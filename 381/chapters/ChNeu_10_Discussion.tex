\chapter{General discussion}\label{sec:genDiscussion}

The results of the experimental study (\chapref{sec:experiments}) and the corpus study (\chapref{sec:diachronic}), alongside the promising outlook of the ProPer toolbox (\chapref{sec:properintroduction}), provide strong support for the synergy of proposals laid out in this work. These include the general PRiORS framework for models of auditory perception in linguistic contexts (\chapref{sec:priors}), the specific treatment of sonority with direct links to perception of pitch and the modeling of syllabic well-formedness with the Nucleus Attraction Principle (\chapref{sec:sonPitch}), as well as the dual-route modeling strategy that considers both top-down and bottom-up inferences with complementary models that can successfully account for both symbolic and dynamic aspects of speech (\chapref{sec:modelimp}).

A few interesting issues deserve elaboration given the above.
In \sectref{sec:division} I discuss the phonotactic division of labor with respect to sonority, which is defined here in very explicit terms, resulting in a narrower approach to what sonority should and should not account for. The discussion uses the case of \emph{/s/-stop} clusters to illustrate this division of labor, making it of special interest as it provides an explanation for the preference of \emph{/s/-stop} clusters over other obstruent clusters, a preference that has been thus far lacking from the current account.
\sectref{sec:projection} is devoted to the classic \emph{nature vs.~nurture} debate as applied to sonority. %There 
I use this context to explicate the contribution of this work to answering the question of %what underlies 
the universality of sonority-based restrictions.
In \sectref{sec:dichotomies} I discuss the complementarity of symbolic/discrete and dynamic/continuous modes in cognitive modeling, suggesting that the top-down--bottom-up distinction exhibits a better fit with the discrete--continuous dichotomy than the classic phonetics--phonology dichotomy.
Finally, I end this book in \sectref{sec:directions} with a brief description of directions for future work.

\section{Phonotactic division of labor}\label{sec:division}

As already mentioned in \chapref{sec:background} (and especially \sectref{sec:problems}), sonority has been widely used to explain practically any type of phonotactic phenomenon, since there is nothing in the standard theory that commits the formal concept of sonority to any specific effect in the perception or articulation of speech. The position taken in this work is very different, drawing explicit links between sonority and the auditory perception of pitch. As a result, sonority in this work is a more specific and more narrowly defined concept. This is important since it is very unlikely that %one force 
a single factor
underlies all the different phonotactic phenomena. 
However, given that there is no consensus on its phonetic basis,
%but without any phonetic basis, 
sonority has become the lightning rod for unrelated phonotactic phenomena. A more well-defined notion of sonority allows us to achieve a better understanding of the phonotactic division of labor between different %forces 
articulatory and perceptual factors
that play a role in the processing of speech.
\emph{/s/-stop} clusters make a good case in point.

\subsection{Towards a holistic account of /s/-stop clusters}\label{towards-a-holistic-account-of-s-stop-clusters}

The well-formedness accounts that are based on the Nucleus Attraction Principle (NAP) do
%NAP's account of the well-formedness of \emph{/s/-stop} clusters does 
not suffice to explain the phonotactic phenomenon of \emph{/s/-stop} clusters since there is nothing in NAP specific to sibilants or stops that would justify assigning a special status to the particular obstruent combination of a sibilant followed by a stop.
In fact, any voiceless element is practically invisible to NAP as it is only sensitive to portions of the speech signal that contain sufficient periodic energy. Indeed, the predictions of NAP, which were corroborated by experimental results (in \sectref{sec:results}), expect non-sibilant counterparts of /s/, like /f/ in the cluster \emph{ftV}, to pattern with \emph{spV} and \emph{ʃpV}. Furthermore, NAP\textsubscript{bu} successfully predicted that all the voiceless-initial clusters in the experiment -- including the \emph{/s/-stop} clusters -- generally pattern together as well-formed, as far as sonority-based restrictions are concerned.
This may suffice to explain why \emph{/s/-stop} clusters are tolerated, but not why they are so often preferred over other obstruent combinations. The complete phonotactic story of \emph{/s/-stop} clusters thus requires an integrative explanation, in which sonority only plays a limited role.

First, there are various reasons to assume that \emph{fricative-stop} clusters are better-formed than \emph{stop-stop} clusters. This generalization is traditionally captured in abstract formal phonological constraints like the \emph{Obligatory Contour Principle} (OCP; going back to \citealt{leben1973suprasegmental, goldsmith1976autosegmental} and \citealt{mccarthy1979formalsk}), which acts as a general dissimilatory requirement banning two successive units of the same type. The OCP in this case may be the reflection of an articulatory disadvantage of the \emph{stop-stop} configuration since it should be harder to coordinate two successive closure and release gestures within the span of a complex onset due to aerodynamic reasons.

Note that this also leads to a disadvantage of \emph{stop-stop} from a perceptual point of view, since the first stop in a \emph{stop-stop} configuration is released into the closure phase of the following stop (see \citealt{surprenant1998perception}). The release of a stop burst into a silent closure phase of another stop, rather than the periodic signal of a vowel, means that many of the acoustic cues to the identity of the first stop consonant are severely attenuated (see \citealt{fujimura1978perception}).

This explanation is essentially based on the concept of perceptual \emph{cue robustness} \citep{wright2004review}, which is less relevant to syllabic organization, but rather
based on adjacency between speech sounds and their chances of being recovered given transitions between them. As 
%Ohala and Kawasaki-Fukumori \citep[361]{kawasaki1997alternativessk}
\citet[361]{kawasaki1997alternativessk}
concluded, \enquote{the degree of salience of modulations created by segmental transitions}, rather than sonority and syllabicity, is the determinant factor of many phonotactic constraints.

%Wright's \citeyear{wright2004review} 
\citegen{wright2004review} 
\emph{cue robustness}
is also critical for the remaining explanation regarding the phonotactic advantage of \emph{/s/-stop} clusters over comparable non-sibilant \emph{fricative-stop} clusters, e.g., \emph{spV} vs.~\emph{ftV}.
Here, the notion of cue robustness serves to explain why sibilants, with their salient and distinctive high frequency aperiodic energy, stand out more than other fricatives, thus allowing more effective recoverability from relatively weak marginal positions (i.e. distant from the vocalic nucleus).

The three phonotactic perspectives are complementary, and although they do not represent an exhaustive list of phonotactic pressures, we need at least these three -- \emph{sonority}, \emph{articulatory dissimilation}, and \emph{cue robustness} -- in order to properly appreciate the phonotactic phenomenon of \emph{/s/-stop} clusters. According to this more holistic account, \emph{/s/-stop} clusters are relatively well-formed in terms of sonority because the syllabic margins are not competing for the nucleus, they are well-formed in terms of articulatory coordination complexity due to the two dissimilar successive gestures
and, finally, they are robust in terms of their acoustic cues:
stops in C\textsubscript{2} can be released into a vowel to optimize the effect of the burst in the release phase, while sibilants retain strong cues to their identity thanks to their unique spectral profile.

\subsection{Revisiting extrasyllabicity}\label{revisiting-extrasyllabicity}

Recall the common extrasyllabic accounts of sibilants in \emph{/s/-stop} clusters, discussed in Section~\ref{sec:failures}, in which marginal sibilants are given a unique status with respect to syllabification to explain why they are not predicted by traditional sonority accounts. NAP-based accounts present an advantage because they do not need to carve out exceptions in order to theoretically remove sibilants from syllables that are not predicted by the model. Under NAP, those sibilants can remain in the structure as members of a well-formed syllable.

On the other hand, NAP-based accounts are compatible with the kinematic findings in \citet{hermes2013phonologysk}, which were taken to support extrasyllabic accounts (having found unique articulatory coordination patterns for sibilants in cluster-initial position in Italian).
In NAP-based accounts, sonority has prosodic roles to play in carrying the pitch and the overall prosodic strength at the nucleus of the syllable. Marginal voiceless elements can, therefore, be timed with different considerations in NAP-based accounts. 
For example, it may be beneficial to prolong duration of marginal voiceless elements to increase their recoverability without the risk of increased nucleus competition. 
This would, indeed, result in some unique timing patterns of marginal sibilants in complex onsets while still fitting comfortably with the rationale of NAP.
%For example, it may be beneficial to prolong marginal voiceless elements to increase their recoverability without the risk of increased nucleus competition. This would, indeed, result in some unique timing patterns that can be uncovered in articulation and fit comfortably with the rationale of NAP.

\section{Universality of sonority}\label{sec:projection}

A consistent interest within theoretical phonology concerns the universality of sonority-based principles. An impressive volume of publications devoted to this question can be found in the works of Iris Berent and her colleagues, starting with \citet{berent2007we}, and followed by many subsequent studies (e.g.~\citealt{berent2008language, berent2015role, berent2012language, berent2012universalsk, berent2011syllable, berent2014languagesk, berent2013phnological, berent2017origins, gomez2014language, lennertz2015onthesonority, zhao2015universalsk}). Berent and her colleagues collected mostly behavioral data from perception tasks, where subjects of various different language backgrounds were found to adhere to the SSP, even when presented with combinations that are not attested in their language. The patterns under Berent's consistent scrutiny are usually limited to a set of initial clusters with an onset rise (e.g.~\emph{blif}), an onset plateau (e.g.~\emph{bdif}) and an onset fall (e.g.~\emph{lbif}). Since /s/-initial clusters and sonorant plateaus are absent from these studies, Berent's experimental results with SSP-based models are largely compatible with NAP, as the hierarchy \emph{blif} (3) \(>\) \emph{bdif} (2) \(>\) \emph{lbif} (0) is maintained in NAP\textsubscript{td}.\footnote{NAP\textsubscript{td} model scores are given in brackets. NAP\textsubscript{bu} cannot make such determinations based on symbolic representations, but it should be expected to generally follow the same trends in the vast majority of cases.}

Berent and her colleagues interpret these findings as supporting the innateness hypothesis, assuming that all humans share a universal linguistic knowledge, which is genetically encoded (the \emph{Universal Grammar} in generative traditions).
The universality of sonority principles thus implies innate knowledge of ordinal sonority hierarchies that map onto a discrete representation of the speech signal, with mechanisms that compute the sonority slopes within syllables to determine well-formedness.

The interpretation of Berent's findings has been a matter of interest in the literature. Some responses, like \citet{daland2011explaining} and \citet{hayes2011interpreting}, have argued that the universal phonotactic behaviors that Berent et al.~present can be shown to result from speakers' ability to generalize categories and distributions from the attested lexicon, and use analogy and probabilities to predict unattested forms.
Such models can successfully apply statistical learning methods based on the lexicon, without a requirement for prior formal knowledge of sonority (e.g., \citealt{jurafsky2009speech, albright2009feature, bailey2001determinants, coleman1997stochastic, futrell2017generative, hayes2011interpreting, hayes2008maximum, jarosz2017inputsk, mayer2019phonotacticsk, vitevitch2004webbasedsk}, and \citealt{mirea2019usingsk}).

While it is relatively clear that statistical learners reflect top-down inferences, it is perhaps less obvious that connectionist models, such as \citet{goldsmith1992local, laks1995connectionistsk,smolensky2014optimization} and \citet{tupper2012sonoritysk}, also seem to be quite compatible with what is considered here as top-down phonology.
Connectionist models can be historically related to an opposition to the classic symbol-based models (see \sectref{sec:risenfall}). However, the inputs and outputs of these models are expressed in discrete symbols and they are designed to capture generalizations in terms of the weights of connections in the system, which may serve as a good mechanistic description of top-down operations. 
%Interestingly, connectionist models, such as \citet{goldsmith1992local, laks1995connectionistsk} \citet{smolensky2014optimization} and \citet{tupper2012sonoritysk}, also seem to be quite compatible with what is considered here as the symbolic version of NAP. Connectionist models can be historically related to an opposition to the classic symbol-based models (see \sectref{sec:risenfall}). However, the inputs and outputs of these models are expressed in discrete symbols and they are designed to capture generalizations in terms of the weights of connections in the system, which may serve as a good mechanistic description of top-down operations.
%Likewise, many other models that successfully describe phonotactic knowledge, such as the Bayesian models in \citet{wilson2013bayesian}, and \citet{wilson2014effects}, essentially describe what is considered here to be the top-down mode of the system, which is somewhat disconnected from the physical and cognitive sources of phonotactic phenomena.

In contrast to traditional sonority principles, NAP was designed to be compatible with general cognitive processes and auditory perception, such that no unique assumptions are required for postulating an innate formal knowledge of sonority.
Sonority-based patterns in NAP arise from the general cognitive process that underlies the parsing of the speech stream into syllables with a pitch-bearing nucleus (i.e.~nucleus competition). This requirement for pitch-bearing units may be explained in evolutionary timescales
as the inevitable result of the important role of pitch in speech communication (\citealt{bolinger1978intonation, cutler1997prosody, house1990tonal, roettger2019tune}) and the observation that tune-text integration occurs with syllable-sized units (e.g.~\citealt{goldsmith1976autosegmental, ladd2008intonational, liberman1975intonationalsk, pierrehumbert1980phoneticssk}).

The PRiORS framework from \chapref{sec:priors}, and especially its take on universal aspects of syllabic structure (Section~\ref{sec:universal}), can contribute greatly to explanations regarding the universality of syllables. This is the case both in terms of their typical duration, which is governed by the temporal regime of perception, and their internal segmental makeup in terms of sonority and pitch perception, which are governed by the timescale of the spectral regime.

The NAP approach appears capable of synthesizing the different views on the origins of universal sonority.
The bottom-up model of NAP can explain the universality of sonority as the natural development of communication systems that exploit 
pitch
%both the spectral regime and the temporal regime of 
perception as they shape language systems. The top-down model of NAP is, at the same time, very much in line with statistical phonotactic learners, in which the regularities of language can be deduced from the symbolic abstractions that reflect the speakers' knowledge in stable forms.
Top-down inferences reflect the history of the distribution of recognized symbols as they appear in the lexicon of the ambient language. They only indirectly express the functional aspects that we see in the bottom-up route since they reflect the surface manifestations of the functionally-motivated (bottom-up) dynamics.

To conclude, bottom-up NAP combines the innateness claims for formal sonority universals with a more general explanation that is based on the workings of the perceptual and cognitive systems and the evolution of languages as pitch-bearing communication systems. At the same time, top-down NAP is in line with the rationale of statistical learners and the mechanics of connectionist models. These explanations require symbolic interpretation of the signal that abstract from variable dynamic events into stable forms (e.g.~consonants, vowels, phonological features) in order to learn and generalize over their distributions.

\section{Reshuffling dichotomies in linguistic models}\label{sec:dichotomies}
\begin{sloppypar}
This work rejects the classic dichotomy between phonetics and phonology, whereby continuous phenomena are considered phonetic, while phonology is exclusively modeled in discrete terms (see \sectref{sec:risenfall}).
As I previously mentioned in \sectref{sec:synthesis}, the integration of dynamic aspects into phonological models has already shown that a phonetics--phonology dichotomy does not fit well with a classic continuous--discrete dichotomy, as we have good reasons to incorporate continuous entities into phonology alongside discrete units, and we have good models to simulate this integration (e.g.~Articulatory Phonology and attractor landscape models).
\end{sloppypar}

%The present work suggests 
I suggest in this work
further avenues to integrate dynamics and continuity in perception-based models of phonology, alongside discrete symbolic entities.
I model the
%In this work, the 
effects of processes that respond to signal-based continuous stimuli %are modeled 
as bottom-up processes, while the effects of processes that are initiated by symbol-based discrete units are separately modeled as top-down processes.
Thus, %this work 
I 
suggest that the continuous--discrete dichotomy in phonology should be linked to the bottom-up--top-down dichotomy, and that both of these distinct types of processes need to coexist in language systems.
It is therefore important to highlight the difference between them.

Bottom-up routes in perception are based on continuous stimuli and they are functional in the sense that they adhere to the laws of physics and to the limitations of the perceptual and cognitive systems of the agents.
Bottom-up processes that seem to systematically characterize language processing may be taken to imply an evolutionary benefit for reliable communication.

In contrast, top-down inferences in perception are based on the history of symbolic representations that speakers learn from experience.
This learning ability has its own universal functional limitations (e.g.~memory-related capacities),
but the learned links between the dynamic and symbolic modes can be largely arbitrary, as they rely on the superficial history of co-occurrence, systematically presented by a given language system (see \sectref{sec:pattee}).
These symbols and their probabilistic distributions may be constantly updated, in a Bayesian fashion, reflecting knowledge about the distribution of categorically analyzable units of speech, and contributing to what is typically considered to be \emph{phonological knowledge}.

In this book, I modeled the notion of sonority and its contribution to linguistic sound systems 
%The notion of sonority and its contribution to linguistic sound systems was modeled in this book 
with the assumption that the two different routes -- bottom-up and top-down -- are both active when speech inferences take place. The bot\-tom-up model uses continuous data (periodic energy), dynamic principles (attraction and competition), and functional motivation (syllables carry pitch information) to model sonority. The top-down model is based on generalizations over the discrete segmental units in the system and their distribution given bottom-up sonority restrictions (note that the top-down NAP model is not a true statistical learner for reasons that are explained in \sectref{sec:complementary}).

The results of the perception experiments may be taken to support the importance of both routes, given the evident relative success of both the NAP\textsubscript{td} and the NAP\textsubscript{bu} model. More specifically, the results of the model comparison in Experiment 2 (see Section~\ref{sec:modComp}) suggest further support for the complementarity of the top-down and bottom-up models.
\tabref{tab:resultsmodels} shows the combined contribution of all the different sonority models to a maximized score, which reflects the combined ability of the models to predict unseen forms (see details in Section~\ref{sec:datanlysis}).
Model comparison in Experiment 2 shows that the combined contribution of the two NAP models exhibits the highest degree of complementarity among all models (65\% for NAP\textsubscript{td} and 14\% for NAP\textsubscript{bu}), even though they represent the same principle.
This is a desirable result for the present framework, which advocates for the need for two complementary models to better account for phonological phenomena.

The division of language perception into signal-based models that adhere to the laws of physics and auditory perception, and symbol-based models that adhere to probability-based inferences in cognitive systems, can have profound implications.
For one, it should allow us to extend traditional models of phonology to be readily compatible with models in related scientific fields, providing more opportunities to share terminologies and models across disciplines.

\section{Directions for future work}\label{sec:directions}

The novelties that are proposed here for models of sonority, and more generally, for models of phonology and auditory perception, will need to amass more supporting evidence from multiple sources in order to be more widely adopted.
%and consequently developed further.
I hope to have laid the foundations for such potential long-term contributions with this book.

There are many different threads 
in this work that call for further research.
%this work leaves open for further research.
Among them are improved characterizations of the competition procedure in NAP. The method presented here for NAP\textsubscript{bu}, using the \emph{center of mass} calculations (see \sectref{sec:napbu}), is not a model of the cognitive process itself, but rather an estimation of its result. A more robust and cognitively-plausible measurement would surely improve our bottom-up model of competition for the syllabic nucleus based on NAP.

Furthermore, there is a potentially vast uncharted ground yet to explore by combining the PRiORS theoretical backbone (see \chapref{sec:priors}) with the methodology of the ProPer toolbox (see \chapref{sec:properintroduction}). Most immediately, this relates to the study of prosody, where the continuous information of F0 and periodic energy, and their interactions, can be effectively exploited to model the  major prosodic effects, namely \emph{intonation}, \emph{prominence} and \emph{speech rate}.

%The latter, \emph{speech rate}, is yet to be fully explored within this context. The implementation and the usefulness of this metric are predicted by the PRiORS framework (see Section~\ref{sec:shiftRhythm}), which expects speech rate effects within the timescale of the temporal regime in the form of a fluctuating trajectory. The ProPer toolbox provides the necessary methodology, but it is still in its early testing phase, and can therefore not yet be considered a reliable tool.

%Given the above, it seems likely that 
Hopefully, the findings and approaches presented in this work will be able to deliver more valuable insights into old and new problems in phonology and linguistic theory.
