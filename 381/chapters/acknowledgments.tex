\addchap{\lsAcknowledgementTitle} 

The work presented in this book is based on my doctoral dissertation which was accepted by the Faculty of Arts and Humanities of the University of Cologne in 2022. This work would not have been possible without a doctoral scholarship from the German Academic Exchange Service (DAAD). 
I am also thankful to the Collaborative Research Centre in Cologne (SFB 1252 \emph{Prominence in Language}) for funding the later stages of this effort. Further thanks should also go to a.r.t.e.s. Graduate School for the Humanities Cologne and the Cologne Center of Language Sciences (CCLS) for their assistance during this project. 

I owe a huge debt to my thesis supervisor, Martine Grice, who opened the doors of Cologne's phonetics lab to me. Martine was able to see the potential behind my scattered thoughts and she pushed me forward throughout the path that was required in order to shape them into cohesive proposals that tell a coherent story. Any failure in achieving these goals is mine.

My second thesis supervisor, Doris Mücke, introduced me to Articulatory Phonology, which facilitated critical developments in the way I think about problems in phonology. Doris encouraged me to do my research without losing sight of my creative interest in music, an interest that we both share.

I found personal friendships and academic inspiration in Cologne's phonetics lab. Timo Roettger left a huge mark on my thinking, in terms of both theory and methodology. The same ought to be said about Francesco Cangemi, with whom I also had the pleasure to co-author some of the work that is presented in this book. Francesco's interaction with my projects had a crucial impact on my path.

Many people that I met in Cologne's phonetics lab deserve my gratitude: Simon Wehrle, who reduced the rate of crimes I commit against the English language and remained a steady pillar of sanity; Caterina Ventura, Simona Sbranna and Maria Lialiou, with whom I had the pleasure to co-author papers using the ProPer toolbox; and Anna Bruggeman, who was always willing to assist me if I asked for help, and I always did!

There are many other members, students and friends of the phonetics lab, past and present, to whom I owe my gratitude: Bastian Auris, Stefan Baumann, Mark Ellison, Luke Galea, Katharina Gayler, Harriet Hanekamp, Anne Hermes, Henrik Hess, Alicia Janz, Constantijn Kaland, Janina Kalbertodt, Theo Klinker, Martina Krüger, Hauke Lindstädt, Jane Mertens, Eduardo Möking, Lena Pagel, Christine Riek, Simon Roessig, Christine Röhr, Tobias Schröer, Mathias Stöber, Tabea Thies, Drenushë Valera-Kurteshi and Esther Weitz.

While working on this research in Cologne, I had the pleasure to meet and to engage with The following eclectic list of inspiring researchers, to whom I also wish to extend my thanks: Dinah Baer-Henney, Jason Bishop, Ioana Chitoran, Jennifer Cole, Sam Hellmuth, Hae-Sung Jeon, Frank Kügler, Leonardo Lancia, Umesh Patil, Elina Savino, Petra Schumacher, Kevin Tang, Francisco Torreira, Sandra Vella, Kai Vogeley, Nigel Ward, Bodo Winter and Katharina Zahner-Ritter.

A huge gratitude must be reserved for my good friend, Bruno Nicenboim. The fruits of our collaborative effort can be found in the experimental evidence I present herein, but his influence was more far-reaching. I learned a great deal from Bruno's analytical skills, intellectual honesty and technical agility.

I thank Carol Espy-Wilson for being so generous in giving me access to the APP Detector code, which allowed this whole project to get off the ground with its first reliable measurements of periodic energy. Many people helped me further in using this code. These include Francesco Cangemi, Yair Lakretz and Doron Veltzer. I also thank Paul Boersma for helping me figure out how to extract periodic energy data using Praat.

I thank Joanna Rączaszek-Leonardi for her kindness and for her role in shaping my understanding of language systems (and I thank Leo for pointing me in that direction).
I thank Yoav Beirach for his friendship and the thought-provoking discussions we had about the notions of harmony, sound and time. I thank Eitan Grossman and Noam Faust for their kindness and for the stimulating discussions we had, including invitations to talks in Jerusalem and in Paris. 

I am also thankful for the consistent assistance, engaging feedback and invitations to talks from my ex-teachers in Tel Aviv University, Outi Bat-El and Evan Cohen. My thanks also go to all the friends and colleagues from Tel Aviv University (past and present) who engaged with me and my work in recent years: Daniel Asherov, Si Berebi, Irena Botwinik, Aya Meltzer-Asscher, Avi Mizrachi, Doron Veltzer, Hadas Yeverechyahu and Hadass Zaidenberg.
Further gratitude should be extended to Shuly Wintner, Bracha Nir and Ruth Berman for their contributions to my academic transition towards a doctoral endeavor.

I am grateful for my loving family, my beloved mom and my two amazing sisters (and their own beautiful families), for their unconditional love and endless support. Finally, I thank my dear partner and my very best friend -- Alma -- for wherever she is, I know that we are home.
