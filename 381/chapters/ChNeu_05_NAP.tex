\chapter{Sonority, pitch and the Nucleus Attraction Principle (NAP)}\label{sec:sonPitch}

\section{Sonority and pitch intelligibility: A causal link}\label{sec:pitchintelligibility}

The observation that sonority summarizes an essential quality that is related to vowels and their propensity to deliver a relatively steady harmonic structure, highlighting pitch and formant information, is by no means new. Previous proposals already defined sonority as either relating to vowels in some general way, more specifically relating it to voicing or glottal fold vibration, or to the clarity/strength of formants.\footnote{A partial list of some prominent examples includes \citet{sigurd1955rank, jakobson1956fundamentals, chomsky1968spesk, foley1972rule, ladefoged1971preliminaries, allen1973accentsk, fujimura1975syllable, Donegan1978onthenatural, ultan1978typological, price1980sonority, lindblom1983production, anderson1986suprasegmental, vennemann1988preferencesk, levitt1991syllable, pierrehumbert1992lenition, fujimura1997acoustic, stemberger1997handbook, boersma1998functional, zhang2001effects, howe2004harmonic, clements2009does, sharma2018significance}.} A few previous accounts went even further, by addressing the function of this evasive vowel-centric feature, suggesting that sonority may be related to periodic energy or pitch/tone (\citealt{heselwood1998unusual, ladefoged1997linguistic, lass1988phonology, nathan1989preliminaries, puppel1992sonority}). What all these proposals share, explicitly or implicitly, is a recurring insight about a strong link between the preferred type of segmental material in syllabic nuclei and a set of features that conspire to optimize pitch intelligibility, a property which characterizes vowels more than consonants.

Pitch is an indispensable communicative dimension of all linguistic sound systems (\citealt{bolinger1978intonation, cutler1997prosody, house1990tonal, roettger2019tune}), whether it is lexically determined as in linguistic \emph{tone},
or post-lexically employed to convey intonation, i.e. the linguistic \emph{tune} (see typological accounts of prosodic systems in \citealt{jun2005prosodicsk, jun2015prosodicsk}).
Tones are used to distinguish lexical items while tunes are used to demarcate units,
to modulate semantics (e.g.~information structure and sentence modality) and to
express a vast array of non-propositional meanings (e.g.~discourse-pragmatic intention, emotional state, socio-indexical identity, and attitudinal stance). The importance of pitch to human communication cannot be overstated.

Crucially, linguistic pitch events are known to target syllable-sized units as their \enquote{docking site}, regardless of the type of pitch event, whether they are lexical tones or post-lexical tunes.
These linguistic pitch events are commonly considered to associate with \emph{tone-bearing units} (see \citealt{leben1973suprasegmental}), that are either syllables or \emph{moras}.\footnote{Moras are used to represent quantitative differences between light and heavy syllables (\emph{weight sensitivity}, see Section~\ref{sec:attraction}), such that light syllables contain one mora while heavier syllables contain two (and sometimes even three) moras (see \citealt{hyman1984atheory, hayes1989compensatory, ito1989prosodic, mccarthy1990footsk, zec1995sonority, zec2003prosodic}).}
These associations between the text on the one hand and tone or tune on the other hand are widely assumed to be mediated by syllabic/moraic units.
For example, intonation pitch contours that highlight and modulate whole words and phrases essentially target privileged syllables -- \emph{heads} (stressed syllables) and \emph{edges} (syllables at initial and final positions of prosodic words and phrases) -- to achieve their communicative goal on textual material of various sizes.
This tone-bearing role of syllables and moras is the hallmark of many prominent theories regarding tone and intonation, following from Autosegmental and Autosegmental-Metrical Phonology (e.g.~\citealt{liberman1975intonationalsk, goldsmith1976autosegmental, ladd2008intonational, pierrehumbert1980phoneticssk}).

The functionally motivated conclusion that emerges with respect to sonority is therefore that syllables require a pitch-bearing nucleus and that sonority is a scalar measure of the ability to bear pitch. In other words, sonority is, most likely, a measure of \emph{pitch intelligibility}.
This hypothesis comes with an underlying assumption that was introduced by the PRiORS theoretical framework in \chapref{sec:priors},
whereby syllables are claimed to have followed an evolutionary trajectory that shaped them to optimally carry pitch in their nuclei (Section~\ref{sec:universal}). Sonority, according to this description, serves as the tool that governs the requirement for intelligible pitch as a fundamental characteristic in the design of the building blocks of prosody (see Section~\ref{sec:shiftSonority}).

It is important to note that this view of sonority is explicitly and exclusively based on perception, rather than articulation of speech. However, it does not exclude articulation-based description of syllables under the assumption that restrictions on syllabic structure must be derived from both the perception and the articulation of speech. A case in point is the \emph{Articulatory Phonology} framework
(see \sectref{sec:synthesis}),
with its valuable descriptions of temporal coordination and phase relations between motor gestures, which can be effectively linked to syllabic organization (see, e.g., \citealt{goldstein2007syllablesk, gafos2014stochastic, goldstein2009coupled, hermes2017variabilitysk, shaw2009syllabificationsk}).

\section{Periodic energy and sonority: Causation by transitivity}\label{sec:periodicenergy}

Pitch is a psychophysical phenomenon based on perception and cognition (see \citealt{plomp1976aspects, plack2005psychophysics}). We can technically obtain pitch-related measurements in terms of neurological and behavioral responses directly from perception. Such measurements are hard to accumulate in very large numbers as they require intricate lab procedures in order to collect data from each subject.
Another avenue for obtaining perception-related measurements is to extract them from acoustics, i.e. not directly from the perceived sensation of a human subject but from the digitally-analyzed description of the physical sound in space.
The benefits of acoustic measurements include the accessibility of recording and processing capabilities and the availability of many existing corpora, which facilitate access to large amounts and diverse types of acoustic speech data.
Using acoustics to cover auditory psychophysical phenomena is not a straightforward task. It requires a consistent and reliable association between acoustics on the one hand, and perception and cognition on the other hand.
This task is potentially complicated further with a complex phenomenon like pitch, which is evidently sensitive to various aspects of the rich acoustic signal as well as to our top-down expectations with regard to learned regularities of pitch behavior in the speech signal (see, e.g., \citealt{houtsma1995pitch, mcpherson2018diversity}, \cite[203]{moore2013anintro, shepard2001pitch}).

Fortunately, there are strong links between pitch and acoustic markers. This is well-known from the extensive use of acoustic F0 measurements to estimate perceived pitch height.
Furthermore, pitch estimations from F0 measurements can become more reliable when dealing with specific types of audio such as speech, as in this case the bulk of pitch information comes from a single source (i.e.~one speaker) within a limited range of fundamental frequencies (mostly between 75--400\,Hz, rarely below 50\,Hz or above 600\,Hz).

To estimate perceived pitch intelligibility from acoustic signals, we need to obtain a measure of \emph{periodic energy}, which is a measurement of the acoustic power of periodic components in the signal. It may be helpful to think of this as a measurement of general intensity that excludes the contribution of aperiodic noise and transient bursts.
Measurements of periodic energy are not very different from widely-used F0 measurements that are commonly based on the ability to detect periodic components in the complex signal. Roughly speaking, rather than resolving the harmonic denominator of detected periodic components in order to estimate F0, a periodic energy meter needs to sum over their power.

To conclude, our ability to detect periodicity in acoustic signals allows us to extract good estimates of F0 and periodic energy from speech data. We stand on firm grounds when we map these acoustic markers to perception in terms of pitch height and pitch intelligibility (respectively).
Given a causal link between perceived pitch height and linguistic tone and intonation contours, it is reasonable and, indeed, commonplace, to assume by transitivity that acoustic F0 maintains a causal link to linguistic tone and intonation.
Likewise, given a causal link between perceived pitch intelligibility and linguistic sonority, it should be reasonable to assume by transitivity that acoustic periodic energy maintains a causal link with the linguistically-loaded notion of sonority.

\section{The Nucleus Attraction Principle}\label{sec:nap}

At the heart of all sonority-based principles lies the idea that the most sonorous segment in a sequence is contained within the nucleus of the syllable. This idea in fact postulates a link between the amount of sonority and the nucleus position of the syllable. I adopt this fundamental insight that guides all other sonority principles in the development of the Nucleus Attraction Principle. However, instead of adding further formal assumptions about non-overlapping segments with fixed sonority values and corresponding sonority slopes in symbolic time, the link between sonority and the syllabic nucleus is simply modeled as a dynamic process in real time. All the portions of the speech signal compete against each other for available nuclei in this process.

Sonority is therefore the quality that is capable of \emph{attracting} the nucleus. The varying quantities of this quality, which temporally fluctuate along the stream of speech, determine which portions of speech are prone to succeed in attracting nuclei given their superior local sonority \emph{mass}. The speech portions that fall between those successful attractors are syllabified in the margins of syllables, at onset and coda positions.

Crucially, NAP treats the postulated link between sonority peaks and syllabic nuclei as the result of a perceptual-cognitive process in real time, rather than describing a geometric state of affairs with symbolic discrete tools.
In fact, by modelling the link between sonority and the syllabic nucleus in dynamic terms it is not necessary to add further theoretical postulates about sonority slopes or discrete segmental categories of consonants and vowels in order to determine well-formedness of syllabic structures. Syllabic ill-formedness in NAP-based models is positively correlated with the degree of nucleus competition that a given syllabified portion incurs.

It is important to note that the informativeness of NAP-based models is not derived from identifying the winner of the nucleus competition, but from quantifying the degree of competition within different portions of speech that stand for potential syllabic parses.
NAP-based models can analyze speech parts that are parsed together as a single syllabic unit in order to estimate the degree of competition they give rise to when they compete for a single nucleus.
In discrete terms, NAP-based models can quantify different sequences of segments to reflect how strongly they compete for a single nucleus.
Either way, the higher the degree of internal competition, the more ill-formed a syllable is predicted to result from this parse.
To simplify this further with respect to the subset of instances discussed in this work (i.e.~syllables with complex consonantal onset clusters), it is possible to say that the winner of the nucleus competition is always the only vowel in the structure. The determination of ill-formedness in these cases is based on quantifying the amount of competition that the winning vowel has to withstand given different consonantal clusters in the onset of the same syllable.

It should be also useful to note that we do not expect serious competition to arise from a consonant adjacent to the vowel in the same syllable, such that in a C\textsubscript{1}C\textsubscript{2}V syllable only C\textsubscript{1} is considered to be the potential competitor to V. The consonant in C\textsubscript{2} position has a crucial impact on the competing potential of C\textsubscript{1} but it is not, in and of itself, a competitor in the data presented in this study.\footnote{We narrowly expect vocoids (i.e. glides) to be able to compete for the nucleus from the vowel-adjacent C\textsubscript{2} position, but this case is likely circular since a glide in the nucleus position would be simply considered a (high) vowel.}
To elucidate this point, consider the case of simple CV syllables. Here, sonority levels are expected to rise from C to V
continuously, with no competition for the nucleus.
Nucleus competition, much like sonority slopes, has a limited impact on syllables with maximally simple onsets and/or codas, (i.e. V, CV, VC and CVC). Principles like SSP and NAP play a role chiefly when sequences of consonants are syllabified within a single syllable as complex onset or coda clusters (e.g.~\textbf{C}CV or VC\textbf{C}). The phonotactics of these possible sequences are determined to a large extent by sonority principles. We interpret this aspect of cluster phonotactics such that sequences within syllables are avoided the more they increase the potential competition for the nucleus in the process of syllabifying/parsing the stream of speech.

\subsection{Schematic NAP sketches}\label{sec:NAPsketch}



\begin{figure}
\includegraphics[width=.8\linewidth]{extrenal_figures/napComb150-smallen}
\caption{Schematic depictions of competition scenarios with symbolic CCV structures. Nucleus competition can be understood as the competition between the blue and the purple areas under the sonority curve. The two examples in the top row -- \emph{plV} and \emph{lpV} -- suggest a replication of successful traditional predictions, while the three examples in the bottom row -- \emph{spV}, \emph{sfV} and \emph{nmV} -- suggest a divergence from SSP-type models (see text for more details).}\label{fig:nap-depictions}
\end{figure}

To understand the rationale of NAP, a series of schematic sketches are presented in \figref{fig:nap-depictions}, accompanied by an impressionistic description. These will eventually be implemented within formal models that are described in detail in \chapref{sec:modelimp}.
The five examples with specified consonantal clusters exhibit their related sonorant energy depicted as the \emph{area under the curve}, whereby the curve itself is an idealized depiction of schematic sonority.
The purple area in each syllable in \figref{fig:nap-depictions} denotes the sonorant energy of the winning vowel in the nucleus position while the blue area denotes the sonorant energy of the losing portions in the onset.
Consider for example the pair \emph{plV} and \emph{lpV}, with schematic NAP-related depictions in the top row of \figref{fig:nap-depictions} (and with more traditional sonority slopes in \figref{fig:slopes-pl-lp}). A consonantal onset cluster with a putatively well-formed rising sonority slope like \emph{plV} should be also considered well-formed under NAP due to the very low potential of competition between the marginal minimally-sonorous onset consonant /p/ and the non-adjacent vowel that wins the competition for the nucleus. The intervening /l/ in this case only promotes a continuous rise in sonority from /p/ to V. Likewise, a consonantal onset cluster with a putatively ill-formed falling sonority slope like \emph{lpV} should be also considered ill-formed under NAP due to the strong potential for competition between the marginal sonorous onset consonant /l/ and the non-adjacent vowel, especially given the intervening /p/ that leads to discontinuity in the sonority trajectory between /l/ and V.

Unlike the examples above, where the rationale of NAP is expected to replicate successful predictions of the SSP with cases like \emph{plV} and \emph{lpV}, NAP is also expected to diverge from traditional sonority sequencing principles in those cases where traditional principles suffer from inherent failures, as detailed in Section~\ref{sec:failures}.
Consider the examples in the bottom row of \figref{fig:nap-depictions}, which were also depicted with traditional sonority slopes in Figures~\ref{fig:slopes-sp-sp} and \ref{fig:slopes-nm-sf}.
Under NAP, neither \emph{/s/-stop} clusters like \emph{spV} nor voiceless obstruent plateaus like \emph{sfV} are expected to incur a strong competition syllable-internally due to the low potential for competition between the minimally-sonorous onset consonant /s/ and the non-adjacent vowel that wins the competition (here, the intervening voiceless obstruents /p/ and /f/ retain a minimally sonorous trajectory throughout the whole onset).
At the same time, a strong competition potential is predicted under NAP for nasal plateaus like \emph{nmV} when compared to obstruent plateaus like \emph{sfV}. This should be expected given the strong potential for competition between the marginal sonorous onset consonant /n/ and the non-adjacent winning vowel (here, the intervening nasal retains a relatively level sonorous trajectory throughout the onset).


As a rough conclusion, it is possible to suggest that by observing the potential competition between blue and purple areas in \figref{fig:nap-depictions}, we should easily see that the two structures on the right-most side (\emph{lpV} and \emph{nmV}) exhibit a stronger competition potential syllable-internally in comparison to the other three structures, in a manner that is not fully predictable from their sonority slopes. For more elaborate competition-based distinctions, see Section~\ref{sec:advantages}.

\subsection{\texorpdfstring{On the roots of prosodic \emph{attraction}}{On the roots of prosodic attraction}}\label{sec:attraction}

The central idea behind NAP, whereby sonority \emph{attracts} syllabic nuclei, is, in fact, well-established in phonological theory.
In various descriptions of stress systems, it is often suggested that some languages exhibit \emph{weight sensitivity}.
This is not a universal process, as stress assignment patterns vary from language to language, and not all languages even have stress to begin with. However, weight sensitivity is one of the naturally occurring stress assignment patterns that various unrelated languages exhibit (e.g.~Arabic, Tibetan (Lhasa), Wolof, Finnish, Latin and many more; see \citealt{goedemans2013weight}, and \cite[23]{gordon2006syllableweight} for more exhaustive lists).

\begin{sloppypar}
Weight sensitivity usually means that a language which regularly assigns the primary stress to a certain syllabic position within phonological words (e.g.~initial/final syllable, etc.) may diverge from this canonical position and assign the stress to an adjacent syllable if it is \emph{heavier} than the syllable at the canonically stressed position.
This is standardly understood as \emph{attraction} of the primary stress by the heavy syllable, where heaviness is mainly the product of a longer vowel in the nucleus, and in some languages heaviness may also result from a (preferably sonorant) consonant in the coda (see, e.g., \citealt{mccarthy1979formalsk, gordon2006syllableweight, hayes1980metrical, prince1990quantitative}). There are also analyses whereby vowel qualities that are considered more sonorous due to degree of opening (i.e.~more \emph{open}/ \emph{lower} vowels)
can contribute to heaviness and attract stress \citep{gordon2012sonority, kenstowicz1997quality, delacy2002formal, zec1995sonority, zec2003prosodic}.
\end{sloppypar}

Importantly, all of these notions of weight are consistent with a hierarchy of sonority.
Structurally, the rime is the locus of weight phenomena, and within the rime -- the nucleus is most important for weight. Segmentally, sonorants contribute more to weight than obstruents, and within sonorants, open vowels are the strongest attractors.
Viewed with NAP in mind, attraction of stress in weight-sensitive systems is simply the special case of a regular procedure, whereby weight -- i.e.~sonority mass -- attracts syllabic nuclei. In other words, given the regular process that NAP assumes, by which syllabic nuclei are attracted to sonorant energy masses, weight sensitivity is simply an extension whereby \emph{heavy} syllabic nuclei are attracted to \emph{heavy} sonorant energy masses.

The stressed syllable in weight-sensitive systems is maintained as highly sonorous, which makes it an optimal syllable for carrying tonal events in intonation, generally serving as the docking site for \emph{pitch accents}.
Attraction in prosody thus follows a consistent rationale: sufficiently pitch-intelligible units satisfy the requirement for a regular syllable by attracting nuclei in general, and exceptionally pitch-intelligible units may satisfy a special requirement for the stressed syllable by attracting the strongest nuclei.

A similar process also occurs post-lexically in many languages. This process is related to text-tune interaction, which can often lead to local sonority enhancements of syllables that need to carry tonal information. The most prominent cases are post-lexical prosodic enhancements through an increase in duration and/or intensity of sonorant material (alongside insertions of transitional vocoids and epenthetic vowels) serving to accommodate certain tonal events in intonation (see \citealt{roettger2019tune}).

To conclude, the understanding that sonority is linked to pitch via syllabic units is well established in phonology.
NAP takes this understanding further than previous insights about prosodic weight and text-tune interactions in proposing a functional theory of prosody and sonority based on pitch intelligibility.
