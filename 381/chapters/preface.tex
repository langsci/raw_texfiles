\addchap{\lsPrefaceTitle}
 
Sonority is a central notion in phonetics and phonology with useful formal applications.
%, yet it has remained vague in too many important respects. 
Sonority is a single hierarchical concept that is most often used to characterize all speech sounds along its scale in a manner that is pivotal for generalizing preferences and restrictions on syllabic organization. However, even after many decades in which sonority has played a crucial role in phonology, there is no consensus with regards to its phonetic basis in articulation or perception of speech. Sonority therefore presents a phonetic challenge.

The most prevalent models of sonority are based on the \emph{Sonority Sequencing Principle} (SSP) and they are characteristic of linguistic models from the second half of the twentieth century, whereby the speech signal is represented with discrete symbolic units, phonological processes are modeled as symbol manipulating rules and time is accounted for in terms of the non-overlapping linear order of the discrete units in the symbolic representation. Since speech signals do not in fact readily lend themselves to analyses of non-overlapping discrete units with associated fixed values in symbolic time, it seems like the classic theory created a formal notion of sonority that simply cannot be found in any phonetic space.

The challenges associated with sonority therefore lie at the intersection of the fields of phonetics and phonology. Chiefly, these are issues pertaining to phonetic substance itself and to phonological theory, which accounts for the parts of the system that are linguistically relevant.
To undertake these challenges, this work aims to determine what sonority \emph{is}, what it \emph{does}, and \emph{how} it does it. To achieve these goals, I present the outline of a novel approach for the extraction of continuous entities from acoustic space in order to model dynamic aspects of phonological perception -- \emph{Perceptual Regimes of Repetitive Sound} (PRiORS). It is used here to advance a functional understanding of sonority as a universal aspect of prosody that is tightly related to the requirement for pitch-bearing syllables as the building blocks of speech.

This book argues that sonority is best understood as 
%Sonority is claimed here to be 
a measurement of \emph{pitch intelligibility} in perception, which is closely linked with \emph{periodic energy} in the acoustic signal. 
It continues by presenting a novel principle for sonority-related determinations of well-formedness, titled the \emph{Nucleus Attraction Principle} (NAP), 
%Moreover, in this project I suggest a novel principle for sonority-related determinations regarding well-formedness, 
based on the premise that different speech sounds within a syllables compete for the nucleus.
%the \emph{Nucleus Attraction Principle} (NAP). 
The implementation of NAP is inspired by the theoretical physicist and biologist, Howard Pattee, in suggesting two complementary models to independently account for the symbolic representations and the continuous aspects of the language system, separately covering both top-down and bottom-up routes in speech perception. Importantly, the dynamic and symbolic NAP models outperform the traditional SSP-based models in terms of empirical coverage. This is shown here with experimental studies with German and Hebrew speakers and in a corpus study of Segholate nouns in Modern Hebrew.

This book also presents a set of tools called ProPer, which stands for \emph{Prosodic Analysis with Periodic Energy}. The ProPer toolbox takes the idea that periodic energy can reflect sonority to its next logical step by exploiting periodic energy for various aspects of prosodic research, covering major issues like \emph{syllabification}, \emph{prominence}, \emph{intonation} and \emph{speech rate}.

I complete this book with short discussions on some of the contributions that this work can offer to the linguistic theory. These include the phonotactic division of labor with respect to \emph{/s/-stop} clusters, the debate about the universality of sonority and the fate of the classic phonetics--phonology dichotomy as it relates to continuity and dynamics in phonology. 
