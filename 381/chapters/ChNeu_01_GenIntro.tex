\chapter{General introduction}\label{sec:genIntro}

\section{The sonority challenge}\label{the-sonority-challenge}\largerpage[2]

%Various speech researchers -- the author of this book included -- have often been displeased with the state of affairs regarding the notion of \emph{sonority} as it is conventionally used in phonetics and phonology.
Sonority is a central notion in phonetics and phonology, with many useful formal applications, yet it has remained vague in too many important respects.
The centrality of sonority is primarily derived from the important theoretical weight that it carries in descriptions of syllables in phonology. Sonority is a single hierarchical concept that is most often used to characterize all speech sounds along a single scale in a manner that is pivotal for generalizing preferences and restrictions on syllabic organization.

However, even after many decades in which sonority has played a crucial role in phonology, there is no consensus with regards to its \emph{phonetic} basis in articulation or perception of speech. Many proposals have been debated, but no real consensus has ever been reached.
Sonority therefore presents an ongoing phonetic challenge.
A good overview of the multiplicity of proposals for the basis of sonority can be found in the various publications by Stephen Parker, from his dissertation \citep{parker2002quantifying}, to subsequent publications \citep[like][]{Parker:TheSonorityControversy:2012, parker2017sounding, parker2018abib} that meticulously document the vast research related to sonority in the linguistic literature.

The most prevalent models of sonority are based on the \emph{Sonority Sequencing Principle} (SSP) and they are very characteristic of linguistic models from the second half of the twentieth century, whereby the speech signal is represented as a sequence of discrete units, phonological processes are modeled as symbol manipulating rules, and time is accounted for in terms of the non-overlapping linear order of the discrete units in symbolic representations (see \sectref{sec:risenfall}).

It may very well be the case that little progress has been made in the theory of sonority since the turn of the century because of the constraining role that SSP-based models play with regards to phonetic dimensions. Specifically, the classic theoretical idea that the speech signal is composed of segments that have a fixed sonority value, which they share with other members of the same category, may have created an impossibility in the traditional theory. This is because the speech signal does not in fact lend itself to such analyses of non-overlapping discrete units with fixed sonority values (see \sectref{sec:risenfall}). Such units can only be extracted from a human mind. In other words, the classic theory may have created a formal notion of sonority that simply cannot be found in any phonetic space.

The challenges of sonority are therefore at the intersection of phonetics and phonology.
Chiefly, these are issues pertaining to phonetic substance itself and to phonological theory, which accounts for the parts of the system that are linguistically relevant.
A better model of sonority would seem to require novelties on both fronts.

\section{Beyond correlates}\label{beyond-correlates}

Another issue highlighted by the notion of sonority, and directly related to the above, is the lack of explanatory power in many phonological models.
A good example of this problem can be gleaned from the common practice of suggesting acoustic \emph{correlates} for various linguistic phenomena, without any related attempt to suggest plausible \emph{causation}. This is very often the case when the suggested acoustic correlates do not seem to have any clear and consistent links to the perception or articulation of speech.
In many other cases, the implied causation can be misguided.
A case in point is the use of physical acoustic intensity as a correlate of linguistic phenomena -- sonority \emph{inter alia} -- although the physical amplitude of the entire signal does not consistently relate to perceived loudness, or any other aspect of perception and/or articulation (see details in Section~\ref{sec:correlusions}).

Thus, it seems already at the outset of this project that in order to suggest a form of phonetic substance for sonority we need to be able to explain its function in the language system, beyond its ability to exhibit statistical correlations with sonority hierarchies. This cannot be achieved with a few tweaks in traditional discrete and symbolic models, as they are too far removed from cognitive plausibility owing to their classic computer-like architectures (see \chapref{sec:lingMod}).
The mainstream models of phonology from the second half of the twentieth century are simply not designed to achieve explanatory goals of this kind.

\section{Goals and motivations of the current endeavor}\label{goals-and-motivations-of-the-current-endeavor}

The challenges that sonority presents are far-reaching, as apparent from the list of problems detailed above. Solving them requires a host of theoretical and methodological novelties that necessitate a relatively large-scale effort.
%, such as this book-length project.
To undertake these challenges, this work aims to determine what sonority \emph{is}, what it \emph{does}, and \emph{how} it does it. To this end, it is also imperative to break away from traditional discrete and symbolic models in linguistics by incorporating continuity and dynamics in a perception-based model of phonology.

%If proven successful, the fruits of this endeavor will also be far-reaching, beyond the specific treatment of sonority. This immediately includes new approaches to the study of prosody, where sonority plays a crucial role.
%Moreover and more generally, this includes new approaches to perception-based models in phonology, suggesting a more principled way to exploit acoustic data in linguistic research.
%Lastly, this endeavor can contribute even more generally to a growing need for new approaches to modeling linguistic systems with both discrete and continuous entities that can coexist and interact in a complementary way.


\section{A note about terminological choices}\label{a-note-about-terminological-choices}

Terminology can be confusing.
Often there are multiple terms for the same thing, and they are sometimes loaded with differing implications in different scientific and professional circles.
The following description of terminological interpretations is intended to reduce confusion for readers of this book.

The terms \textsc{consonant} and \textsc{vowel} are used here broadly to denote the phonological entities, which also consider the position within the syllable. In some cases, when it is specifically relevant to only refer to phonetic features (i.e.~only to the degree of vocal tract stricture), I use the terms \textsc{contoid} and \textsc{vocoid} (respectively), following \citet{pike1943phonetics}.

Successive consonants that follow each other in the phonological description are either referred to as a \textsc{sequence} or a \textsc{cluster}. The latter has a more specific meaning, implying that \emph{clusters} are syllabified together in a single syllable, thus constituting a tautosyllabic complex onset or coda. The term \emph{sequence} is used when no implication is made about the status of the two successive items, which could also be heterosyllabic.

When writing about auditory perception, it can be useful to keep acoustic and perceptual aspects separate via distinct terminology. The terms \textsc{intensity}, \textsc{power} and \textsc{amplitude} relate to the acoustic signal. They are often interchangeable although whenever the distinction between the raw pressure and the log-transformed dB scale is relevant, I use \emph{power} for the raw pressure and \emph{intensity} for the the log-transformed dB scale. Crucially, the related term for perception of acoustic strength is consistently \textsc{loudness}.
Likewise, F0, or the \textsc{fundamental frequency}, always refer to the acoustic signal. The related term in perception is \textsc{pitch}.

Note that the terms \textsc{tone} and \textsc{tune} have specific meanings in linguist contexts. A tone is a pitch target with a communicative function that is either part of the lexical repertoire (e.g.~\emph{lexical tone}) or the post-lexical repertoire (e.g.~a \emph{pitch accent}/\emph{boundary} tone). A \emph{tune} is often used to describe larger intonation contours (that are commonly analyzed as being composed of tones).

This book deals with \textsc{discrete} vs.~\textsc{continuous}, and with \textsc{symbolic} vs.~\textsc{dynamic} entities or elements. These pairs are interchangeable in the majority of the contexts used here.
Likewise, the terms \textsc{trajectory} and \textsc{curve}, as they are used here to denote graphs within plots or, more abstractly, when describing the progression of a certain feature in time, are largely interchangeable.

%\subsection{Other technical notes}\label{other-technical-notes}

\section{Conventions}\label{conventions}

Phonemic depictions of speech are presented in this book with mostly standard IPA conventions. A few diversions from the IPA norm include the use of acute accents (´) on the vowel (e.g.~é, ó, á) to mark the stressed syllable, and the use of the simple grapheme /c/ to denote the voiceless alveolar affricate, which is transcribed with the complex grapheme /t͡s/ in the standard IPA.

Within phonemic transcriptions, a dash (-) is sometimes used to mark morpheme boundaries, while a dot (.) is used to mark syllabic boundaries. For instance, \emph{dla.t-ót} stands for `door-\Pl{}' in Modern Hebrew, where the suffix \emph{-ot} forms a syllable with the final consonant of the base morpheme (/t/), as shown by the
location of the dot.

%	\subsubsection{Methods}\label{methods}
%	
%	This thesis was written with \emph{RStudio} (\citet{rstudio}), using the following packages to combine text and code in a single file that can output LaTeX documents with APA style: R (Version 3.6.3; R Core Team, 2018) and the R-packages \emph{knitr} (Version 1.28; Xie, 2020), \emph{papaja} (Version 0.1.0.9942; Aust \& Barth, 2020), and \emph{rmarkdown} (Version 2.2; Allaire et al., 2020).

\section{Scope of book}\label{scope-of-book}

The current book is divided into five parts in an attempt to address all the necessary issues mentioned above. The first part, \emph{Introduction}, includes this chapter (\chapref{sec:genIntro}) and two more chapters that present the relevant background on sonority (\chapref{sec:background}) and linguistic models more generally (\chapref{sec:lingMod}).

The second part, \emph{Novel theoretical outlooks}, contains three chapters that present the new theoretical proposals that this work develops:
(i) the PRiORS framework for modeling auditory perception (\chapref{sec:priors});
(ii) the proposal for sonority's perceptual cause and acoustic correlate, and the new sonority-based criterion for well-formedness determinations, the \emph{Nucleus Attraction Principle} (\chapref{sec:sonPitch}); and
(iii) the implementation of the Nucleus Attraction Principle in both symbolic and dynamic terms as two separate models (\chapref{sec:modelimp}).

The third part, \emph{Evidence in support of the Nucleus Attraction Principle}, consists of two chapters. \chapref{sec:experiments} presents an experimental study that analyzes data from perception tasks carried out by native speakers of German and Modern Hebrew. \chapref{sec:diachronic} presents a corpus study that looks at some phonologized regularities in Modern Hebrew.

The last chapter in which new data is presented (\chapref{sec:properintroduction}) constitutes the fifth part of this book, titled \emph{Further contributions of periodic energy to the study of prosody}. This chapter is a showcase for the ProPer toolbox, which is an ongoing open-source project building on the assumptions proposed here for sonority in order to develop new acoustic tools for the study of prosody.

This book ends in the \emph{Conclusion} part (\chapref{sec:genDiscussion}), with short discussions on issues that this work can contribute to, followed by the closing section, \emph{Directions for future work}.
%such as a more holistic view of the phonotactic division of labor, and the debate regarding the universality of sonority, followed by the closing section, \emph{Directions for future work}.
%Finally, the book ends by briefly pointing at potential directions for future work. 
The discussions cover implications of the current work on issues such as analysis of phonotactic phenomena, %our understanding of 
the universality of sonority and the typical interpretations of the classic phonetics--phonology dichotomy.
%The discussions cover suggestions for understanding the phonotactic division of labor, implications for the debate regarding the universality of sonority and implications for the interpretation of the classic phonetics--phonology dichotomy.

