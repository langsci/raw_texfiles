\title{Form and formalism in linguistics}  %look no further, you can change those things right here.
\BackBody{``Form'' and ``formalism'' are a pair of highly productive and polysemous terms that occupy a central place in much linguistic scholarship. Diverse notions of ``form''  – embedded in biological, cognitive and aesthetic discourses – have been employed in accounts of language structure and relationship, while ``formalism'' harbours a family of senses referring to particular approaches to the study of language as well as representations of linguistic phenomena. This volume brings together a series of contributions from historians of science and philosophers of language that explore some of the key meanings and uses that these multifaceted terms and their derivatives have found in linguistics, and what these reveal about the mindset, temperament and daily practice of linguists, from the nineteenth century up to the present day.}
\typesetter{James McElvenny}
\proofreader{
Agnes Kim,
Andreas Hölzl,
Brett Reynolds,
Daniela Hanna-Kolbe,
Els Elffers,
Eran Asoulin,
George Walkden,
Ivica Jeđud,
Jeroen van de Weijer,
Judith Kaplan,
Katja Politt,
Lachlan Mackenzie,
Laura Melissa Arnold,
Nick Riemer,
Tom Bossuyt,
Winfried Lechner
}
\author{James McElvenny}
\renewcommand{\lsISBNdigital}{978-3-96110-182-5}
\renewcommand{\lsISBNhardcover}{978-3-96110-183-2}                     
\BookDOI{10.5281/zenodo.2654375}
\renewcommand{\lsSeries}{hpls} % use lowercase acronym, e.g. sidl, eotms, tgdi
\renewcommand{\lsSeriesNumber}{1}
\renewcommand{\lsID}{214}
\renewcommand{\lsImpressionCitationAuthor}{McElvenny, James}
