\title{Advances in formal Slavic linguistics 2022}
\BackBody{\textit{Advances in formal Slavic Linguistics 2022} brings together a collection of 22 articles originating as talks presented at the 15th Formal Description of Slavic Languages conference (FDSL 15) held in Berlin on 5–7 October, 2022.  The contributions cover a broad spectrum of topics, including clitics, nominalizations, l-participles, the dual, verbal prefixes, assibilation, verbal and adjectival morphology, lexical stress, vowel reduction, focus particles,
aspect, multiple wh-fronting, definiteness, polar questions, negation words, and argument structure in such languages as BCMS, Bulgarian, Czech, Macedonian, Polish, Russian, Slovenian, Ukrainian, and Upper Sorbian.
The wide range of topics explored in this volume underscores the  diversity and complexity of Slavic languages. The contributions not only advance our understanding of languages belonging to the Slavic group
but also offer fresh perspectives for linguistics more broadly.
}
\author{Berit Gehrke and Denisa Lenertová and Roland Meyer and Daria Seres and Luka Szucsich and Joanna Zaleska}

\renewcommand{\lsISBNdigital}{978-3-96110-506-9}
\renewcommand{\lsISBNhardcover}{978-3-98554-135-5}
\BookDOI{10.5281/zenodo.15056351}
% \typesetter{}
\proofreader{Alessandro Bigolin,
Alexandr Rosen,
Amir Ghorbanpour,
David Carrasco Coquillat,
Elisa Roma,
Elliott Pearl,
Erk Deniz Vargez,
Ivelina Stoyanova,
Jeroen van de Weijer,
Katja Politt,
Lachlan Mackenzie,
Ludger Paschen,
Mary Ann Walter,
Nicoletta Romeo,
Jean Nitzke,
Patricia Cabredo Hofherr,
Raquel Benítez Burraco,
Silvie Strauß}
% \lsCoverTitleSizes{51.5pt}{17pt}% Font setting for the title page


\renewcommand{\lsID}{481}
\renewcommand{\lsSeries}{osl}
\renewcommand{\lsSeriesNumber}{10}
