\documentclass[output=paper,colorlinks,citecolor=brown]{langscibook}
\ChapterDOI{10.5281/zenodo.15394166}
%\bibliography{localbibliography}

\author{Petr Biskup\orcid{0000-0003-2602-8856}\affiliation{University of Leipzig}}
% replace the above with you and your coauthors
% rules for affiliation: If there's an official English version, use that (find out on the official website of the university); if not, use the original
% orcid doesn't appear printed; it's metainformation used for later indexing

%%% uncomment the following line if you are a single author or all authors have the same affiliation
\SetupAffiliations{mark style=none}

%% in case the running head with authors exceeds one line (which is the case in this example document), use one of the following methods to turn it into a single line; otherwise comment the line below out with % and ignore it
% \lehead{Šimík, Gehrke, Lenertová, Meyer, Szucsich \& Zaleska}
% \lehead{Radek Šimík et al.}

\title[Delimitatives, diminutive-iteratives and the secondary imperfective]{Delimitatives, diminutive-iteratives and the secondary imperfective in North Slavic}
% replace the above with your paper title
%%% provide a shorter version of your title in case it doesn't fit a single line in the running head
% in this form: \title[short title]{full title}
\abstract{This paper is concerned with diminutive-iterative verbs, delimitative verbs with the prefix \textit{po-} and the secondary imperfective suffix. It is argued that diminutive-iterative \textit{po-}verbs are derivationally based on delimitative predicates. Further, the secondary imperfective suffix is not an undifferentiated element. It is argued that the two instances of the imperfectivizing suffix -- the iterative one and the progressive one -- merge in distinct structural positions and that the delimitative prefix \textit{po-} occurs between them. In the derivation of diminutive-iteratives, delimitative \textit{po-} selects a predicate with a scalar structure and the Davidsonian event variable and contributes an extensive measure function to the base predicate. The iterative \textit{-yva}, with its pluractional semantics, then iterates the eventuality denoted by the \textit{po-}predicate.

\keywords{delimitative prefix, diminutive-iterative \textit{po-}verbs, iterativity, progressivity, secondary imperfective}
}

\begin{document}
\maketitle

% Just comment out the input below when you're ready to go.
% For a start: Do not forget to give your Overleaf project (this paper) a recognizable name. This one could be called, for instance, Simik et al: OSL template. You can change the name of the project by hovering over the gray title at the top of this page and clicking on the pencil icon.

\section{Introduction}\label{sim:sec:intro}

Language Science Press is a project run for linguists, but also by linguists. You are part of that and we rely on your collaboration to get at the desired result. Publishing with LangSci Press might mean a bit more work for the author (and for the volume editor), esp. for the less experienced ones, but it also gives you much more control of the process and it is rewarding to see the quality result.

Please follow the instructions below closely, it will save the volume editors, the series editors, and you alike a lot of time.

\sloppy This stylesheet is a further specification of three more general sources: (i) the Leipzig glossing rules \citep{leipzig-glossing-rules}, (ii) the generic style rules for linguistics (\url{https://www.eva.mpg.de/fileadmin/content_files/staff/haspelmt/pdf/GenericStyleRules.pdf}), and (iii) the Language Science Press guidelines \citep{Nordhoff.Muller2021}.\footnote{Notice the way in-text numbered lists should be written -- using small Roman numbers enclosed in brackets.} It is advisable to go through these before you start writing. Most of the general rules are not repeated here.\footnote{Do not worry about the colors of references and links. They are there to make the editorial process easier and will disappear prior to official publication.}

Please spend some time reading through these and the more general instructions. Your 30 minutes on this is likely to save you and us hours of additional work. Do not hesitate to contact the editors if you have any questions.

\section{Illustrating OSL commands and conventions}\label{sim:sec:osl-comm}

Below I illustrate the use of a number of commands defined in langsci-osl.tex (see the styles folder).

\subsection{Typesetting semantics}\label{sim:sec:sem}

See below for some examples of how to typeset semantic formulas. The examples also show the use of the sib-command (= ``semantic interpretation brackets''). Notice also the the use of the dummy curly brackets in \REF{sim:ex:quant}. They ensure that the spacing around the equation symbol is correct. 

\ea \ea \sib{dog}$^g=\textsc{dog}=\lambda x[\textsc{dog}(x)]$\label{sim:ex:dog}
\ex \sib{Some dog bit every boy}${}=\exists x[\textsc{dog}(x)\wedge\forall y[\textsc{boy}(y)\rightarrow \textsc{bit}(x,y)]]$\label{sim:ex:quant}
\z\z

\noindent Use noindent after example environments (but not after floats like tables or figures).

And here's a macro for semantic type brackets: The expression \textit{dog} is of type $\stb{e,t}$. Don't forget to place the whole type formula into a math-environment. An example of a more complex type, such as the one of \textit{every}: $\stb{s,\stb{\stb{e,t},\stb{e,t}}}$. You can of course also use the type in a subscript: dog$_{\stb{e,t}}$

We distinguish between metalinguistic constants that are translations of object language, which are typeset using small caps, see \REF{sim:ex:dog}, and logical constants. See the contrast in \REF{sim:ex:speaker}, where \textsc{speaker} (= serif) in \REF{sim:ex:speaker-a} is the denotation of the word \textit{speaker}, and \cnst{speaker} (= sans-serif) in \REF{sim:ex:speaker-b} is the function that maps the context $c$ to the speaker in that context.\footnote{Notice that both types of small caps are automatically turned into text-style, even if used in a math-environment. This enables you to use math throughout.}$^,$\footnote{Notice also that the notation entails the ``direct translation'' system from natural language to metalanguage, as entertained e.g. in \citet{Heim.Kratzer1998}. Feel free to devise your own notation when relying on the ``indirect translation'' system (see, e.g., \citealt{Coppock.Champollion2022}).}

\ea\label{sim:ex:speaker}
\ea \sib{The speaker is drunk}$^{g,c}=\textsc{drunk}\big(\iota x\,\textsc{speaker}(x)\big)$\label{sim:ex:speaker-a}
\ex \sib{I am drunk}$^{g,c}=\textsc{drunk}\big(\cnst{speaker}(c)\big)$\label{sim:ex:speaker-b}
\z\z

\noindent Notice that with more complex formulas, you can use bigger brackets indicating scope, cf. $($ vs. $\big($, as used in \REF{sim:ex:speaker}. Also notice the use of backslash plus comma, which produces additional space in math-environment.

\subsection{Examples and the minsp command}

Try to keep examples simple. But if you need to pack more information into an example or include more alternatives, you can resort to various brackets or slashes. For that, you will find the minsp-command useful. It works as follows:

\ea\label{sim:ex:german-verbs}\gll Hans \minsp{\{} schläft / schlief / \minsp{*} schlafen\}.\\
Hans {} sleeps {} slept {} {} sleep.\textsc{inf}\\
\glt `Hans \{sleeps / slept\}.'
\z

\noindent If you use the command, glosses will be aligned with the corresponding object language elements correctly. Notice also that brackets etc. do not receive their own gloss. Simply use closed curly brackets as the placeholder.

The minsp-command is not needed for grammaticality judgments used for the whole sentence. For that, use the native langsci-gb4e method instead, as illustrated below:

\ea[*]{\gll Das sein ungrammatisch.\\
that be.\textsc{inf} ungrammatical\\
\glt Intended: `This is ungrammatical.'\hfill (German)\label{sim:ex:ungram}}
\z

\noindent Also notice that translations should never be ungrammatical. If the original is ungrammatical, provide the intended interpretation in idiomatic English.

If you want to indicate the language and/or the source of the example, place this on the right margin of the translation line. Schematic information about relevant linguistic properties of the examples should be placed on the line of the example, as indicated below.

\ea\label{sim:ex:bailyn}\gll \minsp{[} Ėtu knigu] čitaet Ivan \minsp{(} často).\\
{} this book.{\ACC} read.{\PRS.3\SG} Ivan.{\NOM} {} often\\\hfill O-V-S-Adv
\glt `Ivan reads this book (often).'\hfill (Russian; \citealt[4]{Bailyn2004})
\z

\noindent Finally, notice that you can use the gloss macros for typing grammatical glosses, defined in langsci-lgr.sty. Place curly brackets around them.

\subsection{Citation commands and macros}

You can make your life easier if you use the following citation commands and macros (see code):

\begin{itemize}
    \item \citealt{Bailyn2004}: no brackets
    \item \citet{Bailyn2004}: year in brackets
    \item \citep{Bailyn2004}: everything in brackets
    \item \citepossalt{Bailyn2004}: possessive
    \item \citeposst{Bailyn2004}: possessive with year in brackets
\end{itemize}

\section{Trees}\label{s:tree}

Use the forest package for trees and place trees in a figure environment. \figref{sim:fig:CP} shows a simple example.\footnote{See \citet{VandenWyngaerd2017} for a simple and useful quickstart guide for the forest package.} Notice that figure (and table) environments are so-called floating environments. {\LaTeX} determines the position of the figure/table on the page, so it can appear elsewhere than where it appears in the code. This is not a bug, it is a property. Also for this reason, do not refer to figures/tables by using phrases like ``the table below''. Always use tabref/figref. If your terminal nodes represent object language, then these should essentially correspond to glosses, not to the original. For this reason, we recommend including an explicit example which corresponds to the tree, in this particular case \REF{sim:ex:czech-for-tree}.

\ea\label{sim:ex:czech-for-tree}\gll Co se řidič snažil dělat?\\
what {\REFL} driver try.{\PTCP.\SG.\MASC} do.{\INF}\\
\glt `What did the driver try to do?'
\z

\begin{figure}[ht]
% the [ht] option means that you prefer the placement of the figure HERE (=h) and if HERE is not possible, you prefer the TOP (=t) of a page
% \centering
    \begin{forest}
    for tree={s sep=1cm, inner sep=0, l=0}
    [CP
        [DP
            [what, roof, name=what]
        ]
        [C$'$
            [C
                [\textsc{refl}]
            ]
            [TP
                [DP
                    [driver, roof]
                ]
                [T$'$
                    [T [{[past]}]]
                    [VP
                        [V
                            [tried]
                        ]
                        [VP, s sep=2.2cm
                            [V
                                [do.\textsc{inf}]
                            ]
                            [t\textsubscript{what}, name=trace-what]
                        ]
                    ]
                ]
            ]
        ]
    ]
    \draw[->,overlay] (trace-what) to[out=south west, in=south, looseness=1.1] (what);
    % the overlay option avoids making the bounding box of the tree too large
    % the looseness option defines the looseness of the arrow (default = 1)
    \end{forest}
    \vspace{3ex} % extra vspace is added here because the arrow goes too deep to the caption; avoid such manual tweaking as much as possible; here it's necessary
    \caption{Proposed syntactic representation of \REF{sim:ex:czech-for-tree}}
    \label{sim:fig:CP}
\end{figure}

Do not use noindent after figures or tables (as you do after examples). Cases like these (where the noindent ends up missing) will be handled by the editors prior to publication.

\section{Italics, boldface, small caps, underlining, quotes}

See \citet{Nordhoff.Muller2021} for that. In short:

\begin{itemize}
    \item No boldface anywhere.
    \item No underlining anywhere (unless for very specific and well-defined technical notation; consult with editors).
    \item Small caps used for (i) introducing terms that are important for the paper (small-cap the term just ones, at a place where it is characterized/defined); (ii) metalinguistic translations of object-language expressions in semantic formulas (see \sectref{sim:sec:sem}); (iii) selected technical notions.
    \item Italics for object-language within text; exceptionally for emphasis/contrast.
    \item Single quotes: for translations/interpretations
    \item Double quotes: scare quotes; quotations of chunks of text.
\end{itemize}

\section{Cross-referencing}

Label examples, sections, tables, figures, possibly footnotes (by using the label macro). The name of the label is up to you, but it is good practice to follow this template: article-code:reference-type:unique-label. E.g. sim:ex:german would be a proper name for a reference within this paper (sim = short for the author(s); ex = example reference; german = unique name of that example).

\section{Syntactic notation}

Syntactic categories (N, D, V, etc.) are written with initial capital letters. This also holds for categories named with multiple letters, e.g. Foc, Top, Num, etc. Stick to this convention also when coming up with ad hoc categories, e.g. Cl (for clitic or classifier).

An exception from this rule are ``little'' categories, which are written with italics: \textit{v}, \textit{n}, \textit{v}P, etc.

Bar-levels must be typeset with bars/primes, not with an apostrophe. An easy way to do that is to use mathmode for the bar: C$'$, Foc$'$, etc.

Specifiers should be written this way: SpecCP, Spec\textit{v}P.

Features should be surrounded by square brackets, e.g., [past]. If you use plus and minus, be sure that these actually are plus and minus, and not e.g. a hyphen. Mathmode can help with that: [$+$sg], [$-$sg], [$\pm$sg]. See \sectref{sim:sec:hyphens-etc} for related information.

\section{Footnotes}

Absolutely avoid long footnotes. A footnote should not be longer than, say, {20\%} of the page. If you feel like you need a long footnote, make an explicit digression in the main body of the text.

Footnotes should always be placed at the end of whole sentences. Formulate the footnote in such a way that this is possible. Footnotes should always go after punctuation marks, never before. Do not place footnotes after individual words. Do not place footnotes in examples, tables, etc. If you have an urge to do that, place the footnote to the text that explains the example, table, etc.

Footnotes should always be formulated as full, self-standing sentences.

\section{Tables}

For your tables use the table plus tabularx environments. The tabularx environment lets you (and requires you in fact) to specify the width of the table and defines the X column (left-alignment) and the Y column (right-alignment). All X/Y columns will have the same width and together they will fill out the width of the rest of the table -- counting out all non-X/Y columns.

Always include a meaningful caption. The caption is designed to appear on top of the table, no matter where you place it in the code. Do not try to tweak with this. Tables are delimited with lsptoprule at the top and lspbottomrule at the bottom. The header is delimited from the rest with midrule. Vertical lines in tables are banned. An example is provided in \tabref{sim:tab:frequencies}. See \citet{Nordhoff.Muller2021} for more information. If you are typesetting a very complex table or your table is too large to fit the page, do not hesitate to ask the editors for help.

\begin{table}
\caption{Frequencies of word classes}
\label{sim:tab:frequencies}
 \begin{tabularx}{.77\textwidth}{lYYYY} %.77 indicates that the table will take up 77% of the textwidth
  \lsptoprule
            & nouns & verbs  & adjectives & adverbs\\
  \midrule
  absolute  &   12  &    34  &    23      & 13\\
  relative  &   3.1 &   8.9  &    5.7     & 3.2\\
  \lspbottomrule
 \end{tabularx}
\end{table}

\section{Figures}

Figures must have a good quality. If you use pictorial figures, consult the editors early on to see if the quality and format of your figure is sufficient. If you use simple barplots, you can use the barplot environment (defined in langsci-osl.sty). See \figref{sim:fig:barplot} for an example. The barplot environment has 5 arguments: 1. x-axis desription, 2. y-axis description, 3. width (relative to textwidth), 4. x-tick descriptions, 5. x-ticks plus y-values.

\begin{figure}
    \centering
    \barplot{Type of meal}{Times selected}{0.6}{Bread,Soup,Pizza}%
    {
    (Bread,61)
    (Soup,12)
    (Pizza,8)
    }
    \caption{A barplot example}
    \label{sim:fig:barplot}
\end{figure}

The barplot environment builds on the tikzpicture plus axis environments of the pgfplots package. It can be customized in various ways. \figref{sim:fig:complex-barplot} shows a more complex example.

\begin{figure}
  \begin{tikzpicture}
    \begin{axis}[
	xlabel={Level of \textsc{uniq/max}},  
	ylabel={Proportion of $\textsf{subj}\prec\textsf{pred}$}, 
	axis lines*=left, 
        width  = .6\textwidth,
	height = 5cm,
    	nodes near coords, 
    % 	nodes near coords style={text=black},
    	every node near coord/.append style={font=\tiny},
        nodes near coords align={vertical},
	ymin=0,
	ymax=1,
	ytick distance=.2,
	xtick=data,
	ylabel near ticks,
	x tick label style={font=\sffamily},
	ybar=5pt,
	legend pos=outer north east,
	enlarge x limits=0.3,
	symbolic x coords={+u/m, \textminus u/m},
	]
	\addplot[fill=red!30,draw=none] coordinates {
	    (+u/m,0.91)
        (\textminus u/m,0.84)
	};
	\addplot[fill=red,draw=none] coordinates {
	    (+u/m,0.80)
        (\textminus u/m,0.87)
	};
	\legend{\textsf{sg}, \textsf{pl}}
    \end{axis} 
  \end{tikzpicture} 
    \caption{Results divided by \textsc{number}}
    \label{sim:fig:complex-barplot}
\end{figure}

\section{Hyphens, dashes, minuses, math/logical operators}\label{sim:sec:hyphens-etc}

Be careful to distinguish between hyphens (-), dashes (--), and the minus sign ($-$). For in-text appositions, use only en-dashes -- as done here -- with spaces around. Do not use em-dashes (---). Using mathmode is a reliable way of getting the minus sign.

All equations (and typically also semantic formulas, see \sectref{sim:sec:sem}) should be typeset using mathmode. Notice that mathmode not only gets the math signs ``right'', but also has a dedicated spacing. For that reason, never write things like p$<$0.05, p $<$ 0.05, or p$<0.05$, but rather $p<0.05$. In case you need a two-place math or logical operator (like $\wedge$) but for some reason do not have one of the arguments represented overtly, you can use a ``dummy'' argument (curly brackets) to simulate the presence of the other one. Notice the difference between $\wedge p$ and ${}\wedge p$.

In case you need to use normal text within mathmode, use the text command. Here is an example: $\text{frequency}=.8$. This way, you get the math spacing right.

\section{Abbreviations}

The final abbreviations section should include all glosses. It should not include other ad hoc abbreviations (those should be defined upon first use) and also not standard abbreviations like NP, VP, etc.


\section{Bibliography}

Place your bibliography into localbibliography.bib. Important: Only place there the entries which you actually cite! You can make use of our OSL bibliography, which we keep clean and tidy and update it after the publication of each new volume. Contact the editors of your volume if you do not have the bib file yet. If you find the entry you need, just copy-paste it in your localbibliography.bib. The bibliography also shows many good examples of what a good bibliographic entry should look like.

See \citet{Nordhoff.Muller2021} for general information on bibliography. Some important things to keep in mind:

\begin{itemize}
    \item Journals should be cited as they are officially called (notice the difference between and, \&, capitalization, etc.).
    \item Journal publications should always include the volume number, the issue number (field ``number''), and DOI or stable URL (see below on that).
    \item Papers in collections or proceedings must include the editors of the volume (field ``editor''), the place of publication (field ``address'') and publisher.
    \item The proceedings number is part of the title of the proceedings. Do not place it into the ``volume'' field. The ``volume'' field with book/proceedings publications is reserved for the volume of that single book (e.g. NELS 40 proceedings might have vol. 1 and vol. 2).
    \item Avoid citing manuscripts as much as possible. If you need to cite them, try to provide a stable URL.
    \item Avoid citing presentations or talks. If you absolutely must cite them, be careful not to refer the reader to them by using ``see...''. The reader can't see them.
    \item If you cite a manuscript, presentation, or some other hard-to-define source, use the either the ``misc'' or ``unpublished'' entry type. The former is appropriate if the text cited corresponds to a book (the title will be printed in italics); the latter is appropriate if the text cited corresponds to an article or presentation (the title will be printed normally). Within both entries, use the ``howpublished'' field for any relevant information (such as ``Manuscript, University of \dots''). And the ``url'' field for the URL.
\end{itemize}

We require the authors to provide DOIs or URLs wherever possible, though not without limitations. The following rules apply:

\begin{itemize}
    \item If the publication has a DOI, use that. Use the ``doi'' field and write just the DOI, not the whole URL.
    \item If the publication has no DOI, but it has a stable URL (as e.g. JSTOR, but possibly also lingbuzz), use that. Place it in the ``url'' field, using the full address (https: etc.).
    \item Never use DOI and URL at the same time.
    \item If the official publication has no official DOI or stable URL (related to the official publication), do not replace these with other links. Do not refer to published works with lingbuzz links, for instance, as these typically lead to the unpublished (preprint) version. (There are exceptions where lingbuzz or semanticsarchive are the official publication venue, in which case these can of course be used.) Never use URLs leading to personal websites.
    \item If a paper has no DOI/URL, but the book does, do not use the book URL. Just use nothing.
\end{itemize}


\iffalse
noch offen vom Typesestting:

    - examples (10)-(12): asterisk placed on an English translation; here I'd be in favor of doing something like "cf. \textit{*to be reading for a while}" (rationale: it's about the ungrammaticality of the English example, not of the reading/interpretation, that's why I'd use italics (and introduce it by cf.)
    - examples (31), (32), (33), (35), (36) are semantic formulas; (46) is a syntactic tree
    - above (32), after po-delimitatives, there's a dot before "necessary" which, however, is not capitalized... the sentence structure is unclear
    - Biskup (to appear) needs a reference
    - (37) contains a star in the translation; consider reformulation
    - (47) - asterisks within an example, without glosses and translations, please fix (either into one example with slashes, or with full subexamples, but then please with glosses and intended meanings).
    - (48) - intended meaning missing
    - Biskup (2020) needs more specification in the bibliography

\fi



\section{Introduction}\label{biskup:sec:intro}

This section introduces diminutive-iterative \textit{po-}verbs, delimitative \textit{po-}verbs and secondary imperfective verbs and briefly overviews their morphosyntactic and semantic properties relevant to the analysis pursued in following sections.

\subsection{Diminutive-iterative verbs}\label{biskup:sec:intro-dim}

As to their form, diminutive-iterative verbs contain the prefix \textit{po-} and the imperfectivizing/iterative suffix \textsc{-yva}, as shown in \REF{biskup:ex:pref-rus} for Russian, in \REF{biskup:ex:pref-po} for Polish, and in \REF{biskup:ex:pref-cz} for Czech (see also \citealt{bis:Svedova1980}: 600, \citealt{Katny1994}: 66--70, and \citealt{Petr1986}: 398). In what follows, I will use \textsc{-yva} as a shorthand that also stands for other allomorphs, e.g. in Russian, it stands for the allomorphs \textit{-yva/-iva}, \textit{-va} and \textit{-a}.%\footnote{The following abbreviations are used: DEL=delimitative, F=feminine, GEN=genitive, HAB=habitual, IPFV=imperfective, INF= infinitive, INS=instrumental, ITER=iterative, NMLZ= nominalizer, PFV=perfective, PL=plural, PROG=progressive, PST=past, PTCP=participle, SG=singular, SI=secondary imperfective, TH=theme (vowel).}

\ea\label{biskup:ex:pref-rus}\gll po-lёž-iva-t'\\
\textsc{del}-lie-\textsc{iter}-\INF\\
\glt ‘to lie from time to time’\hfill (Russian)
\z
\ea\label{biskup:ex:pref-po}\gll po-płak-iwa-ć\\
\textsc{del}-cry-\textsc{iter}-\INF\\
\glt ‘to cry from time to time’\hfill (Polish)
\z
\ea\label{biskup:ex:pref-cz}\gll po-sed-á-va-t\\
\textsc{del}-sit-\textsc{th}-\textsc{iter}-\INF\\
\glt ‘to sit from time to time’\hfill (Czech)
\z

\noindent Concerning morphological (grammatical/viewpoint) aspect properties, di\-min\-u\-tive-iterative \textit{po-}verbs are always imperfective. With respect to their meaning, di\-min\-ut\-ive-iteratives are usually described as denoting a short action (with low intensity) that is repeated several times; see e.g. \citet[103]{Zaliznjak.Smelev1997} for Russian, \citet[23]{Czochralski1975} for Polish and \citet[194, 209]{Karlik.etal1995} for Czech. The action does not have to be repeated regularly and the iterative suffix brings about an unspecified number of instances of the particular eventuality. Typically, the number of repetitions depends on the context.

\subsection{Secondary imperfective verbs} \label{biskup:sec:intro-sec}

Secondary imperfective verbs contain an \textsc{-yva} allomorph, which derives imperfective predicates from perfective stems, as shown in examples \REF{biskup:ex:second-rus}, \REF{biskup:ex:second-pol} and \REF{biskup:ex:second-cz}. Crucially, \textsc{-yva} allomorphs used in formation of diminutive-iterative verbs are identical to the \textsc{-yva} allomorphs used in secondary imperfective verbs.


\ea\label{biskup:ex:second-rus}\ea\gll za-pis-a-t'\textsuperscript{PFV}\\
behind-write-\textsc{th-inf}\\
\glt `to write down'\label{biskup:ex:second-rus.a}
\ex\gll za-pis-yva-t'\textsuperscript{IPFV}\\
behind-write-\textsc{si-inf}\\
\glt `to write down'\\`to be writing down'\hfill (Russian)\label{biskup:ex:second-rus.b}
\z\z

\ea\label{biskup:ex:second-pol}\ea\gll pod-pis-a-ć\textsuperscript{PFV}\\
below-write-\textsc{th-inf}\\
\glt ‘to sign’\label{biskup:ex:second-pol.a}
\ex\gll pod-pis-ywa-ć\textsuperscript{IPFV}\\
below-write-\textsc{si-inf}\\
\glt ‘to sign’\\‘to be signing’\hfill(Polish)\label{biskup:ex:second-pol.b}
\z\z

\ea\label{biskup:ex:second-cz}\ea\gll vy-ps-a-t\textsuperscript{PFV}\\
out-write-\textsc{th-inf}\\
\glt ‘to excerpt’\label{biskup:ex:second-cz.a}
\ex\gll vy-pis-ova-t\textsuperscript{IPFV}\\
out-write-\textsc{si-inf}\\
\glt ‘to excerpt’\\`to be excerpting’\hfill(Czech)\label{biskup:ex:second-cz.b}
\z\z

\noindent Secondary imperfective verbs can have (at least) four meanings. 1. progressive, expressing that a certain eventuality is in progress; 2. iterative, which expresses the successive occurrence of several instances of a certain eventuality; 3. habitual (generic), which describes an eventuality that is characteristic of an extended time period; and 4. general-factual, which typically refers to a realized or even completed eventuality (in a fashion similar to perfective verbs); see e.g. \citet[24--40]{bis:Comrie1976}, \citet[75--102]{Dahl1985}, \citet[49--125]{bis:Dickey2000}, \citet[22--30]{Groenn2004}, \citet[417--424]{Timberlake2004} and \citet[64--76]{Petruchina2011}. This article is concerned with the progressive and iterative meanings.

\subsection{Delimitative verbs}\label{biskup:sec:delim}

Delimitative verbs are formed with the help of the delimitative prefix \textit{po-} and with the reduplicated form \textit{popo-} in the case of motion verbs in Czech, as demonstrated by \REF{biskup:ex:del-rus}, \REF{biskup:ex:del-pol}, and \REF{biskup:ex:del-cz}. The prefix is adjoined to an unprefixed, imperfective stem and derives a perfective predicate, as shown in the examples under discussion. 

\ea\label{biskup:ex:del-rus}\ea\gll čit-a-t'\textsuperscript{IPFV}\\
read-\textsc{th-inf}\\
\glt ‘to read’\\‘to be reading’\label{biskup:ex:del-rus.a}
\ex\gll po-čit-a-t'\textsuperscript{PFV}\\
\textsc{del}-read-\textsc{th-inf}\\
\glt ‘to read for a while’\hfill(Russian)\label{biskup:ex:del-rus.b}
\z\z

\ea\label{biskup:ex:del-pol}\ea\gll siedzi-e-ć\textsuperscript{IPFV}\\
sit-\textsc{th-inf}\\
\glt ‘to sit’\\‘to be sitting’\label{biskup:ex:del-pol.a}
\ex\gll po-siedzi-e-ć\textsuperscript{PFV}\\
\textsc{del}-sit-\textsc{th-inf}\\
\glt ‘to sit for a while’\hfill(Polish)\label{biskup:ex:del-pol.b}
\z\z

\ea\label{biskup:ex:del-cz}\ea\gll nés-t\textsuperscript{IPFV}\\
carry-\textsc{inf}\\
\glt ‘to carry’\\‘to be carrying’\label{biskup:ex:del.cz.a}
\ex\gll popo-nés-t\textsuperscript{PFV}\\
\textsc{del}-carry-\INF\\
\glt ‘to carry sth. a little’\hfill(Czech)\label{biskup:ex:del-cz.b}
\z\z

\noindent As to the meaning, the prefix delimits the eventuality denoted by the base predicate. Typically, it is a temporal delimitation, as in \REF{biskup:ex:del-rus.b} and \REF{biskup:ex:del-pol.b}, but a property scale and a path scale can be delimited, too, as shown in \REF{biskup:ex:del-cz.b} for the path scale.

Delimitative \textit{po-}verbs are standardly claimed to be perfectiva tantum, i.e. they do not form secondary imperfectives; see \citet[391]{Isačenko1962}, \citet[110]{Kopecny1962}, \citet[94]{Zaliznjak.Smelev1997}, and \citet[77]{Lazinski2020}. Thus, e.g. the delimitative verbs in \REF{biskup:ex:del2-rus.a}, \REF{biskup:ex:del2-pol.a}, and \REF{biskup:ex:del2-cz.a} cannot be imperfectivized and receive a progressive delimitative meaning, as shown by the second translations in examples \REF{biskup:ex:del2-rus.b}, \REF{biskup:ex:del2-pol.b}, and \REF{biskup:ex:del2-cz.b}. The \textit{po-}verb-\textsc{yva} forms in the (b) examples can only have the diminutive-iterative meaning, as demonstrated by the first translations.

\ea\label{biskup:ex:del2-rus}\ea\gll po-čit-a-t'\textsuperscript{PFV}\\
\textsc{del}-read-\textsc{th-inf}\\
\glt ‘to read for a while’ \label{biskup:ex:del2-rus.a}
\ex\gll po-čit-yva-t'\textsuperscript{IPFV}\\
\textsc{del}-read-\textsc{iter-inf}\\
\glt ‘to read from time to time’\\ \textit{Unavailable reading}: `to be reading for a while'\hfill(Russian)\label{biskup:ex:del2-rus.b}
\z\z

\ea\label{biskup:ex:del2-pol}\ea\gll po-płak-a-ć\textsuperscript{PFV}\\
\textsc{del}-cry-\textsc{th-inf}\\
\glt ‘to cry for a while’ \label{biskup:ex:del2-pol.a}
\ex\gll po-płak-iwa-ć\textsuperscript{IPFV}\\
\textsc{del}-cry-\textsc{iter-inf}\\
\glt ‘to cry from time to time’\\ \textit{Unavailable reading}: ‘to be crying for a while’\hfill (Polish)\label{biskup:ex:del2-pol.b}
\z\z

\ea\label{biskup:ex:del2-cz}\ea\gll popo-nés-t\textsuperscript{PFV}\\
\textsc{del}-carry-\textsc{inf}\\
\glt ‘to carry sth. a little’\label{biskup:ex:del2-cz.a}
\ex\gll popo-náš-e-t\textsuperscript{IPFV}\\
\textsc{del}-carry-\textsc{iter-inf}\\
\glt ‘to carry sth. from time to time’\\ \textit{Unavailable reading}: ‘to be carrying a little/for a while’\hfill (Czech)\label{biskup:ex:del2-cz.b}
\z\z

\noindent I leave attenuative \textit{po-}verbs like the Russian \textit{poprideržat'} ‘to hold gently’ aside in this article because the attenuative \textit{po-} and the delimitative \textit{po-} behave as two distinct elements morphosyntactically.\footnote{This contrasts with the semantic analysis by \citet{Souckova2004b, Souckova2004a} and \citet{Kagan2016}, who treat the two types of \textit{po-} prefixes as one and the same semantic element, measuring degrees on a scale. Note that an analysis with two distinct morphosyntactic \textit{po-}s does not preclude the possibility that the two prefixes have identical or very similar semantic properties.}  While the attenuative \textit{po-} selects perfective stems that are prefixed, delimitative \textit{po-}verbs are derived from the base, imperfective stems (see e.g. \citealt[391, 396]{Isačenko1962}, \citealt[101]{Zaliznjak.Smelev1997}, \citealt[398]{Petr1986}). Further, in Czech, the delimitative \textit{po-} often adds a dative reflexive argument that is licensed by the agent and is obligatory.\footnote{Consider e.g. \REF{biskup:ex:footnote3} and the discussion of \REF{biskup:ex:cough.b} in \sectref{biskup:sec:dimin-iter}. The most straightforward analysis would introduce the dative \textit{si} ‘self’ in the specifier of the delimitative \textit{po-} projection, where the argument is c-commanded by the agent placed in the specifier of VoiceP. \ea\label{biskup:ex:footnote3}\gll Strejda \minsp{*(} si) po-lyžoval a odjel domů.\\
uncle {} self \textsc{del}-skied and went home\\
\glt ‘My uncle skied for a while and went home.’\hfill (Czech)\z
} In contrast, attenuative \textit{po-}verbs can have the dative \textit{si} ‘self’ but it is never obligatory. In addition, the two types of \textit{po-}verbs behave differently with respect to the formation of secondary imperfectives. As discussed in the preceding paragraph, delimitative \textit{po-}verbs are considered to be perfectiva tantum. The authors mentioned there do not discuss whether or not attenuative \textit{po-}verbs can be imperfectivized but (at least some) \textit{po-}attenuatives form secondary imperfectives; e.g. Russian \textit{poprideržat'} ‘to hold gently’ derives the imperfective form \textit{popriderživat'} ‘to hold gently/to be holding gently’. \citet[96]{Tatevosov2009} and \citeauthor{KlimekJankowska.Blaszczak2022} (\citeyear{KlimekJankowska.Blaszczak2022}: 7--9, \citeyear{KlimekJankowska.Blaszczak2023}: ex. (85), (92)) place the Russian and Polish, respectively, attenuative \textit{pod-} below the imperfectivizing \textsc{-yva}. If the attenuative \textit{po-} behaves in the same way, then attenuative \textit{po-}verbs should be able to undergo secondary imperfectivization. Given that the attenuative \textit{po-} selects a perfective stem, it must belong to positionally restricted prefixes in terms of \citet{Tatevosov2009}, which means that it merges below the imperfectivizing \textsc{-yva}.

The remainder of the article is structured as follows. \sectref{biskup:sec:dimin-iter} argues that diminutive-iterative \textit{po-}verbs are derived from delimitative predicates. \sectref{biskup:sec:analysis} then offers a morphosyntactic and semantic analysis with the relevant derivational steps. \sectref{biskup:sec:conclusions} concludes the article.

\section{Diminutive-iterative verbs are derived from delimitative predicates} \label{biskup:sec:dimin-iter}

Recall from the preceding section that both delimitative and diminutive-iterative verbs contain the prefix \textit{po-} and include a delimited degree scale. Given this and the imperfectivizing and iterative effects of \textsc{-yva} in diminutive-iterative verbs, a natural idea is that diminutive-iteratives, as in \REF{biskup:ex:dimin-rus.b}, are derived by applying the imperfectivizing \textit{-yva} to the delimitative predicate, as in \REF{biskup:ex:dimin-rus.a}. 

\ea\label{biskup:ex:dimin-rus}\ea\gll po-lež-a-t' \\
\textsc{del}-lie-\textsc{th-inf}\\
\glt ‘to lie for a while’ \label{biskup:ex:dimin-rus.a}
\ex\gll po-lёž-iva-t'\\
\textsc{del}-lie-\textsc{iter-inf}\\
\glt ‘to lie from time to time’ \hfill (Russian)\label{biskup:ex:dimin-rus.b}
\z\z

\noindent Besides the morphological, semantic and aspectual arguments, there are also phonological facts which support such an analysis. Suffixes used in the derivation of diminutive-iteratives are identical with \textsc{-yva} allomorphs used in the “standard” secondary imperfectivization. Also phonological processes involved in the formation of diminutive-iterative \textit{po-}verbs are identical with phonological processes involved in the derivation of “standard” secondary imperfectives. Compare the diminutive-iterative example \REF{biskup:ex:dimin-rus1} with the secondary imperfectivization example in \REF{biskup:ex:dimin-rus2}. In both examples \textit{-yva} shifts the accent from the theme to the root vowel and in both roots, we also observe the vowel gradation (lengthening) from /o/ to /a/.\footnote{The accent is represented with the diacritic length mark.}  Consider also \REF{biskup:ex:dimin-rus}, which displays a vowel gradation in the root, too, and which also manifests the accent shift from the theme to the root.

\ea\label{biskup:ex:dimin-rus1}\ea\gll po-kol-ó-t'\textsuperscript{PFV}\\
\textsc{del}-prick-\textsc{th-inf}\\
\glt ‘to prick a little’ \label{biskup:ex:dimin-rus1.a}
\ex\gll po-kál-yva-t'\textsuperscript{IPFV}\\
\textsc{del}-prick-\textsc{iter-inf}\\
\glt ‘to prick from time to time’\hfill (Russian)\label{biskup:ex:dimin-rus1.b}
\z\z

\ea\label{biskup:ex:dimin-rus2}\ea\gll s-pros-í-t'\textsuperscript{PFV}\\
with-ask-\textsc{th-inf}\\
\glt ‘to ask’ \label{biskup:ex:dimin-rus2.a}
\ex\gll s-práš-iva-t'\textsuperscript{IPFV}\\
with-ask-\textsc{si-inf}\\
\glt ‘to ask’\\‘to be asking’\hfill (Russian)\label{biskup:ex:dimin-rus2.b}
\z\z

\noindent The Polish examples in \REF{biskup:ex:dimin-pol1} and \REF{biskup:ex:dimin-pol2} show that in formation of both diminutive-iteratives and ordinary secondary imperfectives, /j/ is inserted to block hiatus.

\ea\label{biskup:ex:dimin-pol1}\ea\gll po-pi-ć\textsuperscript{PFV}\\
\textsc{del}-drink-\textsc{inf}\\
\glt ‘to drink a little’ \label{biskup:ex:dimin-pol1.a}
\ex\gll po-pi-ja-ć\textsuperscript{IPFV}\\
\textsc{del}-drink-\textsc{iter-inf}\\
\glt ‘to drink from time to time’ \hfill (Polish) \label{biskup:ex:pol1.b}
\z\z

\ea\label{biskup:ex:dimin-pol2}\ea\gll wy-bi-ć\textsuperscript{PFV}\\
out-beat-\textsc{inf}\\
\glt ‘to kill off’ \label{biskup:ex:dimin-pol2.a}
\ex\gll wy-bi-ja-ć\textsuperscript{IPFV}\\
out-beat-\textsc{si-inf}\\
\glt ‘to kill off’ \\ ‘to be killing off’ \hfill (Polish) \label{biskup:ex:dimin-pol2.b}
\z\z

\noindent A hiatus-blocking process is also present in the Czech examples in \REF{biskup:ex:dimin-cz1} and \REF{biskup:ex:dimin-cz2}. This time, /v/ is inserted between the theme vowel \textit{-a} and the imperfectivizing \textit{-a} (according to the standard analysis, /v/ is the imperfectivizing suffix itself; see e.g. \citealt[194]{Karlik.etal1995}; consider also \citeauthor{Matushansky2009}'s \citeyear{Matushansky2009}: 397 unifying analysis of imperfectivizing suffixes in Russian, arguing that /v/ derives from an underlying back rounded yer). In addition, both examples also display a lengthening of the theme after adding the imperfectivizing suffix.

\ea\label{biskup:ex:dimin-cz1}\ea\gll po-kašl-a-t\textsuperscript{PFV} si\\
\textsc{del}-cough-\textsc{th-inf} self\\
\glt ‘to cough a little’ \label{biskup:ex:dimin-cz1.a}
\ex\gll po-kašl-á-va-t\textsuperscript{IPFV}\\
\textsc{del}-cough-\textsc{th-iter-inf}\\
\glt ‘to cough from time to time’ \hfill (Czech) \label{biskup:ex:dimin-cz1.b}
\z\z

\ea\label{biskup:ex:dimin-cz2}\ea\gll při-děl-a-t\textsuperscript{PFV}\\
at-do-\textsc{th-inf}\\
\glt ‘to fix’ \label{biskup:ex:dimin-cz2.a}
\ex\gll při-děl-á-va-t\textsuperscript{IPFV}\\
at-do-\textsc{th-si-inf}\\
\glt ‘to fix’ \\ ‘to be fixing’ \hfill (Czech) \label{biskup:ex:dimin-cz2.b}
\z\z

\noindent However, the standard literature does not adopt the analysis in which diminutive-iterative \textit{po-}verbs are derived from delimitative predicates (see e.g. \citealt{Isacenko1960}: 279--282, \citeyear{Isačenko1962}: 407, \citealt[600]{bis:Svedova1980}, \citealt[94, 104]{Zaliznjak.Smelev1997}). According to them, delimitatives like \REF{biskup:ex:dimin-rus.a} are perfectiva tantum, diminutive-iteratives like \REF{biskup:ex:dimin-rus.b} are imperfectiva tantum and the verbs belong to different Aktionsarten: delimitative and diminutive-iterative. The authors claim that diminutive-iter\-a\-tives are derived by circumfixation of \textit{po-} and \textsc{-yva} to the imperfective simplex predicate, i.e. to \textit{ležat'} ‘to lie’ in the case of \REF{biskup:ex:dimin-rus.b} (see \citealt[58]{Katny1994}ff. and \citealt[419]{Petr1986} for analogous claims with respect to Polish and Czech). This means that there is a strange coincidence. Both types of verbs have \textit{po-} and some sort of a delimitative/diminutive meaning and the perfective verbs with the delimitative \textit{po-} do not have a secondary imperfective counterpart and the imperfective verbs with \textit{po-} (and \textsc{-yva}) do not have a perfective counterpart. \citet[133--134]{Tatevosov2009} shows that \citeposst{Isacenko1960} arguments are not strong enough and argues that there is a clear (derivational) relation between the existence of delimitative \textit{po-}verbs and the existence of diminutive-iterative \textit{po-}verbs in Russian.

\newpage
The generative literature places delimitative \textit{po-} and the imperfectivizing \textsc{-yva} in different structural positions. The delimitative prefix is mostly higher than \mbox{\textsc{-yva}}; see e.g. \citet[271--272]{bis:Romanova2004}, \citet[437--438]{bis:Tatevosov2008} and \citeauthor{KlimekJankowska.Blaszczak2022} (\citeyear{KlimekJankowska.Blaszczak2022}: 9, \citeyear{KlimekJankowska.Blaszczak2023}: ex. (85), (92)) (but see also \citealt[408--409]{Souckova2004b}, who assumes circumfixation, and \citealt[377--381]{Jablonska2004}, who proposes two positions for \textit{po-}, below and above \textsc{-yva}). This placement has the advantage that it can account for why delimitative verbs are not (progressively) imperfectivized. However, it brings about two scope problems. First, it makes a false prediction with respect to morphological aspect. Given the position of the prefix being higher than \textsc{-yva}, one expects diminutive-iterative verbs to be perfective contrary to the facts; see \sectref{biskup:sec:intro-dim} again. Second, if the delimitative prefix scoped over \textsc{-yva}, one would expect an interpretation with the repetition of standard actions in a delimited/short time frame, again contrary to the facts. For instance, in example \REF{biskup:ex:del2-rus.b}, repeated here as \REF{biskup:ex:readrus}, one should receive a repetition of (normal) actions of reading in a short time frame.

\ea\label{biskup:ex:readrus}\gll po-čit-yva-t' \\
\textsc{del}-read-\textsc{iter-inf}\\
\glt ‘to read from time to time’ \hfill (Russian)
\z

\noindent However, \REF{biskup:ex:readrus} is interpreted as a repetition of short actions of reading. This indicates reversed scope properties, with \textsc{-yva} being higher than \textit{po-}. 

Generally, there are three possibilities for how to derive diminutive-iterative \textit{po-}verbs from \textit{po-} delimitatives, as schematized in \REF{biskup:ex:root}.

\ea\label{biskup:ex:root}\ea {[ \textit{po-} {\dots} [\textit{-yva} {\dots} [{\dots} $\sqrt{}$root]]]}\label{biskup:ex:root.a}
\ex {[ \textit{-yva} {\dots} [\textit{-po-} {\dots} [{\dots} $\sqrt{}$root]]]}\label{biskup:ex:root.b}
\ex {[ \textit{-yva} + \textit{po-} {\dots} [{\dots} $\sqrt{}$root]]]}\label{biskup:ex:root.c}
\z\z

\noindent The first possibility, as just discussed, makes wrong predictions with respect to interpretational possibilities and morphological aspect properties. Thus, since diminutive-iteratives are imperfective and since morphological aspect is determined by the last (highest) aspectual morpheme (e.g. \citealt[96]{Karcevski1927}, \citealt{Vinogradov1947}: 500, \citealt{Dostal1954}: 482, \citealt{Isačenko1962}: 416--418, \citealt{Zinova.Filip2015}: 601--602, \citealt{Tatevosov2020}: 28, \citealt{Zinova2021}: 36--38, \citealt{bis:Biskup2020}: 55--61), the second option, shown in \REF{biskup:ex:root.b}, is preferred over the first possibility in \REF{biskup:ex:root.a}. Note that the delimitative \textit{po-} cannot attach to the predicate after the imperfectivizing \textsc{-yva} because there are no exceptions to the perfectivizing effect of prefixation \citep[242]{bis:Smith1997}; there are only apparent exceptions (e.g. \citealt{Schuyt1990}: 292, \citealt{Zaliznjak.Smelev1997}: 67--68, \citealt{Zaliznjak2017}: 4--6).\footnote{\citet[605--607]{Zinova.Filip2015} argue that iterative \textit{pere-} and attenuative \textit{po-} do not have to perfectivize in Russian. They assume that imperfectives like \textit{perezapisyvat’} ‘to (be) rerecord(ing)’ can be derived by attaching \textit{pere-} to the imperfective \textit{zapisyvat’} ‘to (be) record(ing)’. Their supporting argument is however based on borrowings, which are known to be anomalous in various respects. Specifically, when the prefixes discussed attach to a borrowed biaspectual verb, the new verb is still biaspectual like \textit{perekvalificirovat’} ‘to requalify’. A comparison with other languages suggests that in Russian the verb is not adapted enough to the language system to be able to accept the perfectivizing effect of the prefix. In contrast, the Czech \textit{překvalifikovat} ‘to requalify’ is already perfective and there is also its imperfective counterpart \textit{překvalifikovávat}.}

Given that interpretational properties of diminutive-iterative verbs are correctly derived by \REF{biskup:ex:root.b}, in contrast to \REF{biskup:ex:root.a}, the second option is preferred over the derivation in \REF{biskup:ex:root.a} more generally. As to the circumfixation derivation in \REF{biskup:ex:root.c}, this option is generally disfavored, especially if both affixes exist independently (see e.g. \citealt{Marusic2023}), as is the case with diminutive-iterative \textit{po-}verbs. What is more, there is no reason to assume the special operation of circumfixation when \REF{biskup:ex:root.b} successfully derives the relevant facts.

Since diminutive-iterative \textit{po-}verbs cannot receive the progressive delimitative interpretation, only the iterative interpretation, as in \REF{biskup:ex:del2-rus.b}, \REF{biskup:ex:del2-pol.b} and \REF{biskup:ex:del2-cz.b}, I assume that the iterative \textsc{-yva} differs from the progressive \textsc{-yva} with respect to their structural position. The progressive \textsc{-yva} is generated below the delimitative \textit{po-} and the iterative \textsc{-yva} merges above the delimitative prefix, as illustrated in \REF{biskup:ex:root2} (see \citealt{bis:Ramchand2004}: 33, who proposes that \textsc{-yva} can occur in two different projections: Asp and Cuml). 

\ea\label{biskup:ex:root2} {[ \dots Iter \textsc{-yva} {\dots} [{\dots} Del \textit{po-} {\dots} [{\dots} Prog \textsc{-yva} {\dots} [{\dots} $\sqrt{}$root]]]]} \\
\z

\noindent An argument against diminutive-iteratives derived from delimitatives could be based on the fact that there are derivational chains with missing links (see e.g. \citealt{Souckova2004b}: 409). Specifically, in contrast to the Russian \textit{pokašljat'} ‘to cough a little’ and the Polish \textit{pokaszleć/pokasłać} ‘to cough a little’, Czech (and Slovak) do not have the middle step of the derivational chain \textit{pokašlat} (there is only \textit{pokašlat si}), as shown in \REF{biskup:ex:cough.b}.\footnote{The Polish situation is somewhat controversial but \textit{pokaszleć/pokasłać} can be found in \textit{Wielki słownik języka polskiego} \citep{Żmigrodzki2022}.}

\ea\label{biskup:ex:cough}\ea[]{\gll kašl-a-t \\
cough-\textsc{th-inf}\\
\glt ‘to cough’ \label{biskup:ex:cough.a}}
\ex[*] { po-kašl-a-t\\
\textsc{del}-cough-\textsc{th-inf}\label{biskup:ex:cough.b}}
\ex[]{\gll po-kašl-á-va-t\\
\textsc{del}-cough-\textsc{th-iter-inf}\\
\glt ‘to cough from time to time’ \hfill (Czech) \label{biskup:ex:cough.c}}
\z\z

\noindent Such an argument, however, is valid only if we assume a lexicalist framework, in which verbs like diminutive-iteratives are derived from complete word forms (see e.g. the definition of derivational chains in \citealt{Zinova.Filip2015}: 601-602). In morphosyntactic approaches like the one assumed here verbs are derived incrementally, morpheme by morpheme, in a bottom-up fashion, not by attaching the imperfectivizing \textit{-va} in \REF{biskup:ex:cough.c} to the complete verb \textit{*pokašlat}.

The same also holds for other morphemes in derivational chains, e.g. prefixes. Hence, \textit{vypisat'} ‘to excerpt’ in \REF{biskup:ex:vypisat.a} is not derived by prefixation of \textit{vy-} to the complete verb \textit{pisat'} ‘to write’ since in the morphosyntactic approach assumed here, prefixes typically merge before the tense morpheme (infinitival \textit{-t'} in \REF{biskup:ex:vypisat}) and in certain cases also before the theme marker.\footnote{An analogous reasoning also applies to disappearing elements, e.g. \textit{si} ‘self’ in \REF{biskup:ex:footnote7}. One might argue that in cases like \REF{biskup:ex:footnote7}, incrementality is violated because \textit{si} does not have to be present in \REF{biskup:ex:footnote7.c}, in contrast to \REF{biskup:ex:footnote7.b}. Again, since the verb in \REF{biskup:ex:footnote7.c} is not derived from the complete verb with \textit{si} in \REF{biskup:ex:footnote7.b}, there cannot be disappearing \textit{si} in \REF{biskup:ex:footnote7.c}.
\ea\label{biskup:ex:footnote7}\ea\gll křič-e-t\\
shout-\textsc{th-inf}\\
\glt ‘to (be) shout(ing)’\label{biskup:ex:footnote7.a}
\ex\gll po-křič-e-t \minsp{*(} si)\\
\textsc{del}-shout-\textsc{th-inf} {} self\\
\glt ‘to shout a little for oneself’\label{biskup:ex:footnote7.b}
\ex\gll po-křik-ova-t \minsp{(} si)\\
\textsc{del}-shout-\textsc{iter-inf} {} self\\
\glt ‘to shout from time to time’ \hfill (Czech) \label{biskup:ex:footnote7.c}
\z\z}

\ea\label{biskup:ex:vypisat}\ea\gll vy-pis-a-t'\\
out-write-\textsc{th-inf}\\
\glt ‘to excerpt’ \label{biskup:ex:vypisat.a}
\ex\gll pis-a-t'\\
write-\textsc{th-inf}\\
\glt ‘to write’\label{biskup:ex:vypisat.b}
\z\z

\noindent To conclude this section, the imperfectivizing marker \textsc{-yva} can appear in different structural positions and consequently, it can spell out distinct semantic properties.

\section{Analysis} \label{biskup:sec:analysis}

In what follows, I elaborate the proposal from the preceding section, first from the semantic perspective, then from the syntactic point of view.

\subsection{Combining \textit{po-} and the verb stem} \label{biskup:sec:combiningpo}

According to \citet[392]{Isačenko1962}, there are several restrictions on the formation of delimitative \textit{po-}verbs. The restrictions are found not %not found 
only in Russian, as shown in \REF{biskup:ex:stem-pol}, \REF{biskup:ex:stem-cz} and \REF{biskup:ex:stem-rus}. At least partially, they can be accounted for if it is assumed that the delimitative prefix needs a scale (for scale (degree) approaches to \textit{po-} see \citealt{bis:Filip2000}, \citeyear{Filip2003}; \citealt{Jablonska2004,Souckova2004b,Souckova2004a}; \citealt{Kagan2016} and \citealt{Zinova2021}). 

\ea[*]{\gll po-zyskać\\
\textsc{del}-win\\
\glt Intended: ‘to win for a while’ \hfill (Polish)\label{biskup:ex:stem-pol}}
\z

\ea[*]{\gll po-bodnout\\
\textsc{del}-stab\\
\glt Intended: ‘to stab for a while’ \hfill (Czech)\label{biskup:ex:stem-cz}}
\z

\ea[*]{\gll po-stoit'\\
\textsc{del}-cost\\
\glt Intended: ‘to cost for a while’ \hfill (Russian)\label{biskup:ex:stem-rus}}
\z

\noindent Given that achievements denote a momentaneous change of state, there is no (protracted) scale that could be delimited by the prefix and cases like \REF{biskup:ex:stem-pol} are ungrammatical.\footnote{The so-called degree achievements are known for their special behavior; they provide an appropriate change-of-state scale.}  In the same vein, since semelfactive verbs denote punctual eventualities, the delimitative \textit{po-} cannot measure them and examples like \REF{biskup:ex:stem-cz} are ungrammatical as well.

\largerpage[-1]
An interesting case is stative predicates like the one in \REF{biskup:ex:stem-rus}, which also cannot be prefixed with the delimitative \textit{po-}. The ungrammatical status cannot be ascribed to the fact that a temporal extent scale is missing because the base predicate of such states can be modified by durative adverbials, as is the case e.g. with Russian \textit{znat' kogo tri dnja} ‘to know somebody for three days’. The reason also cannot be the homogeneity requirement of delimitative \textit{po-} (\citealt[61]{bis:Filip2000}, \citeyear[91]{Filip2003}; \citealt{Mehlig2006}: 247) because states are homogeneous. Atelicity does not play a role either because states are atelic and other atelic predicates – e.g. activities like the Polish \textit{płakać} ‘to cry’ – can be prefixed with delimitative \textit{po-}. So, one might argue that the problem lies in the concept of change because stative predicates do not entail a change, in contrast to dynamic predicates (activities, accomplishments and achievements). However, this reasoning is not correct, either%too
, since there are also states that can be prefixed with the delimitative \textit{po-}; consider the examples \REF{biskup:ex:stem-rus2}, \REF{biskup:ex:stem-pol2} and \REF{biskup:ex:stem-cz2}. 

\ea Russian\label{biskup:ex:stem-rus2}\\
\begin{tabularx}{.95\textwidth}{l@{~~}X@{~~}l@{~~}X@{~~}l@{~~}X}
     a.&po-stojat'&b.&po-sidet'&c.&po-spat'\\
     & \textsc{del}-stand&&\textsc{del}-sit&&\textsc{del}-sleep\\
\end{tabularx}
\z

\ea Polish\label{biskup:ex:stem-pol2}\\
\begin{tabularx}{.95\textwidth}{l@{~~}X@{~~}l@{~~}X@{~~}l@{~~}X}
     a.&po-stać&b.&po-siedzieć&c.&po-spać\\
     & \textsc{del}-stand&&\textsc{del}-sit&&\textsc{del}-sleep\\
\end{tabularx}
\z

\ea Czech\label{biskup:ex:stem-cz2}\\
\begin{tabularx}{.95\textwidth}{l@{~~}X@{~~}l@{~~}X@{~~}l@{~~}X}
     a.&po-stát&b.&po-sedět&c.&po-spat\\
     & \textsc{del}-stand&&\textsc{del}-sit&&\textsc{del}-sleep\\
     &‘to~stand for a~while’&&`to sit for a while'&&`to sleep for a while'\\
\end{tabularx}
\z

% \ea po-stojat'\\
% \textsc{del}-stand\label{biskup:ex:stem-rus2.a}
% \ex po-sidet'\\
% \textsc{del}-sit\label{biskup:ex:stem-rus2.b}
% \ex po-spat'\\
% \textsc{del}-sit \label{biskup:ex:stem-rus2.c}
% \z\z

% \ea\label{biskup:ex:stem-pol2}\ea po-stać \\
% \textsc{del}-stand\label{biskup:ex:stem-pol2.a}
% \ex po-siedzieć\\
% \textsc{del}-sit\label{biskup:ex:stem-pol2.b}
% \ex po-spać\\
% \textsc{del}-sit\label{biskup:ex:stem-pol2.c}
% \z\z

% \ea\label{biskup:ex:stem-cz2}\ea\gll po-stát\\
% \textsc{del}-stand\\
% \glt All: ‘to stand for a while’ \label{biskup:ex:stem-cz2.a}
% \ex\gll po-sedět\\
% \textsc{del}-sit\\
% \glt ‘to sit for a while’\label{biskup:ex:stem-cz2.b}
% \ex\gll po-spat	si\\
% \textsc{del}-sit self\\
% \glt ‘to sleep for a while’ \label{biskup:ex:stem-cz2.c}
% \z\z


\noindent The unprefixed verbs in \REF{biskup:ex:stem-rus2}--\REF{biskup:ex:stem-cz2} belong to the class of “interval statives” (\citealt{Dowty1979}: 173--180) and differ from “static states” (\citealt{Bach1986}: 6) like ‘to know’, ‘to own’ and ‘to cost’, which are less dynamic and do not accept the delimitative \textit{po-} in North Slavic. \citet{Maienborn2003, Maienborn2005} analyzes dynamic states like ‘to stand’ as “Davidsonian states” because they refer to eventualities in the sense of Davidson, so they introduce a Davidsonian event argument. In contrast, static states belong to “Kimian states” in her analysis. They do not have a Davidsonian event variable but introduce a specific Kimian-state referential argument. In fact, the German modifier \textit{ein bisschen} ‘a little bit’ distinguishing Davidsonian states from Kimian states in her eventuality diagnostic \citep[297--299]{Maienborn2005} behaves like delimitative \textit{po-} with respect to grammaticality judgements. Here I follow Maienborn’s proposal and assume that delimitative \textit{po-} selects a predicate with a Davidsonian event argument (that in addition has some scalar structure, as discussed above). 


\largerpage[-1]
According to \citeauthor{bis:Filip2000} (\citeyear{bis:Filip2000}: 61--66, \citeyear{Filip2003}: 89--90), delimitative \textit{po-} applies to a homogeneous predicate and contributes an extensive measure function, which is contextually specified and meets or falls short of some contextually determined expectation value. \citeauthor{Souckova2004b} (\citeyear{Souckova2004b}: 410, \citeyear{Souckova2004a}: 73) modifies Filip’s proposal and argues that the measure function only applies to events, as shown in \REF{biskup:ex:relsmall}.\footnote{The proposal that the measure function applies to events can be already found in \citet[362--363]{Pinon1994}. \citet{Souckova2004b, Souckova2004a} also argues that the prefix can also apply to non-homogeneous predicates since she unifies delimitative \textit{po-} and the attenuative \textit{po-}. Given that I keep the two prefixes apart (see \sectref{biskup:sec:delim}), I assume that the delimitative \textit{po-} applies to homogeneous predicates. The attenuative \textit{po-} then probably only applies to non-homogeneous predicates, as in the case of the Russian \textit{poprideržat'} ‘to hold gently’.}

\ea\label{biskup:ex:relsmall}
\sib{\textit{po-}} $= \lambda P \lambda e [P(e) \wedge m(e) =
 c\textsubscript{relatively.small}]$
\z

\noindent \textit{P} is a variable over predicates, \textit{m} stands for the extensive measure function applied to an event and \textit{c} means that its value is contextually determined. The function \textit{m} measures events that contain some scalar structure, concretely, it measures the degree of change on the appropriate scale and it can apply to various dimensions (types of scales). It depends on lexical properties of the particular verb, which dimension -- if at all; recall the discussion of achievements and semelfactives in \REF{biskup:ex:stem-pol} and \REF{biskup:ex:stem-cz} -- is accessible. For instance, in the case of motion verbs, there is a scale of progress along the path; in degree achievements, there is an increase in the degree on a property scale; but in most cases, the prefix is applied to a time scale. Note, however, that it is not possible to reduce all delimitative cases to the time scale and that particular scales do not have to coincide with respect to the degree of change. For instance, in the case of Czech \textit{poponést} ‘to carry sth. a little’ in \REF{biskup:ex:del-cz.b}, it is the progress on the appropriate path that is measured and the relatively small value of this progress (the short path) can be in contrast to the duration of the carrying eventuality, which under appropriate circumstances can even be very long.

As just mentioned, \REF{biskup:ex:relsmall} states that the value of the measure function is relatively small in the context. I use the equal relation in \REF{biskup:ex:relsmall} instead of the less-than-or-equal relation (in contrast to e.g. \citealt{Kagan2015}: 47, \citeyear{Kagan2016}: 310; \citealt{KlimekJankowska.Blaszczak2023}: ex. (68)) since it directly brings about the quantization property of \textit{po-}delimitatives. necessary for iteration, as discussed in the following section. More concretely, if a delimited event of crying takes five minutes in the specific context, i.e. the value of the measure function is contextually determined to be equal to five minutes, then there will be no proper part of the event that also falls in the denotation of this crying \textit{po-}predicate. In contrast, if the less-than-or-equal relation were used in \REF{biskup:ex:relsmall}, there could be a proper part – e.g. crying for just four minutes – falling in the denotation of the crying \textit{po-}predicate.

Thus, building on Součková’s proposal, the meaning of the Polish \textit{popłaka(ć)} ‘to cry for a while’ -- after applying the delimitative prefix \textit{po-} to the predicate \textit{płaka(ć)} ‘to cry’ -- will look like \REF{biskup:ex:poplaka}. 

\ea\label{biskup:ex:poplaka}
\sib{\textit{po-płaka-}} $= \lambda e [\textsc{cry}(e) \wedge m(e) =
 c\textsubscript{relatively.small}]$
\z

\noindent Given the meaning in \REF{biskup:ex:relsmall}, with \textit{P} as a predicate over events and the measure function applied to an event, it is obvious that delimitative \textit{po-} cannot combine with states like \textit{stoit'} ‘to cost’ in \REF{biskup:ex:stem-rus}, which only have the Kimian-state referential argument.

\subsection{Combining the delimitative \textit{po-}predicate and the iterative \textsc{-yva}} \label{biskup:sec:combining}

According to the proposal in \sectref{biskup:sec:dimin-iter}, diminutive-iteratives are derived from delimitative predicates in the way that the eventuality delimited/measured by \textit{po-} is iterated by means of \textsc{-yva}. In other words, delimitative \textit{po-} is responsible for individuation and the iterative \textsc{-yva} then for pluralization. Thus, given that for iteration and counting, discrete elements are necessary, the question arises as to which concept is responsible for the individuation here. Is quantization sufficient or is telicity necessary as well? According to \citet[91]{Filip2003}, delimitative \textit{po-} (``attenuative'' in her terms) makes predicates not only quantized, analogously to measure functions like \textit{a} (\textit{relatively}) \textit{small quantity} and \textit{a few}, but also telic given her definition of telicity based on atomicity (see \citealt[60--61]{Filip2003}). However, if the standard adverbial test is a reliable diagnostic for (a)telicity, then delimitative \textit{po-}verbs must be atelic because they are compatible with durative adverbials like ‘an hour (long)’ and incompatible with time-span adverbials like ‘in an hour’ in the relevant reading. In the light of grammatical diminutive-iteratives like \textit{polёživat'} ‘to lie from time to time’ in \REF{biskup:ex:pref-rus}, that in turn means that quantization brought about by an extensive measure function like the delimitative \textit{po-} is sufficient for applying the iterative \textit{-yva-} and that telicity is not a necessary condition in this case.

As mentioned in \sectref{biskup:sec:intro-dim}, the iterative marker brings about an unspecified number of instances of the particular eventuality and the cardinality of repetitions is contextually determined. Since it is difficult to determine the smallest number of repetitions here, I assume the weakest position and use the meaning of plurals as a base. This means that the cardinality of the iterated eventuality is greater than one, as shown in \REF{biskup:ex:tricettri}, where $|e|$ stands for the number of atomic events.

\ea\label{biskup:ex:tricettri} 
\sib{\cnst{iter}} $= \lambda P \lambda e[P(e) \wedge e = \sigma e' [P(e') \wedge e' \subset e \wedge |e| > 1] \wedge \forall e' [\cnst{atom}(e') \rightarrow \neg \exists e''[P(e'') \wedge e'' \subset e \wedge \cnst{atom}(e'') \wedge \tau (e') \supset\subset \tau(e'')]]]$
\z

\noindent For deriving pluralities, usually, \citeposst{bis:Link1983} σ-operator and the ∗-operator are used. Therefore, for the iterative \textsc{-yva}, I use Kratzer’s proposal (\citealt{Kratzer2008}: 296, see also \citealt{Boneh.Doron2008}: ex. \REF{biskup:ex:relsmall} and  \citealt{Ferreira2016}: 358), according to which the sum of all events $e'$ that are proper parts of the event $e$ and have the property $P$ is identical to $e$. Since it brings about a weak notion of plurality, with singularities as special cases, I also use the conjunct $|e| > 1$ in \REF{biskup:ex:tricettri}, as discussed above.

The iterated events are not temporally adjacent, e.g. the meaning of the Polish \textit{popłakiwać} ‘to cry from time to time’ in \REF{biskup:ex:pref-po} is characterized by \citet{Żmigrodzki2022} as \textit{płakać z przerwami} ‘to cry with pauses’ (see also \citealt{Katny1994}: 67). For this reason, I add to the meaning in \REF{biskup:ex:tricettri} the restriction on temporal traces. Specifically, for every atomic proper part $e'$ of the event $e$, it holds that there is no atomic subevent $e''$ with the property $P$ that is a proper part of $e$ and whose temporal trace abuts with the temporal trace of $e'$. Under the assumption that the abut relation (precluding any contact) is stronger than the overlap relation, it holds that if the temporal traces of $e'$ and $e''$ do not abut, then they also do not overlap. The size of the time interval between $e'$ and $e''$ is not defined here since it depends on the lexical meaning of the appropriate predicate and on the context. Moreover, intervals between the particular subevents can be of different sizes.

I do not use the classical non-overlap condition by \citet[256]{Lasersohn1995} (see also e.g. \citealt{Wood2007}: 126), with the function $f$ standing for temporal, spatio-temporal or participant-based distributivity, since in the case of diminutive-itera\-tive \textit{po-}verbs, participants and spaces can overlap. As an illustration, consider example \REF{biskup:ex:polish}, in which the referent of the expression \textit{Kasia} is identical for the crying subevents.\footnote{With plural subjects, distributive readings are possible but not necessary.} The crying subevents of example \REF{biskup:ex:polish} also can (but do not have to) happen in an identical space.

\ea\label{biskup:ex:polish}\gll Kasia cały dzień po-płak-iwa-ł-a.\\
Kasia whole day \textsc{del}-cry-\textsc{iter-ptcp-f}\\
\glt ‘Kasia cried repeatedly the whole day.’\hfill (Polish)
\z

\noindent To receive the separated-in-time reading, \citet[254]{Lasersohn1995} adds a betweenness condition to the non-overlap condition, which introduces a time that intervenes between temporal traces of the singular events. I do not follow his proposal because I assume that the time variable is introduced by the aspectual head later in the derivation. Instead of the betweenness and non-overlap conditions, I use the abut condition, as stated in \REF{biskup:ex:tricettri}.

When the meaning of the iterative operator is applied to the meaning of the delimitative predicate \textit{popłaka(ć)} in \REF{biskup:ex:poplaka}, we obtain a diminutive-iterative predicate over events, as demonstrated in \REF{biskup:ex:poplakiwa}.

\ea\label{biskup:ex:poplakiwa}
\sib{\textit{popłakiwa-}} $= \lambda e[*\textsc{cry}(e) \wedge m(e) = c\textsubscript{relatively.small} \wedge e = \sigma e'[*\textsc{cry}(e') \wedge m(e') = c\textsubscript{relatively.small} \wedge e' \subset e \wedge |e| > 1] \wedge \forall e' [\cnst{atom}(e') \rightarrow \neg \exists e'' [*\textsc{cry}(e'') \wedge m(e'') = c\textsubscript{relatively.small} \wedge e'' \subset e \wedge \cnst{atom}(e'') \wedge \tau (e') \supset\subset \tau(e'')]]]$
\z

\noindent Thus, having established the meaning of iterative \textsc{-yva}, let us now look at the lower instantiation of the marker, progressive \textsc{-yva}.

\subsection{The progressive \textsc{-yva} and imperfectivity} \label{biskup:sec:the-progressive}

As already mentioned, when \textsc{-yva} attaches to a delimitative predicate, the form cannot have the progressive meaning. It can only receive an iterative interpretation; consider e.g. the Russian \textit{počityvat'} ‘to read from time to time’ in \REF{biskup:ex:del2-rus.b} again. This results from the splitting of \textsc{-yva} into two different syntactic positions and from the positioning of the delimitative \textit{po-} between them, as discussed in \sectref{biskup:sec:dimin-iter}. So, what is the meaning of the progressive \textsc{-yva}?

Progressivity is often defined in terms of partitivity (\citealt[171--175, 213]{Filip1999} and references therein), as is (secondary) imperfectivity (e.g. \citealt{bis:Lazorczyk2010}: 134--139). Although progressivity and (Slavic) imperfectivity are close notions (\citealt{Zucchi1999}: 200), they are not identical; see e.g. \citet[33]{bis:Comrie1976} and \citet[92]{Dahl1985}. Recall also from \sectref{biskup:sec:combining} that the aspectual projection, which is going to encode (im)perfectivity in the current proposal, occurs in a higher syntactic position, hence it is not identical to the progressive projection spelled out by the progressive \textit{-yva-}. Thus, since the progressive brings about an internal part of the eventuality, I assume the (for simplicity extensional) meaning in \REF{biskup:ex:prog} for the progressive \textsc{-yva}. 

\ea\label{biskup:ex:prog}\
\sib{\cnst{prog}} $= \lambda P \lambda e' \exists e [P(e) \wedge e' < e]$
\z

\noindent It is based on \citeauthor{Krifka1992} (\citeyear{Krifka1992}: 47) progressive operator but uses the proper-part-of relation instead of just part-of relation. The reason is that we need to exclude the possibility that the event culminates with the progressive \textsc{-yva}.\footnote{\citet{Filip2005} uses the part-of relation for the meaning of the imperfective morphological aspect since it can also derive the meaning of general-factuals, which can refer to a culminated eventuality. Also \citet{bis:Lazorczyk2010} uses the part-of relation in the meaning of her secondary imperfective operator.}

Given the proposal that there are two different syntactic positions (Prog and Iter; see \REF{biskup:ex:root2} again) with distinct meanings that are spelled out as \textsc{-yva}, the question arises whether the progressive \textsc{-yva} can co-occur with the iterative \textsc{-yva}. It is possible to test it with Czech since it allows combinations of more \textsc{-yva} markers. Consider example \REF{biskup:ex:davaval}, in which the imperfectivizing \textit{-vá} is attached to the unprefixed perfective stem \textit{dá-} and then another \textit{-(y)va} marker is adjoined, forming the imperfective \textit{dávával}.\footnote{The imperfective \textit{dáva-(t)} can have either the progressive or the iterative interpretations, hence I gloss \textit{-vá} with \textsc{prog/iter}.}

\ea\label{biskup:ex:davaval}\gll \minsp{(*} Včera) dá-vá-va-l peníze chudým.\\
{} yesterday 	give-\textsc{prog/iter-hab-ptcp} money poor.\textsc{adj.dat.pl}\\
\glt (Intended:) ‘(Yesterday) he had the habit of giving money to the poor.’\\
\z

\noindent Although the lower \textit{-vá} can bring about the progressive or the iterative interpretation, the higher \textit{-va} can only bring about the habitual meaning in \REF{biskup:ex:davaval}. Therefore, the sentence cannot be interpreted non-habitually. The ungrammatical status of the adverbial ‘yesterday’ indicates that an episodic reading is impossible. If an iterative adverbial is used, the sentence must also receive a habitual interpretation, as shown by the translation in \REF{biskup:ex:davavalhab}.

\ea\label{biskup:ex:davavalhab}\gll Dá-vá-va-l peníze chudým dvakrát.\\
give-\textsc{prog/iter-hab-ptcp} money to.poor twice\\
\glt ‘He had the habit of giving money to the poor twice.’\\
\z

\noindent For instance, a person could have the habit of giving money to the poor daily at 9 a.m. and 7 p.m. In contrast, it is not possible that there were only two occasions of giving money to the poor in total. Thus, the outer \textsc{-yva} in cases like \REF{biskup:ex:davavalhab} always instantiates a habitual operator that is structurally higher than the iterative and progressive suffixes.\footnote{For more on the habitual \textsc{-yva}, see \citet{Filip.Carlson1997}, \citet[321--340]{Esvan2007}, \citet{Berger2009}, \citet[132--135]{Nadenicek2011}, \citet{Nubler2017}, and also \citet{Biskup2021}%Biskup (to appear)
, who shows that the habitual \textsc{-yva} differs from the secondary imperfective \textsc{-yva} -- the progressive and iterative \textsc{-yva} in the current approach -- in phonological, aspectual and interpretational properties. Interpretationally, the habitual \textsc{-yva} brings about a (generic) quantificational semantics in contrast to the pluractional iterative \textsc{-yva}. Since the habitual \textsc{-yva} is beyond the scope of this article, I will not discuss it any further here.}

The incompatibility of the iterative and progressive \textsc{-yva} has a semantic reason. As discussed in \sectref{biskup:sec:combining}, the iterative operator applies to quantized predicates. However, the progressive operator derives a homogeneous (i.e. cumulative and divisive) predicate, which cannot serve as an input for the iterative \textsc{-yva}. According to \citet[137--138]{bis:Lazorczyk2010}, the secondary imperfective operator takes a telic predicate and returns a homogeneous subpart of it (see also \citealt{Filip1999}: 167, who argues that progressive sentences are cumulative).\footnote{See also \citet[489]{Tatevosov2015}, who builds on \citet[346]{PaslawskaStechow2003} and treats the imperfectivizing \textsc{-yva} as an eventizer.}  Unsurprisingly, this behavior of the progressive operator corresponds with morphological aspect properties; we saw in \sectref{biskup:sec:intro-sec} that the imperfectivizing \textsc{-yva} selects a perfective stem and derives an imperfective predicate.\footnote{Note that despite the fact that the iterative \textsc{-yva} and the progressive \textsc{-yva} cannot co-occur, sentences with a diminutive-iterative verb can receive a simultaneous interpretation when the reference time (expressed e.g. by the temporal clause) is included in the event time, as in \REF{biskup:ex:kiedyfootnote}, based on \REF{biskup:ex:polish}. 
\ea\label{biskup:ex:kiedyfootnote}\gll Kiedy wszedł do pokoju, Kasia po-płak-iwa-ł-a.\\
when came.3\textsc{sg.m} into room Kasia \textsc{del}-cry-\textsc{iter-ptcp-f}\\
\glt ‘When he came into the room, Kasia was crying a little.’ \\
\z } 

Consequently, since the progressive \textsc{-yva} selects a perfective, telic stem, the \textit{po-}verbs \textit{popisyvat'} ‘to write from time to time’, \textit{popłakiwać} ‘to cry from time to time’	and \textit{posedávat} ‘to sit from time to time’ cannot be derived in the way schematized in \REF{biskup:ex:tricetdevet}, i.e. as “delimited progressives”. The reason for this is that in derivations like \REF{biskup:ex:tricetdevet}, in contrast to what was just said, the progressive \textsc{-yva} would have to apply to stems that are imperfective, homogeneous and atelic: ‘to write’, ‘to cry’ and ‘to sit’.

\ea\label{biskup:ex:tricetdevet}\ea[*]{[po-[pis-yva]]-t'\\
\textsc{del}-write-\textsc{prog-inf}\hfill (Russian) \label{biskup:ex:tricetdevet.a}}
\ex[*]{[po-[płak-iwa]]-ć\\
\textsc{del}-cry-\textsc{prog-inf} \hfill (Polish) \label{biskup:ex:tricetdevet.b}}
\ex[*]{[po-[sed-á-va]]-t\\
\textsc{del}-sit-\textsc{th-prog-inf} \hfill (Czech) \label{biskup:ex:tricetdevet.c}}
\z\z

% \ea vkopirovat
% \z

\noindent Another alternative would be to assume that the progressive reading of \textit{po-}verbs with \textsc{-yva} is excluded because the delimitative \textit{po-} applies first and makes the predicate quantized and individuated, with the consequence that proper parts of the denoted event are not accessible to the progressive \textsc{-yva} (which applies as second). This proposal, however, would be very restricted because it could only work in the case of an incremental theme (see \citealt{Filip2005}: 273). Let us test it with other prefixes. 

\ea\label{biskup:ex:ctyricet}\gll Pavel teď vy-pis-uj-e / o-pis-uj-e celou přednášku.\\
Pavel now out-write-\textsc{prog-3.sg} {} about-write-\textsc{prog-3.sg} whole talk\\
\glt ‘Pavel is excerpting/copying the whole talk right now.’ \hfill (Czech)
\z

\noindent Although there is a secondary imperfective predicate with prefixes making the base verb quantized, combined with a quantified incremental theme in \REF{biskup:ex:ctyricet}, the progressive reading is available. Thus, this alternative explanation of the impossibility of the progressive delimitative interpretation also does not work.

At this point, the question arises as to how the difference between progressivity and imperfectivity is modelled in the current approach. The meaning of the progressive operator in \REF{biskup:ex:prog} shows that progressivity concerns the internal structure of eventualities. As to the morphological aspect, I make the standard assumption that it concerns temporal properties of eventualities, i.e. relates the event time to the reference time via the inclusion relation (see e.g. \citealt{PaslawskaStechow2003}: 322). Concretely, in the case of the imperfective aspect, the reference time is included in the event time.

\subsection{Morphosyntactic derivation} \label{biskup:sec:morphosyntactic}
\largerpage
Building on \citeauthor{bis:Biskup2019} (\citeyear{bis:Biskup2019}: 36--42, \citeyear{bis:Biskup2020}), I assume that the value of the morphological aspect of diminutive-iterative \textit{po-}verbs is determined in the aspectual head via the operation Agree. Combining the structure in \REF{biskup:ex:root2} with the proposal that all aspectual markers are separated from the aspectual interpretation of the aspectual head \citep{Biskup2021}%(Biskup to appear)
, we receive \REF{biskup:ex:ctyricetjedna}, with parts relevant to our discussion.\footnote{With the aspectual interpretation of the Asp head, the relation between the reference time and the event time is meant, as discussed in the last paragraph of \sectref{biskup:sec:the-progressive}. This interpretation is different from the semantics of particular aspectual markers, like prefixes, the semelfactive and habitual suffixes and the iterative and progressive \textsc{-yva}, as discussed in \sectref{biskup:sec:combining} and \sectref{biskup:sec:the-progressive} (see also \citealt{PaslawskaStechow2003}, \citealt{Groenn2004}, \citealt{bis:Tatevosov2011}, \citeyear{Tatevosov2015}).}

\ea\label{biskup:ex:ctyricetjedna}
[Asp\textsubscript{[Asp:?]} ... [Iter \textsc{-yva}\textsubscript{[IPFV]} ... [Del \textit{po-}\textsubscript{[PFV]} ... [Prog \textsc{-yva}\textsubscript{[IPFV]} ... [√root]]]]]
\z

\noindent It shows that aspectual markers like the iterative \textsc{-yva}, the delimitative \textit{po-} and the progressive \textsc{-yva} bear an aspectual feature -- either with the value [perfective] or [imperfective] -- which can value the unvalued aspectual feature of the aspectual head.

Concerning the aspectual projection, it is standardly placed above the projection introducing the agent (see \citealt{BabkoMalaya2003}, \citealt{Blaszcza.KlimekJankowska2012}, \citealt{Gribanova2015}). This placement is also supported by the fact that although Russian \textit{-nie} nominals are aspectless (e.g. \citealt{Schoorlemmer1995}, \citealt{bis:Gehrke2008}), they can have an agent argument, as demonstrated by the agent-oriented modifier in \REF{biskup:ex:agent.a} and the agentive \textit{by-}phase in \REF{biskup:ex:agent.b}.\footnote{Regarding the aspectless status of Russian \textit{-nie} nominals, a reviewer asks whether \REF{biskup:ex:agent.a} can have a durative modifier. Given that \textit{for-}adverbials and \textit{in-}adverbials diagnose (a)telicity rather than (im)perfectivity, this test is not effective.  }

\ea\label{biskup:ex:agent}\ea\gll umyšlennoe prestuplenie\\
deliberate crime\\
\glt ‘a wilful crime’ \label{biskup:ex:agent.a}
\ex\gll soveršenie prestuplenija licom{\dots} \\
perpetration crime.\textsc{gen.sg} person.\textsc{ins.sg}\\
\glt ‘a perpetration of the crime by a person’\\\hfill (Russian; \citealt{Biskup2021}%Biskup to appear
: ex. (69)) \label{biskup:ex:agent.b}
\z\z

\noindent It has been argued that the secondary imperfective suffix merges inside the verbal domain below the agent (see e.g. \citealt{bis:Romanova2004}: 272 and \citealt{Tatevosov2015}: 488 for Russian, \citealt{bis:Kwapiszewski2022} for Polish and \citealt{Biskup2021} %Biskup to appear 
for Czech). However, given the splitting of \textsc{-yva} into the iterative and progressive \textsc{-yva} in the current proposal, we need to know more about the positioning of the agent. There are agent nominalizations ending in \textit{-tel’} in Russian, \textit{-ciel} in Polish and \textit{-tel} in Czech that can contain the imperfectivizing \textsc{-yva}. Semantically, the suffixes \textit{-tel’}, \textit{-ciel} and \textit{-tel} (and others, like the agentive version of the Czech \textit{-č} in \REF{biskup:ex:pojidat}) relate to the projection containing the agent (external) argument since they introduce an entity -- predominantly, a person -- that carries out the action denoted by the predicate to which they are attached.\footnote{In the traditional terminology, nouns in \textit{-tel} are called \textit{nomina agentis}.} Thus, based on \citeauthor{Baker.Vinokurova2009}'s (\citeyear{Baker.Vinokurova2009}: 531) analysis of nominalizing affixes like the English \textit{-er}, I consider the morphemes \textit{-tel’}, \textit{-ciel} and \textit{-tel} to be nominal versions of the agentive Voice head. What is crucial for us is that the imperfectivizing \textsc{-yva} is always closer to the root than the agentive nominalizing suffix, as demonstrated in \REF{biskup:ex:ctyritri.a}.

\ea\label{biskup:ex:ctyritri}\ea\gll do-pis-ova-tel\\
to-write-\textsc{iter-nmlz}\\
\glt ‘correspondent’ \label{biskup:ex:ctyritri.a}
\ex\gll ob-jev-i-tel Plut-a\\
about-show-\textsc{th-nmlz} Pluto-\textsc{gen}\\
\glt ‘the discoverer of Pluto’ \hfill (Czech) \label{biskup:ex:ctyritri.b}
\z\z

\noindent Since \REF{biskup:ex:ctyritri.a} refers to a person repeatedly performing the event of making a report, I take the \textit{-ova} suffix to represent the iterative head. Note that the iterative meaning cannot be a property of \textit{-tel} since this suffix also attaches e.g. to perfective predicates denoting a single event of discovering without changing the cardinality of the action, as shown in \REF{biskup:ex:ctyritri.b}. Thus, the order of the imperfectivizing \textsc{-yva} and the nominalizing suffix can be taken to mean that the projection of Voice -- hosting the agent -- is structurally higher than the iterative projection spelled out by \textsc{-yva}.\footnote{  The vocabulary item \textsc{-yva} can be inserted into the head of the iterative and progressive projections since it is specified as [imperfective].}

I am not aware of diminutive-iterative \textit{po-}predicates with \textit{-tel’}, \textit{-ciel} or \textit{-tel} but there is at least the deverbal nominalization \textit{pojídač} ‘eater’, in which the agentive \textit{-č} occurs outside the diminutive-iterative predicate \textit{pojída(t)} ‘to eat from time to time’, as shown in \REF{biskup:ex:pojidat}. The presence of the iterative meaning (expressed by \textit{-a}) is confirmed by the fact that the singular complement \textit{krevety} is ungrammatical in contrast to the plural \textit{krevet}.

\ea\label{biskup:ex:pojidat}\gll po-jíd-a-č krevet / \minsp{*} krevety\\
\textsc{del}-eat-\textsc{iter-nmlz} shrimp.\textsc{pl} {} {} shrimp.\textsc{sg}\\
\glt ‘shrimp eater’ \hfill (Czech)
\z

\noindent Building on these nominalization facts and the placement of the aspectual head above the projection introducing the agent, the relevant piece of morphosyntactic structure of diminutive-iterative \textit{po-}verbs looks like \REF{biskup:ex:ctyripet}, which is based on \REF{biskup:ex:ctyricetjedna}. Note that I added the standard verbalizing head \textit{v} and that the structure does not contain the progressive projection now since the progressive reading of delimitative \textit{po-}verbs with \textsc{-yva} is excluded.

%\ea\label{biskup:ex:ctyripet}
%[Asp\textsubscript{[Asp:IPFV]} [Voice [Iter \textsc{-yva}\textsubscript{[IPFV]} [Del \textit{po-}\textsubscript{[PFV]} [\textit{v} [√root]]]]]]
%\z

\ea\label{biskup:ex:ctyripet}                              
[Asp%
\tikz[baseline, anchor=base, remember picture] \node [inner xsep=0pt, inner ysep=.25ex] (BiskupASP) {\textsubscript{[Asp:IPFV]}}; 
[Voice [Iter \textsc{-yva}%
\tikz[baseline, anchor=base, remember picture] \node [inner xsep=0pt, inner ysep=.25ex] (BiskupIPF) {\textsubscript{[IPFV]}}; 
[Del \textit{po-}\textsubscript{[PFV]} [\textit{v} [√root]]]]]]     
\tikz[remember picture, overlay, baseline]
\draw [{Triangle[]}-{Triangle[]}] (BiskupASP.south) -- +(0, -1.5ex) -| (BiskupIPF.south); 
\z

\noindent Given that the aspectual value is determined by the aspectual marker that is attached last (i.e. that is closest to Asp; see discussion in \sectref{biskup:sec:dimin-iter}), when the aspectual head probes, then it finds the aspectual feature of the iterative \textsc{-yva} first. Consequently, the Agree operation uses this [imperfective] feature and diminutive-iterative \textit{po-}verbs always occur as imperfective, as shown in \REF{biskup:ex:ctyripet}.

Structures of diminutive-iterative \textit{po-}verbs are not difficult to linearize since all affixes are structurally higher than the root, as illustrated for the elements under discussion in \REF{biskup:ex:ctyripet}. If the Affix-Specific Linearization by Harley (2013) is assumed -- which encodes the prefixal versus affixal property directly in the specific marker -- then no head movement is necessary. Only argument phrases need to be evacuated from the extended verbal projection. Consider e.g. the relevant part of linearization of the Polish verb \textit{popłakiwać} ‘to cry from time to time’ in \figref{biskup:ex:linear}.

\begin{figure}[ht]
% the [ht] option means that you prefer the placement of the figure HERE (=h) and if HERE is not possible, you prefer the TOP (=t) of a page
% \centering
    \begin{forest}
    for tree={s sep=1cm, inner sep=0, l=0}
[...
         [...
         [VoiceP
            [IterP
                [DelP
                    [Del [\textit{po-}]]
                    [\textit{v}P
                        [√P
                            [\textit{płak}]
                        ]
                        [\textit{v}]
                    ]
                ]
            [Iter
                    [\textit{-iwa}]
                ]
            ]
        [Voice
            ]
        ]]
        [T [\textit{-ć}]
        ]
    ]
    % the overlay option avoids making the bounding box of the tree too large
    % the looseness option defines the looseness of the arrow (default = 1)
    \end{forest}
    \vspace{3ex} % extra vspace is added here because the arrow goes too deep to the caption; avoid such manual tweaking as much as possible; here it's necessary
    \caption{Linearization of Polish \textit{popłakiwać} ‘to cry from time to time’}
    \label{biskup:ex:linear}
\end{figure}

The verbalizing head \textit{v} is phonologically empty in this derivation but in other cases a theme vowel can be inserted. Since theme vowels are suffixes, the \textit{v} head is placed to the right also here. Delimitative \textit{po-} and iterative \textit{-iwa} are placed to the left and to the right, respectively, in accordance with their prefixal and suffixal status. I do not decompose imperfectivizing suffixes in this article but it is possible to split them, e.g. the Russian \textit{-yva} into \textit{-yv-a}, analogously the Polish \textit{-ywa} into \textit{-yw-a} and the Czech \textit{-ova} into \textit{-ov-a}. The second element (\textit{-a}) could be analyzed as a theme vowel that, e.g. in the structure of \textit{popłakiwać}, spells out the Voice head. Therefore, I put the head to the right in \REF{biskup:ex:linear}. Finally, \textit{-ć} can be taken to represent the infinitival T head, linearized to the right in accordance with the suffixal status of the marker.

One could assume that \figref{biskup:ex:linear} in fact represents the syntactic structure already before linearization (if one prefers a more powerful syntactic module). This would however go against the standard SVO analysis of Slavic languages, which supposes that heads take their complements to the right in Slavic (but see also \citealt{Haider.Szucsich2022}: 34--35).

Delimitative \textit{po-} is not the only prefix displaying the special, iterative type of behavior. Comitative \textit{pod-}, e.g. in the Russian \textit{podpevat'} and in the Polish \textit{podśpiewywać}, both with the meaning ‘to sing a little with sth.’, seems to behave in the same way.

\citet[41]{Flier1985} shows that perdurative (``delimitative'' in his terms) \textit{pro-} manifests very similar behavior; consider example \REF{biskup:ex:pro}.

\ea\label{biskup:ex:pro}\ea\gll Často vesennie večera ona pro-siž-iva-l-a na vysokom kryl'ce.\\
often spring evenings she through-sit-\textsc{iter-pst-f.sg} on high porch\\
\glt ‘She would often spend entire evenings in the spring sitting on the high porch.’ \label{biskup:ex:pro.a}
\ex[*]{\gll Ona dolgo prosiživala\textsuperscript{IPFV} na vysokom kryl'ce.\\ 
she long through.sit.\textsc{iter.pst.f.sg} on high porch\\ \glt Intended: ‘She would spend a long time sitting on the high porch.’ \label{biskup:ex:pro.b}}
\ex[*]{\gll Ona sejčas prosiživaet\textsuperscript{IPFV} na vysokom 	kryl'ce.\\ 
she now through.sit.\textsc{iter.prs.sg} on high porch\\ \glt Intended: ‘She is now spending time sitting on the high porch.’ \label{biskup:ex:pro.c}} \hfill (Russian; \citealt{Flier1985}: 41)
\z\z

\noindent As shown by \REF{biskup:ex:pro.a}, the event of sitting is iterated. The incompatibility of the imperfective verb with the adverbial \textit{dolgo} ‘for a long time’ in \REF{biskup:ex:pro.b} demonstrates that the durative meaning is excluded and the incompatibility of the verb in present with the adverbial \textit{sejčas} ‘now’ in \REF{biskup:ex:pro.c} shows the impossibility of the progressive meaning. A larger context showing the impossibility of the progressive reading is provided in \REF{biskup:ex:ctyriosm}.

\ea[*]{\label{biskup:ex:ctyriosm}\gll Ona pro-siž-iva-l-a vesennie večera na vysokom kryl'ce v tot moment, kogda bitva načalas’.\\
she through-sit-\textsc{iter-pst-f.sg} spring evenings on high porch in the moment when battle begin.\textsc{pst}\\
\glt Intended: ‘She was spending entire evenings in the spring sitting on the high porch when the battle began.’}
\z

\noindent Thus, this perdurative \textit{pro-} seems to be the next candidate for the positioning between the progressive and iterative \textsc{-yva}.

The current proposal has consequences for the overall architecture of verbal predicates. Before the splitting of the secondary imperfective marker into the progressive and the iterative \textsc{-yva}, there were two structural possibilities for verbal prefixes: below and above the secondary imperfective morpheme. Now, after the splitting, there are three options, as shown in \REF{biskup:ex:ctyridevet}.

\ea\label{biskup:ex:ctyridevet}
[Asp [Voice [SP\textsubscript{high} [Iter \textsc{-yva} [SP\textsubscript{high} (Del \textit{po-}) [Prog \textsc{-yva} [SP\textsubscript{low} [\textit{v} [√root [LP]]]]]]]]]]
\z

\noindent Specifically, (i) below the progressive \textsc{-yva} for (lexical, intermediate and lower superlexical) prefixes in predicates that can receive the progressive and iterative interpretations, like the Russian \textit{pro-davat'}\textsuperscript{IPFV} ‘to sell’; (ii) between the progressive and the iterative \textsc{-yva} for higher superlexical prefixes in predicates which cannot be progressivized but can have an iterative interpretation, like delimitative \textit{po-} in diminutive-iteratives and perdurative \textit{pro-} in \REF{biskup:ex:pro}; (iii) above iterative \textsc{-yva} for higher superlexical prefixes in predicates which cannot be imperfectivized -- like ingressive \textit{roz-} in the Polish \textit{rozboleć}\textsuperscript{PFV} ‘to start to ache’ -- and for higher superlexical prefixes which perfectivize a secondary imperfective predicate, like distributive \textit{po-} in the Russian \textit{po-vytalkivat'}\textsuperscript{PFV} ‘to push out one after another’.\footnote{\citeauthor{Mehlig2007} (\citeyear{Mehlig2007}, \citeyear{bis:Mehlig2012}) discusses examples of delimitative \textit{po-}verbs in Russian like \textit{pootkryvat’ okno} ‘to open a window for a while’, which at first sight, suggest that the delimitative \textit{po-} can also merge higher than the iterative \textit{-yva}. Such predicates denote an attempt to attain the change of state through several different actions. These cases probably are not problematic since they could be analyzed in terms of the partitive progressive operator (which I argued to be lower than the delimitative \textit{po-}), as in \citet{TatevosovIvanov2009}.}

\section{Conclusions} \label{biskup:sec:conclusions}

I have argued that diminutive-iterative \textit{po-}verbs are derived from delimitative \textit{po-}predicates. The secondary imperfective marker is split into two distinct elements, both syntactically and semantically: iterative \textsc{-yva} and progressive \textsc{-yva}. The iterative marker has a pluractional meaning and merges higher than delimitative \textit{po-}, whereas progressive \textsc{-yva} is a partitive operator that occurs below the prefix. In diminutive-iterative verbs, the progressive operator is not present and delimitative \textit{po-}, with its measure function meaning, applies to the simplex predicate. Then, the event denoted by the quantized predicate is iterated by the pluractional imperfectivizing \textsc{-yva}. I have discussed certain restrictions on the formation of delimitative \textit{po-}verbs and argued that only eventualities with a scalar structure and an event variable can be delimited. Static states, which contain the Kimian state referential argument, are not compatible with delimitative \textit{po-}. As to morphosyntactic structure, I have argued that the iterative projection, spelled out by \textsc{-yva}, occurs inside the verbal domain below the Voice projection, which introduces the external argument. Morphological aspect properties of diminutive-iteratives are determined in the aspectual projection via Agree with the closest aspectual feature, i.e. the imperfective feature of the iterative head. We have also seen that the overall (aspectual) architecture of Slavic predicates is more fine-grained and that there are more prefixal positions than usually assumed.

\section*{Abbreviations}

\begin{tabularx}{.5\textwidth}{@{}lQ}
\textsc{del}&delimitative\\
\textsc{f}&feminine\\
\textsc{gen}&genitive\\
\textsc{hab}&habitual\\
\textsc{inf}&infinitive\\
\textsc{ins}&instrumental\\
\textsc{ipfv}&imperfective\\
\textsc{iter}&iterative\\
\textsc{m}&masculine\\
\textsc{nmlz}&nominalizer\\
\end{tabularx}%
\begin{tabularx}{.5\textwidth}{lQ@{}}
\textsc{pfv}&perfective\\
\textsc{pl}&plural\\
\textsc{prog}&progressive\\
\textsc{prs}&present\\
\textsc{pst}&past\\
\textsc{ptcp}&participle\\
\textsc{sg}&singular\\
\textsc{si}&secondary imperfective\\
\textsc{th}&theme (vowel)\\
\\
\end{tabularx}

\section*{Acknowledgments}
Funded by the Deutsche Forschungsgemeinschaft (DFG, German Research Foundation) -- Project-ID 498343796. For helpful suggestions, I would like to thank participants of the FDSL-15 conference at the Humboldt-Universität zu Berlin (October 2022). Special thanks go to two anonymous reviewers for their insightful comments.

\printbibliography[heading=subbibliography,notkeyword=this]

\end{document}
