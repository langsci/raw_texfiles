\documentclass[output=paper,colorlinks,citecolor=brown]{langscibook}
\ChapterDOI{10.5281/zenodo.15394180}
%\bibliography{localbibliography}
%\usepackage{hhline}% http://ctan.org/pkg/hhline

\author{Dorota Klimek-Jankowska\orcid{https://orcid.org/0000-0003-3102-1384}\affiliation{University of Wrocław} and Vesela Simeonova\orcid{https://orcid.org/0000-0002-3841-1265}\affiliation{University of Graz}} 
% replace the above with you and your coauthors
% rules for affiliation: If there's an official English version, use that (find out on the official website of the university); if not, use the foriginal
% orcid doesn't appear printed; it's metainformation used for later indexing

%%% uncomment the following line if you are a single author or all authors have the same affiliation
% \SetupAffiliations{mark style=none}

%% in case the running head with authors exceeds one line (which is the case in this example document), use one of the following methods to turn it into a single line; otherwise comment the line below out with % and ignore it
% \lehead{Šimík, Gehrke, Lenertová, Meyer, Szucsich \& Zaleska}
% \lehead{Radek Šimík et al.}

\title[Secondary imperfectives in Polish and Bulgarian]{Two types of secondary imperfectives: Evidence from Polish and Bulgarian}
% replace the above with your paper title
%%% provide a shorter version of your title in case it doesn't fit a single line in the running head
% in this form: \title[short title]{full title}
\abstract{Secondary imperfective (SI) morphology differs in its productivity in Polish (PL) and  Bulgarian (BG): in PL, the SI morphology combines with some but not all prefixes. By contrast, almost every BG perfective verb has a SI  variant. To our knowledge, there is no research that has attempted to get a closer understanding of the source of this discrepancy. To fill in this niche, we conducted a comparative study of the interaction of SI morphology with different classes of aspectual prefixes in PL and BG and the meaning effects they give rise to. We present novel observations and account for them by proposing that there are two distinct layers at which SI morphemes are generated in BG and only one such layer in PL. 

\keywords{secondary imperfective, aspect, habituality, prefixes, Polish, Bulgarian} %here, come, your, 4 to 6 keywords or keyphrases
}

\begin{document}
\maketitle

\section{Introduction}
There is an ongoing debate in the literature on Slavic aspect concerning the status of aspectual morphemes. Little agreement has been reached as to the status of secondary imperfective (SI) /(y)v/ in Slavic languages. The views vary as to where /(y)v/ is generated in the structure (see \citealt{kli+:schoorlemmer1995,babko1999,istratkova2004,kli+:milicevic2004lexical,svenonius2004b,kli+:svenonius2004a,kli+:romanova2004,kli+:filip2005,sciullo2005,kli+:arsenijevic2006,romanova2007,kli+:ramchand2008a,ramchand2008b,lazorczyk2010,markova2011,tatevosov2011,kli+:tatevosov2015,biskup2012,kli+:biskup2019,wiland2012,zaucer2012,rothstein2020,klimek2022,klimekNLLT,kli+:kwapiszewski2022}). 
In most of these studies, generalizations about the status of secondary imperfective morphology are made based on the data from a single language. However, Slavic languages differ considerably in the productivity of forming SI verbs. For example, the Bulgarian (BG) secondary imperfective is considerably more productive as compared to Polish (PL): almost every BG perfective verb has a SI variant (see \citealt{markova2011,rivero2014}; \citealt{nicolova2017}: 5.3.25), while in PL, SI morphology combines with some but not all prefixed verbs (see \citealt{lazorczyk2010,lazinski11,lazinski20,wiemer-et-al-20,kli+:kwapiszewski2022,klimek2022,klimekNLLT,}). While it is a well-known observation, there are no works that attempt to explain the locus of variation. Our goal in this paper is to fill that gap. We provide two major novel empirical observations: 

\begin{enumerate}
    \item 
     Within Bulgarian: while all previous works on SI in general and in Bulgarian specifically treat it as a single class, we identify two distinct SI classes within Bulgarian with systematically different sets of formal derivational and semantic properties. %One class of SI has only habitual readings, while the other class allows both ongoing and habitual readings.  

     \item 
     Between languages: we identify the locus of cross-linguistic variation in the productivity of SI between Polish and Bulgarian: the two languages share one of the two classes of SI and Polish lacks the second class of SI that Bulgarian has.  
\end{enumerate}


\noindent We conclude that the difference in the productivity of SI in Polish and Bulgarian is not random, but is systematically determined based on the range of derivational possibilities in the two languages, with respective semantic consequences. 

To arrive at these conclusions, we tested the interaction of SI morphology in PL and BG with two classes of prefixes -- lexical and purely perfectivizing prefixes -- and the meaning effects these different combinations of the tested aspectual prefixes with SI morphemes give rise to. In this paper, the empirical scope of the environments tested is limited to past tense contexts. We discuss possible extensions in \S \ref{section:conclusion}.

We show that in BG there are two types of SI morphemes that bear different meanings; PL has only one of these SI morphemes. We propose that the two types of SI in BG are realized in two different syntactic layers, following \citeauthor{cinque1999}'s (\citeyear{cinque1999}) model. 
 
 
 


\section{Background on the secondary imperfective in Polish and Bulgarian}\label{section:background-SI}


\subsection{Similarities between Polish and Bulgarian}\label{subsection:background-SI:similarities}
In both PL and BG, aspectual distinctions are encoded on almost all verbs.\footnote{The encoding of aspectual distinctions can be blocked, e.g. for phonological reasons, %e.g. glasuvam `vote' already has the \textit{-va} morphology in  
or in certain loanwords, e.g. \textit{printiram} `print' only has one form in BG.} The least morphologically complex aspectual forms are primary imperfectives (bare, i.e. ``unprefixed’’ verbs) and they can be perfectivized by means of a prefix, cf. \REF{basics:perfective}. Some prefixes do not change the verb in any way other than its aspectual value; they are called \textsc{purely perfectivizing} or \textsc{empty} prefixes    (\citealt{boguslawski1963,svenonius2004b,kli+:svenonius2004a,mlynarczyk2004,willim2006,kli+:ramchand2008a}).\footnote{\citet{janda2010} emphasize that Russian has at least 16 prefixes forming natural perfectives (those perfectives which are not semantically distinct from the unprefixed base verb), which may suggest that they encode hidden distinctions.
They propose that in the case of natural perfectives there is a semantic overlap between the meaning of the prefix and the meaning of the base verb and the diversity of prefixes used in natural perfectives follows from the fact that the base verbs from which they are derived fall into semantically diverse classes. Building on that, \citet{janda2013} propose that the verbal prefixes act as classifiers in that they select verbs according to broad semantic traits, categorizing them the way numeral classifiers in some languages categorize nouns. We think that irrespective of the terminology used, there is a general consensus that the prefixes in natural perfectives do not modify the meaning of the base verbs but they may only impose selectional restrictions on the base verbs they combine with. Therefore, we will maintain the terminology `purely perfectivizing'.}

%NOTEVES When you add the % sign to something, you delete it from the PDF but you keep it for yourself in the code file.
% We would like to acknowledge that this may very well be the case that purely perfectivizing prefixes are diverse because they attach to different semantic classes of base verbs (and as such they may be treated as classifiers) but even so it is still correct to assume that if the prefixes forming natural perfectives overlap the meaning of the base verb, they do not modify its lexical content and their role is purely aspectual. We think the apparent conflict between these two approaches discussed above is not a real conflict. Both approaches are correct. 

\ea\label{basics:perfective}
\ea  \gll pisać --  napisać  \\
 write.\textsc{ipfv}  {} write.\textsc{pfv} \\ \hfill PL
\ex \gll  piša -- napiša   \\
write.\textsc{ipfv} {} write.\textsc{pfv} \\ \hfill BG
\z % this ends the (sub)example
\z %in this case we need two - one to end the subexample and one to end the example


% (see e.g. BabkoMalaya 1999, Ramchand 2008a,b, Romanova 2004, 2007, Svenonius 2004a,b, Sciullo & Slabakova 2005, Arsenijević 2006, Biskup 2012, 2019, Žaucer 2009, 2012, Markova 2011, Wiland 2012).


\noindent Another class of prefixes are the so-called \textsc{lexical prefixes} (\citealt{babko1999,svenonius2004b,kli+:svenonius2004a,kli+:romanova2004,romanova2007,kli+:ramchand2008a,ramchand2008b,biskup2012,kli+:biskup2019}, a.m.o.): they have an idiosyncratic meaning where the prefix changes the lexical interpretation of the verb, but not in a predictable way, for example the prefix \textit{prze-} in \REF{basics:idiosyncratic:PL} and the prefix \textit{pod-} in \REF{basics:lexical:idiosyncratic-prefixes:BG} have very different meaning in various verbs they participate in.  
Lexical prefixes cause idiosyncratic changes in the meaning of a verbal predicate that is not derivable from either the verb or the prefix, cf. \REF{basics:idiosyncratic:PL} and \REF{basics:lexical:idiosyncratic-prefixes:BG}.\footnote{Because of these properties, \citet{kli+:romanova2004,svenonius2004b,kli+:svenonius2004a,kli+:ramchand2004,kli+:ramchand2008a,ramchand2008b,lazorczyk2010} argue that lexical prefixes are merged vP-internally.}
%in this study, we focus on the interaction of SI morphology with lexical and purely perfectivizing prefixes in PL and BG. 

\ea\label{basics:idiosyncratic:PL} 
\ea
\gll kupić -- \textit{prze}-kupić  \\
buy.\textsc{pfv} {} bribe.\textsc{pfv} \\  \hfill PL
\glt`to buy’ -- `to bribe’
%
\ex \gll grać -- \textit{prze}-grać\\
play.\textsc{ipfv} {} lose.\textsc{pfv} \\ \hfill PL
\glt `play' -- `lose'
%
\ex \gll łączyć -- \textit{prze}-łączyć\\
connect.\textsc{ipfv} {} switch.\textsc{pfv} \\\hfill PL
\glt `connect' -- `switch'\\ 
\z
\z

% \ex \gll pisha -- o-pisha [BG]\\
% write -- o-write.pfv\\
% \glt `write  -- describe'
% \z
% \z 
\ea\label{basics:lexical:idiosyncratic-prefixes:BG}
\ea \gll seštam \minsp{*(} se) -- \textit{pod}-seštam\\
recall {} \textsc{refl} {} remind.{\PFV}\\ \hfill BG
\glt `recall' -- `remind' 
%
\ex \gll budja -- \textit{pod}-budja\\
wake.up.{\IPFV} {} incite.{\PFV}\\ \hfill BG
\glt `wake up' -- `incite, instigate'
%
\ex \gll igraja -- \textit{pod}-igraja\\
play.{\IPFV} {} mock.{\PFV}\\ \hfill BG
\glt `play' -- `mock' \\ 
%
\z
\z

\noindent Furthermore, lexical prefixes (can) alter the argument structure/selectional restrictions of a verb, cf. \REF{basics:argument-structurePL}--\REF{basics:argument-structureBG}.
    

\ea\label{basics:argument-structurePL}
\ea \gll znać \minsp{\{} kogo\'{s} / \minsp{*} uszkodzenia\} \\ 
know.\textsc{ipfv} {} someone {} {} damage \\ \hfill PL
\glt `to know someone'
\ex \gll do-znać \minsp{\{*} kogo\'{s} / uszkodzenia\}\\
suffer.\textsc{pfv} {} someone {} damage\\ 
\glt `to suffer damage'  
\z 
\z 

\ea\label{basics:argument-structureBG} 
\gll mislja \minsp{(*} se) -- za-mislja \minsp{*(} se) \\
think {} \textsc{refl} {} consider {}  \textsc{refl} \\ \hfill BG
\glt `think'  -- `consider'
%
\z 

\iffalse
\ea\label{basics:argument-structure}
\ea \gll znać \minsp{\{} kogo\'{s} / *uszkodzeni\}a -- \textit{do}-znać \minsp{\{*} kogo\'{s} / uszkodzenia\} \\
know.\textsc{ipfv} {} someone {} damage {} suffer.\textsc{pfv} someone {} damage\\ \hfill [PL]
\glt `to know someone' -- `to suffer damage'  
%
\ex \gll mislja \minsp{(*} se) -- \textit{za}-mislja \minsp{*(} se) \\
think {} \textsc{refl} {} consider {}  \textsc{refl} \\ \hfill [BG]
\glt `think'  -- `consider'
%
\z 
\z 
\fi



\noindent Lexically prefixed perfective verbs are imperfectivized by means of an \textit{-yw-} or \textit{-a-} suffix in Polish and by a \textit{-va-} suffix or vowel alternations in Bulgarian, cf. \REF{basics:lexical}. These imperfective forms derived from perfective verbs are called \textsc{secondary imperfective} (SI). Table \ref{table:lexical:PL+BG} shows more verbs from this morphological pattern. 

\ea \label{basics:lexical}
\ea \gll podpisać -- podpisywać \\
sign.\textsc{pfv} {} sign.\textsc{si}\\\hfill PL
\glt `sign' -- `sign'
\ex  \gll  podpiša -- podpisvam   \\
sign.\textsc{pfv} {} sign.\textsc{si}\\ \hfill BG
\glt `sign' -- `sign'
\z 
\z 



\begin{table}
\begin{tabularx}{\textwidth}{ X X  l l  l }
  \lsptoprule
\multicolumn{2}{c}{Polish}  & \multicolumn{2}{c}{Bulgarian} & {English}\\
  \cmidrule(lr){1-2}\cmidrule(lr){3-4}
%
PFV & SI & PFV & SI & \\ 

\midrule
 podpisać  & podpisywać & podpiša & podpisvam & `sign'\\
 %
odpowiedzieć & odpowiadać & otgovorja & otgovarjam  & `reply’ \\
%
naprawić & naprawiać & popravja & popravjam & `repair' \\ 
%
wyjaśnić & wyjaśniać & objasnja & objasnjavam & `explain' \\ 
%
sprzedać & sprzedawać & prodam & prodavam & `sell'\\ 
%
opisać & opisywać & opiša & opisvam & `describe' \\
%
%zarobić & zarabiać & zarabotja & zarabotvam & 'earn' \\ %NoteVES: I am not sure if this is superlexical or lexical. It behaves only habitual in Bg.
\lspbottomrule
\end{tabularx}
\caption{Lexical prefixes and SI in Polish and Bulgarian}
\label{table:lexical:PL+BG}
\end{table}

 




\subsection{A major difference: SI productivity} 



In the previous section, we showed that SI is possible with lexically prefixed verbs both in PL and in BG. 
%\hl{(this needs to be integrated better)} - NoteVes: DONE
However, there is a major difference between PL and BG in that almost every BG verb can form SI (\citealt{dickey-book}: 11; \citealt{nicolova2017}: 5.3.25). Most verbs with empty prefixes also have SI forms, as illustrated in \REF{basics:bg:triplet:possible}. In PL, empty prefixed forms cannot form SI, see the ungrammatical form in \REF{basics:pl:triplet:impossible}. 
 The pattern described in example \REF{basics:pl-bg-distinction} is systematic in the two languages, as demonstrated in  Tables \ref{table:pure:longBG} and~\ref{table:pure:longPL}.\footnote{The fact that one may find some rare instances of these verbs on the internet suggests that someone either used them creatively or mistakenly. Such rare uses may suggest that the two projections that we will argue for in \S\ref{section:cinque} high SI and low SI are universally there in the hierarchy of projections but in some languages such as Polish, for example, the high SI morpheme generally does not merge in this position (it is blocked), but it may exceptionally be unblocked when used creatively or in speech production errors.}



% • have secondary (derived) imperfective counterparts: 

% \ea\label{basics:derived}
% \ea
% \gll pod-pisać -- pod-pis-yw-ać [PL]\\
% pod-write.pfv -- pod-write-SI-inf\\
% \glt `to sign -- to be signing'\\
% %
% \ex \gll pod-pisha -- pod-pis-va-m [BG]\\
% pod-write.pfv -- pod-write-SI-1sg\\
% \glt `to sign -- to be signing'\\
% \z
% \z


% przy-gotow-aćP — przy-gotow-yw-aćSI
% przy-cook.pfv -przy-cook-si.ipfv
% ‘to prepare’ — ‘to be preparing’
% BG examples needed


\ea\label{basics:pl-bg-distinction}	
\ea \gll pisać – napisać -- \minsp{*} napisywać \label{basics:pl:triplet:impossible}\\ \hfill 
 write.\textsc{ipfv}  {} write.\textsc{pfv} {} {} write.\textsc{si} \\\hfill PL 
\ex  \gll piša – napiša -- napisvam \label{basics:bg:triplet:possible} \\
 write.\textsc{ipfv}  {} write.\textsc{pfv} {} write.\textsc{si}\\\hfill BG
 \z
 \z 
\begin{table}
\begin{tabularx}{.8\textwidth}{XXll}
\lsptoprule
\multicolumn{3}{c}{Bulgarian} & {English}\\
\cmidrule{1-3}
%
IPFV & PFV & SI \\ 

\midrule
 stroja & postroja & \cellcolor{gray!15} postrojavam & `build'\\
 piša & napiša & \cellcolor{gray!15} napisvam & `write’ \\ p\v{u}r\v{z}a & izpăr\v{z}a & \cellcolor{gray!15} izpăr\v{z}vam & `fry'\\ gladja & izgladja & \cellcolor{gray!15} izgla\v{z}dam & `iron' \\ broja & prebroja & \cellcolor{gray!15} prebrojavam & `count'\\  molja & pomolja & \cellcolor{gray!15} pomolvam & `ask'\\ četa & pročeta & \cellcolor{gray!15} pročitam & `read' \\gubja & izgubja & \cellcolor{gray!15} izgubvam & `lose' \\ merja & izmerja & \cellcolor{gray!15} izmervam & `measure' \\ zvănja & pozvănja & \cellcolor{gray!15} pozvănjavam & `call'\\ čupja & sčupja & \cellcolor{gray!15} sčupvam & `break' \\
\lspbottomrule
\end{tabularx}
\caption{Purely perfectivizing prefixes and SI in Bulgarian}
\label{table:pure:longBG}
\end{table}

\begin{table}
\begin{tabularx}{.8\textwidth}{XXll}
\lsptoprule
\multicolumn{3}{c}{Polish}  & {English}\\
\cmidrule{1-3}
%
IPFV & PFV & SI\\ 
\midrule
budować  & zbudować & \cellcolor{gray!15}*zbudowywać & `build'\\
pisać & napisać & \cellcolor{gray!15}*napisywać & `write’ \\
smażyć & usmażyć & \cellcolor{gray!15}*usmażywać & `fry'\\
prasować & wyprasować & \cellcolor{gray!15}*wyprasowywać & `iron' \\
liczyć & policzyć & \cellcolor{gray!15}*policzać & `count'\\ 
prosić & poprosić & \cellcolor{gray!15}*popraszać & `ask'\\
czytać & przeczytać & \cellcolor{gray!15}*przeczytywać & `read' \\
gubić & zgubić & \cellcolor{gray!15}*zgubiać & `lose' \\
mierzyć & zmierzyć & \cellcolor{gray!15}*zmierzać & `measure' \\
dzwonić & zadzwonić & \cellcolor{gray!15}*zadzwaniać & `call'\\
łamać & złamać & \cellcolor{gray!15}*złamywać & `break' \\ 
\lspbottomrule
\end{tabularx}
\caption{Purely perfectivizing prefixes and SI in Polish}
\label{table:pure:longPL}
\end{table}


% a paradox - SI is used as a test for lexical prefixes but not in BG
% how to account for the greater productivity of SI forms in Bg, is the meaning of SI of empty prefixed verbs in BG the same as the meaning of SI verbs of lexically prefixed verbs?



%!!! VES: FIX TABLE - ^^FIXED^^
In other words, in BG, there is a morphological triplet for  verbs with purely perfectivizing prefixes and a pair for lexically prefixed forms, as in Tables \ref{table:pure:longBG} and~\ref{table:BG}. 
And in PL, verbs form morphological pairs: either the bare imperfective and a verb with a purely perfectivizing prefix or the lexically prefixed imperfective and the derived SI, as in Tables \ref{table:pure:longPL} and \ref{table:PL}. The crucial difference between the two languages in the two tables is marked with shading.\footnote{
We acknowledge that the type of classification of triplets and pairs that we are using to make this claim is not the only one that exists in the literature. In a very recent study on aspectual triplets in Russian, Czech, Polish, \citet{wiemer-et-al-20} identify triplets based on a different set of criteria. They assume that lexically prefixed verbs also form triplets. We explain how we understand triplets and pairs Section, \S \Ref{subsection:background-SI:similarities} that we use in this paper  are  based on the works cited in \S \ref{section:background-SI}.
  
}
 



\begin{table}
\begin{tabular}{ c c c }
  \lsptoprule
primary imperfective & perfective & secondary imperfective \\ 
  \midrule
 N/A & podpiša `sign' & podpisvam \\
 N/A & poleja `water' & polivam \\
 \tablevspace
 stroja ‘build’ & postroja & \cellcolor{gray!15}postrojavam\\
piša ‘write’ & napiša & \cellcolor{gray!15}napisvam \\
  \lspbottomrule
\end{tabular}
\caption{Purely perfectivizing vs. lexical prefixes and SI in Bulgarian}
\label{table:BG}
\end{table}




\begin{table}
\begin{tabular}{ c c c }
  \lsptoprule
primary imperfective & perfective & secondary imperfective \\ 
  \midrule
 N/A & podpisać `sign' & podpisywać \\  
 N/A & podlać `water' & podlewać \\
 \tablevspace
 budować ‘build’ & zbudować & \cellcolor{gray!15}*zbudowywać\\
pisać ‘write’ & napisać & \cellcolor{gray!15}*napisywać \\
\lspbottomrule
\end{tabular}
\caption{Purely perfectivizing vs. lexical prefixes and SI in Polish}
\label{table:PL}
\end{table}

% . SI is possible in BG but not in PL with prefixes like \textit{na-} in \REF{basics:bg:triplet:possible}. 


While the morphological determinant of the restrictive SI in PL is well-known -- the availability of SI counterparts of perfective verbs depends on the prefix type, as we described above -- and it is also well-known that BG SI is fully productive, these observations raise many questions that remain unanswered to date. The questions we address in this paper are whether there are semantic differences between the SI forms in BG that do not have equivalents in PL and the ones that do have equivalents in PL and why the BG SI forms are not possible in PL.

\section{Novel findings: Two types of SI}


Our first finding is that the pair SI in PL is equivalent to the pair SI in BG and it is ambiguous between the single ongoing and habitual reading. The second finding is that within BG, the triplet SI is qualitatively different from the pair SI. The triplet SI is habitual only, while the pair SI is ambiguous between single ongoing and habitual reading. This means that the properties of SI are not uniform across languages and even within the same language they are not homogeneous. %but we propose, they are uniform across morpho-syntactic constructions.







\subsection{Pair SI in Bulgarian and Polish}\label{section:pair-SI-in-BG-and-PL}


Both in Bulgarian and in Polish, SI forms derived from lexically prefixed verbs are ambiguous between an ongoing reading, as in \REF{ongoing:pair:BG+PL:past}, and a habitual reading, as in \REF{habitual:pair:BG+PL:past}. This is not idiosyncratic of a specific verb, but holds across the morphological paradigm represented in Table \ref{table:lexical:PL+BG}. 


\ea \label{ongoing:pair:BG+PL:past} \textsc{Ongoing context}
\ea 	\gll Kogato 	vljazoh v 		ofisa 	  na Ivan, 	 toj  \minsp{(} točno)\hspace{2.4cm}	\minsp{\{} podpis-va-še	dokumenti / poprav-ja-še koleleta / otgovar-ja-še na imejli\}.	\\
%
when 		enter.\textsc{pfv.aor.1sg} 	in office  of Ivan 	
he {} just {} sign-\textsc{si}-\textsc{impf.3sg}	documents {} repair-\textsc{si}-\textsc{impf.3sg} bikes {} reply-\textsc{si}-\textsc{impf.3sg} to emails\\\hfill BG
%
%
\ex  \gll	Kiedy weszłam do gabinetu Jana, \minsp{(} właśnie)\hspace{2.4cm} \minsp{\{} podpis-yw-ał dokumenty / naprawi-a-ł rower / odpowiad-a-ł na maila\}.\\
%
when 	entered.\textsc{pfv.pst.1sg} 	to office 	John {} just {} sign-\textsc{si}-\textsc{pst.3sg}	documents {} repair-\textsc{si}-\textsc{pst.3sg} bike {} reply-\textsc{si}-\textsc{pst.3sg} to email\\ \hfill PL
%
\glt `When I entered John’s office, he was (in the middle of) \{signing documents / repairing bikes / replying to e-mails\}.'
\z
\z



\ea \label{habitual:pair:BG+PL:past} \textsc{Habitual context}\\
%
\ea \gll Predi obiknoveno	\minsp{\{} podpis-va-še dokumentite / poprav-ja-še koleletata / otgovar-ja-še na imejli\}	po-bărzo 	ot 	men,  no  veče 	ne.  \\
before usually {}  sign-\textsc{si}-\textsc{impf.3sg} documents {} repair-\textsc{si}-\textsc{impf.3sg} bikes.{\DEF} {} reply-\textsc{si}-\textsc{impf.3sg} to emails faster 	    	than 	me,
but already not\\\hfill BG
%
\ex \gll	Kiedyś zwykle	\minsp{\{} podpis-yw-ał dokumenty / na-prawi-a-ł rowery / odpowiad-a-ł na maile\} 	szybciej niż 	ja, ale 	teraz 	już 		nie. \\
before usually	{}	sign-\textsc{si}-\textsc{pst.3sg} documents {} repair-\textsc{si}-\textsc{pst.3sg} bikes {} respond-\textsc{si}-\textsc{pst.3sg} to emails faster 	  than 	I but 	now 	already 	not\\\hfill PL
\glt `In the past, usually he (used to) \{sign (the) documents / repair bikes / respond to e-mails\} faster than me but not anymore.'
%
%
\z
\z



% \ea\label{sim:ex:german-verbs}\gll Hans \minsp{\{} schläft / schlief / \minsp{*} schlafen\}.\\
% Hans {} sleeps {} slept {} {} sleep.\textsc{inf}\\
% \glt `Hans \{sleeps / slept\}.'
% \z

 
\noindent All the verbs in Table \ref{table:lexical:PL+BG} behave in a way analogous to the pattern shown in examples \REF{ongoing:pair:BG+PL:past} and \REF{habitual:pair:BG+PL:past}, allowing both habitual and ongoing readings. We were unable to find any counterexamples. 
%---------------------------------------------------------------



\subsection{Triplet SI in Bulgarian}

The triplet SI in Bulgarian cannot be used with ongoing actions, cf. \REF{ongoing:bg:triplet:past} and only has habitual readings, cf. \REF{habitual:bg:triplet:past}. 
The examples also show that the bare imperfective is grammatical in both environments.\footnote{While we assume that it is always available in ongoing contexts, we do not claim that it is always possible in every habitual context. }


% \ea\label{sim:ex:german-verbs}\gll Hans \minsp{\{} schläft / schlief / \minsp{*} schlafen\}.\\
% Hans {} sleeps {} slept {} {} sleep.\textsc{inf}\\
% \glt `Hans \{sleeps / slept\}.'
% \z


\ea \label{ongoing:bg:triplet:past} \textsc{Ongoing context}\\
%
\ea \gll Kogato telefonăt zvănna, točno \minsp{\{}  păr\v{z}eh /\hspace{1.7cm} \minsp{*} izpăr\v{z}vah\} kjufteta.\\
when phone.{\DEF} rang.\textsc{aor.3sg} just {} fry.\textsc{impf.1sg.ipfv} {} {} fry.\textsc{impf.1sg.si} meatballs \\ \hfill BG 
\glt `When the phone rang, I was (right in the middle of) frying meatballs.'
%
\ex \gll Kogato telefonăt zvănna, točno \minsp{\{} gladeh /\hspace{1.8cm} \minsp{*} izgla\v{z}dah\} drehi.\\
when phone rang.\textsc{aor.3sg} just
{} iron.\textsc{impf.1sg.ipfv} {} {} iron.\textsc{impf.1sg.si} clothes\\ \hfill BG
\glt `When the phone rang, I was (right in the middle of) ironing clothes.'
%
%
\ex \gll Kogato telefonăt zvănna, točno si \minsp{\{} pišeh /\hspace{0.15cm} \minsp{*} napisvah\} domašnoto. \\
when phone rang.\textsc{aor.3sg} just \textsc{refl.gen}
{} write.\textsc{impf.1sg.ipfv} {} {} write.\textsc{impf.1sg.si} homework.\textsc{def}\\\hfill BG
\glt `When the phone rang, I was (right in the middle of)  writing my homework.'
\z
\z



\ea \label{habitual:bg:triplet:past} \textsc{Habitual context}\\
%
\ea \gll Kogato praveh zakuska, obiknoveno \minsp{\{} păr\v{z}eh / izpăr\v{z}vah\} po 3 kjufteta na čovek.\\
when make.\textsc{impf.ipfv.1sg} breakfast usually {} fry.\textsc{impf.ipfv.1sg} {} fry.\textsc{impf.si.1sg} \textsc{distr} 3 meatballs per person\\ \hfill BG
\glt `When I made breakfast, I used to fry 3 meatballs per person.'
%
\ex \gll Predi vinagi \minsp{\{} gladeh / izgla\v{z}dah\} drehite vednaga sled prane.\\
before always {} iron.\textsc{impf.ipfv.1sg} {} iron.\textsc{impf.si.1sg} clothes.\textsc{def} immediately after washing\\ \hfill BG
\glt `Before I always ironed the clothes immediately after washing.'
%
%
\ex \gll Predi obiknoveno \minsp{\{} pišeh / napisvah\} po njakolko  knigi na godina, no sega samo po edna.\\
%
before usually {} wrote.\textsc{impf.ipfv.1sg} {} wrote.\textsc{impf.si.1sg} \textsc{distr} several books per  year, but now only \textsc{distr} one \\ \hfill BG
%
\glt `In the past, I used to write several books per year, but now only one.'
\z
\z




% wrócili się - wracali się - to return -  

\noindent This semantic pattern observed with triplet SI above is valid across the paradigm of triplet SI, a sample of which was presented in Tables \ref{table:pure:longBG} and \ref{table:pure:longPL}. Since there is no triplet SI in Polish, this part of the data is not directly comparable between the two languages. In both single ongoing and habitual scenarios presented for Bulgarian in \REF{ongoing:bg:triplet:past} and \REF{habitual:bg:triplet:past}, Polish uses primary imperfective verbs only. 


To summarize, while PL
uses primary \textsc{ipfv} verbs to render both habitual and ongoing readings in purely perfectivized verbs and the SI forms of such verbs are blocked, BG productively uses the SI forms of those verbs to exclusively ``mark'' the special kind of habitual reading (consisting of a series of temporally non-overlapping bounded events happening on separate occasions).

% This suggests that, thanks to the vast
% productivity of SI markers, BG has repurposed those that  into a specific linguistic means to mark these contexts. Put still
% differently, BG makes use of SI verbs that are „useless” in other Slavic languages due to their
% being synonymous with the respective simplex verbs.


Our  novel observation is that the properties of SI morphology are not semantically uniform across the two languages. In the next section, we propose that the two types of SI morphemes merge at different syntactic positions in BG but not in Polish. 

%---------------------------------------------------------------



\section{The syntax of SI}\label{section:cinque}



In order to formally capture the observations presented in the previous section, we propose that in BG, the two types of SI morphemes merge at two syntactic layers -- one higher and one lower -- while in PL, the low SI morpheme merges only in the lower one. 

\ea \begin{tabular}[t]{l l c c}
BG: & SI\textsubscript{high} &  $>>$  & SI\textsubscript{low}\\
PL: &           &          & SI\textsubscript{low}\\
\end{tabular}
\z

\noindent This proposal allows us to syntactically distinguish between the properties of lexical prefixes and purely perfectivizing ones
(see \S \ref{section:background-SI}). We argue that SI is not a uniform category within Bulgarian because it merges in two different syntactic positions with different properties each. Moreover, pair SI is equivalent across Bulgarian and Polish because it is merged in the same projection SI\textsubscript{low} with the same properties. PL does not have aspectual triplets because the high SI morpheme is blocked and it cannot merge in the SI\textsubscript{high} layer while Bulgarian developed a specialized habitual meaning of the SI\textsubscript{high} morpheme. Because verbs with SI\textsubscript{high} have obligatory habitual readings in BG, which are missing in PL, we propose that the SI\textsubscript{high} morpheme merges in a projection corresponding to \citeauthor{cinque1999}'s (\citeyear{cinque1999}) Asp\textsubscript{HAB}. 

We assume \citeauthor{baker85}'s (\citeyear{baker85}) Mirror Principle, according to which syntax reflects morphology and vice versa and the linearization of functional morphemes is syntactically motivated. Additionally, we follow \citet{cinque1999}, who argues that there is a fixed hierarchy of functional projections which regulates the way adverbs and functional morphemes are merged in syntax. Based on a large survey of languages, \citeauthor{cinque1999} shows that among temporal/aspectual affixes, e.g. repetitive, frequentative, terminative, continuative, retrospective, durative, progressive, completive, those that are specifically dedicated to expressing habituality scope the highest \citep[][p. 56; 70]{cinque1999}. Crucially, this means that the dedicated habitual functional head is syntactically higher than the progressive aspectual head. The complete functional hierarchy is provided below in (\Ref{cinque:hierarchy}), in which the two functional heads are highlighted with boxes.

\ea \label{cinque:hierarchy} Mood\textsubscript{speech~act} $>$ Mood\textsubscript{evaluative} $>$  
Mood\textsubscript{evidential} $>$ 
Mood\textsubscript{epistemic} $>$ 
T\textsubscript{(past)} $>$ 
T\textsubscript{(Future)} $>$
Mood\textsubscript{irrealis} $>$
\fbox{Asp\textsubscript{habitual}} $>$
T\textsubscript{(Anterior)} $>$
Asp\textsubscript{perfect} $>$
 Asp\textsubscript{retrospective} $>$
 Asp\textsubscript{durative} $>$
 \fbox{Asp\textsubscript{progressive}} $>$
 Asp\textsubscript{prospective}/Mod\textsubscript{root} $>$
 Voice $>$ 
 Asp\textsubscript{celerative} $>$ 
 Asp\textsubscript{completive} $>$
 Asp\textsubscript{(semel)repetitive} $>$
 Asp\textsubscript{iterative} \\
%
 \hfill \citet[76]{cinque1999}
\z



\noindent Our examples in the preceding sections showed that this is the case also in BG and PL. In addition, we showed that the adverbs are optional in the case of SI\textsubscript{high}, that is, it encodes the habitual reading itself rather than being merely compatible with it. SI\textsubscript{low}, on the other hand, is compatible with both frequentative and ongoing adverbs just like a null IPFV operator, for example the one proposed in \citet{kli+:ferreira2016} selecting for VPs referring to singular or plural events respectively: IPFV [VPsg / VPpl].\footnote{We assume \citeauthor{tatevosov2011}'s (\citeyear{tatevosov2011,kli+:tatevosov2015}) proposal that aspectual morphology can be lower than the actual aspectual interpretation.}

In this way the proposal offers a formalization in syntactic terms which captures the differences in the morphological productivity of SI in the two languages, the within-language split (in BG), as well as the fact that the lower SI is equivalent in the two languages. % even though as a whole they differ

\section{Discussion and conclusion}
\label{section:conclusion}

This study presented a formal description of a systematic difference in the productivity of SI in PL and BG: the Polish perfectives that do not allow subsequent secondary imperfectivization are precisely those cases where Bulgarian SI forms only have a habitual reading and the single ongoing reading is unavailable. These are the perfective forms which in Polish contain purely perfectivizing prefixes. By contrast, Polish perfectives that allow subsequent secondary imperfectivization are those cases where Bulgarian SI forms are ambiguous between a single ongoing and a habitual reading. These are the perfective forms which both in Polish and Bulgarian contain lexical prefixes. 

Based on these novel observations we proposed that there are two distinct types of SI in BG: SI\textsubscript{high} >> SI\textsubscript{low} and only one in Polish, SI\textsubscript{low}. Crucially, we showed that SI\textsubscript{low} is uniform in the two languages -- it is ambiguous between a single ongoing and habitual reading. SI\textsubscript{high} is merged in a projection corresponding to \citeauthor{cinque1999}'s (\citeyear{cinque1999}) Asp\textsubscript{HAB} and Bulgarian syntax generates this position, while Polish does not.

Previous works on SI cannot capture the novel observations we present here. For example, \citet{rivero2014} assume that all SIs in BG have both habitual and ongoing readings. This is reflected in their formal account, attributing this duality to context. 
As we have shown, this is accurate for SI with lexical prefixes in BG and in Polish, but it overgenerates for SI with purely perfectivizing prefixes. 

Conversely, \citet{markova2011} assumes (but provides no evidence) that the SI morphology in BG is in \citeauthor{cinque1999}'s (\citeyear{cinque1999}) Asp habitual projection. This has the opposite problem: it undergenerates the available ongoing interpretations of SI with lexical prefixes.

One limitation of this study is that the syntactic proposal put forth here still does not explain why it may be that one language is able to generate the higher SI layer, while the other language is not. We leave such a comprehensive explanatory account for future work.  

Additionally, one may reasonably ask why SI\textsubscript{high} is not blocked by primary imperfective forms which can also express a habitual reading. In order to address this issue, we show below that in temporal \textit{after}-clauses imposing sequential ordering between two events only SI\textsubscript{high} is possible and simple imperfective is not, as shown in \REF{bg:when-clause}. By contrast in temporal \textit{while}-clauses with two events temporally overlapping only simple imperfective is possible and  SI\textsubscript{high} is not, as shown in \REF{bg:while-clause}.


\ea\label{bg:when-clause} \gll Vseki păt sled kato \minsp{\{*?} stroeše / postrojavaše\} kăšta, tja se srutvaše. \\
%
every time after when {} built.\textsc{ipfv.3sg} {} built.\textsc{si}.\textsc{3sg} house it.\textsc{f} \textsc{refl} collapsed.\textsc{si}.\textsc{3sg}\\ \hfill BG
%
\glt `Every time ‘after’ he built (=finished building) a house, it collapsed.’
\z


\ea\label{bg:while-clause} \gll  Vseki păt dokato \{ stroeše / \minsp{*} postrojavaše\} kăšta, imaše incidenti.\\
%
every time while {} built.{\IPFV} {} {} built.\textsc{si}.\textsc{3sg} house have.\textsc{ipfv.3sg} incidents\\ \hfill BG
%
\glt ‘Every time while he was building a house, there were incidents.’
\z


\noindent One possible answer could be that there are two homophonous SI morphemes in Bulgarian, the one applying higher in syntax being specialized in expressing habituality consisting of a series of temporally non-overlapping bounded events (instead of serving the more general task of ``undoing the perfectivizing contribution of the prefix"). Panini's Principle (also referred to as Elsewhere Principle), according to which the application of a specific rule or operation overrides the application of a more general rule, would then link the function of expressing habituality to the specialized SI\textsubscript{high} form.

What remains to be studied in more detail is the interaction of SI morphology with different classes of superlexical prefixes. It is also necessary to extend the empirical scope to other Slavic languages in order to identify which languages pattern with Bulgarian and which ones pattern parametrically with Polish, and whether there are other possibilities. It also remains to be tested whether the observations we report in this study for past tense contexts can be extended to non-past tense as well.\footnote{We note that both present tense and imperfective aspect can have a multitude of meanings, e.g. non-actual readings, see  \citet{rivero2014}, \citet{nicolova2017}: 364; due to this, a study of the interaction of SI and present tense deserves a longer paper that we leave to future work. } 

Finally,  Bulgarian and Macedonian have two properties that other Slavic languages lack: they have definite articles and they have preserved the Imperfectum and Aorist tenses. To that end, it would be relevant to test whether the SI morphology interacts with Imperfectum and Aorist and behaves differently in the present and in the past, as well as whether number and referentiality of nominal complements impact the interpretation of either types of SI.  




%------------------------------------------------------------------------------

% \section*{Basics of LaTeX for Dorota}

% \begin{itemize}
% \item
% \textbf{commenting out:}
% Latex has two views: the source where we edit the \textbf{source} text, and the final \textbf{pdf}. All text that follows the sign \% in the source  is invisible in the pdf. We can use that to add comments to each other in the source on things to pay attention to or resolve, eg like this: %vcomm this is a comment by Vesela 
% more text % dcomm this is a comment by Dorota
% or something like this: %!vcomm this is a very important comment by Vesela etc....

% \item
% \textbf{quick commands:}
% Latex has a learning curve, but at least the quick commands like ctrl+B for bold, ctrl+I for italics, ctrl+C and V for copy-paste etc work. Also special Polish alphabet symbols work, which is great. And glossed examples align themselves, which is life-saving. 

% \item
% \textbf{quotation marks}
% in latex, proper quotes don't go like this: 'text' but like this: `text' - see the text in the source and the result in the pdf

% \item
% \textbf{special environments and symbols:}
% All other things require special coding that is not too hard, for example, to make something in caps lock \textsc{text}. We also have some special symbols that Latex uses as its own special environments. These are the symbols \% for commenting, \$ for math environment (formulas and some types of text formatting, like superscript and subscript),  backslashes - a single backslash starts a command, a double backslash gives a forced new line (used in tables), and the \{ brackets also have special use in that they give the scope of commands, such as the command for \textbf{bold} - you can see how the \{ brackets determine the beginning and the end of the bold. If we want to use them simply as brackets or percent sign, we use a backslash (as in this text). The underscore also has special uses, please don't use it in body text (it is used in math environment with curly brackets to give subscript text like this: regular text$_{subscript text}$).

% the math envrionment,  to make something 

%     \item 
% \textbf{new paragraphs:}
% If you want a new paragraph in the file, the text in the code has to be separated by an empty line, for example: 

% these two lines of text are not separated by an empty line
% and therefore they appear on the same line in the pdf

% but now this text is separated by an empty line and that is why it appears properly

% \item
% \textbf{labels:}
% The best part about Latex is that it keeps track of all examples and references, so we can move them around without having to change them individually. This is done by using unique descriptive \textbf{labels} that we choose to our liking and only we see. To label an example, we use the command \label{label-text-here}. For example, I can call an example with habitual in Bg \label{bg:SI:habitual}. To refer to the example bearing that label, we use the command \REF{label-text-here}, for example: "as shown in \REF{bg:SI:habitual}, BG high SI have only habitual readings"...

% \item
% \textbf{examples:} here is how to code examples in this document (it's not equivalent in all latex)

% \ea\label{sim:ex:german-verbs}\gll Hans \minsp{\{} schläft / schlief / \minsp{*} schlafen\}.\\
% Hans {} sleeps {} slept {} {} sleep.\textsc{inf}\\
% \glt `Hans \{sleeps / slept\}.'
% \z

% template:

% Single glossed example: 

% \ea\label{labeltext}\gll here are some source words.\\
% gloss gloss gloss gloss gloss\\ % notice that you don't need to do anything for the glosses to align themselves, the environment already does that for you, you just separate the words with a regular space in both the source line and the gloss line
% \glt `translation.'
% \z % this ends the example

% Example with subexamples:


% \ea\label{labeltext}
% \ea  first subexample \label{also-can-be-labeled} 
% \ex second subexample \label{labeled-differently} %note that the second subexample is with the code \ex not \ea 
% \z % this ends the (sub)example
% \z %in this case we need two - one to end the subexample and one to end the example

% Example with glossed subexamples
% \ea\label{labeltext}
% \ea \gll source words.\\
% gloss gloss\\
% \glt `translation.'
% \ex \gll source words of second subexample.\\
% gloss gloss gloss gloss gloss\\
% \glt `translation.'
% \z % the first z ends the subexample
% \z % this ends the example


% \item 
% \textbf{references:} all references will be replaced by labels. I will do this as we go along. Once there is a label for a reference, please just use the label when using the reference. Here is how to cite:
% \citet{labelname} for refs in the main text
% \citealp{label-name} for refs in parentheses


% \item
% Dorota, add your questions here 



% \end{itemize}



% % % Just comment out the input below when you're ready to go.
  % For a start: Do not forget to give your Overleaf project (this paper) a recognizable name. This one could be called, for instance, Simik et al: OSL template. You can change the name of the project by hovering over the gray title at the top of this page and clicking on the pencil icon.

\section{Introduction}\label{sim:sec:intro}

Language Science Press is a project run for linguists, but also by linguists. You are part of that and we rely on your collaboration to get at the desired result. Publishing with LangSci Press might mean a bit more work for the author (and for the volume editor), esp. for the less experienced ones, but it also gives you much more control of the process and it is rewarding to see the quality result.

Please follow the instructions below closely, it will save the volume editors, the series editors, and you alike a lot of time.

\sloppy This stylesheet is a further specification of three more general sources: (i) the Leipzig glossing rules \citep{leipzig-glossing-rules}, (ii) the generic style rules for linguistics (\url{https://www.eva.mpg.de/fileadmin/content_files/staff/haspelmt/pdf/GenericStyleRules.pdf}), and (iii) the Language Science Press guidelines \citep{Nordhoff.Muller2021}.\footnote{Notice the way in-text numbered lists should be written -- using small Roman numbers enclosed in brackets.} It is advisable to go through these before you start writing. Most of the general rules are not repeated here.\footnote{Do not worry about the colors of references and links. They are there to make the editorial process easier and will disappear prior to official publication.}

Please spend some time reading through these and the more general instructions. Your 30 minutes on this is likely to save you and us hours of additional work. Do not hesitate to contact the editors if you have any questions.

\section{Illustrating OSL commands and conventions}\label{sim:sec:osl-comm}

Below I illustrate the use of a number of commands defined in langsci-osl.tex (see the styles folder).

\subsection{Typesetting semantics}\label{sim:sec:sem}

See below for some examples of how to typeset semantic formulas. The examples also show the use of the sib-command (= ``semantic interpretation brackets''). Notice also the the use of the dummy curly brackets in \REF{sim:ex:quant}. They ensure that the spacing around the equation symbol is correct. 

\ea \ea \sib{dog}$^g=\textsc{dog}=\lambda x[\textsc{dog}(x)]$\label{sim:ex:dog}
\ex \sib{Some dog bit every boy}${}=\exists x[\textsc{dog}(x)\wedge\forall y[\textsc{boy}(y)\rightarrow \textsc{bit}(x,y)]]$\label{sim:ex:quant}
\z\z

\noindent Use noindent after example environments (but not after floats like tables or figures).

And here's a macro for semantic type brackets: The expression \textit{dog} is of type $\stb{e,t}$. Don't forget to place the whole type formula into a math-environment. An example of a more complex type, such as the one of \textit{every}: $\stb{s,\stb{\stb{e,t},\stb{e,t}}}$. You can of course also use the type in a subscript: dog$_{\stb{e,t}}$

We distinguish between metalinguistic constants that are translations of object language, which are typeset using small caps, see \REF{sim:ex:dog}, and logical constants. See the contrast in \REF{sim:ex:speaker}, where \textsc{speaker} (= serif) in \REF{sim:ex:speaker-a} is the denotation of the word \textit{speaker}, and \cnst{speaker} (= sans-serif) in \REF{sim:ex:speaker-b} is the function that maps the context $c$ to the speaker in that context.\footnote{Notice that both types of small caps are automatically turned into text-style, even if used in a math-environment. This enables you to use math throughout.}$^,$\footnote{Notice also that the notation entails the ``direct translation'' system from natural language to metalanguage, as entertained e.g. in \citet{Heim.Kratzer1998}. Feel free to devise your own notation when relying on the ``indirect translation'' system (see, e.g., \citealt{Coppock.Champollion2022}).}

\ea\label{sim:ex:speaker}
\ea \sib{The speaker is drunk}$^{g,c}=\textsc{drunk}\big(\iota x\,\textsc{speaker}(x)\big)$\label{sim:ex:speaker-a}
\ex \sib{I am drunk}$^{g,c}=\textsc{drunk}\big(\cnst{speaker}(c)\big)$\label{sim:ex:speaker-b}
\z\z

\noindent Notice that with more complex formulas, you can use bigger brackets indicating scope, cf. $($ vs. $\big($, as used in \REF{sim:ex:speaker}. Also notice the use of backslash plus comma, which produces additional space in math-environment.

\subsection{Examples and the minsp command}

Try to keep examples simple. But if you need to pack more information into an example or include more alternatives, you can resort to various brackets or slashes. For that, you will find the minsp-command useful. It works as follows:

\ea\label{sim:ex:german-verbs}\gll Hans \minsp{\{} schläft / schlief / \minsp{*} schlafen\}.\\
Hans {} sleeps {} slept {} {} sleep.\textsc{inf}\\
\glt `Hans \{sleeps / slept\}.'
\z

\noindent If you use the command, glosses will be aligned with the corresponding object language elements correctly. Notice also that brackets etc. do not receive their own gloss. Simply use closed curly brackets as the placeholder.

The minsp-command is not needed for grammaticality judgments used for the whole sentence. For that, use the native langsci-gb4e method instead, as illustrated below:

\ea[*]{\gll Das sein ungrammatisch.\\
that be.\textsc{inf} ungrammatical\\
\glt Intended: `This is ungrammatical.'\hfill (German)\label{sim:ex:ungram}}
\z

\noindent Also notice that translations should never be ungrammatical. If the original is ungrammatical, provide the intended interpretation in idiomatic English.

If you want to indicate the language and/or the source of the example, place this on the right margin of the translation line. Schematic information about relevant linguistic properties of the examples should be placed on the line of the example, as indicated below.

\ea\label{sim:ex:bailyn}\gll \minsp{[} Ėtu knigu] čitaet Ivan \minsp{(} často).\\
{} this book.{\ACC} read.{\PRS.3\SG} Ivan.{\NOM} {} often\\\hfill O-V-S-Adv
\glt `Ivan reads this book (often).'\hfill (Russian; \citealt[4]{Bailyn2004})
\z

\noindent Finally, notice that you can use the gloss macros for typing grammatical glosses, defined in langsci-lgr.sty. Place curly brackets around them.

\subsection{Citation commands and macros}

You can make your life easier if you use the following citation commands and macros (see code):

\begin{itemize}
    \item \citealt{Bailyn2004}: no brackets
    \item \citet{Bailyn2004}: year in brackets
    \item \citep{Bailyn2004}: everything in brackets
    \item \citepossalt{Bailyn2004}: possessive
    \item \citeposst{Bailyn2004}: possessive with year in brackets
\end{itemize}

\section{Trees}\label{s:tree}

Use the forest package for trees and place trees in a figure environment. \figref{sim:fig:CP} shows a simple example.\footnote{See \citet{VandenWyngaerd2017} for a simple and useful quickstart guide for the forest package.} Notice that figure (and table) environments are so-called floating environments. {\LaTeX} determines the position of the figure/table on the page, so it can appear elsewhere than where it appears in the code. This is not a bug, it is a property. Also for this reason, do not refer to figures/tables by using phrases like ``the table below''. Always use tabref/figref. If your terminal nodes represent object language, then these should essentially correspond to glosses, not to the original. For this reason, we recommend including an explicit example which corresponds to the tree, in this particular case \REF{sim:ex:czech-for-tree}.

\ea\label{sim:ex:czech-for-tree}\gll Co se řidič snažil dělat?\\
what {\REFL} driver try.{\PTCP.\SG.\MASC} do.{\INF}\\
\glt `What did the driver try to do?'
\z

\begin{figure}[ht]
% the [ht] option means that you prefer the placement of the figure HERE (=h) and if HERE is not possible, you prefer the TOP (=t) of a page
% \centering
    \begin{forest}
    for tree={s sep=1cm, inner sep=0, l=0}
    [CP
        [DP
            [what, roof, name=what]
        ]
        [C$'$
            [C
                [\textsc{refl}]
            ]
            [TP
                [DP
                    [driver, roof]
                ]
                [T$'$
                    [T [{[past]}]]
                    [VP
                        [V
                            [tried]
                        ]
                        [VP, s sep=2.2cm
                            [V
                                [do.\textsc{inf}]
                            ]
                            [t\textsubscript{what}, name=trace-what]
                        ]
                    ]
                ]
            ]
        ]
    ]
    \draw[->,overlay] (trace-what) to[out=south west, in=south, looseness=1.1] (what);
    % the overlay option avoids making the bounding box of the tree too large
    % the looseness option defines the looseness of the arrow (default = 1)
    \end{forest}
    \vspace{3ex} % extra vspace is added here because the arrow goes too deep to the caption; avoid such manual tweaking as much as possible; here it's necessary
    \caption{Proposed syntactic representation of \REF{sim:ex:czech-for-tree}}
    \label{sim:fig:CP}
\end{figure}

Do not use noindent after figures or tables (as you do after examples). Cases like these (where the noindent ends up missing) will be handled by the editors prior to publication.

\section{Italics, boldface, small caps, underlining, quotes}

See \citet{Nordhoff.Muller2021} for that. In short:

\begin{itemize}
    \item No boldface anywhere.
    \item No underlining anywhere (unless for very specific and well-defined technical notation; consult with editors).
    \item Small caps used for (i) introducing terms that are important for the paper (small-cap the term just ones, at a place where it is characterized/defined); (ii) metalinguistic translations of object-language expressions in semantic formulas (see \sectref{sim:sec:sem}); (iii) selected technical notions.
    \item Italics for object-language within text; exceptionally for emphasis/contrast.
    \item Single quotes: for translations/interpretations
    \item Double quotes: scare quotes; quotations of chunks of text.
\end{itemize}

\section{Cross-referencing}

Label examples, sections, tables, figures, possibly footnotes (by using the label macro). The name of the label is up to you, but it is good practice to follow this template: article-code:reference-type:unique-label. E.g. sim:ex:german would be a proper name for a reference within this paper (sim = short for the author(s); ex = example reference; german = unique name of that example).

\section{Syntactic notation}

Syntactic categories (N, D, V, etc.) are written with initial capital letters. This also holds for categories named with multiple letters, e.g. Foc, Top, Num, etc. Stick to this convention also when coming up with ad hoc categories, e.g. Cl (for clitic or classifier).

An exception from this rule are ``little'' categories, which are written with italics: \textit{v}, \textit{n}, \textit{v}P, etc.

Bar-levels must be typeset with bars/primes, not with an apostrophe. An easy way to do that is to use mathmode for the bar: C$'$, Foc$'$, etc.

Specifiers should be written this way: SpecCP, Spec\textit{v}P.

Features should be surrounded by square brackets, e.g., [past]. If you use plus and minus, be sure that these actually are plus and minus, and not e.g. a hyphen. Mathmode can help with that: [$+$sg], [$-$sg], [$\pm$sg]. See \sectref{sim:sec:hyphens-etc} for related information.

\section{Footnotes}

Absolutely avoid long footnotes. A footnote should not be longer than, say, {20\%} of the page. If you feel like you need a long footnote, make an explicit digression in the main body of the text.

Footnotes should always be placed at the end of whole sentences. Formulate the footnote in such a way that this is possible. Footnotes should always go after punctuation marks, never before. Do not place footnotes after individual words. Do not place footnotes in examples, tables, etc. If you have an urge to do that, place the footnote to the text that explains the example, table, etc.

Footnotes should always be formulated as full, self-standing sentences.

\section{Tables}

For your tables use the table plus tabularx environments. The tabularx environment lets you (and requires you in fact) to specify the width of the table and defines the X column (left-alignment) and the Y column (right-alignment). All X/Y columns will have the same width and together they will fill out the width of the rest of the table -- counting out all non-X/Y columns.

Always include a meaningful caption. The caption is designed to appear on top of the table, no matter where you place it in the code. Do not try to tweak with this. Tables are delimited with lsptoprule at the top and lspbottomrule at the bottom. The header is delimited from the rest with midrule. Vertical lines in tables are banned. An example is provided in \tabref{sim:tab:frequencies}. See \citet{Nordhoff.Muller2021} for more information. If you are typesetting a very complex table or your table is too large to fit the page, do not hesitate to ask the editors for help.

\begin{table}
\caption{Frequencies of word classes}
\label{sim:tab:frequencies}
 \begin{tabularx}{.77\textwidth}{lYYYY} %.77 indicates that the table will take up 77% of the textwidth
  \lsptoprule
            & nouns & verbs  & adjectives & adverbs\\
  \midrule
  absolute  &   12  &    34  &    23      & 13\\
  relative  &   3.1 &   8.9  &    5.7     & 3.2\\
  \lspbottomrule
 \end{tabularx}
\end{table}

\section{Figures}

Figures must have a good quality. If you use pictorial figures, consult the editors early on to see if the quality and format of your figure is sufficient. If you use simple barplots, you can use the barplot environment (defined in langsci-osl.sty). See \figref{sim:fig:barplot} for an example. The barplot environment has 5 arguments: 1. x-axis desription, 2. y-axis description, 3. width (relative to textwidth), 4. x-tick descriptions, 5. x-ticks plus y-values.

\begin{figure}
    \centering
    \barplot{Type of meal}{Times selected}{0.6}{Bread,Soup,Pizza}%
    {
    (Bread,61)
    (Soup,12)
    (Pizza,8)
    }
    \caption{A barplot example}
    \label{sim:fig:barplot}
\end{figure}

The barplot environment builds on the tikzpicture plus axis environments of the pgfplots package. It can be customized in various ways. \figref{sim:fig:complex-barplot} shows a more complex example.

\begin{figure}
  \begin{tikzpicture}
    \begin{axis}[
	xlabel={Level of \textsc{uniq/max}},  
	ylabel={Proportion of $\textsf{subj}\prec\textsf{pred}$}, 
	axis lines*=left, 
        width  = .6\textwidth,
	height = 5cm,
    	nodes near coords, 
    % 	nodes near coords style={text=black},
    	every node near coord/.append style={font=\tiny},
        nodes near coords align={vertical},
	ymin=0,
	ymax=1,
	ytick distance=.2,
	xtick=data,
	ylabel near ticks,
	x tick label style={font=\sffamily},
	ybar=5pt,
	legend pos=outer north east,
	enlarge x limits=0.3,
	symbolic x coords={+u/m, \textminus u/m},
	]
	\addplot[fill=red!30,draw=none] coordinates {
	    (+u/m,0.91)
        (\textminus u/m,0.84)
	};
	\addplot[fill=red,draw=none] coordinates {
	    (+u/m,0.80)
        (\textminus u/m,0.87)
	};
	\legend{\textsf{sg}, \textsf{pl}}
    \end{axis} 
  \end{tikzpicture} 
    \caption{Results divided by \textsc{number}}
    \label{sim:fig:complex-barplot}
\end{figure}

\section{Hyphens, dashes, minuses, math/logical operators}\label{sim:sec:hyphens-etc}

Be careful to distinguish between hyphens (-), dashes (--), and the minus sign ($-$). For in-text appositions, use only en-dashes -- as done here -- with spaces around. Do not use em-dashes (---). Using mathmode is a reliable way of getting the minus sign.

All equations (and typically also semantic formulas, see \sectref{sim:sec:sem}) should be typeset using mathmode. Notice that mathmode not only gets the math signs ``right'', but also has a dedicated spacing. For that reason, never write things like p$<$0.05, p $<$ 0.05, or p$<0.05$, but rather $p<0.05$. In case you need a two-place math or logical operator (like $\wedge$) but for some reason do not have one of the arguments represented overtly, you can use a ``dummy'' argument (curly brackets) to simulate the presence of the other one. Notice the difference between $\wedge p$ and ${}\wedge p$.

In case you need to use normal text within mathmode, use the text command. Here is an example: $\text{frequency}=.8$. This way, you get the math spacing right.

\section{Abbreviations}

The final abbreviations section should include all glosses. It should not include other ad hoc abbreviations (those should be defined upon first use) and also not standard abbreviations like NP, VP, etc.


\section{Bibliography}

Place your bibliography into localbibliography.bib. Important: Only place there the entries which you actually cite! You can make use of our OSL bibliography, which we keep clean and tidy and update it after the publication of each new volume. Contact the editors of your volume if you do not have the bib file yet. If you find the entry you need, just copy-paste it in your localbibliography.bib. The bibliography also shows many good examples of what a good bibliographic entry should look like.

See \citet{Nordhoff.Muller2021} for general information on bibliography. Some important things to keep in mind:

\begin{itemize}
    \item Journals should be cited as they are officially called (notice the difference between and, \&, capitalization, etc.).
    \item Journal publications should always include the volume number, the issue number (field ``number''), and DOI or stable URL (see below on that).
    \item Papers in collections or proceedings must include the editors of the volume (field ``editor''), the place of publication (field ``address'') and publisher.
    \item The proceedings number is part of the title of the proceedings. Do not place it into the ``volume'' field. The ``volume'' field with book/proceedings publications is reserved for the volume of that single book (e.g. NELS 40 proceedings might have vol. 1 and vol. 2).
    \item Avoid citing manuscripts as much as possible. If you need to cite them, try to provide a stable URL.
    \item Avoid citing presentations or talks. If you absolutely must cite them, be careful not to refer the reader to them by using ``see...''. The reader can't see them.
    \item If you cite a manuscript, presentation, or some other hard-to-define source, use the either the ``misc'' or ``unpublished'' entry type. The former is appropriate if the text cited corresponds to a book (the title will be printed in italics); the latter is appropriate if the text cited corresponds to an article or presentation (the title will be printed normally). Within both entries, use the ``howpublished'' field for any relevant information (such as ``Manuscript, University of \dots''). And the ``url'' field for the URL.
\end{itemize}

We require the authors to provide DOIs or URLs wherever possible, though not without limitations. The following rules apply:

\begin{itemize}
    \item If the publication has a DOI, use that. Use the ``doi'' field and write just the DOI, not the whole URL.
    \item If the publication has no DOI, but it has a stable URL (as e.g. JSTOR, but possibly also lingbuzz), use that. Place it in the ``url'' field, using the full address (https: etc.).
    \item Never use DOI and URL at the same time.
    \item If the official publication has no official DOI or stable URL (related to the official publication), do not replace these with other links. Do not refer to published works with lingbuzz links, for instance, as these typically lead to the unpublished (preprint) version. (There are exceptions where lingbuzz or semanticsarchive are the official publication venue, in which case these can of course be used.) Never use URLs leading to personal websites.
    \item If a paper has no DOI/URL, but the book does, do not use the book URL. Just use nothing.
\end{itemize}

\section*{Abbreviations}

\begin{tabularx}{.5\textwidth}{@{}lQ}
\textsc{1}&first person\\
\textsc{3}&third person\\
\textsc{aor}&aorist\\
\textsc{def}&definite\\
\textsc{distr}&distributive\\
\textsc{ipfv}&imperfective\\
\textsc{gen}&genitive\\
\textsc{f}&feminine gender\\
%\textsc{n/a}&not applicable\\
\end{tabularx}%
\begin{tabularx}{.5\textwidth}{lQ@{}}
\textsc{ipfv}&imperfective\\
\textsc{pfv}&perfective\\
\textsc{pp}&past participle\\
\textsc{prs}&present tense\\
\textsc{pst}&past\\
\textsc{refl}&reflexive\\
\textsc{sg}&singular\\
\textsc{si}&secondary imperfective\\

%&\\ % this dummy row achieves correct vertical alignment of both tables
\end{tabularx}

\section*{Acknowledgments}
This research was supported by the Polish National Science Center (NCN) grant SONATA BIS-11 HS2 (2021/42/E/HS2/00143). We are very grateful to the audiences of FDSL in Berlin, to the anonymous reviewers, and to the editors and especially Berit Gehrke for their insightful comments.

\printbibliography[heading=subbibliography,notkeyword=this]

\end{document}
