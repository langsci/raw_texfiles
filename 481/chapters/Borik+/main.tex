\documentclass[output=paper,colorlinks,citecolor=brown]{langscibook}
\ChapterDOI{10.5281/zenodo.15394168}
%\bibliography{localbibliography}

\author{Olga Borik\orcid{0000-0003-1255-5962}\affiliation{Universidad Nacional de Educacion a Distancia} and Bert Le Bruyn\orcid{0000-0001-9090-7383}\affiliation{Utrecht University} and Jianan Liu\orcid{}\affiliation{Utrecht University} and Daria Seres\orcid{0000-0002-9044-8516}\affiliation{University of Graz} }
% replace the above with you and your coauthors
% rules for affiliation: If there's an official English version, use that (find out on the official website of the university); if not, use the original
% orcid doesn't appear printed; it's metainformation used for later indexing

%%% uncomment the following line if you are a single author or all authors have the same affiliation
% \SetupAffiliations{mark style=none}

%% in case the running head with authors exceeds one line (which is the case in this example document), use one of the following methods to turn it into a single line; otherwise comment the line below out with % and ignore it

% \lehead{Borik, Le Bruyn, Liu, \& Seres}

%\lehead{John \& Jane }

\title{Bare nouns in Slavic and beyond}
% replace the above with your paper title
%%% provide a shorter version of your title in case it doesn't fit a single line in the running head
% in this form: \title[short title]{full title}
\abstract{The article presents a study of the distribution of singular bare nouns in three Slavic languages, Russian, Polish and Macedonian, based on parallel translation corpora. The distribution of bare singulars in Russian and Polish shows that they freely appear in definite and indefinite contexts, which makes it possible to classify these languages as truly articleless. Macedonian bare singulars frequently appear in indefinite contexts, alongside with nouns accompanied by an indefinite marker \textit{one}, whose status require further scrutiny. The data reported in this study call for a theoretical account of bare nouns which allows for fine-grained variation in their distribution across domains and languages, taking into consideration a broader/narrower use of competing expressions. 

%in its turn, has a definite article and an emerging indefinite marker whose status requires further scrutiny. These cross-linguistic data argue against a uniform definiteness semantics of singular bare NPs and can best be accounted for with a classical blocking analysis in which fine-grained variation in the distribution of bare nominals follows from the broader/narrower use of article-like expressions.  %The abstract shouldn't be longer than 150 words


\keywords{definiteness, indefiniteness, bare nouns, parallel corpora, translation mining}
}

\begin{document}
\maketitle

% Just comment out the input below when you're ready to go.
%For a start: Do not forget to give your Overleaf project (this paper) a recognizable name. This one could be called, for instance, Simik et al: OSL template. You can change the name of the project by hovering over the gray title at the top of this page and clicking on the pencil icon.

\section{Introduction}\label{sim:sec:intro}

Language Science Press is a project run for linguists, but also by linguists. You are part of that and we rely on your collaboration to get at the desired result. Publishing with LangSci Press might mean a bit more work for the author (and for the volume editor), esp. for the less experienced ones, but it also gives you much more control of the process and it is rewarding to see the quality result.

Please follow the instructions below closely, it will save the volume editors, the series editors, and you alike a lot of time.

\sloppy This stylesheet is a further specification of three more general sources: (i) the Leipzig glossing rules \citep{leipzig-glossing-rules}, (ii) the generic style rules for linguistics (\url{https://www.eva.mpg.de/fileadmin/content_files/staff/haspelmt/pdf/GenericStyleRules.pdf}), and (iii) the Language Science Press guidelines \citep{Nordhoff.Muller2021}.\footnote{Notice the way in-text numbered lists should be written -- using small Roman numbers enclosed in brackets.} It is advisable to go through these before you start writing. Most of the general rules are not repeated here.\footnote{Do not worry about the colors of references and links. They are there to make the editorial process easier and will disappear prior to official publication.}

Please spend some time reading through these and the more general instructions. Your 30 minutes on this is likely to save you and us hours of additional work. Do not hesitate to contact the editors if you have any questions.

\section{Illustrating OSL commands and conventions}\label{sim:sec:osl-comm}

Below I illustrate the use of a number of commands defined in langsci-osl.tex (see the styles folder).

\subsection{Typesetting semantics}\label{sim:sec:sem}

See below for some examples of how to typeset semantic formulas. The examples also show the use of the sib-command (= ``semantic interpretation brackets''). Notice also the the use of the dummy curly brackets in \REF{sim:ex:quant}. They ensure that the spacing around the equation symbol is correct. 

\ea \ea \sib{dog}$^g=\textsc{dog}=\lambda x[\textsc{dog}(x)]$\label{sim:ex:dog}
\ex \sib{Some dog bit every boy}${}=\exists x[\textsc{dog}(x)\wedge\forall y[\textsc{boy}(y)\rightarrow \textsc{bit}(x,y)]]$\label{sim:ex:quant}
\z\z

\noindent Use noindent after example environments (but not after floats like tables or figures).

And here's a macro for semantic type brackets: The expression \textit{dog} is of type $\stb{e,t}$. Don't forget to place the whole type formula into a math-environment. An example of a more complex type, such as the one of \textit{every}: $\stb{s,\stb{\stb{e,t},\stb{e,t}}}$. You can of course also use the type in a subscript: dog$_{\stb{e,t}}$

We distinguish between metalinguistic constants that are translations of object language, which are typeset using small caps, see \REF{sim:ex:dog}, and logical constants. See the contrast in \REF{sim:ex:speaker}, where \textsc{speaker} (= serif) in \REF{sim:ex:speaker-a} is the denotation of the word \textit{speaker}, and \cnst{speaker} (= sans-serif) in \REF{sim:ex:speaker-b} is the function that maps the context $c$ to the speaker in that context.\footnote{Notice that both types of small caps are automatically turned into text-style, even if used in a math-environment. This enables you to use math throughout.}$^,$\footnote{Notice also that the notation entails the ``direct translation'' system from natural language to metalanguage, as entertained e.g. in \citet{Heim.Kratzer1998}. Feel free to devise your own notation when relying on the ``indirect translation'' system (see, e.g., \citealt{Coppock.Champollion2022}).}

\ea\label{sim:ex:speaker}
\ea \sib{The speaker is drunk}$^{g,c}=\textsc{drunk}\big(\iota x\,\textsc{speaker}(x)\big)$\label{sim:ex:speaker-a}
\ex \sib{I am drunk}$^{g,c}=\textsc{drunk}\big(\cnst{speaker}(c)\big)$\label{sim:ex:speaker-b}
\z\z

\noindent Notice that with more complex formulas, you can use bigger brackets indicating scope, cf. $($ vs. $\big($, as used in \REF{sim:ex:speaker}. Also notice the use of backslash plus comma, which produces additional space in math-environment.

\subsection{Examples and the minsp command}

Try to keep examples simple. But if you need to pack more information into an example or include more alternatives, you can resort to various brackets or slashes. For that, you will find the minsp-command useful. It works as follows:

\ea\label{sim:ex:german-verbs}\gll Hans \minsp{\{} schläft / schlief / \minsp{*} schlafen\}.\\
Hans {} sleeps {} slept {} {} sleep.\textsc{inf}\\
\glt `Hans \{sleeps / slept\}.'
\z

\noindent If you use the command, glosses will be aligned with the corresponding object language elements correctly. Notice also that brackets etc. do not receive their own gloss. Simply use closed curly brackets as the placeholder.

The minsp-command is not needed for grammaticality judgments used for the whole sentence. For that, use the native langsci-gb4e method instead, as illustrated below:

\ea[*]{\gll Das sein ungrammatisch.\\
that be.\textsc{inf} ungrammatical\\
\glt Intended: `This is ungrammatical.'\hfill (German)\label{sim:ex:ungram}}
\z

\noindent Also notice that translations should never be ungrammatical. If the original is ungrammatical, provide the intended interpretation in idiomatic English.

If you want to indicate the language and/or the source of the example, place this on the right margin of the translation line. Schematic information about relevant linguistic properties of the examples should be placed on the line of the example, as indicated below.

\ea\label{sim:ex:bailyn}\gll \minsp{[} Ėtu knigu] čitaet Ivan \minsp{(} často).\\
{} this book.{\ACC} read.{\PRS.3\SG} Ivan.{\NOM} {} often\\\hfill O-V-S-Adv
\glt `Ivan reads this book (often).'\hfill (Russian; \citealt[4]{Bailyn2004})
\z

\noindent Finally, notice that you can use the gloss macros for typing grammatical glosses, defined in langsci-lgr.sty. Place curly brackets around them.

\subsection{Citation commands and macros}

You can make your life easier if you use the following citation commands and macros (see code):

\begin{itemize}
    \item \citealt{Bailyn2004}: no brackets
    \item \citet{Bailyn2004}: year in brackets
    \item \citep{Bailyn2004}: everything in brackets
    \item \citepossalt{Bailyn2004}: possessive
    \item \citeposst{Bailyn2004}: possessive with year in brackets
\end{itemize}

\section{Trees}\label{s:tree}

Use the forest package for trees and place trees in a figure environment. \figref{sim:fig:CP} shows a simple example.\footnote{See \citet{VandenWyngaerd2017} for a simple and useful quickstart guide for the forest package.} Notice that figure (and table) environments are so-called floating environments. {\LaTeX} determines the position of the figure/table on the page, so it can appear elsewhere than where it appears in the code. This is not a bug, it is a property. Also for this reason, do not refer to figures/tables by using phrases like ``the table below''. Always use tabref/figref. If your terminal nodes represent object language, then these should essentially correspond to glosses, not to the original. For this reason, we recommend including an explicit example which corresponds to the tree, in this particular case \REF{sim:ex:czech-for-tree}.

\ea\label{sim:ex:czech-for-tree}\gll Co se řidič snažil dělat?\\
what {\REFL} driver try.{\PTCP.\SG.\MASC} do.{\INF}\\
\glt `What did the driver try to do?'
\z

\begin{figure}[ht]
% the [ht] option means that you prefer the placement of the figure HERE (=h) and if HERE is not possible, you prefer the TOP (=t) of a page
% \centering
    \begin{forest}
    for tree={s sep=1cm, inner sep=0, l=0}
    [CP
        [DP
            [what, roof, name=what]
        ]
        [C$'$
            [C
                [\textsc{refl}]
            ]
            [TP
                [DP
                    [driver, roof]
                ]
                [T$'$
                    [T [{[past]}]]
                    [VP
                        [V
                            [tried]
                        ]
                        [VP, s sep=2.2cm
                            [V
                                [do.\textsc{inf}]
                            ]
                            [t\textsubscript{what}, name=trace-what]
                        ]
                    ]
                ]
            ]
        ]
    ]
    \draw[->,overlay] (trace-what) to[out=south west, in=south, looseness=1.1] (what);
    % the overlay option avoids making the bounding box of the tree too large
    % the looseness option defines the looseness of the arrow (default = 1)
    \end{forest}
    \vspace{3ex} % extra vspace is added here because the arrow goes too deep to the caption; avoid such manual tweaking as much as possible; here it's necessary
    \caption{Proposed syntactic representation of \REF{sim:ex:czech-for-tree}}
    \label{sim:fig:CP}
\end{figure}

Do not use noindent after figures or tables (as you do after examples). Cases like these (where the noindent ends up missing) will be handled by the editors prior to publication.

\section{Italics, boldface, small caps, underlining, quotes}

See \citet{Nordhoff.Muller2021} for that. In short:

\begin{itemize}
    \item No boldface anywhere.
    \item No underlining anywhere (unless for very specific and well-defined technical notation; consult with editors).
    \item Small caps used for (i) introducing terms that are important for the paper (small-cap the term just ones, at a place where it is characterized/defined); (ii) metalinguistic translations of object-language expressions in semantic formulas (see \sectref{sim:sec:sem}); (iii) selected technical notions.
    \item Italics for object-language within text; exceptionally for emphasis/contrast.
    \item Single quotes: for translations/interpretations
    \item Double quotes: scare quotes; quotations of chunks of text.
\end{itemize}

\section{Cross-referencing}

Label examples, sections, tables, figures, possibly footnotes (by using the label macro). The name of the label is up to you, but it is good practice to follow this template: article-code:reference-type:unique-label. E.g. sim:ex:german would be a proper name for a reference within this paper (sim = short for the author(s); ex = example reference; german = unique name of that example).

\section{Syntactic notation}

Syntactic categories (N, D, V, etc.) are written with initial capital letters. This also holds for categories named with multiple letters, e.g. Foc, Top, Num, etc. Stick to this convention also when coming up with ad hoc categories, e.g. Cl (for clitic or classifier).

An exception from this rule are ``little'' categories, which are written with italics: \textit{v}, \textit{n}, \textit{v}P, etc.

Bar-levels must be typeset with bars/primes, not with an apostrophe. An easy way to do that is to use mathmode for the bar: C$'$, Foc$'$, etc.

Specifiers should be written this way: SpecCP, Spec\textit{v}P.

Features should be surrounded by square brackets, e.g., [past]. If you use plus and minus, be sure that these actually are plus and minus, and not e.g. a hyphen. Mathmode can help with that: [$+$sg], [$-$sg], [$\pm$sg]. See \sectref{sim:sec:hyphens-etc} for related information.

\section{Footnotes}

Absolutely avoid long footnotes. A footnote should not be longer than, say, {20\%} of the page. If you feel like you need a long footnote, make an explicit digression in the main body of the text.

Footnotes should always be placed at the end of whole sentences. Formulate the footnote in such a way that this is possible. Footnotes should always go after punctuation marks, never before. Do not place footnotes after individual words. Do not place footnotes in examples, tables, etc. If you have an urge to do that, place the footnote to the text that explains the example, table, etc.

Footnotes should always be formulated as full, self-standing sentences.

\section{Tables}

For your tables use the table plus tabularx environments. The tabularx environment lets you (and requires you in fact) to specify the width of the table and defines the X column (left-alignment) and the Y column (right-alignment). All X/Y columns will have the same width and together they will fill out the width of the rest of the table -- counting out all non-X/Y columns.

Always include a meaningful caption. The caption is designed to appear on top of the table, no matter where you place it in the code. Do not try to tweak with this. Tables are delimited with lsptoprule at the top and lspbottomrule at the bottom. The header is delimited from the rest with midrule. Vertical lines in tables are banned. An example is provided in \tabref{sim:tab:frequencies}. See \citet{Nordhoff.Muller2021} for more information. If you are typesetting a very complex table or your table is too large to fit the page, do not hesitate to ask the editors for help.

\begin{table}
\caption{Frequencies of word classes}
\label{sim:tab:frequencies}
 \begin{tabularx}{.77\textwidth}{lYYYY} %.77 indicates that the table will take up 77% of the textwidth
  \lsptoprule
            & nouns & verbs  & adjectives & adverbs\\
  \midrule
  absolute  &   12  &    34  &    23      & 13\\
  relative  &   3.1 &   8.9  &    5.7     & 3.2\\
  \lspbottomrule
 \end{tabularx}
\end{table}

\section{Figures}

Figures must have a good quality. If you use pictorial figures, consult the editors early on to see if the quality and format of your figure is sufficient. If you use simple barplots, you can use the barplot environment (defined in langsci-osl.sty). See \figref{sim:fig:barplot} for an example. The barplot environment has 5 arguments: 1. x-axis desription, 2. y-axis description, 3. width (relative to textwidth), 4. x-tick descriptions, 5. x-ticks plus y-values.

\begin{figure}
    \centering
    \barplot{Type of meal}{Times selected}{0.6}{Bread,Soup,Pizza}%
    {
    (Bread,61)
    (Soup,12)
    (Pizza,8)
    }
    \caption{A barplot example}
    \label{sim:fig:barplot}
\end{figure}

The barplot environment builds on the tikzpicture plus axis environments of the pgfplots package. It can be customized in various ways. \figref{sim:fig:complex-barplot} shows a more complex example.

\begin{figure}
  \begin{tikzpicture}
    \begin{axis}[
	xlabel={Level of \textsc{uniq/max}},  
	ylabel={Proportion of $\textsf{subj}\prec\textsf{pred}$}, 
	axis lines*=left, 
        width  = .6\textwidth,
	height = 5cm,
    	nodes near coords, 
    % 	nodes near coords style={text=black},
    	every node near coord/.append style={font=\tiny},
        nodes near coords align={vertical},
	ymin=0,
	ymax=1,
	ytick distance=.2,
	xtick=data,
	ylabel near ticks,
	x tick label style={font=\sffamily},
	ybar=5pt,
	legend pos=outer north east,
	enlarge x limits=0.3,
	symbolic x coords={+u/m, \textminus u/m},
	]
	\addplot[fill=red!30,draw=none] coordinates {
	    (+u/m,0.91)
        (\textminus u/m,0.84)
	};
	\addplot[fill=red,draw=none] coordinates {
	    (+u/m,0.80)
        (\textminus u/m,0.87)
	};
	\legend{\textsf{sg}, \textsf{pl}}
    \end{axis} 
  \end{tikzpicture} 
    \caption{Results divided by \textsc{number}}
    \label{sim:fig:complex-barplot}
\end{figure}

\section{Hyphens, dashes, minuses, math/logical operators}\label{sim:sec:hyphens-etc}

Be careful to distinguish between hyphens (-), dashes (--), and the minus sign ($-$). For in-text appositions, use only en-dashes -- as done here -- with spaces around. Do not use em-dashes (---). Using mathmode is a reliable way of getting the minus sign.

All equations (and typically also semantic formulas, see \sectref{sim:sec:sem}) should be typeset using mathmode. Notice that mathmode not only gets the math signs ``right'', but also has a dedicated spacing. For that reason, never write things like p$<$0.05, p $<$ 0.05, or p$<0.05$, but rather $p<0.05$. In case you need a two-place math or logical operator (like $\wedge$) but for some reason do not have one of the arguments represented overtly, you can use a ``dummy'' argument (curly brackets) to simulate the presence of the other one. Notice the difference between $\wedge p$ and ${}\wedge p$.

In case you need to use normal text within mathmode, use the text command. Here is an example: $\text{frequency}=.8$. This way, you get the math spacing right.

\section{Abbreviations}

The final abbreviations section should include all glosses. It should not include other ad hoc abbreviations (those should be defined upon first use) and also not standard abbreviations like NP, VP, etc.


\section{Bibliography}

Place your bibliography into localbibliography.bib. Important: Only place there the entries which you actually cite! You can make use of our OSL bibliography, which we keep clean and tidy and update it after the publication of each new volume. Contact the editors of your volume if you do not have the bib file yet. If you find the entry you need, just copy-paste it in your localbibliography.bib. The bibliography also shows many good examples of what a good bibliographic entry should look like.

See \citet{Nordhoff.Muller2021} for general information on bibliography. Some important things to keep in mind:

\begin{itemize}
    \item Journals should be cited as they are officially called (notice the difference between and, \&, capitalization, etc.).
    \item Journal publications should always include the volume number, the issue number (field ``number''), and DOI or stable URL (see below on that).
    \item Papers in collections or proceedings must include the editors of the volume (field ``editor''), the place of publication (field ``address'') and publisher.
    \item The proceedings number is part of the title of the proceedings. Do not place it into the ``volume'' field. The ``volume'' field with book/proceedings publications is reserved for the volume of that single book (e.g. NELS 40 proceedings might have vol. 1 and vol. 2).
    \item Avoid citing manuscripts as much as possible. If you need to cite them, try to provide a stable URL.
    \item Avoid citing presentations or talks. If you absolutely must cite them, be careful not to refer the reader to them by using ``see...''. The reader can't see them.
    \item If you cite a manuscript, presentation, or some other hard-to-define source, use the either the ``misc'' or ``unpublished'' entry type. The former is appropriate if the text cited corresponds to a book (the title will be printed in italics); the latter is appropriate if the text cited corresponds to an article or presentation (the title will be printed normally). Within both entries, use the ``howpublished'' field for any relevant information (such as ``Manuscript, University of \dots''). And the ``url'' field for the URL.
\end{itemize}

We require the authors to provide DOIs or URLs wherever possible, though not without limitations. The following rules apply:

\begin{itemize}
    \item If the publication has a DOI, use that. Use the ``doi'' field and write just the DOI, not the whole URL.
    \item If the publication has no DOI, but it has a stable URL (as e.g. JSTOR, but possibly also lingbuzz), use that. Place it in the ``url'' field, using the full address (https: etc.).
    \item Never use DOI and URL at the same time.
    \item If the official publication has no official DOI or stable URL (related to the official publication), do not replace these with other links. Do not refer to published works with lingbuzz links, for instance, as these typically lead to the unpublished (preprint) version. (There are exceptions where lingbuzz or semanticsarchive are the official publication venue, in which case these can of course be used.) Never use URLs leading to personal websites.
    \item If a paper has no DOI/URL, but the book does, do not use the book URL. Just use nothing.
\end{itemize}

\section{Introduction}\label{intro}
Referring is one of the main functions of natural language, and speakers of different languages use a variety of linguistic means and mechanisms to express different types of reference. %The expression of reference in natural language is a vast research topic, which involves a cross-linguistic study of a range of syntactic, semantics and pragmatic linguistic phenomena. The current paper makes a contribution to this research agenda. 
In the empirical study that we present here, we focus on the syntax--semantics interface of bare nouns (BNs) and examine their distributional properties in Russian, Polish and Macedonian, languages that belong to the East Slavic, West Slavic and South Slavic subgroups, respectively. In particular, we address the issue of a comparative distribution of bare singular nouns (BSs) in the definite and indefinite domain across the three languages. 

In terms of definiteness/indefiniteness marking, Russian and Polish are typically classified as articleless languages \citep{wals}, that is, having no dedicated morphosyntactic marker to express definiteness or indefiniteness. We thus expect nominals to appear in their bare form in all argument positions in both languages. This straightforward expectation is in line not only with the traditional descriptive grammars, such as \citet{bor+:Svedova1980}, but also with some formal semantic literature, such as \citet{Chierchia1998}, \citet{Geist2010}, among others. Other formal approaches, most notably \citet{Dayal2004,Dayal2018}, argue that number plays a crucial role in the distribution of BNs in articleless languages, making different predictions for bare plurals (henceforth BPs) and BSs. In particular, \citet{Dayal2004} argues that BSs do not get an indefinite interpretation in languages without articles, while BPs can get narrow scope indefinite readings. Therefore, BSs are predicted to be largely restricted to definite contexts. Our focus on BSs allows us to check the predictions made by \citeauthor{Dayal2004}'s theory as opposed to more traditional approaches. 
%. The former supposition is in line with the tradition view in Slavic linguistics The latter is in line with \citeposst{Dayal2004} proposal, who claims that bare singulars in languages without articles cannot be \textit{bona fide} indefinites.  

Macedonian is usually described as a language with a definite article (\citealt{Friedman1993}, \citealt{Tomić2006}, among many others). The definite article in Macedonian is postpositive and morphologically bound. It is typically added to the first element of a nominal phrase\footnote{We use the term \textit{nominal phrase} to abstract away from the DP/NP debate, prominent mostly in the syntactic literature on Slavic. See, for instance, \citet{bor+:boskovic08}.} (e.g., \textit{kuče-to} \lq the dog', \textit{ubavo-to kuče} \lq the beautiful dog'), and is inflected for number and gender (e.g., \textit{maž-ot} \textsc{m.sg} \lq the man', \textit{žena-ta} \textsc{f.sg} \lq the woman', \textit{dete-to} \textsc{n.sg} \lq the child', \textit{maži-te/ženi-te }\textsc{m/f.pl} \lq the men/the women', \textit{deca-ta} \textsc{n.pl} \lq the children').  BSs are also admissible in argument positions in Macedonian, while there is no agreement on their interpretation in the literature (\citealt{Weiss2004}, \citealt{Topolinjska2009}, among others). The most widely accepted assumption is that BSs in Macedonian appear in indefinite contexts, although it has also been noticed that a determiner \textit{eden} \lq one' is often used to mark indefiniteness in this language (\citealt{Tomić2006}).

Our study aims at answering the following research questions:
\begin{enumerate}
    \item What is the distribution of BSs in Russian and Polish as languages without articles? Do they appear in both definite and indefinite domains or do we observe significant differences in the distribution of BSs across domains? 
\item What is the status of BSs in Macedonian in indefinite contexts as compared to Russian and Polish?
\item What is the status of BSs in Macedonian in definite contexts? 
\end{enumerate}

\noindent To address these questions, we ran a parallel corpus study to analyze nominal phrases that appear in both definite and indefinite contexts in the three languages, with a critical look at the distribution of BSs in each of the domains. 

For the definite domain, the expectations are rather straightforward: both traditional descriptive and formal literature seem to converge on the idea that BSs freely appear in definite contexts in Russian and Polish, whereas in Macedonian we expect the definite article to dominate. However, the status of the definite marker as an article in Macedonian is not uncontroversial: \citet[313]{Rudin2021}, for instance, suggests that it might be a type of demonstrative rather than an article. Semantic literature repeatedly stresses similarities between demonstrative NPs and definite descriptions \citep[e.g.,][]{Roberts2002, bor+:Elbourne2008}, as well as the need to differentiate between the two \citep{Lyons1999}. We include demonstrative nominals in our empirical study and look at the relative distribution of NPs specified by demonstratives vs. definite nominal phrases in Macedonian or BSs in Russian and Polish in the definite domain.%\textcolor{green}{\footnote{\textcolor{green}{By the terms \textit{definite description} or \textit{definite NP} we mean a structure of the form of "the X" where X is a noun phrase or a singular common noun.}}} 

For the indefinite domain, existing analyses diverge when it comes to predictions. Traditional descriptions do not report any irregularities or asymmetries in the distribution of BSs across definite vs. indefinite domains, so they seem to predict that BSs can freely appear in indefinite contexts. However, claims have been made that in Polish and Macedonian, the indefinite marker \textsc{one}\footnote{The English \textsc{one} is used as a cover term for language specific \textit{odin} (Russian), \textit{jeden} (Polish) and \textit{eden} (Macedonian) and their respective forms.} is acquiring (or has acquired) the status of an indefinite article (\citealt{Hwaszcz.Kedzierska2018, Molinari2022} with reference to Polish; \citealt
{Tomić2006} with reference to Macedonian).
 The prediction that these proposals make is that the marker \textsc{one} will frequently appear in the indefinite domain in these languages, competing with or prevailing over BSs. The same prediction is made by \citet{Dayal2004,Dayal2018}, who 
%develops one of the few formal, full-fledged analyses of BNs in languages without articles. In particular, Dayal argues that BSs in such languages do not get an indefinite interpretation at all. She 
takes Hindi as a representative example of an articleless language and argues  that it typically resorts to a construction with \textsc{one} in those contexts where English uses the indefinite article. Applying Dayal's analysis to Russian and Polish,\footnote{Dayal does discuss Russian, and we assume that the proposal extends to other languages without articles like Polish, as it is based on general, language-independent semantic principles and mechanisms.} we expect BSs in these languages to be severely restricted in the indefinite domain, as opposed to the construction with \textsc{one}, which should dominate. In other analyses, the marker \textsc{one} in Russian is assumed to mark specificity rather than function as an indefinite article \citep{Ionin2013}, which predicts its appearance only in specific indefinite contexts, converging with the predictions of \citet{Geist2010}, who argues that BSs in Russian can only get a non-specific reading.\footnote{\citeposst{Geist2010} predictions should be relativized to the information structure since she argues that indefinite BSs cannot serve as aboutness topics.}

To get a broader cross-linguistic perspective, we compare parallel-corpus data for Russian, Polish and Macedonian to corpus results for Mandarin and German, two non-Slavic languages. Mandarin functions as a control language for Russian and Polish, as it is usually assumed to be an articleless language \citep{Li2021}, whereas German functions as a control language for Macedonian, as both have a definite article. 

In order to investigate the distribution of BSs in definite and indefinite contexts in the three Slavic languages we ran a parallel-corpus study, described in detail in \S \ref{study}. We present the results of our study in \S \ref{results}, followed by a general discussion in \S \ref{discussion}. \S \ref{conclusions} concludes the paper. 

%As far as Macedonian is concerned, we first evaluate its status as a language without an indefinite article by comparing the use of bare NPs in indefinite contexts to that of Russian, Polish and Mandarin. And then we evaluate its status as a language with a definite article, for this we use German as a control language.

%The above mentioned aims of cross-linguistic research are achieved by means of an empirical study was carried out using \textit{Translation Mining} methodology, which is described in \S \ref{study}. We further present the data and the results of the parallel-corpus study (Subsections \ref{data} and \ref{results}). The study is followed by a general discussion (\S \ref{discussion}) and conclusions (\S \ref{conclusions}).

\section{Data and methodology}\label{study}
%\subsection{Methodology: Translation Mining}\label{method}

%We adopt a translation mining parallel-corpus approach to cross-linguistic semantics \citep{Bremmers.et.al2021}, a method that allows us to compare the distribution of grammatical items in different languages in parallel. Moreover, this method also allows for tracing interactions of various grammatical items in multiple languages in similar contexts. The translation mining approach builds on the assumption that the meanings of the translations of the different referential expressions in the same contexts are as closely related to each other as the grammars of the respective languages allow them to be. Another important assumption is that translations are representative of their target languages (\textit{the target language representativeness hypothesis)}. For a more detailed discussion of the methodology and its caveats, see \citealt{LeBruyn.et.al2022a}; \citealt{LeBruyn.deSwart2022}).

We use parallel corpora to study the distribution of grammatical items in different languages in parallel, an approach that has recently gained traction in the formal literature for the study
of a variety of empirical domains, for example, tense and aspect (see -- among others --  
 \citealt{Fuchs.Gonzales2022}; \citealt{Gehrke2022}; \citealt{Mo2022}; \citealt{Mulder.et.al2022}), negation (\citealt{deSwart2020}) and
reference \citep{Bremmers.et.al2021}. Parallel corpus research builds on the assumption that the meanings of the original and the translations are as closely related to each other as the grammars of the respective languages allow them to be. Another important assumption is
that translations are representative of their target languages (\textit{the target language
representativeness hypothesis}). For a more detailed discussion of the methodology and its
caveats, see \citet{LeBruyn.et.al2022a,LeBruyn.deSwart2022}.

This study uses a translation corpus built on the first chapter of J. K. Rowling's \textit{Harry Potter and the Philosopher's Stone}, a novel written in English and translated into many typologically diverse languages. English grammatically marks the distinction between definiteness and indefiniteness, which allows us to easily detect all definite and indefinite referential expressions in the source text. We selected all (in)definite referential expressions (\textit{a N, the N, N-s, the N-s}) with their aligned translations in Russian, Polish, Macedonian, Mandarin and German (n=284) and manually annotated the corresponding NP forms in all the target languages.\footnote{Because some referential expressions are not translated and because of issues of automatic alignment, some data are literally lost in translation. Our dataset for this study includes referential expressions that have translations in all five languages under scrutiny. These numbers are expected not to be identical to the ones in \citet{Liuetal2022}, a study that we conducted for a wider set of languages using the same methodology.} At this point, it is important to emphasize that our methodology involves the annotation of forms (but not meanings) in the same contexts across the languages under study. 

%We examined 284 contexts in total, including translations of plural noun phrases: 44 occurrences of \textit{N}\textsubscript{plural} and 34 occurrences of \textit{the N}\textsubscript{plural}. 

Since this paper focuses on the singular domain, we limit our quantitative analysis to the singular paradigm only.\footnote{Although there will be a short discussion of plurals in \S \ref{McPl}.} Apart from theoretical reasons discussed in \S \ref{intro}, plurals were excluded due to their relatively low frequency in our dataset and the interaction of plural definites with proper names (e.g., \textit{The Potters, The Dursleys}). Thus, our final dataset includes the translations of \textit{a N} (n=82) and \textit{the N}\textsubscript{sing} (n=124) constructions into Russian, Polish, Macedonian, as well as Mandarin and German, which are used as control languages in this study. 

%\textit{N-s} (bare plural) and \textit{the N-s} (definite plural) from the first chapter of  \textit{Harry Potter and the Philosopher's Stone}.\footnote{This book was chosen as a source corpus because of its availability in many typologically diverse languages. Also, it contains both narrative discourse and direct speech (fictional dialogues that can serve as a proxy for spoken language).} Then, 
 
%only but to include the plural paradigm in our qualitative interpretation of the data.
%In the current study, we apply the translation mining methodology to three typologically close languages, belonging to the  Slavic group: Russian, Polish (no articles) and Macedonian (definite article only). We further compare the data obtained from the Slavic languages to the data previously collected from German and Mandarin, which are more typologically distinct.

 An example of an English source \textit{the N} expression (\ref{ex:1a}) and its translations from the parallel corpus are shown below. 

\ea \label{ex:1}
\ea \label{ex:1a} Mr Dursley might have been drifting into an uneasy sleep, but \textit{the cat} on the wall outside was showing no sign of sleepiness.
\ex \label{ex:1b} Dolgo\v{z}dannyj i nespokojnyj son u\v{z}e prinjal v svoi ob''jatija mistera Darsli, a sidev\v{s}aja na ego zabore \textit{ko\v{s}ka} spat' sover\v{s}enno ne sobiralas'.\\
\hfill{Russian [N]}
%Долгожданный и неспокойный сон уже принял в свои объятия мистера Дарсли, а сидевшая на его заборе \underline{кошка} спать совершенно не собиралась.
\ex \label{ex:1c} Pan Dursley zapadł w niezbyt zresztą spokojny sen, ale \textit{kot} na murku 
nie okazywał najmniejszych oznak senności. \hfill{Polish [N]}
\ex \label{ex:1d} Gospodinot Darsli mo\v{z}ebi potona vo nemiren son, no \textit{ma\v{c}kata} na dzidot nadvor ne poka\v{z}uva\v{s}e ni tro\v{s}ka sonlivost.
%Господинот Дарсли можеби потона во немирен сон, но \underline{мачката} 
%на ѕидот надвор не покажуваше ни трошка сонливост.
\\ \hfill{Macedonian [N$+$the]}
\ex \label{ex:1e} Mr Dursley mochte in einen unruhigen Schlaf
hinübergeglitten sein,
doch \textit{die Katze} draußen auf der Mauer zeigte keine
Spur von
Müdigkeit. \hfill{German [the N]}
\ex \label{ex:1f} 
%Désīlǐ  xiānshēng mímíhúhú, běnlái kěnéng húluàn shuìshàng yíjiào, kě huāyuán qiángtóushàng \textit{nàzhīmāo} què méiyǒu sīháo shuìyì.\\
Désīlǐ  xiānshēng mímíhúhú, běnlái kěnéng húluàn shuì-shàng yí jiào, kě huāyuán qiángtóu shàng \textit{nà zhī māo} què méiyǒu sīháo shuìyì.\\
%德 思 礼 先生 迷迷糊糊, 本来 可能 胡乱 睡 上 一觉, 可 花园 墙头 
%上 \underline{那 只 猫} 却 没有 丝毫 睡意。\\
\hfill{Mandarin [demonstrative$+$classifier$+$N]}
\z\z

\noindent In the definite domain, we examine the forms that Russian, Polish and Macedonian use for the translation of the English \textit{the N}. In particular, we check whether and to what extent BSs that we expect to find in Russian and Polish, and singular definites that we expect to find in Macedonian, interact with demonstratives in the definite domain. We then contrast the results obtained for the three Slavic languages with Mandarin (as a control for Russian and Polish) and German (as a control for Macedonian).  

In the indefinite domain, we look to determine which forms are used for the translations of the English \textit{a N} in all three languages. We evaluate to which extent BSs are used in singular indefinite contexts in Russian vs. Polish vs. Macedonian and check for the interactions with the forms using the marker \textsc{one}. Once again, we compare the results obtained for Slavic languages with the results for Mandarin in the indefinite domain.

%In order to determine whether Russian and Polish bare nominals take on definite and indefinite readings, we compare Russian and Polish translations of  and  and compare the results with the forms used in the same contexts in Mandarin. We also check whether the distribution of BSs/BNs interacts with or demonstratives for \textit{the N} expressions. 

%To evaluate whether Macedonian bare nominals take on indefinite readings, we look into translations of \textit{a(n) N} and compare the use of Macedonian BSs to that of Russian, Polish and Mandarin BSs. To evaluate the use of BSs in definite contexts in Macedonian and to assess the status of Macedonian as a language with a definite article, we compare the German and Macedonian translations of \textit{the N} and the bi-directional mapping patterns between their respective definite articles.

\iffalse
This approach gives a broad empirical base that further allows to develop a fine-grained semantics of grammatical items in question determined from their distribution as compared to other items within each language, based on the notion of compositionality, and across languages, taking into account both semantic universals and variation. The patterns of of distribution of functional expressions are established in the same contexts across languages under study.  Thus, translation mining is crucially data-driven, that is, the distribution of forms leads towards the hypotheses about their meanings.

It is also important to notice that the translation mining parallel-corpus approach is replication-based, that is, findings from one corpus are used as input predictions for the study of the next.\footnote{The most complete application of this method --up till now-- is to the \textsc{have}-Perfect in Western European languages (\citealt{vanderKlis.et.al.2022}; \citealt{LeBruyn.deSwart2022}).} Replication shows the robustness of translation data and their usefulness for comparative research. 

Translation mining as a methodology relies on the assumption that meaning is by and large preserved in translations (\textit{the constancy of meaning hypothesis}), even though it is a creative process that may lead to some alternations. 

We use translations of English \textit{a N} expressions as a proxy set of singular indefinite contexts, and translations of \textit{the N}\textsubscript{sing} as a proxy set of singular definite contexts. In order to answer our research questions we evaluate the following empirical data: (i) the use of bare nominals in Russian and Polish in singular (in)definite contexts and the interaction of the distribution of bare nominals with that of the numeral \textsc{one} (in singular indefinite contexts) and \textsc{demonstratives} (in singular definite contexts). Russian and Polish data is also compared to the data obtained from Mandarin, our control case;
(ii) the distribution of bare NPs vs. numeral \textsc{one} in Macedonian in singular indefinite contexts by comparing it to the data from Russian, Polish and Mandarin; (iii) the distribution of \textit{def N} in Macedonian by comparing the German and Macedonian translations of \textit{the N}; (iv) the status of Macedonian as a language with a definite article by comparing bi-directional mapping patterns between definite articles in Macedonian and German.
\fi

%\subsection{Data}\label{data}

\section{Results} \label{results}

\subsection{Singular definite contexts}
 As far as definite contexts are concerned, there are no major surprises found in our data. The overall results are presented in Figure \ref{fig:2}, which reflects absolute frequencies and includes all translations in the target languages. The category \textit{Rest} contains all those translations that do not present any immediate interest for us (e.g., pronouns, possessives, etc.).  
 
 As we can observe, BSs are, indeed, the default option for rendering English \textit{the N} both in Russian and in Polish, as shown in Figure \ref{fig:2}. The differences in the occurrence of bare nominals in definite contexts are not significant for these two languages ($p = 0.37$, Fisher's Exact Test (FET)). Regarding Macedonian, the most prominent form in singular definite contexts is the one with a definite article, a result which is also fully in accordance with our initial expectations. In all three languages, there are practically no demonstratives used in singular definite contexts. 

 Comparing the results of Russian and Polish with their control language, Mandarin, we see that BNs in Mandarin are the most frequent form in the definite context as well.  However, we also observe an important difference in the relative distribution of BNs vs. NPs specified by demonstratives in the definite domain: in Mandarin, the tendency to resort to demonstratives is higher. The differences are significant for the comparison of  Mandarin and Polish ($p < 0.001$, FET),  and for Mandarin and Russian ($p = 0.016$, FET). 

As for Macedonian and its control language German, the two languages are quite uniform in the distribution of nominal forms in the definite domain. BSs and demonstrative NPs are either absent or clearly outnumbered in singular definite contexts in both Macedonian and German.\footnote{More specifically, there is only one BS found in German and three in Macedonian.} In \S \ref{McDef}, we will come back to the issue of definiteness marking in Macedonian and discuss some of the examples with BSs. 

\begin{figure}[H]
    \centering
    \includegraphics[scale=0.7]{figures/bor-Picture def.png}
    \caption{Russian, Polish and Mandarin BNs vs. demonstrative-N; \\
Macedonian and German the-N vs. demonstrative-N}
    \label{fig:2}
\end{figure}

\subsection{Singular indefinite contexts}
The parallel-corpus data showed that a bare noun is the default option for rendering singular indefinite nominals in both Russian and Polish (see Figure \ref{fig:1}). These two languages do not use the \textsc{one}$+$N construction in indefinite contexts in a statistically relevant way. The differences in distribution of bare nominals and nominals preceded by \textsc{one} are not significant for Russian and Polish ($p = 0.5$ FET). 

In Macedonian, however, while a BS is still the most frequent form in the indefinite domain, the English \textit{a N} construction is more often translated with the numeral \textsc{one} than in Russian or Polish. The differences are significant for the comparison of both Macedonian and Russian ($p < 0.001$, FET), and  Macedonian and Polish ($p < 0.001$, FET). 

As for the control language, Mandarin, where the numeral \textsc{one} precedes the nominal in a large number of cases in indefinite contexts, it shows a sharp contrast with Russian and Polish, which hardly ever use this structure. Moreover, Mandarin also shows contrast with Macedonian, where the use of \textsc{one} is not as frequent. The differences are significant for Mandarin and Russian ($p < 0.001$, FET), and Mandarin and Polish ($p < 0.001$, FET), as well as for Macedonian and Mandarin ($p < 0.001$, FET).

\begin{figure}[H]
    \centering
    \includegraphics[scale=0.7]{figures/bor-Picture indef.png}
    \caption{Russian, Polish, Macedonian and Mandarin bare nominals vs. \textsc{one}$+$N}
    \label{fig:1}
\end{figure}

\subsection{Recap}
Summing up the results of our parallel corpus study, it can be said that Russian and Polish freely use bare nouns in both singular indefinite and singular definite contexts, in accordance with the Slavic descriptive literature. They are, however, in sharp contrast with Mandarin, where the numeral \textsc{one} seems to be the default option in the indefinite domain and the demonstrative is competing with bare NPs in the definite domain.

 As for Macedonian, in the indefinite domain it seems to occupy an intermediate position between Russian and Polish, on the one hand, and Mandarin on the other: The \textsc{one}$+$N construction appears in the translations of \textit{a N} quite frequently, but not as often as in Mandarin. In the definite domain, Macedonian uses NPs with a definite marker in \textit{the N{\textsubscript{sing}}} contexts as often as German.

\section{Discussion} \label{discussion}

We structure our main discussion points in the same way we presented the results of the study, that is, according to the distribution of various forms in a specific context. We begin by evaluating the parallel corpus results obtained for the definite domain. In the discussion of the indefinite domain, we reflect not only on the distribution of BSs, but also on the role of \textsc{one}$+$N construction in the grammar of all three target languages. 

\subsection {Definite contexts in Russian, Polish and Macedonian} \label{Def discussion}

Both traditional descriptions and formal semantic analyses seem to be in full agreement on attributing a possible definite reading to BSs in languages without articles. Our data cast no doubt on this claim for Russian or Polish: BSs prevail in definite contexts in both languages and the distributional behaviour of BSs is therefore in full accordance with their standard semantic descriptions and/or analyses. In Macedonian, the prevailing form is the definite singular, i.e., our data also confirm the status of Macedonian as a language with a definite article. Even though the interpretation of our main results seems to be rather straightforward, there are two points of interest that we would like to discuss. 

The first observation concerns the distribution of BSs vs. NPs specified by demonstratives in the definite domain. In the previous section, we pointed out that demonstratives do not seem to occupy a prominent place in either of the three Slavic languages in \textit{the N{\textsubscript{sing}}} contexts. Russian and Polish as languages without articles can be contrasted to Mandarin in this respect, one of the two control languages used in this study, where the higher rate of demonstratives in the definite domain suggests that the demonstrative plays a much more significant role in definiteness marking in Mandarin. In fact, \citet{Liuetal2022} hypothesise that Mandarin is developing a definite article (and an indefinite one), but in this paper, we limit ourselves to empirical statements with respect to Mandarin.
%mere observations about the data.
%\footnote{The reader is referred to  \citet{Liuetal2022} for specific details of the proposal.} 
In Macedonian, there were only three contexts where a demonstrative was used, all corresponding to anaphoric uses of \textit{the N} in the source text. 

The second point that we would like to discuss is the status of the definite article in Macedonian, which will be examined in the next subsections. 

%\subsubsection{BSs and demonstratives} \label{demonstratives}

\subsubsection{The Macedonian definite article in the singular domain} \label{McDef}

The status of the Macedonian definite article has been subject to some debate in the semantic literature, as pointed out in \S \ref{intro}. No consensus emerges from the literature concerning the semantic contribution of this marker. One of the features that it exhibits (and that distinguishes it from a typical definite article) is that it morphologically marks a proximal--neutral--distal distinction, just like demonstratives in many languages do \citep{Lyons1999}. It should be noted that the question about the status of a definite marker in any language is essentially semantic and cannot be definitively resolved without looking into the meaning of this expression, but the distribution of any definite marker/article also plays a significant role in a potential analysis and this is what our study can inform about.   

We looked into the properties of the definite article in Macedonian by running a comparative study of the distribution of the definite article in Macedonian and German. In particular, we measured their co-occurrence in the same contexts by calculating Normalized Pointwise Mutual Information \citep[NMPI,][]{Bouma2009}, which is a bidirectional measure for parallel data \citep{LeBruyn.et.al2022a}. The result shows that the NPMI of the two articles reaches 0.48 (with a maximum of 1). That means that the likelihood of the articles in the two languages occurring in the same contexts is higher than chance but not at ceiling. In other words, the bi-directional mapping pattern suggests that the distribution of definite articles in German and Macedonian across the definite contexts is not completely identical. 

%\begin{table}
%    \centering
%    \begin{tabular}{|c|c|c|c|c|}
%    \hline
%    \multicolumn{2}{|c|}{}     &  \multicolumn{2}{c|}{Macedonian} &\\
%    \cline{3-4}
%     \multicolumn{2}{|c|}{} & definite article N & rest & \\
%       \hline
%    \multirow{2}{*}{German} & definite article N & 88 & 24 & 112\\
%    \cline{2-5}
%    & rest & 20 & 74 & 94\\
%    \hline
%    \multicolumn{2}{|c|}{} & 108 & 98 & 206 \\
%    \hline        
%    \end{tabular}
%    \caption{Bi-directional mapping patterns between the German and Macedonian (singular) definite article}
%    \label{tab:1}
%\end{table}

\begin{table}
    \centering
    \begin{tabular}{ccccc}
    \lsptoprule
    \multicolumn{2}{c}{}     &  \multicolumn{2}{c}{Macedonian} &\\
     \multicolumn{2}{c}{} & definite article N & rest & \\
     \midrule
    \multirow{2}{*}{German} & definite article N & 88 & 24 & 112\\
    & rest & 20 & 74 & 94\\
    \multicolumn{2}{c}{} & 108 & 98 & 206 \\
    \lspbottomrule        
    \end{tabular}
    \caption{Bi-directional mapping patterns between the German and Macedonian (singular) definite article}
    \label{tab:1}
\end{table}

Looking into the contexts where Macedonian and German did not coincide in the use of the definite article, we find some interesting examples of BSs. For instance, in (\ref{ex:6}), Macedonian uses a BS, while German opts for a definite article, just like the English source:

\ea \label{ex:6} \ea At half past eight, Mr Dursley picked up his briefcase, pecked Mrs Dursley on \textit{the cheek} and tried to kiss Dudley goodbye but missed...\\ \hfill English (source): [the N]
\ex Vo osum i pol gospodinot Darsli ja zede svojata akten\v{c}anta, ja kolvna gospoǵ\'{a}ta 
Darsli vo \textit{obraz} i se obide da go bakne Dadli za razdelba, no ne uspea... \hfill Macedonian: [N]
\ex Um halb neun griff Mr Dursley nach der Aktentasche, gab seiner Frau einen Schmatz auf \textit{die Wange} und versuchte es auch bei Dudley mit einem Abschiedskuss... \hfill German: [the N]
\z \z

\noindent Another example of the same type of article mismatch is presented in (\ref{ex:7}): 

\ea \label{ex:7} \ea A man appeared on the corner the cat had been watching, appeared so suddenly and silently you'd have thought he'd just popped out of \textit{the ground}. \hfill English: [the N]
\ex Na agolot \v{s}to go nabljuduva\v{s}e ma\v{c}kata se pojavi \v{c}ovek, tolku nenadejno i tivko, kako da izniknal od \textit{zemja}. \hfill Macedonian [N]
\ex An der Ecke, die sie beobachtet hatte, erschien ein Mann, so j\"{a}h und lautlos, als wäre er geradewegs aus \textit{dem Boden} gewachsen. \\
\hfill German: [the N]
\z \z

\noindent Although we cannot reach any firm conclusions on the basis of only few examples, we can hypothesise that they both present cases of weak definites (possessive weak definites, \citealt{Barker05}, in the case of (\ref{ex:6})), so that the contexts where the uniqueness of a definite description is questioned are potentially very good candidates for the absence of the definite article in Macedonian. Needless to say, additional empirical investigation is needed to check this hypothesis. 

%\noindent In (\ref{ex:7}), Macedonian uses a definite article, while German and English use an indefinite one (which is modified and has a specific reading though).

%\ea \label{ex:7} \ea Mr Dursley was the director of \textit{a firm} called Grunnings, which made drills. \hfill English: [a N]
%\ex Mr Dursley war Direktor \textit{einer Firma} namens Grunnings, die Bohrmaschinen herstellte. \hfill German: [a N]
%\ex Gospodinot Darsli beše direktor na \textit{firmata} ``Graningz", koja praveše dupčalki. \hfill Macedonian [N$+$the]
%\z \z

\subsubsection{Some remarks on the plural definite domain} \label{McPl}

Even though we did not run statistical analyses for the plural domain and there were not too many data points in our dataset, we would like to draw attention to some observations concerning the use of the definite article with plurals in Macedonian that appear important. For instance, Macedonian seems to use definite articles in plural generic contexts, while English resorts to bare plurals and German presents variation.

\ea \label{ex:4} \ea \label{ex:4a} \textit{Cats} couldn't read maps or signs. \hfill English (source): [Ns] 
\ex \label{ex:4b} \textit {Mačkite} ne možat da čitaat ni mapi ni oznaki. \hfill Macedonian: [Ns$+$the] 
\ex \label{ex:4c} \textit{Katzen} konnten weder Karten noch Schilder lesen. \hfill German: [Ns] \z\z

\noindent The use of the definite article with generic plurals as illustrated in (\ref{ex:4}) may suggest that Macedonian -- at least in some aspects -- is rather comparable to Romance languages in its use of definite plurals than to Germanic languages. 

Existential contexts in the plural definite domain require further scrutiny. We detected several examples in our dataset where both Macedonian and German use a definite article whereas English uses a bare plural in the same context. One of those examples is (\ref{ex:5}). 

\ea \label{ex:5} \ea \label{ex:5a} And finally, \textit{bird-watchers} everywhere have reported that the nation's owls have been behaving very unusually today. \hfill English: [Ns]\\
\ex \label{ex:5b}
I kone\v{c}no \textit{nabljuduva\v{c}ite} na ptici od site strani javija deka buvo - vite vo na\v{s}ata zemja deneska se ondesuvale mnogu neobi\v{c}no. \\
\hfill Macedonian: [Ns $+$ the]
\ex \label{ex:5c}
Und hier noch eine Meldung. Wie \textit{die Vogelkundler} im ganzen Land berichten, haben sich unsere Eulen heute sehr ungewöhnlich verhalten.
%\textcolor{red}{FILL IN} 
\hfill German: [the Ns] \\

\z\z

\noindent In this particular case, the presence of the article in Macedonian could be due to a specific syntactic construction used in the example (prepositional phrase \textit{watchers of birds} instead of the nominal compound \textit{bird-watchers} in the source text). This, however, would not explain the presence of the definite article in German. It might also be the case of a so-called \textsc{functional} reading of BPs in English, discussed at length for English by \citet{Condoravdi1994}. The availability of this reading for BPs is language-specific, so we conclude that our data demonstrate some cross-linguistic variation worth a more systematic investigation. It is not surprising to see this variation in the distributional patterns, as cross-linguistic differences in the use of the definite article are very well documented and widely discussed in the literature. The corpus data of the current study is not sufficient to arrive at any firm conclusions, but it may be reasonably suggested that German and Macedonian, just like many other languages with grammatical marking of definiteness, do not fully coincide in definiteness marking patterns: the overlap in the use of the definite article is only partial, not absolute. 


%that languages with definite articles (like German and Macedonian) might show certain differences in the use of definite nominals, just like languages without articles do in the use of bare nominals.

%The data from German and Macedonian may also indicate that the semantics of the definite article is not as strongly associated with the uniqueness presupposition across languages as the literature on definiteness usually suggests (but see \citeposst{Simik.Demian2020} conclusions for German; \citet{Coppock.Beaver2015} for an alternative view on definiteness.) 
%The data from our corpus study showed that both Macedonian and German (a control language) used definite NPs in definite contexts, but the overlap in the use of the definite article between these two languages is only partial. The fact that definitely marked NPs in Macedonian are found only in a subset of contexts in which German uses definite articles may suggest caution for the the analysis of the Macedonian definite marker as an article.

\subsection {Indefinite contexts in Russian, Polish and Macedonian} \label{Indef discussion}

Indefinite contexts constitute the most interesting case in our study, as they convincingly illustrate several theoretically relevant points. First, there is variation both within and outside the Slavic family in the distribution of BSs in the indefinite domain, which has direct repercussions for existing theoretical analyses of BSs. Second, intricate interactions of \textsc{one}$+$N with BSs in the indefinite domain can elucidate the grammatical status of \textsc{one} in a given language. Third, our data pose some very specific constraints and requirements for an accurate and empirically adequate theoretical analysis of BSs in languages without articles. We discuss each of these points in the three subsections that follow. 

\subsubsection{BSs in Russian, Polish and Macedonian}

One of the main results of our study concerns the distributional pattern of BSs in Russian and Polish. In particular, the data from the parallel corpus show that in Russian and Polish BSs freely appear in indefinite singular contexts as counterparts of \textit{a N} in the source text. One rather typical example of an indefinite in an existential context is given below: 

\ea \ea \label{ex:cat} English (source): [a N]			 \\
There was \textit{a tabby cat} standing on the corner of Privet Drive, but there wasn't a map in sight.
\ex  Russian: [N]\\
Na uglu Praivet Draiv dejstvitel'no sidela \textit{polosataja} \textit{ko\v{s}ka}, no nikakoj karty vidno ne bylo.\\
%on corner P. D. indeed sat tabby cat but no map seen not was\\
\ex Polish: [N] \\
 Na rogu Privet Drive rzeczywi\'{s}cie stał \textit{bury} \textit{kot}, ale nie studiował żadnej mapy. \\
%on corner P. D. indeed stood tabby cat but not studied no map\\
\z\z

\noindent Our Russian and Polish data directly support traditional descriptive approaches to BSs in Slavic languages without articles and those formal approaches which do not rule out an indefinite interpretation for BSs, e.g., \citet{Chierchia1998, Krifka2003}. The results of our study are also compatible with the proposal that bare NPs in Russian are essentially indefinite and a definite reading is achieved through pragmatic strengthening \citep{Seres.Borik2021}.

On the other hand, our empirical findings are in conflict with \citeposst{Dayal2004} proposal, whose prediction -- as we mentioned in \S \ref{intro} -- is that BSs should never give rise to indefinite readings in regular argument position in languages without articles. Dayal examines the behavior of BSs in Hindi, Russian and Mandarin, and argues that  an overt indefiniteness marker has to appear in those contexts where an indefinite reading has to be expressed. This prediction holds for Hindi, where \textsc{one} functions as such a marker,\footnote{This result has been confirmed by a parallel corpus study reported in \citet{Liuetal2022}.} but it is very clear that Russian and Polish behave differently. In fact, in our data \textsc{one} is only used twice in Russian in the indefinite domain, whereas the Polish data do not contain a single occurrence of this item. Thus, our data allow us to conclude that both Russian and Polish are truly articleless languages where BSs dominate in both definite and indefinite contexts. No competing forms emerge in our study in either of the two contexts in either of the two languages. 

In contrast to Russian and Polish, Macedonian uses both BSs and \textsc{one}$+$N constructions. Our data show that Macedonian differs from truly articleless languages, and the construction \textsc{one}$+$N competes with BSs in the indefinite domain in Macedonian. This difference can be illustrated with the translation of example (\ref{ex:cat}) above into Macedonian: where Russian and Polish use a BS, Macedonian uses \textsc{one}$+$N. 

\ea
 Na agolot na \v{S}im\v{s}irovata uli\v{c}a stoe\v{s}e \textit{edna} \textit{neobi\v{c}na} \textit{\v{s}arena} \textit{ma\v{c}ka}, no nikade nema\v{s}e mapa. \\
%on corner.the on S.the street stood one unusual stripped cat but nowhere was.not map\\
\z

%На аголот на Шимшировата улица стоеше една необична шарена мачка, но никаде немаше мапа.
\noindent If we look outside the Slavic family, our control language, Mandarin, shows a strong tendency for the \textsc{one}$+$N construction to appear in singular indefinite contexts (see Figure \ref{fig:1}). Macedonian clearly occupies an intermediate position between Mandarin (relatively low percentage of BSs) and Russian/Polish (predominantly BSs) with respect to the use of BSs in the singular indefinite domain. 

Note that this kind of variation in the use of BSs comes out unexpected on most analyses. In general, articleless languages are perceived as a homogeneous group that either do or do not use BSs in a certain domain, but the kind of variation that we see in our data is rather challenging for theoretical approaches. We will come back to this point at the end of this section, but first we will take a better look at the closest competitor of a BS in the indefinite domain, the indefinite marker \textsc{one}.

\subsubsection{The status of \textsc{one} in the indefinite domain}

It is well known that the numeral \textsc{one} is a predecessor of the indefinite article in many languages (\cite{Heine1997}, \cite{Gelderen2011}, among many others). Looking once again at the distribution of nominal forms in the indefinite domain in Figure \ref{fig:1}, we observe a clear interaction between BSs and \textsc{one}$+$N constructions: the frequency of \textsc{one}$+$N in our data goes from being at floor in Russian and Polish to a significant percentage in Macedonian and to predominance in Mandarin. This raises a question about the grammatical status of the marker \textsc{one} in different languages. 

The differences in the use of the \textsc{one}$+$N construction across languages may be accounted for by different stages of its grammaticalisation as an article. Typically, the stages of grammaticalisation of the indefinite article are defined in the following order: 1. the numeral, 2. the presentative marker, 3. the specificity marker, 4. the non-specific marker, 5. the generalised article (\citealt{Givon1981}, \citealt{Heine1997}, among others).\footnote{These stages are coarsely defined and may have substages.} Even though defining the exact stage of grammaticalisation of \textsc{one} in the languages under study is out of the scope of this paper, our data offer several discussion points relevant for the issue.   

Our empirical findings for Russian and Polish, where BSs overwhelmingly dominate in the indefinite domain, seem to be in conflict with the proposal of \citet{Hwaszcz.Kedzierska2018}, who claim that in Russian \textsc{one} is grammaticalised as a presentative marker, that is, it marks a newly introduced referent, which is intended to be used in the subsequent discourse and is usually specific and topical. The authors also claim that in Polish \textsc{one} is grammaticalised even further, being used as a specific and sometimes as a non-specific marker. Neither of the two claims is confirmed by our data, as some representative examples can illustrate: 

\ea \label{ex:Dudley} \ea  English (source): [a N]			 \\
The Dursleys had \textit{a small son} called Dudley and in their opinion there was no finer boy anywhere.
\ex  Russian: [N]\\ \label{dudleyru}
U mistera i missis Darsli byl \textit{malen'kij syn} po imeni Dadli, i, po ix mneniju, \.{e}to byl samyj \v{c}udesnyj rebenok na svete. 
\ex Polish: [N] \\ \label{dudleypl}
\textit{Syn} Dursley\'{o}w miał na imię Dudley, a rodzice uważali go za najwspanialszego chłopca na świecie. 
\z\z

\ea \label{ex:son} \ea  English (source): [a N] \\
He was sure there were lots of people called Potter who had \textit{a son} called Harry.
\ex  Russian: [N]\\ \label{sonru}
Mister Darsli legko ubedil sebja v tom, \v{c}to v Anglii \v{z}ivet mno\v{z}estvo semej, nosja\v{s}\v{c}ix familiju Potter i imeju\v{s}\v{c}ix \textit{syna} po imeni Garri.
\ex Polish: [N] \\ \label{sonpl}
Mnóstwo ludzi może się nazywać Potter i mieć \textit{syna} Harry'ego.
\z\z

\noindent Example (\ref{ex:Dudley}) is a typical context where a new specific referent is introduced by a modified indefinite in the source text, which is then rendered by a BS both in Russian and in Polish, just like the non-specific indefinite \textit{a son} in (\ref{ex:son}). At least in Russian, \textsc{one}$+$N cannot be used instead of N in (8) and (9), unless \textsc{one} is interpreted as a numeral.\footnote{We thank an anonymous reviewer for stressing this point.} Our data show no sign of any significant difference between Russian and Polish with respect to the grammatical status of \textsc{one}: this marker does not show up regularly or systematically in either a presentative, specific or any other type of context.

Macedonian \textsc{one}, on the other hand, is more frequent. We have not conducted any specific study of the contexts where \textsc{one} appears in Macedonian, as our dataset is too small to yield sensible results, but we can provide some indicative examples here that can help us map out a path for future research. For instance, Macedonian uses a BS in translations of both example (\ref{ex:Dudley}) and (\ref{ex:son}) above, but there are other specific and non-specific contexts where \textsc{one}$+$N construction appears: 

\ea \ea \label{sensible} English (source): [a N] \hfill (non-specific)\\
My dear Professor, surely \textit{a sensible person} like yourself can call him by his name?
\ex \label{sensiblemk} Macedonian [one N]\\
Draga moja profesorke, ne misli\v{s} li deka \textit{edna tolku razumna li\v{c}nost'} kako \v{s}to si ti slobodno mo\v{z}e da go narekuva po ime? 

%\ex For comparison: \\
	%German: [a N]\\
	%Mandarin: [numeral-yi \lq one' $+$ classifier $+$ N]\\
	%Russian, Polish: [N]
 \z\z

\ea \ea \label{handker} English (source): [a N] \hfill (specific)\\
Professor McGonagall pulled out \textit{a lace handkerchief} and dabbed at her eyes beneath her spectacles.
\ex \label{hankermk} Macedonian [one N]\\
Profesorkata Mekgonagl izvadi \textit{edno tanteleno maramče} i gi protri očite pod očilata.
%\ex For comparison: \\
	%German: [a N]\\
	%Mandarin: [numeral-yi \lq one' $+$ classifier $+$ N]\\
	%Russian, Polish: [N]
 \z\z

\noindent The mixed data across specific and non-specific contexts indicate that the \textsc{one}$+$N construction is not really established in these types of contexts. The data obtained in our study are, in principle, in line with \citet{Hwaszcz.Kedzierska2018}, who claim that \textsc{one} in Macedonian is used with both specific and non-specific indefinite NPs. Our Macedonian data show that both specific and non-specific indefinite NPs may also appear as bare, as illustrated in the above examples, which may indicate a certain degree of optionality in the use of \textsc{one} for marking specific and non-specific nominals.\footnote{One of the limitations of corpus studies is that it is impossible to determine the optionality of an element. In order to research the (non-)obligatoriness of \textsc{one} in certain linguistic environments, linguistic experiments with native speakers need to be carried out.} This flexibility (possibly translated as optionality) provides a contrast with English and German, languages where an indefinite article is obligatorily used in all the examples discussed in this subsection. Thus, Macedonian does differ from languages with established indefinite articles, and we therefore conclude this discussion by saying that the status of \textsc{one} cannot be unequivocally defined as an indefinite article in Macedonian, contra, e.g., \citet{Tomić2006}.\footnote{In this respect, the Macedonian data resemble the situation in Bulgarian, as reported in \citet{Geist2013}.} Rather, \textsc{one} is an indefinite marker that might evolve into an article, but further research is needed to substantiate this claim. 


\iffalse
 \ea \ea \label{ex:2a} English (source): [a N]			\hfill		(non-specific/generalized article) \\
It seemed that Professor McGonagall had reached the point she was most anxious to discuss, the real reason she had been waiting on a cold hard wall all day, for neither as \textit{a cat} nor as a woman had she fixed Dumbledore with such a piercing stare as she did now.
\ex \label{ex:2b} Macedonian: [N]\\
...ni kako \textit{mačka} ni kako žena, ne go beše gledala Dambldor tolku ostro kako sega.
\ex For comparison: \\
Mandarin: [numeral-yi \lq one' $+$ classifier $+$ N]\\
	Russian, Polish: [N]\\
	(curiously, German: [N])\\
\z\z

\ea \ea \label{ex:3a} English (source):  [a  N]			\hfill				(non-specific)\\ He laid Harry gently on the doorstep, took \textit{a letter} out of his cloak, tucked it inside Harry’s blankets and then came back to the other two.
\ex \label{ex:3b} Macedonian: [N]\\
Vnimatelno go spušti Hari na pragot. Od nametkata izvadi \textit{pismo}, go vovleče pod kebeto, a potoa se vrati kaj drugite dvajca.\\
\ex For comparison: \\
	German: [a N]\\
	Mandarin: [numeral-yi \lq one' $+$ classifier $+$ N]\\
	Russian, Polish: [N]
\z\z
\fi

\subsubsection{Theoretical implications}

As the discussion in the previous sections indicates, the main challenge that our data pose for theoretical approaches striving for empirical adequacy is the problem of language variation. The variation in the definite domain, especially in the distribution of the definite article across languages, is relatively well known and discussed in the semantic literature \citep[e.g.,][]{Dryer2005}. Our analysis of the definite article in Macedonian vs. German adds one more study case to this discussion. 

In the indefinite domain, however, variation in the distribution of BSs in languages without articles (or without an indefinite article) is less expected. For instance, the approach to BNs in general and BSs in particular developed in \citet{Dayal2004}, \citet{Dayal2018}, and \citet{DayalSag2020} is based on the claim that BSs do not allow for indefinite readings in articleless languages. The formal machinery of this approach does not leave much room for variation: the denotation of a noun in regular argument positions is derived by type-shifting operators and, crucially, Dayal's analysis cuts off the possibility of an existential type-shift for BSs. The logic behind this move, we believe, applies universally. Our data for Russian and Polish, though, strongly suggest that there should be an easy way to allow for a BS to appear in the singular indefinite domain, which may be achieved via standard type-shifting operations, like an existential type shift. However, allowing for this type shift to be subject to parametric variation will considerably weaken Dayal's formal theory, at least in the absence of any independent principle underlying such variation. 

The Macedonian data, where we see a competition between BSs and the \textsc{one}$+$N construction, suggest that there should be a way to allow for BSs in those contexts where the other construction does not appear on a regular basis. In other words, there should be an account of an interaction between nominal forms that coexist in the indefinite domain. Dayal's approach cannot easily accommodate such interaction either, because \textsc{one}$+$N is predicted to be the only option in the indefinite domain in the absence of an indefinite article.  Thus, we conclude that the semantic theory of bare nominals advocated in \citet{Dayal2004}, \citet{Dayal2018}, and \citet{DayalSag2020} has considerable difficulties accounting for an overall empirical picture that emerges from our data.\footnote{See also \citet{Liuetal2022}.} 

Mandarin, our control language, clearly prefers the \textsc{one}$+$N construction to BNs in the indefinite domain. As \citet{Liuetal2022} argue, this fact does not really follow from Dayal's analysis either, since in Mandarin, which lacks grammatical number, BNs are expected to easily get an indefinite reading, just like BPs in other languages do. If Mandarin BNs behave like BPs rather than BSs, they are predicted to get a narrow scope indefinite reading and hence, they should be visibly prominent in indefinite (singular and plural) contexts. In our data, however, the \textsc{one}$+$N construction wins over BNs in the singular indefinite domain. In fact, it looks like what \citet{Dayal2004, Dayal2018} predicts for Mandarin occurs in Russian and Polish, with a proviso for number marking, and what her analysis predicts for Russian and Polish seems to hold for Mandarin.

An analysis that our data calls for should allow for a formal way to derive an existential interpretation of a BS via type-shifting, but only if there is no competing form with an overt marker that would block this shift. \citeposst{Chierchia1998} or \citeposst{Krifka2003} classical analyses, for instance, state that while in some languages type shifts are indicated by overt determiners, in languages that lack them, type shifts apply covertly whenever the linguistic context requires it. Covert type-shifting is restricted by the Blocking Principle, which roughly states that if a language has an overt means to express a type shift, then it must be used. 
This analysis seems to be much better equipped to handle our data. For instance, we have seen no evidence  that \textsc{one}$+$N  in Russian (\textit{odin N}) and Polish (\textit{jeden N}) function as an article-like expression. 
%that is,  \textsc{one} does not seem to be grammaticalised at all, neither in Russian nor in Polish. 
Thus, the covert application of the existential type-shift is not blocked, which allows for BSs to be freely used in indefinite contexts. For Macedonian, a language with an emerging indefinite marker \textsc{one}, the existential type shift would be blocked for a BN only in those contexts where \textit{eden} appears. Our cross-linguistic data provide a serious argument in favour of a classical blocking semantic analysis of bare nominals, in which fine-grained variation in the distribution of bare nominals follows from the broader/narrower use of article-like expressions.


%\footnote{\citet[360]{Chierchia1998} formulates the Blocking Principle (\lq Type-shifting as Last Resort') as follows:
%\ea For any type-shifting operation $\tau$ and any
%$X$: \\
%$∗\tau(X)$\\ if there is a determiner \textit{D} such that for any set \textit{X} in its domain,\\ $D(X) = \tau(X)$.\z } 

%(which has the following forms in Macedonian \textit{eden} \textsc{masc.sg}, \textit{edna} \textsc{fem.sg}, \textit{edno} \textsc{neut.sg}, \textit{edni} \textsc{pl})

\section{Conclusions} \label{conclusions}

In this paper, we have reported the results of a parallel translation corpus study on the distribution of BSs in three Slavic languages, Russian, Polish and Macedonian. We built our corpus on the text of the first chapter of 
\textit{Harry Potter and the Philosopher's Stone} and complemented the results obtained for Slavic languages with the results for Mandarin as a control language for Russian and Polish, and German as a control language for Macedonian. 

In view of the empirical data presented here, it can be concluded that Russian and Polish are truly articleless languages and freely allow their BSs to take on definite and indefinite readings across domains. In Macedonian, BSs are restricted to the indefinite domain where they compete with the indefinite marker \textsc{one}, whereas in the definite domain, Macedonian uses the definite article, just as expected. Therefore, we conclude that Macedonian is a language with a definite article and with an emerging indefinite marker whose exact grammatical status requires further empirical investigation. 

Slavic languages present challenging theoretically relevant contrasts with their control languages. In case of Macedonian, we have stressed the need to further scrutinize the conditions and the contexts where the definite article is used because we have shown that the overlap between the definite articles in Macedonian and German is partial. We also see the need to extend the investigation to the plural domain to get a full picture of the distribution of the definite article in Macedonian. As for Russian and Polish, they present a striking contrast with Mandarin in the indefinite singular domain, where the two Slavic languages show a clear preference for BSs and Mandarin opts for the \textsc{one}$+$N construction as a counterpart of the English \textit{a} \textit{N}. Macedonian occupies an intermediate position: \textsc{one}$+$N is used rather frequently in Macedonian, but not as often as in Mandarin singular indefinite contexts. 

We have argued that these contrasts call for a theoretical approach where the observed variation in the distribution of BSs and competing forms can be naturally accounted for. We suggest that the Blocking Principle as formulated in \citet{Chierchia1998} can serve as a foundation for such an approach. 





\iffalse

different languages in parallel
The results of our parallel corpus study crucially illustrate that there is variation between languages without articles as far as the distributional pattern of BSs in the indefinite domain is concerned. 
We conclude:

(ii)Macedonian has a definite marker which can be considered an article (however, different from German in some uses). It also has an emerging indefinite marker whose status requires further scrutiny, that is, a bigger set of data is needed to determine the status of the indefinite marker \textsc{one} in Macedonian, and also experimental work with native speakers of this language to determine the degree of its obligatoriness/optionality in certain contexts.

(iii) that our cross-linguistic data argue against a uniform definiteness semantics of singular BNs (Dayal 2004) and can best be accounted for with a classical blocking analysis in which fine-grained variation in the distribution of BNs follows from the broader/narrower use of article-like expressions.

BSs in Russian and Polish are not restricted,  
Our data does not prove or disprove its status of an article but suggests that the context where it appears should be further investigated. 

This might have various theoretical repercussions and consequences for cross-linguistic semantics of the definite article 
more extensive empirical studies are called for. 
In the present corpus study we have observed that the numeral \textsc{one} as an indefinite marker is firmly established in indefinite domain in Macedonian. 

The data from singular and plural definite domains in Macedonian and German give us clues for further investigation into variation among languages with definiteness marking.
\fi

\section*{Abbreviations}

\begin{tabularx}{.5\textwidth}{@{}lQ}

\textsc{f}&feminine\\
\textsc{m}&masculine\\
\textsc{n}&neuter\\
\end{tabularx}%
\begin{tabularx}{.5\textwidth}{lQ@{}}
\textsc{pl}&plural\\
\textsc{sg}&singular\\
&\\ % this dummy row achieves correct vertical alignment of both tables
\end{tabularx}



\section*{Acknowledgments}
The first and the last authors acknowledge financial
support by Spanish MINECO
(PID2020–112801GB–100). The last author also acknowledges support by Gene\-ra\-litat de Catalunya (2021-SGR-00787). The second author acknowledges
support by the Dutch Research Council (NWO) (\#360-80-070). The third author acknow\-ledges support under her CSC grant. At the time of writing the article, the last author was a Margarita Salas fellow, funded by the European Union -- Next Generation EU. 

\printbibliography[heading=subbibliography,notkeyword=this]

\end{document}
