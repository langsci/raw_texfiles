\documentclass[output=paper,colorlinks,citecolor=brown]{langscibook}
\ChapterDOI{10.5281/zenodo.15394201}
%\bibliography{localbibliography}

\author{Marko Simonović\orcid{0000-0002-9651-6399}\affiliation{University of Graz} and 
 Petra Mišmaš\orcid{0000-0001-8659-875X}\affiliation{University of Nova Gorica} and 
 Stefan Milosavljević\orcid{0000-0003-2305-2519}\affiliation{University of Graz} and 
 Boban Arsenijević\orcid{0000-0002-1124-6319}\affiliation{University of Graz} and 
Katarina Gomboc Čeh\orcid{}\affiliation{University of Nova Gorica} and 
Franc Lanko Marušič\orcid{0000-0002-0667-3236}\affiliation{University of Nova Gorica} and
Rok Žaucer\orcid{0000-0001-7771-6937}\affiliation{University of Nova Gorica}}

\lehead{Simonović et al.}

\title[ ``\textsc{l}-participle'' nominalisations]{How departicipial are ``\textsc{l}-participle'' nominalisations in Western South Slavic}


\abstract{We focus on nominalisations seemingly derived from \textsc{l}-participles, illustrated by \textit{lec}-nominalisations in Slovenian, in order to establish the nature and position of the \textsc{l}-morpheme as well as the structure of these nominalisations in general. Our research is situated in the current debates on whether the item \textsc{l} in \textsc{l}-participles and \textsc{l}-nominalisations is the same morpheme or two different morphemes, and if the former, whether \textsc{l}-nominalisations are derived from \textsc{l}-participles. We argue that the \textsc{l}-morpheme is a root in both, but also show that it is not the case that \textit{lec-}nomina\-li\-sa\-tions contain \textsc{l}-participles. The \textit{lec-}nominalisations are argued to contain a smaller structure than the corresponding \textsc{l}-participle, which is also reflected in the set of theme vowels possible in these nominalisations.

\keywords{deverbal nominalisations, departicipial nominalisations, \textsc{l}-participles, agentive nominalisations, Western South Slavic, Slovenian}
} 

%\setlength{\footheight}{42.95pt}

\begin{document}
\maketitle


\section{Introduction}\label{sim+:intro}


One productive strategy to derive deverbal agentive nominalisations in Slovenian (Slo) is with the item \textit{-lec}, which shares its first segment with the \textsc{l}-participle (termed the past participle in much of the traditional literature). This is illustrated in \REF{ex:brati}. 


\ea \label{ex:brati}
\glll br-a-ti -- br-a-l-a -- br-a-l-ec %; obraz-ec  
\\
read-\textsc{tv}-\textsc{inf} {} read-\textsc{tv}-\textsc{l.ptcp}-\textsc{f.sg} {} read-\textsc{tv}-\textsc{l}-\textit{er}  \\
{`to read'} {} {`(she) read'} {} `reader'   \\\hfill (Slo)
\z

\noindent This is not an isolated example of such nominalisations in Western South Slavic. The same type can be found in Bosnian/\-Croatian/\-Montenegrin/Serbian [BCMS], where the item -\textit{lac} has the same structure,  \REF{ex:citati}.


\ea \label{ex:citati}
\glll čit-a-ti -- čit-a-l-a -- čit-a-l-ac \\
        read-\textsc{tv}-\textsc{inf} {} read-\textsc{tv}-\textsc{l.ptcp}-\textsc{f.sg} {} read-\textsc{tv}-\textsc{l}-\textit{er}    \\
 {`to read'} {} {`(she) read'} {} `reader'  \\\hfill (BCMS)
 \z

\noindent In both languages, the string \textit{-ec/-ac} (glossed as \textit{-er} in the examples above)  is also an attested nominal suffix, as illustrated in, e.g., \textit{Slovenec/Slovenac} `a Slovenian'. This fact prompts several authors to analyse the string \textit{-lec} as consisting of two morphological units, \textsc{l} and \textit{-ec/-ac} (see \S \ref{sec:severing} for references or \citealt{birtic2008} for an overview). And while the function/contribution of \textit{-ec/-ac} seems to be unproblematic,  the question whether \textit{lec/lac}-nominalisations (and other comparable derivations) contain the \textsc{l}-par\-ti\-cip\-le has been posed and answered differently both in traditional descriptive work and in formal approaches.\footnote{In some contexts in Slovenian, \textit{-lec} is written and pronounced as \textit{-vec}, specifically, after some roots ending in a vowel (e.g., \textit{pi-$\emptyset$-ti, pi-vec} `to drink, drinker') or in -\textit{l} or \textit{-lj}, e.g., \textit{del-a-ti, del-a-vec} `to work, worker' \citep[163--164]{sim+:toporisic2000}. We take this to be lexically conditioned allomorphy. }

Given the pattern in \REF{ex:brati}, the ``\textsc{l}-is-participial" analysis may be the most straightforward one. Such an approach would mean that \textit{lec}-nominalisations join a broad\-er class of departicipial nominalisations, which also include nominalisations illustrated in \REF{ex:anketiranec}, standardly analysed as derived from the passive participles (e.g., \citealt{sim+:toporisic2000}).

\ea \label{ex:anketiranec}
\glll anketir-a-ti -- anketir-a-n -- anketir-a-n-ec \\
        survey-\textsc{tv}-\textsc{inf} {} survey-\textsc{tv}-\textsc{pass.ptcp} {} survey-\textsc{tv}-\textsc{pass.ptcp}-\textit{er} {} %court-\textit{ac}
        \\
 {`to survey'} {} {`surveyed'} {} `respondent' {} \\\hfill (Slo)
 %\glt 
\z

\noindent Moreover, if derived from a participle, \textit{lec}-nominalisations can be taken to be similar to other agentive nominalisations that have a form from the verbal paradigm as their base. Such an analysis is possible, for example, for agentive  nominalisations in \textit{-telj}, where the base seems to be the short infinitive.\footnote{Note that -\textit{lec} and -\textit{telj} are not allomorphs. First, there are a few pairs with these suffixes combined with the same base (e.g., the Slovenian \textit{brani-telj} -- \textit{brani-lec} `defender' from \textit{braniti} `to defend' or \textit{hrani-telj} -- \textit{hrani-lec} `custodian' from \textit{hraniti} `to keep in custody'). Second,  -\textit{telj} is much more consistently related to an agent interpretation (i.e., animate and human; there are only a few exceptions, such as \textit{pokazatelj} `indicator'). On the other hand, \textit{-lec} can also be associated with an instrument interpretation, see \S \ref{sec:severing}. Finally, nominalisations with -\textit{telj} are far less common than nominalisations with -\textit{lec} \citep[see][]{sim+:WeSoSlaV}. }



\ea \label{ex:predavatlj}
\glll predav-a-t -- predav-a-t-elj \\
        lecture-\textsc{tv}-\textsc{s.inf} {} lecture-\textsc{tv}-\textsc{s.inf}-\textit{er} {} 
        \\
 {`to lecture'} {} `lecturer' {} \\\hfill (Slo)
\z

\ea \label{ex:ucitlj}
\glll uč-i-t -- uč-i-t-elj  \\
        teach-\textsc{tv}-\textsc{s.inf} {} teach-\textsc{tv}-\textsc{s.inf}-\textit{er} {} 
        \\
 {`to teach'} {} `teacher' {} \\\hfill (Slo)
\z

\noindent As will be discussed in detail in \S \ref{sec:whatisl}, despite the many similarities,  \textit{lec-}nomina\-li\-sa\-tions (and nominalisations from short infinitives) are not fully comparable to passive-participle nominalisations, since only the latter preserve the prosody of the participle and allow all theme-vowel classes in the verbal base. In this paper, we therefore revisit the issue of the nature and the contribution of the \textsc{l}-morpheme. The empirical data and the proposed analysis tackle some of the foundational questions of morphology, in particular regarding the status of roots, cycles of computation, and their interactions. While we will focus on Slovenian data, the observations and the analysis can be extended to BCMS. In what follows, the examples are from Slovenian, unless marked otherwise. 
%\setlength{\footheight}{42.94601pt}


Before we continue, a remark is in order on a type of nominalisation that is NOT attested in Western South Slavic, since this gap will inform our analysis. Nominalisations from the three bases shown above (approximately matching the \textsc{l}-participle, passive (\textsc{n/t})-participle and short infinitive) are, to the best of our knowledge, the only deverbal derivations that preserve the theme vowel of the base verb.\footnote{The combination of the root and the theme vowel by themselves, without an overt derivational suffix, is also not attested as a derivational pattern (i.e., something like \textit{predava} or \textit{uči} does not occur as a nominalisation). This naturally means that zero-derived nominals in which the theme vowel is not present, e.g., the Slovenian \textit{popis} `inventory' (related to \textit{popis-a-ti} `to catalogue'), are not at issue here.} In other words, there are no deverbal derivations, such as the hypothetical derivations illustrated in \REF{ex:predavap}, where the root and the theme vowel would directly combine with a hypothetical morpheme \textit{-p} that would not show up in the paradigm of the verb. 
 

\ea \label{ex:predavap} \ea
\glll predav-a-ti -- *\hspace{-2pt} predav-a-p \\
        lecture-\textsc{tv}-\textsc{inf} {} {} lecture-\textsc{tv}-\textit{p} {} 
        \\
 {`to lecture'} {}  {} \\
 %\glt 
\ex \label{ex:ucip}
\glll uč-i-ti -- *\hspace{-2pt} uč-i-p \\
        teach-\textsc{tv}-\textsc{inf} {} {} teach-\textsc{tv}-\textit{p} {} 
        \\
 {`to teach'} {} {} \\
 %\glt 
\z
\z

\noindent In sum, whenever a theme vowel appears in a nominalisation, it appears embedded under additional (seemingly) functional material. In what follows, we advance an account in which, in the nominalisations, this material, i.e., both the \textsc{l}-morpheme and the passive-participle \textsc{n/t}-morpheme, correspond to the same conceptually empty root.\footnote{The notion of a conceptually empty root corresponds to the notion of a light root in the sense of \citet{Quaglia2022}, who use this label for secondary-imperfective suffixes; see also \S \ref{sec:syntax_of_participial_morphemes}.} Focusing on the \textsc{l}, we argue that \textsc{l}-participles contain a richer structure than the corresponding portion of \textsc{l}-nominalisations, while no such difference is found with the passive-participle \textsc{n/t}-morpheme. We leave agentive nominalisations that are derived from the short infinitive for future work.


The paper is organised as follows. In \S \ref{sec:severing}, we discuss (both old and new) reasons for splitting \textsc{l-}initial deverbal suffixes into an \textsc{l}-morpheme and another suffix that is added on top of it (e.g.,\textit{-ec}). In \S \ref{sec:whatisl}, we discuss the nature of the \textsc{l}-morpheme in the nominalisations under consideration. \S \ref{sec:analysisnew} presents our account of the structural position of the \textsc{l}-morpheme and the theme vowel restrictions in the respective nominalisations. \S \ref{sec:conclusions} concludes the paper. 

\section{Internal structure: Severing \textsc{l} from \textit{-ec} (\& other affixes)}\label{sec:severing}

We start the discussion with the internal structure of \textit{-lec} and related derivations in Slovenian. While various other agentive nominalisations also exist, deverbal nominalisations that contain the \textsc{l}-morpheme preceded by a theme vowel are by far the most common in Slovenian (see \citealt{marvin2016, marvin2015} quoting \citealt{stramljic1999}).\footnote{\citet[]{stramljivc1995specializiranost} presents counts in which -\textit{ač} emerges as the most frequent affix in agentive nominalisations. This is due to the fact that the author assumes -\textit{ilec} and -\textit{alec} to be two separate affixes. If -\textit{ilec} and -\textit{alec} are taken to instantiate the same item depending on the theme vowel of the base verb (as it is assumed in this paper), the unified item comes out as more frequent than -\textit{ač} in her counts as well.} Example \REF{ex:brus}, taken from \citet[ 98, (22)]{marvin2002}, illustrates three such nominalisations. According to \citeauthor{marvin2002}, the three affixes added to the \textsc{l}-morpheme (-\textit{ec}, -\textit{k} and -$\emptyset$) are variants of the same affix deriving nouns of three different genders. All three nominalisations are generally related to an external argument, be it an agent or an instrument, with the neuter-gender nominalisation primarily having the instrument interpretation (\cite[ 99, fn. 18]{marvin2002}, but see fn. \ref{fn:unacc} for examples in which \textit{-lec} is not associated with agentivity).\footnote{As pointed out by a reviewer, it is relevant to show at this point that the examples in \REF{ex:brus} indeed have nominal properties. The examples in \REF{ex:adjbrus} give the same nominalisations in the genitive case (case, number and gender being nominal properties in Slovenian) with an agreeing adjective. These items cannot be modified by an adverb such as \textit{hitro} `fast.\textsc{adv}'.

\ea \label{ex:adjbrus} \ea 
\glll *hitro / hitrega brus-i-l-c-a \\ 
{ }fast.\textsc{adv} {} fast.\textsc{gen.sg.m} sharpen-\textsc{tv}-\textsc{l}-\textit{er}-\textsc{gen.sg} \\
`a{ }fast sharpener.\textsc{m}'\\

\ex \glll hitre brus-i-l-k-e \\
 fast.\textsc{gen.sg.f} sharpen-\textsc{tv}-\textsc{l}-\textit{er}-\textsc{gen.sg} \\
 `a{ }fast sharpener.\textsc{f}'\\

\ex \glll hitrega {brus-i-l-$\emptyset$-a  } \\
fast.\textsc{gen.sg.n} sharpen-\textsc{tv}-\textsc{l}-\textit{er}-\textsc{gen.sg}\\ 
`a{ }fast sharpening{ }device' \\
% \glt

\z \z 

\noindent Still, as pointed out in \citet[101]{marvin2002},  such nominalisations can be modified by, for example, manner adverbials, which in fact modify the event included in the nominalisations, in turn implying an event component in these nouns. \label{fn:adverbial}} 


\ea \label{ex:brus}
\ea \glll brus-i-l-ec-$\emptyset$\\
sharpen-\textsc{tv}-\textsc{l}-\textit{er}-\textsc{nom.sg} \\
`a{ }sharpener.\textsc{m}'\\

\ex \glll brus-i-l-k-a \\
sharpen-\textsc{tv}-\textsc{l}-\textit{er}-\textsc{nom.sg}\\
`a{ }sharpener.\textsc{f}'\\


\ex \glll  brus-i-l-$\emptyset$-o  \\
  sharpen-\textsc{tv}-\textsc{l}-\textit{er}-\textsc{nom.sg}\\
 `a{ }sharpening{ }device' \\
\z \z 


\noindent Moreover, it has long been observed that various nominalisations and adjectivisations share the same \textsc{l}-final base. Some further affixes (both nominal and adjectival) that can combine with the \textsc{l}-form are exemplified in (\ref{ex:gledal}). Note that the adjectivising items -\textit{en} and -\textit{n} are exponents of the same item, whereby the exponent -\textit{en} [ən] includes an epenthetic vowel. The distribution of the epenthetic vowel is guided by the same rules as for -\textit{ec} (see fn. \ref{fn:epenthetic}). The two suffixes that we gloss as \textsc{place} are two different items. While -\textit{išč} consistently results in a place interpretation, -\textit{ic} is only associated with this interpretation when in the context of -\textsc{l} and -\textit{n}. While these suffixes are in and of themselves interesting and underexplored, a more detailed account of them is beyond the scope of this paper.   

\ea \label{ex:gledal}
\glll čak-a-l-en | čak-a-l-n-ic-a | čak-a-l-išč-e \\
wait-\textsc{tv}-\textsc{l}-\textsc{adj} {} wait-\textsc{tv}-\textsc{l}-\textsc{adj}-\textsc{place}-\textsc{nom.sg} {} wait-\textsc{tv}-\textsc{l}-\textsc{place}-\textsc{nom.sg}  \\
`waiting.\textsc{a}' | `waiting{ }room' | `waiting{ }spot'\\
\z %\label{fn:lother} 

\noindent The existence of such families of related derivations is a strong argument for the decomposition of \textit{-lec} into \textsc{l} and \textit{ec}, but also an argument for recognising \textsc{l} as a morpheme that is required for the verbal base to combine with derivational affixes, especially given the observation that there are no nominalisations in which the nominalising affix simply combines with the verb stem (i.e. minimally, root + theme vowel), as shown in \REF{ex:predavap}. In other words, as soon as the base is verbal, and marked as such by the presence of the theme vowel, another item needs to `mediate' in the attachment of the nominaliser. 
 
\largerpage
A decomposition of -\textit{lec} is  further supported by the fact that the affixes added to the \textsc{l}-morpheme, as in the examples \REF{ex:brus} and \REF{ex:gledal}, also show up in other environments. For example, \citet{marvin2002} shows that the suffix -\textit{ec} can also be found in various non-verbal environments, i.e., with adjectives, roots, and nouns, as illustrated by \REF{ex: ec}.\footnote{The relevant nominalising suffix in Slovenian is really just -\textit{c}, and the vowel in -\textit{ec} [əts] is an epenthetic vowel inserted to avoid a complex coda. As such, the vowel is absent in many forms of each paradigm, such as the dual \textit{bakr-en-c-a} `copper coin.\textsc{du}' for \REF{ex:bakrenec}, \textit{hod-c-a} `walker.\textsc{du}' for \REF{ex:drob}, \textit{brusil-c-a} `sharpener.\textsc{du}' for \REF{ex:brus} etc. We continue to use -\textit{ec} in the text for simplicity.\label{fn:epenthetic}} An analogous argument can be made for its feminine counterpart \textit{-k}, as shown in \REF{ex:nka}.\footnote{The situation is somewhat more complicated with nouns. When merging with a masculine nP, as in \REF{ex:krogec}, the noun with -\textit{ec} will get a diminutive reading. We leave the diminutive interpretation aside at this point. \label{fn:diminutive}}


 \ea \label{ex: ec} \ea \label{ex:bakrenec}
\glll bakr-en | bakr-en-ec-\emptyset \\
copper-\textsc{adj}  {} copper-\textsc{adj}-\textit{er}-\textsc{nom.sg}\\ 
{`made of copper'}  | {`copper{ }coin'} \\

\ex \glll hod- %\hspace{0.6cm} 
| hod-ec-\emptyset \\ 
 $\sqrt{\textsc{walk}}$  {} walk-\textit{er}-\textsc{nom.sg} {} \\ 
{}  | `walker' \label{ex:drob}\\

\ex \glll krog %\hspace{0.7cm} 
| krog-ec-\emptyset \\ 
 circle {} circle-\textit{er}-\textsc{nom.sg} \\ 
`circle' | `small{ }circle'\label{ex:krogec}\\

\z \z 


\ea \label{ex:nka} \ea 
\glll jekl-en  %\hspace{0.4cm} 
| jekl-en-k-a \\
steel-\textsc{adj} {} steel-\textsc{adj}-\textit{er}-\textsc{nom.sg}\\ 
{`made of steel'}  | `gas{ }cylinder'\\

\ex \glll hod-  %\hspace{0.7cm} 
| hod-k-a \\ 
 $\sqrt{\textsc{walk}}$ {} walk-\textit{er}-\textsc{nom.sg} {} \\ 
{} | `walker' \label{ex:hodka}\\


\ex \glll adidas  %\hspace{0.8cm} 
|  adidas-k-a \\ 
 Adidas {} adidas-\textit{er}-\textsc{nom.sg} \\ 
{} | `adidas-shoe'\\
\z \z


\noindent These facts unequivocally demonstrate that the item -\textit{lec} is complex, comprising two distinct items, namely \textsc{l} and \textit{-ec}. And while the morpheme \textit{-ec} may be treated as invariant between deverbal and other nominalisations in which it occurs (it consistently  restricts the denotation to count objects), the status of the \textsc{l}-morpheme in these nominalisations and its relation to the \textsc{l}-morpheme that surfaces in the \textsc{l}-participle is more complex. We address the issue in what follows and argue that this morpheme universally stativises event predicates of various sizes in order to license the derivation of words that are not verbs and that denote the event described by the verbal expression.


\section{What is \textsc{l}? \label{sec:whatisl}}
\largerpage

\noindent The question of the status of the \textsc{l}-morpheme in \textit{lec}-nominalisations is not a new one. As we show in what follows, both traditional descriptive sources and formal accounts offer a variety of solutions. We start with an overview of traditional accounts.  


\subsection{Traditional accounts of \textsc{l} in nominalisations}
For Slovenian, some authors take \textit{-lec} to be a single morpheme; for example, \citet[]{sim+:toporisic2000} treats -\textit{lec}, as in example \REF{ex:brati}, and -\textit{ec}, as in \REF{ex: ec}, as two separate items. The issue of \textsc{l} in \textit{lec}-nominalisatons is considered in \citet[]{stramljic1999}, who mentions as a possible answer \citeposst{bajec1950besedotvorje} proposal that  \textsc{l} essentially generalises from neuter-gender nominalisations such as \textit{zija-lo} `gawker' (form
\textit{zijati} `to gawk'). On the other hand, \citet[]{bajec1956slovenska} argue that either the neuter-gender nouns with -\textit{lo} or \textsc{l}-participles can serve as the derivational base for \textit{lec}-nominalisations.  

Similar proposals also exist for BCMS, where a common denominator of the accounts which propose a single suffix \textit{-lac} (\cite{maretic1963, babic1986tvorba, klajntvorbasufiksacija}) is the assumption that the \textsc{l}-participle encodes past. Given that \textsc{l}-participles are also used in past tense in Slovenian, such an assumption could easily be extended to Slovenian. However, this assumption has little empirical ground, since the \textsc{l}-participle is used in a variety of syntactic contexts in Slovenian, e.g., with the conditional or the future tense, where it does not receive `past' interpretation.\footnote{In fact, even as a part of the perfect form, traditionally analysed as past tense in BCMS, its meaning varies between the `past' interpretation and the present perfect, as extensively argued in \citet{todorovic2016}. This is exemplified in \REF{neda} and also holds for Slovenian. 
\ea \label{neda} \gll Jeo sam. \\
   eaten.\textsc{m.sg} \textsc{aux.1.sg} \\\hfill (BCMS) 
 \glt `I ate/I've eaten. (i.e., I'm not hungry.)' \\ 
 \z



}

\ea \glll je hodil | bo hodil  \\
   \textsc{aux} walk.\textsc{ptcp} {} \textsc{aux} walk.\textsc{ptcp} \\
  `has walked' | `will walk' \\ 
 %\glt
 \z  

\noindent We therefore do not take this as an argument against the decomposition of -\textit{lec}. 

Summing up, three options seem to emerge in the traditional literature: (i) -\textit{lec} is a single suffix, (ii) \textit{lec}-nominalisations are derived from \textsc{l}-participles and (iii) \textsc{l} in \textit{lec-}nominalisations spreads from \textit{lo}-nominalisations (where it is unclear what \textsc{l} in -\textit{lo} is).    

\subsection{Formal accounts of \textsc{l} and theme vowels in nominalisations (and beyond)} \label{Formal (DM) and beyond}


\subsubsection{There is no single \textsc{l}  \citep{marvin2002} \label{sec:dmmarvin}}
The account, which will in many ways serve as the starting point of our analysis, is the account of Slovenian \textit{lec}-nominalisations in \citet{marvin2002}, couched in Distributed Morphology \citep{Hal1993,Hal1994}. Considering the identity of the \textsc{l}-morpheme, \citet{marvin2002} proposes that \textsc{l} in these nominalisations is the participial \textsc{l}. This \textsc{l} in turn corresponds to the featureless Elsewhere Vocabulary Item that gets inserted in the T\textsubscript{2}/Participle head.\footnote{\citet{marvin2002} distinguishes between two T(ense) heads, T\textsubscript{1} and T\textsubscript{2}, whereby the latter corresponds to participles.} And since in Slovenian -\textit{lec} denotes an external argument, be it an agent or an instrument, \citet{marvin2002} proposes that \textit{-ec} is merged in the agentive position (Spec of \textit{v}P) and undergoes subject movement to the assumed SpecTP, resulting in the correct order in the linearised structure. \figref{ex:marvindrevo} provides the relevant structure before the movement of the nominal -\textit{ec} (as given in \citealt{marvin2002}, 99, (25)).\footnote{Following \citet[105-107]{marvin2002}, the inflectional ending carrying number and case agreement is inserted in the Number head when the nominalisation is used in a sentence, and this head nominalises the structure. The NumberP is, on her account, embedded under a DP. } Placing -\textit{ec} in the external argument position in the described structure can be seen as predicting that only unergative and transitive verbs are able to form these nominalisations, while  unaccusatives will not be able to do so (since, as stated in \citealt[99]{marvin2002}, unaccusatives do not have an external argument position). This tentative prediction is confirmed by the WeSoSlaV database \citep{sim+:WeSoSlaV}, but see also \citet{marjanovivc2013word}. Out of 728 \textit{lec}-nominalisations in the Slovenian sub-base of WeSoSlaV, only 3 can be taken to be derived from unaccusatives, and the majority of unaccusative verbs, such as \textit{porumeneti} `to become yellow', do not derive \textit{lec}-nominalisations (\textit{*porumenelec} `(intended) someone who becomes yellow').\footnote{In fact, even these three examples can be successfully accounted for under an alternative analysis. That is, the set of \textit{lec}-nominalisations that \textit{prima facie} seem to be derived from unaccusatives  consists of  \textit{pogorelec} `victim of a fire' (from \textit{pogoreti} `burn down'), \textit{otrdelec} `something hardened (usually penis)' (from \textit{otrdeti} `harden') and \textit{osamelec} `something isolated (usually tree or hill)' (from \textit{osameti} `become alone'). As is clear from the translations, all of these have a very specific interpretation which is never agentive. Furthermore, all of these items can be argued to be deadjectival. As shown by \citet[]{aljovic2000} for BCMS and \citet{simonovicMismas2022} for Slovenian, unaccusatives can derive adjectival \textsc{l}-participles which have full adjectival paradigms and can serve as bases for further derivation (e.g., -\textit{ost}-nominalisation, in \textit{osamelost} `the property of being left alone'). Taking this into consideration, the three \textit{lec}-nominalisations that seem to be derived from unaccusative verbs may well be derived from adjectives and therefore lack the agentive interpretation. \label{fn:unacc}}


\begin{figure}
{\setstretch{0.7}
\begin{forest}
sn edges/.style={for tree={
parent anchor=south, child anchor=north}},
%sn edges
	[\textcolor{white}{empty}
	[T\textsubscript{2} [-\textit{l}]] [\textit{v}P
	[\textit{n} [-\textit{ec}]][\textit{v}'%\textcolor{white}{n}
        [\textit{v} [ $\emptyset$]][$\sqrt{}$P  [\textit{plav-a},roof]]]]]
\end{forest}

}
\caption{Structure proposed by \citet[ 99, (25)]{marvin2002}.}
\label{ex:marvindrevo}
\end{figure}

The analysis just outlined, under which these nominalisations are derived from \textsc{l}-participles, finds its support in the fact that virtually all \textit{lec/lac-}nomina\-li\-sations can be derived from an \textsc{l}-participle of an existing verb. However, if we take as a starting point existing \textit{lec/lka/lo}-nominalisations and work our way back towards a participial base, we soon  find nominalisations which contain \textsc{l} preceded by a combination of a root and a theme vowel that cannot be found in an attested verb. Such cases, exemplified by the last example in each row in \REF{ex: čudala}, are considered in \citet[]{marvin2002},  who treats them as (non-compositional) ``root \textit{l}-participle nominalisations". In these nominalisations the root together with the theme vowel is the complement of a Part(iciple)P (headed by the \textsc{l}-morpheme), which gets nominalised by \textit{-ec/k/$\emptyset$}. Crucially, \citeauthor{marvin2002} argues that in these root nominalisations (unlike the de\textit{verbal} \textit{lec}-nominalisation) the nominalised structure does not include a \textit{v}-head. Consequently, these nominalisations (e.g. \textit{rezilo}, unlike \textit{rezalo}) are argued to exhibit a lack of an event component (cf. fn. \ref{fn:adverbial}), and of an external agent position (Spec\textit{v}P). 

\ea \label{ex: čudala}
\ea
\gll rez- | rez-a-ti |  rež-e-mo | rez-a-l-$\emptyset$-o | rez-i-l-$\emptyset$-o\\
$\sqrt{\textsc{cut}}$ | `cut.\textsc{inf}' | `cut.\textsc{prs.1pl}' | `cutter' | `blade' \\
\ex \label{ex:barv}
\gll barv- | barv-a-ti | barv-a-mo | barv-a-l-$\emptyset$-o | barv-i-l-$\emptyset$-o\\
$\sqrt{\textsc{colour}}$ | `colour.\textsc{inf}' | `colour.\textsc{prs.1pl}' | `colouring{ }device' | `pigment' \\
\\
\ex \label{ex:god-e-mo}
\gll god- | gos-$\emptyset$-ti{ }/god-$\emptyset$-ti/ | god-e-mo | god-a-l-$\emptyset$-o\\
$\sqrt{\textsc{play}}$ | `play.\textsc{inf}' | `play.\textsc{prs.1pl}' | `string{ }instrument' \\

\z 
\z 

\noindent The fact that we can still observe an \textsc{l} item  and a theme vowel in examples like \textit{rezilo, barvilo, godalo}, despite the lack of a \textit{v}\textsuperscript{0}, is important.

On \citeauthor{marvin2002}'s analysis, neither the theme vowel nor the Part(iciple) projection are inherently linked to the verbal domain. \citet[110]{marvin2002} takes the \textsc{l} (and also the theme vowel) in these nominalisations to be a part of an  extended root and proposes that ``its meaning is non-compositional (encyclopedic) as if it were a regular bare root with some extra pieces of morphology, to which then a nominaliser is added in root nominals in general." She further states that ``it appears that the language is making use of the process of root extension to introduce new non-compositional meaning that for some reason could not be introduced by nominalising just a bare root” \citep[110, 111]{marvin2002}. Finally, as for the theme vowels, which in these nominalisations are restricted to the set of two (\textit{i} and \textit{a}), Marvin states that they are the default theme vowels in the language, but does not further elaborate on how they are assigned. To sum up the proposal in \citet{marvin2002}, some nominalisations that include \textsc{l} are taken to be derived from \textsc{l}-participles, while others include a ``root extending" \textsc{l}.

Extending our empirical base, the small set of \textit{lo}-nominalisations with a theme vowel switch can be complemented by the even smaller set of \textit{lo}-nominalisations for which no corresponding verb can be found. Despite the fact that no independently attested verbal base is available, these nouns are interpreted as instruments and their theme vowels also come from the set of two: \textit{a} and \textit{i}. Examples of these nominalisations given in \REF{ex: superčudala} come from \citet{Simonovic2020}.
 

\ea \label{ex: superčudala}
\ea  \label{ex: glasbilo}
\glll glasb-a | glasb-i-l-$\emptyset$-o\\ 
music-\textsc{nom.sg} {} music-\textsc{tv}-\textsc{l}-\textit{er}-\textsc{nom.sg}\\ 
`music' | `musical{ }instrument' \\
%\glt
\ex
\glll \hspace{1cm} / \hspace{1.2cm} | zrc-a-l-$\emptyset$-o\\
{} {} {}  {} mirror-\textsc{tv}-\textsc{l}-\textit{er}-\textsc{nom.sg}\\ 
{} {} {} | `mirror' \\
 %\glt 
\z 
\z  

\noindent Crucially for the analysis of \textsc{l}, the fact that these examples are not derived from a verb clearly implies that they are also not derived from \textsc{l}-participles. 

\subsubsection{\textsc{l} is a root, and the importance of the theme vowel set \label{sec:dmroot}}

An account of the \textsc{l} that participates in deverbal derivations is given in \citet{simonovicMismas2022}, where the focus is on adjectival \textsc{l}-participles. This work endorses a DM framework, but the authors assume a specific approach to derivational affixes proposed by \citet{lowenstamm2014}. Under this approach, all derivational affixes are viewed as transitive (or ``bound”) roots. This means that, on the one hand, these roots require a complement (either a phrase or a root), but they also project and can be embedded under a categorial head or selected by another root. This approach then crucially separates typical traditional derivational affixes into roots (which are acategorial, as are, in accordance with \citealt{marantz2001words}, all ``free” roots, i.e. roots that do not require a complement, e.g., $\sqrt{\textsc{dog}}$) and categorial heads.\footnote{This division is motivated by derivational affixes that (under a classic DM view) realise different categorial heads. One such example is  the English -\textit{an} which can appear in nouns (\textit{librarian}) or adjectives (\textit{reptilian}), examples from \citet[233]{lowenstamm2014}. An alternative view, according to which only affixes that are associated with different categories are roots, is presented in \citet{creemers2018some}. } In this pairing, categorial heads are typically phonologically empty and roots have semantic and/or phonological content. 


 Assuming this approach, \citet{simonovicMismas2022} discuss two types of participles -- verbal \textsc{l-}participles that we can find in complex tenses, and adjectival \textsc{l-}participles, which only derive from unaccusative verbs (see fn. \ref{fn:unacc}), arguing that \textsc{l} is a root in both. This root can merge with either a root or  a phrase. In adjectival participles, \textsc{l} is merged with a root, whereas in past participles it is merged with a verbalised structure. In addition, taking \textsc{l} to be a root then allows \citet{simonovicMismas2022} to offer a unified account even for \textsc{l} beyond the verbal domain, e.g., in the noun  \textit{krog-l-a} `sphere', %cf. \textit{krog} `circle'. 
related more directly to the noun \textit{krog} `circle' than to the verb \textit{krož-i-ti} `to circle'.

Interestingly, if we zoom in on allowed theme vowel classes, there is a discrepancy between the adjectival \textsc{l}-participles in \citet[]{simonovicMismas2022} and the derivations discussed in this paper. The set of morphemes that can precede the \textsc{l}-morpheme in the adjectival participles in \citet[]{simonovicMismas2022} prominently \textit{excludes} the theme vowels \textit{a} and \textit{i}, which are by far the most common theme vowels in Western South Slavic (in \citealt{WeSoSlaV_inflection}, 1504 out of 3000 verbs in the Slovenian part of the database have the theme vowel \textit{a} in the non-finite forms, which are relevant here, and 863 have \textit{i}; the situation is similar in BCMS). On the other hand, in the nominalisations and adjectivisations discussed in this paper, what can precede the \textsc{l}-morpheme is exactly this set of theme vowels. This holds not only for examples such as those in \REF{ex: superčudala}, where the base verb is not attested independently. It also holds for agentive \textit{lec-}nominalisations. \REF{ex:tvchange1} illustrates \textit{lec-}nominalisations from four transitive verbs where theme vowels $\emptyset$ and \textit{e} are replaced by \textit{i}. 
% grelec?

\ea \label{ex:tvchange1} 
\ea \label{ex:tvchange10}
 \glll ves-$\emptyset$-ti{ }/vez-$\emptyset$-ti/ -- vez-e-l{ }/vez-$\emptyset$-l/ -- vez-i-l-ec   \\
   embroider-\textsc{tv}-\textsc{inf} {} embroider-\textsc{tv}-\textsc{l} {} embroider-\textsc{tv}-\textit{er}.\textsc{nom.sg} \\
  `to{ }embroider' {} `embroidered' {}  `embroiderer' \\
 
\ex \label{ex:tvchange1a}
 \glll ples-$\emptyset$-ti{ }/plet-$\emptyset$-ti/ -- plet-e-l{ }/plet-$\emptyset$-l/ -- plet-i-l-ec   \\
   knit-\textsc{tv}-\textsc{inf} {} knit-\textsc{tv}-\textsc{l} {} knit-\textsc{tv}-\textit{er}.\textsc{nom.sg} \\
   `to{ }knit' {} `knitted' {} `knitter' \\
 
 \ex \glll gnes-$\emptyset$-ti{ }/gnet-$\emptyset$-ti/ -- gnet-e-l{ }/gnet-$\emptyset$-l/ -- gnet-i-l-ec \\
 knead-\textsc{tv}-\textsc{inf} {}  knead-\textsc{tv}-\textsc{l} {} knead-\textsc{tv}-\textit{er}.\textsc{nom.sg}\\
`to{ }knead' {}  `kneaded' {} `kneader'\\

  \ex \glll vrt-e-ti -- vrt-e-l -- vrt-i-l-ec\\ 
  spin-\textsc{tv}-\textsc{inf} {} spun-\textsc{tv}-\textsc{l} {}  spin-\textsc{tv}-\textsc{l}-\textit{er}.\textsc{nom.sg}\\
  `to{ }spin' {} `spun' {} `spinner'\\
  \z \z 

\noindent This change in the theme vowel is noted also in \citet[163--164]{sim+:toporisic2000}, who states that perhaps the affix is not -\textit{lec} but is rather \textit{V-lec}, where V is the final vowel of the stem (e.g., \textit{a} or \textit{i}), but if the stem is consonantal (i.e., the verb has a $\emptyset$ theme vowel), the vowel is realised as \textit{i}. Based on the observation that verbs with a \textit{$\emptyset$} theme vowel are nominalised with -\textit{ilec}, \citet[]{stramljic1999} concludes that this form is the least marked option. 

In the context of formal accounts, on the other hand, the discrepancy in theme vowels present in adjectival \textsc{l}-participles and \textit{lec}-nominalisations can be taken as a consequence of different structures. While \textit{lec}-nominalisations have convincingly been shown to include a verb phrase, see \S \ref{sec:dmmarvin}, in adjectival \textsc{l}-participles the complement of  \textsc{l} has been argued to be a root \citep{simonovicMismas2022}.\footnote{Note that the same restriction to the theme vowels \textit{a} and \textit{i} is attested in much less productive nominalisations derived from short infinitives.}  And yet, if we assume that theme vowels surface as exponents of the verbalising head \textit{v}\textsuperscript{0} (as proposed in \citealt{Quaglia2022} for Slavic, \citealt{MilosArsen2022} for Serbo-Croatian; see also \citealt{Svenonius2004a} for Russian, \citealt{sim+:Biskup2019} for Czech), the presence of the theme vowels \textit{a} and \textit{i}, the two most productive verbal theme vowels, in all nominalisations with \textsc{l} under discussion, even the ones in \REF{ex: čudala}, implies that all these nominalisations are in fact deverbal.
This means that both \textsc{l}-participles and \textit{lec}-nominalisations have a verbal structure, but also need to be different in some way. Put differently, if the set of allowed theme vowels is a reliable diagnostic for differentiating between different structural environments, then \textit{lec}-nominalisations (where, again, only \textit{a} and \textit{i} are allowed) is a different environment from the \textsc{l}-participle, where all theme vowels are allowed.

We propose a specific solution in \S \ref{sec:analysisnew}.




\subsubsection{Change in the secondary imperfective}

Another argument that the base onto which \textit{-ec} is added is not the verbal participle is offered by the fact that in some cases the secondary imperfective morpheme also does not match between the verbal ``base'' and the \textit{-lec} nominalisation. \REF{ex:izterjevalec} gives two such nominalisations.

\ea \label{ex:izterjevalec}
\ea \glll  obračun-a-ti -- obračun-av-a-ti  --  \textsuperscript{??}\hspace{-2pt} obračun-ov-a-ti  --  obračun-ov-a-l-ec \\
calculate-\textsc{tv}-\textsc{inf} {} calculate-\textsc{si}-\textsc{tv}-\textsc{inf}  {} {} calculate-\textsc{si}-\textsc{tv}-\textsc{inf}  {}  calculate-\textsc{si}-\textsc{tv}-\textsc{l}-\textit{er}\\ 
`calculate.\textsc{pfv}' {}  `calculate.\textsc{ipfv}' {}  {} `calculate.\textsc{ipfv}' {} `calculator'    \\
\ex \glll prikim-a-ti -- prikim-av-a-ti -- \textsuperscript{??}\hspace{-2pt} prikim-ov-a-ti -- prikim-ov-a-l-ec\\
nod-\textsc{tv}-\textsc{inf}  {}  nod-\textsc{si}-\textsc{tv}-\textsc{inf} {} {}  nod-\textsc{si}-\textsc{tv}-\textsc{inf}   {} nod-\textsc{si}-\textsc{tv}-\textsc{l}-\textit{er}\\
`nod.\textsc{pfv}' {} `nod.\textsc{ipfv}' {}  {} `nod.\textsc{ipfv}' {} `nodder'\\
\z
\z

\noindent We do not provide a full analysis of these examples here, but rather leave this for future work.

\subsubsection{Prosody and \textsc{l}\label{sec:dmprosody}}
 

The final type of evidence featuring in DM approaches to \textit{lec}-nominalisation is their prosodic behaviour. All nominalisations and adjectivisations containing the \textsc{l-}morpheme share the same prosodic pattern, i.e., stress on the theme vowel, which overrides the lexical prosody of the base verb (if available). This is illustrated in \REF{ex:pros1}, where the verbal bases do not all have the same stress pattern, as can be seen in the \textsc{l}-participles, but these differences get neutralised in all other cases.\footnote{A comparable pattern is observed in \citet{caha2022prefixes} for Czech, where, in terms of vowel length, all verbal forms have the same allomorph, but the nominalisation has a different one.} 



\ea \label{ex:pros1}
\ea
\glll moˈr-i-l vs. | moˈr-i-l-ec -- moˈr-i-l-k-a -- moˈr-i-l-en\\
murder-\textsc{tv-l.m.sg} {} {} murderer-\textsc{tv-l}-\textit{er}.\textsc{nom.sg} {} murderer-\textsc{tv-l}-\textit{er}.\textsc{nom.sg} {} murder-\textsc{tv-l-adj.m.sg}'\\
`murdered' {} | `murderer.\textsc{m}' {} `murderer.\textsc{f}' {} `related{ }to{ }murder'\\
\label{ex:mor1}

\ex
\glll ˈmer-i-l vs. | meˈr-i-l-ec -- meˈr-i-l-k-a -- meˈr-i-l-en\\
measure-\textsc{tv-l.m.sg} {} {} measure-\textsc{tv-l}-\textit{er}.\textsc{nom.sg} {}  measure-\textsc{tv-l}-\textit{er}.\textsc{nom.sg} {} measure-\textsc{tv-l-adj.m.sg} \\
`measured' {} | `measurer.\textsc{m}' {} `measurer.\textsc{f}' {} `related{ }to{ }measuring'\\
\label{ex:mer1}
%
% tuki spodi sem zakomentiral primere, kjer ni razvidno, da se je prosodija nevtralizirala.

\ex
\glll ˈrez-a-l vs. | reˈz-a-l-ec -- reˈz-a-l-$\emptyset$-o -- reˈz-i-l-$\emptyset$-o\\
cut-\textsc{tv-l.m.sg} {} {}  cut-\textsc{tv-l}-\textit{er}.\textsc{nom.sg} {} cut-\textsc{tv-l}-\textit{er}.\textsc{nom.sg} {} cut-\textsc{tv-l}-\textit{er}.\textsc{nom.sg}\\
`cut' {} |  `cutting{ }person.\textsc{m}' {} `cutter'  {} `blade'\\
\ex
\glll iˈgr-a-l vs. | iˈgr-a-l-ec | iˈgr-a-l-k-a | iˈgr-a-l-$\emptyset$-o\\
play-\textsc{tv-l.m.sg} {} {} player-\textsc{tv-l}-\textit{er}.\textsc{nom.sg} {} player-\textsc{tv-l}-\textit{er}.\textsc{nom.sg} {} play-\textsc{tv-l}-\textit{er}.\textsc{nom.sg}\\
`played' {} | `player.\textsc{m}' | `player.\textsc{f}' | `playground{ }equipment'\\
%
%\ex
%\glll zrˈc-a-l-$\emptyset$-o\\
%mirror-\textsc{tv-l}-\textit{er}-\textsc{nom.sg} \\
%`mirror' \\\\
% \glt  
\z 
\z 



\noindent In stark contrast to the nominalisations that contain the \textsc{l}-morpheme, those that contain the passive participle (the \textsc{n/t-}participle) behave as stress-preserving, as illustrated in \REF{ex:pros2}. Here in each case the prosodic pattern of the passive participle is preserved in all further derivations.


\ea \label{ex:pros2}
\ea
\glll ˈmerjen vs. | ˈmerjen-ec -- ˈmerjen-ka -- ˈmerjen-je\\
measure.\textsc{pass.ptcp} {} {} measured-\textit{er}.\textsc{m} {} measured-\textit{er}.\textsc{f} {} measured-\textit{ing}\\
`measured' {} | `measured{ }person.\textsc{m}' {} `measured{ }person.\textsc{f}' {} `measuring'\\
\ex
\glll umorˈjen vs. | umorˈjen-ec {} umorˈjen-ka\\
murder.\textsc{pass.ptcp} {} {} murdered-\textit{er}.\textsc{m}  {} murdered-\textit{er}.\textsc{f} \\
`murdered' {} | `murdered{ }person.\textsc{m}' {} `murdered{ }person.\textsc{f}' \\
\ex
\glll ˈpitan vs. | ˈpitan-ec -- ˈpitan-ka --  ˈpitan-je \\
fatten.\textsc{pass.ptcp} {} {} fattened-\textit{er}.\textsc{m} {} fattened-\textit{er}.\textsc{f}' {} fattened-\textit{ing}\\
`fattened' {} | `fatling.\textsc{m}' {} `fatling.\textsc{f}' {} `fattening'\\
\ex
\glll zgaˈran vs. | zgaˈran{ }-ec {} zgaˈran{ }-ka\\
exhaust.\textsc{pass.ptcp} {} {} exhausted-\textit{er}.\textsc{m} {} exhausted-\textit{er}.\textsc{f}\\
`exhausted' {} | `exhausted{ }person.\textsc{m}' {} `exhausted{ }person.\textsc{f}'\\
% \glt 
\z 
\z 


\noindent \citeposst{marvin2002} account of these facts makes crucial use of phasal spellout. While the \textsc{l}-morpheme is in Part\textsuperscript{0}/T\textsubscript{2}\textsuperscript{0}, which is not a phasal head, the \textsc{pass.part} morpheme is in Pass\textsuperscript{0}, which is an adjectival head. Since categorial heads trigger spellout, the prosody of the passive participle is computed and shipped off to PF, so it cannot be altered by morphemes that get merged later. \textsc{l}-participles, on the other hand, do not constitute phases and therefore allow morphemes like \textit{-ec} to interfere with the prosody of the whole. 

However, while \citeauthor{marvin2002}'s account correctly predicts prosodic faithfulness in derivations from passive participles, it does not predict total neutralisation of lexical prosody in derivations from \textsc{l}-participles (including also adjectives in \textit{-n} and \textit{-sk} and others, see \citealt{Simonovic2020}). Rather, what we would expect is that some of the further affixes are stress-affecting, whereas others are stress-neutral and allow for the preservation of lexical prosody. 

In order to resolve the problem of obligatory stress-shifting behaviour in derivations from \textsc{l}-participles, \citet{Simonovic2020}, who also follows \citet[]{lowenstamm2014} in assuming that derivational affixes are roots, generalises \citeauthor{marvin2002}'s idea of extended roots to all nominalisations that contain the \textsc{l}-morpheme. On this analysis \textsc{l} is a root-selecting root, which appears in a structure that \citet[]{lowenstamm2014} terms a ``radical core'', i.e. a sequence of roots with no intervening categorial heads. In radical cores, default prosody of the language is assigned. \citet{Simonovic2020} argues more generally that all cases where affixal prosody overrides lexical verbal stress should be analysed as cases of radical cores. For  [{meˈr-i-l-əts}] and [{meˈr-i-l-ən}] from \REF{ex:pros1}, the relevant radical cores would be:

\ea \label{ex:brusroots}
\gll $\sqrt\textsc{mer(i)}$+$\sqrt\textsc{l}$+$\sqrt\textsc{c}$ --  $\sqrt\textsc{mer(i)}$+$\sqrt\textsc{l}$+$\sqrt\textsc{n}$ \\
`measurer' {} `measuring.\textsc{adj}' \\
% \glt 
\z 



\noindent In both examples the radical cores span over all the morphemes that have phonological content and are embedded under a silent nominaliser and adjectiviser, respectively. Default stress in Slovenian is final, but schwa is avoided by stress, which is why we get [{meˈr-i-l-əts}] and [{meˈr-i-l-ən}] rather than *[{mer-i-ˈl-əts}] and *[{mer-i-ˈl-ən}] (for a full analysis, see \citealt{simonovic2025}).\footnote{A question that \citet{Simonovic2020} leaves open is the status of theme vowels (e.g., \textit{mer-\underline{i}} in \REF{ex:brusroots}). If the whole structure is a radical core, the theme vowel has to be part of the root, just as in \citet[]{marvin2002}. Then the problem remains why the same root can appear without the theme vowel, e.g., in the noun \textit{mer-a} `measure' and in the adjective \textit{mer-en} `measuring'. Our analysis in \S \ref{sec:analysisnew} addresses this issue. \label{fn:markoproblem}} 


In light of the previous discussion, our approach in this paper departs from \citet{Simonovic2020}, in that we argue that the structure below \textsc{l} is not itself a root, but minimally a \textit{v}P, since it contains a theme vowel. The question, now, is whether we can still account for the uniform prosodic behavior of all \textit{lec-} and related derivations. The answer is that this uniform prosody is predicted as long as we maintain that \textsc{l} is a transitive root which is required to be selected by a root. The presence of a root selected by a root in the structure will always result in a radical core and impose default prosody whenever the radical core is spelled out. As in \citet{Simonovic2020}, the lexical prosody of the \textsc{l}-participle is then a consequence of the fact that participles do not contain any radical cores. 

And finally, consider the following example as an illustration of the assignment of default prosody. The dual form of \textit{meˈr-i-l-ec}, given in \REF{ex:mer1}, is \textit{meˈr-i-l-c-a} (we use it because it has an overt case/number ending). This word is spelled out in three cycles. First the \textit{v}P \textit{mer-i-} gets spelled out, then the \textit{n}P -\textsc{l}-\textit{c}- and then the case/number ending \textit{-a} follows.\footnote{See fn. \ref{fn:epenthetic} for the omission of \textit{e}.} Now, in the first cycle the faithful prosodic pattern wins: \textit{ˈmer-i-}. This output serves as the input to the second cycle, where there is a clash between the lexical prosody (\textit{ˈmerilc}) and the pattern imposed by the radical core (\textit{meˈrilc}). In such cases, the rightmost accent mark wins, so the theme vowel ends up stressed. Finally, the case/number ending \textit{-a} is stress-neutral and does not contain a radical core, so the whole word is realised as \textit{meˈrilca}.


\subsection{A summary}

In this section, we discussed previous approaches to \textit{lec-} and related nominalisations, while also 
articulating our own approach. We follow a host of previous formal approaches in severing the \textsc{l} morpheme from \textit{-ec} and all the other morphemes which it gets combined with. We however depart from the previous analyses in that we assume that all \textit{lec-} and related nominalisations contain verbal structure, while at the same time not containing full \textsc{l}-participles. The exact way in which these two verbal structures differ is the main focus of the following section.


\section{An analysis \label{sec:analysisnew}}
We pursue a unified analysis of the \textsc{l} morpheme in (i) \textsc{l}-participles, (ii) deverbal nominalisations with \textsc{l}, such as \textit{lec-}nominalisations, and (iii) non-deverbal items such as \textit{krog-l-a} `sphere' (see \S \ref{sec:dmroot}). As indicated in \S \ref{sec:whatisl}, we will pursue the idea that \textsc{l} is always a root. 

In this section we will tackle the task of explaining the two main issues that emerged throughout the paper.  First, all deverbal nominalisations require some extension of the verbal base, be it with \textsc{l}, the passive \textsc{n/t} or the \textsc{t} of the short infinitive (see \S \ref{sim+:intro}). The question, then, is why such extension is required and what the difference between the specific extensions is. Here we limit ourselves to \textsc{n/t} and \textsc{l} and leave short infinitives for future work.
Second, prosodic patterns and the set of allowed theme vowels distinguish between \textsc{l}-participles on the one hand and \textit{lec}- and other related nominalisations on the other. We have argued in the previous section that both of these environments involve a verbal structure. The question, then, is what the exact structure of nominalisations is and what the structure of participles is. Depending on the answer to this question, an account needs to be formulated of the way in which the inventory of theme vowels is restricted to \textit{a} and \textit{i} in \textit{lec}-nominalisations.
 


\subsection{Why is base extension needed and how it works}
Given that \textit{lec}-nominalisations are at the centre of this paper and that we have shown the -\textit{ec} in them to constitute a separate, independent suffix, we will limit the discussion to examples with -\textit{ec}. We will also assume that as a derivational affix -\textit{ec} is a transitive root, as argued (for all derivational affixes) in \citet{lowenstamm2014}, summed up in \S \ref{sec:dmroot}. This means that $\sqrt{\textsc{ec}}$ can be categorised (and in fact is categorised by an \textit{n} head) and has the ability to select. It it precisely this ability that leads to the modification of the base.

As shown in \S \ref{sim+:intro} and \S\ref{sec:severing}, nominalisers like  -\textit{ec} appear with different complements, but crucially never select for a \textit{v}P (see \S \ref{sim+:intro}). This is why there are no examples like \textit{*predav-a-p} or \textit{*uč-i-p}, cf. \REF{ex:predavap}. 
Since verbal bases are not acceptable complements for -\textit{ec}, merging -\textit{ec} with a \textit{v}P 
results in a crash and thus requires some extra operation or additional structure. We argue that insertion of the \textsc{l}-root is such an operation, which makes the modified structure parallel to examples such as \REF{ex:drob}. 
On the other hand,  -\textit{ec}  can merge with a passive (\textsc{n/t-}) base, since passives are adjectival (and therefore an acceptable complement), cf. \REF{ex:bakrenec}. 


In proposing root insertion of $\sqrt{\textsc{l}}$ in \textit{lec-}nominalisations, we essentially extend \citeposst{acquaviva2009roots} idea of root extension (and generalise \citeauthor{marvin2002}'s \citeyear{marvin2002} idea of root extension to all \textit{lec}-nominalisations). That is, \citet{acquaviva2009roots} 
argues that items such as \textit{de-stroy} consist of a ``lexical'' root $\sqrt{\textsc{stroy}}$ and a root extension $\sqrt{\textsc{de}}$, which attaches to $\sqrt{\textsc{stroy}}$, modifies it, and in doing so creates a complex root which is only then categorised. 
Since in \citeposst{lowenstamm2014} approach derivational roots are transitive, they are able to take any kind of complement and thus can extend either other roots or phrases. 


Our proposal is given in \figref{fig:Passiven}. $\sqrt{\textsc{l}}$ in nominalisations acts as an extension. However, unlike $\sqrt{\textsc{de}}$, it projects over the categorised (functional) structure with which it merges, and thus can be selected by items that select for roots, such as -\textit{ec}. Recall from \ref{sec:dmprosody} that this approach then also solves the issue of prosody. Since there is no categorial head above $\sqrt{\textsc{l}}$ to trigger spell out, the roots $\sqrt{\textsc{l}}$ and $\sqrt{\textsc{ec}}$ form a radical core.%\footnote{Given that \textit{-ec} can select also for an \textit{a}P as discussed above, we could also claim \textit{a}P projects above $\sqrt{\textsc{l}}$, but we leave this option aside.}
 

\begin{figure}  
{\setstretch{0.7}
\begin{forest}
sn edges/.style={for tree={
parent anchor=south, child anchor=north}},
%sn edges
	[nP[n] 
     [$\sqrt{\textsc{ec}}$
	 [$\sqrt{\textsc{ec}}$] [$\sqrt{\textsc{l}}$
      [$\sqrt{\textsc{l}}$]  [vP
	 [v] [RootP
	 [$\sqrt{\textsc{root}}$] [ObjectP]]]]]]]]
\end{forest}

}
\caption{Lec-nominalisations }
\label{fig:Passiven}
\end{figure} 

As we have seen in \S \ref{sec:severing}, the morpheme \textsc{l} displays extreme multifunctionality in Slovenian, showing up in non-verbal contexts (\textit{krog-l-a} `sphere'%, derived from  \textit{krog} `circle'
), as an extension of the domain of deverbal derivation, and as the participial ending. It behaves as the default elsewhere allomorph of the verbal domain (as proposed already in \citealt{marvin2002}), which makes it comparable to the morpheme \textsc{ov}, previously described as playing a similar role in the nominal and adjectival domain \citep{simonovic2020ov}. In the following subsection, we turn to the nature of \textsc{l} (and its passive counterpart) in participles.   



\subsection{The analysis of participial roots \label{sec:syntax_of_participial_morphemes}}

We propose that traditional participial morphemes,  \textsc{l} and \textsc{n/t}, are conceptually empty roots that are merged (unlike \textsc{l} in \textit{lec-}nominalisations) in the head of AspP. 
Our proposal now enables us to take a further step in investigating the nature of derivational roots. That is, \citet{lowenstamm2014} posits that derivational roots lack semantic content and that their root nature precludes the eventual realisation of syntactic content. The proposal according to which participial morphemes are roots merging in the head of AspP therefore raises two important questions. The first one concerns the possibility of a conceptually empty root to be manifested as either \textsc{l} or \textsc{n/t}. The second one regards the very possibility of a root being merged in the head of a functional projection.\footnote{Another potential issue, peculiar to Slavic languages, concerns a widespread view that secondary-imperfective suffixes are markers of (imperfective) grammatical aspect, and are typically analysed as heads of AspP (e.g. \citealt{sim+:Smith1997, sim+:Ramchand2004, Ramchand2008a, Borer.2005, sim+:Progovac2005, sim+:Borik2006}, among many others). The compatibility of participial morphemes in Slovenian and BCMS with secondary imperfectives at first glance clashes with our proposal that \textsc{l} and \textsc{n/t} are merged in Asp\textsuperscript{0}. However, the problem disappears once we analyse secondary-imperfective suffixes as reverbalisers, i.e. morphemes that combine with perfective verbs, which encode telicity, and return bare \textit{v}Ps (see %\citealt{Arsenijević2018,
\citealt{Arsenijević2023SLE, SimonovićArsMil2021} for detailed argumentation). This means that secondary-imperfective suffixes are merged below grammatical aspect, which is an idea that has also been advocated in \citet{Klein1995, sim+:Lazorczyk2010, tatevosov2015severing, tatevosov2017, Mueller-Reichau2020, sim+:Biskup2023, Milosavljević2023} -- although their exact function varies across approaches.} 


To address the first problem, we  employ allomorphy rules that take the root \textsc{l} to have two phonological realisations (or Vocabulary Items ) -- one that emerges in specific contexts (i.e., the `passive' /n/) and the elsewhere form (i.e., /l/).

 \ea     $\sqrt{\textsc{l}}$  $\leftrightarrow$ /n/ \textbackslash \hspace{0.3cm} \_\textsc{[passive]}  \\
      \textcolor{white}{$\sqrt{\textsc{l}}$} $\leftrightarrow$ /l/ \hspace{0.3cm} elsewhere\\
   
\z 

\noindent This then captures \citeposst{marvin2002} observation that \textsc{l} seems to be the elsewhere allomorph of the verbal domain.\footnote{Given that the passive affixes are referred to as \textsc{n/t} throughout the paper, we are simplifying the Vocabulary Items and referring the reader to \citet[92]{marvin2002}. } Still, the two Vocabulary Items are also associated with different interpretations -- we return to the issue in what follows. 


As for the proposal that the root \textsc{l} is merged in a functional head, we build on the analysis in \citet{Cavirani-Pots2020} (see also \citealt{Cavirani-Pots2021}), who argues that semi-lexicality of some items emerges when (lexical) roots are merged/ incorporated into a functional head. These  items  have both lexical and functional uses. For instance, the word \textit{bunch} is used in English as both a quantifier (\textit{A bunch of chickens were found on the trail}), and a regular lexical noun (\textit{The flowers were arranged in a beautiful bunch}). In the former, functional use, \textit{bunch} is incorporated into a functional head (Q\textsuperscript{0}), whereas in its lexical use, it realises a root position \citep{Cavirani-Pots2021}. And while typical instances of semi-lexicality include clearly meaningful roots such as \textit{bunch}, in the case of participial morphemes, the underlying procedure is the same; what is merged into a functional head (Asp\textsuperscript{0}) is a conceptually empty root. Completely parallel to other semi-lexical items, however, \textsc{l} too has lexical and functional uses. When it serves as an extension in deverbal nominalisations it is lexical (except that, here as well, it is devoid of conceptual content), and functional in participles. 

The obvious question that emerges is what the motivation is behind merging a conceptually empty root into the Aspectual head. To answer this question, but also to further motivate our analysis of \textsc{l} in Western South Slavic as a functional root, we will extend our discussion to the participial morphology more generally, as it exhibits very similar properties across languages. 

Participial morphemes are multifunctional in many languages, i.e., they appear in a variety of contexts. For instance, the same participial forms are found in the verbal/eventive participles, proper adjectives and (present) perfect constructions in languages such as English, German, Italian, Latin (see \citealt{Borik2019, Wegner2019book} for overviews). In Romance languages (Italian, Latin), like in Slavic, the ``participial'' bases are also found in nominalisations (e.g. \citealt{Calabrese2020}, and references therein). This diversity of contexts is the first property that is common to participial morphemes and roots (both traditional ones and affixes). The multifunctionality of participial morphemes has led many authors to propose that they have either a very light meaning or no meaning whatsoever -- which is a property of some affixes as roots in the sense of \citet{creemers2018some} and \citet{simonovic2020ov}. 

There are roughly three families of approaches in the formal literature trying to handle a pure (if any) semantic contribution of participial morphemes. One is to assume that they are exponents of the Asp head, with highly underspecified contexts of application and very abstract semantics (e.g. \citealt{Embick2000, sim+:Embick2004, Embick2005, Remberger2012, Wegner2019article, Wegner2021}).\footnote{For instance, \citet{sim+:Embick2004} postulates different ``flavours'' of the aspectual head to derive different types of passives in English -- eventive, resultative and stative. Notably, \citeauthor{sim+:Embick2004} also analyses what he calls ``stative participles" (but effectively adjectives) like \textit{closed} or \textit{open} as also including a (stative) Asp head that merges directly with the root, i.e. they differ from ``true'' verbal participles in lacking a verbalising head. \citet[286]{Remberger2012} proposes that participial morphemes in Latin are exponents of the nominal aspect n/Asp that has no specific tense value or temporal semantics, and means something like ``concerned/affected". \citet{Wegner2019article, Wegner2019book, Wegner2021} proposes that participial morphemes in English (but possibly also in other languages) are exponents of a single underspecified aspectual head and that specific aspectual values of the given predicate are computed based on its interaction with the telicity properties of the \textit{v}P in the complement of Asp, as well as with the semantics brought about by auxiliaries a particular participle combines with. \citeposst{Wegner2019article}  approach is reminiscent of more general approaches to grammatical aspect as default aspect such as \citet{Bohnemeyer_2004}.} An intermediate stance is that participial morphemes are meaningless at least in some environments (e.g. in the so-called target and resultant state participles in German or English), as proposed in \citet{Kratzer2000} and \citet{ramchand2018}, but their contribution is not vacuous even in such cases. For \citet{Kratzer2000}, the participial morpheme serves only to license the absence of verbal inflection (and consequently the external argument).\footnote{In \citet{Kratzer2000} the lack of verbal inflection explains why adjectival passives lack an external argument, as in her approach the external argument is introduced by verbal inflection. The stative nature of these participles is brought about either by a zero suffix, or by the adjectivising head itself.\label{fn:Kratzer}} According to \citet{ramchand2018}, the participial morpheme \textit{-en/-ed} in English is devoid of conceptual content associated with syntactic information. It is a ``stunted version of the inflected verbal form" \citep[127]{ramchand2018} that can spell out different subparts of the verbal structure up to AspP, i.e. ``any non-tense-information-carrying contiguous subset of the root's features" \citep[81]{ramchand2018}: ResP in the case of the stative passive, InitP for the verbal/eventive passive, and AspP in the present perfect construction. In the spirit of \citet{Kratzer2000}, \citet[92]{ramchand2018} contends that ``the effect of participle formation is not vacuous, presumably because it suspends the continuation of the verb to tense inflection and anchoring, and makes adjectivisation possible". Finally, the third, most radical view is that the convergence in ``participial" form across different syntactic contexts is a consequence of purely morphological rules rather than a reflex of any common semantic/syntactic core \citep{Calabrese2020}. This last view is based on the assumption that there is a separate morphological module, which can manipulate the output of the narrow syntax. Participial morphemes, alongside theme vowels, constitute a crucial piece of evidence for postulating such a module.\footnote{Apart from DM approaches like \citet{Calabrese2020}, this stance is at the heart of A-morphous approaches to morphology such as \citet{Aronoff1994}, who also analyses participial morphemes, alongside theme vowels, as purely morphological entities.}

Our approach, on which participial morphemes are conceptually empty roots merging in the Asp head, combines and further elaborates and motivates the first two of the three families of approaches to participial morphemes presented above. We immediately exclude the third option, i.e., the purely morphological analysis, according to which syntax does not play any role, as the roots under discussion clearly have a syntactic role, although they are conceptually empty.\footnote{The reader is also referred to \citet{MilosArsen2022, Kovacevic2023} for arguments against these being purely ornamental morphemes, much like theme vowels.} Specifically, an analysis in terms of roots explains their multifunctional (and multicategorial) status and their highly abstract/underspecified meaning or lack of meaning. These are, as we have seen, general properties of derivational affixes reanalysed as roots. It also explains the intuition hinted at in \citet{Kratzer2000} and \citet{ramchand2018} that they license the absence of anchoring to a specific context or, more generally, the absence of referential properties. This is in full accordance with the view that the meaning of roots is intensional, and that the referential properties are introduced by functional material (see \citealt{sim+:Arsenijevic2022} and the references therein), probably by the inflectional/person morphemes (cf. \citealt{ramchand2018}). Additionally, given that adjectives, like roots, %have only intensional meaning, i.e. they 
do not refer% \citep[e.g.][]{sim+:Arsenijevic2022}
, this aligns with the observed selectional restrictions of the nominalising suffixes discussed in the present paper. That is, suffixes like -\textit{ec} select for either adjectives or root structures where both are basically structures devoid of referential/extensional meanings. 

Our analysis also explains the intuition that participles generally have a ``stative" or ``adjectival" meaning \citep{Kratzer2000, ramchand2018, tatevosov2017, Borik2019}. Let us spell this out in more technical terms. The aspectual head normally specifies a temporal relation, which obtains between the eventuality described by its complement (i.e., \textit{v}P) and a temporal pronoun in its specifier (referring to the topical time, i.e., the reference time). When this head is filled with a conceptually empty root (such as $\sqrt{\textsc{l}}$), the temporal pronoun cannot be merged in the specifier -- exactly due to the intensional nature of the root, which licenses the absence of referential properties. Instead, the highest c-commanded argument moves to the specifier of AspP (the external argument in transitives and unergatives, the internal one in unaccusatives), deriving the interpretation that the predicate denoted by the \textit{v}P overlaps with the temporal dimension of this argument. The overlap interpretation is default for viewpoint aspect in the absence of an overt specification of non-overlap. This derives exactly the result state interpretation as in \citet{Kratzer2000}: the relevant argument bears the property of having participated, with a particular role, depending on the value of Voice, in events satisfying the description specified by the complement of Asp. This matches exactly the interpretation that verbal participles have.


\subsection{The syntax of \textit{lec}-nominalisations and theme vowel realisation}

As discussed in \S \ref{sec:dmroot}, \textsc{l}-nominalisations and \textsc{l}-participles allow different sets of theme vowels. Specifically, \textsc{l}-nominalisations allow only two theme vowels (\textit{i} and \textit{a}), while \textsc{l}-participles allow all theme vowels available in the language. This pattern is crucial evidence that the root \textsc{l} combines with different sizes of the verbal base in nominalisations and participles, i.e. that \textsc{l}-nominalisations do not actually contain \textsc{l}-participles. Specifically, the structure in \textsc{l}-nominalisations is smaller than that in \textsc{l}-participles. As the most conservative implementation, we assume that \textsc{l}-nominalisations only contain the \textit{v}P (as they always have a theme vowel). Note that this does not hold for \textsc{n/t}-nominalisations: they are always deadjectival, i.e. contain the passive participle, as also attested in their prosodic behaviour (see \S \ref{sec:dmprosody}). As is fully expected, \textsc{n/t}-nominalisations display the same set of theme vowels as passive participles. This then also means that the \textsc{n/t} morpheme never functions as a true \textit{v}P extension, but always as a participial ending.

Our final piece of the puzzle is the exact mechanism behind the variable theme vowel exponence. Recall that in the picture we sketched above (\S \ref{sec:dmroot}), some verbs have three theme vowel exponents.  A verb like \textit{vesti} `embroider' is a case in point: its theme vowel is realised as $\emptyset$ in the \textsc{l}-participle \textit{vez-$\emptyset$-l-a}, as \textit{e} in the present-tense form \textit{vez-e-mo} `we embroider' and as \textit{i} in the nominalisation \textit{vez-i-l-ec} `embroiderer'. 

Our general approach is to follow \citet[]{oltra1999notion} in the assumption that different theme-vowel classes result from root diacritics, such as [$\alpha$] and [$\beta$], or lack thereof. The present-tense version of the theme vowel is not in focus here. Suffice it to say that the spellout of the theme vowel is influenced by additional marked features on \textit{v} or an adjacent head (see \citeauthor{oltra1999notion}'s discussion of the ``marked T''). We are then left with the difference between participles and nominalisations/adjectivisations discussed in this paper. We can first define the most general vocabulary item for West South Slavic, which will apply in all cases where no more specific vocabulary item applies, as given in \REF{ex:VII}.


\ea \label{ex:VII}
       TV $\leftrightarrow$ /i/    
\z 

\noindent This defines \textit{i} as the elsewhere theme vowel in the whole system. The next is the theme vowel \textit{a}, which is defined by \citet{marvin2002} as the other default theme vowel. It is the spellout of all theme vowels that have the unmarked negative value of the diacritic feature [$\alpha$].

\ea \label{ex:VIA}
       TV\textsubscript{[-$\alpha$]}  $\leftrightarrow$   /a/
\z 

\noindent Now we arrive at the vocabulary items for the two classes which have a shift to the elsewhere theme vowel in nominalisations.

\ea 
\label{ex:VIrest}
       
         TV\textsubscript{[+$\alpha$, -$\beta$]}  $\leftrightarrow$    /e/ \textbackslash \hspace{0.3cm} \_ \textsc{[voice]}\\
         TV\textsubscript{[+$\alpha$, +$\beta$]} $\leftrightarrow$ /$\emptyset$/    \textbackslash \hspace{0.3cm} \_ \textsc{[voice]} \\
        ...
   
\z 

\noindent These theme vowels have vocabulary items which, apart from the diacritic features, also refer to the presence of an adjacent Voice projection. These vocabulary items are applicable in participles. However, in non-passive nominalisations and adjectivisations, the more general and least specific vocabulary item applies, and \textit{i} gets inserted.



\section{Conclusions and further developments}\label{sec:conclusions}
The paper focused on nominalisations seemingly derived from \textsc{l}-participles, exemplified by the \textit{lec-}nominalisations in Slovenian, in order to determine the nature and position of the \textsc{l}-morpheme. One important point of comparison was the passive participles and the \textsc{n}-morpheme in nominalisations that are derived from passive participles.


We argue that the supposed \textsc{l}-participle nominalisations are not derived from participles in that there is no perfect containment relation between the \textsc{l}-par\-ti\-ci\-ple and the \textit{lec/lac-} nominalisations. Rather, \textit{lec/lac-}nominalisations contain a smaller structure (\textit{v}P) than the corresponding \textsc{l}-participle. This influences the spell out, and consequently, the insertion of theme vowels. In the proposed structure, \textsc{l} (but also \textsc{n/t} in passive participles and related nominalisations) are realisations of a conceptually empty root. 
The structure of nominalisations, however, does include the verbaliser \textit{v} realised by a theme vowel, which is, in the contexts without Voice, realised as \textit{a} or \textit{i}, the latter being the most general Vocabulary Item. 

It was argued that the \textsc{l}-morpheme is a conceptually empty root that can appear both as an inflectional ending and as a derivational affix. This  
multifunctionality of roots is 
an important innovation of our analysis, but, obviously, also the least explored one. While there have been previous analyses of related phenomena for both Slovenian and BCMS \citep[]{SimArs2014,simonovic2017syntax, simonovic2020ov}, we hope that future research will bring new insight as well as an integral theory of the phenomenon.


\section*{Abbreviations}

\begin{tabularx}{.45\textwidth}{@{}lQ}
\textsc{1}&first person\\
\textsc{adj}&adjective\\
BCMS & Bosnian/Croatian/ Montenegrin/Serbian\\
\textsc{aux}&auxiliary\\
\textsc{f}&feminine\\
\end{tabularx}
\begin{tabularx}{.45\textwidth}{lQ@{}}
\textsc{inf}&infinitive\\
\textsc{ipfv}&imperfective \\
\textsc{m}&masculine\\
\textsc{n}&neuter\\
\textsc{nom}&nominative\\
\textsc{pass}&passive\\
\end{tabularx}

\begin{tabularx}{.45\textwidth}{lQ@{}}
\textsc{pfv}&perfective\\
\textsc{pl}&plural\\
\textsc{prs}&present tense\\
\textsc{ptcp}&participle\\
\textsc{sg}&singular\\
\end{tabularx}
\begin{tabularx}{.45\textwidth}{lQ@{}}
Slo&Slovenian\\
\textsc{si}&secondary imperfective\\
\textsc{s.inf}& short infinitive\\
\textsc{tv}&thematic vowel\\
&\\
\end{tabularx}


\section*{Acknowledgments}
 We acknowledge the financial support by the Austrian Science Fund FWF (grants I 4215 and I 6258), by the Slovenian Research and Innovation Agency ARIS (grants N6-0113, J6-4614, P6-0382 and BI-AT/23-24-021), as well as by the Austrian Federal Ministry of Education, Science and Research (BMBWF; Scientific \& Technological Cooperation grant for the project \textit{Verbal suffixes and theme vowels: Different or the same?}).

\printbibliography[heading=subbibliography,notkeyword=this]


\end{document}
