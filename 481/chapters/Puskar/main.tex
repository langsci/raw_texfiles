\documentclass[output=paper,colorlinks,citecolor=brown]{langscibook}
\ChapterDOI{10.5281/zenodo.15394197}
%\bibliography{localbibliography}

\author{Zorica Puškar-Gallien\orcid{0000-0003-4130-0598}\affiliation{Leibniz-Centre General Linguistics}}

% replace the above with you and your coauthors
% rules for affiliation: If there's an official English version, use that (find out on the official website of the university); if not, use the original
% orcid doesn't appear printed; it's metainformation used for later indexing

%%% uncomment the following line if you are a single author or all authors have the same affiliation
\SetupAffiliations{mark style=none}

%% in case the running head with authors exceeds one line (which is the case in this example document), use one of the following methods to turn it into a single line; otherwise comment the line below out with % and ignore it
%\lehead{Puškar-Gallien}
% \lehead{Radek Šimík et al.}

\title[Morphosemantic mismatches with pronouns]{Morphosemantic mismatches with pronouns as a consequence of their internal structure}
% replace the above with your paper title
%%% provide a shorter version of your title in case it doesn't fit a single line in the running head
% in this form: \title[short title]{full title}
\abstract{In addition to differences in their form and position in a sentence, strong pronouns and clitics in Bosnian/Croatian/Montenegrin/Serbian show systematic form-meaning mismatches. Strong pronouns license only animate referents and strict identity readings, whereas clitics show no such restrictions. This paper focuses on two exceptional contexts in which inanimate interpretation and sloppy identity readings are permitted on strong pronouns: focus contexts (acknowledged in previous literature) and prepositional phrases (novel contribution).  The seemingly unrelated properties of pronominal elements can be accounted for under a unified approach to (pro)nominal syntactic structure. I will argue for a hierarchy of nominal projections: base $\succ$ $\phi$-features $\succ$ case, whereby $\phi$-features further split into a hierarchy (person $\succ$ number $\succ$ gender). Under the additional assumption that the pronominal base (\textit{n}P) is a phase, and that it encodes referentiality and individuation features, its absence from the structure (due to deletion) will account for the spell-out of clitics and sloppy identity readings, while the blocking or deletion will allow for the same with strong pronouns in PPs and focus contexts.    

\keywords{pronouns, clitics, phi-features, animacy}
}

\begin{document}
\maketitle

% Just comment out the input below when you're ready to go.
%For a start: Do not forget to give your Overleaf project (this paper) a recognizable name. This one could be called, for instance, Simik et al: OSL template. You can change the name of the project by hovering over the gray title at the top of this page and clicking on the pencil icon.

\section{Introduction}\label{sim:sec:intro}

Language Science Press is a project run for linguists, but also by linguists. You are part of that and we rely on your collaboration to get at the desired result. Publishing with LangSci Press might mean a bit more work for the author (and for the volume editor), esp. for the less experienced ones, but it also gives you much more control of the process and it is rewarding to see the quality result.

Please follow the instructions below closely, it will save the volume editors, the series editors, and you alike a lot of time.

\sloppy This stylesheet is a further specification of three more general sources: (i) the Leipzig glossing rules \citep{leipzig-glossing-rules}, (ii) the generic style rules for linguistics (\url{https://www.eva.mpg.de/fileadmin/content_files/staff/haspelmt/pdf/GenericStyleRules.pdf}), and (iii) the Language Science Press guidelines \citep{Nordhoff.Muller2021}.\footnote{Notice the way in-text numbered lists should be written -- using small Roman numbers enclosed in brackets.} It is advisable to go through these before you start writing. Most of the general rules are not repeated here.\footnote{Do not worry about the colors of references and links. They are there to make the editorial process easier and will disappear prior to official publication.}

Please spend some time reading through these and the more general instructions. Your 30 minutes on this is likely to save you and us hours of additional work. Do not hesitate to contact the editors if you have any questions.

\section{Illustrating OSL commands and conventions}\label{sim:sec:osl-comm}

Below I illustrate the use of a number of commands defined in langsci-osl.tex (see the styles folder).

\subsection{Typesetting semantics}\label{sim:sec:sem}

See below for some examples of how to typeset semantic formulas. The examples also show the use of the sib-command (= ``semantic interpretation brackets''). Notice also the the use of the dummy curly brackets in \REF{sim:ex:quant}. They ensure that the spacing around the equation symbol is correct. 

\ea \ea \sib{dog}$^g=\textsc{dog}=\lambda x[\textsc{dog}(x)]$\label{sim:ex:dog}
\ex \sib{Some dog bit every boy}${}=\exists x[\textsc{dog}(x)\wedge\forall y[\textsc{boy}(y)\rightarrow \textsc{bit}(x,y)]]$\label{sim:ex:quant}
\z\z

\noindent Use noindent after example environments (but not after floats like tables or figures).

And here's a macro for semantic type brackets: The expression \textit{dog} is of type $\stb{e,t}$. Don't forget to place the whole type formula into a math-environment. An example of a more complex type, such as the one of \textit{every}: $\stb{s,\stb{\stb{e,t},\stb{e,t}}}$. You can of course also use the type in a subscript: dog$_{\stb{e,t}}$

We distinguish between metalinguistic constants that are translations of object language, which are typeset using small caps, see \REF{sim:ex:dog}, and logical constants. See the contrast in \REF{sim:ex:speaker}, where \textsc{speaker} (= serif) in \REF{sim:ex:speaker-a} is the denotation of the word \textit{speaker}, and \cnst{speaker} (= sans-serif) in \REF{sim:ex:speaker-b} is the function that maps the context $c$ to the speaker in that context.\footnote{Notice that both types of small caps are automatically turned into text-style, even if used in a math-environment. This enables you to use math throughout.}$^,$\footnote{Notice also that the notation entails the ``direct translation'' system from natural language to metalanguage, as entertained e.g. in \citet{Heim.Kratzer1998}. Feel free to devise your own notation when relying on the ``indirect translation'' system (see, e.g., \citealt{Coppock.Champollion2022}).}

\ea\label{sim:ex:speaker}
\ea \sib{The speaker is drunk}$^{g,c}=\textsc{drunk}\big(\iota x\,\textsc{speaker}(x)\big)$\label{sim:ex:speaker-a}
\ex \sib{I am drunk}$^{g,c}=\textsc{drunk}\big(\cnst{speaker}(c)\big)$\label{sim:ex:speaker-b}
\z\z

\noindent Notice that with more complex formulas, you can use bigger brackets indicating scope, cf. $($ vs. $\big($, as used in \REF{sim:ex:speaker}. Also notice the use of backslash plus comma, which produces additional space in math-environment.

\subsection{Examples and the minsp command}

Try to keep examples simple. But if you need to pack more information into an example or include more alternatives, you can resort to various brackets or slashes. For that, you will find the minsp-command useful. It works as follows:

\ea\label{sim:ex:german-verbs}\gll Hans \minsp{\{} schläft / schlief / \minsp{*} schlafen\}.\\
Hans {} sleeps {} slept {} {} sleep.\textsc{inf}\\
\glt `Hans \{sleeps / slept\}.'
\z

\noindent If you use the command, glosses will be aligned with the corresponding object language elements correctly. Notice also that brackets etc. do not receive their own gloss. Simply use closed curly brackets as the placeholder.

The minsp-command is not needed for grammaticality judgments used for the whole sentence. For that, use the native langsci-gb4e method instead, as illustrated below:

\ea[*]{\gll Das sein ungrammatisch.\\
that be.\textsc{inf} ungrammatical\\
\glt Intended: `This is ungrammatical.'\hfill (German)\label{sim:ex:ungram}}
\z

\noindent Also notice that translations should never be ungrammatical. If the original is ungrammatical, provide the intended interpretation in idiomatic English.

If you want to indicate the language and/or the source of the example, place this on the right margin of the translation line. Schematic information about relevant linguistic properties of the examples should be placed on the line of the example, as indicated below.

\ea\label{sim:ex:bailyn}\gll \minsp{[} Ėtu knigu] čitaet Ivan \minsp{(} často).\\
{} this book.{\ACC} read.{\PRS.3\SG} Ivan.{\NOM} {} often\\\hfill O-V-S-Adv
\glt `Ivan reads this book (often).'\hfill (Russian; \citealt[4]{Bailyn2004})
\z

\noindent Finally, notice that you can use the gloss macros for typing grammatical glosses, defined in langsci-lgr.sty. Place curly brackets around them.

\subsection{Citation commands and macros}

You can make your life easier if you use the following citation commands and macros (see code):

\begin{itemize}
    \item \citealt{Bailyn2004}: no brackets
    \item \citet{Bailyn2004}: year in brackets
    \item \citep{Bailyn2004}: everything in brackets
    \item \citepossalt{Bailyn2004}: possessive
    \item \citeposst{Bailyn2004}: possessive with year in brackets
\end{itemize}

\section{Trees}\label{s:tree}

Use the forest package for trees and place trees in a figure environment. \figref{sim:fig:CP} shows a simple example.\footnote{See \citet{VandenWyngaerd2017} for a simple and useful quickstart guide for the forest package.} Notice that figure (and table) environments are so-called floating environments. {\LaTeX} determines the position of the figure/table on the page, so it can appear elsewhere than where it appears in the code. This is not a bug, it is a property. Also for this reason, do not refer to figures/tables by using phrases like ``the table below''. Always use tabref/figref. If your terminal nodes represent object language, then these should essentially correspond to glosses, not to the original. For this reason, we recommend including an explicit example which corresponds to the tree, in this particular case \REF{sim:ex:czech-for-tree}.

\ea\label{sim:ex:czech-for-tree}\gll Co se řidič snažil dělat?\\
what {\REFL} driver try.{\PTCP.\SG.\MASC} do.{\INF}\\
\glt `What did the driver try to do?'
\z

\begin{figure}[ht]
% the [ht] option means that you prefer the placement of the figure HERE (=h) and if HERE is not possible, you prefer the TOP (=t) of a page
% \centering
    \begin{forest}
    for tree={s sep=1cm, inner sep=0, l=0}
    [CP
        [DP
            [what, roof, name=what]
        ]
        [C$'$
            [C
                [\textsc{refl}]
            ]
            [TP
                [DP
                    [driver, roof]
                ]
                [T$'$
                    [T [{[past]}]]
                    [VP
                        [V
                            [tried]
                        ]
                        [VP, s sep=2.2cm
                            [V
                                [do.\textsc{inf}]
                            ]
                            [t\textsubscript{what}, name=trace-what]
                        ]
                    ]
                ]
            ]
        ]
    ]
    \draw[->,overlay] (trace-what) to[out=south west, in=south, looseness=1.1] (what);
    % the overlay option avoids making the bounding box of the tree too large
    % the looseness option defines the looseness of the arrow (default = 1)
    \end{forest}
    \vspace{3ex} % extra vspace is added here because the arrow goes too deep to the caption; avoid such manual tweaking as much as possible; here it's necessary
    \caption{Proposed syntactic representation of \REF{sim:ex:czech-for-tree}}
    \label{sim:fig:CP}
\end{figure}

Do not use noindent after figures or tables (as you do after examples). Cases like these (where the noindent ends up missing) will be handled by the editors prior to publication.

\section{Italics, boldface, small caps, underlining, quotes}

See \citet{Nordhoff.Muller2021} for that. In short:

\begin{itemize}
    \item No boldface anywhere.
    \item No underlining anywhere (unless for very specific and well-defined technical notation; consult with editors).
    \item Small caps used for (i) introducing terms that are important for the paper (small-cap the term just ones, at a place where it is characterized/defined); (ii) metalinguistic translations of object-language expressions in semantic formulas (see \sectref{sim:sec:sem}); (iii) selected technical notions.
    \item Italics for object-language within text; exceptionally for emphasis/contrast.
    \item Single quotes: for translations/interpretations
    \item Double quotes: scare quotes; quotations of chunks of text.
\end{itemize}

\section{Cross-referencing}

Label examples, sections, tables, figures, possibly footnotes (by using the label macro). The name of the label is up to you, but it is good practice to follow this template: article-code:reference-type:unique-label. E.g. sim:ex:german would be a proper name for a reference within this paper (sim = short for the author(s); ex = example reference; german = unique name of that example).

\section{Syntactic notation}

Syntactic categories (N, D, V, etc.) are written with initial capital letters. This also holds for categories named with multiple letters, e.g. Foc, Top, Num, etc. Stick to this convention also when coming up with ad hoc categories, e.g. Cl (for clitic or classifier).

An exception from this rule are ``little'' categories, which are written with italics: \textit{v}, \textit{n}, \textit{v}P, etc.

Bar-levels must be typeset with bars/primes, not with an apostrophe. An easy way to do that is to use mathmode for the bar: C$'$, Foc$'$, etc.

Specifiers should be written this way: SpecCP, Spec\textit{v}P.

Features should be surrounded by square brackets, e.g., [past]. If you use plus and minus, be sure that these actually are plus and minus, and not e.g. a hyphen. Mathmode can help with that: [$+$sg], [$-$sg], [$\pm$sg]. See \sectref{sim:sec:hyphens-etc} for related information.

\section{Footnotes}

Absolutely avoid long footnotes. A footnote should not be longer than, say, {20\%} of the page. If you feel like you need a long footnote, make an explicit digression in the main body of the text.

Footnotes should always be placed at the end of whole sentences. Formulate the footnote in such a way that this is possible. Footnotes should always go after punctuation marks, never before. Do not place footnotes after individual words. Do not place footnotes in examples, tables, etc. If you have an urge to do that, place the footnote to the text that explains the example, table, etc.

Footnotes should always be formulated as full, self-standing sentences.

\section{Tables}

For your tables use the table plus tabularx environments. The tabularx environment lets you (and requires you in fact) to specify the width of the table and defines the X column (left-alignment) and the Y column (right-alignment). All X/Y columns will have the same width and together they will fill out the width of the rest of the table -- counting out all non-X/Y columns.

Always include a meaningful caption. The caption is designed to appear on top of the table, no matter where you place it in the code. Do not try to tweak with this. Tables are delimited with lsptoprule at the top and lspbottomrule at the bottom. The header is delimited from the rest with midrule. Vertical lines in tables are banned. An example is provided in \tabref{sim:tab:frequencies}. See \citet{Nordhoff.Muller2021} for more information. If you are typesetting a very complex table or your table is too large to fit the page, do not hesitate to ask the editors for help.

\begin{table}
\caption{Frequencies of word classes}
\label{sim:tab:frequencies}
 \begin{tabularx}{.77\textwidth}{lYYYY} %.77 indicates that the table will take up 77% of the textwidth
  \lsptoprule
            & nouns & verbs  & adjectives & adverbs\\
  \midrule
  absolute  &   12  &    34  &    23      & 13\\
  relative  &   3.1 &   8.9  &    5.7     & 3.2\\
  \lspbottomrule
 \end{tabularx}
\end{table}

\section{Figures}

Figures must have a good quality. If you use pictorial figures, consult the editors early on to see if the quality and format of your figure is sufficient. If you use simple barplots, you can use the barplot environment (defined in langsci-osl.sty). See \figref{sim:fig:barplot} for an example. The barplot environment has 5 arguments: 1. x-axis desription, 2. y-axis description, 3. width (relative to textwidth), 4. x-tick descriptions, 5. x-ticks plus y-values.

\begin{figure}
    \centering
    \barplot{Type of meal}{Times selected}{0.6}{Bread,Soup,Pizza}%
    {
    (Bread,61)
    (Soup,12)
    (Pizza,8)
    }
    \caption{A barplot example}
    \label{sim:fig:barplot}
\end{figure}

The barplot environment builds on the tikzpicture plus axis environments of the pgfplots package. It can be customized in various ways. \figref{sim:fig:complex-barplot} shows a more complex example.

\begin{figure}
  \begin{tikzpicture}
    \begin{axis}[
	xlabel={Level of \textsc{uniq/max}},  
	ylabel={Proportion of $\textsf{subj}\prec\textsf{pred}$}, 
	axis lines*=left, 
        width  = .6\textwidth,
	height = 5cm,
    	nodes near coords, 
    % 	nodes near coords style={text=black},
    	every node near coord/.append style={font=\tiny},
        nodes near coords align={vertical},
	ymin=0,
	ymax=1,
	ytick distance=.2,
	xtick=data,
	ylabel near ticks,
	x tick label style={font=\sffamily},
	ybar=5pt,
	legend pos=outer north east,
	enlarge x limits=0.3,
	symbolic x coords={+u/m, \textminus u/m},
	]
	\addplot[fill=red!30,draw=none] coordinates {
	    (+u/m,0.91)
        (\textminus u/m,0.84)
	};
	\addplot[fill=red,draw=none] coordinates {
	    (+u/m,0.80)
        (\textminus u/m,0.87)
	};
	\legend{\textsf{sg}, \textsf{pl}}
    \end{axis} 
  \end{tikzpicture} 
    \caption{Results divided by \textsc{number}}
    \label{sim:fig:complex-barplot}
\end{figure}

\section{Hyphens, dashes, minuses, math/logical operators}\label{sim:sec:hyphens-etc}

Be careful to distinguish between hyphens (-), dashes (--), and the minus sign ($-$). For in-text appositions, use only en-dashes -- as done here -- with spaces around. Do not use em-dashes (---). Using mathmode is a reliable way of getting the minus sign.

All equations (and typically also semantic formulas, see \sectref{sim:sec:sem}) should be typeset using mathmode. Notice that mathmode not only gets the math signs ``right'', but also has a dedicated spacing. For that reason, never write things like p$<$0.05, p $<$ 0.05, or p$<0.05$, but rather $p<0.05$. In case you need a two-place math or logical operator (like $\wedge$) but for some reason do not have one of the arguments represented overtly, you can use a ``dummy'' argument (curly brackets) to simulate the presence of the other one. Notice the difference between $\wedge p$ and ${}\wedge p$.

In case you need to use normal text within mathmode, use the text command. Here is an example: $\text{frequency}=.8$. This way, you get the math spacing right.

\section{Abbreviations}

The final abbreviations section should include all glosses. It should not include other ad hoc abbreviations (those should be defined upon first use) and also not standard abbreviations like NP, VP, etc.


\section{Bibliography}

Place your bibliography into localbibliography.bib. Important: Only place there the entries which you actually cite! You can make use of our OSL bibliography, which we keep clean and tidy and update it after the publication of each new volume. Contact the editors of your volume if you do not have the bib file yet. If you find the entry you need, just copy-paste it in your localbibliography.bib. The bibliography also shows many good examples of what a good bibliographic entry should look like.

See \citet{Nordhoff.Muller2021} for general information on bibliography. Some important things to keep in mind:

\begin{itemize}
    \item Journals should be cited as they are officially called (notice the difference between and, \&, capitalization, etc.).
    \item Journal publications should always include the volume number, the issue number (field ``number''), and DOI or stable URL (see below on that).
    \item Papers in collections or proceedings must include the editors of the volume (field ``editor''), the place of publication (field ``address'') and publisher.
    \item The proceedings number is part of the title of the proceedings. Do not place it into the ``volume'' field. The ``volume'' field with book/proceedings publications is reserved for the volume of that single book (e.g. NELS 40 proceedings might have vol. 1 and vol. 2).
    \item Avoid citing manuscripts as much as possible. If you need to cite them, try to provide a stable URL.
    \item Avoid citing presentations or talks. If you absolutely must cite them, be careful not to refer the reader to them by using ``see...''. The reader can't see them.
    \item If you cite a manuscript, presentation, or some other hard-to-define source, use the either the ``misc'' or ``unpublished'' entry type. The former is appropriate if the text cited corresponds to a book (the title will be printed in italics); the latter is appropriate if the text cited corresponds to an article or presentation (the title will be printed normally). Within both entries, use the ``howpublished'' field for any relevant information (such as ``Manuscript, University of \dots''). And the ``url'' field for the URL.
\end{itemize}

We require the authors to provide DOIs or URLs wherever possible, though not without limitations. The following rules apply:

\begin{itemize}
    \item If the publication has a DOI, use that. Use the ``doi'' field and write just the DOI, not the whole URL.
    \item If the publication has no DOI, but it has a stable URL (as e.g. JSTOR, but possibly also lingbuzz), use that. Place it in the ``url'' field, using the full address (https: etc.).
    \item Never use DOI and URL at the same time.
    \item If the official publication has no official DOI or stable URL (related to the official publication), do not replace these with other links. Do not refer to published works with lingbuzz links, for instance, as these typically lead to the unpublished (preprint) version. (There are exceptions where lingbuzz or semanticsarchive are the official publication venue, in which case these can of course be used.) Never use URLs leading to personal websites.
    \item If a paper has no DOI/URL, but the book does, do not use the book URL. Just use nothing.
\end{itemize}

\section{Introduction}\label{pus:sec:introduction}

The goal of this paper is to develop a formal description of the morphological distinctions, distribution and form-meaning mismatches of pronominal elements in Bosnian/Croatian/Montenegrin/Serbian (BCMS), based on a unified model of the form, locus and function of their $\phi$- and case features. BCMS personal pronouns distinguish between the so-called \textsc{strong pronouns} (pronouns in their full form) and clitics. The main claim that this paper will advance is that some seemingly unrelated properties of pronominal elements, which will be inspected throughout the paper, can be accounted for as a consequence of a unified approach to (pro)nominal syntactic structure, which relies on the key notion of \textsc{hierarchy}.

Pronominal elements in BCMS differ across two dimensions: local person (1\fst{} and 2\nd{} person) vs. 3\rd{} person  pronouns on the one hand, and strong pronouns vs. clitics on the other. Looking at their morphological structure, clitics are morphologically reduced forms of strong pronouns. For instance, the accusative forms of third person singular pronouns are \textit{nje-ga} `\textsc{3.sg.m.acc}', \textit{nje} `\textsc{3.sg.f.acc}', \textit{nje-ga} `\textsc{3.sg.n.acc}', while the corresponding clitics are realised by a portmanteau morpheme expressing gender, number and case, omitting the base \textit{nj(e)-}, i.e  \textit{ga}, \textit{je}, \textit{ga}. On a different dimension, local person pronouns seem to spell out all their phi-features in the form of a portmanteau and their case separately, while third person pronouns spell out the base separately from gender, number and case, resembling lexical nouns and adjectives. 

Strong pronouns have been argued to license only animate referents and strict identity readings whereas clitics show no such restrictions. While the lack of animacy in focus contexts was acknowledged in previous literature, I will present novel data from prepositional phrases which further blur this seemingly sharp divide by demonstrating that strong pronouns in the complement of a P position may in fact be inanimate and license sloppy identity readings. %Local-person pronouns are restricted to human reference. 

This disparate set of distributional properties of pronominal elements in BCMS raises the question whether there is a way to unite them under a single analysis. The first step towards such an analysis requires us to look at the properties outlined above in further detail, which will be the task of \sectref{pus:sec:propertiesofpronominalelements} below. 
The core of the proposal will be based on the claim that the internal structure of a pronoun involves several hierarchies: (i) Within the pronominal extended projection, consisting of a nominal base, followed by $\phi$-feature-encoding projections, followed in turn by case ([Case [$\Phi$ [NP ]]]); (ii) within $\phi$-features \citep{harleyritter02}, such that person precedes number, which itself precedes gender ([gender [number [person ]]]); and (iii) within case features \citep{cahadiss}, which distinguishes between the following types of case -- unmarked (\textsc{nom}) $\succ$ dependent (\textsc{acc}, \textsc{gen}) $\succ$ oblique (\textsc{dat}) $\succ$ prepositional (\textsc{ins}, \textsc{loc}).
I will further propose that these hierarchies are structurally encoded in the syntax \citep{bejarrezac09,vankoppen12}. Distribution of nominal features across them and the locality domains they define will be shown to have consequences on the morphology of pronouns \citep{moskal15}, interpretation and ability to move. In particular, local-person pronouns will differ from third-person pronouns in whether they encode grammatical gender \citep{puskarglossapredicate}; while the former cannot do it, for the latter it is one of their defining properties. Clitics and strong pronouns share the same structure, but clitics crucially lack the NP base. As I will argue, due to the location of features [animate] and [human] on the NP, and their deterministic role in establishing individuation, as well as N's role in establishing reference, the absence of N (modelled as deletion after \citealt{vanurkpronouns}) will allow for certain semantic flexibility which will lead to the possibility of sloppy readings of clitics. 

The paper is structured as follows. \sectref{pus:sec:propertiesofpronominalelements} introduces the pronominal paradigms and morphosemantic mismatches. A short overview of previous literature and certain issues raised from it will be presented in \sectref{pus:sec:previousaccounts}. The proposal on the internal structure of pronominal elements will occupy \sectref{pus:sec:proposal}. Subsequently, \sectref{pus:sec:interpretiveproperties}  will inspect the consequences of the proposal for syntax and interpretation in more detail. \sectref{pus:sec:conclusion} summarises and concludes. 

\section{Properties of pronominal elements in BCMS}\label{pus:sec:propertiesofpronominalelements}

\subsection{Morphological form}\label{subsec:morphologicalform}

An overview of the BCMS personal pronouns and clitics is presented in Table \ref{tableclitics}; clitics are outlined in boldface. First and second person pronouns share the same set of case endings, and realise their base (comprising of $\pi$ (person) and \# (number)) separately from their case features. I will consider the morphemes \textit{-en-}  and \textit{-eb-} in the singular to be the so-called ``support morphemes''  \citep{cardinalettistarke}, which distinguish the strong pronoun forms from their clitic counterparts. The clitic forms of those pronouns are the simple \textit{me} and \textit{te}, without this extension. The base of first person pronouns undergoes suppletion in all non-nominative cases (cf. \textit{ja} vs. \textit{m-} /  \textit{na-}), as well as in the plural, while second person pronouns undergo suppletion in the plural (\textit{ti} vs. \textit{vi}). 		
The third person pronouns' base undergoes suppletion in non-nominative environments, resulting in the \textit{nj(e)-} allomorph. This morpheme is followed by a portmanteau morpheme that realises gender, number and case features, which shares its paradigm with adjectival inflection.  

As for clitics, they are available in genitive, accusative and dative. Local-person clitics spell out the person, number and case features without the support morpheme, whereas third-person clitics amount to the spellout of the gender, number and case suffix, without the pronominal base \textit{on-}/\textit{nj(e)-}.


\begin{table}[ht]
	\begin{tabular}{l l@{~~}l@{~~}l@{~~}l @{\qquad} l@{~~~}l@{~~~}l}
 \lsptoprule
		&  \textsc{1sg}                                  			& \textsc{2sg}                  			& \textsc{1pl}				&\textsc{2pl} 				& \textsc{3sgm/n} 				& \textsc{3sgf} 		& \textsc{3pl} \\\midrule
	    \textsc{nom} & \textit{ja} 							& \textit{ti}   					& \textit{mi}     			& \textit{vi}				&	\textit{on}-$\emptyset$/-\textit{o}  	& \textit{on-a}        			& \textit{on-i}/-\textit{e}/-\textit{a}   \\
		\textsc{gen} & \textit{m}-\textcolor{Gray}{\textit{en}}-\textit{e} 	& \textit{t}-\textcolor{Gray}{\textit{eb}}-\textit{e}    & \textit{na}-\textit{s}  		&\textit{va}-\textit{s}  		&	\textit{nje}-\textit{ga} 		& \textit{n}\textit{j}-\textit{e} & \textit{nj}-\textit{ih}    \\
		\textsc{dat} & \textit{m}-\textcolor{Gray}{\textit{en}}-\textit{i}  	& \textit{t}-\textcolor{Gray}{\textit{eb}}-\textit{i} 	& \textit{na}-\textit{m}\textit{a}    & \textit{va}-\textit{m}\textit{a}   	& \textit{nje}-\textit{mu} 			& \textit{n}\textit{j}-\textit{oj} & \textit{nj}-\textit{im}\textit{a}\\
		\textsc{acc} & \textit{m}-\textcolor{Gray}{\textit{en}}-\textit{e}      & \textit{t}-\textcolor{Gray}{\textit{eb}}-\textit{e} & \textit{na}-\textit{s} 			& \textit{va}-\textit{s} 		&	\textit{nje}-\textit{ga} 		& \textit{n}\textit{j}-\textit{u} & \textit{nj}-\textit{ih}   \\
		\textsc{ins} & \textit{m}-\textcolor{Gray}{\textit{n}}-\textit{om}      & \textit{t-}\textcolor{Gray}{\textit{ob}}-\textit{om} & \textit{na-ma}   			& \textit{va-ma} 			& \textit{nj-im} 				 & \textit{nj-om}     		 & \textit{nj-ima }   \\
		\textsc{loc} & \textit{m}-\textcolor{Gray}{\textit{en}}-\textit{i}      & \textit{t}-\textcolor{Gray}{\textit{eb}}-\textit{i}     & \textit{na-ma}   			 & \textit{va-ma}   			& \textit{nje-mu} 				& \textit{nj-oj} 		& \textit{nj-ima}    \\
		\lspbottomrule
	\end{tabular}
	\caption{Strong pronouns vs. clitics in BCMS}\label{tableclitics}
\end{table}  

\subsection{Restrictions on reference}\label{subsec:restrictionsonreference}

\subsubsection{Animacy}\label{subsubsec:animacy}

As noted in previous literature (e.g. \citealt{despic11}), a clitic can be interpreted as referring to either an animate (or rather human), or an inanimate referent, in contrast to a strong pronoun, which can only be interpreted as denoting a human entity.  

\ea \label {humannonhuman}\textit{Clitics vs. pronouns, animacy/humanness  \citep[240]{despic11}}
\ea{\gll  Čuo sam \fbox{je}.\\
heard.\textsc{m.sg} \textsc{aux.1.sg} \fbox{\textsc{cl.3.f.sg.acc}}\\
\glt `I heard her/it.' \hfill [+\textsc{hum}] [-\textsc{hum}]}
\ex {\gll Čuo sam \fbox{nju}.\\
heard.\textsc{m.sg} \textsc{aux.1.sg} \fbox{\textsc{3.f.sg.acc}}\\
\glt `I heard her.' \hfill [+\textsc{hum}] *?[-\textsc{hum}]}
\z \z

\noindent Exceptions to this generalization have been shown to appear in prepositional phrases and focus contexts. Specifically, in a PP, it is not possible to realise a clitic, instead a strong pronoun is necessary \REF{cliticpronounPP} (as also discussed by \citealt{abels-phases,milicevbeslin19}).\footnote{\label{footnotetripartitedivision}See \citet{stegovec19} for a tripartite distinction between Slovenian strong, clitic and P-pronouns, present in earlier stages of BCMS.}

\newpage
\ea \label{cliticpronounPP}\textit{Clitics vs. pronouns in a PP}\\
\gll Slavica kupuje poklon za \fbox{njega/nju/*ga/*ju}.\\ 
Slavica buys present for \textsc{3.m.sg.acc}/\textsc{3.f.sg.acc}/\textsc{cl.3.m.sg.acc}/\textsc{cl.3.f.sg.acc}\\
\glt `Slavica is buying a present for him/her.'
\z

\noindent What has, to my knowledge, hitherto escaped closer scrutiny is that such a strong pronoun in a complement of P position can in fact refer to an inanimate entity. The following sentences illustrate this for genitive \REF{ppinanimate1a}, dative \REF{ppinanimate1b}, and accusative case \REF{ppinanimate1c}.

\ea \label{ppinanimate1} \textit{Strong pronouns as complements of P}
\ea {\gll Dok vozi, Ljubica uglavnom koristi svoj \fbox{telefon} za navigaciju, a Tamara se dobro snalazi i [$_\text{PP}$ bez \fbox{njega}].\\
while drives Ljubica mostly uses her phone.\textsc{m.sg} for navigating but Tamara \textsc{refl} good manages and {} without \textsc{3.m.sg.gen}\\
\glt `While driving, Ljubica mostly uses her phone for navigating and Tamara manages well without it.'\hfill  (\textsc{gen}, \textsc{inanim})\label{ppinanimate1a}}
\ex {\gll Jelena mnogo voli svoj novi \fbox{posao}, a Jovana ose\'{c}a izrazitu odbojnost [$_\text{PP}$ prema \fbox{njemu}].\\
Jelena a.lot loves self's new job.\textsc{m.sg} but Jovana feels distinct revulsion {} towards \textsc{3.m.sg.dat}\\
\glt `Jelena likes her new job a lot and Jovana finds it  repulsive.' \\\hfill (\textsc{dat}, \textsc{inanim})\label{ppinanimate1b}}
\ex {\gll Mladen je prošao kroz svoja \fbox{pitanja} za kontrolni, a i Saša je tako{\dj}e prošao [$_\text{PP}$ kroz \fbox{njih}].\\
Mladen is went through self's questions.\textsc{n.pl} for test but and Sasha is also went {} through \textsc{3.n.pl.acc}\\
\glt `Mladen went through his questions for the test and Sasha went through them too.' \hfill (\textsc{acc}, \textsc{inanim})\label{ppinanimate1c}}
\z \z

\noindent Additionally, instrumental and locative strong pronouns (those without clitic counterparts),  show the same behaviour. This has also been noted for Slovenian by \citet{stegovec19}, and can be illustrated by the examples in \REF{instrumentallocative}. By analogy with \REF{ppinanimate1}, I will use this to argue that instrumental and locative are in fact PPs in BCMS.

\newpage
\ea \label{instrumentallocative}\textit{Strong pronouns in instrumental and locative}
\ea {\gll Slavica uglavnom putuje bez svog velikog \fbox{ruksaka}, a Jovan obavezno putuje [$_\text{PP}$ s \fbox{njim}].\\
Slavica mostly travels without self's big backpack.\textsc{m.sg} but Jovan necessarily travels {} with \textsc{3.f.sg.ins} \\
\glt `Slavica mostly travels without her big backpack, but Jovan necessarily travels with it.' \hfill (\textsc{ins}, \textsc{inanim})}
\ex {\gll Lena se rado igra u svojoj \fbox{sobi}, a Matija samo uči [$_\text{PP}$ u \fbox{njoj}].\\
Lena \textsc{refl} gladly play in self's room.\textsc{f.sg} but Matija only studies {} in \textsc{3.f.sg.loc}\\
\glt `Lena likes to play in her room and Matija only studies in it.' \\\hfill (\textsc{loc}, \textsc{inanim})}
\z \z 

\noindent Finally, if a strong pronoun is marked as discourse prominent by focus or topicalisation, it may also be inanimate. The following example illustrates this for a focused pronoun. Compare \REF{focus1} to \REF{humannonhuman} above. 

\ea \label{focus1} \textit{Focused inanimate pronoun \citep[246]{despic11}}\\
\gll Čuo sam čak i \fbox{nju}. \\
heard.\textsc{m.sg} \textsc{aux.1.sg} even and \textsc{3.f.sg.acc}\\
\glt `I heard even it (lit. her).' \hfill [+\textsc{hum}] [-\textsc{hum}]
\z

\noindent It should also be noted that strong pronouns referring to inanimate entities can appear in argument positions even without focus particles, but in this case they normally introduce a contrastive topic, cf. \REF{topicalinanimate}. The generalisation however remains that information structure properties facilitate inanimate interpretations of strong pronouns. 

\ea \label{topicalinanimate} \textit{Topical inanimate pronoun}\\
\gll Ovo je moj novi bicikl. Njega su mi poklonili roditelji za ro\dj{}endan.\\
this is my new bicycle \textsc{3.m.sg.acc} \textsc{aux.3.pl} \textsc{cl.1.sg.dat} given parents for birthday\\
\glt `This is my new bicycle. It was given to me by my parents for my birthday.'
\z

\subsubsection{Sloppy identity readings}\label{subsubsec:sloppyreadings}

Another property that distinguishes strong pronouns from clitics in BCMS is their ability to function as bound variables. Specifically, while strong pronouns may only strictly refer to their antecedent, clitics can license sloppy identity readings (in addition to strict ones).\footnote{The discussion here is restricted to third-person clitics.} According to \citet{franks13}, factors that affect the availability of sloppy identity readings include animacy, modification of the antecedent and regional variant, however \citet{runic14} argues that all that is necessary is the appropriate context, e.g. \REF{sloppy1} (see also \citealt{ruda21pronounstructure,ruda21sloppy} for Polish). Note that examples \REF{runica}--\REF{runicb} may not seem to be entirely parallel, due to the second position requirement on the clitic placement, however see \sectref{subsec:prnounmovement} for further detail.\footnote{The context for sloppy reading in \REF{sloppy1} as suggested by \citet[123]{runic14} is the following: `Nikola and Danilo are cousins who live in two different cities in Serbia. Specifically, Nikola lives in Belgrade, while Danilo lives in Niš. They are both five years old and their parents take them to circus performances whenever a circus is in town. A circus is in both Belgrade and Niš at the same time. Both Nikola and Danilo saw an interesting clown in the circus, albeit not the same one'.}


\ea \label{sloppy1}\textit{Clitics vs. pronouns regarding sloppy readings}
\ea {\gll Nikola je vidio zanimljivog klovna, a vidio \fbox{ga} je i Danilo. \\
Nikola \textsc{aux.3.sg} saw interesting clown and saw \textsc{cl.3.sg.m.acc} \textsc{aux.3.sg} and Danilo\\
\glt `Nikola saw an interesting clown and Danilo saw him/one too.'\\
({\langscicheckmark} Nikola saw an interesting clown and Danilo saw him ($=$the same clown that Nikola saw))\\
({\langscicheckmark} Nikola saw an interesting clown and Danilo saw one ($=$a different clown from Nikola's.)}\label{runica}
\ex {\gll Nikola je vidio zanimljivog klovna, a \fbox{njega} je vidio i Danilo.\\
Nikola \textsc{aux.3.sg} saw interesting clown and \textsc{3.sg.m.acc} \textsc{aux.3.sg} saw and Danilo\\
\glt `Nikola saw an interesting clown, and Danilo saw him/*one too.'\\
({\langscicheckmark} Nikola saw an interesting clown and Danilo saw him ($=$the same clown that Nikola saw).)\\
({\langscicross} Nikola saw an interesting clown and Danilo saw one ($=$a different clown from Nikola's).)\hfill \citep[123-124]{runic14}}\label{runicb}
\z \z

\noindent A novel observation I put forward is that BCMS strong pronouns in complement of P position may also allow for sloppy readings, as the examples repeated in \REF{ppsloppy1} show. Example \REF{ppsloppy1a} illustrates this for genitive case, \REF{ppsloppy1b} for dative, and \REF{ppsloppy1c} for accusative. 

\ea \label{ppsloppy1} \textit{Sloppy readings of strong pronouns as complements of P}
\ea {\gll Dok vozi, Ljubica uglavnom koristi svoj \fbox{telefon} za navigaciju, a Tamara se dobro snalazi i [\textsubscript{PP} bez \fbox{njega}].\\
while drives Ljubica mostly uses her phone.\textsc{m.sg} for navigating but Tamara \textsc{refl} good manages and {} without \textsc{3.sg.m.gen}\\
\glt `While driving, Ljubica mostly uses her phone for navigating and Tamara manages well without Ljubica's phone/Tamara's phone.'}\label{ppsloppy1a}
\ex {\gll Jelena mnogo voli svoj novi \fbox{posao}, a Jovana ose\'{c}a izrazitu odbojnost [\textsubscript{PP} prema \fbox{njemu}].\\
Jelena a.lot loves self's new job.\textsc{m.sg} but Jovana feels distinct revulsion {} towards \textsc{3.m.sg.dat}\\
\glt `Jelena likes her new job a lot and Jovana finds it (Jelena's job/Jovana's job) repulsive.'}\label{ppsloppy1b}
\ex {\gll Mladen je prošao kroz svoja \fbox{pitanja} za kontrolni, a i Saša je tako{\dj}e prošao [\textsubscript{PP} kroz \fbox{njih}].\\
Mladen is went through self's questions.\textsc{n.pl} for test but and Sasha is also went {} through \textsc{3.n.pl.acc}\\
\glt `Mladen went through his questions for the test and Sasha went through them (Sasha's/Mladen's questions) too.'}\label{ppsloppy1c}
\z \z

\noindent The same holds for instrumental and locative, as repeated in \REF{ppsloppy2}.

\ea \label{ppsloppy2} \textit{Sloppy readings of strong pronouns in instrumental and locative}
\ea {\gll Slavica uglavnom putuje bez svog velikog \fbox{ruksaka}, a Jovan obavezno putuje [\textsubscript{PP} s \fbox{njim}].\\
Slavica mostly travels without self's big backpack.\textsc{m.sg} but Jovan necessarily travels {} with \textsc{3.f.sg.ins} \\
\glt `Slavica mostly travels without her big backpack, but Jovan necessarily travels with it (Slavica's/Jovan's backpack).'} 
\ex {\gll Lena se rado igra u svojoj \fbox{sobi}, a Matija samo uči [\textsubscript{PP} u \fbox{njoj}].\\
Lena \textsc{refl} gladly play in self's room.\textsc{f.sg} but Matija only studies {} in \textsc{3.f.sg.loc}\\
\glt `Lena likes to play in her room and Matija only studies in it (Lena's/Matija's room).' }
\z \z

\noindent The sentences in \REF{ppsloppy1}--\REF{ppsloppy2} were included in an informal survey, completed by 35 native speakers, recruited through the online community (a Facebook group) \textit{Kako biste VI rekli?} `How would YOU say?'. Based on a short context, the participants were asked to rate the sentence (thus probing the acceptance of animacy restrictions) and choose the appropriate interpretation in a multiple-choice task (choice between the strict and the sloppy interpretation, or both). For instance, \REF{ppsloppy1a} received an overall rating of 4/5 and 25/35 speakers chose the sloppy identiy reading as the preferred interpretation. This confirms that the context plays a big role, but so does the sentence structure. A more formal and balanced further study is planned in order to confirm and elaborate on these results, considering additional factors such as the position of the PP. Nevertheless, the fact that BCMS speakers accept sloppy identity readings of strong pronouns in this context indicates that the divide between strong pronouns and clitics may not be as sharp as is normally drawn, which any theory that models them should be able to account for.  

\subsubsection{Information structure}\label{subsubsec:informationstructure}

An additional distinction between strong pronouns and clitics in BCMS associates strong pronouns with focus, and clitics with topical interpretation. In BCMS, only strong pronouns may express new-information or contrastive focus (or require an antecedent that carries focus, see \citealt{despic11,jovovic22}), as illustrated in \REF{pronounsfocus}, where the sentence-final position is normally the one where contrastive focus is introduced.  
	
	\ea \label{pronounsfocus} \textit{Strong pronouns and focus} \\
	Who did you see?
	\ea[\#] {\gll Video sam \fbox{ga}.\\
	seen.\textsc{m.sg} \textsc{aux.1.sg} \textsc{cl.3.m.sg} \\
	\glt `I saw him.'}
	\ex[] {\gll Video sam \fbox{njega}.\\
	seen.\textsc{m.sg} \textsc{aux.1.sg} \textsc{3.m.sg} \\
	\glt `I saw him.' \hfill \citep[245]{despic11}}
\z \z

\noindent Clitics, on the other hand, are topical elements, or require antecedents that express discourse-given information \citep{jovovic22}. If contrastive focus is present, a strong pronoun must be used as in \REF{boundvariablefocus}. Note that \REF{boundvariablefocus} remains ungrammatical even if the clitic is moved to its (expected) second position in the clause \REF{boundvariablefocus2}.

\ea \label{cliticstopic} \textit{Clitics and topicality }
\ea {\gll Svaki predsednik$_{i}$ misli da \fbox{ga$_{i}$/??njega$_{i}$} svi vole.\\
every president thinks that \textsc{cl.3.m.sg.acc}/\textsc{3.m.sg.acc} everyone love\\
\glt `Every president$_{i}$ thinks that everybody loves him$_{i}$.'}
\ex {\gll Svaki predsednik$_{i}$ misli da samo \minsp{\{} njega$_{i}$ / \minsp{*} ga$_{i}$\} svi vole.\\
every president thinks that only {} \textsc{3.m.sg.acc} {} {} \textsc{cl.3.m.sg.acc} everyone love\\
\glt `Every president$_{i}$ thinks that everyone loves only him$_{i}$.'}\label{boundvariablefocus} 
\ex[\minsp{*}] {\gll Svaki predsednik$_{i}$ misli da ga samo svi vole.\\
every president thinks that \textsc{cl.3.m.sg.acc} only everyone love\\
\glt Intended: `Every president$_{i}$ thinks that everyone loves only him$_{i}$.' \\\hfill \citep[243]{despic11}}\label{boundvariablefocus2}
\z \z


\noindent Focus in BCMS requires prosodic prominence, which clitics always lack, which in turn makes them illicit in a focus position.\footnote{See \citet{browne74,zecinkelas91,franksprogovac94,godjevac2000} on clitics lacking prosodic prominence, \citet{godjevac2000} on focus requiring prosodic prominence, and \citet[244]{despic11} on further interactions between the two.} If a focused pronoun allows for inanimate reference as in \REF{focusexample}--\REF{intensifierexample}, \citet[244]{despic11} argues that such a pronoun is merely a clitic that has to be spelled out as a strong pronoun due to the phonological requirements on focused constituents. Such a `camouflaged clitic' \citep[244]{despic11} should also be able to act as a bound variable, as illustrated by \REF{boundvariablefocus} above. 


\ea \label{focusexample} \textit{Focused inanimate pronoun}\\
\gll Čuo sam čak i \fbox{nju}. \\
heard.\textsc{m.sg} \textsc{aux.1.sg} even and \textsc{3.f.sg.acc}\\
\glt `I heard even it/her.' \hfill [+\textsc{hum}] [-\textsc{hum}] \citep[246]{despic11}
\z 

\newpage
\ea \label{intensifierexample} \textit{Focused inanimate pronoun}\\
\gll Malo ko obilazi muzeje oko gradske crkve$_{i}$. Nju$_{i}$ \minsp{*(} samu), opet dnevno poseti oko 50 turista.\\
few who visits museums around city church \textsc{3.f.sg.acc} {} alone again daily visits around 50 tourists\\
\glt `A few people visits museums around the city church. (As for the church itself), an average of 50 tourists visits it a day.'\hfill \citep[247]{despic11}
\z 

\noindent The animacy properties, the ability to be bound and the sloppy readings outlined in \sectref{subsubsec:sloppyreadings} indicate a lack of inherent referentiality of strong pronouns in these contexts. This may be the reason why \citet{cardinalettistarke} treat them as weak pronouns, or why \citet[244]{despic11} treats them as clitics in disguise. 

\section{Theoretical puzzles and their treatment in the literature}\label{pus:sec:previousaccounts}

The data presented above pose several basic questions that a unified theory of pronominal elements should be able to answer. For a start, we would like to know how the morphosyntactic differences between strong pronouns and clitics can be accounted for, while specifying how referential properties of strong pronouns vs. clitics should be modelled. In relation to their referential properties, the question arises how animacy is represented, as well as why clitics allow for sloppy interpretations, and how the exceptions in PPs can be accounted for. This should directly extend to the behaviour of pronouns in focus contexts. 

All of the issues raised here have been discussed in relation to the categorial status of the pronoun by being tied to the debate on whether nominal elements in BCMS project a DP. Specifically, \citet{despic11} and \citet{runic14}, among others, argue that pronouns in BCMS are NPs. Some of their arguments come from pronominal modification, argument ellipsis, the ability of clitics to license sloppy readings, etc. Yet, \citet{beslinNPDP} advocates for a parametrised view of nominal categories in BCMS, under which lexical nouns are NPs, but pronouns are DPs in this language. Part of her argument is based on pronominal modification and the fact that Left-Branch Extraction of a nominal modifier is possible with a lexical NP but
not with a pronoun. As we will see shortly below, using modification of a pronoun as a diagnostic has shown to lead to inconclusive results, which makes the parametrised view require closer scrutiny. Finally, some authors reject the NP/DP distinction as a culprit for the difference in the behaviour of nominal and pronominal elements altogether in BCMS, arguing that factors other than the presence of articles in a language may be employed to explain some of \citeposst{pus:boskovic08} typological generalizations. For instance, \citet{jovovic22} does this for binding and Condition B violations present in BCMS (and absent in languages without articles), showing that the empirical picture is more complex and dependent on factors such as information structure, and not necessarily nominal size. 

One way to resolve this puzzle is to apply tests in order to probe the structure of the pronominal phrase. \citet{dechainewiltschko} argue that this structure can be threefold, namely pronouns may be mere NPs (Pro-NP), or DPs (Pro-DP), or of an intermediate size, which they term Pro-PhiP. Unfortunately, the tests provided in their work prove to be inconclusive for BCMS. For instance, for a pronoun to count as a DP, is should allow modification of the type \textit{we linguists} or \textit{you poor thing}, where the pronoun would be the overt realisation of the D head. BCMS pronouns do allow modification (see \citealt{progovac98,pus:boskovic08,despic11,runic14,arsenijevicnominalstructure,beslinNPDP} for detailed descriptions, as well as \citealt{hohn15} on such constructions in general), as illustrated in \REF{overtmaterialprecede}. 

\ea \label{overtmaterialprecede} \textit{Modified personal pronouns}
\ea {\gll Dobri ti me retko \minsp{\{} zove / zoveš\}.\\
good.\textsc{m.sg} \textsc{2.sg.nom} \textsc{1.sg.acc} rarely {} call.\textsc{3.sg} {} call.\textsc{2.sg}\\
\glt `The good you rarely call(s) me.' \hfill \citep{arsenijevic-honorific}}\label{overtmaterialprecede1}
%\ag. Jaka ja mogu to prevazi\'{c}i.\\
%strong.\textsc{f.sg} I can.\textsc{1.sg} that overcome\\
%`The strong me can overcome this.'
\ex {\gll Ja volim onog tebe kojeg poznajem.\\
\textsc{1.sg.nom} love.\textsc{1.sg} that.\textsc{m.sg} \textsc{2.sg.acc} who know.\textsc{1.sg}\\ 
\glt `I love that you that I know.'\hfill \citep[28]{pereltsvaig07}}\label{overtmaterialprecede2}
\z \z 

\noindent Nevertheless, as observed by \citet{arsenijevicnominalstructure}, the mere fact that pronouns can be modified in BCMS and in English is insufficient to diagnose the presence or absence of a DP layer. 
\citet[13]{arsenijevicnominalstructure} argues (contra \citealt{pus:boskovic08,runic14})  that even English pronouns can be modified by adjectives (e.g. \textit{Last night's him was so unlike the him that Sepi had first met}). And since they can be preceded by an article, this would indicate that they do not move to D, contrary to \citet{dechainewiltschko}. 
Moreover, \citet{arsenijevic-honorific} recognises that there are semantic restrictions on the adjectives that can modify pronouns, such that only non-restrictive adjectives can combine with pronouns. Adjectives that are used restrictively can combine with pronouns only if the pronouns themselves semantically shift in interpretation, acquiring the interpretation of nouns (i.e. from type $e$ to $\langle e, t \rangle$, as evident in the different agreement possibilities that such a pronoun can license, demonstrated in \REF{overtmaterialprecede1}). 

Furthermore, a Pro-DP behaves as an R-expression, while a Pro-PhiP behaves as a bound variable, which would qualify strong pronouns as DPs and clitics as PhiPs. However we have seen above that strong pronouns may license sloppy readings in PPs and act as bound variables in focus contexts, which would simultaneously make them PhiPs. Finally, according to \citet{dechainewiltschko}, a Pro-DP cannot be used as a predicate, but only as an argument. Clitics in BCMS can only be used as arguments \REF{cliticargument}, which would qualify them as DPs, while strong pronouns can appear in both contexts \REF{cliticargument}--\REF{pronounpredicate}, which would make them Pro-PhiPs. However note that the very claim that DPs cannot function as predicates, put forward by \citet{longobardi94}, and followed by \citet{dechainewiltschko} has been disputed in the literature (see for instance \citealt[21f.]{pereltsvaig07} and references therein for Slavic).\footnote{Cardinaletti and Starke (\citeyear{cardinalettistarke}) argue for a tripartite distinction between strong, weak, and clitic pronouns; their tests are also insufficient -- we could treat argument pronouns as strong and PP pronouns such as those in example \REF{ppinanimate1} as weak (since they allow for inanimate referents, unlike strong pronouns in argument position), but they should also disallow coordination (see \citealt{beslinNPDP} and \citealt{despic11} for discussion and counterexamples).}

\ea {\gll Video sam \minsp{\{} tebe / te\}.\\
see.\textsc{prt.m.sg} \textsc{aux.1.sg} {} \textsc{2.sg.acc} {} \textsc{cl.2.sg.acc}\\
\glt `I saw you.'}\label{cliticargument} 
\z

\ea {\gll Postala sam ti.\\
become.\textsc{prt.f.sg} \textsc{aux.1.sg} \textsc{2.sg.nom}\\
\glt `I became you.'}\label{pronounpredicate}
\z

\noindent There thus seems to be a lack of clear evidence on what category the pronominal elements could be, but more evidence favours their being PhiPs, than DPs. I will thus take an intermediate position, which is on the one hand, that the DP is not crucial to our understanding of the properties of personal pronouns, and on the other, that $\phi$-features are one of their defining properties. As such, the DP will not play a crucial role in our analysis and will be left out of the pronominal structures proposed below (which will also be in line with recent proposals by \citealt{stegovec19,ruda21pronounstructure}, but also the bulk of recent literature on the morphological realisation of pronouns advocated for by \citealt{moskal15}; Smith et al. \citeyear{smithetal18}; \citealt{mcfadden18}). Their PhiP status will prove to be convenient in accounting for the similarities and differences between strong pronouns and clitics. Eliminating the DP will require other ways to deal with their referentiality, but see \citet{trenkic04,branko14,branko-diss} on reference not requiring D in BCMS. The existence of the DP in the structure and its location in relation to other phases will thus not be essential for the analysis. 

Having established that $\phi$-features are a crucial part of pronouns, we may further inquire about their exact structural encoding and relation to case and animacy features. Several works in the literature have tackled this issue, including \citet{progovac98,franks13,despic-hybrid,stegovec19,caha21,ruda21pronounstructure}. Assuming that they are distributed along the nominal spine, the consensus is mostly on a structure that involves an NP, followed by $\phi$-features and case features on top of them, which I will follow, with some adjustments. 
As for animacy and humanness, they are tied to referential/individuation specification and also connected to natural and grammatical gender and number distinction, as well as person, which makes them generally problematic for the Y-model of syntax.
They have been tied to person by \citet{sicheltoosarvandani21wccfl,sicheltoosarvandani22}, or to gender and classifiers by \citet{harleyritter02,puskarsyntax,puskarglossapredicate,arsenijevic21}, or referential index \citep{stegovec19}. Any successful analysis of the data presented above should be able to account for the optionality of animacy on clitics.
 	
In what follows, I aim to provide an account of the properties of pronouns (animacy restrictions and sloppy readings) outlined above that will be based on a unified syntactic structure with well-defined locality domains.

\section{Proposal: The internal structure of pronouns}\label{pus:sec:proposal}

In this section, I will outline a proposal for the internal structure of pronominal categories based on a combination of the feature geometry approach \citep{harleyritter02}, the size of nominal phrase \citep{dechainewiltschko,caha21}, separate encoding of $\phi$-features and predefined locality domains (e.g. \citealt{moskal15,vanurkpronouns}).\footnote{\citet{puskargallienpronouns} offers a proposal on full syntactic decomposition of pronouns and their subfeatures, as well as their morphological realisation in the Distributed Morphology framework, which is why these will be largely put aside in the discussion below.} 

The general idea is that the (pro)nominal phrase consists of three general zones, a lexical one, followed by $\phi$-feature-hosting projections, topped by case-bearing projections ([KP [$\phi$P [NP ]]]). The $\phi$P will be further dissected into a person phrase (PersP), number phrase (NumP)  and a gender phrase (ClassP). Finally, the case phrases will distinguish between unmarked, dependent, oblique, and prepositional case. 

The base of the noun consists of a nominal root and a nominalizing head \textit{n} (see \citealt{kramerbook} and references therein). 
Following the claims of \citet{moskaldiss,moskal15} and \citet{smithetal18} that the pronominal base crucially differs from the one of nouns in lacking a lexical root, I will treat the pronominal \textit{n}P as consisting solely of the categorizing head \textit{n} (\citealt{vanurkpronouns}, building on \citealt{postal69,elbourne05}; but also \citealt{dechainewiltschko, vankoppen12}). 

\subsection{Phi-features and their distribution}\label{subsec:phifeaturesanddistribution}

In analysing the syntactic representation of $\phi$-features, I will rely on the proposal of \citet{harleyritter02}, who argue that $\phi$-features have complex internal structure in the form of hierarchically organised sub-features. Their proposal is reproduced in \figref{fig:hierarchy}. An important aspect of the hierarchy is feature entailment. Having a deeper-embedded feature implies having the feature dominating it. For instance, if a pronoun has the feature [Addressee] from \figref{fig:hierarchy}, it will also contain the feature [Participant]. Such a  structured geometric representation of morphological features, modelled after that of the phonological ones, is claimed to help constrain pronoun and agreement systems and present interdependence of features in a systematic way.

\begin{figure}
\begin{tikzpicture}[>=latex',scale=.9] \tikzset{every tree node/.style={align=center,anchor=north}} 
\Tree [.{Referring Expression} [.Participant [.Speaker ] [.Addressee ] ] [.Individuation [.Group ] [.Minimal Augmented ] [.Class [.Animate [. Masculine ] [.Feminine ] ] [.Inanimate/Neuter ]]]] ; 
\useasboundingbox (current bounding box.north west) rectangle ([yshift=-2.5ex] current bounding box.south east); 
\end{tikzpicture}
    \caption{Structural hierarchy of $\phi$-features \citep[486]{harleyritter02}}
    \label{fig:hierarchy}
\end{figure}

Accounts that distribute these features across the nominal spine have mostly focused on two types of features, person and number, or number and gender (see \citealt{bejarrezac09,vankoppen12,puskarsyntax,puskarglossapredicate,caha21}). I intend to offer a unified proposal for structural encoding of the hierarchy in \figref{fig:hierarchy} within the nominal phrase that includes all the feature types present in it. 

As a starting assumption, I take it that each feature type is hosted by a separate phrase. Taking the incremental bottom-up approach to syntactic structure building very literally, I interpret the root node of the pronoun, one that the entire hierarchy is built on (the ``Referring Expression'' in \figref{fig:hierarchy}), as the \textit{n}P base. This models the idea that \textit{n}P is responsible for the referentiality of the pronoun.\footnote{Precursors for this idea include \citet{caha21}, who models RefP as an additional syntactic projection above the \textit{n}P, albeit without providing much detail on its purpose or interpretation. \citet{sicheltoosarvandani22} use a more abstract $\sigma$P for individuation purposes, while \citet{ruda21pronounstructure,ruda21sloppy} utilizes a PersP. See also \citet{stegovec19}, who employs a (morphologically) empty node \textit{Index} to introduce the referential index on the pronoun. This node is assumed to be higher in the structure.} 

Disagreement in the literature is present not only in the encoding of referentiality, but also in the encoding of individuation (another complex node in the hierarchy in \figref{fig:hierarchy}). Referentiality and individuation are connected such that reference taking and quantification are dependent on individuation (see e.g. \citealt{sicheltoosarvandani21wccfl}), which differentiates nouns from other lexical categories \citep[94-189]{baker03}. Individuation as a property has received different treatments in the literature. While \citet{harleyritter02} separate it from person and make it a precondition for having number and gender features (cf. \figref{fig:hierarchy}), \citet{sicheltoosarvandani21wccfl,sicheltoosarvandani22} employ a separate syntactic projection to encode this property, which to them mediates between person and animacy features and accounts for their interdependence. The locus of animacy is thus also a matter of debate, or rather crosslinguistic variability. It has been related to person (see also, e.g., \citealt{lochbihleretal21}), but also to gender by \citet{foleytoosarvandani22}, or \citet{puskarsyntax,puskarglossapredicate} for BCMS. 

I follow \citet{puskarsyntax,puskarglossapredicate} in assuming that individuation is related to animacy, both of which are a part of \n. \citet{puskarsyntax} integrates animacy into the representation of natural gender, which is argued to be located on \n. Encoding animacy as part of natural gender on \textit{n} (as opposed to morphological gender which is higher in the structure, see below) correctly derives all available, and rules out unavailable patterns of agreement in BCMS such as hybrid agreement and \citeposst{corbett97} Agreement Hierarchy. \citet{puskarglossapredicate} extends this to agreement with honorific pronouns by arguing that animacy is also an integral part of natural number, which is encoded together with natural gender on \n. They are located under a common node, labeled ``\textsc{ind}'', standing for ``individuation''. I will thus assume that individuation (in addition to referential index) is a property encoded on the nominal base. Recall that \citet[94-189]{baker03} claims that individuation and reference taking differentiate nouns from other lexical categories. Distributed Morphology models this difference by building different categories on different categorising heads (and sharing their extended projections). Making \textit{n} responsible for individuation and reference thus models this connection. More concretely, I will assume that individuation is dependent on properties such as [animate] and [human], which can appear as features of the pronominal base.\footnote{\citet{puskargallienpronouns} offers a revision of this model and provides further detail on how animacy and humanness can be encoded on the \textit{n} base.} 

Disassociating individuation from number and gender requires a reorganisation of the hierarchy in \figref{fig:hierarchy} such that it can ultimately be encoded in terms of syntactic phrase structure.
That person features reside lower than number features has been argued by \citet{noyer-diss,trommer02,harbourpersp07,harbour08phi,harbour16,arreginevins12}. Their argument comes from the ordering of person and number affixes, where it was noticed that person affixes strongly tend to be linearised closer to the stem of the word, and number affixes further from them. Under the Mirror Theory \citep{baker-mirror85,brody00,brodyszabolcsi}, this points to a lower base position of person with respect to number. Additionally, under \citeposst{harbour16} theory of person and number encoding, person being introduced higher than number makes wrong predictions for possible and impossible pronoun inventories. 
Following \citet{vanurkpronouns}; Smith et al. (\citeyear{smithetal18}), I assume $\pi$ to be local to the pronominal base. I take person to head its own projection, $\pi$P, above the \textit{n}P, following recent proposals of \citet{ruda21pronounstructure} for Polish and \citet{stegovec19} for Slovenian. Specifically, I assume that 1\fst{} person comprises the features [$\pi$, Participant, Speaker], 2\nd{} person lacks the [Speaker] feature and 3\rd{} person is represented by the person [$\pi$] node alone, as illustrated in figure \figref{fig:1stplpron1} below. 

Number heads a projection further up, which I will label as \#P \citep{picallo91,bernstein-thesis,borer05vol1,acquaviva09,harbour08}. 
Since BCMS has a simple binary number system, it suffices to assume that it includes the generalised feature [\#], which can have a [\textsc{pl}] feature as its dependant. Singular will be treated as the absence of number (\citealt{pus:nevins2011,pesetsky14}; see \citealt{despic-hybrid} for a claim that singular number is unmarked with respect to plural in Serbian). Technically, \#P will be postulated only in case it specifies plural number, i.e. \#P is not projected if the noun is singular \citep{kratzer07}.

Grammatical gender heads its own projection \textsc{cl}(ass)P above \#P. Here, \textsc{class} will be used as mnemonic for gender, which admittedly has more complex structure and whose further modelling is outside of the scope of this paper. I will simply assume that \textsc{cl}P hosts the morphologically realised \textsc{gender}. In locating morphological gender above number I also follow \citet{puskarsyntax,puskarglossapredicate}, who argues that this constellation is indispensable for BCMS in order to derive the variability of agreement patterns found with different nominals. This position of number in between grammatical gender and individuation (in her case natural gender and number) has a blocking effect on agreement, which can derive agreement mismatches of nouns such as \textit{vladika} `bishop', which agree as masculine in the singular (natural gender), but as feminine in the plural (grammatical gender). This way of modeling gender is also a precondition to deriving all other agreement patterns in the language.\footnote{\textsc{gender} as a category can be dispersed across the nominal spine. For the distinctions in encoding grammatical and natural gender see \citet{steriopolowiltschko,pesetsky14,landau15,kucerova17,steriopolo18russianhybrid,steriopolo18Russian,fassifehri18}, but also \citet{arsenijevic21} for an alternative view, and in particular \citet{puskarsyntax,puskarglossapredicate} for arguments why natural gender must be located lower in the structure.}
	
To sum up the discussion thus far, \figref{fig:1stplpron1} presents the proposal for the basic pronominal functional spine in BCMS. I assume that the features themselves are the syntactic heads that project the corresponding phrases. These features can also include a small hierarchy of sub-features below them.\footnote{\label{footnotenaturalgender} One necessary addition to this model is the representation of natural gender on \n. I assume that it additionally involves a feature [\textsc{cl}] and a feature [F] as its dependant. This directly links gender and the features [\textsc{anim}] and [\textsc{hum}]. For instance, nouns of feminine natural gender will involve all of the available nodes in the hierarchy: [\textsc{cl}[\textsc{anim}[\textsc{hum}]][\textsc{f}]], while grammatically feminine nouns will lack the animate and human specification, leaving them with [\textsc{cl}[\textsc{f}]]. Nouns of masculine grammatical gender will only involve the [\textsc{cl}] node, as an unmarked gender feature. Masculine natural gender will involve the [\textsc{anim}] and [\textsc{hum}] features as well, accounting for the general bias in language under which the default referent of human nouns is male 
%(see \citealt{arsenijevicetalhybrid} for a justification of this claim based on experimental evidence). 
Finally, the absence of the [\textsc{cl}] node signals the absence of gender, thereby modelling neuter gender. As such, gender can also participate in agreement, as 1\fst{} and 2\nd{} person pronouns control natural gender agreement.

\iffalse
\begin{figure}
\begin{tikzpicture}[>=latex'] \tikzset{every tree node/.style={align=center,anchor=north}} 
\Tree [.\textsc{cl} [.\textsc{anim} \textsc{hum} ] F ] ; 
\useasboundingbox (current bounding box.north west) rectangle ([yshift=-2.5ex] current bounding box.south east); 
\end{tikzpicture}
\caption{Gender hierarchy in BCMS}
    \label{fig:genderhierarchy}
\end{figure}
\fi

%\renewcommand{\Exlabelsep}{-0.1em}
\ea \label{genderhierarchy}\leavevmode\vadjust{\vspace{-\baselineskip}}\newline
\begin{tikzpicture}[>=latex'] \tikzset{every tree node/.style={align=center,anchor=north}} 
\Tree [.\textsc{cl} [.\textsc{anim} \textsc{hum} ] F ] ; 
\useasboundingbox (current bounding box.north west) rectangle ([yshift=-2.5ex] current bounding box.south east); 
\end{tikzpicture}\\
\z
} 

\begin{figure}
\begin{tikzpicture}[>=latex', scale=1] \tikzset{every tree node/.style={align=center,anchor=north}} 
\Tree [.\textsc{cl}P \scriptsize{\genderdash} [.\#P \scriptsize{\pluraldash} [.$\pi$P \scriptsize{\speakerdash} [.\node(c){\scriptsize{\ndash}}; ]]]]]]] ; 
\useasboundingbox (current bounding box.north west) rectangle ([yshift=-2.5ex] current bounding box.south east); 
\end{tikzpicture}
    \caption{Basic pronominal functional spine in BCMS}
    \label{fig:1stplpron1}
\end{figure}

\subsection{Case features and their distribution}\label{subsec:casefeaturesanddistribution}

Following \citet{bittnerhale96,cahadiss,neelemanszendroi07,moskaldiss,moskal15,smithetal18}, I assume that case is introduced by a separate projection K(P). K can have a complex structure that encodes \citeposst{cahadiss} \textit{Case Hierarchy}:  \newline\textsc{nominative} $\succ$ \textsc{accusative} $\succ$ \textsc{genitive} $\succ$ \textsc{dative} $\succ$ \textsc{instrumental} $\succ$ \textsc{comitative}.  \citet{smithetal18} simplify this somewhat by assuming a distinction between the \textsc{dependent case} (\textsc{dep}; here encompassing \textsc{acc} and \textsc{gen}) and the \textsc{oblique case} (\textsc{obl}, here \textsc{dat}). To this I add the assumption that BCMS also includes two cases that are realised as prepositional phrases, namely \textsc{ins}trumental and \textsc{loc}ative (see \citealt{milicevbeslin19} for instrumental in BCMS; the assumption on locative is straightforward for BCMS, as it is always syncretic with dative and obligatorily preceded by a preposition). 

 	\ea \label{almostcomplete} [$_\text{PP}$ P 
  [$_\text{K$_{\textsc{obl}}$P}$ K$_{\textsc{obl}}$ [$_\text{K$_{\textsc{dep}}$P}$ K$_{\textsc{dep}}$ 
  [$_\text{K$_{\textsc{unm}}$P}$ K$_{\textsc{unm}}$ 
  [$_\text{\textsc{cl}P}$ \textsc{cl} 
  [$_\text{$\#$P}$ \# 
  [$_\text{$\pi$P}$ $\pi$ 
  [$_\text{\textit{n}P}$ \textit{n} ]]]]]]]]
\z

\noindent To the structure above \citet{mcfadden18} adds the proposal that \textsc{nom} is the absence of case (built on \citealt{bittnerhale96,tomsandhya09}, i.a.), which he models as the absence of the case-bearing projection(s). This eliminates K$_{\textsc{unm}}$, leaving nominative pronouns without any case projections.\footnote{Modelling case features closely follows the assumptions from nanosyntax on the containment of case projections. A reviewer notices though that KP layers differ from the other layers in the NP as they are interdependent. In order to streamline the nature of the projections, it can be assumed that KP is projected by the feature [\textsc{dep}], thus KP would only be present when the feature [\textsc{dep}] is. Other case features, such as [\textsc{obl}] may be introduced as sub-features of [\textsc{dep}], such that the case hierarchy is present within the head node on this projection, just like with $\phi$-features. This would model the dependence of oblique case on the dependent case, as well as the absence of case in the nominative. See \citet{barany17book} for a similar approach.} 

\subsubsection{Interim summary}\label{subsubsec:interimsummary}

To sum up, \figref{fig:complete} represents the complete structure of a BCMS nominal  phrase in the most complex case. This provides a way to distribute the \citet{harleyritter02} hierarchy across the pronominal spine (see also \citealt{vankoppen12,fassifehri2000}).

\begin{figure}
\begin{tikzpicture}[>=latex', scale=1] \tikzset{every tree node/.style={align=center,anchor=north}} 
\Tree  [.PP P [.K$_{\textsc{obl}}$P K$_{\textsc{obl}}$  [.K$_{\textsc{dep}}$P K$_{\textsc{dep}}$  [.\textsc{cl}P \textsc{cl}\\{[\textsc{f}]} [.\#P \#\\{[\textsc{pl}]} [.$\pi$P $\pi$\\{[\textsc{prtcpt}][\textsc{spkr}]} [.\node(c){$n$P\\{[\textsc{anim}] [\textsc{hum}]}}; ]]]]]]] ; 
\useasboundingbox (current bounding box.north west) rectangle ([yshift=-2.5ex] current bounding box.south east); 
\end{tikzpicture}
    \caption{Proposed model of the structure of a BCMS pronoun}
    \label{fig:complete}
\end{figure}

\subsection{The representation of pronoun types}\label{subsec:pronounstructures}

The complete structure of a pronoun given in \figref{fig:complete} offers possibilities for parametrisation, as not all pronouns will require all the available nodes. I propose that local-person pronouns lack \textsc{cl}P in general, which models the lack of grammatical gender. Their singular forms also lack \#P. The $\pi$P is projected, since they must have at the minimum the [\textsc{prtcpt}] feature. The structures in Figures \ref{fig:1stsgpronoun}--\ref{fig:1stplpron} represent the local-person pronouns in the nominative case (hence the lack of KP). First person pronouns differ from second person ones in having the additional [\textsc{spkr}] feature.\footnote{A reviewer wonders how local-person pronouns can control gender agreement without having overt grammatical gender features. Recall from \sectref{subsec:phifeaturesanddistribution} and footnote \ref{footnotenaturalgender} that I assume that natural gender is present on the \textit{n}P of local-person pronouns, following \citet{puskarsyntax,puskarglossapredicate}. From there it can enter agreement relations.} 

This structure offers additional possibilities for parametric variation. While BCMS  does not show gender distinctions on local person due to an assumed lack of \textsc{cl}P, Slovenian does contain this phrase and consequently distinguishes feminine (\textit{m-e} `\textsc{1-f.pl}') and masculine (\textit{m-i} `\textsc{1-m.pl}') versions of local person. Notice that Slovenian incidentally offers evidence for ordering person before number and gender, as the gender and number portmanteau follows the person morpheme.\footnote{Alternatively, we may assume grammatical gender to be universally present and that it gets deleted under Impoverishment in local person contexts, as suggested by \citet{noyer-diss} for Arabic, or \citet{despic-hybrid} for Serbian.}

\begin{figure}
\begin{tikzpicture}[>=latex'] \tikzset{every tree node/.style={align=center,anchor=north}} 
	\Tree  [.$\pi$P {$\pi$\\{[\textsc{prtcpt}]([\textsc{spkr}])}} [.\node(c){$n$P\\{[\textsc{anim}][\textsc{hum}]}}; ]]] ; 
	\useasboundingbox (current bounding box.north west) rectangle ([yshift=-2.5ex] current bounding box.south east); 
	\end{tikzpicture}
 \caption{Singular local-person pronoun}
    \label{fig:1stsgpronoun}
\end{figure}

\begin{figure}
\begin{tikzpicture}[>=latex'] \tikzset{every tree node/.style={align=center,anchor=north}} 
\Tree [.\#P {\#\\{[\textsc{pl}]}} [.$\pi$ {$\pi$\\{[\textsc{prtcpt}]([\textsc{spkr}])}} [.\node(c){$n$P\\{[\textsc{anim}][\textsc{hum}]}}; ]]]]]]] ; 
\useasboundingbox (current bounding box.north west) rectangle ([yshift=-2.5ex] current bounding box.south east); 
\end{tikzpicture}
\caption{Plural local-person pronoun}
    \label{fig:1stplpron}   
\end{figure}

The proposed structures for 3\rd{}-person pronouns are presented in Figures \ref{fig:3rdsgpronounM}--\ref{fig:3rdplpronounM}. In the singular, due to the absence of number, their \textit{n}P will be dominated by $\pi$P and \textsc{cl}P, which bears the [\textsc{f}] node for grammatically feminine nouns or just the [\textsc{cl}] node for masculine ones. In the plural, the \textsc{cl}P will be projected above the \#P. The combination of these two phrases will define the inflectional affixes of the pronouns. The \textit{n}P lacks features if the pronoun denotes an inanimate entity. With an animate (or human) referent, these features will be present on the \textit{n}P.

\begin{figure}
\begin{tikzpicture}[>=latex'] \tikzset{every tree node/.style={align=center,anchor=north}} 
\Tree [.\textsc{cl}P {\textsc{cl}\\{([\textsc{f}])}} [.$\pi$P {$\pi$} [.\node(c){$n$P}; ]]]]]]] ; 
\useasboundingbox (current bounding box.north west) rectangle ([yshift=-2.5ex] current bounding box.south east); 
\end{tikzpicture}
\caption{Singular 3\rd{}-person pronoun}
    \label{fig:3rdsgpronounM}
\end{figure}

\begin{figure}
\begin{tikzpicture}[>=latex'] \tikzset{every tree node/.style={align=center,anchor=north}} 
\Tree [.\textsc{cl}P {\textsc{cl}\\{([\textsc{f}])}} [.\#P \#\\{[\textsc{pl}]} [.$\pi$P $\pi$ [.\node(c){$n$P}; ]]]]]]] ; 
\useasboundingbox (current bounding box.north west) rectangle ([yshift=-2.5ex] current bounding box.south east); 
\end{tikzpicture}
    \caption{Plural 3\rd{}-person pronoun}
    \label{fig:3rdplpronounM}
\end{figure}

The system proposed above may be extended straightforwardly to other languages of the Slavic family. As for further extensions to possible and impossible pronominal systems, the proposal would make similar predictions as those made by \citet{harleyritter02} under the assumption that what they call ``activation'' of a particular node is implemented as the presence of that node in the syntax. 
Just like their model, my model keeps person and number features separate, and the variation in pronominal systems depends on the activation of the \mbox{(sub-)}hierarchies of these nodes.
If the two nodes [Participant] and [\#] are activated together, their combination may yield particular types of person, such as those with inclusive/exclusive distinctions. According to them, the presence of particular features in the pronominal hierarchy may be motivated by the presence of a feature in other areas of grammar too. E.g. Pirah\~{a}, Maxakal{\'{i}} and Kwakiutl do not show number distinctions and consequently do not make use of the Individuation node in their hierarchy. Thus in my system a language that makes person and number distinctions would project $\pi$P and \#P, whose sub-nodes would further model distinctions such as inclusive/exclusive, paucal, etc.

As for gender, \citeauthor{harleyritter02} admit that the \textsc{cl} node in their hierarchy would need further modelling and elaboration due to wide crosslinguistic variation in the representation of gender features. They note that ``1\fst{} or 2\nd{} person features should combine freely with any of the number and gender features, since the latter are dependents of a separate organizing node'' \citep[508]{harleyritter02}. Representation of gender across different (lexical and functional) categories, interaction of gender with other $\phi$- and and case features and interaction of gender with animacy and humanness is thus a task under current research that is outside the scope of this paper.\footnote{First steps of further research involve a crosslinguistic study of pronouns that show gender distinctions on local person. So far, I have identified 54 languages with gender on local person, belonging to 18 families and 2 isolates, based on the World Atlas of Language Structures \citep{siewierska13genderpronouns}. My system predicts that in polymorphemic pronouns, gender should follow person and number, and languages that conform to this include Andi, Arabic, Berber, Bora, Djeebbana, Gagadu, Nama, Provencal, Spanish, Lithuanian, Slovenian, Korana. Other candidates to be studied further include Aramaic, Beja, Coptic, Zari, Paez, Sha, Baniata, Dumo, Murui Huitoto and Tunica. This sample should offer further insight into feature entailment relations by identifying patterns of gender encoding and its limitations.}

\subsection{A note on the morphological realisation of strong pronouns vs. clitics}\label{subsec:morphologicalrealisation}

The general intuition that I would like to outline here is that the spell-out rules for local-person pronouns target the base and $\phi$-features together, whereas in third-person pronouns, the base is spelled out separately from the inflectional affixes, cf. Figures \ref{fig:2ndplpron2}--\ref{fig:3rdplpronounM2}. This is what in principle makes third-person pronouns similar to nouns. The spell-out rules will have to be made more precise in order to be able to account for the suppletion patterns presented in \sectref{subsec:morphologicalform}, however this is outside the scope of the current paper.

\begin{figure}
	\begin{tikzpicture}[>=latex',scale=1] \tikzset{every tree node/.style={align=center,anchor=north}} 
	\Tree [.KP K [.\#P \#\\{[\textsc{pl}]} [.$\pi$P $\pi$\\{[\textsc{prtcpt}][\textsc{spkr}]} [.\node(c){$n$P}; ]]]]]]] ; 
	\useasboundingbox (current bounding box.north west) rectangle ([yshift=-2.5ex] current bounding box.south east); 
	\draw[overlay, dashed] (-3,-1.5)..controls +(north west:1.5) and +(north west:1.5).. (2.5,-0.4);
	\node at (2,-0.5) {\small{case}};
	\draw[overlay, dashed] (-1,-3.5)..controls +(north west:1.5) and +(north west:2).. (2.9,-1.3);
	\node at (2.6,-1.6) {\small{base + $\phi$}};
	\end{tikzpicture}
    \caption{Local person}
    \label{fig:2ndplpron2}
\end{figure}

\begin{figure}
\begin{tikzpicture}[>=latex',scale=1] \tikzset{every tree node/.style={align=center,anchor=north}} 
	\Tree [.KP K [.\textsc{cl}P {\textsc{cl}\\{([\textsc{f}])}} [.\#P \#\\{[\textsc{pl}]} [.$\pi$P {$\pi$}  [.\node(c){$n$P}; ]]]]]]] ; 
	\useasboundingbox (current bounding box.north west) rectangle ([yshift=-2.5ex] current bounding box.south east); 
	\draw[overlay, dashed] (2.4,-5.3)..controls +(north west:1) and +(north west:1.5).. (3.5,-4.3);
	\node at (3.5,-4.7) {\small{base}};
	\draw[overlay, dashed] (-2.5,-1.5)..controls +(north west:1.5) and +(north west:1.5).. (2.5,-0.4);
	\node at (2.3,-0.9) {\small{$\phi$ + case}};
	\end{tikzpicture}
    \caption{3\rd{} person}
    \label{fig:3rdplpronounM2}
\end{figure}


We will furthermore see that spelling out \textit{n}P independently, i.e. effectively deleting it, is what enables a certain amount of flexibility to clitics that strong pronouns lack. Under the assumption that the \textit{n}P is a locality domain and as such it is transferred to the interfaces independently of the rest of the structure, the remaining structure is spelled out in the next cycle as a clitic. Figures \ref{fig:2ndplpronclitic}--\ref{fig:3rdplpronounclpl} illustrate the part of the structure that gets realised as a clitic after \textit{n}P deletion. I will build on this below in exploring the syntactic consequences of the given structures. 

\begin{figure}
 	\begin{tikzpicture}[>=latex',scale=1] \tikzset{every tree node/.style={align=center,anchor=north}} 
	\Tree [.KP K\\{\small{\textit{-e}/\textit{-i}}} [.$\pi$P\\{\small{\textit{-m}/\textit{t-}}} ]]]]]] ; 
	\useasboundingbox (current bounding box.north west) rectangle ([yshift=-2.5ex] current bounding box.south east); 
	%	\draw[overlay, dashed] (-3,-1.5)..controls +(north west:1.5) and +(north west:1.5).. (2.5,-0.4);
	%	\node at (2,-0.5) {\small{\textit{-e}/\textit{-i}}};
	%	\draw[overlay, dashed] (-1,-3.5)..controls +(north west:1) and +(north west:2).. (2.9,-1.3);
	%	\node at (3,-1.5) {\small{\textit{-m}/\textit{t-}}};
	\end{tikzpicture}
   \caption{Local person clitic singular}
    \label{fig:2ndplpronclitic}
\end{figure}

\begin{figure}
\begin{tikzpicture}[>=latex',scale=1] \tikzset{every tree node/.style={align=center,anchor=north}} 
	\Tree [.KP K\\{\small{\textit{-m}/\textit{-s}}} [.\#P \#\\{[\textsc{pl}]} [.$\pi$P\\{\small{\textit{na-}/\textit{va-}}} ]]]]]] ; 
	\useasboundingbox (current bounding box.north west) rectangle ([yshift=-2.5ex] current bounding box.south east); 
	%	\draw[overlay, dashed] (-3,-1.5)..controls +(north west:1.5) and +(north west:1.5).. (2.5,-0.4);
	%	\node at (2,-0.5) {\small{\textit{-m}/\textit{-s}}};
	%	\draw[overlay, dashed] (-1,-3.5)..controls +(north west:1) and +(north west:2).. (2.9,-1.3);
	%	\node at (3,-1.5) {\small{\textit{na-}/\textit{va-}}};
	\end{tikzpicture}
    \caption{Local person clitic plural}
    \label{fig:2ndplproncliticpl}
\end{figure}

\begin{figure}
\begin{tikzpicture}[>=latex',scale=1] \tikzset{every tree node/.style={align=center,anchor=north}} 
	\Tree [.KP K [.\textsc{cl}P {\textsc{cl}}   [.\node(c){$\pi$P\\{\small{\textit{-ga}/\textit{-mu}/\textit{-je}/\textit{-joj}/\textit{-ju}}}}; ]]]]]]] ; 
	\useasboundingbox (current bounding box.north west) rectangle ([yshift=-2.5ex] current bounding box.south east); 
	%	\draw[overlay, dashed] (2,-4.5)..controls +(north west:1.3) and +(north west:1.5).. (3.5,-3.3);
	%	\node at (3.2,-3.7) {\small{base}};
	%	\draw[overlay, dashed] (-3,-1.5)..controls +(north west:1.5) and +(north west:1.5).. (2.5,-0.4);
	%	\node at (2.5,-0.5) {\small{\textit{-ga}/\textit{-mu}/\textit{je}/\textit{-joj}/\textit{-ju}}};
	\end{tikzpicture}
    \caption{3\rd{} person clitic singular}
    \label{fig:3rdplpronounclsg}
\end{figure}

\begin{figure}
\begin{tikzpicture}[>=latex',scale=1] \tikzset{every tree node/.style={align=center,anchor=north}} 
	\Tree [.KP K [.\textsc{cl}P {\textsc{cl}} [.\#P \#\\{[\textsc{pl}]} [.\node(c){$\pi$P\\{\small{-ih}/\textit{-im}}}; ]]]]]]] ; 
	\useasboundingbox (current bounding box.north west) rectangle ([yshift=-2.5ex] current bounding box.south east); 
	%	\draw[overlay, dashed] (2,-4.5)..controls +(north west:1.3) and +(north west:1.5).. (3.5,-3.3);
	%	\node at (3.2,-3.7) {\small{base}};
	%	\draw[overlay, dashed] (-3,-1.5)..controls +(north west:1.5) and +(north west:1.5).. (2.5,-0.4);
	%	\node at (2.5,-0.5) {\small{-ih}/\textit{-im}};
	\end{tikzpicture}
    \caption{3\rd{} person clitic plural}
    \label{fig:3rdplpronounclpl}
\end{figure}

To sum up, what unifies strong pronouns and clitics is their internal structure, which can be parametrised. What differentiates strong pronouns from clitics is the presence of the \textit{n}P, such that with clitics it is not realised.

\subsection{Consequences for animacy and referentiality}\label{subsec:animacyand referentiality}

The proposal above has direct consequences for the interpretational properties of pronouns presented in \sectref{subsec:restrictionsonreference}. Since clitics lack the \textit{n}P, and with it the animate and human features, they are in principle compatible with either interpretation. Recall that clitics also behave as bound variables, which allows for sloppy readings and the ability to be bound. Due to the lack of \textit{n}P, they also lack strict reference, and are thus more flexible. 

Before continuing on to the syntactic consequences of this proposal, a comment on the interpretation of $\phi$-features is in order. As interpretable features, $\phi$-features have been widely assumed to trigger presuppositions (\citealt{cooper83,heim08,kratzer09,jacobson12,sudo-diss}). Pronouns carry a referential index which determines their interpretation (e.g. speaker, hearer, participant in a speech act), and $\phi$-features, which are considered to introduce presuppositions to the values provided by the index (see \citealt{sauerland13}). Even though presuppositions triggered by free and bound pronouns may differ in some aspects, they have been subject to unified analyses (see \citealt{sudo-diss,sauerland13}). 

Since I treat animacy as a part of natural gender, I will follow \citet{merchant14}; Murphy et al. (\citeyear{murphyetal-fdslexperiment}); \citet{sudospathas20,arsenijevic21}, all of whom assume that natural gender features trigger presuppositions on the gender of the referent, although they differ in their treatment of grammatical gender (no presuppositions by \citealt{merchant14,murphyetal-fdslexperiment}, presuppositions but no assertions by \citealt{sudospathas20}, or weak presupposition by  \citealt{arsenijevic21}). \citet{arsenijevic21} and \citet{arsenijevicetalhybrid} argue that features like [human] can also be presupposition triggers in BCMS, mostly in conjunction with and in relation to gender. In particular, they argue that [human] contributes to intepretation of gender by triggering a moderate male presupposition (due to cultural bias). In principle, the absence of a gender presupposition (or an assertion thereof) makes a noun compatible with either male or female referents. In the same vein, we can assume that the absence of animacy and humanness information on the \textit{n}P leads to a pronoun's compatibility with both animate and inanimate referents. This would mean that the deletion mechanism proposed below  applies at LF as well. I will leave further formalisation of this for future research and explore some of the technical consequences below.

\section{Consequences for syntax and interpretation}\label{pus:sec:interpretiveproperties}

This section explores the syntactic consequences of the structures proposed above. In particular, I will argue that the availability of sloppy readings of strong pronouns is related to their inability to move out of the PP. \sectref{subsec:prnounmovement}  explores the general properties of movement of (pro)nominal elements, and \sectref{subsec:pronounsinPP}--\sectref{subsec:pronounsinfocus} develop an account on the interactions of this movement with the pronominal structure and its locality domains. 

\subsection{Pronoun movement}\label{subsec:prnounmovement}

Recall that if a pronoun follows a preposition, it can only appear in its strong form, no clitics are allowed, as illustrated above in \REF{cliticpronounPP}. Yet such strong pronouns in the complement of PP show clitic-like behaviour: They may be inanimate and allow for sloppy readings, as illustrated by examples \REF{ppinanimate1}--\REF{instrumentallocative} and \REF{ppsloppy1}--\REF{ppsloppy2} above. I will argue that such clitic-like behaviour of pronouns in this context is due to a ban on movement out of the PP. 

As a starting point, let us examine the general behaviour of (pro)nominal elements in BCMS with respect to movement. Unlike nouns, pronouns in BCMS have been argued to move outside of the VP, as illustrated in \REF{footnote} for pronouns and \REF{pronounmovementb} for nouns. As \citet[3]{beslinNPDP} suggests, a potential context for \REF{footnote} could be something like `When will Mary meet John next?'. A lexical NP may move, with an effect on its interpretation (the moved instance of \textit{Jovan} in \REF{pronounmovementb} is topical, while the postverbal \textit{in-situ} one is new information focus, as reported in \citealt{beslinNPDP}). Clitics in BCMS are also known to undergo movement to the second position in a sentence \REF{pronounmovementc} (see \citealt{boskovic01book,boskovic04,talic18}). 

\ea \textit{Pronoun movement} 
\ea {\gll Marija \minsp{\{} njega\} sre\'{c}e \minsp{\{?*} njega\} svaki dan.\\
Marija {} \textsc{3.m.sg.acc} meets {} \textsc{3.m.sg.acc} every day. \\
\glt `Marija meets him every day.'}\label{footnote} 
\ex {\gll Marija \minsp{\{} Jovana\} sre\'{c}e \minsp{\{} Jovana\} svaki dan.\\
Marija {} Jovan meets {} Jovan every day. \\
\glt `Marija meets Jovan every day.'\hfill (\citealt[307]{stojanovic97}; \citealt{beslinNPDP})}\label{pronounmovementb} 
\ex {\gll Marija \minsp{\{} ga\} sre\'{c}e \minsp{\{*} ga\} svaki dan.\\
Marija {} \textsc{cl.3.m.sg.acc} meets {} \textsc{cl.3.m.sg.acc} every day. \\
\glt `Marija meets him every day.'}\label{pronounmovementc}
\z \z

\noindent Based on the position of the pronoun relative to adverbs and negation, \citet{beslinNPDP} proposes that the landing site of the moved pronoun is somewhere in the middle field, between \textit{v}P and TP \REF{landingsiteb}. Although the movement of clitics is further affected by phonological considerations such as second position in a prosodic word (see \citealt{talic18} and references therein), assuming that clitics behave like pronominal elements, they should be able to move at least as high as strong pronouns otherwise do. Since the exact position to which the pronominal elements move is not crucial for the further discussion, it will be left for further research. 

\ea \textit{Pronoun movement}
\ea {\gll Marko \minsp{(} juče) ni-je \minsp{\{} NJU / nju\} mudro savetovao.\\
Marko {} yesterday \textsc{neg-aux.3.sg} {} \textsc{3.sg.f.acc} {} \textsc{3.sg.f.acc} wisely advise.\textsc{prt.m.sg}\\
\glt `Yesterday, Marko did not advise \{ HER / her\} in a wise manner.'}
\ex {[$_\text{TP}$ yesterday [$_\text{TP}$ \textsc{neg-aux} [$_\text{XP}$ HER/her$_{i}$ [$_\text{vP/VP}$ wisely [$_\text{vP/VP}$ advised t$_{i}$ ]]]]] \hfill \citep[6]{beslinNPDP}}\label{landingsiteb}
\z \z

\noindent Proposals on the trigger for such a movement include semantically-triggered object shift (moving out of the VP to avoid existential closure and receive a definite interpretation; \citealt{stojanovic97}), or categorially-driven movement (pronouns, unlike lexical nouns, are DPs and as such have to move to Spec, AgrOP to check the D-feature, \citealt{beslinNPDP}). 
Although the source of the trigger requires more elaborate research, it seems to me that the most probable explanation is the one that  \citet{beslinNPDP} rejects, namely information structure. Even though in \REF{footnote} it is argued that the interpretation of the pronoun is neutral (under the context assumed by \citeauthor{beslinNPDP}, the pronoun should refer to the topic of the previous discourse), compared to \REF{pronounmovementb}, the strong pronoun still carries some sort of contrastive interpretation. Thus whereas focus might not necessarily be at play, some sort of contrast is definitely involved, as for instance in a contrastive topic. And these may require movement in BCMS. I will leave this issue for further research and come back to it briefly below in \sectref{subsec:pronounsinfocus}.

\subsection{Pronouns in PP position}\label{subsec:pronounsinPP}

\subsubsection{Assumptions}

Having established that pronouns as complements of verbs move from their base position, we may extend this to pronouns in general, including those that are in the complement of P position. However, with the latter this movement will be blocked by the preposition. Below I will argue that this is exactly what leads to inanimate interpretations and sloppy readings in these specific contexts. 
   
I will largely build my account on \citeposst{vanurkpronouns} proposal for pronoun copying, based on pronoun copying in Dinka Bor (Nilotic).\footnote{See also \citet{boskovic01book} for a copy-based account of clitic placement in Serbo-Croatian.}
This language allows constructions in which a pronoun doubles a noun or another pronoun. This poses the challenge of having multiple copies of the same element in a sentence (as for instance in constructions with multiple copies of a verb that has undergone movement, see \citealt{abels01} for Russian, \citealt{landau06} for Hebrew). What is more, a mismatch can happen as in \REF{dinkabor}. Both examples involve an overt copy of a fronted object pronoun, realised as the \textsc{3.pl} \textit{k\^eek}. This pronoun matches the fronted pronoun only partially -- in number, but not in person. 

%This is a version that uses unicode glyphs (and the corrected i vowel)
\ea \label{dinkabor}
\ea {\gll wɔ̂ɔk cí̤i bôl \minsp{\{} kêek / \minsp{*} wɔ̂ɔk\} tí̤iŋ
\\
\textsc{1.pl} \textsc{prf.ov} Bol.\textsc{gen} {} \textsc{3.pl} {} {} \textsc{1.pl} see.\textsc{inf}\\
\glt `Us, Bol has seen.'}
\ex {\gll wêek cí̤i bôl \minsp{\{} kêek / \minsp{*} wêek\} tí̤iŋ\\
\textsc{2.pl} \textsc{prf.ov} Bol.\textsc{gen} {} \textsc{3.pl} {} {} \textsc{2.pl} see.\textsc{inf}\\
\glt `You all, Bol has seen.' \hfill (Dinka Bor; \citealt[940]{vanurkpronouns})}
\z \z

\noindent \Citet{vanurkpronouns} thus needs to account for pronoun movement and multiple-copy spellout. Building on \citet{landau06}, \citeauthor{vanurkpronouns}'s analysis employs the copy theory of movement and a spellout algorithm that enables prononuciation of multiple copies. There are two conditions on copy-spellout, namely recoverability and economy. Recoverability requires that a copy be pronounced if it is associated with phonetic content and economy ensures that as little structure is spelled out as possible, amounting to one copy in a chain (``all unique phonetic content is realised at least once''; \citealt[964]{vanurkpronouns}). Association with phonetic content is met either if an item has its own phonetic content, or if it appears in a position specified with some phonological requirement \citep[31]{landau06}. These two conditions normally ensure that only one copy in a chain is pronounced and the others deleted. The spellout of multiple copies in Dinka is motivated by the peculiarities of phonological requirements related to the EPP features on \textit{v}P and CP edges, which was taken to be a matter of parametric variation.

In a movement chain some copies will undergo full deletion (a precondition on deletion is that a unit must be a phase). For pronouns, \citeauthor{vanurkpronouns} also proposes a so-called \textsc{partial deletion}. The \textit{n}P may be a phase, which is taken to be a cross-linguistic parameter, and as such it can undergo copy deletion independently of the rest of the NP. The deletion operation includes the phase head as well, see \citet[968f.]{vanurkpronouns}. Deleting the \textit{n}P thus leaves the rest of the projections in the pronoun intact, which results in a partial copy, including KP and NumP in his case. Since person information gets deleted together with \textit{n}P (the locus of $\pi$ under his account), the remaining copy need not match in person. In my account below, deleting the \textit{n}P will exactly amount to spelling out a clitic, and I will assume that deleting the \textit{n}P also deletes all of the contents of its sub-hierarchy. 

\subsubsection{Derivation}

Following \citet{vanurkpronouns}, I will assume that pronominal \textit{n}P in BCMS is a phase. I also assume that the target for movement and copying is the KP as in \figref{fig:derivpronmovement}. This ensures that only objects move. The pronoun moves through the edges of phases, stopping (at least) at the \textit{v}P edge.
Such a movement operation may create multiple copies, some of which must be deleted. I posit that the difference in whether we will get a strong pronoun or a clitic depends on the phonological requirements related to their landing sites (e.g. if a pronoun is in a focus position, \textit{n}P will be realised, resulting in a strong pronoun; if it is in a topical position, it will be deleted, resulting in a clitic).
As a result of partial deletion, only the structure between \textit{n}P and the highest KP gets realised, but not the \textit{n}P itself. In my system this amounts exactly to a realisation of a clitic, as illustrated in \figref{fig:derivpronmovement}. 

\begin{figure}
\begin{tikzpicture}[>=latex', scale=1] \tikzset{every tree node/.style={align=center,anchor=north}} 
	\Tree [.TP T [.... [.\node(kk){\fbox{KP}}; K [.\node(c){\fbox{\textsc{cl}}};  {{\textsc{cl}}\\{[\textsc{f}]}}
	[.\node(n){\fbox{\#}}; \node(c){{\#}\\{[\textsc{pl}]}};   
	[.\node(p){\fbox{$\pi$P}}; $\pi$
	[.{\textcolor{gray}{$n$P}} ] ]]]] [.... ... 
	[.$v$P NP$_{sbj}$
	[.$v$\1 $v$ [.VP V [.\node(k){\fbox{KP}}; K [.\node(c){\fbox{\textsc{cl}}};  {{\textsc{cl}}\\{[\textsc{f}]}} 
	[.\node(n){\fbox{\#}}; \node(c){{\#}\\{[\textsc{pl}]}};   
	[.\node(p){\fbox{$\pi$P}}; $\pi$
	[.\fbox{$n$P} ]]]]] ]]]]]] 
	\draw[overlay, semithick,->] (k.west)..controls +(south west:3.5) and +(east:3.5)..node [midway,fill=white] {\ding{172}} (kk.east);
	\draw[overlay, dashed] (2,-7.3)..controls +(north west:1.5) and +(north west:1).. (3,-6.3);
	\node at (3,-7) {\ding{173}};
	\useasboundingbox (current bounding box.north west) rectangle ([yshift=-2.5ex] current bounding box.south east); 
\end{tikzpicture}
    \caption{Pronoun movement, resulting in a clitic (here e.g. \textsc{3.f.pl})}
    \label{fig:derivpronmovement}
\end{figure}

The deletion of the \textit{n}P makes the animacy and humanness features unavailable, leaving the clitic more flexible in terms of its interpretation by virtue of lacking the individuation information. 

\newpage
Applying the process above to pronouns in the complement of PP position will result in the preposition blocking the first step of the process. Assuming that PP is a phase, I will take the cause of the impossibility of extraction to be antilocality \citep{abels-phases,milicevbeslin19}. The moved pronoun would have to pass through the Spec, PP position, which is too short a movement step. This will in turn enforce the spellout of the full pronoun.  

\ea \label{derivpronmovementblockPp} PP blocking movement \\
$[$$_\text{PP}$ <K\tikzmark{p}P> P [\fbox{$_\text{K}$\tikzmark{k}$_\text{P}$} K [\fbox{$_\text{\textsc{cl}P}$} \textsc{cl} [\fbox{$_\text{\#P}$} \# [\fbox{$_\text{$\pi$P}$} $\pi$ [$_\text{\textit{n}P}$ \textit{n} ]]]]]]
\DrawArrowa{p}{k}{\footnotesize {\langscicross}}
\z\medskip

%\vspace{0.5cm}

\noindent As a result, due to an inherent lack of stress on the prepositions under discussion, a clitic remains without a phonological host (see e.g. \citealt{talic18}) or the possibility to move.  The spellout of a strong pronoun may in this case be thought of as a last-resort strategy due to recoverability in order to satisfy the phonological requirements within the PP. As a result, the \textit{n}P must be realised, and exactly in these contexts the pronoun can also be inanimate and have a sloppy reading \REF{ppsloppy1} (i.e. formally a strong pronoun may functionally be a clitic). As an extension, if instrumental and locative are treated as PPs instead of KPs (e.g. \citealt{milicevbeslin19} for instrumental, or \citealt{stegovec19} for Slovenian), the behaviour of their complement pronouns (inanimate reference and sloppy readings, as in other PPs) follows automatically. 

A further benefit of this analysis is that a clitic need not be animate or human, since those features remain stranded on the \textit{n}P base and undergo deletion with it. A clitic may also act as a bound variable since the projections that are responsible for establishing reference are missing (see also \citealt{ruda21pronounstructure,ruda21sloppy} for a claim that PersP is responsible for specificity and definiteness, which is absent in pronouns with a non-specific reading; on reference not requiring D in BCMS, see \citealt{trenkic04,branko14,branko-diss,arsenijevicetalhybrid}). In addition to this, the position of the DP in the structure is not crucial for the analysis.\footnote{The final issue is the nature and timing of the copy-deletion process. \Citet[968]{vanurkpronouns} entertains the possibility that deletion may be seen as non-Transfer, under the assumption that Transfer applies to phasal units (e.g. as in \citealt{foxpesetsky05}). He admits that this view raises an operation-ordering issue in terms of timing of Transfer and copy deletion, as copy deletion would have to precede Transfer, even though it is assumed to be a PF operation. He also admits that there is an issue of how long the copies actually have to stay visible in the derivation in order to evaluate which one in the chain will be spelled out. Adopting this premise would require that deleting the \textit{n}P essentially means that it avoids Transfer to PF and LF. The absence of the features [\textsc{animate}] and [\textsc{human}] would allow for a more flexible interpretation since they cannot trigger presuppositions on the referent. PF would still need to have access to the \textit{n}P somewhat longer though, at least until the next phase head is merged. This would result in the possibility of realising the nP within the PP phase due to recoverability and economy, while the animacy features would be inaccessible.}

\subsection{Pronouns in focus position}\label{subsec:pronounsinfocus}

This section provides a brief discussion on the extensions of the analysis above on pronouns in focus constructions. Recall that in BCMS only strong pronouns may express contrastive focus (or require a focused antecedent), while clitics are topical elements. We assumed above that if a strong pronoun is present in a context where a clitic is usually banned (PPs, focus contexts), such pronouns can be treated as clitics in disguise \citep[244]{despic11}.

Under my proposal, the presence of focus on the pronoun should somehow be able to prevent the deletion of the \textit{n}P or enforce its phonological realisation. Recall from examples \REF{focusexample}--\REF{intensifierexample} from \sectref{subsubsec:informationstructure} that a pronoun can be focused either by being in a particular position in a sentence (e.g. at the beginning or at the end) or by appearing with a particle. In the former case, under the account above, the focus position would impose a PF requirement that the element in this position must carry stress, thus a strong pronoun will be realised, as per recoverability and economy principles. 

If a pronoun appears with an element that carries stress, as in example \REF{intensifierexample} above, one way to implement this technically is to assume that a pronominal phrase may include an additional functional layer, an FP, which may serve as a landing site for the movement of the clitic, as proposed by \citet{vanalem22}. Van Alem justifies this by the existence of nouns with focus particles in Dutch, which can be accounted for under this kind of structure. This FP essentially adds focus to the DP and provides an escape hatch for the clitic to move through. If Spec, FP is already occupied by the focus material, the clitic cannot move out. Instead, it has to be pronounced \textit{in situ}, which has different effects in different Dutch dialects. \citet[217]{despic11} proposes a similar analysis especially for examples like \REF{intensifierexample} which include an overt focus element, such as the intensifier \textit{sam}, although in his account this element projects its own phrase above the nominal projections. See \citet{despic11} for further examples and discussion.

Applied to the case at hand, the specifier of the FP above KP introduces focus material, such as the intensifier \textit{sam} \REF{derivpronmovementblockfp}, which would disable the movement of the KP. As a focus environment, just like a PP, requires a strong pronoun, the \textit{n}P will have to be pronounced as last resort. Note that in the absence of a DP, movement of the KP to Spec, FP would also independently be banned due to antilocality \citep{abels-phases}. 

\ea \label{derivpronmovementblockfp}  FP blocking movement \\
$[$$_\text{FP}$ X\tikzmark{p}P F 
[\fbox{$_\text{K}$\tikzmark{k}$_\text{P}$} K 
[\fbox{$_\text{\textsc{cl}P}$} \textsc{cl} 
[\fbox{$_\text{$\#$P}$} \# 
[\fbox{$_\text{$\pi$P}$} $\pi$ 
[$_\text{\textit{n}P}$ \textit{n} ]]]]]]
\DrawArrowa{p}{k}{\footnotesize {\langscicross}}
\z
\vspace{0.5cm}

\noindent Recall that sometimes it is not strictly focus, but some sort of contrastive interpretation that is also involved in these kinds of structures. I will tentatively assume that such constructions involve the same kind of structure as presented in \REF{derivpronmovementblockfp}, however further research is necessary to establish their exact nature.\footnote{As noted by a reviewer, Slovenian clitics differ from BCMS ones. For instance, they can stand alone as answers to polar questions, and they can carry stress and appear in focus positions (see \citealt{dvorak07} for a full spectrum of variation and peculiar behaviour of Slovenian clitics). I would nevertheless expect them to behave the same in terms of animacy restrictions and sloppy readings, given their clitic status. The locus of variation would lie in the phonological requirements on the realisation of stress, such that in Slovenian it can be carried by the clitic itself, while in BCMS the realisation of the base is unavoidable. On the other hand, Slovenian makes use of a further type of pronouns such as \textit{z\'{a}\_nj} `for him', which make use of the pronominal base in a PP, with a shift of the stress from the base onto the preposition. Note that it is not so clear-cut what portion of structure these pronouns actually involve, since the feminine version is syncretic with the strong pronoun \textit{z\'{a}\_njo} `for her' (P-pronoun) vs. \textit{za nj\'{o}} `for her' (PP). I will leave this issue as an avenue for further extension (\citealt{stegovec19} analyses these as lacking a referential index and the KP layer).} 

\section{Conclusion and outlook}\label{pus:sec:conclusion}

The aim of this paper was to develop a unified model of the form and structure of pronominal elements in BCMS in order to account for a wide set of their distributional properties, including morphological realisation, animacy restrictions, 
ability to function as bound variables, and the distribution in focus (and contrastive) contexts. In addition to presenting an overview of the data available in the literature on these various properties, I have introduced novel data that show that strong pronouns in the complement of PP position may be inanimate, and may allow for sloppy identity readings, contrary to expectation. The data are based on an informal survey, but nevertheless suggestive of the flexibility of the strong pronouns that has previously been overlooked. 

I have argued that the behaviour of strong pronouns in PPs and focus contexts in terms of allowing for animate referents and bound variable interpretations makes them more clitic-like in these contexts. The mismatch between their form and distribution was resolved based on a proposal for their unified syntactic structure and restrictions on morphological realisation, based on a particular theory of pronominal copying. 

One of the main contributions of this paper is the proposal for a decomposed structure of pronominal elements in BCMS, that is applicable to other Slavic languages, but potentially also wider. I have argued that all pronouns are based on an \textit{n}P, followed by $\phi$-feature-bearing projections, such that person is local to the base, number follows it and gender tops them both ([\textsc{cl} [\# [$\pi$ ]]]). These are followed by case-bearing projections, of which the nominative one is missing, and the others encode \textsc{dep}endent case below \textsc{obl}ique one. Crucially for us, the features [\textsc{anim}] and [\textsc{hum}] are encoded on the \textit{n}P, and as such tied to individuation and referential properties of pronouns. 

As a direct consequence, in case that the pronominal base undergoes deletion, the remaining structure becomes more flexible in terms of its interpretation. Specifically, leaving out the \textit{n}P leaves us with a clitic, interpreted as either animate or inanimate, and either sloppy or strict. The deletion of the \textit{n}P was implemented using \citeposst{vanurkpronouns} theory of pronominal copying. A benefit of this analysis was that cases where the \textit{n}P had to be realised due to phonological reasons (PPs and focus/contrastive contexts) were exactly those in which strong pronouns show clitic-like behaviour. Another benefit of the approach is that it allowed us to treat locative and instrumental as PPs in BCMS, based on the parallels in the behaviour of strong pronouns between them and other cases.  

One issue that remains open concerns dative clitics and sloppy readings. In particular, \citet{runic14} notices that in BCMS only accusative clitics allow for sloppy identity readings, while with dative clitics this is impossible. We have however seen that strong pronouns in the complement of a preposition that inherently assigns dative case do not face such a restriction. One way to account for this may be to assume that the K$_{\textsc{obl}}$ phrase functions as some sort of a locality-domain-determining phrase and as such also restricts the interpretation of dative clitics. This issue will be left for further research. In addition to that, the next steps would include validating this proposal based on the data from other Slavic languages, as well as a broader range of crosslinguistic data. 

\section*{Abbreviations}



\begin{tabularx}{.45\textwidth}{@{}lQ}
\textsc{1}&first person\\
\textsc{2}&second person\\
\textsc{3}&third person\\
\textsc{f}&feminine gender\\
\textsc{acc}&accusative\\
\textsc{anim}&animate\\
\textsc{aux}&auxiliary\\
\textsc{cl}&class \\
\textsc{cl}&clitic\\
\textsc{dat}&dative\\
\textsc{gen}&genitive\\
\textsc{hum}&human \\
\textsc{inanim}&inanimate\\
\textsc{inf}&infinitive\\
\textsc{ins}&instrumental\\
\end{tabularx}
\begin{tabularx}{.45\textwidth}{@{}lQ}
\textsc{loc}&locative\\
\textsc{m}&masculine gender \\
\textsc{n}&neuter gender \\
\textsc{neg}&negative\\
\textsc{nom}&nominative\\
\textsc{ov}& Object Voice\\
\textsc{pl}&plural\\
\textsc{prf}&perfect\\
\textsc{prt}&participle\\
\textsc{prtcpt}&participant\\
\textsc{refl}&reflexive \\
\textsc{sg}&singular \\
\textsc{spkr}&speaker\\
$\pi$&person\\
\#&number \\
%&\\ % this dummy row achieves correct vertical alignment of both tables
\end{tabularx}

\section*{Acknowledgments}
I would like to thank Artemis Alexiadou, Karolina Zuchewicz, the members of ZAS's Research Area 3 ``Syntax and Lexicon'' and the audience at FDSL 15 for their valuable comments and feedback. I also appreciate the comments of two anonymous reviewers and the editors, whose insights were of great help in writing the paper.    

\printbibliography[heading=subbibliography,notkeyword=this]

\end{document}
