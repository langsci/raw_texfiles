\documentclass[output=paper,colorlinks,citecolor=brown]{langscibook}
\ChapterDOI{10.5281/zenodo.15394207}
%\bibliography{localbibliography}

%\author{anonymised version}
\author{Ksenia Zanon\affiliation{University of Cambridge}} 
% replace the above with you and your coauthors
% rules for affiliation: If there's an official English version, use that (find out on the official website of the university); if not, use the original
% orcid doesn't appear printed; it's metainformation used for later indexing
%%% uncomment the following line if you are a single author or all authors have the same affiliation
\SetupAffiliations{mark style=none}

%% in case the running head with authors exceeds one line (which is the case in this example document), use one of the following methods to turn it into a single line; otherwise comment the line below out with % and ignore it
\lehead{Zanon}
% \lehead{Radek Šimík et al.}

\title{Wh-indefinites in Russian}
% replace the above with your paper title
%%% provide a shorter version of your title in case it doesn't fit a single line in the running head
% in this form: \title[short title]{full title}
\abstract{The goal of this paper is to chart the expanse of environments that license wh-indefinites in Russian. Primarily a descriptive endeavor, this study provides a more exhaustive empirical coverage of the phenomenon than what has heretofore been documented. Appearing in a proper subset of \textit{nibud'}-licensing contexts, wh-inde\-finites require a clausebounded nonveridical operator and exhibit sensitivity to scalarity. The central analytical import concerns the dichotomy ``clitic" vs. ``non-clitic". Instead of a rigid binary taxonomy, I endorse the view that there is a continuum clitic$\longleftrightarrow$non-clitic, which accommodates elements of transitional flavor. Wh-inde\-finites are just such elements:  not quite clitics proper, they are not full tonic forms either.     

\keywords{wh-indefinites, nonveridicality, clitics, Russian}
}

\begin{document}
\maketitle

\section{Introduction}\label{zan:sec:intro}
Wh-indefinites have the morphological shape of a wh-word and the interpretation of an indefinite. A postverbal \textit{kto} in a polar question (\ref{zan:ex:prisel kto}) functions as an indefinite pronoun in contrast to the sentence-initial one in (\ref{zan:ex:kto prisel}), interpreted as a wh-word.

\ea 
\ea\label{zan:ex:prisel kto}
\gll Prišel kto?\\
came who.{\INDF} \\\hfill YN question
\glt `Did anybody come?' 

\ex\label{zan:ex:kto prisel}
\gll Kto prišel? \\
       who came \\
        \glt `Who came?'
\z\z

\noindent In addition to (\ref{zan:ex:prisel kto}), four other contexts in (\ref{zan:ex:4 contexts}) reportedly enable licensing of wh-indefinites (examples (\ref{zan:ex:esli kto pridet})--(\ref{zan:ex:ctoby kto ne vosel}) appear in \citealt{yan2005}, (\ref{zan:ex:ne poxoze cto kogo videl}) -- in \citealt{heng2018}). However, it turns out that not all subjunctives tolerate wh-indefinites but only those that embed some negative component. Likewise, matrix negation is insufficient by itself: my informants deem (\ref{zan:ex:ne poxoze cto kogo videl}) degraded.\footnote{Data are elicited from five informants on the scale 1--5. Judgments are presented in the following format: `*' = 1, `\textsuperscript{?}*' = 2, `\textsuperscript{??}' = 3, `\textsuperscript{?}' = 4. In controversial cases, I provide all obtained values (e.g., `*/\textsuperscript{?}*'). Positive data are mostly sourced from the national corpus (ruscorpora.ru) or found online. To keep the exposition unencumbered, I indicate the type of source instead of providing a long URL (specifics should be reconstructible via a reverse search). Angle brackets, i.e. `$\langle$ $\rangle$', are used to indicate elicited alternatives in the naturally occurring or reported examples.}

\ea \label{zan:ex:4 contexts}
\ea\label{zan:ex:esli kto pridet}
    \gll Esli 	kto 	pridet, 	pozovi 	menja. \\
    if 	who.{\INDF} 	comes 	call 	me \\\hfill Antecedent of conditional
    \glt `If anybody comes, call me.'
\ex\label{zan:ex: mozet kto prixodil}
\gll Možet, 	kto 	prixodil. \\
    maybe who.{\INDF} 	came \\\hfill Modal adverbs
    \glt `Maybe somebody came.'
\ex \label{zan:ex:ctoby kto ne vosel}
    \gll Petja 	zaper 	dver’, 	čtoby 	kto 	ne 	vošel.  \\
    Petja 	locked 	door 	that.{\SBJV} who.{\INDF} 	{\NEG} 	entered \\ \hfill Subjunctive
    \glt `Peter locked the door, lest somebody enter.'
\ex[(\textsuperscript{?}*)]
{\gll Ne	poxože,	čto Vasja kogo 	uvidel. \\
{\NEG} similar	that	Vasja whom.{\INDF} saw \\\hfill Matrix negation
\glt `It	does	not	look	like	Vasja saw	anybody.'\label{zan:ex:ne poxoze cto kogo videl}}
\z
\z

\noindent A wh-indefinite shares a requirement for a licenser with a (better studied) \textit{nibud'}-indefinite. Neither is possible in past episodic declaratives like (\ref{zan:ex:vcera kto umer}). 

\ea[*] 
{\gll Včera kto-nibud'  kto umer.\\
yesterday who-\textit{nibud'} {} who.{\INDF} died \\
\glt Intended: `Yesterday someone died.'\label{zan:ex:vcera kto umer}}
\z 

 \noindent \textit{Nibud'}-indefinites are morphologically decomposable into a wh-element and an invariable suffix \textit{-nibud'}: e.g., \textit{kto}-\textit{nidud'} `who.{\NOM}-\textit{nibud'}, \textit{čto}-\textit{nidud'} `what-\textit{nibud'}, etc. Roughly, \textit{-nibud'}-indefinites are eligible in nonveridical contexts (questions, conditionals, imperatives, in modal, future and iterative constructions, subjunctives of all flavors, under propositional attitude verbs like \textit{doubt, hope}) as well as the scope and restriction of universal quantifiers (\citealt{Fitzgibbons}, \citealt{paducheva2016}, \citealt{pereltsvaig2008}). In the next section I show that wh-indefinites appear in a proper subset of \textit{nibud'}-environments and identify the conditions that impede or enable the licensing of wh-indefinites.

Before diving in, two short asides are in order. First, a handful of constructions have been excluded from the present consideration on the grounds that the relevant wh-element does not fit the profile of a prototypical indefinite in an obvious way (or if its status is controversial). These include: (i) Modal-Existential configurations (MECs) like (\ref{zan:ex:MEC}) and (ii) two subspecies of relatives in (\ref{zan:ex:relatives}). On MECs, I refer the reader to \citet{vsimik2017existential} for a concise literature overview on the topic.\footnote{But see \citet{vsimik2009hamblin} for arguments that wh-elements in MECs are (Hamblin) indefinites, after all.} Constructions like (\ref{zan:ex:Rudin-style}) were first noted in \citet{rudin2007multiple} for Bulgarian (see also \citealt{caponigro2022semantics}). Correlatives like (\ref{zan:ex:correlatives}) (example provided by Reviewer 1) are discussed in  \citet{Citko2009} with antecedents in \citet{izvorski1996syntax} (for a more general literature overview see \citealt{lin2020correlatives}).\footnote{\citet{belyaev2020genesis} defend the position that the wh-elements in these constructions owe their provenance to indefinites. \citet{arsenijevic2009relative} treats the wh-elements in correlatives as ``extreme non-specific expression(s)''. It is worth pointing out that the latter two analyses view correlatives as subtypes of conditionals.}       

 \ea\label{zan:ex:MEC}
        \gll Mne est' čto gde počitat'.\\
        to.me is what where to.read \\
        \glt `I have something to read.' 
 \z       
 
\ea \label{zan:ex:relatives}
\ea \label{zan:ex:Rudin-style}\gll My otpravili, kto skol'ko naskreb. \\
we sent who how.much scraped.together\\
\glt `We sent however much each scraped together.'

\ex \label{zan:ex:correlatives} \gll Kto kogo uvidit \minsp{(} na večerinke), tot s tem i pozdorovaetsja.\\
who whom will.see {} at party that.one with that.one and will.greet\\
\glt `Whoever sees whomever at the party will greet them.'

\z\z

\noindent The second point concerns the shape of the indefinite itself. In Russian, it need not be a bare wh-word: complex expressions (i.e., \textit{which X}, as in (\ref{zan:ex:adj indf})) are admissible in all the licensing contexts catalogued in the ensuing sections.        

\ea \label{zan:ex:adj indf}\gll Byt' možet zavtra kakoj ukazik sverxu spustjat i togda kotu pod xvost vse ego trudy (...)\\
to.be may tomorrow which.{\INDF} edict from.above will.issue and then to.cat under tail all his labors\\
\glt `It may well be that tomorrow they'll issue some edict from above and then all his labors are for naught.' \hfill{(S. Xabliev. \textit{Povtornye ogni}. 2002)}
\z

\noindent There are, however, gaps in terms of the membership in major ontological categories (`person', `thing', `time', `place', `manner', `reason', etc., on which see \citealt{haspelmath1997indefinite}: 29--31, and references therein). In particular, reason and manner and categories (attempted in (\ref{zan:ex:reason-2}) and (\ref{zan:ex:manner-2}), respectively) prove to be unfit for bare indefinites but open to the \textit{nibud'}-series (as attested by the (a)-examples).

\ea \label{zan:ex:reason}
\ea[] {\gll Esli že počemu-nibud' emu nel'zja budet priexat' ko vremeni moego priezda,...\\
if {\FOC} why-\textit{nibud'} to.him impossible will.be to.come at time of.my arrival\\
\glt `If it would be impossible for some reason for him to come by the time of my arrival,...' \hfill{(P. Tchaikovskii. Letters. 1884.)}\label{zan:ex:reason-1}}
\ex[*] {\gll Esli emu počemu nel'zja budet priexat'...\\
if to.him why.{\INDF} impossible will.be to.come\\
\glt Intended: `If for any reason it would be impossible for him to come...'\label{zan:ex:reason-2}}
\z
\z

\ea \label{zan:ex:manner}
\ea[]  {\gll Možno ėto kak-nibud' ispravit'?\\
possible this how-\textit{nibud'} to.fix\\
\glt `Is it possible to fix it in some way?' \hfill{(beauty forum. 2023)}\label{zan:ex:manner-1}}
\ex[*]  {\gll Možno eto kak ispravit'?\\
possible this how.{\INDF} to.fix\\
\glt Intended: `Is it possible to fix it in any way?'\label{zan:ex:manner-2}}
\z
\z

\noindent Finally, I would be remiss not to point out the crosslinguistic ubiquity of wh-indefinites (\citealt{Gartner2009}), sometimes accommodated under the rubric of ``indeterminate'' pronouns in the literature (\citealt{kratzer2017indeterminate}). A wh-inde\-finite is easy to spot (it looks just like a wh-word), but the environments that render it happy differ across languages: some, like Dutch (\citealt{postma1994indefinite}) or Passamaquoddy (\citealt{bruening2007wh}), do not impose licensing requirements; others, like Chinese, do (see, e.g., \citealt{bruening2007wh, lin2014wh}; a more recent theoretical debate is found in \citealt{chierchia2015, giannakidou2016mandarin, liu2021modal}). Then there is the question of how wh-words and wh-indefinites are related: whether their syncretism is a matter of homophony or homonymy is addressed in \citet{bhat2004pronouns}. All of this is to say that there is a massive volume of scholarly output on the topic, which I cannot hope to address in any detail here. Mine is a case study of the licensing conditions of the Russian wh-indefinite.


\section{Licensing contexts}\label{zan:sec:licensing contexts}
The default licensing requirement for wh-indefinites is the configuration that enables ``epistemic neutrality", understood as Giannakidou's ``nonveridical equilibrium" defined in (\ref{zan:ex:equilibrium}). Such ``prototypical inquisitiveness", i.e. genuine noncommitment of an epistemic agent to one of the polar values in the partitioned information state, arises in neutral Yes / No (YN) questions, conditionals, and under possibility modals (\textit{might}). 

\ea \label{zan:ex:equilibrium}
An information state $W$ is in nonveridical equilibrium iff $W$ is partitioned into $p$ and $\neg p$, and there is no bias towards $p$ or $\neg p$. \\\hfill(\citealt{giannakidou2013inquisitive}: 121)
\z

\noindent Equilibrium might be disrupted in a variety of ways: intonation, tags, adverbs, NPIs, etc. all tilt the balance, inducing the effect of speaker bias. For instance, although \textit{John speaks English, doesn't he?} retains its nonveridical properties, it also supplies an inference that the proposition is true.  

Generally speaking, while \textit{-nibud'}-items are compatible with nearly all nonveridical contexts (whether biased or not), the conditions on wh-indefinites are more stringent. The ``default" licenser must contribute to the representation consistent with epistemic neutrality (\sectref{zan:subsec:default context}). But there are multiple ways to bypass this requirement: by introducing an extrapropositional (epistemic) speech act adverb (\sectref{zan:subsec:SA phenomena}), by integrating an explicit scale whose value is set to be ``less than" the alternatives (\sectref{zan:subsec:scale}), or by embedding the indefinite in the context of ``high" negation (\sectref{zan:subsec:high negation}). The ensuing exposition is best construed as an empirical exercise, designed to fit the novel data into some general theoretical schemes. In other words, I am not necessarily making any analytical commitments -- rather, I am using the existing theoretical apparatus to systematize the facts. Insofar as the proposals acquire an explicit shape (most notably in \sectref{zan:subsec:high negation}), I attempt no exhaustive treatment of the phenomena involved.                   

\subsection{Default contexts}\label{zan:subsec:default context}
Wh-indefinites are robustly attested in conditional antecedents (\ref{zan:ex:esli kto pridet v futbolke}) and embedded (or root) YN questions (\ref{zan:ex:proverjali smogut li kogo}). The affinity between conditionals and questions has been observed by multiple authors (for an overview and further references see \citealt{bhatt2017conditionals}). In fact, wh-indefinites turn up with admirable regularity in precisely these two contexts. Following standard practice, we may assume that the responsible licensing party here is a Q/conditional operator, merged in CP.


\ea
\ea \label{zan:ex:esli kto pridet v futbolke}
\gll Esli	kto / \minsp{$\langle$} kto-nibud'$\rangle$ 	 pridet v futbolke, vygonju!\\
if	who.{\INDF} {} {} who-\textit{nibud'} will.come in t-shirt kick.out\\
\glt  `If somebody shows up in a t-shirt, I'll kick them out!'\hfill (Twitter. 2019)

\ex \label{zan:ex:proverjali smogut li kogo}
\gll \minsp{\dots} proverjali, \minsp{[} smogut li kogo / \minsp{$\langle$} kogo-nibud'$\rangle$ obmanut']?\\
{} checked {} are.able \textsc{q} whom.{\INDF} {} {} whom-\textit{nibud'} to.cheat\\
\glt  `(They) checked whether they would be able to hoodwink \\somebody.'\hfill (M. Semenova. \textit{Volkodav}. 2003)
\z\z

\noindent From (\ref{zan:ex:deontic modals}) and (\ref{zan:ex:epistemic modals}), we glean that wh-indefinites are unhappy in deontic contexts, independent of the quantificational strength of the modal -- universal in (\ref{zan:ex:deontic necessity modals}) or existential in (\ref{zan:ex:deontic possibility modals}). They are, however, compatible with epistemic modality, provided that the modal is of a possibility (\ref{zan:ex:epistemic possibility modals}) rather than of a necessity (\ref{zan:ex:epistemic necessity modals}) variety. 


\ea \label{zan:ex:deontic modals}
\ea \label{zan:ex:deontic necessity modals}
        \gll Ty dolžen s''est'  čto-nibud' / \minsp{*} s''est' čto.\\
        you must.{\MASC.\SG} to.eat  what-\textit{nibud'} {} {} to.eat what.{\INDF} \\
        \glt  `(I am not letting you out hungry). You must eat something.'

\ex \label{zan:ex:deontic possibility modals}
        \gll Možeš’ posmotret’ čto-nibud’ / \minsp{\textsuperscript{?}*} posmotret' čto. \\
        may.2.{\SG} to.watch what\textit{-nibud'}  {} {} to.watch what.{\INDF} \\
        \glt  `(Because you behaved today), you may watch something.'
\z \z


\ea \label{zan:ex:epistemic modals}
\ea \label{zan:ex:epistemic possibility modals}
        \gll Razmery mogut komu / \minsp{$\langle$} komu-nibud'$\rangle$ i prigodit’sja.\\
        dimensions may.{3.\PL} to.whom.{\INDF}  {} {} to.whom\textit{-nibud'} {\FOC} to.be.of.use \\
        \glt  `(I am sharing this information, because) the dimensions might be of use to somebody.' \hfill {(car forum. 2017)}
       
\ex \label{zan:ex:epistemic necessity modals}
        \gll Lekcija dolžna \minsp{\textsuperscript{?}*\textsuperscript{/??}} kogo / kogo-nibud' zainteresovat'.\\
        lecture must.{\FEM.\SG} {} whom.{\INDF} {} whom\textit{-nibud'} to.be.of.interest \\
        \glt  `The lecture must be of interest to somebody (though there are no guarantees of robust attendance).'\\
\z\z

\noindent An enduring generalization that epistemics consistently outscope other sentential operators (including negation and root modals) formed the basis for formulating the analyses under which epistemics occupy a clause-peripheral position, high enough to take the widest scope (for various implementations see \citealt{butler2003minimalist, drubig2001syntactic, cormack2002modals}, a.o.). 

If so, the patterns above conform to the following generalization. A wh-inde\-finite licenser must be merged in a position presumably related to the (split) C-domain, which houses interrogative, conditional and epistemic operators. But there is a further semantic requirement necessary for convergence: given the contrast in (\ref{zan:ex:epistemic modals}), the operator must be compliant with epistemic neutrality. While the possibility modal ensures epistemic equilibrium, the necessity one coaxes a stronger statement -- one that is biased towards $p$. A similar effect is detectable in future contexts like (\ref{zan:ex:zavtra kto sdelaet}). 

Predictive future in \citet{giannakidou2013two} is likewise nonveridical (since the outcome of the future event is unknown), but positively biased (``probably"), since it ``presupposes confidence [of the speaker] that the actual world to come is a $p$ world" (119).  

\ea \label{zan:ex:zavtra kto sdelaet}
    \gll Zavtra kto-nibud' / \minsp{$\langle$} \minsp{*} kto$\rangle$ sdelaet obaldennoe kino [...].\\
        tomorrow  who-\textit{nibud'} {} {} {} who.{\INDF} will.make exciting movie\\
    \glt `Tomorrow somebody will produce an exciting film.’ \hfill{(kinometro.ru. 2012)}
\z

\noindent Clearly, universal epistemics and the future induce a similar effect: they appear to be too strong for wh-indefinites. In Giannakidou (and Mari)'s work this strength (to wit, bias) arises at the ``not-at-issue" (presuppositional) level, defined as the speaker's measure of the likelihood of the event/actual-world-to-come. Epistemic \textit{must} and the future come with a default positive bias, which can be modified by speech act adverbs: \textit{Maybe John will come} expresses less confidence in the occurrence of the future event than its adverb-less counterpart. 

Tampering with the default bias in Russian yields the following results. A YN question with \textit{razve}, an element strictly specialized for non-neutral questions in (\ref{zan:ex:razve kto somnevalsja}), conveys negative bias (i.e., the speaker believes that nobody had doubts about Putin's intentions). Likewise, in a future configuration (\ref{zan:ex:avos' zavtra kto}), the introduction of \textit{avos'} `perhaps, maybe' weakens the statement enough to render the indefinite appropriate in this context.

\ea
\ea \label{zan:ex:razve kto somnevalsja}
    \gll Razve kto somnevalsja, čto Putin ne ujdet na pensiju?\\
    really  who.{\INDF} doubted that Putin {\NEG} leave on pension\\
    \glt  `Did anybody really doubt that Putin wouldn't retire?'  \hfill {(dk.ru. 2020)}

\ex \label{zan:ex:avos' zavtra kto}
    \gll \minsp{*(} Avos') zavtra kto poučastvuet.\\
    {} maybe tomorrow who.{\INDF} will.participate\\
    \glt  `Maybe someone will participate tomorrow.' \hfill {(car forum. 2009)}
\z
\z

\noindent Assuming a skeletal structure in (\ref{zan:ex:default licening}), we may conclude that the indefinites are licensed by an element above TP -- a high modal or an operator in SpecCP, but not by a root modal. Furthermore, this licenser must introduce epistemic equilibrium. In situations when it does not -- i.e. when the default bias is skewed towards a positive proposition -- a weakening adverbial, overriding the default bias, may salvage the configuration (as demonstrated by the contrast between (\ref{zan:ex:zavtra kto sdelaet}) and (\ref{zan:ex:avos' zavtra kto})).     

\ea \label{zan:ex:default licening}
$[$\textsubscript{CP} \textit{Op}\textsubscript{Q/Conditional} [\textsubscript{FP} Mod\textsubscript{Epis/Fut}  [ ... Mod\textsubscript{root} ... ]]]
\z

\noindent As it turns out, however, the adverbial need not induce weaker bias. In the next section I demonstrate that speech act adverbs are legitimate licensers for wh-indefinites, independent of the direction of their bias.  

\subsection{Speech Act adverbs}\label{zan:subsec:SA phenomena}

\citet{yan2005}, enlisting the paradigm in (\ref{zan:ex:Yanovichs adverbials}), concludes that wh-indefinites are not licensed by certain adverbs like \textit{dolžno byt'} `must be'. 

\ea \label{zan:ex:Yanovichs adverbials}
\ea
        \gll Možet, \minsp{\{} kto / \minsp{$\langle$} \minsp{\textsuperscript{\ding{51}}} kto-nibud'$\rangle$\} prixodil. \\
      maybe {} who.{\INDF} {} {} {} who-\textit{nibud'} came.{\MASC}\\
      \glt `Maybe someone came.' 

\ex \label{zan:ex:dolzno byt' kto prixodil}
        \gll Dolžno byt', \minsp{\{*} kto / \minsp{$\langle$} \minsp{\textsuperscript{\ding{51}}} kto-nibud'$\rangle$\} prixodil. \\
       must be {} who.{\INDF} {} {} {} who-\textit{nibud'} came.{\MASC}\\
      \glt `It must be the case that someone came.' \hfill{(\citealt{yan2005})}
\z
\z

\noindent Indeed, (\ref{zan:ex:dolzno byt' kto prixodil}) is bad, but a small adjustment in the word order, as in (\ref{zan:ex:dolzno byt' prixodil kto}), renders the sentence perfectly natural if a bit quaint. I attribute this contrast to PF constraints to be discussed in \sectref{zan:sec:syntax}. For now, it suffices to concede that an adverbial \textit{dolžno byt'}, which, in contrast to the neutral \textit{možet}, introduces a higher degree of speaker confidence, is in principle compatible with wh-indefinites.       

\ea \label{zan:ex:dolzno byt' prixodil kto}
        \gll Dolžno byt', \minsp{\{} prixodil / prišel\} kto \minsp{(} raz takoj porjadok). \\
      must be {} came.{\IPFV} {} came.{\PFV} who.{\INDF} {} since such order\\
      \glt `Someone must've stopped by, given how clean the place is.'
\z

 \noindent Speech act modal adverbs (SpMAs) like \textit{dolžno byt'} and \textit{možet (byt')} are distinct from the agreeing modals encountered in (\ref{zan:ex:deontic modals}) and (\ref{zan:ex:epistemic modals}): The former are adjuncts, the latter are integral to a proposition. SpMAs have an immutable form and appear in the environments with inflected verbs. Agreeing modals carry phi-features and take on the infinitive complements. Furthermore, SpMAs (\textit{probably, perhaps, certainly}, etc.) are said to express subjective modality in contrast to objective modality, routinely encoded by modal adjectives (\textit{[it is] probable, certain, possible}, etc.) (\citealt{ernst2009speaker, krifka2019layers, wolf2014degrees}, a.o.).\footnote{Modal verbs are frequently ambiguous between the two (see, e.g., \citealt{papafragou2006epistemic} and references therein).} The basic intuition here is that SpMAs convey speakers' internal judgment of/commitment to the embedded proposition. This is opposed to some external (objective) assessment of the event's likelihood. There are also more tangible correlates of subjective modality: modal adverbs are deviant in non-assertive environments like (\ref{zan:ex:subjective modal-YN}) and resist negation, as in (\ref{zan:ex:subjective modal-negation}) (cf. the grammatical counterparts with agreeing modals in (b)). \citet{krifka2017assertions, krifka2019layers} provisions a special syntactic position for SpMAs -- one that is external to the core proposition: for him, objective epistemics are associated with TP (hence, proposition-internal, at-issue), while subjective ones relate to the Judgment Phrase, a position above TP (hence, proposition-external, relaying not-at-issue content).                      

\ea 
\ea[*] 
{\gll Pojdet li segodnja, \minsp{\{} možet / dolžno byt'\}, dožd'?\\
will.go \textsc{q} today {} maybe {} must be rain \\
\glt Intended: `Will it (maybe, certainly) rain today?'\label{zan:ex:subjective modal-YN}} 
\ex[] 
{\gll \minsp{\{} Možet / dolžen\} li segodnja pojti dožd'? \\
{} may.2.{\SG} {} must.{\MASC}.{\SG} \textsc{q} today to.go rain \\
\glt  `Might/ must it rain today?'} 
        
\z
\z

\ea  
\ea \label{zan:ex:subjective modal-negation}
   \gll \{(\minsp{*} Ne) možet / (\minsp{*} Ne) dolžno byt'\}, Ivan doma.\\
       {} {\NEG} maybe {} {} {\NEG} must be Ivan home \\
\glt Intended: `Ivan cannot be home.'
 
\ex \gll Oni \minsp{\{} ne mogut byt' / ne dolžny byt'\} doma. \\
        they {} {\NEG} may.2.{\PL} to.be {} {\NEG} must.{\PL} to.be home\\
     \glt  `It is not possible/probable that they are home.' 
        
\z
\z

\noindent If (\ref{zan:ex:epistemic necessity modals}) externalizes objective modality and (\ref{zan:ex:dolzno byt' prixodil kto}) subjective modality, then the requirement for a weaker speaker commitment only holds of the former. The basic insight here is that universal (objective) epistemics, future, and veridical past contexts are too strong for wh-indefinites. But when explicitly tempered at the illocutionary level, these three contexts become just fine for indefinites, as demonstrated by the trio in (\ref{zan:ex:further SA adverbs}) for each environment, respectively. Note that ``tempering" is equivalent to embedding any subjective modification. SpMAs in (\ref{zan:ex:further SA adverbs}) range from weak (\textit{vrjad li}) to neutral (\textit{možet (byt')}) to strong (\textit{očevidno, dolžno byt'}). As elements of epistemic/evidential/inferential flavor, they form a natural class.         

\ea \label{zan:ex:further SA adverbs}
\ea[\textsuperscript{(?)}] 
{\gll \minsp{\{} Vrjad li / edva li\} segodnja gde dolžen pojti dožd'.  \\
{} hardly \textsc{q} {} hardly \textsc{q} today where.{\INDF} must.{\MASC}.{\SG} to.go rain \\
\glt `It is unlikely that it must rain somewhere today.'\label{zan:ex:vrjad li}}
\ex[] 
{\gll Gljadiš', komu i prigoditsja.  \\
       see.2.{\SG} to.whom.{\INDF} {\FOC} will.be.of.use \\
\glt `Perhaps, (it'll) be of use to someone.' \hfill {(multiple sources)}\label{zan:ex:gljadis}}

 \ex[] 
{\gll \minsp{\{} Stalo byt' / očevidno\}, obidel ee kto \minsp{(} raz plačet).  \\
{} come to.be {} obviously hurt.{\PST} her who.{\INDF} {} since cries\\
\glt `Evidently/obviously, someone hurt her, since she is crying.'\label{zan:ex:stalo byt}}
\z
\z

\noindent Finally, it should be noted that the licenser in the illocutionary domain must be of an epistemic variety, as no other speech act adverbs -- discourse-oriented (\textit{čestno, vkratce}) or evaluative (\textit{k ščast'ju, uvy}) -- are compatible with wh-indefinites (or \textit{nibud'}-indefinites):

\ea[*]
{\gll \minsp{\{} K ščast'ju / uvy / čestno / vkratce\}, prišel kto.\\
{} to fortune {} alas {} honestly {} briefly came who\\
\glt Intended: `Fortunately/alas/honestly/in brief, someone came.'}

\z


\subsection{Role of the scale}\label{zan:subsec:scale}
 The inaugural (\ref{zan:ex:4 contexts}) would have us believe that subjunctives and matrix negation are licit licensers for wh-indefinites. This is not quite accurate. Desiderative and root subjunctives in (\ref{zan:ex:bad subjunctives}), and negated factive verbs in (\ref{zan:ex:ne skazal cto prisel kto}), prove to be unfit for purpose.\footnote{As an aside, \textit{nibud'}-indefinites are perfect in (\ref{zan:ex:bad subjunctives}) (though somewhat awkward in (\ref{zan:ex:ne skazal cto prisel kto})).} By contrast, in the previously reported examples, the subjunctive (\ref{zan:ex:ctoby kto ne vosel}) imparts a meaning somewhat akin to English \textit{lest}-clauses (to be discussed separately in \sectref{zan:subsec:high negation}), while negation in (\ref{zan:ex:ne poxoze cto kogo videl}) accompanies a matrix verb of the epistemic flavor.

\ea \label{zan:ex:bad subjunctives}
\ea[*]
{\gll Ja \minsp{\{} xotel / dobivalsja\} togo,	čtoby	kto	priexal.  \\
I	{} wanted {} strove that that.{\SBJV}	 who.{\INDF} 	came \\
\glt Intended: `I wanted for (tried to get) somebody to come.'}
\ex[*\textsuperscript{/?}*]
{\gll Ja by čto sejčas posmotrel.\\
I {\SBJV}	 what.{\INDF} now	watched \\
\glt Intended: `I would watch something now.’}
\z
\z

\ea[*] 
{\gll Ivan ne \minsp{\{} podtverdil / znal\}, čto prišel kto.\\
Ivan {\NEG} {} confirmed {} knew that came  who.{\INDF} \\
\glt Intended: `Ivan didn't confirm / know that anybody came.'\label{zan:ex:ne skazal cto prisel kto}}
\z

\noindent In fact, matrix negation is not directly relevant -- what matters is the type of the embedding predicate. As it turns out, wh-indefinites may be licensed under epistemic non-factives (\textit{think}), emotives (\textit{hope}), and dubitatives (\textit{doubt}) -- i.e., those verbs that in Romance are variable in selecting either subjunctive or indicative complements (\citealt{AnandPranav2013Eaa, farkas1992semantics}); for arguments that they incorporate nonveridical components see \citet{giannakidou2016emotive}. Crucially, even in these contexts, wh-indefinites cannot simply appear ``as is": they are most natural in the presence of a scalar adverb \textit{xot'} ``even, at least". 
 
As for desideratives, a substantial body of work provisions a comparative semantics for the subjunctive-embedding attitudes (\citealt{AnandPranav2013Eaa, heim1992presupposition, villalta2000spanish, villalta2008mood}, a.o.). Though the proposals vary in details, it will suffice for my purposes that \textit{want}-type predicates introduce a scale, which orders the proposition expressed by the complement relative to the contextually supplied alternatives. Applying this to the contexts in (\ref{zan:ex:bad subjunctives}), one may surmise that wh-indefinites are sensitive to preference ordering: they are incompatible with the contexts where the proposition is ranked as more desirable than the alternatives. Interestingly, desideratives, just like the attitude-embedding predicates, become wh-indefinite-friendly upon the introduction of  \textit{xot'}.    
 
To sum up, though both non-factives and subjunctives are nonveridical (and hence, potential licensers), this alone is not sufficient for the felicity of wh-indefi\-nites -- as we will see, these contexts become appropriate for indefinites if they incorporate a bottom-of-the-scale condition. Moreover, this amelioration procedure is also available in imperative and iterative contexts (which, in the absence of scalar adverbials, are likewise incompatible with wh-indefinites).

I begin with the attitude verbs. Since (\ref{zan:ex:ne skazal cto prisel kto}) established that matrix negation is not a licenser for wh-indefinites, I suggested that the relevant factor is the type of the embedding predicate. The latter claim is ostensibly contradicted by the datasets in (\ref{zan:ex:propositional attitude 1}) and (\ref{zan:ex:propositional attitude 2}): while there is some speaker variation, none of my informants find wh-indefinites under \textit{think}, \textit{doubt} or \textit{hope} (whether negated or not) fully acceptable. 

\ea \label{zan:ex:propositional attitude 1} The weather is awful today. People will probably choose to stay in.
\ea[] 
{\gll \minsp{\{\textsuperscript{?}*} Ne dumaju / \minsp{\textsuperscript{??}} ne poxože\}, čto pridet kto na sobranie.\\
{} {\NEG} think {} {} {\NEG} seems that will.come who.{\INDF} to meeting\\
\glt Intended: `\{I don't think that / It doesn't look like\} anybody will show up to the meeting.'}
\ex[\textsuperscript{??}]  
{\gll Somnjevajus', čto pridet kto na sobranie.\\
doubt that will.come who.{\INDF} to meeting\\
\glt Intended: `I doubt anybody will show up to the meeting.'}
\ex[\textsuperscript{?}*\textsuperscript{/??}]  
{\gll Ne nadejus', čto pridet kto na sobranie.\\
{\NEG} hope that will.come who.{\INDF} to meeting\\
\glt Intended: `I doubt anybody will show up to the meeting.'}
\z
\z

\ea \label{zan:ex:propositional attitude 2} The weather is delightful today. Surely, people will be inclined to get out.
\ea[\textsuperscript{?}*\textsuperscript{/??}] 
{\gll \minsp{\{} Dumaju / Poxože\}, čto pridet kto na sobranie.\\
{} think {}  seems that will.come who.{\INDF} to meeting\\
\glt Intended: `\{I think that / It looks like\} somebody will show up to the meeting.'}
\ex[\textsuperscript{?}*\textsuperscript{/??}]  
{\gll Ne somnjevajus', čto pridet kto na sobranie.\\
{\NEG}  doubt that will.come who.{\INDF} to meeting\\
\glt Intended: `I doubt anybody will show up to the meeting.'}
\ex[\textsuperscript{?}*\textsuperscript{/??}]  
{\gll Nadejus', čto pridet kto na sobranie.\\
hope that will.come who.{\INDF} to meeting\\
\glt Intended: `I don't doubt that somebody will show up to the meeting.'}
\z
\z

\noindent Before I show how to improve (\ref{zan:ex:propositional attitude 1}) and (\ref{zan:ex:propositional attitude 2}),  consider an apparent non-sequitur in (\ref{zan:ex:imperative}), whose purpose will become clear in a moment. Though the imperatives provide a felicitous environment for wh-indefinites in a handful of Slavic languages and beyond (\citealt{haspelmath1997indefinite}), evidently they are not legitimate licensers for wh-indefinites in Russian. \citet{kaufmann2012interpreting} develops a modal semantics for imperatives, where \textit{Eat your broccoli!} is roughly equivalent to \textit{You must eat your broccoli}. For \citet[49]{condoravdi2012imperatives}, imperatives integrate the speaker's ``preferential attitudes -- including his wishes and desires", rendering the imperative operator broadly similar to \textit{want}. If so, the ill-formedness of (\ref{zan:ex:imperative}) with a bare \textit{čto} follows from the same principles that inhibit the appearance of wh-indefinites in either deontic contexts like (\ref{zan:ex:deontic modals}) or desiderative contexts like (\ref{zan:ex:bad subjunctives}).

\ea \label{zan:ex:imperative}
        \gll  Privezi \minsp{\{*} čto / čto-nibud'\} iz Pariža! \\
       bring.{\IMP} {} what.{\INDF}  {} what-\textit{nibud'} from Paris\\
        \glt `Bring [me] something from Paris!'
\z

\noindent The reason for these detours is to do with a uniform procedure that converts all the listed bad contexts into good ones. To recap, the ``bad" contexts for wh-indefinites include: (a) root/desiderative subjunctives in (\ref{zan:ex:bad subjunctives}); (b) complements of affirmative and negated propositional attitude verbs, \textit{think, doubt, hope} in (\ref{zan:ex:propositional attitude 1}) and (\ref{zan:ex:propositional attitude 2}); (c) imperatives in (\ref{zan:ex:imperative}). In all three environments, the degradedness disappears upon the introduction of a scalar adverb \textit{xot'} `at least, even', which evinces two properties. First, its associate is obligatorily focalized (\citealt{haspelmath1997indefinite}). Second, \textit{xot'} is itself eligible only in non-assertive (i.e., nonveridical) situations. For example:

\ea
\ea[*]
{\gll On xot' raz  \minsp{(} ne) ezdil v Pariž.\\
he even once {} {\NEG} travelled to Paris\\
\glt Intended: `He (hasn't) traveled to Paris at least/even once.'}
\ex[] 
{\gll On xot' raz ezdil v Pariž?\\
he even once travelled to Paris\\
\glt `Has he been to Paris even once?'}
\z
\z

\noindent With this in place, observe a considerable transformation induced by \textit{xot'} in all the iffy contexts (\ref{zan:ex:XOT'-environments}): root and desiderative subjunctives in \REF{zan:ex:XOT'-environmentsa}--\REF{zan:ex:XOT'-environmentsb}, the imperative in \REF{zan:ex:XOT'-environmentsc}, and the attitude predicate in \REF{zan:ex:XOT'-environmentsd}, all become quite natural when accompanied by \textit{xot'}.   

\ea \label{zan:ex:XOT'-environments}
\ea\label{zan:ex:XOT'-environmentsa}
\gll Ty by xot' raz komu peredaču snesla.\\
       you {\SBJV} even once to.whom.{\INDF} parcel brought\\
         \glt `You could've taken a care package to someone at least once.'\\ \hfill {(R. Pal'. \textit{Cvety ve\v{c}nosti}. 1990)}
\ex\label{zan:ex:XOT'-environmentsb} 
\gll My	dobivali's togo, 	čtoby xot' stročku	nam  kto napisal.\\
we tried that	 that.{\SBJV}	even line to.us who.{\INDF} wrote\\
    \glt `We tried to get somebody to respond to us at least once.'
\ex\label{zan:ex:XOT'-environmentsc}  
\gll  Ty xot' slovo komu napiši, bezdel'nik! \\
you even word to.whom.{\INDF} write.{\IMP} laggard \\
        \glt `Write at least a word to somebody, you laggard!'
\ex\label{zan:ex:XOT'-environmentsd} 
\gll On s nadeždoj dumal, čto xot' raz ego kto uslyšit.\\
he with hope thought that even once him who.{\INDF} will.hear\\
\glt `He hoped that at least once someone will hear him.'
\z
\z

\noindent In fact, \textit{xot'} need not be overt if the context is appropriate, as demonstrated by the \textit{hope}-type predicate in (\ref{zan:ex: kogo vstretit}).\footnote{The availability of the implicit \textit{xot'} may be the source of speaker variation reported above, as well as the disagreement of my informants with the judgments recorded in \citet{heng2018}. In fact, Reviewer 1 reports that in their judgment, wh-elements under \textit{hope} are not possible whether with or without \textit{xot'}.} In both cases of (\ref{zan:ex: kogo vstretit}), the locative adverb \textit{tam} is focalized, which ensures the identical interpretation of (\ref{zan:ex:TAM}) and (\ref{zan:ex:XOT' TAM}) even in the absence of an explicit \textit{xot'}.


\ea \label{zan:ex: kogo vstretit} Context: `It seems John is unlucky in his romantic pursuits. He never even had a date in our small town. But he's moving to New York soon, \dots' 
\begin{xlist}
\ex[\textsuperscript{\ding{51}/?}] 
{\gll Nadejus', čto \minsp{[} tam] kogo vstretit.\\
hope that {} there who.{\INDF} meets\\
\glt `I hope he meets somebody there (at least).'\label{zan:ex:TAM}}
\ex[\textsuperscript{\ding{51}/?}] 
{\gll Nadejus', čto xot' \minsp{[} tam] kogo vstretit.\\
hope that  even {} there who.{\INDF} meets\\
\glt `I hope he meets somebody there at least.'\label{zan:ex:XOT' TAM}}
\end{xlist}
\z

\noindent Russian \textit{xot'} works just like the Greek variable scale \textit{esto} `even, at least' (\citealt{giannakidou2007landscape}). Giannakidou argues that \textit{esto} carries a negative existential presupposition and a bottom-of-the-scale condition. Unlike other types of \textit{even}, \textit{esto} does not introduce the likelihood scale itself, but rather relies on the context to supply one.

The central take-away point here is sensitivity to scale: In potentially licensing environments, wh-indefinites are possible only in the presence of a scalar element which supplies a (contextual) bottom-of-the-scale condition. If so, iterative contexts with frequency adverbs such as (\ref{zan:ex:frequency adverbs}) likewise comply with this ``less than" requirement: wh-indefinites are only possible with negative frequency adverbs in contrast to their \textit{nibud'}-cousins, which are fine with both, \textit{rarely} and \textit{frequently}.    

\ea \label{zan:ex:frequency adverbs}
\ea  \label{zan:ex:neg frequency adverbs}
 \gll Znakomyx u menja v Moskve mnogo [...], no ja redko\hspace{1.4cm} \minsp{\{} kogo / \minsp{$\langle$} \minsp{\textsuperscript{\ding{51}}} kogo-nibud'$\rangle$\} vižu.\\
      acquaintances at me in Moscow lots {} but I rarely {} whom.{\INDF} {} {} {} whom-\textit{nibud'} see \\
        \glt `I've many acquaintances in Moscow, but I rarely see anybody.’\\ \hfill (M.Bulgakov. Letters.)
        
\ex  \label{zan:ex:pos frequency adverbs}
        \gll Ona často \minsp{\{*} kogo /  \minsp{\textsuperscript{\ding{51}}} kogo-nibud'\} rugaet.  \\
       she frequently {} whom.{\INDF} {} {} whom-\textit{nibud'} chides \\
        \glt `She chides somebody frequently.'
\z
\z

\noindent Finally, I would be remiss not to point out one recurrent theme. In all the licensing contexts discussed so far, wh-indefinites exhibit sensitivity to their syntactic environments -- in that the relevant licenser must be contained in the same clause as the licensee.   


\subsection{``High" negation}\label{zan:subsec:high negation}

Yanovich's subjunctive from (\ref{zan:ex:ctoby kto ne vosel}) belongs in the same semantic cluster as the examples in (\ref{zan:ex:lest}). I will refer to them as \textsc{lest}-clauses. \textsc{Lest}-clauses are special, because they freely admit negative concord items (\textit{nikto}) as well as \textit{nibud'}-inde\-finites (this alternation is treated in \citealt{paducheva2016}).


\ea \label{zan:ex:lest}
\ea \label{zan:ex:lest-pribrala ctoby}
 \gll ... pribrala,	čtoby	\minsp{\{} kto	 / \minsp{$\langle$} \minsp{\textsuperscript{\ding{51}}} kto-nibud'$\rangle$ / \minsp{$\langle$} \minsp{\textsuperscript{\ding{51}}} nikto$\rangle$\} ne	podnjal.  \\
       {}	picked.up	that.{\SBJV} {} who.{\INDF} {} {} {} who-\textit{nibud'} {} {} {} \textit{ni}.who	{\NEG}	took \\
        \glt `(I deliberately) picked [it] up, lest somebody take it.'\\ \hfill {(M. Bulgakov. \textit{Master i Margarita}. 1928--40)}
 
  \ex
  \gll 	Szadi,	čtoby	\minsp{\{} kto / \minsp{$\langle$} \minsp{\textsuperscript{\ding{51}}} kto-nibud'$\rangle$ / \minsp{$\langle$} \minsp{\textsuperscript{\ding{51}}} nikto$\rangle$\}	ne	sbežal	dorogoju,	exali	na	konjax dva	monaxa.  \\
        behind	that.{\SBJV} {}	who.{\INDF} {} {} {} who-\textit{nibud'} {} {} {} \textit{ni}.who	{\NEG}	ran.away	en.route	rode	on	horses
two	monks \\
        \glt `Two monks were riding astride behind [them] lest someone make a run for it en route.' \hfill {(Ju. German. \textit{Rossija molodaja}. 1952)}
       
\z\z

\noindent The point of oddity is that \textit{nibud'}-indefinites are not licensed by clausemate negation. Verbal negation in Slavic famously requires negative concord, as in (\ref{zan:ex:nobody escaped}).  
        \ea \label{zan:ex:nobody escaped}
        \gll \minsp{*\{} Kto-nibud' / \minsp{\textsuperscript{\ding{51}}} Nikto\}	ne	sbežal	dorogoju.\\
        {} who-\textit{nibud'} {} {} \textit{ni}.who    {\NEG} ran.away en.route\\
        \z
\noindent Concerning the meaning differences induced by \textit{kto-nibud'} vs. \textit{nikto} in, e.g., (\ref{zan:ex:lest-pribrala ctoby}), \citeauthor{paducheva2016} offers the paraphrases in (\ref{zan:ex:paraphrases}) and accepts the two as logically equivalent. She argues that although \textit{nibud'}-indefinites appear in the scope of ``global" negation (as opposed to local negation in cases of \textit{nikto}), they are licensed by a nonveridical clausal operator. 

\ea \label{zan:ex:paraphrases}
\ea With \textit{nikto}: I picked it up so that (it is the case that) nobody takes it.
\ex \label{zan:ex:paraphrases 2} With \textit{nibud'}: I picked it up so that it is not the case that somebody takes it.
\z \z

\noindent In this specification for a negative outcome, \textsc{lest}-clauses are akin (though not fully identical) to ``apprehensive subjunctives" like (\ref{zan:ex:predicates of fear}). A handful of verbs, denoting surveillance/supervision/warning (\textit{prismatrivat'} `keep an eye', \textit{karaulit'} `guard', \textit{bereč'sja} `beware, be safe', \textit{smotret'} `watch (out)') or psych states of an unpleasant nature (\textit{bojat'sja} `be afraid', \textit{trevožit'sja} `be anxious', \textit{volnovat'sja} `be uneasy'), select a subjunctive clause headed by \textit{kak (by)} (\citealt{nilsson2012peculiarities}). In fact, the matrix verb may be altogether absent, in which case a bare \textit{kak by}-clause (absolutely coherent as a stand-alone sentence) is understood as an implicit warning or expression of fear. 


\ea \label{zan:ex:predicates of fear}
  \gll \minsp{(} Smotri / Bojus',) kak by \minsp{\{} kto / \minsp{$\langle$} \minsp{\textsuperscript{\ding{51}}} kto-nibud'$\rangle$ / \minsp{$\langle$} \minsp{\textsuperscript{\ding{51}}} nikto$\rangle$\} telefon ne stibril v takoj tolpe! \\
      {} watch.out.{\IMP} / fear how {\SBJV} {} who.{\INDF} {} {} {} who-\textit{nibud'} {} {} {} \textit{ni}.who phone {\NEG} snatched in such crowd\\
         \glt `Watch out lest someone snatch your phone in this crowd./ I fear someone might snatch your phone in this crowd.'      
\z

\noindent In addition to  subjunctive morphology, \textsc{lest}-clauses (\ref{zan:ex:lest}) and apprehensive subjunctives (\ref{zan:ex:predicates of fear}) also pattern alike in syntax -- by requiring verbal negation and admitting NCIs as well as \textit{nibud'-} and wh-indefinites.\footnote{Contrary to the standard claim that NCIs do not embed under fear-predicates (e.g., \citealt{zan:abels2005, zan:brown1995}), many such examples are attested online. My informants likewise indicate that (\ref{zan:ex:predicates of fear}) with the NCI is perfectly on a par with the negative concord version of (\ref{zan:ex:lest}). See also \citet{nilsson2012peculiarities} for further empirical adjudication.} Such similarity, in turn, suggests that the two constructions may be eligible for a uniform analysis.

Complements of fear verbs are said to contain ``expletive negation" (EN), alleged to be devoid of polarity reversing semantics despite the compulsory realization of negation on the verb. The theoretical status of EN remains murky: there is no consensus on what \textit{ne} in (\ref{zan:ex:predicates of fear}) actually does. Is it a semantically contentful element that moves to a high position within its clause to negate the evaluative mood (as in \citealt{zan:abels2005}), or a mood marker licensed by nonveridicality (as in \citealt{yoon2011not}), or a weak epistemic (as in \citealt{makri2016not}), or simply a semantically empty exponent of morphosyntactic negation (as in \citealt{zan:brown1995})? My proposal is closer in spirit to \citet{zan:abels2005} (and consistent with \citeauthor{paducheva2016}'s \citeyear{paducheva2016} insight on ``global" vs. ``local" negation). Suppose that there are multiple merge sites for negation available in Russian, as in (\ref{zan:ex:two negs}). The lower one (NegP2) negates events and delimits the exclusive domain of negative concord. The higher one (NegP1), introduced in the illocutionary field above TP, does not license NCIs, but it is compatible with bare wh-indefinites. If so, (\ref{zan:ex:lest}) and (\ref{zan:ex:predicates of fear}) are ambiguous between the two structures -- and hence, enable a seemingly free alternation of the indefinites and NCIs. 


\ea \label{zan:ex:two negs}
$[$ ... [\textsubscript{NegP1} {\NEG} ... [\textsubscript{TP} ... [\textsubscript{NegP2} {\NEG} [AspectP/\textit{v}P ... ]]]]]
\z

\noindent There are also constructions that are not ambiguous between the two negations, shown in (\ref{zan:ex:POKA}) and (\ref{zan:ex:X-neg}). The former, featuring an \textit{until}-clause, is standardly classified as another species of EN. The latter features expletive negation in a very literal sense -- the negator here is a taboo word (\textit{dick}, glossed as X.{\NEG}). Neither construction tolerates NCIs.


\ea \label{zan:ex:POKA}
\gll Uvjazneš	po	samye	stupitsy	i	zagoraeš,	poka \minsp{\{} kto / \hphantom{xxxxxxxxxx} \minsp{$\langle$ \textsuperscript{\ding{51}}} kto-nibud'$\rangle$ / \minsp{$\langle$ *} nikto$\rangle$\} 	ne	vytaščit.\\
 stuck	to	very	hubs	and	tan until {} who.{\INDF} 
     {} {} {} who-\textit{nibud'} {} {} \textit{ni}.who	{\NEG}	will.pull.out\\
        \glt `Your hubs get stuck and you hang out until somebody pulls you out.'\\
\hfill {(O. Efremov. \textit{Rybak primor'ja}. 2003)}
\z


\ea \label{zan:ex:X-neg}
\gll Xuj \minsp{\{} kto / kto-nibud' / \minsp{$\langle$ *}  nikto$\rangle$\} prišel.\\
X.{\NEG} {} who.{\INDF} {}  who-\textit{nibud'} {} {} \textit{ni}.who came \\
 \glt `It is not the case that anybody came.' \hfill {(\citealt{Ershler})}
 \z


\noindent Per \citeauthor{zan:abels2005}, the matrix proposition and the \textit{until}-clause in (\ref{zan:ex:POKA}) cannot be true at the same time: One is either stuck, in which case the extricating event has not happened, or one is extricated, in which case they are no longer stuck. His proposal is that negation raises at LF to scope over the \textit{poka}-clause, which precludes NCIs (as the licenser ceases to be sufficiently local). My amendment is that high negation merges directly in that position. Similarly, for X-negation in (\ref{zan:ex:X-neg}), \citet{Ershler} argues that the negator sits in the Spec of the TP-external PolP that does not license NCIs.

The data are summarized in Table \ref{zan:tab:wh and NCIs}. I ascribed the alternation NCI $\sim$ wh-indefi\-nite in the first two entries to syntactic ambiguity stemming from the position of merge: The lower negation requires negative concord, the higher one supplies an appropriate context for wh-indefinites. Because the last two contexts do not tolerate NCIs, the negators in both instances must be introduced higher -- above TP.    

\begin{table}
\caption{Distribution of wh-indefinites and NCIs}
\label{zan:tab:wh and NCIs}
 \begin{tabularx}{.70\textwidth}{lcc}%{lYYYY} %.77 indicates that the table will take up 77% of the textwidth
  \lsptoprule
            & NCIs & wh-indefinites \\
  \midrule
  \textsc{lest}-clauses  &   \ding{51}  &   \ding{51}\\
  apprehensive subjunctives &  \ding{51}   &   \ding{51}\\
  \textit{until}-clauses & \ding{55}  & \ding{51} \\
  X-{\NEG} & \ding{55} & \ding{51}\\
  
  \lspbottomrule
 \end{tabularx}
\end{table}

It should be noted that I do not envision a fixed position for ``high negation" -- indeed, its behavior in various contexts is consistent with multiple merge sites in the illocutionary domain. Since considerations of space prevent me from dealing with this topic in any coherent detail, I confine myself to a bare bones sketch of the proposal, leaving the details of implementation or, indeed, a comprehensive justification for future endeavors. Assume the structure in (\ref{zan:ex:illocutionary domain}), adopted from \citet{krifka2019layers}, where ActP is the locus of assertions ($\bullet$) or questions (?), ComP is the domain of the speaker's social commitments to the proposition, and the already familiar J(udgment)P is the province of subjective epistemic attitudes. For explicitness, I also assume that ComP can be headed by a null bouletic element (alternatively, one may posit an independent projection, representing bouletic attitudes of the speaker as in, e.g., \citealt{sode2018verb}).  

\ea \label{zan:ex:illocutionary domain}

$[$\textsubscript{ActP}  [\textsubscript{Act$^0$} $\bullet$][\textsubscript{ComP} [\textsubscript{Com\textsubscript{BOUL}}] [\textsubscript{JP} ...[\textsubscript{TP} ...]]]]
\z

\noindent  Given the above, I suggest that X.{\NEG} and ``global" negation in \textsc{lest}-clauses apply at the level of ActP, which furthermore must contain an assertorial operator to render it consistent with Erschler's observation that X.{\NEG} is impossible in questions \REF{zan:ex:X-neg in questions}.

\ea[*] 
{\gll Xren on xodil na rabotu? \\
X.{\NEG} he went to work\\
\glt Intended: `Did he not go to work?'\label{zan:ex:X-neg in questions}}
\z

\noindent On the other hand, in apprehensive subjunctives under \textit{kak by} and \textit{until}-clauses, negation appears lower -- at the level of ComP or TP. That the {\NEG} of a \textsc{lest}-clause is distinct from the {\NEG} of an apprehensive subjunctive/\textit{until}-clause is confirmed by (\ref{zan:ex:X-neg substitution}):  X.{\NEG} can replace \textit{ne} in a \textsc{lest}-clause (\ref{zan:ex:X-neg in lest}) but not the lower \textit{ne} of the two EN contexts in (\ref{zan:ex:X-neg in fear}) and (\ref{zan:ex:X-neg in until}).

\ea \label{zan:ex:X-neg substitution}
\ea[]
{\gll 	...,	čtoby xren	kto sbežal	dorogoju, ...  \\
{} that.{\SBJV}	X.{\NEG} who.{\INDF} ran.away	  en.route\\
\glt `... so that it is not the case that somebody escapes en route, ...'\label{zan:ex:X-neg in lest}}

\ex[*] 
{\gll Bojus', kak by xren kto telefon stibril. \\
fear how {\SBJV} X.{\NEG} who.{\INDF} phone snatched\\
\glt Intended: `I don't want for anybody to steal the phone.'\label{zan:ex:X-neg in fear}
}

\ex[*]
{\gll ... poka xren kto vytaščit. \\
{} until X.{\NEG} who.{\INDF} will.pull.out\\
\glt Intended: `... until someone pulls (us) out.' \label{zan:ex:X-neg in until}
}
\z
\z

\noindent The exposition is undeniably terse here, but the essential insight should be reasonably clear: Wh-indefinites are licensed by a negative operator, residing in the illocutionary domain. This distinguishes wh-indefinites from NCIs, whose felicity is predicated on the presence of a proposition-internal operator.   


\section{Intermediate summary}\label{zan:subsec:intermediate summary}

Wh-indefinites are possible in polar interrogatives (neutral or biased), conditionals (indicative or hypothetical) and under existential epistemic modals. While the future, episodic past and modal environments (with universal epistemics) are ``too strong", they can be made compatible with wh-indefinites by manipulating subjective modality (i.e., by merging an epistemic speech act adverbial). Desiderative and root subjunctives, attitude predicates, iterative contexts and imperatives likewise create ``potentially licensing" contexts -- only in these situations, the felicity of wh-indefinites is parasitic on the presence of a scalar adverb (encoding a bottom-of-the-scale condition). Finally, wh-indefinites are happy under high (illocutionary) negation. 

The lessons here are two. First, there are no contexts that license wh-indefinites to the exclusion of \textit{nibud'}-indefinites. In fact, the requirements of the latter are substantially less stringent: \textit{nibud'}-indefinites are perfectly acceptable with no additional conditions in desiderative, future, iterative, etc. contexts. In the interest of full disclosure, consider also (\ref{zan:ex:quantified contexts}), which shows that in contrast to wh-indefinites, \textit{nibud'}-indefinites are fine in both the scope and the restriction of a universal.\footnote {I refer the reader to \citet{PaduchevaE.V.ElenaViktorovna1974Oss:}, \citet{pereltsvaig2008} for discussion of Russian \textit{nibud'}-indefinites in quantified contexts.} Conversely, wh-indefinites are routinely banned in universally quantified contexts, independent of the quantifier's syntactic role (subject or object), its surface position or, indeed, its type (\textit{vse} `all', \textit{oba} `both', \textit{každyj} `each' are all deviant with wh-indefinites). Furthermore, my informants are reluctant to accept wh-indefinites even when a quantifier is embedded in an otherwise wh-indefinite-friendly environment, such as a polar interrogative in (\ref{zan:ex:Q with universal}).       

\ea \label{zan:ex:quantified contexts}
\ea
        \gll Každyj \minsp{\{$\langle $*} čto$\rangle$ / čto-nibud'\} slyšal o korole Arture. \\
     each {} what.{\INDF} {} what-\textit{nibud'} heard of king Arthur\\
        \glt `Everybody heard something about king Arthur.'\\ \hfill{(bookstore blurb, modified)}

\ex 
        \gll Každyj, kto \minsp{\{*} komu / komu-nibud'\} zaviduet, obladaet nizkoj samoocenkoj.\\
      each who.{\REL} {} to.whom.{\INDF} {} to.whom-\textit{nibud'} envies possesses low self-esteem\\
        \glt Intended: `Everyone who envies somebody has low self-esteem.'
\z
\z

\ea \label{zan:ex:Q with universal}
\gll Razve vse studenty \minsp{\{*\textsuperscript{/?}*} čto	/ čto-nibud'\}  pročitali?\\
really all students {} what.{\INDF} {} what-\textit{nibud'} read\\
\glt `Didn't all students read something?'
\z

\noindent The second point concerns a recurrent locality issue. The felicitous contexts require a clausemate licenser of the relevant kind -- adverbs aside, all other environments feature an operator associated with the C-domain. For instance, in (\ref{zan:ex:long-distance OP}), with the operator in the superordinate clause, wh-indefinites are unacceptable.  

\ea \label{zan:ex:long-distance OP}
\ea \gll  Razve on govoril, čto \minsp{\{*} kto / kto-nibud'\} sdal \.{e}kzamen? \\
really he said that {} who.{\INDF} {} who-\textit{nibud'} passed  exam\\
\glt `Didn't he say that someone passed the exam?'

\ex
\gll Esli najti v Rossii čeloveka, kotoryj \minsp{\{$\langle$\textsuperscript{?}*} čto$\rangle$ / čto-nibud'\}  sdelal v pol'zu UNSO, to ... .\\
if to.find in Russia person who {} what.{\INDF} {} what-\textit{nibud'} did for benefit UNSO\\
\glt `If one were to find a person who did something to benefit UNSO, then [he might be prosecuted].' \hfill {(gazeta.ru. 2014)}
\z
\z 

\noindent This locality constraint is intuitively logical. While both local and distant licensers require full morphological specification, the medial one enables the spellout of a bare indefinite, provided the environment is sufficiently negative. In other words, we may conceive of the polarity-sensitive pronouns as a hierarchy of sorts, i.e. \textit{kto-nibud'} >> \textit{kto(-nibud')} >> \textit{nikto}, where \textit{nibud'} is compatible with (almost any) nonveridical operator (medial or distant), \textit{ni}-items are required under a local antiveridical operator, and wh-indefinites are somewhat in the middle -- possible in a subset of nonveridical environments in close proximity to their licenser. This ``intermediate" (and morphologically sterile) status also correlates with certain PF-related effects to be discussed in the next section. 


\section{Syntax-PF interactions}\label{zan:sec:syntax} 

That wh-indefinites are crosslinguistically de-focalized is not a revelation (e.g., \citealt{haida2008indefiniteness}, \citealt{hengeveld2022quexistentials}). \citet{hengeveld2022quexistentials}, in fact, state the requirement as a biconditional: wh-elements (``quexistentials" in \citeauthor{hengeveld2022quexistentials}'s terminology) are obligatorily focalized in their interrogative interpretation; in their existential incarnation, on the other hand, they are never focalized.\footnote{Reviewer 2 points out that the first clause of this biconditional is falsified by Czech (see \citealt{vsimik2010interpretation}) and perhaps Slovenian (\citealt{mivsmavs2017slovenian}).} What I will attempt to show here is that Russian wh-indefinites are not simply unable to bear contrastive focus: indeed, they are considerably fussier in selecting surface positions than other indefinites. The basic observation is that in addition to resisting contrastive focus, wh-indefinites prefer to be adjacent to the element that realizes the main sentential stress. This property, along with a preference for clustering in a specific order as well as resistance to coordination, render them akin to clitics. 

Sentential stress here is understood as in \citet{yokoyama1987discourse}, who argues that there are two basic types of intonation in Russian: Type 1 (neutral) and Type 2 (``expressive"), shown in (\ref{zan:ex: no-stress}) and (\ref{zan:ex: stress}), respectively. Type 1 entails an iterating sequence of intonational phrases with LH contour, accompanied by a downstep. The ``new information" (or, in more familiar terms, an element bearing information focus) comes at the end with a falling (HL) contour, which basically corresponds to a neutral declarative sentence with falling intonation. This is demonstrated in (\ref{zan:ex: IC-no stress}).\footnote{This representation is borrowed from \textcite{king1993configuring}.} Obviously, ``new information'' need not be restricted to a single lexical item -- an entire constituent may function in this manner. In Type 2 (\ref{zan:ex: IC-stress}), the fronted constituent (\textit{doždiček}) realizes sentential stress, defined as the ``stress which marks the knowledge item that would occur in utterance-final position, were the same sentence to be uttered with intonation Type 1 instead" (\citealt{yokoyama1987discourse}: 191). Its properties are twofold: (i) It is the last intonational center of the utterance, and (ii) No rising tones can follow it. Abstracting away from the pitch details, `\searrow' 
 will be used to indicate Type 1 intonation, `$\ast$' (and small caps) to mark Type 2 intonation, and `|' to identify phonological phrase boundaries (call it $\pi$P).

\ea
\ea\label{zan:ex: no-stress}
\gll Nad Krakovom nakrapyval doždiček.\\
   over Krakow drizzled rain\\
       \glt `The rain was drizzling over Krakow.' 
       
\ex \label{zan:ex: stress}
    \gll  Nad Krakovom \textsc{doždiček} nakrapyval.\\
    over Krakow rain drizzled\\
\z
\z

\ea
\ea \label{zan:ex: IC-no stress}

\begin{tabbing} Nad \= Krako\=vom \=| na\=krapyval \=| doždiček. \\ 
    \> LH  \>  \>|\textsuperscript{!}  \> LH  \> |  HL (\searrow) \end{tabbing}  

\ex \label{zan:ex: IC-stress}
\begin{tabbing}
     Nad \=Krakovom \= | \=  \textsc{doždiček }\= nakrapyval.\\
     \> LH \> | \>  HL ($\ast$)  \>  
\end{tabbing}
\z
\z

\noindent \citet{yokoyama1987discourse} also shows that indefinite pronouns are ineligible to realize the final HL under neutral Type 1 intonation. Instead, the intonational core shifts to a ``fully specified" constituent. In the case of (\ref{zan:ex: pojdemte}), it is the verb. While the indefinite is ineligible to serve as the default intonational pivot here, it can be pronounced with the contrastive (Type 2) contour. 

\ea \label{zan:ex: pojdemte}
    \gll Pojdemte  kuda-nibud'.\\ 
    lets.go  where-\textit{nibud'} (\searrow)\\ \hfill{(Type 1 with indefinites)}
    \glt      `Let's go somewhere.' 
\z

\noindent  With these preliminaries in place, consider \citeposst{yan2005} data in (\ref{zan:ex:Yanovichs adverbials-repeated}) again (repeated from (\ref{zan:ex:Yanovichs adverbials})). Earlier it was established that a modal adverb like \textit{dolžno byt'} is a legitimate licenser for wh-indefinites after all, provided its licensee complies with certain word order restrictions.

\ea \label{zan:ex:Yanovichs adverbials-repeated}
\ea
        \gll Možet, \minsp{\{} kto\} prixodil \minsp{\{} kto\}. \\
      maybe {} who.{\INDF} came.{\MASC} {} who.{\INDF}\\
      \glt `Maybe someone came.' 

\ex \label{dolzhno byt'} \label{zan:ex:dolzno byt' kto prixodil-repeated}
        \gll Dolžno byt', \minsp{\{*} kto\} prixodil \minsp{\{} kto\}. \\
       must be {} who.{\INDF}  came.{\MASC} {} who.{\INDF}\\
      \glt `It must be the case that someone came.'
\z
\z

\noindent Focusing here on the intransitive verbs, consider the subject position permutations with \textit{dolžno byt'}. A neutral sequence in (\ref{zan:ex: d.b. neutral}) requires a postverbal subject, which takes on the default Type 1 accentuation. On the other hand, (\ref{zan:ex: d.b. focused}) is marked: now, the scrambled (contrastively focused) subject carries Type 2 sentential stress.  

\ea
\ea \label{zan:ex: d.b. neutral}
   \gll Dolžno byt', | umerla | koroleva.\\
       must be | died | queen (\searrow)\\
       \glt `The queen must've died.' 
       
\ex \label{zan:ex: d.b. focused}
    \begin{tabbing}  Dolžno byt', \= | \textsc{korol}\= \textsc{eva} umerla.\\ \>  \> $\ast$
    \end{tabbing}
\z
\z

 \noindent On the other hand, the \textit{nibud'}-indefinite in (\ref{zan:ex: d.b.-nibud}) can be placed either before or after the verb -- but in either case \textit{umer} serves as the default prosodic center of its prosodic phrase, i.e. both (\ref{zan:ex: d.b.-nibud-1}) and (\ref{zan:ex: d.b.-nibud-2}) display the Type 1 pattern, where the intonational pivot shifts along with the verb. The Type 2 scheme, found in (\ref{zan:ex: d.b. focused}), is difficult to get for \textit{nibud'}-indefinites. For whatever reason, in these contexts the \textit{nibud'}-item resists contrastive focalization.

\ea \label{zan:ex: d.b.-nibud}
\ea \label{zan:ex: d.b.-nibud-1}
   \gll Dolžno byt', | umer kto-nibud'.\\
       must be | died who-\textit{nibud'} (\searrow)\\
       \glt `Somebody must've died.' 
       
\ex \label{zan:ex: d.b.-nibud-2}
     \gll Dolžno byt', | kto-nibud' umer.\\
     must be {} who-\textit{nibud'} died (\searrow)\\
\z
\z

\noindent Finally, consider (\ref{zan:ex: d.b.-wh}). The sentence is parsed into two $\pi$Ps. Immediately excluded are instances like (\ref{zan:ex:d.b.-wh-3}) with the Type 2 (contrastive focus) intonation. The two incarnations of Type 1 prosody in (\ref{zan:ex:d.b.-wh-1}) and (\ref{zan:ex:d.b.-wh-2}) correspond to (\ref{zan:ex: d.b.-nibud-1}) and (\ref{zan:ex: d.b.-nibud-2}), respectively. I ascribe the deviance of (\ref{zan:ex:d.b.-wh-2}) to the convergence of two factors: the indefinite sits in the $\pi$P-initial position to the left of the element realizing default declarative prosody.  

\ea \label{zan:ex: d.b.-wh}
\ea\label{zan:ex:d.b.-wh-1}
  \begin{tabbing} Dolž\=no byt', | prixo\=dil kto.\\
        \>  \> HL (\searrow)
    \end{tabbing}
      
       
\ex \label{zan:ex:d.b.-wh-2} \begin{tabbing} \textsuperscript{?}*Dolž\=no byt', | kto prixo\=dil.\\
    \>  \>  HL (\searrow)
        \end{tabbing}

\ex \label{zan:ex:d.b.-wh-3}\begin{tabbing} *Dolžno byt', \= | \textsc{kt}\=\textsc{o} prixodil.\\
    \>  \> $\ast$
        \end{tabbing}
\z
\z

\noindent Taken independently, these two contingencies are no impediment for wh-inde\-finites. For instance, \textit{možet} in (\ref{zan:ex: mozet-wh}) does not require a prosodic boundary after itself, which, in turn, ensures that the indefinite is not stranded in the initial position. In this situation, the indefinite can be left- or right-adjacent to the default prosodic host. Note that a heavier constituent -- like \textit{možet byt'} in (\ref{zan:ex: mozet byt-wh}) -- is tougher to integrate into the utterance: with a pause after the adverbial, the indefinite feels awkward in the preverbal slot.  

\ea \label{zan:ex: mozet-wh}
\ea \label{zan:ex: mozet-wh-1}
   \gll Možet, umer kto.\\
       maybe died who.{\INDF} (\searrow)\\
       
\ex \label{zan:ex: mozet-wh-2}
    \gll Možet, kto umer.\\
     maybe who.{\INDF} died (\searrow)\\

\z
\z

\ea \label{zan:ex: mozet byt-wh}
   \gll \minsp{\textsuperscript{?}*\textsuperscript{/??}} Možet byt', | kto prišel.\\
      {} may be {} who.{\INDF} came (\searrow) \\
    \glt Intended: `Maybe someone came.'
\z

\noindent Conversely, (\ref{zan:ex: YN-wh-initial}) shows that a wh-indefinite may appear in the utterance-initial position but only if its host carries a non-default intonational contour, as is the case in the polar interrogative context schematized in (\ref{zan:ex: YN-wh-initial-2}). 

\ea \label{zan:ex: YN-wh-initial}
\ea 
   \gll Kto \textsc{prišel}?\\
      who.{\INDF} came \\
    \glt `Did somebody come?'

\ex \label{zan:ex: YN-wh-initial-2}
\begin{tabbing} Kto \textsc{pri}\=\textsc{šel}?\\
    \>  $\ast$
        \end{tabbing}

\z
\z

\noindent The analytical payoff here is this. Wh-indefinites can appear neither in the positions of information focus (like other indefinites) nor in the positions of contrastive focus (unlike other indefinites). Additionally, they must obey certain added restrictions, which curb their presence in the $\pi$P-initial positions. These two properties suggest that the elements in question are of a special nature. In the remainder of this section I identify a few additional quirks of wh-indefinites that attest to their clitic-like qualities.

First, multiple wh-indefinites are possible in principle. In such situations, the indefinites prefer the sequence \textsc{nom} $>>$ \textsc{dat/acc} $>>$ adjuncts. Violations are perceived to be non-lethal -- certainly not on the level of ordering infractions in languages with pronominal clitics, yet my informants are consistent in their dislike for the alternative orders. Examples are found in (\ref{zan:ex: ordering restrictions}).

\ea \label{zan:ex: ordering restrictions}
\ea 
        \gll Videl li \minsp{\{} kto kogo / \minsp{\textsuperscript{?}} kogo kto\} včera? \\
        saw \textsc{q} {} who.{\INDF} whom.{\INDF} {} {} whom.{\INDF} who.{\INDF} yesterday  \\
        \glt `Did someone see anybody yesterday?'

\ex 
        \gll Kak by \minsp{\{} kto gde / \minsp{\textsuperscript{?/??}} gde kto\} ne zastrjal!\\
 how {\SBJV} {} who.{\INDF} where.{\INDF} {} {} where.{\INDF} who.{\INDF} {\NEG} get.stuck  \\
   \glt `(I am afraid) someone might get stuck somewhere.'
   
\z
\z

\noindent Second, multiple wh-indefinites tend to form a cluster, as demonstrated by an embedded YN question in (\ref{zan:ex: split cluster-Q}) and a conditional in (\ref{zan:ex: split cluster-Cond}). Under the most natural reading, in the deviant examples, the verbs (i.e., \textit{razboltal} and \textit{rasskažet}) form the  prosodic core in the relevant intonational domains. The oddity of (\ref{zan:ex: split cluster-Q-1}) and (\ref{zan:ex: split cluster-Cond-1}) follows from the non-adjacency of one of the indefinites to its (verbal, in this case) host. There are, however, strategies that improve split clusters. For instance, if \textit{Ivan} from (\ref{zan:ex: split cluster-Cond-1}) receives contrastive focus in the manner of (\ref{zan:ex: split cluster-Cond-2}), the sentence becomes rather natural. In other words, while the default configuration is one in which the indefinites form a bundle, split clusters are possible if the indefinites in question are adjacent to the appropriate host.

\ea \label{zan:ex: split cluster-Q}
\ea[*\textsuperscript{?}] 
{\gll \minsp{(} Ja ne znaju,) razboltal li komu Ivan čto, no vse uže znajut naš sekret. \\
{} I {\NEG} know blabbed \textsc{q} to.whom.{\INDF} Ivan what.{\INDF} but all already know our secret\\
\glt Intended: `I don't know if Ivan blabbed something to someone, but everybody already knows our secret.'\label{zan:ex: split cluster-Q-1}}
\ex[] 
{\gll Ja ne znaju, razboltal li komu čto Ivan,...  \\
I {\NEG} know blabbed \textsc{q} to.whom.{\INDF} what.{\INDF} Ivan\\}
\z
\z

\ea \label{zan:ex: split cluster-Cond}
\ea[\textsuperscript{??}] 
{\gll Esli komu Ivan čto rasskažet, ja budu v jarosti.\\  
if to.whom.{\INDF} Ivan what.{\INDF} will.tell I will.be in fury\\
\glt Intended: `If Ivan tells anybody anything, I will be furious.'\label{zan:ex: split cluster-Cond-1}}
\ex[] 
{\gll Esli Ivan komu čto rasskažet, ...\\
 if Ivan to.whom.{\INDF} what.{\INDF} will.tell\\} 
 \z
\z

\ea \label{zan:ex: split cluster-Cond-2}
\begin{tabbing} \textsuperscript{\ding{51}}  Esli komu \textsc{Iv}\=\textsc{an}  čto rasskažet,...\\
    \>  $\ast$
        \end{tabbing}
\z

\noindent The third property requires a small digression. It is a well-established fact that multiple wh-phrases can be coordinated in Russian in the manner of (\ref{zan:ex: kto i kogo-1}) (e.g., \citealt{gribanova2009structural}). The other cases in (\ref{zan:ex: hybrid coordination}) are, perhaps, less famous (data are due to \citealt{paperno2012semantics}). \citeauthor{paperno2012semantics} shows that the conjuncts in such configurations must be of the same type (i.e., indefinite$+$indefinite, universal$+$universal, etc.), cf. a mismatched indefinite$+$universal in (\ref{zan:ex: kto-nibud i vse}).

\ea \label{zan:ex: hybrid coordination}
\ea \label{zan:ex: kto i kogo-1}
        \gll  Kto i kogo videl?\\
        who and who saw\\
        \glt `Who saw whom?'
 \ex
        \gll Nikto i nikogo ne pobedil.\\
        \textit{ni}-who and \textit{ni}-whom {\NEG} defeated\\
        \glt `Nobody defeated anybody.'
       
      \ex
        \gll Ponjal li kto-nibud' i čto-nibud'?\\
        understood \textsc{q} who-\textit{nibud'} and what-\textit{nibud'}\\
        \glt `Did anybody understand anything?'
\z
\z

\ea[*] 
{\gll Ponjal li kto-nibud' i vse?\\
understood \textsc{q} who-\textit{nibud'} and everything\\
\glt Intended: `Did anybody understand everything?'\label{zan:ex: kto-nibud i vse}}
\z

\noindent The phenomenon of hybrid coordination is poorly understood (and I have nothing to add about the syntax of these structures). But whatever the mechanism, it is clearly unavailable to wh-indefinites: the pattern in (\ref{zan:ex: wh-coordination}) corroborates that wh-indefinites are not fantastic when coordinated, whereas in the absence of the coordinator (\ref{zan:ex: wh-coordination-no coordinator}), they are well-formed. One alternative (with a precedent in the literature) is to attribute their resistance to coordination to PF reasons.  

\ea \label{zan:ex: wh-coordination}
\ea[\textsuperscript{?}*\textsuperscript{/??}] 
{\gll Možet, on komu i čto privezet iz Pariža.\\
maybe he to.whom.{\INDF} and what.{\INDF} will.bring from Paris\\
\glt Intended: `Maybe somebody will bring something from Paris.'}
\ex[\textsuperscript{?}*\textsuperscript{/??}] 
{\gll Esli on komu i čto privezet,... \\
if he to.whom.{\INDF} and what.{\INDF} will.bring\\
\glt Intended: `If he brings anything for anybody,...'}
\z
\z

\ea \label{zan:ex: wh-coordination-no coordinator}
\ea Možet, on komu čto privezet iz Pariža.\\
\ex Esli on komu čto privezet,... \\
\z
\z

\noindent Said precedent is found in \citet{StepanovArthur2020WaWW}, who argue that in Lebanese Arabic, \textit{ʃu} `what' evinces clitic-like properties -- one of which, they suggest, is resistance to coordination, as shown in (\ref{zan:ex: stepanov}) (this also holds of the French \textit{que} `what'). 

\ea \label{zan:ex: stepanov}
\ea[*]
{\gll ʃu w min bta-ʕref b-hal-balad?\\
what and who 2.{\SG}-know in-this-country\\
\glt `What and who(m) do you know in this country?’}

\ex[]
{\gll Amta w kif ʕam t-rooħ-o ʕa-l-masbaħ?\\
when and how \textsc{prog} 2-going-{\PL} to-the-swimming.pool\\
\glt `When and how are you going to the swimming pool?’}  
\z
\hfill {(Lebanese Arabic)}
\z

\noindent Needless to say, coordinating  ``real" clitics (e.g., Bosnian/Croatian/Serbian (BCS) pronominal clitics in (\ref{zan:ex: BCS})) is out of the question. Though the violations in (\ref{zan:ex: wh-coordination}) are not nearly as bad as in BCS (\ref{zan:ex: BCS}), my informants are consistent in assigning the value $\le$ 3 (out of 5) to the coordinated indefinites.  

\ea[*] 
{\gll Poklonila sam mu i ga. \\
gifted am him and it\\ 
\glt Intended: `I gifted it to him.'\hfill {(BCS)}\label{zan:ex: BCS}}
\z

 \noindent All of this is to say that although coordination should not be taken as the sole diagnostic of clitichood, when combined with earlier observations on the dependent prosodic status of wh-indefinites, it may prove to be of explanatory value.\footnote{See \citet{citko2013puzzles} and references therein for a discussion of syntactic factors involved in deriving non-standard coordination dependencies.} Considering also that the mechanism is available to all types of quantified elements in Russian, the baseline assumption would situate wh-indefinites within the same array of elements that are amenable to coordination in principle. Their outlier behavior is hence best accommodated by appealing to their peculiar prosodic status.

 In short, the contention here is that wh-indefinites are reminiscent of (albeit not fully tantamount to) clitics. They require adjacency to a prosodic host, but show flexibility in alignment (left or right). They form clusters which can nevertheless be broken under the right conditions. Within clusters, they tend to appear in a particular order, the violations of which are merely dispreferred (rather than fully unacceptable). Finally, unlike other quantified elements, they are far from ideal when coordinated. All of these properties, in turn, suggest that the binary division into clitic vs. non-clitic is too rigid. There must be room to accommodate items like wh-indefinites, which are not quite clitics proper but neither are they tonic forms. In other words, clitic$\longleftrightarrow$non-clitic represents a scale, with elements occupying various intermediate positions within this continuum.\footnote{Reviewer 1 points out that this property renders them rather akin to weak pronouns in the sense of \citet{CardinalettiStarke1999}} 



\section{Conclusion}

Wh-indefinites are ``not quite" elements: not quite clitics, they require a weakly negative context, created by a clausebounded operator. They can always be replaced with \textit{nibud'}-indefinites, but not \textit{vice versa}. This ``in-between" status correlates with bare morphology: while very local (antiveridical) and superordinate (nonveridical) operators call for full morphological specification (\textit{ni}- or \textit{nibud'}, respectively), the medial ones admit such morphologically deficient elements under certain circumstances. Though I have attempted to catalog what these circumstances are, it would be obviously desirable to uncover a unifying semantic mechanism that ensures the felicity of wh-indefinites in all the contexts from \sectref{zan:sec:licensing contexts}. Dwelling on the topic of further desideratum, it would be productive to establish specific phonetic correlates that underlie the weak prosodic status of non-polar elements within the clitic--non-clitic continuum. 

With these caveats aside, the basic findings are as follows. First, wh-indefinites are possible in a proper subset of \textit{nibud'}-indefinites. Encountered most frequently in polar interrogatives, in conditional antecedents, and under weak epistemic verbs, they can also be introduced in desiderative/root subjunctives, imperatives, iterative and future contexts, as well as under strong epistemics and attitude predicates. However, all the latter (``non-standard") contexts require further modification to render the indefinites happy -- either a (subjective) epistemic or a scalar adverb. Second, Russian wh-indefinites occupy a peculiar PF niche: not only do they resist contrastive focalization (a well-established fact), they evince additional properties consistent with the typical behavior of clitics. Said properties include their preference for clustering (and a specific order within the clusters), their selectivity of hosts, and their inability to coordinate.       


%For a start: Do not forget to give your Overleaf project (this paper) a recognizable name. This one could be called, for instance, Simik et al: OSL template. You can change the name of the project by hovering over the gray title at the top of this page and clicking on the pencil icon.

\section{Introduction}\label{sim:sec:intro}

Language Science Press is a project run for linguists, but also by linguists. You are part of that and we rely on your collaboration to get at the desired result. Publishing with LangSci Press might mean a bit more work for the author (and for the volume editor), esp. for the less experienced ones, but it also gives you much more control of the process and it is rewarding to see the quality result.

Please follow the instructions below closely, it will save the volume editors, the series editors, and you alike a lot of time.

\sloppy This stylesheet is a further specification of three more general sources: (i) the Leipzig glossing rules \citep{leipzig-glossing-rules}, (ii) the generic style rules for linguistics (\url{https://www.eva.mpg.de/fileadmin/content_files/staff/haspelmt/pdf/GenericStyleRules.pdf}), and (iii) the Language Science Press guidelines \citep{Nordhoff.Muller2021}.\footnote{Notice the way in-text numbered lists should be written -- using small Roman numbers enclosed in brackets.} It is advisable to go through these before you start writing. Most of the general rules are not repeated here.\footnote{Do not worry about the colors of references and links. They are there to make the editorial process easier and will disappear prior to official publication.}

Please spend some time reading through these and the more general instructions. Your 30 minutes on this is likely to save you and us hours of additional work. Do not hesitate to contact the editors if you have any questions.

\section{Illustrating OSL commands and conventions}\label{sim:sec:osl-comm}

Below I illustrate the use of a number of commands defined in langsci-osl.tex (see the styles folder).

\subsection{Typesetting semantics}\label{sim:sec:sem}

See below for some examples of how to typeset semantic formulas. The examples also show the use of the sib-command (= ``semantic interpretation brackets''). Notice also the the use of the dummy curly brackets in \REF{sim:ex:quant}. They ensure that the spacing around the equation symbol is correct. 

\ea \ea \sib{dog}$^g=\textsc{dog}=\lambda x[\textsc{dog}(x)]$\label{sim:ex:dog}
\ex \sib{Some dog bit every boy}${}=\exists x[\textsc{dog}(x)\wedge\forall y[\textsc{boy}(y)\rightarrow \textsc{bit}(x,y)]]$\label{sim:ex:quant}
\z\z

\noindent Use noindent after example environments (but not after floats like tables or figures).

And here's a macro for semantic type brackets: The expression \textit{dog} is of type $\stb{e,t}$. Don't forget to place the whole type formula into a math-environment. An example of a more complex type, such as the one of \textit{every}: $\stb{s,\stb{\stb{e,t},\stb{e,t}}}$. You can of course also use the type in a subscript: dog$_{\stb{e,t}}$

We distinguish between metalinguistic constants that are translations of object language, which are typeset using small caps, see \REF{sim:ex:dog}, and logical constants. See the contrast in \REF{sim:ex:speaker}, where \textsc{speaker} (= serif) in \REF{sim:ex:speaker-a} is the denotation of the word \textit{speaker}, and \cnst{speaker} (= sans-serif) in \REF{sim:ex:speaker-b} is the function that maps the context $c$ to the speaker in that context.\footnote{Notice that both types of small caps are automatically turned into text-style, even if used in a math-environment. This enables you to use math throughout.}$^,$\footnote{Notice also that the notation entails the ``direct translation'' system from natural language to metalanguage, as entertained e.g. in \citet{Heim.Kratzer1998}. Feel free to devise your own notation when relying on the ``indirect translation'' system (see, e.g., \citealt{Coppock.Champollion2022}).}

\ea\label{sim:ex:speaker}
\ea \sib{The speaker is drunk}$^{g,c}=\textsc{drunk}\big(\iota x\,\textsc{speaker}(x)\big)$\label{sim:ex:speaker-a}
\ex \sib{I am drunk}$^{g,c}=\textsc{drunk}\big(\cnst{speaker}(c)\big)$\label{sim:ex:speaker-b}
\z\z

\noindent Notice that with more complex formulas, you can use bigger brackets indicating scope, cf. $($ vs. $\big($, as used in \REF{sim:ex:speaker}. Also notice the use of backslash plus comma, which produces additional space in math-environment.

\subsection{Examples and the minsp command}

Try to keep examples simple. But if you need to pack more information into an example or include more alternatives, you can resort to various brackets or slashes. For that, you will find the minsp-command useful. It works as follows:

\ea\label{sim:ex:german-verbs}\gll Hans \minsp{\{} schläft / schlief / \minsp{*} schlafen\}.\\
Hans {} sleeps {} slept {} {} sleep.\textsc{inf}\\
\glt `Hans \{sleeps / slept\}.'
\z

\noindent If you use the command, glosses will be aligned with the corresponding object language elements correctly. Notice also that brackets etc. do not receive their own gloss. Simply use closed curly brackets as the placeholder.

The minsp-command is not needed for grammaticality judgments used for the whole sentence. For that, use the native langsci-gb4e method instead, as illustrated below:

\ea[*]{\gll Das sein ungrammatisch.\\
that be.\textsc{inf} ungrammatical\\
\glt Intended: `This is ungrammatical.'\hfill (German)\label{sim:ex:ungram}}
\z

\noindent Also notice that translations should never be ungrammatical. If the original is ungrammatical, provide the intended interpretation in idiomatic English.

If you want to indicate the language and/or the source of the example, place this on the right margin of the translation line. Schematic information about relevant linguistic properties of the examples should be placed on the line of the example, as indicated below.

\ea\label{sim:ex:bailyn}\gll \minsp{[} Ėtu knigu] čitaet Ivan \minsp{(} často).\\
{} this book.{\ACC} read.{\PRS.3\SG} Ivan.{\NOM} {} often\\\hfill O-V-S-Adv
\glt `Ivan reads this book (often).'\hfill (Russian; \citealt[4]{Bailyn2004})
\z

\noindent Finally, notice that you can use the gloss macros for typing grammatical glosses, defined in langsci-lgr.sty. Place curly brackets around them.

\subsection{Citation commands and macros}

You can make your life easier if you use the following citation commands and macros (see code):

\begin{itemize}
    \item \citealt{Bailyn2004}: no brackets
    \item \citet{Bailyn2004}: year in brackets
    \item \citep{Bailyn2004}: everything in brackets
    \item \citepossalt{Bailyn2004}: possessive
    \item \citeposst{Bailyn2004}: possessive with year in brackets
\end{itemize}

\section{Trees}\label{s:tree}

Use the forest package for trees and place trees in a figure environment. \figref{sim:fig:CP} shows a simple example.\footnote{See \citet{VandenWyngaerd2017} for a simple and useful quickstart guide for the forest package.} Notice that figure (and table) environments are so-called floating environments. {\LaTeX} determines the position of the figure/table on the page, so it can appear elsewhere than where it appears in the code. This is not a bug, it is a property. Also for this reason, do not refer to figures/tables by using phrases like ``the table below''. Always use tabref/figref. If your terminal nodes represent object language, then these should essentially correspond to glosses, not to the original. For this reason, we recommend including an explicit example which corresponds to the tree, in this particular case \REF{sim:ex:czech-for-tree}.

\ea\label{sim:ex:czech-for-tree}\gll Co se řidič snažil dělat?\\
what {\REFL} driver try.{\PTCP.\SG.\MASC} do.{\INF}\\
\glt `What did the driver try to do?'
\z

\begin{figure}[ht]
% the [ht] option means that you prefer the placement of the figure HERE (=h) and if HERE is not possible, you prefer the TOP (=t) of a page
% \centering
    \begin{forest}
    for tree={s sep=1cm, inner sep=0, l=0}
    [CP
        [DP
            [what, roof, name=what]
        ]
        [C$'$
            [C
                [\textsc{refl}]
            ]
            [TP
                [DP
                    [driver, roof]
                ]
                [T$'$
                    [T [{[past]}]]
                    [VP
                        [V
                            [tried]
                        ]
                        [VP, s sep=2.2cm
                            [V
                                [do.\textsc{inf}]
                            ]
                            [t\textsubscript{what}, name=trace-what]
                        ]
                    ]
                ]
            ]
        ]
    ]
    \draw[->,overlay] (trace-what) to[out=south west, in=south, looseness=1.1] (what);
    % the overlay option avoids making the bounding box of the tree too large
    % the looseness option defines the looseness of the arrow (default = 1)
    \end{forest}
    \vspace{3ex} % extra vspace is added here because the arrow goes too deep to the caption; avoid such manual tweaking as much as possible; here it's necessary
    \caption{Proposed syntactic representation of \REF{sim:ex:czech-for-tree}}
    \label{sim:fig:CP}
\end{figure}

Do not use noindent after figures or tables (as you do after examples). Cases like these (where the noindent ends up missing) will be handled by the editors prior to publication.

\section{Italics, boldface, small caps, underlining, quotes}

See \citet{Nordhoff.Muller2021} for that. In short:

\begin{itemize}
    \item No boldface anywhere.
    \item No underlining anywhere (unless for very specific and well-defined technical notation; consult with editors).
    \item Small caps used for (i) introducing terms that are important for the paper (small-cap the term just ones, at a place where it is characterized/defined); (ii) metalinguistic translations of object-language expressions in semantic formulas (see \sectref{sim:sec:sem}); (iii) selected technical notions.
    \item Italics for object-language within text; exceptionally for emphasis/contrast.
    \item Single quotes: for translations/interpretations
    \item Double quotes: scare quotes; quotations of chunks of text.
\end{itemize}

\section{Cross-referencing}

Label examples, sections, tables, figures, possibly footnotes (by using the label macro). The name of the label is up to you, but it is good practice to follow this template: article-code:reference-type:unique-label. E.g. sim:ex:german would be a proper name for a reference within this paper (sim = short for the author(s); ex = example reference; german = unique name of that example).

\section{Syntactic notation}

Syntactic categories (N, D, V, etc.) are written with initial capital letters. This also holds for categories named with multiple letters, e.g. Foc, Top, Num, etc. Stick to this convention also when coming up with ad hoc categories, e.g. Cl (for clitic or classifier).

An exception from this rule are ``little'' categories, which are written with italics: \textit{v}, \textit{n}, \textit{v}P, etc.

Bar-levels must be typeset with bars/primes, not with an apostrophe. An easy way to do that is to use mathmode for the bar: C$'$, Foc$'$, etc.

Specifiers should be written this way: SpecCP, Spec\textit{v}P.

Features should be surrounded by square brackets, e.g., [past]. If you use plus and minus, be sure that these actually are plus and minus, and not e.g. a hyphen. Mathmode can help with that: [$+$sg], [$-$sg], [$\pm$sg]. See \sectref{sim:sec:hyphens-etc} for related information.

\section{Footnotes}

Absolutely avoid long footnotes. A footnote should not be longer than, say, {20\%} of the page. If you feel like you need a long footnote, make an explicit digression in the main body of the text.

Footnotes should always be placed at the end of whole sentences. Formulate the footnote in such a way that this is possible. Footnotes should always go after punctuation marks, never before. Do not place footnotes after individual words. Do not place footnotes in examples, tables, etc. If you have an urge to do that, place the footnote to the text that explains the example, table, etc.

Footnotes should always be formulated as full, self-standing sentences.

\section{Tables}

For your tables use the table plus tabularx environments. The tabularx environment lets you (and requires you in fact) to specify the width of the table and defines the X column (left-alignment) and the Y column (right-alignment). All X/Y columns will have the same width and together they will fill out the width of the rest of the table -- counting out all non-X/Y columns.

Always include a meaningful caption. The caption is designed to appear on top of the table, no matter where you place it in the code. Do not try to tweak with this. Tables are delimited with lsptoprule at the top and lspbottomrule at the bottom. The header is delimited from the rest with midrule. Vertical lines in tables are banned. An example is provided in \tabref{sim:tab:frequencies}. See \citet{Nordhoff.Muller2021} for more information. If you are typesetting a very complex table or your table is too large to fit the page, do not hesitate to ask the editors for help.

\begin{table}
\caption{Frequencies of word classes}
\label{sim:tab:frequencies}
 \begin{tabularx}{.77\textwidth}{lYYYY} %.77 indicates that the table will take up 77% of the textwidth
  \lsptoprule
            & nouns & verbs  & adjectives & adverbs\\
  \midrule
  absolute  &   12  &    34  &    23      & 13\\
  relative  &   3.1 &   8.9  &    5.7     & 3.2\\
  \lspbottomrule
 \end{tabularx}
\end{table}

\section{Figures}

Figures must have a good quality. If you use pictorial figures, consult the editors early on to see if the quality and format of your figure is sufficient. If you use simple barplots, you can use the barplot environment (defined in langsci-osl.sty). See \figref{sim:fig:barplot} for an example. The barplot environment has 5 arguments: 1. x-axis desription, 2. y-axis description, 3. width (relative to textwidth), 4. x-tick descriptions, 5. x-ticks plus y-values.

\begin{figure}
    \centering
    \barplot{Type of meal}{Times selected}{0.6}{Bread,Soup,Pizza}%
    {
    (Bread,61)
    (Soup,12)
    (Pizza,8)
    }
    \caption{A barplot example}
    \label{sim:fig:barplot}
\end{figure}

The barplot environment builds on the tikzpicture plus axis environments of the pgfplots package. It can be customized in various ways. \figref{sim:fig:complex-barplot} shows a more complex example.

\begin{figure}
  \begin{tikzpicture}
    \begin{axis}[
	xlabel={Level of \textsc{uniq/max}},  
	ylabel={Proportion of $\textsf{subj}\prec\textsf{pred}$}, 
	axis lines*=left, 
        width  = .6\textwidth,
	height = 5cm,
    	nodes near coords, 
    % 	nodes near coords style={text=black},
    	every node near coord/.append style={font=\tiny},
        nodes near coords align={vertical},
	ymin=0,
	ymax=1,
	ytick distance=.2,
	xtick=data,
	ylabel near ticks,
	x tick label style={font=\sffamily},
	ybar=5pt,
	legend pos=outer north east,
	enlarge x limits=0.3,
	symbolic x coords={+u/m, \textminus u/m},
	]
	\addplot[fill=red!30,draw=none] coordinates {
	    (+u/m,0.91)
        (\textminus u/m,0.84)
	};
	\addplot[fill=red,draw=none] coordinates {
	    (+u/m,0.80)
        (\textminus u/m,0.87)
	};
	\legend{\textsf{sg}, \textsf{pl}}
    \end{axis} 
  \end{tikzpicture} 
    \caption{Results divided by \textsc{number}}
    \label{sim:fig:complex-barplot}
\end{figure}

\section{Hyphens, dashes, minuses, math/logical operators}\label{sim:sec:hyphens-etc}

Be careful to distinguish between hyphens (-), dashes (--), and the minus sign ($-$). For in-text appositions, use only en-dashes -- as done here -- with spaces around. Do not use em-dashes (---). Using mathmode is a reliable way of getting the minus sign.

All equations (and typically also semantic formulas, see \sectref{sim:sec:sem}) should be typeset using mathmode. Notice that mathmode not only gets the math signs ``right'', but also has a dedicated spacing. For that reason, never write things like p$<$0.05, p $<$ 0.05, or p$<0.05$, but rather $p<0.05$. In case you need a two-place math or logical operator (like $\wedge$) but for some reason do not have one of the arguments represented overtly, you can use a ``dummy'' argument (curly brackets) to simulate the presence of the other one. Notice the difference between $\wedge p$ and ${}\wedge p$.

In case you need to use normal text within mathmode, use the text command. Here is an example: $\text{frequency}=.8$. This way, you get the math spacing right.

\section{Abbreviations}

The final abbreviations section should include all glosses. It should not include other ad hoc abbreviations (those should be defined upon first use) and also not standard abbreviations like NP, VP, etc.


\section{Bibliography}

Place your bibliography into localbibliography.bib. Important: Only place there the entries which you actually cite! You can make use of our OSL bibliography, which we keep clean and tidy and update it after the publication of each new volume. Contact the editors of your volume if you do not have the bib file yet. If you find the entry you need, just copy-paste it in your localbibliography.bib. The bibliography also shows many good examples of what a good bibliographic entry should look like.

See \citet{Nordhoff.Muller2021} for general information on bibliography. Some important things to keep in mind:

\begin{itemize}
    \item Journals should be cited as they are officially called (notice the difference between and, \&, capitalization, etc.).
    \item Journal publications should always include the volume number, the issue number (field ``number''), and DOI or stable URL (see below on that).
    \item Papers in collections or proceedings must include the editors of the volume (field ``editor''), the place of publication (field ``address'') and publisher.
    \item The proceedings number is part of the title of the proceedings. Do not place it into the ``volume'' field. The ``volume'' field with book/proceedings publications is reserved for the volume of that single book (e.g. NELS 40 proceedings might have vol. 1 and vol. 2).
    \item Avoid citing manuscripts as much as possible. If you need to cite them, try to provide a stable URL.
    \item Avoid citing presentations or talks. If you absolutely must cite them, be careful not to refer the reader to them by using ``see...''. The reader can't see them.
    \item If you cite a manuscript, presentation, or some other hard-to-define source, use the either the ``misc'' or ``unpublished'' entry type. The former is appropriate if the text cited corresponds to a book (the title will be printed in italics); the latter is appropriate if the text cited corresponds to an article or presentation (the title will be printed normally). Within both entries, use the ``howpublished'' field for any relevant information (such as ``Manuscript, University of \dots''). And the ``url'' field for the URL.
\end{itemize}

We require the authors to provide DOIs or URLs wherever possible, though not without limitations. The following rules apply:

\begin{itemize}
    \item If the publication has a DOI, use that. Use the ``doi'' field and write just the DOI, not the whole URL.
    \item If the publication has no DOI, but it has a stable URL (as e.g. JSTOR, but possibly also lingbuzz), use that. Place it in the ``url'' field, using the full address (https: etc.).
    \item Never use DOI and URL at the same time.
    \item If the official publication has no official DOI or stable URL (related to the official publication), do not replace these with other links. Do not refer to published works with lingbuzz links, for instance, as these typically lead to the unpublished (preprint) version. (There are exceptions where lingbuzz or semanticsarchive are the official publication venue, in which case these can of course be used.) Never use URLs leading to personal websites.
    \item If a paper has no DOI/URL, but the book does, do not use the book URL. Just use nothing.
\end{itemize}

\section*{Abbreviations}

\begin{tabularx}{.5\textwidth}{@{}lQ}
\textsc{2}&second person\\
\textsc{3}&third person\\
\textsc{foc}&focus\\
\textsc{imp}&imperative\\
\textsc{indf}&indefinite\\
\textsc{ipfv}&imperfective\\
\textsc{m}&masculine\\
\textsc{neg}&negation\\
\end{tabularx}%
\begin{tabularx}{.5\textwidth}{lQ@{}}
\textsc{pfv}&perfective\\
\textsc{pl}&plural\\
\textsc{pst}&past\\
\textsc{prog}&progressive\\
\textsc{q}&question marker\\
\textsc{rel}&relative\\
\textsc{sbjv}&subjunctive\\
\textsc{sg}&singular\\

%&\\ % this dummy row achieves correct vertical alignment of both tables
\end{tabularx}

\section*{Acknowledgments}
I am grateful to the audience of FDSL 15 as well as the two reviewers for helpful comments and suggestions.  

\printbibliography[heading=subbibliography,notkeyword=this]

\end{document}
