\documentclass[output=paper,colorlinks,citecolor=brown]{langscibook}
\ChapterDOI{10.5281/zenodo.15394189}
%\bibliography{localbibliography}

\author{Maria Onoeva\affiliation{Charles University} and Anna Staňková\affiliation{Charles University}}
% replace the above with you and your coauthors
% rules for affiliation: If there's an official English version, use that (find out on the official website of the university); if not, use the original
% orcid doesn't appear printed; it's metainformation used for later indexing

%%% uncomment the following line if you are a single author or all authors have the same affiliation
\SetupAffiliations{mark style=none}

%% in case the running head with authors exceeds one line (which is the case in this example document), use one of the following methods to turn it into a single line; otherwise comment the line below out with % and ignore it
% \lehead{Onoeva \& Staňková}
% \lehead{Radek Šimík et al.}

\title{Polar questions in Czech and Russian: An exploratory corpus investigation}
% replace the above with your paper title
%%% provide a shorter version of your title in case it doesn't fit a single line in the running head
% in this form: \title[short title]{full title}
%\abstract{The topic of polar questions has received considerable attention and has been investigated theoretically as well as experimentally. For example, the complex differences in meaning between \textit{Is Jane coming?}, \textit{Isn't Jane coming?} and \textit{Is Jane not coming?} in English puzzled many researchers. Our study attempts to address these and additional phenomena related to polar questions in two Slavic languages -- Czech and Russian. We examine the formal and semantic/pragmatic features in samples from spoken corpora. This method allowed us to observe some tendencies of polar questions usage, although it was not without difficulty. Based on our material and results, we suggest directions for further research.

\abstract{This study aims to bring new insights into the topic of polar questions in Czech and Russian based on corpus data. What is of particular interest are the complex differences in meaning among the Czech and Russian counterparts of English structures such as \textit{Is Jane coming?}, \textit{Isn't Jane coming?} and \textit{Is Jane not coming?}. We examine the formal and semantic/pragmatic features of polar questions in these two Slavic languages, namely word order, presence and position of negation, presence of question tags, presence of question particles, and their relation to the question's meaning and its bias towards a possible answer. Using authentic data from spoken corpora allowed us to observe some prominent tendencies of polar questions usage. 

\keywords{polar questions, bias, corpus, Czech, Russian}
}

\begin{document}
\maketitle

\section{Introduction}

Polar questions (PQs) have been widely studied from different points of view in recent years.\footnote{The term ``question'' is related rather to a speech act, whilst ``interrogative'' is used for syntactic and semantic descriptions. In this paper, however, we stick to the term ``polar question'' as it is more frequent in the literature.} A number of researchers brought important insights about how the meaning of PQs is affected by negation (e.g. \citealt{Buering2000, Romero2004, ono+:Repp2013, ono+:AnderBois2019}), word order (e.g. \citealt{ono+:Gunlogson2002,}), particles (e.g. \citealt{Sudo2013, Frana2019, Gaertner2022, Gonzalez2023}), intonation or focus (e.g. \citealt{Gyuris2019, Rudin2022, Goodhue2022a}), and other phenomena. The goal of this paper is to contribute to the topic from the perspective of Slavic languages, namely Czech and Russian.

There are two main strategies how to ask a PQ in these languages -- overt and intonational \citep{Simik2023}. In Czech, a PQ can be constructed by interrogative word order \citep{Sticha1995a}, which involves the finite verb preceding an overt subject, as in \REF{ex-cz-wo}. The second strategy is using intonation, either the rise or fall-rise pattern \citep{Danes1987, Palkova1994}. Thanks to this, declarative sentences can be interpreted as PQs, as shown in \REF{ex-cz-int}. The results of the experiment run by \citet{stankovaexpression2023} showed that the choice between these two strategies (interrogative vs. declarative) can be motivated by the presence of evidential bias (more on bias in \sectref{sec-semantic-pragmatic-features}).

\ea \ea \gll Koupil si Petr auto? \\ 
bought \textsc{refl} Petr car \\
\glt `Did Petr buy a car?' 
\label{ex-cz-wo}
\ex \gll  Petr si koupil auto? \\ 
Petr \textsc{refl} bought car \\
\glt `Petr bought a car?' 
\label{ex-cz-int}\hfill(Cz)
\z
\z


\noindent In Russian, the overt strategy is to place the particle \textit{li} after the first phonological word as in \REF{ex-ru1a}. Any word can appear with it and then be in the question focus \citep{King1994}. The intonational strategy is shown in \REF{ex-ru1c}. For out-of-the-blue PQs, word order is declarative but the pitch locus (a steep rise and an immediate fall; \textsc{Q-peak} by \citealt{Esipova2024}) is placed at the verb, here it is \textit{vyigrala} `won', whereas in statements it is usually placed at the most deeply embedded argument (\textit{priz} `prize' in this case) \citep{Meyer2006, Rathcke2006}. \citet{Schwabe2004} and \citet{Brown1995} mention the markedness of \textit{li} in main clauses and its ongoing loss in colloquial Russian. Nevertheless, \textit{li} must be still present in embedded PQs, \REF{ex-ru1b}. \citet{Esipova2024a} argue that \textit{li}-PQs simply present two alternatives and thus are true neutral questions, whereas intonation PQs convey pressure to respond.  

\ea \ea \gll Vyigrala li Daša priz? \\ 
won \textsc{li} Daša prize \\
\glt `Did Daša win a prize?' 
\label{ex-ru1a}
\ex \gll Daša vyigrala priz? \\ 
Daša  won  prize \\
\glt `Did Daša win a prize?' 
\label{ex-ru1c}
\ex \gll  Ja ne znaju, vyigrala li Daša priz. \\ 
I not know won \textsc{li} Daša prize \\
\glt `I don't know whether Daša won the prize.' 
\label{ex-ru1b}\hfill (Ru)
\z\z

\noindent In the present study, we looked at PQs in general through the lens of corpus data. Besides the above-described features, Czech and Russian questions could contain various elements which directly influence their meaning, such as indefinites, different particles, tags, negation etc. Due to the limitations of the Russian corpus, prosody was not taken into account. Answers to PQs were also laid aside. It was an exploratory study, in which we addressed the following research questions: 

\begin{enumerate}
    \item What are the formal properties of PQs in real communication? 
    \item Besides the core interrogative semantics, what semantic/pragmatic implications do PQs have? 
    \item Are there any correlations between the formal and semantic/pragmatic aspect?
\end{enumerate}

To answer the first research question, we annotated each PQ with respect to its structure. For the second research question, we explored question biases \citep{Buering2000, Sudo2013, Gaertner2017} and their distribution among Czech and Russian PQs. The third research question was to check if there is any relation between their form and meaning.

The paper is organized as follows. In \sectref{section-method}, we describe the method of annotation. \sectref{section-results} reports on the absolute values of the annotated features and the results of the inter-annotator agreement. In \sectref{sec-discussion}, we discuss the results. \sectref{sec-conclusion} concludes the paper. 

\section{Method} \label{section-method}
% \begin{itemize}
%     \item Ru and Cz spoken corpora, no audio for Ru, random sample
%     \item 500 x 2
%     \item annotated features 
%     \begin{itemize}
%         \item formal (A), word order, tag (?) literature
%         \item prejacent (two examples Ru and Cz, glosses), judging from context, sem/prag, epistemic and evidential biases (M), literature, what to do with variables SB, SE, SK 
%         \item examples of annotation 
%     \end{itemize}
%     \item manual 
%     \item 100 x 2, inter annotator agreement, why Russian was poor etc 
% \end{itemize}

In this section, we describe the method and procedure of the annotation, which were the same for both languages. We used the spoken corpus of the Russian National Corpus \citep{Grisina2005, Grisina2009} and the ORTOFON v2 corpus \citep{ortofon_v2} of the Czech National Corpus, the latter accessed via the KonText interface \citep{kontext_interface}. Both corpora contain informal everyday conversations with the option to display a limited context around the question. Audio is not available for the spoken part of the Russian corpus, thus intonation was not taken into account. 

For each language, a random sample of 500 instances was manually collected. We queried for the question mark and filtered out \textit{wh}-interrogatives. In order to address the first and second research questions, the annotated features were divided into two groups -- formal and semantic/pragmatic. They are described below.\footnote{The complete annotation is available here: \url{https://bit.ly/3xKM9XX}}

\subsection{Formal features}
We have already mentioned some of the formal features of PQs, such as the specific word order or usage of question particles. In our sample, we annotated word order with respect to the position of the verb -- it was either initial, medial or final. As for particles, for each one we marked their presence (`1' = present, `0' = absent). 

Previous research paid attention to negation in PQs because of non-trivial implications it involves (e.g. \citealt{ono+:Ladd1981, ono+:Repp2013}). It was claimed that there are two types of negation -- inner ($\approx$ semantic) and outer ($\approx$ pragmatic) -- and that they differ in their syntactic and semantic/pragmatic features. Inner negation is interpreted and licenses Negative Polarity Items [NPIs] (Negative Concord Items [NCIs] in Czech and Russian), whereas outer negation does not trigger the negative operator per se and licenses Positive Polarity Items [PPIs] (\cite{Romero2004}; cf. \cite{Goodhue2022a}). Based on these observations, we annotated our data set for the presence of negation as well as certain indefinites. For Czech, these were \textit{ně}-indefinites (considered as PPIs) and \textit{ni}-/\textit{žád}-indefinites (considered as NCIs). In Russian, they were \textit{-nibud', -to, koe-} indefinites and \textit{ni-} NCIs.

The last annotated formal feature was the question tag. Tag questions consist of an anchor (the PQ) and a tag. There are different types of tags based on their polarity \citep[e.g.][]{krifka2015bias}. The first type agrees in polarity with the PQ (matching tags), the second type is of the opposite polarity than the PQ (reverse tags). Moreover, tags can differ in their intonation patterns \citep{ono+:Ladd1981}. In our annotation, we marked their presence, but did not distinguish them any further.

In \REF{ex-formal-annotation}, we provide an example of a Czech PQ annotated from a formal point of view for all the features just mentioned.

\begin{exe}
\ex \label{ex-formal-annotation} \gll Snad ho teďka nebudeš stavět ne? \\
\textsc{snad} him now \textsc{neg}-will build no \\
\glt `You're not going to build it now, are you?' \hfill (Cz)
\begin{xlist}
    \exi{Formal annotation:} \textsc{wo}: xVx; \textsc{prt}: $1$ \textit{snad}; \textsc{neg}: $1$; \textsc{indef}: $0$; \textsc{tag}: $1$ \textit{ne}
\end{xlist}
\end{exe}

\subsection{Semantic/pragmatic features} \label{sec-semantic-pragmatic-features}
%%% TO DO 
% describe biases
% check transliteration (the web is down for some reason)
% describe interannotator 
After \citet{Hamblin1973}, \citet{Karttunen1977}, and \citet{Groenendijk1984}, the semantic interpretation of questions is represented as a set of their (true or possible) answers. In case of PQs, it can be simplified to $\{p, \neg p\}$ where $p$ is a question radical, e.g. for a question \textit{Is it raining?}, $p =$ \textit{it is raining}. 

%\footnote{\cite{ono+:AnderBois2019} suggests for negative PQs $\{\neg p, \neg \neg p\}$, hence to distinguish them from affirmative.} 

% Then PQs might have additional meanings/prompts/hints/implications favoring a particular reply

Aside from that the structure of PQs may indicate a certain favor, or \textsc{bias}, towards a particular reply, which is not captured by the set of their possible answers. So far \textsc{speaker} (or \textsc{epistemic}) and \textsc{evidential} biases are recognized. They usually either support or oppose $p$. Speaker bias is based on prior and private speaker's beliefs,\footnote{\citet{Sudo2013} suggests that epistemic/speaker bias can also include deontic or bouletic states. Since these are too complex to judge based on written text, we only work with speaker's beliefs.} while evidential bias comes from contextual information available to all interlocutors. Not all PQs are equally biased, it is possible that one of the biases or both are absent. If no bias is present, the question is considered to be neutral. Different combinations of biases and their absence represent \textsc{bias profiles} of PQs and could be universal to specific question forms or particles \citep{Sudo2013, Gaertner2017}.

In our annotation, we also distinguished a third type of bias which was related to the speaker's awareness of the answer, and we refer to it as \textsc{knowledge} bias. If the speaker knows the answer for sure, the PQ is biased. This type of bias is sort of in between speaker and evidential biases. It typically occurs in exam \citep{Krifka2011} or surprise echo questions. 

To be able to investigate the bias profiles of Czech and Russian PQs, we manually constructed an affirmative prejacent $\phi$ for each question from their radicals. We performed the following steps to produce it: (i) remove negation if it is present, \REF{ex-prej1}; (ii) remove particles, question tags and other elements that do not appear in statements as in \REF{ex-prej2} and \REF{ex-prej3}; (iii) if the first or second person pronouns appear, replace them with `speaker' or `addressee' as in \REF{ex-prej4}.

\ea \ea \gll Není to kočka? \\
\textsc{neg}.is it cat \\
\glt `Isn't it a cat?' \hfill (Cz) \\
 $\phi={}$It's a cat. 
\label{ex-prej1}

\ex \gll A Daník tady bude ne? \\
and Daník here will.be no \\
\glt`And Daník is going to be here, isn't he?' \hfill (Cz)\\
 $\phi={}$ Daník is going to be here.
 \label{ex-prej2}

% I'll fix cyrillics later :) -- done
 \ex \gll Neuželi oni tože slyšat kak my rugaemsja? \\
 \textsc{neuželi} they also hear how we argue \\
 \glt`Do they also hear how we argue?' \hfill (Ru)\\
 $\phi={}$ They also hear how we argue. 
  \label{ex-prej3}
  
% check transliteration -- done
\ex \gll A u tebja pomimo sobački est' eščë kto-nibud'? \\ 
and at you besides doggie is else anyone \\
\glt `Do you have anyone else besides a doggie?'  \hfill (Ru)\\
$\phi={}$The addressee has someone else besides a doggie. 
 \label{ex-prej4}
\z\z

\noindent We used affirmative prejacents and not question radicals to decide whether or not speakers had any prior belief. The same applies for evidential bias. With the aid of prejacents, it was easier to judge the type of bias and its value in some controversial cases such as PQs with certain particles or outer negation cases, where it was not clear if the radical was affirmative or not.

Judgments about the biases were based on our intuition as native speakers, which were later compared with judgments from three additional annotators (see \sectref{sec-interan}). We always annotated the questions in some amount of context in order to detect evidential bias. Due to the limitations of the corpora, we were able to capture only linguistic cues of evidence. 

To construct the bias profiles, we assigned each bias one of the three values: `$1$', `$0$' and `$-1$'. The value `$1$' was assigned if the bias supported the affirmative prejacent. For instance, if the speaker believed that $\phi$ before posing the question, `$1$' was assigned to speaker bias. On the contrary, the value `$-1$' was assigned when the bias went against $\phi$ or, in the other words, supported that $\neg \phi$. E.g. if in the context there was a cue suggesting that $\neg \phi$, evidential bias for such cases was `$-1$'. The value `$0$' was assigned if no bias was detected. 

The following examples clarify the annotation of the bias profile. In the context of \REF{ex-rubiases}, it is mentioned who is the oldest brother and the youngest, Leonid and Aleksandr Aleksandrovič, respectively. The context supports the prejacent $\phi$, hence, the value assigned to evidential bias is `$1$'. The particle \textit{razve} indicates that the speaker's prior belief was that $\neg \phi$ \citep{Geist2023, Korotkova2023}, so Viktor believed that Aleksandr Aleksandrovič was not the youngest. The value assigned to speaker bias is `$-1$'. Since it is clear from the context in \REF{ex-rubiases} that the speaker now knows that $\phi$ (the speaker mentions the brothers' age difference explicitly), the value assigned to knowledge bias is `$1$'.

\largerpage
\begin{exe}
\ex Context: The addressee says her husband, aged 28 at the time, had two brothers: Leonid, 30, and Aleksandr Aleksandrovič, 27. The speaker asks:
    \begin{xlist}
    \exi{Sp:} \gll \minsp{\{} Kak / razve\} Aleksandr Aleksandrovič mladšij? \\
    {} how {} \textsc{razve} Aleksandr Aleksandrovič youngest \\
    \glt `Wait a second, is Aleksandr Aleksandrovič the youngest?' \hfill (Ru)\\
    $\phi={}$Aleksandr Aleksandrovič is the youngest. 
    \label{ex-rubiases}
    \exi{Semantic/pragmatic annotation:} \textsc{speaker} $-1$, \textsc{evidential} $1$, \textsc{knowledge}~$1$\hspace*{-1cm}
    \end{xlist}

\end{exe}

\noindent In \REF{ex-czbiasEPI}, the speaker has some prior belief when the addressee leaves because they explicitly say the time of leaving, so the epistemic bias value is `$1$'. There were no contextual cues, so the evidential and knowledge bias values are `$0$'.  

\begin{exe}
\ex Context: The speaker promised some sausage to the addressee but did not manage to bring it. They want to do it later and check when the addressee is available.
    \begin{xlist}
    \exi{Sp:} \gll Ty pojedeš kolem vosmý nějak? \\
    you leave around eight somehow \\
    \glt `Are you leaving around eight?' \hfill (Cz)\\
    $\phi={}$  The addressee is leaving around eight. 
    \label{ex-czbiasEPI}
    \exi{Semantic/pragmatic annotation:} \textsc{speaker} $1$, \textsc{evidential} $0$, \textsc{knowledge} $0$
    \end{xlist}

\end{exe}

\noindent In \REF{ex-czbiaseSE}, there is an example of evidential bias only. The speaker bias value is `$0$' because the speaker had no prior belief about the prejacent and guesses the number from the context. Since they do not know for sure how many cars were there, the value assigned to knowledge bias is `$0$'. 
\begin{exe}
\ex Context: The addressee lists how many people were with them on a trip. The speaker assumes the following from the provided number of people:
    \begin{xlist}
    \exi{Sp:} \gll Vy jste jeli třema autama? \\
    you \textsc{aux} went tree cars \\
    \glt `Did you guys travel in three cars?' \hfill (Cz)\\
    $\phi={}$ The addressee and the group travelled in 3 cars. 
    \label{ex-czbiaseSE}
    \exi{Semantic/pragmatic annotation:} \textsc{speaker} $0$, \textsc{evidential} $1$, \textsc{knowledge} $0$
    \end{xlist}

\end{exe}

\noindent The example in \REF{ex-rubiasKNOW} was annotated as carrying knowledge bias only. There is no prior belief about the cat being in the speaker's spot and no linguistic evidence of that.

\begin{exe}
\ex Context: The speaker sees their cat Ryžik sitting in the speaker's spot.
    \begin{xlist}
    \exi{Sp:} \gll Ryžik ty čto moë mesto zanjal? \\
    Ryžik you what my spot taken \\
    \glt `Ryžik, have you taken my spot?' \hfill (Ru)\\
    $\phi={}$  Ryžik has taken the speaker's spot.
    \label{ex-rubiasKNOW}
    \exi{Semantic/pragmatic annotation:} \textsc{speaker} $0$, \textsc{evidential} $0$, \textsc{knowledge} $1$
    \end{xlist}

\end{exe}

\section{Results} \label{section-results}
In this section, we outline the results of the study. We begin with the absolute values for both the formal and semantic/pragmatic features, then we report the results for Czech and Russian separately. The inter-annotator agreement report concludes the section.  

\subsection{Overall values}
Table \ref{tab-formal-numbers} summarizes the overall frequencies of the formal features we annotated. From the \textsc{neg} column follows that, out of 500 PQs, only 89 Czech and 79 Russian were negated. Tag questions were much more frequent in Czech (154) than in Russian (46). 

The occurrence numbers of the possible verb positions are in the third column. For Czech, the medial position was the most frequent one (220), followed by initial (150) and final (97). For Russian, we have found fewer PQs with the verb placed at the initial position (53), medial was the second frequent option (148), the most popular was final (237). Some cases were excluded from the final analysis because they were not full sentences (e.g. only subject or object with no predicate). 


\begin{table}
\caption{Formal features}
\label{tab-formal-numbers}
 \begin{tabularx}{.6\textwidth}{lrrrYY} %.77 indicates that the table will take up 77% of the textwidth
  \lsptoprule
            & \small{\textsc{neg}} & \small{\textsc{tag}}  & \multicolumn{3}{c}{\small{\textsc{verb position}}} \\
            \cmidrule{4-6}
            & \small{\textsc{1}} & \small{\textsc{1}}  & \small{\textsc{ini}} & \small{\textsc{med}} & \small{\textsc{fin}}\\
 \midrule
  Czech    &   89  & 154    & 150    & 220   & 97 \\
  Russian  &   79  & 46     & 53     & 148   & 237 \\
  \lspbottomrule
 \end{tabularx}
\end{table}

The distribution of the biases in the samples is reported in Table \ref{tab-semprag-numbers}. The most striking difference between the languages is present in the column \textsc{speaker} with the value `$1$' representing speaker bias that $\phi$ (104 instances for Czech and only 35 for Russian). Aside from that the other bias values were distributed equally in both languages. 

\begin{table}
\caption{The distribution of the biases in the samples}
\label{tab-semprag-numbers}
 \begin{tabularx}{.8\textwidth}{lrrr @{}X@{} rrr @{}X@{} rrr} %.77 indicates that the table will take up 77% of the textwidth
 \lsptoprule
            & \multicolumn{3}{c}{\small{\textsc{speaker}}} && \multicolumn{3}{c}{\small{\textsc{evidential}}}  && \multicolumn{3}{c}{\small{\textsc{knowledge}}} \\

            \cmidrule(lr){2-4}\cmidrule(lr){6-8}\cmidrule(lr){10-12}
            & \small{\textsc{0}} & \small{\textsc{1}} & \small{\textsc{-1}}& & \small{\textsc{0}} & \small{\textsc{1}} & \small{\textsc{-1}}& & \small{\textsc{0}} & \small{\textsc{1}} & \small{\textsc{-1}} \\
 \midrule
  Czech    &   379  & 104  & 17 & &  353  & 113  & 34 & & 446  & 45  & 9 \\
  Russian  &   453  & 35   & 12 & & 364   & 109  & 27 & & 465  & 31  & 4 \\
  \lspbottomrule
 \end{tabularx}
\end{table}

In the next subsections, we focus on particular form-meaning correlations in Czech and Russian PQs, respectively. For purposes of the following analyses, we pooled non-null speaker bias and evidential bias and excluded the few exceptional cases where the polarity of speaker/evidential bias was not in accord with the question's polarity.

\subsection{Czech}
The two form-meaning correlations we zoomed in on were: (i) the presence of a \textsc{tag} and the value of speaker bias (\textsc{sb}), and (ii) verb position and the value of evidential bias (\textsc{eb}). According to the null hypothesis, the variables in the pairs (i) and (ii) are independent of each other. According to the alternative hypothesis, the variables in those two pairs correlate. To test the alternative hypothesis for each pair, we ran two chi-square tests on the Czech data. Since there were two tests run on the data, we used Bonferroni correction ($\alpha$ divided by $n$, where $n$ is the number of tests) to adjust the alpha level. The adjusted alpha level was $0.025$ ($0.05$ divided by $2$).\footnote{We thank an anonymous reviewer for pointing out the inconsistencies in our reporting of the results.} We report the absolute values in the contingency tables \ref{tab-cz-tag} and \ref{tab-cz-verb}. Values expected in the case the null hypothesis is true are in brackets. Now we comment on the two pairs of variables individually.

As mentioned above, tag questions occurred frequently in the Czech data set, and mostly they (the anchor) exhibited declarative word order. Table \ref{tab-cz-tag} shows the correlation between tag PQs and speaker bias, i.e. previous beliefs of the speaker; which was statistically significant ($\chi^2(1)=120.9; p<.001$). For non-null speaker bias (= biased PQs), tags were present in 77 cases, even though the expected value by the null hypothesis was 31.7. For null speaker bias, the expected value was 120.3, but tags occurred only 75 times. 


\begin{table}
\caption{Tag--belief correlation ($p<.001$)}
\label{tab-cz-tag}
 \begin{tabularx}{.5\textwidth}{lrrrr} %.77 indicates that the table will take up 77% of the textwidth
  \lsptoprule
            & \multicolumn{2}{c}{\textsc{no tag}} & \multicolumn{2}{c}{\textsc{tag}} \\
  \midrule
  \textsc{sb $\pm 1$}   &21 & (66.3)     &77 & (31.7) \\
  \textsc{sb 0}         &297&  (251.7)   &75 & (120.3) \\
  \lspbottomrule
 \end{tabularx}
\end{table}

In our sample, initial verb position (\textsc{V ini}) negatively correlates with non-null evidential bias, and this correlation was significant ($\chi^2(1)=11.8; p<.001$). This is apparent from Table \ref{tab-cz-verb}, where the expected value of \textsc{v ini} in non-null \textsc{eb} PQs is higher (42.6) than the actual value (27). What follows from this is that the interrogative word order is preferred in PQs occurring in neutral context, i.e. those, where evidential bias equals `0'. When it was non-null, the PQ exhibited declarative word order (\textsc{V non-ini}).

\begin{table}
\caption{V-position--evidence correlation ($p<.001$)}
\label{tab-cz-verb}
 \begin{tabularx}{.5\textwidth}{lrrrr} %.77 indicates that the table will take up 77% of the textwidth
  \lsptoprule
            & \multicolumn{2}{c}{\textsc{V non-ini}} & \multicolumn{2}{c}{\textsc{V ini}} \\
  \midrule
  \textsc{eb $\pm 1$}   &103 &(87.4)     &27 &(42.6) \\
  \textsc{eb 0}         &211 &(226.6)    &126& (110.4) \\
  \lspbottomrule
 \end{tabularx}
\end{table}

As for indefinites, they were present in 59 Czech PQs. Fifty-four of them were of the \textit{ně}-type (PPIs) and the rest were of the \textit{ni}- or \textit{žád}-type (NCIs).

Question particles were very sparse in the Czech data set. There was only one occurrence of \textit{náhodou} and one of \textit{snad}.

\subsection{Russian}
In the Russian sample, various particles were found. The most frequent was the particle \textit{čto li} and its variations (26). The initial \textit{čto} (occasionally also medial) was also quite frequent in PQs, we have detected 14 cases. The presence of the particle \textit{razve} suggestively correlates with the speaker bias: all 10 cases displayed it, 6 of them carried evidential bias. Out of 143 cases with particles, \textit{li} was found in 6 of them. Only one question had the particle \textit{neuželi}. Thirty-eight PQs with indefinites were discovered, 18 of them contained the \textit{nibud'} series indefinites, 14 \textit{to}, 6 \textit{ni}. 

Unlike for Czech, verbs (and non-verbal predicates) were not very often at the initial position. The most frequent was the final position but no significant correlation was found between the verb position and any of the biases. 

Due to the low number of speaker bias in the Russian sample, we were unable to draw any conclusive results about the correlations between the form and this type of bias. The same applies to tags. We address it in the discussion. 

\subsection{Inter-annotator agreement} \label{sec-interan}

To test the reliability of our annotation, we recruited three students in order to later conduct the inter-annotator agreement for the semantic/pragmatic features. One Czech and two Russian speakers were paid to perform the same annotation of the bias profiles for 100 random instances from the samples. They were given instructions how to judge the biases and the affirmative prejacents for each PQ.  

We compared the annotation of the recruited students with our own. Table \ref{tab-agreem} summarizes the results, the annotators are in complete agreement if $\kappa = 1$ \citep{cohen60}. The agreement for Czech was moderate to substantial. For Russian, it was poor in the first case, slightly better in the second, moderate for evidential bias but still poor for the other two. Potential reasons for it will be discussed in the next section. 

\begin{table}
\caption{The agreement for the bias annotation, $\kappa$}
\label{tab-agreem}
 \begin{tabularx}{.7\textwidth}{Xrrr} %.77 indicates that the table will take up 77% of the textwidth
  \lsptoprule
            & \textsc{Czech} & \textsc{Russian 1} & \textsc{Russian 2}\\
  \midrule
  \textsc{speaker}         & 0.54   & 0.05   & 0.24 \\
  \textsc{evidential}      & 0.60   & 0.16   & 0.43 \\
  \textsc{knowledge}       & 0.62   & 0.33   & 0.06 \\
  \lspbottomrule
 \end{tabularx}
\end{table}


\section{Discussion} \label{sec-discussion}
In this section we discuss the results and further comment on the process of annotation. Since both Czech and Russian are Slavic languages, we expected them to behave similarly. This assumption was met to some extent. For example, negative PQs were much less frequent than the positive ones in both languages -- out of 500 PQs in each language, there were 89 negative PQs in Czech and 79 in Russian. This finding is consistent with previous research which claims that positive PQs are the unmarked way of requesting information. However, the two languages differed substantially in the frequency of tag PQs. There were also differences in the verb position, which was probably connected to the languages' preferred syntactic mechanisms. As for indefinites, their occurrence in our sample was too sparse to draw any generalizations based on them, although it is an issue that we would definitely like to address in the future. 

Czech and Russian showed comparable distribution of values of evidential bias and knowledge bias, but they differed in speaker bias. We suppose that speaker bias was a category too difficult to be objectively judged based on written material, which might have led to this discrepancy between the languages. The overall number of instances with knowledge bias was small for both languages (around 10\%). It is perhaps uncommon for speakers to know for sure the answer to the question they ask because it violates Interrogativity principle \citep{Goodhue2018a}.\footnote{\textsc{interrogativity principle:} Ask a question $?p$ only if the context set \textit{c} does not entail a complete answer to $?p$.}

We continue with the discussion for each of the languages separately.

\subsection{Czech}

In Czech tag PQs, declarative word order correlated with non-null speaker bias, i.e. tag PQs showed a strong tendency to be biased with respect to what the speaker believed. By uttering such a question, the speaker expresses their prior belief, but as it is not definitive, they shift the commitment onto the addressee at the same time. Since our annotation did not go into so much detail, we could not draw any conclusions about the different types of tags, although it would be an interesting follow-up. Our findings agree with previous research, which claims that tag PQs are mostly biased \citep{bill2022question}.

Additionally, we observed a negative correlation between interrogative word order and null evidential bias. In these PQs, the speaker has no expectation about the possible answer based on public information shared by the participants. If there is, however, a piece of compelling contextual evidence, the declarative word order would be favored. Again we see that declarative PQs tend to carry a bias. Since this is the case, we have a reason to believe that the interrogative word order is the default strategy of forming an unbiased PQ in Czech, supporting previous claims e.g. by \citet{Sticha1995a}. 

The situation is different when it comes to interrogative PQs with negation, as they do come with a bias. By uttering such a question the speaker expresses that they think that one of the alternatives ($p$ or $\neg p$) is possibly true. It seems that this bias is weaker in its meaning than that of English preposed negation, which is claimed to convey that the speaker \textit{believes} that $p$ or $\neg p$. Czech preposed negation is another issue that would deserve a closer look.

%A: I added 1 and a half paragraphs to the Czech discussion

    
\subsection{Russian}
We had to analyze the results of the semantic/pragmatic features for Russian cautiously, since the inter-annotator agreement was poor for both annotators. The potential explanation for this is the nature of the Russian corpus. Compared to the Czech corpus, it displays less context around the question; usually there were two additional lines of text. This could drastically influence the judgments of the bias profiles because evidential bias requires as much context as possible. We suppose that the number of PQs biased with respect to what the speaker believes is different from Czech in Table \ref{tab-semprag-numbers} for the same reason, since it is tricky to judge it considering formal features only. However, we have checked the cases where there was at lease some agreement between the annotators.\footnote{We thank an anonymous reviewer who recommended to check it.} Such PQs usually contained particles or were annotated as carrying no bias. For the cases with particles, agreement is justified since particles are reliable markers of various biases. In no bias agreement cases, we cannot be certain that the PQs are completely neutral due to little context availability and no audio.

Based on our corpus data we cannot conclude that the low number of tags for Russian in Table \ref{tab-formal-numbers} signals their lack in colloquial speech compared to Czech. The spoken Russian corpus contains various texts from the sixties until the present days \citep{Grisina2009} and it seems they were annotated differently in the corpus itself. For instance, tags were often separate one-word questions in the older texts, while in the modern ones they were divided by the pause marking slash `/'. Therefore, more investigation is required, preferably with audio. 

When it comes to particles, it is not surprising that \textit{li} was not very frequent in the spoken corpus. As mentioned in the introduction, it is quite marked in colloquial Russian or used in truly neutral contexts but it was not completely absent. 

The particles \textit{neuželi} and \textit{razve} occurred in our sample and were recently investigated in a series of experiments by \citet{Geist2023}. Their claim is that \textit{neuželi} denotes \textsc{verum}, an epistemic operator indicating the speaker's intention to add the proposition in question to the common ground. \textit{Neuželi} is also incompatible with another illocutionary operator \textsc{falsum} which is responsible for outer negation interpretation. \textit{Razve} is compatible with both \textsc{verum} and \textsc{falsum}. Our findings neither support nor dismiss that since the number of the particles (1 and 10, respectfully) was not sufficient to make any constructive judgments; however, all the cases were biased in one way or another. 

\textit{Čto li} (literally translated as `what whether') was the most frequent particle. \citet{Restan1969} and \citet{Dobrovolskij2014} mention its presumptive and emotional nature, in other words saying it introduces some bias and is infelicitous in out-of-the-blue PQs. Generally, questions with this particle, e.g. \textit{Na ulice dožd' čto li idët?} `It is raining outside?', are used in contexts when there is an evidence for $p$ and the speaker wants to confirm that $p$ (out of 26 cases, 10 questions had the value `$1$' for evidential bias). A private speaker belief that $\neg p$ is possible but not necessary for affirmative PQs but obligatory for negated PQs with the particle. Unlike the mentioned \textit{neuželi} and \textit{razve}, this particle is also available in declaratives and imperatives contributing epistemic modality flavor as non-at-issue meaning (cf. \citealt{Bernasconi2023}). 

%We are not going into details here for the sake of space but \textit{čto li} might be an overt lexical operator \textsc{verum} in the sense of \cite{Romero2004}, \cite{Geist2023} or \cite{Gutzmann2020}. 

\section{Conclusion} \label{sec-conclusion}

The goal of our study was to contribute to the empirical investigation of Slavic PQs. By exploring the properties of PQs through corpora, we addressed three research questions concerning their form and meaning. Corpus proved to be a convenient means of investigation, which allowed us to quickly collect authentic language data. We collected a sample of 500 PQs for each language, in which we were able to observe some tendencies, although it was probably not the best way to evaluate meaning shades, such as the biases. 

We have run the inter-annotator agreement for semantic/pragmatic features for both languages. The agreement was moderate to substantial for Czech but poor for Russian. We hypothesize that variations in the corpora may account for this discrepancy. To avoid it in future, we suggest to ensure that a corpus shows at least 10 lines of text prior to a query, corpus texts are annotated in a unified fashion and audio is available.  

The contribution of our research is mainly empirical. We observed some interesting form-meaning correlations for Czech, for instance, tag PQs tend to express speaker's belief and initial verb PQs mostly do not carry evidential bias. For Russian, we have seen that the intonational strategy is used predominately in spoken language which supports the previous observations. Moreover, we found some different particles that exhibit certain biases, e.g. \textit{neuželi}, \textit{razve} and less studied \textit{čto li}. In future research, we plan to concentrate on particular phenomena, e.g. negation, particles or intonation in Czech and Russian PQs, since we barely touched upon these or did not even consider them in the corpus investigation.


% \begin{itemize}
%     \item kickoff study, authentic data from corpus, not imaginary 
%     \item however careful with the corpus
%     \item exploring all, future concentration on particular features
% \end{itemize}

%\newpage
%For a start: Do not forget to give your Overleaf project (this paper) a recognizable name. This one could be called, for instance, Simik et al: OSL template. You can change the name of the project by hovering over the gray title at the top of this page and clicking on the pencil icon.

\section{Introduction}\label{sim:sec:intro}

Language Science Press is a project run for linguists, but also by linguists. You are part of that and we rely on your collaboration to get at the desired result. Publishing with LangSci Press might mean a bit more work for the author (and for the volume editor), esp. for the less experienced ones, but it also gives you much more control of the process and it is rewarding to see the quality result.

Please follow the instructions below closely, it will save the volume editors, the series editors, and you alike a lot of time.

\sloppy This stylesheet is a further specification of three more general sources: (i) the Leipzig glossing rules \citep{leipzig-glossing-rules}, (ii) the generic style rules for linguistics (\url{https://www.eva.mpg.de/fileadmin/content_files/staff/haspelmt/pdf/GenericStyleRules.pdf}), and (iii) the Language Science Press guidelines \citep{Nordhoff.Muller2021}.\footnote{Notice the way in-text numbered lists should be written -- using small Roman numbers enclosed in brackets.} It is advisable to go through these before you start writing. Most of the general rules are not repeated here.\footnote{Do not worry about the colors of references and links. They are there to make the editorial process easier and will disappear prior to official publication.}

Please spend some time reading through these and the more general instructions. Your 30 minutes on this is likely to save you and us hours of additional work. Do not hesitate to contact the editors if you have any questions.

\section{Illustrating OSL commands and conventions}\label{sim:sec:osl-comm}

Below I illustrate the use of a number of commands defined in langsci-osl.tex (see the styles folder).

\subsection{Typesetting semantics}\label{sim:sec:sem}

See below for some examples of how to typeset semantic formulas. The examples also show the use of the sib-command (= ``semantic interpretation brackets''). Notice also the the use of the dummy curly brackets in \REF{sim:ex:quant}. They ensure that the spacing around the equation symbol is correct. 

\ea \ea \sib{dog}$^g=\textsc{dog}=\lambda x[\textsc{dog}(x)]$\label{sim:ex:dog}
\ex \sib{Some dog bit every boy}${}=\exists x[\textsc{dog}(x)\wedge\forall y[\textsc{boy}(y)\rightarrow \textsc{bit}(x,y)]]$\label{sim:ex:quant}
\z\z

\noindent Use noindent after example environments (but not after floats like tables or figures).

And here's a macro for semantic type brackets: The expression \textit{dog} is of type $\stb{e,t}$. Don't forget to place the whole type formula into a math-environment. An example of a more complex type, such as the one of \textit{every}: $\stb{s,\stb{\stb{e,t},\stb{e,t}}}$. You can of course also use the type in a subscript: dog$_{\stb{e,t}}$

We distinguish between metalinguistic constants that are translations of object language, which are typeset using small caps, see \REF{sim:ex:dog}, and logical constants. See the contrast in \REF{sim:ex:speaker}, where \textsc{speaker} (= serif) in \REF{sim:ex:speaker-a} is the denotation of the word \textit{speaker}, and \cnst{speaker} (= sans-serif) in \REF{sim:ex:speaker-b} is the function that maps the context $c$ to the speaker in that context.\footnote{Notice that both types of small caps are automatically turned into text-style, even if used in a math-environment. This enables you to use math throughout.}$^,$\footnote{Notice also that the notation entails the ``direct translation'' system from natural language to metalanguage, as entertained e.g. in \citet{Heim.Kratzer1998}. Feel free to devise your own notation when relying on the ``indirect translation'' system (see, e.g., \citealt{Coppock.Champollion2022}).}

\ea\label{sim:ex:speaker}
\ea \sib{The speaker is drunk}$^{g,c}=\textsc{drunk}\big(\iota x\,\textsc{speaker}(x)\big)$\label{sim:ex:speaker-a}
\ex \sib{I am drunk}$^{g,c}=\textsc{drunk}\big(\cnst{speaker}(c)\big)$\label{sim:ex:speaker-b}
\z\z

\noindent Notice that with more complex formulas, you can use bigger brackets indicating scope, cf. $($ vs. $\big($, as used in \REF{sim:ex:speaker}. Also notice the use of backslash plus comma, which produces additional space in math-environment.

\subsection{Examples and the minsp command}

Try to keep examples simple. But if you need to pack more information into an example or include more alternatives, you can resort to various brackets or slashes. For that, you will find the minsp-command useful. It works as follows:

\ea\label{sim:ex:german-verbs}\gll Hans \minsp{\{} schläft / schlief / \minsp{*} schlafen\}.\\
Hans {} sleeps {} slept {} {} sleep.\textsc{inf}\\
\glt `Hans \{sleeps / slept\}.'
\z

\noindent If you use the command, glosses will be aligned with the corresponding object language elements correctly. Notice also that brackets etc. do not receive their own gloss. Simply use closed curly brackets as the placeholder.

The minsp-command is not needed for grammaticality judgments used for the whole sentence. For that, use the native langsci-gb4e method instead, as illustrated below:

\ea[*]{\gll Das sein ungrammatisch.\\
that be.\textsc{inf} ungrammatical\\
\glt Intended: `This is ungrammatical.'\hfill (German)\label{sim:ex:ungram}}
\z

\noindent Also notice that translations should never be ungrammatical. If the original is ungrammatical, provide the intended interpretation in idiomatic English.

If you want to indicate the language and/or the source of the example, place this on the right margin of the translation line. Schematic information about relevant linguistic properties of the examples should be placed on the line of the example, as indicated below.

\ea\label{sim:ex:bailyn}\gll \minsp{[} Ėtu knigu] čitaet Ivan \minsp{(} často).\\
{} this book.{\ACC} read.{\PRS.3\SG} Ivan.{\NOM} {} often\\\hfill O-V-S-Adv
\glt `Ivan reads this book (often).'\hfill (Russian; \citealt[4]{Bailyn2004})
\z

\noindent Finally, notice that you can use the gloss macros for typing grammatical glosses, defined in langsci-lgr.sty. Place curly brackets around them.

\subsection{Citation commands and macros}

You can make your life easier if you use the following citation commands and macros (see code):

\begin{itemize}
    \item \citealt{Bailyn2004}: no brackets
    \item \citet{Bailyn2004}: year in brackets
    \item \citep{Bailyn2004}: everything in brackets
    \item \citepossalt{Bailyn2004}: possessive
    \item \citeposst{Bailyn2004}: possessive with year in brackets
\end{itemize}

\section{Trees}\label{s:tree}

Use the forest package for trees and place trees in a figure environment. \figref{sim:fig:CP} shows a simple example.\footnote{See \citet{VandenWyngaerd2017} for a simple and useful quickstart guide for the forest package.} Notice that figure (and table) environments are so-called floating environments. {\LaTeX} determines the position of the figure/table on the page, so it can appear elsewhere than where it appears in the code. This is not a bug, it is a property. Also for this reason, do not refer to figures/tables by using phrases like ``the table below''. Always use tabref/figref. If your terminal nodes represent object language, then these should essentially correspond to glosses, not to the original. For this reason, we recommend including an explicit example which corresponds to the tree, in this particular case \REF{sim:ex:czech-for-tree}.

\ea\label{sim:ex:czech-for-tree}\gll Co se řidič snažil dělat?\\
what {\REFL} driver try.{\PTCP.\SG.\MASC} do.{\INF}\\
\glt `What did the driver try to do?'
\z

\begin{figure}[ht]
% the [ht] option means that you prefer the placement of the figure HERE (=h) and if HERE is not possible, you prefer the TOP (=t) of a page
% \centering
    \begin{forest}
    for tree={s sep=1cm, inner sep=0, l=0}
    [CP
        [DP
            [what, roof, name=what]
        ]
        [C$'$
            [C
                [\textsc{refl}]
            ]
            [TP
                [DP
                    [driver, roof]
                ]
                [T$'$
                    [T [{[past]}]]
                    [VP
                        [V
                            [tried]
                        ]
                        [VP, s sep=2.2cm
                            [V
                                [do.\textsc{inf}]
                            ]
                            [t\textsubscript{what}, name=trace-what]
                        ]
                    ]
                ]
            ]
        ]
    ]
    \draw[->,overlay] (trace-what) to[out=south west, in=south, looseness=1.1] (what);
    % the overlay option avoids making the bounding box of the tree too large
    % the looseness option defines the looseness of the arrow (default = 1)
    \end{forest}
    \vspace{3ex} % extra vspace is added here because the arrow goes too deep to the caption; avoid such manual tweaking as much as possible; here it's necessary
    \caption{Proposed syntactic representation of \REF{sim:ex:czech-for-tree}}
    \label{sim:fig:CP}
\end{figure}

Do not use noindent after figures or tables (as you do after examples). Cases like these (where the noindent ends up missing) will be handled by the editors prior to publication.

\section{Italics, boldface, small caps, underlining, quotes}

See \citet{Nordhoff.Muller2021} for that. In short:

\begin{itemize}
    \item No boldface anywhere.
    \item No underlining anywhere (unless for very specific and well-defined technical notation; consult with editors).
    \item Small caps used for (i) introducing terms that are important for the paper (small-cap the term just ones, at a place where it is characterized/defined); (ii) metalinguistic translations of object-language expressions in semantic formulas (see \sectref{sim:sec:sem}); (iii) selected technical notions.
    \item Italics for object-language within text; exceptionally for emphasis/contrast.
    \item Single quotes: for translations/interpretations
    \item Double quotes: scare quotes; quotations of chunks of text.
\end{itemize}

\section{Cross-referencing}

Label examples, sections, tables, figures, possibly footnotes (by using the label macro). The name of the label is up to you, but it is good practice to follow this template: article-code:reference-type:unique-label. E.g. sim:ex:german would be a proper name for a reference within this paper (sim = short for the author(s); ex = example reference; german = unique name of that example).

\section{Syntactic notation}

Syntactic categories (N, D, V, etc.) are written with initial capital letters. This also holds for categories named with multiple letters, e.g. Foc, Top, Num, etc. Stick to this convention also when coming up with ad hoc categories, e.g. Cl (for clitic or classifier).

An exception from this rule are ``little'' categories, which are written with italics: \textit{v}, \textit{n}, \textit{v}P, etc.

Bar-levels must be typeset with bars/primes, not with an apostrophe. An easy way to do that is to use mathmode for the bar: C$'$, Foc$'$, etc.

Specifiers should be written this way: SpecCP, Spec\textit{v}P.

Features should be surrounded by square brackets, e.g., [past]. If you use plus and minus, be sure that these actually are plus and minus, and not e.g. a hyphen. Mathmode can help with that: [$+$sg], [$-$sg], [$\pm$sg]. See \sectref{sim:sec:hyphens-etc} for related information.

\section{Footnotes}

Absolutely avoid long footnotes. A footnote should not be longer than, say, {20\%} of the page. If you feel like you need a long footnote, make an explicit digression in the main body of the text.

Footnotes should always be placed at the end of whole sentences. Formulate the footnote in such a way that this is possible. Footnotes should always go after punctuation marks, never before. Do not place footnotes after individual words. Do not place footnotes in examples, tables, etc. If you have an urge to do that, place the footnote to the text that explains the example, table, etc.

Footnotes should always be formulated as full, self-standing sentences.

\section{Tables}

For your tables use the table plus tabularx environments. The tabularx environment lets you (and requires you in fact) to specify the width of the table and defines the X column (left-alignment) and the Y column (right-alignment). All X/Y columns will have the same width and together they will fill out the width of the rest of the table -- counting out all non-X/Y columns.

Always include a meaningful caption. The caption is designed to appear on top of the table, no matter where you place it in the code. Do not try to tweak with this. Tables are delimited with lsptoprule at the top and lspbottomrule at the bottom. The header is delimited from the rest with midrule. Vertical lines in tables are banned. An example is provided in \tabref{sim:tab:frequencies}. See \citet{Nordhoff.Muller2021} for more information. If you are typesetting a very complex table or your table is too large to fit the page, do not hesitate to ask the editors for help.

\begin{table}
\caption{Frequencies of word classes}
\label{sim:tab:frequencies}
 \begin{tabularx}{.77\textwidth}{lYYYY} %.77 indicates that the table will take up 77% of the textwidth
  \lsptoprule
            & nouns & verbs  & adjectives & adverbs\\
  \midrule
  absolute  &   12  &    34  &    23      & 13\\
  relative  &   3.1 &   8.9  &    5.7     & 3.2\\
  \lspbottomrule
 \end{tabularx}
\end{table}

\section{Figures}

Figures must have a good quality. If you use pictorial figures, consult the editors early on to see if the quality and format of your figure is sufficient. If you use simple barplots, you can use the barplot environment (defined in langsci-osl.sty). See \figref{sim:fig:barplot} for an example. The barplot environment has 5 arguments: 1. x-axis desription, 2. y-axis description, 3. width (relative to textwidth), 4. x-tick descriptions, 5. x-ticks plus y-values.

\begin{figure}
    \centering
    \barplot{Type of meal}{Times selected}{0.6}{Bread,Soup,Pizza}%
    {
    (Bread,61)
    (Soup,12)
    (Pizza,8)
    }
    \caption{A barplot example}
    \label{sim:fig:barplot}
\end{figure}

The barplot environment builds on the tikzpicture plus axis environments of the pgfplots package. It can be customized in various ways. \figref{sim:fig:complex-barplot} shows a more complex example.

\begin{figure}
  \begin{tikzpicture}
    \begin{axis}[
	xlabel={Level of \textsc{uniq/max}},  
	ylabel={Proportion of $\textsf{subj}\prec\textsf{pred}$}, 
	axis lines*=left, 
        width  = .6\textwidth,
	height = 5cm,
    	nodes near coords, 
    % 	nodes near coords style={text=black},
    	every node near coord/.append style={font=\tiny},
        nodes near coords align={vertical},
	ymin=0,
	ymax=1,
	ytick distance=.2,
	xtick=data,
	ylabel near ticks,
	x tick label style={font=\sffamily},
	ybar=5pt,
	legend pos=outer north east,
	enlarge x limits=0.3,
	symbolic x coords={+u/m, \textminus u/m},
	]
	\addplot[fill=red!30,draw=none] coordinates {
	    (+u/m,0.91)
        (\textminus u/m,0.84)
	};
	\addplot[fill=red,draw=none] coordinates {
	    (+u/m,0.80)
        (\textminus u/m,0.87)
	};
	\legend{\textsf{sg}, \textsf{pl}}
    \end{axis} 
  \end{tikzpicture} 
    \caption{Results divided by \textsc{number}}
    \label{sim:fig:complex-barplot}
\end{figure}

\section{Hyphens, dashes, minuses, math/logical operators}\label{sim:sec:hyphens-etc}

Be careful to distinguish between hyphens (-), dashes (--), and the minus sign ($-$). For in-text appositions, use only en-dashes -- as done here -- with spaces around. Do not use em-dashes (---). Using mathmode is a reliable way of getting the minus sign.

All equations (and typically also semantic formulas, see \sectref{sim:sec:sem}) should be typeset using mathmode. Notice that mathmode not only gets the math signs ``right'', but also has a dedicated spacing. For that reason, never write things like p$<$0.05, p $<$ 0.05, or p$<0.05$, but rather $p<0.05$. In case you need a two-place math or logical operator (like $\wedge$) but for some reason do not have one of the arguments represented overtly, you can use a ``dummy'' argument (curly brackets) to simulate the presence of the other one. Notice the difference between $\wedge p$ and ${}\wedge p$.

In case you need to use normal text within mathmode, use the text command. Here is an example: $\text{frequency}=.8$. This way, you get the math spacing right.

\section{Abbreviations}

The final abbreviations section should include all glosses. It should not include other ad hoc abbreviations (those should be defined upon first use) and also not standard abbreviations like NP, VP, etc.


\section{Bibliography}

Place your bibliography into localbibliography.bib. Important: Only place there the entries which you actually cite! You can make use of our OSL bibliography, which we keep clean and tidy and update it after the publication of each new volume. Contact the editors of your volume if you do not have the bib file yet. If you find the entry you need, just copy-paste it in your localbibliography.bib. The bibliography also shows many good examples of what a good bibliographic entry should look like.

See \citet{Nordhoff.Muller2021} for general information on bibliography. Some important things to keep in mind:

\begin{itemize}
    \item Journals should be cited as they are officially called (notice the difference between and, \&, capitalization, etc.).
    \item Journal publications should always include the volume number, the issue number (field ``number''), and DOI or stable URL (see below on that).
    \item Papers in collections or proceedings must include the editors of the volume (field ``editor''), the place of publication (field ``address'') and publisher.
    \item The proceedings number is part of the title of the proceedings. Do not place it into the ``volume'' field. The ``volume'' field with book/proceedings publications is reserved for the volume of that single book (e.g. NELS 40 proceedings might have vol. 1 and vol. 2).
    \item Avoid citing manuscripts as much as possible. If you need to cite them, try to provide a stable URL.
    \item Avoid citing presentations or talks. If you absolutely must cite them, be careful not to refer the reader to them by using ``see...''. The reader can't see them.
    \item If you cite a manuscript, presentation, or some other hard-to-define source, use the either the ``misc'' or ``unpublished'' entry type. The former is appropriate if the text cited corresponds to a book (the title will be printed in italics); the latter is appropriate if the text cited corresponds to an article or presentation (the title will be printed normally). Within both entries, use the ``howpublished'' field for any relevant information (such as ``Manuscript, University of \dots''). And the ``url'' field for the URL.
\end{itemize}

We require the authors to provide DOIs or URLs wherever possible, though not without limitations. The following rules apply:

\begin{itemize}
    \item If the publication has a DOI, use that. Use the ``doi'' field and write just the DOI, not the whole URL.
    \item If the publication has no DOI, but it has a stable URL (as e.g. JSTOR, but possibly also lingbuzz), use that. Place it in the ``url'' field, using the full address (https: etc.).
    \item Never use DOI and URL at the same time.
    \item If the official publication has no official DOI or stable URL (related to the official publication), do not replace these with other links. Do not refer to published works with lingbuzz links, for instance, as these typically lead to the unpublished (preprint) version. (There are exceptions where lingbuzz or semanticsarchive are the official publication venue, in which case these can of course be used.) Never use URLs leading to personal websites.
    \item If a paper has no DOI/URL, but the book does, do not use the book URL. Just use nothing.
\end{itemize}


\section*{Abbreviations}

\begin{tabularx}{.5\textwidth}{@{}lQ}
% \textsc{3}&third person\\
% \textsc{acc}&accusative\\
% \textsc{inf}&infinitive\\
% \textsc{m}&masculine\\
% \textsc{nom}&nominative\\
% \end{tabularx}%
% \begin{tabularx}{.5\textwidth}{lQ@{}}
% \textsc{prs}&present tense\\
% \textsc{ptcp}&participle\\
\textsc{aux}&auxiliary\\
\textsc{indef}&indefinite\\
\textsc{neg}&negation\\
\end{tabularx}%
\begin{tabularx}{.5\textwidth}{lQ@{}}
\textsc{prt}&particle\\
\textsc{refl}&reflexive\\
\textsc{wo}&word order\\
%&\\ % this dummy row achieves correct vertical alignment of both tables
\end{tabularx}

\section*{Acknowledgments}
The study was funded by the Czech Science Foundation (GA\v{C}R), project No. 21-31488J. We would like to acknowledge the valuable input (alphabetically) of Kateřina Hrdinková, Roland Meyer, Mariia Razguliaeva, and Radek Šimík. We would like to express our gratitude to the organizers and audience of FDSL 15 for the discussion and insightful comments. We also thank our annotators who helped us during the investigation, the reviewers and the editors of the volume for their valuable feedback. 

\printbibliography[heading=subbibliography,notkeyword=this]

\end{document}
