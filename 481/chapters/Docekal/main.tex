\documentclass[output=paper,colorlinks,citecolor=brown]{langscibook}
\ChapterDOI{10.5281/zenodo.15394174}
%\bibliography{localbibliography2}

\author{Mojmír Dočekal\orcid{0000-0002-9993-4756}\affiliation{Masaryk University} }
%\author{John Doe}

% replace the above with you and your coauthors
% rules for affiliation: If there's an official English version, use that (find out on the official website of the university); if not, use the original
% orcid doesn't appear printed; it's metainformation used for later indexing

%%% uncomment the following line if you are a single author or all authors have the same affiliation
\SetupAffiliations{mark style=none}

%% in case the running head with authors exceeds one line (which is the case in this example document), use one of the following methods to turn it into a single line; otherwise comment the line below out with % and ignore it
%\lehead{Šimík, Gehrke, Lenertová, Meyer, Szucsich \& Zaleska}
\lehead{Mojmír Dočekal}
%\lehead{John Doe}


\title{Equatives and two theories of negative concord}
% replace the above with your paper title
%%% provide a shorter version of your title in case it doesn't fit a single line in the running head
% in this form: \title[short title]{full title}
\abstract{This article reports the results of an experiment targeting the acceptability of Czech neg-words and strong NPIs under Neg-Raising predicates and in the complement clauses of equatives. The theoretical consequences of the results are discussed and range from the support of non-standard negative concord theories to the support of non-standard degree semantics for the equative constructions. 

\keywords{experimental semantics, Czech, neg-words, strong NPIs, equatives}
}

\begin{document}
\maketitle

\section{Introduction}\label{doc:intro}

In this article, I explore expressions that are polarity-dependent. The evidence comes from Czech, a strict negative-concord language. I will focus on one recent experiment in the paper.\footnote{But the experiment is a continuation of many previous experimental works which incrementally changed the nature of questions and research goals reflected in the current experiment.} The theoretical ambition of this paper is to examine the distribution of neg-words and strong NPIs in two environments: Neg-Raising predicates and equatives. The acceptability pattern of these two kinds of negative dependent expressions is challenging for standard theories of neg-words but also for the current degree theories of equatives. The data are subtle, and therefore I report results of an acceptability judgment task experiment on Czech native speakers; in this way, I add to the experimental research on Negative Polarity Items (NPIs), like \citet{Chemla-Homer-Rothschild-NPI,gajewski2016another,alexandropoulou2020there}, a.o., more specifically to the experimental research on cross-linguistic variation in NPI licensing as found in \citet{djarv2018cognitive,schwarz2020italian,chierchia2019factivity}. Empirically, the experimental data concern the Czech strong NPIs, like \emph{ani jeden} `even one,' and neg-words, like \textit{žádný} `no', as exemplified in \REF{ex-1}. 

\ea\label{ex-1}\gll Petr nepotkal \minsp{\{} {ani jednoho} / žádného\} studenta.\\
Petr \textsc{neg}.met {} \textsc{strong.npi} {} \textsc{neg.word} student\\
\glt `Petr didn't meet \{even one / any\} student.'
\z

\noindent Both polarity-sensitive items are in the majority of contexts interchangeable, but their meaning differs, and the experiment was focused on the environments where the meaning difference is detectable. In previous works, the acceptability of NPIs was reported as varying between speakers \citep{homer2021domains,sprouse2019}. In the neg-words research, the variation of acceptability was found to correlate with demographic factors such as age or education \citep{burnett2015variable,burnett2018structural}. For these reasons, I also included demographic variables in my experimental research to see whether a more complex picture (that incorporates both grammatical and demographic factors) can explain the distribution of neg-words and strong NPIs more successfully.  

%\ea\label{sim:ex:german-verbs}\gll Hans \minsp{\{} schläft / schlief / \minsp{*} schlafen\}.\\
%Hans {} sleeps {} slept {} {} sleep.\textsc{inf}\\
%\glt `Hans \{sleeps / slept\}.'
%\z

The article is structured as follows: In \sectref{sec:theoretical-backgr}, I introduce the theoretical background for the experiment. In \sectref{sec:experiment}, I describe the experiment. In \sectref{sec:theoretical_consequences}, I discuss the results, and in \sectref{sec:summary}, I conclude the article.

\subsection{Theoretical background}\label{sec:theoretical-backgr}

\subsubsection{Polarity-dependent expressions in equatives and under Neg-Raising predicates}

The theoretical background for the experiment is the contrast between strong NPIs and neg-words in two environments, under Neg-Raising predicates and in equatives. As a baseline against the two environments, I used simple unembedded sentences with negated verbs. The baseline is important since it is the only environment where both strong NPIs and neg-words are grammatical under any theory of neg-words and NPIs. The baseline is illustrated in \REF{ex-1}. In the current section I discuss the theoretical background for the constructions (equatives and Neg-Raising predicates), in \sectref{sec:theoretical-backgr-NPIs-neg-words} Czech strong NPIs and neg-words are discussed, and \sectref{sec:assumptions-licensing-of-strong-npis} is dedicated to the theoretical background for the licensing of strong NPIs and neg-words. 

As for Neg-Raising, we can adapt any current theory of Neg-Raising, be it the presuppositional version of \citet{gajewski2007neg} or the scalar implicature version of \citet{romoli2013scalar}. Both share the insight that Neg-Raising predicates bear the excluded middle inference: for \textit{believe}: $Bel(p) \vee Bel(\neg p)$, adding the negated assertion $\neg Bel(p)$ results in the deductively valid conclusion where the negation scopes in the embedded clause, $Bel(\neg p)$. Non-Neg-Raising predicates then come without the excluded middle inference, resulting in the surface interpretation of the negation. Since the scope of negation ends in the embedded clause, the negation is local, and therefore, the strong NPIs are licensed, as demonstrated with Spanish \textit{ni un} `not even one' in \REF{ex-span-1}.

\ea\label{ex-span-1}\gll No creo que ni un solo soldado pueda lograrlo.\\
not believe.\textsc{1sg} that not even one soldier can achieve\\
\glt `I don't believe that not even one soldier can achieve it.'
\z

\noindent The second environment which was tested in the experiment was equatives. The first thing to note is that the standard theory of equatives is built on the ``$>$'' analysis of comparatives \citep{beck_comparison_nodate,stechow1984comparing} where the core operation is the relation $>$ comparing two maxima: (i) the maximum of the set of degrees from the main clause, (ii) the maximum of the set of degrees from the complement of the comparative clause; see \REF{ex-17} as an illustration. 


\ea\label{ex-17} The dog is taller than the cat. \ea \cnst{MAX}({$d$\vert the height of the dog $\geq d$}) $>$ \cnst{MAX}({$d$\vert the height of the cat $\geq d$})\z\z

\noindent The standard theory of equatives \citep{beck_comparison_nodate,stechow1984comparing,rullmann1995maximality} then follows the ``$>$'' analysis of comparatives, just replacing $>$ with $\geq$ which is in most contexts pragmatically strengthened to ``$=$''; see \REF{ex-18} for an illustration. 

\ea\label{ex-18} The dog is as tall as the cat. \ea \cnst{MAX}({$d$\vert the height of the dog $\geq d$}) $\geq$ \cnst{MAX}({$d$\vert the height of the cat $\geq d$})\z\z

\noindent Comparatives are then theoretically expected to license NPIs in their complement clauses, since the degree argument is downward-monotonic, therefore if some degree $d > d'$ and there is another degree $d''$, such as $d' > d''$, then by transitivity $d > d''$. Intuitively, if the dog from \REF{ex-17} is taller (or of the same height) than the cat from \REF{ex-17}, then he is taller than any cat smaller than that cat. The literature on comparatives  \citep{stechow1984comparing,rullmann1995maximality,gajewski2008more} agrees on the empirical verification of this prediction. Weak NPIs (like English \textit{any}) are licensed in the complement clauses of the comparative. In the case of strong NPIs, the empirical situation is less clear. Still, at least empirically, it is claimed for Germanic languages that strong NPIs appear in the complement clause of comparatives felicitously; see \citeposst{hoeksema2008natural} Dutch example in \REF{ex-19}. \REF{ex-19} contains the Dutch expression \textit{ook maar} `even,' which is taken as a standard example of a strong NPI (see \citealt{zwarts1998three}).

\ea\label{ex-19} \gll Zij was beter dan {ook maar iemand} verwacht had.\\
she \textsc{aux} better than \textsc{strong.npi} expected \textsc{aux}\\
\glt `She was better than anyone could have expected.'
\z

\noindent Under the premise that equatives are built on the ``$>$'' analysis of comparatives, the standard theory of equatives predicts that NPIs should be licensed in the complement clause of equatives. This prediction works for English and supports the standard theories of equatives (see \cite{stechow1984comparing,beck_comparison_nodate}, a.o.), since NPIs in English equatives are licensed; see \REF{ex-5} from \citet{seuren1984comparative}.

\ea\label{ex-5} Paris is as quiet as ever.
\z


\noindent The theories of Neg-Raising and equatives are general and their aim is to model the meaning of the construction. The interaction of the constructions with various classes of polarity dependent expressions is not their primary concern. Nevertheless, as discussed in this section at least for strong NPIs, the predictions of the standard theories of Neg-Raising and standard theories of equatives are clear. The strong NPIs should be licensed. But if we want to apply the predictions of the standard theories of Neg-Raising and equatives to both strong NPIs and neg-words, we have to introduce the theories of neg-words and strong NPIs. This is the topic of the next section.

\subsubsection{Czech strong NPIs and neg-words}\label{sec:theoretical-backgr-NPIs-neg-words}

Let us introduce some background information and intuitions concerning both classes. Starting with strong NPIs (for a theoretical framework, see \cite{gajewski2011licensing}), Czech strong NPIs of the \textit{ani} sort bear the unlikelihood presupposition, discussed concerning English stressed  \emph{ANY} (see \citealt{krifka1995semantics}, a.o.), Hindi \emph{ek bhii} (see \citealt{lahiri1998focus}, a.o.) or English \emph{even one} (see \citealt{crnivc2014non}, a.o.). But unlike the English or Hindi strong NPIs, the Czech strong NPIs are much more limited in distribution, requiring clause-mate negation in most of their occurrences. Nevertheless, this requirement is not obligatory, as will be demonstrated, and Czech \textit{ani} strong NPIs can appear embedded under negated Neg-Raising predicates without any overt clause-mate negation. But in all contexts, \textit{ani} presupposes that its prejacent (a proposition which \textit{ani} modifies) entails all the relevant alternatives. By way of example, \textit{ani jeden} `even one' in \REF{ex-2} is acceptable since not scoring one goal entails not scoring two, three, etc. goals ($\neg \textsc{score}(1) \models \neg \textsc{score}(2:\infty)$), the relevant alternatives. But \textit{ani deset} `even ten' is much less acceptable, since not scoring ten goals is entailed by not scoring 9, 8, etc. goals ($\neg \textsc{score}(1:9) \models \neg \textsc{score}(10)$) and therefore the prejacent does not entail all the relevant alternatives. In this respect, \textit{ani jeden} belongs to the same class of strong NPIs as English \textit{even one}, which yields the scalar presupposition (in \REF{ex-2} the focus alternatives have to be less probable and entailed by the prejacent).
  
\ea\label{ex-2}\gll FC Barcelona nedala \minsp{\{} ani jeden / \minsp{\#} ani deset\} gól/ů.\\
FC Barcelona \textsc{neg}.gave {} even one {} {} even ten goal(s)\\
\glt `FC Barcelona didn't score \{even one/\#ten\} goal(s).'
\z

\noindent Turning now to neg-words, Czech (and generally Slavic) neg-words are similar to Italian neg-words (such as \emph{niente}, e.g., see \citealt{ladusaw1992expressing}). In contrast to strong NPIs like \textit{ani jeden} in \REF{ex-2}, or English \textit{even one}, neg-words do not bear any scalar or additive presupposition. In addition, neg-words have strong syntactic requirements on their licensing, and in Czech, as in all Slavic languages, which are strict negative-concord languages (see \citealt{zeijlstra2004sentential}, a.o.), Czech neg-words in the majority of contexts require verbal negation (in the same clause), the requirements being more strict than in the case of Czech strong NPIs; see \REF{ex-3}. 

\ea\label{ex-3}  
\ea \gll Petr nedal žádný gól.\\
  Petr \textsc{neg}.scored \textsc{neg}.word goal\\
  \glt `Petr didn't score any goal.' 
\ex \gll Nikdo  \minsp{\{} nepřišel / \#přišel\}.\\
  \textsc{neg.word} {} \textsc{neg}.came {} came\\
  \glt `Nobody came.' 
\ex \gll *Petr neřekl, že nikdo přišel.\\
  Petr \textsc{neg}.said that \textsc{neg.word} came\\
  \glt `Petr didn't say that anybody came.'
\z\z 


\noindent Unlike strong NPIs, neg-words do not yield the scalar presupposition but their licensing is more locality-constrained. Therefore Czech neg-words are degraded under negated Neg-Raising predicates (see \citealt{dovcekal2016experimentala,docekaldotlacilsubedinb} for details, and \REF{ex-8-b} for an example from the experiment).

The most influential current analysis of neg-words is the syntactic approach of \citet{zeijlstra2004sentential}, a.o. (the standard theory/\cite{zeijlstra2004sentential} hereinafter). It claims for strict negative-concord languages that all neg-words (and the verbal negation) carry a [uNeg] feature and are checked against an [iNeg] (covert) operator with the semantics of $\neg$. Part of this paper is dedicated to providing experimental support for an alternative semantic theory of neg-words (see \citealt{ovalle2004double,kuhn2022dynamics}, a.o.; I will refer to this theory as the alternative theory/\cite{ovalle2004double}), which will be explained in detail later; see \sectref{sec:assumptions-licensing-of-strong-npis}. The empirical point concerns equatives, one of the contexts where the distribution of strong NPIs and neg-words diverges. It was noted for Polish equatives at least as early as \citet{Blasczak:2001} that neg-words are surprisingly grammatical in them. Czech equatives are similar; they seem not to license strong (and weak) NPIs, resembling German and many other non-English equatives (see \citealt{krifka1992some}, a.o.). Nevertheless,  neg-words are very much acceptable in the complement clauses of Czech equatives; see \REF{ex-4}.  

\ea\label{ex-4} \gll Petr je tak vysoký jako \minsp{\{\#} {ani jeden} / žádný\} jiný student.\\
Petr is so tall how {} \textsc{strong.npi} {} \textsc{neg.word} other student.\\
\glt `Petr is as tall as any other student.'
\z

\noindent The acceptability of neg-words in Czech equatives is also surprising according to the standard theory of neg-words/\citet{zeijlstra2004sentential},  since there is no plausible overt or covert operator with the interpretable [iNeg] feature in the complement clause of Czech equatives. The standard theory of neg-words/\citet{zeijlstra2004sentential} predicts the ungrammaticality of neg-words in Czech equatives, which is empirically wrong. Part of the experimental work reported in this paper is to test the acceptability of neg-words in Czech equatives, such as \REF{ex-4}, and to compare it with the acceptability of strong NPIs.


Concerning Neg-Raising predicates, Czech (like Spanish in \REF{ex-span-1}) allows licensing of strong NPIs in the embedded clause; see \REF{ex-cz-nr}. 

\ea\label{ex-cz-nr} \gll Nechci, aby ani jeden student odešel.\\
\textsc{neg}.want.\textsc{1sg} that even one student left\\
\glt `I don't want even one student to leave.'
\z

\noindent As for neg-words, previous experimental research reported their decreased acceptability under Neg-Raising predicates (see \citealt{dovcekal2016experimentala}). Such a pattern is expected in the standard theory of neg-words/\citet{zeijlstra2004sentential}, since the negation is syntactically localized in the root clause. In terms of the syntactic approach, the root negation bearing a [iNeg] feature is too far away from eventual neg-words in the embedded clause to license them. 

To this end, the experiment reported below scrutinizes the contrast between strong NPIs and neg-words in equatives (and under Neg-Raising predicates). First, Slavic literature observed the acceptability of neg-words in equatives, and second, NPI literature noticed the unacceptability of strong NPIs in Germanic. Nevertheless, the contrast was neither experimentally researched nor theoretically explained. Moreover, equatives are one of the environments where the contrast between Czech neg-words and strong NPIs is most robust, but still, there seems to be speaker variation involved. In simple terms, some speakers treat \emph{ani} as a neg-word and therefore do not accept it as much in equatives, unlike the speakers who use \textit{ani} as a strong NPI. In a bit broader picture, the speaker variation resembles the variation of English NPIs vs. negative quantifiers, e.g., as studied first by functional linguists (see \citealt{tottie1991negation}, a.o.), and more recently in the formal syntactic tradition (see \citealt{burnett2015variable,burnett2018structural}, a.o.). \citet{burnett2015variable,burnett2018structural} show that formal constraints explain the English speaker variation with higher success than historical and social factors (discovered in the functional tradition before). According to \citet{burnett2018structural}, the English negative quantifiers are replaced by NPIs in lower syntactic domains.\footnote{Consider a contrast like \textit{There were no jobs to be had} -- higher syntactic domain vs. \textit{I can't have any form of gluten}, where in the first sentence, the negative quantifier is used, while in the second sentence, an NPI occurs. Examples come from \citet{burnett2018structural}, where it is claimed that while this constraint is soft in contemporary English, it is a hard one in the Scandinavian language family.} This process overrules any demographic factors, like age or education. In a similar vein, \citet{burnett2015variable} describe the variable negative concord in Québec French as explainable by the interplay of grammatical and demographic factors, where the first type of factors is decisive.
  

The speaker variation mentioned above is an intriguing and hard-to-pin-down phenomenon, and one of the reasons for using experimental methods, since it certainly resists any simple intuition-based methods for data collecting.  The emerging picture is that speaker variation concerning negation, negative concord, negative quantifiers, and NPIs comes both from social and grammatical sources, and only experimental work can give some reasonable answers as to their respective strength. In this respect, the experimental work reported below is the first tiny step in explaining Slavic neg-words vs. NPIs speaker variation due to the possible interplay between demographic and grammatical factors.

The following section introduces the licensing conditions for strong NPIs and neg-words in a formal way. At the end of the section the predictions of the theories of neg-words, strong NPIs, Neg-Raising and equatives are summarized.


\subsubsection{Assumptions concerning licensing of (strong) NPIs}\label{sec:assumptions-licensing-of-strong-npis}

Let us assume a standard approach to NPIs and strong NPIs licensing. For the general framework, the so-called \textit{even}-theory of NPIs licensing is naturally the most attractive candidate (see \citealt{krifka1995semantics,lahiri1998focus,crnivc2014non}, a.o.), since \textit{ani} bears the unlikelihood presupposition similar to English \textit{even}. And for strong NPIs, let us follow Gajewski's formalization of strong NPIs  \citep{gajewski2011licensing}. According to \citet{gajewski2011licensing}, strong NPIs are licensed in downward-entailing (DE) environments. But the downward entailments are checked both in Truth Conditions (TC), the at-issue part of the meaning, and in the non-at-issue meaning, presuppositions and implicatures being the most pertinent non-at-issue meaning components. Weak NPIs, on the other hand, require DE environments only in the TC part of the meaning. The conditions for weak and strong NPIs are summarized in \REF{ex:9}.

\ea\label{ex:9} An NPI is licensed in the environment $γ$\\
$[_α exh [_β \ldots [_γ$ NPI $] \ldots ]]$:
\ea the environment $γ$ is DE in β \hfill weak NPIs
\ex the environment $γ$ is DE in α \hfill strong NPIs\z\z

\noindent The standard exhaustifier from \REF{ex:9} is the formalization of the \textit{only}-kind of focus operator which works very well for weak NPIs like unstressed English \textit{any}. But for other weak or strong NPIs with the unlikelihood-presupposition meaning, another kind of exhaustifier, a covert counterpart of English \textit{even}, was proposed (see \citealt{crnic2011getting,crnivc2014against}). The same mechanism is used in formal approaches to focus particles (see \citealt{panizza2020minimal}). The \textit{even} exhaustifier, like its overt version, then comes with two presuppositions. The first is scalar, demonstrated in \REF{ex-10-a} -- the sentence is acceptable in such contexts where a dancing Pope is very unlikely (compatible with the actual world). The second is additive, exemplified with \REF{ex-10-b}. 

\ea\label{ex-10} \ea\label{ex-10-a} Even the Pope$_F$ danced.
\ex\label{ex-10-b} Even one$_F$ cat will make the Pope happy.\z\z

\noindent The sentence is true if two, three, \ldots  cats will make the Pope happy as well. The placement of focus determines the nature of alternatives used in presuppositions. Let us follow the formalization of both presuppositions by \citet{panizza2020minimal}; see \REF{ex-11}. 

\ea\label{ex-11} `Even $\phi$' presupposes:
\ea that $\phi$ is relatively unlikely to be true among Alt($\phi$); and
\ex that there is $\psi \in$ Alt($\phi$) that is not entailed by $\phi$ and is true.\z\z

\noindent For monotonic scales, likelihood from \REF{ex-11} translates into entailment (after \citealt{crnic2011getting}), therefore the predictions of traditional downward-entailing approaches like \citep{ladusaw1992expressing} and \textit{even}-theories of NPIs collapse for downward-monotonic contexts.

Assuming this standard approach to strong NPIs, its predictions are clear for simple negated sentences like \REF{ex-10-neg}, where \textit{even} associates with the weak scalar item (the numeral \textit{one}). The scalar presupposition of \textit{even} has to scope over negation, schematically [$_{even} \neg$[one student arrived] ], and since this logical form entails all other alternatives ([$_{even} \neg$[$n$ students arrived] ], where $n > 1$), the prejacent is both strongest and the least likely from the alternatives and the scalar presupposition (see \REF{ex-11}) is fulfilled. In the experiment, I used baseline sentences of similar form for both strong NPIs and neg-words. Both expressions were unsurprisingly well-accepted in the baseline. 

\ea\label{ex-10-neg} Even one student didn't arrive.
\z

\noindent For Neg-Raising predicates like \textit{want} from \REF{ex-10-nr}, then the schematic scope configuration in the embedded clause is covert ($even$) $>
%\textgreater{} 
\neg >$
%\textgreater{} 
{[$_α$}\ldots{} one \ldots{]} -- the scalar presupposition of \textit{even} is fulfilled. Moreover, the licensing condition for strong NPIs (see \REF{ex:9}) requires that the local domain (α in \REF{ex-10-nr}) is DE, meaning after we factor in all non-at-issue meaning components, which is the case for Neg-Raising predicates.

\ea\label{ex-10-nr} The director doesn't want [$_α$ even one student to depart].
\z


\noindent As for equatives, they are theoretically expected to license NPIs, which seems to be the case for English weak NPIs, as illustrated by \REF{ex-5}, repeated for convenience below as \REF{ex-5-rep}. 

\ea\label{ex-5-rep} Paris is as quiet as ever.
\z

\noindent But it was noticed before that this does not hold cross-linguistically; see \citet{krifka1992some} for German and \citet{penka2016degree} for German and Romance languages. But at least in the comparative/equative ``$>$'' theories, if comparatives license strong NPIs, the expectation is that equatives will behave similarly. Turning now to Slavic equatives, there are many factors at play here, though. First, Slavic equatives are different from English equatives, and their morpho-syntax is very similar to correlatives (like German and Romance equatives). And since it is known at least from \citet {pauline1995quantificational} that correlatives are bad licensors of NPIs, the expectation is that both weak and strong NPIs will be much worse in Slavic equatives (compared to Germanic languages).\footnote{To address this issue, another experiment targeting both weak and strong NPIs in comparatives and equatives is in preparation.} To summarize, the standard degree theory of equatives -- the standard degree theory/\citet{rullmann1995maximality} hereinafter (and assuming the standard theory of NPIs licensing introduced above) -- predicts that NPIs (weak and strong) should be licensed in equatives. I was unable to locate any scholarly discourse pertaining to strong NPIs and equatives, but the following example \REF{ex-5-eq} from \citet{britannica} can be seen as an empirical approval of the standard degree theory prediction for English. In \REF{ex-5-eq}, there is a strong NPI \textit{until recently} in the equative standard; the verb \textit{changed} is telic. Therefore the strong NPI should be licensed by the DE logical properties of the English equative, which seems to be the case. But as will be demonstrated in the next section, the situation is quite different in Czech, which confirms the observations concerning German and Romance equatives \citep{krifka1992some,penka2016degree} and in alternative theories of equatives (the alternative degree theories/\cite{penka2016degree} hereinafter).

\ea\label{ex-5-eq} Under the party system in Canada cabinets changed as often as, until recently, they did in France.
\z

\subsubsection{Assumptions concerning licensing of neg-words}\label{sec:assumptions-licensing-of-neg-words}

As for the licensing of neg-words, I will now introduce the syntactic theory of Negative Concord (NC) developed by \citet{zeijlstra2004sentential,penka2007negative}, and \citet{zeijlstra2008negative} in detail. The standard theory/\citet{zeijlstra2004sentential} is the syntactic tool for dealing with negative concord both in strict and non-strict negative concord languages. Since Czech (like all Slavic languages) is an example of strict negative concord, I will focus on the part of the theory that deals with strict NC. The basic assumption for strict NC languages in Penka/Zeijlstra's syntactic theory is that all morphologically-negated words come without semantic negation. Neg-words and sentential negation carry a so-called uninterpretable {[}uNeg{]} feature, which is in agreement with the logical operator (propositional negation) that has an interpretable {[}iNeg{]} feature.  Sentential negation is a signal of propositional negation, but propositional negation is located higher in the syntactic tree than sentential negation. The syntactic theory then treats neg-words as indefinites, and their negation is purely syntactical (the uninterpretable feature). The purpose of the uninterpretable feature is then to signal the presence of the propositional negation operator. Let us illustrate the mechanism used in the syntactic theory with a Czech example in \REF{ex-neg-concord-a}. The sentence contains three morphological negations, but according to the syntactic theory, none of them bears semantic force, which is delegated to the abstract logical operator with the semantics of classical propositional negation, see \REF{ex-neg-concord-b}. The final logical form is in \REF{ex-neg-concord-c}.

\ea\ea\label{ex-neg-concord-a} \gll Nikdo neviděl nic.\\
\textsc{neg}.person \textsc{neg}.saw \textsc{neg}.thing\\
\glt `Nobody saw anything.'
\ea\label{ex-neg-concord-b} Op$_{\neg \mathrm{[iNeg]}}[$Nikdo$_{\neg \mathrm{[uNeg]}}$ neviděl$_{\neg \mathrm{[uNeg]}}$ nic$_{\neg \mathrm{[uNeg]}}$.$]$
\ex $\llbracket$nikdo$\rrbracket = \lambda P \exists x[\cnst{person}(x) \wedge P(x)]$ 
\ex $\llbracket$nic$\rrbracket = \lambda P \exists x[\cnst{thing}(x) \wedge P(x)]$
\ex $\llbracket$neviděl$\rrbracket = \lambda y.\lambda x.\textsc{see}(x,y)$\z
\ex\label{ex-neg-concord-c} $\neg \exists x \exists y[\cnst{person}(x) \wedge \cnst{thing}(y) \wedge \textsc{see}(x,y)]$\z\z

\noindent The syntactic theory is well-equipped to deal with the locality constraints on negative concord and, of course, easily explains the baseline kind of example like \REF{ex-neg-concord-a}, where the neg-words and verbal negation appear in a root clause. The logical operator has to be local, around the level of the TP projection (of the clause where neg-words or verbal negation appears). 

As for Neg-Raising, the predictions of the syntactic theory are the following: Since the inferential process by which the scope of negation ends on the embedded predicate (schematically: $\neg$ NegRaisingVerb [Predicate] $\rightsquigarrow$ NegRaisingVerb [$\neg$ Predicate]) is pragmatic in nature, the excluded middle inference is (depending on theory) treated either as a presupposition or as an implicature. The valid scope of the invisible operator is the root sentence (Op$_\neg$[NegRaisingVerb [Predicate]]). Therefore, neg-words in the embedded sentence are too far away for agreement between the uninterpretable feature and the interpretable feature of the operator.

Nevertheless, in the case of equatives, the standard theory/\citet{zeijlstra2004sentential} simply predicts the ungrammaticality of neg-words (in the case of the positive main predicate), which is empirically wrong. Let us start with some empirical observations. According to Sketch Engine \citep{kilgarriff2014sketch}, in their csTenTen19 (the most representative Czech corpus in Sketch Engine), there are 28 occurrences of neg-words in the standard clause of equatives.\footnote{The CQL used for the search was: \texttt{[lemma="tak"] [tag="k2.*"] [lemma="jak[o]?"] [lemma="žádný"]}.} One example sentence from the query is in \REF{ex-neg-concord-eq}. This is in contrast to strong NPI: in Sketch Engine, there is no occurrence of strong NPI \textit{ani} in the standard clause of equatives. This asymmetry is also verified by the intuitions of native speakers, as will be reported in the experiment. The empirical inadequacy of the syntactic theory follows from the standard theory of equatives as the $\geq$ relation between two maxima of two sets of degrees -- there is no place for negation in the semantics of equatives, neither in the standard theory nor in the alternative theories of equatives (see \citealt{penka2016degree}, a.o.). And for this reason, I will now introduce the alternative, non-standard theory of neg-words.

\ea\label{ex-neg-concord-eq} \gll Ve zbarvení je pstruh obecný tak variabilní jako žádná naše ryba. \\
in coloration is trout brown as variable like \textsc{neg.word} our fish\\
\glt `The brown trout is as variable in coloration as any of our fish.'
\z

\noindent The alternative theory of neg-words was formulated in \citet{ovalle2004double}, and a more recent reformulation can be found in \citet{kuhn2022dynamics}. It shares some assumptions with the syntactic theory, though. First, both theories agree on the indefinite description status of neg-words. Therefore neg-words denote sortally existential quantifiers like in \REF{ex-13-a} in the alternative theory too. The negative force, which in the syntactic theory is carried by the covert operator (the bearer of the classical logical semantic of $\neg$), is in the semantic/pragmatic theory reformulated as a presupposition of empty reference in the original version; see \REF{ex-13-b}. Or in the dynamic reformulation as a test on the cardinality of discourse referents, like in \REF{ex-13-c}. 

\ea\label{ex-13} \ea\label{ex-13-a} $\llbracket$neg-word$\rrbracket$=$\lambda P.\exists x[\cnst{sort}(x) \wedge P(x)]$
  \hfill TC 
\ex\label{ex-13-b}   $\llbracket$neg-word$\rrbracket$=$\neg \exists x[\cnst{sort}(x) \wedge P(x)]$
  \hfill non-at-issue 
\ex\label{ex-13-c} after \textcite{kuhn2022dynamics}:
  $\wedge 0_x$ \ldots postsupposition (highest scope) \z\z
  
\noindent In this article, we can abstract away from the formal implementations and work with the core assumption: The emptiness of reference is a presupposition with the usual projection properties of presuppositions. One of the main differences concerns the interpretation of verbal negation in strict negative concord languages though. While in the standard theory/\citet{zeijlstra2004sentential}, the verbal negation is just an agreement negation with the active covert logical operator, which carries the logical negation, in the alternative theory, the verbal negation has its semantic interpretation, the classical propositional logic $\neg$.

The original version of the semantic/pragmatic theory/\citet{ovalle2004double} does not come with any locality constraints on the neg-word licensing, which is a problematic assumption since negative concord is, in most cases, limited to the clause-internal dependency between neg-words and verbal negation. This is also one of the reasons why the syntactic approach is so successful and remains the standard theory of neg-words today. \citet{kuhn2022dynamics} improves in many aspects over the original version of the semantic/pragmatic theory; one of them is the delimitation of the emptiness of reference presupposition in terms of previous contexts, and also in tying it to discourse referents and therefore making the presupposition more specific. But most importantly, \citet{kuhn2022dynamics} brings some syntactic constraints into the game. He formalizes neg-words' syntax via split scope around their licensor (prototypically verbal negation). Since the split scope is realized via quantifier raising, some locality constraints on the neg-word emerge. More specifically, \citeposst{kuhn2022dynamics} empirical claim is that the locality constraints on neg-word licensing should correspond to the locality of quantifier raising in the particular language and construction. Whether this is the right theoretical solution is a separate question, which is not answerable in this article. Still, it is definitely a step in the right direction, including some form of syntactic sensitivity for locality into the semantic/pragmatic theory.

Let us go through the predictions the semantic/pragmatic theory/\citet{ovalle2004double} makes concerning the baseline (simple root sentences with ne\-gat\-ed verbs), Neg-Raising sentences, and equatives. For the first environment, the predictions of the standard syntactic approach/\citet{zeijlstra2004sentential} do not differ from the alternative one/\citet{ovalle2004double}. Both approaches agree on the indefinite and positive at-issue meaning of neg-words. The syntactic theory delegates the negative property into the uninterpretable features; the alternative theory explains the negative force as a presupposition. In simple cases, like \REF{ex-neg-concord-a}, both theories predict grammaticality (either via feature checking or by the verification of the emptiness of the reference presupposition).\footnote{ More interesting is how both theories account for the sentences like Spanish \REF{ex-neg-concord-sp}, where the neg-word c-commands a positive predicate. Such configurations are ungrammatical in strict negative concord languages, though. The syntactic approach does not have a straightforward answer for the ungrammaticality of such [Neg-word positive-V] sentences since both neg-words and verbal negation are posited to bear uninterpretable features, so it is not clear why one such feature is not enough to signal the covert Op$_\neg$. A way out is offered by \citet{penka2007negative} in \REF{ex-penka-principle}, but as she herself admits, the principle is not anything else than restating the problem. For the alternative theory, the answer for the ungrammaticality of \REF{ex-neg-concord-sp} in the strict NC languages is straightforward: The presupposition of neg-word clashes with the assertion of the sentence, leading to a contradiction. Nevertheless, the alternative theory has to use more machinery to account for non-strict negative concord languages like Spanish, exemplified in \REF{ex-neg-concord-sp}. The solution, in a nutshell, lies in the accommodation of the emptiness of the reference presupposition, which can happen in specific circumstances. The technical details and extensive discussion can be found in \citet{kuhn2022dynamics}.

\ea\label{ex-neg-concord-sp}\gll Nadie vino.\\
\textsc{neg.word} came\\
\glt `Nobody came.'\z

\ea\label{ex-penka-principle}Principle for the expression of negation: \\
Mark sentential negation on the finite verb, unless this results in a different meaning.\z 
}

The predictions of the standard theory/\citet{zeijlstra2004sentential} concerning Neg-Raising were already introduced. The alternative theory/\citet{ovalle2004double} requires quantifier rising of the neg-word over its licensor (negation) in syntax, but since the scope of negation in Neg-Raising predicates ends on the embedded verb (but in the pragmatic part of the derivation), the alternative theory can predict somehow decreased acceptability of neg-words. Moreover, the emptiness of reference presupposition can be relativized to the belief or other possible worlds. Nevertheless, a full comparison of both theories with respect to Neg-Raising would have to take into account also non-Neg-Raising predicates and islands. Such configurations were not tested in the current experiment, though.

Finally, concerning the equatives, only the alternative theory of neg-words can reasonably explain why neg-words are licensed in the standard clause of equatives. First, the emptiness of reference presupposition can be satisfied in equative sentences like \REF{ex-neg-concord-eq}: It would require that no other fish (with the exception of brown trout) has the particular degree (on the scale of coloration) which is compatible with the truth conditions of the equative. Moreover, the split-scope part of the mechanics would need to quantifier-raise the neg-word over the given operator (MAX), and also, the dynamic properties of equatives would have to be checked off. Precise derivation of this must wait for future work, but the alternative approach has at least a good chance to derive the empirical asymmetry: Strict negative concord languages seem to allow the neg-words in the standard clauses of equatives but do not allow (strong) NPIs there. There are a couple of other environments studied before where such licensing of neg-words goes beyond negation: the complement of prepositions like \textit{without}, and licensing of Spanish neg-words under verbs like \textit{forbid, doubt} and \textit{deny} (see \cite{herburger2001negative}). 

I will end this section via recapitulation of the predictions. As is clear, the predictions are very much theory-dependent, and for many patterns, the non-standard theories (either in the polarity or in the degree theories) are more promising than the established ones. \tabref{tab:expected_responses} represents the predictions for three conditions: \textsc{bas}(eline), \textsc{n}(eg-)\textsc{r}(aising) and \textsc{eq}(uatives). We can expect that baseline will be acceptable for all speakers. Non-standard theories of neg-words/\citet{ovalle2004double} predict acceptance of neg-words in Neg-Raising predicates.  Alternative neg-word theories/\citet{ovalle2004double} and alternative degree theories of equatives/\citet{penka2016degree} predict acceptance of neg-words in equatives and rejection of strong NPIs in equatives. On the other hand, standard syntactic and degree theories predict no neg-word licensing in equatives (rejection) and licensing of strong NPIs (acceptance), due to the downward-entailing environment of the equative clause. And likewise for neg-raising predicates. 

\begin{table}
	\centering
	\caption{Expected acceptability (Czech speakers)}\label{tab:expected_responses}
	\begin{tabular}{lccc}
		\lsptoprule%\hline
		Condition & \textsc{bas} & \textsc{nr} & \textsc{eq} \\ \midrule%\hline
		strong NPIs (standard NPIs theories) & High & High & High \\
		neg-words (standard neg-words theories/ & High & Low & Low \\
    \cite{zeijlstra2004sentential}) & & & \\
		strong NPIs (non-standard equative theories/ & High & High & Low \\
    \cite{penka2016degree}) & & & \\
		neg-words (non-standard  & High & Low & High \\
    neg-words theories/\cite{ovalle2004double}) & & & \\ \lspbottomrule%\hline
    %\hline
	\end{tabular}
\end{table}


\subsection{Research questions}

We will tackle two questions. The first question, in \REF{ex-6}, is the main empirical question behind the experiment and, more generally, the search for the distinction between Czech strong NPIs and neg-words focused on one particular environment. 

\ea\label{ex-6} Question 1: Are Czech equatives acceptable with neg-words and unacceptable with strong NPIs?
\z

\noindent The question is theoretically important since the current standard theories of equatives (like \cite{stechow1984comparing,beck_comparison_nodate}) build upon the analysis of the \textit{as}-clause of the equatives as downward-monotonic, therefore predicting at least grammaticality of weak NPIs and unacceptability of negation, negative quantifiers (and neg-words in languages with negative concord). This is the empirical pattern of English, but as suggested above, exactly the opposite is true for Slavic (as well as for German and other non-English) equatives. The experiment also scrutinizes Neg-Raising. For Neg-Raising the acceptability pattern is expected to be reversed (compared to equatives): Strong NPIs should be more acceptable than neg-words. But since both the standard/\citet{zeijlstra2004sentential} and the alternative theory of neg-words/\citet{ovalle2004double} predict the same pattern in the case of Neg-Raising, the first research question is focused on equatives only. The theoretical consequence of the positive answer to the first research question is empirical support for non-standard theories of neg-words and equatives.

The second question, in \REF{ex-7}, concerns the factors of the variation in the acceptability of strong NPIs. 

\ea\label{ex-7} Question 2: Is speaker variation of Czech strong NPIs caused by grammatical or demographic factors?\z

\noindent As introduced above, previous works on variation in polarity-sensitive expressions revealed that both grammatical and demographic factors could play various roles in the speaker variation of the negative-dependent expressions. \citet{burnett2015variable} convincingly show that next to grammatical (syntactic) factors, a proper analysis of variation should control for age and education level, since these demographic factors explain some portion of the speaker variation for negative concord (absence or presence of negative concord in Montréal French in the case of \citealt{burnett2015variable}). Since variation in speakers and their interpretation of strong NPIs was detected in previous research \citep{docekaldotlacilsubber,dovcekal2020n}, I included age, reading time (as a measure of education level or aspiration), and region as demographic questions in my experiment. The second research question in \REF{ex-7} phrases exactly this research agenda targeting demographic factors. And as will be shown, the experimental data give us precise-enough results to give some answers to both questions.

\section{Experiment}\label{sec:experiment}

% https://apastyle.apa.org/learn/faqs/subjects-and-participants
% as for: Throughout section 2: obsolete term: “subjects” → “participants”

The experiment aimed at answering the two research questions, \REF{ex-6} and \REF{ex-7}: to test acceptability of neg-words and strong NPIs in Neg-Raising and in equatives. Next, the speaker variation was tested as well.

\subsection{Methods}

\subsubsection{Participants \& fillers} 

The experiment was run online on the L-Rex platform \citep{l-rex2023}. The participants were students of Masaryk University (Brno) and Charles University (Prague), and the majority of the students received credit for their participation. 105 participants filled out the experiment. The experiment included practice items to help subjects familiarize themselves with the acceptability judgment task, which was then used in the experiment itself. The experiment also included 64 fillers, half of them grammatical Czech sentences and half clearly ungrammatical sentences. Both halves of the fillers were complexity-wise similar to the items; the ungrammatical fillers included unlicensed anaphors and neg-words unlicensed by constituent negation, a.o. The exclusion rate was 66\% success. 82 of the participants passed the fillers, and their data points were included in the analysis.\footnote {\label{fn:success}The criterion was whether the subject was more than 66\% successful in fillers or not. One of the two anonymous reviewers asked about the details of the exclusion rate and also why the most-used standard rate, 75\%, was not used. The exclusion criterion was measured as follows: For each participant the difference between answers to good and bad fillers was computed -- since the scale is 7-point, the difference was in the interval 0 to 6. The success rate was then computed on the difference scale for each subject. As a sanity check, I ran the analysis with 75\% exclusion rate during the revisions of the article. The descriptive and inferential statistics remained the same as in the original analysis modulo changes in the second digit to the right of the decimal point; also the strength of the effects remained the same.} 

\subsubsection{Materials \& procedure}

Each questionnaire consisted of 64 items, and there were 48 randomized lists generated from the items by L-Rex. The questionnaire started with three demographic-related questions: (i) the age of the participant, (ii) the region of the participant during their first language acquisition, and (iii) their daily reading time (explained as reading time of books and/or journals, not looking at the screen of phones, etc.). Each participant filled out 128 trials (half items, half fillers) in the acceptability part of the experiment, which is the part reported in the present article.\footnote{\label{footnote:the_other}The other part of the experiment was an acceptability judgment task with probability/scalarity manipulated. The results of the second part are not reported in the present article due to space reasons.}

The experiment consisted of two parts: (i) an acceptability judgment task where sentences were judged without context, and (ii) an acceptability judgment task where sentences were judged against a probability/scalarity manipulated context (see footnote \ref{footnote:the_other}). In both parts, participants judged the acceptability of sentences on a 1 to 7-point Likert scale (1 the worst -- the least acceptable, 7 the best -- the most acceptable). In both parts, all conditions were crossed with two conditions: (i) \textsc{neg-words}, (ii) \textsc{strong npi}s. 

An example item from the experiment is in \REF{ex-8}. There were three conditions: (i) \textsc{bas(eline)}, \REF{ex-8-a}, (ii) Neg-Raising, \textsc{nr}, \REF{ex-8-b}, and (iii) equative, \textsc{eq}, \REF{ex-8-c}. All three conditions were crossed with two types of negative polarity expression: (i) neg-words \textit{žádný}, \textsc{z}, and (ii) strong NPIs \textit{ani}, \textsc{a}. Therefore the experiment was a 3x2 design. A mnemonic for crossed conditions for baseline are \textsc{basa} for strong NPIs, and \textsc{basz} for neg-words.

\ea\label{ex-8} \ea\label{ex-8-a} \gll V království nezůstal \minsp{\{} žádný / ani~jeden\} zloděj.\\
in kingdom \textsc{neg}.remained {} \textsc{neg.word} {} \textsc{npi} thief\\
\glt `No thief remained in the kingdom.' 
\ex\label{ex-8-b} \gll Král nechce, aby v království zůstal \minsp{\{} žádný / ani~jeden\} zloděj.\\
King \textsc{neg}.wants that in kingdom remained {} \textsc{neg.word} {} \textsc{npi} thief\\
\glt `The king doesn't want any thief to remain in the kingdom.' 
\ex\label{ex-8-c} \gll Zloděj ze souostroví Qwghlm je tak šikovný jako \minsp{\{} žádný / ani~jeden\} zloděj.\\
thief from archipelago Qwghlm is so clever how {} \textsc{neg.word} {} \textsc{npi} thief\\
\glt `The thief from the Qwghlm archipelago is as clever as any other thief.'
\z\z

\subsection{Predictions}

All the discussed theories predict that neg-words and strong NPIs will be accepted in the baseline condition, \textsc{bas}. The condition is present in the experiment to check how much worse the other two conditions will be compared to the baseline. 

On the contrary, the standard theory of neg-words/\citet{zeijlstra2004sentential} and the alternative theory of neg-words/\citet{ovalle2004double} differ in their predictions for equatives, \textsc{eq}: The standard theory predicts neg-words to be not acceptable in \textsc{eq} while the alternative theory is compatible with their acceptability in \textsc{eq}. The standard theory of equatives/\citet{rullmann1995maximality} predicts the acceptability of strong NPIs in \textsc{eq} while the alternative theory of equatives/\citet{penka2016degree} predicts their unacceptability. 

Finally, both theories of neg-words predict the decreased acceptability of neg-words in \textsc{nr} and much higher acceptability of strong NPIs in \textsc{nr}.

Concerning the speaker variation, there are no theory-specific predictions. But since the speaker variation was observed in previous research, it is expected that the variation will be observed in the current experiment too. The demographic factors were included in the experiment to test whether the variation is related to grammatical or demographic factors.



\subsection{Results}
\largerpage
The descriptive-statistics results can be seen in the graph of acceptance, including error bars, in \figref{fig-error-bar}.\footnote{The graph of acceptance in \figref{fig-error-bar} uses the standard Cartesian coordinate system with the \textit{y}-axis origin at 0. That does not mean that the response scale was 0 to 7 but is simply the default behavior of the \textsc{ggplot2} R package (see \citealt{ggplot}).}
As can be seen with the naked eye, both expressions are nearly at the ceiling in \textsc{bas}, but their acceptability in the two other conditions is reversed. While neg-words are much more acceptable in equatives (\textsc{EqZ}), strong NPIs are preferred in Neg-Raising contexts (\textsc{nrA}).\footnote{One of the two anonymous reviewers asked whether unacceptable fillers were judged worse than neg-words (or strong NPIs) in Neg-Raising contexts. The answer is yes, the ungrammatical fillers' median acceptability was 1, while the median acceptability of neg-words and strong NPIs was 2.}

\begin{figure}
  \caption{Graph of acceptance (+error bars) conditions: \textsc{bas}eline, \textsc{eq}uative, \textsc{N}eg\textsc{R}aising expressions: \textsc{a}ni (strong NPI), \textsc{z}ádný (neg-word)}
  \includegraphics[height=0.45\textheight]{figures/doc-error_bar_without_prob-eps-converted-to2.pdf}
  \label{fig-error-bar}
\end{figure}

\subsubsection{Inferential statistics}\label{sec:experiment-inf}

The Bayesian hierarchical random-effects model with default priors was fit using the R package \textsc{rstanarm} \citep{rstanarm}: The dependent variable was the subject's response; the independent variables were: (i) environment (\textsc{bas, eq, nr}), (ii) type of the polarity-dependent expression (\textsc{a, z}), and their interaction; the reference level was \textsc{bas, a}. The baseline was selected as the condition which should be uncontroversially accepted by speakers, which indeed was the case since no main effect was positive against the baseline. The model included random effects for both subject and item intercepts.

The model was fit to the data and we found that (i) the baseline was very well accepted ($\mathrm{Intercept}=6.67$, 95\% C(redibility) I(nterval)$=[6.38, 6.95]$); there is no distinction between neg-words and strong NPIs in it and sine qua non, both expressions are acceptable to the same extent (posterior main effect in the form of median and 95\% CI: $\hat{\mu}=-0.20$, $\mathrm{CI}=[-0.47, 0.08]$); (ii) neg-words were much better accepted in equatives than strong NPIs (the positive interaction of \textsc{eq} by \textsc{z}: $\hat{\mu}=2.18$, $\mathrm{CI}=[1.81, 2.58]$ -- against the reference level); (iii) strong NPIs were preferred in Neg-Raising (the negative interaction of \textsc{nr} by \textsc{z}: $\hat{\mu}=-0.53$, $\mathrm{CI}=[-0.91, -0.14]$ -- against the reference level).\footnote{As one of the two anonymous reviewers correctly points out, \textsc{nr} (whether it includes strong NPI or neg-word) is surprisingly poorly accepted -- see the row of \textsc{nr} in \tabref{tab:exp1}. I agree, but this seems to be the case generally; the results of my experiment resonate with the experimental finding from \citet{dovcekal2016experimentala}, where the strong negative effect between the acceptable baseline and tested Neg-Raising was observed. The reasons for this negative effect are unclear, but see \citet{alexandropoulou2020there} for some other environments where including NPIs leads to a strong decrease in acceptability even if such an effect is theoretically unexpected.}
The results are also supported by the results of R(egion) O(f) P(ractical) E(quivalence), ROPE: Only \textsc{z} is not significant, since it is 23\% in ROPE. For all medians, confidence intervals, and ROPE percents, see \tabref{tab:exp1}; all percents of ROPE are computed for the interval $[-0.10, 0.10]$. Medians, 95\% credibility intervals, and ROPE are also visually represented by the graph in \figref{fig-rope}. Notice that in \figref{fig-rope}, as is usual in Bayesian modeling, the reference level condition (\textsc{bas, a}) is coded as 0 of the x-axis and each condition (or interaction of condition) has its own y-line (with the distribution and median); the credibility of a condition can be visually inspected via observation of the condition including 0 (e.g. clearly \textsc{z}) or differing from it either positively (e.g. the interaction between \textsc{eq} by \textsc{z}) or negatively (e.g. \textsc{nr}).



\begin{table}
  \begin{tabularx}{.9\textwidth}{lYYYr}
  \lsptoprule
  \textbf{Parameter}&\\\midrule
    &            Median & CI &  \% in ROPE\\  
    Intercept               &$6.67$   &$[6.38,6.95]$ &$0\%$\\  
    \textsc{Eq}               &$-3.27$   &$[-3.53,-3.00]$ &$0\%$\\  
    \textsc{nr}               &$-3.57$   &$[-3.86, -3.30]$ &  $0\%$\\  
    \textsc{z}               &$-0.20$   &$[-0.47,  0.08]$ &$23.08\%$\\  
    \textsc{Eq:z}               &$2.18$   &$[1.81,  2.58]$   &$0\%$\\
    \textsc{nr:z}               &$-0.53$   &$[-0.91, -0.14]$   &$0\%$\\  
    \bottomrule
    \textbf{Random effects}&\\\midrule
    & Name &SD\\
    subject&Intercept&$0.57$\\
    item&Intercept&$0.35$\\
  \lspbottomrule
  \end{tabularx}
  \caption{Bayesian model and its posterior distribution for the experiment}%; log-likelihood: -638.5
  \label{tab:exp1}
\end{table}

\begin{figure}
  \caption{Graph of posterior samples with ROPE (for the experiment)\\conditions: \textsc{bas}eline, \textsc{eq}uative, \textsc{N}eg\textsc{R}aising\\expressions: \textsc{a}ni (strong NPI), \textsc{z}ádný (neg-word)}
  \includegraphics[height=0.45\textheight]{figures/doc-posterior_graph.pdf}
  \label{fig-rope}
\end{figure}

\subsubsection{Demographic factors}

Next, three Bayesian generalized mixed linear models were fitted to detect the effects of demographic factors inhibiting or prohibiting acceptability. This was important since previous work (see \citealt{burnett2015variable} and \citealt{burnett2018structural}, a.o.) revealed that both grammatical and demographic factors are at play when negative polarity variation is linguistically studied. As a reminder, the experiment included three demographic questions: region, age, and daily reading time. The last factor was used as a proxy for investigating educational level. The selection of factors was influenced by the previous work on variation in negative dependent expressions: \citet{burnett2015variable,burnett2015variable} have shown that age, education level, and location of the speakers can have an impact on the variation. And since in the previous experimental work it was revealed that there are idiolects of Czech speakers interpreting the strong NPI \textit{ani} \citep{docekaldotlacilsubber,dovcekal2020n}, I included the three mentioned demographic factors to detect whether the variation can be traced to some extralinguistic sources eventually. Nevertheless, it has to be said that the pool of subjects was rather homogeneous, consisting mainly of university students. Therefore, at least the education-level results should be taken with a grain of salt and mainly as a first step in the general description of polarity items' variation in Slavic languages. Also, for this reason, I move the inferential statistics into footnotes and describe the main outcomes in terms of descriptive statistics.

Let us start with \textsc{age}. Descriptively, \textsc{age} ranged from 19 to 71, with a $\text{median}=23$, $\text{mean}=25.59$, and $\text{sd}=9.47$. The age was first z-transformed and then plugged in as the third interaction variable in the Bayesian model (next to the two conditions, \textsc{z} and \textsc{bas/nr/eq} environment).%
\footnote{The model revealed that acceptability overall was not affected by age at all (main effect of \textsc{age}: $\hat{\mu}=0.01$, $\mathrm{CI}=[-0.25,  0.28]$, ROPE: $58.00\%$ for the $[-0.10, 0.10]$ interval). There was also no significant interaction with any single condition or pair of conditions. The lowest ROPE was 30.55\% for the three-way interaction between \textsc{Eq:Z:age}. All other interactions had an even bigger portion in ROPE and were also less significant.}
But the model did not confirm any effect on overall acceptability or any particular age-related inhibition or prohibition of any construction or negative-dependent expression.  

Next, \textsc{region} was more varied than \textsc{age}, where the data points from the first to the third quantile were in the range of 21 to 25. But since I did not control for the specificity of the values entered into the form, the answers ranged from city-specific to region-specific. For this reason, I aggregated all the answers into a discrete factor with two levels: \textsc{Moravian} and \textsc{nonMoravian}. 67\% of subjects entered as their region \textsc{nonMoravian}; the remaining 33\% identified themselves as being from Moravia. Again the factor \textsc{region} was used as the third interaction variable in the Bayesian model.%
\footnote{The main effect of the region was not credible ($\hat{\mu}=0.33$, $\mathrm{CI}=[-0.18,  0.88]$, ROPE: $14.79\%$ for the $[-0.10, 0.10]$ interval) but this time there was very weak evidence coming from interactions. Namely, there seemed to be a slight tendency for higher acceptance of Neg-Raising in the non-Moravian part of the Czech Republic (the interaction \textsc{nr:Moravian}: $\hat{\mu}=-0.61$, $\mathrm{CI}=[-1.28,  0.04]$, ROPE: $3.89\%$ for the $[-0.10, 0.10]$ interval). All other interactions with \textsc{region} were less significant.}
Overall, the \textsc{region} did not increase or decrease acceptability, but there is some anecdotal evidence for higher acceptance of Neg-Raising in the Czech (non-Moravian) part of the population. Nevertheless, the interaction effect is so weak that I doubt there is any genuine linguistic Neg-Raising isogloss between Czech dialects.

The last demographic factor was reading time. As hinted above, the factor was used to get information about education or education aspirations. The answers (converted to hours) ranged from 0 to 10 hours, with 1 hour as the median, 1.43 hours as the mean, and the range of first and third quantiles being 1 hour and 2 hours, respectively. Similarly to \textsc{age}, data points are centered around the mean with a small standard deviation, 1.26, and few outliers. As in the case of \textsc{region}, I recorded the continuous variable as a factor \textsc{readingTime} with two levels: \textsc{over1hour,under1hour} dividing the sample according to the median value of reading time. The result was two nearly proportional halves: 52\% of the subjects claimed that their daily reading time is under 60 minutes, and the remaining 48\% entered that they read more than one hour. The third demographic factor (\textsc{readingTime}) was plugged into the Bayesian model as an independent (interaction) variable.%
\footnote{And again, as with two previous demographic factors, the main effect of \textsc{readingTime} was not credible ($\hat{\mu}=-0.13$, $\mathrm{CI}=[-0.63,  0.39]$, ROPE: $28.89\%$ for the $[-0.10, 0.10]$ interval). And similarly to \textsc{region}, there was one weakly-credible interaction: subjects claiming to read more than average (over 60 minutes daily) were accepting Neg-Raising constructions more (\textsc{nr:over1hour} interaction: $\hat{\mu}=0.66$, $\mathrm{CI}=[ 0.04,  1.27]$, ROPE: $1.24\%$ for the $[-0.10, 0.10]$ interval). All other interactions were much less credible.}
The modeling results show that there is some weak evidence for the positive correlation between reading time and the acceptance of the Neg-Raising construction: Subjects who claimed to read more were more accepting of the Neg-Raising construction. Such a tendency is intuitively plausible but does not say anything linguistically important about the constructions and polarity-dependent expressions tested in the experiment.


Let us summarize: The design of the experiment and three demographic questions did not reveal any important information concerning the demography-related variation in polarity constructions of Czech speakers. Two weak effects can be interpreted as clues about region and education-level variation concerning Neg-Raising. Still, there seems to be nothing significant in the variation of \textit{ani} vs. \textit{žádný} in the studied constructions. So, whatever speaker variation (in the usage of \textit{ani}) we will discuss further, it seems not to be related to age, region, or education level as revealed by the sample of the experiment (in this respect, the results of the experiment are different from the previous work on speaker variation in polarity dependent expressions, like \citealt{burnett2015variable,burnett2018structural}). 


\subsubsection{Correlations}\label{sec:corr}

The Bayesian model revealed that both non-baseline environments (\textsc{nr} and \textsc{Eq}) were less accepted by speakers, but there was no difference between \textit{ani} and \textit{žádný} in terms of main effects. Nevertheless, speakers accepted in equatives many more neg-words than strong NPIs (the strong effect, and the only one which yielded positive interaction). But speakers are also inclined to reject neg-words in Neg-Raising against strong NPIs (the negative interaction effect between \textsc{z} and \textsc{nr}). The theoretical consequences of these findings will be discussed below, but let us turn to another kind of variation, this time not demographic.

The first important thing to note is that all speakers agreed on their high acceptance of baseline, and in this condition, they accepted neg-words and strong NPIs indistinguishably. But this acceptance of both polarity expressions diverged in the two other conditions. Namely, some speakers rate \textit{ani} high in equatives (unlike the main thrust of speakers; recall the strong positive interaction between \textsc{z} and \textsc{Eq}) but also reject it in NegRaising (again going against the overall acceptance of strong NPIs there: the negative interaction between \textsc{z} and \textsc{nr}). And vice versa, subjects who reject strong NPIs in equatives (behaving according to the negative interaction effect) accept strong NPIs in Neg-Raising (again verifying the negative interaction effect). In both cases, we observe a negative correlation between the acceptability of neg-words/strong NPIs in the two environments, equatives and Neg-Raising. One way to understand this reversed correlation is to assume that the first kind of speaker (those who accept \textit{ani} in equatives) treats \textit{ani} more like a neg-word and not like a strong NPI. The rest of the sample (the majority, in fact) treats \textit{ani} as a strong NPI and therefore accepts it under Neg-Raisers and rejects it in equatives.

Therefore, post-hoc correlation statistics were run and are reported below. Notice, though, that the correlation is post-hoc in the sense of interpreting sub-clusters of speakers (let us say idiolects in linguistic terms) but not in the sense of avoiding the Type I error (mistakenly rejecting the null hypothesis) since the null hypothesis is not an important part of Bayesian statistics; instead the bulk of the inference statistics in Bayesian framework is posterior distribution, which represents the probability of the parameters of a model given the data and which was reported here in \sectref{sec:experiment-inf}. The motivation for the correlation analysis comes both from the observed variation introduced above and from the previous work on Czech strong NPIs \citep{docekaldotlacilsubber,dovcekal2020n}, where it was observed that there are idiolects of Czech speakers concerning their strong-NPI interpretation.

The way the correlations were checked statistically is the following. First, the acceptance of conditions was z-transformed (by subject). Then, such z-transformed variables were checked for correlations across conditions. Indeed, there is a strong negative correlation between the acceptability of \textit{ani} in equatives and its acceptability under Neg-Raising predicates (Pearson's product-moment correlation: $t = -5.93, p < 0.001$). The correlation graph is in \figref{fig-ani-corr}. This means that we can identify two groups of speakers: (i) speakers who accept \textit{ani} under Neg-Raisers and reject it with equatives (top left section in \figref{fig-ani-corr}), and (ii) speakers who accept \textit{ani} in equatives and reject it under Neg-Raisers (the bottom right part of \figref{fig-ani-corr}). But crucially, no speakers are accepting both conditions (the empty top right corner) nor speakers who would reject both conditions (the empty space in the bottom left part). And also, there is no correlation between the acceptability of \textit{ani} in the baseline and equatives, just pure noise, as can be seen in \figref{fig-ani-bas-corr}. This correlation of \textit{ani} between \textsc{nr} and \textsc{Eq} resonates with the previous work \citep{docekaldotlacilsubber} where similar correlations were found (for \textit{ani}) in the case of probability-manipulated conditions and Neg-Raising.

\begin{figure}
    \centering
    \caption{Correlations between Equatives and Neg-Raising for \textit{ani}}
    \includegraphics[scale=0.6]{figures/doc-correlations_ani.png}
    \label{fig-ani-corr}
\end{figure}
  
\begin{figure}
  \caption{Correlations between Equatives and Baseline for \textit{ani}}
  \includegraphics[scale=0.6]{figures/doc-equatives_baseline_corr.png}
  \label{fig-ani-bas-corr}
\end{figure}

\subsubsection{Discussion}

The reported experimental results suggest that Czech strong NPIs are (for most speakers) accepted under Neg-Raising predicates (\textsc{nr} showed a credible negative effect but the negative interaction effect \textsc{nr} by \textsc{z} shows that participants prefer strong NPIs in \textsc{nr}) and rejected in equatives (the main effect of \textsc{eq} is negatively strong, but there is a very strong positive interaction \textsc{eq} by \textsc{z} indicating robust preference for neg-words in equatives). For Czech neg-words, the opposite is true: Most speakers accept them in equatives but reject them under Neg-Raising predicates. In the case of \textit{ani} (strong NPI), there is speaker variation, and some subset of speakers treats it more like a neg-word; nothing similar was found for neg-words. The correlation discussed in \sectref{sec:corr} suggests that there are two kinds of speakers: (i) speakers accepting \textit{ani} in \textsc{nr} and rejecting it in \textsc{eq} (strong NPI treatment of \textit{ani}), and (ii) speakers accepting \textit{ani} in \textsc{eq} and rejecting it in \textsc{nr} (the speakers who use \textit{ani} more like a neg-word).  Moreover, the speaker variation does not seem to be derivable from demographic factors (or at least not from the demographic factors controlled in the experiment).


\section{Theoretical consequences}\label{sec:theoretical_consequences}

The nature of this article is mainly experimental. For this reason and the obvious constraints of space, the consequences of the analyzed experimental data will be discussed only to a limited extent. 

\subsection{Application: strong NPIs}\label{sec:application-strong-npis}

Now, let us demonstrate how the approaches introduced in \sectref{sec:theoretical-backgr} can be applied to the data gathered in the experiment. First, concerning the strong NPIs (Czech \textit{ani}), I will show how the \textit{even}-approach fits the Czech data from the experiment, starting with the baseline. For a reminder, the \tabref{tab:exp1} summarizes predictions of the theories for the experiment. The results of the experiment confirm more the non-standard theories of neg-words/\citet{ovalle2004double} and non-standard theories of equatives/\citet{penka2016degree}.

As discussed above, \textit{ani} comes with the \textit{even}-presuppositions, namely the scalar and the additive. In formal terms, this is translated as an association of \textit{ani} with covert \textit{even} scoping at the propositional level; see a schematic representation of the baseline in \REF{ex-12} and its logical form in \REF{ex-12-a}. Since the sentence is negated, the entailment between numerals is reversed by negation:
$\neg (\llbracket$ one thief $\rrbracket \ldots) \models \neg(\llbracket$two
thieves$\rrbracket \ldots)$. The alternatives to the prejacent come from the focused numeral, since \textit{ani} can associate with nouns, clauses, etc., but as usual in Slavic languages, it associates mostly with its sister node (the numeral in \REF{ex-12}). Because the entailment is reversed, the \emph{even}-approach predictions agree with the downward-entailing explanation. Moreover, since \textit{ani} is a strong NPI, it requires DE/\textit{even}-presuppositions to be satisfied both in truth conditions (configurationally β in \REF{ex-12-a}) but also at the level of non-at-issue meaning (where the exhaustifier, silent \textit{even}, scopes: α in \REF{ex-12-a}). Next, we have to check both presuppositions as schematically formalized in \REF{ex-12-d} and \REF{ex-12-e}, which are also fulfilled. The theoretical prediction of the standard account/\citet{gajewski2011licensing} then agrees with the high baseline acceptability of \textit{ani}: Even the inferential statistic baseline median intercept was 6.67 on the 7-point Likert scale.

  
\ea\label{ex-12} Ani [one]$_F$ thief neg-remained in the kingdom.\\
\ea\label{ex-12-a} ~{[}$_α$ (\textit{even}) {[}$_β$
  \neg [$_γ$ ani one thief remained in the kingdom ]{]} {]}
\ex TC  (in β) DE: {\langscicheckmark}
\ex non-at-issue (in α) DE:
  {\langscicheckmark}
\ex\label{ex-12-d} scalar presupposition of (even): $\rightarrow$
  $\neg$(two thieves remained), $\neg$(three thieves remained),
  \ldots: {\langscicheckmark}
\ex\label{ex-12-e} additive presupposition: $\neg$(two thieves
  remained) $\vee$ $\neg$(three thieves remained), \ldots:
  {\langscicheckmark} \z\z

\noindent Continuing now to the Neg-Raising condition, neg-words were in \textsc{nr} less accepted than strong NPIs; although the effect was not particularly strong, it was still significant. The theories of Neg-Raising \citep{gajewski2007neg,romoli2013scalar} predict that for the Neg-Raising predicates the scope of root negation ends (via presupposition or implicature calculation) in the embedded clause. Schematically we can formalize the important ingredients of the LF for the experiment \textsc{nr} conditions as in \REF{ex-12-nr}. 

\ea\label{ex-12-nr} The king does \sout{not} want [$_{α}$ (\textit{even}) [$_{β} \neg$ [ani [one]$_F$ thief remained in the kingdom]]].
\z

\noindent From the point of view of the standard theory of strong NPIs and the \textit{even}-presuppositions, the logical form of the embedded clause is the same as in the case of the baseline, therefore the explanation of the NR acceptability is the same as in the baseline case. What differs is the actual acceptance by speakers (much lower in the case of Neg-Raising than in the baseline), but as discussed in footnote \ref{fn:success}, in this respect, our experiment nearly replicates the previous findings with respect to Neg-Raising and NPIs generally. The second difference concerns the diverging acceptability of strong NPIs and neg-words in Neg-Raising contexts (both kinds of expressions were indistinguishably well-accepted in the baseline). Still, this point will be discussed in \sectref{sec:neg-words}.

Finally, the standard theory of equatives predicts that strong NPIs can be licensed in the standard clause. This was argued before to be wrong for German and Romance languages \citep{krifka1992some,penka2016degree} and also clashes with the intuition of Czech speakers, for whom strong NPIs were much less accepted than neg-words in the standard clauses of equatives. A proper investigation of Slavic equatives must wait for future work, but let us take the first steps in this direction. If we take seriously the morpho-syntax of Slavic equatives and follow the theoretical hints from \citet{penka2016degree}, it is possible to model the equative conditions from the experiment as in \REF{ex-12-eq}. 

\ea\label{ex-12-eq}The thief from the Qwghlm archipelago is so$^2$ clever [$_{α}$ (even) [$_β$ MAX\textsubscript{inf} how$_1$ ani [one]$_F$ thief $t_1$ \sout{clever is}.]]$^2$
\z

\noindent First, since the morpho-syntax of Czech (and Slavic, Romance, German) equatives is built upon the correlatives, anaphoric \textit{so} (Czech \textit{tak} `so', see \REF{ex-8-c} for full glosses) picks up the referent of the definite degree description (index 2). The definite degree description is yielded by the MAX\textsubscript{inf} operator, similar to free relatives. And since free relatives are known not to license NPIs \citep{pauline1995quantificational}, the correlative standard equative clause is expected not to be a good environment for NPIs. Therefore we can predict, if we assume this non-standard but well-motivated theory of equatives, why in Czech equatives, strong NPIs are not accepted: Since β is most probably not DE, even if the presuppositions of silent \textit{even} in α were satisfied (MAX\textsubscript{inf} makes the standard of equatives most probably non-monotonic), the licensing of strong NPIs is not satisfied. The much better acceptance of neg-words in equatives will be discussed in \sectref{sec:neg-words}.

\subsection{Neg-words}\label{sec:neg-words}

In this section, I will present the application of theories introduced in \sectref{sec:assumptions-licensing-of-neg-words} to the results of the experiment.

The baseline can be explained easily both in the standard theory of neg-words/ \citet{zeijlstra2004sentential} and in the alternative theory of neg-words/\citet{ovalle2004double}. I discuss just the alternative theory explanation here. In a negative sentence, like schematic \REF{ex-14}, the truth conditions (indefinite descriptions) of the neg-word and its presupposition agree: The indefinite description is under the scope of negation, and the presupposition of the emptiness of the discourse referent is compatible with the truth conditions. But in a positive minimal pair sentence, like \REF{ex-15}, the existential quantification over the discourse referent and the presupposition of its reference emptiness would clash. Therefore, the positive minimal pair sentence is predicted to be unacceptable. This is exactly what we observe in the experiment: The baseline is very well accepted, but the positive minimal pair is rejected.

\ea\label{ex-14} neg-word thief neg-remained in the kingdom. \ea {[}$\neg[\exists x[\textsc{thief}(x) \wedge \textsc{remained}(x)]$ {]}{]} $\wedge 0_x$\z\z

\ea\label{ex-15} neg-word thief remained in the kingdom. \ea {[}$\exists x[\textsc{thief}(x) \wedge \textsc{remained}(x)]$ {]} $\wedge 0_x$ \hfill $\bot$\z\z

\noindent Turning now to Neg-Raising, neg-words were less accepted in \textsc{nr} than strong NPIs. From the perspective of the alternative theory of neg-words/\citet{ovalle2004double}, we should expect that this would align with the locality constraints: The neg-words have to quantifier-raise over their licensors, creating a split-scope logical form. But since the root negation ends in the embedded clause in the case of Neg-Raising predicates, as discussed above, the schematic logical form for Neg-Raising conditions can be rendered as \REF{ex-16}. This reasoning explains the difference between relatively freely-licensing (both for neg-words and strong NPIs) Neg-Raising predicates and much worse non-Neg-Raising predicates (verbs of causation or communication) in such configurations like \REF{ex-16} -- see \citet{dovcekal2016experimentala} for experimental data and analysis. Recall that experimental results showed a slight preference for strong NPIs in this condition. The difference can be captured as follows: Since the split scope relies on syntactic mechanisms and the negation ends in the embedded clause in the pragmatic part of the derivation, there is a slight timing issue which maybe can be coerced. A similar kind of explanation can be retold in the standard approach/\citet{zeijlstra2004sentential}, too: The syntactic licensing of features should proceed before the pragmatic mechanisms like presupposition and implicature calculation. But in both kinds of explanation, a lot hinges upon the assumptions about the architecture of grammar and many other assumptions. Moreover, it is not totally clear how to linguistically interpret the effects from the experiment -- as \tabref{tab:exp1} shows, both kinds of expressions (strong NPIs and neg-words) lead to a dramatic acceptability decrease ($-3.57$), but there is the slight preference for strong NPIs (the negative interaction of neg-words with Neg-Raising: $-0.53$).

\ea\label{ex-16} The king wants \sout{not} that  [neg-word thief$_2$ $\neg$ [$t_2$ remained in the kingdom]].
\z

\noindent Finally, concerning the equatives: (i) As tested in the experiment, neg-words are, but strong NPIs are not, acceptable in the complement clause of the Czech equatives; (ii) adding to this, verbal negation is not acceptable either -- see \REF{ex-20}. The high acceptability of neg-words is especially surprising from the perspective of English since negative quantifiers are distinctly odd in this position (in comparatives, but as discussed in \ref{sec:assumptions-licensing-of-strong-npis}, the expectations are -- in the standard theories -- similar for comparatives and equatives), see \REF{ex-21} from \citet{gajewski2008more}.

  
\ea\label{ex-20} \gll Petr je tak chytrý jak \minsp{\{} nikdo jiný / *Marie ne / *ani jeden\}.\\
Petr is so smart as {} \textsc{neg.word} else {} Mary not {} \textsc{strong.npi}\\
\glt `Petr is as smart as anyone.'
\z  

\ea\label{ex-21} *Mary is taller than no boys are.
\z
  
\noindent The ambition of this article is not to solve the above-mentioned theoretical puzzles. But let us at least indicate where a possible solution can be. The experimental results show that Czech neg-words are very much accepted in the complement clause of equatives, while strong NPIs are degraded there. Moreover, intuitions and preliminary results from the follow-up experiment suggest that weak NPIs are not acceptable in the equatives either, following the German and Romance data discussed in \citet{krifka1992some} and \citet{penka2016degree}. One possible explanation is that Czech complement clauses of equatives are not downward-monotonic in either truth conditions or non-at-issue meaning. But for some reason, the emptiness of the neg-words' discourse referents' presupposition can be easily satisfied in this environment. Consider \REF{ex-21-eq}: The emptiness of discourse referents' presupposition here would be that there is no such thief with the degree of cleverness $d$ which would exceed the degree of the Qwghlm archipelago thief. Such a presupposition is plausible, and more generally, it can be said that Czech equatives are one of the rare environments where the neg-words can be licensed by expressions not including negation (similarly to Spanish verbs like \textit{forbid, doubt} or \textit{deny}).

\ea\label{ex-21-eq}The thief from the Qwghlm archipelago is so$^2$ clever [MAX\textsubscript{inf} how$_1$ neg-word thief $t_1$ \sout{clever is}.]$^2$
\z

\noindent This can be compatible with \citeposst{penka2016degree} suggestion to replace the MAX operator in analyzing English equatives with a different relation on the degrees, MAX\textsubscript{inf} discussed in \sectref{sec:application-strong-npis}. But so far, I consider the evidence to be inconclusive regarding the monotonic properties of Czech complement clauses of equatives; they are not downward-entailing for sure, but the resulting two possibilities, upward entailing or non-monotonic, are still open. From a theoretical standpoint, I agree with \citet{penka2016degree} that current degree theories of equatives do not hold the cross-linguistic water. And in the same direction, it is clear that a purely syntactic approach to neg-words faces big trouble when posed with the acceptability of neg-words in equatives. No matter how the cross-linguistically feasible degree theory of equatives will look like (e.g., using MAX\textsubscript{inf} as suggested by \cite{penka2016degree}), there is clearly no room for a sentential negation operator in its version for Slavic (and Romance) equatives, since then the weak or strong NPIs would be admissible there, contrary to facts. Concluding this section, merging the non-standard theory of neg-words/\citet{ovalle2004double} and the non-standard theory of equatives/\citet{penka2016degree} is promising as a theoretical explanation for the high acceptability of neg-words in the standards of Czech equatives and the very low acceptance of strong NPIs there.


% First, I will foll \citet{penka2016degree} in taking the morphosyntactic clues seriously and replace the standard `>' theory MAX operator with $max \rightarrow max_{inf}$, corresponding to the operator used in the analysis of correlatives. The logical form for a schematic \textsc{eq} like \REF{ex-22} is in \REF{ex-22-a}. Czech \textit{tak} `so' is anaphoric to the degree from the main clause, \REF{ex-22-b}. Then the neg-word presupposition (from the semantic/pragmatic theory of neg-words) has to be satisfied: \REF{ex-22-d}.

%   \begin{itemize}
  
%   \item
%     syntactic and semantic ingredients (pseudoCzech in NEXT)
%   \item
%     non-standard: $max \rightarrow max_{inf}$ (otherwise $max$ would
%     lead to $\bot$): \cite{penka2016degree}
%   \end{itemize}

%   Motivation of the ingredients:
  
%   \begin{itemize}
  
%   \item
%     $max_{inf}$: the equative in Czech has exactly the same building
%     blocks (\emph{tak} `so' \ldots{} \emph{jak} `how') as correlative
%     constructions
%   \item
%     \emph{other}: the anaphor similar to reciprocal anaphors
  
%     \begin{itemize}
    
%     \item
%       it identifies the dref
%     \item
%       it is also used in the exceptive phrases from which the
%       presupposition comes: \emph{Nobody other than John neg-came}
%       presupposes that John came (as the only exception)
%     \end{itemize}
%   \item
%     neg-word presupposition ranges over the dref picked up by the
%     reciprocal
%   \end{itemize}
  
 

% LOOK AT IT AGAIN

% \ea\label{ex-22} This thief is so clever how neg-word other thief.\\
%   \ea\label{ex-22-a} {[} so {[}so$_1$ no other thief $t_1$ clever {]}{]}$_2$
%   {[}This thief is $t_2$ clever{]}\\
%   \ex\label{ex-22-b} $\llbracket so\rrbracket$ \ldots{} picks up the degree denoted by
%   the standard clause\\
%   \ex\label{ex-22-c} $\llbracket$ so$_1$ neg-word other thief clever is
%   $\rrbracket$\\
%   \ex\label{ex-22-d} nobody other than the thief is $d$-clever \hfill neg-word
%   presupposition\\
%   \ex the thief is $d$-clever \hfill implicature of \emph{other}\\
%   \z\z
  
%   \ea \ea $\llbracket$ as
%   $\rrbracket = \lambda S\lambda C.max(C) \geq max(S)$\\
%   \ex
%   $S' \subseteq S: max(C) \geq max(S) \rightarrow max(C) \geq max(S')$
%   \hfill English DE \textit{as}\z\z
  
\section{Summary}\label{sec:summary}  

The findings of the current study provide an answer to research question 1, repeated below as \REF{ex-22}. The experiment confirmed the robust acceptability of neg-words in the standard clause of equatives. This can be explained as deriving from the neg-words presupposition relativized to the set of degrees introduced in the main clause if we follow the alternative theory of neg-words/\citet{ovalle2004double}. Neg-words in examples like \REF{ex-20} are accepted since, in this configuration, the presupposition does not require total emptiness of reference, just the emptiness of reference for such discourse referents whose degree would exceed the degree of the subject. Next, the strong NPI unacceptability in Czech equatives is a direct consequence of the Czech equatives complement clauses not being downward-monotonic. The sub-answers to research question 1 are in \REF{ex-23}. The results of the experiment bring empirical support for alternative theories of neg-words/\citet{ovalle2004double} and alternative theories of equatives/\citet{penka2016degree}.

\ea\label{ex-22} Question 1: Are Czech equatives acceptable with neg-words and
non-acceptable with strong NPIs?
\z

\ea\label{ex-23} The non-standard theories of negative concord and equatives give promising answers:\\
  \ea The semantic/pragmatic theory of neg-words allows the presupposition of discourse emptiness to be satisfied (relativized to degrees of the main clause).\\
  \ex The complement clause of the Czech equatives is not downward-entailing.\z\z

\noindent Now, research question 2 is repeated below as \REF{ex-24}. Some speaker variation was observed. Recall that for some speakers \textit{ani} behaved more like a neg-word. Nevertheless, it is not likely that the variation can be related to demographic factors such as age, region, or daily reading time. But there is one way to theoretically explain the variation: We can assume that both neg-words and strong NPIs in Czech come with presuppositions, the emptiness of discourse referents for neg-words, and the scalar presupposition for strong NPIs. Then the flux from strong NPIs to neg-words can be theoretically cashed out as follows: Speakers substitute the scalar presupposition with the emptiness of discourse referents' presupposition. Speculatively, we can try to explain the one-way direction in terms of economy: The scalar presupposition needs a covert exhaustifier, but the emptiness presupposition does not. Therefore it is less costly and more attractive for speakers who oscillate between the two presuppositions.
For this reason, there is no speaker variation concerning neg-words: Adopting the scalar presupposition would mean a less economical logical representation. Why the flux is unrelated to the demographic factors is an issue for future work. The answers are summarized in \REF{ex-25}. 

  
\ea\label{ex-24} Question 2: Is speaker variation of Czech strong NPIs caused by
grammatical or demographic factors?\z

  
\ea\label{ex-25} The speaker variation is explainable as shifting from the scalar to
  the emptiness of the DR presupposition (in the case of \emph{ani jeden}
  `even one'). \ea Social factors don't seem to play a role in this shift.\z\z
  
\noindent Let us end this section with some open questions. The first of them concerns the locality constraints on neg-words' licensing. The alternative theory of neg-words/\citet{ovalle2004double} predicts that the neg-word locality should approximate the quantifier raising. Only syntactic islands (such as relative clauses) should be hard limits for both neg-word licensing and quantifier raising. The syntactic literature on the topic of Slavic quantifier raising seems to argue for a possibility of overt movement (out of non-island clauses) but obligatory reconstruction (see \cite{neeleman2009focus}, a.o.). Still, the experimental research in this direction seems limited to mono-clausal conditions (see, e.g., \cite{ionin2018focus}). So there is space for future research in this direction, and only then can we conclude whether the locality constraints between quantifier raising and neg-word licensing coincide. Another open question concerns the cross-linguistic variation in neg-word licensing: In Romance languages, neg-words are licensed in \textit{before}-clauses and under \textit{doubt}-type predicates; in Slavic languages, this is not the case. The alternative theory of neg-words/\citet{ovalle2004double} predicts that this should follow from the different presupposition projection properties in the two types of languages. Whether this is true remains again a question for future work.
  
\section*{Abbreviations}

\begin{tabularx}{.5\textwidth}{@{}lQ}
\textsc{1}&first person\\
\textsc{aux}&auxiliary\\
\textsc{eq}&equative\\
\textsc{comp}&comparative\\

\end{tabularx}%
\begin{tabularx}{.5\textwidth}{lQ@{}}
\textsc{neg}&negation\\
\textsc{npi}&Negative Polarity Item\\
\textsc{nr}&Neg-Raising\\
\textsc{sg}&singular\\
    
%&\\ % this dummy row achieves correct vertical alignment of both tables
\end{tabularx}

\section*{Acknowledgments}\label{sec:acknowl}
Thanks to the audiences at FDSL-15, two anonymous reviewers, and to  Jakub Dotlačil, Iveta Šafratová, Tereza Slunská, Martin Juřen and many other linguists in Brno and around.  

\printbibliography[heading=subbibliography,notkeyword=this] 
% made capitalization under my name

\end{document}
