\documentclass[output=paper,colorlinks,citecolor=brown]{langscibook}
\ChapterDOI{10.5281/zenodo.15394203}
%\bibliography{localbibliography}

\author{Carla Spellerberg %\orcid{0009-0006-7237-0635}
\affiliation{University of Massachusetts, Amherst}}
% replace the above with you and your coauthors
% rules for affiliation: If there's an official English version, use that (find out on the official website of the university); if not, use the original
% orcid doesn't appear printed; it's metainformation used for later indexing

%%% uncomment the following line if you are a single author or all authors have the same affiliation
 \SetupAffiliations{mark style=none}

%% in case the running head with authors exceeds one line (which is the case in this example document), use one of the following methods to turn it into a single line; otherwise comment the line below out with % and ignore it
%\lehead{Šimík, Gehrke, Lenertová, Meyer, Szucsich \& Zaleska}
% \lehead{Radek Šimík et al.}

% \title{Focus-sensitive particles in Bulgarian: Towards an adverbial-only analysis}
% replace the above with your paper title
%%% provide a shorter version of your title in case it doesn't fit a single line in the running head
\title[Focus-sensitive particles in Bulgarian]{Focus-sensitive particles in Bulgarian: Towards an adverbial-only analysis}
\abstract{This paper is an investigation of the placement and potential adjunction sites of focus-sensitive particles (FSPs) in Bulgarian. In contrast to well-researched languages such as English and German, there is currently no full analysis of FSP-placement in Bulgarian. I propose an analysis here based partly on results of previous analyses by \citet{BuringHartmann2001} for German and \citet{Zanon2018} for Russian, arguing that Bulgarian FSPs adjoin to projections belonging to the extended verbal projection (EVP) as well as a functional projection FP in the nominal domain. In addition, I discuss the implications that right-adjunction of FSPs to F-marked constituents in Bulgarian has for the Particle Theory as proposed here. Future research in this direction could focus on the connection between adjacency of the FSP and overt focus movement as well as semantic restrictions that individual modifiers and particles impose upon the possibility of adjunction of FSPs.

\keywords{focus, focus-sensitive particles, particle placement, Bulgarian}
}

\begin{document}
\maketitle

% Just comment out the input below when you're ready to go.
%For a start: Do not forget to give your Overleaf project (this paper) a recognizable name. This one could be called, for instance, Simik et al: OSL template. You can change the name of the project by hovering over the gray title at the top of this page and clicking on the pencil icon.

\section{Introduction}\label{sim:sec:intro}

Language Science Press is a project run for linguists, but also by linguists. You are part of that and we rely on your collaboration to get at the desired result. Publishing with LangSci Press might mean a bit more work for the author (and for the volume editor), esp. for the less experienced ones, but it also gives you much more control of the process and it is rewarding to see the quality result.

Please follow the instructions below closely, it will save the volume editors, the series editors, and you alike a lot of time.

\sloppy This stylesheet is a further specification of three more general sources: (i) the Leipzig glossing rules \citep{leipzig-glossing-rules}, (ii) the generic style rules for linguistics (\url{https://www.eva.mpg.de/fileadmin/content_files/staff/haspelmt/pdf/GenericStyleRules.pdf}), and (iii) the Language Science Press guidelines \citep{Nordhoff.Muller2021}.\footnote{Notice the way in-text numbered lists should be written -- using small Roman numbers enclosed in brackets.} It is advisable to go through these before you start writing. Most of the general rules are not repeated here.\footnote{Do not worry about the colors of references and links. They are there to make the editorial process easier and will disappear prior to official publication.}

Please spend some time reading through these and the more general instructions. Your 30 minutes on this is likely to save you and us hours of additional work. Do not hesitate to contact the editors if you have any questions.

\section{Illustrating OSL commands and conventions}\label{sim:sec:osl-comm}

Below I illustrate the use of a number of commands defined in langsci-osl.tex (see the styles folder).

\subsection{Typesetting semantics}\label{sim:sec:sem}

See below for some examples of how to typeset semantic formulas. The examples also show the use of the sib-command (= ``semantic interpretation brackets''). Notice also the the use of the dummy curly brackets in \REF{sim:ex:quant}. They ensure that the spacing around the equation symbol is correct. 

\ea \ea \sib{dog}$^g=\textsc{dog}=\lambda x[\textsc{dog}(x)]$\label{sim:ex:dog}
\ex \sib{Some dog bit every boy}${}=\exists x[\textsc{dog}(x)\wedge\forall y[\textsc{boy}(y)\rightarrow \textsc{bit}(x,y)]]$\label{sim:ex:quant}
\z\z

\noindent Use noindent after example environments (but not after floats like tables or figures).

And here's a macro for semantic type brackets: The expression \textit{dog} is of type $\stb{e,t}$. Don't forget to place the whole type formula into a math-environment. An example of a more complex type, such as the one of \textit{every}: $\stb{s,\stb{\stb{e,t},\stb{e,t}}}$. You can of course also use the type in a subscript: dog$_{\stb{e,t}}$

We distinguish between metalinguistic constants that are translations of object language, which are typeset using small caps, see \REF{sim:ex:dog}, and logical constants. See the contrast in \REF{sim:ex:speaker}, where \textsc{speaker} (= serif) in \REF{sim:ex:speaker-a} is the denotation of the word \textit{speaker}, and \cnst{speaker} (= sans-serif) in \REF{sim:ex:speaker-b} is the function that maps the context $c$ to the speaker in that context.\footnote{Notice that both types of small caps are automatically turned into text-style, even if used in a math-environment. This enables you to use math throughout.}$^,$\footnote{Notice also that the notation entails the ``direct translation'' system from natural language to metalanguage, as entertained e.g. in \citet{Heim.Kratzer1998}. Feel free to devise your own notation when relying on the ``indirect translation'' system (see, e.g., \citealt{Coppock.Champollion2022}).}

\ea\label{sim:ex:speaker}
\ea \sib{The speaker is drunk}$^{g,c}=\textsc{drunk}\big(\iota x\,\textsc{speaker}(x)\big)$\label{sim:ex:speaker-a}
\ex \sib{I am drunk}$^{g,c}=\textsc{drunk}\big(\cnst{speaker}(c)\big)$\label{sim:ex:speaker-b}
\z\z

\noindent Notice that with more complex formulas, you can use bigger brackets indicating scope, cf. $($ vs. $\big($, as used in \REF{sim:ex:speaker}. Also notice the use of backslash plus comma, which produces additional space in math-environment.

\subsection{Examples and the minsp command}

Try to keep examples simple. But if you need to pack more information into an example or include more alternatives, you can resort to various brackets or slashes. For that, you will find the minsp-command useful. It works as follows:

\ea\label{sim:ex:german-verbs}\gll Hans \minsp{\{} schläft / schlief / \minsp{*} schlafen\}.\\
Hans {} sleeps {} slept {} {} sleep.\textsc{inf}\\
\glt `Hans \{sleeps / slept\}.'
\z

\noindent If you use the command, glosses will be aligned with the corresponding object language elements correctly. Notice also that brackets etc. do not receive their own gloss. Simply use closed curly brackets as the placeholder.

The minsp-command is not needed for grammaticality judgments used for the whole sentence. For that, use the native langsci-gb4e method instead, as illustrated below:

\ea[*]{\gll Das sein ungrammatisch.\\
that be.\textsc{inf} ungrammatical\\
\glt Intended: `This is ungrammatical.'\hfill (German)\label{sim:ex:ungram}}
\z

\noindent Also notice that translations should never be ungrammatical. If the original is ungrammatical, provide the intended interpretation in idiomatic English.

If you want to indicate the language and/or the source of the example, place this on the right margin of the translation line. Schematic information about relevant linguistic properties of the examples should be placed on the line of the example, as indicated below.

\ea\label{sim:ex:bailyn}\gll \minsp{[} Ėtu knigu] čitaet Ivan \minsp{(} často).\\
{} this book.{\ACC} read.{\PRS.3\SG} Ivan.{\NOM} {} often\\\hfill O-V-S-Adv
\glt `Ivan reads this book (often).'\hfill (Russian; \citealt[4]{Bailyn2004})
\z

\noindent Finally, notice that you can use the gloss macros for typing grammatical glosses, defined in langsci-lgr.sty. Place curly brackets around them.

\subsection{Citation commands and macros}

You can make your life easier if you use the following citation commands and macros (see code):

\begin{itemize}
    \item \citealt{Bailyn2004}: no brackets
    \item \citet{Bailyn2004}: year in brackets
    \item \citep{Bailyn2004}: everything in brackets
    \item \citepossalt{Bailyn2004}: possessive
    \item \citeposst{Bailyn2004}: possessive with year in brackets
\end{itemize}

\section{Trees}\label{s:tree}

Use the forest package for trees and place trees in a figure environment. \figref{sim:fig:CP} shows a simple example.\footnote{See \citet{VandenWyngaerd2017} for a simple and useful quickstart guide for the forest package.} Notice that figure (and table) environments are so-called floating environments. {\LaTeX} determines the position of the figure/table on the page, so it can appear elsewhere than where it appears in the code. This is not a bug, it is a property. Also for this reason, do not refer to figures/tables by using phrases like ``the table below''. Always use tabref/figref. If your terminal nodes represent object language, then these should essentially correspond to glosses, not to the original. For this reason, we recommend including an explicit example which corresponds to the tree, in this particular case \REF{sim:ex:czech-for-tree}.

\ea\label{sim:ex:czech-for-tree}\gll Co se řidič snažil dělat?\\
what {\REFL} driver try.{\PTCP.\SG.\MASC} do.{\INF}\\
\glt `What did the driver try to do?'
\z

\begin{figure}[ht]
% the [ht] option means that you prefer the placement of the figure HERE (=h) and if HERE is not possible, you prefer the TOP (=t) of a page
% \centering
    \begin{forest}
    for tree={s sep=1cm, inner sep=0, l=0}
    [CP
        [DP
            [what, roof, name=what]
        ]
        [C$'$
            [C
                [\textsc{refl}]
            ]
            [TP
                [DP
                    [driver, roof]
                ]
                [T$'$
                    [T [{[past]}]]
                    [VP
                        [V
                            [tried]
                        ]
                        [VP, s sep=2.2cm
                            [V
                                [do.\textsc{inf}]
                            ]
                            [t\textsubscript{what}, name=trace-what]
                        ]
                    ]
                ]
            ]
        ]
    ]
    \draw[->,overlay] (trace-what) to[out=south west, in=south, looseness=1.1] (what);
    % the overlay option avoids making the bounding box of the tree too large
    % the looseness option defines the looseness of the arrow (default = 1)
    \end{forest}
    \vspace{3ex} % extra vspace is added here because the arrow goes too deep to the caption; avoid such manual tweaking as much as possible; here it's necessary
    \caption{Proposed syntactic representation of \REF{sim:ex:czech-for-tree}}
    \label{sim:fig:CP}
\end{figure}

Do not use noindent after figures or tables (as you do after examples). Cases like these (where the noindent ends up missing) will be handled by the editors prior to publication.

\section{Italics, boldface, small caps, underlining, quotes}

See \citet{Nordhoff.Muller2021} for that. In short:

\begin{itemize}
    \item No boldface anywhere.
    \item No underlining anywhere (unless for very specific and well-defined technical notation; consult with editors).
    \item Small caps used for (i) introducing terms that are important for the paper (small-cap the term just ones, at a place where it is characterized/defined); (ii) metalinguistic translations of object-language expressions in semantic formulas (see \sectref{sim:sec:sem}); (iii) selected technical notions.
    \item Italics for object-language within text; exceptionally for emphasis/contrast.
    \item Single quotes: for translations/interpretations
    \item Double quotes: scare quotes; quotations of chunks of text.
\end{itemize}

\section{Cross-referencing}

Label examples, sections, tables, figures, possibly footnotes (by using the label macro). The name of the label is up to you, but it is good practice to follow this template: article-code:reference-type:unique-label. E.g. sim:ex:german would be a proper name for a reference within this paper (sim = short for the author(s); ex = example reference; german = unique name of that example).

\section{Syntactic notation}

Syntactic categories (N, D, V, etc.) are written with initial capital letters. This also holds for categories named with multiple letters, e.g. Foc, Top, Num, etc. Stick to this convention also when coming up with ad hoc categories, e.g. Cl (for clitic or classifier).

An exception from this rule are ``little'' categories, which are written with italics: \textit{v}, \textit{n}, \textit{v}P, etc.

Bar-levels must be typeset with bars/primes, not with an apostrophe. An easy way to do that is to use mathmode for the bar: C$'$, Foc$'$, etc.

Specifiers should be written this way: SpecCP, Spec\textit{v}P.

Features should be surrounded by square brackets, e.g., [past]. If you use plus and minus, be sure that these actually are plus and minus, and not e.g. a hyphen. Mathmode can help with that: [$+$sg], [$-$sg], [$\pm$sg]. See \sectref{sim:sec:hyphens-etc} for related information.

\section{Footnotes}

Absolutely avoid long footnotes. A footnote should not be longer than, say, {20\%} of the page. If you feel like you need a long footnote, make an explicit digression in the main body of the text.

Footnotes should always be placed at the end of whole sentences. Formulate the footnote in such a way that this is possible. Footnotes should always go after punctuation marks, never before. Do not place footnotes after individual words. Do not place footnotes in examples, tables, etc. If you have an urge to do that, place the footnote to the text that explains the example, table, etc.

Footnotes should always be formulated as full, self-standing sentences.

\section{Tables}

For your tables use the table plus tabularx environments. The tabularx environment lets you (and requires you in fact) to specify the width of the table and defines the X column (left-alignment) and the Y column (right-alignment). All X/Y columns will have the same width and together they will fill out the width of the rest of the table -- counting out all non-X/Y columns.

Always include a meaningful caption. The caption is designed to appear on top of the table, no matter where you place it in the code. Do not try to tweak with this. Tables are delimited with lsptoprule at the top and lspbottomrule at the bottom. The header is delimited from the rest with midrule. Vertical lines in tables are banned. An example is provided in \tabref{sim:tab:frequencies}. See \citet{Nordhoff.Muller2021} for more information. If you are typesetting a very complex table or your table is too large to fit the page, do not hesitate to ask the editors for help.

\begin{table}
\caption{Frequencies of word classes}
\label{sim:tab:frequencies}
 \begin{tabularx}{.77\textwidth}{lYYYY} %.77 indicates that the table will take up 77% of the textwidth
  \lsptoprule
            & nouns & verbs  & adjectives & adverbs\\
  \midrule
  absolute  &   12  &    34  &    23      & 13\\
  relative  &   3.1 &   8.9  &    5.7     & 3.2\\
  \lspbottomrule
 \end{tabularx}
\end{table}

\section{Figures}

Figures must have a good quality. If you use pictorial figures, consult the editors early on to see if the quality and format of your figure is sufficient. If you use simple barplots, you can use the barplot environment (defined in langsci-osl.sty). See \figref{sim:fig:barplot} for an example. The barplot environment has 5 arguments: 1. x-axis desription, 2. y-axis description, 3. width (relative to textwidth), 4. x-tick descriptions, 5. x-ticks plus y-values.

\begin{figure}
    \centering
    \barplot{Type of meal}{Times selected}{0.6}{Bread,Soup,Pizza}%
    {
    (Bread,61)
    (Soup,12)
    (Pizza,8)
    }
    \caption{A barplot example}
    \label{sim:fig:barplot}
\end{figure}

The barplot environment builds on the tikzpicture plus axis environments of the pgfplots package. It can be customized in various ways. \figref{sim:fig:complex-barplot} shows a more complex example.

\begin{figure}
  \begin{tikzpicture}
    \begin{axis}[
	xlabel={Level of \textsc{uniq/max}},  
	ylabel={Proportion of $\textsf{subj}\prec\textsf{pred}$}, 
	axis lines*=left, 
        width  = .6\textwidth,
	height = 5cm,
    	nodes near coords, 
    % 	nodes near coords style={text=black},
    	every node near coord/.append style={font=\tiny},
        nodes near coords align={vertical},
	ymin=0,
	ymax=1,
	ytick distance=.2,
	xtick=data,
	ylabel near ticks,
	x tick label style={font=\sffamily},
	ybar=5pt,
	legend pos=outer north east,
	enlarge x limits=0.3,
	symbolic x coords={+u/m, \textminus u/m},
	]
	\addplot[fill=red!30,draw=none] coordinates {
	    (+u/m,0.91)
        (\textminus u/m,0.84)
	};
	\addplot[fill=red,draw=none] coordinates {
	    (+u/m,0.80)
        (\textminus u/m,0.87)
	};
	\legend{\textsf{sg}, \textsf{pl}}
    \end{axis} 
  \end{tikzpicture} 
    \caption{Results divided by \textsc{number}}
    \label{sim:fig:complex-barplot}
\end{figure}

\section{Hyphens, dashes, minuses, math/logical operators}\label{sim:sec:hyphens-etc}

Be careful to distinguish between hyphens (-), dashes (--), and the minus sign ($-$). For in-text appositions, use only en-dashes -- as done here -- with spaces around. Do not use em-dashes (---). Using mathmode is a reliable way of getting the minus sign.

All equations (and typically also semantic formulas, see \sectref{sim:sec:sem}) should be typeset using mathmode. Notice that mathmode not only gets the math signs ``right'', but also has a dedicated spacing. For that reason, never write things like p$<$0.05, p $<$ 0.05, or p$<0.05$, but rather $p<0.05$. In case you need a two-place math or logical operator (like $\wedge$) but for some reason do not have one of the arguments represented overtly, you can use a ``dummy'' argument (curly brackets) to simulate the presence of the other one. Notice the difference between $\wedge p$ and ${}\wedge p$.

In case you need to use normal text within mathmode, use the text command. Here is an example: $\text{frequency}=.8$. This way, you get the math spacing right.

\section{Abbreviations}

The final abbreviations section should include all glosses. It should not include other ad hoc abbreviations (those should be defined upon first use) and also not standard abbreviations like NP, VP, etc.


\section{Bibliography}

Place your bibliography into localbibliography.bib. Important: Only place there the entries which you actually cite! You can make use of our OSL bibliography, which we keep clean and tidy and update it after the publication of each new volume. Contact the editors of your volume if you do not have the bib file yet. If you find the entry you need, just copy-paste it in your localbibliography.bib. The bibliography also shows many good examples of what a good bibliographic entry should look like.

See \citet{Nordhoff.Muller2021} for general information on bibliography. Some important things to keep in mind:

\begin{itemize}
    \item Journals should be cited as they are officially called (notice the difference between and, \&, capitalization, etc.).
    \item Journal publications should always include the volume number, the issue number (field ``number''), and DOI or stable URL (see below on that).
    \item Papers in collections or proceedings must include the editors of the volume (field ``editor''), the place of publication (field ``address'') and publisher.
    \item The proceedings number is part of the title of the proceedings. Do not place it into the ``volume'' field. The ``volume'' field with book/proceedings publications is reserved for the volume of that single book (e.g. NELS 40 proceedings might have vol. 1 and vol. 2).
    \item Avoid citing manuscripts as much as possible. If you need to cite them, try to provide a stable URL.
    \item Avoid citing presentations or talks. If you absolutely must cite them, be careful not to refer the reader to them by using ``see...''. The reader can't see them.
    \item If you cite a manuscript, presentation, or some other hard-to-define source, use the either the ``misc'' or ``unpublished'' entry type. The former is appropriate if the text cited corresponds to a book (the title will be printed in italics); the latter is appropriate if the text cited corresponds to an article or presentation (the title will be printed normally). Within both entries, use the ``howpublished'' field for any relevant information (such as ``Manuscript, University of \dots''). And the ``url'' field for the URL.
\end{itemize}

We require the authors to provide DOIs or URLs wherever possible, though not without limitations. The following rules apply:

\begin{itemize}
    \item If the publication has a DOI, use that. Use the ``doi'' field and write just the DOI, not the whole URL.
    \item If the publication has no DOI, but it has a stable URL (as e.g. JSTOR, but possibly also lingbuzz), use that. Place it in the ``url'' field, using the full address (https: etc.).
    \item Never use DOI and URL at the same time.
    \item If the official publication has no official DOI or stable URL (related to the official publication), do not replace these with other links. Do not refer to published works with lingbuzz links, for instance, as these typically lead to the unpublished (preprint) version. (There are exceptions where lingbuzz or semanticsarchive are the official publication venue, in which case these can of course be used.) Never use URLs leading to personal websites.
    \item If a paper has no DOI/URL, but the book does, do not use the book URL. Just use nothing.
\end{itemize}
\section{Introduction}\label{sec:1}
Focus-sensitive particles (FSPs), particles such as English \textit{only}, \textit{even}, and \textit{also}, have received attention to varying degrees depending on the language studied. While association with focus in English is by now a well-studied phenomenon, especially from a semantic perspective (see \citealt{spe:Rooth1985}, for example), and the syntactic properties of FSPs in Germanic languages such as English and German have been extensively researched \citep{Jacobs1983,BuringHartmann2001,vonStechow2008,Mursell2021}, the same cannot be said of many other language families. In Slavic, for example, information structure-sensitive particles, including FSPs, remain understudied in comparison to other phenomena in information structure (IS), such as the interaction of IS and free word order \citep[731]{Jasinskaja2016}. Additionally, there is generally a strong bias towards Russian data, with other Slavic languages being either less studied or even understudied in comparison to that, as \citet{Jasinskaja2016} notes. \par 
In the following, I provide a first analysis of Bulgarian focus-sensitive particles with an emphasis on \textit{samo} `only’, \textit{s\v{a}\v{s}to} `also’, and \textit{dori} `even’. \REF{Bulgarian:basic1} shows how these three particles associate with a f(ocus)-marked constituent in Bulgarian.\footnote{All non-English examples in this paper are from Bulgarian, unless marked otherwise next to the example.}\textsuperscript{,}\footnote{In the basic cases shown in this paper, the three particles generally behave the same way with respect to their syntactic behavior. However, once they are investigated in further detail, their behavior (unsurprisingly) diverges. I cannot provide a detailed investigation of this in this paper as the purpose is to provide a first analysis of Bulgarian FSPs, and this will be addressed in future research. Throughout the paper, I indicate relevant differences between the particles with respect to their placement when needed.}


\ea\label{Bulgarian:basic1}
\ea
\gll Obadi-h se samo \minsp{[} na IVAN]$_{F}$. \\
call-\textsc{pst.1sg} \textsc{refl} only {} to Ivan \\
\glt ‘I only called Ivan.’ \hfill (\citealt[ex. 8a]{TishevaDzhonova2003})
\ex
\gll V\v{c}era s\v{a}\v{s}to \minsp{[} \v{c}eto-h ROMAN]$_{F}$. \\
yesterday also {} read-\textsc{pst.1sg} novel \\
\glt ‘I also read a novel yesterday.’
\ex
\gll V\v{c}era dori \minsp{[} AZ]$_{F}$ pro\v{c}eto-h edin roman. \\
yesterday even {} I read-\textsc{pst.1sg} one novel \\
\glt ‘Even I read a novel yesterday.’
\z
\z

\noindent In addition to pre-focal association with focus, Bulgarian FSPs are also able to follow the focused constituent they are associated with, as \REF{Bulgarian:basic2} demonstrates. Post-focal association is attested for other Slavic languages such as Russian as well and distinguishes Slavic in this respect from German and other Germanic languages.\footnote{While the differences in interpretation between the two word orders shown here deserve an investigation of their own, I provide some preliminary results concerning these differences in \sectref{sec:4} of this paper.} 

\ea\label{Bulgarian:basic2}
\ea
\gll Obadi-h se \minsp{[} na IVAN]$_{F}$ samo. \\
call-\textsc{pst.1sg} \textsc{refl} {} to Ivan only\\
\glt ‘I only called Ivan.’ \hfill (\citealt[ex. 8a]{TishevaDzhonova2003})
\ex
\gll \minsp{[} PETĂR]$_{F}$ s\v{a}\v{s}to mo\v{z}-e da gotvi. \\
{} Petǎr also able-\textsc{prs.3sg} to cook\\
\glt ‘Petǎr is able to cook as well.’
\ex
\gll \minsp{[} ANA]$_{F}$ dori \v{s}te trjabva da dojd-e. \\
{} Ana even \textsc{aux.fut} have to come-\textsc{prs.3sg} \\
\glt ‘Even Ana will have to come.’
\z
\z

\noindent So far, only few studies have focused on the syntax and semantics of Bulgarian FSPs. For instance, a semantic study of \textit{samo} `only’ is given in \citet{Nicolova2000}. The semantic properties of the FSP \textit{a\.z}/\textit{\v cak}, the scalar opposite of scalar `only' which is present in several Slavic languages, including Bulgarian, is extensively studied in \citet{Tomaszewicz2013}. The syntactic distribution of \textit{samo} `only’ is described in \citet{TishevaDzhonova2003}. Their corpus study provides a detailed description of the adjunction sites of \textit{samo}. However, the study only considers surface word order and does not contain an analysis that goes beyond surface level. In the present study, I close a research gap in this respect and argue that a so-called \textsc{adverbial-only analysis} correctly predicts the possible adjunction sites of Bulgarian FSPs along the lines of what is argued for German in \citet{BuringHartmann2001} and for Russian in \citet{Zanon2018}. \par 
Adverbial-only analyses of FSPs predict that FSPs are only able to adjoin to projections belonging to the Extended Verbal Projection (EVP), although individual analyses of this kind may differ with respect to the projections which they allow adjunction to. The logical alternative to this type of analysis is the so-called \textsc{adnominal} or \textsc{mixed} analysis, which predicts that FSPs can adjoin to any type of phrase, and especially also to argument DPs. For German, it has been argued extensively in \citet{BuringHartmann2001} that an adverbial-only analysis successfully captures the syntactic properties of the language's FSPs (see \citealt{Mursell2021} for an extension and discussion of this proposal, and \citealt{Sudhoff2010} for criticism of this line of analysis). 

In addition to arguing for an adverbial-only analysis of Bulgarian FSPs, I show that a combination of \citeposst{BuringHartmann2001} Particle Theory on the one hand, and \citeposst{Zanon2018} Particle Theory on the other hand can be adapted to account for the placement options of FSPs in Bulgarian. Since Bulgarian is considered to be a language with relatively ``free'’ word order, the successful extension of Büring and Hartmann’s adverbial-only analysis is relevant insofar as it shows that languages with a more flexible word order than German can also impose heavy restrictions on the distribution of FSPs. This indicates that `free’ word order does not have to mean that FSP-adjunction is necessarily free as well. \par 
This article is structured as follows. In \sectref{sec:2}, I briefly present the few studies that have already been conducted on Bulgarian FSPs, give an overview of the placement options of the three particles studied here on the clausal level and in the nominal domain, and summarize the most important aspects of \citeposst{BuringHartmann2001} and \citeposst{Zanon2018} Particle Theories. In \sectref{sec:3}, I present syntactic arguments for an adverbial-only analysis of FSP adjunction in Bulgarian and discuss challenges to conventional adverbial-only analyses caused by the particle placement options available in the nominal domain, before moving on to presenting a Particle Theory for Bulgarian. Since Bulgarian FSPs can also appear post-focally, in contrast to German FSPs, I discuss post-focal FSPs in \sectref{sec:4}. \sectref{sec:5} concludes the paper.

\section{Focus-sensitive particles in Bulgarian and beyond}\label{sec:2}
The aim of this section is twofold. First, I offer a basic description of the placement options of Bulgarian FSPs based on previous work on Bulgarian as well as novel data. This is followed by an overview of the basics of \citeposst{BuringHartmann2001} Particle Theory as well as an analysis of the syntactic behavior of Russian \textit{tol'ko} `only' in \citet{Zanon2018}, both of which I am going to extend to Bulgarian in \sectref{sec:3}. 
\subsection{Previous research on (Bulgarian) FSPs}\label{sec:2.1} 
\subsubsection{Syntactic aspects}\label{sec:2.1.1}
In this section, I provide a basic description of the placement of FSPs in Bulgarian based on \citet{TishevaDzhonova2003}, a descriptive corpus study of Bulgarian \textit{samo} `only', and also provide novel data. 
%\paragraph*{FSPs in Bulgarian: basic facts} 
\textit{Only} and its approximate equivalents are the particles that have received most attention in the literature on FSPs. This is also the case for the corpus study by \citet{TishevaDzhonova2003} already mentioned. The authors argue that \textit{samo} `only' in Bulgarian ``can have scope over NP, PP, AdvP, VP, or part of XP” \citep[65]{TishevaDzhonova2003}.\footnote{As pointed out by a reviewer, this usage of the term ``scope'' is potentially misleading as what the authors describe in their paper is actually possible adjunction sites of \textit{samo} `only' in Bulgarian. I therefore follow the reviewer's suggestion and avoid this usage of the term ``scope''. While I cannot discuss these matters further here due to reasons of space, more in-depth discussion of different usages of the notion of scope can be found in \citet{BrananErlewine2023}.} Their data shows how flexible \textit{samo} seems to be when it comes to the potential adjunction sites of the particle. \citet[109]{Nicolova2000} also acknowledges the flexibility of the placement of FSPs in Bulgarian, remarking that NPs, PPs, VP, verbs, AdvPs, or whole subordinate clauses can associate with an FSP. \REF{Bulgarian:basic3} demonstrates that the FSP can adjoin to NPs and Vs (in addition to the apparent PP adjunction demonstrated in \REF{Bulgarian:basic1}). 

\ea\label{Bulgarian:basic3} 
\ea
\gll Kupi-h samo \minsp{[} KRASTAVIC-I]$_{F}$ za salata-ta. \\
buy-\textsc{pst.1sg} only {} cucumber-\textsc{pl} for salad-\textsc{def}\\
\glt ‘I only bought cucumbers for the salad.’
\ex
\gll Samo \minsp{[} \v{C}ET-A]$_{F}$ roman-i. \\
only {} read-\textsc{prs.1sg} novel-\textsc{pl}\\
\glt ‘I only read novels.’\hfill \citep[exx. 7a, 6a]{TishevaDzhonova2003}
\z
\z

\noindent These examples also show that \textit{samo} generally marks narrow focus \citep{TishevaDzhonova2003}. The authors also note that \textit{samo} can associate with the phrase preceding it when it is placed at the end of the clause, as in \REF{Bulgarian:basic2} above. \REF{Bulgarian:basic2} also shows that the FSPs under discussion here can also follow a subject that they associate with, an option that is not discussed by \citet{TishevaDzhonova2003}. In other positions, \textit{samo} is much more likely to associate with the phrase following it \citep[6--7]{TishevaDzhonova2003}. In general, there are only two cases in which the focused constituent is not right-adjacent to the FSP associated with it. Apart from the case of (apparent) right-adjunction already shown, it is also possible for the focused constituent to move to a position at the left edge of the clause, stranding the FSP that is associated with it. This is shown in \REF{movement1}.  \par

\ea\label{movement1}
\ea
\gll \v{C}et-a samo \minsp{[} ROMAN-I]$_{F}$. \\
read-\textsc{1sg} only {} novel-\textsc{pl} \\
\glt `I read only [novels]$_{F}$.’
\ex
\gll \minsp{[} ROMAN-I]$_{F}$ \v{c}et-a samo \\
{} novel-\textsc{pl} read-\textsc{1sg} only \\
\glt `[Novels]$_{F}$, I read only.’
\z
\z 

\noindent \citet{TishevaDzhonova2003} note that there are two restrictions that the placement of \textit{samo} must adhere to: the prohibition against insertion into PPs and the prohibition against insertion into complex verbal complexes. Examples of both can be seen in \REF{Bulgarian:basic4}. 

\ea\label{Bulgarian:basic4}
\ea[*]
{\gll Obadi-h se na samo \minsp{[} IVAN]$_{F}$.\\
call-\textsc{pst.1sg} \textsc{refl} to only {} Ivan\\
\glt Intended: ‘I called only Ivan.’}
\ex[*]{
\gll Ti \v{s}te samo \minsp{[} SEDI-\v{S}]$_{F}$. \\
you \textsc{aux.fut} only {} sit-\textsc{prs.2sg}\\
\glt Intended: ‘You will just sit.’\hfill \citep[exx. 8d, 11b]{TishevaDzhonova2003}}
\z
\z

\noindent Apart from these prohibitions, the authors argue that \textit{samo} can be placed relatively freely within the Bulgarian clause. However, a crucial restriction of the scope of their investigation is that their study is mostly descriptive and only takes surface word order into consideration. Once the aim is to identify why it should be the case that the restrictions in place in Bulgarian exist and what they reveal about the underlying adjunction sites of Bulgarian FSPs, it becomes evident that the potential adjunction sites for FSPs in Bulgarian are much more restricted than can be seen at the level of surface word order. \par 

In addition to the data discussed by \citet{TishevaDzhonova2003}, several more potential adjunction sites of Bulgarian FSPs can be discovered in the nominal domain. \REF{nominal1} demonstrates that FSPs such as \textit{samo} can adjoin to PPs within NPs:

\ea\label{nominal1}
\ea
\gll samo kotka-ta \minsp{[} na S\v{A}SEDKA-TA]$_{F}$ \\
 only cat-\textsc{def} {} of neighbor-\textsc{def} \\
\glt `only the cat [of the neighbor]$_{F}$’
\ex
\gll kotka-ta samo \minsp{[} na S\v{A}SEDKA-TA]$_{F}$ \\
cat-\textsc{def} only {} of neighbor-\textsc{def} \\
\glt `only the cat [of the neighbor]$_{F}$’
\ex
\gll kotka-ta \minsp{[} na S\v{A}SEDKA-TA]$_{F}$ samo \\
cat-\textsc{def} {} of neighbor-\textsc{def} only \\
\glt `only the cat [of the neighbor]$_{F}$’
\z
\z

\noindent Adjunction to nominal modifiers within NPs sometimes even circumvents the ``no PP-insertion''-prohibition discussed by \citet{TishevaDzhonova2003}, irrespective of the FSP that is being inserted. However, the acceptability of these examples depends on the preposition and modifier involved, as \REF{nominal2} shows (further discussion can be found in \sectref{sec:3.2} of this paper). 

\ea\label{nominal2}
\ea[*]
{\gll săs samo \minsp{[} EDNA]$_{F}$ kola \\
with only {} one car \\
\glt Intended: `with only one car’}
\ex[\textsuperscript{?}]
{\gll sled samo \minsp{[} NJAKOLKO]$_{F}$ sekund-i \\
within only {} few second-\textsc{pl} \\
\glt `within only a few seconds’}
\ex[]
{\gll me\v{z}du samo \minsp{[} DVE]$_{F}$ opci-i \\
between only {} two option-\textsc{pl} \\
\glt `between only two options’}
\ex[]
{\gll me\v{z}du dori \minsp{[} DVE]$_{F}$ opci-i \\
between even {} two option-\textsc{pl} \\
\glt `between even two options’}
\z
\z 

\noindent After briefly discussing semantic research on Bulgarian FSPs, I turn to theoretical approaches that can be adapted to analyze particle placement in Bulgarian in the remainder of the section. (both are more about syntax though)

\subsubsection{Semantic aspects}\label{sec:2.1.2}
With respect to the semantics of Bulgarian FSPs, three particles have been studied in the literature in more depth, namely \textit{samo} `only', \textit{dori} `even', and \textit{\v cak}.\footnote{No translation of \textit{\v cak} is provided here due to its intriguing semantic properties that impede a direct translation into English.} \par 
According to \citet{Nicolova2000}, both the exclusive particle \textit{samo} and the additive particle \textit{dori} mark contrastive focus in Bulgarian.\footnote{\citet[108]{Nicolova2000} argues that additive as well as exclusive FSPs induce contrastive focus as both particles express a difference between a predicted and a real sum, a position that can be (and has been) debated.}\footnote{\citet{Nicolova2000} labels \textit{even}-type as well as \textit{also}-type particles as additive particles, a (terminological) decision that does not seem intuitive to readers nowadays, as a reviewer notes. In \citeauthor{Nicolova2000}'s system, both particles are additive, but \textit{even} is scalar while \textit{also} is non-scalar (which also holds for \textit{only} in her classification). I follow \citeauthor{Nicolova2000}'s terminology here for the sake of correctly presenting her proposal.} The two particles can associate with different types of phrases such as NPs, PPs, VPs, or AdvPs \citep[109]{Nicolova2000}. In addition to that, \citet{Nicolova2000} notes that the contrastively focused constituent can be placed everywhere in the clause and is not restricted to a designated position while the most prominent sentence accent is placed in the domain of the FSP and its adjacent focused constituent. \par 
\citet{Tomaszewicz2013} provides a semantic study of the Slavic FSP \textit{a\.z}/\textit{\v cak}.\footnote{\textit{A\.z} is found in Czech, Polish, Slovak, and Russian (with different spellings), and \textit{\v cak} is found in Bulgarian and other South Slavic languages. Both particles have similar properties and can be treated as two forms of the same particle.} According to the author, \textit{a\.z}/\textit{\v cak} makes three basic contributions, namely the assertion that lower alternatives than the one presented in the clause are excluded as well as the presuppositions that ``the prejacent is high on the scale'' and that ``the prejacent or an alternative at most as strong is true'' \citep[321]{Tomaszewicz2013}. A Bulgarian example is shown in \REF{Tomaszewicz:basic1} \citep[from][302]{Tomaszewicz2013}.\footnote{Transliteration changed to scientific transliteration.}

\ea\label{Tomaszewicz:basic1} 
\gll Govori-h \v{c}ak s \minsp{[} MARY]$_{F}$. \\
talk-\textsc{pst.1sg} \textsc{\v{c}ak} with {} Mary \\
\glt `I talked to somebody so important as [Mary]$_{F}$.'
\z

\noindent In \REF{Tomaszewicz:basic1}, \textit{\v cak} makes a contribution similar to English \textit{even} in that it singles out Mary as a very important person to talk to. However, the particle is not merely presuppositional and can also be the direct opposite of \textit{only}, in contrast to \textit{even} \citep{Tomaszewicz2013}. In contrast to \textit{a\.z}/\textit{\v cak}, \textit{only} would assert that there is no higher, true alternative and presuppose that ``the prejacent is low on the scale'' and that ``the prejacent or an alternative at least as strong is true'' \citep[321]{Tomaszewicz2013}.  
Particles such as \textit{\v cak} are heavily restricted in their usage due to their particular semantic properties. While \textit{only}, \textit{also}, and \textit{even} can be used interchangeably in most examples discussed here, this is not the case for \textit{\v cak}. In the remainder of this paper, I am going to focus on the less semantically restricted FSPs in Bulgarian. In the next section, I turn to \citeposst{BuringHartmann2001} theory of FSPs in German and \citeposst{Zanon2018} account of the syntax of \textit{tol'ko} `only' in Russian, which will provide the basis for the proposed analysis in \sectref{sec:3}. 

\subsection{The Particle Theories of \citet{BuringHartmann2001} and \citet{Zanon2018}}\label{sec:2.2}

\subsubsection{\citeposst{BuringHartmann2001} Particle Theory for German}\label{sec:2.2.1} \citet{BuringHartmann2001} propose an adverbial-only analysis of German focus-sensitive particles. One of their many arguments is that this kind of analysis naturally excludes the adjunction of FSPs to DPs within PPs or embedded within other DPs, which is ungrammatical in German. \REF{BuringHartmann:basic1} shows both ungrammatical cases. 

\ea\label{BuringHartmann:basic1} 
\ea[*]
{\gll mit nur \minsp{[} HANS]$_{F}$ \\
with only {} Hans\\
\glt Intended: `only with Hans’}
\ex[*]{
\gll der Bruder nur \minsp{[} de-s GRAF-EN]$_{F}$ \\
the brother only {} the-\textsc{gen} count-\textsc{gen} \\
\glt Intended: `only the count's brother’}\hfill (German; \citealt{BuringHartmann2001}: exx. 7a, 8a)
\z
\z 

\noindent As I demonstrate in the next section, the adjunction of FSPs to NPs/DPs within PPs is also ungrammatical in Bulgarian in most cases. \par 
The specific adverbial-only analysis that \citet[ex. 6]{BuringHartmann2001} propose allows adjunction of FSPs only to projections belonging to the Extended Verbal Projection (EVP). Their Particle Theory (in its preliminary version) consists of four clauses plus an additional clause concerning left-adjunction of FSPs in German and is shown in \REF{BuringHartmann:basic2}--\REF{BuringHartmann:basic3}. 

\newpage
\ea\label{BuringHartmann:basic2}
For any node $\alpha$ marked F in a phrase marker P, let the set of f-nodes of $\alpha$ consist of all nodes $\beta$ in P such that 
\ea{$\beta$ is an EP (extended projection) of some V $\gamma$}\label{BuringHartmann:basic2a}
\ex{$\beta$ is a maximal projection}\label{BuringHartmann:basic2b}
\ex{$\beta$ dominates $\alpha$ or is identical to $\alpha$}\label{BuringHartmann:basic2c}
\ex{there is no EP $\beta'$ of $\gamma$ such that $\beta$ dominates $\beta'$ and $\beta'$ meets (10b) and (10c). \hfill \citep[ex. 11]{BuringHartmann2001}}\label{BuringHartmann:basic2d}
\z
\z 

\ea\label{BuringHartmann:basic3}
A FSP must be left-adjoined to an f-node of its focus. \\\hfill \citep[ex. 12]{BuringHartmann2001}
\z

\noindent Apart from adjunction to EVP, the Particle Theory predicts that FSPs only adjoin to maximal projections (\ref{BuringHartmann:basic2b}), that the FSP has to dominate the F-marked constituent \REF{BuringHartmann:basic2c} (``dominate'' means ``c-command'' for the purpose of the discussion here), that FSPs adjoin to the focus as closely as possible (\ref{BuringHartmann:basic2d}), and that FSPs can only be left-adjoined in German.\footnote{See \citet{BuringHartmann2001} for a detailed discussion of and argumentation for the individual clauses, and \citet{Sudhoff2010} as well as \citet{Mursell2021} for discussion and criticism of individual aspects of their proposal.} In \sectref{sec:3.3}, I discuss how this Particle Theory could be adapted to Bulgarian, after arguing that an adverbial-only analysis should, in fact, be pursued for this language. \par 
\citeposst{BuringHartmann2001} proposal has been met with criticism in the literature, much of which has implications for a Particle Theory for German, but not necessarily for the Particle Theory for Bulgarian developed here. \citet{Reis2005} remarks that the PT for German sometimes predicts V3 structures that should be ungrammatical according to the strict V2 requirement in German. This is mostly irrelevant for Bulgarian, but see \citet{Mursell2021} for a defense of this aspect of \citet{BuringHartmann2001}. The reconstruction-based arguments employed by \citet[Section 5]{BuringHartmann2001} have also been an object of debate, for example in \citet{MeyerSauerland2009} and \citet{SmeetsWagner2018}. I will leave this aspect of the debate aside for now as I will not employ reconstruction-based arguments for developing my Particle Theory for Bulgarian (but see, again, \citealt{Mursell2021} for an extensive and recent discussion of the reconstruction facts in German). Additionally, \citet{Reis2005} discusses the adjacency requirement already mentioned (which is termed the ``closeness condition'' in \citealt{Reis2005}). The following example is given by her to show that the adjacency requirement can be violated in German:

\ea\label{Reis1} 
\ea
\gll Ich hab nur \minsp{\{} darin / in dem Buch\} \minsp{[} geLESen]]$_{F}$. \\
I have only {} therein {} in the book {} read \\
\glt `I have only read it/the book.’\label{Reis1a}
\ex
\gll Ich hab \minsp{\{} darin / in dem Buch\} nur \minsp{[} geLESen]$_{F}$.\\
I have {} therein {} in the book only {} read \\
\glt `I have only read it/the book.’\label{Reis1b}\hfill (German; \citealt{Reis2005}: ex. 23a) 
\z 
\z 

\noindent Regarding this example, I agree with \citet[230]{Mursell2021} with respect to the questionable grammaticality of \REF{Reis1a}. \REF{Reis1a} is labelled as grammatical in \citet{Reis2005}, a fact with which many German speakers do not agree. More important, however, is that \REF{Reis1a} does not violate the adjacency requirement, as \textit{nur} `only' still adjoins to the projection of the EVP most immediately dominating the F-marked constituent. The only difference between \REF{Reis1a} and \REF{Reis1b} is that there is no scrambling of the argument out of the VP in \REF{Reis1a}, an option which is permitted under \citeposst{BuringHartmann2001} theory \citep[230]{Mursell2021}. Regarding the closeness condition/adjacency requirement, it is interesting to note that evidence for the validity of such a condition can also be found in languages unrelated to the ones discussed in this paper. \citet{Erlewine2017} discusses focus association with \textit{chỉ} `only' in Vietnamese and argues that \textit{chỉ} needs to adjoin as early as possible in each phase during the derivation, relating the closeness condition to cyclic structure-building facts. This is in line with findings concerning the exhaustive focus marker \textit{shì} in Mandarin \citep{Erlewine2022}. These findings are important in the context of \citeposst{BuringHartmann2001} theory as they show that the closeness condition can be motivated independently of the facts found for German, and, crucially, independently of the arguments brought forward by \citeauthor{BuringHartmann2001} that have been criticized so markedly in the literature. \par 

Later on in their paper, \citet[265-266]{BuringHartmann2001} modify their proposal further and argue that FSPs only adjoin to non-arguments (this condition replaces the EVP condition mentioned above). Among tricky CP adjunction data discussed by the authors, this proposal also accounts for cases of adjunction within DPs that would be excluded by the EVP condition, such as the cases in \REF{BuringHartmann:basic4}. 

\ea\label{BuringHartmann:basic4} 
\ea
\gll eine nur an \minsp{[} MUSIK]$_{F}$ interessierte Student-in \\
a only in {} music interested student-\textsc{f}\\
\glt `a student interested only in music’
\ex
\gll der sogar mit \minsp{[} KARL]$_{F}$ verfeindete Förster \\
the even with {} Karl quarreling forest\_ranger \\
\glt `the forest ranger who is quarreling even with Karl’
\ex
\gll unser auch von \minsp{[} Origami]$_{F}$ begeisterter Hausmeister \\
our also of {} Origami enthusiastic janitor \\
\glt `our janitor who is enthusiastic also about Origami’
\z
\hfill (German; \citealt{BuringHartmann2001}: ex. 74) 
\z 

\noindent In these cases, the modified Particle Theory predicts the adjunction of the FSP to the modifier instead of DP, which would then be adjunction to a non-argument. This aspect of \citeauthor{BuringHartmann2001}'s proposal has been met with criticism as well; however, a detailed discussion of this would go beyond the scope of this paper and can be found in \citet{Mursell2021}. Additionally, \citet[247-248]{Mursell2021} discusses DP data such as \REF{Mursell:basic1}, which is not explained by the ``adjunction to non-arguments''-condition either.\footnote{While I judge \REF{Mursell:basic1b} as degraded, it improves for me as an answer to a question such as \textit{What kind of bag would you like to have?}} 

\ea\label{Mursell:basic1}  
\ea[]
{\gll ein nur \minsp{[} MIttelmäßiger]$_{F}$ Student \\
an only {} mediocre student\\
\glt `an only mediocre student’}\label{Mursell:basic1a}
\ex[*]{
\gll eine nur \minsp{[} ROte]$_{F}$ Tasche \\
an only {} red bag \\
\glt Intended: `an only red bag’\hfill (German; \citealt{Mursell2021}: 247, ex. 78)}\label{Mursell:basic1b}
\z
\z 

\noindent As discussed by \citet{Mursell2021}, the fact that individual modifiers provide different adjunction options for FSPs or even prohibit adjunction points towards the fact that there could be additional semantic reasons that permit or prohibit FSP adjunction. Please note as well that \citeposst{BuringHartmann2001} proposal excludes data such as the cases in \REF{nominal1} found in Bulgarian, which demonstrate that the adjunction of FSPs to PPs within NPs/DPs is possible in Bulgarian. A possible solution to this is offered by the next proposal, \citet{Zanon2018}, to be discussed in the following section.  

\subsubsection{\citeposst{Zanon2018} analyis of Russian \textit{tol'ko}}\label{sec:2.2.2}
\citet{Zanon2018} examines the behavior of Russian \textit{tol'ko} `only', arguing that \textit{tol'ko} is always adjacent to the F-marked constituent due to a strong [Foc] feature of \textit{only} that triggers movement of the F-marked constituent to a position adjacent to it. Along the lines of \citeposst{spe:Rudin1988} proposal for Bulgarian multiple wh-questions, \citet[420]{Zanon2018} argues that \textit{tol'ko} and the F-marked constituent form an unsplittable complex, as shown in \figref{fig:Zanon:tree1}.\footnote{As a reviewer notes, \citeauthor{Zanon2018}'s adaptation of \citeauthor{spe:Rudin1988}'s analysis presupposes that the phrase that is the sister to \textit{only} can move into a non-c-commanding position, along the lines of \citeauthor{spe:Rudin1988}'s proposal for the right-adjunction of Bulgarian wh-words in multiple wh-environments. \citeauthor{spe:Rudin1988} refers to \citet{Chomsky1986}, who argues that this type of movement is an option in these environments. While it would be interesting to investigate the disadvantages of this adaptation and the advantages of other analyses, I must postpone this to future research, as this is not the focus of my paper.}\footnote{While the adjacency account correctly derives the particle placement facts for Bulgarian, the result cannot always be an unsplittable complex, as I discuss in \sectref{sec:4} of this paper.} 

\begin{figure}
\begin{forest}
[\textit{v}P
[\textit{only}-phrase
[\textit{only} $\lbrack$Foc$\rbrack$]
[NP]
]
[\textit{v}P[\textit{t}\textsubscript{NP}, roof]]
]
\end{forest}
    \caption{Structure of an \textit{only}-phrase from \citet[ex. 7]{Zanon2018}}
    \label{fig:Zanon:tree1}
\end{figure}

Similarly to my proposal for Bulgarian, \citet{Zanon2018} provides three arguments against \textit{tol'ko} being an NP-adjunct. Just like Bulgarian and German, Russian does not allow the insertion of \textit{only} into a PP:\footnote{A reviewer notes that the insertion of \textit{tol'ko} `only' into a PP is sometimes possible in Russian and provides the following example: 

\ea  
\gll vopreki tol'ko zdravomu smyslu \\
in.spite.of only common.\textsc{dat} sense.\textsc{dat} \\
\glt `in spite of only common sense'
\z 

\noindent This is in line with the occasional circumvention of the ``no PP-insertion''-prohibition that can be found in Bulgarian. Since Russian is not the focus of this paper, I cannot discuss these Russian examples further, but suggest that if Russian and Bulgarian pattern similarly here, these cases are rather infrequent in comparison to the general prohibition at work in both languages.} 

\ea\label{Zanon:basic1}  
\ea[*]
{\gll dlja tol'ko sestry \\
for only sister\\
\glt Intended: `only for the/a sister’}
\ex[*]{
\gll ... s tol'ko krupnymi finansovymi gruppami \\
{} with only large financial groups \\
\glt Intended: `... only with the large financial groups’}
\z
\hfill (Russian; \citealt{Zanon2018}: exx. 8a, c)
\z

\noindent Additionally, \textit{tol'ko} does not pattern with adjectival or adverbial modifiers that would be expected to be NP-adjuncts \citep[422]{Zanon2018}. \REF{Zanon:basic2} and \REF{Zanon:basic3} show that \textit{tol'ko} neither patterns with adverbial modifiers like \textit{o\v{c}en'}\footnote{It is unclear to me why \textit{o\v{c}en'} is analyzed as an adverbial modifier here when it seems to behave more like a degree expression whose purpose is to modify the adjective, as noted by a reviewer. I leave these terminological problems aside for now as this respective example does not affect my analysis of Bulgarian.} nor with adjectival modifiers like \textit{sve\v{z}uju}.\footnote{Transliteration adapted to scientific transliteration in \REF{Zanon:basic2}.} 

\ea\label{Zanon:basic2}
\ea[*]
{\gll Tol'ko vy \minsp{[} SVE\v{Z}UJU]$_{F}$ rybu kupili? \\
only you {} fresh fish bought\\
\glt Intended: `Did you only buy the [FRESH]$_{F}$ fish?’}
\ex[]{
\gll O\v{c}en ty bol'\v{s}uju cenu za \v{s}kury zaprosil.\\
very you big price for pelts asked \\
\glt `You requested too high a price for the pelts.’}
\z
\hfill (Russian; \citealt{Zanon2018}: ex. 11)
\z 

\ea\label{Zanon:basic3}
\ea[*]
{\gll Vy tol'ko kupili \minsp{[} RYBU]$_{F}$? \\
you only bought {} fish\\
\glt Intended: `Did you only buy [FISH]$_{F}$?’\footnote{A reviewer notes that this example is not fully ungrammatical for some Russian speakers, but only degraded. In Bulgarian, it is generally not possible to separate FSPs from the constituents they associate with, although there are (scarce) examples to be discussed in \sectref{sec:3}. Even if the adjacency facts are not as clear-cut in Russian as presented in \citet{Zanon2018}, this does not affect my analysis of Bulgarian.}}
\ex[]{
\gll Vy [SVE\v{Z}UJU]$_{F}$ kupili rybu? \\
you fresh bought fish \\
\glt `You bought [FRESH]$_{F}$ fish?’}\hfill (Russian; \citealt{Zanon2018}: ex. 12)
\z
\z  

\largerpage
\noindent A last argument in favor of \textit{tol'ko} not being adjoined to NP is of a semantic nature: As observed in \citet{Taglicht1984}, NP-adjacent \textit{only} in English causes scope ambiguity. \citet[423-424]{Zanon2018} does not find the analogous ambiguity in Russian, as \REF{Zanon:basic4} shows.  

\ea\label{Zanon:basic4} 
\ea
\gll Ja \v{z}aleju, \v{c}to poceloval tol'ko \minsp{[} MA\v{S}U]$_{F}$. \\
I regret that kissed only {} Ma\v{s}a \\
\glt `I regret that I only kissed [Ma\v{s}a]$_{F}$.’ 
\ea[] {...and no one else.}
\ex[\#] {...but I don't regret that I kissed Anastasia.}
\z 
\ex
\gll Ja tol'ko \minsp{[} MA\v{S}U]$_{F}$ \v{z}aleju, \v{c}to poceloval. \\
I only {} Ma\v{s}a kissed that regret \\
\glt `I only regret that I kissed [Ma\v{s}a]$_{F}$.’
\ea[\#] {...and no one else.}
\ex[] {...but I don't regret that I kissed Anastasia.}
\z
\z
\hfill (Russian; \citealt{Zanon2018}: ex. 15)
\z  

\noindent With respect to the potential adjunction sites of \textit{tol'ko}, Zanon argues that \textit{v}P, CP, and a functional projection in the DP, namely FP, are potential adjunction sites for \textit{tol'ko}. What unites these projections is that they can all be argued to be phases in Russian that are functional projections at the same time. The `only'-complex can adjoin to \textit{v}P in the verbal domain (with the verb optionally raising above the complex then) \citep[426-427]{Zanon2018}.\footnote{This analysis would run into problems under approaches that assume that the finite verb never moves out of \textit{v}P in Russian, as a reviewer remarks. I refer the reader to the discussion of this issue in \citet{Zanon2018}. Since movement of the finite verb to T is assumed for Bulgarian, this debate is less relevant for the analysis that I am pursuing here. See, for example, \citet{Harizanov2019} for a recent discussion and summary of verb position in Bulgarian.} \textit{Tol'ko} is adjoined to CP in \textit{tol'ko}+subject complexes as in \REF{Zanon:basic5} \citep[429]{Zanon2018}.\footnote{A reviewer provides the following example and remarks that \citeauthor{Zanon2018}'s analysis would not be able to account for subjects in embedded CPs that follow a complementizer in C: 

\ea  
\gll Ja znaju čto tol'ko \minsp{[} IVAN]$_{F}$ posmotrel {\.e}tot fil'm. \\
I know that only {} Ivan watched this movie \\
\glt `I know that only Ivan watched this movie.'
\z 

\noindent I leave the solution of this problem under \citeauthor{Zanon2018}'s account open here but would suggest that examples like this perhaps show that \citeauthor{Zanon2018}'s restriction of adjunction sites to CP, \textit{v}P, and FP is too restrictive for Russian. Again, this problem does not affect my more permissive analysis of adjunction sites for Bulgarian FSPs.} 

\ea\label{Zanon:basic5} 
\gll Tol'ko \minsp{[} IVAN]$_{F}$ posmotrel ėtot fil'm. \\
only {} Ivan watched this movie\\
\glt `Only [IVAN]$_{F}$ watched this movie.’\hfill (Russian; \citealt{Zanon2018}: ex. 25)
\z

\largerpage[-2]
\noindent In the nominal domain, \citet[432-433]{Zanon2018} notes that \textit{tol'ko} can be adjoined to the genitive complement inside an NP (or to parts of it such as the numeral modifier shown below), as demonstrated in \REF{Zanon:basic6}. This fact can be accounted for by the existence of a functional projection FP that is able to host \textit{tol'ko}.\footnote{Examples such as \REF{Zanon:basic6} differ in acceptability among Russian speakers, as a reviewer remarks. I must leave open for now why that is the case and how widespread this divergence is in Russian. As examples of this kind are grammatical in all environments in which I have tested them, \citeauthor{Zanon2018}'s prediction is still borne out in Bulgarian.} As I discuss in \sectref{sec:3.3}, this account can be extended to the nominal domain in Bulgarian, but also runs into problems depending on the modifier studied. 

\ea\label{Zanon:basic6} 
\gll Ja znaju \minsp{[} studentov tol'ko PERVOGO$_{F}$ kursa]. \\
I know {} students only first year \\
\glt `I know only the [FIRST]$_{F}$ year students.’\hfill (Russian; \citealt{Zanon2018}: ex. 32c)
\z

\noindent In a way, \citet{Zanon2018} provides an explanation for the adjunction patterns of \textit{only} in Russian that is the exact opposite of what \citet{TishevaDzhonova2003} assume for Bulgarian: Instead of arguing that \textit{only} can adjoin to any type of syntactic constituent, Zanon limits the number of adjunction sites of \textit{only}. As a next step, the F-marked constituent moves towards the position of \textit{only}, instead of \textit{only} adjoining to the respective F-marked constituent anywhere in the clause. This analysis correctly rules out the restrictions on the placement of FSPs in Bulgarian that \citeposst{TishevaDzhonova2003} account leaves unexplained, as I argue in \sectref{sec:3}. 

\section{An adverbial-only analysis of Bulgarian FSPs}\label{sec:3}
In this section, I present arguments against an adnominal analysis of the adjunction behavior of Bulgarian FSPs as well as arguments in favor of an adverbial-only analysis (\sectref{sec:3.1}). \sectref{sec:3.2} provides additional discussion of the behavior of Bulgarian FSPs in the nominal domain and the challenges that this poses for \citeposst{BuringHartmann2001} Particle Theory. Finally, I introduce an adaptation of \citeauthor{BuringHartmann2001}'s Particle Theory to Bulgarian (\sectref{sec:3.3}).  

\subsection{Arguments against an adnominal analysis}\label{sec:3.1}
The biggest advantage of pursuing an adverbial-only analysis of Bulgarian FSPs is that an analysis of this type predicts and explains certain distributional facts that an adnominal analysis struggles to account for. One distributional fact that has been observed for German, as already mentioned in \sectref{sec:2.2}, is the impossibility of adjoining FSPs to DPs within PPs, which adnominal analyses would predict to be an option. \REF{Bulgarianprepositions1} shows that the prohibition, which is judged to be sharply ungrammatical by all my consultants, is not limited to a specific FSP or the involvement of a specific preposition. The prohibition carries over to all FSPs and prepositions that I tested.\footnote{A possible exemption is \textit{vmesto} `instead'. This preposition is the only one which can be inserted within PPs, examples of which can be found in the \textit{Bulgarian National Corpus} \citep{KoevaBNC}. This example is from an excerpt of a (spoken) debate: \textit{P\v{a}rvo da se glasuva p\v{a}rvata \v{c}ast na teksta s predlo\v{z}enieto na gospodin Bu\v{c}kov vmesto samo [LICA]$_{F}$}, `First to vote the first part of the text with Mr. Bu\v{c}kov's proposal instead of only persons'. At this point, it is unclear to me why the PP insertion prohibition does not extend to \textit{vmesto}. However, even with this preposition, examples of FSP-insertion within PPs are scarce and Bulgarian speakers prefer to place the FSP before the preposition.}

\ea\label{Bulgarianprepositions1}
\ea[*]
{\gll Kupi-h krastavic-i za samo \minsp{[} SALATA-TA]$_{F}$. \\
buy-\textsc{pst.1sg} cucumber-\textsc{pl} for only  {} salad-\textsc{def} \\
\glt Intended: ‘I bought cucumbers only for the salad.’}
\ex[*]{
\gll Ana glasuva sre\v{s}tu samo \minsp{[} MARIA]$_{F}$ \\
Ana vote.\textsc{pst.3sg} against  only  {} Maria \\
\glt Intended: ‘Ana voted only against Maria.’}
\ex[*]
{\gll okolo dori golemite \minsp{[} GRAD-OVE]$_{F}$ \\
around even big {} city-\textsc{pl} \\
\glt Intended: `even around big cities’}
\z
\z 

\noindent Adverbial-only analyses neatly predict the PP-insertion prohibition. Furthermore, they also account for the exclusion of adjunction to NPs/DPs in other environments in which the FSP would be forced to adjoin to phrases of this type. An example for this is the coordination test proposed by \citet{Jacobs1983} for German, as shown in \REF{Jacobs:basic1}.

\ea[*] 
{\gll dass Peter und \minsp{\{} nur / sogar / auch\} Luise sich in Straßburg trafen \\
that Peter and {} only {} even {} also Luise \textsc{refl} in Straßburg meet.\textsc{pst} \\ 
\glt Intended: `that Peter and \{only / even / also\} Luise met in Straßburg'}\label{Jacobs:basic1} \hfill (German; \citealt{Jacobs1983}: 45, ex. 3.29b)
\z

\noindent As noted by a reviewer, a purely syntactic account of these examples is unable to account for their ungrammaticality. While I generally agree with this view (and consider it not to be incompatible with my argumentation), I argue that there is a certain component of the ungrammaticality of these examples that can be explained by particle placement. The reviewer gives two reasons for their scepticism: First, \textit{nur} `only' should be incompatible with DP coordination irrespective of syntax due to its exhaustive interpretation. This cannot be entirely true since \REF{Jacobs:basic1} is also degraded/ungrammatical if other FSPs such as \textit{sogar} `even' and \textit{auch} `also' are used. \citet[45]{Jacobs1983} discusses the impact of semantic factors on the ungrammaticality of the example and points out that reversing the order of conjuncts makes the example grammatical: 

\ea[]
{\gll dass nur / sogar / auch \minsp{[} LUISE]$_{F}$ und Peter sich in Straßburg trafen\\ 
that only {} even {} also {} Luise and Peter \textsc{refl} in Straßburg meet.\textsc{pst} \\
\glt `that only / even / also Luise and Peter met in Straßburg'}
\z

\noindent Even if only \textit{Luise} is focused in this example, it is still grammatical, which could be explained by the fact that the FSP can adjoin to an EVP-projection in this case. This extends to examples with non-reflexivized verbs such as \REF{ex:jacobsfn2a}, versus \REF{ex:jacobsfn2b}.

\ea\label{ex:jacobsfn2}
\ea[]{ 
\gll Nur \minsp{[} MARIE]$_{F}$ und Luise haben die Klausur bestanden.\\
only {} Marie and Luise \textsc{aux.pl} the exam pass.\textsc{ptcp} \\
\glt `Only Mary and Luise passed the exam.'} \label{ex:jacobsfn2a}
\ex[*]{
\gll Marie und nur \minsp{[} LUISE]$_{F}$ haben die Klausur bestanden.\\ 
Marie and only {} Luise \textsc{aux.pl} the exam pass.\textsc{ptcp} \\ 
\glt Intended: `Marie and only Luise passed the exam.'}\label{ex:jacobsfn2b}
\z 
\z 

\noindent Using a non-reflexivized verb improves the situation in the case of \textit{even} and \textit{also} (I consider the examples presented here as only slightly degraded in German with these two FSPs instead of \textit{only}). Nevertheless, the ``semantic explanation'' does not fully account for why the reversal of the order of conjuncts should lead to grammaticality here.

In \REF{Jacobs:basic1}, \textit{nur} `only' is forced to adjoin to the coordinated DP \textit{Luise}, which results in ungrammaticality. This extends to Bulgarian, as can be seen in \REF{Bulgariancoordination:1}. 

\ea[\textsuperscript{??/}*] 
{\gll Znaj-a, \v{c}e Peter i samo \minsp{[} ANNA]$_{F}$ se sre\v{s}tna-ha v Berlin. \\
know-\textsc{1sg} that Peter and only {} Anna \textsc{refl} meet-\textsc{pst.3pl} in Berlin \\ 
\glt Intended: `I know that Peter and only Anna met in Berlin.'}\label{Bulgariancoordination:1}
\z 

\newpage
\noindent Consequently, the adverbial-only analysis predicts that adjunction to VP should not be a problem in a coordinating construction. This is borne out, as demonstrated by \REF{Bulgariancoordination:2}.\footnote{While the results of the coordination test fit the predictions made by the adverbial-only analysis, it is important to note that a test of this kind should not be used on its own to make predictions about the correctness of this analysis, since in special constructions such as coordinated structures, there could be other interfering factors at work. Moreover, the intuition of German speakers concerning the German equivalents of examples such as \REF{Bulgariancoordination:2} differ, a problem that I must leave for further research for now. A reviewer points out that adjunction of \textit{only} to the second conjunct should also be difficult in cases of VP-adjunction that lack the purpose reading found in \REF{Bulgariancoordination:2} such as *\textit{John cried and only laughed}. The German equivalent \textit{Jan hat geweint und nur gelacht} is grammatical to me, especially under a temporal interpretation of the conjunction. Again, I conclude that only an analysis that takes semantic and syntactic factors going hand in hand into account can grasp adjunction data of this kind to its full extent. Nevertheless, VP-adjunction seems to be often possible in cases in which DP-adjunction is not, favoring the adverbial-only analysis.}

\ea\label{Bulgariancoordination:2} 
\gll Peter izle-ze i samo \minsp{[} PAZAR-UVA]$_{F}$. \\
Peter go.out-\textsc{3sg.pst} and only {} shop-\textsc{pst}  \\ 
\glt `Peter went out and only did his shopping.'
\z 

\noindent An additional argument against the incorporation of Bulgarian FSPs into the NP/DP is provided by observable stranding phenomena. Stranding of \textit{nur} `only' is possible in German in many instances.\footnote{See \citet{Mursell2021} for discussion.} The FSP can be stranded in Bulgarian, as in \REF{Bulgarian:stranding}.\footnote{A reviewer notes that this example could also be a case of NP-splitting. NP-splitting is possible in Bulgarian, but conflicting judgments are constantly being reported in the literature on these splits so that it is difficult to determine which splits are accepted by a majority of speakers and which ones are not. While the possibility of NP-splitting should be kept in mind when interpreting my examples here, I argue that what we can observe in \REF{Bulgarian:stranding} is not an NP-split as \textit{samo} does not pattern with, for example, adjectival modifiers here. In many NP-splits, it is possible to strand the noun and front the adjective, as in the following example:

\ea
\gll Nova$_{1}$ e kupil [t$_{1}$ kola] \minsp{(} ne stara). \\
new is bought {} car {} not old \\
\glt `He bought a new car, not an old one.’\hfill \citep[ex. 36a]{TassevaDubinsky2018}
\z \par 

\noindent This is not possible with \textit{samo} and other FSPs that I have tested, as fronting \textit{samo} and stranding the noun would mean that \textit{samo} is not associated with the stranded noun anymore, but with the constituent to its right. Additionally, an explanation would be needed for why only the lowest NP can be split in this case so that \textit{samo} is moved to the left periphery. I therefore tentatively conclude that my example does not show an NP-split.} 

\ea\label{Bulgarian:stranding}
\ea
\gll \v{C}et-a samo \minsp{[} ROMAN-I]$_{F}$. \\
read-\textsc{1sg} only {} novel-\textsc{pl} \\
\glt `I read only [novels]$_{F}$.’
\ex
\gll \minsp{[} ROMAN-I]$_{F}$ \v{c}et-a samo. \\
{} novel-\textsc{pl} read-\textsc{1sg} only \\
\glt `[NOVELS]$_{F}$, I read only.’
\z
\z 

\noindent Since \textit{romani} `novels' is placed above the verb in this example, we can conclude that it moved above TP, suggesting that the F-marked constituent moved to the designated FocP in the left periphery. \par 
These arguments taken together suggest that there are not as many adjunction sites for Bulgarian FSPs as descriptive analyses such as \citet{TishevaDzhonova2003} suggest. In fact, adjunction seems to be restricted to projections belonging to the EVP, a proposal which is in line with \citet{BuringHartmann2001} and less restrictive than \citeposst{Zanon2018} analysis of Russian \textit{tol'ko}. However, an additional adjunction site in the nominal domain is needed to account for the adjunction options of Bulgarian FSPs there, as I show in the next section.
\subsection{Bulgarian FSPs in the nominal domain}\label{sec:3.2}
Bulgarian FSPs in the nominal domain show the importance of not only taking syntactic but also semantic factors into account when determining possible adjunction sites for FSPs. As discussed by \citet{BuringHartmann2001}, German FSPs can circumvent the prohibition against adjunction to DPs inside PPs if they are adjoined to an adjectival or numeral modifier, as demonstrated in \REF{DPdata:1}. 

\ea\label{DPdata:1}  
\ea
\gll mit nur \minsp{[} EINEM]$_{F}$ Wagen. \\
with only {} one car \\
\glt `with only [ONE]$_{F}$ car.’
\ex
\gll in nur \minsp{[} WENIGEN]$_{F}$ Sekunden \\
in only {} few seconds \\
\glt `within only [A FEW]$_{F}$ seconds.’ \\\hfill (German; \citealt{BuringHartmann2001}: exx. 82a, c)
\z
\z  

\noindent \citet{BuringHartmann2001} account for this by further generalizing from ``adjunction to EVP'' to ``adjunction to non-arguments'' as the principle governing particle placement in German. However, ``adjunction to non-arguments'' does not explain the fact that German FSPs are unable to adjoin to some modifiers, as discussed in \sectref{sec:2.2}. \citet[247]{Mursell2021} discusses the possibility that there could be a bigger reason explaining adjunction possibilities in general, such as that FSPs only adjoin to elements that introduce a scale (an observation that he attributes to Karen De Clercq), which could be argued for some of the modifiers discussed by him as well as for verbs, accounting for the facts captured by the adverbial-only analysis as well. While it is definitely necessary to consider this bigger reason behind adjunction possibilities that goes beyond a syntactic treatment of the problem, a first step is to successfully capture the adjunction options of FSPs from a syntactic perspective. If we consider Bulgarian data equivalent to the German data discussed above, it becomes evident that Bulgarian does not pattern with German here but shares many characteristics with Russian in this respect. Most importantly, Bulgarian FSPs are able to adjoin to PPs and DPs embedded within DPs, which is impossible in German, as a direct comparison shows (in this example, Bulgarian \textit{samo} adjoined to a PP within a DP while we can observe the impossibility of adjunction to DP within a DP in the German example):  

\ea\label{DPdata:2}
\ea[*]
{\gll die Katze nur \minsp{[} de-s NACHBAR-N]$_{F}$\\
the.\textsc{f} cat only {} the-\textsc{m.gen} neighbor-\textsc{m.gen}  \\\hfill (German)
\glt Intended: `only the cat [of the neighbor]$_{F}$’}
\ex[]
{\gll kotka-ta samo \minsp{[} na S\v{A}SEDKA-TA]$_{F}$ \\
cat-\textsc{def} only {} of neighbor-\textsc{def} \\
\glt `only the cat [of the neighbor]$_{F}$’}
\z
\z  

\noindent This adjunction behavior is a major obstacle for a Particle Theory in the style of \citet{BuringHartmann2001} as data points such as \REF{DPdata:2} are one major argument for excluding adnominal adjunction in German. However, these examples can be reconciled with the help of \citeposst{Zanon2018} proposal for Russian. Under the assumption that Bulgarian FSPs adjoin to a functional projection in the nominal domain, FP, as she proposes for Russian, the examples can be captured by her analysis. Russian and Bulgarian pattern similarly here:\footnote{I prefer \citeauthor{Zanon2018}'s approach over \citeauthor{BuringHartmann2001}'s approach here as \citeauthor{Zanon2018}'s approach does not exclude the adjunction options shown in \REF{DPdata:2} to be possible in Bulgarian. Adopting the ``adjunction to non-arguments''-condition for Bulgarian would mean that the Bulgarian examples in \REF{ex:DPdata:3bulg} could be explained by arguing for adjunction to the numeral, while \REF{DPdata:2} would be predicted to be ungrammatical as there is no modifier present.} 

\ea\label{ex:DPdata:3russ}
\ea
\gll Ja znaju tol’ko \minsp{[} studentov PERVOGO$_{F}$ kursa].\\
I know only {} students.\textsc{acc} first.\textsc{gen} year.\textsc{gen}  \\
\glt `I only know the [FIRST]$_{F}$ year students.’
\ex
\gll Ja znaju \minsp{[} studentov tol’ko PERVOGO$_{F}$ kursa].\\
I know {} students.\textsc{acc} only first.\textsc{gen} year.\textsc{gen}  \\
\glt `I know only the [FIRST]$_{F}$ year students.’ \\\hfill (Russian; \citealt[ex. 32a, c]{Zanon2018})
\z 
\z 

\ea\label{ex:DPdata:3bulg}
\ea
\gll Pozna-vam samo \minsp{[} student-i P\v{A}RVA godina]$_{F}$. \\
know-\textsc{1sg} only {} student-\textsc{pl} first year  \\
\glt `I only know the [FIRST]$_{F}$ year students.’
\ex
\gll Pozna-vam student-i samo \minsp{[} P\v{A}RVA godina]$_{F}$. \\
 know-\textsc{1sg} student-\textsc{pl} only {} first year\\
\glt `I know only the [FIRST]$_{F}$ year students.’
\z
\z  

\noindent While \citeauthor{Zanon2018}'s analysis captures these facts effortlessly, neither this analysis nor newer proposals made for languages such as German, e.g. \citet{Mursell2021}, account for the variation in adjunction behavior to different modifiers at this point. Adjunction to modifiers within PPs in Bulgarian varies depending on the modifier and the preposition involved. \REF{DPdata:4} can be accounted for with the help of \citeauthor{Mursell2021}'s proposal for scalar modifiers mentioned above (except for \REF{DPdata:4a}, which possibly requires additional phonological considerations), as a reviewer notes. The ungrammaticality of the last two examples could then be explained by the fact that `big' is not a scalar modifier. 

\ea\label{DPdata:4}
\ea[*]
{\gll săs samo edna kola \\
with only one car \\
\glt Intended: `with only one car’}\label{DPdata:4a}
\ex[\textsuperscript{?}]
{\gll sled samo njakolko sekund-i \\
within only few second-\textsc{pl} \\
\glt `within only a few seconds’}\label{DPdata:4b}
\ex[]
{\gll me\v{z}du samo dve optsi-i \\
between only two option-\textsc{pl} \\
\glt `between only two options’}\label{DPdata:4c}
\ex[]
{\gll me\v{z}du dori dve optsi-i \\
between even two option-\textsc{pl} \\
\glt `between even two options’}\label{DPdata:4d}
\ex[*]
{\gll okolo samo golemite grad-ove$_{F}$ \\
around only big city-\textsc{pl} \\
\glt Intended: `only around big cities’}\label{DPdata:4e}
\ex[*]
{\gll okolo dori golemite grad-ove$_{F}$ \\
around even big city-\textsc{pl} \\
\glt Intended: `even around big cities’}\label{DPdata:4f}
\z
\z 

\noindent The puzzle involving the restrictions that different prepositions and modifiers impose on FSP placement in Bulgarian cannot be resolved here. However, it allows us to draw a few conclusions for the analysis. First, it points towards the fact that \citeposst{Zanon2018} analysis involving adjunction to FP is on the right track for Bulgarian, while \citeposst{BuringHartmann2001} adjunction to non-arguments would not be able to capture the available adjunction sites of Bulgarian FSPs in the nominal domain. Second, it is evident how challenging nominal data is for adverbial-only approaches. An additional assumption, such as adjunction to FP, is needed in order to capture the empirical facts. There are two possibilities to develop a proper account that is able to capture the difference between individual modifiers. The first would be to aim for a separate treatment of FSP-adjunction in the clausal and nominal domain, which would be the less economic approach. A second option is the one already sketched, namely exploring the relationship between possible adjunction sites of FSPs and elements introducing a scale. Much of the data discussed in this section can be explained by the presence or absence of scalar modifiers, but \REF{DPdata:2} remains unexplained under this approach. Nevertheless, I consider this a fruitful path for future research in this area that aims to not only account for the Bulgarian facts, but for FSP adjunction in other languages such as German as well. 

\subsection{A Particle Theory for Bulgarian}\label{sec:3.3}
Based on the facts already discussed here, I consider it reasonable to develop a Particle Theory for Bulgarian based on \citeposst{BuringHartmann2001} Particle Theory for German, since both languages show surprisingly similar patterns with respect to the adjunction behavior of their FSPs. At the same time, essential parts of \citeposst{Zanon2018} proposal, such as the possibility of adjunction to a functional projection FP in the nominal domain, are needed to account for the Bulgarian data. I therefore argue for a combination of both proposals for Bulgarian. In this section, I discuss the five clauses of \citeauthor{BuringHartmann2001}'s Particle Theory and how these conditions could be adapted to the Bulgarian facts. 
\subsubsection{Adjunction to EVP}\label{sec:3.3.1}
As discussed in the previous subsections, an adverbial-only analysis of the adjunction behavior of Bulgarian FSPs elegantly excludes the insertion prohibitions that can be found in the language. However, not all adverbial-only analyses are alike. \citet{BuringHartmann2001} themselves propose two of them: They first argue that German FSPs only adjoin to the EVP and then further generalize to adjunction to non-arguments, as described in \sectref{sec:2.2}. While this generalization is, as already discussed, not unproblematic for German, the previous section has shown that adjunction to non-arguments would also not be able to capture the adjunction behavior of Bulgarian FSPs in the nominal domain. A related, but distinct option would therefore be \citeposst{Zanon2018} proposal that assumes that Russian \textit{tol'ko} only adjoins to \textit{v}P, CP, and FP. Please note that \citeauthor{Zanon2018}'s analysis excludes adjunction to TP, which \citeauthor{Zanon2018} rules out based on examples such as \REF{TPdata:1}.

\ea\label{TPdata:1} 
\gll Ja ne znaju...\\ 
I \textsc{neg} know \\ 
\glt `I don't know...' \\
\ea[\textsuperscript{?/}*]
{\gll posmotrel li tol'ko \minsp{[} IVAN]$_{F}$ ėtot fil'm. \\
watched Q only {} Ivan this movie  \\
\glt Intended: `if only [Ivan]$_{F}$ watched this movie.’}\label{TPdata:1a}
\ex[]
{\gll posmotrel li Ivan tol'ko \minsp{[} \.{E}TOT]$_{F}$ fil'm. \\
watched Q Ivan only {} this movie  \\
\glt `if Ivan watched only [THIS]$_{F}$ movie.’ \hfill (Russian; \citealt{Zanon2018}: ex. 24)}\label{TPdata:1b}
\z
\z   

\noindent In \REF{TPdata:1a}, \textit{tol'ko} must be adjoined to TP, below \textit{li} in the CP, which is degraded in Russian. The example becomes grammatical once \textit{tol'ko} is adjoined to \textit{v}P, as in \REF{TPdata:1b} \citep[428-429]{Zanon2018}.\footnote{A reviewer notes that TP-adjunction could, in fact, be possible in Russian, and provides the following example: 

\ea 
\gll Ja ne znaju...\\ 
I \textsc{neg} know \\ 
\glt `I don't know...' \\
\ea[]
{\gll posmotrel li tol'ko \minsp{[} \.{E}TOT]$_{F}$ student ėtot fil'm. \\
watched Q only {} this student this movie  \\
\glt `if only [THIS]$_{F}$ student watched this movie.’}
\ex[\textsuperscript{?/}*]
{\gll posmotrel li Ivan tol'ko \minsp{[} FIL'M]$_{F}$. \\
watched Q Ivan only {} movie  \\
\glt `if Ivan watched only a [MOVIE]$_{F}$.’}
\z
\z  \par 

\noindent This example potentially shows TP-adjunction, so Russian could be more permissive than assumed by \citet{Zanon2018}, and therefore also closer to Bulgarian in this respect. In any case, my proposal for Bulgarian allows for TP-adjunction and is therefore not affected by the pattern found for Russian here.} Reproducing these examples in Bulgarian shows that Bulgarian is more permissive here and allows adjunction to TP as well. I conclude that the ``adjunction to EVP''-condition that is more permissive than \citeauthor{Zanon2018}'s proposal makes the correct predictions for Bulgarian.

\ea\label{TPdata:2}
\gll Az ne znaj-a...\\ 
\textsc{1sg} \textsc{neg} know-\textsc{1sg} \\ 
\glt `I don't know...' \\
\ea
\gll dali samo \minsp{[} IVAN]$_{F}$ e gleda-l tozi film. \\
whether only {} Ivan \textsc{aux.3sg} watch-\textsc{ptcp} this movie  \\
\glt `whether only [Ivan]$_{F}$ watched this movie.’
\ex
\gll dali Ivan e gleda-l samo \minsp{[} TOZI]$_{F}$ film. \\
whether Ivan \textsc{aux.3sg} watch-\textsc{ptcp} only {} this movie  \\
\glt `whether Ivan watched only [this]$_{F}$ movie.’
\z
\z   
 
\subsubsection{Adjunction to maximal projections}\label{sec:3.3.2}
As \citet[240-244]{BuringHartmann2001} discuss, there are theory-internal reasons that make it desirable to uphold the requirement that FSPs adjoin to maximal projections only. I argue that the clause should hold for Bulgarian as well, since it naturally excludes cases such as \REF{maximaldata}. Here, the FSP can adjoin to the auxiliary in T, but not to the finite verb on its own. 

\ea\label{maximaldata}
\ea[]
{\gll Ti samo \v{s}te \minsp{[} SEDI-\v{S}]$_{F}$ \\
you only \textsc{aux.fut} {} sit-\textsc{prs.2sg} \\
\glt `You will only sit.’}
\ex[*]
{\gll Ti \v{s}te samo \minsp{[} SEDI-\v{S}]$_{F}$. \\
you \textsc{aux.fut} only {} sit-\textsc{prs.2sg} \\
\glt Intended: `You will only sit.’}
\z
\z 

\subsubsection{The c-command condition}\label{sec:3.3.3}
The c-command criterion can be maintained for Bulgarian if it is adapted in a way that allows for the FSP to not necessarily c-command the F-marked constituent only, but also its trace in cases in which the focused constituent moved above the FSP.\footnote{Association of focus with traces is extensively discussed in \citet{Erlewine2014}, including discussion of previous work on focus association that deemed this to not be a possible operation.} In all other cases, the local feature checking relationship proposed by \citet{Zanon2018} for Russian also holds for Bulgarian. In fact, it is reasonable to argue that in the cases of right-adjunction discussed here, the F-marked constituent moved to adjoin to the FSP to locally check the strong [Foc] feature and moved above the FSP in a second step.\footnote{As I will briefly touch upon in \sectref{sec:4}, this optional step would have to be constrained by discourse-level constraints instead of being driven by a syntactic feature.} I discuss further details of this approach in \sectref{sec:4}. 
\subsubsection{The adjacency requirement}\label{sec:3.3.4}
In contrast to other criteria such as the EVP requirement, the adjacency requirement does not have to be adapted for Bulgarian. There are abundant examples, such as \REF{adjacency}, demonstrating that FSPs need to adjoin to their F-marked constituent as closely as possible.\footnote{The adjacency requirement is very strict in Bulgarian, and it is difficult to find examples in which adjacency of an F-marked constituent to its FSP is not required. The following is a puzzling example (adapted from \citealt{TishevaDzhonova2003}), as adjacency is not required here for at least some of my consultants (although judgments differ). For this group of speakers, all positions of \textit{samo} indicated in the example are possible while the FSP is associated with the F-marked constituent.

\ea 
{\gll Tova \minsp{(} samo) mo\v{z}e \minsp{(} samo) da b\v{a}de \minsp{(} samo) \minsp{[} ofis-\v{a}t na MICROSOFT]$_{F}$. \\
this {} only could {} only to be {} only {} office-\textsc{def} of Microsoft  \\ 
\glt `This could only be the office of Microsoft.'}
\z  

\noindent However, these examples are scarce, and it is unclear at this point for how many Bulgarian speakers they are grammatical.
} 

\ea[*] 
{\gll Kupi-h samo krastavic-i \minsp{[} ZA SALATA-TA]$_{F}$. \\
buy-\textsc{pst} only cucumber-\textsc{pl} {} for salad-\textsc{def}  \\ 
\glt Intended: `I bought cucumbers only [for the salad].'}\label{adjacency}
\z 

\noindent At this point, a typological remark is in order. Although both Bulgarian and German adhere to the adjacency requirement, both languages still differ in the adjunction behavior of their FSPs insofar as Bulgarian FSPs do seem to possess a strong [Foc] feature that triggers movement of the F-marked constituent to the position of the FSP. While \citet[428] {Zanon2018}'s typological generalization, namely that ``in overt focus movement languages, a focalized XP-associate must be adjacent to the F-licensing element" was made with Russian in mind, the same holds for Bulgarian. Just like Russian, Bulgarian possesses overt focus movement and adheres to the adjacency requirement. Despite the similarities between FSP-adjunction in Bulgarian, Russian, and German, German seems to be located in another place in the typological realm here. German does not possess overt focus movement, but still requires adjacency if the maximal projection requirement is not violated and the syntax of the language permits adjacency. German is taking a middle ground here between the two stricter Slavic languages discussed and languages whose FSPs adjoin more loosely in general, such as English. While further developing this discussion would go beyond the scope of this paper, I want to underline the insights that could result from an investigation of FSP placement from a typological perspective.\footnote{I thank \v{Z}eljko Bo\v{s}kovi\'{c} for discussion of this point.}
\subsubsection{Left-adjunction}\label{sec:3.3.5}
While the left-adjunction criterion is absolutely necessary to derive the correct particle placement for German, Bulgarian FSPs do not have to be left-adjoined to the F-marked constituent. Nevertheless, they usually remain close to their focused constituent, even when they surface to the right of the F-marked constituent, the reasons for which I discuss in \sectref{sec:4}. In order to capture the apparent right-adjunction of Bulgarian FSPs, \citeposst{BuringHartmann2001} fifth clause has to be adapted so that not only left-adjunction to an f-node of the FSP's focus is allowed, but also left-adjunction to the trace left behind by the F-marked constituent moving above the FSP. 
\subsubsection{Summary}\label{sec:3.3.6}
Summing up, I propose the following Particle Theory for Bulgarian, based on \citeposst{BuringHartmann2001} proposal combined with the analysis by \citet{Zanon2018}, adapted to account for the Bulgarian data discussed here. 

\ea\label{PTBulgarian1} 
\textsc{The Particle Theory for Bulgarian} \\
For any node $\alpha$ marked F in a phrase marker P, let the set of f-nodes of $\alpha$ consist of all nodes $\beta$ in P such that 
\ea{$\beta$ is an EP (extended projection) of some V $\gamma$ or a functional projection FP within DP}
\ex{$\beta$ is a maximal projection}\label{spellref:PTBulgarian1b}
\ex{$\beta$ dominates $\alpha$ or a trace of $\alpha$ or is identical to $\alpha$}\label{spellref:PTBulgarian1c}
\ex{there is no EP $\beta$' of $\gamma$ such that $\beta$ dominates $\beta$' and $\beta$' meets \REF{spellref:PTBulgarian1b} and \REF{spellref:PTBulgarian1c}.}
\z
\z 

\ea\label{PTBulgarian2} 
{A FSP must be left-adjoined to an f-node of its focus or its trace.}
\z

\section{Post-focal FSPs in Bulgarian}\label{sec:4}
As already discussed, Bulgarian FSPs must adjoin to the F-marked constituent that they belong to as closely as possible. However, there are two different, but, as I argue, related cases in which the F-marked constituent is able to move out of its position right-adjacent to the FSP. The first case, shown in \REF{postfoc1}, involves the focused constituent moving above the FSP, but staying immediately above it. The second case, the stranding case shown in \REF{Bulgarian:stranding} (repeated here as \REF{postfoc2}), consists of the F-marked constituent moving to a high position in the clause, presumably FocP in the left periphery.\footnote{A reviewer asks how optional movement can take place here if a feature-based theory is assumed. While movement of the focused constituent to the position of the FSP seems obligatory, the movement types described here are not. My explanation is that they are, in fact, not feature-driven. The strong [Foc] feature should have already been checked and deleted by the time the focused constituent has moved to the FSP. This optional movement would then be  caused by more discourse-based reasons which would have to be explored in the future.} 

\ea\label{postfoc1}
\ea
\gll \v{C}et-a \minsp{[} ROMAN-I]$_{F}$ samo. \\
read-\textsc{1sg} {} novel-\textsc{pl} only \\
\glt `I read only [novels]$_{F}$.’
\ex
\gll V\v{c}era \minsp{[} az]$_{F}$ s\v{a}\v{s}to \v{c}eto-h roman. \\
yesterday {} I also read-\textsc{pst.1sg} novel \\
\glt `Yesterday, [I]$_{F}$ also read a novel.’
\z
\z

\ea\label{postfoc2}
\ea
\gll \v{C}et-a samo \minsp{[} ROMAN-I]$_{F}$. \\
read-\textsc{1sg} only {} novel-\textsc{pl} \\
\glt `I read only [novels]$_{F}$.’
\ex
\gll \minsp{[} ROMAN-I]$_{F}$ \v{c}et-a samo \\
{} novel-\textsc{pl} read-\textsc{1sg} only \\
\glt `[NOVELS]$_{F}$, I read only.’
\z
\z  

\noindent The first important question that these types of ``movement out of focus'' (meaning cases in which a focused constituent left its original position right-adjacent to its FSP) raise is what consequences they have for our Particle Theory. \citet{BuringHartmann2001} do not assume the possibility of right-adjunction in German, which derives the German adjunction facts correctly. Although the surface word order of the F-marked constituent and the FSP in Bulgarian suggests that right-adjunction is an option in this language, I will not argue for this to be the case for two reasons. First, the F-marked constituent is still interpreted as the constituent associated with the respective FSP, even if it has moved out of its position right-adjacent to the particle. This is surprising given the fact that Bulgarian FSPs are usually interpreted as strictly associating with the constituent following them. This suggests that the FSP associates with the F-marked constituent's trace and that the focused constituent reconstructs at LF when it is interpreted.\footnote{A reviewer asks why it should not be possible for the FSP to operate on the F-marked constituent while being right-adjoined to it. While this is generally an option, I argue that this is not the case here. Allowing for general right-adjunction of FSPs in Bulgarian would overgenerate adjunction options in the middle of the clause that are unavailable, as \REF{nomiddleposition} in this section shows. At the same time, Bulgarian FSPs are able to associate with traces of F-marked constituents, which explains cases such as \REF{postfoc2}. Focus movement to a high and a potentially low focus position is the more economical assumption here.} The (simplified) trees in \figref{fig:movementtree1} and \ref{fig:movementtree2} show how cases such as \REF{postfoc1} and \REF{postfoc2}, respectively, could be represented.

\begin{figure}
\begin{forest}
[TP
[Subject
]
[T$'$
[T [četa] ]
[\textit{v}P 
[\textit{only}-phrase
[FocP [roman-i, roof]]
[\textit{only}P
[{samo} $\lbrack$Foc$\rbrack$]
[\textit{t}\textsubscript{roman-i}]
  ]] [\textit{v}$'$ [\textit{t}\textsubscript{četa} ] [VP [\textit{t}\textsubscript{roman-i}, roof]]]   
  ]
]
]
\end{forest}
    \caption{Low movement out of the \textit{only}-phrase}
    \label{fig:movementtree1}
\end{figure}

\begin{figure}
\begin{forest}
[FocP 
[roman-i]
[TP
[Subject
]
[T$'$
[T [\v{c}eta] ]
[\textit{v}P 
[\textit{only}-phrase
[{samo} $\lbrack$Foc$\rbrack$]
[\textit{t}\textsubscript{roman-i}]
  ] [\textit{v}$'$ [\textit{t}\textsubscript{četa} ] [VP [\textit{t}\textsubscript{roman-i}, roof]]]   
  ]
]
]
]
\end{forest}
    \caption{High movement out of the \textit{only}-phrase}
    \label{fig:movementtree2}
\end{figure}

As discussed by \citet{Erlewine2014}, for example, a similar operation is available in German and Dutch, which are both languages that allow movement of F-marked constituents to clause-initial positions. A second reason for maintaining \citeposst{BuringHartmann2001} left-adjunction condition for a Particle Theory for Bulgarian is that movement out of focus in Bulgarian is restricted in the sense that only the two types of movement shown in \REF{postfoc1} and \REF{postfoc2} seem to be allowed. Moreover, movement out of focus in the case in \REF{postfoc1} is only permitted if the F-marked constituent is the lowest constituent in the clause, as already noted by \citet{TishevaDzhonova2003}, or around the subject position in T. This movement type is not permitted in positions in the middle of the clause since the FSP would then rather be interpreted as being associated with the constituent following it by Bulgarian speakers:\footnote{A reviewer notes that backwards association in the middle of the clause of the kind discussed here is grammatical in Russian once a disambiguating context is employed. To the best of my knowledge, this type of ambiguity does not exist in Bulgarian as all my consultants strongly reject backwards association in the middle of the clause.}

\ea[*] 
{\gll Kupi-h \minsp{[} KRASTAVIC-I]$_{F}$ samo za salata-ta. \\
buy-\textsc{pst} {} cucumber-\textsc{pl} only for salad-\textsc{def}  \\ 
\glt Intended: `I bought only cucumbers for the salad.'}\label{nomiddleposition}
\z 

\noindent Allowing for right-adjunction would overgenerate the options that are there for movement out of focus in Bulgarian and would not predict that there are actually only two positions that the moved F-marked constituent can move to. \par 
At this point, two questions remain open. First, an obvious question is where the F-marked constituent moves to in the low cases of movement out of focus. It is generally noted in the syntactic literature on Bulgarian that while the left periphery of the language is well-researched, much less is known about the verbal domain and the positions that it hosts \citep{Krapova2002}. A possible solution to this problem would be to argue for a low, post-verbal focus position, along the lines of the proposal made by \citet{spe:Belletti2004} for Italian. A second question that I leave open here is whether there are differences in interpretation between the association of an F-marked constituent preceding or following the FSP. As \REF{postfoc1} shows, the usage of a post-focal FSP usually requires focal stress on the F-marked constituent, which helps speakers associate and interpret it as belonging to the FSP following it. This kind of focal stress is not required if the F-marked constituent follows its FSP.\footnote{A reviewer suggests the possibility that a moved focus is always contrastive. At the same time, in-situ would then be ambiguous between a contrastive and a non-contrastive interpretation. While this is certainly a plausible option, it would not explain what the trigger for this optional movement is. An analysis along the lines of \citet{Titov2020} could solve this problem, but I leave this question open for now.} Additionally, my consultants (as well as \citealt{Nicolova2000}) report that there is a register difference between the two low options, with the pre-FSP position that the F-marked constituent can be in being associated with colloquial, informal speech. Future research could focus on further differences between the two positions and what they can tell us about the semantic differences between them. 

\section{Concluding remarks}\label{sec:5}
In this paper, I argued for an adverbial-only analysis of Bulgarian focus-sensitive particles that combines two proposals, namely \citeposst{BuringHartmann2001} analysis of German FSPs and \citeposst{Zanon2018} analysis of the adjunction behavior of Russian \textit{tol'ko} `only'. While several arguments point against the feasibility of an adnominal analysis of Bulgarian FSPs, their adjunction options in the nominal domain suggest that \citeauthor{Zanon2018}'s proposal involving adjunction to FP in the nominal domain is on the right track for Bulgarian, in contrast to \citeauthor{BuringHartmann2001}'s ``adjunction to non-arguments''-condition. A gap that I necessarily leave aside in this paper is the question of semantic properties of individual modifiers constraining the adjunction possibilities of Bulgarian FSPs, a question which I argued to be essential for understanding the additional semantic reasons for adjunction, even beyond Bulgarian. Future research could close this gap at this point with a more detailed semantic investigation. Subsequently, I proposed a Particle Theory for Bulgarian based on \citet{BuringHartmann2001}. \par 
At this point, it becomes evident that \citeauthor{BuringHartmann2001}'s left-adjunction condition that accounts for the rigid exclusion of right-adjunction of FSPs in German cannot be upheld in its original formulation when their analysis is extended to Bulgarian. Bulgarian FSPs are able to move above the FSP dominating them. FSPs adjoined to an F-marked constituent low in the clause can even be stranded while the focused constituent moves to FocP in the left periphery. These two types of movement are, however, highly restricted. F-marked constituents can move above FSPs but have to remain close to them in the first movement type. They are only able to move into the high focus position in the left periphery or must remain low, in a position where a second, low focus position could be assumed in Bulgarian. Future research could focus on finding additional evidence for or counterexamples against the existence of such a projection, as well as possible semantic differences between the two available positions for the F-marked constituent that precedes or follows the FSP dominating it. In any case, Bulgarian FSPs can be split from their F-marked constituents after local checking of the strong [Foc] feature, although in a very controlled manner, as argued above. \par 
Finally, the investigation conducted here, on a par with \citet{Zanon2018}, suggests fruitful paths for typological research investigating the connection between overt focus movement and the strict adjacency requirement that holds in Russian and Bulgarian. While adjacency is not required in English, it is necessary in both languages. Languages such as German can be placed in the middle ground between these two extremes, with German not requiring overt focus movement, but adjunction as close to the F-marked constituent as German syntax allows. Future investigations into FSP placement could focus on other, also typologically unrelated language families in order to learn more about the connection of adjacency and overt focus movement. 

\section*{Abbreviations}

\begin{tabularx}{.5\textwidth}{@{}lQ}
\textsc{1}&first person\\
\textsc{2}&second person\\
\textsc{3}&third person\\
\textsc{acc}&accusative\\
\textsc{aux}&auxiliary\\
\textsc{dat}&dative\\
\textsc{def}&definite\\
%\textsc{f}&feminine\\
\textsc{fut}&future\\
\textsc{gen}&genitive\\
\end{tabularx}%
\begin{tabularx}{.5\textwidth}{lQ@{}}
\textsc{m}&masculine\\
%\textsc{nom}&nominative\\
\textsc{pl}&plural\\
\textsc{prs}&present tense\\
\textsc{pst}&past tense\\
\textsc{ptcp}&participle\\
\textsc{q}&question\\
\textsc{refl}&reflexive\\
\textsc{sg}&singular\\
&\\ % this dummy row achieves correct vertical alignment of both tables
\end{tabularx}

\section*{Acknowledgments}
I would like to express my thanks to my main consultants, Joana Emilova Djadjeva, Iverina Ivanova, Nelli Kerezova, and Evelina Parvanova, for their judgments and discussion of the data. Nothing of this would have been possible without them. I am also thankful to the audiences at Sinfonija 15, FDSL-15, and the syntax colloquium at Goethe University/Frankfurt for feedback and comments. The first draft of this paper was substantially improved by the comments made by two reviewers, whom I would like to thank for their reviews. Additionally, I thank Rajesh Bhatt, \v{Z}eljko Bo\v{s}kovi\'{c}, Katharina Hartmann, and Roumyana Pancheva for discussion of this project. All remaining errors and shortcomings are of course my own. 

\printbibliography[heading=subbibliography,notkeyword=this]

\end{document}
