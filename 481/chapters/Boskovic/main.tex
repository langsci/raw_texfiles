\documentclass[output=paper,colorlinks,citecolor=brown]{langscibook}
\ChapterDOI{10.5281/zenodo.15394170}
%\bibliography{localbibliography}

\author{Željko Bošković\affiliation{University of Connecticut}}

\SetupAffiliations{mark style=none}
\lehead{Željko Bošković}


\title[Multiple wh-fronting in a typological setting]{Multiple wh-fronting in a typological setting: What is behind multiple wh-fronting?}
\abstract{The paper establishes broad typological correlations between multiple wh-fronting (MWF) and other phenomena in an attempt to understand what is behind MWF. In particular, the paper establishes a correlation between MWF and the morphological shape of wh-words, which is argued to be responsible for MWF. MWF languages are also shown to be characterized by a particular status regarding articles: they either lack articles or have affixal articles (the difference is shown to matter for superiority effects). Certain cases of non-wh indefinite interpretations of wh-phrases and the exceptional behavior of D-linked wh-phrases regarding MWF -- they are not subject to it -- are also discussed and captured (including Hungarian, where D-linked wh-phrases are not exceptional in this respect, which is tied to another exceptional property of Hungarian).

\keywords{multiple wh-fronting, indeterminates, indefinites, D-linking, articles, Superiority}}

\begin{document}
\maketitle

\section{What is special about multiple wh-fronting?}\label{sec:bosk:1}

The goal of this paper is to shed light on what is behind one particular language type regarding multiple questions. Most languages front one question word/wh-phrase or leave them all in situ in multiple questions. The former type is illustrated by English \REF{ex:bosk:1} and the latter by Chinese \REF{ex:bosk:2}.

\ea What did John give to who?
\label{ex:bosk:1}
\z

\ea \gll John gei-le shei shenme? \\
John give-\textsc{pfv} who  what\\\hfill(Chinese)
\glt `What did John give to who?' 
\label{ex:bosk:2} 
\z 

\noindent There is another pattern, which is not frequent crosslinguistically: the so-called multiple wh-fronting languages (MWF), which front all wh-phrases in questions. The pattern is illustrated by Serbo-Croatian (SC) examples in \REF{ex:bosk:3} (note that SC is an SVO language).\footnote{There are some highly specific contexts where MWF languages need not front wh-phrases (just like there are contexts where English can employ wh-in-situ). I will generally not be concerned with those exceptional contexts here (apart from D-linking), just with the broad, main pattern. I merely note that, as discussed in \citet{Bošković2002}, several of those exceptional contexts involve PF issues, e.g. the case where the fronted wh-phrases would yield a sequence of homophonous elements, like Romanian \REF{ex:bosk:i}. \citet{Bošković2002} shows that such cases are exceptional only superficially -- they still involve MWF in the syntax, with pronunciation of a lower copy of a moved wh-phrase (second \textit{ce} in \REF{ex:bosk:ia}), which is motivated by PF considerations. Thus, the second wh-phrase in \REF{ex:bosk:ia} licenses parasitic gaps (see \REF{ex:bosk:ii}), which is a test for movement in overt syntax (compare \REF{ex:bosk:iiia} and \REF{ex:bosk:iiib}).

\ea\label{ex:bosk:i} 
\ea[]
{\gll Ce     precede   ce?\\
what precedes what\\\hfill (Romanian)}\label{ex:bosk:ia} 
\ex[*]
{\gll Ce ce precede?\\
what what precedes\\}\label{ex:bosk:ib}
\z\z

\ea \gll Ce    precede   ce     fără       să influenţeze?\\
what precedes what without \textsc{subj.part} influences\\\hfill (Romanian)
\glt `What precedes what without influencing?'\label{ex:bosk:ii}
\z 

\ea\label{ex:bosk:iii} 
\ea[*]{What precedes what without influencing?}\label{ex:bosk:iiia}
\ex[] {What did Mary promote without influencing?}\label{ex:bosk:iiib}
\z\z

}

\ea\label{ex:bosk:3} 
\ea[] {\gll Ko šta kupuje?\\
who what buys\\\hfill(SC)
\glt `Who is buying what?'}\label{ex:bosk:3a}
\ex[*] 
{\gll Ko kupuje šta?\\ 
who buys what\\}\label{ex:bosk:3b}
\z\z

\noindent There have been quite a few works on MWF in the generative tradition since the seminal paper by \citet{bos:Rudin1988} (MWF has been discussed less outside of that tradition, but see e.g. \citealt{Mycock2007}). These works generally focus on examining the structure and the derivation of MWF constructions. However, they do not attempt to understand what is really behind MWF, why some languages employ this strategy.

This paper aims to address that question, but from a broad typological perspective, in particular, by establishing correlations between MWF and other phenomena. Its scope will be limited -- I will not go into the derivation and the structure of MWF constructions; the goal of the paper is simply to establish, and understand, prerequisites for the MWF pattern, in an effort to understand what is behind this strategy of forming multiple questions. The discussion will be based on the following 18 (typologically diverse) MWF languages: SC, Romanian, Polish, Russian, Bulgarian, Macedonian, Czech, Slovenian, Ukrainian, Yiddish, Lithuanian, Hungarian, Basque, Mohawk, Georgian, Ossetic, Svan, and Latin.\footnote{The list includes languages I was able to identify as having MWF (and determine for them the additional information that is needed in the discussion below) based on literature surveys (most of them are well-known as MWF languages; for some less-known cases, see \citealt{Ledgeway2012} for Latin, \citealt{Baker1996} for Mohawk, \citealt{GillonArmoskaite2015} for Lithuanian, \citealt{Erschler2012} for Ossetic, \citealt{Erschler2015} for Georgian and Svan).} Latin will turn out to be particularly useful, since it can be compared with modern Romance languages.

What will be important for our purposes is the notion of indeterminates (the term goes back to \citealt{Kuroda1965}, who actually took it from traditional Japanese grammars, which use the term ``indeterminate words''). In many languages, the same forms that are used for wh-words have a variety of usages, like existentials, universal quantifiers, negative concord/negative polarity items, free choice, depending on the context where they occur (for much relevant discussion, see \citealt{Haspelmath1997}). They are referred to as indeterminates since their exact quantificational force is not inherently determined -- it is determined by the licensing context in which they are found.

\citet{Cheng1991}, a predecessor of this work, observes that Bulgarian, Polish, and Hungarian have indeterminate systems. It turns out that all MWF languages from above have a productive indeterminate system, which suggests that the indeterminate system is a prerequisite for MWF. But there is more to it. There are different types of indeterminate systems. I define here a particular type, which I will refer to as the sub-wh system. It is a fully productive system where addition of an inseparable affix to a wh-phrase results in a series of meanings shown in SC (\ref{ex:bosk:4}).

\ea\label{ex:bosk:4} \ea ko `who'\hfill{}(SC)\\
\ex \textit{i}ko `anyone'\label{ex:bosk:4a}
\ex \textit{n}iko `no one'\label{ex:bosk:4b}
\ex \textit{ne}ko `someone'\label{ex:bosk:4c}
\ex \textit{sva}ko `everyone'\label{ex:bosk:4d}
\ex \textit{bilo} ko `whoever'\label{ex:bosk:4e}
\z\z

\noindent There is a morphological subset-superset relationship between the wh/question usage and other usages, as stated in (\ref{ex:bosk:5}) regarding `who'.
 
\ea\label{ex:bosk:5} sub-wh system: \textit{who}+X for other pronouns (inseparable, fully productive, order doesn't matter)\z

\noindent What is not a sub-wh system is the situation found in Chinese, where the same form can have different functions, as illustrated by (\ref{ex:bosk:6}), or Japanese, where a particle occurs on each function -- in some cases inseparable (namely, existential), in some cases separable -- note that \textit{-ka}, which is always separated on the wh-usage in Standard Japanese, need not be separated in Okinawan, as illustrated by (\ref{ex:bosk:7}).
 
\ea\label{ex:bosk:6} \ea \gll Ni xiang mai \textit{shenme} \minsp{(} ne)?\\
you want  buy what {} \textsc{q}\\\hfill(Chinese)
\glt `What do you want to buy?'
\ex \gll Wo bu  xiang mai \textit{shenme}.\\
I not want buy anything\\
\glt `I don't want to buy anything.'
\ex \gll Wo xiang mai  yi-dian \textit{shenme}.\\
I want buy one-\textsc{cl}  something\\
\glt `I want to buy something.' 
\z\z 
 
\ea\label{ex:bosk:7} 
\ea \gll Taruu-ja \textit{nuu}   koota-\textit{ga}.\\
Taro-\textsc{top} what bought-\textsc{q}\\\hfill (Okinawan)
\glt `What did Taro buy?'
\ex \gll Taruu-ja \textit{nuu-ga} koota-ra.\\
Taro-\textsc{top} what-\textsc{q} bought-\textsc{ra}\\
\glt `What did Taro buy?'\hfill(\citealt{KinjoOseki2016}) 

\z\z

\noindent It should be noted that it has been argued that the \textsc{q} marker starts with the wh-phrase even in Standard Japanese (just as in Okinawan), see e.g. \citet{Hagstrom1998}. This is then a rather different system from SC, where the wh-form is a subset of everything.\footnote{Japanese is, however, more similar to SC in the relevant respect than Chinese is, which may not be surprising in light of the discussion below given that Japanese in fact used to be a MWF language (i.e. Old Japanese appears to have been a MWF language; see \citealt{Aldridge2009}, \citealt{Dadan2019}).}

English also does not have a sub-wh system since the relevant system is not fully productive in English (compare \textit{some}where, \textit{every}where, \textit{no}where, \textit{any}where with *\textit{some}who\textit{/every}who\textit{/no}who, *\textit{no}what/\textit{no}when/\textit{no}how), i.e., it is lexicalized (\citealt{Cheng1991} suggests that the good cases are lexically incorporated forms, essentially compounds).

Returning to MWF languages, it turns out that all MWF languages have exactly the sub-wh type of indeterminates, which leads me to posit (\ref{ex:bosk:8}) (note that this is a one-way correlation).
 
\ea\label{ex:bosk:8} If a language has multiple wh-fronting, it has a sub-wh indeterminate system.\z

\noindent This was illustrated above with SC in \REF{ex:bosk:4}. Additional confirmations of \REF{ex:bosk:8} are provided by the MWF languages in Tables \ref{bosk:tab:1}--\ref{bosk:tab:4} (the data in Tables \ref{bosk:tab:1}--\ref{bosk:tab:7} are from, or based on, \citealt{Haspelmath1997}; only partial paradigms are given below, and not all series are illustrated -- all these languages have additional series; for more complete paradigms, see \citealt{Haspelmath1997}).\footnote{I do not consider German as having a productive sub-wh system since in German only one series, the \textit{irgend}-series (but not the \textit{etwas}- or \textit{n}-series, which are the respective second and fourth examples in \REF{ex:bosk:germ}), is related to wh-words, as shown by \REF{ex:bosk:germ} (data from \citealt{Haspelmath1997}; note, however, that \REF{ex:bosk:8} is a \textit{one-way} correlation).

\ea\label{ex:bosk:germ} 
\ea 
%(i)               interrogative  etwas{}-series  irgend{}-series        n{}-series
\textit{person}: wer, jemand, irgend-wer / irgend-jemand, niemand
\ex \textit{thing}: was, etwas, irgend-was / irgend-etwas, nichts
\ex \textit{place}: wo, --, irgend-wo, nirgends
\ex \textit{time}: wann, --, irgend-wann, nie
\ex \textit{manner}: wie, --, irgend-wie, (auf keine Weise)
\ex \textit{determiner}: welche,  (ein), irgend-ein / irgend-welche, kein
\z
\z

}

\begin{table}
\begin{tabularx}{.90\textwidth}{Xllll}
\lsptoprule
& interrogative & existential & neg-concord & free choice\\\midrule
person & kto & kto-to & ni-kto & kto ugodno\\
thing & čto & čto-to & ni-čto & čto ugodno\\
place & gde & gde-to & ni-gde & gde ugodno\\
time & kogda & kogda-to & ni-kogda & kogda ugodno\\
manner & kak & kak-to & ni-kak & kak ugodno\\
\lspbottomrule
\end{tabularx}
    \caption{Russian indeterminate series}
    \label{bosk:tab:1}
\end{table}
 
\begin{table}
\begin{tabularx}{.90\textwidth}{Xllll}
\lsptoprule
& interrogative & existential & neg-concord & {free choice}\\\midrule
person & koj & nja-koj & ni-koj & koj to i da e\\
thing & što & ne-što & ni-što & što to i da e\\
place & kâde & nja-kâde & ni-kâde & kâde to i da e\\
time & koga & nja-koga & ni-koga & koga to i da e\\
manner & kak & nja-kak & ni-kak & kak to i da e\\
\lspbottomrule
\end{tabularx}
    \caption{Bulgarian indeterminate series}
\label{bosk:tab:2}
\end{table}
 
\begin{table}
\begin{tabularx}{.95\textwidth}{Xllll}
\lsptoprule
& interrogative & existential & neg-concord & {free choice}\\\midrule
person & ki & vala-ki  & sen-ki & akár-ki\\
thing & mi   & vala-mi  & sem-mi & akár-mi\\
place& hol    & vala-hol  & se-hol & akár-hol\\
time & mikor  & vala-mikor & sem-mikor& akár-mikor\\
manner& hogy(an)& vala-hogy(an) & se-hogy(an) & akár-hogy(an)\\
\lspbottomrule
\end{tabularx}
    \caption{Hungarian indeterminate series}
\label{bosk:tab:3}
\end{table}

\begin{table}
\begin{tabularx}{\textwidth}{Xlllll}
\lsptoprule
& interro- & \textit{bait}-series & \textit{i}-series & \textit{edo}-series & \textit{nahi}-series\\
& gative& \begin{footnotesize}(non-emphatic)\end{footnotesize} & \begin{footnotesize}(NPI)\end{footnotesize} & \begin{footnotesize}(free choice)\end{footnotesize} & \begin{footnotesize}(free choice)\end{footnotesize}\\\midrule
person & nor & nor-bait & i-nor & edo-nor & nor-nahi\\
thing & zer & zer-bait & e-zer & edo-zer & zer-nahi\\
place & non & non-bait & i-non & edo-non & non-nahi\\
time & noiz & noiz-bait & i-noiz & edo-noiz & noiz-nahi\\
manner & nola & nola-bait & i-nola & edo-nola & nola-nahi\\
determiner & zein & -- & -- & edo-zein & zein-nahi\\
\lspbottomrule
\end{tabularx}
    \caption{Basque indeterminate series}
\label{bosk:tab:4}
\end{table}

Particularly interesting for our purposes is Romance. Latin was clearly a MWF language (see \citealt{Ledgeway2012} and \citealt{Dadan2019} for extensive discussion) and had a fully productive sub-wh system. The fully productive sub-wh system got lost in all modern Romance languages except one: Romanian, which is the only modern Romance language that still has MWF, a strong confirmation of \REF{ex:bosk:8}. A partial illustration of the Romance situation is given in Tables \ref{bosk:tab:5}--\ref{bosk:tab:7}.


\begin{table}
\begin{tabularx}{.85\textwidth}{Xllll}
\lsptoprule
& interrogative & existential & polarity & {free choice}\\\midrule
person    & quis & ali-quis  & quis-quam & qui-vis\\
thing & quid & ali-quid  & quid-quam & quid-vis\\
place & ubi  &  ali-cubi & usquam &  ubi-vis\\
time & quando  & ali-quando & umquam & --\\
\lspbottomrule
\end{tabularx}
    \caption{Latin indeterminate series}
\label{bosk:tab:5}
\end{table}


\begin{table}
\begin{tabularx}{.90\textwidth}{Xlll}
\lsptoprule
& interrogative & existential & neg-concord \\\midrule
person & chi & qualcuno & nessuno\\
thing & che  & qualche cosa, qualcosa & niente, nulla\\
place & dove & in qualche luogo  & in nessun luogo\\
time  & quando  & qualque volta  & (mai)\\
\lspbottomrule
\end{tabularx}
    \caption{Italian}
\label{bosk:tab:6}
\end{table}

\begin{table}
\begin{tabularx}{.85\textwidth}{Xllll}
\lsptoprule
& interrogative & existential & {free choice} & \textit{oare}-series\\\midrule
person & cine & cine-va & ori-cine & oare-cine\\
thing & ce & ce-va & ori-ce  & oare-ce\\
place & unde & unde-va & ori-unde  & oare-unde\\
time & cînd & cînd-va & ori-cînd  & oare-cînd\\
\lspbottomrule
\end{tabularx}
    \caption{Romanian indeterminate series}
\label{bosk:tab:7}
\end{table}

I conclude therefore that a sub-wh system is a prerequisite for MWF. I will now briefly discuss why that is the case.

The crucial point is that \textit{ko} in \REF{ex:bosk:4a} is actually not `who', i.e. it does not correspond to English \textit{who}. The form is a true indeterminate, which means that it does not have an inherent quantificational force (see below for evidence to this effect). It requires licensing, which also determines its quantificational force (i.e. its exact meaning).

The particles that indeterminates merge with normally do that -- they determine the exact quantificational force, and the meaning of the indeterminate in cases like those given in SC \REF{ex:bosk:16} as a partial illustration of the relevant SC paradigm.\footnote{In these particular cases, the morphology is rather transparent. \textit{I-} also means `even'. On the connection between `even' and NPIs, see e.g. \citet{bos:Rooth1985}, \citet{Haspelmath1997}, \citet{Giannakidou2007}, \citet{Crnič2011}; \textit{n-} may indicate a connection with negation. At any rate, these details are not important for our purposes.}

\ea\label{ex:bosk:16}
\ea i$+$\textit{ko} `anyone'\hfill (SC)
\ex n$+$i$+$\textit{ko} `no one'
\z 
\z 

\noindent Importantly, in a sub-wh system, the \textit{only} usage on which the indeterminate is not merged with a particle is the wh-usage, which means that we are dealing here with an unlicensed indeterminate. I suggest that this is what requires fronting. The indeterminate is licensed as a wh-phrase by moving to an interrogative projection (which determines its meaning). The movement thus does not occur because of a property of the interrogative head (which is the case in English, where only one wh-phrase fronts because of that), but because of indeterminate licensing -- this is why they \textit{all} need to undergo fronting, resulting in MWF.

In short, in the sub-wh system, affixes merged with an indeterminate determine its quantificational force and license the indeterminate. When there is no such affix, the indeterminate is licensed as a wh-phrase by movement to an interrogative projection.

MWF languages do however have certain cases where the wh-phrase itself (so the form that is used in wh-questions) receives a different, non-wh interpretation, like the wh-existential in \REF{ex:bosk:17} (see e.g. \citealt{Izvorski1998}, \citealt{Bošković2002}, \citealt{Šimík2011}).

 
\ea\label{ex:bosk:17}
\ea[] 
{\gll Ima ko   šta      da    ti    proda.\\
has who what  that you sells\\\hfill (SC)
\glt `There is someone who can sell you something.'}
\ex[*]
{\gll Ima ko    da    ti    proda šta.\\
has  who that you sells  what\\}
\z 
\z 

\noindent Importantly, the relevant elements must front here. The fronting does not occur to the interrogative projection, since the relevant clause is simply not interrogative. I suggest that since \textit{ko} and \textit{šta} are not merged with an indefinite particle in these cases, they are licensed as indefinites by moving to a special indefinite licensing position. What is relevant here is languages like Kaqchikel, where the exact same form functions as interrogative or indefinite, and must be fronted on both functions, with the landing site of the interrogative being higher than the indefinite licensing projection, as discussed in detail in \citet{Erlewine2016}. What Kaqchikel shows is that there is a pattern where the indefinite meaning of an indeterminate is licensed by movement to a special projection that licenses this meaning (see \citealt{Erlewine2016}). The suggestion is that this is precisely what happens in \REF{ex:bosk:17} (the movement is not to the interrogative CP projection since the relevant clauses are clearly not interrogative; note that this (i.e. \REF{ex:bosk:17}) can also be taken to confirm that the relevant elements are not inherently wh-phrases but bare indeterminates).\footnote{The movement strategy just discussed and the affixation strategy for licensing indefinites can be combined, though this option is slightly disfavored, possibly due to a parallelism for indefinite licensing being favored.

\ea[?]
{\gll Ima ko   da    ti     proda nešto.\\
has who  that you sells    something\\\hfill (SC)
\glt `There is someone who can sell you something.'}
\z 

}

It is worth noting that a number of Australian languages have the same form for wh-phrases and indefinites but while the morphology is the same the syntax is not: as wh-phrases they must front, as indefinites they stay in situ (these languages cannot be checked for MWF since they do not allow multiple questions in the first place, see \citealt{Cheng1991} for relevant discussion of these languages).

 
\ea Martuthunira\label{ex:bosk:18}
\ea \label{ex:bosk:18a}
\gll ngana nganhu       wartirra nyina-nguru  karra-ngka    muyinu-npi-rra?\\                
who   that.\textsc{nom}  woman  sit-\textsc{prs}       scrub-\textsc{loc}   hidden-\textsc{inch}-\textsc{ctemp}\\%\hfill (Martuthunira)
\glt `Who is that woman hiding in the scrub?'
\ex \label{ex:bosk:18b}
\gll ngayu nyina-lha martama-l.yarra palykura-la nganangu-la.\\
\textsc{1sg.nom} sit-\textsc{prs} press.on-\textsc{ctemp} groundsheet-\textsc{loc} someone.\textsc{gen}-\textsc{loc}\\
\glt `I sat down on someone's groundsheet, holding it down.'\hfill (\citealt{Dench1987})
\z 
\z 

\ea Panyjima\label{ex:bosk:19}
\ea \label{ex:bosk:19a}
\gll ngatha       ngananhalu            nhantha-nnguli-nha.\\
\textsc{1sg.nom} something.\textsc{ins} bit-\textsc{pass}-\textsc{pst}\\%\hfill (Panyjima)
\glt `I was bitten by something.'
\ex \label{ex:bosk:19b}
\gll ngananha-ma-rna   nyinta         ngunhalku?\\  what-\textsc{caus}-\textsc{pst}  \textsc{2sg.nom}  that.\textsc{acc}\\
\glt  `What have you done to him?'          \hfill (\citealt{Dench1981})
\z 
\z 

\noindent There is a parallel situation with MWF languages. In particular, there are similar wh-indefinites in Slavic MWF languages, as illustrated by Russian \REF{ex:bosk:20} (see e.g. \citealt{Zanon2022}, \citealt{HengeveldRoelofsen2023}).

 
\ea \label{ex:bosk:20}
\gll Možet, kto   prixodil.\\
maybe who  came  \\\hfill (Russian)
\glt        `Maybe someone came.' \hfill (\citealt{HengeveldRoelofsen2023})                  \z 

\noindent This usage is very restricted in Slavic; in SC even more so than in Russian -- \REF{ex:bosk:20} is in fact unacceptable in SC; regarding Russian, see especially \citet{Zanon2022}, who argues that the relevant elements are licensed by a semantically motivated and constrained null operator, which essentially plays the role of the licensing affixes discussed above hence this kind of analysis of the usage in question can be adjusted to the system developed here. Alternatively, it is possible that an indeterminate that does not have a licensing particle attached and does not move to an indeterminate-licensing projection or has a linking index (see the discussion right below) is interpreted by a default rule for unlicensed indeterminates, which would apply in the relevant contexts in the languages that allow this usage (they also differ regarding such contexts), as a simple indefinite. In this respect, it is worth noting that such indefinites cannot occur in wh-questions (see e.g. \citealt{Zanon2022} and \citealt{HengeveldRoelofsen2023}), which can be taken to confirm the default nature of the licensing in question -- it is available only if another way is not available.\footnote{Note that these indefinites are different from those in wh-existentials like \REF{ex:bosk:17} -- e.g. Ksenia Zanon (p.c.) notes that the former cannot be coordinated, see \citet{Zanon2022}, while the latter can be.} 

Interestingly, \citet{Zanon2022} and \citet{HengeveldRoelofsen2023} observe that these wh-indefinites cannot be focused. What is important here is that real MWF/wh-fronting in Slavic has been analyzed as focus-movement (e.g. \citealt{Bošković2002}, see also \citealt{Stepanov1998} for Russian as well as the discussion below), i.e. it is essentially focusing. It then makes sense that if the relevant element is focused it would be interpreted as a wh, not a non-wh (i.e. indefinite), hence the non-wh-indefinite usage does not allow focalization. \citet{HengeveldRoelofsen2023} actually observe that the non-focusing requirement is not general -- it does not hold in Dutch. Given the current discussion, Dutch-like exceptions should not be possible in MWF languages.

A different (and independent) exception to MWF concerns D-linked wh-phrases, which need not undergo fronting, as illustrated below by SC \REF{ex:bosk:21}. (Note, however, that this is not the case in all MWF languages; they must front in Hungarian, which is discussed in \sectref{sec:bosk:2}.)\footnote{It
    may be worth noting here that D-linked wh-phrases more generally can be special, and subject to ill-understood language variation. Thus, there are languages that disallow multiple questions, e.g. Hong Kong Sign Language (HKSL), Italian, and Mandinka. (\REF{ex:bosk:hksla}--\REF{ex:bosk:itb} are taken from \citealt{Gan2022}).

    \ea[*]{WHO BUY WHAT?           \hfill  (HKSL)}\label{ex:bosk:hksla}
    \z

    \ea[*]
    {\gll Chi   ha  scritto  che cosa?\\
    who has written what\\         \hfill  (Italian)}\label{ex:bosk:ita}
    \z

    \noindent \citet{Gan2022} shows that D-linking improves multiple questions in HKSL and Mandinka, but not in Italian (it is not out of question that there is some connection here with the SC vs. Hungarian difference regarding D-linked MWF questions).

    \ea
    STUDENT WHO BUY COMPUTER BUY-WHICH? \hfill (HKSL)\\
    `Which student bought which book?'\label{ex:bosk:hkslb}
    \z

    \ea[*]
    {\gll Quale studente comprerà quale  libro? \\
    which student  will-buy   which book\\
    \hfill (Italian)}\label{ex:bosk:itb}
    \z\medskip
}

 
\ea\label{ex:bosk:21}
\gll Ko    kupuje     koju   knjigu?\\
who buys which book\\ \hfill (SC)\\
\glt `Who is buying which book?'
\z 

\noindent Two issues are relevant here. First, \textit{koju} is not an indeterminate but a wh-specific form (this may not be a general situation though). Second, as briefly noted above, \citet{Bošković2002} argues that MWF is actually movement to a focus projection, this means that the relevant licensing takes place in the Spec of a focus-licensing head; this by itself is not surprising -- focus/interrogativity connection has often been noted.\footnote{I assume that as a result of this connection, indeterminates can still be licensed as interrogative in such a projection. (Possibly, being in such a projection would enable them to undergo unselective binding with interrogative C in spite of the issue noted in \sectref{sec:bosk:2.1} (i.e. without a null operator, the intuition being that it is not needed in this case since the relevant element is located in an operator, in fact the right operator, position—SpecFocP), which would license their interrogative interpretation.)} Furthermore, \citet{Bošković2002} observes that D-linking is very different from focus. With D-linked wh-phrases the range of felicitous answers is restricted by a set of objects that is familiar to the speaker and the hearer as a result of it being referred to/salient in the context. In other words, the range of reference of D-linked wh-phrases is discourse-given. Due to their discourse givenness, such wh-phrases are not focused, hence they are not subject to focus movement. (One wh-phrase always must front for clausal typing as discussed in \citealt{Cheng1991} so when only a D-linked wh-phrase is present it fronts but \citealt{Bošković2002} shows that the landing site is different; for special behavior of D-linked wh-phrases regarding MWF see also \citealt{Diesing2003} on Yiddish, which disallows MWF with D-linked wh-phrases.)

Regarding the interpretation of D-linked wh-phrases, \citet{Enç2003} proposes that specific arguments have a linking index $\ell $ which identifies the set of individuals of which the argument is a member (i.e. it gives the set which that argument must belong to). Non-specific arguments have no such index. \citet{Shields2008} extends this to wh-phrases: D-linked wh-phrases are specific and therefore have a set-denoting (linking) index, which non-D-linking wh-phrases do not have. The linking index points to the set of entities in the discourse that a specific expression is required to be a member of.

%\ea\label{ex:bosk:22}
%which book: WH $x$ [$x$ a thing] [$x$ a member of $\ell $]
%\z 

%\noindent 
Indeterminate pronouns are normally non-specific, D-linked ones (i.e. D-linked wh-phrases) are not. The interpretation of the latter is essentially determined by their semantics, no further licensing is needed (essentially, an indeterminate with a linking index is interpreted as D-linked -- the linking index points to the set of entities in the discourse that the relevant element is required to be a member of). It is also possible that the linking index allows D-linked wh-phrases to undergo unselective binding by interrogative C and that they are licensed in that way (see \citealt{Pesetsky1987} on unselective binding of D-linked wh-phrases; see also \sectref{sec:bosk:2}).\footnote{There is an alternative account. A number of authors (e.g. \citealt{bos:Belletti2004}, \citealt{Lacerda2020}) have argued for several languages that they have a low topic projection. It is possible that D-linked wh-phrases are licensed in a low topic-like projection (see \citealt{Grohmann2006} for D-linking as topichood). On this analysis, the D-linked wh-phrase in \REF{ex:bosk:21} would not actually be in situ (SC and Hungarian could then differ here regarding topic movement; see, however, below). It is worth noting here that \REF{ex:bosk:fnSC} is also acceptable. \citet{Bošković2002}, however, shows that the D-linked wh-phrase in such cases is lower than the second wh-phrase in examples like \REF{ex:bosk:3a}, i.e. it is not the case that the D-linked wh-phrase simply optionally undergoes movement that the second wh-phrase must undergo in \REF{ex:bosk:3} (examples like \REF{ex:bosk:fnSC}, i.e. optional fronting, is actually not allowed in all MWF languages, see \citealt{Bošković2002}, \citealt{Pesetsky1987}, \citealt{Wachowicz1974}).

\ea \label{ex:bosk:fnSC}
\gll Ko    koju   knjigu kupuje?\\     
who which book   buys\\ \hfill          (SC)
\glt `Who is buying which book?'
\z

}

\largerpage
In conclusion, this section has established a correlation between MWF and another phenomenon. In particular, MWF languages have been shown to have a sub-wh indeterminate systems, which forces MWF (except with D-linked wh-phrases).

\section{Multiple wh-fronting and articles}\label{sec:bosk:2}

\subsection{Another generalization}\label{sec:bosk:2.1}

I will now show that there is another property that MWF languages have in common, which is in principle independent of the one presented in \sectref{sec:bosk:1} (in the sense that if one of the generalizations in questions turns out not to be correct the other one would not necessarily be affected).\footnote{But see the generalization regarding indeterminates themselves in \citet{Oda2022} that would actually relate \REF{ex:bosk:8} and \REF{ex:bosk:23}. Oda also provides an alternative deduction of \REF{ex:bosk:8} based on my earlier version of this generalization given in \citet{Bošković2020} where the prerequisite for MWF was a broader indeterminate system than the sub wh-system.} In particular, they all either lack definite articles or have affixal definite articles \REF{ex:bosk:23}. The relevant language cut is given in \REF{ex:bosk:24}.\footnote{\label{fnt:ftn11}For most of the languages listed in \REF{ex:bosk:24b}, their affixal status is well-known. For arguments that Hungarian definite article is affixal (more precisely, a prefix), see \citet{MacWhinney1976}, \citet{Oda2022}, and \citet{Lewis2024}. \citeauthor{MacWhinney1976} observes that it undergoes a morphophonemic alternation that is typical of affixes, while \citeauthor{Oda2022} and \citeauthor{Lewis2024} observe typological generalizations where Hungarian patterns with languages with affixal articles (languages with affixal articles actually pattern with languages without articles regarding those generalizations). Regarding the affixal status of the definite article in Yiddish, which might be the least discussed case here, see \citet{Oda2022}. To mention some relevant arguments, \citet{Talić2017} and \citet{Oda2022} observe that languages with affixal definite articles allow article omission in contexts where such omission is not possible in free-standing article languages like English. \citeauthor{Oda2022} notes that this is especially the case in PPs, where due to article omission a bare noun can even receive a definite interpretation in (some) affixal article languages, which is never possible in languages with non-affixal definite articles, where a definite article is required for definite interpretation (see \citealt{Bošković2016}; \citeauthor{Oda2022} argues that in the relevant cases the preposition essentially functions as the definite article). Thus, \citet[119]{Zwicky1984} observes regarding \REF{ex:bosk:yidb}: ``The phrase \textit{in gloz} in `in the glass' is a typical example. The noun \textit{gloz} in this expression is understood definitely, and can even be anaphoric.''

\ea \label{ex:bosk:yid} 
\ea \label{ex:bosk:yida}
\gll lebn tir\\
near door\\\hfill (Yiddish)
\glt `near the door'
\ex in gloz $=$ in the glass\label{ex:bosk:yidb} \hfill (\citealt{Zwicky1984})        
\z 
\z 

\noindent \citet{Bošković2016} also notes that, for the purposes of Bošković's NP/DP generalizations (see below for some relevant discussion), definite articles have a form distinct from demonstratives. Definite articles in Yiddish have the same form as demonstratives, with stress distinguishing them. \citet[122]{Margolis2011} in fact states that: “this/these” is identical to the definite article with added stress. Essentially following \citet{Oda2022}, I thus consider Yiddish to be an affixal article language, the definite article being an affixal, hence unstressed, version of the demonstrative (there may be a change under way regarding the status of the relevant element where dialectal differences may also be relevant; not all dialects of Yiddish in fact have MWF, see \citealt{Diesing2003}).}

 
\ea\label{ex:bosk:23}
MWF languages either lack articles or have affixal definite articles.
\z 

\ea\label{ex:bosk:24}
\ea No articles: SC, Polish, Russian, Czech, Slovenian, Ukrainian, Mohawk, Latin, Georgian, Lithuanian, Ossetic, Swan\label{ex:bosk:24a}
\ex Affixal articles: Romanian, Bulgarian, Macedonian, Basque, Hungarian, Yiddish\label{ex:bosk:24b}
\z 
\z 

\noindent Turning to the deduction of \REF{ex:bosk:23}, in a series of works (e.g. \citealt{Bošković2012}), based on a number of syntactic and semantic typological generalizations, where languages with and without definite articles consistently differ regarding a number of syntactic and semantic phenomena, I argued that languages without definite articles do not project DP (i.e., there are no null definite articles in such languages).

\citet{Talić2017} argues for a refinement of the NP/DP language distinction; she shows that in many respects languages with affixal definite articles behave like a separate type (see also \citealt{Oda2022}, \citealt{Lewis2024}), in that they sometimes behave like languages with articles and sometimes like those without articles.\footnote{Below, for ease of exposition I will simply use the term (affixal) article, though what matters here (and what matters for Bošković's NP/DP generalizations) is definite articles only.}

In \citet{Bošković2020} I suggested an implementation of this observation for the affixal article languages that have MWF: there is D in such languages, but there is no DP. The affixal article is base-generated adjoined to N (more precisely, its host). It should be noted that there is nothing strange about this theoretically: Adjunction through movement can involve either phrasal or head adjunction, the same should hold for adjunction through base-generation (for much relevant discussion regarding definite articles, see also \citealt{Oda2022}; regarding indefinite articles, see \citealt{Wang2019}).


\begin{figure}
\begin{forest}
[book
[book
]
[the
]
]
\end{forest}
    \caption{Noun-article base-generation}
    \label{fig:bosk:1}
\end{figure}

Recall now that in a sub wh-system, only on the wh-usage the indeterminate does not occur with a licensing particle. I suggest then that, in principle, such indeterminates can still be licensed at a distance \textit{in situ}, with a null operator in SpecDP that is unselectively bound by interrogative C. This is not possible in MWF languages due to the lack of a DP projection that would be capable of such licensing. The only way to license the indeterminate on the wh-usage is then to front it to an interrogative position.\footnote{As suggested above, D-linked wh-phrases may be able to undergo unselective binding even in the absence of DP for independent reasons, namely, due to the presence of the linking index.}

A confluence of independent factors, namely the sub wh-system and a particular status regarding articles, is what is behind MWF: MWF languages have a sub-wh indeterminate system, and either lack articles or have affixal articles, which are the typological findings of this paper.

Regarding the relevance of the latter property, in languages without articles and languages with affixal articles the possibility of wh-licensing \textit{in situ} by interrogative C through unselective binding is blocked because such licensing is done through a null operator in SpecDP (except with D-linked wh-phrases), which is absent in languages without articles and languages with affixal articles (in the former, because DP itself is lacking, and in the latter because the affixal article is base-generated adjoined to N, which means that in such languages there is D, but there is still no DP, hence no null operator in SpecDP).

\subsection{Superiority variation regarding basic Superiority effects} \label{sec:bosk:2.2}

I turn now to a case of variation within MWF languages which will also shed light on the exceptional status of Hungarian regarding D-linked wh-phrases, noted in \sectref{sec:bosk:1}. Already \citet{bos:Rudin1988} observed that MWF languages differ regarding whether they show ordering, i.e. Superiority, effects with MWF. Regarding basic cases like those shown in \REF{ex:bosk:26BG}--\REF{ex:bosk:26SC}, SC does not show them, while Bulgarian does show them.

 
\ea\label{ex:bosk:26BG}
\ea\label{ex:bosk:26BGa} 
\gll Koj  kakvo  e kupil?\\
    who  what   is bought\\\hfill (Bulgarian)
\ex[*] 
{\gll Kakvo koj e kupil?\\
what who is bought\\}\label{ex:bosk:26BGb}
\z 
(Intended:) `Who bought what?'            
\z 

\ea\label{ex:bosk:26SC}
\ea 
\gll Ko  šta    kupuje?\\
who what buys\\\hfill (SC)\label{ex:bosk:26SCa} 
\ex 
\gll Šta ko kupuje?\\
what who buys\\\label{ex:bosk:26SCb}
\z 
`Who is buying what?'
\z 

\noindent A survey of the literature shows the following language cut regarding Superiority effects in basic cases of this sort.

 
\ea\label{ex:bosk:27}
\ea No Superiority effects: SC, Polish, Czech, Russian, Slovenian, Ukrainian, Mohawk, Lithuanian, Georgian, Ossetic, Svan, Hungarian
\ex Superiority effects: Romanian, Bulgarian, Macedonian, Basque, Yiddish
\z 
\z

\noindent It turns out that the cut is not arbitrary -- there is a correlation with (the type of) articles. Putting Hungarian aside (taking Hungarian into consideration we would have a one-way correlation in \REF{ex:bosk:29}, which was actually noted in \citealt{Bošković2008}), we have \REF{ex:bosk:28}.

 
\ea\label{ex:bosk:28}
MWF languages without articles do not show basic Superiority effects, those with affixal articles do.
\z 


\ea\label{ex:bosk:29}
MWF languages without articles do not show basic Superiority effects.
\z 

\noindent Below, I will briefly outline a deduction of \REF{ex:bosk:28} that will also accommodate the Hungarian exception (given the affixal status of the Hungarian definite article, see fn. \ref{fnt:ftn11}), tying it to another Hungarian exception, namely the exceptional behavior of Hungarian regarding D-linking.

\citet{Bošković2002} argues that Superiority effects arise with MWF to SpecCP (English-style wh-movement), not with MWF to a lower position, which means that SC MWF targets a lower position than Bulgarian MWF (see \citealt{Bošković2002} for evidence to this effect). Now, if Superiority is taken to be a sign of true, English-style wh-movement, this can be generalized in such a way that languages with articles (non-affixal or affixal) must have true English-style wh-movement to SpecCP when fronting wh-phrases. \citet{Bošković2008} in fact suggests that the D-feature is crucially involved in movement to SpecCP. Affixal article languages still have the D-feature, which means that they have wh-movement to SpecCP, which is Superiority inducing. This then captures \REF{ex:bosk:28}. But what about Hungarian? 

Superiority as a test for wh-movement is confirmed by single-pair (SP)/pair-list (PL) answers. \citet{Bošković2003,Bošković2002} shows that overt wh-movement languages require a PL answer for examples like \REF{ex:bosk:30}. \REF{ex:bosk:30} cannot be felicitously asked in the following situation: John is in a store and sees somebody buying an article of clothing, but does not see who it is and does not see exactly what the person is buying. He goes to the sales clerk and asks \REF{ex:bosk:30}.

 
\ea\label{ex:bosk:30}
Who bought what?
\z 

\noindent Whereas German patterns with English, wh-in-situ languages Japanese, Hindi, and Chinese allow SP answers in such questions (see \citealt{Bošković2003}). Importantly, French allows SP answers, but only with in-situ questions like \REF{ex:bosk:31a}, not \REF{ex:bosk:31b}.

 
\ea\label{ex:bosk:31}
\ea\label{ex:bosk:31a} 
\gll Il   a    donné quoi  à  qui?\\
he has given  what to who\\\hfill (French)
\ex\label{ex:bosk:31b} 
\gll Qu' a-t-il    donné à qui?\\
what has-he given   to who\\
\z 
`What did he give to who?'
\z 

\noindent Based on this, \citet{Bošković2003,Bošković2002} argues that the availability of SP answers depends on the possibility of not moving any wh-phrase to SpecCP overtly (see \citealt{Bošković2003} for an account of this generalization).

Turning to MWF languages, SC allows SP answers, while Bulgarian does not, which confirms that SC MWF lands in a lower position than Bulgarian MWF (see \citealt{Bošković2007} and references therein for additional languages confirming this).

As noted above, \citet{Bošković2002} argues that MWF involves focus. Now, \citet{Bošković1999} argues that movement-attracting heads can differ regarding the specification of the movement-attracting feature. They can be specified to attract one element with the relevant feature, call it F, or all elements with the F feature. English interrogative C is an attract 1-F head -- it attracts one (in particular, the highest) element with the wh-feature. In SC, wh-phrases undergo focus movement; the relevant head has the specification Attract All-focus. Bulgarian is a combination of English and SC: It has single-fronting wh-movement as in English (Attract 1-wh) and MWF for focus (Attract All-focus, see \citealt{Bošković1999} and fn. \ref{fnt:ftn15}). Importantly, from this perspective, Superiority is not a diagnostic of wh-movement, but single fronting.\footnote{\label{fnt:ftn14}Given the economy-of-derivation condition that every requirement be satisfied through the shortest movement possible, Attract 1-F heads will always attract the highest phrase with the relevant feature: thus, in \REF{ex:bosk:fnMary}, the relevant formal inadequacy of the interrogative C is checked through a shorter movement in \REF{ex:bosk:fnMarya} than in \REF{ex:bosk:fnMaryb} (cf. the pre-wh-movement structure in \REF{ex:bosk:fnMaryc}).

\ea\label{ex:bosk:fnMary} 
\ea Who$_i$ did Mary tell t$_i$ to buy the book? \label{ex:bosk:fnMarya}    
\ex[*] {What$_i$ did Mary tell who to buy t$_i$?}     \label{ex:bosk:fnMaryb}
\ex Mary tell who to buy what\label{ex:bosk:fnMaryc}
\z 
\z 

\noindent With Attract All-F heads, like the SC focus-licensing head, all relevant elements must move: Regardless of the order of movement, the same number of nodes are crossed with such movement, hence the order of movement of wh-phrases is free (see \citealt{Bošković1999} for a more detailed discussion).} In this respect, \citet{Bošković2002} shows that there are selective Superiority effects in Bulgarian. Only the first wh-phrase, which is the only wh-phrase that undergoes wh-movement, is subject to Superiority effects, other wh-phrases are not. Thus, the indirect object wh-phrase must precede the direct object wh-phrase in \REF{ex:bosk:32A} (because it is higher than the object wh-phrase before wh-fronting) but not in \REF{ex:bosk:32B}, where a subject wh-phrase, which is higher than both indirect and direct object wh-phrase before wh-fronting, is present.\footnote{\label{fnt:ftn15}Note that, as discussed in \citet{Bošković1999}, it is the same head, interrogative C, that has the relevant properties (Attract 1-wh, Attract All-focus) in Bulgarian. Given that the first wh-phrase that moves to SpecCP automatically satisfies the Attract 1-wh requirement (see \citealt{Bošković1999}), the highest wh-phrase must move first, then the order of movement does not matter, since Attract All-focus does not care about the order of movement, as noted in fn. \ref{fnt:ftn14}. (Note that, as standardly assumed, the order of fronted wh-phrases reflects the order of their movement, see \citet{bos:Rudin1988}, \citet{Richards2001} for different implementations of this, i.e. the wh-phrase that is first in the linear order is the one that moves first, hence the highest wh-phrase must move first when Superiority is in effect.)}

 
\ea\label{ex:bosk:32A}
\ea \label{ex:bosk:32Aa}
\gll Kogo  kakvo e  pital   Ivan? \\
whom what  is asked  Ivan\\\hfill (Bulgarian)
\ex[?*]
{\gll Kakvo kogo e pital Ivan?\\
what whom is asked  Ivan\\}\label{ex:bosk:32Ab}
\z
`Who did Ivan ask what?'
\z 

\ea\label{ex:bosk:32B}
\ea \label{ex:bosk:32Ba}
\gll Koj kogo kakvo e  pital?\\
who whom  what   is asked\\\hfill (Bulgarian)
\ex \label{ex:bosk:32Bb}
\gll Koj kakvo kogo e pital?\\
who what whom is asked\\
\z 
`Who asked who what?'
\z 

\noindent All this raises a question: Is there a MWF language where D-linked wh-phrases also must front? That would be a true MWF counterpart of English, with an Attract All-wh specification (note that Attract All-wh affects D-linked wh-phrases, in contrast to Attract All-focus). As noted above, and as discussed in \citet{Bošković2007} and \citet{É.Kiss2002}, both D-linked and non-D-linked wh-phrases must move in Hungarian. This is illustrated by \REF{ex:bosk:33A}--\REF{ex:bosk:33B}.

\ea\label{ex:bosk:33A}
\ea[*]
{\gll Ki    irt      mit?\\
who wrote what\\\hfill (Hungarian)}
\ex[] 
{\gll Ki   mit    irt?\\
who what wrote\\}
\ex[] 
{\gll  Mit  ki     irt?\\
what who wrote\\}
\z 
(Intended:) `Who wrote what?' \hfill (\citealt{Bošković2007})
\z 

\ea\label{ex:bosk:33B}
\ea[*]
{\gll Ki    irta    melyik levelet?\\
who wrote which   letter \\\hfill (Hungarian)}
\ex[]
{\gll Ki   melyik levelet irta?\\
 who which  letter   wrote\\}
\ex[] 
{\gll  Melyik levelet ki     irta?\\
which   letter   who wrote  \\}
\z 
(Intended:) `Who wrote which letter?' \hfill (\citealt{Bošković2007})
\z 

\noindent Importantly, Hungarian MWF questions also disallow SP answers (see e.g. \citealt{Surányi2005}) and do not show Superiority effects (see \REF{ex:bosk:33A}), which is exactly the behavior expected of a true MWF counterpart of English (there are no Superiority effects since we are dealing only with an Attract-All fronting and SP answers are disallowed because the fronting is to SpecCP).\footnote{\citet{Horváth1998}, \citet{Puskás2000}, \citet{Lipták2001}, and \citet{É.Kiss2002} suggest that the wh-phrase that is closest to the verb in Hungarian MWF questions undergoes focus-movement, other wh-phrases undergo movement that non-wh-quantifiers undergo, but see \citet{Surányi2005} for arguments against this position.} What appeared to be an exceptional behavior of Hungarian regarding Superiority and D-linking is thus explained, in fact in a uniform manner.

At any rate, the discussion from \sectref{sec:bosk:2} is summarized below in table form (where the left column %first line 
gives the relevant language types -- there are two types for affixal article languages, depending on whether D-linked wh-phrases are also subject to MWF).

%\begin{table}
%\begin{tabularx}{\textwidth}{Xlll}
%\lsptoprule
%& MWF & Superiority with MWF & SP with wh-fronting\\\midrule
%Free-standing article & * & N/A & *\\\hline
%Affixal article & & Yes & *\\\hline
%Affixal article $+$ & & No & *\\
%{} {} D-linking MWF &&&\\\hline
%No article && No &\\
%\lspbottomrule
%\end{tabularx}
%    \caption{Summary}
%\label{bosk:tab:final}
%\end{table}

\begin{table}
\fittable{
\begin{tabular}{llll}
\lsptoprule
& MWF & Superiority with MWF & SP with wh-fronting\\\midrule
Free-standing article & * & N/A & *\\
Affixal article & & Yes & *\\
Affixal article $+$ & & No & *\\
{} {} D-linking MWF &&&\\
No article && No &\\
\lspbottomrule
\end{tabular}
}
    \caption{Summary}
\label{bosk:tab:final}
\end{table}

\section{Conclusion}\label{sec:bosk:3}

The paper has established correlations between MWF and other phenomena, in an attempt to understand what is behind MWF. In particular, MWF languages have been shown to have a sub-wh indeterminate system, which was suggested to force MWF. In such a system, an inseparable affix is attached to the indeterminate, with the exact quantificational force of the indeterminate determined by the affix that merges with it. What is traditionally considered to be wh-phrases in the sub-wh indeterminate system are not really wh-phrases but bare indeterminates; they are not licensed \textit{in situ} because they are bare -- no licensing affix is attached to them -- hence they must front to a position in the left periphery to get licensed, which in turn determines their interpretation. This yields MWF.

MWF languages are also characterized by a particular status regarding articles -- they either lack articles or have affixal articles. It was argued that in these language types the possibility of wh-licensing \textit{in situ} by interrogative C through unselective binding is blocked because such licensing is done through a null operator in SpecDP, which is absent in languages without articles and languages with affixal articles for a principled reason. The distinction between the lack of articles and affixal articles in MWF languages was, however, shown to have an effect on the presence/absence of Superiority effects. The exceptional behavior of D-linked wh-phrases regarding MWF (they don't need to undergo it) was also captured (including the Hungarian pattern, where D-linked wh-phrases are not exceptional in this respect -- they are subject to MWF). Certain cases of non-wh indefinite interpretations of wh-phrases were also discussed.

All in all, the paper has established the following generalizations regarding MWF, where Hungarian was shown to be exceptional regarding \REF{ex:bosk:35c} but for a principled reason, which was tied to its exceptional behavior regarding D-linking; the reader should thus bear in mind that the way \REF{ex:bosk:35c} is deduced in the paper does leave room for principled exceptions.

 
\ea\label{ex:bosk:35}
\ea \label{ex:bosk:35a}
If a language has multiple wh-fronting, it has a sub-wh indeterminate system.
\ex \label{ex:bosk:35b}
MWF languages either lack articles or have affixal definite articles.
\ex \label{ex:bosk:35c}
MWF languages without articles do not show basic Superiority effects, those with affixal articles do.
\z 
\z 

\noindent At any rate, the main typological finding of this paper is that a confluence of independent factors, namely the sub-wh indeterminate system and a particular status regarding articles, is what is behind MWF.

\section*{Abbreviations}

\begin{tabularx}{.5\textwidth}{@{}lQ}
\textsc{1}& first person\\
\textsc{2}& second person\\
\textsc{acc}& accusative\\
\textsc{caus}& causative\\
\textsc{cl}& classifier\\
\textsc{ctemp}& contemporaneous relative\\
\textsc{gen}& genitive\\
\textsc{HKSL}& Hong Kong Sign Language\\
\textsc{inch}& inchoative\\
\textsc{ins}& instrumental\\
\textsc{loc}& locative\\
\textsc{nom}& nominative\\
\textsc{MWF}& multiple wh-fronting\\
\end{tabularx}%
\begin{tabularx}{.5\textwidth}{XQ@{}}
\textsc{part}& particle\\
\textsc{pass}& passive\\
\textsc{pfv}& perfective\\
\textsc{PL}& pair-list\\
\textsc{prs}& present\\
\textsc{pst}& past \\
\textsc{q}& question particle\\
%\textsc{ra}& ???\\
\textsc{SC}& Serbo-Croatian\\
\textsc{sg}& singular\\
\textsc{SP}& single-pair\\
\textsc{subj}&subjunctive\\
\textsc{top}& topic\\
&\\ % this dummy row achieves correct vertical alignment of both tables
\end{tabularx}

\section*{Acknowledgments}

For helpful comments and suggestions, I thank two anonymous referees, the audience at FDSL 15, and the participants of my 2020 University of Connecticut seminar.

\printbibliography[heading=subbibliography,notkeyword=this]

\end{document}

