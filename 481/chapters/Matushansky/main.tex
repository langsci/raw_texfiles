\documentclass[output=paper,colorlinks,citecolor=black,koreanfont]{langscibook}
\ChapterDOI{10.5281/zenodo.15394184}
%\bibliography{localbibliography}

\author{Ora Matushansky\orcid{0000-0002-1420-7091}\affiliation{SFL (CNRS/Université Paris-8/UPL)}}
% replace the above with you and your coauthors
% rules for affiliation: If there's an official English version, use that (find out on the official website of the university); if not, use the original
% orcid doesn't appear printed; it's metainformation used for later indexing

%%% uncomment the following line if you are a single author or all authors have the same affiliation
\SetupAffiliations{mark style=none}

%% in case the running head with authors exceeds one line (which is the case in this example document), use one of the following methods to turn it into a single line; otherwise comment the line below out with % and ignore it
%\lehead{Šimík, Gehrke, Lenertová, Meyer, Szucsich \& Zaleska}
%\lehead{Radek Šimík et al.}

\title{Russian verbal stress retraction as induced unstressability}
% replace the above with your paper title
%%% provide a shorter version of your title in case it doesn't fit a single line in the running head
% in this form: \title[short title]{full title}
\abstract{This paper analyzes Russian verbal stress through the prism of the {1\SG} pattern, which characterizes about a third of the productive second conjugation (\textit{i}-verbs), as well as many others. In this pattern the {1\SG} and a few other present-tense forms surface with inflectional stress, while all other cells of the present-tense paradigm appear with stem-final stress. I propose that this pattern arises as a result of the more general hiatus resolution process that deletes a vowel before another vowel, on the assumption that the accentual specification of the deleted vowel is retained. I propose that the vocalic thematic suffix is post-accenting and the vocalic present-tense suffix is accented. Once the former is deleted, the latter is rendered unstressable because it receives two conflicting accentual requirements: to bear stress (accentuation) and to shift it to the next syllable (post-accentuation). This conflict is resolved by the deletion of the present-tense suffix from the metrical tier, which forces the accent onto the ending if available and onto the final syllable of the stem otherwise.

\keywords{lexical stress, accent, hiatus resolution, Russian, verbal stress, stress shift}
}

\begin{document}
\maketitle

\iffalse
1. The author requests that examples with derivations not be split across two pages when the final layout is done.  

2. The issue of underlining (\ul{e}, \ul{~~}, etc) hasn't been resolved yet. In the article, underlining might be one of the exceptional circumstances in which the publisher allows it. However, if necessary, the author would be happy with replacing each of the currently underlined positions with a frame around it or with highlighting it by shading.

2. The author wanted the References sections to appear in the boomkarks. Sebastian said that this “should be fine in the final book” but right now, they do not appear in the bookmarks. 

\fi



\section{The puzzle: The {1\SG} present-tense pattern}\label{mat:sec:intro}
\largerpage[-4]
The Russian verb productively consists of four parts: the lexical stem (henceforth, \textsc{L-stem}), which contains the root and semantically contentful suffixes, the thematic suffix, the tense suffix and agreement morphology.\footnote{The transcriptions below closely follow Russian orthography and do not indicate: (i) palatalization before front
vowels (/Ci/ → [Cʲi], /Ce/ → [Cʲe]), (ii) various vowel reduction phenomena in unstressed syllables, (iii) voicing assimilation and final devoicing. Stress is marked by an acute accent on the vowel. The yers (abstract high lax unrounded vowels) are represented as /ĭ/ (the front yer) and /ŭ/ (the back yer). The letters \textit{ч} (IPA [t͡ɕ], see \citealt{PadgettZygis2007}), \textit{ш} (IPA [ʂ]), \textit{ж} (IPA [ʐ]), \textit{щ} (IPA [ɕɕ], \textit{ц} (IPA [͡ts]) are traditionally rendered as \textit{č}, \textit{š}, \textit{ž}, \textit{šč}, and \textit{c}.} The thematic suffix is a cover term for a morpheme that appears between the verbal stem (potentially including verbalizing or imperfective suffixes) and the tense and agreement suffixes. Slavic thematic suffixes have been analyzed as verbalizers or as semantically null morphological glue.\footnote{See \citet{AntonyukQuaglia2022} for a range of opinions.}

\ea\label{mat:ex:RussianVerb}
\ea\label{mat:ex:RussianVerbA}\gll \parbox{2.65cm}{léz-} \parbox{2.25cm}{~} \parbox{2cm}{l-} a \\
{\textsc{l-stem}: climb} {\textsc{theme}: none} {\textsc{tense}: past} {ϕ: {\FEM.\SG}}\\\hfill (athematic verb)
\z
\ea\gll \parbox{2.65cm}{žértv-ov-} \parbox{2.25cm}{a-} \parbox{2cm}{l-} a \\
{\textsc{l-stem}: sacrifice} {\textsc{theme}: a/i} {\textsc{tense}: past} {ϕ: {\FEM.\SG}}\\\hfill (thematic verb)
\z
\z

\noindent As Russian is a lexical stress language (see \citealt{Zaliznjak1985,Melvold1989,Idsardi1992,Garde1998,Alderete1999,Revithiadou1999,Butska2002}, and \citealt{mat:Dubina2012}), each morpheme potentially introduces an accent, which can appear on it (which would make the morpheme \textsc{accented}), before it (\textsc{pre-accenting}) or after it (\textsc{post-accenting}). The position of the surface stress is determined by the Basic Accentuation Principle \REF{mat:ex:BAP}:

\ea\label{mat:ex:BAP}
\textsc{The Basic Accentuation Principle} \citep{KiparskyHalle1977}:\\
Assign stress to the leftmost accented vowel; if there is no accented vowel, assign stress to the initial vowel.
\z

\noindent An examination of the accentuation of Russian thematic verbs reveals three productive patterns in the present tense correlating with two in the past: consistent stem stress (\tabref{mat:tab:Accents-i}-a), consistent post-stem stress (\tabref{mat:tab:Accents-i}-b) and variable stress in the present (final stress in the first-person singular, stem-final stress elsewhere, henceforth \textsc{the 1sg pattern}) correlated with post-stem stress in the past (\tabref{mat:tab:Accents-i}-c). The pattern in \tabref{mat:tab:Accents-i}-d, involving stem-final stress in the present-tense correlating with stress on the thematic suffix in the past, cannot be called productive because it occurs with only four verbal stems, but as it also characterizes the productive verbalizing suffix \Affix{ow}, it is quite frequent.

\begin{table}
\caption{Accentual interaction in thematic verbs}
\label{mat:tab:Accents-i}
 \begin{tabularx}{1\textwidth}{lL{2.92cm}XXL{1.8cm}X}
  \lsptoprule
    &   &{\PRS-1\SG}        &{\PRS-3\SG}  
        &{\PST-\FEM.\SG}       &{\PST-\PL}   \\
  \midrule
  a.    &stem: \newline\textit{-žal-} ‘sting’  
        &{žálʲ-u}        &{žál-i-t}
        &{žál-i-l-a}     &{žál-i-l-i}  \\\addlinespace[5pt]
  b.    &post-stem: \newline\textit{-govor-} ‘speak’  
        &{govorʲ-ú}      &{govor-í-t}
        &{govor-í-l-a}   &{govor-í-l-i}  \\\addlinespace[5pt]
  c.    &{1\SG}: \newline\textit{-lʲub-} ‘love’  
        &{lʲublʲ-ú}      &{lʲúb-i-t}
        &{lʲub-í-l-a}    &{lʲub-í-l-i}  \\\addlinespace[5pt]
  d.    &stem-final present: \newline\textit{-koleb-} ‘rock’  
        &{koléblʲ-u}     &{koléblʲ-e-t}    
        &{koleb-á-l-a}    &{koleb-á-l-i}  \\
  \lspbottomrule
 \end{tabularx}
\end{table}

The stem-stress pattern in \tabref{mat:tab:Accents-i}-a corresponds to an accented L-stem (which, being leftmost, wins over any suffixal accents). The consistent post-stem stress in the past tense of both \tabref{mat:tab:Accents-i}-b and \tabref{mat:tab:Accents-i}-c suggests that the thematic suffix is accented, while the L-stems can be either unaccented or post-accenting. However, the {1\SG} pattern in \tabref{mat:tab:Accents-i}-c is not predicted by the system sketched so far, and neither is the pattern in \tabref{mat:tab:Accents-i}-d, which only arises with the class of verbs whose thematic suffix surfaces as /a/ in the past and as /i/ (giving rise to the so-called transitive softening mutation) in the present (henceforth, the \Affix{a}/\Affix{i} class). 

In this paper I will link the {1\SG} pattern to the unstressability of the present-tense suffix, which results from its absence in the metrical tier. I will propose that this absence itself arises from an accentual conflict: that with unaccented L-stems the deletion of the thematic vowel before the present-tense suffix creates an accentual conflict that can only be resolved by the deletion of the problematic position from the metrical tier. I will then hypothesize how post-accenting L-stems can produce both the consistent post-stem stress (\tabref{mat:tab:Accents-i}-b) and the stem-final stress in the present (\tabref{mat:tab:Accents-i}-d), and link the difference between the two situations to glide deletion and its timing.

The paper is structured as follows. In \sectref{mat:sec:Background} I will introduce the segmental phonology of Russian verbal conjugation and the hiatus resolution mechanism: vowel-before-vowel deletion. I will also discuss the accentuation of the relevant morphemes revealed by their interplay in the athematic verb and show that in the presence of a thematic suffix a stress pattern arises that is not predicted by the interaction of these morphemes.

\sectref{mat:sec:ThematicVerbs} discusses the role of the thematic suffix. I will show that the thematic suffix usually introduces an accent, which should have the double effect of removing the difference between unaccented and post-accenting L-stems and nullifying the impact of all following suffixes. As this predicts the impossibility of the {1\SG} pattern and removes the possibility of explaining it in the terms of L-stem accentuation, a special lexical property, that of triggering stress retraction, has been appealed to. I will show that this hypothesis does not explain why some thematic classes are more prone to exhibiting the {1\SG} pattern than others or why the paradigm cells that fail to undergo retraction are phonologically defined as simple vocalic suffixes. My explanation of the latter fact will be introduced in section \sectref{mat:sec:Unstressability}: I will suggest that in the first conjugation the {1\SG} pattern arises from induced unstressability of the present-tense suffix.

\sectref{mat:sec:Accentuation} examines the {1\SG} pattern in \Affix{a}/\Affix{i} verbs and argues that verbs manifesting it have an unaccented lexical stem, which further supports a phonological explanation of the {1\SG} pattern. \sectref{mat:sec:AccentualConflict} provides such an explanation by ascribing the unstressability of the present-tense suffix to accentual conflict. As will be discussed below, due to the deletion of the thematic suffix before the vocalic present-tense suffix, the same syllable ends up with conflicting instructions: both to bear an accent and to assign it to the next syllable.  The need to resolve this conflict will be shown to derive not only the {1\SG} pattern but also the stem-final stress in the present tense of some \Affix{a}/\Affix{i} verbs. The treatment of the {1\SG} pattern will be shown to extend to second-conjugation verbs, which have 
been argued to have a null present-tense suffix. 

\sectref{mat:sec:Conclusion} provides the conclusion and discusses potential reasons for the non-pro\-duc\-tivity of the {1\SG} pattern in some verb classes.

\section{Background: Verbal conjugation and the {1\SG} pattern}\label{mat:sec:Background}
\largerpage[-2]
In this section I discuss the conjugation of the Russian verb: first the segmental representation of the two tenses and then their accentual properties. As  \tabref{mat:tab:NestiVinit} illustrates, Russian has two conjugation classes, distinguished by the vowel appearing before the person-number suffix in the present tense: In the first conjugation it is \Affix{e} and in the second, \Affix{i}. 

\begin{table}
\caption{Verbal conjugations, present-tense paradigms: \textit{nestí} ‘to carry’, \textit{vinítʲ} ‘to blame’}
\label{mat:tab:NestiVinit}
 \resizebox{\textwidth}{!}{\begin{tabularx}{1.11\textwidth}{lXL{2.9cm}L{2.86cm}L{3.01cm}} 
  \lsptoprule
    &\multicolumn{2}{c}{first conjugation} &\multicolumn{2}{c}{second conjugation}   \\
  \midrule
    &\multicolumn{1}{c}{singular}    &\multicolumn{1}{c}{plural}    
    &\multicolumn{1}{c}{singular}    &\multicolumn{1}{c}{plural}    \\\midrule
1.  &nes-e-u $\rightarrow$ nesú
    &nes-e-m $\rightarrow$ nesʲóm
    &vin-i-Ø-u $\rightarrow$ vinʲú  
    &vin-i-Ø-m $\rightarrow$ viním  \\
2.  &nes-e-šʲ $\rightarrow$ nesʲóšʲ 
    &nes-e-te $\rightarrow$ nesʲóte   
    &vin-i-Ø-šʲ $\rightarrow$ viníšʲ  
    &vin-i-Ø-te $\rightarrow$ viníte  \\
3.  &nes-e-t $\rightarrow$ nesʲót   
    &nes-e-nt $\rightarrow$ nesút  
    &vin-i-Ø-t $\rightarrow$ vinít  
    &vin-i-Ø-nt $\rightarrow$ vinʲát  \\
  \lspbottomrule
 \end{tabularx}}
\end{table}

While the first-conjugation \Affix{e} corresponds to the present-tense suffix, the second-conjugation \Affix{i} is the thematic suffix (\citealt{Micklesen1973,CoatsLightner1975}, and \citealt[129--130]{Itkin2007}, though alternative analyses exist, see \sectref{mat:subsec:SecondConjugation}). 

The consideration of the past-tense forms shows that the verb \textit{nestí} ‘to carry’ is athematic (no vowel appears between the L-stem and the past-tense suffix \Affix{l}), while the verb \textit{vinítʲ} ‘to blame’ contains the thematic vowel \Affix{i}, see \tabref{mat:tab:NestiVinitPast}. The infinitive suffix (surface  [ti] under stress, [tʲ] otherwise) shows the same behavior.

\begin{table}
\caption{Verbal conjugations, past-tense paradigms: \textit{nestí} ‘to carry’, \textit{vinítʲ} ‘to blame’}
\label{mat:tab:NestiVinitPast}
 \begin{tabularx}{\textwidth}{lXXL{2.72cm}L{2.62cm}} 
  \lsptoprule
    &\multicolumn{2}{c}{first conjugation} &\multicolumn{2}{c}{second conjugation}   \\
  \midrule
    &\multicolumn{1}{c}{singular}    &\multicolumn{1}{c}{plural}    
    &\multicolumn{1}{c}{singular}    &\multicolumn{1}{c}{plural}    \\\midrule
{\MASC}     &nes-l-ŭ $\rightarrow$ nʲós     &  
            &vin-i-l-ŭ $\rightarrow$ viníl  & \\
{\FEM}      &nes-l-a $\rightarrow$ neslá    &nes-l-i $\rightarrow$ neslí 
            &vin-i-l-a $\rightarrow$ viníla &vin-i-l-i $\rightarrow$ viníli  \\
{\N}     &nes-l-o $\rightarrow$ nesló    &  
            &vin-i-l-o $\rightarrow$ vinílo &  \\
  \lspbottomrule
 \end{tabularx}
\end{table}

The past tense (historically, the active past participle form) is segmentally uncontroversial, and its number-gender suffixes are identical to those of pronouns. While the concatenation of the various morphemes in the past tense is relatively straightforward, in the present tense vowel sequences are created that do not surface as such.\footnote{I will not discuss the details of how the consonant cluster created by the stem-final consonant and the past-tense suffix \Affix{l} or the infinitive suffix \Affix{tĭ} is resolved for various consonants (see \citealt{Lightner1965,Lightner1972}). The alternation between the surface back vowel with a palatalized preceding consonant ([ʲo]) under stress and the phonological /e/ in unstressed syllables in \tabref{mat:tab:NestiVinit}, \tabref{mat:tab:NestiVinitPast} and elsewhere is allophonic \citep{Lightner1969,Boyd1997}.}

\subsection{Verbal conjugation and vowel-before-vowel deletion}

While \citet{Lightner1965,Lightner1972} and \citet{Halle1973} propose rather abstract underlying representations for Russian present-tense agreement suffixes, for our purposes the finer details do not matter, and I will follow \citet{Melvold1989} and assume that the underlying representations of these suffixes are nearly always identical to their surface forms, as indicated in \tabref{mat:tab:NestiVinit}.\footnote{The surface representations of the {3\PL} endings, \Affix{ut} and \Affix{at} for the first and second conjugations respectively, arise from the morphologically conditioned merger of the present-tense suffix \Affix{e} (for \Affix{ut}) or the thematic vowel \Affix{i} (\Affix{at}) with the nasal of the ending (cf. \citealt{Lightner1969,Kayne1967}). The same VN-modifications occur in the active present participle, inside some verbal roots and in the declension of the ten nouns in [mʲa] \citep{Lightner1967,Halle2004}. \citet[237]{Melvold1989} assumes this representation for the second conjugation but not for the first one (where she postulates the surface [ut] as the underlying representation), yet the behavior of this ending with respect to stress suggests a consonantal ending in both conjugation classes.} As is easy to see, most but not all agreement suffixes in the present tense are consonantal.

The {1\SG} form, which will be crucial for the discussion below, shows how hiatus is resolved in Russian. If the vowel preceding another vowel is \textit{i}, like in second-conjugation verbs, it turns into a glide before any vowel distinct from \textit{i}.\footnote{\label{mat:fn:Iotation}Examples \xxref{mat:ex:VowelDeletionB}{mat:ex:VowelDeletionC} illustrate the fact that the consonant--glide sequence undergoes a mutation known as \textsc{transitive softening}, or \textsc{iotation} (\citealt{Jakobson1929,Meillet1934,Kortlandt1994,TownsendJanda1996}, \textit{inter alii}; see \citealt{Halle1963,Lightner1972,CoatsLightner1975,Bethin1992,Brown1998} and \citealt{RubachBooij2001} for generativist analyses), which will not be directly relevant here.} Otherwise the first vowel is deleted:

\ea\label{mat:ex:VowelDeletion}
\ea\label{mat:ex:VowelDeletionA}\gll vin- i- Ø u → vin-j-u → \textit{vinʲú} \\
blame {\THEM} {\PRS} {1\SG}\\
\ex\label{mat:ex:VowelDeletionB}\gll pros- i- Ø u → pros-j-u → \textit{prošú} \\
ask {\THEM} {\PRS} {1\SG}\\
\ex\label{mat:ex:VowelDeletionC}\gll nes- e- u → nes-\cancel{e}-u → \textit{nesú} \\
carry {\PRS} {1\SG}\\
\z
\z

\noindent While in \REF{mat:ex:VowelDeletionC} the deleted vowel belongs to the present-tense suffix, this latter can itself trigger vowel deletion when preceded by a vocalic or vowel-final thematic suffix, such as, for instance, the semelfactive suffix \Affix{nu}, whose vowel is deleted before the vocalic present-tense suffix, as in \figref{mat:fig:derivation-stomp-3sg}.

\begin{figure}
    \begin{tikzpicture}[yscale=.75]
        \node[anchor=base west] at (-2.25,0) {\strut};
        \node[anchor=base] at (0,0) {[[[√-{\THEM}]$_1$-{\PRS}]$_2$-{3\SG}]$_3$};
        \node[anchor=base] at (0,-1) {[[[top-nu]$_1$-{e}]$_2$-{t}]$_3$};
        \node[anchor=base] at (0,-2) {[tópnet]};
        \draw[->] (0,-1.25) to (0,-1.70);
        \node[anchor=base west] at (2,-1.70) {Vowel-before-vowel deletion};
        \node[anchor=base east] at (10,0) {(stomp-{\PRS}-{3\SG})};
    \end{tikzpicture}    
    \caption{{3\SG} derivation}
    \label{mat:fig:derivation-stomp-3sg}
\end{figure}

The vowel of the thematic suffix is deleted before the vowel of the present-tense suffix also in the {1\SG}, and then the present-tense suffix is deleted before the vocalic {1\SG} ending \Affix{u}, as in \figref{mat:fig:derivation-stomp-1sg}.\footnote{\label{mat:fn:Null}\citet[83--86]{Melvold1989} points out that there are two arguments for the absence of the present-tense suffix \Affix{e} in the {1\SG} and {3\PL}: the lack of Velar Palatalization and the position of the stress (which she predicts to retract after hiatus resolution). She proposes therefore that the present-tense suffix is null in the {1\SG} and {3\PL}, and the two endings are accented. The correct result ensues, yet the fact that the {1\SG} and the {3\PL} endings behave differently in {1\SG}-pattern verbs then requires an additional stipulation. Since I assume that the present-tense suffix is realized in the entire first-conjugation paradigm, my alternative explanation is that the underlying representation of the present-tense suffix is \Affix{o} and the source of (Velar) Palatalization is a floating [$–$back] feature on the {2\SG}, {3\SG}, {1\PL} and {2\PL} endings. Conversely, palatalization in \REF{mat:ex:VowelDeletionA} results from the consonant-glide sequence \textit{nj} that has undergone transitive softening (see fn. \ref{mat:fn:Iotation}). I will not develop the argument further here.}

\begin{figure}
    \begin{tikzpicture}[yscale=.75]
        \node[anchor=base west] at (-2.25,0) {\strut};
        \node[anchor=base] at (0,0) {[[[√-{\THEM}]$_1$-{\PRS}]$_2$-{1\SG}]$_3$};
        \node[anchor=base] at (0,-1) {[[[top-nu]$_1$-{e}]$_2$-{u}]$_3$};
        \node[anchor=base] at (0,-2) {[[topn-{e}]$_2$-{u}]$_3$};
        \node[anchor=base] at (0,-3) {[tópnu]};
        \draw[->] (0,-1.25) to (0,-1.70);
        \draw[->] (0,-2.25) to (0,-2.70);
        \node[anchor=base west] at (2,-1.70) {Vowel-before-vowel deletion};
        \node[anchor=base west] at (2,-2.70) {Vowel-before-vowel deletion};
        \node[anchor=base east] at (10,0) {(stomp-{\PRS}-{1\SG})};
    \end{tikzpicture}    
    \caption{{1\SG} derivation}
    \label{mat:fig:derivation-stomp-1sg}
\end{figure}

The hypothesis that the derivation of the Russian verb involves the deletion of vowels before other vowels was originally proposed by \citet{Jakobson1948}, who suggested that the longer form of the verbal stem is always the underlying one (see also \citealt{Lightner1965,Halle1973,Melvold1989}, etc.). The natural question to ask here is what happens to the accents when a vowel is deleted or turns into a glide, and this will turn out to be the clue to the {1\SG} pattern. However, before this issue can be addressed (in \sectref{mat:sec:AccentualConflict}), it is necessary to establish the underlying accentuation of Russian verbal suffixes.

I will begin with the closed class of verbs lacking the thematic suffix \REF{mat:ex:RussianVerbA} and on their basis I will show that the {1\SG} pattern is indeed problematic for the assumptions made so far.

\subsection{The Halle--Idsardi stress theory and accent interaction}\label{mat:subsec:HalleIdsardi}

To illustrate accent interactions I use the autosegmental metrical structure notation introduced by \citet{HalleVergnaud1987a,HalleVergnaud1987b} and further developed in \citet{Melvold1989,Idsardi1992,HalleIdsardi1995} and \citet{Halle1997}, where each syllable projected to the metrical tier is indicated by an asterisk and foot edges are marked by parentheses. Feet are unbounded from one accent to the next or to the end of the phonological word, and left-headed, which means that lexical accents can be encoded as underlying left parentheses. The head of each foot is projected to the next line:\bigskip

\ea\label{mat:ex:MetricalTheory}%The author requests that this example not be split across two pages
\ea
\tikzstyle{every picture}+=[remember picture, inner sep=0pt,baseline, anchor=base,execute at begin node=\strut]
z\tikz\node(zim){i};m\hspace{.25cm}\tikz\node(u){u};
 \begin{tikzpicture}[overlay,remember picture]
	 \node [above of = zim, node distance=4mm] (zim-ast) {$\ast$};
  \node [above of = u, node distance=4mm] (u-ast) {$\ast$};
  \node [left of = zim-ast, node distance=2mm] (left-bracket) {(};
  \node [right of = u-ast, node distance=2mm] (right-bracket) {)};
  \coordinate (center) at ($(zim)!0.5!(u)$);
  \node (arrow)   [below of = center,node distance=4mm]  {$\downarrow$};
\end{tikzpicture}\bigskip\bigskip\bigskip
\ex \tikzstyle{every picture}+=[remember picture, inner sep=0pt,baseline, anchor=base,execute at begin node=\strut]
z\tikz\node(zim2){i};m\hspace{.25cm}\tikz\node(u2){u};\hspace{1cm}$\rightarrow$\hspace{1cm} zímu ‘winter.{\SG}.{\ACC}’
 \begin{tikzpicture}[overlay,remember picture]
	 \node [above of = zim2, node distance=4mm] (zim2-ast) {$\ast$};
  \node [above of = zim2-ast, node distance=4mm] (zim2-ast2) {$\ast$};
  \node [above of = u2, node distance=4mm] (u2-ast) {$\ast$};
  \node [left of = zim2-ast, node distance=2mm] (left-bracket) {(};
  \node [right of = u2-ast, node distance=2mm] (right-bracket) {)};
\end{tikzpicture}
\z
\z

\noindent In the \citet{Halle1997} version, which I will be using here, the fact that unaccented words surface with initial stress is implemented by the addition of the right parenthesis at the right edge. The Basic Accentuation Principle \REF{mat:ex:BAP} is implemented by the assumption that feet are left-headed on all lines of the metrical tier, which ensures that only the head of the leftmost foot projects to the next line:

\ea Indo-European stress rules \citep[after][]{Halle1997}:
    \ea Accents are notated in vocabulary representations with left parentheses on line 0
	\ex\label{mat:ex:StressRules-b} Line 0 is subject to the edge-marking rule RRR
	\ex Line 0 is subject to the head-marking rule L
	\ex Line 1 is subject to the edge-marking rule LLL
	\ex Line 1 is subject to the head-marking rule L
	\ex Stress is assigned to the head of the word
\z
\z

\largerpage
\noindent Thus when an accented stem is combined with an unaccented suffix, as in \REF{mat:ex:MetricalTheoryBAP-a}, stress falls on the stem. Conversely, when the stem is unaccented and the suffix, accented, stress surfaces on the suffix \REF{mat:ex:MetricalTheoryBAP-b}. Finally, when both the stem and the suffix are unaccented, the first syllable is stressed \REF{mat:ex:MetricalTheoryBAP-c}:\bigskip

\ea\label{mat:ex:MetricalTheoryBAP}
\ea\label{mat:ex:MetricalTheoryBAP-a}
\tikzstyle{every picture}+=[remember picture, inner sep=0pt,baseline, anchor=base,execute at begin node=\strut]
l\tikz\node(lez){e};z\hspace{.25cm}l\hspace{.25cm}\tikz\node(i){i};
 \begin{tikzpicture}[overlay,remember picture]
	 \node [above of = lez, node distance=4mm] (lez-ast) {$\ast$};
  \node [above of = lez-ast, node distance=4mm] (lez-ast2) {$\ast$};
  \node [above of = i, node distance=4mm] (i-ast) {$\ast$};
  \node [left of = lez-ast, node distance=2mm] (left-bracket) {(};
  \node [right of = i-ast, node distance=2mm] (right-bracket) {)};
  \end{tikzpicture}\bigskip\bigskip
\ex\label{mat:ex:MetricalTheoryBAP-b}\tikzstyle{every picture}+=[remember picture, inner sep=0pt,baseline, anchor=base,execute at begin node=\strut]
kl\tikz\node(klad){a};d\hspace{.25cm}l\hspace{.25cm}\tikz\node(a){a}; \hfill(surface \textit{klalá})
 \begin{tikzpicture}[overlay,remember picture]
	 \node [above of = klad, node distance=4mm] (klad-ast) {$\ast$};
  \node [above of = a, node distance=4mm] (a-ast) {$\ast$};
  \node [above of = a-ast, node distance=4mm] (a-ast2) {$\ast$};
  \node [left of = a-ast, node distance=2mm] (left-bracket) {(};
  \node [right of = a-ast, node distance=2mm] (right-bracket) {)};
\end{tikzpicture}\bigskip\bigskip
\ex\label{mat:ex:MetricalTheoryBAP-c}\tikzstyle{every picture}+=[remember picture, inner sep=0pt,baseline, anchor=base,execute at begin node=\strut]
kl\tikz\node(klad){a};d\hspace{.25cm}l\hspace{.25cm}\tikz\node(i){i}; \hfill(surface \textit{kláli})
 \begin{tikzpicture}[overlay,remember picture]
	 \node [above of = klad, node distance=4mm] (klad-ast) {$\ast$};
  \node [above of = klad-ast, node distance=4mm] (klad-ast2) {$\ast$};
  \node [above of = i, node distance=4mm] (i-ast) {$\ast$};
  \node [right of = i-ast, node distance=2mm] (right-bracket) {)};
\end{tikzpicture}
\z
\z

\noindent Empirically, the combination of a post-accenting stem with an accented suffix does not give rise to a clash: Stress surfaces where both morphemes assign it, i.e., on the suffix. This is illustrated in \REF{mat:ex:AccentCombinations} for the nominal domain: The nominative ending is accented and bears the main stress with both an unaccented and a post-accenting stem (the unaccented accusative ending provides the control distinguishing accented, unaccented and post-accenting stems):

\ea\label{mat:ex:AccentCombinations}
\ea ruká/rúku ‘hand.{\SG.\NOM/\ACC}’\hfill (unaccented stem)
\ex čertá/čertú ‘line.{\SG.\NOM/\ACC}’\hfill (post-accenting stem)
\z
\z

\noindent In the Halle-Idsardi framework this result is obtained by postulating that whenever a sequence of two parentheses obtains that do not group any stress-bearing material, one of them is deleted:\bigskip

\ea\label{mat:ex:MetricalTheory-BracketDeletion}%The author requests that this example not be split across two pages
\ea\label{mat:ex:MetricalTheory-BracketDeletion-a}
\tikzstyle{every picture}+=[remember picture, inner sep=0pt,baseline, anchor=base,execute at begin node=\strut]
r\tikz\node(ruk){u};k\hspace{.5cm}\tikz\node(a){a};
 \begin{tikzpicture}[overlay,remember picture]
	 \node [above of = ruk, node distance=4mm] (ruk-ast) {$\ast$};
  \node [above of = a, node distance=4mm] (a-ast) {$\ast$};
  \node [right of = ruk-ast, node distance=2mm] (left-bracket) {(};
  \node [left of = a-ast, node distance=2mm] (left-bracket2) {(};
  \node [right of = a-ast, node distance=2mm] (right-bracket) {)};
  \coordinate (center) at ($(ruk)!0.5!(a)$);
  \node (arrow)   [below of = center,node distance=4mm]  {$\downarrow$};
\end{tikzpicture}\bigskip\bigskip\bigskip
\ex\label{mat:ex:MetricalTheory-BracketDeletion-b} \tikzstyle{every picture}+=[remember picture, inner sep=0pt,baseline, anchor=base,execute at begin node=\strut]
r\tikz\node(ruk2){u};k\hspace{.5cm}\tikz\node(a2){a};
 \begin{tikzpicture}[overlay,remember picture]
	 \node [above of = ruk2, node distance=4mm] (ruk2-ast) {$\ast$};
  \node [above of = a2, node distance=4mm] (a2-ast) {$\ast$};
  \node [above of = a2-ast, node distance=4mm] (a2-ast2) {$\ast$};
  \node [right of = ruk2-ast, node distance=2mm] (left-bracket) {(};
  \node [right of = a2-ast, node distance=2mm] (right-bracket) {)};
\end{tikzpicture}
\z
\z

\noindent While for examples like \REF{mat:ex:MetricalTheory-BracketDeletion} the choice of the parenthesis to be deleted makes no difference, the interaction between the left parenthesis introduced by post-accenting morphemes and the right parenthesis introduced by \REF{mat:ex:StressRules-b} makes it clear that it is the second parenthesis in a sequence that is deleted, as will be now shown.

Empirically, when a post-accenting morpheme is not followed by any stress-bearing material, stress surfaces on the final syllable. Examples can be readily drawn from nominal declension, where post-accenting nouns surface with stress on the stem-final syllable if the case ending is an unstressable non-vocalized yer, like the genitive plural in \REF{mat:ex:bulava} and the nominative singular in \REF{mat:ex:sekretar}. The same happens in adjectives, as in \REF{mat:ex:adjectives}:

\ea
\ea\label{mat:ex:bulava} bulavá/bulavámi/buláv ‘mace.{\SG.\NOM/\PL.\INS/\PL.\GEN}’
\ex\label{mat:ex:sekretar} sekretárʲ/sekretarʲá/sekretarʲámi ‘secretary.{\SG.\NOM/\SG.\GEN/\PL.\INS}’
\z
\ex\label{mat:ex:adjectives}
\ea zdoróv/zdorová/zdorovó/zdorovɨ́ ‘robust.{\FEM/\MASC/\NEUT/\PL}’
\ex	tʲažʲól/tʲaželá/tʲaželó/tʲaželɨ́ ‘heavy.{\FEM/\MASC/\NEUT/\PL}’
\z 
\z

\noindent Several ways of accounting for this effect are possible and I will not choose between them.\footnote{\label{mat:fn:Yers}The assumption that the nominative singular and genitive plurals endings are underlyingly back yers makes it possible to capitalize on the fact that word-internally an accent assigned to a yer surfaces on the preceding syllable. To capture this, \citet[284]{Halle1997} inserts a left parenthesis on the syllable preceding an accented yer. Alternatively, these stress retraction phenomena have been accounted for by an appeal to iambic feet in Russian (\citealt{Crosswhite1999,mat:Crosswhite2000,Gouskova2010}, and \citealt{mat:Dubina2012}, among others). I will not attempt to address this discussion here.} Importantly, under all approaches this process, distinguishing as it does between vocalized  and non-vocalized yers, is a late one. What is crucial, however, is that the representation of such cases in the Halle-Idsardi framework involves two parentheses on the right edge:

\ea\label{mat:ex:MetricalTheoryYers}
\ea\label{mat:ex:MetricalTheoryYers-a} nominative singular\\\bigskip
\tikzstyle{every picture}+=[remember picture, inner sep=0pt,baseline, anchor=base,execute at begin node=\strut]
b\tikz\node(bu){u};\hspace{.25cm}l\tikz\node(lav){a};v\hspace{.25cm}\tikz\node(a){a};
 \begin{tikzpicture}[overlay,remember picture]
	 \node [above of = bu, node distance=4mm] (bu-ast) {$\ast$};
  \node [above of = lav, node distance=4mm] (lav-ast) {$\ast$};
  \node [above of = a, node distance=4mm] (a-ast) {$\ast$};
  \node [left of = a-ast, node distance=2mm] (left-bracket) {(};
  \node [right of = a-ast, node distance=2mm] (right-bracket) {)};
\end{tikzpicture}
\ex\label{mat:ex:MetricalTheoryYers-b} genitive plural\\\bigskip
\tikzstyle{every picture}+=[remember picture, inner sep=0pt,baseline, anchor=base,execute at begin node=\strut]
b\tikz\node(bu2){u};\hspace{.25cm}l\tikz\node(lav2){a};v\hspace{.25cm}\tikz\node(u2){ŭ};
 \begin{tikzpicture}[overlay,remember picture]
	 \node [above of = bu2, node distance=4mm] (bu2-ast) {$\ast$};
  \node [above of = lav2, node distance=4mm] (lav2-ast) {$\ast$};
  \node [above of = u2, node distance=4mm] (u2-ast) {};
  \node [left of = u2-ast, node distance=2mm] (left-bracket) {(};
  \node [right of = u2-ast, node distance=2mm] (right-bracket) {)};
\end{tikzpicture}
\z
\z

\noindent If the first parenthesis in the sequence were deleted, the outcome would be identical to that for an unaccented stem and stress would be incorrectly predicted to be initial. If, on the other hand, the rightmost parenthesis is deleted, the resulting configuration can be repaired as suggested in fn. \ref{mat:fn:Yers}.\footnote{Yet another alternative would be to move the final left parenthesis before the insertion of the right parenthesis. I reject this option since it requires the same repair strategy with an additional assumption about ordering, and the need to delete one of the two immediately adjoining parentheses is motivated independently.} It will be later demonstrated that the deletion of the second one in the sequence of two immediately adjacent parentheses leads to a correct prediction in another situation where such a configuration arises.


\subsection{The underlying accentuation of Russian verbal suffixes}\label{mat:subsec:URofSuffixes}
\largerpage[2]

Following \citet{Halle1973} and \citet{Melvold1989}, four main accentual classes of athematic verbs can be established, depending on the accentuation of the root, with the positions of the underlying accents indicated by underlining in \tabref{mat:tab:InteractionAthematic}.\footnote{\label{mat:fn:FifthClass}The fifth class consists of just two verbal roots, \textit{-mog-} (\textit{močʲ} ‘to be able’) and the cranberry root \textit{-im-}/\textit{-nʲa-} (e.g., \textit{prinʲátʲ} ‘to accept’) and their derivatives, which exhibit the {1\SG} pattern. I return to this matter in \sectref{mat:subsubsec:AthematicVerbs}.} As discussed above, systematic stem stress (\tabref{mat:tab:InteractionAthematic}-a) is a sign of an accented root, and variable stress (\tabref{mat:tab:InteractionAthematic}-c) is an indicator of an unaccented root. I follow \citet{Melvold1989} and treat (b) and (d) in \tabref{mat:tab:InteractionAthematic} as post-accenting roots, but differ from her in their analysis, as will be seen below.
\clearpage

\begin{table}
\caption{Accentual interaction in athematic (√-T-ϕ) verbs}
\label{mat:tab:InteractionAthematic}
 \begin{tabularx}{.95\textwidth}{lL{2.75cm}lCCC} 
  \lsptoprule
    &&&accented\linebreak{\PST-\FEM.\SG} &unaccented\linebreak{\PST-\PL} &accented\linebreak{\PST-3\SG}   \\
  \midrule
    a.  &stem:\newline\textit{-l\ul{e}z-} ‘climb’ &A
        &l\ul{é}z-l-\ul{a}    &l\ul{é}z-l-i    &l\ul{é}z-\ul{e}-t    \\\addlinespace[5pt]
    b.  &post-stem:\newline\textit{-nes\ul{~~}-} ‘carry’ &PA$+$
        &nes\ul{~~}-l-\ul{á}   &nes\ul{~~}-l-í     &nes\ul{~~}-ʲ\ul{ó}-t   \\\addlinespace[5pt]
    c.  &variable (past):\newline\textit{-klad-} ‘put’ &UA
        &kla-l-\ul{á}    &klá-l-i    &klad-ʲ\ul{ó}-t   \\\addlinespace[5pt]
    d.  &retracting (past):\newline\textit{-grɨz\ul{~~}-} ‘gnaw’ &PA$-$
        &gríz\ul{~~}-l-\ul{a}  &gríz\ul{~~}-l-i  &griz\ul{~~}-ʲ\ul{ó}-t   \\
  \lspbottomrule
 \end{tabularx}
\end{table}


I also follow \citeauthor{Melvold1989} in assuming that the past-tense exponent \Affix{l}, as expected from a consonantal affix, does not introduce an accent. As a result, the contrast between the feminine and the plural in the past of \tabref{mat:tab:InteractionAthematic}-c is derived by treating the plural suffix \Affix{i} as unaccented, while the feminine ending \Affix{a} is accented. The masculine and neuter endings are unaccented as well.

The post-stem pattern in \tabref{mat:tab:InteractionAthematic}-b, with consistent final stress in the past, results, \citeauthor{Melvold1989} argues, from a post-accenting root, whereas the pattern in \tabref{mat:tab:InteractionAthematic}-d involves the special rule of retraction triggered by a subclass of verbal roots. While \citet{Melvold1989} implements this process by moving the relevant parenthesis one syllable to the left, \citet{Halle1997} handles it by inserting a parenthesis before the preceding syllable.

Lexically conditioned retraction does not, however, explain the facts discussed in \citet{Matushansky2024}, namely, that verbs following the pattern in \tabref{mat:tab:InteractionAthematic}-b also violate the Basic Accentuation Principle \REF{mat:ex:BAP} in the infinitive and in the passive past participle. Despite the fact that both these suffixes behave as pre-accenting in other environments, stress is final:

\ea\label{mat:ex:nesti-unesena}
\ea\label{mat:ex:nesti} nestí ‘to carry’,\\
cf. léztʲ ‘to climb’, klástʲ ‘to put’, grɨ́ztʲ ‘to gnaw’
\ex\label{mat:ex:unesena} unesená ‘carried away.{\FEM.\SG}’,\\
cf. perelézena ‘climbed over’, progrɨ́zena ‘gnawed through’, sprʲádena ‘spun’
\z
\z

\noindent The Basic Accentuation Principle \REF{mat:ex:BAP} predicts that in a sequence of a post-ac\-cent\-ing and a pre-accenting morpheme the stress assigned by the latter should win \REF{mat:ex:Nesti-Wrong} (indices are added to indicate which morpheme introduced which parenthesis). Such is in fact the case in other instances of such morpheme sequences.\bigskip

\ea\label{mat:ex:Nesti-Wrong}
\ea
\tikzstyle{every picture}+=[remember picture, inner sep=0pt,baseline, anchor=base,execute at begin node=\strut]
n\tikz\node(nes){e};s\hspace{1cm}t\tikz\node(i){ĭ};
 \begin{tikzpicture}[overlay,remember picture]
	 \node [above of = nes, node distance=4mm] (nes-ast) {$\ast$};
  \node [above of = i, node distance=4mm] (i-ast) {$\ast$};
  \node [right of = nes-ast, node distance=4mm] (left-bracket) {($^1$};
  \node [left of = i-ast, node distance=4mm] (left-bracket2) {($^2$};
  \node [yshift=-1mm] (plus)  at ($(nes-ast)!0.5!(i-ast)$) {$+$};
\end{tikzpicture}\bigskip\bigskip
\ex \tikzstyle{every picture}+=[remember picture, inner sep=0pt,baseline, anchor=base,execute at begin node=\strut]
n\tikz\node(nes2){e};s\hspace{1cm}t\tikz\node(i2){ĭ};
 \begin{tikzpicture}[overlay,remember picture]
	 \node [above of = nes2, node distance=4mm] (nes2-ast) {$\ast$};
    \node [left of = nes2-ast, node distance=4mm] (left-bracket) {($^2$};
  \node [above of = i2, node distance=4mm] (i2-ast) {$\ast$};
  \node [left of = i2-ast, node distance=4mm] (left-bracket2) {($^1$};
  \node [right of = i2-ast, node distance=3cm] (comment) {wrongly winning accent};
  \draw [->] (comment.north west) [out=135,in=45] to (nes2-ast);
\end{tikzpicture}
\z
\z


\noindent To explain the facts in \REF{mat:ex:nesti-unesena}, \citet{Matushansky2024} argues that the pattern in \tabref{mat:tab:InteractionAthematic}-d should be analyzed as involving unaccentable roots, i.e., roots that cannot bear a parenthesis anywhere but at the right edge.\footnote{This is a novel notion introduced to explain the fact that both the unaccentable PPP suffix \Affix{en} and unaccentable roots cannot bear an accent but, as a last resort, can bear stress when not followed by stress-bearing material. This ability to bear stress distinguishes unaccentability from unstressability (to be discussed further). See \citet{Matushansky2023b} for a proposal distinguishing the two in a different framework treating Russian accent as tone (cf. \citealt{mat:Dubina2012}): Unaccentable roots in it are absent from the tonal tier, and unstressable ones, from the metrical tier.} As a result, the accent is forced rightwards, yielding word-final stress in passive past participles and the realization of the yer in the infinitive suffix. To explain the pattern in \tabref{mat:tab:InteractionAthematic}-d \citet{Matushansky2024} proposes that it involves post-accenting stems and that forcing stress retraction is the general property of the past-tense suffix.\footnote{The accent introduced by the feminine ending \nobreakdash-\textit{a} is not affected by this retraction. This is naturally achieved if stress is assigned cyclically, but I will not pursue this line of inquiry here, leaving it for future research.} As is easy to see, under this approach accented stems will retain stress on themselves, unaccented stems will be unaffected, unaccentable stems will still force post-stem stress, and only in \tabref{mat:tab:InteractionAthematic}-d stress will be retracted:\bigskip

\ea\label{mat:ex:Grizla-Derivation}%The author requests that this example not be split across two pages
\ea
\tikzstyle{every picture}+=[remember picture, inner sep=0pt,baseline, anchor=base,execute at begin node=\strut]
gr\tikz\node(griz){ɨ};z\hspace{.25cm}l\hspace{.25cm}\tikz\node(a){a};
 \begin{tikzpicture}[overlay,remember picture]
	 \node [above of = griz, node distance=4mm] (griz-ast) {$\ast$};
  \node [above of = a, node distance=4mm] (a-ast) {$\ast$};
  \node [right of = griz-ast, node distance=2mm] (left-bracket) {(};
  \node [left of = a-ast, node distance=2mm] (left-bracket2) {(};
  \node [right of = a-ast, node distance=2mm] (right-bracket) {)};
  \coordinate (center) at ($(griz)!0.5!(a)$);
  \node (arrow)   [below of = center,node distance=4mm]  {$\downarrow$};
\end{tikzpicture}\bigskip\bigskip
\ex
\tikzstyle{every picture}+=[remember picture, inner sep=0pt,baseline, anchor=base,execute at begin node=\strut]
gr\tikz\node(griz2){ɨ};z\hspace{.25cm}l\hspace{.25cm}\tikz\node(a2){a};
 \begin{tikzpicture}[overlay,remember picture]
	 \node [above of = griz2, node distance=4mm] (griz2-ast) {$\ast$};
  \node [above of = a2, node distance=4mm] (a2-ast) {$\ast$};
  \node [left of = griz2-ast, node distance=2mm] (left-bracket) {(};
  \node [left of = a2-ast, node distance=2mm] (left-bracket2) {(};
  \node [right of = a2-ast, node distance=2mm] (right-bracket) {)};
  \coordinate (center2) at ($(griz2)!0.5!(a2)$);
  \node (arrow2)   [below of = center2,node distance=4mm]  {$\downarrow$};
\end{tikzpicture}\bigskip\bigskip
\ex \tikzstyle{every picture}+=[remember picture, inner sep=0pt,baseline, anchor=base,execute at begin node=\strut]
gr\tikz\node(griz3){ɨ};z\hspace{.25cm}l\hspace{.25cm}\tikz\node(a3){a};
 \begin{tikzpicture}[overlay,remember picture]
  \node [above of = griz3, node distance=4mm] (griz3-ast) {$\ast$};
  \node [above of = a3, node distance=4mm] (a3-ast) {$\ast$};
  \node [left of = griz3-ast, node distance=2mm] (left-bracket) {(};
  \node [left of = a3-ast, node distance=2mm] (left-bracket2) {(};
  \node [right of = a3-ast, node distance=2mm] (right-bracket) {)};
  \node [above of = griz3-ast, node distance=4mm] (a3-ast2) {$\ast$};
\end{tikzpicture}
\z
\z

\noindent Turning now to the present tense, only two patterns can be detected (modulo fn. \ref{mat:fn:FifthClass}): systematic stress on the stem (\tabref{mat:tab:InteractionAthematic}-a) if it is accented, and on the present-tense suffix (\tabref{mat:tab:InteractionAthematic}-b--d) otherwise. This means \citep{Halle1973,Melvold1989} that the present-tense suffix has to introduce an accent: If it were unaccented, the Basic Accentuation Principle \REF{mat:ex:BAP} would predict stem stress both for accented stems (due to the accent of the stem) and for unaccented stems (stress on the leftmost syllable). Conversely, if the present-tense suffix is accented (as assumed by \citealt{Melvold1989}), post-stem stress is correctly predicted for the entire present-tense paradigm for both unaccented \REF{mat:ex:PostStem-Unaccented} and post-accenting \REF{mat:ex:PostStem-Post-Accenting} roots:

\ea\label{mat:ex:PostStem-Unaccented}
\ea\gll klad- \ul{e}- m → kladʲóm \\
put {\PRS} {1\PL}\\
\ex\gll klad- \ul{e}- te → kladʲóte \\
put {\PRS} {2\PL}\\
\z
\ex\label{mat:ex:PostStem-Post-Accenting}
\ea\gll nes\ul{~~}- \ul{e}- m → nesʲóm \\
put {\PRS} {1\PL}\\
\ex\gll nes\ul{~~}- \ul{e}- te → nesʲóte \\
put {\PRS} {2\PL}\\
\z
\z

\noindent Given that the present-tense suffix is accented and deleted before the {1\SG} ending \Affix{u}, the fact that this ending remains stressed with unaccented verbs (e.g., \textit{kladú} ‘put.{1\SG}’) demonstrates, \textit{ceteris paribus}, that the accent of a deleted vowel is neither deleted nor shifted to the left, and this is also what is predicted by the Halle-Idsardi system:\bigskip

\ea\label{mat:ex:Klad-Derivation}%The author requests that this example not be split across two pages
\ea\tikzstyle{every picture}+=[remember picture, inner sep=0pt,baseline, anchor=base,execute at begin node=\strut]
\gll kl\tikz\node(klad){a};d- \tikz\node(e){e};- \tikz\node(u){u}; \\
put {\PRS} {1\SG}\\
 \begin{tikzpicture}[overlay,remember picture]
	 \node [above of = klad, node distance=4mm] (klad-ast) {$\ast$};
  \node [above of = e, node distance=4mm] (e-ast) {$\ast$};
  \node [above of = u, node distance=4mm] (u-ast) {$\ast$};
  \node [left of = e-ast, node distance=2mm] (left-bracket) {(};
  \node [right of = u-ast, node distance=2mm] (right-bracket2) {)};
  \coordinate (center) at ($(klad)!0.5!(u)$);
  \node (arrow)   [below of = center,node distance=9mm]  {$\downarrow$};
\end{tikzpicture}\bigskip\bigskip
\ex\gll kl\tikz\node(klad2){a};d- \tikz\node(e2){\cancel{e}};- \tikz\node(u2){u}; \\
put {\PRS} {1\SG}\\
 \begin{tikzpicture}[overlay,remember picture]
	 \node [above of = klad2, node distance=4mm] (klad2-ast) {$\ast$};
  \node [above of = e2, node distance=4mm] (e2-noast) { };
  \node [above of = u2, node distance=4mm] (u2-ast) {$\ast$};
  \node [left of = e2-noast, node distance=2mm] (left-bracket) {(};
  \node [right of = u2-ast, node distance=2mm] (right-bracket2) {)};
  \coordinate (center2) at ($(klad2)!0.5!(u2)$);
  \node (arrow)   [below of = center2,node distance=9mm]  {$\downarrow$};
\end{tikzpicture}\bigskip\bigskip
\ex\gll kl\tikz\node(klad3){a};d- \tikz\node(e3){}; \tikz\node(u3){u}; \\
put {\PRS} {1\SG}\\
 \begin{tikzpicture}[overlay,remember picture]
	 \node [above of = klad3, node distance=4mm] (klad3-ast) {$\ast$};
  \node [above of = e3, node distance=4mm] (e3-noast) { };
  \node [above of = u3, node distance=4mm] (u3-ast) {$\ast$};
  \node [above of = u3-ast, node distance=4mm] (u3-ast2) {$\ast$};
  \node [left of = e3-noast, node distance=2mm] (left-bracket) {(};
  \node [right of = u3-ast, node distance=2mm] (right-bracket2) {)};
\end{tikzpicture}
\z
\z

\noindent A possible alternative would be that the accent is deleted together with the vowel but the {1\SG} ending is accented, drawing the stress. I will argue, however, that the interaction of accents surviving after hiatus resolution can account for the {1\SG} pattern that would be inexplicable otherwise.

\subsection{Intermediate summary}

In this section I have discussed and motivated my background assumptions about the segmental and accentual properties of Russian tense and agreement morphemes. Segmentally, Russian tense and agreement markers were taken to coincide with their surface forms except for the present-tense suffix \Affix{e}, which surfaces as \Affix{ʲo} (palatalizing [o]) under stress, and the 3{\PL} suffix (which I take to be \Affix{nt}). Sequences of two vowels are resolved, following \citet{Jakobson1948}, by the deletion of the first one (unless the first vowel is an \textit{i}, which turns into a glide before a vowel other than \textit{i}).

The examination of the finite paradigms of athematic verbs, alongside with their infinitive and passive past participle forms, makes it possible to determine the accentual properties of various inflectional suffixes:

\begin{itemize}
    \item the present-tense suffix \Affix{e} and the feminine singular suffix \Affix{a} are accented
    \item the plural suffix \Affix{i} is unaccented (and the same is true for the masculine (\Affix{ŭ}) and neuter (\Affix{o}) suffixes, which show the same accentual behavior; for minor lexically-conditioned variation see \citealt{Melvold1989} and \citealt{MarklundSharapova2000})
    \item the past-tense suffix \Affix{l} is unaccented but retracting (forcing the realization of the stress of a post-accenting stem on the stem-final syllable)
\end{itemize}

\noindent Since in the Halle-Idsardi system feet are left-headed, the deletion of an accented vowel yields rightward stress shift. This prediction is correct for the class of verbs in \tabref{mat:tab:Accents-i}-b, characterized by the post-stem stress pattern. I will now argue that the accentual patterns in \tabref{mat:tab:Accents-i}-c and d cannot be explained by the mechanisms postulated so far. 

\section{Thematic verbs and the {1\SG} pattern}\label{mat:sec:ThematicVerbs}

Except for the two athematic stems in fn. \ref{mat:fn:FifthClass}, the {1\SG} stress pattern in the present is only attested in thematic verbs. Importantly, it can be found with several thematic suffixes, as shown in \tabref{mat:tab:InteractionSemelfactiveNu} and \tabref{mat:tab:InteractionThematicE}.

\begin{table}
\caption{Accentual interaction in thematic verbs, illustrated for the semelfactive suffix \Affix{nu}}
\label{mat:tab:InteractionSemelfactiveNu}
 \begin{tabularx}{\textwidth}{lL{2.36cm}Z{1.8cm}Z{1.54cm}Z{2.08cm}Z{2.08cm}} 
  \lsptoprule
    &&accented\linebreak{\PRS-3\SG} &accented\linebreak{\PRS-1\SG}&accented\linebreak{\PST-\FEM.\SG}  &unaccented\linebreak{\PST-\PL}    \\
  \midrule
    a.  &stem:\newline\textit{-top-} ‘stomp’  
        &tóp-n-e-t    &tóp-n-u    &tóp-n-u-l-a  &tóp-n-u-l-i \\\addlinespace[5pt]
    b.  &post-stem:\newline\textit{-max-} ‘wave’ 
        &max-nʲ-ó-t  &max-n-ú     &max-n-ú-l-a   &max-n-ú-l-i  \\\addlinespace[5pt]
    c.  &{1\SG}:\newline\textit{-obman-} ‘cheat’ 
        &obmá-n-e-t    &obma-n-ú    &obma-n-ú-l-a   &obma-n-ú-l-i   \\
  \lspbottomrule
 \end{tabularx}
\end{table}

\begin{table}
\caption{Accentual interaction in thematic verbs, illustrated for the thematic suffix \Affix{e}}
\label{mat:tab:InteractionThematicE}
 \begin{tabularx}{\textwidth}{lL{2.36cm}Z{1.8cm}Z{1.54cm}Z{2.08cm}Z{2.08cm}} 
  \lsptoprule
    &&accented\linebreak{\PRS-3\SG} &accented\linebreak{\PRS-1\SG}&accented\linebreak{\PST-\FEM.\SG}  &unaccented\linebreak{\PST-\PL}    \\
  \midrule
    a.  &stem:\newline\textit{-vid-} ‘see’ 
        &víd-i-t    &víž-u    &víd-e-l-a  &víd-e-l-i \\\addlinespace[5pt]
    b.  &post-stem:\newline\textit{-vel-} ‘order’ 
        &vel-í-t  &velʲ-ú     &vel-é-l-a   &vel-é-l-i  \\\addlinespace[5pt]
    c.  &{1\SG}:\newline\textit{-vert-} ‘spin’ 
        &vért-i-t    &verč-ú    &vert-é-l-a   &vert-é-l-i   \\
  \lspbottomrule
 \end{tabularx}
\end{table}

\newpage
The fact that first-conjugation verbs (\textit{-ʲo-} in the present tense, exemplified by \tabref{mat:tab:InteractionSemelfactiveNu}) and second-conjugation verbs (/i/ in the present tense, exemplified by \tabref{mat:tab:InteractionThematicE}) can both exhibit the {1\SG} pattern suggests that it is linked not to a given concrete present-tense suffix, but to the morphological feature [{$–$\PST}]. While I will not make such an assumption, the discussion of the present-tense allomorphs will be postponed until \sectref{mat:sec:AccentualConflict}, and in the remainder of this section I will address the thematic suffix, arguing that it plays a crucial role in the emergence of the {1\SG} pattern.

\subsection{The accentuation of thematic suffixes and the {1\SG} pattern}\label{mat:subsec:ThematicAnd1SG}

Since the vowel of the thematic suffix either is deleted or turns into a glide before the vowel of the present-tense suffix, the underlying accentuation of the thematic suffix must be established on the basis of the past tense, where it is left intact. The Basic Accentuation Principle \REF{mat:ex:BAP} means that accented L-stems can be identified by systematic stem stress in \tabref{mat:tab:InteractionSemelfactiveNu}-a and \tabref{mat:tab:InteractionThematicE}-a. If the thematic suffix were unaccented, we would expect to find the varying pattern in the past tense of some verbs, indicating unaccented L-stems, as in \tabref{mat:tab:InteractionAthematic}.\footnote{\label{mat:fn:UnnaccentedA}One thematic suffix, surfacing as \Affix{a} in the past tense and undetectable in the present, is unaccented. Evidence for this comes from the variable position of the stress in its past tense (e.g., \textit{lgalá}/\textit{lgáli} ‘lied.{\FEM.\SG}/{\PL}’). In the present the suffix is undetectable (\textit{lgu}/\textit{lžʲot} ‘lie.{1\SG}/{3\SG}’) due to hiatus resolution before the vocalic present-tense suffix.} The fact that this pattern is unattested in the past tense of verbs exhibiting the {1\SG} pattern strongly suggests that the thematic suffix must 
introduce an accent.

Because a sequence of two parentheses without any stress-bearing elements between them is simplified to a single parenthesis (see \sectref{mat:subsec:HalleIdsardi}), post-accenting stems in \REF{mat:ex:MetricalPostaccenting} are simplified to the same representation as unaccented stems \REF{mat:ex:MetricalUnaccented} by the time a tense suffix is added, so the difference between the {1\SG} pattern and the post-stem pattern in the present is not expected to follow from the accentuation of the L-stem.

\ea\label{mat:ex:MetricalUnaccentedPostaccenting}
\ea\label{mat:ex:MetricalUnaccented} unaccented L-stem\\\bigskip
\tikzstyle{every picture}+=[remember picture, inner sep=0pt,baseline, anchor=base,execute at begin node=\strut]
\tikz\node(root){√-};\hspace{.25cm}\tikz\node(th){\THEM};\hspace{.25cm}{\PST}\hspace{.25cm}\tikz\node(fsg){{\FEM.\SG}};
 \begin{tikzpicture}[overlay,remember picture]
	 \node [above of = root, node distance=4mm] (root-ast) {$\ast$};
  \node [above of = th, node distance=4mm] (th-ast) {$\ast$};
  \node [above of = fsg, node distance=4mm] (fsg-ast) {$\ast$};
  \node [left of = th-ast, node distance=2mm] (left-bracket) {(};
  \node [left of = fsg-ast, node distance=2mm] (left-bracket) {(};
  \node [right of = fsg-ast, node distance=2mm] (right-bracket) {)};
\end{tikzpicture}
\ex\label{mat:ex:MetricalPostaccenting} post-accenting L-stem\\\bigskip
\tikzstyle{every picture}+=[remember picture, inner sep=0pt,baseline, anchor=base,execute at begin node=\strut]
\tikz\node(root2){√-};\hspace{.25cm}\tikz\node(th2){\THEM};\hspace{.25cm}{\PST}\hspace{.25cm}\tikz\node(fsg2){{\FEM.\SG}};\hspace{1cm}$\rightarrow$\hspace{1cm}\tikz\node(root3){√-};\hspace{.25cm}\tikz\node(th3){\THEM};\hspace{.25cm}{\PST}\hspace{.25cm}\tikz\node(fsg3){{\FEM.\SG}};
 \begin{tikzpicture}[overlay,remember picture]
	 \node [above of = root2, node distance=4mm] (root2-ast) {$\ast$};
  \node [above of = th2, node distance=4mm] (th2-ast) {$\ast$};
  \node [above of = fsg2, node distance=4mm] (fsg2-ast) {$\ast$};
  \node [right of = root2-ast, node distance=2mm] (left-bracket) {(};
  \node [left of = th2-ast, node distance=2mm] (left-bracket) {(};
  \node [left of = fsg2-ast, node distance=2mm] (left-bracket) {(};
  \node [right of = fsg2-ast, node distance=2mm] (right-bracket) {)};
  \node [above of = root3, node distance=4mm] (root3-ast) {$\ast$};
  \node [above of = th3, node distance=4mm] (th3-ast) {$\ast$};
  \node [above of = fsg3, node distance=4mm] (fsg3-ast) {$\ast$};
  \node [left of = th3-ast, node distance=2mm] (left-bracket) {(};
  \node [left of = fsg3-ast, node distance=2mm] (left-bracket) {(};
  \node [right of = fsg3-ast, node distance=2mm] (right-bracket) {)};
\end{tikzpicture}
\z
\z

\noindent It therefore seems reasonable to assume (\citealt[328]{Halle1973}, \citealt[291]{Melvold1989}, \citealt[124]{Idsardi1992}, \citealt[114--117]{Gladney1995}, \citealt{Feldstein2015}, among others) that the {1\SG} present-tense pattern is due to something not considered so far.

\subsection{The role of the thematic suffix}\label{mat:subsec:Role}

While \citet{Redkin1965} and \citet{Zaliznjak1985} claim that there is no correlation between the thematic suffix and stress, \citet{Slioussar2012} shows that the three stress patterns in Tables \ref{mat:tab:Accents-i}, \ref{mat:tab:InteractionSemelfactiveNu}, and \ref{mat:tab:InteractionThematicE} are not equally productive in all verb classes and that some thematic suffixes do not produce the {1\SG} pattern at all, as shown in \tabref{mat:tab:StressAndThematic}. (Examples are provided for all thematic classes that can give rise to the {1\SG} pattern; for non-productive thematic classes the numbers given represent the number of unprefixed verbs in that class.\footnote{\label{mat:fn:j-final-verbs}For the 21 \textit{j}-final verbs with the theme \Affix{a} in the past, the shape of the stem makes it impossible to determine if in the present this theme is deleted before the present-tense suffix (cf. fn. \ref{mat:fn:UnnaccentedA}) or undergoes a readjustment rule (cf. \citealt{Matushansky2023a}) turning it into [i] (which would then turn into a glide). The same issue arises for the two verbs with OCS palatalization of the final consonant cluster ([žd] arising from underlying [dj]), \textit{žáždatʲ} ‘to thirst’ and the non-standard \textit{stráždatʲ} ‘to suffer’.

\ea
\ea tájatʲ/táju/táet ‘melt.{\INF}/{1\SG}/{3\SG}’
\ex stráždatʲ/stráždu/stráždet ‘suffer.{\INF}/{1\SG}/{3\SG}’ (Modern Russian \textit{stradátʲ}/\textit{stradáju}/\textit{stradáet}, literary variant with the \Affix{a}/\Affix{i} thematic suffix \textit{stradátʲ}/\textit{stráždu}/\textit{stráždet})
\z
\z

\noindent Nineteen of them have stem stress and, though I have assigned them to the \Affix{a}/\Affix{i} class, their uncertain status is indicated by parentheses in the table. The two \textit{j}-final verbs with systematic post-stem stress, \textit{smejátʲsʲa} ‘to laugh’ and the archaic \textit{vopijátʲ} ‘to clamor’, have been assigned to the \Affix{a}/\Affix{Ø} class because no verb with a detectable \Affix{a}/\Affix{i} thematic suffix shows the post-stem stress pattern in the present tense.})

%\resizebox{\textwidth}{!}{\begin{tabularx}{1.11\textwidth}{lXL{2.9cm}L{2.86cm}L{3.01cm}} 

\begin{table}%Table needs adjusting
\caption{Stress and thematic suffixes}
\label{mat:tab:StressAndThematic}
 \fittable{\begin{tabularx}{1.08\textwidth}{lL{2cm}L{1.35cm}L{1.75cm}L{1.55cm}L{1.8cm}R{1.75cm}} 
  \lsptoprule
    &theme\linebreak({\PST}/{\PRS}) &{\PRS.1\SG}  &{\PRS.2\SG}  &{\INF}  &gloss   &{{1\SG}\linebreak pattern}    \\
  \midrule
\textbf{a.}  &\textbf{a/aj}    &{čit-{á}j-u}    &čit-áj-e-šʲ    &čit-{á}-tʲ     &‘read’     &$0$/${\infty}$\\\addlinespace[5pt]
\textbf{b.}  &\textbf{e/ej}   &bel-éj-u       &bel-éj-e-šʲ    &bel-é-tʲ       &‘be white’ &$0$/${\infty}$\\\addlinespace[5pt]
\textbf{c.}  &\textbf{nu/n}\linebreak\textbf{(semelfactive)}    &tolk-n-ú       &tolk-nʲ-ó-šʲ   &tolk-nú-tʲ & ‘push’ &$6$/${\infty}$\\\addlinespace[5pt]
d.  &none or Ø & mog-ú & m{ó}ž-e\nobreakdash-šʲ & m{ó}čʲ & ‘be able’ & $2$/$84$\\\addlinespace[5pt]
e.  &a/Ø    &ser-ú  &sér-e-šʲ   &sr-á-tʲ    &‘shit (dial.)’     &$1$--$2$/$20$ ($39$)\\\addlinespace[5pt]
f.  &a/i    &piš-ú  &píš-e-šʲ   &pis-á-tʲ   &‘write’    &$60$/$103$ ($84$)\\\addlinespace[5pt]
g.  &o/i    &kolʲ-ú &kól-e-šʲ   &kol-ó-tʲ   &‘stab’     &$5$/$5$\\\addlinespace[5pt]
h.  &nu/n\linebreak(mutative)&gíb-n-u &gíb-n-e-šʲ    &gíb-nu-tʲ  &‘perish’   &$0$/$60$\\\midrule
\textbf{i.}  &\textbf{i}      &proš-ú &prós-i-šʲ  &pros-í-tʲ  &‘ask’      &$23\%$\\\addlinespace[5pt]
j.  &e/Ø    &verč-ú &vért-i-šʲ  &vert-é-tʲ  &‘turn’     &$6$/$83$\\
  \lspbottomrule
 \end{tabularx}}
\end{table}

Several empirical generalizations can be established based on the patterns in \tabref{mat:tab:StressAndThematic}. Of the four productive verb classes in Russian (a, b, c, i; the unproductive class f also contains all the verbs derived with the productive suffix \Affix{ow}) the {1\SG} pattern is productive in one (1556 out of the 6875 \textit{i}-verbs in \citepossalt{Zaliznjak1977} dictionary, according to the calculations in \citealt{Slioussar2012}). It never occurs with the thematic suffixes surfacing as \Affix{aj} and \Affix{ej} in the present, which suggests that it is dependent on the deletion of a vowel. However, verbs derived with the pre-accenting mutative suffix \Affix{nu} (\tabref{mat:tab:StressAndThematic}-h) or with the unaccented thematic suffix \Affix{a} that is deleted in the present tense (\tabref{mat:tab:StressAndThematic}-e) also do not give rise to the {1\SG} pattern.

Given these facts it is reasonable to assume that the {1\SG} pattern is linked to the deletion of a vowel that introduces an accent. Support for this hypothesis comes from the fact that with the accent-bearing vocalic suffixes \Affix{i} and \Affix{a}/\Affix{i} (as well as its allomorph \Affix{o}/\Affix{i}) the {1\SG} pattern is systematic, and with two more accent-bearing thematic suffixes it is marginally possible: with the semelfactive first-conjugation \Affix{nu} (six verbs to be discussed in section \sectref{mat:subsubsec:AthematicVerbs}) and with the second-conjugation \Affix{e} (five verbal roots, see \sectref{mat:subsubsec:FiveEVerbs}).

The most important empirical generalization to be drawn from \tabref{mat:tab:StressAndThematic} is that the {1\SG} pattern is systematically available with some thematic suffixes (\Affix{i}, \Affix{a}/\Affix{i}) and exceptional with others (\Affix{nu}, \Affix{e}), which suggests that the properties of the thematic suffix play a role in determining which stress pattern the verbal stem (L-stem + thematic suffix) gives rise to.\footnote{\citet[28, 380]{Zaliznjak1985} offers a number of lexical generalizations over both patterns and points out that the systematic post-stem pattern is characteristic of the more archaic strata of the vocabulary, providing such near-minimal prefixed verb pairs as the standard \textit{razbužú}/\textit{razbúdit} ‘awaken.{1\SG}/{3\SG}’, \textit{počinʲú}/\textit{počínit} ‘repair.{1\SG}/{3\SG}’ ({1\SG} pattern) vs. the literary \textit{učinʲú}/\textit{učinít} ‘initiate.{1\SG}/{3\SG}’, \textit{vozbužú}/\textit{vozbudít} ‘arouse.{1\SG}/{3\SG}’ (post-stem stress). The prefixes themselves, however, cannot be regarded as the reason for these contrasts.} Nonetheless, as will be presently shown, it cannot be the thematic suffix itself that is responsible, since in no verb class is the {1\SG} stress pattern the only one available.

\subsection{Treating retraction as the lexical property of the stem}
\largerpage
As discussed in \sectref{mat:subsec:HalleIdsardi}, when two parentheses appear in a sequence with no asterisk in between, one is deleted, reflecting the fact that a post-accenting stem and an unaccented stem followed by an accented suffix yield the same surface outcome \REF{mat:ex:MetricalUnaccentedPostaccenting}. This is also the configuration that arises when the vowel of the thematic suffix is followed by the present-tense suffix, as in \REF{mat:ex:Nu-e-t-Derivation-1}. When the thematic vowel is deleted, a sequence of two accents is created that should be resolved into one:\footnote{I use the semelfactive suffix \Affix{nu} for an example despite the exceptionality of the {1\SG} pattern with it because the consonant remaining after hiatus resolution makes it easier to abstract away from the L-stem. While the suffix is represented here as accented (since it also makes for easier representations), I will revise this assumption later.}\bigskip

\ea\label{mat:ex:Nu-e-t-Derivation-1}%The author requests that this example not be split across two pages
\ea\tikzstyle{every picture}+=[remember picture, inner sep=0pt,baseline, anchor=base,execute at begin node=\strut]
\gll \tikz\node(root){√-}; n\tikz\node(nu){u};~ \tikz\node(e){e}; \tikz\node(t){t}; \\
{} {\THEM} {\PRS} {3\SG}\\
 \begin{tikzpicture}[overlay,remember picture]
	 \node [above of = root, node distance=4mm] (root-ast) {$\ast$};
  \node [above of = nu, node distance=4mm] (nu-ast) {$\ast$};
  \node [above of = e, node distance=4mm] (e-ast) {$\ast$};
  \node [left of = nu-ast, node distance=2mm] (left-bracket) {(};
  \node [left of = e-ast, node distance=2mm] (left-bracket) {(};
  \node [right of = e-ast, node distance=2mm] (right-bracket2) {)};
  \coordinate (center) at ($(root)!0.5!(t)$);
  \node (arrow)   [below of = center,node distance=9mm]  {$\downarrow$};
\end{tikzpicture}\bigskip\bigskip
\ex
\gll \tikz\node(root2){√-}; n\tikz\node(nu2){\cancel{u}};~ \tikz\node(e2){e}; \tikz\node(t2){t}; \\
{} {\THEM} {\PRS} {3\SG}\\
 \begin{tikzpicture}[overlay,remember picture]
	 \node [above of = root2, node distance=4mm] (root2-ast) {$\ast$};
  \node [above of = nu2, node distance=4mm] (nu2-ast) {\strut};
  \node [above of = e2, node distance=4mm] (e2-ast) {$\ast$};
  \node [left of = nu2-ast, node distance=2mm] (left-bracket) {(};
  \node [left of = e2-ast, node distance=2mm] (left-bracket) {(};
  \node [right of = e2-ast, node distance=2mm] (right-bracket2) {)};
  \coordinate (center) at ($(root2)!0.5!(t2)$);
  \node (arrow)   [below of = center,node distance=9mm]  {$\downarrow$};
\end{tikzpicture}\bigskip\bigskip
\ex
\gll \tikz\node(root3){√-}; n~ \tikz\node(e3){e}; \tikz\node(t3){t}; \\
{} {\THEM} {\PRS} {3\SG}\\
 \begin{tikzpicture}[overlay,remember picture]
	 \node [above of = root3, node distance=4mm] (root3-ast) {$\ast$};
  \node [above of = e3, node distance=4mm] (e3-ast) {$\ast$};
  \node [left of = e3-ast, node distance=2mm] (left-bracket) {(};
  \node [right of = e3-ast, node distance=2mm] (right-bracket2) {)};
\end{tikzpicture}
\z
\z

\noindent It is easy to see that systematic post-stem stress is predicted here, and the addition of a vocalic suffix (such as the {1\SG} \Affix{u}) instead of a consonantal one (like the {3\SG} \Affix{t}) does not change the outcome. This is why \citet[291]{Melvold1989}, following \citet{Halle1973}, proposes that the stems giving rise to the {1\SG} pattern are marked to undergo retraction in all forms of the present tense except {1\SG}, where the present-tense suffix is null (see fn. \ref{mat:fn:Null}). \citet[124]{Idsardi1992} improves upon this by proposing that retraction fails in the {1\SG} because its trigger, the present-tense marker, is deleted before another vowel. \citet{Halle1997} encodes retraction by inserting an additional parenthesis before the trigger morpheme:\bigskip

\ea\label{mat:ex:Nu-e-t-Derivation-2}%The author requests that this example not be split across two pages
\ea\tikzstyle{every picture}+=[remember picture, inner sep=0pt,baseline, anchor=base,execute at begin node=\strut]
\gll \tikz\node(root){√-}; n\tikz\node(nu){u};~ \tikz\node(e){e}; \tikz\node(t){t}; \\
{} {\THEM} {\PRS} {3\SG}\\
 \begin{tikzpicture}[overlay,remember picture]
	 \node [above of = root, node distance=4mm] (root-ast) {$\ast$};
  \node [above of = nu, node distance=4mm] (nu-ast) {$\ast$};
  \node [above of = e, node distance=4mm] (e-ast) {$\ast$};
  \node [left of = nu-ast, node distance=2mm] (left-bracket) {(};
  \node [left of = e-ast, node distance=2mm] (left-bracket) {(};
  \coordinate (center) at ($(root)!0.5!(t)$);
  \node (arrow)   [below of = center,node distance=9mm]  {$\downarrow$};
\end{tikzpicture}\bigskip\bigskip
\ex
\gll \tikz\node(root2){√-}; n\tikz\node(nu2){\cancel{u}};~ \tikz\node(e2){e}; \tikz\node(t2){t}; \\
{} {\THEM} {\PRS} {3\SG}\\
 \begin{tikzpicture}[overlay,remember picture]
	 \node [above of = root2, node distance=4mm] (root2-ast) {$\ast$};
  \node [above of = nu2, node distance=4mm] (nu2-ast) {\strut};
  \node [above of = e2, node distance=4mm] (e2-ast) {$\ast$};
  \node [left of = e2-ast, node distance=2mm] (left-bracket) {(};
  \coordinate (center) at ($(root2)!0.5!(t2)$);
  \node (arrow)   [below of = center,node distance=9mm]  {$\downarrow$};
\end{tikzpicture}\bigskip\bigskip
\ex
\gll \tikz\node(root3){√-}; n~ \tikz\node(e3){e}; \tikz\node(t3){t}; \\
{} {\THEM} {\PRS} {3\SG}\\
 \begin{tikzpicture}[overlay,remember picture]
	 \node [above of = root3, node distance=4mm] (root3-ast) {$\ast$};
  \node [above of = root3-ast, node distance=4mm] (root3-ast2) {$\ast$};
  \node [above of = e3, node distance=4mm] (e3-ast) {$\ast$};
  \node [left of = root3-ast, node distance=2mm] (left-bracket) {(};
  \node [left of = e3-ast, node distance=2mm] (left-bracket) {(};
\end{tikzpicture}
\z
\z

Even though the present-tense suffix is deleted before the {1\SG} ending only in first-conjugation verbs (in second-conjugation verbs there is glide formation in the {1\SG} (fn. \ref{mat:fn:Iotation})), the connection between the stress failing to retract and a vocalic ending is real and supported by independent evidence. As \citet{Feldstein2015} points out, there exist two more forms with the same final stress as in the {1\SG}: the imperative (surface [i]) and the present tense gerund (surface [ʲa]):\footnote{Stress in the active present participle generally patterns with non-{1\SG}, but sometimes doesn’t (e.g., \textit{učúsʲ}/\textit{účitsʲa} ‘study.{1\SG}/{3\SG}’ vs. \textit{učáščijsʲa} ‘studying.{\MASC.\SG}’, see also \citealt[29, 77]{Zaliznjak1985}).}

\ea
\ea vert-í ‘spin.{\IMP}’, vertʲ-á ‘spin.\textsc{ger}’ (cf. verčú/vértit ‘spin.{1\SG}/{3\SG}’) 
\ex obman-í ‘cheat.{\IMP}’ (cf. obmanú/obmánet ‘cheat.{1\SG}/{3\SG}’)
\ex lʲub-í ‘love.{\IMP}’, lʲubʲ-á ‘love.\textsc{ger}’ (cf. lʲublʲú/lʲúbit ‘love.{1\SG}/{3\SG}’)
\z
\z

\noindent While \citeauthor{Feldstein2015} simply points out that non-retracting forms all have a simple vowel ending of the type -V\#, \citeauthor{Idsardi1992}’s proposal makes retraction failure phonologically predictable, deriving it from hiatus resolution. For this proposal to succeed, however, it is necessary for retraction to happen after hiatus resolution:\bigskip

\ea\label{mat:ex:Nu-e-u-Derivation}%Incorrect glosses?%The author requests that this example not be split across two pages
\ea\tikzstyle{every picture}+=[remember picture, inner sep=0pt,baseline, anchor=base,execute at begin node=\strut]
\gll \tikz\node(root){√-}; n\tikz\node(nu){u};~~~ \tikz\node(e){e}; \tikz\node(u){u}; \\
{} {\THEM} {\PRS} {3\SG}\\
 \begin{tikzpicture}[overlay,remember picture]
	 \node [above of = root, node distance=4mm] (root-ast) {$\ast$};
  \node [above of = nu, node distance=4mm] (nu-ast) {$\ast$};
  \node [above of = e, node distance=4mm] (e-ast) {$\ast$};
  \node [above of = u, node distance=4mm] (u-ast) {$\ast$};
  \node [left of = nu-ast, node distance=2mm] (left-bracket) {(};
  \node [left of = e-ast, node distance=2mm] (left-bracket) {(};
  \node [right of = u-ast, node distance=2mm] (right-bracket2) {)};
  \coordinate (center) at ($(root)!0.5!(u)$);
  \node (arrow)   [below of = center,node distance=9mm]  {$\downarrow$};
\end{tikzpicture}\bigskip\bigskip
\ex
\gll \tikz\node(root2){√-}; n\tikz\node(nu2){\cancel{u}};~~~ \tikz\node(e2){e}; \tikz\node(u2){u}; \\
{} {\THEM} {\PRS} {3\SG}\\
 \begin{tikzpicture}[overlay,remember picture]
	 \node [above of = root2, node distance=4mm] (root2-ast) {$\ast$};
  \node [above of = nu2, node distance=4mm] (nu2-ast) {\strut};
  \node [above of = e2, node distance=4mm] (e2-ast) {$\ast$};
  \node [above of = u2, node distance=4mm] (u2-ast) {$\ast$};
  \node [left of = nu2-ast, node distance=2mm] (left-bracket) {(};
  \node [left of = e2-ast, node distance=2mm] (left-bracket) {(};
  \node [right of = u2-ast, node distance=2mm] (right-bracket2) {)};
  \coordinate (center) at ($(root2)!0.5!(u2)$);
  \node (arrow)   [below of = center,node distance=9mm]  {$\downarrow$};
\end{tikzpicture}\bigskip\bigskip
\ex\label{mat:ex:Nu-e-u-Derivation-c}
\gll \tikz\node(root3){√-}; n\tikz\node(nu3){\cancel{u}};~~~ \tikz\node(e3){\cancel{e}}; \tikz\node(u3){u}; \\
{} {\THEM} {\PRS} {3\SG}\\
 \begin{tikzpicture}[overlay,remember picture]
	 \node [above of = root3, node distance=4mm] (root3-ast) {$\ast$};
  \node [above of = nu3, node distance=4mm] (nu3-ast) {\strut};
  \node [above of = e3, node distance=4mm] (e3-ast) {\strut};
  \node [above of = u3, node distance=4mm] (u3-ast) {$\ast$};
  \node [left of = nu3-ast, node distance=2mm] (left-bracket) {(};
  \node [left of = e3-ast, node distance=2mm] (left-bracket2) {(};
  \node [right of = u3-ast, node distance=2mm] (right-bracket2) {)};
\end{tikzpicture}
\z
\z

\noindent Assuming that the deletion of a vowel removes it from the metrical tier but retains the accent, it is to the representation in \REF{mat:ex:Nu-e-u-Derivation-c} that stress rules apply. While the first parenthesis is deleted by regular processes (since no metrical element follows), \citeauthor{Idsardi1992}’s claim is that the deletion of the present-tense suffix makes it impossible for it to trigger retraction.

The assumption that the past-tense suffix \Affix{l} is retracting (\sectref{mat:subsec:URofSuffixes}) makes it impossible to explain retraction failure in the {1\SG} by the fact that a deleted suffix is removed from the metrical tier: The asyllabic past-tense suffix is not present on the metrical tier either. Furthermore, the restrictions both on the verbal classes exhibiting the pattern (only with deleted accent-bearing thematic suffixes) and on the pattern itself (failing before simple vocalic endings) suggest that it is not due to an arbitrary lexical property of the stem. In the next section I will introduce an explanation for retraction failure with simple vocalic endings: I will propose that the {1\SG} pattern results from induced unstressability.

\section{The {1\SG} pattern as induced unstressability of the present-tense suffix}\label{mat:sec:Unstressability}

I begin this section with an assumption. Suppose that with some verbs the present-tense suffix is not represented on the metrical tier. Once again I use a \Affix{nu} verb to illustrate the matter and I will assume an unaccented L-stem because, as discussed in \sectref{mat:subsec:ThematicAnd1SG}, the combination of a post-accenting stem and an accented suffix produces the same result as that of an unaccented stem and an accented suffix.

Starting out with the {1\SG} form, the assumption that the present-tense suffix is absent from the metrical tier \REF{mat:ex:Nu-e-u-Derivation-2-a} gives rise to word-final stress once the vowel of the thematic suffix is deleted before the vowel of the present-tense suffix \REF{mat:ex:Nu-e-u-Derivation-2-b}. The deletion of the present-tense suffix \REF{mat:ex:Nu-e-u-Derivation-2-c} yields the correct surface form:\bigskip

\ea\label{mat:ex:Nu-e-u-Derivation-2}%Incorrect glosses?%The author requests that this example not be split across two pages
\ea\label{mat:ex:Nu-e-u-Derivation-2-a}\tikzstyle{every picture}+=[remember picture, inner sep=0pt,baseline, anchor=base,execute at begin node=\strut]
\gll \tikz\node(root){√-}; n\tikz\node(nu){u};~~~ \tikz\node(e){e}; \tikz\node(u){u}; \\
{} {\THEM} {\PRS} {3\SG}\\
 \begin{tikzpicture}[overlay,remember picture]
	 \node [above of = root, node distance=4mm] (root-ast) {$\ast$};
  \node [above of = nu, node distance=4mm] (nu-ast) {$\ast$};
  \node [above of = u, node distance=4mm] (u-ast) {$\ast$};
  \node [left of = nu-ast, node distance=2mm] (left-bracket) {(};
  \node [right of = u-ast, node distance=2mm] (right-bracket2) {)};
  \coordinate (center) at ($(root)!0.5!(u)$);
  \node (arrow)   [below of = center,node distance=9mm]  {$\downarrow$};
\end{tikzpicture}\bigskip\bigskip
\ex\label{mat:ex:Nu-e-u-Derivation-2-b}
\gll \tikz\node(root2){√-}; n\tikz\node(nu2){\cancel{u}};~~~ \tikz\node(e2){e}; \tikz\node(u2){u}; \\
{} {\THEM} {\PRS} {3\SG}\\
 \begin{tikzpicture}[overlay,remember picture]
	 \node [above of = root2, node distance=4mm] (root2-ast) {$\ast$};
  \node [above of = nu2, node distance=4mm] (nu2-ast) {\strut};
  \node [above of = u2, node distance=4mm] (u2-ast) {$\ast$};
  \node [left of = nu2-ast, node distance=2mm] (left-bracket) {(};
  \node [right of = u2-ast, node distance=2mm] (right-bracket2) {)};
  \coordinate (center) at ($(root2)!0.5!(u2)$);
  \node (arrow)   [below of = center,node distance=9mm]  {$\downarrow$};
\end{tikzpicture}\bigskip\bigskip
\ex\label{mat:ex:Nu-e-u-Derivation-2-c}
\gll \tikz\node(root3){√-}; n\tikz\node(nu3){\cancel{u}};~~~ \tikz\node(e3){\cancel{e}}; \tikz\node(u3){u}; \\
{} {\THEM} {\PRS} {3\SG}\\
 \begin{tikzpicture}[overlay,remember picture]
	 \node [above of = root3, node distance=4mm] (root3-ast) {$\ast$};
  \node [above of = nu3, node distance=4mm] (nu3-ast) {\strut};
  \node [above of = u3, node distance=4mm] (u3-ast) {$\ast$};
  \node [left of = nu3-ast, node distance=2mm] (left-bracket) {(};
  \node [right of = u3-ast, node distance=2mm] (right-bracket2) {)};
\end{tikzpicture}
\z
\z

\noindent Asyllabic endings are predicted to exhibit different behavior. After the deletion of the thematic vowel \REF{mat:ex:Nu-e-u-Derivation-3-b} the two parentheses at the right edge of the word are not followed by any metrical material. As discussed in \sectref{mat:subsec:HalleIdsardi}, this configuration yields leftward stress shift, which I implement, like in \REF{mat:ex:Nu-e-t-Derivation-2}, by doubling the last left parenthesis. Stem-final stress \REF{mat:ex:Nu-e-u-Derivation-3-c} is therefore correctly predicted with asyllabic endings, and \citeauthor{Feldstein2015}’s generalization (stress retraction in the absence of a vocalic ending) is explained:\bigskip

\ea\label{mat:ex:Nu-e-t-Derivation-3}%The author requests that this example not be split across two pages
\ea\label{mat:ex:Nu-e-u-Derivation-3-a}\tikzstyle{every picture}+=[remember picture, inner sep=0pt,baseline, anchor=base,execute at begin node=\strut]
\gll \tikz\node(root){√-}; n\tikz\node(nu){u};~~~ \tikz\node(e){e}; \tikz\node(t){t}; \\
{} {\THEM} {\PRS} {3\SG}\\
 \begin{tikzpicture}[overlay,remember picture]
	 \node [above of = root, node distance=4mm] (root-ast) {$\ast$};
  \node [above of = nu, node distance=4mm] (nu-ast) {$\ast$};
  \node [above of = t, node distance=4mm] (t-ast) {\strut};
  \node [left of = nu-ast, node distance=2mm] (left-bracket) {(};
  \node [right of = t-ast, node distance=2mm] (right-bracket2) {)};
  \coordinate (center) at ($(root)!0.5!(t)$);
  \node (arrow)   [below of = center,node distance=9mm]  {$\downarrow$};
\end{tikzpicture}\bigskip\bigskip
\ex\label{mat:ex:Nu-e-u-Derivation-3-b}
\gll \tikz\node(root2){√-}; n\tikz\node(nu2){\cancel{u}};~~~ \tikz\node(e2){e}; \tikz\node(t2){t}; \\
{} {\THEM} {\PRS} {3\SG}\\
 \begin{tikzpicture}[overlay,remember picture]
	 \node [above of = root2, node distance=4mm] (root2-ast) {$\ast$};
  \node [above of = nu2, node distance=4mm] (nu2-ast) {\strut};
  \node [above of = t2, node distance=4mm] (t2-ast) {\strut};
  \node [left of = nu2-ast, node distance=2mm] (left-bracket) {(};
  \node [right of = t2-ast, node distance=2mm] (right-bracket2) {)};
  \coordinate (center) at ($(root2)!0.5!(t2)$);
  \node (arrow)   [below of = center,node distance=9mm]  {$\downarrow$};
\end{tikzpicture}\bigskip\bigskip
\ex\label{mat:ex:Nu-e-u-Derivation-3-c}
\gll \tikz\node(root3){√-}; n\tikz\node(nu3){\cancel{u}};~~~ \tikz\node(e3){\cancel{e}}; \tikz\node(t3){t}; \\
{} {\THEM} {\PRS} {3\SG}\\
 \begin{tikzpicture}[overlay,remember picture]
	 \node [above of = root3, node distance=4mm] (root3-ast) {$\ast$};
  \node [above of = nu3, node distance=4mm] (nu3-ast) {\strut};
  \node [above of = u3, node distance=4mm] (u3-ast) {\strut};
  \node [left of = root3-ast, node distance=2mm] (left-bracket) {(};
  \node [left of = nu3-ast, node distance=2mm] (left-bracket) {(};
\end{tikzpicture}
\z
\z

\noindent The final issue to be resolved is that of the {2\PL} ending \Affix{te}, which is wrongly predicted to be stressed:\bigskip

\ea\label{mat:ex:Nu-e-te-Derivation}%The author requests that this example not be split across two pages
\ea\label{mat:ex:Nu-e-te-Derivation-a}\tikzstyle{every picture}+=[remember picture, inner sep=0pt,baseline, anchor=base,execute at begin node=\strut]
\gll \tikz\node(root){√-}; n\tikz\node(nu){\cancel{u}};~~~ \tikz\node(e){e}; t\tikz\node(te){e}; \\
{} {\THEM} {\PRS} {2\PL}\\
 \begin{tikzpicture}[overlay,remember picture]
	 \node [above of = root, node distance=4mm] (root-ast) {$\ast$};
  \node [above of = nu, node distance=4mm] (nu-ast) {\strut};
  \node [above of = te, node distance=4mm] (te-ast) {$\ast$};
  \node [left of = nu-ast, node distance=2mm] (left-bracket) {(};
  \node [right of = te-ast, node distance=2mm] (right-bracket2) {)};
  \coordinate (center) at ($(root)!0.5!(te)$);
  \node (arrow)   [below of = center,node distance=9mm]  {$\downarrow$};
\end{tikzpicture}\bigskip\bigskip
\ex\label{mat:ex:Nu-e-te-Derivation-b}
\gll \tikz\node(root2){√-}; n\tikz\node(nu2){\cancel{u}};~~~ \tikz\node(e2){e}; t\tikz\node(te2){e}; \\
{} {\THEM} {\PRS} {3\SG}\\
 \begin{tikzpicture}[overlay,remember picture]
	 \node [above of = root2, node distance=4mm] (root2-ast) {$\ast$};
  \node [above left of = root2-ast, node distance=4mm] (cross) {✘};
  \node [above of = nu2, node distance=4mm] (nu2-ast) {\strut};
  \node [above of = te2, node distance=4mm] (te2-ast) {$\ast$};
  \node [above of = te2-ast, node distance=4mm] (te2-ast2) {$\ast$};
  \node [left of = nu2-ast, node distance=2mm] (left-bracket) {(};
  \node [right of = te2-ast, node distance=2mm] (right-bracket2) {)};
\end{tikzpicture}
\z
\z

\noindent I hypothesize, in accordance with historical evidence \citep[316--322]{Zaliznjak1985}, that the {2\PL} ending is either extrametrical (i.e., not represented on the metrical tier) or retracting. As a result, stress ends up on the stem-final syllable in all cells of the finite paradigms except for the {1\SG}.

The hypothesis that the {1\SG} pattern is due to induced unstressability of the present-tense suffix rather than to retraction explains why stress is stem-final with consonantal suffixes and final with vocalic ones but does not explain how unstressability is induced. Yet if the {1\SG} pattern is lexically triggered, it is not expected to be productive, contrary to fact (\sectref{mat:subsec:Role}); moreover, \textit{i}-verbs with post-stem stress are being continually shifted into it (see \citealt[29]{Zaliznjak1985}, \citeyear{Zaliznjak2019}, \citealt[57--59]{Feldstein1986}, \citealt[108]{Choi1996}, \citealt[132]{MarklundSharapova2000}, and \citealt[469]{Eskova2008}, \citeyear[469]{Eskova2014}). It is therefore desirable to derive induced unstressability from some independently motivated property of the L-stem, and stem accentuation (i.e., the lack of an accent vs. post-accentuation) seems the best candidate. In the next section I will provide some evidence linking the {1\SG} pattern and the lack of an accent on the L-stem.

\section{The accentuation of the verbal stem}\label{mat:sec:Accentuation}

As shown by \citet{Halle1973,Halle1975,Halle1997} and \citet{Melvold1989}, stem-conditioned stress retraction is also attested in the nominal declension, where some nouns undergo it in the plural, and in adjectival inflection, where it is triggered for most adjectival stems by the long-form suffix. \citet{Melvold1989} further argues that both post-accenting and unaccented stems can undergo retraction:

\ea unaccented feminine stem: final stress in the singular, except in the accusative
\ea {\SG}: ruká/rúku ‘hand.{\SG.\NOM/\ACC}’, {\PL}: rukámi ‘hand.{\PL.\INS}’ \hfill (regular)
\ex {\SG}: dušá/dúšu ‘soul.{\SG.\NOM/\ACC}’, {\PL}: dúšami ‘soul.{\PL.\INS}’ \hfill (retracting)
\z
\z

\ea post-accenting feminine stem: final stress in the singular
\ea {\SG}: čertá/čertú ‘line.{\SG.\NOM/\ACC}’, {\PL}: čertámi ‘line.{\PL.\INS}’ \hfill (regular)
\ex {\SG}: stroká/strokú ‘text line.{\SG.\NOM/\ACC}’, {\PL}: strókami ‘text line.{\PL.\INS}’ \hspace*{\fill} (retracting)
\z
\z

\noindent Verbs are different. While \citet[291]{Melvold1989} proposes that verbal retraction occurs in post-accenting L-stems, in this section I will argue that the {1\SG} pattern correlates with unaccented L-stems.

\subsection{The verbs in \textit{-a-}/\textit{-i-} and L-stem accentuation}\label{mat:subsec:AIandLStems}

The class of first-conjugation \Affix{a}/\Affix{i} verbs is a semi-closed one: This thematic suffix combines with a finite set of stems (103, to the best of my knowledge, as well as five \Affix{o}/\Affix{i} verbs)\footnote{The five verbs in \Affix{o}/\Affix{i} (\textit{kolótʲ} ‘to stab’, \textit{molótʲ} ‘to grind’, \textit{polótʲ} ‘to weed’, \textit{borótʲ} ‘to fight’, and \textit{porótʲ} ‘to whip’) all have stems ending in [olo] or [oro], which are, respectively, pleophonic allomorphs of \Affix{la} and \Affix{ra} in Russian (on pleophony in Slavic see, e.g., \citealt[36--37, 207]{SussexCumberley2006}).} and is also used with the denominal verbalizing suffix \Affix{ow} (on which more below). Instead of the post-stem stress pattern (\tabref{mat:tab:Interaction1stAI}-b), this class contains, in addition to verbs exhibiting the stem and {1\SG} patterns, five verbal roots (cf. \citealt[115]{Gladney1995}) with the pattern in \tabref{mat:tab:Interaction1stAI}-d, where stress is retracted to the stem-final syllable throughout the present-tense paradigm: \textit{kolebátʲ}/\textit{koléblʲu} ‘rock.{\INF}/{1\SG}’, \textit{kolɨxátʲ}/\textit{kolɨ́šu} ‘sway.{\INF}/{1\SG}’, \textit{alkátʲ}/\textit{álču} ‘crave.{\INF}/{1\SG}’, the archaic variant \textit{stradátʲ}/\textit{stráždu} ‘suffer.{\INF}/{1\SG}’ and the two equally archaic prefixed derivatives of the cranberry root \Affix{im}, \textit{vnimátʲ}/\textit{vnémlʲu} ‘heed.{\INF}/{1\SG}’ and \textit{prinimátʲ}/\textit{priémlʲu} ‘accept.{\INF}/{1\SG}’; in modern spoken Russian the last three take the thematic suffix \Affix{aj}. The post-stem pattern, on the other hand, is not attested in this thematic class.

\begin{table}
\caption{Accentual interaction with the 1st conjugation suffix \Affix{a}/\Affix{i}}
\label{mat:tab:Interaction1stAI}
 \begin{tabularx}{\textwidth}{lL{2.95cm}Z{1.6cm}Z{1.5cm}Z{1.8cm}Z{1.8cm}} 
  \lsptoprule
    &&accented\linebreak{\PRS-3\SG} &accented\linebreak{\PRS-1\SG}&accented\linebreak{\PST-\FEM.\SG}  &unaccented\linebreak{\PST-\PL}    \\
  \midrule
    a.  &stem:\newline\textit{-maz-} ‘smear’ 
        &máž-e-t    &máž-u    &máz-a-l-a  &máz-a-l-i 
        \\\addlinespace[5pt]
    b.  &\multicolumn{5}{X}{post-stem: N/A; potential candidates among \textit{j}-final stems can be assigned to the \Affix{a}/\Affix{Ø} class} 
        \\\addlinespace[5pt]
    c.  &{1\SG}:\newline\textit{-vʲaz-} ‘tie’ 
        &vʲáž-e-t    &vʲaž-ú    &vʲaz-á-l-a   &vʲaz-á-l-i   
        \\\addlinespace[5pt]
    d.  &stem-final present:\newline\textit{-koleb-} ‘rock’ 
        &koléblʲ-e-t &koléblʲ-u &koleb-á-l-a  &koleb-á-l-i  \\
  \lspbottomrule
 \end{tabularx}
\end{table}

Inside this class there are two subclasses of derived stems: the non-productive class derived by the suffix \Affix{ot} and the productive class in \Affix{ow}. While the former creates the {1\SG} pattern (\tabref{mat:tab:Interaction1stAI}-c), the latter gives rise to the stem-final one (\tabref{mat:tab:Interaction1stAI}-d).

Starting with the former, all ca. 20 stems ending in \Affix{ot} form sound-emission verbs. While for most verbs in this category no meaningful root can be identified before \Affix{ot}, at least the verbs \textit{vorkotátʲ} ‘to grumble’, \textit{topotátʲ} ‘to stamp’ and \textit{trepetátʲ} ‘to tremble’ can be argued to be built on the roots \Affix{vork}, \Affix{top} and \Affix{trep}, given the verbs \textit{vorkovátʲ} ‘to coo’, \textit{tópatʲ} ‘to stamp, tramp’ and \textit{trepátʲ} ‘to pull, flutter’. The fact that the accented root of \REF{mat:ex:topat} is not stressed in \REF{mat:ex:topotat} could indicate that \Affix{ot} is accentually dominant (and either post-accenting or unaccented, since it is never stressed itself), and this is confirmed by the fact that all verbs with this suffix give rise to the {1\SG} pattern.

\ea 
\ea\label{mat:ex:topat} tópatʲ ‘to stamp, tramp’: tópaju ({1\SG})/tópajet ({3\SG})
\ex\label{mat:ex:topotat} topotátʲ ‘to stamp’: topočú ({1\SG})/topóčet ({3\SG})
\z
\z

\largerpage
\noindent Can it be determined if \Affix{ot} stems are unaccented or post-accenting? Unfortunately, the answer is no, because the thematic suffix \Affix{a}/\Affix{i} introduces an accent, and the fact that action nouns null-derived from \Affix{ot} verbs exhibit initial stress \REF{mat:ex:groxotatxoxotat}, while suggestive of an unaccented L-stem, could also be the artefact of conversion, which favors initial stress.\footnote{None of these nouns naturally forms a plural, which excludes this way of checking their accentuation. The fact that the post-accenting diminutive suffix \Affix{ŭk} derives a post-accenting noun (\textit{xoxotók}) is also non-indicative (cf. \citealt[340]{Halle1973}).} However, the fact that the same suffix uniformly gives rise to the same accentual behavior strongly indicates that the {1\SG} pattern depends on the accentuation of the L-stem.

\ea\label{mat:ex:groxotatxoxotat} 
\ea\label{mat:ex:groxotat} groxotátʲ ‘to bang’: groxočú ({1\SG})/groxóčet ({3\SG}) \\gróxot ‘a bang’
\ex\label{mat:ex:xoxotat} xoxotátʲ ‘to laugh loudly’: xoxočú ({1\SG})/xoxóčet ({3\SG}) \\xóxot ‘laughter’
\z
\z

\noindent The same conclusion can be drawn from the denominal verbalizer \Affix{ow}, which, as \REF{mat:ex:vracevatkritikovat} shows, surfaces as [ov] before the surface [a] in the past and in the infinitive and as [u] (followed by the surface [j]) in the present.\footnote{See \citet{Melvold1989} for the assumption that \Affix{ow} combines with the thematic suffix \Affix{a}/\Affix{i} and a demonstration how its surface realization is determined by the resulting syllable structure. Systematic treatments of (some other instances of) the surface [u] as an underlying /ow/ before consonants are presented in \citet{Lightner1965} and more recently in \citet[147--148]{Itkin2007}.}

\ea\label{mat:ex:vracevatkritikovat} 
\ea\label{mat:ex:vracevat} vračevátʲ ‘to treat, heal’: vračúju ({1\SG})/vračújet ({3\SG}) \hfill (retraction)
\ex\label{mat:ex:kritikovat} kritikovátʲ ‘to critique’: kritikúju ({1\SG})/kritikújet ({3\SG})
\z
\z

\noindent The accentual behavior of \Affix{ow} verbs is strikingly different from that of \Affix{ot} verbs (as well as from that of \textit{i}-verbs): Unless they have systematic stem stress (e.g., \textit{komándovatʲ} ‘to command’), they are stressed on the thematic suffix in the past and exhibit stem-final stress in the present (\tabref{mat:tab:Interaction1stAI}-d). The dependence of this stem-final pattern on the accentuation of the thematic suffix is confirmed by the fact that the \Affix{a}/\Affix{i} subclass contains no verbs that exhibit post-stem stress in the present (modulo fn. \ref{mat:fn:j-final-verbs}).

Given that the \Affix{a}/\Affix{i} thematic suffix introduces an accent, the accentuation of the \Affix{ow} stems in \REF{mat:ex:vracevatkritikovat} is difficult to determine: They can be unaccented or post-accenting. Since \Affix{ow} verbs are denominal, the accentuation of their L-stems should be linked to their nominal bases. However, as noted by \citet{Redkin1965}, \citet[344--347]{Halle1973}, \citet[107]{Zaliznjak1985}, and \citet{Gladney1995}, among others, the relation between the accentuation of a noun and that of the verb that is derived from it is not straightforward, as can also be shown by the following \textit{i}-verbs:

\largerpage
\ea\label{mat:ex:iAccentedNouns} accented nouns
\ea razžáloblʲú/razžálobit ‘move to pity.{1\SG}/{3\SG}’ \hfill (stem)\\
(cf. žáloba/žálobu ‘complaint.\NOM/\ACC’) 
\ex bešú/bésit ‘enrage.{1\SG}/{3\SG}’ \hfill ({1\SG})\\
(cf. bésa/bésami ‘devil.{\SG.\GEN}/{\PL.\INS}’
\ex bomblʲú/bombít ‘bomb.{1\SG}/{3\SG}’ \hfill (inflection)\\
(cf. bómba/bómbu ‘bomb.\NOM/\ACC’)
\z
\z

\ea post-accenting nouns
\ea {kónču}/{kónčit} ‘finish.{1\SG}/{3\SG}’ \hfill (stem)\\
(cf. {koncá}/{koncámi} ‘end.{\SG.\GEN}/{\PL.\INS}’) 
\ex {ženʲú}/{žénit} ‘marry.{1\SG}/{3\SG}’ \hfill ({1\SG})\\
(cf. {žená}/{ženú} ‘wife.{\NOM}/{\ACC}’
\ex {strujú}/{struít} ‘stream.{1\SG}/{3\SG}’ \hfill (inflection)\\
(cf. {strujá}/{strujú} ‘stream.{\NOM}/{\ACC}’) 
\z
\z

\ea\label{mat:ex:iUnaccentedNouns} unaccented nouns
\ea prizemlʲú/prizemlít ‘land.{1\SG}/{3\SG}’ \hfill (stem)\\
(cf. zemlʲá/zémlʲu ‘ground.{\NOM}/{\ACC}’) 
\ex {poručú}/{porúčit} ‘entrust.{1\SG}/{3\SG}’ \hfill ({1\SG})\\
(cf. {ruká}/{rúku} ‘hand.{\NOM}/{\ACC}’
\ex {boronʲú}/{boronít} ‘harrow.{1\SG}/{3\SG}’ \hfill (inflection)\\
(cf. {boroná}/{bóronu} ‘harrow.{\NOM}/{\ACC}’) 
\z
\z

\noindent Derivation with the suffix \Affix{ow} may preserve the accent of the base noun \REF{mat:ex:urodovat}, or may override it \REF{mat:ex:kritikovatAgain}. Yet if the suffix \Affix{ow} were unaccented, we would expect to find at least some verbs derived from a post-accenting noun that would end up with stress on the suffix itself. The fact that instead in the past we find stress on the thematic suffix, as in \REF{mat:ex:vracevatAgain}, strongly suggests that the suffix \Affix{ow} is post-accenting. Indeed, as shown by \citet[126]{Garde1998}, in the sequence of two post-accenting morphemes the second accent wins (thus violating the Basic Accentuation Principle \REF{mat:ex:BAP}).\footnote{\label{mat:fn:Garde}\citet[126]{Garde1998} illustrates this with the derivation in \REF{mat:ex:vracevatAgain}, which can also be accounted for by the assumption that \Affix{ow} 
cannot bear an accent (except by retraction). \citet[131]{Garde1998} further provides an example of a post-accenting root followed by the post-accenting diminutive suffix \Affix{ĭc}, where stress is final \REF{mat:ex:GardeDvor}, and the same result is obtained with the post-accenting diminutive suffix \Affix{ŭk} \REF{mat:ex:GardeKazak}. Since, however, both suffixes contain yers, which are known to be unstressable, these cases are also non-definitive.

\ea
\ea\label{mat:ex:GardeDvor} dvor/dvorɨ́ ‘yard.{\SG.\NOM}/{\PL.\NOM}’\\ dvoréc/dvorcɨ́ ‘palace.{\SG.\NOM}/{\PL.\NOM}’ 
\ex\label{mat:ex:GardeKazak} kazák/kazakí ‘Cossack.{\SG.\NOM}/{\PL.\NOM}’ \\kazačók/kazačkí ‘boy-servant.{\SG.\NOM}/{\PL.\NOM}’
\z
\z

\noindent \citeauthor{Garde1998}’s generalization, however, is supported also by the suffix \Affix{ič} (cf. \textit{moskvičí} ‘denizens of Moscow’ from the post-accenting \textit{Moskvá}/\textit{Moskvú} ‘Moscow.{\NOM/\ACC}’). Conversely, it should also be noted that a sequence of two post-accenting diminutive suffixes \Affix{ŭk} yields stress on the first one (e.g., the unaccented root \Affix{vetĭr} ‘wind’ yields a post-accenting simple diminutive \textit{veterók}/\textit{veterká} ‘wind.{\textsc{dim}.\NOM}/{\GEN}’ and a stem-accented double diminutive \textit{veteróček}/\textit{veteróčka} ‘wind.{\textsc{dim}.\NOM}/{\GEN}’). }

\ea\label{mat:ex:TwoPostAccenting}
\ea\label{mat:ex:urodovat} accented: \\
uród/uródɨ ‘ugly person.{\SG.\NOM}/{\PL.\NOM}’ $\rightarrow$ uródovatʲ ‘to disfigure’
\ex\label{mat:ex:kritikovatAgain} accented:\\
krítika/krítiki ‘critique.{\SG.\NOM}/{\PL.\NOM}’ $\rightarrow$ kritikovátʲ ‘to critique’
\ex\label{mat:ex:vracevatAgain} post-accenting:\\
vrač/vračí ‘doctor.{\SG.\NOM}/{\PL.\NOM}’ $\rightarrow$ vračevátʲ ‘to treat, heal’
\z
\z

\noindent If the stem-final pattern (\tabref{mat:tab:Interaction1stAI}-d) is associated with a post-accenting L-stem, as suggested by verbs in \Affix{ow}, it seems reasonable to hypothesize that the {1\SG} stress correlates with an unaccented L-stem, and stress-initial nouns in \Affix{ot} provide further tentative evidence in favor of this view for verbs in \Affix{ot}. In the next subsection I will offer additional support for the assumption that the {1\SG} stress pattern corresponds to an unaccented L-stem.

\subsection{Two 2nd conjugation \textit{-a-}/\textit{-i-} verbs}

There exist two second-conjugation verbs with the thematic suffix \Affix{a} in the past: \textit{gnátʲ} ‘to chase’ and \textit{spátʲ} ‘to sleep’, which both exhibit accentual variability in the past:

\ea\label{mat:ex:GnatSpatPst}
\ea gnalá/gnáli ‘chase.{\PST.\FEM}{\SG}/{\PL}’
\ex spalá/spáli ‘sleep.{\PST.\FEM}{\SG}/{\PL}’
\z
\z

\noindent As demonstrated in \sectref{mat:sec:Background}, accentual variability in the past is a diagnostic of the lack of a preceding accent, i.e., both the roots and the thematic suffix in \REF{mat:ex:GnatSpatPst} are unaccented. While \textit{gnátʲ} ‘to chase’ shows the {1\SG} pattern in the present, the root of the verb \textit{spátʲ} ‘to sleep’ is asyllabic, so its accentuation could conform to any of the three patterns:

\ea\label{mat:ex:GnatSpatPrs}
\ea gonʲú/gónit ‘chase.{1\SG}/{3\SG}’
\ex splʲú/spít ‘sleep.{1\SG}/{3\SG}’
\z
\z

\noindent The {1\SG} present-tense stress pattern can therefore be taken as an indication that the L-stem is unaccented. The post-accenting stem can then be assumed to give rise either to the systematic post-stem stress (with second-conjugation \textit{i}-verbs and occasional \textit{e}-verbs, as well as with six semelfactive \textit{nu}-verbs) or to consistent stem-final stress (with \Affix{a}/\Affix{i} verbs).

The two questions to address next are (i) how the {1\SG} pattern is derived, and (ii) why \textit{i}-verbs, \textit{nu}-verbs and \textit{e}-verbs also give rise to the post-stem stress pattern while \Affix{a}/\Affix{i} verbs surface with stem-final stress. I will propose that the {1\SG} pattern is due to an accentual conflict arising from the hiatus resolution with an accented vowel, and that post-accenting L-stems create two ways of avoiding this conflict, depending on the fate of the thematic vowel.

\section{The {1\SG} pattern as an accentual conflict}\label{mat:sec:AccentualConflict}

I have suggested (\sectref{mat:sec:Unstressability}) that the {1\SG} pattern arises when the present-tense suffix is absent from the metrical tier (being either not projected there or removed from it). I have also provided evidence (\sectref{mat:sec:Accentuation}) that the {1\SG} pattern is associated with unaccented L-stems and arises when the thematic vowel is deleted before the vowel of the present-tense suffix. The lack of accentual variability in the present tense of athematic verbs as contrasted with their past tense shows that the present-tense suffix \Affix{e} introduces an accent (\sectref{mat:subsec:URofSuffixes}). Likewise, the lack of accentual variability in the past tense of most thematic verbs (\sectref{mat:subsec:ThematicAnd1SG}) entails the same for thematic suffixes, and with the unaccented thematic suffix \Affix{a} and in athematic verbs the {1\SG} pattern is exceptional.

While so far I have been tacitly assuming that the present-tense suffix and the thematic suffix are accented, I am now going to revise this assumption and propose that the thematic suffixes giving rise to the {1\SG}  pattern are post-accenting. Since the past-tense suffix \Affix{l} has been argued to be retracting (\sectref{mat:sec:Background}), this assumption makes no difference in the past tense of thematic verbs, as illustrated in \REF{mat:ex:PstRetraction}; for the sake of intelligibility the foot boundary introduced by post-accentuation is indicated by a square bracket:

\ea\label{mat:ex:PstRetraction} Past-tense retraction\bigskip%The author requests that this example not be split across two pages
\ea\tikzstyle{every picture}+=[remember picture, inner sep=0pt,baseline, anchor=base,execute at begin node=\strut]
\gll \tikz\node(root){√-}; n\tikz\node(nu){u};~~~ l \tikz\node(a){a}; \\
{} {\THEM} {\PST} {{\FEM.\SG}}\\
 \begin{tikzpicture}[overlay,remember picture]
	 \node [above of = root, node distance=4mm] (root-ast) {$\ast$};
  \node [above of = nu, node distance=4mm] (nu-ast) {$\ast$};
  \node [above of = a, node distance=4mm] (a-ast) {$\ast$};
  \node [right of = nu-ast, node distance=2mm] (left-bracket) {[};
  \node [left of = a-ast, node distance=2mm] (left-bracket) {(};
  \node [right of = a-ast, node distance=2mm] (right-bracket) {)};
  \coordinate (center) at ($(root)!0.5!(a)$);
  \node (arrow)   [below of = center,node distance=9mm]  {$\downarrow$};
\end{tikzpicture}\bigskip\bigskip
\ex
\gll \tikz\node(root2){√-}; n\tikz\node(nu2){u};~~~ l \tikz\node(a2){a}; \\
{} {\THEM} {\PST} {{\FEM.\SG}}\\
 \begin{tikzpicture}[overlay,remember picture]
	 \node [above of = root2, node distance=4mm] (root2-ast) {$\ast$};
  \node [above of = nu2, node distance=4mm] (nu2-ast) {$\ast$};
  \node [above of = nu2-ast, node distance=4mm] (nu2-ast2) {$\ast$};
  \node [above of = a2, node distance=4mm] (a2-ast) {$\ast$};
  \node [left of = nu2-ast, node distance=2mm] (left-bracket) {[};
  \node [left of = a2-ast, node distance=2mm] (left-bracket) {(};
  \node [right of = a2-ast, node distance=2mm] (right-bracket) {)};
\end{tikzpicture}
\z
\z

\noindent The situation is different in the present, where the vowel of the post-accenting thematic suffix is deleted before the vowel of the accented present-tense suffix. My intuition here is that the removal of the present-tense suffix from the metrical tier is due to a conflict that is created by this deletion. On the assumption that the accent of a deleted vowel remains after deletion and is associated to the same syllable (defined from the left by the consonant(s) of the first syllable, and from the right, by the nucleus of the second one), this syllable would receive contradictory instructions: to project an accent on the metrical tier and to shift this accent one syllable to the left. I represent this conflict in \xxref{mat:ex:Nu-e-u-Derivation-4-a}{mat:ex:Nu-e-u-Derivation-4-b}, with the right parenthesis at the right edge of the word 
deleted in \REF{mat:ex:Nu-e-u-Derivation-4-b} because it is immediately preceded by another parenthesis (see \sectref{mat:subsec:HalleIdsardi}):\footnote{The order of the two parentheses is changed to emphasize that “[” forces the accent on the next syllable, but the deletion of the present-tense suffix from the metrical tier means that the accent would be assigned to the agreement suffix also if the order is maintained. Crucially, the deletion of an extra parenthesis has to follow hiatus resolution.}\bigskip\bigskip\bigskip

\ea\label{mat:ex:Nu-e-t-Derivation-4}%The author requests that this example not be split across two pages
\ea\label{mat:ex:Nu-e-u-Derivation-4-a}\tikzstyle{every picture}+=[remember picture, inner sep=0pt,baseline, anchor=base,execute at begin node=\strut]
\gll \tikz\node(root){√-}; n\tikz\node(nu){u};~~~ \tikz\node(e){e}; \tikz\node(t){t}; \\
{} {\THEM} {\PRS} {3\SG}\\
 \begin{tikzpicture}[overlay,remember picture]
  \node [above of = root, node distance=8mm] (root-syl) {σ};
  \node [above of = nu, node distance=8mm] (nu-syl) {σ};
  \node [above of = e, node distance=8mm] (e-syl) {σ};
	 \node [above of = root-syl, node distance=4mm] (root-ast) {$\ast$};
  \node [above of = nu-syl, node distance=4mm] (nu-ast) {$\ast$};
  \node [above of = e-syl, node distance=4mm] (e-ast) {$\ast$};
  \node [right of = nu-ast, node distance=2mm] (right-bracket) {[};
  \node [left of = e-ast, node distance=2mm] (right-bracket2) {(};
  \node [right of = e-ast, node distance=2mm] (right-bracket) {)};
  \draw[-] (root-syl) -- (root);
  \draw[-] (nu-syl) -- (nu);
  \draw[-] (e-syl) -- (e);
  \draw[-] (e-syl) -- (t);
  \coordinate (center) at ($(root)!0.5!(t)$);
  \node (arrow)   [below of = center,node distance=9mm]  {$\downarrow$};
\end{tikzpicture}\bigskip\bigskip\bigskip\bigskip
\ex\label{mat:ex:Nu-e-u-Derivation-4-b}
\gll \tikz\node(root2){√-}; \tikz\node(n2){n};\tikz\node(nu2){\cancel{u}};~~~ \tikz\node(e2){e}; \tikz\node(t2){t}; \\
{} {\THEM} {\PRS} {3\SG}\\
 \begin{tikzpicture}[overlay,remember picture]
  \node [above of = root2, node distance=8mm] (root2-syl) {σ};
  \node [above of = e2, node distance=8mm] (e2-syl) {σ};
	 \node [above of = root2-syl, node distance=4mm] (root2-ast) {$\ast$};
  \node [above of = e2-syl, node distance=4mm] (e2-ast) {$\ast$};
  \node [left of = e2-ast, node distance=2mm] (right-bracket2) {(};
  \node [right of = e2-ast, node distance=2mm] (right-bracket) {[};
  \draw[-] (root2-syl) -- (root2);
  \draw[-] (e2-syl) -- (n2);
  \draw[-] (e2-syl) -- (e2);
  \draw[-] (e2-syl) -- (t2);
  \coordinate (center) at ($(root2)!0.5!(t2)$);
  \node (arrow)   [below of = center,node distance=9mm]  {$\downarrow$};
\end{tikzpicture}\bigskip\bigskip\bigskip\bigskip
\ex\label{mat:ex:Nu-e-u-Derivation-4-c}
\gll \tikz\node(root3){√-}; \tikz\node(n3){n};\tikz\node(nu3){\cancel{u}};~~~ \tikz\node(e3){e}; \tikz\node(t3){t}; \\
{} {\THEM} {\PRS} {3\SG}\\
 \begin{tikzpicture}[overlay,remember picture]
  \node [above of = root3, node distance=8mm] (root3-syl) {σ};
  \node [above of = nu3, node distance=8mm] (nu3-syl) {σ};
  \node [above of = e3, node distance=8mm] (e3-syl) {σ};
	 \node [above of = root3-syl, node distance=4mm] (root3-ast) {$\ast$};
  \node [above of = e3-syl, node distance=4mm] (e3-ast) {\strut};
  \node [left of = e3-ast, node distance=2mm] (right-bracket2) {(};
  \node [right of = e3-ast, node distance=2mm] (right-bracket) {[};
  \draw[-] (root3-syl) -- (root3);
  \draw[-] (e3-syl) -- (n3);
  \draw[-] (e3-syl) -- (e3);
  \draw[-] (e3-syl) -- (t3);
\end{tikzpicture}
\z
\z

\noindent The representation in \xxref{mat:ex:Nu-e-u-Derivation-4-b}{mat:ex:Nu-e-u-Derivation-4-c} makes explicit the relation between thematic vowel deletion and an accentual conflict: Since the deletion of the thematic vowel triggers resyllabification of the resulting phonological string while retaining the lexically specified instructions for the metrical tier, the rebuilt syllabic structure is subject to conflicting instructions. This is shown in \REF{mat:ex:Nu-e-u-Derivation-4-b}: The same syllable cannot be simultaneously accented and post-accenting. I propose that the problematic position is deleted from the metrical tier \REF{mat:ex:Nu-e-u-Derivation-4-c}. 

Once again, as no stress-bearing elements are contained between the two parentheses in \REF{mat:ex:Nu-e-u-Derivation-4-c}, the second parenthesis is deleted \REF{mat:ex:Nu-e-u-Derivation-4-d}. Because the remaining parenthesis ends up word-final, stress, like with post-accenting stems, will surface on the final syllable of the stem \REF{mat:ex:Nu-e-u-Derivation-4-e}:

\newpage\vspace*{1em}
\ea\label{mat:ex:Nu-e-t-Derivation-4-continued}
\ea\label{mat:ex:Nu-e-u-Derivation-4-d}\tikzstyle{every picture}+=[remember picture, inner sep=0pt,baseline, anchor=base,execute at begin node=\strut]
\gll \tikz\node(root3){√-}; \tikz\node(n3){n};~~~ \tikz\node(e3){e}; \tikz\node(t3){t}; \\
{} {\THEM} {\PRS} {3\SG}\\
 \begin{tikzpicture}[overlay,remember picture]
  \node [above of = root3, node distance=8mm] (root3-syl) {σ};
  \node [above of = e3, node distance=8mm] (e3-syl) {σ};
	 \node [above of = root3-syl, node distance=4mm] (root3-ast) {$\ast$};
  \node [above of = e3-syl, node distance=4mm] (e3-ast) {\strut};
  \node [left of = e3-ast, node distance=2mm] (right-bracket2) {(};
  \draw[-] (root3-syl) -- (root3);
  \draw[-] (e3-syl) -- (n3);
  \draw[-] (e3-syl) -- (e3);
  \draw[-] (e3-syl) -- (t3);
  \coordinate (center) at ($(root3)!0.5!(t3)$);
  \node (arrow)   [below of = center,node distance=9mm]  {$\downarrow$};
\end{tikzpicture}\bigskip\bigskip\bigskip\bigskip
\ex \label{mat:ex:Nu-e-u-Derivation-4-e}
\gll \tikz\node(root2){√-}; \tikz\node(n2){n};~~~ \tikz\node(e2){e}; \tikz\node(t2){t}; \\
{} {\THEM} {\PRS} {3\SG}\\
 \begin{tikzpicture}[overlay,remember picture]
  \node [above of = root2, node distance=8mm] (root2-syl) {σ};
  \node [above of = e2, node distance=8mm] (e2-syl) {σ};
	 \node [above of = root2-syl, node distance=4mm] (root2-ast) {$\ast$};
  \node [above of = e2-syl, node distance=4mm] (e2-ast) {\strut};
  \node [left of = root2-ast, node distance=2mm] (left-bracket) {(};
  \node [left of = e2-ast, node distance=2mm] (left-bracket) {(};
  \draw[-] (root2-syl) -- (root2);
  \draw[-] (e2-syl) -- (n2);
  \draw[-] (e2-syl) -- (e2);
  \draw[-] (e2-syl) -- (t2);
\end{tikzpicture}
\z
\z

\noindent If, on the other hand, the present-tense suffix is followed by a syllabic suffix, i.e., the {1\SG} \Affix{u}, as in \REF{mat:ex:Nu-e-u-Derivation-5-a}, the present-tense gerund \Affix{ʲa} or the imperative \Affix{i}, the thematic vowel is deleted \REF{mat:ex:Nu-e-u-Derivation-5-b}. After this deletion the present-tense suffix is removed from the metrical tier \REF{mat:ex:Nu-e-u-Derivation-5-c}, and then the present-tense suffix is deleted before another vowel \REF{mat:ex:Nu-e-u-Derivation-5-d} and its accent is realized on the vowel of the {1\SG} suffix:\bigskip\bigskip\bigskip

\ea\label{mat:ex:Nu-e-u-Derivation-5}%The author requests that this example not be split across two pages
\ea\label{mat:ex:Nu-e-u-Derivation-5-a}\tikzstyle{every picture}+=[remember picture, inner sep=0pt,baseline, anchor=base,execute at begin node=\strut]
\gll \tikz\node(root){√-}; \tikz\node(n){n};\tikz\node(nu){u};~~~ \tikz\node(e){e}; \tikz\node(u){u}; \\
{} {\THEM} {\PRS} {1\SG}\\
 \begin{tikzpicture}[overlay,remember picture]
  \node [above of = root, node distance=8mm] (root-syl) {σ};
  \node [above of = nu, node distance=8mm] (nu-syl) {σ};
  \node [above of = e, node distance=8mm] (e-syl) {σ};
  \node [above of = u, node distance=8mm] (u-syl) {σ};
	 \node [above of = root-syl, node distance=4mm] (root-ast) {$\ast$};
  \node [above of = nu-syl, node distance=4mm] (nu-ast) {$\ast$};
  \node [above of = e-syl, node distance=4mm] (e-ast) {$\ast$};
  \node [above of = u-syl, node distance=4mm] (u-ast) {$\ast$};
  \node [right of = nu-ast, node distance=2mm] (left-bracket) {[};
  \node [left of = e-ast, node distance=2mm] (left-bracket) {(};
  \node [right of = u-ast, node distance=2mm] (right-bracket) {)};
  \draw[-] (root-syl) -- (root);
  \draw[-] (nu-syl) -- (nu);
  \draw[-] (e-syl) -- (e);
  \draw[-] (u-syl) -- (u);
  \coordinate (center) at ($(root)!0.5!(u)$);
  \node (arrow)   [below of = center,node distance=9mm]  {$\downarrow$};
\end{tikzpicture}\bigskip\bigskip\bigskip\bigskip
\ex\label{mat:ex:Nu-e-u-Derivation-5-b}\tikzstyle{every picture}+=[remember picture, inner sep=0pt,baseline, anchor=base,execute at begin node=\strut]
\gll \tikz\node(root){√-}; \tikz\node(n){n};\tikz\node(nu){\cancel{u}};~~~ \tikz\node(e){e}; \tikz\node(u){u}; \\
{} {\THEM} {\PRS} {1\SG}\\
 \begin{tikzpicture}[overlay,remember picture]
  \node [above of = root, node distance=8mm] (root-syl) {σ};
  \node [above of = e, node distance=8mm] (e-syl) {σ};
  \node [above of = u, node distance=8mm] (u-syl) {σ};
	 \node [above of = root-syl, node distance=4mm] (root-ast) {$\ast$};
  \node [above of = e-syl, node distance=4mm] (e-ast) {$\ast$};
  \node [above of = u-syl, node distance=4mm] (u-ast) {$\ast$};
  \node [left of = e-ast, node distance=2mm] (left-bracket) {(};
  \node [right of = e-ast, node distance=2mm] (left-bracket) {[};
  \node [right of = u-ast, node distance=2mm] (right-bracket) {)};
  \draw[-] (root-syl) -- (root);
  \draw[-] (e-syl) -- (n);
  \draw[-] (e-syl) -- (e);
  \draw[-] (u-syl) -- (u);
  \coordinate (center) at ($(root)!0.5!(u)$);
  \node (arrow)   [below of = center,node distance=9mm]  {$\downarrow$};
\end{tikzpicture}\bigskip\bigskip\bigskip\bigskip
\ex\label{mat:ex:Nu-e-u-Derivation-5-c}
\gll \tikz\node(root2){√-}; \tikz\node(n2){n};\tikz\node(nu2){\cancel{u}};~~~ \tikz\node(e2){e}; \tikz\node(u2){u}; \\
{} {\THEM} {\PRS} {1\SG}\\
 \begin{tikzpicture}[overlay,remember picture]
  \node [above of = root2, node distance=8mm] (root2-syl) {σ};
  \node [above of = e2, node distance=8mm] (e2-syl) {σ};
  \node [above of = u2, node distance=8mm] (u2-syl) {σ};
	 \node [above of = root2-syl, node distance=4mm] (root2-ast) {$\ast$};
  \node [above of = e2-syl, node distance=4mm] (e2-ast) {\strut};
  \node [above of = u2-syl, node distance=4mm] (u2-ast) {$\ast$};
  \node [left of = e2-ast, node distance=2mm] (left-bracket) {(};
  \node [right of = u2-ast, node distance=2mm] (right-bracket) {)};
  \draw[-] (root2-syl) -- (root2);
  \draw[-] (e2-syl) -- (n2);
  \draw[-] (e2-syl) -- (e2);
  \draw[-] (u2-syl) -- (u2);
  \coordinate (center) at ($(root2)!0.5!(u2)$);
  \node (arrow)   [below of = center,node distance=9mm]  {$\downarrow$};
\end{tikzpicture}\bigskip\bigskip\bigskip\bigskip
\ex\label{mat:ex:Nu-e-u-Derivation-5-d}
\gll \tikz\node(root3){√-}; \tikz\node(n3){n};~~~ \tikz\node(e3){\cancel{e}}; \tikz\node(u3){u}; \\
{} {\THEM} {\PRS} {1\SG}\\
 \begin{tikzpicture}[overlay,remember picture]
  \node [above of = root3, node distance=8mm] (root3-syl) {σ};
  \node [above of = u3, node distance=8mm] (u3-syl) {σ};
	 \node [above of = root3-syl, node distance=4mm] (root3-ast) {$\ast$};
  \node [above of = u3-syl, node distance=4mm] (u3-ast) {$\ast$};
  \node [above of = u3-ast, node distance=4mm] (u3-ast2) {$\ast$};
  \node [right of = root3-ast, node distance=2mm] (left-bracket) {(};
  \draw[-] (root3-syl) -- (root3);
  \draw[-] (u3-syl) -- (n3);
  \draw[-] (u3-syl) -- (u3);
\end{tikzpicture}
\z
\z

\noindent Problematically, the Halle-Idsardi model does not have the means to express the intuition that unstressability results from an accentual conflict. This is not purely a matter of notation: In this model, parentheses on line 0 of the metrical tier represent foot boundaries and post-accentuation is implemented by placing a parenthesis on the next asterisk. There is therefore no difference between the illicit (\textit{ex hypothesi}) structures in \REF{mat:ex:Nu-e-u-Derivation-4-b} or \REF{mat:ex:Nu-e-u-Derivation-5-b} and the licit structures created by a sequence of two accented \REF{mat:ex:Masa} or post-accenting \REF{mat:ex:Moskvica} morphemes:\bigskip\bigskip

\ea\label{mat:ex:MasaMoskvica}
\ea\label{mat:ex:Masa}\tikzstyle{every picture}+=[remember picture, inner sep=0pt,baseline, anchor=base,execute at begin node=\strut]
\gll M\tikz\node(Ma){a};š \tikz\node(a){a}; \\
{Masha} {\SG.\NOM}\\
 \begin{tikzpicture}[overlay,remember picture]
  \node [above of = Ma, node distance=4mm] (Ma-ast) {$\ast$};
  \node [above of = Ma-ast, node distance=4mm] (Ma-ast2) {$\ast$};
  \node [above of = a, node distance=4mm] (a-ast) {$\ast$};
  \node [left of = Ma-ast, node distance=2mm] (left-bracket) {(};
  \node [left of = a-ast, node distance=2mm] (left-bracket) {(};
  \node [right of = a-ast, node distance=2mm] (right-bracket) {)};
\end{tikzpicture}\bigskip\bigskip
\ex\label{mat:ex:Moskvica}
\gll M\tikz\node(o2){o};skv \tikz\node(i2){i};č \tikz\node(a2){a}; \\
{Moscow} {\NMLZ} {{\SG.\GEN}}\\
 \begin{tikzpicture}[overlay,remember picture]
	 \node [above of = o2, node distance=4mm] (o2-ast) {$\ast$};
  \node [above of = i2, node distance=4mm] (i2-ast) {$\ast$};
  \node [above of = a2, node distance=4mm] (a2-ast) {$\ast$};
  \node [above of = a2-ast, node distance=4mm] (a2-ast2) {$\ast$};
  \node [right of = o2-ast, node distance=2mm] (left-bracket) {[};
  \node [right of = i2-ast, node distance=2mm] (left-bracket) {[};
  \node [right of = a2-ast, node distance=2mm] (right-bracket) {)};
\end{tikzpicture}
\z
\z

\noindent As discussed in fn. \ref{mat:fn:Garde}, the Basic Accentuation Principle \REF{mat:ex:BAP} incorrectly predicts initial stress in cases like \REF{mat:ex:Moskvica}. This fact suggests that post-accentuation is indeed a process, as proposed by \citet{Garde1998} and \citet{Melvold1989}: In \citeauthor{Melvold1989}'s approach post-accentuation is represented as a diacritic forcing post-cyclic movement of the appropriate parenthesis one syllable to the right, which is why it yields the correct outcome for \REF{mat:ex:MoskvicaDerivation}. Nonetheless, neither Melvold’s approach nor Garde’s predict that the structure resulting from \REF{mat:ex:Nu-e-u-Derivation-4-b} should be in any way problematic.\bigskip

\ea\label{mat:ex:MoskvicaDerivation}%The author requests that this example not be split across two pages
\ea\tikzstyle{every picture}+=[remember picture, inner sep=0pt,baseline, anchor=base,execute at begin node=\strut]
\gll M\tikz\node(o){o};skv \tikz\node(i){i};č \tikz\node(a){a}; \\
{Moscow} {\NMLZ} {{\SG}.{\GEN}}\\
 \begin{tikzpicture}[overlay,remember picture]
	 \node [above of = o, node distance=5mm] (o-ast) {~~$\ast_{\text{p}}$};
  \node [above of = i, node distance=5mm] (i-ast) {~~$\ast_{\text{p}}$};
  \node [above of = a, node distance=5mm] (a-ast) {$\ast$};
  \node [left of = o-ast, node distance=2mm] (left-bracket) {(};
  %\node [right of = o-ast, node distance=2mm] (p) {{\footnotesize p}};
  %\node [right of = i-ast, node distance=2mm] (p) {{\footnotesize p}};
  \node [left of = i-ast, node distance=2mm] (left-bracket) {(};
  \node [right of = a-ast, node distance=2mm] (right-bracket) {)};
  \coordinate (center) at ($(o)!0.5!(a)$);
  \node (arrow)   [below of = center,node distance=9mm]  {$\downarrow$};
\end{tikzpicture}\bigskip\bigskip
\ex
\gll M\tikz\node(o2){o};skv \tikz\node(i2){i};č \tikz\node(a2){a}; \\
{Moscow} {\NMLZ} {{\SG}.{\GEN}}\\
 \begin{tikzpicture}[overlay,remember picture]
	 \node [above of = o2, node distance=5mm] (o2-ast) {$\ast$};
  \node [above of = i2, node distance=5mm] (i2-ast) {~~$\ast_{\text{p}}$};
  \node [above of = a2, node distance=5mm] (a2-ast) {$\ast$};
  \node [left of = i2-ast, node distance=2mm] (left-bracket) {(};
  \node [left of = left-bracket, node distance=2mm] (left-bracket2) {(};
  \node [right of = a2-ast, node distance=2mm] (right-bracket) {)};
  \coordinate (center) at ($(o2)!0.5!(a2)$);
  \node (arrow)   [below of = center,node distance=9mm]  {$\downarrow$};
\end{tikzpicture}\bigskip\bigskip
\ex
\gll M\tikz\node(o3){o};skv \tikz\node(i3){i};č \tikz\node(a3){a}; \\
{Moscow} {\NMLZ} {{\SG}.{\GEN}}\\
 \begin{tikzpicture}[overlay,remember picture]
	 \node [above of = o3, node distance=5mm] (o3-ast) {$\ast$};
  \node [above of = i3, node distance=5mm] (i3-ast) {$\ast$};
  \node [above of = a3, node distance=5mm] (a3-ast) {$\ast$};
  \node [above of = a3-ast, node distance=5mm] (a3-ast2) {$\ast$};
  \node [left of = a3-ast, node distance=2mm] (left-bracket) {(};
  \node [right of = a3-ast, node distance=2mm] (right-bracket) {)};
\end{tikzpicture}
\z
\z

\noindent I will continue to use the enriched representation with square brackets because I believe that it not only encodes a valuable intuition about the source of the unstressability of the present-tense suffix, but also makes it possible to explain how this effect is nullified when the L-stem is post-accenting (see \sectref{mat:sec:Accentuation} on the correlation between unaccented \Affix{ot} verbs with the {1\SG} pattern and post-accenting \Affix{ow} verbs with the retracting pattern). More specifically, in section \sectref{mat:subsec:PostAccentingRole} I will show how post-stem stress is correctly predicted for post-accenting L-stems, and in section \sectref{mat:subsec:TwoPatternsAI}, how \Affix{a}/\Affix{i} verbs with post-accenting L-stems give rise to the stem-final pattern alternating with the {1\SG} pattern. 

\subsection{The role of a post-accenting stem}\label{mat:subsec:PostAccentingRole}

As discussed in \sectref{mat:subsec:AIandLStems}, the sequence of two post-accenting morphemes, as in \REF{mat:ex:Moskvica}, does not obey the Basic Accentuation Principle \REF{mat:ex:BAP}: Whereas in \citeposst{Halle1997} framework the surface stress is expected to coincide with the first accent, the real outcome is the same as if the first morpheme were unaccented. However, as will be shown now, the hypotheses I developed so far give rise to the correct outcome in a structure like \REF{mat:ex:Nu-e-u-Derivation-6-a}, where a post-accenting L-stem is followed by a post-accenting thematic suffix. 

As the vowel of the thematic suffix is followed by the vocalic present-tense suffix, the former is deleted, \REF{mat:ex:Nu-e-u-Derivation-6-b}. Once again, a sequence of two parentheses with no metrical elements between them (set in a box) is simplified to a single parenthesis and here it is crucial that the one deleted is the second one, yielding \REF{mat:ex:Nu-e-u-Derivation-6-c}. Note that the derivation proceeds left to right, so clash resolution precedes and bleeds the creation of an accentual conflict:\bigskip\bigskip\bigskip

\ea\label{mat:ex:Nu-e-t-Derivation-6}%The author requests that this example not be split across two pages
\ea\label{mat:ex:Nu-e-u-Derivation-6-a}\tikzstyle{every picture}+=[remember picture, inner sep=0pt,baseline, anchor=base,execute at begin node=\strut]
\gll \tikz\node(root){√-}; n\tikz\node(u){u};~~~ \tikz\node(e){e}; \tikz\node(t){t}; \\
{} {\THEM} {\PRS} {3\SG}\\
 \begin{tikzpicture}[overlay,remember picture]
  \node [above of = root, node distance=8mm] (root-syl) {σ};
  \node [above of = u, node distance=8mm] (u-syl) {σ};
  \node [above of = e, node distance=8mm] (e-syl) {σ};
	 \node [above of = root-syl, node distance=4mm] (root-ast) {$\ast$};
  \node [above of = u-syl, node distance=4mm] (u-ast) {$\ast$};
  \node [above of = e-syl, node distance=4mm] (e-ast) {$\ast$};
  \node [right of = root-ast, node distance=2mm] (left-bracket) {[};
  \node [right of = u-ast, node distance=2mm] (left-bracket) {[};
  \node [left of = e-ast, node distance=2mm] (left-bracket) {(};
  \node [right of = e-ast, node distance=2mm] (right-bracket) {)};
  \draw[-] (root-syl) -- (root);
  \draw[-] (u-syl) -- (u);
  \draw[-] (e-syl) -- (e);
  \draw[-] (e-syl) -- (t);
  \coordinate (center) at ($(root)!0.5!(t)$);
  \node (arrow)   [below of = center,node distance=9mm]  {$\downarrow$};
\end{tikzpicture}\bigskip\bigskip\bigskip\bigskip
\ex \label{mat:ex:Nu-e-u-Derivation-6-b}
\gll \tikz\node(root2){√-}; n\tikz\node(u2){\cancel{u}};~~~ \tikz\node(e2){e}; \tikz\node(t2){t}; \\
{} {\THEM} {\PRS} {3\SG}\\
 \begin{tikzpicture}[overlay,remember picture]
  \node [above of = root2, node distance=8mm] (root2-syl) {σ};
  \node [above of = e2, node distance=8mm] (e2-syl) {σ};
	 \node [above of = root2-syl, node distance=4mm] (root2-ast) {$\ast$};
  \node [above of = e2-syl, node distance=4mm] (e2-ast) {$\ast$};
  \node [right of = root2-ast, node distance=2mm] (left-bracket1) {[};
  \node [left of = e2-ast, node distance=2mm] (left-bracket2) {(};
  \node [right of = e2-ast, node distance=2mm] (left-bracket3) {[};
  \draw[-] (root2-syl) -- (root2);
  \draw[-] (e2-syl) -- (e2);
  \draw[-] (e2-syl) -- (t2);
  \node[draw=black, fit=(left-bracket1) (left-bracket2)](FIt1) {};
  \coordinate (center) at ($(root2)!0.5!(t2)$);
  \node (arrow)   [below of = center,node distance=9mm]  {$\downarrow$};
\end{tikzpicture}\bigskip\bigskip\bigskip\bigskip
\ex \label{mat:ex:Nu-e-u-Derivation-6-c}
\gll \tikz\node(root3){√-}; n\tikz\node(u3){\cancel{u}};~~~ \tikz\node(e3){e}; \tikz\node(t3){t}; \\
{} {\THEM} {\PRS} {3\SG}\\
 \begin{tikzpicture}[overlay,remember picture]
  \node [above of = root3, node distance=8mm] (root3-syl) {σ};
  \node [above of = e3, node distance=8mm] (e3-syl) {σ};
	 \node [above of = root3-syl, node distance=4mm] (root3-ast) {$\ast$};
  \node [above of = e3-syl, node distance=4mm] (e3-ast) {$\ast$};
  \node [right of = root3-ast, node distance=2mm] (left-bracket) {[};
  \node [right of = e3-ast, node distance=2mm] (left-bracket) {[};
  \draw[-] (root3-syl) -- (root3);
  \draw[-] (e3-syl) -- (e3);
  \draw[-] (e3-syl) -- (t3);
\end{tikzpicture}
\z
\z

\noindent The structure in \REF{mat:ex:Nu-e-u-Derivation-6-c} is clearly distinct from that in \REF{mat:ex:Nu-e-u-Derivation-4-b}: Here no conflicting instructions are associated to the same syllable. As a result, nothing is deleted from the metrical tier and the final post-accentuation is resolved to final stress.\footnote{If a left parenthesis is inserted before the final syllable, as in \REF{mat:ex:Nu-e-u-Derivation-4-e}, it will be deleted, as in \xxref{mat:ex:Nu-e-u-Derivation-6-b}{mat:ex:Nu-e-u-Derivation-6-c}. This is why I skip these steps in the derivation in \REF{mat:ex:Nu-e-t-Derivation-6}.}

To recap, with a post-accenting L-stem the deletion of the thematic vowel and the subsequent reassignment of its bracket to the present-tense suffix creates a metrical structure \REF{mat:ex:Nu-e-u-Derivation-6-b} that is identical to the combination of a post-accenting stem with an accented suffix \REF{mat:ex:MetricalTheory-BracketDeletion-a}, which is resolved by the deletion of the second accentual mark \REF{mat:ex:Nu-e-u-Derivation-6-c} in a manner fully parallel to \REF{mat:ex:MetricalTheory-BracketDeletion-b}. In other words, a post-accenting stem prevents the creation of an accentual conflict and stress is thus correctly predicted to fall on the present-tense suffix. 

While \textit{e}-verbs and \textit{i}-verbs (combining with the null present-tense suffix but yielding the same metrical structure as \textit{nu}-verbs) will be discussed in \sectref{mat:subsec:SecondConjugation}, in the next subsection I turn to the derivation of the stem-final present-tense pattern of \Affix{a}/\Affix{i} verbs.
 
\subsection{The accentual patterns of \textit{-a-}/\textit{-i-} verbs}\label{mat:subsec:TwoPatternsAI}

The class of \Affix{a}/\Affix{i} verbs is characterized by the thematic suffix \Affix{a} in the infinitive and the past tense and by transitive softening (fn. \ref{mat:fn:Iotation}) in the present. As transitive softening is known to arise from an underlying consonant-glide sequence, the thematic suffix is assumed to surface as [i] in the present tense, either as a result of a readjustment rule \citep{Bethin1992} or due to ablaut triggered by the present-tense suffix \citep{Matushansky2023a}.

As discussed in \sectref{mat:subsec:AIandLStems}, \Affix{a}/\Affix{i} verbs lack the post-stem pattern, which seems to be replaced by the stem-final one, as shown in \tabref{mat:tab:Interaction1stAI}, repeated below as \tabref{mat:tab:Interaction1stAI-rep}.

\begin{table}
\caption{Accentual interaction with the 1st conjugation suffix \Affix{a}/\Affix{i}}
\label{mat:tab:Interaction1stAI-rep}
 \begin{tabularx}{\textwidth}{lL{2.95cm}Z{1.6cm}Z{1.5cm}Z{1.8cm}Z{1.8cm}} 
  \lsptoprule
    &&accented\linebreak{\PRS-3\SG} &accented\linebreak{\PRS-1\SG}&accented\linebreak{\PST-\FEM.\SG}  &unaccented\linebreak{\PST-\PL}    \\
  \midrule
    a.  &stem:\newline\textit{-maz-} ‘smear’ 
        &máž-e-t    &máž-u    &máz-a-l-a  &máz-a-l-i 
        \\\addlinespace[5pt]
    b.  &\multicolumn{5}{X}{post-stem: N/A; potential candidates among \textit{j}-final stems can be assigned to the \Affix{a}/\Affix{Ø} class} 
        \\\addlinespace[5pt]
    c.  &{1\SG}:\newline\textit{-vʲaz-} ‘tie’ 
        &vʲáž-e-t    &vʲaž-ú    &vʲaz-á-l-a   &vʲaz-á-l-i   
        \\\addlinespace[5pt]
    d.  &stem-final present:\newline\textit{-koleb-} ‘rock’ 
        &koléblʲ-e-t &koléblʲ-u &koleb-á-l-a  &koleb-á-l-i  \\
  \lspbottomrule
 \end{tabularx}
\end{table}

The main difference between the thematic suffixes \Affix{a}/\Affix{i} and \Affix{nu} lies in the fate of the vowel: While the thematic vowel of \Affix{nu} is deleted in the present tense, the \Affix{a}/\Affix{i} suffix (or rather, its \Affix{i} allomorph) turns into a glide. I propose that this difference can derive the observed stress retraction with post-accenting stems.

As discussed above, when the thematic suffix \Affix{nu} follows an unaccented L-stem, the deletion of the thematic vowel before another vowel gives rise to an accentual conflict \REF{mat:ex:Nu-e-u-Derivation-4-b}. The problematic position (the present-tense suffix) is then deleted from the metrical tier \REF{mat:ex:Nu-e-u-Derivation-4-c}, and the resulting  post-accenting stem is realized with stem-final stress. The same outcome is correctly expected  to arise with the thematic suffix \Affix{i} \REF{mat:ex:i-e-t-Derivation-1}.

\newpage\vspace*{1em}

\ea\label{mat:ex:i-e-t-Derivation-1}%The author requests that this example not be split across two pages
\ea\label{mat:ex:i-e-t-Derivation-1-a}\tikzstyle{every picture}+=[remember picture, inner sep=0pt,baseline, anchor=base,execute at begin node=\strut]
\gll \tikz\node(root){√-}; \tikz\node(i){i};~~~ \tikz\node(e){e}; \tikz\node(t){t}; \\
{} {\THEM} {\PRS} {3\SG}\\
 \begin{tikzpicture}[overlay,remember picture]
  \node [above of = root, node distance=8mm] (root-syl) {σ};
  \node [above of = i, node distance=8mm] (i-syl) {σ};
  \node [above of = e, node distance=8mm] (e-syl) {σ};
	 \node [above of = root-syl, node distance=4mm] (root-ast) {$\ast$};
  \node [above of = i-syl, node distance=4mm] (i-ast) {$\ast$};
  \node [above of = e-syl, node distance=4mm] (e-ast) {$\ast$};
  \node [right of = i-ast, node distance=2mm] (right-bracket) {[};
  \node [left of = e-ast, node distance=2mm] (right-bracket2) {(};
  \node [right of = e-ast, node distance=2mm] (right-bracket) {)};
  \draw[-] (root-syl) -- (root);
  \draw[-] (i-syl) -- (i);
  \draw[-] (e-syl) -- (e);
  \draw[-] (e-syl) -- (t);
  \coordinate (center) at ($(root)!0.5!(t)$);
  \node (arrow)   [below of = center,node distance=9mm]  {$\downarrow$};
\end{tikzpicture}\bigskip\bigskip\bigskip\bigskip
\ex\label{mat:ex:i-e-t-Derivation-1-b}
\gll \tikz\node(root2){√-}; \tikz\node(j2){j};~~~ \tikz\node(e2){e}; \tikz\node(t2){t}; \\
{} {\THEM} {\PRS} {3\SG}\\
 \begin{tikzpicture}[overlay,remember picture]
  \node [above of = root2, node distance=8mm] (root2-syl) {σ};
  \node [above of = e2, node distance=8mm] (e2-syl) {σ};
	 \node [above of = root2-syl, node distance=4mm] (root2-ast) {$\ast$};
  \node [above of = e2-syl, node distance=4mm] (e2-ast) {$\ast$};
  \node [left of = e2-ast, node distance=2mm] (right-bracket2) {(};
  \node [right of = e2-ast, node distance=2mm] (right-bracket) {[};
  \draw[-] (root2-syl) -- (root2);
  \draw[-] (e2-syl) -- (j2);
  \draw[-] (e2-syl) -- (e2);
  \draw[-] (e2-syl) -- (t2);
  \coordinate (center) at ($(root2)!0.5!(t2)$);
  \node (arrow)   [below of = center,node distance=9mm]  {$\downarrow$};
\end{tikzpicture}\bigskip\bigskip\bigskip\bigskip
\ex\label{mat:ex:i-e-t-Derivation-1-c}
\gll \tikz\node(root3){√-}; \tikz\node(j3){j};~~~ \tikz\node(e3){e}; \tikz\node(t3){t}; \\
{} {\THEM} {\PRS} {3\SG}\\
 \begin{tikzpicture}[overlay,remember picture]
  \node [above of = root3, node distance=8mm] (root3-syl) {σ};
  \node [above of = e3, node distance=8mm] (e3-syl) {σ};
	 \node [above of = root3-syl, node distance=4mm] (root3-ast) {$\ast$};
  \node [above of = e3-syl, node distance=4mm] (e3-ast) {\strut};
  \node [left of = e3-ast, node distance=2mm] (right-bracket2) {(};
  \node [right of = e3-ast, node distance=2mm] (right-bracket) {[};
  \draw[-] (root3-syl) -- (root3);
  \draw[-] (e3-syl) -- (j3);
  \draw[-] (e3-syl) -- (e3);
  \draw[-] (e3-syl) -- (t3);
\end{tikzpicture}
\z
\z

\noindent To obtain the desired outcome (i.e., stem-final stress) for post-accenting stems, I capitalize on the difference between vowel deletion and glide formation. While the accent of the thematic suffix (indicated by shading in \REF{mat:ex:i-e-t-Derivation-2}) remains on the same syllable in both cases, I hypothesize that this is not true for the accent of the post-accenting stem (set in a box). I propose that if a vowel turns into a glide, the accent that would be assigned to it behaves like a word-final accent in that it is realized on the assigning syllable, i.e., on the final syllable of the stem (concurrently with the creation of a conflicting position, as in \xxref{mat:ex:i-e-t-Derivation-2-b}{mat:ex:i-e-t-Derivation-2-c}):\bigskip\bigskip\bigskip

\ea\label{mat:ex:i-e-t-Derivation-2}%The author requests that this example not be split across two pages
\ea\label{mat:ex:i-e-t-Derivation-2-a}\tikzstyle{every picture}+=[remember picture, inner sep=0pt,baseline, anchor=base,execute at begin node=\strut]
\gll \tikz\node(root){√-}; \tikz\node(i){i};~~~ \tikz\node(e){e}; \tikz\node(t){t}; \\
{} {\THEM} {\PRS} {3\SG}\\
 \begin{tikzpicture}[overlay,remember picture]
  \node [above of = root, node distance=8mm] (root-syl) {σ};
  \node [above of = i, node distance=8mm] (i-syl) {σ};
  \node [above of = e, node distance=8mm] (e-syl) {σ};
	 \node [above of = root-syl, node distance=4mm] (root-ast) {$\ast$};
  \node [above of = i-syl, node distance=4mm] (i-ast) {$\ast$};
  \node [above of = e-syl, node distance=4mm] (e-ast) {$\ast$};
  \node [right of = root-ast, node distance=2mm] (right-bracket) {[};
  \node [right of = i-ast, node distance=2mm, fill=gray!50] (right-bracket) {[};
  \node [left of = e-ast, node distance=2mm] (right-bracket2) {(};
  \node [right of = e-ast, node distance=2mm] (right-bracket) {)};
  \draw[-] (root-syl) -- (root);
  \draw[-] (i-syl) -- (i);
  \draw[-] (e-syl) -- (e);
  \draw[-] (e-syl) -- (t);
  \coordinate (center) at ($(root)!0.5!(t)$);
  \node (arrow)   [below of = center,node distance=9mm]  {$\downarrow$};
\end{tikzpicture}\bigskip\bigskip\bigskip\bigskip
\ex\label{mat:ex:i-e-t-Derivation-2-b}
\gll \tikz\node(root2){√-}; \tikz\node(j2){j};~~~ \tikz\node(e2){e}; \tikz\node(t2){t}; \\
{} {\THEM} {\PRS} {3\SG}\\
 \begin{tikzpicture}[overlay,remember picture]
  \node [above of = root2, node distance=8mm] (root2-syl) {σ};
  \node [above of = j2, node distance=8mm] (j2-syl) {\strut};
  \node [above of = e2, node distance=8mm] (e2-syl) {σ};
	 \node [above of = root2-syl, node distance=4mm] (root2-ast) {$\ast$};
  \node [above of = j2-syl, node distance=4mm] (j2-ast) {\strut};
  \node [above of = e2-syl, node distance=4mm] (e2-ast) {$\ast$};
  \node [right of = root2-ast, node distance=2mm] (left-bracket1) {[};
  \node [right of = j2-ast, node distance=2mm, fill=gray!50] (left-bracket2) {[};
  \node [left of = e2-ast, node distance=2mm] (right-bracket2) {(};
  \node [right of = e2-ast, node distance=2mm] (right-bracket) {[};
  \draw[-] (root2-syl) -- (root2);
  \draw[-] (e2-syl) -- (root2);
  \draw[-] (e2-syl) -- (j2);
  \draw[-] (e2-syl) -- (e2);
  \draw[-] (e2-syl) -- (t2);
  \node[draw=black, fit=(left-bracket1) (left-bracket1)](FIt1) {};
  \coordinate (center) at ($(root2)!0.5!(t2)$);
  \node (arrow)   [below of = center,node distance=9mm]  {$\downarrow$};
\end{tikzpicture}\bigskip\bigskip\bigskip\bigskip
\ex\label{mat:ex:i-e-t-Derivation-2-c}
\gll \tikz\node(root3){√-}; \tikz\node(j3){j};~~~ \tikz\node(e3){e}; \tikz\node(t3){t}; \\
{} {\THEM} {\PRS} {3\SG}\\
 \begin{tikzpicture}[overlay,remember picture]
  \node [above of = root3, node distance=8mm] (root3-syl) {σ};
  \node [above of = e3, node distance=8mm] (e3-syl) {σ};
	 \node [above of = root3-syl, node distance=4mm] (root3-ast) {$\ast$};
  \node [above of = e3-syl, node distance=4mm] (e3-ast) {$\ast$};
  \node [left of = root3-ast, node distance=2mm] (left-bracket1) {[};
  \node [left of = e3-ast, node distance=2mm] (right-bracket2) {(};
  \node [right of = e3-ast, node distance=2mm, fill=gray!50] (right-bracket) {[};
  \draw[-] (root3-syl) -- (root3);
  \draw[-] (e3-syl) -- (root3);
  \draw[-] (e3-syl) -- (j3);
  \draw[-] (e3-syl) -- (e3);
  \draw[-] (e3-syl) -- (t3);
  \node[draw=black, fit=(left-bracket1) (left-bracket1)](FIt1) {};
\end{tikzpicture}
\z
\z

\noindent As a result, even though the present-tense suffix is deleted from the metrical tier \REF{mat:ex:i-e-t-Derivation-2-d}, stress surfaces on the final syllable of the L-stem \REF{mat:ex:i-e-t-Derivation-2-e}:\bigskip\bigskip\bigskip

\begin{exe}\addtocounter{xnumi}{-1}%The author requests that this example not be split across two pages
\ex\label{mat:ex:i-e-t-Derivation-2-cont}
\begin{xlist}\addtocounter{xnumii}{3}
\ex\label{mat:ex:i-e-t-Derivation-2-d}\tikzstyle{every picture}+=[remember picture, inner sep=0pt,baseline, anchor=base,execute at begin node=\strut]
\gll \tikz\node(root4){√-}; \tikz\node(j4){j};~~~ \tikz\node(e4){e}; \tikz\node(t4){t}; \\
{} {\THEM} {\PRS} {3\SG}\\
 \begin{tikzpicture}[overlay,remember picture]
  \node [above of = root4, node distance=8mm] (root4-syl) {σ};
  \node [above of = e4, node distance=8mm] (e4-syl) {σ};
	 \node [above of = root4-syl, node distance=4mm] (root4-ast) {$\ast$};
  \node [above of = e4-syl, node distance=4mm] (e4-ast) {\strut};
  \node [left of = root4-ast, node distance=2mm] (right-bracket) {[};
  \node [left of = e4-ast, node distance=2mm] (right-bracket2) {(};
  \node [right of = e4-ast, node distance=2mm] (right-bracket) {[};
  \draw[-] (root4-syl) -- (root4);
  \draw[-] (e4-syl) -- (root4);
  \draw[-] (e4-syl) -- (j4);
  \draw[-] (e4-syl) -- (e4);
  \draw[-] (e4-syl) -- (t4);
  \coordinate (center) at ($(root4)!0.5!(t4)$);
  \node (arrow)   [below of = center,node distance=9mm]  {$\downarrow$};
\end{tikzpicture}\bigskip\bigskip\bigskip\bigskip
\ex\label{mat:ex:i-e-t-Derivation-2-e}
\gll \tikz\node(root5){√-}; \tikz\node(j5){j};~~~ \tikz\node(e5){e}; \tikz\node(t5){t}; \\
{} {\THEM} {\PRS} {3\SG}\\
 \begin{tikzpicture}[overlay,remember picture]
  \node [above of = root5, node distance=8mm] (root5-syl) {σ};
  \node [above of = e5, node distance=8mm] (e5-syl) {σ};
	 \node [above of = root5-syl, node distance=4mm] (root5-ast) {$\ast$};
  \node [above of = root5-ast, node distance=4mm] (root5-ast2) {$\ast$};
  \node [above of = e5-syl, node distance=4mm] (e5-ast) {\strut};
  \node [left of = root5-ast, node distance=2mm] (right-bracket) {[};
  \node [left of = e5-ast, node distance=2mm] (right-bracket2) {(};
  \node [right of = e5-ast, node distance=2mm] (right-bracket) {[};
  \draw[-] (root5-syl) -- (root5);
  \draw[-] (e5-syl) -- (root5);
  \draw[-] (e5-syl) -- (j5);
  \draw[-] (e5-syl) -- (e5);
  \draw[-] (e5-syl) -- (t5);
\end{tikzpicture}
\end{xlist}
\end{exe}

\noindent The natural question arises why glide formation makes a post-accenting stem become accented, while vowel deletion does not. Beyond noting that both strategies (accent retraction and accent advancement) seem equally valid outcomes for the disappearance of an accented vowel, I can provide no answer for the choice of strategy, it could also be lexically determined. Importantly, it has to be the post-accentuation of the preceding syllable that is affected by glide formation, since, as I will now show, the accent of the thematic suffix \Affix{i} is not retracted when it forms a glide in the {1\SG}.

\subsection{Second-conjugation verbs and the derivation of the {1\SG} pattern}\label{mat:subsec:SecondConjugation}

As noted above, the difference between the first and the second conjugations in Russian lies in the realization of the present-tense suffix: While in the first conjugation it is \Affix{e}, in the second conjugation it is zero. The paradigms in \tabref{mat:tab:Accents-i} and \tabref{mat:tab:InteractionThematicE}, repeated in \tabref{mat:tab:Accents-i-rep} and \tabref{mat:tab:InteractionThematicE-rep}, illustrate two facts: firstly, that second-conjugation verbs manifest two thematic suffixes in the past tense, \Affix{e} and \Affix{i}, both corresponding to /i/ in the present, and secondly, that in the present both classes of verbs exhibit the same three stress patterns as \textit{nu}-verbs: stem stress, post-stem stress and the {1\SG} pattern.

\begin{table}
\caption{Accentual interaction in thematic verbs, illustrated for the thematic suffix \textit{-i-}}
\label{mat:tab:Accents-i-rep}
 \begin{tabularx}{1\textwidth}{lL{2.92cm}XXL{1.8cm}X}
  \lsptoprule
    &   &{\PRS-1\SG}        &{\PRS-3\SG}  
        &{\PST-\FEM.\SG}       &{\PST-\PL}   \\
  \midrule
  a.    &stem: \newline\textit{-žal-} ‘sting’  
        &{žálʲ-u}        &{žál-i-t}
        &{žál-i-l-a}     &{žál-i-l-i}  \\\addlinespace[5pt]
  b.    &post-stem: \newline\textit{-govor-} ‘speak’  
        &{govorʲ-ú}      &{govor-í-t}
        &{govor-í-l-a}   &{govor-í-l-i}  \\\addlinespace[5pt]
  c.    &{1\SG}: \newline\textit{-lʲub-} ‘love’  
        &{lʲublʲ-ú}      &{lʲúb-i-t}
        &{lʲub-í-l-a}    &{lʲub-í-l-i}  \\
  \lspbottomrule
 \end{tabularx}
\end{table}

\begin{table}
\caption{Accentual interaction in thematic verbs, illustrated for the thematic suffix \Affix{e}}
\label{mat:tab:InteractionThematicE-rep}
 \begin{tabularx}{\textwidth}{lL{2.36cm}Z{1.8cm}Z{1.54cm}Z{2.08cm}Z{2.08cm}} 
  \lsptoprule
    &&accented\linebreak{\PRS-3\SG} &accented\linebreak{\PRS-1\SG}&accented\linebreak{\PST-\FEM.\SG}  &unaccented\linebreak{\PST-\PL}    \\
  \midrule
    a.  &stem:\newline\textit{-vid-} ‘see’ 
        &víd-i-t    &víž-u    &víd-e-l-a  &víd-e-l-i \\\addlinespace[5pt]
    b.  &post-stem:\newline\textit{-vel-} ‘order’ 
        &vel-í-t  &vel-ú     &vel-é-l-a   &vel-é-l-i  \\\addlinespace[5pt]
    c.  &{1\SG}\newline\textit{-vert-} ‘spin’ 
        &vért-i-t    &verč-ú    &vert-é-l-a   &vert-é-l-i   \\
  \lspbottomrule
 \end{tabularx}
\end{table}

Two types of explanations have been given for the lack of the thematic suffix \Affix{e} in the present tense. One proposal (\citealt{Jakobson1948,Melvold1989}) is that the thematic vowel \Affix{e} is deleted before the present-tense suffix \Affix{i} \REF{mat:ex:GoritVowelDeletion}. The alternative (\citealt{Micklesen1973}, \citealt{CoatsLightner1975}, \citealt[129--130]{Itkin2007}, \citealt{Matushansky2023b}) is that the second-conjugation present-tense suffix is null, and the thematic vowel \Affix{e} is raised to [i] in the present tense \REF{mat:ex:GoritVowelChange}.

\ea
\ea \label{mat:ex:GoritVowelDeletion} [[[gor-e]$_\text{2}$-i]$_\text{3}$-t]$_\text{4}$ $\rightarrow$ [[[gor-\cancel{e}]$_\text{2}$-i]$_\text{3}$-t]$_\text{4}$ → [gorit] \hfill (vowel deletion)
\ex \label{mat:ex:GoritVowelChange} [[[gor-e]$_\text{2}$-Ø]$_\text{3}$-t]$_\text{4}$ → [[[gor-i]$_\text{2}$-Ø]$_\text{3}$-t]$_\text{4}$ → [gorit] \hfill (vowel change)
\z
\z

\noindent With the former approach the derivation of the three accentual patterns proceeds along exactly like for \textit{nu}-verbs. In the latter approach to obtain the {1\SG} pattern and its nullification with post-accenting L-stems it is necessary to assume that the null present-tense suffix also introduces an accent. 

I begin with a concrete verb (\tabref{mat:tab:Accents-i-rep}c) exhibiting the {1\SG} pattern: stem-final stress before consonantal suffixes and final stress on vocalic ones. I propose that, as before, the thematic suffix is post-accenting while the null present-tense suffix introduces an accent \REF{mat:ex:ljubit-Derivation-1-a}. Since the present-tense suffix is non-segmental, its accent is assigned to the syllable of the thematic suffix \REF{mat:ex:ljubit-Derivation-1-b}, and the resulting accentual conflict leads to the deletion of the thematic suffix from the metrical tier \REF{mat:ex:ljubit-Derivation-1-c}. As in \REF{mat:ex:Nu-e-u-Derivation-4-c} and \REF{mat:ex:i-e-t-Derivation-1-a}, superfluous parentheses are removed and stress is realized on the final syllable of the stem:\bigskip\bigskip\bigskip

\ea\label{mat:ex:ljubit-Derivation-1}%The author requests that this example not be split across two pages
\ea\label{mat:ex:ljubit-Derivation-1-a}\tikzstyle{every picture}+=[remember picture, inner sep=0pt,baseline, anchor=base,execute at begin node=\strut]
\gll \tikz\node(lj){lʲ};\tikz\node(ljub){u};\tikz\node(b){b}; \tikz\node(i){i}; \tikz\node(null){∅}; \tikz\node(t){t}; \\
{√} {\THEM} {\PRS} {3\SG}\\
 \begin{tikzpicture}[overlay,remember picture]
  \node [above of = ljub, node distance=8mm] (ljub-syl) {σ};
  \node [above of = i, node distance=8mm] (i-syl) {σ};
	 \node [above of = ljub-syl, node distance=4mm] (ljub-ast) {$\ast$};
  \node [above of = i-syl, node distance=4mm] (i-ast) {$\ast$};
  \node [right of = i-ast, node distance=2mm] (left-bracket) {[};
  \node [right of = left-bracket, node distance=2mm] (left-bracket2) {(};
  \draw[-] (ljub-syl) -- (ljub);
  \draw[-] (ljub-syl) -- (lj);
  \draw[-] (i-syl) -- (i);
  \draw[-] (i-syl) -- (b);
  \coordinate (center) at ($(ljub)!0.5!(t)$);
  \node (arrow)   [below of = center,node distance=9mm]  {$\downarrow$};
\end{tikzpicture}\bigskip\bigskip\bigskip\bigskip
\ex\label{mat:ex:ljubit-Derivation-1-b}
\gll \tikz\node(lj2){lʲ};\tikz\node(ljub2){u};\tikz\node(b2){b}; \tikz\node(i2){i}; \tikz\node(null2){∅}; \tikz\node(t2){t}; \\
{√} {\THEM} {\PRS} {3\SG}\\
 \begin{tikzpicture}[overlay,remember picture]
  \node [above of = ljub2, node distance=8mm] (ljub2-syl) {σ};
  \node [above of = i2, node distance=8mm] (i2-syl) {σ};
	 \node [above of = ljub2-syl, node distance=4mm] (ljub2-ast) {$\ast$};
  \node [above of = i2-syl, node distance=4mm] (i2-ast) {$\ast$};
  \node [right of = i2-ast, node distance=2mm] (left2-bracket) {[};
  \node [left of = i2-ast, node distance=2mm] (left2-bracket2) {(};
  \draw[-] (ljub2-syl) -- (ljub2);
  \draw[-] (ljub2-syl) -- (lj2);
  \draw[-] (i2-syl) -- (i2);
  \draw[-] (i2-syl) -- (b2);
  \coordinate (center) at ($(ljub2)!0.5!(t2)$);
  \node (arrow)   [below of = center,node distance=9mm]  {$\downarrow$};
\end{tikzpicture}\bigskip\bigskip\bigskip\bigskip
\ex\label{mat:ex:ljubit-Derivation-1-c}
\gll \tikz\node(lj3){lʲ};\tikz\node(ljub3){u};\tikz\node(b3){b}; \tikz\node(i3){i}; \tikz\node(null3){∅}; \tikz\node(t3){t}; \\
{√} {\THEM} {\PRS} {3\SG}\\
 \begin{tikzpicture}[overlay,remember picture]
  \node [above of = ljub3, node distance=8mm] (ljub3-syl) {σ};
  \node [above of = i3, node distance=8mm] (i3-syl) {σ};
	 \node [above of = ljub3-syl, node distance=4mm] (ljub3-ast) {$\ast$};
  \node [above of = i3-syl, node distance=4mm] (i3-ast) {\strut};
  \node [right of = i3-ast, node distance=2mm] (left3-bracket) {[};
  \node [left of = i3-ast, node distance=2mm] (left3-bracket2) {(};
  \draw[-] (ljub3-syl) -- (ljub3);
  \draw[-] (ljub3-syl) -- (lj3);
  \draw[-] (i3-syl) -- (i3);
  \draw[-] (i3-syl) -- (b3);
  \draw[-] (i3-syl) -- (t3);
\end{tikzpicture}
\z
\z

\noindent In the {1\SG}, on the other hand, the problematic position is followed by another vowel \REF{mat:ex:ljublju-Derivation-1-a}, which should (and does) turn the thematic suffix into a glide. Recall that in \sectref{mat:subsec:TwoPatternsAI} the thematic suffix \Affix{a}/\Affix{i} formed a glide in the present tense, which was taken as the reason why \Affix{a}/\Affix{i} verbs exhibit stem-final stress: I proposed that if a vowel turns into a glide, the accent assigned to it is shifted one syllable to the left \REF{mat:ex:i-e-t-Derivation-2}. Is \REF{mat:ex:ljublju-Derivation-1} incorrectly predicted to also give rise to stem-final stress?\bigskip\bigskip\bigskip

\ea\label{mat:ex:ljublju-Derivation-1}%The author requests that this example not be split across two pages
\ea\label{mat:ex:ljublju-Derivation-1-a}\tikzstyle{every picture}+=[remember picture, inner sep=0pt,baseline, anchor=base,execute at begin node=\strut]
\gll \tikz\node(lj){lʲ};\tikz\node(ljub){u};\tikz\node(b){b}; \tikz\node(i){i}; \tikz\node(null){∅}; \tikz\node(u){u}; \\
{√} {\THEM} {\PRS} {1\SG}\\
 \begin{tikzpicture}[overlay,remember picture]
  \node [above of = ljub, node distance=8mm] (ljub-syl) {σ};
  \node [above of = i, node distance=8mm] (i-syl) {σ};
  \node [above of = u, node distance=8mm] (u-syl) {σ};
	 \node [above of = ljub-syl, node distance=4mm] (ljub-ast) {$\ast$};
  \node [above of = i-syl, node distance=4mm] (i-ast) {$\ast$};
  \node [above of = u-syl, node distance=4mm] (u-ast) {$\ast$};
  \node [right of = i-ast, node distance=2mm] (left-bracket) {[};
  \node [left of = i-ast, node distance=2mm] (left-bracket) {(};
  \node [right of = u-ast, node distance=2mm] (right-bracket) {)};
  \draw[-] (ljub-syl) -- (ljub);
  \draw[-] (ljub-syl) -- (lj);
  \draw[-] (i-syl) -- (i);
  \draw[-] (i-syl) -- (b);
  \draw[-] (u-syl) -- (u);
  \coordinate (center) at ($(ljub)!0.5!(u)$);
  \node (arrow)   [below of = center,node distance=9mm]  {$\downarrow$};
\end{tikzpicture}\bigskip\bigskip\bigskip\bigskip
\ex\label{mat:ex:ljublju-Derivation-1-b}
\gll \tikz\node(lj2){lʲ};\tikz\node(ljub2){u};\tikz\node(b2){b}; \tikz\node(i2){i}; \tikz\node(null2){∅}; \tikz\node(u2){u}; \\
{√} {\THEM} {\PRS} {1\SG}\\
 \begin{tikzpicture}[overlay,remember picture]
  \node [above of = ljub2, node distance=8mm] (ljub2-syl) {σ};
  \node [above of = i2, node distance=8mm] (i2-syl) {σ};
  \node [above of = u2, node distance=8mm] (u2-syl) {σ};
	 \node [above of = ljub2-syl, node distance=4mm] (ljub2-ast) {$\ast$};
  \node [above of = i2-syl, node distance=4mm] (i2-ast) {\strut};
  \node [above of = u2-syl, node distance=4mm] (u2-ast) {$\ast$};
  \node [right of = i2-ast, node distance=2mm] (left-bracket) {[};
  \node [left of = i2-ast, node distance=2mm] (left-bracket) {(};
  \node [right of = u2-ast, node distance=2mm] (right-bracket) {)};
  \draw[-] (ljub2-syl) -- (ljub2);
  \draw[-] (ljub2-syl) -- (lj2);
  \draw[-] (i2-syl) -- (i2);
  \draw[-] (i2-syl) -- (b2);
  \draw[-] (u2-syl) -- (u2);
\end{tikzpicture}
\z
\z

\noindent The prediction is avoided because glide formation is timed differently in the two derivations. In \REF{mat:ex:i-e-t-Derivation-2-cont} glide formation both precedes and causes the creation of a problematic position, while in \REF{mat:ex:ljublju-Derivation-1-cont} the accentual conflict deleting the thematic suffix from the metrical tier arises before its conversion into a glide: First the accentual conflict is resolved by the deletion of the problematic position from the metrical tier \REF{mat:ex:ljublju-Derivation-1-c}, and then a glide is formed \REF{mat:ex:ljublju-Derivation-1-d}. As this glide formation cannot affect accentuation, stress falls on the vowel of the {1\SG} ending \REF{mat:ex:ljublju-Derivation-1-e}. Stress assignment in \textit{i}-verbs therefore provides an argument for a cyclic approach to Russian accentuation.

\bigskip\bigskip\bigskip

\begin{exe}\addtocounter{xnumi}{-1}%The author requests that this example not be split across two pages
\ex\label{mat:ex:ljublju-Derivation-1-cont}
\begin{xlist}\addtocounter{xnumii}{2}
\ex\label{mat:ex:ljublju-Derivation-1-c}\tikzstyle{every picture}+=[remember picture, inner sep=0pt,baseline, anchor=base,execute at begin node=\strut]
\gll \tikz\node(lj3){lʲ};\tikz\node(ljub3){u};\tikz\node(b3){b}; \tikz\node(i3){i}; \tikz\node(null3){∅}; \tikz\node(u3){u}; \\
{√} {\THEM} {\PRS} {1\SG}\\
 \begin{tikzpicture}[overlay,remember picture]
  \node [above of = ljub3, node distance=8mm] (ljub3-syl) {σ};
  \node [above of = i3, node distance=8mm] (i3-syl) {σ};
  \node [above of = u3, node distance=8mm] (u3-syl) {σ};
	 \node [above of = ljub3-syl, node distance=4mm] (ljub3-ast) {$\ast$};
  \node [above of = i3-syl, node distance=4mm] (i3-ast) {\strut};
  \node [above of = u3-syl, node distance=4mm] (u3-ast) {$\ast$};
  \node [left of = i3-ast, node distance=2mm] (left-bracket) {(};
  \node [right of = u3-ast, node distance=2mm] (right-bracket) {)};
  \draw[-] (ljub3-syl) -- (ljub3);
  \draw[-] (ljub3-syl) -- (lj3);
  \draw[-] (i3-syl) -- (i3);
  \draw[-] (i3-syl) -- (b3);
  \draw[-] (u3-syl) -- (u3);
  \coordinate (center) at ($(ljub3)!0.5!(u3)$);
  \node (arrow)   [below of = center,node distance=9mm]  {$\downarrow$};
\end{tikzpicture}
\newpage\vspace*{1em}
\ex\label{mat:ex:ljublju-Derivation-1-d}
\gll \tikz\node(lj4){lʲ};\tikz\node(ljub4){u};\tikz\node(b4){b}; \tikz\node(i4){j}; \tikz\node(null4){∅}; \tikz\node(u4){u}; \\
{√} {\THEM} {\PRS} {1\SG}\\
 \begin{tikzpicture}[overlay,remember picture]
  \node [above of = ljub4, node distance=8mm] (ljub4-syl) {σ};
  \node [above of = i4, node distance=8mm] (i4-syl) {\strut};
  \node [above of = u4, node distance=8mm] (u4-syl) {σ};
	 \node [above of = ljub4-syl, node distance=4mm] (ljub4-ast) {$\ast$};
  \node [above of = i4-syl, node distance=4mm] (i4-ast) {\strut};
  \node [above of = u4-syl, node distance=4mm] (u4-ast) {$\ast$};
  \node [left of = i4-ast, node distance=2mm] (left-bracket) {(};
  \node [right of = u4-ast, node distance=2mm] (right-bracket) {)};
  \draw[-] (ljub4-syl) -- (ljub4);
  \draw[-] (ljub4-syl) -- (lj4);
  \draw[-] (u4-syl) -- (u4);
  \draw[-] (u4-syl.south) -- (i4.north);
  \draw[-] (u4-syl.south) -- (b4.north);
  \coordinate (center) at ($(ljub4)!0.5!(u4)$);
  \node (arrow)   [below of = center,node distance=9mm]  {$\downarrow$};
\end{tikzpicture}\bigskip\bigskip\bigskip\bigskip
\ex\label{mat:ex:ljublju-Derivation-1-e}
\gll \tikz\node(lj5){lʲ};\tikz\node(ljub5){u};\tikz\node(b5){b}; \tikz\node(i5){j}; \tikz\node(null5){∅}; \tikz\node(u5){u}; \\
{√} {\THEM} {\PRS} {1\SG}\\
 \begin{tikzpicture}[overlay,remember picture]
  \node [above of = ljub5, node distance=8mm] (ljub5-syl) {σ};
  \node [above of = i5, node distance=8mm] (i5-syl) {\strut};
  \node [above of = u5, node distance=8mm] (u5-syl) {σ};
	 \node [above of = ljub5-syl, node distance=4mm] (ljub5-ast) {$\ast$};
  \node [above of = i5-syl, node distance=4mm] (i5-ast) {\strut};
  \node [above of = u5-syl, node distance=4mm] (u5-ast) {$\ast$};
  \node [above of = u5-ast, node distance=4mm] (u5-ast2) {$\ast$};
  \node [left of = i5-ast, node distance=2mm] (left-bracket) {(};
  \node [right of = u5-ast, node distance=2mm] (right-bracket) {)};
  \draw[-] (ljub5-syl) -- (ljub5);
  \draw[-] (ljub5-syl) -- (lj5);
  \draw[-] (u5-syl) -- (u5);
  \draw[-] (u5-syl.south) -- (i5.north);
  \draw[-] (u5-syl.south) -- (b5.north);
\end{tikzpicture}
\end{xlist}
\end{exe}

\noindent If the L-stem is post-accenting, the derivation proceeds as in \REF{mat:ex:Nu-e-t-Derivation-6}: The accentual conflict is prevented because, before the present-tense suffix can influence the outcome, the second of the two accents not divided by metrical material is deleted.

\subsection{Summary}

I have proposed that the {1\SG} pattern arises from induced unstressability: An accented suffix is deleted from the metrical tier when it receives two conflicting accentual specifications, which is what happens when the vowel of a post-accenting suffix turns into a glide or is deleted before an accented vowel. When the L-stem is post-accenting, the accent of the thematic suffix has to be deleted, which straightforwardly derives the post-stem stress with \Affix{nu}verbs. For \Affix{a}/\Affix{i} verbs an additional assumption is required (cf. \citealt[254]{Melvold1989}) that when a glide is formed, the accent assigned to it shifts to the preceding syllable. 

The advantage of this view is that it derives the three stress patterns from the independently motivated property of L-stem accentuation: Unaccented stems exhibit the {1\SG} pattern and post-accenting stems surface with consistent stress position unless a glide is formed.

\section{Conclusion and questions for future research}\label{mat:sec:Conclusion}

I have proposed that the accentual behavior of thematic verbs in the present tense can be linked transparently to the accentual specification of the L-stem and to the accentuation of the thematic suffix. The combination of an unaccented L-stem and a post-accenting thematic suffix creates a configuration that makes the present-tense suffix unstressable by forcing it off the metrical grid. A post-accenting L-stem is hypothesized to remove the problematic thematic accent and so not to create such a problem, yielding post-stem stress for all thematic suffixes, except \Affix{a}/\Affix{i}, which yields stem-final stress because of glide formation. Since both the first- and second-conjugation present-tense suffixes come into conflict with a post-accenting thematic suffix, it is apparently not the concrete vowel that has this property, but rather the abstract morpheme.\footnote{While I have chosen to present this analysis as a series of representations, it can be equally easily cast in a rule-based framework and in OT.} The natural question is whether nominal stress retraction (\citealt{Halle1973,Halle1975,Halle1997,Melvold1989,Revithiadou1999,Alderete1999,Butska2002,mat:Dubina2012,Osadcha2019}, etc.) can be accounted for by the same mechanism. Given that nouns can exhibit retraction in the singular and in the plural and that both unaccented and post-accenting nouns can trigger it, more work is needed to determine if nominal retraction is the same phenomenon. The same issue arises for adjectival retraction.

The empirical contributions of this study include the facts that the {1\SG} pattern is dependent on the deletion of an accent-bearing thematic suffix, that it is not equally frequent with different thematic suffixes and that it correlates with an unaccented L-stem. This approach can explain why the thematic suffixes \Affix{aj} and \Affix{ej} do not give rise to the {1\SG} pattern: As their vowels are not deleted before the present-tense suffix, their accent will not shift. The reason why the non-productive mutative suffix \Affix{nu} and the thematic suffix \Affix{a}/\Affix{Ø} do not yield the {1\SG} pattern is that the latter is not accented, and the former is pre-accenting, so accentuation is not affected by the deletion of their thematic vowel.\footnote{Intuitively, pre-accentuation operates on the already existing structure, unlike post-accentuation, which is an instruction for the structure to be built, so it is reasonable to assume that at the present-tense cycle the accent of a pre-accenting suffix has already been assigned to the stem-final syllable.}

The alternation of the stem-final stress pattern for the \Affix{a}/\Affix{i} suffix with the post-stem pattern for all other thematic vowels triggering the {1\SG} pattern has allowed us to determine the thematic suffix for some \textit{j}-final verbs.

While the intuition that the {1\SG} pattern arises from induced unstressability can be accounted for in the terms of the Halle-Vergnaud framework, the hypothesis that this unstressability is due to an accentual conflict between post-accentuation and accentuation cannot be expressed with the tools of this framework: Accented and post-accenting morphemes in it have the same effect, the only distinction being the position of the accent. Though I have adjusted the notation to encode the postulated difference between accentuation and post-accentuation, this change goes against the core principles of the framework, where a parenthesis indicates a foot boundary rather than an instruction to include or not include the carrier syllable into the foot created. Since I believe that this enrichment makes it possible to account for rather complex phenomena, the question arises whether the Halle-Vergnaud framework can be made compatible with this more complex notation, or another framework should be used, where the simultaneous placement of a bracket 
and a parenthesis on the same metrical position can be represented as a conflict of instructions, e.g., with post-accentuation representing the tail of an iambic foot or by treating the two types of accents as tones \citep{Matushansky2023b}. 

A number of empirical questions remain. Some, like the interaction of vowel deletion and glide formation with secondary imperfective suffixes, will be left for future research. Others, like the persistence of stem accentuation across verb classes and the derivation of the {1\SG} pattern in verb classes where it is an exception, will be discussed in the remainder of this section.

\subsection{Unproductive {1\SG} pattern: Athematic verbs, \textit{e}-verbs, \textit{nu}-verbs, and \textit{a}-verbs}

In this subsection I will discuss verb classes for which the {1\SG} pattern is attested only with a small number of verbs. As in one of these classes (\sectref{mat:subsubsec:AthematicVerbs}) the {1\SG} pattern is manifested only when the stem is prefixed with two specific prefixes, I will propose that in all these cases the {1\SG} pattern arises from idiosyncratic lexical specification.

\subsubsection{Two athematic verbs}\label{mat:subsubsec:AthematicVerbs}

As mentioned in fn. \ref{mat:fn:FifthClass}, there are two athematic stems giving rise to the {1\SG} pattern in the derived verbs: \textit{-mog-} (\textit{močʲ} ‘to be able’, \textit{pomóčʲ} ‘to help’) and the cranberry root \textit{-im-}/\textit{-nʲa-} (\textit{prinʲátʲ} ‘to accept’, \textit{podnʲátʲ} ‘to raise’, \textit{obnʲátʲ} ‘to hug’, etc.).

\begin{table}
\caption{Two athematic {1\SG} pattern verbs}
\label{mat:tab:TwoAthematic}
 \begin{tabularx}{\textwidth}{lL{3.5cm}CCCC} 
  \lsptoprule
    &&{\PST-\FEM.\SG} &{\PST-\PL} &{\PRS-1\SG} &{\PRS-3\SG}   \\
  \midrule
    a.  &\textit{-pri.m/pri.nʲa-} ‘accept’ 
        &pri.nʲa-l-\ul{á}    &prí.nʲa-l-i
        &pri.m-\ul{~~}-ú      &prí.m-\ul{e}-t \\
    b.  &\textit{-mog\ul{~~}-} ‘be able’
        &mog\ul{~~}-l-\ul{á}    &mog\ul{~~}-l-í
        &mog\ul{~~}-\ul{~~}-ú
        &móž\ul{~~}-\ul{e}-t \\
  \lspbottomrule
 \end{tabularx}
\end{table}


The verb \textit{prinʲátʲ} ‘to accept’ in \tabref{mat:tab:TwoAthematic}a exhibits accentual variability in the past, as expected from an unaccented stem, while the verb \textit{močʲ} ‘to be able’ in \tabref{mat:tab:TwoAthematic}b surfaces in the past with consistent word-final stress that is indicative of a post-accenting stem. While at first blush it might seem that these facts argue against the link between an unaccented stem and the {1\SG} pattern, there is no thematic vowel deletion here to create an accentual conflict. These verbs can be regarded as lexically specified to delete the present-tense suffix from the metrical tier.

\subsubsection{Six \textit{-nu-} verbs}

There are only six \textit{nu}-stems that exhibit the {1\SG} stress pattern in the present tense. Four of them form perfective verbs only (\textit{pomʲanútʲ} ‘to remember’, \textit{obmanútʲ} ‘to cheat’, \textit{vzglʲanútʲ} ‘to glance’, and \textit{minútʲ} ‘to elapse’) and can therefore be regarded as semelfactive, while two are imperfective (\textit{tonútʲ} ‘to drown’ and \textit{tʲanútʲ} ‘to pull’) and may involve the same suffix \Affix{nu} as mutative verbs.

As discussed above, the imperfective suffix \Affix{nu} is pre-accenting, so its deletion does not give rise to an accentual conflict. I propose that the reason why the semelfactive \Affix{nu} does not give rise to the {1\SG} pattern is that it is accented. For both types of \Affix{nu} I propose that the six stems above force the suffix to become post-accenting. As \REF{mat:ex:gljad} shows, such exceptional behavior can target some prefixed derivatives of a particular root:

\ea\label{mat:ex:gljad} \textit{-glʲad-} ‘look’
\ea\label{mat:ex:gljanu}
\ea\label{mat:ex:gljanu-i}{glʲánu}/{glʲánet} ‘will glance.{1\SG}/{3\SG}’ (semelfactive suffix \Affix{nu}) \\\hspace*\fill (stem)\\
\ex {proglʲánet} ‘will glance through.{3\SG}, impers.’ (ibid.)\\
\z
\ex\label{mat:ex:gljazu} {glʲažú}/{glʲadít} ‘look.{1\SG}/{3\SG}’ (suffix \Affix{e}) \hfill (post-stem)\\
\ex\label{mat:ex:vzgljanu}
\ea\label{mat:ex:vzgljanu-i} {vzglʲanú}/{vzglʲánet} ‘will glance.{1\SG}/{3\SG}’ \hfill ({1\SG})\\
\ex {zaglʲanú}/{zaglʲánet} ‘will look in on.{1\SG}/{3\SG}’, etc.
\z
\z
\z

\noindent The stem stress in \REF{mat:ex:gljanu} suggests that the root is accented, while the post-stem stress in \REF{mat:ex:gljazu} is explained by the fact that \Affix{e} is dominant (see \sectref{mat:subsubsec:FiveEVerbs}). However, the behavior of \REF{mat:ex:vzgljanu} is unexpected both for an accented root and for the accented \Affix{nu}.\footnote{There is no clear difference in meaning between \REF{mat:ex:gljanu-i} and \REF{mat:ex:vzgljanu-i}, but \REF{mat:ex:gljanu-i} is either archaic or dialectal.} While the stipulation that the thematic suffix \Affix{nu} is exceptionally post-accenting in the prefixed verbs in \REF{mat:ex:vzgljanu} accounts for their stress pattern, it cannot be independently motivated. Nonetheless, given that the combination of a prefixed stem and a thematic suffix can be semantically non-compositional or idiomatic, phonological unpredictability can also be accommodated.

\subsubsection{Five \textit{e-}verbs}\label{mat:subsubsec:FiveEVerbs}

The same issue arises when the exceptional character of accentual variance with \textit{e}-verbs is considered. Only five out of the ca. 80 second-conjugation \textit{e}-verbs surface with stem stress (\textit{slɨ́šatʲ} ‘to hear’, \textit{zavísetʲ} ‘to depend’, \textit{vídetʲ} ‘to see’, \textit{nenavídetʲ} ‘to hate’, and \textit{obídetʲ} ‘to offend’, with the last three diachronically derived from the same root \Affix{vid}), which strongly suggests that the thematic suffix \Affix{e} is accentually dominant. Support for this claim comes from the fact that, on the basis of all \textit{e}-verbs that have corresponding semelfactives (16 verbs) or mutatives (4 verbs), stem stress in \Affix{nu} verbs systematically corresponds to post-stem stress in \Affix{e} verbs. If the thematic suffix \Affix{e} is accented and dominant, it will remove the underlying accent of the L-stem:

\ea\label{mat:ex:krik} \textit{-krik-} ‘shout’
\ea\label{mat:ex:kriknu} {kríknu}/{kríknet} ‘will give a shout.{1\SG}/{3\SG}’ (semelfactive suffix \textit{-nu-}) \\\hspace*{\fill} (stem)\\
\ex\label{mat:ex:kricu} {kričú}/ {kričít} ‘shout.{1\SG}/{3\SG}’ (suffix \Affix{e}) \hfill (post-stem)\\
\z
\ex\label{mat:ex:perd} \textit{-perd-} ‘fart’ (vulgar)
\ea\label{mat:ex:pjordnu} {pʲórdnu}/{pʲórdnet} ‘will give a fart.{1\SG}/{3\SG}’ (semelfactive suffix \Affix{nu}) \hfill (stem)\\
\ex\label{mat:ex:perzu} {peržú}/{perdít} ‘fart.{1\SG}/{3\SG}’ (suffix \Affix{e}) \hfill (post-stem)\\
\z
\ex\label{mat:ex:molk} \textit{-molk-} ‘be silent’
\ea\label{mat:ex:molknu} {mólknu}/{mólknet} ‘be silent.{1\SG}/{3\SG}’ (pre-accenting mutative suffix \Affix{nu}) \hfill ({stem})\\
\ex\label{mat:ex:molcu} {molčú}/{molčít} ‘be silent.{1\SG}/{3\SG}’ (suffix \Affix{e}) \hfill ({post-stem})\\
\z
\z

\noindent If the dominant suffix \Affix{e} were post-accenting, we would wrongly expect systematic accentual variance, as in \tabref{mat:tab:InteractionThematicE-rep}c: if the L-stem accent is removed, it becomes unaccented. However, only five \textit{e}-verbs show the {1\SG} pattern (\textit{deržátʲ} ‘to hold’, \textit{terpétʲ} ‘to tolerate’, \textit{smotrétʲ} ‘to look’, \textit{vertétʲ} ‘to turn’, and \textit{dɨšátʲ} ‘to breathe’). The prevalence of the post-stem pattern \tabref{mat:tab:InteractionThematicE-rep}b in \textit{e}-verbs (ca. 70 verbs out of 80) therefore strongly suggests that the suffix \Affix{e} is accented. No accentual conflict arises with the null accented present-tense suffix, and the Basic Accentuation Principle \REF{mat:ex:BAP} predicts systematic surface stress on the thematic vowel, barring the five accented stems.

To derive the {1\SG} pattern the same analysis can be appealed to as that proposed for the thematic suffix \Affix{e} in the preceding subsection: Suppose that these five \textit{e}-verbs take the post-accenting allomorph of the thematic suffix (or force it to become post-accenting).

\subsubsection{Two {1\SG} \textit{-a-}/\textit{-$\emptyset$-} verbs}

To complete the empirical picture, the unproductive thematic suffix \Affix{a}/\Affix{Ø} is unaccented, as shown by the fact that it permits accentual variability in the past tense (fn. \ref{mat:fn:UnnaccentedA}). Like athematic verbs though, this class also includes two verbs with the {1\SG} pattern:\footnote{The {1\SG} and the gerund forms of the verb \textit{stonátʲ} ‘to moan’ are ineffable (on paradigm gaps in the {1\SG} of Russian verbs see \citet{Sims2006}, \citet{DalandEtAl2007}, \citet{Pertsova2016}, etc.). The form of its imperative is also compatible with the \Affix{a}/\Affix{i} theme, which may be the reason why it exists. The verb \textit{sratʲ} ‘to shit’ has several conjugational variants, \REF{mat:ex:srat} is merely one of them.}

\ea\label{mat:ex:sratstonat}
\ea\label{mat:ex:srat} {sratʲ}/{serú}/{séret} ‘shit.{\INF}.{1\SG}/{3\SG}’ 
\ex\label{mat:ex:stonat} {stonátʲ}/{stoní}/{stónet} ‘moan.{\INF}{1\SG}/{3\SG}’
\z
\z

\noindent Following the reasoning suggested above, I hypothesize that these roots are lexical exceptions triggering post-accentuation of the thematic suffix.

\subsubsection{Summary}

Given that four classes of verbs exceptionally give rise to the {1\SG}, which is regular in two other thematic classes, an appeal to lexical exceptions appears to be inevitable. As it does not seem reasonable to postulate a post-accenting allomorph for each of the three thematic suffixes for which the {1\SG} pattern constitutes an exception, I hypothesize that these stems can force post-accentuation of the thematic suffix.

Of the fifteen verbs in question (2 athematic verbs, 5 \textit{e}-verbs, 6 \textit{nu}-verbs and 2 \textit{a}-verbs) only two have counterparts in other thematic classes that could have given rise to the {1\SG} pattern:

\ea\label{mat:ex:dysatminut} 
\ea\label{mat:ex:dysat} {dɨšátʲ}: ‘to breathe’: \\\Affix{nu}: {dɨxnú}/{dɨxnʲót} ‘provide a breathing sample.{1\SG}/{3\SG}’
\ex\label{mat:ex:minut} {minútʲ}: ‘to elapse’: \\\Affix{ow}: {minúju}/{minúešʲ} ‘elapse.{\IPFV}.{1\SG}/{3\SG}’ ({minovátʲ} {\INF})
\z
\z

\noindent Though these two stems do not yield the {1\SG} pattern with other thematic suffixes, as shown in \REF{mat:ex:dysatminut}, they may be expected not to: \Affix{ow} is post-accenting, and the \Affix{nu} derivation may involve a different, if related, root.

\subsection{L-stem accentuation across verb classes}

The evidence (\sectref{mat:subsec:AIandLStems}) linking the {1\SG} pattern to unaccented L-stems is rather tenuous, but for each L-stem its accentuation, once determined for one verb class, is predicted to persist in another. To exclude some potential lines of further inquiry, I would like to report that I have found no correlation between the {1\SG} pattern and the form of the secondary imperfective. The accentual relation between semelfactive \textit{nu}-verbs and their imperfective counterparts in \Affix{i} does not seem to be predictable either: Although all {1\SG} \textit{i}-verbs that I have looked at have post-stem stress in the semelfactive, other stress patterns do not appear to be linked to each other (though \REF{mat:ex:voskresit} seems to be exceptional in that it involves a valency change):

\ea
\ea katítʲ ‘to roll’ ({1\SG}), katnútʲ (final)
\ex skolʲzítʲ ‘to slide’ (post-stem), skolʲznútʲ (post-stem)
\ex číllitʲ ‘to chill out’ (stem), čilʲnútʲ (post-stem)
\ex\label{mat:ex:voskresit} voskresítʲ ‘to resurrect’ (post-stem), voskrésnutʲ ‘to be resurrected’ (stem)
\z
\z

\noindent A brief examination of stress patterns in minimal pairs composed of semelfactive \textit{nu}-verbs and their imperfective counterparts in \Affix{a}/\Affix{aj} also suggests that one form cannot be predicted from the other:

\ea
\ea brɨ́znutʲ (stem)/brɨ́zgatʲ (stem) ‘to spatter’
\ex zevnútʲ (post-stem)/zevátʲ (post-stem) ‘to yawn’ 
\ex šmɨgnútʲ (post-stem)/šmɨ́gatʲ (stem) ‘to dart’ 
\ex kínutʲ (stem)/kidátʲ (post-stem) ‘to toss’
\z
\z

\noindent Even though the first two patterns with stress retention are the most frequent, the existence of the latter two requires an explanation, which does not seem to be provided by postulating any type of accent or lack thereof on the L-stem.

Derivational morphology is just as inconclusive: As shown by examples \xxref{mat:ex:iAccentedNouns}{mat:ex:iUnaccentedNouns}, there does not seem to be a transparent relation between the accentuation of a noun and that of the verb that it is derived from. I leave the issue of apparently indeterminate accentuation of thematic L-stems for future research.


%Just comment out the input below when you're ready to go.
%For a start: Do not forget to give your Overleaf project (this paper) a recognizable name. This one could be called, for instance, Simik et al: OSL template. You can change the name of the project by hovering over the gray title at the top of this page and clicking on the pencil icon.

\section{Introduction}\label{sim:sec:intro}

Language Science Press is a project run for linguists, but also by linguists. You are part of that and we rely on your collaboration to get at the desired result. Publishing with LangSci Press might mean a bit more work for the author (and for the volume editor), esp. for the less experienced ones, but it also gives you much more control of the process and it is rewarding to see the quality result.

Please follow the instructions below closely, it will save the volume editors, the series editors, and you alike a lot of time.

\sloppy This stylesheet is a further specification of three more general sources: (i) the Leipzig glossing rules \citep{leipzig-glossing-rules}, (ii) the generic style rules for linguistics (\url{https://www.eva.mpg.de/fileadmin/content_files/staff/haspelmt/pdf/GenericStyleRules.pdf}), and (iii) the Language Science Press guidelines \citep{Nordhoff.Muller2021}.\footnote{Notice the way in-text numbered lists should be written -- using small Roman numbers enclosed in brackets.} It is advisable to go through these before you start writing. Most of the general rules are not repeated here.\footnote{Do not worry about the colors of references and links. They are there to make the editorial process easier and will disappear prior to official publication.}

Please spend some time reading through these and the more general instructions. Your 30 minutes on this is likely to save you and us hours of additional work. Do not hesitate to contact the editors if you have any questions.

\section{Illustrating OSL commands and conventions}\label{sim:sec:osl-comm}

Below I illustrate the use of a number of commands defined in langsci-osl.tex (see the styles folder).

\subsection{Typesetting semantics}\label{sim:sec:sem}

See below for some examples of how to typeset semantic formulas. The examples also show the use of the sib-command (= ``semantic interpretation brackets''). Notice also the the use of the dummy curly brackets in \REF{sim:ex:quant}. They ensure that the spacing around the equation symbol is correct. 

\ea \ea \sib{dog}$^g=\textsc{dog}=\lambda x[\textsc{dog}(x)]$\label{sim:ex:dog}
\ex \sib{Some dog bit every boy}${}=\exists x[\textsc{dog}(x)\wedge\forall y[\textsc{boy}(y)\rightarrow \textsc{bit}(x,y)]]$\label{sim:ex:quant}
\z\z

\noindent Use noindent after example environments (but not after floats like tables or figures).

And here's a macro for semantic type brackets: The expression \textit{dog} is of type $\stb{e,t}$. Don't forget to place the whole type formula into a math-environment. An example of a more complex type, such as the one of \textit{every}: $\stb{s,\stb{\stb{e,t},\stb{e,t}}}$. You can of course also use the type in a subscript: dog$_{\stb{e,t}}$

We distinguish between metalinguistic constants that are translations of object language, which are typeset using small caps, see \REF{sim:ex:dog}, and logical constants. See the contrast in \REF{sim:ex:speaker}, where \textsc{speaker} (= serif) in \REF{sim:ex:speaker-a} is the denotation of the word \textit{speaker}, and \cnst{speaker} (= sans-serif) in \REF{sim:ex:speaker-b} is the function that maps the context $c$ to the speaker in that context.\footnote{Notice that both types of small caps are automatically turned into text-style, even if used in a math-environment. This enables you to use math throughout.}$^,$\footnote{Notice also that the notation entails the ``direct translation'' system from natural language to metalanguage, as entertained e.g. in \citet{Heim.Kratzer1998}. Feel free to devise your own notation when relying on the ``indirect translation'' system (see, e.g., \citealt{Coppock.Champollion2022}).}

\ea\label{sim:ex:speaker}
\ea \sib{The speaker is drunk}$^{g,c}=\textsc{drunk}\big(\iota x\,\textsc{speaker}(x)\big)$\label{sim:ex:speaker-a}
\ex \sib{I am drunk}$^{g,c}=\textsc{drunk}\big(\cnst{speaker}(c)\big)$\label{sim:ex:speaker-b}
\z\z

\noindent Notice that with more complex formulas, you can use bigger brackets indicating scope, cf. $($ vs. $\big($, as used in \REF{sim:ex:speaker}. Also notice the use of backslash plus comma, which produces additional space in math-environment.

\subsection{Examples and the minsp command}

Try to keep examples simple. But if you need to pack more information into an example or include more alternatives, you can resort to various brackets or slashes. For that, you will find the minsp-command useful. It works as follows:

\ea\label{sim:ex:german-verbs}\gll Hans \minsp{\{} schläft / schlief / \minsp{*} schlafen\}.\\
Hans {} sleeps {} slept {} {} sleep.\textsc{inf}\\
\glt `Hans \{sleeps / slept\}.'
\z

\noindent If you use the command, glosses will be aligned with the corresponding object language elements correctly. Notice also that brackets etc. do not receive their own gloss. Simply use closed curly brackets as the placeholder.

The minsp-command is not needed for grammaticality judgments used for the whole sentence. For that, use the native langsci-gb4e method instead, as illustrated below:

\ea[*]{\gll Das sein ungrammatisch.\\
that be.\textsc{inf} ungrammatical\\
\glt Intended: `This is ungrammatical.'\hfill (German)\label{sim:ex:ungram}}
\z

\noindent Also notice that translations should never be ungrammatical. If the original is ungrammatical, provide the intended interpretation in idiomatic English.

If you want to indicate the language and/or the source of the example, place this on the right margin of the translation line. Schematic information about relevant linguistic properties of the examples should be placed on the line of the example, as indicated below.

\ea\label{sim:ex:bailyn}\gll \minsp{[} Ėtu knigu] čitaet Ivan \minsp{(} často).\\
{} this book.{\ACC} read.{\PRS.3\SG} Ivan.{\NOM} {} often\\\hfill O-V-S-Adv
\glt `Ivan reads this book (often).'\hfill (Russian; \citealt[4]{Bailyn2004})
\z

\noindent Finally, notice that you can use the gloss macros for typing grammatical glosses, defined in langsci-lgr.sty. Place curly brackets around them.

\subsection{Citation commands and macros}

You can make your life easier if you use the following citation commands and macros (see code):

\begin{itemize}
    \item \citealt{Bailyn2004}: no brackets
    \item \citet{Bailyn2004}: year in brackets
    \item \citep{Bailyn2004}: everything in brackets
    \item \citepossalt{Bailyn2004}: possessive
    \item \citeposst{Bailyn2004}: possessive with year in brackets
\end{itemize}

\section{Trees}\label{s:tree}

Use the forest package for trees and place trees in a figure environment. \figref{sim:fig:CP} shows a simple example.\footnote{See \citet{VandenWyngaerd2017} for a simple and useful quickstart guide for the forest package.} Notice that figure (and table) environments are so-called floating environments. {\LaTeX} determines the position of the figure/table on the page, so it can appear elsewhere than where it appears in the code. This is not a bug, it is a property. Also for this reason, do not refer to figures/tables by using phrases like ``the table below''. Always use tabref/figref. If your terminal nodes represent object language, then these should essentially correspond to glosses, not to the original. For this reason, we recommend including an explicit example which corresponds to the tree, in this particular case \REF{sim:ex:czech-for-tree}.

\ea\label{sim:ex:czech-for-tree}\gll Co se řidič snažil dělat?\\
what {\REFL} driver try.{\PTCP.\SG.\MASC} do.{\INF}\\
\glt `What did the driver try to do?'
\z

\begin{figure}[ht]
% the [ht] option means that you prefer the placement of the figure HERE (=h) and if HERE is not possible, you prefer the TOP (=t) of a page
% \centering
    \begin{forest}
    for tree={s sep=1cm, inner sep=0, l=0}
    [CP
        [DP
            [what, roof, name=what]
        ]
        [C$'$
            [C
                [\textsc{refl}]
            ]
            [TP
                [DP
                    [driver, roof]
                ]
                [T$'$
                    [T [{[past]}]]
                    [VP
                        [V
                            [tried]
                        ]
                        [VP, s sep=2.2cm
                            [V
                                [do.\textsc{inf}]
                            ]
                            [t\textsubscript{what}, name=trace-what]
                        ]
                    ]
                ]
            ]
        ]
    ]
    \draw[->,overlay] (trace-what) to[out=south west, in=south, looseness=1.1] (what);
    % the overlay option avoids making the bounding box of the tree too large
    % the looseness option defines the looseness of the arrow (default = 1)
    \end{forest}
    \vspace{3ex} % extra vspace is added here because the arrow goes too deep to the caption; avoid such manual tweaking as much as possible; here it's necessary
    \caption{Proposed syntactic representation of \REF{sim:ex:czech-for-tree}}
    \label{sim:fig:CP}
\end{figure}

Do not use noindent after figures or tables (as you do after examples). Cases like these (where the noindent ends up missing) will be handled by the editors prior to publication.

\section{Italics, boldface, small caps, underlining, quotes}

See \citet{Nordhoff.Muller2021} for that. In short:

\begin{itemize}
    \item No boldface anywhere.
    \item No underlining anywhere (unless for very specific and well-defined technical notation; consult with editors).
    \item Small caps used for (i) introducing terms that are important for the paper (small-cap the term just ones, at a place where it is characterized/defined); (ii) metalinguistic translations of object-language expressions in semantic formulas (see \sectref{sim:sec:sem}); (iii) selected technical notions.
    \item Italics for object-language within text; exceptionally for emphasis/contrast.
    \item Single quotes: for translations/interpretations
    \item Double quotes: scare quotes; quotations of chunks of text.
\end{itemize}

\section{Cross-referencing}

Label examples, sections, tables, figures, possibly footnotes (by using the label macro). The name of the label is up to you, but it is good practice to follow this template: article-code:reference-type:unique-label. E.g. sim:ex:german would be a proper name for a reference within this paper (sim = short for the author(s); ex = example reference; german = unique name of that example).

\section{Syntactic notation}

Syntactic categories (N, D, V, etc.) are written with initial capital letters. This also holds for categories named with multiple letters, e.g. Foc, Top, Num, etc. Stick to this convention also when coming up with ad hoc categories, e.g. Cl (for clitic or classifier).

An exception from this rule are ``little'' categories, which are written with italics: \textit{v}, \textit{n}, \textit{v}P, etc.

Bar-levels must be typeset with bars/primes, not with an apostrophe. An easy way to do that is to use mathmode for the bar: C$'$, Foc$'$, etc.

Specifiers should be written this way: SpecCP, Spec\textit{v}P.

Features should be surrounded by square brackets, e.g., [past]. If you use plus and minus, be sure that these actually are plus and minus, and not e.g. a hyphen. Mathmode can help with that: [$+$sg], [$-$sg], [$\pm$sg]. See \sectref{sim:sec:hyphens-etc} for related information.

\section{Footnotes}

Absolutely avoid long footnotes. A footnote should not be longer than, say, {20\%} of the page. If you feel like you need a long footnote, make an explicit digression in the main body of the text.

Footnotes should always be placed at the end of whole sentences. Formulate the footnote in such a way that this is possible. Footnotes should always go after punctuation marks, never before. Do not place footnotes after individual words. Do not place footnotes in examples, tables, etc. If you have an urge to do that, place the footnote to the text that explains the example, table, etc.

Footnotes should always be formulated as full, self-standing sentences.

\section{Tables}

For your tables use the table plus tabularx environments. The tabularx environment lets you (and requires you in fact) to specify the width of the table and defines the X column (left-alignment) and the Y column (right-alignment). All X/Y columns will have the same width and together they will fill out the width of the rest of the table -- counting out all non-X/Y columns.

Always include a meaningful caption. The caption is designed to appear on top of the table, no matter where you place it in the code. Do not try to tweak with this. Tables are delimited with lsptoprule at the top and lspbottomrule at the bottom. The header is delimited from the rest with midrule. Vertical lines in tables are banned. An example is provided in \tabref{sim:tab:frequencies}. See \citet{Nordhoff.Muller2021} for more information. If you are typesetting a very complex table or your table is too large to fit the page, do not hesitate to ask the editors for help.

\begin{table}
\caption{Frequencies of word classes}
\label{sim:tab:frequencies}
 \begin{tabularx}{.77\textwidth}{lYYYY} %.77 indicates that the table will take up 77% of the textwidth
  \lsptoprule
            & nouns & verbs  & adjectives & adverbs\\
  \midrule
  absolute  &   12  &    34  &    23      & 13\\
  relative  &   3.1 &   8.9  &    5.7     & 3.2\\
  \lspbottomrule
 \end{tabularx}
\end{table}

\section{Figures}

Figures must have a good quality. If you use pictorial figures, consult the editors early on to see if the quality and format of your figure is sufficient. If you use simple barplots, you can use the barplot environment (defined in langsci-osl.sty). See \figref{sim:fig:barplot} for an example. The barplot environment has 5 arguments: 1. x-axis desription, 2. y-axis description, 3. width (relative to textwidth), 4. x-tick descriptions, 5. x-ticks plus y-values.

\begin{figure}
    \centering
    \barplot{Type of meal}{Times selected}{0.6}{Bread,Soup,Pizza}%
    {
    (Bread,61)
    (Soup,12)
    (Pizza,8)
    }
    \caption{A barplot example}
    \label{sim:fig:barplot}
\end{figure}

The barplot environment builds on the tikzpicture plus axis environments of the pgfplots package. It can be customized in various ways. \figref{sim:fig:complex-barplot} shows a more complex example.

\begin{figure}
  \begin{tikzpicture}
    \begin{axis}[
	xlabel={Level of \textsc{uniq/max}},  
	ylabel={Proportion of $\textsf{subj}\prec\textsf{pred}$}, 
	axis lines*=left, 
        width  = .6\textwidth,
	height = 5cm,
    	nodes near coords, 
    % 	nodes near coords style={text=black},
    	every node near coord/.append style={font=\tiny},
        nodes near coords align={vertical},
	ymin=0,
	ymax=1,
	ytick distance=.2,
	xtick=data,
	ylabel near ticks,
	x tick label style={font=\sffamily},
	ybar=5pt,
	legend pos=outer north east,
	enlarge x limits=0.3,
	symbolic x coords={+u/m, \textminus u/m},
	]
	\addplot[fill=red!30,draw=none] coordinates {
	    (+u/m,0.91)
        (\textminus u/m,0.84)
	};
	\addplot[fill=red,draw=none] coordinates {
	    (+u/m,0.80)
        (\textminus u/m,0.87)
	};
	\legend{\textsf{sg}, \textsf{pl}}
    \end{axis} 
  \end{tikzpicture} 
    \caption{Results divided by \textsc{number}}
    \label{sim:fig:complex-barplot}
\end{figure}

\section{Hyphens, dashes, minuses, math/logical operators}\label{sim:sec:hyphens-etc}

Be careful to distinguish between hyphens (-), dashes (--), and the minus sign ($-$). For in-text appositions, use only en-dashes -- as done here -- with spaces around. Do not use em-dashes (---). Using mathmode is a reliable way of getting the minus sign.

All equations (and typically also semantic formulas, see \sectref{sim:sec:sem}) should be typeset using mathmode. Notice that mathmode not only gets the math signs ``right'', but also has a dedicated spacing. For that reason, never write things like p$<$0.05, p $<$ 0.05, or p$<0.05$, but rather $p<0.05$. In case you need a two-place math or logical operator (like $\wedge$) but for some reason do not have one of the arguments represented overtly, you can use a ``dummy'' argument (curly brackets) to simulate the presence of the other one. Notice the difference between $\wedge p$ and ${}\wedge p$.

In case you need to use normal text within mathmode, use the text command. Here is an example: $\text{frequency}=.8$. This way, you get the math spacing right.

\section{Abbreviations}

The final abbreviations section should include all glosses. It should not include other ad hoc abbreviations (those should be defined upon first use) and also not standard abbreviations like NP, VP, etc.


\section{Bibliography}

Place your bibliography into localbibliography.bib. Important: Only place there the entries which you actually cite! You can make use of our OSL bibliography, which we keep clean and tidy and update it after the publication of each new volume. Contact the editors of your volume if you do not have the bib file yet. If you find the entry you need, just copy-paste it in your localbibliography.bib. The bibliography also shows many good examples of what a good bibliographic entry should look like.

See \citet{Nordhoff.Muller2021} for general information on bibliography. Some important things to keep in mind:

\begin{itemize}
    \item Journals should be cited as they are officially called (notice the difference between and, \&, capitalization, etc.).
    \item Journal publications should always include the volume number, the issue number (field ``number''), and DOI or stable URL (see below on that).
    \item Papers in collections or proceedings must include the editors of the volume (field ``editor''), the place of publication (field ``address'') and publisher.
    \item The proceedings number is part of the title of the proceedings. Do not place it into the ``volume'' field. The ``volume'' field with book/proceedings publications is reserved for the volume of that single book (e.g. NELS 40 proceedings might have vol. 1 and vol. 2).
    \item Avoid citing manuscripts as much as possible. If you need to cite them, try to provide a stable URL.
    \item Avoid citing presentations or talks. If you absolutely must cite them, be careful not to refer the reader to them by using ``see...''. The reader can't see them.
    \item If you cite a manuscript, presentation, or some other hard-to-define source, use the either the ``misc'' or ``unpublished'' entry type. The former is appropriate if the text cited corresponds to a book (the title will be printed in italics); the latter is appropriate if the text cited corresponds to an article or presentation (the title will be printed normally). Within both entries, use the ``howpublished'' field for any relevant information (such as ``Manuscript, University of \dots''). And the ``url'' field for the URL.
\end{itemize}

We require the authors to provide DOIs or URLs wherever possible, though not without limitations. The following rules apply:

\begin{itemize}
    \item If the publication has a DOI, use that. Use the ``doi'' field and write just the DOI, not the whole URL.
    \item If the publication has no DOI, but it has a stable URL (as e.g. JSTOR, but possibly also lingbuzz), use that. Place it in the ``url'' field, using the full address (https: etc.).
    \item Never use DOI and URL at the same time.
    \item If the official publication has no official DOI or stable URL (related to the official publication), do not replace these with other links. Do not refer to published works with lingbuzz links, for instance, as these typically lead to the unpublished (preprint) version. (There are exceptions where lingbuzz or semanticsarchive are the official publication venue, in which case these can of course be used.) Never use URLs leading to personal websites.
    \item If a paper has no DOI/URL, but the book does, do not use the book URL. Just use nothing.
\end{itemize}

\section*{Abbreviations}

\begin{tabularx}{.5\textwidth}{@{}lQ}
\textsc{1}&first person\\
\textsc{2}&second person\\
\textsc{3}&third person\\
{\ACC}  &accusative\\
\textsc{dim}  &diminutive   \\
{\FEM}  &feminine   \\
{\GEN}  &genitive\\
\textsc{ger}  &gerund   \\
{\INF}  &infinitive\\
{\INS}  &instrumental\\
\end{tabularx}%
\begin{tabularx}{.5\textwidth}{lQ@{}}
{\IPFV}  &imperfective\\
{\MASC} &masculine\\
{\NEUT}  &neuter\\
{\NMLZ}  &nominalizer\\
{\NOM}  &nominative\\
{\PL}   &plural  \\
{\PRS}  &present tense  \\
{\PST}  &past tense  \\
{\SG}   &singular   \\
{\THEM} &thematic suffix    \\
%\\ % this dummy row achieves correct vertical alignment of both tables
\end{tabularx}

\section*{Acknowledgments}
I am very grateful to the audiences at FDSL 15 (October 5--7, 2022. Humboldt-Universität zu Berlin), SinFonIJa 15 (September 22--24, 2002, University of Udine), and the SLE workshop ``Lexical and fixed word stress: Representation, Production and Perception'' (August 24--27, 2022, University of Bucharest), where various attempts to describe and account for these phenomena were presented, for their questions and comments.

\printbibliography[heading=subbibliography,notkeyword=this]

\end{document}
