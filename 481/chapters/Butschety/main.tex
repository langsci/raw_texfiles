\documentclass[output=paper,colorlinks,citecolor=brown]{langscibook}
\ChapterDOI{10.5281/zenodo.15394172}
%\bibliography{localbibliography.bib}

\author{Madeleine Butschety\orcid{0009-0006-6439-627X}
\affiliation{University of Nova Gorica}}
% replace the above with you and your coauthors
% rules for affiliation: If there's an official English version, use that (find out on the official website of the university); if not, use the original
% orcid doesn't appear printed; it's metainformation used for later indexing

%%% uncomment the following line if you are a single author or all authors have the same affiliation
\SetupAffiliations{mark style=none}

%% in case the running head with authors exceeds one line (which is the case in this example document), use one of the following methods to turn it into a single line; otherwise comment the line below out with % and ignore it
%\lehead{Šimík, Gehrke, Lenertová, Meyer, Szucsich \& Zaleska}
% \lehead{Radek Šimík et al.}

\title{A quantification-based approach to plural pronoun comitatives}
% replace the above with your paper title
%%% provide a shorter version of your title in case it doesn't fit a single line in the running head
% in this form: \title[short title]{full title}
\abstract{Plural pronoun constructions (PPCs) consist of a plural pronoun and a comitative (i.e. \textit{with-}) phrase. In sentences such as \textit{My s Petej pojdëm domoj} (lit. `We with Petja will-go home') from Russian, PPCs are ambiguous between a default interpretation according to which a plural referent \textit{we} will go home with Petja (=ePPC), and an unexpected interpretation according to which Petja and the speaker will go home (=iPPC). I show that this ambiguity can be derived under the assumption that plural pronouns and (universal) quantifiers have some striking properties in common. In particular, I argue that the unexpected iPPC reading arises if the comitative phrase occurs inside the restrictor of the plural pronoun (which is similar to a quantifier's restrictor), and an ePPC reading arises if it occurs elsewhere in the structure. My account further offers an explanation regarding the availability and distribution of iPPC interpretations within and across Slavic languages.

\keywords{plural pronoun construction, comitative, quantifier, restrictor, Torlakian BCMS, Bulgarian}
}

\begin{document}
\maketitle

% Just comment out the input below when you're ready to go.
%For a start: Do not forget to give your Overleaf project (this paper) a recognizable name. This one could be called, for instance, Simik et al: OSL template. You can change the name of the project by hovering over the gray title at the top of this page and clicking on the pencil icon.

\section{Introduction}\label{sim:sec:intro}

Language Science Press is a project run for linguists, but also by linguists. You are part of that and we rely on your collaboration to get at the desired result. Publishing with LangSci Press might mean a bit more work for the author (and for the volume editor), esp. for the less experienced ones, but it also gives you much more control of the process and it is rewarding to see the quality result.

Please follow the instructions below closely, it will save the volume editors, the series editors, and you alike a lot of time.

\sloppy This stylesheet is a further specification of three more general sources: (i) the Leipzig glossing rules \citep{leipzig-glossing-rules}, (ii) the generic style rules for linguistics (\url{https://www.eva.mpg.de/fileadmin/content_files/staff/haspelmt/pdf/GenericStyleRules.pdf}), and (iii) the Language Science Press guidelines \citep{Nordhoff.Muller2021}.\footnote{Notice the way in-text numbered lists should be written -- using small Roman numbers enclosed in brackets.} It is advisable to go through these before you start writing. Most of the general rules are not repeated here.\footnote{Do not worry about the colors of references and links. They are there to make the editorial process easier and will disappear prior to official publication.}

Please spend some time reading through these and the more general instructions. Your 30 minutes on this is likely to save you and us hours of additional work. Do not hesitate to contact the editors if you have any questions.

\section{Illustrating OSL commands and conventions}\label{sim:sec:osl-comm}

Below I illustrate the use of a number of commands defined in langsci-osl.tex (see the styles folder).

\subsection{Typesetting semantics}\label{sim:sec:sem}

See below for some examples of how to typeset semantic formulas. The examples also show the use of the sib-command (= ``semantic interpretation brackets''). Notice also the the use of the dummy curly brackets in \REF{sim:ex:quant}. They ensure that the spacing around the equation symbol is correct. 

\ea \ea \sib{dog}$^g=\textsc{dog}=\lambda x[\textsc{dog}(x)]$\label{sim:ex:dog}
\ex \sib{Some dog bit every boy}${}=\exists x[\textsc{dog}(x)\wedge\forall y[\textsc{boy}(y)\rightarrow \textsc{bit}(x,y)]]$\label{sim:ex:quant}
\z\z

\noindent Use noindent after example environments (but not after floats like tables or figures).

And here's a macro for semantic type brackets: The expression \textit{dog} is of type $\stb{e,t}$. Don't forget to place the whole type formula into a math-environment. An example of a more complex type, such as the one of \textit{every}: $\stb{s,\stb{\stb{e,t},\stb{e,t}}}$. You can of course also use the type in a subscript: dog$_{\stb{e,t}}$

We distinguish between metalinguistic constants that are translations of object language, which are typeset using small caps, see \REF{sim:ex:dog}, and logical constants. See the contrast in \REF{sim:ex:speaker}, where \textsc{speaker} (= serif) in \REF{sim:ex:speaker-a} is the denotation of the word \textit{speaker}, and \cnst{speaker} (= sans-serif) in \REF{sim:ex:speaker-b} is the function that maps the context $c$ to the speaker in that context.\footnote{Notice that both types of small caps are automatically turned into text-style, even if used in a math-environment. This enables you to use math throughout.}$^,$\footnote{Notice also that the notation entails the ``direct translation'' system from natural language to metalanguage, as entertained e.g. in \citet{Heim.Kratzer1998}. Feel free to devise your own notation when relying on the ``indirect translation'' system (see, e.g., \citealt{Coppock.Champollion2022}).}

\ea\label{sim:ex:speaker}
\ea \sib{The speaker is drunk}$^{g,c}=\textsc{drunk}\big(\iota x\,\textsc{speaker}(x)\big)$\label{sim:ex:speaker-a}
\ex \sib{I am drunk}$^{g,c}=\textsc{drunk}\big(\cnst{speaker}(c)\big)$\label{sim:ex:speaker-b}
\z\z

\noindent Notice that with more complex formulas, you can use bigger brackets indicating scope, cf. $($ vs. $\big($, as used in \REF{sim:ex:speaker}. Also notice the use of backslash plus comma, which produces additional space in math-environment.

\subsection{Examples and the minsp command}

Try to keep examples simple. But if you need to pack more information into an example or include more alternatives, you can resort to various brackets or slashes. For that, you will find the minsp-command useful. It works as follows:

\ea\label{sim:ex:german-verbs}\gll Hans \minsp{\{} schläft / schlief / \minsp{*} schlafen\}.\\
Hans {} sleeps {} slept {} {} sleep.\textsc{inf}\\
\glt `Hans \{sleeps / slept\}.'
\z

\noindent If you use the command, glosses will be aligned with the corresponding object language elements correctly. Notice also that brackets etc. do not receive their own gloss. Simply use closed curly brackets as the placeholder.

The minsp-command is not needed for grammaticality judgments used for the whole sentence. For that, use the native langsci-gb4e method instead, as illustrated below:

\ea[*]{\gll Das sein ungrammatisch.\\
that be.\textsc{inf} ungrammatical\\
\glt Intended: `This is ungrammatical.'\hfill (German)\label{sim:ex:ungram}}
\z

\noindent Also notice that translations should never be ungrammatical. If the original is ungrammatical, provide the intended interpretation in idiomatic English.

If you want to indicate the language and/or the source of the example, place this on the right margin of the translation line. Schematic information about relevant linguistic properties of the examples should be placed on the line of the example, as indicated below.

\ea\label{sim:ex:bailyn}\gll \minsp{[} Ėtu knigu] čitaet Ivan \minsp{(} často).\\
{} this book.{\ACC} read.{\PRS.3\SG} Ivan.{\NOM} {} often\\\hfill O-V-S-Adv
\glt `Ivan reads this book (often).'\hfill (Russian; \citealt[4]{Bailyn2004})
\z

\noindent Finally, notice that you can use the gloss macros for typing grammatical glosses, defined in langsci-lgr.sty. Place curly brackets around them.

\subsection{Citation commands and macros}

You can make your life easier if you use the following citation commands and macros (see code):

\begin{itemize}
    \item \citealt{Bailyn2004}: no brackets
    \item \citet{Bailyn2004}: year in brackets
    \item \citep{Bailyn2004}: everything in brackets
    \item \citepossalt{Bailyn2004}: possessive
    \item \citeposst{Bailyn2004}: possessive with year in brackets
\end{itemize}

\section{Trees}\label{s:tree}

Use the forest package for trees and place trees in a figure environment. \figref{sim:fig:CP} shows a simple example.\footnote{See \citet{VandenWyngaerd2017} for a simple and useful quickstart guide for the forest package.} Notice that figure (and table) environments are so-called floating environments. {\LaTeX} determines the position of the figure/table on the page, so it can appear elsewhere than where it appears in the code. This is not a bug, it is a property. Also for this reason, do not refer to figures/tables by using phrases like ``the table below''. Always use tabref/figref. If your terminal nodes represent object language, then these should essentially correspond to glosses, not to the original. For this reason, we recommend including an explicit example which corresponds to the tree, in this particular case \REF{sim:ex:czech-for-tree}.

\ea\label{sim:ex:czech-for-tree}\gll Co se řidič snažil dělat?\\
what {\REFL} driver try.{\PTCP.\SG.\MASC} do.{\INF}\\
\glt `What did the driver try to do?'
\z

\begin{figure}[ht]
% the [ht] option means that you prefer the placement of the figure HERE (=h) and if HERE is not possible, you prefer the TOP (=t) of a page
% \centering
    \begin{forest}
    for tree={s sep=1cm, inner sep=0, l=0}
    [CP
        [DP
            [what, roof, name=what]
        ]
        [C$'$
            [C
                [\textsc{refl}]
            ]
            [TP
                [DP
                    [driver, roof]
                ]
                [T$'$
                    [T [{[past]}]]
                    [VP
                        [V
                            [tried]
                        ]
                        [VP, s sep=2.2cm
                            [V
                                [do.\textsc{inf}]
                            ]
                            [t\textsubscript{what}, name=trace-what]
                        ]
                    ]
                ]
            ]
        ]
    ]
    \draw[->,overlay] (trace-what) to[out=south west, in=south, looseness=1.1] (what);
    % the overlay option avoids making the bounding box of the tree too large
    % the looseness option defines the looseness of the arrow (default = 1)
    \end{forest}
    \vspace{3ex} % extra vspace is added here because the arrow goes too deep to the caption; avoid such manual tweaking as much as possible; here it's necessary
    \caption{Proposed syntactic representation of \REF{sim:ex:czech-for-tree}}
    \label{sim:fig:CP}
\end{figure}

Do not use noindent after figures or tables (as you do after examples). Cases like these (where the noindent ends up missing) will be handled by the editors prior to publication.

\section{Italics, boldface, small caps, underlining, quotes}

See \citet{Nordhoff.Muller2021} for that. In short:

\begin{itemize}
    \item No boldface anywhere.
    \item No underlining anywhere (unless for very specific and well-defined technical notation; consult with editors).
    \item Small caps used for (i) introducing terms that are important for the paper (small-cap the term just ones, at a place where it is characterized/defined); (ii) metalinguistic translations of object-language expressions in semantic formulas (see \sectref{sim:sec:sem}); (iii) selected technical notions.
    \item Italics for object-language within text; exceptionally for emphasis/contrast.
    \item Single quotes: for translations/interpretations
    \item Double quotes: scare quotes; quotations of chunks of text.
\end{itemize}

\section{Cross-referencing}

Label examples, sections, tables, figures, possibly footnotes (by using the label macro). The name of the label is up to you, but it is good practice to follow this template: article-code:reference-type:unique-label. E.g. sim:ex:german would be a proper name for a reference within this paper (sim = short for the author(s); ex = example reference; german = unique name of that example).

\section{Syntactic notation}

Syntactic categories (N, D, V, etc.) are written with initial capital letters. This also holds for categories named with multiple letters, e.g. Foc, Top, Num, etc. Stick to this convention also when coming up with ad hoc categories, e.g. Cl (for clitic or classifier).

An exception from this rule are ``little'' categories, which are written with italics: \textit{v}, \textit{n}, \textit{v}P, etc.

Bar-levels must be typeset with bars/primes, not with an apostrophe. An easy way to do that is to use mathmode for the bar: C$'$, Foc$'$, etc.

Specifiers should be written this way: SpecCP, Spec\textit{v}P.

Features should be surrounded by square brackets, e.g., [past]. If you use plus and minus, be sure that these actually are plus and minus, and not e.g. a hyphen. Mathmode can help with that: [$+$sg], [$-$sg], [$\pm$sg]. See \sectref{sim:sec:hyphens-etc} for related information.

\section{Footnotes}

Absolutely avoid long footnotes. A footnote should not be longer than, say, {20\%} of the page. If you feel like you need a long footnote, make an explicit digression in the main body of the text.

Footnotes should always be placed at the end of whole sentences. Formulate the footnote in such a way that this is possible. Footnotes should always go after punctuation marks, never before. Do not place footnotes after individual words. Do not place footnotes in examples, tables, etc. If you have an urge to do that, place the footnote to the text that explains the example, table, etc.

Footnotes should always be formulated as full, self-standing sentences.

\section{Tables}

For your tables use the table plus tabularx environments. The tabularx environment lets you (and requires you in fact) to specify the width of the table and defines the X column (left-alignment) and the Y column (right-alignment). All X/Y columns will have the same width and together they will fill out the width of the rest of the table -- counting out all non-X/Y columns.

Always include a meaningful caption. The caption is designed to appear on top of the table, no matter where you place it in the code. Do not try to tweak with this. Tables are delimited with lsptoprule at the top and lspbottomrule at the bottom. The header is delimited from the rest with midrule. Vertical lines in tables are banned. An example is provided in \tabref{sim:tab:frequencies}. See \citet{Nordhoff.Muller2021} for more information. If you are typesetting a very complex table or your table is too large to fit the page, do not hesitate to ask the editors for help.

\begin{table}
\caption{Frequencies of word classes}
\label{sim:tab:frequencies}
 \begin{tabularx}{.77\textwidth}{lYYYY} %.77 indicates that the table will take up 77% of the textwidth
  \lsptoprule
            & nouns & verbs  & adjectives & adverbs\\
  \midrule
  absolute  &   12  &    34  &    23      & 13\\
  relative  &   3.1 &   8.9  &    5.7     & 3.2\\
  \lspbottomrule
 \end{tabularx}
\end{table}

\section{Figures}

Figures must have a good quality. If you use pictorial figures, consult the editors early on to see if the quality and format of your figure is sufficient. If you use simple barplots, you can use the barplot environment (defined in langsci-osl.sty). See \figref{sim:fig:barplot} for an example. The barplot environment has 5 arguments: 1. x-axis desription, 2. y-axis description, 3. width (relative to textwidth), 4. x-tick descriptions, 5. x-ticks plus y-values.

\begin{figure}
    \centering
    \barplot{Type of meal}{Times selected}{0.6}{Bread,Soup,Pizza}%
    {
    (Bread,61)
    (Soup,12)
    (Pizza,8)
    }
    \caption{A barplot example}
    \label{sim:fig:barplot}
\end{figure}

The barplot environment builds on the tikzpicture plus axis environments of the pgfplots package. It can be customized in various ways. \figref{sim:fig:complex-barplot} shows a more complex example.

\begin{figure}
  \begin{tikzpicture}
    \begin{axis}[
	xlabel={Level of \textsc{uniq/max}},  
	ylabel={Proportion of $\textsf{subj}\prec\textsf{pred}$}, 
	axis lines*=left, 
        width  = .6\textwidth,
	height = 5cm,
    	nodes near coords, 
    % 	nodes near coords style={text=black},
    	every node near coord/.append style={font=\tiny},
        nodes near coords align={vertical},
	ymin=0,
	ymax=1,
	ytick distance=.2,
	xtick=data,
	ylabel near ticks,
	x tick label style={font=\sffamily},
	ybar=5pt,
	legend pos=outer north east,
	enlarge x limits=0.3,
	symbolic x coords={+u/m, \textminus u/m},
	]
	\addplot[fill=red!30,draw=none] coordinates {
	    (+u/m,0.91)
        (\textminus u/m,0.84)
	};
	\addplot[fill=red,draw=none] coordinates {
	    (+u/m,0.80)
        (\textminus u/m,0.87)
	};
	\legend{\textsf{sg}, \textsf{pl}}
    \end{axis} 
  \end{tikzpicture} 
    \caption{Results divided by \textsc{number}}
    \label{sim:fig:complex-barplot}
\end{figure}

\section{Hyphens, dashes, minuses, math/logical operators}\label{sim:sec:hyphens-etc}

Be careful to distinguish between hyphens (-), dashes (--), and the minus sign ($-$). For in-text appositions, use only en-dashes -- as done here -- with spaces around. Do not use em-dashes (---). Using mathmode is a reliable way of getting the minus sign.

All equations (and typically also semantic formulas, see \sectref{sim:sec:sem}) should be typeset using mathmode. Notice that mathmode not only gets the math signs ``right'', but also has a dedicated spacing. For that reason, never write things like p$<$0.05, p $<$ 0.05, or p$<0.05$, but rather $p<0.05$. In case you need a two-place math or logical operator (like $\wedge$) but for some reason do not have one of the arguments represented overtly, you can use a ``dummy'' argument (curly brackets) to simulate the presence of the other one. Notice the difference between $\wedge p$ and ${}\wedge p$.

In case you need to use normal text within mathmode, use the text command. Here is an example: $\text{frequency}=.8$. This way, you get the math spacing right.

\section{Abbreviations}

The final abbreviations section should include all glosses. It should not include other ad hoc abbreviations (those should be defined upon first use) and also not standard abbreviations like NP, VP, etc.


\section{Bibliography}

Place your bibliography into localbibliography.bib. Important: Only place there the entries which you actually cite! You can make use of our OSL bibliography, which we keep clean and tidy and update it after the publication of each new volume. Contact the editors of your volume if you do not have the bib file yet. If you find the entry you need, just copy-paste it in your localbibliography.bib. The bibliography also shows many good examples of what a good bibliographic entry should look like.

See \citet{Nordhoff.Muller2021} for general information on bibliography. Some important things to keep in mind:

\begin{itemize}
    \item Journals should be cited as they are officially called (notice the difference between and, \&, capitalization, etc.).
    \item Journal publications should always include the volume number, the issue number (field ``number''), and DOI or stable URL (see below on that).
    \item Papers in collections or proceedings must include the editors of the volume (field ``editor''), the place of publication (field ``address'') and publisher.
    \item The proceedings number is part of the title of the proceedings. Do not place it into the ``volume'' field. The ``volume'' field with book/proceedings publications is reserved for the volume of that single book (e.g. NELS 40 proceedings might have vol. 1 and vol. 2).
    \item Avoid citing manuscripts as much as possible. If you need to cite them, try to provide a stable URL.
    \item Avoid citing presentations or talks. If you absolutely must cite them, be careful not to refer the reader to them by using ``see...''. The reader can't see them.
    \item If you cite a manuscript, presentation, or some other hard-to-define source, use the either the ``misc'' or ``unpublished'' entry type. The former is appropriate if the text cited corresponds to a book (the title will be printed in italics); the latter is appropriate if the text cited corresponds to an article or presentation (the title will be printed normally). Within both entries, use the ``howpublished'' field for any relevant information (such as ``Manuscript, University of \dots''). And the ``url'' field for the URL.
\end{itemize}

We require the authors to provide DOIs or URLs wherever possible, though not without limitations. The following rules apply:

\begin{itemize}
    \item If the publication has a DOI, use that. Use the ``doi'' field and write just the DOI, not the whole URL.
    \item If the publication has no DOI, but it has a stable URL (as e.g. JSTOR, but possibly also lingbuzz), use that. Place it in the ``url'' field, using the full address (https: etc.).
    \item Never use DOI and URL at the same time.
    \item If the official publication has no official DOI or stable URL (related to the official publication), do not replace these with other links. Do not refer to published works with lingbuzz links, for instance, as these typically lead to the unpublished (preprint) version. (There are exceptions where lingbuzz or semanticsarchive are the official publication venue, in which case these can of course be used.) Never use URLs leading to personal websites.
    \item If a paper has no DOI/URL, but the book does, do not use the book URL. Just use nothing.
\end{itemize}


\section{Introduction}
Plural pronoun comitatives (often also dubbed ``Plural pronoun constructions'' in the literature; henceforth PPCs) are complex expressions that consist of a plural pronoun and a comitative (i.e. \textit{with-}) phrase. In many Slavic languages, PPCs can give rise to two different interpretations, paraphrased as (a) and (b) in the examples below. 

\ea {
\gll My s Petej pojdëm domoj. \\
we with Petja.\textsc{inst} go.\textsc{fut.1pl} home \\
\glt \hfill (Russian; \cite{VassilievaLarson2005}: 101)}\label{ini:russian}
\ea `We will go home with Petja.'
\ex `I and Petja will go home.'
\z \z
\ea {
\gll \textit{pro} S Mariju smo otišli u muzej. \\
%\textit{pro}
{} with Maria.\textsc{inst} \textsc{aux.1pl} went in museum \\
\hfill (Torlakian BCMS)}\label{ini:torlakian}
\ea `We went to the museum with Maria.'
\ex `Maria and I went to the museum.'
\z \z 

\noindent The availability of an apparent singular (`I') interpretation of the (dropped) pronoun under reading (b) is unexpected given that its surface form is plural.\footnote{To talk about the pronoun's ``surface form'' is actually a bit misleading here. Example (\ref{ini:torlakian}) from Torlakian BCMS involves \textit{pro}-drop. So, strictly speaking, there is no such thing as a surface form of the plural pronoun in this sentence. Although this example is also felicitous in its version with an overt plural pronoun \textit{mi} `we', I omit it here and in the succeeding examples for reasons that will be discussed in %Section 
\sectref{sec:conclusion}. Furthermore, we can infer the ``underlying'' form, or rather the features of \textit{pro} from the features on the verb -- which is 1st person plural in both readings of (\ref{ini:torlakian}).} This reading (henceforth: iPPC; following \citeauthor{Feldman2003}'s \citeyear{Feldman2003} %\cite{Feldman2003}'s 
terminological distinction) contrasts with the default `we' interpretation (henceforth: ePPC) of the plural pronoun under the reading paraphrased in (a).\footnote{The distinction between ``iPPC'' and ``ePPC'' as made in \citet{Feldman2003} refers to an ``inclusive'' and an ``exclusive'' interpretation of the plural pronoun, respectively. That is, under an ``inclusive'' interpretation of a PPC, the referent from the comitative phrase (e.g. \textit{Petja} in (\ref{ini:russian})) is apparently included in the overall reference of the plural pronoun; whereas under an ``exclusive'' interpretation, the reference of the plural pronoun does not include the referent from the comitative phrase -- i.e., under the ePPC reading of (\ref{ini:russian}), the 1st person plural pronoun refers to the speaker and someone else (but Petja).} In the light of the ambiguity between (a) and (b), it may seem tempting to assume that plural pronouns are ambiguous between a singular and a plural interpretation in general. As \citet{VassilievaLarson2001} pointed out already, however, this cannot be the case. Constructions like (\ref{ini:amb}) not involving a comitative phrase do not give rise to the ambiguity observed for the otherwise parallel example (\ref{ini:russian}).\footnote{The original transliteration of example \REF{ini:amb} was altered to be in line with the scientific transliteration of Cyrillic.} 

\ea {
\gll My pojdëm domoj. \\
 we go.\textsc{fut.1pl} home \\
 \hfill (Russian; \cite{VassilievaLarson2001}: 449)} \label{ini:amb}
\ea `We will go home.'
\ex *`I will go home.'
\z \z

\noindent Common analyses have usually taken one of two explanations: either that the reference of a plural pronoun in a PPC is composed of its singular counterpart and the referent of the comitative (cf. \cite{VassilievaLarson2001}, \cite{VassilievaLarson2005}), or that PPCs involve asymmetric coordination from a syntactic perspective, but have the same (or similar) semantics as symmetric coordination (cf. \cite{Dyla1988}, \cite{McNally1993}, \cite{FeldmanDyla2008}). I will spell out the core assumptions of these and other approaches in \sectref{sec:prev}. 


A real challenge for pretty much any theory of PPCs (apart from apposition-based ones such as \cite{Cable2017}, perhaps) is that the plural pronoun and the comitative phrase can occur as a discontinuous constituent. And yet, they give rise to iPPC interpretations (at least in most Slavic languages). Consider the contrast between (\ref{disc1}) and (\ref{disc2}) from Torlakian, which is mainly syntactic but not semantic. 

\ea {
\gll \textit{pro} S Mariju smo otišli u muzej. \\
%\textit{pro}
{} with Maria.\textsc{inst} \textsc{aux.1pl} went in museum \\
 }\hfill (Torlakian BCMS) \label{disc1}
\ea `We went to the museum with Maria.' \hfill ePPC
\ex `Maria and I went to the museum.' \hfill iPPC 
\z \z
\ea {
\gll \textit{pro} Otišli smo u muzej s Mariju. \\
%\textit{pro}
{} went \textsc{aux.1pl} in museum with Maria.\textsc{inst} \\
 } \hfill (Torlakian BCMS)\label{disc2}
\ea `We went to the museum with Maria.' \hfill ePPC 
\ex `Maria and I went to the museum.' \hfill iPPC
\z \z

\noindent To the best of my knowledge, no analysis has yet come up with a neat answer to the question of how an iPPC reading can arise for such ``split'' PPCs like (\ref{disc2}). My attempt to explain this fact will be outlined in \sectref{sec:qfloat}. In particular, I propose a novel analysis of PPCs in Slavic languages that is based on the assumption that plural pronouns and (universal) quantifiers behave alike in various respects. In particular, I argue that the difference between an iPPC  and an ePPC interpretation can be derived in terms of whether the comitative phrase resides inside the restrictor of the plural pronoun (which is similar to a quantifier's restrictor) or not. My analysis is essentially based on data from Torlakian BCMS and from Bulgarian. These two languages are very suitable as a starting point, as they have different properties regarding available readings for split PPCs -- and from these differences, I derive one of the core assumptions of my analysis.\footnote{However, I also occasionally show how (and to what extent) the generalizations derived from those distinct features are applicable to other Slavic languages such as Russian or Polish.} Moreover, to the best of my knowledge, nothing has been said in the literature yet about the behavior of PPCs in Torlakian BCMS and Bulgarian. 


This article proceeds as follows. In \sectref{sec:prev} I briefly summarize the main findings and claims of previous approaches to PPCs, and point out a few problems that arise from each. I present my proposal in \sectref{sec:proposal}. First, in \sectref{sec:general}, I present my analysis of the internal structure of plural pronouns in general. I claim that plural pronouns, just as quantifiers, select a restrictor argument and that it is the precise internal structure of this restrictor which determines the overall referential properties of the respective plural pronoun. In \sectref{sec:ppc} I then show the implications of those assumptions for an analysis of PPCs. Specifically, I argue that the comitative phrase occurs inside of the plural pronoun's restrictor in iPPCs, but outside of it (that is, elsewhere in the syntactic structure) in ePPCs. I outline my explanation of why split PPCs can have iPPC interpretations in some Slavic languages in \sectref{sec:qfloat}. Furthermore, I also offer an explanation as to why we find iPPC readings for split PPCs in precisely those Slavic languages in which we find them (such as Torlakian BCMS), but not in others (such as Bulgarian). In \sectref{sec:sc} and \sectref{sec:bind} I present data in favour of my analysis related to Subject Control constructions and binding. However, I want to point out in advance that the data from those two sections could probably also be correctly derived under other approaches to PPCs. Nevertheless, it is my intention to discuss them rather than to leave them as implicit evidence. So in the end, the virtue of my analysis is that it can properly predict when, how, and why iPPC readings for split PPCs arise -- and thus, that it fills an explanatory gap that exists among previous analyses of PPCs. 




\section{Previous analyses of PPCs in Slavic}\label{sec:prev}
Existing approaches to PPCs can basically be divided into three categories: those assuming that PPCs have an underlying coordinative structure; those which treat PPCs as a kind of appositive construal involving ellipsis; and finally, those which are based on the idea that the comitative phrase acts as a complement of the plural pronoun in such constructions. In this section, I briefly illustrate the main claims of previous analyses as well as some issues that remain open in the light of the respective theories. My own analysis draws on the assumptions of analyses of the third category, i.e. on approaches which treat the comitative phrase as a complement of the plural pronoun. However, my proposal takes one step further in observing and implementing some parallels between plural pronouns and quantifiers. 


\subsection{Based on coordination}\label{sec:coord}
As has been observed in the literature on (primarily) Russian and Polish, ``ordinary'' or ``regular'' comitative constructions (i.e. those which are not headed by a plural pronoun) behave differently with regard to whether they trigger singular or plural agreement on the verb; see example (\ref{agree:sgpl}). Traditional analyses (cf. \cite{Dyla1988}, \cite{McNally1993}, \cite{FeldmanDyla2008}) anchor the alternation of verbal number agreement in diverging underlying syntactic structures. Specifically, those comitatives that trigger plural agreement on the verb are considered to be coordinative (that is, conjunctive) construals, whereas comitatives that trigger singular agreement on the verb are commonly treated as adjuncts to VP.\footnote{Transliterations were adjusted here.} 

\ea \label{agree:sgpl} \ea
\gll Maša s Dašej xodjat v školu. \\
Maša.\textsc{nom} with Daša.\textsc{inst} go.\textsc{1pl} to school \\
 \glt `Maša and Daša go to school.' \hfill \textit{s}-\textsc{conjunction}
\ex
\gll Anja s Vanej pošla v biblioteku. \\
Anja.\textsc{nom} with Vanja.\textsc{inst} went.\textsc{sg.f} to library \\
 \glt `Anja went to the library with Vanja.' \hfill \textit{s}-\textsc{adjunction}
\z
\hfill (Russian; \cite{Feldman2003})
\z

\noindent I do not want to enter this debate to any extent and I will have little to say about such ``regular'' comitatives in this article. My account is based on the specific properties of plural pronouns and does not intend to make generalizations to any other kinds of DPs. Nonetheless, both structural options from (\ref{agree:sgpl}) have been considered in the light of PPCs -- although coordination-based accounts (such as the influential paper by \cite{McNally1993}) are more numerous than analyses that assume that PPCs have an adjunction structure (see \cite{Ladusaw1989}). 


A recent example of a coordinative treatment of PPCs is \citeposst{Sokolova2019} analysis. It is claimed there that Slovak PPCs classify as ``coordinate comitatives'', i.e. the plural pronoun DP and the comitative phrase are assumed to have the same structural rank. Under \citeposst{Sokolova2019} approach, the denotation of a plural pronoun such as \textit{my} `we' in (\ref{soko19}) consists of two referents (or ``participants'', as dubbed in the original article) -- namely the speaker ($=$ first participant), and a second participant that gets lexically specified by the referent of the comitative.

\ea
\gll My s Evou chodíme do rovnakej školy. \\
we with Eva.\textsc{inst} go.\textsc{1pl} to same school \\
 \glt `Eva and I go to the same school.' \hfill (Slovak; \cite{Sokolova2019}: 101) \label{soko19}
\z

\noindent The iPPC interpretation of a sentence like (\ref{soko19}) is assumed to arise on the basis of an ``absorption'' mechanism. The apparent double occurrence of the second participant conflates into one via the overall reference of the plural pronoun. \citet{Sokolova2019} is not explicit about how this ``absorption of a referent'' (a notion based on \cite{Daniel2000}) is supposed to work out on a derivational level, however. Likewise, \citeposst{Daniel2000} account is not concerned with a precise formal underpinning. The original idea, though, is that plural pronouns are only explicit about the first referent in general (such as ``speaker'' or ``addressee'' in the case of `we' and `you\textsubscript{PL}', respectively). Under some discourse conditions, an explication of the second participant, which otherwise remains implicit, is required. This explication can be resolved via different means in (morpho-)syntax; and so in the end, PPCs are just one (morpho-)syntactic conventionalization to feed these pragmatic needs for specification of the other referent of the plural pronoun (Mikhail Daniel, p.c.).\footnote{Note that an explanation along these lines, once incorporated into a formal discourse framework, would probably suffice the purpose of predicting when 2p-interpretations arise -- namely if and only if the identity of the second participant otherwise remains unclear. Evidence in favor of this view comes from infelicitous sequences such as (\ref{bg:disni}). 
\ea
\gll Nie otidohme v muzeja. \#Večerta nie s Peter gledahme star Disni film. \\
 we went.\textsc{1pl} in museum evening we with Peter watched.\textsc{1pl} old Disney movie \\
 \glt `We$_i$ went to the museum. In the evening, \#[I and Peter]$_i$ watched an old Disney movie.' \\
 \hfill (Bulgarian)\label{bg:disni}
\z
}


Analyses such as \citeposst{McNally1993}, on the other hand, assume an asymmetric kind of coordination. \citet{McNally1993} does not deal with PPCs in particular, but suggests that PPCs such as (\ref{nally93com}) and ``regular'' comitative constructions like (\ref{nally93reg}) share the same underlying structure illustrated in Figure \ref{nally93struc}.

\ea \ea {
\gll Oni s Petej pridut. \\
 they.\textsc{nom} with Petja.\textsc{inst} come.\textsc{3pl} \\
 \glt `He and Petja are coming. \hfill (Russian; \cite{McNally1993}: 359) }\label{nally93com}
\ex {
\gll Anna s Petej napisali pis'mo. \\
 Anna.\textsc{nom} with Petja.\textsc{inst} wrote.\textsc{3pl} letter \\
 \glt `Anna and Petja wrote a letter.' \hfill (Russian; \cite{McNally1993}: 347)} \label{nally93reg}
\z \z

%\ex (\cite{McNally1993}: 359) \label{nally93struc}
\begin{figure}[ht]
% the [ht] option means that you prefer the placement of the figure HERE (=h) and if HERE is not possible, you prefer the TOP (=t) of a page
% \centering
    \begin{forest}
    for tree={s sep=1cm, inner sep=0, l=0}
    [ NP
    [NP] [PP]
    ]
\end{forest}
\caption{Asymmetric coordination structure (\cite{McNally1993}: 359)}
    \label{nally93struc}
    \end{figure}

\citet{McNally1993} claims that from a semantic point of view, the structure in Figure \ref{nally93struc} has the same interpretation as a structure involving symmetric coordination.


Several problems arise from a coordination-based approach to PPCs. A salient challenge has to do with the fact that the plural pronoun can be dropped in a PPC (in Slavic languages that permit \textit{pro}-drop, such as Torlakian BCMS, Bulgarian, Polish, or Slovak), but not in other (regular) forms of conjunction. But if we assume that a PPC is (structurally speaking) a conjunction of the plural pronoun and the comitative phrase, then we would expect that PPCs and regular coordination show the same syntactic pattern -- contrary to the actual facts. Dropping the first conjunct (even if pronominal) in regular coordination is impossible, compare (\ref{drop:ppc}) and (\ref{drop:coord}) from Polish.

\ea \ea {
\gll On/\textit{pro} z bratem poszli do kina. \\
he with brother went to cinema \\
 \glt `He and his brother went to the cinema.' \hfill (Polish; \cite{Trawinski2005}: 385) }\label{drop:ppc}
\ex {
\gll On/*\textit{pro} i Maria poszli do kina. \\
he and Maria went to cinema \\
 \glt (Intended): `He and Maria went to the cinema.' \\
\hfill (Polish; \cite{Trawinski2005}: 384)} \label{drop:coord}
\z \z

\noindent Moreover, it remains puzzling why the order of plural pronoun and the other DP involved is fixed in PPCs; i.e. why a construction like (\ref{rev:ppc}) is ungrammatical. Because, as can be seen from (\ref{rev:coord}), such a reversed order is (although slightly deviant) sometimes possible in coordinated structures. 

\ea \ea[*] {
\gll Petja s nami tancevali. \\
 Petja.\textsc{nom} with us.\textsc{inst} danced.\textsc{pl} \\
 \glt Intended: `Petja and us danced.' \label{rev:ppc}}
\ex[?] {
\gll Petja i ty tancevali. \\
 Petja.\textsc{nom} and you.\textsc{nom} danced.\textsc{pl} \\
 \glt `Petja and you danced.' \hfill (Russian; \cite{VassilievaLarson2005}: 114)
\label{rev:coord}}
\z
\z

\noindent Under the assumption that PPCs have an underlying structure like the one in Figure \ref{nally93struc}, we could potentially explain these differences to regular conjunction structures in terms of asymmetric coordination. But to the best of my knowledge, no such account has been presented yet, and thus, the exact reasons for those differences remain to be spelled out in detail. 


\subsection{Based on apposition}\label{sec:app}
This line of approach originates from \citet{Cable2017}. The basic claim of \citeposst{Cable2017} analysis is that PPCs are appositive constructions involving ellipsis. In particular, he assumes that PPCs in their iPPC versions contain an elided instance of the respective plural pronoun's singular counterpart, such that a sentence like (\ref{appos1}) has the underlying structure in (\ref{appos2}).

\ea \ea
\gll My s Petej pojdëm domoj. \\
 we.\textsc{nom} with Petja.\textsc{inst} go.\textsc{fut.1pl} home \\
 \glt `Petja and I will go home.' \hfill (Russian; \cite{VassilievaLarson2005}: 101) \label{appos1}
\ex
\gll My \minsp{[$\langle$} ja$\rangle$ s Petej] pojdëm domoj. \\
 we.\textsc{nom} {} I.\textsc{nom} with Petja.\textsc{inst} go.\textsc{fut.1pl} home \\
 \glt \textit{Approx.}: `We, I and Petja, will go home.' \hfill (Russian; \cite{Cable2017}: 8) \label{appos2}
\z \z

\noindent A plural pronoun in a PPC is taken to denote the sum of the two referential expressions occurring in the appositive, i.e. the sum of \textit{ja} (`I') and \textit{Petja} in (\ref{appos2}).


While such an account seems appealing, it is not without complications. First of all, it raises the question of what licenses ellipsis in the apposition. It cannot be deletion under identity in a strict sense (see \cite{Liptak2015} for an overview, and the references therein), because according to \citeposst{Cable2017} proposal, the plural pronoun denotes the sum of the elided element and the comitative referent. Hence, if anything, deletion should target both \textit{ja} and \textit{Petej} in (\ref{appos2}); and then, there would be no overt material left in the appositive apart from the comitative element \textit{s} itself. At best, we could assume that only \textit{ja} gets elided because ellipsis would have to target a discontinuous constituent otherwise, or because it bears the same grammatical case as the plural pronoun (whereas \textit{Petej} bears instrumental case, assigned by the comitative element). It is hard to figure out which deletion mechanisms are supposed to be at stake in a structure like (\ref{appos2}) -- especially as long as the assumptions concerning the appositive structure are not embedded in any tradition of analyzing appositions. However, \citet{Cable2017} seems to be aware of this problem and mentions that it could also be the case that the comitative element itself has a meaning akin to appositive structure. How this meaning would need to be defined in particular is left as an open issue. 

Furthermore, it seems that phonological deletion poses a problem to any ap\-po\-si\-tion-based approach to (Slavic, at least) PPCs in general. Because as varied as theories of appositives may be, there is a syntactic property which appositions are uncontroversially assumed to have. Namely, that there is an anchor expression to which the appositive attaches. The anchor expression of the apposition $\langle$\textit{ja}$\rangle$ \textit{s Petej} from (\ref{appos2}) would (also quite undisputedly) have to be \textit{my}, i.e. the plural pronoun. While this is unproblematic for the Russian case at hand, we run into complications as soon as we want to apply this analysis to Slavic languages that allow \textit{pro}-drop structures. Take example (\ref{noapp}) from Torlakian BCMS, for instance. The version in (\ref{noapp1}) with the overt plural pronoun could be treated along the lines of (\ref{appos2}), as shown in (\ref{noapp2}). But for its \textit{pro}-dropped version in (\ref{noapp3}), this does not work out. Because we would have to assume that appositions can attach to silent anchor expressions, as pictured in (\ref{noapp4}).\footnote{The hashtag \# in (\ref{noapp4}) is intended to indicate that it should be regarded with suspicion whether such a syntactic configuration is even possible.}

\ea \label{noapp} 
\ea\label{noapp1} 

\gll Juče smo mi s Mariju otišli u bioskop. \\
yesterday \textsc{aux.1pl} we.\textsc{nom} with Maria.\textsc{inst} went in cinema \\
\ex\label{noapp2}

\gll Juče smo mi \minsp{[$\langle$} ja$\rangle$ s Mariju] otišli u bioskop. \\
 yesterday \textsc{aux.1pl} we.\textsc{nom} {} I.\textsc{nom} with Maria.\textsc{inst} went in cinema \\
\z
\textit{Approx.}: `Yesterday we, I and Maria, went to the cinema.'  \\
 \hfill (Torlakian BCMS)
\z

\ea 
\ea \label{noapp3}

\gll Juče smo \textit{pro} s Mariju otišli u bioskop. \\
 yesterday \textsc{aux.1pl} %\textit{pro}
{} with Maria.\textsc{inst} went in cinema \\
\ex[\#] {
\gll Juče smo \textit{pro} \minsp{[$\langle$} ja$\rangle$ s Mariju] otišli u bioskop. \\
 yesterday \textsc{aux.1pl} %\textit{pro}
{} {} I.\textsc{nom} with Maria.\textsc{inst} went in cinema \\
 }\label{noapp4}
\z \textit{Intended:} `Yesterday we, I and Maria, went to the cinema.' \\
\hfill (Torlakian BCMS)
\z

\noindent If it were indeed possible to adjoin an apposition or any kind of additional syntactic material (such as a PP, for instance) to an attachment site that is not overtly present in the syntactic structure we would expect to find constructions such as (\ref{pro:appos}) or (\ref{pro:adj}) regularly. But as we can see, this expectation is not met. 
\ea 
\ea {
\gll Mi, \minsp{(} naime) ja s Petra, idemo sutra u bioskop.\\
we.\textsc{nom} {} namely I.\textsc{nom} with Peter.\textsc{inst} go.\textsc{1pl} tomorrow in cinema\\
\glt `We, namely I and Peter, will go to the cinema tomorrow.'} \label{pro:app}
\ex[\#] {
\gll \textit{pro} Naime ja s Petra, idemo sutra u bioskop. \\
 %\textit{pro}
{} namely I.\textsc{nom} with Peter.\textsc{inst} go.\textsc{1pl} tomorrow in cinema \\
 \textit{Intended:} `We, namely I and Peter, will go to the cinema tomorrow.'} \label{pro:appos}
\ex {
\gll Ona s plavom kosom ide sutra u bioskop. \\
 she with blond hair go.\textsc{3sg} tomorrow in cinema \\
 \glt `She with the blond hair will go to the cinema tomorrow.'}
\ex[\#] {
\gll \textit{pro} S plavom kosom ide sutra u bioskop. \\
 %\textit{pro}
{} with blond hair go.\textsc{3sg} tomorrow in cinema \\
 \glt %\#
`She will go to the cinema tomorrow with the blond hair.' \\
 Intended: `She with the blond hair will go to the cinema tomorrow.'} \label{pro:adj}
\z \hfill (Torlakian BCMS)\z

\noindent The sentence in (\ref{pro:appos}) is ungrammatical in an out-of-the-blue context as well as in any syntactic context that does not contain a salient anchor or anaphoric expression. More precisely, the only syntactic environment which could save (\ref{pro:appos}) is a preceding sentence like \textit{We/The best of friends will go to the cinema -- naime ja s Petra...}. Example (\ref{pro:adj}) on the other hand is simply infelicitous. It cannot mean what it is supposed to, i.e. that he/she and the blond-haired he/she will go to the cinema tomorrow. Instead, the sentence only has a rather awkward interpretation according to which she (an individual salient from the discourse) will go to the cinema tomorrow taking the blond hair (literally) with her. 


Abstracting from (im)possible anchor expressions, an apposition-based analysis makes unwelcome predictions regarding the spell-out of an iPPC. Appositives are usually articulated with an intonational/phonological break or boundary. Thus, if for instance (\ref{appos1}) has the underlying structure in (\ref{appos2}), such a break or boundary should reflect in the pronunciation of such a sentence; see (\ref{spelloutapp}), where ``(--)'' signals the points where the breaks should occur. 

\ea \label{spelloutapp}

\gll My (--) s Petej (--) pojdem domoj.\\
we.\textsc{nom} {} with Petja.\textsc{inst} {} go.\textsc{fut.1pl} home\\
\glt `We, with Peter, will go home.'
\z

\noindent This prediction is not borne out, since a PPC (no matter which reading is intended) does not surface any kind of apposition-typical intonational breaks in an unmarked context. It rather has a regular pronunciation. 



\subsection{Comitatives as pronoun complements}\label{sec:proncomp}
The analyses in \citet{VassilievaLarson2001} and \citet{VassilievaLarson2005} treat (Russian) plural pronouns as incomplete expressions, comprising a singular nucleus and an unsaturated element $\triangle$ in their meaning. These two components are elements of an ordered pair $\langle X, Y\rangle$. $X$ is obligatorily taken by the plural pronoun's singular counterpart (in accordance with its person feature), and $Y$ gets (per default) saturated by some $\sigma$ from the context. The resulting (distributive) semantics of Russian \textit{my} `we', \textit{vy} `you.\textsc{pl}' and \textit{oni} `they' can be represented as in (\ref{distr}) below.\footnote{The formalism that \citet{VassilievaLarson2001} and \citet{VassilievaLarson2005} make use of was adopted from \citet{LarsonSegal1995}, where it is assumed that the truth value assigned to a sentence is dependent on context sequences $\sigma$ -- such that $\sigma(a), \sigma(b), \sigma(c)$ and $\sigma(d)$ are associated with speaker, addressee, speaker time and speaker location, respectively, from the context. Indexical pronouns are claimed to get their values via these means. Other instances of such a sequence, dubbed $\sigma(n)$, determine the reference of non-indexical pronouns on the other hand. Since my analysis does not make use of this particular formal framework, I will not go into this topic any further here.}

\ea \label{distr} \ea $\cnst{Val}(\langle X, Y\rangle$, [\textsubscript{D} \textit{my}], $\sigma$) iff $\mid$(\{$\sigma(a)$\} $\cup$ $Y$) $- X\mid = 0$ \\
 `(all of) speaker $+$ others $Y$'
\ex $\cnst{Val}(\langle X, Y\rangle$, [\textsubscript{D} \textit{vy}], $\sigma$) iff $\mid$(\{$\sigma(b)$\} $\cup$ $Y$) $- X\mid = 0$ \\
 `(all of) addressee $+$ others $Y$'
\ex $\cnst{Val}(\langle X, Y\rangle$, [\textsubscript{D} \textit{oni}], $\sigma$) iff $\mid$(\{$\sigma(i)$\} $\cup$ $Y$) $- X\mid = 0$ \\
 `(all of) he/she/it $+$ others $Y$'
\z
\hfill (\cite{VassilievaLarson2005}: 119)
\z

\noindent With regard to PPCs and their iPPC interpretations, the idea outlined in \citet{VassilievaLarson2005} is that the comitative phrase occupies a complement position to the plural pronoun. The $Y$-slot in the meaning of the plural pronoun then gets filled by the referent of the comitative, i.e. as a matter of syntactic means. The underlying structure of the iPPC reading is given in Figure \ref{struc:vaslars}, and its semantics in (\ref{sem:vaslars}). 

\begin{figure}
% the [ht] option means that you prefer the placement of the figure HERE (=h) and if HERE is not possible, you prefer the TOP (=t) of a page
% \centering
    \begin{forest}
    for tree={s sep=1cm, inner sep=0, l=0}
    [DP 
        [D [ \textit{my} `we' \\
 I $+ \triangle$, name=triangle] ]
        [PP [ \textit{s} `with' ]
        [ Petej, name=Petej ]
        ]
    ]
\draw[->,overlay] (Petej) to[out=south west, in=south, looseness=1.1] (triangle);
\end{forest} 
\vspace{5ex}
    \caption{iPPC structure (\cite{VassilievaLarson2005}: 120)}
    \label{struc:vaslars}
\end{figure} 

\ea $\cnst{Val}(\langle X,$ \textit{my s Petej}, $\sigma\rangle$ iff $\mid$(\{$\sigma(a)$\} $\cup$ \{Petja\}) $- X\mid = 0$; i.e. \\
 $\cnst{Val}(\langle X,$ \textit{my s Petej}, $\sigma\rangle$ iff $\mid$(\{$\sigma(a)$, Petja\}) $- X\mid = 0$ \\
 `(all of) speaker $+$ Petja' \label{sem:vaslars}
\hfill (\cite{VassilievaLarson2005}: 120)
\z

\noindent Hence, \citeauthor{VassilievaLarson2005} tie the availability of an iPPC interpretation to the specific syntactic configuration depicted in Figure \ref{struc:vaslars}. However, there are two pieces of data that they explicitly leave unaccounted for. Firstly, it remains unclear under this analysis why PPCs (in their iPPC meaning) can occur split, i.e. why the plural pronoun and the comitative phrase can be discontinuous and still an iPPC interpretation is available; see (\ref{split:vaslars}). Secondly, it is puzzling why \textit{wh}-questions such as (\ref{q:vaslars}) do actually not give rise to an iPPC meaning. 
\ea \ea
\gll My pojdëm zavtra s Ivanom v magazin i vsë kupim. \\
 we.\textsc{nom} go.\textsc{fut.1pl} tomorrow with Ivan.\textsc{inst} to store and all buy.\textsc{fut} \\
 \glt `Tomorrow, we will go to the store with Ivan and buy all we need.' \\
 \hfill ePPC \\
 `Tomorrow, Ivan and I will go to the store and buy all we need.' \hfill iPPC \label{split:vaslars}
\ex
\gll S kem my xodili v magazin? \\
 with whom we went to store \\
 \glt `With whom did we go to the store?' \hfill ePPC \\
 \textit{Unavailable iPPC reading}: `I and who went to the store?' %\hfill {\bf iPPC}
\label{q:vaslars}
\z
\hfill (Russian; \cite{VassilievaLarson2005}: 122)
\z

\noindent If the comitative phrase acts as a complement of the plural pronoun, we do not expect (\ref{split:vaslars}) to give rise to an iPPC reading, because the plural pronoun and the comitative phrase do not form a constituent in this sentence -- yet, an iPPC interpretation is available. This poses a serious challenge to \citeposst{VassilievaLarson2005} analysis since the availability of an iPPC reading for split PPCs is %cross-linguistically 
widespread across Slavic languages (with some exceptions). My analysis, to which we turn next, offers an explanation of these facts as well as a novel perspective on PPCs in general. 




\section{Proposal}\label{sec:proposal}
In this section, I present my approach to PPCs in Slavic. My analysis is based on observations regarding similarities in the semantic and syntactic behaviour of plural pronouns and quantifiers; or rather, of plural pronominal DPs and quantificational (noun) phrases (henceforth: QPs).\footnote{It is important to note here that I do not claim that plural pronouns are quantificational elements (in a broad sense) by definition. While these two types of expressions exhibit some striking parallels, my analysis is not intended to treat them as completely analogous. That is, just because a (universal) quantifier has such and such properties or shows such and such behaviour (of a syntactic or semantic nature), this does not automatically also have to apply to plural pronouns. For the time being, I restrict the similarities between plural pronouns and quantifiers to what is explicitly diagnosed in the course of my analysis outlined in this article.} Therefore I first outline my claims regarding the internal structure of plural pronouns in \sectref{sec:general} and show how it is similar to the internal structure of QPs. On this basis, I derive the two readings of PPCs (i.e. iPPC and ePPC) in \sectref{sec:ppc} and then illustrate in \sectref{sec:qfloat} why and how split PPCs give rise to iPPC interpretations in some Slavic languages, but not in others. Data related to Subject Control structures and binding are presented in %Sections 
\sectref{sec:sc} and \sectref{sec:bind}, respectively. 


\subsection{Restrictor sets for plural pronouns}\label{sec:general}
Plural pronouns have often been treated analogously to definite descriptions in that both denote pluralities of individuals (cf. \cite{but:Link1983}, \cite{Nunberg1993}, \cite{but:Elbourne2008}, \cite{Buring2011}, among many others). If we leave figurative uses aside, then this plurality obligatorily includes the speaker (1st person plural pronoun) or addressee (2nd person plural pronoun) of the utterance. In my analysis, I follow \citet{but:Link1983} and others in assuming that plural pronouns denote pluralities of individuals. But the suggested way in which this plurality gets composed differs from previous approaches. I propose a slightly modified variant of the ``plural pronouns as definite descriptions'' view. In particular, I claim that plural pronouns are more similar to quantifiers after all. What I am arguing for in this and subsequent sections is that plural pronouns and (universal) quantifiers have some striking properties in common -- syntactic, semantic, and pragmatic ones.


Let us start with a very basic yet deep parallel regarding the syntactic arguments these expressions take. Quantifiers are uncontroversially assumed to take two arguments: restrictor and scope. The former typically corresponds to the NP the quantifier forms a QP with. The latter typically corresponds to the predicate (i.e. the VP, roughly speaking). While it is probably also quite uncontroversial that a plural pronoun (in subject position) combines with a predicate (a scope argument so to speak), I claim here that the internal syntactic structure and semantic composition of such an expression also involves a restrictor argument. More specifically, the restrictor of a plural pronoun is argued to be the decisive factor in determining the plural pronoun's overall reference. So, what must the restrictor of a plural pronoun look like by analogy with the restrictor of a quantificational expression? And what semantic or pragmatic properties can we find?


There has been a vast debate in the literature on quantifiers concerning the question whether quantificational determiners presuppose that their restrictor sets must not be empty (see \cite{HeimKratzer1998}, and \cite{Szabolcsi2010} for an overview). By now, it is more or less commonly agreed that at least strong quantifiers such as \textit{all, most}, or \textit{each} presuppose non-emptiness of their restrictor's denotation. So if, as claimed here, requiring a restrictor argument is one of the properties that plural pronouns share with quantifiers, we would probably expect plural pronouns to trigger an equivalent presupposition regarding the denotation of their restrictor sets. While there seems to be nothing wrong in principle at first glance with the assumption that a plural pronoun like \textit{we} presupposes that what it ranges over must not be the empty set, such a bare existential presupposition does not suffice under closer inspection. The reason is that non-emptiness alone does not account for referential properties related to a plural pronoun's person feature. Specifically, a felicitous use of \textit{we} not only requires that the pronoun ranges over a plurality made up of whatever individuals, but rather over a plurality which obligatorily contains the speaker of the utterance. I will first and foremost tie these requirements directly to the lexico-semantic properties of plural pronouns.\footnote{So far, nothing hinges on this decision. I will not discuss any data that is concerned with presupposition projection or presupposition failure. Note, though, that presuppositional content of \textit{we} that matches the conditions stated above has occasionally been suggested in the literature (see \cite{Stokke2022} and the references therein).}


The lexical entries of \textit{we} and \textit{you}\textsubscript{PL} in (\ref{we:lex}) and those expressions' underlying structures in Figures \ref{fig:butsch:we:struc}--\ref{fig:butsch:label3}  further illustrate the assumed parallel to a quantifier's restrictor.\footnote{I concentrate on 1st and 2nd person plural pronouns only throughout this article. The reason for this restriction is twofold. On the one hand, 3rd person plural pronouns do not have a fixed first referent like 1st and 2nd person plural pronouns do (i.e. speaker or addressee, respectively). On the other hand, iPPC interpretations are often harder to obtain for PPCs in the Slavic languages I investigate, if they are available at all. And this, in turn, might be related to the lack of a fixed first referent. In particular, I found that iPPC readings are often judged infelicitous in out-of-the-blue contexts -- and only in suitable contexts or in follow-up sentences to an explicit QUD highlighting which individual is intended to be the first referent, iPPC interpretations seemed more readily available.}

\ea \label{we:lex} \ea \sib{\textit{we}}, \sib{\textit{you}\textsubscript{PL}} $= \lambda P. \lambda Q. P \subseteq Q$ 
\ex \sib{\textsc{op}$_\cup$} $= \lambda x.\lambda y.\lambda z. z\leq x \vee z = y$ \hfil such that $x \neq y$
\z
\z 

%\ex \label{we:struc} {(underlying syntactic structure of 1st/2nd person plural pronoun)}
\begin{figure}
% the [ht] option means that you prefer the placement of the figure HERE (=h) and if HERE is not possible, you prefer the TOP (=t) of a page
% \centering
    \begin{forest} for tree= fairly nice empty nodes
%     for tree={s sep=1cm, inner sep=0, l=0}
[ DP
    [ D ]
        [ [ {[1st.\textsc{sg}] / [2nd.\textsc{sg}]} ]
        [ [ \textsc{op}$_\cup$ ]
        [  \textit{pro}$_i$ ]
        ]
        ] 
        ]
\end{forest}
\caption{Underlying syntactic structure of 1st/2nd person plural pronoun}
    \label{fig:butsch:we:struc}
\end{figure}
%\ex {(semantic interpretation 1st person plural pronoun)}
\begin{figure}
% the [ht] option means that you prefer the placement of the figure HERE (=h) and if HERE is not possible, you prefer the TOP (=t) of a page
% \centering
    \begin{forest}for tree= fairly nice empty nodes
%     for tree={s sep=1cm, inner sep=0, l=0}
[ $\lambda Q.(\lambda z. z \leq g(i) \vee \cnst{spkr}(z) \wedge \cnst{atomic}(z)) \subseteq Q$ 
    [ \textit{we} ]
        [ [ $\langle I \rangle$ \\
 $\cnst{spkr}$ ]
        [ [ \textsc{op}$_\cup$ ]
        [  $g(i)$]
        ]
        ] 
        ]
\end{forest}
\caption{Semantic interpretation 1st person plural pronoun}
    \label{fig:butsch:label2}
\end{figure}
%\ex {(semantic interpretation 2nd person plural pronoun)}
\begin{figure}
% the [ht] option means that you prefer the placement of the figure HERE (=h) and if HERE is not possible, you prefer the TOP (=t) of a page
% \centering
    \begin{forest}for tree= fairly nice empty nodes
%     for tree={s sep=1cm, inner sep=0, l=0}
[ $\lambda Q.(\lambda z. z \leq g(i)$ $\vee$ \cnst{addr}$(z)$ $\wedge$ \cnst{atomic}$(z)) \subseteq Q$ 
    [ \textit{you}\textsubscript{PL} ] 
        [ [ $\langle you$\textsubscript{SG}$ \rangle$ \\
 \cnst{addr} ]
        [ [ \textsc{op}$_\cup$ ]
        [  $g(i)$ ]
        ]
        ] 
        ]
\end{forest}
\caption{Semantic interpretation of 2nd person plural pronoun}
    \label{fig:butsch:label3}
\end{figure} 
%\z \z 

The restrictor argument of a plural pronoun thus consists of three components: A silent instance of \cnst{spkr} (i.e. reference to the speaker of the utterance) or \cnst{addr} (reference to the addressee of the utterance), a silent operator \textsc{op}$_\cup$, and a contextual assignment function $g(i)$ which surfaces as a silent pronominal form \textit{pro}$_i$ (interpreted as \sib{\textit{pro}$_i$}\textsuperscript{g}, i.e. $g(i)$ again) in the structure. The central idea is that the two referential instances within the restrictor are conflated by \textsc{op}$_\cup$. Roughly speaking, the mechanism of \textsc{op}$_\cup$ can be seen as set union. \textsc{op}$_\cup$ takes two arguments $x, y$ to form a $z$ such that $z$ is identical to $y$ (i.e. the speaker \cnst{spkr} or addressee \cnst{addr}) or $z$ is less or equal to $x$ (i.e. $g(i)$). Note that being less or equal to $g(i)$ will come out as plain $g(i)$ in any case. Actually, \textsc{op}$_\cup$'s $\lambda z$-part is just a formal workaround. Given how this function is defined it acts more or less like a type-shifter in that it takes two type $e$ expressions as input and forming a set containing precisely and only those individuals. So what the 1st and 2nd plural pronouns refer to according to (\ref{we:lex}) can be paraphrased as `the speaker and other(s)' and `the addressee and other(s)', respectively. 



\subsection{Restrictors in PPCs}\label{sec:ppc}
In the previous section, I suggested that plural pronouns have a (syntactically speaking) restrictor argument (and semantically speaking, introduce a restrictor set), and that this restrictor (or restrictor set) determines the plural pronoun's overall reference. The main hypothesis of my proposal regarding PPCs is that the difference between an iPPC and an ePPC interpretation boils down to whether the comitative phrase resides inside the restrictor of the plural pronoun or not -- in the former case, an iPPC reading arises, whereas in the latter case, only an ePPC reading should be available. In particular, the structure of a PPC such as Russian (\ref{ru:we}) under its iPPC meaning is as pictured in Figure \ref{i:struc}.


The crucial aspect in Figure \ref{i:struc} is that the comitative element \textit{s} `with' is assumed to have the very same semantics as \textsc{op}$_\cup$ and that the referent of the comitative phrase occupies the position of the pronominal element interpreted as $g(i)$ in the default plural pronominal structure. That is to say that the comitative phrase \textit{s Petej} `with Petja.\textsc{inst}' as a whole acts as a spell-out of the more general \textsc{op}$_\cup$ plus $g(i)$-part from Figure \ref{fig:butsch:we:struc} %(\ref{we:struc}) 
-- and consequently, \textit{my} in the structural configuration (\figref{i:struc}) refers to just the speaker and Petja. Put differently, the comitative element serves the same purpose as \textsc{op}$_\cup$ here, i.e. it forms a set from its two arguments, namely, the speaker and Petja.

\ea
\gll My s Petej ... \\
 we.\textsc{nom} with Petja.\textsc{inst} \\
 \glt Intended (iPPC): `I and Petja ...' \label{ru:we}
\z

\begin{figure}
% the [ht] option means that you prefer the placement of the figure HERE (=h) and if HERE is not possible, you prefer the TOP (=t) of a page
    \begin{forest}for tree= fairly nice empty nodes
%     for tree={s sep=1cm, inner sep=0, l=0}
[
[ $\lambda Q.(\lambda z. z \leq \textrm{Petej} \vee \cnst{spkr}(z) \wedge \cnst{atomic}(z)) \subseteq Q$ 
    [ \textit{my} ] 
        [ [ $\langle \textrm{I} \rangle$ \\
 $\cnst{spkr}$ ]
        [ [ \textit{s} ]
        [  \textit{Petej} ]
        ]
        ] 
        ]
        [ $\dots$
        ]
        ]
\end{forest}
where \sib{\textit{s}} $=$ \sib{\textsc{op}$_\cup$}
\caption{Semantic interpretation of iPPC structure}
\label{i:struc}
\end{figure}

However, how to derive the ePPC interpretation of (\ref{ru:we}) is not entirely straightforward. Note that we cannot simply adjoin the comitative phrase to the (fully determined) plural pronominal DP -- because on the one hand, it would not be straightforward how to integrate the comitative phrase into the scope argument of the plural pronoun. On the other hand, we might run into incorrect predictions after all. The data that will be presented in the remainder of this section strongly suggests that the comitative phrase is actually rather an adjunct to VP in the ePPC cases. I thus claim at this point that we are dealing with adjunction to VP, and refer to the forthcoming sections for further evidence in favour of this view. 


The predictions of this theory should now be straightforward: if and only if the comitative phrase occurs inside the restrictor of a plural pronoun, an iPPC interpretation arises. On the other hand, if the comitative phrase occupies a position adjacent to the VP, only an ePPC reading is predicted to be available. We turn to data from Torlakian BCMS and Bulgarian which support the suggested ``quantificational'' treatment of plural pronouns next. This choice regarding investigated languages is not due to arbitrary reasons. As will soon become clear, these two languages have very illustrative distinct properties when it comes to PPCs. Moreover, nothing has been said in the literature yet about PPCs in Bulgarian or Torlakian BCMS to the best of my knowledge -- and hence, they are definitely worth a closer inspection. 


To start with, we find a direct syntactic-semantic reflection of the predictions just stated in Bulgarian. In particular, a structure like (\ref{bu:i}) in which the plural pronoun and the comitative phrase are ``tied together'' has an iPPC reading only. In contrast, a split PPC like (\ref{bu:e}) exclusively gives rise to an ePPC interpretation. 

\ea \ea
\gll Nie/\textit{pro} s Peter otidohme v muzeja. \\
 we.\textsc{nom} with Peter went.\textsc{1pl} in museum.\textsc{def} \\
\hfill (Bulgarian) \glt \textit{Unavailable ePPC reading}: `We went to the museum with Peter.'  \\
 `Peter and I went to the museum.'\hfill iPPC\label{bu:i}
\ex
\gll Nie/\textit{pro} otidohme v muzeja s Peter. \\
 we.\textsc{nom} went.\textsc{1pl} in museum with Peter \\
 \glt `We went to the museum with Peter.' \hfill ePPC\\
 \textit{Unavailable iPPC reading}: `Peter and I went to the museum.'\label{bu:e}
\z \z

\noindent In accordance with the comitative phrase's structural proximity to the plural pronoun, we could argue that \textit{s Peter} occurs within the plural pronoun's restrictor in (\ref{bu:i}), but not in (\ref{bu:e}). But while Bulgarian exhibits a structure-meaning correspondence that perfectly matches the predictions of my theory, it happens to be the case that not all Slavic languages (and we might in fact be talking about a minority here) behave like Bulgarian in this respect. In Torlakian BCMS, for instance, both sentences from (\ref{tor:ambi}) are ambiguous between an iPPC  and an ePPC interpretation. That is, no matter whether the plural pronoun and the comitative phrase appear structurally very close to one another or as separate constituents, both readings are available.

\ea \label{tor:ambi}
\ea
\gll \textit{pro} s Mariju smo otišli u muzej. \\
 %\textit{pro}
{} with Maria.\textsc{inst} \textsc{aux.1pl} went in museum \\
\hfill (Torlakian BCMS) \glt `We went to the museum with Maria.' \hfill ePPC\\
 `Maria and I went to the museum.' \hfill iPPC\label{tor:a1}
\ex
\gll \textit{pro} otišli smo u muzej s Mariju. \\
 %\textit{pro}
{} went \textsc{aux.1pl} in museum with Maria.\textsc{inst} \\
 \glt `We went to the museum with Maria.' \hfill ePPC\\
 `Maria and I went to the museum.' \hfill iPPC\label{tor:a2}
\z \z

\noindent The crucial question is how to account for the iPPC interpretation of (\ref{tor:a2}) under the analysis proposed in this article. Because note that (\ref{tor:a1})'s ePPC reading can be easily explained via the assumption that \textit{s Mariju} is an adjunct to any suitable XP from the sentence except inside the plural pronominal DP, or directly attached to it. 



\subsubsection{Together vs. apart: Split PPCs as floated constructions}\label{sec:qfloat}
If a plural pronominal DP is indeed similar to a QP, then we would expect that these two phrasal expressions do not only share striking semantic properties, but also syntactic ones -- such as the possibility to detach restrictor and head: Quantifier-floating (henceforth: Q-floating; for a general discussion see \cite{Sportiche1988}, \cite{Shlonsky1991}, \cite{Merchant1996}, \cite{Boskovic2004}, \cite{Fitzpatrick2006}, among many others).\footnote{At this point, one could object that this is not an accurate description of Q-floating. In some approaches, for example, it is assumed that a quantifier is not a syntactic head in its own right, but merely occupies the specifier position of the restrictor NP/DP. Yet other accounts suggest that quantifiers might be simply adjoined to their restrictor arguments. To enter such a debate is beyond the scope of this article and must be postponed to another occasion. What I assume, though, is a kind of generalized structure for which the crucial point is this: there is a syntactic element (i.e. the quantifier or the plural pronoun) which selects, or at least combines with a restrictor argument. These two expressions can be separated or dislocated from each other via a syntactic movement operation. Note that this is also to say that I assume with \citet{Sportiche1988}, \citet{Shlonsky1991}, \citet{Merchant1996}, or \citet{Boskovic2004} Q-floating to involve syntactic movement. There, too, is an ongoing debate in the literature concerning the question whether this is indeed the case, or whether the quantifier acts more like an adverbial in such constructions (see for instance \cite{Bobaljik1998}), or whether both are lively options (see especially \cite{Fitzpatrick2006} for a detailed cross-linguistic investigation).} I argue here that a similar movement operation is the source of (\ref{tor:a2})'s iPPC interpretation, i.e. that the plural pronoun and the comitative phrase were base-generated within the same DP, but got detached via syntactic movement. But note that I remain intentionally vague in saying only that it is a ``similar'' movement operation -- because strictly speaking, the floated element would be the restrictor in the case of split PPCs, not the head (as in Q-floating structures). I will set this issue aside for the time being and briefly return to it in \sectref{sec:conclusion}.


However, notably, an explanation along the lines of an analogy to Q-floating cannot be applied to Bulgarian split PPCs such as (\ref{bu:e}) from above since Bulgarian does not have ``true'' Q-floating structures.\footnote{By ``true'', I mean Q-floating structures that do not make use of a resumptive pronoun or any other resumptive linguistic device.} This is shown by the ungrammatical transformation of (\ref{bg:nfloat}) into Q-floated (\ref{bg:float}). What might look like a Q-floating structure in (\ref{bg:topic}) at first glance should rather be considered as topicalization since the clitic would not occur in the non-topicalized structure.

\ea \ea
\gll Vsički bademi sa na masata. \\
 all almonds are on table.\textsc{def} \\
 \glt `All almonds are on the table.' \label{bg:nfloat}
\ex[*] {
\gll Bademite sa vsički na masata. \\
 almonds.\textsc{def} are all on table.\textsc{def} \\
 \glt Intended: `The almonds are all on the table.' \label{bg:float}}
\ex {
\gll Knigite gi pročetoh vsičkite. \\
 books.\textsc{def} them.\textsc{cl} read.\textsc{1sg} all.\textsc{def} \\
 \glt %\#
`As for the books, I read them all.' (Intended: `The books I read all.') \label{bg:topic}}
\z
\hfill (Bulgarian; \cite{VulchanovaGiusti1995}: 55)
\z

\noindent In contrast, Q-floating structures exist in Torlakian BCMS, as illustrated in (\ref{tor:nfloat}-\ref{tor:float}) -- and moreover, there seems to be a remarkable correlation in general between the availability of iPPC readings for split PPCs and whether the respective language also permits Q-floating. Take Russian as a further example, where we find the very same pattern as in Torlakian BCMS: split PPCs can give rise to iPPC interpretations and Q-floating structures are possible; see (\ref{ru:split}) and (\ref{ru:nfloat})--(\ref{ru:float}), respectively.\footnote{One could also mention Polish in this context, which seems to pattern more or less with Bulgarian. The picture is, however, not entirely clear. While it has been stated in \citet{FeldmanDyla2008} that the comitative phrase has to occur adjacent to the plural pronoun in order to obtain iPPC readings, my informants found iPPC interpretations somewhat available for split PPCs. Likewise, structures with floated \textit{wszystkie} `all' were occasionally (and given certain intonational circumstances) found acceptable. After all, it thus seems that Polish occupies an intermediate position.} 

\ea \ea
\gll Svi bademi su na sto. \\
 all almonds \textsc{aux.3pl} on table \\
\hfill (Torlakian BCMS) \glt `All almonds are on the table.' \label{tor:nfloat}
\ex
\gll Bademi su svi na sto. \\
 almonds \textsc{aux.3pl} all on table \\
 \glt `The almonds are all on the table.' \label{tor:float}
\z \z

\ea \ea
\gll My pojdëm zavtra s Ivanom v magazin. \\
 we go.\textsc{fut.1pl} tomorrow with Ivan.\textsc{inst} to store \\
 \glt `Tomorrow, we will go to the store with Ivan.' \hfill ePPC\\
 `Tomorrow, Ivan and I will go to the store.' \hfill iPPC\label{ru:split}
\ex
\gll Prišli vse deti. \\
 came.\textsc{pl} all children \\
 \glt `All the children came.' \label{ru:nfloat}
\ex
\gll Deti prišli vse. \\
 children came.\textsc{pl} all \\
 \glt `The children all came.' \label{ru:float}
\hfill (Russian; \cite{Fitzpatrick2006}: 144)
\z \z 

\noindent The parallel between split PPCs and Q-floating does not only account for the availability of iPPC readings for split PPCs, but also for the other piece of data that remained puzzling under \citeposst{VassilievaLarson2005} approach -- namely the lack of an iPPC interpretation of \textit{wh-}questions like (\ref{q:vaslars}), repeated in (\ref{q:vs2}) for convenience.

\ea
\gll S kem my xodili v magazin? \\
 with whom we went to store \\
 \glt \textit{Unavailable iPPC interpretation}: `With whom did I go to the store?' \\
\hfill (Russian; \cite{VassilievaLarson2005}: 122) \label{q:vs2}
\z

\noindent In the terms of my account, a structure like (\ref{q:vs2}) would have to involve the reverse kind of Q-floating movement operation; and carrying out such a movement does simply not yield grammatical results. We would thus expect it to be unavailable for QPs and plural pronominal DPs alike. Examples analogous to the Russian (\ref{q:vs2}) do also not give rise to iPPC readings in Torlakian BCMS: 

\ea
\gll S koj ste prekjuče išli u prodavnicu? \\
 with who \textsc{aux.2pl} the.day.before.yesterday gone in shop \\
 \glt `With whom did you\textsubscript{PL} go to the store the day before yesterday?'\\
 \textit{Unavailable}: `With whom did you\textsubscript{SG} go to the store the day before yesterday?' \hfill (Torlakian BCMS)
\z

\noindent The same holds, unsurprisingly, for Bulgarian (\ref{bg:q}). The only \textit{wh}-question structure for which iPPC interpretations arise are echo questions like (\ref{bg:eq}), i.e. syntactic constructions in which the \textit{wh}-element remains \textit{in situ}.

\ea \ea
\gll S koj vie ste bili v muzeja? \\
 with who you.\textsc{pl} are.\textsc{2pl} been in museum.\textsc{def} \\
 \hfill (Bulgarian)\glt `With whom have you\textsubscript{PL} been to the museum?' \label{bg:q}
\ex
\gll Vie s KOJ otidokhte v muzeja? \\
 you.\textsc{pl} with who went.\textsc{2pl} in museum.\textsc{def} \\
 \glt `You\textsubscript{SG} went to the museum with WHOM?' \label{bg:eq}
\z \z 

\noindent Such cases are covered by the theory presented in this article, since there is nothing in the structure of (\ref{bg:eq}) that would suggest that the comitative phrase does not occur inside of the plural pronoun's restrictor -- rather, the comitative-internal DP just got replaced by a respective wh-phrase.



\subsubsection{Subject Control}\label{sec:sc}
Further support for the view put forth here comes from Subject Control (henceforth: SC) constructions: Consider the contrast between (\ref{tor:scfin}) and (\ref{tor:scinf}) from Torlakian BCMS with regard to (im)possible interpretations to start with.
\ea \ea
\gll \textit{pro} pokušali smo juče s Mariju da PRO popravljamo krov. \\
 {} tried \textsc{aux.1pl} yesterday with Maria.\textsc{inst} \textsc{comp} {} repair roof \\
 \glt `Yesterday, we tried to repair the roof with Maria.' \hfill ePPC\\
 `Yesterday, Maria and I tried to repair the roof.' \hfill iPPC\label{tor:scfin}
\ex
\gll \textit{pro} pokušali smo juče da PRO popravljamo krov s Mariju. \\
 {} tried \textsc{aux.1pl} yesterday \textsc{comp} {} repair roof with Maria.\textsc{inst} \\
 \glt `Yesterday, we tried to repair the roof with Maria.' \hfill ePPC\\
 \textit{Unavailable iPPC interpretation:} `Yesterday, Maria and I tried to repair the roof.' \label{tor:scinf}
\z \hfill (Torlakian BCMS)\z

\noindent If, as in (\ref{tor:scfin}), the comitative phrase occurs in the matrix, we get the standard ambiguity between an iPPC  and an ePPC reading. But if, on the other hand, the comitative phrase occupies a syntactic position inside of the embedded complex \textit{popravljamo krov} `repair the roof', as in (\ref{tor:scinf}), an iPPC interpretation is not available. Given the observations from the previous section, i.e. that split PPCs usually also give rise to iPPC interpretations in Torlakian BCMS, the question arises why this reading is unavailable for (\ref{tor:scinf}). 


It is commonly assumed that PRO is a null pronoun that lacks any phonological content. Thus, PRO is treated as an empty category (see \cite{Chomsky1981} for the origins of this notion). Other instances falling into this class are traces of moved phrases, including wh-movement traces, and (dropped) \textit{pro}. But in contrast to \textit{pro}, for instance, there is no kind of overt NP or DP that corresponds to PRO.\footnote{Whereas for \textit{pro}, the corresponding overt expression would be the pronoun with the respective number and person features, of course.} PRO is simply a silent anaphoric element that gets bound by an antecedent expression. To the best of my knowledge, no specific assumptions have been made in the literature regarding any sort of internal structure of PRO. Hence we should not go all wrong if we presume that there is none, i.e. if we take PRO just to be a syntactic subject placeholder element that has to occupy SpecTP in SC constructions due to the EPP (Extended Projection Principle). 


So since PRO has no internal structure, the comitative phrase in (\ref{tor:scinf}) cannot occur inside of it -- because PRO has no such thing as what I dubbed a restrictor here. Consequently, \textit{s Mariju} can only be part of the plural pronoun's scope argument in (\ref{tor:scinf}). Given that we are dealing with an SC construction, no movement of the plural pronoun beyond \textit{da} can be argued to be involved. Or, put differently, there is just no instance whatsoever (neither overt nor covert) of the plural pronoun inside of (\ref{tor:scinf})'s PRO-part such that we could assume the comitative phrase to be a leftover of.\footnote{It has been suggested in \citet{Hornstein1999}, however, that SC does indeed involve movement of the subject from the infinitival clause to a higher syntactic position in the matrix. That is, PRO, in \citeposst{Hornstein1999} sense, would rather be a trace than a silent anaphoric pronominal. Under such an analysis, the argument concerning (\ref{tor:scinf}) just made above might seem a bit misguided. However, even if the movement-based approach to SC constructions was the more suitable solution (for a discussion, see \cite{Landau2003}), there is another issue standing in the way of \textit{s Mariju} originating in the restrictor of the plural pronoun in (\ref{tor:scinf}) -- namely clause-boundedness of floated quantifiers. I return to this issue in more detail below.} On the one hand, we cannot claim that the comitative phrase was base-generated within PRO (for the reasons just mentioned) and that the two got separated via a syntactic movement operation akin to Q-floating. On the other hand, we also cannot claim that the (dropped) plural pronoun does, or at any point of the derivation did, appear inside of the \textit{da}-complement of the matrix. So in sum, my analysis correctly rules out iPPC interpretations for sentences like (\ref{tor:scinf}), but correctly predicts the availability of an iPPC reading for sentences such as (\ref{tor:scfin}), where the PPC also occurs split, but the (dropped) plural pronoun and the comitative phrase are inside the same clause -- and that is all the more important: Because if the movement operation behind split PPCs indeed resembles Q-floating, this is another reason why (\ref{tor:scinf}) has no iPPC interpretation, even under an account to SC such as \citeposst{Hornstein1999}. Q-floating is clause-bound (cf. \cite{Kayne1981}), i.e. the moved constituent cannot cross certain syntactic boundaries such as (full) clauses. In this sense, the complementizer \textit{da}, or as we could also say, PRO, constitutes such a crucial border: a PPC that appears (linearly speaking) behind it cannot, and cannot have been part of the plural pronoun's restrictor. Thus, only ePPC readings are available for structures such as (\ref{tor:scinf}). But a PPC that occurs in the matrix, may it be split or not (as long as both parts reside inside the matrix), can give rise to iPPC interpretations. 


The significance of these SC examples can easily be overlooked, and what they indicate can just as easily be misunderstood. So I recap here why those data are relevant and what they show. First of all, one should bear in mind that split PPCs give rise to iPPC readings in Torlakian BCMS -- nevertheless, this interpretation is not available for (\ref{tor:scinf}). And, na\"{i}vely speaking, there is no reason why this should be the case. Hence if iPPC readings arise for split PPCs in Torlakian BCMS, what is (\ref{tor:scinf})'s special trait that blocks this interpretation? Or, asked the other way around, what would have to be the case in order to obtain an iPPC interpretation for (\ref{tor:scinf})? Well, the comitative phrase (inside the \textit{da}- or PRO-part of the sentence) would need to have been base-generated within the DP of the (dropped) plural pronoun that occurs in the matrix. Is it a feasible assumption that at any stage of the derivation, this was the case in (\ref{tor:scinf})? I argue here that it is not. If we stick to the more common treatment of SC constructions in terms of PRO, then neither has the plural pronominal DP ever occurred within the \textit{da}-part of the sentence, nor can the comitative phrase occur inside the restrictor of PRO itself since there is no such thing. If we follow \citet{Hornstein1999} instead in assuming that SC constructions actually do involve syntactic movement, then still, an iPPC interpretation of (\ref{tor:scinf}) is ruled out under my account -- but for a slightly different reason. Under this view, we would have to say that the comitative phrase can nonetheless not have been base-generated within the plural pronoun's DP because Q-floating is a clause-bound operation. And if split PPCs involve a similar kind of movement, \textit{s Mariju} must have originated somewhere else in the structure but within the plural pronoun's restrictor.


Nevertheless, one could object that the decisive difference between (\ref{tor:scinf}) and (\ref{tor:scfin}) lies elsewhere in the structure anyway. Specifically, in the fact that PRO is an anaphoric element that needs a binder, i.e. an antecedent expression from which it gets its reference. Such a binder expression can only occur somewhere higher in the syntactic structure, that is, in the matrix part of the sentence. Now the comitative phrase occurs in (\ref{tor:scfin}), but not in (\ref{tor:scinf}) within the matrix. The consequence is that PRO can be bound in (\ref{tor:scfin}), but not in (\ref{tor:scinf}) by the PPC -- may it have an underlying ePPC  or an underlying iPPC structure in (\ref{tor:scfin}). And then, there is nothing in the structure of (\ref{tor:scinf}) that would even suggest that we are dealing with a split iPPC. I would like to make two comments on such a line of reasoning to conclude this section: First of all, if that was indeed the reason for the difference in terms of available interpretations between (\ref{tor:scfin}) and (\ref{tor:scinf}), it does no harm to the theory proposed here -- \textit{s Mariju} would end up being a part of the plural pronoun's scope argument anyway. Consequently, we correctly predict only an ePPC reading to be available for (\ref{tor:scinf}). It is fair to admit, however, that other analyses of PPCs would probably make similar predictions.\footnote{For example, if an iPPC had an underlying coordinative structure, then one could argue that the comitative phrase cannot be coordinated with PRO; no matter whether we are dealing with an asymmetric or symmetric kind of coordination. The precise predictions apposition-based theories would make in this respect are somewhat more difficult to calculate. But we could at least state the following: Appositives occur quite freely within a sentence in general, they need not occur strictly adjacent to their anchor expressions. A sentence-final position as the comitative phrase in (\ref{tor:scinf}) occupies, however, is one of the positions that is usually readily available for appositions. Since many aspects have not been spelled out in \citeposst{Cable2017} approach, it can only be surmised that this theory (as it currently stands) would probably not preclude an iPPC reading of (\ref{tor:scinf}).} But second, it has not yet been once and for all decided whether SC constructions actually involve a PRO, or whether \citeposst{Hornstein1999} approach involving movement is more practicable in the end -- although one must of course say that there are many aspects about SC in favour of the PRO view. Should the latter option nevertheless be the case, my analysis has a suitable explanation prepared. Lastly, we must not forget that some analyses of PPCs face difficulties in explaining iPPC readings of split PPCs at all. 




\subsubsection{Binding data}\label{sec:bind}
My analysis in its current form makes the following predictions regarding the binding of non-independently referential expressions such as \textit{-self} anaphors. If the plural pronoun's restrictor is made up of a silent instance of reference to the speaker as well as of a comitative phrase including a comitative referent (i.e. in the iPPC case), we expect that the two together can co-bind an anaphor. If, on the other hand, the plural pronoun's restrictor is fully determined by \cnst{spkr} and \textit{pro}$_i$ (i.e. in the ePPC case), we expect that the comitative referent cannot participate in the binding of an anaphor. I use exclusively Russian data in this section to show that these predictions are borne out. The reason for this change in the languages discussed is the following: unlike Torlakian BCMS, Russian has the possessive \textsc{self} anaphor \textit{svoj-}. This element can be bound by antecedents of various morpho-syntactic kinds without changing its surface form. That is to say that \textit{svoj-} can, for instance, be bound by a referential expression in the singular, or by a referential expression in the plural without having to adapt its own morpho-syntactic shape -- it will remain as plain \textit{svoj-} either way. In Torlakian BCMS, pronouns such as \textit{naše} `our' would have to be used instead in analogous examples as the ones from Russian below. The virtue of discussing examples with \textit{svoj-} is thus that we are on the safer side in excluding cases of mere co-reference (instead of binding).


The (un)available referential properties of \textit{svoj-} under an iPPC  and an ePPC reading of the Russian sentence in (\ref{ru:bind}) show that our predictions are correct. Although \textit{svoj-} could also be interpreted as \textit{his} (in the sense of \textit{Petja's}) in principle, this binding configuration is not available under the iPPC reading of (\ref{ru:bind}). Namely, the comitative referent from inside the plural pronoun's restrictor is not available as a sole binder of \textit{svoj-} -- and neither is the silent instance of reference to the speaker. Rather, all referential instances from the restrictor of the plural pronoun jointly bind the \textsc{self} element here.

\newpage
\ea
\gll My s Petej čitaem svoju knigu. \\
 we.\textsc{nom} with Petja.\textsc{inst} read.\textsc{1pl} \textsc{poss.refl} book \\
 \glt \hfill (Russian; \cite{VassilievaLarson2005}: 112) \label{ru:bind}
\ea `We$_j$ are reading \textsc{self}$_{j/*i/*j+i}$'s book with Petja.' \hfill ePPC\label{ru:binde}
\ex `Petja$_i$ and I$_j$ are reading \textsc{self}$_{*j/*i/j+i}$'s book.' \hfill iPPC
\z \z 

\noindent As can be seen from the respective paraphrase in (\ref{ru:binde}), the comitative referent is not at all involved in binding \textit{svoj-} in the ePPC case. We find the same pattern in ``regular'' comitatives where the comitative phrase is commonly analyzed as an adjunct to VP, see (\ref{ru:bindr}).
\ea
\gll Mal'čik s kotënkom ušël v svoju komnatu. \\
 boy.\textsc{nom} with kitten.\textsc{inst} went.\textsc{3sg} to \textsc{poss.refl} room \\
 \glt `The boy$_j$ went to \textsc{self}$_{j/*i/*j+i}$'s room with the kitten$_i$.' \\
 \hfill (Russian; \cite{VassilievaLarson2005}: 109) \label{ru:bindr}
\z

\noindent Therefore the conclusion suggests that the comitative phrase adjoins to VP in ePPC structures as well. 




\section{Conclusion}\label{sec:conclusion}
The analysis presented in this article is based on the assumption that plural pronouns and quantifiers have some crucial semantic and syntactic properties in common. In particular, I suggested that plural pronouns select restrictor arguments just like quantifiers. Analogous to instances of universal quantification, a plural pronoun conveys that for all elements in its restrictor denotation, it holds that $Q$ (where $Q$ is some predicate). Under my approach, the restrictor of a plural pronoun contains the following three ingredients: A silent instance of the plural pronoun's singular counterpart (reference to the speaker/addressee), a silent pronominal element whose reference is determined by means of a contextual assignment function $g(i)$ (reference to other(s)), and a silent operator \textsc{op}$_\cup$. The function of this operator is to form a set consisting of exactly those individuals to which the two referential instances from the restrictor refer. Thus, \textsc{op}$_\cup$ basically has a mechanism amounting to set union. 


I argued that in an iPPC structure, the comitative phrase acts as the spell-out of the more general \textsc{op}$_\cup$ plus $g(i)$ part which is present in the restrictor of a plural pronoun anyway by default. Moreover, I claimed that this was the case since the comitative element in these constructions has the very same semantics as \textsc{op}$_\cup$ -- that is to say that in an iPPC the comitative phrase occurs inside the plural pronominal restrictor, i.e. within the same DP.\footnote{As an anonymous reviewer mentions, this raises the question why (i)PPCs are not more widespread across the world's languages. But one of the crucial points of my analysis is indeed that the comitative element \textit{s} and \textsc{op}$_\cup$ have the same semantics -- and there is no salient reason to assume that a particularly large number of languages should have a comitative element whose semantics matches those of \textsc{op}$_\cup$. Moreover, other languages have found different ways of expressing the same meaning as an iPPC. Just consider Icelandic which has so-called Pro[NP] constructions (cf. \cite{SigurdssonWood2021}). Those Pro[NP]s superficially differ from PPCs only in that they lack a comitative element.} I suggested that we can account for the availability of iPPC interpretations of split PPCs by assuming that these constructions involve a kind of syntactic movement similar to the kind involved in Quantifier-floating structures. And indeed, we found an intriguing correlation: the very Slavic languages that allow iPPC readings of split PPCs also have Quantifier-floating structures. However, as already pointed out in \sectref{sec:qfloat}, the wrong element is ``floated'' under my approach actually. That is, if we really want to assume that split PPCs are related to Q-floating structures, then we are faced with the question of why the restrictor gets floated in a split PPC, whereas the quantifier does in a Q-floating structure. I am aware of this problem, and of the fact that the analogy between these two cannot be 1:1 -- but for the sake of explicitness, I stick with it for the time being; and leave it as an issue for further research which kind of syntactic movement could be involved in split PPCs instead, if any. An obvious candidate is so-called ``Left Branch Extraction'' (LBE; cf. \cite{Boskovic2008}, and the references therein). But whether the generalizations made in relation to LBE in Slavic languages, and the generalizations we could derive in this article are actually compatible with each other remains to be seen. And in any case, it seems remarkable that the positions available for floated quantifiers within a sentence coincide with the positions available for the comitative phrase in a split PPC constructions -- at least in Torlakian BCMS.\footnote{But to make Torlakian BCMS not to appear very exotic in this respect: The same correspondence of available positions can also be found in other languages (belonging to different language families) that have both Quantifier-floating and (split) PPCs -- namely Finnish and Fenno-Swedish; see also \citet{Butschety2023}.}


Other issues remained open here as well. For example, it is puzzling why an iPPC reading in Torlakian BCMS is much more salient when the plural pronoun is dropped, and \textit{vice versa} an ePPC interpretation is much more salient when the pronoun is pronounced; see (\ref{tor:salient}). 
\ea \label{tor:salient} \ea
\gll Mi smo s Mariju otišli u muzej. \\
 we.\textsc{nom} \textsc{aux.1pl} with Maria.\textsc{inst} went in museum \\
 \glt `We went to the museum with Maria.' \hfill (more salient) ePPC\\
 `Maria and I went to the museum.' \hfill (less prominent) iPPC
\ex
\gll \textit{pro} s Mariju smo otišli u muzej. \\
 {} with Maria.\textsc{inst} \textsc{aux.1pl} went in museum \\
 \glt `We went to the museum with Maria.' \hfill (less prominent) ePPC\\
 `Maria and I went to the museum.' \hfill (more salient) iPPC
\z \hfill (Torlakian BCMS)\z 

\noindent I can only speculate here that this is a pragmatic effect. More precisely, one could assume that the pronunciation of the plural pronoun (which could otherwise be dropped) indicates that we are dealing with a fully determined instance of the plural pronoun and that the comitative phrase consequently does not occur inside its restrictor. 


Moreover, it remains unclear why predicates such as \textit{hate broccoli} or \textit{believe in God} are felicitous with PPCs under an iPPC interpretation (but crucially not under an ePPC interpretation) in Russian (see (\ref{ru:brocc}) and (\ref{po:brocc})), but not in Torlakian BCMS, where analogous sentences are simply infelicitous no matter which reading is intended.
\ea \ea
\gll My s Dašei verim v boga. \\
 we.\textsc{nom} with Daša.\textsc{inst} believe.\textsc{1pl} in God \\
 \glt \textit{Unavailable ePPC reading}: `We believe in God with Daša.' \\
 `Daša and I believe in God.' \hfill iPPC \\
 \hfill (Russian; \cite{Feldman2003}) \label{ru:brocc}
\ex
\gll My s Ivanom nenavidim brokkoli. \\
 we.\textsc{nom} with Ivan.\textsc{inst} hate.\textsc{1pl} broccoli \\
 \glt \textit{Unavailable ePPC reading}: `We hate broccoli with Ivan.' \hfill \\
 `Ivan and I hate broccoli.' \hfill iPPC \\
 \hfill (Russian; \cite{VassilievaLarson2005}: 112) \label{po:brocc}
\z \z 

\noindent These topics need a much longer discussion, which I have to defer to another occasion. 


To conclude, based on the observed analogies between plural pronouns and (universal) quantifiers, the two readings PPCs in Slavic can give rise to were derived in terms of whether the comitative phrase occurs inside the plural pronominal restrictor (=iPPC), or outside of it as an adjunct to VP (=ePPC). Assuming that split PPCs involve a similar kind of syntactic movement as floated quantifiers, my analysis could derive in which Slavic languages (namely precisely in those that have Q-floating structures) on the one hand, and in which syntactic configurations on the other hand iPPC readings arise for split PPCs. While this analogy turned out to be not as direct as intended, the correlation between the availability of floated (universal) quantifiers and iPPC readings for split PPCs nonetheless represents a noteworthy datum. I leave it to future research to decide on whether and how these two phenomena can be put into a uniform picture. 



\section*{Abbreviations}

\begin{tabularx}{.5\textwidth}{@{}lQ}
\textsc{1}&first person\\
\textsc{2}&second person\\
\textsc{3}&third person\\
\textsc{addr}&addressee\\
\textsc{aux}&auxiliary\\
\textsc{cl}&clitic\\
\textsc{comp}&complementizer\\
\textsc{def}&definite\\
\end{tabularx}%
\begin{tabularx}{.5\textwidth}{lQ@{}}
\textsc{f}&feminine\\
\textsc{fut}&future tense\\
\textsc{inst}&instrumental\\
\textsc{nom}&nominative\\
\textsc{pl}&plural\\
\textsc{poss.refl}&possessive reflexive\\
\textsc{spkr}&speaker\\
\textsc{sg}&singular
%&\\
 % this dummy row achieves correct vertical alignment of both tables
\end{tabularx}

\section*{Acknowledgments}
Many thanks to Boban Arsenijevi\'{c}, Edgar Onea, Vesela Simeonova, and Radek \v{S}imík for helpful comments and extensive discussions. Thanks also to two anonymous reviewers for their valuable remarks, and to the audiences at FDSL 15 and NELS 53. Part of the work on this article was supported by the Slovenian Research and Innovation Agency (ARIS) grant P6-0382.

%\printbibliography%[heading=bibliography,notkeyword=this]


\printbibliography[heading=subbibliography,notkeyword=this]

\end{document}
