\documentclass[output=paper,colorlinks,citecolor=brown]{langscibook}
\ChapterDOI{10.5281/zenodo.15394182}
%\bibliography{localbibliography}
 
\author{
Franc Lanko Marušič\orcid{0000-0002-0667-3236}\affiliation{University of Nova Gorica} and Petra Mišmaš\orcid{0000-0001-8659-875X}\affiliation{University of Nova Gorica} and Rok Žaucer\orcid{0000-0001-7771-6937}\affiliation{University of Nova Gorica}}
% orcid doesn't appear printed; it's metainformation used for later indexing

\SetupAffiliations{mark style=none}

%% in case the running head with authors exceeds one line (which is the case in this example document), use one of the following methods to turn it into a single line; otherwise comment the line below out with % and ignore it
% \lehead{Marušič, Mišmaš \& Žaucer}
% \lehead{Radek Šimík et al.}

\title{The (un)expectedly stacked prefixes in Slovenian}
% replace the above with your paper title
%%% provide a shorter version of your title in case it doesn't fit a single line in the running head
% in this form: \title[short title]{full title}

\abstract{When a Slavic verb occurs with multiple prefixes their order is often claimed to follow certain restrictions of a fairly formal character. Firstly, lexical prefixes, which can modify the argument structure of the verb and contribute idiosyncratic interpretations, are always found adjacent to the verbal root, while superlexical prefixes, which do not alter the argument structure and whose interpretative contribution is adverbial, can be stacked over the lexicals. And secondly, when multiple superlexicals stack on a verbal stem, they follow a fixed order. We set out to test these two generalizations with a corpus study. We find that there exist a number of verbs which seem to have more than one lexical prefix, in direct contradiction of the standard assumptions about prefixation.

\keywords{verbal prefixes, internal prefixes, external prefixes, prefix stacking}
}

\begin{document}
\maketitle

\section{Introduction}\label{sec:introduction}

In Slovenian and in Slavic languages more generally, simplex verbs consist of a root, a theme vowel [\textsc{tv}] and a tense and agreement ending [\textsc{t/agr}], and are typically imperfective (though this is not a rule, cf. e.g. the Slovenian perfective simplex verb \textit{kupiti} `to buy'). Verbs can also carry one or more prefixes, with the prefixed form generally being interpreted perfectively (unless imperfectivized through, for example, suffixation in the process called secondary imperfectivization [\textsc{si}]). We demonstrate this for the verb \textit{znati} `to know' and some of its derivatives in \tabref{tab:prerazporediti2}.\footnote{Unless indicated otherwise, all examples in this paper are Slovenian.} 
    

\begin{table}
\begin{tabularx}{.8\textwidth}{lllllll}
\lsptoprule
  prefix  & prefix & root & \textsc{si}  & \textsc{tv} & \textsc{t/agr} & Gloss\\
 \midrule
  &  & zn &   & a & ti & `to know.\textsc{ipfv}'\\     
 & po & zn &   & a & ti & `to know.\textsc{pfv}'\\  
 & po & zn &  av & a & ti & `to know.\textsc{ipfv}'\\ 
pre & po & zn  &  & a & ti & `to recognize.\textsc{pfv}'\\ 
pre & po & zn  & av & a & ti & `to recognize.\textsc{ipfv}'\\ \lspbottomrule
\end{tabularx}
\caption{The various parts of the Slavic verb}
\label{tab:prerazporediti2}
\end{table}

Turning to verbal prefixes, these are, in general, all formally related to prepositions (e.g., \textit{ob} `by/next to', \textit{pri} `at', etc., cf. \citealt{matushansky2002formal, gehrke2008ps, caha2021prefixes}, a.o.), but are often assumed to differ among themselves in terms of their position within the verbal domain. Typically, a distinction is made between so-called lexical and superlexical prefixes. The former are often seen as affixal prepositions functioning as VP-internal resultative secondary predicates, similarly to resultative particles in Germanic, the latter as affixal prepositions functioning as VP-external, INFL-level material, e.g., \citet{Ramchand2004, Romanova2004, svenonius2004slavic}, and each type is said to behave uniformly with respect to a number of properties. The tree in \figref{fig:crt} sketches the relevant positions. A more detailed overview is given in \sectref{sec:whatWeKnow}. 

\begin{figure}
\begin{forest}
sn edges/.style={for tree={s sep=10mm, inner sep=0, l=0}},
sn edges
	[
	[superlexical][... 
	[\textcolor{white}{nič}] [VP
	[V\textsuperscript{0}][... 
	[lexical] [...]]]]]
\end{forest}
     \caption{A sketch of the two positions of the two types of prefixes}
     \label{fig:crt}
 \end{figure}

One important distinction between the two types of prefixes that the literature often seems to convey (even if sometimes unintentionally) is that a verb will -- generally -- only have one lexical prefix, while superlexical prefixes can stack. The strong tendency that there will only be a single lexical prefix stems from the fact that there is a single position for lexical prefixes, as in \citet{svenonius2004slavic} or \citet{Romanova2004}, or that the semantics of lexical prefixes preclude there being more than one with a given verb, as in \citet{babko2003perfectivity}. When both types of prefixes appear in a verb, the superlexical prefix(es) linearly precede the lexical prefix, and if a verb has multiple superlexical prefixes, these appear in a certain order \citep[e.g.,][]{mar+:milicevic2004lexical, istratkova2006, wiland2012prefix}.

The main goal of this paper is to see if we can find a reflection of these generalizations in Slovenian corpus data, if we can use Slovenian corpus data to corroborate these generalizations about the lexical--superlexical division, in particular the view that stacked prefixes will generally not be lexical prefixes and that superlexical prefixes are governed by strict ordering constraints. If we find the generalizations reflected in corpus data, this can be seen as support for the theoretical claims; but note that if we do not, the claims can still be correct, as theoretical possibilities for the existence of specific structures per se do not necessarily imply anything about these structures' frequency in use.

Whereas we find that our corpus data are of limited use for testing fine-grained proposals for orderings of superlexicals, we do also find that they offer corpus support for some aspects of the ordering claims. At the same time, our corpus data also reveal some cases that may appear to be at odds with the expected division. Specifically, while isolated examples of verbs that seem to have two lexical prefixes have been pointed out in the past, e.g.,  \textit{iz-pod-riniti} (lit.: from-under-drive) `to push out' and \textit{s-pod-makniti} (lit.: from-under-move) `to jerk away' have been considered in \citet[242]{svenonius2004slavic}, and see also \citet[260]{markova2011nature} for Bulgarian and \citet[20]{biskup2023} for Russian and Czech, our corpus leads us to an expanded set of verbs that display this unexpected combination. Using this set of verbs we then consider how to analyze verbs in which two prefixes both exhibit properties typical of lexical prefixes.  

The paper is organized as follows. In \sectref{sec:whatWeKnow} we review some widely assumed properties ascribed to the two classes of prefixes. \sectref{sec:corpus} presents a corpus study that focuses on stacked verbal prefixes. \sectref{sec:4unexpectedStacking} discusses the data with potentially unexpectedly stacked prefixes, \sectref{mar+:sec:conclusion} presents the conclusions. 


\section{What we know: Lexical and superlexical prefixes in Slavic verbs}\label{sec:whatWeKnow}

A fairly standard division of prefixes that is also characteristic of the more traditional literature \citep[e.g.][]{Muha1993,mar+:toporisic2000}, and is typically assumed to hold for all Slavic languages, establishes two main uses of prefixes. Lexically used prefixes tend to have spatial or idiosyncratic meanings, where ``idiosyncratic'' is meant to capture situations in which the prefix's addition to the verb does not lead to a systematically predictable interpretation of the prefix-verb stem complex, as shown in \REF{ex:InternalPrefixes}. With superlexically used prefixes, on the other hand, the addition of the same prefix predictably adds the same (adverbial) interpretation, and the interpretation of the verb stays transparent and constant across the prefixed verb class, \REF{ex:ExternalPrefixes1}.\footnote{For expository reasons, we ignore Slovenian orthography and separate prefixes from the rest of the verb with a hyphen. Prefixes are glossed on the basis of the basic meanings of their prepositional counterparts.}


\ea 
\glll ob-delati | ob-soditi | ob-noviti | ob-leteti\\
{at}-work {} {at}-judge {} {at}-new {} {at}-fly \\
{`to process'} | {`to sentence'} | {`to renew'} | {`to fly around'} \label{ex:InternalPrefixes} \\
\z

\ea
\gllll {po-sedeti} | {po-bingljati} | {po-plesati} | {po-igrati se}\\
{{over}-sit} {} {{over}-dangle} {} {{over}-dance} {} {{over}-play  \textsc{refl}} \\
{`to sit for} | {`to dangle } |  {`to dance} | {`to play for a}\\
{\phantom{'}a while'} | {\phantom{'}for a while'} |  {\phantom{'}for a while'} | {\phantom{'}while'}\\
\label{ex:ExternalPrefixes1}
\z

%\ea
%\glll {po-sedeti} | {po-bingljati} | {po-plesati} | {po-igrati se}\\
%{{over}-sit} | {{over}-dangle} | {{over}-dance} | {{over}-play  \textsc{refl}} \\
%{`to sit for a while'} | {`to dangle  for a while'} |  {`to dance for a while'} | {`to play for a while'}\\
%\label{ex:ExternalPrefixes1}
%\z

\noindent The two classes are said to differ in a number of other properties. Lexical prefixes are said to appear directly on the verb root while superlexicals can be separated from the root by another prefix, and consequently, lexical prefixes can never be stacked, while there should be no such restriction, across the board, for superlexicals. Also, only lexical prefixes are said to be able to affect argument structure. And only lexical prefixes, but not superlexicals, can form secondary imperfectives, cf., e.g., \citet[229]{svenonius2004slavic} for the diagnostics for superlexical prefixes, though note also that even for \citeauthor{svenonius2004slavic} some subclasses of superlexical prefixes can violate this last constraint  \citep[230]{svenonius2004slavic}. These properties are summarized in \tabref{tab:LexAndSuperLex}.

\begin{table}
\begin{tabular}{l l } 
 \lsptoprule
  {Lexical prefixes} &  {Superlexical prefixes}  \\
  \midrule
  adjacent to the root & outside of lexical prefixes  \\
 idiosyncratic/PP meanings & adverbial meanings \\ 
affect argument structure & don't affect argument structure  \\
form secondary impf. & don't form secondary impf.  \\
 generally don't stack & can stack  \\  
 \lspbottomrule
\end{tabular}
    \caption{Lexical and superlexical prefixes}
    \label{tab:LexAndSuperLex}
\end{table}

Many aspects of these generalizations, however, have also been questioned. \citet{zaucer2009vp}, for example, shows that the cumulative prefix \textit{na-} introduces an unselected object -- generally considered a hallmark of lexicality -- but can, at the same time, also stack over another prefix. A number of authors argued that the split should be in more than two groups: for example, \citet{tatevosov2008intermediate} argues for an independent, third class of \textit{intermediate} prefixes; \citet{babko2003perfectivity} splits lexical prefixes in two groups; \citet{markova2011nature} proposes a four-part division into \textit{outer}, \textit{higher inner}, \textit{lower inner}, and \textit{lexical} prefixes (where the ``traditional" lexical prefixes are split into \textit{lower inner} and \textit{lexical} prefixes).

\subsection{Identity of prefixes}\label{sec:whatWeKnowSlo}
 
What is phonologically one and the same prefix can often be used as either a lexical or a superlexical prefix, as shown in \REF{ex:lexPOsuperPO}--\REF{ex:lexNAsuperNA}. So if prefixes are defined with their phonological shape one should really only talk of their lexical or superlexical uses, rather than of lexical and superlexical prefixes.

\ea \label{ex:lexPOsuperPO}
\ea 
\gll po-liti \\
 {over}-pour\\
 \glt `to spill'\label{ex:lexPOsuperPOlex}
\ex \gll po-sedeti \\
 {over}-sit\\
\glt `to sit for a while'\label{ex:lexPOsuperPOsup} 
\z 
\z

\ea \label{ex:lexNAsuperNA}
\ea
\gll do-staviti \\
 {to}-put\\
 \glt `to deliver' \label{ex:lexDOsup}
\ex \gll do-od-pirati \\
 {to}-{off}-push \\
\glt `to finish opening' \label{ex:superDOlex}
\z 
\z
 
\noindent \textit{Po-} will standardly be analyzed as a lexical prefix resulting in a spatio-id\-i\-o\-syn\-crat\-ic interpretation on the verbal stem in \REF{ex:lexPOsuperPOlex} and as a superlexical prefix with adverbial interpretation in  \REF{ex:lexPOsuperPOsup}, and \textit{do-} as a lexical prefix added to the verbal stem \textit{staviti} (which never occurs on its own without a prefix in most varieties of Slovenian) and as a superlexical prefix added to an already prefixed stem in \REF{ex:superDOlex}. 

Moreover, a prefix can have more than one superlexical use, as shown by the Polish example \REF{ex:PoPoRozKlimek}, where \textit{po-} serves once as a delimitative and once as a distributive prefix \citep[cf. also][]{zaucer2009vp}.

\ea
\gll Kucharze po-po-roz-kładali  przez chwilę naczynia i zajęli się czymś innym.\\ 
cooks po.\textsc{delim}-po.\textsc{dist}-roz-put.\textsc{si} over all tables and began \textsc{refl} something else\\ 
\glt `The cooks put the dishes on the table for a while and they turned their attention to something else.’ \label{ex:PoPoRozKlimek} \\\hfill \citep[Polish;][]{KlimekJankowskaBlaszczak2022}
\z 



\subsection{Stacking}\label{sec:IntroStacking}

As mentioned above, it has been observed that when Slavic verbal prefixes stack, their ordering is not random, but rather reveals certain restrictions of a fairly formal character. For one, lexical prefixes attach  to the verb before superlexical prefixes, and as a consequence, in any form with multiple prefixes, if the form includes a lexical prefix, the lexical prefix will appear closest to the verb, as sketched in \REF{ex:orderprefixes}. The other observation, also sketched in \REF{ex:orderprefixes}, is that superlexical prefixes (and only superlexical prefixes) can stack even over other superlexical prefixes so that a single verb can have more than one superlexical prefix but, normally, just one lexical prefix \citep[cf.][]{Romanova2004,svenonius2004slavic,gehrke2008ps} (though some authors, e.g. \citealt{tatevosov2008intermediate}, argue that Russian actually does not allow stacking of ``genuine" superlexical prefixes (i.e., inceptive \textit{za}-, delimitative \textit{po}-, cumulative \textit{na}- and distributive \textit{pere}-) but only of ``intermediate" prefixes, cf. above).

\ea superlexical prefix $>$  superlexical prefix $>$ lexical prefix $>$ verb \label{ex:orderprefixes}
\z

\noindent The restriction to no more than one lexical prefix is taken to reflect the widely assumed general restriction to one independent resultative secondary predicate per verb \citep[a.o.][]{rappaport2001event,ramchand2008verb}, and suggests a further difference between lexical and superlexical prefixes.\footnote{This restriction is sometimes also suggested to have a non-structural, conceptual explanation, e.g. the Single Delimiting Constraint in \citet{tenny1994aspectual} and \citet{filip2003prefixes}. In this paper, we focus on the structural approach.} Slavic lexical prefixes are parallel to resultative secondary predicates in languages like English, while superlexicals appear to be something different (cf. also \citealt{spencer&zaretskaya1998}).


The superlexical prefixes are also said to follow a fixed order when stacked to the same verbal stem \citep{istratkova2006, wiland2012prefix, endo2014symmetric,KlimekJankowskaBlaszczak2022}. For example, as claimed by \citet{wiland2012prefix}, who develops an even more fined-grained, cartography-inspired differentiation of superlexical prefixes, the cumulative prefix \textit{na-} needs to precede the completive prefix \textit{do-}, as shown in \REF{ex:wilandDoNaDo}. \citet{istratkova2006} proposes the order in \REF{ex:istrakovaOrder} for Bulgarian, \citet{wiland2012prefix} proposes the sequence in \REF{ex:wilandOrder} for Polish, which was later modified by \citet{KlimekJankowskaBlaszczak2022} to \REF{ex:klimekOrder}.


\ea \label{ex:wilandDoNaDo}
\ea \gll na-do-kładaj sobie jeszcze \\ 
\textsc{cuml}-\textsc{compl}-put self more\\ 
\glt ‘get yourself some more (e.g. food)’ \label{ex:wilandNaDo}
\ex *do-na-kładaj sobie jeszcze \label{ex:wilandDoNa} \hfill \citep[Polish;][]{wiland2012prefix}
\z \z 

\ea \gll \textsc{att} $>$ \textsc{incp} $>$ \textsc{compl} $>$ \textsc{dist} $>$ \textsc{cuml} $>$ \textsc{exc} $>$ \textsc{rep} \label{ex:istrakovaOrder} \\
\textit{po}  {} \textit{za} {} \textit{iz} {} \textit{po} {} \textit{na} {} \textit{raz} {} \textit{pre}\\
\glt \strut \hfill \citep[Bulgarian;][]{istratkova2006}
\z

 \ea 
\gll \textsc{dist} $>$ \textsc{att} $>$ \textsc{delim} $>$ \textsc{sat} $>$ \textsc{cuml} $>$ \textsc{exc} $>$ \textsc{rep} $>$ \textsc{perd} $>$ \textsc{compl} $>$ \textsc{term} \\
\textit{po}  {} \textit{pod} {} \textit{po} {} \textit{na} {} \textit{na} {} \textit{na} {} \textit{prze} {} \textit{prze} {} \textit{do} {} \textit{od} \\
\glt \strut \label{ex:wilandOrder} \hfill \citep[Polish;][]{wiland2012prefix}
 \z
 
\ea \gll \textsc{delim} $>$ \textsc{dist} $>$ \textsc{sat} $>$ \textsc{cuml} $>$ \minsp{\{} \textsc{perd}, \textsc{exc}, \textsc{rep}, \textsc{att}, \textsc{term}, \textsc{purely pfv}\}\\
\textit{po}  {} \textit{po} {} \textit{na} {} \textit{na} {} {} \textit{prze} \textit{prze} \textit{prze} \textit{pod} \textit{od} \textit{s/na} \\
\label{ex:klimekOrder} \hfill \citep[Polish;][]{KlimekJankowskaBlaszczak2022}
\z




\section{Corpus-study results}\label{sec:corpus}

In order to get better empirical insight into multiply prefixed verbs in Slovenian, we considered two sets of data. First, we looked at the 3000 most common verbs in Slovenian using the \textit{WeSoSlav} database \citep[see][]{mar+:WeSoSlaV}, to explore the behavior of common verbs with more than one prefix in general (assuming that such a 3000-verb sample is representative of the language). In the second step we created a list of multiply prefixed verbs from the list of all verbs occurring in the \textit{Gigafida 2.0} reference corpus of written standard Slovenian \citep{gigafida_glagoli}.

Starting with \textit{WeSoSlav}, while we were able to confirm that multiple prefixation exists, we found that only 6 out of 3000 verbs had 3 prefixes (no verbs have more), 178 verbs had 2 prefixes, while 2,076 had a single prefix.\footnote{The 6 verbs with three prefixes include two aspectual pairs (i.e. \textit{s-po-raz-umeti} `to agree/\-com\-mu\-nicate.\textsc{pfv}', \textit{s-po-raz-umevati} `to agree/communicate.\textsc{ipfv}' and \textit{s-po-pri-jeti} `to cope/deal with.\textsc{pfv}' \textit{s-po-pri-jemati} `cope/deal with.\textsc{ipfv}') so that there are really only 4 different verbs with three prefixes. Applying this same aspectual-pair exclusion criterium also to verbs with two and with one prefix, there are only around 125 different verbs with two prefixes and around 1500 different verbs with a single prefix. \label{fn:asppair}} \tabref{tab:prefixWeSoSlav} gives the relevant results.\footnote{Verbs that have a non-Slavic prefix like \textit{re-} in \textit{re-organizirati} `to reorganize' or \textit{dis-} in \textit{dis-kvalificirati} `to disqualify' were counted as unprefixed. Similarly we also disregarded the negative prefix \textit{ne-}, as in \textit{o-ne-sposobiti} `to disable'.} Note that each verb was counted only once (that is, verbs with three prefixes were not counted also as verbs with one prefix and as verbs with two prefixes).

\begin{table}[ht]
\centering 
\begin{tabular}{ l r r }
 \lsptoprule
 { number of prefixes}  &  {number of verbs}  &  {percent} \\
 \midrule
zero & 740 & 24.67\% \\ 
 (exactly) one  & 2,076 & 69.2\% \\
 (exactly) two & 178 & 5.93\% \\
 (exactly) three & 6 & 0.2\% \\
 \midrule
 \textsc{Total} & 3,000 & \\ 
 \lspbottomrule
\end{tabular}
\caption{Prefixation in WeSoSlav \citep{mar+:WeSoSlaV}}
    \label{tab:prefixWeSoSlav}
\end{table}

This data leads us to certain conclusions. On the one hand, prefixed verbs are more common than verbs without prefixes (the latter are not necessarily simplex, since some have a suffix, e.g. \textit{kup-ova-ti} `to buy.\textsc{ipfv}'). But more importantly, while verbs with a single prefix are extremely common, multiple prefixation is not. Given the relatively low number of multiply prefixed verbs, no proper quantificational analysis of the relative order of prefixes can be conducted. In order to create a better empirical base for investigating multiple prefixation, we created a larger list of multiply prefixed verbs.


\subsection{Corpus-study results, additional data}\label{sec:corpusResults1}

The new set of data was created from the list of all 90,000+ verbs found in the \textit{Gigafida 2.0} corpus \citep{gigafida_glagoli}. We only looked at verbs that had more than 5 occurrences in the corpus as the number of typos, misspelled words and incorrectly classified words only increases with less frequent strings of characters. Prefixed verbs were automatically extracted from the list using a simple formula that looked at each individual verb and checked whether it begins with one of the prefixes. The prefix was subtracted from the verb and the verb was checked again if the remaining part of the verb starts with one of the listed prefixes. This procedure was repeated five times. The automatically extracted multiply prefixed verbs were then also checked manually, since in some cases the automatic procedure counted some beginnings of stems/roots as prefixes, as in the case of verbs like \textit{stati} (incorrectly analyzed as \textit{s-tati}) `to stand' or \textit{vleči} `to pull' (incorrectly analyzed as \textit{v-leči}), and some combinations of prefixes could be misparsed as combinations of different prefixes, e.g. \textit{pod-o-}... `under-about-...', which is string-homophonous with \textit{po-do-}... `over-to-...', etc. 

\begin{table}
\centering 
\begin{tabular}{ l r r  }
 \lsptoprule
 { number of prefixes}  &  {number of verbs}  &  {percent} \\
 \midrule
zero & 4,186 & 29.45\% \\ 
 one  & 9,181 & 64.58\% \\
 two & 833 & 5.86\% \\
 three & 16 & 0.11\% \\
 \midrule
 \textsc{Total} & 14,216 & \\ 
 \lspbottomrule
\end{tabular}
\caption{Prefixation in the expanded database}
    \label{tab:prefixExtraWeSoSlav}
\end{table}

With this procedure we were able to retrieve a list of 849 multiply-prefixed verbs that exhibit at least 5 occurrences in the corpus. As above, the list contains some aspectual pairs, see footnote \ref{fn:asppair}, but we did not exclude aspectual pairs for the figures we made. Verbs with three prefixes are extremely rare in Slovenian (see \sectref{sec:3prefixes}), and among the verbs with at least 5 occurrences in the corpus, there were no verbs with more than three prefixes. 

In \figref{mar+:fig:1} the prefixes are ordered on the basis of their likelihood, increasing from left to right, to appear as the prefix closest to the verb. The first thing to note is that no prefix is restricted to the root-adjacent position: in the presented set of verbs they all appear in the first position of a pair of prefixes at least once.

\begin{figure}    
\centering
\includegraphics[width=\linewidth]{figures/mar-fig1_chart.pdf}
\caption{The frequency of prefixes relative to their position in a multiply prefixed verb (counting tokens of combinations)}
\label{mar+:fig:1}
\end{figure}

This last observation is very clearly visible also from \figref{fig:3}. Even the prefixes \textit{pod}- `under-' and \textit{vz-} `up-', which can be, based on \citet{sekli2016pomeni}, taken as essentially exclusively lexical prefixes in Slovenian, appear stacked over another prefix in up to 20\% of the cases. Actually, even the prefixes which seem to be most common in the root-adjacent position (\textit{vz-} `up-', \textit{v-} `in-', \textit{ob-} `around-', \textit{pod-} `under-' according to \figref{mar+:fig:1} and \figref{fig:3}) also appear stacked over another prefix in at least 10\% of the cases. Thus, all prefixes that are possible in the root-adjacent position can also be used as stacked prefixes (cf. \citealt{lazinski2011} for a similar dictionary-based result from Polish) and thus -- according to the description so far -- as superlexical prefixes. The implication does not go both ways, as \textit{so-} `co-' is never used as verb-adjacent in multiply prefixed verbs. 

\begin{figure}
\centering
\includegraphics[width=\linewidth]{figures/mar-fig2_chart.pdf}
\caption{Relative amount of prefixes that a prefix can appear with either when it comes first or second in a pair of prefixes (counting types of combinations).}
\label{fig:3}
\end{figure}

%\begin{figure}
%\centering
%\includegraphics[width=\linewidth]{table1_prosto.png}
%\end{figure}

\begin{table}
\scriptsize
\caption{The cross-table of prefix combinations. The first prefix of a pair is listed vertically, the second horizontally. The order of prefixes is a slightly modified sequence from \figref{mar+:fig:1}.}
\label{fig:table1}
\begin{tabularx}{\linewidth}{lXXXXXXXXXXXXXXXXXXX}
&so&	od	&pred&	raz&	iz&	za&	pre&	o&	pri&	s/z&	po&	na&	u&	v&	do&	vz&	pod&	pro&	ob\\ \hline
so&\cellcolor[gray]{0.9}	&\cellcolor{lime}	3&	\cellcolor{lime}2&	&	&	\cellcolor{yellow}1&	&	\cellcolor{lime}3&	\cellcolor{yellow}1&	\cellcolor{yellow}1&	\cellcolor{lime}4&	&	\cellcolor{lime}12&	\cellcolor{lime}3&	\cellcolor{lime}4&	&	\cellcolor{lime}2&	\cellcolor{yellow}1&	\cellcolor{lime}2\\
od&	&\cellcolor[gray]{0.9}	&	&	&	&	&	\cellcolor{yellow}1&	&	\cellcolor{yellow}1&	\cellcolor{lime}2&\cellcolor{lime}	9&	\cellcolor{yellow}1&	&	&	\cellcolor{yellow}1&	\cellcolor{lime}2&	&	\cellcolor{lime}2&	\\
pred&	&	&\cellcolor[gray]{0.9}	&	&	&	&	&	\cellcolor{yellow}1&	\cellcolor{yellow}1&	&	\cellcolor{lime}2&	\cellcolor{lime}3&	&	&	&	&	&	&	\\
raz&	&	&	&\cellcolor[gray]{0.9}	&	&	&	\cellcolor{lime}2&	\cellcolor{lime}3&	&	\cellcolor{lime}5&	\cellcolor{lime}15&	&	&	\cellcolor{lime}2&	&	&	\cellcolor{yellow}1&	\cellcolor{lime}7&	\cellcolor{lime}3\\
iz&	&	&	&	&\cellcolor[gray]{0.9}	&	&	\cellcolor{lime}6&	\cellcolor{lime}4&	&	\cellcolor{lime}4&	\cellcolor{lime}11&	\cellcolor{lime}4&	\cellcolor{yellow}1&	&	\cellcolor{yellow}1&	\cellcolor{lime}2&	\cellcolor{lime}14&	&	\cellcolor{lime}4\\
za&	&	&	\cellcolor{yellow}1&	&	&	\cellcolor[gray]{0.9}&	\cellcolor{lime}10&	\cellcolor{lime}9&	\cellcolor{lime}5&	\cellcolor{lime}127&	\cellcolor{lime}20&	&	\cellcolor{lime}7&	\cellcolor{lime}3&	\cellcolor{lime}8&	\cellcolor{lime}4&	&	\cellcolor{lime}3&	\cellcolor{lime}12\\
pre&	&	\cellcolor{yellow}1&	&	\cellcolor{lime}6&	\cellcolor{lime}5&	\cellcolor{lime}4&\cellcolor[gray]{0.9}	&	\cellcolor{lime}8&\cellcolor{lime}	5&\cellcolor{lime}	6&\cellcolor{lime}	10&\cellcolor{lime}	27&\cellcolor{lime}	14&\cellcolor{lime}	3&	\cellcolor{yellow}1&\cellcolor{lime}	4&	&\cellcolor{lime}	2&\cellcolor{lime}	12\\
o&	&	&	&	&	&	\cellcolor{yellow}1&\cellcolor{lime}	4&\cellcolor[gray]{0.9}	&\cellcolor{lime}	2&\cellcolor{lime}	4&\cellcolor{lime}	11&	&	&	\cellcolor{lime}2&	&	&	&	&	\\
pri&	&	&	&	&	&\cellcolor{lime}	5&	&	\cellcolor{yellow}1&	\cellcolor[gray]{0.9}&	\cellcolor{lime}14&\cellcolor{lime}	11&\cellcolor{yellow}1&	&	&	\cellcolor{lime}5&	\cellcolor{lime}7&	&	\cellcolor{yellow}1&\cellcolor{lime}	3\\
s/z&	&	&	\cellcolor{yellow}1&	&	&	\cellcolor{lime}2&\cellcolor{lime}	36&	&	\cellcolor{lime}5&\cellcolor[gray]{0.9}	1&\cellcolor{lime}	35&\cellcolor{lime}	2&	&	&	\cellcolor{yellow}1&	&\cellcolor{lime}	41&\cellcolor{lime}	8&\cellcolor{lime}	2\\
po&	&	\cellcolor{yellow}1&	&	\cellcolor{lime}9&	\cellcolor{lime}5&	\cellcolor{lime}5&	&	\cellcolor{lime}2&	\cellcolor{lime}6&	\cellcolor{lime}18&\cellcolor[gray]{0.9}	&	\cellcolor{lime}14&	\cellcolor{lime}10&	\cellcolor{lime}8&	\cellcolor{lime}2&	\cellcolor{lime}13&	&	\cellcolor{lime}2&	\cellcolor{lime}2\\
na&	&	&	&	&	&	\cellcolor{yellow}1&	\cellcolor{yellow}1&	&	&	\cellcolor{yellow}1&	\cellcolor{lime}5&\cellcolor[gray]{0.9}	&	\cellcolor{yellow}1&	\cellcolor{lime}7&	\cellcolor{lime}4&	\cellcolor{lime}2&	&	&	\cellcolor{yellow}1\\
u&	&	&	&	&	&	&	&	&	\cellcolor{lime}2&	\cellcolor{lime}5&	\cellcolor{lime}9&	&\cellcolor[gray]{0.9}	&	&	&	&	&	\cellcolor{lime}2&	\cellcolor{yellow}1\\
v&	&	&	&	&	&	&	&	&	&	\cellcolor{lime}2&	\cellcolor{lime}5&	&	&\cellcolor[gray]{0.9}	&	\cellcolor{yellow}1&	&	&	&	\\
do&	&	&	&	&	\cellcolor{lime}2&	\cellcolor{yellow}1&	\cellcolor{yellow}1&	&	\cellcolor{lime}2&	\cellcolor{yellow}1&	\cellcolor{lime}2&	&	&	&\cellcolor[gray]{0.9}	&	&	&	&	\cellcolor{yellow}1\\
vz&	&	&	&	&	&	&	&	&	&	&	\cellcolor{lime}4&	&	&	& &\cellcolor[gray]{0.9}	&	\cellcolor{lime}5&	& \\	
pod&	&	&	\cellcolor{yellow}1&	\cellcolor{yellow}1&	&	&	&	\cellcolor{lime}2&	&	&	\cellcolor{yellow}1&	\cellcolor{lime}2&	\cellcolor{lime}2&	&	&\cellcolor{lime} 2 & \cellcolor[gray]{0.9}	&	&	\\
pro&	&	&	&	&	\cellcolor{lime}2&	&	&	&	&	&	&	&	&	&	\cellcolor{yellow}1&	\cellcolor{yellow}1&	&\cellcolor[gray]{0.9}	&	\\
ob&	&	&	&	\cellcolor{lime}2&	&	&	&	&	&	&	&	&	\cellcolor{yellow}1&	&	&	&	&	&\cellcolor[gray]{0.9}	\\
\hline
%	so	od	pred	raz	iz	za	pre	o	pri	s/z	po	na	u	v	do	vz	pod	pro	ob
\end{tabularx}
\end{table}



\tabref{fig:table1} confirms a tendency for a hierarchy, but does not confirm a true hierarchy. Most pairs of two prefixes only exist in one order, as is evident from the fact that the lower left half of \figref{fig:table1} has fewer cells filled in than the upper right half of the table.  \tabref{fig:table1} shows that many pairs of prefixes exist with both orders of prefixes, so for example, there are 10 different verbs with the sequence \textit{za}-\textit{pre}- `for-over', and 4 different verbs with this sequence reversed, i.e. \textit{pre}-\textit{za}-. Given that certain prefixes have more than one use, that is, that they can be either used as lexical or superlexical prefixes, one would need to determine case by case whether the second prefix of a sequence of two prefixes is indeed an instance of a superlexical prefix or a lexical prefix (which means coding your data on the basis of previous qualitative data analysis, which we wanted to avoid here as much as possible). Further, some prefixes have even more than one superlexical use \citep[cf.][]{wiland2012prefix,KlimekJankowskaBlaszczak2022}, so that they can appear in more than one position within the proposed hierarchy of superlexical prefixes. These two facts presumably explain why we find so many different combinations where both orders of the two prefixes are possible, and we can only conclude that automatic extraction of prefixes cannot produce a clear sequence of superlexical prefixes, and therefore none of the proposed orders can be either confirmed or rejected. 

If we assume that, generally, only the prefix closest to the root will potentially be a lexical prefix (see \sectref{sec:introduction}), we would need to look at verbs with at least three prefixes to be able to get a sequence of superlexical prefixes, but we only have 16 verbs with three prefixes to work with.
    

\subsection{Verbs with three prefixes}\label{sec:3prefixes}
Given that prefixes should be able to stack, and that quite some claims have been made on the basis of the possible and impossible ordering patterns in stacking, we expected that we will find substantial numbers of verbs with three or more prefixes. However, this prediction was not confirmed since out of 849 multiply-prefixed verbs no verb included more than three prefixes and only 16 included three prefixes. Specifically, a closer review of the 16 verbs showed that this number is actually even smaller, as ``deduplication" of aspectual pairs reduces the number to a mere 10 verbs, listed in \REF{ex:ppprediti}--\REF{ex:pppmeriti}.\footnote{Why are verbs with three or more prefixes so rare in actual language is a question we leave for future work. In discussing the rarity of some predicted orders of superlexical stacking, \citet[269]{markova2011nature} suggests that this might have to do with processing constraints.}

Moreover, even some of the 10 verbs in \REF{ex:ppprediti}--\REF{ex:pppmeriti} are odd-looking and unknown to us, such as \textit{priopoteči} in \REF{ex:pppteči}, but as these verbs' few occurrences in the corpus seem to exhibit similar uses, we did not exclude them manually.\footnote{The verb \textit{prisprehoditi} has 5 occurrences in \textit{Gigafida 2.0} and \textit{priopoteči} has 6 occurrences, and these 5/6 occurrences even include more than one example by the same author, so these are possibly forms that have been used/coined by two or three speakers. \textit{Posprehoditi} has 27 occurences and \textit{porazporediti} 30 occurrences in \textit{Gigafida 2.0}. With the exception of \textit{posprehoditi}, none of these are listed in any of the dictionaries available to us; the translations we provide for these verbs are thus our context- and form-based inferences.}\textsuperscript{,}\footnote{One could perhaps also exclude verbs with the prefix \textit{so-}  (similar to the English \textit{co-}), such as \REF{ex:ppprabiti} and \REF{ex:sopovzrociti}. This prefix behaves differently from other verbal prefixes in several respects, can also appear in non-verbal contexts, e.g. \textit{so-avtor} `co-author', and is consequently often not even included in works on verbal prefixation, e.g. \citet{Muha1993}.}

\ea \label{ex:ppprediti}
 \gll pre-raz-po-rediti\textsuperscript{PFV} -- %$\rightarrow$
 pre-raz-po-rejati\textsuperscript{IPFV} \\ 
over-from-over-order {} over-from-over-order\\ 
\glt `to rearrange’\\
\z 

\ea \label{ex:pppjeti}
\gll s-po-pri-jeti\textsuperscript{PFV} -- s-po-pri-jemati\textsuperscript{IPFV} \\ 
with-over-at-hold {} with-over-at-hold\\ 
\glt `to tackle’\\
\z 

\ea \label{ex:ppprniti}
 \gll s-pre-ob-rniti\textsuperscript{PFV} -- s-pre-ob-račati\textsuperscript{IPFV} \\ 
with-over-around-turn {} with-over-around-turn\\ 
\glt `to convert’\\
\z 

\ea \label{ex:ppprabiti}
 \gll so-u-po-rabiti\textsuperscript{PFV}  -- so-u-po-rabljati\textsuperscript{IPFV}  \\ 
co-in-over-use {} co-in-over-use\\ 
\glt `to co-use’\\
\z

\ea \label{ex:sopovzrociti}
 \gll so-po-vz-ročiti\textsuperscript{PFV}  -- so-po-vz-ročati\textsuperscript{IPFV}  \\ 
co-over-up-hand {} co-over-up-hand\\ 
\glt `to co-cause’\\
\z

\ea \label{ex:pppumeti}
 \gll s-po-raz-umeti\textsuperscript{PFV}   -- s-po-raz-umevati\textsuperscript{IPFV} \\ 
with-over-from-understand {} with-over-from-understand\\ 
\glt `to agree/communicate’\\
\z

\ea \label{ex:pppteči}
\gll pri-o-po-teči \\
at-around-over-run \\
\glt `to get somewhere staggering'
\z

\ea\label{ex:porazporediti} 
\gll po-raz-po-rediti\\
over-from-over-order \\
\glt `to distribute' 
\z

\ea\label{ex:posprehoditi}
\gll po-s-pre-hoditi\\
 over-with-over-walk\\
\glt `to take a brief walk'
\z 

\ea \label{ex:pppmeriti}
\gll pri-s-pre-hoditi\\
 at-with-over-walk\\
\glt `to get somewhere taking a walk'
\z


\noindent Ignoring the interpretation of individual prefixes, we can extract several partial orders of prefixes from the above examples. Partial orders are given in \REF{ex:prefixiVTriprefigiranihGlagolih}. Interestingly \textit{s/z}- `with-' and \textit{po-} `over' appear in both orders, which is not surprising if both \textit{po-} and \textit{s/z-} have more than one superlexical use and thus more than one position in the hierarchy of superlexical prefixes.

\ea \label{ex:prefixiVTriprefigiranihGlagolih}
so > u, po \\
s/z > po, pre > raz\\
pri > o, s/z \\
po > raz, s/z 
\z

\noindent But what seems to be going on is probably something else. A closer look at the verbs in \REF{ex:ppprediti}--\REF{ex:pppmeriti} reveals that actually none of them seems to have a sequence of two obvious superlexical prefixes, and that for some of them no prefix seems very much like a standard, VP-external-looking superlexical prefix. In verbs like \REF{ex:pppjeti} and \REF{ex:ppprniti} all three prefixes have some of the properties of lexical prefixes -- they affect the argument structure or have spatial PP meanings.

Automatic extraction of prefixes out of a list of verbs has limitations, and even though we were able to show that there is a tendency for a hierarchy, we did not arrive at a single order of superlexical prefixes; we were just able to show that there are certain prefixes that prefer to stay closer to the root and others that prefer to be further away, and that this preference is different for different prefixes, but different methods of establishing this preference gave different sequences of prefixes.

We will devote the remainder of this paper to the observed unexpected sequences of prefixes. As mentioned above, even the prefixes that have been suggested as being exclusively lexical appear in up to 20\% of cases as the first prefix in a sequence of prefixes. Consider the verb \textit{vz-po-staviti} `to set up'. The prefix \textit{vz}- generally has the meaning `up' and is rarely associated with an adverbial meaning (e.g. \textit{vz-ljubiti} `to start to love') that we generally expect with the outermost prefix of a verb with two prefixes -- certainly such a meaning is absent in \textit{vz-po-staviti}. Similarly, the inner prefix of \textit{vz-po-staviti}, as expected, has a meaning that can only be associated with a lexical prefix (`over'). This type of verbs -- which we will call \textit{vz-po-staviti}-type verbs -- is what we turn to in \sectref{sec:4unexpectedStacking}.  
     


\section{Examples with two seemingly lexical prefixes}\label{sec:4unexpectedStacking}
Considering the mainstream view in the literature on prefixation (\sectref{sec:introduction}), one expectation is that if a verb has two (or more) prefixes, at most one will tend to be a VP-internal, lexical prefix, while the rest will tend to be superlexical (or intermediate). However, our corpus study presented in \sectref{sec:corpus} turned up a sizeable number of multiply prefixed verbs in which the outermost prefix also contributes a typically lexical meaning (i.e., \textit{vz-po-staviti}-type verbs). Examples \REF{ex:klicati} to \REF{ex:umeti} give a sample of such verbs. These examples are presented here in triplets: The first form is the unprefixed version, the last is the relevant example with two prefixes, and the middle example is the form (which is always an attested form) with a single prefix. Example \REF{ex:umeti} stands out somewhat, as it has three seemingly lexical prefixes, and the version of the verb with just two is not attested in modern Slovenian (though it is attested in older versions of Slovenian).  We use a \textsuperscript{\#}hashtag to mark unprefixed forms that are unattested in modern standard Slovenian and also in many dialects, such as \textsuperscript{\#}\textit{staviti}, though they are attested in some present-day dialects of Slovenian, in closely related BCMS, or are historically attested. Note also that in \textit{vz-po-staviti}-type verbs, the verb with a single prefix always seems to exist, which makes these different from 
\citeauthor{zaucer2002role}'s  (\citeyear{zaucer2002role})
examples like \REF{ex:izpodrinit}, discussed in \citet{svenonius2004slavic}, in which the version with a single prefix is not attested.\footnote{Regarding \REF{ex:prijeti}: some varieties do exhibit a verb \textit{jeti}, but only with an aspectual meaning `to start'. While this is the same root, with the aspectual meaning having developed from the root's basic meaning `grab'/`hold' or `take' \citep{snoj2009}, it is not the root's meaning that the prefixed verb is based on, so we mark \textit{jeti} in \REF{ex:prijeti} with a hashtag.}


\ea \glll klicati | po-klicati | v-po-klicati\\
call  {} over-call  {} in-over-call\\
{`to call'} |  {`to call up'}  | {`to enlist'}\label{ex:klicati}\\
\z

\ea \glll \minsp{\textsuperscript{\#}} staviti | po-staviti | vz-po-staviti\\
 {} set {} on-set  {} up-on-set\\
 {} {}  | {`to set'}  | {`to set up/establish'}\label{ex:vzpostaviti}\\
\z 

\ea  \glll  \minsp{\textsuperscript{\#}} jeti | pri-jeti | o-pri-jeti\\
 {} grab  {}  at-grab  {}  around-at-grab\\
  {} {}   |  {`to grab'}  | {`to hold on to'}\label{ex:prijeti}\\
\z

\ea \glll \minsp{\textsuperscript{\#}} peti | vz-peti | po-vz-peti\\
 {} pull {}  up-pull  {}  on-up-pull\\
 {}  {}  | {`to climb'}  | {`to climb’}\label{ex:vzpeti}\\
\z

\ea \glll  \minsp{\textsuperscript{\#}} deti | o-deti | raz-o-deti\\
 {} put  {}  around-put  {}  from-around-put\\
 {}  {}  | {`to wrap'}  | {`to reveal’}  \\
\z

\ea \glll nesti | za-nesti | pri-za-nesti\\
carry  {}  behind-carry  {}  at-behind-carry\\
{`to carry'}  | {`to carry in'}  | {`to spare’}\label{ex:prizanesti}\\
\z

\ea \glll  \minsp{\textsuperscript{\#}} umeti | raz-umeti | \minsp{\textsuperscript{\#}} po-raz-umeti | s-po-raz-umeti\\
 {} get/understand  {}  apart-get  {} {} over-apart-get {}  with-over-apart-get\\
 {} {}  | {`to understand'}  | { } {}  | {`to agree’}\label{ex:umeti}\\
\z

\ea \glll  nesti |  pri-nesti | do-pri-nesti\\
carry  {} at-carry  {}  to-at-carry\\
{`to carry'}  | {`to bring'}  | {`to contribute’}\label{ex:nesti}\\
\z

\ea  \glll  riniti | \minsp{*} pod-riniti | iz-pod-riniti\\
push  {}  {} under-push  {} from-under-push\\
{`to push'}  |  {}  {}  | {`to push out’}\label{ex:izpodrinit}\\
\z

\noindent The meaning contribution of the outermost prefix suggests that these examples contain more than one lexical prefix. In \REF{ex:klicati} and \REF{ex:vzpostaviti} the addition of \textit{v-} and \textit{vz-}, respectively, leads to an idiosyncratic, or perhaps spatial meaning; in \REF{ex:prijeti} the prefix \textit{o-} adds a spatial meaning; in \REF{ex:vzpeti} the contribution of \textit{po-} is not very clear (little discernible meaning change compared to its singly prefixed input); in \REF{ex:prizanesti}, \textit{pri-} adds an idiosyncratic meaning; etc. This situation is suprising in view of the idea that lexical prefixes generally do not stack.

The question is, then, how these prefixes should be analyzed. Possible answers include: (i) they are, despite their meanings, VP-external superlexicals; (ii) they fall into one of the additional categories of prefixes described in the literature (cf. \citealt{babko2003perfectivity,tatevosov2008intermediate,markova2011nature}, etc.); (iii) they are indeed VP-internal lexicals, but can be stacked because some special conditions are met. The last option then further opens several possibilities that could be explored, such as the possibility that these examples, in a sense, only include one prefix (and the inner prefix is somehow incorporated into the root), or that these are in fact two prefixes which either appear in a double-VP structure with two independent ResultPhrase positions for lexical prefixes, that they are result modifiers, or that they even require a completely different approach, perhaps one in which all prefixation is introduced above the VP (cf. \citealt{biskup2023}). In what follows, we explore these options.                 

\subsection{Option 1: They are superlexical}\label{sec:theyaresuperlexical}
If the outer prefixes of the \textit{vz-po-staviti}-type verbs were instances of \textit{v}P-external, superlexical prefixes, then one would expect them to exhibit properties typical of superlexical prefixes. One such property is their placement and the ability to stack -- since they appear on top of a prefix they could, in principle, be taken as superlexical. 

However, there are arguments against this claim. Firstly, they do not carry typical superlexical, adverbial meanings. If we consider the verb \textit{pri-za-nesti} in \REF{ex:prizanesti}, adding the prefix \textit{pri-} results in an idiosyncratic meaning shift from `to carry in' to `to spare', which cannot be the result of one of the two possible adverbial readings that \textit{pri-} has, according to \citet{sekli2016pomeni}, namely, a delimitative or an inchoative reading, as in \textit{pri-preti} `open a little' and \textit{pri-žgati} `to light up', respectively.  

Also, superlexical prefixes are typically said not to allow secondary imperfectivization (see \sectref{sec:whatWeKnow}). Except for \textit{vpoklicati}\textsuperscript{PFV} `to conscript' in \REF{ex:klicati}, all other verbs given in \REF{ex:klicati}--\REF{ex:izpodrinit} have well-attested secondary imperfectives: \textit{vzpostavljati}\textsuperscript{IPFV} `to establish', \textit{oprijemati}\textsuperscript{IPFV} `to hold on to', \textit{povzpenjati}\textsuperscript{IPFV} `to climb',  \textit{razodevati}\textsuperscript{IPFV} `to reveal', \textit{prizanašati}\textsuperscript{IPFV} `to spare', \textit{sporazumevati}\textsuperscript{IPFV} `to communicate', \textit{spozabljati se}\textsuperscript{IPFV} `to forget oneself'.\footnote{The fact that at least for many speakers, \textit{vpoklicati}\textsuperscript{PFV} `to conscript' does not have a natural imperfective counterpart is not problematic, given that it is also not the case that every perfective verb with a single prefix has a secondary imperfective counterpart, e.g., \textit{za-bresti}\textsuperscript{PFV} `to get stuck' does not. In fact, the input of \textit{vpoklicati}\textsuperscript{PFV}, i.e., \textit{poklicati}\textsuperscript{PFV} `to call up', also does not have a secondary imperfective counterpart.} It should be emphasized that these do not seem to be cases of a prefix combining with an imperfective base -- if this were the case, the resulting verb should be, contrary to fact, perfective. Rather, the imperfectivized verbs match the meaning of the perfective form (except in aspect), suggesting that these are in fact imperfectivizations of the doubly prefixed verbs:

 \ea \label{ex:zanestipfv} 
 \ea[]{ 
\gll Veter je \minsp{\{} za-nesel\textsuperscript{PFV} / za-našal\textsuperscript{IPFV}\} listje na dvorišče. \\
wind \textsc{aux} {} behind-carry {} behind-carry leaves.\textsc{acc} on yard \\ 
\glt `The wind carried leaves to the yard.'}
\ex[*] {\gll Veter je \minsp{\{} pri-za-nesel\textsuperscript{PFV} / pri-za-našal\textsuperscript{IPFV}\} listje na dvorišče.  \\ 
 wind \textsc{aux} {} at-behind-carry {} at-behind-carry leaves.\textsc{acc} on yard\\
 \glt \textit{Literally}: `The wind spared leaves to the yard.'}
\z \z 


 \ea \label{ex:prizanestipfv} 
 \ea[*]{ 
\gll Sodišče ni \{ za-neslo\textsuperscript{PFV} / za-našalo\textsuperscript{IPFV}\} osumljencem. \\
court \textsc{neg.aux} {} behind-carry {} behind-carry suspects.\textsc{dat} \\ 
\glt \textit{Literally}: `The court didn't carry to the suspects.'}
\ex[] {\gll Sodišče ni \minsp{\{} pri-za-neslo\textsuperscript{PFV} / pri-za-našalo\textsuperscript{IPFV}\} kriminalcem. \\ 
 court \textsc{neg.aux} {} at-behind-carry {} at-behind-carry criminals\\
\glt `The court didn't spare the criminals.'}
\z \z 

\noindent And finally, according to \citet{svenonius2004slavic} superlexical prefixes normally do not appear in nominalizations, in particular root/zero nominalizations (cf. also \citealt{cahazikova2016}). While it should be noted that not all verbs in Slovenian derive root nominalizations, several of these \textit{vz-po-staviti}-type verbs do:

\ea 
\glll  iz-po-staviti | iz-po-stav-a (/ iz-po-stav-e)\\
 out-over-stand  {} out-over-stand-\textsc{f.sg.nom} {} out-over-stand-\textsc{f.sg.gen}\\
{`to single out'}  |   {`branch'} \label{ex:izpostava} \\
\z

\ea 
\glll  do-pri-nesti  |  do-pri-nos-$\emptyset$ (/ do-pri-nos-a)\\
to-at-carry {} to-at-carry-\textsc{m.sg.nom} {} to-at-carry-\textsc{m.sg.gen}\\
{`to contribute'} |  {`contribution'} \label{ex:doprinos} \\
\z

\ea 
\glll  za-pri-seči  |  za-pri-seg-a (/ za-pri-seg-e)\\
behind-at-reach {} behind-at-reach-\textsc{f.sg.nom} {} behind-at-reach-\textsc{f.sg.gen}\\
{`to pledge'} |  {`pledge'} \label{ex:zaprisega} \\
\z

\ea 
\glll  v-po-klicati  |  v-po-klic-$\emptyset$ (/ v-po-klic-a)\\
in-over-call {} in-over-call-\textsc{m.sg.nom} {} in-over-call-\textsc{m.sg.gen}\\
{`to call in, enlist'} |  {`conscription'} \label{ex:vpoklic} \\
\z

\noindent Root nominalizations are usually assumed not to contain structure above the VP, and following \citet{svenonius2004slavic}, the existence of root nominalizations can be taken as an argument that these prefixes are structurally similar to lexical prefixes, merged inside the verb phrase. 

The only reason to consider the outermost prefix in the verbs under discussion to be superlexical, then, would be their placement, whereas their other properties speak against their being superlexical. In what follows, we will therefore further explore the option that they are not superlexical. 

\subsection{Option 2: They are neither lexical nor superlexical}\label{sec:intermediate}
 
 Whereas a binary split into VP-internal lexical prefixes and a possibly internally diverse group of superlexical prefixes is the most common stance taken in the literature (present also in several cartography-like accounts such as \citealt{wiland2012prefix}), some authors have proposed systems with more than two circumscribed groups of prefixes. In this section, we consider whether the stacked prefix in our \textit{vz-po-staviti}-type verbs could belong to one of these additional classes, and conclude that it could not. Note that we will always leave the highest-merging prefix type of these systems out of the discussion: that the stacked prefix in our \textit{vz-po-staviti}-type verbs cannot be any of these highest merging types follows from the discussion in \sectref{sec:theyaresuperlexical}.
 

 \citet{tatevosov2008intermediate} analyzes lexical prefixes as merging in a result phrase inside the VP and superlexical prefixes as merging outside the vP. He suggests that between the lexical and the superlexical prefixes there is a third group -- intermediate prefixes, such as the Russian completive \textit{do-} -- which merges somewhere above the VP and below the superlexicals. 

 While \textit{vz-po-staviti}-type verbs share certain properties with verbs with intermediate prefixes (e.g. being able to be imperfectivized), they also have characteristics that set them apart. According to \citet{tatevosov2008intermediate}, intermediate prefixes (among other characteristics) yield compositional meanings and never influence argument structure. As we already saw in \sectref{sec:theyaresuperlexical}, the outermost prefix in \textit{vz-po-staviti}-type verbs can lead to non-compositional meanings, such as \textit{vz-} in \REF{ex:vzpostaviti} (`to set’ $>$ `to establish’) or \textit{pri-} in \REF{ex:prizanesti} (`to carry in’ $>$ `to spare’), which come with concomitant argument structure effects (shown with more detail in \sectref{sec:theyarelexical}). As was also already mentioned in \sectref{sec:theyaresuperlexical}, \textit{vz-po-staviti}-type verbs often serve as the basis for root nominalizations, as in \REF{ex:izpostava}, which following \citet{svenonius2004slavic} also suggests that their prefixes do not originate above the VP. We therefore conclude that our \textit{vz-po-staviti}-type verbs are not simply intermediate prefixes.\footnote{In the spirit of \citet{zaucer2013}, an argument could also be made on the basis of relative scope with respect to VP adverbials, the restitutive `again' and adverbs of completion, all of which scope over the outer prefix. For a demonstration of some of this, see \sectref{sec:resultmodifier} below.}

 In a similar vein, \citet{markova2011nature} presents an account in which lexical prefixes, which she merges inside the VP as head adjuncts to V\textsuperscript{0}, are joined by three groups: outer prefixes, which are above vP; higher inner prefixes, which originate between VP and \textit{v}P; and lower inner prefixes, which originate in a PathP complement to V\textsuperscript{0}.
 
 Given that \citeauthor{markova2011nature}'s  (\citeyear{markova2011nature}) higher inner prefixes are positionally the same as \citeauthor{tatevosov2008intermediate}'s  (\citeyear{tatevosov2008intermediate}) intermediate prefixes, the same arguments that we just presented against viewing the stacked prefix in \textit{vz-po-staviti}-type verbs as \citeauthor{tatevosov2008intermediate}'s intermediate prefixes will also apply to the possibility that these prefixes would be \citeauthor{markova2011nature}'s higher inner prefixes. At the same time, the stacked prefixes in \textit{vz-po-staviti}-type verbs will also not be \citeauthor{markova2011nature}'s lower inner prefixes, since she reserves this position for spatial and causative prefixes, whereas the stacked prefixes in a number of our \textit{vz-po-staviti}-type verbs are neither spatial nor causative: see again, for example, \REF{ex:prizanesti}. Also, \citeauthor{markova2011nature}'s lower inner prefixes cannot contribute idiosyncratic meanings, which she reserves for lexical prefixes, but the stacked prefixes in our \textit{vz-po-staviti}-type verbs can contribute idiosyncratic meanings.
 
 Note, however, that somewhat in passing, \citeauthor{markova2011nature} (\citeyear{markova2011nature}: 260) also mentions the possibility that a verb hosts two lexical prefixes, in a V\textsuperscript{0} combining two prefixes and a verb, that is, in a V\textsuperscript{0} to which two prefix heads have been adjoined. From what we can tell, this structure, which assumes the possibility for idiosyncratic meanings for both prefixes, can actually successfully derive our \textit{vz-po-staviti}-type verbs. Though \citeauthor{markova2011nature} does not mention this, her account probably also predicts the possibility that a verb hosts a lexical prefix as well as a stacked lower inner prefix, a structure that presumably can derive some of our \textit{vz-po-staviti}-type verbs. We return to this in \sectref{sec:4.2.1}.
 
 Another account that proposes more than two groups of prefixes was put forth in \citet{babko2003perfectivity}. As a version of the superlexical category, \citeauthor{babko2003perfectivity} has Aktionsart-prefixed verbs, in which the prefix merges outside the VP (for which see \sectref{sec:theyaresuperlexical}). In addition, she has lexically prefixed verbs, in which the prefix is adjoined to V\textsuperscript{0}, and resultatively prefixed verbs, in which the prefix (itself part of a complex head) is adjoined to V\textsuperscript{0}. As explained by \citeauthor{babko2003perfectivity} (\citeyear[27]{babko2003perfectivity}) herself, the semantics derived from those structures is such that double prefixation is only possible when a stacked prefix is an Aktionsart prefix (i.e., a superlexical prefix in the terminology from \sectref{sec:theyaresuperlexical}), while it actually prevents double prefixation with either two lexical prefixes, two resultative prefixes, or a combination of the two. So the stacked prefixes in our \textit{vz-po-staviti}-type verbs will clearly be neither the lexical nor the resultative prefixes of \citet{babko2003perfectivity}.
 
 Note, however, that as pointed out by a reviewer,  the account from \citet{babko2003perfectivity} is presumably not incompatible with the existence of stacked prefixes of the type of \textit{vz-po-staviti} if such stacked prefixes are analyzed as result modifiers in the sense of \citet{zaucer2013} (even though \citeauthor{babko2003perfectivity} herself does not discuss this type of data). This would be a version of the view that these stacked prefixes are VP-internal, lexical prefixes, which is the option we discuss next, having determined now that our \textit{vz-po-staviti}-type verbs can be neither superlexical nor intermediate, or something of the sort.
 
\subsection{Option 3: They are lexical}\label{sec:theyarelexical}
 If prefixes in \textit{vz-po-staviti}-type verbs are VP-internal lexical prefixes,  we expect them to exhibit properties typically ascribed to lexical prefixes. Again, an argument against such an analysis is that the prefixes under discussion stack, while for lexical prefixes it is assumed that they generally do not stack, see \sectref{sec:introduction} and \sectref{sec:whatWeKnow}. The explanation for this restriction is structural. Because lexical prefixes are generally assumed to be resultative and originate in a VP-internal Result Phrase [RP], as shown in \figref{fig:Svenon} (based on \citealt[(80)]{svenonius2004slavic}), and because verbal structure is assumed to be able to host only one result/one RP \citep{rappaport2001event, ramchand2008verb}, it should normally not be possible to have more than one lexical prefix per verb.\footnote{Though see \citet{den1995particles} for a different understanding of the structure used for particles and prefixes and the restrictions it imposes.}

\begin{figure}
\begin{forest}
sn edges/.style={for tree={s sep=10mm, inner sep=0, l=0}},
sn edges	[AspP
	[PP[v-,roof]][Asp' 
	[Asp\textsuperscript{0}] [VP
	[V\textsuperscript{0}\\grad-][RP 
	[DP [okno,roof]] [R'
	[R\textsuperscript{0}][t\textsubscript{PP}]]]]]]
\end{forest}
    \caption{Structure dictates the one-lexical-prefix restriction.}
    \label{fig:Svenon}
\end{figure}

However, as already indicated in \sectref{sec:theyaresuperlexical}, these prefixes display several other properties that can be taken as arguments for a VP-internal, lexical analysis. In addition to non-superlexical interpretations, the availability of secondary imperfectivization and root nominalizations, the outer prefixes in \textit{vz-po-staviti}-type verbs also exhibit some argument-structure effects.  

For example, the ``singly'' prefixed verb \textit{pri-jeti} `to grab' can select for a single accusative object, the reflexive clitic \textsc{se}, and an optional genitive object, or an optional reflexive clitic \textsc{se} and a prepositional phrase, as shown in \REF{ex:prijetiacc}. The ``doubly'' prefixed \textit{o-pri-jeti} `to hold on to', on the other hand, is unacceptable (in most modern varieties) with a single accusative object, requires the genitive object with a reflexive clitic \textsc{se}, and simply does not tolerate prepositional objects, as shown in \REF{ex:oprijetiacc}. Differences in the argument structure of the singly- and doubly-prefixed counterparts are observed also in other cases, as shown in \REF{ex:priseci}--\REF{ex:prizanesti2}.
  
  \begin{comment}
     \ea \ea \gll pri-jeti \minsp{\{} ročaj / se \minsp{(} ročaja) / \minsp{(} se) za ročaj\} \\
   {at}-grab {} handle.\textsc{acc} { } \textsc{refl} {} handle.\textsc{gen} { }   {}  \textsc{refl} for handle.\textsc{acc} \\
  \glt `to grab the handle/ to grab on (to the handle)'%\label{ex:prijetiacc}\\
  \ex   \gll o-pri-jeti \minsp{\{*} ročaj / se \minsp{*(} ročaja) / \minsp{*(} \minsp{*} se) za ročaj\} \\ 
     {around}-{at}-grab {} handle.\textsc{acc} { } \textsc{refl} {} handle.\textsc{gen} { }    {} {}  \textsc{refl}  for handle.\textsc{acc} {} \\
 \glt  `to grab on (to the handle)' %\label{ex:oprijetiacc}
\z \z
\end{comment}
\ea \label{ex:prijetiacc}  \ea  \gll pri-jeti  ročaj  \\
   {at}-grab  handle.\textsc{acc}\\
  \glt `to grab the handle'\\
  \ex   \gll pri-jeti  se \minsp{(} ročaja)  \\ 
     {at}-grab   \textsc{refl} {} handle.\textsc{gen} { }  \\
 \glt  `to grab (on to the handle)' \\
\ex   \gll pri-jeti \minsp{(} se) za ročaj \\ 
     {at}-grab {}  \textsc{refl}  for handle.\textsc{acc} {} \\
 \glt  `to grab on to the handle' 
\z \z


\ea \label{ex:oprijetiacc} \ea [*]{ \gll o-pri-jeti  ročaj  \\
   {around}-{at}-grab  handle.\textsc{acc}\\}
  \ex   \gll o-pri-jeti   se \minsp{*(} ročaja)  \\ 
     {around}-{at}-grab   \textsc{refl} {} handle.\textsc{gen} \\
 \glt  `to grab on to the handle' 
\ex  [*] {\gll o-pri-jeti \minsp{(} se) za ročaj\\ 
     {around}-{at}-grab  {} \textsc{refl}  for handle.\textsc{acc} {} \\}
\z \z 

   \ea \label{ex:priseci} \ea \gll pri-seči \minsp{(*} pričo)  \\
   {at}-reach {}  witness.\textsc{acc} \\
 \glt  `to swear, take an oath'\\
\ex \gll za-pri-seči \minsp{(} pričo) \\
   {behind}-{at}-reach {} witness.\textsc{acc}  \\
  \glt `to take an oath; to swear in a witness'\\
\z \z

\begin{comment}
  \ea %\label{ex:prizanesti2} 
   \ea \gll za-nesti \minsp{\{} skrbi Vidu / \minsp{*} Vidu \minsp{(} s skrbmi)\} \\
   behind-carry {} worries.\textsc{acc} Vid.\textsc{dat} {} {} Vid.\textsc{dat} {} with worries \\
 \glt  `to carry worries to Vid'\\
  \ex  \gll  pri-za-nesti \minsp{\{(*} skrbi) Vidu / Vidu \minsp{(} s skrbmi)\}\\
   at-behind-carry {} worries.\textsc{acc} Vid.\textsc{dat} {} Vid.\textsc{dat} {} with worries\\
 \glt  `to spare Vid the worries'\\
\z   \z
\end{comment}
\ea %\label{ex:prizanesti2} 
\ea \gll za-nesti  skrbi Vidu  \\
   behind-carry worries.\textsc{acc} Vid.\textsc{dat}   \\
 \glt  `to carry worries to Vid'\\
  \ex  [*] {\gll za-nesti Vidu \minsp{(} s skrbmi)\\
   at-behind-carry  Vid.\textsc{dat} {} with worries\\
 \glt  `to carry worries to Vid'\\}
\z   \z


\ea \label{ex:prizanesti2} \ea \gll pri-za-nesti \minsp{(*} skrbi) Vidu \\
    at-behind-carry {} worries.\textsc{acc} Vid.\textsc{dat}  \\
 \glt  `to spare Vid'\\
  \ex  \gll  pri-za-nesti  Vidu \minsp{(} s skrbmi)\\
   at-behind-carry  Vid.\textsc{dat} {} with worries\\
 \glt  `to spare Vid (the worries)'\\
\z   \z
  
\noindent Given that we seem to be led to the conclusion that the outer prefix in \textit{vz-po-staviti}-type verbs is a lexical prefix, it should be noted that different authors have previously observed that VP-internal prefixes are not a homogeneous group. A natural question to ask, then, is whether the outer prefixes in \textit{vz-po-staviti}-type verbs share any of the properties of those proposed subgroups.  



\subsubsection{Option 3.1: They are lexical -- but these verbs contain only one prefix}\label{sec:4.2.1}
This option presents itself as a possibility especially in view of the fact that some of these apparently doubly-prefixed verbs are no longer used without a prefix. For example, while \REF{ex:staviti} exists in some Slovenian dialects (and in BCMS), it does not exist in standard Slovenian, nor in many other dialects that normally use \textit{vz-po-staviti}. Similarly, \REF{ex:peti} does not exist in modern Slovenian (though it does exist in BCMS), and neither does \REF{ex:jeti}.

\ea \label{ex:staviti} \textsuperscript{\#}\textit{staviti} `set' (exists in some Western Slovenian dialects) 
\z

\ea \label{ex:peti} \textsuperscript{\#}\textit{peti} `pull' (but exists in BCMS) 
\z

\ea \label{ex:jeti}*\textit{jeti} `grab'/`hold'
\z

\noindent Given that these simplex forms are not attested (or are at best very limited) synchronically, it could be the case that the innermost prefix, even if historically a prefix, is just a part of the root \citep[cf.][]{fowler1996}, or in other terms, as suggested in \citeauthor{markova2011nature} (\citeyear{markova2011nature}: 260) for all prefixes resulting in idiosyncratic meaning shifts, is adjoined to V\textsuperscript{0}, forming a complex verbal head. According to this analysis, a verb can have more than one lexical/X\textsuperscript{0}-adjoined prefix, and since prefixes are adjoined to \textit{v}\textsuperscript{0}, they are freely ordered.  

 On the one hand, it seems to us that \citeauthor{markova2011nature}'s proposal could be seen as consistent with \textit{vz-po-staviti}-type verbs, especially for those built on verbs like \textit{po-staviti} `to set' or \textit{pri-jeti} `to grab', whose unprefixed bases are not attested syncronically, as well as for those whose outer prefix seems somehow related to a spatial use, such as in \textit{v-po-klicati} `to enlist'. On the other hand, for a number of \textit{vz-po-staviti}-type verbs aspect presents an issue. Several of these verbs, such as \textit{vz-peti} `to climb.\textsc{pfv}', are based on stems that were historically imperfective, and just like most lexically prefixed verbs (and unlike most native unprefixed verbs), these verbs generally form secondary imperfectives, e.g. \textit{po-stavljati} `to stand.\textsc{ipfv}', \textit{vz-penjati} `to climb.\textsc{ipfv}', \textit{pri-jemati} `to hold.\textsc{ipfv}'. This suggests that these inner prefixes trigger perfectivity. It is unclear to us how such adjunction could account for the change of aspect. In \citeposst{svenonius2004slavic} account, for example, the perfectivizing effect arises when a prefix moves from the RP into a VP-external aspect projection; if the prefix is part of a complex V\textsuperscript{0}, such movement does not seem to be possible. For those \textit{vz-po-staviti}-type verbs which exhibit singly-prefixed counterparts even in modern Slovenian, such as \textit{v-po-klicati} `to enlist' or \textit{za-pri-seči} `to take an oath, to swear somebody in', this aspectual concern regarding treating their inner prefix as V\textsuperscript{0}-adjoined is even more obvious. 
 
 In addition, whereas some of these \textit{vz-po-staviti}-type verbs synchronically do not exhibit unprefixed versions, they do occur in a modern Slovenian with several different prefixes, \REF{ex:manystav}--\REF{ex:manyjeti}, resulting in forms with either clearly related or with idiosyncratic meanings. We can take this as an argument against an analysis on which the innermost prefixes are simply part of the root: While we agree with \citet{Romanova2004}, who considers similar examples of ``cranberry roots" in Russian, that these roots are light (according to \citeauthor{Romanova2004} they can have no semantics at all), a comparison of the same root with different prefixes implies some common meaning (for \REF{ex:manystav}, this could be paraphrased as `to place') while the prefixes add a predictable spatial meaning. 

\ea \label{ex:manystav}\glll  \textit{na-staviti} | \textit{po-staviti} | \textit{v-staviti} | \textit{pre-staviti} | \textit{do-staviti} | \textit{od-staviti} ...\\
{on}-set {} {over}-set {}  {in}-set {} over-set {} to-set {} from-set\\
`set' | `set' |   `insert' | `move' | `deliver' | {`remove'}\\
\z

\ea \label{ex:manypeti} \glll \textit{na-peti} | \textit{vz-peti} | \textit{v-peti} | \textit{raz-peti} | \textit{pri-peti} | \textit{od-peti} ...\\
{on}-pull  {} {up}-pull  {}  {in}-pull  {} {apart}-pull {} {at}-pull {} from-pull \\
`stretch'/`string'  | `climb'  |  `fasten' | `spread' | `attach' |  `detach'\\
\z

\ea \label{ex:manyjeti}\glll \textit{na-jeti} | \textit{pri-jeti} |  \textit{za-jeti} |  \textit{ob-jeti}  |    \textit{vz-eti} ... \\
{on}-grab {} {at}-grab  {} {behind}-grab {}  {around}-grab {}    {up}-grab\\
`hire' | `grab' | `scoop' | `hug' | `take' \\
\z

 \noindent And finally, assuming that the forms in \REF{ex:manystav}--\REF{ex:manyjeti} are unprefixed poses a problem for the varieties in which the simplex forms of the verbs in \REF{ex:manystav}--\REF{ex:manyjeti} do exist, and it also does not account for those \textit{vz-po-staviti}-type verbs that are perfectly normally attested both in standard Slovenian and across Slovenian dialects without the prefix (e.g.,  \textit{klicati} `to call', the root of the doubly prefixed verb \textit{v-po-klicati} `to enlist'). We thus conclude that despite some merits, \citeauthor{markova2011nature}'s account falls short of fully explaining our \textit{vz-po-staviti}-type verbs. 
 
  
 
\subsubsection{Option 3.2: They are lexical -- but these verbs have two VPs ($=$double resultative structure)}
As mentioned in \sectref{sec:theyarelexical}, the restriction to a single lexical prefix per verb has been derived as a consequence of the structural position of lexical prefixes; because the clausal structure can only have one RP, there can normally only be one lexical prefix per verb phrase (and consequently per verb). However, \citet{zaucer2009vp} discusses a class of verbs in Slovenian that seem to have two resultative prefixes, and ultimately analyzes these as having a double-VP structure (cf. also \citealt{tatevosov2022}). In the discussion of the cumulative (/accumulative/saturative) prefix \textit{na}-, a crucial piece of support for the double-VP structure is argued to be the two sets of unselected objects, \REF{ex:doublep} and \REF{ex:maradona}. 

\ea \gll \minsp{*(} pre)-igrati\textsuperscript{PFV} Maradono \label{ex:doublep}\\
{} over-play 	Maradona.\textsc{acc}\\
\glt `fake out Maradona’
\z

\ea \gll \minsp{*(} na)-*(pre)-igravati\textsuperscript{PFV/IPFV} se	Maradone \label{ex:maradona}\\
{} on-over-play 	\textsc{refl} 	Maradona.\textsc{gen}\\ 
\glt `get / getting one’s fill of faking out Maradona'
\z

\noindent As is evident from our examples in \sectref{sec:theyarelexical}, the \textit{vz-po-staviti}-type verbs do not behave like this. They do not appear to introduce two unselected objects.

Furthermore, the outermost prefix in \REF{ex:maradona} and this type of examples require an imperfective input, which is not the case in \textit{vz-po-staviti}-type verbs. Also, \REF{ex:maradona} and this type of examples are normally read perfectively, with the outermost prefix there triggering perfectivity; in other words, an example such as \REF{ex:maradona} does not necessarily get an imperfective reading despite the presence of the imperfective suffix -\textit{ava}. At the same time, though, the imperfective affix \textit{can} be interpreted as scoping over the outermost prefix -- in this case the interpretation of \REF{ex:maradona} is `getting one’s fill of faking out Maradona'. Unlike \REF{ex:maradona}, and as shown in \REF{ex:nedovrsnostdvojnih}, the outermost prefix of \textit{vz-po-staviti}-type verbs never perfectivizes its input and the imperfective affix always scopes over the outermost prefix, which further means that the whole verb is interpreted as imperfective.

\ea \label{ex:nedovrsnostdvojnih}
\ea \glll pri-jeti\textsuperscript{PFV} -- pri-jemati\textsuperscript{IPFV}  | o-pri-jeti\textsuperscript{PFV} -- o-pri-jemati\textsuperscript{IPFV}\\
{at}-grab {} {at}-grab.\textsc{si} {}	{around}-{at}-grab {}	{around}-{at}-grab.\textsc{si}\\ 
{`to grab'} {} {} | {'to grab on to'} {} {}\\
\ex \glll pri-nesti\textsuperscript{PFV} -- pri-našati\textsuperscript{IPFV} | do-pri-nesti\textsuperscript{PFV} -- do-pri-našati\textsuperscript{IPFV} \\
{at}-carry {} {at}-carry.\textsc{si} {} {to}-{at}-carry {} {to}-{at}-carry.\textsc{si}\\ 
{`to carry to'} {} {} {} {`to contribute'} {} {}\\
\ex \glll po-staviti\textsuperscript{PFV} -- po-stavljati\textsuperscript{IPFV} | iz-po-staviti\textsuperscript{PFV} -- iz-po-stavljati\textsuperscript{IPFV} \\
{over}-stand {} {over}-stand.\textsc{si} {} {out}-{over}-stand {} {out}-{over}-stand.\textsc{si}\\ 
{`to set'} {} {} | {`to single out'} {} {}\\
\z
\z

\noindent While  \citet{zaucer2009vp} discusses other properties of examples that can be analysed as including two VPs, we take these differences as evidence enough to conclude that prefixes in \textit{vz-po-staviti}-type verbs are not similar to the cumulative \textit{na-}.
      

\subsubsection{Option 3.3: They are lexical -- result modifiers, not main result predicates}\label{sec:resultmodifier}
The literature has identified one further group of prefixes that does not fully respect the standard division into lexical and superlexical. As discussed by \citet{zaucer2013}, prefixes such as excessive (\textit{pre-}), repetitive (\textit{pre-}), attenuative (\textit{pri-}, \textit{po-}), and distributive (\textit{po-}) have adverbial, superlexical-like meanings, can stack, and do not affect argument structure at least when stacked, which makes them look like ordinary superlexical prefixes. An example of this type of prefix is given in \REF{ex:prefix_pre}.

\ea \gll pre-na-polniti\\
{over}-{on}-fill\\
\glt `overfill’ \label{ex:prefix_pre}
\z

\noindent However, \citet{zaucer2013} argues, contrary to what would be expected given the properties listed above, that these prefixes nevertheless merge VP-internally, supporting this claim, for example, with the fact that they scope below VP-adverbials, as shown in \REF{ex:preustekleniciti}. The proposed analysis is that these prefixes are result modifiers, thus a sort of adverbial prefixes, but ones that modify the result phrase directly, before it is merged together with the verb.\footnote{As already mentioned, this is a possibility not considered by \citet{babko2003perfectivity}, whose analysis explicitly rules out stacked lexicals and resultatives, but it is, as pointed out to us by a reviewer, a possibility that is in fact perfectly compatible with that system.}

\ea \label{ex:preustekleniciti} \gll U-stekleničil sem tole vino sicer na roke, pre-u-stekleničil ga bom pa z mašinco.\\
 {in}-bottled \textsc{aux} this wine \textsc{ptcl} on hand {over}-{in}-bottled it will \textsc{ptcl} with machine\\
\glt `Though I bottled this wine manually, I'll re-bottle it with a machine.' \\ \strut \hfill \citep[292]{zaucer2013}
\z

\noindent What \REF{ex:preustekleniciti} says is that the first time the wine was bottled it was bottled manually, while the second time it was bottled this was done with the use of a machine, which indicates that the repetitive \textit{pre-} is inside the scope of the `with'-adverbial, which, in turn, means that \textit{pre}- does not originate above the VP. 

Interestingly, the same scopal facts can be observed with \textit{vz-po-staviti}-type verbs. As shown in \REF{ex:oprijetiZrokavico} the entire verb \textit{oprijeti} `to hold on to' is in the scope of the `with'-adverbial, suggesting that all parts of the verb originate VP-internally.
 
\ea \gll Vejo sem sicer pri-jel z roko, o-pri-jel se je bom pa z rokavico. \\
 branch \textsc{aux} \textsc{ptcl} {at}-hold with hand {around}-{at}-hold \textsc{refl} it \textsc{aux} \textsc{ptcl} with glove\\
 \glt `I grabbed the branch with my hand, but I'll hold on to it with a glove.' \label{ex:oprijetiZrokavico}
\z 

\noindent The two sets of prefixes also behave the same with respect to the restitutive reading of \textit{spet} `again'. That is, both the excessive/measure prefix in \REF{ex:prespet} and the  outer prefix in \textit{vz-po-staviti} `establish' in \REF{ex:spetvzpostaviti} take narrow scope with respect to the restitutive reading of \textit{spet} `again'.


\ea \label{ex:prespet}\gll Juš je hladilnik spet pre-na-polnil.\\
Juš \textsc{aux} fridge again {over-on}-filled\\
\glt `Juš restored the fridge to an overfilled state.'\\
\textit{Not}: Juš was overly involved in filling up the fridge. \hfill \citep[293]{zaucer2013} 
\z

\ea \label{ex:spetvzpostaviti}\gll Miha je stike z očetom spet vz-po-stavil.\\
Miha \textsc{aux} contacts with father again {up-over}-set\\
\glt `Miha restored contacts with his father.'\\
(No other interpretation.)
\z

\noindent  While \citeauthor{zaucer2013}'s (\citeyear{zaucer2013}) result-modifying prefixes have a predictable adverbial interpretation and the outer-most prefixes in \textit{vz-po-staviti}-type verbs do not seem to, both of these types of prefixes behave comparably with respect to scopal tests, suggesting that they share the same structural position.\footnote{\citet{zaucer2013} does not discuss nominalization possibilities, but root nominalizations from verbs with those result-modifying prefixes are not difficult to find, e.g, \textit{pri-vz-dig} `a partial lift', \textit{pre-u-stroj} `remodeling', \textit{pre-u-redba} `reorganization'. The same holds also of our \textit{vz-po-staviti}-type verbs, cf. \REF{ex:izpostava}--\REF{ex:vpoklic} above.} 


\subsubsection{Option 3.4: They are lexical and parallel to particles}
It is well known that there exist parallels between Germanic particles and Slavic prefixes, e.g. \citet{spencer&zaretskaya1998}, \citet{svenonius2004slavic}.
In fact, similarly to doubly-prefixed verbs of the \textit{vz-po-staviti}-type verbs in Slovenian, we can also observe particle recursion in Germanic, see for example \citet[80]{den1995particles}. \citet{den1995particles} claims that particle recursion is structurally
possible but, for unclear reasons, rare. He analyzes recursive particles using his basic structural template from \figref{fig:dendikken} by simply having the second particle as the head of XP, as in \figref{fig:dendikken2}.   
    
\ea I'll send the letter on over to Grandma's
    house. \\ \strut \hfill \citet[(116b)]{den1995particles}, quoting \citet{di1994modifying} \label{ex:onover}
\z

\begin{figure}
\begin{forest}
%sn edges/.style={for tree={
%parent anchor=south, child anchor=north}},
sn edges/.style={for tree={s sep=10mm, inner sep=0, l=0}},
sn edges
    [VP [V] [SC\textsubscript{1} 
    [\textsc{Spec$\theta$} [\textit{ec} ]] 
    [PP [Prt] 
    [SC\textsubscript{2} [ NP ] [XP ] ] ] ] ]  
\end{forest}
    \caption{The basic structural template of \citet{den1995particles}}
    \label{fig:dendikken}
\end{figure}  

\begin{figure}
\begin{forest}
%sn edges/.style={for tree={
%parent anchor=south, child anchor=north}},
sn edges/.style={for tree={s sep=10mm, inner sep=0, l=0}},
sn edges
    [VP [V] [SC\textsubscript{1} 
    [\textsc{Spec$\theta$} [\textit{ec} ]] 
    [PP [Prt\textsubscript{1}] 
    [SC\textsubscript{2} [NP, name=NP1 ] [{XP$=$PP} [Prt\textsubscript{2}] [SC\textsubscript{3} [NP, name= NP2] [XP] ] ] ] ] ] ] 
    \draw[->] (NP2.west) to[out=west, in=south] (NP1.south);
\end{forest}
    \caption{Using \citeauthor{den1995particles}'s (\citeyear{den1995particles}) basic structural template to explain unexpected multiple prefixation}
    \label{fig:dendikken2}
\end{figure}  

\subsubsection{Option 3.5: They are some of the lowest projections above VP}
 
There is yet another set of accounts that we have not discussed, namely, accounts that merge all prefixes, including lexical ones, outside the VP. One part of these accounts is represented by systems which at least implicitly still subscribe to two groups, lexical prefixes and a group of higher prefixes, with a single slot for lexical prefixes (e.g. \citealt{slabakova2005}, \citealt{istratkova2006}, \citealt{wiland2012prefix}); like the accounts discussed above, with lexical prefixes originating VP-internally, these accounts thus generally also end up with a restriction to a single lexical prefix. In addition, it is also not clear to us that such systems can really explain argument structure effects of lexical prefixes well, cf. \citeauthor{zaucer2009vp} (\citeyear{zaucer2009vp}: 16--18). Most recently, \citet{biskup2023} also develops a system with all prefixes merged outside the VP, but his version presumably allows more flexibility than the previous all-prefixes-outside-the-VP accounts as it does not really seem to subscribe to two groups, and it does not limit the number of lexical prefixes structurally but rather by appealing to conceptual reasons; for a similar case as our \textit{vz-po-staviti}-type verbs, it explicitly allows two lexical prefixes hosted in two separate internal-prefix phrases above the VP. The approach looks promising to us for approaching our \textit{vz-po-staviti}-type verbs, however, in addition to the concern regarding argument-structure effects already stated above, it is also not clear to us -- assuming a universal clausal spine -- what the nature of the lexical-prefix projections introducing the multiple lexical prefixes could be, and why they could be freely remergeable.     


\subsection{Instead of a conclusion---a partial proposal}\label{sec:4.3_proposal}
We have shown that the outer prefixes in \textit{vz-po-staviti}-type verbs, even though they are stacked on top of another prefix, do not behave like other superlexical prefixes but rather much more like VP-internal, lexical prefixes. \tabref{tab:LexAndSuperLexVzpostaviti} presents a comparison of our \textit{vz-po-staviti}-type verbs, or rather, their outer prefixes, lexical prefixes, superlexical prefixes and result-modifying prefixes on the basis of the six most typically considered properties. Some of these properties are clearly related to one another, so for example, a prefix's VP-internal position is related to its ability to form a secondary imperfective, which is merged outside the VP and thus scopes over it. Similarly, as already explained in \sectref{sec:theyarelexical}, placing lexical prefixes in a dedicated VP-internal Result Phrase means that a verb should not host a stack of such prefixes. Additionally, idiosyncratic meaning and argument-structure effects of lexical prefixes also seem to be related to their position inside the VP.

\begin{table}
\fittable{
\begin{tabular}{l l l l l}
 \lsptoprule
 &  {Lexical }  &  {\textsc{vz-po-staviti}} &  {result mod.}&  {Superlex. } \\
 \midrule
VP-positioning&  internal  & internal & internal & external   \\
meaning & idiosyn./spati. & idiosyn./spati.  & adverbial & adverbial \\ 
affect arg. struct. & Yes & Yes & No & No  \\
 form sec. imperf. & Yes & Yes & Yes & No   \\
form root nomin. & Yes & Yes & Yes & No \\ 
stacking &  No & Yes & Yes & Yes  \\  
 \lspbottomrule
\end{tabular}
}
    \caption{Lexical, superlexical, and other types of prefixes}
    \label{tab:LexAndSuperLexVzpostaviti}
\end{table}

So far we mentioned 12 different \textit{vz-po-staviti}-type verbs that used 10 different prefixes as the outer prefix. Most likely, then, the outer prefixes of \textit{vz-po-staviti}-type verbs do not form a homogeneous class of prefixes, so we actually need not expect to find a single explanation for all of them. 

The type of verbs that had been discussed by \citet{zaucer2002role} and \citet{svenonius2004slavic}, \textit{iz-pod-riniti} `to push out' and \textit{s-pod-makniti} `to jerk away', are probably just instances of a complex prefix which realizes both \textsc{Path} and \textsc{Place} parts of the preposition phrase inside a single result phrase, as suggested by \citet{svenonius2004slavic}.\footnote{The two combinations \textit{iz-pod-} and \textit{s-pod-} are synonymous. One can find both versions of these two verbs in written Slovenian -- \textit{iz-pod-riniti} and \textit{s-pod-riniti} both with the same meaning `to push out' and likewise \textit{s-pod-makniti} and \textit{iz-pod-makniti} both meaning `to jerk away'. Spoken Slovenian hardly makes a distinction between the two pronunciations of these two forms, so we are treating them as just two realizations of the same lexical unit.}

Some prefixes have a relatively clear spatial meaning, such as \textit{o-} in \textit{o-pri-jeti} `hold on to', which is comparable in meaning to verbs where \textit{o-} is more clearly lexical like \textit{o-kleniti} `grab on to', \textit{o-graditi} `to put a fence around', or \textit{o-črtati} `to draw a line around' (in some cases the (core) spatial meaning got obscured by a more metaphorical interpretation) and \textit{v-} in \textit{v-po-klicati} `to enlist', which can even be doubled by a preposition phrase with the same prefix, as in \REF{ex:vpoklicativVreprezentanco}.

\ea \gll Trener ga je v-po-klical v reprezentanco.\\
coach him \textsc{aux} in-over-call in national-team\\
\glt `The coach called him up into the national team.' \label{ex:vpoklicativVreprezentanco}
\z

In cases like these, the outer prefix may seem to be a proper lexical prefix that would require a result phrase of its own, which would mean that we need two RPs inside the VP, which seems like a problem -- but cf. \citet{markova2011nature} and \citet{biskup2023}. Note that even though these verbs have a different argument structure from their unprefixed counterpart, the contribution of the prefix to the argument-structure change is not very clear, suggesting that potentially one of the two prefixes can receive an alternative interpretation.

In many respects, our \textit{vz-po-staviti}-type verbs seem  to behave similarly to doubly-prefixed verbs in which the prefixes are ``result modifiers", the main difference being the interpretation of prefixes/prefixed verbs -- while the ``result modifiers" in \citet{zaucer2013} have a clear adverbial reading, prefixes in \textit{vz-po-staviti}-type verbs lead to anything between a slight modification in the interpretation of the input to a full-scale idiosyncratic meaning shift compared to the input. Despite this, we propose that the prefixes in \textit{vz-po-staviti}-type verbs should be subsumable under a result-modifier analysis.\footnote{One could say that just like standard lexical prefixes, which sometimes contribute a compositional spatial interpretation and sometimes a non-compositional idiosyncratic interpretation, result-modifying prefixes also have these two options: contributing either a compositional adverbial interpretation or a non-compositional idiosyncratic interpretation, which we observed with many \textit{vz-po-staviti} type verbs.}

Based on \citet{zaucer2013}, we thus propose that the structure in \figref{fig:Svenondva2} captures the two positions for the prefixes in \textit{vz-po-staviti}-type verbs. Note that the result-modifying prefix (on its own) here cannot introduce an unselected object (perhaps unlike the structure in \figref{fig:dendikken2}).

\begin{figure}
\begin{forest}
sn edges/.style={for tree={
parent anchor=south, child anchor=north}},
sn edges/.style={for tree={s sep=10mm, inner sep=0, l=0}},
sn edges
    [VP
	[\textcolor{white}{specifier-blabla}][PP/XP
	[prefix\textsubscript{ResultModifier}][PP
	[prefix\textsubscript{Resultative}] [complement]]]]
\end{forest}
    \caption{The structure with the two positions of the two prefixes of \textit{vz-po-staviti}-type verbs}
    \label{fig:Svenondva2}
\end{figure}
    
           
\section{Conclusions}\label{mar+:sec:conclusion}

 Our corpus data show that even prefixes which have been claimed to serve (almost) exclusively as lexical prefixes appear stacked over another prefix in up to 20\% of their occurrences, which ultimately means that no prefix is used exclusively as a lexical prefix, or that lexical prefixes can sometimes also stack. Our corpus data also confirms a tendency for a hierarchy, but as multiple  prefixes have more than one use and since all of them can be used either as lexical or as superlexical prefixes and can appear in more than one position, a true hierarchy of superlexical prefixes could only be determined, perhaps, if prefix occurrences were coded for specific prefix uses -- a task that unfortunately seems quite unrealistic, but also one that would inevitably end up drawing in individual researcher's subjective decisions. Our corpus study also showed that whereas prefixed verbs are very common in Slovenian, verbs with stacked prefixes are very rare, all in all making the use of corpora rather poorly suited for investigating prefix stacking options in Slovenian.


On the other hand, our corpus investigation also turned up a sizeable set of verbs with two prefixes in which the outer prefix does not seem to have any of the typical superlexical characteristics, other than the fact that it occurs stacked over another prefix. Zooming in on these verbs, which we called \textit{vz-po-staviti}-type verbs, we compared their outer prefixes to superlexical prefixes, to intermediate (and other types of in-between) prefixes, and to some types of stacked prefixes that had previously been proposed to instantiate lexical prefixes despite being stacked. We argued that both the inner and the outer prefix in \textit{vz-po-staviti}-type verbs are lexical and cannot be explained away easily. We found that the outer prefixes in these verbs do not seem to form a homogeneous class, and so it is quite likely that it need not be just one explanation that will solve all of these examples. Some of the discussed cases can be explained relatively easily, and at least for a large part of them they seem best treated as (a version of) result-modifying prefixes, though some cases may need alternative approaches, which we leave for future research.



\section*{Abbreviations}

\begin{tabularx}{.5\textwidth}{@{}lQ}
\textsc{acc}&accusative\\
\textsc{att}&attenuative\\
\textsc{aux}&auxiliary\\
BCMS & Bosnian/Croatian/ Montenegrin/Serbian\\
\textsc{compl}&completive\\
\textsc{cuml}&cumulative\\
\textsc{dat}&dative\\
\textsc{delim}&delimitative\\
\textsc{dist}&distributive\\
\textsc{exc}&excessive\\
\textsc{f}&feminine\\
\textsc{gen}&genitive\\
%\textsc{inf}&infinitive\\
\textsc{incp}&inceptive\\
\end{tabularx}%
\begin{tabularx}{.5\textwidth}{lQ@{}}
\textsc{ipfv}&imperfective\\
\textsc{m}&masculine\\
\textsc{nom}&nominative\\
\textsc{neg}&negation\\
\textsc{perd}&perdurative\\
\textsc{pfv}&perfective\\
%\textsc{prs}&present tense\\
\textsc{ptcl}&particle\\
\textsc{refl}&reflexive\\
\textsc{rep}&repetitive\\
\textsc{sat}&saturative\\
\textsc{sg}&singular\\
\textsc{si} & secondary imperfective\\
\textsc{term}&terminative\\
\textsc{tv}&thematic vowel\\
%&\\ % this dummy row achieves correct vertical alignment of both tables
\end{tabularx}

\section*{Acknowledgments}
We are very grateful to two anonymous reviewers for extensive and helpful comments on the first version of this paper. Work on this paper was supported by the Slovenian Research and Innovation Agency (ARIS) grants J6-4614, N6-0113 and P6-0382. All authors contributed equally to this paper and are listed in alphabetical order.


%\bibliography{localbibliography}%dodano

\printbibliography[heading=subbibliography,notkeyword=this]

\end{document}

%\bibliography{localbibliography}%dodano






\iffalse


\documentclass[output=paper,colorlinks,citecolor=brown]{langscibook}
\bibliography{localbibliography}

\author{Radek Šimík\orcid{0000-0002-4736-195X}\affiliation{Charles University} and Berit Gehrke\orcid{0000-0002-6315-9532}\affiliation{Humboldt-Universität zu Berlin} and Denisa Lenertová\orcid{}\affiliation{Humboldt-Universität zu Berlin} and Roland Meyer\orcid{0000-0003-2023-0527}\affiliation{Humboldt-Universität zu Berlin} and Luka Szucsich\orcid{}\affiliation{Humboldt-Universität zu Berlin} and Joanna Zaleska\orcid{0000-0002-0059-6938}\affiliation{Humboldt-Universität zu Berlin}}
% replace the above with you and your coauthors
% rules for affiliation: If there's an official English version, use that (find out on the official website of the university); if not, use the original
% orcid doesn't appear printed; it's metainformation used for later indexing

%%% uncomment the following line if you are a single author or all authors have the same affiliation
% \SetupAffiliations{mark style=none}

%% in case the running head with authors exceeds one line (which is the case in this example document), use one of the following methods to turn it into a single line; otherwise comment the line below out with % and ignore it
\lehead{Šimík, Gehrke, Lenertová, Meyer, Szucsich \& Zaleska}
% \lehead{Radek Šimík et al.}

\title{Open Slavic Linguistics style guidelines}
% replace the above with your paper title
%%% provide a shorter version of your title in case it doesn't fit a single line in the running head
% in this form: \title[short title]{full title}
\abstract{Abstract goes here and should not have more than 150 words.

\keywords{here, come, your, 4 to 6 keywords or keyphrases}
}

\usepackage{langsci-optional}
\usepackage{langsci-gb4e}
\usepackage{langsci-lgr}

\usepackage{listings}
\lstset{basicstyle=\ttfamily,tabsize=2,breaklines=true}

%added by author
% \usepackage{tipa}
\usepackage{multirow}
\graphicspath{{figures/}}
\usepackage{langsci-branding}


\newcommand{\sent}{\enumsentence}
\newcommand{\sents}{\eenumsentence}
\let\citeasnoun\citet

\renewcommand{\lsCoverTitleFont}[1]{\sffamily\addfontfeatures{Scale=MatchUppercase}\fontsize{44pt}{16mm}\selectfont #1}
  

\togglepaper[42]
% the chapter number will be provided by volume editors; for now keep this way

\begin{document}
\maketitle

% Just comment out the input below when you're ready to go.
For a start: Do not forget to give your Overleaf project (this paper) a recognizable name. This one could be called, for instance, Simik et al: OSL template. You can change the name of the project by hovering over the gray title at the top of this page and clicking on the pencil icon.

\section{Introduction}\label{sim:sec:intro}

Language Science Press is a project run for linguists, but also by linguists. You are part of that and we rely on your collaboration to get at the desired result. Publishing with LangSci Press might mean a bit more work for the author (and for the volume editor), esp. for the less experienced ones, but it also gives you much more control of the process and it is rewarding to see the quality result.

Please follow the instructions below closely, it will save the volume editors, the series editors, and you alike a lot of time.

\sloppy This stylesheet is a further specification of three more general sources: (i) the Leipzig glossing rules \citep{leipzig-glossing-rules}, (ii) the generic style rules for linguistics (\url{https://www.eva.mpg.de/fileadmin/content_files/staff/haspelmt/pdf/GenericStyleRules.pdf}), and (iii) the Language Science Press guidelines \citep{Nordhoff.Muller2021}.\footnote{Notice the way in-text numbered lists should be written -- using small Roman numbers enclosed in brackets.} It is advisable to go through these before you start writing. Most of the general rules are not repeated here.\footnote{Do not worry about the colors of references and links. They are there to make the editorial process easier and will disappear prior to official publication.}

Please spend some time reading through these and the more general instructions. Your 30 minutes on this is likely to save you and us hours of additional work. Do not hesitate to contact the editors if you have any questions.

\section{Illustrating OSL commands and conventions}\label{sim:sec:osl-comm}

Below I illustrate the use of a number of commands defined in langsci-osl.tex (see the styles folder).

\subsection{Typesetting semantics}\label{sim:sec:sem}

See below for some examples of how to typeset semantic formulas. The examples also show the use of the sib-command (= ``semantic interpretation brackets''). Notice also the the use of the dummy curly brackets in \REF{sim:ex:quant}. They ensure that the spacing around the equation symbol is correct. 

\ea \ea \sib{dog}$^g=\textsc{dog}=\lambda x[\textsc{dog}(x)]$\label{sim:ex:dog}
\ex \sib{Some dog bit every boy}${}=\exists x[\textsc{dog}(x)\wedge\forall y[\textsc{boy}(y)\rightarrow \textsc{bit}(x,y)]]$\label{sim:ex:quant}
\z\z

\noindent Use noindent after example environments (but not after floats like tables or figures).

And here's a macro for semantic type brackets: The expression \textit{dog} is of type $\stb{e,t}$. Don't forget to place the whole type formula into a math-environment. An example of a more complex type, such as the one of \textit{every}: $\stb{s,\stb{\stb{e,t},\stb{e,t}}}$. You can of course also use the type in a subscript: dog$_{\stb{e,t}}$

We distinguish between metalinguistic constants that are translations of object language, which are typeset using small caps, see \REF{sim:ex:dog}, and logical constants. See the contrast in \REF{sim:ex:speaker}, where \textsc{speaker} (= serif) in \REF{sim:ex:speaker-a} is the denotation of the word \textit{speaker}, and \cnst{speaker} (= sans-serif) in \REF{sim:ex:speaker-b} is the function that maps the context $c$ to the speaker in that context.\footnote{Notice that both types of small caps are automatically turned into text-style, even if used in a math-environment. This enables you to use math throughout.}$^,$\footnote{Notice also that the notation entails the ``direct translation'' system from natural language to metalanguage, as entertained e.g. in \citet{Heim.Kratzer1998}. Feel free to devise your own notation when relying on the ``indirect translation'' system (see, e.g., \citealt{Coppock.Champollion2022}).}

\ea\label{sim:ex:speaker}
\ea \sib{The speaker is drunk}$^{g,c}=\textsc{drunk}\big(\iota x\,\textsc{speaker}(x)\big)$\label{sim:ex:speaker-a}
\ex \sib{I am drunk}$^{g,c}=\textsc{drunk}\big(\cnst{speaker}(c)\big)$\label{sim:ex:speaker-b}
\z\z

\noindent Notice that with more complex formulas, you can use bigger brackets indicating scope, cf. $($ vs. $\big($, as used in \REF{sim:ex:speaker}. Also notice the use of backslash plus comma, which produces additional space in math-environment.

\subsection{Examples and the minsp command}

Try to keep examples simple. But if you need to pack more information into an example or include more alternatives, you can resort to various brackets or slashes. For that, you will find the minsp-command useful. It works as follows:

\ea\label{sim:ex:german-verbs}\gll Hans \minsp{\{} schläft / schlief / \minsp{*} schlafen\}.\\
Hans {} sleeps {} slept {} {} sleep.\textsc{inf}\\
\glt `Hans \{sleeps / slept\}.'
\z

\noindent If you use the command, glosses will be aligned with the corresponding object language elements correctly. Notice also that brackets etc. do not receive their own gloss. Simply use closed curly brackets as the placeholder.

The minsp-command is not needed for grammaticality judgments used for the whole sentence. For that, use the native langsci-gb4e method instead, as illustrated below:

\ea[*]{\gll Das sein ungrammatisch.\\
that be.\textsc{inf} ungrammatical\\
\glt Intended: `This is ungrammatical.'\hfill (German)\label{sim:ex:ungram}}
\z

\noindent Also notice that translations should never be ungrammatical. If the original is ungrammatical, provide the intended interpretation in idiomatic English.

If you want to indicate the language and/or the source of the example, place this on the right margin of the translation line. Schematic information about relevant linguistic properties of the examples should be placed on the line of the example, as indicated below.

\ea\label{sim:ex:bailyn}\gll \minsp{[} Ėtu knigu] čitaet Ivan \minsp{(} často).\\
{} this book.{\ACC} read.{\PRS.3\SG} Ivan.{\NOM} {} often\\\hfill O-V-S-Adv
\glt `Ivan reads this book (often).'\hfill (Russian; \citealt[4]{Bailyn2004})
\z

\noindent Finally, notice that you can use the gloss macros for typing grammatical glosses, defined in langsci-lgr.sty. Place curly brackets around them.

\subsection{Citation commands and macros}

You can make your life easier if you use the following citation commands and macros (see code):

\begin{itemize}
    \item \citealt{Bailyn2004}: no brackets
    \item \citet{Bailyn2004}: year in brackets
    \item \citep{Bailyn2004}: everything in brackets
    \item \citepossalt{Bailyn2004}: possessive
    \item \citeposst{Bailyn2004}: possessive with year in brackets
\end{itemize}

\section{Trees}\label{s:tree}

Use the forest package for trees and place trees in a figure environment. \figref{sim:fig:CP} shows a simple example.\footnote{See \citet{VandenWyngaerd2017} for a simple and useful quickstart guide for the forest package.} Notice that figure (and table) environments are so-called floating environments. {\LaTeX} determines the position of the figure/table on the page, so it can appear elsewhere than where it appears in the code. This is not a bug, it is a property. Also for this reason, do not refer to figures/tables by using phrases like ``the table below''. Always use tabref/figref. If your terminal nodes represent object language, then these should essentially correspond to glosses, not to the original. For this reason, we recommend including an explicit example which corresponds to the tree, in this particular case \REF{sim:ex:czech-for-tree}.

\ea\label{sim:ex:czech-for-tree}\gll Co se řidič snažil dělat?\\
what {\REFL} driver try.{\PTCP.\SG.\MASC} do.{\INF}\\
\glt `What did the driver try to do?'
\z

\begin{figure}[ht]
% the [ht] option means that you prefer the placement of the figure HERE (=h) and if HERE is not possible, you prefer the TOP (=t) of a page
% \centering
    \begin{forest}
    for tree={s sep=1cm, inner sep=0, l=0}
    [CP
        [DP
            [what, roof, name=what]
        ]
        [C$'$
            [C
                [\textsc{refl}]
            ]
            [TP
                [DP
                    [driver, roof]
                ]
                [T$'$
                    [T [{[past]}]]
                    [VP
                        [V
                            [tried]
                        ]
                        [VP, s sep=2.2cm
                            [V
                                [do.\textsc{inf}]
                            ]
                            [t\textsubscript{what}, name=trace-what]
                        ]
                    ]
                ]
            ]
        ]
    ]
    \draw[->,overlay] (trace-what) to[out=south west, in=south, looseness=1.1] (what);
    % the overlay option avoids making the bounding box of the tree too large
    % the looseness option defines the looseness of the arrow (default = 1)
    \end{forest}
    \vspace{3ex} % extra vspace is added here because the arrow goes too deep to the caption; avoid such manual tweaking as much as possible; here it's necessary
    \caption{Proposed syntactic representation of \REF{sim:ex:czech-for-tree}}
    \label{sim:fig:CP}
\end{figure}

Do not use noindent after figures or tables (as you do after examples). Cases like these (where the noindent ends up missing) will be handled by the editors prior to publication.

\section{Italics, boldface, small caps, underlining, quotes}

See \citet{Nordhoff.Muller2021} for that. In short:

\begin{itemize}
    \item No boldface anywhere.
    \item No underlining anywhere (unless for very specific and well-defined technical notation; consult with editors).
    \item Small caps used for (i) introducing terms that are important for the paper (small-cap the term just ones, at a place where it is characterized/defined); (ii) metalinguistic translations of object-language expressions in semantic formulas (see \sectref{sim:sec:sem}); (iii) selected technical notions.
    \item Italics for object-language within text; exceptionally for emphasis/contrast.
    \item Single quotes: for translations/interpretations
    \item Double quotes: scare quotes; quotations of chunks of text.
\end{itemize}

\section{Cross-referencing}

Label examples, sections, tables, figures, possibly footnotes (by using the label macro). The name of the label is up to you, but it is good practice to follow this template: article-code:reference-type:unique-label. E.g. sim:ex:german would be a proper name for a reference within this paper (sim = short for the author(s); ex = example reference; german = unique name of that example).

\section{Syntactic notation}

Syntactic categories (N, D, V, etc.) are written with initial capital letters. This also holds for categories named with multiple letters, e.g. Foc, Top, Num, etc. Stick to this convention also when coming up with ad hoc categories, e.g. Cl (for clitic or classifier).

An exception from this rule are ``little'' categories, which are written with italics: \textit{v}, \textit{n}, \textit{v}P, etc.

Bar-levels must be typeset with bars/primes, not with an apostrophe. An easy way to do that is to use mathmode for the bar: C$'$, Foc$'$, etc.

Specifiers should be written this way: SpecCP, Spec\textit{v}P.

Features should be surrounded by square brackets, e.g., [past]. If you use plus and minus, be sure that these actually are plus and minus, and not e.g. a hyphen. Mathmode can help with that: [$+$sg], [$-$sg], [$\pm$sg]. See \sectref{sim:sec:hyphens-etc} for related information.

\section{Footnotes}

Absolutely avoid long footnotes. A footnote should not be longer than, say, {20\%} of the page. If you feel like you need a long footnote, make an explicit digression in the main body of the text.

Footnotes should always be placed at the end of whole sentences. Formulate the footnote in such a way that this is possible. Footnotes should always go after punctuation marks, never before. Do not place footnotes after individual words. Do not place footnotes in examples, tables, etc. If you have an urge to do that, place the footnote to the text that explains the example, table, etc.

Footnotes should always be formulated as full, self-standing sentences.

\section{Tables}

For your tables use the table plus tabularx environments. The tabularx environment lets you (and requires you in fact) to specify the width of the table and defines the X column (left-alignment) and the Y column (right-alignment). All X/Y columns will have the same width and together they will fill out the width of the rest of the table -- counting out all non-X/Y columns.

Always include a meaningful caption. The caption is designed to appear on top of the table, no matter where you place it in the code. Do not try to tweak with this. Tables are delimited with lsptoprule at the top and lspbottomrule at the bottom. The header is delimited from the rest with midrule. Vertical lines in tables are banned. An example is provided in \tabref{sim:tab:frequencies}. See \citet{Nordhoff.Muller2021} for more information. If you are typesetting a very complex table or your table is too large to fit the page, do not hesitate to ask the editors for help.

\begin{table}
\caption{Frequencies of word classes}
\label{sim:tab:frequencies}
 \begin{tabularx}{.77\textwidth}{lYYYY} %.77 indicates that the table will take up 77% of the textwidth
  \lsptoprule
            & nouns & verbs  & adjectives & adverbs\\
  \midrule
  absolute  &   12  &    34  &    23      & 13\\
  relative  &   3.1 &   8.9  &    5.7     & 3.2\\
  \lspbottomrule
 \end{tabularx}
\end{table}

\section{Figures}

Figures must have a good quality. If you use pictorial figures, consult the editors early on to see if the quality and format of your figure is sufficient. If you use simple barplots, you can use the barplot environment (defined in langsci-osl.sty). See \figref{sim:fig:barplot} for an example. The barplot environment has 5 arguments: 1. x-axis desription, 2. y-axis description, 3. width (relative to textwidth), 4. x-tick descriptions, 5. x-ticks plus y-values.

\begin{figure}
    \centering
    \barplot{Type of meal}{Times selected}{0.6}{Bread,Soup,Pizza}%
    {
    (Bread,61)
    (Soup,12)
    (Pizza,8)
    }
    \caption{A barplot example}
    \label{sim:fig:barplot}
\end{figure}

The barplot environment builds on the tikzpicture plus axis environments of the pgfplots package. It can be customized in various ways. \figref{sim:fig:complex-barplot} shows a more complex example.

\begin{figure}
  \begin{tikzpicture}
    \begin{axis}[
	xlabel={Level of \textsc{uniq/max}},  
	ylabel={Proportion of $\textsf{subj}\prec\textsf{pred}$}, 
	axis lines*=left, 
        width  = .6\textwidth,
	height = 5cm,
    	nodes near coords, 
    % 	nodes near coords style={text=black},
    	every node near coord/.append style={font=\tiny},
        nodes near coords align={vertical},
	ymin=0,
	ymax=1,
	ytick distance=.2,
	xtick=data,
	ylabel near ticks,
	x tick label style={font=\sffamily},
	ybar=5pt,
	legend pos=outer north east,
	enlarge x limits=0.3,
	symbolic x coords={+u/m, \textminus u/m},
	]
	\addplot[fill=red!30,draw=none] coordinates {
	    (+u/m,0.91)
        (\textminus u/m,0.84)
	};
	\addplot[fill=red,draw=none] coordinates {
	    (+u/m,0.80)
        (\textminus u/m,0.87)
	};
	\legend{\textsf{sg}, \textsf{pl}}
    \end{axis} 
  \end{tikzpicture} 
    \caption{Results divided by \textsc{number}}
    \label{sim:fig:complex-barplot}
\end{figure}

\section{Hyphens, dashes, minuses, math/logical operators}\label{sim:sec:hyphens-etc}

Be careful to distinguish between hyphens (-), dashes (--), and the minus sign ($-$). For in-text appositions, use only en-dashes -- as done here -- with spaces around. Do not use em-dashes (---). Using mathmode is a reliable way of getting the minus sign.

All equations (and typically also semantic formulas, see \sectref{sim:sec:sem}) should be typeset using mathmode. Notice that mathmode not only gets the math signs ``right'', but also has a dedicated spacing. For that reason, never write things like p$<$0.05, p $<$ 0.05, or p$<0.05$, but rather $p<0.05$. In case you need a two-place math or logical operator (like $\wedge$) but for some reason do not have one of the arguments represented overtly, you can use a ``dummy'' argument (curly brackets) to simulate the presence of the other one. Notice the difference between $\wedge p$ and ${}\wedge p$.

In case you need to use normal text within mathmode, use the text command. Here is an example: $\text{frequency}=.8$. This way, you get the math spacing right.

\section{Abbreviations}

The final abbreviations section should include all glosses. It should not include other ad hoc abbreviations (those should be defined upon first use) and also not standard abbreviations like NP, VP, etc.


\section{Bibliography}

Place your bibliography into localbibliography.bib. Important: Only place there the entries which you actually cite! You can make use of our OSL bibliography, which we keep clean and tidy and update it after the publication of each new volume. Contact the editors of your volume if you do not have the bib file yet. If you find the entry you need, just copy-paste it in your localbibliography.bib. The bibliography also shows many good examples of what a good bibliographic entry should look like.

See \citet{Nordhoff.Muller2021} for general information on bibliography. Some important things to keep in mind:

\begin{itemize}
    \item Journals should be cited as they are officially called (notice the difference between and, \&, capitalization, etc.).
    \item Journal publications should always include the volume number, the issue number (field ``number''), and DOI or stable URL (see below on that).
    \item Papers in collections or proceedings must include the editors of the volume (field ``editor''), the place of publication (field ``address'') and publisher.
    \item The proceedings number is part of the title of the proceedings. Do not place it into the ``volume'' field. The ``volume'' field with book/proceedings publications is reserved for the volume of that single book (e.g. NELS 40 proceedings might have vol. 1 and vol. 2).
    \item Avoid citing manuscripts as much as possible. If you need to cite them, try to provide a stable URL.
    \item Avoid citing presentations or talks. If you absolutely must cite them, be careful not to refer the reader to them by using ``see...''. The reader can't see them.
    \item If you cite a manuscript, presentation, or some other hard-to-define source, use the either the ``misc'' or ``unpublished'' entry type. The former is appropriate if the text cited corresponds to a book (the title will be printed in italics); the latter is appropriate if the text cited corresponds to an article or presentation (the title will be printed normally). Within both entries, use the ``howpublished'' field for any relevant information (such as ``Manuscript, University of \dots''). And the ``url'' field for the URL.
\end{itemize}

We require the authors to provide DOIs or URLs wherever possible, though not without limitations. The following rules apply:

\begin{itemize}
    \item If the publication has a DOI, use that. Use the ``doi'' field and write just the DOI, not the whole URL.
    \item If the publication has no DOI, but it has a stable URL (as e.g. JSTOR, but possibly also lingbuzz), use that. Place it in the ``url'' field, using the full address (https: etc.).
    \item Never use DOI and URL at the same time.
    \item If the official publication has no official DOI or stable URL (related to the official publication), do not replace these with other links. Do not refer to published works with lingbuzz links, for instance, as these typically lead to the unpublished (preprint) version. (There are exceptions where lingbuzz or semanticsarchive are the official publication venue, in which case these can of course be used.) Never use URLs leading to personal websites.
    \item If a paper has no DOI/URL, but the book does, do not use the book URL. Just use nothing.
\end{itemize}

\section*{Abbreviations}

\begin{tabularx}{.5\textwidth}{@{}lQ}
\textsc{3}&third person\\
\textsc{acc}&accusative\\
\textsc{inf}&infinitive\\
\textsc{m}&masculine\\
\textsc{nom}&nominative\\
\end{tabularx}%
\begin{tabularx}{.5\textwidth}{lQ@{}}
\textsc{prs}&present tense\\
\textsc{ptcp}&participle\\
\textsc{refl}&reflexive\\
\textsc{sg}&singular\\
&\\ % this dummy row achieves correct vertical alignment of both tables
\end{tabularx}

\section*{Acknowledgments}
Place your acknowledgements here and funding information here.

\printbibliography[heading=subbibliography,notkeyword=this]

\end{document}

\fi
