\documentclass[output=paper,colorlinks,citecolor=black]{langscibook}
\ChapterDOI{10.5281/zenodo.15394193}
%\bibliography{localbibliography}

\author{Margje Post\orcid{0000-0003-3959-5905}\affiliation{University of Bergen}}
% replace the above with you and your coauthors
% rules for affiliation: If there's an official English version, use that (find out on the official website of the university); if not, use the original
% orcid doesn't appear printed; it's metainformation used for later indexing

%%% uncomment the following line if you are a single author or all authors have the same affiliation
\SetupAffiliations{mark style=none}

%% in case the running head with authors exceeds one line (which is the case in this example document), use one of the following methods to turn it into a single line; otherwise comment the line below out with % and ignore it
%\lehead{Author}
%\lehead{Radek Šimík et al.}

\title{Word prosodic structure and vowel reduction in Moscow and Perm Russian}
% replace the above with your paper title
%%% provide a shorter version of your title in case it doesn't fit a single line in the running head
% in this form: \title[short title]{full title}
\abstract{Central Standard Russian is well-known for its vowel reduction in two degrees: the immediately pretonic vowel is much less reduced than vowels in other unstressed positions, both in quality and in quantity, at least when the allophone is a low vowel. This two-degree reduction is expressed clearly in speech from Central Russia, but earlier studies suggest a smaller difference between the degrees in non-central areas. We measured vowel duration and quality of unstressed /a,~o/ in two modern urban Russian varieties: in read speech from 26 adolescents in Moscow (Central Russia) and Perm (Ural region). The Moscow speakers make a sharp distinction between the two degrees in both quantity (duration) and quality (F1), but we found only small, not statistically significant differences in Perm. Perm speech might lack phonological two-degree reduction altogether, in which case two-degree reduction is not a general feature of modern Russian urban speech.

\keywords{Russian, prosody, vowel reduction, regional variation}
}

\begin{document}
\maketitle

% Just comment out the input below when you're ready to go.
%For a start: Do not forget to give your Overleaf project (this paper) a recognizable name. This one could be called, for instance, Simik et al: OSL template. You can change the name of the project by hovering over the gray title at the top of this page and clicking on the pencil icon.

\section{Introduction}\label{sim:sec:intro}

Language Science Press is a project run for linguists, but also by linguists. You are part of that and we rely on your collaboration to get at the desired result. Publishing with LangSci Press might mean a bit more work for the author (and for the volume editor), esp. for the less experienced ones, but it also gives you much more control of the process and it is rewarding to see the quality result.

Please follow the instructions below closely, it will save the volume editors, the series editors, and you alike a lot of time.

\sloppy This stylesheet is a further specification of three more general sources: (i) the Leipzig glossing rules \citep{leipzig-glossing-rules}, (ii) the generic style rules for linguistics (\url{https://www.eva.mpg.de/fileadmin/content_files/staff/haspelmt/pdf/GenericStyleRules.pdf}), and (iii) the Language Science Press guidelines \citep{Nordhoff.Muller2021}.\footnote{Notice the way in-text numbered lists should be written -- using small Roman numbers enclosed in brackets.} It is advisable to go through these before you start writing. Most of the general rules are not repeated here.\footnote{Do not worry about the colors of references and links. They are there to make the editorial process easier and will disappear prior to official publication.}

Please spend some time reading through these and the more general instructions. Your 30 minutes on this is likely to save you and us hours of additional work. Do not hesitate to contact the editors if you have any questions.

\section{Illustrating OSL commands and conventions}\label{sim:sec:osl-comm}

Below I illustrate the use of a number of commands defined in langsci-osl.tex (see the styles folder).

\subsection{Typesetting semantics}\label{sim:sec:sem}

See below for some examples of how to typeset semantic formulas. The examples also show the use of the sib-command (= ``semantic interpretation brackets''). Notice also the the use of the dummy curly brackets in \REF{sim:ex:quant}. They ensure that the spacing around the equation symbol is correct. 

\ea \ea \sib{dog}$^g=\textsc{dog}=\lambda x[\textsc{dog}(x)]$\label{sim:ex:dog}
\ex \sib{Some dog bit every boy}${}=\exists x[\textsc{dog}(x)\wedge\forall y[\textsc{boy}(y)\rightarrow \textsc{bit}(x,y)]]$\label{sim:ex:quant}
\z\z

\noindent Use noindent after example environments (but not after floats like tables or figures).

And here's a macro for semantic type brackets: The expression \textit{dog} is of type $\stb{e,t}$. Don't forget to place the whole type formula into a math-environment. An example of a more complex type, such as the one of \textit{every}: $\stb{s,\stb{\stb{e,t},\stb{e,t}}}$. You can of course also use the type in a subscript: dog$_{\stb{e,t}}$

We distinguish between metalinguistic constants that are translations of object language, which are typeset using small caps, see \REF{sim:ex:dog}, and logical constants. See the contrast in \REF{sim:ex:speaker}, where \textsc{speaker} (= serif) in \REF{sim:ex:speaker-a} is the denotation of the word \textit{speaker}, and \cnst{speaker} (= sans-serif) in \REF{sim:ex:speaker-b} is the function that maps the context $c$ to the speaker in that context.\footnote{Notice that both types of small caps are automatically turned into text-style, even if used in a math-environment. This enables you to use math throughout.}$^,$\footnote{Notice also that the notation entails the ``direct translation'' system from natural language to metalanguage, as entertained e.g. in \citet{Heim.Kratzer1998}. Feel free to devise your own notation when relying on the ``indirect translation'' system (see, e.g., \citealt{Coppock.Champollion2022}).}

\ea\label{sim:ex:speaker}
\ea \sib{The speaker is drunk}$^{g,c}=\textsc{drunk}\big(\iota x\,\textsc{speaker}(x)\big)$\label{sim:ex:speaker-a}
\ex \sib{I am drunk}$^{g,c}=\textsc{drunk}\big(\cnst{speaker}(c)\big)$\label{sim:ex:speaker-b}
\z\z

\noindent Notice that with more complex formulas, you can use bigger brackets indicating scope, cf. $($ vs. $\big($, as used in \REF{sim:ex:speaker}. Also notice the use of backslash plus comma, which produces additional space in math-environment.

\subsection{Examples and the minsp command}

Try to keep examples simple. But if you need to pack more information into an example or include more alternatives, you can resort to various brackets or slashes. For that, you will find the minsp-command useful. It works as follows:

\ea\label{sim:ex:german-verbs}\gll Hans \minsp{\{} schläft / schlief / \minsp{*} schlafen\}.\\
Hans {} sleeps {} slept {} {} sleep.\textsc{inf}\\
\glt `Hans \{sleeps / slept\}.'
\z

\noindent If you use the command, glosses will be aligned with the corresponding object language elements correctly. Notice also that brackets etc. do not receive their own gloss. Simply use closed curly brackets as the placeholder.

The minsp-command is not needed for grammaticality judgments used for the whole sentence. For that, use the native langsci-gb4e method instead, as illustrated below:

\ea[*]{\gll Das sein ungrammatisch.\\
that be.\textsc{inf} ungrammatical\\
\glt Intended: `This is ungrammatical.'\hfill (German)\label{sim:ex:ungram}}
\z

\noindent Also notice that translations should never be ungrammatical. If the original is ungrammatical, provide the intended interpretation in idiomatic English.

If you want to indicate the language and/or the source of the example, place this on the right margin of the translation line. Schematic information about relevant linguistic properties of the examples should be placed on the line of the example, as indicated below.

\ea\label{sim:ex:bailyn}\gll \minsp{[} Ėtu knigu] čitaet Ivan \minsp{(} často).\\
{} this book.{\ACC} read.{\PRS.3\SG} Ivan.{\NOM} {} often\\\hfill O-V-S-Adv
\glt `Ivan reads this book (often).'\hfill (Russian; \citealt[4]{Bailyn2004})
\z

\noindent Finally, notice that you can use the gloss macros for typing grammatical glosses, defined in langsci-lgr.sty. Place curly brackets around them.

\subsection{Citation commands and macros}

You can make your life easier if you use the following citation commands and macros (see code):

\begin{itemize}
    \item \citealt{Bailyn2004}: no brackets
    \item \citet{Bailyn2004}: year in brackets
    \item \citep{Bailyn2004}: everything in brackets
    \item \citepossalt{Bailyn2004}: possessive
    \item \citeposst{Bailyn2004}: possessive with year in brackets
\end{itemize}

\section{Trees}\label{s:tree}

Use the forest package for trees and place trees in a figure environment. \figref{sim:fig:CP} shows a simple example.\footnote{See \citet{VandenWyngaerd2017} for a simple and useful quickstart guide for the forest package.} Notice that figure (and table) environments are so-called floating environments. {\LaTeX} determines the position of the figure/table on the page, so it can appear elsewhere than where it appears in the code. This is not a bug, it is a property. Also for this reason, do not refer to figures/tables by using phrases like ``the table below''. Always use tabref/figref. If your terminal nodes represent object language, then these should essentially correspond to glosses, not to the original. For this reason, we recommend including an explicit example which corresponds to the tree, in this particular case \REF{sim:ex:czech-for-tree}.

\ea\label{sim:ex:czech-for-tree}\gll Co se řidič snažil dělat?\\
what {\REFL} driver try.{\PTCP.\SG.\MASC} do.{\INF}\\
\glt `What did the driver try to do?'
\z

\begin{figure}[ht]
% the [ht] option means that you prefer the placement of the figure HERE (=h) and if HERE is not possible, you prefer the TOP (=t) of a page
% \centering
    \begin{forest}
    for tree={s sep=1cm, inner sep=0, l=0}
    [CP
        [DP
            [what, roof, name=what]
        ]
        [C$'$
            [C
                [\textsc{refl}]
            ]
            [TP
                [DP
                    [driver, roof]
                ]
                [T$'$
                    [T [{[past]}]]
                    [VP
                        [V
                            [tried]
                        ]
                        [VP, s sep=2.2cm
                            [V
                                [do.\textsc{inf}]
                            ]
                            [t\textsubscript{what}, name=trace-what]
                        ]
                    ]
                ]
            ]
        ]
    ]
    \draw[->,overlay] (trace-what) to[out=south west, in=south, looseness=1.1] (what);
    % the overlay option avoids making the bounding box of the tree too large
    % the looseness option defines the looseness of the arrow (default = 1)
    \end{forest}
    \vspace{3ex} % extra vspace is added here because the arrow goes too deep to the caption; avoid such manual tweaking as much as possible; here it's necessary
    \caption{Proposed syntactic representation of \REF{sim:ex:czech-for-tree}}
    \label{sim:fig:CP}
\end{figure}

Do not use noindent after figures or tables (as you do after examples). Cases like these (where the noindent ends up missing) will be handled by the editors prior to publication.

\section{Italics, boldface, small caps, underlining, quotes}

See \citet{Nordhoff.Muller2021} for that. In short:

\begin{itemize}
    \item No boldface anywhere.
    \item No underlining anywhere (unless for very specific and well-defined technical notation; consult with editors).
    \item Small caps used for (i) introducing terms that are important for the paper (small-cap the term just ones, at a place where it is characterized/defined); (ii) metalinguistic translations of object-language expressions in semantic formulas (see \sectref{sim:sec:sem}); (iii) selected technical notions.
    \item Italics for object-language within text; exceptionally for emphasis/contrast.
    \item Single quotes: for translations/interpretations
    \item Double quotes: scare quotes; quotations of chunks of text.
\end{itemize}

\section{Cross-referencing}

Label examples, sections, tables, figures, possibly footnotes (by using the label macro). The name of the label is up to you, but it is good practice to follow this template: article-code:reference-type:unique-label. E.g. sim:ex:german would be a proper name for a reference within this paper (sim = short for the author(s); ex = example reference; german = unique name of that example).

\section{Syntactic notation}

Syntactic categories (N, D, V, etc.) are written with initial capital letters. This also holds for categories named with multiple letters, e.g. Foc, Top, Num, etc. Stick to this convention also when coming up with ad hoc categories, e.g. Cl (for clitic or classifier).

An exception from this rule are ``little'' categories, which are written with italics: \textit{v}, \textit{n}, \textit{v}P, etc.

Bar-levels must be typeset with bars/primes, not with an apostrophe. An easy way to do that is to use mathmode for the bar: C$'$, Foc$'$, etc.

Specifiers should be written this way: SpecCP, Spec\textit{v}P.

Features should be surrounded by square brackets, e.g., [past]. If you use plus and minus, be sure that these actually are plus and minus, and not e.g. a hyphen. Mathmode can help with that: [$+$sg], [$-$sg], [$\pm$sg]. See \sectref{sim:sec:hyphens-etc} for related information.

\section{Footnotes}

Absolutely avoid long footnotes. A footnote should not be longer than, say, {20\%} of the page. If you feel like you need a long footnote, make an explicit digression in the main body of the text.

Footnotes should always be placed at the end of whole sentences. Formulate the footnote in such a way that this is possible. Footnotes should always go after punctuation marks, never before. Do not place footnotes after individual words. Do not place footnotes in examples, tables, etc. If you have an urge to do that, place the footnote to the text that explains the example, table, etc.

Footnotes should always be formulated as full, self-standing sentences.

\section{Tables}

For your tables use the table plus tabularx environments. The tabularx environment lets you (and requires you in fact) to specify the width of the table and defines the X column (left-alignment) and the Y column (right-alignment). All X/Y columns will have the same width and together they will fill out the width of the rest of the table -- counting out all non-X/Y columns.

Always include a meaningful caption. The caption is designed to appear on top of the table, no matter where you place it in the code. Do not try to tweak with this. Tables are delimited with lsptoprule at the top and lspbottomrule at the bottom. The header is delimited from the rest with midrule. Vertical lines in tables are banned. An example is provided in \tabref{sim:tab:frequencies}. See \citet{Nordhoff.Muller2021} for more information. If you are typesetting a very complex table or your table is too large to fit the page, do not hesitate to ask the editors for help.

\begin{table}
\caption{Frequencies of word classes}
\label{sim:tab:frequencies}
 \begin{tabularx}{.77\textwidth}{lYYYY} %.77 indicates that the table will take up 77% of the textwidth
  \lsptoprule
            & nouns & verbs  & adjectives & adverbs\\
  \midrule
  absolute  &   12  &    34  &    23      & 13\\
  relative  &   3.1 &   8.9  &    5.7     & 3.2\\
  \lspbottomrule
 \end{tabularx}
\end{table}

\section{Figures}

Figures must have a good quality. If you use pictorial figures, consult the editors early on to see if the quality and format of your figure is sufficient. If you use simple barplots, you can use the barplot environment (defined in langsci-osl.sty). See \figref{sim:fig:barplot} for an example. The barplot environment has 5 arguments: 1. x-axis desription, 2. y-axis description, 3. width (relative to textwidth), 4. x-tick descriptions, 5. x-ticks plus y-values.

\begin{figure}
    \centering
    \barplot{Type of meal}{Times selected}{0.6}{Bread,Soup,Pizza}%
    {
    (Bread,61)
    (Soup,12)
    (Pizza,8)
    }
    \caption{A barplot example}
    \label{sim:fig:barplot}
\end{figure}

The barplot environment builds on the tikzpicture plus axis environments of the pgfplots package. It can be customized in various ways. \figref{sim:fig:complex-barplot} shows a more complex example.

\begin{figure}
  \begin{tikzpicture}
    \begin{axis}[
	xlabel={Level of \textsc{uniq/max}},  
	ylabel={Proportion of $\textsf{subj}\prec\textsf{pred}$}, 
	axis lines*=left, 
        width  = .6\textwidth,
	height = 5cm,
    	nodes near coords, 
    % 	nodes near coords style={text=black},
    	every node near coord/.append style={font=\tiny},
        nodes near coords align={vertical},
	ymin=0,
	ymax=1,
	ytick distance=.2,
	xtick=data,
	ylabel near ticks,
	x tick label style={font=\sffamily},
	ybar=5pt,
	legend pos=outer north east,
	enlarge x limits=0.3,
	symbolic x coords={+u/m, \textminus u/m},
	]
	\addplot[fill=red!30,draw=none] coordinates {
	    (+u/m,0.91)
        (\textminus u/m,0.84)
	};
	\addplot[fill=red,draw=none] coordinates {
	    (+u/m,0.80)
        (\textminus u/m,0.87)
	};
	\legend{\textsf{sg}, \textsf{pl}}
    \end{axis} 
  \end{tikzpicture} 
    \caption{Results divided by \textsc{number}}
    \label{sim:fig:complex-barplot}
\end{figure}

\section{Hyphens, dashes, minuses, math/logical operators}\label{sim:sec:hyphens-etc}

Be careful to distinguish between hyphens (-), dashes (--), and the minus sign ($-$). For in-text appositions, use only en-dashes -- as done here -- with spaces around. Do not use em-dashes (---). Using mathmode is a reliable way of getting the minus sign.

All equations (and typically also semantic formulas, see \sectref{sim:sec:sem}) should be typeset using mathmode. Notice that mathmode not only gets the math signs ``right'', but also has a dedicated spacing. For that reason, never write things like p$<$0.05, p $<$ 0.05, or p$<0.05$, but rather $p<0.05$. In case you need a two-place math or logical operator (like $\wedge$) but for some reason do not have one of the arguments represented overtly, you can use a ``dummy'' argument (curly brackets) to simulate the presence of the other one. Notice the difference between $\wedge p$ and ${}\wedge p$.

In case you need to use normal text within mathmode, use the text command. Here is an example: $\text{frequency}=.8$. This way, you get the math spacing right.

\section{Abbreviations}

The final abbreviations section should include all glosses. It should not include other ad hoc abbreviations (those should be defined upon first use) and also not standard abbreviations like NP, VP, etc.


\section{Bibliography}

Place your bibliography into localbibliography.bib. Important: Only place there the entries which you actually cite! You can make use of our OSL bibliography, which we keep clean and tidy and update it after the publication of each new volume. Contact the editors of your volume if you do not have the bib file yet. If you find the entry you need, just copy-paste it in your localbibliography.bib. The bibliography also shows many good examples of what a good bibliographic entry should look like.

See \citet{Nordhoff.Muller2021} for general information on bibliography. Some important things to keep in mind:

\begin{itemize}
    \item Journals should be cited as they are officially called (notice the difference between and, \&, capitalization, etc.).
    \item Journal publications should always include the volume number, the issue number (field ``number''), and DOI or stable URL (see below on that).
    \item Papers in collections or proceedings must include the editors of the volume (field ``editor''), the place of publication (field ``address'') and publisher.
    \item The proceedings number is part of the title of the proceedings. Do not place it into the ``volume'' field. The ``volume'' field with book/proceedings publications is reserved for the volume of that single book (e.g. NELS 40 proceedings might have vol. 1 and vol. 2).
    \item Avoid citing manuscripts as much as possible. If you need to cite them, try to provide a stable URL.
    \item Avoid citing presentations or talks. If you absolutely must cite them, be careful not to refer the reader to them by using ``see...''. The reader can't see them.
    \item If you cite a manuscript, presentation, or some other hard-to-define source, use the either the ``misc'' or ``unpublished'' entry type. The former is appropriate if the text cited corresponds to a book (the title will be printed in italics); the latter is appropriate if the text cited corresponds to an article or presentation (the title will be printed normally). Within both entries, use the ``howpublished'' field for any relevant information (such as ``Manuscript, University of \dots''). And the ``url'' field for the URL.
\end{itemize}

We require the authors to provide DOIs or URLs wherever possible, though not without limitations. The following rules apply:

\begin{itemize}
    \item If the publication has a DOI, use that. Use the ``doi'' field and write just the DOI, not the whole URL.
    \item If the publication has no DOI, but it has a stable URL (as e.g. JSTOR, but possibly also lingbuzz), use that. Place it in the ``url'' field, using the full address (https: etc.).
    \item Never use DOI and URL at the same time.
    \item If the official publication has no official DOI or stable URL (related to the official publication), do not replace these with other links. Do not refer to published works with lingbuzz links, for instance, as these typically lead to the unpublished (preprint) version. (There are exceptions where lingbuzz or semanticsarchive are the official publication venue, in which case these can of course be used.) Never use URLs leading to personal websites.
    \item If a paper has no DOI/URL, but the book does, do not use the book URL. Just use nothing.
\end{itemize}

\section{Introduction: Unstressed vowels in East Slavic} 
\subsection{Two-degree reduction in Central Standard Russian}

Standard Russian is known for its typologically unusual word prosodic structure and vowel reduction pattern. Unstressed vowels are partly neutralized: most notably, /o/ merges with /a/. In most languages, vowel merger is combined with phonetic reduction, but in Standard Russian it is combined with an unusually prominent vowel in first pretonic position -- i.e. the vowel immediately preceding the stressed syllable. This vowel can be very long and lack reduction in quality, especially in Central Standard Russian. The term vowel \textit{reduction} can therefore be misleading, as \citet{Dubina2012} and \citet{Iosad2012} remark, although the term is often used not in a purely phonetic, but in a more abstract, phonological sense. In the current paper, \textsc{reduction} will mostly refer to phonetics, but \sectref{post:sec:phonology} will discuss the phonological status of the two degrees of reduction in varieties of Russian. I refer to \textsc{contemporary Central Standard Russian} (CSR), following \citet{Iosad2012}, because almost all studies on Russian unstressed vowels in Standard Russian are based on speakers from Moscow or Saint Petersburg, but I wish to point out that this pronunciation might differ from other, locally coloured varieties of Standard Russian.

In CSR, the immediately pretonic syllable forms a salient contrast, together with the stressed syllable, with unstressed syllables in other, weak positions, which are heavily reduced, both in quality and in quantity \citep{Zlatoustova1981,Kodzasov1999}, especially when the allophone is a low vowel. One could say that the first pretonic and the stressed syllable together form a nucleus in the word (\citealt{Kodzasov1999}, \textit{inter alia}), i.e. a strong centre that is opposed to a weak periphery of the word \citep{Kasatkina1996}, or that word stress is realized over two syllables \citep[cf.][xiii]{Dubina2012}. 

\largerpage
The unusual prominence of the vowel in first pretonic position means that effectively, CSR has two degrees of vowel reduction: a moderate degree for the first pretonic vowel and a radical degree of reduction for unstressed vowels in other positions \citep[e.g.][]{Crosswhite2000}.\footnote{Moderate reduction is also found in other positions than the first pretonic if they allow long vowel durations, notably, in onsetless syllables and, optionally, in phrase-final open syllables \citep[e.g.][]{Barnes2006,Iosad2012}. Based on a purely phonetic study, \citet{Kuznecov1997} discerns an additional, third degree of reduction in duration of /a, o/ after non-palatalized consonants, which is found in posttonic vowels, but only one degree of qualitative reduction, since these vowels were not reduced in quality in first pretonic position.} The term \textsc{two-degree reduction} usually refers to qualitative reduction, but the main distinction might be in duration. The phonological status of the difference in degrees of reduction is discussed in surprisingly few studies, as remarked by \citet[133]{Molczanow2015}. \citet{Barnes2006} and \citet{Iosad2012}, two of the exceptions, argue that only the difference in duration is phonological, the difference in quality being merely an effect of phonetic implementation rules; cf. \sectref{post:sec:phonology} below. One should bear in mind that two-degree reduction accounts first and foremost for the vowels /a/ and /o/ in the position after non-palatalized consonants.\footnote{\label{post:fn:DurationalDifferences}The durational differences between the syllables relative to stress are not found for all vowels, and they are highest for non-high vowels, i.e., for unstressed /a, o/ after non-palatalized consonants \citep{BondarkoEtAl1966,Zlatoustova1981,PadgettTabain2005}. A qualitative distinction into two different allophones according to position relative to stress is mentioned only for these vowels. In the position after palatalized consonants, all vowels except /u/ are produced as front vowels.} When unstressed, they merge, but they merge into two different allophones, depending on their position in the word. In CSR, the word /moloˈko/ `milk' is pronounced as [mәlɐˈko], with the first vowel, the /o/ in second pretonic (antepretonic) position, being substantially shorter than the prominent next vowel, in first pretonic position, which can have the same quality as a stressed /a/ (\citealt{Kuznecov1997} on speakers from Saint Petersburg; \citealt{Knjazev2006} on speakers from Moscow) and may be as long as or even longer than the stressed syllable (in non-focus position; \citealt{Knjazev2006}).\footnote{Most older literature uses the symbol [ʌ] for the first degree reduction of /a, o/ after non-palatalized consonants, but for contemporary CSR, the symbol [a] or [ɐ] is more appropriate; cf. \citet{Kasatkina2005} and \citet{Iosad2012} for a discussion. This accounts for speakers from both Moscow and Saint Petersburg. Most differences in pronunciation between standard speakers from these cities have disappeared; cf. \citet{Verbickaja1977} and \citet{Kuznecov1997}. Speakers from cities with a Northern Russian substrate use vowels further back \citep{Kasatkina2005}.} In addition, it is often singled out by a local high tone, at least by Moscow speakers (on pitch accented words in declaratives, cf. \citealt{Kasatkina2005}). \citet{Dubina2012} and \citet{Molczanow2015} connect the heavy first pretonic syllable in Russian and Belarusian to an abstract phonological high tone that is associated with the first pretonic syllable, comparable to the anticipatory tone spreading in Bosnian/Croatian/Serbian \citep[175]{Dubina2012}. In CSR, however, this high tone need not surface in the phonetic realization of the word (unlike, probably, in some conservative Russian dialects; see \sectref{post:subsec:dissimilation} below).

The distribution of syllable prominence in the word in CSR is captured by \citeposst{Potebnja1866} formula $112\text{ˈ}311$, where $1$ means radical reduction, $2$ means moderate reduction and $\text{ˈ}3$ stands for the unreduced stressed syllable. Empirical studies have confirmed that \citeauthor{Potebnja1866}’s $112\text{ˈ}311$ formula corresponds to a three-way distinction in duration in CSR \citep{BondarkoEtAl1966,Zlatoustova1981,Kuznecov1997,Barnes2006,Knjazev2006}. 

The qualitative reduction of unstressed /a, o/ after non-palatalized consonants in two degrees to a low vowel or to \textit{schwa} is part of the pronunciation standard \citep{Avanesov1984}, which is taught to foreigners learning Russian. 


\largerpage[-1]
Vowel reduction in Central Standard Russian has received due attention in the literature. Examples of phonological accounts of the two-degree reduction in CSR are \citet{Crosswhite2000}, \citet{Iosad2012} and \citet{Molczanow2015}. Empirical phonetic research on vowel reduction in CSR has been done by, among others, \citet{BondarkoEtAl1966,Zlatoustova1981,Kuznecov1997,PadgettTabain2005,Barnes2006,KocharovEtAl2015}. However, most existing acoustic studies are based on a very limited number of speakers, and many questions deserve more attention, among others, the relation between vowel quality, quantity and tone and the phonological status of two-degree reduction (cf. \citealt{Barnes2006,Bethin2006,Molczanow2015}, \textit{inter alia}).

Much less is known about unstressed vowels and the word prosodic structure in other, non-central varieties of Russian (\sectref{post:subsec:regional}) and in the other East Slavic languages (\sectref{post:subsec:belarusian}). Contemporary Russian has little regional variation. Rural dialects show relatively small linguistic distance, compared to dialects of other European languages, and among today’s urban population, geographically based differences are hardly present, but one can expect at least some variation in prosody between the regions \citep{GrammatcikovaEtAl2013,Post2017}. Therefore we decided to study the prosodic word in regional urban varieties of Russian.

\subsection{Regional varieties of Russian}\label{post:subsec:regional}
\subsubsection{Quantitative reduction}
\largerpage[-1]

Studies of traditional rural Russian dialects suggest that all varieties have two-degree reduction in duration, but to a different extent. \citet{Vysotskij1973} measured vowel and consonant durations in words with the structure CV\textsubscript{\tiny{$-2$}}CV\textsubscript{\tiny{$-1$}}ˈCV\textsubscript{\tiny{$0$}}C with the vowels /a, o/ in a large number of Russian varieties. He discerned twelve different groups of rural dialects and three varieties of Russian spoken in Moscow, each with its own rhythmic structure. The first pretonic vowel -- V\textsubscript{\tiny{$-1$}} -- was longer than the preceding vowel -- V\textsubscript{\tiny{$-2$}} -- in all groups, but whereas many dialects in Central Russia combine unusually long first pretonic vowels with very short vowels in second pretonic position (\citealt{Potebnja1866,Vysotskij1973,AlmuxamedovaKulsaripova1980}, \textit{inter alia}), the difference between the two positions was much less pronounced in southern and northern parts of European Russia \citep{Vysotskij1973}. Two-degree reduction in duration is possibly absent in some northern rural dialects, where both pretonic vowels had almost the same duration in Vysotskij’s recordings \citep{Vysotskij1973}.

Most Russians today do not speak a traditional rural dialect, however, but an urban variety with few local characteristics. An unusually long and low first pretonic [a] is typical for the old Moscow dialect and Moscow vernacular speech \citep{Vysotskij1973}. To our knowledge, only two, preliminary, studies compare unstressed vowel durations in cities other than Moscow and Saint Petersburg (\citealt{Erofeeva2005,GrammatcikovaEtAl2013}). Both suggest that the geographical opposition between centre and periphery is retained in modern urban Russian, with a weaker difference in duration between the first and second pretonic vowels in non-central areas than in central European Russia. The number of speakers and vowels measured was very low in these two studies.\footnote{\citeposst{Erofeeva2005} study of vowel duration is based on 600 vowels from spontaneous speech by two male and two female speakers from Perm (city or region) with an audible local accent; \citet{GrammatcikovaEtAl2013} used tokens from a read text that was read once by only 6 individual speakers, one from each city, who were assessed to speak Standard Russian.} More data are therefore needed to confirm their preliminary findings on non-Central modern urban Russian pronunciation.

\subsubsection{Qualitative reduction, dissimilation and tone}\label{post:subsec:dissimilation}
\largerpage[-2]
Two-degree reduction in quality is not possible in traditional Northern dialects, for unstressed /o/ and /a/ do not merge in these dialects. Some Central-Russian dialects combine partial neutralization with strong durational two-degree reduction: The distinction between /a/ and /o/ is retained in the -- prominent -- first pretonic position, but they merge in other, radically reduced positions \citep[cf.][]{AvanesovEtAl1986}. This means that these dialects do distinguish two degrees in quality as well -- with no reduction in first pretonic position and strong reduction (to \textit{schwa}) in second pretonic position. A comparison of studies on qualitative reduction \citep[e.g.][]{AvanesovEtAl1986} with those on quantity (such as \citealt{Vysotskij1973}) suggests that the area with incomplete merger coincides with the area with very strong two-degree durational reduction, with uncommonly short second pretonic and extremely long first pretonic vowels. 

In the remaining Central Russian and in all Southern Russian traditional dialects /o/ and /a/ are neutralized in all unstressed positions. In a large part of these neutralizing dialects (and in some neighbouring dialects in Belarus) the vowel reduction pattern is complicated by vowel dissimilation. Vowel dissimilation means that the quality and duration of the first pretonic vowel are dependent on the quality of the stressed vowel, for the two vowels must be different (\citealt{AvanesovEtAl1986}, \textit{inter alia}). Clear dissimilation tendencies in duration, with a length trade-off between the first pretonic and tonic vowel, are observed even in modern Moscow speech, which does not have vowel dissimilation in quality \citep{Kasatkina2005}. For instance, the first pretonic low vowel [a] is longest before the high vowel [i], and shorter before a low vowel [a], and this difference is substantially larger than what can be accounted for by the intrinsic durational properties of high and low vowels \citep{Zlatoustova1981,Kasatkina2005,Iosad2012}. The strong pretonic vowels were almost invariably marked by a local high tone in some conservative central dialects of Russian and Belarusian (\citealt{Broch1916,Bethin2006}; \textit{inter alia}).\footnote{Note that \citet{Borise2017} could not confirm a local high peak before a fall on the stressed syllable on the words with pretonic strengthening, which was earlier found in the Belarusian dialect she studied (the dialect of Aŭciuki; cf. \citealt{Bethin2006}). Instead, she found a generally higher tone on both first pretonic and tonic syllable (compared to words without pretonic strengthening in this dialect with vowel dissimilation).}

The unreduced, labialized production of unstressed /o/ -- \textit{okanʼe} -- has low social prestige \citep{Andrews1995} and is rare in contemporary urban Russian, especially in read speech \citep{VerbickajaEtAl1984,Erofeeva1993}, but incomplete neutralization, with a distribution of vowel allophones for unstressed /a/ different from /o/, is more common (\citealt{Erofeeva1993} and, less so, \citealt{Erofeeva2005}, on speech from Perm -- a city in the Ural region with Northern Russian traits). Even more common is ``non-normative'' vowel reduction, i.e. a failure to distinguish two different allophones in the second and first pretonic position. This was reported to be frequent in all seven regional cities studied by \citet{VerbickajaEtAl1984}. In a small empirical study, \citet{Erofeeva2005} found substantially lower F1 values for unstressed /a/ and /o/ than for stressed /a/ in Perm, suggesting substantial qualitative reduction. Besides, the allophone in first pretonic position is reported to be further back in cities with a Northern dialect substrate than in CSR \citep{Kasatkina2005}, suggesting low F2 values. No other empirical studies of vowel reduction patterns in regional urban Russian are known to us.

\subsection{Belarusian and Ukrainian: quantitative, but no qualitative two-degree reduction}\label{post:subsec:belarusian}

First pretonic prominence is a feature shared by all three East Slavic languages \citep{Dubina2012}, but this accounts only for duration, not for vowel quality. Neither Standard Belarusian nor Ukrainian has two-degree reduction in quality, depending on the position of the vowel in relation to the stressed syllable. In Ukrainian, unstressed /o/ and /a/ do not merge at all, whereas in Standard Belarusian they merge into a low vowel in all unstressed positions \citep{Černjavskij2012,Dubina2012}. Besides, the durational distinction between the second and first pretonic vowels appears to be smaller in Standard Belarusian and Ukrainian than in Central Standard Russian; cf. \citet{Dubina2012} on Belarusian, referring to an empirical study by \citet{Andreeu1984}, and \citet{LukaszewiczEtAl2022} on Ukrainian. The difference is very small in the latter study, but the word prosodic pattern in Ukrainian is complicated by iterative secondary stresses \citep{LukaszewiczMolczanow2018}. Besides, the study is based on speakers from Western Ukraine. The geographical distribution of strong first pretonic prominence in Russia suggests that the difference might be larger in Central and Eastern Ukraine.

The variation of vowel reduction patterns in the East Slavic languages and their dialects shows that first pretonic vowel prominence in duration in East Slavic need not cooccur with vowel neutralization and qualitative reduction in two degrees, in the way it does in Central Standard Russian. Vowel neutralization, first pretonic prominence and qualitative reduction are distinct processes \citep[cf.][166]{Dubina2012}, which interrelate in various ways.

\subsection{Research question}

We wanted to know how clearly the two-degree reduction in quality and quantity of Central Standard Russian is expressed in today’s urban speech. More particularly, how different are the second and first pretonic vowels /a, o/ after non-palatalized consonants from each other, and from the tonic (stressed) vowel, in Russian regional speech, both in quantity (duration) and in quality (as expressed in vowel formants)?

The scarce existing literature on regional variation in two-degree reduction suggests that we would find a substantial difference between the second pretonic and the first pretonic vowel in Moscow speech, but a smaller difference in Russian speech from other regions. We chose to compare the speech of young urban Russians from Moscow, as a representative of Central European Russian, and Perm (Ural), representing non-central Russian.

\section{Our study: Moscow vs. Perm}

\subsection{Participants}
\largerpage
We recorded speech by young speakers in Moscow (central variety) and Perm (Ural region, non-central variety; cf. \citealt{Erofeeva2005}). We chose Moscow because it is Russia’s capital and because Moscow’s speech, especially its vernacular speech, is known for its long and open realizations of /o/ and /a/ in first pretonic position. The city of Perm is situated in the Ural region on the border between European Russia and Siberia. Perm speech is known for its relatively strong local accent, with traits from Northern Russian \citep{Erofeeva2005}. The quality and duration of the unstressed vowels are claimed to play an important role in its local colouring \citep{ErofeevaEtAl2000}. The phonetics of speech by Russians from Perm have been described extensively (e.g. \citealt{Erofeeva2005}), but mostly auditorily. Acoustic studies measuring durations and formants are all but absent.

We chose to record both sexes, since gender differences are often found in other countries, with a tendency of young urban women speaking with less local colouring than men \citep{Labov2001}. All participants (but one, whose data were left out) were raised in the city they were recorded (Perm or Moscow) and almost all had parents with a high educational level. We did not select our participants for speaking (perceived) Standard Russian or having a local accent. Therefore, the speech of our Moscow participants could differ from Central Standard Russian, and the speech of our adolescents from Perm from \citeposst{Erofeeva2005} results, which were based on speakers with perceived local colouring. We can expect some degree of local colouring for our mean values, but not a strong local vernacular accent, since we use a reading task read by pupils with highly educated parents made in a formal school setting (see \sectref{post:subsec:procedure}).

\subsection{Materials}

The reading task consisted of 14 sentences with 4 target words under several prosodic conditions, read in the same order. The target words were \textit{topotátʼ} `to patter', \textit{pokopátʼ} `to dig (a little)' and \textit{potakátʼ} `to connive', three words that were also analysed in \citet{Vysotskij1973}, and \textit{kopátʼ} `to dig', with a (CV\textsubscript{\tiny{$-2$}})CV\textsubscript{\tiny{$-1$}}ˈCV\textsubscript{\tiny{$0$}}C structure with pretonic /o/ and /a/ after non-palatalized consonants.\footnote{Russian text, including Russian surnames, is transliterated using \citeposst{ComrieCorbett1993} transliteration system.} We will call the vowels in second pretonic position a\textsubscript{\tiny{$-2$}}, since they occur two syllables prior to the stressed syllable, the first pretonics a\textsubscript{\tiny{$-1$}} and the tonic (stressed) vowels á\textsubscript{\tiny{$0$}}. The symbols a\textsubscript{\tiny{$-2$}} and a\textsubscript{\tiny{$-1$}} stand for both /a/ and /o/, which merge in unstressed positions in Standard Russian. Like most previous studies, we confined ourselves to /a/ and /o/, since other vowels might not show reduction in two degrees (cf. footnote \ref{post:fn:DurationalDifferences}). The task was designed preliminary to measure durations rather than formants, which led to some limitations to what we can conclude from our formant data (caused by an uneven distribution of /a/ and /o/, which are surrounded by different plosive consonants leading to different consonant coarticulation effects; cf. \sectref{post:subsec:interpreting} and \sectref{post:subsec:qualitative} below).\footnote{As remarked by one of the anonymous reviewers, the etymology of the pretonic vowels in \textit{potakátʼ} and \textit{topotátʼ} is not certain, let alone their phonemic analysis by the speakers, so the single unstressed /a/ in our target words might in fact represent not /a/, but /o/ for some of the speakers, in which case all pretonic vowels in the reading task are /o/. This is only of minor importance for our analysis, since unstressed /a/ and /o/ merge completely in Moscow speech and probably almost completely in read speech from Perm; cf. \sectref{post:subsec:interpreting} below. Another limitation of our reading task is that \textit{potakátʼ} and \textit{topotátʼ} are infrequent words that may not have been recognized by all speakers and may have provoked unnatural pronunciations. We minimized this problem by leaving out the first occurrences of the target words.}

For the present study we left out the 4 sentences with \textit{kopatʼ}, for lacking vowels in second pretonic position, and the first occurrences of the target words, in citation form, and the sentence with \textit{pokopatʼ} in initial position, to avoid hesitations and boundary phenomena. The remaining 6 sentences cover two prosodic conditions for each of the three target words:

\ea\label{post:ex:nuclear-pitch}utterance-medial position, carrying nuclear pitch accent (here, marked in small caps):\\
\gll \textit{Ja} \textit{\textsc{pokopatʼ}} \textit{pošla}.\\
I.{\NOM} dig.{\INF} go.{\FEM.\SG.\PST}\\
\glt `I went to dig a little.'
\z

\ea\label{post:ex:no-nuclear-pitch}utterance-medial position, not carrying nuclear pitch accent: \\
\gll \textit{Ja} \textit{topotatʼ} \textit{uže} \textit{ne} \textit{\textsc{budu}}.\\
I.{\NOM} patter.{\INF} already {\NEG} {1\SG.\IPFV.\FUT}\\
\glt `I won’t patter anymore.'
\z

\noindent The position of the accents was not marked in the reading task, but all parti\-cipants read sentence \REF{post:ex:nuclear-pitch} with the target word carrying the last, nuclear accent, followed by an unaccented word. In sentence \REF{post:ex:no-nuclear-pitch} the nuclear accent is carried by the last word of the sentence. The target word carries a prenuclear pitch accent in most of the renditions, but other accentuation patterns occur as well, including deaccentuation of \textit{topotatʼ} following a contrastive accent on the preceding word \textit{Ja}. The nuclear pitch accent can vary as well, since the speakers can choose list intonation, which has a falling-rising tune. This variation in accentual patterns ensures that our results are not restricted to one specific prosodic structure but have a broader validity. It can potentially have a small effect on the actual vowel durations.

\subsection{Procedure}\label{post:subsec:procedure}

We recorded a total of 34 adolescents (born in 1996--2000, recorded in 2015), 10 girls and 9 boys in Perm and 7 girls and 8 boys in Moscow. The participants were recorded in quiet rooms at school using digital recorders and head-mounted microphones (Zoom H5 with Shure WH20 in Perm, Zoom H2 with Samson QV in Moscow, set at 44.1 kHz, 16-bit, .wav). The reading task was performed along with recordings of a range of other tasks for, or connected to, Benedikte Vardøy’s PhD project on young Russians’ perception of regional variation in Russian (cf. \citealt{Vardøy2021,PostAndreeva2023}). The utterances were read from paper, only once in Perm, but twice in Moscow. We used only the first iteration of the speakers in Moscow, in order to have comparable data, but their second reading gave us the possibility to replace unsuccessful first renditions.

Several speakers were excluded from the analysis. We left out a boy in Moscow who was raised elsewhere and a girl in Perm because of creaky voice. After the segmentation, six additional speakers were discarded, because they had gaps in the data (see next section).

Six utterances by 13 female and 13 male speakers were analysed statistically (\tabref{post:tab:speakers-tokens}).

\begin{table}
\caption{Number of speakers and tokens used for the statistical analyses}
\label{post:tab:speakers-tokens}
 \begin{tabularx}{\textwidth}{Xrrrrr}
  \lsptoprule
            & speakers & female  & male & word tokens & vowel tokens\\
  \midrule
  Moscow  &   13  &    7  &    6      & 78  &234\\
  Perm & 13 & 6 & 7 &78 & 234\\
Total & 26 & 13 & 13 & 156 & 468\\
  \lspbottomrule
 \end{tabularx}
\end{table}



\subsection{Data analysis and statistics}

The target vowels in the speech samples were segmented manually in Praat \citep{BoersmaWeenink1992} via visual inspection of the waveform and spectrogram. We used standard criteria, among others, vowel boundary labels were placed at zero crossings on the waveform close to the onset and offset of the vowel formants. A number of tokens was discarded, due to misreadings, long pauses or creaky voice. For the Moscow speakers, discarded productions were replaced by their second readings. The speakers with missing data were excluded from the analysis, which left us with 13 speakers from Moscow and 13 speakers from Perm (see \tabref{post:tab:speakers-tokens}).

The durations, F0, F1, F2 and F3 of the target vowels per speaker and location were extracted using Praat scripts. Vowel durations were log-transformed because of positive skewness. F1 and F2 were measured at the temporal midpoint in vocalic nuclei. Speaker-dependent standard normalization was applied to control for differences in formant values due to speaker identity and sex \citep{Lobanov1971}. We used the software JMP 16.2.0 for statistical analysis. Linear mixed models (LMM) were fitted with the respective measure as dependent variable and \textsc{distance-to-stress} with three factor levels (\textit{0}/\textit{1}/\textit{2}), \textsc{location} with two factor levels (\textit{Moscow}/\textit{Perm}) and \textsc{gender} with two factor levels (\textit{male}/\textit{female}) as fixed factors, as well as all their possible interactions. \textsc{speaker}, \textsc{word} (\textit{topotatʼ}{\slash}\textit{po\-ko\-patʼ}{\slash}\textit{pokatatʼ}) and \textsc{position-in-utterance} (\textit{utterance-medial prenuclear}{\slash}\textit{ut\-ter\-ance-me\-di\-al nuclear position}) were taken as random factors. Separate Tukey HSD post-hoc tests were carried out per variable, if appropriate. The confidence level was set at $α = .05$.

\section{Results}
\subsection{Vowel quantity: Duration}

\begin{figure}[b]
  \begin{tikzpicture}
    \begin{axis}[
	%xlabel={Level of \textsc{uniq/max}},
	ylabel={log duration},
	axis lines*=left,
        width  = .8\textwidth,
	height = 5cm,
    	every node near coord/.append style={font=\tiny},
        nodes near coords align={vertical},
	ymin=0,
	ymax=4.2,
	ytick distance=1,
    minor y tick num=5,
	xtick=data,
	ylabel near ticks,
    major x tick style = transparent,
	x tick label style={font=\itshape},
	ybar=0pt,
    bar width=35pt,
	legend pos=outer north east,
	enlarge x limits=0.5,
	symbolic x coords={Moscow, Perm},
	]
	\addplot[fill=langscicol4!30,draw=none,error bars/.cd, y dir=both, y explicit, error mark options={rotate=90,mark size=6pt}] coordinates {
	    (Moscow,3.8119808) += (0,0.05459734) -= (0,0.05459734)
        (Perm,3.8583681) += (0,0.05459467) -= (0,0.05459467)

	};
	\addplot[fill=langscicol4,draw=none,error bars/.cd, y dir=both, y explicit, error mark options={rotate=90,mark size=6pt}] coordinates {
	    (Moscow,3.8353316) += (0,0.05659178) -= (0,0.05659178)
        (Perm,3.5729059) += (0,0.06058881) -= (0,0.06058881)
	};
	\legend{females, males}
    \end{axis}
  \end{tikzpicture}
    \caption{\textsc{gender} vs. \textsc{location}: Mean duration (log-transformed) of vowels (all three positions together), for female speakers (light-coloured) and Perm speakers (dark-coloured), with Moscow speech on the left and Perm speech on the right}
    \label{post:fig:gender-location}
\end{figure}
%https://tikz.dev/pgfplots/reference-errorbars
%https://tex.stackexchange.com/questions/276820/pgfplots-bar-graph-with-confidence-intervals-error
%https://tex.stackexchange.com/questions/470674/adding-error-bars-to-bar-plot


The statistical analysis on the durational data (log-transformed) shows a main effect of \textsc{gender} ($\text{F [1, 22]} = 6.0601$, $p < .02$) on the target vowel duration, with female speakers having somewhat longer overall vowel durations than their male peers (cf. \figref{post:fig:gender-location}, light-coloured vs. dark-coloured bars), and, predictably, of \textsc{distance}{}-\textsc{to}{}-\textsc{stress} ($\text{F [2, 431]} = 815.4856$, $p < .001$), with stressed vowels (á\textsubscript{\tiny{$0$}}) being significantly longer than the vowels in the first pretonic syllable (a\textsubscript{\tiny{$-1$}}), which in turn are longer than the vowels in the second pretonic syllable (a\textsubscript{\tiny{$-2$}}).\footnote{Bar plots of the mean durations (log-transformed) according to gender and according to distance to stress, as well as sound files and details on the data can be provided by the author.} Not significant proved \textsc{location} ($\text{F [1, 22]} = 4.1169$, $p = .055$). The analysis revealed two significant interactions, first, between \textsc{location} and \textsc{gender} ($\text{F [1, 22]} = 8.4120$, $p < .01$). Post-hoc tests show that the girls have significantly longer vowels than the boys only in Perm (\figref{post:fig:gender-location}, right), for the boys and girls in Moscow (left) used similar average vowel durations (see also \citealt{Post2024}). This shows that the earlier mentioned gender effect is due to the difference in Perm.



The second interaction is the highly significant interaction between \textsc{location} and \textsc{distance-to-}\textsc{stress} ($\text{F [2, 431]} = 65.0217$, $p < .001$), which is the main result for our research question. In the realizations of the speakers from Moscow (\figref{post:fig:location-distance-to-stress}, light-coloured bars) the vowel duration becomes significantly shorter with increasing distance from the stressed syllable, whereas in the realizations of the speakers from Perm (\figref{post:fig:location-distance-to-stress}, dark-coloured bars) the only significant opposition we find is between the stressed vowels on the one hand and both unstressed vowels on the other. The small difference in average duration between the two prestressed vowels in Perm -- cf. the minimal difference between second and first pretonics in Perm (dark-coloured bars) in \figref{post:fig:location-distance-to-stress} -- is not statistically significant. This is shown by the letter report from the post-hoc pairwise Tukey HSD test of mean values, the results of which are given above each bar in \figref{post:fig:location-distance-to-stress}. Bars not connected by the same letter are significantly different. The test gave a\textsubscript{\tiny{$-2$}} in Perm the letters C and D, whereas a\textsubscript{\tiny{$-1$}} received C, so the two positions share the same letter C, meaning that their small average difference in duration was not significant.

\begin{figure}[t]
  \begin{tikzpicture}
    \begin{axis}[
	%xlabel={Level of \textsc{uniq/max}},  
	ylabel={log duration}, 
	axis lines*=left, 
        width  = .8\textwidth,
	height = 5cm,
    nodes near coords, 
    point meta=explicit symbolic,
    % 	nodes near coords style={text=black},
    every node near coord/.append style={font=\large},
    nodes near coords align={vertical},
	ymin=0,
	ymax=4.8,
	ytick distance=1,
    minor y tick num=5,
	xtick=data,
	ylabel near ticks,
    major x tick style = transparent,
	x tick label style={font=\itshape},
	ybar=0pt,
    bar width=25pt,
	legend pos=outer north east,
	enlarge x limits=0.35,
	symbolic x coords={second pretonic, first pretonic, stressed},
	]
	\addplot[fill=langscicol4!30,draw=none,error bars/.cd, y dir=both, y explicit, error mark options={rotate=90,mark size=6pt}] coordinates {
	    (second pretonic,3.1268467) += (0,0.04475132) -= (0,0.04475132) [D]
        (first pretonic,3.9156951) += (0,0.01939877) -= (0,0.01939877) [B]
        (stressed,4.4284267) += (0,0.02023690) -= (0,0.02023690) [A]
	};
	\addplot[fill=langscicol4,draw=none,error bars/.cd, y dir=both, y explicit, error mark options={rotate=90,mark size=6pt}] coordinates {
	    (second pretonic,3.2752833) += (0,0.04077388) -= (0,0.04077388) [C,D]
        (first pretonic,3.4001192) += (0,0.05096319) -= (0,0.05096319) [C]
        (stressed,4.4715085) += (0,0.02303410) -= (0,0.02303410) [A]
	};
	\legend{Moscow, Perm}
    \end{axis} 
  \end{tikzpicture} 
    \caption{Mean duration (log-transformed) according to \textsc{location} and \textsc{distance-to-stress} (light-coloured bars = Moscow; dark-coloured bars = Perm; both genders). Error bars represent standard errors, with Compact Letter Display of pairwise comparisons of the mean durations from the post-hoc pairwise Tukey HSD test. Bars not connected by the same letter are significantly different.}
    \label{post:fig:location-distance-to-stress}
    %https://tex.stackexchange.com/questions/307040/assigning-xlabels-to-symbolic-x-coords
\end{figure}


In Moscow, the actual mean durations (not log-transformed) of the three consecutive vowels ($\text{a\textsubscript{\tiny{$-2$}}} : \text{a\textsubscript{\tiny{$-1$}}} : \text{á\textsubscript{\tiny{$0$}}}$) were $24.1\,\text{ms} : 49.5\,\text{ms} : 82.8\,\text{ms}$, which gives relative durations of almost $1 : 2 : 3$. In Perm, the ratio is close to $1 : 1 : 3$, with its mean durations of $28.3\,\text{ms} : 33.1\,\text{ms} : 89.3\,\text{ms}$.\footnote{For details on mean durations and an elaborate discussion of the durational measurements from a sociolinguistic perspective, see \citet{Post2024}.} 

There was no significant interaction in our data between \textsc{gender} and \textsc{distance-to-stress} ($\text{F [2, 431]} = 1.3204$, $p > .05$), nor between all three variables ($\text{F [2, 431]} = 1.4552$, $p > .05$), for the boys used the same relative durations as the girls in both cities (see \citealt{Post2024}).

These numbers show mean values, but the variation between individual tokens is large. Even with this high level of variability, the difference between Moscow and Perm in mean and relative durations of the pretonic vowels is highly significant.


\subsection{Vowel quality: Formants}
\largerpage
\pgfplotstableread[col sep=comma]{
X,Y,E,T,S
second pretonic,-0.994193,0.05954260,E,-.17
first pretonic,0.666930,0.05729071,B,.015
stressed,0.327263,0.05728952,C,.015
}\tMoscow

\pgfplotstableread[col sep=comma]{
X,Y,E,T,S
second pretonic,-0.607915,0.05744826,D,-.17
first pretonic,-0.400861,0.05733466,D,-.17
stressed,1.008776,0.05712402,A,.015
}\tPerm

\begin{figure}[b]
\begin{tikzpicture}
\begin{axis}[
    ylabel={zF1},
    axis lines*=left,
	width  = .8\textwidth,
    height = 5cm,
    nodes near coords,
    point meta=explicit symbolic,
    %
    nodes near coords align={vertical},
    every node near coord/.append style={
       font=\large, yshift={transformdirectiony(\sd)}
    },
    visualization depends on=\thisrow{S} \as \sd,
    ymin=-1.4,
    ymax=1.4,
	ytick distance=.5,
    minor y tick num=1,
	xtick=data,
	ylabel near ticks,
    major x tick style = transparent,
	x tick label style={font=\itshape},
	ybar=0pt,
    bar width=25pt,
	legend pos=outer north east,
    enlarge x limits=0.35,
    symbolic x coords={second pretonic,first pretonic,stressed},
]
    \addplot[fill=langscicol4!30,draw=none, error bars/.cd,y dir=both,y explicit,error mark options={rotate=90,mark size=4pt}]
         table[meta=T,x=X,y=Y,y error=E] {\tMoscow};
    \addplot[fill=langscicol4,draw=none, error bars/.cd,y dir=both,y explicit,error mark options={rotate=90,mark size=4pt}]
         table[meta=T,x=X,y=Y,y error=E] {\tPerm};
    \legend{Moscow, Perm}
\end{axis}
\end{tikzpicture}
\caption{Mean zF1 values according to \textsc{location} and \textsc{distance-to-stress} (both genders, light-coloured bars = Moscow; dark-coloured bars = Perm), with letter report from the pairwise comparison Tukey test}
\label{post:fig:mean-location-distance-to-stress}
\end{figure}
% https://tex.stackexchange.com/questions/642111/labels-above-error-bars
% https://tex.stackexchange.com/questions/200571/expressing-a-y-only-shift-with-axis-direction-cs-with-symbolic-x-coords
% https://tex.stackexchange.com/questions/130313/different-options-for-nodes-for-positive-and-negative-bars-in-pgfplots
%https://tex.stackexchange.com/questions/527968/custom-data-labels-for-each-plot-in-bar-chart


Parallel to the durational data, the statistical analyses of the normalized values of the first formants (zF1 Lobanov normalization) show a main effect of \textsc{distance-to-stress} ($\text{F [2, 484]} = 218.7679$, $p < .001$), with significantly different mean F1 values in each of the three positions (with letters C, B and A for a\textsubscript{\tiny{$-2$}}, a\textsubscript{\tiny{$-1$}} and á\textsubscript{\tiny{$0$}} in the letters report from the Tukey HSD test). 

Unlike for the durational values, a significant interaction for zF1 was found between \textsc{distance-to-stress} and \textsc{gender} ($\text{F [2, 478]} = 6,4192$, $p < .01$). This gender effect is mainly due to a difference in F1 in stressed position. Both the men and women differ significantly in F1 according to each position, but the Tukey HSD test showed that the female speakers produce relatively higher F1 values than the men in á\textsubscript{\tiny{$0$}}.

The most relevant result for our research question, concerning the relative difference between the two pretonic positions in the two cities, is the correlation between \textsc{location} and \textsc{distance-to-stress,} which again is highly significant ($\text{F [2, 484]} = 89.7769$, $p < .001$; \figref{post:fig:mean-location-distance-to-stress}), just like for duration (\figref{post:fig:location-distance-to-stress}). In Perm (dark-coloured bars), an opposition is again found only between stressed position on the one hand (level A in the Tukey pairwise comparison) and unstressed position on the other (level D for both a\textsubscript{\tiny{$-2$}} and a\textsubscript{\tiny{$-1$}}). In Moscow, the F1 values are significantly different in each position, like the durations, a\textsubscript{\tiny{$-2$}}, a\textsubscript{\tiny{$-1$}} and á\textsubscript{\tiny{$0$}} receiving the letters E, B, and C, respectively. Unlike for duration, a\textsubscript{\tiny{$-1$}} has not lower, but higher mean F1 values than á\textsubscript{\tiny{$0$}} in stressed position, and differs most from a\textsubscript{\tiny{$-2$}}, the other unstressed position.


The statistical measurements on our F2 data show no significant difference between the locations, Moscow and Perm. They reveal a main effect only of \textsc{distance-to-stress} ($\text{F [2, 485]} = 384,7697$, $p < .05$; \figref{post:fig:mean-zF2}).

\pgfplotstableread[col sep=comma]{
X,Y,E,T,S
second pretonic,-0.6788944,0.03313592,C,0.36
first pretonic,-0.1869485,0.03273086,B,0.1
stressed,0.8658429,0.03268251,A,0.015
}\tFTwo

\begin{figure}
  \begin{tikzpicture}
    \begin{axis}[
	ylabel={zF2}, 
	axis lines*=left, 
        width  = .9\textwidth,
	height = 5cm,
    nodes near coords, 
    point meta=explicit symbolic,
        every node near coord/.append style={
       font=\large, yshift={transformdirectiony(\sd)}
    },
    visualization depends on=\thisrow{S} \as \sd,
    nodes near coords align={above},
	ymin=-.8,
	ymax=1,
	ytick distance=.5,
    minor y tick num=1,
	xtick=data,
	ylabel near ticks,
    major x tick style = transparent,
	x tick label style={font=\itshape},
	ybar=0pt,
    bar width=35pt,
	enlarge x limits=0.5,
	symbolic x coords={second pretonic, first pretonic, stressed},
	]
     \addplot[fill=langscicol4!70,draw=none, error bars/.cd,y dir=both,y explicit,error mark options={rotate=90,mark size=4pt}]
         table[meta=T,x=X,y=Y,y error=E] {\tFTwo};
    \end{axis} 
  \end{tikzpicture} 
    \caption{Mean zF2 values according to \textsc{distance-to-stress} (both genders and both locations), including letter report from the pairwise comparison Tukey test}
    \label{post:fig:mean-zF2}
\end{figure}

Our data do not support \citeposst{Kasatkina2005} claim of low F2 values in a\textsubscript{\tiny{$-1$}} in varieties with a Northern substrate, but they do not disprove them either, given the limited value of our data for the comparison of actual formant values.

We also plotted F1 against F2 (\figref{post:fig:vowel-plot}, not normalized), to visualize the actual measurements. The plots show a close to full overlap of the two pretonic vowels (square and triangle markers) in Perm, but much less overlap in Moscow, especially for F1 ($y${}-axis). The highest degree of variation is found for a\textsubscript{\tiny{$-1$}} (triangle markers) in Perm, covering a much larger space, especially in F2, than the stressed vowels (circle markers). The female (top) and male speakers (bottom) show the same patterns.


\begin{figure}
    \includegraphics[width=1\textwidth]{figures/post-WordProsodicStructure-fig005.pdf}
    \caption{Vowel plots of F1 ($y$-axis) and F2 ($x$-axis) of the second pretonic (squares; dashed line), first pretonic (triangles; dotted line) and stressed (circles; solid line) vowels in the three target words produced by female speakers and male speakers in Moscow and Perm, with ellipses representing one standard deviation from the mean}
    \label{post:fig:vowel-plot}
\end{figure}


\section{Discussion}

\subsection{Durational differences: Larger than expected}

The durational data confirm our hypothesis of a larger difference between the realizations of a\textsubscript{\tiny{$-2$}} and a\textsubscript{\tiny{$-1$}} in Moscow than in Perm. In fact, the difference between Moscow and Perm is larger than could be expected from previous literature. In Moscow, a\textsubscript{\tiny{$-1$}} is, on average, twice as long as a\textsubscript{\tiny{$-2$}}, whereas a\textsubscript{\tiny{$-2$}} and a\textsubscript{\tiny{$-1$}} hardly differ in Perm. In our data from adolescents, the relative difference in duration between the two pretonic vowels in Moscow is larger than in previous literature on Central Standard Russian (e.g. \citealt{Zlatoustova1981,Kuznecov1997,Barnes2006,Knjazev2006}) and even than in \citeposst{Vysotskij1973} data on traditional Moscow speech (cf. \citealt{Post2024}). As remarked earlier, our speakers from Moscow do not necessarily have a Standard Russian pronunciation and might have a stronger local accent, with a stronger prominence of the first pretonic vowel, than the participants of earlier studies of CSR. However, the first pretonic vowels are still not very prominent. They are shorter than in most previous studies (\citealt{Zlatoustova1981}, \textit{inter alia}), also relative to the duration of the stressed vowel, and both pretonic vowels are much shorter than the surrounding consonants, at least in the utterance in which we measured the durations of both vowels and consonants (cf. \citealt{Post2024}). 

The relative difference in duration between the pretonic vowels in our data from Perm is similar to the difference in the small data set from Perm measured by \citet{Erofeeva2005} (see \citealt{Post2024}) and proved not statistically significant. Unlike our data, Erofeeva’s data were based on spontaneous speech from four speakers with an audible local accent, where one can expect a stronger degree of local colouring and, therefore, a smaller difference between the two pretonic vowels, than in a formal reading task by speakers who were not selected for locally coloured pronunciation. Nevertheless, in our reading task -- recorded in a formal setting at school, which encourages the use of normative speech -- the readers used the same locally coloured word prosodic structure.

\subsection{Interpreting formant values in our data}\label{post:subsec:interpreting}

The first and second formants of a vowel give an indication of its quality, F1 of its height (high or low tongue position) and F2 of the vowel’s frontness (but cf. \citealt{WhalenEtAl2022} for a recent warning not to equate formant measurements with vowel quality). The F1 and F2 values in our data should be approached with caution, since our reading task was not specifically designed to measure formant data. The vowels are surrounded by different consonants, giving different coarticulation effects (cf. \citealt[65]{Bondarko1977} on coarticulation from preceding consonants; \citealt{Kasatkina2005} on effects from consonants following the vowel). Consonant coarticulation is strong in Russian, especially on short unstressed vowels, like the vowels in our data. Another complicating factor is that our reading task contains both /o/ and /a/ in unstressed position (/a/ only in \textit{potakatʼ}). These vowels merge categorically in unstressed position in Moscow speech and in CSR \citep{Barnes2006}, but it is not certain that they have merged completely in Perm. \citet{Erofeeva2005} found a slightly different distribution of unstressed /a/ and /o/ over vowel allophones in Perm, with a lower span of F2 values for unstressed \mbox{/o/} than for unstressed /a/ (\citeyear[220]{Kasatkina2005}). We cannot exclude that unstressed /o/ is still produced differently from unstressed /a/ by our speakers from Perm (or by some of them). Our reading task does not allow a direct comparison of pretonic /a/ with /o/ because of the different adjacent consonants. However, while non-merger might influence the absolute formant values of our data, it does not affect our main research question, which concerns the relative difference between the two pretonic positions, which is very small in Perm in any case, not only for F1, but for F2 as well, as the following section \ref{post:subsec:qualitative} will argue.

\subsection{Qualitative differences}
\label{post:subsec:qualitative}

Our results from the measurements of the first formant are very similar to those on duration as regards the relative difference between the two pretonic positions: Both measures show a large distance between the second and first pretonic vowels in Moscow, suggesting a very different opening grade for a\textsubscript{\tiny{$-2$}} (level E) and a\textsubscript{\tiny{$-1$}} (level B in the pairwise comparison test; \figref{post:fig:mean-location-distance-to-stress}), but no significant difference between them in Perm (with both vowels on level D). The large difference in F1 values in Perm between the unstressed positions -- both level D -- with the stressed position -- level A -- suggests that the unstressed vowels are actually highly reduced in quality in Perm as well as in quantity, but in only one degree.

In Moscow, the main difference lies this time between a\textsubscript{\tiny{$-2$}} and the other two positions. The F1 value of a\textsubscript{\tiny{$-1$}} is just as high or even higher than in á\textsubscript{\tiny{$0$}} (\figref{post:fig:mean-location-distance-to-stress}), suggesting it is not qualitatively reduced, although a\textsubscript{\tiny{$-1$}} is substantially reduced in duration. F1 values as high in a\textsubscript{\tiny{$-1$}} as in á\textsubscript{\tiny{$0$}}, representing a fully open, low vowel, have earlier been found among speakers of Central Standard Russian, most, but not all of them, Muscovites (\citealt{Kuznecov1997,Knjazev2006}; three of the four speakers in \citealt{Barnes2006}). The even higher values in prestressed than in stressed position can be caused by the vowel raising effect of the final palatalized consonant following á\textsubscript{\tiny{$0$}}.

The pretonic vowels have different mean F2 values in both cities (\figref{post:fig:mean-zF2}), but this appears to be due to factors other than relative distance to stress, for the difference between a\textsubscript{\tiny{$-2$}} and a\textsubscript{\tiny{$-1$}} disappears both in Moscow and Perm when one compares the mean values of the vowels in the same CVC string only -- \nobreakdash-\textit{pot}{}- in \textit{po\textsubscript{\tiny{$-2$}}}\textit{t(akatʼ)} vs. \textit{(to)po\textsubscript{\tiny{$-1$}}}\textit{t(atʼ)}. The likely cause of the different F2 values when the other strings are included is consonant coarticulation, especially in Perm, where the vowels are even shorter than in Moscow, thus facilitating even stronger influence of the consonants.\footnote{The labial [p], frequent in second pretonic position, lowers F2, contrary to the laminal [t] and velar [k], the consonants surrounding a\textsubscript{\tiny{$-1$}} in \textit{potakatʼ} (cf. \citealt[65]{Bondarko1977}).} In Perm, there might be an additional cause for the difference between the two pretonic positions: Incomplete merger of /a/ and /o/ would lead to higher F2 values for the phoneme /a/, which in our data only occurs in first pretonic position. 

\subsection{Phonological status of two-degree reduction: Categorical difference in Moscow, not in Perm?}
\label{post:sec:phonology}

\citeauthor{Trubetzkoy1969} (\citeyear{Trubetzkoy1969}, originally published in 1939) claimed that the allophones of /a, o/ in the two positions in Standard Russian are in complementary distribution. This is clearly not the case in (Standard) Belarusian, where we saw that the vowels are low irrespective of their distance to stress. Contrary to Trubetzkoy, \citet{Barnes2006} and \citet{Iosad2012} argue that the distinction between various reduction grades in Standard Russian is mainly one of duration \citep[531--532]{Iosad2012} and that the target vowel of /a, o/ after non-palatalized consonants in Standard Russian is a low vowel in all unstressed positions. This fits well with the Standard Belarusian reduction pattern. \citet{Barnes2006} argues that the difference in quality found in Standard Russian is merely a result of phonetic implementation rules and caused by vowel undershoot: The vowels in second pretonic position are too short, he argues, to reach the low target level, which requires a minimal duration of approx. $60\,\text{ms}$ in his data \citep{Barnes2006}. What did we find in our data from Moscow and Perm?

Our data strongly suggest that the durational difference of /a, o/ between the two pretonic positions is categorical in Moscow speech. Not only is the first pretonic vowel on average twice as long as the second pretonic vowel, but, additionally, the asymmetry between the two positions is very stable: A closer look at the data shows that the second pretonic vowel is shorter than the first pretonic in all but a single token -- in 77 out of the 78 analysed word tokens recorded in Moscow. This stability is remarkable, since these 78 tokens were pronounced with various nuclear and prenuclear pitch accents and focus patterns, resulting in a variety of tonal configurations and degrees of prominence on the target words. Our data obviously confirm \citeposst{Kasatkina2005} observation that the reduction in two degrees is not confined to utterances with a high tone on the first pretonic syllable, so if tone plays a role in the Central Russian word prosodic structure \citep{Dubina2012,Molczanow2015}, it does indeed only on an abstract phonological level.

The large difference in quality between the pretonic positions in Moscow might be categorical as well, and it is unlikely to be caused solely by vowel undershoot.\footnote{The difference in first formant values between the two positions is not only large. In addition, our Moscow speakers tend to produce higher F1 values for /a, o/ in a\textsubscript{\tiny{$-1$}} than in a\textsubscript{\tiny{$-2$}} even in those rare cases when both vowels have similar durations. A presentation of these data and a further discussion of the relation between F1 and duration is outside the scope of this paper.} The high average F1 values of a\textsubscript{\tiny{$-1$}} among our Moscow speakers suggest that our young speakers need less time than $60\,\text{ms}$ to reach the target level of a vowel, since their mean duration is only $51\,\text{ms}$. The target in second pretonic position in Moscow speech might not be a high vowel, as \citet{Barnes2006} suggests for CSR, but a vowel with a quality different from the one in first pretonic position (cf. \citealt{Trubetzkoy1969}), thereby showing both qualitative and quantitative vowel dissimilation, not between the tonic and first pretonic vowels, as in the dissimilating Russian and Belarusian rural dialects, but between the second pretonic and first pretonic position. Further studies are needed to find out whether this pattern is found among Moscow speakers only.

In Perm, on the other hand, the differences between the pretonic positions in duration and in F1 are small and not statistically significant. The Perm speakers make a clear distinction between stressed and unstressed vowels, in both duration and in quality, but there is no evidence of a phonological distinction between first and second pretonic position.

\section{Conclusions}

In Central Standard Russian the prosodic word has a strong nucleus, consisting of the first pretonic and the tonic syllable, which leads to vowel reduction in two degrees, expressed in both quantity and quality, at least when the vowels are /a/ or /o/ after non-palatalized consonants. Our study is, to our knowledge, the first acoustic study of a considerable number of speakers that compares the vowel reduction patterns of these vowels in speech from Central European Russia, represented by Moscow, with modern urban speech from a different region, represented by Perm.

Our main research question concerns the relative difference in duration and quality between the second and first pretonic vowels in the two cities. In our reading task, they have a significantly different word prosodic pattern. Our data corroborate earlier suggestions that the difference between the two degrees is expressed less clearly in non-central varieties of modern urban Russian. In Moscow, the distinction between the first and second pretonics is remarkably large, even larger than in previous studies of Central Standard Russian. On average, the durational relation between second pretonic and first pretonic vowel is almost $1 : 2$ in our data. However, although a\textsubscript{\tiny{$-1$}} -- /a, o/ in first pretonic position -- is twice as long as second pretonic a\textsubscript{\tiny{$-2$}}, it is still much shorter than the stressed vowel á\textsubscript{\tiny{$0$}}. In Perm, the difference between the two pretonic vowels was small and not statistically significant. In Moscow, the vowels also differ greatly in quality: the F1 formant values of a\textsubscript{\tiny{$-2$}} are much lower than for a\textsubscript{\tiny{$-1$}}, where a\textsubscript{\tiny{$-1$}} can have an even higher mean F1 value, suggesting a lower tongue position, than the stressed vowel, even though it is much shorter. In Perm, however, the mean F1 values are low in both pretonic syllables, and much lower than in stressed /a/, confirming \citeposst{Erofeeva2005} findings of considerable qualitative reduction, in addition to a very strong quantitative reduction.

Our data imply that the clear distinction between two degrees of durational reduction of /a, o/ (after non-palatalized consonants) is part of the phonology of Moscow speech. A further study of the relation between F1 and duration might confirm the indications that the qualitative distinction between the pretonic positions is part of Moscow phonology as well, and not a result of phonetic implementation, as \citet{Barnes2006} suggests for Central Standard Russian. In Perm, on the other hand, the differences between the pretonic positions are so small that they suggest there might not be any phonological reduction in two degrees at all. This is remarkable, given the fact that in previous research even all traditional rural dialects had two-degree durational reduction to some degree, with a possible exception of a subgroup of Northern European rural dialects \citep{Vysotskij1973}. One would not expect to find this most extreme pattern in modern urban Russian speech. Unlike these dialects, however, Perm speech shows much qualitative reduction of unstressed /a/ and /o/.

Thus, the normative pronunciation standard \citep{Avanesov1984} of vowel reduction in two clearly distinguished degrees is not a general feature of modern urban Russian speech, even though modern urban Russian shows little regional variation. This norm is not followed by our young speakers from Perm, not even in a reading task, where the tendency to follow high status norms is higher than in informal spontaneous speech. We found no gender differences in the word prosodic structure. Our comparison shows that today’s urban youth still have local prosodic traits in their speech. Both Moscow and Perm speech are known for their local accent, so more research would be welcome on speech from other regions and from the other countries where East-Slavic languages are spoken, to give a better picture of the variation in vowel reduction processes in East Slavic.

\section*{Abbreviations}

\begin{tabularx}{.5\textwidth}{@{}lQ}
1       &first person\\%
{\FEM}  &feminine\\%
{\FUT}  &future\\%
{\INF}  &infinitive\\%
{\IPFV} &imperfective\\%

\end{tabularx}%
\begin{tabularx}{.5\textwidth}{lQ@{}}
{\NEG}  &negation\\%
{\NOM}  &nominative\\%
{\PST}  &past tense\\%
{\SG}   &singular\\%
&\\ % this dummy row achieves correct vertical alignment of both tables
\end{tabularx}

\section*{Acknowledgments}
A range of people have helped me during this dive into phonetics, a field I knew too little about. First of all, I want to thank Benedikte Fjellanger Vardøy for her fieldwork in Perm and for our exciting first exploration of the data, Svetlana Djačenko for doing most segmentations, and Bistra Andreeva for vital statistical analyses and for advice and support. Alexander Krasovitsky, Sergej Knjazev, Elena Erofeeva, Brechtje Post, Mitko Sabev and Carl Börstell answered questions on phonetics and statistics, and Elaine Schmidt and Dirk Jan Vet wrote the Praat scripts. I am also indebted to the members of the Phonetics group at Saarland University for their support during my research stay in 2022, and to Ivan Yuen and to the two anonymous reviewers, whose constructive and insightful comments led to a substantial improvement of this paper. I cannot possibly have incorporated all advice, so all remaining shortcomings are my sole responsibility. Last but not least, I thank the speakers and the secondary schools in Moscow and Perm they attended for their participation.

This study was made possible by various grants from the Meltzer Foundation and from the University of Bergen. The data collection was approved by the Norwegian Social Science Data Services (NSD).

\printbibliography[heading=subbibliography,notkeyword=this]

\end{document}
