\documentclass[output=paper,colorlinks,citecolor=brown]{langscibook}
\ChapterDOI{10.5281/zenodo.15394186}
%\bibliography{localbibliography}

\author{Olav Mueller-Reichau\affiliation{Leipzig University}}
% replace the above with you and your coauthors
% rules for affiliation: If there's an official English version, use that (find out on the official website of the university); if not, use the original
\title{Perfectivity in Russian, Czech and Colloquial Upper Sorbian}
% replace the above with your title
\abstract{Addressing the topic of inner-Slavic variation in aspect, the present paper discusses the issues raised by the seemingly un-Slavic distribution of perfective and imperfective forms in Colloquial Upper Sorbian. An analysis is offered according to which Upper Sorbian perfectives have a weaker semantics than Czech perfectives, which in turn are weaker than Russian perfectives, with the imperfective in all three languages being radically underspecified. It is shown that this approach can successfully model the observed difference in aspect choice between the three languages.    

\keywords{Upper Sorbian, Slavic Aspect, Perfectivity, Microvariation}
}

\begin{document}

%%% provide a shorter version of your title in case it doesn't fit a single line in the running head
% \shorttitlerunninghead{your short title}

%%% uncomment the following line if you are a single author or all authors have the same affiliation
% \SetupAffiliations{mark style=none}

\maketitle

% Just comment out the input below when you're ready to go.

\section{Introduction}\label{mueller:sec:intro}
All Slavic languages have the grammatical category of verbal aspect, contrasting perfective forms with imperfective ones. Language textbooks usually claim that the perfective aspect expresses an action that is (or will be) completed, whereas the imperfective aspect expresses that the action is (or was) not yet completed. From a linguistic point of view, there are many problems with such simplified statements (cf. \citealt[18]{mue:Comrie1976}).

One problem for the analysis that considers perfectivity as completion and imperfectivity as incompletion is that it is simply false, as there are well-known occurrences of imperfectives with reference to completed events, as demonstrated in detail for Russian in \citet{Gronn2004}. Such cases have thus to be listed as exceptions. Another problem is that such a simple characterisation for each language, useful as it might be from a pedagogical perspective, suggests that the aspectual systems of different Slavic languages all work alike. This, however, is not the case, as we know at least since the foundational work of \citet{mue:Dickey2000}. 
Other authors addressing inner-Slavic aspectual differences include \citet{Alvestad13},
\citet{Arregui14}, \citet{Breu00}, \citet{Klimek22},
\citet{Petruchina2000}, \citet{Rivero10}, 
\citet{Wiemer08}.  
As for studies seeking to explain deviating aspectual behaviours between two Slavic languages, most have addressed differences between Czech and Russian, e.g. 
\citet{Berger13}, \citet{Berger16}, 
\citet{Gehrke22},
\citet{Heck18},
\citet{omr18},
\citet{Stunova1991, Stunova1993}.

In the present paper, I would like to draw attention to the aspectual system of Upper Sorbian,
more specifically to Colloquial Upper Sorbian. Papers that include this language into the theoretical discussion on aspectual variation within the Slavic family are rare (but see \citealt{Breu00}, \citealt{Toops1998}, \citealt{Wiemer08}). A proposal for a formal analysis of aspect in Upper Sorbian has, to my knowledge, never been carried out. My motivation is also driven by the belief that a proper theoretical treatment of perfectivity and imperfectivity in Sorbian will shed new light on notions and categories that we as aspectologists believe to know well, and that we, therefore, do not think of  questioning. Central in this respect is the notion of perfectivity, which is why I cannot avoid repeating the following often-quoted statement in this introduction (the author conducted field work on the Sorbian dialect of Mužakov):
``Mne kažetsja, \v{c}to perfektivnost' v tom smysle, kak my ee sebe pred\-stavljaem v russkom jazyke, vovse ne suš\v{c}estvuet v mužakovskom''
    $\,$[It seems to me that perfectivity in the sense we understand it in Russian does not exist at all in the dialect of Mužakov; own translation] (\citealt[121]{Scerba1973}).\footnote{Quoted after \citet[230]{Scholze2008} and \citet[144]{Werner2018}.}
The present paper argues for a nuanced view of perfectivity in Slavic. 

The paper is structured as follows. In \sectref{mueller:sec:cus}
I will first present the relevant facts about Colloquial Upper Sorbian, with special emphasis on where it differs from Russian and Czech patterns. In \sectref{mueller:sec:background} I will then introduce some theoretical background assumptions that I take for granted in the line of argumentation to follow. \sectref{mueller:sec:term} discusses the proposal made in the literature according to which perfectivity in Colloquial Upper Sorbian should be analysed as terminativity. In view of the empirical shortcomings of this kind of approach, \sectref{mueller:sec:det} moves on to assess the explanatory power of the second available proposal, which holds that perfectives in Colloquial Upper Sorbian encode determinateness. I will conclude that this analysis is on the right track, but it has to be made more precise to fully capture the data. To this end, I will offer a proposal in \sectref{mueller:sec:proposal}, showing that it correctly models the distribution of aspectual forms. Having so far laid out my proposal in prose, I will pin it down formally in \sectref{mueller:sec:formalism}. \sectref{scalessec} serves to place my analysis into the broader context of scalarity-based approaches to Slavic aspect. \sectref{mueller:sec:two} introduces two more data points that one might suppose do not fall under the theory developed, and I will explain why they in fact do.
Finally, \sectref{mueller:sec:fin} concludes by integrating the findings based on Colloquial Upper Sorbian into the general picture of inner-Slavic aspectual variation.




\section{What is special about aspect in Upper Sorbian?}\label{mueller:sec:cus}

Besides being proficient in German, speakers of Upper Sorbian live in a situation of diglossia (\citealt{Breu2000}, \citealt{Lewa2002}; see \citealt[39ff.]{Scholze2008} for discussion). On the one hand, there is the codified standard language, often referred to as the literary language (e.g. \citealt{Stone1993}), which is spoken in formal contexts, for instance in school. The codification was oriented towards the Polish and Czech models (\citealt[168]{Werner2003}).
On the other hand, speakers employ the colloquial language in everyday speech.
These two varieties differ quite strongly, not the least with respect to verbal aspect. 

The colloquial language deviates from the literary language in many respects (\citealt{Breu2000},
\citealt{Fasske1981}, \citealt{Lewa2002},
\citealt{Scholze2008}, \citealt{Stone1993}). Besides aspect, this 
concerns, inter alia, the sporadic expression of personal masculine gender, the obligatory presence of articles in noun phrases, a particularly 
high number of lexical borrowings from German, a passive construction with the German loan verb \textit{wordowa\'c} `become', the use of adverbial preverbs as verbal prefixes like in 
\textit{hr\'omad\'zestaje\'c} `put together' (together+put) or 
\textit{nutř\v{c}in\'c} `put in' (inside+do).
%following the German patterns, e.g. XXXX.  
%only (\citealt{Stone1993})
%Influence from German, amount of lexical borrowings particularly high (\citealt[674ff.]{Stone1993}) ; e.g. verbs of motion (\citealt[636]{Stone1993}), adverbs as prefixes (\citealt[637]{Stone1993}) passive with the German loan 
%compound future with a perfective infinitive (\citealt[637]{Stone1993})

With respect to aspect, standard Upper Sorbian has preserved the categories of Aorist and Imperfect. In addition to this ``old" opposition, the literary language also displays the ``new" opposition between perfective and imperfective verb forms (e.g. \citealt{Fasske1981}). Colloquial Upper Sorbian (henceforth: CUS), by contrast, only possesses the distinction between perfective and imperfective aspect.  
In her extensive description of CUS, \citet{Scholze2008} devotes a whole chapter to the category of verbal aspect, see also \citet{Scholze23}. \citet{Breu2000} also discusses the functioning of perfective and imperfective forms in CUS in some detail. Unless indicated otherwise, the examples that I discuss in this paper are taken from these two sources.  

What is \textit{not} special about Upper Sorbian 
is the formal coding of the two aspectual categories. 
To this end, CUS exploits the same system of stem alternations as other 
Slavic languages do (\citealt[248]{Breu2012}). 
As in, for instance, Russian  (e.g. \citealt{AG1980}) or Czech (e.g. \citealt{Karlik95}), 
unprefixed verbs are mostly imperfective. When they are prefixed, they
become perfective. When a secondary imperfective suffix is attached, the formerly perfective verb becomes imperfective and, additionally, we also find suppletive pairs.\footnote{This is, of course, a very simplified summary. Since questions of morphology lie outside of the goals of the present paper, however, it should suffice for our purposes here. See \citet{Werner2003} for a much more detailed picture of Upper Sorbian verbal affixation.}  

(\ref{bsp}) shows examples from CUS (\citealt[230]{Scholze2008}):\footnote{Not surprisingly, Standard Upper Sorbian aspectual morphology works in the same way, see \citet[230]{Scholze2008}.} 

 \ea\label{bsp}
\ea \textit{f\"onwa\'c} (\textsc{ipfv}) -- \textit{sf\"onwa\'c} (\textsc{pfv}) `blow-dry'
\ex \textit{wotu\'ci\'c} (\textsc{pfv}) -- \textit{wotu\'cwa\'c} (\textsc{ipfv}) `wake up'
\ex \textit{kipwa\'c} (\textsc{ipfv}) -- \textit{kipn\'c} (\textsc{pfv}) `tip' 
\z 
\z

%(5) a. kry-t’IPF b. ot-kry-t’PF c. ot-kry-va-t’IPF d. za-ot-kry-va-t’PF
%cover-INF away-cover-INF away-cover-SI-INF behind-away-cover-SI-INF
%‘cover’ ‘open’ ‘open’ ‘start opening’

\noindent What sets CUS aside from the other Slavic languages is the way perfective and imperfective verb forms are used. In many respects, CUS resembles Czech (\citealt[45]{Breu2000}). Consider the following examples:

 \ea\label{uspl}
\gll \textit{W\'on}
\textit{je} \textit{husto} \textit{jenož}
\textit{jednu} \textit{knihu} \textit{předa\l.}\\
he \textsc{aux} often only one book sell.\textsc{pst.\textcolor{black}{pfv}} \\
\glt \normalsize{`He often sold only one book.'}\hfill \small{[CUS]}
\z 
 \ea\label{czpl}
\gll \textit{\v{C}asto}
\textit{prodal} \textit{jen}
\textit{jednu} \textit{knihu.} \\
often sell.\textsc{pst.\textcolor{black}{pfv}} only one book \\
\glt \normalsize{`He often sold only one book.'}\hfill \small{[CZ]}
\z 

\noindent As with speakers of Czech, speakers of CUS will choose a perfective form\footnote{Throughout this text I will speak of "perfective forms" in CUS as if the language had a grammatical category signalling perfectivity understood sensu stricto, i.e. as completed event denotation. As will become clearer soon, however, it has not. My decision might cause confusion, but I think it would be even more confusing if I chose a different label for those forms that have perfective morphology from the point of view of other Slavic languages.} to convey the message that "he often sold only one book", whereas speakers of Russian would select the imperfective form in this context: 

 \ea\label{ele}
\gll \textit{On}
\textit{\v{c}asto} \minsp{\{} \textit{prodaval} / \minsp{*} \textit{prodal}\} 
\textit{tol'ko} \textit{odnu} \textit{knigu.} \\
he often {} sell.\textsc{pst.\textcolor{black}{ipfv}} {} {} sell.\textsc{pst.pfv} only one book \\
\glt \normalsize{`He often sold only one book.'}\hfill \small{[RU]}
\z 

\noindent In other respects, however, CUS is known to be the odd one out in showing ``unslavic" use of aspectual forms. The most striking fact (from the point of view of what one would expect from a ``well-behaved" Slavic language; \citealt[54]{Breu2000}) is perhaps the possibility of referring to an ongoing event by means of a perfective verb:

 \ea\label{usong}
\gll \textit{Jurij} \textit{jo} \textit{rune} \textit{jen} \textit{text} 
\textit{še\l{o}ži\l,} \textit{hdyž} \textit{sym} \textit{ja}
\textit{nutř}
\textit{šišo\l.}\\
J. \textsc{aux} now a text translate.\textsc{pst.pfv} when \textsc{aux} I in come.\textsc{pst.pfv}\\
\glt \normalsize{`When I came in, Jurij \textcolor{black}{was translating} a text.'}
\z


\noindent That sentence (\ref{usong}) can be interpreted so that the translation event temporally includes the moment when the speaker enters, stands in sharp contrast to what its Czech or Russian counterparts would allow for, cf. (\ref{elef}) and (\ref{rutax}). Pay attention to 
the translations enforced by the use of the perfective forms \textit{přeložil} and \textit{perevel}, respectively.

 \ea\label{elef}
\gll \textit{Když}
\textit{jsem} \textit{přišel,}
\textit{přeložil}
\textit{jeden} \textit{text.} \\
when \textsc{aux} come.\textsc{pst.pfv} translate.\textsc{pst.pfv} one text \\
\glt \normalsize{`When I came, he \textcolor{black}{had translated} a text.'}\hfill \small{[CZ]}
\z 
 \ea\label{rutax}
\gll \textit{Kogda}
\textit{ja} \textit{prišel,}
\textit{on} \textit{perevel} \textit{tekst.} \\
when I come.\textsc{pst.pfv} he translate.\textsc{pst.pfv} text \\
\glt \normalsize{`When I came, he \textcolor{black}{had translated} a text.'}\hfill \small{[RU]}
\z 

\noindent The phenomenon is well-known in the literature dedicated to verbal aspect in Upper Sorbian. Apart from \citet{Breu2000} and \citet{Scholze2008}, documentation and discussion can also be found in \citet{Werner2003, Werner2013}. The following is quoted from \citet[43]{Werner2003}.

\ea\label{usong2}
\gll \textit{Sym} \textit{runje} \textit{při} \textit{tym,} \textit{kr\'otke} \textit{powědan\v{c}ko}
\textit{prže\l{o}ži\'c.}\\
\textsc{aux} now at that short tale translate.\textsc{inf.pfv}\\
\glt \normalsize{`I am at the moment engaged in translating a short story.'}
\z

\noindent Within the present paper, the possibility of using perfectives to refer to ongoing events will be the main focus of interest, but let me add two other contexts in which CUS shows remarkable (in the sense of ``unslavic") aspect use.
Both are well discussed in the relevant literature. 

First, CUS has a compound future with a perfective infinitive (\citealt[637]{Stone1993}), as illustrated in (\ref{budu}) from \citet{Scholze2008}. Secondly, there is no ban on perfective infinitives after phase verbs, as can be seen in (\ref{phase}) from \citet{Breu2012}. 

\ea\label{budu}
\gll \textit{A} \textit{potom} \textit{budu} \textit{jej} \textit{pokaza\'c!}\\
and then \textsc{aux} her show.\textsc{inf.pfv}\\
\glt \normalsize{`And then I'll show her!'}
\z

\ea\label{phase}
\gll \textit{T\'on} \textit{jo} \textit{zapo\v{c}a\l{}} \textit{jowo} \textit{za\'ciš\'ce} \textit{napisa\'c.}\\
he \textsc{aux} start.\textsc{pst.pfv} his impressions write.\textsc{inf.pfv}\\
\glt \normalsize{`He began to write down his impressions.'}
\z



\section{Some theoretical background assumptions}\label{mueller:sec:background}
In what follows, I will revisit the data on aspect in CUS and 
discuss the analyses that have been presented so far to account for these data. Jumping ahead, I will argue for a theoretical account that follows the intuitive explanation suggested in \citet{Toops2001}. My proposal will build on three background assumptions. First, I consider the category of aspect to relate reference times and event times (\citealt{Klein1994}).  Second, following \citet{Gronn2004}, I take the imperfective operator 
IPF to introduce a radically underspecified aspectual relation; all that an imperfective form requires is that the reference time overlaps with the event time, whatever the overlap relation will ultimately look like. Third, I assume that different Slavic languages may vary with respect to the content of the perfective category, i.e. with respect to the precise truth-conditional impact of perfectivity on interpretation (\citealt{omr18}). 

Informally, perfective aspect in Russian and other East Slavic languages has been characterised as expressing ``connectedness", which roughly means that the event encoded by a perfective verb will always have to be understood as grounded (``connected") within the chain of particular events preceding and following it (\citealt{Barentsen95, Barentsen98, mue:Dickey2000, Dickey15, Stunova1991, Stunova1993}). \citet{Gronn2004} offers a way to model the intuition behind connectedness in truth-conditional terms. He proposes that, in the case of Russian perfectives, the reference time has to end when the target state of the event is in force. 

Let me elaborate on that.
For a state to be in force at a given moment $t$, it has to be valid immediately before $t$ and immediately after $t$.
Now let $t$ be the final moment of the reference time interval introduced by a Russian perfective, as proposed in \citet{Gronn2004}.
It follows that there has to be a state \textit{after} the reference time interval connected to the state inside the reference time. The kind of connection between the two is trivial: they are parts of one and the same state. In other words, a Russian perfective will almost always\footnote{``Almost always" because there is a systematic exception. Perfectives formed by the delimitative prefix \textit{po-} denote no change-of-state (\citealt{mue:Filip2000}, \citealt{Dickey2006}). In these cases, where the prefix carves out a chunk of the process delivered by the verbal base, connectedness is trivially given because the reference time cannot end but when the process is in force.} 
denote a state-changing event, at the same time entailing the existence of an eventuality subsequent to the denoted change of state. Typically, the subsequent eventuality is a state, which in turn may provide the occasion for further events in the course of the world talked about. The event introduced by a perfective may, however, also be ``connected'' to a process, as with ingressive verbs like \textit{pojti} `start going', for instance. In any event, a Russian perfective will always have to be interpreted as embedded within a chain of eventualities.

Compared to the situation in Russian, the perfective category in Czech has a weaker content. 
What is required by a perfective here is that the event is referred to in its totality (\citealt{Stunova1991, Stunova1993}). Formally, this can be modelled by making use of Filip's maximality operator (\citealt{Filip2008, Filip17}). In Czech, then, perfectives impose the condition that the reference time has to include the endpoint of the event (``maximality requirement''), but not that the time of the eventuality brought about by it was partially covered by the reference time. 

Now how about Sorbian, more precisely, Colloquial Upper Sorbian (CUS)? In (\ref{usong}), we already saw that perfectives in CUS do not require the reference time to include the endpoint of the event. Thus, perfectivity in CUS seems to be an even weaker semantic notion compared to perfectivity in Czech, which, as we saw, is semantically weaker than perfectivity in Russian. 

\section{Terminativity}\label{mueller:sec:term}
\citet{Breu2000, Breu2012} and \citet{Scholze2008} provide an insightful discussion of how speakers of CUS distribute aspectual forms over contexts. When interpreting the presented facts, the authors arrive at the conclusion that a perfective form will be used in CUS if the sentence predicate specifies an ``inherent goal", independently of whether the goal will be reached. 
%\item being perfective means being an accomplishment/achievement 
Here is \citeposst{Scholze2008} definition (own translation from German):\footnote{A more precise definition is not offered. The term ``terminative" seems to be used interchangeably with the terms ``telic" and ``bounded" (cf. \citealt[264]{Breu2012}).} 

\begin{quote}
Terminativity, i.e. the existence of a goal of the action, no matter
whether it is reached or not, is verbalised in CUS by the perfective aspect.
In contrast, the imperfective aspect expresses aterminativity,
i.e. actions without an 
inherent goal. 
(\citealt[232-233]{Scholze2008})\footnote{``Terminativit\"at, d.h. das Vorhandensein eines Ziels der Handlung, 
gleichg\"ultig ob dieses er\-reicht wird oder nicht, wird in der SWR [= CUS]
durch den perfektiven Aspekt versprachlicht. Im Gegensatz hierzu dr\"uckt 
der ipf. Aspekt Aterminativit\"at aus, 
also Handlungen ohne inh\"arentes Ziel" 
(\citealt[232-233]{Scholze2008}).} 
\end{quote}

The proposal appears to be simple and straightforward.
It predicts the non-use of perfective aspect when reference is made to single events on the basis of atelic predicates, i.e. of predicates not specifying an inherent goal. The following would be a case in point: 

 \ea\label{blid}
\gll \textit{Hdyž} \textit{hromadz\'e} \textit{za} \textit{blidom} \textit{sed\'zeštaj,}
\textit{hladaštaj} \textit{sej} \textit{do} \textit{wo\v{c}ow.}\\
when together at table sit.\textsc{3du.pst.ipfv} look.\textsc{3du.pst.ipfv} \textsc{refl} to eyes\\
\glt \normalsize{`As the two sat together at the table, they looked each other in the eye.'}
\z

\noindent At the same time, the possibility of perfectives expressing ongoingness, which was illustrated in (\ref{usong}), is directly accounted for. (\ref{usong3}) shows a further example of that kind: the predicate is telic/terminative, so verbal aspect is perfective. That the event has not yet finished at the end of the reference time is no obstacle for using the perfective.

 \ea\label{usong3}
\gll \textit{W\'on} \textit{rune} \textit{jenu} \textit{kniw} \textit{šeda.} \\
he now a book sell.\textsc{prs.pfv}\\
\glt \normalsize{`He is now selling a book.'}
\z

\noindent \citet[240]{Scholze2008} reports that the imperfective form
\textit{f\"onwe} `blow-dry' in (\ref{foen1}) may be replaced by 
the prefixed perfective \textit{sf\"onwe}, as shown in (\ref{foen2}), but that this replacement induces a meaning shift. Her informants state that, unlike (\ref{foen1}), which describes the pure activity of blow-drying%dry-blowing
, 
(\ref{foen2}) is about an event which is directed toward the goal of having dry hair. One easy way of accounting for this intuition is to associate perfective aspect with the existence of an inherent goal of the action. 

 \ea\label{foen1}
\gll \textit{Na} \textit{t\'on} \textit{so} 
\textit{wěš\'ci} \textit{f\"onwe}
\textit{něke.} \\
well he \textsc{refl} surely blow-dry.\textsc{prs.ipfv}
now\\
\glt \normalsize{`He surely is blow-drying his hair right now.'}
\z

 \ea\label{foen2}
\gll \textit{Na} \textit{t\'on} \textit{so} 
\textit{wěš\'ci} \textit{sf\"onwe}
\textit{něke.} \\
well he \textsc{refl} surely blow-dry.\textsc{prs.pfv}
now\\
\glt \normalsize{`He surely is blow-drying his hair right now (until it will be dry).'}
\z

\noindent Successful as it seems at first sight, the proposal that perfectivity in CUS should be analysed as terminativity runs into problems. 
Look at the following example. Note that the predicate clearly specifies a goal. The goal will be reached when a series of events of the same kind (folding a piece of cloth) will have been performed over a limited set of objects (pieces of cloth):

 \ea\label{sowaesch}
\gll \textit{Ja} \textit{k\l{a}du} \textit{rune} \textit{t\'on} \textit{wešu}
\textit{hromadze.} \\
I put.\textsc{prs.ipfv} now the laundry together\\
\glt \normalsize{`I'm just folding the laundry.'}
\z

\noindent In this case, although the predicate describes an event which has an inherent goal (that the laundry is folded), the imperfective is used, and not the perfective, as one might have expected. 



Example (\ref{so5}) shows a similar case, which is discussed in \citet{Breu2000}. 

 \ea\label{so5}
\gll \textit{Ja} \textit{šedawam} \textit{rune} \textit{pe\'c} \textit{knijow.} \\
I sell.\textsc{prs.ipfv} now five books\\
\glt \normalsize{`I'm selling five books now.'}
\z

\noindent Here, too, the imperfective form is used despite the fact that the reported event has an ``inherent goal" (that five books are sold). Noticing that examples like these run counter to his generalisation, \citet[64]{Breu2000} writes that ``[a] certain complication arises from that the ipf. seems to have another function apart from the expression of aterminativity, namely the expression of distributivity".

This ``complication" is not the only problem that arises from identifying perfectivity with terminativity in CUS. Another one is that the proposal makes the wrong prediction for perfective generics. Consider the following, where the predicate is aterminative, but the perfective is used nevertheless.

 \ea\label{sopfgen}
\gll \textit{T\'on} \textit{basne} \textit{chětř} \textit{nawukne.} \\
he poems quickly learn.\textsc{prs.pfv}\\
\glt \normalsize{`He quickly learns poems.'}
\z

\noindent Learning a poem is an activity which aims at a specific goal. 
The predicate in (\ref{sopfgen}) is about learning poems, however, and \textit{this} activity has no specific goal, or telos. The addition of ``quickly" does not change that: learning poems quickly is still atelic. Despite that, the verb form used in (\ref{sopfgen}) is not imperfective, but perfective, which needs to be explained.    

Interestingly, we also find the mirror image to (\ref{sopfgen}), i.e. sentences where the predicate is telic/terminative, but aspect is imperfective. Consider 
the predicate in (\ref{soipfpoem}). It describes an event that has an inherent goal (that the poem is learned). If Breu's and Scholze's proposal was correct, we would expect a perfective verb form. The verb form that appears in (\ref{soipfpoem}) is, however, imperfective. 

 \ea\label{soipfpoem}
\gll \textit{T\'on} \textit{wukne} \textit{rune} \textit{t\'on} \textit{basejn.} \\
he learn.\textsc{prs.ipfv} now the poem\\
\glt \normalsize{`He is just learning the poem.'}
\z

\noindent To conclude so far, analysing perfective verb forms in CUS as expressing terminativity can account for many, but not all data. Let us therefore move on to see in how far the proposal made by \citet{Toops2001} fares better than Breu's and Scholze's.   

\section{Determinateness}\label{mueller:sec:det}
According to \citet{Toops2001}, perfectives in CUS encode the notion of determinateness, by which he means a one-time action heading for a goal: ``Determinate forms denote a goal-oriented action occuring either once or irregularly" (\citealt[132]{Toops2001}). 
Toops' perfectivity condition is more specific than the one of \citet{Breu2000} and \citet{Scholze2008}. Roughly speaking, determinateness is terminativity (``goal-oriented") plus uniqueness (``once").
%\footnote{Here I ignore the second possible license for a perfective, i.e. that the event occurs more than once in irregular distribution. XXXX Later come back??? XXX}  
As we will see now, \citet{Toops2001} can indeed explain aspect choice in CUS for almost all contexts of use.  

The first thing to note is that \citet{Toops2001} can account for those data points that Breu and Scholze can also account for. Recall example (\ref{usong}). The denoted event is not completed, but it is goal-oriented and single, and this alone suffices for licensing the perfective.

% \ea\label{usong3n}
%\gll \textit{W\'on} \textit{rune} \textit{jenu} \textit{kniw} %\textit{šeda.} \hfill
%(=(\ref{usong3}))\\
%he now a book sell.\textsc{prs.pfv}\\
%\glt \normalsize{`He is now selling a book.'}
%\z

Moving on to (\ref{sodisrep}), the imperfective version of (\ref{usong3}), where the non-use of the perfective implies that either the condition of goal-orientation, or the condition of singularity is not fulfilled (or that both are not). This is indeed the case because the direct object \textit{jenu kniw} `a book' is to be interpreted as type-referring (\citealt[69]{Rakhilina2000}).
The sentence describes several events of selling a copy of the same (kind of) book, leaving open the precise number of events. 
Due to the absence of a goal to which the repetition of book-sellings would be directed, the example is in line not only with \citet{Breu2000,Breu2012} and \citet{Scholze2008}, but also with  \citet{Toops2001}. Both proposals correctly predict the use of an imperfective verb form.  

 \ea\label{sodisrep}
\gll \textit{W\'on} \textit{rune} \textit{jenu} \textit{kniw} \textit{šedawa.} \\
he now a book sell.\textsc{prs.ipfv}\\
\glt \normalsize{`He is now selling a book (copy by copy).'}
\z

\noindent Next, consider (\ref{sogenrep1}). This example is in harmony with interpreting perfectivity in CUS as terminativity, as do Breu and Scholze, because the predicate is aterminative (it is generic). It is also in line with linking the use of a perfective to the two properties of being terminative (``goal-oriented") and being singular (``one-time"), as in \citeposst{Toops2001} account. 

\ea\label{sogenrep1}
\gll \textit{W\'on} \textit{šedawa} \textit{knije.}\\
he sell.\textsc{prs.ipfv} books\\
\glt \normalsize{`He sells books (= is a bookseller).'}
\z

\noindent As a very nice minimal pair, \citet[59]{Breu2000} provides (\ref{soapplrep}) and (\ref{soapplrep1}):

 \ea\label{soapplrep}
\gll \textit{Dyš} \textit{sem} \textit{ja} \textit{do} \textit{lodna}
\textit{šiš\l{a},} \textit{jo} \textit{sej} \textit{wona} \textit{rune} \textit{jabuka}
\textit{bra\l{a}}. \\
when \textsc{aux} I to shop come \textsc{aux} \textsc{refl} she now apples take.\textsc{pst.ipfv}\\
\glt \normalsize{`When I came to the shop, she was taking apples.'}
\z

 \ea\label{soapplrep1}
\gll \textit{Dyš} \textit{sem} \textit{ja} \textit{do} \textit{lodna}
\textit{šiš\l{a},} \textit{jo} \textit{sej} \textit{wona} \textit{rune} \textit{jabuka}
\textit{za\l{a}}. \\
when \textsc{aux} I to shop come \textsc{aux} \textsc{refl} she now apples take.\textsc{pst.pfv}\\
\glt \normalsize{`When I came to the shop, she was taking a pack of apples.'}
\z

\noindent Both examples are identical, except for the fact that the first one ends in an imperfective, whereas the second one ends in a perfective verb. This difference results in the following difference in interpretation. (\ref{soapplrep}) is about the taking of an unspecific set of apples. By contrast, (\ref{soapplrep1}), the perfective version, is understood such that what is being taken is a definite set of apples, most likely a package of apples.\footnote{Note that both main predicates are translated into English by means of the progressive (`was taking').} 

\citet{Breu2000} argues that this contrast provides support for his assumption that CUS perfective forms express terminativity, because packaging the apples into a single unit furnishes the predicate with an inherent goal (that the apples are taken). \citet{Toops2001} could build on this to argue that interpreting the apples as a single unit not only leads to terminativity, but also to understanding the apple-taking as a single event. 

So far, we have looked at data that can be accounted for by appealing to terminativity as well as by appealing to determinateness for perfectivity in CUS. If that was all, the former kind of approach would be preferable because, as we saw, terminativity is a weaker semantic notion than determinateness. However, there are examples that the terminativity approach 
cannot explain, but the determinateness approach can. We already came across such examples above in (\ref{sowaesch}) and (\ref{so5}). Let me repeat one of them for ease of exposition:  

  \ea\label{sowaeschrep}
\gll \textit{Ja} \textit{k\l{a}du} \textit{rune} \textit{t\'on} \textit{wešu}
\textit{hromadze.} (=(\ref{sowaesch}))\\
I put.\textsc{prs.ipfv} now the laundry together\\
\glt \normalsize{`I'm just folding the laundry.'}
\z

\noindent Cases like these represent the ``complications" that force \citet{Breu2000} to assume that imperfective forms not only express aterminativity, but also distributivity. Now we can convince ourselves that, if we follow \citet{Toops2001}, the complication disappears. Since distributivity is a manifestation of pluractionality (\citealt{anamuller20}), and since perfectives are linked to singularity according to the proposal made in \citet{Toops2001}, the non-use of the perfective is correctly predicted. 

Above I have argued with respect to (\ref{sodisrep}) that the use of the perfective is excluded because of the missing goal of the predicate. Now we can add that also Toops' second condition for the use of a perfective, singularity, is not met in (\ref{sodisrep}), due to the pluractionality of the predicate.  


There are, nevertheless, two examples figuring in the discussion about aspect use in CUS that 
\citet{Toops2001}, as it seems, cannot handle. 
The first issue arises in connection to the following generic sentence:

  \ea\label{sopfgenrep}
\gll \textit{T\'on} \textit{basne} \textit{chětř} \textit{nawukne.} (=(\ref{sopfgen})) \\
he poems quickly learn.\textsc{prs.pfv}\\
\glt \normalsize{`He learns poems quickly.'}
\z

\noindent The problem here is that by uttering (\ref{sopfgenrep}) 
the speaker does not refer to a unique event (note the plural form of the direct object). According to the logics of \citet{Toops2001}, this excludes the use of a perfective verb form, yet the verb \textit{is} perfective.  

A second apparent counterexample is 
(\ref{soipfpoemrep}):

 \ea\label{soipfpoemrep}
\gll \textit{T\'on} \textit{wukne} \textit{rune} \textit{t\'on} \textit{basejn.} (=(\ref{soipfpoem})) \\
he learn.\textsc{prs.ipfv} now the poem\\
\glt \normalsize{`He is learning the poem.'}
\z

\noindent This is not a generic, but an episodic sentence which is used to refer to an ongoing 
event. Since the event is a singleton and oriented towards a goal (that the poem is learnt), the choice of a perfective is licensed on the account of \citet{Toops2001}, but the verb actually used is \textit{not} perfective. 

In what follows I want to propose a modification of 
\citet{Toops2001}, or rather a specification of that theory. 
My aim is to integrate the counterexamples presented above into an overall approach in keeping with the spirit of the proposal, and to show that this will give us an account that captures the data correctly. 


\section{Proposal}\label{mueller:sec:proposal}
Let us define determinateness by exploiting the notion of a path (\citealt{Gehrke2008, Krifka1998, Zwarts2005}). 
%$\mathbb{P}$ $\pi$

 \ea\label{defdet}
An event predicate $P$ is \textsc{determinate} iff it is unidimensional, directed and bounded, whereby:
\ea\label{defuni}
$P$ is \textsc{unidimensional} iff the events in its denotation set share a path structure such that all elements of this path structure are parts of a common path which likewise belongs to the part structure.\\ (``all paths are parts of a common path")
\ex\label{defdir}
$P$ is \textsc{directed} iff the events in its denotation set share a path structure such that there are no two non-overlapping elements in this path structure that occupy the same space. \\
(``no return to a formerly traversed region")
\ex\label{defbound}
$P$ is \textsc{bounded} iff the events in its denotation set share a path structure which includes an element that cannot be concatenated by another element of the same path structure such that the resulting path likewise belongs to this path structure.\\
(``there is a maximal path")
\z
\z

\noindent With this definition in mind, we will now reconsider the examples presented above. My claim is that perfective verbs are used in CUS in those cases where the predicate is determinate in the sense of (\ref{defdet}). For ease of exposition I will not refer back to the initial presentation of an example, but instead repeat it with new numbering.  

 \ea\label{sodisreprep}
\gll \textit{W\'on} \textit{rune} \textit{jenu} \textit{kniw} \textit{šedawa.} \hfill
(=(\ref{sodisrep}))\\
he now a book sell.\textsc{prs.ipfv}\\
\glt \normalsize{`He is now selling a book (copy by copy).'}
\z

\noindent The non-use of the perfective in (\ref{sodisreprep}) can be explained by the fact that the predicate does not meet the criteria for being determinate. It does not because the 
paths belonging to different individual selling events do not have a common path. So (\ref{sodisreprep}) violates (\ref{defuni}), the condition of unidimensionality. What has been argued here with respect to (\ref{sodisreprep}) generalises to all cases of event repetition, i.e. to all predicates that describe a plurality of events of the same kind.\footnote{A reviewer wonders whether this ``generalisation" also covers distributive readings with plural subjects performing the same kind of action at the same time along different paths. As I understand it, the question boils down to the issue of whether or not the subject expression belongs to the event predicate. I will have to leave this difficult though important topic for future research.}  

In contrast to (\ref{sodisreprep}), (\ref{usong3reprep}) is about a single selling-event. The noun phrase 
\textit{jenu kniw} `a book' is understood as referring to a 
particular book as the article being sold. 

 \ea\label{usong3reprep}
\gll \textit{W\'on} \textit{rune} \textit{jenu} \textit{kniw} \textit{šeda.} \hfill(=(\ref{usong3}))\\
he now a book sell.\textsc{prs.pfv}\\
\glt \normalsize{`He is now selling a book.'}
\z

\noindent Since reference is made to a single event, there is no violation of unidimensionality. Moreover, there is no reason to assume that an event of selling a particular book would traverse along a path that runs through the same space twice. Thus, there is no violation of directedness. Finally, such an event has an inherent goal as it develops along a maximal path. A path is maximal, or upper-bounded, if it cannot be concatenated by a subpath of itself without the product of this concatenation falling outside of the path structure of the predicate. Accordingly, there is also no violation of boundedness. We may therefore attest that the predicate is determinate. This predicts the use of a perfective form, and this is what we observe. 

So far, so good. Let us now move on to those examples that have turned out to be problematic for \citet{Toops2001}. 
One of them was (\ref{soipfpoemrep}), repeated here: 

  \ea\label{soipfpoemreprep}
\gll \textit{T\'on} \textit{wukne} \textit{rune} \textit{t\'on} \textit{basejn.} (=(\ref{soipfpoemrep}))\\
he learn.\textsc{prs.ipfv} now the poem\\
\glt \normalsize{`He is learning the poem.'}
\z

\noindent The predicate in (\ref{soipfpoemreprep}) describes a one-time, goal-oriented event. According to \citet{Toops2001},
this should license the use of the perfective form. The form actually used is imperfective, however.

The extension, or specification, of the approach of \citet{Toops2001} that I have proposed above solves this problem. Recall that under my definition of determinateness, the described event has to be not only unidimensional ($\approx$ ``one-time") and bounded ($\approx$ ``goal-oriented"), but also directed. Directedness is thereby defined such that there must not exist two or more pieces (subpaths) of the overall event path that would occupy the same region. Very loosely speaking: at no time during the course of the event should the event return to where it was before.\footnote{I believe that what \citet{Toops2001} had in mind when speaking about ``goal-oriented action" was actually an amalgam of directedness and boundedness. This is why I consider my approach to be an elaboration rather than a correction of \citet{Toops2001}.} 

With respect to (\ref{soipfpoemreprep}), two facts need to be noted. First, the movement along a path associated with the learning of a poem is not motion in the physical sense, but rather a metaphorical motion along the words of the poem as they are considered by the learner. Second, the learning of a poem does not usually proceed linearly from the first word to the last word, but rather in cycles. If I want to memorise a poem, I may start by reading through the whole poem first, then I will perhaps return to the first verse, will read it again through, will maybe return to the first line, will reread it, and so on. As should be obvious, this kind of event structure violates directedness, and therefore the use of the perfective is \textit{not} to be expected, thus explaining the use of the imperfective in (\ref{soipfpoemreprep}).  

What remains to be discussed are the generic sentences that 
seem to run counter to the idea that perfectives express determinateness, at least if one takes determinate events to be ``one-time goal-oriented", as proposed in \citet{Toops2001}.
The knowledge about generics that has been accumulated is rich (\citealt{Cohen2022}; \citealt{Krifka1995}; \citealt{Leslie22}). 
Here I can obviously only scratch the surface. Generally speaking, two kinds of generics have been identified in the literature. The first one goes under the expression ``descriptive generics" (e.g. \citealt{Krifka13}), 
or ``inductivist generics" (e.g. \citealt{Carlson1995}).
The labels are motivated by the fact that the generalisation expressed by such a generic \textit{describes} the way certain individuals behave in the world, or that the behaviour of the individuals allows for \textit{inducing} the generalisation. Sentence (\ref{sogenrep}) is a case in point.    

  \ea\label{sogenrep}
\gll \textit{W\'on} \textit{šedawa} \textit{knije.} \hfill (=(\ref{sogenrep1}))\\
he sell.\textsc{prs.ipfv} books\\
\glt \normalsize{`He sells books (= is a bookseller).'}
\z

\noindent Whatever the ultimate analysis of these generics, it is widely agreed that their meaning involves quantification over events, and that the generalisation ex\-pressed by them results from quantifying over ``sufficiently many" (\citealt{Cohen2022}) such events. 
Disagreement concerns the question as to what should count as ``sufficiently many" and how to model this factor (see \citealt{Krifka1995} and \citealt{Cohen2022} for surveys of different types of inductivist approaches). 

In light thereof, the generalisation conveyed by uttering (\ref{sogenrep}) is based on the observation that `he' has acted as seller in sufficiently many events of selling a book. Since the meaning of the predicate \textit{šedawa knije} `sells books' accordingly entails more than one book-selling event, we may note that unidimensionality is violated in (\ref{sogenrep}). This leads us to expect the imperfective form, in line with the facts.

    \ea\label{sopfgenreprep}
\gll \textit{T\'on} \textit{basne} \textit{chětř} \textit{nawukne.} \hfill (=(\ref{sopfgenrep}))\\
he poems quickly learn.\textsc{prs.pfv}\\
\glt \normalsize{`He learns poems quickly.'}
\z

\noindent (\ref{sopfgenreprep}) shows an instance of the second kind of generic sentences. In cases like these, 
which I refer to as ``dispositional generics"
in \citet{omr20concealed}, 
the predicate describes an event which the subject referent is expected to be able to perform given the need to do so. Dispositional generics are logically linked to ``definitional generics" (\citealt{Krifka13}; \citealt{Seres2019}) or ``in virtue-of generics" (\citealt{Greenberg2003}), since recently also known as ``normative generics" (\citealt{Hesni2022}, \citealt{Leslie22}).  
The following Polish examples, which are taken from \citet{Klimek2008}, may serve as illustration. (\ref{truefriend}) is a definitional generic.

\ea\label{truefriend}
\gll \textit{Przyjaciel} \textit{pomo\.{z}e} \textit{w} \textit{potrzebie.}\\
friend help.\textsc{prs.pfv} in need\\
\glt \normalsize{`A friend will help in need.'}  \hfill{[PO]}
\z

\noindent For the purposes of the present paper, we may safely ignore details of ongoing discussion about the best analysis of definitional generics. Let us simply note that a definitional generic will be uttered to express that members of the kind named by the subject nominal have the property described by the predicate because they have a principled connection to the kind named by the subject nominal (\citealt{PrasadaDillingham2006}).
Thus, 
(\ref{truefriend}) expresses that one who qualifies as member of the kind/category `friend' has the property of helping you in case that help is needed \textit{because} he is a friend.  

Related to the definitional generic (\ref{truefriend}) is the dispositional generic (\ref{Janek}), which says that Janek has the property of helping in case that help is needed. 
Given the truth of (\ref{truefriend}), (\ref{Janek}) silently conveys the additional message that Janek is one who deserves being counted as a friend (\citealt{Klimek2008, Klimek2012, omr20concealed}). 

\ea\label{Janek}
\gll \textit{Janek} \textit{pomo\.{z}e} \textit{w} \textit{potrzebie.} \\
Janek help.\textsc{prs.pfv} in need\\
\glt \normalsize{`Janek will help in need.'}  \hfill{[PO]}
\z

\noindent Returning to Sorbian, the generic sentence (\ref{sopfgenreprep}) is a dispositional generic, like (\ref{Janek}). Just like Janek is said to help in case help is needed, `he' in (\ref{sopfgenreprep}) is said to learn a poem quickly in case a poem needs to be learned. And similar to the way (\ref{Janek}) communicates the additional information that Janek deserves being called a friend, (\ref{sopfgenreprep}) silently coarticulates that `he' qualifies for being called a remarkably smart person.

Importantly, since the predicate in (\ref{sopfgenreprep}) is about a single learning of a poem relative to the modal context providing 
the task of learning a poem, 
just a single event is in question, and the condition of unidimensionality is satisfied.
Now, what about directedness and boundedness? Above I said that the path in \textit{learning a poem} goes in cycles, and that this violates directedness, implying the exclusion of perfective aspect.
If so, why is directedness not violated in (\ref{sopfgenreprep})?

The reason is, arguably, the impact of the 
adverb \textit{chětř} `quickly', which triggers an abstraction away from the ``real" profile of the path to the temporal distance between the initial point of the path $p(0)$ and the final point of the path $p(1)$.
It seems plausible to analyse \textit{chětř} as the focus-bearing constituent in (\ref{sopfgenreprep}).
Given this, 
the sentence will be understood as an answer to the implicit question about the speed at which `he' learns poems. 

Speed is distance divided by time ($v=s/t$).
The adverb \textit{chětř} `quickly' denotes a property of how much time it takes to memorise a poem, i.e. to move along a path from not knowing any word ($p(0)$) to knowing the whole text ($p(1)$). The less time is used to proceed on this path, the faster the learning of the poem; the faster the learning of the poem, the shorter the path of the event of learning the poem.
Figure \ref{schnell} shows different speeds compared with each other. 

\begin{figure}
% \begin{center}
% \fbox{
\begin{tikzpicture}
\draw[->] (0, 0) -- (0, 4);
\draw[->] (0, 0) -- (4, 0);
\node at (0, 4.2) {$t$};
\node at (4.125, -0.25) {$s$};
\node at (3,-0.75) {$B$};
\node at (0,-0.75) {$A$};
\draw (3,3,0)node[right]{slow};
\draw (3,2,0)node[right]{normal};
\draw (3,1,0)node[right]{quick};
\draw (0, 0) -- (-0.2, 0);
\draw (0, 0) -- (0, -0.2);
\draw (3, 0) -- (3, -0.2);
%\foreach \x in {0,3}
%{
%\draw (\x,0) -- (\x,-0.2);
%\draw (\x,3) -- (\x,-0.2);
%\node at (\x,-0.75) {$s_x$};
%\draw (0, \x) -- (-0.2, \x);
%\node at (-0.5, \x-0.25) {$\x$};
%}%\coordinate (m) at (2,2);
%\draw (m) circle (0.5);
%\draw[fill] (m) circle (0.25);
%\draw[->] (0, 2) -- (2, 0);
%\draw[->] (2, 0) -- (3, 1);
\draw[-Latex] (0, 0) -- (3, 1);
\draw[-Latex] (0, 0) -- (3, 2);
\draw[-Latex] (0, 0) -- (3, 3);
\end{tikzpicture}
% }
% \end{center}
\caption{Speeds of learning a poem}
\label{schnell}
\end{figure}

\largerpage

Under the assumption that the adverbial \textit{chětř} is the focus constituent, the following propositions constitute the (simplified) set of alternatives relevant for interpretation:

\ea \label{altfoc} Focus alternatives to (\ref{sopfgenreprep}):
\ea that he learns poems very slowly
\ex that he learns poems slowly
\ex that he learns poems at normal speed
\ex that he learns poems quickly\label{altfocd}
\ex that he learns poems very quickly
\z
\z

\noindent One proposition out of these, namely (\ref{altfocd}), will be asserted as true if (\ref{sopfgenreprep}) is uttered. The set of alternatives in (\ref{altfoc}) implies a comparison of different speeds. This comparison presupposes a normalised distance $\overline{AB}$.

Crucial for the present discussion is the fact that the paths of the events compared with each other in the informational background of (\ref{sopfgenreprep}) all support directedness, as can be easily read from Figure \ref{schnell}. This should then also hold for the path of the event denoted by (\ref{sopfgenreprep}): there are no non-overlapping subpaths of this event path that would occupy the same region. 
Moreover, the paths depicted in Figure \ref{schnell} also support boundedness, as there is an upper-bound determined by $B$.  
Since the predicate of (\ref{sopfgenreprep}) satisfies boundedness, directedness, and unidimensionality, the use of perfective aspect is called for. 




\section{Formalisation}\label{mueller:sec:formalism}
The starting point of this paper was the hypothesis, argued for in the literature referred to above, 
that it is two different kinds of perfectivity that figure in the aspectual systems of Czech and Russian. While the Czech perfective category encodes maximality (reminiscent of the concept of ``totality" in the traditional literature), the Russian perfective encodes target state validity (loosely related to the traditional notion of ``resultativity"). Having investigated the use of perfective and imperfective verb forms in CUS, I conclude that this language introduces a third kind of perfectivity into the overall picture of aspect in Slavic. 

CUS features the weakest perfective category within the Slavic family, so weak indeed that the question arises as to whether it should be called ``perfective" at all. The constraint that the use of a CUS perfective form imposes on interpretation is merely that  
the event property has to be unidimensional, directed and bounded. These properties, summarised under the label ``determinateness'', are defined in terms of the path that the denoted event is described as traversing.
% traverse a single upper-bounded directed path!

Czech perfectives come with a more specific requirement.
Here, the denoted event is described as a maximal event. The notion of maximality may be defined mereologically in terms of event stages, but also temporally by requiring the reference time to include the final moment of the event.
The final moment is the moment at which the upper bound of the event path is reached. 
% traverse a single upper-bounded directed path to the end!

Russian has the most specific perfective category. By using a Russian perfective, the speaker refers not only to an event that has been fully realised (up to the upper bound of its path), but in addition to an eventuality (often a state) that the realisation of the event has brought about. The pragmatic effect is that the particular conditions of said event's successor are relevant for the further discourse. 
% traverse a single upper-bounded directed path to the end and make the consequences of it relevant to subsequent conversation

In the remainder of this paper, I want to integrate 
these observations and claims into the overall picture. To do so, 
I take for granted that in every Slavic language, the imperfective category is semantically underspecified (``unmarked"), whereas the perfective category comes with specific content (``marked"). To be concrete, I take the following to be the imperfective operator appearing in every Slavic language:  

    \ea\label{ipfoperator}
\textbf{Imperfective operator}:\\
$IPFV \Rightarrow  \lambda P \lambda t \exists e. P(e) \wedge t \bigcirc \tau(e)$ 
\z

\noindent As should have become clear, the specific content contributed by the perfective category may differ from language to language. (\ref{pfcus}) shows the perfective operator of CUS:

    \ea\label{pfcus}
\textbf{Perfective operator in CUS}:\\
$PFV_{CUS} \Rightarrow  \lambda P \lambda t \exists e . P(e) \wedge \cnst{DET}(P) \wedge t \bigcirc \tau(e)$ 
\z

\noindent In contrast to that, the perfective operator of Czech may formally be pinned down as in (\ref{pfczech}). 

    \ea\label{pfczech}
\textbf{Perfective operator in Czech}:\\
$PFV_{CZ\textcolor{white}{U}} \Rightarrow  \lambda P \lambda t \exists e . P(e) \wedge \cnst{DET}(P) \wedge t \bigcirc \tau(e) \wedge f_{end}(\tau(e)) \subseteq t$
\z

\noindent As for the Russian perfective operator, I propose the following:  

\ea\label{pfrussian}
\textbf{Perfective operator in Russian}:\\
$PFV_{RU\textcolor{white}{U}} \Rightarrow  \lambda P \lambda t \exists e . P(e) \wedge \cnst{DET}(P) \wedge t \bigcirc \tau(e) \wedge f_{end}(\tau(e)) \subseteq t \wedge f_{end}(t) \subseteq f_{target}(e) $
\z

\noindent Of course, for these formulae to be understandable, I have to state precisely what the property $\cnst{DET}$ is supposed to mean. The following summarises the informal discussion presented above: 

\ea\label{DET}
\textbf{Determinateness}:\\
$\forall P. \cnst{DET}(P) \leftrightarrow \cnst{UNI}(P) \wedge \cnst{DIR}(P) \wedge \cnst{BND}(P)$
\z

\noindent According to (\ref{DET}), a property will fulfill $\cnst{DET}$ if it fulfills $\cnst{UNI}$, $\cnst{DIR}$, and $\cnst{BND}$. So we move on to state the semantics of the latter three predicates:

%\ea\label{UNIold}
%\textbf{old-unidimensionality}:\\
%$\forall P. UNI(P) \leftrightarrow \forall e \forall e' \forall %p . P(e) \wedge e' \leq e \wedge \cnst{TRACE}(e') = p \rightarrow p %\leq \cnst{TRACE}(e)$
%\z

\ea\label{UNI}
\textbf{Unidimensionality}:\\
$\forall P. \cnst{UNI}(P) \leftrightarrow \forall e \forall e' \forall q . P(e) \wedge e' \leq e \wedge q = \cnst{TRACE}(e') \rightarrow \exists p . p = \cnst{TRACE}(e) \wedge q \leq p$
\z

\noindent This basically says, to repeat from above, that 
a predicate is unidimensional iff the events in its denotation set each have a path structure such that all paths within it are parts of a common path within it. 

\ea\label{DIR}
\textbf{Directedness}:\\
$\forall P. \cnst{DIR}(P) \leftrightarrow \forall e \forall e' \forall e'' \forall p \forall q . P(e) \wedge e' \leq e \wedge e'' \leq e \wedge \cnst{TRACE}(e') = p \wedge \cnst{TRACE}(e'') = q \wedge \neg (p \bigcirc q) \rightarrow \cnst{SPACE}(p) \neq \cnst{SPACE}(q)  $
\z

\noindent (\ref{DIR}) expresses that a predicate is directed iff the events in its denotation set each have a path structure such that there are no two non-overlapping paths within it that occupy the same space. 

%\ea\label{BNDold}
%\textbf{oldboundedness}:\\
%$\forall P. \cnst{BND}(P) \leftrightarrow \neg(\forall e \forall e' \forall e'' \forall p \forall q . P(e) \wedge e' \leq e \wedge e'' \leq e \wedge \cnst{TRACE}(e') = p \wedge \cnst{TRACE}(e'') = q \rightarrow \exists r \exists e'''. r = p + q \wedge r = \cnst{TRACE}(e''') \wedge e''' \leq e)$
%\z

\ea\label{BND}
\textbf{Boundedness}:\\
$\forall P. \cnst{BND}(P) \leftrightarrow \forall q \forall e . P(e) \wedge q \leq \cnst{TRACE}(e) 
\rightarrow \exists p . p \leq \cnst{TRACE}(e) \wedge \neg(p+q \leq \cnst{TRACE}(e))$
\z

\noindent According to (\ref{BND}), a predicate is bounded iff the events in its denotation set have a path structure which includes a path that cannot be concatenated by another path within it such that the resulting path would belong to the same path structure. 



\section{Paths and scales}\label{scalessec}
A reviewer asked me whether I understand paths as scales, and how my proposal relates to \citeposst{Kagan15} Scale hypothesis. I am grateful for these questions as they give me the opportunity to place my approach in a broader context. The answer to the first question is yes, I do. As for the second question, I will answer it carefully in the following short section. 

\citet{Kagan15} is concerned with the role of prefixation in the grammar of Russian. According to her, 
the semantic contribution of a prefix is such that it effectively fixes a point on a scale.
In the words of the author, the prefix ``imposes a relation between two degrees on a scale, one of which is a degree associated with the event denoted by the verbal predicate, and the other, the standard of comparison" (\citealt[24]{Kagan15}). The second degree, the one that serves as the standard of comparison, is contributed by external sources, that is to say, by means other than the prefix: ``the standard of comparison can be contributed either by a linguistic expression that appears in the sentence, or by the context" (\citealt[24]{Kagan15}).
The scale itself is supplied by the verbal predicate to which the prefix attaches, possibly including the direct object (\citealt[25]{Kagan15}). More precisely, it is contributed by a gradable property ``associated" (\citealt[26]{Kagan15}) with the verbal predicate. One and the same predicate may be ``associated" with different gradable properties, so that different prefixes applying to the same base may operate on different scales (provided by different gradable properties). Since the prefix operates on an independently given scale with an independently given degree that serves as standard of comparison, we may note that, in its core, \citeposst{Kagan15} approach boils down to the claim that prefixes, by relating the comparison degree to some specific degree on the scale, produce upper-bounded scales. 

The conclusion that prefixation amounts to creating upper-bounded scales has been arrived at by others as well (e.g. \citealt{Filip2008,Gehrke2008}). 
Upper-bounded scales are what I call bounded paths in the present paper.
I assume that Slavic-style aspect works with the two 
operators PFV and IPFV.\footnote{See \citet{omr23} for discussion of attempts to reduce the two covert operators to one.} These operators apply to verbal predicates. While IPFV is semantically underspecified in its input conditions, PFV calls for predicates that involve (upper-)bounded scales which are at the same time unidimensional and directed. Whenever at least one of these three conditions is not met, PFV will give way to IPFV. 
And here is now the link to \citet{Filip2008}, \citet{Gehrke2008}, and \citet{Kagan15}. Following these authors, I hold the view that 
prefixation produces event descriptions with upper-bounded scales.
Therefore, prefixed verbal predicates are well-prepared for serving as the input to PFV. Well-prepared, but not fully prepared, because boundedness is no sufficient condition for perfectivity alone. In addition to boundedness, perfectivity also entails directedness and unidimensionality. 
Whether or not unidimensionality and directedness are met in addition to boundedness can be read from the presence or absence of secondary imperfective morphology. 
Secondary imperfective markers on the predicate signal that either unidimensionality or directedness are not met.  

Pay attention to a certain flexibility which is built into \citeposst{Kagan15} theory. 
A verbal predicate will be ``associated" with one gradable property (and hence scale) or the other depending on the element in the context of which it appears, i.e. the prefix.  
In the next section, I will generalise this to other contextual elements in order to explain cases that otherwise would run counter to the predictions.  

\section{Two intricate cases}\label{mueller:sec:two}
Above I have argued for three notions of perfectivity, namely perfectivity as determinateness (the case of CUS), perfectivity as maximality (the case of Czech), and perfectivity as connectedness (the case of Russian). I suggested the following entailments to hold: (i) maximality implies determinateness, and (ii) target state validity (connectedness) implies maximality. While the latter seems uncontroversial, the former is surely not. 
Whether maximality really fully implies (that is: entails) determinateness is questionable. 

\citet[244]{Scholze2008} presents the following dialogue:

    \ea\label{dialogue}
\begin{itemize}
    \item[A:] \gll \textit{\v{S}to} \textit{w\'o} \textit{jow} \textit{\v{c}ini\'ce?} \\
what you.\textsc{du} here do.\textsc{prs.ipfv.du}\\
\glt \normalsize{`What are you two doing here?'}
\item[B:] \gll \textit{Ja} \textit{nawuknem} \textit{rune} \textit{ka} \textit{so} \textit{rajfn} \textit{wekslwe.} \\
I learn.\textsc{prs.pfv} now how \textsc{refl} tyre change.\textsc{prs.ipfv}\\
\glt \normalsize{`I am learning how to change a tyre.'}
\end{itemize}
\z

\noindent This is one more case where a perfective verb form, \textit{nawuknem} `am learning', is used to refer to an ongoing event (recall (\ref{usong}) from above). 
What is remarkable about (\ref{dialogue}) in the context of the present discussion is a comparison to (\ref{soipfpoemreprep}), repeated here one more time. 


  \ea\label{soipfpoemrepreprep}
\gll \textit{T\'on} \textit{wukne} \textit{rune} \textit{t\'on} \textit{basejn.} (=(\ref{soipfpoemreprep}))\\
he learn.\textsc{prs.ipfv} now the poem\\
\glt \normalsize{`He is learning the poem.'}
\z

\noindent With respect to (\ref{soipfpoemrepreprep}), \citet[245]{Scholze2008} notes that the replacement of imperfective \textit{wukne} with perfective \textit{nawukne} is excluded (``ausgeschlossen"). Why, then, is it not excluded in (\ref{dialogue})? Scholze writes that the use of perfective \textit{nawukne} to refer to an ongoing learning event will be possible only if the learner has the intention to acquire a certain skill.\footnote{``Terminative Prozessualit\"at setzt bei dem Lexem \textit{wukn\'c -- nawukn\'c} offensichtlich den beabsichtigten Erwerb einer F\"ahigkeit voraus" (\citealt[245]{Scholze2008}).} But this is hardly convincing, because one would not want to think that the learner in (\ref{soipfpoemrepreprep}) does not intend to acquire the skill to recite the poem. What, then, is the critical difference between the contexts (\ref{dialogue}) and (\ref{soipfpoemrepreprep})?

%First, we should note what appears to be trivial, i.e. that the theme of learning will always be a piece of knowledge. 
%In (\ref{dialogue}) this is quite obvious ("how to change a tyre"), in (\ref{soipfpoemrepreprep}) this is perhaps not so evident. Nevertheless, learning a poem basically means learning how to recite the poem. 

Although both learning events in (\ref{dialogue}) and (\ref{soipfpoemrepreprep}) are about the acquisition of knowledge, they differ in that the acquisition of knowledge about how to change a tyre follows a standardised and thus pre-given plan (it follows a script in the sense of \citealt{SchankAbel1977}), while the acquisition of knowledge about how to recite the poem does not.   
I propose that (\ref{dialogue}) presupposes a plan along the lines of which the learning proceeds, and that this plan supplies a directed path.
This is why the perfective form is licensed. In contrast to that, no directed path is available in the context of (\ref{soipfpoemrepreprep}), because there is no standard way of how to learn a poem. The technique that you choose will always be based on your personal preferences and individual capacities. In the typical case, as argued above, the process of learning a poem will go in cycles.\footnote{The android Data from Star Trek will learn a poem quickly by linearly scanning the text words once.}  

We have arrived at an explanation for the difference between (\ref{dialogue}) and (\ref{soipfpoemrepreprep}). Let me summarise it. If we look at the verbal predicate alone, we will find the predicate being associated with a non-directed scale, which leads us to expect the imperfective. If we take further linguistic material into account, however, the situation can change. In (\ref{dialogue}), the presence of the expression \textit{rajfn wekslwe} `tyre change' evokes a script that ``associates" the predicate with another gradable property, namely the property of how far one has gone through the instructions of a wheel change. This property suggests a learning path that leads through the steps 1 to 8 of the linear script in (\ref{script}).\footnote{\url{https://www.rac.co.uk/drive/advice/car-maintenance/how-to-change-a-tyre}}

  \ea\label{script}
1. Apply the handbrake.\\
2. Position the wheel chocks.\\
3. Loosen the wheel nuts.\\
4. Jack the car up.\\
5. Remove the flat tyre.\\
6. Mount the spare wheel.\\
7. Lower the car and tighten the bolts.\\
8. Fully lower the car.
\z

\noindent Since the path from 1 to 8 is directed, we now expect a perfective verb, which is what we find. 

Let us now look at a second example in which the otherwise unexpected perfective aspect is licensed pragmatically:

  \ea\label{problem}
\gll \textit{Ta} \textit{jo} \textit{t\'on} \textit{basejn} \textit{nawuk\l{a}} \textit{ha} \textit{jo} \textit{so} \textit{hrajka\'c} \textit{š\l{a}.} \\
she \textsc{aux} the poem learn.\textsc{pst.pfv.f} then \textsc{aux} \textsc{refl} play go.\textsc{pst.f}\\
\glt \normalsize{`She learned the poem and went playing.'}
\z

\noindent The form \textit{nawuk\l{a}} `(she) learned' in (\ref{problem}) is perfective. According to the determinateness approach argued for in this paper, the denoted event will have to be understood as proceeding along a single, directed and upper-bounded path. It seems obvious that the learning of the poem in (\ref{problem}) is such that it traverses a single and upper-bounded path. But how about directedness? 

With respect to (\ref{problem}), there is no reason not to assume that `she' has learned the poem ``in the usual way", that is, by revisiting the same lines, verses etc. again and again. Given the cyclic nature of the event, we should expect the imperfective aspect to be usable, and it is indeed possible to replace the perfective in (\ref{problem}) with its imperfective counterpart \textit{wuk\l{a}} (\citealt[245]{Scholze2008}). \citet{Werner2013} presents the following example:\footnote{Note that I treat the motion verb \textit{še\l{}} in CUS as a perfective (!) form, although we know its counterparts in other Slavic languages to be imperfectives.
This is only consequent in view of the present analysis, within which perfectivity boils down to determinateness. See \citet[282]{Scholze2008} for discussion.}

  \ea\label{nocheen}
\gll \textit{W\'on} \textit{je} \textit{list} \textit{\v{c}ita\l{}} \textit{a} \textit{je} \textit{še\l{}} \textit{pre\v{c}.} \\
he \textsc{aux} letter read.\textsc{pst.ipfv} and \textsc{aux} go.\textsc{pst.pfv} away\\
\glt \normalsize{`He read a letter and went away.'}
\z

\noindent Returning to (\ref{problem}), how can we make sense 
of the possibility of perfective \textit{nawuk\l{a}} given that learning a poem typically proceeds in cycles? 
I would propose that the use of the perfective 
in (\ref{problem}) triggers the implicature that the use of the imperfective, although possible as well, is avoided. Avoiding the imperfective is motivated by the fact that the expression of non-directedness is not in the interest of the speaker's message. The interpreter is thus invited to ``associate" the verbal concept with another gradable property, one that implies a directed path/scale. Such a property is plausibly available in the context at hand, namely the property of how far one is in the completion of a tedious task. In the given case the task is the learning of a poem (presumably as homework). Only after that job is done, will `she' be free to do what she likes to do, playing. For the message that the tedious job is done, the question of \textit{how} the poem was learned is irrelevant. Important is only \textit{that} she went from the stage of not being able to recite the poem to the stage of being able to do so, (which is what she will probably have to demonstrate at school the next day). In (\ref{problem}), in other words, the predicate \textit{learn the poem} is understood as abstracted from the details of the real learning process. Here the predicate describes events that traverse along a very simple path consisting of only two relevant points, $p(0)$ and $p(1)$. Figure \ref{change} shows the path relevant for the interpretation of the first sentence of (\ref{problem}). I argue that this path \textit{is} directed and that, therefore, the perfective is the appropriate form. 


\begin{figure}
% \begin{center}
% \fbox{
\begin{tikzpicture}
\draw[-Latex] (0, 0) -- (2, 0);
\node at (2,-0.75) {$p(1)$};
\node at (0,-0.75) {$p(0)$};
\node at (2, 0.9) {$knowing$};
\node at (0,0.9) {$not\, knowing$};
\node at (2, 0.55) {$the\,poem$};
\node at (0,0.55) {$the\,poem$};
\draw (0, 0) -- (0, -0.2);
\draw (2, 0) -- (2, -0.2);
\draw (0, 0) -- (0, 0.2);
\draw (2, 0) -- (2, 0.2);
\end{tikzpicture}
% }
% \end{center}
\caption{Path of learning the poem in (\ref{problem})}
\label{change}
\end{figure}

My discussion of the Upper Sorbian language data ends here. In the following section, I summarise the results once again and place them in a wider context.




\section{Conclusions}\label{mueller:sec:fin}
Although \citeposst{mue:Dickey2000} programmatic study is already more than 20 years old,
the enterprise of carefully investigating inner-Slavic variation in aspect selection has only just begun.  
Until now, most studies have focused on comparing Czech and Russian, because these two well-studied languages are supposed to be good examples of Dickey's 
Western-type and Eastern-type languages. In the present paper, I have drawn attention to Colloquial Upper Sorbian (CUS). I have reviewed the relevant linguistic literature on that language, and I have discussed its aspectual grammar  
against the background of differences between Czech and Russian. 
My overall results may be summarised as follows. 

I started from the idea that every verbal description of a dynamic event involves a path structure. The use of an imperfective  
verb does not impose any restrictions on the path structure 
of the event. The use of a perfective, however, requires the path of the denoted event to meet certain conditions. These conditions are written in the aspectual operator PFV, while its counterpart IPFV remains underspecified. 

More specifically, I concluded that a perfective verb requires 
the path of the denoted event to be unidimensional, directed, and bounded. 
The use of a CUS perfective does not impose any more 
restrictions on interpretation than that (in Czech and Russian, by contrast, perfectives come with additional constraints). 
It follows that the use of a perfective in CUS will be dispreferred if the context suggests that the event path is either not unidimensional, not directed, or not bounded. The perfective form will be the preferred choice, on the other hand, if the context suggests that all three conditions are met. 

If the verbal predicate describes an event that develops along a cyclic (i.e. non-directed) path, as with meanings like \textit{learn a poem}, 
\textit{iron a shirt}, 
\textit{painting a wall with a roll}, or \textit{blow-dry one's hair}, the expected aspectual form will be the imperfective. 
This default may be overridden, however, if the context makes salient an alternative directed path that the verbal description can adapt to. In this paper, we came across three such cases. The first was represented by (\ref{problem}). In this case, the context draws attention to the finishing of the event (here: the finishing of a tedious homework), suggesting the relevant path to be a (directed) two-point scale. In the second case, represented by (\ref{dialogue}), the directed path was introduced by a linear script evoked by overt contextual material (here: the wheel change instructions).  
The third case was represented by (\ref{sopfgen}), where 
the only interpretations pragmatically available involve directed paths due to the impact of a focus-bearing expression (here: the adverb \textit{chětř} `quickly').

I have drawn attention to several examples which show that CUS does not subsume to the aspectologist's common sense understanding according to which perfectives would express completed events. 
Does that mean that CUS is, in some sense, ``unslavic", perhaps due to influence from German?
Not necessarily. 
As \citet{mue:Comrie1976} points out, ``completed" is the wrong feature  
for grasping the content of perfectivity, what is suitable instead is ``complete": 

\begin{quote}
    A very frequent characterisation of perfectivity is that it indicates a completed action. 
 One should note that the word at issue in this definition is ``completed",
not ``complete": despite the formal similarity between the two words, there is an
important semantic distinction which turns out to be crucial in discussing aspect.
The perfective does indeed denote a complete situation, with beginning, middle,
and end. The use of ``completed”, however, puts too much emphasis on the
termination of the situation (\citealt[18]{mue:Comrie1976})
\end{quote}

The notion of a complete (not: completed) event is indeed the appropriate umbrella term to subsume the perfectives in all the three languages discussed in this paper. Differences result from how this general notion manifests itself in each case. We have found three different strengths of what it means to be complete. 

The strongest perfective condition is found in Russian. Here, the event referred to by a perfective has to be complete and its consequences have to be occasions for subsequent events. Czech features a weaker perfective condition. In that language, the event referred to by a perfective has to be complete, and that's all. Since there is no requirement as to ``connecting" the event to neighbouring event tokens, the Czech perfective may be used to refer to a plurality of events, if only the elements of the plurality are to be understood as complete events.
The weakest way of being complete is instantiated by perfectives in CUS. Here, it is not the event/situation as such which has to be complete, contra to what is said in the quote above, but the path along which the denoted event evolves.    

\section*{Abbreviations}

\begin{tabularx}{.45\textwidth}{@{}lX@{}}
\textsc{3}&third person\\
\textsc{aux}&auxiliary\\
\textsc{du}&dual\\
\textsc{f}&feminine\\
\textsc{inf}&infinitive\\
\end{tabularx}%
\begin{tabularx}{.5\textwidth}{@{}lX@{}}
\textsc{ipfv}&imperfective\\
\textsc{pfv}&perfective\\
\textsc{pst}&past\\
\textsc{prs}&present\\
\textsc{refl}&reflexive\\

%&\\ % this dummy row achieves correct vertical alignment of both tables
\end{tabularx}

\section*{Acknowledgements}
For helpful input and discussion, special thanks to Tilman Berger, Petr Biskup, Simon Blum, Berit Gehrke, Matthias Irmer, Fabian Kaulfürst, Alena Paulik,  Matthias Schlegel, Lenka Scholze, Till Voigt, Eduard Werner, and Karolina Zuchewicz.
Thanks also to the audiences of the conferences 
\textit{70 lět sorabistiki -- konferenca w Lipsku} at Leipzig University (May 19-22, 2022)
and
\textit{Formal description of Slavic languages 15} at Humboldt-University Berlin (October 5-7, 2022).
\sloppy
\printbibliography[heading=subbibliography,notkeyword=this]

\end{document}
