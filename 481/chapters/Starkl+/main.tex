\documentclass[output=paper,colorlinks,citecolor=brown]{langscibook}
\ChapterDOI{10.5281/zenodo.15394205}

%\bibliography{localbibliography}

\author{Ema Štarkl\orcid{0000-0002-1328-8395}\affiliation{University of Nova Gorica, ZRC SAZU, University of Ljubljana} and Marko Simonović\orcid{0000-0002-9651-6399}\affiliation{University of Graz} and Stefan Milosavljević\orcid{0000-0003-2305-2519}\affiliation{University of Graz} and Boban Arsenijević\orcid{0000-0002-1124-6319}\affiliation{University of Graz}
}
% replace the above with you and your coauthors
% rules for affiliation: If there's an official English version, use that (find out on the official website of the university); if not, use the original
% orcid doesn't appear printed; it's metainformation used for later indexing

%%% uncomment the following line if you are a single author or all authors have the same affiliation
% \SetupAffiliations{mark style=none}

%% in case the running head with authors exceeds one line (which is the case in this example document), use one of the following methods to turn it into a single line; otherwise comment the line below out with % and ignore it
\lehead{
Štarkl, Simonović, Milosavljević \& Arsenijević
}

\title[The Western South Slavic -\textit{nV}/-\textit{ne} is a diminutive affix with a theme vowel]{The Western South Slavic verbal suffix -\textit{nV}/-\textit{ne} is a diminutive affix with a theme vowel}
% replace the above with your paper title
%%% provide a shorter version of your title in case it doesn't fit a single line in the running head
% in this form: \title[short title]{full title}
\abstract{The paper proposes a novel analysis of the sequence \textit{-nV/-ne} in Western South Slavic (\textit{-nu/-ne} in BCMS and \textit{-ni/-ne} in Slovenian) as a complex morpheme consisting of the diminutive suffix \textit{-n} and the theme vowel \textit{$∅$/e}, whereby the latter realizes the verbal category, like all other verbal themes in Slavic. We argue that the vowel in the suffix \textit{-nV} is a floating vowel that surfaces when it helps optimize the syllable structure. While analyses of \textit{-nV/-ne} as a complex morpheme have been proposed in the literature, the analysis in terms of diminution enables us to account for the peculiar status of the suffix \textit{-nV} among other verbal suffixes, especially its compatibility with other suffixes, including diminutive and secondary imperfectivizing ones, which is either ignored or left unexplained in the previous accounts.

%The West South Slavic verbal suffix \textit{-nV/-ne} (\textit{-nu/-ne} in Bosnian-Croatian-Serbian-Montenegrian (BCMS) and \textit{-ni/-ne} in Slovenian) has been analysed as monomorphemic traditionally and in some formal approaches. We propose \textit{-nu/-ni} in West South Slavic is a complex morpheme that consists of the morpheme \textit{-n\textsuperscript{u}/-n\textsuperscript{i}} (with a floating vowel) and the theme vowel $∅$/e. The morpheme \textit{-nu/-ni} spells out a diminutive feature and the theme vowel spells out the verbal category feature (for the latter, see also \citealt{sta+:Svenonius2004}, \citealt{Biskup2019}, \citealt{KovacevicEtAl2024}, \citealt{SimonovicEtAl2021}, \citealt{MilosavljevicArsenijevic2022}).

\keywords{Western South Slavic, suffix \textit{-n}, verbal morphology, semelfactive, diminution, theme vowel}
}

\begin{document}
\maketitle

% Just comment out the input below when you're ready to go.
%For a start: Do not forget to give your Overleaf project (this paper) a recognizable name. This one could be called, for instance, Simik et al: OSL template. You can change the name of the project by hovering over the gray title at the top of this page and clicking on the pencil icon.

\section{Introduction}\label{sim:sec:intro}

Language Science Press is a project run for linguists, but also by linguists. You are part of that and we rely on your collaboration to get at the desired result. Publishing with LangSci Press might mean a bit more work for the author (and for the volume editor), esp. for the less experienced ones, but it also gives you much more control of the process and it is rewarding to see the quality result.

Please follow the instructions below closely, it will save the volume editors, the series editors, and you alike a lot of time.

\sloppy This stylesheet is a further specification of three more general sources: (i) the Leipzig glossing rules \citep{leipzig-glossing-rules}, (ii) the generic style rules for linguistics (\url{https://www.eva.mpg.de/fileadmin/content_files/staff/haspelmt/pdf/GenericStyleRules.pdf}), and (iii) the Language Science Press guidelines \citep{Nordhoff.Muller2021}.\footnote{Notice the way in-text numbered lists should be written -- using small Roman numbers enclosed in brackets.} It is advisable to go through these before you start writing. Most of the general rules are not repeated here.\footnote{Do not worry about the colors of references and links. They are there to make the editorial process easier and will disappear prior to official publication.}

Please spend some time reading through these and the more general instructions. Your 30 minutes on this is likely to save you and us hours of additional work. Do not hesitate to contact the editors if you have any questions.

\section{Illustrating OSL commands and conventions}\label{sim:sec:osl-comm}

Below I illustrate the use of a number of commands defined in langsci-osl.tex (see the styles folder).

\subsection{Typesetting semantics}\label{sim:sec:sem}

See below for some examples of how to typeset semantic formulas. The examples also show the use of the sib-command (= ``semantic interpretation brackets''). Notice also the the use of the dummy curly brackets in \REF{sim:ex:quant}. They ensure that the spacing around the equation symbol is correct. 

\ea \ea \sib{dog}$^g=\textsc{dog}=\lambda x[\textsc{dog}(x)]$\label{sim:ex:dog}
\ex \sib{Some dog bit every boy}${}=\exists x[\textsc{dog}(x)\wedge\forall y[\textsc{boy}(y)\rightarrow \textsc{bit}(x,y)]]$\label{sim:ex:quant}
\z\z

\noindent Use noindent after example environments (but not after floats like tables or figures).

And here's a macro for semantic type brackets: The expression \textit{dog} is of type $\stb{e,t}$. Don't forget to place the whole type formula into a math-environment. An example of a more complex type, such as the one of \textit{every}: $\stb{s,\stb{\stb{e,t},\stb{e,t}}}$. You can of course also use the type in a subscript: dog$_{\stb{e,t}}$

We distinguish between metalinguistic constants that are translations of object language, which are typeset using small caps, see \REF{sim:ex:dog}, and logical constants. See the contrast in \REF{sim:ex:speaker}, where \textsc{speaker} (= serif) in \REF{sim:ex:speaker-a} is the denotation of the word \textit{speaker}, and \cnst{speaker} (= sans-serif) in \REF{sim:ex:speaker-b} is the function that maps the context $c$ to the speaker in that context.\footnote{Notice that both types of small caps are automatically turned into text-style, even if used in a math-environment. This enables you to use math throughout.}$^,$\footnote{Notice also that the notation entails the ``direct translation'' system from natural language to metalanguage, as entertained e.g. in \citet{Heim.Kratzer1998}. Feel free to devise your own notation when relying on the ``indirect translation'' system (see, e.g., \citealt{Coppock.Champollion2022}).}

\ea\label{sim:ex:speaker}
\ea \sib{The speaker is drunk}$^{g,c}=\textsc{drunk}\big(\iota x\,\textsc{speaker}(x)\big)$\label{sim:ex:speaker-a}
\ex \sib{I am drunk}$^{g,c}=\textsc{drunk}\big(\cnst{speaker}(c)\big)$\label{sim:ex:speaker-b}
\z\z

\noindent Notice that with more complex formulas, you can use bigger brackets indicating scope, cf. $($ vs. $\big($, as used in \REF{sim:ex:speaker}. Also notice the use of backslash plus comma, which produces additional space in math-environment.

\subsection{Examples and the minsp command}

Try to keep examples simple. But if you need to pack more information into an example or include more alternatives, you can resort to various brackets or slashes. For that, you will find the minsp-command useful. It works as follows:

\ea\label{sim:ex:german-verbs}\gll Hans \minsp{\{} schläft / schlief / \minsp{*} schlafen\}.\\
Hans {} sleeps {} slept {} {} sleep.\textsc{inf}\\
\glt `Hans \{sleeps / slept\}.'
\z

\noindent If you use the command, glosses will be aligned with the corresponding object language elements correctly. Notice also that brackets etc. do not receive their own gloss. Simply use closed curly brackets as the placeholder.

The minsp-command is not needed for grammaticality judgments used for the whole sentence. For that, use the native langsci-gb4e method instead, as illustrated below:

\ea[*]{\gll Das sein ungrammatisch.\\
that be.\textsc{inf} ungrammatical\\
\glt Intended: `This is ungrammatical.'\hfill (German)\label{sim:ex:ungram}}
\z

\noindent Also notice that translations should never be ungrammatical. If the original is ungrammatical, provide the intended interpretation in idiomatic English.

If you want to indicate the language and/or the source of the example, place this on the right margin of the translation line. Schematic information about relevant linguistic properties of the examples should be placed on the line of the example, as indicated below.

\ea\label{sim:ex:bailyn}\gll \minsp{[} Ėtu knigu] čitaet Ivan \minsp{(} často).\\
{} this book.{\ACC} read.{\PRS.3\SG} Ivan.{\NOM} {} often\\\hfill O-V-S-Adv
\glt `Ivan reads this book (often).'\hfill (Russian; \citealt[4]{Bailyn2004})
\z

\noindent Finally, notice that you can use the gloss macros for typing grammatical glosses, defined in langsci-lgr.sty. Place curly brackets around them.

\subsection{Citation commands and macros}

You can make your life easier if you use the following citation commands and macros (see code):

\begin{itemize}
    \item \citealt{Bailyn2004}: no brackets
    \item \citet{Bailyn2004}: year in brackets
    \item \citep{Bailyn2004}: everything in brackets
    \item \citepossalt{Bailyn2004}: possessive
    \item \citeposst{Bailyn2004}: possessive with year in brackets
\end{itemize}

\section{Trees}\label{s:tree}

Use the forest package for trees and place trees in a figure environment. \figref{sim:fig:CP} shows a simple example.\footnote{See \citet{VandenWyngaerd2017} for a simple and useful quickstart guide for the forest package.} Notice that figure (and table) environments are so-called floating environments. {\LaTeX} determines the position of the figure/table on the page, so it can appear elsewhere than where it appears in the code. This is not a bug, it is a property. Also for this reason, do not refer to figures/tables by using phrases like ``the table below''. Always use tabref/figref. If your terminal nodes represent object language, then these should essentially correspond to glosses, not to the original. For this reason, we recommend including an explicit example which corresponds to the tree, in this particular case \REF{sim:ex:czech-for-tree}.

\ea\label{sim:ex:czech-for-tree}\gll Co se řidič snažil dělat?\\
what {\REFL} driver try.{\PTCP.\SG.\MASC} do.{\INF}\\
\glt `What did the driver try to do?'
\z

\begin{figure}[ht]
% the [ht] option means that you prefer the placement of the figure HERE (=h) and if HERE is not possible, you prefer the TOP (=t) of a page
% \centering
    \begin{forest}
    for tree={s sep=1cm, inner sep=0, l=0}
    [CP
        [DP
            [what, roof, name=what]
        ]
        [C$'$
            [C
                [\textsc{refl}]
            ]
            [TP
                [DP
                    [driver, roof]
                ]
                [T$'$
                    [T [{[past]}]]
                    [VP
                        [V
                            [tried]
                        ]
                        [VP, s sep=2.2cm
                            [V
                                [do.\textsc{inf}]
                            ]
                            [t\textsubscript{what}, name=trace-what]
                        ]
                    ]
                ]
            ]
        ]
    ]
    \draw[->,overlay] (trace-what) to[out=south west, in=south, looseness=1.1] (what);
    % the overlay option avoids making the bounding box of the tree too large
    % the looseness option defines the looseness of the arrow (default = 1)
    \end{forest}
    \vspace{3ex} % extra vspace is added here because the arrow goes too deep to the caption; avoid such manual tweaking as much as possible; here it's necessary
    \caption{Proposed syntactic representation of \REF{sim:ex:czech-for-tree}}
    \label{sim:fig:CP}
\end{figure}

Do not use noindent after figures or tables (as you do after examples). Cases like these (where the noindent ends up missing) will be handled by the editors prior to publication.

\section{Italics, boldface, small caps, underlining, quotes}

See \citet{Nordhoff.Muller2021} for that. In short:

\begin{itemize}
    \item No boldface anywhere.
    \item No underlining anywhere (unless for very specific and well-defined technical notation; consult with editors).
    \item Small caps used for (i) introducing terms that are important for the paper (small-cap the term just ones, at a place where it is characterized/defined); (ii) metalinguistic translations of object-language expressions in semantic formulas (see \sectref{sim:sec:sem}); (iii) selected technical notions.
    \item Italics for object-language within text; exceptionally for emphasis/contrast.
    \item Single quotes: for translations/interpretations
    \item Double quotes: scare quotes; quotations of chunks of text.
\end{itemize}

\section{Cross-referencing}

Label examples, sections, tables, figures, possibly footnotes (by using the label macro). The name of the label is up to you, but it is good practice to follow this template: article-code:reference-type:unique-label. E.g. sim:ex:german would be a proper name for a reference within this paper (sim = short for the author(s); ex = example reference; german = unique name of that example).

\section{Syntactic notation}

Syntactic categories (N, D, V, etc.) are written with initial capital letters. This also holds for categories named with multiple letters, e.g. Foc, Top, Num, etc. Stick to this convention also when coming up with ad hoc categories, e.g. Cl (for clitic or classifier).

An exception from this rule are ``little'' categories, which are written with italics: \textit{v}, \textit{n}, \textit{v}P, etc.

Bar-levels must be typeset with bars/primes, not with an apostrophe. An easy way to do that is to use mathmode for the bar: C$'$, Foc$'$, etc.

Specifiers should be written this way: SpecCP, Spec\textit{v}P.

Features should be surrounded by square brackets, e.g., [past]. If you use plus and minus, be sure that these actually are plus and minus, and not e.g. a hyphen. Mathmode can help with that: [$+$sg], [$-$sg], [$\pm$sg]. See \sectref{sim:sec:hyphens-etc} for related information.

\section{Footnotes}

Absolutely avoid long footnotes. A footnote should not be longer than, say, {20\%} of the page. If you feel like you need a long footnote, make an explicit digression in the main body of the text.

Footnotes should always be placed at the end of whole sentences. Formulate the footnote in such a way that this is possible. Footnotes should always go after punctuation marks, never before. Do not place footnotes after individual words. Do not place footnotes in examples, tables, etc. If you have an urge to do that, place the footnote to the text that explains the example, table, etc.

Footnotes should always be formulated as full, self-standing sentences.

\section{Tables}

For your tables use the table plus tabularx environments. The tabularx environment lets you (and requires you in fact) to specify the width of the table and defines the X column (left-alignment) and the Y column (right-alignment). All X/Y columns will have the same width and together they will fill out the width of the rest of the table -- counting out all non-X/Y columns.

Always include a meaningful caption. The caption is designed to appear on top of the table, no matter where you place it in the code. Do not try to tweak with this. Tables are delimited with lsptoprule at the top and lspbottomrule at the bottom. The header is delimited from the rest with midrule. Vertical lines in tables are banned. An example is provided in \tabref{sim:tab:frequencies}. See \citet{Nordhoff.Muller2021} for more information. If you are typesetting a very complex table or your table is too large to fit the page, do not hesitate to ask the editors for help.

\begin{table}
\caption{Frequencies of word classes}
\label{sim:tab:frequencies}
 \begin{tabularx}{.77\textwidth}{lYYYY} %.77 indicates that the table will take up 77% of the textwidth
  \lsptoprule
            & nouns & verbs  & adjectives & adverbs\\
  \midrule
  absolute  &   12  &    34  &    23      & 13\\
  relative  &   3.1 &   8.9  &    5.7     & 3.2\\
  \lspbottomrule
 \end{tabularx}
\end{table}

\section{Figures}

Figures must have a good quality. If you use pictorial figures, consult the editors early on to see if the quality and format of your figure is sufficient. If you use simple barplots, you can use the barplot environment (defined in langsci-osl.sty). See \figref{sim:fig:barplot} for an example. The barplot environment has 5 arguments: 1. x-axis desription, 2. y-axis description, 3. width (relative to textwidth), 4. x-tick descriptions, 5. x-ticks plus y-values.

\begin{figure}
    \centering
    \barplot{Type of meal}{Times selected}{0.6}{Bread,Soup,Pizza}%
    {
    (Bread,61)
    (Soup,12)
    (Pizza,8)
    }
    \caption{A barplot example}
    \label{sim:fig:barplot}
\end{figure}

The barplot environment builds on the tikzpicture plus axis environments of the pgfplots package. It can be customized in various ways. \figref{sim:fig:complex-barplot} shows a more complex example.

\begin{figure}
  \begin{tikzpicture}
    \begin{axis}[
	xlabel={Level of \textsc{uniq/max}},  
	ylabel={Proportion of $\textsf{subj}\prec\textsf{pred}$}, 
	axis lines*=left, 
        width  = .6\textwidth,
	height = 5cm,
    	nodes near coords, 
    % 	nodes near coords style={text=black},
    	every node near coord/.append style={font=\tiny},
        nodes near coords align={vertical},
	ymin=0,
	ymax=1,
	ytick distance=.2,
	xtick=data,
	ylabel near ticks,
	x tick label style={font=\sffamily},
	ybar=5pt,
	legend pos=outer north east,
	enlarge x limits=0.3,
	symbolic x coords={+u/m, \textminus u/m},
	]
	\addplot[fill=red!30,draw=none] coordinates {
	    (+u/m,0.91)
        (\textminus u/m,0.84)
	};
	\addplot[fill=red,draw=none] coordinates {
	    (+u/m,0.80)
        (\textminus u/m,0.87)
	};
	\legend{\textsf{sg}, \textsf{pl}}
    \end{axis} 
  \end{tikzpicture} 
    \caption{Results divided by \textsc{number}}
    \label{sim:fig:complex-barplot}
\end{figure}

\section{Hyphens, dashes, minuses, math/logical operators}\label{sim:sec:hyphens-etc}

Be careful to distinguish between hyphens (-), dashes (--), and the minus sign ($-$). For in-text appositions, use only en-dashes -- as done here -- with spaces around. Do not use em-dashes (---). Using mathmode is a reliable way of getting the minus sign.

All equations (and typically also semantic formulas, see \sectref{sim:sec:sem}) should be typeset using mathmode. Notice that mathmode not only gets the math signs ``right'', but also has a dedicated spacing. For that reason, never write things like p$<$0.05, p $<$ 0.05, or p$<0.05$, but rather $p<0.05$. In case you need a two-place math or logical operator (like $\wedge$) but for some reason do not have one of the arguments represented overtly, you can use a ``dummy'' argument (curly brackets) to simulate the presence of the other one. Notice the difference between $\wedge p$ and ${}\wedge p$.

In case you need to use normal text within mathmode, use the text command. Here is an example: $\text{frequency}=.8$. This way, you get the math spacing right.

\section{Abbreviations}

The final abbreviations section should include all glosses. It should not include other ad hoc abbreviations (those should be defined upon first use) and also not standard abbreviations like NP, VP, etc.


\section{Bibliography}

Place your bibliography into localbibliography.bib. Important: Only place there the entries which you actually cite! You can make use of our OSL bibliography, which we keep clean and tidy and update it after the publication of each new volume. Contact the editors of your volume if you do not have the bib file yet. If you find the entry you need, just copy-paste it in your localbibliography.bib. The bibliography also shows many good examples of what a good bibliographic entry should look like.

See \citet{Nordhoff.Muller2021} for general information on bibliography. Some important things to keep in mind:

\begin{itemize}
    \item Journals should be cited as they are officially called (notice the difference between and, \&, capitalization, etc.).
    \item Journal publications should always include the volume number, the issue number (field ``number''), and DOI or stable URL (see below on that).
    \item Papers in collections or proceedings must include the editors of the volume (field ``editor''), the place of publication (field ``address'') and publisher.
    \item The proceedings number is part of the title of the proceedings. Do not place it into the ``volume'' field. The ``volume'' field with book/proceedings publications is reserved for the volume of that single book (e.g. NELS 40 proceedings might have vol. 1 and vol. 2).
    \item Avoid citing manuscripts as much as possible. If you need to cite them, try to provide a stable URL.
    \item Avoid citing presentations or talks. If you absolutely must cite them, be careful not to refer the reader to them by using ``see...''. The reader can't see them.
    \item If you cite a manuscript, presentation, or some other hard-to-define source, use the either the ``misc'' or ``unpublished'' entry type. The former is appropriate if the text cited corresponds to a book (the title will be printed in italics); the latter is appropriate if the text cited corresponds to an article or presentation (the title will be printed normally). Within both entries, use the ``howpublished'' field for any relevant information (such as ``Manuscript, University of \dots''). And the ``url'' field for the URL.
\end{itemize}

We require the authors to provide DOIs or URLs wherever possible, though not without limitations. The following rules apply:

\begin{itemize}
    \item If the publication has a DOI, use that. Use the ``doi'' field and write just the DOI, not the whole URL.
    \item If the publication has no DOI, but it has a stable URL (as e.g. JSTOR, but possibly also lingbuzz), use that. Place it in the ``url'' field, using the full address (https: etc.).
    \item Never use DOI and URL at the same time.
    \item If the official publication has no official DOI or stable URL (related to the official publication), do not replace these with other links. Do not refer to published works with lingbuzz links, for instance, as these typically lead to the unpublished (preprint) version. (There are exceptions where lingbuzz or semanticsarchive are the official publication venue, in which case these can of course be used.) Never use URLs leading to personal websites.
    \item If a paper has no DOI/URL, but the book does, do not use the book URL. Just use nothing.
\end{itemize}

\section{Introduction}

%\subsection{The topic of the paper}

In this paper, we offer a novel analysis of verbs with the suffix \textit{-nV/-ne} in Western South Slavic, specifically in Bosnian/Croatian/Montenegrin/Serbian (BCMS) and Slovenian, illustrated in \REF{str:ex:IllSC} and \REF{str:ex:IllSlo}, respectively. Our focus is on perfective verbs, as in (\ref{str:ex:IllSC-a}, \ref{str:ex:IllSlo-a}), since only they are productive in both languages, although \textit{-nV} is also found in a small number of imperfective degree achievements (DAs), as in (\ref{str:ex:IllSC-b}, \ref{str:ex:IllSlo-b}). We propose that \textit{-nV/-ne} in Western South Slavic is complex and consists of the morpheme \textit{-n\textsuperscript{u}/-n\textsuperscript{i}} (with a floating vowel) and the theme vowel \textit{$∅$/e}. The morpheme \textit{-n\textsuperscript{u}/-n\textsuperscript{i}} spells out a diminutive feature and the theme vowel spells out the verbal category feature, just like all other theme vowels in Slavic languages (for the latter, see also \citealt{sta+:Svenonius2004,Biskup2019, SimonovicEtAl2021,MilosavljevicArsenijevic2022, KovacevicEtAl2024}).

\begin{multicols}{2}
\ea\label{str:ex:IllSC}
	\ea \gll trep-nu-ti \\ 
blink-nV-{\INF} \\
\glt `blink'\label{str:ex:IllSC-a} 

\ex  \gll to(n)-nu-ti  %(BCMS)
\\ 
 sink-nV-{\INF} \\ \jambox*{(BCMS)}
\glt `sink'\label{str:ex:IllSC-b} 
				
	\z
\z 
\end{multicols}

\begin{multicols}{2}

\ea\label{str:ex:IllSlo}
	\ea \gll mežik-ni-ti \\ 
blink-nV-{\INF} \\
\glt `blink'\label{str:ex:IllSlo-a} \columnbreak

\ex  \gll to(n)-ni-ti %(Slovenian)
\\ 
 sink-nV-{\INF} \\ \jambox*{(Slovenian)}
\glt `sink'\label{str:ex:IllSlo-b} 
				
	\z
\z 
\end{multicols}

\noindent Morphological forms of \textit{-nV/-ne}- and \textit{$∅$/e}-verbs that will be relevant for our analysis are summarized in Tables \ref{str:tab:conjugation-nV} and \ref{str:tab:conjugation-0}, respectively.

\begin{table}
\caption{Conjugation of \textit{-nV/-ne} verbs}
\label{str:tab:conjugation-nV}
 \begin{tabularx}{.99\textwidth}{XXXXl}
  \lsptoprule
             Language & {\INF}  & {\PRS.1.\PL} & {\PASS.\PTCP} & Translation \\
  \midrule
  BCMS  &   gur-nu-ti   & gur-ne-mo &  gur-nu-t & `push' \\
  Slovenian & pah-ni-ti & pah-ne-mo& pah-nj-en&`push'\\

  \lspbottomrule
 \end{tabularx}
\end{table}

\begin{table}
\caption{Conjugation of $∅$/e verbs}
\label{str:tab:conjugation-0}
 \begin{tabularx}{.99\textwidth}{XXXXl}
  \lsptoprule
             Language & {\INF}  & {\PRS.1.\PL} & {\PASS.\PTCP} & Translation \\
  \midrule
  BCMS  &   %ukra\sout{d}(s)-$∅$-ti   & ukrad-e-mo &  ukrad-$∅$-en & `steal' 
ves-$∅$-ti   & vez-e-mo &  vez-$∅$-en & `embroider' \\ 
Slovenian  &   %ukra\sout{d}(s)-$∅$-ti   & ukrad-e-mo &  ukrad-$∅$-en & `steal' \\
ves-$∅$-ti   & vez-e-mo &  vez-$∅$-en & `embroider' \\ 

  \lspbottomrule
 \end{tabularx}
\end{table}


The paper is organized as follows. In the remainder of this section, we overview different classes of Slavic verbs derived by \textit{-nV/-ne} that are recognized in the literature. In \sectref{str:sec:previous-analyses}, we critically assess previous accounts of \textit{-nV/-ne}. \sectref{str:sect:Quantitative description} provides a quantitative description of this sequence in BCMS and Slovenian. In \sectref{str:sect:Morpho-phono analysis}, we provide morpho-phonological arguments for our segmentation of \textit{-nV/-ne} into the morpheme \textit{-n\textsuperscript{u}/-n\textsuperscript{i}} and the theme vowel \textit{$∅$/e}. Our syntactic and semantic analysis of \textit{-n\textsuperscript{u}/-n\textsuperscript{i}} as a diminutive suffix that combines with the verbal category (whose exponent is \textit{$∅$/e}) is provided in \sectref{str:sect:syntactic-semantic analysis}. \sectref{str:sect:Conclusion} concludes the paper.

%\subsubsection{Verb classes derived by \textit{-nV} in Slavic languages}

Several verb classes derived by \textit{-nV} have been recognized in the literature on Slavic languages. The most typical and the most productive class comprises \textsc{semelfactives}, illustrated in \REF{str:ex:SemItSC-a} and \REF{str:ex:SemItSlo-a} for BCMS and Slovenian, respectively. Semelfactives are usually definied as ``instantaneous'' actions in the classical sense of \citet{Smith1997}, and in most formal approaches this is the only identified class of perfective \textit{-nV/-ne} verbs (see e.g. \citealt{sta+:Lazorczyk2010} and \citealt{Kwapiszewski2020, Kwapiszewski2022} for Polish, \citealt{Wiland2019} for Czech and Polish, \citealt{sta+:Biskup2023} for Russian and Czech, etc.). In analyses couched in the framework of Cognitive Linguistics, this class of verbs is usually referred to as Single Act Perfectives (see e.g. \citealt{Janda2007,DickeyJanda2009,MakarovaJanda2009,KuznetsovaMakarova2012,Nesset2013,Sokolova2015} for Russian, \citealt{Nesset2012} for Old Church Slavonic, \citealt{Bacz2012} for Polish).

\begin{multicols}{2}

\ea\label{str:ex:SemItSC}
	\ea  \gll mah-nu-ti \\ 
wave-nV-{\INF} \\
\glt `wave once'\label{str:ex:SemItSC-a} 

\ex  \gll mah-a-ti\\ 
 wave-\textsc{tv}-{\INF}\\ \jambox*{(BCMS)}
\glt `wave repeatedly'\label{str:ex:SemItSC-b} 
				
	\z
\z 
\end{multicols}

\begin{multicols}{2}
    
\ea\label{str:ex:SemItSlo}
	\ea  \gll mah-ni-ti \\ 
wave-nV-{\INF} \\
\glt `wave once'\label{str:ex:SemItSlo-a} 

\ex  \gll mah-a-ti\\ 
 wave-\textsc{tv}-{\INF} \\ \jambox*{(Slovenian)}
\glt `wave repeatedly'\label{str:ex:SemItSlo-b} 
				
	\z
\z 
\end{multicols}

\noindent\textsc{Degree achievements}, illustrated for BCMS and Slovenian in \REF{str:ex:IllSC-b} and \REF{str:ex:IllSlo-b} above, are a small class of imperfective verbs derived by \textit{-nV/-ne} (see e.g. \citealt{TaraldsenMedovaWiland2019} for a formal analysis of this class in Czech and Polish). Degree achievements derived by \textit{-nV/-ne} are no longer a productive class across Slavic languages, which is why they will be set aside in the present paper (though we briefly return to them in \sectref{str:subsect: nV - quantitative data}).\footnote{An anonymous reviewer raises the question of whether imperfective \textit{-nV/-ne} verbs should be analyzed on a par with perfective \textit{-nV/-ne} verbs. Since we focus only on perfective \textit{-nV/-ne} verbs in the paper (as only perfectives are productive in contemporary BCMS and Slovenian), we do not delve deeper into the debate on whether \textit{-nV/-ne} in perfectives and imperfectives should be treated as a unified item. Note, however, that once semelfactives and degree achievements are analyzed as sharing the same semantic core based on atomicity (cf. \citealt{Rothstein2008Telicity, Rothstein2008Puzzles}), there might be a semantic justification for treating these two \textit{-nV/-ne} classes as containing the same suffix; see also \citet{TaraldsenMedovaWiland2019} (presented in \sectref{str:sect:bimorphemic analyses}) for a unified syntactic analysis of this suffix within a Nanosyntactic framework. 
%Finally, we also checked the stress/tone pattern of perfective and imperfective \textit{-nV/-ne} verbs and did not identify any differences.
}
\textsc{Natural perfectives} are \textit{-nV} verbs that function as lexicalized perfective counterparts of simplex imperfective verbs (e.g. \citealt{Bacz2012} for Polish, \citealt{Sokolova2015} for Russian). It should be immediately clear that there is no clear-cut boundary between these verbs and ``proper'' semelfactives, since semelfactives act as aspectual counterparts of iterative verbs, as in \REF{str:ex:SemItSC} and \REF{str:ex:SemItSlo}.
Finally, \citet{Sokolova2015} identifies a class of \textsc{(perfective) delimitatives} in Russian, which are \textit{-nV/-ne} verbs that can combine with durative adverbials indicating a short duration, as in \REF{str:ex:RusLet}. Similar examples are available in BCMS, as evidenced by example \REF{str:ex:SCDoz}, whereas in Slovenian this use of \textit{-nV} is not attested. %to the best of our knowledge

\ea\label{str:ex:RusLet}
\gll Ja let-nu-l 2 časa.\\
I fly-nV-{\PST} 2 hours\\ \jambox*{(Russian, from \citealt{Sokolova2015}; our translation)}
\glt `I flew for two hours $[$I took a short two-hour flight$]$.'

\ex\label{str:ex:SCDoz}
\gll Drem-nu-o sam par minuta.\\ 
doze-nV-{\PST} {\AUX.1.\SG} couple minutes \\ \jambox*{(BCMS)}
\glt `I dozed for a few minutes.'

\z

\section{Previous analyses of \textit{-nV/-ne}} \label{str:sec:previous-analyses}

In this section, we %critically
overview previous analyses of the sequence \textit{-nV/-ne}. We first briefly comment on traditional approaches to this item in Slavic in \sectref{str:subsec:traditional-approaches}, after which we provide a detailed discussion of previous formal analyses of \textit{-nV/-ne} in \sectref{str:subsec:previous-formal-approaches}.

\subsection{Traditional approaches to \textit{-nV} verbs in Slavic} \label{str:subsec:traditional-approaches}

In traditional descriptions, \textit{-nV/-ne} is typically analyzed as a monomorphemic theme vowel (TV) defining its own conjugation class (for BCMS, see e.g. \citealt[235]{BaricEtAl1997}, \citealt[253]{IvsicEtAl1970}, \citealt[331]{sta+:Stevanovic1986}, \citealt{StanojcicPopovic2008}; for Slovenian, cf. e.g. \citealt[116, 124]{Breznik1934}, \citealt[364]{sta+:Toporisic2000}, \citealt[64]{Vidovic2011}; a similar point is made for Russian in \citealt{Gladney2013} and references therein). The alternative analysis, whereby \textit{-n} is a separate morpheme and \textit{V/e} is a theme vowel, is usually discarded on the grounds that there is no independently motivated TV class defined by the vowels following \textit{-n} (i.e. \textit{i/e} in Slovenian and \textit{u/e} in BCMS).

\subsection{Previous formal approaches to \textit{-nV} verbs in Slavic} \label{str:subsec:previous-formal-approaches}

In this subsection, we discuss previous formal approaches to \textit{-nV/-ne}, grouping them into those that analyze this segment as a single morpheme -- let us label them \textsc{monomorphemic analyses} (\sectref{str:sect:monomorphemic analyses}), and those arguing that \textit{-nV/-ne} is decomposable into a suffix and a theme vowel -- \textsc{bimorphemic analyses} (\sectref{str:sect:bimorphemic analyses}). 

\subsubsection{Monomorphemic analyses} \label{str:sect:monomorphemic analyses}

\citet{Schoorlemmer2004} analyzes \textit{-nu/-ne} in Russian as a lexical marker of perfectivity, which is one of the two basic ways of how perfectivity arises in her approach (the other way being compositionally, through telicity, as in the case of prefixed perfective verbs; see also \citealt{Schoorlemmer1997}). According to Schoorlemmer, the ``lexical'' (i.e. non-compositional) status of perfective verbs derived by this suffix is confirmed by the fact that (in Russian) they do not derive secondary imperfectives, unlike (prefixed) telic predicates (accomplishments and achievements). For \citet{Borer2005, Borer2005Structuring}, \textit{-nu/-ne}, just as prefixes, assigns quantity to a verbal predicate, hence it is generated in the domain of inner aspect (Borer analyzes Slavic perfectivity as quantity, i.e. telicity). 

An open question for both Schoorlemmer and Borer is the complementary distribution of \textit{-nV/-ne} and (other) theme vowels. In addition, the complementary distribution with the secondary imperfectivizing suffix argued for in \citet{Schoorlemmer2004} cannot be extended to all Slavic languages, as we show in this paper. This means either that Slavic languages vary in this respect, or that this combination is blocked due to some morphological constraint (as hinted at in \citealt{Borer2005Structuring} for Russian and \citealt[236]{Kwapiszewski2022} for Polish), or some kind of semantic incompatibility of the two suffixes is at stake (e.g. \citealt{Jablonska2007} for Polish, \citealt{sta+:Biskup2023} for Czech); see \citet[235--236]{Kwapiszewski2022} for a recent critical assessment of both semantic and morphological constraints.

\citet{sta+:Progovac2005} also analyzes \textit{-nV/-ne} as an aspectual marker (in BCMS), but she claims that it is generated in the domain of grammatical (outer) aspect. More precisely, she proposes that this suffix denotes existential quantification in the outer AspP, where it encodes ``a single event, or, more precisely, at least one event'' \citep[109]{sta+:Progovac2005}. For instance, according to Progovac, the verb in \REF{str:ex:sta+:Progovac2005-a} has the interpretation as in \REF{str:ex:sta+:Progovac2005-b}. She substantiates her analysis of \textit{-nV/-ne} as bearing an existential feature with the fact that verbs with \textit{-nu/-ne} are easily modifiable with the adverbial \textit{jedanput} `once, one time', which she analyzes as an existential quantifier. For Progovac, further support for the analysis of \textit{\nobreakdash-nu/\nobreakdash-ne} as a marker of outer aspect comes from its complementary distribution with secondary imperfectivizing suffixes \REF{str:ex:sta+:Progovac2005-c}, which in her analysis are markers of grammatical (outer) aspect that bear the feature of universal quantification. The incompatibility of these two suffixes follows straightforwardly if they check their quantificational features in the same projection. 

\ea\label{str:ex:sta+:Progovac2005}
	\ea \gll Stefan je \minsp{(} jedanput) kuc-nu-o na prozor.\\ 
Stefan.{\NOM} {\AUX} {} once knock-nV-{\PTCP} on window.{\ACC} \\ \label{str:ex:sta+:Progovac2005-a}
\glt `Stefan knocked (at least once) on the window.' \\
\glt `There was (at least) one time that Stefan knocked on the window.'


\ex  There was some/at least one occasion X for which it is true that Stefan knocked on the window on that occasion X. \label{str:ex:sta+:Progovac2005-b}

\ex[*]{\gll kuc-nu-va-ti\\ 
knock-{\SG}-\textsc{si}-{\INF} \\
				\label{str:ex:sta+:Progovac2005-c}}
	\z
\z 

\noindent However, the compatibility with existential quantifiers such as \textit{jedanput} `once' can hardly be taken as evidence that \textit{-nV/-ne} bears the existential feature, since such adverbials are also compatible with imperfective verbs, as well as other types of perfective verbs, and not only with semelfactives (see \citealt{Milosavljevic2019} for an extensive corpus analysis of these adverbials). When it comes to the compatibility with secondary imperfectivizing suffixes, the reasoning outlined above regarding the proposal in \citet{Schoorlemmer2004} applies to Progovac’s analysis as well.

According to \citet{sta+:Svenonius2004} and \citet{sta+:Biskup2023, Biskup2023Matryoshka,sta+:Biskup2020}, \textit{-nV/-ne} is a verbalizer in Slavic languages (Russian and Czech, respectively). This claim is supported by its complementary distribution with theme vowels, which are analyzed as verbalizers in these works. In addition to its verbalizing role, this suffix also has a perfectivizing effect, i.e. it bears a perfective feature. A question that arises under this family of approaches is why \textit{-nV/-ne} is the only verbalizer with a perfective feature. 

\citet{Kwapiszewski2020}, working within the framework of Distributed Morphology, analyzes \textit{-nV/-ne} in Polish as an exponent of a complex head realizing (fused) verbal and quantity features. This analysis is based on the complementary distribution of \textit{-nV/-ne} with both theme vowels (as verbalizers) and secondary imperfectivizing suffixes in Polish. In a more recent work, \citet[231--237]{Kwapiszewski2022} refines his proposal of the semelfactive \textit{-nV/-ne} in Polish by arguing that this suffix is an exponent of a complex head comprising the verbal category head (more precisely, $v_{DO}$, given the unergative or transitive nature of the relevant verbs), the Voice head, and an aspectual perfective head (he maintains the claim that \textit{-nV/-ne} is in complementary distribution with both verbalizing and secondary imperfectivizing suffixes). While \citeauthor{Kwapiszewski2022}'s approach captures the ``dual'' behavior of this morpheme (verbalization + quantity/perfectivity) and explains its complementary distribution with theme vowels, his approach, being based on morphological operations specific to Polish, cannot be generalized to other Slavic languages since, as discussed above, this suffix is not in complementary distribution with secondary imperfectivizing suffixes in at least some languages. 

\citet{Arsenijevic2006} proposes that \textit{-nV/-ne} in BCMS is a diminutive suffix. Specifically, it introduces some bounded quantity to the interpretation of the eventuality, which is a relatively small part of a larger quantity of the same eventuality. In other words, \textit{-nV/-ne} marks a division into atomic units for the relevant eventuality. \citeauthor{Arsenijevic2006} provides examples similar in spirit to the delimitative uses of \textit{-nu/-ne} illustrated in \xxref{str:ex:RusLet}{str:ex:SCDoz} above, and offers the following explanation: \begin{quote}
    The atomic temporal interval appears as the natural interpretation when the description of an eventuality does not provide any unit of division, but division must still be applied. The natural solution is to take the atomic temporal interval as corresponding to the smallest possible quantity of the eventuality. The atomic interval also provides a partitive interpretation, when related to the mass from which it selects a unit. \citep[219]{Arsenijevic2006}
\end{quote}

Syntactically, according to Arsenijević, \textit{-nV/-ne} is the head of the VP, and marks the presence of a telic template in cases where the description of the eventuality does not define one. As an argument for this position, Arsenijević lists the incompatibility of \textit{-nV/-ne} with internal prefixes, as these morphemes also license telicity. However, \textit{-nV/-ne} can be combined with (internal) prefixes, as will be shown in \sectref{str:subsect: nV - quantitative data} for BCMS and Slovenian (see also \citealt{Nordrum2019} for such combinations in Russian, as well as \citealt{Kwapiszewski2020} for Polish).  

The presented description of the diminutive semantics of the suffix \textit{-nV} closely matches the notion of singularity. In fact, in the semantic approach of \citet{Kagan2008, Kagan2010}, both prefixes and the semelfactive suffix \textit{-nV/-ne} in Russian license singularity, but unlike prefixes, which bring additional meaning and/or argument structure effects, “the suffix -nu seems to introduce no further changes except for the singularity restriction. It takes an imperfective activity predicate and renders a perfective predicate whose denotation contains only the smallest instantiation of this activity, each of which has no proper part which instantiates the same type of event” \citep[11]{Kagan2010}; see also \citet{Milosavljevic2023PhD} for a syntactic implementation of this idea.

Relatedly, according to \citet{Armoskaite2008}, the semelfactive suffix \textit{-nV/-ne} and the secondary imperfectivizing suffix \textit{-yva} in Russian are markers of number in the verbal domain, licensing singularity and pluractionality, respectively, and thus occupy the same syntactic slot. This is supported by their complementary distribution in Russian. \citeauthor{Armoskaite2008} propose that these suffixes, as markers of verbal number, are modifiers, and not heads, contrary to what we find in the nominal domain. The arguments for a modifier analysis are the following. Heads are obligatory, modifiers are not: e.g., on nouns, the number markers are heads, and since they are obligatory, there are no nouns that are neutral with respect to number. Further, on nouns, number marking, as a head, applies even when the number information is redundant, e.g. in the presence of numerals. Finally, number as a head on nouns triggers agreement on dependent constituents, e.g. Subject-Verb Agreement. 
However, the status of number in the verbal and nominal domains can be shown not to be as different as proposed in \citet{Armoskaite2008}. On the contrary, the motivation for seeing Slavic perfectivity as singularity and imperfectivity as plurality in the verbal domain (proposed in \citealt{Kagan2008,Kagan2010}; for related approaches, see \citealt{Arsenijevic2023, sta+:Milosavljevic2022, Milosavljevic2023Advances, Milosavljevic2023PhD}) is argued to rely on compelling parallels between nominal and verbal domains: plural (imperfective) is unspecified for number, while singular (perfective) is the only marked/specified category (in the sense of \citealt{Sauerland2003}). In that sense, all nouns and all verbs are either unspecified for number (if plural, i.e. imperfective) or specified as singular, i.e. perfective. In other words, the absence of the suffix \textit{-nV/-ne} does not imply the absence of singularity, as there can be another way of realizing it (e.g. via Spec--Head agreement in the case of prefixation, see \citealt{Milosavljevic2023Advances, Milosavljevic2023PhD}), licensing a view in which it is not optional. Additionally, verbs suffixed with \textit{-nV/-ne}, just as nouns, appear in the context of numerals, i.e. with the count adverbials like \textit{once} (cf. \citealt{sta+:Progovac2005}).\footnote{On a broader scale, there seems to be a tight cross-linguistic connection between diminutives and singulatives (cf. e.g. \citealt{Rijkhoff1991}; \citealt[$§4$]{Mathieu2012}, and references therein), and more generally a link between diminutives and atomicity (see also \citealt{Wiltschko2006, DeBelder2008, DeBelder2011, Ott2011}). We contend that the link between diminution and singularity reflected through the same morphemes cross-linguistically is due to the fact that they share atomicity as a semantic core.} 

\citet{Markman2008} analyzes both the semelfactive suffix \textit{-nV/-ne} and the secondary imperfectivizing suffix \textit{-iv} in Russian as exponents of a single \textit{v}P-selecting light verb \textit{v} (in the sense of \citealt{Diesing1998}), which denotes an atelic event and is merged above lexical prefixes. The light verb is spelled out as \mbox{\textit{-nV/-ne}} when [$+$Instantaneous] and as a secondary imperfectivizing suffix when [$+$Progressive] or [$+$Habitual]. Markman follows \citet{Smith1997} in assuming that semelfactives are perfective atelic predicates. The single-head approach to the two suffixes is based on the claim that they are in complementary distribution in Russian, whereas their status as light verbs is motivated by similar behavior to light verbs cross-linguistically. A potential problem for \citet{Markman2008}, apart from the issue of complementary distribution with secondary imperfectivizing suffixes discussed above for other approaches, concerns the analysis of semelfactives as atelic predicates. In this paper, we argue that semelfactives are singular telic predicates, like other traditional perfective verbs (see also \citealt{Rothstein2008Telicity, Rothstein2008Puzzles} for an analysis of semelfactives as telic predicates in Russian).

In the next subsection, we turn to bimorphemic analyses of \textit{-nV/-ne}. 

\subsubsection{Bimorphemic analyses} \label{str:sect:bimorphemic analyses}

\citet{sta+:Lazorczyk2010} treats \textit{-nV/-ne} in Polish as composed of two morphemes: the suffix \textit{-n} as a marker of semelfactivity (deriving also a small number of degree achievements), and a theme vowel, which in her approach is a reflex of verbalization through the structure (in the sense of \citealt{Borer2005Structuring}), hence inserted once the inner aspect has been projected (since the root is categorized as a verb in the context of inner aspect). \citet{sta+:Lazorczyk2010}, however, does not elaborate her approach in any detail.

\citet{TaraldsenMedovaWiland2019} and \citet{Wiland2019}, analyzing \textit{-nV/-ne} in Czech within the framework of Nanosyntax (cf. \citealt{Caha2009, Starke2009}), propose that \textit{-n} is a light verb, whereas the vocalic segment is a theme vowel. In their approach, roots, \textit{-n} and the theme vowel can all spell out syntactic structures of different sizes (i.e. of varying syntactic complexity), with the relevant containment relations in syntax specified as in \REF{str:ex:nV-Nano}.

\ea\label{str:ex:nV-Nano}
	\ea  containment of the light verbs: 			
 
\textsc{give} > \textsc{get}

\ex  containment of the lexical categories: 	   

verb > noun > adjective

\ex  argument structure hierarchy:				      

unergative > accusative > unaccusative
				
	\z
\z 

\noindent In semelfactives, the root is nominal, \textit{-n} spells out the light verb GIVE, and the theme vowel spells out the accusative or unergative structure. In degree achievements, the root is adjectival, \textit{-n} spells out the light verb GET, and the theme vowel spells out unaccusative syntax. The relation between semelfactives and degree achievements (hence also \textit{-n} and the theme vowel in semelfactives vs. degree achievements) is regulated by the Superset Principle. According to this principle, a phonological exponent of a lexical item is inserted into a syntactic node if its lexical entry has a (sub-)constituent which matches that node. Where several items meet the conditions for insertion, the item containing fewer features unspecified in the node must be chosen \citep{Starke2009}. Given the containment relations in \REF{str:ex:nV-Nano}, the light verb component in both semelfactives and degree achievements can be spelled out as \textit{-n}.

One problem with this approach concerns the fact that it is extremely difficult to isolate nominal, adjectival, or verbal roots per se, since the same root may be categorized as a noun, verb or an adjective, depending on the categorizing morpheme and/or syntactic context. Further, this approach does not cover the full range of uses of the suffix \textit{-nV/-ne}, which easily combines also with verbal bases, and even with other suffixes (e.g. \textit{bol-uc-nu-ti} `hurt a bit', where \textit{-uc} is a diminutive suffix).

\section{Quantitative description of \textit{-nV} verbs in BCMS and Slovenian} \label{str:sect:Quantitative description}

In this section we first describe our quantitative database in \sectref{str:subsect:WeSoSlaV} and then present the quantitative data on the sequence \textit{-nV/-ne} in \sectref{str:subsect: nV - quantitative data}. In \sectref{str:subsect:Towards an analysis} we summarize the discussion and findings so far to prepare the ground for our morpho-phonological (\sectref{str:sect:Morpho-phono analysis}) and syntactic/semantic analysis (\sectref{str:sect:syntactic-semantic analysis}). 

\subsection{Our empirical source: \textit{WeSoSlaV}} \label{str:subsect:WeSoSlaV}

Our proposal is informed by quantitative insights from the \textit{Annotated Database of the Western South Slavic Verbal System} (\textit{WeSoSlaV}, \citealt{database_Lanko, WeSoSlaV_derivation, database_Boban}). %Our proposal is informed by quantitative insights from the \textit{Annotated Database of the Western South Slavic Verbal System} (\textit{WeSoSlaV}, \citealt{database}). 
The database consists of 5300 BCMS and 3000 Slovenian verbs retrieved from the \textit{srWaC, hrWaC, bsWaC} and \textit{meWaC} corpora for BCMS (\citealt{LjubesicKlubicka2014}) and from \textit{Gigafida}, the Slovenian National Corpus for Slovenian (\citealt{Logar-BergincEgAl2012}). The verbs are selected based on frequency: the top 3000 highest frequency verbs from each of the corpora are included and annotated. As \textit{srWaC, hrWaC, bsWaC} and \textit{meWaC} are corpora of different BCMS varieties, the BCMS database contains the union of the 3000-verb lists from the four corpora. 

Each verb is annotated for a fixed set of over 40 different properties, including 
%frequency, lexical and grammatical aspect as verified by the selected tests, argument structure (taking accusative, genitive, dative, PP, clausal arguments; reflexivity), 
grammatical aspect, the characteristic morphemes (the root, prefixes, suffixes), their special properties (e.g. root allomorphy), deverbal nominalizations, prosodic prominence, TVs and others. Our analysis is mainly based on the derivation subpart of WeSoSlaV \citep{WeSoSlaV_derivation} and an additional \textit{-nV}-verb subpart annotated for the purposes of this paper \citep{data_nV}.

\subsection{\textit{-nV} verbs: the quantitative data} \label{str:subsect: nV - quantitative data}
%S: Ja bih sve kvantitativne podatke ovde, jer su nam i parnjaci važni kao deo argumenta da nije da je -n- načelno nemoguće sa sekundarnom imperfektivizacijom, nego da može, a to što negde ne, jeste što tad ima drukčije parnjake, da u svakom slučaju ograničenje nije sintaksičko kao što se mnogo gde tvrdi
% S: U principu, ništa od podataka nam nije krucijalnu za deminutivnu analizu, tako da ili sve ovde, ja ću probati da motivišem, ili, ako odskače, možda sve tri tabele u appendix; ajd da vidimo na kraju
% S: Pišem opis podataka svakako, pa videćemo gde ćemo same tabele
% E: Meni je super, da je vse tukaj na kupu.

In this section, we present quantitative data on the aspectual properties of verbs formed with \textit{-nV/-ne} in both BCMS and Slovenian. We start with the correlation between (im)perfectivity and the presence of a prefix, as summarized in \tabref{str:tab:table-quant-wesoslav}. 


\begin{table}
\caption{\textit{-nV} verbs in \textit{WeSoSlav}: prefixation and (im)perfectivity}
\label{str:tab:table-quant-wesoslav}
%\vspace{1mm}
\begin{tabularx}{\textwidth}{p{23mm}YYYY}
\lsptoprule
\textit{-nV} verbs\newline  in WeSoSlav & \multicolumn{2}{p{45mm}}{BCMS  (258 in total, 4.87\% of all the verbs in WeSoSlav)} & \multicolumn{2}{p{45mm}}{Slovenian (143 in total, 4.77\% of all the verbs in WeSoSlav)} \\
%\midrule    
\cmidrule(lr){2-3}\cmidrule(lr){4-5}            
  & Unprefixed  &  Prefixed & Unprefixed    &   Prefixed\\ \midrule
All &  91/258 (35.27\%)    & 167/258 (64.73\%)    &  24/143 (16.78\%)  & 119/143 (83.22\%) \\\addlinespace %\midrule
Imperfective  & 9/258 (3.49\%)    & 0 (0\%)    &  3/143 (2.10\%)  &  0 (0\%) \\\addlinespace %\midrule
Perfective  & 82/258 (31.78\%) & 167/258 (64.73\%)  & 21/143 (14.69\%)   & 119/143 (83.22\%) \\
\lspbottomrule
\end{tabularx}
\end{table}

As is clear from the table, all prefixed verbs are perfective.\footnote{Out of 167 prefixed \textit{-nV} verbs in BCMS, 95 (56.89\%) combine with a perfective base, 23 (13.77\%) combine with an imperfective base, while in 49 (29.34\%) cases there is a bound base (i.e. a base that is not attested without a prefix). Out of the 119 prefixed \textit{-nV} verbs in Slovenian, 43 (36.13\%) combine with a perfective base, 13 (10.92\%) combine with an imperfective base, while in 62 (52.1\%) cases the base is bound.} The very existence of prefixed \textit{-nV/-ne} verbs is theoretically significant since it shows that \textit{-nV/-ne} and prefixes can be combined, contrary to some approaches reviewed in \sectref{str:sec:previous-analyses} above.\footnote{The majority of such prefixes are lexical/internal prefixes, e.g. in BCMS: \textit{pod-met-nu-ti} [\textsc{under}-put-nV-{\INF}] `set up, put under', \textit{od-gur-nu-ti} [\textsc{from}-push-nV-{\INF}] `push away', \textit{s-kliz-nu-ti} [\textsc{off}-glide-nV-{\INF}] `slip', \textit{u-tis-nu-ti} [\textsc{in}-press-nV-{\INF}] `press in', \textit{iz-tis-nu-ti} [\textsc{out}-press-nV-{\INF}] `press out'. Although in our main database (WeSoSlaV) there are no typical examples with superlexical prefixes, such verbs are possible, especially in the presence of another prefix, which is expected given that the most typical superlexical prefixes stack on top of other prefixes. Some such examples, taken from \citet{Stojanović2016}, include: \textit{iz-o-kre-nu-ti} [\textsc{out}-\textsc{about}-start-nV-{\INF}] `turn over all', \textit{po-o-smeh-nu-ti} [\textsc{over}-\textsc{about}-laugh-nV-{\INF}-{\REFL}] `laugh a little bit'. However, there are also superlexical-like prefixes, such as the attenuative \textit{pri-}, which combine directly with -\textit{nV} verbs, e.g. \textit{pri-drem-nu-ti} [\textsc{at}-doze-nV-{\INF}] `doze a little bit'. A similar picture is observed in Slovenian. An example of LP-prefixed nV-verbs is \textit{iz-tis-ni-ti} [\textsc{out}-press-nV-{\INF}] `press out', whereas \textit{po-na-tis-ni-ti} [\textsc{over}-\textsc{on}-press-nV-{\INF}] `reprint' illustrates SLPs.} Another important point that \tabref{str:tab:table-quant-wesoslav} makes salient is that the vast majority of unprefixed \textit{-nV/-ne} verbs are perfective. Specifically, out of 91 unprefixed verbs in BCMS, 82 (90.11\%) are perfective, and only 9 (9.89\%) are imperfective. Similarly, out of 23 unprefixed verbs in Slovenian, 18 (78.26\%) are perfective, and only 3 (13.04\%) are imperfective.\footnote{Out of 9 imperfective verbs in BCMS, 7 are degree achievements, and 2 are lexicalized states. Out of the 3 imperfective verbs in Slovenian, 1 is a degree achievement, and 2 are lexicalized states.} These data, together with the fact that new verbs (including the ones with borrowed bases) are always perfective in BCMS and Slovenian (for the former, see also \citealt{Simonovic2015}), strongly indicate that only perfective \textit{-nV/-ne} verbs are productive in the contemporary BCMS and Slovenian. The same has been observed also for other Slavic languages, e.g. Polish (\citealt{KlimekEtAl2018}), Czech (\citealt{TaraldsenMedovaWiland2019, Wiland2019}), Russian (\citealt{Sokolova2015}). This justifies our choice to focus on perfective verbs in this paper.

We now turn to the quantitative patterns of aspectual pairs \textit{-nV/-ne} verbs participate in. Tables \ref{str:tab:table-simple} and \ref{str:tab:table-prefixed} summarize these patterns separately for prefixed and unprefixed verbs.\footnote{The examples of the categories in the first column of this table are in BCMS. There are only 4 and 2 simple perfective verbs with an imperfective secondary imperfective counterpart preserving -\textit{nV} in BCMS and Slovenian, respectively. The remaining three BCMS pairs from WeSoSlav are: \textit{buk-nu-ti -- buk-nj-iva-ti} `erupt', \textit{pla-nu-ti -- pla-nj-ava-ti} `burst into flames', and \textit{ba-nu-ti -- ba-nj-ava-ti} `burst'. The Slovenian verbs and their imperfective counterparts are: \textit{mi-ni-ti} -- \textit{mi-n-eva-ti} `pass', and \textit{ga-ni-ti} -- \textit{ga-nj-ati} `move'.}

\begin{table}
\caption{Imperfective counterparts of unprefixed perfective verbs}
\label{str:tab:table-simple}
%\vspace{1mm}
\begin{tabularx}{\textwidth}{@{}l@{}rrrr}
\lsptoprule
Simple {\PFV} \textit{-nV} verbs in \textit{WeSoSlaV} with \dots\ & \multicolumn{2}{c}{BCMS    (N=82)} &\multicolumn{2}{c}{  Slovenian (N=21)} \\
\midrule     
an {\IPFV} root-TV counterpart\\
(lup-nu-ti – lup-a-ti `slap')   & 43 & (52.44$\%$)   & 10 & (47.62$\%$)  \\ \addlinespace %\midrule
an {\IPFV} \textit{-t-} counterpart\\
(trep-nu-ti – trep-ta-ti `blink')      & 11 & (13.41$\%$)    & 0 & (0$\%$) \\ \addlinespace %\midrule
an {\IPFV} \textit{-k-} counterpart\\
(tres-nu-ti – tres-ka-ti `snap')  & 24 & (29.27$\%$)    & 0 & (0$\%$) \\\addlinespace %\midrule
{\IPFV} SI counterpart, without preserving \textit{-nV}\\
(crk-nu-ti – crk-ava-ti `die')    & 8 & (9.76$\%$)    & 4 & (19.05$\%$) \\
\addlinespace %\midrule
an {\IPFV} apophonical counterpart\\
(mak-nu-ti – mitːc-a-ti `move')   & 4 & (4.88$\%$) & 1 & (4.76$\%$) \\ \addlinespace %\midrule
an {\IPFV} SI counterpart, preserving \textit{-nV}\\
(sva-nu-ti – sva-nj-ava-ti `dawn')    & 4 & (4.88$\%$) & 2 & (9.52$\%$)\\
\lspbottomrule
\end{tabularx}
\end{table}


% \begin{table}
%    \centering
% \caption{Imperfective counterparts of unprefixed perfective verbs in \textit{WeSoSlaV}}
% \label{str:tab:table-simple-suggestion}
% \begin{tabularx}{\textwidth}{llYY}
% \lsptoprule
%  && BCMS  (N=82) & Slovenian (N=21) \\
% \midrule     
% \multicolumn{2}{l}{Simple perfective \textit{-nV} verbs with\ldots}
% \\\addlinespace
% &an {\IPFV} root-TV counterpart & 42 & 10\\
% &(e.g. lup-nu-ti – lup-a-ti `slap')   & (51.22$\%$)   & (47.62$\%$)  \\ \addlinespace \midrule
% &an {\IPFV} \textit{-t-} counterpart & 11 & 0 \\
% &(e.g. trep-nu-ti – trep-ta-ti `blink')      & (13.41$\%$)    &(0$\%$) \\ \addlinespace \midrule
% &an {\IPFV} \textit{-k-} counterpart  & 24 & 0 \\
% &(e.g. tres-nu-ti – tres-ka-ti `snap')  & (29.27$\%$)    &  (0$\%$) \\\addlinespace \midrule
% &an {\IPFV} SI counterpart, without preserving \textit{-nV} & 7 & 5\\
% &(e.g. crk-nu-ti – crk-ava-ti `die')    & (8.54$\%$)    &(23.81$\%$) \\
% \addlinespace %\midrule
% &an {\IPFV} apophonical counterpart & 4 & 1\\
% &(e.g. mak-nu-ti – mi:c-a-ti `move')   & (4.88$\%$) & (4.76$\%$) \\ \addlinespace \midrule
% &an {\IPFV} SI counterpart, preserving \textit{-nV}  & 4  & 2 \\
% &(e.g. sva-nu-ti – sva-nj-ava-ti `dawn')    & (4.88$\%$) &  (9.52$\%$) \\
% \lspbottomrule
% \end{tabularx}
% \end{table}

\begin{table}
\caption{Imperfective counterparts of prefixed perfective verbs}
\label{str:tab:table-prefixed}
%\vspace{1mm}
\begin{tabularx}{\textwidth}{@{}lrrrr}
%\begin{tabularx}{\textwidth}{l@{~~}Y@{~~}Y}
\lsptoprule
{\small Prefixed {\PFV} \textit{-nV} verbs in \textit{WeSoSlaV} with} \dots\ & \multicolumn{2}{c}{BCMS (N=167)} & \multicolumn{2}{c}{Slovenian (N=119)} \\
\midrule                
an {\IPFV} \textit{-t} counterpart      & 34 & (20.36$\%$)    & 0 & (0$\%$) \\ \addlinespace %\midrule
an {\IPFV} \textit{-k} counterpart  & 2 & (1.20$\%$)    & 7 & (5.88$\%$) \\ \addlinespace %\midrule
an {\IPFV} \textit{-p} counterpart  & 0 & (0$\%$)    & 6 & (5.04$\%$) \\ \addlinespace %\midrule
an {\IPFV} SI counterpart,\\without preserving \textit{-nV}    & 60 & (35.93$\%$)    & 56 & (47.06$\%$) \\ \addlinespace %\midrule
an {\IPFV} apophonical counterpart  & 21 & (12.57$\%$) & 20 & (16.81$\%$) \\ \addlinespace %\midrule
an {\IPFV} SI counterpart, preserving \textit{-nV}    & 55 & (32.93$\%$) & 18 & (15.13$\%$) \\
\lspbottomrule
\end{tabularx}
\end{table}



We consider prefixed and unprefixed verbs separately to control for the possible influence of prefixation. For instance, on the one hand, \citet{sta+:Biskup2023} refers to \citet{Isacenko1962} and \citet{Townsend1968} for the claim that prefixed semelfactive verbs are not semelfactive anymore, i.e. they behave like any other prefixed perfective verb. On the other hand, \citet{Kwapiszewski2022} indicates that the presence of a prefix in Polish does not change the fact that in that language \textit{-nV/-ne} verbs cannot undergo secondary imperfectization. For our purposes, two facts evident from Tables \ref{str:tab:table-simple} and \ref{str:tab:table-prefixed} are most significant. First, in the majority of cases, the imperfective aspectual counterpart is either a corresponding unsuffixed verb (i.e. a verb whose root is followed just by a theme vowel), or a verb with some kind of iterative suffix.\footnote{The suffixes \textit{-t} and \textit{-k} that derive diminutive-iterative verbs are traditionally listed as \textit{-ka} and \textit{-ta} in BCMS grammars (e.g., \citealt{StanojcicPopovic2008}). However, these suffixes can also be plausibly decomposed into the proper (diminutive-iterative) suffixes and theme vowels, specifically, \textit{k} + TV \textit{a/a} (e.g. \textit{pip-k-a-ti} ({\INF}), \textit{pip-k-a-mo} ({\PRS.1.\PL}) `touch'), and \textit{t} + TV \textit{a/je} (\textit{trep-t-a-ti} ({\INF}),\textit{ trep-ć-e-mo} < /trep-t-je-mo/ ({\PRS.1.\PL}) `blink'). These two theme vowels (i.e., \textit{a/a} and \textit{a/je}) are two of the three most productive TVs in BCMS that are also found in secondary imperfectivizing suffixes \citep{SimonovicEtAl2021, ArsenijevicEtAl2023}.}
Second, there are both unprefixed and prefixed verbs that undergo secondary imperfectivization at the same time preserving the morpheme \textit{-nV}.\footnote{In addition to the examples used as an illustration in \tabref{str:tab:table-simple}, this pattern can be illustrated by the following prefixed verbs: \textit{na-dah-nu-ti -- na-dah-nj-ivati} `inspire', \textit{za-bezek-nu-ti -- za-bezek-nj-iva-ti} `bewilder' for BCMS; and \textit{s-tr-ni-ti} -- \textit{s-tr-nj-eva-ti} `sum up', \textit{za-mrz-ni-ti} -- \textit{za-mrz-nj-eva-ti} `freeze', \textit{u-ki-ni-ti} -- \textit{u-ki-nj-a-ti} `abolish, cancel', \textit{raz-gr-ni-ti} -- \textit{raz-gri-nj-a-ti} `unfold, spread out' for Slovenian.} The first fact is important in the light of our analysis of verbs derived by \textit{-nV/-ne} as diminutive counterparts of the verbal predicates denoted by the corresponding imperfective verbs, as argued in detail in \sectref{str:sect:syntactic-semantic analysis}. The other fact, i.e. the compatibility of \textit{-nV/-ne} with secondary imperfectivizing suffixes in at least some verbs, corroborates our claim that the two suffixes are not in complementary distribution, contrary to much previous work (see \sectref{str:sec:previous-analyses}).\footnote{See also \citet{Milosavljevic2023PhD} for the discussion of secondary imperfective forms of semelfactive verbs in South-East Serbo-Croatian, where such forms are much more productive.}   


\subsection{Towards an analysis} \label{str:subsect:Towards an analysis}
So far, we have overviewed previous approaches and presented our quantitative data. We have seen that existing analyses, both monomorphemic and bimorphemic, face both empirical and theoretical issues, at least when applied to Western South Slavic. On the empirical side, it was shown by our quantitative data that some central assumptions in the majority of previous approaches (e.g. complementary distribution of \textit{nV/-ne} and secondary imperfectivizing suffixes) do not hold for all the verbs in Western South Slavic. As for the monomorphemic analyses, apart from the issues discussed in \sectref{str:sect:monomorphemic analyses}, we can add that analyzing \textit{\nobreakdash-nV} as a monomorphemic theme vowel leaves open the question of why, unlike all other themes, this theme vowel includes a (non-glide) consonant and is the only theme vowel across Slavic languages that performs a perfectivizing function. An analysis splitting \textit{-nV/-ne} into \textit{-n} as a separate morpheme and \textit{u/e} and \textit{i/e} as a theme vowel in BCMS and Slovenian respectively lends itself as a solution. While a similar segmentation has already been proposed (see \sectref{str:sect:bimorphemic analyses}), the approaches are either not elaborated \citep{sta+:Lazorczyk2010}, or do not cover all the empirical data \citep{TaraldsenMedovaWiland2019, Wiland2019}. In the latter case, it is assumed that \textit{-nV} combines with the nominal bases to derive perfective semelfactive (unergative) verbs (in Czech and Polish), but the same suffix, at least in Western South Slavic, also readily combines with verbal bases, and even with other suffixes, e.g. \textit{bol-uc-nu-ti} `hurt a bit', where \textit{-uc} is a verbal diminutive suffix. In the following sections, we use this compatibility with other (diminutive) suffixes to argue that \textit{-nV} is itself a diminutive suffix \textit{-n\textsuperscript{u}} (BCMS)/\textit{-n\textsuperscript{i}} (Slovenian), which selects the theme vowel \textit{$∅$/e}.

\section{Morpho(-phono)logical analysis} \label{str:sect:Morpho-phono analysis}

In this section, we present morpho-phonological arguments for our main claim that the sequence \textit{-nV/-ne} is composed of the suffix \textit{-n\textsuperscript{u}} (BCMS)/\textit{-n\textsuperscript{i}} (Slovenian), and the theme vowel \textit{$∅$/e}. 
The theme-vowel class \textit{$∅$/e} is  independently attested with simple verbs in both BCMS and Slovenian \citep{database_Boban}. The question remains how to treat the vowels \textit{u} and \textit{i} that appear next to the consonant \textit{-n} when it is combined with the \textit{$∅$}-exponent of the theme vowel, but do not appear when it is combined with the \textit{e}-exponent. We propose that the morpheme under consideration has both a consonantal and vocalic part, but that only the consonantal part is lexically affiliated with a timing slot, whereas the vocalic part is floating. This approach has already been applied to the Polish cognate of the same morpheme in \citet{Zdziebko2017}. We submit that the realization of the floating vowels is regulated by syllable structure constraints. Floating vowels surface in front of consonant-initial endings (helping to prevent consonant clusters) and they do not surface before vowel-initial endings (because realizing them would create a hiatus). This is illustrated in \REF{str:ex:Float}. Specifically, the floating vowel helps avoid the consonant clusters \textit{nt} and \textit{nl} in the infinitive and participle forms (\ref{str:ex:Float-a}, \ref{str:ex:Float-b}). These clusters do not appear in the verbal systems of BCMS and Slovenian. On the other hand, the floating vowels are not realized before vowels \textit{e} or \textit{i} in the present tense and the imperative forms (\ref{str:ex:Float-c}, \ref{str:ex:Float-d}) because in this case full (i.e. non-floating) segments already constitute optimal open syllables and the realization of the floating vowels would lead to a hiatus.

\ea\label{str:ex:Float}
	\ea \gll max-n$^u$-$∅$-ti $→$ maxnuti, *maxnti \\ 
max-n$^i$-$∅$-ti $→$ maxniti, *maxnti \\ \jambox*{(BCMS)}
\glt wave-nV-\textsc{tv}-{\INF}\hfill \raisebox{\baselineskip}[0pt][0pt]{(Slovenian)}\label{str:ex:Float-a}

\ex \gll max-n$^u$-$∅$-l-a $→$ maxnula, *maxnla \\
max-n$^i$-$∅$-l-a $→$ maxnila, *maxnla \\ \jambox*{(BCMS)}
\glt wave-nV-\textsc{tv}-{\PST-\FEM}\hfill \raisebox{\baselineskip}[0pt][0pt]{(Slovenian)}\label{str:ex:Float-b}

\ex \gll max-n$^u$-$e$-mo $→$ maxnemo, *maxnuemo \\ 
max-n$^i$-$e$-mo $→$ maxnemo, *maxniemo \\ \jambox*{(BCMS)}
\glt wave-nV-\textsc{tv}-{\PRS.1\PL}\hfill \raisebox{\baselineskip}[0pt][0pt]{(Slovenian)}\label{str:ex:Float-c}

\ex \gll max-n$^u$-$i$-mo $→$ maxnimo, *maxnuimo \\ 
max-n$^i$-i-mo $→$ maxnimo, *maxniimo \\ \jambox*{(BCMS)}
\glt wave-nV-\textsc{tv}.{\IMP-1.\PL}\hfill \raisebox{\baselineskip}[0pt][0pt]{(Slovenian)}\label{str:ex:Float-d}

\z
\z

\noindent An important argument for adding \textit{-nV/-ne} verbs to the \textit{$∅$/e} class lies in the fact that the forms in \REF{str:ex:Float} (as well as the rest of the paradigm) feature the endings typical of \textit{$∅$/e} verbs in general. The only potential exception is constituted by passive participle forms, which we discuss below.

Before turning to the discussion of the passive participle forms, we need to address an alternative to adding floating vowels to the \textit{n}-morpheme. The same surface result could have been achieved by assuming the \textit{n}-morpheme just with a full consonant and adding the floating vowel to the representation of the theme vowel. In this case, the  \textit{$∅$/e} class would become \textit{$^u$/e} in BCMS and \textit{$^i$/e} in Slovenian. This alternative account encounters an empirical problem, as it would predict the floating vowels to surface in all forms where consonant-final bases combine with consonant-initial endings, e.g. in \textit{pad-$∅$-ti → pasti, *paduti, *paditi} `fall.{\INF}’ or \textit{griz-$∅$-ti → gristi, *grizuti, *griziti} `bite.{\INF}’.   

As mentioned above, adding \textit{-nV/-ne} verbs to the  \textit{$∅$/e} class does appear to face some potential empirical issues. In both languages, the passive participle of \textit{\nobreakdash-nV/\nobreakdash-ne} verbs diverges from most \textit{$∅$/e} verbs. Since BCMS and Slovenian differ at this point, we take a closer look at each language in the following two subsections.

\subsection{BCMS}

The regular passive participle suffix in the \textit{$∅$/e} conjugation in BCMS is \textit{-en}, as illustrated in \REF{str:ex:pass} by the verbs \textit{ukrasti} `steal' and \textit{ugristi} `bite'. Given the vowel-initial ending \textit{-en}, for \textit{-nu/-ne} verbs, we would expect the passive participle form ending in \textit{-nen} (with non-realization of the floating vowel, just like in the present tense and in the imperative in \ref{str:ex:Float-c} and \ref{str:ex:Float-d}). However, the actual passive participles of these verbs end in \textit{-nut}, as shown in \REF{str:ex:pass1} for the verb \textit{dirnuti} `touch'.

\ea\label{str:ex:pass}
	\ea \gll \parbox{2.7cm}{ukra\sout{d}-$∅$-l-a} | ukrad-$∅$-en\\ 
steal-\textsc{tv}-{\PST-\FEM} | steal-\textsc{tv}-{\PASS.\PTCP}\\
\glt \label{str:ex:pass-a}

\ex \gll \parbox{2.7cm}{ugriz-$∅$-l-a} | ugriz-$∅$-en\\ 
bite-\textsc{tv}-{\PST-\FEM} | bite-\textsc{tv}-{\PASS.\PTCP}\\
\glt \label{str:ex:pass-b}

\z

\ex\label{str:ex:pass1}
	
 \gll \parbox{3.3cm}{dir-nu-$∅$-l-a} | dir-nu-$∅$-t, \minsp{*} dir-n\sout{u}-$∅$-en\\ 
touch-nV-\textsc{tv}-{\PST-\FEM} | touch-nV\textsc{-tv}-{\PASS.\PTCP} {} touch-nV-\textsc{tv}-{\PASS.\PTCP}\\
\glt \label{str:ex:pass1-a}

\z


\noindent As it turns out, the \textit{$∅$/e} class is more heterogeneous than our initial overview reveals. If we zoom into verbs whose infinitival stems end in round vowels, we can find three roots that derive verbs with infinitives in \textit{-uti}. These are illustrated in \REF{str:ex:pass2} by the forms of the verbs \textit{obuti} `put shoes on', \textit{načuti} `overhear' and \textit{nasuti} `pour'. As can be observed in \REF{str:ex:pass2}, the passive participle form in such cases can end in \textit{-t} for the first two verbs, and it obligatorily ends in \textit{-t} for the third listed verb. This indicates that \textit{-nuti} verbs do not show atypical behavior with respect to other \textit{-uti} verbs in the system. It can thus be submitted that the passive participle allomorph [-t] is conditioned by the adjacent [$+$round] feature (as one of its contexts of insertion).\footnote{This allomorph shows up in several other environments in the classes \textit{$∅$/e} and \textit{a/a}. As shown in \citet{Beslin2023}, its conditioning is at least partially lexical.} Once this consonantal allomorph is selected, it comes as no surprise that [nu] surfaces as the exponent of  \textit{n$^u$}, since, as stated above, the \textit{nt} cluster is blocked in the verbal forms in general.   

\ea\label{str:ex:pass2}
	\ea \gll obu-$∅$-l-a | obuʋ-$∅$-en, \newline \textsuperscript{?}obu-$∅$-t\\ 
put.shoes.on-\textsc{tv}-{\PST-\FEM} | put.shoes.on-\textsc{tv}-{\PASS.\PTCP} \phantom{\textsuperscript{?}}put.shoes.on-\textsc{tv}-{\PASS.\PTCP}\\
\glt \label{str:ex:pass2-a}

\ex \gll \parbox{3cm}{nat͡ʃu-$∅$-l-a} | nat͡ʃu-$∅$-t, nat͡ʃuʋ-$∅$-en\\ 
overhear-\textsc{tv}-{\PST-\FEM} | overhear-\textsc{tv}-{\PASS.\PTCP} overhear-\textsc{tv}-{\PASS.\PTCP}\\
\glt \label{str:ex:pass2-b}

\ex \gll \parbox{3cm}{nasu-$∅$-l-a} | nasu-$∅$-t\\ 
pour-\textsc{tv}-{\PST-\FEM} | pour-\textsc{tv}-{\PASS.\PTCP}\\
\glt \label{str:ex:pass2-c}

\z
\z


\noindent Based on the facts above, it is safe to conclude that the allomorph selection in passive participle forms of \textit{-nV/-ne} verbs does not constitute an argument for excluding these verbs from the \textit{$∅$/e} theme-vowel class. 

\subsection{Slovenian}

In Slovenian, just like in BCMS, the regular passive participle suffix in the $∅$/e conjugation is \textit{-en} (pronounced as {[-ɛn]} when under stress), as illustrated in \REF{str:ex:pass3} for the verbs \textit{ukrasti} `steal' and \textit{gristi} `bite'.\footnote{The contrast between open-mid vowels {[ɛ, ɔ]} and close-mid vowels {[e, o]} can only be observed in stressed syllables. In unstressed syllables, the neutralized mid vowels are traditionally transcribed as close-mid. For clarity, we mark stress in the examples in this subsection.} Here again, given the vowel-initial ending, we would expect passive participles derived from \textit{-ni/-ne} verbs to end in \textit{-nen}. However, the actual passive participles of these verbs end in \textit{-njen}, as can be observed from \REF{str:ex:mah}. 

 \ea\label{str:ex:pass3}
	\ea \gll \parbox{3cm}{uˈkrad-$∅$-l-a} | uˈkrad-$∅$-en\\ 
steal-\textsc{tv}-{\PST-\FEM} | steal-\textsc{tv}-{\PASS.\PTCP}\\
\glt \label{str:ex:pass3-a}

\ex \gll \parbox{3cm}{ˈgriz-$∅$-l-a} | ˈgriz-$∅$-en\\ 
bite-\textsc{tv}-{\PST-\FEM} | bite-\textsc{tv}-{\PASS.\PTCP}\\
\glt \label{str:ex:pass3-b}

\z

\ex
	
 \gll napix-n$^i$-$∅$-en $→$ naˈpixnjen, *napixn\sout{i}en \\
 inflate-nV-\textsc{tv}-{\PASS.\PTCP}\\ \label{str:ex:mah}
\z

\noindent We suggest that the passive participle morpheme is actually \textit{-\textsuperscript{j}en}, with a floating \textit{j}. This hypothesis is supported by the fact that in the \textit{$∅$/e} class there are verbs (beyond \textit{-ni/-ne} verbs) where the passive participle suffix causes the palatalization of the preceding consonant. Such (admittedly rare) verbs are illustrated in \REF{str:ex:pas4}. We propose that since both the \textit{-n\textsuperscript{i}} morpheme and the passive participle ending \textit{-\textsuperscript{j}en} have floating segments (which in addition have the same features), there is a \textit{cumulative faithfulness effect} \citep{FarrisTrimble2008} strong enough to make the insertion of an additional timing slot and the realization of the [j] obligatory.\footnote{The palatalization in passive participles in \textit{\textsuperscript{(-j)}en} is at least partially lexically determined in Slovenian. This has been discussed for the \textit{i/i} class in \citet{sta+:Toporisic2000}. The \textit{i/i} class features triplets like \textit{ponuditi} `offer', \textit{začuditi} `bewilder', \textit{prisoditi} `attribute', whose passive participles are \textit{ponujen/ponuden}, \textit{začuden} and \textit{prisojen}, respectively ([j] being derived from /dj/). Note that in the \textit{i/i} class palatalization is much more common than in the \textit{$∅$/e} class. This is expected on our account because in the former class both the original theme vowel (\textit{i}) and the morpheme \textit{\textsuperscript{-j}en} favor palatalization.} 

\ea\label{str:ex:pas4}
	\ea \gll \parbox{3.5cm}{preˈnɛs-$∅$-l-a} | preneˈʃ-$∅$-ɛn\\ 
transfer-\textsc{tv}-{\PST-\FEM} | transfer-\textsc{tv}-{\PASS.\PTCP}\\ \label{str:ex:pas4-a}

\ex \gll \parbox{3.5cm}{preˈrast-$∅$-l-a} | preˈraʃt͡ʃ-$∅$-en\\ 
grow.over\textsc{tv}-{\PST-\FEM} | grow.over-\textsc{tv}-{\PASS.\PTCP}\\ \label{str:ex:pas4-b}

\z
\z

\noindent After having provided morpho-phonological evidence for the decomposition of the sequence \textit{-nV/-ne} into the suffix proper (\textit{n\textsuperscript{V}}) and the theme vowel \textit{$∅$/e}, we are now in a position to turn to our syntactic and semantic analysis of the suffix \textit{-n} as a diminutive suffix.

\section{The syntactic-semantic analysis in terms of diminution} \label{str:sect:syntactic-semantic analysis}

As already previewed, our analysis of the verbal suffix \textit{-nV/-ne} is bimorphemic. In this section, we focus on the proposed morpheme \textit{-n\textsuperscript{u}} (BCMS)/\textit{-n\textsuperscript{i}} (Slovenian), which we argue is a diminutive suffix. We start in \sectref{str:subsect:Special status of nu among suffixes} by showing the special status of \textit{-nV} among suffixes: its perfective nature, its possibility to participate in suffix stacking, and the theme vowel it combines with. In \sectref{str:subsect:Diminution in verbs and nouns} we sketch some similarities in the diminution of verbs and nouns that will be important for our analysis of the suffix \textit{\textit{-nV}}. Our syntactic modeling and formal semantic description are provided in \sectref{str:subsect:Syntactic modelling} and \sectref{str:subsect:Semantic formal description}, respectively. \sectref{str:subsect:How_does_nV_analyzed_in_this_way_fit_the_broader_picture} brings a discussion on how the suffix \textit{-nV} fits the broader picture of suffixes in Western South Slavic. Finally, in \sectref{str:subsect:Advantages_of_our_analysis} we compare our analysis to the previous approaches to the suffix \textit{-nV} and outline the advantages of our analysis.

\subsection{Special status of \textit{-nV} among suffixes} \label{str:subsect:Special status of nu among suffixes}

The first important property that sets the suffix \textit{-nV} apart from all other verbal suffixes in BCMS and Slovenian concerns its aspectual effects. Specifically, all other verbal suffixes in BCMS and Slovenian derive verbs that pass tests for imperfectivity and atelicity. This is evidenced in \REF{str:ex:strajk-ova-imperfectivity} and \REF{str:ex:stas-ava-imperfectivity} by the compatibility of BCMS \textit{ova/uje}- and \textit{ava}-verbs with the phasal verb \textit{početi} `begin', as well as by their combinability with durative adverbials (\ref{str:ex:strajk-ova-atelicity}, \ref{str:ex:stas-ava-atelicity}). The suffix \textit{-nV}, by contrast, derives verbs that systematically fail both these tests, as illustrated in \REF{str:ex:vik-ne}. In other words, the suffix \textit{-nV} derives only perfective/telic verbs. 

%All WSS verbal suffixes derive verbs that pass tests for imperfectivity and atelicity, except for one: suffix -nV, which derives verbs that systematically fail both these tests. 

% Treba tight alignment za specifikovanje jezika izvornika

\ea\label{str:ex:strajk-ova}
	\ea \gll Jan je počeo da štrajk-uj-e.\\ 
Jan {\AUX} begun {\COMP} strike-\textsc{suff}-{\PRS.3.\SG}\\ \jambox*{(BCMS)}
\glt `Jan began to strike.' \label{str:ex:strajk-ova-imperfectivity}

\ex \gll Jan štrajk-uj-e dva sata.\\ 
Jan strike-\textsc{suff}-{\PRS.3.\SG} two hours\\
\glt `Jan has been striking for two hours.' \label{str:ex:strajk-ova-atelicity}

\z

\ex\label{str:ex:stas-ava}
	\ea \gll Ovas je počeo da stas-av-a.\\ 
oat {\AUX} begun {\COMP} grow-\textsc{suff}-{\PRS.3.\SG}\\ \jambox*{(BCMS)}
\glt `Oat began to mature.’ \label{str:ex:stas-ava-imperfectivity}

\ex \gll Ovas stas-av-a dva dana.\\ 
oat grow-\textsc{suff}-{\PRS.3.\SG} two days\\
\glt `Oat has been maturing for two days.' \label{str:ex:stas-ava-atelicity}

\z

\ex\label{str:ex:vik-ne}
	\ea[*] {\gll Jan je počeo da vik-n-e.\\ 
Jan {\AUX} begun {\COMP} shout-\textsc{suff}-{\PRS.3.\SG}\\ \jambox*{(BCMS)}
\glt Intended: `Jan began to shout.' \label{str:ex:vik-ne-imperfectivity}}

\ex[*] {\gll Jan vik-n-e dva sata.\\ 
Jan shout-\textsc{suff}-{\PRS.3.\SG} two hours\\
\glt Intended: `Jan has been shouting for two hours.' \label{str:ex:vik-ne-atelicity}}

\z
\z

\noindent Another important property of the suffix \textit{-nV} is its possibility to license the stacking of other verbal suffixes on top of it, unlike most other suffixes, as illustrated by the contrast between \REF{str:ex:tanc1-a} and \REF{str:ex:tanc1-b} on the one hand, and \REF{str:ex:tanc1-c} on the other. Except for \textit{-nV}, the only suffixes that allow stacking of suffixes on top of them are other diminutive suffixes (with which \textit{-nV} forms a natural class), such as \textit{-k} in BCMS \REF{str:zapitkivao}, or \textit{-lj} in Slovenian \REF{str:ex:tanc1-d} (as well as some of the suffixes 
% \textit{-ir} and \textit{-is/-iz}, 
which integrate borrowed verbs).

 \ea\label{str:ex:tanc1}
	\ea[*] {\gll Jan je štrajk-ov-av-a-o.\\ 
Jan {\AUX} strike-\textsc{suff-suff-tv}-{\PST.\MASC}\\ \jambox*{(BCMS)}
 \label{str:ex:tanc1-a}}

\ex[*] {\gll Ovas je stas-av-av-a-o.\\ 
oat {\AUX} grow-\textsc{suff-suff-tv}-{\PST.\MASC}\\ \jambox*{(BCMS)}
 \label{str:ex:tanc1-b}}

\ex \gll Dan je sva-n$^u$-av-a-o [sʋaɲaʋao].\\ 
day {\AUX} dawn-\textsc{suff-suff-tv}-{\PST.\MASC}\\ \jambox*{(BCMS)}
\glt `The day was dawning.'
 \label{str:ex:tanc1-c}

 \ex \gll Pera je za-pit-k-iv-a-o Lazu.\\ 
P {\AUX} \textsc{pref}-ask-\textsc{suff-suff-tv}-{\PST.\MASC} L.\\ \jambox*{(BCMS)}
\glt `Pera was asking Laza questions.'
 \label{str:zapitkivao}

\ex \gll Jan je rez-lj-av-a-l les.\\ %Nisam siguran šta ilustruje ovaj primer
%Isto sto i onaj iznad njega << the possibility of diminutive suffixes (-nu, -lj ...) to license the stacking of other verbal suffixes on top of it
Jan {\AUX} carve-\textsc{suff-suff-tv}-{\PST.\MASC} wood\\ \jambox*{(Slovenian)}
\glt `Jan was carving out wood.'
 \label{str:ex:tanc1-d}

\z
\z

\noindent The final unique property of the suffix \textit{-nV} concerns theme vowel selection. Specifically, all Western South Slavic verbal suffixes take a theme vowel combination which includes the theme \textit{-a} (i.e. \textit{a/a} or \textit{a/je}), as illustrated in \xxref{str:ex:tanc2-a}{str:ex:tanc2-d}, whereas only \textit{-nV} combines with the theme vowel \textit{$∅$/e}, as in (\ref{str:ex:tanc2-e}, \ref{str:ex:tanc2-f}).

 \ea\label{str:ex:tanc2}
	\ea \gll Marija je gril-ov-a-l-a povrće.\\ 
M {\AUX} grill-\textsc{suff-tv}-{\PST-\FEM} vegetables\\ \jambox*{(BCMS)}
\glt `Marija was grilling the vegetables.'\label{str:ex:tanc2-a}

\ex \gll Marija je pre-poruč-iv-a-l-a povrće.\\ 
M {\AUX} \textsc{pref}-message-\textsc{suff-tv}-{\PST-\FEM} vegetables\\ \jambox*{(BCMS)}
\glt `Marija was recommending the vegetables.' \label{str:ex:tanc2-b}

\ex \gll Marija je gril-uc-k-a-l-a povrće.\\ 
M {\AUX} grill-\textsc{suff-suff-tv}-{\PST-\FEM} vegetables\\ \jambox*{(BCMS)}
\glt `Marija was grilling the vegetables a little bit.'
 \label{str:ex:tanc2-c}

\ex \gll Marija je marin-ir-a-l-a povrće.\\ 
M {\AUX} marinate-\textsc{suff-tv}-{\PST-\FEM} vegetables\\ \jambox*{(BCMS)}
\glt `Marija was marinating the vegetables.'
 \label{str:ex:tanc2-d}

\ex \gll Marija je gril-nu-$∅$-l-a povrće.\\ 
M {\AUX} grill-\textsc{suff-tv}-{\PST-\FEM} vegetables\\ \jambox*{(BCMS)}
\glt `Marija grilled the vegetables a little bit.'
 \label{str:ex:tanc2-e}

\ex \gll Marija je ob(-)r-ni-$∅$-l-a kos zelenjave.\\
M {\AUX} (\textsc{pref})-turn-\textsc{suff-tv}-{\PST-\FEM} piece vegetable\\ \jambox*{(Slovenian)}
\glt `Marija turned a piece of vegetables.'
 \label{str:ex:tanc2-f}
\z
\z

\noindent In the following sections, we argue that the special status of \textit{-nV} among other verbal suffixes stems from its diminutive nature.

\subsection{Diminution in verbs and nouns, similarities} \label{str:subsect:Diminution in verbs and nouns}

Diminution is a cross-categorial phenomenon: nouns, verbs and adjectives all undergo this operation, in quite parallel ways. Consider the two structural positions for the diminutive suffix illustrated below for nouns \REF{str:ex:dim-noun}, adjectives \REF{str:ex:dim-adj} and verbs \REF{str:ex:dim-verb}, respectively. 

    %  \ea\label{str:ex:diminutive1-old}
    % 	\ea \label{str:ex:dim-noun-old} \gll \parbox{1cm}{lav} \parbox{1.75cm}{lav-ić} \parbox{2.5cm}{lav-č-e} lav-č-ić\\ 
    % lion lion-\textsc{dim} lion-\textsc{dim-infl} lion-\textsc{dim-dim}\\ \jambox*{(BCMS)}
    % \glt \parbox{1cm}{`lion'} \parbox{1.75cm}{`little lion'} \parbox{2.5cm}{`little lion'} `little{ }lion'
    
    % \ex \label{str:ex:dim-adj-old} \gll \parbox{1.25cm}{smeđ-e} \parbox{3.25cm}{smeđ-ast-o} \parbox{3.25cm}{smeđ-(i)k-av-o} smeđ-(i)k-ast-o\\ 
    % brown brown-\textsc{dim.adj-infl} brown-\textsc{dim-adj-infl} brown-\textsc{dim-dim.adj-infl}\\
    % \glt \parbox{1.25cm}{`brown'} \parbox{3.25cm}{`somewhat brown'} \parbox{3.25cm}{`somewhat brown'} \parbox{3.25cm}{`somewhat brown'}
    
    % \ex \label{str:ex:dim-verb-old} \gll \parbox{2.5cm}{greb-a-ti} \parbox{3cm}{greb-k-a-ti} \parbox{3cm}{greb-uc-a-ti} \parbox{3cm}{greb-uc-k-a-ti}\\ 
    % scratch-\textsc{tv}-{\INF} scratch-\textsc{dim-tv}-{\INF} scratch-\textsc{dim-tv}-{\INF} scratch-\textsc{dim-dim-tv}-{\INF}\\
    % \glt \parbox{2.5cm}{`scratch'} \parbox{3cm}{`scratch a little'} \parbox{3cm}{`scratch a little'} \parbox{3cm}{`scratch a little'}\label{str:ex:noun-c-old}
    
    % \z
    % \z


\ea\label{str:ex:diminutive1}
  \ea \label{str:ex:dim-noun} 
    \ea \gll {lav} \\ 
    lion \\ \jambox*{(BCMS)}
    \glt {`lion'} 
    \ex \gll lav-ić lav-č-e lav-č-ić\\ 
     lion-\textsc{dim} lion-\textsc{dim-infl} lion-\textsc{dim-dim}\\ 
    \glt `little lion'
    \z

  \ex \label{str:ex:dim-adj}
    \ea\gll smeđ-e\\ 
    brown \\
    \glt `brown'
    \ex\gll smeđ-ast-o smeđ-(i)k-av-o smeđ-(i)k-ast-o\\ 
    brown-\textsc{dim.adj-infl} brown-\textsc{dim-adj-infl} brown-\textsc{dim-dim.adj-infl}\\
    \glt `somewhat brown'
    \z

  \ex \label{str:ex:dim-verb} 
    \ea\gll greb-a-ti\\ 
    scratch-\textsc{tv}-{\INF}\\
    \glt `scratch'
    \ex\gll greb-k-a-ti greb-uc-a-ti greb-uc-k-a-ti\\ 
    scratch-\textsc{dim-tv}-{\INF} scratch-\textsc{dim-tv}-{\INF} scratch-\textsc{dim-dim-tv}-{\INF}\\
    \glt `scratch a little'
    \z
  \z
\z

\noindent The illustrated patterns perfectly fit \citeposst{DeBelderEtAl2014} analysis of diminution, where diminutive suffixes may be base-generated at the level of the root or at the level of the category. This is schematically represented in Figures \ref{str:fig:1}--\ref{str:fig:3}, where the maximal structure is given for each of the three categories for the examples in \REF{str:ex:diminutive1}. In all three examples, the higher diminutive is fused with the category, i.e. the diminutive suffix in this position realizes both the diminutive and the category, and can be substituted by a suffix realizing only the category. The lower diminutive, by contrast, is merged directly with the root, before the entire (extended root) structure is categorized. Diminution can be realized by either of the two options, or by a combination, without a (necessary) effect of accumulation.

      \begin{figure}%[H] 
      \caption{Syntactic representation of (double) diminutive nouns}
	     \begin{forest}
[\textit{n}P [-$∅$: ${[n]}$ $/$ -ć${: [n] [dim]}$ ]
[\textit{n}$'$
[{\textit{-č ${: [dim]}$}}]
[$\sqrt{\textsc{lav}}:$ `lion']]]
         \end{forest}\label{str:fig:1}
         \end{figure} 	   	

      \begin{figure}%[H]
      \caption{Syntactic representation of (double) diminutive adjectives}
	     \begin{forest}
[\textit{a}P [-av${: [a]}$ $/$ -ast${: [a] [dim]}$ ]
[\textit{a}$'$
[{\textit{-k ${: [dim]}$}}]
[$\sqrt{\textsc{smeđ}}:$ `brown']]]
         \end{forest}\label{str:fig:2}
         \end{figure} 	   	
         
      \begin{figure}
      \caption{Syntactic representation of (double) diminutive verbs}
	     \begin{forest}
[\textit{v}P [-${∅: [v]}$ $/$ -k${: [v] [dim]}$ ]
[\textit{v}$'$
[{\textit{-uc ${: [dim]}$}}]
[$\sqrt{\textsc{greb}}:$ `scratch']]]
         \end{forest}\label{str:fig:3}
         \end{figure} 	   	


The suffix \textit{-nV} is one of the suffixes used for diminution in the verbal domain. Apart from about a dozen exceptions, mostly degree achievements, as in \REF{str:ex:types-a}, all \textit{-nV} verbs involve the component of a small quantity, as in \REF{str:ex:types-b}.

\ea \label{str:ex:types-a}

\gll \parbox{2.75cm}{to-nu-ti} \parbox{2.75cm}{tru-nu-ti} \parbox{3cm}{bri-nu-ti} sva-nu-ti \\
$\sqrt{\textsc{sink}}$-nV-{\INF} 
$\sqrt{\textsc{rot}}$-nV-{\INF} 
$\sqrt{\textsc{worry}}$-nV-{\INF} 
$\sqrt{\textsc{dawn}}$-nV-{\INF} \\
\glt \parbox{2.75cm}{`sink’} \parbox{2.75cm}{`rot’} \parbox{3cm}{`worry’} `dawn’

\ex

\gll \parbox{3cm}{greb-nu-ti} \parbox{2.75cm}{spav-nu-ti} \parbox{2.75cm}{skok-nu-ti} kuc-nu-ti\\ 
$\sqrt{\textsc{scratch}}$-nV-{\INF} $\sqrt{\textsc{sleep}}$-nV-{\INF} $\sqrt{\textsc{jump}}$-nV-{\INF} $\sqrt{\textsc{knock}}$-nV-{\INF}
\\
\glt \parbox{3cm}{`scratch a little’} \parbox{2.75cm}{`sleep a little’} \parbox{2.75cm}{`jump a little’} `knock a little’\label{str:ex:types-b} \\
\z

\noindent The suffix \textit{-nV} with the diminutive interpretation normally can be combined with the root-level verbal diminutive suffix \textit{-uc} in BCMS. When this is degraded, there typically is an independent reason, such as with the verb \textit{kucnuti} in \REF{str:ex:doub}, where either the stem already involves the suffix \textit{-uc} (so it is actually impossible to have \textit{-nV} without \textit{-uc}), or some process akin to haplology is at play. With the addition of \textit{-uc}, the meaning is not affected, although sometimes the diminutive semantics feels somewhat stronger (which may be a pragmatic effect).


% \ea\label{str:ex:doub-old}
%   \gll greb-uc-nu-ti spav-uc-nu-ti prd-uc-nu-ti \textsuperscript{??}kuc-uc-nu-ti \\
%   $\sqrt{\textsc{scratch}}$-\textsc{dim}-nV-{\INF} $\sqrt{\textsc{sleep}}$-\textsc{dim}-nV-{\INF} $\sqrt{\textsc{fart}}$-\textsc{dim}-nV-{\INF} \phantom{\textsuperscript{??}}$\sqrt{\textsc{knock}}$-\textsc{dim}-nV-{\INF}\\ \jambox*{(BCMS)} 
%   \glt `scratch a little' `sleep a little' `fart a little' `knock a little'
% \z

% \ea\label{str:ex:doub-suggestion}
%   \ea\label{str:ex:doub-scratch}
%   \gll greb-uc-nu-ti\\
%   $\sqrt{\textsc{scratch}}$-\textsc{dim}-nV-{\INF} \\ \jambox*{(BCMS)} 
%   \glt `scratch a little'
%   \ex\label{str:ex:doub-sleep}
%   \gll spav-uc-nu-ti \\ 
%   $\sqrt{\textsc{sleep}}$-\textsc{dim}-nV-{\INF}  \\ 
%   \glt `sleep a little' 
%   \ex\label{str:ex:doub-fart}
%   \gll prd-uc-nu-ti \\
%   $\sqrt{\textsc{fart}}$-\textsc{dim}-nV-{\INF} \\
%   \glt `fart a little' 
%   \ex[\textsuperscript{??}]{\gll kuc-uc-nu-ti \\
%   $\sqrt{\textsc{knock}}$-\textsc{dim}-nV-{\INF}\\ 
%   \glt `knock a little'}\label{str:ex:doub-degraded}
%   \z
% \z


\ea\label{str:ex:doub}
  \gll \parbox{4cm}{greb-uc-nu-ti} spav-uc-nu-ti\\
  $\sqrt{\textsc{scratch}}$-\textsc{dim}-nV-{\INF} $\sqrt{\textsc{sleep}}$-\textsc{dim}-nV-{\INF}\\ \jambox*{(BCMS)} 
  \glt \parbox{4cm}{`scratch a little'} `sleep a little'\smallskip
   
  \gll \parbox{4cm}{prd-uc-nu-ti} \minsp{\textsuperscript{??}} kuc-uc-nu-ti\\
  $\sqrt{\textsc{fart}}$-\textsc{dim}-nV-{\INF} 
  {} 
  $\sqrt{\textsc{knock}}$-\textsc{dim}-nV-{\INF}\\
  \glt \parbox{4cm}{`fart a little'} `knock a little'

\z

\noindent All this points in the direction of having \textit{-nV} as a suffix combining the verbal category with the diminutive component in the category head.

Unlike in BCMS, in Slovenian, the suffix \textit{-nV} does not combine with other diminutive suffixes productively. Judging by the dictionary and corpus data, there is only one verb combining the diminutive suffix \textit{-ic} and \textit{-nV} in Slovenian.

\ea\label{str:ex:doub1}

\gll stop-i-ti	stop-ic-a-ti	stop-ic-ni-$∅$-ti \\
$\sqrt{\textsc{step}}$-\textsc{tv}-{\INF} $\sqrt{\textsc{step}}$-\textsc{dim}-\textsc{tv}-{\INF} $\sqrt{\textsc{step}}$-\textsc{dim}-nV-\textsc{tv}-{\INF}\\\jambox*{(Slovenian)} 
\glt `make a step' `make little steps$/$make steps a little' `make one little step'

\z

\noindent However, verb diminution is common in child-directed speech. The examples in (a) in \xxref{str:ex:chld}{str:ex:chld2} below show diminutive verbs derived from simplex verbs with different diminutive suffixes. The examples in (b) show the grammatical combinations of diminutive suffixes in Slovenian verbs and the examples in (c) show the ungrammatical ones. Just like the suffix \textit{-uc} in BCMS, the diminutive suffixes that combine with \textit{-nV} in Slovenian (i.e. \textit{-k} and \textit{-ic}) are instances of lower diminutives and are merged with the root, i.e. before the categorizing head exponed by a theme vowel.

\ea\label{str:ex:chld}
\ea \gll čič-a-ti	čič-k-a-ti	čič-ni-$∅$-ti (se)\\
$\sqrt{\textsc{sit}}$-\textsc{tv}-{\INF} $\sqrt{\textsc{sit}}$-\textsc{dim}-\textsc{tv}-{\INF} $\sqrt{\textsc{sit}}$-nV-\textsc{tv}-{\INF} {\REFL}\\ \jambox*{(Slovenian)}
\glt `sit' `sit in a small way' `sit down' \label{str:ex:chld-a}

\ex \gll čič-k-ni-$∅$-ti (se)\\
$\sqrt{\textsc{sit}}$-\textsc{dim}-nV-\textsc{tv}-{\INF} {\REFL}\\
\glt `sit in a small way' \label{str:ex:chld-b}

\ex[*] {\gll čič-n(i)-k-a-ti (se)\\
$\sqrt{\textsc{sit}}$-nV-\textsc{dim}-\textsc{tv}-{\INF} {\REFL}\\\label{str:ex:chld-c}}

\z

\ex\label{str:ex:chld1}
\ea \gll cap-a-ti	cap-k-a-ti	cap-lj-a-ti\\
$\sqrt{\textsc{drip}}$-\textsc{tv}-{\INF} $\sqrt{\textsc{drip}}$-\textsc{dim-tv}-{\INF} $\sqrt{\textsc{drip}}$-\textsc{dim-tv}-{\INF}\\ \jambox*{(Slovenian)}
\glt `take steps' `take little steps/step a little' `take little steps/step a little' \label{str:ex:chld1-a}\\

\ex \gll cap-k-lj-a-ti\\
$\sqrt{\textsc{drip}}$-\textsc{dim-dim}-\textsc{tv}-{\INF}\\
\glt `take little steps/step a little' \label{str:ex:chld1-b}

\ex[*] {\gll cap-lj-k-a-ti\\
$\sqrt{\textsc{drip}}$-\textsc{dim-dim-tv}-{\INF}\\\label{str:ex:chld1-c}}

\z

\ex\label{str:ex:chld2}
\ea \gll hop-a-ti\\
$\sqrt{\textsc{hop}}$-\textsc{tv}-{\INF} \\ \jambox*{(Slovenian)}
\glt `hop'\\

\gll hop-k-a-ti hop-lj-a-ti hop-ni-$∅$-ti\\
$\sqrt{\textsc{hop}}$-\textsc{dim}-\textsc{tv}-{\INF} $\sqrt{\textsc{hop}}$-\textsc{dim}-\textsc{tv}-{\INF} $\sqrt{\textsc{hop}}$-nV-\textsc{tv}-{\INF}\\
\glt `take little hops/hop a little' `take little hops/hop a little' `hop once' \label{str:ex:chld2-a}

\ex \gll hop-k-lj-a-ti hop-k-ni-$∅$-ti\\
$\sqrt{\textsc{hop}}$-\textsc{dim}-\textsc{dim}-\textsc{tv}-{\INF} $\sqrt{\textsc{hop}}$-\textsc{dim}-nV-\textsc{tv}-{\INF}\\
\glt `take little hops/hop a little' `take little hops/hop a little' \label{str:ex:chld2-b}

\ex[*] {\gll hop-lj-ni-$∅$-ti \minsp{*} hop-n(i)-lj-(a)-ti\\
$\sqrt{\textsc{hop}}$-nV-\textsc{dim}-\textsc{tv}-{\INF} {} $\sqrt{\textsc{hop}}$-nV-\textsc{dim}-\textsc{tv}-{\INF}\\ \label{str:ex:chld2-c}}

\z
\z


\noindent We take the similarity of the position in the words between the diminutive suffix \textit{-lj} and the suffix \textit{-nV} (i.e. the fact that they both can precede another verbal suffix or follow another diminutive suffix) and their complementary distribution in Slovenian as additional evidence for \textit{-nV} combining a diminutive and verbal component in the category head.

\subsection{Syntactic modeling} \label{str:subsect:Syntactic modelling}

We can now lay out our full structural analysis of the sequence \textit{-nV/-ne}. It is decomposed into two morphemes whose insertion is triggered by two features standing in the head--adjunct configuration: the diminutive feature and the verbal category feature. This is illustrated in \REF{str:ex:tree} and the respective structures in Figures \ref{str:fig:4}--\ref{str:fig:5} on two BCMS verbs, one without and another with the additional diminutive suffix \textit{-uc}.

\ea\label{str:ex:tree} 
    \gll\parbox{3.25cm}{zev-nu-$∅$-ti} zev-uc-nu-$∅$-ti\\
    $\sqrt{\textsc{yawn}}$-nV-\textsc{tv}-{\INF} 
    $\sqrt{\textsc{yawn}}$-\textsc{dim}-nV-\textsc{tv}-{\INF}\\ \jambox*{(BCMS)}
    \glt \parbox{3.25cm}{`yawn a little'} `yawn a little'
\z

\begin{figure}
\caption{Syntactic representation of the verb \textit{zevnuti} in \REF{str:ex:tree}}
    \begin{forest}for tree= fairly nice empty nodes
    [\textit{v}P [ [-nV${: [dim]}$, tier=word] [-${∅: [v]}$, tier=word] ]
    [$\sqrt{\textsc{zev}}$$:$ `yawn', tier=word
]]
    \end{forest}\label{str:fig:4}
\end{figure}

\begin{figure}
\caption{Syntactic representation of the verb \textit{zevucnuti} in \REF{str:ex:tree}}
    \begin{forest}for tree= fairly nice empty nodes
    [\textit{v}P [ [-nV${: [dim]}$] [-${∅: [v]}$] ]
[[-uc${: [dim]}$] [$\sqrt{\textsc{zev}}$$:$ `yawn']
]]
    \end{forest}\label{str:fig:5}
\end{figure} 	   	
		
Subsequent head movement derives the surface order. 

\subsection{Formal semantic description} \label{str:subsect:Semantic formal description}

In line with \citet{Pietroski2005} and \citet{ArsenijevicHinzen2012}, we take all syntactic heads to denote predicates and to mutually combine strictly in terms of predicate modification. We follow \citet{Arsenijevic2017, sta+:Arsenijevic2022} in taking the semantic content of the category feature to be a restriction of the referential domain in terms of the semantic ontological class and unit of counting. The head \textit{v} restricts reference to eventualities, and optionally specifies the quantity structure of the referent of the eventually derived expression at the level of grammatical aspect in terms of neat units, in the sense of \citet{Landman2011}, assuming that the absence of this specification, i.e. the default interpretation, matches the messy quantity structure of the eventually derived description. Formally, hence, it is ambiguous between \REF{str:ex:sem-a} and \REF{str:ex:sem-b}.

\ea\label{str:ex:sem}
\ea $\lambda x. \cnst{event}(x)$\label{str:ex:sem-a}

\ex $\lambda x.\cnst{event}(x) \wedge \cnst{neat}(x)$\label{str:ex:sem-b} 

\z
\z

\noindent In both cases, the category feature is a predicate over entities (x), such that the eventually generated expression refers in terms of units \textit{x}, which are optionally \textit{x} neat. For instance, a verb like \textit{sleep} in its typical use (\ref{ex:slept}) involves a messy quantity structure as in \REF{str:ex:sem-a}, where units are not strictly bounded and two units may share parts or be part of one another. By contrast, for a verb like the  typical use of \textit{blink} (\ref{ex:blinked}), the quantity structure of the predicate is neat, as in \REF{str:ex:sem-b}, where units are strictly bounded and disjoint.

\ea\label{str:ex:neatness}
\ea John slept. \label{ex:slept}

\ex Mary blinked. \label{ex:blinked}

\z
\z

\noindent We analyze the diminutive feature as a specification of a low degree on some measure function, as in \REF{str:ex:sem1-a-new} (where $\cnst{m}(x)$ stands for the measure function applied to $x$). This measure function as well as the standard degree are both provided from the context. In the domain of concrete individuals, the measure function typically targets size, and in the domain of events their temporal duration. In the verbal structure, the diminutive feature may occur in two positions. One is to merge with the base from which the verb derives, typically a root or a complex structure, and apply diminution to it. This typically results in the choice of the measure of intensity of action or of the fit of the description (raising the interpretation of atypical nature of the eventuality with respect to the description used). This is structurally illustrated in \REF{str:ex:sem1-b-new}. 

The other option is that it merges with the category head, typically receiving the measure of duration interpretation, i.e. the unit event has a shorter (temporal or other) interval than the standard for the event kind, as in \REF{str:ex:sem1-c-new}.\footnote{Here we assume that the category head has the nature of a count classifier: it specifies the manner of reference, by specifying reference units (see \citealt{sta+:Arsenijevic2022} for an elaboration and further references). We follow \citet{Milosavljevic2023PhD} in assuming that the verbal structure includes further projections dedicated to atomicity and grammatical number, where the units specified by the category head are further specified and structured to restrict the description and eventually reference too, quite parallel to the way this is traditionally modeled in the nominal domain.} As the relation \textsc{smaller} entails boundedness, this imposes, by presupposition, restriction to neat predicates. As a result, the suffix \textit{-nV} combines with neat \textit{v}’s only, i.e. it accommodates neat quantity structure in the category head. When the diminutive feature adjoins to the category head, it is hence interpreted as specifying the bounded nature and small size of the unit eventuality. This is how for instance \textit{trk-nu-ti} `${\sqrt{run}}$-\textsc{tv}-{\INF}' gets the interpretation of a small (i.e. atomic) instance of running.\footnote{An anonymous reviewer raises the question of whether the neatness condition as part of the semantics of the suffix \textit{-nV} is justified, given that this suffix can combine with non-verbal bases, i.e. may be added to stems that denote uncountable nouns or onomatopoeic words (in Polish). While in Western South Slavic too the suffix \textit{-nV} combines with bases that are attested also as nominal (e.g. \textit{korak-nu-ti} `step'; with the noun \textit{korak} `step' and the Slovenian verb \textit{nasmeh-ni-ti} `smile' with the noun \textit{nasmeh} `smile'), or onomatopoeic (e.g. \textit{tres-nu-ti} `snap, crack' in BCMS and \textit{tresk-ni-ti} in Slovenian, with the respective words \textit{tres!} and \textit{tresk!} also used as interjections expressing a sudden or sharp sound, like the sound of something breaking or snapping), such examples do not constitute a counterargument for our analysis. Namely, in our DM implementation, the suffix \textit{-nV} merges with the category head or with a categorized root. This means that apparent onomatopoeic or nominal bases are verbalized before \textit{-nV} enters the structure, so that these are not counterexamples to the verbal and/or neatness presupposition. More generally, there are two possibilities for ``nominal'' bases: either the root is nominalized by a nominal head, and then verbalized, or the same root appears in both nominal and verbal structures. In both cases, \textit{-nV} would attach to the verbal category head (i.e. verbalized structure). 
}

\newpage
%ORIGINAL VERSION

% \ea\label{str:ex:sem1}

% \ea $[dim] :=  \lambda x [m(x) < std]$ \label{str:ex:sem1-a}

% \ex $\sqrt{\textsc{trk}}:= \lambda x$  [\sib{$\sqrt{\textsc{trk}}$}(x)] \label{str:ex:sem1-b}\\
% by predicate modification:\\
% ${[[dim]}{\sqrt{\textsc{trk}}}{]} := \lambda x $[\sib{$\sqrt{\textsc{trk}}$}(x) $∧ m(x) < std]$

% \ex $[v] := \lambda x [event(x) ∧ neat(x)]$ \label{str:ex:sem1-b}

% by predicate modification:

% ${[[dim][v]]} := \lambda x [event(x) ∧ neat(x) ∧ m(x) < std]$ \label{str:ex:sem1-c}

% \ex by predicate modification with the root: 

% ${[[[dim][v]]\sqrt{\textsc{trk}}]} := \lambda x [event(x) ∧ neat(x) ∧ m(x) < std ∧ $ \sib{$\sqrt{\textsc{trk}}$}(x)] \label{str:ex:sem1-d}

% \z
% \z

\ea\label{str:ex:sem1}

\ea $[dim] :=  \lambda x [\cnst{m}(x) < \cnst{std}]$  \label{str:ex:sem1-a-new}

\ex $\sqrt{\textsc{trk}}:= \lambda x  [$\sib{$\sqrt{\textsc{trk}}$}$(x)]$ \label{str:ex:sem1-b-new}\\
by predicate modification:\\
$[[dim]\sqrt{\textsc{trk}}] := \lambda x [$\sib{$\sqrt{\textsc{trk}}$}$(x) \wedge \cnst{m}(x) < \cnst{std}]$

\ex $[v] := \lambda x [\cnst{event}(x) \wedge \cnst{neat}(x)]$ \label{str:ex:sem1-c-new}

by predicate modification:

${[[dim][v]]} := \lambda x [\cnst{event}(x) \wedge \cnst{neat}(x) \wedge \cnst{m}(x) < \cnst{std}]$ \label{str:ex:sem1-d-new}

\ex by predicate modification with the root: 

${[[[dim][v]]\sqrt{\textsc{trk}}]} := \lambda x [\cnst{event}(x) \wedge \cnst{neat}(x) \wedge \cnst{m}(x) < \cnst{std}\, \wedge$ \sib{$\sqrt{\textsc{trk}}$}$(x)]$ \label{str:ex:sem1-e-new}

\z
\z

\noindent Considering that the suffix \textit{-nV} realizes the diminutive adjoined to the category head and the suffix \textit{-uc} the one composed with the root or other base, this analysis predicts that the suffix \textit{-uc} will be ambiguous, while the suffix \textit{-nV} will not be used with the meaning of low intensity without restriction to neat structure. Indeed, the latter is exactly what is discussed around example \REF{str:ex:tree}, while, as shown in \REF{str:ex:kraj}, \textit{-uc} may also have the pure low intensity interpretation, as all the verbs in \REF{str:ex:kraj} are ambiguous between the durative low intensity interpretation and that of an iteration of pointy intervals of the (low intensity or not) eventuality.

\ea\label{str:ex:kraj}
 
\gll \parbox{3.5cm}{svetl-uc-a-ti}	bel-uc-a-ti\\
$\sqrt{\textsc{light}}$-\textsc{dim-tv}-{\INF} $\sqrt{\textsc{white}}$-\textsc{dim-tv}-{\INF}\\ \jambox*{(BCMS)}
\glt \parbox{3.5cm}{`emit light a little'} `be white a little'

\gll \parbox{3.5cm}{svir-uc-a-ti} šet-uc-a-ti\\
 $\sqrt{\textsc{play}}$-\textsc{dim-tv}-{\INF} $\sqrt{\textsc{walk}}$-\textsc{dim-tv}-{\INF}\\ 
\glt \parbox{3.5cm}{`play a little'} `walk a little' 

\z

%\ea\label{str:ex:kraj}
 
%\gll svetl-uc-a-ti	bel-uc-a-ti	svir-uc-a-ti šet-uc-a-ti\\
%$\sqrt{\textsc{light}}$-\textsc{dim-tv}-{\INF} $\sqrt{\textsc{white}}$-\textsc{dim-tv}-{\INF} $\sqrt{\textsc{play}}$-\textsc{dim-tv}-{\INF} $\sqrt{\textsc{walk}}$-\textsc{dim-tv}-{\INF}\\ \jambox*{(BCMS)}
%\glt `emit light a little' `be white a little' `play a little' `walk a little'

%\z


\subsection{Western South Slavic verbal suffixation} \label{str:subsect:How_does_nV_analyzed_in_this_way_fit_the_broader_picture}

The proposed analysis postulates three syntactic positions in which verbal suffixes are generated in Western South Slavic (and possibly more generally Slavic). These are, bottom up: (i) a position merging with the base, be it a root or a category, in which ambiguous diminutive suffixes are generated (suffixes \textit{-uc}, \textit{-uš} in BCMS, \textit{-ic}, \textit{-k} in Slovenian), (ii) adjunct to the category head, also reserved for the diminutive suffix, but here realized as \textit{-nV}, and (iii) the position of the imperfective (or biaspectual) verbal suffixes, traditionally associated with some aspectual projection. The last type of suffixes has been analyzed in \citet{SimonovicEtAl2021} and \citet{ArsenijevicEtAl2023} as consisting purely of theme vowels, and thus realizing the bare verbal category feature. This reduces the set of possible positions to only two: that below the verbal category head and the verbal category head itself.\footnote{Due to space limitations, we leave aside the status of suffixes that are used for integrating borrowed verbs, such as \textit{-ir} (\textit{kop-ir-a-ti} [copy-ir-\textsc{tv}-{\INF}] `copy') and \textit{-is} (\textit{determin-is-a-ti} [determine-is-\textsc{tv}-{\INF}] `determine') in BCMS.}
% S: U ovoj priči fale jedino sufiksi kojima se adaptiraju strani glagoli: -is, -ir.
% E: Stefan, bi dali to v footnote? Out of the scope of this article? Jaz bi dala.
% S: Dodata fusnota.

\subsection{Comparison to previous analyses} \label{str:subsect:Advantages_of_our_analysis}

Our analysis shares some properties with several others. Like \citet{sta+:Svenonius2004} and \citet{sta+:Biskup2023, Biskup2023Matryoshka,sta+:Biskup2020}, it relates the suffix with the verbal category. As \citet{Kwapiszewski2020}, our analysis attributes to the suffix specification of properties of quantity (the unit of counting), and as \citet{Arsenijevic2006}, it associates it with diminutivity. Finally, in line with \citet{Armoskaite2008}, we associate the suffix with the unit of counting, and with \citet{sta+:Lazorczyk2010,TaraldsenMedovaWiland2019}, and \citet{Wiland2019}, we offer a bimorphemic analysis. Here is how our analysis accounts for the specific properties of the suffix presented above.

In terms of meaning, \textsc{Semelfactives} present the fully compositional interpretation of the suffix \textit{-nV}: they denote one counting unit for the respective event predicate which is smaller than the standard for such an eventuality. \textsc{Natural Perfectives} are a special case, emerging when the event predicate specifies a salient atom. The salience of this interpretation imposes it as a pragmaticized meaning of the diminutive feature applying to the unit of counting specified by the event predicate. The \textsc{Perfective Delimitative} interpretation emerges when the event predicate specifies no salient counting unit. The diminutive feature presupposes such a unit, and by default takes bounded temporal intervals as the unit of counting. The salient natural class of bounded temporal intervals are points in time (no other length or type makes a natural class), resulting in semelfactivity. The \textsc{Degree Achievement} interpretation is not productive anymore, indicating that the suffix no longer contributes a meaning that derives it (see \citealt{Rothstein2008Puzzles} for an explanation of the source of \textit{-nV} degree achievements).
	
The diminutive semantic component, which is at least latently always present with \textit{-nV} (except in the unproductive class of degree achievements) is part of the meaning of the suffix. Telicity is part of the semantic specification of the meaning of the suffix, in the form of the presupposition of a unit of counting required by the meaning of smallness operating over the verbalizer which specifies properties of quantity. Perfectivity is generally strongly associated with telicity in Slavic \citep{Borer2005Structuring,Arsenijevic2006, Arsenijevic2023,sta+:Lazorczyk2010,sta+:Milosavljevic2022, Milosavljevic2023Advances, Milosavljevic2023PhD}, and the same mechanisms are likely at play with \textit{-nV}. Modeling this suffix as the only one with additional syntactic/semantic content next to that borne by the theme vowel \citep{SimonovicEtAl2021,ArsenijevicEtAl2023} enables capturing its being also the only one that imposes telicity and perfectivity.

By our analysis, \textit{-nV} selects the TV \textit{$∅$/e}, i.e. the \textit{-e} in the present stem is not part of the suffix but a TV. This fits the analysis where the diminutive feature realized as \textit{-nV} is left-adjoined to the verbal category feature realized as the TV. Our view obviates the question about the complementary distribution of \textit{-nV/-ne} with theme vowels, since the sequence \textit{-nV/-ne} includes a TV.

The compatibility of the suffix \textit{-nV} with secondary imperfectivizing suffixes in at least some Slavic languages (BCMS included), as well as the ability to stack with other imperfective suffixes, is not a problem for our approach since the suffix does not target the AspP, but a lower head (i.e. in the analysis by \citealt{ArsenijevicEtAl2023}, \textit{-nV} derives telic predicates, which then can be reverbalized).

Finally, unlike other analyses, ours also predicts that the suffix \textit{-nV} combines with the root-level diminutive suffix \textit{-uc} analogous to double diminution in nouns and adjectives.

\section{Conclusion} \label{str:sect:Conclusion}

The paper revisits the Slavic verbal suffix \textit{-nV}, and highlights a range of new qualitative and quantitative observations and generalizations which have not yet been reported or supported by precise quantitative data in previous descriptive and theoretical accounts of this suffix. We observe a unique status of the suffix among verbal suffixes based on the properties of its use (e.g., it may combine with other verbal suffixes, which does not hold for other suffixes; it does not select the theme vowel the other suffixes do). To predict and explain the special properties of the suffix, we propose the decomposition of the suffix into two components, an actual suffix (\textit{-nV}) and a theme vowel (\textit{$∅/e$}), realizing diminution and the verbal category, respectively. We provide a formalization for the diminutive semantics, and a syntactic structure for the position of its base-generation.
%: E: Malo preoblikovala zaključek.

\section*{Abbreviations}

\begin{tabularx}{.5\textwidth}{@{}lQ}
\textsc{1}  &first person   \\
\textsc{3}  &third person   \\
{\ACC}      &accusative   \\
\textsc{adj}&adjective  \\
{\AUX}      &auxiliary    \\
{\COMP}     &complementizer \\  
\textsc{dim}&diminutive \\
{\FEM}      &feminine   \\
{\IMP}      &imperative \\
{\INF}      &infinitive \\
\textsc{infl}   &inflectional ending    \\
{\MASC}     &masculine  \\
\end{tabularx}%
\begin{tabularx}{.5\textwidth}{lQ@{}}
{\NOM}      &nominative \\
{\PASS}     &passive    \\
{\PL}       &plural     \\
\textsc{pref}   &prefix \\
{\PRS}      &present tense\\
{\PST}      &past       \\
{\PTCP}     &participle \\
{\REFL}     &reflexive  \\
{\SG}       &singular   \\
\textsc{si} &secondary imperfective\\
\textsc{suff}&suffix    \\
\textsc{tv}&theme vowel \\
%&\\ % this dummy row achieves correct vertical alignment of both tables
\end{tabularx}

\section*{Acknowledgments}
%Place your acknowledgements here and funding information here.

This research was partly funded by the Austrian Science Fund FWF (grants I 4215 and I 6258), the Austrian Federal Ministry of Education, Science and Research (BMBWF; Scientific \& Technological Cooperation grant for the project \textit{Verbal suffixes and theme vowels: Different or the same?}), as well as the Slovenian Research and Innovation Agency (ARIS; grants P6-0382 and J6-4614).

\printbibliography[heading=subbibliography,notkeyword=this]

\end{document}
