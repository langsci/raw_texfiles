\documentclass[output=paper,colorlinks,citecolor=brown]{langscibook}
\ChapterDOI{10.5281/zenodo.15394164}
%\bibliography{localbibliography}

\author{Alessandro Bigolin\orcid{0000-0003-4209-4263}\affiliation{Centre de Lingüística Teòrica, Universitat Autònoma de Barcelona}}
% replace the above with you and your coauthors
% rules for affiliation: If there's an official English version, use that (find out on the official website of the university); if not, use the original
% orcid doesn't appear printed; it's metainformation used for later indexing

%%% uncomment the following line if you are a single author or all authors have the same affiliation
\SetupAffiliations{mark style=none}

%% in case the running head with authors exceeds one line (which is the case in this example document), use one of the following methods to turn it into a single line; otherwise comment the line below out with % and ignore it
%\lehead{Šimík, Gehrke, Lenertová, Meyer, Szucsich \& Zaleska}
% \lehead{Radek Šimík et al.}

\title{Slavic creation/consumption predicates in light of Talmy's typology}
% replace the above with your paper title
%%% provide a shorter version of your title in case it doesn't fit a single line in the running head
% in this form: \title[short title]{full title}
\abstract{The chapter is concerned with the licensing of creation/consumption predicates in Slavic languages, in light of \citeposst{Talmy2000} typology. I present the results of a pilot study suggesting that Slavic languages behave as verb-framed languages in the domain of creation/consumption predicates, despite these languages being commonly regarded as a type of satellite-framed languages (\citealt{Talmy2000}) referred to as ``weak satellite-framed languages'' (\citealt{Acedo-Matellan2010}; \citeyear{Acedo-Matellan2016}). Assuming a neo-constructionist view on argument structure, I propose a morphosyntactic account of \citeauthor{Talmy2000}'s typology, according to which the verb-framed vs. satellite-framed distinction depends on a specific Phonological Form requirement. In verb-framed languages, the null functional head involved in verbal predication must, by assumption, incorporate its complement as an externalization condition. I propose that so-called weak satellite-framed languages, to which Slavic languages have been argued to belong, are fundamentally verb-framed languages, and that the availability of satellite-framed resultative constructions in these languages is granted by the lexical presence of result morphemes that can incorporate into \textit{v} via prefixation.

\keywords{argument structure, creation/consumption predicates, Path, prefixation, Talmy's typology}
}

\begin{document}
\maketitle

% Just comment out the input below when you're ready to go.
%For a start: Do not forget to give your Overleaf project (this paper) a recognizable name. This one could be called, for instance, Simik et al: OSL template. You can change the name of the project by hovering over the gray title at the top of this page and clicking on the pencil icon.

\section{Introduction}\label{sim:sec:intro}

Language Science Press is a project run for linguists, but also by linguists. You are part of that and we rely on your collaboration to get at the desired result. Publishing with LangSci Press might mean a bit more work for the author (and for the volume editor), esp. for the less experienced ones, but it also gives you much more control of the process and it is rewarding to see the quality result.

Please follow the instructions below closely, it will save the volume editors, the series editors, and you alike a lot of time.

\sloppy This stylesheet is a further specification of three more general sources: (i) the Leipzig glossing rules \citep{leipzig-glossing-rules}, (ii) the generic style rules for linguistics (\url{https://www.eva.mpg.de/fileadmin/content_files/staff/haspelmt/pdf/GenericStyleRules.pdf}), and (iii) the Language Science Press guidelines \citep{Nordhoff.Muller2021}.\footnote{Notice the way in-text numbered lists should be written -- using small Roman numbers enclosed in brackets.} It is advisable to go through these before you start writing. Most of the general rules are not repeated here.\footnote{Do not worry about the colors of references and links. They are there to make the editorial process easier and will disappear prior to official publication.}

Please spend some time reading through these and the more general instructions. Your 30 minutes on this is likely to save you and us hours of additional work. Do not hesitate to contact the editors if you have any questions.

\section{Illustrating OSL commands and conventions}\label{sim:sec:osl-comm}

Below I illustrate the use of a number of commands defined in langsci-osl.tex (see the styles folder).

\subsection{Typesetting semantics}\label{sim:sec:sem}

See below for some examples of how to typeset semantic formulas. The examples also show the use of the sib-command (= ``semantic interpretation brackets''). Notice also the the use of the dummy curly brackets in \REF{sim:ex:quant}. They ensure that the spacing around the equation symbol is correct. 

\ea \ea \sib{dog}$^g=\textsc{dog}=\lambda x[\textsc{dog}(x)]$\label{sim:ex:dog}
\ex \sib{Some dog bit every boy}${}=\exists x[\textsc{dog}(x)\wedge\forall y[\textsc{boy}(y)\rightarrow \textsc{bit}(x,y)]]$\label{sim:ex:quant}
\z\z

\noindent Use noindent after example environments (but not after floats like tables or figures).

And here's a macro for semantic type brackets: The expression \textit{dog} is of type $\stb{e,t}$. Don't forget to place the whole type formula into a math-environment. An example of a more complex type, such as the one of \textit{every}: $\stb{s,\stb{\stb{e,t},\stb{e,t}}}$. You can of course also use the type in a subscript: dog$_{\stb{e,t}}$

We distinguish between metalinguistic constants that are translations of object language, which are typeset using small caps, see \REF{sim:ex:dog}, and logical constants. See the contrast in \REF{sim:ex:speaker}, where \textsc{speaker} (= serif) in \REF{sim:ex:speaker-a} is the denotation of the word \textit{speaker}, and \cnst{speaker} (= sans-serif) in \REF{sim:ex:speaker-b} is the function that maps the context $c$ to the speaker in that context.\footnote{Notice that both types of small caps are automatically turned into text-style, even if used in a math-environment. This enables you to use math throughout.}$^,$\footnote{Notice also that the notation entails the ``direct translation'' system from natural language to metalanguage, as entertained e.g. in \citet{Heim.Kratzer1998}. Feel free to devise your own notation when relying on the ``indirect translation'' system (see, e.g., \citealt{Coppock.Champollion2022}).}

\ea\label{sim:ex:speaker}
\ea \sib{The speaker is drunk}$^{g,c}=\textsc{drunk}\big(\iota x\,\textsc{speaker}(x)\big)$\label{sim:ex:speaker-a}
\ex \sib{I am drunk}$^{g,c}=\textsc{drunk}\big(\cnst{speaker}(c)\big)$\label{sim:ex:speaker-b}
\z\z

\noindent Notice that with more complex formulas, you can use bigger brackets indicating scope, cf. $($ vs. $\big($, as used in \REF{sim:ex:speaker}. Also notice the use of backslash plus comma, which produces additional space in math-environment.

\subsection{Examples and the minsp command}

Try to keep examples simple. But if you need to pack more information into an example or include more alternatives, you can resort to various brackets or slashes. For that, you will find the minsp-command useful. It works as follows:

\ea\label{sim:ex:german-verbs}\gll Hans \minsp{\{} schläft / schlief / \minsp{*} schlafen\}.\\
Hans {} sleeps {} slept {} {} sleep.\textsc{inf}\\
\glt `Hans \{sleeps / slept\}.'
\z

\noindent If you use the command, glosses will be aligned with the corresponding object language elements correctly. Notice also that brackets etc. do not receive their own gloss. Simply use closed curly brackets as the placeholder.

The minsp-command is not needed for grammaticality judgments used for the whole sentence. For that, use the native langsci-gb4e method instead, as illustrated below:

\ea[*]{\gll Das sein ungrammatisch.\\
that be.\textsc{inf} ungrammatical\\
\glt Intended: `This is ungrammatical.'\hfill (German)\label{sim:ex:ungram}}
\z

\noindent Also notice that translations should never be ungrammatical. If the original is ungrammatical, provide the intended interpretation in idiomatic English.

If you want to indicate the language and/or the source of the example, place this on the right margin of the translation line. Schematic information about relevant linguistic properties of the examples should be placed on the line of the example, as indicated below.

\ea\label{sim:ex:bailyn}\gll \minsp{[} Ėtu knigu] čitaet Ivan \minsp{(} často).\\
{} this book.{\ACC} read.{\PRS.3\SG} Ivan.{\NOM} {} often\\\hfill O-V-S-Adv
\glt `Ivan reads this book (often).'\hfill (Russian; \citealt[4]{Bailyn2004})
\z

\noindent Finally, notice that you can use the gloss macros for typing grammatical glosses, defined in langsci-lgr.sty. Place curly brackets around them.

\subsection{Citation commands and macros}

You can make your life easier if you use the following citation commands and macros (see code):

\begin{itemize}
    \item \citealt{Bailyn2004}: no brackets
    \item \citet{Bailyn2004}: year in brackets
    \item \citep{Bailyn2004}: everything in brackets
    \item \citepossalt{Bailyn2004}: possessive
    \item \citeposst{Bailyn2004}: possessive with year in brackets
\end{itemize}

\section{Trees}\label{s:tree}

Use the forest package for trees and place trees in a figure environment. \figref{sim:fig:CP} shows a simple example.\footnote{See \citet{VandenWyngaerd2017} for a simple and useful quickstart guide for the forest package.} Notice that figure (and table) environments are so-called floating environments. {\LaTeX} determines the position of the figure/table on the page, so it can appear elsewhere than where it appears in the code. This is not a bug, it is a property. Also for this reason, do not refer to figures/tables by using phrases like ``the table below''. Always use tabref/figref. If your terminal nodes represent object language, then these should essentially correspond to glosses, not to the original. For this reason, we recommend including an explicit example which corresponds to the tree, in this particular case \REF{sim:ex:czech-for-tree}.

\ea\label{sim:ex:czech-for-tree}\gll Co se řidič snažil dělat?\\
what {\REFL} driver try.{\PTCP.\SG.\MASC} do.{\INF}\\
\glt `What did the driver try to do?'
\z

\begin{figure}[ht]
% the [ht] option means that you prefer the placement of the figure HERE (=h) and if HERE is not possible, you prefer the TOP (=t) of a page
% \centering
    \begin{forest}
    for tree={s sep=1cm, inner sep=0, l=0}
    [CP
        [DP
            [what, roof, name=what]
        ]
        [C$'$
            [C
                [\textsc{refl}]
            ]
            [TP
                [DP
                    [driver, roof]
                ]
                [T$'$
                    [T [{[past]}]]
                    [VP
                        [V
                            [tried]
                        ]
                        [VP, s sep=2.2cm
                            [V
                                [do.\textsc{inf}]
                            ]
                            [t\textsubscript{what}, name=trace-what]
                        ]
                    ]
                ]
            ]
        ]
    ]
    \draw[->,overlay] (trace-what) to[out=south west, in=south, looseness=1.1] (what);
    % the overlay option avoids making the bounding box of the tree too large
    % the looseness option defines the looseness of the arrow (default = 1)
    \end{forest}
    \vspace{3ex} % extra vspace is added here because the arrow goes too deep to the caption; avoid such manual tweaking as much as possible; here it's necessary
    \caption{Proposed syntactic representation of \REF{sim:ex:czech-for-tree}}
    \label{sim:fig:CP}
\end{figure}

Do not use noindent after figures or tables (as you do after examples). Cases like these (where the noindent ends up missing) will be handled by the editors prior to publication.

\section{Italics, boldface, small caps, underlining, quotes}

See \citet{Nordhoff.Muller2021} for that. In short:

\begin{itemize}
    \item No boldface anywhere.
    \item No underlining anywhere (unless for very specific and well-defined technical notation; consult with editors).
    \item Small caps used for (i) introducing terms that are important for the paper (small-cap the term just ones, at a place where it is characterized/defined); (ii) metalinguistic translations of object-language expressions in semantic formulas (see \sectref{sim:sec:sem}); (iii) selected technical notions.
    \item Italics for object-language within text; exceptionally for emphasis/contrast.
    \item Single quotes: for translations/interpretations
    \item Double quotes: scare quotes; quotations of chunks of text.
\end{itemize}

\section{Cross-referencing}

Label examples, sections, tables, figures, possibly footnotes (by using the label macro). The name of the label is up to you, but it is good practice to follow this template: article-code:reference-type:unique-label. E.g. sim:ex:german would be a proper name for a reference within this paper (sim = short for the author(s); ex = example reference; german = unique name of that example).

\section{Syntactic notation}

Syntactic categories (N, D, V, etc.) are written with initial capital letters. This also holds for categories named with multiple letters, e.g. Foc, Top, Num, etc. Stick to this convention also when coming up with ad hoc categories, e.g. Cl (for clitic or classifier).

An exception from this rule are ``little'' categories, which are written with italics: \textit{v}, \textit{n}, \textit{v}P, etc.

Bar-levels must be typeset with bars/primes, not with an apostrophe. An easy way to do that is to use mathmode for the bar: C$'$, Foc$'$, etc.

Specifiers should be written this way: SpecCP, Spec\textit{v}P.

Features should be surrounded by square brackets, e.g., [past]. If you use plus and minus, be sure that these actually are plus and minus, and not e.g. a hyphen. Mathmode can help with that: [$+$sg], [$-$sg], [$\pm$sg]. See \sectref{sim:sec:hyphens-etc} for related information.

\section{Footnotes}

Absolutely avoid long footnotes. A footnote should not be longer than, say, {20\%} of the page. If you feel like you need a long footnote, make an explicit digression in the main body of the text.

Footnotes should always be placed at the end of whole sentences. Formulate the footnote in such a way that this is possible. Footnotes should always go after punctuation marks, never before. Do not place footnotes after individual words. Do not place footnotes in examples, tables, etc. If you have an urge to do that, place the footnote to the text that explains the example, table, etc.

Footnotes should always be formulated as full, self-standing sentences.

\section{Tables}

For your tables use the table plus tabularx environments. The tabularx environment lets you (and requires you in fact) to specify the width of the table and defines the X column (left-alignment) and the Y column (right-alignment). All X/Y columns will have the same width and together they will fill out the width of the rest of the table -- counting out all non-X/Y columns.

Always include a meaningful caption. The caption is designed to appear on top of the table, no matter where you place it in the code. Do not try to tweak with this. Tables are delimited with lsptoprule at the top and lspbottomrule at the bottom. The header is delimited from the rest with midrule. Vertical lines in tables are banned. An example is provided in \tabref{sim:tab:frequencies}. See \citet{Nordhoff.Muller2021} for more information. If you are typesetting a very complex table or your table is too large to fit the page, do not hesitate to ask the editors for help.

\begin{table}
\caption{Frequencies of word classes}
\label{sim:tab:frequencies}
 \begin{tabularx}{.77\textwidth}{lYYYY} %.77 indicates that the table will take up 77% of the textwidth
  \lsptoprule
            & nouns & verbs  & adjectives & adverbs\\
  \midrule
  absolute  &   12  &    34  &    23      & 13\\
  relative  &   3.1 &   8.9  &    5.7     & 3.2\\
  \lspbottomrule
 \end{tabularx}
\end{table}

\section{Figures}

Figures must have a good quality. If you use pictorial figures, consult the editors early on to see if the quality and format of your figure is sufficient. If you use simple barplots, you can use the barplot environment (defined in langsci-osl.sty). See \figref{sim:fig:barplot} for an example. The barplot environment has 5 arguments: 1. x-axis desription, 2. y-axis description, 3. width (relative to textwidth), 4. x-tick descriptions, 5. x-ticks plus y-values.

\begin{figure}
    \centering
    \barplot{Type of meal}{Times selected}{0.6}{Bread,Soup,Pizza}%
    {
    (Bread,61)
    (Soup,12)
    (Pizza,8)
    }
    \caption{A barplot example}
    \label{sim:fig:barplot}
\end{figure}

The barplot environment builds on the tikzpicture plus axis environments of the pgfplots package. It can be customized in various ways. \figref{sim:fig:complex-barplot} shows a more complex example.

\begin{figure}
  \begin{tikzpicture}
    \begin{axis}[
	xlabel={Level of \textsc{uniq/max}},  
	ylabel={Proportion of $\textsf{subj}\prec\textsf{pred}$}, 
	axis lines*=left, 
        width  = .6\textwidth,
	height = 5cm,
    	nodes near coords, 
    % 	nodes near coords style={text=black},
    	every node near coord/.append style={font=\tiny},
        nodes near coords align={vertical},
	ymin=0,
	ymax=1,
	ytick distance=.2,
	xtick=data,
	ylabel near ticks,
	x tick label style={font=\sffamily},
	ybar=5pt,
	legend pos=outer north east,
	enlarge x limits=0.3,
	symbolic x coords={+u/m, \textminus u/m},
	]
	\addplot[fill=red!30,draw=none] coordinates {
	    (+u/m,0.91)
        (\textminus u/m,0.84)
	};
	\addplot[fill=red,draw=none] coordinates {
	    (+u/m,0.80)
        (\textminus u/m,0.87)
	};
	\legend{\textsf{sg}, \textsf{pl}}
    \end{axis} 
  \end{tikzpicture} 
    \caption{Results divided by \textsc{number}}
    \label{sim:fig:complex-barplot}
\end{figure}

\section{Hyphens, dashes, minuses, math/logical operators}\label{sim:sec:hyphens-etc}

Be careful to distinguish between hyphens (-), dashes (--), and the minus sign ($-$). For in-text appositions, use only en-dashes -- as done here -- with spaces around. Do not use em-dashes (---). Using mathmode is a reliable way of getting the minus sign.

All equations (and typically also semantic formulas, see \sectref{sim:sec:sem}) should be typeset using mathmode. Notice that mathmode not only gets the math signs ``right'', but also has a dedicated spacing. For that reason, never write things like p$<$0.05, p $<$ 0.05, or p$<0.05$, but rather $p<0.05$. In case you need a two-place math or logical operator (like $\wedge$) but for some reason do not have one of the arguments represented overtly, you can use a ``dummy'' argument (curly brackets) to simulate the presence of the other one. Notice the difference between $\wedge p$ and ${}\wedge p$.

In case you need to use normal text within mathmode, use the text command. Here is an example: $\text{frequency}=.8$. This way, you get the math spacing right.

\section{Abbreviations}

The final abbreviations section should include all glosses. It should not include other ad hoc abbreviations (those should be defined upon first use) and also not standard abbreviations like NP, VP, etc.


\section{Bibliography}

Place your bibliography into localbibliography.bib. Important: Only place there the entries which you actually cite! You can make use of our OSL bibliography, which we keep clean and tidy and update it after the publication of each new volume. Contact the editors of your volume if you do not have the bib file yet. If you find the entry you need, just copy-paste it in your localbibliography.bib. The bibliography also shows many good examples of what a good bibliographic entry should look like.

See \citet{Nordhoff.Muller2021} for general information on bibliography. Some important things to keep in mind:

\begin{itemize}
    \item Journals should be cited as they are officially called (notice the difference between and, \&, capitalization, etc.).
    \item Journal publications should always include the volume number, the issue number (field ``number''), and DOI or stable URL (see below on that).
    \item Papers in collections or proceedings must include the editors of the volume (field ``editor''), the place of publication (field ``address'') and publisher.
    \item The proceedings number is part of the title of the proceedings. Do not place it into the ``volume'' field. The ``volume'' field with book/proceedings publications is reserved for the volume of that single book (e.g. NELS 40 proceedings might have vol. 1 and vol. 2).
    \item Avoid citing manuscripts as much as possible. If you need to cite them, try to provide a stable URL.
    \item Avoid citing presentations or talks. If you absolutely must cite them, be careful not to refer the reader to them by using ``see...''. The reader can't see them.
    \item If you cite a manuscript, presentation, or some other hard-to-define source, use the either the ``misc'' or ``unpublished'' entry type. The former is appropriate if the text cited corresponds to a book (the title will be printed in italics); the latter is appropriate if the text cited corresponds to an article or presentation (the title will be printed normally). Within both entries, use the ``howpublished'' field for any relevant information (such as ``Manuscript, University of \dots''). And the ``url'' field for the URL.
\end{itemize}

We require the authors to provide DOIs or URLs wherever possible, though not without limitations. The following rules apply:

\begin{itemize}
    \item If the publication has a DOI, use that. Use the ``doi'' field and write just the DOI, not the whole URL.
    \item If the publication has no DOI, but it has a stable URL (as e.g. JSTOR, but possibly also lingbuzz), use that. Place it in the ``url'' field, using the full address (https: etc.).
    \item Never use DOI and URL at the same time.
    \item If the official publication has no official DOI or stable URL (related to the official publication), do not replace these with other links. Do not refer to published works with lingbuzz links, for instance, as these typically lead to the unpublished (preprint) version. (There are exceptions where lingbuzz or semanticsarchive are the official publication venue, in which case these can of course be used.) Never use URLs leading to personal websites.
    \item If a paper has no DOI/URL, but the book does, do not use the book URL. Just use nothing.
\end{itemize}


\section{Introduction}\label{1}

This chapter is concerned with the licensing of creation/consumption predicates in Slavic languages, in light of \citeposst{Talmy2000} typology. In creation/consumption predicates, the direct object is understood as being ``created'' or ``consumed'' during the event denoted by the predicate (\citealt{HaleAndKeyser2002}; \citealt{Volpe2004}; \citealt{Harley2005}; \citealt{Mateu2012}, among others). For instance, \textit{a hole} in \REF{bla} is formed while the digging process takes place, and \textit{the apple} in \REF{bla2} is consumed during the eating process.

\ea \label{blaentrambi} \ea He dug a hole in his garden. \label{bla} \hfill (\citealt[46]{Washio1997})
\ex John ate the apple. \label{bla2} \hfill (\citealt[103]{FolliAndHarley2005})
\z \z

\noindent The typological distinction proposed by \citet{Talmy2000} divides languages into two broad classes, depending on how the Path (or ``change'') core component of resultative events of change of state and location is expressed.\footnote{I consider an event ``resultative'' if it involves a scalar change along a scale that denotes a property or a path (\citealt{RappaportHovav2014}).} In one class of languages, Path is typically encoded in a satellite (e.g., a particle, PP or AP) distinct from the main verb, which in turn may express a co-event.\footnote{PPs are explicitly excluded in \citeposst{Talmy2000} notion of ``satellite'', defined as ``[...] the grammatical category of any constituent other than a noun phrase or prepositional-phrase complement that is in a sister relation to the verb root'' (\citealt[120]{Talmy2000}). Following \citet{Mateu2002}, \citet{BeaversLevinandTham2010}, \citet{Acedo-MatellanandMateu2013}, \citet{Acedo-Matellan2016}, among others, I adopt a broader definition of satellite, which includes non-adjunct result PPs like the one in \REF{sfeninto}.} The co-event usually provides information about the manner in which the main %(`framing'; \citealt{Talmy2000}) 
resultative event unfolds, or about the cause which triggers it.\footnote{See \citet{Talmy2000} for an exhaustive classification of possible conceptual interpretations attributable to co-events.} In the other class of languages, Path is always encoded in the main verb, so that information about a co-event is either not expressed or provided via adjuncts. Languages of the former type are thus referred to as ``satellite-framed'', while languages of the latter type are referred to as ``verb-framed''. The examples from English (a satellite-framed language) and Spanish (a verb-framed language) in \REF{sfen} and \REF{vfsp} illustrate the two patterns, for events of change of location and events of change of state respectively.

\largerpage

\ea Satellite-framed pattern (English): \label{sfen} \ea \label{sfeninto} The bottle [floated]\textsubscript{\textsc{co-event}} [into the cave]\textsubscript{\textsc{path}}. \hfill (\citealt[227]{Talmy2000})
\ex She [shot]\textsubscript{\textsc{co-event}} him [dead]\textsubscript{\textsc{path}}. \hfill (\citealt[136]{Goldberg1995}) \z \z

\ea Verb-framed pattern (Spanish): \label{vfsp} \ea \gll La botella  [entró]\textsubscript{\textsc{path}} ([flotando]\textsubscript{\textsc{co-event}}) a la cueva.\\
the bottle enter.{\PST.\AGR} \hspace{3pt}float.\textsc{ger} to the cave \\
\glt `The bottle moved into the cave (floating).' \hfill (\citealt[227]{Talmy2000})
\ex \gll Lo [mató]\textsubscript{\textsc{path}} (\minsp{[} de un disparo]\textsubscript{\textsc{co-event}}).\\
\textsc{acc.m.sg} kill.{\PST.\AGR} {} of a shot\\\glt `He/she killed him with a shot.’ \hfill (CORPES XXI\footnote{\textit{Corpus del Español del Siglo XXI}, \citet{RAECorpes}.})
\z \z

\noindent Slavic languages, along with Latin, have been classified as ``weak satellite-framed'' (\citealt{Acedo-Matellan2010}; \citeyear{Acedo-Matellan2016}) since, although they allow the expression of Path in a satellite, this must form a prosodic word with the verb. For instance, the object \textit{svoju ručku} `her pen' in \REF{ispisala} is understood to be brought into a state where all its ink is used up by means of the prefixal satellite \textit{iz-} `out', while the verb \textit{pis-} `write' specifies the co-event that causes the transition undergone by the referent of the direct object (\citealt{Spencerandzaretskaya1998}; \citealt{Mateu2008}).

\ea \gll Ona [iz]\textsubscript{\textsc{path}}-[pis]\textsubscript{\textsc{co-event}}-a-l-a svoju ručku.\\
she.{\NOM} out-write-\textsc{th}-{\PST-\AGR} {\POSS} pen.{\ACC}\\
\glt `Her pen has run out of ink.' (Lit. `She has written her pen out (of ink).') \label{ispisala}
\hfill (Russian; \citealt[17]{Spencerandzaretskaya1998}) \z

\noindent The satellite-framed/verb-framed distinction is also found in the domain of predicates denoting events of creation/consumption (\citealt{Mateu2003}; \citeyear{Mateu2012}). In a similar way to \REF{sfen}, satellite-framed languages allow the expression of a co-event in the verb in creation/consumption predicates, giving rise to creation/consumption predicates of the type in \REF{bla5} (hereafter, ``complex creation/consumption predicates''). The predicate in \REF{bla5} can be paraphrased as ``make a hole in the coat by brushing'', whereby it is clear that the main verb of the predicate is understood as specifying a co-event of the main event of creation. Verb-framed languages instead consistently express the event that leads to the creation/consumption of the direct object by means of the main verb, which may be either a light verb (e.g., \textit{make}, as in the Spanish example in \REF{bla6}) or a verb whose meaning is likely to imply the creation/consumption of the object, which in turn is interpreted as a hyponym of the verb (as in \REF{blaentrambi}; see \citealt{HaleAndKeyser1997b}; \citeyear{HaleAndKeyser2002}). The specification of a possible co-event, as in the verb-framed change-of-location/state examples in \REF{vfsp}, is relegated to an optional adjunct.

\ea \label{mainbla} \ea \label{bla5} Brush a hole in one's coat. \hfill (\citealt[279]{LevinAndRapoport1988})
\ex \gll Hizo un agujero en su abrigo (al cepillar=lo).\\
make.{\PST.\AGR} a hole in \textsc{poss} coat \hspace{3pt}at.the brush.{\INF}=\textsc{acc.m.sg}\\\glt
`She made a hole in her coat, by brushing it.' \label{bla6}
\z \z

\noindent A non-trivial difference between creation/consumption predicates and change-of-location/state predicates is that the argument structure of creation/con\-sump\-tion predicates has been argued to lack a Path component (see \citealt{RappaportHovavAndLevin1998}; \citealt{RappaportHovav2008}; \citealt{RappaportHovavAndLevin2010} for works adopting a lexicalist approach; see \citealt{HaleAndKeyser1993}; \citeyear{HaleAndKeyser2002}; \citealt{Mateu2002}; \citealt{Harley2005}; \citealt{FolliAndHarley2005}; \citeyear{FolliAndHarley2008}; \citeyear{FolliAndHarley2020}; \citealt{big:Ramchand2008}; \citealt{Acedo-Matellan2016}, among others, for works adopting a neo-constructionist, syntactic approach). %\footnote{The absence of an explicit lexicalization of the Path component in creation predicates is also acknowledged by \citet[288]{Talmy2000}. Significantly, locative PPs appearing in this type of predicates (e.g., \textit{in her coat} in \REF{bla5}) are typically headed by spatial prepositions expressing a stative relation (e.g., \textit{in}), rather than by prepositions expressing Path (e.g., \textit{into}).} 
Accordingly, in light of contrasts like \REF{mainbla}, \citet{Mateu2012} concludes that a proper descriptive account of the cross-linguistic variation associated with \citeauthor{Talmy2000}'s typology should not be understood in terms of a requirement about the expression of Path (either in the main verb or in a verb's satellite), but rather in terms of whether or not a language allows the expression of a co-event in the main verb.

\largerpage
A prediction of this line of reasoning is that weak satellite-framed languages such as Slavic languages should allow complex creation/consumption predicates of the type in \REF{bla5}, since these languages more generally display constructions where the main verb expresses a co-event (as exemplified in \REF{ispisala}). In this chapter, I present the results of a pilot study investigating the availability of different types of creation/consumption predicates in several Slavic languages, comparing them with data from bona fide satellite-framed languages and verb-framed languages. I provide evidence suggesting that Slavic languages behave as verb-framed languages in the domain of creation/consumption predicates, as they must resort to run-of-the-mill verb-framed strategies to express such predicates and rule out constructions such as complex creation/consumption predicates (\sectref{2}). Assuming a neo-constructionist approach to argument structure (\citealt{Mateu2002}; \citealt{Borer2005b}; \citealt{MateuAndAcedo-Matellan2012}, among others), I propose a morphophonological account of Talmy's typology, which is argued to follow from a Phonological Form (PF) requirement, in verb-framed languages, on the null syntactic head \textit{v} involved in verbal predication. I suggest that Slavic languages, and weak satellite-framed languages in general, should be considered as fundamentally verb-framed languages, predicting the unavailability of complex creation/consumption predicates in this class of languages (\sectref{3}). Next, I explore the prediction -- following from the present account -- that a complex creation/consumption reading is available in Slavic languages for predicates that are perfectivized via so-called “internal” verbal prefixes (\citealt{big:Svenonius2004}; \citealt{big:Borik2006}; \citealt{big:Gehrke2008}, among others), which have been argued to express an abstract result in a resultative construction (\citealt{big:Gehrke2008}; \citealt{Acedo-Matellan2016}; \citealt{big:Kwapiszewski2022}, among others) (\sectref{4}). Finally, I address some potential counterexamples from Latin (another weak satellite-framed language; \citealt{Acedo-Matellan2016}) to the prediction that weak satellite-framed languages lack complex creation/consumption predicates of the type found in satellite-framed languages. I argue that Latin lacked such predicates in the same way as Slavic languages do, \textit{pace} \citet{Acedo-Matellan2016} and consistently with the predictions of the present account (\sectref{5}). I draw general conclusions in \sectref{6}.


\section{Creation/consumption predicates in Slavic languages}\label{2}

In order to investigate the availability of complex creation/consumption predicates in Slavic languages, I carried out a pilot study to check, with the help of native speakers, whether it was possible to directly translate %(namely Russian, Ukrain- ian, Polish, Slovak, Serbian and Croatian) 
different creation/con\-sump\-tion predicates that are licensed in satellite-framed English into several Slavic languages. I further examined whether it was possible to directly translate the English examples into four additional bona fide satellite-framed languages %(German, Dutch, Chinese and Hungarian) 
and five verb-framed languages, %(Italian, Catalan, Spanish, Basque and Greek) 
respectively. %In order to diversify as much as possible the origin of the collected data, effort was invested in gathering evidence from different language families irrespective of their typological classification with respect to Talmy's typology. 
Effort was invested in gathering evidence from different language families, contributing to the diversity of languages represented in the collected data. % and reducing the chances that the results obtained depend on language family-specific constraints unrelated to Talmy’s typology. %the likelihood that language family-specific constraints independent of Talmy's typology influenced the results. the chances that the data collected are the result of language family-specific constraints that are independent of Talmy's typology. % reducing the chances of encountering language family-specific constraints that might bias the data independently of Talmy’s typology. the chances of encountering language family-specific constraints that might affect the data independently of Talmy's typology. 
For the class of satellite-framed languages, data were collected from Dutch, German, Chinese, and Hungarian. Regarding verb-framed languages, data were collected from Italian, Catalan, Spanish, Basque, and Greek. Finally, for the class of Slavic languages, data were collected from Russian and Ukrainian (East Slavic languages), Polish and Slovak (West Slavic languages), and Serbian and Croatian (South Slavic languages).\footnote{Serbian and Croatian are considered individually alongside the other languages examined, notwithstanding classifications that see them as distinct varieties of a single language (e.g., Serbo-Croatian, or BCMS).}


\subsection{The English data} \label{2.1}

The English examples %were taken from the literature, from corpora, or made up and subsequently checked with native speakers of English, and 
range from constructions involving verbs whose meaning can be taken to imply the creation/consumption of the direct object, therefore using a verb-framed strategy, to constructions that can be taken to involve the expression of a manner co-event in the main verb, and which are expected to be ungrammatical in verb-framed languages.\footnote{The selection of the data was primarily based on examples from relevant literature pertaining to hyponymous objects, effected objects, and Talmy's typology. Additionally, some examples were taken from corpora or made up and subsequently checked with native speakers. %Certain data were chosen due to their availability in satellite-framed English and their ungrammaticality in the two verb-framed languages of which I am a native speaker.
Following \citet{Mateu2002}, I have included the examples in \REF{walk} and \REF{swim} as representatives of the class of complex consumption predicates, where the consumption of the direct object constitutes the main event denoted by the predicate, while the verb denotes a co-event. See \citet{Kuno1973} and \citet{Condamines2013} for possible examples of this type in verb-framed Japanese and French, respectively (I thank an anonymous reviewer for bringing my attention to the data analyzed in these works, which deserve further investigation).} The list of the selected examples, starting with verb-framed constructions, is provided in \REF{sang} to \REF{Elnafrowned}.\footnote{The examples have been arranged in the present order based on my own intuitions, as a native speaker of one of the verb-framed languages tested, about the degree of ``manner'' provided by the verb in each of them. Determining the degree of manner provided by the verb in each of the sentences in \REF{sang} to \REF{Elnafrowned} is a complex process that takes place at the conceptual level. %This involves considering the conceptual representation of the creation/consumption event arising from the semantic construal, the conceptual content of the root as listed in the encyclopedia, and the world knowledge-based representation of the entity denoted by the direct object. As I will discuss in \sectref{3.1}, the argument structure of the predicates in these examples gives rise to a reading of the direct object as being either created or consumed during the event. To ascertain whether the event denoted by the verb's root is interpreted as the primary creation/consumption event or as a manner/cause co-event, all three factors listed above must be taken into account. 
To the best of my knowledge, there is currently no objective method to quantitatively measure the degree of manner provided by the verb in a specific construction, leaving the intuition-based approach as the only viable option. %In this respect, it is telling that the results obtained show a clear consistency, cross-linguistically, with the conclusions reached on the basis of my own intuitions.
}

\ea John sang a song.  \label{sang} \hfill (\citealt[1361]{Truswell2007}) \z
\ea They danced a Sligo jig. \hfill (\citealt[98]{Gallego2012}) \z
\ea Ariel ate the mango. \hfill (\citealt[52]{big:Ramchand2008}) \z
\ea He dug a hole in the ground. \label{dughole}\hfill (COCA\footnote{Corpus of Contemporary American English (\citealt{Davies2008}).})\z
\ea She wove the tablecloth. \label{weave} \hfill (adapted from \citealt[452]{FolliAndHarley2020})\z
\ea Marco painted a sky. \label{paint} \hfill (\citealt[438]{FolliAndHarley2020})\z
\ea Maria carved a doll. \label{carve} \hfill (\citealt[439]{FolliAndHarley2020})\z
\ea She burned a hole in her coat. \hfill (made up)\z
\ea He scratched a hole in the ground. \label{scratch} \hfill (COCA)\z
\ea She punctured a wound in her finger. \label{puncture} \hfill (made up)\z
\ea She cut a wound in her foot. \hfill (made up)\z
\ea She bit a hole in the bag. \hfill (COCA)\z
\ea The adventurer walked the trail. \label{walk} \\ \hfill (\citealt[297]{Mateu2002}, adapted from \citealt[17]{Tenny1994})\z
\ea The adventurer swam the channel. \label{swim} \\ \hfill (\citealt[297]{Mateu2002}, adapted from \citealt[17]{Tenny1994})\z
\ea Deanne kicked a hole in the wall. \label{kick} \hfill (COCA)\z
\ea She magicked a cursor. \label{magickednoPP} \hfill (COCA)\z
\ea She brushed a hole in her coat. \label{brushhole2} \\ \hfill (\citealt[213]{MateuandRigau2002}, adapted from \citealt{LevinAndRapoport1988}%[278]
)\z
\ea John smiled his thanks. \label{Johnsmiled} \\ \hfill (\citealt[255]{Mateu2012}, adapted from \citealt{LevinAndRapoport1988})\z
\ea Elna frowned her discomfort. \label{Elnafrowned}\hfill (\citealt[35]{AcedoMatellan&Kwapiszewski2021})
\z


\noindent All the examples in \REF{sang} to \REF{Elnafrowned} are taken to lack a Path component in their argument structure. While this is proposed by much work adopting both the lexicalist approach and the neo-constructionist approach (as pointed out in \sectref{1}), such work is mostly concerned with the argument structure of (verb-framed) predicates in which the meaning of the verb can be taken to imply the creation/consumption of the object. Following \citet{Mateu2012}; \citet{Acedo-Matellan2016}; \citet{FolliAndHarley2020}, among others, I extend such an analysis to satellite-framed predicates of creation/consumption in which the verb is taken to express a co-event. At first sight, predicates of this type might be argued to involve the argument structure of resultative predicates since most of them typically require a locative PP, which is instead omissible in predicates of creation/consumption that involve a verb-framed strategy. See, in this respect, the contrast between \REF{bla}, assumed to be verb-framed, and \REF{bla5}, repeated in \REF{dugnoPP} and \REF{brushnoPP}, respectively.\footnote{The judgment in \REF{brushnoPP} is by an anonymous reviewer.}

\ea \label{dugbrushhole} \ea \label{dugnoPP} He dug a hole (in his garden).
\ex \label{brushnoPP} Brush a hole *(in one's coat). \z \z

\noindent Based on the contrast in \REF{dugbrushhole}, the current assumption that satellite-framed predicates denoting events of creation/consumption do not involve a Path component in their argument structure might be questioned. Specifically, an anonymous reviewer suggests that the PP could be expressing a null Path in English predicates of the type in \REF{bla5} in the same way as it seems to do in predicates denoting events of change such as \textit{walk in the room}, considered by the reviewer to be ambiguous between a locative and a change-of-location reading (but see, e.g., \citealt[83]{FolliAndRamchand2005} and \citealt[90]{big:Gehrke2008} for a different opinion).

The remainder of this subsection is devoted to showing that satellite-framed predicates of creation/consumption should not be taken to involve a null Path element in their argument structure. I argue that several reasons support this conclusion, even though the contrast in \REF{dugbrushhole}, at first sight, might seem to suggest otherwise. %I argue that there are several reasons pointing toward this conclusion, % that satellite-framed predicates of creation/consumption do not involve a Path component in their argument structure, 
%despite the fact that the contrast in \REF{dugbrushhole}, at first sight, might be taken to suggest otherwise. 
First, the claim that the PP in \textit{walk in the room} involves a phonologically null Path is disputable since Path, in such a predicate, has been argued in previous works to be expressed by the verb \textit{walk} (\citealt{Alexiadou2015}; further see \citealt[112, fn. 1]{big:Ramchand2008}; \citealt{Nikitina2008}; \citealt{BeaversLevinandTham2010}). This verb, given the right context, may be coerced by some speakers into an interpretation as involving directionality and hence goal of motion. %\footnote{In this respect, \citet[112, footnote 1]{big:Ramchand2008} notes that the possibility of a change-of-location reading of \textit{walk} seems to depend on the availability of a ``threshold crossing' interpretation of the event, whereby, for instance, an example like \textit{walk in the room} is more likely to be interpreted as a resultative than an example like \textit{walk in the park}. See also \citet{Nikitina2008}; \citet{BeaversLevinandTham2010} for the claim that a change-of-location reading of a predicate like \textit{walk in the room} depends on pragmatic factors linked to the context of utterance. %can acquire a change-of-location reading because of pragmatic factors.} 
This explains the existence of contrasts like the one depicted in \REF{walkdance}. Unlike \textit{walk}, \textit{dance} denotes an activity that typically does not imply directionality. As a result, this verb is less likely to express Path, which must therefore be expressed independently in order for the verb to appear in the change-of-location frame. %lexicalization of Path to be incorporated into the change-of-location frame. whereby it is more likely to be unable to lexicalize Path and require that Path be lexicalized independently of it in order to appear in the change-of-location frame. %: as argued by \citet{Alexiadou2015}, since \textit{walk} denotes an activity which may be used to reach a goal, this verb, given the right context, can license a goal of motion interpretation in the absence of a directional PP. On the other hand, verbs like \textit{dance}, which, based on world knowledge, denotes an activity that does not involve directionality, require Path to be lexicalized independently of them in order to appear in the change-of-location frame. %light of the fact that many other change-of-location predicates with manner denoting verbs require an overt Path in English, as the following contrast shows.

\ea \label{walkdance} \ea \label{walkalexiadou} John walked in the room. \hfill (in a change-of-location reading)
\ex $\#$John danced in the room. \hfill (in a change-of-location reading)\\ \hfill (\citealt[1093]{Alexiadou2015}) \z \z

\noindent Additionally, if the satellite-framed predicates of creation/consumption discussed in this chapter involved a phonologically null Path, the question would arise as to why Path \textit{must} be null in these predicates. Even by assuming that \REF{walkalexiadou} is compatible with a change-of-location reading, Path can optionally be overtly realized independently of the verb in resultative predicates of this type, as \REF{walkininto} shows.

\ea \label{walkininto} John walks in(to) the room. \hfill (in a change-of-location reading) \z

\noindent More strikingly, Path is mandatorily realized by a morpheme different from the verb in transitive resultatives featuring direct objects that are not semantically selected by the verb (meaning that they are not a traditional object of the verb based on what lexicalist approaches consider to be the verb's lexical argument structure, and would not be suitable objects of such a verb outside the resultative construction); see the contrast between the example in \REF{threeone} and the one in \REF{threetwo}, both examples displaying direct objects that are not semantically selected by their respective verb. %(\textit{a hole}, \textit{themselves} and \textit{herself} respectively). 
In \REF{threetwo}, which involves a bona fide resultative predicate, the presence of an overt Path (\textit{to}) is mandatory. This is not the case in \REF{threeone}, in contrast to what one would expect if the predicate in \REF{threeone} was resultative.%The complex creation predicate in \REF{threeone}, instead, is only acceptable without an overt Path. This is in contrast to what one would expect if the predicate in \REF{threeone} was resultative.%are result but only the ones in \REF{threetwo} and \REF{threethree}, I claim, being actual resultatives.
\footnote{Arguably, a literal interpretation of the predicate in \REF{threeone} could be considered grammatical with the presence of \textit{to}, but pragmatically aberrant, as the predicate would be interpreted as roughly meaning `move a hole to the inside of one's coat using a brush-like object / in a brush-like manner' (Jaume Mateu, p.c.).%Arguably, the example in \REF{threeone} would be grammatical, albeit pragmatically aberrant with \textit{to}, as it could be taken to roughly mean that `John moved a hole to the inside of his coat using a brush'.
}


\ea \label{threecontrast} \ea \label{threeone} Brush a hole in($\#$to) one's coat.
\ex \label{threetwo} The children run themselves *in/(in)to exhaustion. \hfill (\citealt[281]{Iwata2020})
%\ex \label{threethree} She talked herself *in/(in)to sleep. \hfill (\citealt{Ono2010}, \textit{apud} \citealt[281]{Iwata2020}) 
\z \z

\noindent A further piece of evidence against considering the locative PP in satellite-framed predicates of creation/consumption as containing a null Path comes from the observation that such a PP can also be headed by the preposition \textit{at}, as shown in \REF{bitpunchat}. Unlike \textit{in}, \textit{at} is only compatible with a non-directional reading and is in complementary distribution with \textit{to}. This strongly suggests that there is no null Path in the locative PPs found in the examples considered in this study.

\ea \label{bitpunchat} \ea They removed the coriaceous bracteoles wrapped outside of the corolla, bit a hole at the base of the corolla where the nectarines are located, and lapped up all the nectar in each flower. \hfill (Web)
\ex To really make it resemble a tea bag, Murphy punched a hole at the top, then added a length of twine and a ``tag''. \hfill (COCA) \z \z

\noindent This said, that the locative PP can be omitted in \REF{dugnoPP} but not in \REF{brushnoPP} is not necessarily due to grammatical reasons. Other factors, e.g., conceptual/pragmatic ones, might be involved. % in the observed contrast. 
Note that only \REF{brushnoPP} involves a direct object which is not semantically selected by the verb. %This object is interpreted as a hyponym of the verb’s root in the sense of Hale & Keyser (1997, 2002): a hole can be conceived of as a kind of dig, and an event of digging, in all probability, results in the creation of a cavity (that is to say, a hole). In contrast, the object in \REF{brushnoPP} is not semantically selected by the verb. 
\textit{Brush} is a verb of surface contact, and it typically appears with direct objects denoting the surface that is brushed. %This is not the case in \REF{brushnoPP}, in which the direct object is created during the event denoted by the verb. %According to the syntactic approach to argument structure assumed in this paper, to be laid out in \sectref{3}, the semantics of creation/consumption in both predicates in \REF{dugbrushhole} is imposed by their syntactic configuration. % of having the direct object \textit{a hole} merged as the complement of a verbal head \textit{v}. 
%Given that \textit{brush} is not typically used as a verb of creation, 
It can then be expected that the \textit{v}P in \REF{brushnoPP} requires additional contextual information in order to be interpreted under a creation reading. %, which is the reading in which \textit{a hole} is not semantically an object of \textit{brush}. 
In the absence of the spatial PP \textit{in the coat}, the default inferable reading would be the pragmatically aberrant (not ungrammatical, in my view) one in which \textit{a hole} is a selected object of \textit{brush} (that is to say, it is an existing entity that undergoes an event of \textit{brushing}). %the reading attributed, for instance, to the object “one’s hair” in “brush one’s hair”).
Such a reading disappears when the locative PP is added, as the PP introduces the semantic argument of the verb (i.e. the surface which is brushed, e.g., \textit{her coat}), favoring the interpretation of the direct object \textit{a hole} as an effected object thanks to the additional context. Further notice, in this respect, that locative PPs do not always appear in predicates of this type. For instance, no locative PP appears in the complex creation predicates in \REF{magickednoPP}, \REF{Johnsmiled} and \REF{Elnafrowned}, nor in the complex consumption predicates in \REF{walk} and \REF{swim}. I suggest that in these predicates, the intended creation/consumption reading arises based on world knowledge/pragmatic considerations regarding the scene denoted by the event which are clear enough without the necessity of additional contextual information.\footnote{This is in contrast to resultative predicates like \REF{threetwo}, where the licensing of a direct object that is not semantically selected by the verb always requires the presence of a phrase (e.g., a result PP) acting as a secondary predicate. Such a contrast can be taken to reflect the different status of the PPs appearing in complex creation/consumption predicates and the result PPs appearing in resultative predicates with non-selected objects, the former being adjuncts while the latter are arguments of the predicate. \label{PPareadjuncts} } %This is in contrast to result PPs in predicates of the type in \REF{threetwo} and \REF{threethree}, whose presence is always necessary for the licensing of a direct object which is not semantically selected by the verb. 



\subsection{Method and results} \label{2.2}

The examples in \REF{sang} to \REF{Elnafrowned} were presented to the speakers in a randomized order. Translations, glosses, and grammaticality judgments were collected by consulting one linguist native speaker per language.\footnote{One exception is the native speaker of Ukrainian, who is not a linguist but who is a proficient speaker of English. %and a language teacher.
} For each of the examples tested, it was ensured that the intended (creation/consumption) meaning of the predicate was clear to the speakers before soliciting a grammaticality judgment. Two caveats were further considered in gathering judgments from the native speakers of the Slavic languages selected. %First, taking into account that perfective aspect, in many Slavic languages, is achieved by means of prefixes that have been argued to play a role in the event domain 
First, considering that, as I will discuss in \sectref{4}, perfective aspect in many Slavic languages is achieved through prefixes which have been argued to play a role in the event domain and interfere with the data being analyzed, the English examples were presented in the imperfective aspect when soliciting corresponding translations from the native speakers of the Slavic languages tested.
%First, taking into account that, as I will discuss in \sectref{4}, perfective aspect in many Slavic languages is achieved by means of prefixes that have been argued to play a role in the event domain, and can be predicted to give rise to a crucial interference with the data under analysis, %, taking into account that, as I will discuss in \sectref{4}, perfective aspect in many Slavic languages is achieved by means of prefixes that also play a predicative role within the event domain, and can be predicted to give rise to a crucial interference with the data under analysis, %taking into account the fact that perfective aspect, in many Slavic languages, is achieved by means of prefixes that also play a predicative role within the event domain of the \textit{v}P (\citealt{RamchandAndSvenonius2002}; \citealt{big:Gehrke2008}; \citealt{Acedo-Matellan2016}; \citealt{big:Kwapiszewski2022}, among others), 
%the English examples were turned into the imperfective aspect when asking the native speakers of the Slavic languages tested for the corresponding translations. 
For instance, the availability of the English example in \REF{sang} was checked in Slavic languages using the imperfective construction \textit{John was singing a song}. Additionally, the speakers were asked to provide translations involving unprefixed verbs only. %Perfective prefixes play a role in the event domain which, as I will discuss in \sectref{4}, can be predicted to give rise to a crucial interference with the data under analysis. %the role played by perfective prefixes in the event domain can be predicted to give rise to a crucial interference with the data under analysis. Namely, these prefixes can license a creation/consumption interpretation of predicates with manner-denoting verbs. 
As a second caveat, when possible, the availability of a transitive non-creation use of those verbs which gave rise to ungrammatical translations in the languages tested was checked for each language, in order to exclude possible cases of ungrammaticality due to unrelated lexical restrictions on the transitivity of the verbs involved.\footnote{Such a non-creation use pertains to transitive predicates where the direct object is understood as a pre-existing entity which undergoes the action named by the verb, and is not created or consumed during the event. Compare, for instance, \REF{carve} with \textit{Maria carved the wood} (\citealt{FolliAndHarley2020}: 439), where the direct object pre-exists the carving event and undergoes the change of state specified by the verb.}

The results obtained are graphically summarized in Table \ref{sim:tab:sfcc}, Table \ref{sim:tab:vfcc}, and Table \ref{sim:tab:slavicccipfv} for satellite-framed languages, verb-framed languages, and Slavic languages, respectively.\footnote{In the tables, empty slots correspond to cases where a direct translation of the English verb is not available in the target language. For reasons of space, the languages examined are identified in the tables using the ISO 639-2/B standardized nomenclature (US Library of Congress).} The data collected %for each of the languages tested 
are provided in the \href{https://osf.io/5a8nw?view_only=ab1508753f9f4c25ba30358819e831b2}{Appendix}.

\begin{table}[p]
\caption{Creation/consumption predicates in satellite-framed languages}
\label{sim:tab:sfcc}
{
 \begin{tabularx}{1\textwidth}{lYYYY} %.77 indicates that the table will take up 77% of the textwidth
  \lsptoprule
        Example    & Dut & Ger  & Chi & Hun\\
  \midrule
  (6) John sang a song     &   \footnotesize\Checkmark    &  \footnotesize\Checkmark     &  \footnotesize\Checkmark     &    \footnotesize\Checkmark   \\
\tablevspace
  (7) They danced a Sligo jig    &   \footnotesize\Checkmark    &    \footnotesize\Checkmark   & \footnotesize\Checkmark      &    \footnotesize\Checkmark   \\
\tablevspace
  (8) Ariel ate the mango    & \footnotesize\Checkmark      & \footnotesize\Checkmark      &   \footnotesize\Checkmark    &  \footnotesize\Checkmark     \\
\tablevspace
  (9) He dug a hole in the ground    &   \footnotesize\Checkmark    &  \footnotesize\Checkmark     &     \footnotesize\Checkmark  &     \footnotesize\Checkmark  \\
\tablevspace
  (10) She wove the tablecloth    &  \footnotesize\Checkmark     &    \footnotesize\Checkmark   & \footnotesize\Checkmark      &    \footnotesize\Checkmark   \\
\tablevspace
  (11) Marco painted a sky    &    \footnotesize\Checkmark   &  \footnotesize\Checkmark     &   \footnotesize\Checkmark    &     \footnotesize\Checkmark  \\
\tablevspace
  (12) Maria carved a doll    & \footnotesize\Checkmark      &   \footnotesize\Checkmark    & \footnotesize\Checkmark      &   \footnotesize\Checkmark    \\
\tablevspace
  (13) She burned a hole in her coat    & \footnotesize\Checkmark      &    \footnotesize\Checkmark   &  \footnotesize\Checkmark     &    \footnotesize\Checkmark   \\
\tablevspace
  (14) He scratched a hole in the ground    &  \footnotesize\Checkmark     &    \footnotesize\Checkmark   &  \footnotesize\Checkmark     &   \footnotesize\Checkmark    \\
\tablevspace
  (15) She punctured a wound in her finger    &   \footnotesize\Checkmark    &  \footnotesize\Checkmark     &  \footnotesize\Checkmark     &    \scriptsize\FiveStar   \\
\tablevspace
  (16) She cut a wound in her foot    &   \footnotesize\Checkmark    &    \footnotesize\Checkmark   &   \footnotesize\Checkmark    &    \footnotesize\Checkmark   \\
\tablevspace
  (17) She bit a hole in the bag    &  \footnotesize\Checkmark     &   \footnotesize\Checkmark    &  \footnotesize\Checkmark     &   \footnotesize\Checkmark    \\
\tablevspace
  (18) The adventurer walked the trail    &    \footnotesize\Checkmark   &    \footnotesize\Checkmark   &  \footnotesize\Checkmark     &  ??     \\
\tablevspace
  (19) The adventurer swam the channel    &   \scriptsize\FiveStar    &   \scriptsize\FiveStar    &   \scriptsize\FiveStar    &  \scriptsize\FiveStar     \\
\tablevspace
  (20) Deanne kicked a hole in the wall    &  \footnotesize\Checkmark     &    \footnotesize\Checkmark   &  \footnotesize\Checkmark     &   \footnotesize\Checkmark    \\
\tablevspace
  (21) She magicked a cursor    &   ??    &       &    \footnotesize\Checkmark   & \footnotesize\Checkmark      \\
\tablevspace
  (22) She brushed a hole in her coat    &    \footnotesize\Checkmark   &  \footnotesize\Checkmark %the speaker accepted it! It appears in the Appendix. There was a typo in the table.
  &     \footnotesize\Checkmark  &    \footnotesize\Checkmark %amended a mistake 
  \\
\tablevspace
  (23) John smiled his thanks    &    \scriptsize\FiveStar   &   \scriptsize\FiveStar    &   \scriptsize\FiveStar    &   \scriptsize\FiveStar    \\
\tablevspace
  (24) Elna frowned her discomfort    &   \scriptsize\FiveStar    &       &       &   \scriptsize\FiveStar    \\
  \lspbottomrule
 \end{tabularx}}
\end{table}


\begin{table}[p]
\caption{Creation/consumption predicates in verb-framed languages}
\label{sim:tab:vfcc}
{
 \begin{tabularx}{1\textwidth}{lYYYYY} %.77 indicates that the table will take up 77% of the textwidth
  \lsptoprule
        Example    & Ita & Cat  & Spa & Baq & Gre\\
  \midrule
  (6) John sang a song     &  \footnotesize\Checkmark    &  \footnotesize\Checkmark    &   \footnotesize\Checkmark   &  \footnotesize\Checkmark    &   \footnotesize\Checkmark   \\
\tablevspace
  (7) They danced a Sligo jig     &   \footnotesize\Checkmark   &    \footnotesize\Checkmark  &   \footnotesize\Checkmark   &   \footnotesize\Checkmark   &    \footnotesize\Checkmark  \\
\tablevspace
  (8) Ariel ate the mango     &   \footnotesize\Checkmark   &  \footnotesize\Checkmark    &  \footnotesize\Checkmark    &  \footnotesize\Checkmark    &   \footnotesize\Checkmark   \\
\tablevspace
  (9) He dug a hole in the ground     &   \footnotesize\Checkmark   &   \footnotesize\Checkmark   &   \footnotesize\Checkmark   &   \footnotesize\Checkmark   &   \footnotesize\Checkmark   \\
\tablevspace
  (10) She wove the tablecloth     &   \footnotesize\Checkmark   &   \footnotesize\Checkmark   &   \footnotesize\Checkmark   &    \footnotesize\Checkmark  &  \footnotesize\Checkmark    \\
\tablevspace
  (11) Marco painted a sky     &  \footnotesize\Checkmark    &    \footnotesize\Checkmark  &   \footnotesize\Checkmark   &    \footnotesize\Checkmark  &   \footnotesize\Checkmark   \\
\tablevspace
  (12) Maria carved a doll     &   \footnotesize\Checkmark   &   \footnotesize\Checkmark   &  \footnotesize\Checkmark    &   \footnotesize\Checkmark   &   \footnotesize\Checkmark   \\
\tablevspace
  (13) She burned a hole in her coat     &  \scriptsize\FiveStar   &  \scriptsize\FiveStar   &  \scriptsize\FiveStar   &  \footnotesize\Checkmark    &  \scriptsize\FiveStar   \\
\tablevspace
  (14) He scratched a hole in the ground     &  \scriptsize\FiveStar   &  \scriptsize\FiveStar   &  \scriptsize\FiveStar   &  \footnotesize\Checkmark    &   \footnotesize\Checkmark   \\
\tablevspace
  (15) She punctured a wound in her finger     &   ??   & \scriptsize\FiveStar    & \scriptsize\FiveStar    &    ??  &  \scriptsize\FiveStar   \\
\tablevspace
  (16) She cut a wound in her foot     & \scriptsize\FiveStar    & \scriptsize\FiveStar    &   \scriptsize\FiveStar  &  ??    &  \scriptsize\FiveStar   \\
\tablevspace
  (17) She bit a hole in the bag     &  \scriptsize\FiveStar   &  \scriptsize\FiveStar   &  \scriptsize\FiveStar   &   ?   &  \scriptsize\FiveStar   \\
\tablevspace
  (18) The adventurer walked the trail     & \scriptsize\FiveStar    &  \scriptsize\FiveStar   &  ?    &   \footnotesize\Checkmark   &    \footnotesize\Checkmark  \\
\tablevspace
  (19) The adventurer swam the channel     &  \scriptsize\FiveStar   &  \scriptsize\FiveStar   &  ?    &      &  \footnotesize\Checkmark    \\
\tablevspace
  (20) Deanne kicked a hole in the wall      & \scriptsize\FiveStar    &      &  \scriptsize\FiveStar   & \scriptsize\FiveStar    &   \scriptsize\FiveStar  \\
\tablevspace
  (21) She magicked a cursor      &      &      &      &      &  \scriptsize\FiveStar   \\
\tablevspace
  (22) She brushed a hole in her coat      &  \scriptsize\FiveStar   &  \scriptsize\FiveStar   &  \scriptsize\FiveStar   &  \scriptsize\FiveStar   &  \scriptsize\FiveStar   \\
\tablevspace
  (23) John smiled his thanks      &  \scriptsize\FiveStar   &  \scriptsize\FiveStar
\tablevspace
  &   ?   &      &   ?   \\
  (24) Elna frowned her discomfort      &   \scriptsize\FiveStar   &  \scriptsize\FiveStar   &   \scriptsize\FiveStar   &      &\scriptsize\FiveStar     \\
  \lspbottomrule
 \end{tabularx}}
\end{table}


\begin{table}[p]
\caption{Creation/consumption predicates in Slavic languages (imperfective, unprefixed predicates)}
\label{sim:tab:slavicccipfv}
{
 \begin{tabularx}{1\textwidth}{lYYYYYY} %.77 indicates that the table will take up 77% of the textwidth
  \lsptoprule
        Example    & Rus & Ukr  & Pol & Slo & Ser & Hrv \\
  \midrule
  (6) John sang a song     &  \footnotesize\Checkmark   &   \footnotesize\Checkmark  &   \footnotesize\Checkmark  &  \footnotesize\Checkmark   &  \footnotesize\Checkmark   & \footnotesize\Checkmark    \\
\tablevspace
  (7) They danced a Sligo jig     &  \footnotesize\Checkmark   &   \footnotesize\Checkmark  &  \footnotesize\Checkmark   &  \footnotesize\Checkmark   & \footnotesize\Checkmark    &   \footnotesize\Checkmark  \\
\tablevspace
  (8) Ariel ate the mango     &  \footnotesize\Checkmark   &   \footnotesize\Checkmark  &  \footnotesize\Checkmark   &   \footnotesize\Checkmark  &  \footnotesize\Checkmark   &  \footnotesize\Checkmark   \\
\tablevspace
  (9) He dug a hole in the ground     &  \footnotesize\Checkmark   &   \footnotesize\Checkmark  &  \footnotesize\Checkmark   &  \footnotesize\Checkmark   &  \footnotesize\Checkmark   &   \footnotesize\Checkmark  \\
\tablevspace
  (10) She wove the tablecloth     &  \footnotesize\Checkmark   &   \footnotesize\Checkmark  &   \footnotesize\Checkmark  &   \footnotesize\Checkmark  &   \footnotesize\Checkmark  &  \footnotesize\Checkmark   \\
\tablevspace
  (11) Marco painted a sky     &   \footnotesize\Checkmark  &  \footnotesize\Checkmark   &  \footnotesize\Checkmark   &   \footnotesize\Checkmark  &  \footnotesize\Checkmark   & \footnotesize\Checkmark    \\
\tablevspace
  (12) Maria carved a doll     &  \scriptsize\FiveStar   &  \scriptsize\FiveStar   &   \footnotesize\Checkmark  &   ?  &  \footnotesize\Checkmark   &  \footnotesize\Checkmark   \\
\tablevspace
  (13) She burned a hole in her coat     & \scriptsize\FiveStar    &   \footnotesize\Checkmark %amended a mistake
  &   ?  & ??    &  \scriptsize\FiveStar   &  \footnotesize\Checkmark   \\
\tablevspace
  (14) He scratched a hole in the ground     &   \footnotesize\Checkmark  &  \footnotesize\Checkmark   & \scriptsize\FiveStar    &  \footnotesize\Checkmark   &  ??   &   \scriptsize\FiveStar  \\
\tablevspace
  (15) She punctured a wound in her finger     &   \scriptsize\FiveStar  &   \footnotesize\Checkmark  &  \scriptsize\FiveStar   &  \scriptsize\FiveStar   &  ??   &   \scriptsize\FiveStar  \\
\tablevspace
  (16) She cut a wound in her foot    &\scriptsize\FiveStar    &  \scriptsize\FiveStar  & \scriptsize\FiveStar   & \scriptsize\FiveStar   & \scriptsize\FiveStar   & \scriptsize\FiveStar   \\
\tablevspace
  (17) She bit a hole in the bag    & \scriptsize\FiveStar   & \scriptsize\FiveStar   & \scriptsize\FiveStar   &  \scriptsize\FiveStar  &  ?   &  \scriptsize\FiveStar  \\
\tablevspace
  (18) The adventurer walked the trail    &  \scriptsize\FiveStar  & \scriptsize\FiveStar   & \scriptsize\FiveStar   & \scriptsize\FiveStar   &  ??   &  \scriptsize\FiveStar  \\
\tablevspace
  (19) The adventurer swam the channel    & \scriptsize\FiveStar   & \scriptsize\FiveStar   & \scriptsize\FiveStar   & \scriptsize\FiveStar   &  ??   &  \scriptsize\FiveStar  \\
\tablevspace
  (20) Deanne kicked a hole in the wall    & \scriptsize\FiveStar   & \scriptsize\FiveStar   &  \scriptsize\FiveStar  & \scriptsize\FiveStar   &  \scriptsize\FiveStar  & \scriptsize\FiveStar   \\
\tablevspace
  (21) She magicked a cursor    &  \scriptsize\FiveStar  &  \scriptsize\FiveStar  & \scriptsize\FiveStar   &  \scriptsize\FiveStar  &  \scriptsize\FiveStar  &  \footnotesize\Checkmark   \\
\tablevspace
  (22) She brushed a hole in her coat    &  \scriptsize\FiveStar  &  \scriptsize\FiveStar  & \scriptsize\FiveStar   &   \footnotesize\Checkmark  & \scriptsize\FiveStar   & \scriptsize\FiveStar   \\
\tablevspace
  (23) John smiled his thanks    &     &     &     &     &     &     \\
\tablevspace
  (24) Elna frowned her discomfort    &     &     &     &     &     &     \\
  \lspbottomrule
 \end{tabularx}}
\end{table}

Overall, the native speakers of the satellite-framed languages tested accepted a literal translation for the vast majority of the complex creation/consumption predicates provided from English (Table \ref{sim:tab:sfcc}), consistently with Talmy's typology.\footnote{I
    assume that Mandarin Chinese is a standard satellite-framed language of the English type. \citet{Acedo-Matellan2016} argues that some varieties of Chinese are weak satellite-framed because the satellite-framed constructions they display present the Path and the co-event components as univerbated in a sort of V-V compound (see also \citealt{Fan2014}). However, the idea that the Path and the co-event components in Chinese resultatives form a complex head is disputed. For instance, \citet{Wang2010} presents evidence of phrasal elements that may intervene between the two members of the V-V compound in Chinese resultatives. We can see this in \REF{Chinesebutai}, where the complex negation \textit{bu tai} `not too' disrupts the adjacency between \textit{da} `hit' and \textit{si} `die'.


    \ea \label{Chinesebutai} \gll Wo da bu tai si na zhi zhanglang.\\
    I hit \textsc{neg} too die that \textsc{cl} cockroach\\
    \glt `I can hardly hit the cockroach to death.' \hfill (Chinese; \citealt[38]{Wang2010}) \z
}
\clearpage

The results obtained from the native speakers of the verb-framed languages tested are considerably different when it comes to predicates that are understood as involving the expression of a co-event by the verb (Table \ref{sim:tab:vfcc}). A literal translation of the English examples gets progressively more difficult to obtain in the verb-framed languages as the predicates shift from a verb-framed strategy (the verb implying the creation/consumption of the object) to a satellite-framed strategy (the verb being understood as specifying a co-event of the main event of creation/consumption), in accordance with the typology.

As Table \ref{sim:tab:slavicccipfv} makes clear, Slavic languages behave on a par with verb-framed languages in disallowing creation/consumption predicates where the meaning of the verb cannot be taken to imply the creation/consumption of the entity denoted by the object. The literal translations in \REF{literalRubrushetc} of the satellite-framed example in \REF{brushhole2} (also in \REF{bla5}) in Russian, Ukrainian, and Polish illustrate this.

\ea \label{literalRubrushetc} \ea[*] {\gll  Ona čes-a-l-a dyrku v pal'to.\\
she.{\NOM} brush.\textsc{ipfv}-\textsc{th}-{\PST-\AGR} hole.{\ACC} in coat.{\LOC} \label{Rubrush}\\ \hfill (Russian)} 
\ex[*] {\gll Vona ter-l-a dyrku na kurtci.\\
she.{\NOM} brush.\textsc{ipfv}-{\PST-\AGR} hole.{\ACC} in coat.{\LOC} \label{Ukbrush}\\ \hfill (Ukrainian)} 
\ex[*] {\gll Ona czes-a-ł-a dziurę w płaszczu.\\
she.{\NOM} brush.\textsc{ipfv}-\textsc{th}-{\PST-\AGR} hole.{\ACC} in coat.{\LOC} \label{Pobrush}\\ \hfill (Polish)}
\z 
\glt Intended: `She was brushing a hole in her coat.'
\z

\noindent In such cases, a verb-framed construction displaying a verb whose meaning implies the creation/consumption of the direct object has to be used instead, the manner co-event being optionally expressed as an adjunct.\footnote{The results obtained further warn against making generalizations about the typological behavior of a language based on individual examples. For instance, the example in \REF{carve} seems to be generally available in the verb-framed languages examined, but it does not fare well in Slavic languages such as Russian, Ukrainian, and Slovak. Instead, the example in \REF{scratch} presents a high degree of variation both in verb-framed languages and in weak satellite-framed Slavic languages, as it is accepted in half of the Slavic languages and in two of the five verb-framed languages examined.  Additionally, none of the native speakers of the satellite-framed languages checked seems to accept the example in \REF{swim}, even though they accept the similar example in \REF{walk} and even though \REF{swim} is accepted by the native speaker of verb-framed Greek. It is also worth noticing that the examples in \REF{Johnsmiled} and \REF{Elnafrowned}, despite being well-formed in English, do not fare well in any of the other satellite-framed languages tested according to the native speakers consulted. Arguably, some level of idiomaticity is present in these two constructions of English, which is not shared by the speakers of the other satellite-framed languages tested. Further similar irregularities are detected, which nonetheless do not affect the emergence of clear trends consistent with the predictions following from Talmy's typology.}

\ea \label{overtlightverb} \ea \gll Ona del-a-l-a dyrku v pal'to \v{s}\v{c}ëtkoj.\\
she.{\NOM} make.\textsc{ipfv}-\textsc{th}-{\PST-\AGR} hole.{\ACC} in coat.{\LOC} brush.\textsc{ins} \label{Rubrush2}\\ \hfill (Rus)%(Russian) %\vspace{4pt}
\ex \gll Vona rob-y-l-a dyrku na kurtci \v{s}\v{c}itkoju.\\
she.{\NOM} make.\textsc{ipfv}-\textsc{th}-{\PST-\AGR} hole.{\ACC} in coat.{\LOC} brush.\textsc{ins} \label{Ukbrush2}\\ \hfill (Ukr)%(Ukrainian) %\vspace{4pt}
\ex \gll Ona rob-i-ł-a dziurę w płaszczu szczotką.\\
she.{\NOM} make.\textsc{ipfv}-\textsc{th}-{\PST-\AGR} hole.{\ACC} in coat.{\LOC} brush.\textsc{ins}\label{Pobrush2}\\ \hfill (Pol)%(Polish) %\vspace{4pt}
\z
\glt `She was making a hole in her coat with a brush.'
\z

\noindent In the next section %, assuming a neo-constructionist view of argument structure, 
I propose a formal account of the patterns observed in terms of a PF requirement holding of the functional head \textit{v} involved in verbal predicates. The requirement is argued to affect both verb-framed and weak satellite-framed languages, explaining the uniformity of results observed in these languages.

\section{A morphophonological account of Talmy's typology}\label{3}

\subsection{A syntactic approach to argument structure}\label{3.1}

I adopt a neo-constructionist view of argument structure along the lines of \citet{MateuAndAcedo-Matellan2012}, according to which argument structure is conceived of as consisting of the relations established between a head and its arguments (i.e. its specifier and complement) in syntax. A fundamental distinction is drawn between functional heads, which are abstract relational elements that are necessary for the building of syntactic structures, and roots, regarded as units of conceptual content that provide real world details to syntactic predicates and are devoid of grammatically relevant information (\citealt{Mateu2002}; \citealt{Borer2005b}; \citealt{Acedo-Matellan2010}; \citeyear{Acedo-Matellan2016}, among others).

In this approach, satellite-framed constructions are understood as involving the conflation, i.e. e(xternal)-merge (\citealt{Haugen2009}), of a root with a phonologically null verbal head \textit{v}, whose complement receives a morphological realization independently of the verb. The root conflated with \textit{v} is understood as specifying a co-event of the main event arising from the predicate (\citealt{big:Embick2004}; \citealt{Harley2005}; \citealt{MateuAndAcedo-Matellan2012}; \citealt{AusensiAndBigolin2023}, among others). In the case of resultative (change-of-state/location) predicates, such as \REF{sc} (whose syntactic structure is illustrated in Figure \ref{sim:fig:sc}), \textit{v} takes a small clause as complement (PredP in Figure \ref{sim:fig:sc}), where the undergoer of the transition and the final state/location are introduced (\citealt{Hoekstra1988}).\footnote{In the structures, I represent roots with small capitals, following \citet{Acedo-Matellan2016}.}

\ea \label{sc} The bottle floated into the cave. \hfill (\citealt[227]{Talmy2000})\z

\begin{figure}[t]
% the [ht] option means that you prefer the placement of the figure HERE (=h) and if HERE is not possible, you prefer the TOP (=t) of a page
% \centering
    \begin{forest}
    for tree={s sep=1cm, inner sep=0, l=0}
    [vP [v
            [\textsc{float}] 
                [v
                ]
        ]
        [PredP
            [DP [the bottle] ]
            [Pred$'$ 
                [Pred] [PP [into the cave] ]
            ] ] ]
    % the overlay option avoids making the bounding box of the tree too large
    % the looseness option defines the looseness of the arrow (default = 1)
    \end{forest}
\caption{Syntactic structure of \REF{sc}}
    \label{sim:fig:sc}
\end{figure}

Verb-framed languages are different from satellite-framed languages in that they never show the conflation pattern depicted in Figure \ref{sim:fig:sc} (\citealt{Mateu2012}). In verb-framed languages, the predicative complement of the small clause always forms a unit with the \textit{v} head, whereby the only resultative predicates attested are those formed via incorporation (\citealt{Mateu2002}; \citeyear{Mateu2017}; \citealt{MateuandRigau2002}; \citealt{FolliAndHarley2020}, among others).\footnote{Following \citet{HaleAndKeyser2002} and \citeauthor{MateuandRigau2002} (\citeyear{MateuandRigau2002}, \citeyear{MateuandRigau2010}), I consider overt PPs expressing the final location of change-of-location events in verb-framed predicates (e.g., \textit{a la cueva} in \REF{sccc2}) as hyponymous arguments that further specify the result provided by the root that incorporates into \textit{v}. In the syntactic structures, hyponymous arguments are omitted for ease of exposition. %hyponymous arguments are represented as e(xternally)-merged with the root of which they are interpreted as hyponyms in the context of the predicate. The analysis reflects the status of these arguments as part of the description of the predicate, but subject to no subevent. of any subevent 
For discussion of possible syntactic representations of hyponymous arguments, see \citeauthor{HaleAndKeyser1997b} (\citeyear{HaleAndKeyser1997b}, \citeyear{HaleAndKeyser2002}); \citet{Mateu2008}; \citet{Haugen2009}; \citet{Gallego2012}; \citet{RealPuigdollers2013}, among others. %What is relevant for the present concerns is that verb-framed constructions appear to involve the incorporation into \textit{v} of a root that is externally merged within \textit{v}'s complement.
} The syntactic argument structure in Figure \ref{sim:fig:sccc2}, relative to the Spanish verb-framed change-of-location predicate in \REF{sccc2}, illustrates this.

\largerpage
\ea \label{sccc2} La botella entró (flotando) a la cueva.\\`The bottle entered the cave (floating).' \hfill (\citealt[227]{Talmy2000}) \z

\begin{figure}
    \begin{forest}
    for tree={s sep=1cm, inner sep=0, l=0}
    [vP [v [Pred [\textsc{entr-}] [Pred] ] [v] ] [PredP [DP [la botella] ] [Pred$'$ [\st{Pred}] [\st{\textsc{entr-}} ] ] ] ]
    \end{forest}
\caption{Syntactic structure of \REF{sccc2}}
    \label{sim:fig:sccc2}
\end{figure}

As for creation/consumption predicates, these are argued to involve an unergative configuration (à la \citealt{HaleAndKeyser1993}) consisting of a \textit{v} head that takes as its complement either a root, which subsequently incorporates into it (the overt object emerging as a hyponym of the verb; \citealt{HaleAndKeyser1997b}; \citeyear{HaleAndKeyser2002}), or an independent DP. In the latter case, \textit{v} may either appear as an overt light verb (e.g., \textit{make}, as in the Spanish example in \REF{bla6}) or conflate with another root, giving rise to the complex creation/consumption predicates that are peculiar to the satellite-framed languages (\citealt{Mateu2012}). The root incorporation pattern, corresponding to predicates of the type in \REF{blaentrambi} (see, e.g., \REF{bla}, repeated in \REF{blab}), is represented in Figure \ref{sim:fig:blab}. The pattern involving conflation is shown in Figure \ref{sim:fig:bla5b}, which represents the syntactic structure of \REF{bla5} (repeated in \REF{bla5b}).\footnote{The spatial PPs in \REF{blab} and \REF{bla5b} are treated as \textit{v}P-external adjuncts (see also footnote \ref{PPareadjuncts}) and are omitted from the syntactic representations for ease of exposition. For the same reason I omit the representation of the external argument, which, following considerations in \citet{Marantz1984}; \citet{big:Kratzer1996}; \citet{Pylkkanen2008}, among others, I assume to be introduced by a functional head Voice merged on top of the \textit{v}P.
}

\ea \label{blab} He dug a hole in his garden.    \hfill (\citealt[46]{Washio1997}) \z

\begin{figure}[ht]
    \begin{forest}
    for tree={s sep=1cm, inner sep=0, l=0}
    [vP [v [\textsc{dig}] [v] ] [ \textsc{dig} ] ]
    \end{forest}
\caption{Syntactic structure of \REF{blab}}
    \label{sim:fig:blab}
\end{figure}

\ea \label{bla5b} She brushed a hole in her coat. \\ \hfill (\citealt[213]{MateuandRigau2002}, based on \citealt{LevinAndRapoport1988}) \z

\begin{figure}[ht]
    \begin{forest}
    for tree={s sep=1cm, inner sep=0, l=0}
    [vP [v [\textsc{brush}] [v] ] [DP [a hole] ] ]
    \end{forest}
\caption{Syntactic structure of \REF{bla5b}}
    \label{sim:fig:bla5b}
\end{figure}


At first sight, the presence vs. absence of the operation conflating a root with \textit{v} in the syntax of a given language might seem to successfully account for the language's behavior with respect to Talmy's typology. However, there are at least two reasons, one theoretical and one empirical, why the availability of this syntactic operation in a given language cannot be taken as such as an effective way of explaining the typology. On the theoretical side, as noted in \citet{FolliAndHarley2020}, parameterizing the availability of a specific syntactic operation comes at the cost of giving up on the basic minimalist assumption that variation is not located in narrow syntax. On the empirical side, the results presented in \sectref{2} show that correlating Talmy's typology with the presence vs. absence of the syntactic operation conflating a root with \textit{v} leads to a wrong prediction when it comes to the possibility of licensing complex creation/consumption predicates in weak satellite-framed languages like Slavic languages (see Table \ref{sim:tab:slavicccipfv}).

In what follows, I propose an account of Talmy's typology which locates the source of the cross-linguistic variation at the PF level, understanding it in terms of differing morphophonological realization conditions of individual functional and lexical items. Not only does such an account seem to make the correct predictions with respect to the relevant patterns of cross-linguistic variation, it also provides a solution to the conundrum whereby verb-framed languages appear to consistently lack a structure-building operation (\textit{viz}. the conflation of a root with \textit{v}) that is instead available in satellite-framed languages.


\subsection{A PF requirement on the \textit{v} head in verb-framed languages}\label{3.2}

I endorse a view of cross-linguistic variation as primarily consisting in differing morphophonological realization conditions of functional heads (\citealt{Acedo-Matellan2016}; \citealt{Mateu2017}, among others). In order to account for the variation observed in relation to Talmy's typology, I posit that the \textit{v} head in verb-framed languages is associated with a PF requirement which imposes the incorporation of \textit{v}'s complement into \textit{v} when \textit{v} is phonologically null.\footnote{The requirement in \REF{PFrequirement} may ultimately be understood as an instance of \citeposst{ArregiAndPietraszko2021} ``Generalized Head Movement'' (GenHM) operation. This operation is captured by \citet{ArregiAndPietraszko2021} by means of a feature [hm] on syntactic heads which, when present, requires them to form a single morphological word with the closest head of their complement. Although \citet{ArregiAndPietraszko2021} formalize GenHM as a syntactic operation, they leave open the possibility that such an operation is carried out in the PF branch of the derivation (see \citealt{big:Kwapiszewski2022} for a PF implementation of GenHM). I am grateful to Víctor Acedo-Matellán for drawing my attention to the work of \citet{ArregiAndPietraszko2021}.}

\ea \label{PFrequirement} \textit{Verb-framed languages' PF requirement}:\\ A phonologically null \textit{v} must incorporate its complement. \z

\noindent The requirement in \REF{PFrequirement} predicts that the typological patterns noted by \citeauthor{Talmy2000} hold regardless of whether a result component is involved (as in the case of change-of-location/state predicates) or not (as in creation/consumption predicates). This is so because the \textit{v} head is found in both resultative predicates and creation/consumption predicates, as discussed in \sectref{3.1}.\footnote{A reviewer wonders whether the PF requirement of verb-framed languages can be argued to apply to phonologically null functional heads in general in these languages. In the remainder of this presentation I continue to focus on the functional head involved in the argument structure of verbal predicates, leaving the exploration of this hypothesis to further research.} In Figure \ref{sim:fig:sccc2b} and Figure \ref{sim:fig:blab2}, illustrating the syntactic structures of the verb-framed resultative predicate in \REF{sccc2} and the verb-framed creation predicate in \REF{blab}, respectively, I represent the PF requirement on the \textit{v} head by means of an index \textit{[i]} which is deleted when the requirement is satisfied.

\begin{figure}[ht]
    \begin{forest}
    for tree={s sep=1cm, inner sep=0, l=0}
    [vP [v\textsuperscript{\st{\textit{[i]}}} [Pred [\textsc{entr-}] [Pred] ] [v\textsuperscript{\textit{[i]}}] ] [PredP [DP [la botella] ] [Pred$'$ [\st{Pred}] [\st{\textsc{entr-}}] ] ] ]
    \end{forest}
\caption{Syntactic structure of \REF{sccc2} (with a visual representation of the PF requirement (\textit{[i]}) on \textit{v})}
    \label{sim:fig:sccc2b}
\end{figure}

\begin{figure}[ht]
    \begin{forest}
    for tree={s sep=1cm, inner sep=0, l=0}
    [vP [v\textsuperscript{\st{\textit{[i]}}} [\textsc{dig}] [v\textsuperscript{\textit{[i]}}] ] [\st{\textsc{dig}}] ]
    \end{forest}
\caption{Syntactic structure of \REF{blab} (with a visual representation of the PF requirement (\textit{[i]}) on \textit{v})}
    \label{sim:fig:blab2}
\end{figure}

In the present account, the absence of the operation conflating a root with \textit{v} in verb-framed languages arises as a by-product of \textit{v}'s PF requirement. No parameterization of specific syntactic operations thus needs to be invoked. Verb-framed languages give the impression of lacking the operation conflating a root with \textit{v}, because the syntactic configuration produced by such an operation is incompatible with the morphophonological context needed for the incorporation of \textit{v}'s complement into \textit{v} at PF in these languages. The syntactic representations in Figure \ref{sim:fig:fregalimpia} and Figure \ref{sim:fig:sonrie}, corresponding to the Spanish ungrammatical %literal translation, in verb-framed Spanish, of the 
satellite-framed resultative predicate in \REF{fregalimpia} and %the 
satellite-framed creation/consumption predicate in \REF{sonrie}, respectively, illustrate this.\footnote{\REF{fregalimpia} is grammatical in Spanish in the irrelevant readings involving a depictive or attributive interpretation of the AP (Jaume Mateu, p.c.).%, as these readings do not involve a satellite-framed strategy (the verb in these cases not expressing a co-event).
} In the case of \REF{fregalimpia}, ungrammaticality arises because neither the AP \textit{limpia} `clean' nor its root \textsc{limp-} can function as prefixes of verbs in Spanish, whereby the fulfillment of \textit{v}'s PF requirement would give rise to an unpronounceable sequence of morphemes.\footnote{In Distributed Morphology terms, one could formalize the context of insertion of the Vocabulary Item associated with \textsc{limp-} as requiring that no roots intervene between \textsc{limp-} and \textit{v}.} %According to the present proposal, 
Similarly, the ungrammaticality of \REF{sonrie} is due to the DP complement of \textit{v} (\textit{su agradecimiento} `his thanks') not being able to incorporate onto \textit{v}, which leaves the PF requirement on \textit{v} unsatisfied.\footnote{See \citet{MartinezVazquez2014} for the claim that, to a certain extent, complex creation/con\-sump\-tion predicates can be found in verb-framed Spanish. Further see \citet{BigolinAndAusensi2021} for an analysis of the examples in \citet{MartinezVazquez2014} as involving a verb-framed strategy.}

\ea[*] {\gll Él fregó la mesa limpia. \label{fregalimpia}\\
he wipe.{\PST.\AGR} the table clean\\
\glt `He wiped the table clean.' \hfill (\citealt[519]{BigolinAndAusensi2021})} \z

\begin{figure}[ht]
    \begin{forest}
    for tree={s sep=1cm, inner sep=0, l=0}
    [vP [v\textsuperscript{!\textit{[i]}} [\textsc{freg-}] [v\textsuperscript{\textit{[i]}}] ] [PredP [DP [la mesa] ] [Pred$'$ [Pred] [\textsc{limp-} / AP\textsubscript{(limpia)} ] ] ] ]
    \end{forest}
\caption{Syntactic structure of \REF{fregalimpia} (with a visual representation of the PF requirement (\textit{[i]}) on \textit{v})}
    \label{sim:fig:fregalimpia}
\end{figure}

\ea[*]  {\gll Juan sonríe su agradecimiento. \label{sonrie}\\
Juan smile.{\PRS.\AGR} \textsc{poss} gratitude\\
\glt `Juan smiles his thanks.' \hfill (\citealt[527]{BigolinAndAusensi2021})} \z

\begin{figure}[ht]
    \begin{forest}
    for tree={s sep=1cm, inner sep=0, l=0}
    [vP [v\textsuperscript{!\textit{[i]}} [\textsc{sonr-}] [v\textsuperscript{\textit{[i]}}] ] [DP [su agradecimiento] ] ]
    \end{forest}
\caption{Syntactic structure of \REF{sonrie} (with a visual representation of the PF requirement (\textit{[i]}) on \textit{v})}
    \label{sim:fig:sonrie}
\end{figure}

I propose that the requirement in \REF{PFrequirement}, found in verb-framed languages, is also responsible for the pattern illustrated in \REF{ispisala} concerning Slavic languages (and weak satellite-framed languages in general), in which the result component of resultative predicates with manner-denoting verbs must form a prosodic word with the verb (\citealt{Talmy2000}; \citealt{Acedo-Matellan2010}; \citeyear{Acedo-Matellan2016}). That is to say, I propose that weak satellite-framed languages are fundamentally verb-framed languages. The reasoning goes as follows. As the account of \REF{fregalimpia} and \REF{sonrie} shows, in addition to PF requirements on functional heads of the type in \REF{PFrequirement}, a relevant factor in determining the availability of specific constructions in a given language is whether or not the constructions in question can be spelled out consistent with the PF restrictions on the individual items that make up the lexical inventory of the language concerned. For instance, I have argued that in Spanish (and, more generally, in verb-framed languages), there are no instances of constructions involving the conflation of a root with \textit{v} because such a syntactic configuration prevents the fulfillment of \textit{v}'s requirement that it incorporate its complement, as the lexical inventory of Spanish does not contain morphemes capable of expressing a Talmian Path in the form of a verbal prefix. I argue that weak satellite-framed languages differ from standard verb-framed languages in the domain of resultative predicates in that their lexicon has result-denoting morphemes which can be realized as verbal prefixes (e.g., \textit{iz-} `out' in \REF{ispisala}). The prefixal nature of such morphemes may satisfy \textit{v}'s requirement that it incorporates its complement, by concomitantly leaving open the possibility of conflating an independent root with \textit{v}.\footnote{The pattern is discussed by \citet{Mateu2017}, who, however, continues to consider Slavic languages (and weak satellite-framed languages in general) as fundamentally satellite-framed languages. The parallelism between prefixed resultative predicates with manner-denoting verbs of Slavic languages and English satellite-framed resultative constructions is proposed in \citet{Spencerandzaretskaya1998} and \citet{Mateu2008}. See \citet{Snyder2012} for the claim that Russian patterns with verb-framed languages with respect to The Compounding Parameter of \citeauthor{Snyder1995} (\citeyear{Snyder1995}; \citeyear{Snyder2001}). Russian and Czech have been argued to be verb-framed languages also by \citet{big:Gehrke2008}, who relates their verb-framedness to their realizing accomplishment structures in the verb (also by means of prefixes). I come back to \citeposst{big:Gehrke2008} proposal in \sectref{4.3}, where I elaborate on its potential relevance %for a qualification of 
for the data dealt with in the present chapter. %light of the present analysis of creation/consumption predicates in Slavic languages.
} 
This gives rise to a satellite-framed behavior in the domain of resultative predicates, as noted in \citet{Talmy2000} and further discussed in \citet{Acedo-Matellan2016}. %\footnote{An anonymous reviewer asks whether the present account of Talmy’s typology predicts that Path prefixes or other Path morphemes can be borrowed into bona fide verb-framed languages. According to the present analysis, the only kind of Path morphemes that can give rise to predicates involving conflation in verb-framed languages are those whose phonological context of insertion allows them to appear as verbal prefixes. A prediction of the analysis is thus that the borrowing of a result morpheme of this kind in a verb-framed language serves as a sufficient condition to license predicates involving conflation in such a language. I leave the investigation of this prediction to future research.} 
The structure in Figure \REF{sim:fig:ispisalab}, corresponding to the Russian predicate in \REF{ispisala} (repeated here as \REF{ispisalab}), illustrates this.

\ea \gll Ona iz-pis-a-l-a svoju ručku.\\
she.{\NOM} out-write-\textsc{th}-{\PST-\AGR} {\POSS} pen.{\ACC}\\
\glt `Her pen has run out of ink.' (Lit. `She has written her pen out (of ink).') \label{ispisalab}
\hfill (Russian; \citealt[17]{Spencerandzaretskaya1998}) \z

\begin{figure}[ht]
    \begin{forest}
    for tree={s sep=1cm, inner sep=0, l=0}
    [vP [v\textsuperscript{\st{\textit{[i]}}} [Pred [\textsc{iz-}] [Pred] ] [v\textsuperscript{\textit{[i]}} [\textsc{pis-}] [v\textsuperscript{\textit{[i]}}] ] ] [PredP [DP [svoju ručku] ] [Pred$'$ [\st{Pred}] [\st{\textsc{iz-}}] ] ] ]
    \end{forest}
\caption{Syntactic structure of \REF{ispisalab} (with a visual representation of the PF requirement (\textit{[i]}) on \textit{v})}
    \label{sim:fig:ispisalab}
\end{figure}

From the hypothesis that weak satellite-framed languages are actually verb-framed languages (in the sense of \REF{PFrequirement}), it also follows that such languages should display a clear verb-framed behavior in the domain of creation/consumption predicates. No prefixal morpheme capable of referring to the object of creation/ consumption predicates is present in the lexicon of these languages, whereby only creation/consumption predicates that involve the incorporation of a root into \textit{v} can be licensed, in addition to predicates involving overt light verbs such as \textit{do} or \textit{make} (e.g., \REF{overtlightverb}). %\footnote{I assume \citeposst{Harley2005} conception of incremental themes as objects denoting entities whose mereological parts map onto the extension of the event denoted by the predicate, measuring it out. In predicates with incremental theme objects, the boundedness of the direct object determines the telicity of the predicate. This is illustrated for English in \REF{incrementalthemeharley}. ADD SPACE:
%\ea \label{incrementalthemeharley} \ea Sue drank/wrote \hfill for hours/$\#$in five minutes.
%\ex Sue drank a pint of beer/wrote a story \hfill $\#$for hours/in five minutes.
%\ex Sue drank beer/wrote stories \hfill for hours/$\#$in five minutes. \\
%\hfill (\citealt{Harley2005}: 43) \z \z ADD SPACE
%} 
See this in Figure \ref{sim:fig:Rubrushbbb}, where the syntactic representation of the ungrammatical Russian predicate in \REF{Rubrush} (repeated in \REF{Rubrushbbb}) is provided.

\ea[*] {\gll  Ona čes-a-l-a dyrku v pal'to.\\
she.{\NOM} brush.\textsc{ipfv}-\textsc{th}-{\PST-\AGR} hole.{\ACC} in coat.{\LOC} \label{Rubrushbbb}\\ \hfill (Russian)}
\glt `She was brushing a hole in her coat.'
\z

\begin{figure}[ht]
    \begin{forest}
    for tree={s sep=1cm, inner sep=0, l=0}
    [vP [v\textsuperscript{!\textit{[i]}} [\textsc{čes-}] [v\textsuperscript{\textit{[i]}}] ] [DP [dyrku] ] ]
    \end{forest}
\caption{Syntactic structure of \REF{Rubrushbbb} (with a visual representation of the PF requirement (\textit{[i]}) on \textit{v})}
    \label{sim:fig:Rubrushbbb}
\end{figure}

The present account provides a solution to the minimalist conundrum whereby verb-framed languages seem to lack a structure-building operation (that of conflating a root with the \textit{v} head) which is instead available in satellite-framed languages (see discussion in \citealt{FolliAndHarley2020}). In present terms, the resultative predicates with manner-denoting verbs and a prefixal result found in weak satellite-framed languages like Slavic languages are precisely to be regarded as constructions where a root is conflated with \textit{v} in a verb-framed system.

\subsection{A comparison with some predecessors}\label{3.3}

Previous neo-constructionist approaches to Talmy's typology emphasize either that verb-framed languages always express the Path component in the main verb (\citealt{Acedo-MatellanandMateu2013}; \citealt{Acedo-Matellan2016}; \citealt{FolliAndHarley2020}, among others) or that verb-framed languages lack predicates where, more generally, the verb expresses a manner co-event (\citealt{Mateu2012}). The former approach runs into problems when considering that verb-framed languages and weak satel\-lite-framed languages do not display complex creation/consumption predicates of the type displayed by satellite-framed languages, as nothing in this approach precludes the realization of such predicates -- where no result component is involved -- in these languages. Put differently, complex creation/consumption predicates are predicted to be universally available by this approach, contrary to facts.\footnote{Aware of this prediction, \citet{FolliAndHarley2020} argue that complex creation/consumption predicates indeed do not give rise to cross-linguistic variation related to Talmy's typology and are generally available in verb-framed languages. This fact would then constitute the empirical proof that the expression of a co-event in the main verb is a universally available linguistic process. Specifically, \citet{FolliAndHarley2020} note that creation/consumption predicates such as \REF{weave}, \REF{paint}, and \REF{carve} are licensed both in satellite-framed English and in verb-framed Italian, and they assume that these predicates involve the expression of a manner co-event in the verb, similar to what is observed in satellite-framed resultative predicates. %Aware of this prediction, \citet{FolliAndHarley2020} argue that complex creation/consumption predicates are indeed unaffected by Talmy's typology and generally available in verb-framed languages, this fact constituting the empirical proof that the conflation of a root with \textit{v} is a universally attested syntactic building process. Specifically, \citet{FolliAndHarley2020} note that creation/consumption predicates such as \REF{weave}, \REF{paint}, and \REF{carve} are licensed both in satellite-framed English and in verb-framed Italian, and they assume that these predicates involve the conflation of a root with \textit{v}, the verb expressing a co-event of the main event of creation.
However, as shown in Table \ref{sim:tab:vfcc}, that these specific examples do not give rise to significant cross-linguistic variation cannot be taken to conclude that no typological variation exists in the domain of creation/consumption predicates. Namely, the examples in \citet{FolliAndHarley2020} can be taken to involve verbs whose conceptual meaning implies the creation of the direct object, which in turn is interpreted as a hyponym of the verb (in the sense of \citealt{HaleAndKeyser1997b}; \citeyear{HaleAndKeyser2002}). As such, they can be argued to involve the verb-framed incorporation pattern exemplified in Figure \ref{sim:fig:blab}, whereby they are allowed in verb-framed Italian.} The latter approach correctly predicts the unavailability of complex creation/consumption predicates in verb-framed languages, but it also predicts that weak satellite-framed languages should behave on a par with standard satellite-framed languages in allowing complex creation/consumption predicates. Furthermore, as discussed in \sectref{3.1}, the generalization provided by this approach can only be taken as a descriptive one, as it cannot itself be considered explanatory without entailing a conception of syntax as a locus of parametric variation.


\section{The role of perfectivizers}\label{4}

%\subsection{Internal and external prefixes}\label{4.1}
\subsection{Internal prefixes and events of creation/consumption}\label{4.2}

In Slavic languages, the contrast between the imperfective and the perfective aspectual viewpoints is typically achieved by means of verbal prefixation and suffixation. In a standard case, basic verbal stems have an imperfective reading, which is turned perfective via the addition of a prefix. The Russian examples in \REF{Ruprefixsmith} illustrate this.

\ea \label{Ruprefixsmith} \ea \gll My pis-a-l-i pis'mo.  \\ 
we.{\NOM} write.\textsc{ipfv}-\textsc{th}-{\PST}-{\AGR} letter.{\ACC} \\
\glt `We were writing a letter.'  \hfill (Russian; \citealt[302]{Smith1991})
\ex \label{Runaprefix} \gll On na-pis-a-l pis'mo. \\
he.{\NOM} \textsc{pfv}-write-\textsc{th}-{\PST} letter.{\ACC} \\
\glt `He wrote a letter.'  \hfill (Russian; \citealt[301]{Smith1991}) \z \z

\noindent Normally, the perfective prefix comes from the same inventory of morphemes which can provide the Talmian Path component in resultative predicates. Indeed, it has been argued that prefixes of this type -- hereafter referred to as ``internal'' prefixes -- denote the incorporation of a non-referential result into the verb, in a resultative structure (\citealt{RamchandAndSvenonius2002}; \citealt{big:Gehrke2008}; \citealt{Acedo-Matellan2016}; \citealt{big:Kwapiszewski2022}, among others).\footnote{Internal (or ``lexical'') prefixes are contrasted with external (or ``superlexical'') ones. The distinction is motivated by a series of factors which point toward the idea that internal prefixes are merged inside the vP (hence the name), while external prefixes are merged higher in the functional spine of the clause. For discussion of the distinction between internal and external verbal prefixes in Slavic languages, see \citet{BabkoMalaya1999}; \citet{big:Romanova2004}; \citet{big:Svenonius2004}; \citet{big:Borik2006}; \citeauthor{Arsenijević2006} (\citeyear{Arsenijević2006}, \citeyear{Arsenijević2007}); \citet{big:Gehrke2008}; \citet{Zaucer2009}; \citet{big:Lazorczyk2010}; \citet{big:Tatevosov2011}; \citet{big:Milosavljevic2022}; \citet{big:Kwapiszewski2022}, among many others. The classification of Slavic prefixes has also been argued to be more nuanced than the traditional bi-partite division found in the literature. For instance, \citet{big:Tatevosov2008} argued that in Russian there exists a class of prefixes (e.g., \textit{do-} and \textit{pere-}) that exhibit an intermediate behavior between internal and external prefixes. %Prefixes of this type discussed are \textit{do-} and \textit{pere-}. 
Since the examples from Russian collected in this study do not involve such prefixes, I do not pursue this issue further here. %I am grateful to an anonymous reviewer for pointing out the work of \citet{big:Tatevosov2008} to me.
} %For instance, internal prefixes attach to verb stems, while external prefixes can attach to already prefixed verbs. Thus, when an internal prefix co-occurs with an external one, the external prefix mandatorily precedes the internal prefix. %The contrast from Polish in \REF{internalafter} illustrates this.
%
%\ea \label{internalafter} \ea[*] {\gll roz(\textsc{int})-po(\textsc{ext})-ład-ow-yw-a-ć\\
%apart-\textsc{dist}-load-v-\textsc{si}-\textsc{th}-\textsc{inf}\\}
%\ex \gll po(\textsc{ext})-roz(\textsc{int})-ład-ow-yw-a-ć\\
%\textsc{dist}-apart-load-v-\textsc{si}-\textsc{th}-\textsc{inf}\\\glt `To unload one by one' \\
% \hfill (Polish; \citealt{big:Kwapiszewski2022}: 108) \z \z
%\noindent Additionally, while external prefixes can stack on top of each other, there can be only one internal prefix per predicate.
%\ea \ea \gll Studenci po(\textsc{ext})-na(\textsc{ext})-za(\textsc{int})-prasz-a(j)-l-i się tego nauczyciela.\\
%students.\textsc{nom} \textsc{dist}-\textsc{sat}-behind-ask-\textsc{th}-\textsc{pst}-\textsc{agr} \textsc{refl} this teacher.\textsc{gen}\\
%\glt `The students each did a lot of inviting of this teacher.' \\\hfill (Polish; \citealt[201]{big:Lazorczyk2010})\\
%\ex[*] {\gll Student wy(\textsc{int})-prze(\textsc{int})-pis-yw-a-ł notatki.\\
%student.\textsc{nom} out-through-write-%\textsc{si}-\textsc{th}-\textsc{pst} notes.\textsc{acc}\\ \glt `The student was coping out the notes.' \\\hfill (Polish; \citealt[199]{big:Lazorczyk2010})}
%\ex[*] {\gll od(\textsc{int})-u(\textsc{int})-stoupit\\
%from-away-step.\textsc{inf}\\} \hfill (Czech; \citealt[170]{big:Gehrke2008}) 
%\z \z
%Another difference between internal and external prefixes is that internal prefixes, in contrast to external prefixes, can alter the argument structure of the predicate, licensing arguments whose presence hinges on that of the prefix. In contrast, external prefixes have no effect on the argument structure of the verb. %Consider again the predicates in \REF{Ruprefixsmith}. While the verb \textit{write} in Russian allows the omission of the object (\REF{writeunerg}), the presence of the internal prefix makes the predicate mandatorily transitive (\REF{writetrans}).
%
%\ea \ea \gll Ivan pis-a-l (pis'mo).\\
%Ivan write.\textsc{ipfv}-\textsc{th}-%\textsc{pst} letter.\textsc{acc}\\
%\glt `Ivan was writing (a letter).' \label{writeunerg}\\
%\ex \gll Ivan na(\textsc{int})-pis-a-l *(pis'mo).\\
%Ivan \textsc{pfv}-write-\textsc{th}-\textsc{pst} letter\\
%\glt `Ivan wrote a letter.' \label{writetrans}\\ \hfill (Russian; \citealt[18]{BabkoMalaya1999}) \z \z
%
%\noindent In contrast, external prefixes have no effect on the valency of the verb.
%
%\ea \gll Po(\textsc{ext})-pis-a-t (pis'mo).\\
%\textsc{po}-write.\textsc{ipfv}-\textsc{th}-\textsc{inf} letter.\textsc{acc}\\
%\glt `To write (a letter)' \hfill (Russian; \citealt[162]{big:Gehrke2008}) \z
%
%Semantically, it has further been argued that only internal prefixes can induce inner-aspectual telicity (\citealt{big:Gehrke2008}). Finally, it is relevant to notice that only internal prefixes may trigger special meanings of the stem they attach to (\citealt{Arsenijević2006}). On the other hand, external prefixes tend to simply modify the eventuality denoted by the predicate. This squares nicely with \citeauthor{Marantz1984}'s (\citeyear{Marantz1984}, \citeyear{Marantz1995}) observation that idiomatic meanings of verbs are triggered by the internal arguments of the predicate.%\footnote{For further discussion of the distinction between internal and external verbal prefixes in Slavic languages, see \citet{BabkoMalaya1999}; \citet{big:Romanova2004}; \citet{big:Svenonius2004}; \citet{big:Borik2006}; \citeauthor{Arsenijević2006} (\citeyear{Arsenijević2006}, \citeyear{Arsenijević2007}); \citet{big:Gehrke2008}; \citet{Zaucer2009}; \citet{big:Lazorczyk2010}; \citet{big:Tatevosov2011}; \citet{big:Milosavljevic2022}; \citet{big:Kwapiszewski2022}, among many others.}

%\ea \label{bitiidiomatic} \ea \gll biti\\
%beat.\textsc{inf}\\
%\glt `To beat'
%\ex \gll u(\textsc{int})-biti\\
%in-beat.\textsc{inf}\\
%\glt `To kill'
%\ex \gll raz(\textsc{int})-biti\\ around-beat.\textsc{inf}\\\glt `To break'\\ \hfill (Serbo-Croatian; \citealt[211]{Arsenijević2006}) \z \z

%\ea \label{kuvatinoidiomatic} \ea \gll kuvati\\
%cook.\textsc{inf}\\
%\glt `To cook'
%\ex \gll na(\textsc{ext})-kuvati\\
%on-cook\\
%\glt `To cook many'
%\ex \gll pre(\textsc{ext})-kuvati\\
%over-cook\\
%\glt `To overcook' \\\hfill (Serbo-Croatian; \citealt[211]{Arsenijević2006}) \z \z

%\noindentFor further discussion of the distinction between internal and external verbal prefixes in Slavic languages, see \citet{BabkoMalaya1999}; \citet{big:Romanova2004}; \citet{big:Svenonius2004}; \citet{big:Borik2006}; \citeauthor{Arsenijević2006} (\citeyear{Arsenijević2006}, \citeyear{Arsenijević2007}); \citet{big:Gehrke2008}; \citet{Zaucer2009}; \citet{big:Lazorczyk2010}; \citet{big:Tatevosov2011}; \citet{big:Milosavljevic2022}; \citet{big:Kwapiszewski2022}, among many others.

%\subsection{Internal prefixes and events of creation/consumption}\label{4.2}

In the present framework, %based on the considerations pointed out above, 
Slavic predicates perfectivized via internal prefixes (such as the Russian one in \REF{Runaprefix}) are thus attributed the syntactic structure in Figure \ref{sim:fig:Runaprefix}. I assume that predicates depicting events of creation/consumption made perfective via internal prefixes consistently involve the argument structure that is found in resultative (change of state/location) predicates, the direct object being interpreted as a created or consumed entity due to pragmatic factors arising from the conceptual interpretation of the construction. Insofar as these predicates involve the incorporation of \textit{v}'s complement into \textit{v}, as shown in Figure \ref{sim:fig:Runaprefix}, they are predicted to be possible in Slavic languages in the same way as resultative predicates with manner-denoting verbs are, the incorporation of the prefix fulfilling the verb-framed requirement of the language as understood in \REF{PFrequirement}. In what follows, I present the results of a study exploring the validity of such a prediction.

\begin{figure}[t]
    \begin{forest}
    for tree={s sep=1cm, inner sep=0, l=0}
    [vP [v [Pred [\textsc{na-}] [Pred] ] [v [\textsc{pis-}] [v] ] ] [PredP [DP [pis'mo] ] [Pred$'$ [\st{Pred}] [\st{\textsc{na-}}] ] ] ]
    \end{forest}
\caption{Syntactic structure of \REF{Runaprefix}}
    \label{sim:fig:Runaprefix}
\end{figure}

In order to verify the prediction, I have conducted the same test run for bona fide creation/consumption predicates which was described in \sectref{2}. This time, however, the English examples have been left in their non-progressive form, to check wheth\-er the presence of the perfective prefixes in their Slavic counterparts affects the grammaticality of their literal translation in the Slavic languages. The results obtained, summarized in Table \ref{sim:tab:slavicccpfv}, show that Slavic languages clearly behave on a par with satellite-framed languages (cf. Table \ref{sim:tab:sfcc}) when a perfective prefix is present, confirming the prediction.\footnote{The native speakers of Serbian and Croatian seem more conservative than the native speakers of the other Slavic languages tested in disallowing a creation/consumption reading for several of the predicates involved. At the moment, I am agnostic as to why the pattern displayed by Serbian and Croatian in this test differs in this way from that of the other Slavic languages.} The grammatical renditions of the English example in \REF{brushhole2} in Russian, Ukrainian, and Polish illustrate this (see \REF{brushholeslavic}). The structure for the Russian example in \REF{Rubrush3}, which is understood to hold also for the rest of the data, is provided in Figure \ref{sim:fig:brushholeslavic}.


\begin{table}[b]
\caption{Perfective predicates with creation/consumption reading in Slavic languages (prefixed predicates)}
\label{sim:tab:slavicccpfv}\resizebox{10.5cm}{!}{
 \begin{tabularx}{1\textwidth}{lYYYYYY} %.77 indicates that the table will take up 77% of the textwidth
  \lsptoprule
        Example    & Rus & Ukr  & Pol & Slo & Ser & Hrv \\
  \midrule
  (6) John sang a song     & \footnotesize\Checkmark   &  \footnotesize\Checkmark   &  \footnotesize\Checkmark   &  \footnotesize\Checkmark   &  \footnotesize\Checkmark   &    \footnotesize\Checkmark \\
  (7) They danced a Sligo jig    &    \footnotesize\Checkmark &  \footnotesize\Checkmark   &  \footnotesize\Checkmark   &  \footnotesize\Checkmark   &  \footnotesize\Checkmark   & \footnotesize\Checkmark    \\
  (8) Ariel ate the mango    &  \footnotesize\Checkmark   &  \footnotesize\Checkmark   &  \footnotesize\Checkmark   &  \footnotesize\Checkmark   &   \footnotesize\Checkmark  & \footnotesize\Checkmark    \\
  (9) He dug a hole in the ground    &    \footnotesize\Checkmark & \footnotesize\Checkmark    &  \footnotesize\Checkmark   &  \footnotesize\Checkmark   &  \footnotesize\Checkmark   &  \footnotesize\Checkmark   \\
  (10) She wove the tablecloth    &  \footnotesize\Checkmark   &   \footnotesize\Checkmark  &  \footnotesize\Checkmark   &  \footnotesize\Checkmark   &  \footnotesize\Checkmark   &  \footnotesize\Checkmark   \\
  (11) Marco painted a sky    &  \footnotesize\Checkmark   &  \footnotesize\Checkmark   & \footnotesize\Checkmark    &  \footnotesize\Checkmark   &   \footnotesize\Checkmark  &  \footnotesize\Checkmark   \\
  (12) Maria carved a doll    &   \footnotesize\Checkmark  &   \footnotesize\Checkmark  &  \footnotesize\Checkmark   &   \footnotesize\Checkmark  &  \footnotesize\Checkmark   &   \footnotesize\Checkmark  \\
  (13) She burned a hole in her coat    &  \footnotesize\Checkmark   &   \footnotesize\Checkmark  &  \footnotesize\Checkmark   &  \footnotesize\Checkmark   &   ??  &   \footnotesize\Checkmark  \\
  (14) He scratched a hole in the ground    &  \footnotesize\Checkmark   &  \footnotesize\Checkmark   &  \footnotesize\Checkmark   &  \footnotesize\Checkmark   &  ??   &  \scriptsize\FiveStar  \\
  (15) She punctured a wound in her finger    &  \footnotesize\Checkmark   &   \footnotesize\Checkmark  &   \footnotesize\Checkmark  &  \footnotesize\Checkmark   &  ??   & \scriptsize\FiveStar   \\
  (16) She cut a wound in her foot    &  \footnotesize\Checkmark   &  ?   &   \footnotesize\Checkmark  & \footnotesize\Checkmark    & \scriptsize\FiveStar   & \scriptsize\FiveStar   \\
  (17) She bit a hole in the bag    &   \footnotesize\Checkmark  &   \footnotesize\Checkmark  &  \footnotesize\Checkmark   &   \footnotesize\Checkmark  &   \footnotesize\Checkmark  &  \scriptsize\FiveStar  \\
  (18) The adventurer walked the trail    &  \footnotesize\Checkmark   &   \footnotesize\Checkmark  & \footnotesize\Checkmark    &   \footnotesize\Checkmark  & \footnotesize\Checkmark    & \footnotesize\Checkmark    \\
  (19) The adventurer swam the channel    &   \footnotesize\Checkmark  & \footnotesize\Checkmark    &   \footnotesize\Checkmark  &  \footnotesize\Checkmark   &  \footnotesize\Checkmark   &  \footnotesize\Checkmark   \\
  (20) Deanne kicked a hole in the wall    & \footnotesize\Checkmark    &  \footnotesize\Checkmark   &   \footnotesize\Checkmark  &   \footnotesize\Checkmark  &  ??   & \scriptsize\FiveStar   \\
  (21) She magicked a cursor    &   \footnotesize\Checkmark  &  \footnotesize\Checkmark   &  \footnotesize\Checkmark   &   \footnotesize\Checkmark  &  ??   &  \footnotesize\Checkmark   \\
  (22) She brushed a hole in her coat    &  \footnotesize\Checkmark   &   \footnotesize\Checkmark  &  \footnotesize\Checkmark   &   \footnotesize\Checkmark  &  ?   & \scriptsize\FiveStar   \\
  (23) John smiled his thanks    &     &     &     &     &     &     \\
  (24) Elna frowned her discomfort    &     &     &     &     &     &     \\
  \lspbottomrule
 \end{tabularx}}
\end{table}

\ea \label{brushholeslavic} \ea \gll Ona pro-čes-a-l-a dyrku v pal'to.\\
she.{\NOM} \textsc{pfv}-brush-\textsc{th}-{\PST-\AGR} hole.{\ACC} in coat.{\LOC} \label{Rubrush3}\\ \hfill (Russian)
\ex \gll Vona pro-ter-l-a dyrku na kurtci.\\
she.{\NOM} \textsc{pfv}-brush-{\PST-\AGR} hole.{\ACC} in coat.{\LOC} \label{Ukbrush3}\\ \hfill (Ukrainian)
\ex \gll Ona wy-czes-a-ł-a dziurę w płaszczu.\\
she.{\NOM} \textsc{pfv}-brush-\textsc{th}-{\PST-\AGR} hole.{\ACC} in coat.{\LOC} \label{Pobrush3}\\ \hfill (Polish) \\
\z
\glt `She brushed a hole in her coat.' \z

\begin{figure}[ht]
    \begin{forest}
    for tree={s sep=1cm, inner sep=0, l=0}
    [vP [v\textsuperscript{\st{\textit{[i]}}} [Pred [\textsc{pro-}] [Pred] ] [v\textsuperscript{\textit{[i]}} [\textsc{čes-}] [v\textsuperscript{\textit{[i]}}] ] ] [PredP [DP [dyrku] ] [Pred$'$ [\st{Pred}] [\st{\textsc{pro-}}] ] ] ]
    \end{forest}
\caption{Syntactic structure of \REF{Rubrush3} (with a visual representation of the PF requirement (\textit{[i]}) on \textit{v})}
    \label{sim:fig:brushholeslavic}
\end{figure}

The contrast in acceptability between predicates with unprefixed verbs and predicates with prefixed verbs in the expression of complex events of creation/ consumption in Slavic languages (compare Table \ref{sim:tab:slavicccipfv} with Table \ref{sim:tab:slavicccpfv}) cannot be argued to depend on the aspectual shift from the imperfective reading of the former type of predicates to the perfective reading of the latter type of predicates. This is proved by the availability, for the examples that are ungrammatical in the imperfective reading provided by unprefixed verbs, of imperfective predicates obtained via secondary imperfectivization. Secondary imperfectivization is a strategy found in Slavic languages whereby a prefixed, perfective verb is turned into an imperfective reading by means, typically (although not necessarily), of a further process of affixation (\citealt{BabkoMalaya1999}; \citealt{big:Romanova2004}; \citealt{big:Svenonius2004}; \citealt{big:Kwapiszewski2022}, among others). In the examples under consideration, secondary imperfectivization gives rise to grammatical predicates also in those cases where an imperfective reading involving unprefixed verbs gives rise to ungrammaticality. This is illustrated in \REF{Ukrukr3} with the Ukrainian translations of \REF{kick}, which is unavailable in the imperfective unprefixed version \REF{imperf1} but is grammatical both in the perfective prefixed version \REF{perf1} and in the imperfective prefixed version obtained via secondary imperfectivization \REF{imperf2}.

\ea \label{Ukrukr3} \ea[*] {\gll Din   byv      dyru    u stini.\\  
Din kick.\textsc{ipfv}.{\PST} hole.{\ACC} in wall.{\LOC} \label{imperf1}\\\hfill (Ukrainian)
\glt Intended: `Din was kicking a hole in the wall.'}
\ex \gll Din pro-byv      dyru  u  stini. \label{perf1}\\
Din \textsc{pfv}-kick.{\PST} hole.{\ACC} in wall.{\LOC}\\
\glt `Din kicked a hole in the wall.'
\ex \gll Din    pro-byv-av      dyru  u  stini.\\
Din \textsc{pro}-kick.{\PST}-\textsc{ipfv} hole.{\ACC} in wall.{\LOC} \label{imperf2}\\
\glt `Din was kicking a hole in the wall.' \z \z

\noindent These facts suggest that the predicate's grammaticality does not rely on the % what is relevant for the grammaticality of the predicate is not the 
perfective reading, but on the presence of the prefix, which fulfills the verb-framed requirement of the language by incorporating into \textit{v} from its complement.%\footnote{The interaction between aspect and Talmy’s typology deserves to be investigated further. The results presented in this chapter form part of an ongoing doctoral dissertation, where such an interaction is explored in greater depth.}

\subsection{Incrementality in complex predicates}\label{4.3}

The meaning contribution of internal prefixes in the licensing of complex predicates in Slavic languages warrants further investigation. For instance, \citet{big:Gehrke2008} posits that complex predicates, in which the main verb denotes an activity, require an accomplishment event structure, which in satellite-framed resultative constructions is licensed by an incremental structure provided by a secondary predicate. She further argues that internal prefixes of Slavic languages derive accomplishment structures, and that in these languages (specifically, she refers to Czech and Russian) accomplishment structures are realized in the verb complex, either by the verb itself or by an internal prefix. \citeauthor{big:Gehrke2008}'s findings may offer an alternative explanation for why complex predicates of creation/consumption are grammatical in Slavic languages only when prefixed. Unprefixed complex predicates of creation/consumption might be infelicitous in Slavic languages due to the absence of an accomplishment structure within the verbal complex. This explanation rests on the assumption that complex predicates, cross-linguistically, require the presence of an accomplishment event structure. However, the idea that an accomplishment event structure is needed to license satellite-framed constructions is not undisputed. For instance, \citet{FolliAndHarley2006} discuss cases of satellite-framed predicates of English, in which a PP denoting an unbounded path appears as the secondary predicate, as in \REF{dancing}. Since the incremental structure associated with PPs of this kind does not have a culmination point, the overall predicate lacks an accomplishment event structure.%Examples of this type, from \citet{FolliAndHarley2006}, are reproduced in \REF{dancing}.
\footnote{The temporal adverbials in the examples in \REF{dancing} show that the PPs in these examples are not understood as referring to a bounded Path. I am not aware of studies concerned with the availability of such examples in Slavic languages. The unavailability of these examples in Polish (Wojciech Lewandowski, p.c.) and in Italian, however, points toward the idea that the satellite-framed/verb-framed division should not be (only) intended as a constraint in the expression of accomplishment structures in some languages and not in others.}

\ea \label{dancing} \ea John waltzed Matilda around and around the room for hours. %\hfill (\citealt[125]{FolliAndHarley2006})
\ex John walked Mary along the river all afternoon. \\\hfill (\citealt[125]{FolliAndHarley2006})
\ex John walked Mary towards her car for 3 hours.\\ \hfill (\citealt[137]{FolliAndHarley2006}) 
\z \z

%\noindent \citet{FolliAndHarley2006} provide independent evidence showing that the PPs in \REF{dancing} are not external adjuncts of the \textit{v}P, but rather behave as internal arguments denoting a secondary predicate in a small clause complement of the verbal head. At the same time, imposing an \textit{ad hoc} requirement on Slavic languages, whereby these languages are required to realize any kind of incremental structure on the verbal complex, is not feasible either, since PPs denoting unbounded paths are phrasal in these languages (\citealt{big:Gehrke2008}).

%Another tentative piece of evidence against the hypothesis that unprefixed complex creation/consumption predicates are not possible in Slavic languages because of the lack of a culmination point in the verb comes from the behavior of perfective unprefixed verbs %when they are forced into a reading as expressing a co-event 
%in complex events of creation/consumption. Verbs of this type could be expected to be able to appear in complex creation/consumption predicates, because they would themselves provide the predicate with the culmination point that by assumption is needed to license satellite-framed constructions. One case in point concerns the Polish verb \textit{strzelić} (`shoot'). This verb gives rise to semelfactive events (argued to behave as achievements, and therefore to be associated with instantaneous transitions; \citealt[127]{big:Gehrke2008}) regardless of whether a prefix is present or not, as shown in the resultative predicate in \REF{shot1}.

%\ea \label{shot1} \gll Strzel-i-ł sobie w nogę.\\
%shoot.\textsc{pfv}-\textsc{th}-\textsc{pst} \textsc{refl} in leg\\
%\glt `He shot himself in the leg.' \hfill (Polish; Web) \z

%\noindent One would thus predict, according to the hypothesis under concern, that this verb can give rise to a complex creation/consumption predicate like the English \REF{shot2}. However, a literal translation of \REF{shot2} is not licensed in Polish with the unprefixed verb \textit{strzelić} (\REF{shot3}). A prefixed verb is needed (Wojciech Lewandowski, p.c.).

%\ea \label{shot2} He shot a hole in the ceiling. \hfill (COCA) \z

%ea \label{shot3} \ea[*] {\gll On strzel-i-ł dziurę w suficie.\\
%he.\textsc{nom} shoot.\textsc{pfv}-\textsc{th}-\textsc{pst} hole.\textsc{acc} in ceiling.\textsc{loc}\\ \label{shot3a}
%\glt `He shot a hole in the ceiling.'}
%\ex \gll On wy-strzel-i-ł dziurę w suficie.\\
%he.\textsc{nom} out-shoot.\textsc{pfv}-\textsc{th}-\textsc{pst} hole.\textsc{acc} in ceiling.\textsc{loc}\\ \label{shot3b} 
%\glt Intended: `He shot a hole in the ceiling.' \z \z

\noindent Another explanation worth considering is that unprefixed complex creation/con\-sump\-tion predicates are not licensed in Slavic languages due to the absence of incrementality in these constructions. I argue that this explanation is not satisfactory either. In these languages, given the right context, predicates of creation/consumption can be telic even if unprefixed (see, e.g., \citealt[179, fn. 41]{big:Gehrke2008}; \citealt{big:Mehlig2012}), the incremental path structure being provided by the direct object (\citealt{RappaportHovav2008, RappaportHovav2014}). For instance, \citet{big:Mehlig2012} argues that such a reading of the direct object in predicates denoting events of creation/consumption is possible in Russian if the extent of the entities denoted by the object has been determined in advance (e.g., from the conversational context) and these entities are referred to in the relevant imperfective predicate by means of a demonstrative (e.g., \textit{{\.e}tot}/\textit{tot} `this/that'). This is illustrated by \citet{big:Mehlig2012} with examples like \REF{banana}, where the consumption predicate \textit{on est {\.e}ti dva banana} `he is eating those two bananas' is successfully modified by the expression \textit{Odin on uže s"el} `He has already eaten one of them', which presupposes that the object has an incremental structure associated with it, because the two conditions listed above are satisfied (see the text preceding the consumption predicate in \REF{banana}, where the amount of \textit{bananas} involved in the eating event is predetermined, and see the presence of the demonstrative \textit{{\.e}ti} `these' in the consumption predicate).

\ea  \label{banana} Segodnja utrom ja dal\textsuperscript{\textsc{pfv}} Saše dva banana. V dannyj moment on est\textsuperscript{\textsc{ipfv}} {\.e}ti dva banana. Odin on uže s"el\textsuperscript{\textsc{pfv}}.\\
\glt `This morning I gave Sasha two bananas. At the moment he is eating those two bananas. He has already eaten one of them.' \\ \hfill (Russian; \citealt[216]{big:Mehlig2012}) \z

\noindent According to the hypothesis under discussion, the complex creation/con\-sump\-tion predicates that gave rise to ungrammaticality in Russian (see Table \ref{sim:tab:slavicccipfv}) %in the test presented in §\ref{2} 
should become acceptable if the contextual conditions identified in \citet{big:Mehlig2012} are met, as the predicates would then be given an incremental structure by the direct object. However, the prediction is not borne out. The same results as those listed in Table \ref{sim:tab:slavicccipfv} are obtained in Russian if the contextual conditions discussed in \citet{big:Mehlig2012} are met, %a literal translation of the English examples in \REF{sang} to \REF{Elnafrowned} in Russian give rise to ungrammaticality in the complex creation/consumption predicates involved (emphasized in italics in \REF{nomore}) in spite of the incremental theme reading of the direct object obtained under the conditions in \citet{big:Mehlig2012} 
as illustrated in \REF{askDaria} with an example based on the predicate in \REF{carve} (Dària Serés, p.c.).

\ea \label{askDaria} \gll Segodnja utrom Deanne zakazali s-delat’ dve reznye kukly. V dannyj  moment \textit{*}\textit{ona} \textit{režet} \textit{{\.e}ti} \textit{dve} \textit{kukly}. Skoree\hspace{3pt}vsego ona uže vy-rez-a-l-a odnu iz nix. \\
today morning Deanne.\textsc{dat} commission.\textsc{pfv}.\textsc{pst}.\textsc{pl} 
\textsc{pfv}-make.\textsc{inf} two carved dolls.\textsc{acc} 
in this moment \hspace{5pt}she.\textsc{nom} carve.\textsc{ipfv}.\textsc{prs} these two dolls.\textsc{acc} probably she.\textsc{nom} already \textsc{pfv}-carve-\textsc{th}-\textsc{pst}-\textsc{agr} one of them \\
\glt `This morning Deanne was commissioned to make two carved dolls. At the moment she is carving those two dolls. She has probably already carved one of them.' \z

%\ea \textit{Russian} \label{nomore} \ea This morning I showed Sasha two trails. At the moment \textit{he is walking those two trails}. He has probably already walked one of them.
%\ex This morning Maria told John she would cut two wounds in his foot. At the moment \textit{she is cutting those two wounds}. She has probably already cut one of them.
%\ex This morning Deanne was ordered two carved dolls. At the moment \textit{she is carving those two dolls}. She has probably already carved one of them. \label{nomorecarved}
%\ex This morning John told Maria he would kick two holes in her garage’s wall. At the moment \textit{he is kicking those two holes}. He has probably already kicked one of them.
%\ex This morning Maria told John she would burn three holes in his coat. At the moment \textit{she is burning those three holes}. She has probably already burned two of them. \z \z

\noindent Similar considerations apply in Slovak, which also seems to license a reading of the object as having an incremental structure associated with it under the conditions in \citet{big:Mehlig2012} but does not allow complex creation/consumption predicates in such contexts (Natália Kolenčíková, p.c.). In the case of Serbian, modifying expressions equivalent to the Russian \textit{Odin on uže s"el} `He has already eaten one of them' in \REF{banana} are compatible with predicates denoting events of consumption regardless of the contextual conditions %pointed out 
in \citet{big:Mehlig2012} (Predrag Kovačević, p.c.), yet complex creation/consumption predicates with unprefixed verbs are not licensed (see Table \ref{sim:tab:slavicccipfv}). %Both 
The ungrammaticality of the predicate in \REF{askDaria} %and the contrast in \REF{shot3} are 
is accounted for by the %morphophonological 
account proposed in the present chapter; the predicate %in \REF{askDaria} 
is not %\REF{shot3b} and not \REF{shot3a} is 
grammatical in Russian %Polish 
because %only \REF{shot3b} displays 
it does not have a verbal prefix which fulfills the language's verb-framed requirement, by incorporating onto the \textit{v} head. %The same reasoning as for the ungrammaticality of the Polish example in \REF{shot3a} applies to the Russian example in \REF{askDaria}. 

In sum, the approach to Talmy's typology proposed in this chapter allows us to account for the kind of cross-linguistic variation related to the typology in the domain of predicates of creation/consumption regardless of whether an incremental structure is provided to the predicate by the verb, by a prefix of the verb or by a phrasal complement of the verb. In the next section, I propose that the present account of Slavic languages as verb-framed languages should be extended to Latin, which was argued to be a weak satellite-framed language, along with Slavic languages, by \citeauthor{Acedo-Matellan2010} (\citeyear{Acedo-Matellan2010}; \citeyear{Acedo-Matellan2016}).


\section{Were there complex creation/consumption predicates in Latin?}\label{5}
 
As observed in \citet{Talmy2000}, and extensively explored in \citeauthor{Acedo-Matellan2010} (\citeyear{Acedo-Matellan2010}; \citeyear{Acedo-Matellan2016}), Latin behaves on a par with Slavic languages in regard to Talmy's typology, in that resultative predicates where the verb denotes a co-event are allowed as long as the Path is expressed by a verbal prefix. As such, Latin is predicted to allow complex creation/consumption predicates by previous neo-constructionist accounts (see \sectref{3.3}). %of the typology, as discussed in \sectref{3.3}. 
Some examples of alleged complex creation predicates from Latin are provided in \citet{Acedo-Matellan2016} to prove this point. This goes against the prediction of the present account of Talmy's typology, according to which neither Latin nor Slavic languages, \textit{qua} weak satellite-framed languages, should be able to license bona fide creation/consumption predicates in the absence of an incorporation process of \textit{v}'s complement onto \textit{v}. Indeed, the data discussed in \citet{Acedo-Matellan2016} are surprising in light of the pattern displayed by Slavic languages in this respect. 
%In this section, I argue that the reasons provided by \citet{Acedo-Matellan2016} for attributing a complex creation/consumption reading to the Latin examples he presents are not convincing. 
%In this section, I argue that the reasons for attributing a complex creation/con- sumption reading to the Latin examples provided in \citet{Acedo-Matellan2016} are not clear. 
In this section, I argue that there is no clear reason for attributing a complex creation/consumption reading to the Latin examples provided in \citet{Acedo-Matellan2016}.
Afterward, I present the results of a corpus search which point toward the conclusion that complex creation/consumption predicates are absent in Latin, in line with the prediction of the present approach.

The examples discussed in \citet{Acedo-Matellan2016} are provided in \REF{Latin} to \REF{Latin5}.

\ea \label{Latin} \gll Qui alteri misceat mulsum.\\
who.\textsc{nom} another.\textsc{dat} mix.\textsc{sbjv}.\textsc{agr} honeyed$\_$wine.\textsc{acc} \label{Latin1}\\
\glt `He who makes honeyed wine for someone else.' \hfill (Latin; \textit{Cic}. Fin. \textit{2, 5, 17}) 
\ex \gll Vulnus [...] quod acu punctum videretur.\\
wound.\textsc{nom} {} which.\textsc{nom} needle.\textsc{abl} puncture.\textsc{ptcp}.\textsc{pfv}.\textsc{nom} seem.\textsc{ipfv}.\textsc{sbjv}.\textsc{agr} \label{Latin2}\\
\glt `A wound that seemed to have been punctured with a needle'.\\\hfill (Latin; \textit{Cic}. Mil. \textit{65})
\ex \gll [Serpens] volubilibus squamosos nexibus
orbes torquet.\\
snake.\textsc{nom} looping.\textsc{abl}.\textsc{pl} scaly.\textsc{acc}.\textsc{pl} writhing.\textsc{abl}.%\textsc{m}.
\textsc{pl} coil.\textsc{acc}.\textsc{pl} twist.{\PRS.\AGR} \label{Latin3}\\
\glt `The snake twists his scaly coils in looping writhings.' \\\hfill  (Latin; \textit{Ov}. Met. \textit{3, 41})
\ex \gll Viam silice sternendam [...] locauerunt.\\
way.\textsc{acc} flint-stone.\textsc{abl} strew.\textsc{ptcp}.\textsc{grdv}.\textsc{acc} {} establish.\textsc{prf}.\textsc{agr} \label{Latin4}\\
\glt `They established that the way was to be paved with flint stone.' \\\hfill (Latin; \textit{Liv. 38, 28, 3})
\ex \gll Aeriam truncis [...] cumulare pyram.\\
high.\textsc{acc} log.\textsc{abl}.\textsc{pl} {}
gather.\textsc{inf} pyre.\textsc{acc} \label{Latin5}\\
\glt `To build a high pyre out of logs.' \hfill (Latin; \textit{Stat.}, Teb. \textit{6, 84}) \z

\noindent I suggest that most of these examples are compatible with a reading as either involving hyponymous objects or displaying resultative predicates of change of state, therefore adopting a verb-framed strategy. For instance, \textit{pyram} `pyre' in \REF{Latin5} could be interpreted as a hyponym of \textit{cumulare} `(lit.) cumulate'. Indeed, a creation reading of this verb is also found in verb-framed Italian, as \REF{cumulare} shows.\footnote{An anonymous reviewer asks me to elaborate on the relevance of the Italian example in \REF{cumulare} for the conclusion that the Latin example in \REF{Latin5} is not a satellite-framed construction. Both the Latin %example in \REF{Latin5} 
and the Italian example %in \REF{cumulare} 
refer to a creation event in which a `cumulation' is formed. As \citet{HaleAndKeyser1997b} noted, the conceptual content of the verb 
in predicates of this kind (in the cases at hand, \textit{cumulare}, meaning `cumulate', or `gather') points non-referentially to the nature of the entity effected during the event (e.g., in \REF{Latin5} and \REF{cumulare}, a `cumulation' of some sort). The object, %(e.g., \textit{pyram} `pyre' in \REF{Latin5} and \textit{esperienza} `experience' in \REF{cumulare}), 
in turn, directly refers to such an effected entity. % of the predicate (namely, a kind of `cumulation'), while the direct object specifies the nature of the cumulation that is created during the event. %the eventAccordingly, as predicated by \citet{HaleAndKeyser1997b,HaleAndKeyser2002}, a non-referential reference to the \textit{effected} object of the event is provided by the verb \textit{cumulare} (`cumulate', `gather'), while the direct object in \REF{cumulare} specifies the nature of the cumulation created during event. 
%Notice that, 
For instance, the predicate in \REF{cumulare} can be paraphrased as `make a gathering that \textit{consists of} experience'. Similarly, the predicate in \REF{Latin5} can be paraphrased as `make a gathering that \textit{consists of} a pyre'. For this reason, direct objects of this type have been referred to in the literature as ``hyponymous arguments'' of the verb. As discussed in \sectref{3}, predicates of this kind have been argued to involve the incorporation into \textit{v} of a root e-merged as the complement of \textit{v} (see Figure \ref{sim:fig:blab}). Thus, they are expected to be well-formed in verb-framed languages. Assuming that Italian is a verb-framed language, %in line with the majority of the literature on Talmy's typology and with the results presented in this study (see Table \ref{sim:tab:vfcc}), 
that the construction in \REF{Latin5} can also be found in Italian, as \REF{cumulare} shows, provides additional evidence to the claim that such a construction is a verb-framed construction, whereby it does not constitute a counterexample to the proposal that Latin should be regarded as a verb-framed language. %under the assumption that Italian is a bona fide verb-framed language. %In light of this, the Latin predicate in \REF{Latin5} can be regarded as a verb-framed construction. The Italian example in \REF{cumulare} supports this conclusion because
} 

\ea \label{cumulare}\gll [...] il primo dovrà aver cumulato esperienza nella grande distribuzione, il secondo sul prodotto e sul contatto con i grandi clienti.\\
{} the first must.{\FUT.\AGR} have.{\INF} gather.\textsc{ptcp}.\textsc{pfv} experience in.the big distribution the second on.the product and on.the contact with the big clients\\
\glt `The first one must have gathered experience in large-scale distribution, the second one on the product and in dealing with large clients.'\\ \hfill (CORIS\footnote{\textit{Corpus di Riferimento dell'Italiano Scritto}, Università di Bologna.}) \z



\noindent As for \REF{Latin2}, the availability of \textit{puncture a wound} (cf. \REF{puncture}) in weak satellite-framed Slavic languages and in verb-framed languages seems to be very limited, but the Ukrainian speaker fully accepts it (Table \ref{sim:tab:slavicccipfv}) and the Basque speaker considers it marginally acceptable (Table \ref{sim:tab:vfcc}), suggesting that %the acceptability of 
this predicate is not entirely precluded in these language types. Finally, I suggest that examples such as \REF{Latin3} and \REF{Latin4} can be compatible with a change-of-state reading of the direct object, which would imply the adoption of a verb-framed resultative structure. For instance, a snake can twist its coils also if the coils have been previously formed, e.g., by the position of the body prior to the \textit{twisting}. Similarly, an existing road can be ordered to be covered with flint stone, supposing, for instance, that it was unpaved before. The proposed syntactic structure of the Latin example in \REF{Latin4}, assuming a change-of-state reading of the predicate, is provided in Figure \REF{sim:fig:Latin4}.

\begin{figure}[ht]
    \begin{forest}
    for tree={s sep=1cm, inner sep=0, l=0}
    [vP [v\textsuperscript{\st{\textit{[i]}}} [Pred [\textsc{stern-}] [Pred] ] [v\textsuperscript{\textit{[i]}}] ] [PredP [DP [viam] ] [Pred$'$ [\st{Pred}] [\st{\textsc{stern-}}] ] ] ]
    \end{forest}
\caption{Syntactic structure of \REF{Latin4} (with a visual representation of the PF requirement (\textit{[i]}) on \textit{v})}
    \label{sim:fig:Latin4}
\end{figure}

In order to further substantiate the prediction that complex creation/con\-sum\-ption predicates could not be licensed in weak satellite-framed Latin, I carried out a corpus-based investigation checking the co-occurrence, in a creation reading, of verbs that can be associated with a manner interpretation with two direct objects that seem to be particularly productive in English complex creation predicates, namely \textit{hole} (Lat. \textit{foramen}) and \textit{wound} (Lat. \textit{vulnus}). The corpus used for Latin, comprising texts from the Early and Classical periods (up to A.D. 200), is the \textit{Classical Latin Texts} by The Packard Humanities Institute.\footnote{\url{https://latin.packhum.org}} The verbs selected, listed in \REF{mannerLatin}, were taken from \citeauthor{Acedo-Matellan2016} (\citeyear{Acedo-Matellan2016}).

\ea \label{mannerLatin} \textit{amburo} `burn', \textit{caedo} `cut, knock', \textit{cremo} `burn', \textit{frico} `rub', \textit{rado} `scrape', \textit{tundo} `beat', \textit{uro} `burn', \textit{verro} `sweep' \z

\noindent Importantly, the English verbs corresponding to the Latin ones in \REF{mannerLatin} can give rise to creation predicates with \textit{hole} or \textit{wound} as effected object, as \REF{EngLat} shows.

\ea \label{EngLat} \ea A discharge of those energies burned a hole in his forehead and killed him. \hfill (\citealt[155]{AusensiAndBigolin2023})
\ex $[$...$]$ his words burned a wound inside her. \hfill (Google Books%\footnote{\textit{Google Books.}}
)
\ex Dad cut a hole in his chest and made me pull his heart out. \hfill (COCA)
\ex The Devil-Is-I pulled the knife he had used to cut a wound on his thumb and lunged forward at the leader of the twelve. \hfill (Google Books)
\ex Weena knocked a hole in the wall. \hfill (COCA)
\ex But I scraped a hole in it so I could see. \hfill (COCA)
\ex $[$...$]$ he scraped a wound on his nose that never cleared up.\\ \hfill (Google Books)
\ex I erased again and again until I had rubbed a hole in the paper. \\\hfill (COCA)
\ex $[$...$]$ the mooring line has rubbed a wound in the willow bark. \\\hfill (Google Books)
\ex My `beloved' boyfriend beat a hole in my roof and now it's awfully cold in there. \hfill (COCA)
\ex A sudden shift in the wind swept a hole in the blowing snow. \\\hfill (Google Books) \z \z

\noindent The verbs in \REF{mannerLatin} were searched for by stem, while the objects were searched for in the nominative/accusative singular and plural forms.\footnote{Being neuter, both \textit{foramen} `hole' and \textit{vulnus} `wound' appear as morphologically identical in their respective nominative and accusative forms.} None of the verb--complement combinations investigated provided relevant results. A verb-framed construction with \textit{vulnus} `wound' was found instead, as shown in \REF{vulneraigne}.

\ea \label{vulneraigne} \gll Sed uulnera facta igne dum sanescunt, defricare bubula urina convenit.\\
But wound.\textsc{nom}.\textsc{pl} make.\textsc{ptcp}.\textsc{pfv}.\textsc{nom}.\textsc{pl} fire.\textsc{abl} while heal.\textsc{prs.agr} off$\_$rub.\textsc{inf} bovine.\textsc{abl} urine.\textsc{abl} fit.{\AGR}\\
\glt `But while the wounds made with fire are healing, it is appropriate to cleanse them with bovine urine.' \hfill (\textit{Col.}, De Re Rustica \textit{6.7.4}) \z

\noindent I take this lack of evidence to tentatively suggest that Latin lacked complex creation/consumption predicates of the type found in satellite-framed languages, and needed to resort to verb-framed strategies in the domain of creation/con\-sump\-tion predicates in the same way as Slavic languages do. This is in line with the prediction, following from the present account, that complex creation/con\-sump\-tion predicates are unavailable in weak satellite-framed languages.

Picking up the discussion in \sectref{4.3} about the possibility that unprefixed complex predicates of creation/consumption may be disallowed in Slavic languages due to their lack of an incremental structure, it is relevant to notice that direct objects could be associated with an incremental structure giving rise to telicity in the predicate in Latin. The compatibility of the consumption predicate in \REF{Plinius} with the time span adverbial \textit{intra duas horas} `within two hours' illustrates this.

\ea \gll [...] nitrosae aut amarae aquae polenta addita mitigantur, ut intra duas horas bibi possint.\\
{} nitrous.\textsc{nom}.\textsc{pl} or bitter.\textsc{nom}.\textsc{pl} water.\textsc{nom}.\textsc{pl} cornmeal.\textsc{abl} add.\textsc{ptcp}.\textsc{pfv}.\textsc{abl} mitigate.\textsc{ipfv}.\textsc{sbjv.pass}.\textsc{agr} that within two.\textsc{acc}\hspace{2em} hour.\textsc{acc}.\textsc{pl} drink.\textsc{inf.pass} can.\textsc{ipfv}.\textsc{sbjv}.\textsc{agr}\\ \label{Plinius}
\glt `Nitrous and bitter waters are softened with added cornmeal, so that they can be drunk within two hours.' \hfill (\textit{Plin.}, Nat. \textit{24, 3, 4}) \z

\noindent Assuming, based on the discussion in this section, that complex creation/con\-sum\-ption predicates were not possible in Latin, such an absence %of such predicates in this language 
cannot be attributed to the predicate's lack of incrementality. The morphophonological account of Talmy's typology proposed in this chapter provides an alternative explanation of the phenomenon that is compatible with the observation that creation/con\-sump\-tion predicates could be telic in Latin without the presence of the prefix (see \REF{Plinius}). %predicts a  this fact cannot thus be attributed to the lack of incrementality in such predicates in Latin. 


\section{Conclusions}\label{6}

I presented the preliminary results of a pilot study concerning the possibility of licensing complex creation/consumption predicates in Slavic languages. The results obtained were further compared with data gathered from native speak\-ers of several satellite-framed languages and verb-framed languages. The study shows that Slavic languages, which are considered as fundamentally satellite-framed in the literature on Talmy's typology (\citealt{Talmy2000}; \citealt{Acedo-Matellan2016}), appear to behave on a par with verb-framed languages in disallowing creation/ consumption predicates that involve a satellite-framed strategy.

Adopting a neo-constructionist perspective on argument structure, I have put forward a morphophonological approach to the variation related to Talmy's typology, understanding verb-framedness in terms of a morphophonological realization condition imposed at PF on the null \textit{v} head involved in verbal predication. A null \textit{v} is required to incorporate its complement in verb-framed languages. I have further argued that Slavic languages, and weak satellite-framed languages in general, should be regarded as fundamentally verb-framed languages, capturing the mandatory prefixation of the Path component in resultative predicates and the absence of complex predicates of creation/consumption in these languages as by-products of the verb-framed PF requirement on the \textit{v} head.

With the present morphophonological account of Talmy's typology, I have additionally provided a solution to the minimalist conundrum whereby verb-framed languages seem to consistently lack the structure-building operation associated with the expression of a co-event in the verb, namely the operation of conflating a root with \textit{v}. To the extent that the verb-framed PF requirement can be satisfied by means of prefixation, the compounding operation can indeed successfully take place in a verb-framed system, as shown by the availability of prefixed satellite-framed resultative predicates in weak satellite-framed languages.

Afterward, I have explored the prediction that a creation/consumption reading of predicates with manner-denoting verbs is available in Slavic languages when the predicate is perfectivized via internal prefixes, which have been argued to involve a resultative structure that receives a reading as involving an event of creation/consumption on the basis of conceptual/pragmatic considerations. The data gathered from the native speakers of the Slavic languages tested confirmed the prediction.

Finally I have argued that Latin, as a weak satellite-framed language (\citealt{Acedo-Matellan2010}; \citeyear{Acedo-Matellan2016}), lacked complex creation/consumption predicates of the type found in bona fide satellite-framed languages in the same way as Slavic languages do. I have argued that this is the case based on the analysis of some alleged Latin complex creation/consumption predicates provided in \citet{Acedo-Matellan2016}, which have been shown to admit a reading either as involving a hyponymous object or as involving a resultative predicate of change of state. Afterward, I have presented the results of a corpus search supporting the prediction that complex creation/consumption predicates are not licensed in Latin. The results strengthen the general hypothesis that Latin and Slavic languages behave alike with respect to Talmy's typology (\citealt{Acedo-Matellan2016}), meanwhile underpinning one of the main conclusion of the present account whereby weak satellite-framed languages should be considered as fundamentally verb-framed languages.


\section*{Abbreviations}

\begin{tabularx}{.5\textwidth}{@{}lQ}
\textsc{abl}&ablative\\
\textsc{acc}&accusative\\
\textsc{agr}&agreement\\
\textsc{cl}&classifier\\
\textsc{dat}&dative\\
\textsc{fut}&future\\
\textsc{ger}&gerund\\
\textsc{grdv}&gerundive\\
\textsc{inf}&infinitive\\
\textsc{ins}&instrumental\\
\textsc{ipfv}&imperfective\\
\textsc{loc}&locative\\
\textsc{m}&masculine\\
\end{tabularx}%
\begin{tabularx}{.5\textwidth}{lQ@{}}
\textsc{neg}&negation\\
\textsc{nom}&nominative\\
\textsc{pass}&passive\\
\textsc{pfv}&perfective\\
\textsc{pl}&plural\\
\textsc{poss}&possessive\\
\textsc{prf}&perfect\\
\textsc{prs}&present\\
\textsc{pst}&past\\
\textsc{ptcp}&participle\\
\textsc{sbjv}&subjunctive\\
\textsc{sg}&singular\\
\textsc{th}&theme vowel\\
%&\\ % this dummy row achieves correct vertical alignment of both tables
\end{tabularx}

\section*{Acknowledgments}
I wish to express my gratitude to Alexandra Ghetallo, Andreas Trotzke, Arkadiusz Kwapiszewski, Dària Serés, Dávid Janik, Éva Kardos, Evripidis Tsiakmakis, Francesc Torres-Tamarit, Hanna de Vries, Irene Fernández-Serrano, Jon Ander Mendia, Katerina Thomopoulou, Marta Petrak, Natália Kolenčíková, Predrag Kovačević, Wojciech Lewandowski, and Ziwen Wang, for offering me their time and help, which were essential to the completion of this investigation. I am especially indebted to Jaume Mateu, for sharing with me his fruitful ideas on the topic and for insightful discussion. I also thank Grant Armstrong, Josep Ausensi and Srabasti Dey for detailed comments on previous drafts of this work, Víctor Acedo-Matellán, Rebecca Shields and Anna Szabolcsi for useful suggestions, and the editors of this volume and two anonymous reviewers for their helpful observations. All errors are my own. This research was supported by funding from a FPI pre-doctoral fellowship (Spanish Agencia Estatal de Investigación/European Social Fund), as well as from the research projects FFI2017-87140-C4-1-P (Spanish Ministerio de Economía, Industria y Competitividad), PID2021-123617NB-C41 (Spanish Ministerio de Ciencia e Innovación), 2017SGR634 (Generalitat de Catalunya), and 2021SGR00787 (Generalitat de Catalunya).


\printbibliography[heading=subbibliography,notkeyword=this]

\end{document}
