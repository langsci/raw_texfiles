\documentclass[output=paper,colorlinks,citecolor=brown]{langscibook}
\ChapterDOI{10.5281/zenodo.15394176}
% \bibliography{localbibliography}

% \author{Kateřina Hrdinková\affiliation{Charles University} and Radek Šimík\affiliation{Charles University}}
\author{Kateřina Hrdinková\affiliation{Charles University} and Radek Šimík\orcid{0000-0002-4736-195X}\affiliation{Charles University}}
% replace the above with you and your coauthors
% rules for affiliation: If there's an official English version, use that (find out on the official website of the university); if not, use the original
% orcid doesn't appear printed; it's metainformation used for later indexing

%%% uncomment the following line if you are a single author or all authors have the same affiliation
\SetupAffiliations{mark style=none}

%% in case the running head with authors exceeds one line (which is the case in this example document), use one of the following methods to turn it into a single line; otherwise comment the line below out with % and ignore it
% \lehead{Šimík, Gehrke, Lenertová, Meyer, Szucsich \& Zaleska}
% \lehead{Radek Šimík et al.}

\title{The meaning of Czech response particles}
% replace the above with your paper title
%%% provide a shorter version of your title in case it doesn't fit a single line in the running head
% in this form: \title[short title]{full title}
\abstract{This article deals with the semantics and interpretation of Czech response particles \textit{
ano} `yes' and \textit{ne} ‘no'. Based on two experiments involving responses to negative polar questions, we argue that \textit{ano} `yes' encodes the relative feature [\textsc{agree}] and \textit{ne} `no' encodes the absolute feature $[−]$, adopting the parlance of \citeposst{roelofsen-farkas15} feature model. This contrasts with the proposal of \citet{gruet2016yes}, who argues, following previous work on English, that both of the Czech response particles are ambiguous between a relative and an absolute reading. We also find some tentative evidence for context affecting the interpretation of response particles, in line with the predictions of \citet{hrd+:krifka13}.

\keywords{response particles, answer, syntax, semantics, pragmatics}
}

\begin{document}
% \pagenumbering{arabic}
% \input{replies.tex}

% \newpage
% \pagenumbering{roman}
\maketitle

% Just comment out the input below when you're ready to go.
% For a start: Do not forget to give your Overleaf project (this paper) a recognizable name. This one could be called, for instance, Simik et al: OSL template. You can change the name of the project by hovering over the gray title at the top of this page and clicking on the pencil icon.

\section{Introduction}\label{sim:sec:intro}

Language Science Press is a project run for linguists, but also by linguists. You are part of that and we rely on your collaboration to get at the desired result. Publishing with LangSci Press might mean a bit more work for the author (and for the volume editor), esp. for the less experienced ones, but it also gives you much more control of the process and it is rewarding to see the quality result.

Please follow the instructions below closely, it will save the volume editors, the series editors, and you alike a lot of time.

\sloppy This stylesheet is a further specification of three more general sources: (i) the Leipzig glossing rules \citep{leipzig-glossing-rules}, (ii) the generic style rules for linguistics (\url{https://www.eva.mpg.de/fileadmin/content_files/staff/haspelmt/pdf/GenericStyleRules.pdf}), and (iii) the Language Science Press guidelines \citep{Nordhoff.Muller2021}.\footnote{Notice the way in-text numbered lists should be written -- using small Roman numbers enclosed in brackets.} It is advisable to go through these before you start writing. Most of the general rules are not repeated here.\footnote{Do not worry about the colors of references and links. They are there to make the editorial process easier and will disappear prior to official publication.}

Please spend some time reading through these and the more general instructions. Your 30 minutes on this is likely to save you and us hours of additional work. Do not hesitate to contact the editors if you have any questions.

\section{Illustrating OSL commands and conventions}\label{sim:sec:osl-comm}

Below I illustrate the use of a number of commands defined in langsci-osl.tex (see the styles folder).

\subsection{Typesetting semantics}\label{sim:sec:sem}

See below for some examples of how to typeset semantic formulas. The examples also show the use of the sib-command (= ``semantic interpretation brackets''). Notice also the the use of the dummy curly brackets in \REF{sim:ex:quant}. They ensure that the spacing around the equation symbol is correct. 

\ea \ea \sib{dog}$^g=\textsc{dog}=\lambda x[\textsc{dog}(x)]$\label{sim:ex:dog}
\ex \sib{Some dog bit every boy}${}=\exists x[\textsc{dog}(x)\wedge\forall y[\textsc{boy}(y)\rightarrow \textsc{bit}(x,y)]]$\label{sim:ex:quant}
\z\z

\noindent Use noindent after example environments (but not after floats like tables or figures).

And here's a macro for semantic type brackets: The expression \textit{dog} is of type $\stb{e,t}$. Don't forget to place the whole type formula into a math-environment. An example of a more complex type, such as the one of \textit{every}: $\stb{s,\stb{\stb{e,t},\stb{e,t}}}$. You can of course also use the type in a subscript: dog$_{\stb{e,t}}$

We distinguish between metalinguistic constants that are translations of object language, which are typeset using small caps, see \REF{sim:ex:dog}, and logical constants. See the contrast in \REF{sim:ex:speaker}, where \textsc{speaker} (= serif) in \REF{sim:ex:speaker-a} is the denotation of the word \textit{speaker}, and \cnst{speaker} (= sans-serif) in \REF{sim:ex:speaker-b} is the function that maps the context $c$ to the speaker in that context.\footnote{Notice that both types of small caps are automatically turned into text-style, even if used in a math-environment. This enables you to use math throughout.}$^,$\footnote{Notice also that the notation entails the ``direct translation'' system from natural language to metalanguage, as entertained e.g. in \citet{Heim.Kratzer1998}. Feel free to devise your own notation when relying on the ``indirect translation'' system (see, e.g., \citealt{Coppock.Champollion2022}).}

\ea\label{sim:ex:speaker}
\ea \sib{The speaker is drunk}$^{g,c}=\textsc{drunk}\big(\iota x\,\textsc{speaker}(x)\big)$\label{sim:ex:speaker-a}
\ex \sib{I am drunk}$^{g,c}=\textsc{drunk}\big(\cnst{speaker}(c)\big)$\label{sim:ex:speaker-b}
\z\z

\noindent Notice that with more complex formulas, you can use bigger brackets indicating scope, cf. $($ vs. $\big($, as used in \REF{sim:ex:speaker}. Also notice the use of backslash plus comma, which produces additional space in math-environment.

\subsection{Examples and the minsp command}

Try to keep examples simple. But if you need to pack more information into an example or include more alternatives, you can resort to various brackets or slashes. For that, you will find the minsp-command useful. It works as follows:

\ea\label{sim:ex:german-verbs}\gll Hans \minsp{\{} schläft / schlief / \minsp{*} schlafen\}.\\
Hans {} sleeps {} slept {} {} sleep.\textsc{inf}\\
\glt `Hans \{sleeps / slept\}.'
\z

\noindent If you use the command, glosses will be aligned with the corresponding object language elements correctly. Notice also that brackets etc. do not receive their own gloss. Simply use closed curly brackets as the placeholder.

The minsp-command is not needed for grammaticality judgments used for the whole sentence. For that, use the native langsci-gb4e method instead, as illustrated below:

\ea[*]{\gll Das sein ungrammatisch.\\
that be.\textsc{inf} ungrammatical\\
\glt Intended: `This is ungrammatical.'\hfill (German)\label{sim:ex:ungram}}
\z

\noindent Also notice that translations should never be ungrammatical. If the original is ungrammatical, provide the intended interpretation in idiomatic English.

If you want to indicate the language and/or the source of the example, place this on the right margin of the translation line. Schematic information about relevant linguistic properties of the examples should be placed on the line of the example, as indicated below.

\ea\label{sim:ex:bailyn}\gll \minsp{[} Ėtu knigu] čitaet Ivan \minsp{(} často).\\
{} this book.{\ACC} read.{\PRS.3\SG} Ivan.{\NOM} {} often\\\hfill O-V-S-Adv
\glt `Ivan reads this book (often).'\hfill (Russian; \citealt[4]{Bailyn2004})
\z

\noindent Finally, notice that you can use the gloss macros for typing grammatical glosses, defined in langsci-lgr.sty. Place curly brackets around them.

\subsection{Citation commands and macros}

You can make your life easier if you use the following citation commands and macros (see code):

\begin{itemize}
    \item \citealt{Bailyn2004}: no brackets
    \item \citet{Bailyn2004}: year in brackets
    \item \citep{Bailyn2004}: everything in brackets
    \item \citepossalt{Bailyn2004}: possessive
    \item \citeposst{Bailyn2004}: possessive with year in brackets
\end{itemize}

\section{Trees}\label{s:tree}

Use the forest package for trees and place trees in a figure environment. \figref{sim:fig:CP} shows a simple example.\footnote{See \citet{VandenWyngaerd2017} for a simple and useful quickstart guide for the forest package.} Notice that figure (and table) environments are so-called floating environments. {\LaTeX} determines the position of the figure/table on the page, so it can appear elsewhere than where it appears in the code. This is not a bug, it is a property. Also for this reason, do not refer to figures/tables by using phrases like ``the table below''. Always use tabref/figref. If your terminal nodes represent object language, then these should essentially correspond to glosses, not to the original. For this reason, we recommend including an explicit example which corresponds to the tree, in this particular case \REF{sim:ex:czech-for-tree}.

\ea\label{sim:ex:czech-for-tree}\gll Co se řidič snažil dělat?\\
what {\REFL} driver try.{\PTCP.\SG.\MASC} do.{\INF}\\
\glt `What did the driver try to do?'
\z

\begin{figure}[ht]
% the [ht] option means that you prefer the placement of the figure HERE (=h) and if HERE is not possible, you prefer the TOP (=t) of a page
% \centering
    \begin{forest}
    for tree={s sep=1cm, inner sep=0, l=0}
    [CP
        [DP
            [what, roof, name=what]
        ]
        [C$'$
            [C
                [\textsc{refl}]
            ]
            [TP
                [DP
                    [driver, roof]
                ]
                [T$'$
                    [T [{[past]}]]
                    [VP
                        [V
                            [tried]
                        ]
                        [VP, s sep=2.2cm
                            [V
                                [do.\textsc{inf}]
                            ]
                            [t\textsubscript{what}, name=trace-what]
                        ]
                    ]
                ]
            ]
        ]
    ]
    \draw[->,overlay] (trace-what) to[out=south west, in=south, looseness=1.1] (what);
    % the overlay option avoids making the bounding box of the tree too large
    % the looseness option defines the looseness of the arrow (default = 1)
    \end{forest}
    \vspace{3ex} % extra vspace is added here because the arrow goes too deep to the caption; avoid such manual tweaking as much as possible; here it's necessary
    \caption{Proposed syntactic representation of \REF{sim:ex:czech-for-tree}}
    \label{sim:fig:CP}
\end{figure}

Do not use noindent after figures or tables (as you do after examples). Cases like these (where the noindent ends up missing) will be handled by the editors prior to publication.

\section{Italics, boldface, small caps, underlining, quotes}

See \citet{Nordhoff.Muller2021} for that. In short:

\begin{itemize}
    \item No boldface anywhere.
    \item No underlining anywhere (unless for very specific and well-defined technical notation; consult with editors).
    \item Small caps used for (i) introducing terms that are important for the paper (small-cap the term just ones, at a place where it is characterized/defined); (ii) metalinguistic translations of object-language expressions in semantic formulas (see \sectref{sim:sec:sem}); (iii) selected technical notions.
    \item Italics for object-language within text; exceptionally for emphasis/contrast.
    \item Single quotes: for translations/interpretations
    \item Double quotes: scare quotes; quotations of chunks of text.
\end{itemize}

\section{Cross-referencing}

Label examples, sections, tables, figures, possibly footnotes (by using the label macro). The name of the label is up to you, but it is good practice to follow this template: article-code:reference-type:unique-label. E.g. sim:ex:german would be a proper name for a reference within this paper (sim = short for the author(s); ex = example reference; german = unique name of that example).

\section{Syntactic notation}

Syntactic categories (N, D, V, etc.) are written with initial capital letters. This also holds for categories named with multiple letters, e.g. Foc, Top, Num, etc. Stick to this convention also when coming up with ad hoc categories, e.g. Cl (for clitic or classifier).

An exception from this rule are ``little'' categories, which are written with italics: \textit{v}, \textit{n}, \textit{v}P, etc.

Bar-levels must be typeset with bars/primes, not with an apostrophe. An easy way to do that is to use mathmode for the bar: C$'$, Foc$'$, etc.

Specifiers should be written this way: SpecCP, Spec\textit{v}P.

Features should be surrounded by square brackets, e.g., [past]. If you use plus and minus, be sure that these actually are plus and minus, and not e.g. a hyphen. Mathmode can help with that: [$+$sg], [$-$sg], [$\pm$sg]. See \sectref{sim:sec:hyphens-etc} for related information.

\section{Footnotes}

Absolutely avoid long footnotes. A footnote should not be longer than, say, {20\%} of the page. If you feel like you need a long footnote, make an explicit digression in the main body of the text.

Footnotes should always be placed at the end of whole sentences. Formulate the footnote in such a way that this is possible. Footnotes should always go after punctuation marks, never before. Do not place footnotes after individual words. Do not place footnotes in examples, tables, etc. If you have an urge to do that, place the footnote to the text that explains the example, table, etc.

Footnotes should always be formulated as full, self-standing sentences.

\section{Tables}

For your tables use the table plus tabularx environments. The tabularx environment lets you (and requires you in fact) to specify the width of the table and defines the X column (left-alignment) and the Y column (right-alignment). All X/Y columns will have the same width and together they will fill out the width of the rest of the table -- counting out all non-X/Y columns.

Always include a meaningful caption. The caption is designed to appear on top of the table, no matter where you place it in the code. Do not try to tweak with this. Tables are delimited with lsptoprule at the top and lspbottomrule at the bottom. The header is delimited from the rest with midrule. Vertical lines in tables are banned. An example is provided in \tabref{sim:tab:frequencies}. See \citet{Nordhoff.Muller2021} for more information. If you are typesetting a very complex table or your table is too large to fit the page, do not hesitate to ask the editors for help.

\begin{table}
\caption{Frequencies of word classes}
\label{sim:tab:frequencies}
 \begin{tabularx}{.77\textwidth}{lYYYY} %.77 indicates that the table will take up 77% of the textwidth
  \lsptoprule
            & nouns & verbs  & adjectives & adverbs\\
  \midrule
  absolute  &   12  &    34  &    23      & 13\\
  relative  &   3.1 &   8.9  &    5.7     & 3.2\\
  \lspbottomrule
 \end{tabularx}
\end{table}

\section{Figures}

Figures must have a good quality. If you use pictorial figures, consult the editors early on to see if the quality and format of your figure is sufficient. If you use simple barplots, you can use the barplot environment (defined in langsci-osl.sty). See \figref{sim:fig:barplot} for an example. The barplot environment has 5 arguments: 1. x-axis desription, 2. y-axis description, 3. width (relative to textwidth), 4. x-tick descriptions, 5. x-ticks plus y-values.

\begin{figure}
    \centering
    \barplot{Type of meal}{Times selected}{0.6}{Bread,Soup,Pizza}%
    {
    (Bread,61)
    (Soup,12)
    (Pizza,8)
    }
    \caption{A barplot example}
    \label{sim:fig:barplot}
\end{figure}

The barplot environment builds on the tikzpicture plus axis environments of the pgfplots package. It can be customized in various ways. \figref{sim:fig:complex-barplot} shows a more complex example.

\begin{figure}
  \begin{tikzpicture}
    \begin{axis}[
	xlabel={Level of \textsc{uniq/max}},  
	ylabel={Proportion of $\textsf{subj}\prec\textsf{pred}$}, 
	axis lines*=left, 
        width  = .6\textwidth,
	height = 5cm,
    	nodes near coords, 
    % 	nodes near coords style={text=black},
    	every node near coord/.append style={font=\tiny},
        nodes near coords align={vertical},
	ymin=0,
	ymax=1,
	ytick distance=.2,
	xtick=data,
	ylabel near ticks,
	x tick label style={font=\sffamily},
	ybar=5pt,
	legend pos=outer north east,
	enlarge x limits=0.3,
	symbolic x coords={+u/m, \textminus u/m},
	]
	\addplot[fill=red!30,draw=none] coordinates {
	    (+u/m,0.91)
        (\textminus u/m,0.84)
	};
	\addplot[fill=red,draw=none] coordinates {
	    (+u/m,0.80)
        (\textminus u/m,0.87)
	};
	\legend{\textsf{sg}, \textsf{pl}}
    \end{axis} 
  \end{tikzpicture} 
    \caption{Results divided by \textsc{number}}
    \label{sim:fig:complex-barplot}
\end{figure}

\section{Hyphens, dashes, minuses, math/logical operators}\label{sim:sec:hyphens-etc}

Be careful to distinguish between hyphens (-), dashes (--), and the minus sign ($-$). For in-text appositions, use only en-dashes -- as done here -- with spaces around. Do not use em-dashes (---). Using mathmode is a reliable way of getting the minus sign.

All equations (and typically also semantic formulas, see \sectref{sim:sec:sem}) should be typeset using mathmode. Notice that mathmode not only gets the math signs ``right'', but also has a dedicated spacing. For that reason, never write things like p$<$0.05, p $<$ 0.05, or p$<0.05$, but rather $p<0.05$. In case you need a two-place math or logical operator (like $\wedge$) but for some reason do not have one of the arguments represented overtly, you can use a ``dummy'' argument (curly brackets) to simulate the presence of the other one. Notice the difference between $\wedge p$ and ${}\wedge p$.

In case you need to use normal text within mathmode, use the text command. Here is an example: $\text{frequency}=.8$. This way, you get the math spacing right.

\section{Abbreviations}

The final abbreviations section should include all glosses. It should not include other ad hoc abbreviations (those should be defined upon first use) and also not standard abbreviations like NP, VP, etc.


\section{Bibliography}

Place your bibliography into localbibliography.bib. Important: Only place there the entries which you actually cite! You can make use of our OSL bibliography, which we keep clean and tidy and update it after the publication of each new volume. Contact the editors of your volume if you do not have the bib file yet. If you find the entry you need, just copy-paste it in your localbibliography.bib. The bibliography also shows many good examples of what a good bibliographic entry should look like.

See \citet{Nordhoff.Muller2021} for general information on bibliography. Some important things to keep in mind:

\begin{itemize}
    \item Journals should be cited as they are officially called (notice the difference between and, \&, capitalization, etc.).
    \item Journal publications should always include the volume number, the issue number (field ``number''), and DOI or stable URL (see below on that).
    \item Papers in collections or proceedings must include the editors of the volume (field ``editor''), the place of publication (field ``address'') and publisher.
    \item The proceedings number is part of the title of the proceedings. Do not place it into the ``volume'' field. The ``volume'' field with book/proceedings publications is reserved for the volume of that single book (e.g. NELS 40 proceedings might have vol. 1 and vol. 2).
    \item Avoid citing manuscripts as much as possible. If you need to cite them, try to provide a stable URL.
    \item Avoid citing presentations or talks. If you absolutely must cite them, be careful not to refer the reader to them by using ``see...''. The reader can't see them.
    \item If you cite a manuscript, presentation, or some other hard-to-define source, use the either the ``misc'' or ``unpublished'' entry type. The former is appropriate if the text cited corresponds to a book (the title will be printed in italics); the latter is appropriate if the text cited corresponds to an article or presentation (the title will be printed normally). Within both entries, use the ``howpublished'' field for any relevant information (such as ``Manuscript, University of \dots''). And the ``url'' field for the URL.
\end{itemize}

We require the authors to provide DOIs or URLs wherever possible, though not without limitations. The following rules apply:

\begin{itemize}
    \item If the publication has a DOI, use that. Use the ``doi'' field and write just the DOI, not the whole URL.
    \item If the publication has no DOI, but it has a stable URL (as e.g. JSTOR, but possibly also lingbuzz), use that. Place it in the ``url'' field, using the full address (https: etc.).
    \item Never use DOI and URL at the same time.
    \item If the official publication has no official DOI or stable URL (related to the official publication), do not replace these with other links. Do not refer to published works with lingbuzz links, for instance, as these typically lead to the unpublished (preprint) version. (There are exceptions where lingbuzz or semanticsarchive are the official publication venue, in which case these can of course be used.) Never use URLs leading to personal websites.
    \item If a paper has no DOI/URL, but the book does, do not use the book URL. Just use nothing.
\end{itemize}

\section{Introduction} \label{hrdsim:sec:introduction}

Response particles like \textit{yes} and \textit{no} are a common way to respond to polar questions. They exhibit anaphoric behavior in that their interpretation crucially depends on previous context and, more specifically, on the form and interpretation of the polar question they respond to. While responses to affirmative questions are largely unproblematic, responses to negative questions give rise to ambiguities (\citealt{Kramer.Rawlins2011,espinal2019response,roelofsen-farkas15,hrd+:krifka13}; etc.); see \REF{hrdsim:ex1} and \REF{hrdsim:ex2}, respectively. (The translations in (\ref{hrdsim:ex2}B) are tentative and will be rectified in view of the experimental results.)

\begin{exe}
\ex
\begin{xlist}
\exi{A:}[]{ 
\gll Zalil Petr květiny?\\  
    watered Petr flowers\\  
    \glt ‘Has Petr watered flowers?'}
\exi{B:}[]{
\gll Ano. (= Zalil.) / Ne. (= Nezalil.)\\
yes {} watered {} no {} \textsc{neg}.watered\\
\glt ‘Yes. (= He has.) / No. (= He hasn't.)'
}\label{hrdsim:ex1}
\end{xlist}
\end{exe}

\begin{exe}
\ex  
\begin{xlist}
\exi{A:}[]{ 
\gll    Nezalil Petr květiny?\\
        \textsc{neg}.watered Petr flowers\\
        \glt ‘Hasn't Peter watered flowers?'}
\exi{B:}[]{
\gll Ano. (= Zalil / Nezalil.) / Ne. (= Zalil / Nezalil.)\\
yes {} watered {} \textsc{neg}.watered {} no {} watered {} \textsc{neg}.watered\\
\glt   ‘Yes. (= He has. / He hasn't.) / No. (= He has. / He hasn't.)'\\\hfill (translations tentative; to be rectified)
}\label{hrdsim:ex2}
\end{xlist}
\end{exe}

\noindent If the polar question is negative, as in \REF{hrdsim:ex2}, both \textit{ano} `yes' and \textit{ne} `no' can in principle correspond to a positive or a negative answer. They can, however, differ in naturalness and likelihood. To give an example from German, \citet{claus2017puzzling} found out that it is more natural to confirm negative questions by \textit{ja} `yes' than by \textit{nein} `no'.

Using a version of the truth-value judgment task, we investigate the meaning of the two Czech response particles \textit{ano} `yes' and \textit{ne} `no', hoping to contribute to related recent literature on Slavic languages (e.g. \citealt{gruet2016yes,esipova2021polar,Geist.Repp2023}). A more specific goal is to evaluate the adequacy of two types of existing accounts of response particle meaning: the feature model of \citet{roelofsen-farkas15}, in which response particles have a lexically specified range of meanings, and the saliency account of \citet{hrd+:krifka13}, in which the meaning is expected to be more context-dependent. We also discuss our results in the light of \citeposst{gruet2016yes} analysis of Czech response particles, which is couched in the feature model. We conclude that our data primarily support a particular version of the feature model, though not the one proposed by \citet{gruet2016yes}. More specifically, we see a very clear tendency for \textit{ano} `yes' to express agreement (the feature [\textsc{agree}]) with its antecedent, be it positive or negative, and \textit{ne} `no' to express a negative proposition (the feature [$-$]), independently of the polarity of the antecedent. What counts as the ``antecedent'' is crucially modulated by the interrogative strategy used: negative polar questions with an interrogative syntax (verb-first) primarily contribute a positive antecedent (i.e., the negation is, by hypothesis, ``pleonastic''), while negative polar questions with a declarative syntax (non-verb-first) contribute a negative antecedent (negation is semantic/propositional). Even though the feature model appears to be more suitable for modelling our results, we also observe -- in a subset of our data -- a statistically significant result predicted by \citeposst{hrd+:krifka13} saliency theory.

The article is structured as follows. \sectref{hrdsim:sec:approaches} briefly introduces the two approaches under consideration -- the feature model \citep{roelofsen-farkas15} and the saliency theory (\citealt{hrd+:krifka13}). We also discuss \citeposst{gruet2016yes} particular application of the feature model to Czech. \sectref{hrdsim:sec:experiment} reports on the experiments we have conducted: experiment 1, in which we investigated responses to negative polar questions with interrogative syntax (V1), and experiment 2, in which we looked at negative polar questions with declarative syntax. In \sectref{hrdsim:sec:discussion} we discuss the results and propose a new implementation of the feature model which is consistent with the results. Finally, \sectref{hrdsim:sec:conclusion} concludes the paper.

\section{Approaches to response particle meaning}\label{hrdsim:sec:approaches}
% From the point of view of syntax, two different approaches to response particles could be distinguished. The basic difference between them is their opposite view on whether response particles have the status of a sentence.\\

% \textsc{Elliptical approaches} (\citealt{Kramer.Rawlins2011}, \citealt{holmberg2013syntax} etc.) assume that response particles are part of a complete clausal structure in which everything except response particle is elided. On the other hand, \textsc{direct derivation approaches} (\citealt{asher1986belief}, \citealt{frank1997context}, \citealt{roelofsen-farkas15}, \citealt{hrd+:krifka13} etc.) think of response particles as independent structures without any additional syntax \citep[262]{espinal2019response}.

% One of the tasks of this paper is to verify whether the hypotheses of the Features model and Saliency account, which were built for English, are also valid in the Czech language context.

\subsection{Feature model} \label{hrdsim:sec:featuresmodel}

\subsubsection{\citet{roelofsen-farkas15}}

The influential \textsc{feature model} of \citet{roelofsen-farkas15} is based on the assumption that a response particle like `yes' or `no' has a lexically specified range of meanings, defined in terms of two types of features -- absolute and relative polarity features. The \textsc{absolute features} $[+]$ and $[-]$ correspond to the polarity of the response. The relative features [\textsc{agree}] and [\textsc{reverse}] indicate a relation between the response and its propositional antecedent (derived from a polar question or an assertion that antecedes the response): the former expresses agreement with the polarity of the antecedent, the latter reverses its polarity. The semantics of the features is presuppositional (see \citealt[385f.]{roelofsen-farkas15} for details). In the lexicon, a response particle can either be specified for a single feature or for a combination of features. Additional complexity may arise in the process of feature realization (spellout), where \citet{roelofsen-farkas15} assume that a feature bundle can be realized by a particle which matches only its proper subset.\footnote{For a recent experimental evaluation of \citeposst{roelofsen-farkas15} model, see \citet{Maldonado.Culbertson2023}.}

\tabref{hrdsim:tab:features} shows the assumed lexical entries and corresponding realization patterns of the English particles \textit{yes} and \textit{no} and the German particle \textit{doch}. By hypothesis, the English particles encode single features, but are lexically ambiguous -- they either encode the respective absolute or relative features. If a feature bundle is generated in the syntax (and interpreted in the semantics), it is realized by a particle whose lexical makeup matches a proper subset of that bundle. In two cases -- [\textsc{agree}, $-$] and [\textsc{reverse}, $+$] -- both \textit{yes} and \textit{no} provide a good match, giving rise to an ambiguity which must be resolved pragmatically.\footnote{\citet{roelofsen-farkas15} employ a set of additional markedness-based rules which nudge the likelihood in one or the other direction.} An example of a bundle-encoding particle is German \textit{doch}, which responds to negative antecedents and at the same time reverses their polarity, whence [\textsc{reverse}, $+$].


% The starting point of the given theory is that response particles function as a morphological realization of two types of polar features -- absolute and relative. \citep[384--385]{roelofsen-farkas15}\\

% The absolute polarity features [+] and [$-$] presuppose that the prejacent of the answer contains a single possibility, which is either positive or negative, and the bare response particle expresses this polarity. In contrast, the relative polarity features [\textsc{agree}] and [\textsc{reverse}] are far more discourse-bound. They assume that the discourse context contains a salient possibility of the antecedent of the answer, and the prejacent then either agrees with the given possibility or refutes it. These features can be combined in their specific lexical realizations, which, according to \citet[384]{roelofsen-farkas15}, gives a total of 4 combinatorial possibilities:\\

\begin{table}%[]
\centering
\begin{tabularx}{.8\textwidth}{Xlll}
\lsptoprule
& Lexically  encoded by & Realized by\\\midrule
$[+]$&\textit{yes}&\textit{yes}\\[0.2em]
$[-]$&\textit{no}&\textit{no}\\[0.2em]
[\textsc{agree}]&\textit{yes}&\textit{yes}\\[0.2em]
[\textsc{reverse}]&\textit{no}&\textit{no}\\[0.2em]
[\textsc{agree}, $+$] &n.a.&\textit{yes}\\ [0.2em]
[\textsc{agree}, $-$] &n.a.&\textit{yes} or \textit{no}\\ [0.2em]
[\textsc{reverse}, $+$] &\textit{doch}&\textit{yes} or \textit{no} / \textit{doch}\\ [0.2em]
[\textsc{reverse}, $-$] &n.a.&\textit{no}\\
\lspbottomrule
\end{tabularx}
    \caption{Feature bundles in the feature model}
    \label{hrdsim:tab:features}
\end{table}

% Polarity features are generated on the syntactic level and they are further given a lexical value using so-called implementation rules. These are language specific, so it is worth noting that the following description will only apply to English and similar languages. The experiment then examines whether they are also applicable to the Czech language.\\

% According to the realization rules made for English, [\textsc{agree}] and [+] can be expressed with \textit{yes}, [\textsc{reverse}] and [$-$] then with \textit{no}. Therefore, if the realization potential of English response particles were combined with the combinatorial possibilities of polar features, the following combinations would occur:
% \begin{itemize}
%  \setlength\itemsep{-1.2 em}
%     \item [] {[\textsc{agree}, +]} can only be realized with yes.\\
%     \item [] {[\textsc{reverse}, $-$]} can only be realized with no.\\
%     \item [] {[\textsc{agree}, $-$]} can be implemented with both yes and no.\\
%     \item [] {[\textsc{reverse}, +]} can be implemented with both yes and no.

% \end{itemize}


Thus, the English \textit{yes} can signal that the answer has a positive polarity [+] or it can agree with its propositional antecedent [\textsc{agree}]. In contrast, \textit{no} can serve either to signal negative polarity [$-$] or to reverse the polarity of its antecedent [\textsc{reverse}]. These double properties of \textit{yes} and \textit{no}, according to \citet[383]{roelofsen-farkas15}, explain why response particles are generally clear after a positive question / positive statement (\ref{hrdsim:ex:positiveantecedent}), while a double interpretation is possible after a negative question / negative statement (\ref{hrdsim:ex:negativeantecedent}).

\begin{exe}
\ex \label{hrdsim:ex:positiveantecedent}
Amy left.\hfill (positive antecedent)\\
Agreement: Yes, she did. / *No, she did.\\
Reversal: *Yes, she didn't. / No, she didn't.   
\end{exe}

\begin{exe}
\ex \label{hrdsim:ex:negativeantecedent}
Amy didn't leave.\hfill ({negative antecedent})\\
Agreement: Yes, she didn't. / No, she didn't.\\
Reversal: Yes, she did. / No, she did.\\
\end{exe}

\subsubsection{The feature model applied to Czech: \citet{gruet2016yes}}\label{hrdsim:sec:czechresponse}

\citet{gruet2016yes} adopts \citeposst{roelofsen-farkas15} feature model and adapts it to Czech. \citeauthor{gruet2016yes} assumes that Czech response particles \textit{ano} `yes' and \textit{ne} `no' exhibit the same ambiguity as the English particles \textit{yes} and \textit{no}, i.e., they can either realize the absolute features ($[+]$ and $[-]$, respectively) or the relative features ([\textsc{agree}] and [\textsc{reverse}], respectively).

\citet{gruet2016yes} further modulates her analysis relative to the form of the question which antecedes the response. She assumes that in interrogative questions, i.e., questions with the verb in clause-initial position (V1), the polarity is neutralized. Response particles used in reaction to V1 questions therefore realize their absolute features. This is illustrated for the case of negative V1 questions in \REF{hrdsim:ex:v1}, where there is no ambiguity in the response: \textit{ano} `yes' indicates positive polarity and \textit{ne} `no' negative polarity. In declarative questions, i.e., questions with the verb in a non-initial position (non-V1), the polarity is salient and the response particles realize their relative features. As a result, response particles are also not ambiguous in this case, but have opposite truth-conditions; see \REF{hrdsim:ex:non-v1}.

\begin{exe}
\ex  {Negative interrogative question (with V1)}\label{hrdsim:ex:v1}
\begin{xlist}
\exi{A:}[]{ 
\gll    Nenapsala Jitka esej?\\
        \textsc{neg}.wrote Jitka essay\\
        \glt ‘Hasn't Jitka written an essay?'}
\exi{B:}[]{
\gll Ano. (= Napsala.) / Ne. (= Nenapsala.)\\
yes {} wrote {} no {} \textsc{neg}.wrote\\
\glt   ‘Yes. (= She has.) / No. (= She hasn't.)' 
}
\end{xlist}
\end{exe}

\begin{exe}
\ex {Negative declarative question (with non-V1)}\label{hrdsim:ex:non-v1}
\begin{xlist}
\exi{A:}[]{ 
\gll    Jitka esej nenapsala?\\
        Jitka essay \textsc{neg}.wrote\\
        \glt ‘Hasn't Jitka written an essay?'}
\exi{B:}[]{
\gll Ano. (= Nenapsala.) / Ne. (= Napsala.)\\
yes {} \textsc{neg}.wrote {} no {} wrote\\
\glt   ‘Yes. (= She hasn't.) / No. (= She has.)' 
}
\end{xlist}
\end{exe}

% In Czech, there are two response particles: \textit{ano} `yes' and \textit{ne} `no'. (as same as in English; by contrast, there are three in German - ja, nein, doch). Research about the meaning of bare \textit{ano, ne} is only at the beginning in Czech linguistics.\\

% However, one theoretical approach to the Czech response particles was created by \citet{gruet2016yes}. Her theory is also based on the existence of absolute and relative features, just like the Features model.  She claims that the negation in the previous question, especially the position of this negation, is essential for the meaning of bare \textit{ano, ne}. If the negation is in initial position (V1), it is neglected and uninterpreted (pleonastic negation). On the contrary, if it occurs in a non-initial position (non-V1), it is perceived as true and therefore interpreted. This negation is then reflected in the interpretation of the answer in terms of its features.\\

% Negative question with the V1 has the same meaning as positive question and thus answers to these two types of questions also have the same meaning. \citet{gruet2016yes} also claims that the polarity of the question is in this case empty (because it contains no negation), thus the polarity of the answer must be absolute (ano = [+] and ne = [$-$]). However, the negation is true and interpreted in negative questions with non-V1 and thus the polarity of the question is specified. The answer then agrees with this polarity or reverses it (ano = [\textsc{agree}] and ne = [\textsc{reverse}]). The application of these assumptions to real linguistic material can be seen in the following examples (\ref{hrdsim:ex:v1}, \ref{hrdsim:ex:non-v1}).\\

% \begin{exe}  \label{hrdsim:ex:v1}
% \ex  {Negative question with V1} \\
% \begin{xlist}
% \exi{A:}[]{ 
% \gll    Nenapsala Jitka esej?\\
%         \textsc{neg}.write Jitka essay\\
%         \glt ‘Hasn't Jitka written an essay?'}
% \exi{B:}[]{
% \gll Ano. (= Napsala.) / Ne. (= Nenapsala.)\\
% yes {} write {} no {} \textsc{neg}.write\\
% \glt   ‘Yes. (= She has.) / No. (= She hasn't.)' 
% }
% \end{xlist}
% \end{exe}

% \begin{exe} \label{hrdsim:ex:non-v1}
% \ex {Negative question with non-V1} \\
% \begin{xlist}
% \exi{A:}[]{ 
% \gll    Jitka esej nenapsala?\\
%         Jitka essay \textsc{neg}.write\\
%         \glt ‘Hasn't Jitka written an essay?'}
% \exi{B:}[]{
% \gll Ano. (= Nenapsala.) / Ne. (= Napsala.)\\
% yes {} \textsc{neg}.write {} no {} write\\
% \glt   ‘Yes. (= She hasn't.) / No. (= She has.)' 
% }
% \end{xlist}
% \end{exe}

% Gruet-Skrabal's predictions are compared with our results in \sectref{hrdsim:sec:discussion}.

% In Czech, the situation would be similar to that in English. \textit{Yes} after a negative antecedent either expresses its positive polarity [+] or agrees with the polarity of the negative antecedent [\textsc{agree}]. \textit{No}, on the contrary, either signals its negative polarity [$-$], or reverses the previous negative question/assertion [\textsc{reverse}] and acquires a positive polarity.

\subsection{Saliency account: \citet{hrd+:krifka13}}\label{hrdsim:sec:saliencyaccount}

\citeposst{hrd+:krifka13} \textsc{saliency account} takes an additional factor into account, namely the role of contextual and more generally pragmatic considerations, co-deter\-min\-ing which proposition is selected as the antecedent for the response particle.

In \citeposst{hrd+:krifka13} theory, response particles are propositional anaphors, not unlike pronouns.\footnote{In this respect, \citeposst{roelofsen-farkas15} and \citeposst{hrd+:krifka13} theories are similar. Both crucially build on an analogy with pronouns -- the former via pronominal-like presuppositions (not discussed here), the latter via the anaphoric potential of pronouns.} If a response particle is preceded by a question which contains negation, there are in principle two possible antecedents for the response particle: either the negative proposition or the negation's prejacent, i.e., the corresponding positive proposition. This is illustrated in \REF{hrdsim:ex:krifka1} (adapted from \citealt[14]{hrd+:krifka13}).\footnote{For questions with high negation (\textit{Didn't Ede steal the cookie?}), \citet[14]{hrd+:krifka13} assumes only one possible antecedent, namely the positive $d'$. The negation in this case is applied outside of the scope of the proposition, making it unavailable for anaphoric pickup.}

\ea \gll [\textsubscript{ActP} did REQUEST [\textsubscript{NegP} Ede not {[\textsubscript{TP} t\textsubscript{Ede}} t\textsubscript{did} steal the cookie]]]?\label{hrdsim:ex:krifka1}\\
{} {} {} {$\hookrightarrow d$} {} {} {$\hookrightarrow d'$}\\
\z

\noindent Response particles used in reaction to a question like \REF{hrdsim:ex:krifka1} can thus be interpreted as in \REF{hrdsim:ex:krifka}, capturing the ambiguity discussed above.

\ea\label{hrdsim:ex:krifka}
\ea \textit{Yes.} $\leadsto$ \cnst{assert}$(d′)$ $\approx$ \textit{Yes, he did!}\hfill (rejecting accent, with clause)
\ex \textit{Yes.} $\leadsto$ \cnst{assert}$(d)$ $\approx$ \textit{Yes, he didn't.} \hfill (natural, but with clause)
\ex \textit{No.} $\leadsto$ \cnst{assert}$(\neg d′)$ $\approx$ \textit{No (he didn’t).} \hfill (natural, clause not necessary)
\ex \textit{No.} $\leadsto$ \cnst{assert}$(\neg d)$ $\approx$ \textit{Well, he did!} \hfill (rejecting accent, with clause)
\z\z

\largerpage[-1]
\noindent What is of interest to us is how the antecedent of the response particle is selected, i.e., whether the response particle denotes $d$ (the negative proposition) or $d'$ (the positive proposition). \citet{hrd+:krifka13} assumes that the saliency of the propositions -- and hence the likelihood of their antecedent status -- can be modulated contextually. Example \REF{hrdsim:ex:krifka2} (adapted from \citealt[14]{hrd+:krifka13}) and the matching example \REF{hrdsim:ex:krifka2b} (created by us) illustrate this point. These examples differ in the question under discussion put on the table by A: in \REF{hrdsim:ex:krifka2}, the issue is negatively defined, in \REF{hrdsim:ex:krifka2b}, the issue is positively defined. Although the default antecedent for the response particles in both cases will be the negative proposition asserted by B, the context is assumed to modulate the availability of the positive antecedent, which leads to an increased likelihood of the truth-conditionally opposite responses in \REF{hrdsim:ex:krifka2b}.\footnote{An anonymous reviewer points out that the positive interpretation (`He climbed it') of response (\ref{hrdsim:ex:krifka2b}A$_1$) might be contingent on \textit{yes} being pronounced with a specific intonation. This is indeed what \citet{Goodhue.Wagner2018} confirmed experimentally; they call the fall rise intonation used in these cases ``contradiction contour'', following \citet{Liberman.Sag1974}.\label{fn:goodhue-wagner}}

\ea Negative context (italicized)\label{hrdsim:ex:krifka2}
\begin{xlist}
\exi{A:}{Which of the mountains on this list \textit{did Reinhold Messner not climb}?}
\exi{B:}{Well, let's see{\dots} He did not climb Mount Cotopaxi in Ecuador.}
\exi{A$_1$:}{Yes.\newline Likely: `He didn't climb it.'\\Unlikely: `He climbed it.'}
\exi{A$_2$:}{No.\\Likely: `He climbed it.'\\Unlikely: `He didn't climb it.'}
\end{xlist}
\z

\ea Positive context (italicized)\label{hrdsim:ex:krifka2b}
\begin{xlist}
\exi{A:}{Which of the mountains on this list \textit{did Reinhold Messner climb}?}
\exi{B:}{Well, let's see\dots He did not climb Mount Cotopaxi in Ecuador.}
\exi{A$_1$:}{Yes.\\Likely: `He didn't climb it.'\\More likely than in \REF{hrdsim:ex:krifka2}: `He climbed it.'}
\exi{A$_2$:}{No.\\Likely: `He climbed it.'\\More likely than in \REF{hrdsim:ex:krifka2}: `He didn't climb it.'}
\end{xlist}
\z

\noindent One of our experiments (experiment 1) will tap not only into the basic meaning of Czech response particles, which can be formulated in terms of the feature model, but also into the influence of the context in determining the antecedent of the response particles.

% \citeposst{hrd+:krifka13} approach presumes that response particles are not related to the prejacent, which is anaphoric to the antecedent, but they are themselves anaphors. They introduce discourse referents related to the salient proposition. \citet{hrd+:krifka13} then distinguishes three types of discourse referents which correspond to three different syntactic projections. \\

% Events are given \textit{v}P and semantically they are event predicates that exist in a temporal or modal relation to the world. In a concrete speech act, the speaker is responsible for the truth value of the proposition, so it is not the event itself that becomes important, but whether the proposition is true. \citet{hrd+:krifka13} also considers this change from a structure to a concrete expression to be an event and assumes that it corresponds to a distinctive layer in the syntax called ActP. According to \citet{hrd+:krifka13}, the last distinctive layer introduced by the discourse referent is negation (NegP). It introduces a second proposition which blocks the original proposition. That is why the sentence \textit{Bill saw it}. is not possible after the statement \textit{Ede didn't steal the cookie.} \citep[3--5]{hrd+:krifka13}\\

% In natural language, there are various types of expressions that can indicate discourse referents. However, in this work we will consider only the answer expressions \textit{ano, ne} (in English ‘yes’, ‘no’).\\
% A negative antecedent (\citet{hrd+:krifka13} uses a negative statement for example, but for a negative question the results would be the same) thus states two discourse referents: a proposition (d') and its negation (d), as can be seen in Fig. (\ref{hrdsim:fig:krifka})\\

% \begin{exe}
% \ex \label{hrdsim:fig:krifka}
% \includegraphics{Pictures/Krifka .png}
% \colorbox{yellow}{(OBRÁZEK JE POTŘEBA PŘEPSAT DO KÓDU)}    
% \end{exe}

% \textit{Yes} or \textit{no} answer to the given statement can therefore result in four analyzes \citet[12]{hrd+:krifka13}:


% He explains the preferences on the basis of two principles \citep[13]{hrd+:krifka13}. The first of them claims that disagreement with the polarity of the antecedent must be explicitly stated. The second principle is based on the salience of propositions in certain contexts. The assumption of the principle is based on the fact that the negative question occurs more often in contexts where the unnegated proposition is salient. (E.g. \textit{Ede stole the cookie.} as one of the possibilities why the cookie is missing.)\\

% Based on these principles, \citet[14]{hrd+:krifka13} concludes that in response to negative questions after a positive context, the preferred meaning of \textit{yes} will rather be its reversion (Yes, he did.) than its confirmation (Yes, he didn't). On the contrary, \textit{no} will be understood as a confirmation of the negative antecedent (No, he didn't). For the opposite meaning, the given response particle would have to be supplemented with an appropriate accompanying clause. \\

% In the case of the preceding negative context, the opposite will be the case. The experiment examines the validity of the given assumption for the Czech language. The predictions are given below.


\section{Experiments} \label{hrdsim:sec:experiment}
The aim of our experiments was to find out the preferred meaning of \textit{ano} `yes' and \textit{ne} `no' in response to polar questions and test the above-mentioned approaches, in particular \citeposst{gruet2016yes} version of the feature-based analysis and \citeposst{hrd+:krifka13} idea that the choice of the response particle an\-te\-ced\-ent is context-dependent. The experimental design was inspired by \citet{kramer2012ellipsis} and \citet{claus2017puzzling}, who investigate the meaning of response particles in English and German.
% \begin{enumerate} \label{hrdsim:researchquestions}
%     \item What is the predominant meaning of \textit{ano} ‘yes’ and \textit{ne} ‘no’ in Czech in response to questions with negative polarity?
%     \item To what extent is the choice of a particular response particle influenced by the previous context?
%     \item Which semantic theory of response particles provides a better match for the Czech data?
%     \item Does \citeposst{gruet2016yes} analysis make the correct predictions?
% \end{enumerate}

Our results suggest a relative ([\textsc{agree}]-based) semantics for \textit{ano} `yes' and absolute ([$-$]-based) semantics for \textit{ne} `no'. This can be easily modeled using the feature model. The particular predictions of \citet{gruet2016yes} were, however, not borne out: we do not see evidence for relative ([\textsc{reverse}]-based) semantics for \textit{ne} `no'. In addition, in a particular corner of our data, we see a pattern which is predicted by \citeposst{hrd+:krifka13} saliency account.

We used 2 experiments combined in a single setup, such that each of the two experiments provided fillers for the other one. This setup makes it possible to draw inferences cross-experimentally. The more complex and powerful experiment 1 uses a $2\times 2\times 2$ design and investigates responses to syntactically interrogative negative polar questions. Experiment 2 uses a $2\times 2$ design and investigates responses to syntactically declarative negative polar questions.

We first describe aspects common to the two experiments (see \sectref{hrdsim:sec:method}) and then turn to the individual experiments (\sectref{hrdsim:sec:e1}--\sectref{hrdsim:sec:e2}).

\subsection{Aspects common to both experiments} \label{hrdsim:sec:method}

As detailed in \tabref{hrdsim:tab:exp.over}, our experimental setup consisted of two experiments with 16 and 8 items, respectively, and an additional set of 16 filler items, giving a total of 40 items. The number matches the number of stimuli seen by each participant.

\begin{table}%[]
\begin{tabular}{lr}
\lsptoprule
Experiment 1 & 16 \\ [0.4em]
Experiment 2 & 8\\ [0.4em]
Fillers & 16\\ [0.4em]
Total & 40\\
\lspbottomrule
\end{tabular}
    \caption{Overall experimental setup}
    \label{hrdsim:tab:exp.over}
\end{table}

\subsubsection{Task, procedure, and dependent variables}

The participants were exposed to written stimuli which consisted of a short narrative (a few sentences) followed by a short dialogue between two people (A and B for ease of reference), in which A opens the dialogue with an assertion associated with the narrative, B asks a relevant polar question, and A responds by saying either `yes' or `no'. The narrative and the dialogues contained mildly colloquial elements, in order to simulate an informal setting -- a dialogue between two friends. The participant's task was twofold: (i) to determine whether A's `yes'/`no' response is consistent with the information provided in the narrative (i.e., a truth-value judgment task) and (ii) to rate the naturalness of that response given the preceding narrative and dialogue (on a scale from 1 = completely unnatural to 7 = completely natural). In this paper, we analyze the responses from task (i) and leave (ii) aside. This is mainly because there was a strong correlation between the two in the sense that responses which were judged as consistent with the information provided were also rated as natural and conversely -- responses judged as inconsistent were rated as unnatural.

\subsubsection{Participants}

Data from 66 adult native speakers of Czech (43 women, 23 men) entered the analyses. We used convenience sampling, recruiting participants from an extended social network of the first author. Most of the participants (44) were 18--29 years old and most (43) had university education. All participants were informed about the purpose of the experiment and all gave informed consent to participate in the experiment and the subsequent processing and anonymous publication of the collected data.

The analyzed sample consists of participants who have passed a preset quality measurement, namely scoring in the expected way on variable (i) in at least \qty{75}{\percent} of the 8 filler items, where the relation between the information provided (in the narrative) and the `yes'/`no' response was particularly transparent.

\subsubsection{Software and administration}

The experiments were prepared and administered using the L-Rex software \citep{strachenko23}. The stimuli from each experiment were distributed on lists using the Latin Square design, so that one participant saw only a single stimulus from each item. The lists from each experiment were then combined and the order of presentation was pseudo-randomized in such a way that two stimuli from a single experiment never directly followed one another and two stimuli from a single condition were always interspersed by at least one stimulus from a different condition.

The experiment was distributed online by sending a link. Participants took part at their own personal computers and most of them needed 25--40 minutes to complete the experiment.

\subsection{Experiment 1: Syntactically interrogative negative polar questions}\label{hrdsim:sec:e1}

\subsubsection{Design and manipulated variables}

This experiment focuses on the most complex and problematic case: responses to negative polar questions which are syntactically interrogative, which means that the finite verb is located in the clause-initial position (V1 for short); see \REF{hrdsim:ex:negV1}. Since negation is obligatorily prefixed to the verb in Czech, there is no reliable formal difference between high and low negation (cf. \citealt{hrd+:Ladd1981}) and its semantic correlate outer (extra-propositional) vs. inner (propositional) negation \citep{hrd+:AnderBois2019,Goodhue2022}. Yet there is a general consensus that negation in V1 polar questions in Czech corresponds to high negation, which is either pleonastic (expletive) or applies at an illocutionary level (\citealt{hrd+:Repp2013}; for a discussion on Czech, see \citealt{Stankova2023}, \citealt{Stankova.Simik2024}, and the references cited therein).

\ea \gll Neprodala Jitka ty staré boty?\label{hrdsim:ex:negV1}\\
\textsc{neg}.sold Jitka \textsc{dem} old shoes\\
\glt `Didn't Jitka / Did Jitka not sell the old shoes?' 
\z

\noindent In a factorial $2\times 2\times 2$ design, we manipulated three variables (all within items and subjects): \textsc{information}, \textsc{context}, and \textsc{response}.
The \textsc{information} variable, with values \textsf{positive} (\textsf{i\_pos}) and \textsf{negative} (\textsf{i\_neg}) is manipulated in the lead-in narrative and fixes the factual state of affairs relative to which the participant judges the consistency of the response. The \textsc{context} variable, likewise with values \textsf{positive} (\textsf{c\_pos}) and \textsf{negative} (\textsf{c\_neg}), was manipulated in the first utterance of the dialogue, which is then followed by the polar question. Finally, the \textsc{response} variable has the values \textit{ano} (\textsf{yes}) and \textit{ne} (\textsf{no}) and is manipulated in the final utterance of the dialogue. For purposes of visualization and statistical analysis, we have found it useful to include an auxiliary variable, namely the \textsc{accordance} between \textsc{information} and \textsc{response}, yielding the value \textsf{accord} for the cases where the \textsf{positive} information is matched by a \textsf{yes} response and \textsf{negative} information by a \textsf{no} response, and the value \textsf{discord} where this is not so. \tabref{hrdsim:tab:exp1-cond} provides an overview of all the 8 unique conditions of experiment 1.

\begin{table}[t]
    \begin{tabularx}{.8\textwidth}{llXXl}
    \lsptoprule
         &\textsc{information}&\textsc{context}&\textsc{response}&\textsc{accordance}\smallskip\\
         % &within items&within items&within items\\
         % &within subjects&within subjects&between subjects\smallskip\\
         % &narrative&1st dialogue utterance&last dialogue utterance\\
         \midrule
a&\textsf{i\_pos}&\textsf{c\_pos}&\textsf{yes}&\textsf{accord}\\
b&\textsf{i\_pos}&\textsf{c\_pos}&\textsf{no}&\textsf{discord}\\
c&\textsf{i\_pos}&\textsf{c\_neg}&\textsf{yes}&\textsf{accord}\\
d&\textsf{i\_pos}&\textsf{c\_neg}&\textsf{no}&\textsf{discord}\\
e&\textsf{i\_neg}&\textsf{c\_pos}&\textsf{yes}&\textsf{discord}\\
f&\textsf{i\_neg}&\textsf{c\_pos}&\textsf{no}&\textsf{accord}\\
g&\textsf{i\_neg}&\textsf{c\_neg}&\textsf{yes}&\textsf{discord}\\
h&\textsf{i\_neg}&\textsf{c\_neg}&\textsf{no}&\textsf{accord}\\
    \lspbottomrule
    \end{tabularx}
    \caption{Conditions in the factorial design of experiment 1}
    \label{hrdsim:tab:exp1-cond}
\end{table}

% We manipulated two dependent variables: a) the consistency of the Answer with the Information and b) the naturalness of the Answer. The first variable was binary (agreement vs. disagreement) and the second variable pseudocontinuous (seven-point Likert scale). In the main experiment, three independent variables were also manipulated. Each of them could take on two values, which added up to eight unique conditions. Manipulation of the given Information was used to make the \textit{ano/ne} meaning clear and to make it possible to follow their more frequent meaning after the negative antecedent. Manipulation of Context is essential for the Saliency account, manipulation of Response for both approaches. In both filler experiments, there was no manipulation of Context.\\

% Each experimental stimulus had the following structure:
% \begin{enumerate}
%     \item {Introduction} (narrator) containing basic information about what the conversation is about and Information available to one of the dialogue partner (positive × negative). If it is necessary for the coherence of the text, the introductory part contains also a short Introduction 2 serving to connect the introductory part with the following one.
%     \item {A short dialogue between two people} (A and B) consisting of three utterances – A: introduction of the topic (= Context, positive × negative), B: negative question (constant across conditions), A: answer (= Answer, Ano × Ne).
% \end{enumerate}

\subsubsection{Materials}

Example \REF{hrdsim:ex:item} provides an example of an item (particularly item 14) in all eight conditions. The values of the \textsc{information} variable is set in small caps, the \textsc{context} in italics, and the \textsc{response} in boldface. The parts that remained constant across the manipulations -- including the negative polar question -- are set in ordinary typeface. The value of the \textsc{information} variable was located in the lead-in narrative, specifically in the position indicated by [\dots].\footnote{All the experiment materials, results, and outputs of statistical models are available at Open Science Framework under the following link: \url{https://doi.org/10.17605/OSF.IO/9VXJS}.}




% Introduction part
% Setting 1: Eva a Lída se zúčastnily vánočního plesu ve svém rodném městě. Lída, která se velmi zajímá o společenský život ve svém rodišti,
% Positive Information: ví, že hlavní organizátorkou byla jejich bývalá spolužačka a zkušená organizátorka plesů Alice}. Když se po nějaké době potkají, probírají spolu proběhlý ples.
% Negative Information: ví, že jejich bývalá spolužačka a zkušená organizátorka plesů Alice se tentokrát na organizaci nepodílela.

% Dialogue:
% Positive context: Lída: Ten ples se jim moc povedl.
% Negative context: Lída: Ten ples se jim moc nepovedl.

% Question: Eva: Neorganizovala ho Alice?

% Yes answer: Lída: Ano.
% No answer: Lída: Ne.

% In conditions a-d, there was positive information and in conditions e-h negative one. Positive context was in conditions a, b, e, f and negative in c, d, g, h. Yes answer in conditions a, c, e, g and No answer in b, d, f, h.

% \colorbox{yellow}{Ten úvod mi takhle přijde hrozně nepřehledný, ale pořádně nemůžu dohledat, jak bych to upravila. Zároveň si říkám, jestli by nestálo za to sem dát tu položku přeloženou do angličtiny s poznámkami, o jaký druh Kontaxtu nebo Informace se jedná a ukázku podmínky v češtině dát až třeba do závěrečných poznámek.}

\begin{exe}
\ex
 \gll Eva a Lída se zúčastnily vánočního plesu ve svém rodném městě. Lída,
která se velmi zajímá o společenský život ve svém {rodišti, [\dots]} Když se po nějaké době potkají, probírají spolu proběhlý ples.\\
Eva and Lída \textsc{refl} took.part Christmas ball in their birth town Lída who \textsc{refl} a.lot interest in social life in her birthplace when \textsc{refl} after some time meet discuss together passed ball\\
\glt ‘Eva and Lída attended a Christmas ball in their hometown. Lída, who is interested in the social life in her hometown very much, [\dots] When Eva and Lída meet after a while, they discuss the ball together.’\label{hrdsim:ex:item}

\begin{xlist}
\exi{a.}{\gll
\textsc{ví}, \textsc{že} \textsc{hlavní} \textsc{organizátorkou} \textsc{byla} \textsc{jejich} \textsc{bývalá} \textsc{spolužačka} \textsc{a} \textsc{zkušená} \textsc{organizátorka} \textsc{plesů} \textsc{Alice}.\\
knows that main organizer was their former classmate and experienced organizer balls Alice\\
\glt `knows that the main organizer was their former classmate and an experienced ball organizer Alice.'\label{hrdsim:ex:item-a}
}
\exi{}{\gll Lída: \textit{Ten} \textit{ples} \textit{se} \textit{jim} \textit{moc} \textit{povedl}.\\
{} \textsc{dem} ball \textsc{refl} them much worked.out\\
\glt ~~~~~~~~~`The ball worked out really well.'}
\exi{}{\gll Eva: Neorganizovala ho Alice?\\
{} \textsc{neg}.organized it Alice\\
\glt ~~~~~~~~`Didn't Alice organize it?'}
\exi{}{\gll Lída: \textbf{Ano.}\\
{} yes\\
\glt ~~~~~~~~~`Yes.'}

\exi{b.}{\gll
\textsc{ví}, \textsc{že} \textsc{hlavní} \textsc{organizátorkou} \textsc{byla} \textsc{jejich} \textsc{bývalá} \textsc{spolužačka} \textsc{a} \textsc{zkušená} \textsc{organizátorka} \textsc{plesů} \textsc{Alice}.\\
knows that main organizer was their former classmate and experienced organizer balls Alice\\
\glt `knows that the main organizer was their former classmate and an experienced ball organizer Alice.'\label{hrdsim:ex:item-b}
}
\exi{}{\gll Lída: \textit{Ten} \textit{ples} \textit{se} \textit{jim} \textit{moc} \textit{povedl}.\\
{} \textsc{dem} ball \textsc{refl} them much worked.out\\
\glt ~~~~~~~~~`The ball worked out really well.'}
\exi{}{\gll Eva: Neorganizovala ho Alice?\\
{} \textsc{neg}.organized it Alice\\
\glt ~~~~~~~~`Didn't Alice organize it?'}
\exi{}{\gll Lída: \textbf{Ne.}\\
{} no\\
\glt ~~~~~~~~~`No.'}

\exi{c.}{\gll
\textsc{ví}, \textsc{že} \textsc{hlavní} \textsc{organizátorkou} \textsc{byla} \textsc{jejich} \textsc{bývalá} \textsc{spolužačka} \textsc{a} \textsc{zkušená} \textsc{organizátorka} \textsc{plesů} \textsc{Alice}.\\
knows that main organizer was their former classmate and experienced organizer balls Alice\\
\glt `knows that the main organizer was their former classmate and an experienced ball organizer Alice.'\label{hrdsim:ex:item-c}
}
\exi{}{\gll Lída: \textit{Ten} \textit{ples} \textit{se} \textit{jim} \textit{moc} \textit{nepovedl}.\\
{} \textsc{dem} ball \textsc{refl} them much \textsc{neg}.worked.out\\
\glt ~~~~~~~~~`The ball worked out really well.'}
\exi{}{\gll Eva: Neorganizovala ho Alice?\\
{} \textsc{neg}.organized it Alice\\
\glt ~~~~~~~~`Didn't Alice organize it?'}
\exi{}{\gll Lída: \textbf{Ano.}\\
{} yes\\
\glt ~~~~~~~~~`Yes.'}

\exi{d.}{\gll
\textsc{ví}, \textsc{že} \textsc{hlavní} \textsc{organizátorkou} \textsc{byla} \textsc{jejich} \textsc{bývalá} \textsc{spolužačka} \textsc{a} \textsc{zkušená} \textsc{organizátorka} \textsc{plesů} \textsc{Alice}.\\
knows that main organizer was their former classmate and experienced organizer balls Alice\\
\glt `knows that the main organizer was their former classmate and an experienced ball organizer Alice.'\label{hrdsim:ex:item-d}
}
\exi{}{\gll Lída: \textit{Ten} \textit{ples} \textit{se} \textit{jim} \textit{moc} \textit{nepovedl}.\\
{} \textsc{dem} ball \textsc{refl} them much \textsc{neg.}worked.out\\
\glt ~~~~~~~~~`The ball worked out really well.'}
\exi{}{\gll Eva: Neorganizovala ho Alice?\\
{} \textsc{neg}.organized it Alice\\
\glt ~~~~~~~~`Didn't Alice organize it?'}
\exi{}{\gll Lída: \textbf{Ne.}\\
{} no\\
\glt ~~~~~~~~~`No.'}

\exi{e.}{\gll
\textsc{ví}, \textsc{že} \textsc{jejich} \textsc{bývalá} \textsc{spolužačka} \textsc{a} \textsc{zkušená} \textsc{organizátorka} \textsc{plesů} \textsc{Alice} \textsc{se} \textsc{tentokrát} \textsc{na} \textsc{organizaci} \textsc{nepodílela}.\\
knows that their former classmate and experienced organizer balls Alice \textsc{refl} this.time in organization \textsc{neg}.was.involved\\
\glt `knows that their former classmate and an experienced ball organizer Alice wasn't involved in the organization this time.'\label{hrdsim:ex:item-e}
}
\exi{}{\gll Lída: \textit{Ten} \textit{ples} \textit{se} \textit{jim} \textit{moc} \textit{povedl}.\\
{} \textsc{dem} ball \textsc{refl} them much worked.out\\
\glt ~~~~~~~~~`The ball worked out really well.'}
\exi{}{\gll Eva: Neorganizovala ho Alice?\\
{} \textsc{neg}.organized it Alice\\
\glt ~~~~~~~~`Didn't Alice organize it?'}
\exi{}{\gll Lída: \textbf{Ano.}\\
{} yes\\
\glt ~~~~~~~~~`Yes.'}

\exi{f.}{\gll
\textsc{ví}, \textsc{že} \textsc{jejich} \textsc{bývalá} \textsc{spolužačka} \textsc{a} \textsc{zkušená} \textsc{organizátorka} \textsc{plesů} \textsc{Alice} \textsc{se} \textsc{tentokrát} \textsc{na} \textsc{organizaci} \textsc{nepodílela}.\\
knows that their former classmate and experienced organizer balls Alice \textsc{refl} this.time in organization \textsc{neg}.was.involved\\
\glt `knows that their former classmate and an experienced ball organizer Alice wasn't involved in the organization this time.'\label{hrdsim:ex:item-f}
}
\exi{}{\gll Lída: \textit{Ten} \textit{ples} \textit{se} \textit{jim} \textit{moc} \textit{povedl}.\\
{} \textsc{dem} ball \textsc{refl} them much worked.out\\
\glt ~~~~~~~~~`The ball worked out really well.'}
\exi{}{\gll Eva: Neorganizovala ho Alice?\\
{} \textsc{neg}.organized it Alice\\
\glt ~~~~~~~~`Didn't Alice organize it?'}
\exi{}{\gll Lída: \textbf{Ne.}\\
{} no\\
\glt ~~~~~~~~~`No.'}

\exi{g.}{\gll
\textsc{ví}, \textsc{že} \textsc{jejich} \textsc{bývalá} \textsc{spolužačka} \textsc{a} \textsc{zkušená} \textsc{organizátorka} \textsc{plesů} \textsc{Alice} \textsc{se} \textsc{tentokrát} \textsc{na} \textsc{organizaci} \textsc{nepodílela}.\\
knows that their former classmate and experienced organizer balls Alice \textsc{refl} this.time in organization \textsc{neg}.was.involved\\
\glt `knows that their former classmate and an experienced ball organizer Alice wasn't involved in the organization this time.'\label{hrdsim:ex:item-g}
}
\exi{}{\gll Lída: \textit{Ten} \textit{ples} \textit{se} \textit{jim} \textit{moc} \textit{nepovedl}.\\
{} \textsc{dem} ball \textsc{refl} them much \textsc{neg}.worked.out\\
\glt ~~~~~~~~~`The ball worked out really well.'}
\exi{}{\gll Eva: Neorganizovala ho Alice?\\
{} \textsc{neg}.organized it Alice\\
\glt ~~~~~~~~`Didn't Alice organize it?'}
\exi{}{\gll Lída: \textbf{Ano.}\\
{} yes\\
\glt ~~~~~~~~~`Yes.'}

\exi{h.}{\gll
\textsc{ví}, \textsc{že} \textsc{jejich} \textsc{bývalá} \textsc{spolužačka} \textsc{a} \textsc{zkušená} \textsc{organizátorka} \textsc{plesů} \textsc{Alice} \textsc{se} \textsc{tentokrát} \textsc{na} \textsc{organizaci} \textsc{nepodílela}.\\
knows that their former classmate and experienced organizer balls Alice \textsc{refl} this.time in organization \textsc{neg}.was.involved\\
\glt `knows that their former classmate and an experienced ball organizer Alice wasn't involved in the organization this time.'\label{hrdsim:ex:item-h}
}
\exi{}{\gll Lída: \textit{Ten} \textit{ples} \textit{se} \textit{jim} \textit{moc} \textit{nepovedl}.\\
{} \textsc{dem} ball \textsc{refl} them much \textsc{neg.}worked.out\\
\glt ~~~~~~~~~`The ball worked out really well.'}
\exi{}{\gll Eva: Neorganizovala ho Alice?\\
{} \textsc{neg}.organized it Alice\\
\glt ~~~~~~~~`Didn't Alice organize it?'}
\exi{}{\gll Lída: \textbf{Ne.}\\
{} no\\
\glt ~~~~~~~~~`No.'}
\end{xlist}
\end{exe}

\noindent 16 items like \REF{hrdsim:ex:item} were created, meaning that each participant was exposed to each unique condition twice (following the Latin Square distribution we used; see \sectref{hrdsim:sec:method}). This number -- admittedly not great judging by current standards \citep{Haussler.Juzek2017} -- resulted from a compromise between statistical power considerations and the significant cognitive load imposed by the task on the participants. The introductory narrative was always presented in stylistically neutral language and the dialogues occasionally contained colloquial expressions. The information at issue (above: whether Alice organized the ball) is known to the first dialogue participant (above: Lída), but not to the second one (above: Eva). The first participant makes a claim relevant to the information at issue (above: how the ball worked out), but does not reveal its value. The first participant's utterance stands in a particular relation to the information: it makes it more likely or less likely. The second participant asks a question about the information, followed by a response from the first participant.

Consider (\ref{hrdsim:ex:item-a}a) for illustration. In this condition it is the case that Lída (the first dialogue participant) knows that Alice organized the ball. Also, as the narrative implies, both Lída and Eva (the second dialogue participant) are aware that Alice is a good ball organizer. Lída's first utterance in the dialogue -- that the ball worked out really well -- implies that the ball was organized by Alice (a case of evidential bias). Eva then asks a polar question, in order to verify or falsify the implication. Lída responds \textit{ano} `yes'. Setting the naturalness rating aside, the participant had the option of either saying that Lída's response is consistent with the information provided (i.e., the response is true), or saying that Lída's response is not consistent with the information provided (i.e., the response is false). In the former case, we assume that the participant either interprets the response absolutely (feature [$+$] / positive polarity = `Alice organized the ball.') or relatively (feature [\textsc{agree}] / agreement with the antecedent `Alice organized the ball' -- made available by the prejacent of the negative polar question asked). In the latter case, the participant interpreted Lída's response as involving the relative feature [\textsc{agree}], agreeing with the negative antecedent -- `Alice didn't organize the ball' -- made available by the prejacent of the polar questions, including the negation. Whether the positive or the negative form of the antecedent is more salient (and hence whether the response is considered as true or false, provided it is interpreted relatively) is, by hypothesis \citep{hrd+:krifka13}, co-determined by the context -- Lída's first utterance (more technically: evidential bias).


% \subsubsection{Predictions} \label{hrdsim:sec:predictions}
% \citet[390]{roelofsen-farkas15} assume that since [\textsc{agree}, --] does not require additional expression, the bare response particle after a negative question will be understood as [\textsc{agree}, --] rather than [\textsc{reverse}, +], regardless of context.

% Relative to our experiment, this means that any response in conversations with negative Information should be consistent with that Information, whereas in conversations with positive Information it should be inconstistent. See \figref{hrdsim:prediction:RF}.

% \begin{exe}
% \ex \label{hrdsim:prediction:RF}
% \includegraphics[width = 30 em]{Pictures/RF.pdf}
% \end{exe}

% \citet[13]{hrd+:krifka13}, on the other hand, assumes that the context will play a large role in determining the meaning of the answers. According to him, negative questions usually appear after a context where the non-negated discourse referent will be more salient than the negated one. \\
% The basic assumption for making predictions was that meaning of \textit{ano} as well as the English ‘yes’ will be understood more as a reversion of the negative after the positive context  (Yes, he came.), and \textit{ne} will be interpreted rather as its confirmation (No, he didn't come). In the case of a previous negative context, there will be reversed tendency (Yes, he didn't come. vs. No, he came.). Predictions for individual conditions are shown in \figref{hrdsim:prediction:Krifka}

% \begin{exe}
% \ex \label{hrdsim:prediction:Krifka}
% \includegraphics[width = 30 em]{Pictures/Krifka.pdf}
% \end{exe}

% At the end of this section, we present a summary table for comparing the predictions of both approaches.\\

% \begin{table}[]
%     \centering
% \begin{tabular}{lcc} 
% \lsptoprule
% \multicolumn{3}{c}{Didn’t John buy a car?}\\
% \hline
% & Positive context & Negative context\\
% Features model & Yes. [= He didn’t.] > Yes. [= He did.] & Yes. [= He didn’t.] > Yes. [= He did.]\\
% & No. [= He didn’t.] No. [= He did.] & No. [= He didn’t .] > No . [= He did.]\\
% Saliency account & Yes. [= He did.] > Yes. [= He didn’t] & Yes. [= He didn’t.] > Yes. [= He did.]\\
% & No. [= He didn’t.] > No. [= He did.] & No. [= He did .] > No. [= He didn’.t]\\
% \lspbottomrule
% \end{tabular}
%     \caption{Summary of predictions of both approaches}
%     \label{hrdsim:tab:predictions}
% \end{table}

% In relation to our experiment, tha data in the table mean that, according to \citet{roelofsen-farkas15}, the Answer will be consistent with the Information in conditions e--h, whereas \citet{hrd+:krifka13} predicts consistency for conditions a, d, e, h.\\


\subsubsection{Results} \label{hrdsim:sec:results-e1}

\figref{hrdsim:fig:results} shows the raw results of experiment 1, in particular of the ratings of consistency between the \textsc{response} (\textsf{yes} vs. \textsf{no}) and the \textsc{information} provided (\textsf{i\_pos} vs. \textsf{i\_neg}). \figref{hrdsim:fig:emmeans-e1} provides the corresponding 95\% confidence intervals (computed with the emmeans function of the emmeans package of R; \citealt{Lenth2024}). The results are visualized -- and also statistically analyzed -- using the auxiliary \textsc{accordance} variable. The value \textsf{accord} combined with the value \textsf{i\_pos} equals the value \textsf{yes}, combined with the value \textsf{i\_neg} equals \textsf{no}, and conversely for \textsf{discord}. The values of the \textsf{response} variable (\textsf{yes} vs. \textsf{no}) are indicated in the top left corner of each of the four panes for clarity. The reason for using \textsc{accordance} rather than \textsc{response} is that from the perspective of the results, the levels of the former variable form more of a natural class than the levels of the latter variable. The results are thus easier to evaluate and interpret.\footnote{See the general discussion (\sectref{hrdsim:sec:discussion}) for a visualization using the \textsc{response} variable.}

\begin{figure}
%     \begin{tikzpicture}
%     \pgftext{
    \includegraphics[height=.45\textheight]{figures/hrd-ano_ne-results-main.pdf}
%     }
%     \node[fill=black, opacity=.2, text opacity=1] at (-3.2,3.35) {\textsf{yes}};
%     \node[fill=black, opacity=.2, text opacity=1] at (-0.25,3.35) {\textsf{no}};
%     \node[fill=black, opacity=.2, text opacity=1] at (-3.25,-0.45) {\textsf{no}};
%     \node[fill=black, opacity=.2, text opacity=1] at (-0.2,-0.45) {\textsf{yes}};
%     \end{tikzpicture}
    \caption{Experiment 1: Response--information consistency ratings}
    \label{hrdsim:fig:results}
\end{figure}

\begin{figure}
    \includegraphics[height=.45\textheight]{figures/hrd-ano-ne_results_ci95.pdf}
    \caption{Experiment 1: 95\% confidence intervals of consistency ratings}
    \label{hrdsim:fig:emmeans-e1}
\end{figure}


Looking at the dependent variable, we note that consistency value 1 indicates that the participant considered the response to be consistent with the information provided, or, in other words, true relative to the information provided. Consistency value 0 indicates a judgment of falsity.

To give a particular example, the top right panel shows that a \textsf{no} response after \textsf{negative} information (corresponding to \textsf{accord} / conditions f/g of our design; see \tabref{hrdsim:tab:exp1-cond} and (\ref{hrdsim:ex:item}f/g)) was considered consistent with the information in about \qty{78}{\percent} of the cases; on the other hand, the bottom left panel shows that a \textsf{no} response after \textsf{positive} information (corresponding to \textsf{discord} / conditions b/d) was considered consistent with the information in only about \qty{15}{\percent} of the cases.

% For statistical analysis, we used R software, namely two mixed effects models. A generalized linear mixed model was used for the variable Consistency with the Information and a cumulative link mixed model was used for the variable Naturalness. The response particle factor was also recoded in terms of accordance with the Information. The \textit{accord} label was used for dialogues with positive Information and \textit{ano} for dialogues with negative Information and \textit{ne}. Conversely, \textit{discord} was used for dialogues with positive Information and \textit{ne} and for dialogues with negative Information and \textit{ano}. \\

We fitted a generalized linear mixed model, using the glmer function of the lme4 package \citep{Bates.etal2015} of the R software \citep{rcore}, to estimate the effect of \textsc{information, context, accordance,} and their mutual interactions on the consistency rating. We included random intercepts and slopes for both items and participants; \textsc{information} and \textsc{context} were sum-coded, \textsc{accordance} treatment-coded (using \textsf{accord} as the reference level).\footnote{The particular formula used was: \textsc{consistency rating} $\sim$ \textsc{information} $*$ \textsc{context} $*$ \textsc{accordance} + (1 + \textsc{information} + \textsc{context} + \textsc{accordance} | participant) + (1 + \textsc{information} + \textsc{context} + \textsc{accordance} | item). Treatment coding was used for \textsc{accordance} because it has a natural reference level (\textsf{accord}) at which we expected high consistency (as compared to the \textsf{discord} level). Such a clear relationship was absent in the other factors, for which reason we applied sum coding to them.\label{hrdsim:fn:coding}} The model confirms the naked-eye-visible effect of \textsc{accordance}: responses which were in accord with the information (the top row in \figref{hrdsim:fig:results}) were rated as true much more often than responses which were in discord with the information ($z=-6.580,p<.001$). For instance, a positive response after positive information (see (\ref{hrdsim:ex:item}a)) was rated as more consistent than after negative information (see (\ref{hrdsim:ex:item}e)). In addition, the model revealed a significant main effect of \textsc{information} ($z=-3.660,p<.001$; not easily interpretable), an interaction between \textsc{information} and \textsc{accordance} ($z=-3.849,p<.001$), between \textsc{information} and \textsc{context} ($z=3.627,p<.001$), and a three-way interaction between all factors ($z=-2.399,p=.016$). The interaction between \textsc{information} and \textsc{accordance} indicates that the effect of \textsc{information} is more pronounced in the \textsf{discord} level of \textsc{accordance}. The interaction between \textsc{information} and \textsc{context} is only visible if the response was in accord with the information (the top pane in \figref{hrdsim:fig:results}), which is also indicated by the significant three-way interaction. In order to see the effect \textsc{context} in a clearer way, we fitted a model onto the \textsf{accord} data subset, including random intercepts and slopes for both items and participants and sum-coding for both predictors -- \textsc{information} and \textsc{context}.\footnote{The formula used was: \textsc{consistency rating} $\sim$ \textsc{information} $*$ \textsc{context} + (1 + \textsc{information} $*$ \textsc{context} | participant) + (1 + \textsc{information} $*$ \textsc{context} | item).} This model confirmed the aforementioned interaction ($z=3.584,p<.001$), and a further statistical analysis (nesting \textsc{context} within the levels of \textsc{information}) revealed that its source is both in \textsf{i\_pos} and \textsf{i\_neg}: if the information was positive, responses were rated as true more often if the context was also positive, as in (\ref{hrdsim:ex:item}a) as opposed to (\ref{hrdsim:ex:item}c) (simple effect of \textsc{context} within \textsf{i\_pos}; $z=1.999,p=.046$), and if the information was negative, responses were rated as true more often if context was also negative, as in (\ref{hrdsim:ex:item}h) as opposed to (\ref{hrdsim:ex:item}f) (simple effect of \textsc{context} within \textsf{i\_neg}; $z=-3.389,p<.001$).\footnote{The formula used for the last model was: \textsc{consistency rating} $\sim$ \textsc{information} $/$ \textsc{context} + (1 + \textsc{information} $*$ \textsc{context} | participant) + (1 + \textsc{information} $*$ \textsc{context} | item).
%The strength of the simple effect of \textsc{context} (stronger in \textsf{i\_neg} than in \textsf{i\_pos}) is not aligned with the numerical difference in consistency rating proportions (greater in \textsf{i\_pos}, \qty{17}{\percent}, than in \textsf{i\_neg} \qty{4}{\percent}). The reason for this is the much greater variance among participants in \textsf{i\_pos} as compared to \textsf{i\_neg}.
}

%Analysis of the results showed that participants rated the Answer as consistent significantly more often if it was also in accordance with the Information – that is, \textit{ano} with positive Information and \textit{ne} with negative Information (main effect of agreement with Information; $z = 13.210, p < 0.0001$) – see graphs in \figref{hrdsim:fig:results}. If the response particle did not match the Information, the answers were rated rather as inconsistent with her.\\

%Furthermore, it was shown that participants rated the Answer as consistent significantly more often with Information if this Information was negative (main effect of Information; $z = -3.800, p < 0.001$), see the generally higher blue columns at the bottom of the graphs in \figref{hrdsim:fig:results} (i\_neg).\\

%The difference between Answers that were in agreement with the Information (accord) and those that were not (discord), was smaller with negative Information (interaction of the factors Information and agreement with Information; $z = –2.376, p = 0.018$). In other words, \textit{ano} was interpreted significantly more often in the sense of \textit{yes, it doesn’t} (approx. 40 \%) than \textit{ne} in the sense of \textit{no, it does} (approx. 15 \%).

%If the Answer was in agreement with the Information (accord), then the Answer was evaluated more often as consistent, if the Context also supported it (effect, nevertheless at the limit of significance, triple interaction between all manipulated factors; $z = –1.940, p = 0.052$) . In other words, \textit{ano} after positive Information and a positive Context was rated as consistent more often (approx. 70 \%) than \textit{ano} after positive Information and a negative Context (approx. 50 \%). The same, but significantly weaker, tendency appeared for \textit{ne} - namely, \textit{ne} was rated more often as consistent after negative Information and negative Context than \textit{ne} after negative Information and positive Context.

\subsubsection{Discussion}

Experiment 1 clearly reveals that particle responses to negative syntactically interrogative questions are judged to be true if their polarity is in accordance with the polarity of the information provided: \textit{ano} `yes' is judged as true if the information is positive and \textit{ne} `no' is judged as true if the information is negative. This result is consistent with the absolute feature analysis, under which \textit{ano} `yes' encodes $[+]$ and \textit{ne} `no' encodes $[-]$. If the negation is pleonastic, as is commonly assumed for interrogative (V1) questions \citep{gruet2016yes}, then the results also follow under the relative feature analysis, under which \textit{ano} encodes [\textsc{agree}] and \textit{ne} [\textsc{reverse}], because in both cases the antecedent is positive.

That said, we should also note that there is a difference in the behavior of \textit{ano} `yes' and \textit{ne} `no'. While the effect of \textsc{accordance} is very clear for \textit{ne} (numerical difference of \qty{63}{\percent} between \textsf{accord} and \textsf{discord}), it is much less pronounced for \textit{ano} (numerical difference of \qty{20}{\percent}). The ratings for \textit{ano} `yes' are  closer to chance in both \textsf{accord} and \textsf{discord}, indicating a greater degree of uncertainty in the consistency ratings. This pattern would be expected under the conjunction of the following two premises: the negative polar question makes the negative proposition available as an antecedent (i.e., negation is not pleonastic) and \textit{ano} `yes' encodes \textsc{[agree]}, i.e., its semantics is relative and agrees either with the positive antecedent (`yes, she did') or with a negative antecedent (`yes, she didn't'). The fact that agreement with the positive antecedent is judged as true significantly more often than agreement with the negative antecedent would then reflect on the relative availability of the two antecedent types. This explanation would further be consistent with the fact that the availability of the positive antecedent is modulated by the context (simple effect of \textsc{context}), in line with \citet{hrd+:krifka13}: if the context is positive, the positive antecedent is available more (\qty{71}{\percent}) than if the context is negative (\qty{56}{\percent}). What is unexpected is that there is no analogous simple effect of \textsc{context} if \textit{ano} `yes' is in discord with the information provided, i.e., if the information provided is negative. In this latter case, we could expect the negative context to make the negative antecedent more accessible and hence increase the consistency judgment relative to the positive context condition. This expectation is not met: \textit{ano} `yes' is judged as consistent in \qty{43}{\percent} of the cases irrespective of the value of the \textsc{context} variable.

Turning to the interpretation of \textit{ne} `no', the overall results are consistent with the absolute semantics ($[-]$). What is unexpected under this view, however, is the simple effect of context in the \textsf{accord} condition, i.e., that \textit{ne} `no' is judged true in more cases if the context is negative (\qty{79}{\percent}) than if it is positive (\qty{75}{\percent}). While this effect is numerically smaller than for `yes' responses, it is statistically stronger. A [\textsc{reverse}]-based semantics would get a handle on this effect, but would leave the very low consistency ratings in the \textsf{discord} condition unexplained.

\subsection{Experiment 2: Syntactically declarative polar questions}\label{hrdsim:sec:e2}

\subsubsection{Design and manipulated variables}

This experiment focuses on responses to negative syntactically declarative polar questions, i.e., questions in which the verb is located after the subject (V2); see \REF{hrdsim:ex:v2-ex}.

\ea \gll Jitka neprodala ty staré boty?\\
Jitka \textsc{neg}.sold \textsc{dem} old shoes\\
\glt `Jitka didn't sell the old shoes?'\label{hrdsim:ex:v2-ex}
\z

\noindent We only used two crossed factors in this experiment -- \textsc{information} and \textsc{response}. Context was always negative because negative declarative questions only sound natural in contexts indicating negative evidential bias \citep{hrd+:Gunlogson2002,Stankova2023} and our primary interest was the interpretation (not so much naturalness) of response particles. The overview of the individual conditions is provided in \tabref{hrdsim:tab:exp2}. The materials were parallel to those in experiment 1; we do not include a token set here in the interest of space.

\begin{table}[h]
    \begin{tabularx}{.65\textwidth}{llXl}
    \lsptoprule
         &\textsc{information}&\textsc{response}&\textsc{accordance}\smallskip\\
         % &within items&within items&within items\\
         % &within subjects&within subjects&between subjects\smallskip\\
         % &narrative&1st dialogue utterance&last dialogue utterance\\
         \midrule
a&\textsf{i\_pos}&\textsf{yes}&\textsf{accord}\\
b&\textsf{i\_pos}&\textsf{no}&\textsf{discord}\\
c&\textsf{i\_neg}&\textsf{yes}&\textsf{discord}\\
d&\textsf{i\_neg}&\textsf{no}&\textsf{accord}\\
    \lspbottomrule
    \end{tabularx}
    \caption{Conditions in the factorial design of experiment 2}
    \label{hrdsim:tab:exp2}
\end{table}

\subsubsection{Results}

\figref{hrdsim:fig:exp2-results} shows the raw results of experiment 2, in particular the ratings of consistency between the \textsc{response} (\textsf{yes} vs. \textsf{no}) and the \textsc{information} provided (\textsf{i\_pos} vs. \textsf{i\_neg}). The values of the auxiliary \textsc{accordance} variable are provided as labels, for completeness. \figref{hrdsim:fig:emmeans-f1} provides the corresponding 95\% confidence intervals. We fitted a generalized linear mixed model to estimate the effect of \textsc{information}, \textsc{response}, and their interaction on the consistency rating. Both factors were sum-coded. Random intercepts and slopes for both items and participants were included.\footnote{The formula used was: \textsc{consistency rating} $\sim$ \textsc{information} $*$ \textsc{response} + (1 + \textsc{information} + \textsc{response} | participant) + (1 + \textsc{information} + \textsc{response} | item).} The model confirms the visually clear main effect of \textsc{information}: responses are judged as more consistent with negative information (\textsf{i\_neg}) than with positive information (\textsf{i\_pos}) ($z=-9.191,p<.001$). Furthermore, there is an interaction between \textsc{information} and \textsc{response}: the effect of \textsc{information} is more pronounced with \textit{ne} than with \textit{ano} ($z=4.091,p<.001$).

\begin{figure}
%     \begin{tikzpicture}
%     \pgftext{
    \includegraphics[width=\textwidth]{figures/hrd-ano_ne-results-f1.pdf}
%     }
%     \node[fill=black, opacity=.2, text opacity=1] at (-3.03,1.95) {\small\textsf{accord}};
%     \node[fill=black, opacity=.2, text opacity=1] at (-1.59,1.95) {\small\textsf{discord}};59
%     \node[fill=black, opacity=.2, text opacity=1] at (0.27,1.95) {\small\textsf{discord}};
%     \node[fill=black, opacity=.2, text opacity=1] at (1.63,1.95) {\small\textsf{accord}};
%     \end{tikzpicture}
    \caption{Experiment 2: Response--information consistency ratings}
    \label{hrdsim:fig:exp2-results}
\end{figure}

\begin{figure}
    \includegraphics[width=\textwidth]{figures/hrd-ano_ne-results-f1_ci95.pdf}
    \caption{Experiment 2: 95\% confidence intervals of consistency ratings}
    \label{hrdsim:fig:emmeans-f1}
\end{figure}

\subsubsection{Discussion}

The most significant result of experiment 2 is that both \textit{ano} `yes' and \textit{ne} `no' have the same truth conditions when responding to negative declarative polar questions: they are both judged as true if the information provided is negative (main effect of \textsc{information}). This result follows from the premise that (i) \textit{ano} `yes' has relative semantics (encodes \textsc{[agree]}) and agrees with the negative antecedent (`yes, she didn't'), and, (ii) \textit{ne} `no' has absolute semantics (encodes $[-]$) and negative polarity (`(no,) she didn't').

The fact that the effect of \textsc{information} is stronger for negative than for positive responses is consistent with this view. The relative semantics of \textit{ano} `yes' leaves some room for uncertainty as to which antecedent functions as the particle's prejacent. While the declarative form of the negative question makes the negative antecedent highly salient (making the `yes, she didn't' interpretation true in \qty{66}{\percent} of the cases in the \textsf{i\_neg/discord} condition and \qty{72}{\percent} in the \textsf{i\_pos/accord} condition), the positive antecedent can also be accessed, at least when compared to the corresponding interpretations in the \textsf{no} condition (the `yes, she did' interpretation is judged as true in \qty{28}{\percent} of the cases in the \textsf{i\_pos/accord} condition and \qty{34}{\percent} in \textsf{i\_neg/discord} condition). Compared to that, the hypothetical `no, she did' interpretation is rather exceptional (only evident in about \qty{17}{\percent} of the \textsf{no} responses overall).

% We will not discuss the results for the filler experiments here, but you can infer them from the graphs in \figref{hrdsim:fig:filler1} and \figref{hrdsim:fig:filler2}. 

% \begin{exe}
% \ex \label{hrdsim:fig:filler1}
% \textbf{Filler 1 experiment}\\
% \includegraphics[width = 30 em]{Pictures/F1.pdf}
% \end{exe}

% \begin{exe}
% \ex \label{hrdsim:fig:filler2}
% \textbf{Filler 2 experiment}\\
% \includegraphics[width = 30 em]{Pictures/F2.pdf}
% \end{exe}

\section{General discussion}\label{hrdsim:sec:discussion}

The results of our two experiments lend solid support to (i) relative, \textsc{[agree]}-based semantics of \textit{ano} `yes' and (ii) absolute, $[-]$-based semantics of \textit{ne} `no'. The proposed lexical encoding and the corresponding realization possibilities are represented in \tabref{hrdsim:tab:features-updated}, an updated version of \tabref{hrdsim:tab:features}. The Czech particles \textit{ano} `yes' and \textit{ne} `no' are framed for clarity. The last column indicates where the evidence for the realization (im)possibilities stems from and whether the evidence is positive (judgment of truth) or negative (judgment of falsity).

\begin{table}%[]
\centering
\begin{tabularx}{\textwidth}{lp{1.8cm}Xp{2.8cm}}
\lsptoprule
& Lexically\newline encoded by & Realized by&(Positive/Negative) evidence from\\\midrule
$[+]$&\textit{yes}&\textit{yes}, \fbox{\textit{\sout{ano}}}&exp 2 (neg)\\[0.2em]
$[-]$&\textit{no}, \fbox{\textit{ne}}&\textit{no}, \fbox{\textit{ne}}&exp 1, 2 (pos)\\[0.2em]
[\textsc{agree}]&\textit{yes}, \fbox{\textit{ano}}&\textit{yes}, \fbox{\textit{ano}}&exp 1, 2 (pos)\\[0.2em]
[\textsc{reverse}]&\textit{no}&\textit{no}, \fbox{\textit{\sout{ne}}}&exp 1, 2 (neg)\\[0.2em]
[\textsc{agree}, $+$] &n.a.&\textit{yes}, \fbox{\textit{ano}}&exp 1 (pos)\\ [0.2em]
[\textsc{agree}, $-$] &n.a.&\textit{yes} or \textit{no}, \fbox{\textit{ano}} or \fbox{\textit{ne}}&exp 1, 2 (pos)\\ [0.2em]
[\textsc{reverse}, $+$] &\textit{doch}&\textit{yes} or \textit{no}/\textit{doch}, \fbox{\textit{\sout{ano}}} / \fbox{\textit{\sout{ne}}}&exp 1, 2 (neg)\\ [0.2em]
[\textsc{reverse}, $-$] &n.a.&\textit{no}, \fbox{\textit{ne}}&exp 1 (pos)\\
\lspbottomrule
\end{tabularx}
    \caption{Feature bundles in the feature model (updated)}
    \label{hrdsim:tab:features-updated}
\end{table}

In order to aid the discussion visually, we insert \figref{hrdsim:fig:agg}, which includes data from both experiments: the top pane visualizes results of experiment 1 (aggregating over both levels of the \textsc{context} variable), in which the question was interrogative (verb-first), and the bottom pane visualizes the results of experiment 2, in which the question was declarative (non-verb-first). For ease of reference, we label the individual stacked bars with capital letters.

\begin{figure}
%     \begin{tikzpicture}
%     \pgftext{
    \includegraphics[height=.45\textheight]{figures/hrd-ano_ne-results-aggregate.pdf}
%     }
%         \node[fill=black, opacity=.2, text opacity=1] at (-2.98,3.34) {\small\textsf{A}};
%         \node[fill=black, opacity=.2, text opacity=1] at (-1.8,3.34) {\small\textsf{B}};
%         \node[fill=black, opacity=.2, text opacity=1] at (-0.15,3.34) {\small\textsf{C}};
%         \node[fill=black, opacity=.2, text opacity=1] at (1.05,3.34) {\small\textsf{D}};
%         \node[fill=black, opacity=.2, text opacity=1] at (-2.98,-.35) {\small\textsf{E}};
%         \node[fill=black, opacity=.2, text opacity=1] at (-1.8,-.35) {\small\textsf{F}};
%         \node[fill=black, opacity=.2, text opacity=1] at (-0.15,-.35) {\small\textsf{G}};
%         \node[fill=black, opacity=.2, text opacity=1] at (1.05,-.35) {\small\textsf{H}};
%     \end{tikzpicture}
    \caption{Both experiments: Response--information consistency ratings}
    \label{hrdsim:fig:agg}
\end{figure}

Let us go through \tabref{hrdsim:tab:features-updated} step-by-step. If \textit{ano} encoded $[+]$ alone, we would expect the \textsf{yes} response in experiment 2 to be judged as consistent with the positive information (bar E in \figref{hrdsim:fig:agg}). The fact that it cannot ``ignore'' the negative antecedent (i.e., agrees with it; bar F), strongly supports its relative (rather than absolute) semantics.

The assumption that \textit{ne} encodes the absolute feature $[-]$ is supported by the stability of its consistency with the negative information, independently of the question type preceding it (bars D and H).

That \textit{ano} encodes \textsc{[agree]} is witnessed primarily by the differential behavior of this particle in experiment 1 and experiment 2. In the former, \textit{ano} is judged more consistent with positive information (bar A, vs. B), and in the latter, \textit{ano} is judged more consistent with negative information (bar F, vs. E). This follows if \textit{ano} agrees with its antecedent and if interrogative (V1) questions make the positive antecedent more salient (cf. pleonastic negation), while declarative (non-V1) questions make the negative antecedent more salient. The fact that the effect of the \textsc{information} variable is less pronounced in the \textsf{yes} condition, as compared to the \textsf{no} condition, is -- or so we hypothesize -- also consistent with the relative vs. absolute semantics of \textit{ano} vs. \textit{ne}, respectively. While the absolute semantics of \textit{ne} remains largely insensitive to the polarity of its antecedent (being sensitive merely to the polarity-free prejacent), the relative semantics of \textit{ano} leaves room for pragmatic and contextual considerations as to which antecedent -- whether positive or negative -- is selected as the prejacent of \textit{ano}, which in turn leads to a greater variance in the consistency judgments and their overall centering around chance. This is especially evident in the results of experiment 1 (see A vs. B), where we also observed the effect of the \textsc{context} variable predicted by \citet{hrd+:krifka13}: positive context (as compared to negative context) supports the selection of a positive prejacent (see \sectref{hrdsim:sec:results-e1}).

The fact that \textit{ne} does not encode \textsc{[reverse]} is supported by the results of both experiments, but especially of experiment 2: while \textit{ano} switched its truth conditions between experiment 1 and experiment 2, the truth conditions of \textit{ne} remain stable. This clearly indicates that the differential availability of the two polar antecedents in these experiments had no effect on the meaning of \textit{ne}, militating against its relative semantics. What supports the relative semantics, and is unexpected under our analysis, is that the consistency of the \textit{ne} response with the negative information (bar D) is modulated by context: the consistency is higher if the context is negative -- a mirror image of what happens in A. This effect is numerically small (only about \qty{4}{\percent}), but statistically significant.

Let us now turn to the realization of the four logically possible feature combinations. The $[\textsc{agree},+]$ bundle is realized by \textit{ano}, which, by the subset principle employed in the feature model, spells out \textsc{[agree]} (leaving $[+]$ unrealized). This case is instantiated by bar A, where agreement is with a positive antecedent (supported by the tendentially pleonastic nature of the negation) and where the polarity of the response is, accordingly, positive (`yes, she did'). The $[\textsc{reverse}, -]$ bundle is realized by \textit{ne}, which spells out the subset $[-]$. This case is instantiated by bar D, where the polarity of the response is negative (`she didn't') and is reversed as compared to the primarily positive polarity of the antecedent. The $[\textsc{agree},-]$ bundle can in principle be realized in two ways -- either by \textit{ano,} which spells out [\textsc{agree}], or by \textit{ne,} which spells out $[-]$. That precisely this is the case is witnessed by the identical truth conditions of the two particles in experiment 2 or, more specifically, by the analogous consistency ratings in bars F and H. In this case, the response agrees with the negative antecedent ($\approx$ \textit{ano}) and thus conveys a response of negative polarity ($\approx$ \textit{ne}). The most problematic case is represented by the bundle $[\textsc{reverse}, +]$, which finds no suitable match in the lexical meanings of \textit{ano} or \textit{ne.} This scenario is represented by bars E and G and, as is evident from the consistency ratings, neither \textit{ano} nor \textit{ne} are capable of reliably conveying it. It follows that a response with positive polarity reacting to a clearly negative antecedent (contributed in experiment 2 by the negative declarative question) cannot be expressed by a standalone particle in Czech. Instead, a more complex structure is warranted, such as a fragment (elliptical) response containing a verb explicitly specified for polarity \citep{Gruetskrabalova2015,gruet2016yes} or the positive particle \textit{ano} `yes' preceded by \textit{ale} `but'; see \REF{hrdsim:ex:verb}. We hypothesize that the particle \textit{ale} `but' reverses the salience of the two polar alternatives, making the positive one, which is otherwise only latently present, more salient and hence available as an antecedent of the relative particle \textit{ano} `yes'.

\ea\label{hrdsim:ex:verb}
\begin{exe}
\exi{A:}{\gll Jitka neprodala ty staré boty?\\
Jitka \textsc{neg.}sold \textsc{dem} old shoes\\
\glt `Did Jitka not sell the old shoes?'}
\exi{B$_1$:}{\gll \minsp{\#} Ano. / \minsp{\#} Ne.\\
{} yes {} {} no\\
\glt Intended: `She sold the old shoes.'}
\exi{B$_2$:}{\gll Prodala.\\
sold\\
\glt `She did.'}
\exi{B$_3$:}{\gll Ale ano.\\
but yes\\
\glt `She did.'}
\end{exe}
\z

\noindent This latter point brings us to a discussion of the predictions made by \citet{gruet2016yes}. \citeauthor{Gruetskrabalova2015} proposes that both of the Czech response particles can realize both the relative features and the absolute features. Our experimental results do not lend support to this claim. More particularly, we see only little evidence for \textsc{[reverse]} being realized by \textit{ne} `no' or for $[+]$ being realized by \textit{ano} `yes'. The infelicity of the response (\ref{hrdsim:ex:verb}B$_1$), which reflects our experimental results, is an example of this.

Our experimental results and the analysis we offer bear implications for the interpretation of Czech negative polar questions. Negative polar interrogatives (V1) are often considered to contain pleonastic negation, i.e., a negation which does not contribute propositional negation (see \citealt{Stankova2023} and the references therein). Such questions can thus be expected to only contribute positive propositions as antecedents available for anaphoric pick-up by the relative particle \textit{ano} `yes'. Counter to this expectation, we see that the negative proposition is not completely unavailable. In experiment 1, \textit{ano} `yes' is considered to be consistent with negative information in \qty{43}{\percent} of the cases (bar B), a proportion which is hardly negligible (esp. when compared to the \textsf{no+i\_pos} condition; see bar C). We take this to indicate that negation attached to a fronted verb in polar interrogatives is not necessarily pleonastic; it can either be marginally read as propositional negation or contributes an illocutionary negation (called \textsc{falsum} by \citealt{hrd+:Repp2013}), which can (marginally) participate in forming an antecedent -- possibly a speech act -- which can in turn function as the prejacent of \textit{ano} `yes'. While a more detailed investigation of the interactions between the semantics of negative polar interrogatives and the semantics of polar responses is still missing, the experimental results reported in \citet{Stankova2023} are consistent with the view just suggested.

Negative declarative questions, on the other hand, primarily contribute a negative antecedent, witnessed by the high consistency of \textit{ano} `yes' with negative information in experiment 2 (bar F). While this is, \textit{prima facie}, an expected result, we also know from \citeposst{Stankova2023} results that negative declarative questions readily contribute not only inner negation (licensing negative concord items) but also outer (``pleonastic'') negation (compatible with positive polarity items). If this is the case, we would expect the positive proposition to be more readily available for anaphoric pickup by \textit{ano} `yes'. Yet this is only possible in \qty{28}{\percent} of the cases (bar E). Admittedly, however, the salience of the positive proposition is reduced by two factors in our experimental design (of experiment 2): the absence of any polarity item indicating outer negation and the contextual negative evidence (bias). It is an open issue whether the manipulation of these factors would have an impact on the availability of the positive interpretation of the particle \textit{ano} (matched by an increased consistency in what would correspond to the E bar).



% The results of the main experiment, supported by the results of the filler experiments (we will not discuss them further here, they can be read from the \figref{hrdsim:fig:filler1} and \figref{hrdsim:fig:filler2}) show that the Features model is more applicable for Czech than the Saliency account. The results did not point to any significant role of context, moreover, as we will show below, Czech response particles can be nicely described on the basis of absolute and relative features.
% However, an unambiguous result was only for \textit{ne}, for which the majority of respondents tended to evaluate it as agreement with a negative antecedent. In contrast, the situation was far more complicated for \textit{ano}. In the main experiment, the ratio was almost 50:50. However, there was a slight tendency to reverse the negative polarity of the question - that is, the meaning of the answer to the question: 'Did you do your homework?' would rather be 'Yes, he did.' than 'Yes, I didn’t.' On contrary, in both filler experiments \textit{ano} rather agreed with the polarity of its antecedent. However, according to \citet{gruet2016yes} and \citet{danes-etal87}, the negation in the question where the verb is in the first position is pleonastic, i.e. it is not reflected and the answer in this case also agrees with the positive antecedent. See example (\ref{ex15}) for an application of this assumption to a real language.

% \begin{exe}
% \ex \label{hrdsim:ex:weV1}
% \textbf{Initial negative verb} - tendency to ignore negation (\citealt{gruet2016yes}, \citealt{danes-etal87}\\
% A: Nenapsala sis úkoly? (Didn’t you write your homework?)\\
% B: Ano. [= Napsala.] – agreement with positive antecedent (negation ignored)\\
% ×\\
% B: Ne. [= Nenapsala.] – absolute negation\\

% \textbf{Negated verb in non-initial position} - negation interpreted\\
% A: Úkoly sis nenapsala? (You didn’t write your homework?)\\
% B: Ano. [= Nenapsala.] – agreement with negative antecedent (negation interpreted)\\
% ×\\
% B: Ne. [= Nenapsala.] – absolute negation\\
% \end{exe}

% According to these results, we suggest that \textit{ne} tends to express its absolute feature [−] and \textit{ano} its relative feature [\textsc{agree}] which is not consistent with the assumptions of \citet{gruet2016yes}. Just repeat that she assumes that in the case of negative questions with V1 \textit{ano} and \textit{ne} will express their absolute polarity, whereas in the case of questions with non-V1 their relative polarity. A summary of her assumptions compared to our results is shown in \tabref{hrdsim:ex16}

% \begin{exe}
% \ex \label{hrdsim:ex16}
% \begin{tabular}{p{2cm}|p{2cm}p{2cm}||p{2cm}p{2cm}}
% \textbf{Word order} & \multicolumn{2}{c}{Gruet-Skrabalova (2017)} & \multicolumn{2}{c}{Our proposal}   \\
% & Ano & Ne & Ano & Ne\\
% \hline
% V1 (Nenapsal sis úkoly?) & [+] & [-] & [\textsc{agree}] & [-] \\
% & Napsal. & Nenapsal. & Napsal. & Nenapsal.\\
% non-V1 (Úkoly sis nenapsal?) & [\textsc{agree}] & [\textsc{reverse}] & [\textsc{agree}] & [-] \\
% & Nenapsal. & Napsal. & Napsal. & Nenapsal.
% \end{tabular}
% \end{exe}
% \colorbox{yellow} {UPRAVIT VZHLED TABULKY}\\

% Our results therefore show that the Gruet-Skrabalova's theory is certainly a great contribution to the Czech context, as it opens up the question of the meaning of answer particles, inverts the Features model into the Czech language context and points to the importance of the position of the negated verb in the question. On the other hand, her assumption about absolute features in V1 questions and relative features in non-V1 ones was not empirically confirmed, so we proposed another solution (\textit{ano} tends to express its relative feature [\textsc{agree}] and \textit{ne} its absolute feature [-]).\\

\section{Conclusion} \label{hrdsim:sec:conclusion}

Our paper contributes the first experimental data pertaining to the semantics and interpretation of the two Czech polar response particles -- \textit{ano} `yes' and \textit{ne} `no'. Building on the feature model of \citet{roelofsen-farkas15} and based on the results of our two experiments, we have argued that \textit{ano} `yes' lexically encodes the relative feature \textsc{[agree]} and \textit{ne} `no' encodes the absolute feature $[-]$. This stands in contrast to what has been proposed for Czech by \citet{gruet2016yes} or for English by \citet{roelofsen-farkas15}, namely that response particles are ambiguous between the relative and the absolute meaning. In addition, the results of experiment 1 reveal tentative evidence in favor of \citeposst{hrd+:krifka13} proposal that context can affect the choice of the antecedent for relative response particles. More specifically, we saw that the relative particle \textit{ano} `yes' is resolved to a positive antecedent more often in cases in which it is preceded by a positive context, as compared to a negative context. What is puzzling is that an inverse effect is observed for the particle \textit{ne} `no', which otherwise exhibits a pattern consistent with absolute lexical semantics (which in turn should be insensitive to contextual manipulations). The effect is numerically much smaller, but statistically stronger.

Finally, we have drawn some implications for the semantics of polar questions. Counter to the common assumption that negation on the fronted verb in interrogative questions is pleonastic (e.g., \citealt{gruet2016yes}), i.e. not interpreted, we have seen some tentative evidence for the availability of a negative structure being contributed by such questions. Whether it is a negative proposition or a negative speech act (as assumed e.g. by \citealt{Stankova2023}) remains an open question. Likewise, it remains open how negative declarative questions (non-V1) in which negation is interpreted as outer negation \citep{Stankova2023} are responded to. The prediction is that the positive interpretation of \textit{ano} `yes' should be available to a greater extent in these cases.

% The experimental study focused on the meaning of \textit{yes} and \textit{no} answers. For these purposes, we adopted two already created theories dealing with response particles and their predictions created for English and examined whether they are applicable in the Czech language context as well. \textit{The saliency account} emphasizes the significant influence of the context for evaluating the meaning of \textit{yes} and \textit{no}, whereas the \textit{Features model} does not assume this influence and emphasizes the existence of absolute and relative features of the answers.\\

% The results of our experiment showed that the answers were more or less independent of the context, therefore we claim that the Features model is much more suitable for the Czech language. Within their terminology, we also claim that \textit{yes} agrees with the polarity of its antecedent. That is, it tends to express its realist feature [\textsc{agree}]. In the case of positive questions and negative questions with V1 (pleonastic negation) it agrees with the positive antecedent, on the contrary in the case of negative non-V1 questions with the negative.\\
% \textit{Ne}, on the other hand, tends to express its absolute negative feature [-]. The results for \textit{ne} turned out to be unproblematic and cůear, while the tendency for \textit{ano} is not so strong.\\


\section*{Abbreviations}

\begin{tabularx}{.5\textwidth}{@{}lQ}
\textsc{dem}&demonstrative\\
\textsc{neg}&negation\\
\end{tabularx}%
\begin{tabularx}{.5\textwidth}{lQ@{}}
\textsc{refl}&reflexive\\
&\\ % this dummy row achieves correct vertical alignment of both tables
\end{tabularx}

\section*{Acknowledgments}
The study was funded by the Czech Science Foundation (GAČR), project No. 21-31488J. We are grateful to the anonymous reviewers of this paper as well as the FDSL audience in Berlin for their useful feedback. We would also like to thank (alphabetically) Tomáš Bořil, Ljudmila Geist, Roland Meyer, Maria Onoeva, Sophie Repp, and Anna Staňková. All remaining errors are our own.

\printbibliography[heading=subbibliography,notkeyword=this]

\end{document}
