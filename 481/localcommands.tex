\makeatletter
\let\thetitle\@title
\let\theauthor\@author
\makeatother

\makeatletter
\renewcommand{\@pnumwidth}{2em}
\makeatother

\newcommand{\togglepaper}[1][0]{
   \bibliography{../localbibliography}
   \papernote{\scriptsize\normalfont
     \theauthor.
     \titleTemp.
     To appear in:
     E. Di Tor \& Herr Rausgeberin (ed.).
     Booktitle in localcommands.tex.
     Berlin: Language Science Press. [preliminary page numbering]
   }
   \pagenumbering{roman}
   \setcounter{chapter}{#1}
   \addtocounter{chapter}{-1}
}

\newbool{bookcompile}
\booltrue{bookcompile}
\newcommand{\bookorchapter}[2]{\ifbool{bookcompile}{#1}{#2}}

\AtBeginDocument{\nogltOffset} %removes extra spacing before trans line in examples


%%%%%%%%%%%%%%

\newcommand{\stkout}[1]{\ifmmode\text{\sout{\ensuremath{#1}}}\else\sout{#1}\fi}

\newcommand{\cnstx}[1]{\textsf{\MakeUppercase{#1}}} %creates sans-serif small-caps, used for typesetting logical constants
%%% \newcommand{\sib}[1]{$\llbracket${#1}$\rrbracket$} %puts text into interpretation brackets
%%% \newcommand{\stb}[1]{\ensuremath{\langle{#1}\rangle}} %puts stuff into semantic type brackets; necessary to use in the form $\stb{}$
%%% \newcommand{\cnst}[1]{\textsf{\MakeUppercase{#1}}} %creates sans-serif small-caps, used for typesetting logical constants
% \newcommand{\cnst}[1]{\textsf{\relsize{-1}\MakeUppercase{#1}}} %same as before but without relsize (does not work)
%%% \newcommand{\minsp}[1]{#1\hspace{-2pt}} %example of use: \-2{(} (produces a bracket followed by reduced horizontal space by 2pt)

%%% \newcommand{\citeposst}[1]{\citegen{#1}}

\NewDocumentCommand{\fracx}{mm}
  {\ensuremath{
    \left\{
    \begin{tabular}{@{\,}l@{\,}}
    #1\\
    #2
    \end{tabular}
    \right\}
  }}

\newcommand\n[1]{`#1'}
\newcommand\nn[1]{``#1''}
\newcommand\prim[1]{\textsl{#1}}
\newcommand\M{\textsc{m}}
\newcommand\F{\textsc{f}}
\newcommand\N{\textsc{n}}
\newcommand\pro{\textsc{pro}}



%%%% MATUSHANSKY

\newcommand{\THEM}{\textsc{th}{}\xspace}	%thematic element
\newcommand{\Affix}[1]{\nobreakdash-\textit{#1}\nobreakdash-\xspace}

%%%% PUSKAR
\newcommand{\ndash}{$\begin{bmatrix} \textit{n}P\\ |\\ \text{Anim}\\ |\\ \text{Human}\end{bmatrix}$}

\newcommand{\participantdash}{$\begin{bmatrix} \pi\\ |\\ \text{Prtcpnt}\end{bmatrix}$} %!!
\newcommand{\speakerdash}{$\begin{bmatrix} \pi\\ |\\ \text{Prtcpnt}\\ |\\ \text{Spkr}\end{bmatrix}$}
\newcommand{\pluraldash}{$\begin{bmatrix} \text{\#}\\ |\\ \text{pl}\end{bmatrix}$} %!!
\newcommand{\singulardash}{$\begin{bmatrix} \text{\#}\\ |\\ \text{sg}\end{bmatrix}$}
\newcommand{\threegendersdash}{$\begin{bmatrix} \text{$\gamma$}\\ |\\ \text{M/F/N}\end{bmatrix}$} %!!
\newcommand{\genderdash}{$\begin{bmatrix} \textsc{cl}\\ |\\ \text{F}\end{bmatrix}$} %!!


\newcommand{\plural}{$\begin{bmatrix} \text{\#}\\ \text{pl}\end{bmatrix}$} %!!
\newcommand{\singular}{$\begin{bmatrix} \text{\#}\\ \text{sg}\end{bmatrix}$}
\newcommand{\gender}{$\begin{bmatrix} \text{$\gamma$}\\ \text{M/F/N}\end{bmatrix}$} %!!


\newcommand{\unmarked}{$\begin{bmatrix} \textsc{unmarked}\\ |\\ \textsc{dependent}\\ |\\ \textsc{oblique}\end{bmatrix}$}
\newcommand{\dependent}{$\begin{bmatrix} \textsc{dependent}\\ |\\ \textsc{oblique}\end{bmatrix}$} %!!


\newcommand{\tikzmark}[1]{\tikz[overlay,remember picture]\node(#1){};}
\newcommand{\bracket}[2][]{\ensuremath{\left\lbrack \mbox{#2} \right\rbrack^{#1}}}
\newcommand{\genbeforenum}{	$\begin{bmatrix}\begin{bmatrix} \textasteriskcentered\textsc{cl}:\boxempty\textasteriskcentered\\  \textasteriskcentered\textsc{anim}:\boxempty\textasteriskcentered\\ \textasteriskcentered\textsc{hum}:\boxempty\textasteriskcentered\end{bmatrix}\\\ [\textasteriskcentered \text{\#}:\boxempty\textasteriskcentered]\end{bmatrix}$}

\newcommand{\numbeforegen}{$\begin{bmatrix}[\textasteriskcentered \#:\boxempty\textasteriskcentered]\\\ \begin{bmatrix} \textasteriskcentered\textsc{cl}:\boxempty\\  \textasteriskcentered\textsc{anim}:\boxempty\textasteriskcentered\\ \textasteriskcentered\textsc{hum}:\boxempty\textasteriskcentered\end{bmatrix}\end{bmatrix}$}


\newcommand{\numbeforegenfail}{$\begin{bmatrix}[\#:\text{pl}]\\\ [\gamma:\emptyset]\end{bmatrix}$}
\newcommand{\biogen}{[$*$\textsc{cl}:$\boxempty$[\textsc{human}:$\boxempty$]$*$]}
\newcommand{\gen}{[$*$\textsc{cl}:$\boxempty$$*$]}
%\newcommand{\num}{[$*$\#:$\boxempty$$*$]}%already defined, but I can't find it - since it does not appear in the file, I commented it out

\newcommand{\masculineanimate}{	$\begin{bmatrix} \textsc{cl}\\  \textsc{anim} \\ \textsc{human}\end{bmatrix}$ }
\newcommand{\feminineanimate}{	$\begin{bmatrix} \textsc{cl}\\  \textsc{anim} \\ \textsc{human} \\ \textsc{f}\end{bmatrix}$ }

\newcommand{\numbeforegenmascsg}{	$\begin{bmatrix}  [ \text{\#}:\emptyset]   \\  \begin{bmatrix} \text{M}\\  \text{anim}\end{bmatrix}\end{bmatrix}$}
\newcommand{\genbeforenummasc}{	$\begin{bmatrix}\begin{bmatrix} \textsc{cl}\\  \textsc{anim}\\ \textsc{human}\end{bmatrix}\\\ [ \textsc{pl}]\end{bmatrix}$}
\newcommand{\numbeforegenmasc}{	$\begin{bmatrix}\ [ \textsc{pl}]\\ \begin{bmatrix} \textsc{cl}\\  \textsc{anim}\\ \textsc{human}\end{bmatrix}\end{bmatrix}$}
\newcommand{\numbeforegenfempl}{	$\begin{bmatrix}\ [ \textsc{pl}]\\\textsc{[cl]}\end{bmatrix}$}


\newcommand{\ssigma}{$\begin{bmatrix} \sigma \end{bmatrix}$} %!!
\newcommand{\person}{$\begin{bmatrix} \sigma\\ \pi\end{bmatrix}$}
\newcommand{\participant}{$\begin{bmatrix} \sigma\\ \pi\\ \text{Prtcpnt}\end{bmatrix}$} %!!
\newcommand{\speaker}{$\begin{bmatrix} \sigma\\ \pi\\ \text{Prtcpnt}\\ \text{Spkr}\end{bmatrix}$}


\newcommand{\fst}{$^{st}$}
\newcommand{\nd}{$^{nd}$}
\newcommand{\rd}{$^{rd}$}

\newcommand*{\DrawArrowa}[4][]{%
	% #1 = draw options
	% #2 = left point
	% #3 = right point
	\begin{tikzpicture}[overlay,remember picture, >=latex']
	% \draw [-latex, #1] ($(#2)+(0.1em,0.5ex)$) to ($(#3)+(0,0.5ex)$);
	\draw[rounded corners, -latex,<-,semithick,#1]  (#2) -- ++(0,-1.2em)coordinate (a) -- 
	%node[below,font=\footnotesize]{#4} 
	node[midway,fill=white,font=\footnotesize]{#4}
	($(a-|#3)$) -- (#3);
	\end{tikzpicture}%
	%\vspace{0.5cm}
}%  

\newcommand*{\DrawArrowaa}[4][]{%
	% #1 = draw options
	% #2 = left point
	% #3 = right point
	\begin{tikzpicture}[overlay,remember picture, >=latex']
	% \draw [-latex, #1] ($(#2)+(0.1em,0.5ex)$) to ($(#3)+(0,0.5ex)$);
	\draw[rounded corners, -latex,<-,semithick,#1]  (#2) -- ++(0,-2em)coordinate (a) -- 
	%node[below,font=\footnotesize]{#4} 
	node[midway,fill=white,font=\footnotesize]{#4}
	($(a-|#3)$) -- (#3);
	\end{tikzpicture}%
	%\vspace{0.5cm}
}%  

\newcommand{\before}{$\succ$}
\definecolor{Gray}{gray}{0.6}


\forestset{
  nice nodes/.style={
  for tree={
  inner sep=1pt, s sep=12pt,
  fit=band,
},
},
% begin fairly nice empty nodes
fairly nice empty nodes/.style={
            delay={where content={}{shape=coordinate,for parent={
                  for children={anchor=north}}}{}}
},
% end fairly nice empty nodes
% begin pretty nice empty nodes
pretty nice empty nodes/.style={
    for tree={
      calign=fixed edge angles,
      parent anchor=children,
      delay={if content={}{
          inner sep=0pt,
          edge path={\noexpand\path [\forestoption{edge}] (!u.parent anchor) -- (.children)\forestoption{edge label};}
        }{}}
    },
  },
% end pretty nice empty nodes
default preamble={
nice nodes,
%nice empty nodes, % uncomment the one you want (and delete the ones you don't)
%fairly nice empty nodes,
%pretty nice empty nodes
}
}
