%add all your local new commands to this file

\newcommand{\smiley}{:)}

\renewbibmacro*{index:name}[5]{%
  \usebibmacro{index:entry}{#1}
    {\iffieldundef{usera}{}{\thefield{usera}\actualoperator}\mkbibindexname{#2}{#3}{#4}{#5}}}

% \newcommand{\noop}[1]{}

\makeatletter
\def\blx@maxline{77}
\makeatother

\newcommand{\appref}[1]{Appendix \ref{#1}}
\newcommand{\fnref}[1]{Appendix \ref{#1}}
\newcommand{\regel}[1]{#1}
\newcommand{\vernacular}[1]{\emph{#1}}
\newcommand{\gloss}[1]{#1}

%MY COMMANDS

\newcommand{\den}[1]{$\llbracket$#1$\rrbracket$}                %for the denotation symbol

\newcommand{\predica}[2]{\text{\textsc{#1}(\textit{#2})}}    %for predicates and arguments in logical notation (inside math mode)


\newcommand{\n}{\noindent}                                      %for paragraphs without indentation

\newcommand{\gvaz}{^{\varnothing}}                                %assignment vazio

\newcommand{\gone}[2]{^{\scriptsize\left[%
		\begin{array}{@{}c@{\ }l@{\ }l@{}}%
			#1 & \rightarrow& \text{#2} \\
		\end{array}\right]}}                                            %assignments com 1 número

\newcommand{\gtwo}[4]{^{\scriptsize\left[%
		\begin{array}{@{}c@{\ }l@{\ }l@{}}%
			#1 & \rightarrow& \text{#2} \\
			#3 & \rightarrow & \text{#4} \\
		\end{array}\right]}}                                            %assignments com 2 números

\qtreecenterfalse

%%%%%%%%%%%%%
%%%%%%%%%%%%%

\newlength{\arrowht}
\setlength{\arrowht}{-2.5ex}% <-- changed
\newcommand*\exdepthstrut{{\vrule height 0.75\arrowht depth -0.25\arrowht width 0pt}}% <-- changed
\newcommand\tikzmark[1]{\tikz[remember picture, baseline=(#1.base)] \node[anchor=base,inner sep=0pt, outer sep=0pt] (#1) {#1\exdepthstrut};}


% This code from http://tex.stackexchange.com/q/55068/2693
\tikzset{
	ncbar angle/.initial=90,
	ncbar/.style={
		to path=(\tikztostart)
		-- ($(\tikztostart)!#1!\pgfkeysvalueof{/tikz/ncbar angle}:(\tikztotarget)$)
		-- ($(\tikztotarget)!($(\tikztostart)!#1!\pgfkeysvalueof{/tikz/ncbar angle}:(\tikztotarget)$)!\pgfkeysvalueof{/tikz/ncbar angle}:(\tikztostart)$)
		-- (\tikztotarget)
	},
	ncbar/.default=\arrowht,% <-- changed
}
% Thanks to Paul Gessler and Percusse for code improvement here
\newcommand{\arrowtop}[2]{\begin{tikzpicture}[remember picture,overlay]
	\draw[->,shorten >=1ex,shorten <=1ex] ([yshift=1.25ex] #1.south) to [ncbar] ([yshift=1.25ex] #2.south);% <-- changed
	\end{tikzpicture}
}

\newcommand{\arrowbottom}[2]{\begin{tikzpicture}[remember picture,overlay]
	\draw[<-,shorten >=1ex,shorten <=1ex] ([yshift=1.25ex] #1.south) to [ncbar] ([xshift=1.75ex,yshift=1.25ex] #2.south);% <-- changed
	\end{tikzpicture}
}

%%%%%%%%%%%%%
%%%%%%%%%%%%%

