%versao de 21-MAR-2019

\chapter{Significado e condições de verdade}


\textsc{Semântica} é o estudo do significado e \textsc{semântica formal} é o estudo
do significado através do uso de ferramentas e técnicas
lógico-matemáticas. Definições assim tão gerais, entretanto,
são pouco esclarecedoras e tendem mais a suscitar do que a responder
questões (Significado de quê? O que é significado? Que tipo de
ferramentas?). O objetivo deste primeiro capítulo é dar uma
resposta preliminar a essas questões, explicitando a acepção que
o termo \textit{significado} tomará no contexto deste livro e,
sobretudo, motivando a relevância de seu estudo através de um
método formal. Sendo este um livro de linguística, nosso interesse
estará no significado de expressões das línguas naturais, mais
especificamente no significado de sentenças e suas partes
(sintagmas). Ainda assim, esse é um terreno vasto e, ao final do
capítulo, falaremos um pouco sobre alguns tipos de sentenças
cujo estudo está além do escopo da teoria que desenvolveremos.



\section{Significado, verdade e mundo}

Um dos fatos mais salientes a respeito das sentenças de uma língua natural é que seus falantes as utilizam a todo momento para falar sobre o mundo em que vivem. Tal fato parece ligado à intuição de que as sentenças têm significado e que, cientes desse significado, somos capazes de estabelecer conexões entre linguagem e realidade. Por exemplo, imagine que alguém diga a você: \textit{Está chovendo em Paris}. Se você souber o significado dessa sentença, e assumindo que a pessoa esteja sendo sincera, você, então, terá aprendido algo sobre o mundo. Da mesma forma, mas no sentido inverso, se você obtiver (digamos, visualmente) informação sobre o tempo chuvoso em Paris, você automaticamente concluirá que a sentença \textit{Está chovendo em Paris} é verdadeira. Ou seja, saber o significado de uma sentença nos permite transitar da linguagem para o mundo e do mundo para a linguagem. Note que nada disso seria possível se não soubéssemos o significado da sentença em questão. Isso nos leva a crer que o significado tem um papel crucial na mediação entre linguagem e realidade. Dada a importância desse papel, é justo cobrar de uma teoria semântica que ela lance luz sobre essa relação linguagem-mundo. É sobre esse aspecto da linguagem, chamado às vezes de referencial, que nos debruçaremos no decorrer deste livro. Nossa premissa é que as expressões de uma língua natural, sentenças em particular, possuem significado, e que a natureza desse significado está na base das conexões que fazemos entre linguagem e realidade. Exploremos um pouco mais esse ponto. 

Voltemos à nossa sentença sobre Paris, destacada em (\ref{paris}) a seguir:

\begin{exe}
	\ex Está chovendo em Paris.\label{paris}
\end{exe}


\n Imagine, agora, que alguém diga a você que sabe o significado dessa sentença do português e que você decida verificar se a pessoa realmente sabe o
que ela significa.


Note, em primeiro lugar, que seria injusto perguntar a
essa pessoa se a sentença é verdadeira ou falsa. Pode ser que ela
não tenha informa\-çõ\-es suficientes sobre o tempo em Paris neste
instante e que, portanto, não possa fornecer a você uma
informa\-ção confiável. Mas isso nada tem a ver com o fato de ela
saber ou não o significado de (\ref{paris}). Saber o significado
de uma sentença não significa saber se a sentença é verdadeira ou
falsa. Imagine agora que nós mostremos várias imagens da cidade de
Paris a essa pessoa e perguntemos a ela se a sentença acima seria
verdadeira em cada uma das circunstâncias. Agora sim seria justo
cobrar dessa pessoa respostas adequadas. Por exemplo, se em uma
das imagens não está chovendo, mas nevando, ou se o chão das ruas
estiver molhado, mas não estiver caindo água do céu, ou ainda se
estiver chovendo próximo a Paris, mas a imagem deixar claro que
está fazendo sol na cidade, nosso informante deve dizer que a
sentença é falsa. Por outro lado, se a imagem for mesmo de Paris e
estiver chovendo, a pessoa deve dizer que a sentença é verdadeira,
independente de outros detalhes, como, por exemplo, se as pessoas
estão ou não carregando guarda-chuvas, ou se há ou não um
arco-íris no céu. Qualquer resposta contrária a essas expectativas
serve de indício de que a pessoa não sabe o significado de
(\ref{paris}).

Poderíamos capturar essa intuição associando o conhecimento do
significado de uma sentença como (\ref{paris}) ao conhecimento das
condi\-çõ\-es para que a sentença seja
verdadeira.

\begin{exe}
	\ex Saber o significado de uma sentença é saber as condi\-çõ\-es necessárias e suficientes para que a sentença seja verdadeira.
\end{exe}

Vamos adotar então como hipótese de trabalho a seguinte
defini\-ção de significado de uma sentença:

\begin{exe}
	\ex O significado de uma sentença é equivalente às condi\-çõ\-es necessárias e suficientes para que a sentença seja verdadeira.
\end{exe}

\n De forma mais reduzida:

\begin{exe}
	\ex O significado de uma sentença é equivalente às suas condi\-çõ\-es de verdade.
\end{exe}


\n Será em torno dessa máxima que construiremos nossa teoria
semântica. Teorias assim, em que o significado de uma sentença equivale às suas condições de verdade, são chamadas de \textsc{vericondicionais}. Nosso objetivo será estabelecer uma maneira de derivar
para cada sentença S de uma língua natural um enunciado da
seguinte forma:

\begin{exe}
	\ex \textit{S} é verdadeira se e somente se p
\end{exe}

\noindent Nesse enunciado,  \textit{S} é uma sentença (ou mais
pre\-ci\-sa\-men\-te, uma estrutura sin\-tá\-ti\-ca, conforme
veremos mais tarde) e \textit{p} descreve condi\-çõ\-es sobre o
mundo.

É importante notar como uma teoria semântica deste tipo é capaz de
estabelecer conexões entre as sen\-ten\-ças de uma língua e o
mundo em que vivemos. No exemplo acima, vimos que o conhecimento
do significado de uma sentença mais o conhecimento das
circunstâncias relevantes do mundo leva ao conhecimento da verdade
ou falsidade da sentença (também chamado de valor de verdade da
sentença). Igualmente, o conhecimento do significado de uma
sentença mais o conhecimento de seu valor de verdade leva ao
conhecimento sobre algo relativo ao mundo. Dito de outra forma, se
sabemos que uma sentença é verdadeira (ou falsa), então sabemos
algo sobre o mundo. Esses dois tipos de inferência centradas em
nosso conhecimento semântico são imediatamente capturados por uma
teoria baseada em condi\-çõ\-es de verdade. Isso está representado
nos esquemas abaixo (abreviamos a expressão \textit{se, e somente
se} por \textit{sse}):

\begin{exe}
	\ex \begin{tabular}[t]{l l l}
		
		\textit{S} é verdadeira \textit{sse} p & \ \ \  & Conhecimento semântico + \\
		
		p & \ \ \  & Conhecimento de mundo (fatos) = \\
		logo, \textit{S} é verdadeira. & \ \ \  & Conhecimento do valor de verdade \\
		
	\end{tabular}
\end{exe}

\begin{exe}
	\ex \begin{tabular}[t]{l l l}

  \textit{S} é verdadeira \textit{sse} p & \ \ \  & Conhecimento semântico + \\
	\textit{S} é verdadeira. & \ \ \  & Conhecimento do valor de verdade = \\
  logo, p & \ \ \  & Conhecimento de mundo (fatos) \\

	\end{tabular}
\end{exe}



\section{Relações semânticas entre sentenças}

Passemos agora a um outro tipo de intuição semântica. Como falantes do português, todos nós temos intuições a respeito
de certas relações entre sentenças, relações essas que, de um ponto
de vista ainda pré-teórico, parecem envolver o significado dessas
sentenças. Assim sendo, seria desejável que uma teoria semântica
auxiliasse na definição e explicação dessas relações.

Considere, por exemplo, o par abaixo:


\begin{exe}
    \ex\label{ac}
    \begin{xlist}
        \ex  João comprou um carro usado.\label{aca}
        \ex  João comprou um carro.\label{acb}
    \end{xlist}
\end{exe}

\n Esse par ilustra a relação de \textsc{acarretamento}. A ideia
por trás dessa noção é a de que uma sentença S$_{1}$ acarreta uma
sentença S$_{2}$ se a verdade de S$_{1}$ já trouxer embutida a verdade
de S$_{2}$. Diz-se também que S$_{2}$ é uma consequência lógica de
S$_{1}$.

Uma semântica baseada em condições de verdade nos permite definir
precisamente essa noção: dizemos que uma sentença S$_{1}$ acarreta
uma sentença S$_{2}$ se, sempre que S$_{1}$ for verdadeira, S$_{2}$
também será verdadeira. Ou, se quisermos empregar explicitamente a
no\-ção de \textsc{condi\-çõ\-es de verdade}, podemos dizer que
uma sentença S$_{1}$ acarreta uma sentença S$_{2}$, se, sempre que as
condi\-çõ\-es de verdade de S$_{1}$ forem satisfeitas (e S$_{1}$ for,
portanto, verdadeira), as condi\-çõ\-es de verdade de S$_{2}$ também
serão satisfeitas (e S$_{2}$ será, portanto, verdadeira).


Assim, dizemos que (\ref{aca}) acarreta (\ref{acb}), já que, se
garantirmos a verdade da primeira, a verdade da segunda fica
automaticamente garantida. Note que não necessitamos saber se as
sen\-ten\-ças em questão são, de fato, verdadeiras ou não. Basta
imaginarmos situa\-çõ\-es possíveis, reais ou não, em que uma das
sen\-ten\-ças seja verdadeira e verificar se, em cada uma delas, a
outra sentença é ou não verdadeira. Se existir ao menos uma
situa\-ção em que S$_{1}$ é verdadeira, mas S$_{2}$ não, então S$_{1}$
não acarreta S$_{2}$.

 Voltando às sen\-ten\-ças vistas acima, já notamos
que (\ref{aca}) acarreta (\ref{acb}). As condi\-çõ\-es de verdade
de (\ref{aca}) nos dizem que a sentença será verdadeira se, e
somente se, João tiver comprado um carro usado. Já as
condi\-çõ\-es de verdade de (\ref{acb}) nos dizem que a sentença
será verdadeira se, e somente se, João tiver comprado um carro.
Mas como quem compra um carro usado necessariamente compra um
carro, então, em toda situa\-ção em que João tiver comprado um
carro usado ele terá comprado um carro. Note, entretanto, que o
inverso não é verdadeiro. Ou seja, alguém que comprou um carro não
necessariamente comprou um carro usado. Basta imaginar uma
situa\-ção em que o carro em questão é novo. Portanto, (\ref{acb})
não acarreta (\ref{aca}). Veremos mais adiante neste capítulo que,
se formos explícitos na formulação das condições de verdade das
sentenças de uma língua, o método que estamos começando a esboçar
também nos auxiliará na explicação dessa nossa intuição tão clara
(e mesmo banal) de que se alguém comprou um carro usado, então
esse alguém necessariamente comprou um carro, mas não o contrário.

Passemos, agora, à no\-ção de equivalência ou sinonímia entre
sen\-ten\-ças. Dizemos de duas sen\-ten\-ças que elas são
\textsc{equivalentes} (ou sinônimas), se as suas condi\-çõ\-es de
verdade são as mesmas. Ou seja, não há como satisfazer as condições de verdade de uma
sem automaticamente satisfazer as da outra. Por exemplo,
(\ref{eqa}) e (\ref{eqb}) abaixo são equivalentes. Não há como
imaginar uma situa\-ção em que uma é verdadeira e a outra falsa, e
vice-versa.


\begin{exe}
    \ex\label{eq}
    \begin{xlist}
        \ex  João visitou Maria.\label{eqa}
        \ex  Maria foi visitada por João.\label{eqb}
    \end{xlist}
\end{exe}


\n Note que é possível definir equivalência em termos de acarretamento. S$_{1}$ e S$_{2}$ são equivalentes se, e somente se, S$_{1}$ acarreta S$_{2}$ e S$_{2}$ acarreta S$_{1}$.

Por fim, mais algumas no\-çõ\-es relevantes que se aplicam a sen\-ten\-ças e que são definidas em termos de condi\-çõ\-es de verdade. Dizemos que uma sentença é uma \textsc{tautologia} se as suas condi\-çõ\-es de verdade forem sempre satisfeitas, não havendo situa\-ção possível em que ela seja falsa. É o caso, por exemplo, da sentença abaixo:

\begin{exe}
    \ex Ou João comprou um carro ou João não comprou
um carro. \label{tau}
\end{exe}

\n Por mais que sejamos criativos, jamais conseguiremos imaginar uma situa\-ção em que essa sentença seja falsa. Podemos também pensar em sen\-ten\-ças que nunca são verdadeiras. Suas condi\-çõ\-es de verdade não são jamais satisfeitas. Nesse caso dizemos que a sentença em questão é uma \textsc{contradi\-ção}. Um exemplo é a sentença abaixo:

\begin{exe}
    \ex O Brasil tem mais habitantes do que Portugal e Portugal tem mais habitantes do que o Brasil. \label{con}
\end{exe}

 \n Restam as sen\-ten\-ças cujas condi\-çõ\-es de verdade podem ou não ser satisfeitas,
dependendo da situa\-ção (fatos) em questão. Essas sen\-ten\-ças
são chamadas de \textsc{contingências}. É o caso, por exemplo, da
sentença \textit{João comprou um carro usado}.

\section{Composicionalidade}

Voltemos a considerar nossa intui\-ção sobre o significado de uma
sentença. Intuitivamente, o significado de uma sentença depende do
significado dos itens lexicais que a compõem. Além de depender do
significado dos itens lexicais que a compõem, o significado de uma
sentença depende também da maneira como esses itens se encontram
organizados na sentença. O significado de \textit{João beijou Maria}
difere não só do significado de \textit{João abraçou Maria}, mas também
do significado de \textit{Maria beijou João}.

Para captar essas intui\-çõ\-es, precisamos dizer algo sobre a
rela\-ção entre o significado de uma sentença e o significado das
partes dessa sentença. Isso parece necessário, já que o contrário
nos levaria a assumir que nossa competência semântica se reduz a
uma lista de sen\-ten\-ças associadas a suas condi\-çõ\-es de
verdade. Mas como explicaríamos, então, nossa capacidade de
compreender sen\-ten\-ças que nunca ouvimos antes? (\ref{wabaq})
abaixo talvez seja um caso em questão:


\begin{exe}
    \ex Uma tartaruga com quatro estômagos engoliu a mãe de um elefante cor de
    abóbora. \label{wabaq}
\end{exe}

\n Um falante competente, mesmo que se depare com a sentença acima
pela primeira vez e fora de qualquer contexto relevante (por
exemplo, escrita em um muro), será capaz de imaginar cenários
hipotéticos que tornariam a sentença verdadeira, por mais bizarros
que sejam.


E, se refletirmos melhor, veremos que somos capazes de compreender
um número infinito de sen\-ten\-ças, ainda que na prática isso não
seja óbvio devido a restri\-çõ\-es de memória e processamento.
Basta pensar em uma série
de sen\-ten\-ças como a seguinte:

\begin{exe}
\ex \begin{xlist}
		\ex O João era poeta.
		\ex O pai do João era poeta.
		\ex O pai do pai do João era poeta.
		\ex O pai do pai do pai do João era poeta.
		\ex \ldots
	\end{xlist}
\end{exe}


\n Note que, mesmo com um léxico bastante reduzido (6 itens), somos
capazes não só de reconhecer como gramaticais, mas também de
interpretar um número infinito de sentenças. A hipótese de uma
lista infinita armazenada em um cérebro finito não parece
consistente. Nossa competência semântica é portanto mais do que
uma simples lista de sen\-ten\-ças pareadas com suas condi\-çõ\-es
de verdade. Em consonância com essas intui\-çõ\-es, vamos adotar o
seguinte prin\-cí\-pio:


	
\begin{exe}
	\ex Princípio de composicionalidade:\\
	O significado de uma sentença é derivado do significado dos itens lexicais que a compõem e da maneira como esses itens estão organizados.
\end{exe}





\n Para representar a maneira como os itens lexicais de uma
sentença estão organizados, ou seja, para representar a estrutura
sintática de uma sentença, vamos nos valer de diagramas em forma
de árvore. Para sen\-ten\-ças simples, assumiremos que são
formadas por um sintagma nominal (NP, do inglês \textit{noun phrase}) e um sintagma verbal (VP, do inglês \textit{verb phrase}).
Sintagmas verbais, por sua vez, são formados por um verbo intransitivo, como em \textit{João trabalha}, ou por um
verbo transitivo seguido de um sintagma nominal, como mostrado a
seguir:

\begin{figure}[H]
	\centerline{ \Tree [.S [.NP [.N João ] ] [.VP [.V ama ] [.NP [.N Maria ] ] ] ] \hspace{1in} \Tree [.S [.NP [.N Maria ] ] [.VP [.V ama ] [.NP [.N João ] ] ] ] } \caption{Sentenças com verbos transitivos }
\end{figure}


\n Note que apesar de conterem os mesmos itens lexicais, as
estruturas acima são distintas, já que seus constituintes (as
sub-árvores que as compõem) não são os mesmos. Por exemplo, o
constituinte correspondente ao nó rotulado de VP engloba (domina)
o item lexical \textit{Maria} na primeira estrutura, mas não na
segunda. À medida que progredirmos, apresentaremos as estruturas
de sen\-ten\-ças mais complexas.

Tendo estruturas como essas em mente, suponha agora que a no\-ção
de significado se aplique não apenas a itens lexicais, mas a todo
e qualquer constituinte sintático (incluindo, claro,
sen\-ten\-ças). Podemos, então, formular uma versão mais específica
do prin\-cí\-pio acima, que continuaremos a chamar simplesmente de prin\-cí\-pio de composicionalidade.

\begin{exe}
	\ex O significado de um constituinte sintático é derivado do significado de seus constituintes imediatos.
\end{exe}

\noindent O prin\-cí\-pio se aplica tanto a
sen\-ten\-ças quanto a constituintes menores, como sintagmas
nominais, proje\-çõ\-es verbais, etc. De certa forma, trata-se de
uma interpreta\-ção radical da no\-ção de composicionalidade, já
que para obter o significado de um constituinte, basta olhar para
o significado de seus constituintes imediatos, sem se preocupar
com o significado das partes que compõem esses constituintes. O
resultado disso é que podemos vislumbrar um procedimento
recursivo, que computa o significado de um constituinte
\textit{C}, baseado nos significados de seus subconstituintes.
Por sua vez, o significado de cada um desses constituintes
(D$_{1}$, ..., D$_{n}$) é obtido a partir do significado de
seus próprios subconstituintes E$_{1}$, ..., E$_{n}$
(``sub-subconstituintes'' de \textit{C}), e assim por diante,
descendo até os itens lexicais, que teriam seus significados
armazenados em uma lista (um léxico).


Como ilustra\-ção, considere a estrutura apresentada mais acima
que corresponde à sentença \textit{Maria ama João}. Nosso objetivo é
obter o significado de S (ou, mais precisamente, da estrutura
rotulada de S). De acordo com o prin\-cí\-pio de
composicionalidade, o significado de S é obtido a partir do
significado de NP$_{1}$ e do significado de VP. O significado de
NP$_{1}$ é obtido a partir do significado de N$_{1}$, que por sua vez
é obtido do significado do item lexical \textit{Maria}. Já o
significado de VP é obtido a partir do significado de V e do
significado de NP$_{2}$. O significado de V é obtido a partir do
significado do verbo \textit{ama} e o significado de NP$_{2}$ a
partir do significado de N$_{2}$, que por sua vez é obtido a partir
do significado de \textit{João}. Tudo o que precisamos agora é de
um lista com as entradas correspondentes ao significado dos itens
lexicais. Feito isso, poderíamos percorrer o caminho inverso ao
que acabamos de fazer e obter o significados de N$_{2}$, de NP$_{2}$,
de V, de VP, de N$_{1}$, de NP$_{1}$ e, finalmente, de S!

Obviamente, dizer que o significado de um constituinte é derivado
do significado de suas partes ainda deixa no ar a questão de como
esse significado é obtido. Falta-nos ainda especificar as regras
ou prin\-cí\-pi\-os de composi\-ção. Serão essas regras que nos
dirão explicitamente de que maneira se obtém o significado do todo a partir do significado das partes.

Tudo isso, claro, ainda é muito programático, mas já sabemos que
formato nossa teoria semântica terá se seguirmos a versão do
prin\-cí\-pio de composicionalidade discutida acima: um léxico, ou
seja, uma lista com o significado das palavras da língua em
questão e um conjunto de regras composicionais, definindo
explicitamente como obter o significado de um constituinte a
partir de seus sub-constituintes imediatos.

É com a defini\-ção dessas regras e com a elabora\-ção de entradas
lexicais para uma série de palavras de diferentes classes
gramaticais que nos ocuparemos a partir do próximo capítulo, tendo
sempre como objetivo a deriva\-ção de condi\-çõ\-es de verdade
condizentes com nossa intui\-ção a respeito do significado das
sen\-ten\-ças analisadas.

\section{Linguagem objeto e metalinguagem}

Daqui em diante, nosso interesse estará voltado para as
sen\-ten\-ças do português, ainda que os métodos utilizados se
apliquem igualmente a outras línguas naturais (essa pelo menos é
nossa esperança). O português será, portanto, nossa
\textsc{linguagem objeto}, a linguagem sobre a qual
iremos teorizar. Precisamos definir ainda uma
\textsc{metalinguagem}, a linguagem na qual iremos
formular nossa teoria. Em particular, precisamos de uma linguagem
adequada para representar as condi\-çõ\-es sobre o mundo que
aparecem à direita do conectivo \textit{se e somente se}  nas
representa\-çõ\-es dos significados das sen\-ten\-ças.

\begin{exe} 
	\ex \textit{S} é verdadeira se e somente se p
\end{exe}

\n Utilizaremos uma mistura de português com alguns
símbolos da lógica de predicados e da teoria dos conjuntos, sempre
almejando obter representa\-çõ\-es sucintas e
sem ambiguidades. O fato de utilizarmos o próprio português em
nossa metalinguagem poderá, às vezes, causar a impressão de que
nossa teoria está apenas relatando o óbvio, como no caso
abaixo:

\begin{exe} 
	\ex \textit{Está chovendo} é verdadeira se e somente se está chovendo.
\end{exe}

\n Tenha em mente que isso é ilusório. Fosse nossa linguagem
objeto o inglês, a ilusão desapareceria:

\begin{exe} 
	\ex \textit{It is raining} é verdadeira se e somente se está chovendo.
\end{exe}

\noindent Note que, nesse segundo caso, não se trata de uma
afirma\-ção óbvia. Note, ainda, que tampouco se trata de uma
tradu\-ção do inglês para o português. O que aparece do lado
direito de \textit{se e somente se} diz respeito ao mundo, e não a
uma sentença do português. Em outras palavras, as afirma\-çõ\-es
acima não equivalem às afirmações abaixo:

\begin{exe} 
	\ex \textit{Está chovendo} é verdadeira se e somente se \textit{Está chovendo} é verdadeira.
\end{exe}

\begin{exe} 
	\ex \textit{It is raining} é verdadeira se e somente se \textit{Está chovendo} é verdadeira.
\end{exe}

\n Note que, nesses casos, estaríamos relacionando linguagem e linguagem, e não linguagem e mundo. Dizer que duas sentenças têm as mesmas condições de verdade não nos informa quais são essas condições. Ou seja, fazer semântica não se limita a apresentar paráfrases ou traduções. Nunca será demais relembrar que sempre que estivermos diante de
uma representa\-ção da forma \textit{S é verdadeira se e somente se p}, o que aparece do lado direito do conectivo \textit{se e somente se} é uma descri\-ção de um estado de coisas, que, acidentalmente, pode estar codificado na própria linguagem objeto.

\section{Extensões}

Assumimos acima que o significado de uma sentença corresponde a
suas condi\-çõ\-es de verdade. Nossa teoria especifica as
condi\-çõ\-es que devem ser satisfeitas para que uma sentença seja
verdadeira. Quando uma sentença é verdadeira, dizemos que seu
valor de verdade é 1 (verdadeiro). Quando uma sentença é falsa,
dizemos que seu valor de verdade é 0 (falso). Diremos de uma
sentença verdadeira que sua \textsc{extensão} ou
\textsc{denotação} é 1. Diremos de uma sentença falsa
que sua extensão ou denota\-ção é 0. Utilizaremos um símbolo
especial para o termo \textit{extensão}: $\llbracket \ \rrbracket$. Além
disso, abreviaremos o conectivo \textit{se e somente se} como
\textit{sse}. As afirma\-çõ\-es abaixo são, portanto,
equivalentes:

\begin{exe} 
	\ex \textit{S} é verdadeira se, e somente se, p
\end{exe}

\begin{exe} 
	\ex \den{S} = 1 \textit{sse} p
\end{exe}

\noindent Note que a extensão de uma sentença não é o seu
significado. Conforme já notamos no início, saber o significado de
uma sentença não é o mesmo que saber se a sentença é verdadeira ou
falsa.

Podemos interpretar o conceito de extensão de uma sentença como
sendo uma fun\-ção (no sentido matemático do termo) que mapeia a
representa\-ção sintática de uma sentença em um valor de verdade.
Uma teoria semântica se incumbe então de especificar as
condi\-çõ\-es para que uma sentença S qualquer seja mapeada no
valor de verdade 1 (ou 0).

Vamos aplicar o conceito de extensão não só a sen\-ten\-ças, mas a
constituintes sintáticos em geral. A ideia é atribuir a cada um
deles (incluindo os itens lexicais) um objeto que funcione como
seu valor semântico, ou seja, algo manipulável pelas regras de
composi\-ção no percurso para a obten\-ção das condi\-çõ\-es de
verdade de uma sentença. Isso nos leva a uma versão extensional do
prin\-cí\-pio
de composicionalidade:

\begin{exe} 
	\ex A extensão de um constituinte sintático é derivada da extensão de seus constituintes imediatos.
\end{exe}

\n Uma teoria semântica é dita \textsc{extensional} se suas regras
de composi\-ção especificam como obter a extensão de um
constituinte a partir das extensões de seus subconstituintes.


Tudo isso ainda está sendo colocado de forma muito abstrata e
ficará mais claro e concreto à medida que formos progredindo na
análise de sen\-ten\-ças específicas. Mas, a título de ilustra\-ção
provisória, vamos considerar um pequeno grupo de sen\-ten\-ças,
criando um sistema composicional rudimentar pautado no conceito de
extensão. Vamos nos valer de algumas ferramentas matemáticas bem simples retiradas da teoria dos conjuntos ensinada nos ensinos fundamental e médio.



Comecemos pela sen\-ten\-ça abaixo:

\begin{exe}
	\ex  João trabalha.\label{ea}
\end{exe}

\n Essa sentença é formada por um sintagma nominal (NP) funcionando como sujeito e um sintagma verbal (VP) funcionando como predicado. O NP sujeito é formado exclusivamente por um nome (N), no caso o nome próprio \textit{João}. O VP é formado exclusivamente por um verbo (V), no caso o verbo intransitivo \textit{trabalhar}. Tudo isso está representado na estrutura abaixo, que servirá de entrada para o sistema interpretativo que estamos criando:



\begin{figure}[H]
	\centerline{ \Tree [.S [.NP [.N João ] ] [.VP [.V trabalha ] ] ] } \caption{Sentença com verbo intransitivo }
\end{figure}

Vamos proceder de baixo para cima, começando pelos itens lexicais. Para o nome próprio, vamos assumir simplesmente que sua extensão é o indivíduo (em carne e osso) portador do nome (faça de conta, para simplificar, que só exista um João no mundo):

\begin{exe}
	\ex \den{João} = o indivíduo João
\end{exe}

\n Ou simplesmente:

\begin{exe}
	\ex \den{João} = João
\end{exe}

Com isso, estamos assumindo que saber o significado de um nome próprio é simplesmente saber a identidade do portador do nome. Ou seja, no caso dos nomes próprios, o sentido é a própria extensão. Já no caso do verbo \textit{trabalhar}, não queremos associá-lo a indivíduos específicos, mas sim a uma classe ou conjunto de indivíduos, a saber, o conjunto dos indivíduos que trabalham.

\begin{exe}
	\ex \den{trabalha} = $\{x | x\ \text{trabalha}\}$
\end{exe}

\n A notação $\{x | \phi \}$ deve ser lida como o conjunto formado única e exclusivamente pelos objetos $x$ que satisfazem a condição $\phi$.

Note que, nessa representação da extensão do verbo, evitamos menção a indivíduos particulares na especificação do conjunto. Ao contrário, optamos pela especificação de uma condição: para que um indivíduo \textit{x} qualquer pertença à extensão desse verbo é necessário (e suficiente) que este indivíduo trabalhe. Com isso, queremos salientar que a extensão de um verbo não equivale a seu significado. De fato, saber o significado do verbo \textit{trabalhar} não implica saber quem são nem quantas são as pessoas que trabalham no mundo. Ao contrário, saber o significado desse verbo implica apenas saber quais as condições que um indivíduo \textit{x} qualquer deve satisfazer para que se possa dizer de \textit{x} que \textit{x} trabalha. Assim, a entrada lexical acima ganha plausibilidade cognitiva, já que mostra que a extensão do verbo é um conjunto de indivíduos, sem, no entanto, especificar a identidade desses indivíduos. Em suma, apesar de estarmos lidando com um sistema extensional, não estamos nos comprometendo com a ideia de que o significado seja equivalente à extensão. Já havíamos visto isso com as sentenças, e agora estamos vendo um ponto semelhante com os verbos.

Feitas essas observações, continuemos com a interpretação de (\ref{ea}). Comecemos pelo NP sujeito. Como se trata de um nó não ramificado, vamos assumir que ele herda a extensão do nó imediatamente dominado por ele, ou seja, N:

\begin{exe}
	\ex $\left \llbracket \vcenter{%
		\hbox{\Tree [.NP [.N João ] ]}} \right \rrbracket$ = $\left \llbracket \vcenter{%
		\hbox{\Tree [.N João ]}} \right \rrbracket$
\end{exe}


\n Como \textbf{N} também é um nó não ramificado, temos:

\begin{exe}
	\ex $\left \llbracket \vcenter{%
	\hbox{\Tree [.N João ]}} \right \rrbracket$ = \den{João}
\end{exe}


\n \textit{João} é um item lexical e, para sabermos sua extensão,
basta consultarmos a entrada lexical correspondente:

\begin{exe}
	\ex \den{João} = João
\end{exe}

\n Em resumo:

\begin{exe}
	\ex $\left \llbracket \vcenter{%
	\hbox{\Tree [.NP [.N João ] ]}} \right \rrbracket$ = $\left \llbracket \vcenter{%
	\hbox{\Tree [.N João ]}} \right \rrbracket$ = \den{João} = João
\end{exe}


\n Efetuando um cálculo análogo para o sintagma verbal, obteremos:

\begin{exe}
	\ex $\left \llbracket \vcenter{%
	\hbox{\Tree [.VP [.V trabalha ] ]}} \right \rrbracket$ = $\left \llbracket \vcenter{%
	\hbox{\Tree [.V trabalha ]}} \right \rrbracket$ = \den{trabalha} = $\{x | x\ \text{trabalha}\}$
\end{exe}



\n Para simplificar a representa\-ção de nossas deriva\-çõ\-es
semânticas, em vez de colocarmos árvores inteiras no interior do
símbolo de extensão, utilizaremos apenas o rótulo correspondente à
raiz da estrutura. Assim, os resultados que acabamos de obter
podem ser apresentados da seguinte forma:

\begin{exe}
	\ex \den{NP} = \den{N} = \den{João} = João
\end{exe}

\begin{exe}
	\ex \den{VP} = \den{V} = \den{trabalha} = $\{x | x\ \text{trabalha}\}$
\end{exe}

\n Resta-nos agora calcular a extensão de \textbf{S}, a partir das extensões de NP e VP. A intuição é que para que a sentença (S) seja verdadeira, a extensão do sujeito (NP) deve pertencer à extensão do predicado (VP):

\begin{exe}
	\ex \den{S} = 1 \textit{sse} \den{NP} $\in$ \den{VP}\\
	\den{S} = 1 \textit{sse} $\text{João} \in \{x | x\ \text{trabalha}\}$
\end{exe}

\n Em outras palavras:

\begin{exe}
	\ex \den{S} = 1 \textit{sse} João trabalha
\end{exe}

\n Esse é exatamente o resultado que queríamos obter. A condi\-ção
necessária e suficiente para que a sentença (\ref{ea}) seja
verdadeira é que o João trabalhe. Precisamos apenas explicitar os mecanismos ou regras que usamos na derivação do resultado. Vamos chamá-los de Regra I e Regra II:

\begin{exe}
	\ex Regra I: \\
	Se X um nó não ramificado, cujo único constituinte imediato é Y, então \den{X} = \den{Y}.
\end{exe}

\begin{exe}
	\ex Regra II: \\
	Seja X um nó ramificado, cujos constituintes imediatos são Y e Z. Se \den{Y} é um indivíduo e \den{Z} é um conjunto, então \den{X} = 1 \textit{sse} \den{Y} $\in$ \den{Z} 
\end{exe}

\n Resumindo: Obtivemos as condições de verdade de uma sentença
valendo-nos exclusivamente de entradas lexicais e regras composicionais explicitamente
formuladas, dentro, portanto, do espírito composicional e extensional que apresentamos mais
acima.



Vejamos agora mais um exemplo, que servirá mais tarde para discutirmos o formalismo que estamos adotando. Considere a sentença (\ref{eb}) abaixo, para a qual vamos assumir a
estrutura a seguir:

\begin{exe}
	\ex João estuda e trabalha.\label{eb}
\end{exe}

\begin{figure}[H]
	\centerline{ \Tree [.S [.NP [.N João ] ] [.VP [.VP$_{1}$ [.V$_{1}$ estuda ] ] e [.VP$_{2}$ [.V$_{2}$ trabalha ] ] ] ] } \caption{ Sentença com coordenação de VPs }
\end{figure}


Para os nós não ramificados da estrutura acima, já sabemos como
proceder por meio da Regra I:

\begin{exe}
	\ex \den{NP} = \den{N} = \den{João} = João\\
	\den{VP$_{1}$} = \den{V$_{1}$} = \den{estuda} = \{\textit{x} | \textit{x} estuda\} \\
	\den{VP$_{2}$} = \den{V$_{2}$} = \den{trabalha} = \{\textit{x} | \textit{x} trabalha\}
\end{exe}

\n Para a conjun\-ção \textit{e}, vamos assumuir que sua extensão é a operação de interseção de conjuntos:

\begin{exe}
	\ex \den{e} = $\cap$
\end{exe}

\n Dados dois conjuntos $A$ e $B$, a interse\-ção de $A$ e
$B$ ($A \cap B$) é o conjunto formado pelos elementos que
pertencem tanto a $A$ quanto a $B$. Passemos, então, à interpreta\-ção do sintagma verbal VP, que tem
três constituintes imediatos. Dois deles têm por extensão
conjuntos, e o outro tem por extensão a opera\-ção de
interse\-ção. A intuição é que, em casos assim, aplicamos a operação aos conjuntos. Vamos explicitar esse procedimento através de uma outra regra:

\begin{exe}
	\ex Regra III: \\
	Seja X um nó ramificado. Se a extensão de um dos constituintes imediatos de X for uma opera\-ção sobre conjuntos e as extensões dos demais constituintes imediatos forem conjuntos, então \den{X} é o conjunto resultante da aplica\-ção daquela opera\-ção a esse(s) conjunto(s).
\end{exe}


Voltando ao exemplo (\ref{eb}), podemos aplicar a Regra III
para obter a extensão de VP:

\begin{exe}
	\ex \den{VP} = \den{VP$_{1}$} $\cap$ \den{VP$_{2}$}\\
	\den{VP} = $\{x | x\ \text{estuda}\} \cap \{x | x\ \text{trabalha}\}$\\
	\den{VP} = $\{x | x\ \text{estuda}\ \text{e}\ x\ \text{trabalha}\}$
\end{exe}

\n Como a Regra III não faz menção à ordem linear dos
constituintes em questão, poderíamos igualmente ter escrito:

\begin{exe}
	\ex \den{VP} = \den{VP$_{2}$} $\cap$ \den{VP$_{1}$}
\end{exe}

\n  No caso acima, isso não tem maiores
consequências, já que a opera\-ção de interse\-ção é comutativa, ou seja, $A \cap B$ é equivalente a $B\cap A$.

De volta à nossa deriva\-ção, devemos agora calcular a extensão de S em fun\-ção das extensões de NP e VP. Podemos aqui aplicar a Regra II:

\begin{exe}
	\ex \den{S} = 1 \textit{sse} \den{NP} $\in$ \den{VP}\\
	\den{S} = 1 \textit{sse} João $\in$ $\{x | x\ \text{estuda}\ \text{e}\ x\ \text{trabalha}\}$
\end{exe}

\n Em palavras:

\begin{exe}
	\ex \den{S} = 1 \textit{sse} João estuda e João trabalha.
\end{exe}

\n Novamente, obtivemos o resultado desejado: a condi\-ção
necessária e suficiente para que a sentença (\ref{eb}) seja
verdadeira é que João estude e trabalhe.

Por fim, passemos à interpreta\-ção da sentença (\ref{ec}) abaixo, cuja estrutura apresentamos em seguida:

\begin{exe}
	\ex João não trabalha.\label{ec}
\end{exe}

\begin{figure}[H]
	\centerline{ \Tree [.S [.NP [.N João ] ] [.VP$_{1}$ não [.VP$_{2}$ [.V trabalha ] ] ] ] } \caption{ Sentença com negação de VP }
\end{figure}


\n A novidade aqui é a negação \textit{não}, que representamos como um adjunto adverbial. A exemplo da conjunção \textit{e}, vamos interpretá-la como uma operação sobre conjuntos. Nesse caso, trata-se da operação de complementação. Dado um conjunto
$A$, o complemento de $A$ ( $\overline{A}$ ) é o conjunto
dos elementos que não pertencem a $A$. Temos então:

\begin{exe}
	\ex \den{não} = $^{\overline{\ \ \ \ }}$ (operação de complementação)
\end{exe}

\n Já para VP$_{2}$, teríamos:

\begin{exe}
	\ex \den{VP$_{2}$} = \den{V} = \den{trabalha} = $\{x | x\ \text{trabalha}\}$
\end{exe}

\n Para VP$_{1}$, já temos tudo o que precisamos: trata-se de um nó ramificado cujos constituintes têm por extensão um conjunto e uma operação. Basta, então, aplicarmos a Regra III:

\begin{exe}
	\ex \den{VP$_{1}$} = $\overline{\llbracket \text{VP}_{2} \rrbracket}$
\end{exe}

\n Pela definição da operação de complementação, teremos:

\begin{exe}
	\ex \den{VP$_{1}$} = $\{x | x \not\in \llbracket \text{VP}_{2} \rrbracket \}$ \\
	\den{VP$_{1}$} = $\{x | x \not\in \{x | x\ \text{trabalha}\} \}$ \\
	\den{VP$_{1}$} = $\{x | x\ \text{não trabalha}\}$
\end{exe}

\n Chegamos, por fim, à relação sujeito-predicado no topo da sentença. Basta aplicarmos a regra II:

\begin{exe}
	\ex \den{S} = 1 \textit{sse} \den{NP} $\in$ 	\den{VP$_{1}$} \\
		\den{S} = 1 \textit{sse} João $\in \{x | x\ \text{não trabalha}\}$ \\
		\den{S} = 1 \textit{sse} João não trabalha.
\end{exe}


\n Uma vez mais, obtivemos o resultado desejado: a condição
necessária e suficiente para que a sentença (\ref{ec}) seja
verdadeira é que João não trabalhe.

É comum apresentar uma derivação semântica com o uso de linhas numeradas e a indicação a cada passo da
justificativa do que está sendo afirmado. Para essa estrutura que acabamos de analisar, teríamos:

\begin{exe}
	\ex Derivação semântica de \textit{João não trabalha}: \\
	1. \den{NP} = \den{N} = \den{João} = João \hfill (Regra I: 2x; Léxico)\\
	2. \den{VP$_{2}$} = \den{V} = \den{trabalha} = $\{x | x\ \text{trabalha}\}$ \hfill (Regra I: 2x; Léxico)\\
	3. \den{não} = $^{\overline{\ \ \ \ }}$ \hfill (Léxico)\\
	4. \den{VP$_{1}$} = $\overline{\llbracket \text{VP}_{2} \rrbracket}$ \hfill (Linha 3; Regra III)\\
	5. \den{VP$_{1}$} = $\{x | x \not\in$ \den{VP$_{2}$}\} \hfill (Linha 4 e defini\-ção de complemento)\\
	6. \den{VP$_{1}$} = $\{x | x\ \not\in \{x | x\ \text{trabalha}\}\}$ \hfill (Linhas 2 e 5)\\
	7. \den{VP$_{1}$} = $\{x | x\ \text{não trabalha}\}$ \hfill (Linha 6)\\
	8. \den{S} = 1 \textit{sse} \den{NP} $\in$ \den{VP$_{1}$} \hfill (Regra II)\\
	9. \den{S} = 1 \textit{sse} João $\in \{x | x\ \text{não trabalha}\}$ \hfill (Linhas 1, 7 e 8)\\
	10. \den{S} = 1 \textit{sse} João não trabalha. \hfill (linha 9 e defini\-ção de conjunto)
\end{exe}

O alcance empírico do sistema que acabamos de ilustrar é
obviamente bastante limitado, já que nosso léxico é constituído
por apenas cinco itens. Ainda assim, o poder do método 
que vimos acima é capaz de fornecer explica\-çõ\-es interessantes
a respeito de algumas intui\-çõ\-es semânticas que todo falante do
português tem sobre as sen\-ten\-ças (\ref{ea})-(\ref{ec}), repetidas abaixo em (\ref{eax})-(\ref{ecx}):

\begin{exe}
	\ex João trabalha.\label{eax}
	\end{exe}

\begin{exe}
	\ex João estuda e trabalha.\label{ebx}
\end{exe}

\begin{exe}
	\ex João não trabalha.\label{ecx}
\end{exe}


\n Por exemplo,
todos nós concordamos que se (\ref{ebx}) for verdadeira, então
(\ref{eax}) será necessariamente verdadeira e (\ref{ecx})
necessariamente falsa. Vejamos o que nosso sistema tem a nos
dizer: de acordo com o que vimos acima, assumir que (\ref{ebx}) é
verdadeira é o mesmo que afirmar que o indivíduo João (que
representaremos abaixo por \textit{j}) pertence à interse\-ção do
conjunto das pessoas que estudam (vamos representar esse
conjunto por \textit{E}) com o conjunto das pessoas que trabalham
(vamos representar esse conjunto
por \textit{T}). Na nota\-ção da teoria dos conjuntos:

\begin{exe}
	\ex j $\in$ (E $\cap$ T) \label{i}
\end{exe}

\n Mas sabemos pelas defini\-çõ\-es da teoria dos conjuntos que se um elemento \textit{x} qualquer pertence à interse\-ção de dois conjuntos \textit{A} e \textit{B} quaisquer, então \textit{x} pertence a \textit{A} e \textit{x} pertence a \textit{B}. Logo, tanto (\ref{ii}) quanto (\ref{iii}) abaixo são consequências de (\ref{i}):

\begin{exe}
	\ex j $\in$ E \label{ii}
\end{exe}

\begin{exe}
	\ex j $\in$ T  \label{iii}
\end{exe}

\n Mas, de acordo com o que vimos mais acima, assumir (\ref{iii}) é o mesmo que assumir que a sentença (\ref{eax}) é verdadeira. Temos
assim uma explica\-ção simples e elegante sobre nossa intui\-ção a
respeito do fato de (\ref{ebx}) acarretar (\ref{eax}). Na mesma
linha, a teoria dos conjuntos nos informa que para qualquer
conjunto $A$, um elemento $x$ qualquer ou pertence a
$A$ ou não pertence a $A$, mas nunca as duas coisas
ao mesmo tempo. Sendo assim, (\ref{iii}) contradiz
(\ref{iv}) abaixo:

\begin{exe}
	\ex j $\not\in$ T  \label{iv}
\end{exe}

\n Mas vimos acima que assumir (\ref{iv}) é o mesmo que afirmar que
(\ref{ecx}) é verdadeira. Logo, a verdade de (\ref{ebx}) implica a
falsidade de (\ref{ecx}). Novamente, temos uma explica\-ção precisa
sobre nossa intui\-ção a respeito da rela\-ção semântica entre
essas duas sen\-ten\-ças.

Cumprimos assim o principal objetivo deste capítulo: motivar a
constru\-ção de um sistema interpretativo vericondicional que atribui
condi\-çõ\-es de verdade para as sen\-ten\-ças de uma língua
baseado no valor semântico (extensões) de suas partes (os
constituintes sintáticos). Motivamos ainda o uso de objetos
matemáticos na representa\-ção destes valores semânticos,
indicando como uma abordagem formal pode lançar luz sobre nossas
capacidade de realizar inferências de cunho semântico.

Continuaremos a proceder dessa forma no restante deste livro,
incrementando gradualmente o alcance empírico da teoria através da
análise de uma série de constru\-çõ\-es linguísticas. Quanto ao
formalismo, faremos uma pequena mudança de perspectiva que nos
permitirá um poder de generaliza\-ção maior na formula\-ção das
regras de composi\-ção. Como o leitor notará já a partir do
próximo capítulo, ao invés de lidarmos diretamente com conjuntos,
manipularemos \textsc{fun\-çõ\-es}. De fato, o uso de fun\-çõ\-es
será tão frequente no restante do livro, que iniciaremos o próximo capítulo com uma discussão sobre elas.

\section{Além de extensões e condi\-çõ\-es de verdade}

A identifica\-ção da extensão de uma sentença com seu valor de verdade e de seu significado com suas
condi\-çõ\-es de verdade pode ser taxada de limitada por uma série
de motivos. Nesta se\-ção vamos olhar para alguns deles. Reconhecer
essas limita\-çõ\-es e refletir sobre suas consequências servirá
para tornar
mais claro nosso real objeto de estudo.

Em primeiro lugar, considere os exemplos abaixo:

\begin{exe}
	\ex\label{vapi}
	\begin{xlist}
		\ex  A Terra é plana.\label{vapia}
		\ex  Pedro acredita que a Terra é plana.\label{vapib}
	\end{xlist}
\end{exe}

\n Note que, apesar de (\ref{vapib}) ter (\ref{vapia}) como um de seus constituintes sintáticos, o valor de verdade de (\ref{vapib}) não depende do valor de verdade de (\ref{vapia}).  Pode-se acreditar em coisas falsas, bem como não acreditar em coisas verdadeiras. Lembre-se, entretanto, de que a extensão de uma sentença é seu valor de verdade e que, de acordo com a versão extensional do princípio de composicionalidade que assumimos, a extensão de um constituinte é calculada a partir da extensão de seus subconstituintes. 

Exemplos como os acima colocam em xeque o sistema extensional que adotamos. Precisamos acessar o significado da oração subordinada, e não seu valor de verdade, para calcular o valor de verdade da oração principal. Para tanto, torna-se necessário atribuir a sentenças e a constituintes sintáticos em geral um outro tipo de valor semântico que possa ser manipulado pelas regras composicionais e que capte o seu significado. Esse outro valor semântico costuma ser chamado de \textsc{intensão} (com a letra \textit{s}). No caso de sentenças declarativas, a intensão é chamada de \textsc{proposição}. Assume-se, assim, que o verbo  \textit{acreditar} em (\ref{vapib}) relaciona indivíduos e proposições. 

É importante salientar que não se trata, simplesmente, de abandonar o sistema extensional que delineamos neste capítulo e que desenvolveremos no restante deste livro. Precisamos, sim, suplementá-lo com ingredientes intensionais a fim de analisarmos sentenças como (\ref{vapib}), por exemplo. Mas, como veremos a partir do próximo capítulo, há uma série de construções linguísticas para as quais um sistema puramente extensional é bem sucedido. 


Passando ao próximo ponto, ser ou não ser verdadeira não é um atributo que
pareça apropriado a sen\-ten\-ças não declarativas, como as
abaixo:


\begin{exe}
    \ex\label{ap}
    \begin{xlist}
        \ex  Quem o João está namorando?\label{apa}
        \ex  O João está namorando?\label{apb}
        \ex  Saia da sala!\label{apc}
        \ex  Passe a salada, por favor. \label{apd}
    \end{xlist}
\end{exe}

\n (\ref{apa}) e (\ref{apb}) são sen\-ten\-ças interrogativas que
exprimem pedidos de informa\-ção. Não faz sentido perguntar se são
verdadeiras ou falsas. Por conseguinte, não faz sentido atribuir a
elas condi\-çõ\-es de verdade. O mesmo se pode dizer das
sen\-ten\-ças imperativas em (\ref{apc}) e (\ref{apd}), que
exprimem uma ordem e um pedido, respectivamente. Se essa crítica
atinge em cheio nossa máxima de que saber o significado de uma
sentença é saber suas condi\-çõ\-es de verdade, devemos nos
perguntar se ela deve afetar também nosso propósito de adotar um
sistema composicional extensional dentro das linhas sugeridas na
se\-ção anterior. A resposta aqui é negativa. A única conclusão a
que devemos chegar frente a essa crítica é que a extensão de uma
sentença não declarativa não é um valor de verdade e seu significado não se reduz a condições de verdade. Aqui podemos apenas
especular que isso se deve, ao menos em parte, à extensão e ao significado de
palavras interrogativas, de formas verbais imperativas, ou mesmo
de morfemas abstratos, não pronunciados, mas presentes na estrutura destas
sen\-ten\-ças (o que tem um certo apelo frente a casos como
(\ref{apb}), por exemplo).

A li\-ção que devemos tirar dessas considera\-çõ\-es é que nossa
máxima deve ser entendida de forma mais restrita: saber o
significado de certas sen\-ten\-ças declarativas é saber suas
condi\-çõ\-es de verdade. Mas nunca é demais reafirmar que em
prin\-cí\-pio isso em nada abala nossa abordagem composicional
baseada na atribui\-ção de extensões a constituintes sintáticos.

O mesmo se pode dizer a respeito de certas sen\-ten\-ças
declarativas chamadas de performativas:


\begin{exe}
    \ex\label{per}
    \begin{xlist}
        \ex  Eu vos declaro marido e mulher.\label{pera}
        \ex  Eu te batizo em nome do Pai, do Filho e do Espírito Santo.\label{perb}
    \end{xlist}
\end{exe}

\n Quando um padre pronuncia sen\-ten\-ças como essas, ele não
está descrevendo um pedaço do mundo. Agindo de acordo com um
ritual e com o poder a ele atribuído, suas próprias palavras têm
valor de transforma\-ção, ou seja, é a própria enuncia\-ção da
sentença que torna os noivos um casal perante a igreja, assim como
a pessoa batizada um cristão. Nesses casos é o próprio ato de fala
que está em questão, e de fato soa estranho perguntar a alguém
presente em uma dessas cerimônias se o que o padre está dizendo é
verdade ou não. Cabem aqui as mesmas observa\-çõ\-es que fizemos
acima a respeito das sen\-ten\-ças em (\ref{ap}), não havendo
razões para abandonarmos nossa abordagem composicional
extensional. Devemos apenas limitar nossa meta inicial (a
atribui\-ção de condi\-çõ\-es de verdade) a um subconjunto das
sen\-ten\-ças declarativas e refletir sobre que objetos atribuir a
sen\-ten\-ças performativas. Novamente, uma discussão mais
profunda sobre essas sen\-ten\-ças e o seu uso foge ao escopo
deste livro (mas ver sugestões de leitura ao final do capítulo).

Por fim, devemos ter em mente que \textit{significado} para nós é
um termo técnico, que certamente não faz jus às diversas
acep\-çõ\-es que essa palavra tem no uso ordinário da língua. Isso
se torna especialmente claro ao se notar que \textit{significado}
como viemos assumindo é algo que se aplica a sen\-ten\-ças
isoladas e que não leva em conta aspectos relacionados ao seu uso
em contextos específicos. Considere por exemplo os dois diálogos
abaixo:

\begin{exe}
    \ex A: Que tipo de roupa a namorada nova do João costuma usar?\\
    B: Ela só usa roupas de grife. \label{fg}
\end{exe}

\begin{exe}
    \ex A: A namorada nova do João é bonita?\\
    B: Ela só usa roupas de grife. \label{gr}
\end{exe}

\n Nos dois casos, \textit{B} responde a \textit{A} com a mesma
sentença. Mas fica claro que os significados (tomados no sentido
ordinário do termo) dessas respostas não são os mesmos. No caso de
(\ref{gr}), mas não no de (\ref{fg}), tendemos a inferir que
\textit{B} não acha a namorada nova do João bonita. A razão para
essa diferença parece clara: em (\ref{gr}), \textit{B} se esquiva
de dar uma resposta direta à pergunta de \textit{A}. Como não
haveria razão para essa esquiva caso ele achasse a namorada de
João bonita e como pessoas tendem a ser polidas ao fazer
julgamentos estéticos (sobretudo negativos), acabamos por concluir
que \textit{B} não acha a namorada de João bonita.

Note agora duas coisas. Em primeiro lugar, ao usarmos o termo
\textit{significado} no parágrafo anterior, nós o aplicamos à
resposta de \textit{B}. Uma resposta não é o mesmo que uma
sentença. Sua caracteriza\-ção faz men\-ção necessária a um
contexto de fala, sobretudo a uma pergunta que a antecede. Nesse
sentido, fica claro porque as respostas de \textit{B} acima têm
significados diferentes: elas respondem perguntas distintas. Em
segundo lugar, na origem das inferências acima está o fato de
reconhecermos que, em (\ref{gr}), \textit{B} se esquivou de dar uma
resposta direta a \textit{A}. Mas isso implica assumir que a
sentença usada por \textit{B} tem uma propriedade que independe do
contexto em que é usada. Afinal, o que queremos dizer ao afirmar
que em (\ref{fg}), mas não em (\ref{gr}), \textit{B} dá uma resposta
direta a \textit{A}? Essa propriedade é justamente o
\textit{significado} da sentença usada por \textit{B} em ambos os
diálogos. Em suma, a compreensão do que se passa nos diálogos
acima requer conhecimento do significado da sentença
usada, bem como dos contextos em que ela é usada, incluindo aí as
intenções do falante. Dessa intera\-ção resulta o que podemos
chamar de significado do uso de uma sentença,
\textsc{significado do enunciado} ou mesmo \textsc{significado do
falante}. Neste livro, é com o significado de sen\-ten\-ças que
estaremos lidando, ainda que continuemos a usar o termo sem
qualificativos. Essa distin\-ção está no cerne da divisão de
trabalho entre duas áreas da linguística preocupadas com a no\-ção
de significado entendida de maneira bastante ampla: a
\textsc{semântica}, que estuda o significado literal ou convencional de sen\-ten\-ças (e suas partes) sem levar em conta o
contexto de fala e as intenções do falante, e a \textsc{pragmática}, que
estuda o significado de enunciados e que, portanto, não pode
prescindir de uma caracteriza\-ção precisa dos contextos em que
usamos sen\-ten\-ças e outras expressões linguísticas, bem como
dos diversos propósitos comunicativos subjacentes a esses usos.

\bigskip

\begin{tcolorbox}[parbox=false,boxrule=0pt,sharp corners,breakable]

\section*{Sugestões de leitura}
\addcontentsline{toc}{section}{Sugestões de leitura}



Sobre os fundamentos de uma teoria semântica baseada em
condições de verdade, as obras clássicas são \cite{tarski44} e
\cite{davidson67}. Introduções acessíveis podem ser encontradas
em \cite{taylor98}, capítulo 3 e \cite{larseg95}, capítulos 1
e 2. Quanto a seu enquadramento nas ciências cognitivas, ver
\cite{larseg95}, capítulo 13.

Sobre a ideia de se tomar um valor de verdade como sendo a
extensão ou o referente de uma sentença, ver \cite{frege92}, outro
grande clássico da área.

A grande figura que impulsionou os trabalhos atuais em semântica
formal é o filósofo americano Richard Montague. Suas principais
obras podem ser encontradas em \cite{thomason74}. Essas, entretanto, são obras
de difícil leitura para o estudante que inicia seus estudos na
área. Os leitores interessados poderão se beneficiar da ótima
introdução em \cite{gamut91}, volume 2.

Sobre noções básicas de teoria dos Conjuntos, ver \cite{partee90}, parte A. Sobre as relações entre
diferentes acepções do termo significado, ver vários textos em \cite{grice89}.
Dois grandes clássicos na área da pragmática em que aspectos não vericondicionais do significado são discutidos são \cite{grice75} e \cite{austin62}.
Para uma boa introdução à Pragmática, ver \cite{levinson83}, traduzido para o português como \cite{levinson07}, e \cite{pirbas14}. Para introduções à sintaxe, ver \cite{carnie13}, \cite{haegeman94} e \cite{mioal05}. 

Por fim, gostaria de mencionar algumas introduções à semântica
formal cujos conteúdos se superpõem parcialmente ao conteúdo deste
livro: em inglês, \cite{cann93}, \cite{larseg95}, \cite{chimcc00},
\cite{heikra98}, \cite{jacobson14}, \cite{deswart03}, \cite{winter16}; em português, \cite{chierchia03} e \cite{pires01}.

\end{tcolorbox}

\bigskip

\begin{tcolorbox}[boxrule=0pt,sharp corners,breakable]

\section*{Exercícios}
\addcontentsline{toc}{section}{Exercícios}


\n\textbf{I.} Dizer se (a) acarreta (b), e se (b) acarreta (a).
Justifique sua resposta. Em particular, se sua resposta for negativa,
 descreva explicitamente situa\-çõ\-es possíveis que suportem essa
resposta.\\

\n (a) O menor estado brasileiro fica na região Nordeste.

\n (b) Sergipe fica na região Nordeste.\\

\n\textbf{II.} Justifique as seguintes afirmativas, em que
\textit{S} é uma sentença qualquer:\\

\n (a) Se \textit{S} é uma tautologia, então qualquer sentença
acarreta \textit{S}.

\n (b) Se \textit{S} é uma contradi\-ção, então \textit{S}
acarreta qualquer sentença.

\n (c) Se \textit{S} é uma contradi\-ção, então nenhuma
contingência
acarreta \textit{S}.\\

\n\textbf{III.} Dizer se a seguinte afirma\-ção é verdadeira ou
falsa, justificando: ``nenhuma sentença acarreta sua nega\-ção."\\

\n\textbf{IV.} Assuma as seguintes extensões para os sintagmas
verbais \textit{viajou} e \textit{viajou de carro}:\\

\n \den{viajou} = $\{x | x\ \text{viajou}\}$

\n \den{viajou de carro} = $\{x | x\ \text{viajou de
carro}\}$\\

\n Valendo-se da teoria dos conjuntos, explique porque (a)
acarreta (b), mas (b) não acarreta (a):\\

\n (a) João viajou de carro.

\n (b) João viajou.\\

\n Dica: dados dois conjuntos $A$ e $B$ quaisquer, $A$ é um subconjunto de $B$ ($A \subseteq B$) se, e somente se, todo elemento que pertence a $A$ também pertence a $B$. \\

\n \textbf{V.} Calcule, passo a passo, as condi\-çõ\-es de verdade
da
sentença abaixo, de acordo com o sistema que desenvolvemos na se\-ção 1.5:\\

\n (a) João estuda e não trabalha.\\

\n\textbf{VI.} Considere a sentença abaixo:\\

\n (a) João não estuda e trabalha.\\

\n Podemos pensar em duas estruturas para essa
sentença:

\begin{center}
	\Tree [.S [.NP [.N João ] ] [.VP [.VP$_{1}$ não [.VP$_{3}$ [.V$_{3}$ estuda ] ] ] e [.VP$_{2}$ [.V$_{2}$ trabalha ] ] ] ]
\end{center}

\begin{center}
	\Tree [.S [.NP [.N João ] ] [.VP não [.VP$_{1}$ [.VP$_{2}$ [.V$_{2}$ estuda ] ] e [.VP$_{3}$ [.V$_{3}$ trabalha ] ] ] ] ]
\end{center}

\n Para cada uma delas, calcule passo a passo as condi\-çõ\-es de
verdade que o sistema da se\-ção 1.5 fornece. Em seguida, diga se
alguma delas (ou ambas) condizem com suas intui\-çõ\-es a respeito
do
significado dessa sentença.\\

\n\textbf{VII.} Considere a sentença abaixo:\\

\n (a) João estuda ou trabalha.\\

\n Assuma que o conectivo \textit{ou} denote a opera\-ção de união ($\cup$) entre conjuntos e calcule passo a passo as condi\-çõ\-es de
verdade de (a). Discuta a adequa\-ção empírica do resultado,
tomando por base sua intui\-ção sobre o significado desta
sentença. (Nota: um objeto \textit{x} qualquer pertencerá à união
entre dois conjuntos se, e somente se, \textit{x} pertencer a pelo
menos um desses conjuntos.)

\end{tcolorbox}










%%%%%
