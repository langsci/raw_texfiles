%% -*- coding:utf-8 -*-
\documentclass[output=paper]{langscibook}
\author{Tilman N. Höhle}

% This would hyphenate the title.
% But it looks ugly.
%\renewcommand*{\raggedsection}{}% default is \raggedright
\title{Vorangestellte Verben und Komplemen- tierer sind eine natürliche\newlineTOC{} Klasse}
% \title{Vorangestellte Verben und Komplementierer sind eine natürliche Klasse und noch eine dritte Zeile}
\abstract{}
\rohead{\thechapter\hspace{0.5em}Vorangestellte Verben und Komplementierer} % Display short title
%% \renewcommand{\lsCollectionPaperCitationText}{%
%% Tilman N. Höhle. 2018. Observing non-finite verbs: Some V3 phenomena in
%% German-Dutch. In Stefan Müller, Marga Reis \& Frank Richter (eds.), \emph{Beiträge
%% zur deutschen Grammatik: Gesammelte Schriften von Tilman N. Höhle}, {\color{red} 1-9999}.
%% Berlin: Language Science Press. {\color{lsGuidelinesGray} DOI: 10.5281/zenodo.1168353}
%% }
\maketitle
\ChapterDOI{10.5281/zenodo.1168353}
\begin{document}
\selectlanguage{german}
\label{chap-komplementierer}


\noindent
This chapter was originally published by Niemeyer, which now belongs to De Gruyter. While all other publishers gave us permission to publish Höhle's papers under a CC-BY
license, Springer and De Gruyter insisted on CC-BY-NC-ND (Attribution"=NonCommercial"=No\-De\-riv\-a\-tives).
This means that we cannot put these papers on commercial platforms like Google Book Store and GitHub. Please
visit \url{http://langsci-press.org/catalog/book/149} to access the complete book.


\pagebreak~
Page left blank intentionally.

\pagebreak~
Page left blank intentionally.

\setcounter{page}{433}

\refstepcounter{mylastpagecount}\label{chap-komplementierer-end}
\pagebreak~
\end{document}
