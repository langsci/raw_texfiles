%% -*- coding:utf-8 -*-
\documentclass[output=paper]{LSP/langsci}
\author{Tilman N. Höhle}

\title{Spuren in {HPSG}}
%\epigram{Change epigram in chapters/01.tex or remove it there}
\abstract{}
\maketitle
% \rohead{\thechapter\hspace{0.5em}short title} % Display short title
\ChapterDOI{10.5281/zenodo.1169687}
\begin{document}
\label{chap-spuren-hpsg}
\selectlanguage{german}
\setcounter{randcount}{0}

\renewcommand*{\thefootnote}{\fnsymbol{footnote}}
\setcounter{footnote}{4}
\footnotetext{%
\emph{Anmerkung der Herausgeber:} Das Papier stellt die bisher unveröffentlichte Textfassung eines
Vortrags auf der GGS"=Tagung in Tübingen am 14.\,Mai 1994 dar.
Formatierung, Absatz- und Beispielnumerierung sowie der
Zitierstil wurden weitgehend an die Konventionen des Bandes
angepasst. Den Beschreibungen der Werte von \textsc{slash} in (\ref{ex:16-2})
wurden Mengenklammern hinzugefügt, die in der Vorlage nicht stehen.%
}

\renewcommand*{\thefootnote}{\arabic{footnote}}
\setcounter{footnote}{0}

\section{Merkmalarchitektur von HPSG}
\label{sec:16-1}

{\randnum}Die Extraktionstheorie von \isi{HPSG} ist in den Grundzügen ähnlich wie die
von \citet{gazdaretal1985a}. Infolge einiger allgemeiner Unterschiede zwischen \isi{GPSG}
und \isi{HPSG} haben Spuren jedoch etwas andere Eigenschaften.

{\randnum}Das \textsq{X-bar System} ist reduziert. Es wird nur zwischen X\textsuperscript{0} (\textit{word}) und
X\textsuperscript{1} (\mbox{\textit{phrase}}) unterschieden. Eine \isi{Konstituente} (\textit{sign}) ist genau dann
ein X\textsuperscript{1}, wenn sie eine Tochter hat (ein \textsc{dtrs}"=Attribut trägt).

{\randnum}Jede \isi{Konstituente} trägt ein Valenzattribut
(hier: \textsc{subcat}, kurz \textsc{sct}), in dessen Wert die noch
nicht abgebundenen selegierten Konstituenten aufgelistet sind. Dieses
Attribut ist in dem Wert eines Attributs \textsc{local} (kurz:
\textsc{l}) enthalten. Anstelle von \textsq{ID-Regeln} wie in \isi{GPSG}
gibt es Wohlgeformtheitsbedingungen über die Kombination von
\textsc{dtrs}"=Werten und \textsc{sct}"=Werten an Mutter und Head"=Tochter.

{\randnum}Die \textsq{Wortstellung} in einem Satz
\textit{Kim likes candy} ist dadurch beobachtbar, daß ihm eine
zeitlich geordnete Lautfolge {kɪm\#laɪks\#kændɪ} entspricht. In
\isi{GPSG} wird diese Ordnung auf sehr indirekte Weise erreicht.
\pagebreak
\begin{exe}
\extab
\begin{tikzpicture}[sibling distance=30pt]
\tikzset{every tree node/.style={align=center,anchor=base}}
\tikzset{level 1+/.style={level distance=30pt}}
\tikzset{frontier/.style={distance from root=90pt}}
\Tree [.{V\textsuperscript{2}[\textsc{subj} $+$]} [.N\textsuperscript{2}nom \edge[dashed]; {kɪm} ] [.{V\textsuperscript{2}[\textsc{subj} $-$]} [.V\textsuperscript{0} \edge[dashed]; {laɪks} ] [.N\textsuperscript{2}acc \edge[dashed]; {kændɪ} ] ] ]
\end{tikzpicture}
\end{exe}
{\randnum}Zwischen den Kategorien
N\textsuperscript{2}nom und V\textsuperscript{2}[\textsc{subj} $-$] und
zwischen den Kategorien V\textsuperscript{0} und N\textsuperscript{2}acc ist eine
transitive (und vermutlich assoziative) \textsq{lineare}
Ordnungsbeziehung definiert (geschrieben: $<$), die möglicherweise als
  zeitliche Ordnung verstanden werden soll. Die phonologischen
  Ausdrücke {kɪm}, {laɪks} und {kændɪ} sind nur als Quasi"=Töchter der
  terminalen Kategorien eingeführt \citep[104]{gazdaretal1985a}; zwischen ihnen
  besteht (deshalb) keine definierte Ordnungsbeziehung.

  {\randnum}In \isi{HPSG} trägt dagegen jede \isi{Konstituente} ein
  \textsc{phon}"=Attribut, dessen Wert eine Liste von phonologischen
  Ausdrücken ist. Die X\textsuperscript{0}"=Konstituenten des Satzes:
\begin{exe}
\extab 
\label{rn:16-8}
\begin{tabular}{l@{\hspace{2em}}l@{\hspace{2em}}l}
N\textsuperscript{0}nom & V\textsuperscript{0} & N\textsuperscript{0}acc \\
\textsc{phon} \nliste{ kɪm } & \textsc{phon} \nliste{ laɪks } & \textsc{phon} \nliste{ kændɪ } \\
\textsc{l}|\ldots{}|\textsc{sct} \textit{elist} & \textsc{l}|\ldots{}|\textsc{sct} \nliste{ Nnom, Nacc } & \textsc{l}|\ldots{}|\textsc{sct} \textit{elist} \\
\end{tabular}
\end{exe}
{\randnum}Die X\textsuperscript{1}"=Konstituenten:
\begin{exe}
\extab 
\begin{tabular}{l@{\hspace{4em}}l}
V\textsuperscript{1} & V\textsuperscript{1} \\
\textsc{phon} \nliste{ laɪks, kændɪ } & \textsc{phon} \nliste{ kɪm, laɪks, kændɪ } \\
\textsc{l}|\ldots{}|\textsc{sct} \nliste{ Nnom } & \textsc{l}|\ldots{}|\textsc{sct} \textit{elist} \\
\textsc{dtrs} \ldots{} & \textsc{dtrs} \ldots{} \\
\end{tabular}
\end{exe}
\randnum\label{rn:16-10}Wie in \isi{GPSG} ist eine \textsq{Spur} ein
lexikalisches Element, das sich von anderen durch 3~stipulierte
Eigenschaften unterscheidet: Es ist phonologisch leer; es trägt ein
\textsc{slash}"=Attribut; und der Wert von \textsc{slash} ist identisch mit grammatisch
relevanten anderen Eigenschaften der Spur, hier: mit dem Wert von
\textsc{local}. Mit Hilfe des Nonlocal Feature Principles (entspricht dem
Foot Feature Principle in \isi{GPSG}) wird der \textsc{slash}"=Wert außerdem mit dem
\textsc{local}"=Wert des Spurbinders identifiziert.
\begin{exe}
\extab
\label{rn:16-11}
\begin{tabular}{l@{\hspace{4em}}l}
\isi{HPSG}"=Spur: & X \\
& \textsc{phon} \textit{elist} \\
& \textsc{local} \avmbox{1} \\
& \textsc{slash} $\{\avmbox{1}\}$
\end{tabular}
\end{exe}

\section{Lineare Position von leeren Kategorien}

{\randnum}\isi{Komplement}"=\isi{Extraktion} ohne Spur: \citet{pollard1985a,pollard1988a}; \citet{Cooper1990}. Mit Spur: \citet{PollardSag1987}, \citet[Kapitel 4--6]{PollardSag1992}.

\randnum\label{rn:16-13}In \cite[Kapitel 9.5.1]{PollardSag1992}
%\intcomm{Check.} 
zitieren P\&S Beobachtungen von \citet{PickeringBarry1991} und motivieren damit eine
Extraktionstheorie ohne Spuren:
\begin{quotation}
  In this section, we will offer a revised theory of filler"=gap
  constructions that begins to make sense of the Pickering"=Barry
  results. [\ldots{}] The basic idea is that \textsc{slash} originates not
  on traces, but rather from the head that licences the
  \textsqe{missing} element.

  Technically, the most straightforward way to implement this idea is
  by lexical rules.
\end{quotation}
{\randnum}\textbf{Klammer auf.} Außer einer
Complement Extraction Lexical Rule (§9.5.1 (62); in \citealt{PollardSag1992iE} gegenüber
\citealt{PollardSag1992} leicht verändert) schlagen P\&S eine Adjunct Extraction Lexical
Rule (§9.5.2 (80)) vor. Diese Regel ist aus syntaktischen Gründen
(vgl.\ \citealt{Hukari1993}) und aus semantischen Gründen nicht haltbar. Die
syntaktischen Mängel kann man durch folgende Umformulierung
beseitigen:
\begin{exe}
\ex
\begin{avm}
\avml  {\[synsem{\[local \avmbox{a} {\[category {\[head \tpv{verb}\]}\\\]}\\nonlocal|inher|slash \tpv{eset} \]}\]} \\[10mm]
\quad $\mapsto$
  {\[synsem{\[local|content \avmbox{2} \\nonlocal|inher|slash {\{Y$^{\prime\prime}$ {\[mod {\[local \avmbox{a} \]} \]}: \avmbox{2} \}} \]}\]}
\avmr
\end{avm}
\end{exe}
{\randnum}Es bleiben jedoch semantische Probleme
bestehen. Bei Verwendung einer \isi{Adjunkt}"=Spur treten die nicht
auf. \textbf{Klammer zu.}

\enlargethispage{\baselineskip}
{\randnum}\citet{Pickering1993}, \citet{PickeringBarry1991}:
{\setlist[enumerate]{leftmargin=*}
\begin{enumerate}[widest=(PB33)]
\item[(PB32)] We \underline{gave} every student capable of answering every single
  tricky question on the details of the new and extremely complicated
  theory about the causes of political instability in small nations
  with a history of military rulers \underline{a prize}.
\item[(PB33)] That's the prize [which]\textsubscript{\textit{i}} we \underline{gave} every student capable
  of answering every single tricky question on the details of the new
  and extremely complicated theory about the causes of political
  instability in small nations with a history of military rulers \underline{{[t]}$_i$}.
\end{enumerate}
}
{\randnum}"`Sentence (PB 32) causes some processing
difficulty, which appears to be because \textit{gave} and \textit{a prize} are separated
by an argument of considerable length and complexity. [\ldots{}] the
gap"=filling account predicts at least the same difficulty for (PB 33)
as (PB 32), and this is clearly not the case."' \citep[169]{Pickering1993}

{\randnum}Das Argument ist richtig für alle Theorien,
in denen leere Kategorien (besonders Spuren) im Grundsatz denselben
Ordnungsbedingungen unterliegen wie entsprechende nicht"=leere
Konstituenten, \zb \isi{GPSG} und Varianten von GB. Es ist nicht richtig
für \isi{HPSG}.

{\randnum}Die relevanten Konstituenten von (\ref{ex:16-1}) sind in (\ref{ex:16-2}) gegeben. (Für \textit{Kim} und \textit{likes} s.\,o.\ (\ref{rn:16-8}).)
\begin{exe}
\ex%(1)
\label{ex:16-1}
who Kim likes t 
\ex%(2)
\label{ex:16-2}
\begin{xlist}
\extab%(2) (a)
\label{ex:16-2a}
\begin{tabular}{p{70pt}l}
\textit{who}: & N\textsuperscript{1} \\
& \textsc{phon} \nliste{ hu } \\
& \textsc{l} \avmbox{1} [\ldots{}|\textsc{sct} \textit{elist}] \\
& \textsc{dtrs} \ldots{}
\end{tabular}\Hack{\vspace*{.5\baselineskip}}
\extabb%(2) (b)
\label{ex:16-2b}
\begin{tabular}{p{70pt}l}
t: & N\textsuperscript{1}acc \\
& \textsc{phon} \textit{elist} \\
& \textsc{local} \avmbox{1} \\
& \textsc{dtrs} \ldots{} \\
& \textsc{slash} $\{\avmbox{1}\}$
\end{tabular}\Hack{\vspace*{.5\baselineskip}}

\extabb%(2) (c)
\label{ex:16-2c}
\begin{tabular}{p{70pt}l}
\textit{likes} t: & V\textsuperscript{1} \\
& \textsc{phon} \nliste{ laɪks } \\
& \textsc{l}|\ldots{}|\textsc{sct} \nliste{ Nnom } \\
& \textsc{dtrs} \ldots{} \\
& \textsc{slash} $\{\avmbox{1}\}$
\end{tabular}\Hack{\vspace*{.5\baselineskip}}

\extabb%(2) (d)
\label{ex:16-2d}
\begin{tabular}[c]{p{70pt}l}
\textit{Kim likes} t: & V\textsuperscript{1} \\
& \textsc{phon} \nliste{ kɪm, laɪks } \\
& \textsc{l}|\ldots{}|\textsc{sct} \textit{elist} \\
& \textsc{dtrs} \ldots{} \\
& \textsc{slash} $\{\avmbox{1}\}$
\end{tabular}\Hack{\vspace*{.5\baselineskip}}

\extabb%(2) (e)
\label{ex:16-2e}
\begin{tabular}{p{70pt}l}
\textit{who Kim likes} t: & V\textsuperscript{1} \\
& \textsc{phon} \nliste{ hu, kɪm, laɪks } \\
& \textsc{l}|\ldots{}|\textsc{sct} \textit{elist} \\
& \textsc{dtrs} \ldots{} \\
\end{tabular}
\end{xlist}
\end{exe}
\addlines[2]
{\randnum}Der wesentliche Punkt zeigt sich in (\ref{ex:16-2c}): Der
\textsc{phon}"=Wert dieser \isi{VP} ist gleich dem \textsc{phon}"=Wert des Worts \textit{likes}
(\ref{rn:16-8}). Das ergibt sich aus dem Constituent Ordering Principle:
\begin{quotation}
  The essential content of the COP is simply that the phonology of a
  phrasal sign is functionally dependent upon the daughters. [\ldots{}]

  In general [\ldots{}] we assume that [the \textsc{phon} value of the mother is] a
  permutation [\ldots{}] of the \textsc{phon} values of the sign's daughters.
\citep[169]{PollardSag1987}
\end{quotation}
{\randnum}Da die Spur (\ref{ex:16-2b}) die leere Liste als
\textsc{phon}"=Wert hat, ist jede Permutation der \textsc{phon}"=Werte
der Töchter von (\ref{ex:16-2c}) gleich dem \textsc{phon}"=Wert von \textit{likes}. Generell
ist der \textsc{phon}"=Wert einer leeren Kategorie im
\textsc{phon}"=Wert der Mutter niemals auf"|findbar.

{\randnum}Offensichtlich kann und muß die Ordnung im
\textsc{phon}"=Wert der maximalen \isi{Konstituente} (\ref{ex:16-2e}) als zeitliche
Ordnung verstanden werden. Man kann die Theorie so präzisieren, daß
dies zugleich die einzige zeitlich interpretierte Ordnungsbeziehung
ist. Eine zeitlich interpretierte Ordnung zwischen
Schwesterkonstituenten ist nicht ausdrücklich vorgesehen und
überflüssig. (Eine zeitliche (transitive assoziative) Ordnung auf
der Menge der Wörter in einem Ausdruck ist bei \citealt{Pollardetal1994b} explizit ausgeschlossen.)
Man braucht deshalb nicht auf Spuren zu verzichten, um "`to make sense of the Pickering"=Barry results"'
(Absatz~\ref{rn:16-13}).

\section{\textit{wanna}"=Kontraktion}
\begin{exe}
\ex%(3) 
\label{ex:16-3}
\begin{xlist}
\ex[]{%(3) (a)
\label{ex:16-3a}
who$_i$ you want (PRO) to leave t$_i$}
\ex[]{%(3) (b)
\label{ex:16-3b}
who$_i$ you wanna leave t$_i$}
\end{xlist}
\ex%(4)
\begin{xlist}
\ex[]{%(4) (a)
\label{ex:16-4a}
who$_i$ you want t$_i$ to leave the room}
\ex[\%]{%(4) (b)
\label{ex:16-4b}
who$_i$ you wanna (t$_i$) leave the room}
\end{xlist}
\end{exe}
{\randnum}Es ist zweifelhaft oder unwahrscheinlich, daß
die \textsq{\textit{wanna}"=Kontraktion} ein produktiver morphophonologischer
Prozeß des heutigen Englischen\il{Englisch} ist. Daher ist es möglich oder
notwendig, für \textit{want} in (\ref{ex:16-3a}) und \textit{wanna} in (\ref{ex:16-3b}) verschiedene
Lexikoneinträge zu formulieren. In (\ref{ex:16-5}) ist die jeweils benötigte
Valenzinformation gegeben.
%% \begin{exe}
%% \ex
%% \label{ex:16-5}
%% \begin{xlist}
%% \ex%(5) (a)
%% \label{ex:16-5a}
%% \textit{want}: \quad Nnom\textsubscript{1} V\textsuperscript{1}[\textit{inft}, \textsc{sct} \nliste{ N\textsubscript{1} }]
%% \ex%(5) (b)
%% \label{ex:16-5b}
%% \textit{wanna}: \quad Nnom\textsubscript{1}, V\textsuperscript{1}[\textit{base}, \textsc{sct} \nliste{ N\textsubscript{1} }]
%% \end{xlist}
%% \end{exe}
\ea
\label{ex:16-5}
\begin{tabular}[t]{@{}l@{~~}l@{\quad}l}
%\ex%(5) (a)
%\label{ex:16-5a}
a. & \textit{want}: & Nnom\textsubscript{1}, V\textsuperscript{1}[\textit{inft}, \textsc{sct} \nliste{ N\textsubscript{1} }]\\
%\ex%(5) (b)
%\label{ex:16-5b}
b. & \textit{wanna}: & Nnom\textsubscript{1}, V\textsuperscript{1}[\textit{base}, \textsc{sct}
  \nliste{ N\textsubscript{1} }]
\end{tabular}
\z
{\randnum}Für das \textit{want} in (\ref{ex:16-4a}) ist ein anderer Lexikoneintrag nötig:
(\mex{1}a). Für die Varianten des Englischen\il{Englisch}, in denen (\ref{ex:16-4b}) akzeptabel ist, ist
dann ein Lexikoneintrag wie in (\mex{1}b) nötig. 
%% \eal
%% \ex%(6) (a)
%% \label{ex:16-6a}
%% \textit{want}: \quad Nnom, \avmbox{2} Nacc, V\textsuperscript{1}[\textit{inft}, \textsc{sct} \nliste{ \avmbox{2} }]
%% \ex%(6) (b)
%% \label{ex:16-6b}
%% \textit{wanna}: \quad Nnom, \avmbox{2} Nacc [\textsc{local} \avmbox{3}, \textsc{slash} \{\avmbox{3}\}], V\textsuperscript{1}[\textit{base}, \textsc{sct} \nliste{ \avmbox{2} }]
%% \zl
\ea
\label{ex:16-6}%
%\ex%(6) (a)
%\label{ex:16-6a}
\begin{tabular}[t]{@{}l@{~~}l@{\quad}l}
a. & \textit{want}: & Nnom, \avmbox{2} Nacc, V\textsuperscript{1}[\textit{inft}, \textsc{sct} \nliste{ \avmbox{2} }]\\
%\ex%(6) (b)
%\label{ex:16-6b}
b. & \textit{wanna}: & Nnom, \avmbox{2} Nacc [\textsc{local} \avmbox{3}, \textsc{slash}
  \{\avmbox{3}\}], V\textsuperscript{1}[\textit{base}, \textsc{sct} \nliste{ \avmbox{2} }]
\end{tabular}
\z
{\randnum}Der Lexikoneintrag (\mex{0}b) ist dadurch
\textsq{markiert}, daß durch ihn eine Spur selegiert wird und kein
entsprechender Lexikoneintrag mit normaler \isi{NP} statt der Spur
existiert. Es ist plausibel, daß er aufgrund positiver Evidenz
\textsq{gelernt}, aber nicht ohne solche Evidenz aufgebaut wird.

\section{Extraktion von V-Projektionen bei \textsq{kohärenter Konstruktion}}

{\randnum}Wenn man annimmt, daß bei \textsq{kohärenter
  Konstruktion} obligatorisch ein Verbalkomplex wie in (\ref{ex:16-7}) existiert,
hat man u.\,U.\ keine Probleme mit Extraktionen wie in (\ref{ex:16-8}); aber alle
Theorien haben Probleme mit (\ref{ex:16-9}):
\begin{exe}
\ex%(7)
\label{ex:16-7}
es möchte\textsubscript{1} jemand den Hund [füttern dürfen t$_1$]
\ex%(8)
\label{ex:16-8}
\begin{xlist}
\ex%(8) (a)
\label{ex:16-8a}
{[füttern]}\textsubscript{2} möchte\textsubscript{1} jemand den Hund [t$_2$ dürfen t$_1$]
\ex%(8) (b)
\label{ex:16-8b}
{[füttern dürfen]}\textsubscript{2} möchte\textsubscript{1} jemand den Hund [t$_2$ t$_1$]
\end{xlist}
\ex%(9)
\label{ex:16-9}
\begin{xlist}
\ex%(9) (a)
\label{ex:16-9a}
{[den Hund füttern]}\textsubscript{2} möchte\textsubscript{1} jemand [t$_2$ dürfen t$_1$]
\ex%(9) (b)
\label{ex:16-9b}
{[den Hund füttern dürfen]}\textsubscript{2} möchte\textsubscript{1} jemand [t$_2$ t$_1$]
\end{xlist}
\end{exe}
 {\randnum}Dieses Problem besteht in \isi{HPSG} nur dann, wenn
 man es durch zusätzliche Stipulationen herbeiführt.

\randnum Wenn man mit Spuren der in Absatz~\ref{rn:16-10} erklärten Art arbeitet, gilt in \isi{HPSG} aus empirischen (nicht aus
formalen) Gründen ein Grundsatz der Nicht"=Zyklizität des \textsc{slash}"=Werts
(NZ1); daraus folgt (NZ2) als Spezialfall:
{\setlist[enumerate]{leftmargin=*}
\begin{enumerate}[widest=(NZ2)]
\item[(NZ1)] Ein Pfad von Attributen \textsc{slash}|$\pi$ darf nicht sich selbst als Wert enthalten.
\item[(NZ2)] Für jeden (Teil-)Pfad $\pi_{1}$ eines Pfads $\pi$|\textsc{slash} ($\pi$ = $\pi_{1}$|$\pi_{2}$) gilt, daß der Wert von $\pi_{1}$ nicht identisch ist mit dem Wert von $\pi$|\textsc{slash}.
\end{enumerate}
}
{\randnum}Die Identifizierung des \textsc{slash}"=Werts mit dem
\textsc{local}"=Wert wie in (\ref{rn:16-11}) genügt dieser Bedingung. Dementsprechend
können Spur und Spurbinder in allen Eigenschaften voneinander
abweichen, die \textsq{links davon}, \dash im Wert eines $\pi_{1}$ lokalisiert sind.

\randnum In den Werten der Präfixe $\pi_{1}$ (von $\pi$|\textsc{slash})
ist nicht nur das \textsc{phon}"=Attribut mit seinem Wert untergebracht, sondern
\zb auch die Klassifikation als \textit{word} oder \textit{phrase} (X\textsuperscript{0} bzw.\ X\textsuperscript{1}). Daher
können die Spuren t$_2$ in (\ref{ex:16-8b}) und in (\ref{ex:16-9}) ohne weiteres vom Typ V\textsuperscript{0} sein
(wie es bei \citealt{Kiss1993} der Fall wäre), obwohl der Spurbinder vom Typ V\textsuperscript{1}
ist. Auch wenn ein selegierter Verbalkomplex durch irgendeine andere
Eigenschaft ausgezeichnet ist~-- etwa durch "`[\textsc{npcomp} \textit{minus}]"' wie bei
\citet{HinrichsNakazawa1993}~-- gibt es kein Problem, solange diese Eigenschaft im
Wert eines $\pi_{1}$ lokalisiert und zugleich (NZ2) eingehalten ist.

\nocite{GibsonHickok1993,Gorrell1993,Pickering1991}

\printbibliography[heading=subbibliography,notkeyword=this]
\refstepcounter{mylastpagecount}\label{chap-spuren-hpsg-end}
\end{document}

