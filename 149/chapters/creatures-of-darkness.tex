%% -*- coding:utf-8 -*-
\documentclass[output=paper]{LSP/langsci}
\author{Tilman N. Höhle}

\title{Featuring creatures of darkness}
%\epigram{Change epigram in chapters/01.tex or remove it there}
\abstract{}
\maketitle
% \rohead{\thechapter\hspace{0.5em}short title} % Display short title
\ChapterDOI{10.5281/zenodo.1169693}
\begin{document}
\label{chap-creatures}
\selectlanguage{english}
\setcounter{randcount}{0}

\renewcommand*{\thefootnote}{\fnsymbol{footnote}}
\setcounter{footnote}{4}
\footnotetext{%
\emph{Editors’ note:} This is the previously unpublished paper version of a talk given at
the IBM Wissenschaftszentrum in Heidelberg, September 7, 1994. It
unifies two textual variants (of the same date) that show some minor
differences. The citation style was adjusted to the conventions
in this volume, and a few abbreviations in the original are systematically
spelled out here (\emph{Scandinavian} for \emph{Scand.}, etc.).%
}
\setcounter{footnote}{1}
\footnotetext{%
	Note: The empirical aspects of the topic are also
  displayed in a contribution to the Heinz Vater Festschrift (Sprache
  im Fokus, ed.\ by C.\ Dürscheid et al., Tübingen: Niemeyer 1997).}

\renewcommand*{\thefootnote}{\arabic{footnote}}
\setcounter{footnote}{0}

\begin{quotation}
``Pollard and Sag (\citeyear[Chapter~5]{PollardSag1994}), pandering shamelessly to the GB masses, propose an analysis of English relative clauses that employs [\ldots] empty relativizers [\ldots]. [\ldots] The lexical entries for these creatures of darkness stipulate numerous ad hoc structure sharings. [\ldots] \\  
Their elimination would be a welcome result.'' \hfill Anonymous, spring 1994. \\ (\textit{pander to the N} `dem (verwerf"|lichen) N  Vorschub leisten')
\end{quotation}
\randnum\label{rn:15-1}Proposition:
\begin{enumerate}
\item[(i)] The essential characteristics of the First Null Relativizer(s) (§\ref{rn:15-7}) are empirically welcome.
\item[(ii)] Fronted verbs (fV, i.e. \isi{V1} and V2) show the same characteristics (§§\ref{rn:15-27}ff.).
\item[(iii)] Hence, fV are related in specific ways to the complementizer system. Note in particular §§\ref{rn:15-27} and \ref{rn:15-32}.
\end{enumerate}
Warning.\randnum\label{rn:15-2} This talk is going to appeal to analytic intuitions in a very sketchy manner. Many important details are not mentioned or/""and left to future work.


\section{English FNRs}

\randnum\label{rn:15-3}\pollardsag's FNR,\il{English|(} nonraising (slash-binding) variety (cf.\ \citeyear[216 (24)]{PollardSag1994}) (partial):
\begin{exe}
\ex
\scalebox{.9}{
\begin{avm}
\onems[word]{phon \tpv{elist} \\ ss \[loc \[cat \[head \onems[rltvzr]{mod n$'$ \ldots} \\
      val \[subj {\<\[\avml loc {\@4} \\ nonloc|inher|rel & \sliste{ \@1 }\avmr\]\>} \\ spr \tpv{elist} \\
        comps \<\[loc \[{cat\[head \onems[verb]{vform \tpv{fin}} \\val\[\avml subj & 
                \tpv{elist}\\spr & \tpv{elist}\\comps & \tpv{elist}\avmr\]\\marking
                \tpv{unmarked}\]} \\ content {\@5} \tpv{psoa}\]\\ nonloc|inher|slash \sliste{ \@4 } \]\>\]\]\\content 
            \[ \avmtype{npro} \\ index & {\@1} \tpv{ref} \\ restr & \[\avml ft & {\@5} \\ rt & \tpv{list} \avmr\]\]\]\\
    nonloc|to-bind|slash \sliste{ \@4 }\]\\}
\end{avm}}
\end{exe}

\pagebreak
\randnum\label{rn:15-4}P\&S's FNR, Subject-to-Subject Raising variety (cf.\ \citeyear[218 (28)]{PollardSag1994}) (partial):
\begin{exe}
\ex
%\scalebox{.95}{
%\oneline
{%
\begin{avm}
\onems[word]{phon \tpv{elist} \\ ss {\[loc {\[cat {\[head \onems[rltvzr]{mod n$'$ \ldots} \\
      val {\[subj {\<{\@7} {\[nonloc|inher|rel \sliste{ \@1 } \]}\>} \\ spr \tpv{elist} \\
        comps \<{\[loc {\[{cat{\[head \onems[verb]{vform \tpv{fin}} \\val{\[\avml subj &
                \sliste{ \@7 }\\spr & \tpv{elist}\\comps & \tpv{elist}\avmr\]}\\marking \tpv{unmarked}\]}}\\content {\@5} \tpv{psoa}\]} \]}\>\]}\]}\\content 
            \[\avmtype{npro} \\ index & {\@1} \tpv{ref} \\ restr & {\[ \avml ft & {\@5} \\ rt & \tpv{list} \avmr\]}\] \]}\]}\\}
\end{avm}}
\end{exe}

\randnum\label{rn:15-5} Comment~1. Ingeniously, P\&S use the \isi{SELR} to characterize (28) as being regularly expected on the basis of (24). I drop this assumption for empirical reasons (e.g., (\ref{ex:15-10}) below).

Comment~2. P\&S specify \avmbox{7} as ``\isi{NP}''. I drop that as it is unjustified on both formal and empirical grounds.
\vspace{1em}

\randnum\label{rn:15-6}The nonraising FNR (identical subscripts indicate token-identical referential indices):
\eal%(1)
\settowidth\jamwidth{(P\&S (11d))}
\label{ex:15-1}
\ex%(1a)
\label{ex:15-1a}
(person$_{i}$) [who$_{i}$ e$_{i}$ [\textsubscript{S}~I talked to t$_{i}$]]$_{i}$ \jambox{(P\&S (1a))}
%\exsourcefive{(1a)}{PollardSag1994}
\ex%(1b)
\label{ex:15-1b}
(person$_{i}$) [ [whose$_{i}$ brother]$_{j}$ e$_{i}$ [\textsubscript{S}~Kim likes t$_j$]]$_{i}$ \jambox{(P\&S (2))}
%\exsourcefive{(2)}{PollardSag1994}
\ex%(1c)
\label{ex:15-1c}
(person$_{i}$) [[to whom$_{i}$] e$_{i}$ [\textsubscript{S}~Kim gave a book t]]$_{i}$ \jambox{(P\&S (25))}
%\exsourcefive{(25)}{PollardSag1994}
\ex%(1d)
\label{ex:15-1d}
(minister$_{i}$) [[in the middle of whose$_{i}$ sermon] e$_{i}$ [\textsubscript{S}~the dog barked
    t]]$_{i}$ \jambox{(P\&S (11d))}
%\exsourcefive{(11d)}{PollardSag1994}
\zl

\pagebreak
{\randnum}The SSR FNR:
\eal%(2)
\settowidth\jamwidth{(P\&S (29))}
\label{ex:15-2}
\ex%(2a)
\label{ex:15-2a}
(person$_{i}$) [who$_{i}$ e$_{i}$ [\textsubscript{VP}~gave a book to Kim]]$_{i}$ \jambox{(P\&S (29))}
%\exsourcefive{(29)}{PollardSag1994}
\ex%(2b)
\label{ex:15-2b}
(person$_{i}$) [[whose$_{i}$ sister] e$_{i}$ [\textsubscript{VP}~gave a book to Kim]]$_{i}$ \jambox{(P\&S (3))}
%\exsourcefive{(3)}{PollardSag1994}
\ex%(2c)
\label{ex:15-2c}
(person$_{i}$) [[pictures of whom$_{i}$] e$_{i}$ [\textsubscript{VP}~were on sale]]$_{i}$ \jambox{(P\&S (9))}
%\exsourcefive{(9)}{PollardSag1994}
\ex%(2d)
\label{ex:15-2d}
([\ldots{} parties]$_{i}$) [[to be admitted to one of which$_{i}$] e$_{i}$ [\textsubscript{VP}~was a privilege]]$_{i}$ \jambox{(P\&S (23))}
%\exsourcefive{(23)}{PollardSag1994}
\zl

\noindent
\randnum\label{rn:15-7-2}Essential characteristics.%\randnum
\label{rn:15-7} There is a natural class of \textsqe{complementizer} elements (in particular, relativizers). Some of them
are SS raising and some are (nonraising) slash-binding. Their
relationship is not accounted for by the \isi{SELR}, but by other means
(which are not yet worked out). They all take S or \isi{VP} as the only
element of the \textsc{comps} value. With nonraising elements, there may -- but need not -- be another non-empty \textsc{valence} value (\textsc{subj}, as with P\&S, or perhaps preferably \textsc{spr}). With raising elements, the \textsc{subj} value has one element.\il{English|)}

\section{German non-3rd Null Relativizer}

\randnum\label{rn:15-8}Relative clauses with (functionally) first or second person pronoun
(\textit{ich}, \textit{wir}, \textit{du}, \textit{ihr}, \textit{Sie}) or vocative antecedent and a nonsubject
relative phrase are obligatorily \textsqe{normal}:

\begin{exe}
\ex%(3)
\label{ex:15-3}
\begin{xlist}
\ex%(3a)
\label{ex:15-3a}
\gll lieber Freund, den               [\textsubscript{S}~wir so gerne t besuchen]~!\\
     dear friend    whom.\textsc{masc} \hphantom{[\textsubscript{S}~}we so gladly {} visit\\
\ex%(3b)
\label{ex:15-3b}
\gll
dich, den [\textsubscript{S}~wir so gerne t besuchen] \\
you.\textsc{sg} whom.\textsc{masc} \hphantom{[\textsubscript{S}~}we so gladly {} visit \\ 
\ex%(3c)
\label{ex:15-3c}
\gll
ich, den [\textsubscript{S}~sie so gerne t besucht] \\
I whom.\textsc{masc} \hphantom{[\textsubscript{S}~}she so gladly {} visits \\
\ex%(3d)
\label{ex:15-3d}
\gll
Ihnen, den [\textsubscript{S}~wir so gerne t besuchen] \\
you.\textsc{honor} whom.\textsc{masc} \hphantom{[\textsubscript{S}~}we so gladly {} visit \\
\end{xlist}
\end{exe}
{\randnum}Expectedly, there are also \textsqe{normal} relative clauses with subject relative pronoun:
\begin{exe}
\ex%(4)
\label{ex:15-4}
\begin{xlist}
\ex[?]{%(4a)
\gll
lieber Freund, der [uns immer so gerne besucht hat]~! \\
dear friend who.\textsc{masc} \spacebr{}us always so gladly visited has \\}
\ex[]{%(4b)
\gll
dich, der [uns so gerne besucht] \\
you who.\textsc{masc} \spacebr{}us so gladly visits \\}
\pagebreak
\ex[]{%(4c)
\gll
ich, der [sie so gerne besucht] \\
I who.\textsc{masc} \spacebr{}her/""them so gladly visits \\}
\ex[]{%(4d)
\gll
Ihnen, der [uns so gerne besucht] \\
you.\textsc{honor} who.\textsc{masc} \spacebr{}us so gladly visits \\}
\end{xlist}
\end{exe}
%
\randnum\label{rn:15-9}But there is an unexpected alternative:
\begin{exe}
\ex%(5)
\label{ex:15-5}
\begin{xlist}
\ex%(5a1)
\label{ex:15-5a1}
\gll
lieber Freund, der [\textsubscript{S}~Sie uns immer so gerne besucht haben]~! \\
dear friend who.\textsc{masc} \hphantom{[\textsubscript{S}~}you.\textsc{honor} us always so gladly visited have \\
\exi{a$^\prime$.}%(5a2)
\gll
lieber Freund, der [\textsubscript{S}~du uns immer so gerne besucht hast]~! \\
dear friend who.\textsc{masc} \hphantom{[\textsubscript{S}~}you us always so gladly visited have \\
\ex%(5b)
\label{ex:15-5b}
\gll
dich, der [\textsubscript{S}~du uns so gerne besuchst] \\
you who.\textsc{masc} \hphantom{[\textsubscript{S}~}you us so gladly visit \\
\ex%(5c)
\label{ex:15-5c}
\gll
ich, der [\textsubscript{S}~ich sie so gerne besuche] \\
I who.\textsc{masc} \hphantom{[\textsubscript{S}~}I her/""them so gladly visit \\
\ex%(5d)
\label{ex:15-5d}
\gll
Ihnen, der [\textsubscript{S}~Sie uns so gerne besuchen] \\
you.\textsc{honor} who.\textsc{masc} \hphantom{[\textsubscript{S}~}you.\textsc{honor} us so gladly visit \\
\end{xlist}
\end{exe}
%
\randnum\label{rn:15-10}There is agreement between the antecedent and the subject of the relative clause as to: 
\begin{quotation}
\textit{c-inds} (speaker, hearer, number), honorification;
\end{quotation}
between the antecedent and the relative pronoun (just as in (\ref{ex:15-3}) and (\ref{ex:15-4})) as to: 
\begin{quotation}
natural number, natural sex;
\end{quotation}
between the relative pronoun and the subject as to: 
\begin{quotation}
case (nominative).
\end{quotation}
\randnum\label{rn:15-11}No attempt is made here to develop precise agreement mechanisms and
modify the properties of the sorts \textit{index} and \textit{c-inds} accordingly. But it can be seen that the arrangement of constituents in (\ref{ex:15-5}) complies with the predictions of §\ref{rn:15-7}.

\pagebreak
\randnum\label{rn:15-12}German non-3rd Null Relativizer (roughly):
\begin{exe}
\ex
\scalebox{.8}{
\begin{avm}
\onems[word]{phon \tpv{elist} \\ ss {\[loc {\[cat {\[head \onems[rltvzr]{mod n$'$ \ldots} \\
      val {\[subj & \tpv{elist} \\ spr & {\<{\[nonloc|inher|rel \<{\@1}\>\]}\>} \\
        comps & \<{\[loc {\[{cat{\[head \onems[verb]{vform \tpv{fin}\\ smor {\[content & \tpv{ppro} \\ context & \ldots{} {\@1} \tpv{n3rd} \]}} \\val{\[\avml subj &
                \tpv{elist}\\spr & \tpv{elist}\\comps & \tpv{elist}\avmr\]}\\marking \tpv{unmarked}\]}}\\content {\@5} \tpv{psoa}\]} \]}\>\]}\]}\\content \[ \avmtype{npro} \\ index & \ldots \\ restr & {\[\avml ft & {\@5} \\ rt & \tpv{nelist} \avmr\]}\] \]}\]}\\}
\end{avm}}
\end{exe}

\section{The attribute \textsc{smor}}

\largerpage
\randnum\label{rn:15-13}In §\ref{rn:15-12}, use is made of a new head attribute \textsc{smor}. Its value is
intended to be the \textsc{local} value of the subject, i.e., of that phrase
that agrees with the finite verb as to person and number, if there is
one. In this way, information about properties of a subject contained
in a phrase is available to anything that selects that phrase. This
attribute allows the case of the subject in \textit{for-to} infinitives to be
selected by \textit{for}:

\begin{exe}
\ex%(6)
\label{ex:15-6}
\begin{xlist}
\ex[]{%(6a)
[for [her to do the ugly work]] is pleasant for him}
\ex[*]{%(6b)
[for [she to do the ugly work]] is pleasant for him}
\end{xlist}
\end{exe}
\addlines
\randnum\label{rn:15-14}It provides information for \textsqe{comp-agreement} as in (\ref{ex:15-7}) (Eastern
Dutch\il{Dutch!Eastern}). Notice that the complementizer's suffix is independent of the
verb's suffix.
\eal%(7)
\settowidth\jamwidth{(9)}
\label{ex:15-7}
\ex%(7a)
\label{ex:15-7a}
\gll az\underline{ze} \underline{wy} de törf niet verkoopn kun\underline{t} \\
     that.1\textsc{pl} we the peat not sell can.1\textsc{pl} \\\jambox{(Haeringen \citeyear[119]{vanHaeringen1958})}
%\exsourcefive{119}{vanHaeringen1958}
\ex%(7b)
\label{ex:15-7b}
\gll az(*-\underline{ze}) \underline{ze}/\underline{zy} de törf niet verkoopn kun\underline{t} \\
     that(-3\textsc{pl}) they the peat not sell can.3\textsc{pl} \\
\zl
{\randnum}And it helps accounting for \isi{VP} topicalization with \textsqe{ergative} subjects in German:
\begin{exe}
\ex%(8)
\label{ex:15-8}
\begin{xlist}
\ex%(8a)
\label{ex:15-8a}
\gll
[der Wein ausgegangen] ist uns diesmal nicht \\
\spacebr{}the.\textsc{nom.sg} wine come.to.an.end is for.us this.time not \\
\ex%(8b)
\label{ex:15-8b}
\gll
[die Argumente ausgegangen] sind/*ist uns diesmal nicht \\
\spacebr{}the.\textsc{nom.pl} arguments come.to.an.end are/*is for.us this.time not \\
\end{xlist}
\end{exe}

\section{Scandinavian \textit{som}}
\randnum\label{rn:15-15}In Scandinavian\il{Scandinavian} languages~-- in particular, in
Norwegian\il{Norwegian}~-- \textit{som} functions as an introduction (i)~to
expressions of comparison, (ii)~to relative clauses, (iii)~to
\emph{wh}-interrogative clauses. (\textit{That}-clauses are introduced by \textit{at},
\textit{whether}-clauses by \textit{om}.)


\subsection{Interrogative clauses}

\randnum\label{rn:15-16}There is no way to predict that \textit{som} can occur
with \emph{wh}-interrogative clauses, and it is unexpected that it is
obligatory with subject interrogatives (\ref{ex:15-9}). With object interrogatives
(\ref{ex:15-10}), it is possible but disprefered in Swedish and impossible in
Norwegian\il{Norwegian}.
\eal
\settowidth\jamwidth{(9)}
%(9)
\label{ex:15-9}
\ex[]{%(9a)
\gll vi vet hvem som [snakker med Marit] \\
     we know who som \spacebr{}talks with Mary \\\jambox{\citep[(7)]{Taraldsen1986}}}
%\exsourcefive{(7)}{Taraldsen1986}
\ex[*]{%(9b)
\gll vi vet hvem [snakker med Marit] \\
     we know who \spacebr{}talks with Mary \\\jambox{(8)}
}
%\exsourcefive{(8)}{Taraldsen1986}
\zl
\eal%(10)
\settowidth\jamwidth{(9)}
\label{ex:15-10}
\ex[\%]{%(10a)
\gll
vi vet hvem som [\textsubscript{S}~Marit snakker med t] \\
we know who som \hphantom{[\textsubscript{S}~}Mary talks with \\\jambox{(10)}}
%\exsourcefive{(10)}{Taraldsen1986}
\ex[]{%(10b)
\gll
vi vet hvem [\textsubscript{S}~Marit snakker med t] \\
we know who \hphantom{[\textsubscript{S}~}Mary talks with \\\jambox{(9)}}
%\exsourcefive{(9)}{Taraldsen1986}
\zl
\randnum\label{rn:15-17}A natural account is to postulate for Norwegian\il{Norwegian} and Swedish\il{Swedish} (i) \textit{om} as an interrogativizer that is neither slash"=binding nor raising, (ii) \textit{som}
as a raising irogvzr, (iii) a null slash"=binding irogvzr; and for
Swedish (iv) another \textit{som} irogvzr that is slash"=binding. Thus, the Swedish \textit{som}
irogvzrs overtly show just the essential properties of the English\il{English} FNRs.

\pagebreak
\randnum\label{rn:15-18}Scandinavian\il{Scandinavian} \textit{om} interrogativizer (partial):
\begin{exe}
\ex
\oneline{%
\begin{avm}
\onems[word]{phon \sliste{ \textnormal{om} } \\ 
             ss {\[loc {\[cat  {\[head \tpv{irogvzr} \\
                                   val {\[subj & \tpv{elist} \\ 
                                     spr & \tpv{elist} \\
                                     comps & \<{\[loc {\[{cat{\[head \onems[verb]{vform \tpv{fin}} \\
                                                                val {\[\avml subj & \tpv{elist}\\
                                                                             spr & \tpv{elist}\\
                                                                           comps &
                                                                           \tpv{elist}\avmr\]}\\
                                                                marking \tpv{unmarked}\]}}\\
                                         content \tpv{psoa}\]} \]}\>\]}\]}\\
                           content \tpv{psoa} \]}\]}\\}
\end{avm}
}
\end{exe}

\randnum\label{rn:15-19}Scandinavian\il{Scandinavian} \textit{som} raising irogvzr (partial):
\begin{exe}
\ex
%\scalebox{.85}{
\oneline{%
\begin{avm}
\onems[word]{phon \sliste{ \textnormal{som} } \\ ss {\[loc {\[cat {\[head \tpv{irogvzr} \\
      val {\[subj & \<{\@7} \[nloc|inher|que \sliste{ \tpv{npro} }\]\> \\ spr & \tpv{elist} \\ 
        comps & \<{\[loc {\[{cat {\[head \onems[verb]{vform \tpv{fin}} \\val {\[\avml subj &
                {\sliste{ \@7 }}\\spr & \tpv{elist}\\comps & \tpv{elist}\avmr\]}\\marking \tpv{unmarked}\]}}\\content \tpv{psoa}\]} \]}\>\]}\]}\\content \tpv{psoa} \]}\]}\\}
\end{avm}}
\end{exe}
\addlines[2]
{\randnum}There is a question as to how the propagation of the nonempty \textsc{que}
value is to be regulated. Since this involves general questions
concerning the function and location of \textsc{to-bind}, these questions are
not discussed here.

\pagebreak
\randnum\label{rn:15-20}Swedish \textit{som} slash-binding irogvzr (partial):
\begin{exe}
\ex
\label{ex:15-99}
\oneline{%
\begin{avm}
\onems[word]{phon \sliste{ \textnormal{som} } \\ ss {\[loc {\[cat & {\[head \tpv{irogvzr} \\
      val {\[subj & \tpv{elist} \\ spr & {\<\[loc {\@4} \\ nonloc|inher|que \sliste{ \tpv{npro} }\]\>} \\
        comps & \<{\[loc {\[{cat{\[head \onems[verb]{vform \tpv{fin}} \\val {\[\avml subj &
                \tpv{elist} \\ spr & \tpv{elist}\\comps & \tpv{elist}\avmr\]}\\marking
                \tpv{unmarked}\]}}\\content \tpv{psoa}\]} \\ nonloc|inher|slash
          \sliste{ \@4 } \]}\>\]}\]}\\content & \tpv{psoa} \]} \\ nonloc|to-bind|slash \sliste{ \@4 } \]}}
\end{avm}}
\end{exe}
%\largerpage

\randnum\label{rn:15-21}Scandinavian\il{Scandinavian} null slash-binding irogvzr (partial) (just like §\ref{rn:15-20}):
\begin{exe}
\ex
\oneline{%
\begin{avm}
\onems[word]{phon \tpv{elist} \\ ss {\[loc {\[cat & {\[head \tpv{irogvzr} \\
      val {\[subj & \tpv{elist} \\ spr & {\<\[loc {\@4} \\ nonloc|inher|que \sliste{ \tpv{npro} }\]\>} \\
        comps & \<{\[loc {\[{cat{\[head \onems[verb]{vform \tpv{fin}} \\val {\[\avml subj &
                \tpv{elist} \\ spr & \tpv{elist}\\comps & \tpv{elist}\avmr\]}\\marking
                \tpv{unmarked}\]}}\\content \tpv{psoa}\]} \\ nonloc|inher|slash \sliste{ \@4
          } \]}\>\]}\]}\\content & \tpv{psoa} \]} \\ nonloc|to-bind|slash \sliste{ \@4 } \]}}
\end{avm}}
\end{exe}

\subsection{Relative clauses}

\randnum\label{rn:15-22}Relative clause introducing \textit{som} is distributed just like the
interrogativizer \textit{som} is in Swedish, except that there is no (overt)
relative phrase. Hence, there are two \textit{som} rltvzrs and one null rltvzr
that are all similar to the Second Null Relativizer of \citet[222 (36)]{PollardSag1994}. Note that \textit{som} in Scandinavian\il{Scandinavian}, as opposed to \textit{that} in English\il{English},
does not occur as an \textsqe{unmarked} complementizer, hence \textit{som} in
relative clauses cannot be treated like P\&S attempt to treat \textit{that} in their §5.2.3.


\eal%(11)
\settowidth\jamwidth{(9)}
\label{ex:15-11}
\ex[]{%(11a)
\label{ex:15-11a}
\gll
vi kjenner den mannen som [snakker med Marit] \\
we know the man som \spacebr{}talks with Mary \\\jambox{(\citeauthor[(1)]{Taraldsen1986})}}
%\exsourcefive{(1)}{Taraldsen1986}
\ex[*]{%(11b)
\label{ex:15-11b}
\gll
vi kjenner den mannen [snakker med Marit] \\
we know the man \spacebr{}talks with Mary \\\jambox{(2)}}
%\exsourcefive{(2)}{Taraldsen1986}
\zl
\largerpage[2]
\eal%(12)
\settowidth\jamwidth{(9)}
\label{ex:15-12}
\ex%(12a)
\label{ex:15-12a}
\gll
vi kjenner den mannen som [\textsubscript{S}~Marit snakker med t] \\
we know the man som \hphantom{[\textsubscript{S}~}Mary talks with \\\jambox{(3)}
%\exsourcefive{(3)}{Taraldsen1986}
\ex%(12b)
\label{ex:15-12b}
\gll
vi kjenner den mannen [\textsubscript{S}~Marit snakker med t] \\
we know the man \hphantom{[\textsubscript{S}~}Mary talks with \\\jambox{(4)}
%\exsourcefive{(4)}{Taraldsen1986}
\zl

\pagebreak
\randnum\label{rn:15-23}Scandinavian\il{Scandinavian} slash-binding \textit{som} relativizer:
\begin{exe}
\ex
%\scalebox{.68}
\oneline{\begin{avm}
\onems[word]{phon \sliste{ \textnormal{som} } \\ ss \[loc \[cat \[head \onems[rltvzr]{mod n$'$ \ldots} \\
      val \[subj & \tpv{elist} \\ spr & \tpv{elist} \\
        comps & \<\[loc \[{cat\[head \onems[verb]{vform \tpv{fin}} \\val\[\avml subj &
                \tpv{elist}\\spr & \tpv{elist}\\comps & \tpv{elist}\avmr\]\\marking
                \tpv{unmarked}\]} \\ content {\@5} \tpv{psoa} \]\\ nonloc|inher|slash \sliste{ \@4 } \]\>\]\]\\content 
            	\[\avmtype{npro} \\ index & {\@1} \tpv{ref} \\ restr & \[\avml ft & {\@5} \\ rt & \tpv{nelist} \avmr\]\]
            	 \]\\
    nonloc|to-bind|slash \<{\@4} \[{cat\[head \tpv{noun} \\val\[\avml subj &
                \tpv{elist}\\spr & \tpv{elist}\\comps & \tpv{elist}\avmr\]\\marking \tpv{unmarked}\]} \\ content {\[index {\@1} \\ restr \tpv{elist}\]} \]  \>\]}
\end{avm}}
\end{exe}

\pagebreak
\randnum\label{rn:15-24}Scandinavian\il{Scandinavian} raising \textit{som} relativizer:
\begin{exe}
\ex
%\scalebox{.75}
\oneline{
\begin{avm}
\onems[word]{phon \sliste{ \textnormal{som} } \\ ss|{loc \[cat \[head \onems[rltvzr]{mod n$'$ \ldots} \\
      val \[subj & \tpv{elist} \\ spr & \tpv{elist} \\
        comps & \<\[loc \[{cat\[head \onems[verb]{vform \tpv{fin}} \\
                                val\[\avml subj {\<\[loc|content|index {\@1}\]\>}\\
                                           spr \tpv{elist}\\
                                           comps \tpv{elist}\avmr\]\\
                                marking \tpv{unmarked}\]} \\ content {\@5} \tpv{psoa} \]\]\>\]\]\\content 
            	\[\avmtype{npro} \\ index & {\@1} \tpv{ref} \\ restr & \[\avml ft & {\@5} \\ rt & \tpv{nelist} \avmr\]\]
            	\]}}
\end{avm}}
\end{exe}

\pagebreak
\randnum\label{rn:15-25}Scandinavian\il{Scandinavian} slash-binding null relativizer (just like §\ref{rn:15-23}):
\begin{exe}
\ex
%\oneline{%
\scalebox{.9}{%
\begin{avm}
\onems[word]{phon \tpv{elist} \\ ss \[loc \[cat \[head \onems[rltvzr]{mod n$'$ \ldots} \\
      val \[subj & \tpv{elist} \\ spr & \tpv{elist} \\
        comps & \<\[loc \[{cat\[head \onems[verb]{vform \tpv{fin}} \\val\[\avml subj &
                \tpv{elist}\\spr & \tpv{elist}\\comps & \tpv{elist}\avmr\]\\marking \tpv{unmarked}\]} \\ content {\@5} \tpv{psoa} \]\\ nonloc|inher|slash \<{\@4}\> \]\>\]\]\\content 
            	\[\avmtype{npro} \\ index & {\@1} \tpv{ref} \\ restr & \[\avml ft & {\@5} \\ rt & \tpv{nelist} \avmr\]\]
            	\]\\
    nonloc|to-bind|slash \<{\@4} \[{cat\[head \tpv{noun} \\val\[\avml subj &
                \tpv{elist}\\spr & \tpv{elist}\\comps & \tpv{elist}\avmr\]\\marking \tpv{unmarked}\]} \\ content {\[index & {\@1} \\ restr & \tpv{elist}\]} \]  \>\]}
\end{avm}}
\end{exe}

%\addlines
%\largerpage[2]
\section{English \textit{that}-relativizers}

\randnum\label{rn:15-26}English\il{English|(} non-\emph{wh}-relative clauses pattern exactly like the Scandinavian\il{Scandinavian}
ones. Cf.\  (\ref{ex:15-13}) and (\ref{ex:15-14}) to (\ref{ex:15-11}) and (\ref{ex:15-12}). Hence, I drop P\&S's
assumption that there is a nominative relative pronoun \textit{that} (\citeyear[220 (33)]{PollardSag1994}). Instead, there are two \textit{that}-relativizers and one null
relativizer corresponding exactly to the Scandinavian\il{Scandinavian} ones
(§§\ref{rn:15-23}--\ref{rn:15-25}).

\eal%(13)
\settowidth\jamwidth{(9)}
\label{ex:15-13}
\ex[]{%(13a)
  (student) that [was telling you about cell structure] \jambox{(P\&S: (32c))}}
%\exsourcefive{(32c)}{PollardSag1994}}
\ex[*]{%(13b)
  (student) [was telling you about cell structure] \jambox{(cf. P\&S: (38))}}
%\exsourcefive[cf.]{(38)}{PollardSag1994}}
\zl
\eal%(14)
\settowidth\jamwidth{(9)}
\label{ex:15-14}
\ex[]{%(14a)
\label{ex:15-14a}
(student) that [\textsubscript{S}~I was telling you about t]  \jambox{(P\&S: (32a))}}
%\exsourcefive{(32a)}{PollardSag1994}
\ex[]{%(14b)
\label{ex:15-14b}
(student) [\textsubscript{S}~I was telling you about t]  \jambox{(P\&S: (32b))}}
%\exsourcefive{(32b)}{PollardSag1994}
\zl\il{English|)}

\section{fV with and without \textsqe{subject inversion}}

\randnum\label{rn:15-27}Fronted finite verbs (fV, i.e. \isi{V1} and V2) in most Germanic languages
show exactly the same inflexional morphophonemics as their non"=fronted
counterparts (uV). (One celebrated exception in Modern English\il{English} is
Hudson's \citeyear{Hudson1977} \textit{aren't I}.) Therefore, many attempts to understand \textsqe{verb movement} proceed from the assumption that fronted and unfronted verbs are in an important sense \textsqe{the same} (and tend to get stuck someplace). This assumption is partially correct for fV (V2) that follow their subject (S-fV); it is incorrect for \textsqe{inverted} fV that precede their subject (\mbox{fV-S}). At the same time, (projections of) S-fV and \mbox{fV-S} share well-known properties that set them apart from (projections of) uV.

\randnum\label{rn:15-28}In Old English\il{Old English} and in Middle Low German\il{Middle Low German}, a 1\textsc{pl} or 2\textsc{pl} \mbox{fV-S} can or must
bear special inflexional properties different from the inflexion of
S-fV and uV (\citealt{Brunner1965}, \citealt{Lasch1974}, \citealt{Sarauw1924}). Probably, the same pattern
underlies the variation in Tatian (Old High German\il{Old High German}) of 1\textsc{pl} \textit{-mes} and
\textit{-n}, although the figures are too small to be conclusive
\citep{Eggenberger1961}. In Middle High German\il{Middle High German}, 1\textsc{pl} \textit{-e} and \textit{-(e)n} alternate accordingly \citep{Paul1989E}.

\randnum\label{rn:15-29}Standard Dutch\il{Dutch} is well-known for its obligatory 2\textsc{sg} alternation:
\begin{exe}
\ex%(15)
\label{ex:15-15}
\begin{xlist}
\ex%(15a)
\label{ex:15-15a}
\gll
dat jij misschien ziek bent/*ben \\
that you perhaps sick are \\
\ex%(15b)
\label{ex:15-15b}
\gll
jij bent/*ben misschien ziek \\
you are perhaps sick \\
\ex%(15c)
\label{ex:15-15c}
\gll
(misschien) ben(*-t) jij ziek \\
\hphantom{(}perhaps are you sick \\
\end{xlist}
\end{exe}
\addlines[2]
\randnum\label{rn:15-30}Since the \mbox{fV-S} here is characterized by the
absence of \textit{-t}, some unusual kind of deletion rule might be
invoked. Nothing like that is possible in dialectal data like (\ref{ex:15-16})
from Eastern Dutch\il{Dutch}:
%\addlines[2]
\eal%(16)
\settowidth\jamwidth{(9)}
\label{ex:15-16}
\ex%(16a)
\label{ex:15-16a}
\gll
dat wii kiikt \\
that we look \\\jambox{\citep[318]{Entjes1970}}
%\exsourcefive{318}{Entjes1970}
\ex%(16b)
\label{ex:15-16b}
wii kiikt
\ex%(16c)
\label{ex:15-16c}
(XP) kiiken wii
\end{xlist}
\end{exe}
There is a broad spectrum of alternations throughout different dialects of Dutch\il{Dutch}.

%\addlines
\randnum\label{rn:15-32}It is apparent that S-fV and \mbox{fV-S} correspond to raising and nonraising complementizers such as seen in §\ref{rn:15-19} and §§\ref{rn:15-18}, \ref{rn:15-20}, respectively.

\randnum\label{rn:15-33}Raising S-fV \textit{kiikt} (partial); cf.\  §\ref{rn:15-19}:
\begin{exe}
\ex
\scalebox{.72}{%
%\oneline{
\begin{avm}
\onems[word]{phon {\<\textnormal{kiikt}\>} \\ ss \[loc \[cat & \[head & \ldots \\
      val & \[subj & {\< {\@2} \[loc \[\avml cat|head & \tpv{noun} \\ content|index & {\[pers & \tpv{1} \\ number & \tpv{pl} \]} \avmr\]\]\>} \\
      spr & \tpv{elist} \\
        comps & \<\[loc \[{cat\[head & {\@3} \onems[verb]{vform \tpv{fin}} \\val & \[\avml subj & 
                {\<{\@2}\>}\\spr & \tpv{elist}\\comps & \tpv{elist}\avmr\]\\marking & \tpv{unmarked}\]}\\ content {\@5} \tpv{psoa} \]\\ nonloc|inher|slash \<{\@4}\> \]\>\]\]\\content & {\@5} \]\\
    nonloc|to-bind|slash \<{\@4} {\[cat|head & {\@3} \\ content & \tpv{psoa}\]} \>\]\\}
\end{avm}}
\end{exe}

\randnum\label{rn:15-34}Nonraising \mbox{fV-S} (\isi{V1}) \textit{kiiken} (partial); cf.\ §§\ref{rn:15-12} and \ref{rn:15-18}:
\begin{exe}
\ex
%\scalebox{.7}
\oneline{
\begin{avm}
\onems[word]{phon {\<\textnormal{kiiken}\>} \\ ss \[loc \[cat & \[head & \ldots \\
      val & \[subj & \tpv{elist} \\ spr & \tpv{elist} \\
        comps & \<\[loc \[{cat  \[head & {\@3} \onems[verb]{\avml vform \tpv{fin} \\
        smor|content|index & {\[pers & \tpv{1}\\ numb & \tpv{pl} \]} \avmr} \\val & \[\avml subj &
                \tpv{elist}\\spr & \tpv{elist}\\comps & \tpv{elist}\avmr\]\\marking & \tpv{unmarked}\]}\\ content  {\@5} \tpv{psoa} \]\\ nonloc|inher|slash \<{\@4}\> \]\>\]\]\\content & {\@5} \]\\
    nonloc|to-bind|slash \<{\@4} {\[cat|head & {\@3} \\ content & \tpv{psoa}\]} \>\]\\}
\end{avm}}
\end{exe}
\randnum\label{rn:15-35}Naturally, an \mbox{fV-S} gets its agreement
information from the \textsc{smor} value of its complement. Significantly,
\citet{vanHaeringen1958} has observed that in those dialects that show inflexion on
complementizers as in (\ref{ex:15-7}), the complementizer inflexion is modelled on
the inflexion of \mbox{fV-S}, not S-fV/""uV. (\citealt[Chapter~III.3]{Zwart1993} provides
\textsqe{minimalist} discussion.)

\randnum\label{rn:15-36}The accounts for \textsqe{raising} complementizers and S-fV presented here entail that the complement's subject cannot
undergo \textsqe{short movement}. This restriction applies in particular to
Norwegian\il{Norwegian} and Dutch\il{Dutch}, two languages that are known to be very liberal
wrt.\ \textsqe{Comp-trace} effects. One may interpret this as a contribution to
the growing evidence that those effects are far from being understood.

\section{Non"=finite fV}

\randnum\label{rn:15-37}In several (geographically scattered) dialects
of Low German\il{Low German}, Dutch\il{Dutch} and Old Frisian\il{Old Frisian} a version of asymmetric
coordination of non"=finite verb projections can be found where the
non"=finite verb in a non-first conjunct is fronted:
\eal%(17)
\settowidth\jamwidth{(9)}
\label{ex:15-17}
\ex%(17a)
\label{ex:15-17a}
\gll
dann wollte ich mir eine Stube mieten und [\underline{verheiraten} mich] \\ 
then wished I for.me a flat rent and \spacebr{}get.married.\textsc{inf} myself \\\jambox{\citep[76]{Teuchert1921}} 
%\exsourcefive{76}{Teuchert1921}
\ex%(17b)
\label{ex:15-17b}
\gll
moust es komen en [\underline{zain} ons vooltje] \\
must.you once come and \spacebr{}see our foal \\\jambox{\citep[(101c)]{Veldman1991}}
%\exsourcefive{(101c)}{Veldman1991}
\ex%(17c)
\label{ex:15-17c}
\gll
ik zoo noo hous goon en [\underline{nemme(n)} ze mei] \\
I should to home go and \spacebr{}take.\textsc{inf} them with \\\jambox{\citep[465]{deBont1962}}
%\exsourcefive{465}{deBont1962}
\zl
\addlines
\randnum\label{rn:15-38}In Modern Frisian, the non"=finite verb, when fronted, takes the form of the imperative:
\eal%(18)
\settowidth\jamwidth{(9)}
\label{ex:15-18}
\ex%(18a)
\label{ex:15-18a}
\gll
de plysje soe bij him komme en [syn papieren mei \underline{nimme}/*nim] \\
the police will to him come and \spacebr{}his papers with take \\\jambox{\citep[(8)]{deHaan1990}}
%\exsourcefive{(8)}{deHaan1990}
\ex%(18b)
\label{ex:15-18b}
\gll
de plysje soe bij him komme en [\underline{nim}/*nimme syn papieren mei] \\
the police will to him come and \spacebr{}take.\textsc{ipv} his papers with \\\jambox{\citep[(9)]{deHaan1990}}
%\exsourcefive{(9)}{deHaan1990}
\zl
\randnum\label{rn:15-39}Evidently, being fronted has been sufficient
for non"=finite verbs to assimilate to a form (imperative) with a
totally different meaning. This seems to indicate that \textsqe{verb movement} involves something much deeper than mere temporal precedence.


\nocite{Hudson1977,Taraldsen1986,Sarauw1924}

\sloppy
\printbibliography[heading=subbibliography,notkeyword=this]
\refstepcounter{mylastpagecount}\label{chap-creatures-end}
\end{document}
