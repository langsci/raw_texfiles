\documentclass[output=paper]{langsci/langscibook}

\author{Tilman N. Höhle}
\title{Topologische Felder}
\abstract{}
\ChapterDOI{10.5281/zenodo.1169661}
\begin{document}
% \rohead{\thechapter\hspace{0.5em}short title} % Display short title
%\chapter{Topologische {F}elder}
\selectlanguage{german}

\setcounter{tocdepth}{3}
\localtableofcontents

\newpage

\section*{Anmerkung der Herausgeber}
\addcontentsline{toc}{section}{\protect\numberline{}Anmerkung der Herausgeber}
\label{fn-herausgeber-topo}

Das hier erstmals publizierte Manuskript "`Topologische Felder"', entstanden 1983 in Köln, ist ein
Fragment; zur im Text angekündigten Fortsetzung und zur Ergänzung der durch "`x"' am Rande
markierten Leerverweise (vgl.\ die Autoren"=Fn.\,\ref{fn-autor-topo}) ist es nie gekommen. Trotzdem
fand diese Arbeit als \textsq{graues Papier} -- in 2 textidentischen, nur in Layout und Paginierung
abweichenden Fassungen (von 1983 und 2003) -- weite Verbreitung.

Die hier vorgelegte Druckfassung ist textuell unverändert, jedoch dem für diesen Band gewählten
Druckformat angepasst. Insbesondere sind die Beispiele jetzt durchgezählt und die
o.\,a.\ Leerverweise als "`X"' in den Text integriert; unspezifische Vorverweise wurden zusätzlich
durch "`[X]"' markiert. Es folgen einige Hinweise zu deren
Entschlüsselung, sowie zum Inhalt der geplanten Fortsetzung.

Die vorhandenen Kapitel~1--8 und darin enthaltene Vorverweise lassen erkennen, dass die weiteren
Abschnitte mindestens zweierlei umfassen sollten: (i) unmittelbar anschließend die Besprechung der
weiteren topologischen Satztypen~-- F1"=Sätze und insbesondere E"=Sätze (verbunden damit auch
Einbettung) bzw.\ der weiteren topologischen Felder (‚\isi{Nachfeld}’ und insbesondere ‚Mittelfeld’); (ii)~in Ausweitung von Abschnitt~\ref{sec-satzanknuepfung}, die genaue Besprechung koordinativer Strukturen.

Über den mutmaßlichen Inhalt dieser Abschnitte geben separate Höhle"=Arbeiten (alle wieder abgedruckt in diesem Band) zumindest partiellen Aufschluss: Zu (i)
vgl.\ \citet{Hoehle86}, dort auch Skizze der Geschichte der Felderterminologie; zu (ii) vgl.\ die
Trias \citet{Hoehle1983}, \citet{Hoehle90a} und vor allem \citet{Hoehle91}. -- Ob (iii)
Wortstellungsregeln für nicht"=verbale Konstituenten nochmals eigens aufgegriffen werden sollten,
ist unklar; jedenfalls waren schon bei Entstehung der vorliegenden Arbeit Höhles Ansichten hierzu
detailliert niedergelegt, vgl.\ \citet{Hoehle1982} (siehe auch Autoren"=Fn.\,\ref{fn-autor-topo}). 

Auf diesem Hintergrund lassen sich die mit X bzw.\ [X] markierten Leer- und Vorverweise grob wie folgt aufschlüsseln:
\begin{itemize}
		\item  S.\,\pageref{fn:1-12}/""Fn.\,\ref{fn:1-12}, \pageref{X:1}, \pageref{X:2},
                  \pageref{X:10}, \pageref{X:3}, \pageref{X:4}, \pageref{ex:1-8.1-20b},
                  \pageref{fn:1-39}/""Fn.\,\ref{fn:1-39}: Bezug auf (i);
		\item S.\,\pageref{ex:1-8.2-3b}: Bezug auf (ii);
		\item S.\,\pageref{ex:1-1-5} [3x], \pageref{fn:1-15}, \pageref{X:5}, \pageref{X:6}, \pageref{X:7}: Bezug auf (iii);
		\item S.\,\pageref{X:8}: Bezug auf (iii), aber auch auf Abschnitt~\ref{sec:1-5} und eventuell (i);
		\item S.\,\pageref{X:9}: Bezug auf \citet{Hoehle78a}.
\end{itemize}
\printbibliography[heading=subbibliography]% print section bibliography

\newrefsection


\maketitle
\label{chap-topo-hoehles-content}
\label{chap:1}\label{chap-topo}

\renewcommand*{\thefootnote}{\fnsymbol{footnote}}
\setcounter{footnote}{0}


\section[Starke und schwache Regeln]{Starke und schwache Regeln\protect\footnote{\label{fn-autor-topo}Dies ist eine vorläufige Fassung der einleitenden Abschnitte eines
		längeren Manuskripts mit gleichem Titel. Einige Verweise auf spätere
		Abschnitte (am Rand durch "`×"' markiert) sind noch nicht
		ausgeführt. Für einige Verweise über stilistisch (nicht-) normale
		Wortstellung vgl.\ vorerst Höhle, Explikationen für "`normale Betonung"'
		und "`normale Wortstellung"', in: \textit{Satzglieder im Deutschen}.~– Tübingen:
		Narr 1982. [s.o.\ Anm.\ S.\,\pageref{fn-herausgeber-topo}]} }\label{sec:1-1}


\renewcommand*{\thefootnote}{\arabic{footnote}}
\setcounter{footnote}{0}

%page 2
\ssubsection{}%1.1.
\label{subsec:1-1.1}

Unter der Rubrik "`Order as a Morphemic Element"' unterscheidet \citet[184--186]{Harris1951} drei Typen von topologischen Beziehungen zwischen Morphem\-(komplex)en:
\begin{exe}
\ex\label{ex:1-1-1}
\begin{xlist}
\ex\label{ex:1-1-1a} contrasting (oder: morphemic) order:\\
where there is a contrast between two arrangements of morphemic elements
\ex\label{ex:1-1-1b} restricted (oder: automatic) order: \\
there is no contrast between two arrangements of a given set of morphemic
segments, but only one of these arrangements occurs
\ex\label{ex:1-1-1c} descriptively equivalent order (oder: not ordered): \\
where the order of morphemic segments in an utterance is free; i.e. the morphemes occur in any order, with no attendant difference in the larger contextual environment or in the social situation.
\end{xlist}
\end{exe}
In Sprachen wie dem Deutschen\il{Deutsch} ist es oft unklar, wie die Unterscheidung zwischen diesen Typen anzuwenden ist. Das hängt u.\,a.\ mit der Frage zusammen, was unter "`contrast"' zu verstehen ist. Harris meint damit: "`differences in form that correlate regularly with differences in environment and meaning"' (S.\,184) und "`differences
in contextual environment and in social situation"' (S.\,186). Aber wie weit oder eng der Begriff \textsq{social situation} zu fassen ist, ist nicht klar.

\ssubsection{}%1.2.
\label{subsec:1-1.2}
\citet{Danes1967} hat eine differenziertere Taxonomie vorgeschlagen. Er unterscheidet vier Typen von topologischen Beziehungen:
\begin{exe}
\ex\label{ex:1-1-2}
\begin{xlist}
\ex\label{ex:1-1-2a} grammaticalized order:

"`In cases where the opposition between two syntactic categories is implemented (realized) by two different positions of the element in the sentence
pattern (the order being thus a distinctive feature), the corresponding rules
may be called \textsqe{functional rules} and the order of elements may be termed
\textsqe{grammaticalized}."' \citep[500f.]{Danes1967}

\ex\label{ex:1-1-2b} fixed order:

"`in some instances the position of an element is \textsqe{fixed}, and yet the violation
of the rule fixing its position in the sentence does not lead to a different sentence (with other grammatical relations between the elements); the result
will only be an \textsqe{ungrammatical} or \textsqe{less grammatical} form of the original
%page 3
sentence. The position of the elements in the sentence is then only a concomitant (\textsqe{redundant}, not distinctive) feature of their syntactic function."'
\citep[501]{Danes1967}

\ex\label{ex:1-1-2c} usual vs.\ marked order:

"`In the third case, a certain order of elements is \textsqe{usual}; any deviation from
this order, permitted by the \textsqe{weak} rule and motivated by special non"=grammatical conditions, is associated with the feature of \textsqe{non"=neutrality} or \textsqe{markedness}."' \citep[501]{Danes1967}

\ex\label{ex:1-1-2d} labile order:

"`In languages with the so"=called \textsqe{free} word order, we must consider a
fourth possibility, i.e., a \textsqe{labile} order. In this case, the order of some elements of the pattern on the grammatical level is irrelevant; in utterances based
on such a pattern, the position of the respective words vacillates according
to non"=grammatical conditions."' \citep[501]{Danes1967}
\end{xlist}
\end{exe}
Die Regeln für grammaticalized und für fixed order~-- also functional
rules und concomitant rules~-- faßt er als \textsq{strong rules} zusammen;
davon zu unterscheiden sind dann weak rules (für usual vs.\ marked
order) und \textsq{free rules} (für labile order) (S.\,506). 

Die grammaticalized order von Dane\v{s} entspricht etwa der contrastive order von Harris, ist jedoch wesentlich schärfer bestimmt, da Dane\v{s} nicht allgemein
von "`contrast"' spricht, sondern von "`opposition between two syntactic
categories"'; sein Beispiel ist das S$>$V$>$O"=Muster\footnote{%
  Wenn A und B syntaktische Elemente sind, bedeutet "`A$>$B"': Das Element A steht vor (aber nicht
  unbedingt: unmittelbar vor) dem Element B. "`$>$"' bezeichnet also eine irreflexive, asymmetrische
  transitive Relation;  ich gehe hier davon aus, daß sie konnex ist. (In der Theorie der
  koordinierten Konstruktionen kann es sinnvoll sein, auf diese letzte Annahme zu verzichten; das
  liegt aber jenseits der Gegenstände dieses Kapitels.)   
  
  Die intransitive Relation "`steht unmittelbar vor"' bezeichne ich durch das Verkettungszeichen
  (z.\,B.:   "`A$^{\smallfrown}$B"'). In der Literatur werden diese zwei  "`vor"'-Relationen gewöhnlich
  nicht explizit unterschieden und beide durch einfaches Hintereinanderschreiben (z.\,B.: "`AB"') 
  bezeichnet.%
}
des Englischen\il{Englisch} (S.\,501). Die labile order von Dane\v{s} fällt sicherlich unter die descriptively equivalent order von Harris, und Dane\v{s}s
fixed order entspricht weitgehend der restricted order von Harris. Die
Einführung von \textsq{usual order} bei Dane\v{s} erlaubt es jedoch, gewisse
Zweifelsfälle besser als bei Harris zu klassifizieren. 

Was er mit "`usual"' bzw.\ "`non"=neutral/""marked"' meint, erläutert Dane\v{s} im Anschluß an \citet{Jakobson1963} mit Hilfe eines russischen\il{Russisch} Satzes,
der 3 Wörter enthält.  Alle 6 logisch möglichen Anordnungen der Wörter
sind (bei konstanten syntaktischen Funktionen der Wörter) grammatisch
möglich; in diesem Fall ist also keine strong rule wirksam. Aber nur
einer unter den 6 Sätzen ist "`stylistically neutral"' \citep[268]{Jakobson1963}; dieser hat nach Dane\v{s} usual order. Die anderen 5 Sätze "`are
%page 4
experienced by native speakers and listeners as diverse emphatic shifts"' (\citeyear[269]{Jakobson1963});
sie haben nach Dane\v{s} marked order. Dane\v{s} erläutert weiter:
\begin{exe}
\ex\label{ex:1-1-3}
"`we come to the conclusion that the variations are motivated by their contextual (and situational) dependence and applicability (even the neutral variant clearly presupposes a certain context, or, more precisely, a certain class
of contexts). In other words: every utterance points to a \textsqe{consituation}"' \citep[504]{Danes1967}
\end{exe}
Bei Harris wäre nicht klar, ob die verschiedenen nicht"=neutralen, markierten Varianten unter contrasting order oder unter descriptively equivalent order fallen.

An den deutschen\il{Deutsch} Sätzen in (\ref{ex:1-1-4}) kann man sich das Problem verdeutlichen (Unterstreichung steht für \isi{Betonung}):
\begin{exe}
\ex\label{ex:1-1-4}
\begin{xlist}
\ex\label{ex:1-1-4a} Karl hat den Pass\underline{\underline{a}}nten den Fund gezeigt
\ex\label{ex:1-1-4b} Karl hat den Fund den Pass\underline{\underline{a}}nten gezeigt
\end{xlist}
\end{exe}
Beide Sätze sind akzeptabel, und die einander entsprechenden Wortgruppen haben
die gleiche \isi{syntaktische Funktion}; die Sätze haben\ -- in einem engen Sinne des Wortes~-- die gleiche Bedeutung (logische Charakterisierung). In einem weiten Sinne von
"`semantisch"' weisen sie jedoch semantische Unterschiede auf; jedenfalls gibt es, wie
man sich leicht klarmachen kann, Unterschiede in den Gebrauchsmöglichkeiten der
Sätze (vgl.\ X, [s.\ Anm.\ S.\,\pageref{fn-herausgeber-topo}]). Reicht das aus, um einen "`contrast"' im Sinne von Harris zu etablieren? Mit dem Unterschied zwischen \emph{John saw Bill} und \emph{Bill saw John} möchte
man den Unterschied zwischen (\ref{ex:1-1-4a}) und (\ref{ex:1-1-4b}) nicht gleichsetzen; aber wenn die Gebrauchsunterschiede zwischen ihnen unter "`differences in contextual environment
and in social situation"' fallen, muß man den topologischen Unterschied zwischen
ihnen als contrasting order bezeichnen. Tut man es nicht, müßte descriptively equivalent order vorliegen. Zu dieser Kategorie gehören vermutlich Satzpaare wie (\ref{ex:1-1-5}):
\begin{exe}
\ex\label{ex:1-1-5}
\begin{xlist}
\ex\label{ex:1-1-5a} Karl hat uns gestern geh\underline{\underline{o}}lfen
\ex\label{ex:1-1-5b} gestern hat Karl uns geh\underline{\underline{o}}lfen
\end{xlist}
\end{exe}
Irgendwelche semantischen Unterschiede sind hier schwer auszumachen; in dieser
Hinsicht besteht ein Unterschied zwischen (\ref{ex:1-1-4}) und (\ref{ex:1-1-5}). Bei Dane\v{s} ist dagegen klar, daß der Unterschied in (\ref{ex:1-1-4}) als usual gegenüber marked order zu beschreiben ist; der Unterschied in (\ref{ex:1-1-5}) dürfte dagegen unter labile order fallen.

Wenn man die Existenz von labile order in Betracht zieht, entsteht allerdings eine
Schwierigkeit, die Erklärung von \textsq{marked order} wortgetreu anzuwenden. Ein Beispiel dafür könnte (\ref{ex:1-1-6}) sein:
\begin{exe}
\ex\label{ex:1-1-6}
geh\underline{\underline{o}}lfen hat Karl uns gestern
\end{exe}
Der Satz scheint gegenüber (\ref{ex:1-1-5}) stilistisch markiert zu sein (vgl.\ X, [s.\ Anm.\ S.\,\pageref{fn-herausgeber-topo}]), müßte also auf die
%page 5
Wirkung einer weak rule zurückgehen. In (\ref{ex:1-1-2c}) ist aber vorausgesetzt, daß es genau 1 usual order gibt, der gegenüber eine nicht"=neutrale \isi{Wortfolge} \textsq{markiert} ist; in (\ref{ex:1-1-5})
haben wir deren zwei. Die Erklärung von \textsq{marked order} bedarf also einer Korrektur, und generell bedürfen die Begriffe der \textsq{usual} bzw.\ \textsq{neutral order} und der \textsq{marked} bzw.\ \textsq{non"=neutral order} einer genaueren Explikation. Es ist auch gar nicht ohne
weiteres klar, ob und gegebenenfalls in welcher Weise diese Begriffe für eine sprachwissenschaftliche Theorie überhaupt von Belang sind. Mit diesen Fragen setzen wir
uns in X [s.\ Anm.\ S.\,\pageref{fn-herausgeber-topo}] auseinander; vorläufig versuchen wir die Begriffe so intuitiv zu verwenden, wie sie eingeführt sind.

In vielen Zusammenhängen kann man auch ohne Schaden darauf verzichten, die
Schwierigkeiten mit usual/""marked order vorweg zu klären. Häufig ist es sinnvoll,
einfach zwischen starken Regeln (= strong rules im Sinne von Dane\v{s}) und allen anderen topologischen Regeln zu unterscheiden. Regeln, die keine starken Regeln sind,
bezeichne ich als schwache Regeln. Dane\v{s}s weak rules und free rules sind also
schwache Regeln in meinem Sinne.
\ssubsection{}%1.3.
\label{subsec:1-1.3}
Es drängt sich auf, in der \isi{Topologie} des Deutschen\il{Deutsch} zwei große Phänomenbereiche zu unterscheiden: (a)
den Bereich der starken Regeln; darunter fallen vor allem wesentliche Teile der Syntax von
Nominalphrasen und die Lehre von den topologischen Feldern des Satzes;\footnote{%
  Damit ist der Teil der Syntax gemeint, der sich mit den Stellungsmöglichkeiten der (finiten oder
  infiniten) Verben und den damit zusammenhängenden Fragen beschäftigt. In diesem Zusammenhang wird
  oft von verschiedenen "`Feldern"' im Satz gesprochen, \zb von Vorfeld, Satzfeld, Mittelfeld,
  Hauptfeld, \isi{Nachfeld}. Die Terminologie ist uneinheitlich und z.\,T.\ nicht ganz adäquat, beruht aber
  auf einigen wichtigen Einsichten.%
} 
und (b) den Bereich der schwachen Regeln; darunter fallen besonders jene Elemente, die mehr oder minder gut ihre Stellung innerhalb
der topologischen Felder oder zwischen ihnen verändern können. Es ist klar, daß schwache Regeln
innerhalb von topologischen Bezirken operieren, die durch starke 
Regeln abgegrenzt und definiert werden. So unterliegen die Bestandteile eines einfachen Satzes in vielen Sprachen (\zb im Russischen\il{Russisch}) keinen oder nur wenigen starken
Regeln, aber innerhalb eines komplexen Satzes bilden die Teile eines Teil"=Satzes gewöhnlich eine zusammenhängende Kette. Ähnliches scheint, wie wir sehen werden,
für die topologischen Felder zu gelten.

In den folgenden Abschnitten bespreche ich einige Grundzüge der Lehre von den
topologischen Feldern des Satzes. Diese Lehre bietet einerseits den Rahmen, der für
eine Erörterung normaler (usual), markierter und freier \isi{Wortstellung} vorauszusetzen ist; die Phänomene, die von dieser Lehre erfaßt werden, sind fundamental für
die gesamte Satzlehre des Deutschen\il{Deutsch}. Zugleich präsentiert sich das Deutsche\il{Deutsch} in dieser Lehre als eine \textsq{exotische} Sprache: Eine Kombination von Phänomenen dieser Art
%page 6
ist unter den Sprachen der Welt außerordentlich selten; genauer: Sie ist in dieser
Weise überhaupt nur aus dem Deutschen\il{Deutsch} und dem Niederländischen\il{Niederländisch} bekannt. Die
wenigen Sprachen, aus denen ähnliche Phänomene bekannt sind (das sind vor allem
die nordgermanischen\il{Germanisch!Nord-} und keltischen\il{Keltisch} Sprachen), zeigen sie in anderer Kombination
oder weniger deutlich. Diese Phänomene stellen einer auf theoretisches Verständnis
abzielenden Syntaxforschung eine Reihe von sehr klaren Aufgaben, die sich bislang
einer befriedigenden Lösung entziehen.

Die Lehre von den topologischen Feldern besteht~-- nicht unter diesem
Namen, aber in ihren wesentlichen Inhalten~-- seit mindestens 100
Jahren in gereifter Form. Das heißt jedoch nicht, daß sie allgemein
verstanden und akzeptiert ist; bis in die jüngste Gegenwart hinein
zeigen sich auch Fachleute manchmal unzureichend unterrichtet. Es
scheint mir deshalb günstig, auf die Exposition der Lehre durch eine
kritische Besprechung einer besonders einflußreichen Abhandlung über
topologische Phänomene hinzuleiten: Greenbergs "`Some universals of
grammar with particular reference to the order of meaningful elements"'
(\citeyear{Greenberg1963}).

Bevor wir uns Greenbergs Bemerkungen über das Deutsche\il{Deutsch} zuwenden, sind jedoch einige Überlegungen zu seiner Terminologie nötig.
\section{Dominante Wortstellung}%2.
\label{sec:1-2}



\ssubsection{}%2.1.
\label{subsec:1-2.1}
Der zentrale Begriff in Greenbergs Ausführungen über die topologischen Eigenschaften von Verben ist
\textsq{dominant}. Der Ausdruck wird folgendermaßen eingeführt:\footnote{%
  Man beachte: Wenn Greenberg "`AB"' oder "`ABC"' schreibt, ist immer "`A$>$B"' bzw.\ "`A$>$B$>$C"' gemeint.%
}
\begin{exe}
\ex\label{ex:1-2-1}
\begin{xlist}
\ex\label{ex:1-2-1a} "`The second [set of criteria] will be the relative order of subject, verb, and
object in declarative sentences with \isi{nominal} subject and object.
\ex\label{ex:1-2-1b} The vast majority of languages have several variant orders but a single dominant\is{Wortstellung!dominante} one.
\ex\label{ex:1-2-1c} Logically, there are six possible orders: \isi{SVO}, \isi{SOV}, \isi{VSO}, \isi{VOS}, \isi{OSV}, and
\isi{OVS}.
\ex\label{ex:1-2-1d} Of these six, however, only three normally occur as dominant\is{Wortstellung!dominante} orders.
\ex\label{ex:1-2-1e} The three which do not occur at all, or at least are excessively rare, are \isi{VOS},
\isi{OSV}, and \isi{OVS}."' \citep[76]{Greenberg1963}
\end{xlist}
\end{exe}
Eine nähere Erläuterung dafür, wodurch sich eine dominante Stellung\is{Wortstellung!dominante} gegenüber anderen Stellungen auszeichnet, gibt es nicht.

Die Formulierung in (\ref{ex:1-2-1b}) läßt mehrere Deutungen zu. Die schwächste~-- eine, die
%page 7
für eine vernünftige Interpretation von Greenbergs Aufsatz auf jeden Fall angenommen werden muß~– formuliere ich in (\ref{ex:1-2-2}):
\begin{exe}
\ex\label{ex:1-2-2}
\begin{xlist}
\ex\label{ex:1-2-2a} Für die meisten L\textsubscript{i} (L\textsubscript{i}: eine natürliche Sprache): \\
In L\textsubscript{i} gibt es mehrere verschiedene Wortstellungen.
\ex\label{ex:1-2-2b} Für die meisten L\textsubscript{i}: \\
In L\textsubscript{i} ist genau 1 \isi{Wortstellung} dominant\is{Wortstellung!dominante}. (Vgl.\ (\ref{ex:1-2-1b}).)
\end{xlist}
\end{exe}
Die Formulierung in (\ref{ex:1-2-2a}) läßt zu, daß in einer Sprache L\textsubscript{j} nur eine einzige Stellung existiert, daß es also grammaticalized oder fixed order im Sinne von (\ref{ex:1-1-2a},b) gibt. Offensichtlich sollen Greenbergs verschiedene Hypothesen auch für einen solchen Fall gelten; ich nehme deshalb an, daß eine Stellung auch dann \textsq{dominant} ist, wenn sie die
einzig mögliche ist. Diese Annahme wird durch die Ausführungen über Adjektive
gestützt:
\begin{exe}
\ex\label{ex:1-2-3}
"`The third basis of classification will be the position of qualifying adjectives
[\ldots] in relation to the noun. [\ldots] Here again there is sometimes variation, but
the vast majority of languages have a dominant\is{Wortstellung!dominante} order."' \citep[77]{Greenberg1963}
\end{exe}
Wenn die Stellung der Adjektive nur "`sometimes"' variiert, muß sie in den meisten
Sprachen stabil sein; wenn "`the vast majority of languages"' eine dominante Stellung\is{Wortstellung!dominante}
von \isi{Substantiv} zu Adjektiv hat, müssen darunter also Sprachen mit nicht"=variierender Stellung sein.

Die Formulierung in (\ref{ex:1-2-2b}) läßt zu, daß in einer Sprache L\textsubscript{k} zwar mehrere verschiedene Stellungen existieren, unter diesen aber keine irgendwie (\zb als \textsq{dominant})
ausgezeichnet ist, daß es also labile order im Sinne von (\ref{ex:1-1-2d}) gibt. Sprachen wie L\textsubscript{j} und L\textsubscript{k}
sind nach (\ref{ex:1-2-2}) aber in der Minderzahl. Ähnlich wie Dane\v{s} in (\ref{ex:1-1-2c}) scheint Greenberg
nach (\ref{ex:1-2-2b}) anzunehmen, daß es dann, wenn in L\textsubscript{i} verschiedene Stellungen existieren
und diese nicht gleichwertig sind, genau eine (als \textsq{dominant}) ausgezeichnete Stellung
gibt. (Dies ist problematisch, wie wir an (\ref{ex:1-1-6}) gesehen haben.)
Das "`normally"' in (\ref{ex:1-2-1d}) muß man (wegen des Zusammenhangs mit (\ref{ex:1-2-1e})) offenbar als "`häufig"' oder "`meistens"' lesen; ich paraphrasiere die Folge (\ref{ex:1-2-1c}–e) so:
\begin{exe}
\ex\label{ex:1-2-4}
\begin{xlist}
\ex\label{ex:1-2-4a} Die Menge der logisch möglichen Wortstellungen ist \\
WO\textsubscript{m} = \{S$>$V$>$O, S$>$O$>$V, V$>$S$>$O, V$>$O$>$S, O$>$S$>$V, O$>$V$>$S\}. (Vgl.\ (\ref{ex:1-2-1c}).)
\ex\label{ex:1-2-4b} Es gibt eine Menge WO\textsubscript{t}, WO\textsubscript{t} $\subset$ WO\textsubscript{m}, für die gilt:
\begin{xlist}
\ex\label{ex:1-2-4bi} Für die weitaus meisten L\textsubscript{i} gilt:
L\textsubscript{i} hat ein Element aus WO\textsubscript{t} als dominante Wortstellung\is{Wortstellung!dominante}. (Vgl.\ (\ref{ex:1-2-1d}).)
\ex\label{ex:1-2-4bii} WO\textsubscript{t} = \{S$>$V$>$O, S$>$O$>$V, V$>$S$>$O\}. (Vgl.\ (\ref{ex:1-2-1e}).)
\end{xlist}
\end{xlist}
\end{exe}
Nach dieser Interpretation ist es durchaus möglich, daß eine Sprache L\textsubscript{1}~-- in Sätzen,
die durch (\ref{ex:1-2-1a}) charakterisiert sind~-- die Stellung V$>$O$>$S, O$>$V$>$S oder O$>$S$>$V
%page 8
aufweist und dennoch zum Normaltypus der Sprachen gehört, wie er in (\ref{ex:1-2-4b}) gekennzeichnet ist; diese Stellungen dürfen in L\textsubscript{l} nur nicht dominant\is{Wortstellung!dominante} sein.

Die in (\ref{ex:1-2-2}) und (\ref{ex:1-2-4}) formulierte Deutung gibt, so weit ich sehe, die einzige Möglichkeit an die Hand, die Abschnitte~2–4 von Greenbergs Aufsatz als konsistenten Text
zu interpretieren.

\ssubsection{}%2.2.
\label{subsec:1-2.2}
Im abschließenden Abschnitt~5 verwendet Greenberg den Ausdruck "`dominant\is{Wortstellung!dominante}"' unvermittelt in völlig anderer Weise. Er spricht dort von "`dominance of a particular order over its alternative"' und sagt u.\,a.:
\begin{exe}
\ex\label{ex:1-2-5}
"`Since the \isi{nominal} object may follow the verb whether the pronoun object
precedes or follows, while the \isi{nominal} object may precede the verb only if
the pronoun precedes, we will say that VO is dominant\is{Wortstellung!dominante} over \isi{OV} since \isi{OV}
only occurs under specified conditions, namely when the pronominal object
likewise precedes, while VO is not subject to such limitations."' \citep[97]{Greenberg1963}
\end{exe}
Hier werden verschiedene Sprachen miteinander verglichen, für die eine dominante
\isi{Wortstellung} im Sinne von (\ref{ex:1-2-2b}) bereits etabliert ist. Für nominale Subjekte und Objekte z.\,B.\ ist im Französischen\il{Französisch} wie im Englischen\il{Englisch} S$>$V$>$O die (einzelsprachlich) dominante Stellung\is{Wortstellung!dominante}, völlig unabhängig davon, welche Stellung ein pronominales Objekt einnimmt. In (\ref{ex:1-2-5}) heißt es, V$>$O sei "`dominant\is{Wortstellung!dominante}"' gegenüber O$>$V \textit{deshalb weil} (oder: insofern als) nominale O$>$V"=Stellung nur zusammen mit pronominaler O$>$V"=Stellung
vorkommt, während nominale V$>$O"=Stellung zusammen mit pronominaler V$>$O"=Stellung (wie im Englischen\il{Englisch}) oder zusammen mit pronominaler O$>$V"=Stellung (wie
im Französischen\il{Französisch}) vorkommt. Das hat nur Sinn, wenn "`dominant\is{Wortstellung!dominante}"' hier in einem typologischen Sinne gemeint ist und nicht, wie in (\ref{ex:1-2-2b}), in Hinsicht auf eine \isi{Einzelsprache}. Dies kann man so erläutern:
\begin{exe}
\ex\label{ex:1-2-6}
Eine \isi{Wortstellung} A$>$B ist (hinsichtlich der Eigenschaft \emph{E}) typologisch dominant\is{Wortstellung!dominante} gegenüber der (\textsq{rezessiven}) \isi{Wortstellung} B$>$A gdw.\ (a) und (b)
gelten:
\begin{xlist}
\ex\label{ex:1-2-6a} 
\begin{xlist}
\ex\label{ex:1-2-6ai} Es gibt Sprachen, die die (einzelsprachlich dominante) \isi{Wortstellung}
A$>$B aufweisen.
\ex\label{ex:1-2-6aii} Einige dieser Sprachen, aber nicht alle, haben die Eigenschaft \emph{E}.
\end{xlist}
\ex\label{ex:1-2-6b} 
\begin{xlist}
\ex\label{ex:1-2-6bi} Es gibt Sprachen, die die (einzelsprachlich dominante) \isi{Wortstellung}
B$>$A aufweisen.
\ex\label{ex:1-2-6bii} Alle diese Sprachen haben die Eigenschaft \emph{E}.
\end{xlist}
\end{xlist}
\end{exe}
Ich sehe nicht, wie man Greenbergs Abschnitt~5 verstehen könnte, wenn man dort
"`dominant\is{Wortstellung!dominante}"' nicht als "`typologisch dominant\is{Wortstellung!dominante}"' im Sinne von (\ref{ex:1-2-6}) interpretiert.


%page 9

Von dem Begriff \textsq{einzelsprachlich dominant} unterscheidet sich \textsq{typologisch dominant} nicht nur durch den notwendigen Bezug auf verschiedene Sprachen, sondern auch dadurch, daß er ein notwendig relationaler Begriff ist. Im Zusammenhang
mit (\ref{ex:1-2-2}) habe ich angemerkt, daß in einer Sprache L\textsubscript{i} eine Stellung A$>$B offenbar auch
(und immer) dann \textsq{dominant} ist, wenn die Stellung B$>$A in L\textsubscript{i} nicht vorkommt, und
unter der inhaltlichen Deutung von \textsq{einzelsprachlich dominant}, die in \ref{subsec:1-2.3} zur Sprache kommt, erweist sich das auch als völlig natürlich. Wenn man eine Stellung auch
dann typologisch dominant\is{Wortstellung!dominante} nennen würde, wenn (\ref{ex:1-2-6bi}) nicht erfüllt wäre, wäre dies
ein inhomogener Begriff. Denn dann wäre es widersinnig, die Forderung (\ref{ex:1-2-6aii}) aufzustellen. Auf diese Forderung kann man aber nicht verzichten, wenn (\ref{ex:1-2-6bi}) erfüllt
ist.

Eine natürliche Folge von (\ref{ex:1-2-6}) ist, daß es typologisch dominante Wortstellungen
gibt, die in manchen Einzelsprachen überhaupt nicht vorkommen. Im Japanischen\il{Japanisch}
etwa ist O$>$V (einzelsprachlich) dominant\is{Wortstellung!dominante} und V$>$O inexistent; trotzdem ist V$>$O
nach (\ref{ex:1-2-5}) typologisch dominant\is{Wortstellung!dominante}. Dementsprechend ist es auch möglich, daß eine typologisch dominante Wortstellung\is{Wortstellung!dominante} A$>$B in weniger Sprachen vorkommt als ihr \textsq{rezessives} Gegenstück.

Diese Interpretation wird unmittelbar bestätigt durch folgende Erläuterung:
\begin{exe}
\ex\label{ex:1-2-7}
"`Note that the notion of dominance is not based on its more frequent occurrence but on the logical factor of a zero in the tetrachoric table. It is not difficult to construct an example in which one of the recessive alternatives is
more frequent than the dominant\is{Wortstellung!dominante}. Dominance [\ldots] relations can be derived
quite mechanically from such a table with a single zero."' \citep[97]{Greenberg1963}
\end{exe}
In einer tetrachoric table werden verschiedene topologische Eigenschaften korreliert;
in den Zellen der Tabelle ist angegeben, wieviele Sprachen die jeweilige Korrelation
(von einzelsprachlich dominanten Wortstellungen verschiedener Art) erfüllen. Für
die in (\ref{ex:1-2-5}) erwähnte Stellung von nominalen und pronominalen Objekten zu Verben sähe das etwa so aus (\textit{i, j, k} $>0$):

\begin{exe}
\extab\label{ex:1-2-8}
\begin{tabular}{ r | c c }
\isi{nominal} & O$>$V & V$>$O \\\hline
pro\isi{nominal} \\
O$>$V & \emph{i} & \emph{j} \\
V$>$O & 0 & \emph{k} \\
\end{tabular}
\end{exe}

Das heißt: In \textit{i} Sprachen stehen nominale wie pronominale Objekte vor dem \isi{Verb}; in \textit{k} Sprachen folgen beide auf das \isi{Verb}; in \textit{j} Sprachen folgt ein nominales Objekt auf das \isi{Verb}, wärend ein pronominales Objekt vor dem \isi{Verb} steht (jeweils in der
%page 10
dominanten Stellung). Aus der Tabelle geht hervor, daß für nominale Objekte die \isi{Wortstellung} V$>$O (hinsichtlich der Stellung von pronominalen Objekten relativ zum
\isi{Verb}) gegenüber O$>$V typologisch dominant\is{Wortstellung!dominante} ist (weil die Kombination pronominal
V$>$O/""\isi{nominal} O$>$V in keiner Sprache vorkommt).\footnote{%
  Aus denselben Gründen ergibt sich für pronominale Objekte umgekehrt, daß O$>$V gegenüber V$>$O
  (hinsichtlich der Stellung von nominalen Objekten zum \isi{Verb}) typologisch dominant\is{Wortstellung!dominante} ist.%
}
Dabei könnte \emph{i} durchaus größer
sein als \emph{j} und/""oder als \emph{k}.

Daß Greenberg zwei ganz verschiedene Begriffe~– \textsq{(einzelsprachlich) dominant}
und \textsq{typologisch dominant}~– verwendet, ist häufig übersehen worden. So beginnt
Dane\v{s} den Aufsatz, aus dem wir in \ref{subsec:1-1.2} zitiert haben, mit den Worten:\footnote{%
  Dane\v{s} zitiert nach der 1.~Auf"|lage von Greenberg (Hrsg.) (\citeyear{Universalsoflanguage}). Die
  beiden Auf"|lagen sind (mit unwesentlichen Ausnahmen) textidentisch, aber verschieden paginiert. Die
  Angabe "`(p.\,76)"' in (\ref{ex:1-2-9}) verweist auf das Textstück, aus dem (\ref{ex:1-2-5}) und
  (\ref{ex:1-2-7}) stammen.%
}
\begin{exe}
\ex\label{ex:1-2-9}
"`In his paper ``Some Universals of Grammar'', J.\,H.\ Greenberg introduces
the notion of \textsc{dominant\is{Wortstellung!dominante} order} of syntactic elements and explains (p.\,76) that the ``dominance is not based on its more frequent occurrence'' (a dominant\is{Wortstellung!dominante} order is not that alternative which is more frequent than its opposite,
the \textsqe{recessive} order) but on the fact that the dominant\is{Wortstellung!dominante} order can always occur while its opposite is present only under specified conditions, i.e., in cooccurrence with another, \textsqe{harmonic}, construction. These conditions are
stated in terms of grammatical notions, such as \isi{Verb}, Object, Pronominal
Object, etc.

\hspace{1em} R.\ Jakobson [\ldots] very aptly shows that in Slavic languages the
\textsqe{recessive alternatives} to a \textsqe{dominant\is{Wortstellung!dominante} order} are
numerous. The Russian\il{Russian}\il{Russisch} sentence corresponding to ``Lenin cites Marx'' may
occur in six different variants. [\ldots] It is worth noting that all the
six logically possible orders may occur, even those three that,
according to Greenberg, ``do not occur at all, or at least are
excessively rare'', namely \isi{VOS}, \isi{OSV}, \isi{OVS}. Jakobson also points out that
the conditions by which the occurrence (selection) of the different
variants is regulated are not of grammatical character"' \citep[499]{Danes1967}
\end{exe}
Dane\v{s} umschreibt hier Greenbergs Ausführungen zum Begriff der
typologischen Dominanz, unterstellt aber offenbar~-- und kritisiert~–,
daß damit der Begriff der (einzelsprachlich) dominanten \isi{Wortstellung}
erläutert werden soll. Dane\v{s} weist zu Recht darauf hin, daß unter
dieser Interpretation widersprüchliche Ergebnisse folgen. Aber
Dane\v{s}s Unterstellung ist offensichtlich ein Irrtum:
"`(einzelsprachlich) dominant\is{Wortstellung!dominante}"' wird bei Greenberg überhaupt nicht
erläutert; "`typologisch dominant\is{Wortstellung!dominante}"' wird zwar erläutert, hat mit
(einzelsprachlich) dominant\is{Wortstellung!dominante} oder mit "`normal"' in Dane\v{s}s Sinn aber
nur indirekt ewas zu tun.
%page 11

\ssubsection{}%2.3.
\label{subsec:1-2.3}
Aber was meint Greenberg mit "`(einzelsprachlich) dominant\is{Wortstellung!dominante}"'? Möglicherweise tatsächlich so etwas wie Jakobsons \textsq{stilistisch neutrale} und Dane\v{s}s \textsq{normale}
(usual) \isi{Wortstellung}. Als Indiz kann man einige andere Adjektive auf"|führen, die
Greenberg, wie es scheint, in demselben Sinn wie "`dominant\is{Wortstellung!dominante}"' gebraucht.

An mehreren Stellen steht "`normal"', wo man aufgrund des Textzusammenhangs
"`dominant\is{Wortstellung!dominante}"' erwarten könnte. So werden die Sprachen gemäß (\ref{ex:1-2-1}) u.\,a.\ danach klassifiziert, ob ihre dominante Abfolge von \isi{Subjekt}, Objekt und Vollverb V$>$S$>$O, S$>$V$>$O
oder S$>$O$>$V ist; diese Stellungstypen werden als I, II und III bezeichnet (S.\,77). Auf
S.\,108 wird dann eine Tabelle mit den Worten erläutert: "`I indicates that normal
word order is verb"=subject"=object"'. Hier wird also "`normal"' in genau derselben Weise gebraucht wie zuvor "`dominant\is{Wortstellung!dominante}"'. Ebenso \zb S.\,79 (Universal~4), S.\,83f (mehrfach,
u.\,a.\ Universals~14 und 15), S.\,91: "`in Masai, whereas normal order for \isi{nominal} object
is \isi{VSO} \ldots"'. In Universal~23 (S.\,89f) findet sich "`usually"'; auch "`usual"' wird verwendet: "`In Nubian, the usual \isi{nominal} order is \isi{SOV}, but the alternative \isi{SVO} is fairly frequent"' (S.\,91). Dort auch "`regularly"': "`No contrary instances occur in the sample of
a pronominal object regularly following the verb \ldots"' (S.\,91).

Es scheint danach, daß durch "`dominant\is{Wortstellung!dominante}"' eine Eigenschaft bezeichnet
wird, die eng verwandt ist mit intuitiven Begriffen der Normalität und
des Üblichen. Wenn man demzufolge annimmt, daß "`dominant\is{Wortstellung!dominante}"' im
wesentlichen das gleiche ist wie Jakobsons "`stylistically neutral"',
löst sich auf jeden Fall eine potentielle Schwierigkeit: daß eine
Stellung A$>$B in einer Sprache L\textsubscript{i} offensichtlich
\textsq{dominant} sein soll, wenn B$>$A in L\textsubscript{i} nicht
vorkommt. Denn eine Stellung für einen fundamentalen \isi{Satztyp} (wie
\textsq{Deklarativsatz}), zu der es keine Alternative gibt, kann gar
nicht anders als stilistisch neutral sein. (Das hindert nicht, daß ein
ganzer \isi{Satztyp} mit speziellen Stellungseigenschaften~-- etwa:
\isi{Interrogativsatz} oder \isi{Imperativsatz}~-- als irgendwie \textsq{markiert}
empfunden wird.)

Zum Abschluß dieser terminologischen Erörterungen ist auf ein
mögliches Mißverständnis hinzuweisen. Um die Form der Universalien zu
erläutern, gibt Greenberg ein fiktives Beispiel:
\begin{exe}
\ex\label{ex:1-2-10}
"`If a language has verb"=subject"=object as its basic word order in main declarative clauses, the dependent genitive always follows the governing noun."'
\citep[74]{Greenberg1963}
\end{exe}
Inhaltlich entspricht das~-- in Verbindung mit dem Universal 2~– dem Universal~3:
\begin{exe}
\ex\label{ex:1-2-11}
"`Languages with dominant\is{Wortstellung!dominante} \isi{VSO} order are always prepositional."' \citep[78]{Greenberg1963}
\end{exe}
Diese Parallelität scheint dafür zu sprechen, daß "`basic"' in (\ref{ex:1-2-10}) synonym mit "`dominant\is{Wortstellung!dominante}"' in (\ref{ex:1-2-11}) ist. Dem steht jedoch der Wortgebrauch S.\,79f entgegen:
\begin{exe}
\ex\label{ex:1-2-12}
"`all \isi{VSO} languages apparently have alternative \underline{basic} orders among which
\isi{SVO} always figures. On the other hand, in a substantial proportion, possibly
a majority, of type III languages, the verb follows all of its modifiers, and if
any other \underline{basic} order is allowed, it is \isi{OSV}. Thus the verb [\ldots] is always at
the end in verbal sentences. It is not logically required, of course, that languages all of whose \underline{basic} orders involve the verb in the third position should
also require all verb modifiers to precede the verb, [\ldots]"'
(\citealt[79]{Greenberg1963}; \isi{Hervorhebung} hinzugefügt)
\end{exe}
%page 12
So auch in Universals 6 und 7. Es kann also in einer Sprache L\textsubscript{i} \zb die \textsq{dominante}
Stellung V$>$S$>$O geben und daneben noch alternative "`basic orders"'; mithin ist nicht
jede \textsq{basic order} zugleich eine \textsq{dominant\is{Wortstellung!dominante} order}. Aber was ist dann eine \textsq{basic order}? Die einzige konsistente Interpretation, die mir möglich scheint, ergibt sich aus
der Einführung in die \textsq{basic order typology}:
\begin{exe}
\ex\label{ex:1-2-13}
\begin{xlist}
\ex\label{ex:1-2-13a}
 "`it is convenient to set up a typology involving certain basic factors of word
order.
\ex\label{ex:1-2-13b} This typology will be referred to as the basic order typology.
\ex\label{ex:1-2-13c} Three sets of criteria will be employed.
\ex\label{ex:1-2-13d} The first of these is the existence of prepositions as against postpositions.
[\ldots]
\ex\label{ex:1-2-13e} The second will be the relative order of subject, verb, and object [\ldots]."' \citep[76]{Greenberg1963}
\end{xlist}
\end{exe}
(Der Ausschnitt (\ref{ex:1-2-13e}) ist in (\ref{ex:1-2-1})
ausführlicher zitiert. Der dritte set of criteria ist die in
(\ref{ex:1-2-3}) zitierte Stellung von Adjektiven zu Substantiven.)

Mit "`basic factors"' in (\ref{ex:1-2-13a}) sind vermutlich
\textsq{grundlegende, fundamentale} Faktoren gemeint und nicht
\textsq{einzelsprachlich dominante} Faktoren; so auch bei "`Basic Data
on the 30"=Language Sample"' (S.\,107). Demnach ist die "`basic order
typology"' von (\ref{ex:1-2-13b}), die im Text häufig erwähnt wird,
eher eine order typology, die \textsq{basic} ist, als eine Typologie
von \textsq{basic orders}. Entsprechend ist auch einfach von "`the
basic typology"' (S.\,91) die Rede. Die "`Basic Order Types"' (S.\,108f),
die sich aus dieser Typologie ergeben, sind nicht Typen von
\textsq{basic orders}, sondern Sprachtypen, die durch die
verschiedenen Kombinationen der "`basic factors of word order"'
charakterisiert sind. Wenn im Zusammenhang mit der Stellung von
\isi{Subjekt}, Objekt und \isi{Verb} von "`basic order"' die Rede ist, wie es in
(\ref{ex:1-2-10}) und (\ref{ex:1-2-12}) der Fall ist, dürften deshalb
die Bedingungen von (\ref{ex:1-2-13e}) gemeint sein: nominales (und
nicht pronominales) \isi{Subjekt} und Objekt in Deklarativsätzen (und nicht
\zb in Imperativ- oder Interrogativsätzen). Gewöhnlich ist dann
eine der Stellungen, die im Sinne von (\ref{ex:1-2-13e})
\textsq{basic} sind, \textsq{dominant} im Sinne von \textsq{normal}
oder \textsq{usual}. Aufgrund dieser dominanten Stellung bezeichnet
Greenberg eine Sprache als \textsq{\isi{VSO}"=Sprache},
\textsq{\isi{SVO}"=Sprache} oder \textsq{\isi{SOV}"=Sprache}.

%page 13

Es wäre hilfreich, wenn sich die Vermutungen, die wir über Greenbergs Terminologie angestellt haben, irgendwie erhärten ließen. Aber ich sehe keine Möglichkeit,
aus seinem Text genaueren Aufschluß zu erhalten. Im folgenden unterstelle ich, daß
unsere Vermutungen nicht in schädlicher Weise unzutreffend sind.


\section{Deklarativsätze: Deutsch als S$>$V$>$O"=Sprache?}%3.
\label{sec:1-3}


Nach diesen Präliminarien kommen wir zu Greenbergs Bemerkungen über das
Deutsche\il{Deutsch}. Das Deutsche\il{Deutsch} spielt in seinem Aufsatz eine eigenartige Rolle. Es gehört
nicht zu den 30 Sprachen, auf deren Analyse die Universalien basieren; aber Greenberg macht im Appendix~II inhaltliche Aussagen über das Deutsche\il{Deutsch}, und an zwei
Textstellen (S.\,82, 104) zieht er das Deutsche\il{Deutsch} heran, um seine Ausführungen zu illustrieren. Dabei verwendet er deutsche Sätze, ohne ihren Auf"|bau genauer zu erläutern oder eine Übersetzung anzugeben.

\ssubsection{Finite und infinite Verben}%3.1.
\label{subsec:1-3.1}

\ssubsubsection{}%3.1.1.
\label{subsubsec:1-3.1.1}

Nach der Übersicht in Appendix~II gehört das Deutsche\il{Deutsch} (wie auch das Niederländische\il{Niederländisch}) zu den S$>$V$>$O"=Sprachen (S.\,109, Nr.\,10). Das bedeutet, wie aus (\ref{ex:1-2-1}) hervorgeht:
\begin{exe}
\ex\label{ex:1-3-1}
In deklarativen Sätzen mit nominalem \isi{Subjekt} und Objekt hat das Deutsche\il{Deutsch}
die Folge S$>$V$>$O als einzige dominante Stellung\is{Wortstellung!dominante}.
\end{exe}
Worauf könnte sich diese Behauptung stützen? Wir finden Beispiele wie (\ref{ex:1-3-2}):
\begin{exe}
\ex\label{ex:1-3-2}
\begin{xlist}
\ex\label{ex:1-3-2a} der Hund erblickte einen Hasen
\ex\label{ex:1-3-2b} Karl brachte die Kartoffeln
\end{xlist}
\end{exe}
Offensichtlich weisen sie S$>$V$>$O"=Stellung auf. Aber ist dies der einzige Typ von deklarativen Sätzen mit nominalem \isi{Subjekt} und Objekt, die im Deutschen\il{Deutsch} dominante
Stellung haben? Wir finden auch Beispiele wie (\ref{ex:1-3-3}):
\begin{exe}
\ex\label{ex:1-3-3}
\begin{xlist}
\ex\label{ex:1-3-3a} der Hund hat einen Hasen erblickt
\ex\label{ex:1-3-3b} Karl soll die Kartoffeln bringen
\end{xlist}
\end{exe}
Diese Sätze sind intuitiv in keiner Weise \textsq{markiert}, sie sind stilistisch neutral in Jakobsons Sinn. Ich sehe keinen Grund, anzunehmen, ihre \isi{Wortstellung} sei nicht dominant\is{Wortstellung!dominante}. Sie weisen jedoch nicht S$>$V$>$O"=Stellung auf, sondern S$>$O$>$V.

Um dies zu erkennen, muß man auf Greenbergs Ausführungen zu Hilfsverben
eingehen:
\begin{exe}
\ex\label{ex:1-3-4}
"`Another relation of verb to verb is that of inflected auxiliary to
main verb. For present purposes, such a construction will be defined
as one in which a closed class of verbs (the auxiliaries) inflected
for both person and number is in construction with an open class of
verbs not inflected for both person and number. For example, in
English\il{English} \textsqe{is going} is such a construction."' \citep[84]{Greenberg1963}
\end{exe}
%page 14
Darauf folgt eine Korrelationstabelle, in der die Eigenschaften "`Auxiliary precedes
verb"' und "`Auxiliary follows verb"' auftreten.

Greenberg unterscheidet also zwischen (flektiertem)
\isi{Hilfsverb}\footnote{%
  Er erkennt auch die Existenz von unflektierten Hilfsverben an: "`Uninflected auxiliaries will be
  considered later in connection with verb inflections."' \citep[85]{Greenberg1963}. Leider geht er,
  entgegen dieser Ankündigung, nirgendwo auf unflektierte Hilfsverben ein.%
} und \isi{Verb} (Vollverb, Hauptverb), und wenn von den topologischen Beziehungen zwischen \isi{Subjekt}, Objekt und \textsq{Verb}
die Rede ist, ist immer das Vollverb gemeint. Das muß man aus der Erörterung von Fragesätzen
schließen. Greenberg will illustrieren, daß in manchen Sprachen eine 
\isi{Wortgruppe}, die ein \isi{Interrogativpronomen} enthält, an den Satzanfang
tritt (so daß die Stellung, die für deklarative Sätze charakteristisch
ist, in solchen Sätzen nicht gilt), und benutzt dafür das Englische\il{Englisch}:
\begin{exe}
\ex\label{ex:1-3-5}
\begin{xlist}
\ex\label{ex:1-3-5a} what did he eat?
\ex\label{ex:1-3-5b} with whom did he go?
\end{xlist}
\end{exe}
Daran schließt sich folgender Text an:
\begin{exe}
\ex\label{ex:1-3-6}
"`Many languages which put interrogatives first likewise invert the order of
verb and subject (e.g., German \textsqe{Wen sah er?}). Such languages sometimes
invert for yes"=no questions, (e.g., \textsqe{Kommt er?})."' \citep[82]{Greenberg1963}
\end{exe}
%\addlines
Wieso führt Greenberg hier das Deutsche\il{Deutsch} ein, warum reicht das Englische\il{Englisch} nicht? Offenbar deshalb, weil das \emph{did} in (\ref{ex:1-3-5}) nicht ein \textsq{Verb} im relevanten Sinne ist; in (\ref{ex:1-3-5a})
sieht Greenberg offensichtlich eine Manifestation der Folge O$>$S$>$V. Dasselbe gilt für
Entscheidungsfragen: In \emph{Kommt er?} hat man die Folge V$>$S; im entsprechenden englischen\il{Englisch} Beispiel \emph{will he come?} hätte man nach Greenberg offenbar S$>$V"=Stellung, so daß die Folge (O)$>$V$>$S, um die es in (\ref{ex:1-3-6}) geht, wieder nur am Deutschen\il{Deutsch} und
nicht am Englischen\il{Englisch} demonstriert werden kann.

Auf die Eigenschaften von deutschen\il{Deutsch} Fragesätzen kommen wir in
\ref{sec:1-4}. zurück. Im Moment halten wir fest: Nach Greenberg ist
ein infinites Vollverb hinsichtlich der topologischen Beziehungen zu
\isi{Subjekt} und Objekt ein \textsq{Verb} (V). Beispiele wie (\ref{ex:1-3-3})
weisen daher S$>$O$>$V"=Stellung auf, oder genauer:\footnote{%
  Wenn man
  gegen Greenberg das \isi{Hilfsverb} als "`V"' kennzeichnet, erhält man die
  Folge S$>$V$>$O$>$V. Das ändert jedoch nichts Wesentliches: Wir
  haben die Folge O$>$V, was bei Abwesenheit des Hilfsverbs unmöglich
  ist.~-- Vielleicht findet sich eine Interpretation von "`dominant\is{Wortstellung!dominante}"',
  nach der solche Sätze keine \textsq{dominante} (sondern vielleicht nur
  \textsq{basic}) \isi{Wortstellung} aufweisen. Wie auch immer eine Typologie
  Greenbergscher Art die Vernachlässigung von Beispielen wie
  (\ref{ex:1-3-3}) rechtfertigen mag, die korrekte Analyse des
  Deutschen\il{Deutsch} kann von ihrer Ignorierung nicht profitieren.%
}
\begin{exe}
\ex\label{ex:1-3-7}
S$>$Aux$>$O$>$V
\end{exe}
%page 15
Wir finden also nebeneinander S$>$V$>$O und S$>$O$>$V. Das sieht aus, als sei die \isi{Wortstellung} im Deutschen\il{Deutsch} frei (\textsq{labile}). Tatsächlich unterliegt sie jedoch strengen Restriktionen. Beispiele wie (\ref{ex:1-3-8}) mit S$>$O$>$V oder (\ref{ex:1-3-9}) mit S$>$V$>$O werden gemeinhin
nicht als akzeptable Sätze des Deutschen\il{Deutsch} betrachtet (oder als literarische Lizenzen
gewertet, die (definitionsgemäß) ein außernormaler Phänomenbereich sind):
\begin{exe}
\ex\label{ex:1-3-8}
\begin{xlist}
\ex[*]{
\label{ex:1-3-8a} der Hund einen Hasen erblickte
}
\ex[*]{
\label{ex:1-3-8b} der Hund einen Hasen erblickt hat
}
\ex[*]{
\label{ex:1-3-8c} der Hund einen Hasen hat erblickt
}
\end{xlist}
\end{exe}
\begin{exe}
\ex\label{ex:1-3-9}
\begin{xlist}
\ex[*]{
\label{ex:1-3-9a} der Hund erblickt einen Hasen hat
}
\ex[*]{
\label{ex:1-3-9b} der Hund erblickt hat einen Hasen
}
\ex[*]{
\label{ex:1-3-9c} der Hund hat erblickt einen Hasen
}
\end{xlist}
\end{exe}
Unter Greenbergs Analyse wirkt das chaotisch und erweckt den Eindruck, daß es im
Deutschen\il{Deutsch} kein begreifbares \isi{Schema} für dominante Wortstellungen gibt. (Dies ist
vermutlich ein Grund gewesen, weshalb Sprecher von S$>$V$>$O"=Sprachen das Deutsche\il{Deutsch} jahrhundertelang als unbegreif"|liche und barbarische Sprache betrachtet haben.)
Aber ein Sprecher des Deutschen\il{Deutsch} beurteilt (\ref{ex:1-3-8}) und (\ref{ex:1-3-9}) als abweichend, obwohl er
über diese oder ähnliche Beispiele nie explizit unterrichtet worden ist. Das zeigt: Der
Sprecher verfügt über Regelwissen, das ihn zu solchen Urteilen veranlaßt.

\ssubsubsection{}%3.1.2.
\label{subsubsec:1-3.1.2}

Welche Regeln sind hier wirksam? Offenbar spielt es eine Rolle, ob V flektiert
(statt dessen künftig: finit) oder infinit ist. Finitheit spielt auch beim \isi{Hilfsverb} eine
Rolle: Bei der Folge S$>$Aux$>$O$>$V in (\ref{ex:1-3-3a}) ist Aux finit; ein infinites Aux dagegen muß
auf V folgen:
\begin{exe}
\ex\label{ex:1-3-10}
der Hund soll einen Hasen erblickt haben

\ex
\label{ex:1-3-11}
\begin{xlist}
\ex[*]{
\label{ex:1-3-11a}
der Hund soll haben einen Hasen erblickt}
\ex[*]{
\label{ex:1-3-11b}
der Hund haben soll einen Hasen erblickt}
\ex[*]{
\label{ex:1-3-11c}
der Hund haben einen Hasen erblickt soll}
\end{xlist}
\end{exe}
In (\ref{ex:1-3-10}) ist das erste Aux (\emph{soll}) finit, und das zweite Aux (\emph{haben}) nach dem V ist infinit. Andere Positionen der Hilfsverben wie in (\ref{ex:1-3-11}) sind nicht akzeptabel.

\addlines[2]
Man kann auf verschiedene Weisen versuchen, diese Zusammenhänge darzustellen. Die Erkenntnis, daß da eine~-- allerdings abstrakte~-- sehr einfache Regularität obwaltet, hat \citet{Herling1821T} besonders klar formuliert:

%page 16
\begin{exe}
\ex\label{ex:1-3-12}
\begin{xlist}
\ex\label{ex:1-3-12a}  "`Die natürliche Folge der wesentlichen Bestandtheile eines Hauptsatzes ist
ganz der bei Bildung eines Urtheils stattfindenden Verrichtung unsers
Denkvermögens gemäß. Mag man diese nun darstellen als das Eintragen eines Begriffs in einen andern [\ldots] oder als die Zerlegung eines Begriffs in seine Bestandtheile [\ldots], immer steht 1) das Subject voran, als das Einzutragende oder zu Zerlegende; ihm folgt 2) die Aussage als die Bezeichnung der
Eintragung (Subsumption) oder der Zerlegung, und dieser 3) das Prädicat,
wofern nicht die Aussage mit dem Ausgesagten in einem Worte vereinigt
ist"' \citep[297]{Herling1821T}
\ex\label{ex:1-3-12b} "`Da die umschriebenen Zeitformen der deutschen\il{Deutsch} Conjugation die Scheidung der Aussage von dem ausgesagten Prädicate am sichtbarsten darstellen, so müssen wir die Stellung des Prädicats gegen seine Nebenbestimmungen, so wie sie bei diesen Zeitformen Statt findet, als die natürliche Stellung
ansehen. Daß bei den Verben, welche mit Adverbien oder Präpositionen
zusammengesetzt sind, diese an jener eigenthümlichen Stelle zurückbleiben,
wenn im \isi{Präsens} oder Imperfect das Prädicat mit der Aussage vereint wird,
rechtfertigt diese Ansicht noch mehr, \zb "`Er hat mir gestern mein Verfahren \so{vorg}eschrieben"' und: "`Er schrieb mir gestern mein Verfahren \so{vor}."' "'
\citep[298f]{Herling1821T}
\ex\label{ex:1-3-12c} "`Wird nun das Ausgesagte, wie im \isi{Präsens} und Imperfect, mit der Aussage
zusammengezogen, so ändert dieses nichts an den vorstehenden Regeln von
der Folge seiner Bestimmungen, und selbst die mit ihm verbundenen Adverbien und Präpositionen bleiben an der ursprünglichen Stelle zurück."' \citep[306f]{Herling1821T}
\end{xlist}
\end{exe}
Die Terminologie bedarf einer Erläuterung. "`Prädicat"' ist für Herling primär ein logischer Begriff: Ein \isi{Prädikat} ist etwas, was ausgesagt wird, und zwar ausgesagt über
einen Gegenstand, das "`Subject"'; daher heißt es auch das Ausgesagte. Sprachlich
wird es dargestellt durch eine Wortform eines Vollverbs (oder durch ein prädikatives Adjektiv, \isi{Substantiv} usw.). Die Verknüpfung von \isi{Subjekt} und \isi{Prädikat} geschieht
durch die \textsq{Aussage}; diese wird sprachlich dargestellt durch die morphologischen
Charakteristika, durch die sich finite Verben auszeichnen.\footnote{%
  Mit "`Aussage"' übersetzt Herling den sonst~-- \zb bei \citet[16ff]{Vater1805}~– üblichen
  Ausdruck "`Copula"'. Später kehrt er zu diesem Ausdruck zurück und schreibt z.\,B.: "`Diesen
  Grundsätzen gemäß ist die \isi{Wortfolge} in den wesentlichen Theilen des \so{Hauptsatzes} [\ldots]:
  \so{Subject, Copula, Prädicat}. Nur, wo diese, wie in den umschriebenen Zeit"= und Modusformen,
  getrennt erscheinen, läßt sich das Gesetz dieser Folge erkennen. Wo Prädicat und Copula zu Einem
  Worte verschmelzen, steht die Verschmelzung an der Stelle der Copula, des eigentlichen Verbs, und
  das Prädicat läßt alle, seinen Begriff näher individualisirenden Bestimmungen [\ldots] an seiner
  eigentlichen Stelle, an der Stelle des Prädicats, zurück [\ldots]. Dies beweist auch für die einfachen Formen, \isi{Präsens} und Imperfect, daß ihre
  eigentliche Stelle am Ende des Hauptsatzes ist."' \citep[84]{Herling1830}

  Genau wie viele heutige Autoren befolgt Herling das (von \citealt{Frege1899} so genannte) \textsq{Prinzip der
  Nichtunterscheidung des Verschiedenen}, indem er \zb sowohl den Gegenstand, über den etwas
  ausgesagt wird, als auch den sprachlichen Ausdruck, der diesen Gegenstand bezeichnet, als \isi{Subjekt}
  bezeichnet. Entsprechend bei "`Ausgesagtes (\isi{Prädikat})"', "`Aussage (Copula)"' usw.%
}
\isi{Adverbiale} Ausdrücke
%page 17
und Objekte betrachtet er als \textsq{(Neben-)Bestimmungen} des Prädikats.

Die Stellung des Hauptverbs, die bei den "`umschriebenen Zeitformen"' zu beobachten ist, sieht Herling als die "`natürliche Stellung"' an. Ein wesentlicher Grund dafür sind die Stellungsregularitäten der "`(Neben-)Bestimmungen"'. (Auf diese Begründung gehen wir in X [s.\ Anm.\ S.\,\pageref{fn-herausgeber-topo}]\label{X:8} ein.) Die \textsq{natürliche} Stellung wäre demnach (\ref{ex:1-3-13}):
\begin{exe}
\ex\label{ex:1-3-13}
\isi{Subjekt}$>$Aussage$>$Bestimmungen$>$\isi{Prädikat}
\end{exe}
Diese Folge ist nahezu gleichwertig mit dem \isi{Schema} (\ref{ex:1-3-7}). Es gibt jedoch einen entscheidenden Unterschied: "`Aux"' in (\ref{ex:1-3-7}) ist ein Symbol für eine Wortklasse; "`Aussage"' in (\ref{ex:1-3-13}) ist eine Bezeichnung für etwas, was als finites \isi{Verb} realisiert wird. Diese
Trennung zwischen Flexionskategorie (Finitheit) und Wortklasse (\isi{Verb}) erlaubt es
Herling, die wesentliche Regularität zu formulieren: Zwischen S und O steht immer
die \textsq{Aussage} (ein finites \isi{Verb}). Nach dem Objekt steht das \isi{Prädikat} (das Vollverb)~–
"`wofern nicht die Aussage mit dem Ausgesagten in einem Worte vereinigt ist"', wie
es im \isi{Präsens} und \isi{Präteritum} ("`Imperfect"') der Fall ist. Wenn wir wie in (\ref{ex:1-3-2}) die Abfolge S$>$V$>$O finden, so manifestiert sie nach Herlings Erkenntnis nicht eine dominante \isi{Wortfolge} (\ref{ex:1-3-14a}), sondern die dominante Folge (\ref{ex:1-3-14b}), und diese ist eine Variante der dominanten \isi{Wortstellung} (\ref{ex:1-3-14c}), die daraus resultiert, daß das \isi{Verb} in (\ref{ex:1-3-2})
finit ist:
\pagebreak
\begin{exe}
\ex\label{ex:1-3-14}
\begin{xlist}
\ex\label{ex:1-3-14a} \isi{Subjekt}$>$\isi{Verb}$>$Objekt
\ex\label{ex:1-3-14b} \isi{Subjekt}$>$finites Element$>$Objekt
\ex\label{ex:1-3-14c} \isi{Subjekt}$>$finites Element$>$Objekt$>$\isi{Verb}
\end{xlist}
\end{exe}
Ob man die Stellung des Vollverbs nach dem Objekt als die \textsq{natürliche} Stellung
betrachtet oder nicht, ist für unseren Zusammenhang weitgehend unwichtig. Wesentlich ist die Erkenntnis, daß die einschlägigen Wortstellungsregularitäten nicht
auf der Unterscheidung "`Vollverb vs.\ \isi{Hilfsverb}"' beruhen, wie Greenberg suggeriert,
sondern auf der Unterscheidung "`finites vs.\ infinites \isi{Verb}"'. So~-- und mir scheint:
nur so~-- kann man begreifen, wie Sprecher des Deutschen\il{Deutsch} zu ihren Urteilen über Beispiele wie (\ref{ex:1-3-2}), (\ref{ex:1-3-3}), (\ref{ex:1-3-8})--(\ref{ex:1-3-11}) kommen.

Wenn wir nun alle finiten Verben durch "`\textsuperscript{f}V"' und alle infiniten Verben durch
"`\textsuperscript{i}V"' symbolisieren (unabhängig davon, ob sie Voll- oder Hilfsverben sind) und
unsere Beobachtungen über das Vorkommen von mehreren Hilfsverben in (\ref{ex:1-3-10})/""(\ref{ex:1-3-11}) berücksichtigen, können wir die dominante Wortstellung\is{Wortstellung!dominante} wie in (\ref{ex:1-3-15}) kennzeichnen:
\begin{exe}
\ex\label{ex:1-3-15}
S$>$\textsuperscript{f}V$>$O$>$(\textsuperscript{i}V)$^{\smallfrown}$(\textsuperscript{i}V)
\end{exe}
%page 18
Daß nach der Objektposition zweimal "`\textsuperscript{i}V"' in Klammern
steht, bedeutet, daß ein \isi{Deklarativsatz} entweder ohne ein infinites
\isi{Verb} auftritt oder mit 1 \textsuperscript{i}V oder mit 2
\textsuperscript{i}V. Dies trägt den bisher betrachteten Beispielen
Rechnung, ist aber überraschend: Warum sollten es grade maximal 2
infinite Verben sein, warum nicht mehr? Tatsächlich findet man
Beispiele mit 3 \textsuperscript{i}V:
\begin{exe}
\ex\label{ex:1-3-16}
\begin{xlist}
\ex\label{ex:1-3-16a}  Karl wird den Hund gefüttert haben wollen
\ex\label{ex:1-3-16b} Karl möchte Herrn Niemöller gekannt haben dürfen
\ex\label{ex:1-3-16c} Karl scheint herangebraust gekommen zu sein
\end{xlist}
\end{exe}
%\addlines[2]
Auch 4 \textsuperscript{i}V können auftreten:
\begin{exe}
\ex\label{ex:1-3-17}
\begin{xlist}
\ex\label{ex:1-3-17a}  Karl soll den Hund geschenkt bekommen haben wollen
\ex\label{ex:1-3-17b} Karl hätte befördert worden sein können
\end{xlist}
\end{exe}
Die Kombinationsmöglichkeiten sind keineswegs völlig frei. Aber es scheint, daß das
nicht an allgemeinen Baugesetzen des Satzes liegt, sondern an Eigenschaften der jeweils beteiligten Hilfsverben; je nachdem, welche Hilfsverben in welcher Reihenfolge kombiniert werden, ist das Ergebnis besser oder schlechter. Dies legt die Vermutung nahe, daß die Zahl der \textsuperscript{i}V im Grundsatz nicht beschränkt ist und tatsächlich
auftretende Beschränkungen als Ausfluß von lexikalischen Eigenschaften der Verben
zu verstehen sind.

Wenn wir dies voraussetzen, können wir (\ref{ex:1-3-15}) als (\ref{ex:1-3-18}) neu formulieren:
\begin{exe}
\ex\label{ex:1-3-18}
S$>$\textsuperscript{f}V$>$O$>$(\textsuperscript{i}V*)
\end{exe}
("`A*"' bedeutet: eine Folge von \emph{n} Elementen A, \emph{n}$>$0.) Dies ist eine erste Korrektur an
Greenbergs Behauptung, das Deutsche\il{Deutsch} habe S$>$V$>$O als dominante Wortstellung\is{Wortstellung!dominante}.

\ssubsubsection{}%3.1.3.
\label{subsubsec:1-3.1.3}

Der in (\ref{ex:1-3-18}) niedergelegten Theorie scheint sich ein
Bedenken entgegenzustellen. Beispiele wie in (\ref{ex:1-3-2-2}) sind
nach Herling radikal anders als bei Greenberg zu analysieren:
\begin{exe}
\exi{(21)}\makeatletter\def\@currentlabel{21}\label{ex:1-3-2-2}\makeatother
\begin{xlist}
\ex\label{ex:1-3-2a-2} der Hund erblickte einen Hasen
\ex\label{ex:1-3-2b-2} Karl brachte die Kartoffeln
\end{xlist}
\end{exe}
Nach Greenberg haben sie die Stellung S$>$V$>$O; nach (\ref{ex:1-3-18}) haben sie die Stellung
S$>$\textsuperscript{f}V$>$O. Wie kann ein Sprecher den Unterschied zwischen diesen beiden Analysen
wahrnehmen? Zweifellos sind ja \textit{erblickte} und \textit{brachte} Vollverben. Wie kann er zu
dem Schluß kommen, daß für Beispiele wie in (\ref{ex:1-3-2-2}) die Tatsache, daß dort zwischen
%page 19
\isi{Subjekt} und Objekt ein Vollverb steht, irrelevant ist (wenn wir voraussetzen, Greenbergs Analysekategorien für topologische Phänomene sind nicht grundsätzlich inadäquat)? Die Anwort ist einfach: Solange dem Sprecher nur Stimuli von Typ (\ref{ex:1-3-2}) zugänglich sind, kann er überhaupt nicht zu diesem Schluß gelangen. Im Deutschen\il{Deutsch}
werden jedoch Sätze mit Hilfsverben (temporale, modale, passivische Hilfsverben)
außerordentlich häufig verwendet; es kann als sicher gelten, daß Sätze vom Typ (\ref{ex:1-3-3-2})
mit S$>$\textsuperscript{f}V$>$O$>$\textsuperscript{i}V zur normalen Stimulusmenge eines
Sprachlerners gehören:\footnote{% 
  \citet{Park1981} bestreitet das: "`However, three German"=speaking children [\ldots] also preferred
  the verb"=final position to the medial one, although one mother never, and the other two mothers
  only exceptionally, produced some embedded and modal sentences"' (S.\,33). Diese Mütter dürften zu
  den ersten gehören, denen es gelungen ist, im kontinuierlichen Umgang mit Kindern Ausdrücke wie
  \emph{du kannst, du darfst, du sollst (nicht) \ldots} konsequent zu vermeiden. Die Annahme, daß
  die Kinder keine Stimuli der Form S$>$Aux$>$(O)$>$V kennengelernt haben, verlangt jedoch nicht nur
  die Annahme, daß die Mütter sich immer so verhalten haben, wie sie sich unter Parks Beobachtung
  verhalten haben, sondern darüber hinaus die Annahme, daß die Kinder ausschließlich solche
  Äußerungen als Stimuli auf"|fassen, die von ihren Müttern (und niemand sonst) an sie (und an
  niemand sonst) gerichtet werden, und daß die Mütter keinerlei nicht"=modale~-- auch keine
  temporalen~-- Hilfsverben verwendeten (obwohl \citeauthor{Park1981} bei einer anderen Mutter
  äußerst häufigen Gebrauch des Perfekts beobachtet hat (\citeyear[86]{Park1981})). Ich sehe keinen
  Grund, auch nur eine dieser Annahmen für glaubhaft zu halten.%
}
\begin{exe}
\exi{(22)}\makeatletter\def\@currentlabel{22}\label{ex:1-3-3-2}\makeatother
\begin{xlist}
\ex\label{ex:1-3-3a-2} der Hund hat einen Hasen erblickt
\ex\label{ex:1-3-3b-2} Karl soll die Kartoffeln bringen
\end{xlist}
\end{exe}
Wenn ein Sprachlerner derartige Sätze überhaupt versteht, versteht er
sie, indem er die Eigenschaften des Vollverbs (in ihren wesentlichen
Zügen) richtig analysiert. Dabei muß man nicht die unplausible Annahme
machen, daß verschiedene Kasusformen und ihre syntaktischen Funktionen
richtig erkannt werden; im typischen Fall reichen Verbsemantik und
situative Indizien aus, ein beträchtliches Maß an Verständigung zu
bewirken (und wenn solche Indizien nicht vorliegen, ist das korrekte
Verständnis der Sätze gewöhnlich auch nicht gesichert). In der Tat
werden solche Sätze verstanden, und in einem Stadium, in dem die
Kinder den Gebrauch von finiten und infiniten Verben nicht korrekt
beherrschen, produzieren sie~-- neben Äußerungen vom Typ
(S)$>$V$>$O~-- typischerweise solche vom Typ (S)$>$O$>$V. Dies ist
aufgrund von Sätzen wie (\ref{ex:1-3-2}) und (\ref{ex:1-3-3}) zu
erwarten; Englisch\il{Englisch} lernende Kinder produzieren dagegen nur ganz
ausnahmsweise solche Wortstellungen. (Zu einschlägigem Material
vgl.\ \citet{Park1981} und dort angegebene Literatur.)

Sobald ein Sprachlerner Hilfsverben und finite Verben analytisch
identifiziert, ist er zugleich im Stande, für Beispiele vom Typ
(\ref{ex:1-3-3}) eine Analyse gemäß (\ref{ex:1-3-18})
durchzuführen. Spätestens in diesem Stadium ist es möglich, Beispiele
vom Typ (\ref{ex:1-3-2}) ebenfalls gemäß (\ref{ex:1-3-18}) zu
analysieren.

Man kann und muß annehmen, daß der Sprachlerner auch aufgrund weiterer Indizien in der Stimulusmenge zu einer Analyse mit \textsuperscript{f}V statt V in zweiter Position
%page 20
geleitet wird; darauf kommen wir im folgenden Abschnitt zu sprechen.\footnote{%
  Ähnliches gilt für keltische\il{Keltisch} Sprachen. Beobachtungen und Erwägungen parallel zu denen, die wir für
  das Deutsche\il{Deutsch} angestellt haben, führen zu der Annahme, daß das Kymrische\il{Kymrisch} und das Irische\il{Irisch} (in
  uneingebetteten und in eingebetteten Sätzen) nach dem topologischen \isi{Schema}
  \textsuperscript{f}V$>$(S)$>$(\textsuperscript{i}V) aufgebaut sind (also nicht, wie allgemein
  angenommen wird, V$>$S$>$O"=Sprachen im Sinne von Greenberg sind); dabei hat offenbar das Kymrische\il{Kymrisch}
  S$>$\textsuperscript{i}V$>$O (vgl.\ \citealt{JonesThomas1977}), während das Irische\il{Irisch} zumindest in
  nördlichen Varianten S$>$O$>$\textsuperscript{i}V hat (vgl.\ Material in
  \citet{McCloskey1980}). Wenn ein Satz kein \isi{Hilfsverb} enthält, ist \textsuperscript{f}V ein
  Vollverb. Die Tatsache, daß die Position am Satzanfang nicht wesentlich eine V"=Position ist, ist
  für den Sprecher jedoch leicht zu erschließen, da diese Sprachen sehr reichen Gebrauch von
  sog.\ verb nouns machen, in denen das Vollverb in einer infiniten Form auftritt.\label{fn:10}%
}

\ssubsection{Subjekt und Objekt}%3.2.
\label{subsec:1-3.2}

\ssubsubsection{}%3.2.1.
\label{subsubsec:1-3.2.1}

Nach (\ref{ex:1-3-18}) weisen die Beispiele in (\ref{ex:1-3-2}) die \isi{Wortstellung} S$>$\textsuperscript{f}V$>$O auf. Wir finden
aber auch Beispiele wie (\ref{ex:1-3-19}):
\begin{exe}
\ex\label{ex:1-3-19}
\begin{xlist}
\ex\label{ex:1-3-19a} da erblickte der Hund einen Hasen
\ex\label{ex:1-3-19b} vielleicht bringt Karl die Kartoffeln
\end{xlist}
\end{exe}
Nichts spricht dafür, dies nicht für eine dominante Wortstellung\is{Wortstellung!dominante} zu halten; die Sätze
sind jedenfalls stilistisch neutral und intuitiv völlig normal. Nach Greenberg hätten
sie die Stellung V$>$S$>$O; korrigiert und expliziter notieren wir das als (\ref{ex:1-3-20}):
\begin{exe}
\ex\label{ex:1-3-20}
Adverbial$>$\textsuperscript{f}V$>$S$>$O
\end{exe}
Hier ist eine ähnliche Schwierigkeit wie bei den Hilfsverben: Nach
Greenbergs Charakterisierung des Deutschen\il{Deutsch} steht das \isi{Subjekt} mal vor
dem \isi{Verb}, mal danach. Gleichwohl ist die \isi{Wortstellung} nicht frei
(\textsq{labile}). Offenbar herrschen strenge Beschränkungen, denn
Beispiele wie (\ref{ex:1-3-21}), die nach Greenberg S$>$V$>$O"=Stellung
aufweisen, werden von Sprechern des Deutschen\il{Deutsch} als abweichend
empfunden,\footnote{\label{fn:1-11}%
  Greenberg hat die Voranstellung von Adverbialen\is{Adverbiale}
  wie auch das Vorkommen von Aux$>$S$>$V bemerkt, aber was er dazu
  schreibt, ist unverständlich:
  \begin{quote}
    "`In general the initial position is the emphatic one, and while there are other methods of emphasis
    (e.g., stress), the initial position always seems to be left free so that an element to which attention is
    directed may come first. [\ldots] It seems probable that in all languages expressions of time and place
    may appear in the initial positions in the sentence.
    
    \hspace{1em} The discontinuity of the predicate, which commonly appears in such instances (e.g., German,
    \textsqe{Gestern ist mein Vater nach Berlin gefahren}), illustrates a further principle."' \citep[103f]{Greenberg1963}
  \end{quote}
  Durch "`in such instances"' wird dieses deutsche Beispiel mit einer in allen Sprachen möglichen
  Voranstellung von temporalen Adverbialen\is{Adverbiale} verknüpft. Aber die \textsq{Diskontinuität des Prädikats}
  (\textit{ist \ldots{} gefahren}) hat nichts mit dem Adverbial zu tun: Wir haben auch \textit{mein
    Vater ist (gestern) nach Berlin gefahren} ohne voranstehendes Adverb, aber mit Diskontinuität
  (vgl.\ (\ref{ex:1-3-3a})); und wir haben (\ref{ex:1-3-19}) mit voranstehendem Adverb, aber ohne
  Diskontinuität. Es ist auch keiner dieser Sätze notwendig mit irgendeiner Art von Emphase verbunden.

  In (\ref{ex:1-3-19}) herrscht zudem, wie wir gesehen haben, V$>$S$>$O"=Stellung; etwas, was es in
  Sprachen mit zweifelsfreier dominanter S$>$V$>$O"=Stellung bei voranstehendem Adverbial sonst
  anscheinend nicht gibt und was im Deutschen\il{Deutsch} für Fragesätze typisch sein soll;
  vgl.\ (\ref{ex:1-3-6}). Es mag sinnvoll sein, zu sagen, die Position vor dem \isi{Subjekt} in
  \textit{yesterday I saw him at the station} sei "`left free"'; aber hinsichtlich der Stellung des
  Verbs gilt das im Deutschen\il{Deutsch} offensichtlich nicht, wie man an (\ref{ex:1-3-21}) sieht.%
} und
%page 21
V$>$S$>$O"=Sätze wie (\ref{ex:1-3-22}) sind keine Deklarativsätze:\footnote{\label{fn:1-12}%
  Aber es gibt Deklarativsätze von dieser Form, vgl.\ X [s.\ Anm.\ S.\,\pageref{fn-herausgeber-topo}].%
}
\begin{exe}
\ex\label{ex:1-3-21}
\begin{xlist}
\ex\label{ex:1-3-21a} da der Hund erblickte einen Hasen
\ex\label{ex:1-3-21b} vielleicht Karl bringt die Kartoffeln
\end{xlist}
\end{exe}
\begin{exe}
\ex\label{ex:1-3-22}
\begin{xlist}
\ex\label{ex:1-3-22a} erblickte der Hund da einen Hasen
\ex\label{ex:1-3-22b} brachte Karl vielleicht die Kartoffeln
\end{xlist}
\end{exe}
Unter Greenbergs Analyse erscheint das erneut als chaotisch.\footnote{%
  Auch solche Fälle haben anscheinend zu der Ansicht beigetragen, das Deutsche\il{Deutsch} sei ein
  unzivilisiertes Idiom. Das Unverständnis für die Stellung Adverbial$>$\textsuperscript{f}V$>$S$>$O
  wird weiter dadurch vermehrt, daß bei Sprechern von S$>$V$>$O"=Sprachen die Meinung vorzuherrschen
  scheint, jedes Nicht"=\isi{Subjekt} am Anfang eines Satzes müsse \textsq{emphatisch} sein; vgl.\ das
  Greenberg"=Zitat in Fn.\,\ref{fn:1-11}. (Allerdings findet sich dasselbe Fehlurteil auch bei Kennern des
  Deutschen\il{Deutsch}, \zb \citet[561]{Adelung1782}.)%
}
Der Sprecher des
Deutschen\il{Deutsch} ist über solche Beispiele jedoch nie instruiert worden, und wenn er ein
Urteil darüber fällt, muß er das aufgrund spezifischer Regeln tun. Eine Analyse,
nach der die Verhältnisse in (\ref{ex:1-3-2}), (\ref{ex:1-3-19}), (\ref{ex:1-3-21}), (\ref{ex:1-3-22}) chaotisch erscheinen, muß deshalb
falsch sein. Wenn man die Sätze in (\ref{ex:1-3-2}) mit Greenberg als S$>$V$>$O"=Stellungen und die
in (\ref{ex:1-3-19}) als V$>$S$>$O"=Stellungen zu begreifen versucht und gleichzeitig unterstellt, daß
die Position von V (dem Hauptverb) ein Parameter ist, der zur Charakterisierung
von Deklarativsätzen wesentlich beiträgt, wäre nicht recht vorstellbar, wie (a) ein
Sprecher zu Urteilen über unakzeptable Wortstellungen kommen könnte, und worin
(b) die charakterisierende Leistung von Vollverben bestehen sollte.

Da wir aber schon gesehen haben, daß in (\ref{ex:1-3-2}) nicht
S$>$V$>$O"=Stellung, sondern S$>$\textsuperscript{f}V$>$O"=Stellung
vorliegt, gibt es zur Verblüffung gar keinen Anlaß. Es spricht nichts
dafür, daß der Position von \textsuperscript{f}V~-- dem Träger der
Kategorie Finit~-- relativ zum \isi{Subjekt} allgemein besondere Relevanz
zukommt; Greenberg behauptet jedenfalls nichts
dergleichen.\footnote{%
  Er sagt nur über finite Hilfsverben etwas: In
  Sprachen, die zweifelsfrei dominante S$>$O$>$V"=Stellung haben,
  stehen sie nach dem V \citep[84]{Greenberg1963}.~-- Herling
  nimmt an, daß \textsuperscript{f}V (die \textsq{Aussage}) naturnotwendig
  zwischen \isi{Subjekt} und \isi{Prädikat} stehen muß;
  vgl.\ (\ref{ex:1-3-12a}). Diese Annahme zwingt zur Aufstellung einer
  komplizierten und schwer überprüfbaren Theorie von "`Inversionen"' für
  Fälle wie in (\ref{ex:1-3-19}) und führt zu Schwierigkeiten bei
  Sätzen mit \isi{Endstellung} des \textsuperscript{f}V (\zb \emph{daß der
  Hund einen Hasen erblickte \ldots}). Herling deutet das so, daß
  Nebensätze nicht den Charakter von Sätzen, sondern den von Begriffen
  haben und mit Nominalisierungen zu vergleichen sind; darum soll das,
  was im Hauptsatz \isi{Subjekt} ist, bei ihnen dann die Funktion einer
  äußeren \textsq{Nebenbestimmung} des (quasi"=nominalisierten) Prädikats
  haben (\citeyear*[319ff]{Herling1821T}; \citeyear[89f]{Herling1830}). Die Bedenken gegen
  diese Theorie liegen auf der Hand.%
}
Es ist daher gar nicht erstaunlich, daß wir nebeneinander S$>$\textsuperscript{f}V$>$O und
Adverbial$>$\textsuperscript{f}V$>$S$>$O finden. Die naheliegende
Interpretation ist, daß in Sätzen dieser Art (a) \textsuperscript{f}V
allgemein die zweite Position im Satz einnimmt und (b) das Element in
der
%page 22
ersten Position hinsichtlich seiner syntaktischen Funktion im Satz frei ist: Es kann~-- mindestens~-- ein adverbialer Ausdruck\is{Adverbiale} oder ein \isi{Subjekt} sein. Wir kommen also zu
dem Schluß, daß die dominante Wortstellung\is{Wortstellung!dominante} durch ein \isi{Schema} wie (\ref{ex:1-3-23}) zu kennzeichnen ist; dabei symbolisiert "`\textit{K}"' die funktional unbestimmte erste Position:
\begin{exe}
\ex\label{ex:1-3-23}
\textit{K}$^{\smallfrown}$\textsuperscript{f}V$>$(S)$>$O$>$(\textsuperscript{i}V*)
\end{exe}
(Man beachte: Wir betrachten nach wie vor nur Sätze, die \isi{Subjekt} und Objekt enthalten. Daß "`S"' in (\ref{ex:1-3-23}) eingeklammert ist, bedeutet hier also: In dieser Position kann ein
\isi{Subjekt} stehen; wenn es dort nicht steht, kann es nur in der ersten Position \textit{K} stehen.)
Dies ist eine zweite Korrektur an Greenbergs Behauptung, das Deutsche\il{Deutsch} habe
S$>$V$>$O als dominante Wortstellung\is{Wortstellung!dominante}.

\ssubsubsection{}%3.2.2.
\label{subsubsec:1-3.2.2}

Nach (\ref{ex:1-3-23}) haben die Sätze in (\ref{ex:1-3-2}) also nicht
die Form S$>$V$>$O, sondern die Form
\textit{K}$^{\smallfrown}$\textsuperscript{f}V$>$O. Fragen wir nun wieder, was einen
Sprachlerner zu dieser Analyse führen kann. Es ist wie bei unserer
ersten Korrektur (\ref{ex:1-3-18}). Wenn die Stimulusmenge nur Sätze
vom Typ (\ref{ex:1-3-2}) enthalten würde, müßte man erwarten, daß der
Sprachlerner ein System entwickelt, das mit dem des Englischen\il{Englisch}
weitgehend identisch ist. Aufgrund der \isi{Spracherwerbsdaten} für das
Englische\il{Englisch} muß man annehmen, daß der Sprachlerner S$>$V$>$O"=Stellungen
frühzeitig identifiziert, wenn diese Stellungen \textsq{kontrastiv}
(Harris, vgl.\ (\ref{ex:1-1-1a})) bzw.\ \textsq{grammatikalisiert}
(Dane\v{s}, vgl.\ (\ref{ex:1-1-2a}))~-- also auch stabil~-- sind, wie
es im Englischen der Fall ist; Englisch\il{Englisch} lernende Kinder produzieren
praktisch keine anderen Wortfolgen als (S)$>$(V)$>$(O). Aber im
Deutschen\il{Deutsch} werden Sätze wie (\ref{ex:1-3-19}) häufig verwendet (und
außerdem Sätze von einem verwandten Typ, den wir in X [s.\ Anm.\ S.\,\pageref{fn-herausgeber-topo}] besprechen) und gehören zur normalen Stimulusmenge jedes
Sprachlerners. Sofern Äußerungen dieses Typs überhaupt verstanden
werden und die \isi{Position des Verbs} relativ zum \isi{Subjekt} wahrgenommen
wird, erzwingen sie die Folgerung, daß die Position von V gegenüber S
im Deutschen\il{Deutsch} nicht stabil, mithin auch nicht \textsq{kontrastiv}
bzw.\ \textsq{grammatikalisiert} ist. Tatsächlich werden solche
Äußerungen ohne Probleme verstanden; und Deutsch\il{Deutsch} lernende Kinder
produzieren u.\,a.\ deklarative Äußerungen mit V$>$S$>$(O)"=Stellung,
wie man es aufgrund von Sätzen wie (\ref{ex:1-3-19})
erwartet.\footnote{\label{fn:1-15}%
  \citet{Park1981} meint, daß V$>$S$>$O"=Stellungen~--
  also deklarative Sätze wie (\ref{ex:1-3-19})~-- im Deutschen\il{Deutsch}
  ungrammatisch sind. Da er außerdem glaubt, daß S$>$O$>$V"=Stellungen
  in der Stimulusmenge der Sprachlerner nicht vorkommen, aber
  u.\,a.\ V$>$S$>$(O)- und (S)$>$O$>$V"=Stellungen bei Kindern
  beobachtet, sieht er sich zu einer Spracherwerbstheorie veranlaßt,
  die in wesentlichen Teilen unverständlich und/""oder inkonsistent
  ist. (Gleiche Kritik hat \citet{Klein1974}~-- aufgrund gleicher
  Überlegungen und Beobachtungen an niederländischem\il{Niederländisch} Material~-- an
  Park geübt.)%
} 
Es gibt deshalb keinen Grund, anzunehmen, daß der
Sprachlerner jemals die Stellung von V (dem Vollverb) relativ zu S als
relevant interpretiert; sobald finite Verben als solche identifiziert
werden, ist daher~-- unabhängig von der Identifizierung der
Hilfsverben)~-- die Analyse \textit{K}$^{\smallfrown}$\textsuperscript{f}V$>$(S)$>$O
möglich.

%page 23
Der Sprachlerner verfügt also über mindestens zwei verschiedene Arten
von Evidenz, die ihn dahin führen, die beiden ersten Positionen von
Deklarativsätzen nicht als S$>$V, sondern als \textit{K}$^{\smallfrown}$\textsuperscript{f}V zu analysieren. Aus Sätzen wie (\ref{ex:1-3-2})
einerseits und wie (\ref{ex:1-3-19}) mit voranstehendem \isi{Adverbiale}
o.\,ä.\ andererseits muß er schließen, daß (i)~vor dem (finiten) \isi{Verb}
eine Position ist, die funktional unbestimmt ist, und daß (ii)~die
Stellung von V zu S irrelevant ist (S steht nicht immer vor V). Aus
Sätzen wie (\ref{ex:1-3-3}) mit \isi{Hilfsverb} muß er den parallelen Schluß
ziehen, daß (i)~die Stellung von V zu O irrelevant ist (V steht nicht
immer vor O) und daß (ii)~V nicht an die zweite Position im Satz
gebunden ist. Sobald Finitheit als Kategorie analytisch verfügbar ist,
muß dies zur Analyse (\ref{ex:1-3-23}) führen.\footnote{%
  Aufgrund
  gleicher Erwägungen wie im Deutschen\il{Deutsch} muß man für das Norwegische\il{Norwegisch} in
  uneingebetteten Deklarativsätzen und in einem Teil von konjunktional
  eingebetteten Sätzen das topologische \isi{Schema} \textit{K}$^{\smallfrown}$\textsuperscript{f}V $>$(S)$>$(\textsuperscript{i}V)$>$(O)
  annehmen. Andere Konjunktionalsätze folgen dem \isi{Schema}
  S$>$\textsuperscript{f}V$>$(\textsuperscript{i}V)$>$O.  (Zum
  Material vgl.\ z.\,B.\ \citet{Faarlund1981}.) Im Schwedischen\il{Schwedisch} sind
  die Verhältnisse ganz ähnlich (vgl.\ \citealt{Andersson1975}); ebenso
  im Dänischen\il{Dänisch}. Auch für das Isländische\il{Isländisch} dürfte das gleiche \isi{Schema}
  gelten (vgl.\ \citealt{MalingZaenen1978,MalingZaenen1981}); die
  Interpretation der Fakten wird allerdings durch gewisse
  Nachstellungen des Subjekts erschwert; vgl.\ \citet{Maling1980}.\label{fn:16}%
}

\ssubsubsection{}%3.2.3.
\label{subsubsec:1-3.2.3}

Fassen wir kurz zusammen. Wenn, gemäß Greenberg, die Kategorie Vollverb
für die topologischen Beziehungen in Deklarativsätzen relevant wäre, würde sich ein
wildes Chaos ergeben. Wir hätten S$>$O$>$V: \textit{Karl hat die Kartoffeln gebracht}, \textit{vielleicht hat Karl die Kartoffeln gebracht}; S$>$V$>$O: \textit{Karl brachte die Kartoffeln}; V$>$S$>$O: \textit{vielleicht brachte Karl die Kartoffeln}. Für den Sprecher des Deutschen\il{Deutsch} herrscht hier jedoch Ordnung: Er hat ein klares Urteil über Beispiele, über die er nie
instruiert worden ist, und beurteilt gewisse Fälle als strikt abweichend, \dash er verfährt nach \textsq{starken Regeln} (genauer: nach \textsq{concomitant rules}, vgl.\ (\ref{ex:1-1-2b})) im Sinne
von Dane\v{s}. Das \isi{Schema} (\ref{ex:1-3-23}) ist eine Repräsentation dieses Regelwissens.

Dieses \isi{Schema} bewährt sich bei Beispielen wie (\ref{ex:1-3-24}), wenn wir von der Spezifikation von O in (\ref{ex:1-3-23}) absehen:
\begin{exe}
\ex\label{ex:1-3-24}
\begin{xlist}
\ex\label{ex:1-3-24a} Topikalisierungen verwende ich nie
\ex\label{ex:1-3-24b} mich hat ein Hund gebissen
\end{xlist}
\end{exe}
Die Sätze haben die Folge O$>$V$>$S bzw.\ O$>$S$>$V, sind ansonsten aber gemäß (\ref{ex:1-3-23}) aufgebaut, und Stellungsvarianten von (\ref{ex:1-3-24}), die nicht im Einklang mit (\ref{ex:1-3-23}) stehen, sind keine deklarativen Sätze oder unakzeptabel. Dabei könnte (\ref{ex:1-3-24b}) eine dominante (stilistisch neutrale) Stellung sein; der Satz muß, wenn \textit{Hund} betont ist, nicht irgendwie \textsq{emphatisch} verstanden werden. Für (\ref{ex:1-3-24a}) gibt es dagegen gute Gründe, diese
Stellung nicht als neutral zu betrachten; vgl.\ X [s.\ Anm.\ S.\,\pageref{fn-herausgeber-topo}].\label{X:5} Das ist sehr beachtlich, denn aufgrund allgemeiner Erfahrung ist man darauf gefaßt, bei nicht"=neutralen Sätzen Abweichungen von der dominanten \isi{Wortstellung} zu finden; so etwa im englischen\il{Englisch} 
%page 24
Gegenstück zu (\ref{ex:1-3-24a}) mit O$>$S$>$V"=Stellung:
\begin{exe}
\ex\label{ex:1-3-25}
topicalization I never use\footnote{%
  Dieses Beispiel verdanke ich Craig Thiersch. Dem Gerücht nach ist es eine spontane und unironische
  Äußerung eines Sprachwissenschaftlers.%
}
\end{exe}
Nicht"=neutrale Stellungen folgen im Deutschen\il{Deutsch} also demselben topologischen \isi{Schema} wie neutrale Stellungen. (\ref{ex:1-3-23}) hat mit der Charakterisierung von dominanten (stilistisch neutralen) Wortstellungen relativ zu \textsq{markierten}, \textsq{nicht"=neutralen} Stellungen gar nicht direkt zu tun.

Dies ist eine folgenreiche Einsicht. Greenberg ist im Recht, wenn er meint, daß
S$>$V$>$O (jedenfalls in zahlreichen Einzelfällen) eine stilistisch neutrale (dominante)
Stellung ist. Diese Charakterisierung ist ihrem Begriff nach~-- in den Fällen, wo es
"`several variant orders"' entsprechend (\ref{ex:1-2-1b}) gibt~-- Ausdruck einer schwachen Regel.
Über die im Deutschen\il{Deutsch} wirksamen starken Regeln sagt diese Beobachtung überhaupt nichts. Zweifellos gibt es Interaktionen zwischen starken und schwachen Regeln, schon deshalb, weil schwache Regeln definitionsgemäß nur innerhalb des Freiraums operieren können, den ihnen die starken Regeln lassen. Aber man kann aus
der Beobachtung einer schwachen Regularität (wie der Tatsache, daß S$>$V$>$O oft stilistisch neutral ist) allenfalls sehr indirekt und unter Voraussetzung vieler hochtheoretischer Annahmen Schlüsse über starke Regeln ziehen. (\ref{ex:1-3-23}) drückt hinsichtlich der
Beziehungen zwischen \textit{K}, \textsuperscript{f}V und \textsuperscript{i}V eine starke Regel (oder: den Effekt von starken
Regeln) aus. Eine Reihe von Realisierungen dieses Schemas (mit V$>$S$>$O, S$>$V$>$O,
S$>$O$>$V, oder auch O$>$S$>$V) sind stilistisch neutral, andere sind es nicht. Dies muß
auf die Wirkung von schwachen Regeln zurückgehen.

\ssubsubsection{}%3.2.4.
\label{subsubsec:1-3.2.4}

Greenberg beschränkt sich für seine typologischen Zwecke weitgehend auf
die Betrachtung von Sätzen, die \isi{Subjekt} und Objekt enthalten. Wie sind Sätze aufgebaut, die kein Objekt enthalten? Die minimale Annahme ist, daß sie sich von Sätzen
mit Objekt nur durch die Abwesenheit des Objekts (und durch das Vorkommen anderer Prädikate) unterscheiden; dies scheint im Deutschen\il{Deutsch} der Fall zu sein. Sie folgen
dem \isi{Schema} (\ref{ex:1-3-23}), wenn wir das "`O"' in (\ref{ex:1-3-23}) als nicht"=obligatorisch betrachten. (Dies
haben wir bereits bei Fällen wie (\ref{ex:1-3-24}) getan.) Unter das derart verallgemeinerte \isi{Schema} (\ref{ex:1-3-26}) fällt \zb auch Greenbergs Beispiel (Fn.\,\ref{fn:1-11}) \emph{gestern ist mein Vater nach Berlin gefahren}:
\begin{exe}
\ex\label{ex:1-3-26}
\textit{K}$^{\smallfrown}$\textsuperscript{f}V$>$(S)$>$(O)$>$(\textsuperscript{i}V*)
\end{exe}
Es gibt im Deutschen\il{Deutsch} Prädikate, die (fakultativ oder obligatorisch) ohne \isi{Subjekt}
vorkommen. Im Aktiv ist ihre Anzahl beschränkt (\textit{jemandem wird schlecht, jemandem graust vor etwas, jemandem liegt (sehr) an etwas} und einige andere), 
%page 25
aber es gibt viele Fälle im Passiv (\textit{jemandem wird geholfen, jemandem ist zu
trauen, irgendwo wird gearbeitet, einer Sache läßt sich abhelfen} usw.). Die
minimale Annahme ist wieder, daß sie sich von Sätzen mit \isi{Subjekt} nur durch die
Abwesenheit des Subjekts (und durch andere Prädikate) unterscheiden; dies scheint
wiederum weitestgehend der Fall zu sein. Man beachte dabei: Eine Theorie, die die
topologischen Eigenschaften von Verben wesentlich in Hinsicht auf das \isi{Subjekt} beschreibt, kann solche subjektlosen Sätze überhaupt nicht erfassen. Insofern ist es relevant, daß sie alle dem \isi{Schema} (\ref{ex:1-3-26}) folgen.

In (\ref{ex:1-3-26}) ist für dominante Wortstellungen die Folge S$>$O angegeben. Aber auch die
Folge O$>$S scheint in manchen Fällen stilistisch neutral zu sein; etwa in (\ref{ex:1-3-27}):
\begin{exe}
\ex\label{ex:1-3-27}
\begin{xlist}
\ex\label{ex:1-3-27a}  vielleicht können einen Physiker Experimente überzeugen
\ex\label{ex:1-3-27b} leider ist dem Meister der Lehrvertrag eingefallen
\end{xlist}
\end{exe}
Jedenfalls sehe ich nicht, daß solche Sätze notwendig irgendwie nicht"=neutral sein
müssen. Dies sind keine vereinzelten Beispiele; für eine ganze Reihe von Prädikaten
gilt, daß die Stellung \textsuperscript{f}V$>$O$>$S nicht generell als stilistisch \textsq{markiert} empfunden
wird. Naturgemäß erhebt sich spätestens hier das Bedürfnis, die Begriffe \textsq{stilistisch neutral
  (normal)} und \textsq{dominant} präzise zu explizieren, damit man sich über derartige intuitive
Beurteilungen in kontrollierter Weise verständigen kann. Dieser Aufgabe wenden wir uns in X
[s.\ Anm.\ S.\,\pageref{fn-herausgeber-topo}]\label{X:6} zu. Man kann aber schon hier mit Gewißheit
voraussetzen: Ob eine bestimmte \isi{Wortstellung} als stilistisch neutral empfunden wird oder nicht,
hängt nicht allein von der syntaktischen Funktion der beteiligten Elemente ab, sondern in hohem Maße
auch von~-- vermutlich semantischen~-- Eigenschaften des jeweiligen Prädikats. 

Das bedeutet, daß es kein allgemeines syntaktisches \isi{Schema} gibt, das ausschließlich stilistisch neutrale Stellungen erfaßt. Wenn wir nach einem allgemeinen topologischen \isi{Schema} für Deklarativsätze des Deutschen\il{Deutsch} suchen, können \isi{Subjekt} und Objekt darin also aus mehreren Gründen nicht auftreten: Es gibt Sätze, die kein \isi{Subjekt},
kein Objekt, oder weder \isi{Subjekt} noch Objekt enthalten; \isi{Subjekt} wie Objekt (und andere Satzbestandteile, etwa adverbiale Ausdrücke) können vor oder nach dem finiten \isi{Verb} auftreten; selbst wenn \isi{Subjekt} und Objekt beide nach dem finiten \isi{Verb} stehen, kann man ihre stilistisch \isi{neutrale Abfolge} nicht allgemein, sondern nur im Verhältnis zu bestimmten Prädikatstypen bestimmen. Das allgemeine topologische
\isi{Schema} für Deklarativsätze muß deshalb auf die Angabe von \isi{Subjekt} und Objekt
verzichten.

Diese Überlegungen führen uns zu der Formulierung (\ref{ex:1-3-28}):
\begin{exe}
\ex\label{ex:1-3-28}
\textit{K}$^{\smallfrown}$\textsuperscript{f}V $>$(\textsuperscript{i}V*)
\end{exe}
Damit stärker ins Auge fällt, daß zwischen \textsuperscript{f}V und den~-- fakultativen~-- \textsuperscript{i}V beliebig
%page 26
viele (auch null) Elemente von beliebiger Art stehen können, will ich in diesem Bereich ein Zeichen "`\textit{KM}"' verwenden und (\ref{ex:1-3-28}) durch (\ref{ex:1-3-29}) ersetzen:
\begin{exe}
\ex\label{ex:1-3-29}
\textit{K}$^{\smallfrown}$\textsuperscript{f}V$>$(\textit{KM}*)$>$(\textsuperscript{i}V*)
\end{exe}
Dies ist~-- im wesentlichen; einige Präzisierungen tragen wir später nach~-- das allgemeine topologische \isi{Schema} für Deklarativsätze.

\section{Interrogativsätze}%4.
\label{sec:1-4}

\ssubsection{}%4.1.
\label{subsec:1-4.1}
Wir haben in (\ref{ex:1-3-6}) schon kurz zwei Typen von Fragesätzen des Deutschen\il{Deutsch} berührt und
gesehen, wie Greenberg sie analysiert: Ergänzungsfragen wie \textit{wen sah er?} haben O$>$V$>$S"=Stellung, Entscheidungsfragen wie \textit{kommt er?} haben V$>$S"=Stellung. Unsere bisherigen Überlegungen haben uns gezeigt, daß die topologische Beziehung zwischen V (dem Vollverb), S und O irrelevant ist und daß es auf die Beziehung zwischen \textsuperscript{f}V, \textsuperscript{i}V und Elementen beliebiger anderer Art ankommt. Es wäre ganz
überraschend, wenn Fragesätze sich da anders verhalten würden. Tatsächlich finden
wir Ergänzungsfragen wie (\ref{ex:1-4-1}):
\begin{exe}
\ex\label{ex:1-4-1}
\begin{xlist}
\ex\label{ex:1-4-1a} wen hat er gesehen?
\ex\label{ex:1-4-1b} wem liegt an ihrem Erfolg?
\ex\label{ex:1-4-1c} wieso sollte den Wählern diese Politik gefallen haben?
\ex\label{ex:1-4-1d} wer hat den Mißerfolg zu verantworten?
\ex\label{ex:1-4-1e} wo wird gearbeitet?
\end{xlist}
\end{exe}
Satz (\ref{ex:1-4-1a}) hat O$>$S$>$V"=Stellung, (\ref{ex:1-4-1b})
enthält kein \isi{Subjekt} und hat O$>$V"=Stellung, (\ref{ex:1-4-1c}) hat
O$>$S$>$V"=Stellung, (\ref{ex:1-4-1d}) hat S$>$O$>$V"=Stellung, und
(\ref{ex:1-4-1e}) enthält weder ein \isi{Subjekt} noch ein Objekt~-- das
bekannte verwirrende Bild, das sich unter Greenbergs Analyse ergibt
und das in genau derselben Form schon bei Deklarativsätzen zu
beobachten war. Man sieht sofort, daß alle Sätze in (\ref{ex:1-4-1})
dem \isi{Schema} (\ref{ex:1-3-29}) genügen. Ergänzungsfragesätze
unterscheiden sich von Deklarativsätzen topologisch allein dadurch,
daß sie in der Position \textit{K} ein Element haben, das ein
\isi{Interrogativpronomen} enthält.

Bei Entscheidungsfragen ist das offensichtlich anders; vor dem finiten \isi{Verb} steht
kein Element. Ansonsten finden wir aber das vertraute Bild, daß sie keineswegs, wie
Greenberg meint, durch V$>$S"=Stellung charakterisiert sind:
\begin{exe}
\ex\label{ex:1-4-2}
\begin{xlist}
\ex\label{ex:1-4-2a} hat Karl die Kartoffeln gebracht?
\ex\label{ex:1-4-2b} liegt euch an ihrem Erfolg?
\ex\label{ex:1-4-2c} graust dir?
\ex\label{ex:1-4-2d} sollte den Wählern die Partei gefallen haben?
\ex\label{ex:1-4-2e} wird gearbeitet?
\end{xlist}
\end{exe}
%page 27
Vielmehr hat (\ref{ex:1-4-2a}) S$>$O$>$V"=Stellung; (\ref{ex:1-4-2b}) enthält kein \isi{Subjekt} und hat V$>$O"=Stellung;
ebenso (\ref{ex:1-4-2c}); (\ref{ex:1-4-2d}) hat O$>$S$>$V"=Stellung; und (\ref{ex:1-4-2e}) enthält weder \isi{Subjekt} noch Objekt. Das
allgemeine topologische \isi{Schema} für Entscheidungsfragen ist also (\ref{ex:1-4-3}); es unterscheidet sich von (\ref{ex:1-3-29}) nur dadurch, daß die Position \textit{K} vor \textsuperscript{f}V fehlt:
\begin{exe}
\ex\label{ex:1-4-3}
\textsuperscript{f}V$>$(\textit{KM}*)$>$(\textsuperscript{i}V*)
\end{exe}
Sätze, die dem \isi{Schema} (\ref{ex:1-3-29}) genügen, haben, so sage ich, F2"=Stellung (oder sind
F2"=Sätze); Sätze, die dem \isi{Schema} (\ref{ex:1-4-3}) genügen, haben F1"=Stellung (oder sind F1"=Sätze). Das "`F"' soll dabei gleichzeitig an "`finit"' und an "`frontal"' erinnern.
\ssubsection{}%4.2.
\label{subsec:1-4.2}
Man kann in einem gewissen Maß auch F2"=Sätze benutzen, um Fragen zu stellen, auf die man die Antwort \textit{ja} oder \textit{nein} erwartet; gewöhnlich sind die dann
durch eine bestimmte \isi{Intonation} gekennzeichnet, die auch für Fragesätze mit F1"=Stellung typisch ist:
\begin{exe}
\ex\label{ex:1-4-4}
\begin{xlist}
\ex\label{ex:1-4-4a} Karl hat die Kartoffeln gebracht?
\ex\label{ex:1-4-4b} dir liegt an ihrem Erfolg?
\end{xlist}
\end{exe}
Zwischen solchen F2"=Sätzen und entsprechenden F1"=Sätzen besteht intuitiv ein gewisser Unterschied, der jedoch schwer zu fassen ist und noch nie befriedigend expliziert worden ist. (Z.\,B.\ dienen Beispiele wie (\ref{ex:1-4-4}) keineswegs immer als \textsq{Echo"=Fragen}~–
was immer mit diesem Ausdruck gemeint ist.) Man könnte vermuten, daß F1"=Sätze
und F2"=Sätze einfach freie Varianten voneinander sind. Das ist jedoch nicht der Fall.

Wir haben schon bei (\ref{ex:1-3-22}) bemerkt, daß F1"=Sätze gewöhnlich nicht als Deklarativsätze verwendet werden können; umgekehrt können Fragesätze mit F1"=Stellung Partikeln wie \emph{etwa} und \emph{denn} enthalten, die in entsprechenden F2"=Sätzen nicht vorkommen können:

\begin{exe}
\ex
\label{ex:1-4-5}
\begin{xlist}
\ex[]{
\label{ex:1-4-5a}
hat Karl etwa die Kartoffeln gebracht?}
\ex[]{
\label{ex:1-4-5b}
liegt dir denn an ihrem Erfolg?}
\end{xlist}

\ex\label{ex:1-4-6}
\begin{xlist}
\ex[*]{
\label{ex:1-4-6a}
Karl hat etwa die Kartoffeln gebracht?}
\ex[*]{
\label{ex:1-4-6b}
dir liegt denn an ihrem Erfolg?}
\end{xlist}
\end{exe}
%\addlines[-1]
Um den Unterschied zwischen (\ref{ex:1-4-4}) und (\ref{ex:1-4-5})
terminologisch zu erfassen, bezeichne ich Sätze wie
(\ref{ex:1-4-5})~-- die F1"=Stellung haben~-- als direkte
(Entscheidungs-){\allowbreak}Interrogativsätze. Sätze wie
(\ref{ex:1-4-4})~-- die F2"=Stellung haben~-- sind dagegen weder
Interrogativsätze noch Deklarativsätze.

Eine ähnliche Unterscheidung wie zwischen (\ref{ex:1-4-4}) und
(\ref{ex:1-4-5}) kann man bei Ergänzungsfragen beobachten. In
Ergänzungsfragen wie (\ref{ex:1-4-1}) kann man wie bei
(\ref{ex:1-4-5}) \textit{denn} einsetzen:
\begin{exe}
\ex\label{ex:1-4-7}
%page 28
\begin{xlist}
\ex\label{ex:1-4-7a} wen hat er denn gesehen?
\ex\label{ex:1-4-7b} wo wird denn morgen gearbeitet?
\end{xlist}
\end{exe}
Ergänzungsfragen müssen das \isi{Interrogativpronomen} nicht in der Position \textit{K} haben;
man kann sie auch wie in (\ref{ex:1-4-8}) formulieren, und dies hat einen ähnlichen Effekt wie
die Fragesätze in (\ref{ex:1-4-4}):
\begin{exe}
\ex\label{ex:1-4-8}
\begin{xlist}
\ex\label{ex:1-4-8a} er hat w\underline{\underline{e}}n gesehen?
\ex\label{ex:1-4-8b} morgen wird w\underline{\underline{o}} gearbeitet?
\end{xlist}
\end{exe}
In solche Sätze kann man aber, so wie bei (\ref{ex:1-4-6}), kein \textit{denn} einsetzen:
\begin{exe}
\ex\label{ex:1-4-9}
\begin{xlist}
\ex[*]{
\label{ex:1-4-9a}
er hat wen denn gesehen?}
\ex[*]{
\label{ex:1-4-9b}
morgen wird denn wo gearbeitet?}
\end{xlist}
\end{exe}
Ein paralleller Unterschied findet sich bei Fragen mit \textit{warum}. Wenn sich dieses Interrogativum in der Position \textit{K} befindet, kann es unbetont sein, wie in (\ref{ex:1-4-10a}), oder
man kann es stark betonen, wie in (\ref{ex:1-4-10b}):
\begin{exe}
\ex\label{ex:1-4-10}
\begin{xlist}
\ex\label{ex:1-4-10a}
 warum hat Karl das get\underline{\underline{a}}n?
\ex\label{ex:1-4-10b}
war\underline{\underline{u}}m hat Karl das getan?
\end{xlist}
\end{exe}
Bei Stellung des Interrogativums nach \textsuperscript{f}V sind beide Betonungsformen unakzeptabel;
möglich ist diese Stellung nur mit \isi{Betonung} auf der ersten Silbe von \textit{warum}:
\begin{exe}
\ex\label{ex:1-4-11}
\begin{xlist}
\ex[*]{
\label{ex:1-4-11a}
Karl hat das warum get\underline{\underline{a}}n?}
\ex[*]{
\label{ex:1-4-11b}
Karl hat das war\underline{\underline{u}}m getan?}
\ex[]{
\label{ex:1-4-11c}
Karl hat das w\underline{\underline{a}}rum getan?}
\end{xlist}
\end{exe}
Ebenso verhalten sich \textit{wozu}, \textit{womit} und einige andere zweisilbige Interrogativa.

Diese Unterschiede zwischen (\ref{ex:1-4-7}) und (\ref{ex:1-4-8}) und
zwischen (\ref{ex:1-4-10}) und (\ref{ex:1-4-11}) will ich wieder so
bezeichnen, daß ich Sätze wie (\ref{ex:1-4-7}) und
(\ref{ex:1-4-10})~-- die F2"=Stellung haben und in der Position
\textit{K} ein Element haben, das ein Interrogativum enthält~-- als
direkte (Ergänzungs-)Interrogativsätze bezeichne. Sätze wie
(\ref{ex:1-4-8}) und (\ref{ex:1-4-11c})~-- die ebenfalls F2"=Stellung
haben, aber ein \isi{Interrogativpronomen} in anderer Position enthalten~--
sind dagegen weder Interrogativ- noch Deklarativsätze.\footnote{%
   Die Form \textit{w\underline{\underline{a}}rum} kann in der Position \textit{K} auftreten
   (\textit{w\underline{\underline{a}}rum hat Karl das getan?}); vermutlich sollten solche
   Sätze aber trotzdem nicht unter \textsq{Interrogativsätze} im intendierten
   Sinn fallen. Ein Grund für diese Vermutung ist, daß es zu normalen
   direkten Interrogativsätzen in aller Regel entsprechende indirekte
   Interrogativsätze gibt. So sind die Parallelen zu (\ref{ex:1-4-10})
   in (\ref{ex:1-fn18i}) akzeptabel:
   \eal
   \label{ex:1-fn18i}
   \ex
   \label{ex:1-fn18ia}
   Karla weiß nicht, warum Karl das get\underline{\underline{a}}n hat
   \ex
   \label{ex:1-fn18ib}
   Karla weiß nicht, war\underline{\underline{u}}m Karl das getan hat
   \zl
   Parallel zu (\ref{ex:1-4-11c}) gibt es zwar (\ref{ex:1-fn18ii}a), aber der indirekte \isi{Interrogativsatz} in (\ref{ex:1-fn18ii}b) ist unakzeptabel:
   \eal
   \label{ex:1-fn18ii}
   \ex
   \label{ex:1-fn18iia}
   Karla weiß nicht, daß Karl das w\underline{\underline{a}}rum getan hat?
   \ex
   \label{ex:1-fn18iib}
   Karla weiß nicht, w\underline{\underline{a}}rum Karl das getan hat
   \zl
   Erwartungsgemäß ist, wenn \textit{w\underline{\underline{a}}rum} in \textit{K} steht, die Einsetzung von \textit{denn} schlecht oder gar nicht möglich:
   \ea
   \label{ex:1-fn18iii}
   * w\underline{\underline{a}}rum hat Karl das denn getan?
   \z
   Ebenso bei \textit{wozu}, \textit{womit} usw.%
}

%page 29
Aus diesen Beobachtungen ergeben sich zwei wichtige Folgerungen. F1"=Stel\-lung
und F2"=Stellung sind topologische Schemata, die sich nicht nur formal, sondern auch
in ihren Gebrauchsmöglichkeiten und, im Zusammenhang damit, hinsichtlich ihrer
lexikalischen Bestandteile unterscheiden können. Vgl.\ (\ref{ex:1-4-5}) gegenüber (\ref{ex:1-4-6}) usw. Andererseits kann ein gegebenes \isi{Schema} ganz verschiedene Gebrauchsmöglichkeiten haben. So wird die F2"=Stellung \zb für Deklarativsätze wie (\ref{ex:1-3-2}) und (\ref{ex:1-3-3}) benutzt, aber
auch für direkte Ergänzungs"=Interrogativsätze wie in (\ref{ex:1-4-1}), (\ref{ex:1-4-7}) und (\ref{ex:1-4-10}) sowie für
nicht"=interrogative Fragesätze wie in (\ref{ex:1-4-4}), (\ref{ex:1-4-8}) und (\ref{ex:1-4-11}). Und eine gegebene semantische oder pragmatische Satzkategorie~-- etwa: \isi{Deklarativsatz} oder \isi{Konditionalsatz}~-- kann, wie wir später sehen werden [X, s.\ Anm.\ S.\,\pageref{fn-herausgeber-topo}]\label{X:1}, durch ganz verschiedene topologische Schemata
realisiert werden.

Besonders auf"|fällig ist der Mangel an Übereinstimmung zwischen topologischem
\isi{Schema} und funktionaler Satzkategorie bei Imperativsätzen. Wir finden Beispiele
wie (\ref{ex:1-4-12}):
\begin{exe}
\ex\label{ex:1-4-12}
\begin{xlist}
\ex\label{ex:1-4-12a} trag (d\underline{\underline{u}}) den Koffer in den Keller
\ex\label{ex:1-4-12b} bringt (\underline{\underline{i}}hr) doch den Müll auf die Straße
\ex\label{ex:1-4-12c} gib mir mal bitte einer einen Flaschenöffner
\ex\label{ex:1-4-12d} seien Sie doch nicht so stur
\end{xlist}
\end{exe}
%\addlines
In Imperativsätzen hat das finite \isi{Verb} im Singular eine besondere Form (\textit{trag},
\textit{gib}); im \isi{Plural} hat es die Form des Indikativs (\ref{ex:1-4-12b}) oder des Konjunktivs (\ref{ex:1-4-12d}). Das \isi{Subjekt} kann, wenn es \isi{morphologisch} 2.\,Person ist, ausgelassen werden ((\ref{ex:1-4-12a}) und (\ref{ex:1-4-12b}));
das \textsq{Höflichkeitspronomen} \textit{Sie} fungiert semantisch als 2.\,Person Singular oder
\isi{Plural}, ist \isi{morphologisch} aber 3.\,Person \isi{Plural} (\ref{ex:1-4-12d}). In den Sätzen unter (\ref{ex:1-4-12}) herrscht
F1"=Stellung. Das ist aber kein notwendiges Charakteristikum von Imperativsätzen,
denn wir finden auch Beispiele wie (\ref{ex:1-4-13}) mit F2"=Stellung:
\begin{exe}
\ex\label{ex:1-4-13}
\begin{xlist}
\ex\label{ex:1-4-13a}  d\underline{\underline{u}} trag bitte den Koffer in den Keller
\ex\label{ex:1-4-13b} den Müll bringt gefälligst selber auf die Straße
\ex\label{ex:1-4-13c} dann gib mir mal einer einen Flaschenöffner
\ex\label{ex:1-4-13d} das nächste Mal seien Sie nicht wieder so stur
\end{xlist}
\end{exe}
Obwohl Imperativsätze sich von Deklarativ"= und von Interrogativ"= bzw.\ Fragesätzen 
%page 30
markant unterscheiden, können sie mit beiden topologischen Schemata vorkommen
(aber nur mit diesen beiden).
\section{Anmerkungen zur Forschungsgeschichte}%5.
\label{sec:1-5}

Bei unserem Bemühen, topologische Charakteristika deutscher\il{Deutsch} Sätze zu finden, haben wir uns so weit wie möglich Greenbergscher Analysekategorien bedient, sind in
ihrer konsequenten Anwendung aber zu völlig anderen Ergebnissen als Greenberg
gelangt. Die Notwendigkeit, Greenbergs Analysen zu verwerfen, resultiert im wesentlichen aus zwei Gründen. (i)~In dem Maße, wie Greenbergs Analyse sich auf Beobachtungsdaten stützen kann, sind diese Daten willkürlich beschränkt; völlig normale Daten wie (\ref{ex:1-3-3}) und (\ref{ex:1-3-19}), die jedem Sprachlerner zugänglich sind, werden
ignoriert (oder in unverständlicher Weise analysiert, vgl.\ Fn.\,\ref{fn:1-11}). (ii)~Greenbergs zentraler Begriff \textsq{(einzelsprachlich) dominant} ist~-- soweit er verständlich erscheint~-- wesentlich mit der Wirkung schwacher Regeln verknüpft. Die für Deklarativ"= und
Interrogativsätze des Deutschen\il{Deutsch} charakteristischen topologischen Eigenschaften sind
jedoch wesentlich mit der Wirkung von starken Regeln (nämlich F1"=Stellung und F2"=Stellung) verknüpft.
\ssubsection{}%5.1.
\label{subsec:1-5.1}
Mängel wie bei Greenberg treten auch in der Spezialliteratur übers Deutsche\il{Deutsch}
selbst in jüngster Zeit immer wieder auf. Das ist einigermaßen überraschend, denn
das, was (\ref{ex:1-3-29}) und (\ref{ex:1-4-3}) ausdrücken, ist spätestens seit \citet{Erdmann1886} bekannt. Er
schreibt:
\begin{exe}
\ex\label{ex:1-5-1}
\begin{xlist}
\ex\label{ex:1-5-1a} "`I. \so{Verbum an zweiter Stelle}, \dash \so{ein Nomen vor dem Verbum,
alle andern ihm folgend}. [\ldots] Enthält [der einfache Aussagesatz] nur \so{einen} nominalen Satzteil, so steht dieser (gewöhnlich also das Subjectivswort) voran, und das Verbum schliesst den Satz ab [\ldots]. Enthält der Satz
aber \so{mehrere} nominale Bestandteile, so steht vor dem Verbum doch ebenfalls immer nur einer [\ldots]; alle andern folgen dem Verbum. Die Auswahl
dieses einen aber ist im Deutschen\il{Deutsch} völlig dem Belieben des Redenden überlassen; er stellt dasjenige Nomen voran, das ihm für den Zweck seiner Rede
gerade den passendsten Ausgangspunkt bietet, um dann mit Anreihung des
Verbums an dasselbe dem Satze die feste Grundlage zu geben, auf welcher
der Auf"|bau der manigfaltigsten und zahlreichsten weiteren Bestimmungen
erfolgen kann. Durchaus unrichtig ist es, wenn manche Grammatiker hier
dem Subjectsnominativ besondern Anspruch auf die erste Stelle einräumen
wollen; ebenso gut wie er kann jeder oblique Casus, jede adverbiale oder
prädicative Bestimmung vorantreten [\ldots], und zwar sowol nachdrücklich
betonte als ganz unwichtige, sowol kurze als sehr umfangreiche 
%page 31
Bestimmungen."' \citep[182f]{Erdmann1886}
\ex\label{ex:1-5-1b} "`II. \so{Verbum an erster Stelle}. Vorangestellt wird das Verbum allen anderen Satzteilen [\ldots] in verschiedenen Fällen."'  \citep[186]{Erdmann1886}

"`3. In \so{Fragesätzen}, die eine Ungewissheit über das Eintreten der
ganzen Handlung ausdrücken, steht das Verbum der Regel nach voran."'
 \citep[188]{Erdmann1886}
\addlines[2]
\ex\label{ex:1-5-1c} "`\so{Anordnung der dem Verbum folgenden
  Satzteile}. Wenn dem an zweiter Stelle stehenden (I) oder
vorangestellten (II) Verbum mehrere nominale Satzteile folgen, so ist
die Anordnung derselben, wenn sie gleiche Wichtigkeit für die Rede und
gleichen Tonwert haben, völlig frei. Die von vielen Grammatikern
aufgestellte Regel, dass das Subjectswort, sobald es nicht (im Typus
I) die erste Stelle einnehme, dem Verbum unmittelbar folgen müsse,
trifft zwar in vielen Fällen zu, hat aber ihre Begründung nicht in der
grammatischen Geltung desselben. Sie gilt nämlich unbedingt nur für
den Fall, dass das Subjectswort ein persönliches \isi{Pronomen} ist. Ein
durch Substantiva und vollbetonte Pronomina (\textit{dieser, jener, jeder}
u.\,a.) bezeichnetes Subject steht durchaus nicht immer unmittelbar
hinter dem Verbum [\ldots]. Die Freiheit der Anordnung der dem Verbum
folgenden Satzteile wird nämlich, soweit sie jedes für sich als
gleichberechtigt dem Verbum gegenüberstehn, nur durch ihren Tonwert
und durch rhetorische Rücksichten bestimmt"'  \citep[189]{Erdmann1886}
\ex\label{ex:1-5-1d} "`3. \so{Participia und Infinitive} werden als
prädicative Bestimmungen des Verbums stets an die letzte Stelle
gesetzt"'  \citep[191]{Erdmann1886}
\end{xlist}
\end{exe}
Die \textsq{Nomen} oder "`nominalen Satzbestandteile können einzelne
Worte (Casus eines Substantivs oder Pronomens und aus ihnen
entstandene Adverbia [\ldots]) sein; sie können aber auch aus mehreren als
grammatische Einheit dem Verbum gegenübertretenden Worten bestehn"'
(S.\,181), entsprechen also dem \textit{K} bzw.\ \textit{KM} von
(\ref{ex:1-3-29}) und (\ref{ex:1-4-3}). Was dort als infinites \isi{Verb}
(\textsuperscript{i}V) gekennzeichnet ist, bezeichnet Erdmann als
\isi{prädikativ} gebrauchte Bestimmung des Verbs; unter "`Verbum"' versteht er
allemal ein finites \isi{Verb} (\textsuperscript{f}V). Was in diesen
Textstellen gesagt ist, ist nichts anderes als eine verbale Fassung
von (\ref{ex:1-3-29})~-- F2"=Stellung, Erdmanns Typ I~-- und von
(\ref{ex:1-4-3})~-- F1"=Stellung, Erdmanns Typ II.

\ssubsection{}%5.2.
\label{subsec:1-5.2}
Viele inhaltliche Einzelheiten aus diesen Zitaten und den hier nicht zitierten
näheren Ausführungen dazu finden sich in ähnlicher Form in zeitgenössischen Werken (\zb bei
\citet{Sanders1883}), bei Herling \citeyearpar{Herling1821T,Herling1828,Herling1830} oder schon bei \citet{Adelung1782}. Dieser stimmt mit Erdmann besonders darin überein, daß er die zweite Position in einem \isi{Deklarativsatz} dem \textsq{eigentlichen Verb} (\dash dem finiten \isi{Verb}) vorbehält und infinite Verben als Bestimmungen des eigentlichen Verbs ansieht:
\begin{exe}
\ex\label{ex:1-5-2}
"`§780. Eben dasselbe gilt auch von den so genannten zusammen gesetzten
Zeiten, wo die Stellung des Participii nach dem Hülfsworte keine Ausnahme von der Regel machen kann, so bald man nur einiger Maßen deutliche
Begriffe von einem Verbo finito und von einem Participio hat. Allein, da
sich Sprachlehrer unter allen Geschöpfen Gottes bisher am wenigsten um
klare und deutliche Begriffe bekümmert haben, so haben sie auch hier die
Sachen ganz verkehrt angesehen, und das Participium für das eigentliche
Verbum, das Hülfswort aber, für~-- nun für das Hülfswort gehalten, und da
wollte denn freylich die Regel, daß das Verbum finitum seine Bestimmungen nach sich nimmt, nirgends passen. Es ist sehr erbaulich zu sehen, wie
sie sich theils winden, theils auf gespitzten Zehen über die Sache wegeilen,
wenn sie auf diesen Punct in der \isi{Wortfolge} kommen. [\ldots]

\hspace{1em} Die so genannten zusammen gesetzten Zeiten der Deutschen\il{Deutsch} Conjugation
bestehen aus einem Verbo finito mit einem unvollständigen Prädicate
[\ldots]. Wer siehet nun nicht, daß dieses Hülfswort hier das eigentliche
Verbum finitum ist, indem alle Veränderungen der Person und Zahl an
demselben allein vorgehen, dagegen das Participium, ob es gleich den
Hauptbegriff enthält, in allen Fällen unverändert bleibt, und folglich
weiter nichts, als ein von dem Verbo abgeleitetes Adverbium oder
Beschaffenheitswort ist"' \citep[525f]{Adelung1782}
\end{exe}
%page 32
Wesentliche Fakten, um die es uns bei der Auseinandersetzung mit Greenberg
gegangen ist, waren also lange vor Herling bekannt. Herlings Ausführungen dazu in
(\ref{ex:1-3-12}) sind in ihrem observationellen Gehalt nicht neu, unterscheiden sich von früheren (und vielen späteren) Darstellungen jedoch durch die Luzidität und Prägnanz in
der Analyse der systematischen Zusammenhänge. Dabei war Herling zu seiner Zeit
in erster Linie für seine Lehre komplexer Sätze bekannt, die er in (\citeyear{Herling1821T})
skizziert und in (\citeyear{Herling1823, Herling1827, Herling1832}) und (\citeyear{Herling1828})
ausführlich entwickelt hat. (Ein weiterer Interessenschwerpunkt war die Lehre von \isi{Tempus} und Modus:
Herling \citeyearpar{Herling1821C,Herling1837,Herling1840}.)
Seine Verbstellungslehre hat nicht überall sofort Aufmerksamkeit gefunden. So hatte
Becker im \textit{Organism} (\citeyear[319f]{Becker1827}) noch eine viel weniger durchsichtige und in mancher Hinsicht zweifelhafte Darstellung gegeben. Erst in der \textit{Grammatik} von \citeyear{Becker1829} (die "`seinem verehrten Freunde"' Herling gewidmet ist) hat er Herlings Analyse übernommen. Dabei verschärft er Herlings Deutung, die \isi{Position des Verbs} am \isi{Satzende}
sei die "`ursprüngliche Stelle"' (vgl.\ \ref{ex:1-3-12c}), in interessanter Weise:
\begin{exe}
\ex\label{ex:1-5-3}\settowidth\jamwidth{[378f]{Becker1829}}
"`wir unterscheiden demnach in der \isi{Topik} des prädikativen Satzverhältnisses \so{drei} Stellen, nämlich die des \so{Subjektes}, der \so{Kopula} und des \so{Prädikativs} z.\,B.
%page 33

\vspace*{.5\baselineskip}
\begin{tabular}{@{}l l l }
\so{Subjekt} & \so{Kopula} & \so{Prädikativ} \\
Die Blume & ist & schön. \\
Der Baum & hat & geblühet. \\
Das Kind & kann & sprechen. \\
Der Knabe & steht & auf. \\
\end{tabular}

\vspace*{.5\baselineskip}
Wenn das \isi{Prädikat} durch ein einfaches \isi{Verb} in einer einfachen \isi{Flexionsform} ausgedrückt ist; so nimmt das \isi{Verb}, welches den Begriff und durch die
\isi{Flexion} zugleich die Beziehungen ausdrückt, immer die Stelle der Kopula
ein. Wir lassen jedoch, um alle Verschiedenheit der deutschen\il{Deutsch} \isi{Topik} unter
wenigen Gesetzen zusammenfassen zu können, auch für diesen Fall \so{drei Stellen} gelten, indem wir annehmen, daß das \isi{Verb} alsdann die Stelle der
\isi{Kopula} einnehme, und daß auf die \so{nicht ausgefüllte} Stelle des Prädikativs andere topische Verhältnisse, wie die des objektiven Faktors, bezogen
werden z.\,B.

\vspace*{.5\baselineskip}
\begin{tabular}[b]{@{}l l l }
\so{Subjekt} & \so{Kopula} & \so{Prädikativ} \\
Der Βaum & blühet & -- \\
Das Kind & spricht & --\phantom{PPrädikativ} "' \citep[378f]{Becker1829} \\
\end{tabular}
\end{exe}
\addlines
(Ganz entsprechend Becker \citeyear[210]{Becker1832} und
\citeyear[310f]{Becker1837}.) In dieser Formulierung tritt womöglich
noch deutlicher als bei Herling hervor, weshalb die Position des
Verbs~-- des Vollverbs~-- am \isi{Satzende} als sein eigentlicher
systematischer Ort betrachtet wird: weil auf diese Weise gewisse
topologische Eigenschaften der Satzbestandteile generell formuliert
werden können. (Aus demselben Grund kann bei Benutzung von
Greenbergschen Analysekriterien die F"=Position des finiten Verbs nicht
zur Charakterisierung einer basic order benutzt werden.) Beckers~--
bzw.\ Herlings~-- Argumentation findet sich in fast unveränderter Form
dann später bei \citet[35]{Bierwisch1963} wieder. Adelung gedachte
dieselben topologischen Regularitäten generell zu formulieren, indem
er die zweite Satzposition (die des finiten Verbs) als Bezugspunkt
nahm. So, wie er seine Lehre formuliert, ist sie zweifellos
inkonsistent; wir gehen darauf nicht näher ein.\footnote{\label{fn:1-19}%
  Herlings Kritik möge als Hinweis genügen:
  \begin{quote}
  "`§12. Es scheint, als habe Adelung (Lehrgeb.\ B.\ II.\ §760. u.\,s.\,w.) beide entgegengesetzte
    Wortfolgen, nämlich der wesentlichen Theile des Hauptsatzes, so wie der Nebenbestimungen und
    Nebensätze, aus einem einzigen Principe erklärt, indem er festsetzt, daß das Unbestimmtere dem
    Bestimmteren und zwar nach dem Grade seiner Bestimmtheit vorstehe; aber bei genauerer Prüfung
    zeigt es sich, daß er von diesem Principe eine doppelte und zum Theil unsichere Anwendung
    macht. Denn in den wesentlichen Theilen des Hauptsatzes nennt er zwar das Subject bestimmt \so{an sich}, jedoch in \so{Absicht der Rede} erst zu bestimmen, und das Prädicat, welches als
    Bestimmung \so{nach folgt}, das Bestimmtere; die Nebenbestimmungen läßt er aber nach dem Grade
    ihres schwächeren Bestimmens den folgenden, welche durch sie bestimmt werden sollen,
    \so{vorausgehen}. Die Verschiedenheit dieser Anwendung seines Princips täuscht uns also in
    Ansehung der Einheit desselben. Die Rechtfertigung der topischen Stellung der Aussage
    (§761.\ Nro.\,3) zeigt die Verlegenheit bei der Anwendung seines Princips noch deutlicher, und
    hätte er [\ldots] die Aussage von dem Ausgesagten getrennt behandelt, so würde seine Behauptung
    "`daß das Regierende dem Regierten vorgesetzt werde, weil die Rection eine wahre Bestimmung sey"'
    unterblieben seyn. Es ist ja auch der Unterschied der \isi{Wortfolge} in beiden Fällen, indem bei den
    wesentlichen Theilen des Hauptsatzes die Bestimmung dem Bestimmten \so{folgt}, bei den
    Nebenbestimmungen aber \so{vorhergeht}, so unverkennbar und bezeichnend, daß es rathsamer ist,
    ihn hervorzuheben, als ihn auf jene Weise verwischen zu wollen."'
    \citep[302f]{Herling1821T}
\end{quote}%
}
Ob und wie weit Herlings Idee fruchtbar ist, werden wir später 
%page 34
[in X, siehe Anm.\ S.\,\pageref{fn-herausgeber-topo}]\label{X:2} untersuchen.

\largerpage
Erdmann beschränkt sich dagegen~-- wie auch (\ref{ex:1-3-29}) und (\ref{ex:1-4-3})~-- darauf, die durch
starke Regeln fixierten Positionen des finiten Verbs (in (\ref{ex:1-5-1a}, b)) und der infiniten Verben (in \ref{ex:1-5-1d}) festzuhalten. Dies ist vor allem im Verhältnis zu jener Arbeit bemerkenswert, über die er ausdrücklich sagt: "`In den meisten Punkten stimme ich überein mit der lesenswerten Abhandlung von E.\, \so{Nordmeyer} [\ldots]"' \citep[182]{Erdmann1886}. Die Abweichungen bestehen vor allem in zwei wichtigen Teiltheorien: der Umschließungslehre
und der Inversionslehre.
\ssubsection{}%5.3.
\label{subsec:1-5.3}
Es ist eine Tradition deutscher Sprachliebhaber, sich von einer angeblichen Eigentümlichkeit des
Deutschen\il{Deutsch} beeindruckt zu zeigen, die Herling so beschreibt:

\begin{exe}
\ex\label{ex:1-5-4}
"`Indem [\ldots] die Nebenbestimmungen, welche bei einem Substantive zwischen die Deuter und das \isi{Substantiv} selbst, bei dem Prädicate zwischen
dasselbe und einen vorhergehenden Begriff \zb die Aussage treten, mit
ihrem Bestimmten schon topisch zusammengehalten, und zu Einem Begriffe
verbunden, in geschlossenen Massen dargestellt werden, bröckeln solche
Sprachen, die diesen Unterschied nicht kennen, in einer gleichmäßigen
\isi{Wortfolge} fortschreitend einen Begriff dem andern nach, und fast nirgends,
als in ihren Nebensätzen, in sofern diese doch ein Ganzes ausmachen, ist eine solche Verknüpfung der Begriffe durch die \isi{Topik} sichtbar."' \citep[302]{Herling1821T}
\end{exe}
Und er sieht eine formale Ähnlichkeit zwischen Nebensätzen und Nominalphrasen:
\begin{exe}
\ex\label{ex:1-5-5}
"`Der Satzartikel "`daß"', die relativen Conjunctionalien stehen, da sie sich auf
den ganzen Satz beziehen und diesen gleichsam individualisirend einschließen, am Anfang des Satzes, wie der Artikel eines Substantivs"' \citep[89]{Herling1830}
\end{exe}
Der Eindruck, daß im Deutschen\il{Deutsch} gewisse Satzbestandteile in spezifischer Weise "`topisch zusammengehalten"' werden, ist natürlich primär durch die topologische Trennung von \textsq{Aussage} und \textsq{Ausgesagtem} begründet (die freilich in ähnlicher Weise
auch z.\,B.\ in keltischen\il{Keltisch} und nordgermanischen\il{Germanisch!Nord-} Sprachen besteht; vgl.\ Fn.\,\ref{fn:10} und Fn.\,\ref{fn:16}); die Parallelisierung mit der topologischen Beziehung zwischen Demonstrativum
oder Artikel ("`Deuter"') und \isi{Substantiv} dürfte durch den Vergleich mit dem Französischen\il{Französisch} motiviert sein. (Das Englische\il{Englisch} stimmt in dieser Hinsicht mit dem Deutschen\il{Deutsch}
%page 35
weitgehend überein.)

\citet{Nordmeyer1883} hat diese schon in Herlings Formulierungen zweifelhafte Lehre
ins Extrem getrieben: Auch bei Nebensätzen sieht er "`dasselbe Princip, welches dem
\isi{Prädikat} des Hauptsatzes seine Form verlieh, nämlich das der Umschließung"' (S.\,70), und Umschließungen erblickt er überall, das Deutsche\il{Deutsch} wird von diesem Prinzip gänzlich durchwaltet. Nordmeyers Umschließungslehre findet sich später bei \citet{Drach1937} unter der Bezeichnung "`Umklammerung"' fast wörtlich wieder.

Erdmann macht von dieser Umschließungslehre nur in relativ nüchterner Form
Gebrauch:
\begin{exe}
\ex\label{ex:1-5-6}

"`Andererseits aber wird im Satztypus I und II eine vom Verbum entfernte
Stelle, und zwar vorzugsweise die letzte, regelmässig gewählt für solche Bestimmungen, die mit dem Verbum finitum \so{grammatisch} oder \so{lexicalisch} eine Einheit bilden, indem sie die im Verbum enthaltene Aussage zunächst ergänzen. Sie sind immer scharf betont; gerade dadurch, dass ihre
Stellung am Schlusse mit der des Verbums am Anfange correspondiert,
wird der enge Zusammenhang beider angedeutet, und indem alle anderen
Satzteile (ausser dem einen nach Typus I.\ vorangestellten), auf welche die so
ergänzte Aussage des Verbums sich bezieht, von demselben und seiner
nächsten Bestimmung umschlossen werden, tritt die Einheit des Satzes
anschaulich hervor."' \citep[191]{Erdmann1886}
\end{exe}
Der Gedanke, daß das finite \isi{Verb} häufig mit einem am \isi{Satzende} stehenden Element
eine grammatische oder lexikalische Einheit bildet, läßt sich präzise explizieren und
empirisch erhärten, vgl.\ X [s.\ Anm.\ S.\,\pageref{fn-herausgeber-topo}].\label{X:9}

\ssubsection{}%5.4.
\label{subsec:1-5.4}
Erdmann weicht noch in einer zweiten wichtigen Einzelheit von Nordmeyer
ab. Nordmeyer nimmt~-- wie fast alle älteren Autoren (\zb Herling) und viele neuere~-- an, daß bei F2"=Sätzen im \textsq{Normalfall} das \isi{Subjekt} die Position \textit{K} einnimmt, daß aber
ein anderer Satzbestandteil an seine Stelle treten kann. Er schreibt:
\begin{exe}
\ex\label{ex:1-5-7}
"`Jetzt wird es auch einleuchtend, warum das grammatische \isi{Subjekt} unmittelbar hinter das Verbum tritt, sobald es von seiner Stelle vor demselben
verdrängt wird"' \citep[9]{Nordmeyer1883}
\end{exe}
Er trifft sich hier (u.\,a.) mit \citet[102]{Bierwisch1963}; beide haben übersehen, daß das
\isi{Subjekt} in F2"= (und in F1"=) Sätzen weit nach dem finiten \isi{Verb} kommen kann, etwa in
(\ref{ex:1-3-27}) und in (\ref{ex:1-5-8}):
\begin{exe}
\ex\label{ex:1-5-8}
zweifellos haben trotz der großen Wohnungsnot auch damals in Berlin und
Umgebung viele Menschen eine Heimat gefunden
\end{exe}
%page 36
Herling trägt solchen Fällen durch eine sehr differenzierte Inversionslehre Rechnung, und es ist kein Zweifel, daß in einem Grammatiksystem, das keinen formalen
Restriktionen unterworfen ist, derartige Umstellungen formulierbar sind. Es ist aber
völlig unklar, welche Evidenz in der Stimulusmenge einen Sprachlerner zu der Annahme führen könnte, daß das \isi{Subjekt} zwar in vielen verschiedenen Positionen im
Satz auftreten kann, aber dennoch primär in die Position \textit{K} vor dem finiten \isi{Verb} gehört. Wir haben schon in \ref{subsec:1-3.2} gesehen, daß die dem Sprachlerner zugängliche Evidenz ganz im Gegenteil darauf hindeutet, daß diese Position funktional unbestimmt
ist. Selbst wenn es \textsq{Subjektinversion} in Nordmeyers Sinn gäbe, wäre zu erwarten,
daß sie spezifische formale Eigenschaften hätte. Eine naheliegende Möglichkeit
wäre, daß durch die Voranstellung eines Nicht"=Subjekts das \isi{Subjekt} so, wie
Nordmeyer und Bierwisch unterstellen, in die zweite nicht"=verbale Position des
Satzes gedrängt wird; aber mit dieser Annahme sind Sätze wie (\ref{ex:1-5-8}) nicht zu erklären. Eine andere~-- eventuell zusätzliche~-- Möglichkeit wäre vielleicht, daß die vorangestellte
\isi{Konstituente} und das \isi{Subjekt} ihren Platz vertauschen. Es ist sehr fraglich, ob solche
Operationen in natürlichen Sprachen existieren können. Klare Beispiele dafür sind
aus keiner Sprache bekannt. In jedem Fall taugt diese Annahme nicht zur Erklärung
der Verhältnisse im Deutschen\il{Deutsch}. Der Satz (\ref{ex:1-5-8}) müßte dann auf (\ref{ex:1-5-9}) zurückgehen; aber
falls das Beispiel überhaupt voll akzeptabel ist, sind (\ref{ex:1-5-8}) und (\ref{ex:1-5-9}) (wegen der Skopusverhältnisse zwischen \textit{zweifellos} und \textit{viele Menschen}) nicht gleichbedeutend:
\begin{exe}
\ex\label{ex:1-5-9}
viele Menschen haben trotz der großen Wohnungsnot auch damals in Berlin
und Umgebung zweifellos eine neue Heimat gefunden
\end{exe}
Und wenn wir bei (\ref{ex:1-5-10a}) \textit{in die Tasche} voranstellen, müßte (\ref{ex:1-5-10b}) (oder \ref{ex:1-5-10c}) resultieren; aber (\ref{ex:1-5-10b}) ist kaum oder gar nicht akzeptabel, und (\ref{ex:1-5-10d}) bleibt ohne Erklärung:
\begin{exe}
\ex\label{ex:1-5-10}
\begin{xlist}
\ex\label{ex:1-5-10a} jemand hat mir einen Zettel in die Tasche gesteckt
\ex\label{ex:1-5-10b} in die Tasche hat mir einen Zettel jemand gesteckt
\ex\label{ex:1-5-10c} in die Tasche hat jemand mir einen Zettel gesteckt
\ex\label{ex:1-5-10d} in die Tasche hat mir jemand einen Zettel gesteckt
\end{xlist}
\end{exe}
Vgl.\ auch (\ref{ex:1-3-27}).

Der Annahme einer \isi{Subjektinversion} fehlt damit nicht nur jede primäre Motivation~-- vgl.\ \ref{subsec:1-3.2}~--; sie trägt auch absolut nichts dazu bei, die Stellungsmöglichkeiten
eines Subjekts nach dem finiten \isi{Verb} in F2"= oder F1"=Sätzen zu erfassen. Insofern verdient es Beachtung, daß Erdmann entgegen einer langen Tradition in (\ref{ex:1-5-1a},c) nachdrücklich betont, daß das \isi{Subjekt} im Allgemeinen weder auf die Position vor noch auf die Position unmittelbar nach dem finiten \isi{Verb} besonderen Anspruch hat; irgendeine Art
von \isi{Subjektinversion} gibt es bei ihm nicht.


%page 37
%\addlines
Daß die Stellung des Subjekts für das Verständnis der Verbstellungsregularitäten
irrelevant ist, hat Erdmann erstmals in \citet{Erdmann1881} hervorgehoben. Dort schreibt er:
\begin{exe}
\ex\label{ex:1-5-11}
"`einen alle sätze umfassenden einteilungsgrund, wie ihn R.\ offenbar sucht,
bietet die \so{stellung des verbums}, das in jedem satze einmal enthalten ist,
gegenüber \so{allen} von ihm abhängigen satzbestandteilen"' \citep[192]{Erdmann1881}
\end{exe}
In einer Fußnote fügt er hinzu: "`klar und treffend ist dies ausgesprochen von Koch
Deutsche grammatik\textsuperscript{3} §435; Wilmanns Deutsche grammatik §209."' \citep[192 Fn.\,1]{Erdmann1881}

Ob Erdmann berechtigt ist, Wilmanns derart als Stütze für seine Ansicht zu zitieren, ist indessen fraglich. Zwar schreibt Wilmanns im angegebenen §209:
\begin{exe}
\ex\label{ex:1-5-12}
"`Dem Verbum finitum geht éin \isi{Satzglied} voran, die übrigen, falls solche
vorhanden, folgen nach."' \citep[181]{Wilmanns1878}
\end{exe}
Wenige Zeilen später folgt jedoch die Bemerkung: "`Das \isi{Subjekt} steht, wenn es nicht
die Spitze des Satzes einnimmt, unmittelbar hinter dem Verbum (vgl.\ §74)."' Und in
§74 heißt es:
\begin{exe}
\ex\label{ex:1-5-13}
"`Alle Satzglieder, die sich zunächst dem Verbum anschliessen (Object,
Subst.\ mit Praep., Adverbia), können auch an die Spitze des Satzes treten.
[\ldots] Das Subject tritt dann im einfachen Aussagesatz hinter das Verbum finitum, in der Regel unmittelbar dahinter."' \citep[42]{Wilmanns1878}
\end{exe}
Es ist offensichtlich, daß Wilmanns hier eine Inversionslehre ganz im Sinne von
Nordmeyer vertritt: Das \isi{Subjekt} steht primär vor dem \textsuperscript{f}V; aber andere Elemente
können an seine Stelle rücken, und dabei tritt das \isi{Subjekt} unmittelbar hinter das
\isi{Verb}~-- dieselbe falsche Behauptung wie in (\ref{ex:1-5-7}). Daß vor dem \textsuperscript{f}V in jedem Fall genau 1
Element steht~-- worauf Erdmann in seinem Hinweis auf (\ref{ex:1-5-12}) wohl abhebt~–, ist allgemein unstrittig und unterscheidet nicht zwischen seiner in (\ref{ex:1-5-11}) ausgedrückten~-- kritisch gegen Ries gerichteten~-- Ansicht und den verschiedenen Inversionslehren.

Korrekt scheint dagegen Erdmanns Hinweis auf Koch zu sein. Der schreibt im angegebenen §435 u.\,a.:
\begin{exe}\settowidth\jamwidth{219f]{Koch1860}}
\ex\label{ex:1-5-14}
"`Im behauptenden Satze und in dem durch ein Interrogativ gebildeten Satze
gilt jetzt das Gesetz: an \so{der zweiten Stelle steht das Verb} und bei zusammengesetzten Verbalformen das \isi{Hilfsverb}. Jeder Satz läßt daher so viele
Umstellungen zu, als er außer dem einfachen \isi{Prädikat} Glieder hat:
\begin{quotation}
Sie zogen gestern lärmend an meinem Hause vorüber. \\
Gestern zogen sie lärmend an meinem Hause vorüber. \\
Lärmend zogen sie gestern an meinem Hause vorüber. \\
An meinem Hause zogen sie gestern lärmend vorüber. \\
Vorüber (an meinem Hause) zogen sie gestern lärmend."' 
\end{quotation}
\jambox{\citep[219f]{Koch1860}}
\end{exe}
%page 38
Hier ist nicht unmittelbar klar, was unter \textsq{Umstellung} zu verstehen
ist. Unter Annahme einer \isi{Subjektinversion} würde man den ersten
Beispielsatz als die Normalform und die folgenden 4 Beispiele als
\textsq{Umstellungen} dazu betrachten. Das würde jedoch heißen, daß der Satz
(außer dem einfachen \isi{Prädikat}) nur die 4 \textsq{Glieder} \textit{gestern},
\textit{lärmend}, \textit{an meinem Hause} und \textit{vorüber (an meinem Hause)} hätte.

Das kann Koch nicht meinen; das \isi{Subjekt} \textit{sie} ist mit Sicherheit
ebenfalls ein \textsq{Glied} des Satzes. Ganz allgemein: Wenn unter einer
\textsq{Umstellung} eine Alternative zu einer Normalstellung (mit
\isi{Subjektinversion}) zu verstehen wäre, könnte ein Satz mit $n$ Gliedern
nur n–1 Umstellungen zulassen. Wenn, wie Koch sagt, die Zahl der
Umstellungen gleich der Zahl der Glieder ist, muß man unter
\textsq{Umstellung} hier verstehen: verschiedene Besetzungen der
\textit{K}"=Position. Nach dieser Darstellung ist mithin keine der verschiedenen
Besetzungen von \textit{K} vor einer anderen grammatisch ausgezeichnet.  Dies
wird auch in den darauf folgenden Ausführungen klar. Es heißt z.\,B.:
\begin{exe}
\ex\label{ex:1-5-15}
"`Bei zwei Adverbialen\is{Adverbiale} steht das betontere nach: {\small er spielt heute gut}; bei mehreren werden sie am besten vertheilt, so daß sie theils dem \isi{Verb} voranstehen, theils demselben nachfolgen."' \citep[220, §435c]{Koch1860}

"`Auf diese Stellung wirkt ein a) [\ldots]~– b) Der Zusammenhang, in welchem
der Satz vorkömmt, indem in dem ersten Worte die Beziehung auf den vorigen Satz aufgenommen wird: So sprach er. Dieses tat er."' \citep[220, §436]{Koch1860}
\end{exe}
Weder hier noch an anderen Stellen ergibt sich ein Indiz für die Annahme einer \isi{Subjektinversion}.

Das tritt noch deutlicher hervor, wenn man frühere Auf"|lagen von Kochs Grammatik
vergleicht.\footnote{%
  Die 4.\,Auf"|lage stimmt im einschlägigen Abschnitt~(\citeyear[239f, §425f]{Koch1862}) mit der 3.\,Auf"|lage 1860
  völlig überein.%
}
In der 2.\,Auf"|lage heißt es:
\begin{exe}
\ex
"`Als Hauptgrundsatz für [die \isi{Wortfolge}] gilt jetzt: an der zweiten Stelle
steht das (\isi{Verb}) \isi{Prädikat} oder, wenn ein solches da ist, sein
Beziehungswort. Jeder Satz läßt daher so viele Umstellungen zu, als er
außer dem Prädikate Glieder hat:
\begin{quotation}
Ich \so{habe} gestern meinem Bruder einen Brief geschrieben. \\
Gestern \so{habe} ich meinem Bruder einen Brief geschrieben. \\
Meinem Bruder \so{hab}' ich gestern einen Brief geschrieben. \\
Einen Brief \so{habe} ich gestern meinem Bruder geschrieben. \\
Geschrieben \so{habe} ich gestern meinem Bruder einen Brief."'
\end{quotation}
\citep[187, §392]{Koch1854}
\end{exe}
Inhaltlich stimmt das mit (\ref{ex:1-5-14}) überein. (Das \textsq{Beziehungswort} ist die \isi{Kopula} bei
%page 39
nominalen Prädikaten; mit der etwas genaueren Formulierung "`einfaches \isi{Prädikat}"' in
(\ref{ex:1-5-14}) ist offensichtlich schlechthin jedes finite \isi{Verb} gemeint.) Auch für (\ref{ex:1-5-15}) gibt es entsprechende Formulierungen. In der 1.\,Auf"|lage heißt es dagegen ganz anders:
\begin{exe}
\ex\label{ex:1-5-17}
"`In seiner Beziehung auf das Subject wird das \isi{Prädikat} erst verständlich, daher steht jenes gewöhnlich voran.

Bei abweichender \isi{Wortstellung} (Inversion) pflegt das \isi{Verb}, mag es nun
\isi{Prädikat} oder Copula sein, an der zweiten Stelle zu stehen"'
\citep[137f, §246]{Koch1848}
\end{exe}
Aus dieser äußerst knappen Wortstellungstheorie kann man zweierlei erschließen:
Erstens nimmt Koch hier klar eine \isi{Subjektinversion} an. Zweitens ist hier ein wesentliches Problem aller Inversionstheorien angedeutet. Wenn die Position vor dem finiten \isi{Verb} primär dem \isi{Subjekt} zukommt, dann wäre es naheliegend, daß ein Satzelement vor das \isi{Subjekt} und das finite \isi{Verb} gestellt werden kann, etwa wie in (\ref{ex:1-3-21});
und sofern eine Nachstellung des Subjekts überhaupt möglich ist, wäre es naheliegend, daß ein \isi{Deklarativsatz} dann mit dem finiten \isi{Verb} beginnt, etwa wie in (\ref{ex:1-3-22}). Daß diese Möglichkeiten im heutigen Deutsch\il{Deutsch} nicht bestehen, muß in der Tat ausdrücklich wie in (\ref{ex:1-5-17}) hervorgehoben werden. Man kann vermuten, daß diese Überlegungen Koch veranlaßt haben, die Inversionstheorie in den späteren Auf"|lagen fallen zu lassen.

Ganz explizit und nachdrücklich hat \citet{Gabelentz1875}~-- von Erdmann offenbar
übersehen~-- die Subjektinversionslehre für das Deutsche\il{Deutsch} bestritten. Er schreibt:
\begin{exe}\settowidth\jamwidth{144f]{Gabelentz1875}}
\ex\label{ex:1-5-18}
"`§28. Grundgesetz scheint mir nach dem heutigen Stande unserer Sprache
zu sein, dass in der mittheilenden Redeweise das verbum finitum des
Hauptsatzes immer die \so{zweite} Stelle einnimmt. Die erste Stelle kann
inne haben:
\begin{enumerate}
\item[1)] \so{das grammatische Subjekt} [\ldots]~– Es kann
\item[2)] das \so{direkte oder indirekte Objekt des Hauptverbums} oder
das \so{verbale Objekt eines Hülfsverbums} den Satz eröffnen [\ldots]
\item[3)] kann ein \so{Adverb} oder dessen Aequivalent [\ldots] zu Anfange des Satzes
stehen [\ldots]~– Endlich
\item[4)] kann das \so{Prädicatsnomen} vorantreten"'
\end{enumerate}
\jambox{\citep[144f]{Gabelentz1875}}

"`Ich möchte den deutschen Satz einem Schranke mit drei Fächern vergleichen. Was das erste Fach enthalten kann, haben wir vorhin gesehen. Das
zweite, engste, enthält ein für allemal das verbum finitum. Das dritte ist das
geräumigste, denn dahinein muss Alles, was noch nicht untergebracht ist."' \citep[146]{Gabelentz1875}
\end{exe}
Dieses \textsq{Grundgesetz} interagiert nach Gabelentz mit zwei weiteren Gesetzmäßigkeiten.
%page 40
Die erste ist (\ref{ex:1-5-19}):
\begin{exe}
\ex\label{ex:1-5-19}
"`\so{dass jedes folgende Glied die vorhergehenden näher}\Hack{\break} \so{bestimmt},
mit andern Worten, das \so{Prädikat zu diesem bildet}, während \so{dieses zu jenen sich als Subjekt verhält}."' \citep[137]{Gabelentz1875}
\end{exe}
Hier sind "`\isi{Subjekt}"' und "`\isi{Prädikat}"' nicht im grammatischen Sinne zu verstehen,
sondern "`psychologisch"' gemeint. Die Gliedfolge A$>$B ist angemessen, wenn der
von A bezeichnete Gegenstand
\begin{exe}
\ex\label{ex:1-5-20}
"`das Thema, worüber ich reden will, also mein \isi{psychologisches Subjekt} bildet"' \citep[137]{Gabelentz1875}
\end{exe}
Die zweite Gesetzmäßigkeit ist, daß eine \isi{Wortgruppe} "`Nebenprädikate"' in sich aufnehmen kann. Dieses Prinzip der "`Infigierung"' oder "`Einschachtelung"' ist identisch
mit der "`Umschließung"', die wir bei \citet{Nordmeyer1883} beobachtet haben: Durch die
einleitende \isi{Konjunktion} und das satzschließende \isi{Verb} werden \textsq{Nebensätze} zu einer
Einheit; in derselben Weise entsteht durch das finite \isi{Verb} und die Elemente am
\isi{Satzende} bei F"=Sätzen eine Einheit. Diese Einheiten sind in sich nach dem Prinzip
(\ref{ex:1-5-19}) gegliedert, fungieren aber in ihrer Gesamtheit als syntaktische Elemente. Bei F2"=Sätzen bildet diese Einheit im Allgemeinen das psychologische \isi{Prädikat} zu dem psychologischen
\isi{Subjekt} in der Position \textit{K}; interrogative F1"=Sätze ermangeln eines solchen psychologischen
Subjekts. Eine Frage ist "`materiell keine selbständige Rede; was Wunder also, wenn sie es auch
formell nicht ist?"' \citep[156]{Gabelentz1875}.\footnote{%
  \citet[9]{Weil1879} beansprucht, alle von \citet{Gabelentz1875} aufgestellten topologischen Prinzipien schon
  25 Jahre früher (in der 1.\,Auf"|lage seines Buches, 1844) formuliert zu haben. Hinsichtlich des Deutschen\il{Deutsch} ist das unkorrekt. Bei seinen Bemerkungen über die deutsche \isi{Topologie} stützt sich Weil weitestgehend auf \citet{Herling1830}; insbesondere geht er eindeutig von einer \isi{Subjektinversion} aus:
  "`in all principal sentences the verb should be put in the midst of the sentence before the attribute
  and after the subject or that part of the proposition which occupies its place."'
  \citep[54]{Weil1879}%
}

\addlines% checked 2018
Die Annahme (\ref{ex:1-5-19}) ist nicht weniger zweifelhaft als die Lehre von der \textsq{Infigierung}
oder \textsq{Umschließung}; wir gehen hier nicht darauf ein. Es ist aber deutlich, daß die
Probleme der Subjektinversionstheorie einzelnen Autoren schon vor Erdmanns
nachdrücklicher Stellungnahme deutlich geworden sind. Trotzdem hat die traditionelle Lehre immer wieder Anhänger gefunden. So schreibt \zb Wunderlich in explizitem Widerspruch zu Erdmann:
\begin{exe}
\ex\label{ex:1-5-21}

"`Unsere eigene Darstellung hofft dieser Lösung noch näher [sc.\ als \citealt{Erdmann1886}] zu kommen.

Die Stellung des \so{Verbum gegen das Subjekt} beherrscht unsere ganze
\isi{Wortstellung} und darüber hinaus noch andere syntaktische Erscheinungen,
wie vor allem die Gliederung des Satzgefüges in Haupt"= und Nebensatz.
%page 41

Wir finden nun \so{drei Hauptformen dieser Stellung}: 1.\ das Verbum
\so{vor} dem Subjekte [\ldots]; 2.\ das Verbum \so{unmittelbar} hinter dem \isi{Subjekt}
[\ldots] und 3.\ das Verbum \so{mittelbar} hinter dem \isi{Subjekt}"' \citep[87f]{Wunderlich1892}
\end{exe}
In jüngerer Zeit ist Erdmanns Erkenntnis erst wieder von \citet{Drach1937} und dann
besonders klar von \citet{Griesbach1960a,Griesbach1961a} mit Nachdruck zur Geltung gebracht worden. Wenn~-- wie etwa bei \citet{Bluemel1914}~-- eine explizite Argumentation zugunsten
der Inversionstheorie versucht wird, wird gewöhnlich die Relevanz der Fakten, die
wir in \ref{subsec:1-3.2} diskutiert haben, außer Acht gelassen, und generell wird nicht mit der gebotenen Deutlichkeit zwischen starken und schwachen Regeln unterschieden. Dem
problematischen Verhältnis zwischen diesen Regeltypen wenden wir uns in X [s.\ Anm.\ S.\,\pageref{fn-herausgeber-topo}]\label{X:7} zu. Vorläufig bleiben wir dabei, die \textsq{grammatischen} Eigenschaften der Satztypen herauszuarbeiten.


\section{Endstellung}%6.
\label{sec:1-6}

Wir haben bisher, Greenberg folgend, nur uneingebettete Sätze von gewissermaßen
prototypischer Art betrachtet: Deklarativsätze und Fragesätze. Es gibt jedoch weitere
Arten von uneingebetteten Sätzen, und diese unterscheiden sich topologisch erheblich von den bisher behandelten:
\begin{exe}
\ex\label{ex:1-6-1}
\begin{xlist}
\ex\label{ex:1-6-1a} daß Karl sich aber auch immer so benehmen muß!
\ex\label{ex:1-6-1b} wenn Karl doch nur endlich kommen wollte!
\ex\label{ex:1-6-1c} ob Karl ihr aber auch nicht zu viel verspricht?
\end{xlist}
\end{exe}
Das auf"|fälligste an solchen Beispielen ist, daß das finite \isi{Verb} hier nicht in einer F1"= oder F2"=Stellung vorkommt, sondern am Ende des Satzes, unmittelbar nach potentiell vorkommenden infiniten Verben. Das topologische \isi{Schema} kann man als (\ref{ex:1-6-2})
formulieren:
\begin{exe}
\ex\label{ex:1-6-2}
C$>$(\textit{KM}*)$>$(\textsuperscript{i}V*)$^{\smallfrown}$\textsuperscript{f}V
\end{exe}
In der Position C stehen in unseren Beispielen die Satzkonjunktionen \textit{daß}, \textit{wenn},
\textit{ob}. Im typischen Fall sind Sätze von der Form (\ref{ex:1-6-2}) in andere Sätze \isi{eingebettet} und stellen sog.\ Nebensätze dar; an (\ref{ex:1-6-1}) sieht man, daß das nicht notwendig so ist. (Man kann diese Beispiele auch nicht ohne weiteres etwa darauf zurückführen, daß ein
Trägersatz, in den sie \isi{eingebettet} waren, ausgelassen worden wäre: Die Elemente
\textit{aber auch} und \textit{doch nur} kommen in eingebetteten Sätzen nicht mit gleicher semantischer Funktion vor.)

Un\isi{eingebettet} treten auch infinite Konstruktionen wie (\ref{ex:1-6-3}) auf:
\begin{exe}
\ex\label{ex:1-6-3}
\begin{xlist}
\ex\label{ex:1-6-3a} sich ständig mit den Trebegängern rumplagen zu müssen!
\ex\label{ex:1-6-3b} wegen so einer Kleinigkeit so einen Wirbel zu machen!
\end{xlist}
\end{exe}
%page 42
Solche Konstruktionen findet man auch \isi{eingebettet} als sog.\ satzwertige Infinitivkonstruktionen wie in (\ref{ex:1-6-4}):
\begin{exe}
\ex\label{ex:1-6-4}
\begin{xlist}
\ex\label{ex:1-6-4a} Karl hat es satt, sich ständig mit den Trebegängern rumplagen zu müssen
\ex\label{ex:1-6-4b} es war dumm von Karl, wegen so einer Kleinigkeit so einen Wirbel zu machen
\end{xlist}
\end{exe}
\addlines
Manche Eigenschaften dieser Konstruktionen sprechen dafür, daß dies
tatsächlich Sätze sind.\footnote{%
  Aber nicht jede Konstruktion, die
  einen \isi{Infinitiv} mit \textit{zu} enthält, ist \isi{satzwertig}; \zb nicht bei
  \textit{schein-} (vgl.\ (\ref{ex:1-6-10b})). Verben, die mit
  nicht"=satzwertigen Infinitkonstruktionen vorkommen, werden
  traditionell meist als Hilfsverben bezeichnet. \citet{Herling1828}
  schreibt dazu: 
  \begin{quote}
  "`Es hängt auch hier von der Bedeutung des Wortes ab,
  ob es wirkliches \isi{Verb} oder Hülfsverb ist. So sind \so{wird} und
  \so{weiß} in: "`er wird sich mit seinen Anlagen schon aus der
  Verlegenheit helfen"' und "`er weiß sich mit seinem Einflusse schon
  Ersatz zu verschaffen"', wie \so{pflegt} in: "`er pflegte sich in
  ähnlichen Fällen besser zu benehmen"' bloße Hülfsverben und ihre
  Infinitive haben nicht den Character des Satzes; aber in: "`er
  \so{wird} erst, was ich schon bin"' und "`er weiß, daß er sich helfen
  kann"', verschieden von "`er weiß sich zu helfen"', sind es beide
  wirkliche Verben. [127.]"' \citep[206]{Herling1828}

  Dazu die Anm.: "`127) Wie
  \so{pflegen} und \so{gewohnt seyn} darin sehr verschieden sind, daß
  letzteres wirkliches \isi{Verb}, ersteres in gleicher Bedeutung nur
  Hülfsverbum ist [\ldots]"' \citep[345]{Herling1828}

  "`Vollständige Nebensätze lassen sich mit
  Auslassung des Subjects, der Bezeichnung der Copula und der
  conjunctionalen \isi{Partikel} zu bloßen Satztheilen \so{verkürzen}, die
  dann noch den Character ganzer Nebensätze behalten; \zb "`er
  behauptete, daß er dies nicht unternehmen könne"' in: "`er behauptete,
  dies nicht unternehmen zu können"'."' \citep[206f]{Herling1828}
\end{quote}%
}
Sie kommen auch mit
Elementen wie \textit{um} vor, die man als Satzkonjunktionen betrachten kann:
\begin{exe}
\ex\label{ex:1-6-5}
um nicht gesehen zu werden, benutzte Karl den Hinterausgang
\end{exe}
%\addlines[2]
Wir können solche Konstruktionen zwanglos als Manifestationen des in (\ref{ex:1-6-1}) \mbox{exemplifizierten} Typs verstehen, wenn wir (\ref{ex:1-6-2}) durch (\ref{ex:1-6-6}) ersetzen:
\begin{exe}
\ex\label{ex:1-6-6}
(C)$>$(\textit{KM}*)$>$(\textsuperscript{i}V*)$^{\smallfrown}$\textsuperscript{u}V
\end{exe}
Dabei steht "`\textsuperscript{u}V"' für "`unabhängiges \isi{Verb}"': Nach den
Vorschlägen von \citet{Hoehle78a} selegiert ein \isi{Hilfsverb} ein \isi{Verb} in
einer bestimmten infiniten Form; das selegierte \isi{Verb} ist von dem
selegierenden \isi{Verb} lexikalisch abhängig. Finite Verben sind nie im
relevanten Sinne selegiert, also immer unabhängig. Die infiniten
Verben am Ende von satzwertigen Infinitivkonstruktionen~– \textit{zu müssen}
in (\ref{ex:1-6-3a}) und (\ref{ex:1-6-4a}), \textit{zu werden} in
(\ref{ex:1-6-5})~-- sind ebenfalls nicht selegiert und damit
unabhängig. (Das \textit{(rum)plagen} in (\ref{ex:1-6-3a}) und
(\ref{ex:1-6-4a}) ist dagegen von \textit{zu müssen} abhängig, und in
(\ref{ex:1-6-5}) ist \textit{gesehen} von \textit{zu werden} abhängig.)

Solche satzwertigen Infinitivkonstruktionen können, wie (\ref{ex:1-6-7}) zeigt, kein \isi{Subjekt}
enthalten:
\begin{exe}
\ex\label{ex:1-6-7}
\begin{xlist}
\ex[*]{
\label{ex:1-6-7a}
(Karl hat es satt,) Karl sich ständig mit den Trebegängern rumplagen zu müssen}
\ex[*]{
\label{ex:1-6-7b}
um Karl nicht gesehen zu werden, benutzte Karl den Hinterausgang}
\end{xlist}
\end{exe}
%page 43
Bemerkenswerterweise müssen sie jedoch Prädikate enthalten, die ein \isi{Subjekt} selegieren. \textit{arbeit-} und \textit{verachtet werd-} selegieren ein \isi{Subjekt} (im \isi{Nominativ}); vgl.\ (\ref{ex:1-6-8}). Die infiniten Sätze in (\ref{ex:1-6-9}) sind dementsprechend einwandfrei:
\begin{exe}
\ex\label{ex:1-6-8}
\begin{xlist}
\ex\label{ex:1-6-8a} hoffentlich arbeitet *(Karl)
\ex\label{ex:1-6-8b} dort scheint *(Karl) verachtet zu werden
\end{xlist}
\end{exe}
\begin{exe}
\ex\label{ex:1-6-9}
\begin{xlist}
\ex\label{ex:1-6-9a} es ist erwünscht, zu arbeiten
\ex\label{ex:1-6-9b} Karl ist zu beliebt, um verachtet zu werden
\end{xlist}
\end{exe}
\textit{gearbeitet werd-} selegiert dagegen kein \isi{Subjekt}, vgl.\ (\ref{ex:1-6-10}); und die infiniten Sätze
in (\ref{ex:1-6-11}) sind unmöglich:
\begin{exe}
\ex\label{ex:1-6-10}
\begin{xlist}
\ex\label{ex:1-6-10a} hoffentlich wird (*Karl) gearbeitet
\ex\label{ex:1-6-10b} dort scheint (*Karl) gearbeitet zu werden
\end{xlist}
\end{exe}
\begin{exe}
\ex\label{ex:1-6-11}
\begin{xlist}
\ex[*]{
\label{ex:1-6-11a}
es ist erwünscht, gearbeitet zu werden}
\ex[*]{
\label{ex:1-6-11b}
es ist zu heiß, um gearbeitet zu werden}
\end{xlist}
\end{exe}
Sehr deutlich ist das bei Prädikaten wie \textit{gefeiert werd-} und \textit{gegrillt werd-}, die
mit oder ohne \isi{Subjekt} vorkommen können wie in (\ref{ex:1-6-12}). Die entsprechenden infiniten Konstruktionen in (\ref{ex:1-6-13}) können nur wie (\ref{ex:1-6-12i}) interpretiert werden:
\begin{exe}
\ex\label{ex:1-6-12}
\begin{xlist}
\ex\label{ex:1-6-12i}
\begin{xlist}
\ex\label{ex:1-6-12ia} es ist ein erhebendes Erlebnis, wenn man gefeiert wird
\ex\label{ex:1-6-12ib} es ist spannend, wenn man gegrillt wird
\end{xlist}
\ex\label{ex:1-6-12ii}
\begin{xlist}
\ex\label{ex:1-6-12iia} es ist ein erhebendes Erlebnis, wenn gefeiert wird
\ex\label{ex:1-6-12iib} es ist spannend, wenn gegrillt wird
\end{xlist}
\end{xlist}
\end{exe}
\pagebreak
\begin{exe}
\ex\label{ex:1-6-13}
\begin{xlist}
\ex\label{ex:1-6-13a} es ist ein erhebendes Erlebnis, gefeiert zu werden
\ex\label{ex:1-6-13b} es ist spannend, gegrillt zu werden
\end{xlist}
\end{exe}
Diese Eigenschaft satzwertiger Infinitivkonstruktionen~-- daß sie in einem systematischen Sinn ein \isi{Subjekt} enthalten müssen, das jedoch phonologisch nicht realisiert werden kann~-- hat eine bemerkenswerte Folge. Es gibt Konstituenten gewisser
Art, die sich auf ein \isi{Subjekt} oder Objekt des eigenen Satzes \textsq{beziehen} können und
dabei mit der Bezugskonstituente im \isi{Kasus} (und eventuell Genus und/""oder Numerus)
übereinstimmen müssen. Besonders deutlich ist das bei dem Ausdruck \textit{ein- nach d- ander-}, der sich auf eine semantisch mehrzahlige \isi{Konstituente} bezieht (sofern er
nicht selber \isi{Subjekt} oder Objekt o.\,ä.\ bildet). In (\ref{ex:1-6-14a}) bezieht sich \textit{eine nach der
%page 44
anderen} auf das \isi{Subjekt} \textit{die Türen}, es ist \isi{Nominativ} und feminin. In (\ref{ex:1-6-14b}) ist \textit{einer nach dem anderen} \isi{Nominativ} und maskulin; es muß sich auf das \isi{Subjekt} \textit{wir}
beziehen, und die Referenten von \textit{wir} müssen männliche Personen sein. In (\ref{ex:1-6-14c}) ist
\textit{einen nach dem anderen} \isi{Akkusativ}; es muß sich auf das \isi{Akkusativobjekt} \textit{die
Burschen} beziehen. Entsprechend in (\ref{ex:1-6-14d}); nur ist \textit{die Burschen} hier ein \textsq{AcI"=Subjekt}. In (\ref{ex:1-6-14e}) ist \textit{einer nach der anderen} \isi{Dativ} und feminin; es muß sich auf das \isi{Dativobjekt} \textit{uns} beziehen, und die Referenten von \textit{uns} müssen weibliche Personen sein. In (\ref{ex:1-6-14f}) bezieht sich \textit{eins nach dem anderen} auf das \isi{Akkusativobjekt} \textit{die Fenster}; es ist \isi{Akkusativ} und neutral:
\begin{exe}
\ex\label{ex:1-6-14}
\begin{xlist}
\ex\label{ex:1-6-14a} die Türen sind eine nach der anderen kaputt gegangen
\ex\label{ex:1-6-14b} einer nach dem anderen haben wir den Burschen runtergeputzt
\ex\label{ex:1-6-14c} einen nach dem anderen haben wir die Burschen runtergeputzt
\ex\label{ex:1-6-14d} ich ließ die Burschen einen nach dem anderen einsteigen
\ex\label{ex:1-6-14e} uns wurde einer nach der anderen der Stuhl vor die Tür gesetzt
\ex\label{ex:1-6-14f} ich soll die Fenster eins nach dem anderen austauschen
\end{xlist}
\end{exe}
%\addlines[2]
Dieselben Bezugsmöglichkeiten kommen in Infinitivkonstruktionen vor. In (\ref{ex:1-6-15a}) bezieht sich \textit{einen nach dem anderen} auf \textit{die Burschen}; in (\ref{ex:1-6-15b}) bezieht sich \textit{einer
nach der anderen} auf \textit{uns}; in (c) bezieht sich \textit{eins nach dem anderen} auf \textit{die
Fenster}:
\begin{exe}
\ex\label{ex:1-6-15}
\begin{xlist}
\ex\label{ex:1-6-15a} er hat uns gedroht, die Burschen demnächst einen nach dem anderen wegzuschicken
\ex\label{ex:1-6-15b} er hat angekündigt, uns dann einer nach der anderen den Stuhl vor die Tür zu setzen
\ex\label{ex:1-6-15c} es ist nötig, die Fenster, sobald es geht, eins nach dem anderen auszutauschen
\end{xlist}
\end{exe}
In (\ref{ex:1-6-16a}) bezieht sich \textit{einer nach dem anderen} offensichtlich auf das implizite \isi{Subjekt} des infiniten Satzes. Das implizite \isi{Subjekt} ist semantisch mit \textit{den Burschen} zu
identifizieren; dies ist jedoch \isi{Dativobjekt} (zu \textit{geraten}), während \textit{einer nach dem
anderen} \isi{Nominativ} ist und sich nicht unmittelbar darauf beziehen kann. In (\ref{ex:1-6-16b}) bezieht sich \textit{eine nach der anderen} syntaktisch entsprechend auf das implizite \isi{Subjekt} des durch \textit{um} eingeleiteten Satzes; ähnlich in (\ref{ex:1-6-16c}). In (\ref{ex:1-6-16d}) kann man das implizite \isi{Subjekt}, auf das sich \textit{einer nach dem anderen} bezieht, semantisch mit \textit{die
Sklavenjäger} identifizieren; dies steht aber im \isi{Akkusativ}, so daß ein unmittelbarer
syntaktischer Bezug schon deshalb ausgeschlossen ist:
\begin{exe}
\ex\label{ex:1-6-16}
\begin{xlist}
\ex\label{ex:1-6-16a}  ich habe den Burschen geraten, im Abstand von wenigen Tagen einer nach dem anderen zu kündigen
%page 45
\ex\label{ex:1-6-16b} die Türen sind viel zu wertvoll, um eine nach der anderen verheizt zu werden
\ex\label{ex:1-6-16c} wir sind es leid, eine nach der anderen den Stuhl vor die Tür gesetzt zu kriegen
\ex\label{ex:1-6-16d} es wäre fatal für die Sklavenjäger, unter Kannibalen zu fallen und einer nach
dem anderen verspeist zu werden
\end{xlist}
\end{exe}
In allen diesen Fällen kann \textit{ein- nach d- ander-} nicht selbst \isi{Subjekt} sein: Satzwertige Infinitivkonstruktionen können kein explizites \isi{Subjekt} haben, vgl.\ (\ref{ex:1-6-7}). Der Ausdruck muß sich vielmehr syntaktisch auf ein \isi{Subjekt} der \isi{Infinitivkonstruktion} beziehen und mit ihm im \isi{Kasus}~-- \isi{Nominativ}~– übereinstimmen. Der Schluß scheint unvermeidlich, daß satzwertige Infinitivkonstruktionen obligatorisch ein phonologisch
leeres \isi{Subjekt} im \isi{Nominativ} enthalten.\footnote{%
  Dies gilt für alle Infinitkonstruktionen, die nicht lexikalisch selegiert sind, also \zb auch
  für adnominale Partizipialkonstruktionen (\textit{der von den Parteifreunden gefeierte Wahlsieg}),
  für infinite Interrogativsätze (\textit{wozu lange darüber nachdenken?}) und für
  \textsq{substantivierte Infinitive}. Es gilt nicht für finite oder für lexikalisch selegierte
  infinite Prädikate, vgl.\ (\ref{ex:1-6-10}). Auch wenn ein lexikalisch selegiertes (infinites)
  \isi{Prädikat} selber ein \isi{Subjekt} selegieren muß (\zb wenn das \isi{Prädikat} von \textit{woll-} abhängig
  ist), muß dieses \isi{Subjekt} nicht als phonologisch leere \isi{Konstituente} repräsentiert werden;
  vgl.\ Höhle (\citeyear[84ff, 173ff]{Hoehle78a}; \citeyear[67ff]{Hoehle80}).%
}
Umso auf"|fälliger sind Beispiele wie (\ref{ex:1-6-17}), die unter das \isi{Schema} (\ref{ex:1-6-6}) fallen:
\begin{exe}
\ex\label{ex:1-6-17}
\begin{xlist}
\ex\label{ex:1-6-17a} der Kleine da vorn (bitte) mal auf die Seite treten
\ex\label{ex:1-6-17b} die Größeren (bitte) im Vordergrund aufstellen
\ex\label{ex:1-6-17c} jeder zweite (bitte) hinter den Vordermann ducken
\end{xlist}
\end{exe}
Der Satz (\ref{ex:1-6-17a}) enthält ein \isi{Subjekt} im \isi{Nominativ} (\textit{der Kleine da vorn}). (\ref{ex:1-6-17b}) ist 2"=deutig: Man kann \textit{die Größeren} als \isi{Akkusativobjekt} verstehen, dann ist das \isi{Subjekt} ausgelassen; oder man kann \textit{die Größeren} als \isi{Subjekt} verstehen, dann ist ein Reflexivpronomen (von \textit{sich aufstellen}) ausgelassen, wie es in (\ref{ex:1-6-17c}) der Fall ist. Die Verben haben die Form des Infinitivs. Wenn (\ref{ex:1-6-17a}) tatsächlich infinit ist, dann ist auf"|fällig und in der deutschen\il{Deutsch} Syntax einzigartig, daß der Satz ein explizites \isi{Subjekt} im \isi{Nominativ} enthält; zugleich ist dann die Auslassung des Reflexivums in (\ref{ex:1-6-17c}) und in der
einen Interpretation von (\ref{ex:1-6-17b}) überraschend. Es dürfte angemessen sein, diese Verbformen trotz ihrer Gleichheit mit dem \isi{Infinitiv} als spezielle finite Formen aufzufassen,
die sich dadurch auszeichnen, daß (i)~ein Reflexivum obligatorisch fehlt und (ii)~das
\isi{Subjekt} (wie im normalen \isi{Imperativ}) fakultativ fehlt.

Sätze, die das topologische \isi{Schema} (\ref{ex:1-6-6}) erfüllen, weisen, wie ich sage, E"=Stellung
auf (oder: sind E-Sätze). Dabei soll das "`E"' an "`End(-stellung)"' erinnern. Sätze mit
F1"= und F2"=Stellung bezeichne ich im Unterschied dazu zusammenfassend als F-Sätze (sie haben F"=Stellung). Über E-Sätze und ihr Verhältnis zu F-Sätzen sind später [X, s.\ Anm.\ S.\,\pageref{fn-herausgeber-topo}]\label{X:10}
%page 46
noch einige Erwägungen anzustellen. In den folgenden Abschnitten lassen wir E-Sätze jedoch außer Betracht und konzentrieren uns ganz auf F2"=Sätze. Um diese Vorgehensweise zu begründen, möchte ich etwas ausholen.

\section{Lernbarkeit}%7.
\label{sec:1-7}

\ssubsection{}%7.1.
\label{subsec:1-7.1}
F2"=Sätzen kommt ein besonderes Interesse zu, weil sie unter den 3
topologischen Satzschemata den kompliziertesten Typ darstellen. Bei
ihnen erheben sich zwei Fragen. (i)~Was ist mit der Stellung des
finiten Verbs; wie ist präzise zu charakterisieren, welchen Ort es in
der Struktur des Satzes einnimmt, und wie ist sein Verhältnis zu
anderen Teilen des Satzes, besonders zu den infiniten Verben, zu
erfassen?  Diese Frage stellt sich ähnlich bei F1"=Sätzen. Bei
F2"=Sätzen kommt eine Frage hinzu: (ii)~Was ist mit der Position
\textit{K}; welches Verhältnis hat sie zum Rest des Satzes, und wie
kann man dieses Verhältnis präzise charakterisieren? Über diese
deskriptiven Fragen hinaus stellt sich die Erklärungsfrage: Gibt es
eine allgemeine Sprachtheorie, aus der~-- unter Annahme von
gesicherten Randbedingungen~-- folgt, daß die Fakten grade so und
nicht irgendwie anders sind?

\addlines% checked 2018
Die deskriptiven und die explanativen Fragen sind nicht völlig
unabhängig voneinander. Natürlich muß man, wenn man nach einer
Erklärung für Tatsachen sucht, die zu erklärenden Tatsachen
kennen. Aber wenn man gewisse Daten hat, ist nicht immer eindeutig,
wie die Zusammenhänge zwischen diesen Daten adäquat zu erfassen sind,
\dash die Daten sind mit verschiedenen Hypothesen über die Fakten
verträglich. Aber häufig sind nur wenige dieser Hypothesen auch mit
einleuchtenden explanatorischen Annahmen verträglich. Die Suche nach
generellen erklärenden Prinzipien kann deshalb die Annahme bestimmter
Analysen von Daten erzwingen, auch wenn diese Daten, für sich und ohne
Einbettung in größere theoretische Zusammenhänge betrachtet, durchaus
verschiedenen Analysen zugänglich wären. Die Meinung, man müsse
grundsätzlich "`erst"' die Fakten klären, um "`danach"' zu einer Erklärung
der Fakten fortschreiten zu können, ist unrichtig. Wie die Fakten
beschaffen sind, kann man manchmal erst im Zusammenhang mit einer
Erklärung der Fakten richtig beurteilen.\footnote{%
	Das heißt natürlich
  nicht, daß die Fakten immer in jeder Hinsicht so undeutlich sind,
  daß sie mit beliebigen allgemeineren Theorien verträglich wären. Im
  Gegenteil; in vielen Einzelfällen sind die Fakten in mancher
  Hinsicht so deutlich, daß sie mit zahllosen allgemeineren Theorien
  nicht verträglich sind, insbesondere mit Theorien, die nicht~--
  \zb in Hinsicht auf Probleme der Lernbarkeit~-- explanatorische
  Potenz beanspruchen können, sondern nur irgendwelchen schlichten
  \textsq{Einfachheits}"=Bedürfnissen entgegenkommen. (Dies darzulegen
  war mein Ziel in \citet{Hoehle80}.) Nur Theorien mit empirisch
  wohlbegründeten Implikationen können eine Wahl zwischen alternativen
  Hypothesen über Fakten erzwingen.%
}

Manche deskriptiven Fragen sind~-- oder scheinen vielmehr~-- leichter
zu
%page 47
beantworten, wenn man die Analyse auf einem systematischen Vergleich
zwischen F2"= und E"=Sätzen aufbaut. Das typische Ergebnis eines solchen
Vergleichs sind Analysen in der Art von \citet{Bierwisch1963},
\citet{Klima1965} oder einer der vielen Varianten davon. Wenn man von
einigen Mängeln im Detail absieht, kann man durchaus einräumen, daß
dies in ihren Grundzügen mögliche Deskriptionen der Daten sind. Der
Ansatz hat jedoch zwei Mängel. Erstens erscheint er so einfach, daß
einige alternative Deskriptionsmöglichkeiten gewöhnlich nicht
hinreichend exploriert werden. Zweitens läßt er die Frage, wie ein
Sprecher dazu kommen könnte, ein solches System zu erwerben, völlig
unbeantwortet. Dieser zweite Gesichtspunkt ist so wichtig, daß er eine
ausführlichere Betrachtung verdient.

\ssubsection{}%7.2.
\label{subsec:1-7.2}
Die Analyse sprachlicher Daten, die ein Sprachwissenschaftler
vornimmt, beruht auf der Gesamtheit aller irgendwie einschlägigen
Informationen, deren der Sprachwissenschaftler habhaft werden
kann. Besonders aufschlußreich sind dabei Urteile von reifen Sprechern
über Akzeptabiltät und Bedeutung von Beispielen, die nicht alltäglich
und stark stereotypisiert sind. Wenn ein Sprecher ein klares Urteil
über solche nicht"=alltäglichen Beispiele hat, dann muß dieses Urteil
darauf beruhen, daß der Sprecher~-- unbewußt~-- Regeln befolgt; diese
Regeln machen grade das aus, was man als (den wesentlichen Aspekt der)
Sprachbeherrschung ansieht.

Wenn der Sprachwissenschaftler aufgrund solcher Urteile dem reifen
Sprecher den Besitz eines komplexen und abstrakten Regelsystems
unterstellt, sollte er sich nicht von der Frage dispensieren, auf
welche Weise der Sprecher in den Besitz dieses Systems gekommen
ist. Selbst wenn man mit \citet{Katz1981} darauf besteht, daß der
Gegenstand der sprachwissenschaftlichen Theorienbildung sprachliche
Regelsysteme sind und diese von der Existenz (psycho-)biologischer
Systeme (Menschen) logisch unabhängig sind, sind die Umstände, unter
denen die Verfügung über solche Regelsysteme normalerweise erworben
wird, von größter Bedeutung für die sprachwissenschaftliche
Theorienbildung. Sie ergänzen die sonstigen Informationen, über die
der Sprachwissenschaftler verfügt, und können dazu beitragen, (i)~die
korrekte Wahl zwischen alternativen Hypothesen über die Fakten zu
determinieren und (ii)~generelle (notwendige) Eigenschaften der
Regelsysteme, die natürlichen Sprachen zugrunde liegen,
aufzudecken. Dabei sind es zwei Umstände, die ganz besondere Beachtung
verdienen.

\ssubsubsection{}%7.2.1.
\label{subsubsec:1-7.2.1}

\addlines% checked 2018
Im natürlichen Spracherwerbsprozeß kommen Instruktionen über unakzeptable Beispiele nach allem, was man weiß, allenfalls in unbedeutendem Maß vor; vgl.\
\citet[64f]{WexlerCulicover1980}. In Fällen, wo solche Instruktionen gegeben werden,
bleiben sie im typischen Fall folgenlos; vgl.\ \citet[160f]{Braine1971}. Der Sprachwissenschaftler dagegen benutzt solche negativen Beurteilungen als besonders wichtige
%page 48
und reiche Datenquelle, eben weil sie das Wirken von Regelbesitz zeigen. Diese Regeln müssen in einer realen Spracherwerbssituation~-- das heißt hier besonders: ohne
negative Instruktionen~-- erworben sein. Im typischen Fall erfüllen nur sehr wenige
unter den logisch möglichen Hypothesen des Sprachwissenschaftlers diese Lernbarkeitsbedingung. (Ein besonders scharfes Problem sind negative Urteile des reifen
Sprechers bei Regeln, deren Anwendung fakultativ ist; vgl.\ \citet{Baker1979}.) Im typischen Fall zeigt sich obendrein, daß relativ zu der Stimulusmenge des Sprachlerners
die korrekte Hypothesenwahl überhaupt nur dann erzwungen wird, wenn man annimmt, daß der Freiraum der hypothetischen Regelbildung aufgrund spezifischer
Prinzipien von vornherein~-- \dash hinsichtlich des Sprachlerners: aufgrund genetisch
bedingter Prädispositionen~-- sehr stark eingeschränkt ist.\footnote{%
  Hier~-- und nur hier~-- spielt die Annahme eine Rolle, daß die Urteile reifer Sprecher über
  gegebene Beispiele in hohem Maße übereinstimmen, wenn die Stimulusmengen, die diesen Sprechern
  während ihrer Spracherwerbsphase zugänglich waren, in wesentlicher Hinsicht ähnlich sind. Es wäre
  an sich möglich, daß ein Sprachlerner sich unter den verschiedenen logisch mit seiner
  Stimulusmenge verträglichen Hypothesen willkürlich die eine oder die andere herausgreift. Dann
  wäre zu erwarten, daß die reifen Sprecher~-- je nachdem, welche Regeln sie in der Erwerbsphase
  gewählt haben~– über Beispiele außerhalb der ursprünglichen Stimulusmenge zu völlig verschiedenen
  Urteilen kommen. So etwas scheint sehr selten vorzukommen. Wenn man bedenkt, wie verschieden die
  Stimulusmengen verschiedener Sprachlerner auch innerhalb einer \textsq{Sprachgemeinschaft}
  tatsächlich sind, ist die Übereinstimmung zwischen den reifen Sprechern höchst eindrucksvoll.%
}
Die Prinzipien, die diesen Freiraum von vornherein einschränken, sind damit Prinzipien, denen die Regelsysteme für alle natürlichen Sprachen unterliegen. Soweit sich solche Prinzipien
nicht aus allgemeinen (nicht"=sprachspezifischen) psychobiologischen Eigenschaften
des Sprachlerners deduzieren lassen, muß man sie als notwendige Prinzipien von
sprachlichen Regelsystemen überhaupt verstehen, und solche Prinzipien sind naturgemäß der Kern jeder sprachwissenschaftlichen Theorienbildung, grade auch der
\textsq{platonistischen} Theorie, für die \citet{Katz1981} plädiert. Es dürfte kaum eine Forschungsstrategie geben, die besser geeignet ist, solche notwendigen Prinzipien aufzudecken, als die strikte Beachtung von realistischen Lernbarkeitsbedingungen für
Regeln, denen der reife Sprecher folgt, ohne darüber instruiert worden zu sein.

\ssubsubsection{}%7.2.2.
\label{subsubsec:1-7.2.2}

Neben unakzeptablen Beispielen\footnote{%
  Damit meine ich nicht nur Beispiele, die schlechthin unakzeptabel sind, sondern auch Beispiele, die
  eine bestimmte geforderte Interpretation nicht haben können, die also \textsq{unter dieser
  Interpretation} unakzeptabel sind.%
}
sind vor allem komplexe Beispiele~-- besonders: Beispiele, die (u.\,U.\ mehrfach) eingebettete Sätze enthalten~-- eine Datenquelle von großer praktischer Bedeutung für den Sprachwissenschaftler; und zwar
wieder deshalb, weil solche Beispiele im typischen Fall außerhalb der alltäglich zu
beobachtenden Sprachproduktion liegen und deshalb im Allgemeinen nicht stereotypisiert oder
gar als solche auswendig gelernt sein können. In dem Maße, wie sie sicher und
%page 49
konsistent beurteilt werden, muß sich hier wieder die Wirkung von allgemeinen Regeln
niederschlagen, denen die Sprecher folgen. Das lernbarkeitstheoretische Problem
dabei ist, daß hochgradig komplexe Sätze beim Erwerb derartiger Regeln im Allgemeinen sicherlich keine Rolle spielen, daß die Urteile über komplexe Beispiele also aufgrund von
Regeln gefällt werden, die anhand von relativ einfachen Beispielen in der Stimulusmenge erworben sind. Dies ergibt sich aus zwei Überlegungen. Erstens sind in
natürlicher Kommunikation auch zwischen reifen Sprechern hochgradig komplexe
Sätze ausgesprochen selten, und im Umgang mit Kindern finden sie noch weniger
Verwendung. Zweitens spricht viel dafür, daß Kinder in frühen Stadien des
Spracherwerbs nicht im Stande sind, komplexe Sätze~-- insbesondere: Sätze, die
eingebettete Sätze enthalten~-- konsistent zu interpretieren. In der Stimulusmenge,
die der Sprachlerner als Basis für den Regelerwerb benutzt, sind in diesen frühen
Stadien also vermutlich gar keine komplexen Sätze enthalten.

In späteren Stadien sind die Kinder fähig, komplexe Sätze zu verstehen, und gebrauchen selber Sätze, die einen eingebetteten Satz enthalten. Diese korrekte Verwendung eingebetteter Sätze setzt gewisse Informationen über die Eigenschaften
solcher Sätze in der Stimulusmenge voraus. Das zeigt sich deutlich bei eng verwandten Sprachen. So können in einigen deutschen\il{Deutsch} Dialekten Relativsätze mit einer
\isi{Partikel} \textit{wo} o.\,ä.\ an Stelle von oder in Kombination mit einem Relativpronomen
gebildet werden (\zb \textit{der Kerl, (den) wo ich gestern getroffen habe}), und einige erlauben ein \textit{daß} bei indirekten Interrogativsätzen (\zb \textit{ich weiß nicht, wen daß du
gestern getroffen hast}); beides ist im Standarddeutschen\il{Deutsch} unmöglich und korreliert
nicht mit irgendwelchen Eigenschaften von nicht"=eingebetteten Sätzen. Sowohl die
Möglichkeit innerhalb solcher Dialekte, derartige Partikeln zu gebrauchen, als auch
die Unmöglichkeit solcher Partikeln im Standarddeutschen\il{Deutsch} muß der Sprachlerner in
irgendeiner Weise anhand der Beispiele mit Relativ- bzw.\ indirekten Interrogativsätzen in seiner Stimulusmenge erschließen.

\addlines% checked 2018
Gewisse Informationen über Sätze mit der Einbettungstiefe D\textsubscript{1} (Degree 1, \dash Sätze, die einen Satz enthalten, der in genau 1 \isi{Konstituente} vom Typ \textsq{Satz} echt enthalten ist) sind also notwendig für den Erwerb der Regeln für eingebettete Sätze. Mit
größter Wahrscheinlichkeit sind diese Informationen aber zugleich hinreichend für
den Erwerb der Regeln, die die Beurteilung von Sätzen mit beliebiger Einbettungstiefe D\textsubscript{$n$} ($n$≥1) erlauben. Denn es spricht nichts dafür, daß der Sprachlerner während
der relevanten Erwerbsphasen überhaupt Beispiele mit größerer Einbettungstiefe als
D\textsubscript{1} in seiner Stimulusmenge hat, und nach aller sprachwissenschaftlichen Erfahrung
sind die Regeln, die in hochgradig komplexen Sätzen wirksam sind, identisch mit
den Regeln, die in D\textsubscript{1}-Sätzen wirksam sind. Für alle faktisch vorkommenden Regeln,
die speziell mit eingebetteten Sätzen zu tun haben, ist deshalb zu fordern: Sie müssen
%page 50
anhand von Beispielen mit der Einbettungstiefe D\textsubscript{1}
erworben werden können.\footnote{%
	Das heißt nicht, daß Regeln, die aus
  irgendwelchen Gründen nur in Sätzen mit größerer Einbettungstiefe
  (etwa bei D\textsubscript{4}-Sätzen) wirksam sind, in einer allgemeinen
  Sprachtheorie prinzipiell unmöglich sein müssen. Aber solche Regeln
  sind unter natürlichen Erwerbsbedingungen nicht erwerbbar, weil so
  komplexe Beispiele in den Stimulusmengen der Sprachlerner nicht in
  hinreichendem Maß präsent sind.

  \citet{WexlerCulicover1980} weisen nach, daß Regelsysteme, die
  gewissen Prinzipien genügen, anhand von Beispielen mit der
  Einbettungstiefe D\textsubscript{2} lernbar (in einem speziellen
  formalen Sinne von "`lernbar"') sind. Meine Forderung ist wesentlich
  schärfer. Allerdings definieren sie die Einbettungstiefe in Hinsicht
  auf Basis"=Phrasemarker, während ich unter \textsq{Satz} hier eine
  Kategorie der \textsq{Oberfläche} verstehe. Wieweit dieser
  Unterschied wichtige empirische Konsequenzen hat, ist schwer zu
  übersehen.%
}

%\addlines
Ohne Frage müssen gewisse Regeln für eingebettete Sätze anhand von eingebetteten Sätzen in der Stimulusmenge eigens erworben werden. Aber sicherlich müssen
nicht alle Eigenschaften eingebetteter Sätze eigens erworben werden. Eine plausible
Vermutung ist, daß der Sprachlerner so weit, wie es irgend möglich ist, auf eingebettete Sätze dieselben Regeln anwendet wie auf uneingebettete Sätze. Diese Vermutung ist nicht nur intuitiv einleuchtend, sondern wird auch durch \isi{Spracherwerbsdaten} nahegelegt. Deutsch\il{Deutsch} lernende Kinder produzieren eingebettete Sätze erst, wenn
sie die Regeln für uneingebettete Sätze bereits weitestgehend beherrschen; gewöhnlich erst etliche Monate später (vgl.\ \citealt{Park1981}, \citealt{Clahsen1982}). Soweit ich weiß, ist
das bei allen Sprachen so. Faktisch sind die Regeln für eingebettete Sätze zum großen Teil identisch mit den Regeln für uneingebettete Sätze. Es ist schwer vorstellbar,
daß der Sprachlerner eingebettete Sätze zu analysieren und zu produzieren versucht,
ohne dabei von den für uneingebettete Sätze bereits erworbenen Regeln Gebrauch
zu machen, und dennoch für eingebettete Sätze Regeln erwirbt, die mit den Regeln
für uneingebettete Sätze so weitgehend übereinstimmen. Es ist im Gegenteil eine
natürliche Annahme, daß die Analyse eingebetteter Sätze dem Kind überhaupt erst
dank der Regeln für uneingebettete Sätze zugänglich wird.

Wenn man diese Vermutung akzeptiert, muß man auch annehmen, daß der Erwerb der Regeln für uneingebettete Sätze gar nicht (oder nur in ganz unbedeutendem Maße) durch die Kenntnis eingebetteter Sätze beeinflußt wird. Dies stimmt wieder mit den \isi{Spracherwerbsdaten} überein: Es ist nicht bekannt, daß ein Sprachlerner
seine Regeln für uneingebettete Sätze zu dem Zeitpunkt, wo er eingebettete Sätze zu
produzieren beginnt, in Übereinstimmung mit seinen Regeln für eingebettete Sätze
geändert hätte.\footnote{%
  Wahrscheinlich gibt es schon vor Beginn der Produktion eingebetteter Sätze eine Phase, in der
  solche Sätze weitgehend korrekt analysiert werden. Es wäre vorstellbar, daß diese
  Analyseergebnisse die Formulierung der Regeln für uneingebettete Sätze beeinflussen. Mir scheint
  diese Überlegung zweifelhaft. In jedem Fall sollten allgemeine Überlegungen zur Lernbarkeit diese
  Vorstellung nicht als notwendig wahr unterstellen.%
}

%page 51
Ich nehme also, zusammenfassend, (\ref{ex:1-7-1}) an:
\begin{exe}
\ex\label{ex:1-7-1}
\begin{xlist}
\ex\label{ex:1-7-1a} Zunächst werden die Regeln für uneingebettete Sätze (D\textsubscript{0}) erworben.
\ex\label{ex:1-7-1b} Danach werden die Regeln für einfach eingebettete Sätze (D\textsubscript{1}) erworben.
\ex\label{ex:1-7-1c} In der Phase (b) werden, soweit es mit den Daten in der Stimulusmenge irgend vereinbar ist, die in der Phase (a) erworbenen Regeln benutzt.
\ex\label{ex:1-7-1d} Diese Regeln werden dabei nicht aufgrund von Beobachtungen geändert,
die sich nur an eingebetteten und nicht an uneingebetteten Sätzen machen
lassen.
\ex\label{ex:1-7-1e} Im Normalfall werden keine speziellen Regeln für mehrfach eingebettete
Sätze (D\textsubscript{\emph{n}}, \emph{n}$>$1) erworben.
\end{xlist}
\end{exe}
Diese Annahmen stehen, wie gesagt, in guter Übereinstimmung mit dem, was man über die Phasen des
Spracherwerbs weiß. Zugleich haben sie einen interessanten explanativen Gehalt. Aus (\ref{ex:1-7-1})
folgt, daß hochgradig komplexe Sätze aufgrund rekursiver Anwendung von Regeln für Sätze von der
Einbettungstiefe D\textsubscript{0} und D\textsubscript{1} analysiert und produziert werden;
hochgradig komplexe Sätze müssen nach (\ref{ex:1-7-1}) kompositionell aufgebaut sein, und wenn sie
nicht"=kompositionelle (idiomatische) Bestandteile enthalten, müssen diese auch in einfachen Sätzen
zu finden sein.\footnote{%
  Ich unterscheide zwischen "`kompositionell"' und "`strikt kompositionell"'. Ein Ausdruck A ist
  kompositionell, wenn seine Bedeutung aufgrund genereller Regeln aus der Bedeutung der Bestandteile
  von A und syntaktischen (und eventuell intonatorischen) Eigenschaften von A resultiert. A ist
  strikt kompositionell, wenn nicht allgemein die Bestandteile von A, sondern nur die unmittelbaren
  Konstituenten von A in die Regeln eingehen. Es ist ein empirisches Faktum, daß hochgradig komplexe
  Ausdrücke in natürlichen Sprachen kompositionell sind; in welchem Maße strikte Kompositionalität
  herrscht, ist dagegen eine offene Frage.%
}
Ausnahmen von diesem Grundsatz sind nur möglich, wenn positive Lernerfahrungen, \dash Beispiele für solche Ausnahmefälle in der Stimulusmenge, vorliegen.

\ssubsubsection{}%7.2.3.
\label{subsubsec:1-7.2.3}

\addlines[2]% checked 2018
Die Annahmen (\ref{ex:1-7-1}) werden, wenn ich es richtig verstehe, von vielen Autoren
geteilt. Überraschenderweise nehmen dieselben Autoren gewöhnlich an, daß derartige Abfolgen verschiedener Erwerbsphasen den Erwerb der korrekten Regeln irgendwie erleichtern würden. Dabei wird jedoch nicht erläutert, inwiefern und warum die Erwerbsaufgabe unter diesen Annahmen leichter sein sollte oder überhaupt
leichter sein könnte, als wenn komplexe Konstruktionen aller Art in gleicher Weise
und zur gleichen Zeit als Basis des Erwerbsprozesses dienen. Tatsächlich wird nach
der Logik der Sachlage das Lernbarkeitsproblem dadurch kolossal verschärft; vgl.\
\citet[§2.7.3]{WexlerCulicover1980}.

Schon dadurch, daß der Sprachlerner~-- im Gegensatz zum Sprachwissenschaftler~-- keinen wesentlichen Gebrauch von negativen Daten macht, wird die Menge der für
den Erwerbsprozeß potentiell hilfreichen Informationen quantitativ und qualitativ
%page 52
gewaltig eingeschränkt. Wenn außerdem gemäß (\ref{ex:1-7-1}) das Konzept der Einbettung eines Satzes als \isi{Konstituente} in einen Trägersatz einzig anhand von einfachen Einbettungen (D\textsubscript{1}) erworben werden muß, entstehen weitere Probleme. Dem naiven Blick
stellen sich komplexe Sätze als lockere Folge von einfachen (uneingebetteten) Sätzen
dar (oder als Sätze mit parenthetischen Einsprengseln). Forschungsgeschichtlich war
es ein weiter Weg bis zu der einfachen und durchsichtigen Lehre von \citet{Herling1821T}, daß (i)~Sätze aus Konstituenten aufgebaut sind und daß eine solche \isi{Konstituente}
(ii)~mit Konstituenten gleicher Art in bestimmter Weise zusammengefügt werden
kann (\isi{Koordination}) und/""oder (iii)~selbst ein Satz sein kann (Einbettung). Im Deutschen\il{Deutsch} lassen sich Sätze von beliebiger Komplexität vollständig und widerspruchsfrei
mit Hilfe der topologischen Schemata für F1"=, F2"= und E"=Sätze analysieren. Aber dies
ist keine triviale Analyseaufgabe und setzt die Konzepte der \isi{Koordination} und der
Einbettung voraus. Der Sprachwissenschaftler kann diese Konzepte durch extensive
Analyse sehr komplexen Beispielmaterials begründen; dem Sprachlerner ist dieses
Verfahren schon aufgrund der Armut seiner Stimulusmenge versagt. Es ist überhaupt nicht zu sehen, wie diese Konzepte anhand der Stimulusmenge erworben
werden könnten. Für die Phase (\ref{ex:1-7-1b}) muß der Sprachlerner diese Konzepte bereits
mitbringen. Wenn er das tut, kann er gemäß (\ref{ex:1-7-1c}) hilfreichen Gebrauch von den Regeln für uneingebettete Sätze machen. Wenn er diese Konzepte nicht voraussetzt, ist
nicht zu verstehen, wie er jemals die Fähigkeit erwerben könnte, die Bildung von
Sätzen größerer Komplexität zu beherrschen.

Darüber hinaus haben eingebettete Sätze Eigenschaften, die sich überhaupt erst
bei einer Einbettungstiefe von D\textsubscript{2} zeigen, \zb sind Extrapositionen~-- besonders: Extrapositionen von Sätzen~– \textsq{rightward bounded}; vgl.\ \citet{Hoehle80}. Solche Eigenschaften können nicht anhand von Beispielen mit der Einbettungstiefe D\textsubscript{1} erworben
werden; obendrein beruht die Kenntnis von dieser Eigenschaft auf negativen Daten
(\zb *\textit{daß die Leute schlafen, ist bekannt, von denen du gesprochen hast}). Wenn
(\ref{ex:1-7-1e}) richtig ist, müssen derartige Eigenschaften zu den Prädispositionen gehören, die
der Sprachlerner beim Erwerb der Regeln bereits voraussetzt. Da nicht zu sehen ist,
wie sie aus allgemeinen (nicht"=sprachspezifischen) Prinzipien folgen könnten, wird
man sie als notwendige Prinzipien von sprachlichen Regelsystemen betrachten müssen, die der Sprachlerner allenfalls dann überwindet, wenn ihn reiche Belege in der
Stimulusmenge dazu zwingen.

Das bei weitem schärfste Lernbarkeitsproblem stellt sich jedoch in der
Phase (\ref{ex:1-7-1a}). Wenn der Sprachlerner beim Erwerb der Regeln
für uneingebettete Sätze~-- im Gegensatz zum Sprachwissenschaftler~--
tatsächlich auf jede Information verzich\-tet, die aus der Analyse
komplexer Sätze (\zb aus dem systematischen Vergleich von
eingebetteten und uneingebetteten Sätzen) gewonnen werden könnte,
verarmt die Menge der potentiell hilfreichen Informationen quantitativ
und qualitativ in einem Maße,
%page 53
daß der Erwerb der korrekten Regeln als unlösbare Aufgabe erscheint~-- es sei denn,
daß die Menge der Regelformulierungen, die mit den Daten kompatibel sind, von
vornherein sehr scharf begrenzt ist. Die Abschnitte X
[s.\ Anm.\ S.\,\pageref{fn-herausgeber-topo}]\label{X:3} werden das reichlich
illustrieren.\footnote{%
  Was wir hier anhand eingebetteter Sätze besprochen haben, gilt natürlich ganz allgemein für
  komplexe Konstituenten, die aufgrund \isi{rekursiv} angewendeter Regeln gebildet werden, \zb komplexe
  Nominalphrasen und auch komplexe Wörter. Allerdings wird dies bei komplexen Wörtern~-- Wörtern,
  die mehrere freie und/""oder gebundene Morpheme enthalten~-- dadurch etwas verdunkelt, daß hier
  der Anteil von nicht"=kompositionellen (idiomatischen) Bildungen außerordentlich groß ist. Ich
  führe das darauf zurück, daß beim Erwerb von Wörtern spezielle Mechanismen wirksam sind, die den
  Sprachlerner einerseits zu einer sehr frühen und sicheren Wahrnehmung der Bedeutung, der
  Wortklasse und der unprädiktablen Kombinationseigenschaften (\textsq{Valenz}) der Wörter befähigen
  und damit eine Grundlage für den Erwerb des syntaktischen Regelsystems legen, andererseits aber
  auch den stereotypisierten Gebrauch der Wörter begünstigen und damit die gewaltige Menge von
  Irregularitäten bei Wortbildungsprodukten verantworten.%
}

\ssubsection{}%7.3.
\label{subsec:1-7.3}

Allerdings muß man im einzelnen unterscheiden. Auch bei uneingebetteten
Sätzen sind gewisse Analysen, die der Sprachlerner durchführt, möglicherweise aufgrund allgemeiner (nicht"=problemspezifischer) Annahmen verständlich. Betrachten
wir (\ref{ex:1-7-2}):
\begin{exe}
\ex\label{ex:1-7-2}
\begin{xlist}
\ex\label{ex:1-7-2a} der Hund erblickte einen Hasen
\ex\label{ex:1-7-2b} the dog saw a rabbit
\end{xlist}
\end{exe}
Beide Sätze haben die Folge (\ref{ex:1-7-3a}), aber (\ref{ex:1-7-2b}) manifestiert das \isi{Schema} (\ref{ex:1-7-3b}), während
(\ref{ex:1-7-2a}) das \isi{Schema} (\ref{ex:1-7-3c}) manifestiert:
\begin{exe}
\ex\label{ex:1-7-3}
\begin{xlist}
\ex\label{ex:1-7-3a} \isi{Subjekt}$>$\isi{Verb}$>$Objekt
\ex\label{ex:1-7-3b} S$>$V$>$O
\ex\label{ex:1-7-3c} \textit{K}$^{\smallfrown}$\textsuperscript{f}V$>$\textit{KM}
\ex\label{ex:1-7-3d} S$>$\textsuperscript{f}V$>$O
\ex\label{ex:1-7-3e} \textit{K}$>$V$>$\textit{KM}
\end{xlist}
\end{exe}
Die \isi{Spracherwerbsdaten} für Englisch\il{Englisch} bzw.\ Deutsch\il{Deutsch} lernende Kinder zeigen, daß die
Sprachlerner die Sätze tatsächlich so analysieren. Aber wie kommen sie zu diesen
Schlüssen; warum könnte (\ref{ex:1-7-2a}) nicht (\ref{ex:1-7-3b}), (\ref{ex:1-7-3d}) oder (\ref{ex:1-7-3e}) manifestieren; warum könnte
(\ref{ex:1-7-2b}) nicht (\ref{ex:1-7-3c}), (\ref{ex:1-7-3d}) oder (\ref{ex:1-7-3e}) manifestieren?
Es scheint, daß man eine Reihe von Annahmen machen muß:
\begin{exe}
\ex\label{ex:1-7-4}
\begin{xlist}
\ex\label{ex:1-7-4a} Es kann genau 1 charakteristisches \isi{topologisches Schema} für Deklarativsätze geben.
\ex\label{ex:1-7-4b} Wenn stilistisch neutrale Deklarativsätze in der Stimulusmenge nur durch 1 bestimmte Abfolge von \isi{Subjekt}, Objekt und \isi{Verb} repräsentiert sind, wird
diese Abfolge als das charakteristische \isi{Schema} für Deklarativsätze interpretiert.
\end{xlist}
\end{exe}
%page 54
Diese Annahmen reichen aus dafür, daß ein Englischlerner\il{Englisch} (\ref{ex:1-7-3c}) und (\ref{ex:1-7-3e}) verwirft,
denn ihm begegnen stilistisch neutrale Deklarativsätze nur mit der Abfolge (\ref{ex:1-7-3a}); ein
Deutschlerner muß dagegen (\ref{ex:1-7-3b}) und (\ref{ex:1-7-3d}) verwerfen, weil ihm auch andere stilistisch neutrale Abfolgen als (\ref{ex:1-7-3a}) begegnen. Ob die Prinzipien (\ref{ex:1-7-4}) spezifisch sprachlich sind oder auf allgemeine kognitive Prinzipien reduziert werden können, ist schwer zu beurteilen, und ich will darauf nicht eingehen. Jedenfalls setzt (\ref{ex:1-7-4}) voraus, daß die Klassifikationskategorien, die Greenberg für die Typologie vorgeschlagen hat, auch dem
Sprachlerner zur Verfügung stehen und von ihm wesentlich benutzt werden.

Bei der weitergehenden Frage, wie der Englischlerner\il{Englisch} zwischen (\ref{ex:1-7-3b}) und (\ref{ex:1-7-3d})
wählt, werden weitere Annahmen nötig. Der interessante Aspekt bei dieser Frage ist,
daß (uneingebettete) stilistisch neutrale Deklarativsätze des Englischen\il{Englisch} tatsächlich
immer das \isi{Schema} (\ref{ex:1-7-3d}) erfüllen. Das tritt deutlich hervor, wenn man direkte Entscheidungs"=Interrogativsätze zum Vergleich heranzieht, etwa (\ref{ex:1-7-5}):
\begin{exe}
\ex\label{ex:1-7-5}
\begin{xlist}
\ex\label{ex:1-7-5a} did the dog see a rabbit?
\ex\label{ex:1-7-5b} has the dog seen a rabbit?
\end{xlist}
\end{exe}
Sie unterscheiden sich von Deklarativsätzen nicht durch die Position des Haupt\-verbs~-- auch die Beispiele (\ref{ex:1-7-5}) erfüllen das \isi{Schema} (\ref{ex:1-7-3b})~–, sondern durch die Position des finiten Verbs. Ihr \isi{topologisches Schema} ist (\ref{ex:1-7-6}):
\begin{exe}
\ex\label{ex:1-7-6}
\textsuperscript{f}V$^{\smallfrown}$S$>$V$>$O
\end{exe}
Um Deklarativsätze im Unterschied zu (\ref{ex:1-7-6}) zu kennzeichnen, müßte man, so scheint
es demnach, entweder (\ref{ex:1-7-3d}) oder (\ref{ex:1-7-7}) wählen:
\begin{exe}
\ex\label{ex:1-7-7}
(X)$^{\smallfrown}$S$>$V$>$O \quad wobei X≠\textsuperscript{f}V
\end{exe}
Diesen Schluß kann man vermeiden, wenn man ein Prinzip wie (\ref{ex:1-7-8}) annimmt:
\begin{exe}
\ex\label{ex:1-7-8}
Wenn es für direkte Interrogativsätze charakteristische topologische Schemata gibt, sind sie durch ihre Abweichung von dem \isi{Schema} für Deklarativsätze gekennzeichnet.
\end{exe}
Eine Annahme wie (\ref{ex:1-7-8}) ist in keiner Weise
selbstverständlich oder trivial, entspricht aber ganz traditionellen
Vorstellungen und ist implizit in Greenbergs Darstellung;
vgl.\ (\ref{ex:1-3-6}). Man kann (\ref{ex:1-7-8}) als Folge der Annahme
verstehen, daß Interrogativsätze im Verhältnis zu Deklarativsätzen als
\isi{Satztyp} \textsq{markiert} sind. Wenn (\ref{ex:1-7-8}) korrekt ist und
(\ref{ex:1-7-6}) das \isi{Schema} für direkte
(Entscheidungs-)Interro\-gativsätze ist, reicht es aus,
%page 55
Deklarativsätze durch (\ref{ex:1-7-3b}) zu kennzeichnen;
(\ref{ex:1-7-3b}) wird dann zwingend als (\ref{ex:1-7-7})
interpretiert, ohne daß dies eigens angegeben werden muß.

Selbst wenn man diese Annahmen soweit für gesichert hält, bleibt immer
noch offen, warum der Sprachlerner nicht (\ref{ex:1-7-3d})
annimmt. Hier wird nun eine Tatsache wichtig, die sich aus
Beobachtungen zum Spracherwerb ergibt:
\begin{exe}
\ex\label{ex:1-7-9}
Flexionseigenschaften von Wörtern werden in einem späteren Stadium erworben als charakteristische topologische Eigenschaften von Satzkonstituenten.
\end{exe}
Aus (\ref{ex:1-7-9}) ergibt sich, daß der Englischlerner\il{Englisch} in einem frühen Stadium überhaupt nur
(\ref{ex:1-7-3b}) und nicht (\ref{ex:1-7-3d}) erwerben kann, weil er das System der (verbalen) \isi{Flexion} noch
gar nicht beherrscht. Was er später zusätzlich erwirbt, ist das (nicht in allen Einzelsprachen gültige) Prinzip (\ref{ex:1-7-10}):
\begin{exe}
\ex\label{ex:1-7-10}
Im typischen Fall ist das unabhängige \isi{Verb} innerhalb eines Satzes finit.
\end{exe}
Aus (\ref{ex:1-7-10}) in Verbindung mit (\ref{ex:1-7-3b}) folgt, daß Deklarativsätze (im typischen Fall) in der
Form (\ref{ex:1-7-3d}) realisiert werden; nichts zwingt den Sprachlerner dazu, (\ref{ex:1-7-3d}) selbst als das
charakteristische \isi{Schema} anzunehmen. Man kann damit rechnen, daß der Sprachlerner an dem \isi{Schema} (\ref{ex:1-7-3b}) auch nach dem Erwerb von (\ref{ex:1-7-10}) festhält, wenn man außerdem die Annahme (\ref{ex:1-7-11}) macht:
\begin{exe}
\ex\label{ex:1-7-11}
Regeln, die einmal erworben worden sind, werden nur dann geändert,
wenn die Informationen in der Stimulusmenge dazu zwingen.
\end{exe}
Die Annahme (\ref{ex:1-7-11}) hat eine wichtige Konsequenz. Dasselbe topologische Sche\-ma wie
für (stilistisch neutrale) uneingebettete Deklarativsätze gilt für einen großen Teil der
eingebetteten Sätze (wobei im Allgemeinen eine \isi{Konjunktion}, etwa \textit{that}, an den Anfang tritt);
nach (\ref{ex:1-7-1c}) und (\ref{ex:1-7-8}) ist das nicht überraschend. Hier gibt es aber \textsq{untypische} Fälle,
nämlich infinite Sätze. Beispiele wie (\ref{ex:1-7-12}) manifestieren (\ref{ex:1-7-3b}) in derselben Weise, wie
finite Sätze es tun:
\begin{exe}
\ex\label{ex:1-7-12}
for the poor to do the job (is pleasant for the rich)
\end{exe}
Es ist eine wichtige Frage, warum gemäß (\ref{ex:1-7-9}) Flexionseigenschaften relativ spät erworben werden. Aber die Tatsache, daß das so ist, erkärt in Verbindung mit (\ref{ex:1-7-11}),
warum (\ref{ex:1-7-3b}) und nicht (\ref{ex:1-7-3d}) gewählt wird.

Für den Deutschlerner haben wir bisher offen gelassen, wie er zwischen (\ref{ex:1-7-3c}) und
(\ref{ex:1-7-3e}) wählen kann. Es wäre möglich, daß er, solange er das (verbale) Flexionssystem
nicht beherrscht, das \isi{Schema} (\ref{ex:1-7-3e}) annimmt. Sobald er die \isi{Flexion} beherrscht, könnte
er durch Beispiele wie (\ref{ex:1-7-13}) gezwungen sein, das erste Vorkommen eines Verbs als finit zu spezifizieren, also (\ref{ex:1-7-3e}) durch (\ref{ex:1-7-3c}) zu ersetzen:

%page 56
\begin{exe}
\ex\label{ex:1-7-13}
\begin{xlist}
\ex\label{ex:1-7-13a} Karl soll die Kartoffeln holen
\ex\label{ex:1-7-13b} der Hund hat einen Hasen erblickt
\end{xlist}
\end{exe}
(Das Prinzip (\ref{ex:1-7-10}) würde zulassen, daß das infinite Vollverb an zweiter und das finite
\isi{Hilfsverb} an letzter Stelle steht.) Aber es ist fraglich, ob der Deutschlerner jemals (\ref{ex:1-7-3e})
annimmt. Falls (\ref{ex:1-7-4}) im Kern richtig ist, sucht der Sprachlerner nach einer charakteristischen Abfolge von \isi{Subjekt}, Objekt und \isi{Verb}. (\ref{ex:1-7-3e}) kann mit einem solchen gesuchten Muster gar nichts zu tun haben, da \isi{Subjekt} und Objekt darin nicht erwähnt sind.

Wenn der Sprachlerner nach einem solchen Muster sucht, kommt im Deutschen\il{Deutsch}
überhaupt nur (\ref{ex:1-7-14}) in Frage:
\begin{exe}
\ex\label{ex:1-7-14}
(S/O)$>$(O/S)$>$V
\end{exe}
Wenn am \isi{Satzende} ein \isi{Verb} steht, wie es in vielen Beispielen innerhalb natürlicher
Stimulusmengen der Fall ist, stehen \isi{Subjekt} und Objekt immerhin auf jeden Fall vor
dem \isi{Verb} (und in einem Großteil der Fälle steht das \isi{Subjekt} vor dem Objekt). Falls
der Sprachlerner (i)~ein ausgeprägtes Bedürfnis hat, die Stellung des Verbs (primär
des Vollverbs) relativ zu \isi{Subjekt} und Objekt zu fixieren, und (ii)~frühzeitig~-- wenn
auch undeutlich~-- bemerkt, daß bei der F"=Stellung des Verbs ein komplizierender
Faktor (Finitheit) eine Rolle spielt und die Stellung des Subjekts in stilistisch neutralen Deklarativsätzen variiert, könnte er dazu kommen, (\ref{ex:1-7-14}) anzunehmen; aber keinesfalls (\ref{ex:1-7-3e}). Wenn man unterstellt, daß die F"=Stellung des Verbs vor Beherrschung
der verbalen \isi{Flexion} für den Sprachlerner weitgehend unverstanden bleiben muß
(weil sie mit keiner konstanten topologischen Relation zwischen \isi{Subjekt}, Objekt und
Vollverb verbunden ist), muß (\ref{ex:1-7-14}) als die einzige sichere Hypothese erscheinen: \isi{Subjekt} und/""oder Objekt vor dem Vollverb ist nie verkehrt, solange das \isi{Verb} nicht finit
ist.

So unsicher diese Überlegungen beim gegenwärtigen Kenntnisstand sein müssen,
sie haben eine interessante \isi{Implikation}. Sie machen verständlich, wieso der Sprachlerner die \isi{Endstellung} des Verbs im Deutschen\il{Deutsch} mit besonderer Sensitivität wahrnimmt. Daß er das tut, geht eindeutig aus den \isi{Spracherwerbsdaten} hervor. Es ist
mehrfach bemerkt worden, daß Äußerungen vom Typ O$>$V in der kindlichen
Sprachproduktion auf"|fällig häufig sind, häufiger, als man nach der Häufigkeit von
Beispielen des Typs (\ref{ex:1-7-2a}) in den Stimulusmengen erwarten könnte; vgl.\ \citet{Park1981},
\citet{Clahsen1982} und dort genannte Literatur. Und es gibt Kinder, die während einer
kurzen Phase nach dem Erwerb der Verbalflexion S$>$O$>$\textsuperscript{f}V"=Äußerungen (\textit{Hansi Hunger hat} u.\,ä.) produzieren. Das ist bemerkenswert; uneingebettete Beispiele dieser Art kommen in den Stimulusmengen nicht vor. Verständlich ist es, wenn diese
Sprachlerner eine in früheren Stadien erworbene Analyse wie (\ref{ex:1-7-14}) bereits mit (\ref{ex:1-7-10})
kombinieren und (\ref{ex:1-7-14}) noch nicht zu (\ref{ex:1-3-29}) weiterentwickelt haben.


%page 57

\ssubsection{}%7.4.
\label{subsec:1-7.4}
Unsere Überlegungen lösen nicht alle Probleme beim Erwerb der
topologischen Eigenschaften von Verben im Englischen\il{Englisch} oder
Deutschen\il{Deutsch}. Es wäre auch verfrüht, ein vollständiges System von
Prinzipien vorzuschlagen, die den Erwerb der korrekten
einzelsprachlichen Regelsysteme ermöglichen. Mir kommt es hier
vielmehr darauf an, deutlich zu machen, daß die Aufgabe des
Sprachlerners, korrekte Hypothesen über das Regelsystem zu entwickeln,
das den Beispielen in der Stimulusmenge zugrunde liegt, unvergleichbar
schwieriger ist als die Aufgabe des Sprachwissenschaftlers. Dem
Sprachlerner fehlt der größte Teil der Indizien, deren sich der
Sprachwissenschaftler bedient: negative Daten und komplexe Sätze. Dies
zwingt zu der Annahme, daß der Sprachlerner von einer Reihe von
Prinzipien geleitet wird. Zu einem Teil folgen diese Prinzipien
vermutlich aus allgemeinen (nicht spezifisch sprachlichen)
psychobiologischen Eigenschaften des heranreifenden Organismus. Aber
es gibt keinerlei Grund zu der Vermutung, daß alle Prinzipien, die für
den Erwerb sprachlicher Regelsysteme angenommen werden müssen, so
allgemeiner Natur sind. Manche von ihnen könnten aus spezifisch
sprachlichen Prädispositionen resultieren, über die der Sprachlerner
aufgrund seiner biologischen Erbanlagen verfügt.

Die Prinzipien, die ich hier und in späteren Abschnitten formuliere,
sind natürlich höchst tentativ. Auch dort, wo sie mit größter
Wahrscheinlichkeit spezifisch sprachlicher Art sind, sind sie nicht
als irreduzible Axiome gedacht. Ganz im Gegenteil: Wenn es notwendige
spezifisch sprachliche Prinzipien gibt, die genetisch vererbt werden,
dann erwartet man nicht, daß es außerordentlich viele Prinzipien
solcher Art gibt, deren Wirkung sich nur in ganz speziellen
Eigenschaften oder gar einzelnen Konstruktionen gewisser Sprachen
niederschlagen. Man erwartet vielmehr, daß es relativ wenige (und
damit notwendig sehr abstrakte) Grundprinzipien gibt, aus deren
Interaktion eine reiche Menge von spezifischen Prinzipien zu
deduzieren ist, und es sollte sich bei hinreichend genauer Analyse in
so gut wie jeder \isi{Einzelsprache} Evidenz für diese Grundprinzipien
finden lassen.

%%\addlines
Ein Standardeinwand gegen alle Lernbarkeitserwägungen lautet: (i)~Im
Laufe der Kindheit erwirbt ein Mensch sehr viele höchst erstaunliche
Fähigkeiten; der Spracherwerb ist in dieser Hinsicht nicht
einzigartig. (ii)~In keinem der vielen Teilbereiche verfügen wir über
eine nennenswert strukturierte Theorie. Über höhere kognitive
"`Lern"'-Vorgänge wissen wir so gut wie nichts, unsere Unkenntnis ist in
diesem Bereich grenzenlos. (iii)~Es ist deshalb nicht auszuschließen,
daß der Mensch über eine enorm leistungsfähige unspezifische Fähigkeit
zur Entwicklung richtiger Theorien verfügt. (iv)~Deshalb ist es müßig,
über Prinzipien zu spekulieren, die speziell für den Erwerb
sprachlicher Regelsysteme relevant sind, und (v)~die Annahme
spezifisch sprachlicher notwendiger Prinzipien entbehrt grundsätzlich
jeden empirischen Gehalts.


%page 58
Bei Einwänden dieser Art imponiert der Mangel an Folgerichtigkeit. Die
Teile (i), (ii) und (iii) mögen zutreffen. Der Teil (iv) folgt nicht
daraus, und der Teil (v) ist falsch. Die Existenz einer hinreichend
starken allgemeinen Theoriebildungsfähigkeit, die in (iii) angenommen
wird, ist nicht demonstriert, und es scheint wenig Aktivität bei der
Konkretisierung dieser Idee zu geben. Viel Fortschritt ist da auch
nicht zu erwarten, solange nicht einmal für einen relativ kleinen
Teilbereich wie den Erwerb sprachlicher Regelsysteme überall völlig
deutlich ist, wie das Verhältnis zwischen der Menge der primären
Stimuli (hier: Äußerungen verhältnismäßig geringer Komplexität) und
dem Endzustand des Erwerbsprozesses (hier: Besitz eines Systems von
spezifisch sprachlichen Regeln) beschaffen ist. Die in (iv) genannte
\textsq{Spekulation} dient zwei Zielen: erstens dazu, das Verhältnis
zwischen Stimuli und Endzustand ganz allgemein genauer zu
charakterisieren; zweitens dazu, die Regeln in einigen Zweifelsfällen
korrekt zu formulieren. Wenn solche Prinzipien ihrer Aufgabe gerecht
werden sollen, sind sie einer extrem starken Adäquatheitsbedingung
unterworfen. Sie müssen zu allen historisch überschaubaren Zeiten in
allen Kulturen für alle Individuen, die sich ansonsten normal
entwickeln und normalen Randbedingungen ausgesetzt sind, Gültigkeit
haben. Man erwartet, in jedem Einzelfall Indizien für die Wirksamkeit
dieser Prinzipien zu finden, und es darf keinen Einzelfall geben, der
mit diesen Prinzipien nicht verträglich ist. Wenn sich alle derartigen
Prinzipien auf allgemeinere kognitive Prinzipien reduzieren lassen,
ist das kein Schaden: Dann hat die allgemeine Theoriebildungsfähigkeit
von (iii) etwas von jener Konkretisierung erfahren, die ihr bislang
fehlt. Gelingt eine solche Reduktion im Einzelfall nicht, ist ein
spezifisch sprachliches Prinzip gefunden.

Es ist denkbar, daß Prinzipien, die diesen Anforderungen genügen,
niemals gefunden werden. Ernste Forschungsanstrengungen auf diesem
Gebiet gibt es erst seit wenigen Jahren. Alles, was man heute erwägen
kann, hat vorläufigen Charakter und bedarf intensiver Überprüfung und
Fortentwicklung im For\-schungs\-pro\-zeß. Aber es gibt ermutigende
Ansätze, und es gibt Wege, sich darüber rational und produktiv
auseinanderzusetzen (vgl.\ \zb die Beiträge in
\citet{LogicalProblem}). Bei aller
Ungewißheit im Einzelnen ist eins sicher: Allgemeine Prinzipien der
besprochenen Art wird man~-- falls es sie gibt~-- nicht aufdecken,
wenn man Fragen der Lernbarkeit ignoriert.

\ssubsection{}%7.5.
\label{subsec:1-7.5}
Der Sprachlerner erwirbt vermutlich als erstes die Regeln für uneingebettete
Sätze. Die prototypischen uneingebetteten Sätze des Deutschen\il{Deutsch} sind deklarative F2"=Sätze. Uneingebettete E"=Sätze, wie wir sie in 6.\ gesehen haben, sind in der Stimulusmenge vermutlich relativ selten vertreten; in jedem Fall gehören sie einem intuitiv
stark \textsq{markierten} Typ an. Es bestehen Zweifel, ob ein Sprachlerner solche intuitiv
\textsq{markierten} Satztypen (dann, wenn er ihnen begegnet) als Evidenz beim Prozeß des
Regelerwerbs für F2"=Sätze heranzieht; jedenfalls sollte eine allgemeine Sprachtheorie
%page 59
dies nicht voraussetzen. Bei der Erörterung von F2"=Sätzen in den nächsten Abschnitten versuchen wir deshalb auch dort, wo es nur um die rein deskriptive Erhellung
der Fakten geht, E"=Sätze ganz außer Betracht zu lassen. Wenn es dem Sprachlerner
gelingt, die Regeln für F2"=Sätze ohne Rückgriff auf E"=Sätze zu erwerben, sollte es
möglich sein, alles Nötige den F2"=Sätzen selbst zu entnehmen, und die Verhältnisse
in E"=Sätzen sollten, wenn (\ref{ex:1-7-1c}) richtig ist, durch minimale Ergänzung der Regeln für
F2"=Sätze gewonnen werden können. Es wird sich zeigen [X, s.\ Anm.\ S.\,\pageref{fn-herausgeber-topo}]\label{X:4}, daß diese Erwartung sich
in einem bestimmten Bereich erfüllt, daß unter diesem Ansatz aber auch ein bestimmtes Problem in der Syntax von E-Sätzen scharf hervortritt.

Aus ähnlichen Gründen lassen wir F1"=Sätze vorläufig ganz beiseite. Man muß annehmen, daß sie in normalen Stimulusmengen nicht selten vertreten sind, aber möglicherweise stellen sie~-- besonders wenn sie Imperativsätze sind~-- einen intuitiv
\textsq{markierten} Typ dar, den man bei der Analyse der \textsq{unmarkierten} F2"=Sätze nicht
voraussetzen sollte. (Unter Lernbarkeitsgesichtspunkten ist der Vergleich zwischen
F1"= und F2"=Sätzen ohnehin wenig ergiebig.) Vielmehr ist zu erwarten, daß sich die
Eigenschaften von F1"=Sätzen durch minimale Ergänzung der Regeln für F2"=Sätze ergeben.
\section{Vorderfeld}%8.
\label{sec:1-8}



Wir haben für Deklarativsätze das \isi{Schema} (\ref{ex:1-3-29}) erarbeitet:
\begin{exe}
\exi{(\ref{ex:1-3-29})}%\makeatletter\def\@currentlabel{48}\label{ex:1-50-2}\makeatother
\textit{K}$^{\smallfrown}$\textsuperscript{f}V$>$(\textit{KM}*)$>$(\textsuperscript{i}V*)
\end{exe}
(Dasselbe \isi{Schema} erfüllen direkte Ergänzungs"=Interrogativsätze und ein Teil der Imperativsätze, wie wir in \ref{sec:1-4}. gesehen haben.) Dieses \isi{Schema} unterscheidet sich von
dem für direkte Entscheidungs"=Interrogativsätze (und einen Teil der Imperativsätze)
dadurch, daß vor dem finiten \isi{Verb} eine \isi{Konstituente} \textit{K} steht; vgl.\ (\ref{ex:1-4-3}):

\begin{exe}
\exi{(\ref{ex:1-4-3})}
\textsuperscript{f}V$>$(\textit{KM}*)$>$(\textsuperscript{i}V*)
\end{exe}
Von dem \isi{Schema} für E-Sätze unterscheidet es sich wesentlich dadurch, daß das unabhängige \isi{Verb} vor beliebig vielen Konstituenten \textit{KM} und vor eventuell vorhandenen infiniten Verben steht; vgl.\ (\ref{ex:1-6-6}):

\begin{exe}
\exi{(\ref{ex:1-6-6})}
(C)$>$(\textit{KM}*)$>$(\textsuperscript{i}V*)$^{\smallfrown}$\textsuperscript{u}V
\end{exe}
Dies führt zu der Annahme, daß innerhalb von Deklarativsätzen vor dem finiten
\isi{Verb} genau eine \isi{Konstituente} zu stehen habe. Diese Annahme ist unrichtig, und das
Deklarativsatzschema bedarf in dieser Hinsicht der Ergänzung.

\subsection{Disjunkte Konstituenten}%8.1.
\label{subsec:1-8.1}

%page 60
Wir finden Sätze wie (\ref{ex:1-8.1-1}):
\begin{exe}
\ex\label{ex:1-8.1-1} 
\begin{xlist}
\ex\label{ex:1-8.1-1a} einem Kerl, der so was tut, dem darf man nicht trauen
\ex\label{ex:1-8.1-1b} für den Preis, dafür kriegt man auch was besseres
\end{xlist}
\end{exe}
Hier entspricht dem \textit{K} des Deklarativsatzschemas das \textit{dem} in (\ref{ex:1-8.1-1a}) bzw.\ das \textit{dafür} in (\ref{ex:1-8.1-1b}); davor steht jeweils noch eine \isi{Konstituente} (\textit{einem Kerl, der so was tut} bzw.\ \textit{für den Preis}). Die Position, die die erste \isi{Konstituente} hier innehat, symbolisiere
ich durch "`\textit{K\textsubscript{L}}"'; die ganze Konstruktion wird gewöhnlich als \isi{Linksversetzung} (Left Dislocation) bezeichnet.

Formal eng verwandt sind Konstruktionen wie in (\ref{ex:1-8.1-2}), die man als Nominativus pendens bezeichnet:
\begin{exe}
\ex\label{ex:1-8.1-2}
\begin{xlist}
\ex\label{ex:1-8.1-2a} ein Kerl, der so was tut, dem darf man nicht trauen
\ex\label{ex:1-8.1-2b} so ein Preis, dafür kriegt man auch was besseres
\end{xlist}
\end{exe}
Die Gemeinsamkeit mit der \isi{Linksversetzung} besteht darin, daß die \isi{Konstituente} in
der Position \textit{K} jeweils eine \isi{Anapher} der \isi{Konstituente} in der Position \textit{K\textsubscript{L}} enthält (und
zwar \textit{d-}, wenn in \textit{K\textsubscript{L}} eine \isi{Nominalphrase} ist).\footnote{%
  Zur \isi{Linksversetzung} werden manchmal auch Beispiele wie (\ref{ex:1-fn31i}) gerechnet:
  \eal
  \label{ex:1-fn31i}
  \ex
  \label{ex:1-fn31ia}
  diesen Kerl, ich kenne ihn nicht
  \ex
  \label{ex:1-fn31ib}
  für den Preis, man kriegt auch was besseres dafür
  \zl
  Manche Sprecher akzeptieren solche Beispiele; für viele Sprecher sind sie aber unakzeptabel oder
  mindestens ganz erheblich schlechter als (\ref{ex:1-8.1-1}) und (\ref{ex:1-8.1-2}). Ich lasse solche Beispiele außer acht; ebenfalls
  solche, wo an Stelle des Personalpronomens (\textit{ihn}) in (a) eine Form von \textit{d-}
  (\textit{den}) steht.%
}
Der Unterschied besteht darin, daß
bei der \isi{Linksversetzung} die Konstituenten in \textit{K\textsubscript{L}} und \textit{K} im \isi{Kasus} übereinstimmen,
wenn \textit{K} durch eine \isi{Nominalphrase} ausgefüllt ist, bzw.\ in der \isi{Präposition} und dem
von ihr selegierten \isi{Kasus} übereinstimmen, wenn \textit{K} durch eine \isi{Präpositionalphrase}
ausgefüllt ist. Dem Nom.\ pendens fehlt diese Übereinstimmung.

Man kann mit \citet{Oertel1923} \isi{Linksversetzung} und Nom.\ pendens als disjunkten
Gebrauch der \isi{Kasus} bezeichnen. Er schreibt:
\begin{quotation}
"`Im normalen idg.\ Satze stehen die \isi{Kasus} direkt zu einem andern Satzelemente in Beziehung, und diese unmittelbare enge Verschränkung kann man
als \so{konjunkten} Gebrauch der \isi{Kasus} bezeichnen. Wenn immer diese unmittelbare Verknüpfung gelockert ist, kann man von einem \so{disjunkten} 
Gebrauch der \isi{Kasus} sprechen. Je nach dem Grade der Lockerung teilen sich
die disjunkten \isi{Kasus} in folgende drei Hauptgruppen: A.\ \so{Proleptischer Gebrauch}. Das Nomen hat die durch seine Beziehung auf ein anderes Satzelement geforderte Kasusform, aber die Verknüpfung ist dadurch gelockert,
%page 61
daß (1) ein Demonstrativpronomen das herausgehobene Nomen wieder aufnimmt [\ldots]. B.\ \so{Casus pendens}. Der Sprechende beginnt den Satz, noch
ehe er die endgültige Form, in die der Gedanke gegossen werden soll, definitiv festgelegt hat; dabei richtet er zunächst seine Aufmerksamkeit nur auf
ein einzelnes nominales Satzelement [\ldots] und fixiert dieses zunächst sprachlich im \isi{Nominativ} als dem einzigen \isi{Kasus}, welcher [\ldots] auch neutral (wie in
Titeln, Listen) ohne syntaktische Beziehung gebraucht werden kann. Erst
später wird diese sprachlich ausgedrückt und zwar am häufigsten durch ein
resumptives Demonstrationspronomen"' \citep[45]{Oertel1923}
\end{quotation}
(Die dritte Hauptgruppe ist der \textsq{absolute Gebrauch}, der im Deutschen\il{Deutsch} keine wesentliche Rolle spielt.) Die \isi{Linksversetzung} entspricht also Oertels proleptischem Gebrauch eines \isi{Kasus} (oder allgemeiner: einer \isi{Konstituente}), während der (auch von
ihm (\citeyear[48]{Oertel1923}) so genannte) Nominativus pendens dem pendenten Gebrauch entspricht. (Zum disjunkten Gebrauch von Konstituenten im Deutschen\il{Deutsch} vgl.\ \citet{Altmann1979} und \citet{Vat1980}.)

Mit der Übereinstimmung zwischen voranstehender \isi{Konstituente} und
\isi{Anapher} beim proleptischen und ihrem Fehlen beim pendenten Gebrauch
hängen weitere Unterschiede zusammen. So ist (\ref{ex:1-8.1-3a}) voll
akzeptabel, wärend (\ref{ex:1-8.1-3b}) zweifelhaft oder unakzeptabel
ist:
\begin{exe}
\ex\label{ex:1-8.1-3}
\begin{xlist}
\ex[]{
\label{ex:1-8.1-3a}
seinen\textsubscript{i} Hund, den sollte man\textsubscript{i} anständig behandeln}
\ex[*]{
\label{ex:1-8.1-3b}
sein\textsubscript{i} Hund, den sollte man\textsubscript{i} anständig behandeln}
\end{xlist}
\end{exe}
%\addlines[2]
Ähnlich ist die \isi{Linksversetzung} (mit \isi{Kasus}übereinstimmung) in (\ref{ex:1-8.1-4a}) einwandfrei,
während der Nom.\ pendens in (\ref{ex:1-8.1-4b}) unakzeptabel ist:
\begin{exe}
\ex\label{ex:1-8.1-4}
\begin{xlist}
\ex[]{
\label{ex:1-8.1-4a}
einen Lieblingsschriftsteller, den hat jeder}
\ex[*]{
\label{ex:1-8.1-4b}
ein Lieblingsschriftsteller, den hat jeder}
\end{xlist}
\end{exe}
Es ist bei Linksversetzungen nicht immer notwendig, daß in \textit{K} eine \isi{Präpositionalphrase} ist, wenn in \textit{K\textsubscript{L}} eine ist. In (\ref{ex:1-8.1-5}) befindet sich in \textit{K} lediglich eine lokale \isi{Anapher}
von \textit{K\textsubscript{L}}; trotzdem haben die Beispiele nicht den Charakter der Inkongruenz, der für
den Nom.\ pendens typisch ist. Dementsprechend kann auch ein gebundenes Possessivum in \textit{K\textsubscript{L}} auftreten, parallel zu (\ref{ex:1-8.1-3}); in den vergleichbaren Nom.\ pendens"=Konstruktionen (\ref{ex:1-8.1-6}) ist das nicht möglich:
\begin{exe}
\ex\label{ex:1-8.1-5}
\begin{xlist}
\ex\label{ex:1-8.1-5a} in diesem Kasten, da liegt ein Zettel
\ex\label{ex:1-8.1-5b} in seinem\textsubscript{i} Haus, da kann jeder\textsubscript{i} machen, was er will
\end{xlist}
\end{exe}
\begin{exe}
\ex\label{ex:1-8.1-6}
\begin{xlist}
\ex[*]{
\label{ex:1-8.1-6a}
dieser Kasten, da liegt ein Zettel}
\ex[*]{
\label{ex:1-8.1-6b}
sein\textsubscript{i} Haus, da(rin) kann jeder\textsubscript{i} machen, was er will}
\end{xlist}
\end{exe}
%page 62
Allerdings müssen bei der \isi{Linksversetzung} \textit{K\textsubscript{L}} und \textit{K} vom gleichen Typ sein. In (\ref{ex:1-8.1-5})
sind beide adverbiale Bestimmungen des Ortes. In (\ref{ex:1-8.1-7}) sind beide adverbiale Bestimmungen des Ziels bzw.\ des Ursprungs:
\begin{exe}
\ex\label{ex:1-8.1-7}
\begin{xlist}
\ex\label{ex:1-8.1-7a} nach Dresden, da fährt Karl gerne hin
\ex\label{ex:1-8.1-7b} aus Dresden, da kommt Karl grade her
\end{xlist}
\end{exe}
 
%\largerpage
In (\ref{ex:1-8.1-8}) stimmen \textit{K\textsubscript{L}} und
\textit{K} in dieser Hinsicht nicht überein, deshalb sind die
Beispiele unakzeptabel:\footnote{%
	Bei der "`gelockerten Verknüpfung"',
  die den disjunkten Gebrauch von Konstituenten kennzeichnet, besteht
  ein enger intonatorischer Zusammenhang zwischen der vorangestellten
  \isi{Konstituente} und dem Rest des Satzes. Wenn man einen starken
  intonatorischen Bruch nach der voranstehenden \isi{Konstituente} hat, sind
  Beispiele wie (\ref{ex:1-8.1-8}) und (\ref{ex:1-8.1-22b}) und
  (\ref{ex:1-8.1-27}) unter gewissen Kontextbedingungen möglich. Unter
  solchen Bedingungen kann man nicht mehr von "`gelockerter
  Verknüpfung"' reden, sondern die so vorangestellten Ausdrücke gehören
  nicht zu demselben Satz; sie verhalten sich so wie die Diskurse
  (\ref{ex:1-8.1-13}) und (\ref{ex:1-8.1-25})
  bzw.\ (\ref{ex:1-8.1-26}). (Solche Fälle gibt es natürlich auch beim
  \isi{Nominativ}; vgl.\ \citet{Havers1926}.)%
}
\begin{exe}
\ex\label{ex:1-8.1-8}
\begin{xlist}
\ex[*]{
\label{ex:1-8.1-8a}
in Dresden, da kommt Karl grade her}
\ex[*]{
\label{ex:1-8.1-8b}
aus Dresden, da fährt Karl gerne hin}
\ex[*]{
\label{ex:1-8.1-8c}
nach Dresden, da kann jeder machen, was er will}
\end{xlist}
\end{exe}
 Bei der \isi{Linksversetzung} könnte die \isi{Konstituente} in \textit{K} deshalb in vielen Fällen fehlen, ohne daß sich an der Akzeptabilität etwas ändern würde (wenn wir von der \isi{Intonation} absehen). Aber nicht in allen Fällen ist das möglich. Wenn man in (\ref{ex:1-8.1-7}) die \isi{Anapher} \textit{da} ausläßt, ergeben sich die für viele Sprecher unakzeptablen Beispiele (\ref{ex:1-8.1-9}):
\begin{exe}
\ex\label{ex:1-8.1-9}
\begin{xlist}
\ex[*]{
\label{ex:1-8.1-9a}
nach Dresden fährt Karl gerne hin}
\ex[*]{
\label{ex:1-8.1-9b}
aus Dresden kommt Karl grade her}
\end{xlist}
\end{exe}
\largerpage
Wenn \textit{K} eine \isi{Nominalphrase} enthält, die nicht selbst eine \isi{Anapher} ist, sondern
eine \isi{Anapher} von \textit{K\textsubscript{L}} echt enthält, ist die \isi{Linksversetzung} schlecht und nur der
Nom.\ pendens möglich:
\begin{exe}
\ex\label{ex:1-8.1-10}
\begin{xlist}
\ex\label{ex:1-8.1-10i}
\begin{xlist}
\ex[*]{
\label{ex:1-8.1-10ia}
dieses Burschen, dessen Aufsätze kenne ich}
\ex[*]{
\label{ex:1-8.1-10ib}
(von) diesem Kerl, ein Bruder von dem arbeitet in Köln}
\end{xlist}
\ex\label{ex:1-8.1-10ii}
\begin{xlist}
\ex\label{ex:1-8.1-10iia} dieser Bursche, dessen Aufsätze kenne ich
\ex\label{ex:1-8.1-10iib} dieser Kerl, ein Bruder von dem arbeitet in Köln
\end{xlist}
\end{xlist}
\end{exe}
Wenn \textit{K} eine \isi{Anapher} enthält, können in \textit{K\textsubscript{L}} freie Relativsätze auftreten, bei
denen sonst geltende Restriktionen für den Gebrauch von freien Relativsätzen nicht
%page 63
zur Geltung kommen; vgl.\ (\ref{ex:1-8.1-11}) gegenüber
(\ref{ex:1-8.1-12}):\footnote{%
	Die \isi{Anapher} für freie Relativsätze
  kongruiert (außer in Kopulasätzen wie (\ref{ex:1-8.1-11b})) mit dem
  Genus und dem Numerus des Relativpronomens;
  vgl.\ (\ref{ex:1-fn33i}). Die \isi{Anapher} für Interrogativsätze ist
  \textit{das} wie in (\ref{ex:1-fn33ii}). Die Ambiguität des
  Komplements in (\ref{ex:1-fn33iii}a) reflektiert sich in dem
  Unterschied zwischen (\ref{ex:1-fn33iii}b) und
  (\ref{ex:1-fn33iii}c):
  \eal
  \label{ex:1-fn33i}%
  \ex[]{%
  \label{ex:1-fn33ia}
  wen er kennt, den begrüßt er}
  \ex[*]{%
  \label{ex:1-fn33ib}
  wen er dort sieht, das begrüßt er}
  \zlmid
  \eal
  \label{ex:1-fn33ii}
  \ex[]{
  \label{ex:1-fn33iia}
  ob er kommt, das weiß niemand}
  \ex[]{
  \label{ex:1-fn33iib}
  wen er dort sieht, das weiß niemand
  }
  \ex[*]{
  \label{ex:1-fn33iic}
  wen er dort sieht, den weiß niemand}
  \zlmid\vspace{-1mm}%Hack
  \eal%
  \label{ex:1-fn33iii}
  \ex[]{\label{ex:1-fn33iiia}%
  er soll uns zeigen, wen er kennt}
  \ex[]{
  \label{ex:1-fn33iiib}
  wen er kennt, den soll er uns zeigen}
  \ex[]{
  \label{ex:1-fn33iiic}
  wen er kennt, das soll er uns zeigen}
  \zllast%
}
\begin{exe}
\ex\label{ex:1-8.1-11}
\begin{xlist}
\ex\label{ex:1-8.1-11a} hinter dem die her sind, der hat den Einbruch begangen
\ex\label{ex:1-8.1-11b} dem die Reporter da nachlaufen, das ist der Kommissar
\end{xlist}
\end{exe}
\begin{exe}
\ex\label{ex:1-8.1-12}
\begin{xlist}
\ex[*]{
\label{ex:1-8.1-12a}
hinter dem die her sind, hat den Einbruch begangen}
\ex[*]{
\label{ex:1-8.1-12b}
dem die Reporter da nachlaufen, ist der Kommissar}
\end{xlist}
\end{exe}
Wegen dieser Nicht"=Übereinstimmung zwischen \textit{K\textsubscript{L}} und \textit{K} sind solche Fälle zum
Nom.\ pendens zu rechnen.

\addlines[-1]% checked 2018
%%\largerpage[2]
Einerseits sind bei der \isi{Linksversetzung} also manche Konstruktionen~-- \zb gebundene Possessivpronomen wie in (\ref{ex:1-8.1-3}) und (\ref{ex:1-8.1-5})~-- möglich, die beim Nom.\ pendens ausgeschlossen sind; andererseits erlaubt der Nom.\ pendens Konfigurationen~-- \zb solche in (\ref{ex:1-8.1-10}) und (\ref{ex:1-8.1-11})~–, die bei der \isi{Linksversetzung} unmöglich sind. Manche Arten von Konstituententypen in \textit{K\textsubscript{L}} sind nur als Linksversetzungen möglich, \zb Präpositionalphrasen wie in (\ref{ex:1-8.1-1b}) und (\ref{ex:1-8.1-7}). Solche Fälle zeigen, daß die \textit{K\textsubscript{L}}"=Position tatsächlich
zu demselben Satz wie die \textit{K}"=Position gehört und nicht als \isi{reines Diskursphänomen}
zu verstehen ist; denn während die Beispiele von (\ref{ex:1-8.1-8}) unakzeptabel sind, sind entsprechende Satzsequenzen wie (\ref{ex:1-8.1-13}) einwandfrei:
\begin{exe}
\ex\label{ex:1-8.1-13}
\begin{xlist}
\ex\label{ex:1-8.1-13a}  ich fahre nach Dresden; da kann jeder machen, was er will
\ex\label{ex:1-8.1-13b} Hans ist grade in Dresden; da fährt er gerne hin
\end{xlist}
\end{exe}
Bei Possessivpronomen ist es ohnehin klar, daß sie nicht innerhalb von Satzsequenzen an Ausdrücke wie \textit{jeder} in (\ref{ex:1-8.1-5b}) gebunden sein können. An \textit{man} kann ein Possessivum so wie in (\ref{ex:1-8.1-3}) sogar nur innerhalb desselben einfachen Satzes gebunden sein; vgl.\ \citet[§5.2]{Hoehle78a}.

Wegen der Restriktionen für Präpositionalphrasen rechne ich Fälle wie (\ref{ex:1-8.1-14a}) mit
\isi{Präpositionalphrase} in \textit{K\textsubscript{L}} und in \textit{K} zur \isi{Linksversetzung}; solche wie (\ref{ex:1-8.1-14b}) mit \isi{Präpositionalphrase} nur in \textit{K} zum Nom.\ pendens:
%page 64
\begin{exe}
\ex\label{ex:1-8.1-14}
\begin{xlist}
\ex\label{ex:1-8.1-14a} davon, daß das ungerecht ist, davon ist Karl überzeugt
\ex\label{ex:1-8.1-14b} daß das ungerecht ist, davon ist Karl überzeugt
\end{xlist}
\end{exe}
Da eine \isi{Präpositionalphrase} in \textit{K\textsubscript{L}} nur bei \isi{Linksversetzung} möglich ist, muß bei
(\ref{ex:1-8.1-14a}) in \textit{K} eine entsprechende \isi{Präpositionalphrase} stehen. Aber warum ist neben
(\ref{ex:1-8.1-14b}) nicht auch (\ref{ex:1-8.1-15}) möglich?
\begin{exe}
\ex[*]{
\label{ex:1-8.1-15}
daß das ungerecht ist, ist Karl überzeugt}
\end{exe}
Die Erklärung scheint einfach: Die \isi{Konstituente} in \textit{K} muß mit den Selektionseigenschaften des Prädikats des Satzes kompatibel sein, und bei \textit{überzeugt\Hack{\break} sei}- muß die
Angabe des Gegenstands der Überzeugung durch eine \isi{Präpositionalphrase} mit der
\isi{Präposition} \textit{von} ausgedrückt werden. Deshalb finden wir auch (\ref{ex:1-8.1-16a}), aber nicht (\ref{ex:1-8.1-16b}):
\begin{exe}
\ex\label{ex:1-8.1-16}
\begin{xlist}
\ex[]{
\label{ex:1-8.1-16a}
Karl ist davon, daß das ungerecht ist, überzeugt}
\ex[*]{
\label{ex:1-8.1-16b}
Karl ist, daß das ungerecht ist, überzeugt}
\end{xlist}
\end{exe}
%\largerpage
Ähnliche Verhältnisse finden wir beim \isi{Prädikat} \textit{stolz sei-}:
\begin{exe}
\ex\label{ex:1-8.1-17}
\begin{xlist}
\ex[]{
\label{ex:1-8.1-17a}
darauf, daß die Partei gewonnen hat, (darauf) ist er stolz}
\ex[*]{
\label{ex:1-8.1-17b}
daß die Partei gewonnen hat, (das) ist er stolz}
\end{xlist}
\ex
\label{ex:1-8.1-18}
\begin{xlist}
\ex[]{
\label{ex:1-8.1-18a}
Karl ist darauf, daß die Partei gewonnen hat, sehr stolz}
\ex[*]{
\label{ex:1-8.1-18b}
Karl ist, daß die Partei gewonnen hat, sehr stolz}
\end{xlist}
\end{exe}
Dementsprechend finden wir auch (\ref{ex:1-8.1-19}):
\begin{exe}
\ex\label{ex:1-8.1-19}
\begin{xlist}
\ex\label{ex:1-8.1-19a} Karl ist überzeugt davon, daß das ungerecht ist
\ex\label{ex:1-8.1-19b} Karl ist stolz darauf, daß die Partei gewonnen hat
\end{xlist}
\end{exe}
Überraschenderweise ist daneben aber auch (\ref{ex:1-8.1-20}) möglich:
\begin{exe}
\ex\label{ex:1-8.1-20}
\begin{xlist}
\ex\label{ex:1-8.1-20a} Karl ist überzeugt, daß das ungerecht ist
\ex\label{ex:1-8.1-20b} Karl ist stolz, daß die Partei gewonnen hat
\end{xlist}
\end{exe}
In gewissen Fällen kann der Gegenstand der Überzeugung bzw.\ des Stolzes also auch
einfach durch einen \textit{daß}"=Satz angegeben werden ohne eine \isi{Präpositionalphrase} mit
\textit{von} bzw.\ \textit{auf}. Dies könnte den Erklärungsversuch für
(\ref{ex:1-8.1-15})–(\ref{ex:1-8.1-18}) in Zweifel ziehen. Wir kommen darauf in 9.\ [X,
  s.\ Anm.\ S.\,\pageref{fn-herausgeber-topo}] zurück. 

Wenn Beispiele vom Typ (\ref{ex:1-8.1-14b}) zum Nom.\ pendens gehören, gilt auch für diese
Konstruktion, daß die \isi{Konstituente} in \textit{K\textsubscript{L}} in einem gewissen Maß den Selektionseigenschaften des Prädikats genügen muß. So ist der \textit{daß}"=Satz in (\ref{ex:1-8.1-14b}) einwandfrei,
und ein \isi{Interrogativsatz} an seiner Stelle ist ausgeschlossen:
\begin{exe}
\ex[*]{
\label{ex:1-8.1-21}
ob das ungerecht ist, davon ist Karl überzeugt}
\end{exe}
%page 65
Bei dem \isi{Prädikat} in (\ref{ex:1-8.1-22}) ist dagegen ein \isi{Interrogativsatz} möglich und ein \textit{daß}"=Satz
schlecht:
\begin{exe}
\ex\label{ex:1-8.1-22}
\begin{xlist}
\ex[]{
\label{ex:1-8.1-22a}
ob das ungerecht ist, darüber sind wir verschiedener Meinung}
\ex[*]{
\label{ex:1-8.1-22b}
daß das ungerecht ist, darüber sind wir verschiedener Meinung}
\end{xlist}
\end{exe}
\addlines
Offensichtlich hängt das damit zusammen, daß (\ref{ex:1-8.1-16a}) bzw.\ (\ref{ex:1-8.1-23}) genauso wie (\ref{ex:1-8.1-14b})
bzw.\ (\ref{ex:1-8.1-22a}) möglich ist, während (\ref{ex:1-8.1-24}) genauso wie (\ref{ex:1-8.1-21}) unmöglich ist:
\begin{exe}
\ex[]{
\label{ex:1-8.1-23}
wir sind darüber, ob das ungerecht ist, verschiedener Meinung}
\ex[*]{
\label{ex:1-8.1-24}
Karl ist davon, ob das ungerecht ist, überzeugt}
\end{exe}
Damit gilt auch für den Nom.\ pendens, daß er nicht als \isi{reines Diskursphänomen} zu
verstehen ist. Denn innerhalb von Satzsequenzen gelten diese Selektionsbeschränkungen nicht:
\begin{exe}
\ex\label{ex:1-8.1-25}
\begin{xlist}
\ex\label{ex:1-8.1-25a} ich weiß nicht, ob das ungerecht ist; davon ist Karl aber überzeugt
\ex\label{ex:1-8.1-25b} Karl behauptet, daß das ungerecht ist; darüber sind wir aber verschiedener Meinung
\end{xlist}
\end{exe}
In der Sequenz (\ref{ex:1-8.1-25a}) kann sich das \textit{davon} auf den vorhergehenden \textit{ob}"=Satz \textsq{beziehen}, und in der Sequenz (\ref{ex:1-8.1-25b}) kann sich das \textit{darüber} auf den vorhergehenden \textit{daß}"=Satz \textsq{beziehen}; eben jene Bezüge sind in (\ref{ex:1-8.1-21}) und (\ref{ex:1-8.1-22b}) nicht möglich.

%%\addlines[2]
%\largerpage[-1]
Ähnliche Zusammenhänge zeigen sich bei Anaphern von infiniten Verben. Bei
Satzsequenzen sind Fälle wie (\ref{ex:1-8.1-26}) möglich, wo sich das \textit{das} auf ein \isi{Partizip} (\ref{ex:1-8.1-26a}), einen einfachen \isi{Infinitiv} (\ref{ex:1-8.1-26b}) oder auf einen \isi{Infinitiv} mit dem \isi{Präfix} \textit{zu} bezieht (\ref{ex:1-8.1-26c}):
\begin{exe}
\ex\label{ex:1-8.1-26}
\begin{xlist}
\ex\label{ex:1-8.1-26a} hat Karl gearbeitet?~– das sollte er jedenfalls
\ex\label{ex:1-8.1-26b} Karl sollte doch arbeiten~-- das hat er noch nie
%\pagebreak
\ex\label{ex:1-8.1-26c} Karl ist zu verurteilen~-- das wird er auch
\ex\label{ex:1-8.1-26d} Karl scheint zu arbeiten~-- das müßte er gar nicht
\end{xlist}
\end{exe}
Entsprechende infinite Verben in der \textit{K\textsubscript{L}}"=Position sind nur möglich, wenn sie die
Selektionseigenschaften des jeweils regierenden Verbs erfüllen, vgl.\ (\ref{ex:1-8.1-27}) gegenüber
(\ref{ex:1-8.1-28}):
\begin{exe}
\ex\label{ex:1-8.1-27}
\begin{xlist}
\ex[*]{
\label{ex:1-8.1-27a}
gearbeitet, das sollte er jedenfalls}
\ex[*]{
\label{ex:1-8.1-27b}
arbeiten, das hat er noch nie}
\ex[*]{
\label{ex:1-8.1-27c}
zu verurteilen, das wird er auch}
\ex[*]{
\label{ex:1-8.1-27d}
zu arbeiten, das müßte er gar nicht}
\end{xlist}
\end{exe}
\begin{exe}
\ex\label{ex:1-8.1-28}
\begin{xlist}
\ex\label{ex:1-8.1-28a}  arbeiten, das sollte er jedenfalls
%page 66
\ex\label{ex:1-8.1-28b} gearbeitet, das hat er noch nie
\ex\label{ex:1-8.1-28c} verurteilt, das wird er auch
\ex\label{ex:1-8.1-28d} arbeiten, das müßte er gar nicht
\end{xlist}
\end{exe}
Hier ist nicht klar, ob das infinite \isi{Verb} in (\ref{ex:1-8.1-28})
pendent oder \isi{proleptisch} gebraucht ist. Mindestens ist jedoch zu
schließen, daß keine infinite \isi{Verbform}~-- auch nicht der einfache
\isi{Infinitiv}~-- in gleicher Weise eine neutrale Form darstellt, wie es
der \isi{Nominativ} ist, der bei eindeutig pendentem Gebrauch wie in
(\ref{ex:1-8.1-2}) und (\ref{ex:1-8.1-10ii}) auftritt.

In anderer Hinsicht teilt die Konstruktion in (\ref{ex:1-8.1-28}) mit
der \isi{Anapher} \textit{das} in \textit{K} jedoch Eigenschaften mit Diskursen. So
kann das \textit{das} sich häufig nicht auf ein \isi{Verb} allein beziehen, wenn es
mit einem (kasuellen oder präpositionalen) Objekt konstruiert ist;
vgl.\ (\ref{ex:1-8.1-29i}) mit
(\ref{ex:1-8.1-29ii}):
\begin{exe}
\ex\label{ex:1-8.1-29}
\begin{xlisti}
\ex\label{ex:1-8.1-29i}
\begin{xlista}
\ex[*]{
\label{ex:1-8.1-29ia}
bestehen, das sollte er auf seiner Forderung besser nicht}
\ex[*]{
\label{ex:1-8.1-29ib}
befürchten, das müßte er die Wahlniederlage nicht}
\ex[*]{
\label{ex:1-8.1-29ic}
entwischt, das ist er leider nur dem ersten Verfolger}
\ex[]{
\label{ex:1-8.1-29id}
ausweichen, das konnte ich (*dem Verrückten) nicht}
\end{xlista}
\ex\label{ex:1-8.1-29ii}
\begin{xlista}
\ex\label{ex:1-8.1-29iia} Karl besteht auf seiner Forderung~– *das sollte er darauf besser nicht
\ex\label{ex:1-8.1-29iib} Karl befürchtet eine Wahlniederlage~– *das müßte er sie nicht
\ex\label{ex:1-8.1-29iic} er möchte den Verfolgern entwischen~– *das ist er leider nur dem ersten
Verfolger
\ex\label{ex:1-8.1-29iid} du hättest dem Kerl ausweichen sollen~-- das konnte ich (*dem Verrückten) nicht
\end{xlista}
\end{xlisti}
\end{exe}
Ohne \isi{Anapher}, mit dem infiniten \isi{Verb} in der \textit{K}"=Position, sind entsprechende Beispiele
dagegen möglich:\footnote{%
  Ähnliches findet sich bei Ergänzungssätzen. Viele Verben lassen \textit{das} als \isi{Anapher} für
  Diskurse und für die \textit{K\textsubscript{L}}"=Position zu; etwa \textit{versprech-} in
  (\ref{ex:1-fn34i}b) und (\ref{ex:1-fn34i}c): 
  \eal
  \label{ex:1-fn34i}
  \ex
  \label{ex:1-fn34ia}
  weniger zu arbeiten hat Karl versprochen
  \ex
  \label{ex:1-fn34ib}
  weniger zu arbeiten, das hat Karl versprochen
  \ex
  \label{ex:1-fn34ic}
  Karl wollte doch weniger arbeiten~-- das hat er jedenfalls versprochen
  \zl
  Andere wie \textit{sich weiger-} lassen \textit{das} dagegen in keinem von beiden Fällen zu:
  \eal
  \label{ex:1-fn34ii}
  \ex[]{
  \label{ex:1-fn34iia}
  weniger zu arbeiten hat Karl sich strikt geweigert}
  \ex[*]{
  \label{ex:1-fn34iib}
  weniger zu arbeiten, das hat Karl sich strikt geweigert}
  \ex[]{
  \label{ex:1-fn34iic}
  Karl sollte doch weniger arbeiten~– *das hat er sich strikt geweigert}
  \zl
  Im Unterschied zwischen (\ref{ex:1-fn34i}) und (\ref{ex:1-fn34ii}) schlagen sich Unterschiede zwischen den Kookkurrenz\-eigenschaften der Prädikate nieder; diese wirken sich~– ähnlich wie bei (\ref{ex:1-8.1-26})/""(\ref{ex:1-8.1-28}) gegenüber (\ref{ex:1-8.1-29})~-- auf die
  Verwendungsmöglichkeiten im Diskurs und die Möglichkeit einer \textit{K\textsubscript{L}}"=Position
  aus.%
}
\begin{exe}
\ex\label{ex:1-8.1-30}
\begin{xlist}
\ex\label{ex:1-8.1-30a} bestehen sollte er auf seiner Forderung besser nicht
\ex\label{ex:1-8.1-30b} befürchten müßte er die Wahlniederlage nicht
%page 67
\ex\label{ex:1-8.1-30c} entwischt ist er leider nur dem ersten Verfolger
\ex\label{ex:1-8.1-30d} ausweichen konnte ich dem Verrückten nicht
\end{xlist}
\end{exe}
%%\addlines
Die Parallelität zwischen der \textit{K\textsubscript{L}}"=Position und Diskursen gilt auch \zb für Ausdrücke
wie \textit{jed-} und \textit{kein-}. Keine \isi{Anapher} kann sich auf sie beziehen, wenn sie in \textit{K\textsubscript{L}} oder in
einem vorangehenden Diskursabschnitt stehen:
\begin{exe}
\ex\label{ex:1-8.1-31}
\begin{xlist}
\ex\label{ex:1-8.1-31i}
\begin{xlist}
\ex\label{ex:1-8.1-31ia} jeder$_i$ packte seine Sachen~– *der$_i$/""er$_i$ bezahlte seine Rechnung
\ex\label{ex:1-8.1-31ib} keiner$_i$ war damit zufrieden~– *der$_i$/""er$_i$ protestierte
\end{xlist}
\ex\label{ex:1-8.1-31ii}
\begin{xlist}
\ex[*]{
\label{ex:1-8.1-31iia}
jeder, der bezahlte seine Rechnung}
\ex[*]{
\label{ex:1-8.1-31iib}
keiner, der protestierte}
\end{xlist}
\end{xlist}
\end{exe}
Offenbar wirken also beim disjunkten Gebrauch von Konstituenten zwei Arten von
Wohlgeformtheitsbedingungen zusammen. Erstens muß die \isi{Konstituente} in \textit{K\textsubscript{L}} (wie natürlich auch die in \textit{K}) in gewisser Weise den Kookkurrenzrestriktionen der relevanten Konstituenten nach der \textit{K}"=Position genügen; vgl.\ (\ref{ex:1-8.1-22}) und (\ref{ex:1-8.1-27})/""(\ref{ex:1-8.1-28}). Zweitens
müssen die Restriktionen erfüllt sein, die auch innerhalb von Diskursen für den Gebrauch von Anaphern gelten; vgl.\ (\ref{ex:1-8.1-29})/""(\ref{ex:1-8.1-30}) und (\ref{ex:1-8.1-31}). Es ist beachtenswert, daß diese zweite Bedingung in Beispielen wie (\ref{ex:1-8.1-3a}), (\ref{ex:1-8.1-4a}), (\ref{ex:1-8.1-5b}) erfüllt ist; bei den Diskursen in
(\ref{ex:1-8.1-32}) kann man \textit{ihn} in (\ref{ex:1-8.1-32a}) als \isi{Anapher} von \textit{seinen Hund}, \textit{den} in (\ref{ex:1-8.1-32b}) als \isi{Anapher}
von \textit{einen Lieblingsschriftsteller} und \textit{da} in (\ref{ex:1-8.1-32c}) als \isi{Anapher} von
\textit{in seinem Haus} verstehen:\footnote{%
  Bei derartigen "`identity of sense"'"=Anaphern sind unter gewissen Bedingungen, die in Beispielen
  wie (\ref{ex:1-8.1-31ib}) nicht erfüllt sind, auch Antezedenskonstituenten mit \textit{kein-}
  möglich. Vgl.\ (\ref{ex:1-fn35i}), wo man \textit{den} als \textit{einen Lieblingsschriftsteller}
  verstehen kann:
  \ea
  \label{ex:1-fn35i}
  angeblich hat Karl keinen Lieblingsschriftsteller~-- den hat doch jeder!
  \z
  Da diese Bedingungen in der \textit{K\textsubscript{L}}"=Position nie erfüllt sind, sind entsprechende Konstituenten dort genau wie in (\ref{ex:1-8.1-31iib}) unmöglich:
  \eal
  \label{ex:1-fn35ii}
  \ex[*]{
  \label{ex:1-fn35iia}
  keinen Lieblingsschriftsteller, den hat Karl}
  \ex[*]{
  \label{ex:1-fn35iib}
  keinen Lieblingsschriftsteller, den hat doch jeder}
  \zllast%
}
\begin{exe}
\ex\label{ex:1-8.1-32}
\begin{xlist}
\ex\label{ex:1-8.1-32a} seinen$_i$ Hund sollte man$_i$ anständig behandeln~-- auf jeden Fall sollte man ihn regelmäßig füttern
\ex\label{ex:1-8.1-32b} Karl hat einen Lieblingsschriftsteller~-- den hat doch jeder
\ex\label{ex:1-8.1-32c} in seinem$_i$ Haus kann jeder$_i$ machen, was er will~-- da kann man sogar Stinktiere züchten
\end{xlist}
\end{exe}
Wie bei (\ref{ex:1-8.1-28}) ist auch bei anderen Fällen offen, ob die Unterscheidung zwischen pendentem und proleptischem Gebrauch sinnvoll anzuwenden ist. Bei (\ref{ex:1-8.1-33}) steht in \textit{K\textsubscript{L}} jeweils ein \isi{Konditionalsatz} und in \textit{K} \textit{dann} bzw.\ \textit{so}; beide Elemente kann man
als \isi{Anapher} des Konditionalsatzes verstehen, etwa mit der Bedeutung von "`unter
%page 68
dieser Bedingung"':
\begin{exe}
\ex\label{ex:1-8.1-33}
\begin{xlist}
\ex\label{ex:1-8.1-33a} wenn er nachdenken will, dann macht er das Radio an
\ex\label{ex:1-8.1-33b} will er nachdenken, so macht er das Radio an
\end{xlist}
\end{exe}
 Auch in (\ref{ex:1-8.1-34}) kann man die \isi{Konstituente} in \textit{K}~-- \textit{diesen Fehler} bzw.\ \textit{so dumm}~-- als \isi{Anapher} der \isi{Konstituente} in \textit{K\textsubscript{L}}~-- des infiniten Satzes~-- auf"|fassen:
\begin{exe}
\ex\label{ex:1-8.1-34}
\begin{xlist}
\ex\label{ex:1-8.1-34a} sich um diesen Posten zu bewerben, diesen Fehler hat Karl nicht gemacht
\ex\label{ex:1-8.1-34b} sich um diesen Posten zu bewerben, so dumm war Karl nicht
\end{xlist}
\end{exe}
Allerdings sind dies nicht \textsq{reine} Anaphern, sondern Ausdrücke, die zugleich in einem prädikativen Verhältnis zu dem infiniten Satz in \textit{K\textsubscript{L}} stehen. Mit den typischen
Linksversetzungskonstruktionen hat (\ref{ex:1-8.1-34}) keine Ähnlichkeit.

In (\ref{ex:1-8.1-35}) kann man das \textit{desto} bzw.\ \textit{umso} als \isi{Anapher} des mit \textit{je} beginnenden Satzes in \textit{K\textsubscript{L}} betrachten:
\begin{exe}
\ex\label{ex:1-8.1-35}
\begin{xlist}
\ex\label{ex:1-8.1-35a} je teurer die Bahn wird, desto mehr Menschen steigen auf das Auto um
\ex\label{ex:1-8.1-35b} je öfter sie ihren Nachbarn begegnet, umso seltener grüßt sie sie
\ex\label{ex:1-8.1-35c} je weniger Autos dort fahren, desto teurer werden die Grundstücke
\end{xlist}
\end{exe}
Ähnlich wie bei (\ref{ex:1-8.1-10}) ist die \isi{Anapher} in der \isi{Konstituente} in \textit{K} echt enthalten.

Offenbar kann man die \textit{K\textsubscript{L}}"=Position iterieren, derart daß jede derartige \isi{Konstituente} eine \isi{Anapher} der vorhergehenden enthält:
\begin{exe}
\ex\label{ex:1-8.1-36}
\begin{xlist}
\ex\label{ex:1-8.1-36a}  dieser Kerl, mit dessen Freundin, mit der habe ich neulich gesprochen
\ex\label{ex:1-8.1-36b} dieser Kerl, dessen Freundin, mit der habe ich neulich gesprochen
\ex\label{ex:1-8.1-36c} dem die Reporter da nachlaufen, der Kerl, das ist der Kommissar
\end{xlist}
\end{exe}
In (\ref{ex:1-8.1-36a}) haben wir in erster Position einen
Nom.\ pendens und in zweiter Position eine \isi{Linksversetzung}; bei
(\ref{ex:1-8.1-36b}) ist in erster und in zweiter Position ein
Nom.\ pendens. Wenn man als \isi{Linksversetzung} nur Fälle mit
\textsq{reiner} \isi{Anapher} betrachtet, sind derartige Iterierungen bei
ihr ausgeschlossen aufgrund der bei (\ref{ex:1-8.1-10}) besprochenen
Beschränkung.

Nach diesen Erörterungen ist das Deklarativsatzschema (\ref{ex:1-3-29}) durch (\ref{ex:1-8.1-37}) zu ersetzen:
\begin{exe}
\ex\label{ex:1-8.1-37}
(\textit{K\textsubscript{L}}*)$^{\smallfrown}$\textit{K}$^{\smallfrown}$\textsuperscript{f}V$>$(\textit{KM}*)$>$(\textsuperscript{i}V*)
\end{exe}
Dabei müssen, wenn eine oder mehrere \textit{K\textsubscript{L}}"=Positionen gefüllt sind, \textit{K} und \textit{K\textsubscript{L}} eine
\isi{Anapher} der \isi{Konstituente} in der unmittelbar vorhergehenden \textit{K\textsubscript{L}}"=Position echt oder
unecht enthalten. Sätze, die dieses \isi{Schema} erfüllen, nenne ich wie bisher F2"=Sätze.


\subsection{Satzanknüpfung}%8.2.
\label{subsec:1-8.2}\label{sec-satzanknuepfung}

\addlines% checked 2018
%page 69
Man kann F2"=Sätze auf verschiedene Weise mit anderen Sätzen verknüpfen. Eine
Möglichkeit ist die Asyndese wie in (\ref{ex:1-8.2-1a}); eine zweite die Verwendung von Partikeln
wie \textit{und, oder, aber, doch, sondern} in (\ref{ex:1-8.2-1b})–(\ref{ex:1-8.2-1f}):
\begin{exe}
\ex\label{ex:1-8.2-1}
\begin{xlist}
\ex\label{ex:1-8.2-1a} Karl füttert den Hund, Maria füttert die Katze
\ex\label{ex:1-8.2-1b} Karl füttert den Hund, und Maria füttert die Katze
\ex\label{ex:1-8.2-1c} Karl füttert den Hund, oder Maria füttert die Katze
\ex\label{ex:1-8.2-1d} Karl füttert den Hund, aber Maria füttert die Katze
\ex\label{ex:1-8.2-1e} Karl füttert den Hund, doch Maria füttert die Katze
\ex\label{ex:1-8.2-1f} Karl füttert nicht den Hund, sondern er füttert die Katze
\end{xlist}
\end{exe}
Alle diese Verknüpfungsmöglichkeiten sind aus anderen Kontexten (wo sie Satzbestandteile verknüpfen) als koordinierende Verknüpfungen bekannt, und wie man es
bei koordinierenden Verknüpfungen erwartet, erlauben sie Linkstilgungen wie in (\ref{ex:1-8.2-2})
und \isi{Gapping} wie in (\ref{ex:1-8.2-3}):

\begin{exe}
\ex\label{ex:1-8.2-2}
\begin{xlist}
\ex\label{ex:1-8.2-2a} Karl füttert \_\_\_ und Maria tränkt den Hund
\ex\label{ex:1-8.2-2b} Karl zeigt nicht seiner Tante \_\_\_, sondern er zeigt seinem Onkel die Briefmarkensammlung
\end{xlist}
\end{exe}
\begin{exe}
\ex\label{ex:1-8.2-3}
\begin{xlist}
\ex\label{ex:1-8.2-3a} Karl füttert den Hund und Maria \_\_ die Katze
\ex\label{ex:1-8.2-3b} Karl füttert den Hund, aber Maria \_\_ die Katze
\end{xlist}
\end{exe}
(Mehr über Koordinationsphänomene in X [s.\ Anm.\ S.\,\pageref{fn-herausgeber-topo}].)

Da die Partikeln in (\ref{ex:1-8.2-1}b-f) verwendet
sind, um jeweils zwei Sätze miteinander zu verknüpfen, könnte man
vermuten, daß sie zwischen den verknüpften Sätzen stehen und nicht mit
einem der verknüpften Sätze eine Einheit bilden. Dies scheint
\zb Nordmeyer zu meinen:

\begin{exe}
\ex\label{ex:1-8.2-4}
"`Konjunktionen sind bekanntlich weiter nichts als Wörter, welche Sätze
mit einander verbinden oder deren Verhältnis zu einander angeben [\ldots].

Sind sie aber wirklich nichts anderes, so folgt daraus ganz unabweisbar,
daß sie nicht Teile der Sätze sind, welche von ihnen eingeleitet werden,
daß sie sich also nicht in sondern \so{vor} denselben befinden."' \citep[4]{Nordmeyer1883}
\end{exe}
(Ähnlich \citealt[35ff]{Drach1937}.) Die hier versuchte Deduktion ist jedoch nicht erfolgreich. Aus der satzverknüpfenden Funktion der Partikeln folgt keineswegs, daß sie
außerhalb der von ihnen verknüpften Sätze stehen. So ist hinsichtlich der Satzverknüpfungsfunktion kein markanter Unterschied zwischen \textit{und} und \textit{obendrein} zu erkennen; trotzdem verhält sich \textit{obendrein} völlig anders als \textit{und}:


%page 70
\begin{exe}
\ex\label{ex:1-8.2-5}
\begin{xlist}
\ex[*]{
\label{ex:1-8.2-5a}
Karl füttert den Hund, obendrein Maria füttert den Kater}
\ex[]{
\label{ex:1-8.2-5b}
Karl füttert den Hund, obendrein füttert Maria den Kater}
\end{xlist}
\end{exe}
%\largerpage
Und neben der Verwendung von \textit{doch} in (\ref{ex:1-8.2-1e}) gibt es die Verwendung in (\ref{ex:1-8.2-6a}), wo
\textit{doch} offenbar in der \textit{K}"=Position steht; dieselbe Position hat das~-- zweifellos satzverknüpfende~– \textit{noch} in (\ref{ex:1-8.2-6b}) inne (und \textit{noch} kann nicht wie \textit{doch} in (\ref{ex:1-8.2-1e}) vor der \textit{K}"=Position stehen):

\begin{exe}
\ex\label{ex:1-8.2-6}
\begin{xlist}
\ex\label{ex:1-8.2-6a} Karl füttert den Hund, doch füttert niemand die Katze
\ex\label{ex:1-8.2-6b} Karl füttert weder den Hund, noch füttert er die Katze
\end{xlist}
\end{exe}
Außer der Position des \textit{aber} in (\ref{ex:1-8.2-1d}) sind auch die Positionen in
(\ref{ex:1-8.2-7}) möglich:\footnote{%
  Es finden sich auch Beispiele wie (i):
  \ea
  \label{ex:1-fn37i}
  sicherlich füttert Karl bald den Hund; oder aber er kommt, um den Ochsen zu tränken
  \z
  Diese Kombination von \textit{oder} und \textit{aber} fällt völlig aus dem Muster, das sich sonst beobachten
  läßt, und wir gehen darauf nicht ein.%
}
\begin{exe}
\ex\label{ex:1-8.2-7}
\begin{xlist}
\ex\label{ex:1-8.2-7a} Karl füttert den Hund, Maria aber füttert die Katze
\ex\label{ex:1-8.2-7b} Karl will den Hund füttern, Maria will ihn aber auch füttern
\end{xlist}
\end{exe}
Auch hier ist kein markanter Unterschied in der Satzverknüpfungsfunktion der
\textit{doch} in (\ref{ex:1-8.2-1e}) und (\ref{ex:1-8.2-6a}) oder der \textit{aber} in (\ref{ex:1-8.2-1d}) und (\ref{ex:1-8.2-7}) zu erkennen, und wieso das
\textit{noch} von (\ref{ex:1-8.2-6b}) diese (und nur diese) Position innehaben kann, ist aus seiner Funktion nicht ohne weiteres zu schließen. Offenbar ist es nicht möglich, die topologischen Eigenschaften der Partikeln von (\ref{ex:1-8.2-1}) allein aufgrund ihrer satzverknüpfenden
Funktion zu erschließen. Wenn man mit \citet[36]{Drach1937} den Partikeln von (\ref{ex:1-8.2-1}) einen
"`besonderen Bewußtseinsinhalt"' zuspricht, ist diese spezielle Eigenschaft offenbar
aus ihrem topologischen Verhalten erschlossen, nicht umgekehrt.

Was man an (\ref{ex:1-8.2-1}) sieht, ist, daß diese Partikeln nicht die Position \textit{K} einnehmen. Daraus folgt jedoch nicht, daß sie nicht zu demselben Satz wie die nachfolgende \textit{K}"=Position gehören. Auch die Position \textit{K\textsubscript{L}} steht vor der \textit{K}"=Position und gehört doch, wie
wir gesehen haben, zu demselben Satz wie sie.

Ein Indiz dafür, daß diese Partikeln Teil des zweiten Satzes sind, ergibt sich aus
parenthetischen Sätzen wie (\ref{ex:1-8.2-8}):
\begin{exe}
\ex\label{ex:1-8.2-8}
\begin{xlist}
\ex\label{ex:1-8.2-8a}  die Menschen in unserem Lande sehnen sich~-- und ich sage das aus tiefer innerer Überzeugung~-- nach geistig"=moralischer Führung
\ex\label{ex:1-8.2-8b} Karl hat~-- aber das weißt du vielleicht schon~-- gestern den Hund gefüttert
\ex\label{ex:1-8.2-8c} Karl muß~– oder vielleicht sollte ich besser sagen: er darf~-- den Hund
füttern
\end{xlist}
\end{exe}
%page 71
\addlines[-1]
Es besteht kein Zweifel, daß \textit{und}, \textit{aber}, \textit{oder} hier jeweils Teil des parenthetischen Satzes sind, und sie haben innerhalb dieser Parenthesen genau dieselben topologischen Eigenschaften wie in (\ref{ex:1-8.2-1}). Der Schluß ist unvermeidbar, daß die Partikeln
von (\ref{ex:1-8.2-1}) innerhalb von Deklarativsätzen eine Anfangsposition innehaben, die ich
durch "`KOORD"' symbolisiere. Ihre Funktion nennt man vielleicht besser Anknüpfung als Verknüpfung.

Interessanterweise scheint es sonst kaum irgendwelche zwingende Evidenz dafür
zu geben, daß die Partikeln Teil des angeknüpften Satzes sind. Dies macht ein Lernbarkeitsproblem deutlich: Wie kommt ein Sprecher dazu, daß er Beispiele wie (\ref{ex:1-8.2-8})
ohne weiteres für einwandfrei hält? Es ist kaum anzunehmen, daß derartige Sätze in
den Stimulusmengen, auf deren Grundlage der Sprachlerner die Regeln für Sätze
wie (\ref{ex:1-8.2-1}) erwirbt, nennenswert vertreten sind. Möglicherweise wird der Sprachlerner
durch ein Prinzip wie (\ref{ex:1-8.2-9}) geleitet:
\begin{exe}
\ex\label{ex:1-8.2-9}
Wenn ein sprachliches Element \textit{A} die Funktion hat, sprachliche Elemente
\textit{B\textsubscript{1}} und \textit{B\textsubscript{2}} zu verknüpfen, ist \textit{A} immer Teil einer \isi{Konstituente} \textit{C}, die \textit{B\textsubscript{i}},
aber nicht \textit{B\textsubscript{j}} enthält. (\textit{i}, \textit{j} ∈ \{1, 2\})
\end{exe}
Verknüpfende Elemente kann man logisch häufig als Ausdrücke für symmetrische
Relationen zwischen den verknüpften Elementen analysieren. Dies gilt für die Partikeln von (\ref{ex:1-8.2-1}), aber auch für die Satzkonjunktion \textit{während} und die gleichlautende
temporale \isi{Präposition}. Nach (\ref{ex:1-8.2-9}) haben auch bei solchen logisch symmetrischen Relationen die verknüpften Elemente \textit{B\textsubscript{1}} und \textit{B\textsubscript{2}} syntaktisch niemals ein gleichartiges
Verhältnis zu dem verknüpfenden Element A. Es ist eine offene Frage, ob das aus tieferen Prinzipien
folgt; aber es scheint eine Tatsache zu sein.\footnote{%
  Wenn (\ref{ex:1-8.2-9}) in voller Allgemeinheit gilt, wie es von einem Prinzip zu erwarten ist,
  das der Sprachlerner beim Auf"|bau seines sprachlichen Regelsystems voraussetzt, ist zu erwägen,
  daß es dann wohl auch für Verben als verknüpfende Elemente zu gelten hat. Ein $n$-stelliges \isi{Verb}
  kann man als Element auf"|fassen, das $n$ Elemente (\isi{Subjekt}, \textsq{direktes Objekt},
  evtl.\ \textsq{indirektes Objekt} usw.) miteinander verknüpft; logisch kann man es als Ausdruck
  einer n-stelligen Relation analysieren. Falls (\ref{ex:1-8.2-9}) für Verben gilt, müßte daraus
  folgen, daß Verben immer Teil einer \isi{Konstituente} C sind, die (mindestens) eins der verknüpften
  Elemente (B\textsubscript{\textit{i}}, \zb ein Objekt) enthält und eins der verknüpften
  Elemente (B\textsubscript{\textit{j}}, \zb das \isi{Subjekt}) nicht enthält.%
}

Aus (\ref{ex:1-8.2-9}) folgt nicht, welches der verknüpften Elemente~-- \textit{B\textsubscript{1}} oder \textit{B\textsubscript{2}}~– Teil von \textit{C}
ist. Bei den anknüpfenden Partikeln kann der Sprachlerner jedoch intonatorische Indizien benutzen. Der Satz vor der \isi{Partikel} hat häufig eine \isi{Intonation}, die er auch hätte, wenn kein weiterer Satz angeknüpft wäre; niemals endet ein Satz mit einer derartigen \isi{Partikel} und satzschließender \isi{Intonation}. Und zwischen einer solchen \isi{Partikel}
und dem durch sie angeschlossenen Satz ist zwar ein starker intonatorischer Bruch
mit Pause möglich, aber das ist untypisch; gewöhnlich ist die \isi{Partikel} vom angeknüpften Satz \isi{intonatorisch} nicht abgehoben. Hinsichtlich der \isi{Intonation} verhalten
%page 72
sich die Partikeln also zum angeknüpften Satz ganz anders als zum vorhergehenden
Satz. Wenn der Sprachlerner (\ref{ex:1-8.2-9}) voraussetzt, ist es naheliegend, daß er diese
Asymmetrie im intonatorischen Verhalten als entsprechende Asymmetrie in der syntaktischen
Zugehörigkeit deutet und die \isi{Partikel} als Teil des angeknüpften Satzes analysiert.\footnote{%
  Genau das gleiche gilt für die \isi{Koordination} von Satzbestandteilen, etwa Nominalphrasen: Die
  koordinierenden Partikeln sind Teil des angeknüpften Bestandteils. Im Japanischen\il{Japanisch} sind die
  entsprechenden Partikeln Teil der vorangehenden \isi{Nominalphrase}. (Satzanknüpfende koordinierende
  Partikeln gibt es im Japanischen\il{Japanisch} nicht; vgl.\ \citet[§8]{Kuno1973}.) Zweifellos hängt das damit
  zusammen, daß das Japanische\il{Japanisch} zur Markierung syntaktischer Funktionen (fast) ausschließlich
  postpositionale Partikeln verwendet. Dies spricht für die Verwandtschaft zwischen koordinierenden
  Partikeln und Satzkonjunktionen sowie (zu Nominalphrasen tretenden) Adpositionen und für die
  Korrektheit von (\ref{ex:1-8.2-9}).%
}
Dabei ist zu beachten, daß diese Analyse ohne Zuhilfenahme von (\ref{ex:1-8.2-9}) nicht
erzwungen würde: Das intonatorische Verhalten der Partikeln wäre mit einem syntaktisch symmetrischen Verhältnis der \isi{Partikel} zu vorangehendem und angeknüpftem Satz durchaus verträglich.

Es erhebt sich die Frage, in welcher Reihenfolge die Positionen KOORD und \textit{K\textsubscript{L}}
auftreten können. Aus (\ref{ex:1-8.2-10}) geht hervor, daß \textit{K\textsubscript{L}} auf KOORD folgt; andere Folgen als
in (\ref{ex:1-8.2-10}) sind nicht möglich:
\begin{exe}
\ex\label{ex:1-8.2-10}
\begin{xlist}
\ex\label{ex:1-8.2-10a}  (Karl spielt mit der Katze) doch mit dem Hund, mit dem spielt Heinz
\ex\label{ex:1-8.2-10b} (wenn man nach Hause kommt) und vor der Tür, da steht der Gerichtsvollzieher (dann ist die gute Laune hin)
\ex\label{ex:1-8.2-10c} (entweder stimmt das) oder dein Freund, der hat Recht
\ex\label{ex:1-8.2-10d} (Karl soll nicht schlafen), aber arbeiten, das soll er
\end{xlist}
\end{exe}
Dadurch, daß KOORD hier vor der \textit{K\textsubscript{L}}"=Position
desselben Satzes auftritt, ergibt sich ein weiteres Indiz dafür, daß
\textit{K\textsubscript{L}} Teil desselben Satzes wie \textit{K}
ist. Man kann aber schwerlich annehmen, daß Beispiele wie
(\ref{ex:1-8.2-10}) in typischen Stimulusmengen so reichlich
repräsentiert sind, daß sie die Sprachlerner zu diesem Schluß führen
würden. Vermutlich gilt umgekehrt: Aufgrund von Indizien, die bislang
nicht aufgeklärt sind, analysiert der Sprachlerner die
\textit{K\textsubscript{L}}"=Position als Teil von F2"=Sätzen, und
ergänzend zu (\ref{ex:1-8.2-9}) gilt ein Prinzip (\ref{ex:1-8.2-11}):
\begin{exe}
\ex\label{ex:1-8.2-11}
Im typischen Fall nimmt ein verknüpfendes Element \textit{A} eine Position an
der Peripherie von \textit{C} ein.
\end{exe}
Dieses Prinzip führt dazu, daß die verknüpfenden Partikeln in
(\ref{ex:1-8.2-1}) als Anfangselemente der angeknüpften Sätze
analysiert werden. Beispiele wie (\ref{ex:1-8.2-7}) müssen dann als
untypische Fälle aufgefaßt werden; der Erwerb der für sie geltenden
Regeln ist nur aufgrund geeigneter Beispiele in der Stimulusmenge
möglich. Beispiele wie (\ref{ex:1-8.2-6}) sind mit (\ref{ex:1-8.2-11})
voll kompatibel; allerdings stehen die Konjunktionen hier nicht in der
%page 73
KOORD"=Position, sondern in der \textit{K}"=Position. Dies ist für den Sprachlerner leicht erkennbar, wenn er entsprechend den Überlegungen von \ref{subsec:1-7.2} die Regeln für einfache
Sätze vor den Regeln für komplexe Sätze erwirbt. Daß ein satzanknüpfendes Element wie \textit{obendrein} nicht wie in (\ref{ex:1-8.2-5}) auf die \textit{K}"=Position beschränkt ist (sondern
auch in einer \textit{KM}"=Position auftreten kann), ist vermutlich ähnlich wie das Verhalten
von \textit{aber} in (\ref{ex:1-8.2-7}) nur aufgrund positiver Beispiele erkennbar.

Nach diesen Beobachtungen ist das \isi{Schema} (\ref{ex:1-8.1-37}) durch (\ref{ex:1-8.2-12}) zu ersetzen:
\begin{exe}
\ex\label{ex:1-8.2-12}
(KOORD)$^{\smallfrown}$(\textit{K\textsubscript{L}}*)$^{\smallfrown}$\textit{K}$^{\smallfrown}$\textsuperscript{f}V$>$(\textit{KM}*)$>$(\textsuperscript{i}V*)
\end{exe}
Dabei kann die Position KOORD von den koordinierenden Konjunktionen eingenommen werden, die in (\ref{ex:1-8.2-1}) auftreten. Sätze, die dieses \isi{Schema} erfüllen, sollen weiter
F2"=Sätze heißen.

Es ist beachtenswert, daß es darüber hinaus weitere Partikeln gibt, die in KOORD
auftreten können. Die bekannteste ist \textit{denn} wie in (\ref{ex:1-8.2-13}):
\begin{exe}
\ex\label{ex:1-8.2-13}
(Karl spielt mit der Katze) denn (mit dem Hund) mit dem spielt Heinz
\end{exe}
Diese \isi{Partikel} ist mit keiner satzanknüpfenden koordinierenden \isi{Partikel} kombinierbar und nimmt genau ihre Stelle ein. Sie kann aber nicht selbst zu den koordinierenden Konjunktionen gerechnet werden, aus zwei Gründen. Erstens finden wir \textit{denn}
sonst niemals in koordinierender Funktion; \textit{denn} verknüpft keine Satzbestandteile. Selbst die Anknüpfung eines F2"=Satzes an einen anderen F2"=Satz mittels \textit{denn}
scheint schlecht zu sein, wenn diese Verknüpfung von F2"=Sätzen \isi{eingebettet}~-- also
echter Bestandteil eines Satzes~-- ist; vgl.\ die eingebetteten Beispiele mit \textit{denn} in
(\ref{ex:1-8.2-14}) mit den uneingebetteten in (\ref{ex:1-8.2-15}) und mit den eingebetteten mit \textit{und} in (\ref{ex:1-8.2-16}):
\begin{exe}
\ex\label{ex:1-8.2-14}
\begin{xlist}
\ex[*]{
\label{ex:1-8.2-14a}
weil Karl meint, wir sollten nach Hause gehen, denn es ist/""sei schon spät, wollen wir bald aufbrechen}
\ex[*]{
\label{ex:1-8.2-14b}
daß Karl glaubt, die Temperatur müsse steigen, denn die Tage würden länger, wissen wir alle}
\end{xlist}
\end{exe}
\begin{exe}
\ex\label{ex:1-8.2-15}
\begin{xlist}
\ex\label{ex:1-8.2-15a} wir sollten nach Hause gehen, denn es ist schon spät
\ex\label{ex:1-8.2-15b} die Temperatur muß steigen, denn die Tage werden länger
\end{xlist}
\end{exe}
\begin{exe}
\ex\label{ex:1-8.2-16}
\begin{xlist}
\ex\label{ex:1-8.2-16a} weil Karl meint, wir sollten nach Hause gehen und es sei schon spät, wollen wir bald aufbrechen
\ex\label{ex:1-8.2-16b} daß Karl glaubt, die Temperatur müsse steigen und die Tage würden länger, wissen wir alle
\end{xlist}
\end{exe}
Zweitens sind bei Satzanknüpfungen mittels \textit{denn} keinerlei Linkstilgungs- oder
\isi{Gapping}"=Phänomene möglich, die man bei einer koordinierenden \isi{Konjunktion} 
%page 74
erwarten würde:
\begin{exe}
\ex\label{ex:1-8.2-17}
\begin{xlist}
\ex[*]{
\label{ex:1-8.2-17a}
Karl füttert \_\_\_, denn Maria tränkt die Katze}
\ex[*]{
\label{ex:1-8.2-17b}
Karl füttert die Katze, denn Maria \_\_ den Hund}
\end{xlist}
\end{exe}
Viele Sprecher verwenden \textit{außer} wie in (\ref{ex:1-8.2-18}):
\begin{exe}
\ex\label{ex:1-8.2-18}
Karl geht jeden Sonntag zur Kirche, außer er hat was besseres vor
\end{exe}
Der durch \textit{außer} angeknüpfte F2"=Satz hat semantisch den Charakter eines Konditionalsatzes, nicht eines Deklarativsatzes. Die Konstruktion hat manche weiteren Besonderheiten, und es ist nicht ganz klar, ob \textit{außer} wirklich die KOORD"=Position
einnimmt, wie es in (\ref{ex:1-8.2-18}) der Fall zu sein scheint.

Manche Sprecher können \textit{als} wie in (\ref{ex:1-8.2-19}) verwenden:
\begin{exe}
\ex\label{ex:1-8.2-19}
es ist mir lieber, wir gehen nach Hause, als wir werden verprügelt
\end{exe}
Es scheint, daß hier der F2"=Satz \textit{wir werden verprügelt} durch \textit{als} an den F2"=Satz \textit{wir gehen nach Hause} angeknüpft wird und daß \textit{als} die KOORD"=Position innehat. Wie in (\ref{ex:1-8.2-18}) haben die F2"=Sätze auch hier den semantischen Charakter von Konditionalsätzen, und es ergeben sich zahlreiche Fragen, die in das komplizierte Gebiet der Komparationskonstruktionen hineinführen. Ich will diese und andere sich aufdrängende Fragen hier nicht verfolgen.

Wir haben festgestellt, daß \textit{K\textsubscript{L}} und KOORD
Positionen sind, die zu demselben einfachen Satz gehören wie die
darauf folgende \textit{K}"=Position. Die Sequenz "`(KOORD)$^{\smallfrown}$(\textit{K\textsubscript{L}}*)$^{\smallfrown}$\textit{K}"' in
(\ref{ex:1-8.2-12}) will ich als Vorderfeld bezeichnen. Diese Bezeichnung ist nicht allgemein
üblich.\footnote{\label{fn:1-39}%
  Zur Geschichte der üblichen "`Feld"'-Terminologie und zu ihrem Verhältnis zu meinen terminologischen
  Vorschlägen vgl.\ X [s.\ Anm.\ S.\,\pageref{fn-herausgeber-topo}].%
}
Das liegt zum Teil einfach daran, daß die \textit{K\textsubscript{L}}"=Position in der Literatur nicht viel Beachtung gefunden hat und die KOORD"=Position häufig nicht als Teil des Satzes betrachtet wird. In gewissem Maße ist das auch verständlich. Zwar können in F2"=Sätzen, wie wir gesehen
haben, erheblich mehr Konstituenten im Vorderfeld stehen als die eine
\isi{Konstituente} in \textit{K}~-- und ob und in welchem Sinne dort genau 1
\isi{Konstituente} steht, ist noch zu überprüfen~–, aber die anderen
Konstituenten sind (i)~fakultativ und (ii)~in ihrer Art und in ihrem
Verhältnis zueinander außerordentlich scharf restringiert, und in
völlig anderer Weise restringiert, als es die Konstituenten
\textit{KM} vor dem \isi{Verb} am Ende des Satzes sind. Es ist durchaus
angemessen, wenn man die Position \textit{K} als den wesentlichen Teil
des Vorderfelds von F2"=Sätzen betrachtet.

\setcounter{footnote}{0}
\renewcommand*{\thefootnote}{YZ}

\vspace*{\baselineskip}
~\hfill (wird fortgesetzt) [s.\ Anm.\ S.\ \pageref{fn-herausgeber-topo}]

\addcontentsline{toc}{section}{Literatur}

\sloppy
\printbibliography[heading=subbibliography,notkeyword=this]
\refstepcounter{mylastpagecount}\label{chap-topo-end}
\end{document}
