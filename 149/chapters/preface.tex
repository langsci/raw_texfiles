%% -*- coding:utf-8 -*-

\chapter{Preface}


\section*{About this volume: introductory remarks}

\begin{refsection}

The idea for this volume was born in 2014 when Stefan Müller reread Tilman N. Höhle's work on
\emph{Topologische Felder} from 1983, and thought it a shame that this fundamental work on German clause
structure was still unpublished. Talking over his plan to change this with Marga Reis and Frank
Richter, who were to become the co-editors of this volume, drew the attention to further important
but unpublished Höhle papers from the eighties and nineties that likewise deserved publication. This
spawned the plan for a bigger volume comprising these papers as well. But it did not take long to
see that, ideally, this volume should also include most of Höhle's already published work: There
are exceedingly close connections between his unpublished and published papers as to topics,
content, theoretical outlook and aims that an attentive reader would want to trace and should be
able to trace easily. This led to the conception of the present volume, which, certain difficulties
notwithstanding (see Postscript, pp.~\pageref{page-postscript} ff.), we pursued steadfastly and finally brought to completion this year. 

\medskip
Before turning to the contents of this volume let us briefly turn to its author and to our motives for (re)publishing his work.  

Tilman N. Höhle, born 1945, studied General Linguistics, Indo-European Linguistics, and German
Philology at the University of Göttingen and the University of Cologne, where he received his
M.A. (1969) and his PhD (1976). Having taught at the German Seminar of the University of Cologne for
a couple of years, he changed to the University of Tübingen in 1984 where, besides teaching German
linguistics, he was involved in training several generations of general and computational linguists
in grammatical theory as well as theoretically oriented descriptive German grammar. A complete list
of his publications is contained in the list of references, pp.~\pageref{page-hoehle-refs} ff. He retired in 2008.

Like many German linguists starting their studies in the sixties and seventies Höhle embraced
Generative Grammar as the most promising way of doing linguistics, and he remained committed
throughout his career to its central theoretical and methodological goals (which later on he found
better realized in Head-Driven Phrase Structure Grammar (HPSG) than in generative linguistics following the Minimalist Program). Without striving for academic prominence he soon became one of the most respected figures, a true grey eminence, in the German generative scene. His written work covers a wide range of syntactic topics, in particular topological and related aspects of clause structure (topological fields and topological clause types, non-finite constructions, coordination, extraction, constituent order, focus projection, verum focus), but also aspects of word syntax, the lexicon, and phonological phenomena, as well as broader issues such as lexicalist syntax, reconstruction, theoretical aspects of phonology, in particular in model-theoretic grammar (HPSG). All of it was highly influential in shaping a theoretically and empirically well-founded grammar of German but also contributed significantly to grammatical theory, in general and in its HPSG variant. 

Linguistics is a fast-moving discipline, so the eighties and nineties of the last century are already history. Still, Höhle's work -- and this is the main motive for the present volume -- is not just historically important we believe, but also worth knowing for contemporary linguists, especially those interested in the grammar of German within the Germanic context. The clearest case in point is \emph{Topologische Felder}, so far unpublished and rather inaccessible, whose wealth of descriptive and theoretical insights still remains to be fully appreciated. But even in the cases in which Höhle papers, published or not, initiated a lively international debate and are still frequently cited (think, e.g., of his papers on asymmetric coordination \citeyearpar{Hoehle1983,Hoehle90a} or verum focus \citeyearpar{Hoehle1988,Hoehle92}), (re)reading the originals leads to observations and ideas worth pursuing that have not found their way into contemporary literature. 

\medskip
\addlines
Let us now turn to the contents of the present volume. 
In order to make the project manageable, we did not include all of Höhle's papers but concentrated
on the -- to our mind central -- contributions to grammar in the narrower sense of (morpho-)syntax
and grammatical theory. Thus, we set aside the early phonological papers \citep{g_Hoehle78b,Hoehle82c}, likewise papers that are, in various ways,
pre-versions to later, often more comprehensive studies on the same topic; this led to the exclusion
of \citew{g_Hoehle79, Hoehle1982a, Hoehle1988, Hoehle96} in favor of
\citew{Hoehle1982,Hoehle1985,Hoehle92,Hoehle2000a} respectively, which are all included here. With these provisos, the present volume is a complete collection of Höhle's work on German grammar and grammatical theory (apart, of course, from his dissertation, published as \citew{Hoehle78a}).

\medskip
The volume is organized in two parts. Part I consists exclusively of \textbf{Topologische Felder} (Topological fields) (=
Chapter~\ref{chap-topo} in this volume) a book-length work written in 1983, which remained unfinished but circulated as a `grey paper' in the generative community. It is a fundamental study of German clause structure in that it establishes in detail the topological properties of German sentences and how they constitute the basic clause types of German. The study also pays detailed attention to the left-peripheral topological extensions of clause types, which includes a thorough discussion of `left dislocation' phenomena and pertinent remarks on coordination. This descriptive enterprise is a) embedded in a critical comparison with Greenberg's word order typology, which is shown to be unable to capture the essentials of German clause structure; b) supplemented by explanatory endeavors turning on astute arguments of learnability; c) enriched by a historical excursus showing that the correct topological picture of German clauses (although sometimes coupled with false beliefs in `subject inversion') had already been achieved in the 19th century (hence Höhle also calls it the `Herling/Erdmann system'); even the idea that the true verb position is clause-final can already be found in Herling's writings. These insights were soon forgotten; it was not before the sixties/seventies of the last century that (more or less independently of this tradition) they came to life again.

\emph{Topologische Felder} is foundational for most of the papers assembled in part II, which justifies its exclusive position in this volume. 

\medskip
Part II (`Selected papers') collects the remaining 17 papers, which are as a rule
presented in chronological order; however, the 5 papers with a distinctly HPSG orientation are grouped together at the end. Our short presentations of their contents follow this order.  

\textbf{Empirische Generalisierung vs.\ ,Einfachheit‘. Zur Zuordnung zwischen formalen und
  logischen Eigenschaften von Sätzen im Deutschen} (Empirical generalization
vs.\ `simplicity'. On the mapping between formal and logical properties of
sentences in German, 1980) (= Chapter 2 in this volume). In this
short paper Höhle argues forcefully against mistaking the form of logical representations of
sentences for their syntactic structure, thereby also demonstrating that the autonomy of syntax
manifests itself most clearly in topological regularities -- wherefore “this part of grammar seems
to merit the utmost theoretical interest” (p.~\pageref{page-besonderes-theoretisches-interesse}). It is clear that this conviction drives Höhle's linguistic research in the following decades.

\textbf{Explikationen für "`normale Betonung"' und "`normale Wortstellung"'}
(Explications of ``normal stress'' and ``normal word order'', 1982) (=
Chapter~\ref{chap-normale-Betonung} in this volume). Unlike the verbal placement patterns involved
in forming topological clause types, the ordering patterns for nonverbal constituents are variable
in German. Nonetheless, there was always the intuition that for every constituent constellation
there are (more or less) `normal' orders, but, as Höhle makes abundantly clear, a satisfactory explication of this intuition is nowhere given. His own explication makes crucial use of the notions `focus' and `focus projection', and proceeds in the two stages indicated by the title of this paper: (i) A sentence S$_i$ has `stylistically normal stress contour' iff it has more possible foci than any other stress contour variant of S$_i$. (ii) A sentence S$_i$ has `stylistically normal word order' iff, given an appropriate stress contour, it has more possible foci than any other ordering or stress contour variant of S$_i$. The reference to `possible foci' ensures that these explications belong to sentence grammar, yet implies, at the same time, that both are inherently pragmatic concepts, for having more possible foci than the respective variants means being able to occur in more context types. This also affords a rather natural \emph{explanation} for the intuition of normalcy. 

This long paper is hard reading but rewarding, in addition to the above, not only for the many observations and generalizations deduced from the above explications but also for its critical discussion of a structural explication of `normal word order', which still does not seem outdated.  

\textbf{Subjektlücken in Koordinationen} (Subject gaps in coordinations, 1983)\linebreak (= Chapter~\ref{chap-subjektluecken} in this volume). This paper is the first study of so-called SLF"=coordinations like (\mex{1}), with `SLF' indicating their salient formal properties: a) there is a subject gap (`Subjektlücke' = \emph{SL}) in the second conjunct, b) both conjuncts are clauses with fronted verb (`\emph{F}-clauses'), with the second conjunct obligatorily being an F1-clause. Crucially, despite \emph{SL}, both conjuncts are inter­pretively related to the overt subject of the first conjunct in the same way. 

%in the second conjunct the subject is lacking, b) the conjuncts are F-clauses (F indicating the clause type with fronted finite verb).   


\ea
\gll  	Hoffentlich sieht  uns keiner   und  meldet uns bei der Polizei. \\
	hopefully    sees   us   nobody and  reports us    at  the police\\
\glt `Hopefully, nobody sees us and reports us to the police.'
\z
%
\addlines
The paper starts with a sketch of `symmetric', i.e., `phrasal coordination' where, roughly speaking, the substitutability criterion holds (every conjunct can substitute for the entire coordination salva grammaticalitate). Against this backdrop, the differing properties of SLF-coordinations are discussed in detail, in particular their most notable `asymmetric' property, which is that the lack of subject in the second conjunct cannot result from ellipsis (hence the second conjuncts violates the substitutability criterion), and its interpretive counterpart, the `fused' reading that all SLF-coordinations share. 
This paper, together with \citew{Hoehle90a} (see below), spurred a still active debate on asymmetric coordination in various Germanic languages. 

\textbf{On composition and derivation: The constituent structure of secondary words in German} (1985) (= Chapter~\ref{chap-composition} in this volume). This study pursues a strictly lexicalist theory of word formation where all morphemes have a lexical entry with the usual (i.a. categorial) specifications. Its most salient claim is that in such a framework the difference between composition and derivation can be entirely reduced to selectional properties of the respective morphemes: bound morphemes (`affixes') select other morphemes/morpheme classes to which they are thereby bound, free morphemes do not. This claim is carefully substantiated by presenting, first, the similarities of affixes to words, then by showing that compounds and derivations behave in a parallel fashion not only with respect to inflection but, on closer inspection, also with respect to boundary-related phenomena (such as the occurrence of linking morphemes, elision, stem formation) and even argument inheritance. Likewise, the detailed examination of formation processes underlying synthetic `compounds' (`Zusammenbildungen'), the verbal complex, `suffixless derivations' like \emph{Stoß} `push', \emph{Unterschied} `difference', and nominal infinitives does not yield any counterevidence either but many new insights into these difficult word-syntactic areas, and, last but not least, an ingenious argument in favor of the strictly lexicalist approach. 

This paper is still the most comprehensive word-syntactic treatment of German word formation to date. 

\textbf{Der Begriff ‚Mittelfeld‘. Anmerkungen über die Theorie der topologischen Felder} (The term `middle field'. Remarks on the theory of topological fields, 1986) (= Chapter~\ref{chap-mittelfeld} in this volume). This paper is a compact version of the descriptive and historical sections of \emph{Topologische Felder}, to which helpful diagrams and examples have been added, likewise extensive clarifying notes (concerning, e.g., the topological treatment of coherent structures). It also contains a brief history of topological `field' terminology and in the final section, important argumentation in favor of a notion ``S-Feld'' comprising the middle field together with VK, which is shown to be descriptively necessary whereas the traditional notion of a separate `middle field' is not. This paper has influenced practically all sections on the topology of German clauses in German syntax textbooks. 

\textbf{Assumptions about asymmetric coordination in German} (1990) (=
Chapter~\ref{chap-asymmetric-coordination} in this volume). This study 
is again about asymmetric coordination, extending the coverage to instances where a) the first conjunct is verb-final, and b) the second conjunct may be a V2-clause, the most typical cases being \emph{wenn}-clauses like (\mex{1}a,b). Because of the subject gap in the second conjunct, which again is irreducible to ellipsis and obligatorily bound up with F1-form, 
Höhle classifies cases like (\mex{1}b) as SLF-coordinations (see above); cases like (\mex{1}a) are dubbed (asymmetric) F2-coordinations. 

\eal
\ex
\gll  Wenn ich heimkomme und  da      steht   der Gerichtsvollzieher \ldots\\
      if   I   home.come and there  stands the bailiff \\
\ex
\gll  Wenn jemand   heimkommt  und sieht den Gerichtsvollzieher \ldots\\
      if   someone  home.comes and sees  the bailiff\\
\zl
%
Since the introductory \emph{wenn} has scope over the entire coordination, what is conjoined are unlike phrases: a V projection with a functional clausal projection (for Höhle an I projection). The entire paper is devoted to making the categories involved more precise and to derive the possible coordinations of this type, as well as their differences to symmetric coordinations, in a principled manner. While distinguishing between their first and asymmetric second conjunct as head vs. non-head, Höhle does not call into doubt that these constructions \emph{are} coordinations, a position not always shared in later literature where adjunction analyses are argued for as well.  

\textbf{On reconstruction and coordination} (1991) (= Chapter~\ref{chap-reconstruction} in this
volume). This paper is primarily concerned with scope and binding phenomena where dislocated
elements D$_i$ appear `reconstructed' into the position of their trace. Höhle considers two
approaches to `reconstruction': (i) D$_i$ is reconstructed into its original position on a level
(`R-structure') different from S-structure, and the relevant scope and binding relations are
computed there (`true reconstruction'), (ii) the definitions of these relations are extended in such
a way that they yield the correct results on S-structure, i.e., they treat D$_i$ as if it were in
the position of its trace (`pseudo-reconstruction'). Coordination comes in when comparing these
approaches: while empirically equivalent in simple cases, Höhle observes that pseudo-reconstruction
is in conflict with standard assumptions on how coordinate structures are to be translated into a
semantic representation. Hence, either (i) is correct, or the translation theory for coordination
needs revision. In settling this issue, Höhle provides first a concise outline of the fundamentals
of coordination theory (including strong arguments against ``forward conjunction reduction'') and of
German clause structure, based on which a comprehensive picture of scope and binding properties of
dislocated phrases in German is given, prominently among them, of course, the reconstruction
cases. These are then evaluated with respect to the two approaches in question. Höhle concludes,
based on cases such as verum focus, lexical anaphors, and in particular parasitic gap phenomena,
that true reconstruction cannot be correct, hence that the translation theory for coordination must
be revised in accordance with what pseudo-reconstruction requires.   

\textbf{Projektionsstufen bei V-Projektionen. Bemerkungen zu Frey/Tappe 1991}\linebreak (Projection levels with V-projections. Remarks on Frey/Tappe 1991, 1991) (=
Chapter~\ref{chap-projektionsstufen} in this volume). Despite its origin as a commentary to a paper
not reprinted here, this short paper is self-contained. It comments astutely on a number of
important issues concerning the structure of the German VP, notably in verb-final clauses, and the
nature of the V-projections in the various positions allowing for them: the clause-final position, the
fronted position (FIN), the pre-field. In particular, there is a forceful plea against identifying
the verb in final position (V$^e$) with the V$^0$ we meet in the FIN position, to which we owe the famous
argument from verbs like \emph{uraufführen} (`stage the first performance'), \emph{bausparen} (`save
for building'), etc., which was already alluded to in \citet[34]{Hoehle78a} but is clearly spelled out here for the first time. 

\textbf{Über Verum-Fokus im Deutschen} (On verum focus in German, 1992) (= Chapter~\ref{chap-verum-fokus} in this volume). The
phenomenon called `verum focus' since \citet{Hoehle1988} is illustrated in (\mex{1}): focus on the
fronted verb may have the effect of stressing the truth of the proposition expressed:   

\ea
\gll	 	Karl HAT bezahlt.   [meaning: es ist \emph{wahr}, dass Karl  bezahlt hat]\\
	       Karl has    paid      {}          it  is  true    that  Karl paid      has\\
\glt           `Karl DID pay.'
\z
%
This suggests that what is stressed is an abstract meaning element VERUM that has the proposition in
its scope. The present paper is a comprehensive discussion of its nature and location. First, it
explores the idea (already proposed in \citew{Hoehle1982} that VERUM is an
`illocution type [= IT] operator' (more exactly a variable over such operators). Despite some
evidence in its favor, Höhle argues that it is untenable: a) main clause \emph{wh}-interrogatives, e.g., have
verum focus only on the fronted verb but the IT operator is (also) associated with the \emph{wh}-phrase; b)
subordinate clauses, which are incompatible with truly illocutionary operators, allow verum focus
(in verb-final clauses located on C-elements like \emph{dass} `that', \emph{ob} `whether'); c) negation may have scope over VERUM, which is unheard of
for true IT operators. So, at best, VERUM is a sort of truth predicate. Finding a segmental
location for it is likewise difficult, given the controversial onset structure of German clauses
and further bewildering data from verum focus in embedded \emph{wh}- and relative clauses. In the end,
Höhle suggests a non-segmental localization of VERUM, at the cost of strict compositionality. 
	
\textbf{Vorangestellte Verben und Komplementierer sind eine natürliche Klasse} (Fronted verbs and
complementizers are a natural class, 1997) (= Chapter~\ref{chap-komplementierer} in this
volume). This paper argues a) that fronted verbs preceding their subject are categorially different
from those following their subject, b) that complementizers are sensitive to the same difference in
relative placement, so that, in this respect, fronted verbs and complementizers form a natural
class. On first glance, either claim seems bizarre but Höhle presents much evidence in their favor:
(a) is supported by data from the West Frisian imperativus pro infinitivo phenomenon, as well as the
many instances of special inverted verb forms in Old English\il{Old English}, Dutch\il{Dutch},
Middle Low German\il{Middle Low German}, Old and Middle High German\il{Old High German}\il{Middle
  High German}. Support for (b) are the distribution of \emph{som} in Scandinavian\il{Scandinavian} relative and interrogative clauses (analogously \emph{that} in English relatives), data from German\il{German} relative clauses to non-3rd person, and from inflected complementizers in Dutch\il{Dutch} dialects that take up the inflection of the inverted form where possible. A schematized analysis is supplied for the relevant structures, which implies, importantly, that a subject in the pre-field of a V2-clause does not bind an Ā-trace. The paper concludes with a description of related facts in Bantu languages, which strongly suggests that the observations and results presented here are of rather general importance. 

\textbf{The \emph{w}- \ldots{} \emph{w}-construction: Appositive or scope indicating?} (2000) (= Chapter~\ref{chap-w-w} in this volume).  This paper is concerned with the analysis of constructions like (\mex{1}):

\ea
\gll	 Was  glaubst du,  wen      er  feuern  wird?\\
	 what think    you  whom  he  fire       will\\
\glt 1) `What do you think with respect to the question who he will fire?'\\ 2) `Who do you think that he will fire?'
\z
%
They became a hot topic in the eighties when the traditional idea that the \emph{wh}-clause was some
sort of apposition to the \emph{was} `what' in the main clause (cf.\ translation 1), was challenged
by the idea that this \emph{was} marked the scope of the embedded \emph{wh}-phrase
(cf. translation 2), thus suggesting a `direct dependency' approach. Höhle was the first to present
a thorough comparative investigation of these analyses for German, which is documented in a series
of influential handouts from 1989/1990, on which (together with an update in 1996) the present paper
is based. After working out the salient characteristics of the \emph{was} \ldots{} \emph{w}-construction and presenting the two competing analyses, Höhle discusses various important empirical phenomena and theoretical issues (\emph{wh}-copy construction, \emph{wh} in-situ, questions of LF-movement and interpretive dependency, exclamative versions of the construction), asking how the two analyses fare with respect to them. As a result, Höhle favors the direct dependency approach but in the course of the discussion, he also makes the `appositive' approach more precise, thus anticipating the `indirect dependency approach' that has become a serious rival of the direct dependency approach in the following years.

\textbf{Observing non-finite verbs: Some 3V phenomena in German-Dutch} (2006) (= Chapter~\ref{chap-v3} in
this volume). This paper is an impressive survey over the systems of non-finite verb forms and the
regularities governing them in numerous German and Dutch\il{Dutch} dialects, covering many dialect areas in
fine-grained detail. It is shown by authentic material taken from pertinent sources that the
differences to the Standard German and Dutch\il{Dutch} systems may be enormous: instead of just three, there
might be six non-finite forms (e.g., bare infinitive, \emph{ge}-infinitive, bare gerund, \emph{ge-} and
\emph{be}-gerund, participle)), in the extreme case even eight; substitutions, e.g., of the participle, may involve different forms (e.g.,
simple and complex supines instead of infinitives) as well as different substitution conditions;
there are considerably different displacement phenomena, likewise order variation and ordering rules
within the verbal complex not found in the Standard systems. Thus, the paper certainly reaches its
professed aim formulated at the outset, which is to provide a more reliable research basis for the
non-finite system than the rather poor systems of Standard German and Dutch\il{Dutch} have
to offer, especially when claims of a more principled nature are at stake.  



%\section*{Introductory remarks on the \isi{HPSG} papers}


\medskip
Turning now to the papers on Head-Driven Phrase Structure Grammar, they were all written in
the 1990ies, starting toward the end of the protracted publication phase of the
canonical presentation of \isi{HPSG} in the book by Pollard and Sag.
They are concerned with properties and the organization of the lexicon
with special emphasis on syntactic traces (\emph{Spuren in \isi{HPSG},
Spurenlose \isi{Extraktion}, Complement extraction lexical rule and variable
argument raising}), with shared properties of relative pronouns, fronted
verbs and complementizers in English, Norwegian\il{Norwegian}, Swedish\il{Swedish}, Dutch\il{Dutch} and
German (\emph{Featuring creatures of darkness}), and with general problems of
phonological theory and the relationship between the abstract
structures characterized by phonotactic rules and observable
empirical phenomena (\emph{An architecture for phonology}).

The first group of papers pursues fundamental questions about lexical
elements. Although the papers remained unpublished, they became highly
influential in some circles of the \isi{HPSG} community, drawing attention
to the grammar"=architectural dependencies in \isi{HPSG} between postulating
traces, different ways of interpreting lexical rule
mechanisms, various options of expressing lexical generalizations, and
the syntax of verb clusters and their dependents in German\il{German} and
English\il{English}.

The series begins with \textbf{Spuren in HPSG} (Traces in
\isi{HPSG}) (= Chapter~14 in this volume), considerations about the nature of traces in \isi{HPSG} that
highlight important differences to assumptions concerning traces in other
frameworks. In particular, \emph{Spuren in HPSG} shows that traces do
not enter into a linear order relation with other words in an
utterance (an observation with major impact on
language processing arguments involving the presumed position of
traces which is often overlooked even today), and it points out
intricate implications of the treatment of traces for
the syntax of verbal projections in coherent constructions in
German.

\textbf{Spurenlose Extraktion} (Traceless extraction) (= Chapter 15 in this volume) embarks
on a thorough analysis of the consequences of eliminating traces from
the theory of extraction by postulating extraction lexical rules. Different
possibilities of implementing lexical rules are explored in great
detail by painstakingly examining concrete lexical
entries and corresponding entries that are derived by lexical
rule. The problems that this discussion reveals with \isi{HPSG}'s
early informal characterizations of lexical rules lead to a review of
various alternatives, which are again explained with great
precision. Many of the insights gained here became influential in
later technical treatments of lexical rules and the lexicon in
\isi{HPSG}.

\textbf{Complement extraction lexical rule and variable
  argument raising} (= Chapter 16 in this volume) builds directly on results of \emph{Spurenlose
  Extraktion} with a precise demonstration that a lexical
rule (in the original framework-internal understanding of the mechanism) for complement
extraction cannot be combined with standard \isi{HPSG} assumptions concerning
argument raising in the verbal complex without leading to massive
problems. Again, the argument is presented with an extraordinary sense
for detail, with exact specifications of the lexical entries that are
involved in the analysis.

%Characteristic of this paper is the high level of explicitness in
%the logical description, which partially adopts a variant of King's
%Speciate Re-entrant Language in an attempt to be as clear as possible
%about the nature and meaning of \isi{HPSG}'s descriptions.

\addlines[2]
\textbf{Featuring creatures of darkness} (= Chapter 17 in this volume) turns to another empty lexical
element of Pollard and Sag's book, the empty relativizer which their
analysis of English\il{English} relative clauses employs. Practitioners of
\isi{HPSG} traditionally
dislike any kind of empty elements in grammar, which meant that the
empty relativizer was immediately met with great skepticism.
Höhle shows that, far from being obscure, the inner structure of
Pollard and Sag's empty relativizer is surprisingly well-suited for a
typological analysis of various elements at the left periphery of
Germanic languages, including Norwegian\il{Norwegian}, Swedish\il{Swedish}, Dutch\il{Dutch} and German. Under this perspective the empty relativizer
serves as a blueprint for \emph{wh}-interrogatives, complementizers, relative
pronouns and fronted verbs alike, and the analysis provides valuable insight into a
very difficult area of Germanic syntax. In many ways \emph{Featuring
  creatures of darkness} is the more technically oriented HPSG twin of
\emph{Vorangestellte Verben und Komplementierer sind eine natürliche
Klasse} (= Chapter 11) published three years later, and they should be read together.

\textbf{An architecture for phonology} (= Chapter 18 in this volume) applies the grammar
architecture and logical apparatus of \isi{HPSG} to
phonology and morphophonology. It argues that model-theoretic grammar
provides a solid foundation for reasoning about complicated empirical
facts in this domain. In addition to outlining a sort hierarchy for
phonology and fundamental principles, examples from German and Russian\il{Russian}
demonstrate the analytical usefulness of the approach. They give rise
to interesting considerations of the complex relationship between
the structures in the denotation of logical grammar theories and the
objects of empirical observation, spelled out with more care here than
anywhere else in the literature.


\section*{Acknowledgments}

Some of the papers (re)published in this book were available in electronic format. Some in Word and
some even in \LaTeX. Others had to be digitized and re-typeset. This involved a lot of work. We are
grateful to our typesetter Luise Dorenbusch, who did the initial conversion of the majority of the
papers. Stefan Müller's student assistants Luise Hiller and Robert Fritzsche did the remaining
chapters and the many cycles of revisions and adaptions. In the final period they were supported by
Nico Lehmann. Sebastian Nordhoff helped with the semiautomatic creation of the index, checked the
final manuscript and provided general technical support. A big ``Thank you'' to them all; without their dedication and
care we would never have been able to produce such a sound publication.


Most of all, however, we would like to thank Tilman Höhle, who generously gave us permission to pursue our ever-expanding plans for publishing his linguistic writings in whatever form we decided on. Although his intellectual interests have moved away from linguistics, we hope that the present volume gives him some satisfaction.

~\medskip

\noindent
Berlin, Tübingen, Bensheim, March 22, 2018

\hfill Stefan Müller, Marga Reis, Frank Richter


\nocite{g_Hoehle78b,g_Hoehle79,Hoehle80,Hoehle83,Hoehle86,Hoehle91,Hoehle91b,hoehle:94c,hoehle1994a-eng,hoehle94b-eng,hoehle:95,g_Hoehle96a,Hoehle97a,Hoehle99a,Hoehle06}

\refstepcounter{mylastpagecount}
\label{page-hoehle-refs}
\printbibliography[heading=subbibliography,notkeyword=this]

\end{refsection}




\section*{Postscript: Rights \& Permissions}
\refstepcounter{mylastpagecount}
\label{page-postscript}
\begin{refsection}

Preparing this book was a lot of work but the three of us took it in stride since we knew it was work done to make great papers of a colleague available for the first time and his published papers more accessible. However, one aspect of this process deserves special mention: the
attempt to get the rights to republish Tilman Höhle's work. This part of the work was extremely time consuming, extremely inefficient and extremely annoying. I want to explain why in
a little more detail. Tilman Höhle published several very influential papers in the 80ies and
90ies. He published with 
Akademie-Verlag,
Benjamins,
Bouvier Grundmann,
CSLI Publications, 
Foris, 
Kluwer, 
Niemeyer, 
Stauffenburg, and
Westdeutscher Verlag. With the exception of Benjamins, Bouvier Grundmann, CSLI Publications and Stauffenburg all these
publishers were bought by De Gruyter or Springer. An overview of this is presented in Table~\ref{tab-publishers-then-and-now}.

\begin{table}
\begin{tabular}{ll}
\lsptoprule
original publisher & now owned by\\
\midrule
Akademie-Verlag      & De Gruyter\\
Benjamins            & \\
Bouvier Grundmann & \\
CSLI Publications    & \\
Foris                & De Gruyter\\
Kluwer               & Springer\\
Niemeyer             & De Gruyter\\
Stauffenburg         & \\
Westdeutscher Verlag & Springer\\
\lspbottomrule
\end{tabular}
\caption{\label{tab-publishers-then-and-now}Concentration in the publishing sector}
\end{table}

The contracts that we signed in the 80ies and 90ies all included a passage saying that authors have
the right to use their articles in collections of their own work or in books authored or coauthored
by them. So we expected that it would not be a problem to get the permissions to put together a
collection of Tilman Höhle's most important papers. I wrote emails to the remaining publishers and
got fast positive responses by CSLI Publications and by Brigitte Narr from Stauffenburg. I called Benjamins, who
were a bit delayed due to the holiday season, but reacted quickly after my call. We got permission
to use the papers we wanted to use with a Creative Commons CC-BY license and free of charge. 

%\largerpage
The interaction with the remaining two publishers was less pleasant. I first thought that Springer
was easy, since they have a web interface for Rights \& Permissions and this web interface
grants you the right to use articles in other compilations and so on provided you are the
author. However, the automatically generated permission letters explicitly exclude online
publications without password protection and refer authors to Springer's Rights \& Permissions department for such usage.

I sent several emails to Springer and got impersonal replies without a name of the sender. I called
several times and I guess I interacted with three or four employees of the Rights \& Permissions
section. The answer I got was: ``We cannot grant you the right to put copyrighted material on a
webpage''. I explained in emails and during phone calls that we did not want to upload the original
articles into repositories or onto any other webpage but that we wanted to edit, reformat, and publish
the papers by Tilman Höhle in a collection of his work, something that is usual and was possible up
to the recent changes in the publishing world. For one paper, it turned out that the rights of the
papers in the relevant publication reverted back to the authors, so we can use \emph{\isi{Verum} focus} in the intended way. But for
\emph{Reconstruction and coordination} I got the repeated reply that there is no way to use this
paper in publications that are available online without a paywall. This would just not get into my
head. Springer sells the open access option (CC-BY) to authors of new articles for \$3.000/""2.200\euro+VAT,\footnote{%
  \url{http://www.springer.com/gp/open-access/springer-open-choice}, 23.09.2015.%
}
but there is no way to turn a book chapter from 1991 into open access? Not even for money? The argument was: it
would be unfair to readers who buy the complete book. What? I really love this appeal to fair play!
What about the readers of journals that appear in print and online? Is their subscription fee of the print version lowered when the journal contains
papers whose authors paid to have their content open access?

My conclusion from this was that Springer is not just a greedy company with a profit rate of more
than 35\,\%, Springer is really an obstacle for science, their interests are fundamentally different from ours. So,
I wrote a letter to Jolanda Vogt, who is responsible for linguistics at Springer, and to Susanne
Wurmbrand, who is the editor of the \emph{Journal of Comparative Germanic Linguistics}, and informed them
that I find Springer's policy regarding rights unbelievable and that I would stop working for
Springer immediately (I am on the board of \emph{Journal of Comparative Germanic Linguistics} and do a lot of reviewing for \emph{NLLT}, \emph{Journal of
Comparative Germanic Linguistics} and \emph{Research on Language and Computation}). Ms.\ Vogt contacted Rights \&
Permissions and we then got a quick reply giving us the permission to use the paper in an open
access publication and with 100 printed copies (which is what we estimated when we filled in the
first permissions form). 

I was happy, but when reading the fine print, we discovered that the rights
were not sufficient. What is needed for open access as we understand it at Language Science Press is a Creative Commons
CC-BY license \citep{Shieber2012a}. There are extensions of this license by NC and ND components. NC
means that commercial use is not allowed without special permission and ND stands for no derivatives
and means that third parties may not produce other compilations that include work under this
license. In order to be able to print books via Print on Demand services, we have to have the right
to sublicense the use of a book to a commercial enterprise. This would be made impossible by the NC
clause. The same is true for uploading books on platforms like Google Books. Since Google is a
commercial enterprise, we cannot sublicense our books to them if we do not have the permissions of
the rights holders.

So, I went into the next round of emailing. The result now is that the Springer paper will be
published under a CC-BY-NC-ND license and that we have the right to sublicense for PoD for 100
books.

\addlines[1]
The interaction with De Gruyter was a little more pleasant, but rather chaotic. We got the offer to
buy the right to republish as open access right away and the prices were \ldots{} shocking. After all
we are talking about papers from the 80ies and 90ies. Nobody will buy these books anymore. Contracts
from Niemeyer stated that the copyright returns to the author once the work goes out of
print. Nowadays nothing goes out of print since we have print on demand, but nobody will buy these
books either.\footnote{%
  The conference volume in which \citew{Hoehle86} appeared is now sold for 119,95\euro{}/""\$168.00,
  which definitely prevents interested readers from buying it.%
}
The only commercial value
of such old papers is bundled content and this is what the bigger commercial publishers are
selling \citep{Shieber2013a}.\footnote{%
 \citew{Hoehle86} is not even available as PDF from De Gruyter. The book is not listed in the directory of
 deliverable books and hence not available in normal bookstores. The content is simply blocked by De Gruyter until somebody pays for digitization.%
}
% cheked amazon, degruyter 02.02.2018
%\todostefan{check VLB, book is not available at amazon, thalia, \ldots}

After several rounds of emailing and a request to Anke Beck, the CEO of De Gruyter, we arrived at a
CC-BY-NC-ND license for six papers for 1.273,30\,\euro\ in total and De Gruyter stated that they explicitly wanted to exclude aggregation of text
material. The negotiation process and its results are summarized in
Table~\ref{tab-interaction-with-publishers}. My mail folder on the rights issue contains 107
emails not including the ones I wrote.
%\todostefan{update numbers at the end, table and text, check price}

\begin{table}
\oneline{%
\begin{tabular}{lrrlr}
\lsptoprule
publisher           & email & calls & result                                & price\\
\midrule
Benjamins           & 1     & 1           & CC-BY                             & 0\,\euro\\
De Gruyter          & 47    & 3           & CC-BY-NC-ND + Print for 500 books & 1.273,30\,\euro\\
CSLI Publications   & 1     & 0           & CC-BY                             & 0\,\euro\\
Springer            & 40    & 3           & CC-BY-NC-ND + Print for 100 books & 0\,\euro\\
Stauffenburg        & 1     & 0           & CC-BY                             & 0\,\euro\\
\lspbottomrule
\end{tabular}}
\caption{\label{tab-interaction-with-publishers}Interaction with publishers and results}
\end{table}

In hindsight I regret that I did not document the time that it took me to do all these negotiations, inform
my co-editors and Language Science Press staff, discuss things and react again. I am sure that these
negotiations wasted at least the same amount of time at the other side (Springer, De Gruyter). This is highly
inefficient. The public sector pays for these publishing houses. We pay the rights and
permissions departments of the publishers. This is part of the book prices that libraries and
individual researchers pay. In a world of true open access all this would be unnecessary.

Due to the restrictive NC-ND license we cannot distribute all papers in the same way. We think that
this is a pity and it ruins the book. We decided to publish this book nevertheless and put blank pages
into versions of the book for which we did not get the permissions we would need. So for instance,
the De Gruyter and Springer papers will not be on Google Books. From the 101st printed copy onwards,
the printed versions of this book will not contain \emph{Reconstruction and coordination}, but blank pages with just the URL to the online
version of this book.

So, the conclusion and the advice to all readers is: do not give your copyright away. Just don't!
Commercial publishers will publish your paper anyway. Or even better, publish with true open access publishers that
license the material under a CC-BY license.

~\medskip

\noindent
Berlin, March 22, 2018\hfill Stefan Müller



% Telefonat Linhard  23.09.2015





\printbibliography[heading=subbibliography,notkeyword=this]
\end{refsection}


%      <!-- Local IspellDict: en_US-w_accents -->
