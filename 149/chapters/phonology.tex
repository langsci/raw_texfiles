%% -*- coding:utf-8 -*-



\documentclass[output=paper]{langsci/langscibook}
\author{Tilman N. Höhle}
\title{An architecture for phonology}
\abstract{}
\maketitle
\rohead{\thechapter\hspace{0.5em}An architecture for phonology} % Display short title
\ChapterDOI{10.5281/zenodo.1169695}
\begin{document}
\label{chap-phonology}
\selectlanguage{english}

%\INS{phonology|(}

\renewcommand*{\thefootnote}{\fnsymbol{footnote}}
\setcounter{footnote}{4}

\footnotetext{%
\emph{Editors’ note:} This paper was originally published in Borsley, Robert D.\ \&
Adam Przepi{\'o}rkowski (eds.). 1999, \textit{Slavic in Head-Driven
Phrase Structure Grammar} (Studies in Constraint-Based Lexicalism), 61--90.
Stanford, CA: CSLI Publications.%
}


\renewcommand*{\thefootnote}{\arabic{footnote}}
\setcounter{footnote}{0}

\section{Why and how}


``Phonology'' is construed here in a broad sense. It comprises the system
of unpredictable (``distinctive'') phonological properties of expressions,
which may be called phonemics, and the system of phonetic properties that
are only predictable in terms of a language's phonemics in conjunction with
phonetic rules specific to the language (but not by a theory of universal
phonetics alone). It also includes ``morphophonology,'' which deals
specifically with phonological phenomena that are observed when signs are
components of larger signs.

%\largerpage[2]
An explicit empirical theory that 
%aspires
is
 to integrate a subject as large
and diversified as phonology\footnote{%
	For information on phonology, useful
  recent reference works are available: for data and analytical ideas that
  have dominated discussions in (morpho-)phonology in the last decade(s),
  \cite{kens:94}\INA{Kenstowicz, Michael} and
  \cite{goldsmith:95:ed}\INA{Goldsmith, John A.}; and for phonetics,
  \cite{lad:mad:96}\INA{Ladefoged, Peter}\INA{Maddieson, Ian} and
  \cite{har:lav:97}\INA{Hardcastle, William J.}\INA{Laver, John}. Some
  information on formal phonology can be found in \cite{bird:95}\INA{Bird,
    Steven} and in \cite{cars:98a}.\INA{Carson-Berndsen, Julie}%
}
needs to
%build 
be built on a very clear but highly expressive formal basis grounded in a
theory of language.  The formal theory of language and grammar developed
in \cite{king:89}\INA{King, Paul John} and work that builds on it,
essential aspects of which also underlie the \isi{HPSG} theory of {\small PS94}
(i.e., \citealt{PS94a})\INA{Pollard, Carl}\INA{Sag, Ivan A.}, is exceptionally
well suited for this task.  Moreover, the structure of signs (i.e., objects
of sort \textit{sign}) proposed in {\small PS94} is an excellent point of
departure for a theory of morphology that is to be able to interact
naturally with morphophonology.

In the sections to come I will outline a frame of reference
%that is meant to be 
expressive enough to discuss empirical phenomena with a useful degree of
explicitness, resulting in a simple architecture for phonology in the broad
sense (summarized in \S\ref{sectSum}). Some aspects of the architecture
(and questions waiting to be explored) are illustrated with examples from
Russian\il{Russian}, German and Miwok\il{Miwok}. No originality is claimed in either respect%, and
%nothing spectacular is going to be proposed
. My intention, on the contrary, is to remind ourselves that a framework in
which to do formal phonology, closely related to {\small PS94}, is at hand
(even though the multistratality that I endorse here has routinely been
rejected). I further believe that considerations from phonology can
contribute to the efforts (e.g., in \citealt{king:99}\INA{King, Paul John} and
\citealt{poll:98}\INA{Pollard, Carl}) to explicate the meaning of grammars,
and that the particular model-theoretic conceptions adopted here can in
turn contribute to attempts to elucidate the relation between the grammar
and physics of phonetics.


\section{Segments and segmental strings}
\label{sectSS}


\subsection{Signature}
\label{sec:1.2.1}


As in {\small PS94}, signs bear an attribute \textsc{phon(olo\-gy)}\INS{phonology@\textsc{phonology}}. Its value is of sort
\textit{phon}. The structure of \textit{phon} objects is described
in Figure~\ref{F1}.
\begin{figure}[ht]
%% \avha{\textit{phon}
%%       \\\textsc{segmental-string} \textit{listofsegment}
%%       \\\textsc{hierarch} \avhr{\textit{hierarch}
%%                       \\\textsc{syllables} \textit{listofsyllable}
%%                       \\\textsc{feet} \textit{listoffoot}
%%                       \\\textsc{phonwords} \textit{listofnelofsegment}}}
\begin{avm}
\onems[phon]{
 segmental-string  \tpv{listofsegment}\\
 hierarch          \[ \tpv{hierarch}\\
                      syllables & \tpv{listofsyllable}\\
                      feet      & \tpv{listof\/foot}\\
                      phonwords & \tpv{listofnelofsegment}\\ \] }
\end{avm}
\caption{\label{F1} Structure of \textit{phon} objects}
\end{figure}

The \textsc{s(egmental)-string} value is a (possibly empty) list of segments,
whose internal structure will be considered below. The string of segments
is hierarchically (``prosodically'') structured, minimally into syllables,
feet and phonological words.

%\addlines[2]
Although phonological words will play an important role
in \S\ref{sectRuss}, the hierarchical structure shown in Figure~\ref{F1} is
provisional. Intuitively, one might expect that a phonological word is
constituted by a sequence of feet, a foot is constituted by a sequence of
syllables, and the s-string (i.e., the \textsc{s-string} value) is
exhaustively syllabified.  Then, the \textsc{phonwords} value should be a list
of \textit{phonword} (rather than \textit{nelofsegment}) objects, and the
attributes \textsc{syllables} and \textsc{feet} would be born by \textit{foot} and
\textit{phonword}\/ objects, respectively.  But phenomena of extrasyllabicity
(cf., e.g., \citealt{bage:91}\INA{Bagemihl, Bruce},
\citealt{guss:92}\INA{Gussman, Edmund}, \citealt{hyma:92}) and\INA{Hyman, Larry M.}
extrametricality, as well as the mismatch of syllables and phonological
words in (\ref{gor}) below, indicate that the relations between s-strings,
syllables, feet, and phonological words can be 
%much less straightforward.
far less simple.
Thus, although it is obvious how members of \textsc{syllables}, \textsc{feet},
and \textsc{phonwords} values can formally be related to s-string members
(cf.\ \citealt{mast:93}\INA{Mastroianni, Michael} for syllables), there is a
host of empirical questions that I will not enter into.%
\begin{figure} 
\begin{exe}
\ex
\label{B2}
\begin{tabbing}
\hspace{1,5em}\=\hspace{1,5em}\=\hspace{1,5em}\=\hspace{1,5em}\=\hspace{6em}
\=\kill
\textit{segment}\>\>\>\>\>\textsc{segmproper} \textit{segmproper} \\
\>\textit{long,  short} \\
\textit{segmproper}\>\>\>\>\>\textsc{airstream} \textit{airstream} \\
\>\>\>\>\>\textsc{voicing} \textit{voicing} \\
\>\>\>\>\>\textsc{velum} \textit{velum} \\
\>\>\>\>\>\textsc{tongue} \textit{tongue} \\
\>\textit{vowel}\>\>\>\>\textsc{airstream} \textit{pulmonic} \\
\>\textit{consonant}\>\>\>\>\textsc{constriction} \textit{nelofplace} \\
\>\>\textit{obstruent} \\
\>\>\>\textit{fricative} \\
\>\>\>\textit{affricate} \\
\>\>\>\textit{plosive}\>\>\textsc{velum} \textit{closedvelum} \\
\>\>\textit{sonorant}\>\>\>\textsc{airstream} \textit{pulmonic} \\
\>\>\>\textit{nasalcon}\>\>\textsc{velum} \textit{openvelum} \\
\>\>\>\textit{liquid} \\
\>\>\>\>\textit{lateral, rhotictrill} \\
\textit{voicing}\>\>\>\>\>\textsc{achievement} \textit{achievement} \\
\>\textit{voiced} \\
\>\>\textit{normalvoice, breathy, creaky} \\
\>\textit{voiceless} \\
\>\>\textit{spreadgl, closedgl} \\
\textit{tongue}\>\>\>\>\>\textsc{vertical} \textit{vertical} \\
\>\>\>\>\>\textsc{horizontal} \textit{horizontal} \\
\>\textit{tense, lax} \\
\textit{horizontal}\>\>\>\>\>\textsc{narrowing} \textit{narrowing} \\
\>\textit{front, central, back} \\
\textit{nelofplace}\>\>\>\>\>\textsc{first} \textit{place} \\
\>\>\>\>\>\textsc{rest} \textit{listofplace} \\
\textit{place}\>\>\>\>\>\textsc{achievement} \textit{achievement} \\
\>\textit{labial}\>\>\>\>\textsc{site} \textit{sitelab} \\
\>\textit{lingual}\>\>\>\>\textsc{laterality} \textit{lateralclosure} \\
\>\>\textit{coronal}\>\>\>\textsc{site} \textit{sitecor} \\
\>\>\>\textit{apical, laminar, retroflex} \\
\>\>\textit{dorsal}\>\>\>\textsc{site} \textit{sitedors} \\
\>\textit{pharyngeal} \\
\>\textit{glottal}
\end{tabbing}
\end{exe}
\end{figure}

%\bigskip  
%
%\noindent
The structure of segments is determined by the inventory of non"=atomic
sorts and their feature declarations in (\ref{B2}) on the following page. Indentation indicates
partitioning into subsorts; thus, e.g., \textit{vowel}\/ and \textit{consonant}\/
partition \textit{segmproper}\/; \textit{obstruent}\/ and \textit{sonorant}\/
partition \textit{consonant}\/; \textit{fricative}, \textit{affricate}\/ and \textit{plosive}\/ partition \textit{obstruent}\/; \textit{nasalcon}\/ and \textit{liquid}\/ partition \textit{sonorant}\/; \textit{lateral}\/ and \textit{rhotictrill}\/ partition \textit{liquid}. For \textit{segmproper}, the
attributes \textsc{airstream}, \textsc{voicing}, \textsc{velum} and \textsc{tongue} are
appropriate; for \textit{consonant}, \textsc{constriction} is appropriate.
Attributes appropriate for a sort are also appropriate for that sort's
subsorts.  Atomic sorts, for which no attribute is appropriate, are
partitioned in (\ref{B3}).
\begin{exe}
\ex
\label{B3}
\textit{achievement}\/: \textit{full, reduced, zero}\/.\\
\textit{airstream}\/: \textit{pulmonic, ejective, implosive, click}\/.\\
\textit{lateralclosure}\/: \textit{sideslocked, sidesunlocked}\/.\\
\textit{listofplace}\/: \textit{elist, nelofplace}\/.\\
\textit{narrowing}\/: \textit{round, nonround}\/.\\
\textit{sideslocked}\/: \textit{grooved, nongrooved}\/.\\
\textit{sitecor}\/: \textit{upperlip, dental, alveolar, postalveolar, palatal}\/.\\
\textit{sitedors}\/: \textit{velar, uvular}\/.\\
\textit{sitelab}\/: \textit{upperlip, upperteeth}\/.\\
\textit{velum}\/: \textit{openvelum, closedvelum}\/.\\
\textit{vertical}\/: \textit{high, mid, low}\/.
\end{exe} 


List supersorts, such as \textit{listofsegment}\/ in Figure~\ref{F1}, are
always partitioned analogously to \textit{listofplace}\/. Construing a string
of segments as a list of tree-like objects with some trees sharing some
branches, as illustrated in Figure~\ref{kuend} below, is a straight
explication of notions that can be found, e.g.,
in \cite[237]{clem:85}\INA{Clements, George N.}; a difference being that
Figure~\ref{F1} (just like~\citealt{scob:97}\INA{Scobbie, James M.}) does not
provide for ``autosegmental tiers'' independently of the \isi{segmental} string.
In detail, there has been (and will continue to be) much discussion on the
proper structure of segments; cf., e.g., \cite[Chapter~4]{bird:95},\INA{Bird,
  Steven} \cite{cle:hum:95},\INA{Clements, George N.}\INA{Hume, Elizabeth
  V.} and \cite{lade:97}\INA{Ladefoged, Peter} for three different
proposals. I cannot hope to discuss here all aspects of~(\ref{B2}),
(\ref{B3}). %Still, some comments are in order to make the intent of certain
%aspects clear.
Certain aspects will be commented upon below.
%\bigskip

%\noindent


\subsection{``Underspecification'' vs.\ total well-typedness}
\label{sec:1.2.2}

I adopt \isi{HPSG}'s standard assumption that the objects in a model of a grammar
are ``totally well-typed,''\footnote{%
	In the formal theory that I rely on
  (\citealt{king:89}\INA{King, Paul John}; cf.\ also \citealt{king:94} and
  note~\ref{RSRL} below), a grammar G consists of a signature (such as
  \small{PS94}, 396--399) and a set of restrictions
  (``constraints/""principles/""rules,'' such as \small{PS94},
  399\hspace{1pt}ff.), each of which is an expression of a formal
  description language.  The 
%(decidable) 
  description language
  of~\cite{king:89} has disjunction, classical negation, sort
  specifications, and path equations. Let U be a set that conforms to the
  signature in that the denotations of the maximally specific sort symbols
  partition U, and each attribute symbol denotes a partial function from U
  to U, respecting appropriateness.  With total well"=typedness, the
  function is total 
%wrt.\ 
with respect to
the denotations of the sorts that the attribute
  is declared to be appropriate for. There is a denotation function \textit{D}\/ from the set of description language expressions to the powerset
  of U\@. Thus, a description language expression denotes a subset of U\@.
  It can require (by a path equation) that on each member of its
  denotation, certain path values are token identical, but it cannot
  identify any particular member of U\@.  U is a ``model'' of G just in
  case U = $\cap \{ D(\delta) \mid \delta \in \mbox{Restrictions} \}$.
  Crucially, the relation of a grammar's models to a natural language is as
  lucid as it can possibly be: the members of the intended models are token
  linguistic objects.  Specifically, for a grammar G, some natural language
  L is intended to be an exhaustive (i.e., maximally inclusive) model of G
  (\citealt{king:95a,king:99}).%
}
hence each object bears all
attributes that are appropriate for its sort (cf.\ {\small PS94}, 396).
Although this assumption has first been explicitly introduced
in \cite{king:89},\INA{King, Paul John} it is less than self-evident 
%wrt.\ 
with respect to
this
theory's intended models.  There are no obvious formal or ontological
reasons for insisting on it. Dropping it would remove a ``foundational
problem'' in {\small PS94}'s theory of coordination ({\small PS94}, 203,
note~39) and would be conducive to an attempt to formally reconstruct
theories of phonemic ``underspecification'' that have loomed large in the
eighties.
 
For an illustration, we may consider some data from~\S\ref{sectRuss}.
Slavic languages are known for a regressive voicing assimilation that
affects obstruents at the end of words. In Russian\il{Russian}\INS{Russian}, the
assimilation is only triggered by obstruents, in which voicing is (mostly)
distinctive, but not by vowels and sonorants, in which voicing is
predictable. To capture the dependence of assimilation on distinctive
voicing, \cite{kipa:85}\INA{Kiparsky, Paul} and others suggest that
phonemically, vowels and sonorants do not have any voicing properties
(i.e., they do not bear a \textsc{voicing} attribute), so that the
assimilation is phonemically triggered just by segments that do have
voicing properties.

Of course, \textit{vowel}\/ and \textit{sonorant}\/ objects that lack voicing
properties cannot surface as such; they must map to corresponding objects
that bear a \textsc{voicing} attribute. We thus get configurations as
schematically described in (\ref{underspec}) for vowels:
\begin{exe}
\ex
\label{underspec}
\begin{xlist}
\ex
\attop{
{\small \avhr{$\varphi$  $\pi_1$\avhr{$\sigma_2$ \\
\textsc{tongue} \avmbox{1} \\
\textsc{voicing} \textit{voiced} } \\ 
$\pi_2$\avhr{\textsc{phon} $\pi_3$\avhr{$\sigma_1$\\\textsc{tongue} \avmbox{1}} } } }\\
}
\ex
{\small $\neg$ \avhr{ $\pi_2$\avhr{\textsc{phon} $\pi_3$\avhr{\textsc{voicing}
\textit{voicing} }}} }
\end{xlist}
\end{exe}
The symbols $\sigma_1$ and $\sigma_2$ are meant to be variables over
maximally specific subsorts of \textit{vowel}\/; $\varphi$ is a cover symbol
for some phonological attribute (such as \textsc{phon} or \textsc{utterance},
introduced below in (\ref{partialoverlap})); the $\pi_i$ are path
variables. The descriptions in (\ref{underspec}a) and (\ref{underspec}b)
are to be understood conjunctively: objects that satisfy (\ref{underspec}a)
also satisfy the negative description (\ref{underspec}b). Thus, the $\pi_3$
value does not bear a \textsc{voicing} attribute. Even if the attribute values
of the $\pi_3$ value are identical to attribute values of the $\pi_1$
value, as shown for \textsc{tongue}, the $\pi_1$ value cannot be identical to
the $\pi_3$ value, as only the former bears a \textsc{voicing} attribute.

When total well-typedness is dropped, $\sigma_1$ and $\sigma_2$ can be
identical; while with total well-typedness retained, they cannot. Sort
identity probably conforms to the intuitions underlying
underspecificational analyses better than nonidentity.

In southwestern variants of Polish\il{Polish!southwestern}, the word-final
voicing assimilation is also triggered by vowels and sonorants; i.e.,
distinctiveness of voicing is not relevant there. This sheds doubt on the
attempt to account for the situation in Russian\il{Russian}\INS{Russian} by an
ontological construct like underspecification.\footnote{%
	Historically,
  ``underspecification'' has more often been motivated by considerations of
  markedness. This motivation has been obsolete at least since
  \cite{kean:81}\INA{Kean, Mary-Louise} and
  \cite{Hoehle82c}\INA{Hohle@H{\"o}hle, Tilman}.%
}
For broader critical discussion, cf.\ \cite[Chapter~9]{broe:93}\INA{Broe, Michael},
\cite{goldsmith:95b}\INA{Goldsmith, John A.},
\cite[{}\S{}6]{cala:95}\INA{Calabrese, Andrea},
\cite[370]{zoll:97}\INA{Zoll, Cheryl}, and references therein.

On inspection, then, the supposed empirical virtues of allowing objects not
to bear attributes they are allowed to bear typically turn out to be
deceptive. Moreover, segments that cannot surface in principle would
contradict what I regard as an important leading conjecture: all segments
are physically interpretable.  Physically uninterpretable segments should
be admitted in the linguistic ontology only in the face of unequivocal
positive evidence. No such evidence is known.

%\addlines[2]
Independently of these empirical considerations, experience indicates that
working with a fully explicit theory that forgoes total well-typedness gets
unwieldy to an extent that outweighs any doubt about its being
ontologically well-motivated. I thus see no reason to drop, but strong
reasons to retain, total well-typedness.
%\bigskip

%\noindent


\subsection{Some comments on the signature}
\label{sec:1.2.3}

According to~(\ref{B2}), then, each \textit{segmproper} object bears an
attribute \textsc{tongue} whose value bears attributes indicating the
%traditional
horizontal and vertical tongue positions that characterize vowel gestures.
I follow \cite{odde:94}\INA{Odden, David} in classing the narrowing of the
frontal oral tract associated with ``rounding'' in vowels together with the
horizontal specification.

Consonants in addition bear an attribute \textsc{constr(iction)} whose value
indicates one or more places of consonantal constriction.  Thus, the \textsc{constr} value is a list that is treated in such a way that the order of
list members is immaterial. In this way, double articulation of stops and
nasals~-- which may be phonemic or may result from contextually induced
coarticulation~-- can be captured.

At the same time, consonants bear the tongue specifications that
characterize vowels. This captures the fact that vowel gestures typically
``act as a kind of background to the ``figure'' of the consonants''
(\citealt[354]{bro:gol:90}\INA{Browman, Catherine P.}\INA{Goldstein, Louis}).
It is exploited phonemically in languages that have distinctive secondary
articulations, such as palatalization, velarization or labialization.

Tense vowels often differ from lax vowels by being articulated with the
tongue root advanced. This is not necessarily true of low vowels, though.
A more general (if more complex) characterization might be centralization
in lax vowels.  In consonants, the \textit{tongue}\/ subsort typically has no
perceptually appreciable consequence. But in some languages, \textsc{tongue}
values of sort \textit{tense}\/ and \textit{lax}\/ cooccur with a battery of
different articulatory properties in consonants,
cf.\ \cite{loc:lod:96}.\INA{Local, John}\INA{Lodge, Ken} Thus, the phonetic
correlates of the subsorts of \textit{tongue}\/ are not fully universal and
are in any case somewhat indirect.

Major class and manner distinctions (vowel vs.\ consonant, sonorant vs.\ 
obstruent, etc.) are captured in (\ref{B2}) by subsorts of \textit{segmproper}. (Inclusion of \textit{affricate}\/ among the manner sorts is
meant to be hypothetical; it raises questions that I cannot discuss here.)
Non-atomic sorts can thus contribute to phonetic characteristics just like
atomic sorts.  

%\cmldcomm[inline]{Der Bib-Eintrag zu jun:96 im nächsten Absatz fehlt mir. Ich vermute, dass es ``Jun, Jongho (1996) Place Assimilation is not the result of Gestural Overlap: Evidence from Korean and English. In Phonology 13.3, 377-407.'' sein könnte. Weiß das zufällig jemand? (Ich habe ``Slavic in \isi{HPSG}'' nicht und kann daher da nicht in die Bibliographie schauen.)}

Considering the attribute \textsc{achievement} that is born by 
%each 
\textit{place} objects, there is 
%ample 
evidence (mainly from stops) that the
degree of constriction that is characteristic of consonants can be reduced
under the control of the speaker, often leading to the perception of
assimilation or loss; see, e.g., \cite{nola:92}\INA{Nolan, Francis},
\cite{barr:91}\INA{Barry, Martin C.}, \cite{jun:96}\INA{Jun, Jongho}. This
can be captured 
%by the value of the attribute \textsc{achievement} that is
%born by each \textit{place} object.
by the \textsc{achievement} value.  While values of sort \textit{full} and
\textit{reduced} should be unproblematic, the value sort \textit{zero} points to
an important topic of research.

\mbox{}\cite{nola:92}\INA{Nolan, Francis} classifies as ``zero'' consonant
productions where no closure is measured (by the method used).
Correspondingly, hearers under ordinary conditions typically do not
perceive a closure in cases like this.  The main difference with mere
absence of closure lies in the fact that under favourable conditions,
hearers do perceive a closure with more than chance frequency. Also, zero
productions are typically found in free variation with reduced productions,
which is natural if zero is understood as extreme reduction. Still, it is
obvious that this concept is empirically difficult, and experiments need to
be replicated under 
the
most careful control of all potentially relevant
factors. But I take it that at least some studies have made it plausible
that zero closure, as distinct from absence of closure, is real.

In fricatives, reduced and zero achievements correspond to ``approximant''
and vowel-like degrees of constriction, respectively. A case in point might
be the rhotic in Standard German\INS{German!Standard}, which is a fricative (or
approximant) when in a syllable onset, as in \textipa{[ti:.K@]} \emph{Tiere}
`animals', but vowel-like in a coda, as in \textipa{[ti:2]} \emph{Tier} `animal'.
(Final obstruent devoicing is only observed with unreduced constrictions.)

Somewhat speculatively, I assume that the \textsc{achievement} attribute is
also appropriate for \textit{voicing}\/ objects. Its phonetic meaning differs
slightly for \textit{voiced}\/ and \textit{voiceless}\/ objects. I follow
\cite{gol:bro:86}\INA{Browman, Catherine P.}\INA{Goldstein, Louis} (and
\citealt[49\hspace*{1pt}ff.\@]{lad:mad:96}\INA{Ladefoged,
  Peter}\INA{Maddieson, Ian}) in considering the state of the glottis to be
the primary characteristic of systematically voiced vs.\ voiceless
segments: in ``voiced'' segments, the vocal folds are adducted, which is
just their neutral state in the speech mode; in ``voiceless'' segments of
sort \textit{spreadgl}\/, they are actively abducted. For there to be voice,
transglottal pressure must in ad\-dition be kept sufficiently high. The
\textsc{achievement} value relates to the active gesture.

\addlines
In a \textit{voiced}\/ object, a value of sort \textit{reduced}\/ or \textit{zero}\/
means that transglottal pressure is reduced. Thus, unvoiced word-initial
plosives as in Standard German\INS{German!Standard} \textipa{[\r*du:]}
\emph{du} `thou' can be analysed as having a \textsc{voicing} value of sort
\textit{voiced}\/ that has an \textsc{achievement} value of sort \textit{zero},
perceptually almost indistinguishable from the unaspirated \textipa{[tu:]}
\emph{tu} `do' of Austrian German\il{German!Austrian}, which has a \textsc{voicing} value of sort \textit{spreadgl}.  (Cf.\ \citealt{fleg:82}\INA{Flege,
  James Emil} for English\il{English}.)

In a \textit{spreadgl}\/ object, an \textsc{achievement} value of a sort other
than \textit{full}\/ means that the vocal folds are less than optimally
abducted; in a \textit{closedgl}\/ object, it means that the vocal folds are
not fully closed. If at the same time transglottal airflow happens to be
large enough, sufficient transglottal pressure may build up to cause voice
to be perceived. 
%\bigskip

%\noindent


\subsection{On token (non-)identities}
\label{sec:1.2.4}


By using subsorts of \textit{segment}\/ to distinguish long from short
segments, I try to forestall any temptation to view slots in a list of
segments (i.e., individual \textit{nelofsegment}\/ objects) as time units. A
physically interpreted list of segments can be viewed to be a ``timing
tier'' in two respects.  First, the articulatory gestures that correspond
to gesturally relevant components\footnote{%
	An object \textit{o}\/ is a
  ``component'' of an object \textit{o'}\/ just in case \textit{o}\/ is the value
  of a path on \textit{o'}\/. (The path can be empty.)%
}
of a segment overlap temporally. A segment thus corresponds to a constellation of gestures
during a temporal interval.  Second, the sequence of segments in the list
corresponds to a temporal sequence of constellations of gestures.  But the
duration of temporal intervals corresponding to segments in a given list
varies drastically in accordance with their position in syllables and other
factors; cf., e.g., \cite{smit:91,smit:95}\INA{Smith, Caroline L.}, and
\cite{bro:gol:88}\INA{Browman, Catherine P.}\INA{Goldstein, Louis}. Thus,
there is no unit of time that could consistently be associated to slots in
a given list. But consistent association is presupposed when a segment is
explained to be long just in case it occupies two adjacent slots, as is
usually done in ``skeletal'' theories. Otherwise this explanation is just a
marking device for phonological length as phonetically arbitrary as any
other.


The main phonological motivation for the double-slot analysis is the fact
that phonotactically, long vowels often pattern with diphthongs and long
consonants, with consonant clusters. Closer inspection, however, suggests
that these phonotactical facts are more adequately captured by syllable
theory in conjunction with a theory of phonological weight. (But I take the
theory of weight to be a research topic rather than a fully developed
theory; cf.\ \citealt[{}\S{}8.4]{kens:94}\INA{Kenstowicz, Michael},
\citealt{perl:95}\INA{Perlmutter, David}, and \citealt{hum:mul:eng:97}\INA{Hume,
  Elizabeth V.}\INA{Muller, Jennifer}\INA{van Engelenhoven, Aone} for
overviews and discussion.)
%\bigskip


%\noindent
I adopt a reformulated version of the ``Sharing Constraint'' in \cite[93,
(3.13)]{scob:97}\INA{Scobbie, James M.}:
\begin{exe}
\ex
\label{a} In a list of segments $\left\langle s_1, \ldots,
    s_n \right\rangle$, if a component \textit{o}\/ of $s_1$ that corresponds
  to an articulatory gesture is also a component of $s_j$, then \textit{o}  is also a component of every $s_i$\/ with $1 < i < j$.
\end{exe}
%
If one token gesture is overlapped by two constellations of overlapping
gestures, it is overlapped by each such constellation in
between.\footnote{%Provided, of course, a physical event is assumed 
I assume physical events to be
  temporally connected by nec\-essity. This assumption is rejected
  in \cite[41]{bir:kle:90}\INA{Bird, Steven}\INA{Klein, Ewan}
  and~\cite[73]{bird:95}.%
}
Thus, (\ref{a}) is necessarily true for
physically interpreted lists of segments (\textsc{utterance s-string} values;
cf.\ \S\ref{sectPI}).  I require not only all segments, but also all lists
of segments to be physically interpretable, hence it must be true for all
such lists. Thus, (\ref{a}) is just a natural consequence of any
empirically clear explication of autosegmental phonology, such as found,
e.g., in \cite{sage:88}.\INA{Sagey, Elizabeth}

As both \cite{sage:88}\INA{Sagey, Elizabeth} and
\cite{scob:97}\INA{Scobbie, James M.} note, (\ref{a}) is inconsistent with
many uses the ``Obligatory Contour Principle'' and ``autosegmental
spreading'' in nonconcatenative morphology have been put to. The obvious
conclusion, drawn in both~\cite[115\hspace{1pt}f.\@]{sage:88} and
\cite[223\hspace{1pt}ff.\@]{scob:97}, is that ``long distance''
identities are not token but type identities.\footnote{%
	Two objects are
  ``type identical'' if they have some significant property in common.
 An important case of type identity can be enforced by ``sort equations'': a
  description language expression of the form $\tau_1 \simeq \tau_2$
  denotes the subset of U such that on each member, paths $\tau_1$ and
  $\tau_2$ are defined and the values of the paths are of the same
  maximally specific sort.  (See~\S\ref{sectMiwok} for an example.)%
}
One can expect to see direct evidence for type (vs.\ token) identity in
particular with consonants, e.g., in reduplication and in Semitic\il{Semitic}
biradicals. It is indeed well-known that long distance geminates are not
true geminates at all%; cf.~\cite[Chapter~6]{scob:97} for discussion of these and
%related facts (also \citealt{gafo:98}\INA{Gafos, Diamandis}).
. Cf.\ \cite[Chapter~6]{scob:97}, and \cite{gafo:98}\INA{Gafos, Diamandis} for
discussion of these and related facts.

One particularly important sort of long distance identity is found in
vowel harmony. Formally, the geometry of (\ref{B2}) allows a \textsc{tongue}
value component to extend over an unrestricted number of segments, and this
seems indeed to be the case with lip rounding in the rounding harmony of
Turkish\il{Turkish} vowels (\citealt{boyc:90}\INA{Boyce, Suzanne E.}). But for several
languages, there is evidence that, e.g., in a sequence such as
\textipa{[ipi]}, there are typically two tokens of the vowel gesture,
rather than one token extending over the whole string
(\citealt[137]{gay:81},\INA{Gay, Thomas}
\citealt[12]{bel:har:81},\INA{Bell-Berti, Fredericka}\INA{Harris, Katherine}
\citealt{mca:eng:91}\INA{McAllister, Robert}\INA{Engstrand, Olle}), and there
is no evidence that vowel harmony languages necessarily disobey this
pattern.  Rather, there is acoustic evidence that some obey it
(\citealt{bess:98}\INA{Bessell, Nicola J.}).  Moreover, vowel harmony systems
often include ``transparent'' vowels that allow harmony to operate across
them even though they fail to have the harmonically relevant phonetic
property.  Thus, \cite[223]{scob:97}\INA{Scobbie, James M.} is clearly
correct in assuming that vowel harmony deals essentially with identities of
types rather than tokens.
%\bigskip

%\noindent


\subsection{Phonemic sorts}
\label{sec:1.2.5}

%\textbf{Test:}
%\begin{itemize}
%\item standard \verb+\_+: \textit{sort}\_\textipa{u}, 
%\item \verb+\underline{\hspace{0.5ex}}+: \textit{sort}\underline{\hspace{0.5ex}}\textipa{u}, 
%\item \verb+\underline{\hspace{1ex}}+: \textit{sort}\underline{\hspace{1ex}}\textipa{u}, 
%\item \verb+\underline{\hspace{1.5ex}}+: \textit{sort}\underline{\hspace{1.5ex}}\textipa{u}.
%\end{itemize}
%\textbf{End of Test}


I posit phonemic sorts (similar to those in \citealt{mast:93}\INA{Mastroianni,
  Michael}) as partitionings of major class and manner sorts; individual
languages may differ (to some extent) in their inventory of phonemic sorts.
As an illustration, some (incomplete) partitionings for German\INS{German}
are introduced in (\ref{B5}), with some of the attending restrictions given
in (\ref{B6}).  (`\textsc{first}' and `\textsc{rest}' are abbreviated \textsc{ft},
\textsc{rt}; other abbreviations are transparent.)
%
\begin{exe}
\ex
\label{B5}
  \textit{vowel}\/: \textit{sort}{\textunderscore}\textipa{i}, \textit{sort}{\textunderscore}\textipa{I}, \textit{sort}{\textunderscore}\textipa{u}, \textit{sort}{\textunderscore}ʊ, \textit{sort}{\textunderscore}\textipa{y},
  \textit{sort}{\textunderscore}ʏ, \ldots \\
  \textit{plosive}\/: \textit{sort}{\textunderscore}\textipa{k}, \textit{sort}{\textunderscore}\textipa{g}, \textit{sort}{\textunderscore}\textipa{t}, \textit{sort}{\textunderscore}\textipa{d}, \textit{sort}{\textunderscore}\textipa{p}, \textit{sort}{\textunderscore}\textipa{b}.\\
  \textit{nasalcon}\/: \textit{sort}{\textunderscore}\textipa{n}, \textit{sort}{\textunderscore}\textipa{m}, \textit{sort}{\textunderscore}ŋ.
\end{exe}
% 
\begin{figure}
  \newsavebox{\tnhone}
  \newsavebox{\tnhtwo}
  \newsavebox{\tnhthree}
  \savebox{\tnhone}{\avhr{
            \textit{sort}{\textunderscore}\textipa{n}
            \\\textsc{voicing} \avmbox{3}
            \\\textsc{velum} \textit{openvelum}
            \\\textsc{tongue} \avmbox{1}
            \\\textsc{constr}  $\left\langle \avmbox{4}\avhr{\textit{apical}\/\\\textsc{laterality} \textit{nongrooved}\\\textsc{site} \textit{alveolar}} \right\rangle$}}
  \savebox{\tnhtwo}{\avhr{\textit{short}
            \\\textsc{segmproper} \avhr{\textit{sort}{\textunderscore}\textipa{d}
              \\\textsc{voicing} \textit{voiced}\/
              \\\textsc{velum} \textit{closedvelum}
              \\\textsc{tongue}  \ldots{}
              \\\textsc{constr} $\left\langle \avmbox{4}
              \right\rangle$}
            }}
  \savebox{\tnhthree}{\avhr{\textit{short} 
        \\\textsc{segmproper} \avhr{
          \textit{sort}{\textunderscore}ʏ
          \\\textsc{voicing} $\avmbox{3}$ \textit{voiced}
          \\\textsc{velum} \avmbox{2}
          \\\textsc{tongue} \avmbox{1}\avhr{\textit{lax}\\\textsc{vertical} \textit{high}
            \\\textsc{horiz} \avhr{\textit{front}
              \\\textsc{narrow} \textit{round}}} }}}
  \avha{\textit{nelofsegment}
    \\\textsc{ft} \avhr{\textit{short}
      \\\textsc{segmproper} \avhr{
        \textit{sort}{\textunderscore}{\textipa{k}}
        \\\textsc{voicing} \textit{spreadgl}
        \\\textsc{velum} $\avmbox{2}$ \textit{closedvelum}
        \\\textsc{tongue} \avmbox{1}
        \\\textsc{constr} $\left\langle \avhr{ \textit{dorsal}\/\\\textsc{laterality}
            \textit{nongrooved}\/\\\textsc{site}
            \textit{velar}} \right\rangle$ }   }
    \\
    \\\textsc{rt} \avhr
    {\textit{nelofsegment}
      \\\textsc{ft} \usebox{\tnhthree}
      \\
      \\\textsc{rt} \avhr{\textit{nelofsegment}
        \\\textsc{ft}
        \avhr{
          \textit{short}
          \\\textsc{segmproper} 
          \usebox{\tnhone}
          }\\
        \\\textsc{rt} 
        \avhr{\textit{nelofsegment}
          \\\textsc{ft}
          \usebox{\tnhtwo}
          }
        }}
    }
  \caption{\label{kuend} Phonemic structure of {[kʏnd]}}
\end{figure}
\begin{exe}
\ex
\label{B6}
  \textit{sort}{\textunderscore}\textipa{i} $\rightarrow $
  \textsc{tongue} {\small \avhr{\textit{tense}
      \\\textsc{vertical}
      \textit{high}
      \\\textsc{horiz} \avhr{\textit{front}
        \\\textsc{narrow} \textit{nonround}}
      }
    }
  \\
  \\\textit{sort}{\textunderscore}ʏ $\rightarrow $ 
  \textsc{tongue} {\small \avhr{\textit{lax}
      \\\textsc{vertical} \textit{high}
      \\\textsc{horiz} \avhr{\textit{front}
        \\\textsc{narrow} \textit{round}}
      }
    }
  \\
  \\\textit{sort}{\textunderscore}\textipa{k} $\rightarrow $ {\small
    \avhr{\textsc{voicing} \textit{voiceless}
      \\\textsc{constr ft site} \textit{velar}} 
    }
  \\
  \\\textit{sort}{\textunderscore}\textipa{d} $\rightarrow $ {\small
    \avhr{\textsc{voicing} \textit{voiced}
      \\\textsc{constr ft site} \textit{alveolar}} 
    }
  \\
  \\\textit{sort}{\textunderscore}\textipa{n} $\rightarrow $ 
  \textsc{constr ft site} \textit{alveolar}
\end{exe}
% 
Sorts like these do not only serve to keep descriptions more compact, but
play a useful role in enforcing type identities and in determining which
components of segments are (not) free to subphonemic variation.\footnote{%
	In
  consonants whose \textsc{constriction} value has more than one member, the
  role of phonemic sorts is formally nontrivial. The order of list members
  being immaterial, a voiced plosive with \textsc{constr} value $\left\langle
    \textit{dorsal}, \textit{labial}\/ \right\rangle$ can be of \textit{sort}{\textunderscore}\textipa{b} or of \textit{sort}{\textunderscore}\textipa{g}. In a language that has a
  phonemic double articulation sort such as \textit{sort}{\textunderscore}{\textipa{gb}}, a
  segment with the same \textsc{constr} value could also be of that sort.
  
  There may be further roles for phonemic sorts. Some speakers of Standard
  German\INS{German!Standard} progressively assimilate syllabic alveolar
  nasals to uvular {\textipa{[K]}}: they have {\textipa{[ti:.K\textscn]}}
  next to {\textipa{[ti:.Kn]}} \emph{Tieren} (dat.\ pl.). What sort should
  the uvular nasal belong to? There are several possible responses, one of
  them being that \textit{sort}{\textunderscore}\textipa{n} objects are not unconditionally
  required to be alveolar, as they are in (\ref{B6}), but only when they
  are in an environment that does not induce assimilation.%
}

A conceivable description of the initial four members of an \textsc{s-string}
value for German\INS{German} \textipa{[kYndIg@]} \emph{k\"undige} `terminate'
is given in Figure~\ref{kuend}. (I assume that in this configuration the
tongue gestures of the vowels overlap the adjacent consonants and that with
voiced obstruents a voicing gesture typically sets in.)
%consonants.)
 





\section{Phonetic strings}
\label{sectPhonstr}


Phonetic strings may exhibit \isi{segmental} phenomena that are not naturally
captured by s-strings. A possible case in point are ``transitional''
segments. For instance, corresponding to the substring \textipa{[Yn]} as
described in Figure~\ref{kuend} there probably is a phonetic string where
the velic opening gesture sets in before the alveolar constriction gesture
does; this might be described as in Figure~\ref{yyn}.
\begin{figure}[ht]
\avha{\textsc{ft 
 segmproper} \avhr{
     \textit{sort}{\textunderscore}ʏ
   \\\textsc{velum} \textit{closedvelum}
   }
\\
\textsc{rt} \avhr{
    \textsc{ft segmproper} \avhr{
       \textit{sort}{\textunderscore}ʏ
     \\\textsc{velum} \avmbox{1} }
    \\
\textsc{rt ft
 segmproper}\avhr{\textit{sort}{\textunderscore}\textipa{n}
                                   \\\textsc{velum} $\avmbox{1}$ \textit{openvelum}}}}
\caption{\label{yyn} Transitional segment in {[ʏʏ̃n]}} %cmldcomm:check
\end{figure}


As noted earlier, segments correspond to time intervals (of varying
durations) of overlapping articulatory gestures.  In this respect,
transitional segments as in Figure~\ref{yyn} are no different from phonemic
segments as in Figure~\ref{kuend}. More particularly, in Figure~\ref{kuend} the
alveolar closure of \textipa{[nd]} and the tongue characteristics of \textipa{[kYn]} are both partially overlapped by the velar opening of \textipa{[n]}.
The only new aspect in Figure~\ref{yyn} is the lack of alveolar closure in
\textipa{[\~Y]}. There is thus no reason to exclude transitional
segments from the universe of linguistic objects in principle.\footnote{%
	On the face of it, transitional segments such as
  {\textipa{[\~Y]}} can be avoided by allowing values of paths
  on segments to be lists whose order of members is temporally interpreted.
  Thus, \textipa{[Y\~Y]} would be one segment with a \textsc{velum}
  value $\left\langle closedvelum, openvelum\/ \right\rangle$. I prefer to
  eschew ``contour'' values like that, though: they complicate matters in
  that they, in effect, introduce something like segments within segments,
  yet do not, as far as I see, solve any problem that transitional segments
  might be attended with.%
}

I assume that transitional segments do not ``feed'' morphophonological
phenomena. It thus should be sufficient to associate strings as
in Figure~\ref{yyn} with signs that are not embedded in another sign.
Unembedded signs, which can be performed as (complete) utterances, need to
be distinguished by a special sort for quite independent syntactic,
semantic and pragmatic (e.g., illocutionary force) reasons. Adapting a
proposal in \cite[134\hspace{1pt}ff.\@]{rich:97}\INA{Richter, Frank}, I
partition the sort \textit{phrase} and require unembedded phrases to bear a
distinguished phonological attribute whose value hosts a non-empty list of
segments: 
\begin{exe}
\ex
\label{partialoverlap}
\begin{tabbing}
\hspace{1,5em}\=\hspace{1,5em}\=\hspace{1,5em}\=\hspace{1,5em}\=\hspace{6em}
\=\kill
\textit{phrase}    \>\>\>\>\>                      \ldots\\
\>\textit{embedded-phrase}\\
\>\textit{unembedded-phrase}       \>\>\>\>\textsc{utterance} \textit{phon}
\end{tabbing}
\ex
\label{UTTnel}
\textit{unembedded-phrase} $\rightarrow$ \textsc{utterance s-string} \textit{nelofsegment} 
\end{exe}
% 
For ease of expression, I reserve the term `s-string' to \textsc{phon
  s-string} values and call an \textsc{utterance s-string} value a
``p-string.'' Figure~\ref{yyn} can now be understood as (an informal
rendering of) a description language expression that denotes (a set of)
components of some p-strings.


In simple cases, a phrase's p-string may be identical to its s-string. In
cases where transitional segments are to be components of signs, the
p-string is related to the s-string in the way suggested by the relation of
\textipa{[Y\~Yn]} to \textipa{[Yn]}. In general, though, the relation is
not necessarily that simple. For German\INS{German} \textipa{[OK.dn@]}
\emph{ordne} `put in order', many speakers have an alternate pronunciation
\textipa{[OK.n@]}. The plosive \textipa{[d]} cannot be overlapped by a
lowered velum gesture, according to (\ref{B2}), hence it cannot appear as a
member of the p-string in the latter case. The same is often observed with
syllabic nasals, as in \textipa{[le:.N]} next to \textipa{[le:.gN]}
\emph{legen} `lay'.  Similarly, \textipa{[b@.am.tn]} \emph{Beamten}
`officials' has an alternate pronunciation \textipa{[b@.am.pm]}, discussed
in \cite{kohl:92}\INA{Kohler, Klaus J.}, where the alveolar constriction of
\textipa{[tn]} is not just overlapped, but in fact replaced, by the
bilabial constriction from \textipa{[m]}. If these variations are to be
accounted for by the relation of p-strings to s-strings, as seems
reasonable, the relation cannot in general preserve segments or even
phonemic sorts.





\section{Physical interpretation} 
\label{sectPI}

%%%Intuitively, in each token \textit{unembedded-phrase}\/ object, components of
%%%its p-string are intended to correspond to (possible) physical events. That
%%%is, there is thought to be a total bijective function ϕ from the set
%%%of \textsc{utterance s-string} values of a language L to the set of possible
%%%utterance events of L such that each complete sequence of {\it
%%%  nelofsegment}\/ objects is mapped to a complete utterance event, each
%%%component of a gestural sort (e.g., \textit{voicing}\/, \textit{velum}\/, {\it
%%%  place}\/, \textit{lateralclosure}\/) is mapped to an articulatory gesture,
%%%and each \textit{segmproper}\/ component is mapped to a constellation of such
%%%gestures during a section of the utterance.

%%Intuitively, in each token \textit{unembedded-phrase}\/ object, components of
%%its p-string  are intended to correspond to (possible) physical events. As
%%an explication, I propose there is a total bijective function ϕ from
%%the set of p-strings of language L to the set of u-equivalences of L, where
%%a ``u-equivalence'' is a  set of possible utterance events. To explain this
%%notion, I temporarily suppose that a u-equivalence is just a singleton set.
%%Thus, each p-string (a maximal \textit{nelofsegment}\/ object) is mapped to a
%%complete utterance event (the member of a u-equivalence) such that each
%%component of a gestural sort (e.g., \textit{voicing, velum, place,
%%lateralclosure}\/) can be mapped to an articulatory gesture and each {\it
%%segment}\/ component, to a constellation of such gestures during a section
%%of the utterance.

%%A possible utterance event, and with it, the u-equivalence containing it,
%%is  individuated by being the utterance of some possible speaker S at some
%%temporal interval t. A set of ``repetitions of an utterance'' is a set of
%%u-equivalences that corresponds (bijectively) to a set of congruent
%%p-strings.

%%More exactly, each repetition can vary along several continuous dimensions
%%that are considered to be ``stylistic,'' i.e., linguistically irrelevant:
%%loudness, pitch level, speaking rate, etc. Thus, a u-equivalence is a
%%maximal non-denumerable set of alternative possible utterances of the same
%%S at the same t.

%%From: hoehle@uni-tuebingen.de (Tilman Hoehle)
%%To: Adam Przepiorkowski <adamp@sfs.nphil.uni-tuebingen.de>
%%Subject: Korrektur der Korrekturen
%%Date: Sat, 23 Jan 1999 19:12:49 +0100

%%Lieber Herr Przepiorkowski,

%%in der Mail "(Letzte ?) Korrekturen", die ich gestern geschickt habe, bitte
%%ich, den ersten Absatz von §4 durch einen Text von 3 Absaetzen zu ersetzen.
%%Bitte nehmen Sie NICHT den dort genannten Text, sondern den HIER folgenden.
%%(Der zweite und dritte Absatz sind geaendert; die Gesamtlaenge ist gleich.)

%%Danke
%%Tilman Hoehle

%Intuitively, in each token \textit{unembedded-phrase}\/ object, components of
%its p-string  are intended to correspond to (possible) physical events. As
%an explication, I propose there is a total bijective function ϕ from
%the set of p-strings of language L to the set of u-equivalences of L, where
%a ``u-equivalence'' is a  set of possible utterance events. To explain this
%notion, I temporarily suppose that a u-equivalence is just a singleton set.
%Thus, each p-string (a maximal \textit{nelofsegment}\/ object) is mapped to a
%complete utterance event (the member of a u-equivalence) such that each
%component of a gestural sort (e.g., \textit{voicing, velum, place,
%lateralclosure}\/) can be mapped to an articulatory gesture and each {\it
%segment}\/ component, to a constellation of such gestures during a section
%of the utterance.

%A u-equivalence and its member(s) are associated with the same possible
%speaker S and the same temporal interval t.  A set of ``repetitions of an
%utterance'' is a set of u-equivalences that corresponds (bijectively) to a
%set of congruent p-strings such that the u-equivalences differ pairwise
%wrt.~S 
%with respect to S 
%or, for the same S, 
%%wrt.~t.
%with respect to~t.  For a given S and a given
%p-string, the set of repetitions is denumerable.

%More exactly, repetitions can vary along  continuous dimensions that are 
%considered to be ``stylistic,'' i.e., linguistically irrelevant: loudness,
%pitch level, speaking rate, etc. 
%%Thus, a u-equivalence is a maximal
%%non-denumerable set of alternative possible utterances.
%Thus, a u-equivalence is a maximal set of alternative possible utterances,
%and it clearly is non-denumerable.

Intuitively, in a token \textit{unembedded"=phrase}\/ object, components of its
p-string  are intended to correspond to (possible) physical events. As an
explication, I propose there is a total bijective function ϕ from the
set of p-strings of language L to the set of u-equivalences of L, where the
latter partitions the  set of possible utterance events of L\@. To explain
this, I temporarily suppose that a u-equivalence is just a singleton set.
Thus, each p-string (a maximal \textit{nelofsegment}\/ object) is mapped to a
complete utterance event (the member of a u-equivalence) such that each
component of a gestural sort (e.g., \textit{voicing}, \textit{velum}\/, \textit{place}\/, \textit{lateralclosure}\/) is mapped to an articulatory gesture and
each \textit{segment}\/ component, to a constellation of such gestures during
a section of the utterance.

More exactly, for a given (token) p-string, I take there to be a choice
among possible utterance events that differ along  continuous dimensions
that are  considered to be ``stylistic,'' i.e., linguistically irrelevant:
loudness, pitch level, speaking rate, etc.  Thus, a u-equivalence actually
is a maximal set of stylistically alternative possible utterance events; it
clearly is non-denumerable.

%%%%%%%%%%%%%%%%%%%%%%%%%%%%%%%%%%%%%%%%%%%%%%%%%%%%%%%%%%%%%%%%%%%%%%%%%%

We need to have ϕ in order to give an accurate empirical explication
to our phonological notions. The apparatus sketched in the previous
sections is unable to characterize well-known properties of gestures
adequately. Velar closure events, e.g., differ considerably (in many
languages) in accordance with adjacent (front or back) vowel gesture
events; conversely, low vowel gesture events that overlap an alveolar
constriction event are very different from the gesture events for the same
vowel without alveolar overlap. That is, the notion of ``gesture'' that we
have relied on is a highly abstract one; the precise properties of
individual gesture events cannot be explicated without a detailed theory of
the coarticulatory interaction of different gestures.

Any such theory must take seriously the fact that gestures are constituted
by actions of spatiotemporal physical entities (``articulators'' and air
volumes) that change their location over time. To be adequate, such a
theory~-- however abstract it might be~-- must have unrestricted recourse to
the relevant physical theories and their attendant mathematical apparatus.
For brevity, I call such a theory a ``physical'' theory.

One 
%(particularly interesting) 
physical theory of that kind is known as
``task dynamics.'' Cf.\ \cite{salt:95}\INA{Saltzman, Elliot L.} for a general
exposition, and \cite{sal:mun:89}\INA{Munhall, Kevin G.} for a detailed one; also
\cite{mcg:sal:95}\INA{McGowan, Richard S.} for an extension.  Adapting it to
the present frame of reference (and simplifying somewhat), we can say that
the physical theory defines a mapping from 
%relevant token linguistic objects to sets of alternative
p-strings and their component objects to maximal sets of stylistically
alternative
possible physical events in accordance with an object's gestural
properties and the influences of contemporaneous and immediately preceding
physical events.\footnote{%
	To capture the role of syllable structure and
  stress for the temporal duration and phasing of gestures, the physical
  theory must access the full \textsc{utterance} value.%
}

I do not take models of grammars to include physical events. Rather,
phonological linguistic objects are as abstract as any other linguistic
object, and (sets of alternative) possible events constitute the
interpretation (in the model-theoretic sense) of the relevant linguistic
objects, with a physical theory defining the interpretation function, i.e.,
ϕ.\footnote{%
	I thus concur with \cite{sage:88}\INA{Sagey, Elizabeth},
  \cite{pier:90}\INA{Pierrehumbert, Janet.}, \cite{cole:98}\INA{Coleman,
    John} and others that the relation of phonological linguistic objects
  to physical events is ``semantic'' in nature. I will not explore here the
  ontologically more parsimonious hypothesis that %phonological linguistic
  those objects are physical events themselves, so that some ϕ$'$ maps events
  as structured by sorts and attributes to the same events as structured by
  a physical theory.%
}

As we have seen in the previous section, and will see again later on, a
component of an s-string is not necessarily also a component of a p-string.
Although I consider it important that each (list of) segment(s) be
physically interpretable in principle (cf.\ (\ref{a}) above), there is no
reason to insist that each of them is actually interpreted. The natural
assumption is that 
%wrt.\ 
with respect to 
the set of \textit{nelofsegment}\/ objects, ϕ is
partial in that it is only p-strings that are in its domain.

There are good reasons (in particular, from acoustically hidden
articulatory gestures; cf., e.g.,
\citealt[363\hspace*{1pt}ff.\@]{bro:gol:90})\INA{Browman, Catherine
  P.}\INA{Goldstein, Louis} to require (sets of alternative) articulatory
events to be in the range of ϕ. But in principle, the range might also
include certain kinds of acoustic events (e.g., for vowels) that are only
secondarily related to articulatory events, as argued for
in \cite{lade:90}\INA{Ladefoged, Peter} and
\cite{per:mat:svi:jor:95}.\INA{Perkell, Joseph S.}\INA{Matthies, Melanie
  L.}\INA{Svirsky, Mario A.}\INA{Jordan, Michael I.}

It seems clear that the defining characteristics of ϕ can in
%substantial parts
large part be universal. The assumption that there is a strictly
universal core to them is in fact a conceptual necessity if (at least major
aspects of)~(\ref{B2}) and~(\ref{B3}) are to have the empirical import they
are intended to have.  But it seems equally clear that they are not fully
universal. Thus, fine details of gestures in different languages just
escape the categorial distinctions that can be adequately captured by sorts
of linguistic objects (cf.\ \citealt{lade:80}\INA{Ladefoged, Peter}). Some
details of ϕ's characteristics apparently have to be learned for
individual languages; but the learning theory for these details may be
conceived to be quite weak.  %\bigskip

%\noindent
According to \cite{sole:95}\INA{Sole@Sol{\'e}, Maria-Josep}, Spanish\il{Spanish} shows
in vowels before nasal consonants significant anticipatory nasalization
which is durationally constant across varying vowel durations in differing
speaking rates. In American English\il{English}, by contrast, anticipatory nasalization
sets in at the beginning of the vowel (at the end of a preceding
consonant), regardless of the vowel's duration.
\cite{sole:95}\INA{Sole@Sol{\'e}, Maria-Josep} suggests that the
nasalization in English\il{English} is ``phonological,'' whereas in Spanish\il{Spanish} it is an
``automatic phonetic'' phenomenon.

We can reconstruct this intuition by assuming that in English\il{English}, there is one
fully nasal \textit{vowel}\/ object in the p-string, whereas in Spanish\il{Spanish}, there
is no nasal vowel in it, not even a transitional one as
in Figure~\ref{yyn}.  Rather, a transitional nasal vowel event is predicted
by ϕ. This is formally possible since the mapping from lists of
segments to events may be less than trivially transparent. It is also
compatible in principle with the fact that some languages otherwise similar
to Spanish\il{Spanish} differ in details of nasalization
(cf.\ \citealt{clum:76}\INA{Clumeck, Harold}), since we recognize that parts of
ϕ's characteristics can be learned.

This construal appears plausible. In English\il{English}, the onset of velic lowering
seems to be phased 
%wrt.\ 
with respect to
the offset of a consonant gesture preceding the
vowel, and there does not seem to be a natural alternative way to express
this. In Spanish\il{Spanish}, the constant phasing of the onset of velic lowering 
%wrt.\ 
with respect to
the onset of the nasal consonant gesture is in fact what we might expect of
gestures that are coupled by a sort (subsort of \textit{nasalcon}\/). Still,
the present state of theorizing about ϕ does not warrant the
conjecture that transitional segments as hypothetically illustrated
in Figure~\ref{yyn} are universally non-existent. (Cf., e.g.,
\citealt[164]{moha:86}\INA{Mohanan, Karuvannur P.} for a candidate example.)

According to \cite{cohn:93}\INA{Cohn, Abigail C.}, the velocity of velic
opening (or closing, in persevering nasalization) in predictably nasalized
vowels adjacent to nasal consonants is (under most circumstances) much
slower in English\il{English} than it is in Sundanese\il{Sundanese}. Although
\cite{cohn:93}\INA{Cohn, Abigail C.} interprets this as a difference
between ``phonetics'' and ``phonology,'' it simply seems that in English\il{English},
there is one (long, and hence, slow\footnote{%
	In long consonants, the
  closing movement is much slower than in short consonants
  (\citealt[217\hspace{1pt}ff.\@]{smit:95}\INA{Smith, Caroline L.}). This
  accords well with the assumption
  in \cite[222]{krog:93}\INA{Kroger@Kr{\"o}ger, Bernd J.} that
  articulator velocity is a function of the gesture's duration and
  amplitude.%
}) velic opening/""closing gesture extending through the
consonant and the vowel, whereas in Sundanese\il{Sundanese}, there is a sequence of two
(overlapping) velic gestures for the consonant and the vowel. If correct,
this means that in Sundanese\il{Sundanese}, but not in English\il{English}, there are two different
\textit{openvelum}\/ objects in adjacent p-string members.  There are thus
many more instances of type (vs.\ token) identities in strings of segments
than assumed in much recent work.
Cf.\ \cite[362--365]{sal:mun:89}\INA{Saltzman, Elliot L.}\INA{Munhall,
  Kevin G.} for related considerations on phasing relations.  %\bigskip

%\noindent
Just as with transitional segments, the question arises whether the work of
\textsc{achievement} values can and should be done by the definition of
ϕ. Two considerations might support this alternative.  Reductions do
not seem to play a role in phonemics;\footnote{However, approximants and
  fricatives are reported to contrast phonemically in some languages
  (\citealt[76 and 324]{lad:mad:96}\INA{Ladefoged, Peter}\INA{Maddieson,
    Ian}).%
}
this would follow from them not being accessible in s-string
members. And when a \textit{place}\/ (or \textit{voicing}\/) component of an
s-string fails to be a component of the p-string, the \textsc{achievement}
value of that component may in many cases be spuriously ambiguous. On the
other hand, languages are known to differ in the conditions and the extent
to which reductions are used (cf.\ \citealt{jun:96}\INA{Jun, Jongho}).
Relegating them to ϕ thus locates further aspects of linguistic
knowledge outside the grammar and presupposes a rather strong learning
theory for ϕ's characteristics.

How the grammar of phonetics and the theory defining ϕ interact is a
large topic of research that can only proceed by working out detailed
theories of both.  The presence of the \textsc{utterance} and \textsc{achievement} attributes is thought to be conducive to this enterprise.






\section{Morphology: Four basic issues}

For morphophonology, {\small PS94}'s notion of signs is highly attractive
in that both simple and complex signs bear a \textsc{phonology} attribute.
This fact by itself provides the multistratality needed to account for
``postlexical'' phonological phenomena (e.g., sandhi). This notion,
moreover, generalizes naturally to morphology such that words are built
from smaller signs which, by bearing their own \textsc{phonology} attributes,
provide the strata that I hypothesize are necessary and sufficient to
account for word-internal phonological alternations.\footnote{%
%These notions
%  bear significant similarity to views propounded
%  in \cite{whee:81}\INA{Wheeler, Deirdre}.  
The way morphology and
  phonology interact here can be viewed as a straight explication of a core
  idea of ``Lexical Phonology,'' e.g.,
  in \cite[142\hspace{1pt}f.\@]{moha:86}\INA{Mohanan, Karuvannur P.}%
}
(See also~\cite{orgu:96}\INA{Orgun, Cemil Orhan} for a vivid defense of the
same view.)%\bigskip


%\noindent
Although many aspects of morphological theory can be left open for the
purpose of morphophonology, four basic topics need to be commented upon:
(i)~lexical licensing, (ii)~combinatorial properties of morphemes, (iii)~phonological effects of morphological combining, (iv)~the triggering of the
phonological effects.

In natural languages the smallest signs are conventionalized tuples of (at
least) semantic, phonological, morphosyntactic and combinatorial
properties. To be able to account for the tuples being conventionalized
(rather than for any formal reason), {\small PS94}, like most grammatical
theories, requires the smallest signs in a model (in {\small PS94}: words,
i.e., objects of sort \textit{word}\/) to be ``licensed'' by ``lexical
entries.'' Although {\small PS94} does not indicate how licensing is to be
formally achieved, one obvious way (taken, e.g.,
in \citealt{Pollard93}\INA{Pollard, Carl}) is to include a ``Word
Principle''\INS{Principle!Word} of the form (\ref{B7}) among the
restrictions of the grammar: 
\begin{exe}
\ex
\label{B7}
% AP:
$word \rightarrow (\mbox{le}_1 \vee   \ldots{} \vee  \mbox{le}_n\/)$   
\end{exe}
%$word \rightarrow (\mbox{le}$_1 \vee   \ldots{} \vee$    \mbox{le}_n\/)$   
%}
Thus, each word must satisfy one of \textit{n-}many lexical entries le$_i$,
each of which is a description language expression that denotes the set of
objects that are ``licensed'' by that entry. The set of disjuncts in the
consequent of~(\ref{B7}), then, constitutes a ``lexicon.''

In a language with productive derivational morphology or/""and compounding,
the set of disjuncts in (\ref{B7}) would typically have to be infinite,
which is impossible by the definition of the description language. The
natural response (given, e.g., in \citealt{kri:ner:93}\INA{Krieger,
  Hans-Ulrich}\INA{Nerbonne, John}) is to recognize formally that words can
have smaller signs as components, and modify~(\ref{B7}) accordingly.

Empirically, larger signs can also be conventionalized in that their
properties do not fully derive from the properties of their component parts
by general rule. This is sometimes seen in phrases, and often in complex
words. To deal with cases like these, we would need to have a general
formal theory of idioms. Although the theory of collocations sketched
in \cite[{}\S{}4]{ric:sai:98}\INA{Richter, Frank}\INA{Sailer, Manfred}
appears to be a promising basis to build such a theory on, I ignore these
cases for the time being.

As for the second topic, minimal signs (morphemes) differ in their
combinatorial properties just 
%like 
as words differ in theirs, except that the
selectional mechanisms for morphemes are considerably more powerful than
those for words. Although there is no shortage of empirical research (e.g.,
\citealt{fabb:88}\INA{Fabb, Nigel}, \citealt{ston:94}\INA{Stonham, John T.}, and
references therein), there does not appear to be a (successful) general
formally explicit theory of selection.

Our primary interest is in the phonological effects on a parent of the
combination of morphological daughters, i.e., the morphophonology.  The
simplest cases of concatenative morphology are similar to phrasal \isi{segmental}
phonology in that just the order in the parent of the daughters'
concatenated \textsc{s-string} values must be specified, possibly including
edge effects (sandhi). But sometimes, the resulting s-string must in
addition conform to phonological rules (such as umlaut) that go beyond edge
effects and may be specific to individual (classes of) daughters. In
nonconcatenative morphology, the result of combination is nontrivial to
begin with. The question, then, is how the particular phonological effects
of combination are formally triggered by the elements combined. Given that
the more complicated phonological effects on a parent are typically induced
by affixes, one may expect this to be a consequence of the more powerful
selectional mechanisms for morphemes.

In a sketch of inflexional morphology, \cite{kri:ner:93}\INA{Krieger,
  Hans-Ulrich}\INA{Nerbonne, John} employs an unusually expressive way of
lexical licensing. Words are supposed to contain a ``stem'' as value of a
path \textsc{morph stem} and an ``ending'' as value of a path \textsc{morph
  ending}. There is a lexicon for objects of sort \textit{word}\/ as indicated
in (\ref{KN}a) (p.~105), and there appears to be a second lexicon for \textit{word}\/ objects as indicated in (\ref{KN}b) (p.~122):
%
\begin{exe}
\ex
\label{KN}
\begin{xlist}
\ex
$word  \rightarrow   \mbox{([\textsc{morph ending} e}_1 \wedge  
\mbox{\textsc{synsem} es}_1 \wedge \mbox{ed}_1$] $\vee\ldots$ \hfill \\ 
        \hspace*{6em} $\ldots{} \vee$ [\textsc{morph ending} e$_m \wedge \mbox{\textsc{synsem} es}_m \wedge \mbox{ed}_m$])
%
\ex
$word  \rightarrow   \mbox{([\textsc{morph stem} s}_1 \wedge  
\mbox{\textsc{synsem} ss}_1 \wedge \mbox{sd}_1$] $\vee\ldots$ \hfill \\ 
        \hspace*{6em} $\ldots{} \vee$ [\textsc{morph stem} s$_r \wedge \mbox{\textsc{synsem} ss}_r \wedge \mbox{sd}_r$])
\end{xlist}
\end{exe}
% 
Thus, (\ref{KN}a) requires each ending e$_i$ with $1\le i\le m$\/ to
cooccur with a certain \textsc{synsem} value of the word, described by
es$_i$\/, and with further (e.g., phonological) properties of the word or
some part of it, described by ed$_i$\/; and (\ref{KN}b) requires the same
of each stem s$_i$ with $1\le i\le r$.

In effect, then, the sets of disjuncts in (\ref{KN}a) and~(\ref{KN}b) are
lexicons for objects (endings and stems) that are proper components of a
larger \textit{word}\/ object that satisfies the antecedent. Therefore, an
embedded object can determine any property of the word it is a component
of. Hence, lexicons like these can easily answer most questions of
selection (including mutual selection of the components, if there is a pair
of such lexicons) and can specify the effects of combination directly,
thereby answering the triggering question. Thus, the four topics remarked
upon above receive a formally simple homogeneous treatment, just by the
particular way of lexical licensing.  
%Still, 
However, expressive though this
approach is, it has its limitations, as will be seen in a moment.

For a sketch of word structure, we assume a simple extension of the sort
hierarchy below \textit{sign}\/:
%
\begin{exe}
\ex
\label{mstruc}
\begin{tabbing}
\hspace{1,5em}\=\hspace{1,5em}\=\hspace{1,5em}\=\hspace{1,5em}\=\hspace{6em}
\=\kill
\textit{sign}\>\>\>\>\>\ldots\\
\>\textit{phrase}\>\>\>\>\ldots\\
\>\textit{word}\>\>\>\>\textsc{morph} \textit{morph}\\
\>\textit{morph}\\
\>\>\textit{simplemorph}\\
\>\>\textit{complexmorph}  \>\>\>\textsc{mhead} \textit{morph}\\
\>\>\>\>\>\textsc{mnonhead} \textit{morph}
\end{tabbing}
\end{exe}
%
This sketch is 
%simplifying
simplified in many ways; e.g., it only allows for
morphologically endocentric structures. An example from German\INS{German}
may illustrate structures motivated on morphological grounds
and their
ensuing complexities.
\begin{figure}[t]
\newsavebox{\tnhfour}
\newsavebox{\tnhfive}
\savebox{\tnhfour}{\avhr{\textit{complexmorph}
        \\\textsc{p s-s} $\left\langle
          \mbox{{\textipa{p, K, y:, f,
                U, N, s}}}
        \right\rangle$\\
        \\\textsc{mhead} \avhr{\textit{simplemorp}h \\\textsc{p s-s}
          $\left\langle 
            \mbox{\textipa{s}}\right\rangle$}\\
        \\\textsc{mnonhead} \avhr{\textit{complexmorph}
          \\\textsc{p s-s}
          $\left\langle \mbox
            {{\textipa{p, K, y:, f, U,
                  N}}}\right\rangle$\\
          \\\textsc{mhead} \avhr{\textit{simplemorph} \\\textsc{p
              s-s} 
            $\left\langle \mbox{{\textipa{U,
                    N}}}\right\rangle$}\\
          \\\textsc{mnonhead} \avhr{\textit{simplemorph} \\\textsc{ p
              s-s} 
            $\left\langle \mbox{{\textipa{p,
                    K, y:, f}}}\right\rangle$}}}}
\savebox{\tnhfive}{\avhr{\textit{simplemorp}h
        \\\textsc{p s-s}  $\left\langle
          \mbox{{\textipa{l, o:, z}}}\right\rangle$}}
  \avha{\textit{word}\/\\\textsc{p s-s} $\left\langle\mbox{{\textipa{p, K,
            y:, f, U, N, s, l,
            o:, s}}}\right\rangle$
    \\\textsc{m} \avhr{\textit{complexmorph}
      \\\textsc{p s-s} $\left\langle \mbox{{\textipa{p, K, y:, f,
              U, N, s, l, o:,
              z}}}
      \right\rangle$\\
      \\\textsc{mhead} \usebox{\tnhfive}\\
      \\\textsc{mnonhead} 
      \usebox{\tnhfour}}}
  \caption{\label{p.los} Morphological structure of \emph{pr\"ufungslos}}
\end{figure}

The adjective \emph{pr\"ufungslos} `examinationless'
in Figure~\ref{p.los}\footnote{Here and later on, phonetic symbols that
  appear as (or in) list members are meant to informally indicate relevant
  properties of \textit{segment}\/ and \textit{segmproper}\/ objects. They do not
  indicate phonemic sorts.  The attribute symbols \textsc{morph},
  \textsc{phon(ology)} and \textsc{s(egmental)-string} are abbreviated
  \textsc{m}, \textsc{p} and \textsc{s-s}.%
}
contains a suffix \emph{-los}
`-less' that derives adjectives from nouns. In it, the regular voicing
alternation in German\INS{German} obstruents (\textipa{[z} $\sim$
\textipa{s]}) can be seen.  I will not go into this; rather, the somewhat
different alternation in Russian\il{Russian} obstruents will be discussed
in \S\ref{sectRuss}.  \emph{Pr\"ufung} `examination' is a noun derived from
the free verbal stem \emph{pr\"uf-} `examine' by the suffix \emph{-ung}
`\mbox{-ation}'.  \emph{-s-} is a ``linking morpheme'' (Fugenmorphem) that
com\-bines with a noun to the left to yield a bound complex that combines
with certain suffixes, as in the present case, or with a free stem to the
right. Thus, if being bound is analysed as a matter of selection,
\emph{pr\"ufungs-} and \emph{-los} must mutually select each other.

\addlines
If the binary branching morphological structure assumed
in Figure~\ref{p.los} is correct, the combinatorial properties of
\emph{-s-} cannot be captured by a set of lexicons for components of \textit{complexmorph} objects in the manner of (\ref{KN}). A lexicon that
contains~(\ref{s}a) (suitably enriched) as a member can account for \emph{-s-}
being bound to its left sister.  But to capture the fact that its parent
constituent is bound to its right sister, there would have to be another
lexicon containing~(\ref{s}b) as a member.  It surely is irritating that
there should be two different lexical entries for \emph{-s-}. But worse, the
lexicon containing~(\ref{s}b) does not contain other members such that
their disjunction could correctly capture the distribution of \emph{-s-}.
\begin{exe}
\ex
\label{s}
\begin{xlist}
\ex
\attop{
  {\small
    \avhr{\textsc{mhead} \avhr{\textit{simplemorph}\\
        \textsc{phon s-string} $\left\langle \mbox{\textipa{s}}\right\rangle$}\\
      \textsc{mnonhead} \ldots}
    }
}
\ex
\attop{
  {\small 
    \avhr{\textsc{mnonhead mhead} \avhr{\textit{simplemorph}\\
        \textsc{phon s-string} $\left\langle \mbox{\textipa{s}} \right\rangle$}\\
      \textsc{mhead} \ldots}}
  }
\end{xlist}
\end{exe}
%
\enlargethispage{\baselineskip}
\begin{sloppypar}
\noindent
Clarifying the interaction of lexical licensing, combinatorial properties
of morphemes, and triggering the phonological effects of the combination of
morphemes is thus left to future research. I henceforth concentrate on the
effects themselves.
\end{sloppypar}







\section{Nonconcatenative morphology}
\label{sectMiwok}



Sierra Miwok\il{Miwok|(} is known for its multiform
nonconcatenative morphology; see \cite{smit:85,smit:86}\INA{Smith, Norval}, \cite{sloa:91}\INA{Sloan,
  Kelly Dawn} and references therein. Part of its inflexional verbal stem
formation has also received an illustrative \isi{HPSG} analysis
in \cite{bir:kle:94}\INA{Bird, Steven}\INA{Klein, Ewan} (based
on~\citealt[83--95]{gold:90}).\INA{Goldsmith, John A.}

For a typical derivational pattern, we consider~(\ref{lVp}) with the
monomorphemic bases in (\ref{lVp2}),
after~\cite[376\hspace{1pt}f.\@]{smit:85}:\INA{Smith, Norval}
\begin{exe}
\ex
\label{lVp}
\begin{xlist}
\ex
\emph{\textipa{P}ojsi-li:p-} `quadruplets' 
%
\ex
\emph{mahko-lo:p-} `quintuplets'  
%
\ex
\emph{na\textipa{P}\v{c}a-la:p-} `ten at a time' 
\end{xlist}
\end{exe}
% 
\begin{exe}
\ex
\label{lVp2}
\begin{xlist}
\ex
\emph{\textipa{P}ojis:a-} `four'
\ex \emph{mah:oka-} `five'
\ex \emph{na\textipa{P}a:\v{c}a-} `ten'
\end{xlist}
\end{exe}
% 
The morpheme complexes in (\ref{lVp}) conform to a template {\small CVCCV}
plus a suffix `l{\small V:}p', regardless of the form of the base, although
the \isi{segmental} substance is that of the base.  The vowel of the suffix is
type identical to the last vowel of the template. The third vowel of the
base is elided, since it does not fit into the template.\footnote{%
  Elisions (and insertions of ``default'' vowels and consonants) are a well-known
  feature of Mi\-wok morphology; cf.\ the references above. Ignoring this,
  \cite[468]{bir:kle:94}\INA{Bird, Steven}\INA{Klein, Ewan} 
%asserts 
  asserts that
  ``the phonology of a complex form can only be produced by either unifying
  or concatenating the phonologies of its parts.''%
}
\begin{figure}
  \avha{\textit{complexmorph}
    \\\textsc{p} \textsc{s-s} $\left\langle \avhr{\textit{short} \\\textsc{s} \textit{con}},
      \avhr{\textit{short} \\\textsc{s}
        \textit{vow}} ,
      \avhr{\textit{short} \\\textsc{s}
        \textit{con}}
      ,
      \avhr{\textit{short} \\\textsc{s} \textit{con}} ,
      \avhr{\textit{short} \\\textsc{s}
        \avmcirc{1}\textit{vow}} 
      \quad | \quad \avmbox{2} 
    \right\rangle$
    \\
    \\\textsc{h} \avhr{\textit{simplemorph}
      \\\textsc{p s-s} \avmbox{2}$\left\langle \avhr{\textit{short}
          \\\textsc{s} l} ,
        \avhr{\textit{long} \\\textsc{s}
          \avmcirc{1}} ,
        \avhr{\textit{short} \\\textsc{s} p}
      \right\rangle$
      }\\
    \\\textsc{n} \avhr{\textit{simplemorph}
      \\\textsc{p s-s} $\left\langle \avhr{\textit{short}\\
          \textsc{s} \textipa{m}},
        \avhr{\textit{short} \\\textsc{s}
          \textipa{a}} ,
        \avhr{\textit{long} \\\textsc{s}
          \textipa{h}} ,
        \avhr{\textit{short} \\\textsc{s}
          \textipa{o}} ,
        \avhr{\textit{short} \\\textsc{s}
          \textipa{k}} ,
        \avhr{\textit{short} \\\textsc{s}
          \textipa{a}} 
      \right\rangle$}}
\caption{\label{FM2} Morphophonological structure of \emph{mahko-lo:p-}}
\end{figure}

Some aspects of the intended analysis of~(\ref{lVp}b) can be seen
in Figure~\ref{FM2}.  
%(Here and in later figures, the attribute symbols
%\textsc{morph}, \textsc{mhead}, \textsc{mnonhead}, \textsc{phon(onology)}, \textsc{segmproper} are abbreviated \textsc{m}, \textsc{h}, \textsc{n}, \textsc{p}, \textsc{s},
%respectively.)
(\textsc{mhead}, \textsc{mnonhead} and \textsc{segmproper} are abbreviated
\textsc{h}, \textsc{n} and \textsc{s}.)
Like~\cite[224\hspace{1pt}\nolinebreak f.\@]{scob:97}\INA{Scobbie, James M.}, and 
%in difference to 
unlike
most work on nonconcatenative morphology, I see no reason to
posit special attributes (``tiers/""planes'') just for consonants or vowels.
The pair of encircled tags, `\avmcirc{1}', indicates sort identity; in the
present case, identity of the phonemic sort that \textipa{[o]} belongs to.
For the reasons noted in the previous section, I leave open how the suffix
(the \textsc{h} value) triggers the phonological form of its parent. We can
see that the \textsc{p s-string} value is the {\small CVCCV} template,
followed by the \textsc{h p s-string} value (tagged \avmbox{2}).  The \textit{consonant}\/ objects of the template are \textit{segmproper}\/ objects of
the \textsc{n p s-string} value, in left to right order; equally for the \textit{vowel}\/ objects.\footnote{\label{RSRL}%
	Here and throughout, the
  exposition of complex restrictions is highly informal.  Explicit
  formulations would rely on the restricted theory of quantification and
  relations (based on~\citealt{king:89}\INA{King, Paul John}) defined
  in \cite[22--35]{rich:97}\INA{Richter, Frank} and explored
  in \cite{ric:kin:97}.  See also \cite{rich:98a}.%
}\il{Miwok|)}






\section{Morphophonology: Russian obstruents}
\label{sectRuss}



\subsection{Background}
\label{sec:1.7.1}


For an illustration of classical morphophonological considerations, we turn
to the voicing alternation in Russian\il{Russian}\il{Russian|(} obstruents. This case
is famous since \cite[21--24]{hall:59}\INA{Halle, Morris} used it in an
argument against ``empiricist'' versions of phonology, effectively wrecking
their distinction between phonemes and morphophonemes.

The case has gained additional fame through two kinds of ill-under\-stood
complications. First, the labiodental fricatives seem Janus-faced in that
they partly pattern like ordinary obstruents, partly like sonorants
(\citealt{jako:56}\INA{Jakobson, Roman}). Second, and somewhat similarly, a
sonorant that immediately precedes an obstruent can, under certain
conditions, pattern like the obstruent it precedes
(\citealt{jako:78}\INA{Jakobson, Roman}). For attempts to capture the full
range of data, see~\cite{whee:88}\INA{Wheeler, Deirdre} and
\cite[{}\S{}2]{kipa:85}\INA{Kiparsky, Paul} (which both rely heavily on
``underspecification''), and references therein.

Exciting though these complications are, I will disregard them and largely
keep to the simplifying introductory description
in \cite[22]{hall:59}:\INA{Halle, Morris} ``voicing is distinctive for
all obstruents except /c/, /\v{c}/ and /x/, which do not possess voiced
cognates. These three obstruents are voiceless un\-less followed by a
voiced obstruent, in which case they are voiced. At the end of the word,
however, this is true of all Russian\il{Russian} obstruents: they are voiceless, unless
the following word begins with a voiced obstruent, in which case they are
voiced.''

\largerpage
We thus find the voicing alternation at the end of the words in (\ref{RB1})
(after Halle~ib.\@\nocite{hall:59}\INA{Halle, Morris}) just like those
in (\ref{RB2}):
\begin{exe}
\ex
\label{RB1}
\begin{xlist}
\ex
\emph{\v{z}e\v{c} li}
\textipa{[ZetSl\super{j}i]} `should one burn?'
%
\ex
\emph{\v{z}e\v{c} by}  \textipa{[ZedZb1]} `were one to burn'
\end{xlist}
\end{exe}
% 
\begin{exe}
\ex
\label{RB2}
\begin{xlist}
\ex
\emph{gorod Ufa} \textipa{[gOr@tufa]}  `(the) town Ufa'
%
\ex
\emph{gorod Baku} \textipa{[gOr@dbaku]} 
%
\ex
\emph{gorod} \textipa{[gOr@t]} (nom.\ sing.)
%
\ex
\emph{goroda} \textipa{[gOr@d@]} (gen.\ sing.)
\end{xlist}
\end{exe}
% 
The affricate in (\ref{RB1}a) as well as the alveolar plosive
in (\ref{RB2}a) are voiceless before the non-obstruent in the following
word, but are voiced before the voiced obstruent in (\ref{RB1}b),
(\ref{RB2}b).  When nothing follows, as in (\ref{RB2}c), obstruents are
voiceless. Before a vowel in the same word, obstruents can be either voiced
or voiceless, such as \textipa{[d]} in (\ref{RB2}d) and \textipa{[k]}
in (\ref{RB2}b); but \textipa{[dZ, dz, G]} do not occur in this environment.

These regularities need to be detailed further in three respects
\citep[63f.]{hall:59}\INA{Halle, Morris}. (i)~By the
``end of a word'' is meant the end of a phonological word (a domain of
accent rules). Most prepositions constitute a proper part of a phonological
word; hence final obstruents in them are not devoiced when followed by a
non-obstruent. (ii)~By a ``following word'' is meant a following
phonological word within the same ``phonemic phrase.''  There is
considerable freedom as to how many phonemic phrases any given utterance
may contain. (iii)~An ``obstruent cluster,'' i.e., a sequence of
consecutive obstruents, always conforms to the last obstruent in the
cluster with regard to voicing. This observation applies to obstruent
clusters within morphological signs and phonological words,\footnote{%
	In a study of obstruent voicing assimilation in prepositions,
  \cite{bur:rob:97}\INA{Burton, Martha W.}\INA{Robblee, Karen E.} has found
  overall acoustic evidence for regressive assimilation, but also subtle
  remnants of the underlying voicing properties. If these subtle effects
  prove to be reliable and significant, the question as to their
  articulatory cause arises. For present purposes I disregard them.%
}
and is generalized in \cite[64]{hall:59}\INA{Halle, Morris} (following Jakobson)
to clusters across boundaries of phonological words within a phonemic
phrase.

To formally account for the fact that the s-strings of unembedded signs are
articulated into phonological phrases, I hypothesize the feature
declaration~(\ref{pphrase}) and the restrictions~(\ref{pphrase2}) and~(\ref{pphrase3}):
\begin{exe}
\ex
\label{pphrase}
\begin{tabbing}
\hspace{1,5em}\=\hspace{1,5em}\=\hspace{1,5em}\=\hspace{1,5em}\=\hspace{6em}
\=\kill
\textit{unembedded-phrase}         \>\>\>\>\>\textsc{phonphrases} \textit{nelofnelofsegment}
\end{tabbing}
\ex
\label{pphrase2} In an \textit{unembedded-phrase} object, the
  \textsc{phon s-string} value is the concatenation of the members of the \textsc{phonphrases} value.
\ex
\label{pphrase3} In an \textit{unembedded-phrase}\/ object, the
  last segment of a phonological phrase (i.e., a member of the \textsc{phonphrases} value) is also the last segment of a phonological word
  (i.e., a member of the \textsc{phon hierarch phonwords} value).
\end{exe}


I thus assume that the end of a phonological phrase coincides with the end
of a phonological word. In the illustrative partial analyses below, I will
assume that in simple cases, a phrase's \textsc{pws} (i.e., \textsc{phonwords})
value is the concatenation of the \textsc{pws} values of its daughters and a
sign's s-string is a concatenation of the members of its \textsc{pws} value,
as in Figure~\ref{g.baku1} below.  (The latter condition fails to be
satisfied in cases of sandhi, as in Figures~\ref{g.baku2} and~\ref{zedzbi}.)  This is not the general rule in morphology, though, and
it is of course not true in the daughters of prepositional phrases in
Russian\il{Russian}.\footnote{% 
	I suggest that the lexical entries of most prepositions
  require the s-string to be a proper prefix of the (single) member of the
  \textsc{pws} value. In a PP headed by a preposition like that, the member of
  the preposition's \textsc{pws} value is the concatenation of its s-string
  and the first member of the complement daughter's \textsc{pws} value. Thus,
  the PP's \textsc{pws} value is the concatenation of the head daughter's and
  the complement daughter's \textsc{pws} values minus the latter's first
  member.  The same sort of analysis can be applied to the indefinite
  articles \textit{a}\/ and \textit{an}\/ in English\il{English} and also, probably, in
  morphology.%
}
Although these cases would deserve a systematic discussion,
I follow the literature in being inexplicit about non"=trivial details.
(Cf.\ \citealt{ink:zec:95}\INA{Inkelas, Sharon}\INA{Zec, Draga} for some
discussion.)

Among the phonemic sorts %that I posit 
for Russian\il{Russian} are the ones in (\ref{Rsort1}), with some of the attending
restrictions given in (\ref{Rsort2}). The \textit{affricate} subsorts \textit{sort}{\textunderscore}\textsc{c} and \textit{sort}{\textunderscore}\textsc{\v{c}} as well as the \textit{fricative}
subsort \textit{sort}{\textunderscore}\textsc{x} are un\-restricted as to voicing, so that,
e.g., both \textipa{[x]} and \textipa{[G]} are of sort \textit{sort}{\textunderscore}\textsc{x}.
\begin{exe}
\ex
\label{Rsort1} \textit{fricative}\/: \textit{sort}{\textunderscore}\textipa{s}, \textit{sort}{\textunderscore}\textipa{z},
  \textit{sort}{\textunderscore}\textipa{S}, \textit{sort}{\textunderscore}\textipa{Z},  \textit{sort}{\textunderscore}\textsc{x}, \ldots\\
  \textit{affricate}\/: \textit{sort}{\textunderscore}\textsc{c}, \textit{sort}{\textunderscore}\textsc{\v{c}}, \ldots
%
\ex
\label{Rsort2}
  \textit{sort}{\textunderscore}\textipa{s} $\rightarrow $ {\small
    \avhr{\textsc{voicing} \textit{voiceless} \\
      \textsc{constr ft site} \textit{alveolar}}} \\
  \textit{sort}{\textunderscore}\textipa{Z} $\rightarrow $ {\small
    \avhr{\textsc{voicing} \textit{voiced} \\
      \textsc{constr ft site} \textit{postalveolar}}}\\
  \textit{sort}{\textunderscore}\textsc{x} $\rightarrow $ 
  \textsc{constr ft site} \textit{velar} \\
  \textit{sort}{\textunderscore}\textsc{c} $\rightarrow $ 
  \textsc{constr ft site} \textit{alveolar} \\
  \textit{sort}{\textunderscore}\textsc{\v{c}} $\rightarrow $ 
  \textsc{constr ft site} \textit{postalveolar}
  
\end{exe}
% \bigskip  

%\noindent


\subsection{Ordinary obstruents}
\label{sec:1.7.2}

As a first step, we consider a set of highly simplified rules that are
close to Halle's~(\shcite{hall:59}) introductory description:
%
\begin{exe}
\ex
\label{R1} In a phonological phrase, the elements of an
  obstruent cluster C have the \textsc{voicing} value of C's last
  element.\footnote{%
  	The assumption that the \textsc{voicing} values are
%Requiring the \textsc{voicing} values to be 
   token identical is a speculation. They might just be type identical,
   with the empirical consequence for the velocity of articulator movement
   noted in our discussion of Sundanese\il{Sundanese} nasalization in \S\ref{sectPI}.%
}
\ex
\label{R2} In a phonological phrase, an obstruent at its end
  is voiceless.
%
\ex
\label{R3} In a phonological phrase, if an obstruent O that
  corresponds to an obstruent at the end of a phonological word is followed
  by a non-obstruent, O is voiceless.
\end{exe}
The ground rule for the relation of a parent's \textsc{phon} value to the \textsc{phon} values of its daughters is of course that the values of
corresponding paths in segments of parent and daughter are identical. To be
able to work proper\-ly, (\ref{R1})--(\ref{R3}) also presuppose that the
ground rule is not obeyed when path values cannot be identical due to the
demands imposed on the parent by a rule. Thus, the rules must be
complemented by appropriate specifications of delinking, a task not
undertaken here.
\begin{figure}
  \newsavebox{\tnhsix}
  \newsavebox{\tnhseven}
  \newsavebox{\tnheight}
  \savebox{\tnhsix}{\avhr{\textit{word}
      \\\textsc{p} \avhr{\textsc{s-s}
        \avmbox{22}$\left\langle 
          \avmbox{6}, \avmbox{7}, \avmbox{8}, \avmbox{9} \right\rangle$
        \\\textsc{ hierarch phonwords} $\left\langle
          \avmbox{22}  \right\rangle$}\\
      \\\textsc{m} \avhr{\textit{simplemorph}
        \\\textsc{p s-s}
        $\left\langle  \avmbox{6}\avhr{\textsc{s} \textit{sort}{\textunderscore}\textipa{b}},
          \avmbox{7}\avhr{\textsc{s} \textipa{a}},
          \avmbox{8}\avhr{\textsc{s} \textipa{k}},
          \avmbox{9}\avhr{\textsc{s} \textipa{u}}\right\rangle$}
      }}
  \savebox{\tnhseven}{\avhr{\textsc{s-s} \avmbox{20}$\left\langle 
        \avmbox{1}, \avmbox{2}, \avmbox{3}, \avmbox{4}, 
        \avmbox{5}\avhr{\textsc{s voi} \avmbox{10}} ,
        \avmbox{6}\avhr{\textsc{s voi} \avmbox{10}} ,
        \avmbox{7}, \avmbox{8}, \avmbox{9}  \right\rangle$
      \\\textsc{hierarch phonwords} $\left\langle
        \avmbox{21}, \avmbox{22}  \right\rangle$}}
  \savebox{\tnheight}{\avhr{\textit{word}
      \\\textsc{p} \avhr{\textsc{s-s}
        \avmbox{21}$\left\langle 
          \avmbox{1}, \avmbox{2}, \avmbox{3}, \avmbox{4}, \avmbox{5}
        \right\rangle$
        \\\textsc{hierarch phonwords} $\left\langle
          \avmbox{21}  \right\rangle$}\\
      \\\textsc{m} \avhr{\textit{simplemorph}
        \\\textsc{p s-s}
        $\left\langle  \avmbox{1}\avhr{\textsc{s} \textipa{g}},
          \avmbox{2}\avhr{\textsc{s} \textipa{O}},
          \avmbox{3}\avhr{\textsc{s} \textipa{r}},
          \avmbox{4}\avhr{\textsc{s} \textipa{O}},
          \avmbox{5}\avhr{\textsc{s} \textit{sort}{\textunderscore}\textipa{d}}\right\rangle$}}}
  \avha{\textit{unembedded-phrase}
    \\\textsc{p}  \usebox{\tnhseven}
    \\\textsc{phonphrases} $\left\langle \avmbox{20} \right\rangle$\\
    \\\textsc{dtrs} $\cdots$\avhr{$\cdots$\usebox{\tnhsix}\\
      \\$\cdots$\usebox{\tnheight}}}
  \caption{\label{g.baku1} Morphophonological structure of \emph{gorod Baku}
    (provisional)}
\end{figure}


Example~(\ref{RB2}b) is partially analysed in accordance
with~(\ref{R1})--(\ref{R3}) in Figure~\ref{g.baku1}. Vowel reduction is
disregarded. I assume that \emph{go\-rod}, \emph{Baku} and \emph{Ufa} are
simple morphemes and hence licensed as such by lexical entries, and
that~(\ref{RB2}a--c) can be performed as complete utterances.

In~Figure~\ref{g.baku1}, no rule requires that any path value in any segment
of a parent be different from a corresponding path value in a daughter. The
only relevant requirement is~(\ref{R1}): the \textsc{voi(cing)} values of the
obstruents at the phonological word boundary (tagged \avmbox{5} and \avmbox{6})
must be identical. This is unproblematic, as they happen to be of the same
sort anyway.

The rules (\ref{R1})--(\ref{R3}) can claim to be ``surface"=true.'' Although
this property is sometimes considered a merit in itself, (\ref{R3}) appears
doubtful. Why should it be that obstruents are voiceless at the end of
words whenever, of all segments, just a vowel or sonorant follows?
Devoicing is often seen in codas. This could in fact be the relevant factor
in (\ref{R2}); but in (\ref{R3}), there is no reason to expect an obstruent
to be in a coda just when followed by a vowel. Actually, in (\ref{RB2}a),
the devoiced obstruent is not in a coda but in a syllable onset, according
to~\cite[76]{weis:87}:\INA{Weisser, Franz}
\begin{exe}
\ex
\label{gor}
\textipa{[gO.r@.tu.fa]}
\end{exe}
%
(If this is correct, the domain of syllabification rules is the
phonological phrase, rather than some smaller unit.) In any case, the
conditioning factor for devoicing is clearly the end of the phonological
word, quite independently of whatever follows.  That is, if a grammar is
not just meant to reproduce the data in some fashion, but also to capture
``natural'' rules that constitute a speaker's knowledge, (\ref{R3}) must be
replaced by (\ref{R3'}). (Thanks to (\ref{pphrase3}), this rule subsumes
the effects of (\ref{R2}).)
%
\begin{exe}
\ex
\label{R3'} 
In a \emph{word} object, an obstruent at the end of a phonological word is
voiceless.
\end{exe}
%

Rule (\ref{R3'}) is of course no longer surface-true: an obstruent that is
voice\-less according to (\ref{R3'}) surfaces voiced just in case a voiced
obstruent follows in the phonological phrase.  This fact suggests that the
generalization expressed by (\ref{R1}) might be spurious and that it should
be replaced by the less general cluster rule~(\ref{R1'}) and a sandhi
rule (\ref{R4}):
%
\begin{exe}
\ex
\label{R1'} In a phonological word and in a \textit{morph}\/
  object, the elements of an obstruent cluster C have the \textsc{voicing}
  value of C's last element.
\end{exe}

\begin{figure}[t]
  \newsavebox{\tnhnine}
  \newsavebox{\tnhten}
  \newsavebox{\tnhtenhalf}
  \newsavebox{\tnheleven}
  \savebox{\tnhnine}{\avhr{\textsc{s-s} \avmbox{20}$\left\langle 
        \avmbox{1}, \avmbox{2}, \avmbox{3}, \avmbox{4}, 
        \avhr{\textsc{s} \avhr{\textit{sort}{\textunderscore}\textipa{d}
            \\\textsc{voi} \avmbox{10}\textit{voiced}}} ,
        \avmbox{6}\avhr{\textsc{s voi} \avmbox{10}} ,
        \avmbox{7}, \avmbox{8}, \avmbox{9}  \right\rangle$
      \\\textsc{hierarch phonwords} $\left\langle
        \avmbox{21}, \avmbox{22}  \right\rangle$}}
  \savebox{\tnhten}{\avhr{\textit{word}
      \\\textsc{p} \textsc{s-s}
      \avmbox{22}$\left\langle 
        \avmbox{6}, \avmbox{7}, \avmbox{8}, \avmbox{9} \right\rangle$
      \\
      \\\textsc{m} \avhr{\textit{simplemorph}
        \\\textsc{p s-s}
        $\left\langle  \avmbox{6}\avhr{\textsc{s} \textit{sort}{\textunderscore}\textipa{b}},
          \avmbox{7}\avhr{\textsc{s} \textipa{a}},
          \avmbox{8}\avhr{\textsc{s} \textipa{k}},
          \avmbox{9}\avhr{\textsc{s} \textipa{u}}\right\rangle$}
      }}
  \savebox{\tnhtenhalf}{\avhr{\textit{simplemorph}
        \\\textsc{p s-s}
        $\left\langle  \avmbox{1}\avhr{\textsc{s} \textipa{g}},
          \avmbox{2}\avhr{\textsc{s} \textipa{O}},
          \avmbox{3}\avhr{\textsc{s} \textipa{r}},
          \avmbox{4}\avhr{\textsc{s} \textipa{O}},
          \avhr{\textsc{s} \avhr{\textit{sort}{\textunderscore}\textipa{d} \\\textsc{voi} \textit{voiced}}}\right\rangle$}}
  \savebox{\tnheleven}{\avhr{\textit{word}
      \\\textsc{p} \textsc{s-s}
      \avmbox{21}$\left\langle 
        \avmbox{1}, \avmbox{2}, \avmbox{3}, \avmbox{4}, 
        \avhr{\textsc{s} \avhr{\textit{sort}{\textunderscore}\textipa{t}
            \\\textsc{voi} \textit{voiceless}}} \right\rangle$  \\
      \\\textsc{m} \usebox{\tnhtenhalf}}}
  \avha{\textit{unembedded-phrase}
    \\\textsc{p}  \usebox{\tnhnine}
    \\\textsc{phonphrases} $\left\langle \avmbox{20} \right\rangle$\\
    \\\textsc{dtrs} $\cdots$\avhr{$\cdots$\usebox{\tnhten}\\
      \\$\cdots$\usebox{\tnheleven}}}
  \caption{\label{g.baku2} Morphophonological structure of \emph{gorod
      Baku} (amended)}
\end{figure}

\begin{exe}
\ex
\label{R4} In a phonological phrase, if an obstruent O that
  corresponds to an obstruent at the end of a phonological word is followed
  by a voiced obstruent B, O has the \textsc{voicing} value of B.
\end{exe}

With (\ref{R3'})--(\ref{R4}) replacing (\ref{R1})--(\ref{R3}),
Figure~\ref{g.baku1} is no longer correct, as shown in Figure~\ref{g.baku2}
(indications of \textsc{pws} values are simplified).  By (\ref{R3'}), the word
\emph{gorod} (tagged \avmbox{21}) ends in a voiceless obstruent, and by (\ref{R4}),
the corresponding obstruent in the phonological phrase is voiced.  %\bigskip

%\noindent


\subsection{Obstruents with predictable voicing}
\label{sec:1.7.3}


The voice-unrestricted segment types \textsc{x}, \textsc{c} and \textsc{\v{c}} have
been problematic for certain theories of phonemics. Within phonological
words and \textit{morph}\/ objects, any such segment is voiceless if it
precedes a non-obstruent.  If it precedes an obstruent, it is subject
to~(\ref{R1'}).  If it is the last segment of a word, it is subject
to~(\ref{R3'}). The situation at the end of morphemes is less obvious, as
is shown in the partial analysis of~(\ref{RB1}b) in Figure~\ref{zedzbi}.
\begin{figure}[t]
  \newsavebox{\tnhtwelf}
  \newsavebox{\tnhthirteen}
  \newsavebox{\tnhthfo}
  \newsavebox{\tnhfourteen}
  \savebox{\tnhtwelf}{\avhr{\textsc{s} \avhr{\textit{sort}{\textunderscore}\textsc{\v{c}}
        \\\textsc{voi} \avmbox{10}\textit{voiced}}}}
  \savebox{\tnhthirteen}{\avhr{\textit{word}
      \\\textsc{p} \textsc{s-s}
      \avmbox{22}$\left\langle 
        \avmbox{4}, \avmbox{5} \right\rangle$ \\
      \\\textsc{m} \avhr{\textit{simplemorph}
        \\\textsc{p s-s}
        $\left\langle 
          \avmbox{4}\avhr{\textsc{s} \textit{sort}{\textunderscore}\textipa{b}},
          \avmbox{5}\avhr{\textsc{s} \textipa{1}}
        \right\rangle$}
      }}
  \savebox{\tnhthfo}{$\left\langle 
          \avmbox{1}\avhr{\textsc{s} \textit{sort}{\textunderscore}\textipa{Z}},
          \avmbox{2}\avhr{\textsc{s} \textipa{e}},
          \avmbox{3}\avhr{\textsc{s} \avhr{\textit{sort}{\textunderscore}\textsc{\v{c}}\\\textsc{voi} \avmbox{8}}}
        \right\rangle$}
  \savebox{\tnhfourteen}{\avhr{\textit{word}
      \\\textsc{p} \textsc{s-s}
      \avmbox{21}$\left\langle 
        \avmbox{1}, \avmbox{2}, \avmbox{6}\avhr{\textsc{s}
          \avhr{\textit{sort}{\textunderscore}\textsc{\v{c}}\\\textsc{voi} 
            \avmbox{9}\textit{voiceless}  }}        \right\rangle$
      \\
      \\\textsc{m} \avhr{\textit{simplemorph}
        \\\textsc{p
          s-s} 
        \usebox{\tnhthfo}}}}
  \avha{\textit{unembedded-phrase}
    \\\textsc{p}  \avhr{\textsc{s-s} \avmbox{20}$\left\langle \avmbox{1}, \avmbox{2},  
        \avmbox{7}\usebox{\tnhtwelf} ,
        \avmbox{4}\avhr{\textsc{s voi} \avmbox{10}},
        \avmbox{5} \right\rangle$
      \\\textsc{hierarch phonwords} $\left\langle \avmbox{21}, \avmbox{22}
      \right\rangle$}
    \\\textsc{phonphrases} $\left\langle \avmbox{20} \right\rangle$\\
    \\\textsc{dtrs} $\cdots$\avhr{$\cdots$\usebox{\tnhthirteen}\\
      \\$\cdots$\usebox{\tnhfourteen}}}
  \caption{\label{zedzbi} Morphophonological structure of \emph{\v{z}e\v{c} by}}
\end{figure}

In the phonological phrase, \textsc{\v{c}} (tagged \avmbox{7}) is voiced,
according to (\ref{R4}).  At the end of the phonological word, \textsc{\v{c}}
(tagged \avmbox{6}) is voiceless, according to (\ref{R3'}). Nothing prevents
\textsc{\v{c}} in the morpheme (tagged \avmbox{3}) to be voiceless too, so that
possibly, $\avmbox{8} = \avmbox{9}$; but nothing enforces this identity either. The
lexical entry for \emph{\v{z}e\v{c}} should certainly not fix the sort of \textsc{\v{c}}'s \textsc{voicing} value. Neither can the ground rule and the
delinking specifications attending the individual rules be balanced in such
a way that necessarily $\avmbox{8} = \avmbox{9}$. (The same problem arises with
obstruent clusters across morpheme boundaries within words.)

Naturally, no incorrect observational predictions ensue if this question is
just left open. But this (non-)decision forces upon us genuinely spurious
ambiguities: for each word that ends in a voice-unrestricted obstruent
sort, there are two candidate \textit{simplemorph} objects that differ in the
sort of the \textsc{voicing} value of the morpheme-final obstruent. If a
grammar is meant to capture 
the
linguistic knowledge of speakers, nothing can
justify an attribution to speakers of knowledge of such an ambiguity.
Therefore, the rule for objects belonging to a voice-unrestricted sort is
posited as in (\ref{R5}):
%
\begin{exe}
\ex
\label{R5} In a \textit{simplemorph} object, a \textsc{p s-string}
  value component of sort \textit{sort}{\textunderscore}\textsc{c}, \textit{sort}{\textunderscore}\textsc{\v{c}} or
  \textit{sort}{\textunderscore}\textsc{x} is voiceless if it does not precede an obstruent.
\end{exe}


Given that the voicing of an obstruent at the end of a morpheme must in
general be inferred on the evidence of the morpheme occurring before a
vowel (or a sonorant) in a phonological word, (\ref{R5}) is perfectly
natural. Thus, by (\ref{R5}), \textsc{\v{c}} in the morpheme
\emph{\v{z}e\v{c}} in Figure~\ref{zedzbi} is voiceless. By the ground rule,
then, $\avmbox{8} = \avmbox{9}$ and hence, $\avmbox{3} = \avmbox{6}$.  %\bigskip

%\noindent


%\addlines[-2]
\subsection{Concluding considerations}
\label{sec:1.7.4}

\begin{sloppypar}
According to the rule system (\ref{R3'})--(\ref{R5}), Russian\il{Russian} has a special
end"=of"=word rule (\ref{R3'}), which is nothing unusual, and a sandhi
rule (\ref{R4}) that is independent of the word"=internal cluster
rule (\ref{R1'}). On comparative grounds, both rules are far more natural
than (\ref{R3}). One can expect, then, that Slavic languages (whose
obstruents show similar behaviour, by and large) vary~(\ref{R4})
independently of (\ref{R1'}). As we have noted in \S\ref{sec:1.2.2}, this is
just what happens in southwestern variants of
Polish\il{Polish} (cf.\ \citealt{beth:84}\INA{Bethin, Christina Y.},
\citealt[34 and~54]{guss:92}\INA{Gussman, Edmund}, \citealt[72
and~82\hspace{1pt}f.\@]{ruba:96}\INA{Rubach, Jerzy}): there, (\ref{R1'})
and~(\ref{R3'}) are (basically) also in force, but in the counterpart
to~(\ref{R4}), `obstruent B' is to be replaced by `segment B'.
\end{sloppypar}

Moreover, it is possible that the sandhi rule~(\ref{R4}) is actually not
phonemic but phonetic in nature; cf.\ \cite{kipa:85}\INA{Kiparsky, Paul} for
references. This might mean that the \textsc{voicing achievement} values of
obstruents, rather than their \textsc{voicing} values, are critically affected
by sandhi. It might also mean that~(\ref{R4}) should actually regulate not
s-strings but p-strings. This is suggested by the observation
in \cite[415]{isac:54}\INA{Isacenko@Isa\v{c}enko, Alexander V.} (confirmed
in \citealt[64, note~15]{hall:59}\INA{Halle, Morris}) that in examples such
as \emph{re\v{z}' bulku} `cut the bread' the fricative typically sets in
voiceless.  This fact appears problematic upon (\ref{R1}) when the
fricative is lexically voiced, but may be easier to understand
upon~(\ref{R3'}) plus a modified version of (\ref{R4}). Conceivably,
though, it is a rather direct consequence of there being just one long
(hence, slow) voicing gesture throughout the obstruent cluster. To clarify
these questions, more detailed data are required than are available to me.
%\bigskip

%\noindent
\begin{sloppypar}
In the discussion of Russian\il{Russian} obstruents above, several noteworthy points
should have become apparent. First, important questions of empirical detail
notwithstanding, the requirements of lexical licensing and of rule
naturalness strongly motivate non-trivial forms of multistratality. Second,
although the relation of a parent's \textsc{phon} value to those of its
daughters (as envisaged in (\ref{R3'})--(\ref{R4}) in conjunction with the
ground rule and the delinking specifications) is intuitively transparent,
its logic is remarkably complex.  Specifying it in full detail is some work
even with the resources referred to in note~\ref{RSRL} above. Third, the
hierarchical structure imposed by \textsc{phonwords} and \textsc{phonphrases}
values is non-trivial just in case it is not homomorphic to
morpho-syntactic structure. But because of the lack of detailed research, we
fail to have a general well"=founded view of how to construe the relation
explicitly for non-trivial cases.\footnote{%
	To account for word order
  phenomena in Serbo-Croatian, \cite{penn:98}\INA{Penn, Gerald} proposes a
  structure for signs that differs significantly from my highly
  conservative proposals (which can, in principle, be combined with the
  approach to linearization put forward in \citealt{rich:97}\INA{Richter,
    Frank}\INA{Sailer, Manfred} or in \citealt{sail:97}). Considerable
  exploratory work will be needed to assess the merits of either
  proposal.%
}
\il{Russian|)}
\end{sloppypar}




\section{Summary of architecture}
\label{sectSum}
 
A natural language is conceived to be a set of totally well-typed token
linguistic objects.  Among them, objects of sort \textit{unembedded"=phrase}\/
are prominent in that they alone can be performed as utterances. They bear
a distinguished phonological attribute \textsc{utterance}. Its value's \textsc{s-string} 
%value is mapped to a 
%set of alternative
%possible physical events by 
value (its ``p-string'') is mapped to a u-equivalence (a maximal
set of stylistically alternative possible utterance events) by
an interpretation function ϕ.

ϕ is defined by a physical theory that has a universal core, but
allows certain details to be learned for individual languages.  For a
``grammar G of natural language L,'' L is (intended to be) an exhaustive
model of G\@.  Knowledge of L is thus partly embodied in G and partly, in
the theory defining ϕ.

An object's \textsc{utterance s-string} value is related, but not necessarily
identical, to its \textsc{phon s-string} value, providing room for phonetic
regularities that are not captured by the \textsc{phon s-string} value (or by
ϕ) alone. Components of \textsc{phon s-string} values need not be
physically interpreted, although they are required to be interpretable.

The \textsc{phon} value of a sign is related to the \textsc{phon} values of its
component signs.  Words embed (morphological) signs, which may again embed
signs (recursively). The relation of a sign's \textsc{phon} value to those of
its component signs thus provides room (hypothesized to be both necessary
and sufficient) for ``postlexical'' as well as ``lexical'' morphophonology.

``Long distance'' identities in strings of segments are necessarily type
(as opposed to token) identities. Identities in adjacent segments can (but
need not) be token identities, since segments allow for massive gestural
overlap. Notions from classical phonemics, which capture constant coupling
of different gestural objects in segments, can be exploited (and explored)
with the help of phonemic sorts 
%(subsorts of \textit{segmproper}\/).
(partitionings of major class and manner sorts).

\INS{phonology|)}

\printbibliography[heading=subbibliography,notkeyword=this]
\refstepcounter{mylastpagecount}\label{chap-phonology-end}
\end{document}
