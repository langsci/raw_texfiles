\chapter{Clause-level syntax}\label{sec:11}

\is{clause|(}

This chapter provides an overview of the \isi{syntax} of Ulwa at the level of the single clause. The interaction of multiple clauses is the focus of \chapref{sec:12}.

\is{clause|)}

\section{Basic constituent order}\label{sec:11.1}

\is{basic constituent order|(}
\is{constituent order|(}
\is{word order|(}
\is{clause|(}

The minimal constituents of an \isi{intransitive} clause are a subject (S) and a verb (V); a \isi{transitive} clause consists of these two elements as well as an object (O). Stated in terms more agnostic with respect to notions of subjecthood and objecthood, an \isi{intransitive} clause consists of a single argument (S) and a verb (V), whereas a \isi{transitive} clause consists of a more \isi{agent}-like argument (A), a more \isi{patient}-like argument (P), and a verb (V). The basic ordering of these elements is given in \REF{ex:clause:1}.

\ea%1
    \label{ex:clause:1}
Basic \isi{constituent order}
\begin{tabbing}
{(Intransitive clauses:)} \= {(SOV)} \= {((APV))}\kill
{\isi{Intransitive} clauses:} \> {SV} \> { }\\
{\isi{Transitive} clauses:} \> {SOV} \> {(APV)}
\end{tabbing}
\z

Although various \isi{pragmatic} factors may affect the ordering (or overt expression) of elements in a clause, Ulwa nevertheless has a fairly rigid ordering of constituents. The order presented in \REF{ex:clause:1} is used for essentially all \isi{active-voice} clauses. In the \isi{intransitive} clauses presented in \REF{ex:clause:2} through \REF{ex:clause:6}, the subject is in \textbf{bold}. The verb is always the final element of the clause.

\ea%2
    \label{ex:clause:2}
            \textit{\textbf{Alum mï} sap.}\\
\gll \textbf{alum}  \textbf{mï}      sa-p\\
    child  \textsc{3sg.subj}  cry-\textsc{pfv}\\
\glt `The baby cried.’ [elicited]
\z

\ea%3
    \label{ex:clause:3}
            \textbf{\textit{Anmoka}} \textit{i.}\\
\gll    \textbf{anmoka}  i\\
    snake    go.\textsc{pfv}\\
\glt `The snake left.’ [elicited]
\z

\ea%4
    \label{ex:clause:4}
            \textbf{\textit{Ndï}} \textit{wowe.}\\
\gll    \textbf{ndï}  wow{}-e\\
    \textsc{3pl}  sleep-\textsc{ipfv}\\
\glt `They are sleeping.’ [elicited]
\z

\ea%5
    \label{ex:clause:5}
            \textit{\textbf{Jukan mï} mbïp.}\\
\gll    \textbf{Jukan}  \textbf{mï}      mbï-p\\
    [name]  \textsc{3sg.subj}  here-be\\
\glt `Jukan is here.’ [elicited]
\z

\ea%6
    \label{ex:clause:6}
            \textbf{\textit{Alum}} \textit{uleplïnda.}\\
\gll    \textbf{alum}  ulep-lï-nda\\
    child  jump-put-\textsc{irr}\\
\glt `The child will jump.’ [elicited]
\z

In the \isi{transitive} clauses given in \REF{ex:clause:7}, \REF{ex:clause:8}, and \REF{ex:clause:9}, the subject (or more \isi{agent}ive participant) is in \textbf{bold}, and the object (or more \isi{patient}ive participant) is \underline{underlined}. The verb is always the final element of the clause.\footnote{Note that \isi{object-marker} \isi{clitic}s have been underlined in examples \REF{ex:clause:8} and \REF{ex:clause:9}, although they \isi{clitic}ize to the following verb. See \sectref{sec:7.2} for more on \isi{object marker}s and \sectref{sec:9.2} for a discussion of their \isi{syntactic} place within the \isi{verb phrase}.}

\ea%7
    \label{ex:clause:7}
            \textit{\textbf{Itom mï} \underline{uta} walinda.}\\
\gll \textbf{itom}  \textbf{mï}      \underline{uta}    wali-nda\\
    father  \textsc{3sg.subj}  bird  hit-\textsc{irr}\\
\glt `Father will shoot a bird.’ [elicited]
\z

\ea%8
    \label{ex:clause:8}
            \textit{\textbf{Alimban mï} \underline{apa ma}yte.}\\
\gll    \textbf{Alimban}  \textbf{mï}      \underline{apa}    \underline{ma}=ita-e\\
    [name]    \textsc{3sg.subj}  house  \textsc{3sg.obj}=build-\textsc{ipfv}\\
\glt `Alimban is building the house.’ [elicited]
\z

\ea%9
    \label{ex:clause:9}
            \textit{\textbf{Apa mï} \underline{alum ma}sap.}\\
\gll    \textbf{apa}  \textbf{mï}      \underline{alum}  \underline{ma}=asa-p\\
    house  \textsc{3sg.subj}  child  3\textsc{sg.obj}=hit-\textsc{pfv}\\
\glt `The house killed the child.’ (e.g., by falling on him) [elicited]
\z

\is{argument alignment}
\is{alignment}

  As is suggested by example \REF{ex:clause:9}, in which the subject/\isi{agent} is \isi{inanimate} and the object/\isi{patient} is \isi{animate}, notions of \isi{agent}ivity or \isi{patient}ivity are not intrinsic to NPs based on their referents. That is, principles such as the \isi{animacy hierarchy} \label{ref:RNDui1Xjf4uVA}\citep{Silverstein1976} play no role in determining \isi{constituent order} (or \isi{core argument alignment}, \sectref{sec:11.2}) in Ulwa.

  This S(O)V order is rigid for most clause types. However, it is possible for the subject (S) constituent to be omitted when its referent is clear from context. Ulwa may thus be considered a \isi{pro-drop} language. However, while subjects may be omitted, it is not common for objects to be omitted: a \isi{transitive} clause generally must, at the very least, contain an \isi{object marker} \isi{proclitic}. Sentences \REF{ex:clause:10} through \REF{ex:clause:20} illustrate sentences with unexpressed subjects.

\ea%10
    \label{ex:clause:10}
          \textit{Wop.}\\
\gll    wo-p\\
    sleep-\textsc{pfv}\\
\glt `[He] slept.’ [ulwa001\_10:46]
\z

\ea%11
    \label{ex:clause:11}
          \textit{Yawe mankap.}\\
\gll    ya-we      ma=nïkï-p\\
    coconut-sago  \textsc{3sg.obj}=dig{}-\textsc{pfv}\\
\glt `[He] made the coconut-sago pancake.’ [ulwa001\_10:44]
\z

\ea%12
    \label{ex:clause:12}
          \textit{Lïmndï wapa ngalala.}\\
\gll    lïmndï  wapa  ngala=ala\\
    eye    leaf  \textsc{pl.prox=}see\\
\glt `[He] saw these leaves.’ [ulwa001\_10:16]
\z

\ea%13
    \label{ex:clause:13}
          \textit{Wombasame maya iye.}\\
\gll    Wombasame  ma=iya      i-e\\
    [name]      \textsc{3sg.obj}=toward  go.\textsc{pfv-dep}\\
\glt `[She] went to Wombasame.’ [ulwa001\_15:30]
\z

\ea%14
    \label{ex:clause:14}
          \textit{Wolka manji numan andanap i.}\\
\gll    wolka  ma-nji      numan    anda=nap    i\\
    again  3\textsc{sg.obj-poss}  husband  \textsc{sg.dist}=for  go.\textsc{pfv}\\
\glt `[She] went [home] again for the sake of her husband.’ [ulwa032\_11:03]
\z

\ea%15
    \label{ex:clause:15}
          \textit{Wambana ndïmokop.}\\
\gll    wambana  ndï=moko-p\\
    fish    \textsc{3pl}=take-\textsc{pfv}\\
\glt `[They] caught fish.’ [ulwa001\_07:01]
\z

\ea%16
    \label{ex:clause:16}
          \textit{Mol mbiye.}\\
\gll    ma=ul      mbï-i-e\\
    3\textsc{sg.obj}=with  here-go.\textsc{pfv{}-dep}\\
\glt `[They] came with her.’ [ulwa014\_16:01]
\z

\ea%17
    \label{ex:clause:17}
          \textit{Manji inom ambi manji wandam may.}\\
\gll    ma-nji      inom  ambi  ma-nji      wandam     ma=i\\
    3\textsc{sg.obj-poss}  mother  big    3\textsc{sg.obj-poss}  jungle  3\textsc{sg.obj}=go.\textsc{pfv}\\
\glt `[We] went to her aunt’s garden.’ [ulwa032\_21:50]
\z

\ea%18
    \label{ex:clause:18}
          \textit{We ndït akïnakape.}\\
\gll    we    ndï=tï    akïnaka=p-e\\
    sago  3\textsc{pl}=take  young=\textsc{cop{}-dep}\\
\glt `[We] took the sago starch when [we] were young.’ [ulwa014\_00:13]
\z

\ea%19
    \label{ex:clause:19}
          \textit{Manji alum mat inde.}\\
\gll    ma-nji      alum  ma=tï      inda-e\\
    3\textsc{sg.obj-poss}  child  3\textsc{sg.obj}=take  walk-\textsc{ipfv}\\
\glt `[I] carried her child around.’ [ulwa032\_16:32]
\z

\ea%20
    \label{ex:clause:20}
          \textit{Ko angwena man inim atïna ne?}\\
\gll    ko  angwena  ma=n      inim  atï-na  na-i\\
    just  why    3\textsc{sg.obj=obl}  water  hit-\textsc{irr}  \textsc{detr-}go.\textsc{pfv}\\
\glt `Why did [you] just go to throw it in the water?’ [ulwa032\_28:44]
\z

Sometimes, in \isi{transitive} clauses, emphasis can be placed on an object by fronting it to the beginning of the clause. Even in such instances, however, the order of the clause (following this arguably preclausal element) is usually still SOV, since the referent of the fronted object invariably appears again in the clause, marked by an \isi{agreement} marker immediately preceding the verb, as in examples \REF{ex:clause:21} through \REF{ex:clause:24}.

\ea%21
    \label{ex:clause:21}
          \textit{\textbf{Nïnji alum ndï} nï Wopata ndape \textbf{ndinap}}.\\
\gll \textbf{nï-nji}    \textbf{alum}  \textbf{ndï}  nï    Wopata  anda=p-e   \textbf{ndï=}ina-p\\
    1\textsc{sg-poss}  child  \textsc{3pl}  \textsc{1sg}  [place]    \textsc{sg.dist}=be\textsc{{}-dep}    3\textsc{pl}=get-\textsc{pfv}\\
\glt `My children -- I had them when I was there at Wopata.’ [ulwa014\_02:23]
\z

\ea%22
    \label{ex:clause:22}
          \textit{\textbf{Talamba ulum ala} ndï \textbf{ndïnap} amblawale.}\\
\gll    \textbf{Talamba}  \textbf{ulum}  \textbf{ala}      ndï  \textbf{ndï}=nap  ambla=wali{}-e\\
    [place]    palm  \textsc{pl.dist}  \textsc{3pl}  \textsc{3pl=}for  \textsc{pl.refl}=hit-\textsc{ipfv}\\
\glt `Those palms at Talamba -- they were fighting on account of them.’ [ulwa034\_00:03]
\z

\ea%23
    \label{ex:clause:23}
          \textit{\textbf{Nïpïl ala} ala \textbf{ndïwale}}.\\
\gll \textbf{nïpïl}  \textbf{ala}      ala      \textbf{ndï}=wali{}-e\\
    vine  \textsc{pl.dist}  \textsc{pl.dist}  \textsc{3pl=}hit-\textsc{ipfv}\\
\glt `Those vines -- people used to beat them.’ [ulwa036\_01:23]
\z

\ea%24
    \label{ex:clause:24}
          \textit{\textbf{Nïnji yenat ngala} nï nan \textbf{ndït}: …}\\
\gll    \textbf{nï-nji}    \textbf{yenat}    \textbf{ngala}    nï    na=n    \textbf{ndï}=ta\\
    1\textsc{sg-poss}  daughter  \textsc{pl.prox}  \textsc{1sg}  talk=\textsc{obl}  \textsc{3pl}=say\\
\glt `My daughters -- I told them: …’ [ulwa037\_40:26]
\z


\is{clause|)}
\is{word order|)}
\is{constituent order|)}
\is{basic constituent order|)}

\is{wh-movement}

  Thus Ulwa maintains a fairly rigid SOV order. This rigidity is not surprising, considering the absence of verbal subject \isi{agreement}, \isi{core argument} \isi{case morphology}, and other clues as to the grammatical relations of NPs -- that is, whether they are subjects or objects. Thus, almost every \isi{indicative} \isi{main clause} with overtly expressed NPs follows this pattern, as do other clause types, such as \isi{interrogative} sentences\footnote{In other words, there is no \textit{wh}-movement for \isi{content question}s or \isi{inverted word order} for \isi{polar question}s.} (\sectref{sec:13.1}) and \isi{imperative} sentences\footnote{The subject need not be expressed in second \isi{person} \isi{imperative}s, but -- when present -- it always precedes the object and verb} (\sectref{sec:13.2}). The most notable divergences from this pattern occur in \isi{passive} constructions. In \isi{passive} constructions, the \isi{basic constituent order} is VS (see \sectref{sec:13.7}).

\section{Core argument alignment}\label{sec:11.2}

\is{alignment|(}
\is{argument alignment|(}
\is{core argument alignment|(}
\is{core argument|(}
\is{core argument|(}
\is{clause|(}

The three basic core arguments of all clause types may be considered to be S, A, and P \REF{ex:clause:24a}.\footnote{P is also sometimes identified as O in the literature.}

\ea%24a
    \label{ex:clause:24a}
The three basic core arguments
\begin{tabbing}
{(A:)} \= {(the more patient-like argument of a \isi{transitive} clause)}\kill
{S:} \> {the single argument of an \isi{intransitive} clause}\\
{A:} \> {the more \isi{agent}-like argument of a \isi{transitive} clause}\\
{P:} \> {the more \isi{patient}-like argument of a \isi{transitive} clause}
\end{tabbing}
\z

\is{accusative alignment}

In Ulwa, the S and A arguments pattern alike in every way -- \isi{syntactic}ally, \isi{morphological}ly, and \isi{phonological}ly. Moreover, \isi{TAM} distinctions have no effect on the marking strategies of core arguments. Nor are arguments marked differently based on different verb classes. Ulwa may thus be considered to exhibit \isi{nominative-accusative alignment}. It is therefore convenient and unproblematic to use terms like “subject” and “object” to refer to various NPs in Ulwa.

  S and A occur in the same position in the clause (namely, clause-initially), whereas P occurs after S and before the verb. Since there is no \is{core argument} core-argument \isi{case morphology} in Ulwa, it is generally fruitless to talk about “\isi{nominative}” and “\isi{accusative}” or “\isi{ergative}” and “\isi{absolutive}” NPs in Ulwa (at least in terms of \isi{morphological} marking). There is, however, one important distinction found in 3\textsc{sg} \isi{pronoun}s and \isi{determiner}s (i.e., \isi{subject marker}s and \isi{object marker}s). Whereas the 3\textsc{sg} subject form is /mï/, the 3\textsc{sg} object form is /ma=/ \REF{ex:clause:24b}. The fact that third-\isi{singular} S and A NPs are both marked with \textit{mï} ‘3\textsc{sg.subj}’, whereas third-\isi{singular} P NPs are marked with \textit{ma=} ‘3\textsc{sg.obj}’ is further indication of \isi{accusative alignment}.

  \ea%24b
    \label{ex:clause:24b}
The form of 3\textsc{sg} \isi{pronoun}s/\isi{determiner}s
 \begin{tabbing}
{(A:)} \= {(\textit{mï})}\kill
{S:} \> {\textit{mï}}\\
{A:} \> {\textit{mï}}\\
{P:} \> {\textit{ma=}}
\end{tabbing}
\z

  Since this \isi{nominative}/\isi{accusative} \isi{morphological} contrast is only apparent with 3\textsc{sg} forms (whether \isi{pronoun}s or NPs marked with these postnominal \isi{determiner}s), we can say that Ulwa exhibits \isi{neutral alignment} through most of its argument flagging: 1st and 2nd \isi{person} \isi{pronoun}s, \isi{non-singular} 3rd \isi{person} \isi{pronoun}s, and \isi{non-singular} NPs do not exhibit \isi{morphological} differences based on their roles as S, A, or P arguments.

  Finally, there is no evidence of Ulwa exhibiting \isi{syntactic ergativity} (\citealt[62--63]{Dixon1979}). Thus, for example, in \isi{coordinate} constructions (\sectref{sec:12.1}), coreference is possible between S and A but not between S and P. In \REF{ex:clause:25}, the omitted S argument of the second clause must be understood to refer to the stated A argument (\textit{yana} ‘woman’) of the first clause.

\ea%25
    \label{ex:clause:25}
          \textit{Yana mï yata masap i.}\\
\gll    yana  mï      yata  ma=asa-p      i\\
    woman  \textsc{3sg.subj}  man  \textsc{3sg.obj=}hit-\textsc{pfv}  go.\textsc{pfv}\\
\glt `The woman hit the man [and] [the woman/*the man] left.’ [elicited]
\z

Similarly, in sentence \REF{ex:clause:26}, the A argument Kolpe must be understood to be the omitted S argument of the second clause, and it would be impossible for the P argument \textit{mana} ‘spear’ to be understood as such.

\ea%26
    \label{ex:clause:26}
          \textit{Kolpe mana motoplïp liyu.}\\
\gll    Kolpe  mana  ma=top-lï-p        li-u\\
    [name]  spear  \textsc{3sg.obj=}throw-put-\textsc{pfv}  down-put\\
\glt `Kolpe threw the spear [but] [Kolpe/*the spear] fell.’ [elicited]
\z

There is also no indication of \isi{split-intransitivity} or related \isi{alignment} types in the language (i.e., there is no active-stative/semantic/\isi{fluid alignment} in Ulwa), nor is there any sign of \isi{direct-inverse alignment} based on \is{animacy hierarchy} animacy or any other \isi{hierarchy}. That is, all types of S arguments pattern more closely with A arguments than with P arguments, regardless of \isi{semantics} or other criteria. Thus, S arguments of clauses are alike both \isi{syntactic}ally and \isi{morphological}ly, irrespective of whether they are more \isi{agent}ive (i.e., \isi{unergative}) \REF{ex:clause:27} or more \isi{patient}ive (i.e., \isi{unaccusative}) \REF{ex:clause:28}.

\is{active-stative alignment}
\is{semantic alignment}

\ea%27
    \label{ex:clause:27}
          \textit{Alum mï uleplïp.}\\
\gll    alum  mï      ulep-lï-p\\
    child  3\textsc{sg.subj}  jump-put-\textsc{pfv}\\
\glt `The child jumped.’ [elicited]
\z

\ea%28
    \label{ex:clause:28}
          \textit{Alum mï liyu.}\\
\gll    alum  mï      li-u\\
    child  \textsc{3sg.subj}  down-put\\
\glt `The child fell.’ [elicited]
\z

This universal treatment of S arguments holds for all NPs, whether full NPs, as in \REF{ex:clause:27} and \REF{ex:clause:28}, or pronominal NPs, as in \REF{ex:clause:29} and \REF{ex:clause:30}.

\ea%29
    \label{ex:clause:29}
          \textit{Nï amun natan.}\\
\gll    nï    amun  na-ta-n\\
    1\textsc{sg}  now  \textsc{detr-}say-\textsc{ipfv}\\
\glt `I am speaking now.’ [elicited]
\z

\ea%30
    \label{ex:clause:30}
          \textit{Nï amun kïkalwana.}\\
\gll    nï    amun  kïkal-wana\\
    1\textsc{sg}  now  ear-feel\\
\glt `I am listening now.’ [elicited]
\z

Finally, S and A arguments are also alike in that both S and A arguments can be \isi{relativize}d, whereas P arguments cannot be \isi{relativize}d (\sectref{sec:12.3}). Neither S nor A arguments can be passivized, whereas P arguments can be passivized (\sectref{sec:13.7}).

\is{clause|)}
\is{core argument|)}
\is{core argument|)}
\is{core argument alignment|)}
\is{argument alignment|)}
\is{alignment|)}

\section{Ditransitive alignment}\label{sec:11.3}

\is{ditransitive alignment|(}
\is{ditransitive|(}
\is{argument alignment|(}
\is{alignment|(}
\is{clause|(}

As well as considering the \isi{morphosyntactic} patterning of S, A, and P arguments, some typologists analyze the relationships among arguments in \isi{ditransitive} constructions. Typological endeavors such as \citet{MalchukovEtAl2010} have largely focused on \isi{dative construction}s -- that is, constructions in which something is given from one participant to another. In some languages, these constructions make use of \isi{ditransitive} verbs, which take three arguments: A, R, and T \REF{ex:clause:30a}. The question of interest is whether the P argument of a \isi{monotransitive} verb patterns more like the R argument or the T argument of a \isi{ditransitive} verb (it is not ever known to pattern like the A argument).

\ea%30a
    \label{ex:clause:30a}
The three basic arguments associated with ‘giving’ events
\begin{tabbing}
{(A:)} \= {(\isi{recipient} (the receiver))}\kill
{A:} \> {\isi{agent} (the \isi{giver})}\\
{R:} \> {\isi{recipient} (the receiver)}\\
{T:} \> {\isi{theme} (the \isi{gift})}
\end{tabbing}
\z

  In Ulwa, however, there are no \isi{ditransitive} verbs. In short, there is no word ‘give’ in the sense of \ili{English} \textit{give}, which may, in some uses, be considered \isi{ditransitive} (as in sentences such as \textit{John gave Mary a rose}). To express ‘giving’ events in Ulwa, two verbs are needed: one verb generally has the meaning ‘take’, and has as its object an NP with a \isi{theme} role (the ‘\isi{gift}’); the other verb is usually \textit{na-} ‘give’,\footnote{It should be noted that the fact that the (\isi{monotransitive}) verb glossed as ‘give’ has as its (sole) object a \isi{recipient} does not imply any sort of \isi{ditransitive alignment} between R and P arguments. The verb \textit{na-} ‘give’, despite being glossed for convenience as ‘give’, is not equivalent to the \ili{English} word \textit{give}. There is, however, unfortunately, no basic \isi{monotransitive} \ili{English} word with which to gloss this \isi{monotransitive} Ulwa word, which means something more like ‘endow’ (although even this \ili{English} gloss is not a very good match, since it can have as its object NP either a \isi{recipient} or a \isi{theme}).} which has as its object an NP with a \isi{recipient} role (the receiver). Sentence \REF{ex:clause:30a} refers to a ‘giving’ event.

 \ea%30b
    \label{ex:clause:30b}
          \textit{Yata mï awal ya \textbf{mat} yananu \textbf{manana}.}\\
\gll    yata mï awal ya ma=\textbf{tï} yananu ma=\textbf{na}-na\\
   man \textsc{3sg.subj} yesterday coconut \textsc{3sg.obj}=take woman \textsc{3sg.obj}=give-\textsc{pfv}\\
\glt `Yesterday the man gave the woman a coconut.’ [elicited]
\z

  Given the nature of real-world scenarios involved in the act of giving, it is common for sentences referring to ‘giving’ events to include three participants: \isi{giver} (\isi{agent}), \isi{recipient} (\isi{benefactive}), and \isi{gift} (\isi{theme}). As such, these three participants are often all expressed in Ulwa ‘giving’ constructions, generally through the use of at least two verbs. It is, however, possible for the verb \textit{na-} ‘give’ to occur without any other verb encoding the \isi{theme} argument; in such instances, the only two roles expressed (as determined by the verb’s argument structure) are the \isi{giver} (the grammatical subject) and the \isi{recipient} (the grammatical object), as in \REF{ex:clause:31} and \REF{ex:clause:32}.

\ea%31
    \label{ex:clause:31}
          \textbf{\textit{Manata}} \textit{we mï man ulum ndïnalin.}\\
\gll    \textbf{ma=na}{}-ta        we    mï      ma=n      ulum     ndï=n    ali-n[da]\\
    3\textsc{sg.obj}=give-\textsc{cond}  then  \textsc{3sg.subj}  \textsc{3sg.obj=obl}  palm    3\textsc{pl=obl}  scrape-\textsc{irr}\\
\glt `If [they] give her [it], then she will scrape sago palms with it.’ [ulwa022\_00:18]
\z

\ea%32
    \label{ex:clause:32}
          \textbf{\textit{Ndïnane}} \textit{mane ndï ndame.}\\
\gll    \textbf{ndï=na}{}-n-e    ma-n-e      ndï  ndï=ama-e\\
    3\textsc{pl}=give-\textsc{pfv-dep}  go-\textsc{ipfv-dep}  \textsc{3pl}  \textsc{3pl=}eat-\textsc{ipfv}\\
\glt `Going and giving them, they would eat them.’ [ulwa018\_03:20]
\z

When the \isi{theme} (\isi{gift}) is also overtly expressed, it is necessary to use another verb. The verb that is most commonly used along with \textit{na-} ‘give’ in these constructions is the verb \textit{tï}- ‘take’ (see \sectref{sec:4.3}). This first verb, \textit{tï}- ‘take’, has as its object the \isi{theme} (that which is given), whereas the second verb, \textit{na-} ‘give’, has as its object the beneficiary (to whom something is given), as in examples \REF{ex:clause:33} through \REF{ex:clause:40}.

\ea%33
    \label{ex:clause:33}
          \textit{Alma mï lamndu} \textbf{\textit{matï}} \textit{Kongos} \textbf{\textit{manan}}.\\
\gll Alma  mï      lamndu  ma=\textbf{tï}      Kongos     ma=\textbf{na}{}-n\\
    [name]  3\textsc{sg.subj}  pig      3\textsc{sg.obj}=take  [name]    3\textsc{sg.obj}=give-\textsc{pfv}\\
\glt `Alma gave a pig to Kongos.’ (Literally ‘Alma took a pig; [Alma] gave Kongos.’) [elicited]
\z

\ea%34
    \label{ex:clause:34}
          \textit{Ndït wa ne} \textbf{\textit{ndït}} \textit{nïnji inom} \textbf{\textit{manana}}.\\
\gll ndï=tï    wa    na-i      ndï=\textbf{tï}    nï-nji     inom     ma=\textbf{na}{}-na\\
    3\textsc{pl}=take  village  \textsc{detr-}go.\textsc{pfv}  \textsc{3pl}=take  1\textsc{sg-poss}  mother    3\textsc{sg.obj}=give-\textsc{pfv}\\
\glt `[He] brought them home and gave them to my mother.’ (Literally ‘took them; gave my mother’) [ulwa013\_01:22]
\z

\ea%35
    \label{ex:clause:35}
          \textit{Ngan tana} \textbf{\textit{mat}} \textbf{\textit{manan}}.\\
\gll ngan    tana  ma=\textbf{tï}      ma=\textbf{na}{}-n\\
    1\textsc{du.excl}  axe    \textsc{3sg.obj}=take  3\textsc{sg.obj}=give-\textsc{pfv}\\
\glt `We gave him the axe.’ (Literally ‘We took the axe; [we] gave him.’) [ulwa014†]
\z

\newpage

\ea%36
    \label{ex:clause:36}
          \textit{Wawana mu} \textbf{\textit{kot}} \textbf{\textit{manane}}.\\
\gll wawana  mu    ko=tï    ma=na-n-e\\
    plant.species  fruit  \textsc{indf}=take  \textsc{3sg.obj}=give-\textsc{pfv-dep}\\
\glt `[They] gave him a fruit.’ [ulwa020\_02:01]
\z

\ea%37
    \label{ex:clause:37}
          \textit{Imba pïta wondi} \textbf{\textit{andat}} \textbf{\textit{unananda}}.\\
\gll imba  p-ta    wondi    anda=\textbf{tï}    unan=\textbf{na}{}-nda\\
    night  be-\textsc{cond}  bandicoot  \textsc{sg.dist}=take  1\textsc{pl.incl}=give-\textsc{irr}\\
\glt `When night comes, [he] will give us that bandicoot.’ [ulwa029\_05:31]
\z

\ea%38
    \label{ex:clause:38}
          \textit{An ango \textbf{kumat unanda}!}\\
\gll    an      ango  kuma=\textbf{tï}  u=\textbf{na}{}-nda\\
    1\textsc{pl.excl}  \textsc{neg}  some=take  \textsc{2sg}=give-\textsc{irr}\\
\glt `We won’t give you any!’ [ulwa032\_56:30]
\z

\ea%39
    \label{ex:clause:39}
          \textit{Mu \textbf{kumatï nïnan}!}\\
\gll    mu    kuma=\textbf{tï}  nï=\textbf{na}{}-n\\
    seed  some=take  1\textsc{sg}=give-\textsc{imp}\\
\glt `Give some seeds to me!’ [ulwa037\_55:02]
\z

\ea%40
    \label{ex:clause:40}
          \textit{Ndï yena} \textbf{\textit{ndït}} \textbf{\textit{ndïnane}} \textit{ndï ndul wop.}\\
\gll    ndï  yena  ndï=\textbf{tï}    ndï=\textbf{na}{}-n-e    ndï  ndï=ul    wo-p\\
    3\textsc{pl}  woman  \textsc{3pl}=take  \textsc{3pl}=give-\textsc{pfv-dep}  \textsc{3pl}  3\textsc{pl}=with  sleep-\textsc{pfv}\\
\glt `They gave the women to them and they slept with them.’ [ulwa002\_06:08]
\z

Since the verb \textit{tï-} ‘take’ is often \isi{defective}, as in most of examples \REF{ex:clause:33} through \REF{ex:clause:40}, it looks very much like these are \isi{separable verb} constructions (\sectref{sec:9.2.1}). However, given the verbal nature of \textit{tï-} ‘take’, these ‘giving’ constructions can instead be described as a modest form of \isi{serial verb construction}s.\footnote{Although not quite fitting some stricter criteria for \isi{serial verb construction}s \citep[8]{Aikhenvald2006}, since (as a \isi{defective verb}) \textit{tï-} ‘take’ does not match \textit{na-} ‘give’ in its \isi{TAM} marking, these Ulwa ‘giving’ constructions qualify as such under definitions such as \citegen[296]{Haspelmath2016}: “a \isi{monoclausal construction} consisting of multiple independent verbs with no element linking them and with no \isi{predicate}-argument relation between the verbs”. That said, when it lacks \isi{TAM} marking, \textit{tï-} ‘take’ is not clearly finite.} Even if we are to consider these ‘giving’ constructions to be instances of \isi{serial verb construction}s in Ulwa, then verb serialization is certainly not a productive \isi{syntactic} construction -- if it indeed exists in the language, then it is restricted to encoding what might be considered “\isi{ditransitive} events”.

  Furthermore, there are some instances in which it seems best to analyze Ulwa ‘giving’ constructions as consisting of two separate clauses. When the first verb \textit{tï-} ‘take’ is marked for \isi{TAM}, it must also receive the \isi{dependent marker} \textit{-e} ‘\textsc{dep}’ (\sectref{sec:12.2.1}), proving, as it were, that this verb belongs to a separate clause. This may be seen in sentence \REF{ex:clause:41}.

\ea%41
    \label{ex:clause:41}
          \textit{Uma} \textbf{\textit{ndïtïne}} \textit{Wombasame} \textbf{\textit{manane}} \textit{mï ndïn ne.}\\
\gll    uma  ndï=\textbf{tï-n}{}-e      Wombasame  ma=\textbf{na-n}{}-e mï      ndï=n    ni{}-e\\
    bone  3\textsc{pl}=take-\textsc{pfv-dep}  [name]      \textsc{3sg.obj}=give-\textsc{pfv-dep}    3\textsc{sg.subj}  3\textsc{pl=obl}  act-\textsc{ipfv}\\
\glt `[They] gave the bones to Wombasame, and he began playing with them.’ [ulwa001\_00:47]
\z

\is{clause|)}
\is{alignment|)}
\is{argument alignment|)}
\is{ditransitive|)}
\is{ditransitive alignment|)}

\is{ditransitive alignment|(}
\is{ditransitive|(}
\is{argument alignment|(}
\is{alignment|(}
\is{clause|(}

Sentence \REF{ex:clause:41} may be rendered more literally as: ‘After [they] took the bones, and after [they] gave Wombasame, he was acting with them.’ Finally, it should be said that the instances in which \textit{tï-} ‘take’ does not receive any \isi{TAM} marking could be considered a very modest form of \isi{clause chaining}, although, if so, then there would be only one \isi{medial clause} involved (see \sectref{sec:12.4} for more on possible \isi{clause-chaining} structures in Ulwa). Furthermore, the fact the \textit{tï-} ‘take’ is so often \isi{defective} even in monoverbal, \isi{monoclausal sentence}s makes its putative dependence on the “\isi{final verb}” less diagnostic.

  It is possible to form other (multi-verb) ‘giving’ constructions in Ulwa with other (\isi{inflect}ed) verbs that mean ‘take’. In sentences \REF{ex:clause:42} and \REF{ex:clause:43}, the verb \textit{moko-} ‘take’, which often has the sense ‘take one by one’, is used along with the verb \textit{na-} ‘give’.

  \ea%42
    \label{ex:clause:42}
          \textit{Mï ani} \textbf{\textit{ndïmokop}} \textbf{\textit{ndïnana}}.\\
\gll mï      ani    ndï=\textbf{moko}{}-p  ndï=\textbf{na}{}-na\\
    3\textsc{sg.subj}  bilum  3\textsc{pl}=take-\textsc{pfv}  3\textsc{pl}=give-\textsc{pfv}\\
\glt `He gave them the \textit{bilum} [= string bags] [one by one].’ [ulwa001\_04:59]
\z

\ea%43
    \label{ex:clause:43}
          \textit{Ndït wa i ndïweyawe} \textbf{\textit{ndïmoke}} \textit{lapun} \textbf{\textit{ndïnane}}.\\
\gll ndï=tï    wa    i    ndï=we-aw-e      ndï=\textbf{moko}{}-e     lapun  ndï=\textbf{na}{}-n-e\\
    3\textsc{pl}=take  village  go.\textsc{pfv}  \textsc{3pl}=cut-put.\textsc{ipfv-dep}  \textsc{3pl=}take\textsc{{}-dep}    old    \textsc{3pl}=give-\textsc{pfv-dep}\\
\glt `[They] used to bring them home, cut them, and give them out to the old people.’ (\textit{lapun} = TP) [ulwa029\_01:46]
\z
  
  Example \REF{ex:clause:42} could be analyzed as a \isi{serial verb construction} if it is assumed that the two verbs belong to a single clause. Indeed, they even match in terms of \isi{TAM} marking. In example \REF{ex:clause:43}, however, the verb \textit{moko-} ‘take’ is marked as belonging to a different clause. Moreover, it does not share \isi{TAM} marking with \textit{na-} ‘give’, suggesting that this is not a \isi{serial verb construction}.

The combination of \textit{moko-} ‘take’ and \textit{na-} ‘give’ is often used to describe the distribution or sharing of items, typically with a \isi{reflexive} object form preceding the verb \textit{na-} ‘give’, as in \REF{ex:clause:44}, \REF{ex:clause:45}, and \REF{ex:clause:46}.

\ea%44
    \label{ex:clause:44}
          \textit{Ndï ndït anmbïlïp \textbf{ndïmoke} \textbf{amblanane}.}\\
\gll ndï  ndï=tï    an-mbï-lï-p    ndï=moko-e     \textbf{ambla}=na-n-e\\
    3\textsc{pl}  \textsc{3pl}=take  out-here-put-\textsc{pfv}  3\textsc{pl}=take-\textsc{dep}    \textsc{pl.refl}=give-\textsc{pfv-dep}\\
\glt `They got them out and were sharing them among themselves.’ [ulwa014\_29:45]
\z

\ea%45
    \label{ex:clause:45}
          \textit{Ndï ilum \textbf{moke} \textbf{amblanane}.}\\
\gll ndï  ilum  moko-e  \textbf{ambla}=na-n-e\\
    \textsc{3pl}  little  take-\textsc{dep}  \textsc{pl.refl=}give-\textsc{pfv-dep}\\
\glt `They would share little [pieces] with each other.’ [ulwa029\_02:35]
\z

\ea%46
    \label{ex:clause:46}\textbf{}
          \textit{Ndï atuma wot ala mundu \textbf{moke} \textbf{amblanane}.}\\
\gll ndï  atuma      wot    ala      mundu  moko-e  \textbf{ambla}=na-n-e\\
    3\textsc{pl}  older.brother  younger  \textsc{pl.dist}  food  take-\textsc{dep}  \textsc{pl.refl}=give-\textsc{pfv-dep}\\
\glt `They, those brothers, shared the food.’ [ulwa033\_01:03]
\z

Also, although not necessarily common, it is possible for \textit{na-} ‘give’ to follow a verb in the preceding clause that means something other than ‘take’. For example, the verb \textit{na-} ‘give’ may follow the verbs \textit{wana-} ‘cook’ \REF{ex:clause:47} or \textit{nïkï-} ‘dig, cut’ \REF{ex:clause:48}.

\ea%47
    \label{ex:clause:47}
          \textit{Ma isi} \textbf{\textit{wanap}} \textit{yawa} \textbf{\textit{lananda}}.\\
\gll ma      isi    \textbf{wana}{}-p  yawa  ala=\textbf{na}{}-nda\\
    3\textsc{sg.obj}  soup  cook-\textsc{pfv}  uncle  \textsc{pl.dist}=give-\textsc{irr}\\
\glt `[They] will cook her soup and give [it] to the uncles.’ [ulwa014\_38:05]
\z

\ea%48
    \label{ex:clause:48}
          \textit{An keka} \textbf{\textit{mankap}} \textbf{\textit{ndïnan}}.\\
\gll an      keka      ma=\textbf{nïkï-}p      ndï=\textbf{na}{}-n\\
    1\textsc{pl.excl}  completely  \textsc{3sg.obj}=dig{}-\textsc{pfv}  \textsc{3pl=}give-\textsc{pfv}\\
\glt `We butchered it and gave it out completely to them.’ [ulwa014\_47:37]
\z

\is{clause|)}
\is{alignment|)}
\is{argument alignment|)}
\is{ditransitive|)}
\is{ditransitive alignment|)}

\is{ditransitive alignment|(}
\is{ditransitive|(}
\is{argument alignment|(}
\is{alignment|(}
\is{clause|(}


While ‘giving’ is the prototypical event to be encoded by \isi{ditransitive} constructions (in languages that exhibit them), there are other verbs that are likely to function similarly crosslinguistically. In the remainder of this section I describe how ‘showing’ events are encoded in Ulwa. Whereas ‘giving’ events in Ulwa are encoded with two \isi{transitive} verbs (typically \textit{tï-} ‘take’ and \textit{na-} ‘give’), ‘showing’ events are encoded with a single \isi{intransitive} verb (\textit{si-} ‘push’). In other contexts, this verb is used \isi{transitive}ly \REF{ex:clause:49}, often in conjunction with the verb \textit{lï-} ‘put’, to covey the sense of something being pushed upon something else, as in \REF{ex:clause:50} and \REF{ex:clause:51}, or in conjunction with the preverbal form [ikali] (literally ‘hand-send’) to convey the act of grabbing, holding, or catching, as in \REF{ex:clause:52} and \REF{ex:clause:53}.\footnote{Like \textit{tï-} ‘take’, \textit{si-} ‘push’ is often \isi{defective} in that it commonly lacks \isi{perfective} or \isi{imperfective} \isi{aspect} marking.}

\ea%49
    \label{ex:clause:49}
          \textit{Ndin u itïtïl} \textbf{\textit{ndïse}}.\\
\gll ndï=in  u    itïtïl  ndï=\textbf{si}{}-e\\
    3\textsc{pl}=in  from  dust  \textsc{3pl}=push-\textsc{ipfv}\\
\glt `[I] was pushing the dust out from them.’ (i.e., shaking out the dust) [ulwa037\_57:21]
\z

\ea%50
    \label{ex:clause:50}
          \textit{Unap} \textbf{\textit{ndïs}} \textit{apïn} \textbf{\textit{lïp}}.\\
\gll u=nap    ndï=\textbf{si}    apïn  \textbf{lï}{}-p\\
    2\textsc{sg=}for  3\textsc{pl}=push  fire    put-\textsc{pfv}\\
\glt `[They] put them on the fire for you.’ [ulwa014\_36:41]
\z

\ea%51
    \label{ex:clause:51}
          \textit{Nawoli mangusuwa imbake apa i wutï} \textbf{\textit{si}} \textbf{\textit{nimbamlïp}}.\\
\gll Nawoli  ma-ngusuwa  imba-ka-e    apa    i    wutï  \textbf{si}     nï=imbam-\textbf{lï}{}-p\\
    [name]  3\textsc{sg.obj{}-}poor  night-at-\textsc{dep}  house  go.\textsc{pfv}  leg    push    1\textsc{sg}=under-put-\textsc{pfv}\\
\glt `Nawoli, the poor thing, came to [my] house at night and put [his] legs under me.’ [ulwa014\_22:02]
\z

\ea%52
    \label{ex:clause:52}
          \textit{Nungol mï} \textbf{\textit{ikali}} \textbf{\textit{mas}}.\\
\gll nungol  mï      \textbf{i-kali}    ma=\textbf{si}\\
    child  3\textsc{sg.subj}  hand-send  \textsc{3sg.obj}=push\\
\glt `The son grabbed it.’  [ulwa006\_00:47]
\z

\newpage

\ea%53
    \label{ex:clause:53}
          \textit{U wa li mama \textbf{ikali masina}?}\\
\gll u    wa  li    ma=ma    \textbf{i-kali}    ma=\textbf{si}{}-na\\
    \textsc{2sg}  just  down  \textsc{3sg.obj}=go  hand-send  \textsc{3sg.obj}=push-\textsc{irr}\\
\glt `You’ll just go down there and grab it?’ [ulwa014\_64:44]
\z

  When used to encode a ‘showing’ event, however, the verb \textit{si-} ‘push’ is \isi{intransitive}: the \isi{agent} (the one showing) is the subject of the verb; the \isi{theme} (that which is shown) is marked by the \isi{oblique marker} \textit{=n} ‘\textsc{obl}’; and the \isi{experiencer} (the one to whom something is shown) is the object of the \isi{postposition} \textit{ul} ‘with’. The preferred order of these two non-core arguments is for the \isi{oblique}-marked \isi{noun phrase} to precede the \isi{postpositional phrase}. Literally such sentences may be rendered as ‘[\isi{agent}] pushes with [\isi{theme}] (along) with [\isi{experiencer}]’. They may be seen in examples \REF{ex:clause:54} through \REF{ex:clause:58}.

\is{clause|)}
\is{alignment|)}
\is{argument alignment|)}
\is{ditransitive|)}
\is{ditransitive alignment|)}

\ea%54
    \label{ex:clause:54}
          \textit{Gwam mï tawa} \textbf{\textit{man}} \textit{Mapana} \textbf{\textit{mol}} \textbf{\textit{si}}.\\
\gll Gwam  mï      tawa  ma=\textbf{n}      Mapana  ma=\textbf{ul} si\\
    [name]  \textsc{3sg.subj}  wound  3\textsc{sg.obj=obl}  [name]    3\textsc{sg.obj}=with    push\\
\glt `Gwam showed her wound to Mapana.’ [elicited]
\z

\ea%55
    \label{ex:clause:55}
          \textit{Gwam mï tawa} \textbf{\textit{ndïn}} \textit{yena} \textbf{\textit{minul}} \textbf{\textit{sina}}.\\
\gll Gwam  mï      tawa  ndï=\textbf{n}    yena    min=\textbf{ul}  \textbf{si}{}-na\\
    [name]  3\textsc{sg.subj}  wound  3\textsc{pl=obl}  mother    3\textsc{du}=with  push-\textsc{irr}\\
\glt `Gwam will show her wounds to the two women.’ [elicited]
\z

\ea%56
    \label{ex:clause:56}
          \textit{Gwam mï tawa} \textbf{\textit{ndïn}} \textit{ndï wopa} \textbf{\textit{lu}} \textbf{\textit{se}}.\\
\gll Gwam  mï      tawa  ndï=\textbf{n}    ndï  wopa  \textbf{lu}    \textbf{si}{}-e\\
    [name]  \textsc{3sg.subj}  wound  3\textsc{pl=obl}  \textsc{3pl}  all    with  push-\textsc{ipfv}\\
\glt `Gwam is showing her wounds to everyone.’ [elicited]
\z

\ea%57
    \label{ex:clause:57}
          \textit{Maya apa i lïmndï} \textbf{\textit{man}} \textbf{\textit{mol}} \textbf{\textit{si}}.\\
\gll ma=iya      apa    i    lïmndï  ma=\textbf{n}      ma=\textbf{ul} \textbf{si}\\
    3\textsc{sg.obj}=toward  house  go.\textsc{pfv}  eye    3\textsc{sg.obj=obl}  \textsc{3sg.obj}=with        push\\
\glt `[It] went to him in the house, and showed him [its] eye.’ [ulwa006\_08:05]
\z

\ea%58
    \label{ex:clause:58}
          \textit{\textbf{Man ndul si}.}\\
\gll ma=\textbf{n}      ndï=\textbf{ul}    \textbf{si}\\
    3\textsc{sg.obj=obl}  \textsc{3pl}=with  push\\
\glt `[He] showed it to them.’ [ulwa037\_25:03]
\z

\section{Obliques}\label{sec:11.4}

\is{oblique|(}
\is{clause|(}

Following from the discussion on possible \isi{ditransitive alignment} (\sectref{sec:11.3}), there is no language-internal reason to refer to any arguments as \isi{indirect object}s in Ulwa. The canonical placement of subjects is at the beginning of clauses, and the canonical placement of \is{direct object} (direct) objects is immediately preceding verbs (which are typically clause-final). All other non-verbal elements in a clause (i.e., \isi{noun phrase}s that are neither subjects nor objects, plus \isi{adverb}s and \isi{adpositional phrase}s) may be referred to as obliques. In Ulwa, obliques typically follow subjects and precede verbs (in \isi{intransitive} clauses) or follow subjects and precede objects (in \isi{transitive} clauses).

\is{clause|)}
\is{oblique|)}

\subsection{The oblique marker \textit{=n} ‘\textsc{obl}’}\label{sec:11.4.1}

\is{oblique marker|(}
\is{oblique|(}
\is{clause|(}

The clearest illustrations of the position and function of obliques in Ulwa are NPs that contain the \isi{enclitic} form \textit{=n} ‘\textsc{obl}’, which may be considered an \isi{oblique marker}. When following a \isi{noun phrase}, this oblique-marker \isi{enclitic} \textit{=n} ‘\textsc{obl}’ can be described as something like a non-core \isi{case marker}. It often encodes \isi{instrumental} functions, and may, in origin, be an \isi{instrumental} marker. Synchronically, however, it can serve other \isi{semantic} and grammatical functions, none of which relates to indicating a \isi{core argument}. The \isi{oblique marker} is realized by a set of similar \isi{allomorph}s, which are mostly in \isi{free variation} \REF{ex:clause:59}.

\ea%59
    \label{ex:clause:59}
          The oblique-marker \isi{enclitic}\\
\begin{tabbing}
{(\textit{=nï})} \= {(‘\textsc{obl}’)}\kill
{\textit{=n}} \> {‘\textsc{obl}’}\\
{\textit{=nï}} \> {‘\textsc{obl}}’\\
{\textit{=ïn}} \> {‘\textsc{obl}}’
\end{tabbing}
\z

In examples \REF{ex:clause:60} through \REF{ex:clause:65}, the \isi{oblique} NP marked by \textit{=n} ‘\textsc{obl}’ appears after the subject (if expressed) and before the object of the verb.

\ea%60
    \label{ex:clause:60}
          \textit{Itom} \textbf{\textit{napnï}} \textit{uta masap.}\\
\gll    itom  nap=\textbf{nï}      uta    ma=asa-p\\
    father  arrow=\textsc{obl}  bird  3\textsc{sg.obj}=hit-\textsc{pfv}\\
\glt `Father shot the bird with an arrow.’ [elicited]
\z

\ea%61
    \label{ex:clause:61}
          \textit{Mï \textbf{manji sina man} mundu maweyup.}\\
\gll    mï      ma-nji      sina  ma=\textbf{n}      mundu ma=we-u-p\\
    3\textsc{sg.subj}  3\textsc{sg.obj-poss}  knife  3\textsc{sg.obj=obl}  food    3\textsc{sg.obj}=cut-put-\textsc{pfv}\\
\glt `He cut the food with his knife.’ [elicited]
\z

\ea%62
    \label{ex:clause:62}
          \textit{Anton mangusuwata \textbf{inimnï} ananap.}\\
\gll    Anton  ma-ngusuwata  inim=\textbf{nï}  an=ana-p\\
    [name]  \textsc{3sg.obj-}poor  water=\textsc{obl}  \textsc{1pl.excl}=scrub-\textsc{pfv}\\
\glt `Anton, the poor thing, baptized us.’ (Literally ‘scrubbed us with water’) [ulwa014\_50:58]
\z

\ea%63
    \label{ex:clause:63}
          \textit{Nï} \textbf{\textit{anamnï}} \textit{ndatïna.}\\
\gll    nï    anam=\textbf{nï}    ndï=atï-na\\
    1\textsc{sg}  lightning=\textsc{obl}  3\textsc{pl}=hit-\textsc{irr}\\
\glt `I will strike them with lightning.’ [ulwa014†]
\z

\ea%64
    \label{ex:clause:64}
          \textit{Mï} \textbf{\textit{yotnï}} \textit{masap.}\\
\gll    mï      yot=\textbf{nï}      ma=asa-p\\
    3\textsc{sg.subj}  machete=\textsc{obl}  \textsc{3sg.obj}=hit-\textsc{pfv}\\
\glt `He hit it with [his] machete.’ [ulwa035\_02:25]
\z

\ea%65
    \label{ex:clause:65}
          \textit{Nïnji apa may nji} \textbf{\textit{ndïn}} \textit{apa up.}\\
\gll    nï-nji    apa    ma=i        nji    ndï=\textbf{n}    apa u-p\\
    \textsc{1sg-poss}  house  \textsc{3sg.obj}=go.\textsc{pfv}  thing  3\textsc{pl=obl}  house    put-\textsc{pfv}\\
\glt    ‘[I] went to my house and put things in the house.’ [ulwa040\_00:17]
\z

Note the argument structure of the word glossed as ‘put’ in \REF{ex:clause:65}: the object of the verb is the place where the item is put; the \isi{theme} is expressed in the \isi{oblique} \isi{phrase} (cf. the argument structure of the \ili{English} verb \textit{load}). Similarly, the word glossed as ‘tie’ in \REF{ex:clause:66} takes as object the thing to which something is tied; that which is tied is encoded in the \isi{oblique} \isi{phrase}.

\ea%66
    \label{ex:clause:66}
          \textit{Lamndu nungol kosape an} \textbf{\textit{man}} \textit{im itap.}\\
\gll    lamndu  nungol  ko=asa-p-e    an      ma=\textbf{n}      im     ita-p\\
    pig      child  \textsc{indf}=hit-\textsc{pfv-dep}  \textsc{1pl.excl}  \textsc{3sg.obj=obl}  tree     tie-\textsc{pfv}\\
\glt `[They] killed a small pig and we tied it to stick.’ [ulwa031\_03:51]
\z

Obliques may occur within \isi{compound verb} \isi{phrase}s or between verbs functioning together in complex \isi{verb phrase}s. Examples \REF{ex:clause:67} and \REF{ex:clause:68} illustrate non-\isi{instrumental} uses of the \isi{oblique marker}: in \REF{ex:clause:67} it has more of a \isi{comitative} meaning, whereas in \REF{ex:clause:68} it has more of a \isi{benefactive} meaning (see \sectref{sec:11.4.2}).

\ea%67
    \label{ex:clause:67}
          \textit{Wa ala lïmndï} \textbf{\textit{unanï}} \textit{mbu mawte.}\\
\gll    wa    ala      lïmndï  unan=\textbf{nï}    mbï-u     ma=uta-e\\
    village  \textsc{pl.dist}  eye    \textsc{1pl.incl=obl}  here-from    3\textsc{sg.obj}=grind-\textsc{ipfv}\\
\glt `Those [people from other] villages see it here among us.’ [ulwa037\_23:05]
\z

\ea%68
    \label{ex:clause:68}
          \textit{Mint} \textbf{\textit{ambïn}} \textit{ani menlïp.}\\
\gll    min=tï    ambï=\textbf{n}    ani    ma=in-lï-p\\
    3\textsc{du}=take  \textsc{sg.refl=obl}  bilum  \textsc{3sg.obj}=in-put-\textsc{pfv}\\
\glt `[I] put them into the \textit{bilum} [= string bag] for myself.’ [ulwa037\_01:57]
\z

The oblique-marked NP may occur alongside other non-core elements in a clause, such as \isi{postpositional phrase}s. \isi{Postpositional phrase}s may either precede oblique-marked NPs -- as in \REF{ex:clause:69} and \REF{ex:clause:70} -- or follow them -- as in \REF{ex:clause:71} and \REF{ex:clause:72} -- but they always occur between subjects and objects.

\ea%69
    \label{ex:clause:69}
          \textit{Nï \textbf{mol apïnï} mame.}\\
\gll    nï    \textbf{ma=ul}      \textbf{apïn=nï}   ma=ama-e\\
    1\textsc{sg}  3\textsc{sg.obj}=with  fire=\textsc{obl}  3\textsc{sg.obj}=eat-\textsc{ipfv}\\
\glt `I burn it with him.’ (Literally ‘I eat it with [= by means of] fire with \mbox{[= along with] him.’)} [ulwa014\_13:05]
\z

\ea%70
    \label{ex:clause:70}
          \textit{Nï \textbf{mawl ndïn} mbup.}\\
\gll    nï    \textbf{ma=ul}      \textbf{ndï=n}    mbï-u-p\\
    1\textsc{sg}  3\textsc{sg.obj}=with  \textsc{3pl=obl}  here-put-\textsc{pfv}\\
\glt `I planted them here with him.’ [ulwa014\_54:28]
\z

\ea%71
    \label{ex:clause:71}
          \textit{\textbf{Ndïn} maka \textbf{ya ndiya} ata unde.}\\
\gll    \textbf{ndï=n}    maka  \textbf{ya}      \textbf{ndï=iya}    ata  unda-e\\
    3\textsc{pl=obl}  thus  coconut  3\textsc{pl=}toward  up  go-\textsc{ipfv}\\
\glt `With them [= straps around their feet] [they] would go up coconut trees like that.’ [ulwa018\_01:05]
\z

\ea%72
    \label{ex:clause:72}
          \textit{Ala \textbf{nïn amba ngo} numbu lïp itana man.}\\
\gll    ala      \textbf{nï=n}    \textbf{amba}      \textbf{nga=u}        numbu lï-p      it-ana    ma-n\\
    \textsc{pl.dist}  1\textsc{sg=obl}  mens.house  \textsc{sg.prox=}from  post    put-\textsc{pfv}  build\textsc{{}-irr} go\textsc{{}-ipfv}\\
\glt `They are going to tie me to a post in this men’s house.’ [ulwa001\_14:14]
\z

The \isi{negator} \textit{ango} ‘\textsc{neg}’ typically follows subjects (when expressed), but precedes any \isi{oblique} NPs, as in \REF{ex:clause:73}, \REF{ex:clause:74}, and \REF{ex:clause:75}.

\ea%73
    \label{ex:clause:73}
          \textit{U \textbf{ango inambanï} ini men.}\\
\gll    u    \textbf{ango}  \textbf{inamba=nï}  ini      ma=in\\
    2\textsc{sg}  \textsc{neg}  money=\textsc{obl}  ground    3\textsc{sg.obj}=get\\
\glt `You did not buy the land.’ (Literally ‘get the land with money’) [ulwa014\_27:11]
\z

\ea%74
    \label{ex:clause:74}
          \textit{\textbf{Ango} maka \textbf{nginï} ute.}\\
\gll    \textbf{ango}  maka  \textbf{ngin=nï}  uta-e\\
    \textsc{neg}  thus  net=\textsc{obl}  grind-\textsc{ipfv}\\
\glt `[They] didn’t catch [fish] with the nets.’ [ulwa036\_02:24]
\z

\ea%75
    \label{ex:clause:75}
          \textit{\textbf{Ango man} ambi itanate.}\\
\gll    \textbf{ango}  \textbf{ma=n}      ambi  ita-na-t-e\\
    \textsc{neg}  3\textsc{sg.obj=obl}  big    build-\textsc{irr-spec-dep}\\
\glt `[I] won’t build it [too] big.’ [ulwa042\_05:46]
\z

Example \REF{ex:clause:75} also illustrates the preverbal placement of an \isi{adjective} when functioning \isi{adverb}ially (\sectref{sec:8.2.6}) and the effect of this on the \isi{semantic} object of the verb: it is \isi{demote}d to an \isi{oblique}, being marked by the \isi{oblique} marker \textit{=n} ‘\textsc{obl}’. This phenomenon also occurs in \REF{ex:clause:76}, \REF{ex:clause:77}, and \REF{ex:clause:78}.

\ea%76
    \label{ex:clause:76}
          \textit{Ndï ango \textbf{ndïn anma} asap.}\\
\gll    ndï  ango  \textbf{ndï=n}    \textbf{anma}  asa-p\\
    3\textsc{pl}  \textsc{neg}  \textsc{3pl=obl}  good  hit-\textsc{pfv}\\
\glt `They did not kill them well.’ [ulwa032\_54:25]
\z

\ea%77
    \label{ex:clause:77}
          \textit{U mat inde \textbf{man anma} tï inde.}\\
\gll    u    ma=tï      inda-e    \textbf{ma=n}      \textbf{anma}  tï     inda-e\\
    2\textsc{sg}  3\textsc{sg.obj}=take  walk-\textsc{dep}  \textsc{3sg.obj=obl}  good  take    walk-\textsc{dep}\\
\glt `You carry her, carry her well.’ [ulwa032\_35:58]
\z

\ea%78
    \label{ex:clause:78}
          \textit{Apa mï ndï \textbf{man tembi} itap.}\\
\gll    apa    mï      ndï  \textbf{ma=n}      \textbf{tembi}  ita-p\\
    house  \textsc{3sg.subj}  \textsc{3pl}  3\textsc{sg.obj=obl}  bad    build-\textsc{pfv}\\
\glt `The house -- they built it poorly.’ [ulwa014\_31:08]
\z

The same \isi{demotion} that occurs with \isi{adjective}s functioning \isi{adverb}ially also occurs when there is an intervening \isi{adpositional phrase} (see \sectref{sec:13.8.9} for examples).

\is{clause|)}
\is{oblique|)}
\is{oblique marker|)}

\subsection{The oblique marker as case marker}\label{sec:11.4.2}

\is{case marker|(}
\is{case|(}
\is{clause|(}
\is{oblique|(}
\is{oblique marker|(}

As described in \sectref{sec:11.4.1}, the primary function of the oblique-marker \isi{enclitic} \textit{=n} ‘\textsc{obl}’ is to encode \isi{non-core NP}s. These oblique-marked NPs may serve a number of functions in a clause, many of which are reminiscent of \isi{case}-marked NPs in languages that employ grammatical \isi{case}. Specifically, the marker \textit{=n} ‘\textsc{obl}’ has certain functions that resemble those of \isi{dative} markers found in other languages (although, importantly, it does not mark the \isi{recipient} in ‘giving’ constructions, \sectref{sec:11.3}). Three such \isi{dative}-like uses of \textit{=n} ‘\textsc{obl}’ are listed in \REF{ex:clause:78a}.

\ea%78a
    \label{ex:clause:78a}
\isi{Dative}-like uses of \textit{=n} \textsc{‘obl’}
\begin{tabbing}
{(-)} \= {(indicate those to whose \isi{disadvantage} something is done)}\kill
{-} \> {indicate \isi{possessor}s}\\
{-} \> {indicate \isi{agent}s}\\
{-} \> {indicate those to whose \isi{disadvantage} something is done}
\end{tabbing}
\z

The use of the \isi{enclitic} \textit{=n} ‘\textsc{obl}’ to encode \isi{possessor}s is discussed in \sectref{sec:9.1.5}.\footnote{Cf. the \isi{dative} of \isi{possession} in classical \ili{Latin}.} For the role of the \isi{enclitic} \textit{=n} ‘\textsc{obl}’ in marking \isi{agent}s in \isi{passive} constructions, see \sectref{sec:13.7}.\footnote{Cf. the \isi{dative} of the \isi{agent} in ancient \ili{Greek}.} In examples \REF{ex:clause:79}, \REF{ex:clause:80}, and \REF{ex:clause:81}, the \isi{oblique marker} indicates \isi{disadvantage}.\footnote{Cf. the \isi{dative} of \isi{disadvantage} in \ili{Latin}, \ili{Greek}, \ili{German}, etc.}

\ea%79
    \label{ex:clause:79}
          \textit{Mï \textbf{unan} mawat pe wombïn ne.}\\
\gll    mï      unan=\textbf{n}    ma=wat    p-e    wombïn=n  ni{}-e\\
    3\textsc{sg.subj}  \textsc{1pl.incl=obl}  \textsc{3sg.obj}=atop  be\textsc{{}-dep} work=\textsc{obl}  act-\textsc{ipfv}\\
\glt `He is hurting us by doing work during it [= this period of mourning].’ [ulwa030\_05:00]
\z

\ea%80
    \label{ex:clause:80}
          \textit{Tembi nji ngala apan \textbf{ndïn} mbïlïp.}\\
\gll    tembi  nji    ngala    apa=n      ndï=\textbf{n}    mbï-lï-p\\
    bad    thing  \textsc{pl.prox}  house=\textsc{obl}  3\textsc{pl=obl}  here-put-\textsc{pfv}\\
\glt `These bad things [= flies] have put [their] house [i.e., nest] here to their disadvantage.’ [ulwa032\_30:19]
\z

\ea%81
    \label{ex:clause:81}
          \textit{Ndï mokum} \textbf{\textit{anïn}} \textit{wandam pe ndam!}\\
\gll    ndï  mokum  an=\textbf{ïn}        wandam  p-e    ndï=ama\\
    3\textsc{pl}  stealth    1\textsc{pl.excl=obl}  jungle    be\textsc{{}-dep} 3\textsc{pl}=eat\\
\glt `They are stealthily in [our] jungles, eating them [= our crops]!’ [ulwa032\_38:45]
\z

Sometimes, as in example \REF{ex:clause:81}, it is not clear whether the \isi{oblique marker} is encoding a \isi{possessor} or the experiencer of some \isi{disadvantage}. In \REF{ex:clause:82}, the oblique-marked NP indicating \isi{disadvantage} stands in its own \isi{embedded clause}, taking as its \isi{predicate} the \isi{locative verb} \textit{p-} ‘be’.

\ea%82
    \label{ex:clause:82}
          \textit{Ala \textbf{anïn pe} ndïwale.}\\
\gll    ala      an=\textbf{ïn}        p-e      ndï=wali{}-e\\
    \textsc{pl.dist}  1\textsc{pl.excl=obl}  be\textsc{{}-dep} 3\textsc{pl}=hit-\textsc{ipfv}\\
\glt `People were killing them [= our dogs], while we were there suffering for it.’ [ulwa032\_11:43]
\z

In \REF{ex:clause:83}, the \isi{dative} of \isi{disadvantage} usage of the \isi{oblique marker} has an almost predicative sense.

\ea%83
    \label{ex:clause:83}
          \textit{Nïpokonampïta} \textbf{\textit{un}} \textit{mapïna.}\\
\gll    nïpokonam=p-ta  u=\textbf{n}    ma=p-na\\
    hard=\textsc{cop{}-cond} 2\textsc{sg=obl}  3\textsc{sg.obj}=be-\textsc{irr}\\
\glt `If [the soil] is hard, [it] will be no good for you.’ (Literally ‘If hard, [it] will be there to your disadvantage.’) [ulwa037\_50:58]
\z

Some of the various uses of the \isi{oblique marker} *n as found in different \ili{Keram-Ramu} languages is provided in \citet[54--55]{KillianBarlow2022}.

\is{oblique marker|)}
\is{oblique|)}
\is{clause|)}
\is{case|)}
\is{case marker|)}

\subsection{Other oblique arguments}\label{sec:11.4.3}

\is{oblique argument|(}
\is{oblique|(}
\is{clause|(}

Other non-core elements (namely, \isi{adverb}s and \isi{adpositional phrase}s) also typically occur between subjects and objects. For examples of this SXOV \isi{word order}, see \sectref{sec:8.1} (on \isi{postposition}s) and \sectref{sec:8.2} (on \isi{adverb}s). When a clause contains both an \isi{adverb} and an \isi{adpositional phrase}, the \isi{adverb} typically precedes the \isi{adpositional phrase}, as in \REF{ex:clause:84}.

\ea%84
    \label{ex:clause:84}
          \textit{Mï \textbf{awal wandam mo} lop.}\\
\gll    mï      \textbf{awal}    \textbf{wandam}  \textbf{ma=u}      lo-p\\
    3\textsc{sg.subj}  yesterday  jungle    3\textsc{sg.obj}=from  go\textsc{{}-pfv}\\
\glt `Yesterday, he went around in jungle.’ [ulwa014\_10:31]
\z

It is possible for several obliques to occur in succession, as in \REF{ex:clause:85}, which contains a \isi{temporal adverb}, an oblique-marked NP, a \isi{postposition}, and a \isi{modal adverb}.

\is{clause|)}
\is{oblique|)}
\is{oblique argument|)}

\ea\label{ex:clause:85}
    \textit{Un \textbf{amun man u maka} wombïn ngamokop.}\\
\gll    un  \textbf{amun}  \textbf{ma=n}      \textbf{u}    \textbf{maka}  wombïn     nga=moko-p\\
    2\textsc{pl}  now  3\textsc{sg.obj=obl}  from  thus  work    \textsc{sg.prox}=take\textsc{{}-pfv}\\
\glt `You recently got this work from him.’ [ulwa037\_19:23]
\z


\section{Non-canonical argument structures}\label{sec:11.5}

\is{non-canonical argument structure|(}
\is{clause|(}

While Ulwa generally maintains a fairly rigid and unified distinction between S/A arguments and P arguments, there is one known construction that exhibits an unusual argument structure. In expressions of experiencing hunger, the \isi{experiencer} role is not the grammatical subject but is rather the object of the verb, whereas ‘hunger’ itself is the grammatical subject. In other words, there is a special \isi{case frame} for expressing the experience of hunger, which takes the form of ‘hunger hits someone’ (as opposed to, say, ‘someone is hungry’ or ‘someone has hunger’). This is illustrated by examples \REF{ex:clause:86}, \REF{ex:clause:87}, and \REF{ex:clause:88}.\footnote{It should be noted that the form \textit{mundu} ‘hunger’ also means ‘food’ or ‘animal’.}

\ea%86
    \label{ex:clause:86}
          \textit{\textbf{Mundu} \textbf{unanas}.}\\
\gll    \textbf{mundu}  unan=\textbf{asa}\\
    hunger  1\textsc{pl.incl}=hit\\
\glt `We’re hungry.’ [ulwa030\_06:13]
\z

\ea%87
    \label{ex:clause:87}
          \textit{An mbïlop wop \textbf{mundu} \textbf{anase}.}\\
\gll    an      mbï-lo-p    wo-p    \textbf{mundu}  an-\textbf{asa}-e\\
    \textsc{1pl.excl}  here-go-\textsc{pfv}  sleep-\textsc{pfv}  hunger  1\textsc{pl.excl}{}-hit-\textsc{ipfv}\\
\glt `We came here, spent the night, and we were hungry.’ [ulwa032\_07:38]
\z

\ea%88
    \label{ex:clause:88}
          \textit{\textbf{Mundu} \textbf{watïna}.}\\
\gll    \textbf{mundu}  u=\textbf{atï}-na\\
    hunger  \textsc{2sg}=hit-\textsc{irr}\\
\glt `You will be hungry.’ [ulwa037\_44:21]
\z

  Although the \isi{experiencer} is always encoded as the object of the verb, it is common for it also to be included as an additional subject argument, as in \REF{ex:clause:89} and \REF{ex:clause:90}.

\ea%89
    \label{ex:clause:89}
          \textbf{\textit{Nï}} \textit{ango mundu} \textbf{\textit{nïwale}}.\\
\gll \textbf{nï}    ango  mundu  \textbf{nï}=wali-e\\
    1\textsc{sg}  \textsc{neg}  hunger  1\textsc{sg}=hit-\textsc{ipfv}\\
\glt `I wasn’t hungry.’ [ulwa014†]
\z

\ea%90
    \label{ex:clause:90}
          \textit{Nï ndïn ka mata} \textbf{\textit{ndï}} \textit{mundu} \textbf{\textit{ndatïna}}.\\
\gll nï    ndï=n    ka  ma-ta    \textbf{ndï}  mundu  \textbf{ndï}=atï-na\\
    1\textsc{sg}  \textsc{3pl=obl}  let  go-\textsc{cond}  3\textsc{pl}  hunger  \textsc{3pl}=hit-\textsc{irr}\\
\glt `If I leave them and go, they will be hungry.’ [ulwa027\_00:49]
\z

When the \isi{experiencer} is a full NP, it is common to include this argument in subject position -- that is, before the grammatical subject of the verb ‘hit’ (i.e., before the noun \textit{mundu} ‘hunger’). The \isi{object marker}, however, which refers to the \isi{experiencer} argument, occurs in object position, \isi{clitic}izing to the verb, as in \REF{ex:clause:91}.

\ea%91
    \label{ex:clause:91}
          \textit{\textbf{Nïnji nungolke ngala} mundu \textbf{ndasape}}.\\
\gll \textbf{nï-nji}    \textbf{nungolke}  \textbf{ngala}    mundu  \textbf{ndï}=asa-p-e\\
    1\textsc{sg}{}-\textsc{poss}  child    \textsc{pl.prox}  hunger  \textsc{3pl}=hit-\textsc{pfv}{}-\textsc{dep}\\
\glt `My children were hungry.’ [ulwa032\_22:53]
\z

Occasionally a verb other than ‘hit’ is used in expressing hunger, as in \REF{ex:clause:92}, which uses the verb \textit{a-} ‘break’ (i.e., ‘hunger is breaking us’).

\ea%92
    \label{ex:clause:92}
          \textit{Unan mbïpe mane \textbf{mundu} \textbf{unanay}.}\\
\gll unan    mbï-p-e    ma-n-e      \textbf{mundu}  unan=\textbf{a}{}-e\\
    1\textsc{pl.incl}  here-be-\textsc{dep}  go-\textsc{ipfv}{}-\textsc{dep}  hunger  1\textsc{pl.incl}{}-break-\textsc{ipfv}\\
\glt `But we are going around here and hunger is breaking us.’ [ulwa037\_44:32]
\z

It is relevant to note that, although the noun \textit{mundu} ‘hunger’ is the grammatical subject in these constructions, it almost never receives a \is{subject marker} subject-marker \isi{determiner}, perhaps because of its low level of \isi{specificity} or \isi{definiteness} (see \sectref{sec:7.2}).

  Aside from this one hunger construction, there does not seem to be a robust \isi{morphosyntactic} distinction made in Ulwa between \isi{predicate}s expressing \isi{controlled} events and \isi{predicate}s expressing \isi{uncontrolled} events or states. Other expressions of physical or emotional states, such as pain, sickness, anger, and sadness, are not expressed with the \isi{experiencer} as the grammatical object, as is done for expressions of hunger. For example, pain is usually expressed by predicating \textit{apïn} ‘pain’ (literally ‘fire’) of the person or body part that is experiencing pain \REF{ex:clause:93}.

\ea%93
    \label{ex:clause:93}
          \textit{Nïnji uma ngala} \textbf{\textit{apïnpe}}.\\
\gll nï-nji    uma  ngala    \textbf{apïn}=p-e\\
    \textsc{1sg-poss}  bone  \textsc{pl.prox}  fire=\textsc{cop}{}-\textsc{dep}\\
\glt `These bones of mine were hurting.’ [ulwa032\_18:52]
\z

Anger may be expressed by predicating the \isi{adjective} \textit{matamal} ‘sharp; difficult, angry’ of the person experiencing anger \REF{ex:clause:94}.

\ea%94
    \label{ex:clause:94}
          \textit{Itom mï} \textbf{\textit{matamalp}}.\\
\gll itom  mï      \textbf{matamal}=p\\
    father  3\textsc{sg.subj}  sharp=\textsc{cop}\\
\glt `Father is angry.’ [elicited]
\z

More commonly, however, anger is expressed by means of a verbal expression with \textit{uni-} ‘shout’. In \REF{ex:clause:95}, the literal expression ‘shout for [something]’ or ‘shout on account of [something]’ conveys the meaning ‘be angry about [something]’. In \REF{ex:clause:96}, the literal expression ‘shout with [someone]’ conveys the meaning ‘be angry with [someone]’.

\ea%95
    \label{ex:clause:95}
          \textit{Mï ndïnap \textbf{unipe}.}\\
\gll    mï      ndï=nap  \textbf{uni}-p-e\\
    3\textsc{sg.subj}  \textsc{3pl}=for  shout-\textsc{pfv-dep}\\
\glt `She was angry about them [= seeds].’ [ulwa014\_12:24]
\z

\ea%96
    \label{ex:clause:96}
        \textit{We nï mol \textbf{une}.}\\
\gll    we    nï    ma=ul      \textbf{uni}-e\\
    then  1\textsc{sg}  3\textsc{sg.obj}=with  shout-\textsc{ipfv}\\
\glt `Then I got angry with her.’ [ulwa032\_02:13]
\z

See \sectref{sec:5.4} for other expressions of emotional and physical states.

\is{clause|)}
\is{non-canonical argument structure|)}

\section{Monoclausal sentences (simple sentences)}\label{sec:11.6}

\is{monoclausal sentence|(}
\is{simple sentence|(}
\is{clause|(}

A \isi{simple sentence} in Ulwa thus consists minimally of one subject and one \isi{predicate}. Since subjects may be pronominal and since subject \isi{pronoun}s may be omitted, it is possible for only the \isi{predicate} to be overt in the clause.

The \isi{predicate} consists minimally of a verb, whether \isi{transitive} or \isi{intransitive}. A \isi{transitive} verb has an object within its \isi{phrase} and may have \isi{object marker}s preceding it. \isi{TAM} \isi{suffix}ation may appear on the verb. A \isi{predicate} may contain more than one verb. There are also a number of \isi{compound verb}s consisting of \isi{discontinuous} elements, between which objects may occur.

The subject, too, when overt, may consist of multiple elements; these typically comprise \isi{noun phrase}s. Subjects often contain \isi{subject marker}s following the \isi{head} of the NP. Other \isi{determiner}s -- that is, in addition to \isi{subject marker}s and \isi{object marker}s -- are possible as well, whether as part of the subject or as part of the object in a \isi{transitive} \isi{verb phrase}.

In addition to the basic elements of the subject and the \isi{verb phrase} (which, if \isi{transitive}, also contains an object), the \isi{monoclausal sentence} may contain \isi{oblique}s. These typically occur between the subject and object, yielding a canonical \isi{word order} of SXOV.

\is{clause|)}
\is{simple sentence|)}
\is{monoclausal sentence|)}