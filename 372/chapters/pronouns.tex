\chapter{Pronouns}\label{sec:6}

\is{pronoun|(}

Various categories of pronouns are treated in the sections in this chapter. \isi{Demonstrative} words, which may also function pronominally, are treated in \sectref{sec:7.3}. Defined in terms of discourse function, these subcategories of pronouns all consist of words that refer to something that is either identified elsewhere in the discourse or thought to be identifiable either from context or from shared knowledge of the \isi{speech act participant}s. The referents of these pronominal forms are \isi{semantic}ally nouns. There are distributional similarities that exist among these subgroups of pronouns (for example, their members can all serve as the \isi{head} of an NP), as well as shared structural features (for example, these forms do not permit \isi{TAM} suffixation).

\is{pronoun|)}

\section{Personal pronouns}\label{sec:6.1}

\is{personal pronoun|(}
\is{pronoun|(}

The paradigm for Ulwa personal pronouns consists of three persons -- first, second, and third -- and three \isi{number}s -- \isi{singular}, \isi{dual}, and \isi{plural}. There is a \isi{clusivity} distinction exhibited in first \isi{person} non-\isi{singular} forms. The \isi{dual} forms denote exactly two referents, whereas \isi{plural} \isi{number} implies more than two (but can, at least for some speakers, be used to refer to exactly two referents as well, \sectref{sec:9.1.2}). The \isi{singular} form, as to be expected, is used when the referent is exactly one. Pronouns do not exhibit any \isi{gender} distinctions or any \isi{politeness} distinctions. The forms of the personal pronouns in Ulwa are given in \tabref{tab::6.1}.


\begin{table}
\caption{Personal pronouns}
\label{tab::6.1}
\begin{tabularx}{.7\textwidth}{lQQQ}
\lsptoprule
& \textsc{sg} & \textsc{du} & \textsc{pl}\\
\midrule
1 & {\itshape nï} & \textit{ngan} [\textsc{excl}]

\textit{ngunan} [\textsc{incl}] & \textit{an} [\textsc{excl}]

\textit{unan} [\textsc{incl}]\\
2 & {\itshape u} & {\itshape ngun} & {\itshape un}\\
3 & {\itshape mï} & {\itshape min} & {\itshape ndï}\\
\lspbottomrule
\end{tabularx}
\end{table}
All (and only the) \isi{non-singular} speech-act personal pronouns end in the formative /n/. The second \isi{person} is marked by the \isi{vowel} /u/, which occurs in each \isi{number} for this \isi{person}. The presence of this \isi{vowel} is also felt in the first \isi{person} \isi{non-singular} \isi{inclusive} forms; this has a certain logic to it, since these forms include the addressee(s) as a referent. The \isi{vowel} /a/ is found in all first \isi{person} forms, except the first \isi{person} \isi{singular}. \isi{Dual} forms are marked by initial /ng/, present in all \isi{dual} pronouns except the third \isi{person}. In fact, all \isi{dual} forms (except the third \isi{person}) can be analyzed as consisting of the \isi{plural} equivalent of these pronouns plus word-initial /ng/. The third \isi{person} \isi{dual} \isi{pronoun} \textit{min} ‘3\textsc{du’} stands out as being formally unusual in this regard.

\is{speech act participant}

The only \isi{polysyllabic} \isi{personal pronoun}s (the \isi{dual} and \isi{plural} first \isi{person} \isi{inclusive} forms) are clearly derived from the combination of two other pronouns, namely a second \isi{person} form (\isi{plural} or \isi{dual}) and an \isi{exclusive} first \isi{person} \isi{plural} form -- i.e., \textit{unan} ‘1\textsc{pl.incl’} < \textit{un} ‘2\textsc{pl’} + \textit{an} ‘\textsc{1pl.excl’} and \textit{ngunan} ‘1\textsc{du.incl’} \linebreak < \textit{ngun} ‘2\textsc{du’} + \textit{an} ‘1\textsc{pl.excl’}.\footnote{If /ng/ is treated as a \isi{dual} formative, /n/ as a \isi{non-singular} \isi{speech act participant} formative, and /a/ as an indicator of the first \isi{person}, then these two pronouns could be further analyzed as /u-n-a-n/ ‘\textsc{2-pl-1-pl’} and /ng-u-n-a-n/ ‘\textsc{du-2-pl-1-pl’}, respectively. These two \isi{exclusive} pronouns are perhaps younger forms, possibly \isi{calque}d from one of the nearby \ili{Yuat} languages, which contrast \isi{inclusive} and \isi{exclusive} first \isi{person} forms. Indeed, \citet[227]{Foley2018} proposes that a pronominal \isi{clusivity} distinction is an areal feature.}

  Some speakers have reported an alternative 3\textsc{du} pronominal form /ndin/. However, it is not well attested in my corpus. The only examples of it seem to reflect an adnominal \isi{demonstrative} rather than a \isi{personal pronoun}. I suspect that it is an abridged form of the \isi{dual} \isi{distal} \isi{demonstrative} \textit{andin} (cf. the form [nda] for the \isi{singular} \isi{distal} \isi{demonstrative} \textit{anda}, \sectref{sec:7.3}). The form [ndin], for example, occurs at the end of an NP in \REF{ex:pron:1}.

\ea%1
    \label{ex:pron:1}
            \textit{Kambok inom ngusuwa} \textbf{\textit{ndin}} \textit{asika ndule.}\\
\gll    Kambok  inom  ngusuwa  \textbf{ndin}  asi-ka  ndï=ula-e\\
    Kambuku  mother  poor    3\textsc{du}?  sit-let  3\textsc{pl}=weave-\textsc{ipfv}\\
\glt `Two poor women from Kambuku [village] used to sit and weave them.’ [ulwa014\_42:11]
\z

In \REF{ex:pron:2}, the functional similarity between \textit{[a]ndin} ‘\textsc{du.dist}’ (‘those two’) and \textit{anda} ‘\textsc{sg.dist}’ (‘that’) is clear, as each form immediately follows a corresponding \isi{numeral}.

\newpage

\ea%2
    \label{ex:pron:2}
         \textit{wik wopa kwa nda nini} \textbf{\textit{ndintïna}}\\
\gll wik  wopa  kwa  anda    nini  \textbf{ndin}=tï-na\\
    week  all    one    \textsc{sg.dist}  two  3\textsc{du}?=take-\textsc{irr}\\
\glt `for all of one week, or two’ (\textit{wik} = TP) [ulwa031\_00:27]
\z

Nevertheless, if there is in fact a pronominal use of \textit{ndin} ‘3\textsc{du} (?)’, then it most likely derives historically from this \isi{demonstrative} word. A similar functional change, whereby a \isi{demonstrative} has come to serve more general pronominal functions without carrying any spatial \isi{deictic} meaning, can be seen in the form \textit{ala} ‘\textsc{pl.dist}’ (‘those’) (\sectref{sec:7.3}).

  Each of the forms given in \tabref{tab::6.1} may serve as the subject of either an \isi{intransitive} or a \isi{transitive} clause. Pronominal objects, on the other hand, are indicated by a paradigm of \isi{clitic}s that precede a verb, \isi{postposition}, or \isi{oblique marker} (\tabref{tab::6.2}). They are almost identical to their subject-form equivalents; the main difference occurs in the third \isi{person} \isi{singular} form, which is /mï/ as a subject, but /ma=/ whenever in \isi{non-subject} roles.


\begin{table}
\caption{Pronominal object markers (non-subject markers)}
\label{tab::6.2}
\begin{tabularx}{.75\textwidth}{lQQQ}
\lsptoprule
& \textsc{sg} & \textsc{du} & {\scshape pl}\\
\midrule
1 & {\itshape nï=} & \textit{ngan=} [\textsc{excl}]
\is{non-subject marker}
\textit{ngunan=} [\textsc{incl}] & \textit{an=} [\textsc{excl}]

\textit{unan=} [\textsc{incl}]\\
2 & {\itshape u=} & {\itshape ngun=} & {\itshape un=}\\
3 & {\itshape ma=} & {\itshape min=} & {\itshape ndï=}\\
\lspbottomrule
\end{tabularx}
\end{table}
These \isi{non-subject} pronominal forms may also be used as \is{adnominal possessor} adnominal possessive pronominal forms, immediately preceding the \isi{possessum} (\sectref{sec:9.1.5}). Further information on \isi{object marker}s (and \isi{non-subject} pronominal forms) is provided in \sectref{sec:7.2}.

  In casual speech, the \isi{dual} and \isi{plural} first \isi{person} \isi{inclusive} pronouns may be pronounced without the final /-n/ -- that is, [nguna] and [una] for /ngunan/ and /unan/, respectively.

\is{pronoun|)}
\is{personal pronoun|)}

\section{Possessive pronouns}\label{sec:6.2}

\is{possessive pronoun|(}
\is{possession|(}
\is{pronoun|(}

Possessive pronominal forms are all clearly derived from the corresponding personal pronominal forms plus the word \textit{nji} ‘thing’. More precisely -- the possessive pronouns correspond to the paradigm of \isi{non-subject} personal pronominal forms, since the third \isi{person} \isi{singular} possessive form is [manji], rather than \textsuperscript{†}[mïnji] (see \tabref{tab::6.1} and \tabref{tab::6.2} for the subject and object personal pronominal paradigms). These possessive forms do not necessarily function as subjects or objects themselves, but rather typically occur within NPs \isi{head}ed by another nominal form – that is, they are typically \is{adnominal possessor} adnominal possessive pronouns. As with all pronominal forms, there is no \isi{gender} distinction, whether in the third \isi{person} or elsewhere.

  \tabref{tab::6.3} provides the forms of the possessive pronouns and possessive demonstratives in Ulwa.


\begin{table}
\caption{Possessive pronouns and demonstratives}
\label{tab::6.3}
\begin{tabularx}{.8\textwidth}{lQQQ}
\lsptoprule
& {\scshape sg} & {\scshape du} & {\scshape pl}\\
\midrule
1 & {\itshape nïnji} & \textit{nganji} [\textsc{excl}]

\textit{ngunanji} [\textsc{incl}] & \textit{anji} [\textsc{excl}]

\textit{unanji} [\textsc{incl}]\\
2 & {\itshape unji} & {\itshape ngunji} & {\itshape unji}\\
3 & {\itshape manji} & {\itshape minji} & {\itshape ndïnji}\\
{\scshape refl} & {\itshape ambïnji} & {\itshape ambinji} & {\itshape amblanji}\\
{\scshape prox} & {\itshape nganji} & {\itshape nginji} & {\itshape ngalanji}\\
{\scshape dist} & {\itshape andanji} & {\itshape andinji} & {\itshape alanji}\\
\lspbottomrule
\end{tabularx}
\end{table}
All forms in \tabref{tab::6.3} are transparently decomposable. There is only one minor \isi{phonological} change, which affects the \isi{dual} and non-third \isi{person} \isi{plural} forms, as well as the \isi{dual} \isi{reflexive} form. This is the shortening (i.e., \isi{quasi-degemination}) of the sequence \isi{nasal}-plus-\isi{prenasalization}. Thus, the possessive first \isi{person} \isi{plural} \isi{exclusive} form has the underlying form /annji/, but is realized as [anji]. Similarly, second \isi{person} \isi{plural} /unnji/ is realized as [unji], making it \isi{homophonous} with the second \isi{person} \isi{singular} possessive form /unji/.

  Throughout this grammar, the pronominal forms are treated \isi{orthographic}ally as individual words, reflecting their \isi{phonological} unity. However, the glossing reflects their composite structure (i.e., object \isi{pronoun} + \textit{{}-nji} ‘\textsc{poss’}). The possessive ending \textit{{}-nji} ‘\textsc{poss’} is also found in full NPs, but only when they end in a \isi{determiner} (i.e., an \isi{object marker} or \isi{deictic}); that is, \textit{{}-nji} ‘\textsc{poss’} cannot attach to bare nouns. Possession with full-NP \isi{possessor}s is discussed further in \sectref{sec:9.1.5}.

  Possessive pronouns do not index anything about the \isi{possessum}. That is, although the \isi{possessive pronoun} encodes the \isi{person} and \isi{number} of the \isi{possessor}, it offers no information about the \isi{person} or \isi{number} (or \isi{gender}) of that which is \isi{possessed}. Furthermore, no \isi{morphological} or \isi{syntactic} distinction is made in Ulwa between \is{alienable possession} alienable and \isi{inalienable possession}.

  Possessive forms may refer to first \isi{person} \REF{ex:pron:3}, second \isi{person} \REF{ex:pron:4}, or third \isi{person} \REF{ex:pron:5} \isi{possessor}s.

\ea%3
    \label{ex:pron:3}
            \textbf{\textit{Nïnji}} \textit{anapa mï atalap.}\\
\gll    \textbf{nï-nji}    anapa  mï      atal-a-p\\
    1\textsc{sg-poss}  sister  3\textsc{sg.subj}  laughter-break-\textsc{pfv}\\
\glt `My sister laughed.’ [elicited]
\z

\ea%4
    \label{ex:pron:4}
            \textbf{\textit{Unji}} \textit{aweta mï anma.}\\
\gll    \textbf{u-nji}    aweta  mï      anma\\
    2\textsc{sg-poss}  friend  3\textsc{sg.subj}  good\\
\glt `Your friend is nice.’ [elicited]
\z

\ea%5
    \label{ex:pron:5}
            \textbf{\textit{Manji}} \textit{wonmi ndï namlip.}\\
\gll    \textbf{ma-nji}      wonmi    ndï  namli=p\\
    \textsc{3sg.obj-poss}  hair    \textsc{3pl}  soft=\textsc{cop}\\
\glt `Her hair is soft.’ [elicited]
\z

\isi{Possessor}s may be \isi{singular}, \isi{dual}, or \isi{plural}, and may occur in object roles as well as subject roles. In example \REF{ex:pron:6}, the possessive form refers to a third \isi{person} \isi{plural} \isi{possessor}, here part of a \isi{direct object} NP.

\ea%6
    \label{ex:pron:6}
            \textit{Ndï lïmndï} \textbf{\textit{ndïnji}} \textit{aweta mala.}\\
\gll    ndï  lïmndï  \textbf{ndï-nji}    aweta  ma=ala\\
    \textsc{3pl}  eye    \textsc{3pl-poss}  friend  \textsc{3sg.obj=}see\\
\glt `They saw their friend.’ [elicited]
\z

Third \isi{person} possessive forms, such as the one in \REF{ex:pron:7}, can have ambiguous reference, pointing either (reflexively) to an \isi{antecedent} in the clause or to a third party not necessarily mentioned in the clause.

\ea%7
    \label{ex:pron:7}
            \textit{Ginam mï inim mo} \textbf{\textit{manji}} \textit{aweta ndït atalp.}\\
\gll    Ginam\textsubscript{i} mï      inim  ma=u      \textbf{ma-nji}\textsubscript{i/j} aweta ndï=tï    ata-lï-p\\
    [name]  \textsc{3sg.subj}  water  \textsc{3sg.obj}=from  \textsc{3sg.obj-poss}  friend \textsc{3pl=}take  up-put-\textsc{pfv}\\

\glt `Ginam pulled her friends out of the water.’ [elicited]
\z

That is, \textit{manji} ‘\textsc{3sg.obj-poss}’ in sentence \REF{ex:pron:7} can refer either to Ginam’s friends or to someone else’s (e.g., Yawat’s). To clarify that the \isi{pronoun} refers to Ginam, a different form may instead be used, composed of the \isi{reflexive} \isi{pronoun} of the appropriate \isi{number} (\sectref{sec:6.3}) and \textit{{}-nji} ‘\textsc{poss}’, giving the meaning ‘X’s own’, as in \REF{ex:pron:8}.

\ea%8
    \label{ex:pron:8}
            \textit{Ginam mï inim mo} \textbf{\textit{ambïnji}} \textit{aweta ndït atalp.}\\
\gll    Ginam\textsubscript{i} mï      inim  ma=u      \textbf{ambï-nji}\textsubscript{i/*j}  aweta ndï=tï    ata-lï-p\\
    [name]  \textsc{3sg.subj}  water  \textsc{3sg.obj}=from  \textsc{sg.refl-poss}  friend \textsc{3pl}=take  up-put-\textsc{pfv}\\
\glt `Ginam pulled her own friends out of the water.’ [elicited]
\z

These forms are similar in function to certain pronouns found in some other languages, such as the \ili{Latin} possessive \isi{reflexive} \isi{pronoun} \textit{suus} ‘his/her/its/their own’. In addition to their use in clarifying third \isi{person} \isi{antecedent}s, however, the Ulwa forms may also be used with first or second \isi{person} reference in order to convey the sense ‘my own’, ‘our own’, ‘your own’, and so on, as in \REF{ex:pron:9}, \REF{ex:pron:10}, and \REF{ex:pron:11}.

\ea%9
    \label{ex:pron:9}
            \textit{Nï lïmndï} \textbf{\textit{ambïnji}} \textit{aweta mala.}\\
\gll    nï    lïmndï  \textbf{ambï-nji}    aweta  ma=ala\\
    1\textsc{sg}  eye    \textsc{sg.refl-poss}  friend  \textsc{3sg.obj=}see\\
\glt `I saw my own friend.’ [elicited]
\z

\ea%10
    \label{ex:pron:10}
          \textit{Min lïmndï} \textbf{\textit{ambinji}} \textit{aweta mala.}\\
\gll    min  lïmndï  \textbf{ambin-nji}    aweta  ma=ala\\
    3\textsc{du}  eye    \textsc{du.refl-poss}  friend  \textsc{3sg.obj=}see\\
\glt `The two of them saw their own friend.’ [elicited]
\z

\ea%11
    \label{ex:pron:11}
          \textit{Un lïmndï} \textbf{\textit{amblanji}} \textit{aweta ndala.}\\
\gll    un  lïmndï  \textbf{ambla-nji}    aweta  ndï=ala\\
    2\textsc{pl}  eye    \textsc{pl.refl-poss}  friend  \textsc{3sg.obj=}see\\
\glt `You saw your own friends.’ [elicited]
\z

These \isi{reflexive} possessive forms are not marked for \isi{person} (or \isi{gender}); they are only marked for \isi{number}. They are included in \tabref{tab::6.3}.

  It may here be noted that possessive forms need not necessarily precede nouns. Although they cannot precede verbs (without the verbs having been \isi{nominalize}d), they can precede \isi{adjective}s. This happens, however, only when the \isi{adjective} is functioning substantively (i.e., nominally) (\sectref{sec:5.3}), as in \REF{ex:pron:12} and \REF{ex:pron:13}. This further suggests that there is little if any \isi{morphosyntactic} distinction between property words (i.e., “\isi{adjective}s”) and \isi{semantic}ally more prototypical nouns.

\ea%12
    \label{ex:pron:12}
          \textbf{\textit{Manji}} \textit{anma ndï apa map.}\\
\gll    \textbf{ma-nji}      anma  ndï  apa    ma=p\\
    3\textsc{sg.obj-poss}  good  \textsc{3pl}  house  3\textsc{sg.obj=}be\\
\glt `His good ones [= daughters] are in the house.’ [elicited]
\z

\ea%13
    \label{ex:pron:13}
          \textbf{\textit{Nïnji}} \textit{njukuta mï wandam i.}\\
\gll    \textbf{nï-nji}    njukuta  mï      wandam  i\\
    1\textsc{sg-poss}  small    \textsc{3sg.subj}  jungle    go.\textsc{pfv}\\
\glt `My small one [= dog] went into the jungle.’ [elicited]
\z

Finally, the possessive forms can be used substantively with an implied noun, as in \REF{ex:pron:14} and \REF{ex:pron:15}.

\ea%14
    \label{ex:pron:14}
          \textit{Unji apa mï njukutap} \textbf{\textit{nïnji}} \textit{mï ambip.}\\
\gll    u-nji    apa  mï        njukuta=p  \textbf{nï-nji}    mï   ambi=p\\
    2\textsc{sg-poss}  house  3\textsc{sg.subj}  small=\textsc{cop}  1\textsc{sg-poss}  \textsc{3sg.subj}  big=\textsc{cop}\\

\glt   ‘Your house is small; mine is big.’ [elicited]
\z

\ea%15
    \label{ex:pron:15}
          \textit{Kayta manji tïn mï} \textbf{\textit{nïnji}} \textit{asap.}\\
\gll    Kayta  ma-nji      tïn    mï      \textbf{nï-nji}    asa-p\\
    [name]  \textsc{3sg.obj-poss}  dog  \textsc{3sg.subj}  \textsc{1sg-poss}  hit-\textsc{pfv}\\
\glt `Kayta’s dog attacked mine.’ [elicited]
\z

\is{pronoun|)}
\is{possession|)}
\is{possessive pronoun|)}

Substantive possessive forms (otherwise more akin to \isi{determiner}s) are thus in some ways rather similar to \ili{English} \isi{possessive pronoun}s; they can, accordingly, be translated as ‘mine’, ‘ours’, ‘yours’, ‘hers’, and so on.

\section{Reflexive and reciprocal pronouns}\label{sec:6.3}

\is{reflexive pronoun|(}
\is{reciprocal pronoun|(}
\is{pronoun|(}

A \isi{reflexive} \isi{pronoun} generally has as an \isi{antecedent} a form occurring (or implied) earlier in the same clause that has the same referent (i.e., the two words are co-indexed for reference). Reflexive pronouns are \isi{inflect}ed for \isi{number}, but not for \isi{person} or \isi{gender}. As these forms function as objects, they typically \isi{clitic}ize to a following verb or \isi{postposition}. The Ulwa \isi{reflexive} pronouns are given in \tabref{tab::6.4}.


\begin{table}
\caption{Reflexive/reciprocal pronouns}
\label{tab::6.4}
\begin{tabularx}{.7\textwidth}{lQQQ}
\lsptoprule
& {\scshape sg} & {\scshape du} & {\scshape pl}\\
\midrule
{\scshape refl} & {\itshape ambï=} & {\itshape ambin=} & {\itshape ambla=}\\
\lspbottomrule
\end{tabularx}
\end{table}
Thus, these pronouns are similar to their \ili{English} translations ending in ‘-self’ or ‘-selves’, in that there is no distinction made among them for \isi{person} -- whether first, second, or third -- but they are distinguished for \isi{number}. These three forms are based on the same root, /amb(ï)-/. The use of \isi{reflexive} pronouns is illustrated by examples \REF{ex:pron:16} through \REF{ex:pron:22}.

\ea%16
    \label{ex:pron:16}
          \textit{Tambana mï} \textbf{\textit{ambuwalinda}}.\\
\gll Tambana  mï      \textbf{ambï}=wali-nda\\
    [name]    3\textsc{sg.subj}  \textsc{sg.refl=}hit-\textsc{irr}\\
\glt `Tambana will hit herself.’ [elicited]
\z

\ea%17
    \label{ex:pron:17}
          \textit{Mï} \textbf{\textit{ambït}} \textit{nïmal ndïlp.}\\
\gll    mï       \textbf{ambï}=tï    nïmal  ndï=lï-p\\
    3\textsc{sg.subj}  \textsc{sg.refl=}take  river  3\textsc{pl}=put-\textsc{pfv}\\
\glt `It has put itself in the rivers.’ [ulwa014\_69:32]
\z

\ea%18
    \label{ex:pron:18}
          \textit{Nï} \textbf{\textit{ambïnakap}} \textit{mol une.}\\
\gll    nï     \textbf{ambï}=nakap  ma=ul      uni-e\\
    1\textsc{sg}  \textsc{sg.refl}=for  3\textsc{sg.obj}=with  shout-\textsc{ipfv}\\
\glt `I was yelling at her on my own behalf.’ [ulwa032\_16:56]
\z

\ea%19
    \label{ex:pron:19}
          \textit{Ngun} \textbf{\textit{ambinkalamp}}.\\
\gll ngun  \textbf{ambin}=kalam=p\\
    2\textsc{du}  \textsc{du.refl=}knowledge=\textsc{cop}\\
\glt `You two know yourselves.’ [elicited]
\z

\ea%20
    \label{ex:pron:20}
          \textit{Nungol ndï} \textbf{\textit{amblat}} \textit{nay.}\\
\gll    nungol  ndï  \textbf{ambla}=tï    na-i\\
    child  \textsc{3pl}  \textsc{pl.refl}=take  \textsc{detr-}go.\textsc{pfv}\\
\glt `The children brought themselves [to go along].’ [ulwa014\_60:09]
\z

\ea%21
    \label{ex:pron:21}
          \textit{Una} \textbf{\textit{amblakolp}}.\\
\gll unan    \textbf{ambla}=kol-p\\
    1\textsc{pl.incl}  \textsc{pl.refl}=break-\textsc{pfv}\\
\glt `We have broken ourselves.’ [ulwa037\_31:10]
\z

\ea%22
    \label{ex:pron:22}
          \textit{Ay ndïnkap ndïn} \textbf{\textit{amblan}} \textit{up.}\\
\gll    ay    ndï=nïkï{}-p    ndï=n    \textbf{ambla}=n    u-p\\
    sago  3\textsc{pl}=dig{}-\textsc{pfv}  \textsc{3pl=obl}  \textsc{pl.refl=obl}  put-\textsc{pfv}\\
\glt `[They] made [packets of] sago and left them for themselves.’ [ulwa014\_49:21]
\z

There are indications that the \is{binding} \isi{binding domain} for \isi{anaphor}s (i.e., \isi{reflexive} pronouns) in Ulwa may be something greater than the clause -- that is, unlike in \ili{English}, it is possible for the \isi{antecedent} of a \isi{reflexive} \isi{pronoun} in Ulwa to be located in a so-called higher clause. All known examples of this occur when the \isi{matrix clause} (containing the \isi{antecedent}) introduces the \isi{embedded clause} (containing the \isi{reflexive} \isi{pronoun}) by means of a verb of speaking (or thinking).\footnote{Thus, Ulwa may be said to exhibit a sort \isi{logophoricity}, since these \isi{reflexive}s must be bound by an \isi{antecedent} whose \isi{speech} (or thought) is being reported. It should be noted, however, that there is no special \isi{logophoric pronoun} in Ulwa.} Examples \REF{ex:pron:23} and \REF{ex:pron:24} illustrate the use of \isi{reflexive} pronouns with \isi{antecedent}s in a “higher” clause.

\ea%23
    \label{ex:pron:23}
          \textit{Wangasa Wore ngala ini tï} \textbf{\textit{ambïnanda}} \textit{nate.}\\
\gll    Wangasa\textsubscript{i}  Wore  ngala    ini    tï    \textbf{ambï}\textsubscript{i}=na-nda na-ta-e\\
    [name]    [place]  \textsc{pl.prox}  ground  take   \textsc{sg.refl}=give-\textsc{irr}    \textsc{detr}{}-say-\textsc{dep}\\
\glt `Wangasa says that these Wore [people] will give him land.’ (Literally ‘… will give himself land’) [ulwa014\_21:49]
\z

\ea%24
    \label{ex:pron:24}
          \textbf{\textit{Ambïwalinda}} \textbf{\textit{ambul}} \textit{undate nakap.}\\
\gll    \textbf{ambï}=wali-nda  \textbf{ambï}=ul    unda-t-e    na-kï{}-p\\
    \textsc{sg.refl}=hit-\textsc{irr}  \textsc{sg.refl}=with  go\textsc{{}-spec-dep}  \textsc{detr-}say-\textsc{pfv}\\
\glt `[He] thought that [the crocodile] might go around with him to kill him.’ [ulwa035\_02:29]
\z

\is{anaphor}
\is{long-distance anaphoric reference}

Example \REF{ex:pron:25} illustrates this long-distance \is{logophoricity} (or logophoric) \isi{anaphoric reference} with a \isi{reflexive} pronominal modifier, here used as a bare \isi{possessor}, without the form \textit{{}-nji} ‘\textsc{poss}’ (\sectref{sec:6.2}).

\ea%25
    \label{ex:pron:25}
          \textit{Kwa mï man} \textbf{\textit{ambaweta}} \textit{kap.}\\
\gll    kwa  mï\textsubscript{i}      ma=n      \textbf{ambï}\textsubscript{i}=aweta     kï{}-p\\
    one    3\textsc{sg.subj}  3\textsc{sg.obj=obl}  \textsc{sg.refl}=friend  say-\textsc{pfv}\\
\glt `Someone said that it was his friend.’ [ulwa020\_00:39]
\z

Also to be mentioned here is an interesting \isi{idiom}atic use of the \isi{reflexive} pronominal \isi{object marker} \textit{ambla=} ‘\textsc{pl.refl}’. When used with the verb \textit{asa-} {\textasciitilde} \textit{wali-} ‘hit, kill’, this marker does not necessarily have a \isi{reflexive} (or reciprocal) sense, but rather gives the entire verb the \isi{intransitive} meaning ‘fight’ (as in battle). The object of fighting (the enemy) can be marked in a \isi{postpositional phrase} with the \isi{postposition} \textit{ul} ‘with’. This can cause ambiguity not unlike what often occurs in \ili{English}, since this same \isi{postpositional phrase} can mark either enemies or allies.\footnote{Cf. \ili{English} sentences such as \textit{The English fought with the French}.} Sentences \REF{ex:pron:26}, \REF{ex:pron:27}, and \REF{ex:pron:28} exemplify this use of the verb \textit{asa-} {\textasciitilde} \textit{wali-} ‘hit, kill’ with the \isi{reflexive} pronominal \isi{object marker} \textit{ambla=} ‘\textsc{pl.refl}’.

\ea%26
    \label{ex:pron:26}
          \textit{Ndiya lop ndiya wa lop ndul} \textbf{\textit{amblasap}}.\\
\gll ndï=iya    lo-p  ndï=iya    wa    lo-p  ndï=ul     \textbf{ambla}=asa-p\\
    \textsc{3pl}=toward  go-\textsc{pfv}  3\textsc{pl=}toward  village  go-\textsc{pfv}  3\textsc{pl}=with    \textsc{pl.refl}=hit-\textsc{pfv}\\
\glt `[They] went to them, went to them in the village, and fought with them [as enemies].’ [ulwa034\_00:27]
\z

\ea%27
    \label{ex:pron:27}
          \textit{Unan ndiya ma ndul \textbf{amblawalinda}!}\\
\gll    unan    ndï=iya    ma  ndï=ul    \textbf{ambla}=wali-nda\\
    1\textsc{pl.incl}  3\textsc{pl}=toward  go  3\textsc{pl=}with  \textsc{pl.refl}=hit-\textsc{irr}\\
\glt `Let’s go to them and fight with them [as allies]!’ [ulwa002\_05:37]
\z

\ea%28
    \label{ex:pron:28}
          \textit{Ndul ndul} \textbf{\textit{amblasap}}.\\
\gll ndï=ul    ndï=ul    \textbf{ambla}=asa-p\\
    3\textsc{pl}=with  3\textsc{pl}=with  \textsc{pl.refl}=hit-\textsc{pfv}\\
\glt `With them [= our allies] we fought with them [= our enemies].’ [ulwa002\_03:13]
\z

The \isi{dual} and \isi{plural} \isi{reflexive} pronouns may, alternatively, convey a reciprocal sense (i.e., ‘each other’, ‘one another’). There may thus arise ambiguity in meaning, typically clarified through context. For example, sentence \REF{ex:pron:19} could be interpreted either as having a \isi{reflexive} sense or as having a reciprocal sense (i.e., either ‘you two know each other’ or ‘you two know yourselves’). Examples of reciprocal meaning are given in sentences \REF{ex:pron:29} through \REF{ex:pron:34}.

\is{pronoun|)}
\is{reciprocal pronoun|)}
\is{reflexive pronoun|)}
\is{reflexive pronoun|(}
\is{reciprocal pronoun|(}
\is{pronoun|(}

\ea%29
    \label{ex:pron:29}
          \textit{Kolpe Womel min} \textbf{\textit{ambinasap}}.\\
\gll Kolpe  Womel  min  \textbf{ambin}=asa-p\\
    [name]  [name]  3\textsc{du}  \textsc{du.refl=}hit-\textsc{pfv}\\
\glt `Kolpe and Womel fought each other.’ [elicited]
\z

\ea%30
    \label{ex:pron:30}
          \textit{Nguna} \textbf{\textit{ambinlu}} \textit{ndïtana.}\\
\gll    ngunan  \textbf{ambin}=lu    ndï=ta-na\\
    1\textsc{du.incl}  \textsc{du.refl=}with  3\textsc{pl}=say-\textsc{irr}\\
\glt `We will tell them [= stories] with each other.’ [ulwa033\_02:36]
\z

\ea%31
    \label{ex:pron:31}
          \textit{Ngan manap} \textbf{\textit{ambinlu}} \textit{une.}\\
\gll    ngan    ma=nap    \textbf{ambin}=lu     uni-e\\
    1\textsc{du.excl}  3\textsc{sg.obj}=for  \textsc{du.refl}=with  shout-\textsc{ipfv}\\
\glt `We argued with each other over her.’ [ulwa032\_17:31]
\z

\ea%32
    \label{ex:pron:32}
          \textit{Wopa} \textbf{\textit{amblol}} \textit{malanda mane.}\\
\gll    wopa  \textbf{ambla}=ul    ma=la-nda      ma-n-e\\
    all    \textsc{pl.refl}=with  3\textsc{sg.obj}=eat-\textsc{irr}  go-\textsc{ipfv-dep}\\
\glt `All are going to eat it with one another.’ [ulwa014\_65:16]
\z

\ea%33
    \label{ex:pron:33}
          \textit{Mundu ndata ndï na} \textbf{\textit{amblakap}}\textit{: …}\\
\gll    mundu  ndï=at-ta    ndï  na    \textbf{ambla}=kï-p\\
    hunger  3\textsc{pl}=hit-\textsc{cond}  \textsc{3pl}  talk  \textsc{pl.refl}=say-\textsc{pfv}\\
\glt `And when [they] got hungry, they said to one another other: …’ [ulwa018\_00:30]
\z

\ea%34
    \label{ex:pron:34}
          \textit{An ambi nape an lïmndï} \textbf{\textit{amblala}}.\\
\gll an      ambi  na-p-e      an      lïmndï  \textbf{ambla}=ala\\
    1\textsc{pl.excl}  big    \textsc{detr-}be\textsc{{}-dep} 1\textsc{pl.excl}  eye    \textsc{pl.refl}=see\\
\glt `When we had gotten big, we looked at one another.’ [ulwa013\_10:25]
\z

Sometimes a \isi{personal pronoun} occurs where a \isi{reflexive}/\isi{reciprocal} \isi{pronoun} may otherwise be expected. It is unclear whether this is a permissible variation in \isi{pronoun} use or an indication of \isi{grammatical attrition}. It is common with the verb \textit{ala-} ‘see’, as in \REF{ex:pron:35} and \REF{ex:pron:36}.

\is{pronoun|)}
\is{reciprocal pronoun|)}
\is{reflexive pronoun|)}

\ea%35
    \label{ex:pron:35}
          \textit{An lïmndï} \textbf{\textit{anala}}.\\
\gll an      lïmndï  \textbf{an}=ala\\
    1\textsc{pl.excl}  eye    \textsc{1pl.excl}=see\\
\glt `We saw ourselves.’ [ulwa013\_04:15]
\z

\ea%36
    \label{ex:pron:36}
          \textit{Olsem nï lïmndï} \textbf{\textit{nala}}.\\
\gll olsem  nï    lïmndï  \textbf{nï}=ala\\
    thus  1\textsc{sg}  eye    \textsc{1sg}=see\\
\glt `I see myself like this.’ (i.e., ‘I view myself as a person from Manu.’) (\textit{olsem} = TP) [ulwa004\_03:03]
\z

\section{Indefinite pronouns}\label{sec:6.4}

\is{indefinite pronoun|(}
\is{pronoun|(}

\isi{Indefinite} referents can be denoted by the \isi{numeral}/\isi{interrogative word} \textit{kwa} ‘one; who?; someone’ when the referent is a human or by the \isi{phrase} \textit{nji kwa} ‘one thing, something’ when the referent is non-human, as illustrated by examples \REF{ex:pron:37} through \REF{ex:pron:40}.

\ea%37
    \label{ex:pron:37}
          \textbf{\textit{Kwa}} \textit{nip.}\\
\gll    \textbf{kwa}  ni-p\\
    one    die-\textsc{pfv}\\
\glt    (a) ‘Someone died.’\\
    (b) ‘Who died?’ [elicited]
\z

\ea%38
    \label{ex:pron:38}
          \textit{Nï} \textbf{\textit{kwa}} \textit{asap.}\\
\gll nï    \textbf{kwa}  asa-p\\
    \textsc{1sg}  one    hit-\textsc{pfv}\\
\glt `I killed someone.’ [elicited]
\z

\ea%39
    \label{ex:pron:39}
          \textit{\textbf{Nji kwa} liyu.}\\
\gll    \textbf{nji}    \textbf{kwa}  li-u\\
    thing  one    down-put\\
\glt `Something fell.’ [elicited]
\z

\ea%40
    \label{ex:pron:40}
         \textit{Nï lïmndï \textbf{nji kwa} ala}\\
\gll nï    lïmndï  \textbf{nji}    \textbf{kwa}  ala\\
    \textsc{1sg}  eye    thing  one    see\\
\glt `I saw something.’ [elicited]
\z

Sentence \REF{ex:pron:37}, if given the right \isi{intonation}, could be interpreted as a \isi{question}, as suggested by the second translation given. This is because the form [kwa] is also used as an \isi{interrogative pronoun} meaning ‘who?’ (\sectref{sec:6.5}).

  \isi{Dual} and \isi{plural} forms do not tend to be used for \isi{indefinite} pronominal referents, at least not on their own. For non-\isi{singular} \isi{indefinite} referents, however, the word \textit{kuma} ‘some’ may follow an NP, whether human \REF{ex:pron:41} or non-human \REF{ex:pron:42}.

\ea%41
    \label{ex:pron:41}
          \textit{Ankam} \textbf{\textit{kuma}} \textit{mbin.}\\
\gll ankam  \textbf{kuma}  mbï-i-n\\
    person  some  here-come-\textsc{pfv}\\
\glt `Some people came.’ [elicited]
\z

\ea%42
    \label{ex:pron:42}
         \textit{Ya} \textbf{\textit{kuma}} \textit{liyu.}\\
\gll ya      \textbf{kuma}  li-u\\
    coconut  some  down-put\\
\glt `Some coconuts fell.’ [elicited]
\z

It is possible for \isi{subject marker}s (\sectref{sec:7.1}) to follow \textit{kuma} ‘some’ \REF{ex:pron:43}. Although \isi{subject marker}s may also follow \textit{kwa} ‘someone’, this is less common. In this way, \textit{kwa} ‘someone’ seems to pattern with what are more likely true pronouns, whereas \textit{kuma} ‘some’ seems to pattern more with \isi{adjective}s.

\ea%43
    \label{ex:pron:43}
          \textit{Ya \textbf{kuma ndï} liyu.}\\
\gll    ya      \textbf{kuma}  \textbf{ndï}  li-u\\
    coconut  some  3\textsc{pl}  down-put\\
\glt `Some coconuts fell.’ [elicited]
\z

It should be noted as well that \isi{object marker}s (\sectref{sec:7.2}) can follow object NPs ending with \textit{kuma} ‘some’, as in \REF{ex:pron:44} and \REF{ex:pron:45}.

\ea%44
    \label{ex:pron:44}
          \textit{Nï lïmndï ankam \textbf{kuma ndala}}.\\
\gll nï    lïmndï  ankam  \textbf{kuma}  \textbf{ndï=}ala\\
    1\textsc{sg}  eye    person  some  3\textsc{pl}=see\\
\glt `I saw some people.’ [elicited]
\z

\ea%45
    \label{ex:pron:45}
         \textit{Nï ya} \textbf{\textit{kuma}} \textbf{\textit{ndamap}}.\\
\gll nï    ya      \textbf{kuma}  \textbf{ndï=}ama-p\\
    1\textsc{sg}  coconut  some  3\textsc{pl}=eat-\textsc{pfv}\\
\glt `I ate some coconuts.’ [elicited]
\z

The \isi{interrogative} form \textit{angos} ‘what?’ can also be used in \isi{negative}-\isi{polarity} sentences (e.g., with the \isi{negator} \textit{ango} ‘\textsc{neg}’) to mean something along the lines of ‘whatever, whatsoever, anything’, as in examples \REF{ex:pron:46} through \REF{ex:pron:49}.

\ea%46
    \label{ex:pron:46}
          \textit{Ango} \textbf{\textit{angos}} \textit{na iye.}\\
\gll ango  \textbf{angos}    na    i-e\\
    \textsc{neg}  what    talk  go.\textsc{pfv-dep}\\
\glt    ‘[They] came to no thought whatsoever.’ (i.e., they came without any particular purpose) [ulwa002\_01:26]
\z

\ea%47
    \label{ex:pron:47}
          \textit{Una ango} \textbf{\textit{angos}} \textit{wombïn ninda.}\\
\gll unan    ango  \textbf{angos}  wombïn=n  ni-nda\\
    1\textsc{pl.incl}  \textsc{neg}  what  work=\textsc{obl}  act-\textsc{irr}\\
\glt `We will not do [just] whatever [sort of] work.’ [ulwa030\_02:30]
\z

\ea%48
    \label{ex:pron:48}
          \textit{Ango} \textbf{\textit{angos}} \textit{na ndït.}\\
\gll    ango  \textbf{angos}  na    ndï=ta\\
    \textsc{neg}  what  talk  3\textsc{pl}=say\\
\glt `[She] didn’t say anything to them.’ [ulwa014†]
\z

\ea%49
    \label{ex:pron:49}
          \textit{Nï ango} \textbf{\textit{angos}} \textit{ame.}\\
\gll nï    ango  \textbf{angos}  ama-e\\
    1\textsc{sg}  \textsc{neg}  what  eat-\textsc{ipfv}\\
\glt `I’m not eating anything.’ [ulwa032\_27:12]
\z

When combined with \textit{nji} ‘thing’, \textit{angos} ‘what?’ can convey the sense ‘whatever’ in \isi{positive}-\isi{polarity} sentences \REF{ex:pron:50}.

\is{pronoun|)}
\is{indefinite pronoun|)}

\ea%50
    \label{ex:pron:50}
          \textit{\textbf{Angos nji} inata una lïmndï mandï ande.}\\
\gll \textbf{angos}  \textbf{nji}    i-na-ta        unan    lïmndï  ma=andï  ande\\
    what  thing  come-\textsc{irr-cond}  \textsc{1pl.incl}  eye    3\textsc{sg}=see  ok\\
\glt `Whatever may come, we would see it [and say:] “OK”.’ [ulwa037\_20:53]
\z

\section{Interrogative pronouns}\label{sec:6.5}

\is{interrogative pronoun|(}
\is{question|(}
\is{pronoun|(}

The forms of the \isi{indefinite pronoun}s \textit{kwa} ‘one, someone’ and \textit{kuma} ‘some’ (\sectref{sec:6.4}) are the same as the forms used in asking \isi{content question}s in Ulwa. The \isi{pronoun} \textit{kwa} ‘one, someone’ thus also means ‘who?’ (for a \isi{singular} human referent), as illustrated by example \REF{ex:pron:51}.

\ea%51
    \label{ex:pron:51}
          \textbf{\textit{Kwa}} \textit{(mï) lamndu ndasap?}\\
\gll \textbf{kwa}  (mï)    lamndu  ndï=asa-p\\
    one    (3\textsc{sg.subj)}  pig      \textsc{3pl=}hit-\textsc{pfv}\\
\glt `Who killed the pigs?’ [elicited]
\z

Similarly, the \isi{pronoun} \textit{kuma} ‘some’ conveys the means ‘who?’ when referring to 
 multiple human referents, as in \REF{ex:pron:52} and \REF{ex:pron:53}.

\ea%52
    \label{ex:pron:52}
          \textbf{\textit{Kuma}} \textit{(min) lamndu ndasap?}\\
\gll \textbf{kuma}  (min)  lamndu  ndï=asa-p\\
    some  (3\textsc{du)}  pig      \textsc{3pl=}hit-\textsc{pfv}\\
\glt `Who [= which (two) people] killed the pigs?’ [elicited]
\z

\newpage

\ea%53
    \label{ex:pron:53}
          \textbf{\textit{Kuma}} \textit{(ndï) lamndu ndasap?}\\
\gll \textbf{kuma}  (ndï)  lamndu  ndï=asa-p\\
    some  (3\textsc{du)}  pig      \textsc{3pl=}hit-\textsc{pfv}\\
\glt `Who [= which (three or more) people] killed the pigs?’ [elicited]
\z
 
For non-human referents, the \isi{question word} \textit{angos} ‘what?’ is used, as in \REF{ex:pron:54}, \REF{ex:pron:55}, and \REF{ex:pron:56}. For all \isi{interrogative} pronouns, \isi{subject marker}s are optional (as are \isi{object marker}s).

\ea%54
    \label{ex:pron:54}
          \textbf{\textit{Angos}} \textit{(mï) lamndu ndasap?}\\
\gll \textbf{angos}  (mï)    lamndu  ndï=asa-p\\
    what  (3\textsc{sg.subj)}  pig      \textsc{3pl=}hit-\textsc{pfv}\\
\glt `What killed the pigs?’ [elicited]
\z

\ea%55
    \label{ex:pron:55}
          \textbf{\textit{Angos}} \textit{(min) lamndu ndasap?}\\
\gll \textbf{angos}  (min)  lamndu  ndï=asa-p\\
    what  (3\textsc{du)}  pig      \textsc{3pl=}hit-\textsc{pfv}\\
\glt `What [= which (two) things] killed the pigs?’ [elicited]
\z

\ea%56
    \label{ex:pron:56}
          \textbf{\textit{Angos}} \textit{(ndï) lamndu ndasap?}\\
\gll    \textbf{angos}  (ndï)  lamndu  ndï=asa-p\\
    what  (3\textsc{pl)}  pig      \textsc{3pl=}hit-\textsc{pfv}\\
\glt `What [= which (three or more) things] killed the pigs?’ [elicited]
\z

\is{wh- question}

In \isi{content question}s (i.e., \textit{wh-} questions), the \textit{wh-} word remains \isi{in situ}; it is not preposed to the front of the clause as in \ili{English}. Accordingly, when the ‘who’ or ‘what’ being asked about is not the grammatical subject, but rather the object of a verb, then the \isi{interrogative} \isi{pronoun} occurs in the typical object position (i.e., immediately before the verb), as in examples \REF{ex:pron:57} through \REF{ex:pron:60}.

\ea%57
    \label{ex:pron:57}
         \textit{U lïmndï} \textbf{\textit{kwa}} \textit{mala?}\\
\gll u    lïmndï  \textbf{kwa}  ma=ala\\
    2\textsc{sg}  eye    one    \textsc{3sg.obj}=see\\
\glt `Whom did you see?’   [elicited]
\z

\ea%58
    \label{ex:pron:58}
          \textit{Yata mï} \textbf{\textit{kuma}} \textit{ndasap?}\\
\gll yata  mï \textbf{\textit{kuma}} ndï=asa-p\\
    man  \textsc{3sg.subj}  some  \textsc{3pl=}hit-\textsc{pfv}\\
\glt `Whom [= which people] did the man hit?’ [elicited]
\z

\ea%59
    \label{ex:pron:59}
          \textit{U} \textbf{\textit{angos}} \textit{matïn?}\\
\gll u    \textbf{angos}  ma=tï-n\\
    \textsc{2sg}  what  \textsc{3sg.obj=}take-\textsc{pfv}\\
\glt `What did you take?’ [elicited]
\z

\ea%60
    \label{ex:pron:60}
          \textit{Nungol mï} \textbf{\textit{angos}} \textit{minanglalop?}\\
\gll nungol  mï      \textbf{angos}  min=angla-lo-p\\
    child  3\textsc{sg.subj}  what  \textsc{3du}=await-go-\textsc{pfv}\\
\glt `What [= which two things] did the boy look for?’ [elicited]
\z

When preceding an NP, the \isi{interrogative word} \textit{ango} ‘which?’ (compare \textit{angos} ‘what?’) conveys the sense ‘which [NP]?’. There is no distinction made based on \isi{animacy} or \isi{number} or \isi{grammatical relation} (i.e., whether the questioned element is a subject or an object). Significantly, whereas modifiers of NPs such as \isi{adjective}s or \isi{determiner}s follow their associated NPs (\sectref{sec:9.1}), the modifying element \textit{ango} ‘which?’ precedes its NP. This could serve the functional means of differentiating between ‘which [NP]?’ and ‘[NP] \textsc{neg}’.\footnote{Cf. \textbf{\textit{ango} \textit{tïn}} \textit{mamap} ‘which dog ate it?’ vs. \textbf{\textit{tïn} \textit{ango}} \textit{mamap} ‘the dog did not eat it’.} Sentences \REF{ex:pron:61} through \REF{ex:pron:64} illustrate the use of \textit{ango} ‘which?’.

\ea%61
    \label{ex:pron:61}
          \textit{\textbf{Ango tïn} (mï) mïnda mamap?}\\
\gll \textbf{ango}  \textbf{tïn}    (mï)    mïnda    ma=ama-p\\
    which  dog  (\textsc{3sg.subj)}  banana    \textsc{3sg.obj}=eat-\textsc{pfv}\\
\glt `Which dog ate the banana?’ [elicited]
\z

\ea%62
    \label{ex:pron:62}
          \textit{U} \textbf{\textit{ango}} \textbf{\textit{mïnda}} \textit{(mï) mamap?}\\
\gll u    \textbf{ango}  \textbf{mïnda}  (mï)    ma=ama-p\\
    \textsc{2sg}  which  banana  (\textsc{3sg.subj)}  \textsc{3sg.obj}=eat-\textsc{pfv}\\
\glt `Which banana did you eat?’ [elicited]
\z

\ea%63
    \label{ex:pron:63}
         \textbf{\textit{Ango}} \textit{nungolke nïnji yot matïn?}\\
\gll \textbf{ango}  nungolke  nï-nji    yot      ma=tï-n\\
    which  child    \textsc{1sg-poss}  machete  \textsc{3sg.obj=}take-\textsc{pfv}\\
\glt `Which child took my machete?’ [elicited]
\z

\ea%64
    \label{ex:pron:64}
          \textit{U} \textbf{\textit{ango}} \textit{apa nditap?}\\
\gll u    \textbf{ango}  apa    ndï=ita-p\\
    \textsc{2sg}  which  house  \textsc{3pl=}build-\textsc{pfv}\\
\glt `Which houses did you build?’ [elicited]
\z

  The \isi{interrogative} \isi{pronoun} ‘whose?’ takes the form \textit{kwanji} ‘whose [\textsc{sg]}’ for \isi{singular} \isi{possessor}s and \textit{kumanji} ‘whose [\textsc{nsg]}’ for \isi{dual} or \isi{plural} \isi{possessor}s (no distinction is made here between the two), as may be seen in examples \REF{ex:pron:65}, \REF{ex:pron:66}, and \REF{ex:pron:67}. These forms are transparently derived from the words \textit{kwa} ‘one’ or \textit{kuma} ‘some’ plus \textit{nji} ‘thing’ (cf. \isi{possessive pronoun}s, \sectref{sec:6.2}).

\is{pronoun|)}
\is{question|)}
\is{interrogative pronoun|)}

\ea%65
    \label{ex:pron:65}
          \textbf{\textit{Kwanji}} \textit{nungol (mï) nïnji yot tïn?}\\
\gll \textbf{kwa-nji}  nungol  (mï)    nï-nji    yot      tï-n\\
    one-\textsc{poss}  child  (\textsc{3sg.subj)}  \textsc{1sg-poss}  machete  take-\textsc{pfv}\\
\glt `Whose child took my machete?’ [elicited]
\z

\ea%66
    \label{ex:pron:66}
          \textit{Anda} \textbf{\textit{kwanji}} \textit{mana?}\\
\gll anda    \textbf{kwa-nji}  mana\\
    \textsc{sg.dist}  one\textsc{{}-poss} spear\\
\glt `Whose spear is that?’ [elicited]
\z

\ea%67
    \label{ex:pron:67}
          \textit{U} \textbf{\textit{kumanji}} \textit{apa maytap?}\\
\gll u    \textbf{kuma{}-nji} apa    ma=ita-p\\
    \textsc{2sg}  some\textsc{{}-poss} house  \textsc{3sg.obj=}build-\textsc{pfv}\\
\glt `Whose [plural] house did you build?’ [elicited]
\z

\section{Intensive pronouns}\label{sec:6.6}

\is{intensive pronoun|(}
\is{pronoun|(}

There are two basic sets of \isi{intensive pronoun}s in Ulwa. The members of one paradigm stress the fact that the referent(s) alone is/are the subject (or object). The members of the other paradigm stress the fact that the referent(s) – out of a group of potential referents -- performed the action; these are called here \isi{partitive-intensive pronoun}s.

  The \isi{suffix} used to derive the set of (non-\isi{partitive}) \isi{intensive pronoun}s is \textit{-awa} ‘\textsc{int}’ (‘-self/-selves’). It may combine with any of the \isi{non-subject} pronominal (or \isi{demonstrative}) forms, generating the paradigm shown in \tabref{tab::6.5}. Throughout this grammar, \isi{intensive pronoun}s are treated \isi{orthographic}ally as individual words; in the \isi{morphological} glossing, however, they are broken into their composite morphemes.


\begin{table}
\caption{Intensive pronominal and demonstrative forms}
\label{tab::6.5}
\begin{tabularx}{.9\textwidth}{lQQQ}
\lsptoprule
& {\scshape sg} & {\scshape du} & {\scshape pl}\\
\midrule
1 & {\itshape nawa} & \textit{nganawa} [\textsc{excl}]

\textit{ngunanawa} [\textsc{incl}] & \textit{anawa} [\textsc{excl}]

\textit{unanawa} [\textsc{incl}]\\
2 & {\itshape wawa} & {\itshape ngunawa} & {\itshape unawa}\\
3 & {\itshape mawa} & {\itshape minawa} & {\itshape ndawa}\\
{\scshape refl} & {\itshape ambawa} & {\itshape ambinawa} & {\itshape amblawa}\\
{\scshape prox} & {\itshape ngawa} & {\itshape nginawa} & {\itshape ngalawa}\\
{\scshape dist} & {\itshape andawa} & {\itshape andinawa} & {\itshape alawa}\\
\lspbottomrule
\end{tabularx}
\end{table}
Sentences \REF{ex:pron:68}, \REF{ex:pron:70}, and \REF{ex:pron:71} illustrate the use of basic \isi{intensive pronoun}s to place emphasis on a subject.

\newpage

\ea%68
    \label{ex:pron:68}
        \textbf{\textit{Nawa}} \textit{lamndu masap.}\\
\gll \textbf{nï-awa}    lamndu  ma=asa-p\\
    1\textsc{sg-int}  pig      \textsc{3sg.obj=}hit-\textsc{pfv}\\
\glt `I myself killed the pig.’ [elicited]
\z

\ea%70
    \label{ex:pron:70}
         \textit{Kayta} \textbf{\textit{mawa}} \textit{lamndu masap.}\\
\gll Kayta  \textbf{ma-awa}    lamndu  ma=asa-p\\
    [name]  \textsc{3sg.obj}{}-\textsc{int}  pig      \textsc{3sg.obj}=hit-\textsc{pfv}\\
\glt `Kayta himself killed the pig.’ [elicited]
\z

\ea%71
    \label{ex:pron:71}
          \textbf{\textit{Wawa}} \textit{utam nduwanap.}\\
\gll \textbf{u-awa}    utam  ndï=wana-p\\
    \textsc{2sg-int}  yam  \textsc{3pl=}cook-\textsc{pfv}\\
\glt `You yourself cooked the yams.’ [elicited]
\z

The \isi{partitive-intensive} pronominal \isi{suffix} is \textit{-we} ‘\textsc{part.int}’ (‘-self/-selves [out of multiple]’). It may combine with any of the \isi{non-subject} pronominal (or \isi{demonstrative}) forms, generating the paradigm given in \tabref{tab::6.6}.

\begin{table}
\caption{Partitive-intensive pronouns and demonstratives}
\label{tab::6.6}
\begin{tabularx}{.9\textwidth}{lQQQ}
\lsptoprule
& {\scshape sg} & {\scshape du} & {\scshape pl}\\
\midrule
1 & {\itshape nuwe} & \textit{nganwe} [\textsc{excl}]

\textit{ngunanwe} [\textsc{incl}] & \textit{anwe} [\textsc{excl}]

\textit{unanwe} [\textsc{incl}]\\
2 & {\itshape uwe} & {\itshape ngunwe} & {\itshape unwe}\\
3 & {\itshape mawe} & {\itshape minwe} & {\itshape nduwe}\\
{\scshape refl} & {\itshape ambuwe} & {\itshape ambinwe} & {\itshape amblawe}\\
{\scshape prox} & {\itshape ngawe} & {\itshape nginwe} & {\itshape ngalawe}\\
{\scshape dist} & {\itshape andawe} & {\itshape andinwe} & {\itshape alawe}\\
\lspbottomrule
\end{tabularx}
\end{table}

\is{partitive-intensive pronoun}

Sentences \REF{ex:pron:72}, \REF{ex:pron:74}, and \REF{ex:pron:75} illustrate the use of \isi{partitive-intensive pronoun}s to emphasize sole participation of a referent (or group of referents).

\ea%72
    \label{ex:pron:72}
          \textbf{\textit{Nuwe}} \textit{lamndu masap.}\\
\gll \textbf{nï-we}      lamndu  ma=asa-p\\
    \textsc{1sg-part.int}  pig      \textsc{3sg.obj=}hit-\textsc{pfv}\\
\glt `I myself [in the group] killed the pig.’ [elicited]
\z

\newpage

\ea%74
    \label{ex:pron:74}
          \textbf{\textit{Nduwe}} \textit{i.}\\
\gll    \textbf{ndï{}-we}    i\\
    \textsc{3pl-part.int}  go.\textsc{pfv}\\
\glt `They themselves [out of a group] went.’ [elicited]
\z

\ea%75
    \label{ex:pron:75}
          \textit{Kolpe Kongos} \textbf{\textit{ambinwe}} \textit{lamndu ndasape nakap.}\\
\gll    Kolpe Kongos    \textbf{ambin-we}      lamndu  ndï=asa-p-e na-kï-p\\
    [name]  [name]  \textsc{du.refl-part.int}  pig      \textsc{3pl}=hit-\textsc{pfv-dep}    \textsc{detr-}say-\textsc{pfv}\\
\glt `Kolpe and Kongos said that they themselves [out of a group] killed the pigs.’ [elicited]
\z

Although both paradigms of \isi{intensive pronoun}s are written as sets of single lexemes, the composite morphemes of each putative word are quite clear and can, at times, occur separately, as in examples \REF{ex:pron:76}, \REF{ex:pron:77}, and \REF{ex:pron:78}, each of which contains both the \is{partitive-intensive pronoun} partitive-intensive pronominal form (of the 3\textsc{pl} \isi{pronoun}) and the basic \isi{intensive} form (as a separate morpheme, without any \isi{person} or \isi{number} marking).

\ea%76
    \label{ex:pron:76}
          \textit{\textbf{Nduwe awa} nïmal ngayte mo liyen.}\\
\gll    \textbf{ndï{}-we}      \textbf{awa}  nïmal  nga=ita-e        ma=u li-i-en\\
    3\textsc{pl-part.int}  \textsc{int}    river  \textsc{sg.prox}=build-\textsc{ipfv}  \textsc{3sg.obj=}from    down-go.\textsc{pfv-nmlz}\\
\glt `They themselves alone were the ones who built [along] this river, having come down along it.’ [ulwa002\_02:47]
\z

\newpage

\ea%77
    \label{ex:pron:77}
          \textit{Manji nji ngala \textbf{nduwe awa}}.\\
\gll ma-nji      nji    ngala    \textbf{ndï{}-we}    \textbf{awa}\\
    3\textsc{sg.obj-poss}  thing  \textsc{pl.prox}  3\textsc{pl-part.int}  \textsc{int}\\
\glt `These things are his.’ (Literally ‘His things are them indeed [out of the group].’) [ulwa014\_05:28]
\z

\ea%78
    \label{ex:pron:78}
          \textit{\textbf{Nduwe awa} man ne.}\\
\gll    \textbf{ndï{}-we}      \textbf{awa} ma=n      ni-e\\
    3\textsc{pl-part.int}  \textsc{int}    \textsc{3sg.obj=obl}  act-\textsc{ipfv}\\
\glt `They themselves do it.’ [ulwa032\_25:42]
\z

It is also possible for the form \textit{we} ‘alone’ to occur as a separate morpheme, \isi{phonological}ly distinct from the preceding word (even if that word is a \isi{pronoun}), as in \REF{ex:pron:79} and \REF{ex:pron:80}.

\is{pronoun|)}
\is{intensive pronoun|)}

\ea%79
    \label{ex:pron:79}
          \textit{Mangusuwa} \textbf{\textit{we}} \textit{i.}\\
\gll ma-ngusuwa  \textbf{we}    i\\
    3\textsc{sg.obj{}-}poor  alone  go.\textsc{pfv}\\
\glt `The poor thing went alone.’ [ulwa035\_01:19]
\z

\ea%80
    \label{ex:pron:80}
          \textit{Nï} \textbf{\textit{we}} \textit{alum ngol mbïka lowonda.}\\
\gll   nï    \textbf{we}    alum  nga=ul      mbï-ka    lo-wo-nda\\
    1\textsc{sg}  alone  child  \textsc{sg.prox}=with  here-thus  \textsc{irr}{}-sleep-\textsc{irr}\\
\glt `I alone will sleep here with this child.’ [ulwa011\_01:46]
\z

\section{Emphatic pronouns}\label{sec:6.7}

\is{emphatic pronoun|(}
\is{pronoun|(}

In addition to these \isi{intensive} pronominal forms, there is a set of what are here called \isi{emphatic} pronominal forms. While there may also exist a full paradigm for such forms in all \isi{person}s and \isi{number}s, both for \isi{personal pronoun}s and for \isi{demonstrative}s (see \sectref{sec:7.3}), only four forms are attested in texts \REF{ex:pron:80a}.

\ea%80a
    \label{ex:pron:80a}
            Attested emphatic pronouns and \isi{demonstrative}s\\
\begin{tabbing}
{(\textit{andanam})} \= {(‘he/she/it is the one’)}\kill
{\textit{mïnam}} \> {‘he/she/it is the one’}\\
{\textit{ndïnam}} \> {‘they are the ones’}\\
{\textit{ngam}} \> {‘this is it’}\\
{\textit{andanam}} \> {‘that is it’}
\end{tabbing}
 \z

Notably, these forms are based on the subject (and not object) forms of the pronouns (e.g., /mï-nam/ and not \textsuperscript{†}/ma-nam/). They all appear to contain the \isi{emphatic} \isi{suffix} \textit{-nam} ‘\textsc{emph}’. The form [ngam] seems to have undergone a \isi{phonological} reduction, assuming that it derives from \textit{nga} ‘\textsc{sg.prox}’ (‘this’) plus \textit{-nam} ‘\textsc{emph’}. The \isi{emphatic} pronominal forms are illustrated by examples \REF{ex:pron:81} through \REF{ex:pron:84}.

\ea%81
    \label{ex:pron:81}
          \textbf{\textit{Mïnam}} \textit{amun masal Dumngul nungol ngawatawe.}\\
\gll    mï-\textbf{nam}  amun  ma=si{}-al      Dumngul  nungol nga=wat-aw-e\\
    3\textsc{sg.subj-emph}  now  3\textsc{sg.obj}=push-\textsc{pfv}  [name]    child    \textsc{sg.prox}=atop-put.\textsc{ipfv-dep}\\
\glt `Now he’s the one -- [they] call Dumngul’s son after him.’ (Literally ‘are pushing it [= the name] onto this child [of] Dumngul’) [ulwa014\_45:50]
\z

\ea%82
    \label{ex:pron:82}
         \textit{Inap ul iyen} \textbf{\textit{ndïnam}}.\\
\gll ina-p    u=ul    i-en      ndï-\textbf{nam}\\
    get-\textsc{pfv}  2\textsc{sg}=with  go.\textsc{pfv-nmlz}  3\textsc{pl-emph}\\
\glt `They were the ones who bore [you] and went with [you].’ [ulwa014†]
\z

\ea%83
    \label{ex:pron:83}
          \textit{A \textbf{andanam}!}\\
\gll a  anda-\textbf{nam}\\
    ah  \textsc{sg.dist-emph}\\
\glt `Ah, that one!’ [ulwa001\_02:41]
\z

\ea%84
    \label{ex:pron:84}
          \textbf{\textit{Ngam}} \textit{u nïn lïmndï ngaka nase.}\\
\gll    nga-\textbf{nam}      u    nï=n    lïmndï  nga=ka nï=asa-e\\
    \textsc{sg.prox-emph}  2\textsc{sg}  1\textsc{sg=obl}  eye    \textsc{sg.prox=}in    1\textsc{sg}=hit-\textsc{dep}\\
\glt `This is it -- you shot me in my eye.’ [ulwa006\_08:08]
\z

The \isi{emphatic} marker \textit{-nam} ‘\textsc{emph’} perhaps also appears in the \isi{interjection} \textit{mawnam} ‘that’s it’ (\sectref{sec:8.3.3}).

\is{pronoun|)}
\is{emphatic pronoun|)}

\section{Topic-marker pronouns}\label{sec:6.8}

\is{topic|(}
\is{topic marker|(}
\is{topic-marker pronoun|(}
\is{pronoun|(}

There is another set of pronominal forms, which can be used to mark the topic of a sentence. They are formed by combining a \isi{pronoun}, whether personal (\sectref{sec:6.1}) or \isi{demonstrative} (\sectref{sec:7.3}), with the topic-marking form \textit{-ambi} ‘\textsc{top}’, which may derive from a strengthened form of the \isi{reflexive} marker \textit{ambï=} ‘\textsc{sg.refl}’ (cf. the \isi{adjective} \textit{ambi} ‘big’, which perhaps also derives from this \isi{intensive} form). Topic-marker forms are presented in \tabref{tab::6.7}.


\begin{table}
\caption{Topic-marker pronominal and demonstrative forms}
\label{tab::6.7}
\begin{tabularx}{\textwidth}{lQQQ}
\lsptoprule
& {\scshape sg} & {\scshape du} & {\scshape pl}\\
\midrule
1 & {\itshape nambi} & \textit{nganambi} [\textsc{excl}]

\textit{ngunanambi} [\textsc{incl}] & \textit{anambi} [\textsc{excl}]

\textit{unanambi} [\textsc{incl}]\\
2 & {\itshape wambi} & {\itshape ngunambi} & {\itshape unambi}\\
3 & {\itshape mambi} & {\itshape minambi} & {\itshape ndambi}\\
{\scshape prox} & {\itshape ngambi} & {\itshape nginambi} & {\itshape ngalambi}\\
{\scshape dist} & {\itshape andambi} & {\itshape andinambi} & {\itshape alambi}\\
\lspbottomrule
\end{tabularx}
\end{table}
These topic-marking forms can be used to contrast one referent from another or to introduce a new referent after, say, a pause in the discourse. Some of their functions are illustrated by sentences \REF{ex:pron:85} through \REF{ex:pron:90}.

\ea%85
    \label{ex:pron:85}
          \textbf{\textit{Ngunanambi}} \textit{ango lïmndï manji pamili ndale.}\\
\gll \textbf{ngunan-ambi}  ango  lïmndï  ma-nji      pamili  ndï=ala-e\\
    1\textsc{du.incl-top}  \textsc{neg}  eye    3\textsc{sg.obj-poss}  family  3\textsc{pl}=see-\textsc{dep}\\
\glt `As for us, we don’t see his family.’ (\textit{pamili} = TP \textit{famili}) [ulwa014\_19:20]
\z

\ea%86
    \label{ex:pron:86}
          \textit{Mï mïnjan} \textbf{\textit{nambi}} \textit{ango misimisi kalamp.}\\
\gll   mï      mïnja=n    \textbf{nï-ambi}  ango  misimisi kalam=p\\
    3\textsc{sg.subj}  speech=\textsc{obl}  1\textsc{sg-top}  \textsc{neg}  story    knowledge=\textsc{cop}\\
\glt `She said: “Me? I don’t know stories.”’ [ulwa014\_01:20]
\z

\ea%87
    \label{ex:pron:87}
          \textbf{\textit{Unanambi}} \textit{unanji wa ilum} \textbf{\textit{ngambi}} \textit{anma ndo.}\\
\gll    \textbf{unan-ambi}  unan-nji     wa    ilum  \textbf{nga-ambi}    anma anda{}=o\\
    1\textsc{pl.incl-top}  \textsc{1pl.incl-poss}  village  little  \textsc{sg.prox-top}  good    \textsc{sg.dist=voc}\\
\glt `But us? As for this little village of ours, it’s good.’ [ulwa037\_32:27]
\z

\ea%88
    \label{ex:pron:88}
          \textit{Nogat Nomnga} \textbf{\textit{mambi}} \textit{kalam anda.}\\
\gll    nogat  Nomnga  \textbf{ma-ambi}    kalam    anda\\
    no    [name]    3\textsc{sg.obj-top}  knowledge  \textsc{sg.dist}\\
\glt `No, Nomnga knows [how to hunt].’ (The speaker was asked whether she was referring to Nomnga as the person who does not know how to hunt.) (\textit{nogat} = TP) [ulwa014\_25:39]
\z

\ea%89
    \label{ex:pron:89}
          \textbf{\textit{Nambi}} \textit{mandïm ma Wopata ma mapïna.}\\
\gll    \textbf{nï-ambi}  ma=andïm  ma    Wopata  ma  ma=p-na\\
    1\textsc{sg-top}  3\textsc{sg.obj=}from  go  [place]    go  3\textsc{sg.obj}=be-\textsc{irr}\\
\glt `I for one will leave her behind and go and stay at Wopata.’ [ulwa032\_35:10]
\z

\ea%90
    \label{ex:pron:90}
          \textit{Tarambi} \textbf{\textit{mambi}} \textit{anmbï mbïpe.}\\
\gll Tarambi  \textbf{ma-ambi}    an-mbï    mbï-p-e\\
    [name]    3\textsc{sg.obj-top}  out-here  here-be\textsc{{}-ipfv}\\
\glt `As for Tarambi, he stays outside.’ [ulwa014†]
\z

The topic-marking pronominal forms are found almost exclusively in subject NPs. However, they appear as \isi{object-marker}s \isi{proclitic}s as part of \isi{irrealis} or \isi{imperative} expressions with the verb \textit{ka-} ‘let, leave, allow’, thereby creating the \isi{idiom}atic meaning ‘forget (about) it/forget (about) them!’. This use is illustrated in examples \REF{ex:pron:91} through \REF{ex:pron:94}.

\is{pronoun|)}
\is{topic-marker pronoun|)}
\is{topic marker|)}
\is{topic|)}

\ea%91
    \label{ex:pron:91}
          \textbf{\textit{Mambilakan}} \textit{nï nakamp.}\\
\gll  \textbf{ma-ambi}=la-ka-n      nï    na-kamb-p\\
    3\textsc{sg.obj-top=irr-}let-\textsc{imp}  1\textsc{sg}  \textsc{detr-}shun-\textsc{pfv}\\
\glt `Forget it; I’ve had enough.’ [ulwa032\_47:46]
\z

\ea%92
    \label{ex:pron:92}
          \textit{Makape i mambi \textbf{mambinalakata}!}\\
\gll    maka=p-e    i    ma-ambi \textbf{ma-ambi}=na-la-ka-ta\\
    thus=\textsc{cop-dep}  way  3\textsc{sg.obj-top}    3\textsc{sg.obj-top-detr-irr-}let-\textsc{cond}\\
\glt `As for behavior like that -- forget it!’ [ulwa037\_29:08]
\z

\ea%93
    \label{ex:pron:93}
          \textit{Nïnji uta la ko} \textbf{\textit{ndambilakata}} \textit{ndï nïn mapïn!}\\
\gll nï-nji    uta    ala      ko  \textbf{ndï-ambi}=la-ka-ta    ndï nï=n     ma=p-n\\
    1\textsc{sg-poss}  bird  \textsc{pl.dist}  just  \textsc{3pl-top=irr-}let\textsc{{}-cond}  \textsc{3pl}    1\textsc{sg=obl}  3\textsc{sg.obj}=be-\textsc{imp}\\
\glt `Those are my birds -- just let them be there with me!’ [ulwa037\_47:32]
\z

\ea%94
    \label{ex:pron:94}
          \textit{\textbf{Mambilakana}!}\\
\gll    \textbf{ma-ambi}=la-ka-na\\
    3\textsc{sg.obj-top}=\textsc{irr-}let-\textsc{irr}\\
\glt `Shocking!’ [ulwa001\_13:36]
\z

\section{Affective pronouns}\label{sec:6.9}

\is{affective pronoun|(}
\is{pronoun|(}

\is{commiserative pronoun}

Ulwa has a set of pronouns used to convey compassion toward a second \isi{person} or third \isi{person} referent. These \isi{affective} (or commiserative) forms are transparently derived from the set of \isi{personal pronoun}s plus the \isi{adjective} \textit{ngusuwa} ‘poor, pitiful’.\footnote{Notably, the 3\textsc{sg} \isi{affective pronoun} derives from the \isi{object-marker} pronominal form (i.e., from /ma=/), and not the subject pronominal form (i.e., not from \textsuperscript{†}/mï/). There is no \isi{phonological} difference between the 2\textsc{sg} and 2\textsc{pl} affective pronominal forms, as this historical difference has been neutralized by the place \isi{assimilation} and \isi{quasi-degemination} of the final /-n/ of \textit{un} ‘2\textsc{pl’} before the immediately following /ng-/.} All the forms may be optionally elongated by the ending [\nobreakdash-ta], which bears no clear \isi{semantic} connection to the \isi{conditional} \isi{suffix} of the same form (\sectref{sec:4.12}). The forms of the affective pronouns are shown in \tabref{tab::6.8}.


\begin{table}
\caption{Affective pronouns}
\label{tab::6.8}
\begin{tabularx}{.8\textwidth}{lQQQ}
\lsptoprule
& {\scshape sg} & {\scshape du} & {\scshape pl}\\
\midrule
2 & \textit{ungusuwa} {\textasciitilde} \textit{ungusuwata} & \textit{ngungusuwa} {\textasciitilde} \textit{ngungusuwata} & \textit{ungusuwa} {\textasciitilde} \textit{ungusuwata}\\
3 & \textit{mangusuwa} {\textasciitilde} \textit{mangusuwata} & \textit{mingusuwa} {\textasciitilde} \textit{mingusuwata} & \textit{ndïngusuwa} {\textasciitilde} \textit{ndïngusuwa}\\
\lspbottomrule
\end{tabularx}
\end{table}
As an \isi{adjective}, \textit{ngusuwa} ‘poor’ has the same distribution to be expected of any \is{attributive adjective} (attributive) \isi{adjective} in Ulwa (\sectref{sec:5.1}): it follows the nominal \isi{head} of the NP and may precede a \isi{determiner}. Its use as an \isi{adjective} is illustrated by sentences \REF{ex:pron:95} through \REF{ex:pron:99}.

\ea%95
    \label{ex:pron:95}
          \textit{Yawa} \textbf{\textit{ngusuwa}} \textit{nda ma unap mat iyap.}\\
\gll    yawa  \textbf{ngusuwa}  anda    ma   u=nap    ma=tï i-ap\\
    uncle  poor    \textsc{sg.dist}  go  2\textsc{sg}=for  3\textsc{sg.obj}=take    go.\textsc{pfv-pfv}\\
\glt `That poor uncle went and brought it for you.’ [ulwa014†]
\z

\ea%96
    \label{ex:pron:96}
          \textit{Moira numan} \textbf{\textit{ngusuwa}} \textit{mï ndala kuma nep.}\\
\gll    Moira  numan    \textbf{ngusuwa}  mï      ndï=ala  kuma ne-p\\
    [name]  husband  poor    3\textsc{sg.subj}  3\textsc{pl=}for  some    harvest-\textsc{pfv}\\
\glt `Moira’s poor husband harvested some [betel nut] for them.’ [ulwa014\_16:22]
\z

\newpage

\ea%97
    \label{ex:pron:97}
          \textit{Paulus} \textbf{\textit{ngusuwa}} \textit{mï numbu ma nan nït.}\\
\gll Paulus  \textbf{ngusuwa}  mï      numbu    ma      na=n nï=ta\\
    [name]  poor    3\textsc{sg.subj}  garamut  \textsc{3sg.obj}  talk\textsc{=obl}    \textsc{1sg=}say\\
\glt `Poor Paulus told me about the ironwood tree.’ [ulwa037\_39:57]
\z

\ea%98
    \label{ex:pron:98}
          \textit{Donna maka wombïn tï tawatïp} \textbf{\textit{ngusuwa}} \textit{lanane.}\\
\gll Donna  maka  wombïn  tï    tawatïp    \textbf{ngusuwa} ala=na-n-e\\
    [name]  thus  work    take   child    poor    \textsc{pl.dist}=give-\textsc{pfv-dep}\\
\glt `Donna, like, gave the work to those poor children.’ [ulwa037\_55:25]
\z

\ea%99
    \label{ex:pron:99}
          \textit{Ngunanji itom} \textbf{\textit{ngusuwa}} \textit{minwe ya ndïn awe.}\\
\gll ngunan-nji      itom  \textbf{ngusuwa}  min-we      ya ndï=n    aw-e\\
    1\textsc{du.incl-poss}  father  poor    3\textsc{du-part.int}   coconut    3\textsc{pl=obl}  put\textsc{{}-ipfv}\\
\glt `Only our two poor fathers used to plant coconuts.’ [ulwa014†]
\z

The \isi{adjective} seen in examples \REF{ex:pron:95} through \REF{ex:pron:99} may be contrasted with the pronominal forms, which never precede \isi{subject marker}s, \isi{object marker}s, or any other \isi{determiner}s belonging to the same \isi{phrase}. The pronominal forms are also capable of being expanded with the ending [-ta], which is never seen in the \isi{adjective} \textit{ngusuwa} ‘poor’. Sentences \REF{ex:pron:100} through \REF{ex:pron:107} illustrate the use of these affective pronouns.

\ea%100
    \label{ex:pron:100}
          \textbf{\textit{Mangusuwa}} \textit{ya ndïn num up.}\\
\gll    \textbf{ma-ngusuwa}  ya      ndï=n    num  u-p\\
    3\textsc{sg.obj-}poor  coconut  3\textsc{pl=obl}  canoe  put-\textsc{pfv}\\
\glt `The poor thing put coconuts in the canoe.’ [ulwa014†]
\z

\ea%101
    \label{ex:pron:101}
          \textbf{\textit{Mangusuwa}} \textit{mbïpe salïn nïsap.}\\
\gll    \textbf{ma-ngusuwa}  mbï-p-e    sal=ïn    nï=sa-p\\
    3\textsc{sg.obj-}poor  here-be-\textsc{dep}  tear=\textsc{obl}  1\textsc{sg}=cry-\textsc{pfv}\\
\glt `When the poor thing was here, [he] cried to me.’ [ulwa014†]
\z

\ea%102
    \label{ex:pron:102}
          \textit{\textbf{Mangusuwata} ngat iye.}\\
\gll    \textbf{ma-ngusuwata}  nga=tï      i-e\\
    3\textsc{sg.obj-}poor    \textsc{sg.prox}=take  go.\textsc{pfv-dep}\\
\glt `The poor thing brought this.’ [ulwa014†]
\z

\ea%103
    \label{ex:pron:103}
          \textbf{\textit{Ungusuwa}} \textit{mat ambul namana.}\\
\gll    \textbf{u-ngusuwa}  ma=tï    ambï=ul      na-ma-na\\
    2\textsc{sg-}poor    3\textsc{sg.obj}=take  \textsc{sg.refl}=with  \textsc{detr-}go-\textsc{irr}\\
\glt `You poor thing will bring it with yourself.’ [ulwa041\_00:45]
\z

\ea%104
    \label{ex:pron:104}
          \textbf{\textit{Ngungusuwa}} \textit{ango luwa u wambana ndït?}\\
\gll    \textbf{ngun-ngusuwa}  ango  luwa  u    wambana  ndï=tï\\
    \textsc{2du-}poor      which  place  from  fish    3\textsc{pl}=take\\
\glt `You two poor things, where did [you] get the fish?’ [ulwa036\_03:45]
\z

\ea%105
    \label{ex:pron:105}
          \textit{Mat \textbf{ungusuwata}!}\\
\gll    ma=ta      \textbf{un-ngusuwata}\\
    3\textsc{sg.obj}=say  2\textsc{pl{}-}poor\\
\glt `[I] said: “You poor things!”’ [ulwa014\_47:29]
\z

\ea%106
    \label{ex:pron:106}
          \textbf{\textit{Ndïngusuwa}} \textit{may we matïn mat mbi.}\\
\gll    \textbf{ndï-ngusuwa}  ma=i         we    ma=tï-n ma=tï      mbï-i\\
    3\textsc{pl-}poor    3\textsc{sg.obj}=go.\textsc{pfv}  sago  3\textsc{sg.obj}=take-\textsc{pfv}    3\textsc{sg.obj}=take  here-go.\textsc{pfv}\\
\glt `The poor things went there, got sago starch, and brought it here.’ [ulwa037\_62:09]
\z

\ea%107
    \label{ex:pron:107}
          \textit{\textbf{Ndïngusuwata} mbïpe matane wapen.}\\
\gll    \textbf{ndï-ngusuwata}  mbï-p-e    ma=ta-n-e wap-en\\
    3\textsc{pl-}poor      here-be\textsc{{}-dep} 3\textsc{sg.obj}=say-\textsc{ipfv-dep}    be.\textsc{pst-nmlz}\\
\glt `When the poor things were here, [they] used to talk about it.’ [ulwa037\_43:44]
\z

\is{pronoun|)}
\is{affective pronoun|)}