\chapter{Phonetics and phonology}\label{sec:2}

In this chapter I describe and analyze Ulwa’s \isi{phonetics} and \isi{phonology}. The basic \isi{phoneme inventory} of Ulwa consists of 19 segments, including 13 \isi{consonant}s and 6 \isi{vowel}s.

\section{Consonants}\label{sec:2.1}

\is{consonant|(}
\is{phonetics|(}
\is{phonology|(}

\tabref{tab:2.1} shows the 13 consonants of Ulwa, presented in the practical \isi{orthography}; where this \isi{orthography} differs from the conventions of the IPA, the IPA equivalent is also given (in parentheses). The form [r] is generally an \isi{allophone} of /l/, but it is the preferred pronunciation in some \isi{proper noun}s (\sectref{sec:1.4}).
\is{phonology|)}
\is{phonetics|)}
\is{consonant|)}

\begin{table}
\caption{Ulwa consonants (in practical orthography)}
\is{consonant}
\is{orthography}
\is{labial}
\is{alveolar}
\is{palatal}
\is{velar}
\label{tab:2.1}
\begin{tabular}{lllll}
\lsptoprule
& labial & alveolar & palatal & velar\\
\midrule
\isi{voiceless} \isi{stop}s & p & t &  & k\\
\isi{prenasalized} \isi{voiced} \isi{stop}s & mb (ᵐb) & nd (ⁿd) &  & ng (ᵑɡ)\\
\isi{prenasalized} \isi{voiced} \isi{affricate} &  &  & nj (ⁿdʒ) & \\
\isi{nasal}s & m & n &  & \\
\isi{liquid} &  & l, [r] &  & \\
\isi{fricative} &  & s &  & \\
\isi{glide}s & w &  & y (j) & \\
\lspbottomrule
\end{tabular}
\end{table}

\subsection{Voiceless stops /p, t, k/}\label{sec:2.1.1}

\is{voiceless|(}
\is{stop|(}
\is{plosive|(}
\is{consonant|(}
\is{phonetics|(}
\is{phonology|(}

There is a three-way place distinction among \isi{voiceless} stops in Ulwa: \isi{labial} /p/, \isi{coronal} /t/, and \isi{dorsal} /k/. These are all quite similar to their \ili{English} equivalents: the /p/ is \isi{bilabial}, like \ili{English} /p/; the /t/ is \isi{alveolar}, like \ili{English} /t/; and the /k/ is \isi{velar}, like \ili{English} /k/. They are all slightly \isi{aspirated}. The following sets of \isi{minimal pair}s illustrate contrasts among \isi{voiceless} stops: /p/ versus /t/ \REF{ex:phon:1}, /p/ versus /k/ \REF{ex:phon:2}, and /t/ versus /k/ \REF{ex:phon:3}.

\ea%1
    \label{ex:phon:1}
            /p/ versus /t/\\
\begin{tabbing}
{(\textit{wo\textbf{p}})} \= {(‘fish species’)} \= {(vs.)} \= {(\textit{wo\textbf{t}})} \= {(‘younger (sibling)’)}\kill
{\textit{\textbf{p}al}} \> {‘palm shoot’} \> {vs.} \> {\textit{\textbf{t}al}} \> {‘tail feather’}\\
{\textit{a\textbf{p}a}} \> {‘house’} \> {vs.} \> {\textit{a\textbf{t}a}} \> {‘up’}\\
{\textit{u\textbf{p}a}} \> {‘fish species’} \> {vs.} \> {\textit{u\textbf{t}a}} \> {‘bird’}\\
{\textit{wo\textbf{p}}} \> {‘sleep [\textsc{pfv]}’} \> {vs.} \> {\textit{wo\textbf{t}}} \> {‘younger (sibling)’}
\end{tabbing}
\z

\ea%2
    \label{ex:phon:2}
            /p/ versus /k/\\
\begin{tabbing}
{(\textit{\textbf{p}alam})} \= {(‘coconut frond’)} \= {(vs.)} \= {(\textit{\textbf{k}alam})} \= {(‘knowledge’)}\kill
{\textit{\textbf{p}alam}} \> {‘cane grass’} \> {vs.} \> {\textit{\textbf{k}alam}} \> {‘knowledge’}\\
{\textit{no\textbf{p}al}} \> {‘coconut frond’} \> {vs.} \> {\textit{no\textbf{k}al}} \> {‘beak’}\\
{\textit{nu\textbf{p}u}} \> {‘bottom’} \> {vs.} \> {\textit{nu\textbf{k}u}} \> {‘flatus’}
\end{tabbing}
\z
            
\ea%3
    \label{ex:phon:3}
            /t/ versus /k/\\
\begin{tabbing}
{(\textit{\textbf{t}ukul})} \= {(‘laughter’)} \= {(vs.)} \= {(\textit{\textbf{k}ukul})} \= {(‘type of basket’)}\kill
{\textit{\textbf{t}a}} \> {‘already’} \> {vs.} \> {\textit{\textbf{k}a}} \> {‘thus’}\\
{\textit{\textbf{t}ukul}} \> {‘fish trap’} \> {vs.} \> {\textit{\textbf{k}ukul}} \> {‘type of basket’}\\
{\textit{a\textbf{t}al}} \> {‘laughter’} \> {vs.} \> {\textit{a\textbf{k}al}} \> {‘ringworm’}
\end{tabbing}
\z

While /p/ and /t/ may appear in all word positions (that is, word-initially, word-medially, or word-finally), /k/ may not appear word-finally. The words in \REF{ex:phon:4} all begin with \isi{voiceless} stops.

\ea%4
    \label{ex:phon:4}
            Word-initial \isi{voiceless} stops\\
\begin{tabbing}
{(\textit{\textbf{k}uman})} \= {(‘mosquito-swatter’)}\kill
{\textit{\textbf{p}iya}} \> {‘banana species’}\\
{\textit{\textbf{p}ul}} \> {‘piece’}\\
{\textit{\textbf{t}embi}} \> {‘bad’}\\
{\textit{\textbf{t}ongan}} \> {‘mosquito-swatter’}\\
{\textit{\textbf{k}uman}} \> {‘large wildfowl’}\\
{\textit{\textbf{k}we}} \> {‘one’}
\end{tabbing}
\z

The words in \REF{ex:phon:5} all have \isi{voiceless} stops in medial position.

\ea%5
    \label{ex:phon:5}
            Word-medial \isi{voiceless} stops\\
\begin{tabbing}
{(\textit{ni\textbf{p}um})} \= {(‘insect species’)}\kill
{\textit{ma\textbf{p}u}} \> {‘fish species’}\\
{\textit{ni\textbf{p}um}} \> {‘\textit{kunai} grass’}\\
{\textit{awe\textbf{t}a}} \> {‘friend’}\\
{\textit{nï\textbf{t}e}} \> {‘\textit{kundu} drum’}\\
{\textit{lu\textbf{k}e}} \> {‘too’}\\
{\textit{ya\textbf{k}al}} \> {‘insect species’}
\end{tabbing}
\z

\newpage

The words in \REF{ex:phon:6} all end with \isi{voiceless} stops.

\is{phonology|)}
\is{phonetics|)}
\is{consonant|)}
\is{plosive|)}
\is{stop|)}
\is{voiceless|)}

\ea%6
    \label{ex:phon:6}
            Word-final \isi{voiceless} stops\\
\begin{tabbing}
{(\textit{moniwo\textbf{t}})} \= {(‘plant species’)}\kill
{\textit{i\textbf{p}}} \> {‘nose’}\\
{\textit{na\textbf{p}}} \> {‘arrow’}\\
{\textit{moniwo\textbf{t}}} \> {‘plant species’}\\
{\textit{nïkï\textbf{t}}} \> {‘lizard’}
\end{tabbing}
\z

\subsection{Prenasalized voiced stops /mb, nd, ng/}\label{sec:2.1.2}

\is{voiced|(}
\is{prenasalization|(}
\is{stop|(}
\is{plosive|(}
\is{consonant|(}
\is{phonetics|(}
\is{phonology|(}

There is a three-place \isi{prenasalized} \isi{voiced} \isi{stop} series, which corresponds in place of articulation to the set of \isi{voiceless} stops: \isi{bilabial} /ᵐb/, \isi{alveolar} /ⁿd/, and \isi{velar} /ᵑɡ/. In the practical \isi{orthography} used in this grammar, these are written as <mb>, <nd>, and <ng>, respectively. These stops are all \isi{prenasalized} -- that is, they are preceded by a \isi{homorganic} \isi{nasal}. \isi{Minimal pair}s between \isi{prenasalized} \isi{voiced} \isi{stop}s and their \isi{voiceless} equivalents are given in \REF{ex:phon:7}. \isi{Minimal pair}s between \isi{prenasalized} \isi{voiced} \isi{stop}s and their simple \isi{nasal} equivalents are given in \REF{ex:phon:8}.\footnote{The \isi{velar} \isi{nasal} component of /ng/, as seen in \textit{nga} ‘this’ \REF{ex:phon:8}, has no simple \isi{nasal} equivalent, as \textsuperscript{†}/ŋ/ is not a phoneme in Ulwa. It occurs only phonetically, in the \isi{prenasalized} \isi{voiced} \isi{velar} \isi{stop} and in the realization of /n/ when preceding /k/ (i.e., the \isi{nasal} \isi{consonant} assimilates in place of articulation). Thus, the third row of example \REF{ex:phon:8} actually contrasts [ᵑɡ] with [n].}

\is{assimilation}

\ea%7
    \label{ex:phon:7}
            Contrasts between \isi{prenasalized} \isi{voiced} \isi{stop}s and \isi{voiceless} \isi{stop}s\\
\begin{tabbing}
{(\textit{a\textbf{nd}ana})} \= {(‘men’s house’)} \= {(vs.)} \= {(\textit{a\textbf{t}ana})} \= {(‘older sister’)}\kill
{\textit{a\textbf{mb}a}} \> {‘men’s house’} \> {vs.} \> {\textit{a\textbf{p}a}} \> {‘house’}\\
{\textit{a\textbf{nd}ana}} \> {‘left’} \> {vs.} \> {\textit{a\textbf{t}ana}} \> {‘older sister’}\\
{\textit{\textbf{ng}a}} \> {‘this’} \> {vs.} \> {\textit{\textbf{k}a}} \> {‘thus’}
\end{tabbing}
\z

\ea%8
    \label{ex:phon:8}
            Contrasts between \isi{prenasalized} \isi{voiced} \isi{stop}s and simple \isi{nasal}s\\
\begin{tabbing}
{(\textit{\textbf{ng}a})} \= {(‘they’)} \= {(vs.)} \= {(\textit{\textbf{n}a})} \= {(‘he/she/it’)}\kill
{\textit{\textbf{mb}ï}} \> {‘here’} \> {vs.} \> {\textit{\textbf{m}ï}} \> {‘he/she/it’}\\
{\textit{\textbf{nd}ï}} \> {‘they’} \> {vs.} \> {\textit{\textbf{n}ï}} \> {‘I’}\\
{\textit{\textbf{ng}a}} \> {‘this’} \> {vs.} \> {\textit{\textbf{n}a}} \> {‘talk’}
\end{tabbing}
\z

There are several reasons for treating these complex articulations as single phonemes, rather than as sequences of \isi{nasal}-plus-\isi{stop}. First, no \isi{voiced} \isi{stop} ever occurs without a preceding \isi{nasal} (although a \isi{nasal} may appear without any adjacent \isi{stop}). Second, when asked to syllabify a word, native speakers do not place a \isi{syllable} boundary between a \isi{nasal} and a following \isi{voiced} \isi{stop}.\footnote{For example, \textit{umbenam} ‘morning’ is broken into [u.ᵐbe.nam], and not into \textsuperscript{†}[um.be.nam]. Note that, while CC \isi{onset}s are possible in Ulwa (\sectref{sec:2.3}), there are no known \isi{nasal}-plus-(heterorganic) \isi{stop} \isi{onset}s; therefore, the interpretation \textsuperscript{†}/u.mbe.nam/ is highly unlikely. It may also be noted that the syllabification of words with \isi{prenasalized} \isi{stop}s can be affected in language attrition, as I have noticed that children, when asked to syllabify Ulwa words, follow the \isi{phonotactics} of \ili{Tok Pisin}, producing forms such as [um.be.nam] for \textit{umbenam} ‘morning’.} Third, in \isi{loanword}s from languages that have a simple voiced-\isi{stop} series, these phonemes are very frequently realized in Ulwa as \isi{prenasalized} \isi{voiced} \isi{stop}s.\footnote{For example, the \ili{Tok Pisin} word \textit{nogat} ‘no’ is often pronounced as [no.ᵑɡat].} Finally, \isi{nasal} segments can precede \isi{voiceless} \isi{stop}s. In these instances, there are in fact two distinct segments, as seen in words illustrating the following CC sequences: /np/ \REF{ex:phon:9}, /nt/ \REF{ex:phon:10}, /nk/ \REF{ex:phon:11}, /mp/ \REF{ex:phon:12}, /mt/ \REF{ex:phon:13}, /mk/ \REF{ex:phon:14}.

\ea%9
    \label{ex:phon:9}
            /np/\\
\begin{tabbing}
{(\textit{wo\textbf{np}})} \= {(‘cut [\textsc{pfv]}’)}\kill
{\textit{i\textbf{np}u}} \> {‘elbow’}\\
{\textit{wo\textbf{np}}} \> {‘cut [\textsc{pfv]}’}
\end{tabbing}
\z

\ea%10
    \label{ex:phon:10}
          /nt/\\
\begin{tabbing}
{(\textit{we\textbf{nt}a})} \= {(‘cassowary bone’)}\kill
{\textit{i\textbf{nt}ïp}} \> {‘cassowary bone’}\\
{\textit{we\textbf{nt}a}} \> {‘bird species’}
\end{tabbing}
\z

\ea%11
    \label{ex:phon:11}
          /nk/\\
\begin{tabbing}
{(\textit{mï\textbf{nk}ïn})} \= {(‘grub species’)} \= {([mïŋkïn])}\kill
{\textit{a\textbf{nk}am}} \> {‘person’} \> {[aŋkam]}\\
{\textit{i\textbf{nk}aw}} \> {‘mountain’} \> {[iŋkaw]}\\
{\textit{mï\textbf{nk}ïn}} \> {‘grub species’} \> {[mïŋkïn]}
\end{tabbing}
 \z

\ea%12
    \label{ex:phon:12}
          /mp/\\
\begin{tabbing}
{(\textit{kala\textbf{mp}})} \= {(‘piece of wood’)}\kill
{\textit{i\textbf{mp}ul}} \> {‘piece of wood’}\\
{\textit{kala\textbf{mp}}} \> {‘know’}\footnotemark
\end{tabbing}
\z
\footnotetext{Literally ‘be knowledgeable’ or ‘have knowledge’.}

\ea%13
    \label{ex:phon:13}
          /mt/\\
\begin{tabbing}
{(\textit{nï\textbf{mt}u})} \= {(‘bird species’)}\kill
{\textit{le\textbf{mt}a}} \> {‘spade’}\\
{\textit{nï\textbf{mt}u}} \> {‘bird species’}
\end{tabbing}
\z

\ea%14
    \label{ex:phon:14}
          /mk/\\
\begin{tabbing}
{(\textit{ya\textbf{mk}we})} \= {(‘sago fried with banana and coconut’)}\kill
{\textit{ilu\textbf{mk}a}} \> {‘a little’}\\
{\textit{ya\textbf{mk}we}} \> {‘sago fried with banana and coconut’}
\end{tabbing}
\z

It should be noted that, except in very slow speech, the sequence /nk/ is realized as [ŋk], the \isi{nasal} \isi{assimilating} in place to the following \isi{velar} \isi{stop}, a typologically very common process. Interestingly, the sequence /np/, as found in \REF{ex:phon:9}, is not realized as \textsuperscript{†}[mp] -- that is, /n/ does not \is{assimilation} assimilate in place to the following \isi{bilabial} \isi{stop} /p/.

  Since it is possible for \isi{homorganic} \isi{nasal}s to precede \isi{voiceless} \isi{stop}s, it is thus also possible to find (pseudo-)\isi{minimal pair}s such as /mb/ versus /mp/, /nd/ versus /nt/, and /ng/ versus /nk/. It must be maintained, however, that the phonetic sequences [ᵐb], [ⁿd], and [ᵑɡ] are each monophonemic, whereas the sequences [mp], [nt], and [ŋk] each consist of two phonemes. There are not many attested examples of such putative \isi{minimal pair}s; however, the contrast between the single phoneme /ng/ and the \isi{consonant cluster} of /nk/ can be seen in a handful of words \REF{ex:phon:15}.

\ea%15
    \label{ex:phon:15}
          Contrasts between /ng/ and /nk/\\
\begin{tabbing}
{(\textit{ta\textbf{ng}am})} \= {(‘banana species’)} \= {(vs.)} \= {(\textit{mo\textbf{nk}in})} \= {(‘vegetable species’)}\kill
{\textit{a\textbf{ng}ïn}} \> {‘vine species’} \> {vs.} \> {\textit{a\textbf{nk}ïn}} \> {‘vegetable species’}\\
{\textit{mo\textbf{ng}i}} \> {‘banana species’} \> {vs.} \> {\textit{mo\textbf{nk}in}} \> {‘gray hair’}\\
{\textit{ta\textbf{ng}am}} \> {‘sprout’} \> {vs.} \> {\textit{a\textbf{nk}am}} \> {‘person’}\\
{\textit{tï\textbf{ng}ïn}} \> {‘many’} \> {vs.} \> {\textit{mï\textbf{nk}ïn}} \> {‘grub species’}
\end{tabbing}
\z

Prenasalized \isi{voiced} \isi{stop}s may occur word-initially or intervocalically, as in \REF{ex:phon:15}, but cannot close a \isi{syllable}, and thus never appear word-finally, at least not in surface forms. There is at least one lexeme, however, that seems underlyingly to end in a \isi{prenasalized} \isi{voiced} \isi{stop}. The underlying form of the \isi{verb stem} /kamb-/ ‘shun’ ends in a \isi{prenasalized} \isi{voiced} \isi{bilabial} \isi{stop} /mb/. When followed by \isi{vowel}-initial \isi{suffix}es, this \isi{verb stem} does not undergo any \isi{phonological} change \REF{ex:phon:15a}.

\ea%15a
    \label{ex:phon:15a}
\gll \normalfont[kambe]\\
/kamb-e/\\
\glt ‘shun [\textsc{ipfv}]’
  \z
  
When no (phonemic) \isi{vowel} follows, however, either: an \isi{epenthetic} [ï] is added to the root, as in the \isi{conditional} form of the word \REF{ex:phon:15b}, or the \isi{stop} gesture of the final phoneme /mb/ is lost, as in the \isi{perfective} form of the word \REF{ex:phon:15c}.

\ea%15b
    \label{ex:phon:15b}
\gll \normalfont[kambïta]\\
/kamb-ta/\\
\glt ‘shun [\textsc{cond}]’
  \z

\ea%15c
    \label{ex:phon:15c}
\gll \normalfont[kamp]\\
/kamb-p/\\
\glt ‘shun [\textsc{pfv}]’
  \z

While the change in \REF{ex:phon:15c} may seem to be conditioned by the following \isi{homorganic} /p/, it also occurs when the root \textit{kamb-} ‘shun’ appears in isolation (i.e., [kam]). These possible phonological strategies for dealing with the illicit word-final /mb/ are summarized in \REF{ex:phon:16}.

\ea%16
    \label{ex:phon:16}
          Possible realizations of an underlying word-final /mb/

\ea Ø → [ï] / mb \_ C [-\isi{labial}]

\ex /mb/ → [m] / \_ \ \{ $\genfrac{}{}{0pt}{}{\textrm{C [+\isi{labial}]}}{\textrm{\#}\>\>\>\>\>\>\>\>\>\>\>\>\>\>\>\>}$
    \z
\z

The change of /mb/ to [m] (especially in word-final surface forms) is particularly interesting, since it implies the splitting of a single segment (/ᵐb/) into a sequence of phonemes (/mp/), a morphophonemic change.

\is{phonology|)}
\is{phonetics|)}
\is{consonant|)}
\is{plosive|)}
\is{stop|)}
\is{prenasalization|)}
\is{voiced|)}

\subsection{The prenasalized voiced palato-alveolar affricate /nj/}\label{sec:2.1.3}

\is{prenasalization|(}
\is{affricate|(}
\is{palato-alveolar|(}
\is{voiced|(}
\is{consonant|(}
\is{phonetics|(}
\is{phonology|(}

There is one \isi{affricate} in Ulwa. This is the \isi{prenasalized} \isi{voiced} \isi{palato-alveolar} \isi{affricate} /ⁿdʒ/, which has no \isi{voiceless} \isi{affricate} counterpart (and no \isi{voiceless} \isi{fricative} counterpart \mbox{either)}. As with the three \isi{prenasalized} \isi{voiced} \isi{stop}s, the sole \isi{voiced} \isi{affricate} is analyzed here as a single phoneme (with multiple articulatory gestures), rather than as a sequence of \isi{nasal}-plus-\isi{affricate} (or \isi{nasal}-plus-\isi{stop}-plus-\isi{fricative}). In the practical \isi{orthography}, it is written as <nj>. It can occur word-initially \REF{ex:phon:17} or word-medially \REF{ex:phon:18}.

\ea%17
    \label{ex:phon:17}
          Word-initial \isi{affricate}\\
\begin{tabbing}
{(\textit{\textbf{nj}ukuta})} \= {(‘thing’)}\kill
{\textit{\textbf{nj}i}} \> {‘thing’}\\
{\textit{\textbf{nj}ukuta}} \> {‘small’}
\end{tabbing}
\z

\ea%18
    \label{ex:phon:18}
          Word-medial \isi{affricate}\\
\begin{tabbing}
{(\textit{tamba\textbf{nj}i})} \= {(‘bird species’)}\kill
{\textit{i\textbf{nj}i}} \> {‘innards’}\\
{\textit{la\textbf{nj}in}} \> {‘fish species’}\\
{\textit{mï\textbf{nj}a}} \> {‘speech’}\\
{\textit{tamba\textbf{nj}i}} \> {‘bird species’}
\end{tabbing}
\z

Like the \isi{prenasalized} \isi{voiced} \isi{stop}s, the \isi{prenasalized} \isi{voiced} \isi{affricate} does not occur word-finally.

  Since almost every instance of [nj] precedes a \isi{high vowel} (/i, u/), it could be argued that the \isi{affricate} is not a distinct phoneme, but rather a \isi{palatalized} \isi{allophone} of /nd/. There is evidence against this hypothesis, however, since there exist contrasts between these two segments \REF{ex:phon:19}.

\ea%19
    \label{ex:phon:19}
            Contrasts between /nj/ and /nd/\\
\begin{tabbing}
{(\textit{\textbf{nj}ukuta})} \= {(‘our [\textsc{excl}]’)} \= {(vs.)} \= {(\textit{\textbf{nd}ukumbu})} \= {(‘palm species’)}\kill
{\textit{a\textbf{nj}i}} \> {‘our [\textsc{excl}]’} \> {vs.} \> {\textit{a\textbf{nd}i}} \> {‘OK’}\\
{\textit{nï\textbf{nj}i}} \> {‘my’} \> {vs.} \> {\textit{nï\textbf{nd}iwe}} \> {‘sago species’}\\
{\textit{\textbf{nj}ukuta}} \> {‘small’} \> {vs.} \> {\textit{\textbf{nd}ukumbu}} \> {‘palm species’}
\end{tabbing}
\z

Indeed, there are a number of words in which /nd/ occurs before a \isi{high vowel} \REF{ex:phon:20}.

\ea%20
    \label{ex:phon:20}
          Words in which /nd/ occurs immediately before a \isi{high vowel}\\
\begin{tabbing}
{(\textit{to\textbf{ndi}way})} \= {(‘plant species’)} \= {(\textit{\textbf{ndundu}ma})} \= {(‘animal, food, hunger’)}\kill
{\textit{mï\textbf{ndi}t}} \> {‘yellow’} \> {\textit{lam\textbf{ndu}}} \> {‘pig’}\\
{\textit{mo\textbf{ndi}n}} \> {‘fruit species’} \> {\textit{mu\textbf{ndu}}} \> {‘animal, food, hunger’}\\
{\textit{to\textbf{ndi}way}} \> {‘plant species’} \> {\textit{\textbf{ndundu}ma}} \> {‘great-grandparent’}\\
{\textit{wo\textbf{ndi}}} \> {‘bandicoot’} \> {\textit{u\textbf{ndu}wan}} \> {‘head’}
\end{tabbing}
\z

An alternative hypothesis could be that [nj] is actually a \isi{palatalized} version of the \is{consonant cluster} cluster /ny/, since this cluster is found only before \isi{low vowel}s (/a/), as seen in \REF{ex:phon:21}.

\ea%21
    \label{ex:phon:21}
          Words with the cluster /ny/\\
\begin{tabbing}
{(\textit{yama\textbf{ny}awi})} \= {(‘bird of paradise’)}\kill
{\textit{ku\textbf{ny}a}} \> {‘yam species’}\\
{\textit{mi\textbf{ny}am}} \> {‘feces’}\\
{\textit{yama\textbf{ny}awi}} \> {‘bird of paradise’}
\end{tabbing}
\z

It should be noted, however, that /ny/ is a very uncommon surface form. The form /nj/, on the other hand, is somewhat more common. Also, it is possible that /n/ and /y/ in these examples fall across a morpheme boundary, or at least a \isi{syllable} boundary.\footnote{There is, however, at least one lexeme in which /ny/ seems to occur within a single \isi{syllable}: \textit{wotnya} ‘bird species’ Since the language does not permit CCC \isi{consonant cluster}s within a \isi{syllable}, this word must syllabify as [wot.nya]. This word, however, is almost certainly \isi{onomatopoetic}, since the bird is described as having the call \textit{wotnya wotnya}.}

  Also, despite its limited distribution, it is not altogether impossible for /nj/ to occur before a \isi{low vowel}. While the form \textit{lumnjap} ‘fish species’ is possibly a \isi{loanword} (\sectref{sec:1.5.6}), the form \textit{mïnja} ‘speech’, which -- despite perhaps having derived from an older from that contained the word \textit{nji} ‘thing’ -- is probably native to Ulwa; it is quite common in speech and not synchronically analyzable as being polymorphemic. Also, it is not unusual for the series [nja] to occur in rapid speech, as in [njala] for /nji ala/ ‘those things’. Therefore, it is most parsimonious to accept the existence of /nj/ as a phoneme, but one whose distribution is mostly limited to environments directly preceding \isi{high vowel}s.

\is{phonology|)}
\is{phonetics|)}
\is{consonant|)}
\is{voiced|)}
\is{palato-alveolar|)}
\is{affricate|)}
\is{prenasalization|)}



\subsection{Nasals /m, n/}\label{sec:2.1.4}

\is{nasal|(}
\is{consonant|(}
\is{voiced|(}
\is{phonetics|(}
\is{phonology|(}

There are two phonemic \isi{nasal} consonants, a \isi{bilabial} /m/ and an \isi{alveolar} /n/. Either \isi{nasal} may occur word-initially \REF{ex:phon:22}, word-medially \REF{ex:phon:23}, or word-finally \REF{ex:phon:24}.

\ea%22
    \label{ex:phon:22}
          Contrasts between word-initial nasals\\
\begin{tabbing}
{(\textit{\textbf{m}il})} \= {(‘sugarcane’)} \= {(vs.)} \= {(\textit{\textbf{n}il})} \= {(‘body hair’)}\kill
{\textit{\textbf{m}ï}} \> {‘he/she/it’} \> {vs.} \> {\textit{\textbf{n}ï}} \> {‘I’}\\
{\textit{\textbf{m}il}} \> {‘sugarcane’} \> {vs.} \> {\textit{\textbf{n}il}} \> {‘body hair’}
\end{tabbing}
\z

\ea%23
    \label{ex:phon:23}
          Contrasts between word-medial nasals\\
\begin{tabbing}
{(\textit{ma\textbf{m}a})} \= {(‘type of basket’)} \= {(vs.)} \= {(\textit{ma\textbf{n}a})} \= {(‘spear’)}\kill
{\textit{a\textbf{m}e}} \> {‘type of basket’} \> {vs.} \> {\textit{a\textbf{n}e}} \> {‘day’}\\
{\textit{ma\textbf{m}a}} \> {‘mouth’} \> {vs.} \> {\textit{ma\textbf{n}a}} \> {‘spear’}
\end{tabbing}
\z

\ea%24
    \label{ex:phon:24}
          Contrasts between word-final nasals\\
\begin{tabbing}
{(\textit{uta\textbf{m}})} \= {(‘neck’)} \= {(vs.)} \= {(\textit{uta\textbf{n}})} \= {(‘you [\textsc{pl}]’)}\kill
{\textit{u\textbf{m}}} \> {‘neck’} \> {vs.} \> {\textit{u\textbf{n}}} \> {‘you [\textsc{pl}]’}\\
{\textit{uta\textbf{m}}} \> {‘yam’} \> {vs.} \> {\textit{uta\textbf{n}}} \> {‘cough’}
\end{tabbing}
 \z

The two nasals can also occur in sequence together, either as /mn/ or as /nm/, although the former is not especially common. Whenever these sequences do occur, the set of two nasals are always split by a \isi{syllable} boundary \REF{ex:phon:25}.

\ea%25
    \label{ex:phon:25}
          Sequences of two heterorganic nasals
          
    \ea  {[nm]}
\begin{tabbing}
{(\textit{nu\textbf{mn}ata})} \= {(‘earthquake’)} \= {([num.na.ta])}\kill
{\textit{a\textbf{nm}a}} \> {‘good’} \> {[an.ma]}\\
{\textit{wo\textbf{nm}i}} \> {‘hair’} \> {[won.mi]}\\
{\textit{a\textbf{nm}oka}} \> {‘snake’} \> {[an.mo.ka]}
\end{tabbing}

    \ex  {[mn]}
\begin{tabbing}
{(\textit{nu\textbf{mn}ata})} \= {(‘earthquake’)} \= {([num.na.ta])}\kill
{\textit{na\textbf{mn}a}} \> {‘afraid’} \> {[nam.na]}\\
{\textit{nu\textbf{mn}ata}} \> {‘earthquake’} \> {[num.na.ta]}
\end{tabbing}
    \z
\z

Also, it may be noted that the \isi{alveolar} \isi{nasal} /n/ may immediately precede the \isi{prenasalized} \isi{labial} \isi{stop} /mb/ \REF{ex:phon:26}.

\ea%26
    \label{ex:phon:26}
          /n/ immediately followed by /mb/\\
\begin{tabbing}
{(\textit{ke\textbf{nmb}u})} \= {(‘betel pepper’)} \= {([ken.mbu])}\kill
{\textit{ke\textbf{nmb}u}} \> {‘heavy’} \> {[ken.mbu]}\\
{\textit{u\textbf{nmb}ï}} \> {‘buttocks’} \> {[un.mbï]}\\
{\textit{wa\textbf{nmb}i}} \> {‘betel pepper’} \> {[wan.mbi]}
\end{tabbing}
\z

Likewise, the \isi{labial} \isi{nasal} /m/ may immediately precede the \isi{prenasalized} \isi{alveolar} \isi{stop} /nd/ \REF{ex:phon:27}.

\ea%27
    \label{ex:phon:27}
          /m/ immediately followed by /nd/\\
\begin{tabbing}
{(\textit{la\textbf{mnd}u})} \= {(‘type of basket’)} \= {([lam.ndu])}\kill
{\textit{i\textbf{mnd}e}} \> {‘type of basket’} \> {[im.nde]}\\
{\textit{la\textbf{mnd}u}} \> {‘pig’} \> {[lam.ndu]}\\
{\textit{lï\textbf{mnd}ï}} \> {‘eye’} \> {[lïm.ndï]}
\end{tabbing}
\z

There is at least one known instance of the \isi{labial} \isi{nasal} /m/ immediately preceding the \isi{prenasalized} \isi{velar} \isi{stop} /ng/ \REF{ex:phon:28}.

\ea%28
    \label{ex:phon:28}
          /m/ immediately followed by /ng/\\
\begin{tabbing}
{(\textit{kïtï\textbf{mng}ïle})} \= {(‘banana species’)} \= {([kï.tïm.ngï.le])}\kill
{\textit{kïtï\textbf{mng}ïle}} \> {‘banana species’} \> {[kï.tïm.ngï.le]}
\end{tabbing}
\z

There are no known instances of the \isi{alveolar} \isi{nasal} /n/ immediately preceding the \isi{prenasalized} \isi{velar} \isi{stop} /ng/. If these ever occur underlyingly, such as across a morpheme boundary, then the \isi{alveolar} /n/ would \is{assimilation} assimilate in place to the \isi{prenasalized} \isi{velar} \isi{stop} and then delete.

  Indeed, it is not possible for a \isi{nasal} immediately to precede a \isi{homorganic} \isi{voiced} \isi{stop} or \isi{affricate}. The phonetic realization of such a series would theoretically include an extra-long \isi{nasal} articulation. These do not occur phonetically in any word. When an underlying \isi{nasal} is immediately followed by a \isi{homorganic} \isi{voiced} \isi{stop} or \isi{affricate}, it \isi{deletes}, as in \REF{ex:phon:28a} (cf. \sectref{sec:4.1}).

  \ea%28a
    \label{ex:phon:28a}
\gll \normalfont[i\textbf{nd}a]\\
/i\textbf{n-nd}a/\\
\glt ‘get [\textsc{irr}]’
  \z
  
  In the set of \is{possessive pronoun} possessive pronominal forms, which all end in /-nji/ (\sectref{sec:6.2}), a similar sort of \isi{quasi-degemination} can be witnessed, since an immediately preceding /n/ \isi{deletes} (presumably after \isi{assimilating} slightly to the \isi{palato-alveolar} place of [nji]). Thus, there is a resulting \isi{homophony} between two of the possessive forms, as illustrated in \REF{ex:phon:28b}.

\newpage

\ea%ex:phon:28b
    \label{ex:phon:28b}
          \isi{Quasi-degemination} in possessive forms\\
\begin{tabbing}    
{(a.)} \= {(‘your [\textsc{sg}]’)} \= {(b.)} \= {(‘your [\textsc{pl}]’)}\kill
 {a.} \> {[unji]} \> {b.} \> {[u\textbf{nj}i]}\\
 { } \> {/u-nji/} \> { } \> {/u\textbf{n-nj}i/}\\
{ } \> {‘your [\textsc{sg}]’}   \> { } \> {‘your [\textsc{pl}]’}
\end{tabbing}
  \z

  The \isi{labial} \isi{nasal} /m/, on the other hand, may precede the \isi{palato-alveolar} \isi{affricate} /nj/ without \isi{assimilating}, as in the word \textit{inimnji} ‘water spirit’.

  There are no sequences of \isi{prenasalized} \isi{voiced} \isi{stop} (or \isi{affricate}) before a \isi{nasal}, whether \isi{homorganic} or heterorganic (i.e., \textsuperscript{†}/mbm, mbn, ndm, ndn, njm, njn/ are all unattested).

\is{phonology|)}
\is{phonetics|)}
\is{voiced|)}
\is{consonant|)}
\is{nasal|)}

\subsection{The liquid /l/}\label{sec:2.1.5}

\is{liquid|(}
\is{lateral|(}
\is{voiced|(}
\is{consonant|(}
\is{phonetics|(}
\is{phonology|(}

The single \isi{liquid} in Ulwa is usually realized as a \isi{voiced} \isi{alveolar} \isi{lateral} \isi{approximant} [l]. It may occur word-initially \REF{ex:phon:29}, intervocalically \REF{ex:phon:30}, and word-finally \REF{ex:phon:31}.

\ea%29
    \label{ex:phon:29}
          Word-initial liquids\\
\begin{tabbing}
{(\textit{\textbf{l}emetam})} \= {(‘tree species’)}\kill
{\textit{\textbf{l}amndu}} \> {‘pig’}\\
{\textit{\textbf{l}emetam}} \> {‘tree species’}\\
{\textit{\textbf{l}i}} \> {‘down’}\\
{\textit{\textbf{l}ïmndï}} \> {‘eye’}\\
{\textit{\textbf{l}uke}} \> {‘too’}
\end{tabbing}
\z

\ea%30
    \label{ex:phon:30}
          Word-medial liquids\\
\begin{tabbing}
{(\textit{wa\textbf{l}a})} \= {(‘vegetable species’)}\kill
{\textit{i\textbf{l}om}} \> {‘day’}\\
{\textit{mï\textbf{l}i}} \> {‘vegetable species’}\\
{\textit{u\textbf{l}et}} \> {‘dish’}\\
{\textit{wa\textbf{l}a}} \> {‘rat species’}
\end{tabbing}
\z

\ea%31
    \label{ex:phon:31}
          Word-final liquids\\
\begin{tabbing}
{(\textit{mïna\textbf{l}})} \= {(‘sugarcane’)}\kill
{\textit{mi\textbf{l}}} \> {‘sugarcane’}\\
{\textit{mïna\textbf{l}}} \> {‘taro’}\\
{\textit{wa\textbf{l}}} \> {‘ribs’}
\end{tabbing}
\z

The \isi{liquid} /l/ may also occur in \isi{consonant cluster}s, as in \REF{ex:phon:32} and \REF{ex:phon:32a}, where \isi{syllable} breaks are included to show that not all of these clusters occur within a single \isi{syllable}.

\newpage

\ea%32
    \label{ex:phon:32}
          Liquid as second member of a \isi{consonant cluster}: [Cl]\\
\begin{tabbing}
{(\textit{ma\textbf{tl}aka})} \= {(‘tree species’)} \= {([mat.la.ka])}\kill
{\textit{a\textbf{mbl}a}} \> {‘tooth’} \> {[a.mbla]}\\
{\textit{nï\textbf{pl}opa}} \> {‘flying fox’} \> {[nï.plo.pa]}\\
{\textit{ma\textbf{tl}aka}} \> {‘rat species’} \> {[mat.la.ka]}\\
{\textit{sa\textbf{kl}up}} \> {‘broom’} \> {[sak.lup]}\\
{\textit{a\textbf{ml}a}} \> {‘tree species’} \> {[am.la]}
\end{tabbing}
\z

\ea%32a
    \label{ex:phon:32a}
          Liquid as first member of a \isi{consonant cluster}: [lC]\\
\begin{tabbing}
{(\textit{ma\textbf{tl}aka})} \= {(‘tree species’)} \= {([mat.la.ka])}\kill
{\textit{a\textbf{lmb}a}} \> {‘hornbill’} \> {[al.mba]}\\
{\textit{wo\textbf{lm}u}} \> {‘nipple’} \> {[wol.mu]}\\
{\textit{mo\textbf{lp}an}} \> {‘tree spirit’} \> {[mol.pan]}\\
{\textit{wo\textbf{lk}a}} \> {‘again’} \> {[wol.ka]}\\
{\textit{a\textbf{ls}a}} \> {‘scorpion’} \> {[al.sa]}
\end{tabbing}
\z

Although the \isi{liquid} is generally realized as a \isi{lateral}, it can, for some speakers, in some environments, be realized as a \isi{rhotic}, either an \isi{alveolar} \isi{flap} [ɾ] or an \isi{alveolar} \isi{trill} [r]. However, the \isi{lateral} phone occurs more frequently overall and in more environments (the \isi{rhotic} variants do not occur word-finally, nor can they act as \isi{syllabic consonant}s, as described for [l] in the following paragraph). Therefore, because of its greater distribution, /l/ is chosen here to represent the basic \isi{liquid} phoneme.\footnote{As further justification for choosing /l/ over /r/ as the basic phoneme, it may be noted that many Ulwa speakers produce [l] for /r/ when speaking \ili{Tok Pisin} (i.e., [lawsim] for \ili{Tok Pisin} \textit{rausim} ‘remove’), but will rarely (if ever) produce [r] for /l/ (i.e., never \textsuperscript{†}[raykim] for \ili{Tok Pisin} \textit{laikim} ‘like’).}

\is{syllabic liquid}
\is{liquid}

Finally, \isi{lateral}s may be \isi{syllabic}. In words in which this is the case, there is almost always variation between a form with \isi{syllabic} [l̩] and one with [ï] preceding the \isi{lateral} [l]. It is proposed here that the \isi{vowel} /ï/ is part of the underlying form, and that those realizations with \isi{syllabic} [l̩] have undergone syncope of this unstressed high \isi{central vowel}. Examples of alternations with \isi{syllabic} [l̩] are given in \REF{ex:phon:33}.

\is{phonology|)}
\is{phonetics|)}
\is{consonant|)}
\is{voiced|)}
\is{lateral|)}
\is{liquid|)}

\ea%33
    \label{ex:phon:33}
          Syllabic [l̩]\\
\begin{tabbing}
\is{syllabic liquid}
{(\textit{tïmb\textbf{ïl}})} \= {(‘careful’)} \= {([tïmbïl {\textasciitilde} tïmb\textbf{l̩}])}\kill
{\textit{andïl}} \> {‘careful’} \> {[andïl {\textasciitilde} and\textbf{l̩}]}\\
{\textit{iw\textbf{ïl}}} \> {‘moon’} \> {[iwïl {\textasciitilde} iw\textbf{l̩}]}\\
{\textit{nïp\textbf{ïl}}} \> {‘vine’} \> {[nïpïl {\textasciitilde} nïp\textbf{l̩}]}\\
{\textit{tïmb\textbf{ïl}}} \> {‘fence’} \> {[tïmbïl {\textasciitilde} tïmb\textbf{l̩}]}
\end{tabbing}
\z


\subsection{The fricative /s/}\label{sec:2.1.6}

\is{fricative|(}
\is{lateral|(}
\is{sibilant|(}
\is{consonant|(}
\is{voiceless|(}
\is{phonetics|(}
\is{phonology|(}

The single \isi{fricative} in Ulwa is a \isi{voiceless} \isi{alveolar} \isi{sibilant} /s/. It is usually pronounced as an \isi{alveolar} \isi{fricative} [s], but may be realized as a \isi{palato-alveolar} \isi{fricative} [ʃ] before a \is{high vowel} high \isi{front vowel} /i/, as in [ʃiwi] for \textit{siwi} ‘grub species’, [ʃina] for \textit{sina} ‘knife’, or [wuʃim] for \textit{wusim} ‘crocodile’. \isi{Palatalization} is an optional rule – that is, for speakers who have this rule, there is \isi{free variation} among the forms they use \REF{ex:phon:34}.

\ea%34
    \label{ex:phon:34}
          Optional \isi{palatalization} of /s/\\
    /s/ → ([ʃ]) / \_ i (optional)
\z

The \isi{voiceless} \isi{alveolar} \isi{fricative} /s/ may occur word-initially, as in \REF{ex:phon:35}, or word-medially, as in \REF{ex:phon:36}.

\ea%35
    \label{ex:phon:35}
          Word-initial /s/\\
\begin{tabbing}
{(\textit{\textbf{s}imïnda})} \= {(‘banana species’)}\kill
{\textit{\textbf{s}awi}} \> {‘saliva’}\\
{\textit{\textbf{s}ikal}} \> {‘fly species’}\\
{\textit{\textbf{s}imïnda}} \> {‘banana species’}\\
{\textit{\textbf{s}okoy}} \> {‘tobacco’}
\end{tabbing}
\z

\ea%36
    \label{ex:phon:36}
          Word-medial /s/\\
\begin{tabbing}
{(\textit{yangu\textbf{s}ole})} \= {(‘plant species’)}\kill
{\textit{a\textbf{s}i}} \> {‘grass’}\\
{\textit{i\textbf{s}i}} \> {‘salt’}\\
{\textit{mi\textbf{s}am}} \> {‘brain’}\\
{\textit{noko\textbf{s}am}} \> {‘tree species’}\\
{\textit{yangu\textbf{s}ole}} \> {‘plant species’}
\end{tabbing}
\z

The \isi{fricative} /s/ does not occur very frequently in word-final position. In fact, only one word with final /s/ has so far been found, \textit{angos} ‘what?’, whose form may be due to \isi{compound}ing (\sectref{sec:8.3.2}).\footnote{\isi{Verb stem}s (e.g., \textit{asa-} ‘hit’ and \textit{si-} ‘push’) may lose their final \isi{vowel}s (\sectref{sec:4.2}), thereby resulting in surface realizations of word-final [s].} The \isi{fricative} /s/ does not occur in \isi{consonant cluster}s.

\is{phonology|)}
\is{phonetics|)}
\is{voiceless|)}
\is{consonant|)}
\is{sibilant|)}
\is{lateral|)}
\is{fricative|)}

\subsection{Glides /w, y/}\label{sec:2.1.7}

\is{glide|(}
\is{semivowel|(}
\is{approximant|(}
\is{voiceless|(}
\is{consonant|(}
\is{phonetics|(}
\is{phonology|(}

There are two \isi{glide}s (or \isi{semivowel}s or \isi{approximant}s) in Ulwa, a \isi{labial-velar} /w/ and a \isi{palatal} /j/. In the practical \isi{orthography} used here, they are written as <w> and <y>, respectively. Whereas /w/ has a fairly wide distribution, /y/ is more restricted, mostly just occurring word-initially and only rarely word-medially. Examples of word-initial \isi{glide}s are given in \REF{ex:phon:37}.

\ea%37
    \label{ex:phon:37}
          Word-initial glides\\
\begin{tabbing}
{(\textit{\textbf{w}usim})} \= {(‘sago starch’)} \= {(\textit{\textbf{y}uname})} \= {(‘bird species’)}\kill
{\textit{\textbf{w}a}} \> {‘village’} \> {\textit{\textbf{y}a}} \> {‘coconut’}\\
{\textit{\textbf{w}i}} \> {‘name’} \> {\textit{\textbf{y}ïwa}} \> {‘mound’}\\
{\textit{\textbf{w}ol}} \> {‘breast’} \> {\textit{\textbf{y}ot}} \> {‘machete’}\\
{\textit{\textbf{w}usim}} \> {‘crocodile’} \> {\textit{\textbf{y}uname}} \> {‘bird species’}\\
{\textit{\textbf{w}e}} \> {‘sago starch’} \> {\textit{\textbf{y}eta}} \> {‘man’}
\end{tabbing}
\z

While there are a number of words that begin with underlying \isi{glide}s, there is also an optional rule among many speakers that generates word-initial \isi{glide} \isi{epenthesis} (i.e., \isi{prothesis}) in words that otherwise would not begin with \isi{glide}s. Thus, [w] may be inserted before /u/, and [y] (IPA [j]) may be inserted before /i/, producing forms such as [wulum] for /ulum/ ‘sago palm’ and [yip] for /ip/ ‘nose’ \REF{ex:phon:38}.

\ea%38
    \label{ex:phon:38}
          Optional word-initial \isi{glide} \isi{epenthesis}\\
    Ø → ([-syl, -cons, {αback])} / \# \_ [+syl, +high, {αback] (optional)}
\z

The words in \REF{ex:phon:39} contain \isi{glide}s in medial position.

\ea%39
    \label{ex:phon:39}
          Word-medial \isi{glide}s\\
\begin{tabbing}
{(\textit{malalï\textbf{w}a})} \= {(‘snake species’)} \= {(\textit{ka\textbf{y}anmali})} \= {(‘lizard species’)}\kill
{\textit{a\textbf{w}al}} \> {‘afternoon’} \> {\textit{asi\textbf{y}a}} \> {‘string’}\\
{\textit{a\textbf{w}eta}} \> {‘friend’} \> {\textit{i\textbf{y}o}} \> {‘yes’}\\
{\textit{a\textbf{w}i}} \> {‘shoulder’} \> {\textit{ka\textbf{y}anmali}} \> {‘lizard species’}\\
{\textit{i\textbf{w}ïl}} \> {‘moon’} \> {\textit{nga\textbf{y}a}} \> {‘far’}\\
{\textit{malalï\textbf{w}a}} \> {‘snake species’} \> { } \> { }
\end{tabbing}
\z

While the distribution of /w/ is fairly broad (it seems to be permitted before or after any \isi{vowel}), /y/ is markedly more restricted. It occurs rarely in medial position, and when it does, the only permissible preceding \isi{vowel}s are /a/ and /i/ (and perhaps also /o/). The status of \isi{glide}s (or \isi{semivowel}s) in \isi{Papuan} languages poses a notoriously difficult problem, and the line between \isi{vowel}s and \isi{glide}s is often blurred, especially in languages (like Ulwa) that exhibit the high \isi{central vowel} [ï].\footnote{See \citet[50--52]{Foley1986} for issues in analyzing this phone in \isi{Papuan} languages.} Nevertheless, it is assumed here that /y/ exists as a phoneme in Ulwa (i.e., it is not, say, strictly underlyingly /i/), even though it has a more limited distribution than /w/, since otherwise it would be necessary to admit unlikely \isi{vowel} sequences into Ulwa’s canonical forms.

  A \isi{glide} may also be preceded by a \isi{consonant}. Although apparently any \isi{consonant} may occur before /w/, the only attested \isi{consonant} to appear before /y/ is the \isi{alveolar} \isi{nasal} /n/, as seen in \REF{ex:phon:40}.

\ea%40
    \label{ex:phon:40}
          \isi{Glide}s following other consonants\\
\begin{tabbing}
{(\textit{mï\textbf{nw}ata})} \= {(‘question’)} \= {(\textit{mi\textbf{ny}am})} \= {(‘yam species’)}\kill
{\textit{mï\textbf{nw}ata}} \> {‘wet’} \> {\textit{u\textbf{lw}a}} \> {‘nothing’}\\
{\textit{i\textbf{pw}at}} \> {‘front’} \> {\textit{mi\textbf{ny}am}} \> {‘feces’}\\
{\textit{a\textbf{tw}ana}} \> {‘question’} \> {\textit{ku\textbf{ny}a}} \> {‘yam species’}
\end{tabbing}
\z

As discussed in \sectref{sec:2.1.3}, /ny/ is a very uncommon surface form. It may (at least in some words) derive from an earlier \isi{palatal} \isi{nasal} *ɲ, which persists in Ulwa’s sister language \ili{Mwakai} (compare Ulwa \textit{minyam} ‘feces’ and \ili{Mwakai} \textit{ɲeri} ‘feces’).

In each of the words presented in \REF{ex:phon:40}, there is a \isi{syllable} break preceding the \isi{glide} (e.g., /mïn.wa.ta/, /min.yam/, etc.). It is also possible for the \isi{labial-velar} \isi{glide} /w/ to occur as the second member of a \isi{complex onset}, as in \REF{ex:phon:41}. The \isi{palatal} \isi{glide} /y/, on the other hand, does not occur as the second element in CC \isi{onset}s.

\ea%41
    \label{ex:phon:41}
          /w/ as the second member of \isi{complex onset}s\\
\begin{tabbing}
{(\textit{i\textbf{ngw}a})} \= {(‘opening’)} \= {([i.ngwa])}\kill
{\textit{\textbf{kw}a}} \> {‘one’} \> { }\\
{\textit{\textbf{mw}a}} \> {‘opening’} \> { }\\
{\textit{i\textbf{ngw}a}} \> {‘spider’} \> {[i.ngwa]}
\end{tabbing}
\z

Finally, word-final \isi{glide}s may be considered. The words in \REF{ex:phon:42} all end in either /w/ or /y/.

\ea%42
    \label{ex:phon:42}
          Word-final \isi{glide}s\\
\begin{tabbing}
{(\textit{wowa\textbf{w}})} \= {(‘lizard species’)} \= {(\textit{langa\textbf{y}})} \= {(‘insect species’)}\kill
{\textit{a\textbf{w}}} \> {‘betel nut’} \> {\textit{a\textbf{y}}} \> {‘jellied sago’}\\
{\textit{ka\textbf{w}}} \> {‘song’} \> {\textit{langa\textbf{y}}} \> {‘bird species’}\\
{\textit{ma\textbf{w}}} \> {‘correct’} \> {\textit{ma\textbf{y}}} \> {‘fish species’}\\
{\textit{nata\textbf{w}}} \> {‘lizard species’} \> {\textit{soko\textbf{y}}} \> {‘tobacco’}\\
{\textit{wopa\textbf{w}}} \> {‘ball’} \> {\textit{tomo\textbf{y}}} \> {‘insect species’}\\
{\textit{wowa\textbf{w}}} \> {‘fish scale’} \> {\textit{wa\textbf{y}}} \> {‘turtle’}
\end{tabbing}
\z

As illustrated by the set of the words in \REF{ex:phon:42}, word-final \isi{glide}s appear almost exclusively after the \isi{low vowel} /a/. Two examples of /-oy/ (\textit{sokoy} ‘tobacco’ and \textit{tomoy} ‘insect species’), however, run counter to this. There are no examples of /w/ following /o/, though, and these two examples of /-oy/ (two of only a few known to exist in the Ulwa \isi{lexicon}) may be problematic.\footnote{The pronunciation of \textit{sokoy} ‘tobacco’ varies greatly among speakers, many pronouncing the word as [sokay] or [soke]. This variation is perhaps due to the presence of many similar-sounding words for ‘tobacco’ in neighboring languages (\sectref{sec:1.5.6}). The origin of \textit{tomoy} ‘insect species’ is unclear, although it is known that several terms for flora and fauna in Ulwa have been \isi{borrow}ed from other languages. A third known word to end in /-oy/ is \textit{sinokoy} ‘crop’, which may be derived from \textit{sokoy} ‘tobacco’; finally, the \isi{adverb} \textit{woyambïn} ‘pointlessly, fruitlessly’, seems to have been derived from other words as well (see \sectref{sec:8.2.5} for a possible etymology).}

  In the \ili{Maruat-Dimiri-Yaul} \isi{dialect}, there also exists a \isi{labiodental} \isi{approximant} [ʋ], which seems to be an \isi{allophone} of /w/. It is perhaps \isi{borrow}ed from the influential neighboring language \ili{Mundukumo}, which has a phonemic \isi{voiced} \isi{labiodental} \isi{consonant} \citep[60]{McElvenny2006}.

\is{phonology|)}
\is{phonetics|)}
\is{consonant|)}
\is{voiceless|)}
\is{approximant|)}
\is{semivowel|)}
\is{glide|)}

\subsection{The glottal stop [ʔ]}\label{sec:2.1.8}

\is{glottal|(}
\is{glottal stop|(}
\is{consonant|(}
\is{voiceless|(}
\is{phonetics|(}
\is{phonology|(}

Finally, although there is no phonemic \isi{glottal} \isi{stop} [ʔ] in Ulwa, this sound appears quite often before \isi{vowel}s when they are utterance-initial, as is typologically common. The words in \REF{ex:phon:43} illustrate the phonetic realization of \isi{vowel}-initial words in unconnected speech.

\ea%43
    \label{ex:phon:43}
          Glottal \isi{stop} before utterance-initial \isi{vowel}s\\
\begin{tabbing}
{(\textit{\textbf{a}nma})} \= {(‘sago palm’)} \= {([\textbf{ʔ}anma])}\kill
{\textit{\textbf{a}nma}} \> {‘good’} \> {[\textbf{ʔ}anma]}\\
{\textit{\textbf{a}pa}} \> {‘house’} \> {[\textbf{ʔ}apa]}\\
{\textit{\textbf{i}m}} \> {‘tree’} \> {[\textbf{ʔ}im]}\\
{\textit{\textbf{i}tom}} \> {‘father’} \> {[\textbf{ʔ}itom]}\\
{\textit{\textbf{u}lum}} \> {‘sago palm’} \> {[\textbf{ʔ}ulum]}\\
{\textit{\textbf{u}tal}} \> {‘worm’} \> {[\textbf{ʔ}utal]}
\end{tabbing}
\z

As seen in \REF{ex:phon:43}, it is possible for the \isi{glottal} \isi{stop} to occur before /i/ or /u/, in addition to occurring before /a/. However, instead of producing initial sequences of [ʔi] or [ʔu], speakers often prefer sequences of [yi] or [wu], respectively -- that is, they employ word-initial \isi{epenthetic} \isi{glide}s (i.e., \isi{prothetic} \isi{glide}s) (\sectref{sec:2.1.7}).

\is{phonology|)}
\is{phonetics|)}
\is{voiceless|)}
\is{consonant|)}
\is{glottal stop|)}
\is{glottal|)}

\newpage

\section{Vowels}\label{sec:2.2}

\is{vowel|(}
\is{phonetics|(}
\is{phonology|(}

There are six vowels in Ulwa with relatively wide distribution, as well as two basic \isi{diphthong}s.

\is{phonology|)}
\is{phonetics|)}
\is{vowel|)}

\subsection{Monophthongs /a, e, i, o, u, ï/}\label{sec:2.2.1}

\is{vowel|(}
\is{monophthong|(}
\is{phonetics|(}
\is{phonology|(}

\tabref{tab:2.2} presents the six vowels of Ulwa. Most \isi{grapheme}s in the practical \isi{orthography} currently match their IPA equivalents. The main exception is <ï>, which represents IPA /ɨ/.\footnote{Another slight exception is <a>, which, as is common in linguistic literature, here represents a \is{low vowel} low \isi{central vowel}, and not a \is{low vowel} low \isi{front vowel}, as the IPA \isi{vowel} chart might suggest.} The seventh form in \tabref{tab:2.2}, <ae>, is included in brackets (and represents IPA /æ/): it is likely not a full-fledged phoneme in Ulwa.

\begin{table}
\caption{Ulwa vowels (in practical orthography)}
\is{vowel}
\is{front vowel}
\is{central vowel}
\is{back vowel}
\is{high vowel}
\is{mid vowel}
\is{low vowel}
\label{tab:2.2}
\begin{tabular}{llll}
\lsptoprule
& front & central & back\\
\midrule
high & i & ï (ɨ) & u\\
mid & e &  & o\\
low & [ae] (æ) & a & \\
\lspbottomrule
\end{tabular}
\end{table}


\is{high vowel}
\is{front vowel}
\is{mid vowel}
\is{front vowel}
\is{high vowel}
\is{back vowel}
\is{mid vowel}
\is{back vowel}
\is{low vowel}
\is{central vowel}

While the phonetic realizations of these vowels may occasionally approximate those of the cardinal vowels (especially in careful speech), they are more often pronounced somewhat more centralized. Thus, \isi{tense vowel}s may be \isi{lax}, especially in closed \isi{syllable}s. Accordingly, the high front \isi{unrounded} \isi{vowel} /i/ has the \isi{allophone} [ɪ]; the high back \isi{rounded} \isi{vowel} /u/ has the \isi{allophone} [ʊ]; the mid front \isi{unrounded} \isi{vowel} /e/ has the \isi{allophone} [ɛ]; and the mid back \isi{rounded} \isi{vowel} /o/ has the \isi{allophone} [ɔ]. Similarly, the  low central \isi{unrounded} \isi{vowel} /a/ may be raised to [ʌ].

Since the \isi{lax} pronunciations of /o/ and /a/ approach each other somewhere in the middle of the \isi{vowel} space, and since a preceding \isi{labial-velar} /w/ has the effect of \isi{rounding} an immediately following non-\isi{front vowel}, the phonetic realizations of /o/ and /a/ after /w/ are often identical \REF{ex:phon:44}.

\ea%44
    \label{ex:phon:44}
          Merger of /o/ and /a/ before /w/\\
    /o, a/ → [ɔ] / w \_
\z

Indeed, it is often near impossible to deduce the underlying form of /o/ or /a/ following /w/ simply from hearing an utterance, and many native speakers themselves seem to have difficulty identifying the phoneme underlying what is often phonetically something like [ɔ].

  That said, there are \isi{minimal pair}s contrasting /wo/ and /wa/ (even if both can be phonetically [wɔ]), as seen in \REF{ex:phon:45}.

\ea%45
    \label{ex:phon:45}
          Minimal pairs contrasting /wo/ and /wa/\\
\begin{tabbing}
{(\textit{\textbf{wo}nmbi})} \= {(‘younger sibling’)} \= {(vs.)} \= {(\textit{\textbf{wa}nmbi})} \= {(‘betel pepper’)}\kill
{\textit{\textbf{wo}l}} \> {‘breast’} \> {vs.} \> {\textit{\textbf{wa}l}} \> {‘ribs’}\\
{\textit{\textbf{wo}n}} \> {‘penis’} \> {vs.} \> {\textit{\textbf{wa}n}} \> {‘sago shoot’}\\
{\textit{\textbf{wo}nmbi}} \> {‘tusk’} \> {vs.} \> {\textit{\textbf{wa}nmbi}} \> {‘betel pepper’}\\
{\textit{\textbf{wo}pa}} \> {‘all’} \> {vs.} \> {\textit{\textbf{wa}pa}} \> {‘leaf’}\\
{\textit{\textbf{wo}t}} \> {‘younger sibling’} \> {vs.} \> {\textit{\textbf{wa}t}} \> {‘ladder’}\\
{\textit{\textbf{wo}wal}} \> {‘chicken’} \> {vs.} \> {\textit{\textbf{wa}wal}} \> {‘hive’}
\end{tabbing}
\z

Other \isi{minimal pair}s reveal the phonemic difference between \isi{high vowel}s and \isi{mid vowel}s \REF{ex:phon:46} as well as between \isi{front vowel}s and \isi{back vowel}s \REF{ex:phon:47}.

\ea%46
    \label{ex:phon:46}
          Contrasts between \isi{high vowel}s and \isi{mid vowel}s\\
    \ea  /i/ versus /e/\\
\begin{tabbing}
{(\textit{n\textbf{u}we})} \= {(‘I myself’)} \= {(vs.)} \= {(\textit{n\textbf{o}we})} \= {(‘\textit{kanda} (rattan)’)}\kill
{\textit{w\textbf{i}}} \> {‘name’} \> {vs.} \> {\textit{w\textbf{e}}} \> {‘sago starch’}\\
{\textit{as\textbf{i}}} \> {‘grass’} \> {vs.} \> {\textit{as\textbf{e}}} \> {‘no’}\\
{\textit{l\textbf{i}}} \> {‘down’} \> {vs.} \> {\textit{l\textbf{e}}} \> {‘\textit{kanda} (rattan)’}
\end{tabbing}

    \ex  /u/ versus /o/\\
\begin{tabbing}
{(\textit{n\textbf{u}we})} \= {(‘I myself’)} \= {(vs.)} \= {(\textit{n\textbf{o}we})} \= {(‘\textit{kanda} (rattan)’)}\kill
{\textit{il\textbf{u}m}} \> {‘little’} \> {vs.} \> {\textit{il\textbf{o}m}} \> {‘day’}\\
{\textit{n\textbf{u}m}} \> {‘canoe’} \> {vs.} \> {\textit{n\textbf{o}m}} \> {‘clay stand’}\\
{\textit{n\textbf{u}we}} \> {‘I myself’} \> {vs.} \> {\textit{n\textbf{o}we}} \> {‘sago species’}
\end{tabbing}
\z
\z

\ea%47
    \label{ex:phon:47}
          Contrasts between \isi{front vowel}s and \isi{back vowel}s\\
    \ea  /i/ versus /u/\\
\begin{tabbing}
{(\textit{and\textbf{e}})} \= {(‘\textit{kanda} (rattan)’)} \= {(vs.)} \= {(\textit{and\textbf{o}})} \= {(‘you [\textsc{pl}]’)}\kill
{\textit{\textbf{i}m}} \> {‘tree’} \> {vs.} \> {\textit{\textbf{u}m}} \> {‘neck’}\\
{\textit{\textbf{i}n}} \> {‘in’} \> {vs.} \> {\textit{\textbf{u}n}} \> {‘you [\textsc{pl}]’}\\
{\textit{ng\textbf{i}n}} \> {‘net’} \> {vs.} \> {\textit{ng\textbf{u}n}} \> {‘you [\textsc{du}]’}
\end{tabbing}


    \ex  /e/ versus /o/\\
\begin{tabbing}
{(\textit{and\textbf{e}})} \= {(‘\textit{kanda} (rattan)’)} \= {(vs.)} \= {(\textit{and\textbf{o}})} \= {(‘you [\textsc{pl}]’)}\kill
{\textit{and\textbf{e}}} \> {‘OK’} \> {vs.} \> {\textit{and\textbf{o}}} \> {‘there’}\\
{\textit{l\textbf{e}}} \> {‘\textit{kanda} (rattan)’} \> {vs.} \> {\textit{l\textbf{o}}} \> {‘cut’}\\
{\textit{w\textbf{e}}} \> {‘sago starch’} \> {vs.} \> {\textit{w\textbf{o}}} \> {‘own’}
\end{tabbing}
\z
\z

The high \isi{central vowel} and \is{low vowel} low \isi{central vowel} also show phonemic contrasts \REF{ex:phon:48}.

\ea%48
    \label{ex:phon:48}
          Contrasts between /ï/ and /a/\\
\begin{tabbing}
{(\textit{and\textbf{ï}})} \= {(‘sago shoot’)} \= {(vs.)} \= {(\textit{and\textbf{a}})} \= {(tail feather’)}\kill
{\textit{and\textbf{ï}}} \> {‘sago shoot’} \> {vs.} \> {\textit{and\textbf{a}}} \> {‘that’}\\
{\textit{n\textbf{ï}}} \> {‘I’} \> {vs.} \> {\textit{n\textbf{a}}} \> {‘talk’}\\
{\textit{t\textbf{ï}l}} \> {‘husk’} \> {vs.} \> {\textit{t\textbf{a}l}} \> {tail feather’}
\end{tabbing}
\z

The high \isi{central vowel} can further be shown to be distinct from other \isi{high vowel}s, both back \REF{ex:phon:49} and front \REF{ex:phon:50}.

\ea%49
    \label{ex:phon:49}
          Contrasts between /ï/ and /u/\\
\begin{tabbing}
{(\textit{t\textbf{ï}l})} \= {(‘he/she/it’)} \= {(vs.)} \= {(\textit{t\textbf{u}l})} \= {(‘bird species’)}\kill
{\textit{m\textbf{ï}}} \> {he/she/it’} \> {vs.} \> {\textit{m\textbf{u}}} \> {‘fruit’}\\
{\textit{n\textbf{ï}}} \> {‘I’} \> {vs.} \> {\textit{n\textbf{u}}} \> {‘near’}\\
{\textit{t\textbf{ï}l}} \> {‘husk’} \> {vs.} \> {\textit{t\textbf{u}l}} \> {‘bird species’}
\end{tabbing}
\z

\ea%50
    \label{ex:phon:50}
          Contrasts between /ï/ and /i/\\
\begin{tabbing}
{(\textit{m\textbf{ï}nam})} \= {(‘he’s the one’)} \= {(vs.)} \= {(\textit{m\textbf{i}nam})} \= {(‘splinter’)}\kill
{\textit{m\textbf{ï}}} \> {‘he/she/it’} \> {vs.} \> {\textit{m\textbf{i}}} \> {‘splinter’}\\
{\textit{m\textbf{ï}nam}} \> {‘he’s the one’} \> {vs.} \> {\textit{m\textbf{i}nam}} \> {‘urine’}\\
{\textit{and\textbf{ï}n}} \> {‘for’} \> {vs.} \> {\textit{and\textbf{i}n}} \> {‘those’}
\end{tabbing}
\z

Despite clearly having phonemic status in the language, the high \isi{central vowel} /ï/ behaves somewhat differently from the other five vowels. When underlyingly present, it is the \isi{vowel} most likely to be \isi{elide}d; and when underlyingly absent, it is the \isi{vowel} most likely to be inserted \isi{epenthetic}ally (e.g., to break up an elicit or disfavored \isi{consonant cluster}).

  There is one last vocalic phone in Ulwa that deserves attention: a \is{lax vowel} lax \is{low vowel} low \is{front vowel} front \isi{unrounded} \isi{vowel} [æ], which has been observed in just a handful of words. It is only found, moreover, in the \ili{Manu} \isi{dialect} (it has not been observed in the \ili{Maruat-Dimiri-Yaul} \isi{dialect}). It is distinctly lower than /e/ and fronter than /a/. So far, four words have been found with this \isi{vowel} sound \REF{ex:phon:51}.

\ea%51
    \label{ex:phon:51}
          Words with the \isi{vowel} [æ]\\
\begin{tabbing}
{(\textit{w\textbf{ae}mbïl})} \= {(‘plant species’)}\kill
{\textit{m\textbf{ae}}} \> {‘shovel’}\\
{\textit{m\textbf{ae}p}} \> {‘bird species’}\\
{\textit{w\textbf{ae}mbïl}} \> {‘white’}\\
{\textit{w\textbf{ae}nkïn}} \> {‘plant species’}
\end{tabbing}
\z

In at least some cases, this \isi{vowel} may derive from sequences of /ea/. The plant \textit{waenkïn} ‘plant species’ is described as being similar to the plant \textit{ankïn} ‘vegetable species’, only having leaves with the (off-)white color of \textit{we} ‘sago starch’ (i.e., \textit{waenkïn} ‘plant species’ < \textit{we} ‘sago starch’ + \textit{ankïn} ‘vegetable species’).

  Likewise, the word \textit{mae} ‘shovel’ may be connected to the word \textit{me} ‘\textit{limbum} palm’ (from which the shovel is made). The etymology of this word might thus be: \textit{mae} ‘shovel’ < \textit{me} ‘\textit{limbum} palm’ + \textit{a-} ‘break’ (?).

\is{phonology|)}
\is{phonetics|)}
\is{monophthong|)}
\is{vowel|)}

\is{monophthong|(}
\is{vowel|(}
\is{phonetics|(}
\is{phonology|(}

  The word \textit{waembïl} ‘white’ also likely contains \textit{we} ‘sago starch’ historically, but here the resulting [æ] may be the product of a formerly underlying /e/ phonetically \isi{nasal}izing (due to the following \isi{nasal} articulation) and consequently lowering (first in perception, then in production) to [æ].\footnote{The other \isi{dialect}s of Ulwa lend some insight here. In \ili{Dimiri}, ‘white’ is [ʋeⁿdum] (cf. [ʋe] ‘sago starch’), and in \ili{Yaul} ‘white’ is [weᵐbal]. The meanings of the forms [ndum] and [mbal] is obscure, but at least the latter is found in the \ili{Manu} \isi{dialect} word \textit{anembal} ‘light (color)’, which clearly contains \textit{ane} ‘sun’. Thus, it may be hypothesized that \ili{Manu} \textit{waembïl} ‘white’ derives from \textit{we} ‘sago starch’ + \textit{mbal} ‘color (?)’, the /a/ in the second \isi{syllable} having reduced to [ï], and the /e/ in the first \isi{syllable} having lowered to [ae] (see \sectref{sec:14.5} for more on \isi{color term}s in Ulwa).}

The word \textit{maep} ‘bird species’, however, offers no ready etymology. It does not seem to be connected in any way with \textit{me} ‘\textit{limbum} palm’. The word could be \isi{onomatopoetic}, as are the names of some other bird species (\sectref{sec:14.3}).

  Given the extremely limited occurrence of [ae] and the fact that it can almost always be explained away as having different underlying vowels, it is not treated as a separate phoneme in this grammatical description. It is, however, written distinctly from both /a/ and /e/, since there are \is{minimal pair} minimal (and near-minimal) pairs contrasting [ae] with both /a/ \REF{ex:phon:52} and /e/ \REF{ex:phon:53}.

\ea%52
    \label{ex:phon:52}
          Minimal pair contrasting /ae/ and /e/\\
\begin{tabbing}
{(\textit{m\textbf{ae}})} \= {(‘shovel’)} \= {(vs.)} \= {(\textit{m\textbf{e}})} \= {(‘\textit{limbum} palm’)}\kill
{\textit{m\textbf{ae}}} \> {‘shovel’} \> {vs.} \> {\textit{m\textbf{e}}} \> {‘\textit{limbum} palm’}
\end{tabbing}
\z

\is{minimal pair}

\ea%53
    \label{ex:phon:53}
          Minimal pair contrasting /ae/ and /a/\\
\begin{tabbing}
{(\textit{m\textbf{ae}})} \= {(‘shovel’)} \= {(vs.)} \= {(\textit{m\textbf{a}})} \= {(‘his/her/its’)}\kill
{\textit{m\textbf{ae}}} \> {‘shovel’} \> {vs.} \> {\textit{m\textbf{a}}} \> {‘his/her/its’}
\end{tabbing}
\z

  There are some rather interesting \isi{phonotactic} constraints placed on vowels. Most notably, the only permissible vowels in \isi{syllable}s without \isi{consonant} \isi{onset}s are /a, i, u/. Furthermore, since many speakers insert \isi{epenthetic} \isi{glide}s before word-initial /i/ and /u/ (namely, [y] and [w], respectively, \sectref{sec:2.1.7}), the only permitted \isi{onset} \isi{vowel} in some \isi{idiolect}s is [a]. Since all vowel-initial \isi{syllable}s begin phonetically with a \is{glottal stop} \isi{glottal} \isi{stop} [ʔ] (when utterance-initial, \sectref{sec:2.1.8}), it could further be argued that the language lacks V(C) \isi{syllable}s altogether, at least phonetically.

  The high \isi{central vowel} /ï/ patterns differently from the other vowels. As mentioned in \sectref{sec:2.1.2}, this \isi{vowel} can serve an \isi{epenthetic} function, breaking up certain \isi{consonant cluster}s. On the other hand, it sometimes seems to be underlyingly present but commonly \isi{elide}d, for example when immediately preceding a \isi{liquid}, thereby resulting in a pronunciation with a \is{syllabic liquid} \isi{syllabic} [l̩] (\sectref{sec:2.1.5}). There are similar examples of words in which /ï/ is commonly \isi{elide}d before \isi{nasal}s, again resulting in a \isi{syllabic} pronunciation of the following \isi{nasal} \isi{consonant}, whether \isi{bilabial} \REF{ex:phon:54} or \isi{alveolar} \REF{ex:phon:54a}.

\ea%54
    \label{ex:phon:54}
          Syllabic [m̩]\\
\begin{tabbing}
\is{syllabic nasal}
{(\textit{ambat\textbf{ïm}})} \= {(‘tongue’)} \= {([ambatïm {\textasciitilde} ambat\textbf{m̩}])}\kill
{\textit{ambat\textbf{ïm}}} \> {‘knee’} \> {[ambatïm {\textasciitilde} ambat\textbf{m̩}]}\\
{\textit{mïn\textbf{ïm}}} \> {‘tongue’} \> {[mïnïm {\textasciitilde} mïn\textbf{m̩} {\textasciitilde} mn̩\textbf{m̩}]}\textbf{}
\end{tabbing}
\z

\ea%54a
    \label{ex:phon:54a}
  Syllabic [n̩]\\
\begin{tabbing}
\is{syllabic nasal}
{(\textit{mïnk\textbf{ïn}})} \= {(‘grub species’)} \= {([mïŋkïn {\textasciitilde} mïŋk\textbf{n̩} {\textasciitilde} m\textbf{ŋ̍}k\textbf{n̩}])}\kill
{\textit{ap\textbf{ïn}}} \> {‘fire’} \> {[apïn {\textasciitilde} ap\textbf{n̩}]}\\
{\textit{mït\textbf{ïn}}} \> {‘egg’} \> {[mïtïn {\textasciitilde} mït\textbf{n̩}]}\\
{\textit{mïnk\textbf{ïn}}} \> {‘grub species’} \> {[mïŋkïn {\textasciitilde} mïŋk\textbf{n̩} {\textasciitilde} m\textbf{ŋ̍}k\textbf{n̩}]}
\end{tabbing}
\z

Like the sometimes \isi{syllabic liquid}s, the \isi{nasal}s in these and similar words are always transcribed in this grammar with the accompanying \isi{vowel} <ï>.

\is{phonology|)}
\is{phonetics|)}
\is{vowel|)}
\is{monophthong|)}

\subsection{Diphthongs /aw, ay/}\label{sec:2.2.2}

\is{vowel|(}
\is{diphthong|(}
\is{phonetics|(}
\is{phonology|(}

The two primary diphthongs in Ulwa are /aw/ and /ay/, each formed through the combination of the \is{low vowel} low \isi{central vowel} /a/ and one of the two \isi{glide}s, /w/ or /y/. On the status of [oy], which may be underlyingly /oi/, see \sectref{sec:2.1.7}.

\is{phonology|)}
\is{phonetics|)}
\is{diphthong|)}
\is{vowel|)}

\section{Syllable structure}\label{sec:2.3}

\is{syllable structure|(}
\is{phonetics|(}
\is{phonology|(}
\is{syllable|(}

Ulwa permits a variety of \isi{syllable} shapes: syllables may or may not have \isi{onset}s, \isi{coda}s, or both. \isi{Complex onset}s are, however, quite limited, and \isi{complex coda}s are generally absent (limited only to certain instances of verbal \isi{suffix}ation). Words that consist of a single \isi{vowel} segment offer the clearest examples of V-only \isi{syllable} structure \REF{ex:phon:55}.

\ea%55
    \label{ex:phon:55}
          Syllables without \isi{onset}s or \isi{coda}s (V)\\
\begin{tabbing}
{(\textit{i})} \= {(‘hand, arm’)} \= {(\textit{u})} \= {(‘in, at, from, around, along’)}\kill
{\textit{i}} \> {‘hand, arm’} \> {\textit{u}} \> {‘you [\textsc{sg}]’}\\
{\textit{i}} \> {‘lime’} \> {\textit{u}} \> {‘ditch, creek’}\\
{\textit{i}} \> {‘go.\textsc{pfv}’} \> {\textit{u}} \> {‘in, at, from, around, along’}
\end{tabbing}
\z

Thus, neither \isi{onset}s nor \isi{coda}s are required in the language. It should be noted, however, that this might be the case only at the underlying level, since otherwise \isi{vowel}-initial words are often realized with an initial \isi{glide} (\sectref{sec:2.1.7}) or with an initial \is{glottal stop} \isi{glottal} \isi{stop} (\sectref{sec:2.1.8}). Note also the high degree of \isi{homophony} among words of the form /i/ and among words of the form /u/ (\sectref{sec:14.2}).

  There are also longer words that may be analyzed as having syllables lacking both \isi{onset}s and \isi{coda}s. Since \isi{prenasalized} \isi{voiced} \isi{stop}s do not occur in \isi{coda} position, it can be assumed for each word in \REF{ex:phon:56} that each \isi{stop} is serving as \isi{onset} to the second \isi{syllable}.

\ea%56
    \label{ex:phon:56}
          Syllables without \isi{onset}s or \isi{coda}s (in longer words) (V)\\
\begin{tabbing}
{(\textit{\textbf{u}mbopa})} \= {(‘stomach’)} \= {([u.mbo.pa])} \= {(\textit{\textbf{a}nda})} \= {(‘put [\textsc{irr]}’)} \= {()}\kill
{\textit{\textbf{a}mbi}} \> {‘big’} \> {[a.mbi]} \> {\textit{\textbf{a}nda}} \> {‘that’} \> {[a.nda]}\\
{\textit{\textbf{i}mba}} \> {‘night’} \> {[i.mba]} \> {\textit{\textbf{i}nga}} \> {‘affine’} \> {i.nga}\\
{\textit{\textbf{u}mbopa}} \> {‘stomach’} \> {[u.mbo.pa]} \> {\textit{\textbf{u}nda}} \> {‘put [\textsc{irr]}’} \> {[u.nda]}
\end{tabbing}
\z

Another possible \isi{syllable} shape is CV (i.e., a simple \isi{onset} and no \isi{coda}) \REF{ex:phon:57}.

\ea%57
    \label{ex:phon:57}
              Syllables with simple \isi{onset}s (CV)
  \begin{tabbing}
  {(\textit{mae})} \= {(‘sago starch’)}   \=       {(\textit{ndï})}  \=  {(‘\textit{limbum} palm’)}\kill
   \textit{li}  \>  ‘down’    \>        \textit{le}  \>  ‘\textit{kanda} (rattan)’\\
    \textit{mae} \> ‘shovel’   \>       \textit{me}  \>  ‘\textit{limbum} palm’\\
    \textit{mï}  \>  ‘he/she/it’   \>       \textit{mu}  \>  ‘fruit’\\
    \textit{nï}  \>  ‘I’      \>        \textit{tï}  \>  ‘take’\\
    \textit{na}  \>  ‘talk’    \>        \textit{ka}  \>  ‘at, in, on’\\
    \textit{pe}  \>  ‘be [\textsc{dep]}’    \>      \textit{se}  \>  ‘cry [\textsc{ipfv}]’\\
    \textit{mbï} \> ‘here’     \>       \textit{ndï}  \>  ‘they’\\
    \textit{nga} \> ‘this’    \>        \textit{nji}  \>  ‘thing’\\
    \textit{ya}  \>  ‘coconut’   \>       \textit{wa}  \>  ‘village’\\
    \textit{we}  \>  ‘sago starch’   \>     \textit{wi}  \>  ‘name’
\end{tabbing}
\z

Each word in \REF{ex:phon:57} is \isi{monosyllabic}, beginning with a \isi{consonant} and ending with a \isi{vowel}. Note that \isi{glide}s may form the \isi{onset} of a CV \isi{syllable}, as, for example, in \textit{ya}  ‘coconut’ or \textit{wi} ‘name’

It is also possible for syllables to contain \isi{coda}s. The words in \REF{ex:phon:58} contain syllables with no \isi{onset}, but with simple \isi{coda}s (which may be \isi{glide}s, as in \textit{ay} ‘jellied sago’). \isi{Disyllabic} words may have initial VC syllables, as illustrated by examples such as \textit{anma} ‘good’.

\ea%58
    \label{ex:phon:58}
          Syllables with simple \isi{coda} and no \isi{onset} (VC)\\
\begin{tabbing}
{(\textit{\textbf{al}mba})} \= {(‘mosquito net’)} \= {([al.mba])} \= {(\textit{\textbf{un}mbï})} \= {(‘betel nut’)} \= {()}\kill
{\textit{im}} \> {‘tree’} \> { } \> {\textit{ip}} \> {‘nose’} \> { }\\
{\textit{al}} \> {‘mosquito net’} \> { } \> {\textit{un}} \> {‘you [\textsc{pl}]’} \> { }\\
{\textit{ay}} \> {‘jellied sago’} \> { } \> {\textit{aw}} \> {‘betel nut’} \> { }\\
{\textit{\textbf{ip}ka}} \> {‘before’} \> {[ip.ka]} \> {\textit{\textbf{un}mbï}} \> {‘buttocks’} \> {[un.mbï]}\\
{\textit{\textbf{al}mba}} \> {‘hornbill’} \> {[al.mba]} \> {\textit{\textbf{an}ma}} \> {‘good’} \> {[an.ma]}
\end{tabbing}
\z

A \isi{syllable} may also have both a simple \isi{onset} and a simple \isi{coda} (CVC), as in \REF{ex:phon:59}.

\ea%59
    \label{ex:phon:59}
          Syllables with both \isi{onset} and \isi{coda} (CVC)
    \begin{tabbing}
    {(\textit{ngan})} \= {(‘we [\textsc{du.excl}]’)}   \=    {(\textit{ndam})} \= {(‘bird species’)}\kill
    \textit{lam}  \> ‘meat’     \>       \textit{ndam} \> ‘bridge’\\
 \textit{tïn}  \>  ‘dog’     \>       \textit{ngin} \> ‘net’\\
 \textit{ngan} \> ‘we [\textsc{du.excl}]’   \>    \textit{ngun} \> ‘you [\textsc{du}]’\\
 \textit{nil}  \>  ‘body hair’   \>     \textit{tul}  \>  ‘bird species’\\
 \textit{pul}  \>  ‘piece’      \>      \textit{kot}  \>  ‘break’\\
 \textit{nap} \> ‘arrow’    \>      \textit{nip}  \>  ‘die [\textsc{pfv}]’\\
 \end{tabbing}
\z

One or both of the \isi{consonant}s in such a CVC \isi{syllable} may be a \isi{glide}, as illustrated by the words in \REF{ex:phon:59a}.

\ea%59a
    \label{ex:phon:59a}
          CVC syllables containing \isi{glide}s
    \begin{tabbing}
    {(\textit{maw})} \= {(‘younger’)}   \=    {(\textit{may})} \= {(‘plant species’)}\kill
 \textit{wal}  \> ‘ribs’    \>         \textit{wan} \> ‘sago shoot’\\
 \textit{wat}  \> ‘ladder’   \>       \textit{wen} \> ‘handle’\\
 \textit{wol}  \> ‘breast’    \>      \textit{won} \> ‘penis’\\
 \textit{wot}  \> ‘younger’    \>      \textit{wun} \> ‘fan’\\
 \textit{yom} \>  ‘heart’    \>        \textit{yot}  \>  ‘machete’\\
 \textit{kaw} \>  ‘song’     \>       \textit{law}  \>  ‘plant species’\\
 \textit{maw} \>  ‘correct’    \>      \textit{may} \> ‘fish species’\\
 \textit{way} \>  ‘turtle’
 \end{tabbing}
\z


Finally, some \isi{complex onset}s are allowed, resulting in the \isi{syllable} shape CCV or CCVC. The attested permissible complex CC \isi{onset}s are /kw-/ \REF{ex:phon:60}, /ngw-/ \REF{ex:phon:60a}, /mbl-/ \REF{ex:phon:60b}, /pl-/ \REF{ex:phon:60c}, and /mw-/ \REF{ex:phon:60d}.

\ea%60
    \label{ex:phon:60}
{\isi{Velar}-plus-\isi{labial-velar} \isi{complex onset}: \textit{kw-}}\\
\begin{tabbing}
{(\textit{\textbf{kw}a})} \= {(‘who?’)}\kill
{\textit{\textbf{kw}a}} \> {‘who?’}
\end{tabbing}
\z

\ea%60a
    \label{ex:phon:60a}
{\isi{Velar}-plus-\isi{labial-velar} \isi{complex onset}: \textit{ngw-}}\\
\begin{tabbing}
{(\textit{i\textbf{ngw}a})} \= {(‘spider’)} \= {([i.ngwa])}\kill
{\textit{i\textbf{ngw}a}} \> {‘spider’} \> {[i.ngwa]}
\end{tabbing}
\z

\ea%60b
    \label{ex:phon:60b}
{\isi{Bilabial} \isi{stop}-plus-\isi{liquid} \isi{complex onset}: \textit{mbl-}}\\
\begin{tabbing}
{(\textit{ko\textbf{mbl}am})} \= {(‘rat species’)} \= {([ko.mblam])}\kill
{\textit{\textbf{mbl}andu}} \> {‘rat species’} \> {[mbla.ndu]}\\
{\textit{a\textbf{mbl}a}} \> {‘tooth’} \> {[a.mbla]}\\
{\textit{ko\textbf{mbl}am}} \> {‘chair’} \> {[ko.mblam]}\\
{\textit{na\textbf{mbl}i}} \> {‘feather’} \> {[na.mbli]}
\end{tabbing}
\z

\ea%60c
    \label{ex:phon:60c}
{\isi{Bilabial} \isi{stop}-plus-\isi{liquid} \isi{complex onset}: \textit{pl-}}\\
\begin{tabbing}
{(\textit{wo\textbf{pl}ota})} \= {(‘flying fox’)} \= {([wo.plo.ta])}\kill
{\textit{nï\textbf{pl}opa}} \> {‘flying fox’} \> {[nï.plo.pa]}\\
{\textit{wo\textbf{pl}ota}} \> {‘lungs’} \> {[wo.plo.ta]}
\end{tabbing}
\z

\ea%6d0
    \label{ex:phon:60d}
{\isi{Bilabial} \isi{nasal}-plus-\isi{labial-velar} \isi{complex onset}: \textit{mw-}}\\
\begin{tabbing}
{(\textit{\textbf{mw}a})} \= {(‘opening’)}\kill
{\textit{\textbf{mw}a}} \> {‘opening’}
  \end{tabbing}
\z


In an alternative analysis, at least some of these apparent CCs could be treated instead as consisting of single (complex) phonemes, such as \is{labialization} labialized \isi{velar} \isi{stop}s [kʷ, ᵑɡʷ] or a labialized \isi{bilabial} \isi{nasal} [mʷ]. But if these are in fact separate phonemes in Ulwa, then they have very limited representation in the \isi{lexicon}.

  No \isi{onset}s of more than two \isi{consonant}s have been found, nor have any \isi{complex coda}s within a single morpheme.

\is{syllable|)}
\is{phonology|)}
\is{phonetics|)}
\is{syllable structure|)}

\section{Stress}\label{sec:2.4}

\is{stress|(}
\is{phonetics|(}
\is{phonology|(}

Stress in Ulwa is not phonemic. In single-word utterances, \isi{disyllabic} words may receive stress either on the \isi{ultima} or on the \isi{penult}, although there is perhaps a slight preference for \isi{penultimate} (\isi{trochaic}) stress. In longer words and \isi{phrase}s, \isi{pragmatic} factors play a significant role in stress assignment, although there is nevertheless a tendency for stress to fall on alternating \isi{syllable}s. Ulwa may be considered a \isi{syllable-timed} language.

  There is no phonemic \isi{tone}, nor are there other \is{suprasegmental feature} suprasegmental phonemic distinctions found in the language, such as \is{vowel length} \isi{vowel} \isi{length}.

\is{phonology|)}
\is{phonetics|)}
\is{stress|)}

\section{Morphophonemic processes}\label{sec:2.5}

\is{morphophonemic process|(}
\is{morphophonology|(}
\is{phonetics|(}
\is{phonology|(}

As there is minimal \isi{affix}ation in Ulwa, there are few opportunities to witness phonological alternations occurring between related word forms. Nevertheless, while morphophonemic processes typically occur within phonological words, almost any such process is possible across lexeme boundaries of all types. Still, for the sake of clarity, phonological changes are noted as they occur within words or \isi{clitic}-host pairs where possible in the following subsections.

\is{phonology|)}
\is{phonetics|)}
\is{morphophonology|)}
\is{morphophonemic process|)}


\subsection{Glide formation}\label{sec:2.5.1}

\is{glide formation|(}
\is{glide|(}
\is{morphophonemic process|(}
\is{morphophonology|(}
\is{phonetics|(}
\is{phonology|(}

Sequences of /au/ and /ai/ \isi{coalesce} into series of \isi{vowel}-plus-\isi{glide}. That is, \isi{high vowel}s /u, i/ strengthen to \isi{approximant}s [w, y] when immediately following a \isi{low vowel} \REF{ex:phon:61}. There are no contexts in which the high \isi{central vowel} /ï/ follows a \isi{low vowel} (or any \isi{vowel}, for that matter).\largerpage

\ea%61
    \label{ex:phon:61}
          \isi{Glide} formation\\
    V [+high] → [-syl] / V [+low] \_
\z

This phonological process is clearly revealed by the addition of \isi{object-marker} \isi{proclitic}s, which index \isi{person} and \isi{number} (\sectref{sec:7.2}). In the examples in \REF{ex:phon:62}, a \isi{glide} is formed wherever the \isi{object-marker} \isi{clitic} ends in /a-/ (such as \textit{ma=} ‘3\textsc{sg.obj’}) and the \isi{verb stem} host begins with a \isi{high vowel} (/i/ or /u/).

\ea%62
    \label{ex:phon:62}
          \isi{Glide} formation in \isi{object-marker} \isi{proclitic}s\\
\begin{tabbing}    
{(a.)} \= {(/ma=uta/)} \= {(d.)} \= {(/min=uta/)}\kill
 {a.} \> {[matï-]} \> {b.} \> {[mintï-]}\\
 { } \> {/ma=tï-/} \> { } \> {/min=tï-/}\\
 { } \> {‘take it'} \> { } \> {‘take two’}\\
 {c.} \> {[m\textbf{ay}ta-]} \> {d.} \> {[minita-]}\\
 { } \> {/ma=ita-/} \> { } \> {/min=ita-/}\\
 { } \> {‘build it’} \> { } \> {‘build two’}\\
{e.} \> {[m\textbf{aw}ta-]} \> {f.} \> {[minuta-]}\\
 { } \> {/ma=uta-/} \> { } \> {/min=uta-/}\\
 { } \> {‘grind it’} \> { } \> {‘grind two’}
   \end{tabbing}
 \z

The \isi{mid vowel}s /e, o/ generally do not condition this \isi{fortition}. Instead, \isi{epenthetic} \isi{glide}s break up forbidden \isi{vowel} sequences such as \textsuperscript{†}[eu, ei, ou, oi], producing forms such as [eyu, eyi, owu, owi]. There is one partial exception, however. Although the \isi{vowel} sequence /oi/ tends to become [owi] when occurring across a word boundary, it is possible for a \isi{glide} to occur when this sequence falls across a \isi{clitic} boundary (yielding [oy]) \REF{ex:phon:63}.

\ea%63
    \label{ex:phon:63}
          \isi{Glide} formation of [oy]\\
    /i/ → [y] / o \_ ]\#
\z

This change can be witnessed when the \isi{indefinite} \isi{object-marker} \isi{proclitic} \textit{ko=} ‘\textsc{indf}’ (\sectref{sec:7.2}) precedes a verb beginning with /i-/, as in \REF{ex:phon:64}.

\ea%64
    \label{ex:phon:64}
          \isi{Glide} formation of [oy] with \textit{ko=} ‘\textsc{indf}’\\
\begin{tabbing}    
{(a.)} \= {(/ko=ita-/)} \= {(b.)} \= {(/ko=tï-/)}\kill
 {a.} \> {[k\textbf{oy}ta-]} \> {b.} \> {[kotï-]}\\
 { } \> {/ko=ita-/} \> { } \> {/ko=tï-/}\\
 { } \> {‘build a’} \> { } \> {‘take a’}
   \end{tabbing}
\z

The \isi{high vowel}s /i, ï, u/ do not condition the \isi{fortition} seen in \REF{ex:phon:62}. Here, too, \isi{epenthetic} \isi{glide}s are formed to break up \isi{vowel} sequences, as in the \isi{perfective} form of the verb ‘fall’, /li-u/, in which an \isi{epenthetic} [y] (IPA [j]) separates the sequence of two \isi{high vowel}s, producing [liyu].

\is{phonology|)}
\is{phonetics|)}
\is{morphophonology|)}
\is{morphophonemic process|)}
\is{glide|)}
\is{glide formation|)}

\subsection{Monophthongization}\label{sec:2.5.2}

\is{monophthong|(}
\is{monophthongization|(}
\is{morphophonology|(}
\is{morphophonemic process|(}
\is{phonetics|(}
\is{phonology|(}
\is{vowel|(}

Sequences of /aw/ and /ay/ may optionally become [o] and [e], respectively, when not immediately followed by a \isi{vowel}. Thus, for many speakers, \textit{yawt} ‘machete’ is pronounced [yot]. The word for ‘time’, \isi{borrow}ed from \ili{Tok Pisin} \textit{taim} ‘time’, has been \isi{fossilized} as [tem]. This change can also occur when the underlying forms are /au/ or /ai/. In other words, this \isi{monophthongization} rule \REF{ex:phon:65} can apply after the \isi{glide formation} rule (\sectref{sec:2.5.1}).

\ea%65
    \label{ex:phon:65}
          Monophthongization\\
    /aw/ → ([o]) / \_ \{ $\genfrac{}{}{0pt}{}{\textrm{C}}{\textrm{\#}}$ {  } (optional)\\
    /ay/ → ([e]) / \_ \ \{ $\genfrac{}{}{0pt}{}{\textrm{C}}{\textrm{\#}}$ {  } (optional)
\z

This \isi{monophthongization} occurs across morpheme boundaries \REF{ex:phon:66}.

\is{vowel|)}
\is{phonology|)}
\is{phonetics|)}
\is{morphophonemic process|)}
\is{morphophonology|)}
\is{monophthongization|)}
\is{monophthong|)}

\ea%66
    \label{ex:phon:66}
          Monophthongization across morpheme boundaries\\
\begin{tabbing}    
{(a.)} \= {(‘with two’)} \= {(vs.)} \= {(d.)} \= {([mayn] or [men])}\kill
 {a.} \> {[minul]} \> {vs.} \> {b.} \> {[mawl] or [m\textbf{o}l]}\\
 { } \> {/min=ul/} \> { } \> { } \> {/ma=ul/}\\
 { } \> {‘with two’} \> { } \> { } \>  {‘with it’}\\
 {c.} \> {[minin]} \> {vs.} \> {d.} \> {[mayn] or [m\textbf{e}n]}\\
 { } \> {/min=in/} \> { } \> { } \> {/ma=in/}\\
 { } \> {‘in two’} \> { } \> { } \> {‘in it’}\\
 {e.} \> {[i]} \> {vs.} \> {f.} \> {[nay] or [n\textbf{e}]}\\
 { } \> {/i/} \> { } \> { } \> {/na-i/}\\
 { } \> {‘went’} \> { } \> { } \> {‘went away’}
 \end{tabbing}
  \z


\subsection{High vowel gliding}\label{sec:2.5.3}

\is{high vowel gliding|(}
\is{high vowel|(}
\is{gliding|(}
\is{glide|(}
\is{morphophonemic process|(}
\is{morphophonology|(}
\is{phonetics|(}
\is{phonology|(}
\is{vowel|(}

The high \isi{back vowel} /u/ becomes a \isi{glide} immediately before a \isi{vowel} occurring in the same \isi{syllable} \REF{ex:phon:67}.

\ea%67
    \label{ex:phon:67}
          High \isi{vowel} \isi{gliding}\\
    /u/ → [w] / \_ V]{\textsubscript{σ}}
\z

Example \REF{ex:phon:68} shows the verbs \textit{asa-} ‘hit’ and \textit{ama-} ‘eat’ occurring with different \isi{object-marker} \isi{clitic}s. High \isi{vowel} \isi{gliding} can be seen in the ‘you [\textsc{sg]}’ forms.

\newpage

\ea%68
    \label{ex:phon:68}
          High \isi{vowel} \isi{gliding} with 2\textsc{sg} object forms\\
\begin{tabbing}    
{(a.)} \= {(‘eat you [\textsc{pl}]’)} \= {(h.)} \= {(‘eat you [\textsc{du}]’)}\kill
{a.} \> {[minasa-]} \> {b.} \> {[ngunasa-]}\\
{ } \> {/min=asa-/} \> { } \> {/ngun=asa-/}\\
{ } \> {‘hit two’} \> { } \> {‘hit you [\textsc{du}]’}\\
{c.} \> {[unasa-]} \> {d.} \> {[\textbf{w}asa-]}\\
{ } \> {/un=asa-/} \> { } \> {/u=asa-/}\\
{ } \> {‘hit you [\textsc{pl}]’} \> { } \> {‘hit you [\textsc{sg}]’}\\
{e.} \> {[minama-]} \> {f.} \> {[ngunama-]}\\
{ } \> {/min=ama-/} \> { } \> {/ngun=ama-/}\\
{ } \> {‘eat two’} \> { } \> {‘eat you [\textsc{du}]’}\\
{g.} \> {[unama-]} \> {h.} \> {[\textbf{w}ama-]}\\
{ } \> {/un=ama-/} \> { } \> {/u=ama-/}\\
{ } \> {‘eat you [\textsc{pl}]’} \> { } \> {‘eat you [\textsc{sg}]’}
\end{tabbing}
\z

This rule \REF{ex:phon:67} should not, however, be taken to suggest that the \isi{glide}s (or at least /w/) are not phonemic in Ulwa. That is, it would be implausible to treat every \isi{syllable} with a \isi{glide} in the \isi{onset} as underlyingly /uV/ or /iV/. First, this would create undesirable and unlikely \isi{vowel} clusters in the underlying forms and would even create double \isi{vowel}s in words such as \textit{wusim} ‘crocodile’ and \textit{wulis} ‘platform’, which would have to be assumed to be underlyingly \textsuperscript{†}/uusime/ and \textsuperscript{†}/uulis/, respectively, despite a total surface absence of sequences of identical \isi{vowel}s (i.e., there are no long \isi{vowel}s). It does not seem that forms such as [wusim] and [wulis] are the product of the optional \isi{glide} \isi{epenthesis} rule, since they are always pronounced with /w/.

\is{length}
\is{vowel length}

Furthermore, there are \isi{minimal pair}s (and near-minimal pairs), distinguishing words with initial /u-/ from words with initial /wu-/ \REF{ex:phon:69}.

\ea%69
    \label{ex:phon:69}
          Contrasts between word-initial /u-/ and word-initial /wu-/\\
\begin{tabbing}
{(\textit{utïl})} \= {(‘you [\textsc{pl}]’)} \= {(vs.)} \= {(\textit{w\textbf{}uta})} \= {(‘shell’ (for some speakers))}\kill
{\textit{uta}} \> {‘bird’} \> {vs.} \> {\textit{\textbf{w}uta}} \> {‘shell’ (for some speakers)}\\
{\textit{un}} \> {‘you [\textsc{pl}]’} \> {vs.} \> {\textit{\textbf{w}un}} \> {‘fan’}\\
{\textit{utïl}} \> {‘refuse’} \> {vs.} \> {\textit{\textbf{w}utï}} \> {‘leg, foot’}
\end{tabbing}
\z

Indeed, for purposes of differentiating these /wu/-initial words from their /u/-initial near-\isi{homophone}s, some speakers pronounce them with an initial \isi{labiodental} [ʋ], as in [ʋuta] ‘shell’ versus [uta] ‘bird’. This [ʋ] sound sometimes colors the following high \isi{back vowel}, producing forms such as [ʋïta] ‘shell’.

  It would of course also seem likely, insofar as the other \isi{glide} /y/ patterns like /w/, that a process might also exist of high \isi{front vowel} \isi{gliding}. However, there are no examples of \isi{proclitic}s or \isi{prefix}es ending in /i-/, and thus no way of knowing how this would apply within phonological words. When /i-/ precedes a \isi{vowel} across a word boundary, though, a number of phonological changes are possible. If the following \isi{vowel} is a \isi{mid vowel}, then the /i/ may \isi{delete} (\sectref{sec:2.5.4}). If the following \isi{vowel} is high, then an \isi{epenthetic} \isi{glide} [y] may break up the following sequence. If the following \isi{vowel} is low, however, it is possible for /i-/ and /-a/ to \isi{coalesce} into [e] (\sectref{sec:2.7}).

\is{vowel|)}
\is{phonology|)}
\is{phonetics|)}
\is{morphophonology|)}
\is{morphophonemic process|)}
\is{glide|)}
\is{gliding|)}
\is{high vowel|)}
\is{high vowel gliding|)}

\subsection{Vowel elision before mid vowels}\label{sec:2.5.4}

\is{vowel elision|(}
\is{elision|(}
\is{mid vowel|(}
\is{morphophonology|(}
\is{morphophonemic process|(}
\is{phonetics|(}
\is{phonology|(}
\is{vowel|(}

All \isi{vowel}s are \isi{deleted} before an immediately following /e/ or /o/ \REF{ex:phon:70}.

\ea%70
    \label{ex:phon:70}
          Vowel \isi{elision} before \isi{mid vowel}s\\
    V → Ø / \_ V [-high, -low]
\z

Since neither /e/ nor /o/ occurs word-initially (\sectref{sec:2.2.1}), the only environments in which this process may be observed are within phonological words. The \isi{elision} of \isi{vowel}s before /e/ may be witnessed when \isi{vowel}-final verbs take the \isi{imperfective} \isi{suffix} \textit{-e} ‘\textsc{ipfv}’(\sectref{sec:4.4}) or the \isi{nominalizer} \textit{-en} ‘\textsc{nmlz}’ (\sectref{sec:3.2}). The final \isi{vowel}s of the \isi{verb stem}s in \REF{ex:phon:71} are lost before the \isi{imperfective} \isi{suffix}, in which the \isi{vowel} immediately precedes /e/, but not in the \isi{perfective} forms, in which the \isi{vowel} immediately precedes the \isi{suffix} \textit{-p} ‘\textsc{pfv}’.

\ea%71
    \label{ex:phon:71}
          Vowel \isi{elision} before the \isi{imperfective} \isi{suffix} \textit{-e} ‘\textsc{ipfv}’\\
\begin{tabbing}    
{(a.)} \= {(‘take [\textsc{ipfv}]’)} \= {(vs.)} \= {(h.)} \= {(‘take [\textsc{pfv}]’)}\kill
{a.} \> {[ase]} \> {vs.} \> {b.} \> {[asap]}\\
{ } \> {/as\textbf{a}-e/} \> { } \> { } \> {/asa-p/}\\
{ } \> {‘hit [\textsc{ipfv}]’} \> { } \> { } \> {‘hit [\textsc{pfv}]’}   
{(a.)} \= {(‘take [\textsc{ipfv}]’)} \= {(vs.)} \= {(h.)} \= {(‘take [\textsc{pfv}]’)}\kill
{c.} \> {[me]} \> {vs.} \> {d.} \> {[mep]}\\
{ } \> {/m\textbf{e}-e/} \> { } \> { } \> {/me-p/}\\
{ } \> {‘sew [\textsc{ipfv}]’} \> { } \> { } \> {‘sew [\textsc{pfv}]’}\\
{e.} \> {[ne]} \> {vs.} \> {f.} \> {[nip]}\\
{ } \> {/n\textbf{i}-e/} \> { } \> { } \> {/ni-p/}\\
{ } \> {‘act [\textsc{ipfv}]’} \> { } \> { } \> {‘act [\textsc{pfv}]’}\\
{g.} \> {[moke]} \> {vs.} \> {h.} \> {[mokop]}\\
{ } \> {/mok\textbf{o}-e/} \> { } \> { } \> {/moko-p/}\\
{ } \> {‘take [\textsc{ipfv}]’} \> { } \> { } \> {‘take [\textsc{pfv}]’}\\
{h.} \> {[ke]} \> {vs.} \> {i.} \> {[kïna]}\\
{ } \> {/k\textbf{ï}-e/} \> { } \> { } \> {/kï-na/}\\
{ } \> {‘say [\textsc{ipfv}]’} \> { } \> { } \> {‘say [\textsc{irr}]’}\footnotemark
\end{tabbing}
\z
\footnotetext{The perfective form of \textit{kï-} ‘say’ is irregular; see \tabref{tab:4.1} in \sectref{sec:4.2}, \tabref{tab:syntax:13} in \sectref{sec:13.4}.}

Among regular verbs, there are no environments in which to observe the \isi{deletion} of /u/ before /e/. The \isi{irregular verb} \textit{li-} ‘fall’, however, has a \isi{perfective} form /li-u/ [liyu], which, when followed by the \isi{imperfective} \isi{suffix} /-e/, is realized as [liye] -- that is, the underlying /u/ is \isi{deleted} before /\nobreakdash-e/.

  The examples in \REF{ex:phon:72} illustrate the loss of \isi{vowel}s before the \isi{nominalizing} \isi{suffix} \textit{-en} ‘\textsc{nmlz}’, which affixes to the \isi{verb stem}.

\ea%72
    \label{ex:phon:72}
          Vowel \isi{elision} before the \isi{nominalizing} \isi{suffix} \textit{-en} ‘\textsc{nmlz}’\\
\begin{tabbing}    
{(a.)} \= {(/inda-en/)} \= {(d.)} \= {(/wana-en/)}\kill
{a.} \> {[iten]} \> {b.} \> {[wanen]}\\
{ } \> {/it\textbf{a}-en/} \> { } \> {/wan\textbf{a}-en/}\\
{ } \> {‘builder’} \> { } \> {‘cook’}\\
{c.} \> {[inden]} \> {d.} \> {[unen]}\\
{ } \> {/ind\textbf{a}-en/} \> { } \> {/un\textbf{i}-en/}\\
{ } \> {‘walker’} \> { } \> {‘shouter’}
\end{tabbing}
\z


There are hardly any environments in which the \isi{mid vowel} /o/ immediately follows another \isi{vowel}. The \isi{vocative} \isi{interjection} \textit{=o} ‘\textsc{voc}’ (\sectref{sec:8.3.3}) may, however, affix to certain words, especially \isi{demonstrative}s, as in the examples given in \REF{ex:phon:73}, which demonstrate the loss of the \isi{vowel} /a/ before /o/.

\ea%73
    \label{ex:phon:73}
          Vowel \isi{elision} before the \isi{vocative} \isi{interjection} \textit{=o} ‘\textsc{voc}’\\
\begin{tabbing}    
{(a.)} \= {(‘those’)} \= {(vs.)} \= {(d.)} \= {(‘those [\textsc{voc}’)}\kill
{a.} \> {[anda]} \> {vs.} \> {b.} \> {[ando]}\\
{ } \> {/anda/} \> { } \> { } \> {/and\textbf{a}=o/}\\
{ } \> {‘that’} \> { } \> { } \> {‘that [\textsc{voc}]’}\\
{c.} \> {[ala]} \> {vs.} \> {d.} \> {[alo]}\\
{ } \> {/ala/} \> { } \> { } \> {/al\textbf{a}=o/}\\
{ } \> {‘those’} \> { } \> { } \> {‘those [\textsc{voc}]’}
\end{tabbing}
\z

Since all known examples of \isi{vowel elision} occurring before /o/ consist of the loss of /a/, the changes in \REF{ex:phon:73} may rather exemplify the process of \isi{central vowel} \isi{elision}, described in \sectref{sec:2.5.5}.

\is{vowel|)}
\is{phonology|)}
\is{phonetics|)}
\is{morphophonemic process|)}
\is{morphophonology|)}
\is{mid vowel|)}
\is{elision|)}
\is{vowel elision|)}

\subsection{Central vowel elision}\label{sec:2.5.5}

\is{central vowel elision|(}
\is{central vowel|(}
\is{vowel elision|(}
\is{elision|(}
\is{morphophonemic process|(}
\is{morphophonology|(}
\is{phonetics|(}
\is{phonology|(}
\is{vowel|(}

The \isi{central vowel}s /a, ï/ \isi{delete} when immediately followed by any other \isi{vowel} -- that is, \isi{central vowel}s \isi{elide} not only before \isi{mid vowel}s (\sectref{sec:2.5.4}), but whenever immediately followed by any other \isi{vowel} \REF{ex:phon:74}.

\ea%74
    \label{ex:phon:74}
          Central \isi{vowel} \isi{elision}

    V [-back, -front] → Ø / \_ V
\z

For example, the \isi{vowel}s /ï/ and /a/ are \isi{deleted} when they occur at the end of \isi{object-marker} \isi{proclitic}s that precede \isi{vowel}-initial \isi{verb stem}s \REF{ex:phon:75}.

\ea%75
    \label{ex:phon:75}
          Central \isi{vowel} \isi{elision} in \isi{object-marker} \isi{proclitic}s\\
\begin{tabbing}    
{(a.)} \= {(/nd\textbf{ï}=asa-/)} \= {(d.)} \= {(‘build them’)}\kill
{a.} \> {[asa-]} \> {b.} \> {[nasa-]}\\
{ } \> {/asa-/} \> { } \> {/n\textbf{ï}=asa-/}\\
{ } \> {‘hit’} \> { } \> {‘hit me’}\\
{c.} \> {[ndasa-]} \> {d.} \> {[masa-]}\\
{ } \> {/nd\textbf{ï}=asa-/} \> { } \> {/m\textbf{a}=asa-/}\\
{ } \> {‘hit them’} \> { } \> {‘hit it’}\\
{e.} \> {[ita-]} \> {f.} \> {[ndita-]}\\
{ } \> {/ita-/} \> { } \> {/nd\textbf{ï}=ita-/}\\
{ } \> ‘build’ \> { } \> {‘build them’}
\end{tabbing}
\z

Note that this rule must be ordered after the \isi{glide formation} rule (\sectref{sec:2.5.1}), which bleeds the otherwise possible change of \textsuperscript{†}/ai/ → [a]. Thus the form /ma=ita-/ ‘build it’ is pronounced as [mayta-] and not as \textsuperscript{†}[mita-].\footnote{Alternatively, it could be argued that the \isi{deletion} of /ï/ and /a/ only takes place before certain \isi{vowel}s, but not before /i/, thus not requiring rule ordering.}

  The fact that the two \isi{central vowel}s in this language pattern distinctly both from \isi{front vowel}s and from \isi{back vowel}s may support the use of a distinctive feature [+/$-$front] in addition to the traditional feature [+/$-$back] (such that a \isi{central vowel} /a, ï/ may be described as [$-$front, $-$back]). Alternatively, however, the feature [+/$-$front] could perhaps be avoided, if this rule is divided into two separate rules \REF{ex:phon:76}. In such an analysis, there would be one \isi{vowel} \isi{degemination} (or shortening) rule, and one /ï/-\isi{elision} rule.

\ea%76
    \label{ex:phon:76}
          A two-rule analysis of \isi{central vowel} \isi{elision}

    \ea  V\textsubscript{i} → Ø / \_ V\textsubscript{i}

    \ex  /ï/ → Ø / \_ V
    \z
\z

Combined with the \isi{glide formation} rule (\sectref{sec:2.5.1}), the first rule in \REF{ex:phon:76} would account for all alternations involving /a/. Since the only \isi{vowel}s permitted in \isi{onset} are /i, u, a/, the only possible \is{low vowel} low-vowel-initial \isi{vowel} combinations would be \textsuperscript{†}/ai/, \textsuperscript{†}/au/, and \textsuperscript{†}/aa/. While the first two sequences would be \isi{diphthong}ized (or become \isi{vowel}-\isi{glide} sequences), the third sequence would undergo the reduction (\isi{deletion} of one \isi{vowel}) suggested by the first rule in \REF{ex:phon:76}. Thus the \isi{vowel} \isi{elision} rule would only need to apply to the high \isi{central vowel}. In this analysis, /ï/ behaves uniquely among \isi{vowel}s. Perhaps this analysis is preferable, considering the distinct behavior of /ï/ (\sectref{sec:2.2.1}).

\is{vowel|)}
\is{phonology|)}
\is{phonetics|)}
\is{morphophonology|)}
\is{morphophonemic process|)}
\is{elision|)}
\is{vowel elision|)}
\is{central vowel|)}
\is{central vowel elision|)}



\subsection{High central vowel assimilation}\label{sec:2.5.6}

\is{central vowel|(}
\is{vowel assimilation|(}
\is{assimilation|(}
\is{high central vowel assimilation|(}
\is{morphophonology|(}
\is{morphophonemic process|(}
\is{phonetics|(}
\is{phonology|(}
\is{vowel|(}
\is{high vowel|(}

\is{high vowel}
\is{back vowel}

Another case in which /ï/ behaves uniquely involves the presence of \isi{glide}s. Immediately before the high back \isi{glide} /w/, this \isi{vowel} often \is{assimilation} assimilates in both roundness and backness, being realized as the high back \isi{rounded} \isi{vowel} [u]. This rule \REF{ex:phon:77} is optional for most speakers.

\ea%77
    \label{ex:phon:77}
          High \isi{central vowel} \isi{assimilation}

    /ï/ → ([u]) / \_ w (optional)
\z

This \isi{assimilation} may be illustrated by verb forms containing \isi{glide}-initial \isi{stem}s \REF{ex:phon:78}.

\ea%78
    \label{ex:phon:78}
          High \isi{central vowel} \isi{assimilation} before \isi{glide}-initial verb stems\\
\begin{tabbing}    
{(a.)} \= {(‘hit you [\textsc{sg}]’)} \= {(vs.)} \= {(d.)} \= {(/ndï=wana-/)}\kill
{a.} \> {[mawana-]} \> {vs.} \> {b.} \> {[nd\textbf{u}wana-]}\\
{ } \> {/ma=wana-/} \> { } \> { } \> {/ndï=wana-/}\\
{ } \> {‘cook it’} \> { } \> { } \> {‘cook them’}\\
{c.} \> {[uwali-]} \> {vs.} \> {d.} \> {[n\textbf{u}wali-]}\\
{ } \> {/u=wali-/} \> { } \> { } \> {/nï=wali-/}\\
{ } \> {‘hit you [\textsc{sg}]’} \> { } \> { } \> {‘hit me’}\\
{e.} \> {[minwe-]} \> {vs.} \> {f.} \> {[nd\textbf{u}we-]}\\
{ } \> {/min=we-/} \> { } \> { } \> {/ndï=we-/}\\
{ } \> {‘cut two’} \> { } \> { } \> {‘cut them’}
      \end{tabbing}
\z

This \isi{assimilation} is most likely primarily one of \isi{rounding} and not backness (with backness tagging along, since the only available \is{high vowel} high \isi{rounded} \isi{vowel} in the language is also [+back]). If, however, this were a case of place-\isi{assimilation} and not \isi{rounding}-\isi{assimilation}, then it might be expected that the \is{high vowel} high \isi{central vowel} /ï/ would also \is{assimilation} assimilate in place to a following high front \isi{glide} /y/, as hypothesized in \REF{ex:phon:79}.

\ea%79
    \label{ex:phon:79}
          A possible place \isi{assimilation} analysis (?)

    /ï/ → [i] / \_ y (?)
\z

Although the sequence /ïy/ never occurs within a single word, it is possible for one word ending in /-ï/ to precede another beginning with /-y/, as in the examples in \REF{ex:phon:80}.

\newpage

\ea%80
    \label{ex:phon:80}
          Sequences of /ï/ + /y/ (across word boundaries)

    \ea  \textit{nï ya} ‘I … coconut …’
    \ex  \textit{mï yana} ‘he … woman …’
    \z
\z

Crucially, this sequence is never pronounced [iy]. That is, there is no place \isi{assimilation} of /ï/ preceding a high front \isi{glide}. Thus, the \isi{rounding} \isi{assimilation} analysis of /ï/ → [u] is preferable.

\is{high vowel|)}
\is{vowel|)}
\is{phonology|)}
\is{phonetics|)}
\is{morphophonemic process|)}
\is{morphophonology|)}
\is{high central vowel assimilation|)}
\is{assimilation|)}
\is{vowel assimilation|)}
\is{central vowel|)}

\subsection{Local vowel assimilation of /a/ to /o/}\label{sec:2.5.7}

\is{local assimilation|(}
\is{vowel assimilation|(}
\is{assimilation|(}
\is{local vowel assimilation|(}
\is{morphophonemic process|(}
\is{morphophonology|(}
\is{phonetics|(}
\is{phonology|(}
\is{vowel|(}

All of the rules mentioned so far (\sectref{sec:2.5.1}--\sectref{sec:2.5.6}), which have been shown to apply within phonological words, may also apply across word boundaries, and thus seem to reflect general phonetic preferences in the languages. Accordingly, \isi{gliding} often occurs in rapid speech when a word ending in a \isi{low vowel} /a/ is immediately followed by a word beginning with a high non-\isi{central vowel} /i, u/. Likewise, the \isi{elision} of \isi{central vowel}s, the \isi{gliding} of /u/ to [w], and the \isi{deletion} of vowels immediately preceding identical \isi{vowel}s can occur across word boundaries, all of which are illustrated in the examples in \REF{ex:phon:81}.

\ea%81
    \label{ex:phon:81}
          Some phonological processes occurring across word boundaries

    \ea  {}[n\textbf{a}mun]\\
      \gll \normalfont/n\textbf{ï}    \normalfont\textbf{a}mun/\\
      1\textsc{sg}  today\\
      \glt ‘Today, I …’

    \ex  {}[\textbf{w}amun]\\
      \gll \normalfont/\textbf{u}    \normalfont\textbf{a}mun/\\
      2\textsc{sg}  today\\
      \glt ‘Today, you …’

    \ex  {}[\textbf{u}mbe]\\
      \gll \normalfont/\textbf{u}    \normalfont\textbf{u}mbe/\\
      2\textsc{sg}  tomorrow\\
      \glt ‘Tomorrow, you …’
      \z
\z

Processes such as these are generally more likely to occur when one of the elements involved is a \isi{clitic} or an \isi{affix}, and this may perhaps be the case in \REF{ex:phon:81}, if in fact all pronominal markers in Ulwa are analyzed as \isi{clitic}s. Nevertheless, these alternations are still possible with full \isi{lexical} items, suggesting a strong phonetic basis for these phonological rules.

  At least one change, however, is extremely limited in its scope, and is perhaps purely \isi{morphological}ly conditioned. This is the change of underlying /a/ to [o]. It has only been observed to occur with one morpheme, the 3\textsc{sg} \isi{object marker} /ma=/. The \isi{allomorphy} of this morpheme when it is followed by a high non-\isi{central vowel} is discussed in (\sectref{sec:2.5.1}) (i.e., it can be realized as [may=] or [maw=]). Before another \isi{low vowel} (\sectref{sec:2.5.5}.), it has the \isi{allomorph} [m=]. The \isi{allomorph} [mo=] occurs when the \isi{proclitic} is immediately followed by a \isi{syllable} containing a mid \isi{back vowel} /o/ \REF{ex:phon:82}.

\ea%82
    \label{ex:phon:82}
          Local \isi{vowel} \isi{assimilation} of /a/ to /o/

    /a/ → [o] / \_ C\textsubscript{0}V [-high, +back], when in the \isi{proclitic} /ma=/
\z

Thus, instead of \textsuperscript{†}[ma=], the surface form [mo=] is found when this 3\textsc{sg} \isi{object marker} immediately precedes \isi{verb stem}s that create this environment \REF{ex:phon:83}.

\ea%83
    \label{ex:phon:83}
          Local \isi{vowel} \isi{assimilation} of /a/ to /o/ in the 3\textsc{sg} \isi{object marker} /ma=/
\begin{tabbing}    
{(a.)} \= {(/ma=moplï-/)} \= {(d.)} \= {(/ma=poplï-/)}\kill
{a.} \> {[m\textbf{o}kot-]} \> {b.} \> {[m\textbf{o}toplï-]}\\
{ } \> {/ma=kot-/} \> { } \> {/ma=toplï-/}\\
{ } \> {‘break it’} \> { } \> {‘throw it’}\\
{c.} \> {[m\textbf{o}moplï-]} \> {d.} \> {[m\textbf{o}poplï-]}\\
{ } \> {/ma=moplï-/} \> { } \> {/ma=poplï-/}\\
{ } \> {‘tie it’} \> { } \> {‘sweep it’}
\end{tabbing}
\z

No other \isi{vowel} in the following \isi{syllable} will condition this change; nor does any similar process occur in the \isi{object marker}s containing other vowels (/i, ï, u/), as illustrated by the examples in \REF{ex:phon:84}.

\ea%84
    \label{ex:phon:84}
          Lack of \isi{vowel} \isi{assimilation} with other \isi{object marker}s\\
\begin{tabbing}    
{(a.)} \= {(‘sweep them’)} \= {(d.)} \= {(‘throw you [\textsc{sg}]’)}\kill
{a.} \> {[m\textbf{i}nkot-]} \> {b.} \> {[\textbf{u}toplï-]}\\
{ } \> {/min=kot-/} \> { } \> {/u=toplï-/}\\
{ } \> {‘break two’} \> { } \> {‘throw you [\textsc{sg}]’}\\
{a.} \> {[\textbf{u}nmoplï-]} \> {b.} \> {[nd\textbf{ï}poplï-]}\\  
{ } \> {/un=moplï-/} \> { } \> {/ndï=poplï-/}\\
{ } \> {‘tie you [\textsc{pl}]’} \> { } \> {‘sweep them’}
\end{tabbing}  
\z

It is also worth noting that no similar process affects other \isi{object marker}s with the same \isi{vowel} /a/, as illustrated by the examples in \REF{ex:phon:85}.

\newpage

\ea%85
    \label{ex:phon:85}
          Lack of \isi{vowel} \isi{assimilation} with other \isi{object marker}s ending in /a-/\\
\begin{tabbing}
{(a.)} \= {(/ala=moplï-/)} \= {(d.)} \= {(‘throw us [\textsc{excl}]’)}\kill
{a.} \> {[and\textbf{a}kot-]} \> {b.} \> {[\textbf{a}ntoplï-]}\\
{ } \> {/anda=kot-/} \> { } \> {/an=toplï-/}\\
{ } \> {‘break that’} \> { } \> {‘throw us [\textsc{excl}]’}\\
{c.} \> {[al\textbf{a}moplï-]} \> {d.} \> {[ng\textbf{a}poplï-]}\\
 { } \> {/ala=moplï-/} \> { } \> {/nga=poplï-/}\\
{ } \> {‘tie those’} \> { } \> {‘sweep this’}     
\end{tabbing}
\z

The result is that this process is restricted to the single morpheme /ma=/. It could thus be argued that this process is \isi{morphological}ly conditioned. If, however, it is indeed a phonologically conditioned process after all, then the most likely explanation is that it is the labiality of the /m/ combined with the presence of /o/ in the following \isi{syllable} that together cause /ma/ to become [mo], as suggested by the rule in \REF{ex:phon:86}.

\ea%86
    \label{ex:phon:86}
          possible phonological motivation for /ma=/ → [mo]

    /a/ → [o] / C [+\isi{labial}] \_ C\textsubscript{0}V [$-$high, +back]
\z

This very well may be the case. It is, of course, difficult to argue from absence of evidence, but I have found no words containing the \isi{low vowel} /a/ immediately following a \isi{labial} \isi{consonant} /p, mb, m/ and preceding a \isi{syllable} with the mid \isi{back vowel} /o/ (i.e., the sequences \textsuperscript{†}[paCo, mbaCo, maCo] all appear to be disallowed). It is thus possible that this rule is not actually \isi{morphological}ly conditioned, but rather applies to every environment in which an underlying /a/ immediately follows a \isi{labial} \isi{consonant} and precedes a mid \isi{back vowel} in the next \isi{syllable} – that is, the preceding \isi{labial} and the following \isi{rounded} \isi{vowel} conspire to condition the change.

\is{vowel|)}
\is{phonology|)}
\is{phonetics|)}
\is{morphophonology|)}
\is{morphophonemic process|)}
\is{local vowel assimilation|)}
\is{assimilation|)}
\is{vowel assimilation|)}
\is{local assimilation|)}

\subsection{Degemination}\label{sec:2.5.8}

\is{geminate|(}
\is{degemination|(}
\is{morphophonology|(}
\is{morphophonemic process|(}
\is{phonetics|(}
\is{phonology|(}
\is{consonant|(}

There is a process in Ulwa by which geminate \isi{consonant}s are reduced to single segments \REF{ex:phon:87}.

\ea%87
    \label{ex:phon:87}
          Degemination

    C\textsubscript{i} → Ø / \_ C\textsubscript{i}
\z

This is mostly observed across word boundaries in rapid speech. There are few instances in which identical \isi{consonant}s would occur underlyingly across a morpheme boundary, but the \isi{oblique marker} \textit{=n} ‘\textsc{obl}’ (\sectref{sec:11.4.1}) can follow words ending in /-n/. Although this marker has the \isi{allomorph} [ïn], which could bleed a possible \isi{degemination}, it is also possible for the sequence of /n=n/ to reduce to a single [n], as seen in \REF{ex:phon:88}.

\ea%88
    \label{ex:phon:88}
          Degemination with the \isi{oblique marker} \textit{=n} ‘\textsc{obl}’
\begin{tabbing}
{(a.)} \= {(‘with the head’)} \= {(d.)} \= {(‘with you [\textsc{pl}]’)}\kill
{a.} \> {[tï\textbf{n}]} \> {b.} \> {[u\textbf{n}]}\\
{ } \> {/tï\textbf{n=n}/} \> { } \> {/u\textbf{n=n}/}\\
{ } \> {‘with the dog’} \> { } \> {‘with you [\textsc{pl}]’}\\
{c.} \> {[unduwa\textbf{n}]} \> {d.} \> {[ngi\textbf{n}]}\\
{ } \> {/unduwa\textbf{n=n}/} \> { } \> {/ngi\textbf{n=n}/}\\
{ } \> {‘with the head’} \> { } \> {‘with the net’}
\end{tabbing}
\z

It is also possible to witness \isi{degemination} within words when certain \isi{separable verb}s (\sectref{sec:9.2.1}) occur with their elements unseparated \REF{ex:phon:89}.

\ea%89
    \label{ex:phon:89}
          Degemination with unseparated \isi{separable verb} forms\\
\begin{tabbing}
{(a.)} \= {(‘bend [\textsc{pfv}]’)} \= {(vs.)} \= {(b.)} \= {(/tumu\textbf{l-l}a-ka-na/)}\kill
{a.} \> {[tumulka]} \> {vs.} \> {b.} \> {[tumu\textbf{l}akana]}\\
{ } \> {/tumul-ka/} \> { } \> { } \> {/tumu\textbf{l-l}a-ka-na/}\\
{ } \> {‘bend [\textsc{pfv]}’} \> { } \> { } \> {‘bend [\textsc{irr]}’}
\end{tabbing}
\z

Degemination also occurs between \isi{object-marker} \isi{proclitic}s that end in /-n/ and \isi{verb stem}s that begin with /n-/, as illustrated by the examples in \REF{ex:phon:90}.

\ea%90
    \label{ex:phon:90}
          Degemination with \isi{object-marker} \isi{proclitic}s ending in /n/\\
\begin{tabbing}
{(a.)} \= {(‘give you [\textsc{pl}] [\textsc{pfv}]’)} \= {(d.)} \= {(‘give us [\textsc{excl}] [\textsc{pfv}]’)}\kill
{a.} \> {[u\textbf{n}an]} \> {b.} \> {[a\textbf{n}an]}\\
{ } \> {/u\textbf{n=n}a-n/} \> { } \> {/a\textbf{n=n}a-n/}\\
{ } \> {‘give you [\textsc{pl}] [\textsc{pfv}]’} \> { } \> {‘give us [\textsc{excl}] [\textsc{pfv}]’}\\
{c.} \> {[u\textbf{n}ip]} \> {d.} \> {[a\textbf{n}ip]}\\
{ } \> {/u\textbf{n=n}i-p/} \> { } \> {/a\textbf{n=n}i-p/}\\
{ } \> {‘beat you [\textsc{pl}] [\textsc{pfv}]’} \> { } \> {‘beat us [\textsc{excl}] [\textsc{pfv}]’}
\end{tabbing}
\z

Other instances of \isi{degemination} occur when the \isi{conditional} \isi{suffix} \textit{-ta} ‘\textsc{cond}’ (\sectref{sec:4.12}) immediately follows a verb form ending in /t/ (e.g., /at-ta/ ‘hit [\textsc{cond}]’ → [ata]) and when the \isi{copular enclitic} \textit{=p} ‘\textsc{cop}’ (\sectref{sec:10.3}) immediately follows a form ending in /p/ (e.g., /awlop=p/ ‘be in vain’ → [awlop]).

\is{consonant|)}
\is{phonology|)}
\is{phonetics|)}
\is{morphophonemic process|)}
\is{morphophonology|)}
\is{degemination|)}
\is{geminate|)}

\subsection{Lexically determined alternations and rules}\label{sec:2.5.9}

\is{phonology|(}
\is{phonetics|(}
\is{lexicon|(}

A few interesting \isi{lexical}ly determined phonological alternations or rules may also be noted. Some common words vary between two pronunciations, even within the speech of individual speakers. Thus, ‘woman’ may be pronounced either as [yena] or as [yana] and, similarly, ‘man’ may be pronounced either as [yeta] or as [yata].

  There may also be \isi{dialect}al differences, even within the rather small \ili{Manu} \isi{dialect}, which is the focus of this grammatical description. For example, some speakers of \ili{Manu} Ulwa use the form \textit{angla} ‘awaiting’, whereas other speakers use the form \textit{andïla} ‘awaiting’ for the same \isi{postposition}al meaning.

  There also appear to be generational differences. For example, older speakers of the \ili{Manu} \isi{dialect} prefer the form \textit{namndu} ‘pig’, whereas younger speakers prefer \textit{lamndu} ‘pig’. Indeed, the form \textit{namndu} ‘pig’ is also used by speakers (of all ages) of the \ili{Maruat-Dimiri-Yaul} \isi{dialect}. Although there are often correspondences of \textit{l} : \textit{n} between the two \isi{dialect}s, they usually occur in the opposite manner -- that is (when there is a difference between the two \isi{dialect}s), typically an /l/ in the \ili{Maruat-Dimiri-Yaul} \isi{dialect} corresponds to an /n/ in the \ili{Manu} \isi{dialect}.\footnote{The form \textit{lamndu} ‘pig’ may thus be a recent innovation of the \ili{Manu} \isi{dialect}, the result perhaps either of \isi{hypercorrection} or of \isi{folk etymology}, in this case based on a perceived connection between \textit{namndu/lamndu} ‘pig’ and \textit{lam} ‘meat’, itself a \isi{loanword} from \ili{Ap Ma} (\sectref{sec:1.5.6}). \chapref{sec:18} contains further discussion of the \ili{Maruat-Dimiri-Yaul} \isi{dialect} of Ulwa.}

  The verb \textit{lï-} ‘put’ shows great variability. It may even be the case that, for some speakers, the \isi{stem}-final \isi{vowel} /ï/ is underlyingly /u/; and that, for some other speakers, this \isi{vowel} is underlyingly /i/ -- at least it is realized as such by these speakers, at least in some environments. Often the \isi{vowel} is lost entirely when the root immediately follows a \isi{vowel} and immediately precedes /-p/, as occurs with the \isi{perfective} \isi{suffix} \textit{-p} ‘\textsc{pfv}’ (\sectref{sec:4.5}). This may be seen in the contrasts presented in \REF{ex:phon:91}: the \isi{vowel} is lost following \textit{ma=} ‘3\textsc{sg.obj}’ (a.) and \textit{ndï=} (b.) ‘3\textsc{pl}’ in the \isi{perfective} forms, but not following \textit{min=} ‘\textsc{3du}’ (c.) and not in the \isi{irrealis} forms (d., e., f.).

\ea%91
    \label{ex:phon:91}
          Loss of [ï] in \isi{perfective} forms of \textit{lï-} ‘put’\\
\begin{tabbing}
{(a.)} \= {(‘put two [\textsc{pfv]}’)} \= {(d.)} \= {(‘put them [\textsc{pfv]}’)}\kill
{a.} \> {[malp]} \> {b.} \> {[ndïlp]}\\
{ } \> {/ma=l\textbf{ï}-p/ } \> { } \> {/ndï=l\textbf{ï}-p/ }\\
{ } \> {‘put it [\textsc{pfv]}’} \> { } \> {‘put them [\textsc{pfv]}’}\\
{c.} \> {[minl\textbf{ï}p]} \> {d.} \> {[mal\textbf{ï}nda]}\\
{ } \> {/min=lï-p/} \> { } \> {/ma=lï-nda/}\\
{ } \> {‘put two [\textsc{pfv]}’} \> { } \> {‘put it [\textsc{irr]}’}\\
{e.} \> {[minl\textbf{ï}nda]} \> {f.} \> {[ndïl\textbf{ï}nda]}\\
{ } \> {/min=lï-nda/ } \> { } \> {/ndï=lï-nda/}\\
{ } \> {‘put two [\textsc{irr]}’} \> { } \> {‘put them [\textsc{irr]}’}
\end{tabbing}
  \z

The \isi{deletion} that occurs in \REF{ex:phon:91} is an obligatory rule for many speakers. For some speakers, there is additionally an optional rule to \isi{delete} this same \isi{vowel} /ï/ in the \isi{irrealis} forms as well, provided that there is a \isi{vowel} immediately preceding the \isi{verb stem}. Thus, /malïnda/ (d.) may at times be pronounced as [malnda]; however, /minlïnda/ (e.) would never be pronounced as \textsuperscript{†}[minlndna], as this would create a \is{phonotactics} phonotactically forbidden cluster.

\is{lexicon|)}
\is{phonetics|)}
\is{phonology|)}

\section{Metathesis}\label{sec:2.6}

\is{metathesis|(}
\is{phonetics|(}
\is{phonology|(}

Sometimes speakers invert the order of two phonological segments in a word. Some instances of this sort of \isi{metathesis} may be simple \isi{speech error}s, but others reflect \isi{free variation} in the pronunciation of certain words or of certain combinations of phonemes. This latter class does not seem to show any phonological or \isi{morphological} conditioning (hence the designation as \isi{free variation}).

  Example \REF{ex:phon:92} illustrates two instances of sporadic \isi{metathesis}, which may represent \isi{speech error}s. The first exhibits the local reversal of two \isi{consonant}s. The second exhibits the change in order of \isi{consonant} and \isi{vowel} (with subsequent \isi{elision} of /ï/).

\ea%92
    \label{ex:phon:92}
          Sporadic \isi{metathesis}

    \ea  {}[amnopa] for /anmopa/ ‘\textit{tulip} greens’ [ulwa029\_01:06]

    \ex  {}[ndumne] for /ndïmune/ ‘throw them’ [ulwa035\_00:23]
    \z
\z

Unlike these somewhat unusual changes, however, there is one word for which \isi{metathesis} is rather common. The \isi{postposition} \textit{ul} ‘with’ is often realized as [lu]. Perhaps counterintuitively, this \isi{allomorphy} generally occurs when the \isi{postposition} immediately follows a \isi{consonant} (as opposed to a \isi{vowel}), resulting in a sequence of two \isi{consonant}s (across a \isi{syllable} boundary). Some examples of metathesized \textit{ul} ‘with’ are given in \REF{ex:phon:94}.

\ea%94
    \label{ex:phon:94}
          Metathesis of \textit{ul} ‘with’ to [lu] following a \isi{consonant}\\
    \ea  \textit{ambin lu} ‘with each other’ [ulwa032\_17:31]

    \ex  \textit{Yaklap lu} ‘with Yaklap’ [ulwa014\_32:00]

    \ex  \textit{Nicholas lu} ‘with Nicholas’ [ulwa018\_00:14]
    \z
\z

The metathesized version also sometimes occurs even when following a \isi{vowel}s provided the object of the \isi{postposition} is not a \isi{clitic} (\isi{object marker}s \textit{ma=} ‘3\textsc{sg.obj}’ and \textit{ndï=} ‘3\textsc{pl}’, for example, never license the metathesized form, but are rather always followed by [ul]). Two examples of \isi{metathesis} in this environment are provided in \REF{ex:phon:95}.

\newpage

\ea%95
    \label{ex:phon:95}
          Metathesis of \textit{ul} to \textit{lu} following a \isi{vowel}

    \ea  \textit{mangusuwa lu} ‘with the poor thing’ [ulwa037\_46:27]

    \ex  \textit{Tema lu} ‘with Tema’ [ulwa037\_60:27]
    \z
\z

Where these two phonemes /l/ and /u/ occur elsewhere in succession, it is also possible (although not necessarily common) to metathesize them \REF{ex:phon:96}.

\ea%96
    \label{ex:phon:96}

          Other examples of [ul] {\textasciitilde} [lu] \isi{metathesis}

    \ea{}  [ulwa] for /luwa/ ‘place’ [ulwa032\_50:31]

    \ex{}  [ulke] for /luke/ ‘too’ [ulwa037\_06:57]

    \ex{}\label{ex:phon:96c}  [nolnda] for /na-lu-nda/ ‘will put’ [ulwa038\_02:53]
    \z
\z

In example \REF{ex:phon:96c} the \isi{metathesis} must precede a \isi{monophthongization} process (\sectref{sec:2.5.2}) -- that is, /nalunda/ > /naulnda/ > [nolnda].

\is{phonology|)}
\is{phonetics|)}
\is{metathesis|)}

\section{Phonetics and phonology of connected speech}\label{sec:2.7}

\is{connected speech|(}
\is{phonetics|(}
\is{phonology|(}

This chapter may be concluded with an impressionistic note on the sounds of \isi{connected speech} in Ulwa. Similar to the \isi{synalepha} found in languages like \ili{Spanish}, there is a tendency in Ulwa for words to blend together, such that it is often impossible (on phonetic grounds) to separate one word from the next. Specifically, sound changes such as \isi{elision} and \isi{coalescence} of \isi{vowel}s at word boundaries are common in rapid speech. Sometimes two \isi{vowel}s \isi{coalesce} when an \isi{epenthetic} \isi{glide} might otherwise be expected. For example, the sequence /i\#a/ may be realized as [e] instead of [iya], as in \REF{ex:phon:97}.

\ea%97
    \label{ex:phon:97}
[ambenda]\\
/ambi anda/\\
‘that big [man]’\footnote{The form [ambiyanda], with an \isi{epenthetic} \isi{glide}, would also be possible.}\\
  \z

\is{paragoge}

  Finally, some phonological phenomena are only observable at the utterance level. For example, many speakers employ an occasional utterance-final \linebreak \isi{epenthetic} (i.e., paragogic) \isi{alveolar} \isi{nasal} /n/. This can cause confusion between \isi{nominalize}d verb forms, which end in /-en/ (\sectref{sec:3.2}), and verbs that end (underlyingly) with the \isi{imperfective} \isi{suffix} /-e/ (\sectref{sec:4.4}) or the \isi{dependent marker} /-e/ (\sectref{sec:12.2.1}), but which also take this utterance-final \isi{epenthetic} [-n], resulting in the \isi{homophonous} ending [-en].

\is{phonology|)}
\is{phonetics|)}
\is{connected speech|)}