\chapter{Verbs}\label{sec:4}

\is{verb|(}

This chapter is dedicated to the description and analysis of verbs in Ulwa. Verbs constitute the \isi{part of speech} that exhibits the most \isi{inflection}, variability, and irregularity. Non-verbal lexemes may function like verbs (that is, they may occur at the end of a clause and fulfill the role of \isi{predicate} of the clause), but only true, underived verbs receive \isi{verbal morphology}. Verbs in Ulwa are thus the words that can be \isi{inflect}ed for the full range of \isi{tense}-\isi{aspect}-\isi{mood} (\isi{TAM}) distinctions in the language \REF{ex:verbs:1a}.\footnote{Here and throughout the book, the abbreviation “\isi{TAM}” is used, largely due to its ubiquity in contemporary grammatical description. \isi{Tense}, however, plays a minimal role in Ulwa \isi{verbal inflection}. Verbs are primary \isi{inflect}ed for distinctions in \isi{aspect} and \isi{mood} (\sectref{sec:4.2}).}

\ea%1a
    \label{ex:verbs:1a}
            Verbs exhibiting \isi{verbal morphology} (\isi{TAM} \isi{suffix}es)\\
\begin{tabbing}
{(\textit{moko-na})} \= {(‘break (\isi{imperfective}))}’\kill
\textit{kol-e} \> {‘break (\isi{imperfective})’}\\
\textit{ita-p} \> {‘build (\isi{perfective})’}\\
\textit{moko-na} \> {‘take (\isi{irrealis})’}
\end{tabbing}
 \z

  There is no \isi{morphological} indexing of subjects (i.e., S or A arguments) on verbs. \isi{Transitive} verbs, however, may be preceded by an \isi{object-marker} \isi{proclitic} (\sectref{sec:7.2}), which may index \isi{direct object}s (i.e., P arguments) \REF{ex:verbs:1b}. These markers can occur with verbs marked for any \isi{TAM} category; they can occur both in \isi{positive} and in \isi{negative} sentences. No \isi{animacy} distinction is found in verbs: \isi{animate} and \isi{inanimate} P arguments are indexed alike.

\ea%1b
    \label{ex:verbs:1b}
            Verbs exhibiting \isi{object-marker} \isi{proclitic}s\\
\begin{tabbing}
{(\textit{ndï=moko-})} \= {(‘take them’)}\kill
{\textit{ma=wali-}} \> {‘hit it’}\\
{\textit{min=lo-}} \> {‘cut two’}\\
{\textit{ndï=moko-}} \> {‘take them’}
\end{tabbing}
 \z

  Although \isi{verb phrase}s may consist of more than one word, a typical unmarked \isi{indicative} clause contains exactly one \isi{inflect}ed verb, which occurs at the end of the clause. In a \isi{transitive} clause, the \is{direct object} (direct) object immediately precedes the verb \REF{ex:verbs:1c}.

\ea%1c
    \label{ex:verbs:1c}
            \textit{Ankam mï tïn \textbf{masap}.}\\
            \gll    ankam  mï      tïn ma=\textbf{asa}-p\\
    person  3\textsc{sg.subj}  dog  3\textsc{sg.obj}=hit-\textsc{pfv}\\
\glt `The person hit the dog.’ [elicited]
\z

  There are no \isi{auxiliary verb}s in Ulwa, at least none that serve as dedicated, productive, and obligatory means of providing grammatical marking. On the \isi{grammaticalization} of ‘go’ to refer to \isi{future} \isi{time}, however, see \sectref{sec:10.3}; and on the perhaps recently innovated use of the \isi{locative verb} \textit{wap} ‘be.\textsc{pst’} as an \isi{auxiliary verb} (or \isi{particle}), see \sectref{sec:15.3}.

  A verb consists, minimally, of a \isi{stem} (\sectref{sec:4.1}), to which an \isi{inflect}ional \isi{TAM} \isi{suffix} may be added (\sectref{sec:4.2}). To any of the \isi{TAM} \isi{suffix}es, the \isi{dependent-marker} \isi{suffix} \textit{{}-e} ‘\textsc{dep}’ may be added as well (\sectref{sec:12.2.1}).

\is{verb|)}

\section{The verb stem}\label{sec:4.1}

\is{verb|(}
\is{verb stem|(}
\is{stem|(}

Monomorphemic verbs in Ulwa (i.e., verbs that are not \isi{compound}s), are generally \isi{disyllabic} or, less commonly, \isi{monosyllabic}. \isi{Verb stem}s generally end in \isi{vowel}s \REF{ex:verbs:1d}.

\ea%1d
    \label{ex:verbs:1d}
            A selection of \isi{vowel}-final \isi{verb stem}s\\
\begin{tabbing}
{(\textit{lopo-})} \= {(‘put in’)}\kill
{\textit{ita-}} \> {‘build’}\\
{\textit{na-}} \> {‘give’}\\
{\textit{ale-}} \> {‘scrape’}\\
{\textit{ne-}} \> {‘harvest’}\\
{\textit{uni-}} \> {‘shout’}\\
{\textit{i-}} \> {‘come’}\\
{\textit{lopo-}} \> {‘rain’}\\
{\textit{wo-}} \> {‘sleep’}\\
{\textit{u-}} \> {‘put’}\\
{\textit{kï-}} \> {‘say’}
\end{tabbing}
 \z

However, for most verbs, this final \isi{vowel} is lost in the \isi{imperfective} form, which takes the \isi{suffix} \textit{{}-e} ‘\textsc{ipfv}’. There is also a small set of verbs that end in \isi{consonant}s. Interestingly, these verbs almost all belong to the same \isi{semantic} domain of cutting, splitting, breaking, and so on \REF{ex:verbs:1e}.

\newpage

\ea%1e
    \label{ex:verbs:1e}
            Verbs with \isi{consonant}-final stems\\
\begin{tabbing}
{(\textit{lomon-})} \= {(‘break; bear, give birth’)}\kill
\textit{kol-} \> {‘break, split’}\\
\textit{tukul-} \> {‘break (\isi{intransitive})’}\\
\textit{kun-} \> {‘break, break off’}\\
\textit{lomon-} \> {‘ignite, set fire to’}\\
\textit{won-} \> {‘cut, cross’}\\
\textit{klop-} \> {‘cross, pass’}\\
\textit{kot-} \> {‘break; bear, give birth’}\\
\textit{kamb-} \> {‘shun, avoid’}
\end{tabbing}
 \z

Some ‘breaking’ verbs may derive from collocations with the \isi{indefinite} marker \textit{ko=} ‘\textsc{indf}’.\footnote{Thus, for example, \textit{kol-} ‘break’ < \textit{ko=} ‘\textsc{indf}’ + \textit{l[ï]-} ‘put’ (i.e., ‘put a [piece]’), or \textit{kot-} ‘break' < \textit{ko=} ‘\textsc{indf}’ + \textit{t[ï]-} ‘take’ (i.e., ‘take a [piece]’). This \isi{compound}ing is in part suggested by the fact that Ulwa generally lost \ili{Proto-Keram} *k- when word-initial in \isi{polysyllabic} words, but preserved it in \isi{monosyllabic} forms, such as \textit{ko=} ‘\textsc{indf}’. Also, it should be noted that the verb \textit{kot-} ‘break, bear’ is analyzed here as ending in /t-/ (and not \textsuperscript{†}/ï-/), despite often exhibiting the \isi{perfective} form [kotïp] and the \isi{irrealis} form [kotïna]. The presence of [ï] in these forms, however, is taken to be \isi{phonetic}ally motivated -- that is, it is an \isi{epenthetic} \isi{vowel} breaking up the \isi{consonant cluster}. Indeed, this \isi{vowel} can be avoided in the \isi{perfective} form in instances in which the /tp/ sequences can be broken across a \isi{syllable} boundary (e.g., in the \isi{dependent-marked} form [kot.pe]). Moreover, the \isi{conditional} form of the verb \textit{kot-} ‘break, bear’ is [kota] (not \textsuperscript{†}[kotïta]), from underlying /kot-ta/ -- in other words, instead of acquiring an \isi{epenthetic} \isi{vowel}, the /tt/ sequence simply \isi{degeminate}s (\sectref{sec:2.5.8}).} Surprisingly (given the \isi{phonotactics} of the language), the verb \textit{kamb-} ‘shun’\footnote{Also its \is{derivation} derivatives \textit{ala-kamb-} ‘dislike’ and \textit{na-kamb-} ‘suffice’.} seems to have an underlying stem-final \isi{voiced} \isi{stop} (\sectref{sec:2.1.2}). In addition to the verbs in \REF{ex:verbs:1e}, the  verb \textit{ina-} ‘get’ seems to have an alternate \isi{consonant}-final stem /in-/, which is used in the \isi{irrealis} form [inda] (< /in-nda/). The \isi{locative verb} \textit{p-} ‘be, be at’, whose stem also ends in a \isi{consonant} (indeed it consists entirely of a single \isi{consonant}), is discussed at the end of \sectref{sec:4.3}.

  Ulwa may be said to contain a relatively small number of verb roots. Although it is impossible to provide an exact count, the total number of verb roots in the language is certainly less than 100. What may be expressed by a single verb in \ili{English} is often expressed in Ulwa by means of \isi{light verb} constructions (\sectref{sec:9.2.1}--\sectref{sec:9.2.3}).

\is{stem|)}
\is{verb stem|)}
\is{verb|)}

\section{Basic verbal morphology}\label{sec:4.2}

\is{verbal morphology|(}
\is{verbal inflection|(}
\is{inflection|(}
\is{verb|(}
\is{morphology|(}

There is a basic three-way distinction in \isi{TAM} in Ulwa, reflected in the choice of verbal \isi{suffix}. The three forms are \isi{imperfective} (\sectref{sec:4.4}), \isi{perfective} (\sectref{sec:4.5}), and \isi{irrealis} (\sectref{sec:4.6}) \REF{ex:verbs:1f}.

\ea%1f
    \label{ex:verbs:1f}
            Basic \isi{TAM} \isi{suffix}es\\
\begin{tabbing}
{(\textit{-na})} \= {(\isi{imperfective} (‘\textsc{ipfv}’))}\kill
{-\textit{e}} \> {\isi{imperfective} (‘\textsc{ipfv}’)}\\
{\textit{-p}} \> {\isi{perfective} (‘\textsc{pfv}’)}\\
{\textit{-na}} \> {\isi{irrealis} (‘\textsc{irr}’)}
\end{tabbing}
 \z

Briefly, the \isi{imperfective} \isi{aspect} encodes events and states that are viewed as incomplete or ongoing; the \isi{perfective} \isi{aspect} encodes events and states that have reached their end (i.e., they are over and done with); and the \isi{irrealis} \isi{mood} encodes events or states not known to the speaker to have happened (i.e., unreal or \isi{hypothetical} events and states), whether \isi{imperfective} or \isi{perfective} in \isi{aspect}. This section provides an overview of the \isi{morphology} of these three basic forms, as they appear in regular verbs. \isi{Irregular verb}s (that is, verbs whose \isi{morphology} does not in some way conform to the generalizations in this section) are discussed in \sectref{sec:4.3}. 

The regular \isi{imperfective} \isi{suffix} \textit{{}-e} ‘\textsc{ipfv}’ is \isi{homophonous} with the \isi{dependent marker} \textit{{}-e} ‘\textsc{dep}’ (\sectref{sec:12.2.1}). It is sometimes ambiguous which of these \isi{suffix}es is marking a given verb or, indeed, whether both are underlyingly present on the same verb, since the double \isi{vowel} sequence /ee/ would reduce to [e] in any case. However, at least with verbs with an irregular \isi{imperfective} \isi{suffix} (\sectref{sec:4.3}), it is clear that the two morphemes can co-occur \REF{ex:verbs:1g}.

\ea%1g
    \label{ex:verbs:1g}
    \isi{Dependent marker} and irregular \isi{imperfective} \isi{suffix} co-occurring \\
\begin{tabbing}
{(a.)} \= {(go-\textsc{ipfv}-\textsc{dep})} \= {(b.)} \=({say-\textsc{ipfv}-\textsc{dep})}\kill
{a.} \> {\textit{mane}} \> {b.} \> {\textit{tane}}\\
{ } \> {\textit{ma-\textbf{n-e}}} \> { } \> {\textit{ta-\textbf{n-e}}}\\
{ } \> {go-\textsc{ipfv}-\textsc{dep}} \> { } \> {say-\textsc{ipfv}-\textsc{dep}}\\
{ } \> {‘going'} \> { } \> {‘saying’}
      \end{tabbing}
\z

Most \isi{verb stem}s end in a \isi{vowel}, which \isi{deletes} when immediately followed by the \isi{imperfective} \isi{suffix} \textit{{}-e} ‘\textsc{ipfv}’ (\sectref{sec:2.5.4}).

  The regular \isi{perfective} \isi{suffix} is \textit{{}-p} ‘\textsc{pfv}’. Aside from in a few \isi{irregular verb}s (\sectref{sec:4.3}), this \isi{suffix} always appears, quite transparently, suffixed to the \isi{verb stem}. Although \isi{stem}-final \isi{vowel}s are never lost before /p/, there is one notable \isi{phonological} change that occurs in certain \isi{verb stem}s before the \isi{perfective} \isi{suffix} \textit{{}-p} ‘\textsc{pfv}’, a change that has a rather specific conditioning environment. Namely, the \is{high vowel} high \isi{central vowel} /ï/ is lowered to [a] when immediately following the \isi{voiceless} \isi{velar} \isi{stop} and immediately preceding the \isi{voiceless} \isi{bilabial} \isi{stop} \REF{ex:verbs:2}.

\ea%2
    \label{ex:verbs:2}

            Lowering of /ï/ to [a] in \isi{perfective} verbs whose stems end in /kï/

    /ï/ → [a] / k \_ p
\z

This is known to affect only two verbs: \textit{kï-} ‘say’ and \textit{nïkï-} ‘dig, cut’. The \isi{low vowel} [a] may be seen in the \isi{perfective} forms in the paradigms of these two verbs (\tabref{tab:4.1}).

\begin{table}
\caption{Lowering of /ï/ to [a] in perfective verbs whose stems end in /kï/}
\is{verb stem}
\is{imperfective}
\is{perfective}
\is{irrealis}
\label{tab:4.1}
\begin{tabularx}{\textwidth}{QQQQQ}
\lsptoprule
gloss & verb stem & imperfective & perfective & irrealis\\
\midrule
‘say’ & {\itshape kï-} & {\itshape ke} & {\itshape k\textbf{a}p} & {\itshape kïna}\\
‘dig, cut’ & {\itshape nïkï-} & {\itshape nïke} & {\itshape nïk\textbf{a}p} & {\itshape nïkïna}\\
\lspbottomrule
\end{tabularx}
\end{table}
As there are no known word forms in Ulwa that contain the sequence /kï/ immediately followed by a \isi{labial}, it could be that \textsuperscript{†}[kïp] {\textasciitilde} \textsuperscript{†}[kïmb] is prohibited in the \isi{phonotactics} of the language.\footnote{While this a rather specific sequence to be prohibited in a language, there is diachronic support for it being disfavored in Ulwa. Forms that began with *kɨp- and *kɨmb- in \ili{Proto-West Keram} are reflected as /nïp-/ and /nïmb-/, respectively, in Ulwa (e.g., *kɨpa > \textit{nïpa} ‘breadfruit’ > and *kɨmbɨ[n] > \textit{nïmban} ‘fish species’).}

\is{morphology|)}
\is{verb|)}
\is{inflection|)}
\is{verbal inflection|)}
\is{verbal morphology|)}

\is{verbal morphology|(}
\is{verbal inflection|(}
\is{inflection|(}
\is{verb|(}
\is{morphology|(}

  Another interesting characteristic of the verb \textit{nïkï-} ‘dig, cut’ is its propensity for eliding its initial \is{high vowel} high \isi{central vowel} /ï/ entirely. In fact, it seems to be \isi{deleted} in all instances where it is \isi{phonotactic}ally permissible to do so. The examples in \REF{ex:verbs:2a} illustrate how the addition of \isi{object marker}s can license \isi{elide}d forms, which would be unpronounceable without these immediately preceding elements.

\ea%2a
    \label{ex:verbs:2a}
            Forms of the verb \textit{nïkï-} ‘dig’ with and without an \isi{elide}d \textit{ï}\\
\begin{tabbing}
{(\textit{ndïnkïna})} \= {(‘dig them [\textsc{irr}]’)}\kill
\textit{nïkap} \> {‘dig [\textsc{pfv}]’}\\
\textit{nïkïna} \> {‘dig [\textsc{irr}]’}\\
\textit{mankap} \> {‘dig it [\textsc{pfv}]’}\\
\textit{ndïnkïna}\> {‘dig them [\textsc{irr}]’}
\end{tabbing}
 \z

  The last basic \isi{TAM} morpheme to be considered is the \isi{irrealis} \isi{suffix} \textit{{}-na} ‘\textsc{irr}’. It has the \isi{phonological}ly conditioned \isi{allomorph} \textit{{}-nda} ‘\textsc{irr}’, which appears only when the preceding \isi{consonant} is a \isi{sonorant} \REF{ex:verbs:3}.

\ea%3
    \label{ex:verbs:3}

            Conditioned \isi{allomorphy} of the \isi{irrealis} \isi{suffix}\\
    /n/ → [nd] / C [+\textsc{son}] V\textsubscript{0} \_ (in the \isi{irrealis} verb form)
\z

The exact \isi{phonetic} underpinnings of this change are unclear. Perhaps this strengthening of /n/ to [nd] is a means of \is{dissimilation} dissimilating a sequence of \isi{sonorant}s, a sequence which would, presumably, cause perceptual or articulatory challenges for listeners or speakers. Whatever the \isi{phonetic} motivation, however, this alternation is quite regular, as illustrated by \isi{irrealis} verb forms that have [nd] after a preceding \isi{sonorant} \isi{consonant}s /l, n, m, w, y/ \REF{ex:verbs:4}.

\ea%4
    \label{ex:verbs:4}
            Irrealis verb forms with the \isi{allomorph} \textit{-nda} ‘\textsc{irr}’\\
\begin{tabbing}
{(\textit{lo\textbf{w}o\textbf{nd}a})} \= {(‘break, split [\textsc{irr}]’)}\kill
\textit{\textbf{l}a\textbf{nd}a} \> {‘eat [\textsc{irr}]’}\\
\textit{ko\textbf{lnd}a} \> {‘break, split [\textsc{irr}]’}\\
\textit{a\textbf{n}a\textbf{nd}a} \> {‘scrub [\textsc{irr}]’}\\
\textit{ku\textbf{nd}a} \> {‘break (off) [\textsc{irr}]}’\\
\textit{\textbf{m}e\textbf{nd}a} \> {‘sew [\textsc{irr}]’}\\
\textit{lo\textbf{w}o\textbf{nd}a} \> {‘sleep [\textsc{irr}]’}\\
\textit{li\textbf{y}u\textbf{nd}a} \> {‘fall [\textsc{irr}]’}
\end{tabbing}
\z

Note that \textit{kunda} ‘break (off) [\textsc{irr}]’ is underlyingly /kun-nda/, having undergone \isi{quasi-degemination} of the sequence /nnd/ (\sectref{sec:2.1.4}).

These forms may be compared with the \isi{irrealis} forms given in \REF{ex:verbs:5}, which have [n] when the preceding \isi{consonant} is an obstruent /p, t, k, nd, ng, s/ (there are no known forms with preceding /mb/ or /nd/) or when there is no preceding \isi{consonant} at all, as in the final example, \textit{ina} ‘come [\textsc{irr}]’.

\ea%5
    \label{ex:verbs:5}
            Irrealis verb forms with the \isi{allomorph} \textit{-na} ‘\textsc{irr}’\\
\begin{tabbing}
{(\textit{tïna\textbf{ng}a\textbf{n}a})} \= {(‘take (one-by-one) [\textsc{irr}]’)}\kill
\textit{lo\textbf{p}o\textbf{n}a} \> {‘rain [\textsc{irr}]’}\\
\textit{i\textbf{t}a\textbf{n}a} \> {‘build [\textsc{irr}]’}\\
\textit{\textbf{t}ï\textbf{n}a} \> {‘take [\textsc{irr}]’}\\
\textit{mo\textbf{k}o\textbf{n}a} \> {‘take (one-by-one) [\textsc{irr}]’}\\
\textit{i\textbf{nd}a\textbf{n}a} \> {‘walk [\textsc{irr}]’}\\
\textit{tïna\textbf{ng}a\textbf{n}a} \> {‘arise [\textsc{irr}]’}\\
\textit{\textbf{s}i\textbf{n}a} \> {‘push [\textsc{irr}]’}\\
\textit{i\textbf{n}a} \> {‘come [\textsc{irr}]’}
    \end{tabbing}
\z

The few exceptions to this pattern are treated with the discussion of \isi{irregular verb}s (\sectref{sec:4.3}). Like the \isi{perfective} \isi{suffix}, the \isi{irrealis} \isi{suffix} does not condition the loss of a \isi{stem}-final \isi{vowel}. There is, however, one very specific environment in which this \isi{vowel} may change. Namely, between /l/ and /nd/, underlying \isi{mid vowel}s /e, o/ are realized as \isi{high vowel}s [i, u] \REF{ex:verbs:6}.

\ea%6
    \label{ex:verbs:6}
            Raising of /e, o/ to [i, u] in \isi{irrealis} verbs whose \isi{stem}s end in /le/ or /lo/\\
    /e, o/ → [i, u] / l \_ nd
\z

The only two verbs that this process is known to affect are \textit{ale-} ‘scrape’ and \textit{lo-} ‘cut, go’, whose basic paradigms are given in \tabref{tab:4.2}. The \isi{phonetic} motivation for this process is unclear. It could result from a form of \isi{hypercorrection} or it could be related to a \isi{phonotactic} constraint.\footnote{The \isi{hypercorrection} hypothesis would run as follows: speakers are raising \isi{vowel}s they assume to have been lowered \isi{phonetic}ally due to secondary nasalization from the following \isi{nasal} gesture giving the \isi{vowel} a perceived lower quality. The alternative hypothesis would be that there is a \isi{phonotactic} constraint at play here, since there are no known forms \textsuperscript{†}[-lend-] or \textsuperscript{†}[-lond-] in Ulwa. Actually, since there are also no known instances of \textsuperscript{†}[-len-] or \textsuperscript{†}[-lon-], it is equally possible that this raising of /e/ and /o/ to [i] and [u], respectively, occurs before the \is{morphophonemic process} morphophonemic change of /n/ to [nd]. This second hypothesis may be supported by the fact that the \isi{imperative} forms of these verbs are \textit{alin} ‘scrape [\textsc{irr]}’ and \textit{lun} ‘cut, go [\textsc{irr]}’ (\sectref{sec:4.7}). It may also be the case, however, that the \isi{imperative} forms are built by \isi{analogy} to the \isi{irrealis} forms; this is almost certainly the case with the \isi{conditional} forms (\sectref{sec:4.12}), which are \textit{alita} ‘scrape [\textsc{cond]}’ and \textit{luta} ‘cut, go [\textsc{cond]}’, respectively.}

\begin{table}
\caption{Raising of /e, o/ to [i, u] in irrealis verbs whose stems end in /le/ or /lo/}
\is{verb stem}
\is{imperfective}
\is{perfective}
\is{irrealis}
\label{tab:4.2}
\begin{tabularx}{\textwidth}{QQQQQ}
\lsptoprule
gloss & verb stem & imperfective & perfective & irrealis\\
\midrule
‘scrape’ & {\itshape ale-}  & {\itshape ale} & {\itshape alep} & {\itshape al\textbf{i}nda}\\
‘cut, go’ & {\itshape lo-} & {\itshape le} & {\itshape lop} & {\itshape l\textbf{u}nda}\\
\lspbottomrule
\end{tabularx}
\end{table}

\is{vowel assimilation}
\is{local assimilation}
\is{local vowel assimilation}

  It is worth noting that the raising of /o/ to [u] in \textit{lunda} ‘go [\textsc{irr}]’ must occur after the local \isi{vowel} \isi{assimilation} of /a/ to /o/ that occurs in the 3\textsc{sg} \isi{object marker} when this marker is present (\sectref{sec:2.5.7}). In examples \REF{ex:verbs:7}, \REF{ex:verbs:8}, and \REF{ex:verbs:9}, the 3\textsc{sg.obj} form [mo=] (as opposed to [ma=]) is seen accompanying each of the three basic \isi{TAM}-marked verbs. The aforementioned rule ordering accounts for the surface form [molunda] in \REF{ex:verbs:9}.

\ea%7
    \label{ex:verbs:7}
            \textit{Yawat mï awal num} \textbf{\textit{molop}}.\\
\gll Yawat  mï      awal    num  \textbf{ma}=lo-p\\
    [name]  3\textsc{sg.subj}  yesterday  canoe  3\textsc{sg.obj}=cut-\textsc{pfv}\\
\glt `Yawat made the canoe yesterday.’ [elicited]
\z

\ea%8
    \label{ex:verbs:8}
            \textit{Yawat mï amun num} \textbf{\textit{mole}}.\\
\gll Yawat  mï      amun    num  \textbf{ma}=lo-e\\
    [name]  3\textsc{sg.subj}  now    canoe  \textsc{3sg.obj}=cut-\textsc{ipfv}\\
\glt `Yawat is making the canoe now.’ [elicited]
\z

\ea%9
    \label{ex:verbs:9}
            \textit{Yawat mï umbe num} \textbf{\textit{molunda}}.\\
\gll Yawat  mï      umbe    num  \textbf{ma}=lo{}-nda\\
    [name]  3\textsc{sg.subj}  tomorrow  canoe  \textsc{3sg.obj}=cut-\textsc{irr}\\
\glt `Yawat will make the boat tomorrow.’ [elicited]
\z

This set of examples also lends further support to the analysis that there are underlying \isi{stem}-final \isi{vowel}s that are lost in \isi{imperfective} forms, since the /o/ of the \isi{stem} /lo-/ must have been present underlyingly in the \isi{imperfective} verb form in order to have conditioned the change of /ma=/ to [mo=] in \REF{ex:verbs:2}.\footnote{  It may be noted that the common \isi{imperative} (or \isi{jussive}) \textit{nol} ‘(let’s) go!’ is likely derived from the verb \textit{lo-} ‘cut, go’. In this analysis, the form derives from \textit{na-lo-n} ‘\textsc{detr}{}-go-\textsc{imp’}, the /o/ first raising to [u] \REF{ex:verbs:6}, then the /l/ and /u/ \is{metathesis} metathesizing (\sectref{sec:2.6}), then the sequence /au/ becoming the \is{monophthongization} \isi{monophthong} [o] (\sectref{sec:2.5.2}), and then the final /n/ of this high-frequency word being \isi{elide}d (i.e., \is{apocope} apocopated).}

  Finally, it should be mentioned that I consider the basic set of three \isi{TAM} \isi{suffix}es to be obligatory \isi{aspect}ual and \isi{modal} \isi{morphology}. However, there are several verbs, all of them high-frequency \isi{lexical} items with rather generic meanings, that seem capable of occurring without any \isi{TAM} \isi{suffix}, even when one is available to them. Usually, when they occur in this bare form, they also \isi{elide} their \isi{stem}-final \isi{vowel}. The most common examples are given in \REF{ex:verbs:9a}.

\ea%9a
    \label{ex:verbs:9a}
            Abbreviated verb forms\\
\begin{tabbing}
{(as)} \= {(‘push’)} \= {(< \textit{asa-})}\kill
\textit{as} \> {‘hit’} \> {< \textit{asa-}}\\
\textit{l} \> {‘put’} \> {< \textit{lï-}}\\
\textit{s} \> {‘push’} \> {< \textit{si-}}\\
\textit{t} \> {‘say’} \> {< \textit{ta-}}\\
\textit{t} \> {‘take’} \> {< \textit{tï-}}
\end{tabbing}
 \z

It may be the case that such verbs are undergoing a process of \isi{grammaticalization}.

\is{morphology|)}
\is{verb|)}
\is{inflection|)}
\is{verbal inflection|)}
\is{verbal morphology|)}

\section{Irregular verbs and suppletion}\label{sec:4.3}

\is{irregular verb|(}
\is{verb|(}

This section is devoted to describing the \isi{morphology} of verbs that in one way or another do not conform to the patterns described in \sectref{sec:4.2} -- that is, verbs that exhibit unexpected \isi{TAM} \isi{suffix}es, verbs that are \isi{defective} in that they lack certain forms or may be un\isi{inflect}ed for certain \isi{TAM} designations, or verbs that have \isi{suppletive} forms that come from unrelated \isi{stem}s.

  First, there is a set of verbs that have \textit{-n} ‘\textsc{pfv}’ as their \isi{perfective} \isi{suffix}, instead of the expected regular form \textit{-p} ‘\textsc{pfv}’, a shown in \tabref{tab:4.3}.

\begin{table}
 \caption{\label{tab:4.3}Verbs with the irregular perfective suffix \textit{-n} ‘\textsc{pfv}’}
\is{verb stem}
\is{imperfective}
\is{perfective}
\is{irrealis}
\begin{tabularx}{\textwidth}{QQQQQ}
\lsptoprule
gloss & verb stem & imperfective & perfective & irrealis\\
\midrule
‘take’ & {\itshape tï-} & {\itshape te {\textasciitilde} tï {\textasciitilde} t}  & {\itshape tï\textbf{n} {\textasciitilde} tï {\textasciitilde} t} & {\itshape tïna}\\
‘give’ & {\itshape na-} & -- & {\itshape na\textbf{n} {\textasciitilde} na\textbf{na}} & {\itshape nanda}\\
‘come’ & {\itshape i-} & {\itshape [man]} & {\itshape i\textbf{n}} & {\itshape ina}\\
\lspbottomrule
\end{tabularx}
\end{table}
The first verb listed in \tabref{tab:4.3}, \textit{tï-} ‘take’, is often \isi{defective}, especially when used in multi-verb constructions with \textit{na-} ‘give’ to express ‘giving’ events (\sectref{sec:11.3}). The final \isi{central vowel} /ï/ of the \isi{stem} is often lost in such \isi{defective verb} forms, but may, alternatively, be present. Its presence in such instances is taken to be \isi{phonetic}ally motivated -- that is, it is assumed that, in the \isi{imperfective} and \isi{perfective} forms, the final \isi{vowel} /ï/ of this verb is always lost, but, when the resulting form [t] is followed by a \isi{consonant}, an \isi{epenthetic} [ï] emerges to break up the forbidden \isi{consonant cluster}. Otherwise, when present, the \isi{perfective} \isi{suffix} is \textit{-n} ‘\textsc{pfv}’. The \isi{irrealis} \isi{suffix} is the expected form \textit{{}-na} ‘\textsc{irr}’. When this verb occurs in its reduced form [t], it often appears to \isi{clitic}ize to a following \isi{vowel}-initial word.

  The second verb, \textit{na-} ‘give’, also has \textit{-n} ‘\textsc{pfv}’ as its \isi{perfective} \isi{suffix}, although this may optionally be realized (perhaps for added emphasis) as \textit{-na} ‘\textsc{pfv}’. The \isi{irrealis} \isi{suffix} \textit{-nda} ‘\textsc{irr}’ is regular -- that is, it is the expected \isi{allomorph} of \textit{-na} ‘\textsc{irr}’, given that the preceding \isi{consonant} is a \isi{sonorant} (\sectref{sec:4.2}). The verb \textit{na-} ‘give’ is \isi{defective} in another sense: there is no distinct \isi{imperfective} form (i.e., there is no form \textsuperscript{†}/ne/); the \isi{perfective} form, however, may be used to convey \isi{imperfective} \isi{aspect}, if needed.

  The third verb, \textit{i-} ‘come’, relies on a \isi{suppletive} form based on the \isi{stem} \textit{ma-} ‘go’ for its \isi{imperfective} form.

  Another verb, \textit{si-} ‘push’, has a unique \isi{perfective} \isi{suffix}, \textit{{}-al} ‘\textsc{pfv}’. It is also common for this verb to use the bare \isi{stem} [si] as the \isi{perfective} form -- that is, the final \isi{vowel} may be retained.\footnote{An alternative analysis would be that this verb results from a historical \is{periphrasis} periphrastic verbal construction consisting of \textit{tï-} ‘take’ and \textit{i} ‘go.\textsc{pfv}’ (\sectref{sec:10.3}), with initial *t having spirantized to [s] immediately preceding the \is{high vowel} high \isi{front vowel} *i. In this analysis, the \isi{irrealis} form [sina] would derive from \textit{tï-} ‘take’ plus \textit{i-na} ‘come-\textsc{irr}’.} The paradigm for \textit{si-} ‘push’ is given in \tabref{tab:4.4}.


\begin{table}
\caption{The irregular verb \textit{si-} ‘push’, with perfective suffix \textit{-al} ‘\textsc{pfv}’}
\is{verb stem}
\is{imperfective}
\is{perfective}
\is{irrealis}
\label{tab:4.4}
\begin{tabularx}{\textwidth}{QQQQQ}
\lsptoprule
gloss & verb stem & imperfective & perfective & irrealis\\
\midrule
‘push’ & {\itshape si-} & {\itshape se} & {\itshape s\textbf{al} \textup{{\textasciitilde}} si} & {\itshape sina}\\
\lspbottomrule
\end{tabularx}
\end{table}
  In yet another set of verbs, there is an unusual \isi{prefix}-like form that occurs in the \isi{irrealis} \isi{mood}, in addition to the expected \isi{irrealis} \isi{suffix}. Thus, these forms appear to have something like \isi{circumfix}es encoding \isi{irrealis} \isi{mood}. The verbs in question are given in \tabref{tab:4.5}.


\begin{table}
\caption{Verbs with the apparent irrealis circumfix \textit{la- … -n(d)a}}
\is{verb stem}
\is{imperfective}
\is{perfective}
\is{irrealis}
\is{circumfix} 
\label{tab:4.5}
\begin{tabularx}{\textwidth}{QQQQQ}
\lsptoprule
gloss & verb stem & imperfective & perfective & irrealis\\
\midrule
‘let’ & {\itshape ka-} & -- & -- & {\itshape \textbf{la}ka\textbf{na}}\\
‘sleep’ & {\itshape wo-} & {\itshape wowe} & {\itshape wop} & {\itshape \textbf{lo}wo\textbf{nda}}\\
‘eat’ & {\itshape ama-} & {\itshape ame} & {\itshape amap} & {\itshape \textbf{landa}}\\
\lspbottomrule
\end{tabularx}
\end{table}
All the \isi{irrealis} forms given in \tabref{tab:4.5} appear, at least historically, to have the same \isi{prefix}-like element, /la-/, which is mostly clearly seen in \textit{lakana} ‘let, leave, allow [\textsc{irr}]’. The form \textit{lowonda} ‘sleep [\textsc{irr}]’ likely derives from *la-wo-nda, the *a having been \is{rounding} rounded and raised by the following \isi{labial-velar} /w/. The \isi{suffix} \textit{-nda} ‘\textsc{irr}’ is the expected \isi{allomorph} following the \isi{sonorant} /w/. The form \textit{landa} ‘eat [\textsc{irr}]’ perhaps derives from *la-am-nda, the initial *a having \isi{deleted} before the following \isi{vowel} (\sectref{sec:2.5.5}) and the *m having been lost after \isi{assimilating} in place to the following \isi{nasal} gesture. Here, too, the \isi{suffix} \textit{-nda} ‘\textsc{irr}’ is expected, following the \isi{sonorant} /l/. This initial [la] may be a \isi{fossilized} form of the \isi{detransitivizing} \isi{prefix} *la- (\sectref{sec:13.8.1}, \sectref{sec:18.4}).

  The verb \textit{ka-} ‘let, leave, allow’ is highly \isi{defective}, lacking \isi{TAM} \isi{morphology} for both \isi{imperfective} and \isi{perfective} forms; for these \isi{TAM} distinctions, the bare \isi{stem} [ka] is used, instead of the predicted forms \textsuperscript{†}[ke] for ‘let [\textsc{ipfv]’} and \textsuperscript{†}[kap] for ‘let [\textsc{pfv]’}. This verb is used in \isi{separable verb} constructions (\sectref{sec:9.2.3}).

  The verb \textit{wo-} ‘sleep’ does not lose its final \isi{vowel} before the \isi{imperfective} \isi{suffix} \textit{{}-e} ‘\textsc{ipfv}’. Rather, the /-o/ of the \isi{stem} remains. This verb has an alternate \isi{stem} /wow-/ used for \isi{imperfective} \isi{aspect}.

  Three verbs have \isi{stem}s consisting of just a \isi{vowel}, as shown in \tabref{tab:4.6}. (The verb \textit{i} ‘come’ is also included in \tabref{tab:4.3}.)

\begin{table}
\caption{Verb stems consisting of just a vowel}
\is{verb stem}
\is{imperfective}
\is{perfective}
\is{irrealis}
\label{tab:4.6}
\begin{tabularx}{\textwidth}{QQQQQ}
\lsptoprule
gloss & verb stem & imperfective & perfective & irrealis\\
\midrule
‘break’ & {\itshape \textbf{a}{}-} & {\itshape aye} & {\itshape ap} & {\itshape anda}\\
‘come’ & {\itshape \textbf{i}{}-} & {\itshape [man]} & {\itshape in} & {\itshape ina}\\
‘put’ & {\itshape \textbf{u}{}-} & {\itshape awe} & {\itshape up} & {\itshape unda}\\
\lspbottomrule
\end{tabularx}
\end{table}

\is{verb stem}
\is{vowel}

 Both \textit{a-} ‘break’ and \textit{u-} ‘put’ exhibit the unexpected \isi{allomorph} \textit{-nda} \textsc{‘irr’} -- unexpected because there is no preceding \isi{sonorant} \isi{consonant}, unless the \isi{glide}s [y] and [w] of the respective \isi{imperfective} forms are somehow in the underlying form of the \isi{irrealis} forms or otherwise influencing a \isi{fortition} of /-na/ to [-nda]. In the \isi{imperfective} forms of these verbs, the \isi{stem}-final \isi{vowel} is not lost. Indeed, that would mean the loss of the entire \isi{phonological} content of the \isi{verb stem}.\footnote{Instead, the \isi{imperfective} forms appear to be derived from the \isi{stem}-plus-\isi{dependent marker}. For the verb \textit{a-} ‘break’ this entails the derivation *a-e > [aye], with an \isi{epenthetic} \isi{glide} /y/ breaking up the VV sequence. For the verb \textit{u-} ‘put’, the derivation of the \isi{imperfective} form is probably as follows: *u-e > *uwe > [awe], the initial *u having been lowered as a means of \is{dissimilation} dissimilating it from the following high back \isi{glide} /w/.} The verb \textit{u-} ‘put’ is used in a number of \isi{separable verb} constructions (\sectref{sec:9.2.2}). The \isi{imperfective} form of this verb is sometimes reduced to [aw].

  Three high-frequency verbs that are also highly irregular are \textit{ma-} ‘go’, \textit{andï-} ‘see’, and \textit{asa-} ‘hit, kill’, the last of which is in a \isi{suppletive} relationship with \textit{wali-} ‘hit, kill’. These verbs are shown in \tabref{tab:4.7}.

  \begin{table}
 \caption{\label{tab:4.7} The irregular verbs ‘go’, ‘see’, and ‘hit, kill’}
\is{verb stem}
\is{imperfective}
\is{perfective}
\is{irrealis}
\begin{tabularx}{\textwidth}{lllll}
\lsptoprule
gloss & verb stem & imperfective & perfective & irrealis\\
\midrule
‘go’ & {\itshape ma-} & {\itshape man} & {\itshape [i]} & {\itshape mana}\\
‘see’ & {\itshape andï-} & {\itshape [ala]} & {\itshape andïm \textup{{\textasciitilde}} [ala]} & {\itshape andïna}\\
‘hit, kill’ & {\itshape asa-} & {\itshape ase} & {\itshape asap {\textasciitilde} as} & {\itshape atïna \textup{{\textasciitilde}} atïm} \\ & & & &{\textup{{\textasciitilde}} \textit{[walinda]}}\\
‘hit, kill’ & {\itshape wali-} & {\itshape wale} & {\itshape [asap]} & {\itshape walinda}\\
\lspbottomrule
\end{tabularx}
\end{table}

\is{verb|)}
\is{irregular verb|)}

\is{irregular verb|(}
\is{verb|(}

First, the verb \textit{ma-} ‘go’ does not have a \isi{perfective} form derived from this \isi{stem}. Rather, the bare \isi{stem} of the verb \textit{i-} ‘come’ is used to encode \isi{perfective} \isi{aspect} for this verb. The \isi{imperfective} form [man] is very strange in that it employs the \isi{suffix} \textit{-n} \textsc{‘ipfv’}, which is otherwise found as an irregular \isi{perfective} marker, as for the verbs \textit{tï-} ‘take’ and \textit{na-} ‘give’ (\tabref{tab:4.3}). The \isi{irrealis} form [mana] is also irregular, in that it exhibits the \isi{suffix} \textit{-na} \textsc{‘irr’} (not \textsuperscript{†}[-nda]), despite the presence of the preceding \isi{sonorant} \isi{consonant} /m/.

  Second, the verb \textit{andï-} ‘see’ has a \isi{suppletive} \isi{imperfective} form [ala], built from a different \isi{stem}. This form is also commonly used with \isi{perfective} meaning. The \isi{perfective} form built on the \isi{stem} [andï-] has the unusual \isi{suffix} \textit{-m} \textsc{‘pfv’}. This form is much less commonly used than [ala], which is used for \isi{perfective} as well as \isi{imperfective} meaning. The \isi{irrealis} form is the regular [andïna], which may be shortened to [andïn].

  Third, the verb \textit{asa-} ‘hit, kill’ does not have the predicted \isi{irrealis} form \textsuperscript{†}[asana]. Instead, one of two irregular forms is used, [atïna] or [atïm], the first of which exhibits the regular \isi{irrealis} \isi{suffix} \textit{{}-na} ‘\textsc{irr}’.\footnote{The nature of the apparent \isi{stem} change (i.e., [atï-] instead of \textsuperscript{†}[asa-]) is not clear, but there could be alternate forms of this root, at least historically, as suggested by the noun \textit{at} ‘fight, battle’.} The final [-m] of the alternate \isi{irrealis} form [atïm] is even harder to account for, especially since this /-m/ ending is used with \isi{perfective} meaning in the verb \textit{andï-} ‘see’.\footnote{This form (with both its \isi{perfective} meaning and its \isi{irrealis} meaning) may derive from a nonfinite verbal \isi{suffix} *-m. Compare a similar form in \ili{Pondi} (\citealt[61--62]{Barlow2020b}).} Most commonly, however, instead of [atïna] or [atïm], the \isi{suppletive} form [walinda] is used for the \isi{irrealis} \isi{mood}. This form comes from the \isi{verb stem} \textit{wali-} ‘hit kill’, which itself relies on \isi{suppletion} for its otherwise lacking \isi{perfective} form. The verb \textit{asa-} ‘hit, kill’ often appears without \isi{TAM} marking, perhaps especially when it has \isi{perfective} meaning.

\is{verb|)}
\is{irregular verb|)}

\is{irregular verb|(}
\is{verb|(}

  \tabref{tab:4.8} includes the paradigms for verbs that are missing basic forms, including the verbs \textit{na-} ‘give’ (cf. \tabref{tab:4.3}) and \textit{ka-} ‘let’ (cf. \tabref{tab:4.5}).

\begin{table}
\caption{Defective verbs}
\is{verb stem}
\is{imperfective}
\is{perfective}
\is{irrealis}
\is{defective verb}
\label{tab:4.8}
\begin{tabularx}{\textwidth}{QQQQQ}
\lsptoprule
gloss & verb stem & imperfective & perfective & irrealis\\
\midrule
‘give’ & {\itshape na-} & -- & {\itshape nan {\textasciitilde} nana} & {\itshape nanda}\\
‘let’ & {\itshape ka-} & -- & -- & {\itshape lakana}\\
‘arise’ & {\itshape tïnanga-} & -- & -- & {\itshape tïnangana}\\
‘feel’ & {\itshape wana-} & -- & -- & {\itshape wananda}\\
‘put’ & {\itshape lumo-} & -- & {\itshape lumop} & --\\
\lspbottomrule
\end{tabularx}
\end{table}
The verb \textit{tïnanga-} ‘arise’ lacks \isi{imperfective} and \isi{perfective} forms; the bare \isi{stem} may be used for these \isi{aspect}s. The \isi{irrealis} form is regular. Similarly, the verb \textit{wana-} ‘feel’ relies on its bare \isi{stem} for the \isi{imperfective} and \isi{perfective} forms. Although the form [wan] sometimes occurs, it seems simply to be a \isi{phonetic}ally reduced form of the \isi{stem}, not a \isi{morphological}ly conditioned \isi{imperfective} form, despite its missing \isi{stem}-final \isi{vowel} -- [wan] is found both with \isi{imperfective} and with \isi{perfective} meaning.

  The verb \textit{lumo-} ‘put’ is only found in the \isi{perfective} form and in \isi{conditional} forms (\sectref{sec:4.12}). This verb is also used in \isi{separable verb} constructions (\sectref{sec:9.2.2}).

  Two verbs remain to be discussed in this section: \textit{ta-} ‘say’ and \textit{li-} ‘fall’. The basic paradigm for \textit{ta-} ‘say’ is given in \tabref{tab:4.9}.


\begin{table}
\caption{The irregular verb \textit{ta-} ‘say’}
\is{verb stem}
\is{imperfective}
\is{perfective}
\is{irrealis}
\label{tab:4.9}
\begin{tabularx}{\textwidth}{QQQQQ}
\lsptoprule
gloss & verb stem & imperfective & perfective & irrealis\\
\midrule
‘say’ & {\itshape ta-} & {\itshape tan {\textasciitilde} t} & {\itshape tap {\textasciitilde} t} & {\itshape tana}\\
\lspbottomrule
\end{tabularx}
\end{table}

Like \textit{ma-} ‘go’, the verb \textit{ta-} ‘say’ has the irregular \isi{imperfective} \isi{suffix} \textit{-n} ‘\textsc{ipfv}’. Furthermore, this high-frequency verb is commonly used (especially in \isi{perfective} \isi{aspect}) without any overt \isi{TAM} marking. Like \textit{tï-} ‘take’, its \isi{stem}-final \isi{central vowel} is lost in such instances. The loss of /a/ is not optional when the verb is used without any \isi{TAM} \isi{suffix}ation (i.e., the form \textsuperscript{†}[ta] ‘say’ is completely unattested).

The verb \textit{li-} ‘fall’ is transparently derived from the \isi{adverb} \textit{li} ‘down’ and the verb \textit{u-} ‘put’ (see \sectref{sec:9.2.2} for similar constructions). A paradigm for \textit{li-} ‘fall’ is given in \tabref{tab:4.10}. It has, however, undergone \isi{phonological change}s in its various forms. The \isi{imperfective} form [liwe] derives from *li-awe, the /a/ having been \isi{elide}d. The \isi{perfective} form [liyu] derives from *li-up: while the \isi{epenthetic} [y] is indeed expected (\sectref{sec:2.5.1}), the loss of final /-p/ is difficult to explain. The \isi{irrealis} form [liyunda], which derives from *li-unda, also employs an \isi{epenthetic} [y] to break up two consecutive \isi{vowel}s.

\begin{table}
\caption{The irregular verb \textit{li-} ‘fall’}
\is{verb stem}
\is{imperfective}
\is{perfective}
\is{irrealis}
\label{tab:4.10}
\begin{tabularx}{\textwidth}{QQQQQ}
\lsptoprule
gloss & verb stem & imperfective & perfective & irrealis\\
\midrule
‘fall’ & {\itshape li-} & {\itshape liwe} & {\itshape liyu} & {\itshape liyunda}\\
\lspbottomrule
\end{tabularx}
\end{table}

  In all such instances of irregularity or \isi{suppletion} in \isi{verb stem}s, the alternations are conditioned by \isi{TAM} distinctions. There are no verbs whose \isi{stem}s alter according to the \isi{person} of an argument, whether subject or object. There may be one verb, however, that exhibits an alternation based on the \isi{number} of the subject argument. The \isi{verb stem} \textit{ni-} ‘die’ appears to have a weakly \isi{suppletive} \isi{stem} \textit{nipinpu-} ‘die.\textsc{pl}’ that is used when the subject is \isi{plural} (or the event is \isi{iterative}), thus representing perhaps a solitary example of \isi{pluractionality}. This \isi{pluractional} verb form appears to be a \isi{compound} containing \textit{u-} ‘put’ (see \sectref{sec:9.2.2}). Examples are given in \REF{ex:verbs:10} through \REF{ex:verbs:13}.

\ea%10
    \label{ex:verbs:10}

          \textit{Ipka ndan matmat mbu ulwa pe ndï ankam ndï} \textbf{\textit{nipinpawe}}.\\
          \gll ipka  anda-n      matmat  mbï-u    ulwa    p-e ndï ankam  ndï  \textbf{nipinp}{}-aw-e\\
    before  \textsc{sg.dist}{}-\textsc{obl}  cemetery  here-from  nothing  be\textsc{{}-dep} 3\textsc{pl}  person  3\textsc{pl}  die.\textsc{pl}{}-put.\textsc{ipfv-dep}\\

\glt `In the past, there was no cemetery here, but they, the people, were dying.’ (\textit{matmat} = TP) [ulwa028\_04:26]
\z

\ea%11
    \label{ex:verbs:11}
          \textit{Ndï} \textbf{\textit{nipunpup}}.\\
\gll ndï  \textbf{nipunp}{}-u-p\\
    3\textsc{pl}  die.\textsc{pl}{}-put-\textsc{pfv}\\
\glt `They have died.’ [ulwa037\_27:16]
\z

\ea%12
    \label{ex:verbs:12}
          \textit{Maka lepen ngusuwa ndï wopa \textbf{nipinpup} ulwap}.\\
\gll maka  lo{}-p-en      ngusuwa  ndï  wopa  \textbf{nipinp}{}-u-p ulwa=p\\
    thus  go-\textsc{pfv}{}-\textsc{nmlz}  poor    3\textsc{pl}  all    die.\textsc{pl}{}-put-\textsc{pfv} nothing=\textsc{cop}\\

\glt `But the poor things who used to go around like that have all died completely.’\footnote{The pronunciation [lepen] is unexpected.} [ulwa032\_48:58]
\z

\ea%13
    \label{ex:verbs:13}
          \textit{Ankam wa} \textbf{\textit{nipunpunda}}.\\
\gll ankam  wa  \textbf{nipunp}{}-u-nda\\
    person  just  die.\textsc{pl}{}-put-\textsc{irr}\\
\glt `People will just die.’ [ulwa037\_25:33]
\z

There is some variability in the pronunciation of the second \isi{vowel} of \textit{nipinpu{}-} ‘die.\textsc{pl}’, often being pronounced [u] (i.e., [nipunpu]) perhaps under influence from the following /u/. It may be possible for \textit{ni-} ‘die’ to be used optionally with \isi{plural} subjects or events, but this seems rarely to be the case (the only examples of \isi{plural} \textit{ni-} ‘die’ come from elicitations). On the other hand, the use of \textit{nipinpu} ‘die.\textsc{pl}’ with a \isi{singular} subject will produce a \isi{semantic}ally unlikely interpretation, as illustrated by the elicited examples \REF{ex:verbs:14}, \REF{ex:verbs:15}, and \REF{ex:verbs:16}.

\ea%14
    \label{ex:verbs:14}
          \textit{Ndï} \textbf{\textit{nipunpup}}.\\
\gll ndï  \textbf{nipunp}{}-u-p\\
    3\textsc{pl}  die.\textsc{pl}{}-put-\textsc{pfv}\\
\glt `They died one by one.’ [elicited]
\z

\ea%15
    \label{ex:verbs:15}
          \textit{Mï awal} \textbf{\textit{nip}}.\\
\gll mï      awal    \textbf{ni}{}-p\\
    3\textsc{sg.subj}  yesterday  die-\textsc{pfv}\\
\glt `He died yesterday.’ [elicited]
\z

\ea[?]{%16
    \label{ex:verbs:16}
            \textit{Mï awal} \textbf{\textit{nipunpup}}.\\
\gll  mï      awal    \textbf{nipunp}{}-u-p\\
      3\textsc{psg.subj}  yesterday  die.\textsc{pl}{}-put-\textsc{pfv}\\
\glt    (a) ‘He kept dying yesterday.’ (?)\\
    (b) ‘He died again and again yesterday.’ (?) [elicited]
}
\z

Similar highly restricted examples of \isi{verbal number} are found throughout the \ili{Keram} family, as in \ili{Pondi} \textit{alas-} {\textasciitilde} \textit{alawa-} ‘fly’ \citep[124]{Barlow2020b}, \ili{Mwakai} \textit{wura-} {\textasciitilde} \textit{wurura-} ‘fly’ \citep[78]{Barlow2020a}, and \ili{Ambakich} \textit{krɨ-} {\textasciitilde} \textit{kano} ‘fall’ \citep[70]{Barlow2021}. Some of these forms (including Ulwa \textit{nipinpu-} ‘die.\textsc{pl}’) may derive from \is{iconicity} \isi{iconic reduplication}. \isi{Reduplication}, however, does not ever seem to have been a productive \isi{morphological} process in the family.

  One final \isi{irregular verb} to be considered is the \isi{locative verb} \textit{p-} ‘be, be at (be located at)’. This verb may be considered \isi{defective} in that it does not encode any \isi{aspect}. It can be used to refer either to \isi{past} or to \isi{present} \isi{time}. There is, however, an \isi{irrealis} form [pïna], which has an \isi{epenthetic} [ï] to break up the /pn/ \isi{consonant cluster}. A weakly \isi{suppletive} form \textit{wap} ‘be.\textsc{pst}’, perhaps derived historically from the \isi{modal adverb} \textit{wa} ‘just’ (?) plus \textit{p-} ‘be’, may be used to refer explicitly to \isi{past} \isi{time}. See \sectref{sec:10.1} for the use of the \isi{locative verb} \textit{p-} ‘be’, see \sectref{sec:10.3} for the formally related \isi{copular enclitic} \textit{=p} ‘\textsc{cop}’, see \sectref{sec:10.4} for the form \textit{wap} ‘be.\textsc{pst}’, and see \sectref{sec:15.3} for possibly \isi{contact}-influenced uses of \textit{wap} ‘be.\textsc{pst}’.


\is{verb|)}
\is{irregular verb|)}

\section{{Imperfective} {aspect}}\label{sec:4.4}

\is{aspect|(}
\is{imperfective|(}
\is{verb|(}

The \isi{imperfective} aspect reflects \isi{atelicity}. If an event did not reach or has not reached its end, whether in \isi{past} or \isi{present} \isi{time}, the verb encoding it usually receives \isi{imperfective} marking. In other words, \isi{imperfective} \isi{morphology} signals \isi{continuous}, \isi{habitual}, or \isi{iterative} happenings or states.

  Sentences \REF{ex:verbs:17} through \REF{ex:verbs:23} exemplify the \isi{imperfective} aspect as it applies to different verbs.

\ea%17
    \label{ex:verbs:17}
          \textit{Malman mï amun lamndu} \textbf{\textit{mase}}.\\
\gll Malman  mï    amun  lamndu  ma=asa-\textbf{e}\\
    [name]  3\textsc{sg.subj}  now  pig      3\textsc{sg.obj=}hit-\textsc{ipfv}\\
\glt `Malman is killing the pig now.’ [elicited]
\z

\ea%18
    \label{ex:verbs:18}
          \textit{Anam mï amun apa} \textbf{\textit{mayte}}.\\
\gll Anam  mï      amun  apa  ma=ita-\textbf{e}\\
    [name]  3\textsc{sg.subj}  now  house  3\textsc{sg.obj=}build-\textsc{ipfv}\\
\glt `‎‎Anam is building the house now.’ [elicited]
\z

\ea%19
    \label{ex:verbs:19}
          \textit{Anam mï awal apa} \textbf{\textit{mayte}}.\\
\gll Anam  mï      awal    apa    ma=ita-\textbf{e}\\
    [name]  3\textsc{sg.subj}  yesterday  house  3\textsc{sg.obj=}build-\textsc{ipfv}\\
\glt `Anam was building the house yesterday.’ (Implication: the process was ongoing yesterday or he has not finished.) [elicited]
\z

\ea%20
    \label{ex:verbs:20}
          \textit{Inom mï ya} \textbf{\textit{ute}}.\\
\gll inom    mï    ya     uta-\textbf{e}\\
    mother  3\textsc{sg.subj}  coconut  grind-\textsc{ipfv}\\
\glt `Mother is grinding coconut.’ [elicited]
\z

\ea%21
    \label{ex:verbs:21}
          \textit{Inom mï ipka ya} \textbf{\textit{ute}}.\\
\gll inom  mï      ipka  ya      uta-\textbf{e}\\
    mother  3\textsc{sg.subj}  before  coconut  grind-\textsc{ipfv}\\
\glt `Mother was grinding coconut earlier.’ (Implication: she was continuing to grind it.) [elicited]
\z

\ea%22
    \label{ex:verbs:22}
          \textit{Itom mï amun} \textbf{\textit{inde}}.\\
\gll itom  mï      amun  inda-\textbf{e}\\
    father  3\textsc{sg.subj}  now  walk-\textsc{ipfv}\\
\glt `Father is walking now.’ [elicited]
\z

\ea%23
    \label{ex:verbs:23}
          \textit{Itom mï utam} \textbf{\textit{mame}}.\\
\gll itom  mï      utam  ma=ama-\textbf{e}\\
    father  3\textsc{sg.subj}  yam  3\textsc{sg.obj}=eat-\textsc{ipfv}\\
\glt `Father was eating the yam.’ [elicited]
\z

Sentence \REF{ex:verbs:24} provides an example of the irregular \isi{imperfective} \isi{suffix} \textit{{}-n} ‘\textsc{ipfv}’, found on the verb \textit{ma-} ‘go’.

\ea%24
    \label{ex:verbs:24}
          \textit{Tïn mï ankam maya} \textbf{\textit{man}}.\\
\gll tïn  mï      ankam  ma=iya      ma-\textbf{n}\\
    dig  3\textsc{sg.subj}  person  \textsc{3sg.obj}=toward  go{}-\textsc{ipfv}\\
\glt `The dog is going toward the person.’ [elicited]
\z

\isi{Habitual} action may also be indicated by use of the \isi{imperfective} \isi{suffix}. In examples \REF{ex:verbs:25} through \REF{ex:verbs:29}, \isi{atelic} verbs are used to refer not to events that occur at one particular \isi{time}, but rather to regular occurrences.

\ea%25
    \label{ex:verbs:25}
          \textit{Inom mï alum nunu ilom mat} \textbf{\textit{inde}}.\\
\gll inom  mï      alum  nunu  ilom  ma=tï      inda-\textbf{e}\\
    mother  3\textsc{sg.subj}  child  every  day    3\textsc{sg.obj=}take  walk-\textsc{iprv}\\
\glt `Mother carries the baby every day.’ [elicited]
\z

\ea%26
    \label{ex:verbs:26}
          \textit{Ginam mï mïnda} \textbf{\textit{ame}}.\\
\gll Ginam  mï      mïnda    ama-\textbf{e}\\
    [name]  3\textsc{sg.subj}  banana    eat-\textsc{ipfv}\\
\glt `Ginam eats bananas.’ [elicited]
\z

\ea%27
    \label{ex:verbs:27}
          \textit{Ginam mï ipka mïnda} \textbf{\textit{ame}}.\\
\gll Ginam  mï      ipka  mïnda  ama-\textbf{e}\\
    [name]  3\textsc{sg.subj}  before  banana  eat-\textsc{ipfv}\\
\glt `Ginam used to eat bananas before.’ [elicited]
\z

\ea%28
    \label{ex:verbs:28}
         \textit{Mï} \textbf{\textit{ane}}.\\
\gll mï      an=na-\textbf{e}\\
    3\textsc{sg.subj}  1\textsc{pl.excl}=feed-\textsc{ipfv}\\
\glt `She would feed us.’ [ulwa013\_00:33]
\z

\ea%29
    \label{ex:verbs:29}
          \textit{Nïnji inom mï nït} \textbf{\textit{inde}}.\\
\gll nï-nji    inom  mï       nï=tï    inda-\textbf{e}\\
    1\textsc{sg-poss}  mother  3\textsc{sg.subj}  1\textsc{sg}=take  walk-\textsc{ipfv}\\
\glt `My mother used to carry me around.’ [ulwa013\_00:34]
\z

Another use of \isi{imperfective} forms is to signal that an action began or is beginning. For the form and function of the \is{inchoative} \isi{inchoative imperfective}, see \sectref{sec:4.10}.

\is{verb|)}
\is{imperfective|)}
\is{aspect|)}

\section{Perfective aspect}\label{sec:4.5}

\is{aspect|(}
\is{perfective|(}
\is{verb|(}

Perfective aspect is applied to events that have reached their logical conclusion. This is, arguably, the \isi{semantic}ally unmarked form for a verb to take when referring to \isi{past} \isi{time}. When a \isi{perfective} form refers to \isi{present} \isi{time}, the \isi{verbal morphology} suggests that an event has just now happened. The regular \isi{perfective} \isi{suffix} is \textit{{}-p} ‘\textsc{pfv}’. Sentences \REF{ex:verbs:30} through \REF{ex:verbs:32} illustrate the use of the \isi{perfective} aspect.

\is{verb|)}
\is{perfective|)}
\is{aspect|)}

\ea%30
    \label{ex:verbs:30}
          \textit{Mï awal mïnda} \textbf{\textit{mamap}}.\\
\gll mï      awal    mïnda  ma=ama-\textbf{p}\\
    \textsc{3sg.subj}  yesterday  banana  \textsc{3sg.obj=}eat-\textsc{pfv}\\
\glt `He ate the banana yesterday.’ [elicited] \is{verb}
\z

\ea%31
    \label{ex:verbs:31}
          \textit{Mï amun mïnda} \textbf{\textit{mamap}}.\\
\gll mï      amun  mïnda  ma=ama-\textbf{p}\\
    \textsc{3sg.subj}  now  banana  \textsc{3sg.obj=}eat-\textsc{pfv}\\
\glt `He just now ate the banana.’ [elicited] \is{perfective}
\z

\ea%32
    \label{ex:verbs:32}
          \textit{Banjiwa mï numbu} \textbf{\textit{manip}}.\\
\gll Banjiwa  mï      numbu    ma=ni-\textbf{p}\\
    [name]    3\textsc{sg.subj}  garamut  3\textsc{sg.obj}=beat-\textsc{pfv}\\
\glt `Banjiwa has beaten the \textit{garamut} drum.’ [elicited] \is{aspect}
\z

\section{Irrealis mood}\label{sec:4.6}

\is{mood|(}
\is{irrealis|(}
\is{verb|(}

The third major \isi{TAM} \isi{suffix} does not encode the \isi{aspect} of an event, but rather its \isi{mood}: \isi{irrealis} \isi{mood} applies to unreal or \isi{hypothetical} events and states. The \isi{irrealis} \isi{suffix} is -\textit{na} ‘\textsc{irr}’, which has the \isi{phonological}ly conditioned \isi{allomorph} -\textit{nda} ‘\textsc{irr}’ (\sectref{sec:4.2}).

  The \isi{irrealis} \isi{mood} can be applied to verbs referring to events thought of as occurring in any \isi{temporal} frame. Examples \REF{ex:verbs:33} and \REF{ex:verbs:34} are both translated in \ili{English} as occurring in the \isi{future}. As a \isi{time} frame that is perforce \isi{hypothetical} or not (yet) real, the \isi{future} is almost always encoded in Ulwa with \isi{irrealis} verb forms. Note that \isi{aspect} -- \isi{perfective} or \isi{imperfective} -- is not specified by the \isi{irrealis} \isi{suffix}, as suggested by the multiple translations in \REF{ex:verbs:33} and \REF{ex:verbs:34}.

\ea%33
    \label{ex:verbs:33}
          \textit{Gambri mï umbe apa} \textbf{\textit{maytana}}.\\
\gll Gambri  mï      umbe    apa  ma=ita-\textbf{na}\\
    [name]    \textsc{3sg.subj}  tomorrow  house  \textsc{3sg.obj=}build-\textsc{irr}\\
\glt    (a) ‘Gambri will build the house tomorrow.’\\
    (b) ‘Gambri will be building the house tomorrow.’ [elicited]
\z

\ea%34
    \label{ex:verbs:34}
          \textit{Nungol ndï wambana} \textbf{\textit{nduwananda}}.\\
\gll nungol  ndï  wambana  ndï=wana-\textbf{nda}\\
    child  \textsc{3pl}  fish    3\textsc{pl}=cook-\textsc{irr}\\
\glt    (a) ‘The children will cook the fish.’\\
    (b) ‘The children will be cooking the fish.’ [elicited]
\z

The \isi{irrealis} \isi{suffix} can express a number of \isi{modal} distinctions, such as \isi{deontic} (‘should’, ‘must’), \isi{abilitative} (‘can’, ‘could’), and \isi{optative} (‘would that’) moods, as illustrated by \REF{ex:verbs:35}.

\ea%35
    \label{ex:verbs:35}
          \textit{Gambri mï (tap) apa} \textbf{\textit{maytana}}.\\
\gll Gambri  mï      (tap)    apa  ma=ita-\textbf{na}\\
    [name]    \textsc{3sg.subj}  (maybe)  house  \textsc{3sg.obj=}build-\textsc{irr}\\
\glt    (a) ‘Gambri should build the house.’\\
    (b) ‘Gambri can build the house.’\\
    (c) ‘Would that Gambri were building the house!’ [elicited]
\z

The \isi{epistemic adverb} \textit{tap} ‘maybe’ is often possible in these sentences, but it is not necessary for conveying \isi{irrealis} \isi{mood} (\sectref{sec:8.2.4}). Note, however, that \textit{tap} ‘maybe’ cannot be used when the speaker wishes to convey that the action will necessarily happen (i.e., ‘will’, ‘must’). Thus, sentence \REF{ex:verbs:35} (with \textit{tap} ‘maybe’ included), cannot mean *‘Gambri will build the house’ or *‘Gambri must build the house’.

  When used in reference to \isi{past} \isi{time}, the \isi{irrealis} \isi{mood} indicates \isi{potential} (i.e., ability) or lack thereof \REF{ex:verbs:36}.

\ea%36
    \label{ex:verbs:36}
          \textit{Ndï ango luwa miniya} \textbf{\textit{mana}}.\\
\gll ndï  ango  luwa  min=iya    ma-\textbf{na}\\
    3\textsc{pl}  \textsc{neg}  place  3\textsc{du}=toward  go-\textsc{irr}\\
\glt `They could not go to them.’ (Literally ‘They nowhere could go to them.’) [ulwa001\_18:34]
\z

When used in reference to \isi{past} \isi{time}, the \isi{irrealis} \isi{mood} can also indicate a \isi{counterfactual} statement, as in \REF{ex:verbs:37}.

\ea%37
    \label{ex:verbs:37}
          \textit{Awal maka nungol ndul li} \textbf{\textit{mana}}.\\
\gll awal     maka   nungol  ndï=ul    li     ma-\textbf{na}\\
    yesterday  thus  child  3\textsc{pl}=with  down  go-\textsc{irr}\\
\glt `[He] would have gone down with [his] children yesterday.’ [ulwa014\_33:22]
\z

Sentence \REF{ex:verbs:37} is proven to be \isi{counterfactual} by the speaker’s immediately following sentence \REF{ex:verbs:38}, which shows that this intended action of the man going with his children was unrealized.

\ea%38
    \label{ex:verbs:38}
          \textit{Ticha ngala mbiye em i stap.}\\
\gll    ticha   ngala     mbï-i-e       em    i     stap\\
    teacher  \textsc{pl.prox}  here-go.\textsc{pfv-dep}  3\textsc{sg}  \textsc{pred}  stay\\
\glt `But these teachers came, so he stayed.’ (\textit{ticha}, \textit{em}, \textit{i}, \textit{stap} = TP) [ulwa014\_33:26]
\z

In some \isi{multiclausal construction}s, the \isi{irrealis} form of a verb may be best translated by an \ili{English} \isi{infinitive}, showing \isi{purpose} or \isi{intention}, as in \REF{ex:verbs:39} and \REF{ex:verbs:40}.

\ea%39
    \label{ex:verbs:39}
          \textit{Wa me ndul} \textbf{\textit{landa}}.\\
\gll wa    ma=i        ndï=ul    la-\textbf{nda}\\
    village  3\textsc{sg.obj}=go.\textsc{pfv}  3\textsc{pl}=with  eat-\textsc{irr}\\
\glt `[They] went home to eat with them.’ [ulwa002\_05:56]
\z

\ea%40
    \label{ex:verbs:40}
          \textit{Malimap matï yawa} \textbf{\textit{mananda}}.\\
\gll  ma=alima-p    ma=tï      yawa  ma=na-\textbf{nda}\\
    3\textsc{sg.obj}=beat-\textsc{pfv}  \textsc{3sg.obj}=take  uncle  3\textsc{sg.obj}=give-\textsc{irr}\\
\glt `[We] beat it [= sago starch] to give to [our] uncle.’ [ulwa014\_00:15]
\z

Sentences such as \REF{ex:verbs:39} and \REF{ex:verbs:40} are analyzed as consisting of two full clauses, so there is actually nothing akin \isi{syntactic}ally to the \ili{English} \isi{infinitive}. The absence of \is{dependent marker} dependent marking (\sectref{sec:12.2.1}) on the first clauses of examples \REF{ex:verbs:39} and \REF{ex:verbs:40} suggests that the clauses containing these \isi{purpose}-denoting \isi{irrealis} verbs are actually \isi{independent} sentences, without any sentences \isi{dependent} upon them. Thus, the \isi{irrealis} \isi{suffix} can be considered here a means of imbuing a \isi{desiderative} or \isi{intentive} meaning to the verb. Thus, \REF{ex:verbs:39} could be translated ‘[They] went home; [they] wanted to eat with them’; and \REF{ex:verbs:40} could be translated ‘[We] beat it; [we] intended to give it to [our] uncle’.

\is{verb|)}
\is{irrealis|)}
\is{mood|)}

\section{Imperative}\label{sec:4.7}

\is{imperative|(}
\is{verb|(}

The three basic \isi{TAM} markings in Ulwa account for much of the \isi{suffix}al \isi{verbal morphology} of all \isi{declarative} and \isi{interrogative} sentences. In \isi{imperative} sentences, however, verbs in Ulwa may receive the \isi{imperative} \isi{suffix} \textit{-n} ‘\textsc{imp}’. For the \isi{syntax} and function of \isi{imperative} clauses in Ulwa, see \sectref{sec:13.2}.

  \tabref{tab:4.11} presents a sample of Ulwa verbs to illustrate their \isi{imperative} forms, shown along with the \isi{irrealis} forms for comparison.

\begin{table}
\caption{Some imperative verb forms}
\is{verb stem}
\is{imperative}
\is{irrealis}
\label{tab:4.11}
\begin{tabularx}{\textwidth}{QQQQ}
\lsptoprule
gloss & verb stem(s) & irrealis & imperative\\
\midrule
‘eat’ & {\itshape ama-, la-} & {\itshape landa} & {\itshape lan}\\
‘let’ & {\itshape ka-, laka-} & {\itshape lakana} & {\itshape lakan}\\
‘say’ & {\itshape kï-, ka-} & {\itshape kïna} & {\itshape kïn}\\
‘cut, go’ & {\itshape lo-, lu-} & {\itshape lunda} & {\itshape lun}\\
‘go’ & {\itshape ma-, i-} & {\itshape mana} & {\itshape man}\\
‘sew’ & {\itshape me-} & {\itshape menda} & {\itshape men}\\
‘give’ & {\itshape na-} & {\itshape nanda} & {\itshape nan}\\
‘put’ & {\itshape u-} & {\itshape unda} & {\itshape un}\\
‘hit’ & {\itshape wali-} & {\itshape walinda} & {\itshape walin}\\
\lspbottomrule
\end{tabularx}
\end{table}

The form [-n] found in \isi{imperative} verbs may be related historically to the \isi{irrealis} \isi{suffix} \textit{-na} ‘\textsc{irr}’. \footnote{At any rate, there is indeed something kindred between \isi{irrealis} and \isi{imperative} forms, since, in \isi{irregular verb}s that exhibit different \isi{stem}s in the \isi{irrealis} \isi{mood}, the \isi{imperative} ending will affix to the \isi{irrealis} \isi{verb stem}, never to the \isi{perfective}/\isi{imperfective} \isi{stem}. Thus, for example, the \isi{imperative} of ‘eat’ is [lan] (cf. \textit{la-nda} ‘eat [\textsc{irr}]’) and not \textsuperscript{†}[aman] (cf. \textit{ama-p} ‘eat [\textsc{pfv}]’) and the \isi{imperative} of ‘let’ is [lakan] (cf. \textit{laka-na} ‘let [\textsc{irr}]’) and not \textsuperscript{†}[kan] (cf. \textit{ka} ‘let [\textsc{ipfv/pfv}]’). Furthermore, there is a \isi{semantic} similarity between the two \isi{suffix}es, since, among other things, the \isi{irrealis} \isi{suffix} can encode \isi{deontic} \isi{mood} (i.e., ‘must’), which, when expressed in an utterance, is not unlike issuing an \isi{imperative}.}

\is{verb|)}
\is{imperative|)}

\section{The double perfective}\label{sec:4.8}

\is{double perfective|(}
\is{perfective|(}
\is{verb|(}

As detailed in sections \sectref{sec:4.4}--\sectref{sec:4.7}, an \isi{inflect}ed verb in Ulwa typically has exactly one \isi{TAM} \isi{suffix} (which may be \is{null suffix} null in certain \isi{imperfective} verb forms). There are circumstances, however, in which \isi{perfective} verbs may be marked twice -- that is, they take the form: [\isi{stem}]-[\textsc{pfv}]-[\textsc{pfv}]. In such instances, the second \isi{perfective} marker adopts the \isi{vowel} from the \isi{verb stem}, preventing the otherwise impossible sequence \textsuperscript{†}[pp]. Thus, for example, verbs with [a]-final \isi{stem}s have the form [\isi{stem}]-[p]-[ap], and verbs with [o]-final \isi{stem}s have the form [\isi{stem}]-[p]-[op] (although there seems to be some variation allowed).

  The \isi{semantic} effect of this \isi{double perfective} is often one of signaling that an action is all-the-more over-and-done-with. Since a single \isi{perfective} marker typically signals that the event is viewed as whole and completed, this double marking could be seen as superfluous.\footnote{Indeed, it may be that -- as speakers use different \isi{TAM} \isi{suffix}es ever more interchangeably, perhaps as the result of \isi{grammatical attrition} -- the extra \isi{perfective} marking is simply redundant (see \chapref{sec:15} for structural changes due to \isi{grammatical attrition}).} There are instances, however, in which the \isi{double perfective} functions something like the \isi{pluperfect} category of some European languages, showing that an event is not only viewed as a completed whole, but that is has been completed before the \isi{time} of some other event in the more-recent \isi{past}. These usages can often be translated with the \ili{English} \isi{auxiliary} ‘had’ plus the past participle, as in examples \REF{ex:verbs:41} through \REF{ex:verbs:45}.


\is{verb|)}
\is{perfective|)}
\is{double perfective|)}
\is{double perfective|(}
\is{perfective|(}
\is{verb|(}

\ea%41
    \label{ex:verbs:41}
          \textit{Man} \textbf{\textit{nïkapap}}.\\
\gll ma=n      nï=kï{}-\textbf{p-ap}\\
    3\textsc{sg.obj=obl}  1\textsc{sg}=say-\textsc{pfv}{}-\textsc{pfv}\\
\glt `[She] had told me.’ [ulwa014\_11:15]
\z

\ea%42
    \label{ex:verbs:42}
          \textit{Mana man} \textbf{\textit{masapap}}.\\
\gll mana  ma=n      ma=asa-\textbf{p-ap}\\
    spear  3\textsc{sg.obj=obl}  3\textsc{sg.obj}=hit-\textsc{pfv-pfv}\\
\glt `[They] had killed him with a spear.’ [ulwa037\_11:41]
\z

\ea%43
    \label{ex:verbs:43}
          \textit{Nï mape Madang pe} \textbf{\textit{ndïlopop}}.\\
\gll nï    ma=p-e      Madang  p-e   ndï=lo-\textbf{p-op}\\
    1\textsc{sg}  3\textsc{sg.obj}=be\textsc{{}-dep}  [place]    be\textsc{{}-dep}  3\textsc{pl}=cut-\textsc{pfv-pfv}\\
\glt `I had made them when I was there in Madang.’ [ulwa014\_11:34]
\z

\ea%44
    \label{ex:verbs:44}
         \textit{Ndïnji inga mol} \textbf{\textit{lopope}}.\\
\gll ndï-nji    inga  ma=ul       lo-\textbf{p-op}{}-e\\
    3\textsc{pl-poss}  affine  3\textsc{sg.obj}=with  go-\textsc{pfv-pfv-dep}\\
\glt `[They] had gone with their in-law.’ [ulwa035\_03:33]
\z

\ea%45
    \label{ex:verbs:45}
          \textit{Asika lïmndï} \textbf{\textit{ndïlpïpe}} \textit{ngala luke asi tï nap ndala une.}\\
   \gll asi-ka  lïmndï  ndï=lï-\textbf{p-ïp}{}-e      ngala    luke   asi   tï    na-p    ndï=ala  uni-e\\
    sit-let  eye    \textsc{3pl}=put-\textsc{pfv-pfv-dep}  \textsc{pl.prox}  too     sit    take  \textsc{detr}{}-be  3\textsc{pl}=for  shout-\textsc{ipfv}\\

\glt `After [they] had sat and watched them, these people also took seats, cheering them on.’ [ulwa032\_36:45]
\z

The \isi{double perfective} can, similarly, provide the sense of ‘already’, and is translated accordingly in \REF{ex:verbs:46}.

\ea%46
    \label{ex:verbs:46}
          \textit{Numbu ala} \textbf{\textit{nungunupop}}.\\
\gll numbu  ala       nungun-u-\textbf{p-op}\\
    post   \textsc{pl.dist}  break-put-\textsc{pfv-pfv}\\
\glt `Those posts have already broken.’ [ulwa042\_05:18]
\z

It is also possible for the word \textit{ta} ‘already’ to appear within a clause exhibiting such a construction \REF{ex:verbs:47}.\footnote{In example \REF{ex:verbs:47}, however, it may be that the form /nip/ has been \isi{reanalyzed} as monomorphemic, a new verb ‘die’, derived from the \isi{perfective} form of the verb \textit{ni-} ‘die’ (cf. the \isi{pluractional} version of this verb, \textit{nipinpu-} ‘die.\textsc{pl}’, \sectref{sec:4.3}).}

\ea%47
    \label{ex:verbs:47}
          \textit{Nïnji wot yena mï ta} \textbf{\textit{nipop}}.\\
\gll nï-nji    wot      yena    mï      \textbf{ta} ni-\textbf{p-op}\\
    \textsc{1sg-poss}  younger  woman    \textsc{3sg.subj}  already    die-\textsc{pfv-pfv}\\
\glt `My younger sister has already died.’ [ulwa027\_00:33]
\z

  Sentences \REF{ex:verbs:46} and \REF{ex:verbs:47} also illustrate the fact that the \isi{vowel}s in the second \isi{perfective} \isi{suffix} do not always match the final \isi{vowel} of the \isi{verb stem}. Indeed, there is variability within certain verb forms. For example, there are attested forms such as \textit{lï-p-ap} ‘put-\textsc{pfv-pfv}’ alongside \textit{lï-p-ïp} ‘put-\textsc{pfv-pfv}’, as seen in \REF{ex:verbs:45}. It may be that some of these putative second \isi{perfective} forms are actually reduced forms of the \isi{past} \isi{locative verb} \textit{wap} ‘be.\textsc{pst}’ (\sectref{sec:10.4}).\footnote{Indeed, it is possible that they have all derived historically from \textit{wap} ‘be.\textsc{pst}’, which is often pronounced [wɔp]: thus, it may be that first the /w/ was lost; then, when following non-low \isi{vowel}s, the \isi{vowel} [ɔ] was colored to [o], and when following the \isi{low vowel}, it was colored to [a]. There are no attested (double) \isi{perfective} forms with the \isi{vowel}s [e] or [u] (i.e., \textsuperscript{†}[-ep] and \textsuperscript{†}[-up] are not found as \isi{perfective} \isi{suffix}es).} Sentences \REF{ex:verbs:48} and \REF{ex:verbs:49} further exemplify the form [-op], following the \isi{stem}-final \isi{vowel}s /i/ and /u/, respectively.

\ea%48
    \label{ex:verbs:48}
          \textit{Ane nda ine nda} \textbf{\textit{nipop}}.\\
\gll ane  anda    i-n-e      anda    n\textbf{i}{}-p-\textbf{op}\\
    sun  \textsc{sg.dist}  come-\textsc{pfv-dep}  \textsc{sg.dist}  die-\textsc{pfv-pfv}\\
\glt `That [woman] died the day before yesterday.’ [ulwa037\_36:02]
\z

\ea%49
    \label{ex:verbs:49}
          \textit{John maweka i Mongima ul ngalan} \textbf{\textit{upop}}.\\
\gll John  maweka  i    Mongima  ul    ngala=n \textbf{u}{}-p-\textbf{op}\\
    [name]  also    go.\textsc{pfv}  [name]    with  \textsc{pl.prox=obl}     put-\textsc{pfv-pfv}\\

\glt `John had also gone and planted these with Mongima.’ [ulwa014\_55:24]
\z

In many instances, as in \REF{ex:verbs:50} and \REF{ex:verbs:51}, it is not clear whether the inclusion of a \isi{double perfective} should be taken to convey any sense different from that of a regular (i.e., single) \isi{perfective} verb form.

\newpage

\ea%50
    \label{ex:verbs:50}
          \textit{An} \textbf{\textit{ndamapape}} \textit{inim nga ambip.}\\
\gll    an      ndï=ama-\textbf{p-ap}{}-e     inim  nga      ambi=p\\
    1\textsc{pl.excl}  3\textsc{pl}=eat-\textsc{pfv-pfv-dep}  water  \textsc{sg.prox}  big=\textsc{cop}\\
\glt `We were eating them [= fish], but [now] the water is high [again].’ [ulwa014\_29:36]
\z

\ea%51
    \label{ex:verbs:51}
          \textit{Mï maka aw ndïn} \textbf{\textit{mopop}}.\\
\gll mï      maka  aw      ndï=n    ma=u-\textbf{p-op}\\
    \textsc{3sg.subj}  thus  betel.nut  \textsc{3pl=obl}  \textsc{3sg.obj}=put-\textsc{pfv{}-pfv}\\
\glt `He had [?] planted the betel nut there.’ [ulwa014†]
\z

Some apparent examples of \isi{double perfective} marking may reflect \isi{grammatical attrition}: as verbal \isi{suffix}es come to be used in increasingly interchangeable ways, they lose their \isi{aspect}ual force. Perhaps such seemingly redundant \isi{perfective} markers are used to show that the meaning intended is truly \isi{perfective}. For example, although the form \textit{i} ‘go.\textsc{pfv}’ is intrinsically marked for \isi{perfective} \isi{aspect}, speakers on occasion add what seems to be a \isi{perfective} \isi{suffix} -- that is, as if they were treating this form as unmarked for \isi{aspect} and thus requiring a \isi{perfective} \isi{suffix}. Example \REF{ex:verbs:52} contains the \isi{perfective} form \textit{i} ‘go.\textsc{pfv}’ with an unnecessary additional \isi{perfective} marker \textit{-ap} ‘\textsc{pfv}’, realized here as [-yap] due to \isi{glide} insertion (\sectref{sec:2.5.1}).

\ea%52
    \label{ex:verbs:52}
          \textit{Li kïkal wopa nda ango} \textbf{\textit{mbiyap}}.\\
\gll li-i        kïkal  wopa  nda      ango  mbï-i-\textbf{ap}\\
    down-go.\textsc{pfv}  ear    all    \textsc{sg.dist}  \textsc{neg}  here-go.\textsc{pfv}{}-\textsc{pfv}\\
\glt `[She] went downstream, but that deaf one did not stay here.’ [ulwa014†]
\z

Example \REF{ex:verbs:53}, on the other hand, shows how \textit{i} ‘go.\textsc{pfv}’ -- with the added \isi{suffix} \textit{-ap} ‘\textsc{pfv}’ -- can function in a \isi{double perfective} construction.

\ea%53
    \label{ex:verbs:53}
          \textit{Ndï lïmndï ute} \textbf{\textit{iyapen}}.\\
\gll ndï  lïmndï  u=uta-e      i-\textbf{ap}{}-en\\
    3\textsc{pl}  eye   2\textsc{sg}=grind-\textsc{ipfv}  go.\textsc{pfv-pfv-nmlz}\\
\glt `They were the ones who had gone and watched over you.’ [ulwa013\_01:25]
\z

Example \REF{ex:verbs:54} shows how a \isi{double perfective} construction can function when \textit{i} ‘go.\textsc{pfv}’ is \isi{reanalyzed} as lacking (intrinsic) \isi{TAM} marking. In this instance, the speaker uses a perfective-marked form of the \isi{past}-\isi{tense} \isi{locative verb} \textit{wap} ‘be.\textsc{pst}’ as something like an \isi{auxiliary verb}.\footnote{I suspect that this is a recent innovation, one influenced by \isi{grammatical attrition} in the face of competing influences from the dominant language, \ili{Tok Pisin} (see \chapref{sec:15}).}

\is{verb|)}
\is{perfective|)}
\is{double perfective|)}

\ea%54
    \label{ex:verbs:54}
          \textit{Ngata ala \textbf{i wapapen}}.\\
\gll ngata  ala       i    wap-\textbf{ap}-en\\
    grand  \textsc{pl.dist}  go.\textsc{pfv}  be.\textsc{pst-pfv-nmlz}\\
\glt `[Our] ancestors are the ones who had gone [there].’ [ulwa014\_23:29]
\z

\section{The irrealis perfective}\label{sec:4.9}

\is{irrealis perfective|(}
\is{irrealis|(}
\is{perfective|(}
\is{verb|(}

An interesting tug-of-war occurs when one must refer to a completed action in \isi{future} \isi{time} (cf. the \isi{future} perfect in \ili{English}, e.g., \textit{we will have eaten}). The three designations in Ulwa’s basic three-way \isi{TAM} system are not mutually exclusive: that is, an event could, theoretically, be viewed both as a completed whole (\isi{perfective} \isi{aspect}) and as something not (yet) real (\isi{irrealis} \isi{mood}). Typically in Ulwa, all \isi{irrealis}-\isi{mood} verbs are treated the same -- that is, there are no \isi{aspect}ual distinctions maintained among them (\sectref{sec:4.6}). Thus, the sentence in \REF{ex:verbs:55} could be translated variously into \ili{English}.

\ea%55
    \label{ex:verbs:55}
          \textit{Nungol ndï} \textbf{\textit{landa}}.\\
\gll nungol  ndï  \textbf{la-nda}\\
    child  3\textsc{pl}  eat-\textsc{irr}\\
\glt    (a) ‘The children will eat.’ (unspecified \isi{aspect})\\
    (b) ‘The children will be eating.’ (\isi{imperfective} \isi{aspect})\\
    (c) ‘The children will have eaten.’ (\isi{perfective} \isi{aspect}) [elicited]
\z

In certain \isi{multiclausal construction}s, however, it may become necessary to designate the \isi{aspect} of an \isi{irrealis} event as being \isi{perfective}. Namely, when one \isi{future} event is contingent on the completion of another, this yet-to-be-completed event can be marked with a \isi{perfective} \isi{suffix} \REF{ex:verbs:56}. Thus, in these tug-of-war scenarios between \isi{aspect} and \isi{mood}, \isi{aspect} wins.

\newpage

\ea%56
    \label{ex:verbs:56}
              \textit{Anji wa} \textbf{\textit{koytap}} \textit{namndu nungol kot ma mat malnda.}\\

    \gll an-nji        wa     ko=ita-\textbf{p}      namndu  nungol ko=tï     ma  ma=tï      ma=lï-nda\\
    1\textsc{pl.excl-poss}  village  \textsc{indf}=build-\textsc{pfv}  pig      child   \textsc{indf}=take  go  \textsc{3sg.obj}=take  3\textsc{sg.obj}=put-\textsc{irr}\\


\glt `Once [we] have built a village for ourselves and gotten a pig, [we] will go and put it there.’ [ulwa014\_07:07]
\z

Such \isi{irrealis} \isi{perfective} constructions are especially common in multiclausal \isi{imperative}s. In \REF{ex:verbs:57}, the \isi{perfective} verb is the \isi{suppletive} form \textit{i} ‘go.\textsc{pfv’.}

\ea%57
    \label{ex:verbs:57}
          \textit{Un} \textbf{\textit{i}} \textit{anul ndul amblawalin.}\\
\gll un  \textbf{i}    an=ul        ndï=ul    ambla=wali-n\\
    2\textsc{pl}  go.\textsc{pfv}  1\textsc{pl.excl}=with  3\textsc{pl}=with  \textsc{pl.refl}=hit-\textsc{imp}\\
\glt `Go and fight with us with [= against] them!’ (Literally ‘You having gone, fight with us with them!’) [ulwa002\_05:19]
\z

A verb in the first clause of such an \isi{imperative} may be marked with the \isi{conditional} \isi{suffix} \textit{-ta} ‘\textsc{cond}’ (\sectref{sec:4.12}). Especially when the subject of the first clause differs from that of the second, the inclusion of this \isi{suffix} may be seen as necessary to convey the \isi{irrealis} \isi{perfective} sense of the \isi{protasis}: in \REF{ex:verbs:58}, the \isi{conditional} \isi{suffix} on the first verb, \textit{anmbi} ‘come out [\textsc{pfv]}’, helps signal that the action of the following verb is contingent on the completion of the action described by this one.

\ea%58
    \label{ex:verbs:58}


          \textit{Ngun \textbf{anmbita} una \textbf{malamape} una lowon.}\\
    \gll ngun  an-mbï-i-\textbf{ta}        unan    ma=la{}-ama-\textbf{p}{}-e      unan    lo-wo-n\\
    2\textsc{du}  out-here-go.\textsc{pfv-cond}  \textsc{pl.incl}    3\textsc{sg.obj}=\textsc{irr}{}-eat-\textsc{pfv-dep}  1\textsc{pl.incl}  \textsc{irr}{}-sleep-\textsc{imp}\\


\glt `Once you two have come and we’ve eaten it, let’s sleep!’ [ulwa041\_01:15]
\z

Sometimes, in the tug-of-war between \isi{perfective} \isi{aspect} and \isi{irrealis} \isi{mood}, there is a \isi{morphological} tie, at least in instances in which the verb form allows some indication of \isi{irrealis} \isi{mood} by means other than \isi{suffix}ation. This applies to verbs that have \isi{circumfix}-like \isi{irrealis} forms beginning with [lo-] or [la-], such as \textit{wo-} ‘sleep’ and \textit{ama-} ‘eat’ (\sectref{sec:4.3}). In \isi{irrealis} \isi{perfective} constructions involving these verbs, the verb combines the \isi{irrealis} \isi{stem} with the \isi{perfective} \isi{suffix}.

  For example, the form \textit{lowop} ‘sleep [\textsc{irr/pfv}]’ combines the beginning of the \isi{irrealis} form \textit{lowonda} ‘sleep [\textsc{irr}]’ with the ending of the \isi{perfective} form \textit{wop} ‘sleep [\textsc{pfv}]’ (cf. \sectref{sec:4.3}). It is thus capable of indicating \isi{perfective} \isi{aspect} and \isi{irrealis} \isi{mood} simultaneously, as in \REF{ex:verbs:59} and \REF{ex:verbs:60}.

\ea%59
    \label{ex:verbs:59}

          \textit{Wa ma} \textbf{\textit{lowop}} \textit{ma siwi anglalunda mane.}\\

\gll    wa     ma  \textbf{lo-wo-p}    ma  siwi     angla-lo{}-nda    ma-n-e\\
    village  go  \textsc{irr}{}-sleep-\textsc{pfv}  go  grub.species  await-go-\textsc{irr}    go-\textsc{ipfv-dep}\\


\glt `[He] was going to the village, and, having slept [there], was going to search for grubs.’ [ulwa038\_03:51]
\z

\ea%60
    \label{ex:verbs:60}

          \textit{Maka} \textbf{\textit{lowop}} \textit{apa mot anda luke itana mane.}\\

\gll    ma=ka      \textbf{lo-wo-p}    apa  mot     anda     luke    ita-na    ma-n-e\\
    \textsc{3sg.obj}=at  \textsc{irr}{}-sleep-\textsc{pfv}  house  awning  \textsc{sg.dist}  too    build-\textsc{irr}  go-\textsc{ipfv-dep}\\


\glt `Having slept there, I was going to build that house awning, too.’ [ulwa042\_04:44]
\z

In example \REF{ex:verbs:61}, the form \textit{lowop} ‘sleep [\textsc{irr/pfv}]’ occurs with some \isi{borrow}ed grammar from \ili{Tok Pisin} (\textit{i no laik}, literally ‘do not want’; here, roughly, ‘should’).

\ea%61
    \label{ex:verbs:61}

          \textit{Un i no laik anul mbï ka} \textbf{\textit{lowop}} \textit{mana?}\\

\gll    un  i    no    laik  an=ul        mbï  ka    \textbf{lo-wo-p}    ma-na\\
    2\textsc{pl}  \textsc{pred}  \textsc{neg}  want  1\textsc{pl.excl}=with  here  thus      \textsc{irr}{}-sleep-\textsc{pfv}  go-\textsc{irr}\\

\glt `Why don’t you spend the night here with us and [then] go?’ (\textit{i no laik} = TP) [ulwa014\_16:04]
\z

In a somewhat different but related manner, the \isi{irrealis} \isi{perfective} form of \textit{ama-} ‘eat’ seems to combine two \isi{stem}s in the same form: the \isi{irrealis} \isi{stem} [la-] and the non-\isi{irrealis} \isi{stem} [ama-] occur together, along with the \isi{perfective} \isi{suffix} \textit{{}-p} ‘\textsc{pfv}’, thus yielding \textit{lamap} ‘eat [\textsc{irr/pfv]}’. Although elsewhere in examples containing this verb the form [la-] is glossed as the \isi{verb stem} ‘eat’, in \isi{irrealis} \isi{perfective} constructions such as those in \REF{ex:verbs:62} and \REF{ex:verbs:63}, it is glossed as ‘\textsc{irr}’, since it is the part of the verb form that is signaling its \isi{irrealis} value. The \isi{vowel} of /la-/ \isi{deletes} before the initial \isi{vowel} of /ama-/ (\sectref{sec:2.5.5}).

\ea%62
    \label{ex:verbs:62}

          \textbf{\textit{Ndïlamap}} \textit{we un namndu atïna.}\\
\gll    ndï=\textbf{la{}-ama-p} we    un  namndu   atï-na\\
    3\textsc{pl}=\textsc{irr}{}-eat-\textsc{pfv}  then  2\textsc{pl}  pig      hit-\textsc{irr}\\
\glt `Once [we] have eaten them, then you will kill pigs!’ [ulwa014\_43:54]
\z

\ea%63
    \label{ex:verbs:63}
          \textit{Mol} \textbf{\textit{lamap}} \textit{mana mat mananda.}\\
\gll    ma=ul       \textbf{la{}-ama-p} mana  ma=tï    ma=na-nda\\
    3\textsc{sg.obj}=with  \textsc{irr}{}-eat-\textsc{pfv}  spear  \textsc{3sg.obj}=take    \textsc{3sg.obj}=give-\textsc{irr}\\


\glt `Having eaten with him, [they] will give him the spear.’ [ulwa014\_62:16]
\z

In the text that contains sentence \REF{ex:verbs:63}, there is another \isi{irrealis} \isi{perfective} construction that immediately follows: \REF{ex:verbs:64}. Here it may be seen again that -- as for most verbs -- the \isi{irrealis} \isi{perfective} form of \textit{ita-} ‘build’ is \isi{morphological}ly identical to the \isi{perfective} form.

\is{verb|)}
\is{perfective|)}
\is{irrealis|)}
\is{irrealis perfective|)}

\ea%64
    \label{ex:verbs:64}
          \textit{Mana} \textbf{\textit{maytap}} \textit{mat mananda.}\\
\gll    mana  ma=\textbf{ita-p}         ma=tï      ma=na-nda\\
    spear  3\textsc{sg.obj}=build-\textsc{pfv}  3\textsc{sg.obj}=take   3\textsc{sg.obj}=give-\textsc{irr}\\
\glt `Having made the spear, [they] will give it to him.’ [ulwa014\_62:19]
\z

\section{The inchoative imperfective}\label{sec:4.10}

\is{inchoative imperfective|(}
\is{inchoative|(}
\is{imperfective|(}
\is{verb|(}

There is a special use of \isi{imperfective} verb forms that may at first seem to run counter to its typically \isi{continuous} \isi{aspect}ual force. Imperfective verbs may be used to signal that an action is beginning or starting. This may be referred to as \isi{inchoative} \isi{aspect} (or \isi{inceptive} \isi{aspect}). Indeed, there is nothing technically \isi{atelic} about verbs denoting the commencement of an action. That said, the \isi{inchoative} \isi{imperfective} verb usually does maintain the sense of uncompleted action (i.e., it encodes that an action was started but interrupted or that an action has begun but has not yet reached its conclusion, both of which actions are ongoing). Sentence \REF{ex:verbs:65} illustrates the use of the \isi{inchoative} \isi{imperfective}.

\ea%65
    \label{ex:verbs:65}

          \textit{We mokotïp mat manane mï} \textbf{\textit{mame}}.\\
\gll we    ma=kot-p        ma=tï      ma=na-n-e mï      ma=ama-\textbf{e}\\
    sago  3\textsc{sg.obj}=break{}-\textsc{pfv}  \textsc{3sg.obj}=take  3\textsc{sg.obj}=give\textsc{{}-pfv-dep}    \textsc{3sg.subj}  \textsc{3sg.obj}=eat-\textsc{ipfv}\\


\glt `[He] broke a piece of sago and gave it to him, and he began to eat it.’ [ulwa001\_11:42]
\z

In \REF{ex:verbs:66}, it is ambiguous whether the ending [-e] on the verb \textit{sa-} ‘cry’ is encoding (only) clausal dependence or (additionally) the \isi{inchoative} \isi{imperfective} \isi{aspect}.

\ea%66
    \label{ex:verbs:66}
          \textit{Mï} \textbf{\textit{se}} \textit{nï mala ndïwanawne.}\\
\gll    mï       sa-\textbf{e}     nï     ma=ala     ndï=wana-uni-e\\
    3\textsc{sg.subj}  cry-\textsc{dep}  \textsc{1sg}  \textsc{3sg.obj=}for  3\textsc{pl}=feel-shout-\textsc{ipfv}\\
\glt `When she started to cry, I called to them to get her.’ [ulwa032\_04:28]
\z

In \REF{ex:verbs:67}, the \isi{inchoative} \isi{imperfective} is marked by the irregular \isi{imperfective} \isi{suffix} \textit{{}-n} ‘\textsc{ipfv}’.

\ea%67
    \label{ex:verbs:67}

          \textit{Tana kot ambïn wutï anmot ngalïp anankïn ala li} \textbf{\textit{mane}}.\\
\gll tana  ko=tï    ambï=n    wutï  anmot  nga=lï-p anankïn  ala       li     ma-\textbf{n}{}-e\\
    axe    \textsc{indf}=take  \textsc{sg.refl=obl}  leg    post  \textsc{sg.prox}=put-\textsc{pfv}    blood    \textsc{pl.dist}  down  go-\textsc{ipfv-dep}\\


\glt `[He] cut his shin with an axe, and blood began to run down.’ [ulwa009\_00:25]
\z

In \REF{ex:verbs:68}, the \isi{inchoative} \isi{imperfective} is indicated by the \isi{suppletive} \isi{imperfective} \isi{stem} of the verb \textit{u-} ‘put’, without any overt \isi{suffix}.

\ea%68
    \label{ex:verbs:68}
          \textit{Ndï kïkal ndïwana} \textbf{\textit{ndïnawte}} \textit{inim} \textbf{\textit{naw}}.\\
\gll ndï  kïkal  ndï=wana  ndï=na-uta-\textbf{e}       inim   na{}-\textbf{aw}\\
    3\textsc{pl}  ear    3\textsc{pl}=feel  3\textsc{pl}=\textsc{detr}{}-grind-\textsc{ipfv}  water  \textsc{detr}{}-put.\textsc{ipfv}\\
\glt `They heard them [= their names] and started grinding them [= coconuts] into water.’ [ulwa018\_02:14]
\z

Likewise, in \REF{ex:verbs:69}, no overt \isi{suffix} is present, but the [w]-final \isi{verb stem} \textit{wow-} ‘sleep.\textsc{ipfv}’ indicates that \isi{imperfective} \isi{aspect} is intended.

\ea%69
    \label{ex:verbs:69}
          \textit{Mawap imba pe mï wolka} \textbf{\textit{nawow}}.\\
\gll ma=wap       imba  p-e   mï       wolka  na-\textbf{wow}\\
    3\textsc{sg.obj}=be.\textsc{pst}  night  be\textsc{{}-dep} 3\textsc{sg.subj}  again  \textsc{detr-}sleep.\textsc{ipfv}\\
\glt `[He] stayed the night there and again he fell asleep.’ [ulwa006\_04:33]
\z

A minor variation to \isi{inceptive} or \isi{ingressive} \isi{aspect} may be termed \isi{resumptive} \isi{aspect}. This \isi{aspect} can also be encoded by \isi{imperfective} verb forms to signal that an action that had stopped has begun again, as in \REF{ex:verbs:70}.

\is{verb|)}
\is{imperfective|)}
\is{inchoative|)}
\is{inchoative imperfective|)}

\ea%70
    \label{ex:verbs:70}
         \textit{Mï numbu} \textbf{\textit{mole}}.\\
\gll mï       numbu   ma=lo-\textbf{e}\\
    3\textsc{sg.subj}  garamut  3\textsc{sg.obj}=cut-\textsc{ipfv}\\
\glt `He resumed making the \textit{garamut} drum.’ [ulwa009\_00:56]
\z

\section{The speculative suffix \textit{-t} ‘\textsc{spec}’}\label{sec:4.11}

\is{speculative|(}
\is{verb|(}

As detailed in \sectref{sec:4.6}, the \isi{irrealis} \isi{suffix} can express a number of \isi{modal}ities, including various predictions, such as that a state or event might be or might happen. There is also, however, a verbal \isi{suffix} \textit{-t} ‘\textsc{spec’}, which can convey a sense of \isi{epistemic} possibility. It immediately follows the \isi{irrealis} \isi{suffix} on the verb, as seen in \REF{ex:verbs:71} and \REF{ex:verbs:72}.

\ea%71
    \label{ex:verbs:71}
          \textit{Nakanaka} \textbf{\textit{nundate}}.\\
\gll na-kanaka    lu{}-nda-\textbf{t}{}-e\\
    \textsc{detr-}unwrap  put-\textsc{irr-spec-dep}\\
\glt `[It] might unwrap.’\footnote{The pronunciation [nundate] – with initial [n] – may represent influence from the \ili{Maruat-Dimiri-Yaul} \isi{dialect}.} [ulwa014\_57:24]
\z

\ea%72
    \label{ex:verbs:72}
          \textit{Mï amun wa mbi an nït mol} \textbf{\textit{inat}}.\\
\gll mï      amun  wa    mbï-i      na=n    nï=ta ma=ul      i-na-\textbf{t}\\
    3\textsc{sg.subj}  now   village  here-go.\textsc{pfv}  talk\textsc{=obl}  \textsc{1sg=}say    3\textsc{sg.obj}=with  come-\textsc{irr-spec}\\


\glt `He recently came here to the village and told me that he might come with her.’ [ulwa037\_46:28]
\z

Examples \REF{ex:verbs:71} and \REF{ex:verbs:72} both illustrate the sense of ‘might’ being conveyed by the \isi{speculative} \isi{suffix} \textit{-t} ‘\textsc{spec}’, in both instances directly following an \isi{irrealis} \isi{suffix}. Example \REF{ex:verbs:72} further illustrates this use in \isi{indirect discourse}. This must not, however, be taken to have \is{evidentiality} evidential force -- that is, the \isi{suffix} is used because the \isi{reported speech} is of someone who himself is speculating about whether or not he would come (not because the person reporting this information can only speculate as to whether or not the person would come). Indeed, the speaker who uttered this sentence followed it with the one given in \REF{ex:verbs:73}, which does not include any \isi{speculative} form.\footnote{For more on \isi{indirect discourse}, see \sectref{sec:13.4.5}.}

\ea%73
    \label{ex:verbs:73}
          \textit{Inim ngol mol ina nït.}\\
\gll    inim  nga=ul     ma=ul       i-na    nï=ta\\
    water  \textsc{sg.prox}=with  3\textsc{sg.obj}=with   come-\textsc{irr}  1\textsc{sg}=say\\
\glt `[He] told me [he] would come with her this year.’ [ulwa037\_46:32]
\z

  The \isi{speculative} \isi{suffix} can also be used in conjunction with the \isi{epistemic adverb} \textit{tap} ‘maybe’, which always comes earlier in the clause, as in \REF{ex:verbs:74}.

\ea%74
    \label{ex:verbs:74}
\textit{Ngunanji yalum anda \textbf{tap} i wa \textbf{mbïpïnate}}.\\
\gll ngunan-nji      yalum    anda     \textbf{tap}   i     wa mbï-p-na-\textbf{t}-e\\
1\textsc{du.incl-poss}  grandchild  \textsc{sg.dist}  maybe  go.\textsc{pfv}  village 
  here-be-\textsc{irr-spec-dep}\\
\glt `Our granddaughter might come and stay here in the village.’ [ulwa014\_11:10]
\z

Sentence \REF{ex:verbs:74} also illustrates that this \isi{suffix} may be followed by the \isi{dependent marker} \textit{-e} ‘\textsc{dep}’. More examples of the \isi{speculative} \isi{suffix} are given in \REF{ex:verbs:75} and \REF{ex:verbs:76}.

\ea%75
    \label{ex:verbs:75}
          \textbf{\textit{Makïnate}}.\\
\gll ma=kï-na-\textbf{t}{}-e\\
    3\textsc{sg.obj}=say-\textsc{irr-spec-dep}\\
\glt `[He] might tell him.’ [ulwa032\_51:01]
\z

\ea%76
    \label{ex:verbs:76}
          \textit{Una iken malka} \textbf{\textit{manate}}.\\
\gll unan    iken  ma=lï-ka     ma-na-\textbf{t}{}-e\\
    1\textsc{pl.incl}  may  3\textsc{sg.obj}=put-let  go-\textsc{irr-spec-dep}\\
\glt `We may go follow it [= an accusation].’ (\textit{iken} < TP \textit{i ken} ‘may, can’) [ulwa037\_08:44]
\z

In sentence \REF{ex:verbs:76}, the force of the \isi{suffix} \textit{-t} ‘\textsc{spec}’ is closer to \ili{English} ‘may’. Indeed, it is used alongside a \ili{Tok Pisin} \isi{loanword} \textit{i ken} ‘may’, often used in granting permission. In sentence \REF{ex:verbs:77}, \textit{-t} ‘\textsc{spec}’ is used along with a different \ili{Tok Pisin} \isi{loanword}, here giving the sentence a \isi{speculative} sense, although one with a \isi{negative} flavor. The \ili{Tok Pisin} word used here, \textit{nogut} ‘bad’, often carries a conjunctive meaning similar to that of \ili{English} ‘lest’.

\ea%77
    \label{ex:verbs:77}
          \textbf{\textit{Nongut}} \textit{mundu tï} \textbf{\textit{unanandat}}.\\
\gll \textbf{nongut}  mundu  tï    unan=na-nda-\textbf{t}\\
    lest     food   take  1\textsc{pl.incl}=give-\textsc{irr-spec}\\
\glt `[It] might give us food.’ (\textit{nongut} < TP \textit{nogut} ‘bad; lest’) [ulwa029\_05:25]
\z

The \isi{speculative} \isi{suffix} is indeed often used in \isi{negative} \isi{irrealis} clauses, typically following the \isi{negative} marker \textit{ango} ‘\textsc{neg}’, which tends to come early in the clause, as in sentences \REF{ex:verbs:78} through \REF{ex:verbs:85}.

\ea%78
    \label{ex:verbs:78}
          \textbf{\textit{Ango}} \textit{maka anma apombam} \textbf{\textit{manate}}.\\
\gll \textbf{ango}  maka  an-ma  apombam      ma-na-\textbf{t}{}-e\\
    \textsc{neg}  thus  out-go  middle.of.house  go-\textsc{irr-spec-dep}\\
\glt `[She] shouldn’t go into the middle of the house, going out like that.’ [ulwa014\_35:26]
\z

\ea%79
    \label{ex:verbs:79}
          \textbf{\textit{Ango}} \textit{apa kwan lusim} \textbf{\textit{manat}}.\\
\gll \textbf{ango}  apa    kwa=n    lusim  ma-na-\textbf{t}\\
    \textsc{neg}  house  one=\textsc{obl}  leave  go-\textsc{irr-spec}\\
\glt `[They] were not going to leave out a single household.’ (\textit{lusim} = TP) [ulwa029\_02:26]
\z

\ea%80
    \label{ex:verbs:80}
          \textit{Una} \textbf{\textit{ango}} \textit{luwa} \textbf{\textit{lundat}}.\\
\gll unan    \textbf{ango}  luwa  lo{}-nda-\textbf{t}\\
    1\textsc{pl.incl}  \textsc{neg}  place  go-\textsc{irr-spec}\\
\glt `We can’t go anywhere.’ [ulwa029\_03:17]
\z

\ea%81
    \label{ex:verbs:81}
          \textit{Nï} \textbf{\textit{ango}} \textit{mbuka wiya} \textbf{\textit{inat}}.\\
\gll nï    \textbf{ango}  mbï-u-ka    u=iya       i-na-\textbf{t}\\
    1\textsc{sg}  \textsc{neg}  here-from-let  2\textsc{sg}=toward  come-\textsc{irr-spec}\\
\glt `I will not come to you quickly.’ [ulwa031\_00:46]
\z

\ea%82
    \label{ex:verbs:82}
          \textbf{\textit{Ango}} \textbf{\textit{nokoplïndat}}.\\
\gll \textbf{ango}  nokop-lï-nda-\textbf{t}\\
    \textsc{neg}  hide-put-\textsc{irr-spec}\\
\glt `[It] couldn’t hide [from us].’ [ulwa037\_14:36]
\z

\ea%83
    \label{ex:verbs:83}
          \textit{Mï \textbf{ango} un apa \textbf{pïnat}}.\\
\gll mï       \textbf{ango}  un=n    apa    p-na-\textbf{t}\\
    3\textsc{sg.subj}  \textsc{neg}  2\textsc{pl=obl}  house  be\textsc{{}-irr-spec}\\
\glt `It won’t last [long] in your house.’ [ulwa037\_52:13]
\z

\ea%84
    \label{ex:verbs:84}
          \textbf{\textit{Ango}} \textit{in} \textbf{\textit{malandate}} \textit{unji ametamal.}\\
\gll    \textbf{ango}  i=n      ma=la-nda-\textbf{t}{}-e          u-nji    ametamal\\
    \textsc{neg}  hand=\textsc{obl}  \textsc{3sg.obj}=eat-\textsc{irr-spec-dep}  2\textsc{sg-poss}  spoon\\
\glt `[You] may not eat it with [your] hand, but [must use] your spoon.’ [ulwa014\_36:27]
\z

\newpage

\ea%85
    \label{ex:verbs:85}
          \textit{Nï} \textbf{\textit{ango}} \textbf{\textit{manat}} \textit{nï mbï napïna.}\\
\gll    nï    \textbf{ango}  ma-na-\textbf{t}    nï    mbï  na-p-na\\
    1\textsc{sg}  \textsc{neg}  go-\textsc{irr-spec}  1\textsc{sg}  here  \textsc{detr-}be-\textsc{irr}\\
\glt `I won’t go; I’ll stay here.’ [ulwa037\_35:57]
\z

The two clauses in \REF{ex:verbs:85} illustrate the contrast between \isi{negative} \isi{irrealis} clauses (here marked with \isi{speculative} \textit{-t} ‘\textsc{spec}’) and \isi{positive} \isi{irrealis} clauses (here, as usual when not needed for extra \isi{speculative} force, not marked with \textit{-t} ‘\textsc{spec}’). In sentence \REF{ex:verbs:86} there is also a contrast between a \isi{negative} \isi{irrealis} clause and a \isi{positive} one. Here, the \isi{negator} \textit{ango} ‘\textsc{neg}’ is missing; the \isi{speculative} \isi{suffix} \textit{-t} ‘\textsc{spec}’ alone is conveying the \isi{negative} force of the first clause.

\ea%86
    \label{ex:verbs:86}

          \textit{Apombam} \textbf{\textit{manate}} \textit{angani wat ando li mana.}

\gll    apombam      ma-na-\textbf{t}{}-e       angani  wat    anda=u    li    ma-na\\
    middle.of.house  go-\textsc{irr-spec-dep}  rear   ladder  \textsc{sg.dist}=from    down   go-\textsc{irr}\\

 \glt ‘[She won’t] go to the middle of the house, but will go down the back ladder.’ [ulwa014\_34:48]
\z

This \isi{suffix} is also commonly used in \isi{negative command}s, which may use the prohibitive marker \textit{wana {\textasciitilde}} \textit{wanap} ‘\textsc{proh}’ (see \sectref{sec:13.2.4} and \sectref{sec:13.3.2} for examples). The \isi{negative} \isi{polarity} function of the \isi{speculative} marker \textit{-t} ‘\textsc{spec}’ may reflect an origin as a postverbal \isi{negator} (cf. \ili{Ap Ma} \textit{-at} ‘\textsc{neg}’). But this historical explanation is itself \isi{speculative}. See \sectref{sec:13.3.2} for the synchronically more common postverbal \isi{negator}s in Ulwa.

  \isi{Question}s, which often contain words derived from the \isi{negative} marker \textit{ango} ‘\textsc{neg}’ (\sectref{sec:13.3}), may also carry \isi{speculative} force, employing the \isi{suffix} \textit{-t} ‘\textsc{spec}’ \REF{ex:verbs:87}.

\ea%87
    \label{ex:verbs:87}
          \textit{A un \textbf{angos tïnat}?!}\\
\gll    a  un  \textbf{angos}  tï-na-\textbf{t}\\
    ah  2\textsc{pl}  what  take-\textsc{irr-spec}\\
\glt `Ah, what will you get?!’ [ulwa018\_06:11]
\z

\is{verb|)}
\is{speculative|)}

This \isi{suffix} can also be used as a device for indicating \isi{politeness}, suggesting more tentativeness in the \isi{question} being asked \REF{ex:verbs:88}.

\ea%88
    \label{ex:verbs:88}
          \textit{Unan \textbf{angos natanate}?}\\
\gll unan     \textbf{angos}  na-ta-na-\textbf{t}{}-e\\
    1\textsc{pl.incl}  what  \textsc{detr-}say-\textsc{irr-spec-dep}\\
\glt `What could we talk about?’ [ulwa037\_58:33]
\z

\section{The conditional suffix \textit{-ta} ‘\textsc{cond}’}\label{sec:4.12}

\is{conditional|(}
\is{verb|(}

Perhaps etymologically related to the \isi{speculative} \isi{suffix} \textit{-t} ‘\textsc{spec}’ (\sectref{sec:4.11}), the \isi{conditional} \isi{suffix} \textit{-ta} ‘\textsc{cond}’ is used to mark the verb in the \isi{protasis} of a \isi{conditional} statement. The \isi{syntax} of such sentences is addressed in \sectref{sec:13.5}; the \isi{morphology} and basic uses of this \isi{suffix} are addressed in this section.

  In the \is{protasis} protases of \isi{conditional} statements, the \isi{suffix} \textit{-ta} ‘\textsc{cond}’ is affixed either to the \isi{stem} of the verb (always including the \isi{vowel}) \REF{ex:verbs:89} or to the \isi{perfective} form (that is, including the \isi{perfective} \isi{suffix}) \REF{ex:verbs:90}.\footnote{Often the two forms appear to be interchangeable. I suspect that, for many speakers, an \isi{aspect}ual or \isi{modal} distinction that perhaps once existed is now being lost. Still, at least in some circumstances, it may be the case that the \isi{perfective} version of the \isi{conditional} verb is required to show a sequence of events.}

  \ea%89
    \label{ex:verbs:89}
          \textit{Inim} \textbf{\textit{lopota}} \textit{nï mana.}\\
\gll    inim      lopo-\textbf{ta}      nï    ma-na\\
    water    rain-\textsc{cond}  \textsc{1sg}  go-\textsc{irr}\\
\glt `If it rains, I’ll leave.’ [elicited]
\z

  \ea%90
    \label{ex:verbs:90}
          \textit{Inim} \textbf{\textit{lopopta}} \textit{nï mana.}\\
\gll    inim      lopo-p-\textbf{ta}      nï    ma-na\\
    water    rain-\textsc{pfv-cond}  \textsc{1sg}  go-\textsc{irr}\\
\glt `If it rains, I’ll leave.’ [elicited]
\z

  In verbs that exhibit different \isi{stem}s for the \isi{irrealis} \isi{mood}, the \isi{conditional} form is never built from the \isi{imperfective}/\isi{perfective} \isi{stem} alone. Either it is built from this \isi{realis} \isi{stem} plus the \isi{perfective} marker or it is built from the \isi{suppletive} \isi{irrealis} \isi{stem} without any additional marker (cf. the discussion of \isi{imperative}s in \sectref{sec:4.7}). Thus, the verb \textit{ama-} ‘eat’ has as its \isi{conditional} form either [amapta] (from the fully \isi{inflect}ed \isi{perfective} form) \REF{ex:verbs:90a} or [lata] (from the \isi{irrealis} \isi{stem}) \REF{ex:verbs:90b}, but never \textsuperscript{†}[amata] (from the \isi{imperfective}/\isi{perfective} \isi{stem} alone).
  
\ea%90a
    \label{ex:verbs:90a}
          \textit{U nïnji mundu} \textbf{\textit{amapta}} \textit{nï uwalinda.}\\
\gll    u  nï-nji  mundu  ama-p-\textbf{ta}      nï  u=wali-nda\\
    2\textsc{sg}  1\textsc{sg-poss} food eat-\textsc{pfv-cond}  1\textsc{sg}  2\textsc{sg}=hit-\textsc{irr}\\
\glt `If you eat my food, I’ll hit you.’ [elicited]
\z

\ea%90b
    \label{ex:verbs:90b}
          \textit{U nïnji mundu} \textbf{\textit{lata}} \textit{nï uwalinda.}\\
\gll    u  nï-nji  mundu  la-\textbf{ta}      nï  u=wali-nda\\
    2\textsc{sg}  1\textsc{sg-poss} food eat-\textsc{cond}  1\textsc{sg}  2\textsc{sg}=hit-\textsc{irr}\\
\glt `If you eat my food, I’ll hit you.’ [elicited]
\z

I have not found any discernible difference in meaning between the alternative \isi{verb stem}s in cases such as these.

Example \REF{ex:verbs:91} illustrates the \isi{conditional} \isi{suffix} following the \isi{suppletive} \isi{perfective} \isi{stem} \textit{i} ‘go.\textsc{pfv}’.

\ea%91
    \label{ex:verbs:91}
          \textit{Mï} \textbf{\textit{ita}} \textit{nï nan makïna.}\\
\gll    mï       i-\textbf{ta}       nï     na=n     ma=kï-na\\
    3\textsc{sg.subj}  go.\textsc{pfv-cond}  1\textsc{sg}  talk=\textsc{obl}  3\textsc{sg.obj}=say-\textsc{irr}\\
\glt `If he comes, I’ll tell him.’ [ulwa014\_09:39]
\z

The \isi{conditional} \isi{suffix} can also appear following the \isi{locative verb} \textit{p-} ‘be at’, as in \REF{ex:verbs:92} and \REF{ex:verbs:93}.

\ea%92
    \label{ex:verbs:92}
          \textit{Kuma lawa} \textbf{\textit{mapta}} \textit{landa.}\\
\gll    kuma  ala{}-awa    ma=\textbf{p-ta}       la-nda\\
    some  \textsc{pl.dist-int}  3\textsc{sg.obj}=be-\textsc{cond}  eat-\textsc{irr}\\
\glt `When some other people are there, [they] might eat [our food].’ [ulwa033\_00:18]
\z

\ea%93
    \label{ex:verbs:93}
          \textit{Ambi \textbf{napta} we mï lïmndï anala.}\\
\gll    ambi  na-\textbf{p-ta}      we    mï       lïmndï  an=ala\\
    big    \textsc{detr-}be\textsc{{}-cond}  then  3\textsc{sg.subj}  eye    \textsc{1pl.excl}=see\\
\glt `Once [we] had gotten big, then she saw us.’ [ulwa013\_00:47]
\z

Especially when needed to break up impossible \isi{consonant cluster}s, the \isi{locative verb} \textit{p-} ‘be at’ may be realized \isi{phonetic}ally as [pï] when preceding the \isi{conditional} \isi{suffix} \textit{-ta} ‘\textsc{cond}’, as in \REF{ex:verbs:94} and \REF{ex:verbs:95}.

\ea%94
    \label{ex:verbs:94}
          \textit{Akum \textbf{pïta} akumnï ndutana.}\\
\gll    akum    \textbf{p-ta}     akum=nï    ndï=uta-na\\
    basket  be\textsc{{}-cond} basket=\textsc{obl}  \textsc{3pl}=grind-\textsc{irr}\\
\glt `If there is basket, [they] scoop them with the basket.’ [ulwa036\_01:43]
\z

\ea%95
    \label{ex:verbs:95}

          \textit{Mawapta mï} \textbf{\textit{anmapïta}} \textit{we ande ndï wolka mol nena.}

\gll    ma=wap-ta         mï      anma=\textbf{p-ta}      we     ande    ndï  wolka  ma=ul      na-i-na\\
    \textsc{3sg.obj}=be.\textsc{pst-cond}  \textsc{3sg.subj}  good=\textsc{cop-cond}  then  ok    \textsc{3pl}  again  3\textsc{sg.obj=}with  \textsc{detr}{}-come-\textsc{irr}\\


\glt `If [the sick person] has stayed there and has gotten well, then OK, they would come back with him.’ [ulwa029\_10:14]
\z

Sentence \REF{ex:verbs:95} also illustrates the use of the \isi{conditional} \isi{suffix} \textit{-ta} ‘\textsc{cond}’ following the \isi{suppletive} \isi{locative} form \textit{wap} ‘be.\textsc{pst}’. More examples of \isi{conditional sentence}s are provided in \sectref{sec:13.5}.

\is{verb|)}
\is{conditional|)}


\section{Derivational morphology: Verbalization}\label{sec:4.13}

\is{derivation|(}
\is{derivational morphology|(}
\is{verbalization|(}
\is{verb|(}

Unlike nouns, which may be formed from other \isi{parts of speech} through the addition of the derivational \isi{suffix} \textit{-en} ‘\textsc{nmlz}’ (\sectref{sec:3.2}), there is no single means of deriving verbs from other \isi{parts of speech}. That said, it is possible for \is{non-verbal predication} non-verbal words to serve as \isi{predicate}s. This is accomplished through the use of the \isi{copular clitic} \textit{=p} ‘\textsc{cop}’ (\sectref{sec:10.2}).

\is{verb|)}
\is{verbalization|)}
\is{derivational morphology|)}
\is{derivation|)}

\section{Compound verbs}\label{sec:4.14}

\is{compound verb|(}
\is{compound|(}
\is{verb|(}

While most verbs are composed of simply a single free root (plus, potentially, \is{bound morpheme} bound \isi{morphology} such as \isi{TAM} \isi{suffix}es, the \isi{detransitivizing} \isi{prefix} \textit{na-} ‘\textsc{detr}’, or \isi{object-marker} \isi{proclitic}s), some can be analyzed as \isi{compound}s -- that is, consisting of more than one free morpheme. The final element of such \isi{compound}s (excluding any \isi{suffix}es) is always a \isi{verb stem}. The first element, on the other hand, may be either a noun \REF{ex:verbs:96} or another verb \REF{ex:verbs:97} (or perhaps a \isi{postposition}, although this may be analyzed otherwise).\footnote{Often it is not clear whether such a combination of non-bound morphemes should best be analyzed as a \isi{compound}. A noun preceding a verb, for example, could simply be the object of the verb. Only when this noun-plus-verb combination permits a \isi{direct object} (or an \isi{object marker}) can it be said to be a \isi{compound}. Similarly, two \isi{verb stem}s in succession may be separate words that are \isi{coordinate}d \isi{paratactic}ally. A true \isi{compound} verb consisting of two verbal elements, however, should permit only one \isi{object marker} (that is, the \isi{object marker} should affix to the beginning of the first member of the \isi{compound}). Often, series of \isi{postposition}-plus-verb-stem seem very much like \isi{compound} verbs, especially when considering their \isi{phonological} tendency to coalesce and reduce. However, there are few if any \isi{morphosyntactic} tests to prove that such forms are true compounds.}

\ea%96
    \label{ex:verbs:96}

          \textit{Ndïn ndiya i iwan} \textbf{\textit{ndïnambïlumope}}.\\
\gll ndï=n    ndï=iya     i     iwa=n ndï=\textbf{nambï-lumo}{}-p-e\\
    3\textsc{pl=obl}  3\textsc{pl}=toward  go.\textsc{pfv}  basket=\textsc{obl}    3\textsc{pl}=skin-put-\textsc{pfv-dep}\\


\glt `[Men] would go to them [= women] with them [= bamboo stalks], blocking them [= fish] with fish trap baskets.’ [ulwa036\_02:11]
\z

\newpage

\ea%97
    \label{ex:verbs:97}

          \textit{Ul wandam ma i tï u kïkal} \textbf{\textit{welunda}}.\\
\gll u=ul     wandam  ma  i     tï     u    kïkal \textbf{we-lo}{}-nda\\
    2\textsc{sg}=with  jungle     go  hand  take  2\textsc{sg}  ear    cut-cut-\textsc{irr}\\
\glt `[They] will go to the jungle with you and box your ears with [their] hands.’ [ulwa014\_00:44]
\z

  Compound verbs may contain nouns as their first element. Evidence that these forms are single (though polymorphemic) \isi{lexical} items comes from the fact that they permit \isi{object marker}s (preceding the entire word) and \isi{TAM} \isi{suffix}es (following the entire word). Examples of noun-plus-verb verbal compounds are given in \REF{ex:verbs:98}, \REF{ex:verbs:99}, and \REF{ex:verbs:100}.


\ea%98
    \label{ex:verbs:98}

          \textbf{\textit{Manambuweyup}} \textit{kuma ndinap ndïtï wa i.}

\gll    ma=\textbf{nambï-we-u}{}-p      kuma  ndï=ina-p    ndï=tï    wa    i\\
    3\textsc{sg.obj}=skin-cut-put-\textsc{pfv}  some  3\textsc{pl}=get-\textsc{pfv}  3\textsc{pl}=take    village  go.\textsc{pfv}\\


\glt `[She] peeled it, got some [greens], and brought them home.’ [ulwa001\_01:06]
\z

\ea%99
    \label{ex:verbs:99}
          \textit{An kaw} \textbf{\textit{mawutïnip}}.\\
\gll an      kaw  ma=\textbf{wutï-ni}{}-p\\
    1\textsc{pl.excl}  song  3\textsc{sg.obj}=leg-beat-\textsc{pfv}\\
\glt `We danced the song.’ [elicited]
\z

\ea%100
    \label{ex:verbs:100}
          \textit{Nï tïn} \textbf{\textit{manambïtwana}}.\\
\gll nï    tïn    ma=\textbf{nambït-wana}\\
    1\textsc{sg}  dog  3\textsc{sg.obj}=odor-feel\\
\glt `I smelled the dog.’ [elicited]
\z

Note that in example \REF{ex:verbs:98} the verb is analyzed as containing not only a nominal element, but also two verbal elements. Similar constructions with ‘put’ verbs are discussed in \sectref{sec:9.2.2}. Additional examples of \isi{compound} verbs consisting of two verb roots are given in \REF{ex:verbs:101} and \REF{ex:verbs:102}. Only the final element receives \isi{TAM} marking.

\ea%101
    \label{ex:verbs:101}
          \textit{Ndï angos tïna nakap} \textbf{\textit{anwanakap}}.\\
\gll ndï  angos  tï-na    na-kï{}-p      an=\textbf{wana-kï}{}-p\\
    3\textsc{pl}  what  take{}-\textsc{irr}  \textsc{detr}{}-say-\textsc{pfv}  1\textsc{pl.excl}=feel-say-\textsc{pfv}\\
\glt `[When] they wanted to get something, [they] called us.’ [ulwa018\_05:01]
\z

\ea%102
    \label{ex:verbs:102}
          \textbf{\textit{Amblawanawne}} \textit{nay.}\\
\gll    ambla=\textbf{wana-uni}{}-e    na-i\\
    \textsc{pl.refl}=feel-shout\textsc{{}-dep}  \textsc{detr}{}-go.\textsc{pfv}\\
\glt `Calling to each other, they went.’ [ulwa001\_07:49]
\z

Compound verbs can be even more complex, containing whole \isi{postpositional phrase}s (that is, units composed of noun-plus-\isi{postposition}). Examples \REF{ex:verbs:103} and \REF{ex:verbs:104} contain the relatively prototypical noun \textit{ina} ‘liver’; example \REF{ex:verbs:105}, on the other hand, includes as the object of the \isi{postposition} a \isi{semantic}ally verb-like noun, \textit{top} ‘throw’.

\ea%103
    \label{ex:verbs:103}
          \textit{Nï kenmbu} \textbf{\textit{maynakawana}}.\\
\gll nï    kenmbu  ma=\textbf{ina-ka-wana}\\
    1\textsc{sg}  problem  \textsc{3sg.obj}=liver-at-feel\\
\glt `I thought about the problem.’ [elicited]
\z

\ea%104
    \label{ex:verbs:104}
          \textit{Nungolke ngala ango} \textbf{\textit{ndinakawana}}.\\
\gll nungolke  ngala    ango  ndï=\textbf{ina-ka-wana}\\
    child    \textsc{pl.prox}  \textsc{neg}  3\textsc{pl}=liver-at-feel\\
\glt `But these children aren’t thinking about them.’ [ulwa038\_01:32]
\z

\ea%105
    \label{ex:verbs:105}
          \textit{Nï} \textbf{\textit{natopinka}}.\\
\gll nï    na-\textbf{top-in-ka}\\
    1\textsc{sg}  \textsc{detr}{}-throw-in-let\\
\glt `I’ve forgotten.’ [ulwa037\_17:58]
\z

The verb \textit{ka-} ‘let, leave, allow’ is presented more fully in the discussion on \isi{separable verb}s (\sectref{sec:9.2.3}). Also see \sectref{sec:9.2.1} for an overview of \isi{compound} verbs that may be composed of \isi{discontinuous} elements.

  Some compounds may be composed of just \isi{postposition}s and verbs. While there often seems to be a close connection (both \isi{semantic} and \isi{phonological}) between these two elements, it is difficult to prove that they indeed form compounds. Although in sentences such as \REF{ex:verbs:106}, they are glossed as \isi{transitive} \isi{compound} verbs with \isi{direct object}s, they could alternatively be analyzed as series of \isi{postpositional phrase}s (which contain objects of the \isi{postposition}) and \isi{intransitive} verbs (having no objects of their own).

\ea%106
    \label{ex:verbs:106}
          \textit{Atana mï ko} \textbf{\textit{malakam}}.\\
\gll atana    mï      ko    ma=\textbf{ala-kamb}\\
    older.sister  3\textsc{sg.subj}  just    3\textsc{sg.obj}=from-shun\\
\glt `The older sister disapproved of it.’ [ulwa011\_02:25]
\z

The verb \textit{u-} ‘put’ usually takes a \isi{goal} argument as its \isi{direct object} (i.e., it does not have the same \isi{semantics} or argument structure as \ili{English} \textit{put}, \sectref{sec:9.2.2}). In \REF{ex:verbs:107}, the \isi{goal} argument directly precedes the possible \isi{compound} \textit{inu-} ‘put in’.

\ea%107
    \label{ex:verbs:107}
          \textit{Ndïn unji uta} \textbf{\textit{menup}}.\\
\gll ndï=n    u-nji    uta    ma=\textbf{in-u}{}-p\\
    \textsc{3pl=obl}  2\textsc{sg-poss}  shell  3\textsc{sg.obj}=in-put-\textsc{pfv}\\
\glt `[They] put them in your dish.’ [ulwa014\_39:19]
\z

The form \textit{watlo-} ‘clear (as land of rubbish, foliage, etc.)’, composed of the \isi{postposition} \textit{wat} ‘atop’ and a form of the verb \textit{lo-} ‘cut’, is also possibly a \isi{compound}. Examples \REF{ex:verbs:108} and \REF{ex:verbs:109} illustrate \textit{watlo-} ‘clear’ as it appears to function as a \isi{compound} verb.

\ea%108
    \label{ex:verbs:108}
          \textit{Ndï amun} \textbf{\textit{nduwatlope}}.\\
\gll ndï  amun  ndï=\textbf{wat-lo}{}-p-e\\
    3\textsc{pl}  now  3\textsc{pl}=atop-cut-\textsc{pfv-dep}\\
\glt `They have just now cleared them.’ [ulwa014\_06:43]
\z

\ea%109
    \label{ex:verbs:109}
          \textit{Nï mape} \textbf{\textit{ndïwatle}}.\\
\gll nï    ma=p-e      ndï=\textbf{wat-lo}{}-e\\
    1\textsc{sg}  3\textsc{sg.obj}=be\textsc{{}-dep} 3\textsc{pl}=atop-cut-\textsc{ipfv}\\
\glt `I cleared them there.’ [ulwa037\_42:27]
\z

For the form and function of \isi{compound} verbs containing \isi{locative adverb}s, such as \textit{mbï} ‘here’, see \sectref{sec:8.2.2}.

\is{verb|)}
\is{compound|)}
\is{compound verb|)}