\chapter{Ulwa: The language context}\label{sec:1}

This a description of Ulwa (ISO 639-3 code [yla], glottocode [yaul1241]), a language of Papua New Guinea. Ulwa is severely \isi{endangered}. It is spoken fluently by fewer than 600 people, almost all of whom live in one of four villages in the East Sepik Province of Papua New Guinea. Ulwa is a member of the \ili{Keram} branch of the \ili{Keram-Ramu} family. It may thus be considered a \isi{Papuan} language (or a non-\ili{Austronesian} language). It was first described in \citet{Barlow2018a}, which serves as the basis for this book. This introductory chapter provides background information on the Ulwa language and its speakers.

\section{The environment}\label{sec:1.5.2}

The \isi{Sepik} region of Papua New Guinea, where Ulwa is spoken, is known for its long, serpentine river and dense, tropical rainforest. All four Ulwa villages are somewhat removed from the \isi{Sepik River} itself, instead being positioned on considerably smaller tributaries, where they are confronted by less boat traffic; however, small canoes can and do ply their waters. The \isi{Keram River} tributary that passes along Manu village is a source of fish, turtles, crayfish, and other seafood, as well as a place to bathe and a source of drinking water during the dry season, when rainwater cannot be collected. The villages of Maruat, Dimiri, and Yaul, however, face harsher conditions, since the \isi{Yuat River} tributary that passes near their villages becomes completely desiccated during the dry season. During the rainy season, the entire area becomes swampy and is besieged with mosquitoes.

\section{The four villages}\label{sec:1.5.3}

Ulwa is spoken in four villages located in Angoram District, East Sepik Province, Papua New Guinea. The map in \figref{fig:1.1} shows the island of \isi{New Guinea}, with the \isi{Sepik} region indicated by a red rectangle. The map in \figref{fig:1.2} (corresponding to the red rectangle in \figref{fig:1.1}) shows the general location of the Ulwa-speaking area, indicated by a red circle. The geographic coordinates of the four Ulwa villages are given in \tabref{tab:1.1a}.

\begin{figure}
\is{Sepik River}
\is{New Guinea}
\caption{The Sepik region (red rectangle) on the island of New Guinea. Contains map data © OpenStreetMap contributors, made available under the terms of the Open Database License (ODbL)}
\label{fig:1.1}
\includegraphics[width=\textwidth]{figures/Ulwa2022September01-img001.png}
\end{figure}

\begin{figure}
\is{Sepik River}
\is{Keram River}
\is{Ramu River}
\is{Yuat River}
\is{Chambri Lake}
\caption{The Ulwa language area (red circle) in the Sepik region. Contains map data © OpenStreetMap contributors, made available under the terms of the Open Database License (ODbL)}
\label{fig:1.2}
\includegraphics[width=\textwidth]{figures/Ulwa2022September01-img002.png}
\end{figure}

\begin{table}
\caption{Geographic coordinates of the four Ulwa villages}
\label{tab:1.1a}


\begin{tabular}{lll}
\lsptoprule
village & sexagesimal degrees & decimal degrees\\
\midrule
Manu & 4°29’00”S, 144°00’55”E & -4.483, 144.015\\
Maruat & 4°25’20”S, 143°54’40”E & -4.422, 143.911\\
Dimiri & 4°24’55”S, 143°54’30”E & -4.415, 143.908\\
Yaul & 4°24’55”S, 143°56’10”E & -4.415, 143.936\\
\lspbottomrule
\end{tabular}
\end{table}

Manu village is located in the Keram Rural Local-Level Government area (LLG). The village sits along the \isi{Keram Black} tributary to the \isi{Keram River}, which is itself a tributary to the \isi{Sepik}. There are no other villages upstream of Manu on the \isi{Keram Black River}. The inhabitants of villages downstream of Manu (including its nearest neighbor Yamen) speak \ili{Ap Ma} as their traditional language. On the main \isi{Keram River} are found villages whose inhabitants traditionally speak \ili{Kanda} (also known as \ili{Angoram}), a member of the \ili{Lower Sepik} family. The \isi{Keram River} leads to the \isi{Sepik River}, meeting this larger river around \linebreak Angoram, the nearest town connected by road (to Wewak). The Ulwa name for the village of Manu is \textit{Nïmalnu} (perhaps derived from \textit{nïmal} ‘river’ plus \textit{nu} ‘near’). The region immediately surrounding Manu village is known to Ulwa speakers as \textit{Bulon}.

The other three Ulwa-speaking villages (Maruat, Dimiri, and Yaul) form a small triangle in the Yuat Rural LLG, west of Manu village. These three villages are closer to the \isi{Yuat River}, another tributary of the \isi{Sepik River}, than they are to the \isi{Keram River}. The \isi{Yuat River} lies west of the villages and is at times accessible from them by creeks. The nearest neighbors on this river speak \ili{Mundukumo} (downstream) and \ili{Bun} (upstream), two closely related members of the \ili{Yuat} family. In Ulwa, Maruat village is known as \textit{Mamala}, Dimiri village is known as \textit{Andïmali}, and Yaul village is known as \textit{Mosombla}.

The villages of Maruat, Dimiri, and Yaul are each within an hour’s walk of the other two, and they share an elementary school located roughly in the middle of the three. Manu is considerably farther from the other three villages. In the dry season (roughly June to November), it is at least a four-hour hike away from any of them; in the wet season (roughly December to May), however, when the jungle paths are mired in swampy water, the journey is much less tractable.

The closest neighbors to Manu, the residents of Yamen village (as well as those of all other villages downstream from Manu), speak \ili{Ap Ma}. This languages is also a member of the \ili{Keram} family, but it is \isi{lexical}ly very different from Ulwa. \mbox{\ili{Ap Ma}} is a considerably larger language, with perhaps as many as 10,000 speakers \mbox{\citep{EberhardEtAl2023}.}\footnote{They cite “2010 PBT [Pioneer Bible Translators]”.} Yamen village is only about 3.5 km (2.2 miles) away, and many Yamen children walk to Manu each day to attend the Manu village elementary school. To the north of Manu there are other \ili{Ap Ma} villages, which are accessible by foot. To the southeast of Manu (and not very accessible) there are villages that speak \ili{Waran} (also known as \ili{Banaro}), which, as a member of the \ili{Ramu} branch of the \ili{Keram-Ramu} family, is very distantly related to Ulwa. \ili{Waran} has perhaps as many as 4,000 speakers \citep{EberhardEtAl2023}.\footnote{They cite “2019 PBT [Pioneer Bible Translators]”.} Farther to the west there are speakers of \ili{Bun}, \ili{Mekmek}, and \ili{Kyenele} (also known as \ili{Miyak}), all of which are members of the \ili{Yuat} family. The other three Ulwa-speaking villages are about 13 km (8 miles) to the northwest of Manu.

The closest neighbors to the three villages of Maruat, Dimiri, and Yaul speak the \ili{Yuat} language \ili{Mundukumo}, which is another relatively large language, of perhaps some 3,040 speakers \citep{EberhardEtAl2023}.\footnote{They cite “2003 SIL”.} Maruat, Dimiri, and Yaul are also relatively close to \ili{Ap Ma}-speaking villages.

The relative locations of the four Ulwa-speaking villages and their neighbors are depicted in the map in \figref{fig:1.3}. In this map, the Ulwa-speaking villages are marked in red, \ili{Ap Ma}-speaking villages are marked in yellow, \ili{Mundukumo}-speaking villages are marked in light green, and the eponymous \ili{Bun}-speaking village is marked in dark green.

\begin{figure}
\is{Yuat River}
\is{Keram Black River}
\caption{The four Ulwa villages and their neighbors}
\label{fig:1.3}
\includegraphics[width=\textwidth]{figures/Ulwa2022September01-img003.jpg}
\end{figure}

\section{The people}\label{sec:1.5.4}

The subsistence pattern of the Ulwa people is a combination of hunting, gathering, fishing, horticulture, and husbandry.

The primary staple carbohydrate is sago, a starch that must be arduously extracted from certain palm species and then prepared either as a jelly (\textit{ay} in Ulwa) or as a chewy pancake (\textit{we} in Ulwa). Traditionally, this entire process was the work of women alone, though men nowadays often help in extracting the pulp -- that is, felling the sago palm, stripping the bark, and hacking the wood into the splinters through which water may subsequently be passed to collect a starchy water to process the sago flour. While men assist in the felling of the sago palms and beating of the sago pulp, it is still generally considered the work of women to press the pulp to extract the starchy water, to carry the starch back to the village, and to cook the sago into a jelly by stirring boiling water into the dry flour. When men wish to cook the sago starch themselves, it is more socially acceptable to prepare \textit{we}, the sago pancake that is made without adding water to the flour.

The second most prevalent source of carbohydrates is the banana (or plantain), of which, according to the folk taxonomy, there are 13 indigenous varieties. There are, in addition, various introduced varieties. Most of the commonly consumed bananas are of the starchy plantain variety that must be cooked (usually boiled), but some are sweet and may be eaten raw when ripe. While sago and bananas account for the bulk of the Ulwa diet and are the only traditional starches, some people today also grow and harvest root crops, such as yams, \textit{kaukau} (sweet potato), taro, and cassava (manioc, yuca). This is more common in Manu village than in Maruat, Dimiri, or Yaul, whose territory is swampier.

Another traditional staple on which the Ulwa people rely greatly is the coconut. Coconut milk is integral to the preparation of most meals; and coconut water may also be drunk, a helpful source of hydration during the dry season. People also grow leafy green vegetables, string beans, corn (maize), and sugarcane, among other crops, including non-food cash crops, such as tobacco and betel nut.

The most important source of protein is fish, especially during the dry season, when the lowered river levels facilitate fishing with nets. Other sources of protein include bandicoot, pig, lizard, the occasional crocodile, turtle, crayfish, wild and domesticated fowl, sago grubs, and eggs. Fat in the diet comes from coconut meat and milk as well as from animal sources. Vegetables are both grown and gathered.

Fishing is a common daily activity, often undertaken by children. A number of small species of fish are caught and then typically boiled, but sometimes (especially during times of great yields) they may be preserved by smoking. Fish, as well as the occasional crayfish or turtle, are caught either by net or by hand.

Hunting is the domain of men. It is usually undertaken at night, though this depends on the quarry: bandicoot, lizard, and crocodile are usually hunted at night, whereas pig is more commonly hunted during the day. All animals are hunted by spear. At night people are aided by battery-powered flashlights (which have replaced traditional flame torches). During the day (that is, to hunt pig) the hunters are assisted by dogs. Birds are also hunted, often by slingshot. This is one of the favorite pastimes of children.

Two major species of grub are harvested from sago palms: the relatively large \textit{siwi} and the smaller \textit{mïnkïn}; the latter is often worked into sago pancakes (\textit{we}). A third species of grub, \textit{mundum}, is taken from the trunks of certain dead tree species.

Few animals are raised, but some people do keep chickens, ducks, or larger fowl, which are used for eggs as well as meat. Despite the inefficiency and cost, some prominent villagers also raise the occasional pig for slaughter. As elsewhere in \isi{New Guinea}, pigs are very valuable and are important in paying bride prices.

A number of vegetables are gathered from the jungle, mostly leafy greens, such as \textit{yomal} (known as \textit{aibika} in \ili{Tok Pisin}) and \textit{anmopa} (known as \textit{tulip} in \ili{Tok Pisin}).

Men, women, and children thus spend much of their day gathering vegetables and insects, fishing and hunting, processing sago, tending their gardens, and cooking (generally two meals a day). Since there is limited food preservation, it is common to eat large meals when food is plentiful (and, of course, to do without when food is scarce). People also very commonly share with other families in the community. A butchered pig will provide meat for more than just the hunter’s family; leftovers are commonly offered to anyone who happens to be around.

The economy is thus fairly self-contained. Indeed, it has to be, since the nearest store is located in the town of Angoram, about six hours away by motorized canoe, a trip that requires an expensive amount of fuel. That said, cash does indeed enter the villages, especially Manu, which is more prosperous than the other three villages. Betel nut (the seed of the \textit{Areca catechu} palm) and tobacco (a crop introduced many generations ago) are grown, both for personal consumption and to be sold in town for the domestic market. Betel nut is especially popular among highland populations, who cannot grow areca palms in the mountains. Cocoa is also grown for sale, ultimately to enter the international market. People use cash to buy commodities such as pots and pans, batteries for flashlights, razors, metal nails, soap, clothing, and nonperishable foodstuffs, such as rice, noodles, canned fish and meat, palm oil, processed sugar, and salt.

Houses are built from jungle materials such as timber, woven bamboo, and vine ropes, but store-bought nails are occasionally used as well. Houses are raised on stilts, preventing the interiors from getting mired in swamp during the rainy season. Houses may contain multiple rooms as well as outdoor verandas, although some consist of just a single large room. They typically need to be rebuilt every five to seven years, an effort that can involve much of the community, who, working together, are able to finish constructing a house in about seven to ten days (although this process can take longer, especially if resources are limited). People sleep inside store-bought mosquito nets, which are especially important during the rainy season, when malaria-carrying mosquitoes plague the villages. Traditionally, people slept in meshwork enclosures (known as \textit{al}), which were made from bark. The insides of these traditional mosquito nets would become sweltering hot, especially when shared by multiple people. Malaria carried by mosquitoes is probably the greatest health risk that the villagers face.

Households can be large, as it is not uncommon for married couples to have six or more children, and grandparents and other relatives commonly live in the same household. Houses in Manu have about five people on average living in them, whereas those in Maruat, Dimiri, and Yaul tend to have more.

Manu village has a single elementary school, attended by most of the children in the village, as well as by many children from the neighboring Yamen village, whose native language is \ili{Ap Ma}, and by a few children from other, more distant, villages as well. Maruat, Dimiri, and Yaul share an elementary school, situated roughly in the middle of the three villages. The two schools provide instruction up through the eighth grade. Few students proceed with their education past that grade, since doing so would require living away from home in a larger town (such as Angoram), a financial burden and logistical difficulty.

The villagers are predominantly Christian, many of them devout and regular churchgoers. Manu and Maruat each have a single Catholic church; Yaul has one Catholic church and one Revival church; and Dimiri has four separate churches: Catholic, Jehovah’s Witness, Lutheran, and Seventh Day Adventist. Still, traditional beliefs in jungle spirits and magical powers persist. However, no traces remain of the ancestral men’s houses (known as \textit{haus tambaran} or \textit{haus boi} in \ili{Tok Pisin}, and as \textit{amba} in Ulwa), where earlier generations of young men were initiated into sacred rites.

In addition to fishing and hunting birds, children divert themselves by swimming in the river or playing sports, mostly soccer for boys, and volleyball and basketball for girls. Annual soccer competitions, to which teams from other villages are invited, are a major source of entertainment for both young and old.

Many other important community activities revolve around the church, which hosts prayer meetings for women, youth gatherings, and occasional feasts for special occasions, during which many families come together for potluck meals. A death in the family is occasion for a long period of mourning, and, depending on the (perceived) circumstances of the death, may require compensation to be paid to the bereaved. Many community conflicts are resolved by paying monetary (or equivalent) compensation, often brokered through the help of respected village leaders. If, for example, a dispute results in a physical altercation, and one party is injured or killed, then the assailant will be expected to pay a certain amount to the victim or the victim’s family.

Marriage, too, is a major cause for celebration. The family of the groom is typically required to pay (in money or goods, often pigs) the family of the bride (i.e., a bride price as opposed to a dowry). Nowadays in Manu, people are usually married by the Catholic church. Traditionally, marriage among Ulwa speakers was exogamous and patrilocal: it was customary for a man to marry a woman belonging to a different \textit{amba} ‘clan’ (literally ‘men’s house’), and for the woman to leave her clan to live with her new husband. Today, however, people practice both exogamy and endogamy. Clan distinctions are no longer recognized, and men and women alike are permitted to marry people from other villages (and of different language backgrounds); sometimes Ulwa speakers leave their village to live with their spouses, and other times their spouses move to the Ulwa village. These marriages are neither exclusively patrilocal nor matrilocal. Due to such exogamy, there are a number of speakers of other languages living in Manu, Maruat, Dimiri, and Yaul. Most marriages, however, are endogamous. Although formerly it was forbidden to marry within one’s clan, nowadays people abide by the simpler rule of avoiding marriages with first cousins or any more closely related \is{kinship} kin.

Although the Ulwa people have adopted many customs of broader modern Papua New Guinea, they are still mostly cut off from long-distance communication. Aside from the power provided by occasional diesel generators, the Ulwa-speaking villages lack electricity. There are no phone lines and, although some residents own cell phones, there is no cell phone service in the villages (or anywhere near them). Occasional newspapers find their way into the villages (often to be used for rolling cigarettes); these are written in either \ili{Tok Pisin} or \ili{English}. It is usually only when villagers visit the town of Angoram that that have access to information about the wider world, which of course finds its way back to the villages.

\section{Relationships with neighboring villages}\label{sec:1.5.5}

Ulwa is a relatively small language, flanked by languages that are more \is{vitality} vital and more widely spoken. All of Manu’s closest neighbors are villages where \mbox{\ili{Ap Ma}} is spoken. Maruat, Dimiri, and Yaul’s closest neighbors (excluding one another) speak \ili{Mundukumo}. While it is common (especially among older men) for Ulwa speakers to have some familiarity with \ili{Ap Ma} or \ili{Mundukumo}, very few \mbox{\ili{Ap Ma}} or \ili{Mundukumo} speakers have any facility with Ulwa. Although tribal warfare was once a regular part of life among villages in the area, the Ulwa villages nowadays enjoy mostly peaceful relations with their neighbors. Children from neighboring Yamen village attend the Manu Elementary School; it is common for travelers to overnight in neighboring villages; people buy and sell goods as they pass neighboring villages on the river; and soccer teams from different villages play against one another in friendly competition.

Manu shares a local representative (called a ward councilor) with the \ili{Ap Ma}-speaking village of Simbri, which is about 11 km (7 miles) northeast of Manu.

A map depicting the language communities that are nearest to Ulwa is provided in \figref{fig:1.4}. In this map, the five \ili{Keram} languages (Ulwa, \ili{Pondi}, \ili{Mwakai}, \ili{Ap Ma}, and \ili{Ambakich}) are presented in the darker shade of red. They constitute the \ili{Keram} branch of the \ili{Keram-Ramu} family. The two languages shown in a lighter shade of red are \ili{Waran} and \ili{Abu}, both members of the \ili{Ramu} branch of the \ili{Keram-Ramu} family. The five members of the \ili{Yuat} family (\ili{Changriwa}, \ili{Mundukumo}, \ili{Mekmek}, \ili{Bun}, and \ili{Kyenele}) are in gray. \ili{Kanda} (in green) is a member of the \ili{Lower Sepik} family. \ili{Buna} (in blue) is a member of the \ili{Marienberg} branch of the \ili{Torricelli} family.

\begin{figure}
\il{Buna}
\il{Kanda}
\il{Ambakich}
\il{Ap Ma}
\il{Mwakai}
\il{Pondi}
\il{Abu}
\il{Waran}
\il{Changriwa}
\il{Mundukumo}
\il{Mekmek}
\il{Bun}
\il{Kyenele}
\is{Sepik River}
\is{Yuat River}
\is{Keram River}
\is{Porapora River}
\caption{The language communities surrounding Ulwa. Contains map data © OpenStreetMap contributors, made available under the terms of the Open Database License (ODbL)}
\label{fig:1.4}
\includegraphics[width=\textwidth]{figures/Ulwa2022September01-img004.png}
\end{figure}

\section{Borrowing}\label{sec:1.5.6}

\is{borrowing|(}
\is{contact|(}
\is{loanword|(}

The language that has traditionally had the greatest influence on Ulwa’s \isi{lexicon}, especially on that of the \ili{Manu} \isi{dialect}, is \ili{Ap Ma}. \ili{Ap Ma} also belongs to the \ili{Keram} family, but it is not very closely related to Ulwa. \tabref{tab:1.1} provides some Ulwa words that likely come from \ili{Ap Ma} (less certain borrowings are indicated with a question mark).


\begin{table}
\caption{Loans from Ap Ma}
\label{tab:1.1}


\begin{tabular}{lll}
\lsptoprule
Ulwa word & gloss & \ili{Ap Ma} source word\\
\midrule
{\itshape ale-} & ‘scrape (sago)’ & <ale> ‘tool for beating sago’ (?)\\
{\itshape almba} & ‘hornbill’ & <alɨmba>\\
{\itshape amangala} & ‘hawk’ & <maŋgal> (?)\\
{\itshape i} & ‘lime’ & <ai>\\
{\itshape ika} & ‘riverbank’ & <jika>\\
{\itshape kaw} & ‘song’ & <ko>\\
{\itshape lam} & ‘meat’ & <lam>\\
{\itshape lanjin}  & ‘perch’ & <landʒin>\\
{\itshape le} & ‘rattan cane’ & <leag> (?)\\
{\itshape li} & ‘down’ & <ji> (?)\\
{\itshape lolop} & ‘just’ & <lolop>\\
{\itshape mïnkïn}  & ‘sago grub’ & <mɨlɨk>\\
{\itshape molombi}  & ‘statuette’ & <molɨmbi> ‘spirit’\\
{\itshape mom} & ‘grandmother’ & <mom> ‘old woman’\\
{\itshape mbinmbin} & ‘grave’ & <mbɨn> ‘land, ground’\\
{\itshape mblandu} & ‘rat species’ & <mbalundo>\\
{\itshape nandu} & ‘grass skirt’ & <nando>\\
{\itshape saklup} & ‘broom’ & <saklup> ‘coconut flower sheath’ \\
{\itshape samnang}  & ‘yam species’ & <semnoŋg>\\
{\itshape umbopa}  & ‘stomach’ & <mbuop> ‘belly’\\
{\itshape way} & ‘turtle’ & <we>\\
{\itshape wonglin} & ‘ladle’ & <koŋgle>\\
{\itshape woplota} & ‘lungs’ & <kolop> (?)\\
{\itshape yaki}  & ‘rat’ & <jake>\\
{\itshape yalum} & ‘grandchild’ & <jalum> ‘grandson’\\
{\itshape yokomakan} & ‘small wildfowl’ & <ɲokok> (?)\\
\lspbottomrule
\end{tabular}
\end{table}

Most of the \ili{Ap Ma} forms in \tabref{tab:1.1} are taken from \citet{Barlow2021}. Four forms (<ale>, <ko>, <leag>, and <lolop>), however, come from \citet{Wade1983}. The forms \textit{nandu} ‘grass skirt’ and \textit{yaki} ‘rat’ are both from the \ili{Maruat-Dimiri-Yaul} \isi{dialect}. The \ili{Ap Ma} form <molɨmbi> ‘spirit’ may ultimately be of \ili{Waran} origin (\tabref{tab:1.4}). The \ili{Ap Ma} form <ai> ‘lime’ is also found in its fellow \ili{East Keram} language \ili{Ambakich} (also known as \ili{Aion}) \citep[60]{Barlow2021}, and similar forms are found in nearby unrelated languages, such as those of the \ili{Lower Sepik} family: for example, <awi> in \ili{Yimas} and <awer> in \ili{Kanda} \citep[215]{Foley1986}. It probably traces back to an \ili{Austronesian} origin (\name{Timothy}{Usher}, p.c.).\footnote{Compare, for example, <avu> in the \ili{Austronesian} language \ili{Gitua} \citep[2]{Lincoln1977}, from \ili{Proto-Oceanic} *qapuR ‘lime’ \citep[612]{Blust2013}.}

The unrelated \ili{Mundukumo} language has also been influential on Ulwa. Some probable loans from \ili{Mundukumo} to Ulwa are given in \tabref{tab:1.2} (less certain borrowings are indicated with a question mark).


\begin{table}
\caption{Loans from Mundukumo}
\label{tab:1.2}


\begin{tabular}{lll}
\lsptoprule
Ulwa word & gloss & \ili{Mundukumo} source word\\
\midrule
{\itshape kalim} & ‘cassowary’ & <kɑlim>\\
{\itshape kokawe} & ‘bird species’ & <kukwɔm> ‘heron’ (?)\\
{\itshape ngïnïm} & ‘chin’ & <ɡənɑmɔŋ>\\
{\itshape sakïma, sakanma} & ‘adze, axe’ & <sɑkɑnmɑ> ‘axe’\\
{\itshape sawi} & ‘saliva’ & [unknown] cf. <sɑɸi> (\ili{Bun})\\
{\itshape wa} & ‘village’ & <watǝk> (?)\\
{\itshape walimot} & ‘pigeon’ & <wɑlim>\\
{\itshape wusim}  & ‘crocodile’ & <ɑsɪn> (?)\\
\lspbottomrule
\end{tabular}
\end{table}
Most of the \ili{Mundukumo} forms in \tabref{tab:1.2} are taken from \citet{Laycock1971a}. One form (<watǝk>), however, comes from \citet[18]{McElvenny2006}. I do not know the \ili{Mundukumo} word for ‘saliva’, but \citet[5056]{Laycock1971b} recorded the closely related language \ili{Bun} as having <sɑɸi> ‘saliva’. The form \textit{sakïma} ‘adze’ is from the \ili{Manu} \isi{dialect}; the form \textit{sakanma} ‘axe’ is from the \ili{Maruat-Dimiri-Yaul} \isi{dialect}. Additional loans from \ili{Mundukumo} that are found only in the \ili{Maruat-Dimiri-Yaul} \isi{dialect} are given in \sectref{sec:18.7}.

\ili{Mwakai}, which is a fellow member of the \ili{West Keram} family, may also be the source of some loans, given in \tabref{tab:1.3} (less certain borrowings are indicated with a question mark).


\begin{table}
\caption{Loans from Mwakai}
\label{tab:1.3}


\begin{tabular}{lll}
\lsptoprule
Ulwa word & gloss & \ili{Mwakai} source word\\
\midrule
{\itshape mbun} & ‘black, blue’ & <kïmbïn>\\
{\itshape katmombe} & ‘ant species’ & <kasïmbum>\\
{\itshape lïngïn, sïngïm} & ‘fog’ & <ingim>\\
{\itshape palam} & ‘cane grass’ & <pïrïm>\\
{\itshape wambana} & ‘fish’ & <sambon> (?)\\
\lspbottomrule
\end{tabular}
\end{table}
All \ili{Mwakai} forms in \tabref{tab:1.3} are taken from \citet{Barlow2020a}. The form \textit{lïngïn} ‘fog’ is from the \ili{Manu} \isi{dialect}; the form \textit{sïngïm} ‘fog’ is from the \ili{Maruat-Dimiri-Yaul} \isi{dialect}. The form <sambon> ‘fish species’ is also found in \ili{Pondi} \linebreak \citep[168]{Barlow2020b}. A similar word for ‘black’ is also found in \ili{Ap Ma}: <pɨnd> \citep[86]{Barlow2021}.

\ili{Pondi}, the third member of the \ili{West Keram} family, however, has not been identified as contributing many loans to Ulwa; nor has \ili{Ambakich} (of the \ili{East Keram} branch of the family), which is spoken somewhat farther away. \ili{Waran} (\ili{Ramu}), which is distantly related, may be the source of some \isi{borrowing} in Ulwa, although perhaps mediated by other languages (most likely \ili{Ap Ma}). Possible loans from these three languages are summarized in \tabref{tab:1.4} (less certain borrowings are indicated with a question mark).


\begin{table}
\caption{Possible loans from other languages}
\label{tab:1.4}


\begin{tabular}{lll}
\lsptoprule
Ulwa word & gloss & source language and word\\
\midrule
{\itshape wemali} & ‘sago pot’ & \ili{Ambakich} <mɨrɨ>\\
{\itshape anasa}  & ‘pick-axe’ & \ili{Pondi} <asangame> (?)\\
{\itshape kalam} & ‘knowledge’ & \ili{Waran} <kɑrɑm> ‘I know’\\
{\itshape lumnjap} & ‘fish species’ & \ili{Waran} <jɑp> ‘fish’ (?)\\
{\itshape molombi}  & ‘statuette’ & \ili{Waran} <mʉrɔm> ‘spirit of ancestors’ (?)\\
\lspbottomrule
\end{tabular}
\end{table}
The \ili{Ambakich} form in \tabref{tab:1.4} is taken from \citet[67]{Barlow2021}. The \ili{Pondi} form is taken from \citet[160]{Barlow2020b}. The first two \ili{Waran} forms are taken from \citet[8--9]{Butler1981a}. The third \ili{Waran} form is taken from \citet[20]{Z’graggen1972}.

\is{loanword|)}
\is{contact|)}
\is{borrowing|)}

\is{areal term|(}
\is{contact|(}
\is{borrowing|(}
\is{loanword|(}

Some words may be considered areal terms, in that similar forms recur among multiple families within the region. These have no doubt diffused due to \isi{contact}, but it is often difficult if not impossible to identify the immediate source for any given language. For example, the Ulwa word \textit{sokoy {\textasciitilde} sokay} ‘tobacco’ is definitely of foreign origin, but it is impossible to discern the immediate source for Ulwa, since this word is so pervasive in the area, as suggested by the mere sample of terms and languages provided in \tabref{tab:1.5}.\footnote{See the following sources: \ili{Ap Ma} \citep[83]{Barlow2021}, \ili{Mwakai} \citep[98]{Barlow2020a}, \ili{Pondi} \mbox{\citep[168]{Barlow2020b},} \ili{Ambakich} \citep[70]{Barlow2021}, \ili{Waran} \citep[9]{Butler1981b}, \ili{Rao} \citep[15]{Stanhope1980}, \ili{Tayap} (\citealt[424]{KulickTerrill2019}), \ili{Monumbo} (\citealt[182]{VormannScharfenberger1914}), \ili{Juwal} \citep[5336]{Laycock1971c}, \ili{Mehek} \citep[483]{Hatfield2016}, \ili{Sos Kundi} (\citealt{JanzenCorbalan2018}), \mbox{\ili{Chambri}} \citep[4974]{Laycock1971b}, \ili{Mundukumo} \citep[43]{McElvenny2006}, \ili{Andai} \citep[5268]{Laycock1971d}, \ili{Haruai} \citep[849]{Laycock1970}, \ili{Kalam} (\citealt[179]{PawleyBulmer2011}), \ili{Korowai} (\citealt[241]{deVriesvanEnk1997}), \ili{Nuk} (\citealt[26]{RetsemaEtAl2009}), \ili{Kis} \citep[3474]{Laycock1971e}.}


\begin{table}
\caption{Some languages with words similar to \textit{sokoy} {\textasciitilde} \textit{sokay} ‘tobacco’}
\label{tab:1.5}


\begin{tabular}{lll}
\lsptoprule
‘tobacco’ & language & family\\
\midrule
<soke> & \ili{Ap Ma} & \ili{Keram} (\ili{Keram-Ramu})\\
<soke> & \ili{Mwakai} & \ili{Keram} (\ili{Keram-Ramu})\\
<sakwe> & \ili{Pondi} & \ili{Keram} (\ili{Keram-Ramu})\\
<tʃuke> & \ili{Ambakich} & \ili{Keram} (\ili{Keram-Ramu})\\
<soke> & \ili{Waran} & \ili{Ramu} (\ili{Keram-Ramu})\\
<čukwai> & \ili{Rao} & \ili{Ramu} (\ili{Keram-Ramu})\\
<sokoi> & \ili{Tayap} & isolate (\ili{Torricelli}?)\\
<tsoḳáe> & \ili{Monumbo} & \ili{Bogia} (\ili{Torricelli})\\
<sɑkeí> & \ili{Juwal} & \ili{Marienberg} (\ili{Torricelli})\\
<sakwe> & \ili{Mehek} & \ili{Tama} (\ili{Sepik})\\
<sakuen> & \ili{Sos Kundi} & \ili{Ndu} (\ili{Sepik})\\
<sɑkwei> & \ili{Chambri} & \ili{Lower Sepik}\\
<sakue> & \ili{Mundukumo} & \ili{Yuat}\\
<tɑɣeik̚> & \ili{Andai} & \ili{Arafundi} (\ili{Upper Yuat})\\
<tʸuŋɡóy> & \ili{Haruai} & \ili{Piawi} (\ili{Upper Yuat})\\
<cgoy> & \ili{Kalam} & \ili{Madang} (\ili{Trans New Guinea}?)\\
<saukh> & \ili{Korowai} & \ili{Asmat-Awyu-Ok} (\ili{Trans New Guinea}?)\\
<seku> & \ili{Nuk} & \ili{Finisterre-Huon} (\ili{Trans New Guinea}?)\\
<sɑkwɛn> & \ili{Kis} & \ili{Oceanic} (\ili{Austronesian})\\
\lspbottomrule
\end{tabular}
\end{table}
Assuming that tobacco was not introduced to \isi{New Guinea} earlier than 1600 CE \citep[15]{Bourke2009}, it is unlikely that a word for ‘tobacco’ can be reconstructed with any significant depth within the \ili{Keram-Ramu} family.

It is possible that the words for ‘tobacco’ given in \tabref{tab:1.5} all derive from \ili{Malay} \textit{sugeh} or \textit{sogeh} or \textit{sugi} “quid (of tobacco …)” (cf. \cite[1128]{Wilkinson1959}).\footnote{I suspect that this same \ili{Malay} word may also be the ultimate origin of certain forms beginning with [j-], such as \ili{Manambu} <yaki> \citep[597]{Aikhenvald2008} (\ili{Sepik} family) and \ili{Yimas} <yaki> \citep[250]{Foley1991} (\ili{Lower Sepik} family). \citet[597]{Aikhenvald2008}, on the other hand, traces these forms back to \ili{Iatmul} as the assumed ultimate source, citing also \citegen[90]{Riesenfeld1951} hypothesis that they somehow derive from a corruption of \ili{English} \textit{tobacco} or \textit{smoke}.} 

Another term that seems to have diffused widely is the word for ‘chicken’ -- that is, the domesticated junglefowl that is generally thought to have been introduced to Oceania by Lapita people \citep{StoreyEtAl2008}. The Ulwa term \textit{wowal} ‘chicken’ probably derives from \ili{Proto-Keram} *kowal. Similar words in other languages are given in \tabref{tab:1.6}.\footnote{See the following sources: \ili{Ap Ma} \citep[77]{Barlow2021}, \ili{Mwakai} \citep[99]{Barlow2020a}, \ili{Pondi} \mbox{\citep[162]{Barlow2020b},} \ili{Ambakich} \citep[64]{Barlow2021}, \ili{Kaian} \citep[102]{Z’graggen1972}, \ili{Rao} \citep[102]{Z’graggen1972}, \ili{Tayap} (\citealt[376]{KulickTerrill2019}), \ili{Monumbo} (\citealt[159]{VormannScharfenberger1914}), \ili{Mehek} \citep[558]{Hatfield2016}, \ili{Kanda} \citep[216]{Foley1986}, \ili{Kalam} (\citealt[296]{PawleyBulmer2011}), \ili{Kis} \citep[3482]{Laycock1971e}.}


\begin{table}
\caption{Some languages with words similar to \textit{wowal} ‘chicken’}
\label{tab:1.6}


\begin{tabular}{lll}
\lsptoprule
‘chicken’ & language & family\\
\midrule
<kokol> & \ili{Ap Ma} & \ili{Keram} (\ili{Keram-Ramu})\\
<yokon> & \ili{Mwakai} & \ili{Keram} (\ili{Keram-Ramu})\\
<kawal> & \ili{Pondi} & \ili{Keram} (\ili{Keram-Ramu})\\
<kokor> & \ili{Ambakich} & \ili{Keram} (\ili{Keram-Ramu})\\
<kɑkur> & \ili{Kaian} & \ili{Ramu} (\ili{Keram-Ramu})\\
<kʌːr> & \ili{Rao} & \ili{Ramu} (\ili{Keram-Ramu})\\
<kokok> & \ili{Tayap} & isolate (\ili{Torricelli}?)\\
<kakatarak> & \ili{Monumbo} & \ili{Bogia} (\ili{Torricelli})\\
<koko> & \ili{Mehek} & \ili{Tama} (\ili{Sepik})\\
<kɨlɨkala> & \ili{Kanda} & \ili{Lower Sepik}\\
<klokl> & \ili{Kalam} & \ili{Madang} (\ili{Trans New Guinea}?)\\
<kʷɔkwɑrɛk> & \ili{Kis} & \ili{Oceanic} (\ili{Austronesian})\\
\lspbottomrule
\end{tabular}
\end{table}
There is most likely an \isi{onomatopoetic} element underlying these forms (based on the sound of a rooster’s crow). However, the fact that so many unrelated languages in a given area share this same general pattern of \textit{koko[l/r]} ‘chicken’ suggests the influence of \isi{contact}, perhaps ultimately an \ili{Austronesian} origin. Compare \ili{Proto-Oceanic} *kokorako \citep[284]{Clark2011}.

Another term that likely owes some of its etymology to \isi{onomatopoeia} is the word for ‘mosquito’. In Ulwa, this is \textit{yangun} ‘mosquito’ (\ili{Manu} \isi{dialect}) or \textit{nangun} ‘mosquito’ (\ili{Maruat-Dimiri-Yaul} \isi{dialect}). Similar forms are given in \tabref{tab:1.7}.\footnote{See the following sources: \ili{Mwakai} \citep[95]{Barlow2020a}, \ili{Pondi} \citep[166]{Barlow2020b}, \ili{Waran} \citep[114]{Z’graggen1972}, \ili{Kaian} \citep[114]{Z’graggen1972}, \ili{Kopar} \citep[215]{Foley1986}, \ili{Tabriak} \citep[215]{Foley1986}, \ili{Monumbo} (\citealt[155]{VormannScharfenberger1914}), \ili{Aisi} \citep[280]{Daniels2020}, \ili{Kis} \citep[3482]{Laycock1971e}.}


\begin{table}
\caption{Some languages with words similar to \textit{yangun} {\textasciitilde} \textit{nangun} ‘mosquito’}
\label{tab:1.7}


\begin{tabular}{lll}
\lsptoprule
‘mosquito’ & language & family\\
\midrule
<nangun> & \ili{Mwakai} & \ili{Keram} (\ili{Keram-Ramu})\\
<nangun> & \ili{Pondi} & \ili{Keram} (\ili{Keram-Ramu})\\
<uɡun> & \ili{Waran} & \ili{Ramu} (\ili{Keram-Ramu})\\
<nɑŋɡit> & \ili{Kaian} & \ili{Ramu} (\ili{Keram-Ramu})\\
<naŋgɨt> & \ili{Kopar} & \ili{Lower Sepik}\\
<yaŋgun> & \ili{Tabriak} & \ili{Lower Sepik}\\
<ng̲̲e̲t> & \ili{Monumbo} & \ili{Bogia} (\ili{Torricelli})\\
<nagur> & \ili{Aisi} & \ili{Madang} (\ili{Trans New Guinea}?)\\
<nʊɡur> & \ili{Kis} & \ili{Oceanic} (\ili{Austronesian})\\
\lspbottomrule
\end{tabular}
\end{table}

\is{loanword|)}
\is{borrowing|)}
\is{contact|)}
\is{areal term|)}

\is{borrowing|(}
\is{contact|(}
\is{loanword|(}

One major and more recent source of \isi{borrowing} into Ulwa is \ili{Tok Pisin}, the \ili{English}-based creole that serves as Papua New Guinea’s primary lingua franca, and is rapidly becoming the first (and only) language for more and more Papua New Guineans. Some \ili{Tok Pisin} words are used to refer to novel referents for which there is no native word (e.g., \textit{koko} ‘cocoa’, \textit{popo} ‘papaya’), although \linebreak \isi{coinage}s based on native words are also possible (e.g., \textit{asimu} ‘grass seed’ for ‘rice’), as are \isi{metaphor}ical extensions of existing lexemes (e.g., \textit{apïn} ‘fire’ for ‘matches, lighter’) (\sectref{sec:14.9}).

\ili{Tok Pisin} forms are also sometimes used where native vocabulary would also be possible. Even among the oldest speakers there is frequent \isi{code-switching} between Ulwa and \ili{Tok Pisin}, and speech in all registers is commonly peppered with \ili{Tok Pisin} words, such as \textit{olsem} ‘thus’, \textit{nogat} ‘no’, and \textit{tok} ‘talk’. Some \ili{Tok Pisin} function words are used where there is no equivalent in Ulwa, such as \textit{na} ‘and’ and \textit{o} ‘or’ (see \sectref{sec:12.1} on \isi{coordination}).

\isi{Loanword}s may be naturalized to the \isi{phonotactics} of Ulwa. In practice, this most commonly results in pronouncing loan \isi{rhotic}s as \isi{lateral}s and pronouncing loan plain \isi{voiced} \isi{stop}s as \isi{prenasalized} \isi{voiced} \isi{stop}.

While many of the aforementioned borrowings may be viewed as natural forms of linguistic change (that is, the type typically experienced by languages that are not \isi{endangered}), there have also been influences on Ulwa that likely reflect its recent decline in usage and current state of severe \isi{endangerment} (see \chapref{sec:15} for the effects of \isi{endangerment} on Ulwa’s grammatical structure).

\is{loanword|)}
\is{contact|)}
\is{borrowing|)}

\section{Dialects}\label{sec:1.5.7}

\is{dialect|(}

There are two major \isi{dialect}s of Ulwa. One is spoken in \ili{Manu} village, while the other is spoken in the three villages of \ili{Maruat}, \ili{Dimiri}, and \ili{Yaul}. Although there are some minor differences among these latter three villages as well, they are each much more similar to one another than to \ili{Manu}. Speakers from all four villages consider all four communities to speak the same language, although each village notes how other villagers “change” the language slightly.

The two \isi{dialect}s are mutually intelligible. Although \isi{morphosyntactic}ally quite similar, they exhibit a number of \isi{lexical} differences, some due to a handful of innovative \isi{sound change}s, most notably of *l to /n/ in many environments in \ili{Manu}. \ili{Maruat-Dimiri-Yaul} is, in general, the more conservative of the two \isi{dialect}s. Other \isi{lexical} differences may be due to \isi{borrowing} or other forms of replacement. Unless otherwise noted, the data in this grammar have been gathered from speakers of the \ili{Manu} \isi{dialect}. Occasional reference to \ili{Maruat-Dimiri-Yaul} is made where relevant, and a brief comparative description of this \isi{dialect} (in particular the variety spoken in Yaul village) is given in \chapref{sec:18}.

\is{dialect|)}

\section{Language vitality}\label{sec:1.6}

\is{endangerment|(}
\is{language endangerment|(}
\is{vitality|(}
\is{language vitality|(}

Ulwa is a severely \isi{endangered} language. Although the population of all four Ulwa-speaking villages is swelling, the language is not being transmitted to children. I estimate that there are fewer than 600 fluent speakers or Ulwa, plus roughly 1,200 semi-speakers. If these numbers were to be naïvely compared to earlier reports of speaker numbers, then these figures could, perhaps, appear relatively high -- that is, as if the language were thriving. Earlier counts of Ulwa speakers, however, may be misleading. When \citet[36]{Laycock1973} first reported on the language, he offered the number 814. As is often the case with apparent speaker number counts, however, this is not intended to mean the number of Ulwa speakers, but rather the combined population of the villages where Ulwa is spoken, in this case, the “population (estimated or censused) as at 1 {January 1970}” \citep[3]{Laycock1973}.

Official census numbers from the Papua New Guinea government are also problematic. The most recently released census (from 2011) lists 669 inhabitants for Manu, 424 for Muruat (= Maruat), and 1,111 for Dimiri; it has no information on Yaul (\citealt[31--32]{NationalStatisticOffice2014}). The census thus seems to be lacking in data. Furthermore, the figure of 669 for “Manu” seems to include the population of the neighboring Simbri village, which shares a government representative with Manu, but whose population speaks a different language \mbox{(\ili{Ap Ma}).}

Population counts conducted by villagers in 2017 at my request yielded the figures given in \tabref{tab:1.8}--\tabref{tab:1.11} for each village.

\begin{table}
\caption{\label{tab:1.8} Age demographics of Manu village in 2017}
\begin{tabular}{rrr}
\lsptoprule
age range & number of people & percentage of total\\
\midrule
0--19 & 186 & 50\%\\
20--40 & 113 & 31\%\\
> 40 & 70 & 19\%\\
\midrule
Total population & 369 & 100\%\\
\lspbottomrule
\end{tabular}
\end{table}

\begin{table}
\caption{\label{tab:1.9} Age demographics of Maruat village in 2017}
\begin{tabular}{rrr}
\lsptoprule
age range & number of people & percentage of total\\
\midrule
0--19 & 406 & 49\%\\
20--40 & 234 & 28\%\\
> 40 & 192 & 23\%\\
\midrule
Total population & 832 & 100\%\\
\lspbottomrule
\end{tabular}
\end{table} \largerpage

\begin{table}
\caption{\label{tab:1.10} Age demographics of  Dimiri village in 2017}
\begin{tabular}{rrr}
\lsptoprule
age range & number of people & percentage of total\\
\midrule
0--19 & 728 & 52\%\\
20--40 & 446 & 32\%\\
> 40 & 225 & 16\%\\
\midrule
Total population & 1,399 & 100\%\\
\lspbottomrule
\end{tabular}
\end{table}

\begin{table}
\caption{Age demographics of Yaul village in 2017}
\label{tab:1.11}
\begin{tabular}{rrr}
\lsptoprule
age range & number of people & percentage of total\\
\midrule
0--19 & 636 & 49\%\\
20--40 & 459 & 35\%\\
> 40 & 204 & 16\%\\
\midrule
Total population & 1,299 & 100\%\\
\lspbottomrule
\end{tabular}
\end{table}

The population of the four Ulwa-speaking villages in 2017 was thus 3,899, as shown in \tabref{tab:1.11a}. This figure of approximately 3,900 was probably about equal to the total (global) ethnic Ulwa population at the time: the small number of non-Ulwas living within these four villages was probably about equal to the number of Ulwas living outside them. Considering the birth and death rates in the area, I estimate the current ethnic Ulwa population to be 4,500 or more (in 2023).

\begin{table}
\caption{Total population of all four Ulwa villages in 2017}
\label{tab:1.11a}
\begin{tabular}{rrr}
\lsptoprule
age range & number of people & percentage of total\\
\midrule
0--19 & 1,956 & 50\%\\
20--40 & 1,252 & 32\%\\
> 40 & 691 & 18\%\\
\midrule
Total population & 3,899 & 100\%\\
\lspbottomrule
\end{tabular}
\end{table}

{In 2017} there were 691 Ulwas older than 40 years old. At that time, I observed that almost all Ulwas older than 40 were fluent (or mostly fluent) speakers of the language, whereas very few Ulwas younger than 40 were fluent (and the number of non-fluent older-than-40 Ulwas was probably about equal to the number of fluent younger-than-40 Ulwas). Given typical mortality rates (and the fact that there are no new Ulwa speakers), the number of fluent Ulwa speakers may be less than 600 today (in 2023). The number of semi-speakers (roughly correlating to the number of Ulwas who were between 20 and 40 years old in 2017) is probably not much higher than 1,200. Thus, of a total ethnic population of roughly 4,500 Ulwas, perhaps less than 15\% are fluent speakers and less than 30\% are semi-speakers.

\newpage

“Semi-speaker” is not a term with a standard definition, and it can be used to mean different things in different contexts. The Ulwa semi-speaker community can be roughly defined by those who:
  
\begin{quote}
\begin{enumerate}[noitemsep, label={(\roman*)}, align=left, widest=190, labelsep=1ex,leftmargin=*]
\item generally understand the spoken language;
\item can produce short responses, generally limited to \isi{formulaic expression}s or words in common \isi{semantic} domains (e.g., food preparation, traveling); and
\item have native \isi{phonology}; but
\item cannot produce long, interrupted speech in Ulwa.
  \end{enumerate}
  \end{quote}
  
Indeed, even many speakers defined here a “fluent” have difficulty speaking Ulwa for long stretches and are generally more comfortable using \ili{Tok Pisin}.

\is{language vitality|)}
\is{vitality|)}
\is{language endangerment|)}
\is{endangerment|)}

\is{endangerment|(}
\is{language endangerment|(}
\is{vitality|(}
\is{language vitality|(}

The group of “non-speakers” (who constitute more than half the ethnic population) consists of those who cannot understand spoken Ulwa or produce any novel utterances. However, some vocabulary items (especially for very culturally salient referents, like sago, betel nut, and string bags) are so integrated into the colloquial Ulwa form of \ili{Tok Pisin}, that probably almost all ethnic Ulwas know their meanings and can produce them.

The estimate of < 600 fluent speakers may be considered “small” in comparison to some of Ulwa’s more expansive neighbors, in particular \ili{Ap Ma}. Furthermore, it is readily apparent that the language is in rapid decline. Less than half of the ethnic population speaks the language to any degree, and less than one-seventh of the ethnic population is fluent.

The most important indicator for the decline in \isi{vitality} of Ulwa is the utter lack of intergenerational transmission. Even though most people older than about 45 (in 2023) are fluent, they nevertheless communicate almost entirely in \ili{Tok Pisin}. Some elders regularly use Ulwa, but even among members of this demographic group there are no monolinguals, and elders often rely heavily on \ili{Tok Pisin} as well. For those who do speak Ulwa, \isi{code-switching} with \ili{Tok Pisin} is common. Even attempts at “pure” speech (offered for the benefit of the outside researcher) are usually riddled with \ili{Tok Pisin} \isi{loanword}s. It is very uncommon for parents to speak Ulwa to their children, aside from in a few set formulae (e.g., \textit{Umbenam anma!} ‘Good morning!’, \textit{U ango mana?} ‘Where are you going?’, \textit{Aw kot nïnan!} ‘Please pass betel the nut!’). Several nouns referring to culturally salient items are commonly used in child-directed speech (and in general) (e.g., \textit{ay} ‘jellied sago’, \textit{we} ‘sago starch’, \textit{ani} ‘string bag’).

Thus, \ili{Tok Pisin} is the language that is currently exerting the greatest influence on Ulwa. Every Ulwa community member speaks \ili{Tok Pisin} (i.e., there are are no monolingual Ulwa speakers), and probably every Ulwa speakers is more comfortable using \ili{Tok Pisin} than Ulwa. \ili{English} is ostensibly the language of instruction at the two elementary schools, but teachers often resort to using \ili{Tok Pisin}, and students do not seem to be acquiring \ili{English}; or, if they are, then they are developing a passive knowledge at best. There are perhaps just a handful of adults with a passing knowledge of \ili{English}. Some members of the older generation are also conversant in one or more of the neighboring languages. As Manu village is quite near a number of \ili{Ap Ma}-speaking villages, some residents have a familiarity with this language. Similarly, Maruat, Dimiri, and Yaul are not far from the \ili{Mundukumo} language community, and some villagers there can speak this language as well.

\ili{Tok Pisin} is the primary language in almost every domain. Church services, classroom instruction (alongside \ili{English}), sporting events, communication with other villages, and even family discussions are typically all conducted in \ili{Tok Pisin}. Most parents exclusively use \ili{Tok Pisin} when addressing their children.

While some adults overestimate the linguistic abilities of the younger generation, assuming that they will naturally become speakers of Ulwa once they become older, many villagers are becoming concerned about the fate of their language, noting the fast decline in intergenerational transmission. There is interest in introducing Ulwa into the Manu school classrooms, but this could prove very difficult, if not impossible, in part due to the dearth of language materials in Ulwa, but mainly due to the fact that most students are not from an Ulwa-speaking village (slightly more than half the students commute from the nearby \ili{Ap Ma}-speaking Yamen village). Furthermore, most teachers in the Manu elementary school come from other parts of Papua New Guinea entirely, such as the highlands, and are thus not speakers of Ulwa.

Local attitudes toward the language are positive. Many ethnic Ulwas lament the decline in language use, as well as the loss of traditional ecological knowledge that they sometimes view as being bound to linguistic knowledge. Although a number of community members have aspirations of economic advancement and, as such, support the use of dominant languages, Ulwa is not viewed as a hindrance to progress. Rather, there is a fairly common view that the spread of \ili{Tok Pisin} has been an unnecessary step in the process of globalization, many wishing their children to be fluent in just two languages, \ili{English} and Ulwa, the former for reasons of socioeconomic betterment and cross-cultural communication, the latter for reasons of cultural preservation and identity. \largerpage[2]

\is{language vitality|)}
\is{vitality|)}
\is{language endangerment|)}
\is{endangerment|)}

\is{endangerment|(}
\is{language endangerment|(}
\is{vitality|(}
\is{language vitality|(}

There have been a number of frameworks proposed for assessing language \linebreak vitality. The following subsections offer assessments of Ulwa’s \isi{vitality} with \linebreak respect to UNESCO’s nine factors (\sectref{sec:1.6.1}), the Language Endangerment Index (LEI) (\sectref{sec:1.6.2}), and the Expanded Graded Intergenerational Disruption Scale \linebreak (EGIDS) (\sectref{sec:1.6.3}).

\subsection{UNESCO’s nine factors}\label{sec:1.6.1}

Based on \citegen{UNESCO2003} framework, Ulwa would be considered severely \isi{endangered} or critically \isi{endangered}. \tabref{tab:1.12} presents Ulwa’s \isi{endangerment} status according to each of UNESCO’s nine factors.


\begin{table}
\caption{Ulwa’s endangerment according to UNESCO’s nine factors (\citealt[7--17]{UNESCO2003})}
\label{tab:1.12}

\begin{tabularx}{\textwidth}{lQQ}
\lsptoprule
factor & description & Ulwa’s status\\
\midrule
1 & “Intergenerational language transmission” & “severely endangered” (2)\\
\tablevspace
2 & “Absolute number of speakers” & “at risk”\\
\tablevspace
3 & “Proportion of speakers within the total population” & “severely endangered” (2)\\
\tablevspace
4 & “Trends in existing language domains” & “limited or formal domains” (2)\\
\tablevspace
5 & “Response to new domains and media” & “inactive” (0)\\
\tablevspace
6 & “Materials for language education and literacy” & “a practical \isi{orthography} is known to the community” (1)\\
\tablevspace
7 & “Governmental and institutional language attitudes and policies, including official status and use” & “equal support” (5)\\
\tablevspace
8 & “Community members’ attitudes toward their own language” & “most members support language maintenance” (4)\\
\tablevspace
9 & “Amount and quality of documentation” & “fair” (3)\\
\lspbottomrule
\end{tabularx}
\end{table}
The first six factors are meant to be taken together to indicate the language’s \isi{vitality}. Factor 2 does not have a grade associated with it. Of the remaining five factors, Ulwa averages a grade of 1.4 out of 5.0, which can probably be taken to mean that the language is “severely endangered” or “critically endangered”.

\subsection{LEI}\label{sec:1.6.2}

According to the LEI (Language Endangerment Index) (\citealt{LeeVanWay2016,LeeVanWay2018}), Ulwa would be classified as “severely endangered”, receiving an \isi{endangerment} score of 68\% (where “severely endangered” equals 61--80\%, with a higher percentile indicating greater \isi{endangerment}). The LEI assessment of Ulwa is summarized in \tabref{tab:1.13}.


\begin{table}
\caption{Ulwa’s endangerment according to the LEI (\citealt[58--62]{LeeVanWay2018})}
\label{tab:1.13}
\begin{tabularx}{\textwidth}{QQ>{\raggedright\arraybackslash}p{.4\textwidth}}
\lsptoprule
LEI factor & Ulwa’s status & description in LEI\\
\midrule
f\textsubscript{1} Intergenerational transmission & 3: endangered & “Some adults in the community are speakers, but the language is not spoken by children.”\\
\tablevspace
f\textsubscript{2} Absolute number of speakers & 3: endangered & “100--999 speakers”\\
\tablevspace
f\textsubscript{3} Speaker number trends & 4: severely endangered & “Less than half of the community speaks the language, and speaker numbers are decreasing at an accelerated pace.”\\
\tablevspace
{f\textsubscript{4} Domains of use} & {4: severely endangered} & {“Used mainly just in the home and/or with family, and may not be the primary language even in these domains for}\\ & & {many community members.”}\\
\tablevspace
calculation of factors:

[(f\textsubscript{1}×2) + f\textsubscript{2} + f\textsubscript{3} + f\textsubscript{4}] ÷ 25 & [(3×2) + 3 + 4 + 4] ÷ 25 = 68\% & “80--61\% = Severely Endangered”\\
\lspbottomrule
\end{tabularx}
\end{table}

\newpage

\subsection{EGIDS}\label{sec:1.6.3}

According to the EGIDS (Expanded Graded Intergenerational Disruption Scale) (\citealt{LewisSimons2010}), Ulwa may be assumed to be either “Level 7 (\isi{shift}ing)” or “Level 8a (moribund)”. If semi-speakers are admitted into the set of people who “can” use the language, then “Level 7” applies (“The child-bearing generation knows the language well enough to use it among themselves but none are transmitting it to their children”, p. 210). If, however, a higher proficiency in the language is required to qualify one as a speaker, then “Level 8a” seems more appropriate (“The only remaining active speakers of the language are members of the grandparent generation”, p. 210). Citing data from \citet{Barlow2018a}, the 26th edition of \textit{Ethnologue} \citep{EberhardEtAl2023} classifies the status Ulwa as “8a (moribund)”.

\is{language vitality|)}
\is{vitality|)}
\is{language endangerment|)}
\is{endangerment|)}

\section{Classification}\label{sec:1.7}

\is{classification|(}
\is{language classification|(}
\is{genealogical classification|(}
\is{genealogical affiliation|(}

In this section I discuss the genealogical \isi{classification} of Ulwa. A more detailed \isi{classification}, which presents the evidence on which the following subsections are based, will be found in \citet{UsherBarlow}.

\subsection{Papuan languages}\label{sec:1.7.1}

First a note on so-called \isi{Papuan} languages is in order. This oft-used category of languages does not refer to a single language family, since its members are not all demonstrably genealogically related. Instead, it is a negative \isi{classification}, referring to all the indigenous (non-sign) languages spoken within a particular area of the southwest Pacific that do not belong to the \ili{Austronesian} language family. \citet[357]{Foley2000}, using the term “\isi{New Guinea} region”, defines this area as roughly running “from the easterly Indonesian islands of Halmahera, Timor, and Alor in the west (125°E), to the westerly island group of New Georgia in the Solomon Islands in the east (155°E), a land area of approximately 850,000 km\textsuperscript{2}”. This heterogeneous group of non-\ili{Austronesian} languages consists of numerous families: even the most liberal classifications (in terms of a researcher’s willingness to accept evidence for genetic relatedness) posit no fewer than 32 \isi{Papuan} families and isolates, as in \citet[30]{Ross2005}. The most conservative classifications, on the other hand, could distinguish more than 120 \isi{Papuan} families and isolates. \textit{Glottolog 4.7}, for example, includes 71 families and 52 isolates that could be considered “\isi{Papuan}” \citep{HammarströmEtAl2022}. Moderate estimates might be closer to a combined 80 families and isolates. \citet{Palmer2018}, for example, identifies 43 families and 37 isolates.\largerpage

These families and isolates are found on the island of \isi{New Guinea} and its smaller satellite islands, as well as in the Bismarck Archipelago, the Solomon Islands Archipelago, the northern Maluku Islands, and the Alor Archipelago. Additionally, one \isi{Papuan} language is spoken within the territory of Australia: \ili{Meriam}, an \ili{Eastern Trans-Fly} language spoken in the Torres Strait. The extinct language \ili{Tambora} once spoken on the island of Sumbawa has been designated as \isi{Papuan} as well \citep{Donohue2007}.

All told, there are well over 800 indigenous spoken non-\ili{Austronesian} languages in this region. \citet[7]{Palmer2018} counts 862; \textit{Glottolog 4.7}, by my count, includes 876 \citep{HammarströmEtAl2022}. These \isi{Papuan} languages have at times suffered from the zealous efforts of comparative linguists to fit them all into a small number of large language families, most notably the “\ili{Indo-Pacific hypothesis}” \citep{Greenberg1971} and the “\ili{Trans-New Guinea Phylum}” (\citealt{McElhanonVoorhoeve1970}; \citealt{WurmEtAl1975}). But the broader claims of genetic affiliations have mostly failed to garner support from rigorous application of the comparative method. While it is certainly possible that all \isi{Papuan} languages descend from just a few \isi{protolanguage}s (or even just a single \isi{protolanguage}), it may simply be impossible (given current methods and the nature of the data available) to prove this. One reason for this is likely the great time depth involved. The ancestors of the modern \isi{Papuan} peoples first migrated to the \isi{Sahul} continent perhaps some 47,000 to 51,000 years ago (\citealt[3]{AllenO’Connell2020}), and, given their subsequent dispersal, their languages were allowed multiple millennia during which to diversify. Perhaps many sister languages have diversified to the point that any \isi{cognacy} (if it were ever present) is now irrecoverable due to the extensive amount of language change over such an extensive amount of time.

\subsection{The West Keram family (Ulmapo family)}\label{sec:1.7.2}
\il{West Keram}
Instead of starting with massive notions like “\isi{Papuan}” or “\ili{Trans New Guinea}”, a more rigorous approach to language \isi{classification} is to work bottom-up. The languages that are the most obviously related to Ulwa are \ili{Mwakai} and \ili{Pondi}. \citet[36]{Laycock1973} was the first to recognize this grouping of three languages, referring to the group as “\ili{Mongol-Langam}” (and placing them within the \ili{Ramu} branch of his “\ili{Sepik-Ramu Phylum}”). \citet[205--206]{Foley2018} referred to them as “\ili{Koam}” and placed them within the \ili{Ramu} branch of his “\ili{Lower Sepik-Ramu} Family”. Previously, I referred to them as “\ili{Ulmapo}”, while remaining agnostic as to their genealogical relationship to other languages (\citealt[11, 34--35, passim]{Barlow2018a}). Here, however, following Timothy \citet{UsherNoDate}, I refer to Ulwa, \ili{Mwakai}, and \ili{Pondi} as “\ili{West Keram}”, as their relationship to the two “\ili{East Keram}” languages has been established (\sectref{sec:1.7.3}). The \ili{West Keram} languages are spoken in the area to the west of the \isi{Keram River}, between the \isi{Keram River} and the \isi{Yuat River}, both of which are tributaries of the \isi{Sepik River} to the north. A grammatical sketch of \ili{Mwakai} is provided by \citet{Barlow2020a}, and a somewhat more developed description of \ili{Pondi} is provided by \citet{Barlow2020b}.

Impressionistically, I would say that the three languages are fairly similar in terms of \isi{morphosyntax}. \isi{Lexical}ly, however, they are somewhat more divergent from one another. Among the items included in the \ia{Swadesh, Morris} \isi{Swadesh} 100-word list and standard SIL-PNG survey \is{word list} word list (so-called “\isi{basic vocabulary}”), around 40--50\% are \isi{cognate} among the \ili{West Keram} language.

The three members of the \ili{West Keram} (\ili{Ulmapo}) group are presented in \figref{fig:1.5}, which provides the seven villages in which the three languages have traditionally been spoken. Note that the terminal nodes in \figref{fig:1.5} are village names, not languages. Also, \ili{Mwakai} and \ili{Pondi} likely constitute a subgroup within \ili{West Keram} (\sectref{sec:1.7.3}), although this is not depicted in \figref{fig:1.5}.


\begin{figure}
\caption{The seven villages of the West Keram (Ulmapo) group}
\il{West Keram}
\il{Ulmapo}
\il{Mwakai}
\il{Pondi}
\il{Maruat-Dimiri-Yaul}
\il{Manu}
\label{fig:1.5}
%\includegraphics[width=\textwidth]{figures/Ulwa2022September01-img005.png}
\begin{forest}
[West Keram
  [Mwakai
    [Mongol, tier=word]
    [Kaimbal, tier=word]
  ]
  [Pondi
    [Langam, tier=word]
  ]
  [Ulwa
    [Maruat-Dimiri-Yaul
      [Maruat, tier=word]
      [Dimiri, tier=word]
      [Yaul, tier=word]
    ]
     [Manu
    [Manu, tier=word]
    ]
  ]
]
\end{forest}
\end{figure}

\subsection{The Keram family}\label{sec:1.7.3}

The \ili{West Keram} family (\sectref{sec:1.7.2}) constitutes one of two branches of the \ili{Keram} family. The other branch is the \ili{East Keram} family, which consists of two languages: \ili{Ambakich} and \ili{Ap Ma}. Although \citet{Laycock1973} and \citet{Foley2018} both included all five \ili{Keram} languages in their respective versions of a “\ili{Sepik-Ramu}” macrofamily, Timothy \citet{UsherNoDate} was the first to propose a \ili{Keram} family consisting of these (and only these) two branches. The \ili{East Keram} languages are spoken mostly along the \isi{Keram River} and in the area immediately to the east, along the \isi{Porapora River} (also known as the \isi{Bien River}), which is also a tributary of the \isi{Sepik River} to the north. A \isi{dialect} survey of \ili{Ambakich} with \isi{wordlist}s and observations on language use is provided by \citet{PotterEtAl2008}. A \isi{phonological} sketch of \ili{Ambakich} (with \isi{wordlist}) is provided by \citet{Barlow2021}. \ili{Ap Ma} is relatively better described. Martha \citet{Wade1984} has written a master’s thesis providing a grammatical description, along with several unpublished manuscripts, including a dictionary \citep{Wade1983}. An \ili{Ap Ma} \isi{wordlist} is also provided by \citet{Barlow2021}.

The two \ili{East Keram} languages are more distantly related, although there are still a fair number of \isi{lexical} \isi{cognate}s (these were largely overlooked in the past due to \ili{Ap Ma}’s loss of initial \isi{syllable}s from \isi{polysyllabic} words). Using the same rough means of describing \isi{lexical} similarity mentioned in \sectref{sec:1.7.2}, I would estimate that around 30\% of “\isi{basic vocabulary}” is \isi{cognate} between \ili{Ambakich} and \ili{Ap Ma} (although \isi{sound change}s rather obscure a number of these). Between the two branches (\ili{East Keram} and \ili{West Keram}), around 20--30\% of “\isi{basic vocabulary}” is \isi{cognate}.

\figref{fig:1.6} is a tree depicting what I consider to be the most likely subgroupings of the \ili{Keram} family. The evidence for the \ili{Mwakai-Pondi} subgrouping is less secure.

\is{genealogical affiliation|)}
\is{genealogical classification|)}
\is{language classification|)}
\is{classification|)}

\begin{figure}
\caption{The five languages of the Keram family}
\label{fig:1.6}
\il{Keram}
\il{West Keram}
\il{East Keram}
\il{Mwakai-Pondi}
\il{Mwakai}
\il{Pondi}
\il{Ap Ma}
\il{Ambakich}

%\includegraphics[width=\textwidth]{figures/Ulwa2022September01-img006.png}
\begin{forest}
[Keram
  [West Keram
    [Mwakai-Pondi
      [Mwakai, tier=word]
      [Pondi, tier=word]
    ]
    [Ulwa, tier=word]
  ]
  [East Keram
    [Ambakich, tier=word]
    [Ap Ma, tier=word]
  ]
]
\end{forest}
\end{figure}

\subsection{The Keram-Ramu family}\label{sec:1.7.4}

\is{classification|(}
\is{language classification|(}
\is{genealogical classification|(}
\is{genealogical affiliation|(}

Whereas the \ili{Keram} family is well established with the support of basic \isi{lexical} \isi{cognate}s and regular sound correspondences, attempts to reconstruct deeper genealogical connections begin to test the limits of historical linguistics. Nevertheless, the evidence for a \ili{Keram-Ramu} family is quite strong \citep{UsherBarlow}. The “\ili{Ramu Phylum}” was first proposed by John A. \citet[27, 137--179]{Z’graggen1969}.\footnote{A revised version is given in \citet[14, 73--92]{Z’graggen1971}.} \citet[88]{Z’graggen1971} only tentatively includes \ili{Ambakich} and \ili{Ap Ma} in the \ili{Ramu} family; of the six villages where Ulwa and \ili{Mwakai} are spoken, he mentions that they “were said to have languages of their own” (no mention is made of \ili{Pondi} or the village where it is spoken, Langam). Z’graggen’s \isi{classification} of \ili{Ramu} is impressively good, and -- with the minor adjustment of moving \ili{Ambakich} and \ili{Ap Ma} to a coordinate \ili{Keram} branch -- it has essentially been upheld by subsequent investigation.

There are not many \isi{lexical} \isi{cognate}s between \ili{Keram} and \ili{Ramu}, although they do exist. Function words and \is{bound morpheme} bound \isi{morphology}, rather, provide the strongest evidence for the \ili{Keram-Ramu} family. In short, the \isi{pronoun} paradigms of \ili{Proto-Keram} and \ili{Proto-Ramu} are completely \isi{cognate}, including \isi{dual} marking, as are the \isi{deictic} \isi{demonstrative}s. There are also two (and only two) nominal \isi{plural} \isi{suffix}es that reconstruct back to each branch’s \isi{protolanguage} and which are \isi{cognate} across the two. There may also be a \isi{singulative} \isi{suffix} *-m, although a possible \isi{derivation} from a medial \isi{deictic} marker leaves open the possibility of later innovation. Likewise, although an \isi{oblique marker} *-n is likely reconstructible back to \ili{Proto-Keram-Ramu}, this is not particularly enlightening, since this morpheme may have long ago diffused across unrelated families, including \ili{Sepik} and \ili{Lower Sepik}, as proposed by \citet[68]{Foley1986}.

 The \ili{Keram-Ramu} family consists of a total of some 29 languages, A genealogical \isi{classification} of the \ili{Keram-Ramu} languages, based on \citet{UsherBarlow}, is provided in \figref{fig:1.7}.


\begin{figure}
\caption{The Keram-Ramu family}
\il{Keram-Ramu}
\il{Keram}
\il{West Keram}
\il{Ulmapo}
\il{Mwakai-Pondi}
\il{Mwakai}
\il{Pondi}
\il{East Keram}
\il{Ambakich}
\il{Ap Ma}
\il{Ramu}
\il{Waran}
\il{Watam}
\il{Bore}
\il{Kaian}
\il{Awar}
\il{Bosngun}
\il{Akukem}
\il{Kire}
\il{Mikarew}
\il{Abu}
\il{Gorovu}
\il{Igom}
\il{Tanggu}
\il{Kaje}
\il{Tanguat}
\il{Romkun}
\il{Breri}
\il{Kominimung}
\il{Igana}
\il{Inapang}
\il{Chini}
\il{Rao}
\il{Aram}
\il{Aren}
\label{fig:1.7}
%\includegraphics[width=\textwidth]{figures/Ulwa2022September01-img007.png}

\begin{forest}
forked edges,
for tree = {anchor=west,
            grow'=east,
            align = left,
            font=\small,
            inner sep = 0pt,
            s sep=3pt,
            },
[Keram-\\Ramu
  [Keram
    [West Keram\\(\ili{Ulmapo})
      [Mwakai-Pondi
        [Mwakai, tier=word]
        [Pondi, tier=word]
      ]
      [Ulwa, tier=word]
    ]
    [East Keram
      [Ambakich, tier=word]
      [Ap Ma, tier=word]
    ]
  ]
  [Ramu
    [Waran, tier=word]
    [Lower Ramu\\(Ruboni)
      [Ramu Coast\\(Ottilien)
        [North Ramu\\Coast
          [Watam, tier=word]
          [Bore, tier=word]
          [Kaian, tier=word]
        ]
        [East Ramu\\Coast
          [Awar, tier=word]
          [Bosngun, tier=word]
        ]
      ]
      [Ruboni Range\\(Misegian)
        [Akukem, tier=word]
        [Kire, tier=word]
        [Mikarew, tier=word]
      ]
    ]
    [Central Ramu
      [Porapora River\\(Agoan)
        [Abu, tier=word]
        [Gorovu, tier=word]
      ]
      [Moam River\\(Ataitan)
        [Igom-Tanggu
          [Igom, tier=word]
          [Tanggu, tier=word]
        ]
        [Kaje, tier=word]
        [Tanguat, tier=word]
      ]
      [Guam River\\(Tamolan)
        [Iski
          [West\\Iski
            [Romkun, tier=word]
            [Breri, tier=word]
          ]
          [East\\Iski
            [Kominimung, tier=word]
            [Igana, tier=word]
          ]
        ]
        [Inapang, tier=word]
        [Chini, tier=word]
      ]
    ]
    [South Ramu\\(Annaberg)
      [Rao, tier=word]
      [Aram-Arem\\(Aian)
        [Aram, tier=word]
        [Aren, tier=word]
      ]
    ]
  ]
]
\end{forest}
\end{figure}

The member languages in this \isi{classification}, along with their ISO codes, glottocodes, and alternative names, are presented in \tabref{tab:1.14}.


\begin{table}
\small
\caption{The Keram-Ramu languages}
\il{Mwakai}
\il{Mongol}
\il{Pondi}
\il{Langam}
\il{Ambakich}
\il{Aion}
\il{Ap Ma}
\il{Kambot}
\il{Botin}
\il{Waran}
\il{Banaro}
\il{Watam}
\il{Bore}
\il{Borei}
\il{Gamei}
\il{Kaian}
\il{Awar}
\il{Bosngun}
\il{Bosmun}
\il{Akukem}
\il{Sepen}
\il{Kire}
\il{Giri}
\il{Mikarew}
\il{Aruamu}
\il{Abu}
\il{Adjora}
\il{Gorovu}
\il{Igom}
\il{Kanggape}
\il{Tanggu}
\il{Tangu}
\il{Kaje}
\il{Andarum}
\il{Tanguat}
\il{Romkun}
\il{Breri}
\il{Kominimung}
\il{Igana}
\il{Inapang}
\il{Chini}
\il{Akrukay}
\il{Rao}
\il{Aram}
\il{Anor}
\il{Aren}
\il{Aiome}
\label{tab:1.14}
\begin{tabularx}{\textwidth}{llQ}
\lsptoprule
\il{Keram-Ramu}
ISO & glottocode & language names\\
\midrule
mgt & mong1344 & Mwakai, Mongol, Kaimba, Momgwo, Mwa\\
lnm & lang1328 & Pondi, Langam, Kaimba, Mwa\\
yla & yaul1241 & Ulwa, Ulwa (Papua New Guinea), Yaul, Yaulu, Dimiri, Dimili, Yaul-Dimiri, Unamama, Andjilowa\\
aew & amba1269 & Ambakich, Aion, No. 1 Porapora\\
kbx & apma1241 & Ap Ma, Kambot, Kambót, Botin, Keram, Kambaramba, Unter-Keram\\
byz & bana1292 & Waran, Banaro, Bánaro, Bakaram, Ober-Keram\\
wax & wata1253 & Watam, Watám, Marangis, Mugdyana Kam\\
gai & bore1247 & Bore, Borei, Mbore, Boroi, Gamai, Gamay, Gamei, \mbox{Mborena Kam}\\
kct & kaia1245 & Kaian, Kayan\\
aya & awar1249 & Awar, Nubia\\
bqs & bosn1248 & Bosngun, Bosman, Bosmun, Mbiermba\\
spm & sepe1240 & Akukem, Akukum, Sepen\\
geb & kire1240 & Kire, Gire, Giri, Kire-Puire\\
msy & arua1260 & Mikarew, Aruamu, Ariawia, Makarub, Makarup, Gumasi, Mikarew-Ariaw\\
ado & abuu1241 & Abu, Adjora, Adjoria, Porapora \#2, Gwɨn\\
grq & goro1261 & Gorovu, Crovu, Yerany, Yerani\\
igm & kang1291 & Igom, Kanggape\\
tgu & tang1355 & Tanggu, Tanggum, Tangu\\
aod & anda1284 & Kaje, Andarum\\
tbs & tang1356 & Tanguat\\
rmk & romk1240 & Romkun, Romkuin, Romakun, Iski\\
brq & brer1240 & Breri, Kwanga, Kuanga, Iski\\
xoi & komi1271 & Kominimung, Komunimung, Iski\\
igg & igan1243 & Igana, Wasmuk, Iski\\
mzu & inap1241 & Inapang, Midsivindi, Itutang, Yigaves, Yigavesakama, Itutang-Inapang\\
afi & akru1241 & Chini, Akrukay, Akrukai, Akruray\\
rao & raoo1244 & Rao, Annaberg, Annanberg, Anaberg, Ndelo\\
anj & anor1241 & Aram, Anor, Atemble, Atemple\\
aki & aiom1240 & Aren, Aran, Aiome, Aomie\\
\lspbottomrule
\end{tabularx}
\end{table}

\subsection{{Evidence} {for} {broader} {genealogical} {affiliations?}}\label{sec:1.7.5}

Following \citegen{Z’graggen1971} \isi{classification} of the \ili{Ramu Phylum}, \citet{Laycock1973} posited many broader genealogical relationship to \ili{Ramu} in his so-called “\ili{Sepik-Ramu Phylum}” that included connections not only to the \ili{Keram} languages, but also to the \ili{Sepik} family (including \ili{Ndu}), the \ili{Lower Sepik} family, the (ill-defined) \ili{Leonhard Schultze} family, the \ili{Yuat} family, the \ili{Upper Yuat} family, and the possible isolates \ili{Biksi} (also known as \ili{Yetfa}) and \ili{Gapun} (also known as \ili{Tayap} or \ili{Taiap}) (\citealt{LaycockZ’graggen1975}). However, the wider-reaching claims of Laycock’s massive \ili{Sepik-Ramu Phylum} have generally been dismissed \citep[24]{Ross2005}, and, at least since \citegen{Foley1986} more conservative \isi{classification} of \isi{Papuan} languages, this grouping has rarely if ever been invoked in its most expansive form.   

\newpage

More recently, however, \citet{Foley1999, Foley2000, Foley2005, Foley2018, Foley2020} has presented arguments for a special connection between the \ili{Lower Sepik} and \ili{Ramu} families.\footnote{In \citegen{Foley2018} \isi{classification}, the \ili{Keram} languages are subsumed within the \ili{Ramu} family.} The evidence supporting Foley’s proposal for a “\ili{Lower Sepik-Ramu} family” consists of:

\begin{quote}
\begin{enumerate}[noitemsep, label={(\roman*)}, align=left, widest=190, labelsep=1ex,leftmargin=*]
\item a small number of \isi{lexical} \isi{cognate}s;
\item \isi{cognate} pronominal forms or formatives; and
\item \isi{cognate} nominal \isi{number} \isi{morphology}.
\end{enumerate}
\end{quote}

However, (i) most of the five putative \isi{lexical} \isi{cognate}s are problematic, including one \ili{Austronesian} \isi{loanword} (‘lime’, \sectref{sec:1.5.6}), (ii) the rather limited resemblances among \isi{pronoun}s mostly boil down to single-segment “\isi{cognacy}” (e.g., the 1\textsc{pl} forms in \ili{Proto-Ramu} and \ili{Proto-Lower Sepik} both begin with *a-); and (iii) the seemingly elaborate \ili{Ramu} nominal \isi{number} system, which \citet[203--204]{Foley2018} suggests to be a vestige of an older \isi{noun class} system as found in the \ili{Lower Sepik} languages, can actually mostly be explained in terms of \isi{phonological} conditioning environments, and it reduces to just two reconstructible \isi{plural} \isi{suffix}es. Moreover, these two \isi{suffix}es do not exhibit any systematic \isi{cognacy} with \ili{Lower Sepik} \isi{suffix}es -- that is, although \ili{Proto-Keram-Ramu} *-i ‘\textsc{pl}’ and *-Vr ‘\textsc{pl}’ may resemble some of the dozen or more \isi{plural} \isi{suffix}es found in a \ili{Lower Sepik} language such as \ili{Yimas} (cf. \citealt[167--168]{Foley1991}), there is no paradigmatic relationship between \isi{suffix}es and bases, as one would expect from a \isi{noun class} system.

\is{genealogical affiliation|)}
\is{genealogical classification|)}
\is{language classification|)}
\is{classification|)}

Therefore, while a deep genealogical connection between the neighboring \ili{Keram-Ramu} and \ili{Lower Sepik} language families may indeed exist, I have yet to see what I consider sufficient evidence to support a “\ili{Lower Sepik-Ramu} family”. Aside from this proposed family, I do not know of any hypotheses of distant genealogical relationships involving \ili{Keram-Ramu} that are current being put forward in the literature.

\section{Typological overview}\label{sec:1.8}

Before examining Ulwa’s grammatical features in greater detail, I provide here a general description of its \isi{phonology}, \isi{morphology}, and \isi{syntax}, in an attempt to place Ulwa’s grammar in a crosslinguistic context.

\subsection{Phonetics and phonology}\label{sec:1.8.1}

\is{phonetics|(}
\is{phonology|(}

Ulwa has a rather small \isi{consonant inventory}, consisting of just 13 \isi{consonant}s (\sectref{sec:2.1}). This figure may be compared to the average of 22.7 \isi{consonant}s in \linebreak \citegen{Maddieson2013a} sample of 562 languages, or to an average of 23.9 \isi{consonant}s among the 3,020 languages in PHOIBLE 2.0 (\citealt{MoranMcCloy2019}). However, Ulwa’s \isi{consonant inventory} is fairly typical for the languages of the region (cf. \cite[243]{Foley2018}). Ulwa’s \isi{vowel inventory}, composed of 6 \isi{monophthong}s (\sectref{sec:2.2}), is closer to the crosslinguistic average of “just fractionally below 6” \linebreak \citep{Maddieson2013b}. Ulwa thus has a “moderately low” \isi{consonant-to-vowel ratio} of 2.17 \citep{Maddieson2013c}. Including \isi{diphthong}s (\sectref{sec:2.2.2}), Ulwa has 8 or 9 \isi{vowel}s in total, which may be compared to the average of 10.3 among the 3,020 languages in PHOIBLE 2.0 (\citealt{MoranMcCloy2019}) (i.e., it is close to the global average). \largerpage

There is nothing particularly unusual about either the \isi{consonant inventory} or the \isi{vowel inventory}. The only notable \is{gap in phoneme inventory} gap in the \isi{consonant inventory} is found among \isi{palato-alveolar}s, since there exists a (\isi{prenasalized}) \isi{voiced} \isi{palato-alveolar} \isi{affricate} /ⁿdʒ/ without a \isi{voiceless} counterpart (i.e., no \textsuperscript{†}/tʃ/ or \textsuperscript{†}/ʃ/, \sectref{sec:2.1.3}). Otherwise, there are no unusual contrasts (or absences of common contrasts) among phonemes. Ulwa distinguishes \isi{stop}s (\isi{plosive}s) in three places of articulation: \isi{labial}, \isi{alveolar}, and \isi{velar} (\sectref{sec:2.1.1}, \sectref{sec:2.1.2}). In each place of articulation, there is a contrast in voicing. Somewhat less common (but not particularly unusual for the region), however, is the fact that the \isi{voiced} \isi{stop}s are all \isi{prenasalized} (\sectref{sec:2.1.2}). Thus, Ulwa’s version of the set of typologically common \isi{stop}s is manifested as: /p, t, k, ᵐb, ⁿd, ᵑɡ/. There is, however, no contrast in voicing among \isi{fricative}s. In fact, the only \isi{fricative} is the \isi{voiceless} \isi{alveolar} /s/ (\sectref{sec:2.1.6}). There are no \isi{uvular} \isi{consonant}s, nor are there \is{glottalization} glottalized \isi{consonant}s nor other \isi{consonant}s with secondary manners of articulation. There is one \isi{liquid} \isi{consonant}: a \isi{voiced} \isi{alveolar} \isi{lateral} \isi{approximant} /l/ (\sectref{sec:2.1.5}). There is no phonemic \isi{velar} \isi{nasal}, although this sound occurs \isi{phonetic}ally, both as part of the \isi{prenasalized} \isi{voiced} \isi{velar} \isi{stop} and when an underlying \isi{alveolar} \isi{nasal} precedes the \isi{voiceless} \isi{velar} \isi{stop} (\sectref{sec:2.1.2}). The \isi{vowel inventory} is likewise fairly typical, consisting of the five standard \isi{vowel}s plus the \is{high vowel} high \isi{central vowel} /ɨ/ (\sectref{sec:2.2.1}), which is also a common “sixth” \isi{vowel} of the region (\citealt[732]{LaycockZ’graggen1975}; \citealt[53]{Foley1986}). The two \isi{back vowel}s are \isi{rounded}; and the two \isi{front vowel}s are \isi{unrounded}. There are no phonemic \isi{nasal vowel}s.

Ulwa generally has a simple \isi{syllable structure}, but the \isi{phonotactics} of the language do permit structures as complex as CCVC (typically only when the CC cluster is composed of a \isi{velar}-plus-\isi{labial-velar} or a \isi{bilabial} \isi{stop}-plus-\isi{liquid}). However, \isi{consonant cluster}s are not common (\sectref{sec:2.3}). There is no phonemic \isi{tone} in Ulwa; nor is \isi{stress} phonemic (\sectref{sec:2.4}).

\is{phonology|)}
\is{phonetics|)}

\subsection{Morphology and word classes}\label{sec:1.8.2}

\is{morphology|(}
\is{word class|(}

Ulwa is a mostly \isi{analytic} (or \isi{isolating}) language, in that it has a relatively low morpheme-to-word ratio. There is not much \isi{inflection}al \isi{morphology} in the language, but some does exist: there are \isi{TAM} \isi{suffix}es on verbs (\sectref{sec:4.2}) and \isi{oblique marker}s on NPs (\sectref{sec:11.4.1}). Since these \isi{affix}es and \isi{clitic}s tend to express one grammatical feature each, Ulwa can be considered more \isi{agglutinative} than \isi{fusional}.

\is{detransitivizer}
\is{detransitivization}

Ulwa employs the \isi{morphological} process of \isi{suffix}ation, both on verbs and on \isi{noun phrase}s. The only known \isi{prefix} is a detransitivizing marker that affixes to verbs (\sectref{sec:13.8.2}). \isi{Object marker}s, while properly \isi{proclitic}s and not \isi{prefix}es, have a close \isi{phonological} affinity with their following host verbs (\sectref{sec:7.2}). Although almost exclusively \isi{suffix}es, some \isi{TAM} \isi{affix}es take forms resembling \isi{circumfix}es (\sectref{sec:4.3}). There are no known productive processes of \isi{infix}ation, \isi{stem modification}, \isi{suprasegmental modification}, or \isi{reduplication}. Some verbs have \isi{suppletive} forms for certain \isi{TAM} distinctions (\sectref{sec:4.3}). \isi{Derivational morphology} includes \isi{nominalizing} \isi{suffix}es that \is{derivation} derive nouns from verbs (\sectref{sec:3.2}). Verbs, in a sense, may be derived from other \isi{parts of speech} through the use of a \isi{copular enclitic} (\sectref{sec:10.3}).

There is little \isi{agreement} marking between \isi{head}s and \isi{dependent}s in Ulwa, but based on what does exist, Ulwa may be considered a \isi{dependent-marking} language: in a \isi{postpositional phrase}, a 3\textsc{sg} object (\isi{dependent}) takes a form that reflects its status as object (\sectref{sec:8.1}, \sectref{sec:9.3.1}); similarly, in \is{possession} possessive \isi{noun phrase}s, the \isi{possessor} (\isi{dependent}) argument can be marked as such by a \isi{suffix} (\sectref{sec:6.2}, \sectref{sec:9.1.5}). If, however, \isi{object marker}s are indeed undergoing a process whereby they are fusing to following verbs, and thereby becoming \isi{prefix}es, then clauses may be considered to be becoming \is{head-marking} head-marked (\sectref{sec:7.2}).

Nouns in Ulwa are not marked in any way for \isi{person}, \isi{number}, \isi{gender}, or \isi{case} (\sectref{sec:3.1}). Subject and object NPs do, however, receive \isi{subject marker}s and \isi{object marker}s -- these are \isi{determiner}s used with third person referents that index \isi{number}: \isi{singular}, \isi{dual}, or \isi{plural} (\sectref{sec:7.1}, \sectref{sec:7.2}). Also, \isi{non-core NP}s can be indicated by an \is{oblique marker} oblique-marker \isi{enclitic} (\sectref{sec:11.4.1}). \isi{Possession} is generally indicated by a separate possessive word, but it can alternatively be signaled by an \is{oblique marker} oblique-marking \isi{enclitic}, or by simple placement, without marking, of the \isi{possessor} immediately before the \isi{possessum} (\sectref{sec:9.1.5}). There are no obligatorily \isi{possessed} nouns (i.e., no special treatment of \isi{inalienable possession}).

The basic paradigm of \isi{personal pronoun}s consists of 11 items (\sectref{sec:6.1}). There is a three-way \isi{number} distinction among \isi{singular}, \isi{dual}, and \isi{plural} forms (in which the category of “\isi{plural}” can, in broader usage, refer to exactly two referents as well as to more than two, \sectref{sec:9.1.2}). Among first \isi{person} \isi{non-singular} \isi{pronoun}s there is a distinction between \isi{inclusive} and \isi{exclusive} forms. \isi{Gender} is not marked in any way in \isi{pronoun}s, nor are there any \isi{politeness} distinctions made among \isi{pronoun}s. There is \isi{colexification} between \isi{reflexive} and \isi{reciprocal} \isi{pronoun}s (\sectref{sec:6.3}), as well as between \is{indefinite pronoun} indefinite and \isi{interrogative pronoun}s (\sectref{sec:6.4}, \sectref{sec:6.5}), and between the \isi{interrogative word} ‘which?’ and the \isi{negator} ‘no, not’ (\sectref{sec:8.2.3}).

\isi{Determiner}s are largely represented by \isi{subject marker}s and \isi{object marker}s, which are free lexemes that follow their respective NPs, marking them for \isi{number} (\sectref{sec:7.1}, \sectref{sec:7.2}). They are not obligatory, nor do they necessarily mark NPs for \isi{definiteness}. There are also a few \isi{demonstrative}s that serve \isi{deictic} function (\sectref{sec:7.3}). There are no \is{classifier} \isi{demonstrative classifier}s, \isi{numeral classifier}s, \isi{possessive classifier}s, or \isi{verbal classifier}s.

  Verbs are marked for a few \isi{aspect} and \isi{mood} distinctions by \isi{suffix}es. There is a basic three-way contrast among \isi{imperfective} (often unmarked), \isi{perfective}, and \isi{irrealis} forms (\sectref{sec:4.2}). There is generally no grammatical \isi{evidentiality}, but \isi{epistemic} possibility can be expressed with a \isi{speculative} \isi{suffix} (\sectref{sec:4.11}). There is also a \isi{conditional} \isi{suffix} that marks the verb in the \isi{protasis} of a \isi{conditional} statement (\sectref{sec:4.12}).

There are no (non-\isi{borrow}ed) \isi{coordinator}s (\sectref{sec:12.1}) and no (obligatory) \isi{subordinator}s in Ulwa (\sectref{sec:12.2}). There is, however, a verbal \isi{suffix} that signals that a given clause is \isi{dependent} (\sectref{sec:12.2.1}). This \isi{suffix} anticipates a following clause, which may be either the \isi{independent clause} of the sentence or another \isi{dependent clause}. There is no \isi{morphological}ly or \isi{phonological}ly defined class of \isi{ideophone}s.

\is{word class|)}
\is{morphology|)}


\subsection{Word order and syntax}\label{sec:1.8.3}

\is{word order|(}
\is{syntax|(}

\is{non-verbal negation}
\is{non-verbal predication}

The \is{basic word order} basic order of basic constituents in Ulwa is SOV (\sectref{sec:11.1}). This order is fairly rigid: there is essentially no variation from this pattern in \isi{active-voice} \isi{main clause}s. \isi{Oblique} \isi{phrase}s follow the subject of the clause and precede the verb (and object if there is one) (i.e., SXOV) (\sectref{sec:11.4}). \isi{Negator}s occur between subjects and objects as well (S-NEG-O-V) (\sectref{sec:13.3.1}). Non-verbal \isi{predicate} \isi{negation} often employs \isi{discontinuous negator}s, one occurring between the subject and the verb, the other occurring clause-finally (\sectref{sec:13.3.2}). \isi{Adposition}s always follow their NPs -- that is, there are only \isi{postposition}s, no \isi{preposition}s (\sectref{sec:8.1}). In \is{possession} possessive constructions, the \isi{possessor} (\isi{genitive}) precedes the \isi{possessum} (\isi{possessed}) (\sectref{sec:9.1.5}). \isi{Adjective}s (or property-denoting words) follow the nouns that they modify (\sectref{sec:5.1}). \isi{Demonstrative}s (\sectref{sec:7.3}) and \isi{numeral}s (\sectref{sec:7.5}) also follow nouns. \isi{Relative clause}s precede their respective \isi{head noun}s (\sectref{sec:12.3}). Ulwa thus conforms very neatly to the typological expectations of \isi{OV language}s.

\is{nominative-accusative alignment}
\is{accusative alignment}

Ulwa has nominative-accusative \isi{morphosyntactic alignment} (\sectref{sec:11.2}). There are no indications of \isi{ergativity}, whether \isi{morphological} or \is{syntactic ergativity} \isi{syntactic}, in any aspect of the grammar. Although it is useful to differentiate \isi{intransitive} and \isi{transitive} verbs in Ulwa, there is no evidence that any verb is \isi{ditransitive} -- that is, a verb may never have more than two \isi{core argument}s: a subject and a \is{direct object} (direct) object (\sectref{sec:11.3}). There is also only minimal evidence to suggest that some verbs may be \isi{ambitransitive} (\sectref{sec:13.8.1}).

\is{discontinuous phrase}

Ulwa does not employ a robust set of \isi{serial verb construction}s, although a modest and restricted form of such constructions may exist (\sectref{sec:11.3}). There are, however, a number of \isi{discontinuous} verbs (or \isi{separable verb}s) in the language, which contain at least one \isi{light verb} element, and which function much like \isi{adjunct}-plus-verb constructions (\sectref{sec:9.2.1}).

Ulwa does not exhibit any particular \isi{comparative} or \isi{superlative} construction, relying instead exclusively on \isi{positive adjective}s.

Both nouns and \isi{adjective}s can function as \isi{predicate complement}s, either with a \isi{zero copula} (\sectref{sec:10.2}) or with a \isi{copular enclitic}, which may take additional marking for (\isi{irrealis}) \isi{mood} (\sectref{sec:10.3}), or with a \isi{suppletive} (\isi{past}-\isi{tense}) \isi{locative verb} form (\sectref{sec:10.4}).

\is{wh- question}
\is{wh-movement}

\is{yes/no question}

\isi{Question}s are formed simply by applying a rising \isi{intonation} to a \isi{declarative} statement. \isi{Polar question}s (‘yes/no’ questions) generally employ no \is{particle} \isi{question particle}, nor do they involve any \isi{inverted word order} (\sectref{sec:13.1.1}); \isi{content question}s (\textit{wh-} questions) contain their \isi{question word} in the \isi{syntactic} slot to be expected from the standard SOV order of \isi{declarative} sentences -- that is, there is no \textit{wh}-movement (\sectref{sec:13.1.2}).

Ulwa may be considered a \isi{pro-drop} language, in that subjects can be omitted from clauses without resulting in ungrammaticality (\sectref{sec:11.1}).

\isi{Passive}s are formed in a very unusual, \isi{syntactic} way: instead of relying on \isi{verbal morphology} to promote the more \isi{patient}ive argument of a \isi{transitive} verb to the grammatical subject of a clause, Ulwa \is{inverted word order} inverts the \isi{word order} to achieve this effect (VS instead of SV) (\sectref{sec:13.7}). The more \isi{agent}ive argument of this \isi{passive} sentence may be encoded as an \isi{oblique} \isi{phrase}, preceding the verb (i.e., XVS). There is also a verbal \isi{prefix} that functions to \is{valency reduction} reduce the \isi{valency} (or \isi{transitivity}) of a verb and may, in a sense, be considered a means of forming \isi{antipassive} constructions (\sectref{sec:13.8.2}, \sectref{sec:13.8.9}). There is no productive process of \isi{noun incorporation} in the language.

\isi{Causative}s in Ulwa are \isi{periphrastic}, always composed of two clauses (\sectref{sec:13.9.1}). They are of the \is{sequential causative} sequential, not the \is{purposive causative} purposive variety -- that is, the two clauses are \isi{juxtaposed} without any linking element: first the clause of the cause and second the clause of the effect. There are no \isi{morphosyntactic} \isi{applicative} constructions in Ulwa.

Only subjects are accessible to \isi{relativization} (\sectref{sec:12.3}). \isi{Relative clause}s in Ulwa immediately precede the \isi{head noun}s they modify, leaving a \is{gap strategy} “gap” in the \isi{relative clause} where the shared nominal element would be expected.

\is{syntax|)}
\is{word order|)}
