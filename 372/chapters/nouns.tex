\chapter{Nouns}\label{sec:3}

\is{noun|(}

This chapter provides a description of the \isi{morphosyntactic} attributes of nouns in Ulwa. Nouns comprise a large, open class of words. There is no canonical syllabic structure peculiar to nouns beyond the general patterns found in the language (\sectref{sec:2.3}). Nouns can vary in length from being \isi{monosyllabic} (even monophonemic) to being triysllabic or even longer, although it is not common for underived nouns to have more than two \isi{syllable}s. Indeed, many longer nouns appear to \isi{derive} from \isi{compound}ing, \isi{onomatopoeia}, or \isi{loan}s. Some \isi{monosyllabic} nouns are provided in \REF{ex:nouns:1}, some \isi{disyllabic} nouns in \REF{ex:nouns:2}, and some \isi{trisyllabic} nouns in \REF{ex:nouns:3}.

\ea%1
    \label{ex:nouns:1}
            \isi{Monosyllabic} nouns\\
\begin{tabbing}
{(\textit{im})} \= {(‘loincloth’)}\kill
{\textit{al}} \> {‘loincloth’}\\
{\textit{i}} \> {‘hand’}\\
{\textit{im}} \> {‘tree’}\\
{\textit{ka}} \> {‘peak’}\\
{\textit{na}} \> {‘talk’}
\end{tabbing}
\z

\ea%2
    \label{ex:nouns:2}
            \isi{Disyllabic} nouns\\
\begin{tabbing}
{(\textit{longom})} \= {(‘type of basket’)}\kill
{\textit{ame}} \> {‘type of basket’}\\
{\textit{ina}} \> {‘liver’}\\
{\textit{longom}} \> {‘dream’}\\
{\textit{mana}} \> {‘spear’}\\
{\textit{nonal}} \> {‘wind’}
\end{tabbing}
\z

\ea%3
    \label{ex:nouns:3}
            Trisyllabic nouns\\
\begin{tabbing}
{(\textit{nambana})} \= {(‘ancestral spirit’)}\kill
{\textit{itïtïl}} \> {‘dust’}\\
{\textit{iwanal}} \> {‘ant species’}\\
{\textit{nambana}} \> {‘ancestral spirit’}\\
{\textit{tambeta}} \> {‘chest’}
\end{tabbing}
\z

There are not many nouns longer than three \isi{syllable}s. Those that do exist seem mainly to refer to particular flora or fauna \REF{ex:nouns:4}.

\ea%4
    \label{ex:nouns:4}
            Nouns of four \isi{syllable}s\\
\begin{tabbing}
{(\textit{sambulumbu})} \= {(‘banana species’)}\kill
{\textit{popotala}} \> {‘frog species’}\\
{\textit{sambulumbu}} \> {‘insect species’}\\
{\textit{supangasa}} \> {‘banana species’}
\end{tabbing}
\z

  Nouns can be defined by their distribution. Only nouns (or \isi{pronoun}s) can serve as subjects or objects of verbs. In practical terms, this means that the first word in a basic \isi{indicative} sentence with overtly expressed arguments will be either a noun or a \isi{pronoun}, since the unmarked SOV \isi{word order} demands a sentence-initial subject and only nouns and \isi{pronoun}s can serve that role.\footnote{However, see \sectref{sec:11.1} on argument omission; \sectref{sec:5.3} on the nominal uses of \isi{adjective}s; and \sectref{sec:8.2} on \isi{adverb}s, whose freer \isi{word order} allows sentence-initial placement.} In examples \REF{ex:nouns:5} and \REF{ex:nouns:6}, both \isi{intransitive} clauses, the noun is the first element in the sentence.

\ea%5
    \label{ex:nouns:5}
            \textbf{\textit{Alum}} \textit{mï se.}\\
    \gll \textbf{alum}  mï      sa-e\\
    child  3\textsc{sg.subj}  cry-\textsc{ipfv}\\
\glt `The baby is crying.’ [elicited]
\z

 \ea%6
    \label{ex:nouns:6}
           \textbf{\textit{Ankam}} \textit{mï nip.}\\
    \gll \textbf{ankam}  mï      ni-p\\
    person  3\textsc{sg.subj}  die-\textsc{pfv}\\
\glt ‘The person died.’ [elicited]
\z
  
  In \isi{transitive} clauses that have overtly expressed objects, there is also typically a noun preceding the verb (again, in accordance with the demands of SOV \isi{word order}). Examples \REF{ex:nouns:7} and \REF{ex:nouns:8} are both \isi{transitive} clauses. In each, there is one noun that is at the beginning of the clause (the subject) and one noun that precedes the verb (the object).

\ea%7
    \label{ex:nouns:7}
            \textbf{\textit{Tïn}} \textit{ndï} \textbf{\textit{mïnda}} \textit{ndamap.}\\
    \gll \textbf{tïn}    ndï  \textbf{mïnda}  ndï=ama-p\\
    dog  3\textsc{pl}  banana  3\textsc{pl=}eat-\textsc{pfv}\\
\glt `The dogs ate the bananas.’ [elicited]
\z

\ea%8
    \label{ex:nouns:8}
            \textbf{\textit{Sina}} \textit{mï} \textbf{\textit{lam}} \textit{maweyunda.}\\
    \gll \textbf{sina}  mï      \textbf{lam}  ma=we-u-nda\\
    knife  \textsc{3sg.subj}  meat  \textsc{3sg.obj}=cut-put-\textsc{irr}\\
\glt `The knife will cut the meat.’ [elicited]
\z

Nouns may also be objects within \isi{postpositional phrase}s, such as the second noun, \textit{itom} ‘father’, in \REF{ex:nouns:9}.

\ea%9
    \label{ex:nouns:9}
            \textbf{\textit{I}} \textit{anma mï keka} \textbf{\textit{itom}} \textit{alol i.}\\
    \gll \textbf{\textit{i}} \textit{anma}  \textit{mï}       \textit{keka} \textbf{\textit{itom}} \textit{ala=ul} \textit{i}\\
    way  good  3\textsc{sg.subj}  completely  father  \textsc{pl.dist}=with go.\textsc{pfv}\\
\glt `Good behavior has completely gone with [our] fathers.’ [ulwa032\_46:16]
\z

Nouns may also occur as the subjects of \isi{non-verbal clause}s, such as the noun \textit{na} ‘talk’ in \REF{ex:nouns:10}.

\ea%10
    \label{ex:nouns:10}
            \textbf{\textit{Na}} \textit{ndï tïngïnpe.}\\
\gll  \textbf{na}    ndï  tïngïn=p-e\\
    talk  3\textsc{pl}  many=\textsc{cop-dep}\\
\glt `There are many arguments.’ [ulwa037\_40:19]
\z

  Nouns, moreover, always precede \isi{subject marker}s or \isi{object marker}s, when present (\sectref{sec:7.1}, \sectref{sec:7.2}), although not always immediately: longer \isi{noun phrase}s, with postnominal modifiers, may contain words intervening between the noun and its \isi{person}/\isi{number} marker. Only nouns may be modified by \isi{adjective}s, which almost always follow the noun (\sectref{sec:5.1}). Nouns may immediately follow \isi{possessive pronoun}s (\sectref{sec:6.2}).
  
  Nouns are not \isi{inflect}ed for \isi{number} or \isi{gender}. They may precede the \isi{oblique marker} \textit{=n} ‘\textsc{obl}’, however, which may be thought of as filling certain \isi{semantic} \isi{case} functions (\sectref{sec:11.4.2}). There is no productive diminutive or augmentative marking on nouns.

Nouns may refer to humans, as in \REF{ex:nouns:5}, \REF{ex:nouns:6}, and \REF{ex:nouns:9}; to non-human \isi{animate}s, as in \REF{ex:nouns:7}; or to \isi{inanimate} objects, as in \REF{ex:nouns:8}. They may also refer to abstract concepts, as in  \REF{ex:nouns:9} and \REF{ex:nouns:10}. Example \REF{ex:nouns:11} illustrates the use of a \isi{proper noun} that signifies a \isi{location}.

\ea%11
    \label{ex:nouns:11}
            \textit{U} \textbf{\textit{Wopata}} \textit{may.}\\
\gll  u     \textbf{Wopata}   ma=i\\
    2\textsc{sg}  [place]    3\textsc{sg.obj}=go.\textsc{pfv}\\
\glt `You went to Wopata.’ [ulwa037\_01:36]
\z

  Although there is no nominal \isi{inflection} in Ulwa (\sectref{sec:3.1}), nouns that are derived from verbs bear \isi{derivation}al (i.e., \isi{nominalizing}) \isi{morphology} (\sectref{sec:3.2}), and multiple nouns can be joined together to form \isi{compound}s (\sectref{sec:3.3}). There does not seem to be any productive process of \isi{reduplication} in the language, but some nouns appear to be composed of reduplicative elements (\sectref{sec:3.4}).

\is{noun|)}

\section{Nominal inflection}\label{sec:3.1}

\is{nominal inflection|(}
\is{inflection|(}
\is{noun|(}

There is no nominal inflectional \isi{morphology} in Ulwa. Nouns are not marked in any way for \isi{gender}, \isi{number}, \isi{case}, or other grammatical attributes. However, some nouns have inherent \isi{gender}, semantically determined by the natural \isi{gender} of the referent, such as \textit{yeta} ‘man’ versus \textit{yena} ‘woman’ and \textit{atuma} ‘older brother’ versus \textit{atana} ‘older sister’. Also, \isi{number} can be signaled by postnominal \isi{subject marker}s (\sectref{sec:7.1}) or \isi{object marker}s (\sectref{sec:7.2}). Although there is no grammatical \isi{case}, an \isi{oblique marker}, which indicates that an argument is functioning as an \isi{adjunct}, can appear to affix to nouns. However, properly, this marker is an \isi{enclitic} that follows entire NPs (\sectref{sec:11.4.1}).

  Sentences \REF{ex:nouns:12} and \REF{ex:nouns:13} illustrate the lack of contrast in nouns according to \isi{grammatical relation} and according to \isi{number}. In \REF{ex:nouns:12}, \textit{uta} ‘bird’ is a grammatical subject and has a \isi{singular} referent (‘the bird’). In \REF{ex:nouns:13}, \textit{uta} ‘bird’ is a grammatical object and has a \isi{plural} referent (‘the birds’).

\ea%12
    \label{ex:nouns:12}
            \textbf{\textit{Uta}} \textit{mï im may.}\\
    \gll \textbf{uta}    mï      im    ma=i\\
    bird  \textsc{3sg.subj}  tree  \textsc{3sg.obj}=go.\textsc{pfv}\\
\glt `The bird flew to the tree.’ [elicited]
\z

\ea%13
    \label{ex:nouns:13}
            \textit{Itom mï} \textbf{\textit{uta}} \textit{nduwalinda.}\\
\gll  itom  mï      \textbf{uta}    ndï=wali-nda\\
    father  \textsc{3sg.subj}  bird  \textsc{3pl}=hit-\textsc{irr}\\
\glt `Father will shoot the birds.’ [elicited]
\z

This absence of \isi{number} and \isi{case} marking holds for nouns with human referents as well. In sentence \REF{ex:nouns:14}, \textit{ankam} ‘person’ is a grammatical object and has a \isi{singular} referent; in sentence \REF{ex:nouns:15}, on the other hand, \textit{ankam} ‘person’ is a grammatical subject and has a \isi{plural} referent.

\ea%14
    \label{ex:nouns:14}
            \textit{Nï lïmndï} \textbf{\textit{ankam}} \textit{ambi mala.}\\
\gll    nï    lïmndï  \textbf{ankam}  ambi  ma=ala\\
    \textsc{1sg}  eye    person  big    \textsc{3sg.obj}=see\\
\glt `I saw the big person.’ [elicited]
\z

\ea%15
    \label{ex:nouns:15}
          \textbf{\textit{Ankam}} \textit{ndï awal imba wondi anglalop.}\\
\gll    \textbf{ankam}  ndï  awal    imba  wondi    angla-lo-p\\
    person  \textsc{3pl}  yesterday  night  bandicoot  await-go-\textsc{pfv}\\
\glt `The people hunted bandicoot(s) last night.’ [elicited]
\z

It should be noted that optional \isi{subject marker}s and \isi{object marker}s can be used to indicate \isi{number} in \isi{noun phrase}s. When such a marker is not included, however, the noun is underspecified for number, as may be seen for the object \textit{wondi} ‘bandicoot’ in \REF{ex:nouns:15}, whose verb lacks an \isi{object-marker} \isi{enclitic} (\sectref{sec:7.2}).

\is{noun|)}
\is{inflection|)}
\is{nominal inflection|)}

\section{Derivational morphology: Nominalization}\label{sec:3.2}

\is{derivational morphology|(}
\is{derivation|(}
\is{nominalization|(}
\is{noun|(}

Nouns can be derived from verbs to denote the \isi{agent} of the action indicated by the verb. An \isi{agent noun} (\isi{nomen agentis}) bares the derivational \isi{suffix} \textit{\nobreakdash-en} ‘\textsc{nmlz}’.\footnote{This \isi{suffix} may perhaps be further analyzable as consisting of the \isi{polysemous} \isi{suffix} /-e/, which can either mark \isi{imperfective} \isi{aspect} (\sectref{sec:4.4}) or signal that a clause is \isi{dependent} (\sectref{sec:12.2.1}), followed by a derivational \isi{suffix} /-n/, which could itself be related to the \isi{oblique marker} of the same form (\sectref{sec:11.4}).} The nominalizing \isi{morphology} in most instances affixes to the end of the \isi{verb stem}, as in \REF{ex:nouns:16}, which may be compared to a sentence illustrating a conjugated form of the same verb \REF{ex:nouns:17}.

\ea%16
    \label{ex:nouns:16}
          \textit{Nï} \textbf{\textit{inden}}.\\
\gll nï    inda-\textbf{en}\\
    \textsc{1sg}  walk-\textsc{nmlz}\\
\glt `I am a walker.’ [elicited]
\z

\ea%17
    \label{ex:nouns:17}
          \textit{Nï indap.}\\
\gll    nï    inda-p\\
    \textsc{1sg}  walk-\textsc{pfv}\\
\glt `I walked.’ [elicited]
\z

The nominalized verb (functioning as a noun) may be followed by a \isi{subject marker} (or an \isi{object marker}). In \REF{ex:nouns:18} and \REF{ex:nouns:19}, the \isi{verb stem} is the \isi{suppletive} \isi{perfective} form of the verb \textit{ma-} ‘go’ -- that is, \textit{i} ‘go.\textsc{pfv}’. In \REF{ex:nouns:18} this form receives the \isi{singular} \isi{subject marker} \textit{mï} ‘\textsc{3sg.subj}’, whereas in \REF{ex:nouns:19} it receives the \isi{plural} \isi{subject marker} \textit{ndï} ‘\textsc{3pl}’.

\ea%18
    \label{ex:nouns:18}
          \textit{Laykos} \textbf{\textit{iyen}} \textit{mï mat ngata lanji anda.}\\
    \gll Laykos  i-\textbf{en}      mï      ma=ta      ngata    ala-nji       anda\\
    [place]  go.\textsc{pfv-nmlz}  \textsc{3sg.subj}  \textsc{3sg.obj=}say  grand    \textsc{pl.dist-poss}  \textsc{sg.dist}\\
\glt `The one who went to the Rai Coast said: “That belongs to those grandparents.”’ (Literally ‘the having-gone-to-the-Rai-Coast one’) (\textit{Laykos} < TP \textit{Raikos} ‘Rai Coast’) [ulwa014\_42:06]
\z

\ea%19
    \label{ex:nouns:19}

          \textit{Ndï mo tïnanga ngap} \textbf{\textit{iyen}} \textit{ndï.}\\
\gll    ndï  ma=u      tïnanga  nga=p       i-\textbf{en}      ndï\\
    3\textsc{pl}  3\textsc{sg.obj}=from  arise  \textsc{sg.prox}=be  go.\textsc{pfv-nmlz}  3\textsc{pl}\\
\glt `They were the ones who got up from there and came to this place.’ (Literally ‘the getting-up-from-there-and-coming-to-this-[place] ones’) [ulwa002\_02:43]
\z

As the translations of sentences \REF{ex:nouns:18} and \REF{ex:nouns:19} suggest, clauses with nominalized verb forms are in some ways akin to (subject) \isi{relative clause}s (\sectref{sec:12.3}).

  In example \REF{ex:nouns:20}, the nominalized verb form is followed by the \isi{demonstrative} \isi{determiner} \textit{anda} ‘that’.

\ea%20
    \label{ex:nouns:20}
          \textit{mundu nïpat} \textbf{\textit{amen}} \textit{anda}\\
\gll    mundu  nïpat  ama-\textbf{en}  anda\\
    food  giant  eat-\textsc{nmlz}  \textsc{sg.dist}\\
\glt `that glutton’ (Literally ‘that giant-food eater’) [ulwa037\_19:32]
\z

Nominalized \isi{transitive} verbs can maintain their objects. Example \REF{ex:nouns:24} illustrates a \isi{transitive} clause. An entire nominalized VP (object plus verb) of such a \isi{transitive} clause can function as the \isi{predicate nominative} \REF{ex:nouns:21}, subject \REF{ex:nouns:22}, or object \REF{ex:nouns:23} of a sentence.

\ea%24
    \label{ex:nouns:24}
          \textit{Nïnji nungol mï \textbf{apa ite}.}\\
\gll    nï-nji    nungol  mï      \textbf{apa}    \textbf{ita-e}\\
    \textsc{1sg-poss}  child  \textsc{3sg.subj}  house  build-\textsc{ipfv}\\
\glt `My son is building a house.’ [elicited]
\z

\ea%21
    \label{ex:nouns:21}
          \textit{Nïnji nungol mï \textbf{apa iten}}.\\
\gll nï-nji    nungol  mï      \textbf{apa}  \textbf{ita-en}\\
    \textsc{1sg-poss}  child  \textsc{3sg.subj}  house  build-\textsc{nmlz}\\
\glt `My son is a house builder.’ [elicited]
\z

\ea%22
    \label{ex:nouns:22}
          \textit{\textbf{Apa iten} mï nip.}\\
\gll    \textbf{apa}  \textbf{ita-en}      mï      ni-p\\
    house  build-\textsc{nmlz}  \textsc{3sg.subj}  die-\textsc{pfv}\\
\glt `The house builder died.’ [elicited]
\z

\newpage

\ea%23
    \label{ex:nouns:23}
          \textit{Nï lïmndï nipe \textbf{ap iten} mala.}\\
\gll    nï    lïmndï  ni-p-e      \textbf{apa}  \textbf{ita-en}      ma=ala\\
    \textsc{1sg}  eye    die-\textsc{pfv-dep}  house  build-\textsc{nmlz}  \textsc{3sg.obj}=see\\
\glt `I saw the dead house builder.’\footnote{The \isi{suffix} \textit{-e} ‘\textsc{dep}’ marks verbs in clauses in a \isi{dependent} relation to the \isi{matrix clause} (\sectref{sec:12.2}), here marking ‘die’ as the main verb of a \isi{relative clause}. There could thus be some relationship (whether diachronic or synchronic) between the /-e/ component of the \isi{nominalizing} \isi{suffix} and this \is{dependent marker} dependent-marking \isi{suffix} \textit{-e} ‘\textsc{dep}’. That is, \textit{apa iten mï} ‘the house builder’ could in effect be (or have evolved from being) a \isi{phrase} meaning something like ‘the one that builds houses’. Since the \isi{suffix} in question consists of more than just /e/, however, (that is, the form is /-en/) it is treated as something other than (or at least more than) a \isi{relativizer}. Furthermore, the \isi{nominalize}d forms that contain /-en/ behave in every way \isi{syntactic}ally as nominal elements, receiving subject marking (or \isi{object marking}), preceding \isi{adjective}s that modify them, and exhibiting all other distributional properties of nouns.} (Literally ‘I saw the having-died house builder.’) [elicited]
\z

  In sentence \REF{ex:nouns:25}, the \isi{nominalize}d verb is modified by an \isi{adjective}, \textit{wutota} ‘tall’, and the entire NP is marked as the subject of the clause by the \isi{subject marker} \textit{mï} ‘3\textsc{sg.subj}’.

\ea%25
    \label{ex:nouns:25}
          \textbf{\textit{Ulepawen}} \textit{wutota mï liyu.}\\
\gll    ulep-aw-\textbf{en}      wutota  mï      li-u\\
    jump-put.\textsc{ipfv-nmlz}  tall    \textsc{3sg.subj}  down-put\\
\glt `The tall jumper fell.’ [elicited]
\z

In sentence \REF{ex:nouns:26}, the nominalized verb is the \isi{head} of an NP that is serving as \isi{direct object} and receives the \isi{object marker} \textit{ma=} ‘3\textsc{sg.obj’}.

\ea%26
    \label{ex:nouns:26}
          \textit{Nï lïmndï mïnda} \textbf{\textit{amen}} \textit{wutota mala.}\\
\gll    nï    lïmndï  mïnda  ama-\textbf{en}  wutota  ma=ala\\
    1\textsc{sg}  eye    banana  eat-\textsc{nmlz}  tall    \textsc{3sg.obj}=see\\
\glt `I saw the tall banana-eater.’ [elicited]
\z

In example \REF{ex:nouns:27}, the nominalizer \isi{suffix} attaches to the \isi{locative verb} \textit{p-} ‘be at’, which here has a \isi{location} as its grammatical object (indexed with the \isi{object-marker} \isi{proclitic} \textit{ma=} ‘3\textsc{sg.obj’}).

\ea%27
    \label{ex:nouns:27}
          \textit{Li} \textbf{\textit{mapen}} \textit{ndï wopa wa i.}\\
\gll li    ma=p-\textbf{en}      ndï  wopa  wa    i\\
    down  3\textsc{sg.obj}=be\textsc{{}-nmlz} 3\textsc{pl}  all    village  go.\textsc{pfv}\\
\glt `The downstream people all came to the village.’ (Literally ‘the being-down-there ones’) [ulwa034\_00:08]
\z

As NPs, these \isi{phrase}s formed with nominalizations can also be \isi{possessed} -- that is, they can be modified by \is{possession} possessive words preceding them (\sectref{sec:6.2}, \sectref{sec:9.1}). Example \REF{ex:nouns:28} is a bit unusual in showing \isi{perfective} marking on the verb.

\ea%28
    \label{ex:nouns:28}
          \textit{manji inom} \textbf{\textit{mokotpen}}\\
\gll    ma-nji      inom  ma=kot-p-\textbf{en}\\
    3\textsc{sg.obj-poss}  mother  3\textsc{sg.obj=}break-\textsc{pfv}{}-\textsc{nmlz}\\
\glt `his biological mother’ (Literally ‘his having-borne-him mother’) [ulwa001\_05:37]
\z

Nominalized verb forms may be used to define or describe people’s characteristics or habits, as illustrated by \REF{ex:nouns:29}, \REF{ex:nouns:30}, and \REF{ex:nouns:31}.

\ea%29
    \label{ex:nouns:29}
          \textit{Nï ango ay nïpat} \textbf{\textit{amen}}.\\
\gll nï    ango  ay    nïpat  ama-\textbf{en}\\
    1\textsc{sg}  \textsc{neg}  sago  giant  eat-\textsc{nmlz}\\
\glt `I don’t eat a lot of sago.’ (Literally ‘I am a not-giant-sago eater.’) [ulwa014†]
\z

\ea%30
    \label{ex:nouns:30}
          \textit{Anambi aw} \textbf{\textit{amen}} \textit{alawa.}\\
\gll    an-ambi     aw       ama-\textbf{en}   ala{}-awa\\
    1\textsc{pl.excl-top}  betel.nut  eat-\textsc{nmlz}  \textsc{pl.dist-int}\\
\glt `As for us, we’re really betel nut chewers.’ [ulwa037\_51:45]
\z

\ea%31
    \label{ex:nouns:31}
          \textit{Nï wandam ngape} \textbf{\textit{wowen}}.\\
\gll nï    wandam  nga=p-e      wow{}-\textbf{en}\\
    1\textsc{sg}  jungle    \textsc{sg.prox=}be\textsc{{}-ipfv} sleep.\textsc{ipfv}{}-\textsc{nmlz}\\
\glt `I live in this jungle.’ (Literally ‘I am an in-this-jungle sleeper.’) [ulwa001\_03:24]
\z

Nominalized verb forms also commonly indicate \isi{habitual} action, as in \REF{ex:nouns:32}, \REF{ex:nouns:33}, and \REF{ex:nouns:34}.

\ea%32
    \label{ex:nouns:32}
          \textit{Mangusuwata ango niya} \textbf{\textit{mbunden}}.\\
\gll ma-ngusuwata  ango  nï=iya      mbï-unda-\textbf{en}\\
    3\textsc{sg.obj{}-}poor  \textsc{neg}  1\textsc{sg}=toward  here-go-\textsc{nmlz}\\
\glt `The poor thing doesn’t come around here to me [anymore].’ [ulwa032\_05:18]
\z

\is{noun|)}
\is{nominalization|)}
\is{derivation|)}
\is{derivational morphology|)}

\is{derivational morphology|(}
\is{derivation|(}
\is{nominalization|(}
\is{noun|(}

\ea%33
    \label{ex:nouns:33}
          \textit{Ala mundun amblol} \textbf{\textit{inden}}.\\
\gll ala       mundu=n  ambla=ul    inda-\textbf{en}\\
    \textsc{pl.dist}  food=\textsc{obl}  \textsc{pl.refl=}with  walk-\textsc{nmlz}\\
\glt `They were ones who walked around with food with one another.’ [ulwa014\_63:23]
\z

\ea%34
    \label{ex:nouns:34}
          \textit{An ango ndiya amba} \textbf{\textit{unden}}.\\
\gll an      ango  ndï=iya    amba       unda-\textbf{en}\\
    1\textsc{pl.excl}  \textsc{neg}  3\textsc{pl=}toward  mens.house  go-\textsc{nmlz}\\
\glt `We didn’t go to them in the men’s house.’ [ulwa018\_04:44]
\z

In example \REF{ex:nouns:35}, the speaker uses a \ili{Tok Pisin} \isi{aspect}ual marker \textit{save} ‘\textsc{hab}’ (literally ‘know’), which indicates \isi{habitual} action, along with a nominalizing \isi{suffix}. For more on the structural influences of \ili{Tok Pisin} on Ulwa, see \chapref{sec:15}.

\ea%35
    \label{ex:nouns:35}
          \textit{Nambi sawe anmoka ala} \textbf{\textit{namnapen}}.\\
\gll nï-ambi   sawe  anmoka  ala namna=p-\textbf{en}\\
    1\textsc{sg-top}  \textsc{hab}  snake    from  afraid=\textsc{cop}{}-\textsc{nmlz}\\
\glt `As for me, I am afraid of snakes.’ (\textit{sawe} < TP \textit{save} ‘know’; \isi{habitual} marker) [ulwa035\_00:03]
\z

While typically forming \is{agent noun} agentive nouns, the nominalizer \isi{suffix} \textit{-en} ‘\textsc{nmlz}’ may also be used to create a more \is{patient noun} \isi{patient}ive noun (\isi{nomen patientis}), which is derived not from the logical subject of the verb \REF{ex:nouns:36}, but rather from the \isi{direct object} \REF{ex:nouns:37}.

\ea%36
    \label{ex:nouns:36}
          \textit{apa} \textbf{\textit{mayten}} \textit{ankam mï}\\
\gll    apa    ma=ita-\textbf{en}        ankam  mï\\
    house  3\textsc{sg.obj}=build-\textsc{nmlz}  person  3\textsc{sg.subj}\\
\glt `the person who is building the house’ [elicited]
\z

\ea%37
    \label{ex:nouns:37}
          \textbf{\textit{iten}} \textit{apa mï}\\
\gll    \textbf{ita-en}      apa    mï\\
    build-\textsc{nmlz}  house  3\textsc{sg.subj}\\
\glt `the house that is being built’ [elicited]
\z

There can therefore at times be ambiguity, as in \REF{ex:nouns:38}.

\ea%38
    \label{ex:nouns:38}
          \textbf{\textit{iten}} \textit{mï}\\
\gll    ita-\textbf{en}      mï\\
    build-\textsc{nmlz}  3\textsc{sg.subj}\\
\glt    (a) ‘the one who is building’\\
    (b) ‘the one that is being built’ [elicited]
\z


Usually, however, the nature of the \isi{derivation} is clear from context. Further examples of \is{patient noun} \isi{patient}ive nominalizations are given in \REF{ex:nouns:39} and \REF{ex:nouns:40}.

\ea%39
    \label{ex:nouns:39}

          \textit{Ambawanam Ngata ankam ambi anda ankam ango lïmndï} \textbf{\textit{uten}} \textit{me.}\\

    \gll Ambawanam  Ngata  ankam  ambi  anda    ankam  ango  lïmndï  uta-\textbf{en}       me\\
    [name]      grand  person  big    \textsc{sg.dist }  person  \textsc{neg} eye    grind-\textsc{nmlz}  \textsc{neg}\\


\glt `Ambawanam Ngata is that big person, a person who has never been seen.’ (This could also mean ‘a person who   has never seen [something/anything]’.) [ulwa009\_00:00]
\z

\ea%40
    \label{ex:nouns:40}
          \textbf{\textit{Nungunupen}} \textit{ndï ngamana.}\\
\gll    nungun-u-p-\textbf{en}    ndï  nga=ma-na\\
    break-put-\textsc{pfv-nmlz}  3\textsc{pl}  \textsc{sg.prox}=go-\textsc{irr}\\
\glt `The broken ones will go here.’ [ulwa014\_68:51]
\z

\is{action}

Beyond \isi{agent}s and \isi{patient}s, the nominalizing \isi{suffix} is not known to derive other \isi{semantic} categories of nouns, such as \isi{instrument}s, \isi{location}s, \isi{action}s, or \isi{state}s.

  Although the nominalizing \isi{suffix} may serve certain \isi{aspect}ual functions, often it is difficult to discern the particular function of verbal ending with [\nobreakdash-en]. Moreover, speakers may sometimes employ a \is{paragoge} paragogic /n/ at the end of clauses, especially those marked by the \isi{dependent marker} \textit{-e} ‘\textsc{dep}’ \REF{ex:nouns:41} or the \isi{homophonous} \isi{imperfective} \isi{suffix} \textit{-e} ‘\textsc{ipfv}’ \REF{ex:nouns:42}.

\ea%41
    \label{ex:nouns:41}
          \textit{Ala yotnï mase mï} \textbf{\textit{nipen}}.\\
\gll ala      yot=nï      ma=asa-e      mï      ni-p-e-\textbf{n}\\
    \textsc{pl.dist}  machete=\textsc{obl}  3\textsc{sg.obj}=hit-\textsc{dep}  3\textsc{sg.subj}  die-\textsc{pfv-dep}{}-?\\
\glt `They hit him with a machete and he died.’ [ulwa039\_00:03]
\z

\ea%42
    \label{ex:nouns:42}
          \textit{Ndï ango kïkal} \textbf{\textit{nïwanen}}.\\
\gll ndï  ango  kïkal  nï=wana-e-\textbf{n}\\
    3\textsc{pl}  \textsc{neg}  ear    1\textsc{sg}=feel-\textsc{ipfv{}-?}\\
\glt `They weren’t listening to me.’ [ulwa037\_08:10]
\z

\is{noun|)}
\is{nominalization|)}
\is{derivation|)}
\is{derivational morphology|)}

\newpage

\section{Compound nouns}\label{sec:3.3}

\is{compound noun|(}
\is{compound|(}
\is{noun|(}

Although lacking \isi{inflect}ional \isi{morphology}, nouns can nevertheless be polymorphemic, provided that they are formed by combining two or more \isi{lexical} roots in a single \isi{compound} word. In most instances, such compounds are formed exclusively from nouns, although it is also possible for compounds to include non-nominal elements. Some compounds are readily analyzable as being composed of two distinct \isi{lexical} elements, whereas the sources of others are obscured somewhat by \isi{sound change}s, and still others contain at least one entirely obscure element. Each of the compounds in \REF{ex:nouns:43} is transparently composed of two nouns.\footnote{Some of these are writeen as multiple words (with a space between \isi{compound} members), whereas others are written as single \isi{orthographic} units. This decision is not always easy. When a \isi{phonological change} (especially an irregular or strictly historical one) has obscured one or more elements of the \isi{compound}, then it is written as one word. When the complete \isi{phonological} integrity of all the \isi{compound} members is maintained, however, then the \isi{compound} members may be written with spaces between them. Complications arise, however, when regular \isi{phonological} processes occur where two \isi{compound} members meet: this is especially common when one member ends with a \isi{vowel} and the following member begins with a \isi{vowel}. In some cases, I have taken speaker preferences into consideration when deciding how to write compounds.}

\ea%43
    \label{ex:nouns:43}
          Compound nouns\\
          \begin{tabular}[t]{lll}
    \textit{apa ini}   &  ‘floor’ & < \textit{apa} ‘house’ + \textit{ini} ‘ground’\\

    \textit{apaka}   &   ‘roof’ & < \textit{apa} ‘house’ + \textit{ka} ‘peak’\\

    \textit{asiyot}  &    ‘grass knife’ & < \textit{asi} ‘grass’ + \textit{yot} ‘machete, knife’\\

    \textit{im nambï} &  ‘bark’ & < \textit{im} ‘tree’ + \textit{nambï} ‘skin’\\

    \textit{im nangïn} &   ‘branch’ & < \textit{im} ‘tree’ + \textit{nangïn} ‘tongs’\\

    \textit{inimnji}  &    ‘dew’ & < \textit{inim} ‘water’ + \textit{nji} ‘thing’\\

 \textit{lïmndï inim} & ‘tear’ & < \textit{lïmndï} ‘eye’ + \textit{inim} ‘water’\\

    \textit{nil nopa} &   ‘beard’ & < \textit{nil ‘}body hair’ + \textit{nopa} ‘cheek’\\

 \textit{unduwan apïn} & ‘headache’ & < \textit{unduwan} ‘head’ + \textit{apïn} ‘fire’\\

 \textit{wala uta}  &  ‘bat’ & < \textit{wala} ‘rat species’ + \textit{uta} ‘bird’\\

 \textit{won inim}  &  ‘semen’ & < \textit{won} ‘penis’ + \textit{inim} ‘water’\\

 \textit{wutïmu}  &    ‘toe’ & < \textit{wutï} ‘leg, foot’ + \textit{mu} ‘fruit, seed’\\

    \textit{wutï yombam} & ‘sole’ & < \textit{wutï} ‘leg, foot’ + \textit{yombam} ‘palm’\\

 \textit{ya inim} &  ‘coconut milk’ & < \textit{ya} ‘coconut’ + \textit{inim} ‘water’\\

    \textit{yawe}  &    ‘sago fried with & < \textit{ya} ‘coconut’ + \textit{we} ‘sago’\\
    & coconut’ &\\
\end{tabular}
\z

Most of these are compositional, \isi{endocentric compound}s. For example, a ‘roof’ is the ‘peak of a house’, ‘bark’ is the ‘skin (i.e., outside covering) of a tree’, and a ‘grass knife’ is a ‘knife for (cutting) grass’. Although the \isi{head} is almost always the second element, it is also possible for the \isi{head} to come first, as in \textit{nil nopa} ‘beard’, in which \textit{nil} ‘body hair’, precedes \textit{nopa} ‘cheek’. While many compounds are completely literal, some contain a more \isi{metaphor}ical element. For example, a ‘branch’ is the ‘tongs of a tree’, a ‘headache’ is ‘fire of the head’, and a ‘toe’ is the ‘fruit of the foot’. Also, not all compounds are strictly endocentric. The word \textit{yawe} ‘sago pancake cooked with coconut’ is a \isi{copulative compound}, and the word for ‘bat’ is an \isi{exocentric compound}.\footnote{That is, unless \textit{uta} in the Ulwa taxonomy means ‘flying non-insect animal’ rather than ‘bird’, in which case the word for ‘bat’ is a regular \isi{endocentric compound}, with the second element serving as the \isi{head}.}

  Further compounding is possible – that is, \isi{compound} nouns may consist of more than two \isi{lexical} elements, as in \REF{ex:nouns:43a}

\ea%43a
    \label{ex:nouns:43a}
          Compound nouns containing three members\\
          \begin{tabular}[t]{lll}
    \textit{imu unduwan}   &  ‘thumb’& {< \textit{i} ‘hand, arm’ + \textit{mu} ‘fruit, seed’ +}\\
    &  & {\textit{unduwan} ‘head’}\\

    \textit{wutïmu unduwan}   &   ‘big toe’ & {< \textit{wutï} ‘leg, foot’ \textit{mu} ‘fruit, seed’ +}\\
    & & {\textit{unduwan} ‘head’}\\
\end{tabular}
\z

  Compounds can be used productively to coin words for novel things, such as introduced foods. The word for ‘rice’, for example, is \textit{asimu}, derived from \textit{asi} ‘grass’ plus \textit{mu} ‘fruit, seed’ -- that is, ‘seed of grass’ (\sectref{sec:14.9}).

  Sometimes, although a \isi{compound} may be easily analyzable into two discrete \isi{lexical} elements, the \isi{semantic} derivation is nevertheless obscure. That is, it is not always clear how the meanings of two component morphemes interact to produce the resultant \isi{exocentric compound}, as in the examples given in \REF{ex:nouns:44}.

\ea%44
    \label{ex:nouns:44}
          Semantically obscure nominal compounds\\
\begin{tabbing}
{(\textit{nipum amba})} \= {(‘veranda, awning’)} \= {(<)} \= {(\textit{nipum} ‘\textit{kunai}’ + \textit{amba} ‘men’s house’)}\kill
{\textit{apa imot}} \> {‘veranda, awning’} \> {<} \> {\textit{apa} ‘house’ + \textit{imot} ‘log’}\\
{\textit{nipum amba}} \> {‘grassland’} \> {<} \> {\textit{nipum} ‘\textit{kunai}’ + \textit{amba} ‘men’s house’}
\end{tabbing}
    \z

Some compounds, however, have undergone historical \isi{sound change}s that have altered the shape of one or both constituent lexemes, as in the compounds given in \REF{ex:nouns:45}.

\newpage

\ea%45
    \label{ex:nouns:45}
          Nominal compounds exhibiting sound change\\
\begin{tabbing}
\is{sound change}
{(\textit{sinananangïn})} \= {(‘middle of house’)} \= {(<)} \= {(textit{wandam} ‘garden’ + \textit{wapata} ‘old, dry’)}\kill
{\textit{apep}} \> {‘front of house’} \> {<} \> {\textit{apa} ‘house’ + \textit{ip} ‘nose’}\\
{\textit{apïnsi}} \> {‘ashes’} \> {<} \>  {\textit{apïn} ‘fire’ + \textit{isi} ‘ashes, salt’}\\
{\textit{apombam}} \> {‘middle of house’} \> {<} \> {\textit{apa} ‘house’ + \textit{wombam} ‘middle’}\\
{\textit{sinananangïn}} \> {‘claw’} \> {<} \> {\textit{sinanan} ‘nail’ + \textit{nangïn} ‘tongs’}\\
{\textit{tïlwa}} \> {‘road, path’} \> {<} \> {\textit{utï} ‘foot’ + \textit{luwa} ‘place’}\\
{\textit{wandapata}} \> {‘fallow garden’} \> {<} \> {\textit{wandam} ‘garden’ + \textit{wapata} ‘old, dry’}
\end{tabbing}
    \z

The first word in \REF{ex:nouns:45}, \textit{apep} ‘front of house’ is the product of a still productive \isi{phonological} process of coalescence, which may optionally yield [e] from /a\#i/ (cf. the similar coalescence of /i\#a/ to [e], \sectref{sec:2.7}). The word \textit{sinananangïn} ‘claw’ has likewise undergone only a minor change: the \isi{degemination} of consecutive \isi{consonant}s (\sectref{sec:2.5.8}). The other words listed in \REF{ex:nouns:45}, however, have undergone more drastic changes – that is, changes not apparently motivated by any regular \isi{phonological} rules of the language.\footnote{Some of these changes may reflect the sort of \isi{phonological} reductions common among high-frequency \isi{lexical} items -- that is, the case could be made that such compounds have more fully \isi{lexical}ized than others. The word \textit{wandapata} ‘fallow garden’ has lost both the final /m/ of \textit{wandam} ‘jungle, garden’ and the initial /wa/ of \textit{wapata} ‘old, dry’; and the word for \textit{apombam} ‘middle of house’ has lost both the final /a/ of \textit{apa} ‘house’ and the initial /w/ of \textit{wombam} ‘middle’. I assume that \textit{apïnsi} ‘ashes’ was coined to disambiguate *isi ‘ashes’ from \textit{isi} ‘salt’.}

  Although most compounds exhibit two (or more) nominal elements, there are also possibly examples of nominal compounds consisting of one non-nominal element \REF{ex:nouns:45a}.\footnote{Cf. also \textit{wandapata} ‘fallow garden’ in \REF{ex:nouns:45}.}
  
\ea%45a
    \label{ex:nouns:45a}
          Compound nouns possibly containing non-nominal members\\
\begin{tabbing}
{(limama)} \= {(‘woman, wife’)} \= {(<)} \= {(‘woman, wife’ + \textit{nu} ‘near’)}\kill
{\textit{limama}} \> {‘jaw’} \> {<} \> {\textit{li} ‘down’ + \textit{mama} ‘mouth’}\\
{\textit{yenanu}} \> {‘woman, wife’} \> {<} \> {\textit{yena} ‘woman, wife’ + \textit{nu} ‘near’}
\end{tabbing}
\z

However, examples such as these can be problematic. Thus, although \textit{li} ‘down’ may be considered an \isi{adverb}, it can also be used as a noun, meaning, among other things, ‘the downstream part of the village’. One complication in the form \textit{yenanu} ‘woman, wife’ is the fact that, synchronically, it is \isi{synonymous} with its putative \isi{compound} member \textit{yena} ‘woman, wife’ -- thus, both can mean either ‘woman’ or ‘wife’.\footnote{While it is possible that there was once a derivation of \textit{yenanu} (*‘wife’) from \textit{yena} (*‘woman’) plus \textit{nu} (‘near’), no \isi{semantic} distinction currently exists between the words (\sectref{sec:14.7}).}

In some cases -- either because the \isi{phonological change} has been too great or a \isi{lexical} item remains too obscure -- only one element of the \isi{compound} is identifiable or the \isi{semantic} derivation from two putative elements is unclear, as in \REF{ex:nouns:45b}. Some of these unknown forms could be \isi{loanword}s.

\is{noun|)}
\is{compound noun|)}
\is{compound|)}

\is{compound|(}
\is{compound noun|(}
\is{noun|(}

\ea%45b
    \label{ex:nouns:45b}
          Compound nouns including members of unknown meaning\\
\begin{tabbing}
{(\textit{kïkal indam})} \= {(‘temple (of the head)’)} \= {(<)} \= {(\textit{tumbu} ‘?’ + \textit{itïm} ‘trash’)}\kill
{\textit{im kal}} \> {‘sap’} \> {<} \> {\textit{im} ‘tree’ + \textit{kal} ‘?’}\\
{\textit{kat ambla}} \> {‘molar’} \> {<} \> {\textit{kat} ‘?’ + \textit{ambla} ‘tooth’}\\
{\textit{kïkal indam}} \> {‘temple (of the head)’} \> {<} \> {\textit{kïkal} ‘ear’ + \textit{indam} ‘?’}\\
{\textit{tumbu itïm}} \> {‘outhouse, toilet’} \> {<} \> {\textit{tumbu} ‘?’ + \textit{itïm} ‘trash’}
\end{tabbing}
\z

Sometimes the forms of both members in a \isi{compound} have known meanings, but it is not clear how those two meanings combine to form the meaning of the \isi{compound}. Perhaps the form of one of the \isi{compound} members has some other meaning (i.e., a \isi{homophone}) that is unknown to me; or perhaps the \isi{semantics} of the \isi{compound} are simply obscure \REF{ex:nouns:45c}.

\is{noun|)}
\is{compound noun|)}
\is{compound|)}

\ea%45c
    \label{ex:nouns:45c}
          Compound nouns with unclear derivations\\
    \begin{tabular}[t]{lll}
    \textit{aymoma}   &  ‘stick for stirring jellied sago’ & {< \textit{ay} ‘jellied sago’}\\
    & & {+ \textit{moma} ‘leaf species’}\\
  \textit{nipum amba}   &  ‘grassland’ & {< \textit{nipum} ‘sword grass’}\\
  & & {+ \textit{amba} ‘men’s house’}\\
\end{tabular}
\z

\section{Reduplication?}\label{sec:3.4}

\is{reduplication|(}
\is{noun|(}

There does not appear to be any productive \isi{morphological} process of \isi{reduplication} in Ulwa. There are, however, a number of nouns that -- at least \isi{phonological}ly -- appear to exhibit full \isi{reduplication}. If in fact any of these is derived from a single non-reduplicated \isi{lexical} root, then this history has been lost to time, as the presumed root of the seemingly reduplicated word is meaningless on its own. Examples of nouns with apparent \isi{reduplication} are given in \REF{ex:nouns:46}.

\ea%46
    \label{ex:nouns:46}
          Nouns with apparent \isi{reduplication}\\
\begin{tabbing}
{(\textit{masamasa})} \= {(‘red ant’ (assuming < *ngun-ngun))}\kill
{\textit{masamasa}} \> {‘tree species’}\\
{\textit{mbatmbat}} \> {‘tilapia’}\\
{\textit{metmet}} \> {‘swamp dwarf’}\\
{\textit{misimisi}} \> {‘story’}\\
{\textit{natnat}} \> {‘greens’}\\
{\textit{ngungun}} \> {‘red ant’ (assuming < *ngun-ngun)}
\end{tabbing}
\z

The noun \textit{mbinmbin} ‘grave’ apparently results from the \isi{reduplication} of a \isi{borrowing} from \ili{Ap Ma}, \textit{mbɨn} ‘land, ground’ (\sectref{sec:1.5.6}), perhaps modeled as a superficial \isi{calque} of sorts of \ili{Tok Pisin} \textit{matmat} ‘grave’ (itself an unanalyzable “\isi{reduplication}” in \ili{Tok Pisin}).

There are, on the other hand, a few nouns that appear to be decomposable into two morphemes each, one a duplicate of the other \REF{ex:nouns:46a}.

  \ea%46a
    \label{ex:nouns:46a}
          Nouns with possibly meaningful \isi{reduplication}\\
\begin{tabbing}
{(\textit{manjimanji})} \= {(‘maggot’)} \= {(<)} \= {(\textit{njimana} ‘housefly’ (?))}\kill
{\textit{wutïwutï}} \> {‘duck’} \> {<} \> {\textit{wutï} ‘leg, foot’ (?)}\\
{\textit{manjimanji}} \> {‘maggot’} \> {<} \> {\textit{njimana} ‘housefly’ (?)}
\end{tabbing}
\z
  
Given the salience of the duck’s waddle and the feet that accomplish it, it seems possible that its name was derived from the word for ‘foot’ (although it could just be a case of accidental \isi{homophony}). Likewise, it is possible that \textit{manjimanji} ‘maggot’ derives from \isi{reduplication} (and reduction) of \textit{njimana} ‘housefly’.\footnote{The form \textit{njimana} ‘housefly’ might itself be derivative, possibly stemming from \textit{nji} ‘thing’ plus \textit{ma-na} ‘go-\textsc{irr}’.} At any rate, \isi{reduplication} is certainly not a productive \isi{morphological} process in Ulwa.

Apparent \isi{reduplication} (and partial \isi{reduplication}) also occurs in a several \isi{onomatopoetic} words (\sectref{sec:14.3}), most notably in a number of terms for various frog and bird species \REF{ex:nouns:46b}.

    \ea%46b
    \label{ex:nouns:46b}
          \isi{Onomatopoetic} nouns with \isi{reduplication}\\
\begin{tabbing}
{(\textit{wandïwandï})} \= {(‘frog species’)}\kill
{\textit{wandïwandï}} \> {‘frog species’}\\
{\textit{kïlakïli}} \> {‘frog species’}\\
{\textit{popotala}} \> {‘frog species’}\\
{\textit{kulkul}} \> {‘bird species’}\\
{\textit{awalawa}} \> {‘bird species’}\\
{\textit{kukumali}} \> {‘bird species’}
\end{tabbing}
      
\z

Beyond the \isi{lexical class} of nouns, we find a few other possible example of (nonproductive) \isi{reduplication} \REF{ex:nouns:46c}.

      \ea%46c
    \label{ex:nouns:46c}
          Apparent \isi{reduplication} in words other than nouns\\
\begin{tabbing}
{(\textit{nipinpu-})} \= {‘one each, one by one’}\kill
{\textit{nini}} \> {‘two’}\\
{\textit{lele}} \> {‘three’}\\
{\textit{kekaka}} \> {‘one each, one by one’}\\
{\textit{nunu}} \> {‘every’}\\
{\textit{nipinpu-}} \> {‘die.\textsc{pl}’}
\end{tabbing}
  \z

\is{universal quantifier}
\is{quantifier}


The \isi{reduplication} in \textit{nini} ‘two’ may have been \isi{iconic}ally motivated. The \isi{reduplication} in \textit{lele} ‘three’ may have then resulted by \isi{analogy} (\sectref{sec:7.5}). The distributive \isi{quantifier} \textit{kekaka} (or \textit{kwekaka}) ‘one each, one by one’ seems to derive from \isi{reduplication} of the \isi{numeral} \textit{kwe} {\textasciitilde} \textit{kwa} ‘one’.\footnote{Cf. \ili{Tok Pisin} \textit{wanwan} ‘one by one’, from \textit{wan} ‘one’.} The \isi{distributive universal quantifier} \textit{nunu} ‘every’ apparently derives from \isi{reduplication}, but its exact etymology is obscure. The historical derivation of the \is{pluractional} verb \textit{nipinpu-} ‘die.\textsc{pl}’ may involve \isi{reduplication} (and \isi{metathesis}) of the \isi{inflect}ed verb form \textit{nip} ‘die.\textsc{sg} \textsc{[pfv]}’ (\sectref{sec:4.3}).

\is{noun|)}
\is{reduplication|)}
