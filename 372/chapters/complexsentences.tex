\chapter{Complex sentences}\label{sec:12}

\is{complex sentence|(}

In this chapter I examine how clauses are combined in Ulwa to form complex sentences. The combination of clauses of equal grammatical status (\isi{coordination}) is discussed in \sectref{sec:12.1}. Then, in \sectref{sec:12.2}, I consider Ulwa’s means for showing the dependence of one clause on another (\isi{subordination}). One special subtype of subordinate clause (the \isi{relative clause}) is investigated in \sectref{sec:12.3}. A very restricted form of \isi{clause chaining} is discussed in \sectref{sec:12.4}.

\is{complex sentence|)}

\section{Coordination}\label{sec:12.1}

\is{coordination|(}
\is{complex sentence|(}

There is no \isi{lexical class} of\isi{coordinator}s or \isi{coordinating conjunction}s in Ulwa -- that is, there are, generally speaking, no words equivalent to the \ili{English} word \textit{and} used to connect elements of equal grammatical status, whether to link words within a \isi{phrase}, \isi{phrase}s within a clause, or clauses within a sentence. Coordination (at all \isi{syntactic} levels) is accomplished through \isi{parataxis} -- coordinate elements are presented one after the other without any \isi{morphological} connector, whether \isi{bound morpheme} or independent word.\footnote{There are, however, some possible exceptions to this generalization: speakers commonly \isi{borrow} \ili{Tok Pisin} \isi{conjunction}s while speaking Ulwa; and there is also a rather infrequently used word \textit{ma} ‘and’, which I suspect to be a recent innovation (\sectref{sec:12.1.3}.)}

\is{complex sentence|)}
\is{coordination|)}

\subsection{Coordination within phrases}\label{sec:12.1.1}

\is{coordination|(}
\is{complex sentence|(}
\is{phrase|(}

Before examining \isi{coordination} between clauses, I consider how elements within \isi{phrase}s may be coordinated, starting with nouns within a \isi{noun phrase}. Nouns are coordinated within a \isi{noun phrase}s without any overt \isi{conjunction} or \isi{morphosyntactic} marking to indicate \isi{conjunction}. When the total \isi{number} of referents among the coordinated nouns in the NP equals two, then the entire NP may receive \isi{dual} subject marking or \isi{object marking}. When the total \isi{number} of referents is more than two, then the NP may receive \isi{plural} subject marking or \isi{object marking}. Examples \REF{ex:complex:1} through \REF{ex:complex:5} illustrate NPs in which two nouns are coordinated.

\ea%1
    \label{ex:complex:1}
            \textit{\textbf{Dimes Susan min} luke i mapta minji itana mane.}\\
\gll    \textbf{Dimes}  \textbf{Susan}  \textbf{min}  luke  i    ma=p-ta      min-nji ita-na    ma-n-e\\
    [name]  [name]  3\textsc{du}  too    go.\textsc{pfv}  \textsc{3sg.obj}=be\textsc{{}-cond}  \textsc{3du-poss}    build-\textsc{irr}  go\textsc{{}-ipfv-dep}\\
\glt `If Dimes and Susan go there, too, [then they] are going to build their [house there].’ [ulwa042\_06:08]
\z

\ea%2
    \label{ex:complex:2}
           \textit{\textbf{Imnde ame latï} inde.}\\
\gll    \textbf{imnde}    \textbf{ame}    \textbf{ala}=tï      inda-e\\
    basket  basket  \textsc{pl.dist}=take  walk-\textsc{ipfv}\\
\glt `[They] carried around baskets.’ [ulwa014\_67:27]
\z

\ea%3
    \label{ex:complex:3}
            \textit{\textbf{Yeta yena la} nakuklunda.}\\
\gll    \textbf{yeta}  \textbf{yena}  \textbf{ala}      na-kuk-lu-nda\\
    man  woman  \textsc{pl.dist}  \textsc{detr-}gather-put-\textsc{irr}\\
\glt `The boys and girls would gather.’ [ulwa032\_43:37]
\z

\ea%4
    \label{ex:complex:4}
            \textit{Manji \textbf{atana atuma ndiya} wa i.}\\
\gll    ma-nji      \textbf{atana}    \textbf{atuma}      \textbf{ndï=}iya    wa i\\
    3\textsc{sg.obj-poss}  older.sister  older.brother  3\textsc{pl}=toward  village    go.\textsc{pfv}\\
\glt `[He] went to his older brothers and sisters in the village.’ [ulwa001\_03:35]
\z

\ea%5
    \label{ex:complex:5}
            \textit{Nïtïne nïnji \textbf{wutï i} tembipe.}\\
\gll    nï=tï-n-e        nï-nji    \textbf{wutï}  \textbf{i}    tembi=p-e\\
    1\textsc{sg}=take-\textsc{pfv-dep}  1\textsc{sg-poss}  leg    hand  bad=\textsc{cop{}-dep}\\
\glt `When it got me, my legs and arms were injured.’ [ulwa026\_00:15]
\z

Although the \isi{noun phrase} in \REF{ex:complex:6} has exactly two referents, the \isi{plural} \isi{subject marker} \textit{ndï} ‘\textsc{3pl}’ is used (\sectref{sec:9.1.2}).

\ea%6
    \label{ex:complex:6}
            \textit{\textbf{Bill Elvis ndï} molop.}\\
\gll    \textbf{Bill}  \textbf{Elvis}  \textbf{ndï}  ma=lo-p\\
    [name]  [name]  3\textsc{pl}  3\textsc{sg.obj}=go-\textsc{pfv}\\
\glt `Bill and Elvis went there.’ [ulwa037\_47:30]
\z

Sentences \REF{ex:complex:7} and \REF{ex:complex:8} provide examples of more than two nouns being coordinated within a single NP.

\ea%7
    \label{ex:complex:7}
            \textit{\textbf{Awaka Mukamba Kawat ndï} mol i.}\\
\gll    \textbf{Awaka}  \textbf{Mukamba}  \textbf{Kawat}  \textbf{ndï}  ma=ul       i\\
    [name]  [name]      [name]  3\textsc{pl}  3\textsc{sg.obj}=with  go.\textsc{pfv}\\
\glt `Awaka, Mukamba, and Kawat came with him.’ [ulwa002\_03:27]
\z

\ea%8
    \label{ex:complex:8}
            \textit{\textbf{Anapa yawa ngata ndunduma ndï} wopa malanda.}\\
\gll    \textbf{anapa}  \textbf{yawa}  \textbf{ngata}  \textbf{ndunduma}    \textbf{ndï}  wopa  ma=la-nda\\
    sister  uncle  grand  great-grandparent  3\textsc{pl}  all    3\textsc{sg.obj}=eat-\textsc{irr}\\
\glt `Sisters, uncles, grandparents, and great-grandparents would all eat it.’ [ulwa014\_65:02]
\z

\isi{Pronoun}s may also signal \isi{coordination}, namely, in \isi{inclusory construction}s, whereby a \isi{non-singular} referent is signaled by the \isi{juxtaposition} of one or more referents with either a \isi{dual} or \isi{plural} \isi{pronoun} immediately following it. In \REF{ex:complex:9}, the \isi{pronoun} \textit{an} ‘1\textsc{pl.excl}’ signals the inclusion of the speaker in the set of ‘we’.

\ea%9
    \label{ex:complex:9}
            \textit{\textbf{Ganmali Dumngul an} molop lilïp.}\\
\gll    \textbf{Ganmali}  \textbf{Dumngul}  \textbf{an}      ma=lo-p      li-lï-p\\
    [name]    [name]    1\textsc{pl.excl}  3\textsc{sg.obj}=cut-\textsc{pfv}  down-put-\textsc{pfv}\\
\glt `Ganmali, Dumngul, and I cut it down.’ [ulwa032\_55:45]
\z

Sentences \REF{ex:complex:10} and \REF{ex:complex:11} exemplify the use of \isi{dual inclusory pronoun}s.

\ea%10
    \label{ex:complex:10}
          \textit{\textbf{Manama ngant}: …}\\
\gll    \textbf{Manama}  \textbf{ngan}=ta\\
    [name]  1\textsc{du.excl}=say\\
\glt `[He] told Manama and me: …’ [ulwa014\_22:14]
\z

\ea%11
    \label{ex:complex:11}
          \textit{\textbf{Awandana nganwe} indape ngan iwa tï ndïnambïlïp indap.}\\
\gll    \textbf{Awandana}  \textbf{ngan-we}        inda-p-e    ngan     iwa      tï    ndï=nambï-lï-p    inda-p\\
    [name]      \textsc{1du.excl-part.int}  walk-\textsc{pfv-dep}  \textsc{1du.excl}    basket  take  3\textsc{pl}=skin-put-\textsc{pfv}  walk-\textsc{pfv}\\
\glt `When Awandana and I alone went, we two got the fish trap, blocked them, and went.’ [ulwa036\_03:23]
\z

\isi{Inclusory construction}s are probably not structurally distinct from \isi{associative plural} constructions (\sectref{sec:7.1}).

  \isi{Adjective}s may also be coordinated. In example \REF{ex:complex:12}, two \isi{adjective}s that are functioning as nouns are coordinated in the same NP.

\ea%12
    \label{ex:complex:12}
          \textit{\textbf{Njukuta ambi} nen.}\\
\gll    \textbf{njukuta}  \textbf{ambi}  na-i-n\\
    small    big    \textsc{detr-}come-\textsc{pfv}\\
\glt `Both big and small [people] came.’ [ulwa029\_03:52]
\z

\isi{Adjective}s may also be coordinated within a single \isi{noun phrase} \isi{head}ed by a single noun. Multiple \isi{adnominal adjective}s can be coordinated within subject NPs, as in \REF{ex:complex:13}, as well as within object NPs, as in \REF{ex:complex:14} and \REF{ex:complex:15}, or in isolated \isi{noun phrase}s, as in \REF{ex:complex:16}.

\ea%13
    \label{ex:complex:13}
          \textit{Tïn \textbf{mbunmana ambi} mï unip.}\\
\gll    tïn    \textbf{mbunmana}  \textbf{ambi}  mï      uni-p\\
    dog  black      big    3\textsc{sg.subj}  shout-\textsc{pfv}\\
\glt `The big, black dog barked.’ [elicited]
\z

\ea%14
    \label{ex:complex:14}
          \textit{Nï lïmndï wambana \textbf{ambi anma} mala.}\\
\gll    nï    lïmndï  wambana  \textbf{ambi}  \textbf{anma}  ma=ala\\
    1\textsc{sg}  eye    fish    big    good  3\textsc{sg.obj}=see\\
\glt `I saw a nice, big fish.’ [elicited]
\z

\ea%15
    \label{ex:complex:15}
          \textit{Tïmbïl \textbf{ambi nïpat ngata} ndaytana.}\\
\gll    tïmbïl  \textbf{ambi}  \textbf{nïpat}  \textbf{ngata}  anda=ita-na\\
    fence  big    giant  grant  \textsc{sg.dist}=build-\textsc{irr}\\
\glt `[You] will build that big, huge, giant fence.’ [ulwa042\_00:43]
\z

\ea%16
    \label{ex:complex:16}
          \textit{Tokples \textbf{njukuta ilum} wa ndïtane.}\\
\gll    tokples  \textbf{njukuta}  \textbf{ilum}  wa  ndï=ta-n-e\\
    tokples  small    little  just  3\textsc{pl}=say-\textsc{ipfv-dep}\\
\glt `Little, short \textit{tokples} [= vernacular] stories -- [I] am just telling them.’ (\textit{tokples} = TP) [ulwa032\_14:59]
\z

Example \REF{ex:complex:17} illustrates coordinated \isi{predicative adjective}s.

\ea%17
    \label{ex:complex:17}
          \textit{Namndu mï \textbf{ambi ngatape}.}\\
\gll namndu  mï      \textbf{ambi}  \textbf{ngata}=p-e\\
    pig      3\textsc{sg.subj}  big    grand=\textsc{cop{}-dep}\\
\glt `The pig was really big.’ [ulwa037\_03:21]
\z

\isi{Verb phrase}s may also be coordinated. When multiple verbs are truly coordinated in the same \isi{verb phrase}, then the \isi{TAM} marking should match on all the verbs, as in \REF{ex:complex:18} and \REF{ex:complex:19}.

\ea%18
    \label{ex:complex:18}
          \textit{Alimban mï lamndu \textbf{masap mamap}.}\\
\gll Alimban  mï      lamndu  \textbf{ma=asa-p}      \textbf{ma=ama-p}\\
    {}[name]    3\textsc{sg.subj}  pig      3\textsc{sg.obj}=hit-\textsc{pfv}  3\textsc{sg.obj}=eat-\textsc{pfv}\\
\glt `Alimban killed and ate the pig.’ [elicited]
\z

\ea%19
    \label{ex:complex:19}
          \textit{Yawana mï utam \textbf{mawanap mamap}.}\\
\gll Yawana  mï      utam  \textbf{ma=wana-p}      \textbf{ma=ama-p}\\
    [name]    3\textsc{sg.subj}  yam  3\textsc{sg.obj}=cook-\textsc{pfv}  \textsc{3sg.obj}=eat-\textsc{pfv}\\
\glt `Yawana cooked and ate the yam.’ [elicited]
\z

The first verb may be unmarked, however, especially if it is an occasionally \isi{defective verb} (\sectref{sec:4.3}), as in \REF{ex:complex:20} and \REF{ex:complex:21}.

\ea%20
    \label{ex:complex:20}
          \textit{Min ko \textbf{mas mamap}.}\\
\gll min  ko  \textbf{ma=asa}    \textbf{ma=ama-p}\\
    3\textsc{du}  just  3\textsc{sg.obj}=hit  3\textsc{sg.obj}=eat-\textsc{pfv}\\
\glt `The two killed and ate it.’ [ulwa001\_00:44]
\z

\ea%21
    \label{ex:complex:21}
          \textit{Guren mï lïmndï lamndu \textbf{mala masap}.}\\
\gll Guren  mï      lïmndï  lamndu  \textbf{ma=ala}    \textbf{ma=asa-p}\\
    [name]  3\textsc{sg.subj}  eye    pig      3\textsc{sg.obj}=see  \textsc{3sg.obj}=hit-\textsc{pfv}\\
\glt `Guren saw and killed the pig.’ [elicited]
\z

\is{phrase|)}
\is{complex sentence|)}
\is{coordination|)}

Moreover, there should be no dependent marking \is{dependent marker}(\sectref{sec:12.2}) on anything other than the \isi{final verb} in the \isi{phrase}. Of course, given the \isi{homophony} of the \isi{imperfective} and \isi{dependent} forms (/-e/), it is sometimes difficult to determine whether a verb is receiving \isi{imperfective} marking (\sectref{sec:4.4}), dependent marking (\sectref{sec:12.2.1}), or both. In \REF{ex:complex:22}, I analyze \textit{wali-} ‘hit’ as taking the \isi{imperfective} \isi{suffix} \textit{-e} ‘\textsc{ipfv}’.

\ea%22
    \label{ex:complex:22}
          \textit{Lamndu \textbf{wale ndame}.}\\
\gll lamndu  \textbf{wali-e}    \textbf{ndï=ama-e}\\
    pig      hit-\textsc{ipfv}  3\textsc{pl}=eat-\textsc{ipfv}\\
\glt `[They] would kill and eat pigs.’ [ulwa014\_62:54]
\z

Although it is common for both verbs in a coordinated \isi{phrase} to receive \isi{object marking}, this is not mandatory: in example \REF{ex:complex:22}, only the second of the two coordinated verbs takes the \isi{object marker}.

\subsection{Coordination of clauses}\label{sec:12.1.2}

\is{complex sentence|(}
\is{coordination|(}

If a sentence contains two verbs that have different objects, then it is assumed that the \isi{coordination} occurs not between two verbs within a single \isi{verb phrase} but rather between two \isi{verb phrase}s. However, it is not always clear whether there are two \isi{verb phrase}s being coordinated within a single clause or there are two clauses being coordinated within a larger sentence. This is because it is common in Ulwa to omit subjects (\sectref{sec:11.1}). Thus, although example \REF{ex:complex:23} is translated as though the \isi{coordination} occurs within a single clause, it could alternatively be the case that there are two full clauses being coordinated, but that the subject in the second clause is omitted (i.e., ‘Alimban killed the pig and [he] cooked the meat’).

\ea%23
    \label{ex:complex:23}
          \textit{Alimban mï \textbf{lamndu masap mundu nduwanap}.}\\
\gll Alimban  mï      lamndu  ma=\textbf{asa}-p      mundu   ndï=\textbf{wana}-p\\
    [name]    3\textsc{sg.subj}  pig      3\textsc{sg.obj}=hit-\textsc{pfv}  food    3\textsc{pl}=cook-\textsc{pfv}\\
\glt `Alimban killed the pig and cooked the meat.’ [elicited]
\z

This point leads to the focus of this section: the \isi{coordination} of clauses in Ulwa. When two clauses are presented on equal grammatical footing, there is no distinction made between the two. They are presented \isi{paratactic}ally, one after the other, without any \is{dependent marker} dependent marking -- that is, Ulwa employs \isi{asyndetic coordination}, as illustrated by sentences \REF{ex:complex:24}, \REF{ex:complex:25}, and \REF{ex:complex:26}.

\ea%24
    \label{ex:complex:24}
          \textit{Mangusuwa as mï nip.}\\
\gll    {[ma-ngusuwa}  {asa]}  {[mï}      {ni-p]}\\
    {[3\textsc{sg.obj-}poor}  {hit]}    {[\textsc{3sg.subj}}  {die-\textsc{pfv]}}\\
\glt `[They] struck the poor thing and he died.’ [ulwa037\_30:42]
\z

\ea%25
    \label{ex:complex:25}
          \textit{Nï mbïwap mokotïp.}\\
\gll    {[nï}    {mbï-wap]}    {[ma=kot-p]}\\
    {[1\textsc{sg}}  here-be.\textsc{pst]}  {[3\textsc{sg.obj}=break-\textsc{pfv]}}\\
\glt `I stayed here and [I] bore her.’ [ulwa014†]
\z

\ea%26
    \label{ex:complex:26}
          \textit{Mangusuwa mbïwap mï amun naman.}\\
\gll    {[ma-ngusuwa}  {mbï-wap]}    {[mï}      amun  {na-ma-n]}\\
    {[3\textsc{sg.obj-}poor}  here-be.\textsc{pst]}  {[3\textsc{sg.subj}}  now  \textsc{detr-}go-\textsc{ipfv]}\\
\glt `The poor thing stayed here and today she’s leaving.’ [ulwa032\_06:25]
\z

Sentences \REF{ex:complex:24}, \REF{ex:complex:25}, and \REF{ex:complex:26} are all translated with ‘and’. Coordinated clauses can have \is{concessive coordination} \isi{concessive} (i.e., ‘but’) senses as well. Again, this is achieved without any overt \isi{coordinating conjunction} \REF{ex:complex:27}.

\ea%27
    \label{ex:complex:27}
          \textit{Mï ango maka Nïmalnu wa map mï nay.}\\
\gll    {[mï}      ango  maka  Nïmalnu  wa    {ma=p]}      {[mï} {na-i]}\\
    [3\textsc{sg.subj}  \textsc{neg}  thus  Manu    village  \textsc{3sg.obj}=be]  \textsc{[3sg.subj}    \textsc{detr-}go.\textsc{pfv]}\\
\glt `He didn’t stay in Manu village, but he went.’ [ulwa023\_00:32]
\z

\is{concessive coordination}

Coordination of clauses is not, however, especially common: speakers generally prefer to mark one or more clauses as \isi{dependent} (\sectref{sec:12.2}).

\is{coordination|)}
\is{complex sentence|)}

\subsection{Other means of coordination}\label{sec:12.1.3}

\is{coordination|(}
\is{complex sentence|(}

It is common for speakers to \isi{borrow} words from \ili{Tok Pisin} when coordinated structures are desired, especially when they are \is{disjunctive coordination} disjunctive (i.e., ‘or’) structures, as in sentences \REF{ex:complex:28}, \REF{ex:complex:29}, and \REF{ex:complex:30}, which \isi{borrow} \ili{Tok Pisin} \textit{o} ‘or’.

\ea%28
    \label{ex:complex:28}
          \textit{U wandam mana} \textbf{\textit{o}} \textit{nï wandam mana.}\\
\gll    u    wandam  ma-na  \textbf{o}  nï    wandam  ma-na\\
    2\textsc{sg}  jungle    go-\textsc{irr}  or  \textsc{2sg}  jungle    go-\textsc{irr}\\
\glt `Either you will go to the jungle or I will go to the jungle.’ (\textit{o} < TP \textit{o} ‘or’) [elicited]
\z

\ea%29
    \label{ex:complex:29}
        \textit{Wambana tïn malanda} \textbf{\textit{o}} \textit{an ma wanwane angop i ndïwanap ndïlanda.}\\
\gll    wambana  tï-n      ma=la-nda      \textbf{o}  an      ma     wanwane  ango-p      i    ndï=wana-p  ndï=la-nda\\
    fish    take-\textsc{pfv}  \textsc{3sg.obj}=eat-\textsc{irr}  or  \textsc{1pl.excl}  go    mushroom  pull.out-\textsc{pfv}  go\textsc{.pfv}  \textsc{3pl=}cook\textsc{{}-pfv}  \textsc{3pl=}eat\textsc{{}-irr}\\
\glt `Either [they] would catch a fish and [we] would eat it or we would go, pick mushrooms, go cook them, and eat them.’ (\textit{o} < TP \textit{o} ‘or’) [ulwa032\_01:53]
\z

\ea%30
    \label{ex:complex:30}
          \textit{Nï mana} \textbf{\textit{o}} \textit{nï mbïpïna nï ango kalam.}\\
\gll    nï    ma-na  \textbf{o}  nï    mbï-p-na    nï    ango  kalam\\
    1\textsc{sg}  go-\textsc{irr}  or  \textsc{1sg}  here-be-\textsc{irr}  \textsc{1sg}  \textsc{neg}  knowledge\\
\glt `Should I go or should I stay? I don’t know.’ (\textit{o} < TP \textit{o} ‘or’) [ulwa037\_49:07]
\z

This \ili{Tok Pisin} \isi{loanword} \textit{o} ‘or’ is used not only to connect clauses, but also to connect elements within \isi{phrase}s, as in \REF{ex:complex:31}, \REF{ex:complex:32}, and \REF{ex:complex:33}.

\ea%31
    \label{ex:complex:31}
          \textit{Kawana mï mïnda} \textbf{\textit{o}} \textit{utam amap.}\\
\gll    Kawana  mï      mïnda  \textbf{o}  utam  ama-p\\
    [name]    3\textsc{sg.subj}  banana  or  yam  eat-\textsc{pfv}\\
\glt `Kawana ate either a banana or a yam.’ (\textit{o} < TP \textit{o} ‘or’) [elicited]
\z

\ea%32
    \label{ex:complex:32}
          \textit{U} \textbf{\textit{o}} \textit{nï wandam mana.}\\
\gll    u    \textbf{o}  nï    wandam  ma-na\\
    2\textsc{sg}  or  1\textsc{sg}  jungle    go-\textsc{irr}\\
\glt `Either you or I will go to the jungle.’ (\textit{o} < TP \textit{o} ‘or’) [elicited]
\z

\ea%33
    \label{ex:complex:33}
          \textit{Mïnal} \textbf{\textit{o}} \textit{mil} \textbf{\textit{o}} \textit{utam} \textbf{\textit{o}} \textit{nongontam mï keka ndïn up.}\\
\gll    mïnal  \textbf{o}  mil      \textbf{o}  utam  \textbf{o}  nongontam   mï      keka      ndï=n    u-p\\
    taro  or  sugarcane  or  yam  or  kaukau    3\textsc{sg.subj}  completely  3\textsc{pl=obl}  put-\textsc{pfv}\\
\glt `[Whether it be] taro or sugarcane or yam or \textit{kaukau} [= sweet potato], he planted them all.’ (\textit{o} < TP \textit{o} ‘or’) [ulwa006\_02:53]
\z

The \ili{Tok Pisin} \isi{loanword} \textit{na} ‘and’ is also used in discourse to coordinate elements, whether words within a \isi{phrase} \REF{ex:complex:34}, \isi{phrase}s within a clause, or clauses within a sentence \REF{ex:complex:35}.

\ea%34
    \label{ex:complex:34}
          \textit{Bopten} \textbf{\textit{na}} \textit{Yar ngusuwa ndï wome mat ndïnane.}\\
\gll    Bopten  \textbf{na}  Yar    ngusuwa  ndï   wome    ma=tï ndï=na-n-e\\
    [place]  and  [place]  poor    3\textsc{pl}   middle  \textsc{3pl}=take    \textsc{3pl}=give-\textsc{pfv-dep}\\
\glt `The poor [people from] Bopten and Yar gave them the middle [piece of land between Bopten and Yar villages].’ (\textit{na} < TP \textit{na} ‘and’) [ulwa014\_17:02]
\z

\ea%35
    \label{ex:complex:35}
          \textit{Tïklika} \textbf{\textit{na}} \textit{anmbi.}\\
\gll    tïkli-ka  \textbf{na}  an-mbï-i\\
    turn-let  and  out-here-go.\textsc{pfv}\\
\glt `[I] turned and came out.’ (\textit{na} < TP \textit{na} ‘and’) [ulwa040\_00:32]
\z

The \isi{borrowing} of \ili{Tok Pisin} \isi{loan}s for grammatical functions such as \isi{coordination} is further described in \chapref{sec:15}.

  Some speakers use \textit{ma} ‘and’ in certain coordinate structures. This seems more frequent among younger speakers and is perhaps a recent innovation. It bears a superficial resemblance to \ili{Tok Pisin} \textit{na} ‘and’, but could instead be derived from the \isi{object marker} \textit{ma=} ‘\textsc{3sg.obj}’ or the \isi{subject marker} \textit{mï} ‘\textsc{3sg.subj}’.\footnote{This \isi{coordinator} is sometimes pronounced in a seemingly reduced way as [mï].} Regardless of its origins, as a connector within \isi{noun phrase}s, \textit{ma} ‘and’ is limited in its usage, appearing almost exclusively after proper names, as in sentences \REF{ex:complex:36} through \REF{ex:complex:39}.

\ea%36
    \label{ex:complex:36}
          \textit{Nicko} \textbf{\textit{ma}} \textit{Danny min niya i.}\\
\gll    Nicko  \textbf{ma}  Danny  min  nï=iya      i\\
    [name]  and  [name]  3\textsc{du}  \textsc{1sg=}toward  go.\textsc{pfv}\\
\glt `Nicko and Danny came to me.’ [ulwa014\_40:02]
\z

\ea%37
    \label{ex:complex:37}
          \textit{Pisuwa} \textbf{\textit{ma}} \textit{Yaluwa minul le.}\\
\gll    Pisuwa  \textbf{ma}  Yaluwa  min=ul    lo-e\\
    [name]  and  [name]    3\textsc{du}=with  go-\textsc{ipfv}\\
\glt `[He] was following Pisuwa and Yaluwa.’ [ulwa014\_49:53]
\z

\ea%38
    \label{ex:complex:38}
          \textit{Tupuk} \textbf{\textit{ma}} \textit{Bay min man mat.}\\
\gll    Tupuk  \textbf{ma}  Bay    min  ma=n      ma=ta\\
    [name]  and  [name]    3\textsc{du}  3\textsc{sg.obj=obl}  \textsc{3sg.obj}=say\\
\glt `Tupuk and Bay told her.’ [ulwa014†]
\z

\ea%39
    \label{ex:complex:39}
          \textit{Nambul} \textbf{\textit{ma}} \textit{Wangasa min mawat pe ambinasap.}\\
\gll    Nambul  \textbf{ma}  Wangasa  min  ma=wat    p-e     ambin=asa-p\\
    [name]    and  [name]    \textsc{3du}  \textsc{3sg.obj=}atop  be\textsc{{}-dep}    \textsc{du.refl=}hit-\textsc{pfv}\\
\glt `Nambul and Wangasa fought over it.’ [ulwa014†]
\z

The connector \textit{ma} ‘and’ may be used to connect more than two \isi{proper noun} NPs, as in \REF{ex:complex:40}.

\ea%40
    \label{ex:complex:40}
          \textit{Lïmndï Ambayam} \textbf{\textit{ma}} \textit{Josephine} \textbf{\textit{ma}} \textit{Susan ndala.}\\
\gll    lïmndï  Ambayam  \textbf{ma}  Josephine  \textbf{ma}  Susan  ndï=ala\\
    eye    [name]    and  [name]    and  [name]  3\textsc{pl}=see\\
\glt `[I] saw Ambayam, Josephine, and Susan.’ [ulwa037\_02:03]
\z

As a connector of NPs, \textit{ma} ‘and’ may follow a \isi{proper noun} even when the other NP is a \isi{pronoun} \REF{ex:complex:41}.

\ea%41
    \label{ex:complex:41}
          \textit{Donna} \textbf{\textit{ma}} \textit{ndï molop.}\\
\gll    Donna  \textbf{ma}  ndï  ma=lo-p\\
    [name]  and  3\textsc{pl}  3\textsc{sg.obj}=go-\textsc{pfv}\\
\glt `Donna and they went there.’ [ulwa042\_04:04]
\z

As a clausal \isi{coordinator}, \textit{ma} ‘and’ may even be derived from \textit{ma} ‘go’, perhaps \isi{calque}d from \ili{Tok Pisin} uses of \textit{go} ‘go’ as a discourse connector. Sentence \REF{ex:complex:42} suggests the ambiguity of the form [ma], which as a connector here could mean either ‘go’ or ‘and’.

\is{complex sentence|)}
\is{coordination|)}

\ea%42
    \label{ex:complex:42}
          \textit{Ay nïkap} \textbf{\textit{ma}} \textit{ndïmokota ndïnata mana.}\\
\gll    ay    nïkï-p    \textbf{ma}  ndï=moko-ta    ndï=na-ta      ma-na\\
    sago  dig{}-\textsc{pfv}  and  3\textsc{pl}=take\textsc{{}-cond}  \textsc{3pl=}give-\textsc{cond}  go-\textsc{irr}\\
\glt `[I] have made sago and will go and give them [= servings of sago] to them.’ [ulwa013\_01:03]
\z

\section{Subordination}\label{sec:12.2}

\is{subordination|(}
\is{subordinate clause|(}
\is{dependent clause|(}
\is{complex sentence|(}
\is{complement clause}

Ulwa makes prolific use of clause-linking, connecting \isi{dependent clause}s to following \isi{independent clause}s (or to further \isi{dependent} clauses) with the verbal \isi{suffix} \textit{-e} ‘\textsc{dep}’, which is glossed here as “\isi{dependent}”. The order of constituents in a subordinate clause is the same as is found in \isi{main clause}s (i.e., SOV). Complement clauses (i.e., \isi{clausal object}s) occur in the same position as nominal objects (i.e., between the subject and the verb of the \isi{matrix clause}). There are no \isi{complementizer}s in Ulwa.


\is{complex sentence|)}
\is{dependent clause|)}
\is{subordinate clause|)}
\is{subordination|)}

\subsection{The dependent marker \textit{-e} ‘\textsc{dep}’}\label{sec:12.2.1}

\is{subordination|(}
\is{subordinate clause|(}
\is{dependent clause|(}
\is{complex sentence|(}
\is{dependent marker|(}

The \isi{dependent marker} \textit{-e} ‘\textsc{dep}’ is a \isi{suffix} that can affix to fully \isi{inflect}ed verb forms (that is, to verbs with \isi{TAM} \isi{suffix} marking).\footnote{Whether the \isi{dependent marker} \textit{-e} ‘\textsc{dep}’ affixes to the \isi{imperfective} \isi{suffix} \textit{-e} ‘\textsc{ipfv}’ is, however, an insoluble question, since a posited underlying form of /-e-e/ would reduce to [e]. In this grammar, verbs with the form [\isi{stem}][e] are glossed sometimes as ‘[\isi{stem}]-\textsc{ipfv}’ and sometimes as ‘[\isi{stem}]-\textsc{dep}’, according to context. At times the decision is arbitrary. The glossing should never be taken to be a definitive statement on which of the two \isi{homophonous} forms is being used. Moreover, in many instances it is possible that both forms are underlyingly present in the same verb form.} The use of the \isi{dependent marker} in Ulwa is not considered an indication of the prototypical \isi{clause chaining} (or \isi{medial clause}s) found in many languages of \isi{New Guinea}, since the de\-pendent-marked verbs in these clauses do not have “more restricted structures”, nor do they indicate “\isi{switch reference}” \citep[399]{Longacre2007}. This is, nevertheless, clearly a kindred phenomenon. See \sectref{sec:12.4} for what may be a better candidate of \isi{clause chaining} in Ulwa.

  As just implied, the subject of the \isi{dependent clause} may be the same as or different from the subject of a subsequent \isi{independent clause} without any \isi{morphological} indication one way or the other. When one clause is subordinated to another, it almost always precedes it. A subordinate clause marked by the \isi{dependent marker} \textit{-e} ‘\textsc{dep}’ may bear one of a few \isi{semantic} relations to the \isi{main clause} on which it depends: causal (\sectref{sec:12.2.2}), \isi{concessive} (\sectref{sec:12.2.3}), \isi{temporal} (\sectref{sec:12.2.4}), and so on.\footnote{\ili{Proto-Keram} probably contrasted two \isi{dependent marker}s: *-a, which would have marked a \isi{sequential} relationship to the following clause, and *\nobreakdash-e, which would have marked a \isi{simultaneous} relationship to the following clause. The \ili{Keram} language \ili{Ambakich} still retains the distinction \citep[74]{Barlow2021}. In Ulwa, however, this contrast has been lost: the sole \isi{dependent marker} \textit{-e} ‘\textsc{dep}’ can indicate either a \isi{sequential} or a \isi{simultaneous} \isi{temporal} relationship, as well as other logical relationships between clauses.}

\is{dependent marker|)}
\is{complex sentence|)}
\is{dependent clause|)}
\is{subordinate clause|)}
\is{subordination|)}

\subsection{Causal subordinate clauses}\label{sec:12.2.2}

\is{causal subordinate clause|(}
\is{subordination|(}
\is{subordinate clause|(}
\is{dependent clause|(}
\is{complex sentence|(}

Sentences \REF{ex:complex:43} through \REF{ex:complex:48} contain \isi{dependent clause}s that bear causal relations to their respective \isi{independent clause}s.

\ea%43
    \label{ex:complex:43}
          \textit{\textbf{Nïnji yanat mï tembipe} nonganup.}\\
\gll    \textbf{nï-nji}    \textbf{yanat}    \textbf{mï}      \textbf{tembi=p-e}    nongan-u-p\\
    1\textsc{sg-poss}  daughter  3\textsc{sg.subj}  bad=\textsc{cop{}-dep} vomit-put-\textsc{pfv}\\
\glt `My daughter vomited because she was sick.’ (Literally ‘Since my daughter was sick, she vomited.’) [elicited]
\z

\ea%44
    \label{ex:complex:44}
          \textit{\textbf{Itom mundu mase} utam mamap.}\\
\gll    \textbf{itom}  \textbf{mundu}  \textbf{ma=asa-e}      utam  ma=ama-p\\
    father  hunger    3\textsc{sg.obj}=hit-\textsc{dep}  yam  3\textsc{sg.obj}=eat-\textsc{pfv}\\
\glt `Father ate the yam because he was hungry.’ (Literally ‘Since father was hungry, he ate the yam.’) [elicited]
\z

\ea%45
    \label{ex:complex:45}
         \textit{\textbf{Nu pe} Kumba la unanlu amblawale.}\\
\gll    \textbf{nu}    \textbf{p-e}    Kumba  ala      unan=lu      ambla=wali-e\\
    near  be\textsc{{}-dep} Bun  \textsc{pl.dist}  1\textsc{pl.incl}=with    \textsc{pl.refl}=hit-\textsc{ipfv}\\
\glt `Since [Bun village] is close, the Bun people fight with us.’ [ulwa014\_24:48]
\z

\ea%46
    \label{ex:complex:46}
          \textbf{\textit{Nipe}} \textit{nganwe nini ngan mbïp.}\\
\gll    \textbf{ni-p-e}      ngan-we        nini  ngan    mbï-p\\
    die-\textsc{pfv-dep}  \textsc{1du.excl-part.int}  two  \textsc{1du.excl}  here-be\\
\glt `Since [our two siblings] have died, we two alone -- we stay here.’ [ulwa028\_00:21]
\z

\ea%47
    \label{ex:complex:47}
         \textit{\textbf{Ya ulwape} an wa inimnï ndïwane.}\\
\gll    \textbf{ya}      \textbf{ulwa=p-e}      an      wa  inim=nï     ndï=wana-e\\
    coconut  nothing=\textsc{cop{}-dep}  \textsc{1pl.excl} just  water=\textsc{obl}    3\textsc{pl}=cook-\textsc{dep}\\
\glt `Since there were no coconuts, we just cooked them in water.’ [ulwa032\_01:25]
\z

\ea%48
    \label{ex:complex:48}
          \textit{\textbf{Wanmbi ulwape} nï wa aw ngan wa akïnaka landa man.}\\
\gll    \textbf{wanmbi}  \textbf{ulwa=p-e}      nï    wa  aw      nga=n     wa  akïnaka  la-nda  ma-n\\
    daka    nothing=\textsc{cop}{}-\textsc{dep}  \textsc{1sg}  just  betel.nut  \textsc{sg.prox=obl}    just  new    eat-\textsc{irr}  go-\textsc{ipfv}\\
\glt `Since there’s no \textit{daka} [= betel pepper], I’m just going to chew this betel nut fresh.’ (i.e., without \textit{daka} pepper or lime) [ulwa037\_35:03]
\z

Instead of the \isi{dependent marker} \textit{-e} ‘\textsc{dep}’, the \isi{conditional} \isi{suffix} \textit{-ta} ‘\textsc{cond}’ (\sectref{sec:4.12}, \sectref{sec:13.5}) may be affixed to the \isi{final verb} in a \isi{dependent clause}, providing a similar causal function as the \isi{dependent marker} \textit{-e} ‘\textsc{dep}’ \REF{ex:complex:49}.

\ea%49
    \label{ex:complex:49}
          \textit{\textbf{Unanji ngata lanji luwa lawapta} maka apa ndaytana.}\\
\gll    unan-nji    ngata   ala-nji luwa ala=wap-\textbf{ta}     maka  apa    anda=ita-na\\
    1\textsc{pl.incl-poss}  grand  \textsc{pl.dist-poss}  place  \textsc{pl.dist}=be.\textsc{pst-cond}    thus  house  \textsc{sg.dist}=build-\textsc{irr}\\
\glt `Since those were our ancestors’ lands, [we] will thus build that house.’ [ulwa037\_38:58]
\z

The \isi{conditional} \isi{suffix} \textit{-ta} ‘\textsc{cond}’ is not known to co-occur with the \isi{dependent marker} \textit{-e} ‘\textsc{dep}’ (i.e., \textsuperscript{†}/-ta-e/ ‘\textsc{cond-dep}’ is unattested).

\is{complex sentence|)}
\is{dependent clause|)}
\is{subordinate clause|)}
\is{subordination|)}
\is{causal subordinate clause|)}

\subsection{Concessive subordinate clauses}\label{sec:12.2.3}

\is{concessive subordinate clause|(}
\is{subordination|(}
\is{subordinate clause|(}
\is{dependent clause|(}
\is{complex sentence|(}

In sentences \REF{ex:complex:50} and \REF{ex:complex:51}, the \isi{dependent clause}s bear a \isi{concessive} relation to their associated \isi{independent clause}s.

\ea%50
    \label{ex:complex:50}
          \textit{\textbf{Ndï ndïl kumat ine} kuma wa mïnwata wandam lïp.}\\
\gll    \textbf{ndï}     \textbf{ndï=lï}    \textbf{kuma=tï}   \textbf{i-n-e}        kuma  wa mïnwata  wandam  lï-p\\
    3\textsc{pl}    3\textsc{pl=}put  some=take   come-\textsc{pfv-dep}    some  just    rotting    jungle    put-\textsc{pfv}\\
\glt `Although they have brought some of them [home], [they] have left others just rotting in the jungle.’ [ulwa032\_54:29]
\z

\ea%51
    \label{ex:complex:51}
        \textit{\textbf{Wot mï maka lïmndï matïne} atuma mï nupu matïn.}\\
\gll    \textbf{wot}    \textbf{mï}      \textbf{maka}  \textbf{lïmndï}  \textbf{ma=tï-n-e}     atuma      mï      nupu  ma=tï-n\\
    younger  3\textsc{sg.subj}  thus  eye    3\textsc{sg.obj}=take-\textsc{pfv-dep}  older.brother  \textsc{3sg.subj}  base  3\textsc{sg.obj}=take-\textsc{pfv}\\
\glt `Whereas the younger [brother] got the eye [side of the coconut], the older brother got the base [side of the coconut].’ [ulwa010\_01:22]
\z

Much like \isi{causal subordinate clause}s (\sectref{sec:12.2.2}), \isi{concessive} subordinate clauses may on occasion employ \isi{conditional} \isi{suffix}es in place of the \isi{dependent marker}, as in \REF{ex:complex:52}.

\ea%52
    \label{ex:complex:52}
          \textit{\textbf{Wa mïnomapïta ndin pïta} tem mat an mokolpe mï wa nambïtïn ninda!}\\
\gll    {wa}  {mïnoma=p-\textbf{ta}}    {ndï=in}  {p-ta}    {tem}  {ma=tï}      {an} {ma=kol-p-e}        mï      wa  nambït=ïn  ni-nda\\
    just  cold=\textsc{cop}{}-\textsc{cond}  \textsc{3pl}=in  be\textsc{{}-cond} time  3\textsc{sg.obj}=take  out    3\textsc{sg.obj}=break-\textsc{pfv-dep}  \textsc{3sg.subj}  just  smell=\textsc{obl}  act-\textsc{irr}\\
\glt `Even though [the meat] will get cold in them [= pots], when [you] have taken it out and broken it, it will just smell [good]!’ (\textit{tem} < TP \textit{taim} ‘time’) [ulwa014\_64:11]
\z

Example \REF{ex:complex:52} actually illustrates two \isi{dependent clause}s in succession: the first, a \isi{concessive} clause, is marked by the \isi{conditional} \isi{suffix} \textit{-ta} ‘\textsc{cond}’; the second, a \isi{temporal} clause, is marked by the \isi{dependent marker} \textit{-e} ‘\textsc{dep}’. This \isi{temporal} clause contains the \ili{Tok Pisin} \isi{loanword} \textit{taim} ‘time’ (> Ulwa \textit{tem} ‘time, when’), here functioning as a \isi{subordinator}. This \isi{loanword}, however, is not needed to form \isi{temporal subordinate clause}s, as shown in \sectref{sec:12.2.4}.

\is{complex sentence|)}
\is{dependent clause|)}
\is{subordinate clause|)}
\is{subordination|)}
\is{concessive subordinate clause|)}

\subsection{Temporal subordinate clauses}\label{sec:12.2.4}

\is{temporal subordinate clause|(}
\is{subordination|(}
\is{subordinate clause|(}
\is{dependent clause|(}
\is{complex sentence|(}

In sentence \REF{ex:complex:53}, the \isi{dependent marker} helps signal that the event referred to in the \isi{dependent clause} occurred \isi{simultaneous}ly to the action of the associated \isi{independent clause} (i.e., signaling the sense of ‘while’).

\ea%53
    \label{ex:complex:53}
          \textit{\textbf{Plas mambi ango mbïpe} nji tïngïn up.}\\
\gll    \textbf{Plas}   \textbf{ma-ambi}    \textbf{ango}  \textbf{mbï-p-e}    nji    tïngïn=n u-p\\
    [name]  3\textsc{sg.obj-top}  \textsc{neg}  here-be\textsc{{}-dep} thing  many=\textsc{obl}    put-\textsc{pfv}\\
\glt `As for Plas, he didn’t plant many things while he was here.’ [ulwa014†]
\z

\newpage

This \isi{temporal} sense is sometimes translated with ‘when’ in \ili{English} \REF{ex:complex:54}.

\ea%54
    \label{ex:complex:54}
          \textit{\textbf{An njukutape} ndul inde.}\\
\gll    \textbf{an}      \textbf{njukuta=p-e}    ndï=ul    inda-e\\
    1\textsc{pl.excl}  small=\textsc{cop{}-dep} 3\textsc{pl}=with  walk-\textsc{ipfv}\\
\glt `When we were small, we went with them.’ [ulwa029\_00:56]
\z

As with other subordinate clause types, it is possible in \isi{temporal} constructions for the \isi{conditional} \isi{suffix} \textit{-ta} ‘\textsc{cond}’ to occur at the end of the subordinate clause instead of the \isi{dependent marker} \textit{-e} ‘\textsc{dep}’ \REF{ex:complex:55}.

\ea%55
    \label{ex:complex:55}
          \textit{\textbf{Ala ndandïla mapta} suwan ndïnap nawlunda mane.}\\
\gll    ala      ndï=andïla  ma=p-\textbf{ta}      suwan  ndï=nap na-u-lo-nda      ma-n-e\\
    \textsc{pl.dist}  \textsc{3pl}=await    \textsc{3sg.obj}=be\textsc{{}-cond} rack  3\textsc{pl}=for    \textsc{detr-}from-cut-\textsc{irr}  go-\textsc{ipfv-dep}\\
\glt `So, while they are there waiting for them, [they] are going to cut [things] for the mesh racks.’ [ulwa014\_67:53]
\z

Dependent marking \is{dependent marker} can also be used when the action of the \isi{main clause} occurred at a particular point in \isi{time} \isi{simultaneous} with that of the \isi{dependent clause}, generally yielding an \ili{English} translation with ‘when’, as in \REF{ex:complex:56} and \REF{ex:complex:57}.

\ea%56
    \label{ex:complex:56}
          \textit{\textbf{Nï tembipe} u malasin alakali nïn anmbï lïp.}\\
\gll    \textbf{nï}    \textbf{tembi=p-e}    u    malasin  ala=kali    nï=n     an-mbï    lï-p\\
    1\textsc{sg}  bad=\textsc{cop{}-dep}  \textsc{2sg}  medicine  \textsc{pl.dist}=send  1\textsc{sg=obl}    out-here  put-\textsc{pfv}\\
\glt `When I was sick, you sent medicine to me.’ (\textit{malasin} = TP \textit{marasin}) [ulwa026\_00:03]
\z

\ea%57
    \label{ex:complex:57}
          \textit{\textbf{Anmbi atwana te} ndï man nït.}\\
\gll    \textbf{an-mbï-i}      \textbf{atwana}  \textbf{ta-e}     ndï  ma=n      nï=ta\\
    out-here-go\textsc{.pfv}  question  say-\textsc{dep}  \textsc{3pl}  \textsc{3sg.obj=obl}  \textsc{1sg=}say\\
\glt `When [I] came out and asked, they told me.’ [ulwa037\_39:36]
\z

Very commonly, there is a simple \isi{sequential} \isi{temporal} relationship between a subordinate clause and the clause that follows it. That is, the \isi{dependent marker} on the subordinate clause marks the event described within it as occurring before the event described in the associated \isi{independent clause}, as in examples \REF{ex:complex:58} through \REF{ex:complex:64}.

\newpage

\ea%58
    \label{ex:complex:58}
          \textit{\textbf{Ala apïn mamape} nï wa mbi ndïmoni pe.}\\
\gll    \textbf{ala}      \textbf{apïn=n}    \textbf{ma=ama-p-e}      nï    wa      mbï-i      ndï=moni    p-e\\
    \textsc{pl.dist}  fire=\textsc{obl}  3\textsc{sg.obj}=eat-\textsc{pfv-dep}  \textsc{1sg}  just    here-go\textsc{.pfv}  \textsc{3pl}=between  be-\textsc{ipfv}\\
\glt `After they burned it, I just came to this place and live among them.’ [ulwa014\_22:24]
\z

\ea%59
    \label{ex:complex:59}
          \textit{\textbf{Mï mawap liye} na ndïtïna.}\\
\gll    \textbf{mï}      \textbf{ma=wap}      \textbf{li-i-e}        na    ndï=tï-na\\
    3\textsc{sg.subj}  3\textsc{sg.obj}=be.\textsc{pst}  down-go.\textsc{pfv-dep}  talk  \textsc{3pl}=take-\textsc{irr}\\
\glt `After he’s stayed there and [then] come down, [he] will get the conversations.’ [ulwa014\_33:58]
\z

\ea%60
    \label{ex:complex:60}
          \textit{\textbf{Ndï ndamap inim lopop ataye} an anmbi uniya wa molop.}\\
\gll    \textbf{ndï}  \textbf{ndï=ama-p}  \textbf{inim}  \textbf{lopo-p}    \textbf{ata-i-e}      an an-mbï-i      un=iya      wa    ma=lo-p\\
    3\textsc{pl}  3\textsc{pl}=eat-\textsc{pfv}  water  wash-\textsc{pfv}  up-go.\textsc{pfv-dep} \textsc{1pl.excl}    out-here-go.\textsc{pfv}  \textsc{2pl=}toward  village  3\textsc{sg.obj}=go-\textsc{pfv}\\
\glt `After they ate them, washed and came up, we came out to you in the village.’ [ulwa032\_24:43]
\z

\ea%61
    \label{ex:complex:61}
          \textit{\textbf{Mï mankape} ndï moko amblanan.}\\
\gll    \textbf{mï}      \textbf{ma=nïkï-p-e}      ndï  moko  ambla=na-n\\
    3\textsc{sg.subj}  3\textsc{sg.obj}=dig{}-\textsc{pfv-dep}  \textsc{3pl}  take  \textsc{pl.refl=}give-\textsc{pfv}\\
\glt `After he butchered it, they shared [it] among themselves.’ [ulwa035\_05:29]
\z

\ea%62
    \label{ex:complex:62}
          \textit{\textbf{Mbi wa mbï itape} ndï Yetani lan u matïn.}\\
\gll    \textbf{mbï-i}      \textbf{wa}    \textbf{mbï}  \textbf{ita-p-e}      ndï  Yetani  ala=n     u    ma=tï-n\\
    here-go.\textsc{pfv}  village  here  build-\textsc{pfv-dep}  \textsc{3pl}  Yamen  \textsc{pl.dist=obl}    from  \textsc{3sg.obj}=take-\textsc{pfv}\\
\glt `After [they] came here and made this village, they got it [= sorcery] from the Yamen people.’ [ulwa037\_10:59]
\z

\ea%63
    \label{ex:complex:63}
         \textit{\textbf{Nï inim lopope} nï mana.}\\
\gll    \textbf{nï}  \textbf{inim}    \textbf{lopo-p-e}    nï    ma-na\\
    1\textsc{sg}  water  wash-\textsc{pfv-dep}  \textsc{1sg}  go-\textsc{irr}\\
\glt `After I’ve bathed, then I will go.’ [ulwa040\_01:46]
\z

\ea%64
    \label{ex:complex:64}
         \textit{\textbf{Min anmbï naye an mïnanamape} an tawnam nolop.}\\
\gll    \textbf{min}  \textbf{an-mbï}    \textbf{na-i-e}        \textbf{an}     \textbf{mï=na-na-ama-p-e}           an      tawnam     na-u-lo-p\\
    3\textsc{du}  out-here  \textsc{detr-}go.\textsc{pfv-dep}  \textsc{1pl.excl}  \textsc{3sg.subj=detr-detr-}eat-\textsc{pfv-dep}  \textsc{1pl.excl}  net \textsc{detr-}from-go-\textsc{pfv}\\
\glt `After the two came out, we ate, and then we went to [our] mosquito nets.’ (\textit{tawnam} = TP \textit{taunam}) [ulwa041\_01:20]
\z

As example \REF{ex:complex:64} illustrates, multiple \isi{dependent clause}s may be strung together in succession.

  It is never the case that a \isi{dependent-marked} clause signals a \isi{time} after the \isi{time} of the \isi{independent clause}. In other words, a \isi{dependent-marked} clause will never be translated as ‘before …’.

\is{complex sentence|)}
\is{dependent clause|)}
\is{subordinate clause|)}
\is{subordination|)}
\is{temporal subordinate clause|)}

\subsection{Tail-head linkage}\label{sec:12.2.5}

\is{tail-head linkage|(}
\is{complex sentence|(}
\is{subordination|(}
\is{subordinate clause|(}
\is{dependent clause|(}

Subordinate clauses marked with the \isi{dependent marker} \textit{-e} ‘\textsc{dep}’ are used extensively in the rhetorical structure known as tail-head linkage, whereby the final clause of one sentence is more or less repeated at the start of the following sentence. In these structures, the \isi{final verb} of the first sentence is fully repeated somewhere in the first clause of the second sentence (i.e., it has the same exact \isi{object marker} and \isi{TAM} \isi{suffix}); the addition of the \isi{dependent marker} \textit{-e} ‘\textsc{dep}’, however, allows the clause with the repeated verb to serve as a transition into a new \isi{independent clause}. In tail-head linkage constructions, it is possible for the entire pivot to be repeated exactly, as in \REF{ex:complex:65}.

\ea%65
    \label{ex:complex:65}
          \textit{\textbf{Min nay wambana ndutap}. \textbf{Min nay wambana ndutape} wa namane.}\\
\gll    \textbf{min}  \textbf{na-i}      \textbf{wambana}  \textbf{ndï=uta-p}       \textbf{min} \textbf{na-i}      \textbf{wambana}  \textbf{ndï=uta-p-e}      wa     na-ma-n-e\\
    3\textsc{du}  \textsc{detr-}go.\textsc{pfv}  fish    3\textsc{pl}=grind-\textsc{pfv}  \textsc{3du}    \textsc{detr-}go\textsc{.pfv}  fish    3\textsc{pl}=grind-\textsc{pfv-dep}  village    \textsc{detr-}go-\textsc{ipfv-dep}\\
\glt `The two went and caught fish. After the two went and caught fish, [they] headed home.’ [ulwa011\_00:34]
\z

It is more common, however, for the recapitulatory clause to be a reduced form of its model, eliding, for example, the subject or one or more coordinated \isi{verb phrase}s. Such reductions in tail-head linkage constructions may be seen in examples \REF{ex:complex:66} through \REF{ex:complex:70}.

\ea%66
    \label{ex:complex:66}
          \textit{Mï \textbf{wolka nawow}. \textbf{Wolka nawowe} mï mala yana angla nol.}\\
\gll    mï      \textbf{wolka}  \textbf{na-wow}      \textbf{wolka}  \textbf{na-wow{}-e}     mï      ma=ala    yana  angla  na-lo\\
    \textsc{3sg.subj}  again  \textsc{detr-}sleep.\textsc{ipfv}  again  \textsc{detr-}sleep.\textsc{ipfv{}-dep}    \textsc{3sg.subj}  \textsc{3sg.obj=}for  woman  await  \textsc{detr}{}-go\\
\glt `Again it fell asleep. After again sleeping, it went searching for a wife for him.’ [ulwa006\_04:35]
\z

\ea%67
    \label{ex:complex:67}
          \textit{Mï \textbf{mol wop}. \textbf{Mol wope} yana mï tïnanga lïmndï wa mala.}\\
\gll    mï      \textbf{ma=ul}      \textbf{wo-p}    \textbf{ma=ul}      \textbf{wo-p-e}     yana    mï      tïnanga  lïmndï  wa  ma=ala\\
    3\textsc{sg.subj}  3\textsc{sg.obj}=with  sleep-\textsc{pfv}  \textsc{3sg.obj=}with  sleep\textsc{{}-pfv-dep}    woman    \textsc{3sg.subj}  arise  eye    just  3\textsc{sg.obj}=see\\
\glt `She slept with him. Having slept with him, the woman got up and noticed him.’ [ulwa006\_05:09]
\z

\ea%68
    \label{ex:complex:68}
          \textit{Mat i \textbf{matï nowe ndo malïp}. \textbf{Matï nowe ndo malïpe} mï mawat pe.}\\
\gll    ma=tï      i    \textbf{ma=tï}      \textbf{nowe}    \textbf{anda=u}     \textbf{ma=lï-p}      \textbf{ma=tï}      \textbf{nowe}    \textbf{anda=u}     \textbf{ma=lï-p-e}        mï      ma=wat    p-e\\
    3\textsc{sg.obj}=take  go.\textsc{pfv}  \textsc{3sg.obj}=take  sago.species  \textsc{sg.dist}=from  3\textsc{sg.obj}=put-\textsc{pfv}  3\textsc{sg.obj}=take  sago.species  \textsc{sg.dist}=from    3\textsc{sg.obj}=put-\textsc{pfv-dep}  3\textsc{sg.subj}  3\textsc{sg.obj}=atop  be\textsc{{}-ipfv}\\
\glt `[It] brought him and put him on a sago palm. Having put him on the sago palm, he stayed atop it.’ [ulwa006\_01:26]
\z

\ea%69
\label{ex:complex:69}
\textit{Kowe mol anmbi \textbf{nïmal mbi}. \textbf{Nïmal mbiye} anmbïwap.}\\
\gll    Kowe  ma=ul      an-mbï-i      \textbf{nïmal}  \textbf{mbï-i}      \textbf{nïmal}     \textbf{mbï-i-e}      an-mbï-wap\\
    [name]  3\textsc{sg.obj}=with  out-here-go.\textsc{pfv}  river  here-go.\textsc{pfv}  river    here-go.\textsc{pfv-dep}  out-here-be.\textsc{pst}\\
\glt `[We] came out with Kowe, came here to the river. After coming here to the river, [we] stayed here.’ [ulwa013\_11:05]
\z

\newpage

\ea%70
    \label{ex:complex:70}
          \textit{Alkumot yana mï alum mokotïp mat \textbf{al malp}. \textbf{Al malpe} mï i.}\\
\gll    Alkumot  yana    mï      alum  ma=kot-p     ma=tï      \textbf{al}  \textbf{ma=lï{}-p}      \textbf{al}  \textbf{ma=lï{}-p-e}   mï      i\\
    [name]    woman    \textsc{3sg.subj}  child  3\textsc{sg.obj}=break-\textsc{pfv}    3\textsc{sg.obj=}take  net  3\textsc{sg.obj}=put-\textsc{pfv}  net  \textsc{3sg.obj=}put\textsc{{}-pfv-dep}  \textsc{3sg.subj}  go.\textsc{pfv}\\
\glt `The woman Alkumot bore the child and put it in a mosquito net. Having put it in the mosquito net, she went.’ [ulwa001\_00:36]
\z

It is also possible for multiple verbs in a single \isi{verb phrase} to be repeated in tail-head linkage patterns, as in \REF{ex:complex:71}.

\ea%71
    \label{ex:complex:71}
          \textit{Wondi mï i mawat \textbf{inmi may}. \textbf{Inmi maye} mï mïnda mokotïp \textbf{mat li lïp}. \textbf{Mat li lïpe} mï inmi mawap.}\\
\gll    wondi    mï      i    ma=wa    \textbf{inmi}  \textbf{ma=i}     \textbf{inmi}  \textbf{ma=i-e}        mï      mïnda     ma=kot-p        \textbf{ma=tï}      \textbf{li}    \textbf{lï-p}      \textbf{ma=tï}     \textbf{li}    \textbf{lï-p-e}      mï      inmi  ma=wap\\
    bandicoot  \textsc{3sg.subj}  go.\textsc{pfv}  3\textsc{sg.obj}=atop  hole  \textsc{3sg.obj}=go.\textsc{pfv}    hole  3\textsc{sg.obj}=go.\textsc{pfv-dep}  3\textsc{sg.subj}  banana    \textsc{3sg.obj}=break-\textsc{pfv}  \textsc{3sg.obj=}take  down  put\textsc{{}-pfv}  \textsc{3sg.obj=}take    down  put-\textsc{pfv-dep}  \textsc{3sg.subj}  hole  3\textsc{sg.obj}=be.\textsc{pst}\\
\glt `The bandicoot went onto her in the hole. [After it] went into the hole, he cut the banana tree and put it down. When [he] put it down, she was [still] in the hole.’ [ulwa001\_02:17]
\z

As example \REF{ex:complex:71} illustrates, it is possible for such chains of dependent and \isi{independent clause}s to continue for linkages of longer than two sentences.

\is{dependent clause|)}
\is{subordinate clause|)}
\is{subordination|)}
\is{complex sentence|)}
\is{tail-head linkage|)}

\subsection{Dependent markers for floor-holding}\label{sec:12.2.6}

\is{complex sentence|(}
\is{dependent clause|(}
\is{subordinate clause|(}
\is{subordination|(}
\is{floor-holding|(}
\is{dependent marker|(}

It is also common for seemingly \isi{independent clause}s to receive the \isi{dependent marker} \textit{-e} ‘\textsc{dep}’. In this way, when added almost as an afterthought, this \isi{suffix} can serve a sort of coordinating function, equivalent almost to a \isi{conjunction} ‘and’ in use. By affixing \textit{-e} ‘\textsc{dep}’ to the end of a clause (and in so doing signaling that another clause is to follow), a speaker may have a better chance at holding the floor. Indeed, some speakers commonly insert the sound [-e] in the silence following a clause to signal that they are not yet done talking, as in examples \REF{ex:complex:72}, \REF{ex:complex:73}, and \REF{ex:complex:74}.

\ea%72
    \label{ex:complex:72}
          \textit{Rays muku kot nïn ani lïp.} \textbf{\textit{E}} \textit{Dora lïmndï nala.}\\
\gll    rays  muku    ko=tï    nï=n    ani    lï-p      \textbf{e}     Dora  lïmndï  nï=ala\\
    rice  package  \textsc{indf}=take  1\textsc{sg=obl}  bilum  put-\textsc{pfv}  \textsc{dep}    [name]  eye    1\textsc{sg}=see\\
\glt `[He] put a package of rice into my \textit{bilum} [= string bag]. And Dora saw me.’ (\textit{rays} = TP \textit{rais}) [ulwa014\_29:27]
\z

\ea%73
    \label{ex:complex:73}
          \textit{Min mat i pul ko i matlïp wulïnup.} \textbf{\textit{E}} \textit{wolka tïnanga matïn mat.}\\
\gll    min  ma=tï      i    pul    ko  i    ma=tï lï-p      wulïn-u-p    \textbf{e}    wolka  tïnanga  ma=tï-n     ma=tï\\
    3\textsc{du}  3\textsc{sg.obj}=take  go.\textsc{pfv}  piece  one  go.\textsc{pfv}  \textsc{3sg.obj=}take    put\textsc{{}-pfv} rest-put-\textsc{pfv}  \textsc{dep}  again  arise  3\textsc{sg.obj}=take-\textsc{pfv}    3\textsc{sg.obj}=take\\
\glt `The two brought it, went to a place, put it down, and rested. And then [they] got up again, got it, were getting it.’\footnote{The use of the noun \textit{pul} ‘piece’ to mean ‘place’ may be \isi{calque}d from TP \textit{hap} ‘piece, place’.} [ulwa035\_04:20]
\z

\ea%74
    \label{ex:complex:74}
          \textit{Ndï maka lop.} \textbf{\textit{E}} \textit{ndï we ndïmokop.} \textbf{\textit{E}} \textit{ndï mbïlop.}\\
\gll    ndï  maka  lo-p  \textbf{e}    ndï   we    ndï=moko-p  \textbf{e}    ndï     mbï-lo-p\\
    \textsc{3pl}  thus  go-\textsc{pfv}  \textsc{dep}  \textsc{3pl}    sago  \textsc{3pl=}take\textsc{{}-pfv}  \textsc{dep}  \textsc{3pl}    here-go-\textsc{pfv}\\
\glt `They went like that. And then they got the sago starch. And then they came here.’ [ulwa037\_63:29]
\z

In similar fashion, the form [pe] is sometimes used. I take this to be the \isi{locative verb} \textit{p-} ‘be’ plus the \isi{dependent marker} \textit{-e} ‘\textsc{dep}’. Thus this structure roughly means something like ‘that being [the case]’ and can, accordingly, function as a connector between clauses or as a \isi{floor-holding} \isi{particle}. Its use is illustrated in \REF{ex:complex:75}, \REF{ex:complex:76}, and \REF{ex:complex:77}.

\is{dependent marker|)}
\is{floor-holding|)}
\is{subordination|)}
\is{subordinate clause|)}
\is{dependent clause|)}
\is{complex sentence|)}

\ea%75
    \label{ex:complex:75}
          \textit{We ndït anmbï mbi Taw mbi.} \textbf{\textit{Pe}} \textit{Brian manji inom mï wolka tïklika lïmndï tïn mala.}\\
\gll    we    ndï=tï    an-mbï    mbï-i      Taw  mbï-i \textbf{p-e}    Brian  ma-nji      inom  mï      wolka  tïkli-ka     lïmndï  tïn    ma=ala\\
    sago  3\textsc{pl}=take  out-here  here-go.\textsc{pfv}  [place]  here-go.\textsc{pfv}    be-\textsc{dep}  [name]  3\textsc{sg.obj-poss}  mother  \textsc{3sg.subj}  again  turn-let    eye    dog  3\textsc{sg.obj}=see\\
\glt `[They] brought sago starch out there, went there to Taw. And [after they had gone] Brian’s mother turned back and saw the dog.’ [ulwa037\_61:14]
\z

\ea%76
    \label{ex:complex:76}
          \textit{Ndï mape malep amun wa mbïlop.} \textbf{\textit{Pe}} \textit{nï tïnanga anmbï mbi.}\\
\gll    ndï  ma=p-e      ma=ale-p        amun  wa mbï-lo-p    \textbf{p-e}    nï    tïnanga  an-mbï    mbï-i\\
    3\textsc{pl}  \textsc{3sg.obj=}be\textsc{{}-dep} 3\textsc{sg.obj}=scrape\textsc{{}-pfv} now  village    here-go-\textsc{pfv}  be-\textsc{dep}  \textsc{1sg}  arise  out-here  here-go.\textsc{pfv}\\
\glt `They were there scraping it and now came home. And then I got up and came out here.’ [ulwa040\_01:29]
\z

\ea%77
    \label{ex:complex:77}
\is{dependent marker}
\is{floor-holding}
\is{subordination}
\is{subordinate clause}
\is{dependent clause}
\is{complex sentence}
          \textit{Ndï ango anmap tembip.} \textbf{\textit{Pe}} \textit{ndï nena.}\\
\gll    ndï  ango  anma=p  tembi=p  \textbf{p-e}    ndï  na-i-na\\
    \textsc{3pl}  \textsc{neg}  good=\textsc{cop}  bad=\textsc{cop}  be-\textsc{dep}  \textsc{3pl}  \textsc{detr-}come\textsc{{}-irr}\\
\glt `They were not healthy, but sick. And [when they were sick] they would come.’ [ulwa029\_09:34]
\z

\subsection{Other means of subordination}\label{sec:12.2.7}

\is{complex sentence|(}
\is{dependent clause|(}
\is{subordinate clause|(}
\is{subordination|(}

In addition to the \isi{dependent-marker} \isi{suffix} \textit{-e} ‘\textsc{dep}’ and the afterthought-like forms [e] and [pe], there is a form \textit{we} ‘(and) then’, which can connect clauses. It is often used in \isi{conditional} statements, occurring between the verb of the \isi{apodosis} (marked by the \isi{conditional} \isi{suffix} \textit{-ta} ‘\textsc{cond}’) and the start of the \isi{protasis} \REF{ex:complex:78}. \isi{Phonological}ly (i.e., in terms of \isi{prosodic unit}s), this word \textit{we} ‘then’ belongs to the \isi{apodosis}.

\ea%78
    \label{ex:complex:78}
          \textit{Ndï ita} \textbf{\textit{we}} \textit{unan matïna.}\\
\gll    ndï  i-ta        \textbf{we}    unan    ma=atï-na\\
    3\textsc{pl}  go.\textsc{pfv-cond}  then  \textsc{1pl.incl}  3\textsc{sg.obj}=hit-\textsc{irr}\\
\glt `If they come, then we will kill him.’ [ulwa001\_15:24]
\z

This form may occur in other sentence types besides just \isi{conditional sentence}s, however. Sometimes it is not perfectly clear whether it is a separate lexeme (i.e., \textit{we} ‘then’) or an elongated version of the \isi{dependent marker} \textit{-e} ‘\textsc{dep}’.

  The word \textit{we} ‘(and) then’ also functions like a \isi{coordinator}. It may be used to connect sentences in discourse, helping the speaker to \is{floor-holding} hold the floor. Examples \REF{ex:complex:79} and \REF{ex:complex:80} illustrate the use of \textit{we} ‘then’ in connecting \isi{independent clause}s.

\is{subordination|)}
\is{subordinate clause|)}
\is{dependent clause|)}
\is{complex sentence|)}

\newpage

\ea%79
    \label{ex:complex:79}
          \textit{Utam ndïn mankap} \textbf{\textit{we}} \textit{Kowe mangusuwa amun ngolop.}\\
\gll    utam  ndï=n  ma=nïkï-p        \textbf{we}    Kowe  ma-ngusuwa amun  nga=u-lo-p\\
    yam  3\textsc{pl=obl}  3\textsc{sg.obj}=dig-\textsc{pfv}  then  [name]  3\textsc{sg.obj-}poor    now  \textsc{sg.prox}=from-cut-\textsc{pfv}\\
\glt `[I] planted yams there and then Kowe, the poor thing, only recently cleared this place.’ [ulwa014\_53:05]
\z

\ea%80
    \label{ex:complex:80}
\is{subordination}
\is{subordinate clause}
\is{dependent clause}
\is{complex sentence}
          \textit{Mundu wanata ndangla lumop ndï anmbi} \textbf{\textit{we}} \textit{nalanda.}\\
\gll    mundu  wana-ta    ndï=angla  lumo-p    ndï  an-mbï-i     \textbf{we}    na-la-nda\\
    food  cook-\textsc{cond}  3\textsc{pl}=await  put-\textsc{pfv}  \textsc{3pl}  out-here-go.\textsc{pfv}    then  \textsc{detr-}eat-\textsc{irr}\\
\glt `Once [you] have cooked food and put it [there] for them, they will come out and then eat.’ [ulwa030\_00:39]
\z

\section{Relative clauses}\label{sec:12.3}

\is{relative clause|(}
\is{complex sentence|(}
\is{dependent clause|(}
\is{subordinate clause|(}
\is{subordination|(}

 In Ulwa, there is no overt \isi{morphological} marker for relative clauses -- that is, there are no \isi{relative pronoun}s or \isi{relativizer}s, nor are there \isi{resumptive pronoun}s or other means of overtly coreferencing an argument in the \isi{relative clause} with an argument in the \isi{matrix clause}. A \isi{relative clause} immediately precedes the \isi{head noun} of the \isi{matrix clause}, and the verb in the \isi{relative clause} is marked for \isi{TAM} as any finite verb in a clause would be. Thus I analyze relative clauses as prenominal \isi{dependent clause}s with unexpressed subjects.\footnote{Relative clauses in Ulwa may thus be said to employ the \isi{gap strategy}, since the \isi{syntactic} spot where the \isi{head noun} of the \isi{antecedent} clause should be found in the \isi{relative clause} (i.e., before the verb) is empty (i.e., there is no overt \isi{phonological} form).} The basic structure of Ulwa relative clauses is outlined in \REF{ex:complex:80a}.

 As an argument in the \isi{matrix clause}, the \isi{head noun} of the \isi{matrix clause} may fulfill any \isi{grammatical relation} -- that is, it may be a subject, an object, or an \isi{oblique}. The \isi{noun phrase} in the \isi{relative clause} that refers to this \isi{antecedent}, however, must be the grammatical subject of the clause. Thus, viewed crosslinguistically in terms of the \is{hierarchy} \isi{accessibility hierarchy} (\citealt{KeenanComrie1977}), Ulwa has a rather limited set of grammatically possible \isi{relative clause} constructions, as only subjects can be \isi{relativize}d.

\newpage

\ea%80a
    \label{ex:complex:80a}
The structure of relative clauses\\
\begin{tabbing}
{( )} \= {( )} \= {(\isi{intransitive} \isi{relative clause}:)} \= {(S[OV]XOV)}\kill
{\isi{relative clause} modifying the subject of an \isi{intransitive} clause:}\\
{ } \> { } \> {\isi{intransitive} \isi{relative clause}:} \> {[V]SV}\\
{ } \> { } \> {\isi{transitive} \isi{relative clause}:} \> [{OV]SV}\\
{\isi{relative clause} modifying an \isi{oblique argument} in an \isi{intransitive} clause:}\\
{ } \> { } \> {\isi{intransitive} \isi{relative clause}:} \> {S[V]XV}\\
{ } \> { } \> {\isi{transitive} \isi{relative clause}:} \> {S[OV]XV}\\
{\isi{relative clause} modifying the subject of a \isi{transitive} clause:}\\
{ } \> { } \> {\isi{intransitive} \isi{relative clause}:} \> {[V]SOV}\\
{ } \> { } \> {\isi{transitive} \isi{relative clause}:} \> {[OV]SOV}\\
{\isi{relative clause} modifying the object of a \isi{transitive} clause:}\\
{ } \> { } \> {\isi{intransitive} \isi{relative clause}:} \> {S[V]OV}\\
{ } \> { } \> {\isi{transitive} \isi{relative clause}:} \> {S[OV]OV}\\
{\isi{relative clause} modifying an \isi{oblique argument} in a \isi{transitive} clause:}\\
{ } \> { } \> {\isi{intransitive} \isi{relative clause}:} \> {S[V]XOV}\\
{ } \> { } \> {\isi{transitive} \isi{relative clause}:} \> {S[OV]XOV}
\end{tabbing}
\z

  There are no \isi{correlative relative clause}s in Ulwa, nor are there \isi{adjoined relative clause}s (i.e., \isi{non-adjacent relative clause}s).\footnote{In an alternative analysis relative clauses in Ulwa may be considered \is{head-internal relative clause} head-internal, with the \isi{head} being expressed as a full NP only within the \isi{relative clause}, namely postverbally. A schematization of the head-internal analysis is given in \REF{ex:syntax:278} in \sectref{sec:13.7}. In this analysis, relative clauses would exhibit a different \isi{word order} from that found in most Ulwa clauses: whereas the \isi{word order} of \isi{pragmatic}ally neutral \is{active voice} active clauses is S(O)V (\sectref{sec:11.1}), the \isi{word order} of relative clauses would be (O)VS. While this analysis seems typologically unusual, it perhaps has some support when considered alongside Ulwa’s \isi{passive} constructions (\sectref{sec:13.7}). However, it does not seem best to analyze the \isi{head} as being internal, since it takes grammatical marking according to its role in the \isi{matrix clause}, not according to what its role would be within the \isi{embedded clause}. For example, we find the \isi{object marker} \textit{ma=} ‘\textsc{3sg.obj}’ as opposed to the \isi{subject marker} \textit{mï} ‘\textsc{3sg.subj}’ in sentences such as \REF{ex:complex:82}. Thus, it seems best to me to analyze relative clauses, like other subordinate clauses, as maintaining canonical S(O)V order and employing a \isi{gap strategy}.}

  Example \REF{ex:complex:81} consists of a simple \isi{intransitive} sentence. The \isi{word order} is the canonical SV.

\ea%81
    \label{ex:complex:81}
          \textit{Itom ngata mï nip.}\\
\gll    itom  ngata  mï      ni-p\\
    father  grand  \textsc{3sg.subj}  die-\textsc{pfv}\\
\glt `The old man died.’ [elicited]
\z

Example \REF{ex:complex:82} shows how the sentence given in \REF{ex:complex:81} might appear in a \isi{relative clause}. Here, \textit{itom ngata} ‘old man’ is both the subject of the \isi{relative clause} and the object of the \isi{matrix clause}. The brackets in \REF{ex:complex:82} enclose the \isi{relative clause}. Thus, sentence \REF{ex:complex:82} is considered to contain a noun-modifying clause, the verb \textit{nipe} ‘died’ thus constituting the entire \isi{relative clause}, with a gap for the subject occurring immediately before the verb.

\ea%82
    \label{ex:complex:82}
          \textit{Nï nipe itom ngata makamp.}\\
\gll    nï    [ni-p-e]    itom  ngata  ma=kamb-p\\
    1\textsc{sg}  [die-\textsc{pfv-dep]}  father  grand  3\textsc{sg.obj}=shun-\textsc{pfv}\\
\glt `I avoided the old man who died.’ [elicited]
\z

Note also that the \isi{dependent marker} \textit{-e} ‘\textsc{dep}’ is employed on the verb in the \isi{dependent} \isi{relative clause} (\sectref{sec:12.2.1}). This lends further support to the idea that the structure in question is indeed a clause.

  A \isi{relative clause} can also serve as the subject of a \isi{matrix clause}, as in \REF{ex:complex:83}.

\ea%83
    \label{ex:complex:83}
          \textit{Nipe itom ngata mï ankam anma.}\\
\gll    [ni-p-e]    itom  ngata  mï      ankam  anma\\
    [die-\textsc{pfv}{}-\textsc{dep]}  father  grand  \textsc{3sg.subj}  person  good\\
\glt `The old man who died is a good person.’ [elicited]
\z

Note that \isi{verb phrase}s that consist of \isi{discontinuous} elements (i.e., \isi{separable verb}s, \sectref{sec:9.2.1}) will surround the \isi{relative clause} if the \isi{relative clause} is the object of the \isi{verb phrase} \REF{ex:complex:84}.

\ea%84
    \label{ex:complex:84}
          \textit{Ndï \textbf{lïmndï} nipe \textbf{itom ngata mala}.}\\
\gll ndï  \textbf{lïmndï}  [ni-p-e]    \textbf{itom}  \textbf{ngata}  \textbf{ma=ala}\\
    3\textsc{pl}  eye    [die{}-\textsc{pfv-dep]}  father  grand  3\textsc{sg.obj}=see\\
\glt `They saw the old man who died.’ [elicited]
\z

Finally, it may be shown that, in addition to subjects and objects, relative clauses may function as \isi{oblique argument}s within \isi{matrix clause}s, such as objects of \isi{postposition}s, as in \REF{ex:complex:85}.

 \ea\label{ex:complex:85}  \textit{Damnda mï nipe itom ngata maya i.}\\
\gll    Damnda  mï      [ni-p-e]    itom  ngata  ma=iya i\\
    [name]    3\textsc{sg.subj}  [die-\textsc{pfv-dep]}  father  grand  3\textsc{sg.obj}=toward    go.\textsc{pfv}\\
\glt `Damnda went to the old man who died.’ [elicited]
\z

Just like \isi{intransitive} clauses, \isi{transitive} clauses may function as relative clauses. Example \REF{ex:complex:86} illustrates a simple \isi{transitive} sentence. The \isi{word order} is SOV.

\ea%86
    \label{ex:complex:86}
          \textit{Ankam mï lamndu masap.}\\
\gll    ankam  mï      lamndu  ma=asa-p\\
    person  3\textsc{sg.subj}  pig      3\textsc{sg.obj}=hit-\textsc{pfv}\\
\glt `The person killed the pig.’ [elicited]
\z

\is{subordination|)}
\is{subordinate clause|)}
\is{dependent clause|)}
\is{complex sentence|)}
\is{relative clause|)}

\is{relative clause|(}
\is{complex sentence|(}
\is{dependent clause|(}
\is{subordinate clause|(}
\is{subordination|(}

This transitive-verb sentence may serve as the object of a verb in a \isi{matrix clause}, as in sentence \REF{ex:complex:87}, which has a \isi{relative clause} exhibiting the \isi{word order} [S]OV (where “[S]” represents a gap in the clause).

\ea%87
    \label{ex:complex:87}
          \textit{Damnda mï lïmndï lamndu masape ankam mala.}\\
\gll Damnda  mï      lïmndï  [lamndu  ma=asa-p-e]      ankam     ma=ala\\
    [name]    3\textsc{sg.subj}  eye    [pig    3\textsc{sg.obj}=hit-\textsc{pfv-dep]}  person    3\textsc{sg.obj}=see\\
\glt `Damnda saw the person who killed the pig.’ [elicited]
\z

Again note the use of the \isi{dependent marker} \textit{-e} ‘\textsc{dep}’ \isi{suffix}ed to the verb in the \isi{relative clause}.

Sentence \REF{ex:complex:88} is an example of a transitive-verb \isi{relative clause} serving as the subject of a \isi{matrix clause}. Note the use of the \isi{subject marker} \textit{mï} ‘\textsc{3sg.subj}’.

\ea%88
    \label{ex:complex:88}
          \textit{Lamndu masape ankam mï wandam may.}\\
\gll    {[lamndu}  ma=asa-p-e]      mï      wandam ma=i\\
    {[pig}    \textsc{3sg.obj}=hit-\textsc{pfv-dep]}  \textsc{3sg.subj}  jungle    3\textsc{sg.obj}=go.\textsc{pfv}\\
\glt `The person who killed the pig went to the jungle.’ [elicited]
\z

Sentence \REF{ex:complex:89} is an example of a transitive-verb \isi{relative clause} serving as an \isi{oblique argument} within the \isi{matrix clause}.

\ea%89
    \label{ex:complex:89}
          \textit{Sinda mï lamndu masape ankam maya i.}\\
\gll    Sinda  mï      [lamndu  ma=asa-p-e]      ankam ma=iya      i\\
    [name]  3\textsc{sg.subj}  [pig    3\textsc{sg.obj}=hit-\textsc{pfv-dep]}  person    3\textsc{sg.obj}=toward  go.\textsc{pfv}\\
\glt `Sinda went to the person who killed the pig.’ [elicited]
\z

It is possible for \isi{oblique}s to occur within the \isi{dependent} \isi{relative clause}s as well, whether they contain a \isi{transitive} verb \REF{ex:complex:90} or an \isi{intransitive} verb \REF{ex:complex:91}.

\ea%90
    \label{ex:complex:90}
          \textit{Mï lïmndï \textbf{mananï} lamndu masape ankam mala.}\\
\gll    mï      lïmndï  [\textbf{mana=nï}    lamndu  ma=asa-p-e] ankam  ma=ala\\
    3\textsc{sg.subj}  eye    [spear=\textsc{obl}  pig      3\textsc{sg.obj}=hit-\textsc{pfv-dep]}    person  3\textsc{sg.obj}=see\\
\glt `She saw the man who stabbed the pig with the spear.’ [elicited]
\z

\ea%91
    \label{ex:complex:91}
          \textit{Mï lïmndï \textbf{ankam ul} natane yana mala.}\\
\gll    mï      lïmndï  [\textbf{ankam}  \textbf{ul}    na-ta-n-e]        yana   ma=ala\\
    3\textsc{sg.subj}  eye    [person  with  \textsc{detr-}say\textsc{{}-ipfv-dep]} woman    \textsc{3sg.obj}=see\\
\glt `She saw the woman who is talking with the man.’ [elicited]
\z



Relative clauses occur rarely in discourse, and some speakers (especially the younger ones) probably never employ them. It could the case that these fairly complex \isi{syntactic} structures are being lost as the language experiences \isi{grammatical attrition} due to rapid replacement by \ili{Tok Pisin}, a language that also -- for many speakers -- has no formal structures for \isi{relativization} (see \chapref{sec:15}). Nevertheless, relative clauses do occasionally occur in the speech of some older speakers. Sentences \REF{ex:complex:92} through \REF{ex:complex:96} provide examples of relative clauses taken from texts.

\ea%92
    \label{ex:complex:92}
          \textit{Ndï manji mawl anmbiye ndï kwa masap.}\\
\gll    ndï  [ma-nji      ma=ul      an-mbï-i-e]      ndï  kwa     ma=asa-p\\
    3\textsc{pl}  [3\textsc{sg.obj-poss}  3\textsc{sg.obj}=with  out-here-go.\textsc{pfv-dep]}  3\textsc{pl}  one    3\textsc{sg.obj}=hit-\textsc{pfv}\\
\glt `They killed one [of] his [brothers] who came along with him.’ [ulwa001\_12:39]
\z

\ea%93
    \label{ex:complex:93}
          \textit{Awal men pe nji ndïkuklïp.}\\
\gll    awal    [ma=in      p-e]    nji    ndï=kuk-lï-p\\
    yesterday  [3\textsc{sg.obj}=in  be\textsc{{}-dep]} thing  3\textsc{pl}=gather-put-\textsc{pfv}\\
\glt `Yesterday [we] gathered [our] things that were in it [= the house].’ [ulwa042\_05:16]
\z

\ea%94
    \label{ex:complex:94}
          \textit{Anga mape numïni mï angani mape.}\\
\gll    {[anga}  {ma=p-e]}      numïni  mï      angani  ma=p-e\\
    {[side}  3\textsc{sg.obj}=be\textsc{{}-dep]} ditch  \textsc{3sg.subj}  behind  \textsc{3sg.obj}=be\textsc{{}-ipfv}\\
\glt `The ditch that is on the other side [of the river] is behind it.’ [ulwa014†]
\z

\ea%95
    \label{ex:complex:95}
          \textit{Nul mbiye yanat mambi umbenam nay.}\\
\gll    {[nï=ul}    {mbï-i-e]}      yanat    ma-ambi    umbenam na-i\\
    {[1\textsc{sg}=with}  here-go.\textsc{pfv-dep]}  daughter  3\textsc{sg.obj-top}  morning    \textsc{detr{}-}go.\textsc{pfv}\\
\glt `As for the daughter who came with me, she left this morning.’ [ulwa032\_11:01]
\z

\ea%96
    \label{ex:complex:96}
          \textit{Apa mbïpe itom inom min luke nji ulwap.}\\
\gll    {[apa}  {mbï-p-e]}    itom  inom  min  luke  nji ulwa=p\\
    {[house}  here-be\textsc{{}-dep]} father  mother  \textsc{3du}  too    thing    nothing=\textsc{cop}\\
\glt `The two home-owners have nothing either.’ (Literally ‘The father and mother who are in the house here, too, have no things.’) [ulwa032\_20:01]
\z

One possible reason for the relative rarity of these constructions in discourse is that fact that the \isi{pragmatic} function of relative clauses can be assumed by \isi{nominalization} (\sectref{sec:12.3.1}), of which speakers tend to make more frequent use.\footnote{Relative clauses may have their historical origins in \isi{nominalized verb phrase}s. The formal distinction between the two is slight, basically hinging on the presence (in \isi{nominalization}) versus the absence (in \isi{relativization}) of a final /-n/. It is thus possible that in some examples the sound has simply been \isi{elide}d. Still, based on speaker perceptions and on the careful pronunciations of elicited sentences, these are treated (at least synchronically) as two separate structures: \isi{nominalized verb phrase}s and relative clauses.} Furthermore, speakers may employ \isi{paratactic relative clause}s as an alternative to this more complicated \isi{syntactic} structure (\sectref{sec:12.3.2}).

\is{subordination|)}
\is{subordinate clause|)}
\is{dependent clause|)}
\is{complex sentence|)}
\is{relative clause|)}

\subsection{Nominalized verb phrases}\label{sec:12.3.1}

\is{nominalized verb phrase|(}
\is{nominalization|(}
\is{relative clause|(}
\is{complex sentence|(}


Nominalized \isi{verb phrase}s may serve the \isi{pragmatic} function of relative clauses. Examples such as \REF{ex:nouns:18} and \REF{ex:nouns:19} in \sectref{sec:3.2} illustrate how \isi{nominalized verb phrase}s may function similarly to relative clauses. Often, these nominalized forms are used with \isi{locative verb}s, as in \REF{ex:complex:97}.


\ea%97
    \label{ex:complex:97}
        \textit{\textbf{Wandam wapen} ndï wa nen.}\\
\gll    {[wandam}  {wap-\textbf{en}]}    ndï  wa    na-i-n\\
    {[jungle}    be.\textsc{pst-nmlz]}  \textsc{3pl}  village  \textsc{detr}{}-come-\textsc{pfv}\\
\glt `Those who were in the jungle came home.’ (Literally ‘the having-been-in-the-jungle [people] …’) [ulwa018\_04:09]
\z

The \isi{verb phrase} that is nominalized may consist of more than one verb \REF{ex:complex:98}. Only the \isi{final verb} receives the nominalizing \isi{morphology}.

\is{complex sentence|)}
\is{relative clause|)}
\is{nominalization|)}
\is{nominalized verb phrase|)}

\ea%98
    \label{ex:complex:98}
          \textit{\textbf{Ata ngape wowen} anda mo anmbunde.}\\
\gll    {[ata}  nga=p-e      {wow{}-\textbf{en}]}      anda     ma=u      an-mbï-unda-e\\
    {[up}    \textsc{sg.prox}=be\textsc{{}-dep} sleep.\textsc{ipfv{}-nmlz]}  \textsc{sg.dist}    3\textsc{sg.obj}=from  out-here-go-\textsc{ipfv}\\
\glt `That one who lives upstream is coming around here from there.’ (Literally ‘that sleeping-up-(in)-this-(place) [person] …’) [ulwa032\_15:13]
\z

The nominalized \isi{phrase} may have its own object NP, as exemplified by \REF{ex:complex:99}.

\ea%99
    \label{ex:complex:99}
\is{complex sentence}
\is{relative clause}
\is{nominalization}
\is{nominalized verb phrase}
          \textit{\textbf{Tïrïngïn inen} i man nït.}\\
\gll    {[Tïrïngïn}  {ina-\textbf{en}]}    i    ma=n      nï=ta\\
    {[[name]}  get-\textsc{nmlz]}  go.\textsc{pfv}  3\textsc{sg.obj=obl}  1\textsc{sg}=say\\
\glt `The one who married Tïrïngïn came and told me.’ (Literally ‘the Tïrïngïn-getting [one]’) [ulwa014\_21:16]
\z


\subsection{Paratactic relative clauses}\label{sec:12.3.2}

\is{parataxis|(}
\is{paratactic relative clause|(}
\is{relative clause|(}
\is{complex sentence|(}

There is yet another means of accomplishing the \isi{pragmatic} task of narrowing the reference of a noun. In addition to relative clauses (\sectref{sec:12.3}) and \isi{nominalized verb phrase}s (\sectref{sec:12.3.1}), speakers of Ulwa can make use of paratactic relative clauses (\citealt{ComrieKuteva2013}). In these constructions, there is no formal \isi{morphological} or \isi{syntactic} \isi{relativization}; rather, what could otherwise be expressed as \isi{matrix clause}s with \is{embedded clause} embedded relative clauses are here expressed by sets of two paratactically \isi{juxtaposed} clauses. Sentences \REF{ex:complex:100} through \REF{ex:complex:103} provide examples of paratactic relative clauses. The clauses in each example are enclosed in brackets.

\ea%100
    \label{ex:complex:100}
          \textit{Tembi la ndï wa mbïp.}\\
\gll    {[tembi}  {ala]}    {[ndï}  wa    {mbï-p]}\\
    {[bad}  \textsc{pl.dist]}  \textsc{[3pl}  village  here-be]\\
\glt `Those people here in the village are bad.’ (Literally ‘Those [people] are bad; they are here in the village.’) [ulwa032\_47:06]
\z

\ea%102
    \label{ex:complex:102}
          \textit{Ango mundu kom un mat nïnan!}\\
\gll    {[ango}  mundu  {kom]}  {[un}    ma=tï      {nï=na-n]}\\
    {[\textsc{neg}}  food  \textsc{neg]}  \textsc{[2pl}  \textsc{3sg.obj}=take  \textsc{1sg}=give-\textsc{pfv]}\\
\glt `That’s not food you gave me!’ (Literally ‘Not food; you gave it to me.’) [ulwa020\_02:04]
\z

\ea%101
    \label{ex:complex:101}
          \textit{Anda nji tembi wa mï unaniya wa ine.}\\
\gll    {[anda}    nji    tembi  {wa]}  {[mï}      unan=iya wa    {i-n-e]}\\
    {[\textsc{sg.dist}}  thing  bad    {just]}  [3\textsc{sg.subj}  1\textsc{pl.incl}=toward    village  come-\textsc{pfv-dep]}\\
\glt `That’s a bad thing that’s come to our village.’ (Literally ‘That is a bad thing; it has come to us, to the village.’ There is no \isi{prosodic break} between the clauses.) [ulwa037\_20:57]
\z

\ea%103
    \label{ex:complex:103}
        \textit{Numbu anma nda u mole.}\\
\gll    {[numbu}  anma  {anda]}    {[u}    {ma=lo-e]}\\
    [garamut  good  \textsc{sg.dist]}  \textsc{[2sg}  \textsc{3sg.obj}=cut-\textsc{ipfv]}\\
\glt `That’s a good \textit{garamut} drum that you’re carving.’ (Literally ‘That is a good \textit{garamut}; you are carving it.’) [ulwa009\_02:08]
\z

These paratactic relative clauses are, for some speakers, the exclusive means of creating relative-clause-like structures -- that is, they lack the formal relative clauses described in \sectref{sec:12.3}. It is possible that paratactic relative clauses are a relatively recent \isi{syntactic} innovation, having emerged as the formal \isi{relative clause} structures have become obscure to younger speakers (see \chapref{sec:15}).

\is{complex sentence|)}
\is{relative clause|)}
\is{paratactic relative clause|)}
\is{parataxis|)}

\section{Clause chaining?}\label{sec:12.4}

\is{complex sentence|(}
\is{clause chain|(}
\is{clause chaining|(}

Whereas \isi{dependent-marked} clauses do not participate in prototypical clause-chaining structures, there is one verb, \textit{tï-} ‘take’, which does appear to behave as a \isi{medial verb}, albeit in a very restricted way. It may occur as the first of two verbs within a series of two clauses, taking its own object argument, but lacking \isi{TAM} marking, as in \REF{ex:complex:104}.

\ea%104
    \label{ex:complex:104}
          \textit{Un ango ame \textbf{tï inde}.}\\
\gll un    ango  ame    \textbf{tï}    \textbf{inda-e}\\
    2\textsc{pl}    \textsc{neg}  basket  take  walk-\textsc{ipfv}\\
\glt `You don’t carry baskets around.’ [ulwa014\_34:31]
\z

It is rare for \textit{tï-} ‘take’ to occur with other \isi{medial verb}s (if it itself may be considered to be a \isi{medial verb}); rather, such “clause chains” formed with \textit{tï-} ‘take’ are generally restricted to this verb plus one verb in an immediately following final clause.\footnote{Often, \textit{tï-} ‘take’ is \isi{phonological}ly reduced to [t] in such constructions. The verb is perhaps in the diachronic process of \isi{grammaticalizing} to become a \isi{postposition}.} The most common \isi{medial-clause} use of \textit{tï-} ‘take’ is found in ‘giving’ constructions, which typically use \textit{na-} ‘give’ as the verb in the final clause (see \sectref{sec:11.3} for examples as well as further discussion on other possible “\isi{ditransitive}” constructions in the language). It may also be used with a \isi{motion} verb (typically \textit{inda-} ‘walk’) as the \isi{final verb}, giving the meaning ‘carry’, as in \REF{ex:complex:104}, and further illustrated by \REF{ex:complex:105} and \REF{ex:complex:106}.

\ea%105
    \label{ex:complex:105}
          \textit{Nï ul unji alum nïpat \textbf{ngat indape}.}\\
\gll nï    u-lo    un-nji    alum  nïpat   \textbf{nga=tï}     inda-p-e\\
    1\textsc{sg}  from-go  2\textsc{sg-poss}  child  giant  \textsc{sg.prox}=take    walk-\textsc{pfv-dep}\\
\glt `I carried that giant daughter of yours around.’ [ulwa032\_17:07]
\z

\ea%106
    \label{ex:complex:106}
          \textit{Apïn ngïl tembi nji ngala ndï \textbf{ndït inde}.}\\
\gll apïn  ngïn  tembi  nji    ngala    ndï  \textbf{ndï=tï}    \textbf{inda-e}\\
    fire    cloud  bad    thing  \textsc{pl.prox}  3\textsc{pl}  3\textsc{pl}=take  walk-\textsc{ipfv}\\
\glt `These marijuana cigarettes -- they carry them around.’\footnote{The speaker here pronounces \textit{apïn ngïn} `smoke’ with a final [l], presumably a \isi{speech error}.} (Literally ‘these bad smoke things’) [ulwa037\_21:52]
\z

The \isi{medial verb} \textit{tï-} ‘take’ can also be used with a \isi{motion} verb (typically \linebreak \textit{ma-} {\textasciitilde} \textit{i-} ‘go’) as the \isi{final verb} to give the meaning ‘bring’, as in examples \REF{ex:complex:107}, \REF{ex:complex:108}, and \REF{ex:complex:109}.

\ea%107
    \label{ex:complex:107}
          \textit{Nï \textbf{ndït i}.}\\
\gll nï  \textbf{ndï=tï}  \textbf{i}\\
    1\textsc{sg}  \textsc{3pl=}take  go.\textsc{pfv}\\
\glt `I brought them.’ [ulwa014\_12:06]
\z

\ea%108
    \label{ex:complex:108}
          \textit{Nï upin \textbf{mat anmbi} mat manane.}\\
\gll    nï    upin  \textbf{ma=tï}      \textbf{an-mbï-i}      ma=tï ma=na-n-e\\
    1\textsc{sg}  crowned.pigeon  3\textsc{sg.obj}=take  out-here-go.\textsc{pfv}  3\textsc{sg.obj}=take    3\textsc{sg.obj}=give-\textsc{pfv-dep}\\
\glt `I brought the crowned pigeon out here and gave it to her.’ [ulwa037\_03:51]
\z

\ea%109
    \label{ex:complex:109}
          \textit{Uma} \textbf{\textit{ndït}} \textit{li} \textbf{\textit{unde}.}\\
\gll uma  \textbf{ndï=tï}    li    \textbf{unda-e}\\
    bone  3\textsc{pl}=take  down  go-\textsc{ipfv}\\
\glt `[She] would bring [their] bones down.’ [ulwa020\_00:13]
\z

If the \isi{goal} argument (the \isi{location} to which something is brought) is encoded, then it is typically done so as the argument of a subsequent clause -- that is, not within a chained clause, but rather within another final clause following the chained group of \isi{medial clause} and final clause, as in \REF{ex:complex:110}.

\ea%110
  \label{ex:complex:110}
          \textit{Ndït i \textbf{Wopata may}.}\\
\gll ndï=tï    i    \textbf{Wopata}  \textbf{ma=i}\\
    3\textsc{pl}=take  go.\textsc{pfv}  [place]    3\textsc{sg.obj}=go.\textsc{pfv}\\
\glt `I brought them [= fish] to Wopata.’ (Literally ‘I brought them [and I] went to Wopata.’) [ulwa014\_29:41]
\z

When the \isi{goal} is functioning rather more \isi{adverb}ially, as in \REF{ex:complex:109} (i.e., \textit{li} ‘down’) or in \REF{ex:complex:111} (i.e., \textit{wa} ‘village’), then it may be included once within a \isi{clause chain} consisting of a \isi{medial clause} and a final clause.

\ea%111
    \label{ex:complex:111}
          \textit{Mat} \textbf{\textit{wa}} \textit{ita una malan!}\\
\gll    ma=tï      \textbf{wa}    i-ta        unan    ma=la-n\\
    3\textsc{sg.obj}=take  village  go.\textsc{pfv-cond}  \textsc{1pl.incl}  \textsc{3sg.obj=}eat-\textsc{imp}\\
\glt `If [we] bring it [= a crocodile] home, we’ll eat it!’ [ulwa038\_04:29]
\z

The \isi{recipient} argument (the person to whom something is brought) can be encoded in a \isi{postpositional phrase}, either with a \isi{direction}al \isi{postposition} such as \textit{iya} ‘toward’, as in \REF{ex:complex:112} and \REF{ex:complex:113}, or with a \isi{benefactive} \isi{postposition} such as \textit{ala} ‘for’ \REF{ex:complex:114} or \textit{nap} ‘for’ \REF{ex:complex:115}.

\ea%112
    \label{ex:complex:112}
          \textit{Inga la mat \textbf{aniya} i.}\\
\gll inga  ala      ma=tï      an=\textbf{iya}     i\\
    affine  \textsc{pl.dist}  3\textsc{sg.obj}=take  1\textsc{pl.excl=}toward  go.\textsc{pfv}\\
\glt `[My] in-laws brought it to us.’ [ulwa037\_36:56]
\z

\ea%113
    \label{ex:complex:113}
          \textit{Ndïtï \textbf{wiya} mana}.\\
\gll ndï=tï    u=\textbf{iya}      ma-na\\
    3\textsc{pl}=take  2\textsc{sg}=toward  go-\textsc{irr}\\
\glt `[They] would bring them to you.’ [ulwa014\_36:45]
\z

\ea%114
    \label{ex:complex:114}
          \textit{\textbf{Mala} numan kot i.}\\
\gll    ma=\textbf{ala}    numan    ko=tï    i\\
    3\textsc{sg.obj}=for  husband  \textsc{indf}=take  go.\textsc{pfv}\\
\glt `[They] brought a husband for her.’ [ulwa019\_00:53]
\z

\ea%115
    \label{ex:complex:115}
          \textit{Ndï ndinap ndït \textbf{nïnap} iye.}\\
\gll    ndï  ndï=ina-p    ndï=tï    nï=\textbf{nap}  i-e\\
    3\textsc{pl}  3\textsc{pl}=get-\textsc{pfv}  3\textsc{pl}=take  1\textsc{sg}=for  go.\textsc{pfv-dep}\\
\glt `They got them and brought them for me.’ [ulwa014\_47:24]
\z

\is{clause chaining|)}
\is{clause chain|)}
\is{complex sentence|)}
