\chapter{The Maruat-Dimiri-Yaul dialect of Ulwa}\label{sec:18}

\il{Maruat-Dimiri-Yaul|(}
\il{Yaul|(}
\il{Manu|(}
\is{dialect|(}

This chapter provides information on the \ili{Maruat-Dimiri-Yaul} \isi{dialect} of Ulwa, focusing on the ways in which it differs from the \ili{Manu} \isi{dialect}, which is otherwise the basis of description in this grammar. As its name implies, the \ili{Maruat-Dimiri-Yaul} \isi{dialect} of Ulwa is spoken in the three villages of Maruat, Dimiri, and Yaul, each of which is within an hour’s walk of the other two. Manu village is at least a four-hour walk from these three villages. The trip is considerably longer when trail conditions are more challenging, such as during the rainy season.

  The data for this description come mainly from Yaul villagers, some of whom I visited in Yaul in 2015, and some of whom I met living in the town of Angoram in 2018. I have also consulted \citegen{Laycock1971a} field notes, which are reproduced as Appendix \ref{sec:app.g}. Finally, I have considered data that I collected at Maruat and Dimiri in 2015, although I have spent less time with speakers from those two villages.

  Although \isi{lexical}ly somewhat divergent, the two main dialects of Ulwa appear to be rather similar grammatically. That said, I have much less information on the grammar of this \isi{dialect} than I do for the \ili{Manu} \isi{dialect}, especially concerning less common and more complex grammatical structures. It is possible that more differences exist than the ones I am aware of. Nevertheless, the basic \isi{syntactic} structures and \isi{morphology} appear to be very similar.

\is{dialect|)}
\il{Manu|)}
\il{Yaul|)}
\il{Maruat-Dimiri-Yaul|)}

\section{\label{sec:18.1}  Lexical similarity among the Ulwa dialects}

\il{Maruat-Dimiri-Yaul|(}
\il{Yaul|(}
\il{Manu|(}
\is{dialect|(}
\is{lexicon|(}
\il{Maruat|(}
\il{Dimiri|(}

The \isi{lexical} differences that exist among the three villages of Maruat, Dimiri, and Yaul are very minor (and some of the apparent differences in my data are probably due to errors of elicitation or to \isi{synonymy} rather than to a true lack of \isi{cognacy}). 

\noindent{Of 74 words\footnote{26 words were discarded due to either lack of data or duplication of data elsewhere in the list.} from the \ia{Swadesh, Morris} \isi{Swadesh} 100-word wordlist:}

\begin{quote}69 words (93\%) are \isi{cognate} between Yaul and Maruat,\\
69 words (93\%) are \isi{cognate} between Yaul and Dimiri, and\\
71 words (96\%) are \isi{cognate} between Maruat and Dimiri.\end{quote}

\noindent{Similarly, from the SIL-PNG list of 170 words:}

\begin{quote}120 out of 128 words (94\%) are \isi{cognate} between Yaul and Maruat,\\
123 out of 129 words (95\%) are \isi{cognate} between Yaul and Dimiri, and\\
123 out of 128 words (96\%) are \isi{cognate} between Maruat and Dimiri.\end{quote}

\noindent{Thus, at least based on this crude metric, we can say that the varieties of Ulwa spoken at the three villages of Maruat, Dimiri, and Yaul are roughly 95\% \isi{lexical}ly similar to one another.}

  The \ili{Manu} \isi{dialect}, however, is clearly distinct, at least in terms of vocabulary. Of 84 words\footnote{16 words were discarded due to either lack of data or duplication of data elsewhere in the list.} from the \ia{Swadesh, Morris} \isi{Swadesh} 100-word wordlist, 72 (86\%) are \isi{cognate} between Manu and Yaul. Similarly, of 141 words from the SIL-PNG list, 121 (86\%) are \isi{cognate} between Manu and Yaul. Likewise, between Manu and Dimiri, 62 of 74 words (84\%) of the \isi{Swadesh list} are \isi{cognate} and 109 of 129 words (84\%) of the SIL-PNG list are \isi{cognate}. Between Manu and Maruat, 61 of 74 words (82\%) of the \isi{Swadesh list} are \isi{cognate} and 106 of 128 words (83\%) of the SIL-PNG list are \isi{cognate}. These percentages are summarized in \tabref{tab:18.1}. Percentages are averaged between those of the \isi{Swadesh} and SIL-PNG lists, where different.


\begin{table}
\caption{Cognacy rates among Ulwa varieties}
\is{cognacy}
\label{tab:18.1}


\begin{tabular}{lllll}
\lsptoprule
& Manu & Yaul & Dimiri & Maruat\\
\midrule
Manu & 100\% & 86\% & 84\% & 83\%\\
Yaul &  & 100\% & 94\% & 93\%\\
Dimiri &  &  & 100\% & 96\%\\
Maruat &  &  &  & 100\%\\
\lspbottomrule
\end{tabular}
\end{table}

It should be noted, however, that although such lists are intended to reflect so-called \isi{basic vocabulary} items (i.e., words that are theoretically less likely to be replaced over time), some items on these lists are probably not terribly basic, at least not in the \isi{New Guinea} context. It is remarkable that, of the 12 non-\isi{cognate} words among the 84 words used from the \ia{Swadesh, Morris} \isi{Swadesh}-100 list, seven are \isi{adjective}s (‘hot’, ‘cold’, ‘red’, ‘black’, ‘long’, ‘small’, ‘many’).\footnote{The remaining five non-\isi{cognate} words are ‘person’, ‘fish’, ‘grease’, ‘head’, and ‘liver’.}

  For the sake of convenience (and because all of the following data come specifically from Yaul villagers), I will henceforth simply use “Yaul” in referring to the \ili{Maruat-Dimiri-Yaul} \isi{dialect} of Ulwa.

\il{Dimiri|)}
\il{Maruat|)}
\is{lexicon|)}
\is{dialect|)}
\il{Manu|)}
\il{Yaul|)}
\il{Maruat-Dimiri-Yaul|)}


\section{\label{sec:18.2}  Sound changes}

\is{sound change|(}
\is{phonological change|(}
\is{dialect|(}
\il{Manu|(}
\il{Yaul|(}
\il{Maruat-Dimiri-Yaul|(}

A few sound changes have led to differences in pronunciation between the two dialects. By far the most salient is the change of non-final *l to /n/ in Manu. Speakers of both dialects are very aware of this \textit{l} : \textit{n} correspondence and are quick to offer examples of how speakers of the other \isi{dialect} “mispronounce” certain words. Word-initial examples of the correspondence are given in \tabref{tab:18.2}.


\begin{table}
\caption{Word-initial \textit{l} : \textit{n} correspondences between Yaul and Manu}
\label{tab:18.2}


\begin{tabular}{lll}

\lsptoprule

gloss & Yaul word & Manu word\\
\midrule
‘talk, speech’ & {\itshape \textbf{l}a} & {\itshape \textbf{n}a}\\
‘die’ & {\itshape \textbf{l}i-} & {\itshape \textbf{n}i-}\\
‘near’ & {\itshape \textbf{l}u} & {\itshape \textbf{n}u}\\
‘lizard’ & {\itshape \textbf{l}ïkït} & {\itshape \textbf{n}ïkït}\\
‘thorn’ & {\itshape \textbf{l}in} & {\itshape \textbf{n}in}\\
‘vine’ & {\itshape \textbf{l}ïpïl} & {\itshape \textbf{n}ïpïl}\\
‘dig’ & {\itshape \textbf{l}ïkï-} & {\itshape \textbf{n}ïkï-}\\
‘beak’ & {\itshape \textbf{l}okal} & {\itshape \textbf{n}okal}\\
‘\textsc{irr}’ & {\itshape {}-\textbf{l}a} & {\itshape {}-\textbf{n}a}\\
\lspbottomrule
\end{tabular}
\end{table}
Word-medial examples of the correspondence are given in \tabref{tab:18.3}.


\begin{table}
\caption{Word-medial \textit{l} : \textit{n} correspondences between Yaul and Manu}
\label{tab:18.3}


\begin{tabular}{lll}

\lsptoprule

gloss & Yaul word & Manu word\\
\midrule
‘sky, cloud’ & {\itshape a\textbf{l}am} & {\itshape a\textbf{n}am}\\
‘sun’ & {\itshape a\textbf{l}e} & {\itshape a\textbf{n}e}\\
‘snake’ & {\itshape a\textbf{l}moka} & {\itshape a\textbf{n}moka}\\
‘boil, abscess’ & {\itshape ka\textbf{l}a\textbf{l}um} & {\itshape ka\textbf{n}a\textbf{n}um}\\
‘spear’ & {\itshape ma\textbf{l}a} & {\itshape ma\textbf{n}a}\\
‘hot water’ & {\itshape ma\textbf{l}al} & {\itshape ma\textbf{n}al}\\
‘navel’ & {\itshape u\textbf{l}et} & {\itshape u\textbf{n}et}\\
‘feel’ & {\itshape wa\textbf{l}a-} & {\itshape wa\textbf{n}a-}\\
\lspbottomrule
\end{tabular}
\end{table}
In a few instances, the change of *l to /n/ in Manu was accompanied by \isi{metathesis} of the following \isi{vowel} and \isi{consonant}, as illustrated in \tabref{tab:18.4}.


\begin{table}
\caption{Metathesis in some Manu words following *l > \textit{n}}
\is{metathesis}
\label{tab:18.4}


\begin{tabularx}{\textwidth}{lllQ}

\lsptoprule

gloss & Yaul word & Manu word & changes in Manu\\
\midrule
‘lime gourd’ & {\itshape al\textbf{is}} & {\itshape an\textbf{si}} & *alis > *anis > \textit{ansi}\\
‘elbow’ & {\itshape il\textbf{up}} & {\itshape in\textbf{pu}} & *i-lup ‘arm-base’ > *inup > \textit{inpu}\\
\lspbottomrule
\end{tabularx}
\end{table}

At least for one word, however, it appears that the Yaul \isi{dialect} was the one that underwent a transposition, namely, of *l (\tabref{tab:18.5}).

\begin{table}
\caption{Metathesis of *l in Yaul}
\is{metathesis}
\label{tab:18.5}


\begin{tabularx}{\textwidth}{lllQ}

\lsptoprule

gloss & Yaul word & Manu word & changes in Yaul\\
\midrule
‘\textit{tulip} greens’ & {\itshape amo\textbf{l}apa} & {\itshape a\textbf{n}mopa} & *alma-wapa ? (‘good-leaf’?) > *almoapa > \textit{amolapa}\\
\lspbottomrule
\end{tabularx}
\end{table}
Some examples of word-medial /l/ found in contemporary Manu are likely due to \isi{metathesis} of formerly word-final *l (\tabref{tab:18.6}).


\begin{table}
\caption{Metathesis of word-final *l in Manu}
\is{metathesis}
\label{tab:18.6}


\begin{tabular}{llll}

\lsptoprule

gloss & Yaul word & Manu word & changes in Manu\\
\midrule
‘\textsc{pl.refl}’ & {\itshape amba\textbf{l}} & {\itshape amb\textbf{l}a} & *ambal > \textit{ambla}\\
‘root’ & {\itshape iwï\textbf{l}} & {\itshape i\textbf{l}u} & *iwul > *iwlu > \textit{ilu}\\
\lspbottomrule
\end{tabular}
\end{table}
In at least one case, Manu preserves an original word-final /l/ where Yaul has undergone \isi{metathesis} (\tabref{tab:18.7}).


\begin{table}
\caption{Metathesis of word-final *l in Yaul}
\label{tab:18.7}
\is{metathesis}
\begin{tabular}{llll}

\lsptoprule

gloss & Yaul word & Manu word & changes in Yaul\\
\midrule
‘sugarcane’ & {\itshape mï\textbf{l}i} & {\itshape mi\textbf{l}} & *mil > *mli > \textit{mïli}\\
\lspbottomrule
\end{tabular}
\end{table}
In three instances, there appears to be a reverse correspondence of \textit{n} : \textit{l} – that is, Yaul exhibits /n/ where Manu has /l/ (\tabref{tab:18.8}).


\begin{table}
\caption{“Reverse correspondence” of \textit{n} : \textit{l} between Yaul and Manu}
\label{tab:18.8}


\begin{tabular}{lll}

\lsptoprule

gloss & Yaul word & Manu word\\
\midrule
‘pig’ & {\itshape \textbf{n}amndu} & {\itshape \textbf{l}amndu}\\
‘dream’ & {\itshape \textbf{n}ongam} & {\itshape \textbf{l}ongam}\\
‘eye’ & {\itshape \textbf{n}ïmndï} & {\itshape \textbf{l}ïmndï}\\
\lspbottomrule
\end{tabular}
\end{table}
In \textit{lamndu} ‘pig’, the presence of /l/ in Manu is likely due to \isi{folk etymology} from \textit{lam} ‘meat’ (itself a \isi{loan} from \ili{Ap Ma}). I do not know the etymologies of the other two words. It should be noted, though, that some Manu speakers produce [namndu] for ‘pig’; moreover, some Yaul speakers produce [lïmndï] for ‘eye’. Indeed, variation between [l] and [n] -- sometimes within even a single speaker’s realization of a single word -- is common in the \ili{Keram} family. Contemporary Manu has plenty of examples of word-initial and word-medial /l/. Some of these are due no doubt to \isi{borrowing}, but some may also reflect language-internal sporadic changes of *n to /l/, in some instances reversions back to the \ili{Pre-Ulwa} form.

  Aside from this \textit{l} : \textit{n} correspondence, there are not many robust phonemic differences between the Yaul and Manu dialects. The realization of initial *y- in the two dialects is somewhat variable. Generally, when the \isi{onset} of the following \isi{syllable} was not a \isi{nasal} (or \isi{prenasalized}) \isi{consonant}, *y- became /n/ in Manu, but was retained as /y/ in Yaul (\tabref{tab:18.9}).


\begin{table}
\caption{*y > \textit{n} / \# \_ VC [-nasal] in Manu}
\label{tab:18.9}


\begin{tabular}{llll}

\lsptoprule

gloss & Yaul word & Manu word & changes in Manu\\
\midrule
‘vegetables’ & {\itshape \textbf{y}at\textbf{l}at} & {\itshape \textbf{n}at\textbf{n}at} & *y- > \textit{n-} ; *-l- > \textit{{}-n}\\
‘body hair’ & {\itshape \textbf{y}il} & {\itshape \textbf{n}il} & *y- > \textit{n-}\\
‘bamboo species’ & {\itshape \textbf{y}okam} & {\itshape ani-\textbf{n}okam \textup{(‘throat’)}} & *y- > \textit{n-}\\
\lspbottomrule
\end{tabular}
\end{table}
However, when the following \isi{onset} was a \isi{nasal} (or \isi{prenasalized}) \isi{consonant}, Manu seems to have retained initial *y-, whereas Yaul behaves less predictably, reflecting either /l/ or /n/ (\tabref{tab:18.10}).


\begin{table}
\caption{*y > \textit{l} {\textasciitilde} \textit{n} / \# \_ VC [+nasal] in Yaul}
\label{tab:18.10}


\begin{tabular}{llll}

\lsptoprule

gloss & Yaul word & Manu word & changes in Yaul\\
\midrule
‘\textit{aibika} greens’ & {\itshape \textbf{l}omol} & {\itshape \textbf{y}omal} & *y- > \textit{l-}\\
‘mosquito’ & {\itshape \textbf{n}angun} & {\itshape \textbf{y}angun} & *y- > \textit{n-}\\
\lspbottomrule
\end{tabular}
\end{table}
It is difficult to disentangle what is happening here. Family-internal \isi{borrowing} could be complicating the matter. It could also be that the \isi{approximant} /y/ is part of a more general pattern of sporadic \textit{l} : \textit{n} : \textit{y} alternations in the \ili{Keram} languages. The verb ‘carve’, for example, is difficult to explain in Ulwa, especially since comparison to other \ili{Keram} languages seems to indicate a proto-form *lo (\tabref{tab:18.11}).

\begin{table}


\caption{\label{tab:18.11} The verb ‘carve’ in Ulwa dialects}
\il{Pre-Ulwa}

\begin{tabular}{llll}

\lsptoprule

gloss & Yaul word & Manu word & Pre-Ulwa word\\
\midrule
‘carve’ & {\itshape lo- \textup{{\textasciitilde}} yo-} & {\itshape lo-} & *lo- (?)\\
\lspbottomrule
\end{tabular}
\end{table}
Thus, Manu reflects \textit{lo-} ‘carve’ instead of the expected \textsuperscript{†}/no-/; and, alongside the expected reflex \textit{lo-} ‘carve’, Yaul has an alternate form /yo-/ that points to a change of *l- to /y-/ (the reverse of the aforementioned Yaul sound change).

  The \isi{approximant} /w/ also exhibits some unusual behavior. In some instances, final *w has become /m/ in Yaul (\tabref{tab:18.12}).

\begin{table}
\caption{\label{tab:18.12} Sporadic changes of final *-w > \textit{{}-m} in Yaul}


\begin{tabularx}{\textwidth}{lllQ}

\lsptoprule

gloss & Yaul word & Manu word & changes in Yaul\\
\midrule
‘betel nut’ & {\itshape a\textbf{m}} & {\itshape a\textbf{w}} & *-w > \textit{{}-m} (*aw-mu > *awm > \textit{am})\\
‘belly’ & {\itshape inapa\textbf{m}} & {\itshape inapa\textbf{w}} & *-w > \textit{{}-m}\\
‘paddle’ & {\itshape ana\textbf{m}} & {\itshape ana\textbf{w}} & *-w > \textit{{}-m}\\
\lspbottomrule
\end{tabularx}
\end{table}
At least in the case of \textit{am} ‘betel nut’, an apparent change of *-w to /-m/ probably instead reflects \isi{compound}ing with the \isi{suffix}-like element \textit{mu} ‘fruit, seed, nut’. That is, *aw-mu ‘areca palm-nut’ was reduced to one \isi{syllable} (*awm), and then the \is{consonant cluster} *wm cluster was reduced to [m]. The attested alternative Yaul pronunciation of [awm] strongly supports this idea.

  In other instances, a medial /w/ in Yaul corresponds to a medial /m/ in Manu (\tabref{tab:18.13}).

\il{Maruat-Dimiri-Yaul|)}
\il{Yaul|)}
\il{Manu|)}
\is{dialect|)}
\is{phonological change|)}
\is{sound change|)}

\is{sound change|(}
\is{phonological change|(}
\is{dialect|(}
\il{Manu|(}
\il{Yaul|(}
\il{Maruat-Dimiri-Yaul|(}

\begin{table}
\caption{Occasional correspondence of \textit{w} : \textit{m} between Yaul and Manu}
\label{tab:18.13}
\begin{tabular}{lll}

\lsptoprule

gloss & Yaul word & Manu word\\
\midrule
‘housefly’ & {\itshape nji\textbf{w}ala} & {\itshape nji\textbf{m}ana}\\
‘nape of the neck’ & {\itshape tumbun\textbf{w}a} & {\itshape tumbun\textbf{m}a}\\
\lspbottomrule
\end{tabular}
\end{table}
Here I suspect that there was an original medial *mw \is{consonant cluster} cluster, which was simplified in different ways in the two dialects (i.e., *mw > \textit{w} in Yaul; *mw > \textit{m} in Manu). Indeed, \citet[3220]{Laycock1971a} records <tumbʊnmwɔ> for ‘nape of the neck’ in Yaul. This word itself may reflect a \isi{compound} containing *umwa ‘neck’. Compare Manu \textit{um} ‘neck’, and the variable Yaul forms \textit{umo} {\textasciitilde} \textit{umwo} ‘neck’. \linebreak \citet[3220]{Laycock1971a} records <wuwɑ> ‘neck’ for Yaul.

  In one case, a final *-l seems to have become /-w/ in Manu, whereas in another case a final *-w seems to have become /-l/ (\tabref{tab:18.14}).


\begin{table}
\caption{Sporadic change of final *-l > \textit{{}-w} and final *-w > \textit{{}-l} in Manu}
\il{Pre-Ulwa}
\label{tab:18.14}


\begin{tabular}{llll}

\lsptoprule

gloss & Yaul word & Manu word & Pre-Ulwa word\\
\midrule
‘scale’ & {\itshape wowa\textbf{l}} & {\itshape wowa\textbf{w}} & *wowal (?)\\
‘afternoon’ & {\itshape awa\textbf{w}} & {\itshape awa\textbf{l}} & *awaw (?)\\
\lspbottomrule
\end{tabular}
\end{table}
The reconstructions in \tabref{tab:18.14} are made based on comparison to Ulwa’s sister languages. While the change of Manu *wowal to /wowaw/ ‘scale’ may be due to a sporadic \is{assimilation} \isi{retrograde assimilation} to the medial /w/; the change of *awaw to /awal/ is more difficult to explain.

  Finally, there is variation in the presence or absence of word-initial [w-] when immediately preceding /u/ (\tabref{tab:18.15}).
  
\begin{table}
\caption{Idiosyncrasies of initial [wu- {\textasciitilde} u-] in Yaul and Manu}
\label{tab:18.15}


\begin{tabular}{lll}

\lsptoprule

gloss & Yaul word & Manu word\\
\midrule
‘bird’ & {\itshape \textbf{w}uta} & {\itshape uta}\\
‘worm’ & {\itshape \textbf{w}utal} & {\itshape utal}\\
‘fan’ & {\itshape un} & {\itshape \textbf{w}un}\\
\lspbottomrule
\end{tabular}
\end{table}

Since there is likely both \isi{glide} \isi{epenthesis} and \isi{glide} \isi{deletion} at play in both dialects, it is difficult to make much of these apparent differences: they may mostly reflect idiosyncrasies, or perhaps arbitrary decisions that I myself made in transcribing different forms.

Sometimes Yual [wï-] corresponds to Manu [wu- {\textasciitilde} u-], as in \tabref{tab:18.16}.


\begin{table}
\caption{Some correspondences of \textit{wï-} : \textit{(w)u-} between Yaul and Manu}
\label{tab:18.16}


\begin{tabular}{lll}

\lsptoprule

gloss & Yaul word & Manu word\\
\midrule
‘coconut shell’ & {\itshape \textbf{wï}ta} & {\itshape \textbf{wu}ta \textup{{\textasciitilde}} \textbf{u}ta}\\
‘leg’ & {\itshape \textbf{wï}tï} & {\itshape \textbf{wu}tï}\\
‘with’ & {\itshape \textbf{wï}l} & {\itshape \textbf{u}l}\\
\lspbottomrule
\end{tabular}
\end{table}
In addition to these sometimes regular, sometimes seemingly sporadic \isi{consonant} correspondences, there is some variability in \isi{vowel} quality between the two dialects. Again, it is unclear to what extent this reflects actual dialectal differences as opposed to, say, \isi{idiolect}al differences, especially considering the limited number of Yaul speakers I worked with. In a number of cases, it seems that Manu has \is{rounding} rounded *a to /o/ when either directly preceding or directly following a \isi{labial} \isi{consonant} (/w, m, mb, p/), as in \tabref{tab:18.17}.


\begin{table}
\caption{Manu *a > \textit{o}, when directly preceding or following a labial}
\is{labial}
\label{tab:18.17}


\begin{tabular}{llll}

\lsptoprule

gloss & Yaul word & Manu word & notes\\
\midrule
‘bandicoot’ & {\itshape \textbf{wa}ndi} & {\itshape \textbf{wo}ndi} & \\
‘hair’ & {\itshape \textbf{wa}nïmi} & {\itshape \textbf{wo}nmi} & \\
‘tusk’ & {\itshape \textbf{wa}nmbi} & {\itshape \textbf{wo}nmbi} & \\
‘breast’ & {\itshape \textbf{wa}lum} & {\itshape \textbf{wo}l} & < *wal-um ‘breast-fruit’\\
‘penis’ & {\itshape \textbf{wa}l} & {\itshape \textbf{wo}n} & < *walV (?); Manu \textit{n} < *l\\
‘father’ & {\itshape it\textbf{am}} & {\itshape it\textbf{om}} & \\
‘mother’ & {\itshape in\textbf{am}} & {\itshape in\textbf{om}} & \\
‘day’ & {\itshape il\textbf{am}} & {\itshape il\textbf{om}} & \\
‘forehead’ & {\itshape mon\textbf{amb}am} & {\itshape mon\textbf{omb}am} & \\
‘stomach’ & {\itshape u\textbf{mbap}a} & {\itshape u\textbf{mbop}a} & \\
‘cockatoo’ & {\itshape y\textbf{ap}uta} & {\itshape y\textbf{op}a} & \\
‘throw’ & {\itshape t\textbf{ap}} & {\itshape t\textbf{op}} & \\
\lspbottomrule
\end{tabular}
\end{table}
However, this same \textit{a} : \textit{o} correspondence may also be found in the absence of neighboring \isi{labial}s (\tabref{tab:18.18}).


\begin{table}
\caption{Correspondence of \textit{a} : \textit{o} between Yaul and Manu (in a non-labial environment)}
\label{tab:18.18}


\begin{tabular}{lll}

\lsptoprule

gloss & Yaul word & Manu word\\
\midrule
‘sweet potato’ & {\itshape n\textbf{a}ngontam} & {\itshape n\textbf{o}ngontam}\\
\lspbottomrule
\end{tabular}
\end{table}

The reverse correspondence (i.e., \textit{o} : \textit{a}) is also attested (\tabref{tab:18.19}). This may reflect \isi{hypercorrection} on the part of Manu \isi{dialect}, since in all examples the \isi{vowel} in question directly neighbors a \isi{labial} \isi{consonant}.


\begin{table}
\caption{“Reverse correspondence” of \textit{o} : \textit{a} between Yaul and Manu (in labial environments)}
\label{tab:18.19}

\begin{tabular}{lll}

\lsptoprule

gloss & Yaul word & Manu word\\
\midrule
‘strap’ & {\itshape w\textbf{o}m} & {\itshape w\textbf{a}m}\\
‘village’ & {\itshape w\textbf{o}} & {\itshape w\textbf{a}}\\
‘wallaby’ & {\itshape w\textbf{o}kan} & {\itshape w\textbf{a}kan}\\
‘bedbug’ & {\itshape m\textbf{o}mbïn} & {\itshape m\textbf{a}mbun}\\
\lspbottomrule
\end{tabular}
\end{table}
In two words, both \textit{o} : \textit{a} and \textit{a} : \textit{o} correspondences can be seen, perhaps reflecting both \isi{rounding} and \isi{hypercorrection} in the same word, or perhaps resulting from \isi{metathesis} of the \isi{vowel}s /a/ and /o/ (\tabref{tab:18.20}). Both examples exhibit word-initial /y-/; I do not know whether this is relevant.


\begin{table}
\caption{Correspondence of \textit{a … o} : \textit{o … a} between Yaul and Manu}
\label{tab:18.20}


\begin{tabular}{lll}

\lsptoprule

gloss & Yaul word & Manu word\\
\midrule
‘palm of the hand’ & {\itshape y\textbf{a}mb\textbf{o}m} & {\itshape y\textbf{o}mb\textbf{a}m}\\
‘snake species’ & {\itshape y\textbf{a}m\textbf{o}} & {\itshape y\textbf{o}m\textbf{a}}\\
\lspbottomrule
\end{tabular}
\end{table}
Examples of other sporadic \isi{vowel} correspondences are given in \tabref{tab:18.21}. Note that \textit{lenjin} {\textasciitilde} \textit{lanjin} ‘fish species’ is likely a \isi{loan} from \ili{Ap Ma} \textit{landʒin} ‘fish species (perch) (TP \textit{nilpis})’ \citep[77]{Barlow2021}, thus explaining the word-initial /l-/ in the Manu form.


\begin{table}
\caption{Sporadic vowel correspondences between Yaul and Manu}
\is{vowel}
\label{tab:18.21}


\begin{tabular}{llll}

\lsptoprule

gloss & Yaul & Manu & correspondences\\
\midrule
‘bad’ & {\itshape t\textbf{a}mbi} & {\itshape t\textbf{e}mbi} & \textit{a} : \textit{e}\\
‘cough’ & {\itshape ut\textbf{e}n} & {\itshape ut\textbf{a}n} & \textit{e} : \textit{a}\\
‘some’ & {\itshape kum\textbf{e}} & {\itshape kum\textbf{a}} & \textit{e} : \textit{a}\\
‘fish species’ & {\itshape l\textbf{e}njin} & {\itshape l\textbf{a}njin} & \textit{e} : \textit{a}\\
‘\textit{pangal}’ & {\itshape w\textbf{o}ma} & {\itshape w\textbf{e}ma} & \textit{o} : \textit{e}\\
‘bow’ & {\itshape w\textbf{e}ng\textbf{u}ta} & {\itshape w\textbf{o}ng\textbf{ï}ta} & \textit{e} : \textit{o} ; \textit{u} : \textit{ï}\\
‘buttocks’ & {\itshape unmb\textbf{u}} & {\itshape unmb\textbf{ï}} & \textit{u} : \textit{ï}\\
‘hunger’ & {\itshape mund\textbf{ï}} & {\itshape mund\textbf{u}} & \textit{ï} : \textit{u}\\
‘frog species’ & {\itshape kïl\textbf{i}kïli} & {\itshape kïl\textbf{a}kïli} & \textit{i} : \textit{a}\\
‘yellow’ & {\itshape mïnd\textbf{ï}t} & {\itshape mïnd\textbf{i}t} & \textit{ï} : \textit{i}\\
‘all’ & {\itshape w\textbf{u}pa} & {\itshape w\textbf{o}pa} & \textit{u} : \textit{o}\\
‘river’ & {\itshape n\textbf{u}m\textbf{u}l} & {\itshape n\textbf{ï}m\textbf{a}l} & \textit{u} : \textit{ï} ; \textit{u} : \textit{a}\\
‘white’ & {\itshape w\textbf{ae}mb\textbf{ï}l} & {\itshape w\textbf{e}mb\textbf{a}l} & \textit{ae} : \textit{e} ; \textit{ï} : \textit{a}\\
\lspbottomrule
\end{tabular}
\end{table}

\il{Maruat-Dimiri-Yaul|)}
\il{Yaul|)}
\il{Manu|)}
\is{dialect|)}
\is{phonological change|)}
\is{sound change|)}

\section{\label{sec:18.3}  Phonetics and phonology}

\is{phonetics|(}
\is{phonology|(}
\il{Maruat-Dimiri-Yaul|(}
\il{Yaul|(}
\il{Manu|(}
\is{dialect|(}

One simply phonetic difference between the Manu and Yaul dialects is found in the pronunciation of the \isi{glide} /w/. In Manu this phoneme is almost always pronounced as a \isi{labial-velar} \isi{approximant} [w], whereas in Yaul it tends to be pronounced as a \isi{labiodental} \isi{approximant} [ʋ]. It may be noted that the \ili{Mundukumo} language, which has been especially influential on the Yaul \isi{dialect} of Ulwa, has a phonemic /v/. Otherwise, the two dialects are phonetically rather similar.

  The phonemic inventories of the two dialects are exactly the same. The Yaul \isi{dialect} does not appear to have the very limited phone [æ] that has been observed in just four Manu \isi{dialect} words (\sectref{sec:2.2.1}).

  In terms of \isi{phonological} rules, the only difference between the dialects that I know of concerns \isi{glide formation}. In Manu, \isi{vowel} sequences of /ai/ and /au/ become [ay] and [aw], respectively. Otherwise, \isi{central vowel}s /a, ï/ are \isi{deleted} before immediately following \isi{vowel}s, suggesting that this glide-formation rule in Manu bleeds the \isi{vowel}-\isi{deletion} rule (\sectref{sec:2.5.1}). In Yaul, however, \isi{central vowel} \isi{deletion} occurs unimpeded. Thus, we see contrasts such as those in \REF{ex:mdy:1}.

\is{dialect|)}
\il{Manu|)}
\il{Yaul|)}
\il{Maruat-Dimiri-Yaul|)}
\is{phonology|)}
\is{phonetics|)}


\ea%1
    \label{ex:mdy:1}
            Central \isi{vowel} \isi{deletion} in /ai/ and /au/ sequences in Yaul\\
\begin{tabbing}
{(c.)} \= {(/ma=uta/)} \= {(vs.)} \= {(d.)} \= {(/ma=uta/)}\kill
{a.} \> {Yaul} \> {vs.} \> {b.} \> {Manu}\\
{ } \> {[m\textbf{i}ta]} \> { } \> { } \> {[m\textbf{ay}ta]}\\
{ } \> {/ma=ita/} \> { } \> { } \> {/ma=ita/}\\
{ } \> {‘build it’} \> { } \> { } \> {‘build it’}\\
{c.} \> {Yaul} \> {vs.} \> {d.} \> {Manu}\\
{ } \> {[m\textbf{u}ta]} \> { } \> { } \> {[m\textbf{aw}ta]}\\
{ } \> {/ma=uta/} \> { } \> { } \> {/ma=uta/}\\
{ } \> {‘grind it’} \> { } \> { } \> {‘grind it’}
\end{tabbing}
\z

\section{\label{sec:18.4}  Morphology}

\is{morphology|(}
\il{Maruat-Dimiri-Yaul|(}
\il{Yaul|(}
\il{Manu|(}
\is{dialect|(}


The two dialects are very similar in terms of \isi{morphology} as well. Like Manu, Yaul exhibits a basic three-way \isi{TAM} distinction in verbal \isi{suffix}es \REF{ex:mdy:2}.

\ea%2
    \label{ex:mdy:2}
                \isi{TAM} \isi{suffix}es in Yaul
\begin{tabbing}
{(\textit{-la} {\textasciitilde} \textit{-nda})} \= {(\isi{imperfective} (‘\textsc{ipfv}’))}\kill
{\textit{-e} {\textasciitilde} \textit{-i}} \> {\isi{imperfective} (‘\textsc{ipfv}’)}\\
{\textit{-pe} {\textasciitilde} \textit{-pi}} \> {\isi{perfective} (‘\textsc{pfv}’)}\\
{\textit{-la} {\textasciitilde} \textit{-nda}} \> {\isi{irrealis} (‘\textsc{irr}’)}
\end{tabbing}
\z

For its basic \isi{imperfective} \isi{suffix}, the Yaul \isi{dialect} employs either \textit{-e}, ‘\textsc{ipfv}’ (as in Manu) or the \isi{allomorph} \textit{-i} ‘\textsc{ipfv}’. These two forms appear to be in \isi{free variation}. Similarly, instead of Manu’s \isi{perfective} \isi{suffix} \textit{-p} ‘\textsc{pfv}’, Yaul generally employs either \textit{-pe} ‘\textsc{pfv}’ or \textit{-pi} ‘\textsc{pfv}’, again, in \isi{free variation}. Less commonly, I have also observed \textit{-pa} ‘\textsc{pfv}’, \textit{-pï} ‘\textsc{pfv}’, and indeed \textit{-p} ‘\textsc{pfv}’ as variations of the Yaul \isi{perfective} \isi{suffix}.\footnote{It is unclear whether the \isi{vowel}-final Yaul forms are more archaic, reflecting the final \isi{vowel} segment of \ili{Proto-Keram} *-apɨ ‘\textsc{pfv}’, or whether they are innovative, adding a final \isi{vowel} to what would -- according to this hypothesis -- be \ili{Pre-Ulwa} *-p ‘\textsc{pfv}’. The Yaul \isi{dialect} appears to show a stronger aversion to \isi{coda}s than the Manu \isi{dialect}, so its \isi{phonotactics} may have encouraged such \is{paragoge} paragogic \isi{vowel}s.} Finally, the \isi{irrealis} form \textit{-la} ‘\textsc{irr}’ in Yaul is the expected \isi{cognate} to Manu \textit{-na} ‘\textsc{irr}’ (\ili{Pre-Ulwa} *-la). The \isi{allomorph} \nobreakdash-\textit{nda} ‘\textsc{irr’}, however, appears to behave less predictably in Yaul than in Manu. Whereas the Manu \isi{allomorph} is conditioned by a preceding \isi{sonorant} \isi{consonant} (\sectref{sec:4.2}), I can find no clear conditioning environment for \textit{-nda} ‘\textsc{irr’} in Yaul. My data here could be unreliable.

  Sentences \REF{ex:mdy:3}, \REF{ex:mdy:4}, and \REF{ex:mdy:5} exemplify \isi{imperfective} verb forms in Yaul.

\ea%3
    \label{ex:mdy:3}
            \textit{Itam ma apa \textbf{mite}.}\\
    \gll itam  mï{}-a      apa    ma=ita-\textbf{e}\\
    father  3\textsc{sg.subj-int}  house  3\textsc{sg.obj}=build-\textsc{ipfv}\\
    \glt ‘Father is building a house.’ [elicited]
\z
\ea%4
    \label{ex:mdy:4}
            \textit{Sïmboy ma ya \textbf{mute}.}\\
    \gll sïmboy  mï-a      ya      ma=uta-\textbf{e}\\
    child  3\textsc{sg.subj-int}  coconut  3\textsc{sg.obj}=grind-\textsc{ipfv}\\
    \glt ‘The child grinds the coconut.’ [elicited]
\z
\ea%5
    \label{ex:mdy:5}
            \textit{Yeta wokïn mï tïn samola \textbf{masi}.}\\
    \gll yeta  wokïn  mï       tïn    samola  ma=asa-\textbf{i}\\
    man  huge    3\textsc{sg.subj}  dog  small  3\textsc{sg.obj}=hit-\textsc{ipfv}\\
    \glt ‘The big man is hitting the small dog.’ [elicited]
\z
Sentences \REF{ex:mdy:6} and \REF{ex:mdy:7} exemplify \isi{perfective} verb forms in Yaul.

\ea%6
    \label{ex:mdy:6}
            \textit{Ndï awaw kambïn \textbf{masapi}.}\\
    \gll ndï  awaw    kambïn  ma=asa-\textbf{pi}\\
    3\textsc{pl}  yesterday  child    3\textsc{sg.obj}=hit-\textsc{pfv}\\
    \glt ‘Yesterday they hit the child.’ [elicited]
\z
\ea%7
    \label{ex:mdy:7}
            \textit{Inam ngata ma \textbf{lipe}.}\\
    \gll inam  ngata  mï-a      li-\textbf{pe}\\
    mother  grand  3\textsc{sg.subj}{}-\textsc{int}  die-\textsc{pfv}\\
    \glt ‘The old woman is dead [= has died].’ [elicited]
\z

Sentences \REF{ex:mdy:8} and \REF{ex:mdy:9} exemplify \isi{irrealis} verb forms in Yaul.

\ea%8
    \label{ex:mdy:8}
            \textit{Nï \textbf{mawalinda} mï \textbf{sala} ne.}\\
    \gll nï    ma=wali-\textbf{nda}    mï      sa-\textbf{la}  n-e\\
    1\textsc{sg}  3\textsc{sg.obj}=hit-\textsc{irr}  3\textsc{sg.subj}  cry-\textsc{irr}  want-\textsc{ipfv}\\
    \glt ‘I’ll hit him and he’ll cry.’ [elicited]
\z
\ea%9
    \label{ex:mdy:9}
            \textit{ɑsi \textbf{kɑndɑ}}\\
    \gll asi  ka-\textbf{nda}\\
    sit  let-\textsc{irr}\\
    \glt ‘[We] will sit down.’ [\citep[3256]{Laycock1971a}; glossing mine]
\z
\isi{Irregular verb}s (with their associated irregular \isi{TAM} \isi{suffix}es) mostly behave the same in Yaul as they do in Manu. Moreover, the same \isi{suppletive} verbal forms (i.e., for ‘hit’ and ‘go’) are attested. The verbs \textit{ka-} ‘let’ \REF{ex:mdy:9a} and \textit{wo-} ‘sleep’ \REF{ex:mdy:9b}, however, do not seem to employ the Manu \isi{irrealis} forms with the \isi{prefix}-like element [la-] (cf. \sectref{sec:4.3}).\footnote{These [la- {\textasciitilde} lo-] forms in Manu may be innovations, perhaps reflecting a \isi{fossilized} form of the archaic \isi{detransitivizing} \isi{prefix} *la- (contemporary Manu \textit{na-} ‘\textsc{detr}’). Yaul does, however, employ the same possibly related \isi{irrealis} form of ‘eat’ as found in Manu (i.e., \textit{ama-} ‘eat’, \textit{landa} ‘eat [\textsc{irr]}’).}

\ea%9a
    \label{ex:mdy:9a}
The verb \textit{ka-} ‘let’ in Yaul\\
\begin{tabbing}
{(\textit{ka-nda})} \= {(‘\textsc{ipfv/pfv}’)} \= {((cf. Manu \textit{la-ka-na}))}\kill
{\textit{ka}} \> {‘\textsc{ipfv/pfv}’} \> { }\\
{\textit{ka-nda}} \> {‘\textsc{irr}’} \> {(cf. Manu \textit{la-ka-na})}
\end{tabbing}
\z

\ea%9b
    \label{ex:mdy:9b}
The verb \textit{wo-} ‘sleep’ in Yaul\\
\begin{tabbing}
{(\textit{wow-e})} \= {(‘\textsc{ipfv}’)} \= {((cf. Manu \textit{lo-wo-nda}))}\kill
{\textit{wow-e}} \> {‘\textsc{ipfv}’} \> { }\\
{\textit{wo-pi}} \> {‘\textsc{pfv}’} \> { }\\
{\textit{wo-la}} \> {‘\textsc{irr}’} \> {(cf. Manu \textit{lo-wo-nda})}
\end{tabbing}
\z

  Yaul \isi{pronoun}s match their Manu counterparts, with the slight distinction that Yaul more consistently drops the final /-n/ from the two \isi{disyllabic} forms that have it (i.e., [una] for /unan/ ‘\textsc{1pl.incl}’ and [nguna] for /ngunan/ ‘\textsc{1du.incl}’). Yaul also makes more frequent use of the \isi{intensive} \isi{suffix} \textit{-awa} ‘\textsc{int}’, typically reduced to [-a], thereby producing forms such as those in \tabref{tab::18.21.a}.\footnote{The \isi{deletion} of final /n/ and the insertion of final /a/ could both be motivated by Yaul’s aversion to \isi{coda}s.}


\begin{table}
\caption{Intensive pronominal forms in Yaul}
\is{intensive pronoun}
\label{tab::18.21.a}
\begin{tabularx}{.9\textwidth}{lQQQ}
\lsptoprule
& {\scshape sg} & {\scshape du} & {\scshape pl}\\
\midrule
1 & {\itshape na} & \textit{ngana} [\textsc{excl}] & \textit{ana} [\textsc{excl}]\\
& & \textit{nguna} [\textsc{incl}] & \textit{una} [\textsc{incl}]\\
2 & {\itshape wa} & {\itshape nguna} & {\itshape una}\\
3 & {\itshape ma} & {\itshape mina} & {\itshape nda}\\
\lspbottomrule
\end{tabularx}
\end{table}

   Thus there is often apparent \isi{homophony} between the forms \textit{una(n)} \textsc{‘1pl.incl’} and \textit{una(wa)} \textsc{‘2pl-int’} and between \textit{nguna(n)} \textsc{‘1du.incl}’ and \textit{nguna(wa)} \linebreak \textsc{‘2du-int’}. Furthermore, the distinction between subject and non-subject forms, generally only observable in \textsc{3sg} forms, is mostly eroded in Yaul, since the subject form /mï/ is frequently pronounced [ma] (< /mï-a/), thereby becoming \isi{homophonous} with the non-subject form /ma/. However, the distinction seems to be partially maintained in the \textsc{3du}, where the non-subject form has the \isi{allomorph} [mini=] when immediately preceding /n-/. In fact, in Yaul, as opposed to in Manu, /l-/ conditions this \isi{allomorphy} as well as /n-/, as illustrated by \REF{ex:mdy:10}.

\ea%10
    \label{ex:mdy:10}
          \textit{Itam ma num nini \textbf{minilope}.}\\
    \gll itam  mï-a      num  nini  \textbf{mini}=lo-pe\\
    father  \textsc{3sg.subj-int}  canoe  two  \textsc{3du.obj}=cut-\textsc{pfv}\\
    \glt ‘Father carved two canoes.’ [elicited]
\z
Example \REF{ex:mdy:10} also exhibits the use of the \isi{intensive} \isi{suffix} \textit{-a} ‘\textsc{int}’ on \isi{subject marker}s, which is common in Yaul.

  The \is{reflexive pronoun} \is{reciprocal pronoun} reflexive/reciprocal pronominal forms in Yaul are given in \tabref{tab::18.21.b}.\footnote{Note that the \isi{singular} reflexive form may actually be /ambï/, but with the \isi{intensive} \isi{suffix} \textit{-a} ‘\textsc{int}’ added. The form \textit{ambal=} ‘\textsc{pl.refl}’ is alternatively recorded as [ambal-a=].}

\begin{table}
\caption{Reflexive/reciprocal pronouns in Yaul}
\is{reflexive pronoun}
\is{reciprocal pronoun}
\label{tab::18.21.b}
\begin{tabularx}{.7\textwidth}{lQQQ}
\lsptoprule
& {\scshape sg} & {\scshape du} & {\scshape pl}\\
\midrule
{\scshape refl} & {\itshape amba=} & {\itshape ambin=} & {\itshape ambal=}\\
\lspbottomrule
\end{tabularx}
\end{table}

The Yaul \isi{plural} formative [-al] provides evidence that the Manu \isi{plural} form [-la] in the \isi{plural} \isi{reflexive} form [ambla] has undergone \isi{metathesis}, thus explaining the preservation of /l/, otherwise expected to change to /n/, were it originally in word-medial position.

The \is{topic marker} topic-marking pronominal \isi{suffix} is \textit{-ambo} ‘\textsc{top}’, which probably also derives from *ambï, but with an \isi{emphatic} \isi{suffix} \textit{-o} ‘\textsc{emph}’ added.

As in Manu, the \is{possession} possessive \isi{suffix} in Yaul is \textit{-nji} ‘\textsc{poss}’ (\sectref{sec:6.2}). There are no other known differences between the two dialects in terms of grammatical morphemes.

\is{dialect|)}
\il{Manu|)}
\il{Yaul|)}
\il{Maruat-Dimiri-Yaul|)}
\is{morphology|)}


\section{\label{sec:18.5}  Syntax}

\is{syntax|(}
\il{Maruat-Dimiri-Yaul|(}
\il{Yaul|(}
\il{Manu|(}
\is{dialect|(}

The two dialects are rather similar in their \isi{syntax} as well. The order of basic constituents is the same, as is the order of elements within \isi{noun phrase}s and within \isi{verb phrase}s. Yaul exhibits \isi{discontinuous} verb constructions of the kind used in Manu (\sectref{sec:9.2.1}), as well as verbal \isi{compound}ing with \isi{light verb}s like ‘put’ (\sectref{sec:9.2.2}) and ‘let’ (\sectref{sec:9.2.3}). \is{equative predication} Equative sentences can be formed without an overt \isi{copula}. Alternatively, the \isi{copular enclitic} \textit{=p} ‘\textsc{cop}’ can be used. Yaul also has the related \isi{locative verb} \textit{p-} ‘be, be at’, but no \isi{past}-\isi{tense} form \textsuperscript{†}[wap] is attested. This may be a Manu innovation.

While \isi{periphrastic} ‘going’ \isi{verb phrase}s are not attested in Yaul, it is very common for \isi{future}-\isi{time} sentences to employ what appears to be a \isi{volitive}/\isi{conative} \isi{auxiliary verb} \textit{ne} {\textasciitilde} \textit{ni} ‘want, will, try’, as in examples \REF{ex:mdy:12} through \REF{ex:mdy:15}.

\ea%12
    \label{ex:mdy:12}
          \textit{Nï uwïl la tanda \textbf{ne}.}\\
    \gll nï    u=wïl    la-ta-nda    \textbf{n-e}\\
    1\textsc{sg}  2\textsc{sg}=with  \textsc{detr}{}-say-\textsc{irr}  want-\textsc{ipfv}\\
    \glt ‘I want to speak with you.’ [elicited]
\z
\ea%13
    \label{ex:mdy:13}
          \textit{Mï utam ndïlanda \textbf{ni}.}\\
    \gll mï      utam  ndï=la-nda    \textbf{n-i}\\
    3\textsc{sg.subj}  yam  3\textsc{pl}=eat\textsc{{}-irr} want-\textsc{ipfv}\\
    \glt ‘She will eat the yams.’ [elicited]
\z
\ea%14
    \label{ex:mdy:14}
          \textit{Na umbe indï wandïla \textbf{ne}.}\\
    \gll nï-a    umbe    indï  u=andï-la    \textbf{n-e}\\
    1\textsc{sg}{}-\textsc{int}  tomorrow  eye    2\textsc{sg}=see-\textsc{irr}  want-\textsc{ipfv}\\
    \glt ‘I’ll see you tomorrow.’\footnote{In example \REF{ex:mdy:14} we see a reduction of \textit{nïmndï} ‘eye’ to [indï], which seems to occur commonly in Yaul when this word is used as part of a \isi{discontinuous} verb construction.} [elicited]
\z
\ea%15
    \label{ex:mdy:15}
          \textit{Na umbe wo mana \textbf{ne}.}\\
    \gll nï-a    umbe    wo    ma-na  \textbf{n-e}\\
    1\textsc{sg}{}-\textsc{int}  tomorrow  village  go-\textsc{irr}  want-\textsc{ipfv}\\
    \glt ‘I want to go to the village tomorrow.’\footnote{It should be observed that the Yaul \isi{irrealis} form of ‘go’ is [ma-na], and not \textsuperscript{†}[ma-la], indicating that the \ili{Pre-Ulwa} irregular \isi{irrealis} \isi{suffix} for ‘go’ is *-na. This explains why the Ulwa form is not the expected \textsuperscript{†}[-nda] following the \isi{sonorant} /m/. Indeed, I suspect that this irregular \textit{-na} ‘\textsc{irr}’ \isi{suffix} may derive from this \isi{volitive} \isi{auxiliary verb}, still productive in Yaul, but only present as the \isi{grammaticalized} \isi{suffix} in Manu. This same verb *n- likely also lies behind the irregular \isi{perfective} forms \textit{tï-n} ‘take-\textsc{pfv}’, \textit{na-n(a)} ‘give-\textsc{pfv}’, and \textit{i-n} ‘come-\textsc{pfv}’ (\sectref{sec:4.3}). It should also be noted that the Yaul \isi{irrealis} form of ‘come’ is (as in Manu) \textit{i-na} ‘come-\textsc{irr}’, suggesting that this verb (like \textit{ma-na} ‘go-\textsc{irr}’) contains the \isi{fossilized} \isi{auxiliary verb} *n- ‘want, will, try’, as opposed to the regular \isi{irrealis} \isi{suffix} *-la ‘\textsc{irr}’. This \isi{auxiliary verb} *n- may ultimately be related to the \isi{light verb} \textit{ni-} ‘act, do’.}
 [elicited]
\z

In example \REF{ex:mdy:8}, which features the \is{volitionality} non-volitional verb ‘cry’, the \isi{auxiliary} \textit{ne} ‘want, will, try’ seems to have little or no \isi{volitive} or \isi{conative} force, presumably simply indicating futurity or \isi{potentiality}. Example \REF{ex:mdy:12} additionally shows the use of the \isi{detransitivizing} \isi{prefix} \textit{la-} ‘\textsc{detr’}, the expected \isi{cognate} of Manu \textit{na-} ‘\textsc{detr’}. 

  As in Manu, ‘giving’ constructions are formed with two clauses, the first containing the verb \textit{tï-} ‘take’, the second containing the verb \textit{na-} ‘give’, as in \REF{ex:mdy:15a}.

  \ea%15a
    \label{ex:mdy:15a}
          \textit{Yana mï ya \textbf{matï} numon \textbf{manane}.}\\
    \gll yana mï ya ma=\textbf{tï} numon ma=\textbf{na-ne}\\
    woman  3\textsc{sg.subj} coconut 3\textsc{sg.obj}=take husband 3\textsc{sg.obj}=give-\textsc{pfv}\\
    \glt ‘The woman gave the coconut to [her] husband.’ [elicited]
\z

  Oblique marking \is{oblique marker} works the same in Yaul as in Manu (\sectref{sec:11.4}). Moreover, the form of the marker is \textit{=n} ‘\textsc{obl}’, suggesting that the \ili{Pre-Ulwa} form was *=n, as opposed to \textsuperscript{†}/=l/. An example of \isi{oblique} marking in Yaul is given in \REF{ex:mdy:16}.

\ea%16
    \label{ex:mdy:16}
          \textit{Itam mala \textbf{man} namndï mase.}\\
    \gll itam  mala  ma=\textbf{n}      namndï  ma=asa-e\\
    father  spear  \textsc{3sg.obj=obl}  pig      3\textsc{sg.obj}=hit-\textsc{ipfv}\\
    \glt ‘Father shoots the pig with the spear.’ [elicited]
\z

\is{dependent marker}

The only examples of overt \isi{coordination} I have seen for Yaul contain the \isi{borrow}ed \ili{Tok Pisin} \isi{conjunction} \textit{na} ‘and’. \isi{Dependent clause}s do seem to contain the \isi{dependent marker} \textit{-e} ‘\textsc{dep}’, although it is difficult to say for sure, since this \isi{phonetic} form is pervasive in Yaul verb endings: essentially every \isi{realis} verb form ends with [-e] or [-i]. It is possible that this preponderance of endings with \linebreak {[-e {\textasciitilde} -i]} is itself the result of an overextension of the dependent-marking \isi{suffix}.

\is{discontinuous negator}

  \isi{Question}s and \isi{command}s are formulated the same in Yaul as in Manu (\sectref{sec:13.1}, \sectref{sec:13.2}). \isi{Negation} likewise occurs with a \isi{negator} \textit{ango} ‘\textsc{neg}’, which follows the subject. Discontinuous \isi{negative} constructions are also possible, namely when \is{negation} negating \is{non-verbal predication} non-verbal \isi{predicate}s. The final element is, as in Manu, \textit{me} ‘\textsc{neg}’, as seen in \REF{ex:mdy:17}.

\ea%17
    \label{ex:mdy:17}
          \textit{Mï \textbf{ango} yana \textbf{me}.}\\
    \gll mï      \textbf{ango}  yana    \textbf{me}\\
    3\textsc{sg.subj}  \textsc{neg}  woman    \textsc{neg}\\
    \glt ‘He is not a woman.’ [elicited]
\z

Interesting, \citet{Laycock1971a} records a \isi{non-verbal negator} <meko>, which reflects the opposite ordering of morphemes as found in the Manu \isi{non-verbal negator} \textit{ko-me} ‘\textsc{neg}’ (\sectref{sec:13.3.1}) \REF{ex:mdy:18}.

\ea%18
    \label{ex:mdy:18}
          \textit{\textbf{ɑŋɡwɔ} nindʒi \textbf{meko}}\\
    \gll \textbf{ango}  nï-nji    \textbf{me-ko}\\
    \textsc{neg}  1\textsc{sg-poss}  \textsc{neg}{}-just\\
    \glt ‘[It] is not mine.’ [\citep[3262]{Laycock1971a}; glossing mine]
\z

\citet{Laycock1971a} also records <ko> alone as an occasional postverbal element, namely in \isi{negative} \isi{declarative} sentences, such as \REF{ex:mdy:19} and \REF{ex:mdy:20}.

\ea%19
    \label{ex:mdy:19}
          \textit{nɑ \textbf{ŋɡɔ} mundə ɑmɑpɛ \textbf{ko}}\\
    \gll nï    \textbf{ango}  mundï  ama-pe    \textbf{ko}\\
    1\textsc{sg}  \textsc{neg}  food  eat-\textsc{pfv}  \textsc{neg}\\
    \glt ‘I have not eaten.’ [\citep[3246]{Laycock1971a}; glossing mine]
\z
\ea%20
    \label{ex:mdy:20}
          \textit{nɑ \textbf{ŋɡwɔ} kəkɑl mɑƀɑli \textbf{ko}}\\
    \gll nï    \textbf{ango}  kïkal  ma=wala{}-i      \textbf{ko}\\
    1\textsc{sg}  \textsc{neg}  ear    3\textsc{sg.obj}=feel-\textsc{ipfv}  \textsc{neg}\\
    \glt ‘I do not hear it.’ [\citep[3258]{Laycock1971a}; glossing mine]
\z

In my own notes, I have one example of the \isi{negator} \textit{ko} ‘\textsc{neg}’ occurring optionally in a \isi{negative command} \REF{ex:mdy:21}. I have not found any equivalent of Manu \textit{wana} ‘\textsc{proh}’ being used in Yaul.

\ea%21
    \label{ex:mdy:21}
          \textit{U \textbf{ango} malanda ne (\textbf{ko}).}\\
    \gll u    \textbf{ango}  ma=la-nda    n-e      (\textbf{ko})\\
    2\textsc{sg}  \textsc{neg}  3\textsc{sg}=eat-\textsc{irr}  want-\textsc{ipfv}  (\textsc{neg})\\
    \glt ‘Do not eat it!’ [elicited]
\z

I do not have enough data to remark on less-common \isi{syntactic} constructions, such as \isi{relative clause}s, \isi{passivization}, or \isi{detransitivization} in Yaul.

\is{dialect|)}
\il{Manu|)}
\il{Yaul|)}
\il{Maruat-Dimiri-Yaul|)}
\is{syntax|)}

\newpage

\section{\label{sec:18.6}  {Loanwords and other lexical differences}}

\is{loanword|(}
\is{borrowing|(}
\il{Maruat-Dimiri-Yaul|(}
\il{Yaul|(}
\il{Manu|(}
\is{dialect|(}
\is{lexicon|(}

The \ili{Manu} and \ili{Maruat-Dimiri-Yaul} dialects share several loans from other languages, some of which very well may have entered at the stage of \ili{Pre-Ulwa}, if not earlier. Some loans that are only attested in the \ili{Maruat-Dimiri-Yaul} \isi{dialect}, however, include those given in \tabref{tab:18.22}.


\begin{table}
\caption{Loanwords in Yaul}
\is{loanword}
\is{borrowing}
\label{tab:18.22}


\begin{tabular}{llll}

\lsptoprule

gloss & Yaul word & Manu word & source of Yaul borrowing\\
\midrule
‘person’ & {\itshape mbalanji} & {\itshape ankam} & \ili{Yuat}\\
‘child’ & {\itshape kambïn} & {\itshape nungol} & \ili{Yuat}\\
‘machete’ & {\itshape itïpïn} & {\itshape yot} & \ili{Yuat}\\
‘guts’ & {\itshape ngïnda} & {\itshape inji} & \ili{Yuat}\\
‘frog’ & {\itshape kita} & {\itshape womotana} & \ili{Yuat}\\
‘hornbill’ & {\itshape sipal} & {\itshape almba} & \ili{Yuat}\\
‘grass skirt’ & {\itshape nandu {\textasciitilde} landu} & {\itshape ana} & \ili{Ap Ma}\\
‘rat’ & {\itshape yaki} & {\itshape matlaka} & \ili{Ap Ma}\\
‘small’ & {\itshape samola} & {\itshape njukuta} & \ili{Tok Pisin} (?)\\
\lspbottomrule
\end{tabular}
\end{table}
As \tabref{tab:18.22} suggests, the \ili{Yuat} family has had a greater influence on Yaul than on Manu. Compared to Manu, the three villages of Maruat, Dimiri, and Yaul are much closer geographically to the \ili{Yuat} family languages, in particular \ili{Mundukumo} and \ili{Bun}.

The Manu word \textit{ankam} ‘person’ derives from \ili{Proto-West Keram} form *alka-m. The Yaul word \textit{mbalanji} ‘person’, however, is borrowed from \ili{Yuat}: compare \ili{Bun} <mbɑlɑdʒi> \citep[5024]{Laycock1971b}. In Manu, \textit{mbalanji} has taken on the meaning ‘enemy’, an interesting \is{semantic change} \isi{semantic} shift, although not a surprising one, considering the traditional animosity between Manu village and the \ili{Yuat}-speaking people.

Yaul \textit{kambïn} ‘child’ likewise comes from \ili{Yuat}: compare \ili{Mundukumo} <kabǝn> \citep[29]{McElvenny2006}. I do not know the origin of the alternative Yaul form \mbox{\textit{sïmboy} ‘child’.} The alternative Manu form \textit{alum} ‘child’ may be a loan from \linebreak \ili{Ap Ma} <jalum> ‘grandson’ \citep[76]{Barlow2021}, although this word was apparently (also) borrowed as Manu \textit{yalum} ‘grandchild’. I do not know the origin of Manu \textit{tawatïp} ‘child’.

Yaul \textit{itïpïn} ‘machete’ seems to derive from \isi{compound}ing with a \ili{Yuat} word for ‘machete, knife’: compare \ili{Bun} <pi·n> \citep[5044]{Laycock1971b}.

Yaul \textit{ngïnda} ‘innards, guts’ may be compared, for example, with \ili{Mundukumo} <ŋɡəndɑ> ‘guts’ \citep[3128]{Laycock1971a}.

Yaul \textit{kita} ‘frog’ may be compared, for example, with \ili{Bun} <kitɑk> \citep[5040]{Laycock1971b}.

Yaul \textit{sipal} ‘hornbill’ probably also comes from \ili{Yuat}: compare \ili{Mundukumo} <sifwɑt>, \isi{plural} <sifɔlɛ> \citep[3194]{Laycock1971b}. A similar term is also found in the \ili{Lower Sepik} family: compare \ili{Chambri} <səbʊl> \citep[4982]{Laycock1971b}. The Manu form \textit{almba} ‘hornbill’, is a loan from \ili{Ap Ma} <alɨmba> ‘hornbill’ \citep[75]{Barlow2021}.

  Although the \ili{Ap Ma} language has probably exerted greater influence on Manu than it has on Yaul, there are a couple of borrowings from \ili{Ap Ma} attested only in the Yaul \isi{dialect}. Yaul \textit{nandu} {\textasciitilde} \textit{landu} ‘grass skirt’ likely comes from \ili{Ap Ma} <nando> ‘grass skirt’ \citep[80]{Barlow2021}. Yaul \textit{yaki} ‘rat’ likely comes from \linebreak \ili{Ap Ma} <jake> ‘rat species’ \citep[76]{Barlow2021}; this \ili{Ap Ma} form may ultimately be related somehow to Manu \textit{matlaka} ‘rat species’, but, if so, the exact nature of the relationship is unclear to me.
  
  Finally, Yaul \textit{samola} ‘small’ may derive from \ili{Tok Pisin} \textit{smol} {\textasciitilde} \textit{smolpela} ‘small’, if it is not a chance similarity.

\is{lexicon|)}
\is{dialect|)}
\il{Manu|)}
\il{Yaul|)}
\il{Maruat-Dimiri-Yaul|)}
\is{borrowing|)}
\is{loanword|)}

\il{Maruat-Dimiri-Yaul|(}
\il{Yaul|(}
\il{Manu|(}
\is{dialect|(}
\is{lexicon|(}

  In addition to \isi{lexical} differences due to \isi{borrowing}, there are some differences due to \is{semantic change} \isi{semantic} shift. For example, Manu \textit{misam} ‘brain’ has, through \isi{metonymy}, come to mean ‘head’ in Yaul. To refer to the internal organ, Yaul uses \textit{misam mu} ‘brain’ (literally ‘fruit of the head’). Yaul \textit{inangïn} ‘red’ is not used in Manu; rather \textit{ngungun} ‘red ant’ is used to refer to the color red as well as to the insect known for having that color. Whereas the Manu word \textit{yenanu {\textasciitilde} yananu} ‘woman’ can also mean ‘wife’, in Yaul this word may be restricted to the meaning ‘sister’; this could reflect differences in \isi{semantic extension}, if not simply a laxity in distinctions made within the \isi{semantic} category of ‘women’.

  Some Yaul words appear to have “extra” \isi{phonological} material, not reflected in the corresponding Manu forms (\tabref{tab:18.23}). \largerpage

\begin{table}
\caption{Yaul words with “extra” phonological material}
\is{phonology}
\label{tab:18.23}
\begin{tabular}{lll}
\lsptoprule
gloss & Yaul word & Manu word\\
\midrule
‘breast’ & {\itshape wal\textbf{um}} & {\itshape wol}\\
‘heart’ & {\itshape yam\textbf{un}} & {\itshape yom}\\
‘wing’ & {\itshape wapa\textbf{lup}} & {\itshape wapa}\\
‘shoulder’ & {\itshape awi\textbf{nam}} & {\itshape awi}\\
‘cockatoo’ & {\itshape yapu\textbf{ta}} & {\itshape yopa}\\
\lspbottomrule
\end{tabular}
\end{table}

The [-um] ending in Yaul \textit{walum} ‘breast’ likely results from \isi{metathesis} of the \isi{suffix}-like word \textit{mu} ‘fruit, seed, nut’.\footnote{This is sometimes realized as a \isi{suffix}-like element [um] in \ili{Mwakai} as well \citep[65]{Barlow2020a}.} The [-un] ending in Yaul \textit{yamun} ‘heart’, however, is more obscure. It may be a corruption of \textit{mu {\textasciitilde} um} ‘fruit, seed, nut’. Or perhaps it results from a shortening of some other internal organ meaning (cf. \textit{inji} ‘liver’ [Yaul], \textit{inji} ‘innards’ [Manu], \textit{ina} ‘liver’ [Manu]).

The [-lup] ending of Yaul \textit{wapalup} ‘wing’ may be related to \textit{lup} ‘base of a shell, bottom’ (cf. \textit{ilup} ‘elbow’, from \textit{i} ‘arm’), perhaps added as a means of disambiguating \textit{wapa(-lup)} ‘wing’ from the \isi{homophonous} word \textit{wapa} ‘leaf’.

Perhaps the [-nam] ending of Yaul \textit{awinam} ‘shoulder’ is related somehow to Manu \textit{nambï} ‘skin, body’, although this might be a stretch \isi{semantic}ally.

Finally, the [-ta] in Yaul \textit{yaputa} ‘cockatoo’ probably results from \isi{compound}ing with \textit{wuta} ‘bird’. The word for ‘cockatoo’ in Ulwa may ultimately be \isi{borrow}ed from \ili{Yuat}: compare \ili{Bun} <yɑƀɸwɑk> ‘cockatoo’ \citep[5040]{Laycock1971b}. It was possibly a \isi{borrowing} into \ili{Proto-West Keram}, since Ulwa’s sisters in this branch appear to exhibit \isi{cognate} forms.

  In one case, Manu seems to be the \isi{dialect} to have added some additional \isi{phonological} material (\tabref{tab:18.24}).


\begin{table}
\caption{A Manu word with “extra” phonological material}
\is{phonology}
\label{tab:18.24}


\begin{tabular}{lll}

\lsptoprule

gloss & Yaul word & Manu word\\
\midrule
‘pigeon’ & {\itshape walim} & {\itshape walim\textbf{ot}}\\
\lspbottomrule
\end{tabular}
\end{table}
This word is a \isi{loan} from \ili{Yuat}: compare \ili{Mundukumo} <wɑlim> ‘pigeon’ \linebreak \citep[3192]{Laycock1971a}. In this case, the Manu form \textit{walimot} ‘pigeon’ probably results from the \isi{compound}ing of \textit{walim} ‘pigeon’ and \textit{uta} ‘bird’.

  For whatever reason, insects seem to be a particularly unstable \isi{semantic} category in Ulwa, exhibiting not only \isi{lexical} replacement but also irregular \isi{sound change}s (\tabref{tab:18.25}).

Manu \textit{mïmin} ‘louse’ refers to lice on humans; \textit{sïmin} ‘louse’ is used to refer to lice on animals. The Yaul form \textit{mïn} ‘louse’, however, is a hypernym referring to all types of lice. Stable insect names across both dialects seem to be limited to \textit{kïka} ‘white ant, termite’, \textit{ngungun} ‘red ant’, and \textit{mu} ‘blowfly’.


\begin{table}
\caption{Insect words in Yaul and Manu}
\label{tab:18.25}


\begin{tabular}{lll}

\lsptoprule

gloss & Yaul word & Manu word\\
\midrule
‘centipede’ & {\itshape imb\textbf{a}p\textbf{ïlat}} & {\itshape y\textbf{a}mbïp\textbf{al}}\\
‘cockroach’ & {\itshape kusimba} & {\itshape ngunguswa}\\
‘firefly’ & {\itshape la\textbf{l}i} & {\itshape na\textbf{l}i}\\
‘louse’ & {\itshape m\textbf{ï}n} & {\itshape \textbf{mï}m\textbf{i}n}\\
‘grasshopper’ & {\itshape mïndisinam} & {\itshape kukum}\\
‘brown ant’ & {\itshape m\textbf{o}la\textbf{n}} & {\itshape m\textbf{u}na}\\
‘bedbug’ & {\itshape m\textbf{o}mb\textbf{ï}n} & {\itshape m\textbf{a}mb\textbf{u}n}\\
‘bee’ & {\itshape \textbf{m}unapïn} & {\itshape unapïn}\\
‘mosquito’ & {\itshape \textbf{n}angun} & {\itshape \textbf{y}angun}\\
‘wasp’ & {\itshape numbum} & {\itshape numbu\textbf{nu}m}\\
‘millipede’ & {\itshape ngunjimba} & {\itshape aylat}\\
‘housefly’ & {\itshape nji\textbf{w}ala} & {\itshape nji\textbf{m}ana}\\
‘butterfly’ & {\itshape yakalapana} & {\itshape awpane}\\
\lspbottomrule
\end{tabular}
\end{table}

\is{lexicon|)}
\is{dialect|)}
\il{Manu|)}
\il{Yaul|)}
\il{Maruat-Dimiri-Yaul|)}

  Finally, as already suggested (\sectref{sec:18.2}), the class of property-denoting words (i.e., \isi{adjective}s) is not very stable in Ulwa. Some examples of \isi{lexical} divergence within this class are given in \tabref{tab:18.26}.

\begin{table}
\caption{Adjectives in Yaul and Manu}
\is{adjective}
\label{tab:18.26}


\begin{tabular}{lll}

\lsptoprule

gloss & Yaul word & Manu word\\
\midrule
‘black’ & {\itshape imkïl} & {\itshape mbun}\\
‘hot’ & {\itshape imamal} & {\itshape wananum}\\
‘cold’ & {\itshape tangaliwa} & {\itshape mïnoma}\\
‘heavy’ & {\itshape nanal} & {\itshape kenmbu}\\
‘short’ & {\itshape wanum} & {\itshape mundotoma}\\
‘wide’ & {\itshape lalame} & {\itshape palmana}\\
‘small’ & {\itshape samola} & {\itshape njukuta}\\
\lspbottomrule
\end{tabular}
\end{table}

\is{lexicon}
\is{dialect}
\il{Manu}
\il{Yaul}
\il{Maruat-Dimiri-Yaul}


\pagebreak
\section{\label{sec:18.7}  Yaul dialect wordlist}

\is{wordlist|(}
\is{dialect|(}
\il{Yaul|(}
\il{Maruat-Dimiri-Yaul|(}

This chapter concludes with a wordlist of 500 items from the Yaul \isi{dialect} of Ulwa. The lexicographical conventions followed in \sectref{sec:17.1} apply here as well.

\il{Maruat-Dimiri-Yaul|)}
\il{Yaul|)}
\is{dialect|)}
\is{wordlist|)}

\begin{enumerate}[noitemsep, label={}, align=left, widest=190, labelsep=1ex,leftmargin=*,itemindent=-10pt]

\item 
\textbf{\textit{-a}} [\isi{intensive} pronominal \isi{suffix}, ‘\textsc{int}’] (also \textbf{\textit{-awa}}) \item
\textbf{\textit{akat}} (\textsc{n}) ‘hoof’ \item
\textbf{\textit{akïlaka}} (\textsc{adj}) ‘new’ \item
\textbf{\textit{akïlisa}} (\textsc{adj)} ‘sharp’ \item
\textbf{\textit{akum}} (\textsc{n)} ‘back of the skull; cassowary casque’ \item
\textbf{\textit{al}} (\textsc{n)} ‘mosquito net’ \item
\textbf{\textit{ala-}} (\textsc{v}) ‘see’ (with \textbf{\textit{nïmndï}} ‘eye’) (also \textbf{\textit{andï-}}) \item
\textbf{\textit{alam}} \textsc{(n)} ‘cloud, sky’ \item
\textbf{\textit{alam kot-} }(\textsc{v)} ‘thunder’ \item
\textbf{\textit{alamas}} (\textsc{n)} ‘snake species (python) (TP \textit{moran})’ \item
\textbf{\textit{ale}} (\textsc{n)} ‘sun, day (daytime)’ \item
\textbf{\textit{ale-}} (\textsc{v)} ‘scrape (sago)’ (possibly < \ili{Ap Ma}) \item
\textbf{\textit{ale ngume}} (\textsc{n)} ‘rainbow’ \item
\textbf{\textit{alekwal}} (\textsc{n)} ‘uvula’ \item
\textbf{\textit{alesa}} (\textsc{n)} ‘pick-axe (for hacking at sago palms)’ (possibly < \ili{Pondi}) \item
\textbf{\textit{ali}} (\textsc{n)} ‘string bag, net bag (TP \textit{bilum}); uterus, womb’ \item
\textbf{\textit{alïnam}} (\textsc{n)} ‘cloth, clothing’ \item
\textbf{\textit{alis}} (\textsc{n)} ‘lime gourd’ \item
\textbf{\textit{alkïn}} (\textsc{n)} ‘blood’ \item
\textbf{\textit{alkïn}} (\textsc{n)} ‘vegetable species (TP \textit{kumu mosong})’ \item
\textbf{\textit{almo}} (\textsc{adj)} ‘good’ (also \textbf{\textit{anmo}}) \item
\textbf{\textit{almoka}} (\textsc{n)} ‘snake’ \item
\textbf{\textit{am}} (\textsc{n)} ‘betel nut’ (also \textbf{\textit{awm}}) \item 
\textbf{\textit{am inji}} (\textsc{n)} ‘betel nut meat’ \item
\textbf{\textit{ama-}} (\textsc{v)} ‘eat, drink, suck’ \item
\textbf{\textit{amakaya}} (\textsc{n)} ‘hoe, digging tool’ \item
\textbf{\textit{amba}} (\textsc{n)} ‘men’s house, spirit house (TP \textit{haus tambaran} or \textit{haus boi})’ \item
\textbf{\textit{amba=}} (\textsc{pro)} ‘myself, yourself, himself, herself, itself’ (\isi{singular} \isi{reflexive} \linebreak \isi{pronoun}) \item
\textbf{\textit{ambal=}} (\textsc{pro)} ‘ourselves [\textsc{pl}], yourselves [\textsc{pl}], themselves [\textsc{pl}]; one another’ \linebreak(\isi{plural} \isi{reflexive}/\isi{reciprocal} \isi{pronoun})  \item
\textbf{\textit{ambas}} (\textsc{n)} ‘chewed-up betel nut’ \item
\textbf{\textit{ambatïm}} (\textsc{n)} ‘joint’ \item
\textbf{\textit{ambi}} (\textsc{adj)} ‘big, large, thick’ \item
\textbf{\textit{ambïla}} (\textsc{n)} ‘tooth’ \item
\textbf{\textit{ambïla wïla}} (\textsc{n)} ‘gums’ \item
\textbf{\textit{ambïla wokïn}} (\textsc{n)} ‘molar’  \item
\textbf{\textit{ambin=}} (\textsc{pro)} ‘ourselves [\textsc{du}], yourselves [\textsc{du}], themselves [\textsc{du}]; each other’ \linebreak(\isi{dual} \isi{reflexive}/\isi{reciprocal} \isi{pronoun})  \item
\textbf{\textit{ambo}} [\isi{topic marker}, ‘\textsc{top’}] \item
\textbf{\textit{ambol}} (\textsc{n)} ‘snake species (small snake)’ \item
\textbf{\textit{ame}} (\textsc{n)} ‘basket’ \item
\textbf{\textit{amolapa}} (\textsc{n)} ‘vegetable species (TP \textit{tulip})’ \item
\textbf{\textit{amun}} (\textsc{adv)} ‘now, today’ \item
\textbf{\textit{an}} (\textsc{pro)} ‘we’ (1\textsc{pl.excl} subject \isi{pronoun}) \item
\textbf{\textit{an=}} (\textsc{pro)} ‘us’ (1\textsc{pl.excl} non-subject \isi{pronoun}) \item
\textbf{\textit{ana-}} (\textsc{v)} ‘scratch, rub’ \item
\textbf{\textit{anam}} (\textsc{n)} ‘paddle’ \item
\textbf{\textit{anda}} (\textsc{dem)} ‘that, that one’ (\isi{singular} \isi{distal} \isi{demonstrative}) \item
\textbf{\textit{anda=}} (\textsc{dem)} ‘that, that one’ (\isi{singular} \isi{distal} \isi{demonstrative}) \item
\textbf{\textit{andan}} (\textsc{n)} ‘left (not right)’ \item
\textbf{\textit{ande}} (\textsc{interj)} ‘yes’ (expresses affirmation) (also \textbf{\textit{andi}}) \item
\textbf{\textit{andi}} (\textsc{interj)} ‘yes’ (expresses affirmation) (also \textbf{\textit{ande}}) \item
\textbf{\textit{andï}} (\textsc{n)} ‘affine, in-law’ \item
\textbf{\textit{andï}} (\textsc{n)} ‘sago shoot’ \item
\textbf{\textit{andï-}} (\textsc{v}) ‘see’ (with \textbf{\textit{nïmndï}} ‘eye’) (also \textbf{\textit{ala-}}) \item
\textbf{\textit{anga}} (\textsc{n)} ‘piece’ \item
\textbf{\textit{angenka}} (\textsc{adv)} ‘afterwards, later’ (also \textbf{\textit{laka}}, \textbf{\textit{naka}}) \item
\textbf{\textit{ango}} (\textsc{neg)} ‘no, not’ \item
\textbf{\textit{ango lïwa}} (\textsc{q)} ‘where?’ \item
\textbf{\textit{angos}} (\textsc{q)} ‘what?’ \item
\textbf{\textit{angos lakap}} (\textsc{q)} ‘why?’ \item
\textbf{\textit{angun}} (\textsc{n)} ‘tail, fin’ \item
\textbf{\textit{anjikeka}} (\textsc{q)} ‘when?’ \item
\textbf{\textit{anmo}} (\textsc{adj)} ‘good’ (also \textbf{\textit{almo}}) \item
\textbf{\textit{apa}} (\textsc{n)} ‘house’ \item
\textbf{\textit{apatam}} (\textsc{n)} ‘table, shelf’ \item
\textbf{\textit{apïka}} (\textsc{adv)} ‘very’ \item
\textbf{\textit{apïn}} (\textsc{n)} ‘fire’ \item
\textbf{\textit{apïn ama-}} (\textsc{v)} ‘burn’ \item
\textbf{\textit{apïngïn}} (\textsc{n)} ‘smoke’ \item
\textbf{\textit{apïnji}} (\textsc{n)} ‘ashes’ \item
\textbf{\textit{as}} (\textsc{n)} ‘grass’ \item
\textbf{\textit{asa-}} (\textsc{v)} ‘hit, shoot; kill’ (also \textbf{\textit{wali-}}) \item
\textbf{\textit{ase}} (\textsc{n)} ‘string, trap (for catching land animals)’ \item
\textbf{\textit{asi ka-}} (\textsc{v)} ‘sit’ \item
\textbf{\textit{asum}} (\textsc{n)} ‘rice’ \item
\textbf{\textit{ata}} (\textsc{adv)} ‘up’; (\textsc{n)} ‘top’ \item
\textbf{\textit{atal}} (\textsc{n)} ‘laughter’ \item
\textbf{\textit{atal a-}} (\textsc{v)} ‘laugh’ \item
\textbf{\textit{atana}} (\textsc{n)} ‘sister (older sister)’ \item
\textbf{\textit{atumo}} (\textsc{n)} ‘brother (older brother)’ \item
\textbf{\textit{-awa}} [\isi{intensive} pronominal \isi{suffix}, ‘\textsc{int}’] (also \textbf{\textit{-a}}) \item
\textbf{\textit{awasingïn}} (\textsc{n)} ‘bird species (eagle, hawk) (TP \textit{tarangau})’ \item
\textbf{\textit{awaw}} (\textsc{n)} ‘afternoon’; (\textsc{adv)} ‘yesterday’ \item
\textbf{\textit{aweta}} (\textsc{n)} ‘friend’ \item
\textbf{\textit{awinam}} (\textsc{n)} ‘shoulder’ \item
\textbf{\textit{awm}} (\textsc{n)} ‘betel nut’ (also \textbf{\textit{am}}) \item
\textbf{\textit{ay}} (\textsc{n)} ‘sago jelly, jellied sago’ \item
\textbf{\textit{aymom}} (\textsc{n)} ‘sago stick (for stirring sago)’ \item
\textbf{\textit{aypul}} (\textsc{n)} ‘scoop of jellied sago’ \item
\textbf{\textit{-e}} [\isi{imperfective} \isi{suffix}, ‘\textsc{ipfv}’] (also \textbf{\textit{-i}}) \item
\textbf{\textit{i}} (\textsc{n)} ‘arm’ \item
\textbf{\textit{i}} (\textsc{n)} ‘lime (calcium hydroxide) (TP \textit{kambang})’ (< \ili{Ap Ma}) \item
\textbf{\textit{i}} (\textsc{v)} ‘go’ (\isi{suppletive} \isi{perfective} form of \textbf{\textit{ma-}}) (also \textbf{\textit{iye}}) \item
\textbf{\textit{i-}} (\textsc{v)} ‘come’ \item
\textbf{\textit{-i}} [\isi{imperfective} \isi{suffix}, ‘\textsc{ipfv}’] (also \textbf{\textit{-e}}) \item
\textbf{\textit{ila}} (\textsc{n)} ‘thatch (TP \textit{morota})’ \item
\textbf{\textit{ilam}} (\textsc{n)} ‘day (countable)’ \item
\textbf{\textit{ilapum}} (\textsc{n)} ‘right (not left)’ (also \textbf{\textit{inapum}}) \item
\textbf{\textit{ilaw}} (\textsc{n)} ‘fat, grease’ \item
\textbf{\textit{ilis}} (\textsc{n)} ‘dust’ \item
\textbf{\textit{ilup}} (\textsc{n)} ‘elbow’ \item
\textbf{\textit{im}} (\textsc{n)} ‘tree’ \item
\textbf{\textit{im nambim}} (\textsc{n)} ‘bark’ \item
\textbf{\textit{imamal}} (\textsc{adj)} ‘hot, warm’ \item
\textbf{\textit{imba}} (\textsc{n)} ‘night’ \item
\textbf{\textit{imbam}} (\textsc{p)} ‘under, below’ \item
\textbf{\textit{imbapïlat}} (\textsc{n)} ‘insect species (centipede)’ \item
\textbf{\textit{imkïl}} (\textsc{adj)} ‘black’ \item
\textbf{\textit{imot}} (\textsc{n)} ‘firewood, stick, log’ \item
\textbf{\textit{in}} (\textsc{n)} ‘ground, land’ \item
\textbf{\textit{in}} (\textsc{p)} ‘in, inside’ \item
\textbf{\textit{ina}} (\textsc{v)} ‘come’ (irregular \isi{irrealis} form of \textbf{\textit{i-}}) \item
\textbf{\textit{inam}} (\textsc{n)} ‘mother’ \item
\textbf{\textit{inam ngata}} (\textsc{n)} ‘grandmother, old woman’ \item
\textbf{\textit{inanan}} (\textsc{n)} ‘nail, fingernail’ \item
\textbf{\textit{inangïn}} (\textsc{adj)} ‘red’ \item
\textbf{\textit{inapam}} (\textsc{n)} ‘belly’ \item
\textbf{\textit{inapum}} (\textsc{n)} ‘right (not left)’ (also \textbf{\textit{ilapum}}) \item
\textbf{\textit{inat}} (\textsc{n)} ‘earth, soil’ \item
\textbf{\textit{indï}} (\textsc{n)} ‘eye’ (when used with \textbf{\textit{ala-}} \textit{{\textasciitilde}} \textbf{\textit{andï-}} ‘see’) \item
\textbf{\textit{indïp}} (\textsc{n)} ‘cassowary bone’ \item
\textbf{\textit{inim}} (\textsc{n)} ‘water, rain’ \item
\textbf{\textit{inim lopo-}} (\textsc{v)} ‘bathe’ \item
\textbf{\textit{inim nji}} (\textsc{n)} ‘dew’ \item
\textbf{\textit{inim pul}} (\textsc{n)} ‘pond’ \item
\textbf{\textit{inïmbï}} (\textsc{n)} ‘vulva’ \item
\textbf{\textit{inïmi}} (\textsc{n)} ‘hole’ \item
\textbf{\textit{inji}} (\textsc{n)} ‘liver’ \item
\textbf{\textit{ip}} (\textsc{n)} ‘nose’ \item
\textbf{\textit{ipïka}} (\textsc{adv)} ‘before’ \item
\textbf{\textit{ita-}} (\textsc{v)} ‘build’ \item
\textbf{\textit{itam}} (\textsc{n)} ‘father’ \item
\textbf{\textit{itam ngata}} (\textsc{n)} ‘grandfather, old man’ \item
\textbf{\textit{itam wot}} (\textsc{n)} ‘father’s brother (paternal uncle)’ \item
\textbf{\textit{itïpïn}} (\textsc{n)} ‘machete, knife’ (< \ili{Yuat}) \item
\textbf{\textit{ititïm}} (\textsc{n)} ‘insect species (black ant)’ \item
\textbf{\textit{iwa}} (\textsc{n)} ‘fish trap’ \item
\textbf{\textit{iwal}} (\textsc{n)} ‘beam’ \item
\textbf{\textit{iwali}} (\textsc{n)} ‘armband, legband’ \item
\textbf{\textit{iwïl}} (\textsc{n)} ‘moon’ \item
\textbf{\textit{iwïl}} (\textsc{n)} ‘root’ \item
\textbf{\textit{iye}} (\textsc{v)} ‘go’ (\isi{suppletive} \isi{perfective} form of \textbf{\textit{ma-}}) (also \textbf{\textit{i}}) \item
\textbf{\textit{ka-}} (\textsc{v)} ‘let’ \item
\textbf{\textit{kakas}} (\textsc{adj)} ‘long, tall’ \item
\textbf{\textit{kalalum}} (\textsc{n)} ‘boil, abscess’ \item
\textbf{\textit{kalim}} (\textsc{n)} ‘cassowary (TP \textit{muruk})’ (< \ili{Yuat}) \item
\textbf{\textit{kambïn}} (\textsc{n)} ‘child’ (< \ili{Yuat}) (also \textbf{\textit{sïmboy}}) \item
\textbf{\textit{kï-}} (\textsc{v)} ‘talk, say’ (also \textbf{\textit{ta-}}) \item
\textbf{\textit{kïka}} (\textsc{n)} ‘insect species (white ant, termite)’ \item
\textbf{\textit{kïkal}} (\textsc{n)} ‘ear’ \item
\textbf{\textit{kïkal wala-}} (\textsc{v)} ‘hear’ \item
\textbf{\textit{kïlikïli}} (\textsc{n)} ‘frog species (small frog)’ \item
\textbf{\textit{kita}} (\textsc{n)} ‘frog’ (< \ili{Yuat}) \item
\textbf{\textit{ko}} [\isi{non-verbal negator}, ‘\textsc{neg}’] (also \textbf{me}) \item
\textbf{\textit{ko=}} [\isi{indefinite} marker, ‘\textsc{indf}’] \item
\textbf{\textit{koke}} (\textsc{q)} ‘who?’ (also \textbf{\textit{kwo}}) \item
\textbf{\textit{kot-}} (\textsc{v)} ‘break’ \item
\textbf{\textit{koy}} (\textsc{n)} ‘chest’ \item
\textbf{\textit{kume}} (\textsc{quant)} ‘some’ \item
\textbf{\textit{kusimba}} (\textsc{n)} ‘insect species (cockroach)’ \item
\textbf{\textit{kwe}} (\textsc{num)} ‘one’ \item
\textbf{\textit{kwo}} (\textsc{q)} ‘who?’ (also \textbf{\textit{koke}}) \item
\textbf{\textit{la}} (\textsc{n)} ‘talk, speech, story’ \item
\textbf{\textit{-la}} [\isi{irrealis} \isi{suffix}, ‘\textsc{irr}’] (also \textbf{\textit{-nda}}) \item
\textbf{\textit{laka}} (\textsc{adv)} ‘afterwards, later’ (also \textbf{\textit{naka}}, \textbf{\textit{angenka}}) \item
\textbf{\textit{lakap}} (\textsc{n)} ‘reason’ \item
\textbf{\textit{lalame}} (\textsc{adj)} ‘wide’ \item
\textbf{\textit{lali}} (\textsc{n)} ‘insect species (firefly); star’ \item
\textbf{\textit{lam}} (\textsc{n)} ‘meat, flesh’ (< \ili{Ap Ma}) \item
\textbf{\textit{landa}} (\textsc{v)} ‘eat, drink, suck’ (irregular \isi{irrealis} form of \textbf{\textit{ama-}}) \item
\textbf{\textit{landu}} (\textsc{n)} ‘skirt, woman’s grass skirt (TP \textit{purpur})’ (also \textbf{\textit{nandu}}) (< \ili{Ap Ma}) \item
\textbf{\textit{lap}} (\textsc{n)} ‘fishing spear’ \item
\textbf{\textit{law}} (\textsc{n)} ‘bunch (of bananas)’ \item
\textbf{\textit{law}} (\textsc{n)} ‘tree species (TP \textit{tanget})’ \item
\textbf{\textit{le}} (\textsc{n)} ‘rattan, cane (TP \textit{kanda})’ (possibly < \ili{Ap Ma}) \item
\textbf{\textit{lemol}} (\textsc{n)} ‘vegetable species (TP \textit{aibika})’ \item
\textbf{\textit{lenjin}} (\textsc{n)} ‘fish species (perch) (TP \textit{nilpis})’ (< \ili{Ap Ma}) \item
\textbf{\textit{li}} (\textsc{adv}) ‘down’; (\textsc{n}) ‘bottom’ (possibly < \ili{Ap Ma}) \item
\textbf{\textit{li}} \textsc{(n)} ‘crayfish, prawn’ \item
\textbf{\textit{li-}} (\textsc{v}) ‘beat; blow (of wind)’ \item
\textbf{\textit{li-}} (\textsc{v}) ‘die’ \item
\textbf{\textit{li u-}} (\textsc{v}) ‘fall’ \item
\textbf{\textit{lïkï-}} (\textsc{v}) ‘dig’ \item
\textbf{\textit{lïkït}} (\textsc{n}) ‘lizard’ \item
\textbf{\textit{lil}} (\textsc{n}) ‘fish species (eel)’ \item
\textbf{\textit{lila}} (\textsc{num}) ‘three’ \item
\textbf{\textit{lïmndï}} (\textsc{n}) ‘eye’ (also \textbf{\textit{nïmndï}}) \item
\textbf{\textit{lin}} (\textsc{n}) ‘thorn’ \item
\textbf{\textit{lïngïn}} (\textsc{n}) ‘spider’ \item
\textbf{\textit{lïpa}} (\textsc{n}) ‘breadfruit’ \item
\textbf{\textit{lïpïl}} (\textsc{n}) ‘vine, rope’ \item
\textbf{\textit{lïpïlopa}} (\textsc{n}) ‘flying fox’ \item
\textbf{\textit{lisa}} ‘drum (type of drum: small hand drum) (TP \textit{kundu})’ \item
\textbf{\textit{lïwa}} (\textsc{n}) ‘dawn’ \item
\textbf{\textit{lïwa}} (\textsc{n}) ‘jungle, woods, forest, bush; place’ (also \textbf{\textit{wandam}}) \item
\textbf{\textit{lo-}} (\textsc{v}) ‘cut, carve; go around’ (also \textbf{\textit{yo-}}) \item
\textbf{\textit{lokal}} (\textsc{n}) ‘beak’ \item
\textbf{\textit{lom}} (\textsc{n}) ‘stand (used to hold a pot)’ \item
\textbf{\textit{lopa}} (\textsc{n}) ‘cheek’ \item
\textbf{\textit{lu}} (\textsc{adj}) ‘near, close’ \item
\textbf{\textit{lup}} (\textsc{n}) ‘base of a shell, bottom’ \item
\textbf{\textit{ma=}} (\textsc{pro}) ‘him, her, it’ (3\textsc{sg} non-subject \isi{pronoun} / \isi{singular} \isi{object marker}) \item
\textbf{\textit{ma-}} (\textsc{v}) ‘go’ \item
\textbf{\textit{mala}} (\textsc{n}) ‘spear’ \item
\textbf{\textit{malal}} (\textsc{n}) ‘hot water’ \item
\textbf{\textit{malal ama-}} (\textsc{v}) ‘boil’ \item
\textbf{\textit{malo}} (\textsc{n}) ‘loincloth’ (< \ili{Tok Pisin} \textit{malo} ‘loincloth’) \item
\textbf{\textit{mama}} (\textsc{n}) ‘mouth’ \item
\textbf{\textit{mamapa}} (\textsc{n}) ‘bird species (owl)’ \item
\textbf{\textit{mana}} (\textsc{v}) ‘go’ (irregular \isi{irrealis} form of \textbf{\textit{ma-}}) \item
\textbf{\textit{manana}} (\textsc{n}) ‘snail’ \item
\textbf{\textit{may}} (\textsc{n}) ‘fish species (catfish) (TP \textit{mausgras pis})’ \item
\textbf{\textit{me}} [\isi{non-verbal negator}, ‘\textsc{neg}’] (also \textbf{\textit{ko}}) \item
\textbf{\textit{me}} (\textsc{n}) ‘betel nut palm stem (TP \textit{buai limbum})’ \item
\textbf{\textit{mï}} (\textsc{pro}) ‘he, she, it’ (3\textsc{sg} subject \isi{pronoun}) \item
\textbf{\textit{mïlale}} (\textsc{n}) ‘guts, intestines’ \item
\textbf{\textit{mïli}} (\textsc{n}) ‘sugarcane’ \item
\textbf{\textit{mïlïkïn}} (\textsc{n}) ‘grub species (sago grub)’ (< \ili{Ap Ma}) \item
\textbf{\textit{mïlïm}} (\textsc{n}) ‘tongue’ (also \textbf{\textit{mïnïm}}) \item
\textbf{\textit{miminya}} (\textsc{n}) ‘feces, excrement’ \item
\textbf{\textit{min}} (\textsc{pro}) ‘they two’ (3\textsc{du} subject \isi{pronoun}) \item
\textbf{\textit{min=}} (\textsc{pro}) ‘them two’ (3\textsc{du} non-subject \isi{pronoun} / \isi{dual} \isi{object marker}) \item
\textbf{\textit{mïn}} (\textsc{n}) ‘louse’ \item
\textbf{\textit{mïnal}} (\textsc{n}) ‘taro’; (\textsc{adj}) ‘green’ \item
\textbf{\textit{mïnam}} (\textsc{n}) ‘urine’ \item
\textbf{\textit{mïnawata}} (\textsc{adj}) ‘wet; ripe’ \item
\textbf{\textit{mïnda}} (\textsc{n}) ‘banana’ \item
\textbf{\textit{mïndam}} (\textsc{n}) ‘pus’ \item
\textbf{\textit{mïndapa}} (\textsc{n}) ‘banana leaf’ \item
\textbf{\textit{mïndisinam}} (\textsc{n}) ‘insect species (grasshopper)’ \item
\textbf{\textit{mïndït}} (\textsc{adj}) ‘yellow’ \item
\textbf{\textit{mini=}} \textsc{(pro}) ‘them two’ (3\textsc{du} non-subject \isi{pronoun} / \isi{dual} \isi{object marker}) \linebreak(\isi{allomorph} of \textbf{\textit{min=}}) \item
\textbf{\textit{mïnïm}} (\textsc{n}) ‘tongue’ (also \textbf{\textit{mïlïm}}) \item
\textbf{\textit{misam}} (\textsc{n}) ‘head’ \item
\textbf{\textit{misam mum}} (\textsc{n}) ‘brain’ \item
\textbf{\textit{mïtïn}} (\textsc{n}) ‘egg’ \item
\textbf{\textit{mïtïn}} (\textsc{n}) ‘testicles’ \item
\textbf{\textit{mo}} (\textsc{n}) ‘forehead, face; middle’ \item
\textbf{\textit{molam}} (\textsc{n}) ‘tree species (tree with red sap)’ \item
\textbf{\textit{molan}} (\textsc{n}) ‘insect species (brown ant)’ \item
\textbf{\textit{molap}} (\textsc{adj}) ‘full, sated’ \item
\textbf{\textit{molawi}} (\textsc{n}) ‘plant species (stinging nettle) (TP \textit{salat})’ \item
\textbf{\textit{molïkïn}} (\textsc{n}) ‘gray hair’ \item
\textbf{\textit{molpan}} (\textsc{n}) ‘spirit (type of spirit: tree spirit)’ \item
\textbf{\textit{mombïn}} (\textsc{n}) ‘insect species (bedbug)’ \item
\textbf{\textit{mombïn}} (\textsc{n}) ‘vegetable species (TP \textit{aupa})’ \item
\textbf{\textit{mon}} (\textsc{n}) ‘betel pepper species (wild betel pepper) (TP \textit{wel daka})’ \item
\textbf{\textit{monambam}} (\textsc{n}) ‘forehead’ \item
\textbf{\textit{monata}} (\textsc{n}) ‘earthquake’ \item
\textbf{\textit{mongi}} (\textsc{n}) ‘mask’ \item
\textbf{\textit{motam}} (\textsc{num}) ‘five’ \item
\textbf{\textit{motam kwe ndïwatke}} (\textsc{num}) ‘six’ \item
\textbf{\textit{motam lila ndïwatke}} (\textsc{num}) ‘eight’ \item
\textbf{\textit{motam nange ndïwatke}} (\textsc{num}) ‘nine’ \item
\textbf{\textit{motam nini}} (\textsc{num}) ‘ten’ (also \textbf{\textit{motam wupa}}) \item
\textbf{\textit{motam nini ndïwatke}} (\textsc{num}) ‘seven’ \item
\textbf{\textit{motam wupa num}} ‘ten’ (also \textbf{\textit{motam nini}}) \item
\textbf{\textit{mu}} (\textsc{n}) ‘fruit, seed, nut’ (also \textbf{\textit{mum}}) \item
\textbf{\textit{mu}} (\textsc{n}) ‘insect species (blowfly)’ \item
\textbf{\textit{mum}} (\textsc{n}) ‘fruit, seed, nut’ (also \textbf{\textit{mu}}) \item
\textbf{\textit{munapïn}} (\textsc{n}) ‘insect species (bee)’ \item
\textbf{\textit{mundï} }(\textsc{n}) ‘hunger, food’ \item
\textbf{\textit{mundï asa-}} (\textsc{v}) ‘be hungry’ \item
\textbf{\textit{mungul}} (\textsc{n}) ‘plant species (fern)’ \item
\textbf{\textit{mup}} (\textsc{n}) ‘core of a tree’ \item
\textbf{\textit{mutam}} (\textsc{n}) ‘back (of the body)’ \item
\textbf{\textit{mutulum}} (\textsc{n}) ‘mud’ \item
\textbf{\textit{mbalanji}} (\textsc{n}) ‘person, human’ (< \ili{Yuat}) \item
\textbf{\textit{mbï}} (\textsc{adv}) ‘here’ \item
\textbf{\textit{mbïl ngom}} (\textsc{n}) ‘vegetable species (tall ginger) (TP \textit{gorgor})’ \item
\textbf{\textit{mbomala}} (\textsc{n}) ‘insect species (large firefly)’ \item
\textbf{\textit{mbone}} (\textsc{n}) ‘crab’ \item
\textbf{\textit{mbonem}} (\textsc{n}) ‘morning’ (also \textit{umbenam}) \item
\textbf{\textit{mbosanga}} (\textsc{n}) ‘pot (made of clay)’ \item
\textbf{\textit{mbosanga mu}} (\textsc{n}) ‘money’ \item
\textbf{\textit{n-}} (\textsc{v}) ‘want, will, try’ (\isi{volitive} \isi{auxiliary verb}, ‘\textsc{vol}’) \item
\textbf{\textit{=n}} \is{oblique marker} [oblique-marking \isi{enclitic}, ‘\textsc{obl}’] \item
\textbf{\textit{na}} (\textsc{conj}) ‘and’ (< \ili{Tok Pisin} \textit{na} ‘and’) \item
\textbf{\textit{na-}} (\textsc{v}) ‘give’ \item
\textbf{\textit{-na}} [irregular \isi{irrealis} \isi{suffix}, ‘\textsc{irr}’] (for \textbf{\textit{i-}} ‘come’) \item
\textbf{\textit{naka}} (\textsc{adv}) ‘afterwards, later’ (also \textbf{\textit{laka}}, \textbf{\textit{angenka}}) \item
\textbf{\textit{nambana}} (\textsc{n}) ‘spirit, ghost’ \item
\textbf{\textit{nambï}} (\textsc{adj}) ‘dirty’ \item
\textbf{\textit{nambim}} (\textsc{n}) ‘skin’ \item
\textbf{\textit{nambis}} (\textsc{n}) ‘odor, smell’ \item
\textbf{\textit{namïli}} (\textsc{adj}) ‘soft’ \item
\textbf{\textit{namndï}} (\textsc{n}) ‘pig’ (also \textbf{\textit{namndu}}) \item
\textbf{\textit{namndu}} (\textsc{n}) ‘pig’ (also \textbf{\textit{namndï}}) \item
\textbf{\textit{nana}} (\textsc{adj}) ‘afraid’ \item
\textbf{\textit{nanal}} (\textsc{adj}) ‘heavy’ \item
\textbf{\textit{nanange}} (\textsc{num}) ‘four’ \item
\textbf{\textit{nandu}} (\textsc{n}) ‘skirt, woman’s grass skirt (TP \textit{purpur})’ (also \textbf{\textit{landu}}) (< \ili{Ap Ma}) \item
\textbf{\textit{nangïn}} (\textsc{n}) ‘tongs (for cooking)’ \item
\textbf{\textit{nangontam}} (\textsc{n}) ‘sweet potato (TP \textit{kaukau})’’ \item
\textbf{\textit{nangun}} (\textsc{n}) ‘mosquito’ \item
\textbf{\textit{-ne}} [irregular \isi{perfective} \isi{suffix}, ‘\textsc{pfv}’] (for \textbf{\textit{i-}} ‘come’, \textbf{\textit{na-}} ‘give’) (also \textbf{\textit{-ni}}) \item
\textbf{\textit{-ni}} [irregular \isi{perfective} \isi{suffix}, ‘\textsc{pfv}’]  (for \textbf{\textit{i-}} ‘come’, \textbf{\textit{na-}} ‘give’) (also \textbf{\textit{-ne}})  \item
\textbf{\textit{nï}} (\textsc{pro}) ‘I’ (1\textsc{sg} subject \isi{pronoun}) \item
\textbf{\textit{nï=}} (\textsc{pro}) ‘me’ (1\textsc{sg} non-subject \isi{pronoun}) \item
\textbf{=\textit{nï}} \is{oblique marker} [oblique-marking \isi{enclitic}, ‘\textsc{obl}’] (\isi{allomorph} of \textbf{\textit{=n}}) \item
\textbf{\textit{nïmban}} (\textsc{n}) ‘fish’ \item
\textbf{\textit{nïmïn}} (\textsc{n}) ‘mucus’ \item
\textbf{\textit{nïmndï}} (\textsc{n}) ‘eye’ (also \textbf{\textit{lïmndï}}) \item
\textbf{\textit{nïmndï ala-}} (\textsc{v}) ‘see’ (also \textbf{\textit{nïmndï andï-}}) \item
\textbf{\textit{nïmndï andï-} }(\textsc{v}) ‘see’ (also \textbf{\textit{nïmndï ala-}}) \item
\textbf{\textit{nini}} (\textsc{num}) ‘two’ \item
\textbf{\textit{nongam}} (\textsc{n}) ‘dream’ \item
\textbf{\textit{nongan}} (\textsc{n}) ‘vomitus’ \item
\textbf{\textit{nongan u-}} (\textsc{v}) ‘vomit’ \item
\textbf{\textit{num}} (\textsc{n}) ‘canoe’ \item
\textbf{\textit{numan}} (\textsc{n}) ‘husband’ (also \textbf{\textit{numon}}) \item
\textbf{\textit{numbu}} (\textsc{n}) ‘drum (type of drum: large slit drum) (TP \textit{garamut})’ \item
\textbf{\textit{numbum}} (\textsc{n}) ‘insect species (wasp)’ \item
\textbf{\textit{numon}} (\textsc{n}) ‘husband’ (also \textbf{\textit{numan}}) \item
\textbf{\textit{numul}} (\textsc{n}) ‘river’ \item
\textbf{\textit{numulwa}} (\textsc{n}) ‘ditch, creek’ \item
\textbf{\textit{nungun}} (\textsc{n}) ‘seedling’ \item
\textbf{\textit{-nda}} [\isi{irrealis} \isi{suffix}, ‘\textsc{irr}’] (also \textbf{\textit{-la}}) \item
\textbf{\textit{ndam} }(\textsc{n}) ‘bridge’ \item
\textbf{\textit{ndï}} (\textsc{pro}) ‘they’ (3\textsc{pl} subject \isi{pronoun}) \item
\textbf{\textit{ndï=}} (\textsc{pro}) ‘them’ (3\textsc{pl} non-subject \isi{pronoun} / \isi{plural} \isi{object marker}) \item
\textbf{\textit{nga}} (\textsc{dem}) ‘this, this one’ (\isi{singular} \isi{proximal} \isi{demonstrative}) \item
\textbf{\textit{ngan}} (\textsc{pro}) ‘we two’ (\textsc{1du.excl} subject \isi{pronoun}) \item
\textbf{\textit{ngan=}} (\textsc{pro}) ‘us two’ (\textsc{1du.excl} non-subject \isi{pronoun}) \item
\textbf{\textit{ngangali}} (\textsc{n}) ‘poison, magic’ \item
\textbf{\textit{ngata}} (\textsc{adj}) ‘great, big, large’ \item
\textbf{\textit{ngïnda}} (\textsc{n}) ‘innards, guts’ (< \ili{Yuat}) \item
\textbf{\textit{ngïnïmo}} (\textsc{n}) ‘chin’ (< \ili{Yuat}) \item
\textbf{\textit{ngïtam}} (\textsc{n}) ‘mallet’ \item
\textbf{\textit{ngun}} (\textsc{pro}) ‘you two’ (2\textsc{du} subject \isi{pronoun}) \item
\textbf{\textit{ngun=}} (\textsc{pro)} ‘you two’ (2\textsc{du} non-subject \isi{pronoun}) \item
\textbf{\textit{nguna}} (\textsc{pro}) ‘we two’ (1\textsc{du.incl} subject \isi{pronoun}) \item
\textbf{\textit{nguna=}} (\textsc{pro}) ‘us two’ (1\textsc{du.incl} non-subject \isi{pronoun}) \item
\textbf{\textit{ngungun}} (\textsc{n}) ‘insect species (red ant) (TP \textit{karakum})’ \item
\textbf{\textit{ngungun}} (\textsc{n}) ‘wind, whirlwind, cyclone’ \item
\textbf{\textit{ngunjimba}} (\textsc{n}) ‘insect species (millipede)’ \item
\textbf{\textit{njanjani}} (\textsc{n}) ‘grub species (large grub)’ \item
\textbf{\textit{njanjni}} (\textsc{n}) ‘shadow, shade’ \item
\textbf{\textit{nji}} (\textsc{n}) ‘thing’ \item
\textbf{\textit{-nji}} [possessive \isi{suffix}, ‘\textsc{poss}’] \item
\textbf{\textit{njiwala}} (\textsc{n}) ‘fly, housefly’ \item
\textbf{\textit{p-}} (\textsc{v}) ‘be, be at’ \item
\textbf{\textit{=p}} [\isi{copular enclitic}, \textsc{‘cop’}] \item
\textbf{\textit{palam}} (\textsc{n}) ‘cane grass (TP \textit{pitpit})’ (< \ili{Mwakai}) \item
\textbf{\textit{-pe}} [\isi{perfective} \isi{suffix}, ‘\textsc{pfv}’] (also \textbf{\textit{-pi}}) \item
\textbf{\textit{-pi} }[\isi{perfective} \isi{suffix}, ‘\textsc{pfv}’] (also \textbf{\textit{-pe}}) \item
\textbf{\textit{pul}} (\textsc{n}) ‘piece’ \item
\textbf{\textit{sa-}} (\textsc{v}) ‘cry’ \item
\textbf{\textit{sakanma}} (\textsc{n}) ‘axe (metal)’ (< \ili{Yuat}) \item
\textbf{\textit{samban}} (\textsc{n}) ‘pot for cooking’ \item
\textbf{\textit{samola}} (\textsc{adj}) ‘small, little’ (possibly < \ili{Tok Pisin} \textit{smol} ‘small’) \item
\textbf{\textit{si}} (\textsc{n}) ‘salt, soup’ \item
\textbf{\textit{sïmboy}} (\textsc{n}) ‘child’ (also \textbf{\textit{kambïn}}) \item
\textbf{\textit{sïngïm}} (\textsc{n}) ‘fog’ (< \ili{Mwakai}) \item
\textbf{\textit{sipal}} (\textsc{n}) ‘bird species (hornbill) (TP \textit{kokomo})’ (< \ili{Yuat}; possibly an \isi{areal term}) \item
\textbf{\textit{sokay}} (\textsc{n}) ‘tobacco’ (\isi{areal term}) \item
\textbf{\textit{ta}} (\textsc{n}) ‘hearth, stove’ \item
\textbf{\textit{ta-}} (\textsc{v}) ‘talk, say’ (also \textbf{\textit{kï-}}) \item
\textbf{\textit{tale li-}} (\textsc{v)} ‘stand, be standing’ \item
\textbf{\textit{talum}} (\textsc{n}) ‘lips’ \item
\textbf{\textit{tambi}} (\textsc{adj}) ‘bad’ \item
\textbf{\textit{tambïnmot}} (\textsc{n}) ‘chest’ \item
\textbf{\textit{tambumana}} (\textsc{adj}) ‘dull’ \item
\textbf{\textit{tana}} (\textsc{n}) ‘axe, adze’ \item
\textbf{\textit{tangaliwa}} (\textsc{adj}) ‘cold, cool’ \item
\textbf{\textit{tap li-}} (\textsc{v}) ‘throw’ \item
\textbf{\textit{taw}} (\textsc{n}) ‘flute’ \item
\textbf{\textit{tawa}} (\textsc{n}) ‘wound, sore’ \item
\textbf{\textit{tawi}} (\textsc{n}) ‘saliva’ \item
\textbf{\textit{tï-}} (\textsc{v}) ‘take’ \item
\textbf{\textit{tïl}} (\textsc{n}) ‘coconut husk’ \item
\textbf{\textit{tïn}} (\textsc{n}) ‘dog’ \item
\textbf{\textit{tïnanga-}} (\textsc{v}) ‘stand, arise’ \item
\textbf{\textit{tongan}} (\textsc{n}) ‘mosquito swatter’ \item
\textbf{\textit{tumbunwa}} (\textsc{n}) ‘nape of the neck’ \item
\textbf{\textit{tup}} (\textsc{n}) ‘fish species (TP \textit{bikmaus})’ \item
\textbf{\textit{tuwalïm}} (\textsc{n}) ‘possum, cuscus (TP \textit{kapul})’ \item
\textbf{\textit{u}} (\textsc{pro}) ‘you [\textsc{sg}]’ (2\textsc{sg} subject \isi{pronoun}) \item
\textbf{\textit{u=}} (\textsc{pro}) ‘you [\textsc{sg}]’ (2\textsc{sg} non-subject \isi{pronoun}) \item
\textbf{\textit{u-}} (\textsc{v}) ‘put’ \item
\textbf{\textit{ulamban}} (\textsc{n}) ‘ladder’ \item
\textbf{\textit{ulanda}} (\textsc{n}) ‘riverbank’ \item
\textbf{\textit{ulet}} (\textsc{n}) ‘navel, umbilical cord’ \item
\textbf{\textit{uli-}} (\textsc{v}) ‘shout, call out’ \item
\textbf{\textit{ulpan}} (\textsc{n}) ‘fish species (small fish)’ \item
\textbf{\textit{ulum}} (\textsc{n}) ‘sago palm; sago pith’ \item
\textbf{\textit{ulwa}} (\textsc{neg}) ‘nothing’ \item
\textbf{\textit{uma}} (\textsc{n}) ‘bone’ \item
\textbf{\textit{umbapa}} (\textsc{n}) ‘stomach’ (< \ili{Ap Ma}) \item
\textbf{\textit{umbe}} (\textsc{adv}) ‘tomorrow’ \item
\textbf{\textit{umbenam}} (\textsc{n}) ‘morning’ (also \textbf{\textit{mbonem}}) \item
\textbf{\textit{umwa}} (\textsc{n}) ‘neck’ (also \textbf{\textit{uwa}}) \item
\textbf{\textit{un}} (\textsc{n}) ‘fan’ \item
\textbf{\textit{un}} (\textsc{pro}) ‘you [\textsc{pl}]’ (\textsc{2pl} subject \isi{pronoun}) \item
\textbf{\textit{un=}} (\textsc{pro}) ‘you [\textsc{pl}]’ (\textsc{2pl} non-subject \isi{pronoun}) \item
\textbf{\textit{una}} (\textsc{pro}) ‘we’ (1\textsc{pl.incl} subject \isi{pronoun}) \item
\textbf{\textit{una=}} (\textsc{pro}) ‘us’ (1\textsc{pl.incl} non-subject \isi{pronoun}) \item
\textbf{\textit{unda}} (\textsc{n}) ‘enemy’ \item
\textbf{\textit{unden}} (\textsc{n}) ‘betel nut palm stem (TP \textit{buai limbum})’ \item
\textbf{\textit{unmbu}} (\textsc{n}) ‘buttocks’ \item
\textbf{\textit{upin}} (\textsc{n}) ‘bird species (crowned pigeon) (TP \textit{guria})’ \item
\textbf{\textit{uta-}} (\textsc{v}) ‘grind (coconut)’ \item
\textbf{\textit{utam}} (\textsc{n}) ‘yam’ \item
\textbf{\textit{uten}} (\textsc{n}) ‘cough, phlegm’ \item
\textbf{\textit{uten wïte-}} (\textsc{v}) ‘cough’ \item
\textbf{\textit{uwa}} (\textsc{n}) ‘neck’ (also \textbf{\textit{umwa}}) \item
\textbf{\textit{wal}} (\textsc{n}) ‘penis’ \item
\textbf{\textit{wala}} (\textsc{adj}) ‘far’ \item
\textbf{\textit{wala-}} (\textsc{v}) ‘perceive’ \item
\textbf{\textit{wali-}} (\textsc{v}) ‘hit, shoot; kill’ (also \textbf{\textit{asa-}}) \item
\textbf{\textit{walim}} (\textsc{n}) ‘bird species (dove, pigeon) (TP \textit{balus})’ (< \ili{Yuat}) \item
\textbf{\textit{walïm}} (\textsc{n}) ‘segment (of sugarcane)’ \item
\textbf{\textit{walum}} (\textsc{n}) ‘breast’ \item
\textbf{\textit{walum uma}} (\textsc{n}) ‘ribs’ \item
\textbf{\textit{walwale}} (\textsc{n}) ‘mushroom species’ \item
\textbf{\textit{wanambam}} (\textsc{n}) ‘armpit’ \item
\textbf{\textit{wandam}} (\textsc{n}) ‘garden; jungle, woods, forest, bush’ \item
\textbf{\textit{wandi}} (\textsc{n}) ‘bandicoot (TP \textit{mumut})’ \item
\textbf{\textit{wanïmi}} (\textsc{n}) ‘hair’ \item
\textbf{\textit{wanmbi}} (\textsc{n}) ‘betel pepper (TP \textit{daka})’ \item
\textbf{\textit{wanmbi}} (\textsc{n}) ‘tusk (of a boar)’ \item
\textbf{\textit{wanum}} (\textsc{adj)} ‘short’ \item
\textbf{\textit{wapa}} (\textsc{n}) ‘leaf’ \item
\textbf{\textit{wapalup}} (\textsc{n}) ‘wing’ \item
\textbf{\textit{wapata}} (\textsc{adj}) ‘dry; old’ \item
\textbf{\textit{wat} }(\textsc{p)} ‘above, atop’ \item
\textbf{\textit{watangïn}} (\textsc{n}) ‘little finger, pinky finger’ \item
\textbf{\textit{wawalmalape}} (\textsc{adj}) ‘hard’ \item
\textbf{\textit{way}} (\textsc{n}) ‘turtle’ (< \ili{Ap Ma}) \item
\textbf{\textit{we}} (\textsc{n}) ‘sago starch, sago flour; sago pancake’ \item
\textbf{\textit{wemala}} (\textsc{n}) ‘lizard species (small colorful gecko)’ \item
\textbf{\textit{wembal}} (\textsc{adj}) ‘white’ (cf. \textit{wendum} in \ili{Dimiri}) \item
\textbf{\textit{wenguta}} (\textsc{n}) ‘bow’ \item
\textbf{\textit{wepal}} (\textsc{n}) ‘sago palm (dead, dry sago palm)’ \item
\textbf{\textit{wi}} (\textsc{n}) ‘name’ \item
\textbf{\textit{wïl}} (\textsc{p)} ‘with (\isi{comitative})’ \item
\textbf{\textit{wïlingïn}} (\textsc{n}) ‘taro (wild taro)’ \item
\textbf{\textit{wipam}} (\textsc{n}) ‘arrow’ \item
\textbf{\textit{wïta}} (\textsc{n}) ‘coconut shell’ \item
\textbf{\textit{wïtï}} (\textsc{n}) ‘leg, foot’ \item
\textbf{\textit{wïtï ambatïm}} (\textsc{n}) ‘knee’ (also \textbf{\textit{wïtal mot}}) \item
\textbf{\textit{wïtï lema}} (\textsc{n}) ‘upper leg, thigh’ \item
\textbf{\textit{wïtï mot}} (\textsc{n}) ‘knee’ (also \textbf{\textit{wïtal ambatïm}}) \item
\textbf{\textit{wïtïlwa}} (\textsc{n}) ‘path, road’ \item
\textbf{\textit{wïtum}} (\textsc{n}) ‘foot’ \item
\textbf{\textit{wiwila}} (\textsc{adj}) ‘light (not heavy)’ \item
\textbf{\textit{wiwina-}} (\textsc{v}) ‘fly’ \item
\textbf{\textit{wo}} (\textsc{n}) ‘village’ (possibly < \ili{Yuat}) \item
\textbf{\textit{wo-}} (\textsc{v}) ‘sleep’ \item
\textbf{\textit{=wo}} [\isi{intensifier}/\isi{vocative} \isi{enclitic}, ‘\textsc{voc}’] \item
\textbf{\textit{wokan}} (\textsc{n}) ‘banana flower’ \item
\textbf{\textit{wokan}} (\textsc{n}) ‘wallaby (TP \textit{sikau})’ (likely an \isi{areal term}) \item
\textbf{\textit{wokïn}} (\textsc{adj}) ‘huge’; (\textsc{n}) ‘big man, important person’ \item
\textbf{\textit{wom}} (\textsc{n}) ‘strap (for climbing trees)’ \item
\textbf{\textit{woma}} (\textsc{n}) ‘woven fronds (TP \textit{pangal})’ \item
\textbf{\textit{womba}} (\textsc{n}) ‘tree species’ \item
\textbf{\textit{wombi}} (\textsc{adj}) ‘many’ \item
\textbf{\textit{wombïn}} (\textsc{n}) ‘work, job, task, activity’ \item
\textbf{\textit{wot}} (\textsc{n}) ‘younger sibling’ \item
\textbf{\textit{wotena}} (\textsc{n}) ‘sister (younger sister)’ \item
\textbf{\textit{wowal}} (\textsc{n}) ‘chicken’ (\isi{areal term}) \item
\textbf{\textit{wowal}} (\textsc{n}) ‘fish scale’ \item
\textbf{\textit{wowi}} (\textsc{n}) ‘tree species (ilima tree) (\textit{Octomeles sumatrana}) (TP \textit{erima})’ \item
\textbf{\textit{wupa}} (\textsc{quant}) ‘all’ \item
\textbf{\textit{wusim}} (\textsc{n}) ‘crocodile’ (possibly < \ili{Yuat}) \item
\textbf{\textit{wuta}} (\textsc{n}) ‘bird’ \item
\textbf{\textit{wutal}} (\textsc{n}) ‘worm’ \item
\textbf{\textit{wutïwutï}} (\textsc{n}) ‘bird species (duck)’ \item
\textbf{\textit{ya}} (\textsc{n}) ‘coconut’ \item
\textbf{\textit{ya mïnandïn}} (\textsc{n}) ‘coconut (young, drinking coconut) (TP \textit{kulau})’ \item
\textbf{\textit{yakalapana}} (\textsc{n}) ‘butterfly’ \item
\textbf{\textit{yakeka}} (\textsc{n}) ‘bean’ \item
\textbf{\textit{yaki}} (\textsc{n}) ‘rat’ (< \ili{Ap Ma}) \item
\textbf{\textit{yalïs}} (\textsc{n}) ‘coconut flower sheath’ \item
\textbf{\textit{yamangïla}} (\textsc{n}) ‘spine, backbone’ \item
\textbf{\textit{yambi}} (\textsc{n}) ‘tree species (tree with white sap)’ \item
\textbf{\textit{yambom}} (\textsc{n}) ‘palm of the hand’ \item
\textbf{\textit{yamo}} (\textsc{n}) ‘grandchild’ \item
\textbf{\textit{yamo}} (\textsc{n}) ‘snake species (short, black snake)’ \item
\textbf{\textit{yamun}} (\textsc{n}) ‘heart’ \item
\textbf{\textit{yana}} (\textsc{n}) ‘woman’ (also \textbf{\textit{yena}}) \item
\textbf{\textit{yana kambïn}} (\textsc{n}) ‘girl’ (also \textbf{\textit{yena kambïn}}) \item
\textbf{\textit{yananu}} (\textsc{n}) ‘sister’ (also \textbf{\textit{yenanu}}) \item
\textbf{\textit{yangum}} (\textsc{n}) ‘hand’ (also \textbf{\textit{yum}}) \item
\textbf{\textit{yapalum}} (\textsc{n}) ‘fish species (TP \textit{palang pis})’ \item
\textbf{\textit{yaposa}} (\textsc{n}) ‘coconut frond’ \item
\textbf{\textit{yaputa}} (\textsc{n}) ‘bird species (cockatoo) (TP \textit{koki})’ \item
\textbf{\textit{yata}} (\textsc{n}) ‘man, brother’ (also \textbf{\textit{yeta}}) \item
\textbf{\textit{yata kambïn}} (\textsc{n}) ‘boy’ (also \textbf{\textit{yeta kambïn}}) \item
\textbf{\textit{yatlat}} (\textsc{n}) ‘greens, vegetables’ \item
\textbf{\textit{yena}} (\textsc{n}) ‘woman’ (also \textbf{\textit{yana}}) \item
\textbf{\textit{yena kambïn}} (\textsc{n}) ‘girl’ (also \textbf{\textit{yana kambïn}}) \item
\textbf{\textit{yenanu}} (\textsc{n}) ‘sister’ (also \textbf{\textit{yananu}}) \item
\textbf{\textit{yeta}} (\textsc{n}) ‘man, brother’ (also \textbf{\textit{yata}}) \item
\textbf{\textit{yeta kambïn}} (\textsc{n}) ‘boy’ (also \textbf{\textit{yata kambïn}}) \item
\textbf{\textit{yil}} (\textsc{n}) ‘body hair’ \item
\textbf{\textit{yo-}} (\textsc{v}) ‘cut, carve; go around’ (also \textbf{\textit{lo-}}) \item
\textbf{\textit{yokam}} (\textsc{n}) ‘bamboo’ \item
\textbf{\textit{yokomakan}} (\textsc{n}) ‘bird species (wildfowl)’ (possibly < \ili{Ap Ma}) \item
\textbf{\textit{yum}} (\textsc{n}) ‘hand’ (also \textbf{\textit{yangum}}) \item
\textbf{\textit{yumip}} (\textsc{n}) ‘finger’
\end{enumerate}
