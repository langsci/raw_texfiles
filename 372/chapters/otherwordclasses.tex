\chapter{Other word classes}\label{sec:8}

\is{word class|(}
\is{lexical class|(}
\is{part of speech|(}

In this chapter I discuss the function, structure, and distribution of various word types that do not fit neatly into other groupings. They are all relatively small, closed classes. On both \isi{semantic} and \isi{morphosyntactic} grounds, they are trickier to define than nouns or verbs. After discussing \isi{postposition}s (\sectref{sec:8.1}) and \isi{adverb}s (\sectref{sec:8.2}), I provide an overview of the remaining small classes: \isi{negator}s, \isi{question word}s, and \isi{interjection}s (\sectref{sec:8.3}).

\is{part of speech|)}
\is{lexical class|)}
\is{word class|)}

\section{Postpositions}\label{sec:8.1}

\is{postposition|(}
\is{adposition|(}

The most frequent postpositions in Ulwa are listed in \REF{ex:otherwc:1}.

\ea%1
    \label{ex:otherwc:1}
            Postpositions
\begin{tabbing}
{(\textit{andï} {\textasciitilde} \textit{andïm} {\textasciitilde} \textit{andïn})} \= {(‘for’ (\isi{benefactive}), ‘from’ (\isi{ablative}))}\kill
    \textit{ala}  \>          ‘for’ (\isi{benefactive}), ‘from’ (\isi{ablative})\\
    \textit{andï} {\textasciitilde} \textit{andïm} {\textasciitilde} \textit{andïn} \> ‘for’ (\isi{benefactive}), ‘from’ (\isi{ablative})\\
    \textit{andïla} {\textasciitilde} \textit{angla}   \>   ‘waiting for, awaiting’\\
    \textit{angani}   \>       ‘behind, after’\\
    \textit{imbam}   \>       ‘under, below’\\
    \textit{in}    \>        ‘in, into’\\
    \textit{ipka}    \>      ‘before, in front of’ (spatial or \isi{temporal})\\
    \textit{iya}   \>         ‘to, toward’\\
    \textit{ka}   \>         ‘at, in, on’\\
    \textit{kana} {\textasciitilde} \textit{kanam}  \>    ‘beside, near, next to’\\
    \textit{moni}   \>       ‘between, among’\\
    \textit{nakap} {\textasciitilde} \textit{nap}   \>   ‘on account of, because of, for’\\
    \textit{u}   \>         ‘from, in, at, around, along’\\
    \textit{ul} {\textasciitilde} \textit{lu}   \>       ‘with’ (\isi{comitative})\\
    \textit{wan}   \>       ‘over, above’\\
    \textit{wat}   \>         ‘atop, onto’
\end{tabbing}
\z

Although considered a grammatical category in Ulwa, postpositions may function at times as verbs. Furthermore, there may not be a clear line between postpositions and the \isi{oblique-marker} \isi{enclitic} \textit{=n} ‘\textsc{obl}’, which functions something like a \isi{case marker} (\sectref{sec:11.4}).\footnote{This ambiguity is perhaps unsurprising given the crosslinguistically common diachronic relationship between postpositions and \isi{case}-marking \isi{suffix}es.} 

Sentences \REF{ex:otherwc:2} through \REF{ex:otherwc:21} illustrate various use of these postpositions. When an NP ends in (or consists entirely of) an \isi{object marker}, this \isi{object marker} \isi{clitic}izes to the following \isi{postposition}.

\is{adposition|)}
\is{postposition|)}
\is{postposition|(}
\is{adposition|(}

\ea%2
    \label{ex:otherwc:2}
            \textbf{\textit{Mala}} \textit{ay mankap.}\\
\gll    ma=\textbf{ala}    ay    ma=nïkï-p\\
    3\textsc{sg.obj}=for  sago  3\textsc{sg.obj}=dig{}-\textsc{pfv}\\
\glt `[They] made sago for him.’ [ulwa011\_01:21]
\z

\ea%3
    \label{ex:otherwc:3}
            \textit{Nï} \textbf{\textit{wala}} \textit{wa man.}\\
\gll    nï    u=\textbf{ala}    wa    ma-n\\
    1\textsc{sg}  2\textsc{sg}=from  village  go-\textsc{ipfv}\\
\glt `I’m going from you to the village.’ [ulwa040\_01:56]
\z

\ea%4
    \label{ex:otherwc:4}
            \textbf{\textit{Mandï}} \textit{sakla itap matï mal unda mane.}\\
\gll    ma=\textbf{andï}    sakla    ita-p    ma=tï       ma=lï unda  ma-n-e\\
    3\textsc{sg.obj}=for  platform  build-\textsc{pfv}  3\textsc{sg.obj}=take  3\textsc{sg.obj}=put    go    go-\textsc{ipfv-dep}\\
\glt `[They] were going to build a stretcher for him, put [him] on it, and go.’ [ulwa029\_09:57]
\z

\ea%5
    \label{ex:otherwc:5}
            \textit{Ndïlakan ndï} \textbf{\textit{ndandïla}} \textit{ndïpïn!}\\
\gll    ndï=la-ka-n    ndï  ndï=\textbf{andïla}  ndï=p-n\\
    3\textsc{pl}=\textsc{irr-}let-\textsc{imp}  3\textsc{pl}  3\textsc{pl}=await    3\textsc{pl}=be-\textsc{imp}\\
\glt `Let them be there waiting for them!’ [ulwa032\_50:02]
\z

\ea%6
    \label{ex:otherwc:6}
            \textit{Kuman} \textbf{\textit{ndangla}} \textit{kontena menup.}\\
\gll    kuma=n  ndï=\textbf{angla}  kontena  ma=in-u-p\\
    some=\textsc{obl}  3\textsc{pl}=await  container  3\textsc{sg.obj}=in-put-\textsc{pfv}\\
\glt `[I] put some [bananas] in the container to wait for them.’ (\textit{kontena} = TP) [ulwa014\_17:39]
\z

\is{adposition|)}
\is{postposition|)}
\newpage

\is{postposition|(}
\is{adposition|(}

\ea%7
    \label{ex:otherwc:7}
            \textit{An luke} \textbf{\textit{unangani}} \textit{ata i.}\\
\gll    an      luke  un=\textbf{angani}  ata  i\\
    1\textsc{pl.excl}  too    \textsc{3pl}=behind  up  go.\textsc{pfv}\\
\glt `We, too, came up behind you.’ [ulwa037\_26:33]
\z

\ea%8
    \label{ex:otherwc:8}
            \textit{Namndu wa anmbi apa} \textbf{\textit{imbam}} \textit{iye.}\\
\gll    namndu  wa  an-mbï-i      apa    \textbf{imbam}  i-e\\
    pig      just  out-here-go.\textsc{pfv}  house  under    go.\textsc{pfv-dep}\\
\glt `The pigs have just come out and gone under the houses.’ [ulwa037\_43:23]
\z

\ea%9
    \label{ex:otherwc:9}
            \textit{Sinokoynï} \textbf{\textit{men}} \textit{nïkïna mane.}\\
\gll    sinokoy=nï  ma=\textbf{in}      nïkï-na  ma-n-e\\
    crop=\textsc{obl}    3\textsc{sg.obj}=in  dig{}-\textsc{irr}  go-\textsc{ipfv-dep}\\
\glt `[I] am going to plant crops in it [= the garden].’ [ulwa042\_03:38]
\z

\ea%10
    \label{ex:otherwc:10}
          \textit{U mat ma mat} \textbf{\textit{nipka}} \textit{malïta!}\\
\gll    u    ma=tï      ma  ma=tï      nï=\textbf{ipka} ma=lï-ta\\
    \textsc{2sg}  3\textsc{sg.obj}=take  go  \textsc{3sg.obj}=take  1\textsc{sg}=before    3\textsc{sg.obj}=put-\textsc{cond}\\
\glt `Take her, go, and put her ahead of me!’ [ulwa032\_18:39]
\z

\ea%11
    \label{ex:otherwc:11}
          \textit{Ndï ndïtï ulum} \textbf{\textit{ndiya}} \textit{unde.}\\
\gll    ndï  ndï=tï    ulum  ndï=iya    unda-e\\
    3\textsc{pl}  3\textsc{pl}=take  palm  3\textsc{pl}=toward  go-\textsc{ipfv}\\
\glt `They take them and go to the sago palms.’ [ulwa014\_56:33]
\z

\ea%12
    \label{ex:otherwc:12}
          \textit{Samban} \textbf{\textit{ka}} \textit{ndïwanap.}\\
\gll    samban  \textbf{ka}  ndï=wana-p\\
    pot      at  3\textsc{pl}=cook-\textsc{pfv}\\
\glt `[They] cooked them in the pot.’ [ulwa037\_44:54]
\z

\ea%13
    \label{ex:otherwc:13}
          \textit{Min tane inmi} \textbf{\textit{makanam}} \textit{lïp.}\\
\gll    min  tane  inmi  ma=\textbf{kanam}  lï-p\\
    3\textsc{du}  stand  hole  3\textsc{sg.obj}=near  put-\textsc{pfv}\\
\glt `The two were standing near the hole.’ [ulwa001\_05:26]
\z

\ea%14
    \label{ex:otherwc:14}
          \textit{Nï matane ndïl} \textbf{\textit{ndïmoni}} \textit{lïp.}\\
\gll    nï    ma=tane    ndïl    ndï=\textbf{moni}    lï-p\\
    1\textsc{sg}  3\textsc{sg.obj}=stand  pandanus  3\textsc{pl}=among  put-\textsc{pfv}\\
\glt `I stood it among the pandanus.’ [ulwa001\_15:35]
\z

\is{adposition|)}
\is{postposition|)}
\is{postposition|(}
\is{adposition|(}

\ea%15
    \label{ex:otherwc:15}
          \textit{Itom mï way} \textbf{\textit{manakap}} \textit{tïnanga se.}\\
\gll    itom  mï      way  ma=\textbf{nakap}  tïnanga  sa-e\\
    father  \textsc{3sg.subj}  turtle  3\textsc{sg.obj}=for  arise  cry-\textsc{ipfv}\\
\glt `The father got up and began to cry on account of the turtle.’ [ulwa006\_08:02]
\z

\ea%16
    \label{ex:otherwc:16}
          \textit{Kalam nga ndï} \textbf{\textit{manap}} \textit{anwale.}\\
\gll    kalam    nga      ndï  ma=\textbf{nap}    an=wali-e\\
    knowledge  \textsc{sg.prox}  \textsc{3pl}  \textsc{3sg.obj}=for  1\textsc{pl.excl}=hit-\textsc{ipfv}\\
\glt `This knowledge -- they are killing us on account of it.’ [ulwa014\_28:58]
\z

\ea%17
    \label{ex:otherwc:17}
          \textit{Mï tïlwa} \textbf{\textit{mo}} \textit{mat ine.}\\
\gll    mï      tïlwa  ma=\textbf{u}      ma=tï      i-n-e\\
    3\textsc{sg.subj}  road  3\textsc{sg.obj}=from  3\textsc{sg.obj}=take  come-\textsc{pfv-dep}\\
\glt `She carried her along the road.’ [ulwa032\_17:27]
\z

\ea%18
    \label{ex:otherwc:18}
          \textit{Nï} \textbf{\textit{mol}} \textit{may mawap.}\\
\gll    nï    ma=\textbf{ul}      ma=i        ma=wap\\
    1\textsc{sg}  3\textsc{sg.obj}=with  3\textsc{sg.obj}=go.\textsc{pfv}  3\textsc{sg.obj}=be.\textsc{pst}\\
\glt `I went with him there and stayed there.’ [ulwa027\_00:05]
\z

\ea%19
    \label{ex:otherwc:19}
          \textit{Ndï ipka man ango alum tïngïn} \textbf{\textit{lu}} \textit{inde.}\\
\gll    ndï  ipka  ma=n      ango  alum  tïngïn  \textbf{lu}    inda-e\\
    3\textsc{pl}  before  3\textsc{sg.obj=obl}  \textsc{neg}  child  many  with  walk-\textsc{ipfv}\\
\glt `In the past, they wouldn’t go around with lots of children.’ [ulwa014\_49:27]
\z

\ea%20
    \label{ex:otherwc:20}
          \textit{Nï apïn malamap} \textbf{\textit{mawan}} \textit{utape …}\\
\gll    nï    apïn  ma=la{}-ama-p      ma=\textbf{wan} uta-p-e\\
    1\textsc{sg}  fire    3\textsc{sg.obj}=\textsc{irr}{}-eat-\textsc{pfv}  3\textsc{sg.obj}=above    grind-\textsc{pfv-dep}\\
\glt `When I’ve burned it and cleared over it …’ [ulwa037\_49:51]
\z

\ea%21
    \label{ex:otherwc:21}
          \textit{Ata ma mïka} \textbf{\textit{ndawat}} \textit{namana.}\\
\gll    ata  ma  mïka  anda=\textbf{wat}    na-ma-na\\
    up  go  tree.species  \textsc{sg.dist}=atop  \textsc{detr-}go-\textsc{irr}\\
\glt `[He] will go up, go onto that tree.’ [ulwa029\_04:51]
\z

Postpositions may be followed by the \isi{locative verb} \textit{p-} ‘be at’ (or its \isi{suppletive} \isi{past} form \textit{wap} ‘be.\textsc{pst}’) to encode predicative spatial meaning. When \isi{spatial postposition}s can convey either stationary or \isi{direction}al meaning (e.g., \textit{in} ‘in, into’), generally only the static sense is felt (e.g., \textit{in p-} ‘is in’), as in examples \REF{ex:otherwc:22} through \REF{ex:otherwc:27}.

\ea%22
    \label{ex:otherwc:22}
          \textit{Yawat mï Sinda \textbf{kanam pe} ame.}\\
\gll    Yawat  mï      Sinda  \textbf{kanam}    \textbf{p-e}    ama-e\\
    [name]  3\textsc{sg.subj}  [name]  beside    be-\textsc{dep}  eat-\textsc{ipfv}\\
\glt `The spear is next to the tree.’ [elicited]
\z

\ea%23
    \label{ex:otherwc:23}
          \textit{Mana mï im \textbf{makanam wap}}.\\
\gll mana  mï      im    ma=\textbf{kanam}    \textbf{wap}\\
    spear  3\textsc{sg.subj}  tree  3\textsc{sg.obj}=beside  be.\textsc{pst}\\
\glt `The spear was next to the tree.’ [elicited]
\z

\ea%24
    \label{ex:otherwc:24}
          \textit{Nï manji ya \textbf{ngalaymbam pe}}.\\
\gll nï    ma-nji      ya      ngala=\textbf{imbam}    \textbf{p}{}-e\\
    1\textsc{sg}  3\textsc{sg.obj-poss}  coconut  \textsc{pl.prox}=under  be\textsc{{}-ipfv}\\
\glt `I am under his coconut trees.’ [ulwa013\_11:55]
\z


\ea%25
    \label{ex:otherwc:25}
          \textit{Ngata nda unde \textbf{ndïwat pe}}.\\
\gll ngata  anda    unda-e  ndï=\textbf{wat}  \textbf{p}{}-e\\
    grand  \textsc{sg.dist}  go-\textsc{dep}  \textsc{3pl}=atop  be\textsc{{}-ipfv}\\
\glt `Our ancestor used to go around over them.’ [ulwa014\_69:56]
\z

\ea%26
    \label{ex:otherwc:26}
          \textit{Ngïm \textbf{nden pe} wa layte iye.}\\
\gll    ngïm  anda=\textbf{in}  \textbf{p}{}-e    wa    ala=ita-e        i-e\\
    cloud  \textsc{sg.dist=}in  be\textsc{{}-dep} village  \textsc{pl.dist}=build-\textsc{dep}  go.\textsc{pfv-dep}\\
\glt `Living in that cloud, [he] was building village after village.’ [ulwa009\_00:06]
\z

\ea%27
    \label{ex:otherwc:27}
          \textit{Inom ndï umbe nungol \textbf{ndul pïna}}.\\
\gll inom  ndï  umbe    nungol  ndï=\textbf{ul}    \textbf{p-na}\\
    mother  3\textsc{pl}  tomorrow  child  3\textsc{pl}=with  be-\textsc{irr}\\
\glt `The mothers will be with the children tomorrow.’ [elicited]
\z

These \isi{verb phrase}s \isi{head}ed by \isi{locative verb}s and containing \isi{postpositional phrase}s can further take the \isi{nominalizing} \isi{suffix} \textit{-en} ‘\textsc{nmlz}’ (\sectref{sec:3.2}), as in \REF{ex:otherwc:28} and \REF{ex:otherwc:29}.

\ea%28
    \label{ex:otherwc:28}
          \textit{ngunan ato inkaw \textbf{ngawat pen} ngala}\\
\gll    ngunan=n      ata-u    inkaw    nga=\textbf{wat}    \textbf{p-en} ngala\\
    1\textsc{du.incl=obl}  up-from  mountain  \textsc{sg.prox}=atop  be\textsc{{}-nmlz}    \textsc{pl.prox}\\
\glt `these [people] who live atop the mountains above us’ [ulwa014\_59:55]
\z

\ea%29
    \label{ex:otherwc:29}
          \textit{sïtik mï kïka tïlwa \textbf{men pen}}\\
\gll    sïtik  mï      kïka    tïlwa  ma=\textbf{in}      \textbf{p-en}\\
    stick  \textsc{3sg.subj}  white.ant  road  3\textsc{sg.obj}=in  be\textsc{{}-nmlz}\\
\glt `the stick that is in the white ant track’ (\textit{sïtik} = TP \textit{stik}) [ulwa029\_05:18]
\z

Like verbs (and unlike nominal elements, \isi{adjective}s, and so on), postpositions do not permit the \isi{oblique marker} \textit{=n} ‘\textsc{obl}’ (\sectref{sec:11.4.1}).

  Some postpositions seem to function like verbs even without being followed by a \isi{locative verb} form. When such forms occur clause-finally and express the action or event of the \isi{predicate}, they may be considered to be verbs (albeit somewhat \isi{defective} ones). Indeed, they are likely verbs in origin, having begun a process of \isi{grammaticalization}, losing their \isi{verbal morphology} as they come to function more as postpositions. This seems to be the case with the \isi{suppletive} verb \textit{ala} {\textasciitilde} \textit{andï(m/n)} ‘see’, both \isi{stem}s of which are used as postpositions with the meaning ‘for’ or ‘from’. However, as \isi{separable verb}s (\sectref{sec:9.2.1}), these forms -- along with the \isi{nominal adjunct} \textit{lïmndï} ‘eye’ -- convey the verbal meaning ‘see’, as in \REF{ex:otherwc:30}.

\ea%30
    \label{ex:otherwc:30}
          \textit{Unan amun lïmndï makape i} \textbf{\textit{mandïm}}.\\
\gll unan    amun  lïmndï  maka=p-e    i    ma=\textbf{andï-m}\\
    1\textsc{pl.incl}  now  eye    thus=\textsc{cop-dep}  way  3\textsc{sg.obj}=see-\textsc{pfv}\\
\glt `We have now seen this kind of behavior.’ [ulwa037\_64:29]
\z

Indeed, even when functioning as a verb, \textit{ala} {\textasciitilde} \textit{andï(m/n)} ‘see’ generally does not take any \isi{TAM} \isi{suffix}ation \REF{ex:otherwc:31}. It can, however, receive a \isi{dependent marker} \REF{ex:otherwc:32}.

\ea%31
    \label{ex:otherwc:31}
          \textit{Nïnji itom mï lïmndï} \textbf{\textit{nala}}.\\
\gll nï-nji    itom  mï      lïmndï  nï=\textbf{ala}\\
    1\textsc{sg-poss}  father  \textsc{3sg.subj}  eye    1\textsc{sg}=see\\
\glt `My father saw me.’ [ulwa013\_01:10]
\z

\ea%32
    \label{ex:otherwc:32}
          \textit{Ndï wa i lïmndï wa} \textbf{\textit{male}}.\\
\gll ndï  wa    i    lïmndï  wa    ma=\textbf{ala-e}\\
    3\textsc{pl}  village  go.\textsc{pfv}  eye    village  \textsc{3sg.obj}=see-\textsc{dep}\\
\glt `They went home and saw the village.’ [ulwa001\_17:53]
\z

The \isi{stem} \textit{andï-} ‘see’ does, however, permit at least one \isi{TAM} \isi{suffix}, the \isi{irrealis} \isi{suffix} \textit{-na} ‘\textsc{irr}’, as in \REF{ex:otherwc:33}.

\ea%33
    \label{ex:otherwc:33}
          \textit{Ankam moweka ango lïmndï} \textbf{\textit{mandïna}}.\\
\gll ankam  moweka  ango  lïmndï  ma=\textbf{andï-na}\\
    person  also    \textsc{neg}  eye    \textsc{3sg.obj}=see-\textsc{irr}\\
\glt `Nor would people see it.’ [ulwa014\_42:27]
\z

Postpositions may also be used as elements in \isi{compound verb}s. See \sectref{sec:4.14}, however, for problems surrounding this issue.

\is{adposition|)}
\is{postposition|)}

\section{Adverbs}\label{sec:8.2}

\is{adverb|(}

Adverbs often provide additional information on the manner in which an action occurs or situate an event in \isi{time} or space. They are never required by the argument structure of a verb. In terms of distribution, adverbs can be defined by their unique ability to precede subjects. Although the canonical placement of adverbs is following subjects and preceding objects (that is, in the position of \isi{oblique}s, i.e., SXOV, \sectref{sec:11.4}), it is possible for adverbs to come first in a given clause. In terms of structure, adverbs may be defined by their inability to take \isi{TAM} \isi{suffix}es (as verbs do) and their inability to take \isi{oblique} marking (as nouns do). The following subsections describe the major subclasses of adverbs.

\is{adverb|)}

\subsection{Temporal adverbs}\label{sec:8.2.1}

\is{temporal adverb|(}
\is{adverb|(}

The most frequent \isi{temporal adverb}s are given in \REF{ex:otherwc:34}.

\ea%34
    \label{ex:otherwc:34}
          Temporal adverbs\\
\begin{tabbing}
{(\textit{anganika})} \= {(‘now, today, nowadays, recently, still’)}\kill
{\textit{amun}} \> {‘now, today, nowadays, recently, still’}\\
{\textit{awal}} \> {‘yesterday’}\\
{\textit{umbe}} \> {‘tomorrow’}\\
{\textit{ta}} \> {‘already’}\\
{\textit{ipka}} \> {‘before, beforehand, earlier, first’}\\
{\textit{anganika}} \> {‘after, afterwards, later, soon’}
\end{tabbing}
\z

Sentences \REF{ex:otherwc:35} through \REF{ex:otherwc:41} illustrate some uses of these \isi{temporal adverb}s.

\ea%35
    \label{ex:otherwc:35}
          \textit{Una} \textbf{\textit{amun}} \textit{mbi.}\\
\gll    unan    \textbf{amun}  mbï-i\\
    1\textsc{pl.incl}  now  here-go.\textsc{pfv}\\
\glt `We’ve now come here.’ [ulwa037\_21:19]
\z

\ea%36
    \label{ex:otherwc:36}
          \textbf{\textit{Amun}} \textit{una kalam.}\\
\gll    \textbf{amun}  unan    kalam\\
    now  1\textsc{pl.incl}  knowledge\\
\glt `Now we know.’ [ulwa014\_57:35]
\z

\ea%37
    \label{ex:otherwc:37}
          \textit{Nï} \textbf{\textit{amun}} \textit{anmbi wema weyunda.}\\
\gll    nï    \textbf{amun}  an-mbï-i      wema  we-u-nda\\
    1\textsc{sg}  now  out-here-go.\textsc{pfv}  pangal  cut-put-\textsc{irr}\\
\glt `I came out recently to cut \textit{pangal} [= palm fronds].’ [ulwa038\_04:55]
\z

\ea%38
    \label{ex:otherwc:38}
          \textit{U} \textbf{\textit{awal}} \textit{mawap.}\\
\gll    u    \textbf{awal}    ma=wap\\
    2\textsc{sg}  yesterday  3\textsc{sg.obj}=be.\textsc{pst}\\
\glt `You were there yesterday.’ [ulwa040\_00:04]
\z

\ea%39
    \label{ex:otherwc:39}
          \textbf{\textit{Awal}} \textit{anambi keka we ulwap.}\\
\gll    \textbf{awal}    an-ambi    keka      we    ulwa=p\\
    yesterday  1\textsc{pl.excl-top}  completely  sago  nothing=\textsc{cop}\\
\glt `As for us, we were completely out of sago yesterday.’ [ulwa037\_60:18]
\z

\ea%40
    \label{ex:otherwc:40}
          \textit{Una} \textbf{\textit{umbe}} \textit{wolka ina.}\\
\gll    unan    \textbf{umbe}    wolka  i-na\\
    1\textsc{pl.incl}  tomorrow  again  come-\textsc{irr}\\
\glt `We’ll come again tomorrow.’ [ulwa030\_04:12]
\z

\ea%41
    \label{ex:otherwc:41}
          \textbf{\textit{Umbe}} \textit{una angos wombïn ninda?}\\
\gll    \textbf{umbe}    unan    angos  wombïn=n  ni-nda\\
    tomorrow  1\textsc{pl.incl}  what  work=\textsc{obl}  act-\textsc{irr}\\
\glt `What [sort of] work will we do tomorrow?’ [ulwa030\_01:43]
\z

Although the three basic \isi{temporal adverb}s (\textit{amun} ‘now’, \textit{awal} ‘yesterday’, and \textit{umbe} ‘tomorrow’) generally occur immediately after the subject (when it is expressed), they may alternatively occur before the subject (i.e., clause-initially), as in \REF{ex:otherwc:36}, \REF{ex:otherwc:39}, and \REF{ex:otherwc:41}. There is a tendency to place the \isi{temporal adverb} before \isi{postpositional phrase}s, as in \REF{ex:otherwc:42} and \REF{ex:otherwc:43}.

\ea%42
    \label{ex:otherwc:42}
          \textit{Nï} \textbf{\textit{umbe}} \textit{mol mana.}\\
\gll    nï    \textbf{umbe}    ma=ul      ma-na\\
    1\textsc{sg}  tomorrow  3\textsc{sg.obj}=with  go-\textsc{irr}\\
\glt `I would go with her tomorrow.’ [ulwa040\_02:21]
\z

\ea%43
    \label{ex:otherwc:43}
          \textit{Nï} \textbf{\textit{amun}} \textit{wiya may wap.}\\
\gll    nï    \textbf{amun}  u=iya      ma=i        wap\\
    1\textsc{sg}  now  2\textsc{sg}=toward  3\textsc{sg.obj}=go.\textsc{pfv}  be.\textsc{pst}\\
\glt `I went there to you today.’ [ulwa014\_53:04]
\z

Similarly, \isi{temporal adverb}s tend to precede \isi{oblique}-marked NPs, as in \REF{ex:otherwc:44}.

\ea%44
    \label{ex:otherwc:44}
          \textit{Nï} \textbf{\textit{amun}} \textit{man ndït.}\\
\gll    nï    \textbf{amun}  ma=n      ndï=ta\\
    1\textsc{sg}  now  3\textsc{sg.obj=obl}  3\textsc{pl}=say\\
\glt    ‘I just recently told them.’ [ulwa014\_04:36]
\z

When \isi{temporal adverb}s occur with other adverbs, however, the order seems rather flexible. In example \REF{ex:otherwc:40}, the \isi{adverb} \textit{wolka} ‘again’ follows the \isi{temporal adverb} \textit{umbe} ‘tomorrow’. It is possible, however, for this \isi{adverb} to precede the \isi{temporal adverb} as well. The variability in ordering of adverbs may be seen in \REF{ex:otherwc:45} and \REF{ex:otherwc:46}.

\is{adverb|)}
\is{temporal adverb|)}

\is{temporal adverb|(}
\is{adverb|(}

\ea%45
    \label{ex:otherwc:45}
          \textit{Ngan ango} \textbf{\textit{amun}} \textbf{\textit{wolka}} \textit{maye.}\\
\gll    ngan    ango  \textbf{amun}  \textbf{wolka}  ma=i-e\\
    1\textsc{du.excl}  \textsc{neg}  now  again  3\textsc{sg.obj}=go.\textsc{pfv-dep}\\
\glt `We have not gone there again lately.’ [ulwa027\_00:16]
\z

\ea%46
    \label{ex:otherwc:46}
          \textit{Nï ango} \textbf{\textit{wolka}} \textbf{\textit{amun}} \textit{may.}\\
\gll    nï    ango  \textbf{wolka}  \textbf{amun}  ma=i\\
    1\textsc{sg}  \textsc{neg}  again  now  3\textsc{sg.obj}=go.\textsc{pfv}\\
\glt `I have not gone there again lately.’ [ulwa027\_00:11]
\z

As in example \REF{ex:otherwc:39}, there may also be a preference among some speakers to place the \isi{temporal adverb} before the subject in clauses containing multiple \isi{oblique} expressions, such as adverbs. In example \REF{ex:otherwc:47}, the sentence contains the adverbs \textit{amun} ‘now’ and \textit{wolka} ‘again’, the former occurring clause-initially.

\ea%47
    \label{ex:otherwc:47}
          \textbf{\textit{Amun}} \textit{yalum ngala wolka mbulop.}\\
\gll    \textbf{amun}  yalum    ngala    wolka  mbï-u-lo-p\\
    now  grandchild  \textsc{pl.prox}  again  here-from-go-\textsc{pfv}\\
\glt `Now these grandsons came around here again.’ [ulwa014†]
\z

Similarly, \isi{modal adverb}s such as \textit{wa} ‘just’ may either follow \REF{ex:otherwc:48} or precede \REF{ex:otherwc:49} \isi{temporal adverb}s.

\ea%48
    \label{ex:otherwc:48}
        \textit{Ndï \textbf{amun wa} ndale.}\\
\gll    ndï  \textbf{amun}  \textbf{wa}  ndï=ale-e\\
    3\textsc{pl}  now  just  \textsc{3pl}=scrape-\textsc{ipfv}\\
\glt `Nowadays they just scrape them.’ [ulwa014\_59:36]
\z

\ea%49
    \label{ex:otherwc:49}
          \textit{Ndï \textbf{wa amun} kuli atap.}\\
\gll    ndï \textbf{wa}  \textbf{amun}  kuli  ata  [na]-p\\
    3\textsc{pl}  just  now  throw  up  [\textsc{detr]-}be\\
\glt `Now they are just coming up well.’ [ulwa037\_42:51]
\z

Although one of the defining characteristics of the class of adverbs is that its members do not permit any nominal or \isi{verbal morphology}, this claim is confounded by the fact that words such as \textit{amun} ‘today’, \textit{awal} ‘yesterday’, and \textit{umbe} ‘tomorrow’ may also function as nouns, as illustrated in \REF{ex:otherwc:50} and \REF{ex:otherwc:51}, where the \isi{temporal} words appear to be \isi{head}ing subject NPs.

\ea%50
    \label{ex:otherwc:50}
          \textit{Ay ngam} \textbf{\textit{amun}} \textit{Prayde.}\\
\gll    ay  nga-nam      \textbf{amun}  Prayde\\
    ay  \textsc{sg.prox{}-emph} now  Friday\\
\glt `Ay, that’s it, today is Friday.’ (\textit{Prayde} = TP \textit{Fraide}) [ulwa014\_27:20]
\z

\ea%51
    \label{ex:otherwc:51}
          \textbf{\textit{Umbe}} \textit{anmbi angos mundu mï anmapïta u malanda?}\\
\gll    \textbf{umbe}    an-mbï-i      angos  mundu  mï anma=p-ta      u    ma=la-nda\\
    tomorrow  out-here-go.\textsc{pfv}  what  food  \textsc{3sg.subj}    good=\textsc{cop}{}-\textsc{cond}  \textsc{2sg}  3\textsc{sg.obj}=eat-\textsc{irr}\\
\glt `When tomorrow comes what food will be good for you to eat?’ [ulwa014\_64:31]
\z

These nominal forms may receive the \isi{copular enclitic}. When occurring with the word \textit{amun} ‘today’, this marker can give the sense of ‘still’ (or, in \isi{negative} \isi{polarity}, ‘yet’), as illustrated by examples \REF{ex:otherwc:52} through \REF{ex:otherwc:56}.

\ea%52
    \label{ex:otherwc:52}
          \textit{Unji nungol ngala} \textbf{\textit{amunpe}} \textit{kalam ngol mane.}\\
\gll    u-nji    nungol  ngala    \textbf{amun=p}{}-e    kalam nga=ul      ma-n-e\\
    2\textsc{sg-poss}  child  \textsc{pl.prox}  now=\textsc{cop-dep}  knowledge  \textsc{sg.prox=}with  go-\textsc{ipfv-dep}\\
\glt `Your children are still in school.’ (Literally ‘going with this knowledge’) [ulwa014\_09:31]
\z

\ea%53
    \label{ex:otherwc:53}
          \textit{Olsem nï} \textbf{\textit{amunpe}} \textit{njukutape …}\\
\gll    olsem  nï    \textbf{amun=p}-e    njukuta=p-e\\
    thus  \textsc{1sg}  now=\textsc{cop-dep}  small=\textsc{cop-dep}\\
\glt `Like, when I was still small …’ (\textit{olsem} = TP) [ulwa029\_00:01]
\z

\ea%54
    \label{ex:otherwc:54}
          \textit{Wowal \textbf{amunpïta} ata pïta una ko nol!}\\
\gll    wowal    \textbf{amun=p-}ta    ata  p-ta    unan    ko na-lo\\
    chicken  now=\textsc{cop-cond}  up  be\textsc{{}-cond}  \textsc{1pl.incl} just    \textsc{detr-}go\\
\glt `When the chickens are still up [in the trees], let’s just go!’ [ulwa031\_03:31]
\z

\ea%55
    \label{ex:otherwc:55}
          \textit{Ango} \textbf{\textit{amunpe}} \textit{atay matïna.}\\
\gll    ango  \textbf{amun=p}{}-e    ata  i    ma=tï-na\\
    \textsc{neg}  now=\textsc{cop-dep}  up  go.\textsc{pfv}  3\textsc{sg.obj}=take-\textsc{irr}\\
\glt `[It] wouldn’t go up and get him immediately.’ [ulwa006\_02:32]
\z

\ea%56
    \label{ex:otherwc:56}
          \textit{U} \textbf{\textit{amunpe}} \textit{wol ulwap.}\\
\gll    u    \textbf{amun=p}{}-e    wol  ulwa=p\\
    2\textsc{sg}  now=\textsc{cop-dep}  breast  nothing=\textsc{cop}\\
\glt `You don’t have breasts yet.’ [ulwa011\_01:12]
\z

When \textit{awal} ‘yesterday’ is followed by [p], however, it generally has the sense of ‘afternoon’, as in \REF{ex:otherwc:57} and \REF{ex:otherwc:58}. This, rather, seems to be an instantiation of the \isi{locative verb} \textit{p-} ‘be at’, here being \isi{metaphor}ically extended to \isi{temporal} meaning. Something similar seems to occurs with the noun \textit{imba} ‘night’, in phrases such as \textit{imba pe} ‘at night’.\footnote{There are no attested uses of \textit{umbe} ‘tomorrow’ with \isi{copula}r or \isi{locative} marking.}

\ea%57
    \label{ex:otherwc:57}
          \textit{\textbf{Awal pe} inim ndïn apïn up ay ndïnkap.}\\
\gll    \textbf{awal}    \textbf{p-e}    inim  ndï=n    apïn  u-p      ay ndï=nïkï-p\\
    afternoon  be\textsc{{}-dep} water  3\textsc{pl=obl}  fire    put-\textsc{pfv}  sago    3\textsc{pl}=dig{}-\textsc{pfv}\\
\glt `In the afternoon, [we] put water on the fire and made sago.’ [ulwa031\_03:17]
\z

\ea%58
    \label{ex:otherwc:58}
          \textit{Mundu anglaluta mawap \textbf{awal pïta}}.\\
\gll mundu  angla-lo-ta      ma=wap      \textbf{awal}    \textbf{p-}ta\\
    food  await-go-\textsc{cond}  \textsc{3sg.obj}=be.\textsc{pst}  afternoon  be\textsc{{}-cond}\\
\glt `If [they] were hunting for food, [they] would stay until afternoon.’ [ulwa029\_07:20]
\z

  The other \isi{temporal adverb}s, which never take either nominal or \isi{verbal morphology}, are perhaps better exemplars of adverbs. Like the three adverbs already described, they may appear either before \REF{ex:otherwc:59} or after \REF{ex:otherwc:60} subject NPs.

  \ea%59
    \label{ex:otherwc:59}
          \textbf{\textit{Ipka}} \textit{ankam ango ulum alepen.}\\
\gll    \textbf{ipka}  ankam  ango  ulum   ale-p-en\\
    before  person  \textsc{neg}  palm  scrape-\textsc{pfv-nml}\\
\glt `Before, people didn’t use to scrape sago palms.’ [ulwa008\_00:25]
\z

\ea%60
    \label{ex:otherwc:60}
          \textit{Nï} \textbf{\textit{ipka}} \textit{alan malan upe.}\\
\gll nï    \textbf{ipka}  ala=n      malan    u-p-e\\
    1\textsc{sg}  before  \textsc{pl.dist=obl}  hot.water  put-\textsc{pfv-dep}\\
\glt `I boiled those first.’ [ulwa032\_24:31]
\z

\is{adverb|)}
\is{temporal adverb|)}

\is{temporal adverb|(}
\is{adverb|(}

  Whereas \textit{ta} ‘already’ is clearly monomorphemic, \textit{ipka} ‘before’ and \textit{anganika} ‘after’ are each apparently derived from multiple morphemes: the former consisting of the noun \textit{ip} ‘nose’ and the \isi{postposition} \textit{ka} ‘at, in, on’, the latter consisting of the noun \textit{angani} ‘rear’ and the \isi{postposition} \textit{ka} ‘at, in, on’. While \textit{ipka} ‘before’ is derived from a crosslinguistically common body-part \isi{metaphor}, \mbox{\textit{anganika}} ‘after’ (often shortened to [naka]) is not necessarily so, since \textit{angani} ‘rear’ is not typically used to refer to any part of the human body (cf. \textit{mutam} ‘back’ and \textit{unmbï} ‘buttocks’). Sentences \REF{ex:otherwc:61} through \REF{ex:otherwc:64} illustrate the adverbial use of \textit{ipka} ‘before’ and \textit{anganika} ‘after’.

\ea%61
    \label{ex:otherwc:61}
          \textit{Nïnji inom mï} \textbf{\textit{ipka}} \textit{apa mo li.}\\
\gll    nï-nji    inom  mï      \textbf{ipka}  apa    ma=u li-i\\
    1\textsc{sg-poss}  mother  \textsc{3sg.subj}  before  house  3\textsc{sg.obj}=from    down-go.\textsc{pfv}\\
\glt `My mother went down around the house first.’ [ulwa004\_03:10]
\z

\ea%62
    \label{ex:otherwc:62}
          \textit{Nï} \textbf{\textit{anganika}} \textit{ma wanam mana.}\\
\gll    nï    \textbf{anganika}  ma      wanam  ma-na\\
    1\textsc{sg}  after    3\textsc{sg.obj}  side  go-\textsc{irr}\\
\glt `I will go alongside her later.’ (Literally ‘go to her side’) [ulwa032\_18:41]
\z

\ea%63
    \label{ex:otherwc:63}
          \textit{Yaka} \textbf{\textit{anganika}} \textit{li.}\\
\gll    Yaka  \textbf{anganika}  li-i\\
    [name]  after    down-go.\textsc{pfv}\\
\glt `Yaka came down after.’ [ulwa004\_03:12]
\z

\ea%64
    \label{ex:otherwc:64}
          \textit{U} \textbf{\textit{anganika}} \textit{ndïtana!}\\
\gll    u    \textbf{anganika}  ndï=ta-na\\
    2\textsc{sg}  after    3\textsc{pl}=say-\textsc{irr}\\
\glt `Tell them later!’ [ulwa014†]
\z

Whereas \textit{anganika} ‘after’ is viewed here as a single \isi{adverb} (that is, not composed of [angani] and [ka], at least not synchronically) and thus should not accept any \isi{morphological} \isi{inflection}, the \isi{postposition} \textit{angani} ‘behind’ (as a \isi{postposition}) can indeed have an \isi{object-marker} \isi{clitic}, as in \REF{ex:otherwc:65} and \REF{ex:otherwc:66}.

\ea%65
    \label{ex:otherwc:65}
          \textit{Anambi itom} \textbf{\textit{alangani}} \textit{i.}\\
\gll    an-ambi    itom  \textbf{ala=angani}    i\\
    1\textsc{pl.excl-top}  father  \textsc{pl.dist}=behind  go.\textsc{pfv}\\
\glt `As for us, we came after [our] fathers.’ [ulwa037\_15:43]
\z

\ea%66
    \label{ex:otherwc:66}
          \textit{Nïnji aweta nda} \textbf{\textit{nangani}} \textit{wonp!}\\
\gll    nï-nji    aweta  anda    \textbf{nï=angani}    won-p\\
    \textsc{1sg-poss}  friend  \textsc{sg.dist}  1\textsc{sg}=behind  cut-\textsc{pfv}\\
\glt `That friend of mine has gone behind my back!’ (Literally ‘That friend of mine has cut behind me.’) [ulwa020\_01:49]
\z

More troubling for this analysis of \textit{ipka} ‘before’ and \textit{anganika} ‘after’ as adverbs, however, is the occasional use of \textit{ipka} ‘before’ as a \isi{postposition} as well, as seen in \REF{ex:otherwc:67} and \REF{ex:otherwc:68}.\footnote{It could be, however, that in such instances the postpositional force of \textit{ka} ‘at, in, on’ is still felt, creating a \isi{postpositional phrase} that means something along the lines of ‘at the nose of’. An additional complication is the clear relationship between the adverbs \textit{ipka} ‘before’ and \textit{anganika} ‘after’ and the verbs \textit{ip ka-} ‘precede’ and \textit{angani ka-} ‘follow’, respectively (\sectref{sec:9.2.3}). These putative verbs are found in \is{ordinal numeral} ordinal constructions with \isi{nominalizing} \isi{morphology} (\sectref{sec:7.5}).}

\ea%67
    \label{ex:otherwc:67}
          \textit{E an tïn alol} \textbf{\textit{unipka}} \textit{mbiye!}\\
\gll    e  an      tïn    ala=ul      \textbf{un=ipka}    mbï-i-e\\
    hey  \textsc{1pl.excl}  dog  \textsc{pl.dist}=with  \textsc{2pl}=before  here-go.\textsc{pfv-dep}\\
\glt `Hey, we came here with those dogs before you!’ [ulwa031\_03:47]
\z

\ea%68
    \label{ex:otherwc:68}
          \textit{Ngan} \textbf{\textit{ndipka}} \textit{iyen.}\\
\gll    ngan    \textbf{ndï=ipka}    i-en\\
    1\textsc{du.excl}  3\textsc{pl}=before  go.\textsc{pfv-nmlz}\\
\glt `We two went ahead of them.’ [ulwa032\_29:02]
\z

  Thus, perhaps \textit{ta} ‘already’, which permits no verbal \isi{TAM} \isi{suffix}ation, \isi{nominalize}d forms, or \isi{object-marker} \isi{clitic}s, and which is able to occur either before or after the subject, is the most prototypical \isi{temporal adverb} in Ulwa. The use of \textit{ta} ‘already’ is illustrated in \REF{ex:otherwc:69}, \REF{ex:otherwc:70}, and \REF{ex:otherwc:71}.

\is{adverb|)}
\is{temporal adverb|)}

\ea%69
    \label{ex:otherwc:69}
          \textit{U} \textbf{\textit{ta}} \textit{kalampe.}\\
\gll    u    \textbf{ta}    kalam=p-e\\
    2\textsc{sg}  already  knowledge=\textsc{cop-dep}\\
\glt `You already know.’ [ulwa014†]
\z

\ea%70
    \label{ex:otherwc:70}
          \textit{E mï} \textbf{\textit{ta}} \textit{keka wapatap.}\\
\gll    e  mï      \textbf{ta}    keka      wapata=p\\
    hey  3\textsc{sg.subj}  already  completely  dry=\textsc{cop}\\
\glt `Hey! It’s already completely dry.’ [ulwa038\_04:59]
\z

\ea%71
    \label{ex:otherwc:71}
          \textbf{\textit{Ta}} \textit{unji anapa ndï u inim nïkape.}\\
\gll    \textbf{ta}    u-nji    anapa  ndï  u    inim  nïkï-p-e\\
    already  2\textsc{sg-poss}  sister  \textsc{3pl}  2\textsc{sg}  water  dig{}-\textsc{pfv-dep}\\
\glt `Already, your sisters have celebrated you.’ (Literally ‘have cut your water’) [ulwa014\_62:29]
\z

\subsection{Locative adverbs}\label{sec:8.2.2}

\is{locative adverb|(}
\is{adverb|(}

There is a small set of \isi{locative adverb}s in Ulwa, which are used to indicate position or \isi{direction} \REF{ex:otherwc:72}.

\ea%72
    \label{ex:otherwc:72}
          Locative adverbs
\begin{tabbing}
{\textit{(ngaya)}}  \=  {(‘far’)}\kill
{\textit{ata}}    \>  {‘up, upward, upstream’}\\
{\textit{li}}  \>    {‘down, downward, downstream’}\\
{\textit{mbï}}  \>  {‘here, to here, hither’}\\
{\textit{mbu}}  \>  {‘here, from here, hence’}\\
{\textit{anda}}  \>  {‘there, to there, thither’}\\
{\textit{ando}} \>   {‘there, from there, thence’}\\
{\textit{nu}}  \>    {‘near’}\\
{\textit{ngaya}}  \>  {‘far’}\\
{\textit{wala}} \>   {‘far, far-off’}
\end{tabbing}
\z

The adverbs \textit{ata} ‘up’ and \textit{li} ‘down’ may refer either to literal vertical-axis \isi{location}s and \isi{direction}s or to relative \isi{location}s and \isi{direction}s along the river -- that is, ‘upstream’ and ‘downstream’, respectively. In \REF{ex:otherwc:73} and \REF{ex:otherwc:74}, \textit{ata} ‘up’ and \textit{li} ‘down’ are being used to refer to vertical-axis \isi{direction}s.

\ea%73
    \label{ex:otherwc:73}
          \textit{Ulum maya} \textbf{\textit{ata}} \textit{i.}\\
\gll    ulum  ma=iya      \textbf{ata}  i\\
    palm  3\textsc{sg.obj}=toward  up  go.\textsc{pfv}\\
\glt `[It] went up the sago palm.’ [ulwa006\_03:12]
\z

\ea%74
    \label{ex:otherwc:74}
          \textit{Nungolke ngala kuli} \textbf{\textit{li}} \textit{malp.}\\
\gll    nungolke  ngala    kuli  \textbf{li}    ma=lï{}-p\\
    child    \textsc{pl.prox}  throw  down  3\textsc{sg.obj}=put-\textsc{pfv}\\
\glt `These children have thrown [themselves] down there [= the water].’ [ulwa038\_01:24]
\z

In \REF{ex:otherwc:75} and \REF{ex:otherwc:76}, \textit{ata} ‘up’ refers to \isi{location}s or \isi{direction}s upstream from a point of reference. In \REF{ex:otherwc:77}, \textit{li} ‘down’ refers to a \isi{location} or \isi{direction} downstream from the point of reference.

\ea%75
    \label{ex:otherwc:75}
          \textit{Wot ngo} \textbf{\textit{ata}} \textit{mane.}\\
\gll    wot    nga=u        \textbf{ata}  ma-n-e\\
    younger  \textsc{sg.prox}=from  up  go-\textsc{ipfv-dep}\\
\glt `[They] were going upstream from this younger [village].’ [ulwa002\_05:04]
\z

\ea%76
    \label{ex:otherwc:76}
          \textit{Nï mat} \textbf{\textit{ata}} \textit{ndo i.}\\
\gll    nï    ma=tï      \textbf{ata}  anda=u    i\\
    1\textsc{sg}  3\textsc{sg.obj}=take  up  \textsc{sg.dist=}from  go.\textsc{pfv}\\
\glt `I brought it from up there [i.e., from upstream].’ [ulwa037\_34:26]
\z

\ea%77
    \label{ex:otherwc:77}
          \textit{Ndïmepe ndït} \textbf{\textit{li}} \textit{may.}\\
\gll    ndï=me-p-e    ndï=tï    \textbf{li}    ma=i\\
    3\textsc{pl}=sew-\textsc{pfv-dep}  \textsc{3pl}=take  down  3\textsc{sg.obj}=go.\textsc{pfv}\\
\glt `He sewed them and brought them down there [i.e., downstream].’ [ulwa037\_35:48]
\z

Example \REF{ex:otherwc:78} contains both \textit{li} ‘down’ and \textit{ata} ‘up’. Here, \textit{li} ‘down’ occurs within a \isi{relative clause} meaning something like ‘these who stay there downstream’.

\ea%78
    \label{ex:otherwc:78}
          \textbf{\textit{Li}} \textit{mape ngala ngalaya} \textbf{\textit{ata}} \textit{mbi.}\\
\gll    \textbf{li}    ma=p-e      ngala    ngala=iya      \textbf{ata} mbï-i\\
    down  3\textsc{sg.obj}=be\textsc{{}-dep}  \textsc{pl.prox}  \textsc{pl.prox=}toward  up    here-go.\textsc{pfv}\\
\glt `These people from downstream came upstream here to these people.’ [ulwa032\_43:44]
\z

Furthermore, upward \isi{motion} and downward \isi{motion} are often \isi{synonymous} in Ulwa with entering and exiting houses, respectively. Since houses are elevated on stilts, one must physically move along the vertical axis in order to enter or exit one. In \REF{ex:otherwc:79}, \textit{ata} ‘up’ is being used to refer to entering a house. In \REF{ex:otherwc:80} and \REF{ex:otherwc:81}, \textit{li} ‘down’ connotes exiting a house.

\ea%79
    \label{ex:otherwc:79}
          \textit{Mat i} \textbf{\textit{ata}} \textit{apa may.}\\
\gll    ma=tï      i    \textbf{ata}  apa    ma=i\\
    3\textsc{sg.obj}=take  go.\textsc{pfv}  up  house  3\textsc{sg.obj}=go.\textsc{pfv}\\
\glt `[It] brought him up to the house and went with him.’ [ulwa006\_03:51]
\z

\ea%80
    \label{ex:otherwc:80}
          \textit{Yana mï} \textbf{\textit{li}} \textit{membam i atwana mat.}\\
\gll    yana  mï      \textbf{li}    ma=imbam    i    atwana ma=ta\\
    woman  3\textsc{sg.subj}  down  3\textsc{sg.obj}=under  go.\textsc{pfv}  question    3\textsc{sg.obj}=say\\
\glt `[His] wife came down under him and asked him a question.’ (The man in the story is up in a house.) [ulwa001\_15:10]
\z

\ea%81
    \label{ex:otherwc:81}
          \textit{Anda ngunaya} \textbf{\textit{li}} \textit{nayn.}\\
\gll    anda    ngunan=iya    \textbf{li}    na-i-n\\
    \textsc{sg.dist}  1\textsc{du.incl=}toward  down  \textsc{detr}{}-come-\textsc{pfv}\\
\glt `That one has come down to us [from the house].’ [ulwa014†]
\z

The \isi{locative adverb} \textit{mbï} ‘here’ may be used to indicate \isi{direction} toward the speaker (i.e., ‘hither’) \REF{ex:otherwc:82}.

\ea%82
    \label{ex:otherwc:82}
          \textit{Na manji yalum ngala} \textbf{\textit{mbï}} \textit{indap.}\\
\gll    na    ma-nji      yalum    ngala    \textbf{mbï}  inda-p\\
    and    \textsc{3sg.obj-poss}  grandchild  \textsc{pl.prox}  here  walk-\textsc{pfv}\\
\glt `And his grandchildren walked here.’ (\textit{na} < TP \textit{na} ‘and’) [ulwa014†]
\z

\is{adverb|)}
\is{locative adverb|)}

\is{locative adverb|(}
\is{adverb|(}


Often, as in \REF{ex:otherwc:82}, the \isi{adverb} \textit{mbï} ‘here’ occurs as the first member of a \isi{compound verb}. The second member is usually a \isi{motion} verb, such as \textit{ma-} {\textasciitilde} \textit{i-} ‘go’, and the \isi{compound} has the sense ‘come (here)’ as seen in \REF{ex:otherwc:83}.

\ea%83
    \label{ex:otherwc:83}
          \textit{Atuma numan anda mï} \textbf{\textit{mbi}}.\\
\gll Atuma  numan    anda    mï      \textbf{mbï-i}\\
    [name]  husband  \textsc{sg.dist}  \textsc{3sg.subj}  here-go.\textsc{pfv}\\
\glt `Atuma’s husband -- he came.’ [ulwa014†]
\z

\ea%84
    \label{ex:otherwc:84}
          \textit{Ngata la Wopata ndo} \textbf{\textit{mbi}}.\\
\gll ngata      ala       Wopata  anda=u    \textbf{mbï-i}\\
    grandparent  \textsc{pl.dist}  [place]    \textsc{sg.dist=}from  here-go.\textsc{pfv}\\
\glt `The ancestors came here from Wopata.’ [ulwa037\_10:55]
\z

Although such adverbs may form \isi{compound}s with verbs, I do not think it would be correct to say that there exists any \isi{locative} or \isi{direction}al \isi{morphological} marking on verbs in the language.

  \isi{Compound}s formed with \textit{mbï} ‘here’ and other verbs are possible as well, as in \REF{ex:otherwc:85}, where the \isi{compound} is headed by the verb \textit{lï-} ‘put’. In the verb in this example, the \isi{adverb} \textit{mbï} ‘here’ combines with \textit{an} ‘out’, which is not known to occur independently as an \isi{adverb}.

\ea%85
    \label{ex:otherwc:85}
          \textit{Ndï ndït} \textbf{\textit{anmbïlïp}} \textit{ndïmoke amblanane.}\\
\gll    ndï  ndï=tï    \textbf{an-mbï-lï-p}    ndï=moko-e ambla-na-n-e\\
    3\textsc{pl}  3\textsc{pl}=take  out-here-put-\textsc{pfv}  3\textsc{pl}=take-\textsc{dep}    \textsc{pl.refl}=give-\textsc{pfv-dep}\\
\glt `They got them out and shared them among themselves.’ [ulwa014\_29:46]
\z

With \isi{motion} verbs such as \textit{ma-} {\textasciitilde} \textit{i-} ‘go’, \isi{compound}s containing the components \textit{an} ‘out’ and \textit{mbï} ‘here’ give the sense of going outside or coming outside (from being within a house, jungle region, etc.), as in \REF{ex:otherwc:86} and \REF{ex:otherwc:87}.

\ea%86
    \label{ex:otherwc:86}
          \textit{Ndï wolka} \textbf{\textit{anmbi}}.\\
\gll ndï  wolka  \textbf{an-mbï-i}\\
    3\textsc{pl}  again  out-here-go.\textsc{pfv}\\
\glt `They went out again.’ [ulwa014\_16:59]
\z

\ea%87
    \label{ex:otherwc:87}
          \textit{Wolka maya} \textbf{\textit{anmbi}} \textit{mol natana.}\\
\gll    wolka  ma=iya      \textbf{an-mbï-i}      ma=ul na-ta-na\\
    again  3\textsc{sg.obj}=toward  out-here-go.\textsc{pfv}  3\textsc{sg.obj}=with    \textsc{detr-}say-\textsc{irr}\\
\glt `Having come out to him, [we] will talk with him again.’ [ulwa037\_59:34]\\
\z

To indicate \isi{direction} away from the speaker (i.e., ‘hence’), the \isi{adverb} \textit{mbï} ‘here’ may be combined with the \isi{postposition} \textit{u} ‘from, in, at, around, along’ \REF{ex:otherwc:88}.

\ea%88
    \label{ex:otherwc:88}
          \textit{Ngo Ganmalin u} \textbf{\textit{mbu}} \textit{matïn mana.}\\
\gll    nga=u        Ganmali=n  u    \textbf{mbï-u} ma=tï-n      ma-na\\
    \textsc{sg.prox=}from  [name]=\textsc{obl}  from  here-from    3\textsc{sg.obj}=take-\textsc{pfv}  go-\textsc{irr}\\
\glt `From this [place], having gotten it from here, from Ganmali, [they] will go.’ [ulwa014\_33:13]
\z

The same \isi{compound} \textit{mbu} ‘from here, hence’ (< \textit{mbï-u}) can have not only \isi{ablative}, but also \isi{locative} sense (i.e., ‘[at] here’ in addition to ‘from here’), as in \REF{ex:otherwc:89} and \REF{ex:otherwc:90}.

\ea%89
    \label{ex:otherwc:89}
         \textit{Ndïn} \textbf{\textit{mbu}} \textit{inum awe.}\\
\gll    ndï=n    \textbf{mbï-u}    inum  aw-e\\
    3\textsc{pl=obl}  here-from  ground  put.\textsc{ipfv-dep}\\
\glt `[They] bury them here.’ [ulwa028\_04:39]
\z

\ea%90
    \label{ex:otherwc:90}
          \textit{Una we apa} \textbf{\textit{mbu}} \textit{ulwap.}\\
\gll    unan    we    apa    \textbf{mbï-u}    ulwa=p\\
    1\textsc{pl.incl}  sago  house  here-from  nothing=\textsc{cop}\\
\glt `We don’t have any sago here at home.’ [ulwa037\_60:47]
\z

Especially when occurring with the \isi{locative verb} \textit{p-} ‘be at’, \textit{mbï} ‘here’ can serve more of a nominal function -- that is, ‘here’ in the sense of ‘this place’ \REF{ex:otherwc:91}.

\ea%91
    \label{ex:otherwc:91}
          \textit{Owet yena ngusuwa anda} \textbf{\textit{mbïpe}}.\\
\gll Owet  yena    ngusuwa  anda    \textbf{mbï-p}{}-e\\
    [name]  woman    poor    \textsc{sg.dist}  here-be\textsc{{}-ipfv}\\
\glt `Owet’s wife, the poor thing, was here.’ [ulwa014\_38:42]
\z

Like other \isi{deictic} words, \textit{mbï} ‘here’ can also be used by speakers to project a \isi{deictic center} to a point other than the ego (\sectref{sec:7.3}), as in \REF{ex:otherwc:92}, where it is translated in \ili{English} as ‘there’.

\ea%92
    \label{ex:otherwc:92}
          \textit{Alum mokotïp an mol} \textbf{\textit{mbïwap}}.\\
\gll alum  ma=kot-p        an      ma=ul      \textbf{mbï-wap}\\
    child  3\textsc{sg.obj}=break-\textsc{pfv}  1\textsc{pl.excl}  3\textsc{sg.obj}=with  here-be.\textsc{pst}\\
\glt `She bore a child, and we were there with her.’ [ulwa014\_38:44]
\z

The \isi{locative} word \textit{ando} ‘there, thence’ is a \isi{compound}, composed of the \isi{deictic} word \textit{anda} ‘that’ and the \isi{postposition} \textit{u} ‘from, in, at, around, along’. Examples \REF{ex:otherwc:93}, \REF{ex:otherwc:94}, and \REF{ex:otherwc:95} illustrate its use. See also example \REF{ex:det:153} in \sectref{sec:7.3} and example \REF{ex:pred:1} in \sectref{sec:10.1}.

\ea%93
    \label{ex:otherwc:93}
          \textbf{\textit{Ando}} \textit{una mape.}\\
\gll    \textbf{anda=u}    unan    ma=p-e\\
    \textsc{sg.dist}=from  \textsc{1pl.incl}  3\textsc{sg.obj}=be\textsc{{}-ipfv}\\
\glt `We are there.’ [ulwa042\_06:22]
\z

\ea%94
    \label{ex:otherwc:94}
          \textit{Nga nganji pul} \textbf{\textit{ando}}.\\
\gll nga      nga-nji      pul    \textbf{anda=u}\\
    \textsc{sg.prox}  \textsc{sg.prox-poss}  piece  \textsc{sg.dist}=from\\
\glt `This is this one’s piece [of the river] over there.’ [ulwa014\_66:29]
\z

\ea%95
    \label{ex:otherwc:95}
          \textit{Tïlwa mï} \textbf{\textit{ando}} \textit{i.}\\
\gll    tïlwa  mï      \textbf{anda=u}    i\\
    road  3\textsc{sg.subj}  \textsc{sg.dist}=from  go.\textsc{pfv}\\
\glt `The track went from there.’ [ulwa029\_07:03]
\z

The \isi{locative} words \textit{nu} ‘near’, \textit{ngaya} ‘far’, and \textit{wala} ‘far, far-off’ generally function as \isi{adjective}s (\sectref{sec:5.4}), but they do contain some curious distributional properties, such as variable \isi{word order} with respect to other constituents. This makes them seem somewhat adverb-like. Moreover, since their etymologies seem to reflect origins as \isi{postpositional phrase}s, it should not be surprising that they behave more like \isi{oblique}s than like prototypical \isi{adjective}s (\sectref{sec:11.4}). I propose \mbox{etymologies} for these three words in \REF{ex:otherwc:96}.

\ea%96
    \label{ex:otherwc:96}
          Possible etymologies of some \isi{locative} words\\
\begin{tabbing}
{(\textit{ngaya})} \= {(‘far(-off)’)} \= {(< \textit{nga} ‘\textsc{sg.prox}’ + \textit{iya} ‘toward’)} \= {(= ‘toward this (place)’)}\kill
{\textit{nu}} \> {‘near’} \> {< \textit{nï} ‘1\textsc{sg}’ + \textit{u} ‘from, around’} \> {= ‘around me’}\\
{\textit{ngaya}} \> {‘far’} \> {< \textit{nga} ‘\textsc{sg.prox}’ + \textit{iya} ‘toward’} \> {= ‘toward this (place)’}\\
{\textit{wala}} \> {‘far(-off)’} \> {< \textit{u} ‘2\textsc{sg}’ + \textit{ala} ‘for, from’} \> {= ‘(away) from you’}
\end{tabbing}
\z

In \REF{ex:otherwc:97}, \textit{nu} ‘near’ is modifying a verb -- that is, it is functioning as an \isi{adverb}.

\ea%97
    \label{ex:otherwc:97}
          \textit{Iwïl nga \textbf{nu} kukawe.}\\
\gll iwïl  nga      \textbf{nu}    kuk-aw-e\\
    moon  \textsc{sg.prox}  near  gather-put.\textsc{ipfv-dep}\\
\glt `[The end of] this month is drawing near.’ [ulwa037\_59:15]
\z

Likewise, \REF{ex:otherwc:98} illustrates the adverbial use of \textit{ngaya} ‘far’, which can also be seen in example \REF{ex:syntax:43} in \sectref{sec:13.1.2} and example \REF{ex:syntax:110} in \sectref{sec:13.2.4}.

\ea%98
    \label{ex:otherwc:98}
          \textit{Nï ndul \textbf{ngaya} mana awlop.}\\
\gll    nï    ndï=ul    \textbf{ngaya}  ma-na  awlop\\
    1\textsc{sg}  3\textsc{pl}=with  far    go-\textsc{irr}  in.vain\\
\glt `I want to go far with them but can’t.’ [ulwa032\_49:20]
\z

There are no instances in the Ulwa corpus of texts in which \textit{wala} ‘far(-off)’ functions as an \isi{adverb}. It always precedes the noun \textit{luwa} ‘place’, which it modifies; thus, it may be most parsimonious to analyze \textit{wala luwa} ‘far-off place’ as a single \isi{compound noun}, one which follows the general trend in Ulwa of the \isi{head} of the \isi{endocentric compound} occurring as the final member (\sectref{sec:3.3}).

\is{adverb|)}
\is{locative adverb|)}

\subsection{Adverbs of manner}\label{sec:8.2.3}

\is{manner adverb|(}
\is{adverb|(}
\is{adverb of manner|(}

Adverbs of manner modify sentences by providing additional information on the way in which an event occurs or a state exists. Although this information is often conveyed through other means (e.g., \isi{adjective}s, \isi{postpositional phrase}s, or even whole clauses), there is a small set of manner adverbs, the most frequent of which are given in \REF{ex:otherwc:99}.

\ea%99
\label{ex:otherwc:99}
Manner adverbs\\
\begin{tabbing}
{(\textit{maweka} {\textasciitilde} \textit{moweka})} \= {(‘thus, in this manner, in that manner’)}\kill
\textit{apka} \> ‘very’\\
\textit{keka} {\textasciitilde} \textit{kaka} \> ‘completely’\\
\textit{maka} \> ‘thus, in this manner, in that manner’\\
\textit{wolka} \> ‘again, in turn’\\
\textit{maweka} {\textasciitilde} \textit{moweka} \> ‘also, moreover’\\
\textit{luke} \> ‘also, too’
\end{tabbing}
\z
  
One readily apparent formal trait shared by all these words is their ending in [ka], which, in these words, is taken to be a formative meaning something like ‘thus, in this manner, in that manner’.\footnote{The form \textit{luke} ‘also, too’ may derive from \textit{ul} {\textasciitilde} \textit{lu} ‘with’ + \textit{ka} ‘thus’ + \textit{{}-e} ‘\textsc{dep}’ -- that is, the form [ka] is probably present in this \isi{adverb} as well, at least diachronically.} These adverbs of manner seem much less amenable to pre-subject position than the \isi{temporal adverb}s are, and their inclusion within the larger class of adverbs is, admittedly, largely based on \isi{semantic} grounds. Sentences \REF{ex:otherwc:100} through \REF{ex:otherwc:112} illustrate their use.
  
  \ea%100
    \label{ex:otherwc:100}
          \textit{Woni mï} \textbf{\textit{apka}} \textit{wutota.}\\
\gll    Woni  mï      \textbf{apka}  wutota\\
    [name]  3\textsc{sg.subj}  very  tall\\
\glt `Woni is very tall.’ [elicited]
\z

\ea%101
    \label{ex:otherwc:101}
          \textit{Amun ane ngo} \textbf{\textit{apka}} \textit{nïpat awlu ato!}\\
\gll    amun  ane  nga{}=o      \textbf{apka}  nïpat  awlu  ata-u\\
    now  sun  \textsc{sg.prox{}=voc} very  giant  step  up-from\\
\glt `Now, this sun has really come out very strong!’ [ulwa038\_00:05]
\z

\ea%102
    \label{ex:otherwc:102}
          \textit{Ango} \textbf{\textit{apka}} \textit{nu luwa me.}\\
\gll    ango  \textbf{apka}  nu    luwa  me\\
    \textsc{neg}  very  close  place  \textsc{neg}\\
\glt `[It] wasn’t a close place.’ [ulwa029\_10:01]
\z

\ea%103
    \label{ex:otherwc:103}
          \textit{Apïn} \textbf{\textit{keka}} \textit{ndïn mol amap.}\\
\gll    apïn  \textbf{keka}      ndï=n    ma=ul      ama-p\\
    fire    completely  3\textsc{pl=obl}  3\textsc{sg.obj}=with  eat-\textsc{pfv}\\
\glt `They [= saucepans] were totally burned with it [= the house].’ (Literally ‘Fire completely ate with them along with it.’) [ulwa014\_32:40]
\z

\ea%104
    \label{ex:otherwc:104}
          \textit{Ala} \textbf{\textit{keka}} \textit{tïlwa le.}\\
\gll    ala      \textbf{keka}      tïlwa  lo-e\\
    \textsc{pl.dist}  completely  road  go-\textsc{ipfv}\\
\glt `Those [children] make tracks all around.’ [ulwa032\_13:09]
\z

\ea%105
    \label{ex:otherwc:105}
          \textit{Mï} \textbf{\textit{keka}} \textit{nungunup.}\\
\gll    mï      \textbf{keka}      nungun-u-p\\
    3\textsc{sg.subj}  completely  break-put-\textsc{pfv}\\
\glt `It broke completely.’ [ulwa037\_01:21]
\z

\ea%106
    \label{ex:otherwc:106}
          \textit{Nï} \textbf{\textit{maka}} \textit{man ndït.}\\
\gll    nï    \textbf{maka}  ma=n      ndï=ta\\
    1\textsc{sg}  thus  3\textsc{sg.obj=obl}  3\textsc{pl}=say\\
\glt `I told them like this.’ [ulwa014\_72:29]
\z

\ea%107
    \label{ex:otherwc:107}
          \textit{Ndï} \textbf{\textit{maka}} \textit{i.}\\
\gll    ndï  \textbf{maka}  i\\
    3\textsc{pl}  thus  go.\textsc{pfv}\\
\glt `They went like this.’ [ulwa032\_13:15]
\z

\ea%108
    \label{ex:otherwc:108}
          \textit{Mbiyen} \textbf{\textit{maka}} \textit{nï kaka mbïpe.}\\
\gll    mbï-i-en      \textbf{maka}  nï    kaka      mbï-p-e\\
    here-go.\textsc{pfv-nmlz}  thus  1\textsc{sg}  completely  here-be\textsc{{}-ipfv}\\
\glt `Having come here, I have thus always stayed here.’ [ulwa027\_00:09]
\z

\ea%109
    \label{ex:otherwc:109}
          \textit{Mï} \textbf{\textit{wolka}} \textit{impul matïn.}\\
\gll    mï      \textbf{wolka}  {im-pul}      ma=tï-n\\
    3\textsc{sg.subj}  again  wood-piece  3\textsc{sg.obj}=take-\textsc{pfv}\\
\glt `It again got a piece of wood.’ [ulwa006\_03:08]
\z

\ea%110
    \label{ex:otherwc:110}
          \textit{Nï} \textbf{\textit{wolka}} \textit{man mat: …}\\
\gll    nï    \textbf{wolka}   ma=n      ma=ta\\
    1\textsc{sg}  again  3\textsc{sg.obj=obl}  3\textsc{sg.obj}=say\\
\glt `I in turn said to her: …’ [ulwa014\_20:49]
\z

\ea%111
    \label{ex:otherwc:111}
          \textit{Ngata ngusuwa nga} \textbf{\textit{moweka}} \textit{wa i.}\\
\gll    ngata  ngusuwa  nga      \textbf{moweka}  wa     i\\
    grand  poor    \textsc{sg.prox}  also    village  go.\textsc{pfv}\\
\glt `This poor grandfather also came home.’ [ulwa014\_72:24]
\z

\ea%112
    \label{ex:otherwc:112}
          \textit{Inom ndï} \textbf{\textit{moweka}} \textit{ango unan tïngïn inap.}\\
\gll    inom  ndï  \textbf{moweka}  ango  unan    tïngïn  ina-p\\
    mother  3\textsc{pl}  also    \textsc{neg}  \textsc{1pl.incl}  many  get-\textsc{pfv}\\
\glt `And another thing: [our] mothers didn’t have many of us.’ [ulwa014\_46:29]
\z

The \isi{adverb} \textit{luke} ‘also, too’ can be found in examples \REF{ex:verbs:45} in \sectref{sec:4.8}, \REF{ex:verbs:60} in \sectref{sec:4.9}, \REF{ex:otherwc:7} in \sectref{sec:8.1}, \REF{ex:complex:1} in \sectref{sec:12.1.1}, \REF{ex:complex:96} in \sectref{sec:12.3}, and \REF{ex:syntax:175} in \sectref{sec:13.3.4}.

  In addition to these, there is another \isi{adverb} of manner, one which seems to be permitted only in \isi{negative} \isi{polarity}, and which is in some ways the \isi{negative} counterpart to \textit{wolka} ‘again’. This \isi{adverb}, \textit{tïki} ‘(ever) again, anymore, else’, may be seen in \isi{negative} sentences, such as \REF{ex:otherwc:113} and \REF{ex:otherwc:114}.

\is{adverb of manner|)}
\is{adverb|)}
\is{manner adverb|)}

\is{manner adverb|(}
\is{adverb|(}
\is{adverb of manner|(}

\ea%113
    \label{ex:otherwc:113}
          \textit{Ndï} \textbf{\textit{ango}} \textbf{\textit{tïki}} \textit{itom luwa ndule.}\\
\gll    ndï  \textbf{ango}  \textbf{tïki}    itom  luwa  ndï=u-lo-e\\
    3\textsc{pl}  \textsc{neg}  again  father  place  3\textsc{pl}=from-go-\textsc{ipfv}\\
\glt `They don’t go around in [their] father’s places anymore.’ [ulwa032\_12:41]
\z

\ea%114
    \label{ex:otherwc:114}
          \textit{Nï} \textbf{\textit{ango}} \textbf{\textit{tïki}} \textit{ikali usina.}\\
\gll    nï    \textbf{ango}  \textbf{tïki}    i-kali    u=si-na\\
    1\textsc{sg}  \textsc{neg}  again  hand-send  2\textsc{sg=}push-\textsc{irr}\\
\glt `I won’t hold you again.’ [ulwa032\_18:18]
\z

Notably, \textit{tïki} ‘(ever) again, anymore, else’ is permitted in \isi{question}s as well, which, at least historically, seem to have derived from clauses of \isi{negative} \isi{polarity} (\sectref{sec:13.1.2}). Questions \REF{ex:otherwc:115} and \REF{ex:otherwc:116} both contain  \textit{tïki} ‘(ever) again, anymore, else’.

\ea%115
    \label{ex:otherwc:115}
          \textbf{\textit{Ango}} \textit{luwa} \textbf{\textit{tïki}} \textit{ko nji kuma ndïtïna?}\\
\gll    \textbf{ango}  luwa  \textbf{tïki}    ko  nji    kuma  ndï=tï-na\\
    which  place  more  just  thing  some  3\textsc{pl}=take-\textsc{irr}\\
\glt `Where else could [we] get some things?’ [ulwa032\_20:45]
\z

\ea%116
    \label{ex:otherwc:116}
          \textbf{\textit{Tïki}} \textit{unan} \textbf{\textit{angos}} \textit{natana?}\\
\gll    \textbf{tïki}    unan    \textbf{angos}  na-ta-na\\
    again  1\textsc{pl.incl}  what  \textsc{detr}{}-say-\textsc{irr}\\
\glt `What else should we say?’ [ulwa037\_34:10]
\z

The \isi{adverb} \textit{maka} ‘thus’, unlike most other adverbs, somewhat commonly occurs with the \isi{copular enclitic} \textit{=p} ‘\textsc{cop}’. The effectively \isi{verbalized} form [makap] has the meaning ‘be like’, as in \REF{ex:otherwc:117}.

\ea%117
    \label{ex:otherwc:117}
          \textit{Kalim mï} \textbf{\textit{makap}}.\\
\gll kalim    mï      \textbf{maka=p}\\
    cassowary  3\textsc{sg.subj}  thus=\textsc{cop}\\
\glt `The cassowary is like that.’ [ulwa014\_46:59]
\z

\is{adverb of manner|)}
\is{adverb|)}
\is{manner adverb|)}

\is{manner adverb|(}
\is{adverb|(}
\is{adverb of manner|(}

In \REF{ex:otherwc:118}, the form [makap] ‘be like’ is further marked with the \isi{dependent marker} \textit{-e} ‘\textsc{dep}’.

\ea%118
    \label{ex:otherwc:118}
          \textit{Amun una keka} \textbf{\textit{makape}}.\\
\gll amun  unan    keka      \textbf{maka=p-e}\\
    now  1\textsc{pl.incl}  completely  thus=\textsc{cop-dep}\\
\glt `But nowadays we are completely like this.’ [ulwa014\_39:46]
\z

This \isi{verbalized} form of \textit{maka} ‘thus’ can even, in turn, be \isi{nominalize}d, as in \REF{ex:otherwc:119}.

\ea%119
    \label{ex:otherwc:119}
          \textbf{\textit{Makapen}} \textit{mï nay.}\\
\gll    \textbf{maka=p-en}    mï      na-i\\
    thus=\textsc{cop-nmlz}  \textsc{3sg.subj}  \textsc{detr-}go.\textsc{pfv}\\
\glt `That way has gone.’ [ulwa014\_62:56]
\z

This \isi{verbalized} form of \textit{maka} ‘thus’ is often used in \isi{relative clause}s (\sectref{sec:12.3}), as in \REF{ex:otherwc:120} and \REF{ex:otherwc:121}.

\ea%120
    \label{ex:otherwc:120}
          \textit{Yetani lan u} \textbf{\textit{makape}} \textit{ambet matïn.}\\
\gll    Yetani  ala=n      u    [\textbf{maka=p-e}]    ambet ma=tï-n\\
    Yamen  \textsc{pl.dist=obl}  from  [thus=\textsc{cop-dep]}  magic    \textsc{3sg.obj}=take-\textsc{pfv}\\
\glt `[They] got magic like this from the Yamen people.’ [ulwa037\_11:02]
\z

\ea%121
    \label{ex:otherwc:121}
          \textit{U} \textbf{\textit{makape}} \textit{nji ulwata u awlop!}\\
\gll    u    [\textbf{maka-p-e}      nji]    ulwa-ta    u    awlop\\
    2\textsc{sg}  [thus=\textsc{cop-dep}  thing]  nothing-\textsc{cond}  2\textsc{sg}  in.vain\\
\glt `If you don’t have things like this, you’re lost!’ [ulwa032\_58:53]
\z

Often the \isi{embedded clause} formed with [makape] has a similar grammatical function to the plain \isi{adverb} \textit{maka} ‘thus’ \REF{ex:otherwc:122}.

\ea%122
    \label{ex:otherwc:122}
          \textit{Un} \textbf{\textit{makape}} \textit{imba wombam niya ita …}\\
\gll    un  [\textbf{maka=p-e}]    imba  wombam  nï=iya      i-ta\\
    2\textsc{pl}  [thus=\textsc{cop-dep]}  night  middle    1\textsc{sg=}toward  go\textsc{.pfv-cond}\\
\glt `If you come to me like this in the middle of the night …’ [ulwa014\_40:16]
\z

In addition to its use as an \isi{adverb}, \textit{maka} ‘thus’ is very frequently used as a filler word (cf. \ili{Tok Pisin} \textit{olsem} ‘thus’, \ili{German} \textit{also} ‘thus’, etc.). When used as such, it is generally translated as ‘like’, following contemporary \ili{English} \isi{idiom} \REF{ex:otherwc:123}. As a filler word, \textit{maka} ‘thus’ can occur in any position in a sentence, even within NPs, as in \REF{ex:otherwc:124}.

\is{adverb of manner|)}
\is{adverb|)}
\is{manner adverb|)}

\ea%123
    \label{ex:otherwc:123}
          \textit{Wusimali} \textbf{\textit{maka}} \textit{in tï Kayta nane.}\\
\gll    Wusimali  \textbf{maka}  in  tï    Kayta  na-n-e\\
    [name]    thus  get  take  [name]  give-\textsc{pfv-dep}\\
\glt `Wusimali, like, bought [an axe] and gave [it] to Kayta.’ [ulwa014\_53:34]
\z

\ea%124
    \label{ex:otherwc:124}
          \textit{Anji} \textbf{\textit{maka}} \textit{ngata ndï ndul iyen.}\\
\gll    an-nji        \textbf{maka}  ngata  ndï  ndï=ul    i-en\\
    1\textsc{pl.excl-poss}  thus  grand  \textsc{3pl}  3\textsc{pl}=with  go.\textsc{pfv-nmlz}\\
\glt `Our, like, ancestors were the ones who went with them.’ [ulwa002\_04:09]
\z

\subsection{{The} {epistemic} {adverb} {\textit{tap}} {‘maybe’}}\label{sec:8.2.4}

\is{epistemic adverb|(}
\is{adverb|(}
\is{epistemic|(}

The \isi{adverb} \textit{tap} ‘maybe’ is used to show the possibility of an event’s occurrence, whether \isi{present}, \isi{past}, or \isi{future}. Fittingly, since its use signals speculation on the part of the speaker, it often accompanies a verb with the \isi{speculative} \isi{suffix} \textit{-t} ‘\textsc{spec}’ (\sectref{sec:4.11}). Like other adverbs, it often occurs immediately after the subject, when expressed. It tends to precede \isi{temporal adverb}s, when these occur in the same clause. It does not permit any form of \isi{inflection}. Sentences \REF{ex:otherwc:125}, \REF{ex:otherwc:126}, and \REF{ex:otherwc:127} illustrate the use of the \isi{epistemic adverb} \textit{tap} ‘maybe’.

\ea%125
    \label{ex:otherwc:125}
          \textit{Mï} \textbf{\textit{tap}} \textit{amun ina.}\\
\gll    mï      \textbf{tap}    amun  i-na\\
    3\textsc{sg.subj}  maybe  now  come-\textsc{irr}\\
\glt `He might come today.’ [ulwa032\_03:44]
\z

\ea%126
    \label{ex:otherwc:126}
          \textbf{\textit{Tap}} \textit{umbe Kumba mana.}\\
\gll    \textbf{tap}    umbe    Kumba  ma-na\\
    maybe  tomorrow  Bun  go-\textsc{irr}\\
\glt `Maybe tomorrow [I] will go to Bun [village].’ [ulwa037\_48:48]
\z

\ea%127
    \label{ex:otherwc:127}
          \textbf{\textit{Tap}} \textit{manji yawa ngawl i.}\\
\gll    \textbf{tap}    ma-nji      yawa  nga=ul      i\\
    maybe  3\textsc{sg.obj-poss}  uncle  \textsc{sg.prox}=with  go.\textsc{pfv}\\
\glt `[He] might have gone with [his] uncle.’ [ulwa014\_33:07]
\z

The \isi{adverb} \textit{tap} ‘maybe’ is \isi{homophonous} with the \isi{perfective} form of the verb \textit{ta-} ‘say’, and the \isi{adverb} very well may derive from this form -- after all, that which has merely been ‘said’ -- but which is not known to be true -- can easily be taken to be \isi{speculative}.

\is{epistemic|)}
\is{adverb|)}
\is{epistemic adverb|)}

\subsection{Other modal and discourse adverbs}\label{sec:8.2.5}

\is{modal adverb|(}
\is{discourse adverb|(}
\is{adverb|(}
\is{mood|(}

The most frequent discourse adverbs are given in \REF{ex:otherwc:128}. It is notoriously difficult to provide accurate translations of words that serve modal or discourse functions. The glosses provided here represent my best approximation of their meaning and function.

\ea%128
    \label{ex:otherwc:128}
          Discourse adverbs\\
\begin{tabbing}
{(\textit{kwa} {\textasciitilde} \textit{ko} {\textasciitilde} \textit{wa})} \= {(‘pointlessly, fruitlessly’)}\kill
\textit{kop} \> ‘please’\\
\textit{kwa} {\textasciitilde} \textit{ko} {\textasciitilde} \textit{wa} \> ‘just’\\
\textit{lolop} \> ‘just’\\
\textit{woyambïn} \> ‘pointlessly, fruitlessly’
\end{tabbing}
  \z

The \isi{adverb} \textit{kop} ‘please’ is often used to soften \isi{command}s -- that is, to make \isi{polite} \isi{request}s \REF{ex:otherwc:129}.

\ea%129
    \label{ex:otherwc:129}
          \textbf{\textit{Kop}} \textit{nambï wiwila lakana!}\\
\gll    \textbf{kop}  nambï  wiwila  la-ka-na\\
    please  skin  light  \textsc{irr-}let-\textsc{irr}\\
\glt `Let [your] body [become] light!’ (i.e., wait until you are no longer pregnant [before attempting to play sports]) [ulwa032\_34:43]
\z

More examples and details relating to this use of \textit{kop} ‘please’ may be found in the section on \isi{command}s and \isi{request}s (\sectref{sec:13.2.2}). As an \isi{adverb}, \textit{kop} ‘please’ can also be used in statements. Here it can convey a sense of care or patience \REF{ex:otherwc:130}.

\ea%130
    \label{ex:otherwc:130}
          \textit{Mï} \textbf{\textit{kop}} \textit{lïmndï anul pe.}\\
\gll    mï      \textbf{kop}  lïmndï  an=ul        p-e\\
    3\textsc{sg.subj}  please  eye    1\textsc{pl.excl}=with  be\textsc{{}-ipfv}\\
\glt `She stays with us, watching [us] patiently.’ [ulwa013\_00:42]
\z

Three forms that are frequently used in discourse are \textit{kwa} {\textasciitilde} \textit{ko} {\textasciitilde} \textit{wa} ‘just’, the first of which is identical to the \isi{numeral} \textit{kwa} ‘one’, and the second of which is clearly derived from the first.\footnote{The form \textit{wa} ‘just’ may also be related to \textit{kwa} ‘one’, although its derivation is less clear.} All three forms share essentially the same set of functions. Often translated as ‘just’, they add a degree of casualness to a statement. Sometimes they convey a sense of ‘simply’, other times a mildly \isi{negative} sense of ‘without care’ or ‘without reason’. Very often, however, it is hard for me to ascribe any clear meaning to them (at least in the \ili{English} translation). These three forms -- \textit{kwa} {\textasciitilde} \textit{ko} {\textasciitilde} \textit{wa} ‘just’ -- are illustrated by examples \REF{ex:otherwc:131}, \REF{ex:otherwc:132}, and \REF{ex:otherwc:133}, respectively.

\ea%131
    \label{ex:otherwc:131}
          \textit{Ay nï} \textbf{\textit{kwa}} \textit{apa mbïpe mane?}\\
\gll    ay  nï    \textbf{kwa}  apa    mbï-p-e    ma-n-e\\
    ay  1\textsc{sg}  just    house  here-be-\textsc{dep}  go-\textsc{ipfv-dep}\\
\glt `Ay, am I just going to stay here?’ [ulwa014\_60:53]
\z

\ea%132
    \label{ex:otherwc:132}
          \textit{Lamndu} \textbf{\textit{ko}} \textit{minamap.}\\
\gll    lamndu  \textbf{ko}  min=ama-p\\
    pig      just  3\textsc{du}=eat-\textsc{pfv}\\
\glt `A pig ate them.’ [ulwa032\_22:11]
\z

\ea%133
    \label{ex:otherwc:133}
          \textbf{\textit{Wa}} \textit{inde le.}\\
\gll    \textbf{wa}  inda-e    lo-e\\
    just  walk-\textsc{dep}  go-\textsc{ipfv}\\
\glt `[We] would just walk around.’ [ulwa013\_03:30]
\z

Serving the same function as \textit{kwa} {\textasciitilde} \textit{ko} {\textasciitilde} \textit{wa} ‘just’ is the \isi{adverb} \textit{lolop} ‘just’, which has been \isi{borrow}ed from the neighboring language \ili{Ap Ma}. It often occurs immediately following \textit{wa} ‘just’, but may occur independently as well. In examples \REF{ex:otherwc:134}, \REF{ex:otherwc:135}, and \REF{ex:otherwc:136}, it has a \isi{frustrative} sense.

\ea%134
    \label{ex:otherwc:134}
          \textit{Nï wa} \textbf{\textit{lolop}} \textit{i mangusuwa nji molop lïp.}\\
\gll    nï    wa  \textbf{lolop}  i    ma-ngusuwa  nji    ma=lo-p lï-p\\
    1\textsc{sg}  just  just    go.\textsc{pfv}  3\textsc{sg.obj-}poor  thing  \textsc{3sg.obj}=cut-\textsc{pfv}    put-\textsc{pfv}\\
\glt `Frustratedly, I just went and cut the poor thing’s thing [= sago palm jungle].’ [ulwa014\_06:37]
\z

\ea%135
    \label{ex:otherwc:135}
            \textit{Una wa} \textbf{\textit{lolop}} \textit{wa pe.}\\
\gll    unan    wa  \textbf{lolop}  wa    p-e\\
    1\textsc{pl.excl}  just  just    village  be-\textsc{ipfv}\\
\glt `We are just [hanging around] in the village.’ [ulwa037\_09:40]
\z

\ea%136
    \label{ex:otherwc:136}
          \textit{Nambi tembi nape nï wa} \textbf{\textit{lolop}} \textit{indana.}\\
\gll    nï-ambi  tembi  na-p-e      nï    wa  \textbf{lolop}  inda-na\\
    1\textsc{sg-top}  bad    \textsc{detr}{}-be-\textsc{dep}  1\textsc{sg}  just  just    walk-\textsc{irr}\\
\glt `As for me, I’m becoming unfit, so I’ll just go around [without worrying about other people].’ [ulwa032\_47:49]
\z

Similar in function to \textit{kwa} {\textasciitilde} \textit{ko} {\textasciitilde} \textit{wa} ‘just’ and \textit{lolop} ‘just’ is the word \textit{woyambïn} ‘pointlessly, fruitlessly’, which has a much more \isi{negative} connotation. This word looks very much like it has derived from other words, in part because of the unusual \isi{diphthong} [oy] (\sectref{sec:2.1.7}). It may derive from /wa-i-ambï=n/ ‘just-go.\textsc{pfv-sg.refl=obl’} -- that is, a \isi{phrase} having meant something like ‘just went with himself/herself/itself’. This is only speculative. Examples \REF{ex:otherwc:137} and \REF{ex:otherwc:138} illustrate the use of \textit{woyambïn} ‘pointlessly, fruitlessly’.

\ea%137
    \label{ex:otherwc:137}
          \textit{Nï} \textbf{\textit{woyambïn}} \textit{ndul ndïnanape.}\\
\gll    nï    \textbf{woyambïn}  ndï=ul    ndï=na-na-p-e\\
    1\textsc{sg}  pointlessly    3\textsc{pl}=with  3\textsc{pl}=\textsc{detr}{}-feed-\textsc{pfv-dep}\\
\glt `I fed them along with them [= my biological children] for nothing.’ (said in reference to ungrateful foster children) [ulwa032\_47:28]
\z

\ea%138
    \label{ex:otherwc:138}
          \textit{Na} \textbf{\textit{woyambïn}} \textit{matane.}\\
\gll na    \textbf{woyambïn}  ma=ta-n-e\\
    and    pointlessly    3\textsc{sg.obj}=say-\textsc{ipfv-dep}\\
\glt `But [we] are just wasting time talking about it.’ (\textit{na} < TP \textit{na} ‘and’) [ulwa037\_07:36]
\z

Some of the adverbs described in \sectref{sec:8.2.3} also seem to behave at times much like \isi{modal} or discourse adverbs, carrying subtle connotations or serving various discourse functions. The \isi{adverb} \textit{wolka} ‘again, in turn’ may be used in narratives to tie together events in series, especially when they are somewhat repetitive. It may thus often be translated with the \ili{English} expression ‘and then …’. In \REF{ex:otherwc:139}, the narrator recounts a narrative of people traveling frome one place to another, using \textit{wolka} ‘again, in turn’ as a linking element.

\is{mood|)}
\is{adverb|)}
\is{discourse adverb|)}
\is{modal adverb|)}

\newpage

\is{modal adverb|(}
\is{discourse adverb|(}
\is{adverb|(}
\is{mood|(}

\ea%139
    \label{ex:otherwc:139}
          \textit{Biwat inim menklop i atay. Ataye} \textbf{\textit{wolka}} \textit{ngo nay.} \textbf{\textit{Wolka}} \textit{ngo anji wandam ngayte i.}\\
\gll    Biwat  inim  ma=in-klop    i    ata-i    ata-i-e \textbf{wolka}  nga=u        na-i       \textbf{wolka}  nga=u     an-nji        wandam  nga=ita-e        i\\
    [place]  water  3\textsc{sg.obj}=in-cross  go.\textsc{pfv}  up-go.\textsc{pfv}  up-go\textsc{.pfv-dep} again  \textsc{sg.prox}=from  \textsc{detr}{}-go.\textsc{pfv}  again  \textsc{sg.prox}=from    1\textsc{pl.excl-poss}  jungle    \textsc{sg.prox}=build-\textsc{dep}  go.\textsc{pfv}\\
\glt `[They] went following the Biwat river, went up. Having gone up, [they] came this way. And then from here, [they] came and built our jungle [area].’ [ulwa002\_01:08]
\z

The \isi{adverb} \textit{maweka} {\textasciitilde} \textit{moweka} ‘also, moreover’ also seems at times to serve \isi{modal} functions. Its usage here seems parallel to \isi{modal} uses of \ili{Tok Pisin} \textit{tu} ‘also, too’, and it is possibly a \isi{calque} of that word (cf. similar phenomena in \chapref{sec:15}). It can be used to add a degree of incredulity, to strengthen a \isi{request} for confirmation in a \isi{question}, or to add a sense of wonder to a statement. Its use is illustrated by examples \REF{ex:otherwc:140} and \REF{ex:otherwc:141}.

\ea%140
    \label{ex:otherwc:140}
          \textit{Nambi} \textbf{\textit{maweka}} \textit{nïnji ala wala luwa ndap.}\\
      \gll nï-ambi  \textbf{maweka}  nï-nji    ala      wala  luwa anda=p\\
    1\textsc{sg-top}  also    1\textsc{sg-poss}  \textsc{pl.dist}  far.off  place    \textsc{sg.dist}=be\\
\glt `As for me, those [relatives] of mine are in a far-off place.’ [ulwa014\_72:14]
\z

\ea%141
    \label{ex:otherwc:141}
          \textit{Kanangula} \textbf{\textit{moweka}} \textit{ango wa mbïwap.}\\
\gll    Kanangula  \textbf{moweka}  ango  wa    mbï-wap\\
    [name]    also    \textsc{neg}  village  here-be\textsc{.pst}\\
\glt `Kanangula did not [even bother to] stay in the village.’ [ulwa014\_42:48]
\z

It has already been shown how the \isi{adverb} \textit{maka} ‘thus’ can also function as a filler word (\sectref{sec:8.2.3}). In a somewhat similar fashion, the \isi{placeholder word} \linebreak \textit{mïngamata} ‘whatchamacallit’ can be used when a speaker is trying to retrieve a word, as in \REF{ex:otherwc:142} and \REF{ex:otherwc:143}.

\is{mood|)}
\is{adverb|)}
\is{discourse adverb|)}
\is{modal adverb|)}

\ea%142
    \label{ex:otherwc:142}
          \textit{Kolpe manji} \textbf{\textit{mïngamata}} \textit{wonmi ndïwonpop.}\\
\gll    Kolpe  ma-nji      \textbf{mïngamata}    wonmi  ndï=won-p-op\\
    [name]  3\textsc{sg.obj-poss}  whatchamacallit  hair  3\textsc{pl}=cut-\textsc{pfv-pfv}\\
\glt `Kolpe had cut his -- what’s it? -- hair.’ [ulwa014\_13:51]
\z

\ea%143
    \label{ex:otherwc:143}
\is{mood}
\is{adverb}
\is{discourse adverb}
\is{modal adverb}
          \textit{Ndul i} \textbf{\textit{mïngamata}} \textit{Yalamba may.}\\
\gll    ndï=ul    i    \textbf{mïngamata}    Yalamba  ma=i\\
    3\textsc{pl}=with  go.\textsc{pfv}  whatchamacallit  Korokopa  3\textsc{sg.obj}=go.\textsc{pfv}\\
\glt `[We] went with them, went to -- what’s it? -- Korokopa [village].’ [ulwa002\_03:48]
\z

\subsection{Functional equivalents of adverbial constructions}\label{sec:8.2.6}

\is{adverb|(}

Finally, to conclude this overview of adverbs, it may be shown how concepts that are sometimes conveyed with adverbs in other languages can be expressed in different ways in Ulwa.

  First, it is possible to use \isi{dependent clause}s to express adverbial notions. Such clauses typically contain \isi{verbalized} forms of \isi{adjective}s or nouns that express properties \REF{ex:otherwc:144}.

\ea%144
    \label{ex:otherwc:144}
          \textbf{\textit{Andïlpe}} \textit{ndïmisisinap.}\\
\gll    \textbf{andïl=p-e}      ndï=misisina-p\\
    careful=\textsc{cop-dep}  3\textsc{pl}=arrange-\textsc{pfv}\\
\glt `[They] carefully arranged them.’ (Literally ‘Being careful, [they] arranged them.’) [ulwa014\_68:39]
\z

Additionally, adverb-like notions can be expressed with \isi{postpositional phrase}s or \isi{oblique}-marked NPs; these usages are often \isi{metaphor}ical, as in \REF{ex:otherwc:145} and \REF{ex:otherwc:146}.

\ea%145
    \label{ex:otherwc:145}
          \textit{\textbf{Nambli lu} manen.}\\
\gll    \textbf{nambli}  \textbf{lu}    ma-n-en\\
    feather    with  go-\textsc{ipfv-nmlz}\\
\glt `[The water] is going quickly.’ (Literally ‘going with feather’) [ulwa038\_04:40]
\z

\ea%146
    \label{ex:otherwc:146}
          \textbf{\textit{Apïnï}} \textit{mowonlïp.}\\
\gll    \textbf{apïn=nï}  ma=won-lï-p\\
    fire=\textsc{obl}  3\textsc{sg.obj}=cut-put-\textsc{pfv}\\
\glt `[He] cut it down quickly.’ (Literally ‘cut it down with fire’) [ulwa013\_08:17]
\z

Of particular interest, however, is Ulwa’s method of placing \isi{adjective}s in object positions to be used adverbially. When this occurs with \isi{transitive} verbs, the putative \isi{direct object} of the verb is \isi{demote}d to an \isi{oblique} and is marked by the \isi{oblique marker} \textit{=n} ‘\textsc{obl}’. See \sectref{sec:11.4.1} for examples of this phenomenon.

\is{adverb|)}

\section{Other small classes}\label{sec:8.3}

\is{word class}
\is{lexical class}
\is{part of speech}

Finally, in this chapter I consider a few other small closed classes, namely \isi{negator}s (\sectref{sec:8.3.1}), \isi{interrogative word}s (\sectref{sec:8.3.2}), and \isi{interjection}s (\sectref{sec:8.3.3}).

\subsection{Negators}\label{sec:8.3.1}

\is{negator|(}
\is{negation|(}

There is in Ulwa a small set of negators, words that indicate that the \isi{polarity} of a sentence is \isi{negative} as opposed to \isi{positive}, which is taken to be the unspecified, default \isi{polarity}. The basic \isi{negative} marker used in \isi{verbal negation} is \textit{ango} ‘\textsc{neg}’ (\sectref{sec:13.3.1}). To express \isi{non-verbal negation}, one of two clause-final negators is often used, sometimes co-occurring with \textit{ango} ‘\textsc{neg}’. These are \textit{me} ‘\textsc{neg’} and \textit{kom} {\textasciitilde} \textit{kome} ‘\textsc{neg}’ (\sectref{sec:13.3.2}). In \isi{negative command}s, the prohibitive marker \textit{wana} {\textasciitilde} \textit{wanap} ‘\textsc{proh’} is used (\sectref{sec:13.3.3}). The \isi{interjection} \textit{ase} ‘no’ is used to respond \isi{negative}ly to \isi{question}s (\sectref{sec:8.3.3}, \sectref{sec:13.3.5}). The negators are summarized in \REF{ex:otherwc:147}.

\ea%147
    \label{ex:otherwc:147}
          Negators\\
\begin{tabbing}
\is{verbal negation}
{(\textit{wana} {\textasciitilde} \textit{wanap})} \= {(\textsc{neg} (‘no, not’, \isi{non-verbal negator}))}\kill
\textit{ango} \> \textsc{neg} (‘not’, verbal negator)\\
\textit{wana} {\textasciitilde} \textit{wanap} \> \textsc{proh} (‘don’t!’)\\
\textit{me} \> \textsc{neg} (‘no, not’, \isi{non-verbal negator})\\
\textit{kom {\textasciitilde} kome} \> \textsc{neg} (‘no, not’, \isi{non-verbal negator})\\
\textit{ase} \> {\textsc{interj} (‘no’, \isi{negative response} word)}
\end{tabbing}
 \z

In addition to these, the word \textit{ulwa} ‘nothing’ has a \isi{negative} meaning. It may also occur clause-finally, either as an \isi{interjection} or as a \isi{verbalized} element with the \isi{copular enclitic} \textit{=p} (\sectref{sec:10.2}) (with the meaning ‘there is nothing’ or ‘there was nothing’). Either way, it often carries emphatic meaning when used clause-finally. This same word \textit{ulwa} ‘nothing’ may also function as a \isi{negative response} word (\sectref{sec:13.3.5}), in particular when someone is being asked for something (e.g., a \isi{request} for betel nut could be met with \textit{ulwa} ‘nothing’, i.e., ‘I do not have any.’).

\is{negation|)}
\is{negator|)}

\subsection{Interrogative words}\label{sec:8.3.2}

\is{interrogative word|(}
\is{question word|(}
\is{question|(}

\is{wh- question}

There is also in Ulwa a small set of \isi{interrogative word}s, which are used in \textit{wh-} questions (i.e., \isi{content question}s). Their forms and functions are described more fully in \sectref{sec:13.1}, but they may be presented together here in terms of their \isi{word class} membership. They are all functionally similar in that they help form \isi{interrogative} sentences. However, they are likely not a \isi{morphosyntactic}ally distinct class, but rather a group composed of different grammatical categories. These \isi{interrogative word}s are given in \REF{ex:otherwc:148}.

\ea%148
    \label{ex:otherwc:148}
          Interrogative words and their possible etymologies
          \begin{tabbing}
{(\textit{ango luwa})} \= {(‘whose? [\textsc{nsg}]’)}  \=    {(< \textit{ango} ‘which?’ + \textit{tem} ‘time’)}\kill
    \textit{kwa}  \>  ‘who? [\textsc{sg}]’ \>   < \isi{cardinal numeral} ‘one’ (\sectref{sec:7.4})\\
    \textit{kuma} \>   ‘who? [\textsc{nsg}]’ \>   < modifier/\isi{quantifier} ‘some’ (\sectref{sec:7.2})\\
    \textit{kwanji} \>   ‘whose? [\textsc{sg}]’ \>   < \textit{kwa} ‘who? [\textsc{sg}]’ + \textit{-nji} ‘\textsc{poss}’ (\sectref{sec:6.2})\\
    \textit{kumanji} \> ‘whose? [\textsc{nsg}]’ \> < \textit{kuma} ‘who? [\textsc{nsg}]’ + \textit{-nji} ‘\textsc{poss}’ (\sectref{sec:6.2})\\
    \textit{ango}  \>  ‘which?’ \>     < \isi{negator} ‘not’ (\sectref{sec:8.3.1})\\
    \textit{angos} \>   ‘what?’ \>     < \textit{ango} ‘which?’ + \textit{s} (?)\footnotemark{}\\
    \textit{ango luwa} \> ‘where?’  \>    < \textit{ango} ‘which?’ + \textit{luwa} ‘place’\\
    \textit{ango tem} \> ‘when?’  \>    < \textit{ango} ‘which?’ + \textit{tem} ‘time’\footnotemark{}\\
    \textit{angwena} \> ‘why?’  \>      < \textit{ango} ‘which?’ + \textit{na} ‘talk, reason, cause’\\
    \textit{anjika} \>   ‘how many?’ \>   < \textit{an-nji} ‘1\textsc{pl.excl-poss’} + \textit{ka} ‘thus’ (?)\\
    \textit{anjikaka} \> ‘how?’   \>     < \textit{anjika} ‘how many’ + \textit{ka-} ‘let’ (?)
    \end{tabbing}
    \z
\footnotetext[6]{The origin of this [s] element is unclear. It possibly derives from \ili{Proto-Keram} *si ‘things’.}
\footnotetext[7]{The word \textit{tem} ‘time’ is a \isi{loanword}; it comes from \ili{Tok Pisin} \textit{taim} ‘time’.}

The etymologies of these \isi{question word}s are discussed in \sectref{sec:13.1.2}. In addition to these \isi{interrogative word}s, the \isi{interjection}s \textit{a} ‘eh?’ or \textit{e} ‘eh?’ may be used at the end of \isi{interrogative} sentences as \isi{tag word}s (\sectref{sec:13.1.1}).

\is{question|)}
\is{question word|)}
\is{interrogative word|)}

\subsection{Interjections}\label{sec:8.3.3}

\is{interjection|(}

Finally, there are in Ulwa a number of \isi{interjection}s, usually short words used to express a variety of thoughts or emotions. The words equivalent to ‘yes’, ‘no’, and ‘OK’ are considered here as well. In the list of \isi{interjection}s given in \REF{ex:otherwc:149}, the exclamation point (!) indicates emphatic pronunciation, the question mark (?) indicates rising \isi{intonation}, and the triangular colon (ː) indicates extended \is{vowel length} \isi{vowel} \isi{length}. Note also that the \isi{interjection} \textit{mm} ‘uh-uh’ is pronounced as two \isi{syllabic nasal}s separated by a \is{glottal stop} \isi{glottal} \isi{stop} (i.e., [mʔm]).

In this grammar, \isi{interjection}s are sometimes glossed with translations like those in \REF{ex:otherwc:149}, and at other times they are glossed with the abbreviation ‘\textsc{interj}’, depending on what is clearer. The two forms on this list that deserve the most comment are \textit{=o} ‘\textsc{voc}’ and \textit{mawnam} ‘that’s it’, the former since it seems limited to use as an \isi{enclitic}, and the latter because it seems to be polymorphemic, at least historically.

\newpage

\ea%149
    \label{ex:otherwc:149}
          Interjections
    \begin{tabbing}
{(\textit{mawnam})} \= {(‘hey!’ (expresses excitement, either positive or negative))}\kill
    \textit{iyo}   \>   ‘yes’ (expresses affirmation)\\
    \textit{iya}  \>    ‘yeah’ (expresses affirmation)\\
    \textit{ase}  \>    ‘no’ (expresses denial)\\
    \textit{asa}  \>    ‘nah’ (expresses denial)\\
    \textit{ande}  \>  ‘OK’ (expresses agreement, etc.)\\
    \textit{andi}  \>  ‘OK’ (expresses agreement, etc.)\\
    \textit{a!}   \>   ‘ah!’ (expresses shock or disbelief)\footnotemark{}\\
    \textit{aː}    \>  ‘uh …’ (filler \isi{interjection})\\
    \textit{a?}  \>    ‘eh?’ (\isi{tag question} \isi{interjection}) (also \textit{e?})\\
    \textit{ay}   \>   ‘ow’ (expresses pain or shock)\\
    \textit{aya}  \>    ‘ah me’ (expresses compassion)\\
    \textit{e!}   \>   ‘hey!’ (expresses excitement, either positive or negative)\\
    \textit{e?}  \>    ‘eh?’ (\isi{tag question} \isi{interjection}) (also \textit{a?})\\
    \textit{i}   \>   ‘alas; yay!’ (expresses dejection or joy)\\
    \textit{u}    \>  ‘ooh’ (expresses amazement)\\
    \textit{m!}   \>   ‘hm!’ (expressed disapproval)\\
    \textit{m}   \>   ‘mhm’ (signals agreement)\\
    \textit{mm}   \>   ‘uh-uh’ (signals disagreement)\\
    \textit{=o}  \>    ‘hey!’ (\isi{vocative} form / \isi{intensifier}, as \isi{enclitic})\\
    \textit{mawnam} \> ‘that’s it!’
\end{tabbing}
\z
\footnotetext[8]{This \isi{interjection} is often used to introduce quoted \isi{speech} (\sectref{sec:13.4.4}).}

  The \isi{interjection} \textit{=o} ‘\textsc{voc}’ is possibly a \isi{loan} from \ili{Tok Pisin}.\footnote{Ulwa has no native words that begin with \isi{mid vowel}s /e/ or /o/. If not a \isi{loan} from \ili{Tok Pisin}, then the form [=o] may at least be influenced by \ili{Tok Pisin} pronunciation. This \isi{vocative} \isi{interjection} has an \isi{allomorph} \textit{=wo} ‘\textsc{voc}’, which may represent an indigenous \isi{interjection}.} Examples \REF{ex:otherwc:150} through \REF{ex:otherwc:154} illustrate the use of \textit{=o} ‘\textsc{voc’}, both as an \isi{interjection} of emphasis and as a \isi{vocative} form used when calling to people.

\ea%150
    \label{ex:otherwc:150}
          \textit{Mawanat \textbf{Supamo}!}\\
\gll    ma=wana-ta    Supam=\textbf{o}\\
    3\textsc{sg.obj}=feel-say  [name]=\textsc{voc}\\
\glt `[They] called to her: “Supam!”’ [ulwa001\_07:52]
\z

\ea%151
    \label{ex:otherwc:151}
          \textit{Ndï ndïwanate wot \textbf{alo}!}\\
\gll    ndï  ndï=wana-ta-e    wot    ala{}=\textbf{o}\\
    \textsc{3pl}  3\textsc{pl}=feel-say-\textsc{dep}  younger  \textsc{pl.dist{}=voc}\\
\glt `They called to them: “Younger brothers!”’ [ulwa002\_05:05]
\z

\ea%152
    \label{ex:otherwc:152}
          \textit{\textbf{Alo} un \textbf{ino}!}\\
\gll    ala{}=\textbf{o}      un  i-n=\textbf{o}\\
    \textsc{pl.dist{}=voc}  \textsc{2pl} come-\textsc{irr=voc}\\
\glt `You all, come!’ [ulwa013\_03:45]
\z

\ea%153
    \label{ex:otherwc:153}
          \textit{\textbf{Tembiwo}!}\\
\gll    tembi=\textbf{o}\\
    bad=\textsc{voc}\\
\glt `It’s bad!’ [ulwa037\_26:53]
\z

\ea%154
    \label{ex:otherwc:154}
          \textit{Alanji amba \textbf{ndo}!}\\
\gll    ala-nji      amba      anda{}=\textbf{o}\\
    \textsc{pl.dist{}-poss} mens.house  \textsc{sg.dist{}=voc}\\
\glt `Over there they have magic!’ [ulwa037\_21:10]
\z

The \isi{interjection} \textit{mawnam} ‘that’s it’ is used to signal the \isi{emphatic} identification of a referent or to show approval of a thought or action.\footnote{Cf. \ili{Tok Pisin} \textit{em nau} ‘it now’, used as an \isi{interjection} to mean ‘that’s it’.} The word may contain the word \textit{maw} ‘correct’, although I consider it also possible that this shorter word is itself a \isi{backformation} from \textit{mawnam} ‘that’s it’. The [-nam] ending suggests a pronominal origin of this word (cf. the \isi{emphatic} \isi{suffix} \textit{-nam} ‘\textsc{emph}’, \sectref{sec:6.7}). Perhaps it is an elaboration of the form \textit{mï-nam} ‘3\textsc{sg.subj}-\textsc{emph}’. The form \textit{mawnam} ‘that’s it’ may take the ending [-e]. It is unclear whether this is the \isi{dependent marker} \textit{-e} ‘\textsc{dep}’ (\sectref{sec:12.2.1}) or simply a further \isi{emphatic} \isi{syllable}. Sentences \REF{ex:otherwc:155} through \REF{ex:otherwc:158} exemplify the use of \textit{mawnam} ‘that’s it’.

\ea%155
    \label{ex:otherwc:155}
          \textit{Makape i} \textbf{\textit{mawnam}}.\\
\gll maka=p-e    i    \textbf{mawnam}\\
    thus=\textsc{cop-dep}  way  thats.it\\
\glt `Behavior like that -- that’s it.’ [ulwa037\_31:35]
\z

\ea%156
    \label{ex:otherwc:156}
          \textbf{\textit{Mawname}} \textit{mï kalam.}\\
\gll    \textbf{mawnam-e}  mï      kalam\\
    thats.it-\textsc{dep?}  \textsc{3sg.subj}  knowledge\\
\glt `That’s it, he knows.’ [ulwa014\_07:56]
\z

\ea%157
    \label{ex:otherwc:157}
          \textbf{\textit{Mawnam}}.\\
\gll mawnam\\
    thats.it\\
\glt `That’s it.’ [ulwa011\_03:15]
\z

\ea%158
    \label{ex:otherwc:158}
          \textbf{\textit{Mawname}}.\\
\gll mawnam-e\\
    thats.it-\textsc{dep?}\\
\glt `That’s right.’ [ulwa037\_33:13]
\z

\is{interjection|)}
