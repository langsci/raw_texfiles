\chapter{Topics in semantics}\label{sec:14}

\is{semantics|(}

In this chapter I describe a few topics in \isi{lexical semantics} in order to facilitate a clearer understanding of the Ulwa language in general, as well as to provide data that may be of use for crosslinguistic typological comparisons.

\is{semantics|)}

\section{Polysemy and homonymy}\label{sec:14.1}

\is{semantics|(}
\is{polysemy|(}
\is{homonymy|(}
\is{homophony|(}
\is{colexification|(}

Perhaps unsurprisingly for a language with a relatively small \isi{phoneme inventory} and many \isi{monosyllabic} and \isi{disyllabic} morphemes and lexemes, the Ulwa \isi{lexicon} has many sets of same-sounding forms that have different meanings (i.e., \isi{colexification}s). It is not always possible to determine whether these pairs represent different meanings of a single word (\isi{polysemy}) or are truly separate words that, due to historical accident, share the same \isi{phonological} form (\isi{homonymy}).

  Given Ulwa’s \isi{phonotactic} constraints, the three \isi{phonological}ly shortest possible words are [i], [u], and [a]; no other \isi{vowel}s are permitted word-initially (\sectref{sec:2.2.1}), and a word requires at least one vowel to be pronounceable. Indeed, there is considerable homophony among morphemes of the form [i] \REF{ex:sem:1a} and [u] \REF{ex:sem:1b}, as well as some homophony among morphemes of the form [a] \REF{ex:sem:1c}.
  
\ea%1a
    \label{ex:sem:1a}
            Homophony: morphemes of the form [i]
\is{suppletion}
\is{perfective}
\is{interjection}
\is{predicate marker}
\il{Tok Pisin}
\is{loan}
\begin{tabbing}
    {(i-)} \= {(verb, ‘go’ (suppletive perfective form of \textit{ma-} ‘go’))}\kill
{\textit{i}} \> {noun, ‘behavior, habit, custom, way’}\\
{\textit{i}} \> {noun, ‘hand, arm’}\\
{\textit{i}} \> {noun, ‘lime (calcium hydroxide)’}\\
{\textit{i}} \> {verb, ‘go’ (suppletive perfective form of \textit{ma-} ‘go’)}\\
{\textit{i}} \> {interjection expressing dejection (‘alas’)}\\
{\textit{i}} \> {predicate marker (Tok Pisin loan)}\\
{\textit{i-}} \> {verb, ‘come’}
\end{tabbing}
\z

\ea%1b
    \label{ex:sem:1b}
            Homophony: morphemes of the form [u]
\is{pronoun}
\is{postposition}
\is{interjection}
\begin{tabbing}
{(u-)} \= {(postposition, ‘from, in, at, around, along’)}\kill
{\textit{u}} \> {noun, ‘ditch, creek’}\\
{\textit{u}} \> {pronoun, ‘you’ (\textsc{2sg})}\\
{\textit{u}} \> {postposition, ‘from, in, at, around, along’}\\
{\textit{u}} \> {interjection expressing amazement (‘ooh’)}\\
{\textit{u-}} \> {verb, ‘put’}
\end{tabbing}
\z

\ea%1c
    \label{ex:sem:1c}
            Homophony: morphemes of the form [a]
\is{tag question}
\is{interjection}
\begin{tabbing}
{(a-)} \= {(interjection for tag questions (‘eh?’))}\kill
{\textit{a}} \> {interjection expressing shock (‘ah!’)}\\
{\textit{a}} \> {filler interjection (‘uh …’)}\\
{\textit{a}} \> {interjection for tag questions (‘eh?’)}\\
{\textit{a-}} \> {verb, ‘break’}
\end{tabbing}
\z

Although not pronounceable on their own, single-\isi{consonant} morphemes are possible (in some instances only as abbreviations of other forms). Examples of homophony can be seen in morphemes with the forms [n] \REF{ex:sem:2a}, [p] \REF{ex:sem:2b}, and [t] \REF{ex:sem:2c}.

\ea%2a
    \label{ex:sem:2a}
            Homophony: morphemes of the form [n]
\is{epenthetic}
\is{TAM}
\is{suffix}
\is{imperative}
\is{imperfective}
\is{perfective}
\is{nominalizer}
\is{allomorph}
\is{oblique marker}
\begin{tabbing}
{(=n)} \= {(epenthetic utterance-final sound for some speakers)}\kill
{\textit{n}} \> {epenthetic utterance-final sound for some speakers}\\
{\textit{-n}} \> {TAM suffix: imperative, ‘\textsc{imp}’}\\
{\textit{-n}} \> {irregular TAM suffix: imperfective, ‘\textsc{ipfv}’}\\
{\textit{-n}} \> {irregular TAM suffix: perfective, ‘\textsc{pfv}’}\\
{\textit{-n}} \> {nominalizer, allomorph of \textit{-en} ‘\textsc{nmlz’}}\\
{\textit{=n}} \> {oblique marker, ‘\textsc{obl}’}
\end{tabbing}
\z

\ea%2b
    \label{ex:sem:2b}
            Homophony: morphemes of the form [p]
\is{epenthetic}
\is{TAM}
\is{suffix}
\is{perfective}
\is{copular enclitic}
\begin{tabbing}
{(=p)} \= {(epenthetic utterance-final sound for some speakers)}\kill
{\textit{p}} \> {epenthetic utterance-final sound for some speakers}\\
{\textit{p-}} \> {verb, ‘be, be at’}\\
{\textit{-p}} \> {TAM suffix: perfective, ‘\textsc{pfv}’}\\
{\textit{=p}} \> {copular enclitic, ‘\textsc{cop}’}
\end{tabbing}
\z

\ea%2c
    \label{ex:sem:2c}
            Homophony: morphemes of the form [t]
\is{speculative}
\is{suffix}
\begin{tabbing}
{(-t)} \= {(verb, ‘take’, abbreviated form of \textit{tï-} ‘take’)}\kill
{\textit{t}} \> {verb, ‘say’, abbreviated form of \textit{ta-} ‘say’}\\
{\textit{t}} \> {verb, ‘take’, abbreviated form of \textit{tï-} ‘take’}\\
{\textit{-t}} \> {speculative suffix, ‘\textsc{spec}’}
\end{tabbing}
\z

Examples of homophony with slightly longer \isi{phonological} forms can be found among forms such as [ala] \REF{ex:sem:2d}, [ka] \REF{ex:sem:2e}, [ma] \REF{ex:sem:2f}, and [na] \REF{ex:sem:2g}.

\ea%2d
    \label{ex:sem:2d}
           Homophony: morphemes of the form [ala]
\is{demonstrative}
\is{postposition}
\is{determiner}
\begin{tabbing}
{(ala-)} \= {(demonstrative determiner, ‘those’)}\kill
{\textit{ala}} \> {demonstrative determiner, ‘those’}\\
{\textit{ala}} \> {postposition, ‘for, from’}\\
{\textit{ala-}} \> {verb, ‘see’}
\end{tabbing}
\z

\ea%2e
    \label{ex:sem:2e}
           Homophony: morphemes of the form [ka]
\is{postposition}
\is{adverb}
\begin{tabbing}
{(\textit{ka-})} \= {(noun, ‘peak’ (as in \textit{apaka} ‘roof’, literally ‘house peak’))}\kill 
{\textit{ka}} \> {noun, ‘peak’ (as in \textit{apaka} ‘roof’, literally ‘house peak’)}\\
{\textit{ka}} \> {postposition, ‘at, in, on’}\\
{\textit{ka}} \> {adverb, ‘thus, in this manner, in that manner’}\\
{\textit{ka-}} \> {verb, ‘let, leave, allow’}
\end{tabbing}
\z

\ea%2f
    \label{ex:sem:2f}
           Homophony: morphemes of the form [ma]
\is{possessive pronoun}
\is{coordinator}
\is{object marker}
\begin{tabbing}
{(ma=)} \= {(possessive pronoun} (3\textsc{sg}), equivalent to \textit{manji} ‘3\textsc{sg.obj-poss’)}\kill
{\textit{ma}} \> {possessive pronoun (3\textsc{sg}), equivalent to \textit{manji} ‘3\textsc{sg.obj-poss’}}\\
{\textit{ma}} \> {coordinator, ‘and’ (perhaps a recent innovation)}\\
{\textit{ma-}} \>  {verb, ‘go’}\\
{\textit{ma=}} \>  {object marker (3\textsc{sg.obj)}}
\end{tabbing}
\z

\ea%2g
    \label{ex:sem:2g}
           Homophony: morphemes of the form [na]
\is{coordinator}
\is{loan}
\il{Tok Pisin}
\is{detransitivizing}
\is{prefix}
\is{TAM}
\is{suffix}
\is{irrealis}
\begin{tabbing}
{(na-)} \= {(noun, ‘talk, speech, story, message, thought, reason, etc.’)}\kill
{\textit{na}} \> {noun, ‘talk, speech, story, message, thought, reason, etc.’}\\
{\textit{na}} \> {coordinator, ‘and’ (Tok Pisin loan)}\\
{\textit{na-}} \> {verb, ‘give’}\\
{\textit{na-}} \> {detransitivizing prefix, ‘\textsc{detr}’}\\
{\textit{-na}} \> {TAM suffix: irrealis, ‘\textsc{irr}’}
\end{tabbing}
\z

Thus ample \isi{colexification} can be found among \isi{function word}s and \isi{basic vocabulary} items. The pairs of identical forms given in \REF{ex:sem:3} represent colexification between functional morphemes and concrete nouns. These are almost certainly all examples of true homonymy, as opposed to polysemy.

\ea%3
    \label{ex:sem:3}
            Additional pairs of homonyms
\begin{tabbing}
{(\textit{ambla})} \= {(‘come [\textsc{irr]}’)} \= {(vs.)} \= {(\textit{ambla})} \= {(armband)}\kill
{\textit{ambla}} \> {‘\textsc{pl.refl’}} \> {vs.} \> {\textit{ambla}} \> {‘tooth’}\\
{\textit{ina}} \> {‘come [\textsc{irr]}’} \> {vs.} \> {\textit{ina}} \> {‘liver’}\\
{\textit{mana}} \> {‘go [\textsc{irr]’}} \> {vs.} \> {\textit{mana}} \> {‘spear’}\\
{\textit{min}} \> {‘3\textsc{du’}} \> {vs.} \> {\textit{min}} \> {‘armband’}\\
{\textit{un}} \> {‘\textsc{2pl’}} \> {vs.} \> {\textit{un}} \> {‘tree species’}
\end{tabbing}
\z

Other pairs of identical forms that are very likely polysemes, as opposed to homonyms, are given in \REF{ex:sem:4}.

\ea%4
    \label{ex:sem:4}
            Likely polysemes
\begin{tabbing}
{(\textit{mbomala})} \= {(‘large firefly’)} \= {(or)} \= {(‘large star (or planet)’)}\kill
{\textit{anga}} \> {‘piece’} \> {or} \> {‘side’}\\
{\textit{apïn}} \> {‘fire’} \> {or} \> {‘pain’}\\
{\textit{mu}} \> {‘seed’} \> {or} \> {‘fruit (or nut)’\footnotemark{}}\\
{\textit{mbomala}} \> {‘large firefly’} \> {or} \> {‘large star (or planet)’}\\
{\textit{nali}} \> {‘small firefly’} \> {or} \> {‘small star’\footnotemark{}}
\end{tabbing}
\z
\footnotetext[1]{The meaning ‘kidney’ is probably derived metaphorically; the meaning ‘blowfly’ may be unrelated though.}
\footnotetext[2]{The meanings ‘spine of a sago frond’ and ‘ten’ are likely related to each other (\sectref{sec:7.5}), but they are probably unrelated to these other meanings of [nali].}

Some words in Ulwa have much greater ranges of meaning than any of their possible \ili{English} equivalents. While these are not properly polysemes or homonyms, it may nevertheless prove useful to provide a few examples of these words \REF{ex:sem:5}.

\ea%5
    \label{ex:sem:5}
            Words with broad meaning
\begin{tabbing}
{(\textit{akïnaka})} \= {(‘talk, speech, story, message, thought, reason, language’)}\kill
{\textit{akïnaka}} \> {‘new, fresh, alive, raw, young’}\\
{\textit{anma}} \> {‘good, nice, true, smart, straight, healthy, well’}\\
{\textit{na}} \> {‘talk, speech, story, message, thought, reason, language’}\\
{\textit{tembi}} \> {‘bad, sick, poor, dirty’}
\end{tabbing}
\z

Often a word derives a new meaning based on a \isi{metaphor}ical or \is{metonymy} metonymic relationship (\sectref{sec:14.2}). Ulwa \isi{coinage}s for foreign concepts that employ \isi{metaphor} or \isi{metonymy} are discussed in \sectref{sec:14.9}. There are also examples of \isi{polysemous} relationships among \isi{color term}s (\sectref{sec:14.5}), \isi{body part term}s (\sectref{sec:14.6}), and terms expressing various \isi{temporal} concepts (\sectref{sec:14.8}).

\is{colexification|)}
\is{homophony|)}
\is{homonymy|)}
\is{polysemy|)}
\is{semantics|)}

\section{Metaphor and metonymy}\label{sec:14.2}

\is{metaphor|(}
\is{metonymy|(}
\is{semantics|(}

Although many metaphors and metonyms have become \isi{fossilized} as the primary term used for certain referents (and thus are perhaps no longer viewed as \isi{semantic extension}s), it is still possible for speakers to employ both \isi{metaphor} and \isi{metonymy} creatively. This may be done even when another word for a referent already exists; however, it is more common as a means of coining terminology for new concepts (\sectref{sec:14.9}). Examples of metaphors are given in \REF{ex:sem:6}.

\ea%6
    \label{ex:sem:6}
            Metaphors\\
\begin{tabbing}
{(\textit{mundotoma})} \= {((< \textit{ya} ‘coconut’ + \textit{iwïl} ‘moon’, the latter ‘round’ like the former))}\kill
{\textit{ana}} \> {‘parasitic person’, literally ‘grass skit’}\\
{ } \> {(an article of clothing that ‘hangs onto’ a person)}\\
{\textit{mundotoma}} \> {‘lacking’, literally ‘short’}\\
{ } \> {(same \isi{metaphor} as in \ili{English}, e.g., ‘in short supply’)}\\
{\textit{unduwan}} \> {‘elder’, literally ‘head’}\\
{ } \> {(the part of the body that comes ‘first’)}\\
{\textit{yawïl}} \> {‘full moon’, literally ‘coconut moon’}\\
{ } \> {(< \textit{ya} ‘coconut’ + \textit{iwïl} ‘moon’, the latter ‘round’ like the former)}
\end{tabbing}
\z

Metonymy is very common in Ulwa. Often the material from which something is made is used to refer to the end product, as in \REF{ex:sem:7}.

\newpage

\ea%7
    \label{ex:sem:7}
            Material-based metonyms
\begin{tabbing}
{(\textit{numbu})} \= {(‘\textit{garamut} drum’)} \= {((made from \textit{numbu} ‘ironwood tree’))}\kill
\textit{asiya} \> {‘animal trap’}  \> {(made with \textit{asiya} ‘string’)}\\
\textit{numbu} \> {‘\textit{garamut} drum’} \> {(made from \textit{numbu} ‘ironwood tree’)}\\
\textit{we} \> {‘sago pancake’} \> {(made from \textit{we} ‘sago starch’)}
\end{tabbing}
  \z

Other forms of \isi{metonymy} are possible as well, such as \isi{synecdoche}, in which either the part comes to represent the whole (\isi{pars pro toto}, as with \textit{isi} ‘soup’) or the whole comes to represent the part (\isi{totum pro parte}, as with \textit{ulum} ‘sago pith’) \REF{ex:sem:8}.

\ea%8
    \label{ex:sem:8}
            Synecdoche
\begin{tabbing}
{(\textit{ulum})} \= {(‘sago pith’)} \= {((the soft, white insides of the \textit{ulum} ‘sago palm’))}\kill
{\textit{isi}} \> {‘soup’} \> {(typically containing \textit{isi} ‘salt’)}\\
{\textit{ulum}} \> {‘sago pith’} \> {(the soft, white insides of the \textit{ulum} ‘sago palm’)}
\end{tabbing}
 \z

Other forms of \isi{metonymy} are used as well, including those in \REF{ex:sem:9}.

\ea%9
    \label{ex:sem:9}
            Other metonyms
\begin{tabbing}
{(\textit{nambana})} \= {(‘menstruation’)} \= {(<)} \= {(\textit{nambana} {‘ancestral spirit’})}\kill
{\textit{iwïl}} \> {‘menstruation’} \> {<} \> {\textit{iwïl} ‘moon’}\\
{\textit{nambana}}\> {‘mask’} \> {<} \> {\textit{nambana} ‘ancestral spirit’}\\
{\textit{yopa}} \> {‘peace’} \> {<} \> {\textit{yopa} ‘cockatoo’}
 \end{tabbing}
\z

The connection between \textit{iwïl} ‘menstruation’ and \textit{iwïl} ‘moon’ is due to their comparble cycles (reflected in \ili{English} etymology as well). As a further extension of meaning, \textit{iwïl} ‘moon’ can also be used euphemistically to refer to female genitalia, otherwise called \textit{inmbï} ‘vulva’. The word \textit{nambana} ‘ancestral spirit’ can refer to a mask, since these are often used to represent the faces of ancestral spirits. The word \textit{yopa} ‘cockatoo’ can mean ‘peace’, since peace is customarily signaled by painting oneself white to resemble a cockatoo.

\is{semantics|)}
\is{metonymy|)}
\is{metaphor|)}

\section{Onomatopoeia}\label{sec:14.3}

\is{onomatopoeia|(}
\is{sound symbolism|(}
\is{semantics|(}

Some words in Ulwa likely derive from \isi{sound symbolism}. However, there is no special class of \isi{ideophone}s in the language (i.e., there is no \isi{morphosyntactic}ally definable class of words that evoke sounds). A number of \isi{onomatopoetic} words can be found in fauna terms, in most cases the word being derived from the sound the particular animal makes. Frog \REF{ex:sem:10} and bird \REF{ex:sem:10a} species in particular seem to lend themselves to \isi{onomatopoeia}.

\newpage

\ea%10
    \label{ex:sem:10}
          Onomatopoeia in names for frogs
\begin{tabbing}
{(\textit{wandïwandï})} \= {(‘very small frog that lives on leaves’)}\kill
{\textit{kïlakïli}} \> {‘very small frog that lives on leaves’}\\
{\textit{popotala}} \> {‘large brown frog’}\\
{\textit{wali}} \> {‘small green, yellow, or brown frog’}\\
{\textit{wandïwandï}} \> {‘small brown frog’}
\end{tabbing}
\z

\ea%10a
    \label{ex:sem:10a}
          Onomatopoeia in names for birds
\begin{tabbing}
{(\textit{kukumali})} \= {(‘red or green parrot (TP \textit{kalangal})’)}\kill
{\textit{awalawa}} \> {‘red or green parrot (TP \textit{kalangal})’}\\
{\textit{kokawe}} \> {‘bird species’}\\
{\textit{kukumali}} \> {‘bird species’}\\
{\textit{kulkul}} \> {‘bird species’}\\
{\textit{maep}} \> {‘bird species’}\\
{\textit{wotnya}} \> {‘type of black bird’}
\end{tabbing}
\z

As can be seen in \REF{ex:sem:10} and \REF{ex:sem:10a}, such \isi{onomatopoetic} words often involve \isi{reduplication} (\sectref{sec:3.4}).

\is{semantics|)}
\is{sound symbolism|)}
\is{onomatopoeia|)}

\section{Formulaic expressions, greetings, and farewells}\label{sec:14.4}

\is{formulaic expression|(}
\is{semantics|(}

In Ulwa, as in \ili{Tok Pisin} and many other languages of the Pacific, it is common to \is{greeting} greet people with descriptions of what they are doing (e.g., ‘you are bathing’, ‘you are chopping wood’, etc.) or questions regarding what they have just done or are about to do (e.g., ‘where were you?’, ‘where are you going?’, etc.). It is not common, as in some European languages, to inquire into one’s physical or emotional state. Traditional Ulwa \isi{greeting}s include, for example, those in \REF{ex:sem:11}.

\ea%11
    \label{ex:sem:11}
          Traditional \isi{greeting}s
\begin{tabbing}
{(\textit{U ango mana?})} \= {(‘Where are you going?’)}\kill
{\textit{Inim lope.}} \> {‘[You] are bathing.’}\\
{\textit{U ango mana?}} \> {‘Where are you going?’}
\end{tabbing}
\z

In addition, there is a set of formulae used to greet people at various times of the day. They are all formed with the \isi{adjective} \textit{anma} ‘good’. I suspect that they are \isi{calque}s from \ili{Tok Pisin}, which, like \ili{English}, employs \isi{greeting}s built from the \isi{adjective} ‘good’ (\ili{Tok Pisin} \textit{gut} {\textasciitilde} \textit{gutpela} ‘good’) and the time of day. The Ulwa time-of-day \isi{greeting}s are given in \REF{ex:sem:12}.

\ea%12
    \label{ex:sem:12}
          Time-of-day \isi{greeting}s
\begin{tabbing}
{(\textit{Umbenam anma!})} \= {(‘Good day!’ (literally ‘good sun’))}\kill
{\textit{Umbenam anma!}} \> {‘Good morning!’}\\
{\textit{Ane anma!}} \> {‘Good day!’ (literally ‘good sun’)}\\
{\textit{Awal anma!}} \> {‘Good afternoon!’\footnotemark{}}\\
{\textit{Imba anma!}} \> {‘Good evening!; Good night!’\footnotemark{}}
\end{tabbing}
\z
\footnotetext[3]{There is also a longer form: \textit{Awal nambï anma!} (literally ‘good body of the afternoon’).}
\footnotetext[4]{The formulaic time-of-day \isi{greeting}s may also be used as \isi{farewell}s, especially at nighttime (i.e., \textit{imba anma} ‘good night’ can be used either to greet people or to bid them \isi{farewell}).}

\isi{Farewell}s in Ulwa are typically proclamations that one is leaving or \isi{command}s for the other party to go (or to stay). These, too, parallel traditional \ili{Tok Pisin} valedictions. Examples are presented in \REF{ex:sem:13}.

\ea%13
    \label{ex:sem:13}
          \isi{Farewell}s
\begin{tabbing}
{(\textit{Un mbïpïna!})} \= {(‘Stay here!’ (addressed to multiple people))}\kill
{\textit{An mana!}} \> {‘We (1\textsc{pl.excl}) shall go!’}\\
{\textit{Un mbïpïna!}} \> {‘Stay here!’ (addressed to multiple people)}\\
{\textit{U mana!}} \> {‘Go!’ (addressed to one person)}\\
{\textit{Namanu!}} \> {‘Goodbye!’ (addressed to someone leaving)\footnotemark{}}
\end{tabbing}
\z
\footnotetext[5]{The form \textit{namanu} ‘goodbye’ has taken on a formulaic usage, although it seems derived from a \isi{command}: \textit{na-} ‘\textsc{detr’} + \textit{ma-} ‘go’ + \textit{n-} ‘\textsc{imp’} + \textit{u} ‘2\textsc{sg}’ (or \textit{{}=o} ‘\textsc{voc’}).}

  Some \isi{polite} formulaic expressions that are common among European languages like \ili{English} (e.g., ‘please’, ‘thank you’, etc.) do not have direct equivalents in Ulwa. It is common, for example, for an Ulwa speaker not to say anything when receiving something from another person.
  
  To express strong gratitude, however, one may say \textit{nïnji anma} ‘my good’, which is akin to \ili{English} ‘thank you’. As a \isi{response} to this, one might say \textit{u anma} ‘you [are] good’, which is akin to ‘you’re welcome’. However, there is probably no tradition of formulaic exchanges of such sayings.

  To make a \isi{polite} \isi{request}, the \isi{modal adverb} \textit{kop} ‘please’ may be used along with an \isi{imperative}, somewhat like the use of \ili{English} ‘please’ (\sectref{sec:13.2.2}).

\is{semantics|)}
\is{formulaic expression|)}

\section{Color terms}\label{sec:14.5}

\is{color term|(}
\is{semantics|(}

Color terms occur very infrequently in the Ulwa corpus. Given the paucity of relevant data and the variability in interpretation of the term “basic”, it is not possible to place Ulwa with perfect certainty within \citegen{BerlinKay1969} \isi{hierarchy} of stages of \isi{basic color term}s. That said, Ulwa seems to employ very few \isi{basic color term}s, and the terms that it does employ -- including those for ‘white’ and ‘black’ -- seem to be either derived or \isi{borrow}ed.

Terms for colors in Ulwa are given in \REF{ex:sem:14}. Some of these words have been obtained through elicitation alone, either by asking speakers to generate lists of color terms or by obtaining translations of \ili{Tok Pisin} color terms; these are thus perhaps more suspect and are therefore identified in \REF{ex:sem:14} as “[elicited]”.

\newpage

\ea%14
    \label{ex:sem:14}
          Color terms\\
    \begin{tabbing}
    {(\textit{mbunmana})} \= {(‘yellow, light’)}\kill
    \textit{waembïl}  \>  ‘white’\\
    \textit{mbun}   \>   ‘black, blue, dark’\\
    \textit{mbunmana} \> ‘black’\\
    \textit{ngungun}  \>  ‘red’\\
    \textit{anem}  \>  ‘blue, purple’ [elicited]\\
    \textit{ane}    \>    ‘yellow, light’\\
    \textit{anembal}  \>  ‘light’\\
    \textit{andwana}  \>  ‘yellow’\\
    \textit{mïndit}  \>  ‘yellow’ [elicited]\\
    \textit{mïnal}  \>  ‘green’ [elicited]\\
    \textit{tondiway} \> ‘orange’ [elicited]\\
    \textit{lemetam} \>   ‘brown’ [elicited]
    \end{tabbing}
\z

Many of these are obviously derived from other words, typically nouns that refer to entities that exhibit the relevant color. The word \textit{ngungun} ‘red’, for example, also refers to a species of red ant.\footnote{This ant species is used as a traditional medicine, boiled in a solution to treat coughs. The same word also refers to a species of plant with red seeds. Yet another meaning of the form [ngungun], ‘whirlwind, cyclone’, is not clearly connected to the color red and may simply be \isi{homophonous}.} The word \textit{mïnal} ‘green’ also means ‘taro’, a plant whose leaves are boiled to make a soup of very saliently green color. Similarly, \textit{tondiway} ‘orange’ has as its more basic meaning a plant species with orange seeds used to make dyes. The word \textit{lemetam} ‘brown’ also refers to a large hardwood tree, whose brown bark is used to bandage wounds. The color word \textit{ane} ‘yellow, light’ also means ‘sun’.\footnote{The word \textit{anembal} ‘light’ clearly contains \textit{ane} ‘sun’ as well, but the form [mbal] is an obscure element; it possibly underlies the form \textit{waembïl} ‘white’ as well.} The word \textit{anem} ‘blue, purple’ is also the name of a yam variety with purple flesh, as well as a necklace bead made from a blue seed.

  Some of the color terms given in \REF{ex:sem:14}, although not completely \isi{homonymous} with other forms, bear very strong resemblances to nominals associated with those colors. Thus, \textit{andwana} ‘yellow’ may be related to \textit{anduwan} ‘young sago palm’ and \textit{mïndit} ‘yellow’ may be related to \textit{mïnda} ‘banana’. The form \textit{mbunmana} ‘black’ seems to have derived from \textit{mbun} ‘black, blue, dark’, but exactly how this has occurred (or why) is unclear.\footnote{Perhaps \textit{mbunmana} ‘black’ derives from \textit{mbun} ‘dark’ plus \textit{mana} ‘go [\textsc{irr]}’ (i.e., ‘going dark’), but this is speculative. The meaning ‘scar’ that belongs to the form \textit{mbun} ‘black, blue, dark’, is, however, more likely derived from the \isi{color term} than vice versa (if, of course, this is not just a matter of \isi{homophony}). In any case, \textit{mbun} ‘black, blue, dark’ itself was likely \isi{borrow}ed from \ili{Mwakai} (\sectref{sec:1.5.6}).} For a possible etymology of \textit{waembïl} ‘white’, which contains the unusual \is{low vowel} low \isi{front vowel} [ae], see \sectref{sec:2.2}.\largerpage

\is{semantics|)}
\is{color term|)}


\section{Body part terms}\label{sec:14.6}

\is{body part term|(}
\is{semantics|(}

In this section I discuss terminology for the parts of the body, a domain that is often of interest to semantic typologists, anthropologists, and others.

  First, several body-part concepts that may be encoded with distinct lexemes in some languages are \isi{colexified} in Ulwa. Indeed, the term for ‘body’ itself is \isi{colexified} with the word for ‘skin’. Thus it seems the word \textit{nambï} ‘skin’ has extended in meaning to include everything encased in the skin, a \isi{semantic extension} that seems to be common for the area.

Likewise, Ulwa makes no \isi{lexical} differentiation between ‘hand’ and ‘arm’, relying rather on \textit{i} ‘hand, arm’ for either meaning. Similarly, the word \textit{wutï} ‘leg, foot’ refers to a part of the body that could be translated as either ‘leg’ or ‘foot’ in \ili{English}.

  While neither ‘leg’ nor ‘foot’ is taken to be a more basic meaning for \textit{wutï} ‘leg, foot’ (and neither ‘hand’ nor ‘arm’ is taken to be a more basic meaning for \textit{i} ‘hand, arm’), the word \textit{monombam} ‘forehead, face’, which can mean either ‘face’ or ‘forehead’, is assumed to have ‘forehead’ as its primary meaning, based on the typologically common \isi{semantic change} of deriving a term for ‘face’ from a term referring to one particular part of the face, very often from ‘forehead’. See \sectref{sec:14.1} on \isi{pars pro toto} \isi{synecdoche}.

  Conversely, there are distinctions that are made in Ulwa that are not commonly made in \ili{English}. For example, there is no general term to cover ‘hair’ in Ulwa: the word \textit{wonmi} ‘head hair’ refers only to the hair on the top of the head, whereas \textit{nil} ‘body hair’ refers to hair everywhere else on the body, including facial hair.

  Also, as is attested in many languages, body part terms may be used \isi{metaphor}ically in Ulwa, often to express spatial reference, as in \REF{ex:sem:15}.

\ea%15
    \label{ex:sem:15}
          Metaphorical extensions of body part terms
\begin{tabbing}
{(\textit{unmbï})} \= {(‘the side of’)} \= {((literally ‘shoulder’))}\kill
{\textit{awi}} \> {‘the side of’} \> {(literally ‘shoulder’)}\\
{\textit{ip}} \> {‘front’} \> {(literally ‘nose’)}\\
{\textit{unmbï}} \> {‘back’} \> {(literally ‘buttocks’)}
\end{tabbing}
\z

The spatial \isi{metaphor} of ‘nose’ to mean ‘front’ has been extended to a \isi{temporal} \isi{metaphor} to mean ‘earlier, former’, as in \REF{ex:sem:16}.

\ea%16
    \label{ex:sem:16}
          \textit{Mat \textbf{ip} ul manata …}\\
\gll    ma=tï      \textbf{ip}    ul  ma=na-ta\\
    3\textsc{sg.obj}=take  nose  with  3\textsc{sg.obj}=give-\textsc{cond}\\
\glt `If [we] bring it first …’ (Literally ‘take it and give with nose’) [ulwa037\_65:32]
\z

Indeed, the \isi{postposition}/\isi{adverb} \textit{ipka} ‘before, earlier, first’ is transparently derived from the noun \textit{ip} ‘nose’ plus the formative/\isi{postposition} \textit{ka} ‘thus, in this manner, in that manner; at, in, on’.

  There are a number of \isi{idiom}s based on body part terms, two of which are given in \REF{ex:sem:17}.

\ea%17
    \label{ex:sem:17}
          \isi{Idiom}s based on body part terms
\begin{tabbing}
{(\textit{tï ip lï-})} \= {(‘be strong’)} \= {((literally ‘take and put nose to’))}\kill
{\textit{uma tï-}} \> {‘be strong’} \> {(literally ‘take bone’)}\\
{\textit{tï ip lï-}} \> {‘destroy’} \> {(literally ‘take and put nose to’)}
\end{tabbing}
\z

In other cases, a word whose primary meaning does not relate to the human body may be used \isi{metaphor}ically to refer to a body part, as in \REF{ex:sem:18}.

\ea%18
    \label{ex:sem:18}
          Words used metaphorically to refer to body parts
\begin{tabbing}
{(\textit{tïmbïl})} \= {(‘diaphragm’)} \= {((literally ‘bamboo species’, cf. also \textit{aninokam} ‘throat’))}\kill
{\textit{mïtïn}} \> {‘testicle’} \> {(literally ‘egg’)}\\
{\textit{mota}} \> {‘throat’} \> {(literally ‘bamboo species’, cf. also \textit{aninokam} ‘throat’)}\\
{\textit{mu}} \> {‘kidney’} \> {(literally ‘fruit’)}\\
{\textit{tïmbïl}} \> {‘diaphragm’} \> {(literally ‘fence’)}
\end{tabbing}
\z

It seems that the \isi{metaphor}ical use of \textit{mïtïn} ‘egg’ to refer to testicles has\is{pejoration} pejorated the word in all its senses. Many speakers thus avoid using \textit{mïtïn} ‘egg’ when referring to actual fowl or reptile eggs, instead using \textit{yokomtïn} ‘egg’ for all types of eggs, regardless of species, as a means of \isi{taboo} avoidance.\footnote{The word \textit{yokomtïn} ‘egg’ seems to derive from \textit{yokomakan} ‘small wild fowl’ plus \textit{mïtïn} ‘egg’. It thus probably originally referred specifically to wild fowl eggs, but it has been extended in meaning to refer to the eggs of any animal.}

  The word \textit{imu} ‘finger’ is a \isi{compound}, consisting of \textit{i} ‘hand’ and a \isi{metaphor}ical use of \textit{mu} ‘fruit’ (literally ‘fruit of the hand’). The individual fingers have mostly \isi{metaphor}ically derived names as well. They are given in \REF{ex:sem:19}.

\ea%19
    \label{ex:sem:19}
          Names of the fingers
\begin{tabbing}
{(\textit{imu watangïn})} \= {(‘index finger, pointer finger’)} \= {(<)} \= {(\textit{ankam} ‘person’)}\kill
{\textit{imu unduwan}} \> {‘thumb’} \> {<} \> {\textit{unduwan} ‘head’}\\
{\textit{imu ankam}} \> {‘index finger, pointer finger’} \> {<} \> {\textit{ankam} ‘person’}\\
{\textit{imu wome}} \> {‘middle finger’} \> {<} \> {\textit{wome} ‘middle’}\\
{\textit{imu law}} \> {‘ring finger’} \> {<} \> {\textit{law} ‘cordyline, ti plant’}\\
{\textit{imu watangïn}} \> {‘pinky finger, little finger’} \> {<} \> {\textit{watangïn} ‘last’}
\end{tabbing}
\z

Similarly, the word \textit{wutïmu} ‘toe’ is literally ‘fruit of the foot’. The individual toes follow a similar naming scheme to that for the individual fingers \REF{ex:sem:20}.

\newpage

\ea%20
    \label{ex:sem:20}
          Names of the toes
\begin{tabbing}
{(\textit{wutïmu watangïn})} \= {(‘pinky toe, little toe’)} \= {(= ‘last fruit of the foot’)}\kill
{\textit{wutïmu unduwan}} \> {‘big toe’} \> {= ‘head fruit of the foot’}\\
{\textit{wutïmu ankam}} \> {‘second toe’} \> {= ‘person fruit of the foot’}\\
{\textit{wutïmu wome}} \> {‘middle toe’} \> {= ‘middle fruit of the foot’}\\
{\textit{wutïmu law}} \> {‘fourth toe’} \> {= ‘cordyline fruit of the foot’}\\
{\textit{wutïmu watangïn}} \> {‘pinky toe, little toe’} \> {= ‘last fruit of the foot’}
\end{tabbing}
\z

A list of some of the most commonly used body part vocabulary is provided in \REF{ex:sem:22}.

\ea%22
    \label{ex:sem:22}
          Body part terms
\begin{tabbing}
    {(\textit{monombam})}   \=   {(‘back of the skull’)} \= {(\textit{tumbunma})}    \=  {(‘back’)}\kill
    \textit{akunpu}   \>   ‘back of the skull’ \> \textit{mutam}    \>  ‘back’\\
    \textit{ambatïm}  \>  ‘joint’ \>       \textit{nambï}  \>    ‘skin, body’\\
    \textit{anangum}  \>  ‘spine’  \>      \textit{nil}  \>    ‘body hair’\\
    \textit{anankïn}  \>  ‘blood’   \>     \textit{nopa}    \>    ‘cheek’\\
    \textit{anen}   \>   ‘fat’   \>     \textit{ngïnïm}  \>    ‘chin’\\
    \textit{aninokam} \>   ‘throat’   \>   \textit{sinanan}  \>    ‘nail’\\
    \textit{atal}   \>   ‘anus’   \>     \textit{tambeta}   \> ‘chest’\\
    \textit{awi}   \>     ‘shoulder’  \>    \textit{tanum} \>   ‘lips’\\
    \textit{i}   \>     ‘hand, arm’  \>   \textit{tumbunma}   \>   ‘nape’\\
    \textit{ina}    \>    ‘liver’  \>      \textit{um}   \> ‘neck’\\
    \textit{inapaw} \>     ‘belly’  \>      \textit{uma}  \>      ‘bone’\\
    \textit{inji}   \>     ‘innards’ \>     \textit{umbopa}  \>    ‘stomach’\\
    \textit{inmbï}  \>    ‘vulva’  \>     \textit{unduwan}  \>  ‘head’\\
    \textit{inpu}   \>   ‘elbow’  \>    \textit{unet}  \>  ‘navel’\\
    \textit{ip}  \>    ‘nose’   \>     \textit{unmbï}   \>   ‘clavicle’\\
    \textit{kïkal}   \>     ‘ear’   \>     \textit{unum}   \>   ‘buttocks’\\
    \textit{limama}  \>  ‘jaw’    \>    \textit{wal}  \>    ‘ribs’\\
    \textit{lïmndï}   \>   ‘eye’  \>      \textit{wanamba}  \>      ‘armpit’\\
    \textit{mama}  \>    ‘mouth’   \>    \textit{wol}  \>  ‘breast’\\
    \textit{mïnandïn}  \>  ‘gallbladder’ \>   \textit{won}   \>     ‘penis’\\
    \textit{mïnane}   \>   ‘intestines’  \>  \textit{wonmi}  \>    ‘head hair’\\
    \textit{mïnïm}   \>   ‘tongue’  \>    \textit{woplota}   \>   ‘lungs’\\
    \textit{mïnopal}   \>  ‘bladder’  \>     \textit{wutï}  \>  ‘leg, foot’\\
    \textit{misam}  \>    ‘brain’   \>     \textit{yom}  \>    ‘heart’\\
    \textit{monombam} \> ‘forehead, face’  \>    { }  \>    { }
    \end{tabbing}
\z

Finally, it is worth noting that the liver has certain importance in Ulwa culture as the seat of emotion and thought. Thus, \textit{ina} ‘liver’ functions much like either ‘heart’ or ‘mind’ in \ili{English}, capable of referring to one’s center of feelings. It forms part of the \isi{compound verb} \textit{inakawana-} ‘think’ (\sectref{sec:9.2.1}), and may also play a part in the etymology of \textit{angwena} ‘why?’ (\sectref{sec:13.1.2}).

  Similarly, the more general term \textit{inji} ‘innards’ (likely derived from \textit{in} ‘in, inside’ plus \textit{nji} ‘thing’), which can refer to the inside of anything, but typically refers to internal organs, can also have a \isi{metaphor}ical sense (cf. \ili{English} \textit{guts}), as in \REF{ex:sem:21}.

  \is{semantics|)}
\is{body part term|)}

\ea%21
    \label{ex:sem:21}
          \textit{Una wa lolop wa \textbf{inji} wopapta, nan wa mbï napïn.}\\
    \gll unan    wa  lolop  wa  \textbf{inji}      wopa=p-ta    unan wa    mbï  na-p-na\\
    1\textsc{pl.incl}  just  just    just  innards    all=\textsc{cop-cond}  \textsc{1pl.incl}    village  here  \textsc{detr}{}-be-\textsc{irr}\\
\glt `If we just have full hearts, then we will stay here [safely] in the village.’ [ulwa037\_30:45]
\z

\section{Kinship terms}\label{sec:14.7}

\is{kinship term|(}
\is{semantics|(}
\is{kinship|(}

The system of \isi{kinship} terminology in Ulwa is fairly \is{classificatory kinship} classificatory (as opposed to \is{descriptive kinship} descriptive) in that a single term may refer to a large number of different types of relatives. It is, however, possible for Ulwa to employ more descriptive terminology by expanding upon the basic system with nominal modifiers. Gender distinctions are found among most of the basic \isi{kinship} terms. When \isi{gender} is not intrinsically encoded in the meaning of a \isi{kinship} term, however, it may be specified by additional modifiers. Some \isi{kinship} terms also indicate relative age, such as \textit{atana} ‘older sister’.

  Relatives of the ego’s parents’ generation can all be referred to as \textit{itom} ‘father’ or \textit{inom} ‘mother’, according to gender. That is, all male siblings of one’s father and mother are \textit{itom} ‘uncle’ (literally ‘father’), and all female siblings of one’s father and mother are \textit{inom} ‘aunt’ (literally ‘mother’). The spouses of one’s parents’ siblings are not seen as familial relations per se; however, in the extended \isi{kinship} system, they can be referred to as ‘father’ and ‘mother’ as well, since they belong to that same generation.

  However, one member of this parents’ generation receives a special designation: the ego’s mother’s brother is called \textit{yawa} ‘maternal uncle’ (cf. \ili{Tok Pisin} \textit{kandere} ‘maternal uncle’). Although it is possible to refer to this relation as \textit{itom} ‘father’, it is more common to use the term \textit{yawa} ‘maternal uncle’. This maternal uncle holds special responsibilities to his sister’s children.

  The ego’s mother’s brother’s wife is known as \textit{ansi inom} ‘\textit{red buai} (betel nut) mother’.\footnote{The word \textit{ansi} ‘\textit{red buai}’ (i.e., the combination of betel nut, betel pepper, and lime) appears in a number of \isi{kinship} terms relating to the \textit{yawa} ‘maternal uncle’, but its exact meaning in these contexts is unclear. In addition to ‘\textit{red buai}’, this word also refers to a gourd-like plant that can be used to store lime and which was previously used to cover the penis. The word may also be used as slang to refer to the penis itself.} The counterpart to the \textit{yawa} ‘maternal uncle’ is the \textit{ansi nungol} ‘sororal nibling’ (the child of a man’s sister).

  The ego’s father’s sister does not have the same status as the mother’s brother; there is, however, a \isi{periphrastic} way of referring to this relation: \textit{ane inom} ‘paternal aunt’ (literally ‘sun mother’). Nor does the ego’s father’s brother have similar responsibilities to his brother’s children. This relation may be referred to with the general term \textit{itom} ‘father’.

  For one’s biological parents, it is common to use the \isi{nursery form}s for direct address -- that is, as \isi{vocative} forms. These are \textit{tata} ‘papa’ and \textit{nana} ‘mama’.

  There are a few different terms to refer to the ego’s children, but the distinctions among them are not clear to me. A child may be called \textit{nungol} ‘child’, \mbox{\textit{nungolke} ‘child’,} \textit{alum} ‘child’, or \textit{tawatïp} ‘child’. Any one of these may refer either to one’s biological child (that is, ‘son’ or ‘daughter’) of any age or to any person of young age (whether related or not). Although none of these terms is strictly limited to a particular gender, \textit{nungol} ‘child’ often implies a male child. There is a gender-specific word \textit{yenat} {\textasciitilde} \textit{yanat} ‘daughter’, which refers to one’s biological daughter or to other females of that generation in the extended \isi{kinship} system. It is clearly related to \textit{yena} {\textasciitilde} \textit{yana} ‘woman, female’.

  When referring to one’s siblings, it is common to make distinctions both based on gender and based on relative age. There is no cover term for ‘sibling’ (of any gender or age), nor is there a cover term either for ‘brother’ or for ‘sister’ (unspecified for relative age). It is, however, possible to refer to younger siblings, regardless of gender, with the word \textit{wot} ‘younger sibling’. This relation may be further specified as \textit{wot yeta} ‘younger brother’ (literally ‘younger man’) or \textit{wot yena} ‘younger sister’ (literally ‘younger woman’). For older siblings there are the gender-specific words \textit{atuma} ‘older brother’ and \textit{atana} ‘older sister’.\footnote{The form \textit{atana} ‘older sister’ probably derives from \textit{ata} ‘up, upper’ (cf. \textit{wat} ‘top’) plus \textit{yana} ‘woman’. The form \textit{atuma} ‘older brother’ may derive from \textit{ata} ‘up, upper’ plus \textit{uma} ‘bone’.} Although a man has no way of speaking generally about a brother (whether younger or older), a woman may refer to any of her male siblings (regardless of his relative age) simply as \textit{yeta} ‘man’.

  The words \textit{wot} ‘younger (sibling)’, \textit{atuma} ‘older brother’, and \textit{atana} ‘older sister’ may be used to add specificity to family relations of the parents’ generation (i.e., aunts and uncles), as shown in \REF{ex:sem:23}.

\ea%23
    \label{ex:sem:23}
          Terms for aunts and uncles
\begin{tabbing}
{(\textit{ane inom atana})} \= {(‘father’s younger brother’)}\kill
{\textit{itom wot}} \> {‘father’s younger brother’}\\
{\textit{yawa wot}} \> {‘mother’s younger brother’}\\
{\textit{inom wot}} \> {‘parent’s younger sister’}\\
{\textit{ane inom wot}} \> {‘father's younger sister’}\\
{\textit{itom atuma}} \> {‘father’s older brother’}\\
{\textit{yawa atuma}} \> {‘mother’s older brother’}\\
{\textit{inom atana}} \> {‘parent’s older sister’}\\
{\textit{ane inom atana}} \> {‘father’s older sister’}
\end{tabbing}
\z

\is{kinship|)}
\is{semantics|)}
\is{kinship term|)}

\is{kinship term|(}
\is{semantics|(}
\is{kinship|(}

For parents’ older male siblings, it is also possible to use the modifier \textit{ambi} ‘big’ instead of \textit{atuma} ‘older brother’, as in \REF{ex:sem:24}.

\ea%24
    \label{ex:sem:24}
          Alternative terms for parents’ older siblings
\begin{tabbing}
{(\textit{yawa ambi})} \= {(‘father’s older brother’)}\kill
{\textit{itom ambi}} \> {‘father’s older brother’}\\
{\textit{yawa ambi}} \> {‘mother’s older brother’}
\end{tabbing}
\z

Grandparents may be referred to with the \isi{adjective}/noun \textit{ngata} ‘grand; grandparent’, irrespective of gender. More specifically, though, the ego’s male grandparents are called \textit{itom ngata} ‘grandfather’, and the ego’s female grandparents are called \textit{inom ngata} ‘grandmother’. The term \textit{ngata} ‘grand’ is also used generally to refer to any old man or woman (cf. \ili{Tok Pisin} \textit{lapun} ‘old person’). It may also refer, broadly, to ‘ancestors’ or to members of a past generation. Sometimes the word \textit{mom} ‘grandmother’ is used as a \isi{vocative} form; it is a \isi{loan} from \ili{Ap Ma}.

  Grandchildren are known as \textit{yalum} ‘grandchild’, also a \isi{loan} from \ili{Ap Ma}. \linebreak Great-grandparents and great-grandchildren alike are called \textit{ndunduma} ‘great-grandparent, great-grandchild’. This latter term is also commonly used with the general sense of ‘ancestors’, usually those from the distant past.

  There is no special term for ‘wife’ that is distinct from general terms meaning ‘woman’. To refer to a wife, one may use either \textit{yena} ‘woman’ or \textit{yenanu} ‘woman’, or their alternate pronunciations, [yana] and [yananu]. To refer to one’s husband, however, the special form \textit{numan} ‘husband’ is used. The general term \textit{yeta} {\textasciitilde} \textit{yata} ‘man’ may be used by women to refer to their brothers, but generally not to their husbands. The form \textit{yenanu} ‘woman, wife’ is clearly related to \textit{yena} ‘woman, wife’.\footnote{It can probably be assumed that \textit{yena} ‘woman’ (clearly the \is{analogy} analogue of \textit{yeta} ‘man’) was the original word for ‘woman’. The form \textit{yenanu} ‘woman’ probably thus emerged as a word meaning ‘wife’, but in contemporary usage, \textit{yena} ‘woman’ and \textit{yenanu} ‘woman’ are completely interchangeable: both can mean either ‘woman’ or ‘wife’, and neither meaning seems to be more basic to either of the forms.}

  To refer to people related to the ego by marriage, the general term \textit{inga} ‘affine, in-law’ is used. It may be combined with other \isi{kinship} terms to add specificity, as in \REF{ex:sem:25}.

\ea%25
    \label{ex:sem:25}
          Examples of more specific terms for affines (in-laws)
\begin{tabbing}          
{(\textit{atuma inga yena})} \= {(‘younger brother’s wife’)}\kill
{\textit{wot inga yena}} \> {‘younger brother’s wife’}\\
{\textit{atuma inga yena}} \> {‘older brother’s wife’}
\end{tabbing}
\z

A number of \isi{taboo}s dictate the proper relationship that one has with one’s affines. For example, it is forbidden to utter an in-law’s name. Instead, one will typically employ one or another circumlocution to refer to a person related by marriage.

  It may also be noted that the term \textit{tamndï} ‘owner’ has importance in \isi{kinship} terminology. While otherwise referring to owners of physical property (e.g., land), \textit{tamndï} ‘owner’ may refer broadly to any kin, but especially to the next of kin following a death in the family (i.e., children, parents, siblings, and spouse). Incidentally, when there is a death in a family, other relatives belonging to the extended family are referred to as \textit{nambana ankam} ‘extended family member’ (literally ‘spirit person’).

  I summarize and conclude this section with a glossary of \isi{kinship} terms in Ulwa, starting with relations one generation older than the ego \REF{ex:sem:26}.

\ea%26
    \label{ex:sem:26}
          Kinship terms: ego’s parents’ generation
\begin{tabbing}
      {(\textit{ane inom atana})} \= {(‘younger sister’s husband’)}\kill
     \textit{itom} \> ‘father’\\
     \textit{inom} \> ‘mother’\\
     \textit{tata} \> ‘papa’ (nursery term for ‘father’; \isi{vocative} form)\\
    \textit{nana} \> ‘mama’ (nursery term for ‘mother’; \isi{vocative} form)\\
    \textit{yawa} \> ‘mother’s brother’\\
    \textit{yawa wot} \> ‘mother’s younger brother’\\
    \textit{yawa atuma} \> ‘mother’s older brother’ (or \textit{yawa ambi})\\
    \textit{itom wot} \> ‘father’s younger brother’\\
    \textit{itom atuma} \> ‘father’s older brother’ (or \textit{itom ambi})\\
    \textit{ane inom} \> ‘father’s sister’\\
    \textit{ane inom wot} \> ‘father’s younger sister’\\
    \textit{ane inom atana} \> ‘father’s older sister’\\
    \textit{inom wot} \> ‘parent’s younger sister’\\
    \textit{inom atana} \> ‘parent’s older sister’\\
    \textit{ansi inom} \> ‘mother’s brother’s wife’
\end{tabbing}
\z

It should be noted that \textit{itom} ‘father’ is also a general term for uncles (usually only paternal uncles), as well as a term of respect for any older man. It sometimes means simply ‘man’. Similarly, \textit{inom} ‘mother’ is also a general term for aunts, as well as a term of respect for any older woman. It sometimes means simply ‘woman’.

The terms in \REF{ex:sem:26a} represent relations that are of the same generation as the ego.

\ea%26a
    \label{ex:sem:26a}
          Kinship terms: ego’s generation
\begin{tabbing}
      {(\textit{atuma inga yena})} \= {(‘younger sister’s husband’)}\kill
    \textit{wot} \> ‘younger (sibling)’\\
    \textit{wot yeta} \> ‘younger brother’\\
    \textit{wot yena} \> ‘younger sister’\\
    \textit{atuma} \> ‘older brother’\\
    \textit{atana} \> ‘older sister’\\
    \textit{yeta} \> ‘brother’ (said only by women) (or \textit{yata}) (literally ‘man’)\\
    \textit{yena} \> ‘wife’ (literally ‘woman’)\\
    \textit{yenanu} \> ‘wife’ (< \textit{yena} ‘woman’, also means ‘woman’)\\
    \textit{numan} \> ‘husband’\\
    \textit{inga} \> ‘affine, in-law’ (i.e., any relation through marriage)\\
    \textit{wot inga yena} \> ‘younger brother’s wife’\\
    \textit{atuma inga yena} \> ‘older brother’s wife’\\
    \textit{wot yena numan} \> ‘younger sister’s husband’\\
    \textit{atana numan} \> ‘older sister’s husband’
\end{tabbing}
\z

It should be noted that \textit{inga} ‘affine, in-law’ is not strictly limited to relations of the ego’s generation, but refers more generally to any relations obtained through marriage, regardless of generation.

The terms in \REF{ex:sem:26b} represent relations that are one generation younger than the ego.

\ea%26b
    \label{ex:sem:26b}
          Kinship terms: ego’s children’s generation
\begin{tabbing}
{(ansi nungol)} \= {(‘younger sister’s husband’)}\kill
    \textit{nungol} \> ‘child’ (often ‘son’) (or \textit{nungolke})\\
    \textit{alum} \> ‘child’\\
    \textit{tawatïp} \> ‘child’\\
    \textit{yetalum} \> ‘son, boy’\\
    \textit{yenalum} \> ‘daughter, girl’\\
    \textit{yenat} \> ‘daughter’ (or \textit{yanat})\\
    \textit{ansi nungol}  \> ‘nephew, niece’ (refers to a man’s sister’s child)\\
    \textit{ansi yanat} \> ‘niece’ (refers to a man’s sister’s daughter)
\end{tabbing}
\z

Finally, terms in \REF{ex:sem:26c} represent relations that are more than one generation removed from the ego.

\ea%26c
    \label{ex:sem:26c}
          Kinship terms: more than one generation removed from ego
\begin{tabbing}
\is{loan}
\il{Ap Ma}
{(ngata yawa)} \= {(‘younger sister’s husband’)}\kill
    \textit{ngata} \> ‘grandparent, old person, ancestor’\\
    \textit{itom ngata} \> ‘grandfather, old man’\\
    \textit{inom ngata} \> ‘grandmother, old woman’\\
     \textit{mom} \> ‘grandmother’ (loan from Ap Ma)\\
    \textit{ngata yawa} \> ‘mother’s mother’s brother’\\
        \textit{yalum} \> ‘grandchild’ (loan from Ap Ma)\\
    \textit{ndunduma} \> ‘great-grandparent, great-grandchild, ancestor’
   \end{tabbing}
\z

The term \textit{ndunduma} ‘great-grandparent, great-grandchild’ refers to a relation that is three generations (or more) removed from the ego. It is unspecified as to whether the reference is to an older or a younger relation.

\is{kinship|)}
\is{semantics|)}
\is{kinship term|)}

\section{Expressions of time}\label{sec:14.8}

\is{time|(}
\is{semantics|(}

Ulwa’s vocabulary reflects some of its speakers’ traditional methods of marking time. The word for ‘year’, for example, is the same as the word \textit{inim} ‘water’. Living in the tropics, Ulwa speakers do not experience significant seasonal changes in temperature or amount of sunlight per day; the most salient demarcation of the passing of years is the annual rainy season, which generally starts in November or December. During the rainy season the rivers swell and much of the land becomes swampy.

  The word used for ‘month’ is \textit{iwïl} ‘moon’, reflecting the common division of time based on the synodic month (roughly 29.5 days). Contemporary speakers use the term \textit{iwïl} ‘moon’ to refer to the months of the Gregorian calendar, not to lunar cycles.

  There are also a number of interesting \isi{polyseme}s and derivatives within the semantic domain of ‘time’. The form \textit{amun} ‘now’ means both ‘now’ and ‘today’ (cf. colloquial \ili{Tok Pisin} \textit{nau} ‘now, today’). Similarly, \textit{awal} ‘afternoon’ also means ‘yesterday’. Based in part on the existence of \isi{cognate}s in the \ili{Keram} languages for ‘afternoon’ but not for ‘yesterday’, I assume that the word \textit{awal} ‘afternoon’ in Ulwa originally meant ‘afternoon’ and subsequently took on the meaning ‘yesterday’ (cf. \ili{English} \textit{eve}). In the formula for ‘good afternoon’, it is possible to clarify ‘afternoon’ as \textit{awal nambï} ‘body of the afternoon’. Finally, the words \mbox{\textit{umbenam} ‘morning’} and \textit{umbe} ‘tomorrow’ are clearly related. Here, too, I suspect that the time-of-day meaning preceded the different-day meaning.\footnote{In other words, \textit{umbe} ‘tomorrow’ probably originally meant ‘morning’, but was extended in meaning to mean ‘tomorrow’ as well (cf. \ili{Spanish} \textit{mañana} ‘morning, tomorrow’, \ili{German} \mbox{\textit{Morgen}} ‘morning’ and \textit{morgen} ‘tomorrow’, \ili{English} \textit{morrow} and \textit{tomorrow}, etc.).} The form \textit{umbenam} ‘morning’ may result from a \isi{compound} of *umbe ‘morning’ and \textit{anam} ‘sky’, literally ‘sky of the morning’.

  As mentioned, \textit{amun} ‘now’ can mean either ‘now’ or ‘today’; within the domain of this former meaning, \textit{amun} ‘now’ can be employed to convey a range of \isi{temporal} meanings, sometimes through the help of the \isi{copular enclitic} (see \sectref{sec:8.2.1} on \isi{temporal adverb}s). For example, it can mean ‘recently’ \REF{ex:sem:27} or ‘still’ \REF{ex:sem:28}, among other things.

\ea%27
    \label{ex:sem:27}
          \textit{Ala \textbf{amun} manap lop.}\\
\gll    ala      \textbf{amun}  ma=nap    lo-p\\
    \textsc{pl.dist}  now  3\textsc{sg.obj}=for  go-\textsc{pfv}\\
\glt `They recently went [to Madang] for his sake.’ [ulwa032\_29:18]
\z

\ea%28
    \label{ex:sem:28}
          \textit{Nï \textbf{amunpe} wol ame.}\\
\gll    nï    \textbf{amun}=p-e    wol  ama-e\\
    1\textsc{sg}  now=\textsc{cop-dep}  breast  eat-\textsc{ipfv}\\
\glt `I was still nursing.’ [ulwa013\_00:28]
\z

In Ulwa, the passage of time is generally expressed with verbal constructions. The verb \textit{wo-} ‘sleep’, usually in the \isi{perfective} form \textit{wop} ‘sleep [\textsc{pfv}]’, has become almost \isi{fossilized} as an \isi{adverb} meaning ‘the next day’. Examples \REF{ex:sem:29} and \REF{ex:sem:30} illustrate the use of \textit{wop} ‘sleep [\textsc{pfv]’} to indicate the passage of one day.

\ea%29
    \label{ex:sem:29}
          \textit{\textbf{Wope} nï man Chris mat.}\\
\gll    \textbf{wo}{}-p-e      nï    ma=n      Chris  ma=ta\\
    sleep-\textsc{pfv-dep}  \textsc{1sg}  \textsc{3sg.obj=obl}  [name]  3\textsc{sg.obj}=say\\
\glt `The next day, I told Chris.’ [ulwa014\_22:30]
\z

\ea%30
    \label{ex:sem:30}
          \textit{Nï ndïwanap \textbf{wop} wolka ndït tamndï ndïn up.}\\
\gll    nï    ndï=wana-p  \textbf{wo}{}-p    wolka  ndï=tï    tamndï     ndï=n    u-p\\
    1\textsc{sg}  \textsc{3pl=}cook-\textsc{pfv}  sleep-\textsc{pfv}  again  \textsc{3pl}=take  owner    3\textsc{pl=obl}  put-\textsc{pfv}\\
\glt `I cooked them, and the next day in turn gave them to the owners.’ [ulwa014\_47:26]
\z

\newpage

It is also possible to use other, often very long, expressions to convey the passage of a day, as in \REF{ex:sem:31}.

\ea%31
    \label{ex:sem:31}
          \textit{Awlu ilom ngawat u mat awe …}\\
\gll    awlu  ilom  nga=wat    u    ma=tï      aw-e\\
    step  day    \textsc{sg.prox}=atop  from  \textsc{3sg.obj}=take  put.\textsc{ipfv-dep}\\
\glt `On the next day …’ (Literally ‘taking a step away from this day’) [ulwa014\_70:42]
\z

The verb form \textit{wop} ‘sleep [\textsc{pfv]’} can also be used to express longer passages of time. In \REF{ex:sem:32}, this verb is used \isi{transitive}ly, with the amount of time passed as its \isi{direct object}.

\ea%32
    \label{ex:sem:32}
          \textit{Ilom lele \textbf{ndïwope} atana mï nan wot yena mat: …}\\
\gll    ilom  lele    ndï=\textbf{wo}{}-p-e      atana    mï na=n    wot    yena  ma=ta\\
    day    three  3\textsc{pl}=sleep-\textsc{pfv-dep}  older.sister  3\textsc{sg.subj}    talk=\textsc{obl}  younger  woman    3\textsc{sg.obj}=say\\
\glt `After three nights, the older sister said to the younger sister: …’ [ulwa011\_01:37]
\z

In \REF{ex:sem:33}, the passage of time is marked with the verb \textit{tï-} ‘take’, which has as its \isi{direct object} the amount of time passed. Here, as in \REF{ex:sem:32}, the \isi{object marker} is \isi{plural} to agree with the number of units of time (days, months, etc.) that have passed.

\is{semantics|)}
\is{time|)}

\ea%33
    \label{ex:sem:33}
          \textit{Iwïl lele \textbf{ndïtïne} yeta nga nan mat: …}\\
\gll    iwïl  lele    ndï=\textbf{tï}{}-n-e      yeta  nga     na=n ma=ta\\
    moon  three  \textsc{3pl}=take-\textsc{pfv-dep}  man  \textsc{sg.prox}  talk=\textsc{obl}    \textsc{3sg.obj}=say\\
\glt `After three months, the man told her: …’ (Literally ‘having taken three months’) [ulwa006\_05:29]
\z


\section{Coinages}\label{sec:14.9}

\is{semantics|(}
\is{coinage|(}

Most contemporary Ulwa speakers do not commonly coin words. Instead, when speaking Ulwa, people will generally use a \ili{Tok Pisin} \isi{loanword} to refer to any concept that lacks an Ulwa name. In the past, however, when confronted with new concepts, like ‘money’ or ‘matches’, speakers employed at least two basic methods for identifying such referents:
  
\begin{quote}
\begin{enumerate}[noitemsep, label={(\roman*)}, align=left, widest=190, labelsep=1ex,leftmargin=*]
\item extending the meaning (\isi{metaphor}ically or \is{metonymy} metonymically) of an existing Ulwa word to refer to the new concept; or
\item forming a \isi{compound noun}, often one that describes \isi{periphrastic}ally the new concept.
\end{enumerate}
\end{quote}
  
  Examples of words whose meanings have been extended to include new concepts are given in \REF{ex:sem:34}.

\ea%34
    \label{ex:sem:34}
          \isi{Semantic extension}s to refer to new concepts
\begin{tabbing}
{(\textit{mïndapan})} \= {(‘patrol officer, police officer’)} \= {(= ‘banana leaf’)}\kill
{\textit{apïn}} \> {‘matches, lighter’} \> {= ‘fire’}\\
{\textit{mïndapan}} \> {‘paper’} \> {= ‘banana leaf’}\\
{\textit{nïpïl}} \> {‘rope’} \> {= ‘vine’}\\
{\textit{wanwane}} \> {‘patrol officer, police officer’} \> {= ‘mushroom’\footnotemark}
\end{tabbing}
\z
\footnotetext[14]{The old colonial patrol officers (\textit{kiap} in \ili{Tok Pisin}) wore hats (perhaps pith helmets) that made them resemble mushrooms.}

Examples of \isi{compound}s formed to describe new concepts are given in \REF{ex:sem:35}. When a \isi{compound} is typically written elsewhere as a single \isi{orthographic} word, a hyphen is included here to show the breaks between members of the \isi{compound}.

\ea%35
    \label{ex:sem:35}
          Compounds to refer to new concepts
\begin{tabbing}
{(\textit{mbomala nangum})} \= {(‘official, civil servant’)} \= {(= ‘going claw hand’ (?))}\kill
{\textit{asi-mu}} \> {‘rice’} \> {= ‘grass seed’}\\
{\textit{i-nangïn-mana}} \> {‘official, civil servant’} \> {= ‘going claw hand’ (?)\footnotemark}\\
{\textit{inim tembi}} \> {‘alcohol’} \> {= ‘bad water’}\\
{\textit{mbomala nangum}} \> {‘flashlight’} \> {= ‘firefly shoot’}\\
{\textit{tïlwa num}} \> {‘car’} \> {= ‘road canoe’\footnotemark}
\end{tabbing}
\z
\footnotetext[15]{The etymology is of \textit{inangïnmana} ‘official, civil servant’ is obscure, but if it does in fact derive from something like ‘going claw hand’, then it is perhaps related to the official’s ability to catch people (cf. the \ili{English} expression \textit{the long arm of the law}).}
\footnotetext[16]{The element \textit{tïlwa} ‘road’ is itself a \isi{compound}, derived from \textit{utï} ‘foot’ plus \textit{luwa} ‘place’.}

Sometimes multiple means of coining words coexist for a single referent. There are, for example, a number of ways to refer to money. The word \textit{inamba} ‘ceremonial armband’ may be extended in meaning, presumably due to the armband’s material value. Another form, \textit{palapal} ‘shell; ceremonial armband; money’ is probably a \isi{loan} from \ili{Tok Pisin} \textit{palpal} {\textasciitilde} \textit{balbal} ‘Indian coral tree’. Alternatively, a \isi{compound} may be used, such as \textit{wombasa anga} ‘piece of clay pot’ or \textit{ata monam mu} ‘high fruit of the rain tree’, both of whose etymologies are obscure to me.

\is{coinage|)}
\is{semantics|)}


\section{{Traditional} {names}}\label{sec:14.10}

\is{name|(}
\is{traditional name|(}
\is{semantics|(}

The people of Manu village typically have multiple names. Almost everyone has one or more traditional names, but people are most commonly referred to and addressed by given names of \ili{English} or Biblical origin. If a person has more than one \isi{traditional name}, one of these is considered primary. As mentioned in \sectref{sec:14.7}, it is \isi{taboo} for someone to utter the primary name of someone related by marriage.

  The use of last names (family names) is a relatively new practice, and many of the current oldest living generation -- those born before around 1950 -- do not have last names that they use. Those who first adopted last names did so by selecting one of their own names or the name of a relative; this then became the name that would be passed down, in patrilineal fashion, to their children.

  The meanings of most names are unknown. While the etymologies of some may have been obscured through time, it is likely that many names are \isi{loan}s from neighboring languages. This is especially suspected to be the case where names contain sounds that are foreign to Ulwa, such as the \isi{velar} \isi{nasal} [ŋ], which occurs in the name \textit{Kanang} (pronounced [kanaŋ]). Also, while there is generally \isi{free variation} in pronunciation between [l] and [r] in the Ulwa \isi{liquid} phoneme /l/, there is a strong preference for some proper names to be pronounced with the \isi{rhotic} [r]. Accordingly, these names are written with <r> and not \textsuperscript{†}<l> (\sectref{sec:1.4}). Personal names in Ulwa are designated either for men or for women. Some common traditional male names are given in \REF{ex:sem:36}. Some common traditional female names are given in \REF{ex:sem:37}.

\is{semantics|)}
\is{traditional name|)}
\is{name|)}


\ea%36
    \label{ex:sem:36}
          Traditional male names
\begin{tabbing}
{(Awandana)} \= {(Awandana)} \= {(Awandana)} \= {(Awandana)} \= {(Awandana)} \= {(Awandana)}\kill
Alimban \> Ayndin \> Kanang \> Konawa \> Mongima \> Yaruwa \\
Alma \> Banjiwa \> Kapos \> Kongos \> Nomnga \> Yawat \\
Ambïnme \> Gambri \> Kawat \> Kowe \> Sambome \> Yokombla\\
Amiwa \> Ganmali \> Kayta \> Malman \> Wekumba \> Yolomban\\
Amombi \> Guren \> Kolpe \> Manama \> Womel \> Yomali\\
Anam \> { } \> { } \> { } \> { } \> { }
\end{tabbing}
\z

\ea%37
    \label{ex:sem:37}
          Traditional female names
\begin{tabbing}
{(Awandana)} \= {(Awandana)} \= {(Awandana)} \= {(Awandana)} \= {(Awandana)} \= {(Awandana)}\kill
Ambonda \> Gami \> Kawana \> Sinda \> Tanom \> Yambït\\
Asingona \> Ginam \> Mapana \> Tambana \> Woni \> Yanapi\\
Awandana \> Gwam \> Maple \> Tangin \> Yambin \> Yawana\\
Damnda \> Jukan \> Mawna \> { } \> { } \> { }
\end{tabbing}
\z

\section{{Toponyms}}\label{sec:14.11}

\is{toponym|(}
\is{place name|(}
\is{semantics|(}

Some place names found in and around the Ulwa-speaking area are given in \REF{ex:sem:38}. All forms presented here reflect the \ili{Manu} \isi{dialect}. As is the case for personal names (\sectref{sec:14.10}), some toponoyms may derived from names used by members of other speech communities. Other toponyms, however, may have language-internal etymologies. For example, \textit{Nïmalnu} ‘Manu village’ may derive from \linebreak \textit{nïmal} ‘river’ (\sectref{sec:1.5.3}). Similarly, \textit{Wopata} (the name of the ‘old’ Manu village) is a contraction of \textit{wa wapata} ‘old village’ (Appendix \ref{sec:app.f}).


\ea%38
    \label{ex:sem:38}
          Place names
\begin{tabbing}
    {(Kanangwa)} \= {(alternative name of Amali village)}\kill
    Amali \> site of the third Manu village\\
    Ambwat \> Kambot (village)\\
    Andïmali \> Dimiri (village)\\
    Bulon \> region immediately surrounding the current Manu village\\
    Dim \> Biwat (village); name of the original Manu village\\
    Imwa \> region surrounding Wopata village\\
    Mïkïlwe \> jungle region near Manu village\\
    Morombi \> Raten (village)\\
    Mosombla \> Yaul (village)\\
    Kambok \> Kambuku (village)\\
    Kamen \> ancestral village of the Ulwa peopla\\
    Kanangwa \> alternative name of Amali village\\
    Kumba \> Bun (village)\\
    Mamala \> Maruat (village)\\
    Mamanu \> downstream half of the old Wopata village\\
    Nanïmwat \> name of the old Yetani (Yamen) village\\
    Nïmalnu \> Manu (village)\\
    Talamba \> jungle region near Manu village\\
    Taw  \> jungle region near Manu village\\
    Tïwen \> jungle region near Wopata village\\
    Tïponïm \> section of Manu village where the school was built\\
    Wopata \> site of the fourth Manu village\\
    Yalamba \> Korokopa (village)\\
    Yambiwa \> upstream half of the old Wopata village\\
    Yambul \> site of the second Manu village\\
    Yetani \> Yamen (village)
  \end{tabbing}
\z


\is{semantics|)}
\is{place name|)}
\is{toponym|)}
