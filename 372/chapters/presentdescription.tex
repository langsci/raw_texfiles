\chapter{The present description}\label{sec:pres}

In this chapter I offer a brief account of how this grammatical description was written and how the data within it are organized and presented. I begin by situating the present description in the wider context of research relating to Ulwa.

\section{Previous research on the language}\label{sec:1.1}

As far as I am aware, the earliest published mention of the Ulwa people occurs in \citet[199]{Mead1938}, who describes the pottery of “the Dimili and Yaulu peoples”. For the Dimili people she offers the alternative name “Dimiri”, describing them as “‘Grass-men’ people living southeast of the Mundugumor [= \ili{Mundukumo}], inland from the \isi{Yuat River}” \citep[347]{Mead1938}. For the Yaulu people she offers the alternative name Yaul, describing them as “‘Grass-man,’ people living east of the Mundugumor between the \isi{Yuat} and the Little Ramu (Potter’s River) [= \isi{Keram River}]” \citep[349]{Mead1938}. Although Mead does not discuss their language, the names Yaul and Dimiri refer to two villages where Ulwa was spoken.\footnote{Mead’s description was presumably the source for \citegen[30]{Loukotka1957} inclusion of Yaulu and Dimili within a list of 21 unknown “languages” of the \isi{Sepik} Basin: “Les autres langues dans le bassin du \isi{Sepik} sont jusqu’à présent inconnues. Les principales sont les suivantes … le \textit{Yaulu} entre le \isi{Yuat} et le \isi{Kerám}, le \textit{Dimili} au sud-est du Mundokuma, …” [“The other languages in the \isi{Sepik} Basin are thus far unknown. The main ones are the following … \textit{Yaulu} between the \isi{Yuat} and the \isi{Kerám}, \textit{Dimili} southeast of the Mundokuma [= \ili{Mundukumo}], …”].}

Following World War II, the \isi{Sepik} area, as part of the Australian-administered Territory of New Guinea, was regularly visited by patrol officers (called \textit{kiap} in \ili{Tok Pisin}), whose patrol reports occasionally contained linguistic \isi{classification}s, of varying accuracy. One of these reports, written in 1950 by officer \name{F. D.}{Anderson}, provides a rather correct identification of some of the languages spoken in the Grass Country Census Group (\citealt[no. 19]{PatrolReports1950}). No language names are provided, but villages are sorted into columns, according to language. One such column contains five villages: Dimiri, Yaul, Manu, Bruten, and Marawat [= Maruat]. These are the villages where Ulwa was spoken.\footnote{I do not know what has become of “Bruten” village, which was apparently situated very close to the former location of Manu village. I can find no record of it later than the 1950s.} Later reports, however, were not as accurate, often grouping Ulwa together with different languages. However, a patrol report written in 1969 by officer \name{Jon R.}{Bartlett} for the Yuat Census Division, correctly groups together the four Ulwa-speaking villages, even assigning a name to the language:

\begin{quote}
UNAMAMA Language: is spoken by YAUL, DIMIRI and MARAWAT -- a total of 714 people. The only other group to speak this language is MANU Village in the Grass Census Division. The older men from DIMIRI are fluent in the MUNDUGOMOR language as they used to live very close to BRANDA Village pre-war. (\citealt[no. 4]{PatrolReports1969})
\end{quote}

These are indeed the four villages where Ulwa was, and still is, spoken. The name \textit{Unamama} (i.e., \textit{una[n]} ‘\textsc{1pl.incl’} + \textit{mama} ‘mouth’) literally means ‘our mouth’. “Mundugomor” here refers to \ili{Mundukumo} (also known as \ili{Biwat}), which is a member of the \ili{Yuat} family.

However, these two patrol reports did not apparently inform any linguistic descriptions or indeed even any future patrol reports, which continued to conflate Ulwa with other languages, in particular with \ili{Ap Ma} (also known as \ili{Kambot} or \ili{Botin}).

  In his linguistic survey, \citet[48]{Capell1962} writes: “The languages on the south-east of the \isi{Sepik River}, reported in the first edition of this work [\citep{Capell1954}] as unknown, still remain in practically the same situation.” No mention is made of Ulwa, nor does the language appear on his map of the \isi{Sepik} Basin (\citealt[38--39]{Capell1962}). In this map, the two language areas of \ili{Biwat} [= \ili{Mundukumo}] and \ili{Kambot} [= \ili{Ap Ma}] are shown as being adjacent, whereas the Ulwa-speaking area should appear somewhere between them. As far as I can tell, the earliest documentation of the Ulwa language itself consists of a single word by \linebreak \citet[95]{Haberland1966}, who, apparently based on a letter from \name{Karl}{Laumann} in 1965, records the word <\textit{wä}> ‘sago palm’ as being used in the villages of Yaul and \mbox{Dimeri [= Dimiri].}\footnote{Although flagging these villages with a question mark, he suggests that they could be part of the \ili{Mundukumo} \isi{dialect} area.} This recorded word no doubt refers to Ulwa \textit{we} ‘sago starch’.

The first person to mention Ulwa in the linguistic material was Donald \linebreak \citet[36]{Laycock1973}, who refers to the language as Yaul. He classifies Yaul [= Ulwa] as being related to \ili{Mongol} [= \ili{Mwakai}] and \ili{Langam} [= \ili{Pondi}], lists the four villages where Ulwa is spoken, estimates the total population of these four villages to have been 814 in 1970, and names the two speakers (Silami and Ansamari) from whom he collected data in Yaul village in 1971. \citegen{Laycock1971a} unpublished handwritten field notes on the \ili{Yaul} \isi{dialect} of Ulwa are reproduced in \mbox{Appendix \ref{sec:app.g}.}

In 2005, while researching the \ili{Mundukumo} language, \name{James}{McElvenny} was able to record some speakers of “Yaul” and “Dimili”. These Ulwa recordings are archived with PARADISEC \citep{McElvenny2005}. The collection consists of four recordings:

\begin{quote}
\begin{enumerate}[noitemsep, label={(\roman*)}, align=left, widest=190, labelsep=1ex,leftmargin=*]
\item an elicitation session of \isi{basic vocabulary} and sentences conducted in Dimiri village (01:29);
\item a story told in Ulwa and \ili{Tok Pisin} recorded in Dimiri village (02:38);
\item an elicitation session of \isi{basic vocabulary} and sentences conducted in the \ili{Mundukumo}-speaking Kinakatem village with an Ulwa speaker from Yaul village (40:00); and
\item a story told in Ulwa by the same speaker, also recorded in Kinakatem village (03:04).
\end{enumerate}
\end{quote}

I began researching Ulwa in 2015, thanks to the good advice and guidance of \name{William A.}{Foley}. See \sectref{sec:1.2} for discussion of my sources and data collection.

\section{The name of the language}\label{sec:1.5.1}

Speakers from all four villages where the language is spoken agree upon \textit{Ulwa} as a \isi{glottonym}. When Laycock conducted his survey work of the \isi{Sepik} area between 1970 and 1971, he recorded the name of this language as “Yaul”, which is the name of one of the four villages. In doing so, \citet[3]{Laycock1973} seems to have contravened one of his principles in choosing language names: “The name should not be that of a village, clan or locality that is significantly smaller than the language area, or that is not accepted by the whole group without feelings of rivalry”. This name lent itself to the formation of the ISO 639-3 code [yla] and the glottocode [yaul1241]. Nevertheless, I do not use it to refer to the language described by this grammar, since it is not the preferred name for the language among its speakers. Furthermore, the term “\ili{Yaul}” creates confusion between reference to the village (and \isi{dialect}) of that name and reference to the language as a whole. That is, I agree with the principle of not naming a language for a village, particularly in cases such as this one, in which the language is spoken in multiple villages. \mbox{\citet[206]{Foley2018} refers to Ulwa as “Yaul-Dimiri”,} which is indeed more \isi{inclusive}, but still does not cover the two other villages where Ulwa is spoken. \textit{Glottolog 4.7} \citep{HammarströmEtAl2022} refers to the language as “Ulwa (Papua New Guinea)”, presumably to differentiate it from a wholly unrelated \mbox{\ili{Misumalpan} language} of \il{Ulwa (Nicaragua)} Nicaragua also known as Ulwa.

As is common among languages of the \isi{Sepik} (and is indeed attested in various languages across the globe), the \isi{glottonym} \textit{Ulwa} is based on a word that means ‘no’ or ‘nothing’. However, although this naming strategy is recurrent throughout the \isi{Sepik} area, there is no evidence that Ulwa speakers have had a long tradition of employing it to refer to their language. It is my impression that it is, and was, more common for people of the region to identify themselves and others by village affiliation rather than by language. \citet[3218]{Laycock1971a} presumably did not encounter the name \textit{Ulwa} when interviewing his consultants in Yaul village. He did, however, record what was perhaps an ad hoc \isi{glottonym}: <ANDJI LOWA> (probably /aⁿdʒi la wo/, i.e., ‘our own talk’). As mentioned in \sectref{sec:1.1}, an Australian patrol officer in 1969 recorded the name of the language as <UNAMAMA> ‘our mouth’ (\citealt{PatrolReports1969}).

\section{Sources and data}\label{sec:1.2}

I began collecting Ulwa language data in 2015, during a two-month trip to Papua New Guinea. During that trip as well as during subsequent trips, I worked with native speakers of the language, all of whom were at least 30 years old and were raised in Ulwa-speaking villages. Most of my consultants were between 40 and 60 years old and were from Manu village. Although I collected texts from both men and women, my elicitation sessions were mainly conducted with male speakers. I also collected data from speakers from the other three Ulwa villages, as well as from speakers who were older than 60 or 70.

In total, I spent about twelve months living and working with Ulwa speakers, divided among four research trips:

\begin{quote}
\begin{enumerate}[noitemsep, label={(\roman*)}, align=left, widest=190, labelsep=1ex,leftmargin=*]
\item two months in 2015 (June to August)
\item six months in 2016 (June to December)
\item three months in 2017 (April to June)
\item one month in 2018 (August to September)
\end{enumerate}
\end{quote}

In total, I recorded over 60 hours of audio (including about 6 hours with accompanying video). Most of my time was spent in Manu village, but I also visited the three other Ulwa-speaking villages (Maruat, Dimiri, and Yaul) in 2015. In 2018, I worked with speakers originally from Yaul but living in the town of Angoram. Although I consider data from a variety of speakers, this is foremost a grammatical description of the \ili{Manu} \isi{dialect} of the Ulwa language as spoken by older speakers. A brief comparative study of the \ili{Yaul} variety of Ulwa is provided in \chapref{sec:18}.\footnote{I have also conducted research with the other members of \ili{Keram} language family, which constitute Ulwa’s sister languages. The grammars of these languages have at times informed my analysis of Ulwa’s grammar.}

The data gathered during my research trips consist of various types: elicited words and sentences, grammaticality judgments, prepared texts, and texts of more naturalistic speech, including both monologues and conversations. I have tried to base my descriptions on actual language use -- that is, analyzing the language based on a corpus of naturalistic speech. Most of the examples in this grammar are therefore drawn from a corpus of about 6 hours of transcribed, translated, and glossed texts. Nevertheless, elicited sentences and grammaticality judgments offer invaluable insights into the nuances of certain grammatical distinctions; and sets of elicited sentences often provide the most illustrative examples of grammatical phenomena. Accordingly, in addition to examples culled from a recorded corpus of texts, there are a number of example sentences taken from elicitation sessions. All examples, whether they were elicited or not, have been vetted by native speakers for their grammaticality (or, in the case of starred sentences, for their ungrammaticality).

  My analysis of the language is not limited to or by any particular theoretical framework. Rather, my overarching goal in writing this grammar has been to describe the language “in its own terms” \citep[211]{Dryer2006}, drawing insights from different approaches where appropriate, in keeping with what I consider to be the best practices of typologically informed descriptive linguistics. My primary concern has been to make the description and analysis clear and accessible to a broad range of linguists and others who may have interest in this language.

\section{Recordings}\label{sec:app.e}

Audio recordings that were produced in 2015 are archived with the \mbox{Kaipuleohone} Language Archive at the University of Hawaiʻi \citep{Barlow2015}. Audio and video recordings, transcriptions and translations of texts, and some photographs of village life that were produced between 2015 and 2018 are archived with the Endangered Languages Archive (ELAR) at the School of Oriental and African Studies, University of London (SOAS) \citep{Barlow2018b}. Although only recordings contained within the ELAR archive have been selected as sources for examples in this description (\sectref{sec:1.3}), my understanding of the language has also benefited from the recordings contained within the Kaipuleohone archive.

  \tabref{tab:K} provides a list of the audio recordings for Ulwa that are deposited with the Kaipuleohone Language Archive:\\

\url{https://hdl.handle.net/10125/37432}\\

These are the earliest recordings I made of Ulwa, during my first visit to the speaker community in 2015. Only speakers’ initials are included in the table; their full names are given in \tabref{tab:speakers}.

%\begin{table}


\begin{xltabular}{\textwidth}{llllQ}
\caption{\label{tab:K} Ulwa recordings archived with Kaipuleohone}\\
\lsptoprule
		File & Speaker(s) & Date & Village & Title\\
		\midrule\endfirsthead
    \midrule
    File & Speaker(s) & Date & Village & Title\\
		\midrule\endhead
    \midrule\endfoot
    \lspbottomrule
    \endlastfoot

RXB1-001 & AA & 25.06.2015 & Manu & Crocodile story\\
RXB1-002 & TG & 29.06.2015 & Manu & Bandicoot story\\
RXB1-003 & TG, MW & 29.06.2015 & Manu & Conversation\\
RXB1-004 & LA & 02.07.2015 & Manu & The two sisters\\
RXB1-005 & CK & 02.07.2015 & Manu & Life in the “stone age”\\
RXB1-006 & LA & 02.07.2015 & Manu & The turtle\\
RXB1-007 & MS & 17.07.2015 & Yaul & World War II story\\
RXB1-008 & EU & 17.07.2015 & Yaul & The song “Kamul”\\
RXB1-009 & KM & 17.07.2015 & Dimiri & Dimiri’s new location\\
RXB1-010 & KM & 17.07.2015 & Dimiri & Song\\
RXB1-011 & JM & 17.07.2015 & Maruat & Warrior story\\
RXB1-012 & PS & 17.07.2015 & Maruat & Song about water\\
RXB1-013 & PS & 17.07.2015 & Maruat & Welcome song to Maruat\\
RXB1-014 & TG & 23.07.2015 & Manu & Cannibalism\\
RXB1-015 & TG & 23.07.2015 & Manu & How to harvest grubs\\
RXB1-016 & TG & 23.07.2015 & Manu & A trip to Maruat village\\
RXB1-017 & TG & 23.07.2015 & Manu & Origin of the Ulwas\\
RXB1-018 & SG & 23.07.2015 & Manu & Men’s house song\\
RXB1-019 & MW & 25.07.2015 & Manu & Water spirit song\\
RXB1-020 & MW & 25.07.2015 & Manu & Ornament song\\
RXB1-021 & MW & 25.07.2015 & Manu & Welcome song to Manu\\
RXB1-022 & MW & 27.07.2015 & Manu & The lullaby song “Nane”\\
RXB1-023 & PT & 27.07.2015 & Manu & Song about moving\\
RXB1-024 & FA & 27.07.2015 & Manu & Francis’s move\\
RXB1-025 & FA & 27.07.2015 & Manu & Description of his father\\
RXB1-026 & FA & 27.07.2015 & Manu & Missionaries\\
RXB1-027 & FA & 27.07.2015 & Manu & Life in Manu long ago\\
RXB1-028 & LA & 27.07.2015 & Manu & How to prepare sago\\
RXB1-029 & LA & 27.07.2015 & Manu & Funeral customs\\
RXB1-030 & EM & 27.07.2015 & Manu & The song “Jambisa”\\
RXB1-031 & FA & 27.07.2015 & Manu & Men’s houses\\
RXB1-032 & SM & 28.07.2015 & Manu & Resurrection\\
RXB1-033 & SM & 28.07.2015 & Manu & A trip to Madang\\
RXB1-034 & SM & 28.07.2015 & Manu & Mock debate\\
RXB1-035 & SM, OW & 28.07.2015 & Manu & Language policies\\
RXB1-036 & SM, OW & 28.07.2015 & Manu & Sago story\\
RXB1-037 & PM & 28.07.2015 & Manu & The story of the moon\\
RXB1-038 & PM & 28.07.2015 & Manu & The meal song “Lawlo”\\
RXB1-039 & OW & 28.07.2015 & Manu & Otto’s move\\
RXB1-040 & PM & 28.07.2015 & Manu & Pig story\\
RXB1-041 & TA & 02.08.2015 & Manu & The story of sago\\
RXB1-042 & TA & 02.08.2015 & Manu & Epilogue\\
RXB1-043 & TA & 02.08.2015 & Manu & The first people\\
RXB1-044 & FA & 02.08.2015 & Manu & Stone axe\\
\end{xltabular}
%\end{table}


\tabref{tab:E.1} provides a list of the audio and video recordings of Ulwa that are deposited with the ELAR archive:\\

\url{http://hdl.handle.net/2196/cffec915-d63a-4482-a0c7-bb606c504b2a}

\url{http://hdl.handle.net/2196/00-0000-0000-000F-CB61-A}\\

All texts used in examples in this grammar can be found in the ELAR archive.\footnote{Some material from the very beginning and end of the text ulwa014 are not included in ELAR: example sentences taken from these sections are marked with an obelisk (†) in the relevant examples.} The recordings in this collection were all made in Manu village. Only speakers’ initials are included in the table; their full names are given in \tabref{tab:speakers}.

%\begin{table}


\begin{xltabular}{\textwidth}{lllQ}
\caption{\label{tab:E.1} Ulwa recordings archived with ELAR}\\
\lsptoprule
		File & Speaker(s) & Date & Title\\
		\midrule\endfirsthead
    \midrule
    File & Speaker(s) & Date & Title\\
		\midrule\endhead
    \midrule\endfoot
    \lspbottomrule
    \endlastfoot

ulwa001 & YK & 22.06.2016 & Wonmelma\\
ulwa002 & AB & 16.11.2016 & Origins of the Ulwa people\\
ulwa003 & AB & 16.11.2016 & Tobacco\\
ulwa004 & AB & 16.11.2016 & Ayndin’s grandmother\\
ulwa005 & AB & 16.11.2016 & Yambalpa\\
ulwa006 & AB & 16.11.2016 & The mother of the turtle (\sectref{sec:16.1})\\
ulwa007 & AB & 16.11.2016 & Ayndin’s personal history\\
ulwa008 & AB & 16.11.2016 & Sago\\
ulwa009 & AB & 16.11.2016 & Ambawanam Ngata\\
ulwa010 & AB & 16.11.2016 & Splitting the coconut\\
ulwa011 & AB & 16.11.2016 & Molpan Ngata\\
ulwa012 & AB & 16.11.2016 & Metmet\\
ulwa013 & TG & 20.11.2016 & Tarambi’s personal history\\
ulwa014 & TK, YK & 28.04.2017 & Tangin and Yanapi\\
ulwa015 & TG & 21.05.2017 & Shell armbands\\
ulwa016 & TG & 21.05.2017 & Boar tusks\\
ulwa017 & TG & 21.05.2017 & Shell bilum\\
ulwa018 & TG & 21.05.2017 & Eating in the men’s house\\
ulwa019 & YK & 26.05.2017 & Itïtïl Yena\\
ulwa020 & YK & 26.05.2017 & The woman Amblom (\sectref{sec:16.2})\\
ulwa021 & YK & 26.05.2017 & Gasuwa’s trip to the spirit world\\
ulwa022 & YK & 26.05.2017 & Scraping sago\\
ulwa023 & YK & 26.05.2017 & Ulimal makes the river\\
ulwa024 & YK & 26.05.2017 & Yanapi’s children\\
ulwa025 & AJ & 27.05.2017 & Sago palms at Wopata\\
ulwa026 & AJ & 27.05.2017 & When Ambasap was sick\\
ulwa027 & AJ & 27.05.2017 & Trips to Angoram\\
ulwa028 & AJ & 27.05.2017 & Going between Wopata and Manu\\
ulwa029 & AJ & 27.05.2017 & Getting tattoos\\
ulwa030 & AJ & 27.05.2017 & Preparing food\\
ulwa031 & AJ & 27.05.2017 & Making plans for tomorrow\\
ulwa032 & TK & 29.05.2017 & Tangin’s trip to Bun village\\
ulwa033 & GT, TK & 01.06.2017 & Gweni’s childhood\\
ulwa034 & GT, TK & 01.06.2017 & Battle at Talamba\\
ulwa035 & GT, TK & 01.06.2017 & Snakes (\sectref{sec:16.3}) / crocodile hunt\\
ulwa036 & GT, TK & 01.06.2017 & Child’s death / poisoning fish\\
ulwa037 & AB, TG & 12.06.2017 & Ayndin and Tarambi\\
ulwa038 & AB, AJ & 19.06.2017 & Dry season\\
ulwa039 & AB, AJ & 19.06.2017 & Murder at Maruat village\\
ulwa040 & AB, AJ & 19.06.2017 & Yesterday’s activities\\
ulwa041 & AB, AJ & 19.06.2017 & Plans for the evening\\
ulwa042 & AB, AJ & 19.06.2017 & Ayndin’s plans to grow tobacco\\
ulwa043 & Manu village & 20.09.2016 & Singsing\\
\end{xltabular}
%\end{table}

The set of examples used in this description draws from most of the recordings listed in \tabref{tab:E.1}. Six of these recordings, however, although available in the ELAR archive, were not used for any examples in this book: ulwa005, ulwa007, ulwa012, ulwa017, ulwa025, ulwa043. The ELAR archive also includes 15 photographs of the Ulwa language community and its environment (\tabref{tab:photos}).

\begin{xltabular}{\textwidth}{llQ}
\caption{\label{tab:photos} Photographs archived with ELAR}\\
\lsptoprule
		File & Date & Title\\
		\midrule\endfirsthead
    \midrule
    File & Date & Title\\
		\midrule\endhead
    \midrule\endfoot
    \lspbottomrule
    \endlastfoot

ulwa044 & 27.06.2015 & Manu village\\
ulwa045 & 27.06.2015 & \isi{Keram Black River}\\
ulwa046 & 02.07.2015 & Fishing\\
ulwa047 & 06.07.2015 & Harvesting sago\\
ulwa048 & 07.07.2015 & Cocoa\\
ulwa049 & 16.07.2015 & Maruat village\\
ulwa050 & 17.07.2015 & Yaul village\\
ulwa051 & 17.07.2015 & Dimiri village\\
ulwa052 & 20.07.2015 & Sago and grubs\\
ulwa053 & 23.07.2015 & Bandicoot meat\\
ulwa054 & 27.07.2015 & Masks\\
ulwa055 & 29.07.2015 & House\\
ulwa056 & 31.07.2015 & Burning a pig\\
ulwa057 & 02.08.2015 & Pick-axe\\
ulwa058 & 05.08.2015 & Watching a boat depart\\
\end{xltabular}
%\end{table}



\tabref{tab:speakers} gives the full names of the Ulwa speakers recorded in the files listed in \tabref{tab:K} and \tabref{tab:E.1}.

%\begin{table}


\begin{xltabular}{\textwidth}{lllQ}
\caption{\label{tab:speakers} Speakers}\\
\lsptoprule
		Initials & Speaker & Also known as\\
		\midrule\endfirsthead
    \midrule
    Initials & Speaker & Also known as\\
		\midrule\endhead
    \midrule\endfoot
    \lspbottomrule
    \endlastfoot

AA & Alus Amombi & {}\\
AB & Ayndin Bram & {Joseph Bram}\\
AJ & Ambasap Jomia & {Christina Jomia}\\
CK & Cecilia Sikimba & {}\\
EM & Elias Mangeme & {}\\
EU & Elias Usimari & {}\\
FA & Francis Ambata & {}\\
GT & Gweni Tungun & {}\\
JM & John Morangi & {}\\
KM & Kunam Malaku & {}\\
LA & Lucy Ambata & {}\\
MS & Mathew Sango & {}\\
MW & Morris Womel & {}\\
OW & Otto Wandangin & {}\\
PM & Paulina Mapul & {}\\
PS & Philip Siwingin & {}\\
PT & Philo Tatu & {}\\
SG & Samuel Gambri & {}\\
SM & Stephen Mawipa & {}\\
TA & Thomas Ambata & {Alimban Ambata}\\
TG & Tarambi Gambri & {David Gambri}\\
TK & Tangin Kapos & {Rosa Kapos}\\
YK & Yanapi Kua & {Yaka Kua}\\
\end{xltabular}
%\end{table}

\section{Orthography}\label{sec:1.4}

\is{orthography|(}

The Ulwa writing system was proposed in \citet{Barlow2018a}. In developing this \isi{orthography}, I had a number of interests in mind. First, as much as possible, I tried to maintain an isomorphic relation between sound and symbol. Indeed, each phoneme can be written in only one way. There is thus exactly one \isi{grapheme} for every phoneme and one phoneme for every \isi{grapheme}. Second, I considered the practicalities of reading and writing the language, and I thus mostly avoided using unusual characters. The Ulwa alphabet consists of 19 letters, almost all of which are basic Latin characters, found both in \ili{English} and in \ili{Tok Pisin}, and are easily typed on any keyboard. The one exception is the \isi{grapheme} <ï>, which represents the \is{high vowel} high \isi{central vowel} (written in the IPA as <ɨ>). Although it would be preferable to avoid diacritics entirely, there is no readily available alternative to this form (which contains a dieresis), since all five basic \isi{vowel}s of the Latin alphabet are used to represent other phonemes in Ulwa’s \isi{orthography}. The form <ï> was chosen over the IPA form <ɨ>, since it is easier to type on a computer (on a PC: Alt 139; on a Mac: option u + i) or with a smartphone (by pressing and holding the i button) and hopefully also less easily confused with the form <i>.

  Aside from <ï>, the phonemic values of Ulwa’s 19 letters should not be difficult for a general audience to intuit. The only \isi{digraph}s in the \isi{orthography} are the four that are used to represent the language’s three \isi{prenasalized} \isi{voiced} \isi{stop}s and one \isi{prenasalized} \isi{voiced} \isi{affricate}. On phonemic grounds, these could have been written as <b, d, g, j> as opposed to <mb, nd, ng, nj>, since there are no phonemic contrasts between \isi{prenasalized} and plain \isi{voiced} \isi{stop}s in the language.\footnote{Similarly, there is no need to write the \isi{voiceless} stops (all of which are \isi{aspirated}) as <pʰ, tʰ, kʰ>, since there is no contrast between \isi{aspirated} and \isi{unaspirated} \isi{voiceless} \isi{stop}s in the language.} Nevertheless, I decided to represent the \isi{nasal} gesture in these phonemes (i.e., the \isi{nasal} sub-segments) overtly in the \isi{orthography} with a \isi{digraph}, so as to avoid any possible mispronunciation. As the language faces attrition, younger speakers and language learners may fail to note the \isi{prenasalized} quality of the \isi{voiced} \isi{stop}s, and they are likely to read Ulwa by following \ili{Tok Pisin} and \ili{English} spelling conventions, not those chosen explicitly for Ulwa, thus pronouncing <b, d, g> as [b, d, ɡ], rather than with their prenasal gesture as [ᵐb, ⁿd, ᵑɡ]. Indeed, whereas the oldest speakers are inclined to pronounce plain \isi{voiced} \isi{stop}s in \ili{Tok Pisin} as \isi{prenasalized} \isi{voiced} \isi{stop}s (e.g., [ⁿdok] for \ili{Tok Pisin} /dok/ ‘dog’), younger speakers, whose first language is usually \ili{Tok Pisin}, do just the opposite -- that is, they fail to pronounce the \isi{nasal} portion of Ulwa’s \isi{voiced} \isi{stop}s, especially when word-initial (e.g., [dunduma] instead of [ⁿduⁿduma] ‘great-grandparent’). Also, regarding these \isi{grapheme}s, it may be noted that the \isi{phonetic} realization of <ng> is [ᵑɡ] -- that is, with a \is{velar} \isi{nasal} \isi{velar} (as opposed to \isi{alveolar}) \isi{nasal} element. However, there is no need to write this \isi{grapheme} with an engma (<ŋ>), since there is no phonemic \isi{velar} \isi{nasal} in the language, and writing one would require a less common character. Furthermore, a natural \isi{phonological} process  \is{assimilation} assimilates \isi{alveolar} \isi{nasal}s to the place of following \isi{velar} \isi{stop}s (i.e., /nk/ → [ŋk]). It should likewise be noted that <ng> is always pronounced [ᵑɡ] and never \textsuperscript{†}[ŋ] or \textsuperscript{†}[ŋ.ɡ] (as in, say, \ili{English}, \textit{si}\textbf{\textit{ng}}\textit{er} or \textit{fi}\textbf{\textit{ng}}\textit{er}). Similarly, the <n> component of the \isi{grapheme} <nj> represents a \isi{palato-alveolar} \isi{nasal} gesture, as opposed to an \isi{alveolar} \isi{nasal} gesture.

  When a \isi{proper noun} (such as the \isi{name} of a person or place) begins with a \isi{prenasalized} \isi{stop} (or \isi{affricate}), however, only the \isi{stop} (or \isi{affricate}) gesture of the phoneme is written. Thus, for example, the personal names [ᵐbaⁿdʒiwa, ⁿdamⁿda, ᵑɡanmali, ⁿdʒukan] are written <Banjiwa, Damnda, Ganmali, Jukan>. This is in keeping with earlier Ulwa name-writing practices, which were themselves likely influenced by the perceptions or preferences of the colonial Australian officers charged with taking census and writing names. Whatever its origin, this practice is maintained here, since it is in keeping with the preferences of current Ulwa speakers. But since the present work also maintains the convention of capitalizing the first letter of \isi{proper noun}s, the \isi{grapheme}s <B, D, G, J> may simply be thought of as representing [ᵐb, ⁿd, ᵑɡ, ⁿdʒ].

\is{place name}

There is one further point to make concerning \isi{proper noun}s: while the \isi{liquid}s [l] and [r] are almost always in \isi{free variation} (as \isi{allophone}s of the phoneme /l/), there is a strong preference among speakers that certain \isi{proper noun}s be pronounced with a \isi{rhotic} [r] sound and never with a \isi{lateral} [l] sound (even though speakers themselves, in casual speech, may pronounce the sound in question as closer to [l] in these names). Since many \isi{proper noun}s are apparently shared with neighboring language communities, it is not unreasonable to assume that such names are in origin \isi{loanword}s. Regardless of their history, these names are written here with the \isi{grapheme} <r>: for example, the proper names <Gambri, Guren, Yaruwa>.

When a \isi{phonological} rule changes the underlying form of a word, the \isi{orthography} reflects the \isi{phonological} realization, not the underlying form. Thus, when the shape of one or more morphemes in an underlying form alters due to a \isi{phonological} rule that occurs within a \isi{phonological} word, the resultant \isi{phonological} realization is written. In practice, this mainly only affects verbs, which take a number of \isi{TAM} \isi{suffix}es. \isi{Object marker}s, though properly \isi{proclitic}s (and not \isi{prefix}es), are nevertheless so closely connected to the following verb, that they are written immediately preceding the verb, without any space. \isi{Phonological} rules that apply across this \isi{clitic} boundary are also reflected in the \isi{orthography}.

Finally, the basic \ili{English} (and \ili{Tok Pisin}) conventions of capitalization and punctuation have been adopted for Ulwa.

\is{orthography|)}

\section{Presentation of examples}\label{sec:1.3}

Throughout the work, numbered examples are presented in four lines. The first line consists of an utterance in Ulwa as written in the \isi{orthography}, including capitalization and punctuation. Words are spelled such that they reflect any word-internal \isi{phonological} rules that have applied. Parentheses indicate optional material (i.e., versions of the utterance both with and without the parenthetical material are attested).

The second line contains a morpheme-by-morpheme \isi{morphological} analysis of the utterance: morphemes are separated such that tabbed spaces are placed between all \isi{phonological} words, an equal sign (=) is placed between \isi{clitic}s and their hosts, and a dash (-) is placed between \isi{bound morpheme}s within a single grammatical word. Square brackets occasionally enclose one or more words to indicate clause boundaries (or, when indicated, to enclose other constituents, such as \isi{phrase}s). Brackets are also sometimes used (within words) to indicate suspected \isi{elide}d \isi{phonological} or \isi{morphological} material.

 The third line contains the \isi{morphological} gloss of the utterance. In glossing the language, I have followed the conventions of the Leipzig Glossing Rules \citep{ComrieEtAl2008}. Lexical items are given a basic \ili{English} translation. If more than one \ili{English} word is required to gloss a single Ulwa morpheme, then a period (.) is used to separate the words in a gloss (e.g., <older.brother> for \textit{atuma}). Functional morphemes are glossed with a standard abbreviation, written in \textsc{small} \textsc{capitals}. When a single morpheme encodes more than one grammatical feature, these are separated in the gloss by a period (e.g., <1\textsc{pl.excl}> for the first person \isi{plural} \isi{exclusive} \isi{pronoun} \textit{an}).

Finally, the fourth line provides a translation of the utterance into \ili{English}, usually designed to be as literal as possible, though still flowing. Although these translations are almost entirely in \ili{English}, an occasional word from \ili{Tok Pisin} will be used (in \textit{italics}) when it provides a clearer or more accurate translation of the Ulwa word (e.g., \ili{Tok Pisin} \textit{bilum} ‘string bag’ to translate Ulwa \textit{ani}). A glossary of these words is given in Appendix \ref{sec:app.d}.

Where further clarification or context is deemed helpful, this is provided, parenthetically, following the translation. \isi{Loanword}s from \ili{Tok Pisin} (TP) are identified in parentheses as well. The \ili{Tok Pisin} \isi{orthography} used here follows \citet{Volker2008}. More literal alternate translations are also sometimes provided within parentheses in this fourth line. Examples taken from recorded texts are identified as such, following the \ili{English} translation (and any other parenthetical explanations, if provided). The identification of texts takes the form “[ulwa000\_00:00]”: the numbers preceding the underscore refer to the text number as it occurs in ELAR \citep{Barlow2018b}; the numbers following the underscore refer to the time (minutes:seconds) in the recording where the quoted part of the speech begins. A list of these recordings is given in \tabref{tab:E.1}. Examples taken from elicitation sessions, on the other hand, are identified as “[elicited]”.

Ungrammatical utterances are indicated by an asterisk (*) at the start of the first line of the example. Note, however, that, elsewhere in the grammar, reconstructed forms are also indicated by an asterisk. This ambiguity is at times mitigated by using a superscript obelisk (\textsuperscript{†}) to indicate forms that may be expected to occur but do not. Note that a normal obelisk (†) is used when citing material from the text ulwa014 that is not included in the ELAR archive and thus does not have a timecode. A sentence preceded by a  question mark (?) is taken to be somehow doubtful: it may be grammatically acceptable but \isi{semantic}ally bizarre, or some speakers may be uncertain as to whether it is grammatical or not.

Morphemes, words, or \isi{phrase}s of particular relevance to the topic being discussed are emphasized in the first two lines of numbered examples by \textbf{bold} font. Prosody is generally not reflected in the transcriptions of examples.

\is{prosody}

  Whenever reference is made to forms in Ulwa, these forms are written in \textit{italics}. Where necessary, slashes (/…/) are used to enclose phonemic transcriptions and square brackets ([…]) are used to enclose \isi{phonetic} transcriptions. Angle bracket (<…>) are used when presenting a form exactly as written by someone else or when needed to draw special attention to the \isi{orthography} of a form.

\section{Organization of this book}\label{sec:1.4.1}

This remainder of this book is organized as follows. The following chapter (\chapref{sec:overview}) provides a brief grammar sketch, intended to equip the reader with the knowledge needed to understand the examples throughout this book. Then, the main body of the grammar proceeds from describing the \isi{phonetics} and \isi{phonology} of the language (\chapref{sec:2}) to detailing the \isi{morphology} of various \is{part of speech} parts of speech: nouns (\chapref{sec:3}), verbs (\chapref{sec:4}), \isi{adjective}s (\chapref{sec:5}), \isi{pronoun}s (\chapref{sec:6}), \isi{determiner}s (\chapref{sec:7}), and other smaller \isi{word class}es (\chapref{sec:8}). The grammar then describes Ulwa’s \isi{syntax}, beginning at the \isi{phrase} level (\chapref{sec:9}), with particular attention paid to the \isi{predicate} (\chapref{sec:10}), proceeding to the clause level (\chapref{sec:11}), covering \isi{complex sentence}s (\chapref{sec:12}), and then considering additional topics in \isi{syntax} (\chapref{sec:13}). This is followed by a discussion of some topics in \isi{semantics} (\chapref{sec:14}). Finally, the structural consequences of \isi{language loss} are discussed (\chapref{sec:15}).

Following these chapters of grammatical description, there is a \isi{lexicon} containing 1,429 entries, included as both an Ulwa-to-\ili{English} \isi{wordlist} and as an \ili{English}-to-Ulwa finder list (\chapref{sec:17}); a selection of three texts from the corpus of texts, transcribed in Ulwa with interlinear morpheme-by-morpheme glossing and translated into \ili{English} (\chapref{sec:16}); and a brief description of the \ili{Maruat-Dimiri-Yaul} \isi{dialect} of Ulwa (\chapref{sec:18}).

The following appendices are included at the end of the book: a \ia{Swadesh, Morris} \isi{Swadesh} 100-word list (Appendix \ref{sec:app.a}), a \isi{Swadesh} 200-word list (Appendix \ref{sec:app.b}), a standard SIL-PNG survey word list (Appendix \ref{sec:app.c}), a glossary of \ili{Tok Pisin} words encountered in this book (Appendix \ref{sec:app.d}), an account of the Ulwa people’s traditional origin stories (Appendix \ref{sec:app.f}), and a reproduction of \citegen{Laycock1971a} field notes on the \ili{Yaul} \isi{dialect} of Ulwa (Appendix \ref{sec:app.g}).