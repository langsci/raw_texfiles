%\chapter{Appendices}\label{sec:app}

\chapter{Swadesh 100-word list}\label{sec:app.a}

\is{wordlist|(}
\is{Swadesh List|(}

The following is a list of 100 \isi{basic vocabulary} items in Ulwa, following \citegen[283]{Swadesh1971} list of 100 words. Where deemed useful, alternate words or clarifications of meaning are provided in parentheses.\\

\begin{enumerate}[noitemsep, label={\arabic*}, align=left, widest=190, labelsep=1ex,leftmargin=*]
\item ‘I’ \textit{nï}

\item ‘you’ \textit{u} (2\textsc{sg;} \textit{ngun} 2\textsc{du,} \textit{un} \textsc{2pl})

\item ‘we’ \textit{an} (1\textsc{pl.excl;} \textit{unan} 1\textsc{pl.incl}, \textit{ngan} 1\textsc{du.excl,} \textit{ngunan} 1\textsc{du.incl})

\item ‘this’ \textit{nga}

\item ‘that’ \textit{anda}

\item ‘who’ \textit{kwa} (\textsc{sg;} \textit{kuma} \textsc{nsg)}

\item ‘what’ \textit{angos}

\item ‘not’ \textit{ango}

\item ‘all’ \textit{wopa}

\item ‘many’ \textit{tïngïn}

\item ‘one’ \textit{kwe} (also \textit{kwa})

\item ‘two’ \textit{nini}

\item ‘big’ \textit{ambi}

\item ‘long’ \textit{wutota}

\item ‘small’ \textit{njukuta}

\item ‘woman’ \textit{yena} (also \textit{yana})

\item ‘man’ \textit{yeta} (also \textit{yata})

\item ‘person’ \textit{ankam}

\item ‘fish’ \textit{wambana}

\item ‘bird’ \textit{uta}

\item ‘dog’ \textit{tïn}

\item ‘louse’ \textit{mïmin} (on humans; \textit{sïmin} refers to lice on animals)

\item ‘tree’ \textit{im}

\item ‘seed’ \textit{mu}

\item ‘leaf’ \textit{wapa}

\item ‘root’ \textit{ilu}

\item ‘bark’ \textit{im nambï} (= \textit{im} ‘tree’ + \textit{nambï} ‘skin’)

\item ‘skin’ \textit{nambï}

\item ‘flesh’ \textit{lam} (\isi{loan} from \ili{Ap Ma})

\item ‘blood’ \textit{anankïn}

\item ‘bone’ \textit{uma}

\item ‘grease’ \textit{anen}

\item ‘egg’ \textit{mïtïn}

\item ‘horn’ \textit{kokal} (actually ‘casque’, as of a cassowary)

\item ‘tail’ \textit{angun}

\item ‘feather’ \textit{nambli}

\item ‘hair’ \textit{wonmi} (hair on the top of the head; \textit{nil} refers to other hair)

\item ‘head’ \textit{unduwan}

\item ‘ear’ \textit{kïkal}

\item ‘eye’ \textit{lïmndï}

\item ‘nose’ \textit{ip}

\item ‘mouth’ \textit{mama}

\item ‘tooth’ \textit{ambla}

\item ‘tongue’ \textit{mïnïm}

\item ‘claw’ \textit{sinananangïn} (< \textit{sinanan} ‘nail’ + \textit{nangïn} ‘tongs’)

\item ‘foot’ \textit{wutï}

\item ‘knee’ \textit{wutï ambatïm} (= \textit{wutï} ‘leg, foot’ + \textit{ambatïm} ‘joint’)

\item ‘hand’ \textit{i}

\item ‘belly’ \textit{inapaw}

\item ‘neck’ \textit{um}

\item ‘breasts’ \textit{wol}

\item ‘heart’ \textit{yom}

\item ‘liver’ \textit{ina}

\item ‘drink’ \textit{ama-} (also means ‘eat’, ‘bite’)

\item ‘eat’ \textit{ama-} (also means ‘drink’, ‘bite’)

\item ‘bite’ \textit{ama-} (also means ‘drink’, ‘eat’)

\item ‘see’ \textit{lïmndï ala-} (= \textit{lïmndï} ‘eye’ + \textit{ala-} ‘see’)

\item ‘hear’ \textit{kïkal wana-} (= \textit{kïkal} ‘ear’ + \textit{wana-} ‘feel’)

\item ‘know’ \textit{kalamp} (= \textit{kalam} ‘knowledge’ + \textit{=p} ‘\textsc{cop}’; \isi{loan} from \ili{Waran})

\item ‘sleep’ \textit{wo-}

\item ‘die’ \textit{ni-}

\item ‘kill’ \textit{asa-} (also \textit{wali-})

\item ‘swim’ \textit{inim mo ma-} (= \textit{inim} ‘water’ + \textit{ma=} ‘3\textsc{sg.obj}’ + \textit{u} ‘on’ + \textit{ma-} ‘go’)

\item ‘fly’ \textit{wiwina-}

\item ‘walk’ \textit{inda-}

\item ‘come’ \textit{i-}

\item ‘lie’ \textit{lop ka-} (= \textit{lop} ‘lying’ + \textit{ka-} ‘let’?)

\item ‘sit’ \textit{asi ka-} (= \textit{asi} ‘seat’ + \textit{ka-} ‘let’?)

\item ‘stand’ \textit{tane lï-} (= \textit{tane} ‘stance’ + \textit{lï-} ‘put’?)

\item ‘give’ \textit{na-}

\item ‘say’ \textit{ta-} (also \textit{kï-})

\item ‘sun’ \textit{ane}

\item ‘moon’ \textit{iwïl}

\item ‘star’ \textit{nali} (small star; \textit{mbomala} refers to big stars)

\item ‘water’ \textit{inim} (also means ‘rain’)

\item ‘rain’ \textit{inim} (also means ‘water’)

\item ‘stone’ \textit{tana}

\item ‘sand’ \textit{tana isi}  (= \textit{tana} ‘stone’ + \textit{isi} ‘ashes’)

\item ‘earth’ \textit{ini}

\item ‘cloud’ \textit{ngïn}

\item ‘smoke’ \textit{apïn ngïn} (= \textit{apïn} ‘fire’ + \textit{ngïn} ‘cloud’)

\item ‘fire’ \textit{apïn}

\item ‘ash’ \textit{apïnsi} (< \textit{apïn} ‘fire’ + \textit{isi} ‘ashes’)

\item ‘burn’ \textit{wo-} (intransitive; transitive is \textit{apïn ama-} ‘eat [with] fire’)

\item ‘path’ \textit{tïlwa} (< \textit{utï} ‘leg, foot’ + \textit{luwa} ‘place’)

\item ‘mountain’ \textit{inkaw}

\item ‘red’ \textit{ngungun} (also refers to red ants and to a plant species with red seeds)

\item ‘green’ \textit{mïnal} (also refers to taro)

\item ‘yellow’ \textit{mïndit} (also \textit{andwana} and \textit{ane})

\item ‘white’ \textit{waembïl}

\item ‘black’ \textit{mbunmana} (also \textit{mbun}; \isi{loan} from \ili{Mwakai})

\item ‘night’ \textit{imba}

\item ‘hot’ \textit{wananum}

\item ‘cold’ \textit{mïnoma}

\item ‘full’ \textit{monop}

\item ‘new’ \textit{akïnaka}

\item ‘good’ \textit{anma}

\item ‘round’ \textit{wopaw}

\item ‘dry’ \textit{wapata}

\item ‘name’ \textit{wi}
\end{enumerate}

\is{Swadesh List|)}
\is{wordlist|)}


\chapter{Swadesh 200-word list}\label{sec:app.b}

\is{wordlist|(}
\is{Swadesh List|(}

The following is a list of 200 \isi{basic vocabulary} items in Ulwa, following \citegen[456--457]{Swadesh1952} list of 200 words. Where deemed useful, alternate words or clarifications of meaning are provided in parentheses.\\

\begin{enumerate}[noitemsep, label={\arabic*}, align=left, widest=190, labelsep=1ex,leftmargin=*]
\item ‘all’ \textit{wopa}

\item ‘and’ \textit{ma} (possibly a recent innovation)

\item ‘animal’ \textit{mundu}

\item ‘ashes’ \textit{apïnsi} (< \textit{apïn} ‘fire’ + \textit{isi} ‘ashes’)

\item ‘at’ \textit{ka} (also \textit{u})

\item ‘back’ \textit{mutam}

\item ‘bad’ \textit{tembi} (also means ‘dirty’)

\item ‘bark’ \textit{im nambï} (= \textit{im} ‘tree’ + \textit{nambï} ‘skin’)

\item ‘because’ \textit{angwena} (means ‘why’, but may function like ‘because’)

\item ‘belly’ \textit{inapaw}

\item ‘berry’ \textit{mu} (also means ‘seed’)

\item ‘big’ \textit{ambi}

\item ‘bird’ \textit{uta}

\item ‘to bite’ \textit{ama-} (also means ‘to drink’, ‘to eat’, ‘to suck’)

\item ‘black’ \textit{mbunmana} (also \textit{mbun}; \isi{loan} from \ili{Mwakai})

\item ‘blood’ \textit{anankïn}

\item ‘to blow’ \textit{nonalni-} (= \textit{nonal} ‘wind, breath’ + \textit{ni{}-} ‘do’)

\item ‘bone’ \textit{uma}

\item ‘to breathe’ \textit{nonal u-} (= \textit{nonal} ‘breath’ + \textit{u-} ‘put’)

\item ‘to burn’ \textit{wo-} (intransitive; transitive is \textit{apïn ama-} ‘eat [with] fire’; same form as ‘to swell’)

\item ‘child’ \textit{nungol} (also \textit{nungolke}, \textit{alum}, \textit{tawatïp})

\item ‘cloud’ \textit{ngïn}

\item ‘cold’ \textit{mïnoma}

\item ‘to come’ \textit{i-}

\item ‘to count’ \textit{ika uta-} (= \textit{ika} ‘instance’ + \textit{uta} ‘rub’)

\item ‘to cut’ \textit{lo-} (also \textit{nïkï-}, \textit{we u-}, \textit{won-})

\item ‘day’ \textit{ane} (‘sun, day’; \textit{ilom} refers to the countable unit of time)

\item ‘to die’ \textit{ni-}

\item ‘to dig’ \textit{nïkï-}

\item ‘dirty’ \textit{tembi} (also means ‘bad’)

\item ‘dog’ \textit{tïn}

\item ‘to drink’ \textit{ama-} (also means ‘to bite’, ‘to eat’, ‘to suck’)

\item ‘dry’ \textit{wapata} (also means ‘old’)

\item ‘dull’ \textit{tambumana} (also \textit{pon})

\item ‘dust’ \textit{itïtïl}

\item ‘ear’ \textit{kïkal}

\item ‘earth’ \textit{ini}

\item ‘to eat’ \textit{ama-} (also means ‘to bite’, ‘to drink’, ‘to suck’)

\item ‘egg’ \textit{mïtïn}

\item ‘eye’ \textit{lïmndï}

\item ‘to fall’ \textit{li u-} (= \textit{li} ‘down’ + \textit{u-} ‘put’)

\item ‘far’ \textit{ngaya}

\item ‘fat’ \textit{anen}

\item ‘father’ \textit{itom}

\item ‘to fear’ \textit{namnap} (= \textit{namna} ‘afraid’ + \textit{=p} ‘\textsc{cop}’)

\item ‘feather’ \textit{nambli}

\item ‘few’ \textit{ilum}

\item ‘to fight’ \textit{amblawali-} (= \textit{ambla=} ‘\textsc{pl.refl}’ + \textit{wali-} ‘hit’)

\item ‘fire’ \textit{apïn}

\item ‘fish’ \textit{wambana}

\item ‘five’ \textit{angay} (< \textit{anga} ‘piece’ + \textit{i} ‘hand’)

\item ‘to float’ \textit{ul watka-} (= \textit{ul} ‘with’ + \textit{wat} ‘atop’ + \textit{ka-} ‘let’)

\item ‘to flow’ \textit{ma-} (the verb \textit{ma-} ‘to go’ is used for ‘to flow’)

\item ‘flower’ \textit{woka} (‘banana flower’; there is no hypernym ‘flower’)

\item ‘to fly’ \textit{wiwina-}

\item ‘fog’ \textit{lïngïn} (\isi{loan} from \ili{Mwakai})

\item ‘foot’ \textit{wutï} (also means ‘leg’)

\item ‘four’ \textit{watangïnila} (< \textit{watangïn} ‘last’ + \textit{ila} ‘sago palm frond’)

\item ‘to freeze’ [there is no word that means ‘to freeze’]

\item ‘to give’ \textit{na-}

\item ‘good’ \textit{anma} (also means ‘straight’)

\item ‘grass’ \textit{asi}

\item ‘green’ \textit{mïnal} (also refers to taro)

\item ‘guts’ \textit{inji}

\item ‘hair’ \textit{wonmi} (hair on the top of the head; \textit{nil} refers to other hair)

\item ‘hand’ \textit{i}

\item ‘he’ \textit{mï}

\item ‘head’ \textit{unduwan}

\item ‘to hear’ \textit{kïkal wana-} (= \textit{kïkal} ‘ear’ + \textit{wana-} ‘feel’)

\item ‘heart’ \textit{yom}

\item ‘heavy’ \textit{kenmbu}

\item ‘here’ \textit{mbï}

\item ‘to hit’ \textit{wali-} (also \textit{asa-}; both forms also mean ‘to kill’, ‘to stab’)

\item ‘to hold’ \textit{ikali lï-} (= \textit{i} ‘hand’ + \textit{kali} ‘send’ + \textit{lï-} ‘put’)

\item ‘how’ \textit{anjikaka}

\item ‘to hunt’ \textit{anglalo-} (= \textit{angla} ‘awaiting’ + \textit{lo-} ‘go’; also \textit{andïlalo-})

\item ‘husband’ \textit{numan}

\item ‘I’ \textit{nï}

\item ‘ice’ [there is no word that means ‘ice’]

\item ‘if’ \textit{{}-ta} (verbal \isi{suffix} that signals the \isi{apodosis} of a condition, ‘\textsc{cond}’)

\item ‘in’ \textit{in} (also \textit{ka}, \textit{u})

\item ‘to kill’ \textit{asa-} (also \textit{wali-}; both forms also mean ‘to hit’, ‘to stab’)

\item ‘to know’ \textit{kalamp} (= \textit{kalam} ‘knowledge’ + \textit{=p} ‘\textsc{cop}’; \isi{loan} from \ili{Waran})

\item ‘lake’ \textit{inimpul} (= \textit{inim} ‘water’ + \textit{pul} ‘piece’)

\item ‘to laugh’ \textit{atal a-} (= \textit{atal} ‘laughter’ + \textit{a-} ‘break’?)

\item ‘leaf’ \textit{wapa} (same form as ‘wing’)

\item ‘left’ \textit{andana}

\item ‘leg’ \textit{wutï} (also means ‘foot’)

\item ‘to lie’ \textit{lop ka-} (= \textit{lop} ‘lying’ + \textit{ka-} ‘let’?)

\item ‘to live’ \textit{p-} (\isi{locative verb})

\item ‘liver’ \textit{ina}

\item ‘long’ \textit{wutota}

\item ‘louse’ \textit{mïmin} (on humans; \textit{sïmin} refers to lice on animals)

\item ‘man’ \textit{yeta} (also \textit{yata})

\item ‘many’ \textit{tïngïn}

\item ‘meat’ \textit{lam} (\isi{loan} from \ili{Ap Ma})

\item ‘mother’ \textit{inom}

\item ‘mountain’ \textit{inkaw}

\item ‘mouth’ \textit{mama}

\item ‘name’ \textit{wi}

\item ‘narrow’ \textit{njukuta} (also means ‘small’, ‘thin’)

\item ‘near’ \textit{nu}

\item ‘neck’ \textit{um}

\is{Swadesh List|)}
\is{wordlist|)}
\is{wordlist|(}
\is{Swadesh List|(}

\item ‘new’ \textit{akïnaka}

\item ‘night’ \textit{imba}

\item ‘nose’ \textit{ip}

\item ‘not’ \textit{ango}

\item ‘old’ \textit{wapata} (also means ‘dry’)

\item ‘one’ \textit{kwe} (also \textit{kwa})

\item ‘other’ \textit{kwa} (actually ‘one; someone; who?’, the closest equivalent)

\item ‘person’ \textit{ankam}

\item ‘to play’ \textit{sini-} (= \textit{si-} ‘push’ + \textit{ni-} ‘do’?)

\item ‘to pull’ \textit{angom lï-} (= \textit{angom} ‘pull’ + \textit{lï-} ‘put’?)

\item ‘to push’ \textit{si-}

\item ‘to rain’ \textit{lopo-} (also means ‘to wash’)

\item ‘red’ \textit{ngungun} (also refers to red ants and to a plant species with red seeds)

\item ‘right (correct)’ \textit{maw}

\item ‘right (hand)’ \textit{inapum}

\item ‘river’ \textit{nïmal}

\item ‘road’ \textit{tïlwa} (< \textit{utï} ‘leg, foot’ + \textit{luwa} ‘place’)

\item ‘root’ \textit{ilu}

\item ‘rope’ \textit{nïpïl}

\item ‘rotten’ \textit{mïnwata} (also means ‘wet’)

\item ‘to rub’ \textit{uta-} (also means ‘to wipe’)

\item ‘salt’ \textit{isi} (native ‘salt’, made from burnt banana leaves)

\item ‘sand’  \textit{tana isi} (= \textit{tana} ‘stone’ + \textit{isi} ‘ashes’)

\item ‘to say’ \textit{ta-} (also \textit{kï-})

\item ‘to scratch’ \textit{ana-}

\item ‘sea’ \textit{angumoni nïmal} (= \textit{angumoni} ‘swelling’ + \textit{nïmal} ‘river’)

\item ‘to see’ \textit{lïmndï ala-} (= \textit{lïmndï} ‘eye’ + \textit{ala-} ‘see’)

\item ‘seed’ \textit{mu} (also means ‘berry’)

\item ‘to sew’ \textit{me-}

\item ‘sharp’ \textit{matamal}

\item ‘short’ \textit{mundotoma}

\item ‘to sing’ \textit{kawni-} (= \textit{kaw} ‘song’ + \textit{ni-} ‘do’)

\item ‘to sit’ \textit{asi ka-} (= \textit{asi} ‘seat’ + \textit{ka-} ‘let’?)

\item ‘skin’ \textit{nambï}

\item ‘sky’ \textit{anam}

\item ‘to sleep’ \textit{wo-}

\item ‘small’ \textit{njukuta} (also means ‘narrow’, ‘thin’)

\item ‘to smell’ \textit{nambït wana-} (= \textit{nambït} ‘odor’ + \textit{wana-} ‘feel’)

\item ‘smoke’ \textit{apïn ngïn} (= \textit{apïn} ‘fire’ + \textit{ngïn} ‘cloud’)

\item ‘smooth’ \textit{namli}

\item ‘snake’ \textit{anmoka}

\item ‘snow’ [there is no word that means ‘snow’]

\item ‘some’ \textit{kuma}

\item ‘to spit’ \textit{ngom lï-} (= \textit{ngom} ‘spitting’ + \textit{lï-} ‘put’?)

\item ‘to split’ \textit{kol-}

\item ‘to squeeze’ \textit{mïmïl u-} (= \textit{mïmïl} ‘squeeze’ + \textit{u-} ‘put’?)

\item ‘to stab’ \textit{asa-} (also \textit{wali-}; both forms also mean ‘to hit’, ‘to kill’)

\item ‘to stand’ \textit{tane lï-} (= \textit{tane} ‘stance’ + \textit{lï-} ‘put’?)

\item ‘star’ \textit{nali} (small star; \textit{mbomala} refers to big stars)

\item ‘stick’ \textit{im nali} (= \textit{im} ‘tree’ + \textit{nali} ‘sago frond spine’)

\item ‘stone’ \textit{tana}

\item ‘straight’ \textit{anma} (also means ‘good’)

\item ‘to suck’ \textit{ama-} (also means ‘to bite’, ‘to drink’, ‘to eat’)

\item ‘sun’ \textit{ane}

\item ‘to swell’ \textit{wo-} (same form as ‘to burn’)

\item ‘to swim’ \textit{inim mo ma-} (= \textit{inim} ‘water’ + \textit{ma} ‘3\textsc{sg.obj}’ + \textit{u} ‘on’ + \textit{ma-} ‘go’)

\item ‘tail’ \textit{angun}

\item ‘that’ \textit{anda}

\item ‘there’ \textit{ando} (= \textit{anda} ‘that’ + \textit{u} ‘from, in, at, around, along’)

\item ‘they’ \textit{ndï} (3\textsc{pl;} \textit{min} 3\textsc{du})

\item ‘thick’ \textit{palmana} (also means ‘wide’)

\item ‘thin’ \textit{njukuta} (also means ‘small’, ‘narrow’)

\item ‘to think’ \textit{inakawana}{}- (= \textit{ina} ‘liver’ + \textit{ka} ‘in’ + \textit{wana-} ‘feel’)

\item ‘this’ \textit{nga}

\item ‘thou’ \textit{u}

\item ‘three’ \textit{lele}

\item ‘to throw’ \textit{kïke u-} (= \textit{kïke} ‘throwing’ + \textit{u-} ‘put’?; also \textit{kuli lï-}, \textit{mune u-}, \textit{top lï-})

\item ‘to tie’ \textit{mop lï-} (= \textit{mop} ‘tying’ + \textit{lï-} ‘put’?; also \textit{ita-} ‘build, tie’)

\item ‘tongue’ \textit{mïnïm}

\item ‘tooth’ \textit{ambla}

\item ‘tree’ \textit{im}

\item ‘to turn’ \textit{tïkli ka-} (= \textit{tïkli} ‘turn’ + \textit{ka-} ‘let’?)

\item ‘two’ \textit{nini}

\item ‘to vomit’ \textit{nongan u-} (= \textit{nongan} ‘vomitus’ + \textit{u-} ‘put’)

\item ‘to walk’ \textit{inda-}

\item ‘warm’ \textit{wananum}

\item ‘to wash’ \textit{lopo-} (also means ‘to rain’)

\item ‘water’ \textit{inim} (also means ‘year’)

\item ‘we’ \textit{an} (1\textsc{pl.excl;} \textit{unan} 1\textsc{pl.incl}, \textit{ngan} 1\textsc{du.excl,} \textit{ngunan} 1\textsc{du.incl})

\item ‘wet’ \textit{mïnwata} (also means ‘rotten’)

\item ‘what?’ \textit{angos}

\item ‘when?’ \textit{ango tem} (= \textit{ango} ‘which?’ + \textit{tem} ‘time’ < \ili{Tok Pisin} \textit{taim} ‘time’)

\item ‘where?’ \textit{ango luwa} (= \textit{ango} ‘which?’ + \textit{luwa} ‘place’)

\item ‘white’ \textit{waembïl}

\item ‘who?’ \textit{kwa} (\textsc{sg;} \textit{kuma} \textsc{nsg)}

\item ‘wide’ \textit{palmana} (also means ‘thick’)

\item ‘wife’ \textit{yenanu} (also \textit{yananu}, \textit{yena}, \textit{yana}; all also mean ‘woman’)

\item ‘wind’ \textit{nonal}

\item ‘wing’ \textit{wapa} (same form as ‘leaf’)

\item ‘to wipe’ \textit{uta-} (also means ‘to rub’)

\item ‘with’ \textit{ul} (\isi{comitative} \isi{postposition}; \isi{instrumental} meaning can be indicated with \isi{oblique marker} \textit{=n} ‘\textsc{obl}’)

\item ‘woman’ \textit{yena} (also \textit{yana}, \textit{yenanu}, \textit{yananu}; all also mean ‘wife’)

\item ‘woods’ \textit{wandam}

\item ‘worm’ \textit{utal}

\item ‘ye’ \textit{un} (2\textsc{pl}; \textit{ngun} 2\textsc{du})

\item ‘year’ \textit{inim} (also means ‘water’)

\item ‘yellow’ \textit{mïndit} (also \textit{andwana}, \textit{ane})
\end{enumerate}

Compared to an older study with 215 words (\citealt{Swadesh1950}), there are 17 items that are not included among these 200 words.\footnote{\citet[457]{Swadesh1952} mentions having excluded 16 words, but he seems to have miscounted, apparently having omitted reference to ‘spear (war)’, which was also among the original 215 words \citep[161]{Swadesh1950}. Two items occur in the 200-word list that were not included in the 215-word list: ‘to say’ (a replacement for ‘to speak’) and ‘heavy’ (added to create an even 200 items).} The closest Ulwa forms for these 17 \isi{lexical} concepts are as follows.

\begin{enumerate}[noitemsep, label={\arabic*}, align=left, widest=190, labelsep=1ex,leftmargin=*]
\item ‘six’ \textit{angay kwe kwe mowon ndïwatlïp} (= 5⋅1+1)

\item ‘seven’ \textit{angay kwe nini minwon ndïwatlïp} (= 5⋅1+2)

\item ‘eight’ \textit{angay kwe lele ndïwon ndïwatlïp} (= 5⋅1+3)

\item ‘nine’ \textit{angay kwe watangïnila ndïwon ndïwatlïp} (= 5⋅1+4)

\item ‘ten’ \textit{angay nini} (= 5⋅2; also \textit{nali} ‘sago frond spine’)

\item ‘twenty’ \textit{angay watangïnila} (= 5⋅4; also \textit{nali nini} ‘two sago frond spines’, \linebreak \textit{lamndu unduwan} ‘pig head’)

\item ‘hundred’ \textit{uta} (also means ‘bird’)

\item ‘brother’ \textit{atuma} (refers to older brothers; there is no hypernym for ‘brother’)

\item ‘clothing’ \textit{al nambï} (= \textit{al} ‘mosquito net’ + \textit{nambï} ‘skin’)

\item ‘to cook’ \textit{wana-}

\item ‘to cry’ \textit{sa-}

\item ‘to dance’ \textit{wutïni-} (= \textit{wutï} ‘leg, foot’ + \textit{ni-} ‘beat’)

\item ‘to shoot’ \textit{asa-} (also \textit{wali-}; both forms also mean ‘to hit’, ‘to kill’, ‘to stab’)

\item ‘sister’ \textit{anapa}

\item ‘to speak’ \textit{ta-} (also \textit{kï-}; both forms also mean ‘to say’)

\item ‘spear’ \textit{mana} (also \textit{lungum})

\item ‘to work’ \textit{wombïn ni-} (= \textit{wombïn} ‘work’ + \textit{ni-} ‘do’)
\end{enumerate}

\is{Swadesh List|)}
\is{wordlist|)}

\chapter{Standard SIL-PNG survey word list (190 items)}\label{sec:app.c}

\is{wordlist|(}

The following is a list of 190 items (170 words and 20 sentences) in Ulwa, based on the standard survey word list used by SIL in Papua New Guinea. The list, developed by \citet{BeePence1962}, was revised in 1999 such that the items are grouped according to \isi{semantic} domains. Where deemed useful, alternate words or clarifications of meaning are provided in parentheses.

\begin{enumerate}[noitemsep, label={\arabic*}, align=left, widest=190, labelsep=1ex,leftmargin=*]
\item  ‘head’ \textit{unduwan}

\item  ‘hair’ \textit{wonmi} (hair on the top of the head; \textit{nil} refers to other hair)

\item  ‘mouth’ \textit{mama}

\item  ‘nose’ \textit{ip}

\item  ‘eye’ \textit{lïmndï}

\item  ‘neck’ \textit{um}

\item  ‘belly’ \textit{inapaw}

\item  ‘skin’ \textit{nambï}

\item  ‘knee’ \textit{wutï ambatïm} (= \textit{wutï} ‘leg, foot’ + \textit{ambatïm} ‘joint’)

\item  ‘ear’ \textit{kïkal}

\item  ‘tongue’ \textit{mïnïm}

\item  ‘tooth’ \textit{ambla}

\item  ‘breast’ \textit{wol}

\item  ‘hand’ \textit{i}

\item  ‘foot’ \textit{wutï} (also means ‘leg’)

\item  ‘back’ \textit{mutam}

\item  ‘shoulder’ \textit{awi}

\item  ‘forehead’ \textit{monombam}

\item  ‘chin’ \textit{ngïnïm}

\item  ‘elbow’ \textit{inpu}

\item  ‘thumb’ \textit{imu unduwan} (= \textit{i} ‘hand’ + \textit{mu} ‘fruit’ + \textit{unduwan} ‘head’)

\item  ‘leg’ \textit{wutï} (also means ‘foot’)

\item  ‘heart’ \textit{yom}

\item  ‘liver’ \textit{ina}

\item  ‘bone’ \textit{uma}

\item  ‘blood’ \textit{anankïn}

\item  ‘baby’ \textit{alum} (includes older children; also \textit{nungol}, \textit{nungolke}, \textit{tawatïp})

\item  ‘girl’ \textit{yenalum} (also \textit{yena}, \textit{yana}; or any words for ‘child’)

\item  ‘boy’ \textit{yetalum} (also \textit{yeta}, \textit{yata}; or any words for ‘child’)

\item  ‘old woman’ \textit{inom ngata} (= \textit{inom} ‘mother’ + \textit{ngata} ‘grand’)

\item  ‘old man’ \textit{itom ngata} (= \textit{itom} ‘father’ + \textit{ngata} ‘grand’)

\item  ‘woman’ \textit{yena} (also \textit{yana})

\item  ‘man’ \textit{yeta} (also \textit{yata})

\item  ‘father’ \textit{itom}

\item  ‘mother’ \textit{inom}

\item  ‘brother’ \textit{atuma} (refers to older brothers; there is no hypernym for ‘brother’)

\item  ‘sister’ \textit{anapa}

\item  ‘name’ \textit{wi}

\item  ‘bird’ \textit{uta}

\item  ‘dog’ \textit{tïn}

\item  ‘pig’ \textit{lamndu} (also \textit{namndu})

\item  ‘cassowary’ \textit{kalim} (\isi{loan} from \ili{Mundukumo})

\item  ‘wallaby’ \textit{wakan} (likely an \isi{areal term}, perhaps ultimately from \ili{Austronesian})

\item  ‘flying fox’ \textit{nïplopa}

\item  ‘rat’ \textit{wala} (also \textit{matlaka}, \textit{mblandu})

\item  ‘frog’ \textit{womotana}

\item  ‘snake’ \textit{anmoka}

\item  ‘fish’ \textit{wambana} (possibly a \isi{loan} from \ili{Mwakai})

\item  ‘person’ \textit{ankam}

\item  ‘to sit’ \textit{asi ka-} (= \textit{asi} ‘seat’ + \textit{ka-} ‘let’?)

\item  ‘to stand’ \textit{tane lï-} (= \textit{tane} ‘stance’ + \textit{lï-} ‘put’?)

\item  ‘to lie down’ \textit{lop ka-} (= \textit{lop} ‘lying’ + \textit{ka-} ‘let’?)

\item  ‘to sleep’ \textit{wo-}

\item  ‘to walk’ \textit{inda-}

\item  ‘to bite’ \textit{ama-} (also means ‘to eat’, ‘to drink’)

\item  ‘to eat’ \textit{ama-} (also means ‘to bite’, ‘to drink’)

\item  ‘to give’ \textit{na-}

\item  ‘to see’ \textit{lïmndï ala-} (= \textit{lïmndï} ‘eye’ + \textit{ala-} ‘see’)

\item  ‘to come’ \textit{i-}

\item  ‘to say’ \textit{ta-} (also \textit{kï-})

\item  ‘to hear’ \textit{kïkal wana-} (= \textit{kïkal} ‘ear’ + \textit{wana-} ‘feel’)

\item  ‘to know’ \textit{kalamp} (= \textit{kalam} ‘knowledge’ + \textit{=p} ‘\textsc{cop}’; \isi{loan} from \ili{Waran})

\item  ‘to drink’ \textit{ama-} (also means ‘to bite, ‘to eat’)

\item  ‘to hit’ \textit{wali-} (also \textit{asa-}; both forms also mean ‘to kill’)

\item  ‘to kill’ \textit{asa-} (also \textit{wali{}-}; both forms also mean ‘to hit’)

\item  ‘to die’ \textit{ni-}

\item  ‘to burn’ \textit{wo-} (intransitive; transitive is \textit{apïn ama-} ‘eat [with] fire’)

\item  ‘to fly’ \textit{wiwina-}

\item  ‘to swim’ \textit{inim mo ma-} (= \textit{inim} ‘water’ + \textit{ma} ‘3\textsc{sg.obj}’ + \textit{u} ‘on’ + \textit{ma-} ‘go’)

\item  ‘to run’ \textit{imbam ka-} (= \textit{imbam} ‘under’ + \textit{ka-} ‘let’)

\item  ‘to fall down’  \textit{li u-} (= \textit{li} ‘down’ + \textit{u-} ‘put’)

\item  ‘to catch’ \textit{ikali lï-} (= \textit{i} ‘hand’ + \textit{kali} ‘send’ + \textit{lï-} ‘put’)

\item  ‘to cough’ \textit{utan uta} (= \textit{utan} ‘cough’ + \textit{uta} ‘rub’)

\item  ‘to laugh’ \textit{atal a-} (= \textit{atal} ‘laughter’ + \textit{a-} ‘break’?)

\item  ‘to dance’ \textit{wutïni-} (= \textit{wutï} ‘leg, foot’ + \textit{ni-} ‘beat’)

\item  ‘big’ \textit{ambi}

\item  ‘small’ \textit{njukuta}

\item  ‘good’ \textit{anma}

\item  ‘bad’ \textit{tembi}

\item  ‘long’ \textit{wutota}

\item  ‘short’ \textit{mundotoma}

\item  ‘heavy’ \textit{kenmbu}

\item  ‘light’ \textit{wiwila}

\item  ‘cold’ \textit{mïnoma}

\item  ‘hot’ \textit{wananum}

\item  ‘new’ \textit{akïnaka}

\item  ‘old’ \textit{wapata} (also means ‘dry’)

\item  ‘round’ \textit{wopaw}

\item  ‘wet’ \textit{mïnwata}

\item  ‘dry’ \textit{wapata} (also means ‘old’)

\item  ‘full’ \textit{monop}

\is{wordlist|)}
\is{wordlist|(}

\item  ‘road’ \textit{tïlwa} (< \textit{utï} ‘leg, foot’ + \textit{luwa} ‘place’)

\item  ‘stone’ \textit{tana}

\item  ‘earth’ \textit{ini}

\item  ‘sand’ \textit{tana isi} (= \textit{tana} ‘stone’ + \textit{isi} ‘ashes’)

\item  ‘mountain’ \textit{inkaw}

\item  ‘fire’ \textit{apïn}

\item  ‘smoke’ \textit{apïn ngïn} (= \textit{apïn} ‘fire’ + \textit{ngïn} ‘cloud’)

\item  ‘ashes’ \textit{apïnsi} (< \textit{apïn} ‘fire’ + \textit{isi} ‘ashes’)

\item  ‘sun’ \textit{ane}

\item  ‘moon’ \textit{iwïl}

\item  ‘star’ \textit{nali} (small star; \textit{mbomala} refers to big stars)

\item  ‘cloud’ \textit{ngïn}

\item  ‘rain’ \textit{inim} (also means ‘water’)

\item  ‘wind’ \textit{nonal}

\item  ‘water’ \textit{inim} (also means ‘rain’)

\item  ‘vine’ \textit{nïpïl}

\item  ‘tree’ \textit{im}

\item  ‘stick’ \textit{im nali} (= \textit{im} ‘tree’ + \textit{nali} ‘sago frond spine’)

\item  ‘bark’ \textit{im nambï} (= \textit{im} ‘tree’ + \textit{nambï} ‘skin’)

\item  ‘seed’ \textit{mu}

\item  ‘root’ \textit{ilu}

\item  ‘leaf’ \textit{wapa} (same form as ‘wing’)

\item  ‘meat’ \textit{lam} (\isi{loan} from \ili{Ap Ma})

\item  ‘fat’ \textit{anen}

\item  ‘egg’ \textit{mïtïn}

\item  ‘louse’ \textit{mïmin} (on humans; \textit{sïmin} refers to lice on animals)

\item  ‘feather’ \textit{nambli}

\item  ‘horn’ \textit{kokal} (actually ‘casque’, as of a cassowary)

\item  ‘wing’ \textit{wapa} (same form as ‘leaf’)

\item  ‘claw’ \textit{sinananangïn} (< \textit{sinanan} ‘nail’ + \textit{nangïn} ‘tongs’)

\item  ‘tail’ \textit{angun}

\item  ‘one’ \textit{kwe} (also \textit{kwa})

\item  ‘two’ \textit{nini}

\item  ‘three’ \textit{lele}

\item  ‘four’ \textit{watangïnila} (< \textit{watangïn} ‘last’ + \textit{ila} ‘sago palm frond’)

\item  ‘five’ \textit{angay} (< \textit{anga} ‘piece’ + \textit{i} ‘hand’)

\item  ‘ten’ \textit{angay nini} (= \textit{angay} ‘five’ [times] \textit{nini} ‘two’; also \textit{nali} ‘sago frond spine’)

\item  ‘taro’ \textit{mïnal} (also means ‘green’)

\item  ‘sugarcane’ \textit{mil}

\item  ‘yam’ \textit{utam}

\item  ‘banana’ \textit{mïnda}

\item  ‘sweet potato’ \textit{nongontam}

\item  ‘bean’ \textit{yakeka}

\item  ‘axe’ \textit{tana} (literally ‘stone’)

\item  ‘knife’ \textit{yawt} (large knife or machete; also \textit{yot}; \textit{sina} ‘bamboo species’ refers to a smaller knife)

\item  ‘arrow’ \textit{wipam} (also \textit{nap}; \textit{wongïta} refers to a bow and arrow)

\item  ‘net bag’ \textit{ani}

\item  ‘house’ \textit{apa}

\item  ‘tobacco’ \textit{sokoy} (\isi{areal term})

\item  ‘morning’ \textit{umbenam}

\item  ‘afternoon’ \textit{awal}

\item  ‘night’ \textit{imba}

\item  ‘yesterday’ \textit{awal}

\item  ‘tomorrow’ \textit{umbe}

\item  ‘white’ \textit{waembïl}

\item  ‘black’ \textit{mbunmana} (also \textit{mbun}; \isi{loan} from \ili{Mwakai})

\item  ‘yellow’ \textit{mïndit} (also \textit{andwana}, \textit{ane})

\item  ‘red’ \textit{ngungun} (also refers to red ants and to a plant species with red seeds)

\item  ‘green’ \textit{mïnal} (also means ‘taro’)

\item  ‘many’ \textit{tïngïn}

\item  ‘all’ \textit{wopa}

\item  ‘this’ \textit{nga}

\item  ‘that’ \textit{anda}

\item  ‘what?’ \textit{angos}

\item  ‘who?’ \textit{kwa} (\textsc{sg;} \textit{kuma} \textsc{nsg)}

\item  ‘when?’ \textit{ango tem} (= \textit{ango} ‘which?’ + \textit{tem} ‘time’ < \ili{Tok Pisin} \textit{taim} ‘time’)

\item  ‘where?’ \textit{ango luwa} (= \textit{ango} ‘which?’ + \textit{luwa} ‘place’)

\item  ‘yes’ \textit{iyo}

\item  ‘no’ \textit{ase}

\item  ‘not’ \textit{ango}

\item  ‘I’ \textit{nï}

\item  ‘you (singular)’ \textit{u}

\item  ‘he’ \textit{mï}

\item  ‘we two’ \textit{ngan} (\textsc{1du.excl}; \textit{ngunan} \textsc{1du.incl})

\item  ‘you two’ \textit{ngun}

\item  ‘they two’ \textit{min}

\item  ‘we’ \textit{an} (\textsc{1pl.excl}; \textit{unan} \textsc{1pl.incl})

\item  ‘you (plural)’ \textit{un}

\item  ‘they’ \textit{ndï}
\end{enumerate}

\noindent{171  ‘He is hungry.’}
\begin{quote}\textit{Mundu mase.}\\
\gll    mundu  ma=asa-e\\
hunger  3\textsc{sg.obj}=hit-\textsc{ipfv}\\
\glt    ‘Hunger is hitting him/her.’\end{quote}

\newpage

\noindent{172  ‘He eats sugarcane.’}
\begin{quote}\textit{Mï mil ame.}\\
\gll    mï      mil    ama-e\\
    \textsc{3sg.subj}  sugar  eat-\textsc{ipfv}\\
\glt    ‘He/she eats sugarcane.’\end{quote}

\noindent{173  ‘He laughs a lot.’}
\begin{quote}\textit{Mï nunu ika atalaye.}\\
\gll    mï      nunu  ika      atal-a-e\\
    \textsc{3sg.subj}  every  instance  laughter-break-\textsc{ipfv}\\
\glt    ‘He/she laughs often.’\end{quote}

\noindent{174  ‘One man stands.’}
\begin{quote}\textit{Yeta kwe tanelïp.}\\
\gll    yeta  kwe  tane-lï-p\\
    man  one    stand-put-\textsc{pfv}\\
\glt    ‘One man stands.’\footnote{More colloquial Ulwa would more likely use \textit{ankam} ‘person’ in sentences such as this and the following. The \isi{singular} \isi{subject marker} \textit{mï} ‘3\textsc{sg.subj}’ could be used instead of the \isi{numeral} \textit{kwe} {\textasciitilde} \textit{kwa} ‘one’.}\end{quote}

\noindent{175  ‘Two men stand.’}
\begin{quote}\textit{Yeta nini tanelïp.}\\
\gll    yeta  nini  tane-lï-p\\
    man  two  stand-put-\textsc{pfv}\\
\glt    ‘Two men stand.’\footnote{The \isi{dual} \isi{subject marker} \textit{min} ‘\textsc{3du}’ could be used instead of the \isi{numeral} \textit{nini} ‘two’.}\end{quote}

\noindent{176  ‘Three men stand.'}
\begin{quote}\textit{Yeta lele tanelïp.}\\
\gll    yeta  lele    tane-lï-p\\
    man  three  stand-put-\textsc{pfv}\\
\glt    ‘Three men stand.’\end{quote}

\newpage

\noindent{177  ‘The man goes.’}
\begin{quote}\textit{Yeta mï man.}\\
\gll    yeta  mï      ma-n\\
    man  3\textsc{sg.subj}  go-\textsc{ipfv}\\
\glt    ‘The man goes.’\end{quote}

\noindent{178  ‘The man went yesterday.’}
\begin{quote}\textit{Yeta mï awal i.}\\
\gll    yeta  mï      awal    i\\
    man  \textsc{3sg.subj}  yesterday  go.\textsc{pfv}\\
\glt    ‘The man went yesterday.’\end{quote}

\noindent{179  ‘The man will go tomorrow.’}
\begin{quote}\textit{Yeta mï umbe mana.}\\
\gll    yeta  mï      umbe    ma-na\\
    man  \textsc{3sg.subj}  tomorrow  go-\textsc{irr}\\
\glt    ‘The man will go tomorrow.’\end{quote}

\noindent{180  ‘The man eats the yam.’}
\begin{quote}\textit{Yeta mï utam mame.}\\
\gll    yeta  mï      utam  ma=ama-e\\
    man  \textsc{3sg.subj}  yam  3\textsc{sg.obj}=eat-\textsc{ipfv}\\
\glt    ‘The man eats the yam.’\end{quote}

\noindent{181  ‘The man ate the yam yesterday.’}
\begin{quote}\textit{Yeta mï awal utam mamap.}\\
\gll    yeta  mï      awal    utam  ma=ama-p\\
    man  \textsc{3sg.subj}  yesterday  yam  3\textsc{sg.obj}=eat-\textsc{pfv}\\
\glt    ‘The man ate the yam yesterday.’\end{quote}

\noindent{182  ‘The man will eat the yam tomorrow.’}
\begin{quote}\textit{Yeta mï umbe utam malanda.}\\
\gll    yeta  mï      umbe    utam  ma=la-nda\\
    man  \textsc{3sg.subj}  tomorrow  yam  3\textsc{sg.obj}=eat-\textsc{irr}\\
\glt    ‘The man will eat the yam tomorrow.’\end{quote}

\noindent{183  ‘The man hit the dog.’}
\begin{quote}\textit{Yeta mï tïn masap.}\\
\gll    yeta  mï      tïn    ma=asa-p\\
    man  \textsc{3sg.subj}  dog  \textsc{3sg.obj}=hit-\textsc{pfv}\\
\glt    ‘The man hit the dog.’\end{quote}

\noindent{184  ‘The man didn’t hit the dog.’}
\begin{quote}\textit{Yeta mï ango tïn masap.}\\
\gll    yeta  mï      ango  tïn    ma=asa-p\\
    man  \textsc{3sg.subj}  \textsc{neg}  dog  \textsc{3sg.obj}=hit-\textsc{pfv}\\
\glt    ‘The man didn’t hit the dog.’\end{quote}

\noindent{185  ‘The big man hit the little dog.’}
\begin{quote}\textit{Yeta ambi mï tïn njukuta masap.}\\
\gll    yeta  ambi  mï      tïn    njukuta  ma=asa-p\\
    man  big    \textsc{3sg.subj}  dog  small    \textsc{3sg.obj}=hit-\textsc{pfv}\\
\glt    ‘The big man hit the small dog.’\end{quote}

\noindent{186  ‘The man gave the dog to the boy.’}
\begin{quote}\textit{Yeta mï tïn matï nungol manan.}\\
\gll    yeta  mï      tïn    ma=tï      nungol  ma=na-n\\
    man  \textsc{3sg.subj}  dog  \textsc{3sg.obj}=take  child  \textsc{3sg=obj}=give-\textsc{pfv}\\
\glt    ‘The man gave the dog to the child.’\end{quote}

\noindent{187  ‘The man hit the dog and went.’}
\begin{quote}\textit{Yeta mï tïn masap i.}\\
\gll    yeta  mï      tïn    ma=asa-p      i\\
    man  \textsc{3sg.subj}  dog  \textsc{3sg.obj}=hit-\textsc{pfv}  go.\textsc{pfv}\\
\glt    ‘The man hit the dog [and] went.’\end{quote}

\noindent{188  ‘The man hit the dog when the boy went.’}
\begin{quote}\textit{Nungol iye yeta mï tïn masap.}\\
\gll    nungol  i-e        yeta  mï      tïn    ma=asa-p\\
    child  go.\textsc{pfv-dep}  man  \textsc{3sg.subj}  dog  \textsc{3sg.obj}=hit-\textsc{pfv}\\
\glt    ‘When the child went, the man hit the dog.’\footnote{The order of the clauses in the translation is reversed, reflecting more natural Ulwa \isi{syntax}. The Ulwa sentence could also mean: ‘After the child went, the man hit the dog’.}\end{quote}

\noindent{189  ‘The man hit the dog and it went.’}
\begin{quote}\textit{Yeta mi tïn masap mï i.}\\
\gll    yeta  mi      tïn    ma=asa-p      mï      i\\
    man  \textsc{3sg.subj}  dog  \textsc{3sg.obj}=hit-pfv  \textsc{3sg.subj}  go.\textsc{pfv}\\
\glt    ‘The man hit the dog [and] it went.’\end{quote}

\noindent{190  ‘The man shot and ate the pig.’}
\begin{quote}\textit{Yeta mï lamndu masap mamap.}\\
\gll    yeta  mï      lamndu  ma=asa-p      ma=ama-p\\
    man  \textsc{3sg.subj}  pig      \textsc{3sg.obj}=hit-\textsc{pfv}  \textsc{3sg.obj}=eat-\textsc{pfv}\\
\glt    ‘The man shot [and] ate the pig.’\end{quote}

\is{wordlist|)}

\chapter{Glossary of Tok Pisin words}\label{sec:app.d}

The following is an annotated glossary of \ili{Tok Pisin} words sometimes used in this grammar because they more closely capture the meanings of certain Ulwa words or because they are more familiar to people who live or work in Papua New Guinea. In the list provided here, the Ulwa translation is given in \textit{italics}, along with an \ili{English} explanation.\\

\noindent \textbf{aibika} (\textit{yomal}). A leafy green vegetable (\textit{Abelmoschus manihot}) that is harvested in the jungle. Its long, soft leaves are commonly cooked in coconut milk (cf. \textbf{tulip}).\\

\noindent \textbf{bilum} (\textit{ani}). A net bag woven of strings and typically worn around the neck; smaller ones are often used to carry items such as tobacco and betel nut. The Ulwa term has come to be applied to modern, factory-made bags as well.\\

\noindent \textbf{buai} (\textit{aw}). The \textit{Areca catechu} palm, whose seed (or ‘nut’, i.e., ‘betel nut’) is chewed as a stimulant, especially when combined with \textbf{daka} (\textit{wanmbi}) and lime (calcium hydroxide). The palm is grown in Manu both for personal consumption and for export. Both the \ili{Tok Pisin} term and the Ulwa term can be applied to the palm or to the nut harvested from it or to the combination of the nut with \textbf{daka} and lime, which, alternatively, may be called \textbf{red} \textbf{buai} (\textit{ansi}).\\

\noindent \textbf{daka} (\textit{wanmbi}). The leaf or flower of the \textit{Piper betle} (‘pepper’) vine, commonly chewed with \textbf{buai} (\textit{aw}) and lime (calcium hydroxide) to make \textbf{red} \textbf{buai} (\textit{ansi}).\\

\noindent \textbf{garamut} (\textit{numbu}). A large slit-drum made from the hollowed log of an ironwood tree. The term, both in \ili{Tok Pisin} and in Ulwa, may also be applied to the trees themselves. The drum is struck as a gong to communicate messages or to summon people to a location. The drums may be decorated with carvings. In Ulwa, the word \textit{numbu} may also be used to refer to the vertical posts of a house, since these, too, are made from the timber of these trees.\\

\noindent \textbf{haus} \textbf{tambaran} (\textit{amba}). A traditional ancestral worship house (‘men’s house’ or ‘spirit house’). Before being abolished in the latter half of the twentieth century – that is, after the arrival of Christian missionaries – these ‘spirit houses’ were the exclusive domain of Ulwa men who had been initiated with secret rites, which included body scarification. The practices of these initiates have largely remained secret, but they are known to have included singing, dancing, and communal dining, sometimes on human flesh. In the \isi{Sepik} area, ‘spirit houses’ are also referred to as \textbf{haus} \textbf{boi}.\\

\noindent \textbf{kanda} (\textit{le}). Several species of climbing palm (rattan cane) that are used to weave the internal walls of houses.\\

\noindent \textbf{kandere} (\textit{yawa}). A \isi{kinship} term that varies in usage based on local custom. In the Ulwa-speaking area it refers to one’s mother’s brother (i.e., maternal uncle), a relation who has special obligations to his sister’s children, who are called \textit{ansi nungol}.\\

\noindent \textbf{kaukau} (\textit{nongontam}). Any of the varieties of sweet potato (\textit{Ipomoea batatas}) that are harvested and consumed in Papua New Guinea. Although more common in the Highlands, some Manu villagers do grow this crop at home in the \isi{Sepik} lowlands. The varieties grown are typically white sweet potatoes, with flesh and skin that are whiter than the more orange-colored North American varieties.\\

\noindent \textbf{kiap} (\textit{wanwane}). Patrol officers (or district officers) of the British (and later Australian) colonial territories in \isi{New Guinea}. The Ulwa name \textit{wanwane} literally means ‘mushroom’, referring to the traveling officers’ headwear, which apparently resembled mushroom caps.\\

\noindent \textbf{kunai} (\textit{nipum}). A blade-like grass (\textit{Imperata cylindrica}) not found directly in the Ulwa area, but in nearby grasslands.\\

\noindent \textbf{kundu} (\textit{nïte}). A small hand drum with a body of wood and vibrating membrane of lizard skin that is struck with the hand. It is used in traditional dances and to accompany singers.\\

\largerpage
\noindent \textbf{limbum} (\textit{me}). A species of palm whose stems are split and flattened to be used for flooring and baskets. The term typically refers not to the palm itself, but to the flattened product derived from it, or – possibly – to a strip of this flattened stem.\\

\noindent \textbf{morota} (\textit{ila}). Sago palm fronds, used in house construction to make thatch roofs. Traditionally, these were also used by the Ulwa people to keep track of time by breaking a frond for each day that has passed.\\

\noindent \textbf{pangal} (\textit{wema}). Woven sago palm fronds, used to make the outside walls of houses.\\

\noindent \textbf{tokples} (\textit{na} ‘talk’, \textit{mïtïn} ‘egg; language’, \textit{unanji na} ‘our [\textsc{excl]} talk’, etc.). Any of the hundreds of vernacular languages of Papua New Guinea, often contrasted with \textbf{Tok} \textbf{Pisin}, the nation’s lingua franca. There is no clear equivalent for this word in Ulwa, and it is commonly used as a \isi{loan} in that language. However, \textit{na} ‘talk, speech, story, message, thought, reason, language’ may convey this meaning, especially when used with a possessive marker. The word \textit{mïtïn} ‘egg’ may also be used to mean ‘language’.\\

\noindent \textbf{tulip} (\textit{anmopa}). A leafy green vegetable (\textit{Gnetum gnemon}) that is harvested in the jungle. It is commonly cooked in coconut milk (cf. \textbf{aibika}).\\

\chapter{The Ulwa cosmogony}\label{sec:app.f}

The Ulwa people have a traditional story that tells of the origin of the universe and the creation of the first people. It runs roughly as follows.

  Long ago there was Ambawanam Ngata, a great man who lived in the universe all alone. He built village after village, until finally he built the current village -- that is, our world. Still, he had no wife and no children. Living alone, he set out to build a \textit{garamut} drum. While hacking at the wood with his stone axe to carve the drum, he accidentally cut his leg. When blood began to pour out, he grabbed a leaf to tie around the wound. After staunching most of the blood, he took half of a split coconut shell, put it under his leg, and let the rest of the blood flow into it. When the bleeding stopped, he took the other half of the coconut shell and enclosed his blood between the two halves. He put the blood-filled coconut under the awning of his house and resumed building his drum. Meanwhile, the coconut, which had transformed into an egg, hatched. Inside the egg were a man and a woman.

  The man left the egg and headed out to see Ambawanam Ngata working on his drum. Shocked to see another human, Ambawanam Ngata asked the man who he was and where he had come from. The man led him back to the awning of the house, whereupon Ambawanam Ngata shot the broken coconut-egg with an arrow, and the woman fell out. Having forbidden the woman to follow him, Ambawanam Ngata went back to work on his drum. But disregarding his order, the woman came upon him while he was carving the drum. He shouted at her to leave, as it is \isi{taboo} for a woman to be present while a \textit{garamut} drum is being made. The woman and the man left together. The woman found a yam and cooked it in the fire. She scraped off the ashes and put the cooked yam in a coconut shell. She gave this to the man, telling him to bring it to their supposed father. The man did just that: he went to Ambawanam Ngata and called to him, “Papa!” But Ambawanam Ngata told the man: “I am not your father; I am your grandfather.” And Ambawanam Ngata left for good, flying off to live in the clouds.

  Within the contemporary Christian Ulwa community, the man and woman who hatched from the coconut-egg are sometimes equated with Adam and Eve. Ambawanam Ngata is sometimes identified with the Christian god.

\bigskip
\noindent
* * *

\bigskip
\noindent
The Ulwa people have another traditional story that tells of the origins of the peoples of \isi{New Guinea} and, perhaps, the wider world. It runs roughly as follows.

  Long ago, alone in this world were an old man and an old woman, who lived together as husband and wife. The old woman desperately wanted a child, but the couple was unable to conceive one. So the old woman prayed to the gods and, in a dream, she was told what to do. She was to gather clay and mold it into the shape of a man; then she was to put the clay man into the fire to bake and take him out once his body had cooked to a fine golden brown.

  The next morning, the woman set out to do just that. She gathered some clay, molded it into the shape of a man, and put this clay man into the fire. Having put the clay man in the fire, she headed out to go fishing. She fished and fished, losing track of the time. Meanwhile, the clay man continued to bake, turning brown, then browner, and then -- since he was in the fire much too long -- black as night. Once fully blackened, the clay man -- now a living boy -- jumped out of the fire and began to run. The old man, who was home, saw this black child and shrieked in fright. The boy, startled by the old man’s yelling, ran away into the jungle, where he became a jungle spirit.

  Eventually, the old woman returned to find the fire having died down, but with no child inside. After her husband explained what he had seen, the old woman, not at all deterred, tried again, this time resolving to keep watch by the fire. She gathered more clay, molded it into a second man, and placed this second clay man into the fire. She watched as the clay started to darken. When the clay man had reached a nice golden brown, she removed him from the fire. He came to life, and she considered him her son.

  Very pleased with the results, the old woman decided to try to make one final child -- only now the fire had died down completely. So she decided to bake this man in the sun instead. She gathered clay, molded the man, and put him out in the sun. He baked and baked, but his color never managed fully to darken. Nevertheless, he too came alive -- another son. He was like his two brothers, only white in complexion. The old woman took the two sons that remained and brought them home to introduce them to their father.

  The old man was also pleased with his new sons, and so he decided then and there to allot to each his inheritance. He called the two boys over. Grabbing a coconut, he split it in two: one side held the eyes of the coconut, the other side the rear. He threw the two halves before the children, telling the older son (the brown-skinned one) that he may choose first. Foolishly, the older son chose the rear end of the coconut. The younger son was left with no choice but to take the eye side. The father spoke to them as follows: “Ah, my son, you are older, but you have chosen foolishly. For you must hold this closed end of the coconut before your face, unable to see far; you will not have an easy life; you must work hard for your livelihood; but this land here will be yours, and it is good land. And you, my younger son, you have before you the eyes of the coconut; you will hold this side before your face, and you will see far; you will make great advances compared to your brother, but you must live far away from here.”

  And with that he sent his sons off into the world. The brown-skinned one was to be the father of all people alive today in the region. Years later, when white people came to \isi{New Guinea}, they were recognized as the descendants of the white-skinned child. Similarly, Solomon Islanders, whose skin is notably darker than that of the people of \isi{New Guinea}, are sometimes said to be the descendants of the first, black-skinned child.

  \bigskip
  \noindent
  * * *

  \bigskip
  \noindent
The village of Manu has an account of its origins as well, extending into the legendary past, which runs roughly as follows.

  Long ago, the ancestors of everyone -- Ulwa-speakers and everyone else who now lives along the \isi{Sepik River} -- came from far off, in unknown lands lying to the distant west. Eventually, they settled in a place called Kamen, which is near present-day Kambaramba village. All the clans and all the language groups lived together: Ulwa, \ili{Mundukumo}, \ili{Ap Ma}, \ili{Kanda}, \ili{Mwakai}, \ili{Pondi}, and so on. But the leading clan in this massive village was called Kamen, after which the settlement was named. One day, the Kamen clan killed a huge crocodile. But, contrary to custom, the leaders of the clan did not share the meat with the other clans. Greatly angered by this, the other clans declared war on the Kamen clan, killing some of them in the battle. In the disorderly fighting that ensued, people from other clans were killed as well. Eventually the entire settlement was at war, every clan fighting for itself. With peace no longer tenable at Kamen, all the clans split up.

  The Ulwa clan was itself divided into four sub-clans: Nïmalnu (Manu), Mamala (Maruat), Andïmali (Dimiri), and Mosombla (Yaul). The Nïmalnu clan first settled in a place called Dim, which is near present-day \ili{Biwat}. The \ili{Mundukumo} people, who lived at \ili{Biwat}, began to enter this land, and wars ensued between the two groups.\footnote{The \ili{Mundukumo} (or Mundugumor) people were famously described as being aggressive by \citet{Mead1935}.} Avoiding further warfare, the Nïmalnu clan moved to a second village, called Yambul, which is in the area of the present-day Maruat, Dimiri, and Yaul villages.

  When the other Ulwa sub-clans moved to this area, too, however, the Nïmalnu clan moved to yet another village (their third), called Amali, which is about a five-hour walk away from the current village location, in the direction of the Bun clan.

  This proved a very desirable location, but incessant warring with the \ili{Bun} community, who are closely related linguistically to the \ili{Mundukumo} community, prompted the Nïmalnu clan to move yet again, to the fourth village, which itself was divided into two areas: Yambiwa and Mamanu. This two-part village, which is about a two-hour walk away from the current one, is still known to the people of Manu, and it is often visited and used as a base from which to hunt. Its full name is \textit{Wa Wapata} (literally ‘old village’), but it is usually called by a shortened form, \textit{Wopata}. By the time of its arrival to this fourth village, the Nïmalnu clan had grown so large as to consist of seven sub-clans: three clans lived at the \linebreak Yambiwa part of the village and four clans lived at the Mamanu part of the village. This fourth village, although a refuge from enemy groups, proved unhygienic. In the swampy climate, the Nïmalnu clan suffered countless deaths due to disease. The sub-clans were reduced in number from seven to four, which is the current number of Manu clans.

  It was because of this poor climate, as well as a desire for better access to water and to colonial Australian administrative services that the Nïmalnu people started moving in the 1960s to their fifth (and current) village, in the area known locally as Bulon, but now commonly referred to as Manu.

\chapter{Laycock’s Yaul field notes}\label{sec:app.g}

\il{Yaul|(}
\il{Maruat-Dimiri-Yaul|(}
\ia{Laycock, Donald C.|(}

This appendix offers a typed transcription of \citegen[3218--3264]{Laycock1971a}\footnote{Laycock mostly skipped odd-numbered pages in his notebooks.} handwritten field notes on the Yaul \isi{dialect} of Ulwa, recorded around February or March of 1971. Laycock generally did not write glosses in his field notes. Rather he followed a mostly fixed elicitation order, which is recorded in \citet[70--71]{Laycock1973}. As signposts he included some occasional item numbers in his notes. He often used a dash (--) to indicate that a word has been skipped. He used a tilde ({\textasciitilde}) to indicate that part or all of the preceding line is to be interpreted as duplicated in the current line.

  In this present transcription, I have included the glosses from \citegen[70--71]{Laycock1973} standard elicitation list. Italicized words are those that Laycock notes to be frequently omitted, since their translations proved difficult to elicit. Other bracketed content (ending in “RB”) contains my own annotations. Digital scans of the original handwritten notes are stored at the PARADISEC archive:\\

\url{https://dx.doi.org/10.4225/72/56F2B4E751BAF}\\

\noindent [page] 3218\\

\noindent YAUL ?\\
(ANDJI LOWA) our language\\
Silami, Ansamari [Laycock’s two consultants -- RB]\\
Yaul, Dimiri, Maruat, Manu [the villages where “Yaul” (i.e., Ulwa) is spoken -- RB]\\

\noindent mbɑrɑndʒi    [‘man’] [‘person’ -- RB]\\
yɑnɑ      [‘woman’]\\
itɑmɡɑdɑ, itɑmɡɑdɛ  [‘old man’]\footnote{Laycock notes here: “one informant has ɑ, other ɛ – one informant has final Ø, other ə”. This note is probably meant to refer to the \isi{idiolect}s of the two consultants more generally.}\\
inɑmɡɑdɑ, inɑmɡɑdɛ  [‘old woman’]\\
sɪmbwɑy     [‘child’]\\
itɑum       [‘father’]\\
inɑum       [‘mother’]\\
itɑum ɑundum     [‘grandfather’]\\
inɑum {\textasciitilde}     [‘grandmother’] [i.e., \textit{inɑum ɑundum} -- RB]\\
ɑtumow     [‘elder brother (of man)’]\\
wɔtu       [‘younger brother (of man)’]\\
yɛnɑnuwə     [‘sister (of man)’]\\
nindʒinɑm    ‘MB’ [i.e., ‘mother’s brother’ -- RB]\footnote{The form here probably actually means ‘my mother’ (or ‘my aunt’), suggesting some confusion in elicitation.}\\
nindʒinɑum wɔtu  ‘SS’ [i.e., ‘sister’s son’ -- RB]\footnote{The form here might actually mean ‘my parent’s younger sister’.}\\
inɑum yɛnɑnu    ‘kantire ♀’ [i.e., ‘mother’s sister’ -- RB]\footnote{Laycock’s gloss uses the \ili{Tok Pisin} word \textit{kandere}, which in local usage generally means ‘mother’s brother’ but can refer to other relations in other areas.}\\
nindʒi ɑnduwɑ  [‘wife’s brother/brother’s wife’] [actually ‘my in-law’ -- RB]\\
 --         [‘\textit{sorcerer}’]\\
 --         [‘\textit{person}’]\\
nəməndi    [‘eye’]\\
yip(ə)      [‘nose’]\\
kəkɑr      [‘ear’]\\
ɑmb{\'{ә}l\textlatin{ɑ}}    [‘tooth’]\\

\ia{Laycock, Donald C.|)}
\il{Maruat-Dimiri-Yaul|)}
\il{Yaul|)}
\il{Yaul|(}
\il{Maruat-Dimiri-Yaul|(}
\ia{Laycock, Donald C.|(}


\noindent [page] 3220\\

\noindent mələm      [‘tongue’]\footnote{Numbered “21”.}\\
mɑmɑ      [‘mouth’]\\
ɡənúmɔ    [‘chin’]\\
wɑnɪmi    [‘head hair’]\\
misɑm      [‘head’]\\
wuwɑ      [‘neck’]\\
tumbʊnmwɔ  [‘nape’]\\
 --         [‘\textit{tear}’]\\
ɑƀinɑm    [‘shoulder’]\\
yɑŋɡwʊm    [‘arm’]\\
yɪrup       [‘elbow’]\\
yoʊmbɔm    [‘hand/palm’]\\
yumí·p      [‘finger’]\\
inɛnɑn      [‘fingernail’]\\
 --         [‘\textit{fist}’]\\
kuɔyi (kwɔyi)  [‘chest’]\\
mutɑmɑ    ‘spine’ [‘back’ -- RB]\\
wɑlʊm      [‘breast (female)’]\\
ina\specialchar{ᵽ}wɔm    [‘belly’]\\
mutɑmɑ wʊmɑ  [‘back’] [‘back bone’, i.e., ‘spine’ -- RB]\\
wʊnbu·p    [‘rump’]\\

\noindent [page] 3222\\

\noindent --         [‘\textit{anus}’]\footnote{Numbered “41”.}\\
bətʊm      [‘leg’]\\
bətɑlmʊtwʊmɑ  [‘knee’]\\
nɑmbiyʊm    [‘skin’]\\
yɪl        [‘body hair’]\\
ɑrᵊkən      [‘blood’]\\
wʊmʷɑ    [‘bone’]\\
ƀɑlʊm wʊmɑ  ‘ribs’\\
kwɔy      ‘bros’ [i.e., ‘chest’ -- RB]\\
mənɑndən    ‘heart’ [actually ‘gallbladder’? -- RB]\\
 --         [‘\textit{lungs}’]\\
índʒɑŋɡɑ    [‘liver’ -- RB’]\\
mɛlɛlɑ, mɛlɛlɛ  [‘intestines’ -- RB]\footnote{Numbered “50”.}\\
ŋɡəndɑ    [‘guts’]\\
ilɑw      [‘fat’]\\
lɑm      ‘mit’ [i.e., ‘meat, flesh’ -- RB]\\
wɑl       [‘penis’]\\
mətέn       [‘testicles’]\\
ɪnɪmbə     [‘vulva’]\\
tɑƀɑ       [‘sore’]\\
məndɑm     [‘pus’]\\
 --        [‘\textit{ghost}’]\\
 --        [‘\textit{ancestral spirit}’]\\
 --        [‘\textit{natural spirit}’]\\

\noindent [page] 3224\footnote{At the top of the page Laycock has the note: “l = r”.}\\

\noindent ɑlei       [‘sun’]\footnote{Numbered “61”.}\\
yiƀəl       [‘moon’]\\
lɑ·ri       [‘star’]\\
 --        [‘\textit{sky}’]\\
ɑrɑm       [‘cloud’]\\
ɑlesíŋɡəm     [‘fog’]\\
yɪnəm       [‘rain’]\\
yimbɑ       [‘night’]\\
mɑtɑkwulᵊp ʊmbwenɑm  ‘(tulait bruk nau.)’ [‘day’]\footnote{Laycock’s \ili{Tok Pisin} translation is roughly ‘twilight has broken’ in \ili{English}. He glosses \mbox{<ʊmbwenɑm>} as ‘tomara’ (i.e., ‘tomorrow’).}\\
bʷɔnɛmbi    [‘morning’]\footnote{Numbered “70”.}\\
ɑƀɑnɑmɡe    [‘evening’]\\
i·nəm      [‘water’]\\
inum pul    [‘pond’]\\
inəm mɑ tɑit pɪn i·n, inəm mɑ ɑmbitey ini ‘wara i tait na i kam’ [‘current’]\footnote{Laycock’s \ili{Tok Pisin} translation is roughly ‘the water is strong and coming’ in \ili{English}. The first sentence in Ulwa is probably actually ‘the water \textit{tait pinis} has come’ (\textit{tait} ‘flood/current’ and \textit{pinis} ‘already’ are both \ili{Tok Pisin}), and the second sentence in Ulwa is probably ‘the water is big, has come’.}\\
 --        [‘\textit{sea}’]\\
 --        [‘\textit{beach}’]\\
yinɑt       [‘ground’]\\
minɛm     [‘stone’]\\
yin yirís     [‘sand’]\\
inɑt mɑ· tɑk   [‘mountain’]\\
 --        [‘\textit{ridge}’]\\
 
\noindent [page] 3226\\

\noindent --         [‘\textit{valley}’]\footnote{Numbered “81”.}\\
wɑniɑt ɑmbɑ·kei  [‘plain’]\\
ƀɑntɑm, ruƀɑ    [‘bush’] [i.e., ‘jungle, woods, forest’ -- RB]\\
{\textasciitilde}        [‘garden’] [i.e., \textit{ƀɑntɑm} -- RB]\\
 --         [‘\textit{fence}’]\\
 --         [‘river’]\\
inɑt tɑmbíŋɡɑtɑ  [‘swamp’] [literally ‘bad ground’ -- RB]\\
ŋɡʊŋɡwʊn    [‘wind’]\\
ŋɡʊŋɡwʊn mɑyinɛ  [‘wind has come’ -- RB]\\
ɑpən      [‘fire’]\\
yimɔt      [‘firewood’]\footnote{Numbered “90”.}\\
ɑpərə́ŋɡən   [‘smoke’]\\
ɑpən inɑŋɡən mɑ́lɑŋ   [‘white ash’]\\
iməkəl     [‘black ash’]\\
sɑkwɑ͔y     [‘tobacco’]\\
 --        [‘\textit{cigarette}’]\\
ƀətə́lƀɑ     [‘road’]\\
ləƀɑ́ {\textasciitilde}      [‘track’] [i.e., \textit{ləƀɑ́ ƀətə́lƀɑ}; literally ‘jungle road’ -- RB]\\
mʊndə     [‘food’]\\
yɪm       [‘tree’]\\
yɪm mɑmɡwʊmu   [‘branch’] [literally ‘tree stick’? -- RB]\\
mʊndə nɑsi  ‘hanggri’ [i.e., ‘hungry’; literally ‘hunger hits me’ -- RB]\\

\noindent [page] 3228\\

\noindent yɪm bɑpɑ     [‘leaf’] [literally ‘tree leaf’ -- RB]\footnote{Numbered “101”.}\\
{\textasciitilde} misɑm    [‘tree-top’] [i.e., \textit{yɪm misɑm}; literally ‘tree head’ -- RB]\\
 --        [‘\textit{vine}’]\\
{\textasciitilde} mʊm       [‘fruit’] [i.e., \textit{yɪm mʊm}; literally ‘tree fruit’ -- RB]\\
 --        [‘flower’]\\
nɑmbiyum    [‘bark’] [= ‘skin’ -- RB]\\
yɪm kɑl    [‘tree sap’ -- RB]\\
ɑsɑpɑ      [‘grass (kunai)’]\\
pɑlɑm      [‘wild sugarcane (pitpit)’]\\
məlí      [‘sugarcane’]\\
yɛkwɑm    [‘bamboo’]\footnote{Numbered “110”.}\\
rɛ        [‘rattan’]\\
wʊlʊm      ‘palm’ [‘sago’]\\
ƀɛ        ‘food (for frying)’ [‘sago flour, sago pancake’ -- RB]\\
ɑy        ‘hatwara’ [‘sago broth’] [i.e., ‘jellied sago’ -- RB]\\
nɪŋɡwɔmtɑm  [‘sweet potato’]\\
mənɑr       [‘taro’]\\
wʊtɑm     [‘yam’]\\
kʊmbulɛ     [‘small yam (mami)’]\\
məndə       [‘banana’]\\
ŋɡənɑm     [‘\textit{pandanus}’]\\
yɛ         [‘coconut’]\\
{\textasciitilde} ƀəpɑtɑ     [‘dry coconut’] [i.e., \textit{yɛ ƀəpɑtɑ} -- RB]\\
{\textasciitilde} mənɑndən   [‘green coconut’] [i.e., \textit{yɛ mənɑndən} -- RB]\\

\noindent [page] 3230\\

\noindent --         [‘\textit{edible fern (kumu)}’]\footnote{Numbered “121”.}\\
ɑ·mu      [‘areca nut’]\\
wɑnbi       [‘betel pepper vine’]\\
{\textasciitilde} ƀɑpɑ      [‘betel pepper leaf’] [i.e., \textit{wɑnbi ƀɑpɑ} -- RB]\\
{\textasciitilde} ŋɡərɑŋɡɑn   [‘betel pepper fruit’] [i.e., \textit{wɑnbi ŋɡərɑŋɡɑn} -- RB]\\
i·         [‘lime’]\\
ɑris       [‘lime gourd’]\\
 --        [‘\textit{lime stick}’]\\
si·         [‘salt’]\\
lə́pɑ      [‘breadfruit’]\\
 --        [‘\textit{Gnetum (sayor)}’]\\
ɑmwɔlɑɸɑ   [‘Gnetum (tulip)’]\\
yɔmɔl       [‘Hibiscus (epika)’]\\
 --        [‘Amaranthus (grinlip)’]\\
mɔmbʊn     [‘Amaranthus (aupa)’]\\
 --        [‘tree sap’]\footnote{This was already elicited, after ‘bark’. Laycock often elicited ‘tree sap’ earlier than in his prescribed order.}\\
rɑᵘƀu       [‘cordyline’]\\
 --        [‘\textit{capsicum}’]\\
mə́rəkən     [‘nipa’] [i.e., ‘sago palm species’ -- RB]\footnote{The form given refers to the sago palm species from which grubs are harvested.}\\
wʊndɛn     [‘arecoid palm (limbum)’]\\

\ia{Laycock, Donald C.|)}
\il{Maruat-Dimiri-Yaul|)}
\il{Yaul|)}
\il{Yaul|(}
\il{Maruat-Dimiri-Yaul|(}
\ia{Laycock, Donald C.|(}

\noindent [page] 3232\\

\noindent dɑŋɡwɑn     ‘black palm’ [‘wild arecoid (wail limbum)’]\\
yimɑ·ƀəl     [‘cassava’]\\
tɨn (tɪn)     [‘dog’]\\
nɑmʊndə     [‘pig’]\\
{\textasciitilde} yinɔm     ‘mother pik’ [i.e., \textit{nɑmʊndə yinɔm} ‘sow’ -- RB]\\
{\textasciitilde} kɑmbən     ‘baby pig’ [i.e., \textit{nɑmʊndə kɑmbən} ‘piglet’ -- RB]\\
tɪn ɑŋɡwʊn  [‘dog’s tail’]\\
{\textasciitilde} yɪl      [‘dog’s fur’] [i.e., \textit{tɪn yɪl} -- RB]\\
wutɑ, wutɛ  [‘bird’]\\
{\textasciitilde} ƀɑpɑ      [‘wing’] [i.e., \textit{wutɑ ƀɑpɑ}; literally ‘bird wing’ -- RB]\\
{\textasciitilde} yɪl      [‘feather’] [i.e., \textit{wutɑ yɪl}; literally ‘bird body hair’ -- RB]\\
{\textasciitilde} yip(ᵊ)    [‘beak’] [i.e., \textit{wutɑ yip(ᵊ)}; literally ‘bird nose’ -- RB]\footnote{Numbered “150”.}\\
mətən       [‘egg’]\\
ɡwi·mɑkɑn   [‘tree kangaroo’]\\
tə\specialchar{ᵽ}wɔ·ləm   [‘possum’]\\
kɑrim       [‘cassowary’]\\
kɑlim ʷɔmbi  ‘many muruk’ [i.e., ‘many cassowaries’ -- RB]\\
wɑndi       [‘bandicoot’]\\
yɑki       [‘rat’]\\
yapɔlɔpɑ     [‘flying fox’]\\
{\textasciitilde} sɔmɔrə, mənəmnɛ   [‘small bat’] [i.e., \textit{yapɔlɔpɑ sɔmɔrə, mənəmnɛ}]\footnote{The first form is literally ‘small flying fox’. The second form may be an otherwise unidentified bat species, or else there was some confusion in the elicitation.}\\
wɔlim       [‘pigeon’]\\
ʊpɪ́n       [‘goura’] [i.e., ‘crowned pigeon’ -- RB]\\

\noindent [page] 3234\\

\noindent yɑpótɑ     [‘cockatoo’]\\
 --        [‘\textit{crow}’]\\
ɑsipɑl       [‘hornbill’]\\
woƀɑl       [‘fowl’]\\
ɑƀɑsɪ́ŋɡən     [‘hawk’]\\
ƀitíwt       [‘duck’]\\
yɑkomɑkɑn   [‘wildfowl’]\\
 --        [‘\textit{owl}’]\\
wɑy       [‘parrot’]\\
(kʊmʊl) kumulmɑu  [‘Bird of Paradise’]\footnote{Numbered “170”. Laycock used parentheses to indicate that this form in Ulwa (presumably two variants given?) derives from a \ili{Tok Pisin} \isi{loan}, \textit{kumul} ‘bird of paradise’.}\\
 -- \footnote{Laycock skipped a line here, but it is not clear what was intended, since there is no corresponding item that occurs here in his test list.}\\
ɑrəmwɔkɑ   [‘snake’]\\
{\textasciitilde} ɑmbiŋɡɑtɑ   [‘python’] [i.e., \textit{ɑrəmwɔkɑ ɑmbiŋɡɑtɑ}; literally ‘big snake’ -- RB]\\
wʊsʸɪm     [‘crocodile’]\\
rəkət       [‘lizard’]\\
 --        [‘monitor lizard’]\\
kitɑ       [‘frog’]\\
ƀəlipɑn     [‘fish’]\\
ri·         [‘crayfish’]\\
dʒiƀɑ·lə    [‘fly’]\\
mu·      ‘blulang’ [‘March fly’]\footnote{Laycock’s \ili{Tok Pisin} gloss \textit{blulang} is often translated in \ili{English} as ‘blowfly’ (Calliphoridae). The item that occurs at this point in his test list (‘March fly’) is known elsewhere in \ili{English} as the ‘horsefly’ (Tabanidae).}\\
nɑŋɡwʊn     [‘mosquito’]\\

\noindent [page] 3236\\

\noindent yɑkərɑƀɑƀɑnə  [‘butterfly’]\footnote{Numbered “181”.}\\
kəkɑ       [‘ant’] [‘white ant’ -- RB]\\
ŋɡu·ŋɡʊn     [‘red ant’]\\
mɨn       [‘louse’]\\
 --        [‘\textit{spider}’]\\
 --        [‘termite’]\\
ɑpɑ       [‘house’]\\
wuɑ (wɔɑ)   [‘village’]\\
ɑpɑtɑm    [‘bed’]\\
 --        [‘fireplace’]\\
 -- \footnote{Laycock skipped a line here, but it is not clear what was intended, since there is no corresponding item that occurs here in his test list.}\\
mɑrɑ       [‘spear’]\\
wɑŋɡutɑ     [‘bow’]\\
rɑpə, wiɸɑm  ‘supsup’ [i.e., ‘fishing spear’ -- RB]\\
 --         [‘\textit{bowstring}’]\\
 --        [‘club’]\\
 --        [‘shield’]\\
mɑsɑ²       [‘string’] [i.e., \textit{mɑsɑmɑsɑ} -- RB]\\
{\textasciitilde} ɑmbiŋɡɑtɑ   [‘rope’] [i.e., \textit{mɑsɑmɑsɑ ɑmbiŋɡɑtɑ}; literally ‘big string’ -- RB]\\
ɑri·       [‘man’s netbag’] [i.e., ‘\textit{bilum}, string bag, net bag’ -- RB]\\
nɑndu·     [‘woman’s skirt’]\\

\noindent [page] 3238\\

\noindent lɔplɔp      ‘laplap’ [‘cloth’, i.e., a \isi{loan} from \ili{Tok Pisin} \textit{laplap} ‘cloth’ -- RB]\\
ɑnɑmbiyum  [‘male clothing’]\\
sɑkɑnmɑ    [‘axe’]\\
yitə\specialchar{ᵽ}ən      [‘bushknife’] [i.e., ‘machete’ -- RB]\\
lísə        [‘hand-drum’] [i.e., ‘\textit{kundu} (small hand drum)’ -- RB]\\
nəmbu      [‘slit-gong’] [i.e., ‘\textit{garamut} (large slit drum)’ -- RB]\\
kwoƀ(u)     [‘singsing’] [i.e., ‘song, song and dance’ -- RB]\\
 --        [‘\textit{decorations}’]\\
 --        [‘\textit{oil}’]\\
nʊmu       [‘canoe’]\\
ɑnɑƀ       [‘paddle’]\\

\noindent
[Editor’s note: The remainder of Laycock’s Yaul notes are devoted to “preliminary grammatical testing” \citep[71]{Laycock1973}. Laycock does not include many translations, but those that he does include are presented as the second line of each entry. When his translation is abbreviated or not fully presented in \ili{English} (i.e., it is at least partly written in \ili{Tok Pisin}), I have included a translation on the second line as well (in brackets). This should be taken as a translation of what Laycock probably understood the Yaul form to mean in \ili{English}. As the third line of each entry, I have included my own translation of what I think the Yaul form most likely means in \ili{English} -- RB.]\\

\ia{Laycock, Donald C.|)}
\il{Maruat-Dimiri-Yaul|)}
\il{Yaul|)}
\il{Yaul|(}
\il{Maruat-Dimiri-Yaul|(}
\ia{Laycock, Donald C.|(}

\noindent kwo wɑmɑ?

\noindent [‘Who are you [\textsc{sg}]?’]

\noindent [‘Who are you [\textsc{sg}] [who is] going?’ -- RB]\\

\newpage

\noindent nɑwu

\noindent mi [= ‘me’]

\noindent [‘[It’s] me.’] -- RB\\

\noindent kwʊmʊŋɡu nɑ·mə

\noindent [‘Who are you two?’]

\noindent [‘Who are you two [who are] going?’ -- RB]\\

\noindent ŋɡɑnɑwɔ

\noindent [‘us two’]

\noindent [‘[It’s] us [\textsc{du.excl}].’ -- RB]\\

\noindent ŋɡɑnbi

\noindent mi 2 i kam [= ‘We [\textsc{du.excl}] come.’]

\noindent [‘We [\textsc{du.excl}] have come here.’ -- RB]\\

\noindent kwumo wunɑ·mɑlɑ?

\noindent [‘Who are you [\textsc{pl}]?’]

\noindent [‘Who are you [\textsc{pl}] [who are] going?’ -- RB]\\

\noindent wʊn biyɛ

\noindent yupela ol i kam [= ‘You [\textsc{pl}] come.’]

\noindent [‘You [\textsc{pl}] have come here.’ -- RB]\\

\noindent ɑnɑw wuɸɑ mbi

\noindent mipela olgeta i kam [= ‘We all come.’]

\noindent [‘We all have come here.’ -- RB]\\

\noindent [page] 3240\\

\noindent kwoyɛ

\noindent 1

\noindent [‘one’ -- RB]\\

\noindent nini

\noindent 2

\noindent [‘two’ -- RB]\\

\noindent rirɑ

\noindent 3

\noindent [‘three’ -- RB]\\

\noindent nənɑnɡɛ

\noindent 4

\noindent [‘four’ -- RB]\\

\noindent mwɔtɑm

\noindent 5

\noindent [‘five’; literally ‘stick’ or ‘bundle’? -- RB]\\

\noindent kwey ɑƀɔk

\noindent 6

\noindent [‘six’; literally ‘one again’? -- RB]\\

\noindent nini {\textasciitilde} [i.e., \textit{nini ɑƀɔk}]

\noindent 7

\noindent [‘seven’; literally ‘two again’? -- RB]\\

\noindent rirɑ {\textasciitilde} [i.e., \textit{rirɑ ɑƀɔk}]

\noindent 8

\noindent [‘eight’; literally ‘three again’? -- RB]\\

\noindent nənɑnɡɛ {\textasciitilde} [i.e., \textit{nənɑnɡɛ ɑƀɔk}]

\noindent 9

\noindent [‘nine’; literally ‘four again’? -- RB]\\

\noindent mwɔtɑm nini

\noindent 10

\noindent [‘ten’; literally ‘two bundles’? -- RB]\\

\noindent {\textasciitilde} kweyɑ ƀɑtək [i.e., \textit{mwɔtɑm nini kweyɑ ƀɑtək}]

\noindent 11

\noindent [‘eleven’; literally ‘two bundles, one let atop’? -- RB]\\

\newpage

\noindent ( ) nini {\textasciitilde} [i.e., \textit{mwɔtɑm nini nini ƀɑtək}]

\noindent 12

\noindent [‘twelve’; literally ‘two bundles, two let atop’? -- RB]\\

\noindent ( ) rirɑ {\textasciitilde} [i.e., \textit{mwɔtɑm nini rirɑ ƀɑtək}]

\noindent 13

\noindent [‘thirteen’; literally ‘two bundles, three let atop’? -- RB]\\

\noindent ( ) nənɑnɡɛ {\textasciitilde} [i.e., \textit{mwɔtɑm nini nənɑnɡɛ ƀɑtək}]

\noindent 14

\noindent [‘fourteen’; literally ‘two bundles, four let atop’? -- RB]\\

\noindent {\textasciitilde} rirɔ [i.e., \textit{mwɔtɑm rirɔ}]

\noindent 15

\noindent [‘fifteen’; literally ‘three bundles’? -- RB]\\

\noindent kwey ɑƀɔk mwɔtɑm rirɑ dəƀɑtək

\noindent 16

\noindent [‘sixteen’; literally ‘one again, three bundles let atop them’? -- RB]\\

\noindent nini {\textasciitilde} {\textasciitilde} {\textasciitilde} {\textasciitilde} [i.e., \textit{nini ɑƀɔk mwɔtɑm rirɑ dəƀɑtək}]

\ia{Laycock, Donald C.|)}
\il{Maruat-Dimiri-Yaul|)}
\il{Yaul|)}
\il{Yaul|(}
\il{Maruat-Dimiri-Yaul|(}
\ia{Laycock, Donald C.|(}

\noindent 17

\noindent [‘seventeen’; literally ‘two again, three bundles let atop them’? -- RB]\\

\noindent rirɑ [i.e., \textit{rirɑ ɑƀɔk mwɔtɑm rirɑ dəƀɑtə}]

\noindent 18

\noindent [‘eighteen’; literally ‘three again, three bundles let atop them’? -- RB]\\

\noindent nənɑnɡɛ [i.e., \textit{nənɑnɡɛ ɑƀɔk mwɔtɑm rirɑ dəƀɑtək}]

\noindent 19

\noindent [‘nineteen’; literally ‘four again, three bundles let atop them’? -- RB]\\

\noindent mwɔtɑm nənɑnɡɛ

\noindent 20

\noindent [‘twenty’; literally ‘four bundles’? -- RB]\\

\newpage

\noindent[page] 3242\\

\noindent nɑ mʊndə rɑndɑ

\noindent [‘I eat.’]

\noindent [‘I will eat food.’ -- RB]\\

\noindent wu {\textasciitilde} {\textasciitilde} nɑ? [i.e., \textit{wu mʊndə rɑndɑ nɑ?}]

\noindent [‘Do you eat?’]

\noindent [‘Will you eat food?’ -- RB]\\

\noindent ɑndi, nɑ {\textasciitilde} {\textasciitilde} [i.e., \textit{ɑndi, nɑ mʊndə rɑndɑ}]

\noindent yes, mi kaikai [= ‘Yes, I eat.’]

\noindent [‘Yes, I will eat food.’ -- RB]\\

\noindent ninsikɑmbən {\textasciitilde} {\textasciitilde} [i.e., \textit{ninsikɑmbən mʊndə rɑndɑ}]

\noindent child eats [= ‘The child eats.’]

\noindent [‘My child will eat food.’ -- RB]\\

\noindent wi nu, ɡwunə {\textasciitilde} {\textasciitilde} ni [i.e., \textit{wi nu, ɡwunə mʊndə rɑndɑ ni}]

\noindent yu kam, mi 2 kaikai [= ‘You come, we two eat.’]

\noindent [‘Come [\textsc{sg}]! We [\textsc{du.incl}] will eat food.’ -- RB]\\

\noindent mu mʊndərɑndɑ

\noindent em i kaikai [= ‘He eats.’]

\noindent [‘He will eat food.’ -- RB]\\

\noindent yɑtəkɑmbən mu {\textasciitilde} [i.e., \textit{yɑtəkɑmbən mu rɑndɑ}]

\noindent male child eats [= ‘The male child eats.’]

\noindent [‘The male child will eat.’ -- RB]\\

\noindent ɡwʊnə mundərɑndɑ

\noindent we 2 eat [= ‘We two eat.’]

\noindent [‘We [\textsc{du.incl}] will eat food.’ -- RB]\\

\noindent ɡwunɑ mundɑ mɑ?

\noindent yu 2 kaikai pinis [= ‘Have you two eaten?]

\noindent [‘Do you two eat food?’ -- RB]\\

\noindent wɑyu, ŋɡɑnɑ ŋɡɔ mundəmɑ ɸeiko

\noindent nogat, mi 2 no kaikai [= ‘No, we [\textsc{du.excl}] do not eat.’]

\noindent [‘No, we [\textsc{du.excl}] do not eat food.’ -- RB]\\

\noindent [page] 3244\\

\noindent PRONOUNS

\begin{tabbing}
{([\textsc{3sg.subj} -- RB])} \= {([\textsc{1du.excl-int} -- RB])} \= {([\textsc{1pl.excl-int} -- RB])}\kill
nɑ(w) \> ŋɡɑnɑ(w) \> ɑnɑ\\
\textsc{[1sg]} \> \textsc{[1du]} \> \textsc{[1pl]}\\
{[\textsc{1sg-int} -- RB]} \> {[\textsc{1du.excl-int} -- RB]} \> {[\textsc{1pl.excl-int} -- RB]}\\
wu \> ɡwuna \> wʊnɑ\\
\textsc{[2sg]} \> \textsc{[2du]} \> \textsc{[2pl]}\\
{[\textsc{2sg} -- RB]} \> {[\textsc{2du-int} -- RB]} \> {[\textsc{2pl-int} -- RB]}\\
mu \> mina \> ndɑ(pwɑ)\\
\textsc{[3sg]} \> \textsc{[3du]} \> \textsc{[3pl]}\\
{[\textsc{3sg.subj} -- RB]} \> {[\textsc{3du-int} -- RB]} \> {[\textsc{3pl(-int)} -- RB]}
\end{tabbing}

\noindent wey nɑmbowᵘ

\noindent mi wanpela [= ‘I myself’]

\noindent [‘only I myself’ -- RB]\\

\noindent nini

\noindent 2

\noindent [‘two’ -- RB]\\

\noindent ɡwunɑ mundɑ mɑ pɑ?

\noindent yu 2 kaikai pinis [= ‘Have you two eaten?’]

\noindent [‘Have you two eaten food?’ -- RB]\\

\noindent ɑnɑ mundɑ mɑp

\noindent we all [have eaten] [= ‘We [\textsc{pl}] have eaten.’]

\noindent [‘We [\textsc{pl.excl}] have eaten food.’ -- RB]\\

\noindent wunɑ pwɑ mʊndɑ mɑ·pɑ

\noindent [‘Have you all eaten?’]

\noindent [‘Have you [\textsc{pl.excl}] already eaten food? -- RB]\\

\newpage

\noindent ɑndi, ɑnɑ mʊndɑ mɑpi

\noindent yes, mipela ol i kaikai pinis [= ‘Yes we [\textsc{pl}] have eaten.’]

\noindent [‘Yes, we [\textsc{pl.excl}] have eaten food.’ -- RB]\\

\noindent ŋɡɑnɑ mʊndɑ mɑpi

\noindent we 2 eat [= ‘We two eat.’]

\noindent [‘We [\textsc{du.excl}] have eaten food.’ -- RB]\\

\noindent minɑ ᵅmɑp(i)

\noindent 2 pela kaikai [‘They two eat.’]

\noindent [‘They two have eaten.’ -- RB]\\

\noindent ndɑɸwɑ ɑmɑp

\noindent ol i kaikai pinis [= ‘They [\textsc{pl}] have eaten.’]

\noindent [‘They [\textsc{pl}] have already eaten.’ -- RB]\\

\noindent [page] 3246\\

\noindent w ɑŋɡwo mɑnɑ?

\noindent yu go we? [= ‘Where do you [\textsc{sg}] go?’]

\noindent [‘Where will you [\textsc{sg}] go?’ -- RB]\\

\noindent nɑ wɔ mɑni

\noindent mi go long ples [= ‘I go to the village.’]

\noindent [‘I am going to the village.’ -- RB]\\

\noindent ɑndi nɑ ləƀɑ mɑni

\ia{Laycock, Donald C.|)}
\il{Maruat-Dimiri-Yaul|)}
\il{Yaul|)}
\il{Yaul|(}
\il{Maruat-Dimiri-Yaul|(}
\ia{Laycock, Donald C.|(}

\noindent yes mi go long bus [= ‘Yes, I go to the jungle.’]

\noindent [‘Yes, I am going to the jungle.’ -- RB]\\

\noindent wumbelepə ləƀɑ mɑnɑni

\noindent tumara mi go long bus [= ‘Tomorrow I go to the jungle.’]

\noindent [‘[I] will go to the jungle tomorrow.’ -- RB]\\

\noindent nɑ ɑŋɡwɔlɑm mɑni

\noindent mi go Angoram [= ‘I go to Angoram.’]

\noindent [‘I am going to Angoram.’ -- RB]\\

\noindent nɑmʊn mɑnɑni

\noindent nau bai mi go [= ‘I will go now.’]

\noindent [‘I will go now.’ -- RB]\\

\noindent wɑndʒikɛkɑ mɑnɑnɑ?

\noindent wanem taim bai [go] [= ‘When will you [\textsc{sg}] go?’]

\noindent [‘When will you [\textsc{sg}] go?’ -- RB]\\

\noindent wʊmbelɑp mɑnɑni

\noindent tumara bai mi go [= ‘Tomorrow I will go.’]

\noindent [‘[I] will go tomorrow.’ -- RB]\\

\noindent wɑndʒikɛkɑ yinɑne?

\noindent when yu kam [= ‘When do you [\textsc{sg}] come?’]

\noindent [‘When will you [\textsc{sg}] come?’ -- RB]\\

\noindent ɑndumbelɑp yinɑme

\noindent mi kam bek haptumara [= ‘I return the day after tomorrow.’]

\noindent [‘[I] will come the day after tomorrow.’ -- RB]\\

\noindent nə mundərɑnde

\noindent [‘I will eat.’]

\noindent [‘I will eat food.’ -- RB]\\

\noindent nɑ ŋɡɔ mundə ɑmɑpɛ ko, nə mbəɸi

\noindent mi no kaikai, mi stap nating [= ‘I do not eat, I just stand around.’]

\noindent [‘I have not eaten; I am here.’ -- RB]\\

\noindent nə ɑŋɡɛŋɡɑ nɑkɑ mundərɑnde

\noindent mi kaikai bihain [= ‘I eat later.’]

\noindent [‘I will eat food later.’ -- RB]\\

\noindent [page] 3248\\

\noindent wʊni ƀəkɑ mundə rɑndu

\noindent yupela kaikai pastaim [= ‘You [\textsc{pl}] eat first.’]

\noindent [‘You [\textsc{pl}] will eat food first.’ -- RB]\\

\noindent wʊmbelɑp mʊndə rɑndu

\noindent I eat tumara [= ‘I eat tomorrow.’]

\noindent [‘[I] will eat food tomorrow.’ -- RB]\\

\noindent nɑ ɑƀɑ mundɑ mɑpe

\noindent I ate yesterday.

\noindent [‘I have eaten food yesterday.’ -- RB]\\

\noindent nə ƀuləƀa wɔpi

\noindent mi slip nating [= ‘I just sleep.’]

\noindent [‘I have just slept.’ -- RB]\\

\noindent [page] 3250\\

\noindent ADJECTIVES\\

\noindent ɑpɑ ɑmbiɡɑtɑ

\noindent [‘big house’]

\noindent [‘big house’ -- RB]\\

\noindent {\textasciitilde} sɔmɔlə [i.e., \textit{ɑpɑ sɔmɔlə}]

\noindent liklik [= ‘small’]

\noindent [‘small house’ -- RB]\\

\noindent {\textasciitilde} kərɑkɑ [i.e., \textit{ɑpɑ kərɑkɑ}]

\noindent nupela [= ‘new’]

\noindent [‘new house’ -- RB]\\

\noindent {\textasciitilde} wɔƀɑt [i.e., \textit{pɑ wɔƀɑt}]

\noindent olpela [= ‘old’]

\noindent [‘old house’ -- RB]\\

\noindent {\textasciitilde} ɑlmow [i.e., \textit{ɑpɑ ɑlmow}]

\noindent gutpela [= ‘good’]

\noindent [‘good house’ -- RB]\\

\newpage

\noindent {\textasciitilde} tɑmbiɡɑtɑ [i.e., \textit{ɑpɑ tɑmbiɡɑtɑ}]

\noindent haus nogut [= ‘bad house’]

\noindent [‘bad house’ -- RB]\\

\noindent yɪm kɑkɑs

\noindent longpela [= ‘long’]

\noindent [‘long tree’ -- RB]\\

\noindent {\textasciitilde} wɑnum [i.e., \textit{yɪm wɑnum}]

\noindent siotpela [= ‘short’]

\noindent [‘short tree’ -- RB]\\

\noindent yɑŋɡəƀɑtɑ

\noindent liklik (siotpela) [= ‘small (short)’]

\noindent [‘small, short’ (?) -- RB]\\

\noindent nɑ· rinɑi

\noindent mi sik [= ‘I am sick.’]

\noindent [‘I am sick.’ (?) -- RB]\\

\noindent wu li wɑyɑ?

\noindent yu sik? [= ‘Are you sick?’]

\noindent [‘Are you sick?’ (?) -- RB]\\

\noindent ɑnde, nɑ· rinɑyi

\noindent yes, mi sik [= ‘Yes, I am sick.’]

\noindent [‘Yes, I am sick.’ (?) -- RB]\\

\noindent nɑ misɑm ɑpənpi

\noindent het i pen [= ‘[My] head hurts.’]

\noindent [‘My head hurts.’ -- RB]\\

\noindent nɑ· ɑrəmow

\noindent mi no sik [= ‘I am not sick.’]

\noindent [‘I am good [= healthy]’ -- RB]\\

\newpage

\noindent inɑŋɡən

\noindent red

\noindent [‘red’ -- RB]\\

\noindent ƀɛndʊm

\noindent white

\noindent [‘white’ -- RB]\\

\noindent yiməkər

\noindent black

\noindent [‘black’ -- RB]\\

\noindent [page] 3252\\

\noindent inəm yiməmɑr

\noindent wara i hat [= ‘The water is hot.’]

\noindent [‘The water is hot.’ -- RB]\\

\noindent {\textasciitilde} tɑŋɡɑnmələ [i.e., \textit{inəm tɑŋɡɑnmələ}]

\noindent wara i kol [= ‘The water is cold.’]

\noindent [‘The water is cold.’ -- RB]\\

\noindent nɑmbim yimɑmɑr

\noindent skin i hat [= ‘[My] skin is hot.’]

\noindent [‘[My] skin is hot.’ -- RB]\\

\noindent {\textasciitilde} tɑŋɡɑnmɑlow [i.e., \textit{nɑmbim tɑŋɡɑnmɑlow}]

\noindent skin i kol [= ‘[My] skin is cold.’]

\noindent [‘[My] skin is cold.’ -- RB]\\

\noindent wʊɡən pɑt

\noindent strongpela man [= ‘strong man’]

\noindent [‘giant big man’ -- RB]\\

\noindent wʊmowurəƀow

\noindent nogat strong [= ‘weak’]

\noindent [‘weak’ (?) -- RB]\\

\noindent yirɑpʊmɑ

\noindent han siut [= ‘right hand’]

\noindent [‘right (not left)’ -- RB]\\

\noindent ɑndɑn yimow

\noindent han kais [= ‘left hand’]

\noindent [‘left (not right)’ -- RB]\\

\noindent yim nɑnɑru

\noindent [‘heavy tree’]

\noindent [‘heavy tree’ -- RB]\\

\noindent {\textasciitilde} bɑpɑŋɡɑw [i.e., \textit{yim bɑpɑŋɡɑw}]

\noindent [‘light tree’]

\noindent [‘light tree’ -- RB]\\

\noindent nɑmbərəm nɑmbium

\noindent skin i doti [= ‘[My] skin is dirty.’]

\noindent [‘[My] skin is dirty.’ -- RB]\\

\noindent [page] 3254\\

\noindent mundə dɑ ɑrəmɔpi

\noindent kaikai i dan [= ‘The food is cooked.’]

\noindent [‘The foods have cooked.’ -- RB]\\

\noindent mɑlɑl dɑ mɑp

\noindent ol i boilim pinis [= ‘They already boiled.’]

\noindent [‘They have boiled.’ (Literally ‘Hot water has eaten them.’) -- RB]\\

\noindent ɑkəlɑkowɑ

\noindent raw

\noindent [‘raw’ -- RB]\\

\noindent dɑŋɡwɔ ɑrəmɔpɛk

\noindent not cooked

\noindent [‘They have not cooked.’ -- RB]\\

\noindent məndɑ mənɑtɑ

\noindent banana mau [= ‘ripe banana’]

\noindent [‘ripe banana’ -- RB]\\

\noindent ɑkolɑkɑ

\noindent nupela [= ‘new’; intended, here: ‘unripe’]

\noindent [‘new’ -- RB]\\

\noindent wɔlʊm lɑndɑ

\noindent drinks breast [= ‘[The child] nurses.’]

\noindent [‘[The child] will nurse.’ (Literally ‘[The child] will eat breast.’) -- RB]\\

\noindent [page] 3256\\

\noindent VERBS\\

\noindent inəm lɑndɑne

\ia{Laycock, Donald C.|)}
\il{Maruat-Dimiri-Yaul|)}
\il{Yaul|)}
\il{Yaul|(}
\il{Maruat-Dimiri-Yaul|(}
\ia{Laycock, Donald C.|(}

\noindent I drink water

\noindent [‘[I] will drink water.’ -- RB]\\

\noindent tənɔ ŋɡɑndɑ

\noindent stand [= ‘stand’]

\noindent [‘will stand’ -- RB]\\

\noindent ɑsi kɑndɑ

\noindent yumi sindaun [= ‘We [\textsc{incl}] sit.’]

\noindent [‘will sit’ -- RB]\\

\noindent wɔrɑni

\noindent slip [= ‘sleep’]

\noindent [‘will sleep’ -- RB]\\

\noindent mɑri·pi

\noindent em i dai [= ‘He dies.’]

\noindent [‘He died.’ -- RB]\\

\newpage

\noindent ɑnɑ lɑtɑndi

\noindent we tok [‘We talk.’]

\noindent [‘We will talk.’ -- RB]\\

\noindent sɑkwey wʊ sɑkwɔy tinɑ (nɑnɑndu!)

\noindent give me brus! [= ‘Give me tobacco!’]

\noindent [‘Tobacco – [give] [\textsc{sg}] me tobacco!’ -- RB]\\

\noindent sɑkwey wuluƀɑpe

\noindent nogat brus [= ‘There is no tobacco.’]

\noindent [‘There is no tobacco.’ -- RB]\\

\noindent nɑŋɡɔ tətunɑndɑ

\noindent mi no ken gipim yu [= ‘I cannot give you [\textsc{sg}].’]

\noindent [‘I cannot give you [\textsc{sg}].’ -- RB]\\

\noindent wu nəməndə ɑŋɡwɔsɑli?

\noindent [‘What do you [\textsc{sg}] see?’]

\noindent [‘What do you [\textsc{sg}] see?’ -- RB]\\

\noindent nɑ nimdi wʊtɑmɑli

\noindent I see a bird

\noindent [‘I see a bird.’ -- RB]\\

\noindent mɑ· mbu\specialchar{ᵽ}i

\noindent em i stap [= ‘It stays.’]

\noindent [‘It is here.’ -- RB]\\

\noindent winɛ mɑsindɑ

\noindent yu kam siutim [= ‘Come [\textsc{sg}] shoot [it]!’]

\noindent [‘Come [\textsc{sg}] shoot it!’ -- RB]\\

\noindent [page] 3258\\

\noindent nɑ ŋɡɔ nəməndə mɑli

\noindent mi no lukim [= ‘I do not see.’]

\noindent [‘I do not see it.’ -- RB]\\

\noindent nə mɑ ɑndərɔli

\noindent mi lukautim [= ‘I look after [it].’ (?)]

\noindent [‘I see it.’ -- RB]\\

\noindent wi·nu, winɛ mɑrɑ nəkəlu ni mɑɑndərɔl

\noindent yu kam soim mi mi lukautim [= ‘Come [\textsc{sg}] show me! I look after [it].’ (?)]

\noindent [‘Come [\textsc{sg}]! Come [\textsc{sg}] show me! I see it.’ -- RB]\\

\noindent nə nəməndɑ mɑ·li [mɑ ɑli]\footnote{These brackets are Laycock’s.}

\noindent mi lukim pinis [= ‘I have seen.’]

\noindent [‘I see it.’ -- RB]\\

\noindent mɑ yiyɛ

\noindent em i go pinis [= ‘It has gone.’]

\noindent [‘It has gone.’ -- RB]\\

\noindent nə nəmdɑmɑ ɑndrɑni

\noindent bai mi lukim [= ‘I will see.’]

\noindent [‘I will see it.’ -- RB]\\

\noindent mɑ ɑpəlu

\noindent yu (em?) stap [= ‘You [\textsc{sg}] (he?) stay(s).’]

\noindent [‘Stay [\textsc{sg}] there! -- RB]\\

\noindent nə kəɡɑl mɑ·ƀɑli

\noindent mi harim [= ‘I hear.’]

\noindent [‘I hear it.’ -- RB]\\

\noindent nɑ ŋɡwɔ kəkɑl mɑƀɑli ko

\noindent mi no harim [= ‘I do not hear.’]

\noindent [‘I do not hear it.’ -- RB]\\

\noindent nə mɑ· nɑmbis ƀɑli

\noindent mi harim smel [= ‘I hear a smell.’]

\noindent [‘I smell it.’ (Literally ‘I perceive its smell.’) -- RB]\\

\noindent [page] 3260\\

\noindent ɑnəndʒi tokples mɑƀow

\noindent tokples bilong mipela [= ‘our [\textsc{excl}] language’]

\noindent [‘our [\textsc{excl}] very own language’ -- RB]\\

\noindent low(u)

\noindent tokples [= ‘language’]

\noindent [‘language’ -- RB]\\

\noindent wɑŋɡɔrɑ nindʒi tɨn mɑƀɑli?

\noindent why yu hit my dog? [= ‘Why do you [\textsc{sg]} hit my dog?’]

\noindent [‘Why do you [\textsc{sg}] hit my dog?’ -- RB]\\

\noindent mə nəl ƀutumwɔmpi nə mɑsi

\noindent it bit me – I hit it [‘It bit me; I hit it.’]

\noindent [‘It bit me, [so] I am hitting it.’ -- RB]\\

\noindent wɑ ŋɡɔdɑ mɑƀɑle nindʒi tɨnə, mɑrəkɑndɑ!

\noindent yu no ken paitim dok bilong mi, lusim! [= ‘Do [\textsc{sg}] not hit my dog! Leave it!’]

\noindent [‘Do [\textsc{sg}] not hit it! Leave my dog [alone]! -- RB]\\

\noindent nu wɑlindɑni

\noindent I will hit yu (?) [= ‘I hill hit you [\textsc{sg}] (?)’]

\noindent [‘I will hit you [\textsc{sg]}.’ -- RB]\\

\noindent ɡwʊnɑn ɑmbin ƀɑlindɑ

\noindent bai yu mi pait [= ‘We [\textsc{incl]} will fight.’]

\noindent [‘We [\textsc{du.incl}] will fight.’ (Literally ‘We [\textsc{du.incl}] hit each other.’) -- RB]\\

\noindent [page] 3262\\

\noindent nɪndʒi tɨn

\noindent [‘my dog’]

\noindent [‘my dog’ -- RB]\\

\newpage

\noindent wʊndʒi tɨn?

\noindent [‘your [\textsc{sg}] dog?’]

\noindent [‘your [\textsc{sg}] dog?’ -- RB]\\

\noindent mɑndʒi tɨn u

\noindent his dog

\noindent [‘his/her/its dog’ -- RB]\\

\noindent ŋɡɑnəndʒi tɨnɑ

\noindent our 2 dog [= ‘our [\textsc{du}] dog’]

\noindent [‘our [\textsc{du.excl}] dog’ -- RB]\\

\noindent ndidʒi tɨnɑ

\noindent their dog [= ‘their [\textsc{pl}] dog’]

\noindent [‘their [\textsc{pl}] dog’ -- RB]\\

\noindent ɑŋɡwɔ nindʒi meko

\noindent not my dog [= ‘not my dog’]

\noindent [‘not mine’ -- RB]\\

\noindent ɡwʊnəndʒi tɨnᵊ mɑ?

\noindent your 2 dog? [= ‘your [\textsc{du}] dog?’]

\noindent [‘[Is] it your [\textsc{du}] dog?’ -- RB]\\

\noindent minəndʒi tɨn kwɑƀo

\noindent bilong tupela narapela [= ‘two others’’]

\noindent [‘some other two’s dog’ (?) -- RB]\\

\noindent ɑnəndʒi tɨnu

\noindent our all dog [= ‘our [\textsc{pl}] dog’]

\noindent [‘our [\textsc{pl.excl}] dog’ -- RB]\\

\noindent wunəndʒi tɨnu?

\noindent bilong yupela [= ‘yours [\textsc{pl}]?’]

\noindent [‘your [\textsc{pl}] dog?’ -- RB]\\

\newpage

\noindent ndɑƀɑndʒi tɨnu (ndɑndʒi)

\noindent their dog [= ‘their [\textsc{pl}] dog’]

\noindent [‘those [people]’s dog’ -- RB]\\

\noindent miminyɑ tu·ndɑn

\noindent mi pekpek [= ‘I defecate.’]

\noindent [‘[I] will defecate.’ -- RB]\\

\noindent minɑm tundɑni

\noindent mi pispis [= ‘I urinate.’]

\noindent [‘[I] will urinate.’ -- RB]\\

\noindent yɛnɑwɑlindɑni

\noindent I fuck woman [= ‘I copulate with a woman.’]

\noindent [‘[I] will hit a woman.’ -- RB]\\

\noindent [page] 3264\\

\noindent nɑ mɑnɑne

\ia{Laycock, Donald C.|)}
\il{Maruat-Dimiri-Yaul|)}
\il{Yaul|)}
\il{Yaul|(}
\il{Maruat-Dimiri-Yaul|(}
\ia{Laycock, Donald C.|(}

\noindent bai me go [= ‘I will go.’]

\noindent [‘I will go.’ -- RB]\\

\noindent ɑnɑ ɸwɑ {\textasciitilde} [i.e., \textit{ɑnɑ ɸwɑ mɑnɑne}]

\noindent mipela olgeta i go [= ‘We all go.’]

\noindent [‘We all will go.’ -- RB]\\

\noindent nɑ yinɑne

\noindent mi kam [= ‘I come.’]

\noindent [‘I will come.’ -- RB]\\

\noindent nə mɑmɑpe wey nɑne

\noindent mi go bai mi kambek [= ‘I go; I will come back.’]

\noindent [‘I went and stayed there, [but] I will go to the village.’ -- RB]\\

\noindent wi·nɑ!

\noindent yu kam [= ‘Come! [\textsc{sg}]’]

\noindent [‘Come! [\textsc{sg}]’ -- RB]\\

\noindent wʊ mɑnɑne!

\noindent yu go [= ‘Go! [\textsc{sg}]’]

\noindent [‘Go! [\textsc{sg}]’ -- RB]\\

\noindent wʊ mɑnɑ!

\noindent yu go! [= ‘Go! [\textsc{sg}]’]

\noindent [‘Go! [\textsc{sg}]’ -- RB]\\

\noindent wʊ rɑndɑnɑ! (wʊrɑndɑ!)

\noindent yu kaikai [= ‘Eat! [\textsc{sg}]’]

\noindent [‘Eat! [\textsc{sg}]’ -- RB]\\

\noindent nə mbəplɑn

\noindent I stay

\noindent [‘I will be here.’ -- RB]\\

\noindent ŋɡɑn bəplɑn

\noindent we stay

\noindent [‘We [\textsc{du.excl}] will be here.’ -- RB]\\

\noindent wʊm bəplɑne?

\noindent yu stap? [= ‘Do you [\textsc{sg}] stay?’]

\noindent [‘Will you [\textsc{sg}] be here?’ -- RB]\\

\noindent wʊ nəmdɑmɑ·rɑ?

\noindent yu lukim wanpela na tu? [= ‘Do you [\textsc{pl}] one and two?’ (?)]

\noindent [‘Do you see it?’ -- RB]\\

\noindent nə nəmdɑmɑri

\noindent mi lukim pinis [= ‘I have seen.’]

\noindent [‘I see it.’ -- RB]\\

\noindent nə kəkɑlmɑ ƀɑli

\noindent mi harim [= ‘I hear.’]

\noindent [‘I hear it.’ -- RB]\\

\noindent nə sikul mɑkɑn

\noindent mi save [= ‘I know.’]

\noindent [‘I learn.’ -- RB]

\ia{Laycock, Donald C.|)}
\il{Maruat-Dimiri-Yaul|)}
\il{Yaul|)}