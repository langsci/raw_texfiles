\chapter{Lexicon}\label{sec:17}

\is{wordlist|(}
\is{lexicon|(}

This chapter provides an Ulwa wordlist. First, \sectref{sec:17.1}, presents 1,429 Ulwa \isi{lexical} entries, each with an \ili{English} translation or explanation. This list includes every Ulwa word and morpheme mentioned in this grammar, as well as a number of other words taken from texts or recorded during elicitation sessions. Then, \sectref{sec:17.2} provides an \ili{English}-to-Ulwa finder list. This is intended to be a quick and simple means of finding words in Ulwa and, as such, does not provide lengthy definitions. Finally, for convenient reference, \sectref{sec:17.3} presents a list of \isi{bound morpheme}s (i.e., \isi{affix}es and \isi{clitic}s) along with their glossing abbreviations.

\is{lexicon|)}
\is{wordlist|)}

\section{Ulwa-to-English wordlist}\label{sec:17.1}

\is{wordlist|(}
\is{lexicon|(}

In the following wordlist, the Ulwa words are organized alphabetically, following the conventions of \ili{English} and \ili{Tok Pisin} alphabetization. For ease of use, the \isi{digraph}s <mb>, <nd>, <ng>, <nj>, and <ae> are treated as series of two characters each. That is, although each \isi{digraph} represents a single phoneme in Ulwa, they are alphabetized as if they were composed of separate letters.\footnote{Thus, for example, the words \textit{ana-} ‘scrub’, \textit{anda} ‘that’, and \textit{ane} ‘sun’ are presented in that order, even though /ana-/ and /ane/ share the first two phonemes, whereas /anda/ has a different second phoneme.} This separation of \isi{phonological}ly more similar words is made in the interest of facilitating the discovery of \isi{lexical} items. The one exception to this scheme is that word-initial \isi{prenasalized} \isi{voiced} \isi{stop}s are treated as distinct \isi{grapheme}s and, as such, received their own letter headings (\textbf{<mb>}, \textbf{<nd>}, \textbf{<ng>}, \textbf{<nj>}). \isi{Proper noun}s that begin with these phonemes, are written, however, without the \isi{nasal} component, and they are alphabetized accordingly (cf. \sectref{sec:1.4}). The letter <ï> is alphabetized along with <i>.

  Word classes are identified following the Ulwa word and preceding the \mbox{\ili{English}} translation. These should not be taken as definitive statements about \isi{lexical class}es in the language, but are rather, in many instances, rather roughly defined, based in part on \isi{semantic} criteria. More detailed information on \isi{word class} is provided throughout the grammar. The abbreviations used for roughly classifying \isi{lexical} entries are given in \REF{ex:lexicon:1}.

\ea%1
    \label{ex:lexicon:1}
            Word class abbreviations used in the wordlist\\

  \begin{tabbing}
{((\textsc{interj}))} \= {(\isi{question word})} \= {((\textsc{interj}))} \= {(\isi{question word})}\kill
{(\textsc{adj})} \> {\isi{adjective}} \> {(\textsc{num})} \> {\isi{numeral}}\\
{(\textsc{adv})} \> {\isi{adverb}} \> {(\textsc{p})} \> {\isi{postposition}}\\
{(\textsc{conj})} \> {\isi{conjunction}} \> {(\textsc{pro})} \> {\isi{pronoun}}\\
{(\textsc{dem})} \> {\isi{demonstrative}} \> {(\textsc{q})} \> {\isi{question word}}\\
{(\textsc{interj})} \> {\isi{interjection}} \> {(\textsc{quant})} \> {\isi{quantifier}}\\
{(\textsc{n})} \> {noun} \> {(\textsc{v})} \> {verb}\\
{(\textsc{neg})} \> {negator} \> {} \> {}
\end{tabbing}
\z

  Entries for verbs take the form of the verb’s \isi{stem}. See \chapref{sec:4} for information on the conjugation of verbs. If a verb uses different \isi{stem}s in its paradigm (e.g., the \isi{irregular verb} \textit{ama-} {\textasciitilde} \textit{la-} ‘eat’), each \isi{stem} receives its own entry in the wordlist. \isi{Separable verb}s are written with a space between the separable elements, to help show how these words may be used in \isi{verb phrase}s (see \sectref{sec:9.2.1}--\sectref{sec:9.2.3}).

  When the \ili{English} gloss is not a translation of the Ulwa word but rather a description (i.e., a grammatical gloss), it is set in square brackets. For example, the Ulwa entry \textbf{\textit{-p}} is glossed as “[\isi{perfective} \isi{suffix}, ‘\textsc{pfv}’]”. \isi{Proper noun}s are also indicated as such with brackets: for example, \textbf{\textit{Alimban}} is glossed as “[male name]”, \textbf{\textit{Alkumot}} is glossed as “[female name]”, and \textbf{\textit{Talamba}} is glossed as “[place] jungle region near Manu village”.

  \isi{Loanword}s are flagged as such when known or suspected, with the arrow symbol (<) indicating the source language or language family. Where deemed helpful, \ili{Tok Pisin} translations are provided for some words (in parentheses following the abbreviation “TP”), in addition to the \ili{English} translation. Literal meanings or \mbox{etymologies} of \isi{compound} forms are occasionally provided (also in parentheses).\\

\is{lexicon|)}
\is{wordlist|)}

\begin{enumerate}[noitemsep, label={}, align=left, widest=190, labelsep=1ex,leftmargin=*,itemindent=-10pt]

\item
\noindent \textbf{<A, a>        [a]}\\ \item

\textbf{\textit{a}} (\textsc{interj}) ah (expresses shock or disbelief; also can introduce quoted \isi{speech}); uh (filler \isi{interjection}); eh? (\isi{tag question} \isi{interjection}) \item
\textbf{\textit{a-}} (\textsc{v}) break \item
\textbf{\textit{akal}} (\textsc{n}) ringworm, tinea; any white, ashy skin condition \item
\textbf{\textit{akat}} (\textsc{n}) hoof (of a pig) \item
\textbf{\textit{akatoma}} (\textsc{n}) fork \item
\textbf{\textit{akïnaka}} (\textsc{adj}) new, fresh, alive, raw, young \item
\textbf{\textit{akïnanga}} (\textsc{n}) palm frond \item
\textbf{\textit{akum}} (\textsc{n}) type of basket (basket made from sago fronds, used as a container) \item
\textbf{\textit{akunpu}} (\textsc{n}) back of the skull (occipital bone) (cf. \textbf{\textit{inpu}} ‘elbow’, \textbf{\textit{wutïnpu}} ‘heel’) \item
\textbf{\textit{al}} (\textsc{n}) type of beam (long beam in a house, running to the roof) \item
\textbf{\textit{al}} (\textsc{n}) loincloth, man’s grass skirt (TP \textit{malo}) \item
\textbf{\textit{al}} (\textsc{n}) mosquito net (traditional covering made of woven sago shoots) \item
\textbf{\textit{-al}} [irregular \isi{perfective} \isi{suffix}, ‘\textsc{pfv}’] (for \textbf{\textit{si-}} ‘push’)
\item \textbf{\textit{al nambï}} (\textsc{n}) bedsheet; cloth, clothing (literally ‘mosquito net skin’) \item
\textbf{\textit{ala}} (\textsc{dem}) those (\isi{plural} \isi{distal} \isi{demonstrative}, ‘\textsc{pl.dist}’) \item
\textbf{\textit{ala}} (\textsc{p}) for, from (also \textbf{\textit{andï}}, \textbf{\textit{andïn}}, \textbf{\textit{andïm}}) \item
\textbf{\textit{ala namnap}} (\textsc{v}) fear, be afraid of, be scared of (literally ‘be afraid from’) \item
\textbf{\textit{ala-}} (\textsc{v}) see (used with \textbf{\textit{lïmndï}} ‘eye’) (also \textbf{\textit{andï-}}) \item
\textbf{\textit{ala=}} (\textsc{dem}) those (non-subject \isi{plural} \isi{distal} \isi{demonstrative}, ‘\textsc{pl.dist}’) \item 
\textbf{\textit{alakamb-}} (\textsc{v}) dislike, disapprove of, hate (literally ‘shun from’) \item 
\textbf{\textit{alalama}} (\textsc{n}) maturing coconut fruit (older than an \textbf{\textit{andïmoni}} ‘young coconut’, but not yet \textbf{\textit{wapata}} ‘dry’) \item 
\textbf{\textit{alaman}} (\textsc{n}) sago species (large sago palm with spines) \item 
\textbf{\textit{alambi}} (\textsc{dem}) as for those ones (\isi{plural} \isi{distal} \isi{topic-marker} \isi{demonstrative}, \linebreak‘\textsc{pl.dist-top}’) \item 
\textbf{\textit{alanji}} (\textsc{dem}) those ones’ (\isi{plural} \isi{distal} possessive \isi{demonstrative}, ‘\textsc{pl.dist-poss}’) \item 
\textbf{\textit{alata}} (\textsc{adj}) rotting, decaying \item 
\textbf{\textit{alawa}} (\textsc{dem}) those themselves (\isi{intensive} \isi{plural} \isi{distal} \isi{demonstrative}, \linebreak ‘\textsc{pl.dist-int}’) \item 
\textbf{\textit{alawe}} (\textsc{dem}) those themselves (from among several) (\isi{plural} \isi{distal} \linebreak \isi{partitive-intensive} \isi{demonstrative}, ‘\textsc{pl.dist-part.int}’) \item 
\textbf{\textit{ale-}} (\textsc{v}) scrape (sago) (possibly < \ili{Ap Ma}) \item 
\textbf{\textit{ali-}} (\textsc{v}) scrape (sago) (\isi{irrealis} \isi{stem} of \textbf{\textit{ale-}}) (possibly < \ili{Ap Ma}) \item 
\textbf{\textit{alima-}} (\textsc{v}) beat (sago pulp) \item 
\textbf{\textit{Alimban}} [male name] \item 
\textbf{\textit{Alkumot}} [female name] \item 
\textbf{\textit{Alma}} [male name] \item 
\textbf{\textit{almba}} (\textsc{n}) bird species (hornbill) (TP \textit{kokomo}) (< \ili{Ap Ma}) \item 
\textbf{\textit{almbïne}} (\textsc{n}) banana species (plantain banana plant with bunches containing very many fruit) \item 
\textbf{\textit{alsa}} (\textsc{n}) scorpion \item 
\textbf{\textit{alum}} (\textsc{n}) child, baby (possibly < \ili{Ap Ma}) \item 
\textbf{\textit{alwoma}} (\textsc{n}) type of beam (support beam in a house) \item 
\textbf{\textit{ama-}} (\textsc{v}) eat, drink; chew, bite, suck; smoke (tobacco) \item 
\textbf{\textit{Amali}} [place] site of the third Manu village, near present-day Bun village \item 
\textbf{\textit{amam}} (\textsc{n}) insect species (insect similar to a ladybug that lives in the water) \item 
\textbf{\textit{amangala}} (\textsc{n}) bird species (brown eagle, hawk) (TP \textit{tarangau}) (possibly \linebreak < \ili{Ap Ma}) \item 
\textbf{\textit{amba}} (\textsc{n}) men’s house, spirit house (TP \textit{haus tambaran} or \textit{haus boi}); clan; magic \item 
\textbf{\textit{ambalka}} (\textsc{adj}) flat, equal (possibly < \textbf{\textit{ambla}} [\isi{plural} \isi{reciprocal} \isi{pronoun}] + \textbf{\textit{ka}} ‘thus’) \item 
\textbf{\textit{ambatïm}} (\textsc{n}) joint \item 
\textbf{\textit{ambawa}} (\textsc{pro}) myself, yourself, himself, herself, itself (\isi{intensive} \isi{singular} \linebreak \isi{reflexive} \isi{pronoun}, ‘\textsc{sg.refl-int}’) \item 
\textbf{\textit{Ambawanam}} [male name] \item 
\textbf{\textit{Ambayam}} [female name] \item 
\textbf{\textit{ambep}} (\textsc{n}) front of the men’s house (< \textbf{\textit{amba}} ‘men’s house’ + \textbf{\textit{ip}} ‘nose’) \item 
\textbf{\textit{ambet}} (\textsc{n}) magic; poison \item 
\textbf{\textit{ambi}} (\textsc{adj}) big, large; much; (\textsc{n}) big man, God \item 
\textbf{\textit{-ambi}} [\isi{topic-marker} \isi{suffix}, ‘\textsc{top}’] \item 
\textbf{\textit{ambï=}} (\textsc{pro}) myself, yourself, himself, herself, itself (\isi{singular} \isi{reflexive} \isi{pronoun}, ‘\textsc{sg.refl}’) \item 
\textbf{\textit{ambin}}= (\textsc{pro}) ourselves, yourselves, themselves, (\isi{dual} \isi{reflexive} \isi{pronoun}, \linebreak ‘\textsc{du.refl}’); each other (\isi{dual} \isi{reciprocal} \isi{pronoun}, ‘\textsc{du.refl}’) \item 
\textbf{\textit{ambinawa}} (\textsc{pro}) ourselves, yourselves, themselves (\isi{intensive} \isi{dual} \isi{reflexive} \linebreak \isi{pronoun}, ‘\textsc{du.refl-int}’) \item 
\textbf{\textit{ambinji}} (\textsc{pro}) our own, your own, their own (\isi{dual} \isi{reflexive} \isi{possessive pronoun}, ‘\textsc{du.refl-poss}’) \item 
\textbf{\textit{ambïnji}} (\textsc{pro}) my own, your own, his own, her own, its own (\isi{singular} \isi{reflexive} \isi{possessive pronoun}, ‘\textsc{sg.refl-poss}’) \item 
\textbf{\textit{Ambïnme}} [male name] \item 
\textbf{\textit{ambinwe}} (\textsc{pro}) ourselves, yourselves, themselves (partitive-intensive \isi{dual} \linebreak \isi{reflexive} \isi{pronoun}, ‘\textsc{du.refl-part.int}’) \is{partitive-intensive pronoun}\item 
\textbf{\textit{ambla}} (\textsc{n}) tooth; stinger (of an insect) \item 
\textbf{\textit{ambla lam}} (\textsc{n}) gums (literally ‘tooth flesh’) \item 
\textbf{\textit{ambla=}} (\textsc{pro}) ourselves, yourselves, themselves (\isi{plural} \isi{reflexive} \isi{pronoun}, \linebreak ‘\textsc{pl.refl}’); one another (\isi{plural} \isi{reciprocal} \isi{pronoun}, ‘\textsc{pl.refl}’) \item 
\textbf{\textit{amblanji}} (\textsc{pro}) our own, your own, their own (\isi{plural} \isi{reflexive} \isi{possessive pronoun}, ‘\textsc{pl.refl-poss}’) \item 
\textbf{\textit{amblawa}} (\textsc{pro}) ourselves, yourselves, themselves (\isi{intensive} \isi{plural} \isi{reflexive} \isi{pronoun}, ‘\textsc{pl.refl-int}’) \item 
\textbf{\textit{amblawali-}} (\textsc{v}) fight, battle (literally ‘hit one another’) \item 
\textbf{\textit{amblawe}} (\textsc{pro}) ourselves, yourselves, themselves (partitive-intensive \isi{plural} \linebreak \isi{reflexive} \isi{pronoun}, ‘\textsc{pl.refl-part.int}’) \is{partitive-intensive pronoun}\item 
\textbf{\textit{Amblom}} [female name] \item 
\textbf{\textit{Ambonda}} [female name] \item 
\textbf{\textit{ambunmbï}} (\textsc{n}) back of the men’s house (< \textbf{\textit{amba}} ‘men’s house’ + \textbf{\textit{unmbï}} \linebreak ‘buttocks’) \item 
\textbf{\textit{ambuwe}} (\textsc{pro}) myself, yourself, himself, herself, itself (partitive-intensive \linebreak \isi{singular} \isi{reflexive} \isi{pronoun}, ‘\textsc{sg.refl-part.int}’) \is{partitive-intensive pronoun}\item 
\textbf{\textit{Ambwat}} [place] Kambot village \item 
\textbf{\textit{ame}} (\textsc{n}) type of basket (basket made from sago shoots, used for carrying sago \linebreak starch); uterus, marsupial pouch \item 
\textbf{\textit{amendum}} (\textsc{n}) plant species (stinging nettle with small leaves) (TP \textit{salat}) \item 
\textbf{\textit{ametamal}} (\textsc{n}) spoon made from a coconut shell \item 
\textbf{\textit{Amiwa}} [male name] \item 
\textbf{\textit{amla}} (\textsc{n}) tree species (Pacific walnut) (TP \textit{mon}) \item 
\textbf{\textit{Amombi}} [male name] \item 
\textbf{\textit{amun}} (\textsc{adv}) now, today, nowadays, recently, still, yet \item 
\textbf{\textit{amunji}} (\textsc{n}) young person \item 
\textbf{\textit{an}} (\textsc{pro}) we (\textsc{1pl.excl} subject \isi{pronoun}, ‘\textsc{1pl.excl}’) \item 
\textbf{\textit{an}} (\textsc{adv}) out (only occurs with \textbf{\textit{mbï}} ‘here’; cf. \textbf{\textit{anmbï}} ‘outside’) \item 
\textbf{\textit{an=}} (\textsc{pro}) us (\textsc{1pl.excl} non-subject \isi{pronoun}, ‘\textsc{1pl.excl}’) \item 
\textbf{\textit{ana}} (\textsc{n}) skirt, woman’s grass skirt (TP \textit{purpur}); a parasitic person; hair on the tip of an animal’s tail \item 
\textbf{\textit{ana-}} (\textsc{v}) scrub, scratch \item 
\textbf{\textit{anam}} (\textsc{n}) sky, cloud; lightning \item 
\textbf{\textit{Anam}} [male name] \item 
\textbf{\textit{anam wapata}} (\textsc{n}) thunder (literally ‘dry sky’) \item 
\textbf{\textit{anambi}} (\textsc{pro}) as for us (\textsc{1pl.excl} \isi{topic-marker pronoun}, ‘\textsc{1pl.excl-top}’) \item 
\textbf{\textit{anangum}} (\textsc{n}) spine, backbone \item 
\textbf{\textit{anankïn}} (\textsc{n}) blood \item 
\textbf{\textit{anapa}} (\textsc{n}) sister \item 
\textbf{\textit{anapot}} (\textsc{n}) type of skirt (short grass skirt for men) \item 
\textbf{\textit{anasa}} (\textsc{n}) pick-axe (for hacking at sago palms) (possibly < \ili{Pondi}) \item 
\textbf{\textit{anat}} (\textsc{n}) vegetable species (ginger) (TP \textit{kawawar}) \item 
\textbf{\textit{anaw}} (\textsc{n}) paddle; fishtail; outboard motor of a canoe; motorboat \item 
\textbf{\textit{anawa}} (\textsc{pro}) we ourselves, us ourselves (1\textsc{pl.excl} \isi{intensive pronoun}, \linebreak ‘\textsc{1pl.excl-int}’) \item 
\textbf{\textit{anda}} (\textsc{dem}) that (\isi{singular} \isi{distal} \isi{demonstrative}, ‘\textsc{sg.dist}’) \item 
\textbf{\textit{anda}} (\textsc{adv}) there; to there, thither \item 
\textbf{\textit{anda=}} (\textsc{dem}) that (non-subject \isi{singular} \isi{distal} \isi{demonstrative}, ‘\textsc{sg.dist}’) \item 
\textbf{\textit{andambi}} (\textsc{dem}) as for that one (\isi{singular} \isi{distal} \isi{topic-marker} \isi{demonstrative}, \linebreak ‘\textsc{sg.dist-top}’) \item 
\textbf{\textit{andana}} (\textsc{n}) left, left-hand side \item 
\textbf{\textit{andanam}} (\textsc{dem}) that is it (\isi{singular} \isi{distal} \isi{emphatic} \isi{demonstrative}, \linebreak‘\textsc{sg.dist-emph}’) \item 
\textbf{\textit{andanji}} (\textsc{dem}) that one’s (\isi{singular} \isi{distal} possessive \isi{demonstrative}, \linebreak‘\textsc{sg.dist-poss}’) \item 
\textbf{\textit{andawa}} (\textsc{dem}) that itself (\isi{intensive} \isi{singular} \isi{distal} \isi{demonstrative}, ‘\textsc{sg.dist-int}’) \item 
\textbf{\textit{andawe}} (\textsc{dem}) that itself (from among several) (\isi{singular} \isi{distal} \linebreak \isi{partitive-intensive} \isi{demonstrative}, ‘\textsc{sg.dist-part.int}’) \item 
\textbf{\textit{ande}} (\textsc{interj}) OK, okay (expresses agreement, etc.) (also \textbf{\textit{andi}}) \item 
\textbf{\textit{andi}} (\textsc{interj}) OK, okay (expresses agreement, etc.) (also \textbf{\textit{ande}}) \item 
\textbf{\textit{andï}} (\textsc{n}) sago shoot \item 
\textbf{\textit{andï}} (\textsc{p}) for, from (also \textbf{\textit{andïm}}, \textbf{\textit{andïn}}; \textbf{\textit{ala}}) \item 
\textbf{\textit{andï-}} (\textsc{v}) see (used with \textbf{\textit{lïmndï}} ‘eye’) (also \textbf{\textit{ala-}}) \item 
\textbf{\textit{andïl}} (\textsc{adj}) careful, slow, quiet \item 
\textbf{\textit{andïla}} (\textsc{p}) waiting for, awaiting (also \textbf{\textit{angla}}) \item 
\textbf{\textit{andïlalo-}} (\textsc{v}) hunt, seek (literally ‘go awaiting’) (also \textbf{\textit{anglalo}}) \item 
\textbf{\textit{andïm}} (\textsc{p}) for, from (also \textbf{\textit{andï}}, \textbf{\textit{andïn}}; \textbf{\textit{ala}}) \item 
\textbf{\textit{Andïmali}} [place] Dimiri village \item 
\textbf{\textit{andïmoni}} (\textsc{n}) young coconut, drinking coconut (TP \textit{kulau}) \item 
\textbf{\textit{andin}} (\textsc{dem}) those (\isi{dual} \isi{distal} \isi{demonstrative}, ‘\textsc{du.dist}’) \item 
\textbf{\textit{andïn}} (\textsc{p}) for, from (also \textbf{\textit{andï}}, \textbf{\textit{andïm}}; \textbf{\textit{ala}}) \item 
\textbf{\textit{andin=}} (\textsc{dem}) those (non-subject \isi{dual} \isi{distal} \isi{demonstrative}, ‘\textsc{du.dist}’) \item 
\textbf{\textit{andinambi}} (\textsc{dem}) as for those ones (\isi{dual} \isi{distal} \isi{topic-marker} \isi{demonstrative}, ‘\textsc{du.dist-top}’) \item 
\textbf{\textit{andinawa}} (\textsc{dem}) those themselves (\isi{intensive} \isi{dual} \isi{distal} \isi{demonstrative}, \linebreak‘\textsc{du.dist-int}’) \item 
\textbf{\textit{andinji}} (\textsc{dem}) those ones’ (\isi{dual} \isi{distal} possessive \isi{demonstrative}, ‘\textsc{du.dist-poss}’) \item 
\textbf{\textit{andinwe}} (\textsc{dem}) those themselves (from among several) (\isi{dual} \isi{distal} \linebreak \isi{partitive-intensive} \isi{demonstrative}, ‘\textsc{du.dist-part.int}’) \item 
\textbf{\textit{andïpipi}} (\textsc{n}) pimple \item 
\textbf{\textit{ando}} (\textsc{adv}) there; from there, thence \item 
\textbf{\textit{anduwan}} (\textsc{n}) young sago palm \item 
\textbf{\textit{andwana}} (\textsc{adj}) yellow \item 
\textbf{\textit{ane}} (\textsc{n}) sun; midday, day (daytime); (\textsc{adj}) yellow, light (color) \item 
\textbf{\textit{ane anma}} (\isi{greeting}) good day \item 
\textbf{\textit{ane inom}} (\textsc{n}) father’s sister (paternal aunt) (literally ‘sun mother’) \item 
\textbf{\textit{ane inom atana}} (\textsc{n}) father’s older sister (paternal aunt) (literally ‘older sister sun mother’) \item 
\textbf{\textit{ane inom wot}} (\textsc{n}) father’s younger sister (paternal aunt) (literally ‘younger sun mother’) \item 
\textbf{\textit{ane mongi}} (\textsc{n}) banana species (banana plant with sweet, red fruit, traditionally eaten only by men) (literally ‘sun \textbf{\textit{mongi}} banana species’) \item 
\textbf{\textit{ane ngungun ane}} (\textsc{n}) rainbow (literally ‘sun red sun’, possibly related to \textbf{\textit{ngum}} ‘snake species’) \item 
\textbf{\textit{ane uta}} (\textsc{n}) bird species (small brown bird with a beak like a parrot’s that sings in the dry season) (literally ‘sun bird’) \item 
\textbf{\textit{ane wapata}} (\textsc{n}) dry season (literally ‘dry sun’) \item 
\textbf{\textit{ane wombam}} (\textsc{n}) noon, midday (literally ‘middle sun’) \item 
\textbf{\textit{anem}} (\textsc{n}) plant species (plant with seeds used for making necklace beads); \linebreak (\textsc{adj}) blue, purple \item 
\textbf{\textit{anem}} (\textsc{n}) yam species (yam with purple flesh); (\textsc{adj}) blue, purple \item 
\textbf{\textit{anembal}} (\textsc{adj}) light (color) (possibly a \isi{compound} containing \textbf{\textit{ane}} ‘sun’) \item 
\textbf{\textit{anen}} (\textsc{n}) fat, grease \item 
\textbf{\textit{anenisi}} (\textsc{n}) torch \item 
\textbf{\textit{anga}} (\textsc{n}) piece, side \item 
\textbf{\textit{angani}} (\textsc{p}) behind, after; (\textsc{n}) rear, back \item 
\textbf{\textit{angani ka-}} (\textsc{v}) follow (literally ‘let behind’) \item 
\textbf{\textit{anganika}} (\textsc{adv}) after, afterwards, later, soon \item 
\textbf{\textit{angay}} (\textsc{num}) five (literally ‘side [of] hand’) \item 
\textbf{\textit{angay angay}} (\textsc{num}) twenty-five (= 5${\cdot}$5) \item 
\textbf{\textit{angay kwe kwe mowon ndïwatlïp}} (\textsc{num}) six (literally ‘one side of hand; cut one and put atop them’) \item 
\textbf{\textit{angay kwe lele ndïwon ndïwatlïp}} (\textsc{num}) eight (literally ‘one side of hand; cut three and put atop them’) \item 
\textbf{\textit{angay kwe nini minwon ndïwatlïp}} (\textsc{num}) seven (literally ‘one side of hand; cut two and put atop them’) \item 
\textbf{\textit{angay kwe watangïnila ndïwon ndïwatlïp}} (\textsc{num}) nine (literally ‘one side of hand; cut four and put atop them’) \item 
\textbf{\textit{angay lele}} (\textsc{num}) fifteen (= 5${\cdot}$3) \item 
\textbf{\textit{angay lele kwe mowon ndïwatlïp}} (\textsc{num}) sixteen (literally ‘three sides of hands; cut one and put atop them’) \item 
\textbf{\textit{angay lele lele ndïwon ndïwatlïp}} (\textsc{num}) eighteen (literally ‘three sides of hands; cut three and put atop them’) \item 
\textbf{\textit{angay lele nini minwon ndïwatlïp}} (\textsc{num}) seventeen (literally ‘three sides of hands; cut two and put atop them’) \item 
\textbf{\textit{angay lele watangïnila ndïwon ndïwatlïp}} (\textsc{num}) nineteen (literally ‘three sides of hands; cut four and put atop them’) \item 
\textbf{\textit{angay nini}} (\textsc{num}) ten (= 5${\cdot}$2) \item 
\textbf{\textit{angay nini kwe mowonndïwatlïp}} (\textsc{num}) eleven (literally ‘two sides of hands; cut one and put atop them’) \item 
\textbf{\textit{angay nini lele ndïwon ndïwatlïp}} (\textsc{num}) thirteen (literally ‘two sides of hands; cut three and put atop them’) \item 
\textbf{\textit{angay nini nini minwon ndïwatlïp}} (\textsc{num}) twelve (literally ‘two sides of hands; cut two and put atop them’) \item 
\textbf{\textit{angay nini watangïnila ndïwon ndïwatlïp}} (\textsc{num}) fourteen (literally ‘two \linebreak sides of hands; cut four and put atop them’) \item 
\textbf{\textit{angay watangïnila}} (\textsc{num}) twenty (= 5${\cdot}$4) \item 
\textbf{\textit{angïn}} (\textsc{n}) vine species \item 
\textbf{\textit{angla}} (\textsc{p}) waiting for, awaiting (also \textbf{\textit{andïla}}) \item 
\textbf{\textit{anglalo-}} (\textsc{v}) hunt, seek (literally ‘go awaiting’) (also \textbf{\textit{andïlalo-}}) \item 
\textbf{\textit{ango}} (\textsc{neg}) no, not \item 
\textbf{\textit{ango}} (\textsc{q}) which?; where? \item 
\textbf{\textit{ango luwa}} (\textsc{q}) where? (literally ‘which place?’) \item 
\textbf{\textit{ango tem}} (\textsc{q}) when? (literally ‘which time?’) \item 
\textbf{\textit{ango-}} (\textsc{v}) pull out, pick \item 
\textbf{\textit{angom lï-}} (\textsc{v}) pull, pull out, uproot (literally ‘put a pull’?) \item 
\textbf{\textit{angos}} (\textsc{q}) what?; (\textsc{p}) whatever, whatsoever, anything \item 
\textbf{\textit{angos nji}} (\textsc{p}) whatever (literally ‘whatever thing’) \item 
\textbf{\textit{angumoni}} (\textsc{adj}) swelling (waves) \item 
\textbf{\textit{angumoni nïmal}} (\textsc{n}) ocean, sea (literally ‘swelling river’) \item 
\textbf{\textit{angun}} (\textsc{n}) tail; fin, fishtail \item 
\textbf{\textit{angwena}} (\textsc{q}) why? (< \textbf{\textit{ango}} ‘which?’ + \textbf{\textit{na}} ‘reason, cause’) \item 
\textbf{\textit{ani}} (\textsc{n}) string bag, net bag (TP \textit{bilum}) \item 
\textbf{\textit{anïm}} (\textsc{n}) forking stick \item 
\textbf{\textit{anïmasi}} (\textsc{n}) snake species (python) (TP \textit{moran}) \item 
\textbf{\textit{anïmbu}} (\textsc{n}) mango \item 
\textbf{\textit{aninokam}} (\textsc{n}) throat, windpipe \item 
\textbf{\textit{anji}} (\textsc{pro}) our, ours (\textsc{1pl.excl} \isi{possessive pronoun}, ‘\textsc{1pl.excl-poss}’) \item 
\textbf{\textit{anjika}} (\textsc{q}) how many? \item 
\textbf{\textit{anjikaka}} (\textsc{q}) how?; what’s the matter? (possibly < \textbf{\textit{anjika}} ‘how many?’ + \textbf{\textit{ka}} ‘thus’) \item 
\textbf{\textit{ankam}} (\textsc{n}) person, human \item 
\textbf{\textit{ankam unduwan}} (\textsc{num}) fifty (literally ‘person head’) \item 
\textbf{\textit{ankam unduwan nali}} (\textsc{num}) sixty (50+10) \item 
\textbf{\textit{ankam unduwan nali lele}} (\textsc{num}) eighty (50+30) \item 
\textbf{\textit{ankam unduwan nali nini}} (\textsc{num}) seventy (50+20) \item 
\textbf{\textit{ankam unduwan nali watangïnila}} (\textsc{num}) ninety (50+40) \item 
\textbf{\textit{ankïn}} (\textsc{n}) vegetable species (TP \textit{kumu mosong}) \item 
\textbf{\textit{anma}} (\textsc{adj}) good, nice, true, smart, intelligent, straight, healthy, well \item 
\textbf{\textit{anma wanani-}} (\textsc{v}) be happy (literally ‘feel-act good’) \item 
\textbf{\textit{anma-}} (\textsc{v}) go out (literally ‘go out’) \item 
\textbf{\textit{anmbasa-}} (\textsc{v}) chase (literally ‘hit outside’) \item 
\textbf{\textit{anmbï}} (\textsc{adv}) outside (literally ‘out here’) \item 
\textbf{\textit{anmbi-}} (\textsc{v}) come out (literally ‘go outside’) \item 
\textbf{\textit{anmoka}} (\textsc{n}) snake \item 
\textbf{\textit{anmopa}} (\textsc{n}) vegetable species (\textit{Gnetum gnemon}) (TP \textit{tulip}) \item 
\textbf{\textit{anmot}} (\textsc{n}) post used in the middle of a house to support the roof (literally ‘out awning’) \item 
\textbf{\textit{ansi}} (\textsc{n}) mix of betel nut, betel pepper, and lime; chewed-up betel nut (TP \textit{red buai}); lime gourd (a gourd-like plant used to store lime), previously used to cover the penis; penis (slang) \item 
\textbf{\textit{ansi inom}} (\textsc{n}) mother’s brother’s wife (aunt) (literally ‘lime gourd mother’) \item 
\textbf{\textit{ansi nungol}} (\textsc{n}) sister’s child (sororal nibling), nephew or niece (only used to refer to a man’s sister’s child, i.e., the reciprocal relation of the \textbf{\textit{yawa}}) (literally ‘lime gourd child’) \item 
\textbf{\textit{ansi yanat}} (\textsc{n}) niece (only used to refer to a man’s sister’s daughter) (literally ‘lime gourd daughter’) \item 
\textbf{\textit{ansimu}} (\textsc{n}) type of drum (gourd-like drum) (literally ‘lime gourd fruit’) \item 
\textbf{\textit{anul}} (\textsc{n}) grass, grassland (also \textbf{\textit{nipum}} \textbf{\textit{amba}}) \item 
\textbf{\textit{anwe}} (\textsc{pro}) we ourselves, us ourselves (from among several) (\textsc{1pl.excl} \linebreak \isi{partitive-intensive pronoun}, ‘\textsc{1pl.excl-part.int}’) \item 
\textbf{\textit{-ap}} [\isi{perfective} \isi{suffix}, ‘\textsc{pfv}’] (in \isi{double perfective} constructions; also \textbf{\textit{-ïp}}, \textbf{\textit{-op}}) \item 
\textbf{\textit{apa}} (\textsc{n}) house, building \item 
\textbf{\textit{apa ini}} (\textsc{n}) floor of a house (literally ‘house ground’) \item 
\textbf{\textit{apa nambï}} (\textsc{n}) wall of a house (literally ‘house skin’) \item 
\textbf{\textit{apaka}} (\textsc{n}) roof of a house (literally ‘house peak’) \item 
\textbf{\textit{apembam}} (\textsc{n}) area beneath a stilted house (literally ‘house under’) \item 
\textbf{\textit{apep}} (\textsc{n}) front of the house (< \textbf{\textit{apa}} ‘house’ + \textbf{\textit{ip}} ‘nose’) \item 
\textbf{\textit{apïn}} (\textsc{n}) fire, matches, lighter; pain \item 
\textbf{\textit{apïn ama-}} (\textsc{v}) burn (\isi{transitive}) (literally ‘eat [with] fire’) \item 
\textbf{\textit{apïn inim}} (\textsc{n}) perspiration, sweat (literally ‘fire water’) \item 
\textbf{\textit{apïn mïnda}} (\textsc{n}) banana species (banana plant with sweet, small, red fruit, \linebreak traditionally eaten only by men) (literally ‘fire banana’) \item 
\textbf{\textit{apïn nangïn}} (\textsc{n}) large fire tongs (literally ‘fire tongs’) \item 
\textbf{\textit{apïn ngïn}} (\textsc{n}) smoke (literally ‘fire cloud’) \item 
\textbf{\textit{apïn we}} (\textsc{n}) sago cooked on the fire (literally ‘fire sago’) \item 
\textbf{\textit{apïnal}} (\textsc{n}) swamp (also \textbf{\textit{mïka itïm}}) \item 
\textbf{\textit{apïnsi}} (\textsc{n}) ash, ashes (< \textbf{\textit{apïn}} ‘fire’ + \textbf{\textit{isi}} ‘ashes, salt’) \item 
\textbf{\textit{apka}} (\textsc{adv}) very, really \item 
\textbf{\textit{aplatam}} (\textsc{n}) table, shelf \item 
\textbf{\textit{apombam}} (\textsc{n}) middle of the house (< \textbf{\textit{apa}} ‘house’ + \textbf{\textit{wombam}} ‘middle’) \item 
\textbf{\textit{apot}} (\textsc{n}) shelf that hangs above the hearth, used for drying and smoking meat and fish \item 
\textbf{\textit{apunmbï}} (\textsc{n}) back of the house (< \textbf{\textit{apa}} ‘house’ + \textbf{\textit{unmbï}} ‘buttocks’) \item 
\textbf{\textit{apwanam}} (\textsc{n}) side of the house (< \textbf{\textit{apa}} ‘house’ + \textbf{\textit{wanam}} ‘side’) \item 
\textbf{\textit{apwane}} (\textsc{n}) insect species (the adult form of the \textbf{\textit{mïnkïn}} grub species) \item 
\textbf{\textit{as}} (\textsc{v}) hit, stab, shoot; kill (abbreviated form of \textbf{\textit{asa-}}) \item 
\textbf{\textit{asa}} (\textsc{interj}) nah, no (expresses denial) \item 
\textbf{\textit{asa-}} (\textsc{v}) hit, stab, shoot; kill \item 
\textbf{\textit{ase}} (\textsc{interj}) no (expresses denial) \item 
\textbf{\textit{asi}} (\textsc{n}) grass \item 
\textbf{\textit{asi ka-}} (\textsc{v}) sit, sit down (literally ‘let sit’?) \item 
\textbf{\textit{asïmïna}} (\textsc{n}) nosering traditionally worn by men; sneeze \item 
\textbf{\textit{asimu}} (\textsc{n}) rice (literally ‘rice seed’) \item 
\textbf{\textit{Asingona}} [female name] \item 
\textbf{\textit{asiya}} (\textsc{n}) string, thread; animal trap made of string; fishing line \item 
\textbf{\textit{asiyot}} (\textsc{n}) grass knife (literally ‘grass machete’) \item 
\textbf{\textit{at}} (\textsc{n}) end, piece \item 
\textbf{\textit{at}} (\textsc{n}) fight, battle \item 
\textbf{\textit{ata}} (\textsc{adv}) up, upper, upward, upstream; (\textsc{adj)} high \item 
\textbf{\textit{ata monam mu}} (\textsc{n}) money (literally ‘high rain tree fruit’) \item 
\textbf{\textit{ata tanum}} (\textsc{n}) upper lip, area above the mouth (literally ‘upper lip’) \item 
\textbf{\textit{atal}} (\textsc{n}) laughter; anus \item 
\textbf{\textit{atala-}} (\textsc{v}) laugh (literally ‘break a laugh’?) \item 
\textbf{\textit{atalï-}} (\textsc{v}) put up (literally ‘put up’) \item 
\textbf{\textit{atana}} (\textsc{n}) older sister (probably < \textbf{\textit{ata}} ‘upper’ + \textbf{\textit{yana}} ‘woman’) \item 
\textbf{\textit{atana numan}} (\textsc{n}) older sister’s husband (brother-in-law) (literally ‘older sister husband’) \item 
\textbf{\textit{atate}} (\textsc{n}) Singapore taro (members of the genus \textit{Xanthosoma}) (TP \textit{kongkong}) \item 
\textbf{\textit{atay}} (\textsc{v}) go up (< \textbf{\textit{ata}} ‘up’ + \textbf{\textit{i}} ‘go.\textsc{pfv}’) \item 
\textbf{\textit{atï-}} (\textsc{v}) hit, stab, shoot; kill (irregular \isi{irrealis} \isi{stem}) \item 
\textbf{\textit{atuma}} (\textsc{n}) older brother (possibly < \textbf{\textit{ata}} ‘upper’ + \textbf{\textit{uma}} ‘bone’) \item 
\textbf{\textit{Atuma}} [female name] \item 
\textbf{\textit{atuma inga yena}} (\textsc{n}) older brother’s wife (sister-in-law) (literally ‘older brother affine woman’) \item 
\textbf{\textit{atwana}} (\textsc{n}) question \item 
\textbf{\textit{atwana kï-}} (\textsc{v}) ask (literally ‘say a question’) (also \textit{\textbf{atwana ta-}}) \item 
\textbf{\textit{atwana ta-}} (\textsc{v}) ask (literally ‘say a question’) (also \textit{\textbf{atwana kï-}}) \item 
\textbf{\textit{aw}} (\textsc{n}) betel nut (\textit{Areca catechu}) palm or fruit (TP \textit{buai}) \item 
\textbf{\textit{aw ilowan}} (\textsc{n}) young betel nut palm, just grown from a shoot (\textbf{\textit{aw}} is ‘betel nut’; meaning of \textit{ilowan} is unknown) \item 
\textbf{\textit{aw imbïn}} (\textsc{n}) betel nut spittle (literally ‘betel nut refuse’) \item 
\textbf{\textit{aw lïmndï}} (\textsc{n}) youngest (immature) stage of betel nut fruit (literally ‘eye betel nut’) \item 
\textbf{\textit{aw ulum}} (\textsc{n}) young, somewhat wet betel nut fruit (the stage following \textbf{\textit{kakïla}} ‘young betel nut’) (literally ‘sago palm nut’) \item 
\textbf{\textit{aw wapata}} (\textsc{n}) mature, dry betel nut fruit (the stage following \textbf{\textit{pïsima}} ‘older betel nut’) (literally ‘dry betel nut’) \item 
\textbf{\textit{aw-}} (\textsc{v}) put (\isi{imperfective} \isi{stem} of \textbf{\textit{u-}}) \item 
\textbf{\textit{awa}} [\isi{intensive} marker, ‘\textsc{int}’) \item 
\textbf{\textit{-awa}} [\isi{intensive} \isi{suffix}, ‘\textsc{int}’) \item 
\textbf{\textit{Awaka}} [male name] \item 
\textbf{\textit{awal}} (\textsc{n}) afternoon, evening; (\textsc{adv}) yesterday \item 
\textbf{\textit{awal anma}} (\isi{greeting}) good afternoon \item 
\textbf{\textit{awal nambï}} (\textsc{n}) afternoon (literally ‘afternoon body’) \item 
\textbf{\textit{awal nambï anma}} (\isi{greeting}) good afternoon \item 
\textbf{\textit{awalawa}} (\textsc{n}) bird species (red or green parrot) (TP \textit{kalangal}) \item 
\textbf{\textit{awame}} (\textsc{n}) seed species (rice-like seed of a palm species, commonly eaten by children) \item 
\textbf{\textit{Awandana}} [female name] \item 
\textbf{\textit{awaw}} (\textsc{n}) lie, falsehood  \item 
\textbf{\textit{awe}} (\textsc{n}) tree species (ilima tree) (\textit{Octomeles sumatrana}) (TP \textit{erima}) \item 
\textbf{\textit{awena}} (\textsc{n}) female friend (of a woman) \item 
\textbf{\textit{aweta}} (\textsc{n}) (male) friend \item 
\textbf{\textit{awi}} (\textsc{n}) shoulder; the side of something \item 
\textbf{\textit{awïl}} (\textsc{n}) yam species (white, thin, very long yam) \item 
\textbf{\textit{awindal}} (\textsc{n}) reeds (TP \textit{tiktik}) \item 
\textbf{\textit{awlop}} (\textsc{adv}) in vain \item 
\textbf{\textit{awlu}} (\textsc{n}) step \item 
\textbf{\textit{awnaka}} (\textsc{n}) tree species \item 
\textbf{\textit{awngala}} (\textsc{n}) bird species (small black, yellow-breasted bird) \item 
\textbf{\textit{awpane}} (\textsc{n}) butterfly \item 
\textbf{\textit{awsingïn}} (\textsc{n}) bird species (eagle, hawk) (TP \textit{tarangau}) \item 
\textbf{\textit{ay}} (\textsc{interj}) ow, ay (expresses pain or shock) \item 
\textbf{\textit{ay}} (\textsc{n}) sago, jellied sago \item 
\textbf{\textit{aya}} (\textsc{interj}) ah me (expresses compassion) \item 
\textbf{\textit{aylat}} (\textsc{n}) insect species (millipede) \item 
\textbf{\textit{aymoma}} (\textsc{n}) sago stick (stick used to stir sago) (\textit{\textbf{ay}} is ‘sago’; relationship, if any, to \textbf{\textit{moma}} ‘leaf tied in a knot’ is unknown) \item 
\textbf{\textit{ayna}} (\textsc{n}) scarf worn by women in mourning; string bag used for carrying babies \item 
\textbf{\textit{Ayndin}} [male name] \item 
\textbf{\textit{aypul}} (\textsc{n}) scoop of jellied sago (literally ‘sago piece’)\\ \item

\noindent \textbf{<B>        [ᵐb]}\\ \item

\textbf{\textit{Banjiwa}} [male name] \item 
\textbf{\textit{Bay}} [male name] \item 
\textbf{\textit{Bulon}} [place] region immediately surrounding the fifth (and current) Manu \linebreak village\\ \item

\noindent \textbf{<D>        [ⁿd]}\\ \item

\textbf{\textit{Damnda}} [female name] \item 
\textbf{\textit{Dim}} [place] Biwat village; name of the original Manu village \item 
\textbf{\textit{Dimes}} [male name] \item 
\textbf{\textit{Dingo}} [male name] \item 
\textbf{\textit{Dumngul}} [male name]\\ \item

\noindent \textbf{<E, e>        [e]}\\ \item

\textbf{\textit{e}} [free \isi{dependent marker}, ‘\textsc{dep}’] \item 
\textbf{\textit{e}} (\textsc{interj}) hey, ay (expresses excitement, either positive or negative); eh? (\isi{tag question} \isi{interjection}) \item 
\textbf{\textit{-e}} [\isi{dependent marker} \isi{suffix}, ‘\textsc{dep}’] \item 
\textbf{\textit{-e}} [\isi{imperfective} \isi{suffix}, ‘\textsc{ipfv}’] \item 
\textbf{\textit{eklak}} (\textsc{n}) tree species (Malay apple) (\textit{Syzygium malaccense}) (TP \textit{laulau}) (\isi{loan} of unknown origin) \item 
\textbf{\textit{-en}} [\isi{nominalizing} \isi{suffix}, ‘\textsc{nmlz}’]\\ \item

\newpage

\noindent \textbf{<G>        [ᵑɡ]}\\ \item

\textbf{\textit{Gambri}} [male name] \item 
\textbf{\textit{Gami}} [female name] \item 
\textbf{\textit{Ganmali}} [male name] \item 
\textbf{\textit{Ginam}} [female name] \item 
\textbf{\textit{Guren}} [male name] \item 
\textbf{\textit{Gwam}} [female name]\\ \item

\noindent \textbf{<I, i>        [i]}\\ \item

\textbf{\textit{i}} [\isi{predicate marker}, ‘\textsc{pred}’] (< \ili{Tok Pisin} \textit{i}, \isi{predicate marker}) \item 
\textbf{\textit{i}} (\textsc{n}) behavior, habit, custom, way \item 
\textbf{\textit{i}} (\textsc{n}) hand, arm \item 
\textbf{\textit{i}} (\textsc{n}) lime (calcium hydroxide) (TP \textit{kambang}) (< \ili{Ap Ma}; probably ultimately \linebreak < \ili{Austronesian}) \item 
\textbf{\textit{i}} (\textsc{v}) go, flow (\isi{suppletive} \isi{perfective} form of \textbf{\textit{ma-}}) \item 
\textbf{\textit{i}} (\textsc{interj}) alas; yay (expresses dejection or joy) \item 
\textbf{\textit{i-}} (\textsc{v}) come \item 
\textbf{\textit{i ambatïm}} (\textsc{n}) elbow (literally ‘arm joint’) \item 
\textbf{\textit{i mutam}} (\textsc{n}) back of the hand (literally ‘hand back’) \item 
\textbf{\textit{i mwa}} (\textsc{n}) palm of the hand (literally ‘hand opening’) (also \textbf{\textit{yombam}}) \item 
\textbf{\textit{i name}} (\textsc{n}) upper arm (cf. \textbf{\textit{wutï name}} ‘thigh’, \textbf{\textit{lam}} ‘muscle’) \item 
\textbf{\textit{i nangum}} (\textsc{n}) forearm (literally ‘arm shoot’) \item 
\textbf{\textit{ika}} (\textsc{n}) instance, time (literally ‘way-thus’?) \item 
\textbf{\textit{ika}} (\textsc{n}) riverbank (< \ili{Ap Ma}) \item 
\textbf{\textit{ika uta-}} (\textsc{v}) count (literally ‘rub instances’?) \item 
\textbf{\textit{ikali lï-}} (\textsc{v}) grab, hold, catch (literally ‘send hand’) \item 
\textbf{\textit{iken}} (\isi{modal} marker) may, can (< \ili{Tok Pisin} \textit{i ken}, \isi{predicate marker} + ‘may’) \item 
\textbf{\textit{ila}} (\textsc{n}) sago palm frond, thatch (TP \textit{morota}) \item 
\textbf{\textit{ilom}} (\textsc{n}) day \item 
\textbf{\textit{ilu}} (\textsc{n}) root \item 
\textbf{\textit{ilum}} (\textsc{n}) piece; (\textsc{quant}) little, few \item 
\textbf{\textit{ilumka}} (\textsc{adv}) a little \item 
\textbf{\textit{im}} (\textsc{n}) tree \item 
\textbf{\textit{im kal}} (\textsc{n}) sap (\textbf{\textit{im}} is ‘tree’; meaning of \textit{kal} is unknown) \item 
\textbf{\textit{im} \textbf{nali}} (\textsc{n}) stick (literally ‘tree frond spine’) \item 
\textbf{\textit{im nambï}} (\textsc{n}) bark (literally ‘tree skin’) \item 
\textbf{\textit{im nangïn}} (\textsc{n}) branch (literally ‘tree tongs’) \item 
\textbf{\textit{imba}} (\textsc{n}) night, evening \item 
\textbf{\textit{imba anma}} (\isi{greeting}) good evening, good night \item 
\textbf{\textit{imbam}} (\textsc{p}) under, below \item 
\textbf{\textit{imbam ka-}} (\textsc{v}) run (literally ‘let under’) \item 
\textbf{\textit{imbïn}} (\textsc{n}) refuse water when washing sago pith \item 
\textbf{\textit{imnde}} (\textsc{n}) type of basket (basket used for straining sago) \item 
\textbf{\textit{imot}} (\textsc{n}) log, firewood (possibly < \textbf{\textit{im}} ‘tree’ + \textbf{\textit{wat}} ‘top’) \item 
\textbf{\textit{impul}} (\textsc{n}) piece of wood (literally ‘tree piece’) \item 
\textbf{\textit{imu}} (\textsc{n}) finger, digit (literally ‘hand fruit’) \item 
\textbf{\textit{imu ankam}} (\textsc{n}) index finger (literally ‘person finger’) \item 
\textbf{\textit{imu law}} (\textsc{n}) ring finger (literally ‘cordyline finger’) \item 
\textbf{\textit{imu unduwan}} (\textsc{n}) thumb (literally ‘head finger’) \item 
\textbf{\textit{imu watangïn}} (\textsc{n}) pinky finger, little finger (literally ‘last finger’) \item 
\textbf{\textit{imu wome}} (\textsc{n}) middle finger (literally ‘middle finger’) \item 
\textbf{\textit{Imwa}} [place] region surrounding \textbf{\textit{Wopata}} village \item 
\textbf{\textit{in}} (\textsc{p}) in, inside, into, within \item 
\textbf{\textit{in-}} (\textsc{v}) get, collect (\isi{irrealis} \isi{stem} of \textbf{\textit{ina-}}) \item 
\textbf{\textit{ina}} (\textsc{n}) liver; the seat of reasoning and emotion \item 
\textbf{\textit{ina-}} (\textsc{v}) get, collect \item 
\textbf{\textit{inakawana-}} (\textsc{v}) think (literally ‘feel in the liver’) \item 
\textbf{\textit{inamba}} (\textsc{n}) armband; money \item 
\textbf{\textit{inane}} (\textsc{n}) grub species (mature edible grub, either of the \textbf{\textit{mïnkïn}} or \textbf{\textit{mundum}} grub species) \item 
\textbf{\textit{inangïnmana}} (\textsc{n}) official, civil servant (literally ‘going claw hand’?) \item 
\textbf{\textit{inapaw}} (\textsc{n}) belly, waist \item 
\textbf{\textit{inapum}} (\textsc{n}) right, right-hand side \item 
\textbf{\textit{inda-}} (\textsc{v}) walk \item 
\textbf{\textit{inga}} (\textsc{n}) affine, in-law (TP \textit{tambu}) \item 
\textbf{\textit{inga yena}} (\textsc{n}) brother’s wife (sister-in-law) (literally ‘affine woman’) \item 
\textbf{\textit{ingwa}} (\textsc{n}) spider \item 
\textbf{\textit{ini}} (\textsc{n}) ground, land, earth, soil \item 
\textbf{\textit{inim}} (\textsc{n}) water, liquid, rain; rainy season, wet season; year \item 
\textbf{\textit{inim ambi}} (\textsc{n}) flood (literally ‘big water’) \item 
\textbf{\textit{inim mo ma-}} (\textsc{v}) swim (literally ‘go on the water’) \item 
\textbf{\textit{inim nïkï-}} (\textsc{v}) celebrate (literally ‘dig water’) \item 
\textbf{\textit{inim tembi}} (\textsc{n}) alcohol (literally ‘bad water’) \item 
\textbf{\textit{inimndum}} (\textsc{n}) sago species (small sago palm with short spines) \item 
\textbf{\textit{inimnji}} (\textsc{n}) type of spirit (water spirit); dew (literally ‘water thing’) \item 
\textbf{\textit{inimpul}} (\textsc{n}) lake, pond (literally ‘water piece’) \item 
\textbf{\textit{inji}} (\textsc{n}) innards, insides, guts (literally ‘inside things’) \item 
\textbf{\textit{inkaw}} (\textsc{n}) mountain \item 
\textbf{\textit{inmbï}} (\textsc{n}) vulva, vagina \item 
\textbf{\textit{inmbï mïnïm}} (\textsc{n}) clitoris (literally ‘vulva tongue’) \item 
\textbf{\textit{inmi}} (\textsc{n}) hole \item 
\textbf{\textit{inom}} (\textsc{n}) mother; term of respect for older women; general term for aunts; any adult woman \item 
\textbf{\textit{inom atana}} (\textsc{n}) parent’s older sister (aunt) (literally ‘older sister mother’) \item 
\textbf{\textit{inom ngata}} (\textsc{n}) grandmother, old woman (literally ‘grand mother’) \item 
\textbf{\textit{inom wot}} (\textsc{n}) parent’s younger sister (aunt) (literally ‘younger mother’) \item 
\textbf{\textit{inpu}} (\textsc{n}) elbow (cf. \textbf{\textit{wutïnpu}} ‘heel’, \textbf{\textit{akunpu}} ‘back of the skull’) \item 
\textbf{\textit{intïp}} (\textsc{n}) cassowary bone (often sharpened to be used as a tool or weapon) \item 
\textbf{\textit{inu-}} (\textsc{v}) put in, put into (literally ‘put in’) \item 
\textbf{\textit{inum}} (\textsc{n}) ground, burial spot (cf. \textbf{\textit{ini}} ‘ground’) \item 
\textbf{\textit{ip}} (\textsc{n}) nose, front \item 
\textbf{\textit{ip ka-}} (\textsc{v}) precede (literally ‘let nose’ or ‘let front’) \item 
\textbf{\textit{ip nonal}} (\textsc{n}) snore (literally ‘nose breath’) \item 
\textbf{\textit{ipka}} (\textsc{p}) before, in front of; (\textsc{adv}) beforehand, earlier, first \item 
\textbf{\textit{ipwat}} (\textsc{n}) front (literally ‘nose top’) \item 
\textbf{\textit{isi}} (\textsc{n}) ash, ashes (usually only as part of the \isi{compound} \textbf{\textit{apïnsi}} ‘ashes’); salt (traditional salt made from the ashes of burnt banana leaves); broth, soup \item 
\textbf{\textit{isi}} (\textsc{n}) young palm frond used for weaving (a younger form of \textbf{\textit{wema}} ‘palm frond’) (TP \textit{pangal}); fuzz (as found on some plants) \item 
\textbf{\textit{isi monombam u-}} (\textsc{v}) pray (literally ‘push hand on forehead’) \item 
\textbf{\textit{ita-}} (\textsc{v}) build, make; tie \item 
\textbf{\textit{itenmbu}} (\textsc{n}) bamboo species; bamboo container, cup \item 
\textbf{\textit{itïm}} (\textsc{n}) trash, rubbish, garbage \item 
\textbf{\textit{itïtïl}} (\textsc{n}) dust \item 
\textbf{\textit{itom}} (\textsc{n}) father; term of respect for older men; general term for uncles (usually only paternal uncles); any adult man \item 
\textbf{\textit{itom ambi}} (\textsc{n}) father’s older brother (paternal uncle) (literally ‘big father’) \item 
\textbf{\textit{itom atuma}} (\textsc{n}) father’s older brother (paternal uncle) (literally ‘older brother father’) \item 
\textbf{\textit{itom ngata}} (\textsc{n}) grandfather, old man (literally ‘grand father’) \item 
\textbf{\textit{itom wot}} (\textsc{n}) father’s younger brother (paternal uncle) (literally ‘younger \linebreak father’) \item 
\textbf{\textit{iwa}} (\textsc{n}) type of basket (vase-shape basket woven from sago fronds, used to catch fish); fish trap \item 
\textbf{\textit{iwal}} (\textsc{n}) type of beam (horizontal beam in a house) \item 
\textbf{\textit{iwanal}} (\textsc{n}) insect species (small red or brown ant) \item 
\textbf{\textit{iwïl}} (\textsc{n}) moon; month; menstruation; vulva (euphemism) \item 
\textbf{\textit{iya}} (\textsc{p}) to, toward \item 
\textbf{\textit{iya}} (\textsc{interj}) yeah, yes (expresses affirmation) \item 
\textbf{\textit{iyo}} (\textsc{interj}) yes (expresses affirmation)\\ \item 

\noindent \textbf{<ï>        [ɨ]}\\ \item

\textbf{\textit{=ïn}} [\isi{oblique marker}, ‘\textsc{obl}’] (\isi{allomorph} of \textbf{\textit{=n}}) \item 
\textbf{\textit{-ïp}} [\isi{perfective} \isi{suffix}, ‘\textsc{pfv}’] (in \isi{double perfective} constructions; also \textbf{\textit{-ap}}, \textbf{\textit{-op}})\\ \item 

\noindent \textbf{<J>        [ⁿdʒ]}\\ \item 

\textbf{\textit{Jukan}} [female name]\\ \item 

\noindent \textbf{<K, k>      [k]}\\ \item 

\textbf{\textit{ka}} (\textsc{n}) peak \item 
\textbf{\textit{ka}} (\textsc{p}) at, in, on \item 
\textbf{\textit{ka}} (\textsc{adv}) thus, in this way, in that way; (filler word) like (TP \textit{olsem}) (also \textbf{\textit{mïka}}, \textbf{\textit{maka}}) \item 
\textbf{\textit{ka-}} (\textsc{v}) let, leave (behind), allow \item 
\textbf{\textit{ka-}} (\textsc{v}) say (\isi{perfective} \isi{stem} of \textbf{\textit{kï-}}) \item 
\textbf{\textit{kaka}} (\textsc{adv}) completely (also \textbf{\textit{keka}}) \item 
\textbf{\textit{kakïla}} (\textsc{n}) young, wet betel nut fruit (the stage following \textbf{\textit{aw lïmndï}} ‘youngest betel nut’) \item 
\textbf{\textit{kalam}} (\textsc{n}) knowledge, wisdom; (\textsc{adj}) knowledgeable, knowing, wise (< \ili{Waran}) \item 
\textbf{\textit{kalamp}} (\textsc{v}) know (literally ‘be knowledgeable’) \item 
\textbf{\textit{kali lï-}} (\textsc{v}) send (literally ‘put a send’?) \item 
\textbf{\textit{kalim}} (\textsc{n}) cassowary (TP \textit{muruk}) (< \ili{Yuat}) \item 
\textbf{\textit{kalingana}} (\textsc{n}) insect species (mantis) \item 
\textbf{\textit{Kalingana}} [male name] \item 
\textbf{\textit{kalum}} (\textsc{n}) egg yolk \item 
\textbf{\textit{kamb-}} (\textsc{v}) shun, avoid \item 
\textbf{\textit{Kambok}} [place] Kambuku village \item 
\textbf{\textit{Kamen}} [place] ancestral village of the Ulwa and neighboring language \linebreak communities, near present-day Kambaramba village \item 
\textbf{\textit{kana}} (\textsc{p}) beside, near, next to (also \textbf{\textit{kanam}}) \item 
\textbf{\textit{kanaka lumo-}} (\textsc{v}) unwrap (literally ‘put an unwrapping’?) \item 
\textbf{\textit{kanam}} (\textsc{p}) beside, near, next to (also \textbf{\textit{kana}}) \item 
\textbf{\textit{Kanang}} [male name] \item 
\textbf{\textit{Kanangula}} [male name] \item 
\textbf{\textit{Kanangwa}} [place] alternative name of \textbf{\textit{Amali}} village \item 
\textbf{\textit{kananum}} (\textsc{n}) boil, blister, abscess \item 
\textbf{\textit{Kapos}} [male name] \item 
\textbf{\textit{kat ambla}} (\textsc{n}) molar (\textbf{\textit{ambla}} is ‘tooth’; meaning of \textit{kat} is unknown) \item 
\textbf{\textit{katmombe}} (\textsc{n}) insect species (black stinging ant) (< \ili{Mwakai}) \item 
\textbf{\textit{kaw}} (\textsc{n}) song, song and dance (TP \textit{singsing}) (< \ili{Ap Ma}) \item 
\textbf{\textit{kawa}} (\textsc{n}) nut species (small green nut that is chewed) (possibly < \ili{Tok Pisin} \textit{kawiwi} ‘wild betel nut’) \item 
\textbf{\textit{Kawana}} [female name] \item 
\textbf{\textit{Kawat}} [male name] \item 
\textbf{\textit{kawni-}} (\textsc{v}) sing (literally ‘do song’) \item 
\textbf{\textit{kayanmali}} (\textsc{n}) lizard species (lizard with a horn on the back of its head) \item 
\textbf{\textit{Kayngam}} [male name] \item 
\textbf{\textit{Kayta}} [male name] \item 
\textbf{\textit{keka}} (\textsc{adv}) completely (also \textbf{\textit{kaka}}) \item 
\textbf{\textit{kekaka}} (\textsc{quant}) one each, one by one, just a few (also \textbf{\textit{kwekaka}}) \item 
\textbf{\textit{kenmbu}} (\textsc{adj}) heavy; (\textsc{n}) problem \item 
\textbf{\textit{kï-}} (\textsc{v}) say, speak, tell, talk, think \item 
\textbf{\textit{kïka}} (\textsc{n}) insect species (white ant, termite); white ant nest \item 
\textbf{\textit{kïkal}} (\textsc{n}) ear \item 
\textbf{\textit{kïkal indam}} (\textsc{n}) temple (of the head) (\textbf{\textit{kïkal}} is ‘ear’; meaning of \textit{indam} is \linebreak unknown) \item 
\textbf{\textit{kïkal wana-}} (\textsc{v}) hear, listen (literally ‘feel [by means of] ear’) \item 
\textbf{\textit{kïkal wopa}} (\textsc{adj}) deaf (literally ‘whole ear’) \item 
\textbf{\textit{kïkalsina}} (\textsc{adj}) sharp \item 
\textbf{\textit{kïke u-}} (\textsc{v}) throw (literally ‘put a throw’?) \item 
\textbf{\textit{kïlakïli}} (\textsc{n}) frog species (very small frog that lives on leaves) \item 
\textbf{\textit{Kïtalwe}} [male name] \item 
\textbf{\textit{kïtïmngïle}} (\textsc{n}) banana species (banana plant with very large fruit, second in size only to the \textbf{\textit{wowi}} banana species) \item 
\textbf{\textit{klop-}} (\textsc{v}) cross, pass \item 
\textbf{\textit{ko}} (\textsc{adv}) just, simply, without care, without reason (also \textbf{\textit{kwa}}, \textbf{\textit{wa}}) \item 
\textbf{\textit{ko=}} [\isi{indefinite} marker, ‘\textsc{indf}’] \item 
\textbf{\textit{kokal}} (\textsc{n}) casque (of a cassowary) (horn), comb (of a rooster) \item 
\textbf{\textit{kokawe}} (\textsc{n}) bird species (possibly < \ili{Yuat}) \item 
\textbf{\textit{koko}} (\textsc{n}) cocoa (< \ili{Tok Pisin} \textit{koko} ‘cocoa’) \item 
\textbf{\textit{kol-}} (\textsc{v}) break, split \item 
\textbf{\textit{Kolpe}} [male name] \item 
\textbf{\textit{kom}} [\isi{non-verbal negator}, ‘\textsc{neg}’] \item 
\textbf{\textit{komblam}} (\textsc{n}) chair \item 
\textbf{\textit{kome}} [\isi{non-verbal negator}, ‘\textsc{neg}’] \item 
\textbf{\textit{kon}} (\textsc{n}) corn (maize) (< \ili{Tok Pisin} \textit{kon} ‘corn’) \item 
\textbf{\textit{Konawa}} [male name] \item 
\textbf{\textit{Kongos}} [male name] \item 
\textbf{\textit{kop}} (\textsc{adv}) please \item 
\textbf{\textit{kot-}} (\textsc{v}) break; bear, give birth \item 
\textbf{\textit{Kowe}} [male name] \item 
\textbf{\textit{kuk u-}} (\textsc{v}) gather, pile; assemble, unite (literally ‘put a gathering’?) \item 
\textbf{\textit{kukul}} (\textsc{n}) type of basket (basket used for carrying sago) \item 
\textbf{\textit{kukum}} (\textsc{n}) insect species (grasshopper) \item 
\textbf{\textit{kukumali}} (\textsc{n}) bird species \item 
\textbf{\textit{kukumbe}} (\textsc{n}) clay pot used to hold water \item 
\textbf{\textit{kukumbe}} (\textsc{n}) sago species (sago palm with no spines) \item 
\textbf{\textit{kukun}} (\textsc{n}) type of beam (horizontal beam on the top of a house, under the roof) \item 
\textbf{\textit{kuli lï-}} (\textsc{v}) throw (literally ‘put a throw’?) \item 
\textbf{\textit{kulkul}} (\textsc{n}) bird species \item 
\textbf{\textit{kuma}} (\textsc{quant}) some \item 
\textbf{\textit{kuma}} (\textsc{q}) who? [\textsc{nsg}] (\isi{non-singular} \isi{interrogative pronoun}) \item 
\textbf{\textit{kuman}} (\textsc{n}) bird species (large wildfowl) \item 
\textbf{\textit{kumanji}} (\textsc{q}) whose? [\textsc{nsg}] (\isi{non-singular} \isi{interrogative pronoun}) \item 
\textbf{\textit{Kumba}} [place] Bun village \item 
\textbf{\textit{kumblima}} (\textsc{n}) betel pepper species (TP \textit{daka}) (long bean-like betel pepper) \item 
\textbf{\textit{kun-}} (\textsc{v}) break, break off \item 
\textbf{\textit{kundan}} (\textsc{n}) fish species (eel) \item 
\textbf{\textit{kunya}} (\textsc{n}) yam species (yam with red skin and reddish-pink flesh) \item 
\textbf{\textit{kwa}} (\textsc{adv}) just, simply, without care, without reason (also \textbf{\textit{ko}}, \textbf{\textit{wa}}) \item 
\textbf{\textit{kwa}} (\textsc{num}) one (also \textbf{\textit{kwe}}) \item 
\textbf{\textit{kwa}} (\textsc{pro}) someone; other, another \item 
\textbf{\textit{kwa}} (\textsc{q}) who? [\textsc{sg}] (\isi{singular} \isi{interrogative pronoun}) \item 
\textbf{\textit{kwanji}} (\textsc{q}) whose? [\textsc{sg}] (\isi{singular} \isi{interrogative pronoun}) \item 
\textbf{\textit{kwe}} (\textsc{num}) one (also \textbf{\textit{kwa}}) \item 
\textbf{\textit{kwekaka}} (\textsc{quant}) one each, one by one, just a few (also \textbf{\textit{kekaka}})\\ \item 

\newpage

\noindent \textbf{<L, l>        [l]}\\ \item 

\textbf{\textit{l}} (\textsc{v}) put (abbreviated form of \textbf{\textit{lï-}}) \item 
\textbf{\textit{la}} (\textsc{dem}) those (\isi{plural} \isi{distal} \isi{demonstrative}, ‘\textsc{pl.dist}’) (abbreviated form of \textbf{\textit{ala}}) \item 
\textbf{\textit{la-}} (\textsc{v}) eat, drink; chew, bite, suck; smoke (tobacco) (irregular \isi{irrealis} \isi{stem} of \textbf{\textit{ama-}}) \item 
\textbf{\textit{la-}} [irregular \isi{irrealis} \isi{prefix}, ‘\textsc{irr}’] (for \textbf{\textit{ka-}} ‘let’, \textbf{\textit{wo-}} ‘sleep’) \item 
\textbf{\textit{la=}} (\textsc{dem}) those (non-subject \isi{plural} \isi{distal} \isi{demonstrative}, ‘\textsc{pl.dist}’) (abbreviated form of \textbf{\textit{ala=}}) \item 
\textbf{\textit{laka-}} (\textsc{v}) let, leave, allow (irregular \isi{irrealis} \isi{stem} of \textbf{\textit{ka-}}) \item 
\textbf{\textit{lam}} (\textsc{n}) meat, flesh, muscle (< \ili{Ap Ma}) \item 
\textbf{\textit{lamban}} (\textsc{n}) nut species (nut larger than betel nut and also chewed) \item 
\textbf{\textit{lamndu}} (\textsc{n}) pig (also \textbf{\textit{namndu}}) \item 
\textbf{\textit{lamndu mu}} (\textsc{n}) insect species (blowfly that follows pigs and stings) (literally ‘pig blowfly’) \item 
\textbf{\textit{lamndu unduwan}} (\textsc{num}) twenty (literally ‘pig head’) \item 
\textbf{\textit{lamndu uta}} (\textsc{n}) bird species (literally ‘pig bird’) \item 
\textbf{\textit{langay}} (\textsc{n}) bird species (red-and-black parrot-like bird) \item 
\textbf{\textit{lanjin}} (\textsc{n}) fish species (perch) (TP \textit{nilpis}) (< \ili{Ap Ma}) \item 
\textbf{\textit{lapa-}} (\textsc{v}) plant \item 
\textbf{\textit{law}} (\textsc{n}) bunch of bananas \item 
\textbf{\textit{law}} (\textsc{n}) plant species (cordyline, ti plant) (TP \textit{tanget}) \item 
\textbf{\textit{layk}} (\isi{modal} marker) be about to (< \ili{Tok Pisin} \textit{laik} ‘want’, \isi{future} marker) \item 
\textbf{\textit{le}} (\textsc{n}) rattan cane (TP \textit{kanda}); bowstring (possibly < \ili{Ap Ma}) \item 
\textbf{\textit{lele}} (\textsc{num}) three \item 
\textbf{\textit{lemetam}} (\textsc{n}) tree species (large hardwood tree) (TP \textit{ton}); (\textsc{adj)} brown \item 
\textbf{\textit{lemta}} (\textsc{n}) spade \item 
\textbf{\textit{lemum}} (\textsc{n}) wart \item 
\textbf{\textit{li}} (\textsc{adv}) down, downward, downstream; (\textsc{adj}) low, lower (possibly < \ili{Ap Ma}) \item 
\textbf{\textit{li}} (\textsc{v}) go down (< \textbf{\textit{li}} ‘down’ + \textbf{\textit{i}} ‘go.\textsc{pfv}’) \item 
\textbf{\textit{li tanum}} (\textsc{n}) lower lip, area below the mouth (literally ‘lower lip’) \item 
\textbf{\textit{li u-}} (\textsc{v}) fall (literally ‘put down’) \item 
\textbf{\textit{lï-}} (\textsc{v}) put \item 
\textbf{\textit{limama}} (\textsc{n}) jaw (literally ‘down mouth’) \item 
\textbf{\textit{lïmndï}} (\textsc{n}) eye \item 
\textbf{\textit{lïmndï ala-}} (\textsc{v}) look, see, watch (literally ‘see [by means of] eye’) (also \textbf{\textit{lïmndï andï-}}) \item 
\textbf{\textit{lïmndï andï-}} (\textsc{v}) look, see, watch (literally ‘see [by means of] eye’) (also \textbf{\textit{lïmndï ala-}}) \item 
\textbf{\textit{lïmndï inim}} (\textsc{n}) tear, teardrop (literally ‘eye water’) (also \textbf{\textit{sal}}) \item 
\textbf{\textit{lïmndï lï-}} (\textsc{v}) watch, look at (literally ‘put eye’) \item 
\textbf{\textit{lïmndï minyam}} (\textsc{n}) eye mucus (literally ‘eye excrement’) \item 
\textbf{\textit{lïmndï mu}} (\textsc{n}) iris, pupil (literally ‘eye fruit’) \item 
\textbf{\textit{lïmndï uta-}} (\textsc{v}) check, examine (literally ‘grind eye’) \item 
\textbf{\textit{lïmndï wopa}} (\textsc{adj}) blind (literally ‘whole eye’) \item 
\textbf{\textit{lindïn}} (\textsc{n}) plant species (edible fern) \item 
\textbf{\textit{lïngïn}} (\textsc{n}) fog (< \ili{Mwakai}) \item 
\textbf{\textit{lingïnane}} (\textsc{n}) spider web \item 
\textbf{\textit{lïwa}} (\textsc{n}) dawn \item 
\textbf{\textit{lo-}} (\textsc{v}) cut, carve, cut down, chop, fell; go \item 
\textbf{\textit{lo-}} [irregular \isi{irrealis} \isi{prefix}, \textsc{irr}’] (for \textbf{\textit{wo-}} ‘sleep’) (\isi{allomorph} of \textbf{\textit{la-}}) \item 
\textbf{\textit{lolop}} (\textsc{adv)} just (< \ili{Ap Ma}) \item 
\textbf{\textit{lomon-}} (\textsc{v}) ignite, set fire to \item 
\textbf{\textit{longom}} (\textsc{n}) dream \item 
\textbf{\textit{lop ka-}} (\textsc{v}) lie, lie down (literally ‘let lie’?) \item 
\textbf{\textit{lopo-}} (\textsc{v}) rain, wash, bathe \item 
\textbf{\textit{lowo-}} (\textsc{v}) sleep (irregular \isi{irrealis} \isi{stem} of \textbf{\textit{wo-}}) \item 
\textbf{\textit{lu}} (\textsc{p}) with (\isi{comitative}) (\isi{allomorph} of \textbf{\textit{ul}}) \item 
\textbf{\textit{lu-}} (\textsc{v}) cut, carve, cut down, chop, fell; go (\isi{irrealis} \isi{stem} of \textbf{\textit{lo-}}) \item 
\textbf{\textit{luke}} (\textsc{adv}) also, too \item 
\textbf{\textit{lumnjap}} (\textsc{n}) fish species (garfish) (possibly < \ili{Waran}) \item 
\textbf{\textit{lumo-}} (\textsc{v}) put \item 
\textbf{\textit{lungum}} (\textsc{n}) long spear made of sharpened palm stem, used to fight \item 
\textbf{\textit{luwa}} (\textsc{n}) place\\ \item 

\noindent \textbf{<M, m>      [m]}\\ \item 

\textbf{\textit{m}} (\textsc{interj}) hm (expressed disapproval); mhm (signals agreement) \item 
\textbf{\textit{-m}} [irregular \isi{irrealis} \isi{suffix}, ‘\textsc{irr}’] (for \textbf{\textit{asa-}} {\textasciitilde} \textbf{\textit{atï-}} ‘hit’) \item 
\textbf{\textit{-m}} [irregular \isi{perfective} \isi{suffix}, ‘\textsc{pfv}’] (for \textbf{\textit{andï-}} ‘see’) \item 
\textbf{\textit{ma}} (\textsc{conj}) and \item 
\textbf{\textit{ma-}} (\textsc{v}) go, flow \item 
\textbf{\textit{ma=}} (\textsc{pro}) him, her, it (3\textsc{sg} non-subject \isi{pronoun}, ‘\textsc{3sg.obj}’; 3\textsc{sg} \isi{object marker}, ‘\textsc{3sg.obj}’) \item 
\textbf{\textit{mae}} (\textsc{n}) shovel, spade (possibly a \isi{compound} containing \textbf{\textit{me}} ‘palm species’) \item 
\textbf{\textit{maep}} (\textsc{n}) bird species (possibly \isi{onomatopoetic}) \item 
\textbf{\textit{mak}} (\textsc{n}) tattoo (< \ili{Tok Pisin} \textit{mak} ‘mark, tattoo’) \item 
\textbf{\textit{maka}} (\textsc{adv)} thus, in this way, in that way; (filler word) like (TP \textit{olsem}) (also \textbf{\textit{ka}}, \textbf{\textit{mïka}}) \item 
\textbf{\textit{malalïwa}} (\textsc{n}) snake species \item 
\textbf{\textit{Malman}} [male name] \item 
\textbf{\textit{mama}} (\textsc{n}) mouth \item 
\textbf{\textit{mamal}} (\textsc{n}) yawn \item 
\textbf{\textit{Mamala}} [place] Maruat village \item 
\textbf{\textit{maman}} (\textsc{n}) insect species (dragonfly) \item 
\textbf{\textit{Mamanu}} [place] downstream half of the old \textbf{\textit{Wopata}} village \item 
\textbf{\textit{mambi}} (\textsc{pro}) as for him, as for her, as for it (3\textsc{sg} \isi{topic-marker pronoun}, \linebreak‘\textsc{3sg.obj-top}’) \item 
\textbf{\textit{mambilakan}} (expression) forget about it! (literally ‘as for it, let it!’) \item 
\textbf{\textit{mambun}} (\textsc{n}) insect species (bedbug) \item 
\textbf{\textit{mambun}} (\textsc{n}) vegetable species (amaranth) (TP \textit{aupa}) \item 
\textbf{\textit{mamnda}} (\textsc{n}) plant species (stinging nettle with large leaves) (TP \textit{salat}) \item 
\textbf{\textit{mamwapa}} (\textsc{n}) bird species (owl) \item 
\textbf{\textit{mana}} (\textsc{n}) spear \item 
\textbf{\textit{manal}} (\textsc{n}) hot water \item 
\textbf{\textit{manal u-}} (\textsc{v}) boil (literally ‘put in hot water’) \item 
\textbf{\textit{Manama}} [male name] \item 
\textbf{\textit{manana}} (\textsc{n}) snail species (river snail) \item 
\textbf{\textit{manangum}} (\textsc{n}) stick with decorations used in dances \item 
\textbf{\textit{mangusuwa}} (\textsc{pro}) the poor thing (3\textsc{sg} \isi{affective pronoun}, ‘\textsc{3sg.obj}-poor’) 
 \linebreak (also \textbf{\textit{mangusuwata}}) \item 
\textbf{\textit{mangusuwata}} (\textsc{pro}) the poor thing (3\textsc{sg} \isi{affective pronoun}, ‘\textsc{3sg.obj}-poor’)  \linebreak (also \textbf{\textit{mangusuwa}}) \item 
\textbf{\textit{manji}} (\textsc{pro}) his, her, hers, its (3\textsc{sg} \isi{possessive pronoun}, ‘\textsc{3sg.obj-poss}’) \item 
\textbf{\textit{manjimanji}} (\textsc{n}) maggot (cf. \textbf{\textit{njimana}} ‘fly’) \item 
\textbf{\textit{Mapana}} [female name] \item 
\textbf{\textit{Maple}} [female name] \item 
\textbf{\textit{mapu}} (\textsc{n}) fish species (gudgeon) (TP \textit{bikmaus}) \item 
\textbf{\textit{Marungun}} [male name] \item 
\textbf{\textit{mas}} (\isi{modal} marker) should, must (< \ili{Tok Pisin} \textit{mas} ‘should, must’) \item 
\textbf{\textit{masamasa}} (\textsc{n}) tree species \item 
\textbf{\textit{maski}} (\textsc{conj}) although, even though (< \ili{Tok Pisin} \textit{maski}, ‘although’) \item 
\textbf{\textit{matamal}} (\textsc{adj}) sharp; difficult; angry \item 
\textbf{\textit{matlaka}} (\textsc{n}) rat species \item 
\textbf{\textit{maw}} (\textsc{adj}) correct, right \item 
\textbf{\textit{mawa}} (\textsc{pro}) himself, herself, itself, he himself, she herself, it itself, him himself, her herself (3\textsc{sg} \isi{intensive pronoun}, ‘\textsc{3sg.obj-int}’) \item 
\textbf{\textit{mawe}} (\textsc{pro}) himself, herself, itself, he himself, she herself, it itself, him himself, her herself (from among several) (3\textsc{sg} \isi{partitive-intensive pronoun}, \linebreak‘\textsc{3sg.obj-part.int}’) \item 
\textbf{\textit{maweka}} (\textsc{adv)} also, moreover (also \textbf{\textit{moweka}}) \item 
\textbf{\textit{Mawna}} [female name] \item 
\textbf{\textit{mawnam}} (\textsc{interj}) that’s it (signals \isi{emphatic} identification or approval) \item 
\textbf{\textit{may}} (\textsc{n}) fish species (catfish) (TP \textit{mausgras pis}) \item 
\textbf{\textit{me}} [\isi{non-verbal negator}, ‘\textsc{neg}’] \item 
\textbf{\textit{me}} (\textsc{n}) palm species; flattened palm stem (TP \textit{limbum}) \item 
\textbf{\textit{me-}} (\textsc{v}) sew \item 
\textbf{\textit{membul}} (\textsc{n}) bird species (small pigeon-like bird with brown sides) \item 
\textbf{\textit{metmet}} (\textsc{n}) type of spirit (swamp dwarf) \item 
\textbf{\textit{mi}} (\textsc{n}) crayfish species (small crayfish) \item 
\textbf{\textit{mi}} (\textsc{n}) splinter, strand, fiber (inside the husk of a coconut) \item 
\textbf{\textit{mï}} (\textsc{pro}) he, she, it (3\textsc{sg} subject \isi{pronoun}, ‘\textsc{3sg.subj}’; 3\textsc{sg} \isi{subject marker}, \linebreak ‘\textsc{3sg.subj}’) \item 
\textbf{\textit{mïka}} (\textsc{adv)} thus, in this way, in that way; (filler word) like (TP \textit{olsem}) (also \textbf{\textit{ka}}, \textbf{\textit{maka}}) \item 
\textbf{\textit{mïka}} (\textsc{n}) tree species (fig tree) (TP \textit{fikus}) \item 
\textbf{\textit{mïka itïm}} (\textsc{n}) swamp (literally ‘fig tree trash’) (also \textbf{\textit{apïnal}}) \item 
\textbf{\textit{Mïkïlwe}} [place] jungle region near Manu village \item 
\textbf{\textit{mil}} (\textsc{n}) sugarcane, sugar \item 
\textbf{\textit{mïli}} (\textsc{n}) vegetable species (tall ginger) (TP \textit{gorgor}) \item 
\textbf{\textit{mïmïl u-}} (\textsc{v}) wring (as sago fibers), squeeze, strain (literally ‘put a squeeze’?) \item 
\textbf{\textit{mïmin}} (\textsc{n}) louse (on humans) \item 
\textbf{\textit{min}} (\textsc{n}) armband, belt, joint for pick-axe \item 
\textbf{\textit{min}} (\textsc{pro}) they (3\textsc{du} subject \isi{pronoun}, ‘\textsc{3du}’; 3\textsc{du} \isi{subject marker}, ‘\textsc{3du}’) \item 
\textbf{\textit{min=}} (\textsc{pro}) them (3\textsc{du} non-subject \isi{pronoun}, ‘\textsc{3du}’; 3\textsc{du} \isi{object marker}, ‘\textsc{3du}’) \item 
\textbf{\textit{mïnal}} (\textsc{n}) taro; (\textsc{adj}) green \item 
\textbf{\textit{mïnal anmoka}} (\textsc{n}) snake species (green snake) (literally ‘taro snake’) \item 
\textbf{\textit{minam}} (\textsc{n}) urine \item 
\textbf{\textit{mïnam}} (\textsc{pro}) he is the one, she is the one, it is the one (3\textsc{sg} \isi{emphatic pronoun}, ‘\textsc{3sg.subj-emph}’) \item 
\textbf{\textit{minambi}} (\textsc{pro}) as for them (3\textsc{du} \isi{topic-marker pronoun}, ‘\textsc{3du-top}’) \item 
\textbf{\textit{mïnandïn}} (\textsc{n}) gallbladder \item 
\textbf{\textit{mïnane}} (\textsc{n}) intestines, guts \item 
\textbf{\textit{mïnanum}} (\textsc{n}) mature, fully ripe betel nut fruit (the stage following \textbf{\textit{aw wapata}} ‘mature betel nut’) \item 
\textbf{\textit{mïnap}} (\textsc{adj}) rotting \item 
\textbf{\textit{minawa}} (\textsc{pro}) themselves, they themselves, them themselves (\textsc{3du} \isi{intensive pronoun}, ‘\textsc{3du-int}’) \item 
\textbf{\textit{mïnda}} (\textsc{n}) banana (plant or fruit) \item 
\textbf{\textit{mïndam}} (\textsc{n}) pus \item 
\textbf{\textit{mïndapan}} (\textsc{n}) banana leaf; paper (probably < \textbf{\textit{mïnda}} ‘banana’ + \textbf{\textit{wapa}} ‘leaf’; origin of \textit{n} is unknown) \item 
\textbf{\textit{mïndit}} (\textsc{adj}) yellow \item 
\textbf{\textit{mïngamata}} (\isi{placeholder word}) whatchamacallit \item 
\textbf{\textit{mingusuwa}} (\textsc{pro}) the poor things [\textsc{du}] (3\textsc{du} \isi{affective pronoun}, ‘\textsc{3du}-poor’) (also \textbf{\textit{mingusuwata}}) \item 
\textbf{\textit{mingusuwata}} (\textsc{pro}) the poor things [\textsc{du}] (3\textsc{du} \isi{affective pronoun}, ‘\textsc{3du}-poor’) (also \textbf{\textit{mingusuwa}}) \item 
\textbf{\textit{mini=}} (\textsc{pro}) them (3\textsc{du} non-subject \isi{pronoun}, ‘\textsc{3du.obj}’; 3\textsc{du} \isi{object marker}, ‘\textsc{3du.obj}’) (\isi{allomorph} of \textbf{\textit{min=}}) \item 
\textbf{\textit{mïnïm}} (\textsc{n}) tongue; strap of a bag \item 
\textbf{\textit{mïnja}} (\textsc{n}) speech \item 
\textbf{\textit{minji}} (\textsc{pro}) their, theirs (\textsc{3du} \isi{possessive pronoun}, ‘\textsc{3du-poss}’) \item 
\textbf{\textit{mïnjika}} (\textsc{n}) this kind of speech, that kind of speech \item 
\textbf{\textit{mïnkïn}} (\textsc{n}) grub species (small edible sago grub, the larva of the \textbf{\textit{apwane}} insect species) (< \ili{Ap Ma}) \item 
\textbf{\textit{mïnkïn ulum}} (\textsc{n}) sago species (sago palm with many spines, used for harvesting \textbf{\textit{mïnkïn}} grubs) (literally ‘sago grub palm’) \item 
\textbf{\textit{mïnkïn we}} (\textsc{n}) sago pancake fried with \textbf{\textit{mïnkïn}} grubs (literally ‘sago grub sago’) \item 
\textbf{\textit{mïnoma}} (\textsc{adj}) cold, cool \item 
\textbf{\textit{mïnopal}} (\textsc{n}) bladder \item 
\textbf{\textit{mïnwata}} (\textsc{adj}) wet, ripe, rotting, rotten, spoiled \item 
\textbf{\textit{minwe}} (\textsc{pro}) themselves, they themselves, them themselves (from among \linebreak several) (3\textsc{du} \isi{partitive-intensive pronoun}, ‘\textsc{3du-part.int}’) \item 
\textbf{\textit{minyam}} (\textsc{n}) feces, excrement \item 
\textbf{\textit{misam}} (\textsc{n}) brain, brains \item 
\textbf{\textit{misimisi}} (\textsc{n}) story \item 
\textbf{\textit{misisina-}} (\textsc{v}) arrange \item 
\textbf{\textit{mïtïn}} (\textsc{n}) egg; testicle \item 
\textbf{\textit{mïtïn}} (\textsc{n}) language (TP \textit{tokples}) \item 
\textbf{\textit{mïtïn ame}} (\textsc{n}) scrotum (literally ‘testicle bag’) \item 
\textbf{\textit{mm}} (\textsc{interj}) uh-uh (signals disagreement) \item 
\textbf{\textit{mo=}} (\textsc{pro}) him, her, it (3\textsc{sg} non-subject \isi{pronoun}, ‘\textsc{3sg.obj}’; 3\textsc{sg} \isi{object marker}, ‘\textsc{3sg.obj}’) (\isi{allomorph} of \textbf{\textit{ma=}}) \item 
\textbf{\textit{moko-}} (\textsc{v}) take, take one-by-one, catch \item 
\textbf{\textit{mokum}} (\textsc{n}) stealth \item 
\textbf{\textit{mokum moko-}} (\textsc{v}) steal (literally ‘take stealth’) \item 
\textbf{\textit{molombi}} (\textsc{n}) statuette, spirit idol (< \ili{Ap Ma}; possibly ultimately < \ili{Waran}) \item 
\textbf{\textit{molpan}} (\textsc{n}) type of spirit (tree spirit) \item 
\textbf{\textit{mom}} (\textsc{n}) grandmother (< \ili{Ap Ma}) \item 
\textbf{\textit{moma}} (\textsc{n}) leaf tied in an overhand knot, used to summon the spirit of the \linebreak deceased \item 
\textbf{\textit{momul}} (\textsc{n}) glowing fungus, mold \item 
\textbf{\textit{monam}} (\textsc{n}) tree species (rain tree) (TP \textit{marmar}) \item 
\textbf{\textit{Monde}} [male name] \item 
\textbf{\textit{mondin}} (\textsc{n}) fruit species (fruit similar to a watermelon) \item 
\textbf{\textit{mondo-}} (\textsc{v}) dry, smoke \item 
\textbf{\textit{mongi}} (\textsc{n}) banana species (banana plant with sweet, thin, long fruit) \item 
\textbf{\textit{Mongima}} [male name] \item 
\textbf{\textit{moni}} (\textsc{n}) bird species (red bird with a beak like a parrot’s) \item 
\textbf{\textit{moni}} (\textsc{p}) between, among \item 
\textbf{\textit{moniwot}} (\textsc{n}) plant species (croton shrub) (TP \textit{purpur}) \item 
\textbf{\textit{monkin}} (\textsc{n}) gray hair, white hair \item 
\textbf{\textit{monombam}} (\textsc{n}) forehead, face \item 
\textbf{\textit{monop}} (\textsc{adj}) full, sated \item 
\textbf{\textit{mop lï-}} (\textsc{v}) tie (literally ‘put a tie’?) \item 
\textbf{\textit{Morombi}} [place] Raten village \item 
\textbf{\textit{Mosombla}} [place] Yaul village \item 
\textbf{\textit{mot}} (\textsc{n}) awning of a house; porch, veranda under the awning \item 
\textbf{\textit{mota}} (\textsc{n}) bamboo species used for cooking fish; bamboo flute; throat \item 
\textbf{\textit{motam}} (\textsc{n}) stick, bundle; bunch of coconuts \item 
\textbf{\textit{moweka}} (\textsc{adv)} also, moreover (also \textbf{\textit{maweka}}) \item 
\textbf{\textit{mu}} (\textsc{n}) fruit, seed, nut, berry; bump, mosquito bite; head or tip of a tool, striking end of a pick-axe \item 
\textbf{\textit{mu}} (\textsc{n}) insect species (blowfly) (TP \textit{blulang}) \item 
\textbf{\textit{Mukamba}} [male name] \item 
\textbf{\textit{muku}} (\textsc{n}) package, packet (as of jellied sago, wrapped in a leaf) \item 
\textbf{\textit{mukuwi}} (\textsc{n}) older sago palm with flowers \item 
\textbf{\textit{mulwat}} (\textsc{n}) bird species \item 
\textbf{\textit{mumne}} (\textsc{adj}) cold and dark \item 
\textbf{\textit{muna}} (\textsc{n}) insect species (large brown ant) \item 
\textbf{\textit{mundotoma}} (\textsc{adj}) short, lacking \item 
\textbf{\textit{mundu}} (\textsc{n}) food, animal; hunger \item 
\textbf{\textit{mundu asa-}} (\textsc{v}) be hungry (literally ‘hunger hits [someone]’) \item 
\textbf{\textit{mundum}} (\textsc{n}) grub species (edible, mid-sized grub, the larva of the \textbf{\textit{nitïl}} insect species) \item 
\textbf{\textit{mune u-}} (\textsc{v}) throw (literally ‘put a throw’?) \item 
\textbf{\textit{mungul}} (\textsc{n}) plant species (edible fern with small leaves) \item 
\textbf{\textit{mungun}} (\textsc{n}) earring, ring; earwax \item 
\textbf{\textit{mupu}} (\textsc{n}) core of a tree or palm, pulp; meat of a coconut, sago palm, or betel nut fruit \item 
\textbf{\textit{mutam}} (\textsc{n}) back (of the body) \item 
\textbf{\textit{mutam}} (\textsc{n}) tree species (tree with leaves used to wrap sago or bandage wounds) \item 
\textbf{\textit{mutoma}} (\textsc{n}) backbone, spine (probably < \textbf{\textit{mutam}} ‘back’ + \textbf{\textit{uma}} ‘bone’) \item 
\textbf{\textit{mutulum}} (\textsc{n}) mud \item 
\textbf{\textit{mwa}} (\textsc{n}) opening, door, window, eye of a needle; face\\ \item 

\noindent \textbf{<Mb, mb>    [ᵐb]}\\ \item 

\textbf{\textit{mbalanji}} (\textsc{n}) enemy, stranger (< \ili{Yuat} word for ‘person’) \item 
\textbf{\textit{mbalus}} (\textsc{n}) airplane (< \ili{Tok Pisin} \textit{balus} ‘dove, airplane’) \item 
\textbf{\textit{mbatmbat}} (\textsc{n}) fish species (tilapia) (TP \textit{makau}) \item 
\textbf{\textit{mbay}} (\isi{modal} marker) will (< \ili{Tok Pisin} \textit{bai} ‘will’) \item 
\textbf{\textit{mbï}} (\textsc{adv)} here; to here, hither \item 
\textbf{\textit{mbi-}} (\textsc{v}) come here (< \textbf{\textit{mbï}} ‘here’ + \textbf{\textit{i-}} ‘come’) \item 
\textbf{\textit{mbïlanda}} (\textsc{n}) palm species (palm used to make bows) \item 
\textbf{\textit{mbinmbin}} (\textsc{n}) grave, cemetery \item 
\textbf{\textit{mblandu}} (\textsc{n}) rat species (rat that lives in the water) (< \ili{Ap Ma}) \item 
\textbf{\textit{mbomala}} (\textsc{n}) insect species (large firefly); large star, planet \item 
\textbf{\textit{mbomala nangum}} (\textsc{n}) flashlight (literally ‘firefly shoot’) \item 
\textbf{\textit{mbone}} (\textsc{n}) crab \item 
\textbf{\textit{mbu}} (\textsc{adv)} here; from here, hence \item 
\textbf{\textit{mbuka}} (\textsc{adv}) quickly \item 
\textbf{\textit{mbun}} (\textsc{adj}) black, blue, dark; (\textsc{n}) scar (< \ili{Mwakai}) \item 
\textbf{\textit{mbunmana}} (\textsc{adj}) black\\ \item 

\newpage

\noindent \textbf{<N, n>        [n]}\\ \item 

\textbf{\textit{n}} [\isi{epenthetic} utterance-final sound for some speakers] \item 
\textbf{\textit{-n}} [\isi{imperative} \isi{suffix}, ‘\textsc{imp}’] \item 
\textbf{\textit{-n}} [irregular \isi{imperfective} \isi{suffix}, ‘\textsc{ipfv}’] (for \textbf{\textit{ma-}} ‘go’) \item 
\textbf{\textit{-n}} [irregular \isi{perfective} \isi{suffix}, ‘\textsc{pfv}’] (for \textbf{\textit{i-}} ‘come’, \textbf{\textit{na-}} ‘give’, \textbf{\textit{tï-}} ‘take’) \item 
\textbf{\textit{-n}} [\isi{nominalizing} \isi{suffix}, ‘\textsc{nmlz}’] (\isi{allomorph} of \textbf{\textit{-en}}) \item 
\textbf{\textit{=n}} [\isi{oblique marker}, ‘\textsc{obl}’] \item 
\textbf{\textit{=n =n kalam me-}} (\textsc{v}) teach (literally ‘sew knowledge about [something] to [someone]’) \item 
\textbf{\textit{=n ul si-}} (\textsc{v}) show (literally ‘push with [something] with [someone]’) \item 
\textbf{\textit{na}} (\textsc{n}) talk, speech, story, message, thought, reason, cause, language \item 
\textbf{\textit{na}} (\textsc{conj}) and (< \ili{Tok Pisin} \textit{na} ‘and’) \item 
\textbf{\textit{na-}} (\textsc{v}) feed \item 
\textbf{\textit{na-}} (\textsc{v}) give \item 
\textbf{\textit{na-}} [\isi{detransitivizing} \isi{prefix}, ‘\textsc{detr}’] \item 
\textbf{\textit{-na}} [\isi{irrealis} \isi{suffix}, ‘\textsc{irr}’] \item 
\textbf{\textit{-na}} [irregular \isi{perfective} \isi{suffix}, ‘\textsc{pfv}’] (for \textbf{\textit{na-}} ‘give’) (\isi{allomorph} of \textbf{\textit{-n}}) \item 
\textbf{\textit{naka}} (\textsc{adv}) after, afterwards, later, soon (abbreviated form of \textbf{\textit{anganika}}) \item 
\textbf{\textit{nakam wanmbi}} (\textsc{n}) betel pepper species (wild betel pepper) (TP \textit{wel daka}) (\textbf{\textit{wanmbi}} is ‘betel pepper’; meaning of \textit{nakam} is unknown) \item 
\textbf{\textit{nakamb-}} (\textsc{v}) suffice, have enough (< \textbf{\textit{na-}} ‘\textsc{detr}’ + \textbf{\textit{kamb-}} ‘shun, avoid’) \item 
\textbf{\textit{nakap}} (\textsc{p}) on account of, because of, for the sake of, for (also \textbf{\textit{nap}}) \item 
\textbf{\textit{nali}} (\textsc{n}) insect species (small firefly); small star \item 
\textbf{\textit{nali}} (\textsc{n}) spine of a sago frond used to make baskets or arrows \item 
\textbf{\textit{nali}} (\textsc{num}) ten (literally ‘sago frond spine’) \item 
\textbf{\textit{nali angay}} (\textsc{num}) fifty (= 10+5) \item 
\textbf{\textit{nali kwe kwe}} (\textsc{num}) eleven (= 10${\cdot}$1+1) \item 
\textbf{\textit{nali kwe lele}} (\textsc{num}) thirteen (= 10${\cdot}$1+3) \item 
\textbf{\textit{nali kwe nini}} (\textsc{num}) twelve (= 10${\cdot}$1+2) \item 
\textbf{\textit{nali kwe watangïnila}} (\textsc{num}) fourteen (= 10${\cdot}$1+4) \item 
\textbf{\textit{nali lele}} (\textsc{num}) thirty (= 10${\cdot}$3) \item 
\textbf{\textit{nali nini}} (\textsc{num}) twenty (= 10${\cdot}$2) \item 
\textbf{\textit{nali nini angay}} (\textsc{num}) twenty-five (= 10${\cdot}$2+5) \item 
\textbf{\textit{nali watangïnila}} (\textsc{num}) forty (= 10${\cdot}$4) \item 
\textbf{\textit{-nam}} [\isi{emphatic} \isi{suffix}, ‘\textsc{emph}’] \item 
\textbf{\textit{namanu}} (\isi{farewell}) goodbye (addressed to someone who is leaving) \item 
\textbf{\textit{nambana}} (\textsc{n}) ancestral spirit, ghost; mask depicting a spirit’s face \item 
\textbf{\textit{nambana}} (\textsc{n}) sago palm flower \item 
\textbf{\textit{nambana}} (\textsc{n}) yam species (large white yam) \item 
\textbf{\textit{nambana ankam}} (\textsc{n}) extended family member (literally ‘spirit person’) \item 
\textbf{\textit{nambana mwa}} (\textsc{n}) mask (literally ‘spirit face’) \item 
\textbf{\textit{nambi}} (\textsc{pro}) as for me (1\textsc{sg} \isi{topic-marker pronoun}, ‘\textsc{1sg-top}’) \item 
\textbf{\textit{nambï}} (\textsc{n}) skin, hide; body \item 
\textbf{\textit{nambïlumo-}} (\textsc{v}) block (literally ‘put body’) \item 
\textbf{\textit{nambïnïkï-}} (\textsc{v}) make, nag (literally ‘dig [at someone’s] skin’) \item 
\textbf{\textit{nambït}} (\textsc{n}) odor, smell \item 
\textbf{\textit{nambït wana-}} (\textsc{v}) smell, sniff (literally ‘feel a smell’) \item 
\textbf{\textit{nambli}} (\textsc{n}) feather, fur \item 
\textbf{\textit{Nambu}} [male name] \item 
\textbf{\textit{nambum}} (\textsc{n}) inner membrane of an egg shell \item 
\textbf{\textit{nambuwe u-}} (\textsc{v}) peel (literally ‘put a skin-cutting’?) \item 
\textbf{\textit{namle}} (\textsc{n}) plant species (plant that grows in swamps) \item 
\textbf{\textit{namli}} (\textsc{adj}) soft, smooth \item 
\textbf{\textit{namna}} (\textsc{adj}) afraid, fearful, scared \item 
\textbf{\textit{namnap}} (\textsc{v}) be afraid, be scared (literally ‘be afraid’) \item 
\textbf{\textit{namndu}} (\textsc{n}) pig (also \textbf{\textit{lamndu}}) \item 
\textbf{\textit{nana}} (\textsc{n}) mama (a nursery term for mother); the \isi{vocative} form of \textbf{\textit{inom}} ‘mother’ for speakers of all ages \item 
\textbf{\textit{nanama}} (\textsc{adj}) bitter \item 
\textbf{\textit{nangïn}} (\textsc{n}) tongs (for cooking), scissors \item 
\textbf{\textit{nangu}} (\textsc{n}) lizard species (venomous brown lizard with a diamond-shaped head) \item 
\textbf{\textit{nangum}} (\textsc{n}) shoot, seedling (possibly a variant of \textbf{\textit{nungum}}) \item 
\textbf{\textit{nanïm}} (\textsc{n}) tree species (ironwood tree) (TP \textit{kwila}) \item 
\textbf{\textit{Nanïmwat}} [place] name of the old Yamen village \item 
\textbf{\textit{nanïwe}} (\textsc{n}) banana species (banana plant with small sweet fruit) \item 
\textbf{\textit{nap}} (\textsc{n}) arrow, fishing spear; yam thorn \item 
\textbf{\textit{nap}} (\textsc{p}) on account of, because of, for the sake of, for (also \textbf{\textit{nakap}}) \item 
\textbf{\textit{nasalïwa}} (\textsc{n}) leech \item 
\textbf{\textit{nataw}} (\textsc{n}) lizard species (large brown gecko) \item 
\textbf{\textit{nataw}} (\textsc{n}) white spot on the skin \item 
\textbf{\textit{natnat}} (\textsc{n}) vegetable, vegetables, greens (TP \textit{kumu}) \item 
\textbf{\textit{nawa}} (\textsc{pro}) I myself, me myself (1\textsc{sg} \isi{intensive pronoun}, ‘\textsc{1sg-int}’) \item 
\textbf{\textit{Nawoli}} [male name] \item 
\textbf{\textit{ne-}} (\textsc{v}) harvest \item 
\textbf{\textit{netïl}} (\textsc{n}) plant species (plant with black seeds) \item 
\textbf{\textit{ni}} (\textsc{n}) crayfish species (large crayfish) \item 
\textbf{\textit{ni-}} (\textsc{v}) act, do; beat \item 
\textbf{\textit{ni-}} (\textsc{v}) die (\isi{singular} subject) \item 
\textbf{\textit{nï}} (\textsc{pro}) I (1\textsc{sg} subject \isi{pronoun}, ‘\textsc{1sg}’) \item 
\textbf{\textit{nï=}} (\textsc{pro}) me (1\textsc{sg} non-subject \isi{pronoun}, ‘\textsc{1sg}’) \item 
\textbf{\textit{=nï}} [\isi{oblique marker}, ‘\textsc{obl}’] (\isi{allomorph} of \textbf{\textit{=n}}) \item 
\textbf{\textit{nïka-}} (\textsc{v}) dig, break up (ground), hoe; cut, butcher; prepare (sago) (\isi{perfective} \isi{stem} of \textbf{\textit{nïkï-}}) \item 
\textbf{\textit{nïkï-}} (\textsc{v}) dig, break up (ground), hoe; cut, butcher; prepare (sago) \item 
\textbf{\textit{nïkïn}} (\textsc{n}) hiccup; belch, burp \item 
\textbf{\textit{nïkït}} (\textsc{n}) lizard \item 
\textbf{\textit{nil}} (\textsc{n}) body hair \item 
\textbf{\textit{nil nopa}} (\textsc{n}) beard (literally ‘cheek hair’) \item 
\textbf{\textit{nim}} (\textsc{n}) nest \item 
\textbf{\textit{nïmal}} (\textsc{n}) river \item 
\textbf{\textit{Nïmalnu}} [place] Manu village \item 
\textbf{\textit{nïmban}} (\textsc{n}) fish species \item 
\textbf{\textit{nïmïn}} (\textsc{n}) mucus \item 
\textbf{\textit{nïmtu}} (\textsc{n}) bird species (very small green, yellow-breasted bird) \item 
\textbf{\textit{nin}} (\textsc{n}) thorn, spine \item 
\textbf{\textit{nïndiwe}} (\textsc{n}) sago species (small sago palm with no spines) \item 
\textbf{\textit{nini}} (\textsc{num}) two \item 
\textbf{\textit{nïnil}} (\textsc{n}) sago species (sago palm) \item 
\textbf{\textit{nïnji}} (\textsc{pro}) my, mine (1\textsc{sg} \isi{possessive pronoun}, ‘\textsc{1sg-poss}’) \item 
\textbf{\textit{nïnji anma}} (expression) thank you, thanks (literally ‘my good’) \item 
\textbf{\textit{nïpa}} (\textsc{n}) breadfruit \item 
\textbf{\textit{nïpat}} (\textsc{adj}) giant \item 
\textbf{\textit{nïpïl}} (\textsc{n}) vine, rope \item 
\textbf{\textit{nipinp u-}} (\textsc{v}) die (\isi{plural} subject) (literally ‘put deaths’?) \item 
\textbf{\textit{nïplopa}} (\textsc{n}) flying fox, large bat \item 
\textbf{\textit{nïpokonam}} (\textsc{adj}) hard \item 
\textbf{\textit{nipum}} (\textsc{n}) sword grass (\textit{Imperata cylindrica}) (TP \textit{kunai}) \item 
\textbf{\textit{nipum amba}} (\textsc{n}) grassland (\textbf{\textit{nipum}} is ‘sword grass’; relationship, if any, to \textbf{\textit{amba}} ‘men’s house’ is unknown) (also \textbf{\textit{anul}}) \item 
\textbf{\textit{nipunp u-}} (\textsc{v}) die (\isi{plural} subject) (literally ‘put deaths’?) (alternative form of \textbf{\textit{nipinp u-}}) \item 
\textbf{\textit{nisi}} (\textsc{n}) coconut flower sheath (TP \textit{pandol}); bunch of betel nut \item 
\textbf{\textit{nïte}} (\textsc{n}) type of drum (small hand drum) (TP \textit{kundu}) \item 
\textbf{\textit{nitïl}} (\textsc{n}) insect species (the adult form of the \textbf{\textit{mundum}} grub species) \item 
\textbf{\textit{nka-}} (\textsc{v}) dig, break up (ground), hoe; cut, butcher; prepare (sago) (abbreviated form of \textbf{\textit{nïka-}}) \item 
\textbf{\textit{nkï-}} (\textsc{v}) dig, break up (ground), hoe; cut, butcher; prepare (sago) (abbreviated form of \textbf{\textit{nïkï-}}) \item 
\textbf{\textit{nokal}} (\textsc{n}) beak \item 
\textbf{\textit{nokop lï-}} (\textsc{v}) hide (literally ‘put a hiding’?) \item 
\textbf{\textit{nokosam}} (\textsc{n}) tree species (Java almond tree) (TP \textit{galip}) \item 
\textbf{\textit{nol}} (expression) go!; let’s go! \item 
\textbf{\textit{nom}} (\textsc{n}) clay stand used to hold pots over a fire \item 
\textbf{\textit{Nomnga}} [male name] \item 
\textbf{\textit{nonal}} (\textsc{n}) wind, breath; the Holy Spirit \item 
\textbf{\textit{nonal u-}} (\textsc{v}) breathe (literally ‘put a breath’) \item 
\textbf{\textit{nonalni-}} (\textsc{v}) blow (of wind) (literally ‘do wind’) \item 
\textbf{\textit{Nongami}} [male name] \item 
\textbf{\textit{nongan}} (\textsc{n}) vomitus \item 
\textbf{\textit{nongan u-}} (\textsc{v}) vomit (literally ‘put vomitus’) \item 
\textbf{\textit{nongat}} (\textsc{interj)} no (expresses denial) (< TP \textit{nogat} ‘no’) \item 
\textbf{\textit{nongontam}} (\textsc{n}) sweet potato (TP \textit{kaukau}) \item 
\textbf{\textit{nongut}} (\textsc{conj}) lest (< \ili{Tok Pisin} \textit{nogut} ‘bad; lest') \item 
\textbf{\textit{nopa}} (\textsc{n}) cheek \item 
\textbf{\textit{nopal}} (\textsc{n}) coconut frond, used in roofing \item 
\textbf{\textit{nopal u-}} (\textsc{v}) crush, mash (literally ‘put a frond’?) \item 
\textbf{\textit{nowe}} (\textsc{n}) sago species (large sago palm with no spines) \item 
\textbf{\textit{nu}} (\textsc{adv)} near, close \item 
\textbf{\textit{nuku}} (\textsc{n}) flatus, fart \item 
\textbf{\textit{num}} (\textsc{n}) canoe, boat \item 
\textbf{\textit{numan}} (\textsc{n}) husband \item 
\textbf{\textit{numbu}} (\textsc{n}) tree species (ironwood tree) (TP \textit{garamut}); type of drum (large slit drum) (TP \textit{garamut}); post of a house \item 
\textbf{\textit{numbu motam}} (\textsc{n}) mallet used to beat the large slit drum (literally ‘slit drum stick’) \item 
\textbf{\textit{numbunum}} (\textsc{n}) insect species (large red bee, wasp) \item 
\textbf{\textit{numïni}} (\textsc{n}) ditch \item 
\textbf{\textit{numnata}} (\textsc{n}) earthquake \item 
\textbf{\textit{nuna}} (\textsc{n}) insect species (large mosquito-like insect) \item 
\textbf{\textit{nungol}} (\textsc{n}) child (often son, but may refer to any young person, boy or girl) (also \textbf{\textit{nungolke}}) \item 
\textbf{\textit{nungolke}} (\textsc{n}) child (often son, but may refer to any young person, boy or girl) (also \textbf{\textit{nungol}}) \item 
\textbf{\textit{nungum}} (\textsc{n}) sucker of a plant, used to plant new bananas, sago palms, etc. \linebreak (possibly a variant of \textbf{\textit{nangum}}) \item 
\textbf{\textit{nungun u-}} (\textsc{v}) break (\isi{intransitive}) (literally ‘put a break’?) \item 
\textbf{\textit{nunu}} (\textsc{quant}) every; various, many \item 
\textbf{\textit{nunu ika}} (\textsc{adv}) always, often, regularly (literally ‘every instance’) \item 
\textbf{\textit{nunu ilom}} (\textsc{adv}) every day (literally ‘every day’) \item 
\textbf{\textit{nunu nji}} (\textsc{p}) everything (literally ‘every thing’) \item 
\textbf{\textit{nupu}} (\textsc{n}) bottom, base; side of the coconut fruit without eyes; part of the yam that is planted in soil \item 
\textbf{\textit{nuwe}} (\textsc{pro}) I myself, me myself (from among several) (1\textsc{sg} \isi{partitive-intensive pronoun}, ‘\textsc{1sg-part.int}’)\\ \item 

\noindent \textbf{<Nd, nd>      [ⁿd]}\\ \item 

\textbf{\textit{nda}} (\textsc{dem}) that (\isi{singular} \isi{distal} \isi{demonstrative}, ‘\textsc{sg.dist}’) (abbreviated form of \textbf{\textit{anda}}) \item 
\textbf{\textit{nda=}} (\textsc{dem}) that (non-subject \isi{singular} \isi{distal} \isi{demonstrative}, ‘\textsc{sg.dist}’) \linebreak (abbreviated form of \textbf{\textit{anda=}}) \item 
\textbf{\textit{-nda}} [\isi{irrealis} \isi{suffix}, ‘\textsc{irr}’] (\isi{allomorph} of \textbf{\textit{-na}}) \item 
\textbf{\textit{ndal}} (\textsc{n}) vein, tendon, ligament \item 
\textbf{\textit{ndam}} (\textsc{n}) bridge \item 
\textbf{\textit{ndambi}} (\textsc{pro}) as for them (\textsc{3pl} \isi{topic-marker pronoun}, ‘\textsc{3pl-top}’) \item 
\textbf{\textit{ndanandum mu}} (\textsc{n}) kidney (\textbf{\textit{mu}} is ‘fruit’; meaning of \textit{ndanandum} is unknown) \item 
\textbf{\textit{ndande}} (\textsc{n}) shadow, shade \item 
\textbf{\textit{ndawa}} (\textsc{pro}) themselves, they themselves, them themselves (\textsc{3pl} intensive \linebreak pronoun, ‘\textsc{3pl-int}’) \is{intensive pronoun}\item 
\textbf{\textit{ndï}} (\textsc{pro}) they (\textsc{3pl} subject \isi{pronoun}, ‘\textsc{3pl}’; \textsc{3pl} \isi{subject marker}, ‘\textsc{3pl}’) \item 
\textbf{\textit{ndï=}} (\textsc{pro}) them (\textsc{3pl} non-subject \isi{pronoun}, ‘\textsc{3pl}’; \textsc{3pl} \isi{object marker}, ‘\textsc{3pl}’) \item 
\textbf{\textit{ndïl}} (\textsc{n}) pandanus \item 
\textbf{\textit{ndïlpot}} (\textsc{n}) type of basket \item 
\textbf{\textit{ndin}} (\textsc{dem}) those (\isi{dual} \isi{distal} \isi{demonstrative}, ‘\textsc{du.dist}’) (abbreviated form of \textbf{\textit{andin}}); (\textsc{pro?}) (possible alternative of \textsc{3du} subject \isi{pronoun} \textbf{\textit{min}}) \item 
\textbf{\textit{ndin=}} (\textsc{dem}) those (non-subject \isi{dual} \isi{distal} \isi{demonstrative}, ‘\textsc{du.dist}’) \linebreak (abbreviated form of \textbf{\textit{andin=}}); (\textsc{pro?}) (possible alternative of \textsc{3du} non-subject \isi{pronoun} \textbf{\textit{min=}}) \item 
\textbf{\textit{ndïnam}} (\textsc{pro}) they are the ones (\textsc{3pl} \isi{emphatic pronoun}, ‘\textsc{3pl-emph}’) \item 
\textbf{\textit{ndïngonim}} (\textsc{n}) insect species (brown ant) \item 
\textbf{\textit{ndïngusuwa}} (\textsc{pro}) the poor things [\textsc{pl}] (3\textsc{pl} \isi{affective pronoun}, ‘\textsc{3pl}-poor’) (also \textbf{\textit{ndïngusuwata}}) \item 
\textbf{\textit{ndïngusuwata}} (\textsc{pro}) the poor things [\textsc{pl}] (3\textsc{pl} \isi{affective pronoun}, ‘\textsc{3pl}-poor’) (also \textbf{\textit{ndïngusuwa}}) \item 
\textbf{\textit{ndïnji}} (\textsc{pro}) their, theirs (\textsc{3pl} \isi{possessive pronoun}, ‘\textsc{3pl-poss}’) \item 
\textbf{\textit{ndolum}} (\textsc{n}) bird species \item 
\textbf{\textit{ndukumbu}} (\textsc{n}) palm species (palm used in construction) \item 
\textbf{\textit{ndunduma}} (\textsc{n}) great-grandparent, ancestor; great-grandchild \item 
\textbf{\textit{nduwe}} (\textsc{pro}) themselves, they themselves, them themselves (from among \linebreak several) (\textsc{3pl} \isi{partitive-intensive pronoun}, ‘\textsc{3pl-part.int}’)\\ \item 

\noindent \textbf{<Ng, ng>      [ᵑɡ]}\\ \item 

\textbf{\textit{nga}} (\textsc{dem}) this (\isi{singular} \isi{proximal} \isi{demonstrative}, ‘\textsc{sg.prox}’) \item 
\textbf{\textit{nga=}} (\textsc{dem}) this (non-subject \isi{singular} \isi{proximal} \isi{demonstrative}, ‘\textsc{sg.prox}’) \item 
\textbf{\textit{ngala}} (\textsc{dem}) these (\isi{plural} \isi{proximal} \isi{demonstrative}, ‘\textsc{pl.prox}’) \item 
\textbf{\textit{ngala=}} (\textsc{dem}) these (non-subject \isi{plural} \isi{proximal} \isi{demonstrative}, ‘\textsc{pl.prox}’) \item 
\textbf{\textit{ngalambi}} (\textsc{dem}) as for these ones (\isi{plural} \isi{proximal} \isi{topic-marker} \isi{demonstrative}, ‘\textsc{pl.prox-top}’) \item 
\textbf{\textit{ngalanji}} (\textsc{dem}) these ones’ (\isi{plural} \isi{proximal} possessive \isi{demonstrative}, \linebreak‘\textsc{pl.prox-poss}’) \item \textbf{\textit{ngalawa}} (\textsc{dem}) these themselves (\isi{plural} \isi{proximal} \isi{intensive} \isi{demonstrative}, \linebreak‘\textsc{pl.prox-int}’) \item 
\textbf{\textit{ngalawe}} (\textsc{dem}) these themselves (from among several) (\isi{plural} \isi{proximal} \linebreak \isi{partitive-intensive} \isi{demonstrative}, ‘\textsc{pl.prox-part.int}’) \item 
\textbf{\textit{ngam}} (\textsc{dem}) this is it (\isi{singular} \isi{proximal} \isi{emphatic} \isi{demonstrative}, \linebreak‘\textsc{sg.prox-emph}’) \item 
\textbf{\textit{ngambi}} (\textsc{dem}) as for this one (\isi{singular} \isi{proximal} \isi{topic-marker} \isi{demonstrative}, ‘\textsc{sg.prox-top}’) \item 
\textbf{\textit{ngan}} (\textsc{pro}) we (\textsc{1du.excl} subject \isi{pronoun}, ‘\textsc{1du.excl}’) \item 
\textbf{\textit{ngan=}} (\textsc{pro}) us \textsc{(1du.excl} non-subject \isi{pronoun}, ‘\textsc{1du.excl}’) \item 
\textbf{\textit{nganambi}} (\textsc{pro}) as for us (\textsc{1du.excl} \isi{topic-marker pronoun}, ‘\textsc{1du.excl-top}’) \item 
\textbf{\textit{nganangan}} (\textsc{n}) betel pepper seed \item 
\textbf{\textit{nganawa}} (\textsc{pro}) we ourselves, us ourselves (\textsc{1du.excl} \isi{intensive pronoun}, \linebreak‘\textsc{1du.excl-int}’) \item 
\textbf{\textit{nganji}} (\textsc{pro}) our, ours (\textsc{1du.excl} \isi{possessive pronoun}, ‘\textsc{1du.excl-poss}’) \item 
\textbf{\textit{nganji}} (\textsc{dem}) this one’s (\isi{singular} \isi{proximal} possessive \isi{demonstrative}, \linebreak‘\textsc{sg.prox-poss}’) \item 
\textbf{\textit{nganwe}} (\textsc{pro}) we ourselves, us ourselves (from among several) (\textsc{1du.excl} \linebreak partitive-\isi{intensive pronoun}, ‘\textsc{1du.excl-part.int}’) \item 
\textbf{\textit{ngata}} (\textsc{adj}) grand, big, huge; (\textsc{n}) grandparent, old person, ancestor \item 
\textbf{\textit{ngata yawa}} (\textsc{n}) mother’s mother’s brother (maternal great-uncle) (literally \linebreak ‘great uncle’) \item 
\textbf{\textit{ngawa}} (\textsc{dem}) this itself (\isi{singular} \isi{proximal} \isi{intensive} \isi{demonstrative}, \linebreak‘\textsc{sg.prox-int}’) \item 
\textbf{\textit{ngawe}} (\textsc{dem}) this itself (from among several) (\isi{singular} \isi{proximal} \linebreak \isi{partitive-intensive} \isi{demonstrative}, ‘\textsc{sg.prox-part.int}’) \item 
\textbf{\textit{ngaya}} (\textsc{adv)} far; long (time) \item 
\textbf{\textit{ngin}} (\textsc{n}) net, fishing net; fish trap woven around a cane hoop \item 
\textbf{\textit{ngin}} (\textsc{dem}) these (\isi{dual} \isi{proximal} \isi{demonstrative}, ‘\textsc{du.prox}’) \item 
\textbf{\textit{ngin=}} (\textsc{dem}) these (non-subject \isi{dual} \isi{proximal} \isi{demonstrative}, ‘\textsc{du.prox}’) \item 
\textbf{\textit{ngïn}} (\textsc{n}) cloud (usually only as part of the \isi{compound} \textbf{\textit{apïn ngïn}} ‘smoke’) \item 
\textbf{\textit{nginambi}} (\textsc{dem}) as for these ones (\isi{dual} \isi{proximal} \isi{topic-marker} \isi{demonstrative}, ‘\textsc{du.prox-top}’) \item 
\textbf{\textit{nginawa}} (\textsc{dem}) these themselves (\isi{dual} \isi{proximal} \isi{intensive} \isi{demonstrative}, \linebreak‘\textsc{du.prox-int}’) \item 
\textbf{\textit{ngïnïm}} (\textsc{n}) chin (< \ili{Yuat}) \item 
\textbf{\textit{nginji}} (\textsc{dem}) these ones’ (\isi{dual} \isi{proximal} possessive \isi{demonstrative}, \linebreak‘\textsc{du.prox-poss}’) \item 
\textbf{\textit{nginwe}} (\textsc{dem}) these themselves (from among several) (\isi{dual} \isi{proximal} \linebreak \isi{partitive-intensive} \isi{demonstrative}, ‘\textsc{du.prox-part.int}’) \item 
\textbf{\textit{ngom lï-}} (\textsc{v}) spit (literally ‘put spit’?) \item 
\textbf{\textit{ngowil}} (\textsc{n}) insect species (black ant) \item 
\textbf{\textit{ngum}} (\textsc{n}) snake species (venomous snake that lives both in water and on land) \item 
\textbf{\textit{ngum}} (\textsc{n}) yam species (long white yam) \item 
\textbf{\textit{ngun}} (\textsc{pro}) you (\textsc{2du} subject \isi{pronoun}, ‘\textsc{2du}’) \item 
\textbf{\textit{ngun=}} (\textsc{pro}) you (\textsc{2du} non-subject \isi{pronoun}, ‘\textsc{2du}’) \item 
\textbf{\textit{nguna}} (\textsc{pro}) we (\textsc{1du.incl} subject \isi{pronoun}, ‘\textsc{1du.incl}’) (abbreviated form of \textbf{\textit{ngunan}}) \item 
\textbf{\textit{ngunambi}} (\textsc{pro}) as for you (\textsc{2du} \isi{topic-marker pronoun}, ‘\textsc{2du-top}’) \item 
\textbf{\textit{ngunan}} (\textsc{pro}) we (\textsc{1du.incl} subject \isi{pronoun}, ‘\textsc{1du.incl}’) \item 
\textbf{\textit{ngunan=}} (\textsc{pro}) us (\textsc{1du.incl} non-subject \isi{pronoun}, ‘\textsc{1du.incl}’) \item 
\textbf{\textit{ngunanambi}} (\textsc{pro}) as for us (\textsc{1du.incl} \isi{topic-marker pronoun}, ‘\textsc{1du.incl-top}’) \item 
\textbf{\textit{ngunanawa}} (\textsc{pro}) we ourselves, us ourselves (\textsc{1du.incl} \isi{intensive pronoun}, \linebreak‘\textsc{1du.incl-int}’) \item 
\textbf{\textit{ngunanji}} (\textsc{pro}) our, ours (\textsc{1du.incl} \isi{possessive pronoun}, ‘\textsc{1du.incl-poss}’) \item 
\textbf{\textit{ngunanwe}} (\textsc{pro}) we ourselves, us ourselves (from among several) (\textsc{1du.incl} \isi{partitive-intensive pronoun}, ‘\textsc{1du.incl-part.int}’) \item 
\textbf{\textit{ngunawa}} (\textsc{pro}) you yourselves (\textsc{2du} \isi{intensive pronoun}, ‘\textsc{2du-int}’) \item 
\textbf{\textit{ngungun}} (\textsc{n}) insect species (red ant) (TP \textit{karakum}); (\textsc{adj}) red \item 
\textbf{\textit{ngungun}} (\textsc{n}) plant species (plant with red seeds); (\textsc{adj}) red \item 
\textbf{\textit{ngungun}} (\textsc{n}) whirlwind, cyclone \item 
\textbf{\textit{ngungusuwa}} (\textsc{pro}) you poor things [\textsc{du}] (2\textsc{du} \isi{affective pronoun}, ‘\textsc{2du}-poor’) (also \textbf{\textit{ngungusuwata}}) \item 
\textbf{\textit{ngungusuwata}} (\textsc{pro}) you poor things [\textsc{du}] (2\textsc{du} \isi{affective pronoun}, ‘\textsc{2du}-poor’) (also \textbf{\textit{ngungusuwa}}) \item 
\textbf{\textit{ngunguswa}} (\textsc{n}) insect species (cockroach) \item 
\textbf{\textit{ngunji}} (\textsc{pro}) your, yours (\textsc{2du} \isi{possessive pronoun}, ‘\textsc{2du-poss}’) \item 
\textbf{\textit{ngunmbi}} (\textsc{n}) banana species (plantain banana plant with medium-sized fruit in large bunches) \item 
\textbf{\textit{ngunwe}} (\textsc{pro}) you yourselves (from among several) (\textsc{2du} \isi{partitive-intensive pronoun}, ‘\textsc{2du-part.int}’) \item 
\textbf{\textit{ngusuwa}} (\textsc{adj}) poor, pitiful \item 
\textbf{\textit{ngwimakan}} (\textsc{n}) black possum, black cuscus (TP \textit{black kapul})\\ \item 

\noindent \textbf{<Nj, nj>      [ⁿdʒ]}\\ \item 

\textbf{\textit{nji}} (\textsc{n}) thing, something \item 
\textbf{\textit{-nji}} [possessive \isi{suffix}, ‘\textsc{poss}’] \item 
\textbf{\textit{njimana}} (\textsc{n}) fly, housefly (literally ‘going thing’?) (cf. \textbf{\textit{manjimanji}} ‘maggot’) \item 
\textbf{\textit{njukuta}} (\textsc{adj}) small, little, thin, narrow\\ \item 

\noindent \textbf{<O, o>        [o]}\\ \item 

\textbf{\textit{o}} (\textsc{conj}) or (< \ili{Tok Pisin} \textit{o} ‘or’) \item 
\textbf{\textit{=o}} [\isi{intensifier}; \isi{vocative} \isi{enclitic}, ‘\textsc{voc}’] \item 
\textbf{\textit{oke}} (\textsc{interj}) OK, okay (expresses agreement, etc.) (< \ili{Tok Pisin} \textit{oke} ‘OK’) \item 
\textbf{\textit{-op}} [\isi{perfective} \isi{suffix}, ‘\textsc{pfv}’] (in \isi{double perfective} constructions; also \textbf{\textit{-ap}}, \textbf{\textit{-ïp}})\\ \item 

\noindent \textbf{<P, p>        [p]}\\ \item 

\textbf{\textit{p}} [\isi{epenthetic} utterance-final sound for some speakers] \item 
\textbf{\textit{p-}} (\textsc{v}) be, be at (be located at), stay, stay at, live, live at, reside, reside at, inhabit \item 
\textbf{\textit{-p}} [\isi{perfective} \isi{suffix}, ‘\textsc{pfv}’] \item 
\textbf{\textit{=p}} [\isi{copular enclitic}, ‘\textsc{cop}’] \item 
\textbf{\textit{pal}} (\textsc{n}) main shoot of a sago palm; type of beam  (horizontal beam in houses to \linebreak support the floor, made from the main shoot of the sago palm) \item 
\textbf{\textit{palam}} (\textsc{n}) cane grass (TP \textit{pitpit}) (< \ili{Mwakai}) \item 
\textbf{\textit{palapal}} (\textsc{n}) shell; ceremonial armband; money (possibly < \ili{Tok Pisin} \textit{balbal} {\textasciitilde} \textit{palpal} ‘Indian coral tree’) \item 
\textbf{\textit{palmana}} (\textsc{adj}) thick, wide \item 
\textbf{\textit{pan}} (\textsc{n}) clay \item 
\textbf{\textit{pat}} (\textsc{n}) shoot that emerges from the bulb of a yam \item 
\textbf{\textit{pawla}} (\textsc{n}) yam species (wild yam with a long bulb) \item 
\textbf{\textit{pïna}} (\textsc{v}) be, be at (be located at), stay, stay at, live, live at, reside, reside at, inhabit (\isi{irrealis} \isi{mood}) \item 
\textbf{\textit{pïsima}} (\textsc{n}) older, somewhat dry betel nut fruit (the stage following \textbf{\textit{aw ulum}} ‘young betel nut’) \item 
\textbf{\textit{Pisuwa}} [male name] \item 
\textbf{\textit{piya}} (\textsc{n}) banana species (plantain banana plant with small fruit) \item 
\textbf{\textit{Plas}} [male name] \item 
\textbf{\textit{pon}} (\textsc{adj}) dull, blunt (also \textit{\textbf{tambumana}}) \item 
\textbf{\textit{pop lï-}} (\textsc{v}) sweep (literally ‘put a sweep’?) \item 
\textbf{\textit{popo}} (\textsc{n}) papaya (< \ili{Tok Pisin} \textit{popo} ‘papaya’) \item 
\textbf{\textit{popotala}} (\textsc{n}) frog species (large brown frog) \item 
\textbf{\textit{pul}} (\textsc{n}) piece, place\\ \item 

\noindent \textbf{<S, s>        [s]}\\ \item 

\textbf{\textit{s}} (\textsc{v}) push (abbreviated form of \textbf{\textit{si-}}) \item 
\textbf{\textit{sa-}} (\textsc{v}) cry, weep \item 
\textbf{\textit{sakïma}} (\textsc{n}) adze, tool for carving canoes (< \ili{Yuat}) \item 
\textbf{\textit{sakla}} (\textsc{n}) platform; stretcher \item 
\textbf{\textit{saklup}} (\textsc{n}) broom (< \ili{Ap Ma}) \item 
\textbf{\textit{sal}} (\textsc{n}) tear, teardrop (also \textbf{\textit{lïmndï inim}}) \item 
\textbf{\textit{samban}} (\textsc{n}) pot for cooking (possibly a variant of \textbf{\textit{sïmbïn}}) \item 
\textbf{\textit{Sambome}} [male name] \item 
\textbf{\textit{sambulumbu}} (\textsc{n}) insect species (flying insect) \item 
\textbf{\textit{samnang}} (\textsc{n}) yam species (yam with pink flesh) (< \ili{Ap Ma}) \item 
\textbf{\textit{sandïpal}} (\textsc{n}) type of basket (basket made from coconut fronds) \item 
\textbf{\textit{sapos}} (\textsc{conj}) ‘if’ (< \ili{Tok Pisin} \textit{sapos} ‘if’) \item 
\textbf{\textit{sasi}} (\textsc{n}) initiation rites \item 
\textbf{\textit{sasi yena}} (\textsc{n}) type of spirit (TP \textit{masalai}) (literally ‘initiation woman’) \item 
\textbf{\textit{sawe}} [\isi{habitual} marker, ‘\textsc{hab}’] (< \ili{Tok Pisin} \textit{save} ‘know’; \isi{habitual} marker) \item 
\textbf{\textit{sawi}} (\textsc{n}) saliva, spit; magic (< \ili{Yuat}) (cf. \textbf{\textit{tawi}} ‘magic’) \item 
\textbf{\textit{si-}} (\textsc{v}) push \item 
\textbf{\textit{sikal}} (\textsc{n}) insect species (type of fly) \item 
\textbf{\textit{sikul}} (\textsc{n}) school (< \ili{Tok Pisin} \textit{skul} ‘school’) \item 
\textbf{\textit{sikulmakan ni-}} (\textsc{v}) learn (literally ‘do school’?) (\textit{sikul} is from \ili{Tok Pisin}; origin of \textit{makan} is unknown) \item 
\textbf{\textit{Simban}} [female name] \item 
\textbf{\textit{simbïli}} (\textsc{n}) lizard species (blue-and-brown-striped lizard) \item 
\textbf{\textit{sïmbïn}} (\textsc{n}) large storage pot (possibly a variant of \textbf{\textit{samban}}) \item 
\textbf{\textit{sïmin}} (\textsc{n}) louse (on animals) \item 
\textbf{\textit{simïnda}} (\textsc{n}) banana species (plantain banana plant with large bunches, second only to the \textbf{\textit{almbïne}} banana species in number of fruits) \item 
\textbf{\textit{sina}} (\textsc{n}) small young bamboo stalk; small knife made from bamboo \item 
\textbf{\textit{Sinanam}} [female name] \item 
\textbf{\textit{sinanan}} (\textsc{n}) nail, fingernail \item 
\textbf{\textit{sinananangïn}} (\textsc{n}) claw (< \textbf{\textit{sinanan}} ‘nail’ + \textbf{\textit{nangïn}} ‘tongs’) \item 
\textbf{\textit{sinangul}} (\textsc{n}) Jew’s harp, mouth harp \item 
\textbf{\textit{Sinda}} [female name] \item 
\textbf{\textit{sini-}} (\textsc{v}) play (literally ‘push-do’?) \item 
\textbf{\textit{sinokoy}} (\textsc{n}) crop \item 
\textbf{\textit{siwi}} (\textsc{n}) grub species (large edible sago grub, the larva of the \textbf{\textit{tambïn}} insect species) \item 
\textbf{\textit{sokoy}} (\textsc{n}) tobacco (\isi{areal term}) \item 
\textbf{\textit{somïn}} (\textsc{n}) fish species \item 
\textbf{\textit{sum}} (\textsc{n}) grub species (either the edible \textbf{\textit{mïnkïn}} or \textbf{\textit{siwi}} sago grub species in a slightly more mature state) \item 
\textbf{\textit{Supam}} [female name] \item 
\textbf{\textit{supangasa}} (\textsc{n}) banana species (plantain banana plant with the second smallest fruit after the \textbf{\textit{yokomakan}} banana species) \item 
\textbf{\textit{suwan}} (\textsc{n}) mesh rack made of palm fronds, used to smoke fish \item 
\textbf{\textit{Suwol}} [male name]\\ \item 

\noindent \textbf{<T, t>        [t]}\\ \item 

\textbf{\textit{t}} (\textsc{v}) say, speak, tell, talk, think (abbreviated form of \textbf{\textit{ta-}}) \item 
\textbf{\textit{t}} (\textsc{v}) take, get (abbreviated form of \textbf{\textit{tï-}}) \item 
\textbf{\textit{-t}} [\isi{speculative} \isi{suffix}, ‘\textsc{spec}’] \item 
\textbf{\textit{ta}} (\textsc{n}) type of beam (floor-supporting beam in a house) \item 
\textbf{\textit{ta}} (\textsc{adv)} already \item 
\textbf{\textit{ta-}} (\textsc{v}) say, speak, tell, talk, think \item 
\textbf{\textit{-ta}} [\isi{conditional} \isi{suffix}, ‘\textsc{cond}’] \item 
\textbf{\textit{tal}} (\textsc{n}) tail feather \item 
\textbf{\textit{Talamba}} [place] jungle region near Manu village \item 
\textbf{\textit{taman}} (\textsc{n}) type of beam (roof beam in a house that sits atop supports) \item 
\textbf{\textit{Tambana}} [female name] \item 
\textbf{\textit{tambanji}} (\textsc{n}) bird species (black, sharp-beaked bird) \item 
\textbf{\textit{tamben}} (\textsc{n}) ladder used to climb trees \item 
\textbf{\textit{tambeta}} (\textsc{n}) chest, sternum \item 
\textbf{\textit{tambïn}} (\textsc{n}) insect species (the adult form of the \textbf{\textit{siwi}} grub species) \item 
\textbf{\textit{tambïn ulum}} (\textsc{n}) sago species (tall, thin sago palm that, when fallen and dry, often contains \textbf{\textit{siwi}} grubs) (literally ‘sago grub palm’) \item 
\textbf{\textit{tambontam}} (\textsc{n}) yam species (yam with yellow-whitish skin and white flesh) \item 
\textbf{\textit{tambumana}} (\textsc{adj}) dull, blunt (also \textit{\textbf{pon}}) \item 
\textbf{\textit{tamndï}} (\textsc{n}) owner, kin, next of kin \item 
\textbf{\textit{tana}} (\textsc{n}) stone, rock; axe \item 
\textbf{\textit{tana isi}} (\textsc{n}) sand (literally ‘stone ashes’) \item 
\textbf{\textit{tanatmu}} (\textsc{n}) stone axe; axe head \item 
\textbf{\textit{tanawen}} (\textsc{n}) hoe, digging tool \item 
\textbf{\textit{tane lï-}} (\textsc{v}) stand, be standing (literally ‘put a stance’?) \item 
\textbf{\textit{tanen}} (\textsc{n}) bird species (brown, yellow-legged bird) \item 
\textbf{\textit{tangam}} (\textsc{n}) sprout, bud \item 
\textbf{\textit{Tangin}} [female name] \item 
\textbf{\textit{Tanom}} [female name] \item 
\textbf{\textit{tanum}} (\textsc{n}) lips, area around the mouth \item 
\textbf{\textit{tap}} (\textsc{adv)} maybe \item 
\textbf{\textit{Tapon}} [male name] \item 
\textbf{\textit{Tarambi}} [male name] \item 
\textbf{\textit{tasol}} (\textsc{conj}) but (< \ili{Tok Pisin} \textit{tasol} ‘but’) \item 
\textbf{\textit{tata}} (\textsc{n}) papa (a nursery term for father); the \isi{vocative} form of \textbf{\textit{itom}} ‘father’ for speakers of all ages \item 
\textbf{\textit{Taw}} [place] jungle region near Manu village \item 
\textbf{\textit{tawa}} (\textsc{n}) wound, sore \item 
\textbf{\textit{tawatal}} (\textsc{n}) scab \item 
\textbf{\textit{tawatïp}} (\textsc{n}) child \item 
\textbf{\textit{tawi}} (\textsc{n}) magic, venom (also \textbf{\textit{sawi}}) \item 
\textbf{\textit{tem}} (\textsc{n}) time; (\textsc{conj}) when, whenever (< \ili{Tok Pisin} \textit{taim} ‘time’) \item 
\textbf{\textit{tembi}} (\textsc{adj}) bad, sick, ill, poor, dirty; (\textsc{n}) badness, sickness, illness \item 
\textbf{\textit{tï inda-}} (\textsc{v}) carry (literally ‘take [and] walk’) \item 
\textbf{\textit{tï-}} (\textsc{v}) take, get \item 
\textbf{\textit{tïke}} (\textsc{adj}) small \item 
\textbf{\textit{tïki}} (\textsc{adv}) again, anymore, else \item 
\textbf{\textit{tïkli ka-}} (\textsc{v}) turn, turn around (literally ‘let turn’?) \item 
\textbf{\textit{tïl}} (\textsc{n}) husk (coconut husk), shell \item 
\textbf{\textit{tïlwa}} (\textsc{n}) road, path, trail, track (< \textbf{\textit{utï}} ‘foot’ + \textbf{\textit{luwa}} ‘place’) \item 
\textbf{\textit{tïlwa num}} (\textsc{n}) car (literally ‘road canoe’) \item 
\textbf{\textit{tïmal}} (\textsc{n}) buttress root \item 
\textbf{\textit{tïmbïl}} (\textsc{n}) fence; diaphragm \item 
\textbf{\textit{tïn}} (\textsc{n}) dog \item 
\textbf{\textit{tïnanga-}} (\textsc{v}) arise, get up, stand up \item 
\textbf{\textit{tïngïn}} (\textsc{adj}) many \item 
\textbf{\textit{tïnum}} (\textsc{n}) game, play \item 
\textbf{\textit{tïpal}} (\textsc{n}) hip \item 
\textbf{\textit{tïponïm}} (\textsc{n}) possum, cuscus (TP \textit{kapul}) \item 
\textbf{\textit{Tïponïm}} [place] section of Manu village where the school was built \item 
\textbf{\textit{Tïrïngïn}} [male name] \item 
\textbf{\textit{Tïwen}} [place] jungle region near Wopata \item 
\textbf{\textit{tomal u-}} (\textsc{v}) pour (literally ‘put a pour’?) \item 
\textbf{\textit{tomona}} (\textsc{n}) frog species (brown, sharp-nosed frog) \item 
\textbf{\textit{tomoy}} (\textsc{n}) insect species (insect that lives around hearth ashes) \item 
\textbf{\textit{tondiway}} (\textsc{n}) plant species (plant with orange seeds used to make dyes); \linebreak (\textsc{adj}) orange \item 
\textbf{\textit{tongan}} (\textsc{n}) mosquito-swatter \item 
\textbf{\textit{tongla}} (\textsc{n}) headdress \item 
\textbf{\textit{tongonat}} (\textsc{n}) frog species (small black frog) \item 
\textbf{\textit{tongonat}} (\textsc{n}) insect species (flying insect) \item 
\textbf{\textit{top lï-}} (\textsc{v}) throw (literally ‘put a throw’?) \item 
\textbf{\textit{topinka-}} (\textsc{v}) forget (literally ‘let in throw’?) \item 
\textbf{\textit{tukul}} (\textsc{n}) fish trap made from bamboo posts shaped into a vee \item 
\textbf{\textit{tukul-}} (\textsc{v}) break (\isi{intransitive}) \item 
\textbf{\textit{tul}} (\textsc{n}) bird species (crow-like black bird) \item 
\textbf{\textit{tumbu itïm}} (\textsc{n}) outhouse, toilet (\textbf{\textit{itïm}} is ‘trash’; meaning of \textit{tumbu} is unknown) \item 
\textbf{\textit{tumbunma}} (\textsc{n}) nape of the neck \item 
\textbf{\textit{tumopa}} (\textsc{n}) heap, pile \item 
\textbf{\textit{tumul ka-}} (\textsc{v}) bend (literally ‘let bend’?) \item 
\textbf{\textit{Tupuk}} [male name] \item 
\textbf{\textit{twa}} (\textsc{n}) hearth, stove\\ \item 

\noindent \textbf{<U, u>        [u]}\\ \item 

\textbf{\textit{u}} (\textsc{n}) ditch, creek \item 
\textbf{\textit{u}} (\textsc{pro}) you (2\textsc{sg} subject \isi{pronoun}, ‘\textsc{2sg}’) \item 
\textbf{\textit{u}} (\textsc{p}) from, in, at, around, along \item 
\textbf{\textit{u}} (\textsc{interj}) ooh (expresses amazement) \item 
\textbf{\textit{u-}} (\textsc{v}) put \item 
\textbf{\textit{u=}} (\textsc{pro}) you (2\textsc{sg} non-subject \isi{pronoun}, ‘\textsc{2sg}’) \item 
\textbf{\textit{u anma}} (expression) you’re welcome (literally ‘you [are] good’) \item 
\textbf{\textit{ul}} (\textsc{p}) with (\isi{comitative}) \item 
\textbf{\textit{ul ni-}} (\textsc{v}) make, force, pressure (literally ‘act with’) \item 
\textbf{\textit{ul watka-}} (\textsc{v}) float (literally ‘let atop with’) \item 
\textbf{\textit{ula-}} (\textsc{v}) weave \item 
\textbf{\textit{ulep lï-}} (\textsc{v}) jump (literally ‘put a jump’?) \item 
\textbf{\textit{ulet}} (\textsc{n}) clay bowl, dish \item 
\textbf{\textit{ulo-}} (\textsc{v}) peel (literally ‘cut from’) \item 
\textbf{\textit{ulum}} (\textsc{n}) sago palm; sago pith \item 
\textbf{\textit{ulumbi}} (\textsc{n}) taro species (wild taro) \item 
\textbf{\textit{ulwa}} (\textsc{neg}) nothing, none; (\textsc{adj}) empty \item 
\textbf{\textit{um}} (\textsc{n}) neck \item 
\textbf{\textit{uma}} (\textsc{n}) bone; fish hook; needle \item 
\textbf{\textit{umba}} (\textsc{n}) garbage heap \item 
\textbf{\textit{umbe}} (\textsc{adv)} tomorrow \item 
\textbf{\textit{umbenam}} (\textsc{n)} morning \item 
\textbf{\textit{umbenam anma}} (\isi{greeting}) good morning \item 
\textbf{\textit{umbopa}} (\textsc{n}) stomach (< \ili{Ap Ma}) \item 
\textbf{\textit{umo-}} (\textsc{v}) put (abbreviated form of \textbf{\textit{lumo-}}) \item 
\textbf{\textit{un}} (\textsc{n}) tree species (okari nut tree) (TP \textit{talis}) \item 
\textbf{\textit{un}} (\textsc{pro}) you (\textsc{2pl} subject \isi{pronoun}, ‘\textsc{2pl}’) \item 
\textbf{\textit{un=}} (\textsc{pro}) you (\textsc{2pl} non-subject \isi{pronoun}, ‘\textsc{2pl}’) \item 
\textbf{\textit{una}} (\textsc{pro}) we (\textsc{1pl.incl} subject \isi{pronoun}, ‘\textsc{1pl.incl}’) (abbreviated form of \textbf{\textit{unan}}) \item 
\textbf{\textit{unambi}} (\textsc{pro}) as for you (\textsc{2pl} \isi{topic-marker pronoun}, ‘\textsc{2pl-top}’) \item 
\largerpage
\textbf{\textit{unan}} (\textsc{pro}) we (\textsc{1pl.incl} subject \isi{pronoun}, ‘\textsc{1pl.incl}’) \item 
\textbf{\textit{unan=}} (\textsc{pro}) us (\textsc{1pl.incl} non-subject \isi{pronoun}, ‘\textsc{1pl.incl}’) \item 
\textbf{\textit{unanambi}} (\textsc{pro}) as for us (\textsc{1pl.incl} \isi{topic-marker pronoun}, ‘\textsc{1pl.incl-top}’) \item 
\textbf{\textit{unanawa}} (\textsc{pro}) we ourselves, us ourselves (\textsc{1pl.incl} \isi{intensive pronoun}, \linebreak‘\textsc{1pl.incl-int}’) \item 
\textbf{\textit{unanji}} (\textsc{pro}) our, ours (\textsc{1pl.incl} \isi{possessive pronoun}, ‘\textsc{1pl.incl-poss}’) \item 
\textbf{\textit{unanwe}} (\textsc{pro}) we ourselves, us ourselves (from among several) (\textsc{1pl.incl} \linebreak \isi{partitive-intensive pronoun}, ‘\textsc{1pl.incl-part.int}’) \item 
\textbf{\textit{unapïn}} (\textsc{n}) insect species (bee) \item 
\textbf{\textit{unawa}} (\textsc{pro}) you yourselves (\textsc{2pl} \isi{intensive pronoun}, ‘\textsc{2pl-int}’) \item 
\textbf{\textit{unda}} (\textsc{n}) enemy; vital spot, target \item 
\textbf{\textit{unda-}} (\textsc{v}) go around \item 
\textbf{\textit{unden}} (\textsc{n}) stem of the areca palm (TP \textit{buai limbum}); container for catching water from strained sago \item 
\textbf{\textit{unduwan}} (\textsc{n}) head; elder \item 
\textbf{\textit{unduwan apïn}} (\textsc{n}) headache (literally ‘head fire’) \item 
\textbf{\textit{unduwan nambï}} (\textsc{n}) scalp of the head (literally ‘head skin’) \item 
\textbf{\textit{unet}} (\textsc{n}) navel, umbilical cord \item 
\textbf{\textit{ungusuwa}} (\textsc{pro}) you poor thing (2\textsc{sg} \isi{affective pronoun}, ‘\textsc{2sg}-poor’) (also \linebreak\textbf{\textit{ungusuwata}}) \item 
\textbf{\textit{ungusuwa}} (\textsc{pro}) you poor things [\textsc{pl}] (2\textsc{pl} \isi{affective pronoun}, ‘\textsc{2pl}-poor’) (also \linebreak\textbf{\textit{ungusuwata}}) \item 
\textbf{\textit{ungusuwata}} (\textsc{pro}) you poor thing (2\textsc{sg} \isi{affective pronoun}, ‘\textsc{2sg}-poor’) (also \linebreak \textbf{\textit{ungusuwa}}) \item 
\textbf{\textit{ungusuwata}} (\textsc{pro}) you poor things [\textsc{pl}] (2\textsc{pl} \isi{affective pronoun}, ‘\textsc{2pl}-poor’) (also \textbf{\textit{ungusuwa}}) \item 
\textbf{\textit{uni-}} (\textsc{v}) shout \item 
\textbf{\textit{unji}} (\textsc{pro}) your, yours (2\textsc{sg} \isi{possessive pronoun}, ‘\textsc{2sg-poss}’) \item 
\textbf{\textit{unji}} (\textsc{pro}) your, yours (2\textsc{pl} \isi{possessive pronoun}, ‘\textsc{2pl-poss}’) \item 
\textbf{\textit{unmbï}} (\textsc{n}) buttocks; rear \item 
\textbf{\textit{unum}} (\textsc{n}) clavicle; crevice \item 
\textbf{\textit{unwe}} (\textsc{pro}) you yourselves (from among several) (\textsc{2pl} partitive-intensive \linebreak pronoun, ‘\textsc{2pl-part.int}’) \is{partitive-intensive pronoun}\item 
\textbf{\textit{upa}} (\textsc{n}) fish species (mosquitofish) (probably a variant of \textbf{\textit{upan}}) \item 
\textbf{\textit{upan}} (\textsc{n}) fish species (small fish) (probably a variant of \textbf{\textit{upa}}) \item 
\textbf{\textit{upin}} (\textsc{n}) bird species (crowned pigeon) (TP \textit{guria}) \item 
\textbf{\textit{uta}} (\textsc{n}) bird; (\textsc{num}) hundred, one hundred \item 
\textbf{\textit{uta}} (\textsc{n}) coconut shell; plate (also \textbf{\textit{wuta}}) \item 
\textbf{\textit{uta kwe}} (\textsc{num}) one hundred (= 100${\cdot}$1) \item 
\textbf{\textit{uta lele}} (\textsc{num}) three hundred (= 100${\cdot}$3) \item 
\textbf{\textit{uta nini}} (\textsc{num}) two hundred (= 100${\cdot}$2) \item 
\textbf{\textit{uta-}} (\textsc{v}) grind (coconut), rub, wipe, scoop, catch (fish) with a net \item 
\textbf{\textit{utal}} (\textsc{n}) worm \item 
\textbf{\textit{utam}} (\textsc{n}) yam \item 
\textbf{\textit{utan}} (\textsc{n}) cough, phlegm \item 
\textbf{\textit{utan uta-}} (\textsc{v}) cough (literally ‘rub a cough’?) \item 
\textbf{\textit{utï}} (\textsc{n}) leg, foot (also \textbf{\textit{wutï}}) \item 
\textbf{\textit{utï moni}} (\textsc{n}) groin (literally ‘between the legs’) \item 
\textbf{\textit{utïl}} (\textsc{n}) refuse, leftovers \item 
\textbf{\textit{uwe}} (\textsc{n}) tree species (tree whose oil is used to clean rusted metal) \item 
\largerpage
\textbf{\textit{uwe}} (\textsc{pro}) you yourself (from among several) (2\textsc{sg} \isi{partitive-intensive pronoun}, ‘\textsc{2sg-part.int}’)\\ \item 

\noindent\textbf {<W, w>      [w]}\\ \item 

\textbf{\textit{wa}} (\textsc{adv)} just, simply, without care, without reason (also \textbf{\textit{ko}}, \textbf{\textit{kwa}}) \item 
\textbf{\textit{wa}} (\textsc{n}) village (possibly < \ili{Yuat}) \item 
\textbf{\textit{waembïl}} (\textsc{adj}) white (possibly a \isi{compound} containing \textbf{\textit{we}} ‘sago starch’) \item 
\textbf{\textit{waembïl ankam}} (\textsc{n}) white person (literally ‘white person’) \item 
\textbf{\textit{waenkïn}} (\textsc{n}) plant species (plant similar to the \textbf{\textit{ankïn}} vegetable species, but with leaves with white backsides) (possibly < \textbf{\textit{we}} ‘sago starch’ + \textbf{\textit{ankïn}} ‘vegetable species’) \item 
\textbf{\textit{wakan}} (\textsc{n}) wallaby (TP \textit{sikau}) (likely an \isi{areal term}, perhaps ultimately \linebreak < \ili{Austronesian}) \item 
\textbf{\textit{wal}} (\textsc{n}) ribs \item 
\textbf{\textit{wala}} (\textsc{adv)} far; (\textsc{adj)} far-off \item 
\textbf{\textit{wala}} (\textsc{n}) rat species \item 
\textbf{\textit{wala luwa}} (\textsc{n}) far-off place (literally ‘far-off place’) \item 
\textbf{\textit{wala uta}} (\textsc{n}) bat (literally ‘rat bird’) \item 
\textbf{\textit{wali}} (\textsc{n}) frog species (small green, yellow, or brown frog) \item 
\textbf{\textit{wali-}} (\textsc{v}) hit, stab, shoot; kill \item 
\textbf{\textit{walimot}} (\textsc{n}) bird species (dove, pigeon) (TP \textit{balus}) (< \ili{Yuat}) \item 
\textbf{\textit{wam}} (\textsc{n}) bark strap used for climbing palms \item 
\textbf{\textit{wambana}} (\textsc{n}) fish (possibly < \ili{Mwakai}) \item 
\textbf{\textit{wambi}} (\textsc{pro}) as for you (2\textsc{sg} \isi{topic-marker pronoun}, ‘\textsc{2sg-top}’) \item 
\textbf{\textit{wambïn}} (\textsc{n}) nut species (small green nut that is chewed) \item 
\textbf{\textit{wan}} (\textsc{n}) sago shoot; sago frond stalk \item 
\textbf{\textit{wan}} (\textsc{p}) over, above \item 
\textbf{\textit{wana}} [prohibitive marker, ‘\textsc{proh}’] (also \textbf{\textit{wanap}}) \item 
\textbf{\textit{wana-}} (\textsc{v}) cook \item 
\textbf{\textit{wana-}} (\textsc{v}) feel, taste, sense, perceive; think \item 
\textbf{\textit{wanakï-}} (\textsc{v}) call (literally ‘feel-say’) \item 
\textbf{\textit{wanam}} (\textsc{n}) side; wooden shield \item 
\textbf{\textit{wanamba}} (\textsc{n}) armpit \item 
\textbf{\textit{wananum}} (\textsc{adj}) hot, warm \item 
\textbf{\textit{wanap}} [prohibitive marker, ‘\textsc{proh}’] (also \textbf{\textit{wana}}) \item 
\textbf{\textit{wanawni-}} (\textsc{v}) call, summon (literally ‘feel-shout’) \item 
\textbf{\textit{wandam}} (\textsc{n}) jungle, woods, forest, bush; garden \item 
\textbf{\textit{wandana}} (\textsc{n}) vegetable species (curry-flavored vegetable, used for treating \linebreak coughs) \item 
\textbf{\textit{wandapata}} (\textsc{n}) fallow garden (< \textbf{\textit{wandam}} ‘garden’+ \textbf{\textit{wapata}} ‘old, dry’) \item 
\textbf{\textit{wandïl}} (\textsc{n}) bird species \item 
\textbf{\textit{wandïwandï}} (\textsc{n}) frog species (small brown frog) \item 
\textbf{\textit{Wangasa}} [male name] \item 
\textbf{\textit{wanmbi}} (\textsc{n}) betel pepper (TP \textit{daka}) \item 
\textbf{\textit{wanmbi mutam}} (\textsc{n}) betel pepper vine (TP \textit{rop daka}) (literally ‘betel pepper back’) \item 
\textbf{\textit{wanmbi wapa}} (\textsc{n}) betel pepper leaf (TP \textit{lip daka}) (literally ‘betel pepper leaf’) \item 
\textbf{\textit{wanwane}} (\textsc{n}) mushroom; patrol officer, district officer (TP \textit{kiap}), police officer \item 
\textbf{\textit{wap}} (\textsc{v}) be, be at (be located at), stay, stay at, live, live at, reside, reside at, inhabit (\isi{past} \isi{tense}) \item 
\textbf{\textit{wapa}} (\textsc{n}) leaf \item 
\textbf{\textit{wapa}} (\textsc{n}) wing \item 
\textbf{\textit{wapal}} (\textsc{n}) insect species (caterpillar) \item 
\textbf{\textit{wapata}} (\textsc{adj}) old, dry \item 
\textbf{\textit{wasi}} (\textsc{n}) tree species (tree whose seeds are used to repel cockroaches) \item 
\textbf{\textit{wat}} (\textsc{n}) ladder, log with steps carved into it leading to a stilted home \item 
\textbf{\textit{wat}} (\textsc{p}) atop, onto; (\textsc{n}) top \item 
\textbf{\textit{watangïn}} (\textsc{n}) last, final; last bunch (of bananas) to emerge \item 
\textbf{\textit{watangïnila}} (\textsc{num}) four (literally ‘last frond’) \item 
\textbf{\textit{watlo-}} (\textsc{v}) clear, cut down (literally ‘cut atop’) \item 
\textbf{\textit{wawa}} (\textsc{pro}) you yourself (2\textsc{sg} \isi{intensive pronoun}, ‘\textsc{2sg-int}’) \item 
\textbf{\textit{wawal}} (\textsc{n}) hive (for ants, bees, etc.) \item 
\textbf{\textit{wawana}} (\textsc{n}) plant species (plant with fruit eaten by flying foxes) \item 
\textbf{\textit{wawat}} (\textsc{n}) segment (as between joints in a sugarcane) \item 
\textbf{\textit{way}} (\textsc{n}) turtle (< \ili{Ap Ma}) \item 
\textbf{\textit{way sokoy}} (\textsc{n}) tobacco species (tobacco with short, oval-shaped leaves) (literally ‘turtle tobacco’) \item 
\textbf{\textit{we}} (\textsc{n}) sago starch, sago flour, fresh sago; sago pancake \item 
\textbf{\textit{we}} (\textsc{adv}) alone, only \item 
\textbf{\textit{we}} (\textsc{conj}) then, and then \item 
\textbf{\textit{-we}} [\isi{partitive-intensive} \isi{suffix}, ‘\textsc{part.int}’] \item 
\textbf{\textit{we nangïn}} (\textsc{n}) small cooking tongs (literally ‘sago tongs’) \item 
\textbf{\textit{we u-}} (\textsc{v}) cut (literally ‘put a cut’?) \item 
\textbf{\textit{Wekumba}} [male name] \item 
\textbf{\textit{welo-}} (\textsc{v}) box (as one’s ears) (literally ‘cut a cut’?) \item 
\textbf{\textit{wema}} (\textsc{n}) palm frond used for weaving (TP \textit{pangal}) \item 
\textbf{\textit{wemali}} (\textsc{n}) large pot for stirring sago (< \ili{Ambakich}) \item 
\textbf{\textit{wemana}} (\textsc{n}) lizard species (small colorful gecko) \item 
\textbf{\textit{wen}} (\textsc{n}) handle (as of a pick-axe) \item 
\textbf{\textit{wenta}} (\textsc{n}) bird species (small black bird whose call is believed to announce a visitor’s arrival) \item 
\textbf{\textit{wepal}} (\textsc{n}) dry, dead sago palm \item 
\textbf{\textit{wewun}} (\textsc{n}) clay pot for storing dry sago starch \item 
\textbf{\textit{wi}} (\textsc{n}) name \item 
\textbf{\textit{wipam}} (\textsc{n}) arrow, arrowhead; bullet \item 
\textbf{\textit{wiwila}} (\textsc{adj}) light (not heavy) \item 
\textbf{\textit{wiwina-}} (\textsc{v}) fly \item 
\textbf{\textit{wo}} (\textsc{adv}) very own (used with possessives) \item 
\textbf{\textit{wo-}} (\textsc{v}) burn (\isi{intransitive}), blaze; swell \item 
\textbf{\textit{wo-}} (\textsc{v}) sleep \item 
\textbf{\textit{=wo}} [\isi{intensifier}; \isi{vocative} \isi{enclitic}, ‘\textsc{voc}’] (\isi{allomorph} of \textbf{\textit{=o}}) \item 
\textbf{\textit{woka}} (\textsc{n}) banana flower \item 
\textbf{\textit{wokïn}} (\textsc{n}) big man, important person \item 
\textbf{\textit{wokomana}} (\textsc{n}) conch shell \item 
\textbf{\textit{wokomana}} (\textsc{n}) plant species (orchid with large leaves) \item 
\textbf{\textit{wol}} (\textsc{n}) breast \item 
\textbf{\textit{wol mïnda}} (\textsc{n}) banana species (alternative name for \textbf{\textit{wowi mïnda}}) (literally ‘breast banana’) \item 
\textbf{\textit{wol mïndam}} (\textsc{n}) milk (literally ‘breast pus’) \item 
\textbf{\textit{wolka}} (\textsc{adv)} again, in turn \item 
\textbf{\textit{wolmu}} (\textsc{n}) nipple (literally ‘breast fruit’) \item 
\textbf{\textit{wolname}} (\textsc{n}) tadpole, larval frog \item 
\textbf{\textit{womba}} (\textsc{n}) lizard species (large brown lizard) \item 
\textbf{\textit{womba}} (\textsc{n}) tree species (tree whose sap is drunk to treat illness) \item 
\textbf{\textit{wombam}} (\textsc{n}) middle \item 
\textbf{\textit{wombasa}} (\textsc{n}) clay pot, clay pan \item 
\textbf{\textit{wombasa anga}} (\textsc{n}) money (literally ‘piece of clay pot’) \item 
\textbf{\textit{Wombasame}} [male name] \item 
\textbf{\textit{wombïn}} (\textsc{n}) work, job, task, activity \item 
\textbf{\textit{wombïn ni-}} (\textsc{v}) work (literally ‘do work’) \item 
\textbf{\textit{wombulalaw}} (\textsc{n}) bird species (kingfisher) \item 
\textbf{\textit{wome}} (\textsc{n}) middle, trunk \item 
\textbf{\textit{Womel}} [male name] \item 
\textbf{\textit{womotana}} (\textsc{n}) frog \item 
\textbf{\textit{won}} (\textsc{n}) penis \item 
\textbf{\textit{won inim}} (\textsc{n}) semen (literally ‘penis water’) \item 
\textbf{\textit{won-}} (\textsc{v}) cut, cross \item 
\textbf{\textit{wondi}} (\textsc{n}) bandicoot (TP \textit{mumut}) \item 
\textbf{\textit{wongïta}} (\textsc{n}) bow, bow and arrow \item 
\textbf{\textit{wonglin}} (\textsc{n}) cup, ladle for hot water used in making jellied sago (< \ili{Ap Ma}) \item 
\textbf{\textit{Woni}} [female name] \item 
\textbf{\textit{wonmbi}} (\textsc{n}) tusk (of a boar) \item 
\textbf{\textit{Wonmelma}} [male name] \item 
\textbf{\textit{wonmi}} (\textsc{n}) hair \item 
\textbf{\textit{wop}} (\textsc{adv}) the next day (probably < \textbf{\textit{wo-}} ‘sleep’) \item 
\textbf{\textit{wopa}} (\textsc{quant}) all; everything, everyone; (\textsc{adj}) whole, entire, full \item 
\textbf{\textit{wopana}} (\textsc{n}) type of skirt (waist skirt) \item 
\textbf{\textit{Wopata}} [place] site of the fourth Manu village, still used as a hunting campsite \item 
\textbf{\textit{wopaw}} (\textsc{n}) ball; (\textsc{adj}) round \item 
\textbf{\textit{woplota}} (\textsc{n}) lungs (possibly < \ili{Ap Ma}) \item 
\textbf{\textit{wot}} (\textsc{n}) younger; younger sibling \item 
\textbf{\textit{wot inga yena}} (\textsc{n}) younger brother’s wife (sister-in-law) (literally ‘younger affine woman’) \item
\textbf{\textit{wot yana}} (\textsc{n}) younger sister (literally ‘younger woman’) (also \textbf{\textit{wot yena}}) \item
\textbf{\textit{wot yata}} (\textsc{n}) younger brother (literally ‘younger man’) (also \textbf{\textit{wot yeta}}) \item 
\textbf{\textit{wot yena}} (\textsc{n}) younger sister (literally ‘younger woman’) (also \textbf{\textit{wot yana}}) \item 
\textbf{\textit{wot yena numan}} (\textsc{n}) younger sister’s husband (brother-in-law) (literally \linebreak ‘younger woman husband’) \item 
\textbf{\textit{wot yeta}} (\textsc{n}) younger brother (literally ‘younger man’) (also \textbf{\textit{wot yata}}) \item 
\textbf{\textit{wotnya}} (\textsc{n}) bird species (type of black bird) \item 
\textbf{\textit{wow}} (\textsc{v}) sleep (\isi{imperfective} form of \textbf{\textit{wo-}}) \item 
\textbf{\textit{wowal}} (\textsc{n}) chicken \item 
\textbf{\textit{wowane}} (\textsc{n}) feathers worn ceremonially \item 
\textbf{\textit{wowaw}} (\textsc{n}) fish species (rainbow fish); fish scale \item 
\textbf{\textit{wowi mïnda}} (\textsc{n}) banana species (banana plant with the largest fruit of all, \linebreak traditionally eaten only by men and used to make \textbf{\textit{yamkwe}} ‘sago fried with banana and coconut’) (\textbf{\textit{mïnda}} is ‘banana’; meaning of \textit{wowi} is unknown) (also \textbf{\textit{wol mïnda}} \item 
\textbf{\textit{woyambïn}} (\textsc{adv)} pointlessly, fruitlessly \item 
\textbf{\textit{wulïn u-}} (\textsc{v}) rest, relax, pause (literally ‘put a rest’?) \item 
\textbf{\textit{wulis}} (\textsc{n}) platform; raft \item 
\textbf{\textit{wun}} (\textsc{n}) fan \item 
\textbf{\textit{wusim}} (\textsc{n}) crocodile (possibly < \ili{Yuat}) \item 
\textbf{\textit{Wusimali}} [male name] \item 
\textbf{\textit{wusimi}} (\textsc{n}) bamboo panpipes \item 
\textbf{\textit{wuta}} (\textsc{n}) coconut shell, plate (also \textbf{\textit{uta}}) \item 
\textbf{\textit{wutï}} (\textsc{n}) leg, foot (also \textbf{\textit{utï}}) \item 
\textbf{\textit{wutï ambatïm}} (\textsc{n}) knee (literally ‘leg joint’) \item 
\textbf{\textit{wutï anmot}} (\textsc{n}) shin, lower leg (literally ‘leg post’) \item 
\textbf{\textit{wutï awi}} (\textsc{n}) ankle (joint) (literally ‘side of foot’) \item 
\textbf{\textit{wutï lïmndï}} (\textsc{n}) anklebone (literally ‘foot eye’) \item 
\textbf{\textit{wutï mutam}} (\textsc{n}) top of the foot (literally ‘foot back’) \item 
\textbf{\textit{wutï name}} (\textsc{n}) thigh, upper leg, lap (cf. \textbf{\textit{i name}} ‘upper arm’, \textbf{\textit{lam}} ‘muscle’) \item 
\textbf{\textit{wutï sinanan}} (\textsc{n}) toenail (literally ‘foot nail’) \item 
\textbf{\textit{wutï yïwa}} (\textsc{n}) footprint (literally ‘foot hill’) \item 
\textbf{\textit{wutï yombam}} (\textsc{n}) sole of the foot (literally ‘foot palm’) \item 
\textbf{\textit{wutïmu}} (\textsc{n}) toe (literally ‘foot fruit’) \item 
\textbf{\textit{wutïmu ankam}} (\textsc{n}) second toe (literally ‘person toe’) \item 
\textbf{\textit{wutïmu law}} (\textsc{n}) fourth toe (literally ‘cordyline toe’) \item 
\textbf{\textit{wutïmu unduwan}} (\textsc{n}) big toe (literally ‘head toe’) \item 
\textbf{\textit{wutïmu watangïn}} (\textsc{n}) pinky toe, little toe (literally ‘last toe’) \item 
\textbf{\textit{wutïmu wome}} (\textsc{n}) middle toe (literally ‘middle toe’) \item 
\textbf{\textit{wutïni-}} (\textsc{v}) dance (literally ‘beat leg’) \item 
\textbf{\textit{wutïnpu}} (\textsc{n}) heel of the foot (cf. \textbf{\textit{inpu}} ‘elbow’, \textbf{\textit{akunpu}} ‘back of the skull’) \item 
\textbf{\textit{wutïwutï}} (\textsc{n}) bird species (duck) (possibly ‘foot-foot’) \item 
\textbf{\textit{wutota}} (\textsc{adj}) tall, long\\ \item 

\noindent\textbf{<Y, y>        [j]}\\ \item 

\textbf{\textit{ya}} (\textsc{n}) coconut; egg white \item 
\textbf{\textit{ya inim}} (\textsc{n}) coconut water (literally ‘coconut water’) \item 
\textbf{\textit{ya wapata}} (\textsc{n}) mature coconut fruit (older than an \textbf{\textit{alalama}} ‘maturing coconut’) (literally ‘dry coconut’) \item 
\textbf{\textit{Yaka}} [female name] \item 
\textbf{\textit{yakal}} (\textsc{n}) insect species (edible, black-and-yellow, caterpillar-like insect) \item 
\textbf{\textit{yakal inom}} (\textsc{n}) bird species (\textbf{\textit{inom}} is ‘mother’; relationship, if any, to \textbf{\textit{yakal}} ‘insect species’ is unknown) \item 
\textbf{\textit{yakam}} (\textsc{n}) shoe made from sago fronds \item 
\textbf{\textit{yakeka}} (\textsc{n}) bean \item 
\textbf{\textit{Yalamba}} [place] Korokopa village \item 
\textbf{\textit{yalum}} (\textsc{n}) grandchild, grandson, granddaughter (< \ili{Ap Ma}) \item 
\textbf{\textit{Yaluwa}} [male name] \item 
\textbf{\textit{yamangla}} (\textsc{n}) cloth-like part of the coconut tree bark \item 
\textbf{\textit{yamanyawi}} (\textsc{n}) bird species (bird of paradise) (TP \textit{kumul}) \item 
\textbf{\textit{yambalpa}} (\textsc{n}) type of spirit (devil-like spirit in the form of a man) \item 
\textbf{\textit{yambi}} (\textsc{n}) tree species (tall, white tree) \item 
\textbf{\textit{Yambin}} [female name] \item 
\textbf{\textit{yambïpal}} (\textsc{n}) insect species (centipede) \item 
\textbf{\textit{yambisa}} (\textsc{n}) yam species (large white, soft yam) \item 
\textbf{\textit{Yambït}} [female name] \item 
\textbf{\textit{Yambiwa}} [place] upstream half of the old \textbf{\textit{Wopata}} village \item 
\textbf{\textit{yambola}} (\textsc{n}) rash, scabies \item 
\textbf{\textit{Yambul}} [place] site of the second Manu village, near present-day Maruat, \linebreak Dimiri, and Yaul villages \item 
\textbf{\textit{yami}} (\textsc{n}) bird species \item 
\textbf{\textit{yami}} (\textsc{n}) insect species \item 
\textbf{\textit{yamkwe}} (\textsc{n}) sago fried with banana and coconut \item 
\textbf{\textit{yana}} (\textsc{n}) woman, wife, female (also \textbf{\textit{yena}}) \item 
\textbf{\textit{yanalum}} (\textsc{n}) daughter, girl (also \textbf{\textit{yenalum}}) (literally ‘female child’) \item 
\textbf{\textit{yananu}} (\textsc{n}) woman, wife (also \textbf{\textit{yenanu}}) \item 
\textbf{\textit{Yanapi}} [female name] \item 
\textbf{\textit{yanat}} (\textsc{n}) daughter (also \textbf{\textit{yenat}}) \item 
\textbf{\textit{yanaw}} (\textsc{n}) wrist \item 
\textbf{\textit{yangïmot}} (\textsc{adj}) tasty, sweet \item 
\textbf{\textit{yangle}} (\textsc{adj}) strong \item 
\textbf{\textit{yangun}} (\textsc{n}) mosquito \item 
\textbf{\textit{yangusole}} (\textsc{n}) plant species (green stinging nettle) (TP \textit{salat}) \item 
\textbf{\textit{yanïmana}} (\textsc{n}) plant species (plant with round leaves, used to perfume the body during dances) \item 
\textbf{\textit{Yaruwa}} [male name] \item 
\textbf{\textit{yata}} (\textsc{n}) man, male; brother (said only by women) (also \textbf{\textit{yeta}}) \item 
\textbf{\textit{yatalum}} (\textsc{n}) son, boy (also \textbf{\textit{yetalum}}) (literally ‘male child’) \item 
\textbf{\textit{yawa}} (\textsc{n}) mother’s brother (maternal uncle) (TP \textit{kandere}) \item 
\textbf{\textit{yawa}} (\textsc{n}) sago strainer \item 
\textbf{\textit{yawa ambi}} (\textsc{n}) mother’s older brother (maternal uncle) (also \textbf{\textit{yawa atuma}}) 
\linebreak (literally ‘big uncle’) \item 
\textbf{\textit{yawa atuma}} (\textsc{n}) mother’s older brother (maternal uncle) (also \textbf{\textit{yawa ambi}}) \linebreak (literally ‘older brother uncle’) \item 
\textbf{\textit{yawa wot}} (\textsc{n}) mother’s younger brother (maternal uncle) (literally ‘younger uncle’) \item 
\textbf{\textit{Yawana}} [female name] \item 
\textbf{\textit{Yawat}} [male name] \item 
\textbf{\textit{yawatalin}} (\textsc{n}) fish species (small eel) \item 
\textbf{\textit{yawe}} (\textsc{n}) sago pancake cooked with coconut (literally ‘coconut sago’) \item 
\textbf{\textit{yawïl}} (\textsc{n}) full moon (literally ‘coconut moon’) \item 
\textbf{\textit{yawïn}} (\textsc{n}) sugar glider \item 
\textbf{\textit{yawt}} (\textsc{n}) machete, knife (also \textbf{\textit{yot}}) \item 
\textbf{\textit{yena}} (\textsc{n}) woman, wife, female (also \textbf{\textit{yana}}) \item 
\textbf{\textit{yena utam}} (\textsc{n}) yam species. (class of yam varieties with spines) (TP \textit{mami}) (literally ‘female yam’) \item 
\textbf{\textit{yenalum}} (\textsc{n}) daughter, girl (also \textbf{\textit{yanalum}}) (literally ‘female child’) \item 
\textbf{\textit{yenanu}} (\textsc{n}) woman, wife (also \textbf{\textit{yananu}}) \item 
\textbf{\textit{yenat}} (\textsc{n}) daughter (also \textbf{\textit{yanat}}) \item 
\textbf{\textit{yeta}} (\textsc{n}) man, male; brother (said only by women) (also \textbf{\textit{yata}}) \item 
\textbf{\textit{yeta utam}} (\textsc{n}) yam species. (class of yam varieties without spines) (literally ‘male yam’) \item 
\textbf{\textit{yetalum}} (\textsc{n}) son, boy (also \textbf{\textit{yatalum}}) (literally ‘male child’) \item 
\textbf{\textit{Yetani}} [place] Yamen village \item 
\textbf{\textit{yïwa}} (\textsc{n}) mound (as for planting yams), hill \item 
\textbf{\textit{yokam}} (\textsc{n}) arrow shaft \item 
\textbf{\textit{yokomakan}} (\textsc{n}) banana species (plantain banana plant with the smallest fruit of all) \item 
\textbf{\textit{yokomakan}} (\textsc{n}) bird species (small wildfowl) (possibly < \ili{Ap Ma}) \item 
\textbf{\textit{Yokombla}} [male name] \item 
\textbf{\textit{yokomtïn}} (\textsc{n}) wildfowl egg (< \textbf{\textit{yokomakan}} ‘wildfowl’ + \textbf{\textit{mïtïn}} ‘egg’) \item 
\textbf{\textit{Yolomban}} [male name] \item 
\textbf{\textit{yom}} (\textsc{n}) heart \item 
\textbf{\textit{yoma}} (\textsc{n}) snake species (brown snake) \item 
\textbf{\textit{yomal}} (\textsc{n}) vegetable species (TP \textit{aibika}) \item 
\textbf{\textit{Yomali}} [male name] \item 
\textbf{\textit{yomba}} (\textsc{n}) vegetable species (Indian coral tree) (\textit{Erythrina variegata}) (TP \textit{balbal}) \item 
\textbf{\textit{yombam}} (\textsc{n}) palm of the hand (possibly < \textbf{\textit{i}} ‘hand’ + \textbf{\textit{wombam}} ‘middle’) (also \linebreak \textbf{\textit{i mwa}}) \item 
\textbf{\textit{yopa}} (\textsc{n}) bird species (cockatoo) (TP \textit{koki}); peace, peace treaty \item 
\textbf{\textit{yot}} (\textsc{n}) machete, knife (also \textbf{\textit{yawt}}) \item 
\textbf{\textit{yuname}} (\textsc{n}) bird species (small brown bird that sings in the morning)
\end{enumerate}

\newpage

\section{English-to-Ulwa finder list}\label{sec:17.2}

The following finder list provides translations from \ili{English} to Ulwa. It is organized alphabetically by the basic \ili{English} translations for words in the Ulwa \isi{lexicon}. It is intended to be used as a general guide and is by no means exhaustive. More detailed definitions of Ulwa words are provided in \sectref{sec:17.1}.\\


\begin{enumerate}[noitemsep, label={}, align=left, widest=190, labelsep=1ex,leftmargin=*,itemindent=-10pt]

\item
\noindent \textbf{A – a}\\ \item

‘a’ \textbf{\textit{ko=}} \item
‘a little’ \textbf{\textit{ilumka}} \item
‘about to, be’ \textbf{\textit{layk}} \item
‘above’ \textbf{\textit{wan}} \item
‘abscess’ \textbf{\textit{kananum}} \item
‘act’ \textit{v.} \textbf{\textit{ni-}} \item
‘activity’ \textbf{\textit{wombïn}} \item
‘adze’ \textbf{\textit{sakïma}} \item
‘affine’ \textbf{\textit{inga}} \item
‘afraid’ \textbf{\textit{namna}} \item
‘afraid, be’ \textit{v.} \textbf{\textit{namnap}} \item
‘after’ \textbf{\textit{angani}}, \textbf{\textit{anganika}}, \textbf{\textit{naka}} \item
‘afternoon’ \textbf{\textit{awal}}, \textbf{\textit{awal nambï}} \item
‘afterwards’ \textbf{\textit{anganika}}, \textbf{\textit{naka}} \item
‘again’ \textbf{\textit{tïki}}, \textbf{\textit{wolka}} \item
‘ah’ \textbf{\textit{a}}, \textbf{\textit{aya}} \item
‘\textit{aibika}’ \textbf{\textit{yomal}} \item
‘airplane’ \textbf{\textit{mbalus}} \item
‘alas’ \textbf{\textit{i}} \item
‘alcohol’ \textbf{\textit{inim tembi}} \item
‘alive’ \textbf{\textit{akïnaka}} \item
‘all’ \textbf{\textit{wopa}} \item
‘allow’ \textit{v.} \textbf{\textit{ka-}}, \textbf{\textit{laka-}} \item
‘alone’ \textbf{\textit{we}} \item
‘along’ \textbf{\textit{u}} \item
‘already’ \textbf{\textit{ta}} \item
‘also’ \textbf{\textit{luke}}, \textbf{\textit{maweka}}, \textbf{\textit{moweka}} \item
‘although’ \textbf{\textit{maski}} \item
‘always’ \textbf{\textit{nunu ika}} \item
‘amaranth’ \textbf{\textit{mambun}} \item
‘among’ \textbf{\textit{moni}} \item
‘an’ \textbf{\textit{ko=}} \item
‘ancestor’ \textbf{\textit{ndunduma}}, \textbf{\textit{ngata}} \item
‘and’ \textbf{\textit{ma}}, \textbf{\textit{na}} \item
‘and then’ \textbf{\textit{we}} \item
‘angry’ \textbf{\textit{matamal}} \item
‘animal’ \textbf{\textit{mundu}} \item
‘animal trap’ \textbf{\textit{asiya}} \item
‘ankle’ \textbf{\textit{wutï awi}}, \textbf{\textit{wutï lïmndï}} \item
‘anklebone’ \textbf{\textit{wutï lïmndï}} \item
‘another’ \textbf{\textit{kwa}} \item
‘ant species’ \textbf{\textit{iwanal}}, \textbf{\textit{katmombe}}, \textbf{\textit{kïka}}, \textbf{\textit{muna}}, \textbf{\textit{ndïngonim}}, \textbf{\textit{ngowil}}, \textbf{\textit{ngungun}} \item
‘anus’ \textbf{\textit{atal}} \item
‘anymore’ \textbf{\textit{tïki}} \item
‘anything’ \textbf{\textit{angos}} \item
‘areca nut’ \textbf{\textit{aw}} \item
‘areca palm’ \textbf{\textit{aw}} \item
‘arise’ \textit{v.} \textbf{\textit{tïnanga-}} \item
‘arm’ \textbf{\textit{i}} \item
‘armband’ \textbf{\textit{inamba}}, \textbf{\textit{min}}, \textbf{\textit{palapal}} \item
‘armpit’ \textbf{\textit{wanamba}} \item
‘around’ \textbf{\textit{u}} \item
‘arrange’ \textit{v.} \textbf{\textit{misisina-}} \item
‘arrow’ \textbf{\textit{nap}}, \textbf{\textit{wipam}} \item
‘arrow shaft’ \textbf{\textit{yokam}} \item
‘arrowhead’ \textbf{\textit{wipam}} \item
‘as for her’ \textbf{\textit{mambi}} \item
‘as for him’ \textbf{\textit{mambi}} \item
‘as for it’ \textbf{\textit{mambi}} \item
‘as for me’ \textbf{\textit{nambi}} \item
‘as for that one’ \textbf{\textit{andambi}} \item
‘as for them’ \textbf{\textit{minambi}}, \textbf{\textit{ndambi}} \item
‘as for these ones’ \textbf{\textit{ngalambi}}, \textbf{\textit{nginambi}} \item
‘as for this one’ \textbf{\textit{ngambi}} \item
‘as for those ones’ \textbf{\textit{alambi}}, \textbf{\textit{andinambi}} \item
‘as for us’ \textbf{\textit{anambi}}, \textbf{\textit{nganambi}}, \textbf{\textit{ngunanambi}}, \textbf{\textit{unanambi}} \item
‘as for you’ \textbf{\textit{ngunamb}}, \textbf{\textit{unambi}}, \textbf{\textit{wambi}} \item
‘ash’ \textbf{\textit{apïnsi}}, \textbf{\textit{isi}} \item
‘ashes’ \textbf{\textit{apïnsi}}, \textbf{\textit{isi}} \item
‘ask’ \textit{v.} \textbf{\textit{atwana kï-}}, \textbf{\textit{atwana ta-}} \item
‘assemble’ \textit{v.} \textbf{\textit{kuk u-}} \item
‘at’ \textbf{\textit{ka}}, \textbf{\textit{u}} \item
‘at, be’ \textit{v.} \textbf{\textit{p-}}, \textbf{\textit{wap}} \item
‘atop’ \textbf{\textit{wat}} \item
‘aunt’ \textbf{\textit{ane inom}}, \textbf{\textit{ane inom atana}}, \textbf{\textit{ane inom wot}}, \textbf{\textit{ansi inom}}, \textbf{\textit{inom}}, \textbf{\textit{inom atana}}, \textbf{\textit{inom wot}} \item
‘\textit{aupa}’ \textbf{\textit{mambun}} \item
‘avoid’ \textit{v.} \textbf{\textit{kamb-}} \item
‘awaiting’ \textbf{\textit{andïla}}, \textbf{\textit{angla}} \item
‘awning’ \textbf{\textit{apa mot}} \item
‘axe’ \textbf{\textit{tana}}, \textbf{\textit{tanatmu}} \item
‘axe head’ \textbf{\textit{tanatmu}} \item
‘ay’ \textbf{\textit{ay}}, \textbf{\textit{e}}\\ \item

\noindent \textbf{B – b}\\ \item

‘baby’ \textbf{\textit{alum}} \item
‘back’ \textbf{\textit{angani}}, \textbf{\textit{mutam}} \item
‘back of the hand’ \textbf{\textit{i mutam}} \item
‘backbone’ \textbf{\textit{anangum}}, \textbf{\textit{mutoma}} \item
‘bad’ \textbf{\textit{tembi}} \item
‘badness’ \textbf{\textit{tembi}} \item
‘bag’ \textbf{\textit{ni}} \item
‘\textit{balbal}’ \textbf{\textit{yomba}} \item
‘ball’ \textbf{\textit{wopaw}} \item
‘\textit{balus}’ \textbf{\textit{walimot}} \item
‘bamboo species’ \textbf{\textit{itenmbu}}, \textbf{\textit{mota}}, \textbf{\textit{sina}} \item
‘banana’ \textbf{\textit{mïnda}} \item
‘banana flower’ \textbf{\textit{woka}} \item
‘banana leaf’ \textbf{\textit{mïndapan}} \item
‘banana species’ \textbf{\textit{almbïne}}, \textbf{\textit{ane mongi}}, \textbf{\textit{apïn mïnda}}, \textbf{\textit{kïtïmngïle}}, \textbf{\textit{mongi}}, \linebreak \textbf{\textit{nanïwe}}, \textbf{\textit{ngunmbi}}, \textbf{\textit{piya}}, \textbf{\textit{simïnda}}, \textbf{\textit{supangasa}}, \textbf{\textit{wol mïnda}}, \textbf{\textit{wowi mïnda}}, \linebreak \textbf{\textit{yokomakan}} \item
‘band’ \textbf{\textit{inamba}}, \textbf{\textit{min}}, \textbf{\textit{palapal}} \item
‘bandicoot’ \textbf{\textit{wondi}} \item
‘bank’ \textbf{\textit{ika}} \item
‘bark’ \textbf{\textit{im nambï}}, \textbf{\textit{yamangla}} \item
‘bark strap’ \textbf{\textit{wam}} \item
‘base’ \textbf{\textit{nupu}} \item
‘basket, type of’ \textbf{\textit{akum}}, \textbf{\textit{ame}}, \textbf{\textit{imnde}}, \textbf{\textit{iwa}}, \textbf{\textit{kukul}}, \textbf{\textit{ndïlpot}}, \textbf{\textit{sandïpal}} \item
‘bat’ \textbf{\textit{nïplopa}}, \textbf{\textit{wala uta}} \item
‘bathe’ \textit{v.} \textbf{\textit{lopo-}} \item
‘battle’ \textbf{\textit{at}} \item
‘battle’ \textit{v.} \textbf{\textit{amblawali-}} \item
‘be’ \textit{v.} \textbf{\textit{p-}}, \textbf{\textit{wap}} \item
‘beak’ \textbf{\textit{nokal}} \item
‘beam, type of’ \textbf{\textit{al}}, \textbf{\textit{alwoma}}, \textbf{\textit{iwal}}, \textbf{\textit{kukun}}, \textbf{\textit{pal}}, \textbf{\textit{ta}}, \textbf{\textit{taman}} \item
‘bean’ \textbf{\textit{yakeka}} \item
‘bear’ \textit{v.} \textbf{\textit{kot-}} \item
‘beard’ \textbf{\textit{nil nopa}} \item
‘beat’ \textit{v.} \textbf{\textit{alima-}}, \textbf{\textit{ni-}} \item
‘because of’ \textbf{\textit{nakap}}, \textbf{\textit{nap}} \item
‘bedbug’ \textbf{\textit{mambun}} \item
‘bedsheet’ \textbf{\textit{al nambï}} \item
‘bee’ \textbf{\textit{numbunum}}, \textbf{\textit{unapïn}} \item
‘before’ \textbf{\textit{ipka}} \item
‘beforehand’ \textbf{\textit{ipka}} \item
‘behavior’ \textbf{\textit{i}} \item
‘behind’ \textbf{\textit{angani}} \item
‘belch’ \textbf{\textit{nïkïn}} \item
‘belly’ \textbf{\textit{inapaw}} \item
‘below’ \textbf{\textit{imbam}} \item
‘belt’ \textbf{\textit{min}} \item
‘bend’ \textit{v.} \textbf{\textit{tumul ka-}} \item
‘berry’ \textbf{\textit{mu}} \item
‘beside’ \textbf{\textit{kana}}, \textbf{\textit{kanam}} \item
‘betel nut’ \textbf{\textit{aw}}, \textbf{\textit{aw ilowan}}, \textbf{\textit{aw lïmndï}}, \textbf{\textit{aw ulum}}, \textbf{\textit{aw wapata}}, \textbf{\textit{kakïla}}, \linebreak \textbf{\textit{mïnanum}}, \textbf{\textit{pïsima}} \item
‘betel nut spittle’ \textbf{\textit{aw imbïn}} \item
‘betel palm’ \textbf{\textit{aw}}, \textbf{\textit{aw ilowan}} \item
‘betel pepper’ \textbf{\textit{kumblima}}, \textbf{\textit{nakam wanmbi}}, \textbf{\textit{nganangan}}, \textbf{\textit{wanmbi}} \item
‘betel pepper leaf’ \textbf{\textit{wanmbi wapa}} \item
‘betel pepper vine’ \textbf{\textit{wanmbi mutam}} \item
‘between’ \textbf{\textit{moni}} \item
‘big’ \textbf{\textit{ambi}}, \textbf{\textit{ngata}} \item
‘big man’ \textbf{\textit{ambi}}, \textbf{\textit{wokïn}} \item
‘big toe’ \textbf{\textit{wutïmu unduwan}} \item
‘\textit{bikmaus}’ \textbf{\textit{mapu}} \item
‘\textit{bilum}’ \textbf{\textit{ani}} \item
‘bird’ \textbf{\textit{uta}} \item
‘bird of paradise’ \textbf{\textit{yamanyawi}} \item
‘bird species’ \textbf{\textit{almba}}, \textbf{\textit{amangala}}, \textbf{\textit{ane uta}}, \textbf{\textit{awalawa}}, \textbf{\textit{awngala}}, \textbf{\textit{awsingïn}}, \linebreak \textbf{\textit{kalim}}, \textbf{\textit{kokawe}}, \textbf{\textit{kukumali}}, \textbf{\textit{kulkul}}, \textbf{\textit{kuman}}, \textbf{\textit{lamndu uta}}, \textbf{\textit{langay}}, \textbf{\textit{maep}}, \textbf{\textit{mamwapa}}, \textbf{\textit{membul}}, \textbf{\textit{moni}}, \textbf{\textit{mulwat}}, \textbf{\textit{ndolum}}, \textbf{\textit{ïmtu}}, \textbf{\textit{tambanji}}, \textbf{\textit{tanen}}, \textbf{\textit{tul}}, \textbf{\textit{upin}}, \textbf{\textit{walimot}}, \textbf{\textit{wandïl}}, \textbf{\textit{wenta}}, \textbf{\textit{wombulalaw}}, \textbf{\textit{wotnya}}, \textbf{\textit{wowal}}, \textbf{\textit{wutïwutï}}, \textbf{\textit{yakal inom}}, \textbf{\textit{yamanyawi}}, \textbf{\textit{yami}}, \textbf{\textit{yokomakan}}, \textbf{\textit{yopa}}, \textbf{\textit{yuname}} \item
‘bite’ \textit{v.} \textbf{\textit{ama-}}, \textbf{\textit{la-}} \item
‘bitter’ \textbf{\textit{nanama}} \item
‘black’ \textbf{\textit{mbun}}, \textbf{\textit{mbunmana}} \item
‘bladder’ \textbf{\textit{mïnopal}} \item
‘blaze’ \textit{v.} \textbf{\textit{wo-}} \item
‘blind’ \textbf{\textit{lïmndï wopa}} \item
‘blister’ \textbf{\textit{kananum}} \item
‘block’ \textit{v.} \textbf{\textit{nambïlumo-}} \item
‘blood’ \textbf{\textit{anankïn}} \item
‘blow’ \textit{v.} \textbf{\textit{nonalni-}} \item
‘blowfly’ \textbf{\textit{lamndu mu}}, \textbf{\textit{mu}} \item
‘blue’ \textbf{\textit{anem}}, \textbf{\textit{mbun}} \item
‘\textit{blulang}’ \textbf{\textit{mu}} \item
‘blunt’ \textbf{\textit{pon}}, \textbf{\textit{tambumana}} \item
‘boat’ \textbf{\textit{anaw}}, \textbf{\textit{num}}, \textbf{\textit{wulis}} \item
‘body’ \textbf{\textit{nambï}} \item
‘body hair’ \textbf{\textit{nil}} \item
‘boil’ \textbf{\textit{kananum}} \item
‘boil’ \textit{v.} \textbf{\textit{manal u-}} \item
‘bone’ \textbf{\textit{uma}} \item
‘bottom’ \textbf{\textit{nupu}} \item
‘bow’ \textbf{\textit{wongïta}} \item
‘bow and arrow’ \textbf{\textit{wongïta}} \item
‘bowl’ \textbf{\textit{ulet}} \item
‘bowstring’ \textbf{\textit{le}} \item
‘box’ \textit{v.} \textbf{\textit{welo-}} \item
‘boy’ \textbf{\textit{nungol}}, \textbf{\textit{nungolke}}, \textbf{\textit{yatalum}}, \textbf{\textit{yetalum}} \item
‘brain’ \textbf{\textit{misam}} \item
‘brains’ \textbf{\textit{misam}} \item
‘branch’ \textbf{\textit{im nangïn}} \item
‘breadfruit’ \textbf{\textit{nïpa}} \item
‘break’ \textit{v.} \textbf{\textit{a-}}, \textbf{\textit{kol-}}, \textbf{\textit{kot-}}, \textbf{\textit{kun-}}, \textbf{\textit{nungun u-}}, \textbf{\textit{tukul-}} \item
‘break off’ \textit{v.} \textbf{\textit{kun-}} \item
‘break up’ \textit{v.} \textbf{\textit{nïka-}}, \textbf{\textit{nïkï-}}, \textbf{\textit{nka-}}, \textbf{\textit{nkï-}} \item
‘breast’ \textbf{\textit{wol}} \item
‘breath’ \textbf{\textit{nonal}} \item
‘breathe’ \textit{v.} \textbf{\textit{nonal u-}} \item
‘bridge’ \textbf{\textit{ndam}} \item
‘broom’ \textbf{\textit{saklup}} \item
‘broth’ \textbf{\textit{isi}} \item
‘brother’ \textbf{\textit{atuma}}, \textbf{\textit{wot yata}}, \textbf{\textit{wot yeta}}, \textbf{\textit{yata}}, \textbf{\textit{yeta}} \item
‘brother-in-law’ \textbf{\textit{atana numan}}, \textbf{\textit{wot yena numan}} \item
‘brown’ \textbf{\textit{lemetam}} \item
‘\textit{buai}’ \textbf{\textit{ansi}}, \textbf{\textit{aw}} \item
‘\textit{buai limbum}’ \textbf{\textit{unden}} \item
‘bud’ \textbf{\textit{tangam}} \item
‘build’ \textit{v.} \textbf{\textit{ita-}} \item
‘building’ \textbf{\textit{apa}} \item
‘bullet’ \textbf{\textit{wipam}} \item
‘bump’ \textbf{\textit{mu}} \item
‘bunch’ \textbf{\textit{law}}, \textbf{\textit{motam}}, \textbf{\textit{nisi}}, \textbf{\textit{watangïn}} \item
‘bundle’ \textbf{\textit{motam}} \item
‘burial spot’ \textbf{\textit{inum}} \item
‘burn’ \textit{v.} \textbf{\textit{apïn ama-}}, \textbf{\textit{wo-}} \item
‘burp’ \textbf{\textit{nïkïn}} \item
‘bush’ \textbf{\textit{wandam}} \item
‘but’ \textbf{\textit{tasol}} \item
‘butcher’ \textit{v.} \textbf{\textit{nïka-}}, \textbf{\textit{nïkï-}}, \textbf{\textit{nka-}}, \textbf{\textit{nkï-}} \item
‘butterfly’ \textbf{\textit{awpane}} \item
‘buttocks’ \textbf{\textit{unmbï}} \item
‘buttress root’ \textbf{\textit{tïmal}}\\ \item

\noindent \textbf{C – c}\\ \item

‘calcium hydroxide’ \textbf{\textit{i}} \item
‘call’ \textit{v.} \textbf{\textit{wanakï-}}, \textbf{\textit{wanawni-}} \item
‘can’ \textbf{\textit{ken}} \item
‘cane’ \textbf{\textit{le}}, \textbf{\textit{mil}}, \textbf{\textit{palam}} \item
‘cane grass’ \textbf{\textit{palam}} \item
‘canoe’ \textbf{\textit{num}} \item
‘car’ \textbf{\textit{tïlwa num}} \item
‘careful’ \textbf{\textit{andïl}} \item
‘carry’ \textit{v.} \textbf{\textit{tï inda-}} \item
‘carve’ \textit{v.} \textbf{\textit{lo-}} \item
‘casque’ \textbf{\textit{kokal}} \item
‘cassowary’ \textbf{\textit{kalim}} \item
‘cassowary bone’ \textbf{\textit{intïp}} \item
‘catch’ \textit{v.} \textbf{\textit{ikali lï-}}, \textbf{\textit{moko-}}, \textbf{\textit{uta-}} \item
‘caterpillar’ \textbf{\textit{wapal}}, \textbf{\textit{yakal}} \item
‘catfish’ \textbf{\textit{may}} \item
‘cause’ \textbf{\textit{na}} \item
‘celebrate’ \textit{v.} \textbf{\textit{inim nkï-}} \item
‘cemetery’ \textbf{\textit{mbinmbin}} \item
‘centipede’ \textbf{\textit{yambïpal}} \item
‘chair’ \textbf{\textit{komblam}} \item
‘chase’ \textit{v.} \textbf{\textit{anmbasa-}} \item
‘check’ \textit{v.} \textbf{\textit{lïmndï uta-}} \item
‘cheek’ \textbf{\textit{nopa}} \item
‘chest’ \textbf{\textit{tambeta}} \item
‘chew’ \textit{v.} \textbf{\textit{ama-}}, \textbf{\textit{la-}} \item
‘chewed-up betel nut’ \textbf{\textit{ansi}} \item
‘chicken’ \textbf{\textit{wowal}} \item
‘child’ \textbf{\textit{alum}}, \textbf{\textit{nungol}}, \textbf{\textit{nungolke}}, \textbf{\textit{tawatïp}} \item
‘chin’ \textbf{\textit{ngïnïm}} \item
‘chop’ \textit{v.} \textbf{\textit{lo-}} \item
‘civil servant’ \textbf{\textit{inangïnmana}} \item
‘clan’ \textbf{\textit{amba}} \item
‘clavicle’ \textbf{\textit{unum}} \item
‘claw’ \textbf{\textit{sinananangïn}} \item
‘clay’ \textbf{\textit{pan}} \item
‘clay bowl’ \textbf{\textit{ulet}} \item
‘clay pan’ \textbf{\textit{wombasa}} \item
‘clay pot’ \textbf{\textit{kukumbe}}, \textbf{\textit{wewun}}, \textbf{\textit{wombasa}} \item
‘clay stand’ \textbf{\textit{nom}} \item
‘clear’ \textit{v.} \textbf{\textit{watlo-}} \item
‘clitoris’ \textbf{\textit{inmbï mïnïm}} \item
‘close’ \textbf{\textit{nu}} \item
‘cloth’ \textbf{\textit{al nambï}} \item
‘clothing’ \textbf{\textit{al nambï}} \item
‘cloud’ \textbf{\textit{anam}}, \textbf{\textit{ngïn}} \item
‘cockatoo’ \textbf{\textit{yopa}} \item
‘cockroach’ \textbf{\textit{ngunguswa}} \item
‘cocoa’ \textbf{\textit{koko}} \item
‘coconut’ \textbf{\textit{alalama}}, \textbf{\textit{andïmoni}}, \textbf{\textit{ya}} \item
‘coconut husk’ \textbf{\textit{tïl}} \item
‘coconut meat’ \textbf{\textit{mupu}} \item
‘coconut shell’ \textbf{\textit{wuta}}, \textbf{\textit{uta}} \item
‘coconut water’ \textbf{\textit{ya inim}} \item
‘cold’ \textbf{\textit{mïnoma}}, \textbf{\textit{mumne}} \item
‘collect’ \textit{v.} \textbf{\textit{in-}}, \textbf{\textit{ina-}} \item
‘comb’ \textbf{\textit{kokal}} \item
‘come’ \textit{v.} \textbf{\textit{anmbi-}}, \textbf{\textit{i-}}, \textbf{\textit{mbi-}} \item
‘come here’ \textit{v.} \textbf{\textit{mbi-}} \item
‘come out’ \textit{v.} \textbf{\textit{anmbi-}} \item
‘completely’ \textbf{\textit{kaka}}, \textbf{\textit{keka}} \item
‘conch shell’ \textbf{\textit{wokomana}} \item
‘container’ \textbf{\textit{akum}}, \textbf{\textit{itenmbu}}, \textbf{\textit{unden}} \item
‘cook’ \textit{v.} \textbf{\textit{wana-}} \item
‘cooking tongs’ \textbf{\textit{nangïn}}, \textbf{\textit{we nangïn}} \item
‘cool’ \textbf{\textit{mïnoma}} \item
‘cordyline’ \textbf{\textit{law}} \item
‘core’ \textbf{\textit{mupu}} \item
‘corn’ \textbf{\textit{kon}} \item
‘correct’ \textbf{\textit{maw}} \item
‘cough’ \textbf{\textit{utan}} \item
‘cough’ \textit{v.} \textbf{\textit{utan uta-}} \item
‘count’ \textit{v.} \textbf{\textit{ika uta-}} \item
‘crab’ \textbf{\textit{mbone}} \item
‘crayfish’ \textbf{\textit{mi}}, \textbf{\textit{ni}} \item
‘creek’ \textbf{\textit{u}} \item
‘crevice’ \textbf{\textit{unum}} \item
‘crocodile’ \textbf{\textit{wusim}} \item
‘crop’ \textbf{\textit{sinokoy}} \item
‘cross’ \textit{v.} \textbf{\textit{klop-}}, \textbf{\textit{won-}} \item
‘crossbeam’ \textbf{\textit{iwal}}, \textbf{\textit{kukun}}, \textbf{\textit{pal}} \item
‘croton’ \textbf{\textit{moniwot}} \item
‘crow’ \textbf{\textit{tul}} \item
‘crowned pigeon’ \textbf{\textit{upin}} \item
‘crush’ \textit{v.} \textbf{\textit{nopal u-}} \item
‘cry’ \textit{v.} \textbf{\textit{sa-}} \item
‘cup’ \textbf{\textit{itenmbu}}, \textbf{\textit{wonglin}} \item
‘cuscus’ \textbf{\textit{ngwimakan}}, \textbf{\textit{tïponïm}} \item
‘custom’ \textbf{\textit{i}} \item
‘cut down’ \textit{v.} \textbf{\textit{lo-}}, \textbf{\textit{watlo-}} \item
‘cut’ \textit{v.} \textbf{\textit{lo-}}, \textbf{\textit{nïka-}}, \textbf{\textit{nïkï-}}, \textbf{\textit{nka-}}, \textbf{\textit{nkï-}}, \textbf{\textit{we u-}}, \textbf{\textit{won-}} \item
‘cyclone’ \textbf{\textit{ngungun}}\\ \item

\noindent \textbf{D – d}\\ \item

‘\textit{daka}’ \textbf{\textit{kumblima}}, \textbf{\textit{nakam wanmbi}}, \textbf{\textit{nganangan}}, \textbf{\textit{wanmbi}} \item
‘dance’ \textit{v.} \textbf{\textit{wutïni-}} \item
‘dark’ \textbf{\textit{mbun}}, \textbf{\textit{mumne}} \item
‘daughter’ \textbf{\textit{yanalum}}, \textbf{\textit{yanat}}, \textbf{\textit{yenalum}}, \textbf{\textit{yenat}} \item
‘dawn’ \textbf{\textit{lïwa}} \item
‘day’ \textbf{\textit{ane}}, \textbf{\textit{ilom}} \item
‘daytime’ \textbf{\textit{ane}} \item
‘deaf’ \textbf{\textit{kïkal wopa}} \item
‘decaying’ \textbf{\textit{alata}} \item
‘decoration’ \textbf{\textit{manangum}} \item
‘devil’ \textbf{\textit{yambalpa}} \item
‘dew’ \textbf{\textit{inimnji}} \item
‘diaphragm’ \textbf{\textit{tïmbïl}} \item
‘die’ \textit{v.} \textbf{\textit{ni-}}, \textbf{\textit{nipinp u-}}, \textbf{\textit{nipunp u-}} \item
‘difficult’ \textbf{\textit{matamal}} \item
‘dig’ \textit{v.} \textbf{\textit{nïka-}}, \textbf{\textit{nïkï-}}, \textbf{\textit{nka-}}, \textbf{\textit{nkï-}} \item
‘digit’ \textbf{\textit{imu}} \item
‘dirty’ \textbf{\textit{tembi}} \item
‘disapprove of’ \textit{v.} \textbf{\textit{alakamb-}} \item
‘dish’ \textbf{\textit{ulet}} \item
‘dislike’ \textit{v.} \textbf{\textit{alakamb-}} \item
‘district officer’ \textbf{\textit{wanwane}} \item
‘ditch’ \textbf{\textit{numïni}}, \textbf{\textit{u}} \item
‘do’ \textit{v.} \textbf{\textit{ni-}} \item
‘dog’ \textbf{\textit{tïn}} \item
‘don’t!’ \textbf{\textit{wana}}, \textbf{\textit{wanap}} \item
‘door’ \textbf{\textit{mwa}} \item
‘dove’ \textbf{\textit{walimot}} \item
‘down’ \textbf{\textit{li}} \item
‘downstream’ \textbf{\textit{li}} \item
‘downward’ \textbf{\textit{li}} \item
‘dragonfly’ \textbf{\textit{maman}} \item
‘dream’ \textbf{\textit{longom}} \item
‘drink’ \textit{v.} \textbf{\textit{ama-}}, \textbf{\textit{la-}} \item
‘drinking coconut’ \textbf{\textit{andïmoni}} \item
‘drum’ \textbf{\textit{ansimu}}, \textbf{\textit{nïte}}, \textbf{\textit{numbu}} \item
‘dry’ \textbf{\textit{wapata}} \item
‘dry’ \textit{v.} \textbf{\textit{mondo-}} \item
‘dry season’ \textbf{\textit{ane wapata}} \item
‘duck’ \textbf{\textit{wutïwutï}} \item
‘dull’ \textbf{\textit{pon}}, \textbf{\textit{tambumana}} \item
‘dust’ \textbf{\textit{itïtïl}} \item
‘dwarf’ \textbf{\textit{metmet}}\\ \item

\noindent \textbf{E – e}\\ \item

‘each other’ \textbf{\textit{ambin=}} \item
‘eagle’ \textbf{\textit{amangala}}, \textbf{\textit{awsingïn}} \item
‘ear’ \textbf{\textit{kïkal}} \item
‘earlier’ \textbf{\textit{ipka}} \item
‘earring’ \textbf{\textit{mungun}} \item
‘earth’ \textbf{\textit{ini}} \item
‘earthquake’ \textbf{\textit{numnata}} \item
‘earwax’ \textbf{\textit{mungun}} \item
‘eat’ \textit{v.} \textbf{\textit{ama-}}, \textbf{\textit{la-}} \item
‘eel’ \textbf{\textit{kundan}}, \textbf{\textit{yawatalin}} \item
‘egg’ \textbf{\textit{mïtïn}}, \textbf{\textit{yokomtïn}} \item
‘egg shell’ \textbf{\textit{nambum}} \item
‘egg white’ \textbf{\textit{ya}} \item
‘egg yolk’ \textbf{\textit{kalum}} \item
‘eh?’ \textbf{\textit{a}}, \textbf{\textit{e}} \item
‘eight’ \textbf{\textit{angay kwe lele ndïwon ndïwatlïp}} \item
‘eighteen’ \textbf{\textit{angay lele lele ndïwon ndïwatlïp}} \item
‘eighty’ \textbf{\textit{ankam unduwan nali lele}} \item
‘elbow’ \textbf{\textit{i ambatïm}}, \textbf{\textit{inpu}} \item
‘elder’ \textbf{\textit{unduwan}} \item
‘eleven’ \textbf{\textit{angay nini kwe mowon ndïwatlïp}}, \textbf{\textit{nali kwe kwe}} \item
‘else’ \textbf{\textit{tïki}} \item
‘empty’ \textbf{\textit{ulwa}} \item
‘end’ \textbf{\textit{at}} \item
‘enemy’ \textbf{\textit{mbalanji}}, \textbf{\textit{unda}} \item
‘enough’ \textit{v.} \textbf{\textit{nakamb-}} \item
‘entire’ \textbf{\textit{wopa}} \item
‘equal’ \textbf{\textit{ambalka}} \item
‘\textit{erima}’ \textbf{\textit{awe}} \item
‘even though’ \textbf{\textit{maski}} \item
‘evening’ \textbf{\textit{awal}}, \textbf{\textit{imba}} \item
‘every’ \textbf{\textit{nunu}} \item
‘every day’ \textbf{\textit{nunu ilom}} \item
‘everyone’ \textbf{\textit{wopa}} \item
‘everything’ \textbf{\textit{nunu nji}}, \textbf{\textit{wopa}} \item
‘examine’ \textit{v.} \textbf{\textit{lïmndï uta-}} \item
‘excrement’ \textbf{\textit{minyam}} \item
‘eye’ \textbf{\textit{lïmndï}} \item
‘eye mucus’ \textbf{\textit{lïmndï minyam}} \item
‘eye of a needle’ \textbf{\textit{mwa}}\\ \item

\noindent \textbf{F – f}\\ \item

‘face’ \textbf{\textit{monombam}}, \textbf{\textit{mwa}} \item
‘fall’ \textit{v.} \textbf{\textit{li u-}} \item
‘fallow garden’ \textbf{\textit{wandapata}} \item
‘falsehood’ \textbf{\textit{awaw}} \item
‘family member’ \textbf{\textit{nambana ankam}}, \textbf{\textit{tamndï}} \item
‘fan’ \textbf{\textit{wun}} \item
‘far’ \textbf{\textit{ngaya}}, \textbf{\textit{wala}} \item
‘far-off’ \textbf{\textit{wala}} \item
‘fart’ \textbf{\textit{nuku}} \item
‘fast’ \textbf{\textit{mbuka}} \item
‘fat’ \textbf{\textit{anen}} \item
‘father’ \textbf{\textit{itom}}, \textbf{\textit{tata}} \item
‘fear’ \textit{v.} \textbf{\textit{ala namnap}} \item
‘fearful’ \textbf{\textit{namna}} \item
‘feather’ \textbf{\textit{nambli}}, \textbf{\textit{tal}}, \textbf{\textit{wowane}} \item
‘feces’ \textbf{\textit{minyam}} \item
‘feed’ \textit{v.} \textbf{\textit{na-}} \item
‘feel’ \textit{v.} \textbf{\textit{wana-}} \item
‘fell’ \textit{v.} \textbf{\textit{lo-}} \item
‘female’ \textbf{\textit{yana}}, \textbf{\textit{yena}} \item
‘fence’ \textbf{\textit{tïmbïl}} \item
‘fern’ \textbf{\textit{lindïn}}, \textbf{\textit{mungul}} \item
‘few’ \textbf{\textit{ilum}} \item
‘fiber’ \textbf{\textit{mi}} \item
‘fifteen’ \textbf{\textit{angay lele}} \item
‘fifty’ \textbf{\textit{ankam unduwan}}, \textbf{\textit{nali angay}} \item
‘fig tree’ \textbf{\textit{mïka}} \item
‘fight’ \textbf{\textit{at}} \item
‘fight’ \textit{v.} \textbf{\textit{amblawali-}} \item
‘\textit{fikus}’ \textbf{\textit{mïka}} \item
‘fin’ \textbf{\textit{angun}} \item
‘final’ \textbf{\textit{watangïn}} \item
‘finger’ \textbf{\textit{imu}} \item
‘fingernail’ \textbf{\textit{sinanan}} \item
‘fire’ \textbf{\textit{apïn}} \item
‘fire tongs’ \textbf{\textit{apïn nangïn}}, \textbf{\textit{nangïn}} \item
‘firefly’ \textbf{\textit{mbomala}}, \textbf{\textit{nali}} \item
‘firewood’ \textbf{\textit{imot}} \item
‘first’ \textbf{\textit{ipka}} \item
‘fish’ \textbf{\textit{wambana}} \item
‘fish hook’ \textbf{\textit{uma}} \item
‘fish scale’ \textbf{\textit{wowaw}} \item
‘fish species’ \textbf{\textit{kundan}}, \textbf{\textit{lanjin}}, \textbf{\textit{lumnjap}}, \textbf{\textit{mapu}}, \textbf{\textit{may}}, \textbf{\textit{mbatmbat}}, \textbf{\textit{nïmban}}, \linebreak \textbf{\textit{somïn}}, \textbf{\textit{upa}}, \textbf{\textit{upan}}, \textbf{\textit{wowaw}}, \textbf{\textit{yawatalin}} \item
‘fish trap’ \textbf{\textit{iwa}}, \textbf{\textit{ngin}}, \textbf{\textit{tukul}} \item
‘fishing line’ \textbf{\textit{asiya}} \item
‘fishing net’ \textbf{\textit{ngin}} \item
‘fishing spear’ \textbf{\textit{nap}} \item
‘fishtail’ \textbf{\textit{anaw}}, \textbf{\textit{angun}} \item
‘five’ \textbf{\textit{angay}} \item
‘flashlight’ \textbf{\textit{mbomala nangum}} \item
‘flat’ \textbf{\textit{ambalka}} \item
‘flatus’ \textbf{\textit{nuku}} \item
‘flesh’ \textbf{\textit{lam}} \item
‘float’ \textit{v.} \textbf{\textit{ul watka-}} \item
‘flood’ \textbf{\textit{inim ambi}} \item
‘floor’ \textbf{\textit{apa ini}} \item
‘flour’ \textbf{\textit{we}} \item
‘flow’ \textit{v.} \textbf{\textit{i}}, \textbf{\textit{ma-}} \item
‘flower’ \textbf{\textit{nambana}}, \textbf{\textit{woka}} \item
‘flower sheath’ \textbf{\textit{nisi}} \item
‘flute’ \textbf{\textit{mota}}, \textbf{\textit{wusimi}} \item
‘fly’ \textbf{\textit{njimana}}, \textbf{\textit{sikal}} \item
‘fly’ \textit{v.} \textbf{\textit{wiwina-}} \item
‘flying fox’ \textbf{\textit{nïplopa}} \item
‘fog’ \textbf{\textit{lïngïn}} \item
‘follow’ \textit{v.} \textbf{\textit{angani ka-}} \item
‘food’ \textbf{\textit{mundu}} \item
‘foot’ \textbf{\textit{utï}}, \textbf{\textit{wutï}} \item
‘footprint’ \textbf{\textit{wutï yïwa}} \item
‘for’ \textbf{\textit{ala}}, \textbf{\textit{andï}}, \textbf{\textit{andïm}}, \textbf{\textit{andïn}}, \textbf{\textit{nakap}}, \textbf{\textit{nap}} \item
‘for the sake of’ \textbf{\textit{nakap}}, \textbf{\textit{nap}} \item
‘force’ \textit{v.} \textbf{\textit{ul ni-}} \item
‘forearm’ \textbf{\textit{i nangum}} \item
‘forehead’ \textbf{\textit{monombam}} \item
‘forest’ \textbf{\textit{wandam}} \item
‘forget’ \textit{v.} \textbf{\textit{topinka-}} \item
‘forget about it!’ \textbf{\textit{mambilakan}} \item
‘fork’ \textbf{\textit{akatoma}} \item
‘forty’ \textbf{\textit{nali watangïnila}} \item
‘four’ \textbf{\textit{watangïnila}} \item
‘fourteen’ \textbf{\textit{angay nini watangïnila ndïwon ndïwatlïp}}, \textbf{\textit{nali kwe watangïnila}} \item
‘fourth toe’ \textbf{\textit{wutïmu law}} \item
‘fresh’ \textbf{\textit{akïnaka}} \item
‘friend’ \textbf{\textit{awena}}, \textbf{\textit{aweta}} \item
‘frog’ \textbf{\textit{wolname}}, \textbf{\textit{womotana}} \item
‘frog species’ \textbf{\textit{kïlakïli}}, \textbf{\textit{popotala}}, \textbf{\textit{tongonat}}, \textbf{\textit{tomona}}, \textbf{\textit{wali}}, \textbf{\textit{wandïwandï}} \item
‘from’ \textbf{\textit{ala}}, \textbf{\textit{andï}}, \textbf{\textit{andïm}}, \textbf{\textit{andïn}}, \textbf{\textit{u}} \item
‘from here’ \textbf{\textit{mbu}} \item
‘from there’ \textbf{\textit{ando}} \item
‘frond’ \textbf{\textit{akïnanga}}, \textbf{\textit{isi}}, \textbf{\textit{nopal}}, \textbf{\textit{wema}} \item
‘front’ \textbf{\textit{ip}}, \textbf{\textit{ipwat}} \item
‘fruit’ \textbf{\textit{mu}} \item
‘fruit species’ \textbf{\textit{mondin}} \item
‘fruitlessly’ \textbf{\textit{woyambïn}} \item
‘full’ \textbf{\textit{monop}}, \textbf{\textit{wopa}} \item
‘fungus’ \textbf{\textit{momul}} \item
‘fur’ \textbf{\textit{nambli}} \item
‘fuzz’ \textbf{\textit{isi}}\\ \item

\noindent \textbf{G – g}\\ \item

‘\textit{galip}’ \textbf{\textit{nokosam}} \item
‘gallbladder’ \textbf{\textit{mïnandïn}} \item
‘game’ \textbf{\textit{tïnum}} \item
‘\textit{garamut}’ \textbf{\textit{numbu}} \item
‘garbage’ \textbf{\textit{itïm}}, \textbf{\textit{umba}} \item
‘garden’ \textbf{\textit{wandam}}, \textbf{\textit{wandapata}} \item
‘garfish’ \textbf{\textit{lumnjap}} \item
‘gather’ \textit{v.} \textbf{\textit{kuk u-}} \item
‘gecko’ \textbf{\textit{nataw}}, \textbf{\textit{wemana}} \item
‘get’ \textit{v.} \textbf{\textit{in-}}, \textbf{\textit{ina-}}, \textbf{\textit{t}}, \textbf{\textit{tï-}} \item
‘get up’ \textit{v.} \textbf{\textit{tïnanga-}} \item
‘ghost’ \textbf{\textit{nambana}} \item
‘giant’ \textbf{\textit{nïpat}} \item
‘ginger’ \textbf{\textit{anat}}, \textbf{\textit{mïli}} \item
‘girl’ \textbf{\textit{nungol}}, \textbf{\textit{nungolke}}, \textbf{\textit{yanalum}}, \textbf{\textit{yenalum}} \item
‘give birth’ \textit{v.} \textbf{\textit{kot-}} \item
‘give’ \textit{v.} \textbf{\textit{na-}} \item
‘glowing fungus’ \textbf{\textit{momul}} \item
‘go’ \textit{v.} \textbf{\textit{anma-}}, \textbf{\textit{atay}}, \textbf{\textit{i}}, \textbf{\textit{li}}, \textbf{\textit{lo-}}, \textbf{\textit{lu-}}, \textbf{\textit{a-}}, \textbf{\textit{unda-}} \item
‘go!’ \textbf{\textit{nol}} \item
‘go around’ \textit{v.} \textbf{\textit{unda-}} \item
‘go down’ \textit{v.} \textbf{\textit{li}} \item
‘go out’ \textit{v.} \textbf{\textit{anma-}} \item
‘go up’ \textit{v.} \textbf{\textit{atay}} \item
‘God’ \textbf{\textit{ambi}} \item
‘good’ \textbf{\textit{anma}} \item
‘good afternoon’ \textbf{\textit{awal anma}}, \textbf{\textit{awal nambï anma}} \item
‘good day’ \textbf{\textit{ane anma}} \item
‘good evening’ \textbf{\textit{imba anma}} \item
‘good morning’ \textbf{\textit{umbenam anma}} \item
‘good night’ \textbf{\textit{imba anma}} \item
‘goodbye’ \textbf{\textit{namanu}} \item
‘\textit{gorgor}’ \textbf{\textit{mïli}} \item
‘gourd’ \textbf{\textit{ansi}} \item
‘grab’ \textit{v.} \textbf{\textit{ikali lï-}} \item
‘grand’ \textbf{\textit{ngata}} \item
‘grandchild’ \textbf{\textit{yalum}} \item
‘granddaughter’ \textbf{\textit{yalum}} \item
‘grandfather’ \textbf{\textit{itom ngata}} \item
‘grandmother’ \textbf{\textit{inom ngata}}, \textbf{\textit{mom}} \item
‘grandparent’ \textbf{\textit{ngata}} \item
‘grandson’ \textbf{\textit{yalum}} \item
‘grass’ \textbf{\textit{anul}}, \textbf{\textit{asi}}, \textbf{\textit{nipum}} \item
‘grass knife’ \textbf{\textit{asiyot}} \item
‘grass skirt’ \textbf{\textit{al}}, \textbf{\textit{ana}}, \textbf{\textit{anapot}} \item
‘grasshopper’ \textbf{\textit{kukum}} \item
‘grassland’ \textbf{\textit{anul}} \item
‘grave’ \textbf{\textit{inum}}, \textbf{\textit{mbinmbin}} \item
‘gray hair’ \textbf{\textit{monkin}} \item
‘grease’ \textbf{\textit{anen}} \item
‘great-grandchild’ \textbf{\textit{ndunduma}} \item
‘great-grandparent’ \textbf{\textit{ndunduma}} \item
‘great-uncle’ \textbf{\textit{ngata yawa}} \item
‘green’ \textbf{\textit{mïnal}} \item
‘greens’ \textbf{\textit{natnat}} \item
‘grind’ \textit{v.} \textbf{\textit{uta-}} \item
‘groin’ \textbf{\textit{utï moni}} \item
‘ground’ \textbf{\textit{ini}}, \textbf{\textit{inum}} \item
‘grub species’ \textbf{\textit{inane}}, \textbf{\textit{mïnkïn}}, \textbf{\textit{mundum}}, \textbf{\textit{siwi}}, \textbf{\textit{sum}} \item
‘gudgeon’ \textbf{\textit{mapu}} \item
‘gums’ ambla \textbf{\textit{lam}} \item
‘\textit{guria}’ \textbf{\textit{upin}} \item
‘guts’ \textbf{\textit{inji}}, \textbf{\textit{mïnane}}\\ \item

\noindent \textbf{H – h}\\ \item

‘habit’ \textbf{\textit{i}} \item
‘hair’ \textbf{\textit{monkin}}, \textbf{\textit{nil}}, \textbf{\textit{wonmi}} \item
‘hand’ \textbf{\textit{i}} \item
‘hand drum’ \textbf{\textit{nïte}} \item
‘handle’ \textbf{\textit{wen}} \item
‘happy, be’ \textit{v.} \textbf{\textit{anma wanani-}} \item
‘hard’ \textbf{\textit{nïpokonam}} \item
‘hardwood tree’ \textbf{\textit{lemetam}} \item
‘harvest’ \textit{v.} \textbf{\textit{ne-}} \item
‘hate’ \textit{v.} \textbf{\textit{alakamb-}} \item
‘\textit{haus boi}’ \textbf{\textit{amba}} \item
‘\textit{haus tambaran}’ \textbf{\textit{amba}} \item
‘hawk’ \textbf{\textit{amangala}}, \textbf{\textit{awsingïn}} \item
‘he’ \textbf{\textit{mï}} \item
‘he is the one’ \textbf{\textit{mïnam}} \item
‘head’ \textbf{\textit{mu}}, \textbf{\textit{unduwan}} \item
‘headache’ \textbf{\textit{unduwan apïn}} \item
‘headdress’ \textbf{\textit{tongla}} \item
‘healthy’ \textbf{\textit{anma}} \item
‘heap’ \textbf{\textit{tumopa}}, \textbf{\textit{umba}} \item
‘hear’ \textit{v.} \textbf{\textit{kïkal wana-}} \item
‘heart’ \textbf{\textit{ina}}, \textbf{\textit{yom}} \item
‘hearth’ \textbf{\textit{twa}} \item
‘heavy’ \textbf{\textit{kenmbu}} \item
‘heel’ \textbf{\textit{wutïnpu}} \item
‘hence’ \textbf{\textit{mbu}} \item
‘her’ \textbf{\textit{ma=}}, \textbf{\textit{mo=}}, \textbf{\textit{manji}} \item
‘her own’ \textbf{\textit{ambïnji}} \item
‘here’ \textbf{\textit{mbï}}, \textbf{\textit{mbu}} \item
‘hers’ \textbf{\textit{manji}} \item
‘herself’ \textbf{\textit{mawa}}, \textbf{\textit{mawe}} \item
‘hey’ \textbf{\textit{e}} \item
‘hiccup’ \textbf{\textit{nïkïn}} \item
‘hide’ \textbf{\textit{nambï}} \item
‘hide’ \textit{v.} \textbf{\textit{nokop lï-}} \item
‘high’ \textbf{\textit{ata}} \item
‘hill’ \textbf{\textit{yïwa}} \item
‘him’ \textbf{\textit{ma=}}, \textbf{\textit{mo=}} \item
‘himself’ \textbf{\textit{ambawa=}}, \textbf{\textit{ambï=}}, \textbf{\textit{ambuwe}}, \textbf{\textit{mawa}}, \textbf{\textit{mawe}} \item
‘hip’ \textbf{\textit{tïpal}} \item
‘his’ \textbf{\textit{manji}} \item
‘his own’ \textbf{\textit{ambïnji}} \item
‘hit’ \textit{v.} \textbf{\textit{as}}, \textbf{\textit{asa-}}, \textbf{\textit{atï-}}, \textbf{\textit{wali-}} \item
‘hither’ \textbf{\textit{mbï}} \item
‘hive’ \textbf{\textit{wawal}} \item
‘hm’ \textbf{\textit{m}} \item
‘hoe’ \textbf{\textit{tanawen}} \item
‘hoe’ \textit{v.} \textbf{\textit{nïka-}}, \textbf{\textit{nïkï-}}, \textbf{\textit{nka-}}, \textbf{\textit{nkï-}} \item
‘hold’ \textit{v.} \textbf{\textit{ikali lï-}} \item
‘hole’ \textbf{\textit{inmi}} \item
‘Holy Spirit’ \textbf{\textit{nonal}} \item
‘hoof’ \textbf{\textit{akat}} \item
‘hook’ \textbf{\textit{uma}} \item
‘horn’ \textbf{\textit{kokal}} \item
‘hornbill’ \textbf{\textit{almba}} \item
‘hot’ \textbf{\textit{wananum}} \item
‘hot water’ \textbf{\textit{manal}} \item
‘house’ \textbf{\textit{apa}} \item
‘housefly’ \textbf{\textit{njimana}} \item
‘how many?’ \textbf{\textit{anjika}} \item
‘how?’ \textbf{\textit{anjikaka}} \item
‘huge’ \textbf{\textit{ngata}} \item
‘human’ \textbf{\textit{ankam}} \item
‘hundred’ \textbf{\textit{uta}} \item
‘hunger’ \textbf{\textit{mundu}} \item
‘hungry, be’ \textit{v.} \textbf{\textit{mundu asa-}} \item
‘hunt’ \textit{v.} \textbf{\textit{andïlalo-}}, \textbf{\textit{anglalo-}} \item
‘husband’ \textbf{\textit{numan}} \item
‘husk’ \textbf{\textit{tïl}}\\ \item

\noindent \textbf{I – i}\\ \item

‘I’ \textbf{\textit{nï}} \item
‘I myself’ \textbf{\textit{nawa}}, \textbf{\textit{nuwe}} \item
‘idol’ \textbf{\textit{molombi}} \item
‘if’ \textbf{\textit{sapos}} \item
‘ignite’ \textit{v.} \textbf{\textit{lomon-}} \item
‘ilima’ \textbf{\textit{awe}} \item
‘ill’ \textbf{\textit{tembi}} \item
‘illness’ \textbf{\textit{tembi}} \item
‘important person’ \textbf{\textit{wokïn}} \item
‘in’ \textbf{\textit{in}}, \textbf{\textit{ka}}, \textbf{\textit{u}} \item
‘in front of’ \textbf{\textit{ipka}} \item
‘in that way’ \textbf{\textit{ka}}, \textbf{\textit{maka}}, \textbf{\textit{mïka}} \item
‘in this way’ \textbf{\textit{ka}}, \textbf{\textit{maka}}, \textbf{\textit{mïka}} \item
‘in turn’ \textbf{\textit{wolka}} \item
‘in vain’ \textbf{\textit{awlop}} \item
‘in-law’ \textbf{\textit{inga}} \item
‘index finger’ \textbf{\textit{imu ankam}} \item
‘inhabit’ \textit{v.} \textbf{\textit{p-}}, \textbf{\textit{wap}} \item
‘initiation rites’ \textbf{\textit{sasi}} \item
‘innards’ \textbf{\textit{inji}} \item
‘insect species’ \textbf{\textit{amam}}, \textbf{\textit{apwane}}, \textbf{\textit{aylat}}, \textbf{\textit{iwanal}}, \textbf{\textit{kalingana}}, \textbf{\textit{katmombe}}, \textbf{\textit{kïka}}, \textbf{\textit{kukum}}, \textbf{\textit{lamndu mu}}, \textbf{\textit{maman}}, \textbf{\textit{mambun}}, \textbf{\textit{manjimanji}}, \textbf{\textit{mïmin}}, \textbf{\textit{mu}}, \textbf{\textit{muna}}, \textbf{\textit{mbomala}}, \textbf{\textit{nali}}, \textbf{\textit{nitïl}}, \textbf{\textit{numbunum}}, \textbf{\textit{nuna}}, \textbf{\textit{ndïngonim}}, \textbf{\textit{ngowil}}, \textbf{\textit{ngungun}}, \textbf{\textit{ngunguswa}}, \textbf{\textit{njimana}}, \textbf{\textit{sambulumbu}}, \textbf{\textit{sikal}}, \textbf{\textit{sïmin}}, \textbf{\textit{tambïn}}, \textbf{\textit{tomoy}}, \linebreak \textbf{\textit{tongonat}}, \textbf{\textit{unapïn}}, \textbf{\textit{wapal}}, \textbf{\textit{yakal}}, \textbf{\textit{yambïpal}}, \textbf{\textit{yami}}, \textbf{\textit{yangun}} \item
‘inside’ \textbf{\textit{in}} \item
‘insides’ \textbf{\textit{inji}} \item
‘instance’ \textbf{\textit{ika}} \item
‘intelligent’ \textbf{\textit{anma}} \item
‘intestines’ \textbf{\textit{mïnane}} \item
‘into’ \textbf{\textit{in}} \item
‘iris’ \textbf{\textit{lïmndï mu}} \item
‘ironwood’ \textbf{\textit{nanïm}}, \textbf{\textit{numbu}} \item
‘it’ \textbf{\textit{ma=}}, \textbf{\textit{mo=}}, \textbf{\textit{mï}} \item
‘it is the one’ \textbf{\textit{mïnam}} \item
‘its’ \textbf{\textit{manji}} \item
‘its own’ \textbf{\textit{ambïnji}} \item
‘itself’ \textbf{\textit{ambawa=}}, \textbf{\textit{ambï=}}, \textbf{\textit{ambuwe}}, \textbf{\textit{mawa}}, \textbf{\textit{mawe}}\\ \item

\noindent \textbf{J – j}\\ \item

‘Java almond’ \textbf{\textit{nokosam}} \item
‘jaw’ \textbf{\textit{limama}} \item
‘jellied sago’ \textbf{\textit{ay}}, \textbf{\textit{aypul}} \item
‘Jew’s harp’ \textbf{\textit{sinangul}} \item
‘job’ \textbf{\textit{wombïn}} \item
‘joint’ \textbf{\textit{ambatïm}}, \textbf{\textit{min}} \item
‘jump’ \textit{v.} \textbf{\textit{ulep lï-}} \item
‘jungle’ \textbf{\textit{wandam}} \item
‘just’ \textbf{\textit{lolop}}, \textbf{\textit{ko}}, \textbf{\textit{kop}}, \textbf{\textit{kwa}}, \textbf{\textit{wa}} \item
‘just a few’ \textbf{\textit{kekaka}}, \textbf{\textit{kwekaka}}\\ \item

\noindent \textbf{K – k}\\ \item

‘\textit{kalangal}’ \textbf{\textit{awalawa}} \item
‘\textit{kambang}’ \textbf{\textit{i}} \item
‘\textit{kanda}’ \textbf{\textit{le}} \item
‘\textit{kandere}’ \textbf{\textit{yawa}} \item
‘\textit{kapul}’ \textbf{\textit{ngwimakan}}, \textbf{\textit{tïponïm}} \item
‘\textit{karakum}’ \textbf{\textit{ngungun}} \item
‘\textit{kaukau}’ \textbf{\textit{nongontam}} \item
‘\textit{kawawar}’ \textbf{\textit{anat}} \item
‘\textit{kiap}’ \textbf{\textit{wanwane}} \item
‘kidney’ \textbf{\textit{ndanandum mu}} \item
‘kill’ \textit{v.} \textbf{\textit{as}}, \textbf{\textit{asa-}}, \textbf{\textit{atï-}}, \textbf{\textit{wali-}} \item
‘kin’ \textbf{\textit{nambana ankam}}, \textbf{\textit{tamndï}} \item
‘kingfisher’ \textbf{\textit{wombulalaw}} \item
‘knee’ \textbf{\textit{wutï ambatïm}} \item
‘knife’ \textbf{\textit{asiyot}}, \textbf{\textit{sina}}, \textbf{\textit{yawt}}, \textbf{\textit{yot}} \item
‘know’ \textit{v.} \textbf{\textit{kalamp}} \item
‘knowing’ \textbf{\textit{kalam}} \item
‘knowledge’ \textbf{\textit{kalam}} \item
‘knowledgeable’ \textbf{\textit{kalam}} \item
‘\textit{koki}’ \textbf{\textit{yopa}} \item
‘\textit{kokomo}’ \textbf{\textit{almba}} \item
‘\textit{kongkong}’ \textbf{\textit{atate}} \item
‘\textit{kulau}’ \textbf{\textit{andïmoni}} \item
‘\textit{kumu}’ \textbf{\textit{natnat}} \item
‘\textit{kumu mosong}’ \textbf{\textit{ankïn}} \item
‘\textit{kumul}’ \textbf{\textit{yamanyawi}} \item
‘\textit{kunai}’ \textbf{\textit{nipum}} \item
‘\textit{kundu}’ \textbf{\textit{nïte}} \item
‘\textit{kwila}’ \textbf{\textit{nanïm}}\\ \item

\newpage

\noindent \textbf{L – l}\\ \item

‘lacking’ \textbf{\textit{mundotoma}} \item
‘ladder’ \textbf{\textit{tamben}}, \textbf{\textit{wat}} \item
‘ladle’ \textbf{\textit{wonglin}} \item
‘ladybug’ \textbf{\textit{amam}} \item
‘lake’ \textbf{\textit{inimpul}} \item
‘land’ \textbf{\textit{ini}} \item
‘language’ \textbf{\textit{mïtïn}}, \textbf{\textit{na}} \item
‘lap’ \textbf{\textit{wutï name}} \item
‘large’ \textbf{\textit{ambi}} \item
‘last’ \textbf{\textit{watangïn}} \item
‘later’ \textbf{\textit{anganika}}, \textbf{\textit{naka}} \item
‘laugh’ \textit{v.} \textbf{\textit{atala-}} \item
‘laughter’ \textbf{\textit{atal}} \item
‘\textit{laulau}’ \textbf{\textit{eklak}} \item
‘leaf’ \textbf{\textit{mïndapan}}, \textbf{\textit{moma}}, \textbf{\textit{wapa}} \item
‘learn’ \textit{v.} \textbf{\textit{sikulmakan ni-}} \item
‘leave’ \textit{v.} \textbf{\textit{ka-}}, \textbf{\textit{laka-}} \item
‘leech’ \textbf{\textit{nasalïwa}} \item
‘left’ \textbf{\textit{andana}} \item
‘left-hand’ \textbf{\textit{andana}} \item
‘leftovers’ \textbf{\textit{utïl}} \item
‘leg’ \textbf{\textit{utï}}, \textbf{\textit{wutï}} \item
‘lest’ \textbf{\textit{nongut}} \item
‘let’ \textit{v.} \textbf{\textit{ka-}}, \textbf{\textit{laka-}} \item
‘let’s go!’ \textbf{\textit{nol}} \item
‘lie’ \textbf{\textit{awaw}} \item
‘lie’ \textit{v.} \textbf{\textit{lop ka-}} \item
‘lie down’ \textit{v.} \textbf{\textit{lop ka-}} \item
‘ligament’ \textbf{\textit{ndal}} \item
‘light’ \textbf{\textit{anembal}}, \textbf{\textit{wiwila}} \item
‘lighter’ \textbf{\textit{apïn}} \item
‘lightning’ \textbf{\textit{anam}} \item
‘\textit{limbum}’ \textbf{\textit{me}} \item
‘lime’ \textbf{\textit{i}} \item
‘lime gourd’ \textbf{\textit{ansi}} \item
‘\textit{lip daka}’ \textbf{\textit{wanmbi wapa}} \item
‘lips’ \textbf{\textit{tanum}} \item
‘liquid’ \textbf{\textit{inim}} \item
‘listen’ \textit{v.} \textbf{\textit{kïkal wana-}} \item
‘little’ \textbf{\textit{ilum}}, \textbf{\textit{njukuta}} \item
‘little finger’ \textbf{\textit{imu watangïn}} \item
‘little toe’ \textbf{\textit{wutïmu watangïn}} \item
‘live’ \textit{v.} \textbf{\textit{p-}}, \textbf{\textit{wap}} \item
‘liver’ \textbf{\textit{ina}} \item
‘lizard’ \textbf{\textit{nïkït}} \item
‘lizard species’ \textbf{\textit{kayanmali}}, \textbf{\textit{nangu}}, \textbf{\textit{nataw}}, \textbf{\textit{simbïli}}, \textbf{\textit{wemana}}, \textbf{\textit{womba}} \item
‘located, be’ \textit{v.} \textbf{\textit{p-}}, \textbf{\textit{wap}} \item
‘log’ \textbf{\textit{imot}} \item
‘loincloth’ \textbf{\textit{al}} \item
‘long’ \textbf{\textit{ngaya}}, \textbf{\textit{wutota}} \item
‘look’ \textit{v.} \textbf{\textit{lïmndï ala-}} \item
‘look at’ \textit{v.} \textbf{\textit{lïmndï lï-}}, \textbf{\textit{lïmndï uta-}} \item
‘louse’ \textbf{\textit{mïmin}}, \textbf{\textit{sïmin}} \item
‘low’ \textbf{\textit{li}} \item
‘lower’ \textbf{\textit{li}} \item
‘lower leg’ \textbf{\textit{wutï anmot}} \item
‘lower lip’ \textbf{\textit{li tanum}} \item
‘lungs’ \textbf{\textit{woplota}}\\ \item

\noindent \textbf{M – m}\\ \item

‘machete’ \textbf{\textit{yawt}}, \textbf{\textit{yot}} \item
‘maggot’ \textbf{\textit{manjimanji}} \item
‘magic’ \textbf{\textit{amba}}, \textbf{\textit{ambet}}, \textbf{\textit{sawi}}, \textbf{\textit{tawi}} \item
‘maize’ \textbf{\textit{kon}} \item
‘\textit{makau}’ \textbf{\textit{mbatmbat}} \item
‘make’ \textit{v.} \textbf{\textit{ita-}}, \textbf{\textit{nambïnïkï-}}, \textbf{\textit{ul ni-}} \item
‘Malay apple’ \textbf{\textit{eklak}} \item
‘male’ \textbf{\textit{yata}}, \textbf{\textit{yeta}} \item
‘mallet’ \textbf{\textit{numbu motam}} \item
‘\textit{malo}’ \textbf{\textit{al}} \item
‘mama’ \textbf{\textit{nana}} \item
‘\textit{mami}’ \textbf{\textit{yena utam}} \item
‘man’ \textbf{\textit{itom}}, \textbf{\textit{yata}}, \textbf{\textit{yeta}} \item
‘mango’ \textbf{\textit{anïmbu}} \item
‘mantis’ \textbf{\textit{kalingana}} \item
‘many’ \textbf{\textit{nunu}}, \textbf{\textit{tïngïn}} \item
‘\textit{marmar}’ \textbf{\textit{monam}} \item
‘\textit{masalai}’ \textbf{\textit{sasi yena}} \item
‘mash’ \textit{v.} \textbf{\textit{nopal u-}} \item
‘mask’ \textbf{\textit{nambana}}, \textbf{\textit{nambana mwa}} \item
‘matches’ \textbf{\textit{apïn}} \item
‘\textit{mausgras pis}’ \textbf{\textit{may}} \item
‘may’ \textbf{\textit{ken}} \item
‘maybe’ \textbf{\textit{tap}} \item
‘me’ \textbf{\textit{nï=}} \item
‘me myself’ \textbf{\textit{nawa}}, \textbf{\textit{nuwe}} \item
‘meat’ \textbf{\textit{lam}}, \textbf{\textit{mupu}} \item
‘men’s house’ \textbf{\textit{amba}} \item
‘menstruation’ \textbf{\textit{iwïl}} \item
‘mesh’ \textbf{\textit{suwan}} \item
‘message’ \textbf{\textit{na}} \item
‘mhm’ \textbf{\textit{m}} \item
‘midday’ \textbf{\textit{ane}}, \textbf{\textit{ane wombam}} \item
‘middle’ \textbf{\textit{wombam}}, \textbf{\textit{wome}} \item
‘middle finger’ \textbf{\textit{imu wome}} \item
‘middle toe’ \textbf{\textit{wutïmu wome}} \item
‘milk’ \textbf{\textit{wol mïndam}} \item
‘millipede’ \textbf{\textit{aylat}} \item
‘mind’ \textbf{\textit{ina}} \item
‘mine’ \textbf{\textit{nïnji}} \item
‘mix of betel nut’ \textbf{\textit{ansi}} \item
‘molar’ \textbf{\textit{kat ambla}} \item
‘mold’ \textbf{\textit{momul}} \item
‘\textit{mon}’ \textbf{\textit{amla}} \item
‘money’ \textbf{\textit{ata monam mu}}, \textbf{\textit{inamba}}, \textbf{\textit{palapal}}, \textbf{\textit{wombasa anga}} \item
‘month’ \textbf{\textit{iwïl}} \item
‘moon’ \textbf{\textit{iwïl}}, \textbf{\textit{yawïl}} \item
‘\textit{moran}’ \textbf{\textit{anïmasi}} \item
‘moreover’ \textbf{\textit{maweka}}, \textbf{\textit{moweka}} \item
‘morning’ \textbf{\textit{umbenam}} \item
‘\textit{morota}’ \textbf{\textit{ila}} \item
‘mosquito’ \textbf{\textit{yangun}} \item
‘mosquito bite’ \textbf{\textit{mu}} \item
‘mosquito net’ \textbf{\textit{al}} \item
‘mosquito-swatter’ \textbf{\textit{tongan}} \item
‘mosquitofish’ \textbf{\textit{upa}} \item
‘mother’ \textbf{\textit{inom}}, \textbf{\textit{nana}} \item
‘motorboat’ \textbf{\textit{anaw}} \item
‘mound’ \textbf{\textit{yïwa}} \item
‘mountain’ \textbf{\textit{inkaw}} \item
‘mouth’ \textbf{\textit{mama}}, \textbf{\textit{tanum}} \item
‘mouth harp’ \textbf{\textit{sinangul}} \item
‘much’ \textbf{\textit{ambi}} \item
‘mucus’ \textbf{\textit{nïmïn}} \item
‘mud’ \textbf{\textit{mutulum}} \item
‘\textit{mumut}’ \textbf{\textit{wondi}} \item
‘\textit{muruk}’ \textbf{\textit{kalim}} \item
‘muscle’ \textbf{\textit{lam}} \item
‘mushroom’ \textbf{\textit{wanwane}} \item
‘must’ \textbf{\textit{mas}} \item
‘my’ \textbf{\textit{nïnji}} \item
‘my own’ \textbf{\textit{ambïnji}} \item
‘myself’ \textbf{\textit{ambawa=}}, \textbf{\textit{ambï=}}, \textbf{\textit{ambuwe}}\\ \item

\noindent \textbf{N – n}\\ \item

‘nah’ \textbf{\textit{asa}} \item
‘nail’ \textbf{\textit{sinanan}} \item
‘name’ \textbf{\textit{wi}} \item
‘nape’ \textbf{\textit{tumbunma}} \item
‘narrow’ \textbf{\textit{njukuta}} \item
‘navel’ \textbf{\textit{unet}} \item
‘near’ \textbf{\textit{kana}}, \textbf{\textit{kanam}}, \textbf{\textit{nu}} \item
‘neck’ \textbf{\textit{tumbunma}}, \textbf{\textit{um}} \item
‘needle’ \textbf{\textit{uma}} \item
‘nephew’ \textbf{\textit{ansi nungol}} \item
‘nest’ \textbf{\textit{kïka}}, \textbf{\textit{nim}}, \textbf{\textit{wawal}} \item
‘net’ \textbf{\textit{al}}, \textbf{\textit{ngin}} \item
‘net bag’ \textbf{\textit{ani}} \item
‘nettle’ \textbf{\textit{amendum}}, \textbf{\textit{mamnda}}, \textbf{\textit{yangusole}} \item
‘new’ \textbf{\textit{akïnaka}} \item
‘next day’ \textbf{\textit{wop}} \item
‘next of kin’ \textbf{\textit{tamndï}} \item
‘next to’ \textbf{\textit{kana}}, \textbf{\textit{kanam}} \item
‘nibling’ \textbf{\textit{ansi nungol}} \item
‘nice’ \textbf{\textit{anma}} \item
‘niece’ \textbf{\textit{ansi nungol}}, \textbf{\textit{ansi yanat}} \item
‘night’ \textbf{\textit{imba}} \item
‘\textit{nilpis}’ \textbf{\textit{lanjin}} \item
‘nine’ \textbf{\textit{angay kwe watangïnila ndïwon ndïwatlïp}} \item
‘nineteen’ \textbf{\textit{angay lele watangïnila ndïwon ndïwatlïp}} \item
‘ninety’ \textbf{\textit{ankam unduwan nali watangïnila}} \item
‘nipple’ \textbf{\textit{wolmu}} \item
‘no’ \textbf{\textit{ango}}, \textbf{\textit{ase}}, \textbf{\textit{nongat}} \item
‘none’ \textbf{\textit{ulwa}} \item
‘noon’ \textbf{\textit{ane wombam}} \item
‘nose’ \textbf{\textit{ip}} \item
‘nosering’ \textbf{\textit{asïmïna}} \item
‘not’ \textbf{\textit{ango}} \item
‘nothing’ \textbf{\textit{ulwa}} \item
‘now’ \textbf{\textit{amun}} \item
‘nowadays’ \textbf{\textit{amun}} \item
‘nut’ \textbf{\textit{mu}} \item
‘nut species’ \textbf{\textit{kawa}}, \textbf{\textit{lamban}}, \textbf{\textit{wambïn}}\\ \item

\noindent \textbf{O – o}\\ \item

‘occipital bone’ \textbf{\textit{akunpu}} \item
‘ocean’ \textbf{\textit{angumoni nïmal}} \item
‘odor’ \textbf{\textit{nambït}} \item
‘official’ \textbf{\textit{inangïnmana}} \item
‘often’ \textbf{\textit{nunu ika}} \item
‘OK’ \textbf{\textit{ande}}, \textbf{\textit{andi}}, \textbf{\textit{oke}} \item
‘okari nut’ \textbf{\textit{un}} \item
‘okay’ \textbf{\textit{ande}}, \textbf{\textit{andi}}, \textbf{\textit{oke}} \item
‘old’ \textbf{\textit{wapata}} \item
‘old man’ \textbf{\textit{itom ngata}} \item
‘old person’ \textbf{\textit{ngata}} \item
‘old woman’ \textbf{\textit{inom ngata}} \item
‘older brother’ \textbf{\textit{atuma}} \item
‘older sister’ \textbf{\textit{atana}} \item
‘\textit{olsem}’ \textbf{\textit{ka}}, \textbf{\textit{maka}}, \textbf{\textit{mïka}} \item
‘on’ \textbf{\textit{ka}} \item
‘on account of’ \textbf{\textit{nakap}}, \textbf{\textit{nap}} \item
‘one’ \textbf{\textit{kwa}}, \textbf{\textit{kwe}} \item
‘one another’ \textbf{\textit{ambla=}} \item
‘one by one’ \textbf{\textit{kekaka}}, \textbf{\textit{kwekaka}} \item
‘one each’ \textbf{\textit{kekaka}}, \textbf{\textit{kwekaka}} \item
‘one hundred’ \textbf{\textit{uta}}, \textbf{\textit{uta kwe}} \item
‘only’ \textbf{\textit{we}} \item
‘onto’ \textbf{\textit{wat}} \item
‘ooh’ \textbf{\textit{u}} \item
‘opening’ \textbf{\textit{mwa}} \item
‘or’ \textbf{\textit{o}} \item
‘orange’ \textbf{\textit{tondiway}} \item
‘orchid’ \textbf{\textit{wokomana}} \item
‘other’ \textbf{\textit{kwa}} \item
‘our’ \textbf{\textit{anji}}, \textbf{\textit{nganji}}, \textbf{\textit{ngunanji}}, \textbf{\textit{unanji}} \item
‘our own’ \textbf{\textit{ambinji}}, \textbf{\textit{amblanji}} \item
‘ours’ \textbf{\textit{anji}}, \textbf{\textit{nganji}}, \textbf{\textit{ngunanji}}, \textbf{\textit{unanji}} \item
‘ourselves’ \textbf{\textit{ambin=}}, \textbf{\textit{mbinawa}}, \textbf{\textit{ambinwe}}, \textbf{\textit{ambla=}}, \textbf{\textit{amblawa}}, \textbf{\textit{amblawe}} \item
‘out’ \textbf{\textit{an}} \item
‘outboard motor’ \textbf{\textit{anaw}} \item
‘outhouse’ \textbf{\textit{tumbu itïm}} \item
‘outside’ \textbf{\textit{anmbï}} \item
‘over’ \textbf{\textit{wan}} \item
‘ow’ \textbf{\textit{ay}} \item
‘owl’ \textbf{\textit{mamwapa}} \item
‘own’ \textbf{\textit{wo}} \item
‘owner’ \textbf{\textit{tamndï}}\\ \item

\noindent \textbf{P – p}\\ \item

‘Pacific walnut’ \textbf{\textit{amla}} \item
‘package’ \textbf{\textit{muku}} \item
‘packet’ \textbf{\textit{muku}} \item
‘paddle’ \textbf{\textit{anaw}} \item
‘pain’ \textbf{\textit{apïn}} \item
‘palm’ \textbf{\textit{anduwan}}, \textbf{\textit{mukuwi}}, \textbf{\textit{ulum}}, \textbf{\textit{wepal}} \item
‘palm flower’ \textbf{\textit{nambana}} \item
‘palm frond’ \textbf{\textit{akïnanga}}, \textbf{\textit{isi}}, \textbf{\textit{wema}} \item
‘palm of the hand’ \textbf{\textit{i mwa}}, \textbf{\textit{yombam}} \item
‘palm species’ \textbf{\textit{me}}, \textbf{\textit{mbïlanda}}, \textbf{\textit{ndukumbu}} \item
‘palm stem’ \textbf{\textit{me}} \item
‘pan’ \textbf{\textit{wombasa}} \item
‘pancake’ \textbf{\textit{we}} \item
‘pandanus’ \textbf{\textit{ndïl}} \item
‘\textit{pandol}’ \textbf{\textit{nisi}} \item
‘\textit{pangal}’ \textbf{\textit{isi}}, \textbf{\textit{wema}} \item
‘panpipes’ \textbf{\textit{wusimi}} \item
‘papa’ \textbf{\textit{tata}} \item
‘papaya’ \textbf{\textit{popo}} \item
‘paper’ \textbf{\textit{mïndapan}} \item
‘parasite’ \textbf{\textit{ana}} \item
‘parrot’ \textbf{\textit{ane uta}}, \textbf{\textit{awalawa}}, \textbf{\textit{langay}}, \textbf{\textit{moni}} \item
‘pass’ \textit{v.} \textbf{\textit{klop-}} \item
‘path’ \textbf{\textit{tïlwa}} \item
‘patrol officer’ \textbf{\textit{wanwane}} \item
‘pause’ \textit{v.} \textbf{\textit{wulïn u-}} \item
‘peace’ \textbf{\textit{yopa}} \item
‘peak’ \textbf{\textit{ka}} \item
‘peel’ \textit{v.} \textbf{\textit{nambuwe u-}}, \textbf{\textit{ulo-}} \item
‘penis’ \textbf{\textit{ansi}}, \textbf{\textit{won}} \item
‘penis gourd’ \textbf{\textit{ansi}} \item
‘pepper’ \textbf{\textit{kumblima}}, \textbf{\textit{nakam}}, \textbf{\textit{nganangan}}, \textbf{\textit{wanmbi}} \item
‘perceive’ \textit{v.} \textbf{\textit{wana-}} \item
‘perch’ \textbf{\textit{lanjin}} \item
‘perfume’ \textbf{\textit{yanïmana}} \item
‘person’ \textbf{\textit{ankam}} \item
‘perspiration’ \textbf{\textit{apïn inim}} \item
‘phlegm’ \textbf{\textit{utan}} \item
‘pick’ \textit{v.} \textbf{\textit{ango-}} \item
‘pick-axe’ \textbf{\textit{anasa}} \item
‘piece’ \textbf{\textit{anga}}, \textbf{\textit{at}}, \textbf{\textit{ilum}}, \textbf{\textit{pul}} \item
‘piece of wood’ \textbf{\textit{impul}} \item
‘pig’ \textbf{\textit{lamndu}}, \textbf{\textit{namndu}} \item
‘pigeon’ \textbf{\textit{membul walimot}} \item
‘pile’ \textbf{\textit{tumopa}} \item
‘pile’ \textit{v.} \textbf{\textit{kuk u-}} \item
‘pimple’ \textbf{\textit{andïpipi}} \item
‘pinky finger’ \textbf{\textit{imu watangïn}} \item
‘pinky toe’ \textbf{\textit{wutïmu watangïn}} \item
‘pith’ \textbf{\textit{ulum}} \item
‘pitiful’ \textbf{\textit{ngusuwa}} \item
‘\textit{pitpit}’ \textbf{\textit{palam}} \item
‘place’ \textbf{\textit{luwa}}, \textbf{\textit{pul}} \item
‘plane’ \textbf{\textit{mbalus}} \item
‘planet’ \textbf{\textit{mbomala}} \item
‘plant species’ \textbf{\textit{amendum}}, \textbf{\textit{anem}}, \textbf{\textit{law}}, \textbf{\textit{lindïn}}, \textbf{\textit{mamnda}}, \textbf{\textit{moniwot}}, \textbf{\textit{mungul}}, \textbf{\textit{namle}}, \textbf{\textit{netïl}}, \textbf{\textit{ngungun}}, \textbf{\textit{tondiway}}, \textbf{\textit{waenkïn}}, \textbf{\textit{wawana}}, \textbf{\textit{wokomana}}, \linebreak \textbf{\textit{yangusole}}, \textbf{\textit{yanïmana}} \item
‘plant’ \textit{v.} \textbf{\textit{lapa-}} \item
‘plate’ \textbf{\textit{uta}}, \textbf{\textit{wuta}} \item
‘platform’ \textbf{\textit{sakla}}, \textbf{\textit{wulis}} \item
‘play’ \textbf{\textit{tïnum}} \item
‘play’ \textit{v.} \textbf{\textit{sini-}} \item
‘please’ \textbf{\textit{kop}} \item
‘pointlessly’ \textbf{\textit{woyambïn}} \item
‘poison’ \textbf{\textit{ambet}} \item
‘police officer’ \textbf{\textit{wanwane}} \item
‘pond’ \textbf{\textit{inimpul}} \item
‘poor’ \textbf{\textit{ngusuwa}}, \textbf{\textit{tembi}} \item
‘poor thing’ \textbf{\textit{mangusuwa}}, \textbf{\textit{mangusuwata}} \item
‘poor things’ \textbf{\textit{mingusuwa}}, \textbf{\textit{mingusuwata}}, \textbf{\textit{ndïngusuwa}}, \textbf{\textit{ndïngusuwata}} \item
‘porch’ \textbf{\textit{apa mot}} \item
‘possum’ \textbf{\textit{ngwimakan}}, \textbf{\textit{tïponïm}} \item
‘post’ \textbf{\textit{anmot}}, \textbf{\textit{numbu}} \item
‘pot’ \textbf{\textit{kukumbe}}, \textbf{\textit{samban}}, \textbf{\textit{sïmbïn}}, \textbf{\textit{wemali}}, \textbf{\textit{wewun}}, \textbf{\textit{wombasa}} \item
‘pouch’ \textbf{\textit{ame}} \item
‘pour’ \textit{v.} \textbf{\textit{tomal u-}} \item
‘pray’ \textit{v.} \textbf{\textit{isi monombam u-}} \item
‘precede’ \textit{v.} \textbf{\textit{ip ka-}} \item
‘prepare’ \textit{v.} \textbf{\textit{nïka-}}, \textbf{\textit{nïkï-}}, \textbf{\textit{nka-}}, \textbf{\textit{nkï-}} \item
‘pressure’ \textit{v.} \textbf{\textit{ul ni-}} \item
‘problem’ \textbf{\textit{kenmbu}} \item
‘pull’ \textit{v.} \textbf{\textit{angom lï-}} \item
‘pull out’ \textit{v.} \textbf{\textit{ango-}}, \textbf{\textit{angom lï-}} \item
‘pulp’ \textbf{\textit{mupu}} \item
‘pupil’ \textbf{\textit{lïmndï mu}} \item
‘purple’ \textbf{\textit{anem}} \item
‘\textit{purpur}’ \textbf{\textit{ana}}, \textbf{\textit{moniwot}} \item
‘pus’ \textbf{\textit{mïndam}} \item
‘push’ \textit{v.} \textbf{\textit{s}}, \textbf{\textit{si-}} \item
‘put’ \textit{v.} \textbf{\textit{atalï-}}, \textbf{\textit{aw}}, \textbf{\textit{l}}, \textbf{\textit{lï-}}, \textbf{\textit{lumo-}}, \textbf{\textit{u-}}, \textbf{\textit{umo-}} \item
‘put in’ \textit{v.} \textbf{\textit{inu-}} \item
‘put up’ \textit{v.} \textbf{\textit{atalï-}} \item
‘python’ \textbf{\textit{anïmasi}}\\ \item

\noindent \textbf{Q – q}\\ \item

‘question’ \textbf{\textit{atwana}} \item
‘quickly’ \textbf{\textit{mbuka}} \item
‘quiet’ \textbf{\textit{andïl}}\\ \item

\noindent \textbf{R – r}\\ \item

‘rack’ \textbf{\textit{suwan}} \item
‘raft’ \textbf{\textit{wulis}} \item
‘rain’ \textbf{\textit{inim}} \item
‘rain tree’ \textbf{\textit{monam}} \item
‘rain’ \textit{v.} \textbf{\textit{lopo-}} \item
‘rainbow’ \textbf{\textit{anem nambum}} \item
‘rainbow fish’ \textbf{\textit{wowaw}} \item
‘rainy season’ \textbf{\textit{inim}} \item
‘rash’ \textbf{\textit{yambola}} \item
‘rat species’ \textbf{\textit{matlaka}}, \textbf{\textit{mblandu}}, \textbf{\textit{wala}} \item
‘rattan’ \textbf{\textit{le}} \item
‘raw’ \textbf{\textit{akïnaka}} \item
‘really’ \textbf{\textit{apka}} \item
‘rear’ \textbf{\textit{angani}}, \textbf{\textit{unmbï}} \item
‘reason’ \textbf{\textit{na}} \item
‘recently’ \textbf{\textit{amun}} \item
‘red’ \textbf{\textit{ngungun}} \item
‘red ant’ \textbf{\textit{ngungun}} \item
‘\textit{red buai}’ \textbf{\textit{ansi}} \item
‘reeds’ \textbf{\textit{awindal}} \item
‘refuse’ \textbf{\textit{imbïn}}, \textbf{\textit{utïl}} \item
‘regularly’ \textbf{\textit{nunu ika}} \item
‘relax’ \textit{v.} \textbf{\textit{wulïn u-}} \item
‘reside’ \textit{v.} \textbf{\textit{p-}}, \textbf{\textit{wap}} \item
‘rest’ \textit{v.} \textbf{\textit{wulïn u-}} \item
‘ribs’ \textbf{\textit{wal}} \item
‘rice’ \textbf{\textit{asimu}} \item
‘right’ \textbf{\textit{inapum}}, \textbf{\textit{maw}} \item
‘right-hand’ \textbf{\textit{inapum}} \item
‘ring’ \textbf{\textit{asïmïna}}, \textbf{\textit{mungun}} \item
‘ring finger’ \textbf{\textit{imu law}} \item
‘ringworm’ \textbf{\textit{akal}} \item
‘ripe’ \textbf{\textit{mïnwata}} \item
‘rites’ \textbf{\textit{sasi}} \item
‘river’ \textbf{\textit{nïmal}} \item
‘river snail’ \textbf{\textit{manana}} \item
‘riverbank’ \textbf{\textit{ika}} \item
‘road’ \textbf{\textit{tïlwa}} \item
‘rock’ \textbf{\textit{tana}} \item
‘roof’ \textbf{\textit{apaka}} \item
‘root’ \textbf{\textit{ilu}}, \textbf{\textit{tïmal}} \item
‘\textit{rop daka}’ \textbf{\textit{wanmbi mutam}} \item
‘rope’ \textbf{\textit{nïpïl}} \item
‘rotten’ \textbf{\textit{mïnwata}} \item
‘rotting’ \textbf{\textit{alata}}, \textbf{\textit{mïnap}}, \textbf{\textit{mïnwata}} \item
‘round’ \textbf{\textit{wopaw}} \item
‘rub’ \textit{v.} \textbf{\textit{uta-}} \item
‘rubbish’ \textbf{\textit{itïm}}, \textbf{\textit{umba}} \item
‘run’ \textit{v.} \textbf{\textit{imbam ka-}}\\ \item

\noindent \textbf{S – s}\\ \item

‘sago’ \textbf{\textit{ay}} \item
‘sago flour’ \textbf{\textit{we}} \item
‘sago jelly’ \textbf{\textit{ay}}, \textbf{\textit{aypul}} \item
‘sago palm’ \textbf{\textit{anduwan}}, \textbf{\textit{mukuwi}}, \textbf{\textit{ulum}}, \textbf{\textit{wepal}} \item
‘sago pancake’ \textbf{\textit{we}} \item
‘sago pith’ \textbf{\textit{ulum}} \item
‘sago shoot’ \textbf{\textit{andï}}, \textbf{\textit{wan}} \item
‘sago species’ \textbf{\textit{alaman}}, \textbf{\textit{inimndum}}, \textbf{\textit{kukumbe}}, \textbf{\textit{mïnkïn ulum}}, \textbf{\textit{nïndiwe}}, \textbf{\textit{nïnil}}, \textbf{\textit{nowe}}, \textbf{\textit{tambïn ulum}} \item
‘sago starch’ \textbf{\textit{we}} \item
‘sago stick’ \textbf{\textit{aymoma}} \item
‘sago strainer’ \textbf{\textit{yawa}} \item
‘\textit{salat}’ \textbf{\textit{amendum}}, \textbf{\textit{mamnda}}, \textbf{\textit{yangusole}} \item
‘saliva’ \textbf{\textit{sawi}} \item
‘salt’ \textbf{\textit{isi}} \item
‘sand’ \textbf{\textit{tana isi}} \item
‘sap’ \textbf{\textit{im kal}} \item
‘sated’ \textbf{\textit{monop}} \item
‘say’ \textit{v.} \textbf{\textit{ka-}}, \textbf{\textit{kï-}}, \textbf{\textit{t}}, \textbf{\textit{ta-}} \item
‘scab’ \textbf{\textit{tawatal}} \item
‘scabies’ \textbf{\textit{yambola}} \item
‘scale’ \textbf{\textit{wowaw}} \item
‘scalp’ \textbf{\textit{unduwan nambï}} \item
‘scar’ \textbf{\textit{mbun}} \item
‘scared’ \textbf{\textit{namna}} \item
‘scared, be’ \textit{v.} \textbf{\textit{namnap}} \item
‘scarf’ \textbf{\textit{ayna}} \item
‘school’ \textbf{\textit{sikul}} \item
‘scissors’ \textbf{\textit{nangïn}} \item
‘scoop’ \textbf{\textit{aypul}} \item
‘scoop’ \textit{v.} \textbf{\textit{uta-}} \item
‘scorpion’ \textbf{\textit{alsa}} \item
‘scrape’ \textit{v.} \textbf{\textit{ale-}} \item
‘scratch’ \textit{v.} \textbf{\textit{ana-}} \item
‘scrotum’ \textbf{\textit{mïtïn ame}} \item
‘scrub’ \textit{v.} \textbf{\textit{ana-}} \item
‘sea’ \textbf{\textit{angumoni nïmal}} \item
‘second toe’ \textbf{\textit{wutïmu ankam}} \item
‘see’ \textit{v.} \textbf{\textit{lïmndï ala-}} \item
‘seed’ \textbf{\textit{mu}} \item
‘seed species’ \textbf{\textit{awame}} \item
‘seedling’ \textbf{\textit{nangum}} \item
‘seek’ \textit{v.} \textbf{\textit{andïlalo-}}, \textbf{\textit{anglalo-}} \item
‘segment’ \textbf{\textit{wawat}} \item
‘semen’ \textbf{\textit{won inim}} \item
‘send’ \textit{v.} \textbf{\textit{kali lï-}} \item
‘sense’ \textit{v.} \textbf{\textit{wana-}} \item
‘seven’ \textbf{\textit{angay kwe nini minwon ndïwatlïp}} \item
‘seventeen’ \textbf{\textit{angay lele nini minwon ndïwatlïp}} \item
‘seventy’ \textbf{\textit{ankam unduwan nali nini}} \item
‘sew’ \textit{v.} \textbf{\textit{me-}} \item
‘shade’ \textbf{\textit{ndande}} \item
‘shadow’ \textbf{\textit{ndande}} \item
‘shaft’ \textbf{\textit{yokam}} \item
‘sharp’ \textbf{\textit{kïkalsina}}, \textbf{\textit{matamal}} \item
‘she’ \textbf{\textit{mï}} \item
‘she is the one’ \textbf{\textit{mïnam}} \item
‘sheet’ \textbf{\textit{al nambï}} \item
‘shelf’ \textbf{\textit{aplatam}}, \textbf{\textit{apot}} \item
‘shell’ \textbf{\textit{palapal}}, \textbf{\textit{tïl}}, \textbf{\textit{uta}}, \textbf{\textit{wokomana}}, \textbf{\textit{wuta}} \item
‘shield’ \textbf{\textit{wanam}} \item
‘shin’ \textbf{\textit{wutï anmot}} \item
‘shoe’ \textbf{\textit{yakam}} \item
‘shoot’ \textbf{\textit{andï}}, \textbf{\textit{nangum}}, \textbf{\textit{pal}}, \textbf{\textit{pat}}, \textbf{\textit{wan}} \item
‘shoot’ \textit{v.} \textbf{\textit{as}}, \textbf{\textit{asa-}}, \textbf{\textit{atï-}}, \textbf{\textit{wali-}} \item
‘short’ \textbf{\textit{mundotoma}} \item
‘should’ \textbf{\textit{mas}} \item
‘shoulder’ \textbf{\textit{awi}} \item
‘shout’ \textit{v.} \textbf{\textit{uni-}} \item
‘shovel’ \textbf{\textit{mae}} \item
‘show’ \textit{v.} \textbf{\textit{=n ul si-}} \item
‘shun’ \textit{v.} \textbf{\textit{kamb-}} \item
‘sibling’ \textbf{\textit{wot}} \item
‘sick’ \textbf{\textit{tembi}} \item
‘sickness’ \textbf{\textit{tembi}} \item
‘side’ \textbf{\textit{anga}}, \textbf{\textit{awi}}, \textbf{\textit{wanam}} \item
‘\textit{sikau}’ \textbf{\textit{wakan}} \item
‘simply’ \textbf{\textit{ko}}, \textbf{\textit{kop}}, \textbf{\textit{kwa}}, \textbf{\textit{wa}} \item
‘sing’ \textit{v.} \textbf{\textit{kawni-}} \item
‘Singapore taro’ \textbf{\textit{atate}} \item
‘\textit{singsing}’ \textbf{\textit{kaw}} \item
‘sister’ \textbf{\textit{anapa}}, \textbf{\textit{atana}}, \textbf{\textit{wot yana}}, \textbf{\textit{wot yena}} \item
‘sister-in-law’ \textbf{\textit{atuma inga yena}}, \textbf{\textit{inga yena}}, \textbf{\textit{wot inga yena}} \item
‘sit’ \textit{v.} \textbf{\textit{asi ka-}} \item
‘sit down’ \textit{v.} \textbf{\textit{asi ka-}} \item
‘six’ \textbf{\textit{angay kwe kwe mowon ndïwatlïp}} \item
‘sixteen’ \textbf{\textit{angay lele kwe mowon ndïwatlïp}} \item
‘sixty’ \textbf{\textit{ankam unduwan nali}} \item
‘skin’ \textbf{\textit{nambï}} \item
‘skirt’ \textbf{\textit{al}}, \textbf{\textit{ana}}, \textbf{\textit{anapot}}, \textbf{\textit{wopana}} \item
‘skull’ \textbf{\textit{akunpu}} \item
‘sky’ \textbf{\textit{anam}} \item
‘sleep’ \textit{v.} \textbf{\textit{lowo-}}, \textbf{\textit{wo-}}, \textbf{\textit{wow}} \item
‘slit drum’ \textbf{\textit{numbu}} \item
‘slow’ \textbf{\textit{andïl}} \item
‘small’ \textbf{\textit{njukuta}}, \textbf{\textit{tïke}} \item
‘smart’ \textbf{\textit{anma}} \item
‘smell’ \textbf{\textit{nambït}} \item
‘smell’ \textit{v.} \textbf{\textit{nambït wana-}} \item
‘smoke’ \textbf{\textit{apïn ngïn}} \item
‘smoke’ \textit{v.} \textbf{\textit{ama-}}, \textbf{\textit{la-}}, \textbf{\textit{mondo-}} \item
‘smooth’ \textbf{\textit{namli}} \item
‘snail species’ \textbf{\textit{manana}} \item
‘snake’ \textbf{\textit{anmoka}} \item
‘snake species’ \textbf{\textit{anïmasi}}, \textbf{\textit{malalïwa}}, \textbf{\textit{mïnal anmoka}}, \textbf{\textit{ngum}}, \textbf{\textit{yoma}} \item
‘sneeze’ \textbf{\textit{asïmïna}} \item
‘sniff’ \textit{v.} \textbf{\textit{nambït wana-}} \item
‘snore’ \textbf{\textit{ip nonal}} \item
‘soft’ \textbf{\textit{namli}} \item
‘soil’ \textbf{\textit{ini}} \item
‘sole of the foot’ \textbf{\textit{wutï yombam}} \item
‘some’ \textbf{\textit{kuma}} \item
‘someone’ \textbf{\textit{kwa}} \item
‘something’ \textbf{\textit{nji}} \item
‘son’ \textbf{\textit{nungol}}, \textbf{\textit{nungolke}}, \textbf{\textit{yatalum}}, \textbf{\textit{yetalum}} \item
‘song’ \textbf{\textit{kaw}} \item
‘soon’ \textbf{\textit{anganika}}, \textbf{\textit{naka}} \item
‘sore’ \textbf{\textit{tawa}} \item
‘soup’ \textbf{\textit{isi}} \item
‘spade’ \textbf{\textit{lemta}}, \textbf{\textit{mae}} \item
‘speak’ \textit{v.} \textbf{\textit{ka-}}, \textbf{\textit{kï-}}, \textbf{\textit{t}}, \textbf{\textit{ta-}} \item
‘spear’ \textbf{\textit{lungum}}, \textbf{\textit{mana}}, \textbf{\textit{nap}} \item
‘speech’ \textbf{\textit{mïnja}}, \textbf{\textit{na}} \item
‘spider’ \textbf{\textit{ingwa}} \item
‘spider web’ \textbf{\textit{lingïnane}} \item
‘spine’ \textbf{\textit{anangum}}, \textbf{\textit{mutoma}}, \textbf{\textit{nali}}, \textbf{\textit{nin}} \item
‘spirit house’ \textbf{\textit{amba}} \item
‘spirit mask’ \textbf{\textit{nambana}} \item
‘spirit, type of’ \textbf{\textit{inimnji}}, \textbf{\textit{metmet}}, \textbf{\textit{molpan}}, \textbf{\textit{nambana}}, \textbf{\textit{sasi yena}}, \textbf{\textit{yambalpa}} \item
‘spit’ \textbf{\textit{sawi}} \item
‘spit’ \textit{v.} \textbf{\textit{ngom lï-}} \item
‘splinter’ \textbf{\textit{mi}} \item
‘split’ \textit{v.} \textbf{\textit{kol-}} \item
‘spoiled’ \textbf{\textit{mïnwata}} \item
‘spoon’ \textbf{\textit{ametamal}} \item
‘sprout’ \textbf{\textit{tangam}} \item
‘squeeze’ \textit{v.} \textbf{\textit{mïmïl u-}} \item
‘stab’ \textit{v.} \textbf{\textit{as}}, \textbf{\textit{asa-}}, \textbf{\textit{atï-}}, \textbf{\textit{wali-}} \item
‘stalk’ \textbf{\textit{sina}}, \textbf{\textit{wan}} \item
‘stand’ \textbf{\textit{nom}} \item
‘stand’ \textit{v.} \textbf{\textit{tane lï-}}, \textbf{\textit{tïnanga-}} \item
‘stand up’ \textit{v.} \textbf{\textit{tane lï-}} \item
‘star’ \textbf{\textit{nali}}, \textbf{\textit{mbomala}} \item
‘starch’ \textbf{\textit{we}} \item
‘statuette’ \textbf{\textit{molombi}} \item
‘stay’ \textit{v.} \textbf{\textit{p-}}, \textbf{\textit{wap}} \item
‘steal’ \textit{v.} \textbf{\textit{mokum moko-}} \item
‘stealth’ \textbf{\textit{mokum}} \item
‘step’ \textbf{\textit{awlu}} \item
‘steps’ \textbf{\textit{wat}} \item
‘sternum’ \textbf{\textit{tambeta}} \item
‘stick’ \textbf{\textit{anïm}}, \textbf{\textit{aymoma im nali}}, \textbf{\textit{manangum}}, \textbf{\textit{motam}} \item
‘still’ \textbf{\textit{amun}} \item
‘stinger’ \textbf{\textit{ambla}} \item
‘stinging nettle’ \textbf{\textit{amendum}}, \textbf{\textit{mamnda}}, \textbf{\textit{yangusole}} \item
‘stomach’ \textbf{\textit{umbopa}} \item
‘s\isi{tone}’ \textbf{\textit{tana}} \item
‘s\isi{tone} axe’ \textbf{\textit{tanatmu}} \item
‘story’ \textbf{\textit{misimisi}}, \textbf{\textit{na}} \item
‘stove’ \textbf{\textit{twa}} \item
‘straight’ \textbf{\textit{anma}} \item
‘strain’ \textit{v.} \textbf{\textit{mïmïl u-}} \item
‘strainer’ \textbf{\textit{imnde}}, \textbf{\textit{yawa}} \item
‘strand’ \textbf{\textit{mi}} \item
‘stranger’ \textbf{\textit{mbalanji}} \item
‘strap’ \textbf{\textit{mïnïm}}, \textbf{\textit{wam}} \item
‘strap of a bag’ \textbf{\textit{mïnïm}} \item
‘stretcher’ \textbf{\textit{sakla}} \item
‘string’ \textbf{\textit{asiya}} \item
‘string bag’ \textbf{\textit{ani}}, \textbf{\textit{ayna}} \item
‘strong’ \textbf{\textit{yangle}} \item
‘suck’ \textit{v.} \textbf{\textit{ama-}}, \textbf{\textit{la-}} \item
‘sucker’ \textbf{\textit{nungum}} \item
‘suffice’ \textit{v.} \textbf{\textit{nakamb-}} \item
‘sugar’ \textbf{\textit{mil}} \item
‘sugar glider’ \textbf{\textit{yawïn}} \item
‘sugarcane’ \textbf{\textit{mil}} \item
‘summon’ \textit{v.} \textbf{\textit{wanawni-}} \item
‘sun’ \textbf{\textit{ane}} \item
‘swamp’ \textbf{\textit{apïnal}}, \textbf{\textit{mïka itïm}} \item
‘swatter’ \textbf{\textit{tongan}} \item
‘sweat’ \textbf{\textit{apïn inim}} \item
‘sweep’ \textit{v.} \textbf{\textit{pop lï-}} \item
‘sweet’ \textbf{\textit{yangïmot}} \item
‘sweet potato’ \textbf{\textit{nongontam}} \item
‘swell’ \textit{v.} \textbf{\textit{wo-}} \item
‘swelling’ \textbf{\textit{angumoni}} \item
‘swim’ \textit{v.} \textbf{\textit{inim mo ma-}} \item
‘sword grass’ \textbf{\textit{nipum}}\\ \item

\noindent \textbf{T – t}\\ \item

‘table’ \textbf{\textit{aplatam}} \item
‘tadpole’ \textbf{\textit{wolname}} \item
‘tail’ \textbf{\textit{anaw}}, \textbf{\textit{angun}} \item
‘tail feather’ \textbf{\textit{tal}} \item
‘take’ \textit{v.} \textbf{\textit{moko-}}, \textbf{\textit{t}}, \textbf{\textit{tï-}} \item
‘take one-by-one’ \textit{v.} \textbf{\textit{moko-}} \item
‘\textit{talis}’ \textbf{\textit{un}} \item
‘talk’ \textbf{\textit{na}} \item
‘talk’ \textit{v.} \textbf{\textit{ka-}}, \textbf{\textit{kï-}}, \textbf{\textit{t}}, \textbf{\textit{ta-}} \item
‘tall’ \textbf{\textit{wutota}} \item
‘tall ginger’ \textbf{\textit{mïli}} \item
‘\textit{tambu}’ \textbf{\textit{inga}} \item
‘\textit{tanget}’ \textbf{\textit{law}} \item
‘\textit{tarangau}’ \textbf{\textit{amangala}}, \textbf{\textit{awsingïn}} \item
‘\textit{target}’ \textbf{\textit{unda}} \item
‘taro’ \textbf{\textit{mïnal}} \item
‘taro species’ \textbf{\textit{ulumbi}} \item
‘task’ \textbf{\textit{wombïn}} \item
‘taste’ \textit{v.} \textbf{\textit{wana-}} \item
‘tasty’ \textbf{\textit{yangïmot}} \item
‘tattoo’ \textbf{\textit{mak}} \item
‘teach’ \textit{v.} \textbf{\textit{=n =n kalam me-}} \item
‘tear’ \textbf{\textit{lïmndï inim}}, \textbf{\textit{sal}} \item
‘teardrop’ \textbf{\textit{lïmndï inim}}, \textbf{\textit{sal}} \item
‘tell’ \textit{v.} \textbf{\textit{ka-}}, \textbf{\textit{kï-}}, \textbf{\textit{t-}}, \textbf{\textit{ta-}} \item
‘temple’ \textbf{\textit{kïkal indam}} \item
‘ten’ \textbf{\textit{angay nini}}, \textbf{\textit{nali}} \item
‘tendon’ \textbf{\textit{ndal}} \item
‘termite’ \textbf{\textit{kïka}} \item
‘testicle’ \textbf{\textit{mïtïn}} \item
‘thank you’ \textbf{\textit{nïnji anma}} \item
‘thanks’ \textbf{\textit{nïnji anma}} \item
‘that’ \textbf{\textit{anda}}, \textbf{\textit{anda=}}, \textbf{\textit{nda}}, \textbf{\textit{nda=}} \item
‘that is it’ \textbf{\textit{andanam}} \item
‘that itself’ \textbf{\textit{andawa}}, \textbf{\textit{andawe}} \item
‘that one’s’ \textbf{\textit{andanji}} \item
‘that’s it’ \textbf{\textit{mawnam}} \item
‘thatch’ \textbf{\textit{ila}} \item
‘their’ \textbf{\textit{minji}}, \textbf{\textit{ndïnji}} \item
‘their own’ \textbf{\textit{ambinji}}, \textbf{\textit{amblanji}} \item
‘theirs’ \textbf{\textit{minji}}, \textbf{\textit{ndïnji}} \item
‘them’ \textbf{\textit{min=}}, \textbf{\textit{mini=}}, \textbf{\textit{ndï=}} \item
‘them two’ \textbf{\textit{min=}}, \textbf{\textit{mini=}} \item
‘themselves’ \textbf{\textit{ambin=}}, \textbf{\textit{ambinawa}}, \textbf{\textit{ambinwe}}, \textbf{\textit{ambla=}}, \textbf{\textit{amblawa}}, \textbf{\textit{amblawe}}, \linebreak \textbf{\textit{minawa}}, \textbf{\textit{minwe}}, \textbf{\textit{ndawa}}, \textbf{\textit{nduwe}} \item
‘then’ \textbf{\textit{we}} \item
‘thence’ \textbf{\textit{ando}} \item
‘there’ \textbf{\textit{anda}}, \textbf{\textit{ando}} \item
‘these’ \textbf{\textit{ngala}}, \textbf{\textit{ngala=}}, \textbf{\textit{ngin}}, \textbf{\textit{ngin=}} \item
‘these ones’’ \textbf{\textit{ngalanji}}, \textbf{\textit{nginji}} \item
‘these themselves’ \textbf{\textit{ngalawa}}, \textbf{\textit{ngalawe}}, \textbf{\textit{nginawa}}, \textbf{\textit{nginwe}} \item
‘these two’ \textbf{\textit{ngin}}, \textbf{\textit{ngin=}} \item
‘they’ \textbf{\textit{min}}, \textbf{\textit{ndin}}, \textbf{\textit{ndï}} \item
‘they are the ones’ \textbf{\textit{ndïnam}} \item
‘they two’ \textbf{\textit{ndin}}, \textbf{\textit{min}} \item
‘thick’ \textbf{\textit{palmana}} \item
‘thigh’ \textbf{\textit{wutï name}} \item
‘thin’ \textbf{\textit{njukuta}} \item
‘thing’ \textbf{\textit{nji}} \item
‘think’ \textit{v.} \textbf{\textit{inakawana-}}, \textbf{\textit{ka-}}, \textbf{\textit{kï-}}, \textbf{\textit{t}}, \textbf{\textit{ta-}}, \textbf{\textit{wana-}} \item
‘thirteen’ \textbf{\textit{angay nini lele ndïwon ndïwatlïp}}, \textbf{\textit{nali kwe lele}} \item
‘thirty’ \textbf{\textit{nali lele}} \item
‘this’ \textbf{\textit{nga}}, \textbf{\textit{nga=}} \item
‘this is it’ \textbf{\textit{ngam}} \item
‘this itself’ \textbf{\textit{ngawa}}, \textbf{\textit{ngawe}} \item
‘this one’s’ \textbf{\textit{nganji}} \item
‘thither’ \textbf{\textit{anda}} \item
‘thorn’ \textbf{\textit{nap}}, \textbf{\textit{nin}} \item
‘those’ \textbf{\textit{ala}}, \textbf{\textit{ala=}}, \textbf{\textit{andin}}, \textbf{\textit{andin=}}, \textbf{\textit{la}}, \textbf{\textit{la=}}, \textbf{\textit{ndin}}, \textbf{\textit{ndin=}} \item
‘those ones’’ \textbf{\textit{alanji}}, \textbf{\textit{andinji}} \item
‘those themselves’ \textbf{\textit{alawa}}, \textbf{\textit{alawe}}, \textbf{\textit{andinawa}}, \textbf{\textit{andinwe}} \item
‘those two’ \textbf{\textit{andin}}, \textbf{\textit{andin=}}, \textbf{\textit{ndin}}, \textbf{\textit{ndin=}} \item
‘thought’ \textbf{\textit{na}} \item
‘thread’ \textbf{\textit{asiya}} \item
‘three’ \textbf{\textit{lele}} \item
‘three hundred’ \textbf{\textit{uta lele}} \item
‘throat’ \textbf{\textit{aninokam}}, \textbf{\textit{mota}} \item
‘throw’ \textit{v.} \textbf{\textit{kïke u-}}, \textbf{\textit{kuli lï-}}, \textbf{\textit{mune u-}}, \textbf{\textit{top lï-}} \item
‘thumb’ \textbf{\textit{imu unduwan}} \item
‘thunder’ \textbf{\textit{anam wapata}} \item
‘thus’ \textbf{\textit{ka}}, \textbf{\textit{maka}}, \textbf{\textit{mïka}} \item
‘ti plant’ \textbf{\textit{law}} \item
‘tie’ \textit{v.} \textbf{\textit{ita-}}, \textbf{\textit{mop lï-}} \item
‘\textit{tiktik}’ \textbf{\textit{awindal}} \item
‘tilapia’ \textbf{\textit{mbatmbat}} \item
‘time’ \textbf{\textit{nunu}}, \textbf{\textit{tem}} \item
‘tinea’ \textbf{\textit{akal}} \item
‘tip’ \textbf{\textit{ana}}, \textbf{\textit{mu}} \item
‘to’ \textbf{\textit{iya}} \item
‘to here’ \textbf{\textit{mbï}} \item
‘to there’ \textbf{\textit{anda}} \item
‘tobacco’ \textbf{\textit{sokoy}} \item
‘tobacco species’ \textbf{\textit{way sokoy}} \item
‘today’ \textbf{\textit{amun}} \item
‘toe’ \textbf{\textit{wutïmu}} \item
‘toenail’ \textbf{\textit{wutï sinanan}} \item
‘toilet’ \textbf{\textit{tumbu itïm}} \item
‘\textit{tokples}’ \textbf{\textit{mïtïn}} \item
‘tomorrow’ \textbf{\textit{umbe}} \item
‘\textit{ton}’ \textbf{\textit{lemetam}} \item
‘tongs’ \textbf{\textit{apïn nangïn}}, \textbf{\textit{nangïn}}, \textbf{\textit{we nangïn}} \item
‘tongue’ \textbf{\textit{mïnïm}} \item
‘too’ \textbf{\textit{luke}} \item
‘tooth’ \textbf{\textit{ambla}} \item
‘top’ \textbf{\textit{wat}} \item
‘top of the foot’ \textbf{\textit{wutï mutam}} \item
‘torch’ \textbf{\textit{anenisi}} \item
‘toward’ \textbf{\textit{iya}} \item
‘track’ \textbf{\textit{tïlwa}} \item
‘trail’ \textbf{\textit{tïlwa}} \item
‘trap’ \textbf{\textit{asiya}}, \textbf{\textit{iwa}}, \textbf{\textit{ngin}}, \textbf{\textit{tukul}} \item
‘trash’ \textbf{\textit{itïm}}, \textbf{\textit{umba}} \item
‘treaty’ \textbf{\textit{yopa}} \item
‘tree’ \textbf{\textit{im}} \item
‘tree species’ \textbf{\textit{amla}}, \textbf{\textit{awe}}, \textbf{\textit{awnaka}}, \textbf{\textit{eklak}}, \textbf{\textit{lemetam}}, \textbf{\textit{masamasa}}, \textbf{\textit{mïka}}, \linebreak \textbf{\textit{monam}}, \textbf{\textit{mutam}}, \textbf{\textit{nanïm}}, \textbf{\textit{nokosam}}, \textbf{\textit{numbu}}, \textbf{\textit{un}}, \textbf{\textit{uwe}}, \textbf{\textit{wasi}}, \textbf{\textit{womba}}, \textbf{\textit{yambi}} \item
‘tree spirit’ \textbf{\textit{molpan}} \item
‘true’ \textbf{\textit{anma}} \item
‘trunk’ \textbf{\textit{wome}} \item
‘\textit{tulip}’ \textbf{\textit{anmopa}} \item
‘turn’ \textit{v.} \textbf{\textit{tïkli ka-}} \item
‘turn around’ \textit{v.} \textbf{\textit{tïkli ka-}} \item
‘turtle’ \textbf{\textit{way}} \item
‘tusk’ \textbf{\textit{wonmbi}} \item
‘twelve’ \textbf{\textit{angay nini nini minwon ndïwatlïp}}, \textbf{\textit{nali kwe nini}} \item
‘twenty’ \textbf{\textit{angay watangïnila}}, \textbf{\textit{lamndu unduwan}}, \textbf{\textit{nali nini}} \item
‘twenty-five’ \textbf{\textit{angay angay}}, \textbf{\textit{nali nini angay}} \item
‘two’ \textbf{\textit{nini}} \item
\largerpage
‘two hundred’ \textbf{\textit{uta nini}}\\ \item

\noindent \textbf{U – u}\\ \item

‘uh’ \textbf{\textit{a}} \item
‘uh-uh’ \textbf{\textit{mm}} \item
‘umbilical cord’ \textbf{\textit{unet}} \item
‘uncle’ \textbf{\textit{itom}}, \textbf{\textit{itom ambi}}, \textbf{\textit{itom atuma}}, \textbf{\textit{itom wot}}, \textbf{\textit{ngata yawa}}, \textbf{\textit{yawa}}, \textbf{\textit{yawa ambi}}, \textbf{\textit{yawa atuma}}, \textbf{\textit{yawa wot}} \item
‘under’ \textbf{\textit{imbam}} \item
‘unite’ \textit{v.} \textbf{\textit{kuk u-}} \item
‘unwrap’ \textit{v.} \textbf{\textit{kanaka lumo-}} \item
‘up’ \textbf{\textit{ata}} \item
‘upper’ \textbf{\textit{ata}} \item
‘upper arm’ \textbf{\textit{i name}} \item
‘upper leg’ \textbf{\textit{wutï name}} \item
‘upper lip’ \textbf{\textit{ata tanum}} \item
‘uproot’ \textit{v.} \textbf{\textit{angom lï-}} \item
‘upstream’ \textbf{\textit{ata}} \item
‘upward’ \textbf{\textit{ata}} \item
‘urine’ \textbf{\textit{minam}} \item
‘us’ \textbf{\textit{an=}}, \textbf{\textit{ngan=}}, \textbf{\textit{ngunan=}}, \textbf{\textit{unan=}} \item
‘us ourselves’ \textbf{\textit{anawa}}, \textbf{\textit{anwe}}, \textbf{\textit{nganawa}}, \textbf{\textit{nganwe}}, \textbf{\textit{ngunanawa}}, \textbf{\textit{ngunanwe}}, \linebreak \textbf{\textit{unanawa}}, \textbf{\textit{unanwe}} \item
‘us two’ \textbf{\textit{ngan=}}, \textbf{\textit{ngunan=}} \item
‘uterus’ \textbf{\textit{ame}}\\ \item

\noindent \textbf{V – v}\\ \item

‘vagina’ \textbf{\textit{inmbï}} \item
‘various’ \textbf{\textit{nunu}} \item
‘vegetable’ \textbf{\textit{natnat}} \item
‘vegetable species’ \textbf{\textit{anat}}, \textbf{\textit{ankïn}}, \textbf{\textit{anmopa}}, \textbf{\textit{mambun}}, \textbf{\textit{mïli}}, \textbf{\textit{wandana}}, \textbf{\textit{yomal}}, \textbf{\textit{yomba}} \item
‘vegetables’ \textbf{\textit{natnat}} \item
‘vein’ \textbf{\textit{ndal}} \item
‘venom’ \textbf{\textit{tawi}} \item
‘veranda’ \textbf{\textit{apa mot}} \item
‘very’ \textbf{\textit{apka}} \item
‘very own’ \textbf{\textit{wo}} \item
‘village’ \textbf{\textit{wa}} \item
‘vine’ \textbf{\textit{nïpïl}} \item
‘vine species’ \textbf{\textit{angïn}} \item
‘vital spot’ \textbf{\textit{unda}} \item
‘vomit’ \textit{v.} \textbf{\textit{nongan u-}} \item
‘vomitus’ \textbf{\textit{nongan}} \item
‘vulva’ \textbf{\textit{inmbï}}, \textbf{\textit{iwïl}}\\ \item

\noindent \textbf{W – w}\\ \item

‘waist’ \textbf{\textit{inapaw}} \item
‘waist skirt’ \textbf{\textit{wopana}} \item
‘waiting for’ \textbf{\textit{andïla}}, \textbf{\textit{angla}} \item
‘walk’ \textit{v.} \textbf{\textit{inda-}} \item
‘wall’ \textbf{\textit{apa nambï}} \item
‘wallaby’ \textbf{\textit{wakan}} \item
‘warm’ \textbf{\textit{wananum}} \item
‘wart’ \textbf{\textit{lemum}} \item
‘was’ \textbf{\textit{wap}} \item
‘wash’ \textit{v.} \textbf{\textit{lopo-}} \item
‘wasp’ \textbf{\textit{numbunum}} \item
‘watch’ \textit{v.} \textbf{\textit{lïmndï ala-}}, \textbf{\textit{lïmndï lï-}} \item
‘water’ \textbf{\textit{inim}} \item
‘water spirit’ \textbf{\textit{inimnji}} \item
‘way’ \textbf{\textit{i}} \item
‘we’ \textbf{\textit{an}}, \textbf{\textit{ngan}}, \textbf{\textit{nguna}}, \textbf{\textit{ngunan}}, \textbf{\textit{una}}, \textbf{\textit{unan}} \item
‘we ourselves’ \textbf{\textit{anawa}}, \textbf{\textit{anwe}}, \textbf{\textit{nganawa}}, \textbf{\textit{nganwe}}, \textbf{\textit{ngunanawa}}, \textbf{\textit{ngunanwe}}, \linebreak \textbf{\textit{unanawa}}, \textbf{\textit{unanwe}} \item
‘we two’ \textbf{\textit{ngan}}, \textbf{\textit{nguna}}, \textbf{\textit{ngunan}} \item
‘weave’ \textit{v.} \textbf{\textit{ula-}} \item
‘web’ \textbf{\textit{lingïnane}} \item
‘weep’ \textit{v.} \textbf{\textit{sa-}} \item
‘\textit{wel daka}’ \textbf{\textit{nakam wanmbi}} \item
‘well’ \textbf{\textit{anma}} \item
‘wet’ \textbf{\textit{mïnwata}} \item
‘wet season’ \textbf{\textit{inim}} \item
‘what?’ \textbf{\textit{angos}} \item
‘what’s the matter?’ \textbf{\textit{anjikaka}} \item
‘whatchamacallit’ \textbf{\textit{mïngamata}} \item
‘whatever’ \textbf{\textit{angos}}, \textbf{\textit{angos nji}} \item
‘whatsoever’ \textbf{\textit{angos}} \item
‘when’ \textbf{\textit{tem}} \item
‘when?’ \textbf{\textit{ango tem}} \item
‘whenever’ \textbf{\textit{tem}} \item
‘where?’ \textbf{\textit{ango}}, \textbf{\textit{ango luwa}} \item
‘which?’ \textbf{\textit{ango}} \item
‘whirlwind’ \textbf{\textit{ngungun}} \item
‘white’ \textbf{\textit{waembïl}} \item
‘white ant’ \textbf{\textit{kïka}} \item
‘white hair’ \textbf{\textit{monkin}} \item
‘white person’ \textbf{\textit{waembïl ankam}} \item
‘white spot’ \textbf{\textit{nataw}} \item
‘who?’ \textbf{\textit{kuma}}, \textbf{\textit{kwa}} \item
‘whole’ \textbf{\textit{wopa}} \item
‘whose?’ \textbf{\textit{kumanji}}, \textbf{\textit{kwanji}} \item
‘why?’ \textbf{\textit{angwena}} \item
‘wide’ \textbf{\textit{palmana}} \item
‘wife’ \textbf{\textit{yana}}, \textbf{\textit{yananu}}, \textbf{\textit{yena}}, \textbf{\textit{yenanu}} \item
‘wild betel pepper’ \textbf{\textit{nakam wanmbi}} \item
‘wild taro’ \textbf{\textit{ulumbi}} \item
‘wildfowl species’ \textbf{\textit{kuman}}, \textbf{\textit{yokomakan}} \item
‘wildfowl egg’ \textbf{\textit{yokomtïn}} \item
‘will’ \textbf{\textit{mbay}} \item
‘wind’ \textbf{\textit{nonal}} \item
‘window’ \textbf{\textit{mwa}} \item
‘windpipe’ \textbf{\textit{aninokam}} \item
‘wing’ \textbf{\textit{wapa}} \item
‘wipe’ \textit{v.} \textbf{\textit{uta-}} \item
‘wisdom’ \textbf{\textit{kalam}} \item
‘wise’ \textbf{\textit{kalam}} \item
‘with’ \textbf{\textit{lu}}, \textbf{\textit{ul}} \item
‘within’ \textbf{\textit{in}} \item
‘without care’ \textbf{\textit{ko}}, \textbf{\textit{kop}}, \textbf{\textit{kwa}}, \textbf{\textit{wa}} \item
‘without reason’ \textbf{\textit{ko}}, \textbf{\textit{kop}}, \textbf{\textit{kwa}}, \textbf{\textit{wa}} \item
‘woman’ \textbf{\textit{inom}}, \textbf{\textit{yana}}, \textbf{\textit{yananu}}, \textbf{\textit{yena}}, \textbf{\textit{yenanu}} \item
‘wood’ \textbf{\textit{impul}} \item
‘woods’ \textbf{\textit{wandam}} \item
‘work’ \textbf{\textit{wombïn}} \item
‘work’ \textit{v.} \textbf{\textit{wombïn ni-}} \item
‘worm’ \textbf{\textit{utal}} \item
‘wound’ \textbf{\textit{tawa}} \item
‘wring’ \textit{v.} \textbf{\textit{mïmïl u-}} \item
‘wrist’ \textbf{\textit{yanaw}}\\ \item

\noindent \textbf{Y – y}\\ \item

‘yam’ \textbf{\textit{utam}} \item
‘yam species’ \textbf{\textit{anem}}, \textbf{\textit{awïl}}, \textbf{\textit{kunya}}, \textbf{\textit{nambana}}, \textbf{\textit{ngum}}, \textbf{\textit{pawla}}, \textbf{\textit{samnang}}, \linebreak \textbf{\textit{tambontam}}, \textbf{\textit{yambisa}}, \textbf{\textit{yena utam}}, \textbf{\textit{yeta utam}} \item
‘yam thorn’ \textbf{\textit{nap}} \item
‘yawn’ \textbf{\textit{mamal}} \item
‘yay’ \textbf{\textit{i}} \item
‘yeah’ \textbf{\textit{iya}} \item
‘year’ \textbf{\textit{inim}} \item
‘yellow’ \textbf{\textit{andwana}}, \textbf{\textit{ane}}, \textbf{\textit{mïndit}} \item
‘yes’ \textbf{\textit{iyo}} \item
‘yesterday’ \textbf{\textit{awal}} \item
‘yet’ \textbf{\textit{amun}} \item
‘yolk’ \textbf{\textit{kalum}} \item
‘you’ \textbf{\textit{ngun}}, \textbf{\textit{ngun=}}, \textbf{\textit{un}}, \textbf{\textit{un=}}, \textbf{\textit{u}}, \textbf{\textit{u=}} \item
‘you poor thing’ \textbf{\textit{ungusuwa}}, \textbf{\textit{ungusuwata}} \item
‘you poor things’ \textbf{\textit{ngungusuwa}}, \textbf{\textit{ngungusuwata}}, \textbf{\textit{ungusuwa}}, \textbf{\textit{ungusuwata}} \item
‘you two’ \textbf{\textit{ngun}}, \textbf{\textit{ngun=}} \item
‘you yourself’ \textbf{\textit{uwe}}, \textbf{\textit{wawa}} \item
‘you yourselves’ \textbf{\textit{ngunawa}}, \textbf{\textit{ngunwe}}, \textbf{\textit{unawa}}, \textbf{\textit{unwe}} \item
‘you’re welcome’ \textbf{\textit{u anma}} \item
‘young’ \textbf{\textit{akïnaka}}, \textbf{\textit{amunji}} \item
‘young coconut’ \textbf{\textit{andïmoni}} \item
‘young person’ \textbf{\textit{amunji}}, \textbf{\textit{nungol}}, \textbf{\textit{nungolke}} \item
‘younger’ \textbf{\textit{wot}} \item
‘younger brother’ \textbf{\textit{wot yata}}, \textbf{\textit{wot yeta}} \item
‘younger sister’ \textbf{\textit{wot yana}}, \textbf{\textit{wot yena}} \item
‘your’ \textbf{\textit{ngunji}}, \textbf{\textit{unji}} \item
‘your own’ \textbf{\textit{ambinji}}, \textbf{\textit{ambïnji}}, \textbf{\textit{amblanji}} \item
‘yours’ \textbf{\textit{ngunji}}, \textbf{\textit{unji}} \item
‘yourself’ \textbf{\textit{ambawa=}}, \textbf{\textit{ambï=}}, \textbf{\textit{ambuwe}} \item
‘yourselves’ \textbf{\textit{ambin=}}, \textbf{\textit{ambinawa}}, \textbf{\textit{ambinwe}}, \textbf{\textit{ambla=}}, \textbf{\textit{amblawa}}, \textbf{\textit{amblawe}}
\end{enumerate}


\section{List of bound morphemes}\label{sec:17.3}

\is{bound morpheme|(}

\tabref{tab:17.1} provides a list of the bound morphemes found in this grammar. Phonologically \is{phonology} conditioned \isi{allomorph}s are listed in the column labeled “allomorphs”. Lexically \is{lexicon} conditioned allomorphs, however, such as irregular \isi{TAM} suffixes, are given their own entries in the table. The “gloss” column shows how these bound morphemes are glossed in example sentences. More information on both the form and function of these bound morphemes can be found by following the cross-references listed in the “sections” column.

Most of these morphemes are \isi{suffix}es. Suffixes may be identified as those forms that are followed by a hyphen (-). Prefixes, on the other hand, may be identified as those forms that are preceded by a hyphen (-). It should be noted, however, that the only proper \isi{prefix} in Ulwa is \textit{na-} ‘\textsc{detr}’. The form \textit{la-} {\textasciitilde} \textit{lo-} ‘\textsc{irr}’ only occurs as part of an irregular \isi{circumfix}-like marker for two verb forms; it is perhaps a \is{fossilization} fossilized form of the sole \isi{prefix} \textit{na-} ‘\textsc{detr}’ < *la-.

\tabref{tab:17.1} also contains \isi{clitic}s: four \isi{proclitic}s and two \isi{enclitic}s. The proclitics consist of three \isi{object marker}s/\isi{non-subject marker}s and one \isi{indefinite} marker. The enclitics consist of one \isi{oblique marker} and one \isi{copula}. It should be noted that \isi{non-subject} pronominal \is{pronoun} forms (\chapref{sec:6}) and non-subject \isi{demonstrative} forms (\chapref{sec:7}) are not included in this table.

\begin{table}
\small
\caption{Bound morphemes}
\is{allomorph}
\is{perfective}
\is{topic marker}
\is{intensifier}
\is{imperfective}
\is{dependent}
\is{nominalizer}
\is{indefinite}
\is{irrealis}
\is{non-subject}
\is{imperative}
\is{oblique}
\is{detransitivizer}
\is{emphatic}
\is{possession}
\is{vocative}
\is{copula}
\is{speculative}
\is{conditional}
\is{partitive-intensive}
\is{bound morpheme|)}
\label{tab:17.1}
\begin{tabularx}{\textwidth}{lllQl}
\lsptoprule
form & allomorphs & gloss & description & sections\\
\midrule
{\itshape {}-al} &  & {\scshape pfv} & perfective (irregular) & \sectref{sec:4.3}\\
{\itshape {}-ambi} &  & {\scshape top} & topic marker & \sectref{sec:6.8}\\
{\itshape {}-awa} &  & {\scshape int} & intensifier & \sectref{sec:6.6}\\
{\itshape {}-e} & {\itshape {}-Ø, {}-ye} & {\scshape ipfv} & imperfective & \sectref{sec:4.4}\\
{\itshape {}-e} & {\itshape {}-ye} & {\scshape dep} & dependent & \sectref{sec:12.2.1}\\
{\itshape {}-en} & {\itshape {}-n, {}-wen, \nobreakdash-yen} & {\scshape nmlz} & nominalizer & \sectref{sec:3.2}, \sectref{sec:12.3.1}\\
{\itshape ko=} &  & {\scshape indf} & indefinite & \sectref{sec:7.2}\\
{\itshape la-} & {\itshape l-, lo-} & {\scshape irr} & irrealis (irregular) & \sectref{sec:4.3}\\
{\itshape {}-m} &  & {\scshape irr} & irrealis (irregular) & \sectref{sec:4.3}\\
{\itshape {}-m} &  & {\scshape pfv} & perfective (irregular) & \sectref{sec:4.3}\\
{\itshape ma=} & {\itshape m=, mo=} & {\scshape 3sg.obj} & 3\textsc{sg} non-subject & \sectref{sec:7.2}\\
{\itshape mï=} & {} & {\scshape 3sg.subj} & 3\textsc{sg} subject (only before {\itshape na-}) & \sectref{sec:13.8.7}\\
{\itshape min=} & {\itshape mini=} & {\scshape 3du} & 3\textsc{du} non-subject & \sectref{sec:7.2}\\
{\itshape {}-n} & {\itshape {}-na} & {\scshape pfv} & perfective (irregular) & \sectref{sec:4.3}\\
{\itshape {}-n} &  & {\scshape imp} & imperative & \sectref{sec:4.7}, \sectref{sec:13.2}\\
{\itshape {}-n} &  & {\scshape ipfv} & imperfective (irregular) & \sectref{sec:4.3}\\
{\itshape =n} & {\itshape =ïn, =nï} & {\scshape obl} & oblique & \sectref{sec:11.4.1}\\
{\itshape {}-na} & {\itshape {}-nda} & {\scshape irr} & irrealis & \sectref{sec:4.2}, \sectref{sec:4.6}\\
{\itshape na-} & {\itshape n-} & {\scshape detr} & detransitivizer & \sectref{sec:13.8.2}\\
{\itshape {}-nam} &  & {\scshape emph} & emphatic & \sectref{sec:6.7}\\
{\itshape ndï=} & {\itshape nd=} & {\scshape 3pl} & 3\textsc{pl} non-subject & \sectref{sec:7.2}\\
{\itshape {}-nji} &  & {\scshape poss} & possessive & \sectref{sec:6.2}, \sectref{sec:9.1.5}\\
{\itshape {}=o} & {\itshape {}=wo} & {\scshape voc} & vocative & \sectref{sec:8.3.3}\\
{\itshape {}-p} & {\itshape {}-ap, {}-ïp, {}-op} & {\scshape pfv} & perfective & \sectref{sec:4.5}, \sectref{sec:4.8}\\
{\itshape =p} &  & {\scshape cop} & copula & \sectref{sec:10.3}\\
{\itshape {}-t} &  & {\scshape spec} & speculative & \sectref{sec:4.11}, \sectref{sec:13.2.4}\\
{\itshape {}-ta} &  & {\scshape cond} & conditional & \sectref{sec:4.11}, \sectref{sec:13.5}\\
{\itshape {}-we} &  & {\scshape part.int} & partitive intensifier & \sectref{sec:6.6}\\
\lspbottomrule
\end{tabularx}
\end{table}