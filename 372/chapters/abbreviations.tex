\addchap{\lsAbbreviationsTitle}

\section*{Interlinear glossing}

The conventions of the Leipzig Glossing Rules \citep{ComrieEtAl2008} are followed, with additions made where needed. The following is a complete list of the abbreviations used in the glosses of Ulwa morphemes.
\begin{tabbing}
{(\textsc{interj})} \=   {(\isi{interjection})}\kill
1   \>   1st \isi{person}\\
2   \>   2nd \isi{person}\\
3  \>    3rd \isi{person}\\
\textsc{cond} \>   \isi{conditional}\\
\textsc{cop} \>   \isi{copula}\\
\textsc{dep}  \>  \isi{dependent marker}\\
\textsc{detr} \>   \isi{detransitivizer}\\
\textsc{dist} \>   \isi{distal}\\
\textsc{du}   \>   \isi{dual}\\
\textsc{emph} \>   \isi{emphatic}\\
\textsc{excl} \>   \isi{exclusive}\\
\textsc{hab} \>    \isi{habitual} (in \isi{loan}s from \ili{Tok Pisin})\\
\textsc{imp}   \>   \isi{imperative}\\
\textsc{incl}  \>  \isi{inclusive}\\
\textsc{indf} \>   \isi{indefinite}\\
\textsc{interj} \>   \isi{interjection}\\
\textsc{int}  \>    \isi{intensive}\\
\textsc{ipfv} \>   \isi{imperfective}\\
\textsc{irr}   \>   \isi{irrealis}\\
\textsc{neg} \>   \isi{negative}/\isi{negator}\\
\textsc{nmlz} \>   \isi{nominalizer}\\
\textsc{nsg}  \>  \isi{non-singular}\\
\textsc{obj}   \>   object (or \isi{non-subject})\\
\textsc{obl}  \>  \isi{oblique}\\
\textsc{part}  \>  \isi{partitive}\\
\textsc{pfv}  \>    \isi{perfective}\\
\textsc{pl}  \>    \isi{plural}\\
\textsc{poss} \>   possessive/\isi{possessor}\\
\textsc{pred} \>   \isi{predicate marker} (in \isi{loan}s from \ili{Tok Pisin})\\
\textsc{proh} \>   \isi{prohibitive}\\
\textsc{prox} \>   \isi{proximal}\\
\textsc{pst}  \>    \isi{past}\\
\textsc{refl}  \>  \isi{reflexive}\\
\textsc{sg}   \>   \isi{singular}\\
\textsc{spec}  \>  \isi{speculative}\\
\textsc{subj} \>   subject\\
\textsc{top}  \>  \isi{topic}\\
\textsc{voc} \>   \isi{vocative}
\end{tabbing}

\section*{Other abbreviations}
\begin{tabbing}
{(PARADISEC)}  \= {(Pacific and Regional Archive for Digital Sources in Endangered Cultures)}\kill
A   \>   the more \isi{agent}-like argument of a \isi{transitive} clause\\
ADJ  \>  \isi{adjective}\\
ADV  \>  \isi{adverb}\\
CONJ  \>  \isi{conjunction}\\
DEM  \>  \isi{demonstrative}\\
LEI  \>  Language Endangerment Index\\
EGIDS  \>  Expanded Graded Intergenerational Disruption Scale\\
ELAR  \>  Endangered Languages Archive\\
INTERJ \> \isi{interjection}\\
IPA  \>  International Phonetic Alphabet\\
ISO  \>  International Organization for Standardization\\
LLG  \>  Local-Level Government area\\
N  \>    noun\\
NP  \>    \isi{noun phrase}\\
NUM \>   \isi{numeral}\\
O   \>   object\\
P  \>    the more \isi{patient}-like argument of a \isi{transitive} clause\\
P   \>   \isi{postposition}\\
PARADISEC  \> Pacific and Regional Archive for Digital Sources in Endangered\\
 \> Cultures\\
PNG  \>  Papua New Guinea\\
PP   \>   \isi{postpositional phrase}\\
PRO  \>  \isi{pronoun}\\
Q   \>   \isi{question word} / \isi{interrogative word}\\
QUANT \> \isi{quantifier}\\
R  \>    \isi{recipient}\\
S   \>   subject (or, the single argument of an \isi{intransitive} clause)\\
SOAS \>   School of Oriental and African Studies\\
T   \>   \isi{theme}\\
\isi{TAM}  \>  \isi{tense}-\isi{aspect}-\isi{mood}\\
TP   \>   \ili{Tok Pisin}\\
UNESCO \> United Nations Educational, Scientific and Cultural Organization\\
V   \>   verb\\
VP  \>    \isi{verb phrase}\\
X  \>    \isi{oblique}
\end{tabbing}
