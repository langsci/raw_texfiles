\addchap{\lsPrefaceTitle}
This book is a revision of my PhD dissertation, \textit{A grammar of Ulwa}, which I successfully submitted in May 2018 at the University of Hawaiʻi at Mānoa. Since then, I have returned again to Papua New Guinea, where I learned more about the Ulwa language. In particular, I met with more speakers of the \ili{Maruat-Dimiri-Yaul} \isi{dialect}. I also conducted research with Ulwa’s four sister languages in the \ili{Keram} family: \ili{Mwakai}, \ili{Pondi}, \ili{Ambakich}, and \ili{Ap Ma}. Although the general shape of this present grammatical description is the same as that of the earlier dissertation, I have made a number of revisions throughout, thanks in part to an improved understanding of the diachronic grammar of the \ili{Keram} family. The introductory chapter has been divided into two chapters: the first discusses the sociolinguistics and historical linguistics of the language, and the second explains the data sources and conventions used throughout the book. I have also added two new chapters: one is an overview of Ulwa grammar, and the other is a description of the \ili{Maruat-Dimiri-Yaul} \isi{dialect}. Finally, I have added time codes to all examples taken from archived texts, so as to improve accessibility.
