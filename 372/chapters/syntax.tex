\chapter{Additional topics in syntax}\label{sec:13}

\is{syntax|(}

This chapter covers an assortment of syntactic constructions, organized rather more by functional concerns than by syntactically motivated principles. Thus this chapter explains how a speaker of Ulwa may accomplish certain communicative goals, such as formulating \isi{question}s, issuing \isi{command}s, \is{negation} negating propositions, and reporting on the \isi{speech} of others.

\is{syntax|)}

\section{Questions}\label{sec:13.1}

\is{syntax|(}
\is{question|(}

\is{yes/no question}
\is{wh- question}

There are, as in most languages, two basic types of \isi{question}s in Ulwa: \isi{polar question}s (‘yes/no’ questions) (\sectref{sec:13.1.1}) and \isi{content question}s (\textit{wh-} questions) (\sectref{sec:13.1.2}).

\is{question|)}
\is{syntax|)}

\subsection{Polar questions (‘yes/no’ questions)}\label{sec:13.1.1}

\is{syntax|(}
\is{question|(}
\is{polar question|(}
\is{yes/no question|(}

Polar questions in Ulwa are identical in form to their \isi{declarative} counterparts. They are recognizable, however, through \isi{pragmatic} factors and through \isi{phonology} (\isi{intonation}). First, context often makes it apparent that a \isi{question}, rather than a statement, is being formed by the speaker. Second, polar questions are identifiable through a rising \isi{intonation}. The questions presented in \REF{ex:syntax:1} through \REF{ex:syntax:4}, if given the right context and said without a rising \isi{intonation}, could all also function as statements.

\ea%1
    \label{ex:syntax:1}
            \textit{Itom mï awal tembi wap.}\\
\gll    itom  mï      awal    tembi  wap\\
    father  \textsc{3sg.subj}  yesterday  bad    be.\textsc{pst}\\
\glt    (a) ‘Was father sick yesterday?’

    (b) ‘Father was sick yesterday.’ [elicited]
\z

\ea%2
    \label{ex:syntax:2}
           \textit{U namap.}\\
\gll    u    na-ama-p\\
    \textsc{2sg}  \textsc{detr}{}-eat-\textsc{pfv}\\
\glt    (a) ‘Have you eaten?’

    (b) ‘You’ve already eaten.’ [elicited]
\z

\ea%3
    \label{ex:syntax:3}
            \textit{Inom mï amun ya ute.}\\
\gll    inom  mï      amun  ya      uta-e\\
    mother  3\textsc{sg.subj}  now  coconut  grind-\textsc{ipfv}\\
\glt    (a) ‘Is mother grinding coconut now?’

    (b) ‘Mother is grinding coconut now.’ [elicited]
\z

\ea%4
    \label{ex:syntax:4}
            \textit{Alum mï ikali ya ndïsina.}\\
\gll    alum  mï      i-kali    ya      ndï=si-na\\
    child  \textsc{3sg.subj}  hand-send  coconut  \textsc{3pl=}push-\textsc{irr}\\
\glt    (a) ‘Can the child catch the coconuts?’

    (b) ‘The child can catch the coconuts.’ [elicited]
\z

Perhaps especially in \isi{leading question}s, Ulwa can employ the \isi{interjection}s \textit{a} ‘eh?’ or \textit{e} ‘eh?’ as a \isi{tag word} at the end of the \isi{interrogative} sentence. These forms are common also in \ili{Tok Pisin}, which possibly played a role in their use in Ulwa. These \isi{tag word}s serve as an additional means of indicating that a sentence is a \isi{question}, as seen in \REF{ex:syntax:5}, \REF{ex:syntax:6}, and \REF{ex:syntax:7}.

\ea%5
    \label{ex:syntax:5}
            \textit{Alo un apa map \textbf{a}?}\\
\gll ala{}=o      un  apa    ma=p      \textbf{a}\\
    \textsc{pl.dist{}=voc} 2\textsc{pl}  house  3\textsc{sg.obj}=be  \textsc{interj}\\
\glt    ‘Hey, you all -- are you home?’ [ulwa018\_01:37]
\z

\ea%6
    \label{ex:syntax:6}
            \textit{Ngun andin ngun mundu ngunas \textbf{a}?}\\
\gll ngun  andin    ngun  mundu  ngun=asa  \textbf{a}\\
    2\textsc{du}  \textsc{du.dist}  2\textsc{du}  hunger  2\textsc{du}=hit  \textsc{interj}\\
\glt `You two, you two over there -- you’re hungry, yeah?’ [ulwa041\_01:09]
\z

\ea%7
    \label{ex:syntax:7}
            \textit{U ango anmbï mbi \textbf{e}?}\\
\gll u    ango  an-mbï    mbï-i    \textbf{e}\\
    2\textsc{sg}  \textsc{neg}  out-here  out-go.\textsc{pfv}  \textsc{interj}\\
\glt `You didn’t come out, eh?’ [ulwa040\_00:06]
\z

There are not many examples of these tag \isi{interjection}s in my corpus. Although examples \REF{ex:syntax:5} through \REF{ex:syntax:7} might suggest a contrast in \isi{polarity} between the \isi{interjection}s \textit{a} ‘eh?’ and \textit{e} ‘eh?’, I do not suspect that such a contrast exists.

  Polar questions may be answered with full sentences, with \isi{paralinguistic gesture}s, with general-purpose \is{exclamation} exclamatory \isi{interjection}s (such as \textit{m} ‘mhm’), or with the designated \isi{response} \isi{interjection}s ‘yes’ or ‘no’. The word \textit{iyo} ‘yes’ (with the alternate form \textit{iya} ‘yeah’) is used for the \isi{affirmative}, and the word \textit{ase} ‘no’ (with the alternate form \textit{asa} ‘nah’) is used for the \isi{negative}. To disagree with a \isi{negative} proposition in a \isi{question}, a speaker may answer ‘yes’. Thus, for example, the answer to \REF{ex:syntax:7} is provided in \REF{ex:syntax:8}.

\ea%8
    \label{ex:syntax:8}
            \textbf{\textit{Iya}} \textit{nï awal mbi lïmndï tawatïp ndale.}\\
\gll    \textbf{iya}  nï    awal    mbï-i      lïmndï  tawatïp    ndï=ala-e\\
    yes  1\textsc{sg}  yesterday  here-go.\textsc{pfv}  eye    child    \textsc{3pl}=see-\textsc{dep}\\
\glt `Yes, I came out yesterday and watched the children.’ [ulwa040\_00:08]
\z

Here the responder answers ‘yes’ to mean: ‘No, I did come out.’

\is{yes/no question|)}
\is{polar question|)}
\is{question|)}
\is{syntax|)}

\subsection{{Content} {questions} {(\textit{wh-}} {questions)}}\label{sec:13.1.2}

\is{wh- question|(}
\is{content question|(}
\is{question|(}
\is{syntax|(}

Content questions in Ulwa rely on several different \textit{wh-} words, which are presented in \REF{ex:syntax:9}. For more on \isi{interrogative pronoun}s, see \sectref{sec:6.5}. There are no \isi{interrogative verb}s in Ulwa.

\ea%9
    \label{ex:syntax:9}
            \isi{Interrogative word}s
\begin{tabbing}
{(\textit{ango (luwa)})} \= {(how many?)}\kill
    \textit{kwa}  \>    ‘who? [\textsc{sg]’}\\
    \textit{kuma} \>     ‘who? [\textsc{nsg]’}\\
    \textit{kwanji} \>     ‘whose? [\textsc{sg]’}\\
    \textit{kumanji} \>   ‘whose? \textsc{[nsg]’}\\
    \textit{angos}  \>    ‘what?’\\
\textit{ango}    \>  ‘which?’\\
\textit{ango (luwa)} \> ‘where?’\\
\textit{ango tem}   \> ‘when?’\\
\textit{angwena}  \>  ‘why?’\\
\textit{anjika}   \>   ‘how many?’\\
\textit{anjikaka}  \>  ‘how?’
\end{tabbing}
\z

  The \isi{interrogative pronoun} \textit{angos} ‘what?’ is discussed in \sectref{sec:6.5}. Examples \REF{ex:syntax:10} and \REF{ex:syntax:11} demonstrate its use in texts.

\ea%10
    \label{ex:syntax:10}
          \textit{A nïnji nungol ala} \textbf{\textit{angos}} \textit{landa?}\\
\gll    a  nï-nji  nungol  ala      \textbf{angos}  la-nda\\
    ah  1\textsc{sg-poss}  child  \textsc{pl.dist}  what  eat-\textsc{irr}\\
\glt `Ah, what will my children eat?’ [ulwa014\_64:56]
\z

\ea%11
    \label{ex:syntax:11}
          \textit{U} \textbf{\textit{angos}} \textit{natan?}\\
\gll    u    \textbf{angos}  na-ta-n\\
    2\textsc{sg}  what  \textsc{detr}{}-say-\textsc{ipfv}\\
\glt `What are you saying?’ [ulwa014†]
\z

As mentioned in \sectref{sec:8.3.2}, the \isi{question word}s \textit{angos} ‘what?’ and \textit{ango} ‘which?’ likely derive from the \isi{negation} marker \textit{ango} ‘\textsc{neg}’. This may suggest that content questions in general may derive from \isi{polar question}s (e.g., a \isi{question} like ‘what will my children eat?’ in \REF{ex:syntax:10} would have its origin in something like ‘will my children not eat?’).\footnote{Similarly, \isi{interrogative pronoun}s like \textit{kwa} ‘who? [\textsc{sg]}’ and \textit{kuma}  ‘who? [\textsc{nsg]}’ are \isi{colexified} with \isi{indefinite pronoun}s. Thus, content questions such as the one in \REF{ex:syntax:13} may thus derive from \isi{polar question}s (i.e., ‘who is carrying it?’ < ‘is someone carrying it?’).}

Sentences \REF{ex:syntax:12} and \REF{ex:syntax:13} provide examples of \textit{kwa} ‘who? [\textsc{sg}]’ (further discussed in \sectref{sec:6.5}). This form is often shortened to [ko]. Whereas the form \textit{kwa} ‘who? [\textsc{sg}]’ refers to exactly one referent, the form \textit{kuma} ‘who? [\textsc{nsg]}’ refers to two or more referents.\footnote{Thus the number distinction made in the words meaning ‘who?’ (or ‘whose?’) is a binary distinction of \isi{singular} versus \isi{non-singular}, as opposed to the three-way contrast of \isi{singular}, \isi{dual}, and \isi{plural} that runs throughout many other paradigms in the language.}

\ea%12
    \label{ex:syntax:12}
          \textbf{\textit{Kwa}} \textit{tïki man tïnangana?}\\
\gll \textbf{kwa}  tïki    ma=n      tïnanga-na\\
    one    again  3\textsc{sg.obj=obl}  arise-\textsc{irr}\\
\glt `Who will get it [= the school] up again?’ [ulwa014\_54:01]
\z

\ea%13
    \label{ex:syntax:13}
          \textbf{\textit{Ko}} \textit{mat inde?}\\
\gll    \textbf{ko}    ma=tï      inda-e\\
    one    3\textsc{sg.obj}=take  walk-\textsc{ipfv}\\
\glt `Who is carrying it around?’ [ulwa037\_17:22]
\z

\is{wh-movement}

As mentioned in \sectref{sec:11.1}, there is no so-called \textit{wh}-movement in Ulwa; all content questions are asked \isi{in-situ} -- that is, with the questioned element occurring in the same place where it would occur in an equivalent \isi{declarative} sentence. Thus, \textit{kwa} {\textasciitilde} \textit{kuma} ‘who?’ or \textit{angos} ‘what?’ occur in the subject position when the questioned element is the subject of a clause, and they occur in the object position when the questioned element is an object. Likewise, \textit{kwanji} {\textasciitilde} \textit{kumanji} ‘whose?’ occurs immediately before the \isi{possessed} NP, just as would any \is{possessive pronoun} possessive pronominal marker. Thus, for example, in \REF{ex:syntax:14}, the ‘who(m)?’ element occurs in the position typically held by objects.

\ea%14
    \label{ex:syntax:14}
          \textit{U man} \textbf{\textit{ko}} \textit{lïp sina?}\\
\gll    u    ma=n      \textbf{ko}  lï-p      si-na\\
    2\textsc{sg}  3\textsc{sg.obj=obl}  one  put-\textsc{pfv}  push-\textsc{irr}\\
\glt `Whom will you blame?’ (Literally ‘Onto whom will you push with it?’) [ulwa014†]
\z

  In questions of countable quantity, the \isi{question word} \textit{anjika} ‘how many?’ appears after the \isi{noun phrase} whose quantity is the topic of questioning. This could be either a subject or an object (or even an \isi{oblique} \isi{phrase}). The word \textit{anjika} ‘how many?’ is thus syntactically identical to any modifying \isi{adjective} and -- in particular -- to \isi{numeral}s, which immediately follow the enumerated NP. Its use is exemplified in \REF{ex:syntax:15} and \REF{ex:syntax:16}.

\ea%15
    \label{ex:syntax:15}
          \textit{Wambana} \textbf{\textit{anjika}} \textit{inim mo man?}\\
\gll    wambana  \textbf{anjika}    inim  ma=u      ma-n\\
    fish    how.many  water  \textsc{3sg.obj}=from  go-\textsc{ipfv}\\
\glt `How many fish are swimming?’ [elicited]
\z

\ea%16
    \label{ex:syntax:16}
          \textit{U wambana} \textbf{\textit{anjika}} \textit{tïn?}\\
\gll    U    wambana  \textbf{anjika}    tï-n\\
    \textsc{2sg}  fish    how.many  take-\textsc{pfv}\\
\glt `How many fish did you catch?’ [elicited]
\z

\is{yes/no question}

It should be noted that questions of non-countable quantity -- that is, questions about \isi{mass noun}s (i.e., ‘how much?’) -- are not asked with \textit{anjika} ‘how many?’. Rather, such \isi{interrogative}s can only be formed as ‘yes/no’ questions, in which an inquiry is made whether the amount in question is ‘big’ or ‘little’, as seen in examples \REF{ex:syntax:17}, \REF{ex:syntax:18}, and \REF{ex:syntax:19}.

\ea%17
    \label{ex:syntax:17}
          \textit{U inim} \textbf{\textit{ambi}} \textit{amap?}\\
\gll    u    inim  \textbf{ambi}  ama-p\\
    2\textsc{sg}  water  big    eat-\textsc{pfv}\\
\glt `How much water did you drink?’ (Literally ‘Did you drink big [i.e., much] water?’) [elicited]
\z

\ea%18
    \label{ex:syntax:18}
          \textit{U inim} \textbf{\textit{ilum}} \textit{amap?}\\
\gll    u    inim  \textbf{ilum}  ama-p\\
    2\textsc{sg}  water  little  eat-\textsc{pfv}\\
\glt `How much water did you drink?’ (Literally ‘Did you drink little water?’) [elicited]
\z

\ea%19
    \label{ex:syntax:19}
          \textit{Nungol mï inim} \textbf{\textit{ambi}} \textit{ame?}\\
\gll    nungol  mï      inim  \textbf{ambi}  ama-e\\
    child  \textsc{3sg.subj}  water  big    eat-\textsc{ipfv}\\
\glt `How much water does the child drink?’ (Literally ‘Does the child drink big [i.e., much] water?’) [elicited]
\z

Other \isi{question word}s, such as \textit{anjikaka} ‘how?’ and \textit{angwena} ‘why?’ cannot serve as either subject or object of a \isi{predicate}. Accordingly, they may be considered \isi{oblique}s. Their position in a clause is thus akin to the positioning of \isi{adverb}s -- that is, following the subject (when expressed) and preceding the entire verb \isi{phrase}, including the object of the verb if the verb is \isi{transitive}, as seen with the form \textit{anjikaka} ‘how?’ in \REF{ex:syntax:20} and \REF{ex:syntax:21}.

\ea%20
    \label{ex:syntax:20}
          \textit{U} \textbf{\textit{anjikaka}} \textit{apa maytap?}\\
\gll    u    \textbf{anjikaka}  apa    ma=ita-p\\
    \textsc{2sg}  how    house  \textsc{3sg.obj=}build-\textsc{pfv}\\
\glt `How did you build the house?’ [elicited]
\z

\ea%21
    \label{ex:syntax:21}
          \textit{Alimban mï} \textbf{\textit{anjikaka}} \textit{lamndu masap?}\\
\gll    Alimban  mï      \textbf{anjikaka}  lamndu  ma=asa-p\\
    [name]    \textsc{3sg.subj}  how    pig      \textsc{3sg.obj}=hit-\textsc{pfv}\\
\glt `How did Alimban kill the pig?’ [elicited]
\z

This is the same position as other \isi{oblique}s, such as \isi{postpositional phrase}s, as illustrated by \REF{ex:syntax:22}, or \isi{oblique}-marked NPs, as illustrated by \REF{ex:syntax:23} (\sectref{sec:11.4}).

\ea%22
    \label{ex:syntax:22}
          \textit{Alimban mï \textbf{tïn mol} lamndu masap.}\\
\gll    Alimban  mï      \textbf{tïn}    \textbf{ma=ul}    lamndu  ma=asa-p\\
    [name]    \textsc{3sg.subj}  dog  \textsc{3sg=}with  pig      \textsc{3sg.obj}=hit-\textsc{pfv}\\
\glt `Alimban killed the pig with the dog.’ [elicited]
\z

\ea%23
    \label{ex:syntax:23}
          \textit{Alimban mï} \textbf{\textit{mananï}} \textit{lamndu masap.}\\
\gll    Alimban  mï      \textbf{mana=nï}  lamndu  ma=asa-p\\
    [name]    \textsc{3sg.subj}  spear=\textsc{obl}  pig      \textsc{3sg.obj}=hit-\textsc{pfv}\\
\glt `Alimban killed the pig with the spear.’ [elicited]
\z


\is{syntax|)}
\is{question|)}
\is{content question|)}
\is{wh- question|)}

\is{wh- question|(}
\is{content question|(}
\is{question|(}
\is{syntax|(}

Although glossed in \REF{ex:syntax:20} and \REF{ex:syntax:21} as a monomorphemic word, \textit{anjikaka} ‘how?’ is actually analyzable as \textit{anjika-ka} ‘how.many-let’.\footnote{It may even be analyzable further (cf. the possible etymology of \textit{anjika} ‘how many?’ presented in \REF{ex:otherwc:148} in
\sectref{sec:8.3.2}).} In other words, the final element is taken to be the \isi{perfective}/\isi{imperfective} form of the \isi{irregular verb} \textit{ka-} ‘let, leave, allow’ (\sectref{sec:9.2.3}). Though perhaps having undergone a process of \isi{grammaticalization} and now often analyzed simply as ‘how?’, the word’s \isi{verbal morphology} is apparent in sentences such as \REF{ex:syntax:24} through \REF{ex:syntax:29}, which reflect the \isi{irrealis} form \textit{laka(na)} ‘let [\textsc{irr}]’ of the verb.

\ea%24
    \label{ex:syntax:24}
          \textit{Itom mï} \textbf{\textit{anjikalaka}} \textit{apa maytana?}\\
\gll    itom  mï      anjika-\textbf{la-ka}    apa    ma=ita-na\\
    father  3\textsc{sg.subj}  how.many-\textsc{irr}{}-let  house  \textsc{3sg.obj}=build-\textsc{irr}\\
\glt `How will father build the house?’ [elicited]
\z

\ea%25
    \label{ex:syntax:25}
          \textit{Nungol ndï} \textbf{\textit{anjikalaka}} \textit{wambana ndutana?}\\
\gll    nungol  ndï  anjika-\textbf{la-ka}    wambana  ndï=uta-na\\
    child  3\textsc{pl}  how.many-\textsc{irr}{}-let  fish    3\textsc{pl}=grind-\textsc{irr}\\
\glt `How will the boys catch the fish?’ [elicited]
\z

\ea%26
    \label{ex:syntax:26}
          \textit{Nga kwa} \textbf{\textit{anjikalakana}} \textit{mane?}\\
\gll    nga      kwa  anjika-\textbf{la-ka-na}      ma-n-e\\
    \textsc{sg.prox}  one    how.many-\textsc{irr}{}-let-\textsc{irr}    go-\textsc{ipfv-dep}\\
\glt `What is this one going to do?’ [ulwa001\_05:53]
\z

\ea%27
    \label{ex:syntax:27}
          \textit{U manï \textbf{anjikalakana}?}\\
\gll u    ma=nï      anjika-\textbf{la-ka-na}\\
    2\textsc{sg}  3\textsc{sg.obj=obl}  how.many-\textsc{irr-}let-\textsc{irr}\\
\glt `What will you do with it?’ [elicited]
\z

\ea%28
    \label{ex:syntax:28}
          \textit{Itom mï mana manï \textbf{anjikalakana}?}\\
\gll itom  mï      mana  ma=nï      anjika-\textbf{la-ka-na}\\
    father  3\textsc{sg.subj}  spear  3\textsc{sg.obj=obl}  how.many-\textsc{irr}{}-let-\textsc{irr}\\
\glt `What is father going to do with the spear?’ [elicited]
\z

\ea%29
    \label{ex:syntax:29}
          \textit{U ndït indata ndïn \textbf{anjikalakana}?}\\
\gll u    ndï=tï    inda-ta      ndï=n    anjika-\textbf{la-ka-na}\\
    2\textsc{sg}  3\textsc{pl}=take  walk-\textsc{cond}  3\textsc{pl=obl}  how.many-\textsc{irr}{}-let-\textsc{irr}\\
\glt `What will you do with them if you carry them around?’ [ulwa014\_14:06]
\z

Note that these \isi{irrealis}-marked forms of this \isi{question word} often convey a sense other than strictly ‘how?’, as examples \REF{ex:syntax:26} through \REF{ex:syntax:29}, which are translated as ‘what will [someone] do?’. The \isi{irrealis} examples notwithstanding, elsewhere throughout this grammar the form \textit{anjikaka} ‘how?’ is glossed simply as ‘how?’, without being analyzed as being polymorphemic.

\is{idiom}

  Idiomatically, \textit{anjika} ‘how many?’ can also be used to ask a \isi{question} somewhat akin to \ili{English} ‘what happened to [someone]?’, or ‘what’s up with [someone]?’, as seen in \REF{ex:syntax:30}.

\ea%30
    \label{ex:syntax:30}
          \textit{Mï nan mat a u \textbf{anjika}?}\\
\gll mï      na=n    ma=ta      a  u    \textbf{anjika}\\
    \textsc{3sg.subj}  talk=\textsc{obl}  3\textsc{sg.obj}=say  ay  2\textsc{sg}  how.many\\
\glt    ‘He said to her: “Ay, what happened to you?”’ [ulwa001\_15:48]
\z

As mentioned in the discussion of \isi{interrogative pronoun}s (\sectref{sec:6.5}), questions of ‘which?’ are formed with \textit{ango} ‘which?’, \isi{homophonous} with the \isi{negative} marker and likely derived from it. The two differ, however, in terms of syntactic position: whereas the \isi{negative} marker typically follows the grammatical subject, the \isi{question word} ‘which?’ precedes the NP it modifies (whether subject, object, or \isi{oblique}). Sentences \REF{ex:syntax:31}, \REF{ex:syntax:32}, and \REF{ex:syntax:33} provide additional examples of \textit{ango} ‘which?’ as it is used in questions.

\ea%31
    \label{ex:syntax:31}
          \textbf{\textit{Ango}} \textit{wa makape wombïn?}\\
\gll    \textbf{ango}  wa     maka=p-e    wombïn\\
    which  village  thus=\textsc{cop-dep}  work\\
\glt `Which village has work like this?’ [ulwa014\_61:04]
\z

\ea%32
    \label{ex:syntax:32}
          \textit{U} \textbf{\textit{ango}} \textit{tïlwa u mbi?}\\
\gll    u    \textbf{ango}  tïlwa  u    mbï-i\\
    2\textsc{sg}  which  road  from  here-go.\textsc{pfv}\\
\glt `Along which road have you come here?’ [ulwa037\_29:44]
\z

\ea%33
    \label{ex:syntax:33}
          \textit{Mbïpïta} \textbf{\textit{ango}} \textit{ini mawat pïta?}\\
\gll    mbï-p-ta    \textbf{ango}  ini    ma=wat    p-ta\\
    here-be\textsc{{}-cond} which  ground  3\textsc{sg.obj}=atop  be\textsc{{}-cond}\\
\glt `If [they] stay, which ground will they live on?’ [ulwa014\_21:01]
\z

The \isi{interrogative word} \textit{angos} ‘what?’ may be used in a similar fashion, modifying an NP (by preceding it) to ask ‘what kind of?’ or ‘what sort of?’, as in \REF{ex:syntax:34} and \REF{ex:syntax:35}.

\ea%34
    \label{ex:syntax:34}
          \textit{Ayndin nï} \textbf{\textit{angos}} \textit{na ukïna?}\\
\gll    Ayndin  nï    \textbf{angos}  na    u=kï-na\\
    [name]  1\textsc{sg}  what  talk  2\textsc{sg}=say-\textsc{irr}\\
\glt `Ayndin, what should I say to you?’ (Literally ‘what talk?’) [ulwa037\_00:01]
\z

\newpage

\ea%35
    \label{ex:syntax:35}
          \textit{Una wandam mawap} \textbf{\textit{angos}} \textit{wombïn ninda?}\\
\gll    unan    wandam  ma=wap      \textbf{angos}  wombïn=n  ni-nda\\
    1\textsc{pl.incl}  jungle    3\textsc{sg.obj}=be.\textsc{pst}  what  work\textsc{=obl}  act-\textsc{irr}\\
\glt `[When] we are in the jungle, what [sort of] work will [we] do?’ [ulwa030\_01:08]
\z

Questions of \isi{time} are asked by combining \textit{ango} ‘which?’ with \textit{tem} ‘time’, the latter word \isi{borrow}ed from \ili{Tok Pisin} \textit{taim} ‘time, when’. Thus, quite transparently, \isi{temporal} questions in Ulwa are rooted in a \isi{phrase} meaning ‘which time?’. This \isi{phrase} occurs in the canonical position for \isi{temporal adverb}s (e.g., \textit{umbe} ‘tomorrow’, \textit{amun} ‘now’, etc.). In other words, \textit{ango tem} ‘when?’ occurs immediately following the subject NP, as illustrated by \REF{ex:syntax:36}, whose answer is given in \REF{ex:syntax:37}.

\ea%36
    \label{ex:syntax:36}
          \textit{Kapos mï} \textbf{\textit{ango}} \textbf{\textit{tem}} \textit{lamndu masap?}\\
\gll Kapos  mï      \textbf{ango}  \textbf{tem}  lamndu  ma=asa-p\\
    [name]  \textsc{3sg.subj}  which  time  pig      \textsc{3sg.obj}=hit-\textsc{pfv}\\
\glt `When did Kapos kill the pig?’ (\textit{tem} < TP \textit{taim} ‘time’) [elicited]
\z

\ea%37
    \label{ex:syntax:37}
          \textit{Kapos mï} \textbf{\textit{awal}} \textit{lamndu masap.}\\
\gll    Kapos  mï      \textbf{awal}    lamndu  ma=asa-p\\
    [name]  \textsc{3sg.subj}  yesterday  pig      \textsc{3sg.obj=}hit-\textsc{pfv}\\
\glt `Kapos killed the pig yesterday.’ [elicited]
\z

Sentence \REF{ex:syntax:38} and \REF{ex:syntax:39} provide additional examples of \textit{ango tem} ‘when?’.

\ea%38
    \label{ex:syntax:38}
          \textit{Itom mï} \textbf{\textit{ango}} \textbf{\textit{tem}} \textit{utam mamap?}\\
\gll    itom  mï      \textbf{ango}  \textbf{tem}  utam  ma=ama-p\\
    father  \textsc{3sg.subj}  which  time  yam  3\textsc{sg.obj=}eat-\textsc{pfv}\\
\glt `When did father eat the yam?’ (\textit{tem} < TP \textit{taim} ‘time’) [elicited]
\z

\ea%39
    \label{ex:syntax:39}
          \textit{\textbf{Ango tem} man ninda?}\\
\gll    \textbf{ango}  \textbf{tem}  ma=n      ni-nda\\
    which  time  3\textsc{sg.obj=obl}  act-\textsc{irr}\\
\glt `When will [we] do it?’ (\textit{tem} < TP \textit{taim} ‘time’) [ulwa031\_01:27]
\z

Alternatively, the \isi{phrase} \textit{ango tem} ‘which time?’ (i.e., ‘when?’) can take the \isi{oblique marker} \textit{=n} ‘\textsc{obl}’, thus forming a phase meaning ‘with which time?’ (i.e., ‘at which time?’), as in \REF{ex:syntax:40} and \REF{ex:syntax:41}.

\ea%40
    \label{ex:syntax:40}
          \textit{Kapos mï} \textbf{\textit{ango}} \textbf{\textit{temnï}} \textit{lamndu masap?}\\
\gll    Kapos  mï      \textbf{ango}  \textbf{tem=nï}  lamndu  ma=asa-p\\
    [name]  \textsc{3sg.subj}  which  time=\textsc{obl}  pig      \textsc{3sg.obj}=hit-\textsc{pfv}\\
\glt `When did Kapos kill the pig?’ (\textit{tem} < TP \textit{taim} ‘time’) [elicited]
\z

\ea%41
    \label{ex:syntax:41}
          \textit{\textbf{Ango temnï} natana?}\\
\gll    \textbf{ango}  \textbf{tem=nï}  na-ta-na\\
    which  time=\textsc{obl}  \textsc{detr}{}-say-\textsc{irr}\\
\glt `When will the meeting start?’ (Literally ‘[They] will at which time talk?’) (\textit{tem} < TP \textit{taim} ‘time’) [elicited]
\z

Spatial questions are also formed with the \isi{question word} \textit{ango} ‘which?’. Unlike \isi{temporal} questions, however, \textit{ango} ‘which?’ usually occurs without overtly modifying any noun, such as, for example, a word meaning ‘place’. That is, when taken alone, \textit{ango} ‘which?’ is understood to mean ‘which \isi{location}?’. Again, the spatial \isi{question word} (or, possibly, abbreviated phrase) occurs in the same position as spatial modifiers in \isi{indicative} clauses, as illustrated by the pair of examples \REF{ex:syntax:42} and \REF{ex:syntax:43}.

\is{syntax|)}
\is{question|)}
\is{content question|)}
\is{wh- question|)}

\is{wh- question|(}
\is{content question|(}
\is{question|(}
\is{syntax|(}

\ea%42
    \label{ex:syntax:42}
          \textit{Ankam mï} \textbf{\textit{ango}} \textit{i?}\\
\gll    ankam  mï      \textbf{ango}  i\\
    person  \textsc{3sg.subj}  which  go.\textsc{pfv}\\
\glt `Where did the person go?’ [elicited]
\z

\ea%43
    \label{ex:syntax:43}
          \textit{Ankam mï} \textbf{\textit{ngaya}} \textit{i.}\\
\gll    ankam  mï      \textbf{ngaya}  i\\
    person  \textsc{3sg.subj}  far    go.\textsc{pfv}\\
\glt `The person went far away.’ [elicited]
\z

\isi{Motion} verbs in Ulwa can be \isi{transitive}, taking a \isi{goal} argument as their \isi{direct object}. Thus, in questions of ‘where to?’ or ‘whither?’, the \isi{question word} \textit{ango} ‘which?’ occurs in object position, as illustrated by the pair of examples \REF{ex:syntax:44} and \REF{ex:syntax:45}.

\ea%44
    \label{ex:syntax:44}
          \textit{Tangin mï} \textbf{\textit{ango}} \textit{i?}\\
\gll    Tangin  mï      \textbf{ango}  i\\
    [name]  \textsc{3sg.subj}  which  go.\textsc{pfv}\\
\glt `Where did Tangin go?’ [elicited]
\z

\ea%45
    \label{ex:syntax:45}
          \textit{Tangin mï} \textbf{\textit{wa}} \textit{may.}\\
\gll    Tangin  mï      \textbf{wa}    ma=i\\
    [name]  \textsc{3sg.subj}  village  \textsc{3sg.obj=}=go.\textsc{pfv}\\
\glt `Tangin went to the village.’ [elicited]
\z

As the object of the verb \textit{i} ‘go.\textsc{pfv}’, the noun \textit{wa} ‘village’ can be indexed with the \isi{object marker} \textit{ma=} ‘\textsc{3sg.obj}’ in \REF{ex:syntax:45}. Critically, however, the \isi{object marker} is not permitted in questions such as in \REF{ex:syntax:44}. Its inclusion would render an interpretation of [ango] as ‘\textsc{neg}’ rather than as ‘which?’, as illustrated by \REF{ex:syntax:46}.

\ea%46
    \label{ex:syntax:46}
          \textit{Tangin mï} \textbf{\textit{ango}} \textit{may.}\\
\gll    Tangin  mï      \textbf{ango}  ma=i\\
    [name]  \textsc{3sg.subj}  \textsc{neg}  \textsc{3sg.obj=}go.\textsc{pfv}\\
\glt    (a) ‘Tangin did not go [there].’

    (b) * ‘Where did Tangin go?’ [elicited]
\z

The \isi{source} of \isi{motion} (i.e., \isi{origin}), on the other hand, can be indicated as the object of the \isi{postposition} \textit{u} ‘from’. Thus, in questions of ‘where from?’ or ‘whence?’, the \isi{question word} \textit{ango} ‘which?’ occurs as the object of the \isi{postposition} \textit{u} ‘from’, as seen in the pair of sentences \REF{ex:syntax:47} and \REF{ex:syntax:48}.

\ea%47
    \label{ex:syntax:47}
          \textit{U} \textbf{\textit{ango}} \textbf{\textit{u}} \textit{mbi?}\\
\gll    u    \textbf{ango}  \textbf{u}    mbï-i\\
    \textsc{2sg}  which  from  here-go.\textsc{pfv}\\
\glt `Where did you come from?’ [elicited]
\z

\ea%48
    \label{ex:syntax:48}
          \textit{Nï} \textbf{\textit{wandam}} \textbf{\textit{u}} \textit{mbi.}\\
\gll    nï    \textbf{wandam}  \textbf{u}    mbï-i\\
    \textsc{1sg}  jungle    from  here-go.\textsc{pfv}\\
\glt `I came from the jungle.’ [elicited]
\z

Indications that \textit{ango} ‘where?’ is \isi{elliptical} for ‘which place?’ come from sentences such as \REF{ex:syntax:49}, which contains the entire \isi{phrase} \textit{ango luwa} ‘which place?’. This lengthier method of asking ‘where?’, however, seems to be relatively uncommon.

\ea%49
    \label{ex:syntax:49}
          \textit{Popo ndï un \textbf{ango luwa} pe.}\\
\gll    popo  ndï  u=n    \textbf{ango}  \textbf{luwa}  p-e\\
    papaya  3\textsc{pl}  \textsc{2sg=obl}  which  place  be\textsc{{}-ipfv}\\
\glt `Where are your papayas?’ (Literally ‘The papayas for you are at which place?’; \textit{popo} < TP \textit{popo} ‘papaya’) [ulwa014\_07:36]
\z

It may be noted that something of the \isi{negative} sense of [ango] is perhaps preserved in example \REF{ex:syntax:49}, since this is a \isi{rhetorical question} meant to imply ‘you have no papayas’. Sentence \REF{ex:syntax:50} is another example in which the full \isi{phrase} \textit{ango luwa} ‘which place?’ occurs.

\ea%50
    \label{ex:syntax:50}
          \textit{Ngun \textbf{ango luwa} wap?}\\
\gll    ngun  \textbf{ango}  \textbf{luwa}  wap\\
    2\textsc{du}  which  place  be.\textsc{pst}\\
\glt `Where were you?’ (Literally ‘You were at which place?’) [ulwa014\_40:07]
\z

As a modifying element, the \textit{ango} ‘which?’ component of the abbreviated \isi{phrase} ‘which place?’ can receive the \isi{copular enclitic} or be followed by a \isi{locative verb}, thereby serving as the \isi{predicate} of its own clause, as in examples \REF{ex:syntax:51}, \REF{ex:syntax:52}, and \REF{ex:syntax:53}.

\ea%51
    \label{ex:syntax:51}
          \textit{Unan \textbf{angop}?}\\
\gll unan    ango=\textbf{p}\\
    \textsc{1pl.incl}  which=\textsc{cop}\\
\glt `Where are we?’ (Literally ‘We are [at] which [place]?’) [elicited]
\z

\ea%52
    \label{ex:syntax:52}
          \textit{U \textbf{ango wap}?}\\
\gll u    ango  \textbf{wap}\\
    \textsc{2sg}  which  be.\textsc{pst}\\
\glt `Where were you?’ [elicited]
\z

\ea%53
    \label{ex:syntax:53}
          \textit{Yanapi mï \textbf{angopïna}?}\\
\gll Yanapi  mï      ango=\textbf{p-na}\\
    [name]  \textsc{3sg.subj}  which=\textsc{cop}{}-\textsc{irr}\\
\glt `Where will Yanapi be?’ [elicited]
\z

Such clauses with \isi{verbalized} ‘where?’ constructions can combine with other clauses, as in the \isi{question} in \REF{ex:syntax:54}.

\ea%54
    \label{ex:syntax:54}
          \textit{Itom mï} \textbf{\textit{angope}} \textit{lamndu masap?}\\
\gll    itom  mï      ango=\textbf{p-e}      lamndu  ma=asa-p\\
    father  \textsc{3sg.subj}  which=\textsc{cop-dep}  pig      \textsc{3sg.obj}=hit-\textsc{pfv}\\
\glt `Where did father kill the pig?’\footnote{Note that the \isi{verbalized} \textit{ango} ‘which?’ now functions as the linking element between two clauses, and accordingly receives both the \isi{copular enclitic} \textit{=p} ‘\textsc{cop}’ and the \isi{dependent marker} \textit{-e} ‘\textsc{dep}’.} (Literally something like ‘Father killed the pig, having been where?’) [elicited]
\z

  In a similar sentence, but with \isi{irrealis} \isi{modal}ity, the verb in each of the two clauses would be marked for \isi{irrealis} or \isi{conditional} \isi{mood} \REF{ex:syntax:55}.

\ea%55
    \label{ex:syntax:55}
          \textit{Itom mï} \textbf{\textit{angopïta}} \textit{lamndu mawalinda?}\\
\gll    itom  mï      ango=\textbf{p-ta}      lamndu  ma=wali-nda\\
    father  \textsc{3sg.subj}  which=\textsc{cop}{}-\textsc{cond}  pig      \textsc{3sg.obj}=hit-\textsc{irr}\\
\glt `Where will father kill the pig?’ (Literally ‘Father will kill the pig if [he] is where?’) [elicited]
\z

Finally, ‘why’ questions are formed with the \isi{question word} \textit{angwena} ‘why?’. Although this is pronounced as a single word, it, too, likely derives from a \isi{phrase} containing \textit{ango} ‘which?’. The second element probably derives from \textit{na} ‘talk, \isi{speech}, story, message, thought, reason, language’, here having the sense of ‘reason’ (i.e., ‘[for] what reason?’).\footnote{It is unclear why the form is pronounced as [angwena] as opposed to the expected \textsuperscript{†}[angona], but the pronunciation may have changed due to a folk etymological association with \textit{ina} ‘liver’, the seat of reasoning and emotion in the Ulwa conception of the human body. This form also appears in words such as \textit{inakawana-} ‘think’ (see \sectref{sec:9.2.1} for a proposed etymology). \citegen[3260]{Laycock1971a} field notes seem to indicate that the \ili{Yaul} \isi{dialect} form is \textit{angola} ‘why?’ (Appendix \ref{sec:app.g}), thereby supporting the etymology of *ango ‘which?’ + *la ‘talk, reason’.} The \isi{question}s in \REF{ex:syntax:56} through \REF{ex:syntax:59} all contain \textit{angwena} ‘why?’.

\is{syntax|)}
\is{question|)}
\is{content question|)}
\is{wh- question|)}

\ea%56
    \label{ex:syntax:56}
          \textit{U} \textbf{\textit{angwena}} \textit{mbi?}\\
\gll    u    \textbf{angwena}  mbï-i\\
    \textsc{2sg}  why    here-go.\textsc{pfv}\\
\glt `Why did you come here?’ [elicited]
\z

\ea%57
    \label{ex:syntax:57}
          \textit{Itom mï} \textbf{\textit{angwena}} \textit{apa maytap?}\\
\gll    itom  mï      \textbf{angwena}  apa    ma=ita-p\\
    father  3\textsc{sg.subj}  why    house  3\textsc{sg.obj}=build-\textsc{pfv}\\
\glt `Why did father build the house?’ [elicited]
\z

\ea%58
    \label{ex:syntax:58}
          \textit{Mï ndïn} \textbf{\textit{angwena}} \textit{ndït inde?}\\
\gll    mï      ndï=n    \textbf{angwena}  ndï=tï    inda-e\\
    3\textsc{sg.subj}  3\textsc{pl=obl}  why    3\textsc{pl}=take  walk-\textsc{ipfv}\\
\glt `Why is he walking around with them?’ [ulwa014\_14:02]
\z

\ea%59
    \label{ex:syntax:59}
          \textit{Un} \textbf{\textit{angwena}} \textit{mawat pe ne?}\\
\gll    un    \textbf{angwena}  ma=wat    p-e    ni{}-e\\
    2\textsc{pl}    why    3\textsc{sg.obj}=atop  be\textsc{{}-dep} act-\textsc{ipfv}\\
\glt `Why are you doing [things] during it [= this period of mourning]?’ [ulwa032\_03:15]
\z

\subsection{Multiple questions}\label{sec:13.1.3}

\is{wh- question|(}
\is{content question|(}
\is{question|(}
\is{syntax|(}
\is{multiple question|(}

Ulwa \isi{interrogative} constructions have the productive ability to \isi{question} multiple things simultaneously. Like \ili{English} constructions such as \textit{who gave what to whom?}, Ulwa constructions may inquire into multiple unknowns. An example of an Ulwa multiple-\isi{question} construction is given in \REF{ex:syntax:60}.

\ea%60
    \label{ex:syntax:60}
          \textbf{\textit{Ango}} \textit{luwa} \textbf{\textit{angos}} \textit{nji ndïlanda?}\\
\gll    \textbf{ango}  luwa  \textbf{angos}  nji    ndï=la-nda\\
    which  place  what  thing  3\textsc{pl}=eat-\textsc{irr}\\
\glt `Where will [we] find something to eat?’ (Literally ‘[We] will eat what things [at] which place?’) [ulwa032\_07:52]
\z

Whereas \ili{English} constructions like \textit{who gave what to whom?} are mostly limited to situations in which it is assumed by the asker that each \isi{question} component has a known referent, Ulwa multiple-\isi{question} constructions are more flexible. Thus, for example, the two \isi{question}s ‘where will we find food?’ and ‘what food will we find?’ may be combined into something like ‘where will we find what food?’, a sentence that would stretch the capacities of \ili{English} multiple-\isi{question} constructions. Examples \REF{ex:syntax:61},\footnote{Admittedly, this sample sentence is not a prototypical multiple-\isi{question} construction, since one of the two questioned elements is perhaps more properly considered an \isi{indefinite pronoun} (\sectref{sec:6.4}) as opposed to a \textit{wh-} word. An alternative analysis of these multiple \isi{question}s would be that these are sets of conjoined phrases with no overt \isi{conjunction} (e.g., ‘at what place and what thing will we eat?’, ‘from which place and what thing can we two get for them?’, etc.).} \REF{ex:syntax:62}, and \REF{ex:syntax:63} illustrate more multiple-\isi{question} constructions of this type.

\ea%61
    \label{ex:syntax:61}
          \textit{U \textbf{ango} luwa \textbf{angos} matïn?}\\
\gll    u    \textbf{ango}  luwa  \textbf{angos}  ma=tï-na\\
2\textsc{sg}  which  place  what  3\textsc{sg.obj}=take-\textsc{irr}\\
\glt `Where will you get something?’ [ulwa032\_12:10]
\z

\ea%62
    \label{ex:syntax:62}
          \textit{Ngan ndandï} \textbf{\textit{ango}} \textit{luwa u} \textbf{\textit{angos}} \textit{tïna?}\\
\gll    ngan    ndï=andï  \textbf{ango}  luwa  u    \textbf{angos}  tï-na\\
    1\textsc{du.excl}  3\textsc{pl}=for  which  place  from  what  take-\textsc{irr}\\
\glt `From which place can we two get what for them?’ [ulwa032\_19:58]
\z

\newpage

\ea%63
    \label{ex:syntax:63}
          \textit{E ngusuwa ko} \textbf{\textit{angwena}} \textbf{\textit{angos}} \textit{mundu wananda nat?}\\
\gll    e  ngusuwa  ko  \textbf{angwena}  \textbf{angos}  mundu  wana-nda  na-ta\\
    ay  poor    just  why    what  food  cook-\textsc{irr}  \textsc{detr}{}-say\\
\glt `Ay, why did that poor thing say that he would cook whatever kind of food?’ (Literally ‘Why did the poor thing say that [he] would cook what food?’) [ulwa014\_15:28]
\z

  Multiple \isi{question}s can also be expressed in what are clearly multiple clauses. In \REF{ex:syntax:64}, the \isi{conditional} form \textit{-ta} ‘\textsc{cond}’ marks the end of the first clause – that is, the \isi{protasis}.

\is{multiple question|)}
\is{syntax|)}
\is{question|)}
\is{content question|)}
\is{wh- question|)}

\ea%64
    \label{ex:syntax:64}
          \textit{Ndï} \textbf{\textit{ango}} \textit{luwa wandam luta} \textbf{\textit{angos}} \textit{mundu malan?}\\
\gll    ndï  \textbf{ango}  luwa  wandam  lo-ta    \textbf{angos}  mundu     ma=la-n[da]\\
    3\textsc{pl}  which  place  jungle    go-\textsc{cond}  what  food    3\textsc{sg.obj}=eat-\textsc{irr}\\
\glt `Where will they go and what will they eat?’ (Literally ‘If they go to which jungles, what food will [they] eat?’) [ulwa032\_14:00]
\z

\subsection{Rhetorical questions}\label{sec:13.1.4}

\is{wh- question|(}
\is{content question|(}
\is{question|(}
\is{syntax|(}
\is{rhetorical question|(}

Questions often serve rhetorical purposes – that is, a speaker may not be actually requesting information, but rather may be making an argument (usually anticipating a \isi{negative response} to the \isi{rhetorical question}). Example \REF{ex:syntax:65} illustrates how these may be made in Ulwa.

\ea%65
    \label{ex:syntax:65}
          \textit{Ndï nji ndïwatlunda?}\\
\gll ndï  nji    ndï=wat-lo-nda\\
    3\textsc{pl}  thing  3\textsc{pl=}atop-cut-\textsc{irr}\\
\glt `Will they clear the things?’ (The anticipated \isi{response} is: ‘No, they will not.’) [ulwa014\_53:02]
\z

Rhetorical questions can be either \isi{polar question}s or \isi{content question}s. In polar rhetorical questions, the anticipated \isi{response} is ‘no’; in content rhetorical questions, the anticipated \isi{response} is ‘nothing’, ‘nowhere’, ‘nobody’, and so on. Example \REF{ex:syntax:66} contains first a \isi{polar question}, and then a \isi{content question}.

\ea%66
    \label{ex:syntax:66}
          \textit{U ko wandam nji ndï \textbf{ango luwa} pe? U ko lïmndï ndala?}\\
\gll    u    ko  wandam  nji    ndï  \textbf{ango}  \textbf{luwa}  p-e    u    ko     lïmndï  ndï=ala\\
    2\textsc{sg}  just  jungle    thing  3\textsc{pl}  which  place  be\textsc{{}-dep}  \textsc{2sg} just    eye    3\textsc{pl}=see\\
\glt `Where are your jungle properties? Do you see them?’ [ulwa032\_39:45]
\z

The first \isi{question} is literally ‘Your jungle things have which place?’ The anticipated \isi{response} to it is: ‘No place’. The anticipated \isi{response} to the second \isi{question} is: ‘No’. Example \REF{ex:syntax:66} also illustrates the use of the \isi{modal adverb} \textit{ko} ‘just’, which may be used for emphasis in rhetorical questions.

\is{rhetorical question|)}
\is{syntax|)}
\is{question|)}
\is{content question|)}
\is{wh- question|)}

\section{Commands and requests}\label{sec:13.2}

\is{command|(}
\is{request|(}
\is{syntax|(}

Commands (or requests) are, generally, built around an \isi{imperative} form of a verb (\sectref{sec:4.7}). Imperative sentences may contain an expressed subject (typically a second \isi{person} \isi{pronoun}), but, as in all sentence types, it is possible for the subject to be omitted. Examples \REF{ex:syntax:67} through \REF{ex:syntax:71} illustrate how second \isi{person} \isi{pronoun}s may be included in \isi{imperative} sentences.

\ea%67
    \label{ex:syntax:67}
          \textbf{\textit{U}} \textit{nul} \textbf{\textit{man}}\textit{!}\\
\gll    \textbf{u}    nï=ul    ma-\textbf{n}\\
    2\textsc{sg}  1\textsc{sg}=with  go-\textsc{imp}\\
\glt `Go with me!’ (said to one person) [ulwa014\_70:49]
\z

\ea%68
    \label{ex:syntax:68}
          \textbf{\textit{Ngun}} \textbf{\textit{naman}}\textit{!}\\
\gll    \textbf{ngun}  na-ma-\textbf{n}\\
    2\textsc{du}  \textsc{detr-}go-\textsc{imp}\\
\glt `Go!’ (said to two people) [ulwa001\_09:46]
\z

\ea%69
    \label{ex:syntax:69}
          \textbf{\textit{U}} \textit{ikali} \textbf{\textit{ngasin}}\textit{!}\\
\gll    \textbf{u}    i-kali    nga=si-\textbf{n}\\
    2\textsc{sg}  hand-send  \textsc{sg.prox}=push-\textsc{imp}\\
\glt `Hold this!’ (said to one person) [ulwa014\_62:22]
\z

\ea%70
    \label{ex:syntax:70}
          \textbf{\textit{U}} \textit{manji ndï nan} \textbf{\textit{makïn}}\textit{!}\\
\gll    \textbf{u}    ma-nji      ndï  na=n    ma=kï-\textbf{n}\\
    2\textsc{sg}  3\textsc{sg.obj-poss}  3\textsc{pl}  talk=\textsc{obl}  3\textsc{sg.obj}=say-\textsc{imp}\\
\glt `Tell her about her [sago palms]!’ (said to one person) [ulwa037\_42:10]
\z

\ea%71
    \label{ex:syntax:71}
          \textbf{\textit{Un}} \textit{maya wa} \textbf{\textit{nayn}}\textit{!}\\
\gll    \textbf{un}  ma=iya      wa    na-i-\textbf{n}\\
    2\textsc{pl}  3\textsc{sg.obj=}toward  village  \textsc{detr-}come-\textsc{imp}\\
\glt `Come home to her!’ (said to multiple people) [ulwa032\_04:30]
\z

In the \isi{imperative} sentences shown in \REF{ex:syntax:72}, \REF{ex:syntax:73}, and \REF{ex:syntax:74}, the second \isi{person} subject is not expressed.

\ea%72
    \label{ex:syntax:72}
          \textit{Amun} \textbf{\textit{man}}\textit{!}\\
\gll    amun  ma-\textbf{n}\\
    now  go-\textsc{imp}\\
\glt `Go now!’ [elicited]
\z

\ea%73
    \label{ex:syntax:73}
          \textit{Unji mat} \textbf{\textit{indan}}\textit{!}\\
\gll    u-nji    ma=tï      inda-\textbf{n}\\
    2\textsc{sg-poss}  \textsc{3sg.obj=}take  walk-\textsc{imp}\\
\glt `Carry your [child] around!’ [ulwa032\_17:15]
\z

\ea%74
    \label{ex:syntax:74}
          \textit{Unji al kwa ndawa ka} \textbf{\textit{lowon}}\textit{!}\\
\gll    u-nji    al  kwa  anda-awa    ka  lo-wo-\textbf{n}\\
    2\textsc{sg-poss}  net  one    \textsc{sg.dist-int}  in  \textsc{irr-}sleep-\textsc{imp}\\
\glt `Sleep in that other mosquito net of yours!’ [ulwa011\_01:42]
\z

\isi{Third person imperative}s (or \isi{jussive}s) are also possible. These are no different from prototypical second \isi{person} imperatives: they, too, contain a verb with the \isi{imperative} \isi{suffix}; the only difference is that the \isi{command} is issued to a third \isi{person} referent. Sentences \REF{ex:syntax:75}, \REF{ex:syntax:76}, and \REF{ex:syntax:77} are examples of \isi{third person imperative}s in Ulwa.

\ea%75
    \label{ex:syntax:75}
          \textit{Mï} \textbf{\textit{lan}}\textit{!}\\
\gll    mï      la{}-\textbf{n}\\
    \textsc{3sg.subj}  eat-\textsc{imp}\\
\glt `Let him eat!’ [elicited]
\z

\ea%76
    \label{ex:syntax:76}
          \textit{Ndï} \textbf{\textit{wutïnin}}\textit{!}\\
\gll    ndï  wutï-ni-\textbf{n}\\
    \textsc{3pl}  leg-beat-\textsc{imp}\\
\glt `Let them dance!’  [elicited]
\z

\ea%77
    \label{ex:syntax:77}
          \textit{Kalingana kalilïta mï} \textbf{\textit{man}}\textit{!}\\
\gll    Kalingana  kali-lï-ta      mï      ma-\textbf{n}\\
    [name]    send-put\textsc{{}-cond} 3\textsc{sg.subj}  go-\textsc{imp}\\
\glt `Send Kalingana and he’ll go!’ (Literally ‘If [you] send Kalingana, let him go!’) [ulwa018\_01:00]
\z

\isi{First person imperative}s (or \isi{hortative}s) are possible as well, but only for \isi{non-singular} \isi{inclusive} forms. That is, at least one addressee must be included in the \isi{exhortation}. Sentences \REF{ex:syntax:78}, \REF{ex:syntax:79}, and \REF{ex:syntax:80} are examples of \isi{first person imperative}s in Ulwa.

\ea%78
    \label{ex:syntax:78}
          \textit{Ngunan} \textbf{\textit{lan}}\textit{!}\\
\gll ngunan    la{}-\textbf{n}\\
    1\textsc{du.incl}  eat-\textsc{imp}\\
\glt `Let’s eat!’ [elicited]
\z

\ea%79
    \label{ex:syntax:79}
          \textit{Unan} \textbf{\textit{ndïlan}}\textit{!}\\
\gll    unan    ndï=la-\textbf{n}\\
    1\textsc{pl.incl}  \textsc{3pl}=eat-\textsc{imp}\\
\glt `Let’s eat them!’ [ulwa037\_45:40]
\z

\ea%80
    \label{ex:syntax:80}
          \textit{Una} \textbf{\textit{man}}\textit{!}\\
\gll    unan    ma-\textbf{n}\\
    1\textsc{pl.incl}  go-\textsc{imp}\\
\glt `Let’s go!’ [ulwa014\_66:54]
\z

Indeed, the only referents that cannot be the subjects of \isi{imperative}s are first \isi{person} non-\isi{inclusive} forms -- that is, first \isi{person} \isi{singular}, first \isi{person} \isi{dual} \isi{exclusive}, and first \isi{person} \isi{plural} \isi{exclusive}, as illustrated by the ungrammatical sentences \REF{ex:syntax:81} and \REF{ex:syntax:82}. Similar constructions containing these pronominal forms, however, can be created with the \isi{irrealis} \isi{suffix}, as illustrated by sentences \REF{ex:syntax:83} and \REF{ex:syntax:84}.

\ea[*]{%81
    \label{ex:syntax:81}
           \textit{Nï} \textbf{\textit{lan}}\textit{!}\\
\gll    nï    la\textbf{{}-n}\\
    \textsc{1sg}  eat-\textsc{imp}\\
\glt `Let me eat!’ [elicited]
}
\z

\ea[*]{%82
    \label{ex:syntax:82}
           \textit{An} \textbf{\textit{lan}}\textit{!}\\
\gll    an      la-\textbf{n}\\
    \textsc{1pl.excl}  eat-\textsc{imp}\\
\glt `Let’s eat!’ [elicited]}
\z

\ea%83
    \label{ex:syntax:83}
          \textit{Nï} \textbf{\textit{landa}}.\\
\gll nï    la-\textbf{nda}\\
    \textsc{1sg}  eat-\textsc{irr}\\
\glt `I should eat.’ [elicited]
\z

\ea%84
    \label{ex:syntax:84}
          \textit{An} \textbf{\textit{landa}}.\\
\gll an      la-\textbf{nda}\\
    \textsc{1pl.excl}  eat-\textsc{irr}\\
\glt `We should eat.’ [elicited]
\z

The issue is, however, complicated, since, in casual speech, speakers commonly drop verbal endings, especially of \isi{irrealis} verb forms. Thus, among the collected texts there are examples of \isi{irrealis} clauses with, for example, 1\textsc{sg} subjects that do appear to employ the \isi{imperative} \isi{suffix} \textit{-n} `\textsc{imp}’, as in \REF{ex:syntax:85}. I consider it more likely, however, that this \isi{alveolar} \isi{nasal} represents an abbreviated form of the \isi{irrealis} \isi{suffix} \textit{-na} {\textasciitilde} \textit{-nda} `\textsc{irr}’.

\ea%85
    \label{ex:syntax:85}
          \textit{Nï ma ndïn} \textbf{\textit{lun}}.\\
\gll nï    ma  ndï=n    lo-\textbf{n[da]}\\
    1\textsc{sg}  go  3\textsc{pl=obl}  cut-\textsc{irr}\\
\glt `I will go and plant them.’ [ulwa014\_08:10]
\z

\isi{Prohibition}s (i.e., \isi{negative command}s) are treated separately from true \isi{imperative}s, not only since they require a special word, \textit{wana} {\textasciitilde} \textit{wanap} ‘\textsc{proh’}, but also because they do not permit the \isi{imperative} \isi{suffix}. \isi{Prohibition}s may be issued to any referent, including first \isi{person} non-\isi{inclusive} forms (see \sectref{sec:13.2.4} for examples).

\is{syntax|)}
\is{request|)}
\is{command|)}

\subsection{Irrealis for imperative}\label{sec:13.2.1}

\is{irrealis|(}
\is{imperative|(}
\is{command|(}
\is{request|(}
\is{syntax|(}

The fact that the \isi{irrealis} \isi{suffix} can encode \isi{deontic modality} (\sectref{sec:4.6}) -- and, specifically, a \isi{directive mood} -- means that it may function very much like an \isi{imperative} \isi{suffix}. Indeed, it is possible that the \isi{imperative} \isi{suffix} derives historically from the \isi{irrealis} \isi{suffix} -- that is, as an \is{apocope} apocopated version, which could be expected to occur in emphatic direct address.

  Thus, some clauses containing \isi{irrealis} verbs may be functionally equivalent to \isi{imperative}s, and they may therefore be translated as such in \ili{English}, as in the first translation of examples \REF{ex:syntax:86}, \REF{ex:syntax:87}, and \REF{ex:syntax:88}.

\ea%86
    \label{ex:syntax:86}
          \textit{U} \textbf{\textit{landa}}\textit{!}\\
\gll    u    la-\textbf{nda}\\
    2\textsc{sg}  eat-\textsc{irr}\\
\glt    (a) ‘Eat!’

    (b) ‘You must eat.’ [elicited]
\z

\ea%87
    \label{ex:syntax:87}
          \textit{Asa u mat} \textbf{\textit{nïnanda}}\textit{!}\\
\gll    asa  u    ma=tï      nï=na-\textbf{nda}\\
    no  2\textsc{sg}  3\textsc{sg.obj}=take  \textsc{1sg}=give-\textsc{irr}\\
\glt    (a) ‘No, give it to me!’

    (b) ‘No, you should give it to me.’ [ulwa032\_28:41]
\z

\ea%88
    \label{ex:syntax:88}
          \textit{Kïkal misimisi} \textbf{\textit{ngawananda}}\textit{!}\\
\gll    kïkal  misimisi  nga=wana-\textbf{nda}\\
    ear    story    \textsc{sg.prox}=feel-\textsc{irr}\\
\glt    (a) ‘Listen to this story!’

    (b) ‘[You] must/should listen to this story.’

    (c) ‘Would that [you] were listening to this story!’ [elicited]
\z

This use of the \isi{irrealis} \isi{suffix} also applies to \isi{third person imperative}s \REF{ex:syntax:89} and \isi{first person imperative}s, whether \isi{dual} \REF{ex:syntax:90} or \isi{plural} \REF{ex:syntax:91}.

\is{syntax|)}
\is{request|)}
\is{command|)}
\is{imperative|)}
\is{irrealis|)}

\ea%89
    \label{ex:syntax:89}
          \textit{Mï} \textbf{\textit{landa}}\textit{!}\\
\gll    mï      la-\textbf{nda}\\
    \textsc{3sg.subj}  eat-\textsc{irr}\\
\glt    (a) ‘Let him eat!’

    (b) ‘He must eat.’

    (c) ‘Oh that he would eat!’ [elicited]
\z

\ea%90
    \label{ex:syntax:90}
          \textit{Ngunan} \textbf{\textit{mana}}\textit{!}\\
\gll    ngunan    ma-\textbf{na}\\
    1\textsc{du.incl}  go-\textsc{irr}\\
\glt    (a) ‘We shall go.’

    (b) ‘Let’s go!’ [ulwa001\_03:28]
\z

\ea%91
    \label{ex:syntax:91}
          \textit{Una} \textbf{\textit{mana}}\textit{!}\\
\gll    unan    ma-\textbf{na}\\
    1\textsc{pl.incl}  go-\textsc{irr}\\
\glt    (a) ‘We must go.’

    (b) ‘Let’s go!’ [ulwa014\_24:41]
\z


\subsection{The modal adverb \textit{kop} ‘please’}\label{sec:13.2.2}

\is{modal adverb|(}
\is{adverb|(}
\is{command|(}
\is{request|(}
\is{syntax|(}
\is{mood|(}

Generally, no distinction is made between commands and requests -- that is, there is no common \is{formulaic expression} formulaic \isi{question} form (as in, for example, \ili{English} \textit{can you please pass the salt?}) to signal a gentle \isi{request} as opposed to a stern \isi{command}. Typically, \isi{intonation} and context alone define an \isi{imperative} form as serving the \isi{pragmatic} functions of either \isi{command} or \isi{request}. There are, however, two other formal devices for indicating requests as opposed to commands: the \isi{adverb} \textit{kop} ‘please’ and the \isi{conditional} \isi{suffix} \textit{-ta} ‘\textsc{cond}’ (\sectref{sec:13.2.3}). Since these devices are softer than commands made with only the \isi{imperative} verb form, they may be considered akin to requests.

  The \isi{modal adverb} \textit{kop} ‘please’ (\sectref{sec:8.2.5}) may be used to soften a \isi{command}, as seen in sentences \REF{ex:syntax:92} and \REF{ex:syntax:93}, which contain \isi{imperative} verb forms (\sectref{sec:4.7}).

\is{mood|)}
\is{syntax|)}
\is{request|)}
\is{command|)}
\is{adverb|)}
\is{modal adverb|)}

\ea%92
    \label{ex:syntax:92}
          \textit{I apa i} \textbf{\textit{kop}} \textit{lamap we un} \textbf{\textit{man}}\textit{!}\\
\gll    i    apa    i    \textbf{kop}  la{}-ama-p    we    un  ma-\textbf{n}\\
    go.\textsc{pfv}  house  go.\textsc{pfv}  please  \textsc{irr}{}-eat-\textsc{pfv}  then  \textsc{2pl}  go-\textsc{imp}\\
\glt `Come, come to the house, eat, and then go!’ [ulwa013\_03:47]
\z

\ea%93
    \label{ex:syntax:93}
          \textbf{\textit{Kop}} \textbf{\textit{malakan}}\textit{!}\\
\gll    \textbf{kop}  ma=la-ka-\textbf{n}\\
    please  3\textsc{pl}=\textsc{irr-}let-\textsc{imp}\\
\glt `Just leave him!’ [ulwa014\_08:15]
\z

\subsection{Conditionals used for requests}\label{sec:13.2.3}

\is{conditional|(}
\is{command|(}
\is{request|(}
\is{syntax|(}

Another method of softening a \isi{command} is using a \isi{conditional} verb form – that is, one with the ending \textit{-ta} ‘\textsc{cond}’ (\sectref{sec:4.12}, \sectref{sec:13.5}), as illustrated by \REF{ex:syntax:94}.

\ea%94
    \label{ex:syntax:94}
          \textit{Nï umbe Supam ul} \textbf{\textit{mata}} \textit{mï maya ata mana.}\\
\gll    nï    umbe    Supam  ul    ma-\textbf{ta}    mï     ma=iya      ata  ma-na\\
    \textsc{1sg}  tomorrow  [name]  with  go-\textsc{cond}  3\textsc{sg.subj}    \textsc{3sg.obj}=toward  up  go-\textsc{irr}\\
\glt `I’ll go with Supam tomorrow and she’ll climb it [= a tree].’ [ulwa001\_01:14]
\z

In the story from which sentence \REF{ex:syntax:94} is taken, a mother is addressing her children, including Supam. While the literal meaning of the first clause is ‘if I go with Supam …’, it has the \isi{pragmatic} value of ‘Supam, you shall go with me …!’ This \isi{imperative} use of a typically \isi{dependent} \isi{conditional} clause may thus be taken as an example of \isi{insubordination} \citep{Evans2007}. Further examples of \isi{conditional sentence}s functioning as softened commands are given in \REF{ex:syntax:95}, \REF{ex:syntax:96}, and \REF{ex:syntax:97}.

\ea%95
    \label{ex:syntax:95}
          \textbf{\textit{Nïlakata}} \textit{nï mawl malanda!}\\
\gll    nï=la-ka-\textbf{ta}    nï    ma=ul      ma=la-nda\\
    \textsc{1sg}=\textsc{irr-}let-\textsc{cond}  \textsc{1sg}  3\textsc{sg.obj}=with  3\textsc{sg.obj}=eat-\textsc{irr}\\
\glt `Let me eat with him!’ (Literally ‘If [you] let me, I will eat with him.’) [ulwa001\_06:11]
\z

\newpage

\ea%96
    \label{ex:syntax:96}
          \textit{Yena ngalat} \textbf{\textit{ndïnata}} \textit{ndï ndul lowope lunda!}\\
\gll    yena  ngala=tï    ndï=na-\textbf{ta}      ndï  ndï=ul     lo-wo-p-e      lo-nda\\
    woman  \textsc{pl.prox}=take  \textsc{3pl}=give-\textsc{cond}  \textsc{3pl}  \textsc{3pl=}with    \textsc{irr}{}-sleep-\textsc{pfv-dep}  go-\textsc{irr}\\
\glt `Give them these women, and they, having slept with them, will go!’ (Literally ‘If [you] give them these women, they, having slept with them, will go.’) [ulwa002\_06:04]
\z

\ea%97
    \label{ex:syntax:97}
          \textit{Kwa nïnji mol niya wa} \textbf{\textit{ita}} \textit{nï ko lïmndï mandïn.}\\
\gll    kwa  nï-nji    ma=ul      nï=iya      wa    i-\textbf{ta} nï    ko  lïmndï  ma=andï-na\\
    just    1\textsc{sg-poss}  \textsc{3sg.obj=}with  1\textsc{sg}=toward  village  go.\textsc{pfv-cond}    1\textsc{sg}  just  eye    3\textsc{sg.obj}=see-\textsc{irr}\\
\glt `If [you] come home to me with my [cousin], I will see her.’ (i.e., ‘Please bring my cousin to me so that I can see her!’) [ulwa037\_46:36]
\z

The \isi{conditional} form may also be used with first \isi{person} commands (i.e., \isi{exhortation}s, \sectref{sec:13.2}). Often, only the \isi{protasis} (marked with the \isi{conditional} \isi{suffix} \textit{-ta} ‘\textsc{cond}’) is expressed, leaving the \isi{apodosis} only implied, as in \REF{ex:syntax:98}.

\ea%98
    \label{ex:syntax:98}
          \textit{Unan na kali wa alan} \textbf{\textit{lïta}}\textit{!}\\
\gll    unan    na    kali  wa    ala=n      lï-\textbf{ta}\\
    1\textsc{pl.incl}  talk  send  village  \textsc{pl.dist=obl}  put\textsc{{}-cond}\\
\glt `Let’s send a message to those villages!’ (Literally ‘If we send a message to those villages …’) [ulwa001\_15:22]
\z

The \isi{modal adverb} \textit{kop} ‘please’ may be used in conjunction with the \isi{conditional} verb form \REF{ex:syntax:99}.

\ea%99
    \label{ex:syntax:99}
          \textbf{\textit{Kop}} \textit{ma wa na} \textbf{\textit{ndïtata}} \textit{mata!}\\
\gll    \textbf{kop}  ma  wa  na    ndï=ta-\textbf{ta}      ma-ta\\
    please  go  just  talk  3\textsc{pl}=say-\textsc{cond}  go-\textsc{cond}\\
\glt `Please, just go and tell stories!’ (Literally ‘If [you] please just go and say the talks, [it] will go.’) [ulwa014†]
\z

\newpage

The form \textit{kop} ‘please’ may be shortened to [ko], as in \REF{ex:syntax:100} and \REF{ex:syntax:101}.

\is{syntax|)}
\is{request|)}
\is{command|)}
\is{conditional|)}

\ea%100
    \label{ex:syntax:100}
          \textbf{\textit{Ko}} \textbf{\textit{ngapta}} \textit{apa itap nji ngalembam pïn!}\\
\gll    \textbf{kop}  nga=p-\textbf{ta}        apa    ita-p    nji ngala=imbam    p-n\\
    please  \textsc{sg.prox}=be\textsc{{}-cond} house  build-\textsc{pfv}  thing    \textsc{pl.prox=}under  be-\textsc{imp}\\
\glt `Please build a house here under these things!’ (Literally ‘If [you] please be this, build a house under these things.’) [ulwa014\_53:10]
\z

\ea%101
    \label{ex:syntax:101}
          \textbf{\textit{Ko}} \textbf{\textit{amblakalampïta}} \textit{lun!}\\
\gll    \textbf{kop}  ambla=kalam=p-\textbf{ta}        lo-n\\
    please  \textsc{pl.refl}=knowledge=\textsc{cop-cond}   go-\textsc{imp}\\
\glt `Please look after yourselves and go!’ (Literally ‘If [you] please know yourselves, go!’) [ulwa037\_63:08]
\z

\subsection{Negative commands}\label{sec:13.2.4}

\is{negative command|(}
\is{prohibition|(}
\is{command|(}
\is{request|(}
\is{syntax|(}

Negative commands are formed with the \isi{prohibitive marker} \textit{wana} \textsc{‘proh}’ or \textit{wanap} \textsc{‘proh}’, which occurs along with an \isi{irrealis} verb form (and not with an \isi{imperative} verb form, \sectref{sec:8.3.1}). The prohibitive marker occurs in the same position as the standard \isi{negator} \textit{ango} \textsc{‘neg}’, which is used to negate, for example, \isi{declarative} statements. In other words, the prohibitive marker follows the subject and precedes the verb \isi{phrase}, including any expressed object in \isi{transitive} clauses. As in \isi{positive} commands, it is common for the second \isi{person} subject of negative commands to be omitted, as in \REF{ex:syntax:102} and \REF{ex:syntax:103}.

\ea%102
    \label{ex:syntax:102}
          \textbf{\textit{Wana}} \textit{nunu nji tï ip lïp mana!}\\
\gll    \textbf{wana}  nunu  nji    tï    ip    lï-p      ma-na\\
    \textsc{proh}  every  thing  take  nose  put-\textsc{pfv}  go-\textsc{irr}\\
\glt `Don’t go destroying everything!’ (Literally ‘Don’t go, having put nose to everything!’) [ulwa014†]
\z

\ea%103
    \label{ex:syntax:103}
          \textit{Angani i} \textbf{\textit{wanap}} \textit{makape na!}\\
\gll    angani  i    \textbf{wanap}  maka=p-e    na\\
    behind  go.\textsc{pfv}  \textsc{proh}  thus=\textsc{cop-dep}  talk\\
\glt `Later, when [you] have come, do not [make] talk like this!’ [ulwa014†]
\z

Prohibitions, however, are not limited to second \isi{person} forms, but may apply to any \isi{person} or \isi{number}, as seen in examples \REF{ex:syntax:104} through \REF{ex:syntax:109}.

\ea%104
    \label{ex:syntax:104}
          \textit{(U)} \textbf{\textit{wana}} \textit{nuwalinda!}\\
\gll    (u)    \textbf{wana}  nï=wali-nda\\
    (\textsc{2sg)}  \textsc{proh}  \textsc{1sg=}hit-\textsc{irr}\\
\glt `Don’t hit me!’ (commanded to one person) [elicited]
\z

\ea%105
    \label{ex:syntax:105}
          \textit{(Un)} \textbf{\textit{wana}} \textit{nïnji utam malanda!}\\
\gll    (un)  \textbf{wana}  nï-nji    utam  ma=la-nda\\
    (\textsc{2pl)}  \textsc{proh}  \textsc{1sg-poss}  yam  \textsc{3sg.obj=}eat-\textsc{irr}\\
\glt `Don’t eat my yam!’ (commanded to more than two people) [elicited]
\z

\ea%106
    \label{ex:syntax:106}
          \textit{Mï} \textbf{\textit{wana}} \textit{landa!}\\
\gll    mï      \textbf{wana}  la-nda\\
    \textsc{3sg.subj}  \textsc{proh}  eat-\textsc{irr}\\
\glt `Don’t let him eat!’ [elicited]
\z

\ea%107
    \label{ex:syntax:107}
          \textit{Unan} \textbf{\textit{wana}} \textit{mana!}\\
\gll    unan    \textbf{wana}  ma-na\\
    \textsc{1pl.incl}  \textsc{proh}  go-\textsc{irr}\\
\glt `Let’s not go!’ [elicited]
\z

\ea%108
    \label{ex:syntax:108}
          \textit{An} \textbf{\textit{wana}} \textit{nakïna!}\\
\gll    an      \textbf{wana}  na-kï-na\\
    \textsc{1pl.excl}  \textsc{proh}  \textsc{detr-}say-\textsc{irr}\\
\glt `We shouldn’t talk.’ [elicited]
\z

\ea%109
    \label{ex:syntax:109}
          \textit{Nï} \textbf{\textit{wana}} \textit{mana!}\\
\gll    nï    \textbf{wana}  ma-na\\
    \textsc{1sg}  \textsc{proh}  go-\textsc{irr}\\
\glt `I shouldn’t go.’ [elicited]
\z

Prohibitions may include the \isi{speculative} \isi{suffix} \textit{-t} ‘\textsc{spec}’ on the \isi{irrealis} verb form (\sectref{sec:4.11}), as in examples \REF{ex:syntax:110} through \REF{ex:syntax:115}.

\ea%110
    \label{ex:syntax:110}
          \textit{Tarambi \textbf{wana} apka nïklop ma ngaya \textbf{manat}!}\\
\gll    Tarambi  \textbf{wana}  apka  nï=klop  ma  ngaya  ma-na-\textbf{t}\\
    [name]    \textsc{proh}  very  1\textsc{sg}=cross  go  far    go-\textsc{irr-spec}\\
\glt `Tarambi, don’t go completely bypass me and go far away!’ [ulwa014†]
\z

\ea%111
    \label{ex:syntax:111}
          \textit{\textbf{Wana} \textbf{ndïwalindat}!}\\
\gll    \textbf{wana}  ndï=wali-nda-\textbf{t}\\
    \textsc{proh}  3\textsc{pl}=hit-\textsc{irr-spec}\\
\glt `Don’t shoot them!’ [ulwa037\_47:35]
\z

\ea%112
    \label{ex:syntax:112}
          \textit{\textbf{Wanap} \textbf{mbïpïnate}!}\\
\gll    \textbf{wanap}  mbï-p-na-\textbf{t}-e\\
    \textsc{proh}  here-be-\textsc{irr-spec-dep}\\
\glt `Don’t stay here!’ [ulwa001\_01:40]
\z

\ea%113
    \label{ex:syntax:113}
          \textit{\textbf{Wana} ata ma Kambaramba \textbf{manat}!}\\
\gll    \textbf{wana}  ata  ma  Kambaramba  ma-na-\textbf{t}\\
    \textsc{proh}  up  go  [place]      go-\textsc{irr-spec}\\
\glt `Don’t go up to Kambaramba [village]!’ [ulwa014†]
\z

\ea%114
    \label{ex:syntax:114}
          \textit{Inim \textbf{wana} \textbf{malakanat} ko man ambi ndalan!}\\
\gll    inim  \textbf{wana}  ma=la-ka-na-\textbf{t}          ko  ma=n      ambi     anda=la-n\\
    water  \textsc{proh}  3\textsc{sg.obj}=\textsc{irr}{}-let-\textsc{irr-spec}  just  3\textsc{sg.obj=obl}  big    \textsc{sg.dist}=eat-\textsc{imp}\\
\glt `Water -- don’t avoid it; drink a lot of it!’ [ulwa014†]
\z

\ea%115
    \label{ex:syntax:115}
          \textit{\textbf{Wana} imba pïta niya \textbf{mbundanat}!}\\
\gll    \textbf{wana}  imba  p-ta    nï=iya      mbï-unda-na-\textbf{t}\\
    \textsc{proh}  night  be\textsc{{}-cond} 1\textsc{sg}=toward  here-go-\textsc{irr-spec}\\
\glt `Don’t come around here to me at night!’ (Literally ‘Don’t, when it is night, come around here to me!’) [ulwa014\_14:22]
\z

Example \REF{ex:syntax:115} also illustrates the use of the \isi{conditional} \isi{suffix} \textit{-ta} ‘\textsc{cond}’ (\sectref{sec:4.12}). Although here it is used to show an actual condition (along with the \isi{speculative} \isi{suffix} on the final \isi{irrealis}-marked verb), it may also be used \isi{idiom}atically in prohibitions, presumably to present an implied \isi{apodosis} (i.e., ‘or else …!’), as in \REF{ex:syntax:116} and \REF{ex:syntax:117}.

\ea%116
    \label{ex:syntax:116}
          \textbf{\textit{Wana}} \textbf{\textit{mapta}}\textit{!}\\
\gll    \textbf{wana}  ma=p-\textbf{ta}\\
    \textsc{proh}  \textsc{3sg.obj}=be\textsc{{}-cond}\\
\glt `Don’t live there!’ [ulwa014†]
\z

\ea%117
    \label{ex:syntax:117}
          \textbf{\textit{Wana}} \textbf{\textit{mbundata}} \textit{inim lata} \textbf{\textit{makapta}}\textit{!}\\
\gll    \textbf{wana}  mbï-unda\textbf{{}-ta} inim  la-ta    maka=p-\textbf{ta}\\
    \textsc{proh}  here-go-\textsc{cond}  water  eat\textsc{{}-cond} thus=\textsc{cop-cond}\\
\glt `Don’t come around here and drink beer like that!’ [ulwa014†]
\z

The prohibitive marker \textit{wana} {\textasciitilde} \textit{wanap}  \textsc{‘proh}’ probably originated as the verb \textit{wana-} ‘hear’, likely originally occurring sentence-finally, but later being \isi{reanalyzed} as a negative marker (as opposed to a verb) and thus migrating to the canonical sentence position for \isi{negator}s. See \citet[118]{Barlow2020b} for discussion of a similar \isi{grammaticalization} process in \ili{Pondi}.

\is{syntax|)}
\is{request|)}
\is{command|)}
\is{prohibition|)}
\is{negative command|)}

\section{Negation}\label{sec:13.3}

\is{negation|(}
\is{syntax|(}

This section concerns sentences that exhibit negative \isi{polarity}. There is no \isi{verbal morphology} in Ulwa used to indicate \isi{polarity}, whether \isi{positive} or negative. Although sentences with negative \isi{polarity} contain propositions concerning events or states that are contrary to perceived reality, they need not be marked as being \isi{irrealis} through \isi{verbal morphology}. Indeed, negative sentences may reflect the same basic three-way \isi{TAM} distinction that occurs in \isi{positive} sentences (\sectref{sec:4.2}). The marking of negative sentences can differ depending on the type of \isi{predication}: verbal (\sectref{sec:13.3.1}) or non-verbal \is{non-verbal predication} (\sectref{sec:13.3.2}).

\is{syntax|)}
\is{negation|)}


\subsection{Verbal negation}\label{sec:13.3.1}

\is{syntax|(}
\is{negation|(}
\is{verbal negation|(}

Negative \isi{declarative} sentences in Ulwa are typically readily identifiable by the \isi{negator} word \textit{ango} ‘\textsc{neg’} (‘no, not’), which comes immediately after the subject NP (or, potentially, after other postnominal modifying elements, such as \isi{temporal adverb}s). Only when a subject NP is omitted can the \isi{negator} occur clause-initially. Examples \REF{ex:syntax:118} and \REF{ex:syntax:119} illustrate the variable ordering of \textit{ango} ‘\textsc{neg}’ with other \isi{adverb}ial-like words.

\ea%118
    \label{ex:syntax:118}
          \textit{Kolpe mï amun} \textbf{\textit{ango}} \textit{apa mayte.}\\
\gll    Kolpe  mï      amun  \textbf{ango}  apa    ma=ita-e\\
    [name]  \textsc{3sg.subj}  now  \textsc{neg}  house  \textsc{3sg.obj}=build-\textsc{ipfv}\\
\glt `Kolpe is not building the house now.’ [elicited]
\z

\ea%119
    \label{ex:syntax:119}
          \textit{Kolpe mï} \textbf{\textit{ango}} \textit{amun apa mayte.}\\
\gll    Kolpe  mï      \textbf{ango}  amun  apa    ma=ita-e\\
    [name]  \textsc{3sg.subj}  \textsc{neg}  now  house  \textsc{3sg.obj}=build-\textsc{ipfv}\\
\glt `Kolpe is not building the house now.’ [elicited]
\z

Sentences \REF{ex:syntax:120} through \REF{ex:syntax:135} provide examples of negative constructions in Ulwa, all of which use the form \textit{ango} ‘\textsc{neg}’. Many of these would be translated in \ili{English} (or many other languages) variously (e.g., with words such as ‘no one’, ‘not … anything’, ‘nothing’, etc.). Where relevant, parallel \isi{positive}-\isi{polarity} sentences are provided to illustrate contrasts.

\is{verbal negation|)}
\is{negation|)}
\is{syntax|)}

\ea%120
    \label{ex:syntax:120}
          \textit{Kwa} \textbf{\textit{ango}} \textit{nip.}\\
\gll    kwa  \textbf{ango}  ni-p\\
    one    \textsc{neg}  die-\textsc{pfv}\\
\glt `No one died.’ (cf. \textit{Kwa nip} ‘Someone died.’) [elicited]
\z

\ea%121
    \label{ex:syntax:121}
          \textit{Nï} \textbf{\textit{ango}} \textit{lïmndï kwa ala.}\\
\gll    nï    \textbf{ango}  lïmndï  kwa  ala\\
    \textsc{1sg}  \textsc{neg}  eye    one    see\\
\glt    (a) ‘I didn’t see anyone.’

    (b) ‘I saw no one.’ (cf. \textit{Nï lïmndï kwa ala} ‘I saw someone.’) [elicited]
\z

\ea%122
    \label{ex:syntax:122}
          \textit{Nji (Ø / mï / ndï)} \textbf{\textit{ango}} \textit{liyu.}\\
\gll    nji    (Ø / mï / ndï)      \textbf{ango}  li-u\\
    thing  (Ø / \textsc{3sg.subj} \textsc{/} \textsc{3pl)}  \textsc{neg}  down-put\\
\glt `Nothing fell.’ (cf. \textit{Nji kwa liyu} ‘Something fell.’) [elicited]
\z

\ea%123
    \label{ex:syntax:123}
          \textit{Nï} \textbf{\textit{ango}} \textit{lïmndï nji ala.}\\
\gll    nï    \textbf{ango}  lïmndï  nji    ala\\
    \textsc{1sg}  \textsc{neg}  eye    thing  see\\
\glt    (a) ‘I didn’t see anything.’

    (b) ‘I saw nothing.’ (cf. \textit{Nï lïmndï nji kwa ala} ‘I saw something.’) [elicited]
\z

\ea%124
    \label{ex:syntax:124}
          \textit{Nï} \textbf{\textit{ango}} \textit{lïmndï minul kwa ala.}\\
\gll    nï    \textbf{ango}  lïmndï  min=ul    kwa  ala\\
    \textsc{1sg}  \textsc{neg}  eye    \textsc{3du=}with  one    see\\
\glt `I didn’t see either [of them].’ (Literally ‘I did not see one with [i.e., of] the two.’; cf. \textit{Nï lïmndï minala} ‘I saw both.’) [elicited]
\z

\ea%125
    \label{ex:syntax:125}
          \textit{Nï} \textbf{\textit{ango}} \textit{lïmndï nungol minul kwa ala.}\\
\gll    nï    \textbf{ango}  lïmndï  nungol  min=ul    kwa  ala\\
    \textsc{1sg}  \textsc{neg}  eye    child  \textsc{3du=}with  one    see\\
\glt `I didn’t see either child.’ (Literally ‘I did not see one with [i.e., of] the two children.’) [elicited]
\z

\ea%126
    \label{ex:syntax:126}
          \textit{Nï} \textbf{\textit{ango}} \textit{lïmndï minala.}\\
\gll    nï    \textbf{ango}  lïmndï  min=ala\\
    \textsc{1sg}  \textsc{neg}  eye    \textsc{3du=}see\\
\glt `I saw neither [of them].’ (Literally ‘I did not see the two.’) [elicited]
\z

\ea%127
    \label{ex:syntax:127}
          \textit{Nï} \textbf{\textit{ango}} \textit{lïmndï ankam minala.}\\
\gll    nï    \textbf{ango}  lïmndï  ankam  min=ala\\
    \textsc{1sg}  \textsc{neg}  eye    person  \textsc{3du=}see\\
\glt `I saw neither person.’ (Literally ‘I did not see the two people.’) [elicited]
\z

\ea%128
    \label{ex:syntax:128}
          \textit{Nï} \textbf{\textit{ango}} \textit{lïmndï mïnda ndul kwa ala.}\\
\gll    nï    \textbf{ango}  lïmndï  mïnda  ndï=ul    kwa  ala\\
    \textsc{1sg}  \textsc{neg}  eye    banana  \textsc{3pl=}with  one    see\\
\glt `I saw none of the bananas.’ (Literally ‘I did not see one with [i.e., of] the [more than two] bananas.’) [elicited]
\z

\ea%129
    \label{ex:syntax:129}
          \textit{Anul kwa} \textbf{\textit{ango}} \textit{wandam i.}\\
\gll    an=ul        kwa  \textbf{ango}  wandam  i\\
    \textsc{1pl.excl=}with  one    \textsc{neg}  jungle    go.\textsc{pfv}\\
\glt `None of us went to the jungle.’ (Literally ‘With [i.e., among] us, one did not go to the   jungle.’) [elicited]
\z

\ea%130
    \label{ex:syntax:130}
\is{verbal negation}
\is{negation}
\is{syntax}
          \textit{Ndul kwa} \textbf{\textit{ango}} \textit{wombïn ne.}\\
\gll    ndï=ul    kwa  \textbf{ango}  wombïn=n  ni-e\\
    \textsc{3pl=}with  one    \textsc{neg}  work=\textsc{obl}  act-\textsc{ipfv}\\
\glt `None of them is working.’ (Literally ‘With [i.e., among] them, one is not working.’) [elicited]
\z

\ea%131
    \label{ex:syntax:131}
          \textit{Ndïnji kwa} \textbf{\textit{ango}} \textit{nipe.}\\
\gll    ndï-nji    kwa  \textbf{ango}  ni-p-e\\
    3\textsc{pl-poss}  one    \textsc{neg}  die-\textsc{pfv-dep}\\
\glt `Not one of them died.’ (Literally ‘Their one did not die.’) [ulwa002\_06:42]
\z

\ea%132
    \label{ex:syntax:132}
          \textit{Ndïnji kwa} \textbf{\textit{ango}} \textit{tïnanga wolka tïklika i.}\\
\gll    ndï-nji    kwa  \textbf{ango}  tïnanga  wolka  tïkli-ka  i\\
    3\textsc{pl-poss}  one    \textsc{neg}  arise  again  turn-let  go.\textsc{pfv}\\
\glt `Not one of their [men] got up and came back again.’ (Literally ‘Their one did not arise again and go back.’) [ulwa004\_00:59]
\z

\ea%133
    \label{ex:syntax:133}
          \textit{Mawna mï keka} \textbf{\textit{ango}} \textit{mïnkïn amap.}\\
\gll    Mawna  mï      keka      \textbf{ango}  mïnkïn    ama-p\\
    [name]    \textsc{3sg.subj}  completely  \textsc{neg}  grub.species  eat-\textsc{pfv}\\
\glt `Mawna has never eaten sago grubs.’ (Literally ‘Mawna has completely not eaten sago grubs.’) [elicited]
\z

\ea%134
    \label{ex:syntax:134}
          \textit{Nï keka} \textbf{\textit{ango}} \textit{ya ame.}\\
\gll    nï    keka      \textbf{ango}  ya      ama-e\\
    \textsc{1sg}  completely  \textsc{neg}  coconut  eat-\textsc{ipfv}\\
\glt `I never eat coconut.’ (Literally ‘I completely do not eat coconut.’) [elicited]
\z

\ea%135
    \label{ex:syntax:135}
\is{verbal negation}
\is{negation}
\is{syntax}
          \textit{Mawna mï} \textbf{\textit{ango}} \textit{nunu ika mïnda ame.}\\
\gll    Mawna  mï      \textbf{ango}  nunu  ika      mïnda  ama-e\\
    [name]    \textsc{3sg.subj}  \textsc{neg}  every  instance  banana  eat-\textsc{ipfv}\\
\glt `Mawna sometimes/rarely eats bananas.’ (Literally ‘Mawna does not always eat bananas.’; cf. \textit{Mawna mï nunu ika mïnda ame} ‘Mawna always/often eats bananas.’) [elicited]
\z

\subsection{Non-verbal negation}\label{sec:13.3.2}

\is{non-verbal negation|(}
\is{syntax|(}
\is{negation|(}

Constructions that negate \isi{non-verbal predicate}s occasionally work the same as those that negate verbal \isi{predicate}s -- that is, \isi{non-verbal negation} may be expressed simply by means of the \isi{negator} \textit{ango} ‘\textsc{neg}’ (see \sectref{sec:10.2} for \isi{non-verbal predication}). Both \is{classificatory predication} classificatory \REF{ex:syntax:136} and \is{possessive predication} possessive \REF{ex:syntax:137} \isi{predication} can be expressed with \isi{zero copula}. In both of these examples, \isi{negation} is accomplished by means of \textit{ango} ‘\textsc{neg}’ alone.

\ea%136
    \label{ex:syntax:136}
          \textit{Kolpe mï} \textbf{\textit{ango}} \textit{yana.}\\
\gll    Kolpe  mï      \textbf{ango}  yana\\
    [name]  \textsc{3sg.subj}  \textsc{neg}  woman\\
\glt `Kolpe is not a woman.’ [elicited]
\z

\ea%137
    \label{ex:syntax:137}
          \textit{Nambi} \textbf{\textit{ango}} \textit{wandam ambi.}\\
\gll    nï-ambi  \textbf{ango}  wandam  ambi\\
    1\textsc{sg-top}  \textsc{neg}  jungle    big\\
\glt `As for me, I don’t have a big garden.’ [ulwa037\_50:05]
\z

Both \is{attributive predication} attributive \REF{ex:syntax:138} and \is{classificatory predication} classificatory \REF{ex:syntax:139} \isi{predication} can, alternatively, be expressed with the \isi{copular enclitic} \textit{=p} ‘\textsc{cop}’. Here, too, \isi{negation} is marked by \textit{ango} ‘\textsc{neg’}.

\ea%138
    \label{ex:syntax:138}
          \textbf{\textit{Ango}} \textit{anmap.}\\
\gll    \textbf{ango}  anma=p\\
    \textsc{neg}  good=\textsc{cop}\\
\glt `[It] is not good.’ [ulwa001\_09:18]
\z

\ea%139
    \label{ex:syntax:139}
          \textit{U} \textbf{\textit{Ango}} \textit{ulum ulwape.}\\
\gll    u    \textbf{ango}  ulum  ulwa=p-e\\
    2\textsc{sg}  \textsc{neg}  palm  nothing=\textsc{cop-dep}\\
\glt `You had no lack of sago palms.’\footnote{This sentence offers a nice example of \isi{litotes}.} [ulwa014\_50:02]
\z

More commonly, however, non-verbal \isi{negation} is accomplished by means of a \isi{clause-final negator}, either \textit{me} ‘\textsc{neg’} or \textit{kom} ‘\textsc{neg’} (sometimes \textit{kome} ‘\textsc{neg}’, perhaps reflecting the \isi{dependent marker} \textit{-e} ‘\textsc{dep}’). Typically, the \isi{clause-final negator} occurs in conjunction with the general \isi{negator} \textit{ango} ‘\textsc{neg’} in its typical post-subject clause position. Thus non-verbal \isi{negation} in Ulwa is generally accomplished by means of a \isi{discontinuous} structure (cf. \ili{French} \textit{ne … pas}). The \isi{discontinuous} structure \textit{ango … me} ‘\textsc{neg}’ can be used to negate different kinds of \is{non-verbal predication} non-verbal \isi{predicate}s: \is{equative predication} equative \REF{ex:syntax:140}, \is{attributive predication} attributive \REF{ex:syntax:141}, \is{identificational predication} identificational \REF{ex:syntax:142}, \is{existential predication} existential \REF{ex:syntax:143}, and \is{classificatory predication} classificatory \REF{ex:syntax:144}; as well as different kinds of \isi{possessive predication}, as in \REF{ex:syntax:145} and \REF{ex:syntax:146}.

\ea%140
    \label{ex:syntax:140}
          \textit{Nï} \textbf{\textit{ango}} \textit{unji itom} \textbf{\textit{me}}.\\
\gll nï    \textbf{ango}  un-nji    itom  \textbf{me}\\
    1\textsc{sg}  \textsc{neg}  \textsc{2pl-poss}  father   \textsc{neg}\\
\glt    ‘I am not your father.’ [ulwa009\_02:55]
\z

\ea%141
    \label{ex:syntax:141}
          \textit{Way} \textbf{\textit{ango}} \textit{ambi} \textbf{\textit{me}}.\\
\gll way  \textbf{ango}  ambi  \textbf{me}\\
    turtle  \textsc{neg}  big  \textsc{neg}\\
\glt `The turtle wasn’t big.’ [ulwa006\_00:02]
\z

\ea%142
    \label{ex:syntax:142}
          \textbf{\textit{Ango}} \textit{Taw} \textbf{\textit{me}}.\\
\gll \textbf{ango}  Taw  \textbf{me}\\
    \textsc{neg}  [place]  \textsc{neg}\\
\glt `[It] is not Taw.’ [ulwa014\_25:06]
\z

\ea%143
    \label{ex:syntax:143}
          \textit{Ipka} \textbf{\textit{ango}} \textit{wambana ambi} \textbf{\textit{me}}.\\
\gll ipka  \textbf{ango}  wambana  ambi  \textbf{me}\\
    before  \textsc{neg}  fish    big    \textsc{neg}\\
\glt `Before, there weren’t any big fish.’ [ulwa014\_69:19]
\z

\newpage

\ea%144
    \label{ex:syntax:144}
          \textbf{\textit{Ango}} \textit{nu luwa} \textbf{\textit{me}}.\\
\gll \textbf{ango}  nu    luwa  \textbf{me}\\
    \textsc{neg}  close  place  \textsc{neg}\\
\glt `[It] was not a close place.’ [ulwa031\_04:53]
\z

\ea%145
    \label{ex:syntax:145}
          \textit{Unanambi} \textbf{\textit{ango}} \textit{unanji amba} \textbf{\textit{me}}.\\
\gll unan-ambi    \textbf{ango}  unan-nji      amba      \textbf{me}\\
    1\textsc{pl.incl-top}  \textsc{neg}  \textsc{1pl.incl-poss}    mens.house  \textsc{neg}\\
\glt `As for us, we don’t have any magic.’ [ulwa037\_21:09]
\z

\ea%146
    \label{ex:syntax:146}
          \textbf{\textit{Ango}} \textit{unji} \textbf{\textit{me}}.\\
\gll \textbf{ango}  u-nji    \textbf{me}\\
    \textsc{neg}  2\textsc{sg-poss}  \textsc{neg}\\
\glt `[They] are not yours.’ [ulwa037\_42:38]
\z

Sentence \REF{ex:syntax:147} offers an example of the \isi{discontinuous negator} \textit{ango … me} ‘\textsc{neg}’ being used with a \isi{nominalized verb phrase}.

\ea%147
    \label{ex:syntax:147}
          \textit{Mï} \textbf{\textit{ango}} \textit{nan nïkapen} \textbf{\textit{me}}.\\
\gll mï      \textbf{ango}  na=n    nï=kï-p-en        \textbf{me}\\
    3\textsc{sg.subj}  \textsc{neg}  talk=\textsc{obl}  \textsc{1sg}=say-\textsc{pfv-nmlz}  \textsc{neg}\\
\glt `She didn’t reply to me.’ (Literally ‘She was not a having-spoken-to-me [person].’) [ulwa032\_21:36]
\z

It is interesting to note that there are also examples of clauses with \isi{nominalized verb phrase}s in which \textit{ango} ‘\textsc{neg}’ is used without any \isi{clause-final negator}. The distinction between the presence and absence of such \isi{clause-final negator} could reflect a possible difference in \isi{scope} for the \isi{negator} \textit{ango} ‘\textsc{neg’}: the presence of the \isi{clause-final negator} \textit{me} ‘\textsc{neg’} (or \textit{kom} ‘\textsc{neg’}) would thus suggest that the non-verbal \isi{predicate} (resulting from a \is{deverbalization} deverbalized verb) is being negated; and the absence of the \isi{clause-final negator} would suggest that the verb itself has been negated (before being \isi{deverbalized}). For example, \REF{ex:syntax:148} contains \textit{me} \textsc{‘neg’}, whereas \REF{ex:syntax:149} lacks it.

\ea%148
    \label{ex:syntax:148}
          \textit{Nambi \textbf{ango} alanji wandam \textbf{unden} \textbf{me}.}\\
\gll nï-ambi  \textbf{ango}  ala-nji      wandam  unda{}-\textbf{en}  \textbf{me}\\
    1\textsc{sg-top}  \textsc{neg}  \textsc{pl.dist-poss}  jungle    go-\textsc{nmlz}  \textsc{neg}\\
\glt `As for me, I’m not one to go around in other people’s jungles.’ [ulwa037\_41:09]
\z

\ea%149
    \label{ex:syntax:149}
          \textit{Nambi \textbf{ango} ndiya \textbf{mawnden}.}\\
\gll nï-ambi  \textbf{ango}  ndï=iya    ma=unda{}-\textbf{en}\\
    1\textsc{sg-top}  \textsc{neg}  3\textsc{pl}=toward  3\textsc{sg.obj}=go-\textsc{nmlz}\\
\glt `As for me, I don’t go around to them there.’ (Literally ‘As for me, I am not a to-them-there goer.’) [ulwa037\_63:56]
\z

Perhaps \REF{ex:syntax:148} could thus be translated as something like ‘I am not a going-around-in-other-people’s-jungles person’, whereas \REF{ex:syntax:149} could be translated as something like ‘I am a not-going-around-to-them-there person’. \isi{Negative scope} is discussed further in \sectref{sec:13.3.4}. Alternatively, it is possible that the absence of \isi{clause-final negator}s with \isi{nominalized verb phrase}s simply reflects a more general optionality of such marking.

\is{scope}

  Sentence \REF{ex:syntax:150} offers an example of the \isi{discontinuous negator} \textit{ango … me} ‘\textsc{neg}’ being used with a \isi{relative clause}.

\ea%150
    \label{ex:syntax:150}
          \textbf{\textit{Ango}} \textit{kambe nji} \textbf{\textit{me}}.\\
\gll \textbf{ango}  [kamb-e]  nji    \textbf{me}\\
    \textsc{neg}  [shun-\textsc{dep]}  thing  \textsc{neg}\\
\glt `[It] wasn’t something that [they] neglected.’ [ulwa037\_44:45]
\z

The other clause-final \isi{non-verbal negator} is \textit{kom} ‘\textsc{neg’}. It, too, occurs in \isi{discontinuous} constructions with the \isi{negator} \textit{ango} ‘\textsc{neg’}. Its use is illustrated in examples \REF{ex:syntax:151} through \REF{ex:syntax:154}.

\is{negation|)}
\is{syntax|)}
\is{non-verbal negation|)}

\is{non-verbal negation|(}
\is{syntax|(}
\is{negation|(}

\ea%151
    \label{ex:syntax:151}
          \textbf{\textit{Ango}} \textit{wala luwa} \textbf{\textit{kom}}.\\
\gll \textbf{ango}  wala  luwa  \textbf{kom}\\
    \textsc{neg}  far.off  place  \textsc{neg}\\
\glt `[It] is not a far-off place.’ [ulwa001\_18:22]
\z

\ea%152
    \label{ex:syntax:152}
          \textit{Unan} \textbf{\textit{ango}} \textit{wa ambi} \textbf{\textit{kom}}.\\
\gll unan    \textbf{ango}  wa    ambi  \textbf{kom}\\
    1\textsc{pl.incl}  \textsc{neg}  village  big    \textsc{neg}\\
\glt `We are not a big village.’ [ulwa037\_24:08]
\z

\ea%153
    \label{ex:syntax:153}
          \textbf{\textit{Ango}} \textit{wutota} \textbf{\textit{kom}} \textit{mundotoma ando.}\\
\gll    \textbf{ango}  wutota  \textbf{kom}  mundotoma  anda{}=o\\
    \textsc{neg}  long  \textsc{neg}  short      \textsc{sg.dist=voc}\\
\glt `[The story] is not long; it’s a short one.’ [ulwa010\_00:00]
\z

\ea%154
    \label{ex:syntax:154}
          \textbf{\textit{Ango}} \textit{unanji amba} \textbf{\textit{kom}}.\\
\gll \textbf{ango}  unan-nji    amba      \textbf{kom}\\
    \textsc{neg}  \textsc{1pl.incl-poss}  mens.house  \textsc{neg}\\
\glt `[It] is not our magic.’ [ulwa037\_09:59]
\z

I have not identified any differences in usage or meaning between \textit{me} ‘\textsc{neg}’ and \textit{kom} ‘\textsc{neg}’.

  Sometimes when \is{non-verbal predication} non-verbal \isi{predicate}s are negated, the \isi{clause-final negator} is used alone -- that is, the only \isi{negative} element in the sentence is \textit{me} ‘\textsc{neg’} or \textit{kom} ‘\textsc{neg’}, without \textit{ango} ‘\textsc{neg’} being used at all.\footnote{This may resemble \isi{Jespersen’s Cycle} \citep{Dahl1979}, a \is{syntactic change} grammatical change whereby a preverbal \isi{negative} marker is replaced by a postverbal \isi{negative} marker via a stage of two-part \isi{negation} (again, as in \ili{French}). However, if in fact different diachronic stages are represented by the presence or absence of preverbal or postverbal \isi{negator}s, then their relative chronologies are not readily discernible. Discontinuous negation \is{discontinuous negator} is prevalent throughout the \ili{Keram} family, although the \isi{morphology} associated with it is perhaps not reconstructible. Given the rather rigid verb-final nature of these languages, the \isi{clause-final negator} may have more likely emerged via \isi{grammaticalization} of an older verb form. The particular history of this grammatical structure remains unknown. \ia{Jespersen, Otto}} Sentence \REF{ex:syntax:155} illustrates the use of \textit{me} ‘\textsc{neg}’ alone (i.e., without \textit{ango} ‘\textsc{neg’}) as a \isi{non-verbal negator}.

\ea%155
    \label{ex:syntax:155}
          \textit{Un ini} \textbf{\textit{me}}.\\
\gll un  ini    \textbf{me}\\
    2\textsc{pl}  ground  \textsc{neg}\\
\glt `[It] is not your land.’ [ulwa014\_23:19]
\z

Sentences \REF{ex:syntax:156} and \REF{ex:syntax:157} demonstrate \textit{me} ‘\textsc{neg}’ being used with the \isi{semantic}ally \isi{negative} word \textit{ulwa} ‘nothing’, perhaps exemplifying a sort of \isi{negative concord}.

\ea%156
    \label{ex:syntax:156}
          \textbf{\textit{Ulwa}} \textbf{\textit{me}}.\\
\gll \textbf{ulwa}    \textbf{me}\\
    nothing  \textsc{neg}\\
\glt `[It] is nothing.’ [ulwa032\_41:11]
\z



\ea%157
    \label{ex:syntax:157}
          \textbf{\textit{Ulwapen}} \textbf{\textit{me}} \textit{nï un ka naman.}\\
\gll    \textbf{ulwa}=p-en        \textbf{me}    nï    u=n    ka  na-ma-n\\
    nothing=\textsc{cop{}-nmlz}  \textsc{neg}  \textsc{1sg}  \textsc{2sg=obl} let  \textsc{detr-}go-\textsc{ipfv}\\
\glt `There’s nothing here, so I’m leaving you.’ [ulwa031\_01:07]
\z

It seems somewhat more common for the \isi{negator} \textit{kom} ‘\textsc{neg’} (or \textit{kome} ‘\textsc{neg’}) to be used alone -- that is, as the only \isi{negator} element in a \isi{negative} non-verbal clause. Sentences \REF{ex:syntax:158} through \REF{ex:syntax:161} illustrate the use of \textit{kom} {\textasciitilde} \textit{kome} ‘\textsc{neg}’ alone (i.e., without \textit{ango} ‘\textsc{neg’}) as a \isi{non-verbal negator}.

\ea%158
    \label{ex:syntax:158}
          \textit{Mïkï itïm} \textbf{\textit{kome}}.\\
\gll mïkï  itïm  \textbf{kome}\\
    tree.species  trash  \textsc{neg}\\
\glt `[It] is not a swamp at all.’ [ulwa014\_31:25]
\z

\ea%159
    \label{ex:syntax:159}
          \textit{Ndïnji} \textbf{\textit{kome}} \textit{ndï matïna.}\\
\gll    ndï-nji    \textbf{kome}  ndï  ma=tï-na\\
    3\textsc{pl-poss}  \textsc{neg}  \textsc{3pl}  \textsc{3sg.obj=}take-\textsc{irr}\\
\glt `But [it] isn’t theirs, so they [won’t] get it.’ [ulwa037\_23:17]
\z

\ea%160
    \label{ex:syntax:160}
          \textit{Isin wane mundu} \textbf{\textit{kom}}.\\
\gll isi=n    wana-e    mundu  \textbf{kom}\\
    soup=\textsc{obl}  cook-\textsc{dep}  food  \textsc{neg}\\
\glt `[This] is not [the kind of] food that is cooked in soup.’ [ulwa014†]
\z

\ea%161
    \label{ex:syntax:161}
          \textit{Kwe wat u iyen} \textbf{\textit{kom}}.\\
\gll kwe  wat    u    i-en      \textbf{kom}\\
    one    atop  from  go.\textsc{pfv-nmlz}  \textsc{neg}\\
\glt `It wasn’t just one who came onto [it].’ (Literally ‘One was not a having-gone onto [it] [one].’) [ulwa014\_21:43]
\z

\is{clause-final negator}
\is{discontinuous negator}

Perhaps bearing on the diachronic question of these clause-final (and discontinuous) \isi{negator}s is the fact that it appears that \textit{kome} ‘\textsc{neg’} itself may be separable into two parts, \textit{ko} ‘just’ and \textit{me} ‘\textsc{neg’}, which may surround the negated \is{non-verbal predication} non-verbal \isi{predicate}. In examples \REF{ex:syntax:162}, \REF{ex:syntax:163}, and \REF{ex:syntax:164}, the prenominal element is glossed as the \isi{modal adverb} \textit{ko} ‘just, simply’, although it is formally identical to the \isi{indefinite} marker \textit{ko=} ‘\textsc{indf’}. Its origin is unclear.

\ea%162
    \label{ex:syntax:162}
          \textit{Un} \textbf{\textit{ko}} \textit{nïnji ankam} \textbf{\textit{me}}.\\
\gll un  \textbf{ko}    nï-nji    ankam  \textbf{me}\\
    2\textsc{pl}  just    1\textsc{sg-poss}  person  \textsc{neg}\\
\glt `You are not my people.’ [ulwa032\_28:00]
\z



\ea%163
    \label{ex:syntax:163}
          \textit{Ngun} \textbf{\textit{ko}} \textit{ini anma} \textbf{\textit{me}}.\\
\gll ngun  \textbf{ko}  ini      anma  \textbf{me}\\
    2\textsc{du}  just  ground    good  \textsc{neg}\\
\glt `You two, [it] is not good land.’ [ulwa014\_12:32]
\z

\newpage

\ea%164
    \label{ex:syntax:164}
          \textit{Nguna} \textbf{\textit{ko}} \textit{ndul amba kwe in wap} \textbf{\textit{ko}} \textit{ndïkalampen} \textbf{\textit{me}}.\\
\gll ngunan    \textbf{ko}  ndï=ul    amba      kwe  in  wap  \textbf{ko}     ndï=kalam=p-en        \textbf{me}\\
    1\textsc{du.incl}  just  3\textsc{pl}=with  mens.house  one    in  be.\textsc{pst}  just  3\textsc{pl}=knowledge=\textsc{cop{}-nmlz}  \textsc{neg}\\
\glt `We have not lived with them in even one men’s house nor [do we] know about them.’ [ulwa037\_16:31]
\z

Example \REF{ex:syntax:163} illustrates the peculiar behavior of \textit{kalam} ‘knowledge, knowledgeable’, a \isi{loan} verb from \ili{Waran} that has taken on nominal/adjectival \is{adjective} features (\sectref{sec:5.4}). Here it is first \isi{verbalized} with the \isi{copular enclitic} \textit{=p} ‘\textsc{cop’}, before receiving the \isi{nominalizing} \isi{suffix} \textit{-en} ‘\textsc{nmlz’}. Finally, this \isi{nominalize}d form is negated with the \isi{non-verbal negator} \textit{me} ‘\textsc{neg’}.

  Finally, it must be noted here that there are also instances in which \textit{me} ‘\textsc{neg}’ is used alone without any apparent \isi{negative} sense. Such uses seem more common with \isi{adjective}s designating the greatness of someone or something, as in \REF{ex:syntax:165} and \REF{ex:syntax:166}.

\ea%165
    \label{ex:syntax:165}
          \textit{E an namndu nïpat} \textbf{\textit{me}}\textit{!}\\
\gll    e  an      namndu  nïpat  \textbf{me}\\
    ay  1\textsc{pl.excl}  pig      giant  \textsc{neg}\\
\glt `Ay, we [had] really giant pigs!’ [ulwa014†]
\z

\ea%166
    \label{ex:syntax:166}
          \textit{Ambi ngata nda yangle} \textbf{\textit{me}} \textit{kenmbu nïpat.}\\
\gll    ambi  ngata  anda    yangle  \textbf{me}    kenmbu  nïpat\\
    big    grand  \textsc{sg.dist}  strong  \textsc{neg}  heavy    giant\\
\glt `That big huge [child] was very strong, terribly heavy.’ [ulwa032\_19:15]
\z


Perhaps sentences such as \REF{ex:syntax:165} and \REF{ex:syntax:166} should be taken to mean, for example, ‘not [merely] giant, [but rather] really, really giant’. Alternatively, they could perhaps be \is{irony} ironical statements.

  See \sectref{sec:4.11} for the \isi{speculative} \isi{suffix} \textit{-t} ‘\textsc{spec}’, which may in origin be a postverbal \isi{negator}, although this is far from certain.


\is{negation|)}
\is{syntax|)}
\is{non-verbal negation|)}

\subsection{Prohibitions}\label{sec:13.3.3}

\is{prohibition|(}
\is{prohibitive marker|(}
\is{negative command|(}
\is{syntax|(}
\is{negation|(}

In prohibitions (i.e., \isi{negative} \isi{command}s), the regular \isi{negator} \textit{ango} ‘\textsc{neg’} is not used at all, but rather the prohibitive marker \textit{wana} {\textasciitilde} \textit{wanap} ‘\textsc{proh’} is used, as in \REF{ex:syntax:167}.

\ea%167
    \label{ex:syntax:167}
          \textbf{\textit{Wanap}} \textit{apka nïklop mana!}\\
\gll    \textbf{wanap}  apka  nï=klop  ma-na\\
    \textsc{proh}  very  1\textsc{sg}=cross  go-\textsc{irr}\\
\glt `Don’t go and bypass me completely!’ [ulwa014†]
\z

More examples of prohibitive statements can be found in the sections on \isi{negative} \isi{command}s (\sectref{sec:13.2.4}) and on the \isi{speculative} \isi{suffix} \textit{-t} ‘\textsc{spec}’ (\sectref{sec:4.11}).

\is{negation|)}
\is{syntax|)}
\is{negative command|)}
\is{prohibitive marker|)}
\is{prohibition|)}

\subsection{Negative scope}\label{sec:13.3.4}

\is{negative scope|(}
\is{syntax|(}
\is{negation|(}
\is{scope|(}

An interesting fact about Ulwa \isi{negation} concerns the \isi{scope} of the \isi{negator}. The tendency in Ulwa is to place \textit{ango} ‘\textsc{neg}’ within the first clause of \isi{multiclausal construction}s, even when the \isi{scope} of \isi{negation} is smaller than the whole series of clauses. In other words, a subsequent clause or clausal element may be negated, without any \isi{negation} implied concerning the clause in which \textit{ango} ‘\textsc{neg}’ occurs, as illustrated in examples \REF{ex:syntax:168} through \REF{ex:syntax:170}.

\ea%168
    \label{ex:syntax:168}
          \textit{An} \textbf{\textit{ango}} \textit{apa mbïlop mbïwap.}\\
\gll    an      \textbf{ango}  apa    mbï-lo-p    mbï-wap\\
    1\textsc{pl.excl}  \textsc{neg}  house  here-go-\textsc{pfv}  here-be.\textsc{pst}\\
\glt `We came home, but didn’t stay.’ (Literally ‘We did not come home and stay.’) [ulwa032\_33:21]
\z

\ea%169
    \label{ex:syntax:169}
          \textbf{\textit{Ango}} \textit{ulum ale we wandam pen.}\\
\gll    \textbf{ango}  ulum  ale-e    we    wandam  p-en\\
    \textsc{neg}  palm  scrape-\textsc{dep}  sago  jungle    be\textsc{{}-nmlz}\\
\glt `When [they] scrape sago palms, the sago starch is not [left behind] in the jungle.’ (Literally ‘[It is] not [the case that], having scraped sago palms, the sago starch is [left behind] in the jungle.’) [ulwa014\_60:03]
\z

\ea%170
    \label{ex:syntax:170}
          \textbf{\textit{Ango}} \textit{mat mïnjikan kïna: …}\\
\gll    \textbf{ango}  ma=tï      mïnjika=n    kï-na\\
    \textsc{neg}  \textsc{3sg.obj}=take  speech=\textsc{obl}  say-\textsc{irr}\\
\glt `Having gotten it, [they] wouldn’t say [the following]: …’ (Literally ‘[It is] not [the case that they] get it and would say [the following]: …’) [ulwa032\_46:52]
\z

This rather early placement of \textit{ango} ‘\textsc{neg}’ occurs in \isi{conditional} statements as well -- that is, the \isi{negator} may occur within the \isi{protasis}, even when the verbal element to be negated belongs in the \isi{apodosis}. In each of the \isi{conditional} statements in \REF{ex:syntax:171}, \REF{ex:syntax:172}, and \REF{ex:syntax:173}, \textit{ango} ‘\textsc{neg}’ occurs within the \isi{protasis}.

\ea%171
    \label{ex:syntax:171}
          \textbf{\textit{Ango}} \textit{maka apwanam mapta inim landa.}\\
\gll    \textbf{ango}  maka  apwanam    ma=p-ta      inim  la-nda\\
    \textsc{neg}  thus  side.of.house  3\textsc{sg.obj}=be\textsc{{}-cond} water  eat-\textsc{irr}\\
\glt `As long as [she] is staying at the side of the house, [a recent mother] may not drink water.’ [ulwa014\_36:12]
\z

\ea%172
    \label{ex:syntax:172}
          \textbf{\textit{Ango}} \textit{mat ita nduwe malanda.}\\
\gll    \textbf{ango}  ma=tï      i-ta        ndï-we      ma=la-nda\\
    \textsc{neg}  \textsc{3sg.obj}=take  go.\textsc{pfv-cond}  \textsc{3pl-part.int}  \textsc{3sg.obj}=eat-\textsc{irr}\\
\glt `If [he] brings it, they will not eat it alone.’ [ulwa014\_64:59]
\z

\ea%173
    \label{ex:syntax:173}
          \textbf{\textit{Ango}} \textit{amunpïta ikali masinate.}\\
\gll    \textbf{ango}  amun=p-ta      i-kali    ma=si-na-t-e\\
    \textsc{neg}  now=\textsc{cop{}-cond} hand-send  \textsc{3sg.obj}=push-\textsc{irr-spec-dep}\\
\glt `If [a baby] is still very young, [then fathers] will not hold it.’ [ulwa014\_37:45]
\z

In the two \isi{conditional} prohibitive statements given in \REF{ex:syntax:174} and \REF{ex:syntax:175}, the \isi{negative} marker \textit{wana} ‘\textsc{proh’} occurs in the \isi{protasis}, even though the \isi{negation} properly occurs in the \isi{apodosis}.

\is{scope|)}
\is{negation|)}
\is{syntax|)}
\is{negative scope|)}

\ea%174
    \label{ex:syntax:174}
          \textbf{\textit{Wana}} \textit{ambipïta wa lolop ala wandam pïta alanji nji landa!}\\
\gll    \textbf{wana}  ambi=p-ta      wa  lolop  ala      wandam  p-ta     ala-nji      nji    la-nda\\
    \textsc{proh}  big=\textsc{cop{}-cond} just  just    \textsc{pl.dist}  jungle    be\textsc{{}-cond}    \textsc{pl.dist-poss}  thing  eat-\textsc{irr}\\
\glt `When [you] are grown and are just [going around] in other people’s gardens, don’t eat   their things!’ [ulwa032\_40:38]
\z

\ea%175
    \label{ex:syntax:175}
          \textit{A un} \textbf{\textit{wana}} \textit{apa mapta luke natana!}\\
\gll    a    un  \textbf{wana}  apa    ma=p-ta      luke  na-ta-na\\
    \textsc{interj}  \textsc{2pl}  \textsc{proh}  house  3\textsc{sg.obj}=be\textsc{{}-cond} too    \textsc{detr-}say-\textsc{irr}\\
\glt `Hey, if you’re in the house, don’t talk either!’ [ulwa032\_42:57]
\z

\subsection{Negative responses}\label{sec:13.3.5}

\is{negative response|(}
\is{syntax|(}
\is{negation|(}

It is relatively uncommon to answer ‘yes’ or ‘no’ to \isi{question}s in Ulwa: rather, interlocutors tend to respond with full answers or \isi{paralinguistic gesture}s (such as head movements) or \isi{interjection}s (such as \textit{m} ‘hm!’). It is nevertheless possible to use the word \textit{ase} ‘no’ (sometimes realized as [asa]), whether as a \isi{response} to a \isi{question}, or as a simple denial (without any \isi{question} necessarily having been posed). Sentences \REF{ex:syntax:176} through \REF{ex:syntax:179} provide examples of its use.

\ea%176
    \label{ex:syntax:176}
          \textit{Ndï man nan nït} \textbf{\textit{ase}}.\\
\gll ndï  ma=n      na=n    nï=ta    \textbf{ase}\\
    3\textsc{pl}  3\textsc{sg.obj=obl}  talk=\textsc{obl}  \textsc{1sg=}say  no\\
\glt `They told me “no”.’ [ulwa014\_20:39]
\z

\ea%177
    \label{ex:syntax:177}
          \textbf{\textit{Ase}} \textit{unan tïngïnpe.}\\
\gll    \textbf{ase}  unan    tïngïn=p-e\\
    no  1\textsc{pl.incl}  many=\textsc{cop{}-dep}\\
\glt `No, we are many [now].’ [ulwa014\_64:27]
\z

\ea%178
    \label{ex:syntax:178}
          \textit{Nï ango wa mbïpta ul wombïn ninda.} \textbf{\textit{Ase}} \textit{nï umbe un ka wandam namana.}\\
\gll nï    ango  wa    mbï-p-ta    u=ul    wombïn=n  ni-nda     \textbf{ase}  nï    umbe    u=n    ka  wandam  na-ma-na\\
    1\textsc{sg}  \textsc{neg}  village  here-be\textsc{{}-cond}  \textsc{2sg=}with  work=\textsc{obl}  act-\textsc{irr}    no  1\textsc{sg}  tomorrow  2\textsc{sg=obl}  let  jungle    \textsc{detr-}go-\textsc{irr}\\
\glt `I won’t stay in the village and work with you. No, tomorrow I’ll leave you and go to the jungle.’ [ulwa031\_00:12]
\z

\ea%179
    \label{ex:syntax:179}
          \textbf{\textit{Asa}} \textit{mï mïnjikan ngunankap: …}\\
\gll    \textbf{asa}  mï      mïnjika=n    ngunan=kï-p\\
    no  3\textsc{sg.subj}  speech=\textsc{obl}  1\textsc{du.incl}=say-\textsc{pfv}\\
\glt `No, he said the following to us: …’ [ulwa014\_23:38]
\z

The word \textit{ulwa} ‘nothing’, or a \isi{verbalized} form \textit{ulwa=p} ‘there is nothing’, may also be used as a \isi{negative response} word, particularly when declining a \isi{request} for a physical item. This may be compared with the \ili{Tok Pisin} word \textit{nogat} ‘no’ which derives from \textit{no gat} ‘there is not [something/anything]’.

\is{negation|)}
\is{syntax|)}
\is{negative response|)}


\section{Reported speech}\label{sec:13.4}

\is{reported speech|(}
\is{speech|(}
\is{syntax|(}

In Ulwa, \isi{direct discourse} is constructed around at least two separate clauses: one containing the quoted utterance (typically the second clause) and one reporting who uttered it (typically the first clause). \isi{Direct discourse} constructions are thus of the form: ‘speaker says/said: “[what that person says/said]”’, as in \REF{ex:syntax:179a}.

\ea%179a
    \label{ex:syntax:179a}
          \textit{Nï mat ndï amun up.}\\
\gll    nï ma=ta ndï amun u-p\\
   1\textsc{sg} 3\textsc{sg.obj}=say  3\textsc{pl} now put-\textsc{pfv}\\
\glt `I said: “They’ve only now planted (them).”’ (Literally ‘I said it ...’) [ulwa014\_10:29]
\z

\isi{Direct discourse} constructions are generally formed with one of two mostly \isi{synonymous} verbs, either \textit{ta-} ‘say’ or \textit{kï-} ‘say’. The basic three-way \isi{TAM} paradigms for these verbs (as well as the \isi{imperative} and \isi{conditional} forms) are presented in \tabref{tab:syntax:13}.


\begin{table}
\caption{Two ‘saying’ verbs}
\is{stem}
\label{tab:syntax:13}
\begin{tabularx}{\textwidth}{QQQQQQl}
\lsptoprule
gloss & stem & {\scshape ipfv} & {\scshape pfv} & {\scshape irr} & {\scshape imp} & {\scshape cond}\\
\midrule
‘say’ & {\itshape ta-} & {\itshape tan {\textasciitilde} t} & {\itshape tap \textup{{\textasciitilde}} t} & {\itshape tana} & {\itshape tan} & {\itshape tata}\\
‘say’ & {\itshape kï-} & {\itshape ke} & {\itshape kap} & {\itshape kïna} & {\itshape kïn} & {\itshape kïta {\textasciitilde} kapta}\\
\lspbottomrule
\end{tabularx}
\end{table}

The verb \textit{ta-} ‘say’ often occurs in a reduced (\isi{defective}) form [t], although only when permitted by the \isi{phonotactics} of the utterance in which it occurs. The verb \textit{ka-} ‘say’, however, never occurs reduced as \textsuperscript{†}[k] (indeed, word-final [k] is illicit, \sectref{sec:2.1.1}). The \isi{conditional} form for \textit{kï-} ‘say’ may be built alternatively from the \isi{perfective} form \textit{kap} ‘say [\textsc{pfv}]’, which exhibits \isi{vowel} lowering (\sectref{sec:4.2}). There is no attested \isi{conditional} form of \textit{ta-} ‘say’ built from its \isi{perfective} form (i.e., \textsuperscript{†}[tapta] does not occur).

\is{syntax|)}
\is{speech|)}
\is{reported speech|)}

\subsection{Intransitive uses of verbs of speaking}\label{sec:13.4.1}

\is{reported speech|(}
\is{speech|(}
\is{syntax|(}
\is{intransitive|(}

In examples \REF{ex:syntax:180}, \REF{ex:syntax:181}, and \REF{ex:syntax:182}, the verb of speaking \textit{ta-} ‘say’ or \textit{kï-} ‘say’ is used \isi{intransitive}ly. The form [na], when present, is glossed as the \isi{detransitivizing} marker (\sectref{sec:13.8.2}), although the argument could be made that it is the noun \textit{na} ‘talk’ functioning as the first half of a \isi{compound verb} form. Since the verb of speaking is often \isi{transitive}, however, the analysis of [na] here as the \isi{detransitivizing} marker \textit{na-} \textsc{‘detr’} is preferred, since the form seems to be helping to \is{valency reduction} reduce the \isi{valency} of the verb.

\is{intransitive|)}
\is{syntax|)}
\is{speech|)}
\is{reported speech|)}

\ea%180
    \label{ex:syntax:180}
          \textit{Awal imba pe una} \textbf{\textit{natap}}.\\
\gll awal    imba  p-e    unan    \textbf{na-ta}{}-p\\
    yesterday  night  be\textsc{{}-dep}  \textsc{1pl.incl}  \textsc{detr-}say-\textsc{pfv}\\
\glt `Last night we talked.’ [ulwa037\_06:28]
\z

\ea%181
    \label{ex:syntax:181}
          \textit{Inom mï} \textbf{\textit{nakap}}.\\
\gll inom  mï      \textbf{na-kï-p}\\
    mother  \textsc{3sg.subj}  \textsc{detr-}say-\textsc{pfv}\\
\glt `Mother spoke.’ [elicited]
\z

\ea%182
    \label{ex:syntax:182}
\is{intransitive}
\is{syntax}
\is{speech}
\is{reported speech}
          \textit{Wiya mbi ul} \textbf{\textit{natana}} \textit{mbi.}\\
\gll    u=iya      mbï-i      u=ul    \textbf{na-ta}{}-na    mbï-i\\
    2\textsc{sg=}toward  here-go.\textsc{pfv}  2\textsc{sg}=with  \textsc{detr-}say-\textsc{irr}  here-go.\textsc{pfv}\\
\glt `[I] came to you here, came to speak with you here.’ [ulwa026\_00:35]
\z

\subsection{Transitive uses of verbs of speaking}\label{sec:13.4.2}

\is{reported speech|(}
\is{speech|(}
\is{syntax|(}
\is{transitive|(}

It is, however, much more common for the verb of speaking to be \isi{transitive}, taking as the object either the thing said or the person addressed. In examples \REF{ex:syntax:183} through \REF{ex:syntax:186}, the object of the verb is the thing said. Note that the \isi{detransitivizing} marker \textit{na-} \textsc{‘detr’} is not present in these \isi{transitive} clauses. The noun \textit{na} ‘talk’, however, may occur.

\ea%183
    \label{ex:syntax:183}
          \textit{Min na kuma} \textbf{\textit{tap}}.\\
\gll min  na    kuma  \textbf{ta}{}-p\\
    3\textsc{du}  talk  some  say-\textsc{pfv}\\
\glt `The two planned something.’ (Literally ‘The two said some talks.’) [ulwa014\_03:37]
\z

\ea%184
    \label{ex:syntax:184}
          \textit{Nï ango na tïngïn} \textbf{\textit{tana}}.\\
\gll nï    ango  na    tïngïn  \textbf{ta}{}-na\\
    1\textsc{sg}  \textsc{neg}  talk  many  say-\textsc{irr}\\
\glt `I won’t tell many stories.’ [ulwa024\_01:55]
\z

\ea%185
    \label{ex:syntax:185}
          \textit{Nï mol na} \textbf{\textit{ndïtane}}.\\
\gll nï    ma=ul      na    ndï=\textbf{ta}{}-n-e\\
    1\textsc{sg}  3\textsc{sg.obj}=with  talk  3\textsc{pl}=say-\textsc{ipfv-dep}\\
\glt `I was telling the stories with him.’ [ulwa014\_40:12]
\z

\ea%186
    \label{ex:syntax:186}
          \textit{Ndï ndïnap atwana} \textbf{\textit{kïna}}.\\
\gll ndï  ndï=nap  atwana    \textbf{kï}{}-na\\
    3\textsc{pl}  3\textsc{pl=}for  question  say-\textsc{irr}\\
\glt `They will ask about them.’ [ulwa032\_52:41]
\z

Example \REF{ex:syntax:186} illustrates how reported \isi{question}s can be expressed: namely, one uses the \isi{verb phrase} ‘say a question’, where ‘question’ is the object of the verb of speaking.

  Often, the verb of speaking takes as an object the thing said, without there being much \isi{semantic} value to this object. That is, the object (always a bare \textsc{3sg} \isi{object-marker} \isi{clitic}) functions as an \isi{expletive}, as in examples \REF{ex:syntax:187} through \REF{ex:syntax:191}.

\ea%187
    \label{ex:syntax:187}
          \textit{Nï} \textbf{\textit{mat}} \textit{a!}\\
\gll    nï    \textbf{ma=ta}      a\\
    1\textsc{sg}  \textsc{3sg.obj=}say  ah\\
\glt `I said, “ah!”’ (Literally ‘I said it: “Ah!”’) [ulwa014\_70:46]
\z

\ea%188
    \label{ex:syntax:188}
          \textit{Mï} \textbf{\textit{mate}} \textit{ankam alanji ala!}\\
\gll    mï      \textbf{ma=ta}{}-e      ankam  ala-nji      ala\\
    3\textsc{sg.subj}  \textsc{3sg.obj}=say-\textsc{dep}  person  \textsc{pl.dist-poss}  \textsc{pl.dist}\\
\glt `He said it [that he would kill their pigs], but those are other people’s [pigs]!’ [ulwa014\_27:04]
\z

\ea%189
    \label{ex:syntax:189}
          \textit{Mï ambïwana} \textbf{\textit{mat}} \textit{a nï ta tata tïn mol li ina mane.}\\
\gll    mï      ambï=wana  \textbf{ma=ta}      a  nï    ta    tata     tï-n      ma=ul      li    i-na    ma-n-e\\
    3\textsc{sg.subj}  \textsc{sg.refl=}feel  3\textsc{sg.obj=}say  ah  1\textsc{sg}  already  papa    take-\textsc{pfv}  3\textsc{sg.obj}=with  down  come-\textsc{irr}  go-\textsc{ipfv-dep}\\
\glt `It thought to itself and said: “Ah! I’m already able to get papa and come down with him.”’ [ulwa006\_02:22]
\z

\ea%190
    \label{ex:syntax:190}
          \textbf{\textit{Makape}} \textit{mï i.}\\
\gll    \textbf{ma=kï-p}{}-e        mï      i\\
    3\textsc{sg.obj}=say-\textsc{pfv-dep}  3\textsc{sg.subj}  go.\textsc{pfv}\\
\glt `Having spoken, he went.’ [ulwa039\_00:28]
\z

\ea%191
    \label{ex:syntax:191}
          \textit{Nï lïmndï minlïpe} \textbf{\textit{mat}}\textit{: …}\\
\gll    nï    lïmndï  min=lï-p-e      \textbf{ma=ta}\\
    1\textsc{sg}  eye    3\textsc{du}=put-\textsc{pfv-dep}  \textsc{3sg.obj}=say\\
\glt `I saw the two of them and said: …’ [ulwa014\_42:15]
\z

As illustrated by examples such as \REF{ex:syntax:191}, among others, it is common for verbs of speaking (or at least the verb \textit{ta-} ‘say’) to be \isi{defective} -- that is, the \isi{verb stem} is often left unmarked for \isi{TAM} (and loses its \isi{stem}-final \isi{vowel}), thus being pronounced as just [t]. This occurs especially in situations in which the object of \isi{speech} is an \isi{expletive} (or \isi{dummy}) object.

  The role of the object of the \isi{transitive} verb of speaking, however, need not be the thing spoken, but may instead be the person addressed, as in \REF{ex:syntax:192}.

  \ea%192
    \label{ex:syntax:192}
          \textit{Nï} \textbf{\textit{nan mat}} \textit{nï ango makïke lunda.}\\
\gll    nï    \textbf{na=n}    ma=\textbf{ta}      nï    ango  ma=kïke     lo-nda\\
    1\textsc{sg}  talk=\textsc{obl}  3\textsc{sg.obj}=say  \textsc{1sg}  \textsc{neg}  \textsc{3sg.obj=}throw     go-\textsc{irr}\\
\glt `I told her: “I won’t sell it.”’ [ulwa042\_01:46]
\z
   
  In such constructions, the word \textit{na} ‘talk’ is often present before the \isi{object marker}. This word may or may not be followed by the \isi{oblique marker} \textit{=n} ‘\textsc{obl}’. When this marker is present, then the construction is analyzed as a clause that consists of a \isi{transitive} verb taking the person addressed as its \isi{direct object} and an \isi{oblique} \isi{phrase} composed of the word \textit{na} ‘talk’ plus the \isi{oblique marker} (i.e., literally ‘tell [someone] with/by means of speech/talk’). In examples \REF{ex:syntax:193} through \REF{ex:syntax:199}, the verbs \textit{ta-} ‘say’ and \textit{kï-} ‘say’ are used \isi{transitive}ly, taking as an object the person addressed. The \isi{verb phrase} follows the \isi{oblique} \isi{phrase} \textit{na=n} ‘with talk’.

\ea%193
    \label{ex:syntax:193}
          \textit{Nï \textbf{nan ndït} nga unji.}\\
\gll    nï    \textbf{na=n}    ndï=\textbf{ta}    nga      un-nji\\
    1\textsc{sg}  talk=\textsc{obl}  3\textsc{pl}=say  \textsc{sg.prox}  \textsc{2pl-poss}\\
\glt `I said to them: “This is yours.”’ [ulwa014\_06:44]
\z

\ea%194
    \label{ex:syntax:194}
          \textit{Nïnji yanat mï} \textbf{\textit{nan}} \textbf{\textit{ndïkap}}\textit{: …}\\
\gll    nï-nji    yanat    mï      \textbf{na=n}    ndï=\textbf{kï-p}\\
    1\textsc{sg-poss}  daughter  \textsc{3sg.subj}  talk=\textsc{obl}  \textsc{3pl=}say-\textsc{pfv}\\
\glt `My daughter told them: …’ [ulwa032\_22:49]
\z

\ea%195
    \label{ex:syntax:195}
          \textit{Nï ine Tarambi \textbf{nan nït}: …}\\
\gll    nï    i-n-e        Tarambi  \textbf{na=n}    nï=\textbf{ta}\\
    \textsc{1sg}  come-\textsc{pfv-dep}    [name]    talk=\textsc{obl}  1\textsc{sg}=say\\
\glt `When I came, Tarambi told me: …’ [ulwa014†]
\z

\ea%196
    \label{ex:syntax:196}
          \textit{An \textbf{nan amblakap}: …}\\
\gll    an      \textbf{na=n}    ambla=\textbf{kï-p}\\
    1\textsc{pl.excl}  talk=\textsc{obl}  \textsc{pl.refl}=say-\textsc{pfv}\\
\glt `We said to each other: …’ [ulwa032\_24:57]
\z

\ea%197
    \label{ex:syntax:197}
          \textit{Nï angos \textbf{nan ukïn}?}\\
\gll nï    angos  \textbf{na=n}    u=\textbf{kï}{}-na\\
    1\textsc{sg}  what  talk=\textsc{obl}  2\textsc{sg}=say-\textsc{irr}\\
\glt `What should I tell you?’ [ulwa037\_16:20]
\z

\ea%198
    \label{ex:syntax:198}
          \textbf{\textit{Nan}} \textit{nungolke} \textbf{\textit{ngalakapta}} \textit{ndï kalampïn!}\\
\gll    \textbf{na=n}    nungolke  ngala=\textbf{kï-p}{}-ta        ndï     kalam=p-na\\
    talk=\textsc{obl}  child    \textsc{pl.prox}=say-\textsc{pfv-cond}  \textsc{3pl}    knowledge=\textsc{cop}{}-\textsc{irr}\\
\glt `Tell these children so that they’ll know!’ (Literally ‘If [you] tell these children with speech, they will know.’) [ulwa006\_07:23]
\z

\ea%199
    \label{ex:syntax:199}
          \textit{Mï \textbf{nan minte} ngun naman!}\\
\gll mï      \textbf{na=n}    min=\textbf{ta}{}-e    ngun  na-ma-n\\
    3\textsc{sg.subj}  talk=\textsc{obl}  3\textsc{du}=say-\textsc{dep}  \textsc{2du}  \textsc{detr-}go-\textsc{imp}\\
\glt `He told the two of them: “Go!”’ [ulwa001\_09:45]
\z

Dependent marking \is{dependent marker} is not necessary in such \isi{reported speech} constructions, although it may be possible, as in \REF{ex:syntax:199}.

  Since the word \textit{na} ‘talk’ functions as the \isi{head noun} of its own \isi{oblique} \isi{phrase} in these constructions, it is possible for it to be modified by an \isi{adjective} \REF{ex:syntax:200}.

\is{transitive|)}
\is{syntax|)}
\is{speech|)}
\is{reported speech|)}

\is{reported speech|(}
\is{speech|(}
\is{syntax|(}
\is{transitive|(}

\ea%200
    \label{ex:syntax:200}
          \textit{Ndï \textbf{na ilumnï} ukïnat.}\\
\gll    ndï  \textbf{na}    \textbf{ilum=nï}  u=kï-na-t\\
    3\textsc{pl}  talk  little=\textsc{obl}  \textsc{2sg}=say-\textsc{irr-spec}\\
\glt `They might tell you a little story.’ (i.e., ‘They might try to deceive you.’) [ulwa014\_71:00]
\z

  When the \isi{oblique marker} is absent, on the other hand, then the construction is analyzed as a \is{compound verb} \isi{compound} \isi{verb phrase} in which the nominal element occurs before the object (that is, they are \isi{separable verb} constructions; see \sectref{sec:9.2.1}).\footnote{In certain situations, however, it is impossible to tell whether the form [na] contains the \isi{enclitic} \textit{=n} ‘\textsc{obl}’ or not: if [na] is followed by a word that begins with /n-/ or /nd-/, then the sequence /nn/ -- if ever present -- would \isi{degeminate} to [n].} In examples \REF{ex:syntax:201} through \REF{ex:syntax:206}, the verbs \textit{ta-} ‘say’ and \textit{kï-} ‘say’ are used \isi{transitive}ly, taking as an object the person addressed. Although the word \textit{na} ‘talk’ is present, it does not take the \isi{oblique marker} \textit{=n} `\textsc{obl}’. Accordingly, these sentences are interpreted as containing \isi{compound} \isi{verb phrase}s, in which the nominal component is separate from the \isi{verb stem}, occurring before the object.

\ea%201
    \label{ex:syntax:201}
          \textit{Nï} \textbf{\textit{na}} \textbf{\textit{makïna}} \textit{ase.}\\
\gll    nï    \textbf{na}    ma=\textbf{kï}{}-na  ase\\
    1\textsc{sg}  talk  \textsc{3sg.obj}=say-\textsc{irr}  no\\
\glt `I will tell him “no”.’ [ulwa031\_00:23]
\z

\ea%202
    \label{ex:syntax:202}
          \textbf{\textit{Na}} \textit{Joanna} \textbf{\textit{kap}} \textit{inom ngol man!}\\
\gll    na    Joanna    \textbf{kï-p}    inom  nga=ul      ma-n\\
    talk  [name]    say-\textsc{pfv}  mother  \textsc{sg.prox}=with  go-\textsc{imp}\\
\glt `[I] told Joanna: “Go with this woman!”’ [ulwa037\_60:45]
\z

\ea%203
    \label{ex:syntax:203}
          \textit{Yanat mï} \textbf{\textit{na}} \textbf{\textit{makap}}\textit{: …}\\
\gll    yanat    mï      \textbf{na}    ma=\textbf{kï-p}\\
    daughter  \textsc{3sg.subj}  talk  \textsc{3sg.obj}=say-\textsc{pfv}\\
\glt `[My] daughter told her: …’ [ulwa032\_28:41]
\z


\ea%204
    \label{ex:syntax:204}
          \textit{Nï} \textbf{\textit{na}} \textbf{\textit{mate}} \textit{mï li.}\\
\gll    nï    \textbf{na}    ma=\textbf{ta}{}-e      mï      li    i\\
    \textsc{1sg}  talk  \textsc{3sg.obj}=say-\textsc{dep}  \textsc{3sg.subj}  down  go.\textsc{pfv}\\
\glt `I told her and she went down.’ [ulwa037\_04:50]
\z

\ea%205
    \label{ex:syntax:205}
          \textit{Awal} \textbf{\textit{na}} \textit{yenanu ambi} \textbf{\textit{ndate}}.\\
\gll awal    \textbf{na}    yenanu  ambi  anda=\textbf{ta}{}-e\\
    yesterday  talk  woman  big    \textsc{sg.dist}=say-\textsc{dep}\\
\glt `I told that big woman yesterday.’ (i.e., ‘my older sister’) [ulwa037\_49:35]
\z

\ea%206
    \label{ex:syntax:206}
          \textit{Nï ango angos} \textbf{\textit{na}} \textbf{\textit{ukïnate}}.\\
\gll nï    ango  angos  \textbf{na}    u=\textbf{kï}{}-na-t-e\\
    \textsc{1sg}  \textsc{neg}  what  talk  2\textsc{sg}=say-\textsc{irr-spec-dep}\\
\glt `I don’t have anything to tell you.’ [ulwa006\_05:43]
\z

In another version of \isi{transitive} clauses based on the verbs \textit{ta-} ‘say’ and \textit{kï-} ‘say’, an \isi{expletive} pronominal \isi{clitic} \textit{ma=} ‘\textsc{3sg.obj}’ is used in place of \textit{na} ‘talk’ and receives the \isi{oblique marker} \textit{=n} ‘\textsc{obl}’. The literal meaning of these constructions could be rendered as ‘tell [someone] with it’ (with ‘speech’ understood as the \isi{antecedent} of ‘it’). This construction is illustrated by examples \REF{ex:syntax:207} through \REF{ex:syntax:211}.

\ea%207
    \label{ex:syntax:207}
          \textit{Nï} \textbf{\textit{man}} \textbf{\textit{ngunte}}\textit{: …}\\
\gll    nï    \textbf{ma=n}      ngun=\textbf{ta}{}-e\\
    1\textsc{sg}  3\textsc{sg.obj=obl}  \textsc{2du}=say-\textsc{dep}\\
\glt `I told you two: …’ [ulwa014\_07:48]
\z

\ea%208
    \label{ex:syntax:208}
          \textit{Nï \textbf{man mate}.}\\
\gll nï    \textbf{ma=n}      ma=\textbf{ta}{}-e\\
    1\textsc{sg}  3\textsc{sg.obj=obl}  3\textsc{sg.obj}=say-\textsc{dep}\\
\glt `I told him.’ [ulwa014\_08:05]
\z

\ea%209
    \label{ex:syntax:209}
          \textit{Tïponïm ini mï tembipe nï} \textbf{\textit{man}} \textit{Danny} \textbf{\textit{mat}}.\\
\gll Tïponïm  ini    mï      tembi=p-e    nï    \textbf{ma=n} Danny  ma=\textbf{ta}\\
    [place]    ground  \textsc{3sg.subj}  bad=\textsc{cop{}-dep}  \textsc{1sg} 3\textsc{sg.obj}=\textsc{obl}     [name]  3\textsc{sg.obj=}say\\
\glt `“The Tïponïm ground is bad,” I told Danny.’ [ulwa014\_09:07]
\z

\ea%210
    \label{ex:syntax:210}
          \textit{Mï \textbf{man mat}: …}\\
\gll    mï      \textbf{ma=n}      \textbf{ma=ta}\\
    3\textsc{sg.subj}  \textsc{3sg.obj=obl}  \textsc{3sg.obj}=say\\
\glt `He told her: …’ [ulwa001\_15:13]
\z

\ea%211
    \label{ex:syntax:211}
          \textit{Nï \textbf{man unate}: …}\\
\gll nï    \textbf{ma=n}      u=na-\textbf{ta}-e\\
    1\textsc{sg}  \textsc{3sg.obj=obl}  \textsc{2sg}=talk-say-\textsc{dep}\\
\glt `I’m telling you: …’ [ulwa014\_33:50]
\z

Example \REF{ex:syntax:211} illustrates the \isi{compound verb} form \textit{na-ta} ‘say talk’ co-occurring with the \isi{oblique}-marked \isi{expletive} construction \textit{ma=n} ‘with it’.

\is{transitive|)}
\is{syntax|)}
\is{speech|)}
\is{reported speech|)}

\subsection{Expressing the topic of speech}\label{sec:13.4.3}

\is{reported speech|(}
\is{speech|(}
\is{syntax|(}

A topic spoken about can be referred to as a \isi{phrase} consisting of the topic, \is{possession} possessive marking, and the word \textit{na} ‘talk’ (literally something like ‘X’s story’, where X can be any kind of referent: a person, thing, or concept). Various means of indicating \isi{possession} can be used in such constructions: the possessive \isi{suffix} \textit{{}-nji} ‘\textsc{poss’} \textsc{\REF{ex:syntax:212}}, the \isi{oblique} \isi{enclitic} \textit{=n} ‘\textsc{obl’} \textsc{\REF{ex:syntax:213}}, or a \isi{non-subject} pronominal \REF{ex:syntax:214} or \isi{deictic} \REF{ex:syntax:215} form.

\ea%212
    \label{ex:syntax:212}
          \textit{\textbf{Manji na} latane.}\\
\gll    \textbf{ma-nji}      \textbf{na}    ala=ta-n-e\\
    3\textsc{sg.obj-poss}  talk  \textsc{pl.dist}=say-\textsc{ipfv-dep}\\
\glt `[We] were talking about her.’ (Literally ‘were saying those talks of her’) [ulwa037\_00:19]
\z

\ea%213
    \label{ex:syntax:213}
          \textit{Nï \textbf{ini man na} tane Wore un ango wap?}\\
\gll    nï    \textbf{ini}    \textbf{ma=n}      \textbf{na}    ta-n-e      Wore  un  ango wap\\
    1\textsc{sg}  ground  3\textsc{sg.obj=obl}  talk  say-\textsc{ipfv-dep}  [place]  2\textsc{pl}  which        be.\textsc{pst}\\
\glt `When I was talking about the land, where were you, [people from] Wore?’ [ulwa014\_22:43]
\z

\ea%214
    \label{ex:syntax:214}
          \textit{Nï amun maka \textbf{lamndu wonmbi ma na} tana manen.}\\
\gll    nï    amun  maka  \textbf{lamndu}   \textbf{wonmbi}  \textbf{ma}      \textbf{na}     ta-na    ma-n-en\\
    1\textsc{sg}  now  thus  pig      tusk    3\textsc{sg.obj}  talk    say-\textsc{irr}  go-\textsc{ipfv-nmlz}\\
\glt `Now I’m thus going to tell the story of the boar tusk.’ [ulwa016\_00:03]
\z

\ea%215
    \label{ex:syntax:215}
          \textit{Sande ndan \textbf{apa nda na} te.}\\
\gll  Sande    anda=n    \textbf{apa}  \textbf{anda}    \textbf{na}    ta-e\\
    Sunday  \textsc{sg.dist=obl}  house  \textsc{sg.dist}  talk  say-\textsc{dep}\\
\glt `Last Sunday, [he] was talking about that church.’ (\textit{Sande} = TP) [ulwa014\_26:07]
\z

If there is a person addressed in such constructions, then this person typically occurs as the \isi{direct object} of the verb of speaking and the topic is included as an \isi{oblique} \isi{phrase} marked by \textit{=n} ‘\textsc{obl}’ following the word \textit{na} ‘talk’, as in \REF{ex:syntax:216}, \REF{ex:syntax:217}, and \REF{ex:syntax:218}. The \is{possession} possessive marker does not seem to be necessary.

\ea%216
    \label{ex:syntax:216}
          \textit{Ndï \textbf{isi nan} antane.}\\
\gll    ndï  \textbf{isi}  \textbf{na=n}    an=ta-n-e\\
    3\textsc{pl}  salt  talk=\textsc{obl}  1\textsc{pl.excl}=say-\textsc{ipfv-dep}\\
\glt `They were asking us about salt.’ [ulwa032\_27:51]
\z

\ea%217
    \label{ex:syntax:217}
          \textit{\textbf{Ndunduma nan} nïte nï mat: …}\\
\gll    \textbf{ndunduma}  \textbf{na=n}    nï=ta-e      nï    ma=ta\\
    ancestor  talk=\textsc{obl}  1\textsc{sg}=say-\textsc{dep}  \textsc{1sg}  \textsc{3sg.obj}=say\\
\glt `When [they] asked me about [their] ancestors, I said: …’ [ulwa014\_40:05]
\z

\ea%218
    \label{ex:syntax:218}
          \textit{Nïnji ulum \textbf{ndï nan} nïkap.}\\
\gll    nï-nji    ulum  \textbf{ndï}  \textbf{na=n}    nï=kï-p\\
    1\textsc{sg-poss}  palm  \textsc{3pl}  talk=\textsc{obl}  1\textsc{sg}=say-\textsc{pfv}\\
\glt `[She] told me about my sago palms.’ [ulwa032\_37:38]
\z

\subsection{Omission of verbs of speaking}\label{sec:13.4.4}

In casual speech, the verb of speaking is sometimes omitted, presumably implied by the word \textit{na} ‘talk’ plus the \isi{oblique marker} \textit{=n} ‘\textsc{obl}’ (or by the \isi{expletive} \isi{oblique} \isi{phrase} \textit{ma=n} ‘with it’), as in \REF{ex:syntax:219} and \REF{ex:syntax:220}.

\newpage

\ea%219
    \label{ex:syntax:219}
          \textit{Nï} \textbf{\textit{nan}}\textit{: …}\\
\gll    nï    \textbf{na=n}\\
    1\textsc{sg}  talk=\textsc{obl}\\
\glt `I said: …’ [ulwa014†]
\z

\ea%220
    \label{ex:syntax:220}
          \textit{Nï wolka} \textbf{\textit{man}} \textit{Carobim u nul man!}\\
\gll    nï    wolka  \textbf{ma=n}      Carobim  u    nï=ul    ma-n\\
    1\textsc{sg}  again  3\textsc{sg.obj=obl}  [name]    2\textsc{sg}  1\textsc{sg}=with  go-\textsc{imp}\\
\glt `I in turn [told] Carobim: “Go with me!”’ [ulwa014\_70:48]
\z

Sometimes \isi{speech} is reported without any word of speaking at all to signal the quotation -- that is, there is neither the verb \textit{ta-} ‘say’ or \textit{kï-} ‘say’ nor the noun \textit{na} ‘talk’ or an \isi{expletive} in its place. Such quotations are often signaled by \isi{intonation} or by \isi{paralinguistic sound}s or \is{paralinguistic gesture} gestures. Often they follow a \isi{phrase} of ‘seeing’, which may be used \isi{idiom}atically to signal thought or reflection, as in \REF{ex:syntax:221} and \REF{ex:syntax:222}.

\ea%221
    \label{ex:syntax:221}
          \textit{Nï \textbf{lïmndï ndala} ungusuwata wombïn ambi nda.}\\
\gll    nï    \textbf{lïmndï}  \textbf{ndï=ala}  un-ngusuwata  wombïn  ambi  anda\\
    1\textsc{sg}  eye    3\textsc{pl}=see  2\textsc{pl-}poor    work    big    \textsc{sg.dist}\\
\glt `I saw them [and said:] “You poor things – that’s big work.”’ [ulwa014\_59:50]
\z

\ea%222
    \label{ex:syntax:222}
          \textit{Itom ndï \textbf{lïmndï anala a} anji nungol ala ambi nape.}\\
\gll    itom  ndï  \textbf{lïmndï}  \textbf{an=ala}      \textbf{a}  an-nji        nungol ala      ambi  na-p-e\\
    father  3\textsc{pl}  eye    \textsc{1pl.excl}=see  ah  1\textsc{pl.excl-poss}  child    \textsc{pl.dist}  big    \textsc{detr-}be\textsc{{}-dep}\\
\glt `[Our] fathers saw us [and said:] “Ah! Our sons have gotten big.”’ [ulwa013\_04:54]
\z

As illustrated by example \REF{ex:syntax:222}, the \isi{exclamation} \textit{a} ‘ah!’ often signals \isi{speech} as well (\sectref{sec:8.3.3}). It typically belongs at the end of a \isi{prosodic unit}, with the rest of the quoted \isi{speech} continuing at the start of the subsequent \isi{prosodic unit}.

  When recounting stories, people may also omit a verb of speaking to make the action livelier \REF{ex:syntax:223}.

\ea%223
    \label{ex:syntax:223}
          \textit{Ne may tata!}\\
\gll    na-i      ma=i        tata\\
    \textsc{detr-}go.\textsc{pfv}  \textsc{3sg.obj}=go.\textsc{pfv}  papa\\
\glt `[He] went, went to him [and said:] “Papa!”’ [ulwa009\_02:06]
\z

Also, when conversations are recounted, the back-and-forth between two or more quoted speakers need not contain verbs of speaking between each turn, as in \REF{ex:syntax:224}.

\ea%224
    \label{ex:syntax:224}
          \textit{Nï atwana mat a un ango luwa? An ma we ndatïna le. Ande nol!}\\
\gll    nï    atwana    ma=ta      a  un  ango  luwa  an ma  we    anda=tï-na      lo-e  ande  na-lo\\
    1\textsc{sg}  question  3\textsc{sg.obj}=say  ah  2\textsc{pl}  which  place  1\textsc{pl.excl}    go  sago  \textsc{sg.dist}=take-\textsc{irr}  go-\textsc{dep}  ok    \textsc{detr-}go\\
\glt `I asked her: “Ah! Where are you [going]?” [And she said:] “We’re going to get sago starch.” [And I said:] “All right, go!”’ [ulwa037\_63:00]
\z

The word \textit{mïnja} ‘speech’ often appears in clauses introducing \isi{reported speech}. Much like \textit{na} ‘talk’, it may be used with an \isi{oblique marker} along with a verb of speaking. It may serve a discourse-\isi{deictic} function, pointing to what has just been reported or to what is about to be reported (i.e., ‘[someone] said this’), as in examples \REF{ex:syntax:225}, \REF{ex:syntax:226}, and \REF{ex:syntax:227}.

\ea%225
    \label{ex:syntax:225}
          \textit{Nï} \textbf{\textit{mïnjan}} \textbf{\textit{ndït}} \textit{mambinalakan!}\\
\gll    nï    \textbf{mïnja=n}    ndï=\textbf{ta}    ma-ambi=na-la-ka-n\\
    1\textsc{sg}  speech=\textsc{obl}  3\textsc{pl}=say  3\textsc{sg.obj-top}\textsc{=detr-irr-}let-\textsc{imp}\\
\glt `I told them: “Leave it alone!”’ [ulwa037\_07:09]
\z

\ea%226
    \label{ex:syntax:226}
          \textit{Ndï \textbf{mïnjan ke}: Mï unanï wa mbïpe.}\\
\gll    ndï    \textbf{mïnja=n}    \textbf{kï}{}-e    mï      unan=nï    wa mbï-p-e\\
    3\textsc{pl}    speech=\textsc{obl}  say-\textsc{dep}  \textsc{3sg.subj}  \textsc{1pl.incl=obl}  village    here-be\textsc{{}-ipfv}\\
\glt `They’re saying this: “It’s here in the village with us.”’ [ulwa037\_17:18]
\z

\ea%227
    \label{ex:syntax:227}
          \textit{Thomas mï na nïte nï \textbf{mïnjan mat}: …}\\
\gll    Thomas  mï      na    nï=ta-e      nï    \textbf{mïnja=n} ma=\textbf{ta}\\
    [name]    3\textsc{sg.subj}  talk  1\textsc{sg}=say-\textsc{dep}  \textsc{1sg}  speech=\textsc{obl}    3\textsc{sg.obj}=say\\
\glt `When Thomas told me, I said to him: …’ [ulwa037\_05:54]
\z

Frequently, however, the word \textit{mïnja} ‘speech’ occurs in a more \isi{elliptical} construction, in which it takes \isi{oblique} marking but where there is no expressed verb, as in \REF{ex:syntax:228} and \REF{ex:syntax:229}.

\is{syntax|)}
\is{speech|)}
\is{reported speech|)}
\is{reported speech|(}
\is{speech|(}
\is{syntax|(}

\newpage

\ea%228
    \label{ex:syntax:228}
          \textit{Nïnji inom mï} \textbf{\textit{mïnjan}} \textit{a nï inim lopop anmbï nay.}\\
\gll    nï-nji    inom    mï      \textbf{mïnja=n}    a  nï    inim lopo-p    an-mbï    na-i\\
    1\textsc{sg-poss}  mother    \textsc{3sg.subj}  speech=\textsc{obl}  ah  1\textsc{sg}  water    wash-\textsc{pfv}  out-here  \textsc{detr-}go.\textsc{pfv}\\
\glt `My mother said: “Ah! I bathed and came out.”’ [ulwa013\_03:01]
\z

\ea%229
    \label{ex:syntax:229}
          \textit{Kowe Marungun min} \textbf{\textit{mïnjan}} \textit{a una yeta la unan ma maytana!}\\
\gll    Kowe  Marungun  min  \textbf{mïnja=n}    a  unan    yeta ala      unan    ma  ma=ita-na\\
    [name]  [name]    \textsc{3du}  speech=\textsc{obl}  ah  1\textsc{pl.incl}  man    \textsc{pl.dist}  \textsc{1pl.incl}  go  3\textsc{sg.obj}=build-\textsc{irr}\\
\glt    ‘Kowe and Marungun said: “Ah! We are men; let’s go and build it!”’ [ulwa013\_07:05]
\z

\is{syntax|)}
\is{speech|)}
\is{reported speech|)}
\is{reported speech|(}
\is{speech|(}
\is{syntax|(}

Note the use of the \isi{interjection} \textit{a} ‘ah!’ in \REF{ex:syntax:228} and \REF{ex:syntax:229}. Sometimes the word \textit{mïnja} ‘speech’ stands alone, without any \isi{oblique} marking, to introduce \isi{reported speech}. In examples \REF{ex:syntax:230} and \REF{ex:syntax:231}, the \isi{interjection} \textit{m} ‘hm!’ helps signal the start of quoted \isi{speech}.

\is{syntax|)}
\is{speech|)}
\is{reported speech|)}

\ea%230
    \label{ex:syntax:230}
          \textit{Mï \textbf{mïnja m}!}\\
\gll    mï      \textbf{mïnja}  \textbf{m}\\
    3\textsc{sg.subj}  speech  hm\\
\glt `She said: “Hm!”’ [ulwa037\_53:08]
\z

\ea%231
    \label{ex:syntax:231}
          \textit{An lïmndï ndala \textbf{mïnja m} ala ankam kuma lawo.}\\
\gll    an      lïmndï  ndï=ala  \textbf{mïnja}  \textbf{m}  ala      ankam  kuma     ala{}-awa{}=o\\
    1\textsc{pl.excl}  eye    3\textsc{pl}=see  speech  hm  \textsc{pl.dist}  person  some    \textsc{pl.dist-int=voc}\\
\glt `We saw them and said: “Hm! Those really are some different people!”’ [ulwa013\_05:38]
\z

\subsection{Indirect discourse}\label{sec:13.4.5}

\is{reported speech|(}
\is{speech|(}
\is{syntax|(}
\is{indirect discourse|(}

When \isi{speech} is reported indirectly, two clauses are employed: a \isi{matrix clause} containing the verb of speaking and a \isi{dependent clause} containing the \isi{reported speech}. The \isi{dependent clause}, which consists of the indirect \isi{speech}, is embedded within the \isi{matrix clause}. Embedded \isi{dependent clause}s of \isi{indirect discourse} maintain Ulwa’s canonical S(O)V \isi{word order}. As in any clause, the subject of the \isi{embedded clause} may be omitted; this may be especially common when the subject of the \isi{embedded clause} matches that of the \isi{matrix clause} (e.g., ‘he\textsubscript{i} said that he\textsubscript{i} …’). In \REF{ex:syntax:232}, the form [na] (which is not necessarily required in such constructions) is analyzed as the \isi{detransitivizing} marker \textit{na-} ‘\textsc{detr}’.

\ea%232
    \label{ex:syntax:232}
          \textit{Alma mï Guren mï apa maytape} \textbf{\textit{natap}}.\\
\gll Alma  mï      {[Guren}    mï      apa     {ma=ita-p-e]}        na-\textbf{ta}{}-p\\
    [name]  \textsc{3sg.subj}  [[name]  3\textsc{sg.subj}  house    3\textsc{sg.obj}=build-\textsc{pfv-dep]}  \textsc{detr-}say-\textsc{pfv}\\
\glt `Alma said that Guren built the house.’ [elicited]
\z

The verb in the \isi{embedded clause} may be marked for \isi{irrealis} \isi{mood} if the reported statement refers to something that has not necessarily already transpired, as in \REF{ex:syntax:233} and  \REF{ex:syntax:234}.

\ea%233
    \label{ex:syntax:233}
          \textit{Mï mol} \textbf{\textit{malanda}} \textit{nate.}\\
\gll    mï      [ma=ul      ma=la-\textbf{nda}]    na-ta-e\\
    3\textsc{sg.subj}  [3\textsc{sg.obj}=with  3\textsc{sg.obj}=eat-\textsc{irr]}  \textsc{detr-}say-\textsc{dep}\\
\glt `She said that [she] would eat with him.’ [ulwa001\_06:05]
\z

\ea%234
    \label{ex:syntax:234}
          \textit{Ndï i} \textbf{\textit{mana}} \textit{nakap.}\\
\gll    ndï  i    [ma-\textbf{na}]  na-kï-p\\
    3\textsc{pl}  go.\textsc{pfv}  [go-\textsc{irr]}  \textsc{detr-}say-\textsc{pfv}\\
\glt `They came and talked about going.’ [ulwa001\_18:26]
\z

Example \REF{ex:syntax:235} suggests that it may also be possible for the \isi{embedded clause} of \isi{indirect discourse} to be embedded within a \isi{noun phrase} \isi{head}ed by the word \textit{na} ‘talk’. In this analysis, the literal rendering of this sentence would be something like ‘yesterday, Dorothy told me would-sell-it talk’.

\ea%235
    \label{ex:syntax:235}
          \textit{Dorothy awal makïke} \textbf{\textit{lunda}} \textit{na nïte.}\\
\gll    Dorothy  awal    [ma=kïke      lo-\textbf{nda}  na]    nï=ta-e\\
    [name]    yesterday  [3\textsc{sg.obj}=throw  go-\textsc{irr}  talk]  \textsc{1sg}=say-\textsc{dep}\\
\glt `Yesterday, Dorothy told me that [she] would sell it.’ [ulwa042\_01:45]
\z

  A \isi{reflexive} \isi{pronoun} may be used within the \isi{embedded clause} of \isi{speech} to refer to the speaker, which is the subject of the \isi{matrix clause}.\footnote{See examples \REF{ex:pron:23}, \REF{ex:pron:24}, and \REF{ex:pron:25} in \sectref{sec:6.3} for illustrations of how \isi{binding} principles apply within \isi{indirect discourse}.} This only seems to occur when the \isi{pronoun} referring to the speaker in the \isi{embedded clause} has a role other than subject; otherwise, no \isi{pronoun} is used at all. There is no designated \isi{logophoric pronoun} in Ulwa.

  Often a verb of speaking is used to refer to thinking (or other non-vocal events), as in examples \REF{ex:syntax:236}, \REF{ex:syntax:237}, and \REF{ex:syntax:238}. This use seems to be more common with the verb \textit{kï-} ‘say’ than with the verb \textit{ta-} ‘say’, with which it is otherwise mostly \isi{synonymous}. Note that there is no overt \isi{complementizer} used in complements of verbs of thinking.

\ea%236
    \label{ex:syntax:236}
          \textit{Alma mï Guren mï apa mayte} \textbf{\textit{nakap}}.\\
\gll Alma  mï      {[Guren}    mï      apa    {ma=ita-e]}  na-\textbf{kï-p}\\
    [name]  3\textsc{sg.subj}  [[name]  3\textsc{sg.subj}  house  3\textsc{sg.obj}=build-\textsc{dep]}     \textsc{detr-}say-\textsc{pfv}\\
\glt `Alma thought that Guren was building the house.’ [elicited]
\z

\ea%237
    \label{ex:syntax:237}
          \textit{Nï anmbï ina} \textbf{\textit{nakap}}.\\
\gll nï    [an-mbï  i-na]    na-\textbf{kï-p}\\
    1\textsc{sg}  [out-here  come-\textsc{irr]}  \textsc{detr-}say-\textsc{pfv}\\
\glt `I thought about coming out here.’ (i.e., ‘I thought that [I] would come out here.’) [ulwa040\_00:55]
\z

\ea%238
    \label{ex:syntax:238}
          \textit{Im maya ata mana} \textbf{\textit{nakap}}.\\
\gll {[im}    ma=iya      ata  {ma-na]}    na-\textbf{kï-p}\\
    {[tree}  3\textsc{sg.obj=}toward  up  go-\textsc{irr]}  \textsc{detr-}say-\textsc{pfv}\\
\glt `[He] thought about going up a tree.’ (i.e., ‘[He] thought that [he] would go up a tree.’) [ulwa035\_02:17]
\z

  In sentence \REF{ex:syntax:239}, the verb of speaking \textit{kï-} ‘say’ is assisted by the form \textit{wana-} ‘feel’, creating a \isi{compound} surrounding the \isi{embedded clause} of indirect \isi{speech} or thought, thus functioning as a \isi{discontinuous} \isi{compound verb} form (\sectref{sec:9.2.1}).

  \is{indirect discourse|)}
\is{syntax|)}
\is{speech|)}
\is{reported speech|)}

\ea%239
    \label{ex:syntax:239}
          \textit{Nï} \textbf{\textit{wana}} \textit{ndï ndine lïpe ndï anmape} \textbf{\textit{nakap}}.\\
\gll nï    \textbf{wana}  {[ndï}  ndï=in-e    lï-p-e      ndï {anma=p-e]}      na-\textbf{kï-p}\\
    1\textsc{sg}  feel  [3\textsc{pl}  3\textsc{pl}=get-\textsc{dep}  put-\textsc{pfv-dep}  \textsc{3pl} good=\textsc{cop{}-dep]}  \textsc{detr}{}-say-\textsc{pfv}\\
\glt `I thought that they got them down and that they were good.’ [ulwa014†]
\z

\section{Conditional sentences}\label{sec:13.5}

\is{conditional|(}
\is{conditional sentence|(}
\is{syntax|(}

 A basic \isi{conditional} statement in Ulwa consists of two clauses, the first (the \isi{protasis}) expressing the condition, and the second (the \isi{apodosis}) expressing the consequence. There are variations to this pattern, though, such as sentences that include more than one \isi{protasis}, sentences that include more than one \isi{apodosis}, and sentences in which the result clause is not a statement, but rather a \isi{question} or a \isi{command}. An example of a simple \isi{conditional} sentence is given in \REF{ex:syntax:240}.

\ea%240
    \label{ex:syntax:240}
          \textit{Inim} \textbf{\textit{lopota}} \textit{nï mana.}\\
\gll    inim  lopo-\textbf{ta}        nï    ma-na\\
    water  rain-\textsc{cond}    \textsc{1sg}  go-\textsc{irr}\\
\glt `If it rains, I’ll go.’ [elicited]
\z

  In the prototypical \isi{conditional sentence}, the verb in the \isi{protasis} is marked with the \isi{conditional} \isi{suffix} \textit{-ta} ‘\textsc{cond}’, whether affixed to the full \isi{perfective} form of the verb or to the \isi{verb stem} (\sectref{sec:4.12}). The verb in the \isi{apodosis} is always marked as \isi{irrealis}. This verb may additionally receive the \isi{suffix} \textit{{}-ta} ‘\textsc{cond}’, but only when built from the \isi{irrealis} form of the verb. That is, the verb in the \isi{apodosis} cannot be in any way perfective-marked or imperfective-marked. Thus, \isi{conditional} sentences in Ulwa are taken always to be \is{hypothetical condition} \isi{hypothetical}. For \isi{implicative condition}s (or \isi{factual condition}s), Ulwa does not employ the \isi{suffix} \textit{{}-ta} ‘\textsc{cond}’, and thus, on grammatical grounds, these are not taken to be \isi{conditional sentence}s.\footnote{Proclamations such as ‘if it rains, the ground gets wet’ would not be expressed as conditions in Ulwa; instead, a speaker would likely connect two clauses by subordinating one to the other and employing the \isi{dependent marker} on the first clause (i.e., ‘[when] it rains, the ground gets wet’).}

  Conditional clauses in Ulwa can variously be translated in \ili{English} with ‘if’, ‘when’, ‘whenever’, ‘once’, ‘lest’, ‘even if’, or ‘even though’, depending on the context and intended meaning of the utterance. Sentences \REF{ex:syntax:241}, \REF{ex:syntax:242}, and \REF{ex:syntax:243} are all translated with ‘if’. The \isi{protasis} always precedes the \isi{apodosis} in an Ulwa \isi{conditional sentence}.

\ea%241
    \label{ex:syntax:241}
          \textit{U atwana} \textbf{\textit{nïkïta}} \textit{nï utana.}\\
\gll    u    atwana    nï=kï-\textbf{ta}      nï    u=ta-na\\
    2\textsc{sg}  question  1\textsc{sg}=say\textsc{{}-cond}  \textsc{1sg}  \textsc{2sg}=say-\textsc{irr}\\
\glt `If you ask me, I’ll tell you.’ [elicited]
\z

\newpage

\ea%242
    \label{ex:syntax:242}
          \textit{Itom mï} \textbf{\textit{mbita}} \textit{unan landa.}\\
\gll    itom  mï      mbï-i-\textbf{ta}      unan    la-nda\\
    father  \textsc{3sg.subj}  here-go.\textsc{pfv-cond}  1\textsc{pl.incl}  eat-\textsc{irr}\\
\glt `If father comes, we’ll eat.’ [elicited]
\z

\ea%243
    \label{ex:syntax:243}
          \textit{Mï} \textbf{\textit{anmapïta}} \textit{we ande ndï wolka mol nena.}\\
\gll    mï      anma=p-\textbf{ta}      we    ande  ndï  wolka     ma=ul      na-i-na\\
    3\textsc{sg.subj}  good=\textsc{cop{}-cond} then  ok    \textsc{3pl}  again    3\textsc{sg.obj}=with  \textsc{detr-}come-\textsc{irr}\\
\glt `If he is well, then, OK, they would come back with him.’\footnote{This example also illustrates how \isi{conditional sentence}s may contain the \isi{subordinator} \textit{we} ‘then’ to connect the two clauses (\sectref{sec:12.2.7}).} [ulwa029\_10:14]
\z

  The \isi{conditional sentence}s given in \REF{ex:syntax:244}, \REF{ex:syntax:245}, and \REF{ex:syntax:246} are better translated with ‘when’.

\ea%244
    \label{ex:syntax:244}
          \textit{Nï anganika ma maya} \textbf{\textit{mata}} \textit{ngan lowonda.}\\
\gll    nï    anganika  ma  ma=iya      ma-\textbf{ta}    ngan lo-wo-nda\\
    1\textsc{sg}  after    go  3\textsc{sg.obj}=toward  go-\textsc{cond}  1\textsc{du.excl}    \textsc{irr-}sleep-\textsc{irr}\\
\glt `When I later go, go to her, we two will sleep.’ [ulwa032\_18:45]
\z

\ea%245
    \label{ex:syntax:245}
          \textit{We mï akïnakapïta u} \textbf{\textit{mankapta}} \textit{mï anmapïna.}\\
\gll    we    mï      akïnaka=p-ta    u    ma=nïkï-p-\textbf{ta}     mï      anma=p-na\\
    sago  3\textsc{sg.subj}  young=\textsc{cop}{}-\textsc{cond}  2\textsc{sg}  3\textsc{sg.obj}=dig{}-\textsc{pfv-cond}    3\textsc{sg.subj}  good=\textsc{cop}{}-\textsc{irr}\\
\glt `When the sago starch is fresh, [then] when you prepare it, it will be good.’ [ulwa014\_60:23]
\z

\ea%246
    \label{ex:syntax:246}
          \textbf{\textit{Ndïnkïta}} \textit{ndul wa undana mane.}\\
\gll    ndï=nïkï-\textbf{ta}      ndï=ul    wa    unda-na  ma-n-e\\
    3\textsc{pl}=dig{}-\textsc{irr-cond}  3\textsc{pl}=with  village  go-\textsc{irr}    go-\textsc{ipfv-dep}\\
\glt `When we have butchered them, we’re going to go home with them.’ [ulwa038\_03:18]
\z

As illustrated by \REF{ex:syntax:243}, the \isi{conditional} marker \textit{-ta} ‘\textsc{cond}’ may follow the \isi{copular enclitic} \textit{=p} ‘\textsc{cop}’ (\sectref{sec:10.2}). Sentence \REF{ex:syntax:247} also contains a \isi{copular enclitic}, as well as a \isi{periphrastic} ‘going’ verb in the \isi{apodosis} in lieu of a simple \isi{irrealis} verb.

\ea%247
    \label{ex:syntax:247}
          \textit{\textbf{Tembipïta} ndï mo \textbf{ina mane}.}\\
\gll tembi=p-\textbf{ta}    ndï  ma=u      \textbf{i-na}    \textbf{ma-n-e}\\
    bad=\textsc{cop{}-cond}  \textsc{3pl}  \textsc{3sg.obj=}from  come-\textsc{irr}  go-\textsc{ipfv-dep}\\
\glt `Whenever [people] were sick, they were going to come from there.’ [ulwa029\_09:50]
\z

The use of \isi{periphrastic} ‘going’ verbs in an \isi{apodosis} is further illustrated by sentence \REF{ex:syntax:248}, which also illustrates some of the complexity that is possible among \isi{conditional sentence}s in Ulwa. It is not uncommon for either the \isi{protasis} or the \isi{apodosis} (or both) to be multiclausal. Whereas the \isi{protasis} in \REF{ex:syntax:248} is \is{monoclausal construction} monoclausal (consisting of just a single verb, marked with the \isi{conditional} \isi{suffix} \textit{-ta} ‘\textsc{cond}’), the \isi{apodosis} is multiclausal (consisting of first a perfective-marked verb and then a \isi{periphrastic} construction that gives the entire multiclausal \isi{apodosis} its \isi{irrealis} \isi{mood}).\footnote{Note that it is rare for the \isi{dependent-marker} \isi{suffix} to occur within \isi{conditional} clauses, even within multiclausal \is{apodosis} apodoses or \is{protasis} protases -- that is, it does not occur anywhere except at the very end of the \isi{apodosis}, as in \REF{ex:syntax:248}. These clauses are thus considered to be \isi{coordinate} structures (\sectref{sec:12.1}). The perfective-marked verb in this example can thus be considered an \isi{irrealis perfective} (\sectref{sec:4.9}); and the \isi{periphrastic} construction (technically an \isi{irrealis}-marked verb plus an imperfective-marked verb) can thus be considered the requisite \isi{irrealis} construction of the \isi{apodosis}.}

\ea%248
    \label{ex:syntax:248}
          \textit{Nduwe unïn anmbïlumopta un nul ndinap \textbf{ndulunda mane}.}\\
\gll ndï=we  un=ïn  an-mbï-lumo-p-ta        un  nï=ul     ndï=ina-p    \textbf{ndï=u-lo-nda}    \textbf{ma-n-e}\\
    3\textsc{pl}=cut  2\textsc{pl=obl}  out-here-put-\textsc{pfv-cond}  \textsc{2pl}  \textsc{1sg=}with    3\textsc{pl}=get-\textsc{pfv}  3\textsc{pl}=from-cut-\textsc{irr}  go-\textsc{ipfv-dep}\\
\glt `Once [I] have cut them [= tobacco leaves] out for you, you, having gotten them with me, are going to peel them.’ [ulwa042\_03:44]
\z

  Like example \REF{ex:syntax:248}, example \REF{ex:syntax:249} illustrates an \isi{irrealis perfective} in the \isi{apodosis}, here \isi{morphological}ly clearer since the form of the verb is \textit{lamap} {‘eat} \linebreak {[\textsc{irr/pfv}]’} (cf. \sectref{sec:4.9}).

\is{syntax|)}
\is{conditional sentence|)}
\is{conditional|)}

\is{conditional|(}
\is{conditional sentence|(}
\is{syntax|(}

\ea%249
    \label{ex:syntax:249}
          \textit{U mat ita nï} \textbf{\textit{malamap}} \textit{wa} \textbf{\textit{mana}}.\\
\gll u    ma=tï      i-ta        nï    \textbf{ma=la{}-ama-p} wa      \textbf{ma-na}\\
    2\textsc{sg}  3\textsc{sg.obj}=take  go.\textsc{pfv-cond}  \textsc{1sg}  3\textsc{sg.obj}=\textsc{irr}{}-eat-\textsc{pfv}    village    go-\textsc{irr}\\
\glt `If you bring it, I’ll eat it and go home.’ [ulwa032\_28:45]
\z

In addition to exhibiting a multiclausal \isi{apodosis}, sentence \REF{ex:syntax:249} exemplifies a \isi{protasis} that contains two verbs. The first verb \textit{tï-} ‘take’, however, is often \isi{defective} and \isi{semantic}ally closely connected to the following verb (often \textit{ma-} {\textasciitilde} \textit{i-} ‘go’ or \textit{na-} ‘give’), so this is perhaps not the clearest example of multiple clauses. In the \isi{conditional sentence}s given in \REF{ex:syntax:250} and \REF{ex:syntax:251}, the verb \textit{tï-} ‘take’ is used both in the \isi{protasis} and in the \isi{apodosis}.

\ea%250
    \label{ex:syntax:250}
          \textit{Olsem u} \textbf{\textit{ngalat}} \textit{nïnata nï nïnji} \textbf{\textit{ngalat}} \textit{unanda.}\\
\gll    olsem  u    ngala=\textbf{tï}    nï=na-ta      nï    nï-nji ngala=\textbf{tï}    u=na-nda\\
    thus  \textsc{2sg}  \textsc{pl.prox}=take  1\textsc{sg}=give-\textsc{cond}  \textsc{1sg}  \textsc{1sg-poss}    \textsc{pl.prox}=take  2\textsc{sg}=give-\textsc{irr}\\
\glt `So if you give these to me, I’ll give mine to you.’ (\textit{olsem} = TP) [ulwa029\_09:25]
\z

\ea%251
    \label{ex:syntax:251}
          \textit{Nï ko nji} \textbf{\textit{tï}} \textit{unata u ko nji} \textbf{\textit{tï}} \textit{nïnanda.}\\
\gll    nï    ko  nji    \textbf{tï}    u=na-ta      u    ko    nji \textbf{tï}    nï=na-nda\\
    1\textsc{sg}  just  thing  take  2\textsc{sg}=give-\textsc{cond}  \textsc{2sg}  just    thing    take  1\textsc{sg}=give-\textsc{irr}\\
\glt `If I give you something, you should give me something.’ [ulwa032\_28:01]
\z

In sentence \REF{ex:syntax:252}, the verb \textit{tï-} ‘take’ is indeed marked for \isi{TAM} (here, \isi{perfective}).

\ea%252
    \label{ex:syntax:252}
          \textit{Kalam} \textbf{\textit{ngatïn}} \textit{mol luta ngaya ndapïna.}\\
\gll    kalam    nga=\textbf{tï-n}        ma=ul      lo-ta    ngaya anda=p-na\\
    knowledge  \textsc{sg.prox}=take-\textsc{pfv}  3\textsc{sg.obj}=with  go-\textsc{cond}  far    \textsc{sg.dist}=be-\textsc{irr}\\
\glt `If he gets this knowledge and goes around with it, he will be far away.’ [ulwa014\_73:58]
\z

Notably, there is no \isi{conditional} marking on the first verb in the \isi{protasis}, even though it is not \isi{defective} here. There are, however, instances in which multiple verbs in the \isi{protasis} may be marked with the \isi{conditional} \isi{suffix} \textit{-ta} ‘\textsc{cond}’. In such sentences, it can be assumed that each verb represents a condition that must be met for the state or event in the \isi{apodosis} to be or occur. Often these are best translated in \ili{English} with a single clause, often with a single verb. The Ulwa sentence, however, contains multiple verbs in the \isi{protasis} that may constitute either a single clause with multiple \isi{verb phrase}s or a multiclausal \isi{protasis}, as in \REF{ex:syntax:253}.\footnote{This sentence has two verbs marked with the \isi{conditional} \isi{suffix} \textit{-ta} ‘\textsc{cond}’ in the \isi{protasis} (literally ‘if you are just here and if [you] eat food …’). Note that a simple \isi{irrealis} \isi{stem} of \textit{ama-} ‘eat’ (i.e., [la-]) is employed, and not an \isi{irrealis perfective} \isi{compound} \isi{stem} (i.e., not \textsuperscript{†}/la-ama-/). This is because the \isi{hypothetical} event in the \isi{protasis} is not \isi{perfective} – that is, the event need not have been completed for the situation in the \isi{apodosis} to be true.}

\ea%253
    \label{ex:syntax:253}
          \textit{U kwa} \textbf{\textit{mapta}} \textit{mundu} \textbf{\textit{lata}} \textit{tamndï ko mundu ndïwalin.}\\
\gll    u    kwa  ma=p-\textbf{ta}      mundu  la-\textbf{ta}    tamndï   ko mundu  ndï=wali-n[da]\\
    2\textsc{sg}   just    3\textsc{sg.obj}=be\textsc{{}-cond} food  eat\textsc{{}-cond} owner    just    hunger  3\textsc{pl}=hit-\textsc{irr}\\
\glt `If you just eat the food there, the owners will go hungry.’ [ulwa032\_13:30]
\z

  Note, however, that verbs in the \isi{protasis} clause or clauses should properly be marked as \isi{conditional} only if the \isi{apodosis} is in fact contingent on them. Thus, in sentence \REF{ex:syntax:254}, the first verb (in fact a \isi{verbalized} noun) receives no \isi{conditional} \isi{suffix}, but rather is \isi{dependent-marked}.

\ea%254
    \label{ex:syntax:254}
          \textbf{\textit{Nungolkepe}} \textit{nï mandïm sata mï nït awi lïp nul wandam mana.}\\
\gll    nungolke=p-\textbf{e}    nï    ma=andïm    sa-ta    mï nï=tï    awi      lï-p      nï=ul    wandam  ma-na\\
    child=\textsc{cop{}-dep}  \textsc{1sg}  \textsc{3sg.obj=}for  cry\textsc{{}-cond}  \textsc{3sg.subj}    \textsc{1sg=}take  shoulder  put-\textsc{pfv}  \textsc{1sg=}with  jungle    go-\textsc{irr}\\
\glt `When [I] was a child and I would cry for him [= my father], he would put me on his shoulder and go with me to the jungle.’ [ulwa033\_01:47]
\z

The \isi{protasis} may have any number of \isi{conditional}ly marked verbs, however. Sentence \REF{ex:syntax:255} contains three.

\ea%255
    \label{ex:syntax:255}
          \textbf{\textit{Mambilakata}} \textbf{\textit{mankïta}} \textit{keka itïm} \textbf{\textit{nomopta}} \textit{una wo lolop wa pïn.}\\
\gll    ma-ambi=la-ka-\textbf{ta}        ma=nïkï-\textbf{ta}      keka     itïm  na-lumo-p-\textbf{ta}      unan    wa  lolop  wa    p-na\\
    3\textsc{sg.obj-top}=\textsc{irr-}let-\textsc{cond}  \textsc{3sg.obj}=dig{}-\textsc{cond}  completely    trash  \textsc{detr-}put-\textsc{pfv-cond}  \textsc{1pl.incl}  just  just    village  be-\textsc{irr}\\
\glt `If [we] abandon it, cut it [out], and throw [it] completely in the trash, [then] we will just stay [fine here] in the village.’ [ulwa037\_07:19]
\z

It is also possible for the \isi{conditional} \isi{suffix} to appear on one or more verbs in the \isi{apodosis}. I suspect that this is a form of overextension, marking a clause as \isi{conditional} simply because it is connected to a \isi{conditional} clause. This may be seen in examples \REF{ex:syntax:256} and \REF{ex:syntax:257}.

\ea%256
    \label{ex:syntax:256}
          \textbf{\textit{Nditapta}} \textit{kalam mï} \textbf{\textit{natïnangata}}.\\
\gll ndï=ita-p-\textbf{ta}      kalam    mï      na-tïnanga-\textbf{ta}\\
    3\textsc{pl}=build-\textsc{pfv-cond}  knowledge  3\textsc{sg.subj}  \textsc{detr}{}-arise-\textsc{cond}\\
\glt `When [they] build them [= school buildings], knowledge will increase.’ [ulwa014\_26:35]
\z

\ea%257
    \label{ex:syntax:257}
          \textit{U mat nonal luwa} \textbf{\textit{malta}} \textit{mï lowop} \textbf{\textit{tembipïta}}.\\
\gll u    ma=tï      nonal  luwa  ma=lï{}-\textbf{ta} mï      lo-wo-p    tembi=p-\textbf{ta}\\
    2\textsc{sg}  \textsc{3sg.obj}=take  wind  place  3\textsc{sg.obj}=put-\textsc{irr-cond}  3\textsc{sg.subj}  \textsc{irr}{}-sleep-\textsc{irr}  bad=\textsc{cop{}-cond}\\
\glt `If you put it out in the open air, it will go bad overnight.’ [ulwa014\_64:39]
\z

\is{syntax|)}
\is{conditional sentence|)}
\is{conditional|)}

\is{conditional|(}
\is{conditional sentence|(}
\is{syntax|(}

Since clauses in the \isi{apodosis} may also be marked with the \isi{conditional} \isi{suffix} \textit{-ta} ‘\textsc{cond}’, it is sometimes not entirely clear whether a clause belongs to the \isi{protasis} or to the \isi{apodosis} – that is, in instances in which a \isi{conditional} sentence contains more than two clauses. Sentence \REF{ex:syntax:258} illustrates this possible ambiguity.

\ea\label{ex:syntax:258}  \textit{Mat} \textbf{\textit{ndïnata}} \textit{ndï} \textbf{\textit{mankïta}} \textit{malanda.}\\
\gll    ma=tï      ndï=na-\textbf{ta}      ndï  ma=nïkï-\textbf{ta} ma=la-nda\\
    3\textsc{sg.obj}=take  3\textsc{pl}=give-\textsc{cond}  3\textsc{pl}  \textsc{3sg.obj=}dig{}-\textsc{cond}    \textsc{3sg}=eat-\textsc{irr}\\
\glt    (a) ‘If [she] gives it to them, then they will prepare it and eat it.’

    (b) ‘If [she] gives it to them, and if they prepare it, then [they] will eat it.’ [ulwa014\_40:35]
\z

While many \isi{conditional sentence}s have statements as \is{apodosis} apodoses, it is also possible for a \isi{conditional sentence} to have a \isi{question} or a \isi{command} as the \isi{apodosis}. In example \REF{ex:syntax:259}, the \isi{apodosis} takes the form of a \isi{question}.

\newpage

\ea%259
    \label{ex:syntax:259}
          \textit{Nï} \textbf{\textit{mamata}} \textit{olsem ko kwa} \textbf{\textit{mbïpïta}} \textit{lïmndï} \textbf{\textit{ndutata}} \textit{ay} \textbf{\textit{nïkïta}} \textit{ndul landa?}\\
\gll    nï    ma=ma-\textbf{ta}        olsem  ko  kwa  mbï-p-\textbf{ta} lïmndï  ndï=uta-\textbf{ta}      ay    nïkï-\textbf{ta}    ndï=ul    la-nda\\
    1\textsc{sg}  3\textsc{sg.obj}=go-\textsc{cond}    thus  just  one    here-be\textsc{{}-cond}    eye    3\textsc{pl}=grind\textsc{{}-cond} sago  dig{}-\textsc{cond}  3\textsc{pl}=with  eat-\textsc{irr}\\
\glt `If I were to go there, who would stay here, watch after them, prepare sago, and eat with them?’ (\textit{olsem} = TP) [ulwa027\_00:27]
\z

Note the extensive use of \isi{conditional}-marked verbs throughout example \REF{ex:syntax:259}; indeed, every verb except the final \isi{irrealis} verb in the \isi{apodosis} exhibits the \isi{conditional} \isi{suffix} \textit{-ta} ‘\textsc{cond}’. Example \REF{ex:syntax:260} also has a \isi{question} for its \isi{apodosis}. Here, only one verb is marked with the \isi{suffix} \textit{-ta} ‘\textsc{cond}’.

\ea%260
    \label{ex:syntax:260}
          \textit{Un nambï} \textbf{\textit{kenmbupïta}} \textit{un anjikaka imbamka lunda?}\\
\gll un  nambï  kenmbu=p-\textbf{ta}    un  anjikaka  imbam-ka  lo-nda\\
    2\textsc{pl}  skin  heavy=\textsc{cop{}-cond} 2\textsc{pl}  how    run-let    go-\textsc{irr}\\
\glt `But if your body is heavy, how can you run around?’ [ulwa032\_34:46]
\z

It is possible for the \isi{question} word to occur in the \isi{protasis} even when (at least in the \ili{English} translation) the \isi{interrogative} would be expected to occur in the \isi{apodosis}, as is the case in \REF{ex:syntax:261}. This may have something to do with the placement and \isi{scope} of the \isi{negator} \textit{ango} ‘\textsc{neg}’ in \isi{multiclausal construction}s (\sectref{sec:13.3.3}), since the \isi{question word}s are etymologically related to \textit{ango} ‘\textsc{neg}’.

\ea%261
    \label{ex:syntax:261}
          \textit{Una} \textbf{\textit{ango}} \textit{luwa pïta inim malanda?}\\
\gll    unan    \textbf{ango}  luwa  p-ta    inim  ma=la-nda\\
    1\textsc{pl.incl}  \textsc{neg}  place  be\textsc{{}-cond} water  3\textsc{sg.obj}=eat-\textsc{irr}\\
\glt `Where will we drink water?’ (Literally ‘We, if at which place, will drink water?’) [ulwa029\_03:26]
\z

It is also common for \is{apodosis} apodoses to take the form of \isi{imperative}s. In fact, \isi{conditional} constructions often serve the \isi{pragmatic} function of softening a \isi{request} or \isi{command} (\sectref{sec:13.2}), as seen in \REF{ex:syntax:262}.

\newpage

\ea%262
    \label{ex:syntax:262}
          \textit{Un ma ya koya ata ma} \textbf{\textit{maynapta}} \textit{nditap ndïtïl nap ndït} \textbf{\textit{ita}} \textit{una} \textbf{\textit{ndutata}} \textit{inim uta ndïlan!}\\
\gll    un  ma  ya      ko=iya      ata  ma  ma=ina-p-\textbf{ta} ndï=ita-p    ndï=tïl    na-p    ndï=tï    i-\textbf{ta}     unan    ndï=uta-\textbf{ta}      inim  u-ta    ndï=la-n\\
    \textsc{2pl}  go  coconut  \textsc{indf=}toward  up  go  3\textsc{sg.obj}=get-\textsc{pfv-cond}    3\textsc{pl}=build-\textsc{pfv}  \textsc{3pl}=husk  \textsc{detr-}be  3\textsc{pl}=take  go.\textsc{pfv-cond}    1\textsc{pl.incl}  \textsc{3pl=}grind\textsc{{}-cond} water  put\textsc{{}-cond}  \textsc{3pl}=eat-\textsc{imp}\\
\glt `If you go, go up a coconut tree, and get it, [and] if [you] tie them [= the coconuts], husk them, and bring them, [then] we will grind them into water and eat them!’ [ulwa018\_00:35]
\z

Although presented as a series of conditions, the first clauses in sentence \REF{ex:syntax:262} are \isi{pragmatic}ally tantamount to a \isi{request} (i.e., ‘please get coconuts so that we may grind them and eat them’). Note also how the \isi{conditional} form \textit{-ta} ‘\textsc{cond}’ occurs in (what is translated here as) the \isi{apodosis}. Another example of a \isi{conditional} form used in a \isi{command} is given in \REF{ex:syntax:263}.

\ea%263
    \label{ex:syntax:263}
          \textit{Un keka nul ndïn umop ulwap ndïwat} \textbf{\textit{itapta}} \textit{we un nol!}\\
\gll    un  keka      nï=ul    ndï=n    lumo-p    ulwa=p ndï=wat  ita-p-\textbf{ta}      we    un  na-lo\\
    2\textsc{pl}  completely  1\textsc{sg}=with  3\textsc{pl=obl}  put-\textsc{pfv}  nothing=\textsc{cop}    3\textsc{pl}=atop  build-\textsc{pfv-cond}  then  2\textsc{pl}  \textsc{detr-}go\\
\glt `If you plant all of them [= the tobacco seedlings] with me and cover them [with leaves], then you [may] go!’ [ulwa040\_01:15]
\z

Sometimes, even without an \isi{imperative} form, a \isi{conditional sentence} can serve \isi{pragmatic}ally as a \isi{request} (see \sectref{sec:13.2.3}).

  In some instances, the use of the \isi{conditional} in \isi{imperative}s may be seen as necessary to clarify a sequence of tasks that the speaker wishes the interlocutor to undertake, as in \REF{ex:syntax:264}.

\ea%264
    \label{ex:syntax:264}
          \textit{Inim ngan apïn} \textbf{\textit{ta}} \textit{we inim ngan ndanan!}\\
\gll    inim  nga=n      apïn  [lï]-\textbf{ta}    we    inim  nga=n ndï=ana-n\\
    water  \textsc{sg.prox=obl}  fire    put-\textsc{cond}  then  water  \textsc{sg.prox=obl}    \textsc{3pl}=scrub-\textsc{imp}\\
\glt `Put this water on the fire, and then scrub with this water!’ [ulwa014\_50:35]
\z

Finally, in \isi{prohibition}s, the \isi{conditional} \isi{suffix} \textit{-ta} ‘\textsc{cond}’ may be employed without any following \isi{apodosis}, as in \REF{ex:syntax:265}. This may be considered a form of \isi{ellipsis}. Alternatively, it could be the case that the \isi{suffix} here is related to the \isi{speculative} \isi{suffix} \textit{-t} ‘\textsc{spec}’ (\sectref{sec:4.11}), which is also often used in \isi{negative command}s.

\ea%265
    \label{ex:syntax:265}
          \textit{Wanap nji ndïn} \textbf{\textit{umopta}}\textit{!}\\
\gll    wanap  nji    ndï=n    lumo-p-\textbf{ta}\\
    \textsc{proh}  thing  3\textsc{pl=obl}  put-\textsc{pfv-cond}\\
\glt `Don’t grow things!’ [ulwa014\_08:15]
\z

See \sectref{sec:13.2.4} and \sectref{sec:13.3.2} for more on \isi{prohibition}s.

\is{syntax|)}
\is{conditional sentence|)}
\is{conditional|)}

\section{Counterfactual sentences}\label{sec:13.6}

\is{counterfactual|(}
\is{syntax|(}

Counterfactual sentences in Ulwa generally employ a verb with \isi{irrealis} marking, as illustrated by \REF{ex:syntax:266}.

\ea%266
    \label{ex:syntax:266}
          \textit{Apïn kali} \textbf{\textit{malnda}} \textit{inim ngalope nï makam.}\\
\gll    apïn  kali  ma=lï\textbf{{}-nda} inim  nga=lo-p-e        nï ma=kamb\\
    fire    send  3\textsc{sg.obj}=put-\textsc{irr}  water  \textsc{sg.prox}=cut-\textsc{pfv-dep}  1\textsc{sg}         3\textsc{sg.obj}=shun\\
\glt `[I] would have burned it, but this rain came, so I didn’t want to.’ [ulwa037\_50:25]
\z

  The \isi{irrealis} \isi{mood} is a natural resource for designating \isi{counterfactual} statements. While one prolific use of the \isi{irrealis} \isi{mood} is to mark \isi{future} states or events, which are in a sense always \isi{counterfactual}, the \isi{irrealis} \isi{mood} can also be applied to \isi{hypothetical} states or events in \isi{present} or \isi{past} \isi{time} that are known not to be true. Sentence \REF{ex:syntax:267} posits a \isi{hypothetical} event in the \isi{past} that is known not to have occurred. In this sentence, an \isi{irrealis} form is used. This may be contrasted with the \isi{perfective} forms used in the second clause to mark what was known to have occurred.

\ea%267
    \label{ex:syntax:267}
          \textit{Wa} \textbf{\textit{mbïpïna}} \textit{Kowe awa mangusuwa asape mï i.}\\
\gll    wa    mbï-p-\textbf{na}    Kowe  awa  ma-ngusuwa    asa-p-e     mï      i\\
    village  here-be-\textsc{irr}  [name]  \textsc{int}    3\textsc{sg.obj-}poor    hit-\textsc{pfv-dep}    3\textsc{sg.subj}  go.\textsc{pfv}\\
\glt    ‘[Kïtalwe] would have stayed in the village, but Kowe himself hit the poor thing   [= Kïtalwe] and he [= Kïtalwe] left.’ [ulwa037\_46:06]
\z

Counterfactual statements are frequently used in \isi{conditional sentence}s, presenting \isi{hypothetical} states or events, whether in the \isi{past} or in the \isi{present}. Examples \REF{ex:syntax:268} and \REF{ex:syntax:269} illustrate \isi{counterfactual} \isi{conditional} statements referring to \isi{past} \isi{time}.

\ea%268
    \label{ex:syntax:268}
          \textit{Nï ndïn ndul \textbf{sita} ndï ango uta tï \textbf{nïnanda}.}\\
\gll nï    ndï=n    ndï=ul    si-\textbf{ta}      ndï  ango  uta    tï nï=na-\textbf{nda}\\
    1\textsc{sg}  3\textsc{pl=obl}  3\textsc{pl}=with  push\textsc{{}-cond}  \textsc{3pl}  \textsc{neg} bird  take    1\textsc{sg}=give-\textsc{irr}\\
\glt    ‘If I had shown them [= the birds] to them, I would not have been able to take the birds for myself.’ [ulwa032\_56:25]
\z

\ea%269
    \label{ex:syntax:269}
          \textit{U uk undana} \textbf{\textit{nakïta}} \textit{u ilum atnï wambana} \textbf{\textit{mokona}}.\\
\gll u    uk    unda-na  na-kï-\textbf{ta}      u     ilum at=nï    wambana  moko-\textbf{na}\\
    2\textsc{sg}  hook  go-\textsc{irr}    \textsc{detr-}say\textsc{{}-cond}  \textsc{2sg} little    fight=\textsc{obl}  fish    take-\textsc{irr}\\
\glt `If you had thought of fishing, you would have gotten many fish.’ (Literally ‘would have taken with little fight’; \textit{uk} = TP \textit{huk}) [ulwa032\_22:59]
\z

Examples \REF{ex:syntax:270} and \REF{ex:syntax:271} illustrate \isi{counterfactual} \isi{conditional} statements referring to \isi{present} \isi{time}.

\is{syntax|)}
\is{counterfactual|)}

\ea%270
    \label{ex:syntax:270}
          \textit{Ndï} \textbf{\textit{ndandïlaluta}} \textbf{\textit{ndalin}}\textit{. Ndï ango ndale.}\\
\gll    ndï  ndï=andïla-lo-\textbf{ta}      ndï=ali-\textbf{n[da]}    ndï    ango ndï=ali{}-e\\
    3\textsc{pl}  \textsc{3pl=}await-go-\textsc{irr-cond}  3\textsc{pl}=scrape-\textsc{irr}  \textsc{3pl}    \textsc{neg}    \textsc{3pl}=scrape-\textsc{ipfv}\\
\glt `If they looked for them, they would scrape them. But they don’t scrape them.’ [ulwa032\_38:31]
\z

\ea%271
    \label{ex:syntax:271}
          \textit{A! Ala num tï} \textbf{\textit{mbïlta}} \textit{nï mbu} \textbf{\textit{wonlakana}}.\\
\gll a  ala      num  tï    mbï-lï{}-\textbf{ta}      nï    mbï-u won-\textbf{la}{}-ka-\textbf{na}\\
    ah  \textsc{pl.dist}  canoe  take  here-put-\textsc{cond}  \textsc{1sg}  here-from    cut-\textsc{irr}{}-let-\textsc{irr}\\
\glt `Ah! If only those folks had a canoe here, I would cross from there.’ [ulwa037\_02:50]
\z

\newpage

\section{Passive voice}\label{sec:13.7}

\is{passive voice|(}
\is{passivization|(}
\is{voice|(}
\is{syntax|(}

Syntactically, \isi{passive} sentences are remarkable in Ulwa, since they do not comply with the canonical verb-final clause structure. Despite their crosslinguistically unusual formation (in that they rely solely on the manipulation of \isi{word order}), these constructions in Ulwa are considered here to be \isi{passive}, since they satisfy common criteria for defining passives (see \cite{Siewierska2013}; \cite{Barlow2019a}).

  \is{active voice} Active sentences in Ulwa have a fairly rigid SOV \isi{constituent order} (\sectref{sec:11.1}), as in \REF{ex:syntax:272}. In \isi{passive} sentences, on the other hand, the verb occupies a different position: it precedes the subject, as in \REF{ex:syntax:273}.

\ea%272
    \label{ex:syntax:272}
          \textit{Yeta mï lamndu} \textbf{\textit{masap}}.\\
\gll yeta  mï      lamndu  \textbf{ma=asa-p}\\
    man  3\textsc{sg.subj}  pig      3\textsc{sg.obj=}hit-\textsc{pfv}\\
\glt `The man killed the pig.’ [elicited]
\z

\ea%273
    \label{ex:syntax:273}
          \textbf{\textit{Asape}} \textit{lamndu mï.}\\
\gll    \textbf{asa-p-e}    lamndu  mï\\
    hit-\textsc{pfv-dep}  pig      3\textsc{sg.subj}\\
\glt `The pig was killed.’ [elicited]
\z

Likewise, in the \isi{active-voice} sentence in \REF{ex:syntax:274}, the \isi{word order} is SOV, whereas in the passive-\isi{voice} sentence in \REF{ex:syntax:275}, the \isi{word order} is VS. The subject argument in the \isi{passive} sentence corresponds to the object argument of its \is{active voice} active equivalent.

\ea%274
    \label{ex:syntax:274}
          \textit{Inom utam} \textbf{\textit{nduwanap}}.\\
\gll inom  utam  \textbf{ndï=wana-p}\\
    mother  yam  3\textsc{pl=}cook-\textsc{pfv}\\
\glt `Mother cooked the yams.’ [elicited]
\z

\ea%275
    \label{ex:syntax:275}
          \textbf{\textit{Wanape}} \textit{utam ndï.}\\
\gll    \textbf{wana-p-e}    utam  ndï\\
    cook-\textsc{pfv-dep}  yam  3\textsc{pl}\\
\glt `The yams were cooked.’ [elicited]
\z

As the grammatical subject in the \isi{passive} sentence, the postverbal \isi{patient}-like argument is the only obligatory argument of the verb. As the subject, it is marked not with the \isi{object marker} but with the \isi{subject marker}.\footnote{Although these markers are usually \isi{homophonous}, there is a distinction in the 3\textsc{sg} form: \textit{mï} ‘3\textsc{sg.subj’} for subjects and \textit{ma=} ‘3\textsc{sg.obj’} for \isi{non-subject}s (\sectref{sec:7.2}, \sectref{sec:11.2}). Crucially, as in example \REF{ex:syntax:273}, the form [mï] (and not \textsuperscript{†}[ma]) follows the subject of the \isi{passive} clause.} Furthermore, the verb in the \isi{passive} sentence does not permit an \isi{object-marker} \isi{proclitic}, since the \isi{semantic} \isi{patient} (the object of the equivalent \is{active voice} active sentence) has been promoted to the role of subject. The difference in \isi{word order} between \isi{active-voice} and \isi{passive-voice} clauses is presented in \REF{ex:syntax:275a}.

\ea%275a
    \label{ex:syntax:275a}
            Basic \isi{constituent order} of active and passive clauses
\is{basic constituent order}
\begin{tabbing}
{(Passive-\isi{voice} clauses (\isi{intransitive}):)} \= {(SOV)} \= {((APV))}\kill
{\isi{Intransitive} \isi{active-voice} clauses:} \> {SV} \> {}\\
{\isi{Transitive} \isi{active-voice} clauses:} \> {SOV} \> ({APV)}\\
{\isi{Passive-voice} clauses (\isi{intransitive}):} \> {VS} \> {(VP)}
\end{tabbing}
\z

The verb in the \isi{passive} clause generally requires the \isi{suffix} -\textit{e} ‘\textsc{dep}’.\footnote{The form [-e] serves several functions in Ulwa. Although it is glossed here as ‘\textsc{dep}’ (i.e., “\isi{dependent}”), these \isi{passive} sentences are analyzed as being \isi{independent}, since they serve as complete sentences, without needing any additional clause, stated or implied.} Although not functioning synchronically as \isi{dependent clause}s, these \isi{passive} sentences may have developed diachronically from a type of \isi{dependent clause}, namely, \isi{relative clause}s. Since \isi{relative clause}s in Ulwa employ a \is{gap strategy} gapping strategy, they leave themselves open to \isi{reanalysis} as \isi{head-internal relative clause}s with \isi{inverted word order} (\sectref{sec:12.3}). For example, a set of \isi{matrix clause} and \isi{relative clause} as in \REF{ex:syntax:276} (with the subject “gapped”) could be \isi{reanalyzed} as the structure presented in \REF{ex:syntax:277} (with \isi{inverted word order}).

\is{embedded clause}
\is{head-internal relative clause}

\ea%276
    \label{ex:syntax:276}
          Gapping strategy analysis of relative clauses\\
    {[S[\_\textsubscript{i}OV]O\textsubscript{i}V]\\
    the gap “\_” represents the S of the embedded \isi{relative clause}\\
    the gap is co-indexed with the O (external \isi{head}) of the \isi{matrix clause}}
\z

\ea%277
    \label{ex:syntax:277}
          Head-internal analysis of relative clauses\\
    {[S[OVS/O]V]\\
    “S/O” is the S of the \isi{relative clause}\\
    “S/O” is also the O (internal \isi{head}) of the \isi{matrix clause}}
\z

Perhaps examples such as \REF{ex:syntax:273} and \REF{ex:syntax:275} could be analyzed as \isi{relative clause}s. For example, sentence \REF{ex:syntax:275} could actually mean something like ‘the yams that were cooked’. However, since these examples are all fully capable of serving as \isi{independent} sentences, not \isi{dependent} on any other clause, they are analyzed here as indeed \isi{passive} sentences and not as \isi{relative clause}s.

The \isi{suffix} \textit{{}-e} ‘\textsc{dep}’ does not appear, at least not overtly, in \isi{irrealis}-\isi{mood} passives. Since the \isi{irrealis} \isi{suffix} invariably ends in [-a], and a \isi{phonological} rule would syncopate a following /e/, it is, however, possible that there is an underlying \isi{suffix} /-e/ even in these \isi{irrealis}-\isi{mood} passives. Sentence \REF{ex:syntax:278} provides an example of an \isi{irrealis}-\isi{mood} \isi{passive} sentence.

\ea%278
    \label{ex:syntax:278}
          \textit{Umbe} \textbf{\textit{walinda}} \textit{lamndu.}\\
\gll    umbe    wali-\textbf{nda}  lamndu\\
    tomorrow  hit-\textsc{irr}    pig\\
\glt `The pig will be killed tomorrow.’ [elicited]
\z

When \isi{passive} clauses contain \isi{discontinuous} verb forms (\sectref{sec:9.2.1}), the entire verbal unit occurs prenominally, as in \REF{ex:syntax:279}.

\ea%279
    \label{ex:syntax:279}
          \textit{\textbf{Lïmndï ale} ankam.}\\
\gll    \textbf{lïmndï}  \textbf{ala-e}    ankam\\
    eye    see{}-\textsc{dep}  person\\
\glt `The man was seen.’ [elicited]
\z

Sentences \REF{ex:syntax:280} and \REF{ex:syntax:281}, which are examples of simple \isi{passive} sentences taken from texts, reveal some of the \isi{pragmatic} functions of \isi{passive} sentences. In sentence \REF{ex:syntax:280}, the speaker is introducing a new topic and placing emphasis on the action (the killing of pigs) and not on the \isi{agent}s of this action. In sentence \REF{ex:syntax:281}, the role of the \isi{agent} (the people who eat food in the dry season) is negligible; rather, the important point is -- quite impersonally -- that the dry season is a time when there is plenty food.

\ea%280
    \label{ex:syntax:280}
          \textit{Asape nungol!}\\
\gll    asa-p-e      nungol\\
    hit-\textsc{pfv-dep}  child\\
\glt `Piglets were killed!’ (Literally ‘Children [i.e., offspring of pigs] were killed!’) [ulwa014\_47:11]
\z

\ea%281
    \label{ex:syntax:281}
          \textit{Ane se ame mundu.}\\
\gll    ane  sa-e    ama-e    mundu\\
    sun  cry-\textsc{dep}  eat-\textsc{dep}  food\\
\glt `When the sun is shining, food is eaten.’ (i.e., the dry season is a good time for finding   food) [ulwa041\_02:10]
\z

\is{syntax|)}
\is{voice|)}
\is{passivization|)}
\is{passive voice|)}

\is{passive voice|(}
\is{passivization|(}
\is{voice|(}
\is{syntax|(}

Although the \isi{agent} of a \isi{passive} sentence need not be expressed, it can be included as an \isi{oblique} \isi{phrase}. In \is{active voice} active sentences, \isi{oblique}s (such as \isi{temporal adverb}s) occur either in clause-initial position or immediately before the \isi{verb phrase} -- that is, before the verb in \isi{intransitive} clauses, and before the object of the verb in \isi{transitive} clauses. Likewise, the \isi{agent} \isi{oblique} \isi{phrase}, if included, appears at the beginning of the \isi{passive} clause, immediately before the verb. The \isi{oblique marker} \textit{=n} ‘\textsc{obl}’ (\sectref{sec:11.4.1}) is used to identify the \isi{agent} of \isi{passive} verbs, as in sentences \REF{ex:syntax:282}, \REF{ex:syntax:283}, and \REF{ex:syntax:284}.

\ea%282
    \label{ex:syntax:282}
          \textbf{\textit{Ankamnï}} \textit{toplïpe mana.}\\
\gll    \textbf{ankam=nï}    top-lï-p-e        mana\\
    person=\textsc{obl}  throw-put-\textsc{pfv-dep}  spear\\
\glt `The spear was thrown by the man.’ [elicited]
\z

\ea%283
    \label{ex:syntax:283}
          \textbf{\textit{Ndïn}} \textit{asape lamndu.}\\
\gll    \textbf{ndï=n}    asa-p-e      lamndu\\
    3\textsc{pl=obl}  hit-\textsc{pfv-dep}  pig\\
\glt `The pig was killed by them.’ [elicited]
\z

\ea%284
    \label{ex:syntax:284}
          \textbf{\textit{Nungolnï}} \textit{lukawtim mape nga.}\\
\gll    \textbf{nungol=nï}    lukawtim  ma=p-e      nga\\
    child=\textsc{obl}    look.after  \textsc{3sg.obj=}be\textsc{{}-dep}  \textsc{sg.prox}\\
\glt `This one was looked after by [my] son.’ (\textit{lukawtim} = TP \textit{lukautim}) [ulwa014\_03:20]
\z

It should be noted that \isi{passive} sentences are very rare in Ulwa discourse. I suspect that they are being lost as the language experiences \isi{grammatical attrition} in the face of \isi{obsolescence} (see \chapref{sec:15}). In many situations in which one might expect to find a \isi{passive} construction (i.e., situations in which the role of the \isi{agent} of a \isi{transitive} sentence is to be downplayed), alternative structures are often used. For example, some speakers use \isi{impersonal construction}s. Since a pronominal subject can be omitted, it is possible to say something along the lines of ‘[they] did something’, in which the non-specific subject ‘they’ is unstated altogether, as in \REF{ex:syntax:285} and \REF{ex:syntax:286}.

\ea%285
    \label{ex:syntax:285}
          \textit{Nip malpe.}\\
\gll    ni-p    ma=lï{}-p-e\\
    die-\textsc{pfv}  3\textsc{sg.obj}=put-\textsc{pfv-dep}\\
\glt `[He] died and [they] buried him.’ [ulwa037\_45:52]
\z

\ea%286
    \label{ex:syntax:286}
          \textit{Lungum anda matï Tapon nana.}\\
\gll    lungum  anda    ma=tï      Tapon  na-na\\
    long.spear  \textsc{sg.dist}  \textsc{3sg.obj=}take  [name]  give-\textsc{pfv}\\
\glt `[They] gave a long spear to Tapon.’ [ulwa003\_00:32]
\z

Although often not necessary to convey information, \isi{passive} clauses fulfill a very useful role in discourse, since they enable certain \isi{relative clause} constructions that would otherwise be impossible. In \isi{relative clause} constructions, only the subject argument is accessible to being \isi{relativize}d (\sectref{sec:12.3}). An \isi{antecedent} \isi{noun phrase} cannot serve as the \isi{direct object} of the \isi{relative clause}. Therefore, although it would be possible directly to translate into Ulwa a sentence like \mbox{‘Ginam} saw the man who killed the pig’, it would not be possible directly to translate into Ulwa a sentence like ‘Ginam saw the pig that the man killed’.

  Passivization, however, which can promote a \isi{direct object} to subject, provides a means for conveying the meaning of a sentence like ‘Ginam saw the pig that the man killed’, changing the sentence, as it were, to a sentence like ‘Ginam saw the pig that was killed by the man.’ The sentence in Ulwa would appear as in \REF{ex:syntax:287}.

\ea%287
    \label{ex:syntax:287}
          \textit{Ginam lïmndï ankamnï asape lamndu mala.}\\
\gll    Ginam  lïmndï  [ankam=nï    asa-p-e]    lamndu  ma=ala\\
    [name]  eye    [person=\textsc{obl}  hit-\textsc{pfv-dep]}  pig      3\textsc{sg.obj}=see\\
\glt `Ginam saw the pig that the man killed.’ (Literally ‘Ginam saw the pig that was killed by the man.’) [elicited]
\z

It is a rather straightforward process to have the \isi{head noun} in an \isi{independent clause}, such as \textit{lam} ‘meat’ in \REF{ex:syntax:288} or \REF{ex:syntax:289}, function as the subject of a \isi{relative clause}, as in \REF{ex:syntax:290}.

\ea%288
    \label{ex:syntax:288}
          \textit{Inom mï \textbf{lam} mawanap.}\\
\gll    inom  mï      \textbf{lam}  ma=wana-p\\
    mother  3\textsc{sg.subj}  meat  3\textsc{sg.obj}=cook-\textsc{pfv}\\
\glt `Mother cooked the meat.’ [elicited]
\z

\ea%289
    \label{ex:syntax:289}
          \textit{\textbf{Lam} mï nungol masap.}\\
\gll    \textbf{lam}  mï      nungol  ma=asa-p\\
    meat  3\textsc{sg.subj}  child  3\textsc{sg.obj}=hit-\textsc{pfv}\\
\glt `The meat killed the child.’ (i.e., it poisoned him and he died) [elicited]
\z

\ea%290
    \label{ex:syntax:290}
          \textit{Inom mï nungol masape \textbf{lam} mawanap.}\\
\gll    inom  mï      [nungol  ma=asa-p-e]      \textbf{lam} ma=wana-p\\
    mother  3\textsc{sg.subj}  [child    3\textsc{sg.obj}=hit-\textsc{pfv-dep]}  meat     \textsc{3sg.obj}=cook-\textsc{pfv}\\
\glt `Mother cooked the meat that killed the child.’ [elicited]
\z

It is not, however, possible, for the \isi{head noun} in the \isi{matrix clause} to correspond to a \isi{direct object} in the \isi{relative clause}. Thus, \textit{lam} ‘meat’ in \REF{ex:syntax:291} cannot be \isi{relativize}d. However, a comparable meaning can be conveyed by using a \isi{passive} construction in the \isi{relative clause}, as in \REF{ex:syntax:292}.

\ea%291
    \label{ex:syntax:291}
          \textit{Nungol mï \textbf{lam} mamap.}\\
\gll    nungol  mï      \textbf{lam}  ma=ama-p\\
    [name]  3\textsc{sg.subj}  meat  \textsc{3sg.obj}=eat-\textsc{pfv}\\
\glt `The child ate the meat.’ [elicited]
\z

\ea%292
    \label{ex:syntax:292}
          \textit{Inom mï nungolnï amape \textbf{lam} mawanap.}\\
\gll    inom  mï      [nungol=nï  ama-p-e]    \textbf{lam}     ma=wana-p\\
    mother  \textsc{3sg.subj}  [child=\textsc{obl}  eat-\textsc{pfv-dep]}  meat    \textsc{3sg.obj}=cook-\textsc{pfv}\\
\glt `Mother cooked the meat that the child ate.’ (Literally ‘Mother cooked the meat that was eaten by the child.’) [elicited]
\z

It is perhaps due to this usefulness that the \isi{passive} \isi{voice} does still appear in discourse, often in other complex constructions that employ \isi{relative clause}s, like in \REF{ex:syntax:293}.

\is{syntax|)}
\is{voice|)}
\is{passivization|)}
\is{passive voice|)}

\ea%293
    \label{ex:syntax:293}
          \textit{U ko nananï nïwat lape mïnda ngawonp.}\\
\gll    u    ko  nana=nï    nï=wat    la-p-e      mïnda nga=won-p-e\\
    2\textsc{sg}  just  mama=\textsc{obl}  1\textsc{sg=}atop  plant-\textsc{pfv-dep}  banana    \textsc{sg.prox}=cut-\textsc{pfv-dep}\\
\glt `You just cut this banana tree that was planted above me by mama.’ [ulwa001\_02:32]
\z

\section{Valency reduction and decreased transitivity}\label{sec:13.8}

\is{valency reduction|(}
\is{valency|(}
\is{decreased transitivity|(}
\is{transitivity|(}
\is{syntax|(}

\isi{Passive} sentences (\sectref{sec:13.7}) can be thought of as reducing the \isi{valency} of a verb. Since their \is{active voice} active, \isi{transitive} equivalents have two \isi{core argument}s (a subject and a \isi{direct object}), whereas they themselves have only one argument (a \isi{patient}-like subject), the \isi{valency} of the verb is considered to be decreased. This section is concerned with other means of reducing \isi{valency} or decreasing \isi{transitivity}.

\is{syntax|)}
\is{transitivity|)}
\is{decreased transitivity|)}
\is{valency|)}
\is{valency reduction|)}

\subsection{Transitivity classes of verbs}\label{sec:13.8.1}

\is{valency reduction|(}
\is{valency|(}
\is{decreased transitivity|(}
\is{transitivity|(}
\is{syntax|(}
\is{transitivity class|(}

  There are no formal properties, such as \isi{phonological} differences, by which \isi{transitivity} classes may be differentiated in Ulwa. That is, there are no formal distinctions between verbs that typically exhibit \isi{semantic} properties associated with high \isi{transitivity} and verbs that typically exhibit \isi{semantic} properties associated with low \isi{transitivity}.\footnote{I follow here a scalar definition of “\isi{transitivity}”, based on \isi{semantic} properties including \isi{telicity}, \isi{punctuality}, \isi{volitionality} of the \isi{agent}, and \isi{affectedness} of the \isi{patient}, among others (\citealt{HopperThompson1980}). Therefore, \isi{semantic}ally, \isi{transitivity} is taken to be a property of an entire clause. Syntactically, on the other hand, I understand a \isi{transitive} clause (as opposed to an \linebreak \isi{intransitive} clause) as being “a construction with two syntactically privileged arguments” \citep[6]{Næss2007}.}

In examining the degree to which verbs may be \isi{ambitransitive} in Ulwa, we may consider both \isi{A-lability} and \isi{P-lability}.\footnote{Here I follow, for example, \citet[109]{Kibrik1996}, in differentiating two varieties of \isi{ambitransitivity} or \isi{lability}. In \isi{A-lability} (i.e., \isi{agent}-preserving \isi{lability}), the S argument of the \isi{intransitive} use of the verb is \isi{semantic}ally the same as the A argument of the \isi{transitive}. In \isi{P-lability} (i.e., \isi{patient}-preserving \isi{lability}), on the other hand, the S argument of the \isi{intransitive} use of the verb is \isi{semantic}ally the same as the P argument of the \isi{transitive}.}

 In considering \isi{A-lability}, we may note that it seems that essentially any verb in Ulwa may occur without an expressed object. Although this could be taken to suggest that there exist no truly \isi{transitive} verbs in Ulwa, an alternative view would be that at least some instances in which a putatively \isi{ambitransitive} verb occurs without an overtly expressed object are examples of \is{object deletion} \isi{indefinite object deletion} (\citealt[124--125]{Næss2007}). For example, the verb \textit{ama-} ‘eat’ may occur either with \REF{ex:syntax:294} or without \REF{ex:syntax:295} an expressed object; when no object is stated, however, it could be argued that there is an \isi{indefinite} object (i.e., ‘food’) that has simply been deleted.\footnote{Crosslinguistically, the verb ‘eat’ often behaves idiosyncratically in terms of \isi{transitivity} \citep{Næss2009}. Thus, it should perhaps not be taken as indicating too much about \isi{transitivity} in the language in general. Nevertheless, Ulwa may here be contrasted with other \ili{Keram} languages, such as \ili{Mwakai} and \ili{Ap Ma}, in which the \isi{semantic}ally light noun \textit{si} ‘things’ (in both languages) is commonly used as the object of the verb ‘eat’ when no particular food is being specified. Thus, in contrast to its sister languages, Ulwa -- or at least the \ili{Manu} \isi{dialect} of Ulwa -- seems to have an \isi{ambitransitive} verb meaning ‘eat’. In the \ili{Maruat-Dimiri-Yaul} \isi{dialect}, on the other hand, ‘eat’ generally takes the object \textit{mundï} ‘food’, when no particular food is being specified. Although contrasts such as the one exemplified by \REF{ex:syntax:294} and \REF{ex:syntax:295} suggest the possible \isi{ambitransitivity} (namely, \isi{A-lability}) of the verb ‘eat’, it is common in the \ili{Manu} \isi{dialect} for the \isi{detransitivizing} \isi{prefix} \textit{na-} \textsc{‘detr’} to occur with ‘eat’ when no particular food is being specified (\sectref{sec:13.8.2}).}

\newpage

\ea%294
    \label{ex:syntax:294}
          \textit{Nï ta lamndu amap.}\\
\gll    nï    ta      lamndu  ama-p\\
    1\textsc{sg}  already    pig      eat-\textsc{pfv}\\
\glt `I have already eaten pork.’ [elicited]
\z


\ea%295
    \label{ex:syntax:295}
          \textit{Nï ta amap.}\\
\gll    nï    ta      ama-p\\
    1\textsc{sg}  already    eat-\textsc{pfv}\\
\glt `I have already eaten.’ [elicited]
\z

  Next, we may consider \isi{P-lability}. Ulwa does not have a class of \isi{patient}-\linebreak preserving \is{lability} labile verbs. Verbal notions such as ‘break’ and ‘burn’, which are crosslinguistically more likely to be expressed with P-labile verbs, occur in Ulwa as pairs of unrelated verbs (i.e., ‘break [\isi{transitive}]’ vs. ‘break [\isi{intransitive}]’ and ‘burn [\isi{transitive}]’ vs. ‘burn [\isi{intransitive}]’). In the terms of \citet{NicholsEtAl2004}, Ulwa could be said to have an \is{indeterminate valence orientation} “indeterminate” \isi{valence orientation}, since the correspondences between “plain” and “induced” verbs are generally of the \isi{suppletive} variety \REF{ex:syntax:295a}.

  \is{plain verb}
  \is{induced verb}
  
  \ea%295a
    \label{ex:syntax:295a}
    Underived pairs of verbs with \isi{transitive} versus \isi{intransitive} meanings\\
    \begin{tabbing}
{(\textit{apïn ama-})} \= {(‘break (cause to break)’)} \= {(\textit{lïmndï ala-})} \= {(‘die’)}\kill
\textit{asa-} \> ‘kill’ \> \textit{ni-} \> ‘die’\\
\textit{na-} \> ‘feed (give food)’ \> \textit{ama-} \> ‘eat’\\
\textit{=n ul si-} \> ‘show’ \> \textit{lïmndï ala-} \> ‘see’\\
\textit{apïn ama-} \> ‘burn (set fire)’ \> \textit{wo-} \> ‘burn (catch fire)’\\
\textit{a-} \> ‘break (cause to break)’ \> \textit{nungun u-} \> ‘break (get broken)’
    \end{tabbing}
\z

  There are, for example, several ways of encoding ‘breaking’ events in Ulwa, but none of them instantiates \isi{P-lability}. The verbs \textit{a-} ‘break’, \textit{kol-} ‘break, split’, \textit{kot-} ‘break, bear’, and \textit{kun-} ‘break, break off’ are all \isi{transitive}, with the P argument referring to that which is broken; the verbs \textit{nungun u-} ‘break (\isi{intransitive})’ and \textit{tukul-} ‘break (\isi{intransitive})’, on the other hand, are both \isi{intransitive}, with the S argument referring to that which is broken. Sentence \REF{ex:syntax:296} exemplifies the use of the \isi{intransitive} verb \textit{tukul-} ‘break’.

\ea%296
    \label{ex:syntax:296}
          \textit{Amun nïnji maka palapal min mï maka \textbf{tukulp}.}\\
\gll    amun  nï-nji    maka  palapal      min  mï    maka     \textbf{tukul}-p\\
    now  1\textsc{sg-poss}  thus  decoration?  band  3\textsc{sg.subj}  thus    break-\textsc{pfv}\\
\glt `Now, my, like, shell armband has broken like this.’ (\textit{palapal} < TP \textit{balbal} {\textasciitilde} \textit{palpal} ‘Indian coral tree’?) [ulwa015\_00:59]
\z

Example \REF{ex:syntax:296} may be compared with example \REF{ex:otherwc:105} in \sectref{sec:8.2.3}, which illustrates the use of the \isi{intransitive} verb \textit{nungun u-} ‘break’.

  Although \textit{tukul-} ‘break’ may be etymologically related to \textit{kol-} ‘break’, \textit{kot-} ‘break’, or \textit{kun-} ‘break’, it is not synchronically derived by any process (i.e., there is no known \isi{prefix} \textsuperscript{†}/tu-/). At any rate, the \isi{transitive} verb \textit{a-} ‘break’ is not related to any of these forms, whether diachronically or synchronically. An example of this \isi{transitive} verb is given in \REF{ex:syntax:297}, which may be compared with example \REF{ex:det:133} in \sectref{sec:7.3} and example \REF{ex:det:199} in \sectref{sec:7.5}, which contain similar uses of this verb.

\ea%297
    \label{ex:syntax:297}
          \textit{May mïnkïn \textbf{map}}\\
\gll    ma=i        mïnkïn    ma=\textbf{a}-p\\
    3\textsc{sg.obj}=go.\textsc{pfv}  sago.species  3\textsc{sg.obj}=break-\textsc{pfv}\\
\glt `[We] went there and broke a sago palm.’ [ulwa037\_03:43]
\z

The verb \textit{kot-} ‘break’ often seems to exhibit \isi{ambitransitivity}, namely when it has the secondary meaning ‘bear, give birth’. This, however, is indicative of \isi{A-lability}, not \is{lability} \isi{P-lability}.

  Like ‘break’, the verbal notion of ‘burn’ in Ulwa is encoded with different verbs for \isi{transitive} and \isi{intransitive} meanings. The \isi{intransitive} verb is \textit{wo-} ‘burn, blaze’, which does not take an object. \isi{Transitive} constructions, on the other hand, are formed with the \isi{phrase} \textit{apïn=n ama-} ‘eat with fire’, with the A argument referring to the person setting fire to something and the P argument referring to that which is burned (\textit{apïn} ‘fire’ occurs in an \isi{oblique} \isi{phrase}) \REF{ex:syntax:297a}.

  \ea%297a
    \label{ex:syntax:297a}
          \textit{Nï \textbf{apïn} im \textbf{ngamap}.}\\
\gll    nï \textbf{apïn=n} im nga=\textbf{ama}-p\\
    1\textsc{sg} fire=\textsc{obl} tree \textsc{sg.prox}=eat-\textsc{pfv}\\
\glt `I burned this tree.’ (Literally ‘I ate this tree with fire.’) [elicited]
\z

  Alternatively, \textit{apïn} ‘fire’ can itself be the A argument (i.e., subject), with the P argument referring to that which is burned (i.e., ‘fire eats [something]’); since there is no \isi{animate} \isi{agent} here (the fire is the subject), this construction has, at least \isi{semantic}ally, something like an \isi{intransitive} meaning, although syntactically it is \isi{transitive} \REF{ex:syntax:297b}.

  \ea%297b
    \label{ex:syntax:297b}
          \textit{\textbf{Apïn} mï amun im \textbf{ndame}.}\\
\gll   \textbf{apïn} mï amun im ndï=\textbf{ama}-e\\
    fire 3\textsc{sg.subj} now tree 3\textsc{pl}=eat-\textsc{ipfv}\\
\glt `The trees are burning now.’ (Literally ‘Fire is now eating the trees.’) [elicited]
\z
  
  Finally, although potentially any verb may be used without an overtly stated object, there are some prototypically low-\isi{transitivity} verbs that never occur with an object, such as ‘die’. It would thus not be unreasonable to consider such verbs to be \isi{intransitive}. There is no class of \isi{ditransitive} verbs in Ulwa (\sectref{sec:11.3}).

\is{transitivity class|)}
\is{syntax|)}
\is{transitivity|)}
\is{decreased transitivity|)}
\is{valency|)}
\is{valency reduction|)}

\subsection{The detransitivizing prefix \textit{na-} ‘\textsc{detr}’}\label{sec:13.8.2}

\is{valency reduction|(}
\is{valency|(}
\is{decreased transitivity|(}
\is{transitivity|(}
\is{syntax|(}
\is{detransitivization|(}
\is{detransitivizer|(}

There is an important \isi{bound morpheme} in Ulwa that serves a number of grammatical functions, often with nuances that are difficult to explain, but whose basic function seems to be to reduce the \isi{transitivity} of verbs \citep{Barlow2019b}. This is the verbal \isi{prefix} \textit{na-} ‘\textsc{detr’} (i.e., “\isi{detransitivizer}”). Reasons for treating the detransitivizing form \textit{na-} \textsc{‘detr’} as a \isi{prefix} rather than a \isi{clitic} include the fact that it only occurs before verbs and the fact that \isi{object-marker} \isi{proclitic}s may precede it. It may be seen in sentence \REF{ex:syntax:297c}.

\ea%297c
    \label{ex:syntax:297c}
          \textit{Ndï \textbf{naytap}.}\\
\gll    ndï \textbf{na}-ita-p\\
    3\textsc{pl} \textsc{detr}-build-\textsc{pfv}\\
\glt `They built [something].’ [ulwa014\_26:43]
\z

  In Ulwa, there are not strong distributional or structural differences between what may be thought of as \isi{transitive} and \isi{intransitive} verbs. Many verbs with meanings that are often considered prototypically \isi{intransitive} can, in Ulwa, have \isi{direct object}s and, as such, may be marked with \isi{object-marker} \isi{proclitic}s. For example, the verb \textit{ma-} {\textasciitilde} \textit{i-} ‘go’ can function simply as an \isi{intransitive} verb, requiring no object \REF{ex:syntax:298}.

  \ea%298
    \label{ex:syntax:298}
          \textit{Nï} \textbf{\textit{i}}.\\
\gll nï    \textbf{i}\\
    1\textsc{sg}  go.\textsc{pfv}\\
\glt `I went.’ [ulwa014\_12:42]
\z

  As an \isi{intransitive} verb, \textit{ma-} {\textasciitilde} \textit{i-} ‘go’ may accept a \isi{postpositional phrase} to demarcate a \isi{goal} argument \REF{ex:syntax:299}.

  \ea%299
    \label{ex:syntax:299}
          \textit{Nï \textbf{ndiya i}}.\\
\gll nï    ndï=\textbf{iya}    \textbf{i}\\
    1\textsc{sg}  3\textsc{pl}=toward  go.\textsc{pfv}\\
\glt `I went to them.’ [ulwa014\_05:39]
\z

  Alternatively, \textit{ma-} {\textasciitilde} \textit{i-} ‘go’ can function as a \isi{transitive} verb, with the \isi{goal} argument as its object. As an object, the \isi{goal} argument can, accordingly, receive \isi{object marking} \REF{ex:syntax:300}.


\ea%300
    \label{ex:syntax:300}
          \textit{Nï Kumba} \textbf{\textit{may}}.\\
\gll nï    Kumba  \textbf{ma=i}\\
    1\textsc{sg} Bun  3\textsc{sg.obj}=go.\textsc{pfv}\\
\glt `I went to Bun [village].’ [ulwa032\_00:16]
\z
  
  As a \isi{transitive} verb, \textit{ma-} {\textasciitilde} \textit{i-} ‘go’ may take an object even without an \isi{object-marker} \isi{proclitic} \REF{ex:syntax:301}.

\ea%301
    \label{ex:syntax:301}
          \textit{Mï i wandam} \textbf{\textit{iye}}.\\
\gll mï      i    wandam  \textbf{i}{}-e\\
    3\textsc{sg.subj}  go.\textsc{pfv}  jungle    go.\textsc{pfv-dep}\\
\glt `She went, went to the jungle.’ [ulwa001\_00:42]
\z

Furthermore, as a \isi{transitive} verb, \textit{ma-} {\textasciitilde} \textit{i-} ‘go’ can even have both a \isi{direct object} and a \isi{postpositional phrase} marking an additional destination (i.e., \isi{goal}), as in \REF{ex:syntax:302}. Thus the verb \textit{ma-} {\textasciitilde} \textit{i-} ‘go’ may be considered \isi{transitive} (or at least capable of being \isi{transitive}), taking as its \isi{direct object} a \isi{goal} argument. Even when there is no expressed object, the claim can be made that the verb is still \isi{transitive}, only that the \isi{direct object} has been left unexpressed.

\ea%302
    \label{ex:syntax:302}
          \textit{Nï} \textbf{\textit{maya}} \textit{wa} \textbf{\textit{may}}.\\
\gll nï    ma=\textbf{iya}      wa    \textbf{ma=i}\\
    1\textsc{sg}  3\textsc{sg.obj}=toward  village  3\textsc{sg.obj}=go.\textsc{pfv}\\
\glt `I went to him in the village.’ [ulwa037\_37:38]
\z

  Although it is possible for the verb \textit{ma-} {\textasciitilde} \textit{i-} ‘go’ to function as an \isi{intransitive} verb without any special marking, it very commonly receives the detransitivizing \isi{prefix} \textit{na-} \textsc{‘detr’}, which seems to serve the primary purpose of reducing \isi{transitivity}, in this case changing the verb’s meaning from something perhaps better glossed as ‘go to’ to something meaning simply ‘go’, as in \REF{ex:syntax:303} and \REF{ex:syntax:304}.

\ea%303
    \label{ex:syntax:303}
          \textit{Ndï} \textbf{\textit{nay}}.\\
\gll ndï  \textbf{na}{}-i\\
    3\textsc{pl}  \textsc{detr-}go.\textsc{pfv}\\
\glt `They went.’ [ulwa032\_31:28]
\z

\ea%304
    \label{ex:syntax:304}
          \textit{Mangusuwata} \textbf{\textit{namana}}.\\
\gll ma-ngusuwata  \textbf{na}{}-ma-na\\
    3\textsc{sg.obj-}poor  \textsc{detr-}go-\textsc{irr}\\
\glt `The poor thing will be going.’ [ulwa037\_59:19]
\z

This same \isi{prefix} is seen on the verb ‘go’ also when a single argument is expressed in a \isi{postpositional phrase} (i.e., the \isi{goal} is not expressed as the \isi{direct object} of the verb), as in \REF{ex:syntax:305}.

\ea%305
    \label{ex:syntax:305}
          \textit{Mï maya} \textbf{\textit{nay}}.\\
\gll mï      ma=iya      \textbf{na}{}-i\\
    3\textsc{sg.subj}  3\textsc{sg.obj}=toward  \textsc{detr-}go.\textsc{pfv}\\
\glt `He went to her.’ [ulwa009\_02:26]
\z

The same \isi{prefix} \textit{na-} ‘\textsc{detr’} can occur with other verbs, also marking them as \isi{intransitive}. In example \REF{ex:syntax:306}, the verb \textit{ama-} {\textasciitilde} \textit{la-} ‘eat’ is \isi{transitive} (i.e., it has an overt object). This sentence may be compared with examples \REF{ex:syntax:307} and \REF{ex:syntax:308}, in which the same verb is \isi{intransitive}, in these examples being marked with the \isi{prefix} \textit{na-} \textsc{‘detr’}.

\ea%306
    \label{ex:syntax:306}
          \textit{Tïn mï} \textbf{\textit{utam}} \textbf{\textit{mamap}}.\\
\gll tïn    mï      \textbf{utam}  \textbf{ma=}ama-p\\
    dog  3\textsc{sg.subj}  yam  3\textsc{sg.obj}=eat-\textsc{pfv}\\
\glt `The dog ate the yam.’ [elicited]
\z

\ea%307
    \label{ex:syntax:307}
          \textit{Nï ta} \textbf{\textit{namap}}.\\
\gll nï    ta      \textbf{na}{}-ama-p\\
    1\textsc{sg}  already    \textsc{detr}{}-eat-\textsc{pfv}\\
\glt `I’ve already eaten.’ [elicited]
\z

\ea%308
    \label{ex:syntax:308}
          \textit{Ndul} \textbf{\textit{nalanda}}\textit{!}\\
\gll    ndï=ul    \textbf{na}{}-la-nda\\
    3\textsc{pl}=with  \textsc{detr-}eat-\textsc{irr}\\
\glt `[Let’s] eat with them!’ [ulwa029\_04:11]
\z

\is{nominative-accusative alignment}
\is{accusative alignment}

As a verbal \isi{affix} that allows an otherwise \isi{transitive} verb to lose its \isi{direct object} argument, the \isi{prefix} \textit{na-} \textsc{‘detr’} could theoretically be described as an \isi{antipassive} morpheme, even though this term is not commonly used in descriptions of languages with nominative-accusative \isi{morphosyntactic alignment}, such as Ulwa.\footnote{See \citet[149--163]{Heaton2017} for discussion of \isi{antipassive}s in nominative-accusative \is{nominative-accusative alignment} languages.}

The morpheme \textit{na-} \textsc{‘detr’} is often better described as reducing the \isi{transitivity} of a verb rather than changing its \isi{valency}. In example \REF{ex:syntax:309}, the object of the verb is technically the \isi{question word} \textit{angos} ‘what?’. However, given the fact that the event that the verb is encoding is far from being prototypically \isi{transitive} (i.e., the situation is non-punctual, \isi{irrealis}, etc.), it is not surprising that the detransitivizing \isi{prefix} \textit{na-} \textsc{‘detr’} is employed.

\ea%309
    \label{ex:syntax:309}
          \textit{Una angos \textbf{nalanda}?}\\
\gll unan    angos  \textbf{na}{}-la-nda\\
    1\textsc{pl.incl}  what  \textsc{detr}{}-eat-\textsc{irr}\\
\glt `What shall we eat?’ [ulwa040\_00:46]
\z

Examples \REF{ex:syntax:310} and \REF{ex:syntax:311} illustrate how the verb \textit{ita-} ‘build’ can likewise be detransitivized with the \isi{prefix} \textit{na-} \textsc{‘detr’}. In example \REF{ex:syntax:310}, the verb is \isi{transitive} and receives the \isi{object marker} \textit{ma=} ‘3\textsc{sg.obj}’; in example \REF{ex:syntax:311}, the verb is \isi{intransitive}, and receives the detransitivizing \isi{prefix} \textit{na-} \textsc{‘detr’}.

\is{detransitivizer|)}
\is{detransitivization|)}
\is{syntax|)}
\is{transitivity|)}
\is{decreased transitivity|)}
\is{valency|)}
\is{valency reduction|)}
\is{valency reduction|(}
\is{valency|(}
\is{decreased transitivity|(}
\is{transitivity|(}
\is{syntax|(}
\is{detransitivization|(}
\is{detransitivizer|(}

\ea%310
    \label{ex:syntax:310}
          \textit{Mï \textbf{wat maytap}.}\\
\gll mï      \textbf{wat}  \textbf{ma}=ita-p\\
    3\textsc{sg.subj}  ladder  3\textsc{sg.obj}=build-\textsc{pfv}\\
\glt `He built the ladder.’ [ulwa001\_10:24]
\z

\ea%311
    \label{ex:syntax:311}
          \textit{Mï} \textbf{\textit{naytap}}.\\
\gll mï      \textbf{na}{}-ita-p\\
    3\textsc{sg.subj}  \textsc{detr-}build-\textsc{pfv}\\
\glt `He built [something].’ [ulwa014\_25:01]
\z

Verbs glossed as ‘put’ in Ulwa, which take as their \isi{direct object} a \isi{goal} argument, are also commonly marked with the \isi{prefix} \textit{na-} \textsc{‘detr’}, either when there is no specific \isi{goal} or when the speaker does not wish to include a \isi{goal} argument, as in examples \REF{ex:syntax:312}, \REF{ex:syntax:313}, and \REF{ex:syntax:314}.


\ea%312
    \label{ex:syntax:312}
          \textit{I ndïn} \textbf{\textit{nop}}.\\
\gll i    ndï=n    \textbf{na}{}-u-p\\
    go.\textsc{pfv}  3\textsc{pl=obl}  \textsc{detr}{}-put-\textsc{pfv}\\
\glt `[They] went and planted them [somewhere].’ [ulwa032\_31:25]
\z

\newpage

\ea%313
    \label{ex:syntax:313}
          \textbf{\textit{Nay}} \textit{mat} \textbf{\textit{nalp}} \textit{mat wapa nduwatlïpe.}\\
\gll    \textbf{na}{}-i      ma=tï      \textbf{na}{}-lï{}-p      ma=tï      wap ndï=wat-lï-p-e\\
    \textsc{detr}{}-go.\textsc{pfv}  3\textsc{sg.obj}=take  \textsc{detr-}put-\textsc{pfv}  \textsc{3sg.obj=}take  leaf          \textsc{3pl}=atop-put-\textsc{pfv-dep}\\
\glt `[They] came, took him, put him [somewhere], put him on the leaves.’ [ulwa001\_11:04]
\z

\ea%314
    \label{ex:syntax:314}
          \textit{Ndï namlipe mï ndït anmbï} \textbf{\textit{nalpe}}.\\
\gll ndï  namli=p-e    mï      ndï=tï    an-mbï     \textbf{na}{}-lï{}-p-e\\
    3\textsc{pl}  soft=\textsc{cop-dep}  \textsc{3sg.subj}  \textsc{3pl}=take  out-here    \textsc{detr}{}-put-\textsc{pfv-dep}\\
\glt `When they were soft, she took them out.’ [ulwa013\_02:29]
\z

This \isi{prefix} may also be used when these ‘put’ verbs are used as the second element of verbal compounds, also with the effect of downplaying the \isi{direct object} (\isi{goal} argument) of the ‘put’ verb, as in \REF{ex:syntax:315} and \REF{ex:syntax:316}.

\ea%315
    \label{ex:syntax:315}
          \textit{Ndï} \textbf{\textit{mamune}} \textbf{\textit{nop}}.\\
\gll ndï  ma=mune      \textbf{na}{}-u-p\\
    3\textsc{pl}  \textsc{3sg.obj}=throw  \textsc{detr}{}-put-\textsc{pfv}\\
\glt `They threw it around.’ [ulwa032\_36:3]
\z

\ea%316
    \label{ex:syntax:316}
          \textit{Ndï nji ngalan \textbf{ndïnambï nop}.}\\
\gll ndï  nji    ngala=n    ndï=nambï  \textbf{na}{}-u-p\\
    \textsc{3pl}  thing  \textsc{pl.prox=obl}  \textsc{3pl}=skin  \textsc{detr-}put-\textsc{pfv}\\
\glt `They blocked them with these things.’ [ulwa036\_02:17]
\z

For uses of the form [nay] or [ne] (both from \textit{na-i} ‘\textsc{detr}{}-go.\textsc{pfv}’) as a \isi{TAM} or discourse marker, see \sectref{sec:15.3}.

\is{detransitivizer|)}
\is{detransitivization|)}
\is{syntax|)}
\is{transitivity|)}
\is{decreased transitivity|)}
\is{valency|)}
\is{valency reduction|)}

\subsection{The prefix \textit{na-} ‘\textsc{detr}’ as middle voice marker}\label{sec:13.8.3}

\is{valency reduction|(}
\is{valency|(}
\is{decreased transitivity|(}
\is{transitivity|(}
\is{syntax|(}
\is{detransitivization|(}
\is{detransitivizer|(}
\is{middle voice|(}

One function of the \isi{prefix} \textit{na-} \textsc{‘detr’} seems to be to create something like a \isi{middle voice} construction, indicating that the \isi{agent} of the verb is also affected by the verb, without being its grammatical object. Thus the verb \textit{kuk-} ‘gather’ can have a \isi{middle voice} sense when marked with the \isi{prefix} \textit{na-} \textsc{‘detr’}, something like ‘assemble, unite, gather oneself’, as in \REF{ex:syntax:317}.

\newpage

\ea%317
    \label{ex:syntax:317}
          \textit{Mape} \textbf{\textit{nakukawe}}.\\
\gll ma=p-e      \textbf{na}{}-kuk-aw-e\\
    3\textsc{sg.obj}=be\textsc{{}-dep}  \textsc{detr-}gather-put.\textsc{ipfv-dep}\\
\glt    `While [he] was there, [they] were gathering.’ [ulwa021\_00:06]
\z

Further examples are provided in the discussion of \isi{separable verb}s (\sectref{sec:9.2.1}).


\subsection{The prefix \textit{na-} ‘\textsc{detr}’ with the verbs \textit{ni-} ‘act’ and \textit{ni-} ‘die’}\label{sec:13.8.4}

Sometimes the role of the \isi{prefix} \textit{na-} \textsc{‘detr’} is not entirely clear. It occurs at times, for example, with the verb \textit{ni-} ‘act, do’. It is not, however, always present; and it is difficult to explain its presence as a form of \isi{detransitivization}, as the verb \textit{ni-} ‘act’ is not particularly \isi{transitive}. When it does select an argument (i.e., when the verb has the sense of ‘do [something]’), this argument is marked with the \isi{oblique marker} \textit{=n} \textsc{‘obl’}, as in \REF{ex:syntax:318}.

\ea%318
    \label{ex:syntax:318}
          \textit{Ndï makape wombïn} \textbf{\textit{man}} \textbf{\textit{ne}}.\\
\gll ndï  maka=p-e    wombïn  ma=\textbf{n}      \textbf{ni}{}-e\\
    3\textsc{pl}  thus=\textsc{cop-dep}  work    3\textsc{sg.obj=obl}  act-\textsc{ipfv}\\
\glt `They used to do work like this.’ [ulwa029\_06:12]
\z

It could be argued that example \REF{ex:syntax:318} is actually \isi{transitive}, with the \isi{object marker} \textit{ma=} \textsc{‘3sg.obj’} attaching directly to the \isi{verb stem}. Indeed, the surface form is [wo.mbïn.ma.ne], which does not offer any \isi{phonological} evidence of the \isi{oblique marker} /=n/, which -- I argue -- has \isi{deleted} before the immediately following /n/. However, its existence as an underlying form is supported by examples in which the \isi{allomorph} /=nï/ appears as the \isi{oblique marker} \REF{ex:syntax:319}.

\ea%319
    \label{ex:syntax:319}
          \textit{Wombïn \textbf{anmanï ne}.}\\
\gll wombïn  anma=\textbf{nï}  \textbf{ni}{}-e\\
    work    good=\textsc{obl}  act-\textsc{ipfv}\\
\glt `[They] were doing good work.’ [ulwa032\_31:07]
\z

Nevertheless, it is possible that the verb \textit{ni-} ‘act, do’ is evolving to become more prototypically \isi{transitive}, helped in part by the \isi{phonological} ambiguity of forms such as those in \REF{ex:syntax:318}. This can perhaps explain what otherwise seems like redundancy in marking \textit{ni-} ‘act’ with the detransitivizing \isi{prefix} \textit{na-} \textsc{‘detr’}, seemingly without any change of meaning, as in \REF{ex:syntax:350}.

\ea%320
    \label{ex:syntax:320}
          \textit{Una umbe makape wombïn} \textbf{\textit{man}} \textbf{\textit{naninda}}.\\
\gll unan    umbe    maka=p-e    wombïn  ma=\textbf{n} \textbf{na}{}-ni-nda\\
    1\textsc{pl.incl}  tomorrow  thus=\textsc{cop-dep}  work    3\textsc{sg.obj=obl}    \textsc{detr}{}-act-\textsc{irr}\\
\glt `Tomorrow we will do work like this.’ [ulwa030\_02:27]
\z

Similarly, the verb \textit{ni-} ‘die’, which I take to be \isi{homophonous} rather than \isi{polysemous} with \textit{ni-} ‘act, do’, nevertheless exhibits similar patterning, often occurring either with or without the \isi{prefix} \textit{na-} \textsc{‘detr’}. Example \REF{ex:syntax:321} illustrates the presence of the \isi{prefix} \textit{na-} \textsc{‘detr’} with the verb \textit{ni-} ‘die’, whereas example \REF{ex:syntax:322} illustrates its absence.

\is{middle voice|)}
\is{detransitivizer|)}
\is{detransitivization|)}
\is{syntax|)}
\is{transitivity|)}
\is{decreased transitivity|)}
\is{valency|)}
\is{valency reduction|)}

\ea%321
    \label{ex:syntax:321}
          \textit{Mï} \textbf{\textit{nanip}}.\\
\gll mï      \textbf{na-ni}{}-p\\
    3\textsc{sg.subj}  \textsc{detr-}die-\textsc{pfv}\\
\glt `She died.’ [ulwa014\_43:05]
\z

\ea%322
    \label{ex:syntax:322}
          \textit{Mï} \textbf{\textit{nip}}.\\
\gll mï      \textbf{ni}{}-p\\
    3\textsc{sg.subj}  die-\textsc{pfv}\\
\glt `She died.’ [ulwa028\_00:16]
\z

\subsection{The prefix \textit{na-} ‘\textsc{detr}’ with locative verbs}\label{sec:13.8.5}

\is{valency reduction|(}
\is{valency|(}
\is{decreased transitivity|(}
\is{transitivity|(}
\is{syntax|(}
\is{detransitivization|(}
\is{detransitivizer|(}
\is{locative verb|(}

The detransitivizing morpheme \textit{na-} \textsc{‘detr’} is often used with the \isi{locative verb} \textit{p-} {\textasciitilde} \textit{wap} ‘be, be at (be located at)’ (\sectref{sec:10.1}, \sectref{sec:10.4}), commonly in conjunction with the form \textit{mbï} ‘here’. It may serve the function of making the identification of the \isi{location} less definite, as in \REF{ex:syntax:323}. However, this is not always clearly the case. In \REF{ex:syntax:324}, it is not clear to me why the \isi{location} would be marked as less definite.

\is{locative verb|)}
\is{detransitivizer|)}
\is{detransitivization|)}
\is{syntax|)}
\is{transitivity|)}
\is{decreased transitivity|)}
\is{valency|)}
\is{valency reduction|)}

\ea%323
    \label{ex:syntax:323}
          \textit{Una ango luwa lunda? Mbï} \textbf{\textit{nawap}}.\\
\gll unan    ango  luwa  lo-nda  mbï  \textbf{na}{}-wap\\
    1\textsc{pl.incl}  which  place  go-\textsc{irr}  here  \textsc{detr}{}-be.\textsc{pst}\\
\glt `Where should we have gone? We stayed.’ (i.e., ‘just stayed around’?) [ulwa029\_04:00]
\z

\ea%324
    \label{ex:syntax:324}
          \textit{Wolka mo nay anmbï mbi mbï} \textbf{\textit{nap}}.\\
\gll wolka  ma=u      na-i      an-mbï    mbï-i      mbï     \textbf{na}{}-p\\
    again  3\textsc{sg.obj}=from  \textsc{detr}{}-go.\textsc{pfv}  out-here  here-go\textsc{.pfv}  here    \textsc{detr}{}-be\\
\glt `Again, [we] came from there, came out here, and are staying here.’ [ulwa002\_02:40]
\z

\subsection{The prefix \textit{na-} ‘\textsc{detr}’ for ‘become’}\label{sec:13.8.6}

\is{valency reduction|(}
\is{valency|(}
\is{decreased transitivity|(}
\is{transitivity|(}
\is{syntax|(}
\is{detransitivization|(}
\is{detransitivizer|(}

Sometimes when the morpheme \textit{na-} \textsc{‘detr’} affixes to the \isi{locative verb} \textit{p-} ‘be at’, the verb seems to have a meaning closer to that of the generic \isi{copula} =\textit{p} ‘\textsc{cop}’, which likely derived from it (\sectref{sec:10.3}). When used along with /p/, the \isi{prefix} \textit{na-} \textsc{‘detr’} often gives the sense of ‘become’ rather than ‘be’, although this is not always the case. Although \isi{semantic}ally more similar to the \isi{copula}, the form [p] here is functionally more similar to the verb in that it allows a verbal \isi{prefix} (\textit{na-} ‘\textsc{detr’}). This sort of ambiguity that exists between the \isi{locative verb} \textit{p-} ‘be’ and the \isi{copular enclitic} \textit{=p} ‘\textsc{cop}’ may be due to undergoing a process of \isi{grammaticalization}. Sentences \REF{ex:syntax:325} through \REF{ex:syntax:328} all convey the sense of ‘becoming’.

\is{detransitivizer|)}
\is{detransitivization|)}
\is{syntax|)}
\is{transitivity|)}
\is{decreased transitivity|)}
\is{valency|)}
\is{valency reduction|)}

\ea%325
    \label{ex:syntax:325}
          \textit{Mï wandam} \textbf{\textit{nap}}.\\
\gll mï      wandam  \textbf{na-p}\\
    3\textsc{sg.subj}  jungle    \textsc{detr}{}-be\\
\glt `It’s become a jungle.’ [ulwa014\_54:02]
\z

\ea%326
    \label{ex:syntax:326}
          \textit{Asiya mï mundotoma} \textbf{\textit{nape}}.\\
\gll asiya  mï      mundotoma  \textbf{na-p}{}-e\\
    string  3\textsc{sg.subj}  short      \textsc{detr}{}-be-\textsc{dep}\\
\glt `The string has gotten short.’ [ulwa015\_02:51]
\z

\ea%327
    \label{ex:syntax:327}
          \textit{Ndï ambi} \textbf{\textit{nap}} \textit{kalam} \textbf{\textit{nap}}.\\
\gll ndï  ambi  \textbf{na-p}    kalam    \textbf{na-p}\\
    3\textsc{pl}  big    \textsc{detr}{}-be  knowledge    \textsc{detr}{}-be\\
\glt `They are already big and know.’ (Literally ‘have become knowledgeable’) [ulwa014†]
\z

\ea%328
    \label{ex:syntax:328}
\is{detransitivizer}
\is{detransitivization}
\is{syntax}
\is{transitivity}
\is{decreased transitivity}
\is{valency}
\is{valency reduction}
          \textit{Ane naman awal} \textbf{\textit{nap}}.\\
\gll ane  na-ma-n    awal    \textbf{na-p}\\
    sun  \textsc{detr}{}-go-\textsc{ipfv}  afternoon  \textsc{detr}{}-be\\
\glt `The sun is going; it’s becoming evening.’ [ulwa018\_03:28]
\z

\subsection{The prefix \textit{na-} ‘\textsc{detr}’ with object-marker proclitics}\label{sec:13.8.7}

\is{object marker|(}
\is{object marking|(}
\is{valency reduction|(}
\is{valency|(}
\is{decreased transitivity|(}
\is{transitivity|(}
\is{syntax|(}
\is{detransitivization|(}
\is{detransitivizer|(}

Rather more challenging to explain, the detransitivizing \isi{prefix} \textit{na-} ‘\textsc{detr}’ may be used in conjunction with \isi{object-marking} \isi{proclitic}s. When present, the \isi{object marker} always precedes the \isi{prefix} \textit{na-} ‘\textsc{detr}’. Interestingly, when the 3\textsc{sg} marker is used, it takes the subject form [mï=] and not the object form \textsuperscript{†}[ma=]. In other words, although functioning to indicate the object of the verb, the marker in such constructions has the formal appearance of a \isi{subject marker}. The fact that the morpheme \textit{na-} ‘\textsc{detr}’ immediately precedes \isi{verb stem}s and follows \isi{object-marker} \isi{proclitic}s is support for the claim that it is a verbal \isi{prefix}. Sentences \REF{ex:syntax:329} through \REF{ex:syntax:332} exemplify the use of \isi{object marker}s along with the detransitivizing \isi{prefix} \textit{na-} `\textsc{detr}’.

\ea%329
    \label{ex:syntax:329}
          \textit{Mï mol anmbi inim naye} \textbf{\textit{mïnape}}.\\
\gll mï      ma=ul      an-mbï-i      inim  na-i-e \textbf{mï=na}{}-p-e\\
    3\textsc{sg.subj}  3\textsc{sg.obj}=with  out-here-go.\textsc{pfv}  water  \textsc{detr}{}-go.\textsc{pfv-dep}    3\textsc{sg.subj=detr}{}-be\textsc{{}-ipfv}\\
\glt `It went with it out into the water and stayed around there.’ [ulwa006\_07:38]
\z

\ea%330
    \label{ex:syntax:330}
          \textit{Mingusuwa mat nay ndï} \textbf{\textit{mïnanïkape}}.\\
\gll min-ngusuwa  ma=tï      na-i      ndï \textbf{mï=na}{}-nïkï-p-e\\
    3\textsc{du-}poor    3\textsc{sg.obj}=take  \textsc{detr}{}-go.\textsc{pfv}  \textsc{3pl}    3\textsc{sg.subj}=\textsc{detr}{}-dig{}-\textsc{pfv-dep}\\
\glt `The two poor things took it and they butchered it.’ [ulwa014\_51:32]
\z

\ea%331
    \label{ex:syntax:331}
          \textit{Ay} \textbf{\textit{ndïnamap}}.\\
\gll ay    \textbf{ndï=na}{}-ama-p\\
    sago  3\textsc{pl}=\textsc{detr}{}-eat-\textsc{pfv}\\
\glt `[They] have eaten the sago.’ [ulwa014\_67:33]
\z

\ea%332
    \label{ex:syntax:332}
          \textit{Min} \textbf{\textit{ndïnasap}}.\\
\gll min  \textbf{ndï=na}{}-asa-p\\
    3\textsc{du}  3\textsc{pl}=\textsc{detr}{}-hit{}-\textsc{pfv}\\
\glt `The two killed them.’ [ulwa001\_18:49]
\z

It may be that these forms have some level of reduced \isi{transitivity} or that the object of the \isi{transitive} verb is less definite. Sometimes, however, the \isi{direct object} of the verb marked with both the \isi{prefix} \textit{na-} ‘\textsc{detr}’ and an \isi{object marker} is expressed as a full NP, as in examples \REF{ex:syntax:333}, \REF{ex:syntax:334}, and \REF{ex:syntax:335}. It is difficult to see the morpheme \textit{na-} ‘\textsc{detr}’ as a means of reducing either \isi{transitivity} or \isi{definiteness} in examples such as these. That said, example \REF{ex:syntax:335} does seem best translated with an \isi{indefinite} article.

\ea%333
    \label{ex:syntax:333}
          \textit{Ande an \textbf{wa mïnapïna}.}\\
\gll ande  an      \textbf{wa}    \textbf{mï=na}-p-na\\
    ok    \textsc{1pl.excl}  village  3\textsc{sg.subj}=\textsc{detr}{}-be-\textsc{irr}\\
\glt `OK, we’ll stay in the village.’ [ulwa013\_06:27]
\z


\ea%334
    \label{ex:syntax:334}
\textit{Yokombla mï nay \textbf{numbu mïnanip}.}\\
\gll Yokombla  mï      na-i      \textbf{numbu}     \textbf{mï=na}-ni-p\\
    [name]    3\textsc{sg.subj}  \textsc{detr}{}-go.\textsc{pfv}  garamut 3\textsc{sg.subj}=\textsc{detr}-beat-\textsc{pfv}\\
\glt `Yokombla went and beat the \textit{garamut} drum.’ [ulwa014\_50:40]
\z

\ea%335
    \label{ex:syntax:335}
          \textit{\textbf{Apa ambi mïnaytana}.}\\
\gll \textbf{apa}  \textbf{ambi}  \textbf{mï=na}-ita-na\\
    house  big    3\textsc{sg.subj}=\textsc{detr}{}-build-\textsc{irr}\\
\glt `[I] will build a big house.’ [ulwa042\_05:01]
\z

In some instances, it seems that the simultaneous use of the detransitivizing \isi{prefix} \textit{na-} ‘\textsc{detr}’ and an \isi{object marker} may be attributable to frequent use of the \isi{prefix} \textit{na-} ‘\textsc{detr}’ with certain verbs. For example, verb forms such as [nay] (< \textit{na-} ‘\textsc{detr’} + \textit{i} ‘go.\textsc{pfv’}) are so common, that it could be that, for some speakers, the \isi{prefix} \textit{na-} ‘\textsc{detr}’ has \isi{fossilized} to the verb root, having lost its original detransitivizing meaning, as for example in \REF{ex:syntax:336}.

\ea%336
    \label{ex:syntax:336}
          \textit{Nay i nay Imwa} \textbf{\textit{mïnay}}.\\
\gll na-i      i    na-i      Imwa  \textbf{mï=na}{}-i\\
    \textsc{detr-}go.\textsc{pfv}  go.\textsc{pfv}  \textsc{detr-}go.\textsc{pfv}  [place]  3\textsc{sg.subj}=\textsc{detr}{}-go.\textsc{pfv}\\
\glt `[They] went and went, went to Imwa.’ [ulwa014\_24:52]
\z

The hypothesis that [nay] has \isi{fossilized} as a monomorphemic form may be supported by the fact that it itself may receive the \isi{prefix} \textit{na-} \textsc{‘detr’}, in effect giving the \isi{verb stem} two detransitivizing \isi{prefix}es, as in \REF{ex:syntax:337}.

\ea%337
    \label{ex:syntax:337}
          \textit{Nï mol nay wa mbï} \textbf{\textit{nanay}}.\\
\gll nï    ma=ul      na-i      wa  mbï     \textbf{na-na}{}-i\\
    1\textsc{sg}  \textsc{3sg.obj}=with  \textsc{detr-}go.\textsc{pfv}  village  here    \textsc{detr-detr-}go\textsc{.pfv}\\
\glt `I went with her and came home here.’ [ulwa032\_18:08]
\z

The \isi{stem} \textit{kamb-} ‘shun’ also frequently seems to have a \isi{fossilized} \isi{prefix} \textit{na-} \textsc{‘detr’}, especially when the verb has the sense ‘suffice, have enough’, as in \REF{ex:syntax:338}.

\ea%338
    \label{ex:syntax:338}
          \textit{Nambi} \textbf{\textit{nakamp}}.\\
\gll nï-ambi  \textbf{na}{}-kamb-p\\
    1\textsc{sg-top}  \textsc{detr-}shun-\textsc{pfv}\\
\glt `As for me, I’ve had enough.’ [ulwa032\_24:30]
\z

This form \textit{nakamb-} ‘shun [\textsc{detr]}’ can also take an additional \isi{object marker}, as in examples \REF{ex:syntax:339} and \REF{ex:syntax:340}.

\is{detransitivizer|)}
\is{detransitivization|)}
\is{syntax|)}
\is{transitivity|)}
\is{decreased transitivity|)}
\is{valency|)}
\is{valency reduction|)}
\is{object marking|)}
\is{object marker|)}

\ea%339
    \label{ex:syntax:339}
          \textit{I ndï una} \textbf{\textit{ndïnakam}}.\\
\gll i    ndï  unan    \textbf{ndï=na}{}-kamb\\
    way  3\textsc{pl}  \textsc{1pl.incl}  \textsc{3pl=detr-}shun\\
\glt `The [traditional] customs – we shun them.’ [ulwa014\_39:48]
\z

\ea%340
    \label{ex:syntax:340}
          \textit{Una} \textbf{\textit{ndïnakam}} \textit{nay.}\\
\gll    unan    \textbf{ndï=na-}kamb    na-i\\
    1\textsc{pl.incl}  \textsc{3pl=detr-}shun  \textsc{detr}{}-go.\textsc{pfv}\\
\glt `We left them and came.’ [ulwa037\_26:13]
\z

\subsection{Multiple \textit{na-} ‘\textsc{detr}’ prefixes on a single verb}\label{sec:13.8.8}

\is{valency reduction|(}
\is{valency|(}
\is{decreased transitivity|(}
\is{transitivity|(}
\is{syntax|(}
\is{detransitivization|(}
\is{detransitivizer|(}

At times, however, the sheer number of \textit{na-} \textsc{‘detr’} \isi{prefix}es in a given verb can be hard to account for \isi{morphosyntactic}ally -- even diachronically -- and may be most simply explained as a sort of filler, as in examples \REF{ex:syntax:341}, \REF{ex:syntax:342}, and \REF{ex:syntax:343}.

\is{detransitivizer|)}
\is{detransitivization|)}
\is{syntax|)}
\is{transitivity|)}
\is{decreased transitivity|)}
\is{valency|)}
\is{valency reduction|)}

\ea%341
    \label{ex:syntax:341}
          \textit{Unan} \textbf{\textit{ndïnanalanda}}.\\
\gll unan    \textbf{ndï=na-na}{}-la-nda\\
    1\textsc{pl.incl}  \textsc{3pl=detr-detr-}eat-\textsc{irr}\\
\glt `We will eat them.’ [ulwa030\_01:16]
\z

\ea%342
    \label{ex:syntax:342}
          \textit{Mbï} \textbf{\textit{nanap}}.\\
\gll mbï  \textbf{na-na}{}-p\\
    here  \textsc{detr-detr-}be\\
\glt `[We] stayed around.’ [ulwa032\_20:48]
\z

\ea%343
    \label{ex:syntax:343}
          \textit{Na ambi ndï mï} \textbf{\textit{ndïnanatïn}}.\\
\gll na    ambi  ndï  mï      \textbf{ndï=na-na}{}-tï-n\\
    talk  big    \textsc{3pl}  \textsc{3sg.subj}  \textsc{3pl=detr-detr-}take-\textsc{pfv}\\
\glt `The big stories – he got them [already].’ [ulwa032\_05:31]
\z

\subsection{Objects demoted by preverbal obliques}\label{sec:13.8.9}

\is{demotion|(}
\is{oblique|(}
\is{valency reduction|(}
\is{valency|(}
\is{decreased transitivity|(}
\is{transitivity|(}
\is{syntax|(}
\is{detransitivization|(}
\is{detransitivizer|(}

In this section I examine a phenomenon in Ulwa that may be analyzed as a change in \isi{valency} or as the \isi{demotion} of a verbal argument. It is possible for the \isi{semantic} object of a verb to appear as part of an \isi{oblique} \isi{phrase}. This occurs when an element intervenes between the (otherwise immediately preverbal) \isi{direct object} and the verb. The element that motivates this \isi{demotion} may be a \isi{postpositional phrase} or an \isi{adjective} functioning \isi{adverb}ially. In examples \REF{ex:syntax:344} through \REF{ex:syntax:347}, the logical object of the verb contains \isi{oblique} marking.

\ea%344
    \label{ex:syntax:344}
          \textit{Ndï} \textbf{\textit{ndïn}} \textit{we ndul landa.}\\
\gll    ndï  \textbf{ndï=n}    we    ndï=ul    la-nda\\
    3\textsc{pl}  3\textsc{pl=obl}  sago  3\textsc{pl}=with  eat-\textsc{irr}\\
\glt `[They] would eat them [= pieces of meat] with sago.’ [ulwa033\_02:31]
\z

\ea%345
    \label{ex:syntax:345}
          \textbf{\textit{Man}} \textit{al mol tïn.}\\
\gll \textbf{ma=n}      al  ma=ul      tï-n\\
    3\textsc{sg.obj=obl}  net  3\textsc{sg.obj}=with  take-\textsc{pfv}\\
\glt `[It] got her with the mosquito net.’ [ulwa006\_04:56]
\z

\ea%346
    \label{ex:syntax:346}
          \textit{Nïnji yenat ngala ango apka} \textbf{\textit{ndïn}} \textit{anma kalampe.}\\
\gll    nï-nji    yenat    ngala    ango  apka  \textbf{ndï=n}    anma kalam=p-e\\
    1\textsc{sg-poss}  daughter  \textsc{pl.prox}  \textsc{neg}  very  3\textsc{pl=obl}  good    knowledge=\textsc{cop-dep}\\
\glt `My daughters do not know them very well.’ [ulwa032\_38:29]
\z

\ea%347
    \label{ex:syntax:347}
          \textit{Ndï wa} \textbf{\textit{sokoyn}} \textit{akïnaka ine.}\\
\gll    ndï  wa  \textbf{sokoy=n}    akïnaka  ina-e\\
    3\textsc{pl}  just  tobacco=\textsc{obl}  new    get-\textsc{ipfv}\\
\glt `They just harvest the tobacco prematurely.’ [ulwa037\_51:22]
\z

In example \REF{ex:syntax:348}, it seems that even the \isi{question word} \textit{anjikaka} ‘how?’ can intervene, thereby motivating the \isi{demotion} of the object.

\ea%348
    \label{ex:syntax:348}
          \textit{U} \textbf{\textit{man}} \textit{anjikaka tï inde iye mï ko liyu?}\\
\gll    u    \textbf{ma=n}      anjikaka  tï  inda-e    i-e mï      ko  li\textit{{}-}u\\
    2\textsc{sg}   3\textsc{sg.obj=obl}  how    take  walk\textsc{{}-dep} go.\textsc{pfv-dep}    3\textsc{sg.subj}  just  down-put\\
\glt `How were you carrying it around such that it just fell?’ [ulwa037\_01:13]
\z

Constructions such as these may, in a way, be considered \isi{antipassive}s, since the logical object of the \isi{transitive} verb is demoted to an \isi{oblique} \isi{phrase}. It should be noted, however, that there is no \isi{verbal morphology}, such as an \isi{affix}, to signal this change.

\is{detransitivizer|)}
\is{detransitivization|)}
\is{syntax|)}
\is{transitivity|)}
\is{decreased transitivity|)}
\is{valency|)}
\is{valency reduction|)}
\is{oblique|)}
\is{demotion|)}

\section{Valency expansion?}\label{sec:13.9}

\is{valency expansion|(}
\is{valency|(}
\is{increased transitivity|(}
\is{transitivity|(}
\is{syntax|(}

Ulwa has no known valency-increasing constructions. The addition of any \isi{core argument}s requires the addition, as well, of an \isi{inflect}ed verb -- that is, the addition of a clause. In other words, there is no verbal \isi{affix} or \isi{clitic} that can turn an \isi{intransitive} verb into a \isi{transitive} one or that can create an \isi{applicative} or \isi{causative} construction. Thus, what are sometimes expressed through valency-increasing operations in other languages have as functional equivalents in Ulwa \isi{multiclausal construction}s, as illustrated in the following subsections.

\is{syntax|)}
\is{transitivity|)}
\is{increased transitivity|)}
\is{valency|)}
\is{valency expansion|)}

\subsection{Causative constructions}\label{sec:13.9.1}

\is{causative|(}
\is{valency expansion|(}
\is{valency|(}
\is{increased transitivity|(}
\is{transitivity|(}
\is{syntax|(}

  Events in which one participant causes another to act are expressed in Ulwa by a minimum of two clauses: one relating the \isi{causer} to the \isi{causee}, the other detailing the action of the \isi{causee}. In examples \REF{ex:syntax:349} and \REF{ex:syntax:350}, the verb \textit{ni-} ‘act, do’ is used along with a \isi{postpositional phrase} \isi{head}ed by \textit{ul} ‘with’ to convey the sense ‘force’. In these constructions, the clause with the \isi{causer} as subject is marked as \isi{dependent} with the \isi{dependent marker} \textit{-e} ‘\textsc{dep}’ following the verb. This first clause may thus be translated with a causal sense (i.e., ‘since …’) (\sectref{sec:12.2.2}).

\ea%349
    \label{ex:syntax:349}
          \textit{Itom mï Kongos} \textbf{\textit{mol}} \textbf{\textit{nipe}} \textit{mï apa itap.}\\
\gll    itom  mï      Kongos  ma=\textbf{ul}      \textbf{ni}{}-p-e      mï apa    ita-p\\
    father  3\textsc{sg.subj}  [name]    3\textsc{sg.obj}=with  act-\textsc{pfv-dep}  \textsc{3sg.subj}    house  build-\textsc{pfv}\\
\glt `Father made Kongos build a house.’ (Literally ‘[Since] father acted with [i.e., forced] Kongos, he built a house.’) [elicited]
\z

\ea%350
    \label{ex:syntax:350}
          \textit{Yena mï numan \textbf{mol nipe} mï asimu inap.}\\
\gll    yena  mï      numan    ma=\textbf{ul}      \textbf{ni}{}-p-e mï      asi-mu    ina-p\\
    woman  3\textsc{sg.subj}  husband  3\textsc{sg.obj}=with  act-\textsc{pfv-dep}    \textsc{3sg.subj}  grass-seed  get-\textsc{pfv}\\
\glt `The woman made [her] husband buy rice.’ (Literally ‘[Since] the woman acted with [i.e., forced] [her] husband, he got rice.’) [elicited]
\z

In example \REF{ex:syntax:351}, a \isi{conditional} statement is used to convey the \isi{irrealis} sense of a \isi{causative}.

\ea%351
    \label{ex:syntax:351}
          \textit{Itom mï Kongos \textbf{mol nipta} mï apa itana.}\\
\gll itom  mï      Kongos  ma=\textbf{ul}      \textbf{ni}{}-p-ta      mï apa    ita-na\\
    father  3\textsc{sg.subj}  [name]    3\textsc{sg.obj}=with  act-\textsc{pfv}{}-\textsc{cond}  \textsc{3sg.subj}    house  build-\textsc{irr}\\
\glt `Father will make Kongos build a house.’ (Literally ‘if father forces Kongos, he will build a house.’) [elicited]
\z

  The \isi{idiom} ‘act with’ (i.e., ‘force’) may be used in a single clause, without any other clause divulging what the person is forced to do, as in \REF{ex:syntax:352}. This lends further support to the claim that the \isi{causative} constructions in \REF{ex:syntax:349}, \REF{ex:syntax:340}, and \REF{ex:syntax:351} are all truly composed of two clauses each.

\ea%352
    \label{ex:syntax:352}
          \textit{Itom mï Kongos \textbf{mol nip}.}\\
\gll itom  mï      Kongos  ma=\textbf{ul}      \textbf{ni}{}-p\\
    father  3\textsc{sg.subj}  [name]    3\textsc{sg.obj}=with  act-\textsc{pfv}\\
\glt `Father forced Kongos.’ [elicited]
\z

In addition to ‘act with’, there is another \isi{idiom} used in Ulwa to express compulsion. The form is \textit{nambïnïkï-} ‘make, nag’ (literally ‘dig at [one’s] skin’). It conveys a weaker level of pressure than \textit{ul … ni-} ‘make, force’, and may be seen in \REF{ex:syntax:353}.

\ea%353
    \label{ex:syntax:353}
          \textit{Yena mï numan} \textbf{\textit{manambïnkape}} \textit{mï asimu inap.}\\
\gll    yena  mï      numan    ma=\textbf{nambï-nïkï-}p-e mï      asi-mu    ina-p\\
    woman  3\textsc{sg.subj}  husband  3\textsc{sg.obj}=skin-dig-\textsc{pfv-dep}    3\textsc{sg.subj}  grass-seed  get-\textsc{pfv}\\
\glt `The woman made her husband buy rice.’ [elicited]
\z

The fact that examples such as \REF{ex:syntax:350} and \REF{ex:syntax:351} are sets of two clauses is suggested by uses of these \isi{causative} \isi{verb phrase}s in situations where the would-be \isi{causee} fails to complete the action, as in \REF{ex:syntax:354} and \REF{ex:syntax:355}.

\newpage

\ea%354
    \label{ex:syntax:354}
          \textit{Yena mï numan \textbf{mol nipe} mï ango asimu inap.}\\
\gll    yena  mï      numan    ma=\textbf{ul}      \textbf{ni}{}-p-e      mï ango  asi-mu    ina-p\\
    woman  3\textsc{sg.subj}  husband  3\textsc{sg.obj}=with  act-\textsc{pfv-dep}  \textsc{3sg.subj}    \textsc{neg}  grass-seed  get-\textsc{pfv}\\
\glt `Even though the woman pressured [her] husband, he didn’t buy rice.’ [elicited]
\z

\ea%355
    \label{ex:syntax:355}
          \textit{Yena mï numan} \textbf{\textit{manambïnkape}} \textit{mï ango asimu inap.}\\
\gll    yena  mï      numan    ma=\textbf{nambï-nïkï-}p-e      mï ango  asi-mu    ina-p\\
    woman  3\textsc{sg.subj}  husband  3\textsc{sg.obj}=skin-dig-\textsc{pfv-dep}  \textsc{3sg.subj}    \textsc{neg}  grass-seed  get-\textsc{pfv}\\
\glt `Even though the woman nagged [her] husband, he didn’t buy rice.’ [elicited]
\z

In other words, any putative ‘causing’ verb is really an ‘asking’ or ‘persuading’ verb, and in no way suggests any increase in \isi{valency}.

\is{syntax|)}
\is{transitivity|)}
\is{increased transitivity|)}
\is{valency|)}
\is{valency expansion|)}
\is{causative|)}

\subsection{Causatives in indirect discourse}\label{sec:13.9.2}

\is{indirect discourse|(}
\is{causative|(}
\is{valency expansion|(}
\is{valency|(}
\is{increased transitivity|(}
\is{transitivity|(}
\is{syntax|(}

Commands or \isi{request}s made in \isi{reported speech} may be viewed as forms of causatives, provided that the \isi{command} or \isi{request} being made leads to an action being performed. An example is given in \REF{ex:syntax:358}.

\ea%358
    \label{ex:syntax:358}
          \textit{\textbf{Nan mate} mï i masamasa mowonp.}\\
\gll    \textbf{na=n}    ma=\textbf{ta}{}-e      mï      i      masamasa ma=won-p\\
    talk=\textsc{obl}  3\textsc{sg.obj}=say-\textsc{dep}  \textsc{3sg.subj}  go.\textsc{pfv}    tree.species    3\textsc{sg.obj}=cut-\textsc{pfv}\\
\glt `[She] told him to go cut the tree.’ (Literally ‘[Since] [she] told him, he went and cut the tree.’) [ulwa001\_03:57]
\z

  In Ulwa, \isi{command}s expressed in \isi{reported speech} reveal a distinction between \isi{realis} and \isi{irrealis} \isi{mood}s. In \ili{English}, for example, there is a degree of ambiguity created by sentences that employ \isi{non-finite verb form}s (i.e., \isi{infinitive}s), such as the following: \textit{Mary told John} \textbf{\textit{to leave}}. Namely, it is not clear whether John actually left or not. In Ulwa, however, the \isi{TAM} of the verb (e.g., \textit{leave} in the \linebreak \ili{English} example) reveals whether the \isi{imperative} led to the desired outcome (\isi{realis}) or not (\isi{irrealis}). This may be seen with a contrasting pair of sentences, one with \isi{perfective} \isi{aspect} (i.e., \isi{realis} \isi{mood}) \REF{ex:syntax:356} and another with \isi{irrealis} \isi{mood} \REF{ex:syntax:357}.

\ea%356
    \label{ex:syntax:356}
          \textit{Mawna mï} \textbf{\textit{nan}} \textit{Yawat} \textbf{\textit{mate}} \textit{mï} \textbf{\textit{i}}.\\
\gll Mawna  mï      \textbf{na=n}    Yawat  ma=\textbf{ta}{}-e      mï \textbf{i}\\
    [name]    3\textsc{sg.subj}  talk=\textsc{obl}  [name]  3\textsc{sg.obj}=say-\textsc{dep}  3\textsc{sg.subj}    go.\textsc{pfv}\\
\glt `Mawna told Yawat to leave [and he did].’ (Literally ‘[Since] Mawna told Yawat, he left.’) [elicited]
\z

\ea%357
    \label{ex:syntax:357}
          \textit{Mawna mï} \textbf{\textit{nan}} \textit{Yawat} \textbf{\textit{mate}} \textit{mï} \textbf{\textit{mana}}.\\
\gll Mawna  mï      \textbf{na=n}    Yawat  ma=\textbf{ta}{}-e      mï ma-\textbf{na}\\
    [name]    3\textsc{sg.subj}  talk=\textsc{obl}  [name] 3\textsc{sg.obj}=say-\textsc{dep}  \textsc{3sg.subj}    go-\textsc{irr}\\
\glt `Mawna told Yawat to leave [but it is unclear whether or not he did].’ (Literally ‘[Since] Mawna told Yawat, he might have left’; or ‘[Since] Mawna told Yawat, he will leave.’) [elicited]
\z

Sentence \REF{ex:syntax:359} provides another example of a \isi{causative} in \isi{indirect discourse}, this one illustrating \isi{irrealis} \isi{mood}.

\is{syntax|)}
\is{transitivity|)}
\is{increased transitivity|)}
\is{valency|)}
\is{valency expansion|)}
\is{causative|)}
\is{indirect discourse|)}

\ea%359
    \label{ex:syntax:359}
          \textit{Unan} \textbf{\textit{na}} \textbf{\textit{makïta}} \textit{mï \textbf{ndambikulilïnda}!}\\
\gll    unan    \textbf{na}    ma=\textbf{kï}{}-ta        mï ndï-ambi=kuli-lï-\textbf{nda}\\
    1\textsc{pl.incl}  talk  3\textsc{sg.obj}=say\textsc{{}-cond}  \textsc{3sg.subj}    \textsc{3pl-top=}throw-put-\textsc{irr}\\
\glt `Let’s tell him to throw them away!’ (Literally ‘If we tell him, he will throw them away.’) [ulwa014\_13:59]
\z

\subsection{Factitive constructions}\label{sec:13.9.3}

\is{factitive|(}
\is{valency expansion|(}
\is{valency|(}
\is{increased transitivity|(}
\is{transitivity|(}
\is{syntax|(}

When someone or something is caused to have a certain attribute, Ulwa uses an \isi{idiom} with the verb \textit{me}{}- ‘sew’. The object of this verb is the acquired attribute, and that which acquires it is expressed as an \isi{oblique} \isi{phrase} designated by the \isi{oblique marker} \textit{=n} ‘\textsc{obl}’ (literally ‘sew [the attribute] to [that which acquires it]’). Examples of such \isi{factitive} (or \isi{translative}) constructions are given in \REF{ex:syntax:360}, \REF{ex:syntax:361}, and \REF{ex:syntax:362}.

\ea%360
    \label{ex:syntax:360}
          \textbf{\textit{Ndïn}} \textit{wapata} \textbf{\textit{mep}}.\\
\gll ndï=\textbf{n}    wapata  \textbf{me}{}-p\\
    3\textsc{pl=obl}  dry    sew-\textsc{pfv}\\
\glt `[He] made them [= sores] dry.’ (Literally ‘[He] sewed dry[ness] to them.’ In other words, ‘He healed the sores.’) [ulwa014†]
\z

\newpage

\ea%361
    \label{ex:syntax:361}
          \textbf{\textit{Amblan}} \textit{mundotoma} \textbf{\textit{menda}}.\\
\gll ambla=\textbf{n}    mundotoma  \textbf{me}{}-nda\\
    \textsc{pl.refl=obl}  short      sew-\textsc{irr}\\
\glt `[We] will make ourselves short.’ (Literally ‘[We] will sew short[ness] to ourselves.’ In other words, ‘We will become less populous as a village.’) [ulwa037\_33:57]
\z

\ea%362
    \label{ex:syntax:362}
          \textit{Kïka mï awlu apa mo} \textbf{\textit{man}} \textit{tembi} \textbf{\textit{mep}}.\\
\gll kïka    mï      awlu  apa    ma=u      ma=\textbf{n} tembi  \textbf{me}{}-p\\
    white.ant  3\textsc{sg.subj}  step  house  3\textsc{sg.obj=}from  3\textsc{sg.obj=obl}           bad    sew-\textsc{pfv}\\
\glt `The white ant nest has come to the house and made it bad.’ (Literally ‘… has sewn bad[ness] to it.’ In other words, ‘… has worsened it.’) [ulwa042\_05:37]
\z

In examples \REF{ex:syntax:360}, \REF{ex:syntax:361}, and \REF{ex:syntax:362}, the \isi{adjective}s either may be functioning as \isi{abstract noun}s or may (as is common in \isi{translative} constructions in other languages) be functioning as \isi{predicate adjective}s. Example \REF{ex:syntax:363} contains the noun \textit{kalam} ‘knowledge’, which also commonly functions either as an \isi{abstract noun} (‘knowledge’) or as an \isi{adjective} (‘knowledgeable, knowing’) (\sectref{sec:5.4}).

\ea%363
    \label{ex:syntax:363}
          \textit{Nan ndïtap} \textbf{\textit{ndïn}} \textit{kalam} \textbf{\textit{mendat}}.\\
\gll na=n    ndï=ta-p    ndï=\textbf{n}    kalam      \textbf{me}{}-nda-t\\
    talk=\textsc{obl}  \textsc{3pl=}say-\textsc{pfv}  \textsc{3pl=obl}  knowledge    sew-\textsc{irr-spec}\\
\glt `[We] told them so that [we] might teach them.’ (Literally ‘… might sew knowledge to them.’) [ulwa018\_05:09]
\z

The ‘teaching’ construction in \REF{ex:syntax:363} encodes in its \isi{oblique} \isi{phrase} the \isi{recipient} of the knowledge. It is also possible for such constructions to admit two \isi{oblique} phrases, one denoting the \isi{recipient} of the knowledge and the other denoting the material being taught (literally ‘sew knowledge to someone with [respect to] something’), as is illustrated by example \REF{ex:syntax:364}.

\ea%364
    \label{ex:syntax:364}
          \textit{Nï nji} \textbf{\textit{ngalan}} \textbf{\textit{unï}} \textit{kalam} \textbf{\textit{men}}.\\
\gll nï    nji    ngala=\textbf{n}    un=\textbf{nï}    kalam    \textbf{me}{}-n[da]\\
    1\textsc{sg}  thing  \textsc{pl.prox=obl}  \textsc{2pl=obl}  knowledge  sew-\textsc{irr}\\
\glt `I will teach you these things.’ [ulwa014\_41:46]
\z

\newpage

In example \REF{ex:syntax:365}, the object of the verb is a title that has been acquired.

\ea%365
    \label{ex:syntax:365}
          \textbf{\textit{Amblan}} \textit{ini tamndï} \textbf{\textit{mep}}.\\
\gll ambla=\textbf{n}    ini    tamndï  \textbf{me}{}-p\\
    \textsc{pl.refl=obl}  ground  owner  sew-\textsc{pfv}\\
\glt `[They] made themselves the owners of the land.’ (Literally ‘[They] sewed [the title] of land-owner to themselves.’) [ulwa014\_22:55]
\z

While the verb \textit{me}{}- ‘sew’ is the most common verb used in these constructions, the same \isi{factitive} concept can be expressed with other verbs that show that a new quality is being ‘attached’, as in examples \REF{ex:syntax:366} and \REF{ex:syntax:367}, which use the \isi{compound verb} \textit{watlï-} ‘put atop’.

\is{syntax|)}
\is{transitivity|)}
\is{increased transitivity|)}
\is{valency|)}
\is{valency expansion|)}
\is{factitive|)}

\ea%366
    \label{ex:syntax:366}
          \textit{Simban yeta tï} \textbf{\textit{ambïwatlïpe}}.\\
\gll Simban  yeta  tï    ambï=\textbf{wat-lï}{}-p-e\\
    [name]    man  take  \textsc{sg.refl=}atop-put-\textsc{pfv-dep}\\
\glt `Simban made herself [like] a man.’ (Literally ‘Simban took “man” and put [it] atop herself.’) [ulwa034\_00:35]
\z

\ea%367
    \label{ex:syntax:367}
          \textit{Mï yeta ambi tï} \textbf{\textit{ambïwatlïp}}.\\
\gll mï      yeta  ambi  tï    ambï=\textbf{wat-lï}{}-p\\
    \textsc{3sg.subj}  man  big    take  \textsc{sg.refl}=atop-put-\textsc{pfv}\\
\glt `He’s [like] a grown man!’ (Literally ‘He took “big man” and put [it] atop himself.’) [ulwa014\_13:03]
\z

\subsection{Permissive constructions}\label{sec:13.9.4}

\is{permissive|(}
\is{valency expansion|(}
\is{valency|(}
\is{increased transitivity|(}
\is{transitivity|(}
\is{syntax|(}

Constructions expressing permission function similarly to \is{biclausal construction} biclausal \isi{causative} constructions (\sectref{sec:13.9.1}). In the first clause is the verb \textit{ka-} ‘let, leave, allow’, which takes as its object the person or thing being granted permission; in the second clause, the subject is this person or thing being granted permission, and the verb explains what this subject is being permitted to do, as in \REF{ex:syntax:374}.

\ea%374
    \label{ex:syntax:374}
          \textbf{\textit{Ndïlakan}} \textit{ndï mapïn!}\\
\gll    ndï=la-\textbf{ka}-n    ndï  ma=p-na\\
    3\textsc{pl}=\textsc{irr-}let-\textsc{imp}  3\textsc{pl}  3\textsc{sg.obj}=be-\textsc{irr}\\
\glt `Let them stay there!’ (Literally ‘Let them! They will be there.’) [ulwa030\_01:55]
\z

  First, it may be shown how the verb \textit{ka-} ‘let, leave, allow’ functions in simple \isi{monoclausal construction}s. It should be noted that, in these clauses, the object of the verb is the \isi{location} in which someone or something is being left. That which is being left, on the other hand, may be expressed in an \isi{oblique} \isi{phrase} using the \isi{oblique marker} \textit{=n} ‘\textsc{obl}’.\footnote{Compare the argument structure of the verb \textit{lï-} ‘put’ (\sectref{sec:9.2.2}). For the irregular \isi{circumfix}-like \isi{irrealis} form of the verb \textit{ka-} ‘let, leave, allow’, see \sectref{sec:4.3} and \sectref{sec:9.2.3}.} Sentences \REF{ex:syntax:368} through \REF{ex:syntax:372} illustrate the use of \textit{ka-} ‘let, leave, allow’ in simple \isi{monoclausal construction}s.

\ea%368
    \label{ex:syntax:368}
        \textit{Mol i man Simundo} \textbf{\textit{maka}}.\\
\gll ma=ul      i    ma=n      Simundo  ma=\textbf{ka}\\
    3\textsc{sg.obj}=with  go.\textsc{pfv}  3\textsc{sg.obj=obl}  [place]    \textsc{3sg.obj}=let\\
\glt `[They] went with him and left him at Simundo [village].’ [ulwa002\_04:27]
\z

\ea%369
    \label{ex:syntax:369}
          \textit{Dingo man} \textbf{\textit{maka}}.\\
\gll Dingo  ma=n      ma=\textbf{ka}\\
    [name]  3\textsc{sg.obj=obl}  3\textsc{sg.obj}=let\\
\glt `[They] left Dingo there.’ [ulwa002\_04:12]
\z

\ea%370
    \label{ex:syntax:370}
          \textit{Mï nul mbi nïn} \textbf{\textit{ka}} \textit{wolka nay.}\\
\gll    mï      nï=ul    mbï-i      nï=n      \textbf{ka}    wolka na-i\\
    3\textsc{sg.subj}  1\textsc{sg}=with  here-go.\textsc{pfv}  1\textsc{sg=obl}   let.\textsc{pfv}    again    \textsc{detr-}go.\textsc{pfv}\\
\glt `She came with me, left me, and went again.’ [ulwa032\_11:07]
\z

\ea%371
    \label{ex:syntax:371}
          \textit{Ulum pul male we ndïn} \textbf{\textit{maka}}.\\
\gll ulum  pul    ma=ale-e        we    ndï=n    ma=\textbf{ka}\\
    palm  piece  3\textsc{sg.obj}=beat-\textsc{dep}    sago  3\textsc{pl=obl}  3\textsc{sg.obj}=let\\
\glt `[They] were scraping a piece of sago palm but left the sago starch there.’ [ulwa037\_60:30]
\z

\ea%372
    \label{ex:syntax:372}
          \textit{Wana} \textbf{\textit{malakana}}\textit{!}\\
\gll    wana  ma=la-\textbf{ka}{}-na\\
    \textsc{proh}  \textsc{3sg.obj}=\textsc{irr-}let-\textsc{irr}\\
\glt `Don’t abandon it!’ [ulwa014\_57:39]
\z

Notably, when functioning in \is{biclausal construction} biclausal \isi{permissive} constructions, the verb \textit{ka-} ‘let, leave, allow’ takes as its object the thing being permitted, as in \REF{ex:syntax:373}, as opposed to a \isi{location}, as in \isi{monoclausal sentence}s such as \REF{ex:syntax:368} and \REF{ex:syntax:369}.

\ea%373
    \label{ex:syntax:373}
          \textit{\textbf{Ndïnji ndïlaka} ndï mïnap.}\\
\gll    \textbf{ndï-nji}    \textbf{ndï}=la-ka    ndï  mï=na-p\\
    3\textsc{pl-poss}  3\textsc{pl}=\textsc{irr-}let  \textsc{3pl}  \textsc{3sg.subj=detr-}be\\
\glt `[They] let their possessions [just] stay [as they are].’ (Literally ‘[They] let theirs; they stay.’) [ulwa014\_71:31]
\z

Such constructions often make use of \isi{conditional} clauses, especially in \isi{command}s \REF{ex:syntax:375}.

\ea%375
    \label{ex:syntax:375}
          \textit{\textbf{Unanji malakata} mï ina!}\\
\gll    \textbf{unan-nji}    \textbf{ma=la-ka-ta}      mï      i-na\\
    1\textsc{pl.incl-poss}  \textsc{3sg.obj=irr-}let-\textsc{cond}  3\textsc{sg.subj}  come-\textsc{irr}\\
\glt `Let our [granddaughter] come!’ (Literally ‘If [you] let our [granddaughter], she will come.’) [ulwa014\_11:14]
\z

These \isi{permissive} \isi{conditional sentence}s may be contrasted with sentence \REF{ex:syntax:376}, in which the \isi{conditional} verb form \textit{lakata} ‘let, leave, allow [\textsc{cond}]’ is used in a \isi{protasis} to mean, simply, ‘leave’ (i.e., it is not a \isi{permissive} construction); here, the object of \textit{lakata} ‘let [\textsc{cond]}’ is the \isi{location} where something is left.

\ea%376
    \label{ex:syntax:376}
          \textit{Ndïn mumnopen luwa} \textbf{\textit{lakata}} \textit{tomoy ndïwat mana.}\\
\gll    ndï=n    mumne-u-p-en        luwa  la-\textbf{ka}{}-ta tomoy    ndï=wat  ma-na\\
    3\textsc{pl=obl}  cold.and.dark-put-\textsc{pfv-nmlz}  place  \textsc{irr-}let-\textsc{cond}    insect.species  3\textsc{pl=}atop  go-\textsc{irr}\\
\glt `If [they] were to leave them in a cold and dark place, [then] insects would go onto them.’ [ulwa014\_69:01]
\z

Finally, it may be noted that the verb \textit{ka-} ‘let, leave, allow’ is used frequently in an \isi{idiom} meaning something like ‘forget about it!’, ‘don’t’ even mention it!’, or ‘amazing!’. In such expressions, the \isi{object marker} typically takes the \is{topic-marker pronoun} topic-marked pronominal form (\sectref{sec:6.6}) and the verb takes an \isi{irrealis} or \isi{imperative} form. This use of \textit{ka-} ‘let’ is illustrated by examples \REF{ex:syntax:377} and \REF{ex:syntax:378}.

\is{syntax|)}
\is{transitivity|)}
\is{increased transitivity|)}
\is{valency|)}
\is{valency expansion|)}
\is{permissive|)}

\ea%377
    \label{ex:syntax:377}
          \textit{A} \textbf{\textit{mambilakan}}\textit{!}\\
\gll    a  ma-ambi=la-\textbf{ka}{}-n\\
    ah  3\textsc{sg.obj-top}=\textsc{irr-}let-\textsc{imp}\\
\glt `Ah, forget it!’ [ulwa014\_21:36]
\z

\ea%378
    \label{ex:syntax:378}
\is{syntax}
\is{transitivity}
\is{increased transitivity}
\is{valency}
\is{valency expansion}
\is{permissive}
          \textbf{\textit{Mambilakan}} \textit{anankïn ngala!}\\
\gll    ma-ambi-la-\textbf{ka}{}-n      anankïn  ngala\\
    3\textsc{sg.obj-top-irr}{}-let-\textsc{imp}  blood    \textsc{pl.prox}\\
\glt `Amazing, the blood!’ [ulwa001\_17:55]
\z

\subsection{Desiderative constructions}\label{sec:13.9.5}

\is{desiderative|(}
\is{valency expansion|(}
\is{valency|(}
\is{increased transitivity|(}
\is{transitivity|(}
\is{syntax|(}

The expression of wants follows patterns very similar to those of \isi{indirect discourse} (\sectref{sec:13.4.5}). Indeed, the most common way of expressing that one wants something to happen is to use a verb of speaking or thinking, typically \textit{kï-} ‘say’ and typically expressed in the \isi{perfective} \isi{mood} and with the detransitivizing \isi{prefix} \textit{na-} \textsc{‘detr’} (thus: [nakap], literally ‘said’ or ‘thought’). This form has likely been somewhat \isi{fossilized} as a word used to express desires. The clause expressing the desire is a \isi{dependent clause} embedded within a \isi{matrix clause} that has as its subject the person who desires something. The verb in the \isi{dependent clause} is always marked as \isi{irrealis}, as in sentences \REF{ex:syntax:379} through \REF{ex:syntax:382}. Brackets enclose the \isi{embedded clause}s.

\ea%379
    \label{ex:syntax:379}
          \textit{Sokoy ulwape nï nïnji wa mana} \textbf{\textit{nakap}}.\\
\gll sokoy    ulwa=p-e      nï    [nï-nji    wa    ma-na] \textbf{na-kï-p}\\
    tobacco  nothing=\textsc{cop-dep}  1\textsc{sg}  [1\textsc{sg-poss}  village  go-\textsc{irr]}    \textsc{detr-}say-\textsc{pfv}\\
\glt `Since there’s no tobacco, I want to go to my village.’ [ulwa032\_02:15]
\z

\ea%380
    \label{ex:syntax:380}
          \textit{Kaukaunï mankïna} \textbf{\textit{nakap}}.\\
\gll {[kaukau=nï}  {ma=nïkï-na]}    \textbf{na-kï-p}\\
    {[kaukau=\textsc{obl}}  \textsc{3sg.obj=}dig-\textsc{irr]}  \textsc{detr-}say-\textsc{pfv}\\
\glt `[They] wanted to plant \textit{kaukau} [= sweet potato].’ [ulwa037\_55:30]
\z

\ea%381
    \label{ex:syntax:381}
          \textit{Nïn u na tïna} \textbf{\textit{nakap}}.\\
\gll {[nï=n}    u    na    {tï-na]}    \textbf{na-kï-p}\\
    {[1\textsc{sg=obl}}  from  talk  take-\textsc{irr]}  \textsc{detr-}say-\textsc{pfv}\\
\glt `[He] wants to get stories from me.’ [ulwa032\_05:24]
\z

\ea%382
    \label{ex:syntax:382}
          \textit{Na ndan nïkïna \textbf{nakap}?}\\
\gll {[na}    anda=n    {nï-kï-na]}    \textbf{na-kï-p}\\
    {[talk}  \textsc{sg.dist=obl}  1\textsc{sg}=say-\textsc{irr]}  \textsc{detr-}say-\textsc{pfv}\\
\glt `Do [you] want to tell me something?’ [ulwa014\_03:57]
\z

The subject of the \isi{matrix clause} (the person desiring something) need not be the subject of the \isi{embedded clause} (the \isi{agent} desired to do something). In sentence \REF{ex:syntax:383}, the subject of the \isi{matrix clause} is an understood third party, whereas the subject of the \isi{embedded clause} is the speaker (1\textsc{sg)}.

\ea%383
    \label{ex:syntax:383}
          \textit{Nï mana} \textbf{\textit{nakap}} \textit{nï mïnjan mat: …}\\
\gll    {[nï}    {ma-na]}    \textbf{na-kï-p}    nï    mïnja=n    ma=ta\\
    {[1\textsc{sg}}  go-\textsc{irr]}  \textsc{detr-}say-\textsc{pfv}  \textsc{1sg}  speech=\textsc{obl}  3\textsc{sg.obj}=say\\
\glt `[Wala] wanted me to go, but I told him: …’ [ulwa037\_35:55]
\z

The form [nakap], as seen in \REF{ex:syntax:383}, can be used regardless of \isi{TAM} distinctions: thus, for example, many sentences with this form have \isi{imperfective} force, despite the typically perfective-marking \isi{suffix} \textit{-p} \textsc{‘pfv’}. Moreover, the form [nakap] may be used without any \isi{conditional} marking, even in the \isi{protasis} of a \isi{conditional sentence}, as in \REF{ex:syntax:384}.

\ea%384
    \label{ex:syntax:384}
          \textit{Nan nïkïna} \textbf{\textit{nakap}} \textit{na kali nïwatlïta.}\\
\gll    {[na=n}    {nï=kï-na]}    \textbf{na-kï-p}    na    kali     nï=wat-lï-ta\\
    {[talk=\textsc{obl}}  \textsc{1sg=}say-\textsc{irr]}  \textsc{detr-}say-\textsc{pfv}  talk  send    1\textsc{sg}=atop-put-\textsc{cond}\\
\glt `If [you] wanted to talk to me, [then you] should have sent a message to me.’ [ulwa014\_14:24]
\z

In example \REF{ex:syntax:385}, the \isi{conditional} marker \textit{-ta} \textsc{‘cond’} occurs within the \isi{embedded clause}, instead of being affixed to the \isi{matrix clause} verb form [nakap] (cf. issues of \isi{scope} in \sectref{sec:13.3.4}).

\ea%385
    \label{ex:syntax:385}
          \textit{Wutï munta lunda} \textbf{\textit{nakap}} \textit{…}\\
\gll    {[wutï}  mune-ta    {lu-nda]}    \textbf{na-kï-p}\\
    {[leg}  throw-\textsc{cond}  put-\textsc{irr]}  \textsc{detr}{}-say-\textsc{pfv}\\
\glt `If you want to throw your legs around …’ (i.e., play sports) [ulwa032\_34:42]
\z

  The \isi{semantic} connection between verbs of speaking or thinking and verbs of desiring is understandable. Often, when one wants something, one talks about it (and almost certainly thinks about it). While [nakap] seems to be a \isi{fossilized} form used in \isi{desiderative} clauses, it is nevertheless possible to use other verbs of speaking to express desires, as in the \isi{desiderative} sentence given in \REF{ex:syntax:386}, which uses the verb \textit{ta-} ‘say’.

\newpage

\ea%386
    \label{ex:syntax:386}
          \textit{Nul mana} \textbf{\textit{nate}}.\\
\gll {[nï=ul}  {ma-na]}      \textbf{na-ta-e}\\
    {[1\textsc{sg}=with}   go-\textsc{irr]}  \textsc{detr-}say-\textsc{dep}\\
\glt `[He] wanted to go with me.’ [ulwa014†]
\z

In addition to these \is{biclausal construction} biclausal \isi{desiderative} constructions, it is possible to express a desire in a single clause, simply by using an \isi{irrealis} verb form. In such instances, it is not necessarily clear whether the person desiring the event encoded by the verb is the subject of the verb, the speaker of the clause, or both. In examples \REF{ex:syntax:387} through \REF{ex:syntax:391}, all translated with ‘want’, the \isi{irrealis} verb forms could, in other contexts, impart other meanings (e.g., ‘will’, ‘should’, ‘can’, etc.; see \sectref{sec:4.6}).

\ea%387
    \label{ex:syntax:387}
          \textit{Nï lamndu} \textbf{\textit{mawalinda}}.\\
\gll nï    lamndu  ma=wali-\textbf{nda}\\
    1\textsc{sg}  pig      3\textsc{sg.obj}=hit-\textsc{irr}\\
\glt `I want to kill a pig.’ [elicited]
\z


\ea%388
    \label{ex:syntax:388}
          \textit{Nï awal we} \textbf{\textit{landa}}.\\
\gll nï    awal    we    la-\textbf{nda}\\
    1\textsc{sg}  yesterday  sago  eat-\textsc{irr}\\
\glt `I wanted to eat sago yesterday.’ [elicited]
\z

\ea%389
    \label{ex:syntax:389}
          \textit{An inamba sokoy} \textbf{\textit{inda}}.\\
\gll an      inamba[=n]  sokoy    in-\textbf{nda}\\
    1\textsc{pl.excl}  money[=\textsc{obl]}  tobacco  get-\textsc{irr}\\
\glt `We want to buy tobacco.’ [ulwa037\_52:59]
\z

\ea%390
    \label{ex:syntax:390}
          \textit{Apa} \textbf{\textit{mana}} \textit{i liwe umbu anïm nga mas.}\\
\gll    apa    ma-\textbf{na}  i    li-aw-e          numbu  anïm  nga ma=asa\\
    house  go-\textsc{irr}  go.\textsc{pfv}  down{}-put.\textsc{ipfv-dep}  post  fork  \textsc{sg.prox}    3\textsc{sg.obj}=hit\\
\glt `[He] wanted to go home, but [he] went and fell, and the fork of the post pierced him.’ [ulwa021\_00:10]
\z

\ea%391
    \label{ex:syntax:391}
          \textit{Nï ango wa} \textbf{\textit{lunda}}.\\
\gll nï    ango  wa    lo-\textbf{nda}\\
    1\textsc{sg}  \textsc{neg}  village  go-\textsc{irr}\\
\glt `I don’t want to go around in villages.’ [ulwa037\_49:14]
\z

Example \REF{ex:syntax:391} illustrates a \isi{negative desire}. Often, to express that something is not desired, the verb \textit{kamb-} ‘shun’ (\sectref{sec:2.1.2}) is used. It may take either the \isi{imperfective} (or unmarked) form \textit{kam} ‘shun [\textsc{ipfv}]’, the \isi{imperfective} form \textit{kambe} ‘shun [\textsc{ipfv]’}, or the \isi{perfective} form \textit{kamp} ‘shun [\textsc{pfv]’}, as shown in \REF{ex:syntax:392}.

\ea%392
    \label{ex:syntax:392}
          \textit{Nï \textbf{kam(be/p)}!}\\
\gll    nï    \textbf{kamb({}-e/-p)}\\
    1\textsc{sg}  shun(-\textsc{ipfv}/-\textsc{pfv)}\\
\glt    (a) ‘I don’t want to!’

    (b) ‘I don’t want it!’ [elicited]
\z

To express that an object is desired, Ulwa simply employs the verb \textit{tï-} ‘take’ in the \isi{irrealis} \isi{mood}. After all, to say, for example, that one ‘wants a spear’ means that one ‘wants to take (i.e., obtain, have) a spear’. In other words, a proclamation such as ‘I would take’ links, by inference, to ‘I want’. This is illustrated in \REF{ex:syntax:393} and \REF{ex:syntax:394}.

\ea%393
    \label{ex:syntax:393}
          \textit{Nï mana} \textbf{\textit{tïna}}.\\
\gll nï    mana  \textbf{tï-na}\\
    1\textsc{sg}  spear  take-\textsc{irr}\\
\glt `I want a spear.’ [elicited]
\z

\ea%394
    \label{ex:syntax:394}
          \textit{Nï awal mana akïnaka} \textbf{\textit{tïna}}.\\
\gll nï    awal    mana  akïnaka  \textbf{tï-na}\\
    1\textsc{sg}  yesterday  spear  new    take-\textsc{irr}\\
\glt `I wanted a new spear yesterday.’ [elicited]
\z

Often the distinction between ‘want’ and ‘need’ in such instances is not explicit. Sentence \REF{ex:syntax:395} thus may be translated variably.

\ea%395
    \label{ex:syntax:395}
          \textit{Nï mana akïnaka} \textbf{\textit{tïna}}.\\
\gll nï    mana  akïnaka  \textbf{tï-na}\\
    1\textsc{sg}  spear  new    take-\textsc{irr}\\
\glt    (a) ‘I want a new spear.’

    (b) ‘I need a new spear.’ [elicited]
\z

\is{syntax|)}
\is{transitivity|)}
\is{increased transitivity|)}
\is{valency|)}
\is{valency expansion|)}
\is{desiderative|)}