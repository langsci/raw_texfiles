\addchap{\lsAcknowledgementTitle}
First I would like to thank the speakers of Ulwa, without whose help and indefatigable patience this book would not have been possible. In particular, I owe very much to the Ambata family in Manu village: Thomas, Betty, Bradley, Mel, Valentine, Davis, Eugene, and Shannon. I must also thank everyone else in Manu who ensured that I was well-fed and safe during my time there: Carobim, Cecilia, Chris, David, Elias, Fidelma, John G., John K., Janet, Jayson, Justin, Lucy, Nigel, Paula, Philo, Ruben, Samuel, Trisha, and Yambin, among others. Thanks also to Soti and Sylvia for their hospitality in Angoram.

Many people helped me learn about the Ulwa language; some of them are: Francis Ambata, Lucy Ambata, Alus Amombi, Michael Amombi, Kanisus Binda, Joseph Bram, Wesley Dovina, David Gambri, Gema Gambri, Janet Gambri, \linebreak Samuel Gambri, Corona Gami, Christina Jomia, Danny Kanangula, Chris \linebreak Kapos, Rosa Kapos, Yanapi Kua, Kunam Malaku, Elias Mangeme, Arnold \linebreak Mangri, Ginam Mapipa, Paulina Mapul, Stephen Mawipa, John Morangi, Linneth Nungami, Mathew Sango, Cecilia Sikimba, Nicholas Sikimba, Philip Siwingin, Philo Tatu, Gweni Tungun, Elias Usimari, Otto Wandangin, Morris Womel, and Leo Yokombla. Above all, Mr. Thomas Ambata has been an exceptional teacher, as well as a good friend.


I also wish to thank the teachers who have guided and encouraged me along my academic career, especially John Van Sickle at Brooklyn College and Michael Janda at the Universität Münster. At the University of Hawaiʻi I have been fortunate to have had a number of amazing teachers. In particular, I must thank Lyle Campbell, the best of all possible doctoral advisors. I must also thank my other committee members at UH, a descriptive-linguistics dream team: Gary Holton, William O’Grady, and the late Robert Blust. Thanks to Alex Golub for his anthropological perspective. Thanks also to Jen Kanda and Nora Lum at UH for all their help and deft handling of administrative matters. And many thanks to William A. Foley at the University of Sydney, who helped me from the very start, guiding and advising me before I ever set foot in the Sepik.

After Hawaiʻi I was fortunate to have found my way to the Max Planck Institute in Germany, first in Jena, then in Leipzig, and I wish to thank Russell D. Gray for his continued support and encouragement.

A number of colleagues have improved my understanding of Ulwa thanks to our linguistic discussions; among them are Joseph Brooks, Don Killian, Timothy Usher, and Martha Wade.

This book has benefited greatly from Martin Haspelmath’s sage editorial guidance. I must also thank Sebastian Nordhoff and Yanru Lu at Language Science Press for their keen help in technical matters.

Finally, I am grateful for the generous organizations that have helped fund my research: the Bilinski Educational Foundation, the Firebird Foundation, the Endangered Languages Documentation Programme (ELDP), and the Department of Linguistic and Cultural Evolution at the Max Planck Institute for the Science of Human History.
