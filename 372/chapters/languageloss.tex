\chapter{The structural consequences of language loss}\label{sec:15}

\is{language loss|(}

This chapter provides some hypotheses about the nature of \isi{contact}-induced language change in Ulwa.\footnote{See \citet{CampbellMuntzel1989} for discussion and examples of changes in \isi{obsolescent} languages.} These must all remain merely hypotheses, since, with hardly any documentation of the language prior to 2015, it is impossible to know with any certainty what the structure of the language was like in the past. Still, based in part on older speakers’ grammaticality judgements of younger speakers’ speech and in part on suspicious structural similarities to (or clear \isi{borrow}ings from) \ili{Tok Pisin}, I outline in this chapter some of the most significant changes that Ulwa likely has faced and is facing in light of rapid \isi{language loss}.

  Although Ulwa has likely always been in \isi{contact} with other languages and has probably undergone changes due to areal influences, I assume that the greatest external force affecting the language has been \ili{Tok Pisin}, which first came to the Ulwa community in the twentieth century. It has become the first language of all ethnic Ulwas and is the only language of the majority of ethnic Ulwas.

\is{language loss|)}

\section{Lexical changes}\label{sec:15.1}

\is{lexicon|(}
\is{lexical change|(}
\is{language loss|(}

\is{intersentential code-switching}
\is{intrasentential code-switching}

The most obvious linguistic effect of \ili{Tok Pisin} can be seen in the \isi{lexicon}. It is very common for Ulwa speakers to infuse their speech with \ili{Tok Pisin} \isi{loanword}s. Sometimes these \isi{borrow}ings are clearly motivated by the lack of native vocabulary for certain concepts (e.g., \ili{Tok Pisin} \textit{balus} ‘airplane’, \textit{hausik} ‘hospital’, etc.). Often, however, speakers use \ili{Tok Pisin} words simply because they come more readily to mind or because they do not know the Ulwa word. In some instances, it may be better to view the use of \ili{Tok Pisin} words as a form of \isi{code-switching}, as indeed some speakers switch between Ulwa and \ili{Tok Pisin} both intersententially and intrasententially. \ili{Tok Pisin} words are generally adapted to accommodate the \isi{phonology} of Ulwa (\sectref{sec:15.2}).

\is{language loss|)}
\is{lexical change|)}
\is{lexicon|)}

\section{Phonological changes}\label{sec:15.2}

\is{phonology|(}
\is{phonological change|(}
\is{language loss|(}

Even though the number of native \isi{lexical} items used in speech appears to have diminished, even among the oldest speakers, Ulwa’s native \isi{phonology} seems still to be intact. In other words, there are not any indications that older speakers have \isi{shift}ed their \is{phonology} phonologies due to influences from \ili{Tok Pisin}. In fact, many speakers impose Ulwa \isi{phonotactics} on their variety of \ili{Tok Pisin}. For example, older Ulwa speakers often produce [l] for /r/ in \ili{Tok Pisin} words, as in \REF{ex:loss:1a}.

\ea%1a
    \label{ex:loss:1a}
     Pronunciation of \ili{Tok Pisin} /r/ as [l] among some Ulwa speakers
\begin{tabbing}
{([lawsim])} \= {(for \ili{Tok Pisin} \textit{rausim} ‘remove’)}\kill
{[kal]} \> for \ili{Tok Pisin} \textit{kar} ‘car’\\
{[lalim]} \> for \ili{Tok Pisin} \textit{larim} ‘let’\\
{[lawsim]}  \>  for \ili{Tok Pisin} \textit{rausim} ‘remove’
  \end{tabbing}
\z

Ulwa speakers often also \isi{prenasalize} all \isi{voiced} \isi{stop}s in \ili{Tok Pisin}, as in \REF{ex:loss:1b}.

\ea%1b
    \label{ex:loss:1b}
     Pronunciation of \ili{Tok Pisin} \isi{voiced} \isi{stop}s as \isi{prenasalized} among some Ulwa speakers
\begin{tabbing}
    {([ngutpela])} \= {(for \ili{Tok Pisin} \textit{gutpela} ‘good’)}\kill
    {[mbilas]} \> for \ili{Tok Pisin} \textit{bilas} ‘decoration’\\
   {[ndok]} \> for \ili{Tok Pisin} \textit{dok} ‘dog’\\
    {[ngutpela]}  \>  for \ili{Tok Pisin} \textit{gutpela} ‘good’
  \end{tabbing}
\z

On the other hand, many younger Ulwas, who are generally non-speakers but perhaps know a few words, do not seem to have acquired the \isi{phonology} of Ulwa. They often fail to \isi{prenasalize} word-initial \isi{voiced} \isi{stop}s, as this is \isi{phonotactic}ally prohibited in \ili{Tok Pisin}, their first language. Furthermore, when \isi{prenasalized} \isi{voiced} \isi{stop}s occur intervocalically, they syllabify the word such that the \isi{nasal} gesture belongs to the \isi{coda} of one \isi{syllable} and the \isi{stop} gesture belongs to the \isi{onset} of the following \isi{syllable}. In other words, they fail to treat the \isi{prenasalized} \isi{voiced} \isi{stop} as a single segment.

\is{language loss|)}
\is{phonological change|)}
\is{phonology|)}

\section{Morphological changes}\label{sec:15.3}

\is{morphological change|(}
\is{morphology|(}
\is{language loss|(}

Ulwa also seems to be undergoing morphological changes due to \isi{contact} with \ili{Tok Pisin} and to \isi{language loss} in general. For example, speakers may be less likely to use the appropriate \isi{TAM} \isi{suffix}es on verbs, or they may omit such \isi{verbal morphology} entirely. Whereas Ulwa exhibits a mandatory three-way distinction in \isi{TAM}, manifested by verbal \isi{suffix}es, \ili{Tok Pisin} does not \isi{inflect} verbs at all for such grammatical categories. As speakers \isi{shift} more to \ili{Tok Pisin}, they observe these distinctions in Ulwa less. Similarly, when speakers do make use of a \isi{perfective} or \isi{irrealis} \isi{suffix}, they sometimes use an unexpected \isi{stem}-final \isi{vowel}. The underlying \isi{stem}-final \isi{vowel} of many verbs is never seen in the \isi{imperfective} form (\sectref{sec:4.2}), and it may be the case that younger speakers are not acquiring this underlying form, instead creating \isi{perfective} and \isi{irrealis} forms based on \isi{analogy} to other verbs. For example, a speaker might say [indep] for \textit{indap} ‘walk [\textsc{pfv}]’, unaware that the underlying root is /inda/-, since this form never appears unaffixed as a surface form.

  While the aforementioned changes may be viewed as reflecting a general reduction of grammatical forms, there are also \isi{morphosyntactic} innovations that appear to be due to \isi{contact} and \isi{language shift} as well. These are mostly \isi{calque}s from \ili{Tok Pisin}, a highly \isi{analytic} language. Thus, even though Ulwa has the ability to express several \isi{aspect}ual and \isi{modal} meanings through its \isi{verbal morphology}, speakers have begun to incorporate periphrases to express such distinctions. These may be used in place of or in addition to the more \isi{synthetic} Ulwa structures.\footnote{This appears to be in keeping with the hypothesis that speakers in \isi{endangered} language situations tend to rely more on \isi{analytic} constructions, replacing or reducing the number of their \isi{morphological}ly marked (i.e., \isi{synthetic}) alternatives \citep[97]{Andersen1982}.}

\is{calque}

  A very common form of this \isi{morphosyntactic} calquing from \ili{Tok Pisin} is the use of the \isi{locative verb} form \textit{wap} ‘be.\textsc{pst}’ as an \isi{auxiliary verb} following the main verb to signal \isi{continuous}, \isi{progressive}, or \isi{habitual} \isi{aspect}. This may have been derived thanks to the very similar role of \textit{stap} ‘be’ in \ili{Tok Pisin}. Not only does the Ulwa \isi{suffix} parallel the \ili{Tok Pisin} word in meaning (both are used in \isi{locative clause}s), but it also resembles the \ili{Tok Pisin} word \isi{phonological}ly (the two forms rhyme). In example \REF{ex:loss:1}, the \isi{perfective} form of the verb is used, but with the form \textit{wap} ‘be.\textsc{pst’} following it to signal \isi{continuous} action in the past.

\ea%1
    \label{ex:loss:1}
            \textit{Iya nï imba pe \textbf{i wap}.}\\
\gll iya  nï    imba  p-e    \textbf{i}  \textbf{wap}\\
    yes  1\textsc{sg}  night  be-\textsc{dep}  go.\textsc{pfv}  be.\textsc{pst}\\
\glt `Yes, I was going at night.’ [ulwa037\_01:26]
\z

In the immediately following sentence, however, the speaker “corrects” to the more traditional means of expressing \isi{past} \isi{continuous} action, which is an \isi{imperfective} verb form \REF{ex:loss:2}.

\newpage

\ea%2
    \label{ex:loss:2}
            \textit{Imba pe \textbf{mane} …}\\
\gll imba  p-e    \textbf{ma-n-e}\\
    night  be-\textsc{dep}  go-\textsc{ipfv-dep}\\
\glt `[I] was going at night and …’ [ulwa037\_01:29]
\z

Sentences such as \REF{ex:loss:1} are thus likely influenced by structures in \ili{Tok Pisin}. To express \isi{continuous} action, \ili{Tok Pisin} employs the marker \textit{i stap} (the \isi{predicate marker} \textit{i} ‘\textsc{pred’} + \textit{stap} ‘be’). Moreover, the chance similarity between the \ili{Tok Pisin} \isi{predicate marker} \textit{i} ‘\textsc{pred’} and the Ulwa verb \textit{i} ‘go.\textsc{pfv}’ has perhaps further influenced the adoption of this construction.

  In \REF{ex:loss:3}, \textit{wap} ‘be.\textsc{pst’} serves a \isi{habitual} function. Here it follows a verb that is already marked as \isi{imperfective}.

\ea%3
    \label{ex:loss:3}
           \textit{Kapos wapata anda matï inde \textit{wap}.}\\
\gll Kapos  wapata    anda    ma=tï      inda-e    \textbf{wap}\\
    [name]  old      \textsc{sg.dist}  3\textsc{sg.obj}=take  walk-\textsc{ipfv}  be.\textsc{pst}\\
\glt `That old [man] Kapos used to carry it.’ [ulwa014\_45:35]
\z

Some speakers of Ulwa also make use of \isi{iconic} repetition of verbs to signal \isi{iterative} (or, occasionally, \isi{durative}) \isi{aspect}. This, too, seems influenced from \ili{Tok Pisin}, in which verbs may be repeated to signal \isi{iterative} action (although this could, of course, also reflect a general tendency in languages towards \isi{iconic} representation of iterated activity). Thus, repeated verbs may be used to signal repetitive \REF{ex:loss:4} or \isi{durative} \REF{ex:loss:5} action.

\ea%4
    \label{ex:loss:4}
            \textit{\textbf{Ndalep ndalep ndalep} yawt tï nduwep.}\\
\gll    \textbf{ndï=ale-p}      \textbf{ndï=ale-p}      \textbf{ndï=ale-p}      yawt tï    ndï=we-[u-]p\\
    3\textsc{pl}=scrape-\textsc{pfv}  3\textsc{pl}=scrape-\textsc{pfv}  \textsc{3pl}=scrape-\textsc{pfv}  machete    take  3\textsc{pl}=cut-[put-]\textsc{pfv}\\
\glt `[They] scraped and scraped and scraped them, [and then] got machetes and cut them.’ [ulwa014\_59:39]
\z

\ea%5
    \label{ex:loss:5}
            \textit{Kwa Yalamba wa wap \textbf{mawap mawap mawap}.}\\
\gll kwa  Yalamba  wa    wap  \textbf{ma=wap}      \textbf{ma=wap}     \textbf{ma=wap}\\
    just    Korokopa  village  be.\textsc{pst}  \textsc{3sg.obj=}be.\textsc{pst}  \textsc{3sg.obj=}be.\textsc{ps}    \textsc{3sg.obj=}be.\textsc{pst}\\
\glt `[He] just stayed at Korokopa village for quite some time.’ [ulwa037\_19:48]
\z

A form of the verb ‘go’ (often either the root \textit{ma-} ‘go’ alone or the \isi{suppletive} \isi{perfective} form \textit{i} ‘go.\textsc{pfv’} with the \isi{detransitivizing} \isi{prefix}, i.e., [nay] or [ne]) may be used to show \isi{iterative} or \isi{durative} \isi{aspect} as well. This, too, parallels some uses of the \ili{Tok Pisin} \isi{progressive} marker \textit{i go}, and is illustrated in \REF{ex:loss:6} and \REF{ex:loss:7}.

\ea%6
    \label{ex:loss:6}
            \textit{Mï minyam tï ambïlïp \textbf{naye}.}\\
\gll mï      minyam  tï    ambï=lï-p      \textbf{na-i-e}\\
    3\textsc{sg.subj}  feces    take  \textsc{sg.refl}=put-\textsc{pfv}  \textsc{detr}{}-go.\textsc{pfv-dep}\\
\glt `He kept soiling himself.’ [ulwa036\_00:09]
\z

\ea%7
    \label{ex:loss:7}
            \textit{Wopata mapen makape wombïn ndïn ne} \textbf{\textit{i}}.\\
\gll Wopata  ma=p-en      maka=p-e    wombïn  ndï=n     ni-e    \textbf{i}\\
    [place]    3\textsc{sg.obj}=be\textsc{{}-nmlz} thus=be\textsc{{}-dep} work    3\textsc{pl=obl}    act-\textsc{dep}  go.\textsc{pfv}\\
\glt `Those who lived in Wopata were like this, doing work, for some time.’ [ulwa018\_04:00]
\z

As in the \ili{Tok Pisin} \isi{progressive} \textit{i go} construction, the ‘go’ element in Ulwa may be repeated, even multiple times, as in \REF{ex:loss:8} and \REF{ex:loss:9}.

\ea%8
    \label{ex:loss:8}
            \textit{Una awal matane \textbf{nay nay nay nay}.}\\
\gll unan    awal    ma=ta-n-e          \textbf{na-i} \textbf{na-i}      \textbf{na-i}      \textbf{na-i}\\
    1\textsc{pl.incl}  yesterday  3\textsc{sg.obj}=say-\textsc{ipfv-dep}  \textsc{detr-}go.\textsc{pfv}    \textsc{detr}{}-go.\textsc{pfv}  \textsc{detr}{}-go.\textsc{pfv}  \textsc{detr}{}-go.\textsc{pfv}\\
\glt `We kept discussing it yesterday, over and over.’ [ulwa037\_08:21]
\z

\ea%9
    \label{ex:loss:9}
            \textit{Una wombïn nita \textbf{ma ma ma} …}\\
\gll    unan    wombïn=n  ni-ta    \textbf{ma}  \textbf{ma}  \textbf{ma}\\
    1\textsc{pl.incl}  work=\textsc{obl}  act\textsc{{}-cond} go  go  go\\
\glt `And when we work on and on …’ [ulwa037\_24:24]
\z

Also likely \isi{calque}d from \ili{Tok Pisin} are some \isi{idiom}atic expressions, such as using the equivalent of \textit{stap na kam} ‘be (in a place) and come’ to express the notion of coming from a place (here, too, employing \textit{wap} ‘be.\textsc{pst}’). The Ulwa equivalent of the \ili{Tok Pisin} construction may be seen in \REF{ex:loss:10}.

\is{language loss|)}
\is{morphology|)}
\is{morphological change|)}

\newpage

\ea%10
    \label{ex:loss:10}
\is{language loss}
\is{morphology}
\is{morphological change}
          \textit{Nambi Madang \textbf{wap mbiye} nï maka Wombasame mï Wonmelma mintap.}\\
\gll    nï-ambi  Madang  \textbf{wap}  \textbf{mbï-i-e}      nï    maka Wombasame  mï      Wonmelma    min=ta-p\\
    1\textsc{sg-top}  [place]    be.\textsc{pst}  here-go.\textsc{pfv-dep}  \textsc{1sg}  thus    [name]      \textsc{3sg.subj}  [name]      \textsc{3du}=say-\textsc{pfv}\\
\glt `As for me, when I came from Madang, I talked about Wombasame and Wonmelma.’ [ulwa014†]
\z

\section{Syntactic changes}\label{sec:15.4}

\is{syntactic change|(}
\is{syntax|(}
\is{language loss|(}

Speakers may also be employing fewer and fewer syntactic structures in Ulwa. Thus, the more complex constructions in the language, such as \isi{relative clause}s (\sectref{sec:12.3}) and \isi{passive} constructions (\sectref{sec:13.7}), may be avoided entirely by some speakers, or they may be simply unknown to them.

  Other syntactic changes may be due specifically to \ili{Tok Pisin} influence. Although the order of basic clausal constituents does not seem to have been affected by the prevalence of \ili{Tok Pisin} (i.e., Ulwa’s SOV \isi{word order} has not \isi{shift}ed towards \ili{Tok Pisin}’s SVO \isi{word order}), the structure of NPs may be changing due to \ili{Tok Pisin} influence, as some speakers occasionally place \isi{adjective}s before their nominal \isi{head}s (following \ili{Tok Pisin} \isi{syntax}) instead of after them (as more traditionally in Ulwa) (\sectref{sec:5.4}).

\is{language loss|)}
\is{syntax|)}
\is{syntactic change|)}



\section{Borrowed function words}\label{sec:15.5}

\is{borrowing|(}
\is{function word|(}
\is{language loss|(}
\is{loanword|(}

In addition to grammatical \isi{calque}s such as those detailed in \sectref{sec:15.3}, speakers of Ulwa frequently employ \ili{Tok Pisin} \isi{loanword}s for grammatical functions. For example, the borrowed \isi{coordinating conjunction}s \textit{na} ‘and’ \REF{ex:loss:11} and \textit{o} ‘or’ \REF{ex:loss:12} are commonly used.

\ea%11
    \label{ex:loss:11}
          \textit{Mat lïp \textbf{na} amun wolka kwa ngol ne.}\\
\gll    ma=tï      lï-p      \textbf{na}  amun  wolka  kwa  nga=ul     ni-e\\
    3\textsc{sg.obj}=take  put-\textsc{pfv}  and  now  again  one    \textsc{sg.prox}=with    act-\textsc{ipfv}\\
\glt `[I] left it and now in turn [I] am making this one.’ (\textit{na} < TP \textit{na} ‘and’) [ulwa015\_01:41]
\z

\newpage

\ea%12
    \label{ex:loss:12}
          \textit{U imba pe i \textbf{o} ane pe i?}\\
\gll    \textit{u}    imba  p-e    i    \textbf{o}  ane  p-e    i\\
    2\textsc{sg}  night  be\textsc{{}-dep}  go.\textsc{pfv}  or  sun  be\textsc{{}-dep} go.\textsc{pfv}\\
\glt `Did you go at night or go during the day?’ (\textit{o} < TP \textit{o} ‘or’) [ulwa037\_01:25]
\z

The adoption of the \ili{Tok Pisin} \isi{conjunction}s \textit{na} ‘and’ and \textit{o} ‘or’ can be seen as filling a gap in the Ulwa \isi{lexicon}, since, prior to \isi{contact} with \ili{Tok Pisin}, the language did not have any word used to \isi{coordinate} \isi{phrase}s or clauses. This was accomplished rather through \isi{juxtaposition} (\sectref{sec:12.1}).

Some commonly borrowed function words from \ili{Tok Pisin} are given in  \REF{ex:loss:10}.

\ea%10a
    \label{ex:loss:10a}
     Borrowed function words from \ili{Tok Pisin}
\begin{tabbing}
{(\textit{nongut})} \= {(‘when, whenever’)} \= {(<)} \= {(\textit{taim})}\kill
{\textit{i}} \> {‘\textsc{pred}’} \> {<} \> {\textit{i}}\\
{\textit{iken}} \> {‘may, can’} \> {<} \> {\textit{i ken}}\\
{\textit{layk}} \> {‘be about to’} \> {<} \> {\textit{laik}}\\
{\textit{mas}} \> {‘should, must’} \> {<} \> {\textit{mas}}\\
{\textit{maski}} \> {‘although’} \> {<} \> {\textit{maski}}\\
{\textit{mbay}} \> {‘will’} \> {<} \> {\textit{bai}}\\
{\textit{na}} \> {‘and’} \> {<} \> {\textit{na}}\\
{\textit{nongut}} \> {‘lest’} \> {<} \> {\textit{nogut}}\\
{\textit{o}} \> {‘or’} \> {<} \> {\textit{o}}\\
{\textit{sapos}} \> {‘if’} \> {<} \> {\textit{sapos}}\\
{\textit{sawe}} \> {‘\textsc{hab}’} \> {<} \> {\textit{save}}\\
{\textit{tasol}} \> {‘but’} \> {<} \> {\textit{tasol}}\\
{\textit{tem}} \> {‘when, whenever’} \> {<} \> {\textit{taim}}
\end{tabbing}
\z



\isi{Loan} \isi{subordinator}s from \ili{Tok Pisin} include \textit{maski} ‘although’ and \textit{taim} ‘when, whenever’. These function words occur at the beginning of a \isi{dependent clause}. In traditional forms of Ulwa, the \isi{dependent marker} \textit{-e} ‘\textsc{dep}’ would have sufficed to convey such \isi{concessive} or \isi{temporal} notions. With these words, however, the \isi{dependent marker} may be used as well \REF{ex:loss:13}, or it may be omitted \REF{ex:loss:14}.

\ea%13
    \label{ex:loss:13}
          \textit{\textbf{Tem} ndï ndïnji ngin motop inim \textbf{pe} ambana nungol kotïne mbay an malanda.}\\
\gll    \textbf{tem}  ndï  ndï-nji    ngin  ma=top      inim  [lï-]p-\textbf{e}     wambana  nungol  ko=tï-n-e        mbay  an ma=la-nda\\
    time  \textsc{3pl}  3\textsc{pl-poss}  net    \textsc{3sg.obj}=throw  water  [put-]\textsc{pfv-dep}    fish    child  \textsc{indf}=take-\textsc{pfv-dep}  will  1\textsc{pl.excl}    \textsc{3sg.obj}=eat-\textsc{irr}\\
\glt `Whenever they threw their net into the water and got a small fish, [then] we would eat it.’ (\textit{tem} < TP \textit{taim} ‘time, when’, \textit{mbay} < TP \textit{bai} ‘will’) [ulwa032\_01:36]
\z

\ea%14
    \label{ex:loss:14}
          \textit{\textbf{Maski} u ma awlop maka lowonda.}\\
\gll    \textbf{maski}     u    ma  awlop  ma=ka      lo-wo-nda\\
    although  \textsc{2sg}  go  in.vain  3\textsc{sg.obj}=at  \textsc{irr}{}-sleep-\textsc{irr}\\
\glt `Even if you go and get lost, [you] can sleep there.’ (\textit{maski} < TP \textit{maski} ‘although’) [ulwa029\_02:54]
\z

Example \REF{ex:loss:14} also illustrates the use of the \ili{Tok Pisin} \isi{auxiliary verb} \textit{bai} ‘will’. \ili{Tok Pisin} \isi{modal} verbs such as \textit{bai} ‘will’, \textit{iken} ‘may, can’, or \textit{mas} ‘should, must’ may occur along with \isi{irrealis}-marked Ulwa verbs. The \ili{Tok Pisin} verb \textit{save} \linebreak ‘know’, which can function as an \isi{auxiliary} in that language to mark \isi{habitual} \linebreak \isi{aspect}, is also a common \isi{loanword} in Ulwa, usually used along with an \linebreak imperfective-marked verb, the traditional means of marking \isi{habitual} \isi{aspect} in Ulwa. Examples of the borrowed \ili{Tok Pisin} function words \textit{mas} ‘should, must’ \REF{ex:loss:15}, \textit{iken} ‘may, can’, \REF{ex:loss:16}, and \textit{save} ‘\textsc{hab’} \textsc{(1}7) are commonly found in texts.

\ea%15
    \label{ex:loss:15}
          \textit{U \textbf{mas} matan!}\\
\gll    u    \textbf{mas}  ma=ta-n[a]\\
    2\textsc{sg}  must  3\textsc{sg.obj}=say-\textsc{irr}\\
\glt `You should tell it!’\footnote{The form [matan] contains either an \isi{elide}d form of the \isi{irrealis} \isi{suffix} \textit{-na} ‘\textsc{irr}’, or else it contains the \isi{imperative} \isi{suffix} \textit{-n} ‘\textsc{imp}’.} (\textit{mas} < TP \textit{mas} ‘should, must’) [ulwa032\_05:47]
\z

\ea%16
    \label{ex:loss:16}
          \textit{Un \textbf{iken} mawan utap ma ndïn mankïna.}\\
\gll    un  \textbf{iken}  ma=wan      uta-p    ma  ndï=n ma=nïkï-na\\
    2\textsc{pl}  may  3\textsc{sg.obj=}above  grind-\textsc{pfv}  go  3\textsc{pl=obl}    3\textsc{sg.obj}=dig-\textsc{irr}\\
\glt `You can clear over it and plant them there.’ (\textit{iken} < TP \textit{i ken} ‘may, can’) [ulwa042\_04:16]
\z

\ea%17
    \label{ex:loss:17}
          \textit{Nambi nï \textbf{sawe} inim lope.}\\
\gll    nï-ambi  nï    \textbf{sawe}  inim  lopo-e\\
    1\textsc{sg-top}  \textsc{1sg}  \textsc{hab}  water  wash-\textsc{ipfv}\\
\glt `As for me, I bathe.’ (\textit{sawe} < TP \textit{save} ‘know’; \isi{habitual} marker) [ulwa014†]
\z

Example \REF{ex:loss:16} illustrates the use of the \ili{Tok Pisin} \isi{predicate marker} \textit{i} ‘\textsc{pred’}, here probably just adopted along with the verb \textit{ken} ‘can, may’ -- that is, reanalyzed as a unitary \isi{auxiliary verb} [iken]. The \isi{predicate marker} \textit{i} ‘\textsc{pred’} does appear elsewhere in Ulwa discourse, but, due to its \isi{homophony} with the \isi{suppletive} \isi{perfective} form of the verb ‘go’, it is often difficult to determine whether the form [i] is being used as the \isi{predicate marker} or as a \isi{calque} of \ili{Tok Pisin} \textit{go} ‘go’, which is used to achieve similar grammatical functions.

\is{loanword|)}
\is{language loss|)}
\is{function word|)}
\is{borrowing|)}


\section{Detransitivization of loan verbs}\label{sec:15.6}

\is{detransitivization|(}
\is{loan|(}
\is{language loss|(}

When \ili{Tok Pisin} verbs are \isi{borrow}ed into Ulwa, they are typically treated as \isi{intransitive}, regardless of their \isi{semantics}. The logical object of the verb is not indexed by an \isi{object marker}, but instead appears as the \isi{head} of an \isi{oblique} \isi{phrase} marked by the \isi{oblique marker} \textit{=n} ‘\textsc{obl}’.\footnote{This occurs even with \ili{Tok Pisin} verbs that are marked with the \isi{transitive} \isi{suffix} \textit{-im}, which of course is not a \isi{transitive} marker in Ulwa.} The \ili{Tok Pisin} verb is usually employed without any \isi{TAM} marking, as in \ili{Tok Pisin}. Examples \REF{ex:loss:18}, \REF{ex:loss:19}, and \REF{ex:loss:20} illustrate the use of \ili{Tok Pisin} loan verbs without any \isi{suffix} or \isi{auxiliary verb}. Note the absence of any grammatical object and the use of the \isi{oblique marker} \textit{=n} `\textsc{obl}’.

\ea%18
    \label{ex:loss:18}
          \textit{Ndï i awnï \textbf{tambuwim}.}\\
\gll ndï    i    aw=nï      \textbf{tambuwim}\\
    3\textsc{pl}    go.\textsc{pfv}  betel.nut=\textsc{obl}  taboo\\
\glt `They went and forbade [taking] the betel nut.’ (\textit{tambuiwim} = TP \textit{tambuim}) [ulwa014†]
\z

\ea%19
    \label{ex:loss:19}
          \textit{Nï ta wa man \textbf{pilim}.}\\
\gll nï    ta      wa  ma=n      \textbf{pilim}\\
    1\textsc{sg}  already    just  3\textsc{sg.obj=obl}  feel\\
\glt `I had already just felt it.’ (\textit{pilim} = TP) [ulwa037\_00:34]
\z

\ea%20
    \label{ex:loss:20}
          \textit{Unji yena unji inin \textbf{paynim}!}\\
\gll    un-nji    yena    un-nji    ini=n      \textbf{paynim}\\
    \textsc{2pl-poss}  woman    \textsc{2pl-poss}  ground=\textsc{obl}  find\\
\glt `[They] are your women; [so] find your land!’ (\textit{payinim} = TP \textit{painim}) [ulwa014†]
\z

Sometimes, however, the \isi{aspect} or \isi{mood} of a detransitivized \ili{Tok Pisin} loan verb may be conveyed by means of an \isi{auxiliary} ‘going’ verb: (\textit{ma-} {\textasciitilde} \textit{i} ‘go’ or \textit{unda-} ‘go around’). As elsewhere with such detransitivized verb constructions, the logical object is expressed in an \isi{oblique} \isi{phrase}. Examples \REF{ex:loss:22} through \REF{ex:loss:27} illustrate various ‘going’ verbs used with detransitivized \ili{Tok Pisin} loan verbs.

\ea%22
    \label{ex:loss:22}
          \textit{Ndïn \textbf{mboylim i} ndala ya motap.}\\
\gll    ndï=n    \textbf{mboylim}  \textbf{i}    ndï=ala  ya ma=uta-p\\
    3\textsc{pl=obl}  boil    go.\textsc{pfv}  3\textsc{pl}=for  coconut    3\textsc{sg.obj}=grind-\textsc{pfv}\\
\glt `[I] boiled them and ground a coconut for them.’ (\textit{mboylim} = TP \textit{boilim}) [ulwa014\_17:31]
\z

\ea%23
    \label{ex:loss:23}
          \textit{Una gaden ngalan \textbf{pinisim iye}.}\\
\gll unan    gaden  ngala=n    \textbf{pinisim}  \textbf{i-e}\\
    1\textsc{pl.incl}  garden  \textsc{pl.prox=obl}  finish    go.\textsc{pfv-dep}\\
\glt `We have finished these gardens.’ (\textit{pinisim}, \textit{gaden} = TP) [ulwa030\_02:59]
\z

\ea%24
    \label{ex:loss:24}
          \textit{Ala ndïn \textbf{labisim mana}.}\\
\gll ala      ndï=n    \textbf{labisim}  \textbf{ma-na}\\
    \textsc{pl.dist}  \textsc{3pl=obl}  rubbish  go-\textsc{irr}\\
\glt `They will mess with them.’ (\textit{labisim} = TP \textit{rabisim}) [ulwa014\_23:07]
\z

\ea%25
    \label{ex:loss:25}
          \textit{Una unanji grup ngan \textbf{pasim ma}!}\\
\gll    unan    unan-nji      grup  nga=n      \textbf{pasim}  \textbf{ma}\\
    1\textsc{pl.incl}  \textsc{1pl.incl-poss}    group  \textsc{sg.prox=obl}  tie    go\\
\glt `Let’s form a group!’ (Literally `form our group’; \textit{pasim}, \textit{grup} = TP) [ulwa029\_01:16]
\z

\ea%26
    \label{ex:loss:26}
          \textit{Ala amblol le amblan \textbf{winim unde}.}\\
\gll ala      ambla=ul    lo-e  ambla=n    \textbf{winim}  \textbf{unda-e}\\
    \textsc{pl.dist}  \textsc{pl.refl=}with  go-\textsc{dep}  \textsc{pl.refl=obl}  win  go-\textsc{ipfv}\\
\glt `They go around with each other, competing with each other.’ (\textit{winim} = TP) [ulwa032\_54:02]
\z

\ea%27
    \label{ex:loss:27}
          \textit{Ya, i mas tokples ngan \textbf{laynim unda}.}\\
\gll ya    i    mas  tokples  nga=n      \textbf{laynim}  \textbf{unda}\\
    yeah  \textsc{pred}  must  tokples  \textsc{sg.prox=obl}  teach  go\\
\glt `Yeah, [we] have to teach [them] this \textit{tokples} [= vernacular].’ (\textit{laynim} = TP \textit{lainim}; \textit{ya}, \textit{i}, \textit{mas}, \textit{tokples} also = TP) [ulwa014\_02:46]
\z

Notably, one \ili{Tok Pisin} loan verb does seem to permit objects: this is the \ili{Tok Pisin} verb \textit{helpim} ‘help’, which is generally pronounced [alpim] in Ulwa. As in some varieties of \ili{Tok Pisin}, Ulwa lacks the \isi{glottal} \isi{fricative} [h]. Ulwa furthermore forbids the \is{mid vowel} mid \isi{front vowel} [e] word-initially. In example \REF{ex:loss:21}, the 2\textsc{sg} \isi{object marker} is the \isi{direct object} of the verb.

\ea%21
    \label{ex:loss:21}
          \textit{\textbf{Walpim} unji wombïn man ninda.}\\
\gll    \textbf{u=alpim}  u-nji    wombïn  ma=n      ni-nda\\
    2\textsc{sg}=help  2\textsc{sg-poss}  work    3\textsc{sg.obj=obl}  act-\textsc{irr}\\
\glt `[I] will help you do your work.’ (\textit{alpim} = TP \textit{helpim}) [ulwa031\_00:55]
\z

Sentence \REF{ex:loss:28} illustrates the use of the verb \textit{alpim} ‘help’ with the \isi{auxiliary verb} \textit{unda-} ‘go around’.

\ea%28
    \label{ex:loss:28}
          \textit{Nungol ndï \textbf{malpim unde} mol inamban nji ndine.}\\
\gll    nungol  ndï  \textbf{ma=alpim}    \textbf{unda-e}  ma=ul      inamba=n nji    ndï=ina-e\\
    child  \textsc{3pl}  3\textsc{sg.obj}=help  go-\textsc{dep}  \textsc{3sg.obj=}with  money\textsc{=obl}    thing  3\textsc{pl}=get-\textsc{ipfv}\\
\glt `The children are helping him buy things.’ (\textit{alpim} = TP \textit{helpim}) [ulwa032\_07:21]
\z

The loan verb \textit{lukawtim} ‘look after’ is also exceptional in that it seems sometimes to take the \isi{copular enclitic} rather than using a \isi{periphrastic} construction with a verb of ‘going’ to convey \isi{TAM} meaning. This perhaps reflects the fact that this verb has been adopted into Ulwa as a non-verbal element (cf. \textit{kalam} ‘knowledge’, \isi{borrow}ed from \ili{Waran}, \sectref{sec:5.4}). Still, like most loan verbs that come from \ili{Tok Pisin}, \textit{lukawtim} ‘look after’ does not permit an object, but rather makes use of \isi{oblique} \isi{phrase}s marked by \textit{=n} ‘\textsc{obl}’, as in \REF{ex:loss:29}.

\ea%29
    \label{ex:loss:29}
          \textit{Nï ango tïki \textbf{ankam kuman} \textbf{lukawtimpïna}.}\\
\gll nï    ango  tïki    ankam  kuma=\textbf{n}  \textbf{lukawtim=p}-na\\
    1\textsc{sg}  \textsc{neg}  again  person  some=\textsc{obl}  look.after\textsc{=cop}{}-\textsc{irr}\\
\glt `I won’t look after other people anymore.’ (\textit{lukawtim} = TP \textit{lukautim}) [ulwa032\_47:51]
\z

In \REF{ex:loss:30}, the \isi{verbalized} form of this word (with the \isi{copular enclitic} \textit{=p} ‘\textsc{cop}’) is further \isi{nominalize}d by the \isi{nominalizing} \isi{suffix} \textit{-en} ‘\textsc{nmlz}’ and then treated as a \is{non-verbal predication} non-verbal \isi{predicate}, being negated with \isi{clause-final negator} \textit{kom} ‘\textsc{neg}’. A similar phenomenon is seen with \textit{kalam} ‘knowledge’ in example \REF{ex:syntax:164} in \sectref{sec:13.3.2}.

\ea%30
    \label{ex:loss:30}
          \textit{Tembi anda wa njin anmanï \textbf{lukawtimpen} kom.}\\
\gll    tembi  anda    wa    nji=n    anma=nï  \textbf{lukawtim=p-en} kom\\
    bad    \textsc{sg.dist}  just    thing=\textsc{obl}  good=\textsc{obl}  look.after=\textsc{cop-nmlz}    \textsc{neg}\\
\glt `That bad one just doesn’t look after things well.’ [ulwa014\_07:28]
\z

\is{language loss|)}
\is{loan|)}
\is{detransitivization|)}

