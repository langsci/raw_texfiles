\chapter{Phrase-level syntax}\label{sec:9}

\is{phrase|(}

This chapter is dedicated to the description of phrase-level \isi{syntax}. Although a \isi{phrase} may consist of a single word, the following sections will mostly be concerned with multi-word phrases, as the point of interest here is how words interact with one other within a single \isi{phrase}.

\is{phrase|)}

\section{Noun phrases}\label{sec:9.1}

\is{noun phrase|(}
\is{phrase|(}

A \isi{noun phrase} (NP) consists minimally of a noun or a \isi{pronoun}, but may contain additional elements, such as \is{possession} possessive markers, \isi{adjective}s, \isi{numeral}s, or \isi{determiner}s. The canonical order of elements in an Ulwa NP is given in \REF{ex:phrase:1}.

\ea%1
    \label{ex:phrase:1}
            The Ulwa \isi{noun phrase}\\
    ([\isi{possessor}]) [noun] ([\isi{adjective}(s)]) ([\isi{numeral}]) ([\isi{determiner}])
\z

The \isi{noun phrase} given in \REF{ex:phrase:2} illustrates all these elements occurring together. The \isi{head noun} is indicated in \textbf{bold}.

\ea%2
    \label{ex:phrase:2}
            \textit{nïnji \textbf{lamndu} ambi kwe anda}\\
\gll    nï-nji    \textbf{lamndu}  ambi  kwe  anda\\
    1\textsc{sg-poss}  pig      big    one    \textsc{sg.dist}\\
\glt `that one big pig of mine’ [elicited]
\z

NPs \isi{head}ed by nouns may have a phrase-final \isi{determiner} (\isi{subject marker}, \isi{object marker}, or \isi{demonstrative} \isi{determiner}). NPs \isi{head}ed by pronouns do not permit \isi{subject marker}s or \isi{object marker}s but may have \isi{demonstrative}s.\footnote{In instances in which an \isi{object marker} immediately follows a \isi{personal pronoun}, the two are taken to be in \isi{apposition} to each other, not part of the same NP. There should be a \isi{prosodic break} signaling this phrasal separation.} The canonical position for \isi{adjective}s in NPs is following the noun and preceding the \isi{determiner} (if present); there are, however, examples of \isi{adjective}s preceding nouns (\sectref{sec:5.1}).\footnote{This may be due to influence from the \isi{word order} of \ili{Tok Pisin}; indeed, some speakers consider the order \isi{adjective}-noun to be ungrammatical.} \isi{Numeral}s also follow nouns (and \isi{adjective}s, if present) (\sectref{sec:7.5}). \isi{Numeral}s, too, precede \isi{determiner}s such as \isi{subject marker}s (\sectref{sec:7.1}), \isi{object marker}s (\sectref{sec:7.2}), or \isi{demonstrative}s (\sectref{sec:7.3}), when present. NPs also permit \is{possession} possessive markers (\sectref{sec:6.2}, \sectref{sec:9.1.5}). These precede nouns and, thus, when present, are the first elements in their respective NPs.\footnote{The issue of the peculiar placement -- outside the NP -- of the universal \isi{quantifier} \textit{wopa} ‘all’ is addressed in \sectref{sec:7.4}.}

\is{phrase|)}
\is{noun phrase|)}

\subsection{The head of the noun phrase}\label{sec:9.1.1}

\is{head noun|(}
\is{head|(}
\is{noun phrase|(}
\is{phrase|(}

The \isi{head} of a \isi{noun phrase} need not be a prototypical noun. It may, instead, be an \isi{adjective} functioning as a noun \REF{ex:phrase:3} (see \sectref{sec:5.3} on substantive uses of \isi{adjective}s). Similarly, like \isi{personal pronoun}s, \isi{possessive pronoun}s (\sectref{sec:6.2}) may also function as the \isi{head} of an NP \REF{ex:phrase:4}.

\ea%3
    \label{ex:phrase:3}
            \textbf{\textit{Ambi}} \textbf{\textit{anda}} \textit{wa lolop man.}\\
\gll    \textbf{ambi}  anda    wa  lolop  ma-n\\
    big    \textsc{sg.dist}  just  just    go-\textsc{ipfv}\\
\glt `That big [man] just goes around.’ [ulwa014\_04:51]
\z

\ea%4
    \label{ex:phrase:4}
            \textit{Nï \textbf{nïnji ngalat} unanda.}\\
\gll    nï    \textbf{nï-nji}    ngala=tï    u=na-nda\\
    1\textsc{sg}  1\textsc{sg-poss}  \textsc{pl.prox}=take  2\textsc{sg}=give-\textsc{irr}\\
\glt `I will give mine to you.’ [ulwa029\_09:27]
\z

The \isi{head} of an NP can also be a noun derived from a verb that has been \isi{nominalize}d with the \isi{suffix} \textit{-en} ‘\textsc{nmlz}’ (see \sectref{sec:3.2} for examples).

\is{phrase|)}
\is{noun phrase|)}
\is{head|)}
\is{head noun|)}


\subsection{Plural for dual}\label{sec:9.1.2}

\is{noun phrase|(}
\is{phrase|(}

As discussed in \sectref{sec:7.1} and \sectref{sec:7.2}, \isi{subject marker}s and \isi{object marker}s do not always occur in NPs that function as subjects or as objects. When they do occur, however, they mark their respective NPs for \isi{number}. The three \isi{number} categories in Ulwa are \isi{singular}, \isi{dual}, and \isi{plural} (\sectref{sec:7.1}--\sectref{sec:7.3}). Whereas the category of \isi{dual} is never employed when there are three or more referents, the category of \isi{plural} is sometimes found even when there are exactly two referents, as in \REF{ex:phrase:5}.

\is{phrase|)}
\is{noun phrase|)}

\ea%5
    \label{ex:phrase:5}
            \textit{Wonmbi} \textbf{\textit{ndïtumulka}}.\\
\gll wonmbi  \textbf{ndï}=tumul-ka\\
    tusk    3\textsc{pl}=bend-let\\
\glt `[They] bent the tusks.’ (This refers to a pair of tusks belonging to a single boar.) [ulw001\_12:30]
\z

\subsection{Multiple adjectives}\label{sec:9.1.3}

\is{noun phrase|(}
\is{phrase|(}

Multiple \isi{adjective}s may occur within a single NP. When there are multiple \isi{adjective}s, they simply stack up after the \isi{head noun} (and before any \isi{determiner}s), as seen in \REF{ex:phrase:6} and \REF{ex:phrase:7}.

\is{phrase|)}
\is{noun phrase|)}

\ea%6
    \label{ex:phrase:6}
            \textit{Wapa \textbf{ambi tembi} ndawe nï mat inde.}\\
\gll    wapa  \textbf{ambi}  \textbf{tembi}  anda-we      nï    ma=tï inda-e\\
    leaf  big    bad    \textsc{sg.dist-part.int}  \textsc{1sg}  \textsc{3sg.obj=}take    walk-\textsc{ipfv}\\
\glt `That big, bad leaf alone -- I’m taking it.’ [ulwa037\_55:10]
\z

\ea%7
    \label{ex:phrase:7}
            \textit{Tïmbïl \textbf{ambi nïpat ngata} maytana mane.}\\
\gll    tïmbïl  \textbf{ambi}  \textbf{nïpat}  \textbf{ngata}  ma=ita-na        ma-n-e\\
    fence  big    giant  grand  3\textsc{sg.obj}=build-\textsc{irr}  go-\textsc{ipfv-dep}\\
\glt `[You] are going to build a big, huge, giant fence.’ [ulwa042\_00:48]
\z

\subsection{Apposition}\label{sec:9.1.4}

\is{noun phrase|(}
\is{phrase|(}

Two noun phrases may be in \isi{apposition} to each other, as in \REF{ex:phrase:8}. Here, the two NPs in \isi{apposition} are the \isi{compound noun} \textit{wot yana} ‘younger sister’ and the \isi{proper noun} \textit{Sinanam}, a woman’s name. Each NP serves as the \isi{singular} grammatical object of the verb \textit{na-} ‘give’.

\is{phrase|)}
\is{noun phrase|)}

\ea%8
    \label{ex:phrase:8}
            \textit{\textbf{Wot yana Sinanam} manana.}\\
\gll    \textbf{wot}    \textbf{yana}    \textbf{Sinanam}  ma=na-na\\
    younger  woman    [name]    3\textsc{sg.obj}=give-\textsc{pfv}\\
\glt `[He] gave it to the younger sister Sinanam.’ [ulwa001\_08:15]
\z

\subsection{Ways of indicating possession}\label{sec:9.1.5}

\is{possession|(}
\is{noun phrase|(}
\is{phrase|(}

Nominal \isi{possession} may be indicated in one of three basic ways: with a pronominal form with the possessive \isi{suffix} \textit{{}-nji} \textsc{`poss’} (\sectref{sec:6.2}), with the \isi{oblique-marker} \isi{enclitic} \textit{=n} \textsc{`obl’} (\sectref{sec:11.4}), or with a \isi{non-subject marker}, such as \textit{ma} \textsc{`3sg.obj’} (\sectref{sec:7.3}).\footnote{Possessive predicates \is{possessive predication} are discussed in \sectref{sec:10.2} and \sectref{sec:10.3}.} Regardless of the marking, the order of elements is always the same: the \isi{possessor} argument (i.e., the \isi{genitive}) precedes the \isi{possessed} argument (i.e., the \isi{possessum}).\footnote{Occasionally (and more commonly for \isi{proper noun} \isi{possessor}s), \isi{possession} can be indicated by \isi{juxtaposition} alone, without any special marking: the \isi{possessor} simply immediately precedes the \isi{possessum}.}

The possessive \isi{suffix} \textit{{}-nji} \textsc{`poss’} affixes to \isi{pronoun}s and \isi{determiner}s to create \isi{possessive pronoun}s and possessive \isi{demonstrative}s \sectref{sec:6.2}. These so-formed \isi{possessive pronoun}s may in turn occur in NPs (following the \isi{possessor} argument) to indicate \isi{possession}. Sentences \REF{ex:phrase:9} through \REF{ex:phrase:12} illustrate \isi{possession} marked on full NPs. The \isi{possessor} is indicated in \textbf{bold}.

\ea%9
    \label{ex:phrase:9}
          \textit{\textbf{Itom manji} lamndu mï nip.}\\
\gll    \textbf{itom}  \textbf{ma-nji}      lamndu  mï      ni-p\\
    father  \textsc{3sg.obj-poss}  pig      3\textsc{sg.subj}  die-\textsc{pfv}\\
\glt `Father’s pig died.’ [elicited]
\z

\ea%10
    \label{ex:phrase:10}
          \textit{Tïn ndï \textbf{itom manji} lamndu masap.}\\
\gll    tïn    ndï  \textbf{itom}  \textbf{ma-nji}      lamndu  ma=asa-p\\
    dog  3\textsc{pl}  father  \textsc{3sg.obj-poss}  pig      3\textsc{sg.obj}=hit-\textsc{pfv}\\
\glt `The dogs killed father’s pig.’ [elicited]
\z

\ea%11
    \label{ex:phrase:11}
          \textit{\textbf{Manama manji} wot mï mana motoplïp.}\\
\gll    \textbf{Manama}  \textbf{ma-nji}      wot    mï      mana ma=top-lï-p\\
    [name]    3\textsc{sg.obj-poss}  younger  3\textsc{sg.subj}  spear    3\textsc{sg.obj}=throw-put-\textsc{pfv}\\
\glt `Manama’s younger brother threw the spear.’ [elicited]
\z

\ea%12
    \label{ex:phrase:12}
          \textit{\textbf{Nïnji atuma manji} aweta mï tembip.}\\
\gll    \textbf{nï-nji}    \textbf{atuma}      \textbf{ma-nji}      aweta  mï      tembi=p\\
    1\textsc{sg-poss}  older.brother  3\textsc{sg.obj-poss}  friend  3\textsc{sg.subj}  bad=\textsc{cop}\\
\glt `My older brother’s friend is sick.’ [elicited]
\z

 To intensify \isi{possessor} NPs that contain \isi{possessive pronoun}s, the modifier \textit{wo} ‘very own’ may be added. Whereas the \isi{possessive pronoun} precedes the \isi{head noun}, the \isi{emphatic} modifier \textit{wo} ‘very own’ follows it, as in examples \REF{ex:phrase:13} through \REF{ex:phrase:16}.

\ea%13
    \label{ex:phrase:13}
            \textbf{\textit{Manji}} \textit{tïn} \textbf{\textit{wo}} \textit{lamndu masap.}\\
\gll \textbf{ma-nji}      tïn    \textbf{wo}    lamndu  ma=asa-p\\
    3\textsc{sg.obj-poss}  dog  own  pig      3\textsc{sg.obj}=hit-\textsc{pfv}\\
\glt `His very own dog killed the pig.’ [elicited]
\z

\ea%14
    \label{ex:phrase:14}
          \textbf{\textit{nïnji}} \textit{na} \textbf{\textit{wo}}\\
\gll    \textbf{nï-nji}    na    \textbf{wo}\\
    1\textsc{sg-poss}  talk  own\\
\glt `my very own story’ [ulwa013\_00:10]
\z

\ea%15
    \label{ex:phrase:15}
          \textbf{\textit{Anji}} \textit{wi} \textbf{\textit{wo}}.\\
\gll \textbf{an-nji}        wi    \textbf{wo}\\
    1\textsc{pl.excl-poss}  name  own\\
\glt `It’s really our name.’ [ulwa002\_01:42]
\z

\ea%16
    \label{ex:phrase:16}
          \textit{Yetani} \textbf{\textit{lanji}} \textbf{\textit{wo}}.\\
\gll Yetani  \textbf{ala-nji}      \textbf{wo}\\
    Yamen  \textsc{pl.dist-poss}  own\\
\glt `[He was] the Yamen people’s very own [ancestor].’ [ulwa002\_03:19]
\z

Another means of signaling that an NP has a \isi{possessor} role is the \isi{oblique-marker} \isi{enclitic} \textit{=n} \textsc{‘obl'} (\sectref{sec:11.4.1}). This strategy is mainly limited to pronominal \isi{possessor}s. In such constructions, the \isi{pronoun} or \isi{demonstrative} marked with \textit{=n} \textsc{‘obl'} is the \isi{possessor} of the NP that immediately follows, as in examples \REF{ex:phrase:17} through \REF{ex:phrase:20}.

\ea%17
    \label{ex:phrase:17}
          \textit{Upan nungol ndïn} \textbf{\textit{nïn}} \textit{ani up.}\\
\gll    upan  nungol  ndï=\textbf{n}    nï=n    ani    u-p\\
    fish.species  child  3\textsc{pl=obl}  1\textsc{sg=obl}  bilum  put-\textsc{pfv}\\
\glt `[She] put some small fish in my \textit{bilum} [= string bag].’ [ulwa014\_29:32]
\z

\ea%18
    \label{ex:phrase:18}
          \textit{Mawl i} \textbf{\textit{man}} \textit{wandam malïp.}\\
\gll    ma=ul      i    ma=\textbf{n}    wandam  ma=lï-p\\
    3\textsc{sg.obj}=with  go.\textsc{pfv}  3\textsc{sg.obj=obl}  jungle  3\textsc{sg.obj}=put-\textsc{pfv}\\
\glt `[He] went with her and put her in his jungle [home].’ [ulwa001\_03:30]
\z

\ea%19
    \label{ex:phrase:19}
          \textit{Way mï asi} \textbf{\textit{man}} \textit{wat wan make.}\\
\gll    way  mï      asi  ma=\textbf{n}      wat    wan  ma=ka-e\\
    turtle  \textsc{3sg.subj}  sit  3\textsc{sg.obj=obl}  ladder  above  3\textsc{sg.obj}=let\textsc{{}-ipfv}\\
\glt `The turtle was sitting at the top of his ladder.’ [ulwa006\_04:05]
\z

\ea%20
    \label{ex:phrase:20}
          \textit{Atana mï liyu matïn mat} \textbf{\textit{ambïn}} \textit{ame menlïp.}\\
\gll    atana    mï      li-u        ma=tï-n ma=tï      ambï=\textbf{n}    ame    ma=in-lï-p\\
    older.sister  3\textsc{sg.subj}  down-from  3\textsc{sg.obj}=take-\textsc{pfv}    3\textsc{sg.obj=}take  \textsc{sg.refl=obl}  basket  3\textsc{sg.obj}=in-put-\textsc{pfv}\\
\glt `The older sister got him down and put him in her basket.’ [ulwa011\_01:15]
\z

See \sectref{sec:11.4.2} for other \isi{case}-like uses of the \isi{oblique marker}. It may be noted here that there is some overlap in function between the \isi{oblique marker} and the possessive marker. Just as the \isi{oblique marker} \textit{=n} \textsc{‘obl'} can function very much like a possessive marker, \isi{possessive pronoun}s marked with \textit{-nji} \textsc{‘poss'} can serve \isi{oblique}-like functions -- namely, they may indicate a beneficiary, as in sentences \REF{ex:phrase:21}, \REF{ex:phrase:22}, and \REF{ex:phrase:23}.

\ea%21
    \label{ex:phrase:21}
          \textit{Nï i} \textbf{\textit{ngunji}} \textit{mundu ilum kuma wananda.}\\
\gll    nï    i    ngun-\textbf{nji}  mundu  ilum  kuma  wana-nda\\
    1\textsc{sg}  go.\textsc{pfv}  \textsc{2du-poss}  food  little  some  cook-\textsc{irr}\\
\glt `I will come and cook some food for you.’ (Literally ‘cook some little food of yours’) [ulwa014\_15:27]
\z

\ea%22
    \label{ex:phrase:22}
          \textbf{\textit{Ndïnji}} \textit{na tïna mbïlop.}\\
\gll    ndï-\textbf{nji}    na    tï-na    mbï-lo-p\\
    3\textsc{pl-poss}  talk  take-\textsc{irr}  here-go-\textsc{pfv}\\
\glt `[They] came here to have a talk for them [= their children].’ [ulwa032\_32:54]
\z

\ea%23
    \label{ex:phrase:23}
          \textit{Nï} \textbf{\textit{manji}} \textit{ana mat manana.}\\
\gll    nï    ma-\textbf{nji}      ana      ma=tï      ma=na-na\\
    1\textsc{sg}  3\textsc{sg.obj-poss}  grass.skirt  3\textsc{sg.obj}=take  3\textsc{sg.obj}=give-\textsc{pfv}\\
\glt `I gave her a grass skirt.’ [ulwa014\_11:29]
\z

Possession can also be indicated with a \isi{non-subject marker} (\sectref{sec:7.3}), without any additional possessive or \isi{oblique} marking. In such instances, these “\isi{object marker}s” do not seem to be \isi{clitic}s: they are not necessarily \isi{phonological}ly dependent on the following word. Like the \isi{oblique-marker} strategy, this strategy is mainly limited to pronominal \isi{possessor}s, as in examples \REF{ex:phrase:24} through \REF{ex:phrase:28}.

\ea%24
    \label{ex:phrase:24}
          \textbf{\textit{Ma}} \textit{yawa mï i makalilïpe.}\\
\gll    \textbf{ma}      yawa  mï      i    ma=kali-lï-p-e\\
    3\textsc{sg.obj}  uncle  \textsc{3sg.subj}  go.\textsc{pfv}  \textsc{3sg.obj}=send-put-\textsc{pfv-dep}\\
\glt `Her uncle went and sent her.’ [ulwa014\_11:17]
\z

\ea%25
    \label{ex:phrase:25}
          \textit{Mï \textbf{ma} inim ame.}\\
\gll    mï      \textbf{ma}      inim  ama-e\\
    3\textsc{sg.subj}  \textsc{3sg.obj}  water  eat-\textsc{ipfv}\\
\glt `He was drinking its nectar.’ [ulwa001\_11:10]
\z

\is{phrase|)}
\is{noun phrase|)}
\is{possession|)}

\newpage

\is{possession|(}
\is{noun phrase|(}
\is{phrase|(}

\ea%26
    \label{ex:phrase:26}
          \textit{May lïmndï} \textbf{\textit{ndï}} \textit{we imbïn ndutap.}\\
\gll    ma=i        lïmndï  \textbf{ndï}      we     imbïn  ndï=uta-p\\
    3\textsc{sg.obj}=go.\textsc{pfv}  eye    3\textsc{sg.obj}  sago  refuse  3\textsc{pl}=grind-\textsc{pfv}\\
\glt `[I] went and saw their sago refuse.’ (i.e., the water run-off from strained sago;   literally ‘eye-ground’ for ‘saw’) [ulwa037\_63:43]
\z

\ea%27
    \label{ex:phrase:27}
          \textbf{\textit{Ndï}} \textit{ini nda.}\\
\gll    \textbf{ndï}  ini    anda\\
    3\textsc{pl}  ground  \textsc{sg.dist}\\
\glt `That’s their land.’ [ulwa032\_37:53]
\z

\ea%28
    \label{ex:phrase:28}
          \textit{I lïmndï} \textbf{\textit{min}} \textit{luwa ala.}\\
\gll    i    lïmndï  \textbf{min}  luwa  ala\\
    go.\textsc{pfv}  eye    3\textsc{du}  place  for\\
\glt `[He] went and saw their place.’\footnote{It is possible that the \isi{oblique}-marking strategy is being used here, but that a sequence of /nn/ resulting from the final /n/ of the \isi{dual} pronominal form and the /=n/ that is the \isi{oblique marker} has \isi{degeminate}d.} [ulwa001\_10:15]
\z

  Occasionally, \isi{possession} is indicated without any marking. Rather, the \isi{possessum} simply follows the \isi{possessor}. This strategy is mainly restricted to phrases with \isi{proper noun} \isi{possessor}s, but there are a few known examples with \isi{common noun} \isi{possessor}s as well (all are \isi{kinship} terms). Sentences \REF{ex:phrase:29} through \REF{ex:phrase:34} illustrate this \isi{juxtaposition} method.

\ea%29
    \label{ex:phrase:29}
          \textbf{\textit{Yawana}} \textit{numan ngusuwa nda}\\
\gll \textbf{Yawana}  numan    ngusuwa  anda\\
    [name]    husband  poor    \textsc{sg.dist}\\
\glt `Yawana’s poor husband’ [ulwa014†]
\z

\ea%30
    \label{ex:phrase:30}
          \textbf{\textit{Samuel}} \textit{yena nda sokoy mu ndïkuklïp.}\\
\gll    \textbf{Samuel}  yena  anda    sokoy    mu    ndï=kuk-lï-p\\
    [name]    wife  \textsc{sg.dist}  tobacco  seed  3\textsc{pl}=gather-put.\textsc{pfv}\\
\glt `Samuel’s wife gathered the tobacco seeds.’ [ulwa037\_54:17]
\z

\ea%31
    \label{ex:phrase:31}
          \textit{Nï wa man mïka} \textbf{\textit{Manama}} \textit{nungol ngalat: …}\\
\gll    nï    wa  ma=n      mïka  \textbf{Manama}  nungol  ngala=ta\\
    1\textsc{sg}  just  3\textsc{sg.obj}=\textsc{obl}  thus  [name]    child  \textsc{pl.prox}=say\\
\glt `I just, like, told Manama’s children: …’ [ulwa014\_70:12]
\z

\ea%32
    \label{ex:phrase:32}
          \textit{Aya nï} \textbf{\textit{Kayta}} \textit{yanat anda.}\\
\gll    aya    nï    \textbf{Kayta}  yanat    anda\\
    \textsc{interj}  1\textsc{sg}  [name]  daughter  \textsc{sg.dist}\\
\glt `Aya, I am Kayta’s daughter.’ [ulwa014†]
\z

\ea%33
    \label{ex:phrase:33}
          \textit{Nï ma} \textbf{\textit{Eltik}} \textit{wandam ma.}\\
\gll    nï    ma  \textbf{Eltik}  wandam  ma\\
    1\textsc{sg}  go  [name]  jungle    go\\
\glt `I will go, go to Eltik’s garden.’ [ulwa014\_07:50]
\z

\ea%34
    \label{ex:phrase:34}
         \textit{Nï ango wa i \textbf{nïnji itom} wandam alo lop.}\\
\gll    nï    ango  wa    i    \textbf{nï-nji}    \textbf{itom}  wandam     ala=u      lo-p\\
    1\textsc{sg}  \textsc{neg}  village  go.\textsc{pfv}  1\textsc{sg-poss}  father  jungle    \textsc{pl.dist}=from  go-\textsc{pfv}\\
\glt `I haven’t gone to the village, gone around in my father’s jungles.’ [ulwa032\_49:28]
\z

In such constructions it appears to be common for the \isi{possessed} \isi{phrase} to contain a postnominal \isi{demonstrative} -- for example, ‘these children of Manama’ in \REF{ex:phrase:31} -- although this is not mandatory.

  In addition to these methods of indicating \isi{possession} with noun phrases, it is possible to indicate \isi{possession} with \isi{verb phrase}s. These \isi{verb phrase}s function in such instances as \isi{relative clause}s (\sectref{sec:12.3}), as in \REF{ex:phrase:35}.

\ea%35
    \label{ex:phrase:35}
          \textbf{\textit{Kaytape}} \textit{anapa mï}\\
\gll    [\textbf{Kayta=p-e}]      anapa  mï\\
    [[name]=\textsc{cop-dep]}  sister  \textsc{3sg.subj}\\
\glt `Kayta’s sister’ (Literally ‘the sister that is [of] Kayta’) [ulwa014\_21:14]
\z

These \isi{verb phrase}s may in turn be \isi{nominalize}d, as in examples \REF{ex:phrase:36} and \REF{ex:phrase:37}.

\ea%36
    \label{ex:phrase:36}
          \textbf{\textit{Lucypen}} \textit{anda.}\\
\gll    \textbf{Lucy=p-en}      anda\\
    [name]=\textsc{cop-nmlz}  \textsc{sg.dist}\\
\glt `That [pot] is Lucy’s.’ (Literally ‘The one that is [of] Lucy [is] that [one].’) [ulwa014\_17:38]
\z

\ea%37
    \label{ex:phrase:37}
          \textbf{\textit{Albertpen}} \textit{maka yena inom mï maka mu kumat nïnana.}\\
\gll    \textbf{Albert=p-en}      maka  yena    inom    mï      maka mu    kuma=tï  nï=na-na\\
    [name]=\textsc{cop-nmlz}  thus  woman    mother    3\textsc{sg.subj}  thus    seed  some=take  1\textsc{sg}=give-\textsc{pfv}\\
\glt `Albert’s wife’s mother, like, gave me some seeds.’ [ulwa037\_56:43]
\z

\is{phrase|)}
\is{noun phrase|)}
\is{possession|)}

\subsection{Noun phrases as clauses}\label{sec:9.1.6}

\is{phrase|(}
\is{noun phrase|(}

Noun phrases may serve a number of grammatical functions within a clause: the subject, the object of a verb, the \isi{object of a postposition}, or part of an \isi{oblique} \isi{phrase} marked by the \isi{oblique marker} \textit{=n} ‘\textsc{obl}’ (\chapref{sec:11}). In addition to these clause-internal functions, noun phrases may occasionally stand alone as entire clauses. In this use, the grandeur of an event is stressed, the \isi{predicate} itself being merely implied, as in \REF{ex:phrase:38} and \REF{ex:phrase:39}.

\is{noun phrase|)}
\is{phrase|)}

\ea%38
    \label{ex:phrase:38}
          \textit{Min map. \textbf{Mïnda wandam}!}\\
\gll    min ma=p \textbf{mïnda}  \textbf{wandam}\\
    \textsc{3du} \textsc{3sg.obj}=be banana  jungle\\
\glt `The two were staying there. Banana garden!’ (i.e., ‘Oh what a banana garden they made there!’) [ulwa001\_06:39]
\z

\ea%39
    \label{ex:phrase:39}
          \textit{Ndï apïn anul ndame.} 
          \textit{\textbf{Namndu}!}\\
\gll    ndï  apïn=n    anul    ndï=ama-e    \textbf{namndu}\\
    3\textsc{pl}  fire=\textsc{obl}  grassland  3\textsc{pl}=eat-\textsc{ipfv}  pig\\
\glt `They were burning the grassland. Pigs!’\footnote{Setting fire to the grassland is a strategy for chasing out pigs to hunt.} (i.e., ‘Oh how many pigs they killed!’) [ulwa001\_12:22]
\z

\section{Verb phrases}\label{sec:9.2}

\is{verb phrase|(}
\is{phrase|(}

A verb \isi{phrase} (VP) consists minimally of a verb, a non-verbal element with \isi{copular enclitic} (\sectref{sec:10.2}), or a \isi{postposition} functioning as a verb (\sectref{sec:8.2}). In some instances, a VP may also contain a \isi{noun phrase} (NP) or a \isi{postpositional phrase}s (PP). The order of potential elements of the Ulwa VP is given in \REF{ex:phrase:40}.

\ea%40
    \label{ex:phrase:40}
          The Ulwa verb \isi{phrase}

    ([PP]) ([NP]) [verb]
\z

The verb (or verbal element) is always the final element in the VP. If the verb is \isi{transitive} and contains an overt object, then this object \isi{noun phrase} occurs before the verb. This NP may be marked with an \isi{object marker}, which \isi{clitic}izes to the verb. Other \isi{determiner}s, such as \isi{demonstrative}s, potentially \isi{clitic}ize as well (\sectref{sec:7.3}).\footnote{The fact that \isi{object marker}s are \isi{phonological}ly closely connected with their verbs makes it difficult to assign them to positions within NPs, insofar as they resemble \isi{agreement}-marking verbal \isi{prefix}es. Nevertheless, largely by \isi{analogy} to their \isi{subject marker} equivalents, they are considered to be \isi{syntactic} constituents of NPs, albeit NPs that are themselves constituents of verb phrases.} In addition to NPs, \isi{postpositional phrase}s (PPs) may also be considered constituents of VPs. When present, they always occur before the verb (and before the \isi{direct object}, if the verb is \isi{transitive}).

The phrases in \REF{ex:phrase:41} and \REF{ex:phrase:42} are both \isi{intransitive}, whereas those in \REF{ex:phrase:43} and \REF{ex:phrase:44} are both \isi{transitive}. The phrases in \REF{ex:phrase:42} and \REF{ex:phrase:44} both contain \isi{postpositional phrase}s, whereas those in \REF{ex:phrase:41} and \REF{ex:phrase:43} do not.

\ea%41
    \label{ex:phrase:41}
          \textit{man}\\
\gll    ma-n\\
    go-\textsc{ipfv}\\
\glt `is going’ [elicited]
\z

\ea%42
    \label{ex:phrase:42}
          \textit{im maya man}\\
\gll    im    ma=iya      ma-n\\
    tree  3\textsc{sg.obj}=toward  go-\textsc{ipfv}\\
\glt `is going toward the tree’ [elicited]
\z


\ea%43
    \label{ex:phrase:43}
          \textit{utam mawanap}\\
\gll    utam  ma=wana-p\\
    yam  \textsc{3sg.obj}=cook-\textsc{pfv}\\
\glt `cooked the yam’ [elicited]
\z

\ea%44
    \label{ex:phrase:44}
          \textit{apïn mawat mawanap}\\
\gll    apïn  ma=wat    ma=wana-p\\
    fire    3\textsc{sg.obj}=atop  3\textsc{sg.obj}=cook-\textsc{pfv}\\
\glt `cooked the yam on the fire’ [elicited]
\z

The order and relation of elements within NPs is discussed in \sectref{sec:9.1}. The order and relation of elements within PPs is discussed \sectref{sec:9.3}.

\is{phrase|)}
\is{verb phrase|)}

\subsection{Separable verbs}\label{sec:9.2.1}

\is{separable verb|(}
\is{verb phrase|(}
\is{phrase|(}

In \sectref{sec:4.14} it was shown that \isi{compound verb} forms can be constructed with \isi{postposition}al or nominal elements (in additional to at least one verbal component). Some \isi{compound} verbs containing nominal elements can actually be \isi{discontinuous} -- that is, words may intervene between the nominal component and the verbal component, as illustrated by \REF{ex:phrase:55}.

\ea%55
    \label{ex:phrase:55}
          \textit{An} \textbf{\textit{kïkal}} \textit{inom itom} \textbf{\textit{ndïwana}}.\\
\gll an      \textbf{kïkal}  inom    itom  ndï=\textbf{wana}\\
    1\textsc{pl.excl}  ear    mother    father  3\textsc{pl}=feel\\
\glt `We listened to our parents.’ [ulwa013\_03:52]
\z

In \REF{ex:phrase:55}, the verb \textit{wana-} ‘feel, taste, sense, think’ combines with \textit{kïkal} ‘ear’ to form the the \isi{compound} \textit{kïkalwana-} ‘hear’, which thus consists of two separable parts. The object of the verb occurs between the two elements \textit{kïkal} ‘ear’ and \textit{wana-} ‘feel’.

It may first be demonstrated how \textit{wana-} ‘feel’ can function as a verb on its own, with a variety of related meanings, as in \REF{ex:phrase:45}, \REF{ex:phrase:46}, and \REF{ex:phrase:47}.

\ea%45
    \label{ex:phrase:45}
          \textit{Nï} \textbf{\textit{wana}} \textit{Raten ndï ita.}\\
\gll    nï    \textbf{wana}  Raten  ndï  i-ta\\
    1\textsc{sg}  feel  [place]  \textsc{3pl}  go.\textsc{pfv-cond}\\
\glt `I thought that the Raten people would come.’ [ulwa014\_29:48]
\z

\ea%46
    \label{ex:phrase:46}
          \textit{Ankam} \textbf{\textit{mawane}} \textit{mambi …}\\
\gll    ankam  ma=\textbf{wana}{}-e    ma-ambi\\
    person  3\textsc{sg.obj}=feel-\textsc{dep}  3\textsc{sg.obj-top}\\
\glt `As for the person who tastes it …’ [ulwa037\_52:11]
\z

\ea%47
    \label{ex:phrase:47}
          \textbf{\textit{Mawana}}.\\
\gll ma=\textbf{wana}\\
    3\textsc{sg.obj}=feel\\
\glt `[She] smelled it.’ [ulwa037\_53:08]
\z

Thus, \textit{wana-} ‘feel’ can function either as an \isi{intransitive} verb introducing a clausal complement \REF{ex:phrase:45} or as a \isi{transitive} verb, as in \REF{ex:phrase:46} and \REF{ex:phrase:47}. The form \textit{wana-} ‘feel’, however, can combine with a nominal element to create a new meaning. For example, sentence \REF{ex:phrase:48} illustrates the noun-plus-verb \isi{compound} \textit{nambïtwana-} ‘smell’, which here takes as its \isi{direct object} the \isi{pronoun} \textit{nï=} ‘1\textsc{sg}’.\footnote{The \isi{compound} \textit{nambïtwana-} ‘smell (\isi{transitive})’ consists of \textit{nambït} ‘odor’ and \textit{wana-} ‘sense’ (i.e., ‘sense odor’).}

\ea%48
    \label{ex:phrase:48}
          \textit{Mï} \textbf{\textit{nïnambïtwana}} \textit{ko anmbu i.}\\
\gll    mï      nï=\textbf{nambït-wana}  ko  an-mbï-u    i\\
    3\textsc{sg.subj}  1\textsc{sg}=odor-feel    just  out-here-from  go.\textsc{pfv}\\
\glt `It smelled me and just went out from there.’ [ulwa037\_03:23]
\z

The verb \textit{inakawana-} ‘think’ is a \isi{compound} that contains an entire \isi{postpositional phrase} (including a noun).\footnote{The \isi{compound} \textit{inakawana-} ‘think’ consists of \textit{ina} ‘liver [the seat of reason]’, \textit{ka} ‘at, in, on’, and \textit{wana-} ‘feel’, (i.e., ‘feel in [one’s] mind’).} Its use is illustrated by examples \REF{ex:phrase:49} and \REF{ex:phrase:50}.

\ea%49
    \label{ex:phrase:49}
          \textit{Mï i atay mawap} \textbf{\textit{inakawanap}}.\\
\gll mï      i    ata-i    ma=wap      \textbf{ina-ka-wana}{}-p\\
    3\textsc{sg.subj}  go.\textsc{pfv}  up-go.\textsc{pfv}  \textsc{3sg.obj}=be.\textsc{pst}  liver-at-feel-\textsc{pfv}\\
\glt `He went, went up, stayed there, and thought.’ [ulwa035\_02:57]
\z

\ea%50
    \label{ex:phrase:50}
          \textit{Nï} \textbf{\textit{inakawana}} \textit{nï unul mbïpïta …}\\
\gll    nï    \textbf{ina-ka-wana}  nï    un=ul    mbï-p-ta\\
    1\textsc{sg}  liver-at-feel  1\textsc{sg}  2\textsc{pl}=with  here-be\textsc{{}-cond}\\
\glt `So I thought: if I stay here with you …’ [ulwa037\_40:46]
\z

The verb \textit{inakawana-} ‘think’ may take as an object the topic about which one thinks \REF{ex:phrase:51}. Here, the object occurs before the entire \isi{compound} (i.e., this is not a \isi{separable verb}).

\ea%51
    \label{ex:phrase:51}
          \textit{Atana nda nipe ndï ango} \textbf{\textit{ninakawan}}.\\
\gll atana    anda    ni-p-e      ndï  ango  nï=\textbf{ina-ka-wana}\\
    older.sister  \textsc{sg.dist}  die-\textsc{pfv-dep}  \textsc{3pl}  \textsc{neg}  1\textsc{sg}=liver-at-feel\\
\glt `When that older sister died, they didn’t think of me.’ [ulwa014†]
\z

The \isi{compound} \textit{kïkalwana-} ‘hear’, on the other hand, can be separated when it has an object. First, it may be seen that, when there is no object, the nominal element \textit{kïkal} ‘ear’ occurs immediately before the verbal element. Examples \REF{ex:phrase:52} and \REF{ex:phrase:53} illustrate \textit{kïkalwana-} ‘hear’ functioning as an \isi{intransitive} verb (meaning something like ‘listen’).

\ea%52
    \label{ex:phrase:52}
          \textit{Ndï} \textbf{\textit{kïkalwana}} \textit{ngunaniya ita.}\\
\gll    ndï  \textbf{kïkal-wana}  ngunan=iya    i-ta\\
    3\textsc{pl}  ear-feel    1\textsc{du.incl}=toward  go.\textsc{pfv-cond}\\
\glt `If only they would listen and come to us.’ [ulwa037\_21:56]
\z

\ea%53
    \label{ex:phrase:53}
          \textit{Ango} \textbf{\textit{kïkalwana}}.\\
\gll ango  \textbf{kïkal-wana}\\
    \textsc{neg}  ear-feel\\
\glt `[They] don’t listen.’ [ulwa014\_37:03]
\z

However, when \textit{kïkalwana-} ‘hear’ is \isi{transitive}, we see that its component parts are separable. The nominal element \textit{kïkal} ‘ear’ is separate from the \isi{verb stem} \textit{wana-} ‘feel’, with the object of the verb occurring between these two parts, as in examples \REF{ex:phrase:54} and \REF{ex:phrase:56}.

\ea%54
    \label{ex:phrase:54}
          \textit{Ndï mbi nïmal mbïpen ndï} \textbf{\textit{kïkal}} \textit{na} \textbf{\textit{mawana}}.\\
\gll ndï  mbï-i      nïmal  mbï-p-en    ndï  \textbf{kïkal}  na ma=\textbf{wana}\\
    3\textsc{pl}  here-go.\textsc{pfv}  river  here-be\textsc{{}-nmlz}  \textsc{3pl}  ear    talk    3\textsc{sg.obj}=feel\\
\glt `Those who came here and stay here at the river would hear the message.’ [ulwa028\_01:44]
\z

\ea%56
    \label{ex:phrase:56}
          \textit{U} \textbf{\textit{kïkal}} \textbf{\textit{mawane}}.\\
\gll u    \textbf{kïkal}  ma=\textbf{wana}{}-e\\
    2\textsc{sg}  ear    3\textsc{sg.obj}=feel-\textsc{dep}\\
\glt `You heard it.’ [ulwa037\_00:29]
\z

Viewed from an alternative perspective, such verbal constructions can be said to be lacking “incorporation” rather than exhibiting “separation”. In this view, sentences such as \REF{ex:phrase:54} and \REF{ex:phrase:56} would be said to have “unincorporated” verbal structures.

  In constructions with \isi{separable verb}s, the first element always appears at the absolute beginning of the verb \isi{phrase}. \isi{Postposition}s, which are also properly constituents of VPs, appear after the first element, as in \REF{ex:phrase:57}, where \textit{kïkal} ‘ear’ is the first element in the \isi{separable verb} construction.

\ea%57
    \label{ex:phrase:57}
          \textit{Ango yeta ndï} \textbf{\textit{kïkal}} \textit{nïn u na} \textbf{\textit{ngalawan}}.\\
\gll ango  yeta  ndï  \textbf{kïkal}  nï=n    u    na    ngala=\textbf{wana}\\
    \textsc{neg}  man  3\textsc{pl}  ear    1\textsc{sg=obl}  from  talk  \textsc{pl.prox}=feel\\
\glt `No men hear these stories from me.’ [ulwa014\_48:48]
\z

Such \isi{separable verb} constructions are especially common with \is{perception verb} verbs of perception. Thus, verbs of seeing function similarly to this verb of hearing. Such ‘seeing’ constructions combine the word \textit{lïmndï} ‘eye’ with the irregular \isi{suppletive} verb \textit{ala-} {\textasciitilde} \textit{andï-} ‘see’. If there is an expressed object, it occurs between these two elements. Example \REF{ex:phrase:58} demonstrates an \isi{intransitive} use of this verb.

\ea%58
    \label{ex:phrase:58}
          \textit{An ambi nape \textbf{lïmndï ala}}.\\
\gll an      ambi  na-p-e      \textbf{lïmndï}  \textbf{ala}\\
    1\textsc{pl.excl}  big    \textsc{detr-}be\textsc{{}-dep} eye    see\\
\glt `When we had gotten big, [we] looked around.’ [ulwa013\_04:46]
\z

Examples \REF{ex:phrase:59} through \REF{ex:phrase:63}, on the other hand, are \isi{transitive} sentences, which illustrate the separable element \textit{lïmndï} ‘eye’ occurring before the \isi{direct object} of the verb.

\is{phrase|)}
\is{verb phrase|)}
\is{separable verb|)}
\is{separable verb|(}
\is{verb phrase|(}
\is{phrase|(}

\ea%59
    \label{ex:phrase:59}
          \textit{Unan ango \textbf{lïmndï} ankam \textbf{ala}}.\\
\gll unan    ango  \textbf{lïmndï}   ankam  \textbf{ala}\\
    1\textsc{pl.incl}  \textsc{neg}  eye      person  see\\
\glt `We haven’t seen anyone.’ [ulwa037\_07:11]
\z

\ea%60
    \label{ex:phrase:60}
          \textit{Mï wa i \textbf{lïmndï minala}}.\\
\gll mï      wa    i    \textbf{lïmndï}  min=\textbf{ala}\\
    3\textsc{sg.subj}  village  go.\textsc{pfv}  eye    3\textsc{du}=see\\
\glt `She came home and saw the two.’ [ulwa001\_00:52]
\z

\ea%61
    \label{ex:phrase:61}
          \textit{U amun \textbf{lïmndï} unji atuma \textbf{ngal}!}\\
\gll    u    amun  \textbf{lïmndï}  u-nji    atuma      nga=\textbf{al}\\
    2\textsc{sg}  now  eye    2\textsc{sg-poss}  older.brother  \textsc{sg.prox}=see\\
\glt `Now look at your older brother!’ [ulwa014†]
\z

\ea%62
    \label{ex:phrase:62}
          \textit{U amun \textbf{lïmndï} Gambri \textbf{andï}!}\\
\gll    u    amun  lïmndï  Gambri  \textbf{andï}\\
    2\textsc{sg}  now  eye    [name]    see\\
\glt `Now, take a look at Gambri!’ [ulwa014\_13:42]
\z

\ea%63
    \label{ex:phrase:63}
          \textit{Una} \textbf{\textit{lïmndï}} \textit{mangusuwa} \textbf{\textit{andïna}}.\\
\gll unan    \textbf{lïmndï}  ma-ngusuwa  \textbf{andï-na}\\
    1\textsc{pl.incl}  eye    3\textsc{sg.obj-}poor  see{}-\textsc{irr}\\
\glt `We will see the poor thing.’ [ulwa037\_46:49]
\z

In \REF{ex:phrase:64}, the first element \textit{lïmndï} ‘eye’ precedes a \isi{direct object} NP that itself contains a VP.

\ea%64
    \label{ex:phrase:64}
          \textit{Una} \textbf{\textit{lïmndï}} \textit{makape i} \textbf{\textit{mandïm}}.\\
\gll unan    \textbf{lïmndï}  [maka=p-e]    i    ma=\textbf{andï-m}\\
    1\textsc{pl.incl}  eye    [thus=\textsc{cop-dep]}  way  \textsc{3sg.obj}=see-\textsc{pfv}\\
\glt `We’ve seen this kind of behavior.’ [ulwa037\_10:23]
\z

Crucial for the argument that the nouns \textit{kïkal} ‘ear’ and \textit{lïmndï} ‘eye’ are truly (separable) parts of \isi{compound verb}s is the fact that they never receive \isi{postposition}s or \isi{oblique marker}s in these constructions. That is, they cannot be interpreted as belonging to other phrases. For example, constructions such as \REF{ex:phrase:65} are never found.

\ea[*]{%65
    \label{ex:phrase:65}
           \textbf{\textit{kïkaln(ï)}} \textit{mawana}\\
\gll    kïkal=\textbf{n(ï)}  ma=wana\\
    ear=\textsc{obl}  3\textsc{sg.obj}=feel\\
\glt    ‘sense with ear’ (i.e., ‘hear’) [elicited]}
\z

Expressions of visual perception can also be formed with other verbs \is{perception verb}, such as \textit{lï-} ‘put’ and \textit{uta-} ‘grind’. In all instances, the nominal element \textit{lïmndï} ‘eye’ behaves the same -- that is, it never receives any \isi{oblique} marking, as illustrated by \REF{ex:phrase:66}, \REF{ex:phrase:67}, and \REF{ex:phrase:68}.

\ea%66
    \label{ex:phrase:66}
          \textit{Mï \textbf{lïmndï malïp}}.\\
\gll mï      \textbf{lïmndï}  ma=\textbf{lï}{}-p\\
    3\textsc{sg.subj}  eye    3\textsc{sg.obj}=put-\textsc{pfv}\\
\glt `She watched it.’ [elicited]
\z

\ea%67
    \label{ex:phrase:67}
          \textit{Mï ndala wonka} \textbf{\textit{lïmndï}} \textit{manji asiya} \textbf{\textit{ndute}}.\\
\gll mï      ndï=ala  won-ka  \textbf{lïmndï}  ma-nji      asiya     ndï=\textbf{uta}{}-e\\
    3\textsc{sg.subj}  3\textsc{pl}=from  cut-let    eye    3\textsc{sg.obj-poss}  string    3\textsc{pl}=grind-\textsc{ipfv}\\
\glt `He left them and crossed [the river] and was checking his string traps.’ [ulwa032\_30:27]
\z

\ea%68
    \label{ex:phrase:68}
          \textit{Una} \textbf{\textit{lïmndï}} \textbf{\textit{ndutape}}.\\
\gll unan    \textbf{lïmndï}  ndï=\textbf{uta}{}-p-e\\
    1\textsc{pl.incl}  eye    3\textsc{pl}=grind-\textsc{pfv-dep}\\
\glt `We’ve examined them.’ [ulwa037\_31:38]
\z

While \is{perception verb} verbs of perception constitute one of the most common subclasses of verbs that exhibit the separable structure, other \isi{compound verb} forms behave similarly. The verb ‘ask’ is composed of the word \textit{atwana} ‘question’ and some form of a verb of speaking (e.g., \textit{ta-} ‘say’ or \textit{kï-} ‘say’) as \isi{discontinuous} elements, with no \isi{oblique} marking on the nominal component \textit{atwana} ‘question’, as in \REF{ex:phrase:69} and \REF{ex:phrase:70}.\footnote{More information on \isi{reported speech} with \textit{na} ‘talk’ as a nominal component is provided in \sectref{sec:13.4}.}

\ea%69
    \label{ex:phrase:69}
          \textit{Mï li} \textbf{\textit{atwana}} \textit{manji yana} \textbf{\textit{mat}}.\\
\gll mï      li-i        \textbf{atwana}  ma-nji      yana     ma=\textbf{ta}\\
    3\textsc{sg.subj}  down-go.\textsc{pfv}  question  3\textsc{sg.obj-poss}  woman    3\textsc{sg.obj}=say\\
\glt `He went down and asked his wife.’ [ulwa001\_13:59]
\z

\ea%70
    \label{ex:phrase:70}
          \textit{Dumngul imba pe i} \textbf{\textit{atwana}} \textbf{\textit{ankap}}.\\
\gll Dumngul  imba  p-e    i    \textbf{atwana}  an=\textbf{kï-p}\\
    [name]    night  be\textsc{{}-dep} go.\textsc{pfv}  question  1\textsc{pl.excl}=say-\textsc{pfv}\\
\glt `Dumngul came at night and asked us.’ [ulwa032\_19:37]
\z

  In a somewhat more complicated fashion, the verb ‘catch, grab, hold’ is formed with the \isi{irregular verb} \textit{si-} ‘push’, which follows the \isi{discontinuous} element [ikali], which is itself composed of \textit{i} ‘hand’ and \textit{kali} ‘send’, and may thus not so clearly be labeled a nominal element. Sentences \REF{ex:phrase:71} and \REF{ex:phrase:72} exemplify this structure.

\ea%71
    \label{ex:phrase:71}
          \textit{Una} \textbf{\textit{ikali}} \textbf{\textit{ndïsina}}.\\
\gll unan    \textbf{i-kali}    ndï=\textbf{si}{}-na\\
    1\textsc{pl.incl}  hand-send  3\textsc{pl}=push-\textsc{irr}\\
\glt `We can grab them.’ [ulwa037\_08:01]
\z

\ea%72
    \label{ex:phrase:72}
          \textit{Ngunan ango} \textbf{\textit{ikali}} \textit{ndïn u ani} \textbf{\textit{kos}}.\\
\gll ngunan    ango  \textbf{i-kali}    ndï=n    u    ani    ko=\textbf{si}\\
    \textsc{1du.incl}  \textsc{neg}  hand-send  3\textsc{pl=obl}  from  bilum  \textsc{indf}=push\\
\glt `We haven’t gotten a single \textit{bilum} [= string bag] from them.’ [ulwa037\_16:44]
\z

The use of \isi{light verb}s (such as \textit{wana-} ‘feel’, \textit{ta-} ‘say’, \textit{kï-} ‘say’, and \textit{si-} ‘push’) to generate a larger \isi{semantic} range than would otherwise be possible within Ulwa’s small set of verbs is reminiscent of many languages of \isi{New Guinea}. Indeed, this resembles the common \isi{adjunct}-plus-verb construction (\citealt[117--128]{Foley1986}), in which an \isi{adjunct nominal} combines with a generic verb to make the meaning of the generic verb more specific. One notable feature of these Ulwa constructions, however, is that the \isi{adjunct nominal} component is often \isi{morphological}ly very much like a verb -- that is, it can take \isi{verbal morphology}. This feature is described further in \sectref{sec:9.2.2} and \sectref{sec:9.2.3}. For the role of the verb \textit{tï-} ‘take’ in similar bipartite constructions, see the discussion in \sectref{sec:11.3} of possible \isi{serial verb construction}s.

\is{phrase|)}
\is{verb phrase|)}
\is{separable verb|)}

\subsection{The verbs \textit{u-} ‘put’, \textit{lï-} ‘put’, and \textit{lumo-} ‘put’}\label{sec:9.2.2}

\is{separable verb|(}
\is{verb phrase|(}
\is{phrase|(}

This section covers a small subclass of \isi{separable verb}s involving words with meanings somewhat like \ili{English} \textit{put}. I say “somewhat like” since there are two important \isi{semantic} distinctions. First, these Ulwa verbs select only two arguments -- that is, they are not three-place \isi{predicate}s. Indeed, they may even be able to function \isi{intransitive}ly as well. Second, the object of these Ulwa verbs is not a \isi{theme} argument, but rather a goal, the place to which a \isi{theme} is put. If a \isi{theme} argument is expressed in a clause, it occurs in an \isi{oblique} \isi{phrase}. One such verb is \textit{u-} ‘put’, which, as in \REF{ex:phrase:73} and \REF{ex:phrase:74}, takes a \isi{goal} as its object argument.

\ea%73
    \label{ex:phrase:73}
          \textit{Inom mï wa unde iwa lan \textbf{inim andawe}}.\\
\gll inom  mï      wa  unda-e  iwa      ala=n      \textbf{inim} \textbf{anda=aw}{}-e\\
    mother  3\textsc{sg.subj}  just  go-\textsc{dep}  basket  \textsc{pl.dist}=\textsc{obl}  water    \textsc{sg.dist}=put.\textsc{ipfv-dep}\\
\glt `A woman used to just go around, setting fish traps in the water.’ [ulwa006\_00:08]
\z

\ea%74
    \label{ex:phrase:74}
          \textit{Wen} \textbf{\textit{ndawe}}.\\
\gll we=n    \textbf{ndï=aw}{}-e\\
    sago=\textsc{obl}  3\textsc{pl}=put.\textsc{ipfv-dep}\\
\glt `[They] used to put sago starch in them.’ [ulwa014\_34:38]
\z

Another major ‘put’ verb is \textit{lï-} ‘put’, which also takes a \isi{goal} as its \isi{direct object} (and may include a \isi{theme} argument as an \isi{oblique} \isi{phrase}), as in \REF{ex:phrase:76}.


\ea%76
    \label{ex:phrase:76}
         \textit{\textbf{Al malpe} mï i.}\\
\gll    \textbf{al}  \textbf{ma=lï}{}-p-e        mï      i\\
    net  3\textsc{sg.obj}=put-\textsc{pfv-dep}  \textsc{3sg.subj}  go.\textsc{pfv}\\
\glt `Having put [the baby] in the mosquito net, she went.’ [ulwa001\_00:40]
\z


 In addition to these two verbs, there is the \isi{defective} \isi{stem} \textit{lumo-} ‘put’, which seems only to occur in the \isi{perfective} form [lumop] and in \isi{conditional} forms [lumota] and [lumopta]; sometimes, in casual speech, the initial [l-] is lost (i.e., the \isi{stem} may be \is{apheresis} apheresized to [umo-]). It is shown in \REF{ex:phrase:75}.

\ea%75
    \label{ex:phrase:75}
          \textit{Ndï malimap ndïn} \textbf{\textit{ame ndïlumop}}.\\
\gll ndï  ma=alima-p    ndï-n    \textbf{ame}    \textbf{ndï=lumo}{}-p\\
    3\textsc{pl}  \textsc{3sg.obj=}beat-\textsc{pfv}  \textsc{3pl=obl}  basket  3\textsc{pl}=put-\textsc{pfv}\\
\glt `They beat it [= the sago] and put them [= the starch] in the baskets.’ [ulwa014†]
\z

These ‘put’ verbs may combine with other elements in \isi{separable verb} constructions. When they do so, the first element resembles verbs in one crucial way: it permits an \isi{object marker}. For example, the (unseparated) verb \textit{kalilï-} ‘send’ takes as its object that which is sent (i.e., a \isi{theme} argument) \REF{ex:phrase:77}; however, as a \isi{discontinuous} verb, the first element \textit{kali} ‘send’ takes this \isi{theme} argument as its object, whereas the second element \textit{lï-} ‘put’ takes as its object the place to which someone or something is sent (i.e., a \isi{goal} argument) \REF{ex:phrase:78}.

\ea%77
    \label{ex:phrase:77}
          \textit{Wot} \textbf{\textit{makalilïpe}}.\\
\gll wot    ma=\textbf{kali-lï}{}-p-e\\
    younger  3\textsc{sg.obj}=send-put-\textsc{pfv-dep}\\
\glt `[They] sent the younger brother.’ [ulwa001\_13:58]
\z

\ea%78
    \label{ex:phrase:78}
          \textbf{\textit{Makali}} \textit{Nanïmwat} \textbf{\textit{malp}}.\\
\gll ma=\textbf{kali}    Nanïmwat  ma=\textbf{lï}{}-p\\
    \textsc{3sg.obj}=send  [place]    3\textsc{sg.obj}=put-\textsc{pfv}\\
\glt `[They] sent him to Nanïmwat [village].’\footnote{This example illustrates yet another peculiarity of the verb \textit{lï-} ‘put’: the common \isi{elision} of its only \isi{vowel} (\sectref{sec:2.5.9}).} [ulwa002\_03:34]
\z

  Similarly, the form \textit{kuk} ‘gather’ may take an \isi{object marker} when appearing with a separable ‘put’ verb. This form mostly appears with the verb \textit{u-} ‘put’, but may alternatively appear with a blended version containing the element /l/. Examples \REF{ex:phrase:79} and \REF{ex:phrase:80} show an \isi{intransitive} use of the verb.\footnote{According to some terminology, this usage may be considered “\isi{middle voice}”.} Example \REF{ex:phrase:79} lacks the \isi{detransitivizing} marker \textit{na-} ‘\textsc{detr}’, whereas \REF{ex:phrase:80} includes it (\sectref{sec:13.8.2}).

\ea%79
    \label{ex:phrase:79}
          \textit{Kuma} \textbf{\textit{kukup}}.\\
\gll kuma  \textbf{kuk-u}-p\\
    some  gather-put-\textsc{pfv}\\
\glt `Some gathered.’ [ulwa032\_43:36]
\z

\ea%80
    \label{ex:phrase:80}
          \textit{An} \textbf{\textit{nakukunda}}.\\
\gll an      na-\textbf{kuk-u}-nda\\
    1\textsc{pl.excl}  \textsc{detr-}gather-put-\textsc{irr}\\
\glt `We would gather.’ (i.e., ‘gather together, assemble’) [ulwa013\_05:53]
\z

As a \isi{transitive} verb, however, \textit{kuk u-} ‘gather’ has as its object that which is ‘gathered’ (or ‘piled up’, etc.), and this argument may be indexed by an \isi{object marker} preceding the form \textit{kuk} ‘gather’. The place in(to) which things are being gathered or piled is, in turn, the object of the verb ‘put’, and thus occurs as an NP between the separable form \textit{kuk} ‘gather’ and the \isi{verb stem} \textit{u-} ‘put’, as in examples \REF{ex:phrase:81} and \REF{ex:phrase:82}.

\is{phrase|)}
\is{verb phrase|)}
\is{separable verb|)}
\is{separable verb|(}
\is{verb phrase|(}
\is{phrase|(}

\ea%81
    \label{ex:phrase:81}
          \textit{Siwi} \textbf{\textit{kuk}} \textit{wa} \textbf{\textit{nolnda}}.\\
\gll siwi    \textbf{kuk}  wa  na-\textbf{lu}{}-nda\\
    grub.species  gather  village  \textsc{detr}{}-put-\textsc{irr}\\
\glt `[We] will gather grubs home.’\footnote{Note the \isi{metathesis} of the alternate form [lu-] of the \isi{verb stem} \textit{lï-} ‘put’ (\sectref{sec:2.5.9}), which enables the formation of the \is{monophthongization} \isi{monophthong} [o] (from /au/) (\sectref{sec:2.5.2}).} [ulwa038\_03:35]
\z

\ea%82
    \label{ex:phrase:82}
          \textit{Mï} \textbf{\textit{ndïkuk}} \textit{nïn ani} \textbf{\textit{mope}}.\\
\gll mï      ndï=\textbf{kuk}    nï=n    ani    ma=\textbf{u-}p-e\\
    3\textsc{sg.subj}  3\textsc{pl}=gather  1\textsc{sg=obl}  bilum  3\textsc{sg.obj}=put-\textsc{pfv-dep}\\
\glt `She piled them into my \textit{bilum} [= string bag].’ [ulwa037\_04:03]
\z

\is{phrase|)}
\is{verb phrase|)}
\is{separable verb|)}
\is{separable verb|(}
\is{verb phrase|(}
\is{phrase|(}

As a \isi{phonotactic}ally prohibited word-final \isi{consonant} (\sectref{sec:2.1.1}), the final [k] in \textit{kuk} ‘gather’ may be \isi{deleted} when this word occurs as a separate form \REF{ex:phrase:83}.\footnote{The \isi{nominal adjunct}s in \isi{separable verb} constructions such as this one may be forms that occur only in verbal \isi{compound}s -- that is, unlike \textit{lïmndï} ‘eye’, which occurs frequently as a noun in its own right (e.g., \textit{mï nïnji lïmndï masap} ‘he hit my eye’), there is no indication that forms like \textit{kuk} ‘gather’ (or ‘gathering’?), ever appear on their own without such verbs.}

\ea%83
    \label{ex:phrase:83}
         \textit{Nïpïl ndïwale} \textbf{\textit{ndïku}} \textit{inim} \textbf{\textit{awe}}.\\
\gll nïpïl  ndï=wali-e    ndï=\textbf{kuk}    inim  \textbf{aw}{}-e\\
    vine  3\textsc{pl}=hit-\textsc{ipfv}  3\textsc{pl}=gather  water  put.\textsc{ipfv-dep}\\
\glt `[We] used to break vines and gather them into the water.’ [ulwa036\_01:02]
\z

Other \isi{separable verb}s with \isi{stem}s meaning ‘put’ have as their first elements words that seem less likely to permit \isi{object marker}s. For example, \textit{tane} ‘stand’ has as its object the place where one stands, and this is marked on the ‘put’ verb, rather than on the form \textit{tane} ‘stand’. In \REF{ex:phrase:84}, this \isi{locative} argument is expressed by the \isi{object-marker} \isi{proclitic} \textit{ma=} ‘\textsc{3sg.obj}’, which in this \isi{phrase} has the sense ‘there’.

\ea%84
    \label{ex:phrase:84}
          \textit{Ngala imba pe \textbf{tane malpe}}.\\
\gll ngala    imba  p-e    \textbf{tane}  ma=\textbf{lï}{}-p-e\\
    \textsc{pl.prox}  night  be\textsc{{}-dep} stand  3\textsc{sg.obj}=put-\textsc{pfv-dep}\\
\glt `These people stand there at night.’ [ulwa032\_15:35]
\z

That said, this verb can at times permit two objects -- that is, the first element may permit as an object that which is stood (i.e., erected, positioned, etc.), as in \REF{ex:phrase:85}.

\ea%85
    \label{ex:phrase:85}
          \textit{I apa kongomlïp mat i} \textbf{\textit{matanelïp}}.\\
\gll i    apa    ko=angom-lï-p      ma=tï      i     ma=\textbf{tane-lï}{}-p\\
    go.\textsc{pfv}  house  \textsc{indf}=pull.out-put-\textsc{pfv}  3\textsc{sg.obj}=take  go.\textsc{pfv}    \textsc{3sg.obj=}stand-put-\textsc{pfv}\\
\glt `[It] went and pulled out a house, brought it, and stood it up.’ [ulwa006\_02:43]
\z

There are some other verbs that appear on \isi{morphological} grounds to belong to this class of separable ‘put’ verbs, but which never seem to occur as \isi{discontinuous} elements. This could simply be due to their \isi{semantics}. For example, the object of a verb such as \textit{mïmïlu-} ‘wring, strain’ is more likely to be a \isi{theme} argument than a \isi{goal} argument. This verb may be seen in sentences \REF{ex:phrase:86} and \REF{ex:phrase:87}.\footnote{The fact that ‘put’ verbs can, however, so commonly permit separable constructions has a certain rationale to it, especially considering that the object of these verbs glossed as ‘put’ is always the \isi{goal} and not the \isi{theme}, which, when overt, is expressed as an \isi{oblique} \isi{phrase}. Thus, in expressions like ‘send to a place’ or ‘gather/pile up to a place’, it is appropriate that the object of the second element in the \isi{separable verb} (i.e., the ‘put’ verb) is a destination.}

\ea%86
    \label{ex:phrase:86}
          \textit{Ulum tamndï mawa} \textbf{\textit{ndïmïmïlunda}}.\\
\gll ulum  tamndï  ma-awa    ndï=\textbf{mïmïl-u}{}-nda\\
    palm  owner  3\textsc{sg.obj-int}  \textsc{3pl=}wring-put-\textsc{irr}\\
\glt `The owner of the sago palms herself will wring them.’ [ulwa014\_59:45]
\z

\ea%87
    \label{ex:phrase:87}
          \textit{Ndï} \textbf{\textit{ndïmïmïlawe}}.\\
\gll ndï  ndï=\textbf{mïmïl-aw}{}-e\\
    3\textsc{pl}  \textsc{3pl=}wring-put.\textsc{ipfv-dep}\\
\glt `They would be wringing them.’ [ulwa014\_34:21]
\z

  The \isi{semantic} origins of ‘put’ verbs being used as the second component in such \isi{separable verb}s may be as follows: verbs like ‘throw’, ‘break’, and so on, could derive from phrases such as ‘put a throw’, ‘put a break’, and so on, where the first element in each \isi{phrase} is in origin an \isi{abstract noun}.

\tabref{tab::9.1} presents \isi{separable verb}s that most commonly use the main verb form \textit{u-} ‘put’. \tabref{tab::9.2} presents \isi{separable verb}s that most commonly use the main verb form \textit{lï-} ‘put’.

The two paradigms, however, are not completely distinct -- that is, although separable ‘put’ verbs mostly contain either one set of endings or the other set of endings, sometimes speakers mix forms, producing, for example, [tane-yu-p] ‘stand [\textsc{pfv]}’ (for /tane-lï-p/) or [kuk-lï-p] ‘gather [\textsc{pfv}]’ (for /kuk-u-p/). Generally, the verbs based on the stem \textit{lï-} ‘put’ lack \isi{imperfective} forms. If needed, however, they seem capable of adopting the [-awe] ending from the other ‘put’ verb paradigm.

\begin{table}
\caption{Separable ‘put’ verbs with verb stem \textit{u-}}
\is{imperfective}
\is{perfective}
\is{irrealis}
\is{verb stem}
\is{separable verb}
\label{tab::9.1}
\begin{tabularx}{\textwidth}{QQQQ}
\lsptoprule
gloss & imperfective & perfective & irrealis\\
\midrule
‘throw’ & {\itshape kïkeyawe} & {\itshape kïkeyup} & {\itshape kïkeyunda}\\
‘gather’ & {\itshape kukawe} & {\itshape kukup} & {\itshape kukunda}\\
‘wring’ & {\itshape mïmïlawe} & {\itshape mïmïlup} & {\itshape mïmïlunda}\\
‘die.\textsc{pl}’ & {\itshape nipinpawe} & {\itshape nipinpup} & {\itshape nipinpunda}\\
‘vomit’ & {\itshape nonganawe} & {\itshape nonganup} & {\itshape nonganunda}\\
‘crush’ & {\itshape nopalawe} & {\itshape nopalup} & {\itshape nopalunda}\\
‘break’ & {\itshape nungunawe} & {\itshape nungunup} & {\itshape nungununda}\\
‘pour’ & {\itshape tomalawe} & {\itshape tomalup} & {\itshape tomalunda}\\
‘cut’ & {\itshape weyawe} & {\itshape weyup} & {\itshape weyunda}\\
\lspbottomrule
\end{tabularx}
\end{table}


\begin{table}
\caption{Separable ‘put’ verbs with verb stem \textit{lï-}}
\is{imperfective}
\is{perfective}
\is{irrealis}
\is{verb stem}
\is{separable verb}
\label{tab::9.2}
\begin{tabularx}{\textwidth}{QQQQ}
\lsptoprule
gloss & imperfective & perfective & irrealis\\
\midrule
‘pull’ & {--} & {\itshape angomlïp} & {\itshape angomlïnda}\\
‘send’ & {--} & {\itshape kalilïp} & {\itshape kalilïnda}\\
‘throw’ & {--} & {\itshape kulilïp} & {\itshape kulilïnda}\\
‘tie’ & {--} & {\itshape moplïp} & {\itshape moplïnda}\\
‘hide’ & {--} & {\itshape nokoplïp} & {\itshape nokoplïnda}\\
‘spit’ & {--} & {\itshape ngomlïp} & {\itshape ngomlïnda}\\
‘stand’ & {--} & {\itshape tanelïp} & {\itshape tanelïnda}\\
‘throw’ & {--} & {\itshape toplïp} & {\itshape toplïnda}\\
‘jump’ & {--} & {\itshape uleplïp} & {\itshape uleplïnda}\\
\lspbottomrule
\end{tabularx}
\end{table}

\is{phrase|)}
\is{verb phrase|)}
\is{separable verb|)}

\newpage

\subsection{The verb \textit{ka-} ‘let’}\label{sec:9.2.3}

\is{separable verb|(}
\is{verb phrase|(}
\is{phrase|(}

The verb \textit{ka-} ‘let, leave, allow’ is another verb that is frequently used in \isi{separable verb} constructions. As a verb with \is{telicity} telic \isi{Aktionsart}, there is no distinction made between \isi{perfective} and \isi{imperfective} \isi{aspect}: both \isi{aspect}s are encoded with the un\isi{inflect}ed form of the verb [ka].\footnote{This form is \isi{homophonous} with the \isi{adverb} \textit{ka} ‘thus, in this manner, in that manner’ as well as with the \isi{postposition} \textit{ka} ‘at, in, on’. The \isi{irrealis} form [lakana] exhibits what appears to be \isi{circumfix}ation, \textit{la- … -na} (\sectref{sec:4.3}). The final [-na] of \textit{lakana} ‘let [\textsc{irr]’} is often \isi{elide}d. For the use of this verb in \isi{permissive} constructions, see \sectref{sec:13.9.4}.}

  Members of the class of \textit{ka-} ‘let’ \isi{separable verb}s tend to be \isi{intransitive}. When they do take objects, however, these are \isi{goal} arguments, much like their counterparts in ‘put’ \isi{separable verb}s (\sectref{sec:9.2.2}). These objects occur between the first element and the \isi{verb stem}. Examples \REF{ex:phrase:88} through \REF{ex:phrase:93} illustrate some uses of \isi{verb phrase}s formed with \textit{ka-} ‘let’.

\ea%88
    \label{ex:phrase:88}
          \textit{Nawa ndul} \textbf{\textit{asike}} \textit{ndï matap.}\\
\gll    nï-awa    ndï=ul    \textbf{asi-ka}{}-e  ndï  ma=ta-p\\
    1\textsc{sg-int}  3\textsc{pl}=with  sit-let\textsc{{}-dep}  \textsc{3pl}  \textsc{3sg.obj}=say-\textsc{pfv}\\
\glt `I myself sat with them, and they talked about it.’ [ulwa014\_29:14]
\z

\ea%89
    \label{ex:phrase:89}
          \textit{Nï wa ndul \textbf{asi maka}}.\\
\gll nï    wa  ndï=ul    \textbf{asi}  ma=\textbf{ka}\\
    1\textsc{sg}  just  3\textsc{pl}=with  sit  3\textsc{sg.obj}=let\\
\glt `I just sat there with them.’ [ulwa032\_18:56]
\z

\ea%90
    \label{ex:phrase:90}
          \textit{Nï ma} \textbf{\textit{loplakana}}.\\
\gll nï    ma  \textbf{lop}{}-la-\textbf{ka}{}-na\\
    1\textsc{sg}  go  lie-\textsc{irr}{}-let-\textsc{irr}\\
\glt `I will go and rest.’ [ulwa032\_18:21]
\z

\ea%91
    \label{ex:phrase:91}
          \textbf{\textit{Lop}} \textit{wulis} \textbf{\textit{maka}}.\\
\gll \textbf{lop}  wulis    ma=\textbf{ka}\\
    lie  platform  3\textsc{sg.obj}=let \\
\glt    ‘[I] lay on the platform.’ [ulwa037\_62:39]
\z

\ea%92
    \label{ex:phrase:92}
          \textit{Ngan wolka} \textbf{\textit{tïklika}} \textit{mbi.}\\
\gll    ngan    wolka  \textbf{tïkli-ka}  mbï-i\\
    1\textsc{du.excl}  again  turn-let    here-go.\textsc{pfv}\\
\glt `The two of us turned again and came here.’ [ulwa037\_01:43]
\z

\ea%93
    \label{ex:phrase:93}
          \textit{Una} \textbf{\textit{tïkli}} \textbf{\textit{amblalaka}} \textit{wolka amblawalinda man.}\\
\gll    unan    \textbf{tïkli}  ambla=la-\textbf{ka}    wolka  ambla=wali-nda ma-n\\
    1\textsc{pl.incl}  turn  \textsc{pl.refl=irr}{}-let  again  \textsc{pl.refl}=hit-\textsc{irr}    go-\textsc{ipfv}\\
\glt `We’re going to turn on one another and fight one another again.’ [ulwa037\_08:46]
\z

Often, however, instead of being expressed as the \isi{direct object} of the verb, the \isi{goal} argument in such constructions is expressed in a \isi{postpositional phrase}. Still, these phrases occur between the first element of the \isi{separable verb} and the \isi{verb stem}, as in \REF{ex:phrase:94}, \REF{ex:phrase:95}, and \REF{ex:phrase:96}.

\ea%94
    \label{ex:phrase:94}
          \textit{Nï} \textbf{\textit{asi}} \textit{unji komblam mayn} \textbf{\textit{ka}}.\\
\gll nï    \textbf{asi}  u-nji    komblam  ma=in      \textbf{ka}\\
    1\textsc{sg}  sit  2\textsc{sg-poss}  chair    3\textsc{sg.obj}=in  let\\
\glt `I sat in your chair.’ [elicited]
\z

\ea%95
    \label{ex:phrase:95}
          \textit{Wa} \textbf{\textit{asi}} \textit{nïmal kanam} \textbf{\textit{ka}}.\\
\gll wa  \textbf{asi}  nïmal  kanam  \textbf{ka}\\
    just  sit  river  beside  let\\
\glt `[They] just sit beside the river.’ [ulwa037\_45:28]
\z


\ea%96
    \label{ex:phrase:96}
          \textit{\textbf{Lop} ndïkana \textbf{ka} ko nip.}\\
\gll    \textbf{lop}  ndï=kana    \textbf{ka}  ko  ni-p\\
    lie  3\textsc{pl}=beside  let  just  die-\textsc{pfv}\\
\glt `[She] lay beside them and just died.’ [ulwa014†]
\z

\tabref{tab:9.3} provides a list of some separable \textit{ka-} ‘let’ verbs, showing their form in the \isi{imperfective}/\isi{perfective} and in the \isi{irrealis}.


\begin{table}
\caption{Separable \textit{ka-} ‘let’ verbs}
\is{imperfective}
\is{perfective}
\is{irrealis}
\label{tab:9.3}
\begin{tabularx}{\textwidth}{QQQ}
\lsptoprule
gloss & imperfective/perfective & irrealis\\
\midrule
‘sit’ & {\itshape asika} & {\itshape asilakana}\\
‘lie (down)’ & {\itshape lopka} & {\itshape loplakana}\\
‘turn (around)’ & {\itshape tïklika} & {\itshape tïklilakana}\\
‘bend’ & {\itshape tumulka} & {\itshape tumulakana}\\
‘cut’ & {\itshape wonka} & {\itshape wonlaka}\\
\lspbottomrule
\end{tabularx}
\end{table}

\is{phrase|)}
\is{verb phrase|)}
\is{separable verb|)}

Note the \isi{degemination} (\sectref{sec:2.5.8}) that occurs in the \isi{irrealis} form /tumul-la-ka-na/ ‘bend [\textsc{irr}]’). Also note that \textit{won-} ‘cut’ can alternatively take a set of regular \isi{TAM} endings (e.g., \textit{wonp} ‘cut [\textsc{pfv}]’, \textit{wonda} ‘cut [\textsc{irr}]’). The \isi{temporal} verbs \textit{ip ka-} ‘precede’ (literally ‘let nose’ or ‘let front’) and \textit{angani ka-} ‘follow’ (literally ‘let rear’), which are \isi{homophonous} with (and clearly related to) the \isi{adverb}s \textit{ipka} ‘before’ and \textit{anganika} ‘after’, also belong to this class of \isi{separable verb}s (see \sectref{sec:7.5} for their use in \is{ordinal numeral} ordinal number expressions).

\newpage

\section{Other phrasal constructions}\label{sec:9.3}

\is{phrase|(}

Besides noun phrases and \isi{verb phrase}s, the most important phrasal constituents of clauses are \isi{postpositional phrase}s (PPs). After describing PPs in Ulwa (\sectref{sec:9.3.1}), I consider the utility of describing \isi{adjectival phrase}s and \isi{adverbial phrase}s in the language (\sectref{sec:9.3.2}).

\is{phrase|)}

\subsection{Postpositional phrases}\label{sec:9.3.1}

\is{postpositional phrase|(}
\is{adpositional phrase|(}
\is{phrase|(}

A \isi{postpositional phrase} (PP) consists minimally of a \isi{postposition} and the object of the \isi{postposition}, which always immediately precedes it. The object of the \isi{postposition} may consist of a full NP, whether with or without an \isi{object marker}, or it may consist only of an \isi{object-marker} \isi{proclitic}. A number of examples of postpositions are provided in \sectref{sec:8.1}.

  Most \isi{postpositional phrase}s consist of just a single \isi{postposition}. However, it is also possible for multiple \isi{postposition}s to occur within a single \isi{phrase}, sharing a single object. This use of multiple \isi{postposition}s often conveys a specific (usually spatial) relationship between two NPs. A common component of such complex \isi{postpositional phrase}s is \textit{u} ‘from, in, at, around, along’, which, when following another \isi{postposition}, may add to it an \isi{ablative} sense, as in examples \REF{ex:phrase:97} through \REF{ex:phrase:100}.\footnote{An alternative analysis might be that at least some of these putative \isi{postposition}s are (or can function as) nouns. Thus, what is translated as ‘from atop the table’ in \REF{ex:phrase:97} may perhaps be better translated as ‘from the table’s top’.}

\ea%97
    \label{ex:phrase:97}
          \textit{Nï \textbf{aplatam mawat u} ani matïn.}\\
\gll    nï  \textbf{aplatam}  \textbf{ma=wat}    \textbf{u}    ani    ma=tï-n\\
    1\textsc{sg}  table  3\textsc{sg.obj}=atop  from  bilum  3\textsc{sg.obj}=take-\textsc{pfv}\\
\glt `I took the \textit{bilum} [= string bag] from atop the table.’ [elicited]
\z

\ea%98
    \label{ex:phrase:98}
          \textit{Lïwa ta \textbf{nïwat u} anmbi.}\\
\gll    lïwa  ta    \textbf{nï=wat}  \textbf{u}    an-mbï-i\\
    dawn  already  1\textsc{sg=}atop  from  out-here-go.\textsc{pfv}\\
\glt `Dawn already came out upon me.’ [ulwa041\_01:39]
\z

\ea%99
    \label{ex:phrase:99}
          \textit{Ndï \textbf{wimbam u} inim ma.}\\
\gll    ndï  \textbf{u=imbam}   \textbf{u}    inim  ma\\
    3\textsc{pl}  2\textsc{sg}=under  \textsc{2sg}  water  go\\
\glt `They go from under you to the water.’ (i.e., people go under your legs to lift you up   and take you to the water) [ulwa014\_00:35]
\z

\ea%100
    \label{ex:phrase:100}
          \textit{\textbf{Ndin u} siwi lomoke.}\\
\gll    \textbf{ndï=in}  \textbf{u}    siwi    ala=moko-e\\
    3\textsc{pl=}in  from  grub.species  \textsc{pl.dist}=take-\textsc{ipfv}\\
\glt `[He] would get grubs from within them.’ [ulwa004\_02:41]
\z

Sometimes, however, two \isi{postposition}s may occur in a single \isi{phrase} without any sense of \isi{motion}. In \REF{ex:phrase:101}, the two \isi{postposition}s \textit{wan} ‘over, above’ and \textit{wat} ‘atop, onto’ combine to give the sense of something hovering above something else.

\ea%101
    \label{ex:phrase:101}
          \textit{Yangun mï \textbf{aplatam mawan wat} wap.}\\
\gll    yangun    mï      \textbf{aplatam}  \textbf{ma=wan}      \textbf{wat}  wap\\
    mosquito  3\textsc{sg.subj}  table    3\textsc{sg.obj}=above  atop  be.\textsc{pst}\\
\glt `The mosquito was above the table.’ [elicited]
\z

It is even possible for three \isi{postposition}s to occur within a single \isi{phrase}, as demonstrated by examples \REF{ex:phrase:102} and \REF{ex:phrase:103}.

\ea%102
    \label{ex:phrase:102}
          \textit{Yangun mï \textbf{aplatam mawan wat u} mbi.}\\
\gll    yangun    mï      \textbf{aplatam}  \textbf{ma=wan}      \textbf{wat}  \textbf{u} mbï-i\\
    mosquito  3\textsc{sg.subj}  table    3\textsc{sg.obj}=above  atop  from    here{}-go.\textsc{pfv}\\
\glt `The mosquito came from above the table.’ [elicited]
\z

\ea%103
    \label{ex:phrase:103}
          \textit{Nongami \textbf{mawan wat u} molop.}\\
\gll    Nongami  \textbf{ma=wan}      \textbf{wat}  \textbf{u}    ma=lo-p\\
    [name]    3\textsc{sg.obj}=above  atop  from  3\textsc{sg.obj}=cut-\textsc{pfv}\\
\glt `Nongami cut it from above it.’ (i.e., he cut a sago palm by positioning himself above the palm) [ulwa014\_72:44]
\z

These PPs consisting of multiple \isi{postposition}s should not be confused with series of multiple PPs occurring within a single clause. The latter construction type always contains multiple objects (one per \isi{head postposition} in each PP), as in \REF{ex:phrase:104} and \REF{ex:phrase:105}.

\is{phrase|)}
\is{adpositional phrase|)}
\is{postpositional phrase|)}

\ea%104
    \label{ex:phrase:104}
          \textbf{\textit{Maya}} \textit{al} \textbf{\textit{men}} \textit{i.}\\
\gll    ma=\textbf{iya}      al  ma=\textbf{in}      i\\
    3\textsc{sg.obj}=toward  net  3\textsc{sg.obj=}in  go.\textsc{pfv}\\ 
\glt `[It] went to him into [his] mosquito net.’ [ulwa006\_05:07]
\z

\ea%105
    \label{ex:phrase:105}
\is{phrase}
\is{adpositional phrase}
\is{postpositional phrase}
          \textit{Min \textbf{mawl mawat} wap.}\\
\gll    min  \textbf{ma=ul}      \textbf{ma=wat}    wap\\
    3\textsc{du}  3\textsc{sg.obj}=with  3\textsc{sg.obj}=atop  be.\textsc{pst}\\
\glt `The two stayed with her on top of it.’ [ulwa019\_00:33]
\z

\subsection{Adjectival or adverbial phrases?}\label{sec:9.3.2}

\is{adjectival phrase|(}
\is{adverbial phrase|(}
\is{phrase|(}

This chapter may be concluded with a consideration of other possible phrasal units. Although constituents such as adjectival phrases and adverbial phrases are sometimes described for languages, there does not seem to be much utility in doing so for Ulwa. When multiple \isi{adjective}s occur in sequence, either:
  
\begin{quote}
\begin{enumerate}[noitemsep, label={(\roman*)}, align=left, widest=190, labelsep=1ex,leftmargin=*]
\item they are all in the same NP, together modifying the same \isi{head noun}; or
\item they are in the same \isi{predicate}, being predicated of the same subject; or
\item at least one is a substantive, with the other(s) modifying it or being predicated of it.
\end{enumerate}
\end{quote}

  First, when multiple \isi{adjective}s modify the same \isi{head noun}, it is not clear whether one or another \isi{adjective} has a closer affinity to the \isi{head noun}. In other words, it is not clear what the constituent structure is. An example of a noun being modified by multiple \isi{adjective}s is given in \REF{ex:phrase:106}.

\ea%106
    \label{ex:phrase:106}
          \textit{lamndu ambi anma mï}\\
\gll    lamndu  ambi  anma  mï\\
    pig      big    good  3\textsc{sg.subj}\\
\glt    (a) ‘the big [good pig]’ (?)

    (b) ‘the good [big pig]’ (?)

    (c) ‘the [big (and) good pig]’ (?) [elicited]
\z

Second, when an NP has multiple \isi{predicate adjective}s, it may be most parsimonious to analyze them as \isi{coordinate}d \isi{paratactic}ally, since Ulwa does not contain overt \isi{coordinator}s (\sectref{sec:12.1}). Such an analysis is employed for sentences such as \REF{ex:phrase:107}.

\ea%107
    \label{ex:phrase:107}
          \textit{Lamndu mï ambi anma.}\\
\gll    lamndu  mï      ambi  anma\\
    pig      3\textsc{sg.subj}  big    good\\
\glt `The pig is big [and] good.’ [elicited]
\z

Third, when one \isi{adjective} in a series is functioning as a substantive, it may indeed be the \isi{head} of a \isi{phrase} -- but this \isi{phrase} in question is a \isi{noun phrase}, not an \isi{adjectival phrase}, as illustrated by \REF{ex:phrase:108} and \REF{ex:phrase:109}.

\ea%108
    \label{ex:phrase:108}
          \textit{ambi anma mï}\\
\gll    ambi  anma  mï\\
    big    good  \textsc{3sg.subj}\\
\glt `the good big [one]’ [elicited]
\z

\ea%109
    \label{ex:phrase:109}
          \textit{Ambi (mï) anma.}\\
\gll    ambi  (mï)    anma\\
    big    (3\textsc{sg.subj)}  good\\
\glt `The big [one] is good.’ [elicited]
\z

Nor does there seem to be much value in analyzing a set of adverbial phrases. The class of \isi{adverb}s in Ulwa consists of modifiers that are mostly all considered to be \is{sentential modifier} sentential -- that is, insofar as they are modifiers, they modify on the level of the sentence or clause, and do not modifier smaller constituents, such as verbs or \isi{adjective}s. Thus, they are generally not themselves constituents of larger phrases. Furthermore, when multiple \isi{adverb}s occur in the same clause, they each, independently, modify this clause. Therefore, they do not seem to belong to any multi-word constituent unit smaller than the clause or sentence.

\is{phrase|)}
\is{adverbial phrase|)}
\is{adjectival phrase|)}

