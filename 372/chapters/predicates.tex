\chapter{Predicates}\label{sec:10}

\is{predicate|(}

In this chapter I describe how different types of predicates are formed. I begin by discussing \isi{locational predication} (\sectref{sec:10.1}) before turning to types of \isi{non-verbal predication}, whether accomplished without any overt predicative marking (\sectref{sec:10.2}), accomplished with a \isi{copular clitic} (\sectref{sec:10.3}), or accomplished with a special \isi{past}-\isi{tense} verb form (\sectref{sec:10.4}). I conclude with a discussion of a limited number of \isi{complex predicate}s involving auxiliary verbs (\sectref{sec:10.5}).

\is{predicate|)}

\section{Locational predication}\label{sec:10.1}

\is{locational predication|(}
\is{predicate|(}
\is{predication|(}
\is{locative clause}

Locative clauses are indicated with the \isi{locative verb} \textit{p-} ‘be, be at (be located at)’ (\sectref{sec:4.3}). The verb \textit{p-} ‘be’ may be used to express \isi{location} at a place, whether permanent or temporary. The \isi{locative verb} \textit{p-} ‘be’ can follow either proper \isi{place name}s \REF{ex:pred:1} or \isi{common noun} \isi{location}s \REF{ex:pred:2}.

\ea%1
    \label{ex:pred:1}
            \textit{Tumbuna la mbïpe \textbf{Wopata pe} ala ando.}\\
\gll    tumbuna    ala       mbï-p-e    \textbf{Wopata}  \textbf{p-e} ala      anda=u\\
    grandparent  \textsc{pl.dist}  here-be-\textsc{dep}  [place]    be-\textsc{dep}    \textsc{pl.dist}  \textsc{sg.dist}=from\\
\glt `When the grandparents were here at Wopata they [went] from there.’ (\textit{tumbuna} = TP) [ulwa029\_07:01]
\z

\ea%2
    \label{ex:pred:2}
            \textit{Alum ala ndï wa lolop ala \textbf{wandam pe} mundu ame.}\\
\gll    alum  ala      ndï  wa   lolop  ala      \textbf{wandam}  \textbf{p-e} mundu  ama-e\\
    child  \textsc{pl.dist}  3\textsc{pl}  just  just    \textsc{pl.dist}  jungle    be-\textsc{dep}    food  eat-\textsc{ipfv}\\
\glt `Those children -- they just, they’re just in the jungle eating food.’ [ulwa032\_12:50]
\z

This verb is irregular in both form and function. First, it is \isi{defective} in that it lacks a \isi{perfective} form.\footnote{This is perhaps unsurprising, given both the \isi{semantic} properties of the verb and the fact that a \isi{perfective} form like \textsuperscript{†}/pp/ very well might just \isi{degeminate} to [p] anyway. On the other hand, \isi{double perfective} forms such as [p-ap], [p-op], and [p-ïp], manage to solve this problem by means of \isi{vowel} insertion (\sectref{sec:4.8}).} Also, given the function of the verb -- often used along with other, non-\isi{locative verb}s -- it is often difficult to interpret the ending [-e] as being either \isi{imperfective} marking or \is{dependent marker} dependent marking. The \isi{irrealis} form of the verb, however, is rather simple to analyze: it takes the form [pïna] -- that is, the \isi{stem} \textit{p-} ‘be’ plus the regular \isi{irrealis} \isi{suffix} \textit{-na} ‘\textsc{irr}’, with an \isi{epenthetic} [ï] to break up the underlying \isi{consonant cluster}. The forms of \textit{p-} ‘be’ are given in \REF{ex:pred:2a}.

\ea%2a
    \label{ex:pred:2a}
The \isi{locative verb} \textit{p-} \normalfont{‘be, be at (be located at)’}\\
\begin{tabbing}
{(\textit{p-na})} \= {(‘be (\textsc{ipfv}?) (\textsc{dep}?)’)}\kill
{\textit{p}} \> {‘be’} (unmarked for \isi{tense} or \isi{aspect})\\
{\textit{p-e}} \> {‘be (\textsc{ipfv}?) (\textsc{dep}?)’}\\
{\textit{p-na}} \> {‘be (\textsc{irr})’ [pïna]}\\
{\textit{wap}} \> {‘be (\textsc{pst})’}
\end{tabbing}
\z

  Although there are no \isi{aspect}ual distinctions overtly encoded for this verb, there is a sort of \isi{tense} distinction that can be made, by means of the weakly \isi{suppletive} form \textit{wap} ‘be.\textsc{pst}’, which may be used to make explicit reference to \isi{past} \isi{time}. This form, itself perhaps derived from \textit{p-} ‘be’, is discussed further in \sectref{sec:10.4}. The verb \textit{p-} ‘be’ is almost certainly the source of the \isi{copular clitic} \textit{=p} ‘\textsc{cop}’, which is discussed in \sectref{sec:10.3}.
  
 The un\isi{inflect}ed form [p] may refer either to \isi{past} \isi{time} or to \isi{present} \isi{time}. With reference to \isi{location} in the \isi{future}, however, the \isi{locative verb}, as other verbs, typically receives the \isi{irrealis} \isi{suffix} \textit{-na} ‘\textsc{irr}’ \REF{ex:pred:3}.

\ea%3
    \label{ex:pred:3}
            \textit{Uta nungol kwa \textbf{wandam pïna}}.\\
\gll uta    nungol  kwa  \textbf{wandam}  \textbf{p-na}\\
    bird  child  one    jungle    be-\textsc{irr}\\
\glt `[Not] one little bird will be [left] in the jungle!’ [ulwa032\_53:17]
\z

As with other verbs, the \isi{irrealis} \isi{suffix} encodes not only \isi{future} \isi{time} for the verb \textit{p-} ‘be’, but also other non-real \isi{modal}ities, such as \isi{counterfactual}s \REF{ex:pred:4}.

\ea%4
    \label{ex:pred:4}
            \textit{Ango kwe kuma wa} \textbf{\textit{mapïna}}.\\
\gll ango  kwe  kuma  wa    ma=\textbf{p-na}\\
    \textsc{neg}  one    some  just    3\textsc{sg.obj}=be-\textsc{irr}\\
\glt `Not just one or a few would be staying there.’ [ulwa014\_65:18]
\z

Sentence \REF{ex:pred:4} further illustrates how the \isi{locative verb} \textit{p-} ‘be’ can be indexed with \isi{proclitic} \isi{object marker}s or \isi{deictic} forms. In some instances, it may perhaps even be considered a \isi{transitive} verb, taking as its object the place at which something is or was located. In such situations, the verb can be translated as ‘be located at’, ‘stay at’, ‘live at’, ‘reside at’, and so on. This function is illustrated by sentences \REF{ex:pred:5} through \REF{ex:pred:12}.

\is{predication|)}
\is{predicate|)}
\is{locational predication|)}
\is{locational predication|(}
\is{predicate|(}
\is{predication|(}

\ea%5
    \label{ex:pred:5}
            \textit{Mï \textbf{wa mape}}.\\
\gll mï      \textbf{wa}    \textbf{ma=p}-e\\
    3\textsc{sg.subj}  village  3\textsc{sg.obj}=be-\textsc{ipfv}\\
\glt `She stayed in the village.’ [ulwa020\_00:09]
\z

\ea%6
    \label{ex:pred:6}
            \textit{Ndï maka \textbf{ndïnji wa mape}}.\\
\gll ndï  maka  \textbf{ndï-nji}    \textbf{wa}    \textbf{ma=p}-e\\
    3\textsc{pl}  thus  3\textsc{pl-poss}  village  3\textsc{sg.obj}=be-\textsc{ipfv}\\
\glt `They thus stayed in their village.’ [ulwa018\_06:04]
\z

\ea%7
    \label{ex:pred:7}
            \textit{Ambi ngata nduwe ndï \textbf{amba mape}}.\\
\gll ambi  ngata  ndï-we      ndï    \textbf{amba}      \textbf{ma=p}-e\\
    big    grand  \textsc{3pl-part.int}  3\textsc{pl}    mens.house  3\textsc{sg.obj}=be-\textsc{ipfv}\\
\glt `Only the big grandparents – they stayed in the men’s house.’ [ulwa018\_04:55]
\z

\ea%8
    \label{ex:pred:8}
            \textit{Inom manji mï \textbf{ata ngap}}.\\
\gll inom  ma-nji      mï      \textbf{ata}  \textbf{nga=p}\\
    mother  3\textsc{sg.obj-poss}  3\textsc{sg.subj}  up  \textsc{sg.prox}=be\\
\glt `The mother’s [garden] was upstream.’ [ulwa001\_06:54]
\z

\ea%9
    \label{ex:pred:9}
            \textit{Nga \textbf{Tïwen ngape}}.\\
\gll nga      \textbf{Tïwen}  \textbf{nga=p}-e\\
    \textsc{sg.prox}  [place]  \textsc{sg.prox}=be-\textsc{ipfv}\\
\glt `Here -- [they] stayed here in Tïwen.’ [ulwa014\_23:57]
\z

\ea%10
    \label{ex:pred:10}
          \textit{Mï amunpe} \textbf{\textit{andape}}.\\
\gll   mï      amun=p-e    \textbf{anda=p}-e\\
    3\textsc{sg.subj}  now=\textsc{cop-dep}  \textsc{sg.dist}=be-\textsc{ipfv}\\
\glt `He is still there.’ [ulwa014\_08:14]
\z

\ea%11
    \label{ex:pred:11}
          \textit{An wa ndimbam li \textbf{apembam ndape}}.\\
\gll an      wa    ndï=imbam  li    \textbf{apembam anda=p}-e\\
    1\textsc{pl.excl}  just    3\textsc{pl}=under    down  area.beneath.house    \textsc{sg.dist}=be-\textsc{ipfv}\\
\glt `We were just staying in the area beneath the house down underneath them.’ [ulwa018\_04:47]
\z

\ea%12
    \label{ex:pred:12}
          \textit{Ase itom ala mïka \textbf{apa ndape}}.\\
\gll ase  itom  ala      mïka  \textbf{apa}  \textbf{anda=p}-e\\
    no  father  \textsc{pl.dist}  thus  house  \textsc{sg.dist}=be-\textsc{ipfv}\\
\glt `No, the fathers are in the house.’ [ulwa018\_05:12]
\z

The \isi{locative verb} \textit{p-} ‘be’ can take other verbal \isi{affix}ation, such as the \isi{dependent marker} \textit{-e} ‘\textsc{dep}’, as exemplified in examples \REF{ex:pred:1} and \REF{ex:pred:2}. Examples \REF{ex:pred:13} and \REF{ex:pred:14} show \textit{p-} ‘be’ with the \isi{conditional} marker \textit{-ta} ‘\textsc{cond}’.

\ea%13
    \label{ex:pred:13}
          \textit{Ndï wa} \textbf{\textit{pïta}}\\
\gll    ndï  wa    p-\textbf{ta}\\
    3\textsc{pl}  village  be-\textsc{cond}\\
\glt `If they were home …’ [ulwa033\_01:07]
\z

\ea%14
    \label{ex:pred:14}
          \textit{… maka amba} \textbf{\textit{ngapta}} \textit{ndïlanda man.}\\
\gll    maka  amba      nga=p-\textbf{ta}        ndï=la-nda    ma-n\\
    thus  mens.house  \textsc{sg.prox}=be-\textsc{cond}  3\textsc{pl}=eat-\textsc{irr}  go-\textsc{ipfv}\\
\glt `… if [we] stay like this in the men’s house, then [we] are going to eat them.’ [ulwa018\_01:23]
\z

It is also possible for \textit{p-} ‘be’ to take the \isi{nominalizing} \isi{suffix} \textit{-en} ‘\textsc{nmlz}’. The force of the \isi{nominalization} is not always felt \REF{ex:pred:15}.

\ea%15
    \label{ex:pred:15}
          \textit{A nïplopa im in pe wandam} \textbf{\textit{pen}}.\\
\gll a    nïplopa  im    in  p-e    wandam  p-\textbf{en}\\
    \textsc{interj}  flying.fox  tree  in  be-\textsc{dep}  jungle    be-\textsc{nmlz}\\
\glt `Ah, there are [still] flying foxes in the trees in the jungle.’ [ulwa032\_57:17]
\z

Example \REF{ex:pred:15} further illustrates the use of the \isi{locative verb} in combination with a \isi{postpositional phrase} (here, with the \isi{postposition} \textit{in} ‘in’). Further examples of \isi{locative predication} with \isi{spatial postposition}s are given in \REF{ex:pred:16}, \REF{ex:pred:17}, and \REF{ex:pred:18}.

\ea%16
    \label{ex:pred:16}
          \textit{Nïtet nïmal \textbf{kanam mapen} ndï manji wo.}\\
\gll    Nïtet  nïmal  \textbf{kanam}    \textbf{ma=p}{}-en      ndï  ma-nji      wo\\
    [place]  river  beside    3\textsc{sg.obj}=be-\textsc{nmlz}  3\textsc{pl}  3\textsc{sg.obj-poss}  own\\
\glt `Those [palms] that are next to Nïtet river are her very own.’\footnote{This example additionally illustrates the use of the \isi{nominalizing} \isi{suffix} \textit{-en} ‘\textsc{nmlz}’.} [ulwa037\_42:24]
\z

\ea%17
    \label{ex:pred:17}
          \textit{Way mï minïn twa} \textbf{\textit{kana map}}.\\
\gll way  mï      min=n twa \textbf{kana} \textbf{ma=p}\\
    turtle  3\textsc{sg.subj}  3\textsc{du=obl}  hearth  beside  3\textsc{sg.obj}=be\\
\glt `The turtle stayed there with them next to the hearth.’\footnote{Sentences such as \REF{ex:pred:16} and \REF{ex:pred:17}, which contain the \isi{object marker} \textit{ma=} ‘\textsc{3sg.obj}’ following the \isi{postposition} \textit{kana {\textasciitilde} kanam} ‘beside’, are perhaps better translated as ‘there beside …’ or ‘there next to …’.} [ulwa006\_06:09]
\z

\ea%18
    \label{ex:pred:18}
          \textit{Nï ango unul ini} \textbf{\textit{ngawat pïna}}.\\
\gll nï    ango  un=ul    ini    nga=\textbf{wat}    \textbf{p-}na\\
    1\textsc{sg}  \textsc{neg}  2\textsc{pl=}with  ground  this.\textsc{sg}=atop  be-\textsc{irr}\\
\glt `I will not live on this land with you.’ [ulwa014\_03:42]
\z

  The distinction between \isi{locative predication} and \isi{existential predication} is often not clear in Ulwa, since there are no formal distinctions between the two constructions. Like \isi{locative predication}, \isi{existential predication} can be expressed with the \isi{locative verb} \textit{p-} ‘be’, as in \REF{ex:pred:19}, \REF{ex:pred:20}, and \REF{ex:pred:21}.

\ea%19
    \label{ex:pred:19}
          \textit{Anmoka ndï wandam} \textbf{\textit{map}}\\
\gll    anmoka  ndï  wandam  ma=\textbf{p}\\
    snake    3\textsc{pl}  jungle    3\textsc{sg.obj}=be\\
\glt `There are snakes in the jungle.’ [elicited]
\z

\ea%20
    \label{ex:pred:20}
          \textit{Inim mï ini mawat} \textbf{\textit{pe}}.\\
\gll inim  mï      ini    ma=wat    \textbf{p}{}-e\\
    water  3\textsc{sg.subj}  ground  3\textsc{sg.obj}=atop  be\textsc{{}-ipfv}\\
\glt `There is water on the ground.’ [elicited]
\z

\ea%21
    \label{ex:pred:21}
          \textit{Wanmbi ani} \textbf{\textit{mapta}} \textit{u mat nïnata nï ansi lan.}\\
\gll    wanmbi  ani    ma=\textbf{p}{}-ta      u    ma=tï     nï=na-ta      nï    ansi    la{}-n[da]\\
    daka    bilum  3\textsc{sg.obj}=be\textsc{{}-cond}  \textsc{2sg}  \textsc{3sg.obj=take}  1\textsc{sg}=give-\textsc{cond}  1\textsc{sg}  red.buai  eat-\textsc{irr}\\
\glt `If there is \textit{daka} [= betel pepper] in [your] \textit{bilum} [= string bag], [then] give it to me so I can chew \textit{red buai} [= betel nut].’ [ulwa037\_34:22]
\z

Often context alone can determine whether a \isi{copula}r \isi{suffix} is being used in an existential construction or a \isi{locative} construction. Thus, for example, a sentence such as \REF{ex:pred:20} could be interpreted as meaning ‘the water is on the ground’ as well as ‘there is water on the ground’.

  The verb \textit{p-} ‘be’ (or the \isi{copular enclitic} \textit{=p} ‘\textsc{cop}’) may be used with \isi{temporal adverb}s or nouns as well, here having a \isi{temporal}, rather than a spatial meaning, as in the expression \textit{amunpe} ‘still’ in \REF{ex:pred:10}. This can be understood as having the literal meaning ‘being at now’. See \sectref{sec:8.2.1} for more examples of \isi{temporal} expressions. The special \isi{past}-\isi{tense} form of the \isi{locative verb} (\textit{wap} ‘be.\textsc{pst}’) is discussed in \sectref{sec:10.4}.

\is{predication|)}
\is{predicate|)}
\is{locational predication|)}

\section{Non-verbal predication}\label{sec:10.2}

\is{non-verbal predication|(}
\is{predication|(}
\is{predicate|(}

Aside from \isi{locational predication}, which requires the \isi{locative verb} \textit{p-} {\textasciitilde} \textit{wap} ‘be’, \isi{non-verbal predication} (as well as other \isi{non-verbal clause} constructions) can be expressed without any overt verbal marking. Thus, Ulwa can be said to allow \isi{zero copula} constructions (see \sectref{sec:10.3}, however, for the use of a \isi{copular enclitic}). For example, \is{classificational clause} classificational, \is{identificational clause}  identificational, or \is{equational clause} equational (or \is{equative clause} equative) clauses may all be expressed by simply \isi{juxtaposing} two NPs without any marking, as in examples \REF{ex:pred:22} through \REF{ex:pred:27}. See also example \REF{ex:det:159} in \sectref{sec:7.3}.

\ea%22
    \label{ex:pred:22}
          \textit{Mongima mï} \textbf{\textit{yata}}.\\
\gll Mongima  mï      \textbf{yata}\\
    [name]    3\textsc{sg.subj}  man\\
\glt `Mongima is a man.’ [elicited]
\z

\ea%23
    \label{ex:pred:23}
          \textit{Mongima mï \textbf{ankam anma}}.\\
\gll Mongima  mï    \textbf{ankam}    \textbf{anma}\\
    [name]    3\textsc{sg}  person    good\\
\glt `Mongima is a good person.’ [elicited]
\z

\ea%24
    \label{ex:pred:24}
          \textit{Kowe mï \textbf{nïnji atuma}}.\\
\gll Kowe  mï      \textbf{nï-nji}    \textbf{atuma}\\
    [name]  3\textsc{sg.subj}  \textsc{1sg-poss}  older.brother\\
\glt `Kowe is my older brother.’ [elicited]
\z

\ea%25
    \label{ex:pred:25}
          \textit{Kowe Mongima min \textbf{nïnji atuma wot}}.\\
\gll Kowe  Mongima  min  \textbf{nï-nji}    \textbf{atuma}      \textbf{wot}\\
    [name]  [name]    3\textsc{du}  \textsc{1sg-poss}  older.brother  younger\\
\glt `Kowe and Mongima are my brothers.’\footnote{The combination of ‘older brother’ and ‘younger [brother]’ can be used to refer to multiple male siblings, unspecified for relative age.} [elicited]
\z

\ea%26
    \label{ex:pred:26}
          \textit{Kowe mï} \textbf{\textit{atuma}}.\\
\gll Kowe  mï      \textbf{atuma}\\
    [name]  3\textsc{sg.subj}  older.brother\\
\glt `Kowe is an older brother.’ (i.e., he is an older brother to some unspecified person) [elicited]
\z

\newpage

\ea%27
    \label{ex:pred:27}
          \textit{Ngata yeta mï} \textbf{\textit{Suwol}}.\\
\gll ngata  yeta  mï      \textbf{Suwol}\\
    grand  man  3\textsc{sg.subj}  [name]\\
\glt `The male ancestor was Suwol.’ [ulwa002\_03:19]
\z

  Similarly, \isi{attributive clause}s (or \isi{attributional clause}s) can be formed by placing the property-denoting \isi{phrase} (i.e., the \isi{predicate adjective}) immediately after the subject NP, without any special marking, as in \REF{ex:pred:28} and \REF{ex:pred:29}.

\ea%28
    \label{ex:pred:28}
          \textit{Kowe mï} \textbf{\textit{wutota}}.\\
\gll Kowe  mï      \textbf{wutota}\\
    [name]  3\textsc{sg.subj}  tall\\
\glt `Kowe is tall.’ [elicited]
\z

\ea%29
    \label{ex:pred:29}
          \textit{Kowe mï \textbf{apka wutota}}.\\
\gll Kowe  mï      \textbf{apka}  \textbf{wutota}\\
    [name]  3\textsc{sg.subj}  very  tall\\
\glt `Kowe is very tall.’ [elicited]
\z

Possessive predicates \is{possessive predication} can likewise be expressed without any marking. There is no verb like the \ili{English} verb \textit{have} in Ulwa. Rather, the \isi{possessed} item is simply predicated of the person who possesses it. The \isi{possessor} is indicated by the \isi{possessive pronoun}. Thus, predicative \isi{possession} is expressed with the \isi{possessum} as the single argument of a \isi{monotransitive} clause; the \isi{possessor} is treated just like any \isi{adnominal possessor} in a possessive NP (i.e., it precedes the \isi{head noun}, \sectref{sec:9.1.5}). In examples \REF{ex:pred:30} through \REF{ex:pred:33}, the \isi{possessum} is indicated in \textbf{bold}.

\ea%30
    \label{ex:pred:30}
          \textit{Alimban manji \textbf{yeta watangïnila}}.\\
\gll Alimban  ma-nji      \textbf{yeta}  \textbf{watangïnila}\\
    [name]    3\textsc{sg.obj-poss}  man  four\\
\glt `Alimban has four sons.’ [ulwa024\_01:39]
\z

\ea%31
    \label{ex:pred:31}
          \textit{Nï amun nïnji \textbf{palapal min nga}}.\\
\gll nï    amun  nï-nji    \textbf{palapal}    \textbf{min}  \textbf{nga}\\
    1\textsc{sg}  now  1\textsc{sg-poss}  decoration?  band  \textsc{sg.prox}\\
\glt    ‘Now I have this shell armband.’ (\textit{palapal} < TP \textit{balbal} {\textasciitilde} \textit{palpal} ‘Indian coral tree’?) [ulwa015\_02:21]
\z

\ea%32
    \label{ex:pred:32}
          \textit{Depina manji \textbf{samban andawa}}.\\
\gll Depina  ma-nji      \textbf{samban}  \textbf{anda-awa}\\
    [name]  3\textsc{sg.obj-poss}  pot      \textsc{sg.dist{}-int}\\
\glt `Depina had her own pot.’ [ulwa032\_23:29]
\z

\ea%33
    \label{ex:pred:33}
          \textit{Leobaha min minji \textbf{samban andawa}}.\\
\gll Leobaha  min  min-nji    \textbf{samban}  \textbf{anda-awa}\\
    [name]    3\textsc{du}  3\textsc{du-poss}  pot      \textsc{sg.dist{}-int}\\
\glt `Leobaha [and another child] had their own pots.’ [ulwa032\_23:34]
\z

\isi{Negation} of \isi{non-verbal clause}s is discussed in \sectref{sec:13.3.2}.

\is{predicate|)}
\is{predication|)}
\is{non-verbal predication|)}

\section{The enclitic copula \textit{=p} ‘\textsc{cop}’}\label{sec:10.3}

\is{copula|(}
\is{enclitic copula|(}
\is{copular enclitic|(}
\is{copular clitic|(}
\is{predicate|(}
\is{predication|(}
\is{non-verbal predication|(}

Although \isi{non-verbal clause}s can be formed without any overt \isi{verb phrase} (\sectref{sec:10.2}), it is also possible for a \isi{copular enclitic} \textit{=p} ‘\textsc{cop}’ to be to be added to a noun, \isi{adjective}, or other \isi{parts of speech} to create a \isi{predicate}. Although clearly derived historically from the \isi{locative verb} \textit{p-} ‘be at’ (< \ili{Proto-Keram} *ip), the \isi{copular enclitic} is not used for \isi{locational predication}. Rather, it is used -- always optionally -- for other kinds of \isi{non-verbal clause}s. For example, \is{classificational clause} classificational, \is{identificational clause} identificational, or \is{existential clause} existential clauses may include the \isi{clitic} \textit{=p} ‘\textsc{cop}’ at the end of the second NP, as in \REF{ex:pred:34}, \REF{ex:pred:35}, and \REF{ex:pred:36}.

\ea%34
    \label{ex:pred:34}
          \textit{Kowe mï} \textbf{\textit{atumap}}.\\
\gll Kowe  mï      atuma=\textbf{p}\\
    [name]  \textsc{3sg.subj}  older.brother=\textsc{cop}\\
\glt `Kowe is an older brother.’ [elicited]
\z

\ea%35
    \label{ex:pred:35}
          \textit{Mï} \textbf{\textit{wandampe}} \textit{mï wolka molop.}\\
\gll    mï      wandam-\textbf{p}{}-e    mï      wolka  ma=lo-p\\
    3\textsc{sg.subj}  jungle=\textsc{cop-dep}  3\textsc{sg.subj}  again  3\textsc{sg.obj}=cut\textsc{{}-pfv}\\
\glt `It was a jungle; but he cleared it again.’ [ulwa014\_55:13]
\z

\ea%36
    \label{ex:pred:36}
          \textit{Wondi} \textbf{\textit{ulwap}}.\\
\gll wondi    ulwa=\textbf{p}\\
    bandicoot  nothing=\textsc{cop}\\
\glt `There were no bandicoots.’ (Literally ‘Bandicoots are/were nothing.’) [ulwa032\_25:29]
\z

No \isi{tense} or \isi{aspect} distinction is made with this \isi{clitic}. However, the \isi{irrealis} \isi{suffix} \textit{-na} ‘\textsc{irr}’ is generally added for \isi{irrealis} clauses, and other verbal \isi{suffix}ation may be added as well, such as the \isi{dependent marker} \textit{-e} ‘\textsc{dep}’ in \REF{ex:pred:35}. The \isi{copular enclitic} can, in effect, function as a \isi{verbalizing} morpheme. In \REF{ex:pred:37} the \isi{irrealis} \isi{suffix} attaches to the \isi{copula}.

\ea%37
    \label{ex:pred:37}
          \textit{Kowe mï} \textbf{\textit{atumapïna}}.\\
\gll Kowe  mï      atuma=\textbf{p-na}\\
    [name]  \textsc{3sg.subj}  older.brother=\textsc{cop}{}-\textsc{irr}\\
\glt `Kowe will be an older brother.’ [elicited]
\z

Likewise, \isi{attributive clause}s (or \isi{attributional clause}s) receive the \isi{copular enclitic} after the \isi{phrase} containing the property word, as illustrated by examples \REF{ex:pred:38} through \REF{ex:pred:41}.

\ea%38
    \label{ex:pred:38}
          \textit{Itom mï} \textbf{\textit{ambip}}.\\
\gll itom  mï      ambi=\textbf{p}\\
    father  3\textsc{sg.subj}  big=\textsc{cop}\\
\glt `Father is big.’ [elicited]
\z

\ea%39
    \label{ex:pred:39}
          \textit{Itom mï} \textbf{\textit{ambipïna}}.\\
\gll itom  mï      ambi=\textbf{p-}na\\
    father  3\textsc{sg.subj}  big=\textsc{cop}{}-\textsc{irr}\\
\glt `Father will be big.’ [elicited]
\z

\ea%40
    \label{ex:pred:40}
          \textit{Na mï ango} \textbf{\textit{anmape}}.\\
\gll na    mï      ango  anma=\textbf{p}{}-e\\
    talk  \textsc{3sg.subj}  \textsc{neg}  good\textsc{=cop{}-dep}\\
\glt `The talk wasn’t good.’ [ulwa037\_00:39]
\z

\ea%41
    \label{ex:pred:41}
          \textit{Un ndïlakata kuma} \textbf{\textit{wapatapïta}!}\\
\gll    un  ndï=la-ka-ta    kuma  wapata=\textbf{p-}ta\\
    2\textsc{pl}  \textsc{3pl=irr-}let-\textsc{cond}  some  dry=\textsc{cop-cond}\\
\glt `Let some of them dry!’ (Literally ‘If you let them, some will be dry.’) [ulwa014\_54:34]
\z

It may be possible to use the \isi{copular enclitic} to indicate \isi{possessive predication}, although I have not found any clear examples of this. Sentence \REF{ex:pred:42}, for example, may rather represent the use of the \isi{locative verb} \textit{p-} ‘be’ (i.e., ‘My big one is located there.’).

\ea%42
    \label{ex:pred:42}
          \textit{Nïnji ambi \textbf{kwe mape}}.\\
\gll nï-nji    ambi  kwe  ma=\textbf{p}{}-e\\
    1\textsc{sg-poss}  big    one    3\textsc{sg.obj}=\textsc{cop?-dep}\\
\glt `I have one big one there.’ [ulwa042\_01:33]
\z

Example \REF{ex:pred:43}, which uses a \isi{nominalizing} \isi{suffix} after the \isi{copular clitic}, is perhaps better analyzed as \is{classificatory predication} classificatory or \isi{attributive predication} (i.e., ‘His land is nothing/nonexistent.’).

\is{non-verbal predication|)}
\is{predication|)}
\is{predicate|)}
\is{copular clitic|)}
\is{copular enclitic|)}
\is{enclitic copula|)}
\is{copula|)}

\ea%43
    \label{ex:pred:43}
          \textit{Manji ini} \textbf{\textit{ulwapen}}.\\
\gll ma-nji      ini    \textbf{ulwa=p-en}\\
    3\textsc{sg.obj-poss}  ground  nothing=\textsc{cop-nmlz}\\
\glt `He doesn’t have land.’ [ulwa014†]
\z

\section{{The} {past-tense} {locative} {verb} {\textit{wap}} {‘be.\textsc{pst}}{’}}\label{sec:10.4}

\is{locative|(}
\is{past|(}
\is{locative verb|(}
\is{non-verbal predication|(}
\is{predication|(}
\is{predicate|(}

Although the \isi{locative verb} \textit{p-} ‘be’ is unmarked for \isi{tense} and, as such, may be used to refer to either \isi{past} or \isi{present} \isi{time}, there is a weakly \isi{suppletive} form \textit{wap} ‘be.\textsc{pst}’, which may be used to make explicit reference to \isi{location}s in \isi{past} \isi{time}. Its use in \is{locative predication} locative predicates is illustrated by sentences \REF{ex:pred:44}, \REF{ex:pred:45}, and \REF{ex:pred:46}.

\ea%44
    \label{ex:pred:44}
          \textit{Amombi maka \textbf{Rabaul wap}}.\\
\gll Amombi  maka  \textbf{Rabaul}  \textbf{wap}\\
    [name]    thus  [place]    be.\textsc{pst}\\
\glt `Amombi was, like, in Rabaul.’ [ulwa037\_24:55]
\z

\ea%45
    \label{ex:pred:45}
          \textit{Ngay mawap Ramu i} \textbf{\textit{mawap}}.\\
\gll nga=i        ma=wap      Ramu  i    \textbf{ma=wap}\\
    \textsc{sg.prox}=go.\textsc{pfv}  3\textsc{sg.obj}=be.\textsc{pst}  [place]  go.\textsc{pfv}  3\textsc{sg.obj}=be.\textsc{pst}\\
\glt `[He] went there, stayed there, went to the Ramu area and stayed there.’ [ulwa037\_45:51]
\z

\ea%46
    \label{ex:pred:46}
          \textit{Kambaramba wa ambi maytap} \textbf{\textit{mawap}}. \textbf{\textit{Mawape}} \textit{wusim anden pe amblasap.}\\
\gll    Kambaramba  wa    ambi  ma=ita-p        \textbf{ma=wap}     \textbf{ma=wap}{}-e      wusim    anda=in    p-e ambla=asa-p\\
    [place]      village  big    3\textsc{sg.obj}=build-\textsc{pfv}  3\textsc{sg.obj}=be.\textsc{pst}    3\textsc{sg.obj}=be.\textsc{pst-dep}  crocodile  \textsc{sg.dist}=in    be\textsc{{}-dep}    \textsc{pl.refl}=hit-\textsc{pfv}\\
\glt `[They] built the big village Kambaramba and stayed there. While staying there, they fought one another over the crocodile.’ [ulwa002\_00:46]
\z

As with the form \textit{p-} ‘be’, the use of the form \textit{wap} ‘be.\textsc{pst}’ may at times be ambiguous between \isi{locative predication} and \isi{existential predication}, as in \REF{ex:pred:47} and \REF{ex:pred:48}.

\ea%47
    \label{ex:pred:47}
          \textit{Anmoka mï apa \textbf{mawap} i.}\\
\gll    anmoka  mï      apa    \textbf{ma=wap}      i\\
    snake    3\textsc{sg.subj}  house  3\textsc{sg.obj}=be.\textsc{pst}  go.\textsc{pfv}\\
\glt `There was a snake in the house, [but it] left.’ [elicited]
\z

\ea%48
    \label{ex:pred:48}
          \textit{Inim mï awal ini mawat} \textbf{\textit{wap}}.\\
\gll inim  mï      awal    ini    ma=wat    \textbf{wap}\\
    water  3\textsc{sg.subj}  yesterday  ground  3\textsc{sg.obj}=atop  be.\textsc{pst}\\
\glt `There was water on the ground yesterday.’ [elicited]
\z

It is perhaps possible for the \isi{past}-\isi{tense} \isi{locative verb} form \textit{wap} ‘be.\textsc{pst}’ to be used as a \isi{copula}r verb (i.e., for non-\isi{locative predication}, such as \is{classificatory predication} classificatory, \is{equative predication} equative, or \isi{attributive predication}), but the only examples of this that I have seen come from targeted elicitation and may not be representative of traditional or naturalistic speech. Sentences \REF{ex:pred:49}, \REF{ex:pred:50}, and \REF{ex:pred:51} provide examples of \isi{copula}r uses of \textit{wap} ‘be.\textsc{pst’}.

\is{predicate|)}
\is{predication|)}
\is{non-verbal predication|)}
\is{locative verb|)}
\is{past|)}
\is{locative|)}


\ea%49
    \label{ex:pred:49}
          \textit{Kowe mï \textbf{atuma wap}}.\\
\gll Kowe  mï      atuma      \textbf{wap}\\
    [name]  \textsc{3sg.subj}  older.brother  be.\textsc{pst}\\
\glt `Kowe was an older brother.’ [elicited]
\z

\ea%50
    \label{ex:pred:50}
          \textit{Itom mï \textbf{ambi wap}}.\\
\gll itom  mï      ambi  \textbf{wap}\\
    father  3\textsc{sg.subj}  big    be.\textsc{pst}\\
\glt `Father was big.’ [elicited]
\z

\ea%51
    \label{ex:pred:51}
          \textit{Banjiwa mï \textbf{ankam anma wap} amun tembip.}\\
\gll    Banjiwa  mï      ankam  anma  \textbf{wap}  amun  tembi=p\\
    [name]    \textsc{3sg.subj}  person  good  be.\textsc{pst}  now  bad=\textsc{cop}\\
\glt `Banjiwa was a good person, but now he is bad.’ [elicited]
\z

\section{Complex predicates}\label{sec:10.5}

\is{complex predicate|(}
\is{predicate|(}

There are two relatively common \isi{auxiliary-verb} constructions in Ulwa, both of which appear to be relatively recently innovated. One of these constructions, which uses the verb ‘go’ as an \isi{auxiliary} to signal \isi{future} \isi{time}, parallels similar developments of \isi{motion} verbs in other languages, such as several European languages. The other construction, which involves the \isi{locative}/\isi{copula}r form \textit{wap} ‘be.\textsc{pst}’ is suspected to be a more recent \isi{contact}-induced innovation.

  The \isi{imperfective} or \isi{irrealis} form of \textit{ma-} ‘go’ may be used as an \isi{auxiliary verb} along with an \isi{irrealis} form of a main verb to signal the \isi{future}, nearly paralleling the \ili{English} \isi{periphrastic} \isi{future} construction with \textit{going to} and an \isi{infinitive}. These Ulwa \isi{future} constructions are illustrated in examples \REF{ex:pred:52} through \REF{ex:pred:56}.

\ea%52
    \label{ex:pred:52}
          \textit{Nï ma na \textbf{tana man}}.\\
\gll nï    ma      na    \textbf{ta-na}    \textbf{ma-n}\\
    1\textsc{sg}  3\textsc{sg.obj}  talk  say-\textsc{irr}  go-\textsc{ipfv}\\
\glt `I am going to tell its story.’ [ulwa015\_00:10]
\z

\ea%53
    \label{ex:pred:53}
          \textit{Un \textbf{maytana man}}.\\
\gll un  ma=\textbf{ita-na}        \textbf{ma-n}\\
    2\textsc{pl}  3\textsc{sg.obj}=build-\textsc{irr}  go-\textsc{ipfv}\\
\glt `You are going to build it.’ [ulwa014\_71:48]
\z

\ea%54
    \label{ex:pred:54}
          \textit{Wombïn ambi nga \textbf{ina mane}}.\\
\gll wombïn  ambi  nga      \textbf{i-na}    \textbf{ma-n-e}\\
    work    big    \textsc{sg.prox}  come-\textsc{irr}  go-\textsc{ipfv-dep}\\
\glt `This big work is going to come.’ [ulwa014\_10:52]
\z

\ea%55
    \label{ex:pred:55}
          \textit{Ndï men \textbf{pïna mane}}.\\
\gll ndï  ma=in      \textbf{p-na}  \textbf{ma-n-e}\\
    3\textsc{pl}  3\textsc{sg.obj}=in  be-\textsc{irr}  go-\textsc{ipfv-dep}\\
\glt `They are going to stay inside it.’ [ulwa032\_55:38]
\z

\ea%56
    \label{ex:pred:56}
          \textit{Apwanam ita wa \textbf{mapïna mane}}.\\
\gll apwanam    i-ta        wa  ma=\textbf{p-na}      \textbf{ma-n-e}\\
    side.of.house  go.\textsc{pfv}{}-\textsc{cond}  just  3\textsc{sg.obj}=be-\textsc{irr}  go-\textsc{ipfv-dep}\\
\glt `If [you] go to the side of the house, [you] are just going to stay there.’ [ulwa014\_36:15]
\z

Examples such as \REF{ex:pred:55} and \REF{ex:pred:56}, in which the main verb is the static \isi{locative verb} \textit{p-} ‘be’, illustrate the purely \isi{temporal} or \isi{modal} use of this verb \textit{ma-} ‘go’ -- that is, a use without any sense of \isi{motion}. While the \isi{imperfective} form of \textit{ma-} ‘go’ is most often used in these constructions, it is alternatively possible to use an \isi{irrealis} (or even \isi{conditional}) form of the verb, as shown by \REF{ex:pred:57} and \REF{ex:pred:58}.

\ea%57
    \label{ex:pred:57}
          \textit{U angos \textbf{tïna mana}?}\\
\gll u    angos  \textbf{tï-na}    \textbf{ma-na}\\
    2\textsc{sg}  what  take-\textsc{irr}  go-\textsc{irr}\\
\glt `What could you be going to get?’ [ulwa032\_00:32]
\z

\ea%58
    \label{ex:pred:58}
          \textit{Una tïmbïl men \textbf{pïta manata} …}\\
\gll    unan    tïmbïl  ma=in      \textbf{p-ta}    \textbf{ma-na-ta}\\
    1\textsc{pl.incl}  fence  3\textsc{sg.obj}=in  be\textsc{{}-cond} go-\textsc{irr-cond}\\
\glt `If we are going to be within the fence …’ [ulwa037\_22:31]
\z

While it is theoretically possible that \isi{perfective} forms of the verb may be used in such constructions as well (i.e., the \isi{suppletive} form \textit{i} ‘go.\textsc{pfv’}), I do not have clear examples of this. Some apparent examples of \isi{periphrastic} \isi{future} constructions with \textit{i} ‘go.\textsc{pfv}’ are in fact constructions with \isi{purpose clause}s, which also contain \isi{irrealis} verb forms. For example, sentences \REF{ex:pred:59} and \REF{ex:pred:60} are not analyzed as \isi{periphrastic} \isi{future} constructions, but rather as \isi{purpose} constructions.

\ea%59
    \label{ex:pred:59}
          \textit{Ala \textbf{unanwalinda i}}.\\
\gll ala      \textbf{unan=wali-nda}  \textbf{i}\\
    \textsc{pl.dist}  \textsc{1pl.incl=}hit-\textsc{irr}  go.\textsc{pfv}\\
\glt `Those people have come to fight us.’ [ulwa014\_03:08]
\z

\ea%60
    \label{ex:pred:60}
          \textit{Min tïlwanï wandam \textbf{kolnda iye}}.\\
\gll min  tïlwa=nï  wandam  \textbf{kol-nda}  \textbf{i}{}-e\\
    3\textsc{du}  road=\textsc{obl}  jungle    split-\textsc{irr}  go.\textsc{pfv-dep}\\
\glt `The two went to split the jungle from the path.’ (i.e., to clear a trail) [ulwa014\_33:09]
\z

Indeed, it is often difficult to determine whether a ‘going’ verb is marking futurity or \isi{purpose}. This is perhaps unsurprising, if one assumes the likely historical change of: verb of \isi{motion} > \isi{purpose} > \isi{future}. In examples \REF{ex:pred:61} and \REF{ex:pred:62}, the \ili{English} translations capture the ambiguity well. In \REF{ex:pred:61}, a reading in which the ‘going’ verb marks futurity would suggest that the first clause is \isi{counterfactual}.

\ea%61
    \label{ex:pred:61}
          \textit{Ndï apïn anul \textbf{landa mane} mï ipka i.}\\
\gll    ndï  apïn=n    anul    \textbf{la-nda}  \textbf{ma-n}{}-e    mï      ipka i\\
    3\textsc{pl}  fire=\textsc{obl}  grassland  eat-\textsc{irr}  go-\textsc{ipfv-dep}  3\textsc{sg.subj}  before  go.\textsc{pfv}\\
\glt `They were going to burn the grassland, but he went ahead [of them].’ [ulwa001\_13:01]
\z


\ea%62
    \label{ex:pred:62}
          \textit{Magendo lol \textbf{amblawalinda mane}}.\\
\gll Magendo  ala=ul      \textbf{ambla=wali-nda}  \textbf{ma-n}{}-e\\
    [place]    \textsc{pl.dist}=with  \textsc{pl.refl}=hit-\textsc{irr}  go-\textsc{ipfv-dep}\\
\glt `[They] were going to fight with the people from Magendo [village].’ [ulwa002\_05:11]
\z

Whereas only \textit{ma-} ‘go’ seems to be permitted in \isi{periphrastic} \isi{future} constructions, it is possible for other \isi{motion} verbs to indicate \isi{purpose} as well, such as the verb \textit{lo-} ‘cut, go’ and \textit{in-} ‘come’, as seen in \REF{ex:pred:63} and \REF{ex:pred:64}.\footnote{It should be noted, however, that the exact \isi{syntactic} nature of these \isi{purpose} constructions is not entirely clear. It seems that only \isi{motion} verbs permit this \isi{embedded clause} structure to express \isi{purpose}. Elsewhere, the clause indicating \isi{purpose} simply follows the clause detailing the action performed for that \isi{purpose} (see \sectref{sec:4.6}).}

\ea%63
    \label{ex:pred:63}
          \textit{Wongïta man \textbf{matïna lope}}.\\
\gll wongïta  ma=n      \textbf{ma=atï-na}      \textbf{lo-p}{}-e\\
    bow    3\textsc{sg.obj=obl}  3\textsc{sg.obj}=take-\textsc{irr}  go-\textsc{pfv-dep}\\
\glt `[I] went to hit it with [my] bow.’ [ulwa037\_03:16]
\z

\ea%64
    \label{ex:pred:64}
          \textit{Ndït} \textbf{\textit{ndïnanda}} \textit{ndïnap} \textbf{\textit{ine}}.\\
\gll ndï=tï    \textbf{ndï=na-nda}  ndï=nap  \textbf{i-n}{}-e\\
    3\textsc{pl}=take  3\textsc{pl}=give-\textsc{irr}  3\textsc{pl=}for  come-\textsc{pfv-dep}\\
\glt `[They] came for their sake to give them [= fish] to them [= their people].’ [ulwa032\_48:36]
\z

  The other major \isi{auxiliary-verb} construction in Ulwa employs the \isi{locative verb} \textit{wap} ‘be.\textsc{pst}’ (\sectref{sec:10.4}) as the \isi{auxiliary verb} (perhaps better considered an auxiliary \isi{particle}, since it does not \isi{inflect}). This construction is discussed in \sectref{sec:15.3}.

\is{predicate|)}
\is{complex predicate|)}
