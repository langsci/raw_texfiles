\chapter{Texts}\label{sec:16}

This chapter contains three Ulwa texts: \textit{Way Inom} (‘The Mother of the Turtle’, \sectref{sec:16.1}), \textit{Amblom Yena} (‘The Woman Amblom’, \sectref{sec:16.2}), and \textit{Anmoka} (‘Snakes’, \sectref{sec:16.3}). The versions of the texts included here are all based on recordings that I have collected in Manu village. These recordings can all be found online in the collections of the Endangered Languages Archive (ELAR):\\

\url{http://hdl.handle.net/2196/00-0000-0000-000F-CB61-A}\\

The transcriptions in this chapter have been made in the practical phonemic \isi{orthography} that is used throughout the book. Minor \isi{speech error}s and non-linguistic vocalizations such as coughs have not been included in these clean versions. The translations are meant to be fairly literal, while still capturing the spirit of the stories being told. Where it is thought helpful, footnotes are included to explicate relevant cultural information, clarify aspects of the narrative, or indicate words \isi{borrow}ed from \ili{Tok Pisin}.

\section{\label{sec:16.1}  {\textit{Way Inom}} {(‘The} {Mother} {of} {the} {Turtle’)}}

This is a traditional story told by Ayndin Bram on 16  {November 2016}, at his home in Manu village. Examples from this text that appear elsewhere in this book are labeled “ulwa006\_mm:ss”. The audio recording can be found on the ELAR website (file name: ulwa006.wav). It is about eight-and-a-half minutes long (08:40).

  The story is an etiology of sea turtles. The Ulwa people live at a considerable remove from the ocean and may not have had much direct familiarity with the ocean traditionally. That said, there must have been a long history of trade routes leading to the sea and its contents. For example, the lime (calcium hydroxide) used in chewing betel nut is produced from seashells.

  The story runs roughly as follows: A woman lives alone with her son. Every morning she goes out on the river with him in her canoe to check her fish traps. One day she finds a small turtle caught in a trap. The boy becomes fond of the turtle and keeps it as a pet. He feeds fish to the turtle, and it grows bigger and bigger. One day, however, an eagle swoops down and snatches both the boy and his turtle. It carries them far away, ultimately dropping them on the top of a sago palm. With no way down, the two live together in a crevice at the top of the palm. The turtle continues to grow and grow. Once it has become rather large, it begins testing its strength, climbing up and down the stalk of the palm with pieces of wood on its shell. When it gets strong enough, it climbs down the palm, uproots a house from a village, and carries it off. It goes back up the palm to fetch its owner, carries him down, and puts him in the house. The turtle then goes off to find a wife for its owner, who is by now a grown man. It picks up a young woman fast asleep and carries her back to the owner in his new house. The man and woman live together with the turtle and have children of their own. The children grow up, but the man never tells them about the special nature of this turtle. One day, one of his sons shoots an arrow at the turtle, hitting it in the eye. The turtle decides to leave the family forever, running off to the sea, where it can still be seen to this day as the giant sea turtle.

\ea
{\itshape Way inom.}\\
\gll way  inom\\
turtle  mother\\
\glt ‘The mother of the turtle.’\footnote{This is something like a title to the story. This is a story about the ‘mother’ of the turtle since it explains the origin of the (sea) turtle.}

\ex {\itshape Way ango ambi me.}\\
\gll way  ango  ambi  me\\
turtle  \textsc{neg}  big    \textsc{neg}\\
\glt ‘The turtle wasn’t big.’

\ex {\textit{Njukuta ndoy.}}\\
\gll njukuta  anda{{}=o}\\
small    \textsc{sg.dist=voc}\\
\glt ‘It was small!’

\ex {\itshape
Inom mï --}\\
\gll inom  mï\\
mother  3\textsc{sg.subj}\\
\glt ‘A woman --’

\newpage

\ex {\itshape
Inom mï wa unde iwa lan inim andawe.}\\
\gll inom  mï      wa  unda-e   iwa      ala{=}n{} inim anda=aw{{}-}e\\
mother  3\textsc{sg.subj}  just  go-\textsc{dep}    basket  \textsc{pl.dist=obl}  water \textsc{sg.dist}=put.\textsc{ipfv-dep}\\
\glt ‘A woman used to just go around, setting fish traps\footnote{The \textit{iwa} basket (here translated as ‘fish trap’) is a traditional basket woven from sago fronds; it is shaped like a vase (or funnel) and is used to catch fish in the river overnight as they swim into the wide mouth and get trapped at the other end of the basket.} in the water.’

\ex \textit{Iwa lan inim andawe umbenam unde ndi we nd}{\textit{i}}\textit{n} {\textit{u}} \textit{kundan nïmban ndïwale ndïkuk nji awe.}\\
\gll iwa      ala{{}-}n      inim  anda=aw{{}-e} umbenam    unda-e     ndï={i} we    ndï={i}n  {u} kundan    nïmban ndï{=}wali-e  ndï{=}kuk      nj{i} aw-e\\
{basket} \textsc{pl.dist=obl}  water \textsc{sg.dist}{=put.}\textsc{ipfv-dep}  morning 3\textsc{pl}=go.\textsc{pfv}  go-\textsc{dep}  then  3\textsc{pl=}in  from  eel      fish.species 3\textsc{pl}=hit-\textsc{dep}  \textsc{3pl}=gather  thing  put.\textsc{ipfv-dep}\\
\glt ‘[She] would put fish traps in the water, go to them in the morning, and then, {from} within them, {kill} eels and fish\footnote{The speaker refers to \textit{nïmban}, an unidentified fish species. This form might actually reflect the generic \ili{Proto-West Keram} word for fish (cf. \ili{Pondi} \textit{kimbe} ‘fish’, with which it is \isi{cognate}).} and gather them into something [i.e., a basket].’

\ex {\itshape
Ndïkuk nji awe mï wolka i ndïn up.}\\
\gll ndï{=}kuk    nj{i} aw-e      mï      wolka  i    ndï{=}n u-p\\
3\textsc{pl}=gather  thing  put.\textsc{ipfv-dep}  \textsc{3sg.subj}  again  go.\textsc{pfv}  \textsc{3pl=obl} put-\textsc{pfv}\\
\glt ‘After gathering them into something, she again went and set them.’

\ex {\textit{Iye wolka i ndïkukaw.}}\\
\gll i-e        wolka  i    ndï{=}kuk{{}-aw}\\
go.\textsc{pfv-dep}  again  go.\textsc{pfv}  \textsc{3pl=}gather-put.\textsc{ipfv}\\
\glt ‘Having gone, [she] again went and gathered them [= the fish].’

\ex \textit{Ndi we i}{\textit{y}}\textit{e way nungol kotïn.}\\
\gll ndï={i} we    i-e        way  nungol  ko{=}tï{{}-}n\\
3\textsc{pl}={go.}\textsc{pfv}  {then} go.\textsc{pfv-dep}  turtle  child  \textsc{indf=}take-\textsc{pfv}\\
\glt ‘[She] went to them, and having gone, [she] caught a little tur{t}le.’\footnote{Literally a ‘child turtle’.}

\newpage

\ex {\itshape
Way nungol mï iwa mene.}\\
\gll way  nungol  mï       iwa      m{a=i}n{{}-}e\\
turtle  child  3\textsc{sg.subj}  basket  3\textsc{sg.obj}=in-\textsc{ipfv}\\
\glt ‘The little turtle was in the trap.’

\ex \textit{Manji nungol mat ambïn} {\textit{nu}}\textit{m} {\textit{malï}}\textit{p.}\\
\gll ma-nji      nungol  ma{=}tï      ambï=n num  ma=lï-p\\
3\textsc{sg.obj-poss}  child  3\textsc{sg.obj}=take  \textsc{sg.refl=obl}  canoe  3\textsc{sg.obj}=put-\textsc{pfv}\\
\glt ‘[She] got her child and put [him] in her canoe.’\footnote{Here the story backs up to what the woman had done before catching the turtle in her trap.}

\ex \textit{Alum ulwap} {\textit{numan}} \textit{ulwape}{\textit{no.}}\\
\gll alum  ulwa=p    numan ulwa{=}p{{}-}e{n=o}\\
child  nothing=\textsc{cop}  husband  nothing=\textsc{cop}{}-\textsc{nmlz=voc}\\
\glt ‘[She] didn’t have a child\footnote{The speaker makes a mistake (‘child’) but corrects it (‘husband’).} – didn’t have a husband.’

\ex \textit{Mawe aw}{\textit{a}} \textit{iyen.}\\
\gll ma-we        aw{a} i-e{n}\\
3\textsc{sg.obj-part.int}  \textsc{int}    go.\textsc{pfv-nmlz}\\
\glt ‘She herself used to go out alone.’

\ex \textit{Yanat matï} \textit{nungol matï ambïn} {\textit{n}}\textit{um} {\textit{ma}}\textit{l}{\textit{ï}}\textit{p.}\\
\gll yanat    ma=tï      nungol  ma{=}tï      ambï=n {n}um {ma=}l{ï-}p\\
daughter  3\textsc{sg.obj}=take  child  3\textsc{sg.obj}=take  \textsc{sg.refl=obl}  canoe 3\textsc{sg.obj}=put-\textsc{pfv}\\
\glt ‘[She] put her daughter\footnote{The speaker makes another mistake (‘daughter’) but again corrects it (‘child/son’).} -- her son into her canoe.’

\ex \textit{Wolka mol i iwa nd}{\textit{i}} \textit{we iye way mï matïne matï nungol manane.}\\
\gll wolka  m{a=u}l      i    iwa      ndï={i} we    i-e  way  mï      ma{=}tï{{}-}n{{}-}e          ma{=}tï      nungol ma{=}na-n{{}-}e\\
again  3\textsc{sg.obj}=with  go.\textsc{pfv}  basket  3\textsc{pl}=go.\textsc{pfv}  then  go.\textsc{pfv-dep} turtle  3\textsc{sg.subj}  3\textsc{sg.obj}=take-\textsc{pfv-dep}  3\textsc{sg.obj}=take  child 3\textsc{sg.obj}=give-\textsc{pfv-dep}\\
\glt ‘[She] in turn went with him, went to the fish traps and then, having gone -- the turtle -- when [she] got it, [she] gave it to her son.’

\newpage

\ex {\itshape
Nungol mï ikali mas mat uta ndenlïp.}\\
\gll nungol  mï      i-kali    ma=si      ma=tï      uta anda=in-lï-p\\
child  3\textsc{sg.subj}  hand-send  3\textsc{sg.obj}=push  3\textsc{sg.obj}=take  shell \textsc{sg.dist}=in-put-\textsc{pfv}\\
\glt ‘The son grabbed it and put it in a [coconut] shell.’

\ex {\itshape
Mat uta ndalp mala unde.}\\
\gll ma{=}tï      uta    anda{=}lï{{}-}p      ma={a}la    unda-e\\
3\textsc{sg.obj}=take  shell  \textsc{sg.dist}=put-\textsc{pfv}  3\textsc{sg.obj}=for  go-\textsc{ipfv}\\
\glt ‘[He] put it in the shell and started going around for the sake of it.’\footnote{That is, the boy started going around the river to look for food for his pet turtle.}

\ex {\itshape
Mala wambana mïnwata ndïmoke manane.}\\
\gll ma=ala    wambana  mïnwata  ndï{=}moko-e  ma{=}na-n{{}-}e\\
3\textsc{sg.obj=}for  fish    rotten    3\textsc{pl}=take-\textsc{dep}  \textsc{3sg.obj}=give-\textsc{pfv-dep}\\
\glt ‘For the sake of it, [he] gave rotten fish to it.’

\ex \textit{Mï ndame n}{\textit{ay}} \textit{n}{\textit{ay}}.\\
\gll mï      ndï={a}ma-e    n{a-i} n{a-i}\\
3\textsc{sg.subj}  \textsc{3pl}=eat-\textsc{dep}  \textsc{detr}{}-go.\textsc{pfv}  \textsc{detr-}go.\textsc{pfv}\\
\glt ‘It ate them for quite some time.’

\ex {\itshape
Way mï nay ambi nap.}\\
\gll way  mï       na-i ambi na-p\\
turtle  3\textsc{sg.subj}  \textsc{detr-}go.\textsc{pfv}  big    \textsc{detr}{}-be\\
\glt ‘And the turtle went and got big.’

\ex {\textit{Inom mï mol nay.}}\\
\gll inom  mï       ma=ul      na-i\\
mother  3\textsc{sg.subj}  \textsc{3sg.obj=}with  \textsc{detr}{}-go.\textsc{pfv}\\
\glt ‘The mother went with him [= her son].’

\ex \textit{Inom mol iyen} {\textit{mambi}} \textit{nungol m}{\textit{ï}} \textit{ambi nap.}\\
\gll inom  m{a=u}l      i-e{n} {ma-ambi} nungol  m{ï} ambi na{{}-}p\\
mother  3\textsc{sg.obj}=with  go.\textsc{pfv-nmlz}  3\textsc{sg.obj-top}  child  \textsc{3sg.subj}  big \textsc{detr}{}-be\\
\glt ‘And as for the mother who went around with him, [her] son got big.’

\newpage

\ex {\itshape
Ambi nape way mï luke ambi nap.}\\
\gll ambi  na{{}-}p{{}-}e      way  mï      luke  ambi  na-p\\
big    \textsc{detr}{}-be-\textsc{dep}  turtle  3\textsc{sg.subj}  too    big    \textsc{detr}{}-be\\
\glt ‘[He] got big and the turtle {got big, too.}’


\ex {\itshape
Ambi nape nogat!}\\
\gll ambi na-p-e      nogat\\
big    \textsc{detr}{}-be-\textsc{dep}  no\\
\glt ‘[It] got big, but no!’\footnote{The \ili{Tok Pisin} \isi{interjection} \textit{nogat} ‘no’ is signaling that something bad is about to happen.}

\ex {\itshape
Wolka wop mol iye nogat!}\\
\gll wolka  wo-p    m{a=u}l      i-e        nogat\\
again  sleep-\textsc{pfv}  3\textsc{sg.obj}=with  go.\textsc{pfv-dep}  no\\
\glt ‘Again, the next day,\footnote{Literally ‘slept’.} [the mother] went with him [= the son], but no!’\footnote{The \isi{interjection} \textit{nogat} ‘no’ is from \ili{Tok Pisin}.}

\ex {\textit{Mï ikali way nungol man uta mol si.}}\\
\gll mï      i-kali    way  nungol  ma=n uta    ma=ul si\\
3\textsc{sg.subj}  hand-send  turtle  child  3\textsc{sg.obj=obl}  shell  3\textsc{sg.obj}=with push\\
\glt ‘He held the little turtle with the {[coconut]} shell.’

\ex \textit{Amangala nda kw}{\textit{a i}} \textit{w}{\textit{a}}\textit{pa li ka i.}\\
\gll amangala  anda    kwa  i wapa  li    ka  i\\
eagle    \textsc{sg.dist}  just    go.\textsc{pfv}  wing  down  let  go.\textsc{pfv}\\
\glt ‘But an eagle\footnote{A large, brown predatory bird, similar to an eagle.} just came, came with [its] wings pointing down.’

\ex {\itshape
Kwa mangusuwa --}\\
\gll kwa  ma-ngusuwa\\
 just    3\textsc{sg.obj-}poor\\
\glt ‘Just, the poor thing --’

\ex \textit{N}{\textit{u}}\textit{m mo nungol man} {\textit{kwa}} \textit{way mol tïn.}\\
\gll n{u}m  m{a=u} nungol  ma=n      {kwa} way  m{a=u}l tï{{}-}n\\
canoe  3\textsc{sg.obj}=from  child  3\textsc{sg.obj=obl}  just    turtle  3\textsc{sg.obj}=with take-\textsc{pfv}\\
\glt ‘{[The eagle] got the boy with the turtle from the canoe.}’

\newpage

\ex {\textit{Mat i matï nowe ndo malïp.}}\\
\gll ma{=}tï      i    ma{=}tï      nowe    anda=u ma=lï-p\\
3\textsc{sg.obj}=take  go.\textsc{pfv}  \textsc{3sg.obj}=take  sago.species  \textsc{sg.dist}=from 3\textsc{sg.obj}=put-\textsc{pfv}\\
\glt ‘And [it] brought him and put him on a sago palm.’\footnote{The narrator specifies that the palm is a \textit{nowe} palm, a large sago palm species that has no spines on its stem. In some versions of the story, the eagle wishes to remove the boy and the turtle from the river because the boy has been feeding the turtle all the fish that the eagle would otherwise hunt.}

\ex \textit{Matï nowe nd}{\textit{o}} \textit{m}{\textit{a}}\textit{lïpe mï maw}{\textit{a}}\textit{t pe.}\\
\gll ma{=}tï      nowe    and{a=u} m{a=}lï-p{{}-}e        mï m{a=}w{at} p{{}-}e\\
3\textsc{sg.obj}=take  sago.species  \textsc{sg.dist}=from  3\textsc{sg.obj}=put-\textsc{pfv-dep}  \textsc{3sg.subj} \textsc{3sg.obj=}atop  be-\textsc{ipfv}\\
\glt ‘Having put him on the sago palm, he [= the boy] stayed on top of it.’

\ex \textit{Mï maw}{\textit{a}}{t pe way mat ambul inde.}\\
\gll mï      m{a=}w{a}t p{{}-}e    way  ma=tï      ambï{=u}l {i}nda-e\\
3\textsc{sg.subj}  3\textsc{sg.obj}=atop  be-\textsc{dep}  turtle  3\textsc{sg.obj}=take  \textsc{sg.refl}=with walk-\textsc{ipfv}\\
\glt ‘While he was staying on top of it, [he] carried the turtle around with himself.’

\ex {\itshape
Way mï nay ambi nap.}\\
\gll way  mï      na-i ambi na-p\\
turtle  3\textsc{sg.subj}  \textsc{detr-}go.\textsc{pfv}  big    \textsc{detr}{}-be\\
\glt ‘The turtle went and got big.’

\ex \textit{Way mï nay ambi nape inom} {\textit{mï wa}} \textit{ma na tap ma sa}{\textit{l}} \textit{sap.}\\
\gll way  mï      na-i ambi na-p-e      inom  mï wa ma      na    ta-p    ma sal sa-p\\
turtle  3\textsc{sg.subj}  \textsc{detr}{}-go.\textsc{pfv}  big    \textsc{detr}{}-be-\textsc{dep}  mother  3\textsc{sg.subj} just  3\textsc{sg.obj}  talk  say-\textsc{pfv}  3\textsc{sg.obj}  tear  cry-\textsc{pfv}\\
\glt ‘After the turtle went and got big, the mother just spoke about him and cried about him [= the boy].’\footnote{Literally ‘said his talk’ and ‘cried his tears’, the second of which may be considered a \isi{cognate accusative}.}

\ex \textit{M}{\textit{ï}} \textit{nay.}\\
\gll m{ï} na{{}-i}\\
3\textsc{sg.subj}  \textsc{detr}{}-go.\textsc{pfv}\\
\glt ‘He went.’

\ex {\textit{Ay}} \textit{nungol m}{\textit{ï}} \textit{ulum mo mape.}\\
\gll {ay} nungol  m{ï} ulum  m{a=u} ma{=}p{{}-}e\\
ay  child  3\textsc{sg.subj}  palm  3\textsc{sg.obj}=from  3\textsc{sg.obj}=be-\textsc{ipfv}\\
\glt ‘Ay, the child was living within [{a crevice in]} the sago palm.’

\ex {\itshape
Way mï mo map mol mïnawap.}\\
\gll way  mï       m{a=u} ma{=}p      m{a=u}l mï{=}na{{}-}wap\\
turtle  3\textsc{sg.subj}  \textsc{3sg.obj=}from  3\textsc{sg.obj}=be  3\textsc{sg.obj}=with 3\textsc{sg.subj}=\textsc{detr}{}-be.\textsc{pst}\\
\glt ‘The turtle was [also] within it, living with him.’

\ex \textit{Mawap mawap way mï keka ambi ngata} {\textit{n}}\textit{ap.}\\
\gll ma{=}wap      ma{=}wap      way  mï      keka      ambi ngata  na{{}-}p\\
3\textsc{sg.obj}=be.\textsc{pst}  \textsc{3sg.obj}=be.\textsc{pst}  turtle  3\textsc{sg.subj}  completely  big grand  \textsc{detr}{}-be\\
\glt ‘[They] stayed and stayed there, and the turtle got really huge.’

\ex {\itshape
Way mï keka ambi nape imba pe mï mol awlu unum kwa men u.}\\
\gll way  mï      keka      ambi na{{}-}p{{}-}e      imba  p{{}-}e mï      m{a=u}l      awlu  unum    kwa  m{a=i}n      u\\
turtle  3\textsc{sg.subj}  completely  big    \textsc{detr}{}-be-\textsc{dep}  night  be-\textsc{dep} 3\textsc{sg.subj}  3\textsc{sg.obj}=with  step  crevice    one    3\textsc{sg.obj}=in  from\\
\glt ‘Once the turtle was really big, one night, he [= the boy] stepped out with it from within one crevice [to another].’

\ex \textit{Awlu at}{\textit{o}} \textit{unum kwa men u lowonda mane.}\\
\gll awlu  at{a-u} unum    kwa  m{a=i}n      u    l{o-}w{o-}nda ma{{}-}n{{}-}e\\
step  up-from  crevice    one    3\textsc{sg.obj}=in  from  \textsc{irr}{}-sleep-\textsc{irr} go-\textsc{ipfv-dep}\\
\glt ‘Having stepped up into another crevice, [he] was going to sleep [there].’

\ex {\itshape
Lowonda mane way mï mala ne.}\\
\gll l{o-}w{o-}nda    ma{{}-}n{{}-}e      way  mï      ma=ala       n{a=i}\\
\textsc{irr}{}-sleep-\textsc{irr}  go-\textsc{ipfv-dep}  turtle  3\textsc{sg.subj}  \textsc{3sg.obj}=for  \textsc{detr-}go.\textsc{pfv}\\
\glt ‘As [he] was going to sleep, the turtle went for his sake.’

\ex {\itshape
Li ne.}\\
\gll li    n{a-i}\\
down  \textsc{detr}{}-go.\textsc{pfv}\\
\glt ‘[It] went down.’

\ex \textit{Ulum m}{\textit{a}} \textit{nambï k}{\textit{a}} \textit{l}{\textit{i}} \textit{wandam nay.}\\
\gll ulum  m{a} nambï  k{a} li-i wandam  na{{}-i}\\
palm  3\textsc{sg}.\textsc{obj}  skin  on  down-go.\textsc{pfv}  jungle    \textsc{detr}{}-go.\textsc{pfv}\\
\glt ‘[It] went down along the bark of the sago palm and went toward the jungle.’

\ex {\itshape
Li wandam may molop.}\\
\gll li    wandam  ma{=i} ma=lo-p\\
down  jungle    3\textsc{sg.obj}=go.\textsc{pfv}  3\textsc{sg.obj}=go-\textsc{pfv}\\
\glt ‘[It] went down to the jungle and went around.’

\ex \textit{Molop} {\textit{impul kotïn mas ambï awi lïp.}}\\
\gll ma=lo-p      im-pul      ko=tï-n      ma=si      ambï awi      lï-p\\
3\textsc{sg.obj}=go-\textsc{pfv}  wood-piece  \textsc{indf}=take-\textsc{pfv}    \textsc{3sg.obj=}push  \textsc{sg.refl} shoulder  put-\textsc{pfv}\\
\glt ‘[It] went around, got a piece of wood, and put it on its shoulder.’

\ex \textit{Mas at}{\textit{o a}}\textit{mb}{\textit{ï}} \textit{m}{\textit{u}}\textit{t}{\textit{oma}} \textit{watlïp.}\\
\gll ma{=}si      at{a-u} {a}mb{ï} m{u}t{oma} wat{{}-}l{ï-}p\\
3\textsc{sg.obj}=push  up-from  \textsc{sg.refl}  backbone  atop-put-\textsc{pfv}\\
\glt ‘[It] put it [= the wood] up onto {its back.}’

\ex {\textit{Mat i atay ulum maya ata i.}}\\
\gll ma{=}tï      i    ata-{i} ulum  ma=iya      ata  i\\
3\textsc{sg.obj}=take  go.\textsc{pfv}  up-go.\textsc{pfv}  palm  3\textsc{sg.obj=}toward  up  go.\textsc{pfv}\\
\glt ‘[It] brought it and went up, went up the sago palm.’

\ex \textit{Ata i w}{\textit{a}}\textit{p a wolka mat i li.}\\
\gll ata  i w{ap} wolka  ma{=}tï      i    li-i\\
up  go.\textsc{pfv}  be.\textsc{pst}  again  3\textsc{sg.obj}=take  go.\textsc{pfv}  down-go.\textsc{pfv}\\
\glt ‘Having gone up, [the turtle] again brought it [= the wood] down.’

\newpage

\ex {\itshape
Mat i li nay matï li wandam nay inakawana.}\\
\gll ma{=}tï      i    li    na{{}-i} ma{=}tï      li-i wandam  na{{}-i} ina-ka-wana\\
3\textsc{sg.obj}=take  go.\textsc{pfv}  down  \textsc{detr}{}-go.\textsc{pfv}  \textsc{3sg.obj=}take  down-go.\textsc{pfv} jungle    \textsc{detr-}go.\textsc{pfv}  liver-in-feel\\
\glt ‘[It] brought it, went down, brought it down, went to the jungle, and thought.’

\ex \textit{Mï ambï}{\textit{wana}} \textit{mat: “A!”}\\
\gll mï      ambï{=wana} ma{=}ta      a\\
3\textsc{sg.subj}  \textsc{sg.refl}=feel  3\textsc{sg.obj}=say  ah\\
\glt ‘It thought to itself and said: “Ah!”’

\ex \textit{“Nï ta tata tïn mol} {\textit{l}}\textit{i} {\textit{i}}\textit{na mane.”}\\
\gll nï    ta      tata    tï{{}-}n      m{a=u}l      {l}i    {i-}na ma{{}-}n{{}-}e\\
1\textsc{sg}  already    papa  take-\textsc{pfv}  3\textsc{sg.obj}=with  down  come-\textsc{irr} go-\textsc{ipfv-dep}\\
\glt ‘“{I’m already able to get papa and come down with him.”}’\footnote{The turtle was carrying the wood as a test to see whether it would be able to carry the boy (its ‘papa’) down from atop the palm.}

\ex \textit{“Nï ta}{\textit{ta}} \textit{tïn mol} {\textit{l}}\textit{i} {\textit{i}}\textit{na!”}\\
\gll nï    ta{ta} tï{{}-}n      m{a=u}l      {l}i    {i-}na\\
1\textsc{sg}  papa  take-\textsc{pfv}  3\textsc{sg.obj}=with  down  come-\textsc{irr}\\
\glt ‘“{So I’ll get papa and come down with him!”}’

\ex {\itshape
Makap.}\\
\gll ma{=}kï-p\\
3\textsc{sg.obj}=say-\textsc{pfv}\\
\glt ‘[It] thought this.’

\ex {\itshape
Ango amunpe.}\\
\gll ango  amun{=}p{{}-}e\\
\textsc{neg}  now=\textsc{cop}{}-\textsc{dep}\\
\glt ‘But not immediately.’

\ex {\textit{Ango amunpe atay matïna.}}\\
\gll ango  amun{=}p{{}-}e    ata-i    ma=tï-na\\
\textsc{neg}  now=\textsc{cop}{}-\textsc{dep}  up-go.\textsc{pfv}  \textsc{3sg.obj}=take-\textsc{irr}\\
\glt ‘[It] wouldn’t go up and get him immediately.’

\newpage

\ex \textit{Kop mala} {\textit{i}} \textit{inim i i wa ndï li}{ \textit{ndule i apa kongomlïp.}}\\
\gll kop  ma=ala    {i} inim  i    i    wa    ndï  li-i ndï={u-}l{{}-}e      i    apa    ko=angom{{}-}lï-p\\
just    \textsc{3sg.obj=}for  go.\textsc{pfv}  water  go.\textsc{pfv}  go.\textsc{pfv}  village  3\textsc{pl}  down-go.\textsc{pfv} 3\textsc{pl}=from-go-\textsc{dep}  go.\textsc{pfv}  house  \textsc{indf}=pull.out-put-\textsc{pfv}\\
\glt ‘[It] just went for his sake, went to the water, went, went down to the villages, went around in them, {went,} and pulled out a house.’

\ex {\itshape
Apa kongomlïp wa molop.}\\
\gll apa    ko=angom{{}-}l{ï-}p      wa    ma=lo-p\\
house  \textsc{indf}=pull.out-put-\textsc{pfv}  village  3\textsc{sg.obj}=cut-\textsc{pfv}\\
\glt ‘[It] pulled out a house and cleared a village.’

\ex \textit{I apa kongomlïp mat i matan}{\textit{e}}\textit{lïp.}\\
\gll i    apa    ko=angom{{}-}lï-p      ma{=}tï      i ma{=}tan{e-}l{ï-}p\\
go.\textsc{pfv}  house  \textsc{indf=}pull.out-put-\textsc{pfv}  3\textsc{sg.obj}=take  go.\textsc{pfv} 3\textsc{sg.obj}=stand-put-\textsc{pfv}\\
\glt ‘[It] went and pulled out a house, brought it, and stood it up.’

\ex {\textit{Ke}}\textit{ka wandam ndï way mawa wandam nd}{\textit{ï}}\textit{lop.}\\
\gll {ke}ka      wandam  ndï  way  ma-awa    wandam  nd{ï=}lo-p\\
completely  jungle    3\textsc{pl}  turtle  3\textsc{sg.obj-int}  jungle    3\textsc{pl}=cut-\textsc{pfv}\\
\glt ‘Completely, the gardens -- the turtle itself cut the gardens.’

\ex {\itshape
Mïnal o mil o utam o nongontam --}\\
\gll mïnal  o  mil    o  utam  {o} nongontam\\
taro  or  sugar  or  yam  or  kaukau\\
\glt ‘[Whether it be] taro or sugarcane or yam or \textit{kaukau} [= sweet potato] --’\footnote{The \isi{coordinator} \textit{o} ‘or’ is \isi{borrow}ed from \ili{Tok Pisin} \textit{o} ‘or’.}

\ex {\itshape
Mï keka ndïn up.}\\
\gll mï      keka      ndï{{}-}n    u-p\\
3\textsc{sg.subj}  completely  3\textsc{pl=obl}  put-\textsc{pfv}\\
\glt ‘It planted them all.’

\ex {\textit{Ndï keka ambip.}}\\
\gll ndï  keka      ambi=p\\
3\textsc{pl}  completely  big=\textsc{cop}\\
\glt ‘They [= the crops] were really big.’

\ex {\itshape
Wowal lamndu mï ndïtïn ndït i ndïkuk wandam mop.}\\
\gll wowal    lamndu  mï      ndï{=}tï{{}-}n    ndï{=}t{ï} i ndï=kuk    wandam  m{a=u-}p\\
chicken  pig      3\textsc{sg.subj}  3\textsc{pl}=take-\textsc{pfv}  3\textsc{pl}=take  go.\textsc{pfv} 3\textsc{pl}=gather  jungle    3\textsc{sg.obj}=put-\textsc{pfv}\\
\glt ‘Chickens and pigs -- it got them, brought them, and gathered them into the garden.’

\ex {\itshape
Apa membamup.}\\
\gll apa  m{a=i}mbam{{}-}u-p\\
house  3\textsc{sg.obj}=under-put-\textsc{pfv}\\
\glt ‘[The turtle] put [the livestock] under the house.’

\ex {\itshape
Ande.}\\
\gll ande\\
ok\\
\glt ‘OK.’

\ex {\textit{Mï wolka impul matïn mat ambï mutoma watlïpe.}}\\
\gll mï      wolka  im-pul      ma=tï-n      ma=tï    ambï mutoma  wat-lï-p-e\\
3\textsc{sg.subj}  again  wood-piece  3\textsc{sg.obj}=take-\textsc{pfv}  3\textsc{sg.obj}=take  \textsc{sg.refl} back    atop-put-\textsc{pfv-dep}\\
\glt ‘It again got a piece of wood, got it and put it on its shell.’

\ex \textit{Mol i atay ulum may}{\textit{a}} \textit{atay.}\\
\gll m{a=u}l      i      ata-i    ulum  ma{=iya} ata-i\\
3\textsc{sg.obj}=with  go.\textsc{pfv}    up-go.\textsc{pfv}  palm  3\textsc{sg.obj}=toward  up=go.\textsc{pfv}\\
\glt ‘[It] went with it, went up, went up the sago palm.’

\ex {\textit{Ulum mat atay wape mat nay li nay.}}\\
\gll ulum  ma{=}tï      ata-{i} wap{{}-}e    ma{=}tï{} na{{}-i} li    n{a-i}\\
palm  3\textsc{sg.obj}{=take}  up-go.\textsc{pfv}  be.\textsc{pst-dep}  3\textsc{sg.obj}=take  \textsc{detr}{}-go.\textsc{pfv} down  \textsc{detr-}go.\textsc{pfv}\\
\glt ‘{[On the] palm, got it [= the piece of wood]}, went up, brought it down, and went down.’

\newpage

\ex {\textit{Mat nay li wandam i ambïwana.}}\\
\gll ma{=}tï      na{{}-i} li-i wandam  i {ambï=wana}\\
3\textsc{sg.obj}=take  \textsc{detr-}go.\textsc{pfv}  down-go.\textsc{pfv}  jungle    go.\textsc{pfv} \textsc{sg.refl=}feel\\
\glt ‘[The turtle] bought it [= the wood] down, went to the jungle, and thought to itself.’

\ex {\textit{Mat: “Nï a}}\textit{t}{\textit{a}} \textit{ma matï}{\textit{n}} \textit{m}{\textit{o}}\textit{l} {\textit{l}}\textit{i} {\textit{i}}\textit{na.”}\\
\gll ma=ta      nï    at{a} ma  ma{=}tï{{}-n} m{a=u}l       {l}i {i-}na\\
3\textsc{sg.obj}=say  1\textsc{sg}  up  go  3\textsc{sg.obj}=take-\textsc{pfv}  3\textsc{sg.obj}=with  down come-\textsc{irr}\\
\glt ‘[It] thought: “I shall go up, get him, and come down with him.”’

\ex {\itshape
Al matïn.}\\
\gll al  ma{=}tï{{}-}n\\
net  3\textsc{sg.obj}=take-\textsc{pfv}\\
\glt ‘{[It]} got a mosquito {net.}’

\ex \textit{Al mï apa mï alanda men wape}{\textit{n}} \textit{nda.}\\
\gll al  mï      apa    mï      al  anda    m{a=i}n wap{{}-}e{n} {a}nda\\
net  3\textsc{sg.subj}  house  3\textsc{sg.subj}  net  \textsc{sg.dist}  3\textsc{sg.obj}=in  be.\textsc{pst-nmlz} {\scshape sg.dist}\\
\glt ‘The mosquito {net}, the house -- {there was a mosquito net in it.}’\footnote{The speaker is clarifying what the turtle had done: the house that the turtle had pulled out for the growing young man had a mosquito net inside it.}

\ex {\itshape
Man mol tï i.}\\
\gll ma{=}n      m{a=u}l       tï    i\\
3\textsc{sg.obj=obl}  \textsc{3sg.obj=}with  take  go.\textsc{pfv}\\
\glt ‘{[The turtle]} brought it {[= the mosquito net]} with it {[= the house]}.’

\ex \textit{Man mol tï i matan}{\textit{e}}\textit{lïpape mï imba pe nay mï nawowe.}\\
\gll ma{=}n      m{a=u}l      tï    i    ma{=}tan{e-}l{ï-}p{{}-}ap{{}-}e mï      imba{} p{{}-}e    na{{}-i} mï      n{a-}wow{}-e\\
3\textsc{sg.obj=obl}  \textsc{3sg.obj=}with  take  go.\textsc{pfv}  \textsc{3sg.obj}=stand-put-\textsc{pfv-pfv-dep} 3\textsc{sg.subj}  night  be-\textsc{dep}  \textsc{detr-}go.\textsc{pfv}  3\textsc{sg.subj}  \textsc{detr-}sleep-\textsc{ipfv}\\
\glt ‘{After [t}he turtle] {had} brought it [= the mosquito {net]} with it [= the house] {and} stood it up, it went at night, while he [= the man] was sleeping.’

\newpage

\ex \textit{Un}{\textit{u}}\textit{m pe men pe wowe mï i m}{\textit{o}}\textit{kum ne i membam i.}\\
\gll un{u}m{} p{{}-}e    m{a=i}n{} p{{}-}e    wow{}-e    mï      i    mokum n{a-i} i    m{a=i}mbam    i\\
crevice  be-\textsc{dep}  \textsc{3sg.obj}=in  be-\textsc{dep}  sleep-\textsc{dep}  \textsc{3sg.subj}  go\textsc{.pfv}  stealth \textsc{detr-}go.\textsc{pfv}  go.\textsc{pfv}  \textsc{3sg.obj=}under  go.\textsc{pfv}\\
\glt ‘{While [the man] was sleeping in the crevice, inside it, i}t [= the turtle] went, went stealthily, went, went under him.’

\ex \textit{Mat at}{\textit{o a}}\textit{mb}{\textit{ï}} \textit{m}{\textit{u}}\textit{ta}{\textit{m w}}\textit{atlïp.}\\
\gll ma{=}tï      ata-u    amb{ï} m{u}tam  wat{{}-}l{ï-}p\\
3\textsc{sg.obj}=take  up-from  \textsc{sg.refl}  back  atop-put-\textsc{pfv}\\
\glt ‘[The turtle] got him and put him up onto its back.’

\ex \textit{Ankam ngatan} {\textit{ambï mutam w}}\textit{atlïp.}\\
\gll ankam  ngata=n    ambï    mutam  wat{{}-}l{ï-}p\\
person  grand=\textsc{obl}  \textsc{sg.refl}  back  atop-put-\textsc{pfv}\\
\glt ‘[It] put the huge person on its back.’

\ex \textit{Ankam ngatan} {\textit{a}}\textit{mb}{\textit{ï m}}\textit{utam} {\textit{w}}\textit{at}{\textit{l}}\textit{ïp ande m}{\textit{o}}\textit{kum mat le.}\\
\gll ankam  ngata=n    {a}mbï    mutam  {w}at{{}-lï-}p    ande  m{o}kum ma{=}tï lo-e\\
person  grand=\textsc{obl}  \textsc{sg.refl}  back  atop-put-\textsc{pfv}  ok    {stealth} 3\textsc{sg.obj}=take  go-\textsc{ipfv}\\
\glt ‘Having put the huge man on its back, OK, [it] began bringing him carefully.’

\ex \textit{Naye ulum ma  na}{\textit{mbï ka}} \textit{nay li wandam nay.}\\
\gll na{{}-i-e} ulum  ma{} nambï  ka na{{}-i} li-i      wandam  na{{}-i}\\
\textsc{detr-}go.\textsc{pfv-dep}  palm  3\textsc{sg.obj}  skin  on  \textsc{detr-}go.\textsc{pfv} down-go.\textsc{pfv}  jungle \textsc{detr-}go.\textsc{pfv}\\
\glt ‘{[It] went and went down the bark of the palm and went to the jungle.}’

\ex {\itshape
Keka matï i atay apa may.}\\
\gll keka      ma{=}tï      i    ata-{i} apa    ma{=i}\\
completely  3\textsc{sg.obj}=take  go.\textsc{pfv}  up-go.\textsc{pfv}  house  3\textsc{sg.obj}=go.\textsc{pfv}\\
\glt ‘[The turtle] brought him all the way up to the house.’

\ex \textit{Mat i ata apa may mo}{\textit{l}} \textit{i.}\\
\gll ma{=}tï      i    ata  apa  ma{=i} m{a=ul} i\\
3\textsc{sg.obj}=take  go.\textsc{pfv}  up  house  3\textsc{sg.obj}=go.\textsc{pfv}  3\textsc{sg.obj}=with  go.\textsc{pfv}\\
\glt ‘[It] brought him up to the house and went with him.’

\ex \textit{A}{\textit{l}} {\textit{men i}} \textit{mat}{\textit{ï}} \textit{menlïp.}\\
\gll a{l} ma=in      i ma{=}t{ï} m{a=i}n{{}-}lï{{}-}p\\
net  3\textsc{sg.obj=}in  go.\textsc{pfv}  3\textsc{sg.obj}=take  3\textsc{sg.obj}=in-put-\textsc{pfv}\\
\glt ‘{[It] went into the mosquito net and} put him inside {it}.’

\ex \textit{M}{\textit{ï al}} \textit{men ka wop.}\\
\gll mï      al m{a=i}n ka  wo-p\\
3\textsc{sg.subj}  net  3\textsc{sg.obj}=in  at  sleep-\textsc{pfv}\\
\glt ‘He slept in the mosquito net.’

\ex {\textit{M}}{\textit{ï al}}{ \textit{men ka wop}} {\textit{awlu}}{ }{\textit{anmbu}}{ \textit{inakawanap: “A!”}}\\
\gll mï      al ma{=i}n ka  wo-p    awlu  {an-mbï-u} ina-ka-wana-p    a\\
3\textsc{sg.obj}  net  3\textsc{sg.obj=}in  at  sleep-\textsc{pfv}  step  out-here-from liver-in-feel{{}-}\textsc{pfv}  ah\\
\glt ‘He slept in the mosquito net, came out, and thought: “Ah!”’

\ex \textit{“Nï anjikaka li wap?”}\\
\gll nï    anjikaka  li-i        wap\\
{1}\textsc{sg}  how    down-go.\textsc{pfv}  {be.}\textsc{pst}\\
\glt ‘“{How did I come down like this?”}’

\ex \textit{Way m}{\textit{ï}} \textit{asi man w}{\textit{a}}\textit{t wan make.}\\
\gll way  m{ï} asi  ma=n      w{a}t    wan  ma{=}k{a-e}\\
turtle  3\textsc{sg.subj}  sit  3\textsc{sg.obj=obl}  ladder  above  3\textsc{sg.obj}=let-\textsc{ipfv}\\
\glt ‘The turtle was sitting at the top of his ladder.’\footnote{This is the ladder (or stairs) leading up to the house, which, like all houses in the region, would have been built on stilts.}

\ex \textit{Lop man wat wan maka map.}\\
\gll lop  ma=n{} wat    wan  ma{=}ka    ma{=}p\\
lie  3\textsc{sg.obj=obl}  ladder  above  3\textsc{sg.obj}=let  3\textsc{sg.obj}=be\\
\glt ‘[The turtle] lay at the top of his ladder and stayed there.’

\ex \textit{Mape lïmndï mal}{\textit{ï}}\textit{pe mï anmbi} {\textit{inakawana:}}\\
\gll ma{=}p{{}-}e      lïmndï  ma{=}l{ï-}p{{}-}e        mï      an{{}-}mbï-{i} {ina-ka-wana}\\
3\textsc{sg.obj}=be\textsc{{}-dep} eye    3\textsc{sg.obj}=put-\textsc{pfv-dep}  3\textsc{sg.subj}  out-here-go.\textsc{pfv} liver-in-feel\\
\glt ‘[It] was there watching him, when he [= the man] came out and thought:’

\ex \textit{“Nï anjikaka li?”}\\
\gll nï    anjikaka  li-i\\
1\textsc{sg}  how    down-go.\textsc{pfv}\\
\glt ‘“How did I get down like this?”’

\ex {\itshape
“A!”}\\
\gll a\\
ah\\
\glt ‘“{Ah!”}’

\ex {\itshape
Inakawane: “Nï anjikaka li?”}\\
\gll ina-ka-wana-e    nï    anjikaka  li-i\\
liver-in-feel-\textsc{dep}  1\textsc{sg}  how    down-go.\textsc{pfv}\\
\glt ‘[He] was thinking: “{How did I get down like this?”}’

\ex \textit{“N}{\textit{ï ata}} \textit{ndaw}{\textit{ap}} \textit{nanjikaka liye?”}\\
\gll nï    ata  anda=wap nï    anjikaka  li-i-e\\
1\textsc{sg}  up  \textsc{sg.dist}=be.\textsc{pst}  1\textsc{sg}  how    down-go.\textsc{pfv-dep}\\
\glt ‘“{I was up there, so how did I get down like this?”}’

\ex {\itshape
Lïmndï way mala.}\\
\gll lïmndï  way  ma=ala\\
eye    turtle  3\textsc{sg.obj}=see\\
\glt ‘[He] saw the turtle.’

\ex \textit{“Way ngangusuwa} {\textit{ngapïnate mï ko}} \textit{mï mase --”}\\
\gll way  nga      ngusuwa  nga=p-na-t-e           mï      ko mï      mas{{}-e}\\
turtle  \textsc{sg.prox}  poor    \textsc{sg.prox}=be-\textsc{irr-spec-dep}  3\textsc{sg.subj}  just 3\textsc{sg.subj}  must-\textsc{dep}\\
\glt ‘[And he thought:] “This turtle, the poor thing; it might be this [turtle] -- it just; it must\footnote{The \isi{modal} word \textit{mas} ‘should, must’ is \isi{borrow}ed from \ili{Tok Pisin} \textit{mas} ‘should, must’.} have -- ”’

\ex \textit{“Ata i ko} {\textit{nïtïn}} \textit{n}{\textit{ït}} \textit{li w}{\textit{a}}\textit{p.”}\\
\gll ata  i    ko  {nï=tï-n} n{ï=tï} li-i        w{a}p\\
up  go.\textsc{pfv}  just  1\textsc{sg}=take-\textsc{pfv}  1\textsc{sg}=take  down-go-\textsc{pfv}  be.\textsc{pst}\\
\glt ‘“{[It] went up and just got me and brought me down.”}’

\ex {\itshape
Makap inakawanap.}\\
\gll ma=kï-p      ina-ka-wana-p\\
3\textsc{sg.obj}=say-\textsc{pfv}  liver-in-feel-\textsc{pfv}\\
\glt ‘[He] said it and thought.’

\ex {\itshape
Mawap imba pe mï wolka nawow.}\\
\gll ma{=}wap      imba p{{}-}e    mï      wolka  na{{}-}w{ow}\\
3\textsc{sg.obj}=be.\textsc{pst}  night  be\textsc{{}-dep}  \textsc{3sg.subj} again  \textsc{detr-}sleep\textsc{.ipfv}\\
\glt ‘[He] stayed the night there and again he {fell a}sle{e}p.’

\ex \textit{Wolka nawowe mï mal}{\textit{a}} \textit{yana} {\textit{a}}\textit{ng}{\textit{l}}\textit{a n}{\textit{o}}\textit{l.}\\
\gll wolka  na{{}-}wow{}-e        mï      ma=al{a} yana     {a}ng{l}a n{a-}lo\\
again  \textsc{detr-}sleep\textsc{.ipfv{}-dep}  \textsc{3sg.subj}  \textsc{3sg.obj=}for  woman    await \textsc{detr}{}-go\\
\glt ‘After again sleeping, it [= the turtle] went searching for a wife for him.’

\ex {\textit{Way nga wa mala yana anglale.}}\\
\gll way  nga      wa   ma=al{a} yana    {a}ng{la-lo-e}\\
turtle  \textsc{sg.prox}  just  3\textsc{sg.obj=}for  woman    await-go-\textsc{ipfv}\\
\glt ‘This turtle was just searching for a wife for him.’

\ex \negmedspace \textit{I wa kwa may inim i li wa kwa may atay.}\\
\gll i    wa    kwa  ma{=i} inim  i    li-i wa    kwa  ma{=i} ata-{i}\\
go.\textsc{pfv}  village  one    3\textsc{sg.obj}=go.\textsc{pfv}  water  go.\textsc{pfv}  down-go.\textsc{pfv} village  one    3\textsc{sgobj}=go.\textsc{pfv}  up-go.\textsc{pfv}\\
\glt ‘{[It] went, went to one village, went downstream, went to another village, and then went up [into yet another village].}’\footnote{The turtle was going to village after village along the river to find a wife for its master.}

\ex \negmedspace \textit{Atay wa mo} {\textit{le}} \textit{yana amunji nungol anma ndawa.}\\
\gll ata-{i} wa    m{a=u} {lo-e} yana    amu{n}ji  nungol  anma ndï-awa\\
up-go.\textsc{pfv}  village  3\textsc{sg.obj=}from  go-\textsc{dep}  woman    young  child  good 3\textsc{pl-int}\\
\glt ‘[It] went up, going around the village, [looking for] {suitable} young women.’

\ex \negmedspace \textit{Mï i apa nd}{\textit{in u le i}} \textit{lïmndï mala.}\\
\gll mï      i    apa  ndï=in  u      lo-e {i} lïmndï  ma=ala\\
3\textsc{sg.subj}  go.\textsc{pfv}  house  3\textsc{pl}=in  from  go-\textsc{dep}  go.\textsc{pfv}  eye    3\textsc{sg.obj}=see\\
\glt ‘It went, went around inside the house{s}, {went,} and saw her.’\footnote{That is, the turtle finally saw the woman who it thought would make a good wife for its master.}

\ex \negmedspace {\textit{Lïmndï mala m}}{\textit{o}}{\textit{kum al men u matïn.}}\\
\gll lïmndï  ma=ala    m{o}kum  al  m{a=i}n      u {ma=tï-n}\\
eye    3\textsc{sg.obj}=see  stealth    net  3\textsc{sg.obj}=in  from 3\textsc{sg.obj}=take-\textsc{pfv}\\
\glt ‘[It] saw her and stealthily got her from out of [her] mosquito net.’

\ex \negmedspace {\itshape
Man al mol tïn.}\\
\gll ma{=}n al  m{a=u}l      tï{{}-}n\\
3\textsc{sg.obj=obl}  net  3\textsc{sg.obj}=with  take-\textsc{pfv}\\
\glt ‘[It] got her with the mosquito net.’

\ex \negmedspace \textit{Mat amb}{\textit{ï mutam w}}\textit{atlïp mat i.}\\
\gll ma{=}tï      ambï  mutam  wat{{}-}l{ï-}p    ma{=}tï      i\\
3\textsc{sg.obj}=take  \textsc{sg.ref}  back  atop-put-\textsc{pfv}  3\textsc{sg.obj}=take  go.\textsc{pfv}\\
\glt ‘[It] got her onto its back and brought her.’

\ex \negmedspace {\itshape
Mat i itom maya apa i.}\\
\gll ma{=}tï      i    itom  ma{=i}ya      apa    i\\
3\textsc{sg.obj}=take  go.\textsc{pfv}  father  3\textsc{sg.obj=}toward  house  go.\textsc{pfv}\\
\glt ‘[It] brought her and went home to the man.’

\ex \negmedspace {\itshape
Itom maya apa i mat makanalïp.}\\
\gll itom  ma{=i}ya      apa    i    ma{=}tï m{a=}kana{{}-}l{ï-}p\\
father  3\textsc{sg.obj=}toward  house  go.\textsc{pfv}  3\textsc{sg.obj}=take 3\textsc{sg.obj}=beside-put-\textsc{pfv}\\
\glt ‘[It] went home to the man and put her next to him.’

\ex \negmedspace \textit{Mat iye keka mol i maya al} {\textit{men i mat}} \textit{mon}{\textit{i}}\textit{lïp.}\\
\gll ma{=}tï i-e        keka      ma={u}l       i ma{=i}ya      al  ma=in       i    ma=tï mon{i-}l{ï-}p\\
3\textsc{sg.obj}=take  go.\textsc{pfv-dep}  completely  3\textsc{sg.obj}=with  go.\textsc{pfv} 3\textsc{sg.obj=}toward  net  3\textsc{sg.obj=}in  go.\textsc{pfv}  \textsc{3sg.obj}=take  among-put-\textsc{pfv}\\
\glt ‘Having brought her, [it] went all the way with her, went to him into [his] mosquito net, and put her within [it].’

\ex \negmedspace \textit{M}{\textit{ï}} \textit{mol wop.}\\
\gll m{ï} m{a=u}l      wo-p\\
3\textsc{sg.subj}  3\textsc{sg.obj}=with  sleep-\textsc{pfv}\\
\glt ‘She slept with him.’\footnote{That is, she slept in the same bed as him. The young woman was asleep the whole time she was being transported.}

\newpage

\ex \negmedspace \textit{Mol w}{\textit{o}}\textit{pe y}{\textit{a}}\textit{na mï tïnanga lïmndï wa mala.}\\
\gll m{a=u}l      w{o-}p{{}-}e      y{a}na  mï      tïnanga    lïmndï  wa ma=ala\\
3\textsc{sg.obj}=with  sleep-\textsc{pfv-dep}  woman  3\textsc{sg.subj}  arise    eye    just 3\textsc{sg.obj}=see\\
\glt ‘Having slept {with him}, the woman got up and noticed him.’

\ex \negmedspace {\itshape
Lïmndï ankam ngala.}\\
\gll lïmndï  ankam  nga=ala\\
eye    person  \textsc{sg.prox}=see\\
\glt ‘[She] saw this person.’

\ex \negmedspace \textit{Mï keka se wap se wap se wap se wap keka awal pe imba pe} {\textit{wop.}}\\
\gll mï      keka      sa-e wap  sa-e    wap  sa-e wap  sa-e wap  keka      awal p{{}-}e    imba p{{}-}e {wo-p}\\
3\textsc{sg.subj}  completely  cry-\textsc{dep}  be.\textsc{pst}  cry-\textsc{dep}  be.\textsc{pst}  cry-\textsc{dep} be.\textsc{pst}  cry-\textsc{dep}  be.\textsc{pst}  completely  afternoon  be-\textsc{dep}  night  be\textsc{{}-dep} sleep-\textsc{pfv}\\
\glt ‘{And she cried and cried and cried} throughout the afternoon, throughout the night, and into the next day.’

\ex \negmedspace \textit{Keka maka wap makape imba pe w}{\textit{o}}\textit{p.}\\
\gll keka      maka wap  maka=p{{}-}e    imba p{{}-}e    w{o-}p\\
completely  thus  be.\textsc{pst}  thus=\textsc{cop}{}-\textsc{dep}  night  be-\textsc{dep}  sleep-\textsc{pfv}\\
\glt ‘It was totally like that, like this, {every} night.’

\ex \negmedspace {\itshape
Inim iwïl lele ndïtïn.}\\
\gll inim  iwïl  lele    ndï{=}tï{{}-}n\\
water  moon  three  3\textsc{pl}=take-\textsc{pfv}\\
\glt ‘Three years\footnote{The speaker makes a mistake (‘year’) but corrects it (‘month’).} -- months passed.’

\ex \negmedspace {\itshape
Iwïl lele ndïtïne yeta nga nan mat:}\\
\gll iwïl  lele    ndï{=}tï{{}-}n{{}-}e       yeta  nga    na{=}n    ma{=}ta\\
moon  three  3\textsc{pl}=take-\textsc{pfv-dep}  man  \textsc{sg.prox}  talk=\textsc{obl}  3\textsc{sg.obj}=say\\
\glt ‘And after three months, the man told her:’

\ex \negmedspace \textit{“N}{\textit{awa ango}} \textit{kalam m}{\textit{e nï i}} \textit{ungusuwal}{\textit{u}} \textit{i wap.”}\\
\gll nï-awa    ango kalam    {me} nï    i    u-ngusuwa{=}lu  i wap\\
1\textsc{sg-int}  \textsc{neg}  knowledge  \textsc{neg}  1\textsc{sg}  go.\textsc{pfv}  \textsc{2sg-}poor=with go.\textsc{pfv}  be.\textsc{pst}\\
\glt ‘“I really don’t know how I went and came to stay with you, you poor thing.”’

\ex \negmedspace \textit{“N}{\textit{ï}} \textit{ango kalam m}{\textit{e}} \textit{u anjikaka i wap.”}\\
\gll nï ango  kalam      me    u anjikaka  i wap\\
1\textsc{sg}  \textsc{neg}  knowledge    \textsc{neg}  2\textsc{sg}  how    go.\textsc{pfv}  be.\textsc{pst}\\
\glt ‘“{I don’t know how you got here.”}’

\ex \negmedspace \textit{“N}{\textit{ï}} \textit{ango kalam me.”}\\
\gll n{ï} ango  kalam      {m}e\\
1\textsc{sg}  \textsc{neg}  knowledge    \textsc{neg}\\
\glt ‘“I don’t know.”’

\ex \negmedspace {\textit{“Way nga tap ungusuwa tï i wapape.”}}\\
\gll wa{y} nga      tap    u-ngusuwa  t{ï} i wa{p-ap-e}\\
turtle  \textsc{sg.prox}  maybe  2\textsc{sg-}poor  take  go.\textsc{pfv}  be.\textsc{pst}{}-\textsc{pfv}{}-\textsc{dep}\\
\glt ‘“{Maybe this turtle brought you, you poor thing.”}’

\ex \negmedspace {\textit{“}}\textit{N}{\textit{ï}} \textit{ango angos na} {\textit{ukïn}}\textit{ate.”}\\
\gll \textit{n}{\textit{ï}} ango  angos  na    {u=kï-n}a{{}-}t{{}-}e\\
1\textsc{sg}  \textsc{neg}  what  talk  2\textsc{sg}=say-\textsc{irr-spec-dep}\\
\glt ‘“{I don’t have anything to tell you.”}’

\ex \negmedspace {\textit{“Awlopen luwa nda nguna map.”}}\\
\gll awlop=p{{}-}e{n} luwa  {a}nda    ngunan  ma=p\\
in.vain=\textsc{cop}{}-\textsc{nmlz}  place  \textsc{sg.dist}  1\textsc{du.incl}  3\textsc{sg.obj}=be\\
\glt ‘“{That strange place -- we are in it.”}’

\ex \negmedspace {\textit{“Nguna}} \textit{mbï nanap.”}\\
\gll {ngunan} mbï  na{{}-}na{{}-}p\\
1\textsc{du.incl}  here  \textsc{detr-detr-}be\\
\glt ‘“{We are staying here.”}’

\ex \negmedspace \textit{Mï nasap}{\textit{e}} \textit{mï mala li lamndu masap.}\\
\gll mï      na{{}-}sa{{}-}p{{}-e} mï       ma=ala    li-i {l}amndu  ma=asa-p\\
3\textsc{sg.subj}  \textsc{detr-}cry-\textsc{pfv-dep}  3\textsc{sg.subj}  3\textsc{sg.obj=}for  down-go.\textsc{pfv} pig      3\textsc{sg.obj}=hit-\textsc{pfv}\\
\glt ‘After she cried, he went down for her and killed a pig.’

\newpage

\ex \negmedspace {\itshape
Yeta mï mala li lamndu masap manke man up.}\\
\gll yeta  mï      ma=ala    li-i lamndu  ma=as{a-}p ma{=}nïkï{{}-}e      ma{=}n      u-p\\
man  3\textsc{sg.subj}  3\textsc{sg.obj}=for  down-go.\textsc{pfv}  pig      3\textsc{sg.obj}=hit-\textsc{pfv} 3\textsc{sg.obj}=dig{}-\textsc{dep}  3\textsc{sg.obj=obl}  put-\textsc{pfv}\\
\glt ‘The man went down for her, killed a pig, butchered it, and put {it [}in the house] for her.’

\ex \negmedspace {\itshape
Mol min ndïmondop.}\\
\gll m{a=u}l      min  ndï{=}mondo-p\\
3\textsc{sg.obj}=with  3\textsc{du}  3\textsc{pl}=dry-\textsc{pfv}\\
\glt ‘With her -- the two dried them [= the butchered pieces of meat].’

\ex \negmedspace {\itshape
Ndïmonde ndame ndïwat pe.}\\
\gll ndï{=}mondo-e    ndï={a}ma-e    ndï{=}wat  p{{}-}e\\
3\textsc{pl}=dry-\textsc{dep}    \textsc{3pl}=eat-\textsc{dep}  \textsc{3pl}=atop  be-\textsc{dep}\\
\glt ‘[They continued] drying them, eating them, and {relying}’\footnote{Literally ‘being atop’.} on them.

\ex \negmedspace \textit{Way nga min}{\textit{ï}}\textit{n t}{\textit{wa}} \textit{k}{\textit{a}}\textit{na map.}\\
\gll way  nga      min{=}{ï}{n} t{wa} k{a}na  {m}a{=}p\\
turtle  \textsc{sg.prox}  3\textsc{du=obl}  hearth  beside  3\textsc{sg.obj}=be\\
\glt ‘And this turtle stayed {there} with them next to the hearth.’

\ex \negmedspace \textit{Way nga min}{\textit{ï}}\textit{n twa k}{\textit{a}}\textit{na mape mï nan mat:}\\
\gll way  nga      min{=}{ï}{n} t{wa} k{a}na  {m}a{=}p{{}-}e      mï na{=}n ma{=}ta\\
turtle  \textsc{sg.prox}  3\textsc{du=obl}  hearth  beside  3\textsc{sg.obj}=be-\textsc{dep}  \textsc{3sg.subj} \textsc{detr=obl}  3\textsc{sg.obj}=say\\
\glt ‘And while this turtle stayed there with them next to the hearth, he [= the man] {said} to her:’

\ex \negmedspace {\itshape
“Tsk!”}\\
\gll tsk\\
tsk\\
\glt ‘“Tsk!”’\footnote{\isi{Phonetic}ally this is a \is{dental click} dental \isi{click} [ǀ]. In Ulwa it is a \isi{paralinguistic sound} used to express shock, compassion, or dismay. Here it signals the man’s sympathy for the woman.}

\newpage

\ex \negmedspace {\textit{“Way nga mï tap utïn utï i wap.”}}\\
\gll way  nga      mï      tap    u=tï-n      u=tï    i    wap\\
turtle  \textsc{sg.prox}  3\textsc{sg.subj}  maybe  2\textsc{sg}=take-\textsc{pfv}  \textsc{2sg}=take  go.\textsc{pfv}  be.\textsc{pst}\\
\glt ‘“This turtle, maybe he got you and brought you.”’

\ex \negmedspace {\textit{“Nï ango kalam me.”}}\\
\gll nï     ango  kalam      me\\
1\textsc{sg}  \textsc{neg}  knowledge    \textsc{neg}\\
\glt ‘“I don’t know.”’

\ex \negmedspace \textit{Way mï min}{\textit{ï}}\textit{n twa kana map.}\\
\gll way  mï      min=n t{wa} kana  ma{=}p\\
turtle  3\textsc{sg.subj}  3\textsc{du=obl}  hearth  beside  3\textsc{sg.obj}=be\\
\glt ‘The turtle stayed {there} with them next to the hearth.’

\ex \negmedspace \textit{Min}{\textit{ï}}\textit{n twa kana mape m}{\textit{i}}\textit{n ame.}\\
\gll min{=}{ï}{n} t{wa} kana  {m}a{=}p{{}-}e      m{i}n  ama-e\\
3\textsc{du=obl}  hearth  beside  \textsc{3sg.obj=}be-\textsc{dep}  3\textsc{du}  eat-\textsc{ipfv}\\
\glt ‘While [it] stayed with them {there} by the hearth, the two would eat.’

\ex \negmedspace \textit{Mundu ndïkuk man a}{\textit{w}}\textit{e mï ndame.}\\
\gll mundu  ndï{=}kuk    ma{=}n    a{w-}e      mï      ndï={a}ma-e\\
food  3\textsc{pl}=gather  3\textsc{gsg}=\textsc{obl}  put.\textsc{ipfv-dep}  \textsc{3sg.subj}  3\textsc{pl}=eat-\textsc{ipfv}\\
\glt ‘[They] would gather food for it and it would eat them [= the items of food].’

\ex \negmedspace \textit{Ndame nay nay way mï keka ne ambi nïpat nga}{\textit{t}}\textit{a nap.}\\
\gll ndï={a}ma-e    na{{}-i} na{{}-i} way  mï      keka      n{a-i} ambi  nïpat  nga{t}a na{{}-}p\\
3\textsc{pl}=eat-\textsc{dep}  \textsc{detr-}go.\textsc{pfv}  \textsc{detr}{}-go.\textsc{pfv} turtle  3\textsc{sg.subj}  completely  \textsc{detr-}go.\textsc{pfv}  big    giant  grand \textsc{detr-}{be}\\
\glt ‘[It] ate them and ate them until {the turtle} totally went {and} got big, giant, huge.’

\ex \negmedspace {\textit{Way mï keka ne ambi nïpat ngata nap ande.}}\\
\gll way  mï      keka       na-i      ambi  nïpat  ngata na-p    ande\\
turtle  3\textsc{sg.subj}  completely  \textsc{detr-}go.\textsc{pfv}  big    huge  giant \textsc{detr-}be  ok\\
\glt ‘The turtle completely went big, giant, huge, OK.’

\ex \negmedspace {\textit{Mï inakawane} --}\\
\gll mï      {i}na-ka-wana-e\\
3\textsc{sg.subj}  liver-in-feel-\textsc{dep}\\
\glt ‘He was thinking --’

\ex \negmedspace \textit{Ita tata mï inakawan}{\textit{e}} \textit{m}{\textit{ï}}\textit{nape.}\\
\gll {i-ta} tata    mï       ina-ka-wana-e m{ï=}na{=}p{{}-}e\\
go\textsc{.pfv-cond}  papa  3\textsc{sg.subj}  liver-in-feel-\textsc{dep}  3\textsc{sg.subj=detr}{}-be-\textsc{dep}\\
\glt ‘If [he] went -- the papa was thinking{ around there.}’\footnote{This line is hard to follow.}

\ex \negmedspace {\textit{Min yena mol mïnap min alum ndïnanayn.}}\\
\gll min  yena    m{a=u}l      m{ï=}na{{}-}p        min  alum ndï{=}na{{}-}n{a-in}\\
3\textsc{du}  woman    3\textsc{sg.obj}=with  \textsc{3sg.subj}=\textsc{detr-}be  3\textsc{du}  child {3\textsc{pl=detr-detr-}get}\\
\glt ‘They -- [he] stayed around there with his wife, and they had children.’

\ex \negmedspace \textit{M}{\textit{i}}\textit{n}{ }\textit{yena mol m}{\textit{ï}}\textit{nap min alum ndïnan}{\textit{ayn.}}\\
\gll min  alum  ndï{=}na{{}-}na{{}-i}n{{}-}e\\
3\textsc{du}  child  3\textsc{pl}=\textsc{detr-detr-}get-\textsc{ipfv}\\
\glt ‘They had children.’

\ex \negmedspace \textit{E yeta uwe ko way ma na}{\textit{n}} \textit{alum ndïkïna!}\\
\gll e  yeta  {u-}we      ko  way  ma na{=n} alum ndï{=}kï-na\\
hey  man  2\textsc{sg-part.int}  just  turtle  3\textsc{sg.obj}  talk=\textsc{obl}  child 3\textsc{pl}=say-\textsc{irr}\\
\glt ‘Hey, man, you yourself should have {just told the children about the turtle}!’\footnote{The narrator is \is{apostrophe} apostrophically addressing the man in the story, who should have told his children that this turtle that lives around them is no ordinary turtle, but something like a foster parent to their own parents. The line seems to have a false start.}

\ex \negmedspace {\textit{“U}}{\textit{n}}{ \textit{wana mbï}}{\textit{wa}}{\textit{p angos ngan anjikalakana!”}}\\
\gll u{n} wana  mbï{{}-wa}p     angos     {n}ga=n    anjika-la-{ka-na}\\
2\textsc{pl}  \textsc{proh}  here-be.\textsc{pst}  what    \textsc{sg.prox=obl}  how.many-\textsc{irr}{}-let-\textsc{irr}\\
\glt ‘“Don’t do something [bad] to this [turtle] here!”’\footnote{This is something along the lines of what the father should have told his children.}

\ex \negmedspace \textit{Way mï nï min ndïwa}{\textit{na}} \textit{ande.}\\
\gll way   mï       nï   min     ndï{=}wa{na} ande\\
turtle  3\textsc{sg.subj}  1\textsc{sg}  3\textsc{du}  3\textsc{pl}=feel  ok\\
\glt ‘The turtle, I, they two, heard them, OK.’\footnote{This line seems to be confused.}

\newpage

\ex \negmedspace \textit{Mï alum ndï}{\textit{nayne}} \textit{way mï mala inim namana man.}\\
\gll mï      alum  ndï{=na-in-e} way   mï       ma=ala inim  na{{}-}ma{{}-}na   ma{{}-}n\\
3\textsc{sg.subj}  child  3\textsc{pl=detr-}get\textsc{{}-dep} turtle  3\textsc{sg.obj}  3\textsc{sg.obj=}from water  \textsc{detr-}go-\textsc{irr}  go-\textsc{ipfv}\\
\glt ‘[After] he [= {the man]} had children, the turtle {was going to go away from him, [back] to the water.}’

\ex \negmedspace {\itshape
Mala inim namana mane.}\\
\gll ma=ala     inim   na{{}-}ma{{}-}na     ma{{}-}n{{}-}e\\
3\textsc{sg.obj=}from  water  \textsc{detr-}go-\textsc{irr}  go-\textsc{ipfv-dep}\\
\glt ‘[It] was going to go {away from him} to the water.’

\ex \negmedspace {\itshape
Imba pe ala maka longom tï manana.}\\
\gll imba{} p{{}-}e     ala       maka   longom  tï    ma{=}na-na\\
night  be-\textsc{dep}    \textsc{pl.dist}  thus  dream    take  3\textsc{sg.obj}=give-\textsc{pfv}\\
\glt ‘At night, they\footnote{The \isi{plural} \isi{demonstrative} \isi{pronoun} here perhaps refers to spirits that grant people dreams at night.}{} gave him [= the man] a dream like this.’

\ex \negmedspace \textit{Nan mat: “Nï} {\textit{wandïm}} \textit{inim namana.”}\\
\gll na{=}n{} ma{=}ta      nï    {u=andïm} inim  na{{}-}ma{{}-}na\\
talk=\textsc{obl}  3\textsc{sg.obj}=say  1\textsc{sg}  2\textsc{sg}=from  water  \textsc{detr-}go-\textsc{irr}\\
\glt ‘[It] told him: “I will go {from you [back] to the wat}e{r}.”’\footnote{The man has a premonitory dream, in which the turtle tells him that it will leave him.}

\ex \negmedspace {\itshape
Itom mï wop umbenam lamndu masap.}\\
\gll itom  mï       wo-p    umbenam  lamndu  ma=asa-p\\
father  3\textsc{sg.subj}  sleep-\textsc{pfv}  morning  pig      3\textsc{sg.obj}=hit-\textsc{pfv}\\
\glt ‘The man slept and in the morning killed a pig.’

\ex \negmedspace {\itshape
Wonmelma.}\\
\gll Wonmelma\\
[name]\\
\glt ‘Wonmelma.’

\ex \negmedspace \textit{Lamndu manji wi Wonmel}{\textit{m}}\textit{a.}\\
\gll lamndu  ma-nji      wi    Wonmel{m}a\\
pig      3\textsc{sg.obj-poss}  name  [name]\\
\glt ‘The pig’s name was Wonmelma.’\footnote{Text “ulwa001” tells the tale of Wonmelma, a popular figure in Ulwa storytelling.}

\ex \negmedspace {\itshape
Masape way mol mïnanamap.}\\
\gll ma=asa{{}-}p{{}-e} way  m{a=u}l      mï{=}na{{}-}na-ama-p\\
3\textsc{sg.obj}=hit-\textsc{pfv-dep}  turtle  3\textsc{sg.obj}=with  3\textsc{sg.subj}{}-\textsc{detr-detr}{}-eat-\textsc{pfv}\\
\glt ‘After [he] killed it, [he] ate it with the turtle.’

\ex \negmedspace \textit{A}{\textit{t}} \textit{kwa man m}{\textit{ï}} \textit{mat man ani lïp.}\\
\gll a{t} kwa  ma{=n} m{ï} ma{=}tï{} ma{=}n      ani lï-p\\
end  one    3\textsc{sg.obj=obl}  3\textsc{sg.subj}  3\textsc{sg.obj}=take  3\textsc{sg.obj=obl}  bilum put-\textsc{pfv}\\
\glt ‘{One} piece [of the meat]{ --} he put {it} in the \textit{bilum} [= string bag] for it \mbox{[= the turtle].’}

\ex \negmedspace \textit{Way mï ango man ka} {\textit{li}} \textit{mana.}\\
\gll way  mï      ango  ma{=}n{} ka  {li} ma{{}-}na\\
turtle  3\textsc{sg.subj}  \textsc{neg}  3\textsc{sg.obj}=\textsc{obl}  let  down  {go-}\textsc{irr}\\
\glt ‘The turtle would not [yet] {leave him and go down.}’

\ex \negmedspace {\itshape
Mï kop mol mape.}\\
\gll mï      kop  m{a=u}l    ma{=}p{{}-}e\\
3\textsc{sg.subj}  just  \textsc{3sg.obj}=with  3\textsc{sg.obj}=be-\textsc{ipfv}\\
\glt ‘It just stayed with him.’

\ex \negmedspace {\itshape
Mï ko mol mape nogat!}\\
\gll mï      ko  m{a=u}l      ma{=}p{{}-}e      nogat\\
3\textsc{sg.subj}  just  3\textsc{sg.obj}=with  3\textsc{sg.obj}=be-\textsc{ipfv}  no\\
\glt ‘It just stayed with him -- no!’\footnote{The \isi{interjection} \textit{nogat} ‘no’ is from \ili{Tok Pisin}.}

\ex \negmedspace {\textit{Itom manji alum ndï ndï} --}\\
\gll itom  ma-nji      alum  ndï  {ndï}\\
father  3\textsc{sg.obj-poss}  child  3\textsc{pl}  \textsc{3pl}\\
\glt ‘The father’s children, {they –}’

\ex \negmedspace \textit{Min numan} {\textit{yena ul}} \textit{wandam mane.}\\
\gll min  numan    yena    ul wandam  ma{{}-}n{{}-}e\\
3\textsc{du}  husband  woman    with  jungle    go-\textsc{ipfv-dep}\\
\glt ‘They -- the husband was going around the jungle with [his] wife.’

\ex \negmedspace {\textit{I nan nungolke ngalakapta ndï kalampïn!}}\\
\gll i  na{=n} nungolke  ngala{=}kï-p{}-ta        ndï kalam{=}p-{na}\\
ay  talk=\textsc{obl}  child    \textsc{pl.prox}=say-\textsc{pfv-cond}  3\textsc{pl}  knowledge=\textsc{cop}{}-\textsc{irr}\\
\glt ‘Ay, tell these children so that they’ll know!’\footnote{The narrator is again addressing the man in the story.}

\ex \negmedspace {\itshape
Nogat.}\\
\gll nogat\\
no\\
\glt ‘No.’\footnote{The \isi{interjection} \textit{nogat} ‘no’ is from \ili{Tok Pisin}, here used to express the fact that something did not occur (i.e., the father never did tell his children about the turtle.)}

\ex \negmedspace \textit{Alum yeta mï ko m}{\textit{awap}} \textit{nali wongïta tïn ko way ngusu}{\textit{wa}} \textit{ma}{\textit{n lïmndï}} \textit{m}{\textit{o}} \textit{m}{\textit{a}}\textit{ka mas.}\\
\gll alum  yeta  mï      ko   m{a=wap} nali  wongïta  tï{{}-}n ko  way  ngusu{wa} ma=n      lïmndï m{a=u} m{a}ka ma=asa\\
child  man  3\textsc{sg.subj}  just  3\textsc{sg.obj}=be.\textsc{pst}  spine  bow    take-\textsc{pfv} just  turtle  poor    3\textsc{sg.obj=obl}  eye    3\textsc{sg.obj=}from  thus 3\textsc{sg.obj}=hit\\
\glt ‘{The son just stayed there and got a sago frond bow}\footnote{That is, a set of bow and arrow. The arrow is made from a \textit{nali} ‘sago frond spine’.} and just hit the poor turtle {like this} in the eye.’

\ex \negmedspace {\textit{Yeta mï way mï anmbi inim i.}}\\
\gll yeta  mï      way  mï      an{{}-}mbï-{i} inim  i\\
man  3\textsc{sg.subj}  turtle  3\textsc{sg.subj}  out-here-go.\textsc{pfv}  water  go.\textsc{pfv}\\
\glt ‘The man\footnote{This seems to be another false start.} -- the turtle went out into the water.’

\ex \negmedspace {\textit{Mase m}}{\textit{ï}}{\textit{ka nali nungun ma lïmndï upe.}}\\
\gll ma=asa-e      m{ï}ka  nali  mï      nungun  ma      lïmndï {u-p-e}\\
3\textsc{sg.obj}=hit-\textsc{dep}  thus  spine  3\textsc{sg.subj}  break    3\textsc{sg.obj}  eye put-\textsc{pfv-dep}\\
\glt ‘Having hit it, the spine thus broke into its eye.’

\ex \negmedspace {\textit{Mï mol anmbi inim nay.}}\\
\gll m{ï} m{a=u}l      an{{}-}mbï-{i} inim  {na-i}\\
3\textsc{sg.subj}  \textsc{3sg.obj}=with  out-here-go.\textsc{pfv}  water  \textsc{detr-}go.\textsc{pfv}\\
\glt ‘It went with it [the spine arrow] out into the water.’

\ex \negmedspace {\textit{Mï mol anmbi inim naye}} {\textit{mïnape.}}\\
\gll m{ï} m{a=u}l      an{{}-}mbï-{i} inim  na{=i-e} {mï=na-p-e}\\
3\textsc{sg.subj}  \textsc{3sg.obj}=with  out-here-go.\textsc{pfv}  water  \textsc{detr-}go.\textsc{pfv-dep} \textsc{3sg.subj=detr-}be-\textsc{ipfv}\\
\glt ‘It went with it out into the water and stayed around there.’

\newpage

\ex \negmedspace {\itshape
Itom mï wa i manglalop.}\\
\gll itom  mï      wa    i ma={a}ngla{{}-}lo-p\\
father  3\textsc{sg.subj}  village  go.\textsc{pfv}  3\textsc{sg.obj}=await-go-\textsc{pfv}\\
\glt ‘The man went to the village and searched for it.’

\ex \negmedspace \textit{“Inom} {\textit{ngata}} \textit{ngusuwa nda} {\textit{a}}\textit{ng}{\textit{o luwa}} \textit{nay?”}\\
\gll inom  {ngata} ngusuwa  {a}nda    {a}ngo  luwa na{{}-i}\\
mother  grand  poor    \textsc{sg.dist}  which  place  \textsc{detr-}go.\textsc{pfv}\\
\glt ‘“Where did that poor grandmother go?”’\footnote{The man and woman refer to the turtle as ‘grandmother’, since it was a foster parent to them and therefore a foster grandparent to their children.}

\ex \negmedspace {\textit{Ndï atwana}} \textit{nungolke ndïte.}\\
\gll ndï  atwana nungolke  ndï{=}ta-e\\
3\textsc{pl}  question  child    3\textsc{pl}=say-\textsc{dep}\\
\glt ‘They asked the children.’

\ex \negmedspace \textit{Alum ndïte alum ndï} {\textit{nat:}} \textit{“An ango kalam me.”}\\
\gll alum  ndï{=}ta-e    alum  ndï  {na-ta} an      ango  kalam me\\
child  3\textsc{pl}=say-\textsc{dep}  child  3\textsc{pl}  \textsc{detr}{}-say  1\textsc{pl.excl}  \textsc{neg}  knowledge {\scshape neg}\\
\glt ‘[They] asked the children and the children replied: “We don’t know.”’

\ex \negmedspace {\textit{Yenanu nungol}} {\textit{m}}{\textit{awape nan mat: “Nogat ya!”}}\\
\gll yenanu    nungol  {m}a{=}wap{{}-}e        na{=}n    {m}a=ta    nogat {ya}\\
woman    child  3\textsc{sg.obj}=be.\textsc{pst-dep}  talk=\textsc{obl}  \textsc{3sg.obj}=say  no yes\\
\glt ‘But the daughter later\footnote{Translated here as ‘later’, \textit{mawape} ‘3\textsc{sg.obj}=be.\textsc{pst-dep}’ literally means something like ‘having been there’.} told him [= the father]: “Not at all!”’\footnote{In other words, she confesses that she and the other children had been lying; the expression \textit{nogat ya} (literally ‘no yes’) is from \ili{Tok Pisin}.}

\ex \negmedspace {\itshape
“Yeta nda mas wape!”}\\
\gll yeta  {a}nda    ma=asa    wap{{}-}e\\
man  \textsc{sg.dist}  3\textsc{sg.obj}=hit  be.\textsc{pst-dep}\\
\glt ‘“That boy hit it!”’

\ex \negmedspace \textit{“Nali} {\textit{wongïta ndan mangusuwa}} \textit{lïmnd}{\textit{ï anda}}\textit{ka mas wap.”}\\
\gll nali  wongïta  anda=n    ma-ngusuwa lïmndï  anda=ka ma=asa    wap\\
spine  bow    \textsc{sg.dist=obl}  \textsc{3sg.obj-}poor    eye    \textsc{sg.dist}=in 3\textsc{sg.obj}=hit  be.\textsc{pst}\\
\glt ‘“{[He] hit the poor thing in the eye with that sago frond bow.”}’

\ex \negmedspace {\itshape
Itom mï way manakap tïnanga se.}\\
\gll itom  mï      way  ma{=}nakap    tïnanga  sa-e\\
father  3\textsc{sg.subj}  turtle  3\textsc{sg.obj=}for  arise  cry-\textsc{ipfv}\\
\glt ‘The father got up and began to cry on account of the turtle.’

\ex \negmedspace {\textit{Way mï ta awal pe imba pe i may}}{\textit{a apa}}{ \textit{i lïmndï man mol}} {\textit{si.}}\\
\gll way  mï      ta      awal    p-e    imba p{{}-}e    {i} ma=iya      apa {i} lïmndï  ma{=}n m{a=u}l      {si}\\
turtle  \textsc{3sg.subj}  already    afternoon  be-\textsc{dep}  night  be-\textsc{dep}  go.\textsc{pfv} 3\textsc{sg.obj=}toward  house  go.\textsc{pfv}  eye    3\textsc{sg.obj=obl}  3\textsc{sg.obj=}with  push\\
\glt ‘The turtle -- already in the afternoon\footnote{The speaker again corrects his word choice.} -- went at night, went to him in the house, and showed him [its injured] eye.’

\ex \negmedspace \textit{“Ngam u} {\textit{nïn l}}\textit{ïmndï} {\textit{ngaka}} \textit{nase.”}\\
\gll nga{{}-na}m      u    {nï=n} {l}ïmndï  {nga=ka} nï=asa-e\\
\textsc{sg.prox-emph}  \textsc{2sg}  \textsc{1sg=obl}  eye    \textsc{sg.prox=}in  \textsc{1sg=}hit-\textsc{dep}\\
\glt ‘“This is it -- you shot me in my eye.”’\footnote{The turtle presumably speaks this line, as well as the following one.}

\ex \negmedspace \textit{“Nï wand}{\textit{ï}}\textit{m namana man.”}\\
\gll nï    {u=a}nd{ï}m  na{{}-}ma{{}-}na    ma{{}-}n\\
1\textsc{sg}  2\textsc{sg=}from  \textsc{detr-}go-\textsc{irr}  go-\textsc{ipfv}\\
\glt ‘“So {I’m going to go to away from you.”}’

\ex \negmedspace \textit{Itom mï mala wo}{\textit{p}} \textit{wolka li lamndu masap.}\\
\gll itom  mï      ma=ala    wo-{p} wolka  li-i lamndu ma=as{a-}p\\
father  3\textsc{sg.subj}  3\textsc{sg.obj=}for  sleep-\textsc{pfv}  again  down-go-\textsc{pfv}  pig 3\textsc{sg.obj}=hit-\textsc{pfv}\\
\glt ‘The next day, the man went down again and killed a pig for it.’

\newpage

\ex \negmedspace \textit{Wolka lamndu kwa masape mol mïnanamap a}{\textit{t}} \textit{kuma ndïn man} {\textit{ame}} \textit{naytap.}\\
\gll wolka  lamndu  kwa  ma=asa-p{{}-}e      m{a=u}l mï{=}na{{}-}na{}-am{a-}p          a{t} kuma  ndï={n} ma=n ame		na-ita-p\\
again  pig      one    3\textsc{sg.obj}=hit-\textsc{pfv-dep}  3\textsc{sg.obj}=with 3\textsc{sg.subj=detr-detr-}eat-\textsc{pfv}  end    some  \textsc{3pl=obl}  \textsc{3sg.obj=obl} basket  \textsc{detr-}tie-\textsc{pfv}\\
\glt ‘{Having killed another pig, [he] ate it with it [= the turtle]} and tied some pieces [of meat] up in his {basket}.’

\ex \negmedspace {\itshape
Ndït manane mï ndït nay inim nay.}\\
\gll ndï=tï ma=na-n-e          mï      ndï=tï na-i inim  na-i\\
3\textsc{pl}=take  3\textsc{sg.obj}=give-\textsc{pfv-dep}  \textsc{3sg.subj}  \textsc{3pl}=take  \textsc{detr-}go.\textsc{pfv} water  \textsc{detr-}go.\textsc{pfv}\\
\glt ‘[He] gave them [= the pieces of meat] to it [= the turtle], and it took them and went, went into the water.’

\ex \negmedspace {\textit{Inim naye una amun lïmndï way ambi ndanden.}}\\
\gll inim  na{{}-i-e} u{n}a{n} {a}mun  {l}ïmndï  way  ambi {a}nda{=andï-en}\\
water  \textsc{detr-}go.\textsc{pfv-dep}  \textsc{1pl.incl}  now  eye    turtle  big \textsc{sg.dist}=see-\textsc{nmlz}\\
\glt ‘Having gone into the water, we now see that big turtle.’

\ex \negmedspace \textit{Nd}{\textit{ï}} \textit{ang}{\textit{u}}\textit{mo}{\textit{ni nïmal ando}} \textit{inimp.}\\
\gll nd{ï} ang{u}moni  nïmal  anda=u inim p[-e]\\
3\textsc{pl}  swelling  river  \textsc{sg.dist}=from  water  be[-\textsc{ipfv]}\\
\glt ‘They are in the swelling river,\footnote{That is, the ocean: \textit{angumoni nïmal} ‘swelling river, ocean, sea’ may perhaps best be thought of as a \isi{compound noun}, especially since it has the (non-canonical) order \isi{adjective}-noun. The big turtles referred to here are sea turtles.} in the water.’

\ex \negmedspace {\textit{Nd}}{\textit{ï}} {\textit{ang}}{\textit{u}}{\textit{mon}}{\textit{i nï}}{\textit{mal mo inim pe.}}\\
\gll nd{ï} ang{u}moni  nïmal  ma=u      inim  p-e\\
3\textsc{pl}  swelling  river  3\textsc{sg.obj=}from  water  be-\textsc{ipfv}\\
\glt ‘They are in the water in the swelling river.’

\newpage

\ex \negmedspace {\itshape
Una way ambi way ambi ndï --}\\
\gll unan way  ambi  way  ambi  ndï\\
1\textsc{pl.incl}  turtle  big    turtle  big    3\textsc{pl}\\
\glt ‘We -- big turtles, the big turtles --’

\ex \negmedspace \textit{Ndïna}{\textit{m}} \textit{nd}{\textit{ï angumoni nïmal}} \textit{map.}\\
\gll ndï{{}-}na{m} ndï  angumoni  nïmal ma{=}p\\
3\textsc{pl-emph}  \textsc{3pl}  swelling  river  3\textsc{sg.obj}{=be}\\
\glt ‘They’re the ones; they live in the swelling river.’

\ex \negmedspace {\itshape
Mawnam.}\\
\gll mawnam\\
thats.it\\
\glt ‘That’s it.’
\z

\section{\label{sec:16.2}  {\textit{Amblom} \textit{Yena}} {(‘The} {Woman} {Amblom’)}}

This is a traditional story told by Yanapi Kua on 26  {May 2017}, at her home in Manu village. Examples from this text that appear elsewhere in this book are labeled “ulwa020\_mm:ss”. The audio recording can be found on the ELAR website (file name: ulwa020.wav). It is a little over two minutes long (02:17).

  The story is, among other things, an etiology of the moon. In the tale, a wicked woman named Amblom lives in the village. Whenever the men and women of the village go off into the jungle to harvest sago starch, she captures their children, decapitates them, and eats them. She then hides their bones in the top of a sago palm. Eventually, however, the parents discover Amblom’s secret and decide to kill her. She evades them, however, by climbing the palm where she keeps the children’s bones. The parents run to the palm but she exerts a magical force over it, so there is little that they can do to get her down. In some versions of the story she taunts the parents by throwing feces at them -- that is, the final product of their eaten children. They cannot chop down the palm, nor can they shoot her down with arrows. Finally, a mysterious stranger comes to the village, supposedly a friend of one of the villagers. Versed in magic, he is able to shoot down Amblom with an enchanted arrow. She falls to her death and the villagers butcher her body. They begin distributing her flesh as meat, offering to the stranger whichever body part he prefers. He refuses all the choicest cuts, requesting instead Amblom’s vulva. He places the vulva on a frond of the palm where she was hiding, whereupon it transforms into a glowing torch. Using this torch, he has great success hunting pigs. The stranger’s friend learns of his fruitful hunts, but does not know his secret. The stranger, not wanting to reveal his magical glowing vulva, instructs his friend simply to build a regular torch from palm fronds. The friend has some initial success killing a small pig, but, when he tries to kill a larger one, he himself is nearly killed by the boar. Suspecting that he has been tricked, he spies on the stranger’s home, discovering the magical vulva. However, while trying to grab it, he clumsily disrupts the vulva and sends it flying up into the sky where it remains to this day as the moon. The Ulwa conception that there is something vaginal about the moon can still be seen in the fact that the word \textit{iwïl} ‘moon’ also means ‘menstruation’, no doubt related to the similarity in duration of the lunar cycle and menstrual cycle.

  The story also contains an epilogue in which the stranger builds a huge ladder to rescue the magical vulva (now the moon). He manages to reach the moon and dangle from it. While hanging there, however, a colony of bats comes by to inspect this strange new being. When he declines the fruit that they offer him, they become suspicious that he is not one of them, so they yank him away. And no human since has been able to reach the moon. In some versions of the story, these bats are said to be the stars that surround the moon in the night sky.

\setcounter{equation}{0}
\ea {\itshape
Amblom Yena mï --}\\
\gll Amblom  Yena    mï\\
[name]    woman    3\textsc{sg.subj}\\
\glt ‘Amblom Yena --’

\ex {\itshape
Ndï nungolke ndïn man lïp.}\\
\gll ndï  nungolke  ndï=n    ma=n      lï-p\\
3\textsc{pl}  child    3\textsc{pl=obl}  3\textsc{sg.obj=obl}  put-\textsc{pfv}\\
\glt ‘They [= other villagers] left [their] children with her.’

\ex {\itshape
Wandam unde ulum ale mï wa mape.}\\
\gll wandam  unda-e  ulum  ali-e    mï      wa    ma=p-e\\
jungle    go-\textsc{dep}  palm  scrape-\textsc{dep}  \textsc{3sg.subj}  village  \textsc{3sg.obj}=be-\textsc{ipfv}\\
\glt ‘When [they] would go around in the jungle and scrape sago palms, she stayed in the village.’

\newpage

\ex \textit{Ndïnji unduwan nduwe we ndame uma ndït li unde ndïkuk maka ulum nowe nda ndïn} --\\
\gll ndï-nji    unduwan  ndï=we  we     ndï=ama-e    uma  ndï=tï li    unda-e  ndï=kuk    maka  ulum  nowe    anda ndï=n\\
3\textsc{pl-poss}  head    3\textsc{pl}=cut  then  3\textsc{pl}=eat-\textsc{dep}  bone  3\textsc{pl}=take down  go-\textsc{dep}  3\textsc{pl}=gather  thus  palm  sago.species  \textsc{sg.dist} \textsc{3pl=obl}\\
\glt ‘[She] would cut off their heads and then eat them, bring [their] bones down, and pile them -- like, that sago palm\footnote{The narrator specifies that the palm is a \textit{nowe} palm, a large sago palm species that has no thorns on its stem.} -- with them --’

\ex {\itshape
Ndïkuk mo ma awi up.}\\
\gll ndï=kuk    ma=u      ma      awi      u-p\\
3\textsc{pl}=gather  3\textsc{sg.obj}=from  3\textsc{sg.obj}  shoulder  put-\textsc{pfv}\\
\glt ‘[She] piled them there onto its shoulder.’\footnote{That is, she piles the eaten children’s bones into a crevice in the frond of the palm.}

\ex {\itshape
Ndï nokoplïp lïmndï mala.}\\
\gll ndï  nokop-lï-p    lïmndï  ma=ala\\
3\textsc{pl}  hide-put-\textsc{pfv}  eye    3\textsc{sg.obj}=see\\
\glt ‘But they [= the parents] hid and saw her.’

\ex {\itshape
I ma nan amblakap.}\\
\gll i    ma      na=n    ambla=kï-p\\
go.\textsc{pfv}  3\textsc{sg.obj}  talk=\textsc{obl}  \textsc{pl.refl}=say-\textsc{pfv}\\
\glt ‘They went and talked about her.’

\ex {\itshape
Matïna nakap iye.}\\
\gll ma=atï-na      na-kï-p      i-e\\
3\textsc{sg.obj}=hit-\textsc{irr}  \textsc{detr-}say-\textsc{pfv}  go.\textsc{pfv-dep}\\
\glt ‘[They] wanted to kill her.’

\ex {\itshape
Mï li awlu ulum mo ma we ulum mï keka i wutotap.}\\
\gll mï      li    i    awlu  ulum  ma=u       ma  we    ulum mï      keka      i     wutota=p\\
3\textsc{sg.subj}  down  go.\textsc{pfv}  step  palm  3\textsc{sg.obj=}from  go  then  palm 3\textsc{sg.subj}  completely  go.\textsc{pfv}  tall=\textsc{cop}\\
\glt ‘But she went down [from her house] and stepped onto the palm, and then the palm went and got really tall.’\footnote{Amblom performs some magic to make the palm grow tall.}

\ex {\itshape
Wutotape ndï wongïta tïn mol asap ulwape.}\\
\gll wutota=p-e  ndï  wongïta  tï-n      ma=ul      asa-p ulwa=p-e\\
tall=\textsc{cop}{}-\textsc{dep}  3\textsc{pl}  bow    take-\textsc{pfv}  3\textsc{sg.obj}=with  hit-\textsc{pfv} nothing=\textsc{cop}{}-\textsc{dep}\\
\glt ‘Since [it] was tall, they could hit nothing when they got a bow and shot with it.’

\ex {\itshape
Kwa ngawa wandam ngo i ndiya wa i.}\\
\gll kwa  nga-awa    wandam  nga=u        i    ndï=iya wa    i\\
one    \textsc{sg.prox-int}  jungle    \textsc{sg.prox=}from  go.\textsc{pfv}  3\textsc{pl}=toward village  go.\textsc{pfv}\\
\glt ‘But someone just came from out of the jungle and went to them in the village.’

\ex {\itshape
Wongïta matïn man mawl as.}\\
\gll wongïta  ma=tï-n      ma=n      ma=ul      asa\\
bow    3\textsc{sg.obj}=take-\textsc{pfv}  3\textsc{sg.obj=obl}  3\textsc{sg.obj}=with  hit\\
\glt ‘[He] took the bow and shot at her with it.’

\ex {\itshape
Manji sawi manip mawl ase.}\\
\gll ma-nji      sawi  ma=ni-p      ma=ul      asa-e\\
\textsc{3sg.obj-poss}  magic  3\textsc{sg.obj}=act-\textsc{pfv}  3\textsc{sg.obj}=with  hit-\textsc{dep}\\
\glt ‘[He] sang his magic charm and shot with it.’

\ex {\itshape
Kwa mï man ambï aweta kap.}\\
\gll kwa  mï      ma=n      ambï    aweta  kï-p\\
one    3\textsc{sg.subj}  3\textsc{sg.obj=obl}  \textsc{sg.refl}  friend  say-\textsc{pfv}\\
\glt ‘Someone\footnote{That is, one of the villagers identifies the stranger as his friend.} said that it was his friend.’

\ex {\itshape
“A nïnji aweta anda ko matïna!”}\\
\gll a  nï-nji    aweta  anda    ko  ma=atï-na\\
ah  1\textsc{sg-poss}  friend  \textsc{sg.dist}  just  3\textsc{sg.obj=}hit-\textsc{irr}\\
\glt ‘“Ah, that friend of mine will really hit her!”’

\ex {\itshape
Mï asika sawi manip ulwape.}\\
\gll mï      asi-ka  sawi  ma=ni-p      ulwa=p-e\\
3\textsc{sg.subj}  sit-let  magic  3\textsc{sg.obj}=act-\textsc{pfv}  nothing=\textsc{cop}{}-\textsc{dep}\\
\glt ‘He [= the stranger] sat and sang the magic spell to the end --’

\ex {\itshape
Keka man u wongïta matïn keka mase mï keka nip.}\\
\gll keka      ma=n      u    wongïta  ma=tï-n keka      ma=asa-e      mï       keka      ni-p\\
completely  3\textsc{sg.obj=obl}  from  bow    3\textsc{sg.obj}=take-\textsc{pfv} completely  3\textsc{sg.obj}=hit-\textsc{dep}  3\textsc{sg.subj}  completely  die-\textsc{pfv}\\
\glt ‘Totally -- [he] got the bow from him [= his friend] and totally hit her and she died completely.’

\ex {\itshape
Ulum molop li lïp men u uma ndïkuk anmbup.}\\
\gll ulum  ma=lo-p      li    lï-p      ma=in      u    uma ndï=kuk    an-mbï-u-p\\
palm  3\textsc{sg.obj}=cut-\textsc{pfv}  down  put-\textsc{pfv}  \textsc{3sg.obj=}in  from  bone 3\textsc{pl}=gather  out-here-put-\textsc{pfv}\\
\glt ‘[They] cut the sago palm down and gathered the bones out from within it.’

\ex {\itshape
Mankap at kot mananda nate.}\\
\gll ma=nïkï-p      at  ko=tï    ma=na-nda    na-ta-e\\
3\textsc{sg.obj}=dig{}-\textsc{pfv}  end  \textsc{indf}=take  3\textsc{sg.obj}=give-\textsc{irr}  \textsc{detr-}say-\textsc{dep}\\
\glt ‘[They] butchered her and talked about giving a piece [of her body] to him [= the stranger].’\footnote{Having butchered Amblom Yena’s fallen body, the villagers distribute her body parts as food. The stranger, as the hero of the day, is offered various body parts.}

\ex {\itshape
Mï kambï man ndït: “M!”}\\
\gll mï      kambï  ma=n      ndï=ta    m\\
3\textsc{sg.subj}  shun  3\textsc{sg.obj=obl}  3\textsc{pl}=say  \textsc{interj}\\
\glt ‘But he didn’t want it and told them: “No!”’

\ex {\itshape
“Un maka ma nambï pen ngat nïnata nï mat mana.”}\\
\gll un  maka   ma      nambï  p-en    nga=tï      nï=na-ta nï    ma=tï    ma-na\\
2\textsc{pl}  thus  3\textsc{sg.obj}  body  be-\textsc{nmlz}  \textsc{sg.prox}=take  1\textsc{sg}=give-\textsc{cond} 1\textsc{sg}  \textsc{3sg.obj}=take  go-\textsc{irr}\\
\glt ‘“If you, like this, give me this thing on her body, I will take it and go.”’\footnote{The stranger rejects the offers of various body parts, asking instead for Amblom Yena’s vulva.}

\newpage

\ex {\itshape
Mï mat i mas isi pul mat lïp.}\\
\gll mï      ma=tï      i    ma=si      isi        pul ma=tï      lï-p\\
3\textsc{sg.subj}  3\textsc{sg.obj}=take  go.\textsc{pfv}  3\textsc{sg.obj}=push  young.pangal  piece 3\textsc{sg.obj}=take  put-\textsc{pfv}\\
\glt ‘He brought it and pushed it onto a piece of palm frond.’\footnote{The word \textit{isi} ‘young palm frond’ refers to a younger form of the \textit{wema} ‘palm frond’. Here the frond is being used as a pike.}

\ex {\itshape
Kukumbe isi pul mat lïpe mï tembip.}\\
\gll kukumbe  isi    pul    ma=tï      lï-p-e       mï      tembi=p\\
sago.species  pangal  piece  3\textsc{sg.obj}=take  put-\textsc{pfv-dep}  3\textsc{sg.subj}  bad=\textsc{cop}\\
\glt ‘[He] put it on a piece of \textit{kukumbe} sago frond,\footnote{The narrator specifies the species of sago palm to which the  palm frond pike belongs: \textit{kukumbe} ‘sago species (sago palm with no spines)’.} but it was bad.’

\ex {\itshape
Mï mas nowe isi pul mat lïpe mï anmap!}\\
\gll mï      ma=si      nowe    isi        pul    ma=tï lï-p-e      mï       anma=p\\
3\textsc{s.subj}  3\textsc{sg.obj}=push  sago.species  young.pangal  piece  3\textsc{sg.obj}=take put-\textsc{pfv-dep}  3\textsc{sg.subj}  good=\textsc{cop}\\
\glt ‘So he pushed it onto a piece of a \textit{nowe} sago fond,\footnote{The narrator again specifies the species of sago palm: \textit{nowe} ‘sago species (large sago palm with no spines)’.} and it was good!’\footnote{The mysterious hero, clearly familiar with magic, knows that Amblom Yena’s vulva has special properties. He is experimenting with different species of palm fronds to discover how to harness its power.}

\ex {\itshape
Pe mï ma anenisin namndu nduwalep.}\\
\gll p-e    mï      ma      ane-nisi=n          namndu ndï=wali-e=p\\
be-\textsc{dep}  3\textsc{sg.subj}  3\textsc{sg.obj}  sun-flower.sheath=\textsc{obl}  pig 3\textsc{pl}=hit-\textsc{dep}=\textsc{cop}\\
\glt ‘And then he, with its torch,\footnote{Literally its ‘sun-coconut-flower-sheath’. The flower pods of coconut palms were traditionally used as torches.} was killing pigs.’\footnote{Now affixed to the proper species of palm frond, the vulva emits a light like a torch, which the stranger can use to help him hunt pigs at night.}

\newpage

\ex {\itshape
Manji aweta mï i.}\\
\gll ma-nji      aweta  mï      i\\
3\textsc{sg.obj-poss}  friend  3\textsc{sg.subj}  go.\textsc{pfv}\\
\glt ‘But then his friend came.’

\ex {\itshape
Mangop ana mangop.}\\
\gll ma=ango-p    ana      ma=ango-p\\
3\textsc{sg.obj}=\textsc{neg-pfv}  grass.skirt  3\textsc{sg.obj}=pull.out-\textsc{pfv}\\
\glt ‘[And the stranger] lied to him, tricked him:’\footnote{The verbs here are difficult to parse. The verb ‘lie’ may be a \isi{verbalized} form of the \isi{negator} \textit{ango} ‘\textsc{neg}’. Alternatively, there may be an \isi{idiom} ‘pull off one’s grass skirt’ meaning ‘trick’, containing the component \textit{ango-} ‘pull out’.}

\ex {\itshape
“U ma ila we apïn lumope namndu kotïn!”}\\
\gll u    ma  ila    we  apïn  lumo-p-e    namndu  ko=tï-n\\
2\textsc{sg}  go  morota  cut  fire    put-\textsc{pfv-dep}  pig      \textsc{indf}=take-\textsc{imp}\\
\glt ‘“Go and cut sago palm fronds, put them on the fire, and kill a pig!”’\footnote{The stranger’s friend has come in order to learn how the stranger has been so successful in hunting pigs. The stranger tricks him, however, in that he tells him to make a regular torch from sago palm fronds, not revealing anything of the incandescent vulva.}

\ex {\itshape
Mï i ila we apïn up namndu tïke mase mï nip.}\\
\gll mï      i    ila    we  apïn  u-p      namndu  tïke ma=asa-e      mï      ni-p\\
3\textsc{sg.subj}  go.\textsc{pfv}  morota  cut  fire    put-\textsc{pfv}  pig      small 3\textsc{sg.obj}=hit-\textsc{dep}  3\textsc{sg.subj}  die-\textsc{pfv}\\
\glt ‘He went and cut sago palm fronds, put them on the fire, hit a small pig, and it died.’

\ex {\itshape
Mï numbu mane mï i mankape.}\\
\gll mï      numbu    ma=ni-e      mï    i ma=nïkï-p-e\\
3\textsc{sg.subj}  garamut  3\textsc{sg.obj}=beat-\textsc{dep}  3\textsc{sg}  go.\textsc{pfv} 3\textsc{sg.obj}=dig{}-\textsc{pfv-dep}\\
\glt ‘He beat the \textit{garamut} drum until he came, and [they] butchered it.’\footnote{Having successfully killed a small pig, the friend summons the stranger by beating the \textit{garamut} drum, so that the two may work together to butcher the pig.}

\ex {\itshape
Mat: “Mawnam.”}\\
\gll ma=ta      mawnam\\
3\textsc{sg.obj}=say  thats.it\\
\glt ‘[And the stranger] said: “That’s it.”’

\newpage

\ex {\itshape
Mï nay awlu ambi mo ma awi we ambi mï keka mat nin ndïl.}\\
\gll mï       na-i      awlu  ambi  ma=u      ma      awi i     we    ambi  mï       keka      ma=tï     nin ndï=lï\\
3\textsc{sg.subj}  \textsc{detr-}go.\textsc{pfv}  step  big    3\textsc{sg.obj}=from  3\textsc{sg.obj}  shoulder go.\textsc{pfv}  then  big    3\textsc{sg.subj}  completely  3\textsc{sg.obj}=take  thorn 3\textsc{pl}=put\\
\glt ‘He went and stepped onto the shoulder of a big one [= a pig],\footnote{The friend goes out again, this time trying to kill a larger pig.} but then the big one completely got him and pushed [him] into some thorns.’

\ex {\itshape
Ka atay nipe.}\\
\gll ka    ata  i    ni-p-e\\
thus  up  go.\textsc{pfv}  die-\textsc{pfv-dep}\\
\glt ‘[He] went up like that and [nearly] died.’\footnote{This second pig is too much for the hunter to handle. While he is climbing onto the pig’s shoulders to attack it, the pig bucks, pushing the hunter into a thorny tree, injuring him severely.}

\ex {\itshape
Mï wa i tawa ndul mawap.}\\
\gll mï      wa    i    tawa  ndï=ul    ma=wap\\
3\textsc{sg.subj}  village  go.\textsc{pfv}  wound  3\textsc{pl}=with  3\textsc{sg.obj}=be.\textsc{pst}\\
\glt ‘He went home and stayed there with his wounds.’

\ex {\itshape
I mangani wonp.}\\
\gll i    ma=angani    won-p\\
go.\textsc{pfv}  3\textsc{sg.obj}=behind  cut-\textsc{pfv}\\
\glt ‘[He] went behind his back.’\footnote{Literally ‘cut behind him’. The injured friend goes to the home of the stranger without him knowing.}

\ex {\itshape
Njin iwïl mase mï keka i atay anam i.}\\
\gll nji=n    iwïl  ma=asa-e      mï      keka      i ata-i    anam  i\\
thing=\textsc{obl}  moon  3\textsc{sg.obj}=hit-\textsc{dep}  3\textsc{sg.subj}  completely  go.\textsc{pfv} up-go.\textsc{pfv}  sky    go.\textsc{pfv}\\
\glt ‘[He] hit the moon with something and it went completely up, went to the sky.’\footnote{While spying, the injured friend spots the magical vulva, here referred to euphemistically (or with foreshadowing) as the ‘moon’. He somehow disrupts it and it flies up into the sky.}

\ex {\itshape
Anam maye mï anmbi.}\\
\gll anam  ma=i-e          mï      an-mbï-i\\
sky    3\textsc{sg.obj}=go.\textsc{pfv-dep}  3\textsc{sg.subj}  out-here-go.\textsc{pfv}\\
\glt ‘When [it] went to the sky, he [= the stranger] came out.’

\ex {\itshape
“Nïnji aweta nda nangani wonp!”}\\
\gll nï-nji    aweta  anda    nï=angani    won-p\\
1\textsc{sg-poss}  friend  \textsc{sg.dist}  1\textsc{sg}=behind  cut-\textsc{pfv}\\
\glt ‘“That friend of mine has gone behind my back!”’

\ex {\itshape
Mï tamben mayte i atay.}\\
\gll mï      tamben  ma=ita-e        i    ata-i\\
3\textsc{sg.subj}  ladder  3\textsc{sg.obj}=build-\textsc{dep}  go.\textsc{pfv}  up-go.\textsc{pfv}\\
\glt ‘He [= the stranger] went and built a ladder\footnote{A \textit{tamben} ‘ladder’ is very tall, used for climbing trees, as opposed to a \textit{wat} ‘ladder’, which is shorter and leads up to the entrance of a house.} and climbed up.’

\ex {\itshape
I si membamlïpe ato ul ka.}\\
\gll i    si    ma=imbam-lï-p-e        ata-u    ul    ka\\
hand  push  3\textsc{sg.obj}=under-put-\textsc{pfv-dep}  up-from  with  let\\
\glt ‘[He] put his hand under it [= the moon] and hung [onto it].’

\ex {\itshape
Ato ul ke nïplopa ngala i.}\\
\gll ata-u    ul    ka-e  nïplopa  ngala    i\\
up-from  with  let-\textsc{dep}  flying.fox  \textsc{pl.prox}  go.\textsc{pfv}\\
\glt ‘As [he] hung, some flying foxes came.’

\ex {\itshape
Wapan masine i.}\\
\gll wapa=n  ma=si-ni-e          i\\
wing=\textsc{obl}  3\textsc{sg.obj}=push-beat-\textsc{dep}  go.\textsc{pfv}\\
\glt ‘[They] came and played with him with [their] wings.’

\ex {\itshape
Wawana mu kot manane.}\\
\gll wawana  mu    ko=tï  ma=na-n-e\\
plant.species  fruit  \textsc{indf}=take  3\textsc{sg.obj}=give-\textsc{pfv-dep}\\
\glt ‘[They] gave him a \textit{wawana} fruit.’\footnote{The flying foxes are curious about this new creature (a man) that has come to join their realm in the sky. They give him a \textit{wawana} fruit, which is something a flying fox (but not a human) would typically eat.}

\ex {\itshape
Mï man ndït: “Ango mundu kom un mat nïnan!”}\\
\gll mï      ma=n      ndï=ta    ango  mundu  kom  un  ma=tï nï=na-n\\
3\textsc{sg.subj}  3\textsc{sg.obj=obl}  3\textsc{pl}=say  \textsc{neg}  food  \textsc{neg}  2\textsc{pl}  3\textsc{sg.obj}=take 1\textsc{sg}=give-\textsc{pfv}\\
\glt ‘But he told them: “That’s not food you gave me!”’

\ex {\itshape
Ndï i ma nan amblakap.}\\
\gll ndï  i    ma      na=n    ambla=kï-p\\
3\textsc{pl}  go.\textsc{pfv}  3\textsc{sg.obj}  talk=\textsc{obl}  \textsc{pl.refl}=say-\textsc{pfv}\\
\glt ‘They went and talked about him.’\footnote{When the man refuses the \textit{wawana} fruit as something inedible, the flying foxes become suspicious and are wary about having him around.}

\ex {\itshape
Wop wolka i umbenam i.}\\
\gll wo-p    wolka  i    umbenam  i\\
sleep-\textsc{pfv}  again  go.\textsc{pfv}  morning  go.\textsc{pfv}\\
\glt ‘The next day, [they] came again, came in the morning.’

\ex {\itshape
Ato mawlop.}\\
\gll ata-u    ma=u-lo-p\\
up-from  3\textsc{sg.obj}=from-go-\textsc{pfv}\\
\glt ‘[They] grabbed onto him.’

\ex {\itshape
Iwïl membam u motop anmbïlïp.}\\
\gll iwïl  ma=imbam    u    ma=top      an-mbï-lï-p\\
moon  3\textsc{sg.obj}=under  from  3\textsc{sg.obj}=throw  out-here-put-\textsc{pfv}\\
\glt ‘[And they] threw him out from under the moon.’

\ex {\itshape
Em Amblom manji mïnam.}\\
\gll em    Amblom  ma-nji    mï-nam\\
3\textsc{sg}  [name]    3\textsc{sg.obj-poss}  3\textsc{sg.subj-int}\\
\glt ‘That’s it;\footnote{The \isi{pronoun} \textit{em} ‘3\textsc{sg}’, used here as an \isi{interjection}, is a \isi{loan} from \ili{Tok Pisin}.} that’s Amblom’s [story].’
\z

\section{\label{sec:16.3}  {\textit{Anmoka}} {(‘Snakes’)}}

This is a description of a traditional cultural practice, as told by Tangin Kapos on 1 {June 2017}, at her home in Manu village. This text is part of a larger conversation between Tangin Kapos and Gweni Tungun. Examples from this text that appear elsewhere in this book are labeled “ulwa035\_mm:ss”. The audio recording can be found on the ELAR website (file name: ulwa035.wav). The entire recording is almost six minutes long (05:51). The following text, however, represents about the first minute (00:57) of the recording; in the rest of the recording (not transcribed here), the speaker recounts a crocodile hunt.

  In this text, Tangin describes a traditional method of inducing labor, which would be used when a husband suspected that his wife was overdo in carrying their child. The husband would kill a snake and wrap its body in a banana leaf, as if it were cooked food. He would then give this package to his wife, who, thinking it was food, would unwrap it, see the snake, and get a shock, which -- it was believed -- would induce her to bear the child on that very night.

\setcounter{equation}{0}
\ea {\itshape
Anmoka stori.}\\
\gll anmoka  stori\\
snake    story\\
\glt ‘A snake story.’\footnote{\textit{Stori} ‘story’ is from \ili{Tok Pisin}.}

\ex {\itshape
Nambi sawe anmoka ala namnapen.}\\
\gll nï-ambi  sawe  anmoka  ala  namna=p-en\\
1\textsc{sg-top}  \textsc{hab}  snake    from  afraid=\textsc{cop}{}-\textsc{nmlz}\\
\glt ‘As for me, I’m afraid of snakes.’\footnote{The \isi{habitual} marker \textit{sawe} ‘\textsc{hab}’ is \isi{borrow}ed from the \ili{Tok Pisin} \isi{habitual} marker \textit{save} (literally ‘know’).}

\ex {\itshape
Nï wandam mata ankam anmoka matïm mapta nï mandï namnap unip.}\\
\gll nï    wandam  ma-ta    ankam  anmoka  ma=atï-m ma=p-ta      nï    ma=andï    namna=p    uni-p\\
1\textsc{sg}  jungle    go-\textsc{cond}  person  snake    3\textsc{sg.obj}=hit-\textsc{irr} \textsc{3sg.obj=}be\textsc{{}-cond} 1\textsc{sg}  3\textsc{sg.obj}=from  afraid=\textsc{cop}  shout-\textsc{pfv}\\
\glt ‘Whenever I go to the jungle and people kill a snake there, I shout in fear about it.’

\ex {\itshape
Wopa ndawa u mana mane.}\\
\gll wopa  anda-awa    u    ma-na  ma-n-e\\
all    \textsc{sg.dist-int}  from  go-\textsc{irr}  go-\textsc{ipfv-dep}\\
\glt ‘[I] am going to go far from there.’

\ex {\itshape
Ango anmoka ndala nambï nïpatpe.}\\
\gll ango  anmoka  ndï=ala  nambï  nïpat=p-e\\
\textsc{neg}  snake    3\textsc{pl}=for  skin  giant=\textsc{cop}{}-\textsc{dep}\\
\glt ‘[I] don’t have thick skin for snakes.’\footnote{Literally ‘not for the snakes is skin giant’ (i.e., the speaker is easily frightened by snakes).}

\newpage

\ex {\itshape
Anmoka ndï ala ipka inom ala nambï kenmbupe itom ala ndïwale ndïn muku ite ndït wa unde ndïmune ndïwat awe.}\\
\gll anmoka  ndï  ala      ipka  inom  ala      nambï  kenmbu=p-e itom  ala      ndï=wali-e    ndï=n    muku    ita-e ndï=tï    wa    unda-e  ndï=mune  ndï=wat  aw-e\\
snake    3\textsc{pl}  \textsc{pl.dist}  before  mother  \textsc{pl.dist}  skin  heavy=\textsc{cop}{}-\textsc{dep} father  \textsc{pl.dist}  3\textsc{pl}=hit-\textsc{dep}  3\textsc{pl=obl}  package  tie-\textsc{dep} 3\textsc{pl}=take  village  go-\textsc{dep}  2\textsc{pl}=throw  3\textsc{pl=}atop  put.\textsc{ipfv-dep}\\
\glt ‘Snakes\footnote{After the preceding prologue about the speaker’s fear of snakes, she begins now to describe a traditional means of inducing labor that relies on a woman’s fear of snakes.} -- people in the past, when mothers were pregnant,\footnote{Literally ‘bodies (have) heaviness’.} the fathers used to kill them [= snakes], tie them up into packages [with leaves], bring them home, and toss them [= the wrapped snakes] to them [= the pregnant women].’\footnote{That is, the husband would toss the leaf-wrapped snake to his wife as if it were food.}

\ex {\itshape
“U alum man nambï ka wap ngayap.”}\\
\gll u    alum  ma=n      nambï  ka  wap  ngaya=p\\
2\textsc{sg}  child  3\textsc{sg.obj=obl}  body  on  be.\textsc{pst}  far=\textsc{cop}\\
\glt ‘“You’ve been with a child on your body for [too] long.”’\footnote{This is what a husband might say to his presumed overdo wife.}

\ex {\itshape
“Wap ngayape oke.”}\\
\gll wap  ngaya=p-e    oke\\
be.\textsc{pst}  far=\textsc{cop}{}-\textsc{dep}  ok\\
\glt ‘“[You] have been [that way] for [too] long, OK.”’\footnote{The \isi{interjection} \textit{oke} ‘OK’ is \isi{borrow}ed from \ili{Tok Pisin} \textit{oke} ‘OK’.}

\ex {\itshape
Manap anmoka matïm map man muku itap matï ma angop mundu tï mawatlïp mat mananda.}\\
\gll ma=nap    anmoka  ma=atï-m      ma=p      ma=n muku    ita-p  ma=tï      ma  ango=p  mundu  tï ma=wat-lï-p        ma=tï    ma=na-nda\\
3\textsc{sg.obj=}for  snake    3\textsc{sg.obj}=hit-\textsc{irr}  \textsc{3sg.obj=}be  3\textsc{sg.obj=obl} package  tie-\textsc{pfv}  3\textsc{sg.obj}=take  go  \textsc{neg=cop}  food  take 3\textsc{sg.obj}=atop-put-\textsc{pfv}  3\textsc{sg.obj}=take  3\textsc{sg.obj}=give-\textsc{irr}\\
\glt ‘Having killed a snake for her there, made a package with it, and brought it [home], [the husband], pretending that it was food, would give it to her.’\footnote{Literally ‘in a false way taking [the name] “food” and putting [it] on it [= the snake]’.}

\ex {\itshape
Mat manata mï makanakawmop lïmndï mandï mandï unipïna: “Yi!”}\\
\gll ma=tï      ma=na-ta        mï      ma=kanaka-lumo-p lïmndï  ma=andï    ma=andï    uni=p-na      yi\\
3\textsc{sg.obj}=take  3\textsc{sg.obj}=give-\textsc{cond}  3\textsc{sg.subj}  3\textsc{sg.obj}=unwrap-put-\textsc{pfv} eye    3\textsc{sg.obj}=see  3\textsc{sg.obj}=for  shout=\textsc{cop}{}-\textsc{irr}  \textsc{interj}\\
\glt ‘When [he] has given it to her, she would unwrap it, see it, and shout\footnote{The form \textit{unipïna} ‘shout [\textsc{irr]}’ appears to be an alternate form of \textit{uninda} ‘shout [\textsc{irr]}’, seemingly formed with the \isi{copular enclitic} plus the \isi{irrealis} \isi{suffix}.} about it: “Eek!”’

\ex {\itshape
Mala namnap unipïna.}\\
\gll ma=ala    namna=p    uni=p-na\\
3\textsc{sg.obj=}from  afraid=\textsc{cop}  shout=\textsc{cop}{}-\textsc{irr}\\
\glt ‘[She] would shout in fear of it.’

\ex {\itshape
Ta manji alum mï tïnangata mï mokotnda.}\\
\gll ta    ma-nji      alum  mï      tïnanga-ta     mï ma=kot-nda\\
already  3\textsc{sg.obj-poss}  child  3\textsc{sg.subj}  arise-\textsc{cond}  3\textsc{sg.subj} 3\textsc{sg.obj}=break-\textsc{irr}\\
\glt ‘Immediately, her baby would get up, and she would bear\footnote{The form \textit{kotnda} ‘break [\textsc{irr}]’ appears to be an alternate form of \textit{kotïna} ‘break [\textsc{irr}]’, exhibiting the \isi{allomorph} [-nda] of the \isi{irrealis} \isi{suffix} \textit{-na} ‘\textsc{irr}’].} it.’

\ex {\itshape
Olsem mï amun imba pïta mï mandï unipta mï imba pïta alum mï tïnangana.}\\
\gll olsem  mï      amun  imba  p-ta    mï      ma=andï uni-p-ta      mï      imba  p-ta    alum  mï tïnanga-na\\
thus  3\textsc{sg.subj}  now  night  be-\textsc{cond}  3\textsc{sg.subj}  3\textsc{sg.obj}=for shout-\textsc{pfv-cond}  3\textsc{sg.subj}  night  be-\textsc{cond}  child  3\textsc{sg.subj} arise-\textsc{irr}\\
\glt ‘And so,\footnote{The connector word \textit{olsem} ‘thus’ is from \ili{Tok Pisin}.} it -- that night, when she shouted about it, it -- that night -- the baby would get up.’

\ex {\itshape
Men u tïnangata mï imba pïta mokotnda.}\\
\gll ma=in      u    tïnanga-ta    mï      imba  p-ta ma=kot-nda\\
3\textsc{sg.obj}=in  from  arise-\textsc{cond}  3\textsc{sg.subj}  night  be-\textsc{cond} 3\textsc{sg.obj}=break-\textsc{irr}\\
\glt ‘When [the baby] would get up inside her, she would bear it that night.’

\z
