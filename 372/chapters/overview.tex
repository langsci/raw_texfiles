\chapter{A grammatical overview of Ulwa}\label{sec:overview}

This chapter provides an overview of the grammar of Ulwa. It is intended to help readers understand the descriptions and glossed examples presented elsewhere. The examples in this chapter have either been taken from elicited responses or have been adapted from other examples in this book, designed to be presented as simply as possible.

\section{Phonology}\label{sec:overview:2.1}

\is{phonology|(}

\tabref{tab:overview:1} presents the 13 \isi{consonant}s of Ulwa. \tabref{tab:overview:2} presents the 6 \isi{vowel}s of Ulwa.

\is{consonant}
\begin{table}
\caption{Ulwa consonants}
\is{labial}
\is{alveolar}
\is{palatal}
\is{velar}
\label{tab:overview:1}
\begin{tabular}{lllll}
\lsptoprule
 & labial & alveolar & palatal & velar\\
\midrule
\isi{voiceless} \isi{stop}s & p & t &  & k\\
\isi{prenasalized} \isi{voiced} \isi{stop}s & mb & nd &  & ng\\
\isi{prenasalized} \isi{voiced} \isi{affricate} &  &  & nj & \\
nasals & m & n &  & \\
\isi{liquid} &  & l &  & \\
\isi{fricative} &  & s &  & \\
\isi{glide}s & w &  & y & \\
\lspbottomrule
\end{tabular}
\end{table}

\is{vowel}
\is{front vowel}
\is{central vowel}
\is{back vowel}
\is{high vowel}
\is{mid vowel}
\is{low vowel}
\begin{table}
\caption{Ulwa vowels}
\label{tab:overview:2}

\begin{tabular}{llll}
\lsptoprule
& front & central & back\\

\midrule
high & i & ï & u\\
mid & e &  & o\\
low &  & a & \\
\lspbottomrule
\end{tabular}
\end{table}



In addition to these 19 basic phonemes, there is a \isi{rhotic} \isi{consonant} [r], which is the preferred pronunciation in some \isi{proper noun}s and is thus written as <r>. Elsewhere [r] is an \isi{allophone} of /l/. There is also a very infrequent \is{low vowel} low \isi{front vowel} [æ], which is not considered here to be phonemic. It is written as <ae> in the four words known to exhibit it. There are two basic \isi{diphthong}s, /aw/ and /ay/, and potentially also /oy/.


All \isi{consonant}s may occur word-initially and word-medially. \isi{Voiced} \isi{stop}s  /mb, nd, ng/, the \isi{voiced} \isi{affricate} /nj/, and the \isi{voiceless} \isi{velar} \isi{stop} /k/ do not occur word-finally. There is no \isi{velar} \isi{nasal} phoneme, but \isi{alveolar} /n/ \is{assimilation} assimilates in place to [ŋ] when immediately preceding the \isi{velar} \isi{stop} /k/. The sole \isi{fricative} /s/ may optionally be \isi{palatalized} to [ʃ] when immediately preceding the \is{high vowel} high \isi{front vowel} /i/.


All \isi{vowel}s may occur word-medially and word-finally, but only /a, i, u/ may occur word-initially. Speakers often insert optional \isi{glide}s [y] and [w] before word-initial /i/ and /u/, respectively. The contrast between /a/ and /o/ is often neutralized in the environment immediately preceding /w/, where they may both be pronounced as [ɔ]. The \is{high vowel} high \isi{central vowel} /ï/ is often \isi{elide}d.


\isi{Consonant cluster}s are relatively uncommon, but they do occur. \isi{Complex onset}s are more common than \isi{complex coda}s. The \isi{sonorant}s /m, n, l/ can all behave as \isi{syllabic consonant}s, generally only when a preceding /ï/ has been \isi{elide}d. \isi{Stress} is not phonemic. There is a preference for \isi{penultimate} \isi{stress} in \isi{polysyllabic} words and for alternating \isi{stress} in longer utterances.

\is{phonology|)}

\section{Morphophonology}\label{sec:overview:2.2}

\is{morphophonology|(}
\is{phonology|(}

There are several \isi{morphophonemic process}es that result in different surface forms. The \isi{high vowel}s /i/ and /u/ become \isi{glide}s when immediately following the \is{low vowel} low \isi{central vowel} /a/ \REF{ex:overview:1}.

\ea%1
    \label{ex:overview:1}
    \ea   /maita/    → [mayta]      ‘build it’\\
    \ex  /mauta/  → [mawta]    ‘grind it’
    \z
\z

The sequences [ay] and [aw] may optionally \isi{monophthongize} to [e] and [o], respectively \REF{ex:overview:2}.

\ea%2
    \label{ex:overview:2}
    \ea  /nai/    → [nay]    (→ [ne])    ‘went away’

    \ex  /maul/    → [mawl]    (→ [mol])    ‘with it’
    \z
\z

When the \is{low vowel} low \isi{central vowel} /a/ is immediately followed by a non-\isi{high vowel}, however, /a/ is \isi{deleted} \REF{ex:overview:3}.

\ea\label{ex:overview:3}
  \ea  /wanaen/  → [wanen]    ‘cook’

  \ex  /andao/    → [ando]      ‘that!’

  \ex  /maasa/  → [masa]      ‘hit it’
  \z
\z

The \is{high vowel} high \isi{central vowel} /ï/ is \isi{deleted} when immediately followed by any \isi{vowel} \REF{ex:overview:4}.

\ea\label{ex:overview:4}
    \ea  /ndïin/    → [ndin]      ‘in them’

    \ex  /ndïul/    → [ndul]      ‘with them’

    \ex  /lïe/    → [le]        ‘put’

    \ex  /nïasa/    → [nasa]      ‘hit me’
    \z
\z

All \isi{vowel}s are \isi{deleted} before an immediately following \is{mid vowel} mid \isi{front vowel} /e/ \REF{ex:overview:5}.

\ea%5
    \label{ex:overview:5}
    \ea  /nie/      → [ne]      ‘act’

    \ex  /mokoe/    → [moke]    ‘take’

    \ex  /alee/      → [ale]    ‘scrape’

    \ex  /itaen/      → [iten]    ‘builder’
    \z
\z

The \is{high vowel} high \isi{back vowel} /u/ becomes a \isi{glide} [w] when immediately before a \isi{vowel} occurring in the same \isi{syllable} \REF{ex:overview:6}.

\ea%6
    \label{ex:overview:6}
    \ea  /uasa/      → [wasa]    ‘hit you’

    \ex  /uama/      → [wama]    ‘eat you’
    \z
\z

\is{high vowel}
\is{assimilation}
\is{high vowel}
\is{back vowel}

The high \isi{central vowel} /ï/ optionally assimilates in place and \isi{rounding} to the high back rounded \isi{vowel} [u] when immediately preceding the \isi{glide} /w/ \REF{ex:overview:7}.

\ea%7
    \label{ex:overview:7}
    \ea  /ndïwana/    (→ [nduwana])  ‘cook them’

    \ex  /nïwali/    (→ [nuwali])    ‘hit me’
    \z
\z

The \is{low vowel} low \isi{central vowel} /a/ harmonizes to the \is{mid vowel} mid \isi{back vowel} [o] when immediately following \isi{labial} /m/ and preceding /o/ in the subsequent \isi{syllable} \REF{ex:overview:8}.

\ea%8
    \label{ex:overview:8}
    \ea  /mako/      → [moko]      ‘break it’

    \ex  /matoplï/    → [motoplï]    ‘throw it’
    \z
\z

\newpage

\isi{Geminate} \isi{consonant}s are reduced to single segments \REF{ex:overview:9}.

\ea%9
    \label{ex:overview:9}
    \ea  /tïnn/      → [tïn]      ‘with the dog’

    \ex  /tumullakana/  → [tumulakana]  ‘will bend’
    \z
\z

There is also a sort of \isi{quasi-degemination}, whereby a \isi{nasal} \isi{consonant} is \isi{elide}d when immediately preceding a \is{homorganic consonant} homorganic \isi{prenasalized} \isi{voiced} \isi{stop} \REF{ex:overview:10}.

\ea%10
    \label{ex:overview:10}
    \ea  /annji/      → [anji]      ‘our’

    \ex   /kunnda/    → [kunda]      ‘will break’
    \z
\z

There are \isi{morphophonological} alternations that either are fairly restricted in their environments or are \isi{morphological}ly conditioned. For example, in \isi{perfective} verbs whose \isi{stem}s end in /kï/, the final /ï/ is lowered to [a] \REF{ex:overview:11}.

\ea%11
    \label{ex:overview:11}
    \ea  /kïp/      → [kap]      ‘said’

    \ex  /nïkïp/      → [nïkap]      ‘dug’
    \z
\z

In \isi{irrealis} verbs whose \isi{stem}s end in /le/ or /lo/, the final \isi{vowel} is raised to [i] or [u], respectively \REF{ex:overview:12}.

\ea%12
    \label{ex:overview:12}
    \ea  /alenda/    → [alinda]      ‘will scrape’

    \ex  /londa/      → [lunda]      ‘will cut’
    \z
\z

Finally, there are a number of \isi{lexical}ly determined \isi{phonological} alternations. Notably, in the verb \textit{lï-} ‘put’, the \is{high vowel} high \isi{central vowel} [ï] is often \isi{elide}d, sometimes obligatorily, sometimes only optionally \REF{ex:overview:13}.

\is{phonology|)}
\is{morphophonology|)}

\ea%13
    \label{ex:overview:13}
    \ea  /ndïlïp/      → [ndïlp]      ‘put them’

    \ex  /malïnda/    (→ [malnda])    ‘will put it’
    \z
\z

\section{Nouns and noun phrases}\label{sec:overview:2.3}

\is{noun phrase|(}

There is no designated nominal \isi{morphology}. Nouns, however, may host the \isi{oblique} \isi{enclitic} \textit{=n} ‘\textsc{obl}’, which signals that a \isi{noun phrase} is functioning in some role other than that of subject or object \REF{ex:overview:14}. The \isi{oblique} \isi{enclitic} \textit{=n} ‘\textsc{obl}’ has the \isi{allomorph}s \textit{=ïn} ‘\textsc{obl}’ and \textit{=nï} ‘\textsc{obl}’.

\ea%14
    \label{ex:overview:14}
    \textit{mundun}\\
\gll    mundu=\textbf{n}\\
    food=\textsc{obl}      \\
\glt    ‘with food’
\z

Nouns may be \isi{derived} from verbs by means of the \isi{nominalizer} \isi{suffix} \textit{{}-en} ‘\textsc{nmlz’} \REF{ex:overview:15}. The \isi{nominalizer} \isi{suffix} \textit{{}-en} ‘\textsc{nmlz’} has the \isi{allomorph}s \textit{{}-n} ‘\textsc{nmlz}’, \textit{{}-wen} ‘\textsc{nmlz’}, and \textit{{}-yen} ‘\textsc{nmlz’.}

\ea%15
    \label{ex:overview:15}
    \textit{iyen}\\
\gll    i-\textbf{en}\\
    go-\textsc{nmlz}\\
\glt    ‘the one going’
\z

  \isi{Noun phrase}s are built around either a noun or a \isi{pronoun}. If there is a \isi{possessor} argument, it precedes the \isi{possessum} \REF{ex:overview:16}.

\ea%16
    \label{ex:overview:16}
    \textbf{\textit{manji}} \textit{yana}\\
\gll    \textbf{ma-nji}      yana\\
    3\textsc{sg.obj-poss}  woman\\
\glt    ‘his wife’
\z

\isi{Adnominal adjective}s generally follow \isi{head noun}s \REF{ex:overview:17}.

\ea%17
    \label{ex:overview:17}
    \textit{tïn} \textbf{\textit{njukuta}}\\
\gll tïn    \textbf{njukuta}\\
    dog  small\\
\glt    ‘small dog’
\z

Adnominal \isi{numeral}s also follow nouns (and any \isi{adjective}s, if present) \REF{ex:overview:18}.

\ea%18
    \label{ex:overview:18}
    \textit{mïnda} \textbf{\textit{lele}}\\
\gll    mïnda    \textbf{lele}\\
    banana    three\\
\glt    ‘three bananas’
\z

\isi{Determiner}s, when they occur in \isi{noun phrase}s, are usually the final element of the NP \REF{ex:overview:19}.

\ea%19
    \label{ex:overview:19}
    \textit{ankam ambi} \textbf{\textit{anda}}\\
\gll    ankam  ambi  \textbf{anda}\\
    person  big    \textsc{sg.dist}\\
\glt    ‘that big person’
\z

The \isi{collective universal quantifier} \textit{wopa} ‘all’, however, follows \isi{determiner}s when it is used adnominally \REF{ex:overview:20}.

\is{universal quantifier}
\is{quantifier}

\ea%20
    \label{ex:overview:20}
    \textit{tawatïp ngala} \textbf{\textit{wopa}}\\
\gll    tawatïp    ngala    \textbf{wopa}\\
    child    \textsc{pl.prox}  all\\
\glt    ‘all these children’
\z

The \isi{distributive universal quantifier} \textit{nunu} ‘every’, on the other hand, precedes the noun \REF{ex:overview:21}.

\is{universal quantifier}
\is{quantifier}
\is{noun phrase|)}

\ea%21
    \label{ex:overview:21}
    \textbf{\textit{nunu}} \textit{wombïn tembi}\\
\gll    \textbf{nunu}  wombïn  tembi\\
    every  work    bad\\
\glt    ‘every bad job’
\z

\section{Verbs and verb phrases}\label{sec:overview:2.4}

\is{verb phrase|(}

Verbs typically receive one of three basic \isi{TAM} \isi{suffix}es \REF{ex:overview:22}.

\ea%22
    \label{ex:overview:22}
Basic \isi{TAM} \isi{suffix}es\\
\begin{tabbing}
{(\textit{-na})} \= {(\isi{imperfective} (‘\textsc{ipfv}’))}\kill
{\textit{-e}} \> {\isi{imperfective} (‘\textsc{ipfv}’)}\\
{\textit{-p}} \> {\isi{perfective} (‘\textsc{pfv}’)}\\
{\textit{-na}} \> {\isi{irrealis} (‘\textsc{irr}’)}
\end{tabbing}
\z

\isi{Imperfective} verbs are used for events and states that are viewed as incomplete or ongoing \REF{ex:overview:23}.

\ea%23
    \label{ex:overview:23}
    \textit{Nï amun we} \textbf{\textit{ame}}.\\
\gll    nï    amun  we    ama-\textbf{e}\\
    1\textsc{sg}  now  sago  eat-\textsc{ipfv}\\
\glt    ‘I am eating sago now.’
\z

\isi{Imperfective} \isi{morphology} thus signals \isi{continuous}, \isi{habitual}, or \isi{iterative} happenings or states. In addition to the \isi{imperfective} \isi{suffix} \textit{{}-e} ‘\textsc{ipfv}’, which has the \isi{phonological}ly conditioned \isi{allomorph} \textit{{}-ye} ‘\textsc{ipfv}’, there is an irregular \isi{imperfective} form \textit{{}-n} ‘\textsc{ipfv}’.

\isi{Perfective} verbs, on the other hand, are used for events and states that have reached their end \REF{ex:overview:24}.

\ea%24
    \label{ex:overview:24}
    \textit{Itom mï lamndu} \textbf{\textit{masap}}.\\
\gll    itom  mï      lamndu  ma=asa-\textbf{p}\\
    father  3\textsc{sg.subj}  pig      3\textsc{sg.obj}=hit-\textsc{pfv}\\
\glt    ‘Father killed the pig.’
\z

\isi{Perfective} \isi{morphology} thus often signals \isi{past} \isi{time}, although it may also occur with \isi{present}-\isi{time} reference. In addition to the \isi{perfective} \isi{suffix} \textit{{}-p} ‘\textsc{pfv}’, which has \isi{phonological}ly conditioned \isi{allomorph}s \textit{{}-ap} ‘\textsc{pfv}’, \textit{{}-ïp} ‘\textsc{pfv}’, and \textit{{}-op} ‘\textsc{pfv}’, there are irregular \isi{perfective} forms \textit{{}-al} ‘\textsc{pfv}’, \textit{{}-m} ‘\textsc{pfv}’, and \textit{{}-n} ‘\textsc{pfv}’.

\isi{Irrealis} verbs are used for events or states that are not known to the speaker to have happened \REF{ex:overview:25}.

\ea%25
    \label{ex:overview:25}
    \textit{Itom mï apa} \textbf{\textit{itana}}.\\
\gll    itom  mï      apa    ita-\textbf{na}\\
    father  3\textsc{sg.subj}  house  build-\textsc{irr}\\
\glt    ‘Father can build house.’
\z

Although all \isi{future} events are encoded with \isi{irrealis} verbs, \isi{irrealis} \isi{morphology} can be applied to any \isi{time} frame, provided the verb is referring to something unreal or \isi{hypothetical}. The \isi{irrealis} \isi{suffix} \textit{{}-na} ‘\textsc{irr}’ has the \isi{allomorph} \textit{{}-nda} ‘\textsc{irr}’, which occurs when the preceding \isi{consonant} is a \isi{sonorant}. There is also an irregular \isi{irrealis} form \textit{{}-m} ‘\textsc{irr}’, as well as a \isi{prefix}-like element \textit{la-} ‘\textsc{irr}’ or \textit{lo-} ‘\textsc{irr}’ or \textit{l-} ‘\textsc{irr}’ that occurs in some \isi{irrealis} forms along with the \isi{suffix}ing \isi{morphology}, thereby creating a sort of \isi{circumfix}.

\isi{Imperative}s are formed with the verbal \isi{suffix} \textit{{}-n} ‘\textsc{imp’}, which follows the \isi{verb stem} \REF{ex:overview:26}.

\ea%26
    \label{ex:overview:26}
    \textit{U} \textbf{\textit{man!}}\\
\gll    u    ma-\textbf{n}\\
    2\textsc{sg}  go-\textsc{imp}\\
\glt    ‘Go!\textsc{’}
\z

\isi{Conditional} statements are formed with the verbal \isi{suffix} \textit{{}-ta} ‘\textsc{cond’}, which occurs on the verb in the \isi{protasis}, and which follows a \isi{TAM} \isi{suffix}, if present \REF{ex:overview:27}.

\ea%27
    \label{ex:overview:27}
    \textit{Sinda nji} \textbf{\textit{wanapta}} \textit{…}\\
\gll    Sinda  nji    wana-p-\textbf{ta}\\
    [name]  thing  cook-\textsc{pfv-cond}\\
\glt    ‘If Sinda cooks something …\textsc{’}
\z

The \isi{speculative} \isi{suffix} \textit{{}-t} ‘\textsc{spec’} may immediately follows the \isi{irrealis} \isi{suffix} on the verb to convey a sense of \isi{epistemic} possibility \REF{ex:overview:28}.

\ea%28
    \label{ex:overview:28}
    \textit{Mï mol} \textbf{\textit{inat}}.\\
\gll    mï      ma=ul      i-na-\textbf{t}\\
    3\textsc{sg.subj}  3\textsc{sg.obj}=with  come-\textsc{irr-spec}\\
\glt    ‘He might come with her.’
\z

There is only one verbal \isi{prefix}, the \isi{detransitivizer} \textit{na-} ‘\textsc{detr’}, which may be used to reduce the \isi{transitivity} of a verb \REF{ex:overview:29}.

\ea%29
    \label{ex:overview:29}
    \textit{Nï ta} \textbf{\textit{namap}}.\\
\gll    nï    ta      \textbf{na}{}-ama-p\\
    1\textsc{sg}  already    \textsc{detr}{}-eat-\textsc{pfv}\\
\glt    ‘I’ve already eaten.’
\z

\isi{Transitive} \isi{verb phrase}s often contain \isi{object marker}s, which are \isi{pronoun}-like verbal \isi{proclitic}s that index whether an object argument is \isi{singular} \REF{ex:overview:30}, \isi{dual} \REF{ex:overview:31}, or \isi{plural} \REF{ex:overview:32}.

\ea%30
    \label{ex:overview:30}
    \textit{mïnda} \textbf{\textit{mame}}\\
\gll    mïnda    \textbf{ma}=ama-e\\
    banana    3\textsc{sg.obj}=eat-\textsc{ipfv}\\
\glt    ‘eating the banana’
\z

\ea%31
    \label{ex:overview:31}
    \textit{utam} \textbf{\textit{minwanap}}\\
\gll utam  \textbf{min}=wana-p\\
    yam  3\textsc{du}=cook-\textsc{pfv}\\
\glt    ‘cooked the (two) yams’
\z

\ea%32
    \label{ex:overview:32}
    \textit{lamndu} \textbf{\textit{ndasap}}\\
\gll    lamndu  \textbf{ndï}=asa-p\\
    pig      3\textsc{pl}=hit-\textsc{pfv}\\
\glt    ‘killed the pigs’
\z

The \isi{indefinite} marker \textit{ko=} ‘\textsc{indf’} is another \isi{proclitic} that may precede a \isi{verb stem}. It indicates that the object of the verb is \isi{indefinite} \REF{ex:overview:33}.

\ea%33
    \label{ex:overview:33}
    \textit{way nungol \textbf{kotïn}}\\
\gll    way  nungol  \textbf{ko}{=}tï{{}-}n\\
    turtle  child  \textsc{indf=}take-\textsc{pfv}\\
\glt    ‘caught a little tur{t}le’
\z

Some \isi{light verb}s co-occur with a preceding nominal element. These \isi{adjunct}-verb pairings may result in \isi{compound}-like structures. However, it is also possible for the \isi{nominal adjunct} to occur \isi{discontinuous}ly in the \isi{verb phrase} -- that is, with the object of the verb interceding \REF{ex:overview:34}.

\ea%34
    \label{ex:overview:34}
    \textbf{\textit{kïkal}} \textit{na} \textbf{\textit{mawana}}\\
\gll    \textbf{kïkal}  na    ma=\textbf{wana}\\
    ear    talk  3\textsc{sg.obj}=feel\\
\glt    ‘hear the message’
\z

\is{irregular verb}

These \isi{separable verb} forms occur commonly with \is{perception verb} verbs of perception. The irregular \isi{suppletive} verb \textit{ala-} {\textasciitilde} \textit{andï-} ‘see’, although itself glossed as ‘see’, must nevertheless co-occur with the nominal element \textit{lïmndï} ‘eye’ in order to complete its meaning \REF{ex:overview:35}.

\ea%35
    \label{ex:overview:35}
    \textbf{\textit{lïmndï}} \textit{apa} \textbf{\textit{mala}}\\
\gll    \textbf{lïmndï}  apa    ma=\textbf{ala}\\
    eye    house  3\textsc{sg.obj}=see\\
\glt    ‘see the house’
\z

Other \isi{light verb}s that are commonly used in \is{separable verb} separable construction are \textit{lï-} ‘put’, \textit{u-} ‘put’, and \textit{ka-} ‘let’.

\is{verb phrase|)}

\section{Adjectives}\label{sec:overview:2.5}

\is{adjective|(}

\isi{Attributive adjective}s follow the \isi{head noun} of the \isi{noun phrase} but precede any \isi{determiner}s \REF{ex:overview:36}.

\ea%36
    \label{ex:overview:36}
    \textit{ankam} \textbf{\textit{ambi}} \textit{mï}\\
\gll    ankam    \textbf{ambi}  mï\\
    person    big    3\textsc{sg.subj}\\
\glt    ‘the big person’
\z

\isi{Predicative adjective}s, on the other hand, exist \isi{syntactic}ally outside the \isi{noun phrase} -- that is, they follow all elements in the NP \REF{ex:overview:37}.

\ea%37
    \label{ex:overview:37}
    \textit{Ankam mï} \textbf{\textit{ambi}}.\\
\gll    ankam    mï      \textbf{ambi}\\
    person    3\textsc{sg.subj}  big\\
\glt    ‘The person is big.’
\z

\isi{Adjective}s are not \isi{syntactic}ally very different from nouns. Thus, property-denoting words can themselves serve as the \isi{head}s of \isi{noun phrase}s \REF{ex:overview:38}.

\is{adjective|)}

\ea%38
    \label{ex:overview:38}
    \textbf{\textit{Ambi}} \textit{mï nip.}\\
\gll    \textbf{ambi}  mï      ni-p\\
    big    3\textsc{sg.subj}  die-\textsc{pfv}\\
\glt    ‘The big one died.’
\z

\section{Pronouns}\label{sec:overview:2.6}

\is{pronoun|(}

\isi{Pronoun}s index first (‘1’), second (‘2’), and third (‘3’) \isi{person}, and \isi{singular} (‘\textsc{sg}’), \isi{dual} (‘\textsc{du}’), and \isi{plural} (‘\textsc{pl}’) \isi{number}. There is a distinction made between \isi{inclusive} (‘\textsc{incl}’) and \isi{exclusive} (‘\textsc{excl}’) \isi{non-singular} first \isi{person} pronouns. Subject (‘\textsc{subj}’) and \isi{non-subject} (‘\textsc{obj}’) pronouns are nearly identical in form. Only the 3\textsc{sg} forms are consistently differentiated. \isi{Reflexive} \isi{pronoun}s (‘\textsc{refl}’) encode \isi{number} but do not encode \isi{person}. \isi{Possessive pronoun}s are all derived from the set of \isi{non-subject} pronouns plus the element \textit{{}-nji} ‘\textsc{poss}’. The most common Ulwa \isi{pronoun}s are given in \tabref{tab:overview:3}.

\begin{table}
\caption{Ulwa pronouns}
\label{tab:overview:3}
\is{pronoun}
\is{personal pronoun}
\is{non-subject}
\is{possessive pronoun}
\begin{tabularx}{\textwidth}{QQQQ}
\lsptoprule
& subject

(‘\textsc{subj}’) & non-subject (‘\textsc{obj}’) & possessive (‘\textsc{poss}’)\\
\midrule
1\textsc{sg} & {\itshape nï} & {\itshape nï=} & {\itshape nïnji}\\
2\textsc{sg} & {\itshape u} & {\itshape u=} & {\itshape unji}\\
3\textsc{sg} & {\itshape mï} & {\itshape ma=} & {\itshape manji}\\
\textsc{1du.excl} & {\itshape ngan} & {\itshape ngan=} & {\itshape nganji}\\
{\scshape 1du.incl} & {\itshape ngunan} & {\itshape ngunan=} & {\itshape ngunanji}\\
2\textsc{du} & {\itshape ngun} & {\itshape ngun=} & {\itshape ngunji}\\
3\textsc{du} & {\itshape min} & {\itshape min=} & {\itshape minji}\\
1\textsc{pl.excl} & {\itshape an} & {\itshape an=} & {\itshape anji}\\
1\textsc{pl.incl} & {\itshape unan} & {\itshape unan=} & {\itshape unanji}\\
2\textsc{pl} & {\itshape un} & {\itshape un=} & {\itshape unji}\\
3\textsc{pl} & {\itshape ndï} & {\itshape ndï=} & {\itshape ndïnji}\\
{\scshape sg.refl} & -- & {\itshape ambï=} & {\itshape ambïnji}\\
{\scshape du.refl} & -- & {\itshape ambin=} & {\itshape ambinji}\\
{\scshape pl.refl} & -- & {\itshape ambla=} & {\itshape amblanji}\\
\lspbottomrule
\end{tabularx}
\end{table}

There are other pronominal forms that may be created by combining these more basic forms with various \isi{suffix}es. The \isi{intensive} \isi{suffix} \textit{{}-awa} ‘\textsc{int}’ combines with \isi{non-subject} \isi{pronoun}s to stress their role in a sentence \REF{ex:overview:39}.

\ea%39
    \label{ex:overview:39}
    \textit{mawa}\\
\gll    ma-\textbf{awa}\\
    3\textsc{sg.obj-int}\\
\glt    ‘he himself’
\z

The \isi{partitive-intensive} \isi{suffix} \textit{{}-we} ‘\textsc{part.int}’ combines with \isi{non-subject} \isi{pronoun}s to stress their role as sole participants in an event or action \REF{ex:overview:40}.

\ea%40
    \label{ex:overview:40}
    \textit{mawe}\\
\gll    ma-\textbf{we}\\
    3\textsc{sg.obj-part.int}\\
\glt    ‘he himself [from among several]’
\z

The \isi{emphatic} \isi{suffix} \textit{{}-nam} ‘\textsc{emph’} serves similar functions to these two \isi{intensive} \isi{suffix}es, although it combines with the subject form of a \isi{pronoun} and it may be restricted to third person referents \REF{ex:overview:41}.

\ea%41
    \label{ex:overview:41}
    \textit{mïnam}\\
\gll    mï-\textbf{nam}\\
    3\textsc{sg.subj-emph}\\
\glt    ‘he’s the one’
\z

The \isi{topic-marker} \isi{suffix} \textit{{}-ambi} ‘\textsc{top}’ may be used to mark the \isi{topic} of a sentence. It combines with \isi{non-subject} pronominal forms \REF{ex:overview:42}.

\ea%42
\label{ex:overview:42}
    \textit{nambi …}\\
\gll    nï-\textbf{ambi}\\
    1\textsc{sg-top}\\
\glt    ‘as for me, I …’
\z

\isi{Affective pronoun}s, which are used to convey compassion toward a second person or third person referent, are formed by combining \isi{non-subject} pronominal forms with the \isi{adjective} \textit{ngusuwa} ‘poor, pitiful’ \REF{ex:overview:43}.

\is{pronoun|)}

\newpage

\ea%43
    \label{ex:overview:43}
\is{pronoun}
    \textit{ungusuwa}\\
\gll    u-\textbf{ngusuwa}\\
 2\textsc{sg-}poor\\
\glt    ‘you poor thing’
\z

\section{Determiners}\label{sec:overview:2.7}

\is{determiner|(}

Ulwa \isi{noun phrase}s often contain postnominal \isi{determiner}s that index the \isi{number} of referents in a full NP. These \isi{subject marker}s (‘{\textsc{subj}}{’) and \isi{non-subject marker}s (‘}{\textsc{obj}}{’) are derived from the set of third person pronominal forms and seem to play a role in indicating \isi{topic} or \isi{focus}. \isi{Demonstrative} \isi{determiner}s occupy the same position in the NP. \isi{Proximal} (‘}{\textsc{prox}}{’) forms are used when the referent is near the speaker. \isi{Distal} (‘}{\textsc{dist}}{’) forms are used when the referent is not near the speaker. \isi{Determiner}s are given in \tabref{tab:overview:4}.}

\begin{table}
\caption{Ulwa determiners}
\label{tab:overview:4}
\is{determiner}
\is{proximal}
\is{distal}
\begin{tabular}{lllll}
\lsptoprule
& {\scshape subj} & {\scshape obj} & {\scshape prox} & {\scshape dist}\\
\midrule
\textsc{sg} & {\itshape mï} & {\itshape ma=} & {\itshape nga} & {\itshape anda}\\
\textsc{du} & {\itshape min} & {\itshape min=} & {\itshape ngin} & {\itshape andin}\\
{\scshape pl} & {\itshape ndï} & {\itshape ndï=} & {\itshape ngala} & {\itshape ala}\\
\lspbottomrule
\end{tabular}
\end{table}


The main formal difference between \isi{subject marker}s and \isi{non-subject marker}s occurs in the \isi{singular} forms: \textit{mï} ‘3\textsc{sg.subj}’ versus \textit{ma=} ‘3\textsc{sg.obj}’, the latter also exhibiting the \isi{phonological}ly conditioned \isi{allomorph} \textit{mo=} ‘3\textsc{sg.obj}’. The \isi{dual} \isi{non-subject marker} \textit{min=} ‘3\textsc{du}’ similarly exhibits an \isi{allomorph} \textit{mini=} ‘\textsc{3du.obj’} when immediately preceding /n/. The \isi{non-subject marker}s \isi{clitic}ize to immediately following verbs or \isi{postposition}s \REF{ex:overview:44}.


\ea\label{ex:overview:44}
 \textit{al} \textbf{\textit{men}}\\
\gll    al \textbf{ma}=in\\
    net  3\textsc{sg.obj}{=in}\\
\glt    ‘into the mosquito net’
\z

The \isi{demonstrative}s show no formal distinctions based on their grammatical roles. However, like the \isi{object marker}s, they tend to \isi{clitic}ize to the following word when functioning in \isi{non-subject} roles \REF{ex:overview:45}.

\is{determiner|)}

\ea\label{ex:overview:45}
\is{determiner}
 \textit{tïn} \textbf{\textit{andol}}\\
\gll    tïn    \textbf{anda}=ul\\
    dog {\textsc{sg.dist=}}{with}\\
\glt    ‘with that dog’
\z

\section{Postpositions}\label{sec:overview:2.8}

\is{postposition|(}

\isi{Adposition}s always follow their object in Ulwa -- that is, there are only \isi{postposition}s, no \isi{preposition}s. \isi{Postpositional phrase}s occur within \isi{verb phrase}s, always preceding the \isi{head verb}. If the verb is \isi{transitive}, then the \isi{postpositional phrase} also precedes the object of the verb \REF{ex:overview:46}.

\is{postposition|)}

\ea%46
    \label{ex:overview:46}
    \textbf{\textit{tïn mol}} \textit{lamndu masap}\\
\gll    \textbf{tïn}    \textbf{ma=ul}    lamndu  ma=asa-p\\
    dog  \textsc{3sg=}with  pig      \textsc{3sg.obj}=hit-\textsc{pfv}\\
\glt    ‘killed the pig with the dog’
\z

\section{Adverbs}\label{sec:overview:2.9}

\is{adverb|(}

\isi{Adverb}s are non-\isi{inflect}ing words that are never required by the argument structure of a verb, but which can be employed to give additional information about things like the manner, \isi{time}, or space in which an action or event transpires. They typically follow subjects and precede objects \REF{ex:overview:47} but may alternatively appear as the first element within a clause \REF{ex:overview:48}.

\is{adverb|)}

\ea%47
    \label{ex:overview:47}
    \textit{Mï} \textbf{\textit{wolka}} {\textit{impul matïn.}}\\
\gll    mï      \textbf{wolka}  {im-pul}      {ma=tï-n}\\
    3\textsc{sg.subj}  again  wood-piece  3\textsc{sg.obj}=take-\textsc{pfv}\\
\glt    ‘It again got a piece of wood.’
\z

\ea%48
    \label{ex:overview:48}
    \textbf{\textit{Amun}} \textit{tïn mï mïnda mame.}\\
\gll    \textbf{amun}  tïn    mï      mïnda    ma=ama-e\\
    now  dog  3\textsc{sg.subj}  banana    3\textsc{sg.obj=}eat-\textsc{ipfv}\\
\glt    ‘The dog is eating the banana now.’
\z

\newpage

\section{Negators}\label{sec:overview:2.10}

\is{negator|(}
\is{negation|(}

The \is{verbal negation} verbal negator \textit{ango} ‘\textsc{neg’} (‘no, not’) follows subjects and precedes objects \REF{ex:overview:49}.

\ea%49
    \label{ex:overview:49}
    \textit{Tïn mï} \textbf{\textit{ango}} \textit{mïnda ndamap.}\\
\gll    tïn    mï      \textbf{ango}  mïnda    ndï=ama-p\\
    dog  3\textsc{sg.subj}  \textsc{neg}  banana    3\textsc{pl=}eat-\textsc{pfv}\\
\glt    ‘The dog did not eat the bananas.’
\z

\isi{Non-verbal negation}, on the other hand, usually occurs with a \isi{clause-final negator}, whether \textit{me} ‘\textsc{neg’}, \textit{kom} ‘\textsc{neg}’, or \textit{kome} ‘\textsc{neg’}. These may co-occur with the preverbal \isi{negator} \textit{ango} ‘\textsc{neg}’ (‘no, not’), thereby creating a \isi{discontinuous} structure \REF{ex:overview:50}.

\ea%50
    \label{ex:overview:50}
    \textit{Way} \textbf{\textit{ango}} \textit{ambi} \textbf{\textit{me}}.\\
\gll    way  \textbf{ango}  ambi  \textbf{me}\\
    turtle  \textsc{neg}  big    \textsc{neg}\\
\glt    ‘The turtle was not big.’
\z

\isi{Negative command}s (i.e., \isi{prohibition}s) are formed with the \isi{prohibitive} marker \textit{wana} ‘\textsc{proh}’ (‘don’t!’) \REF{ex:overview:51}. It occurs in the same position as the \is{verbal negation} verbal negator \textit{ango} ‘\textsc{neg’} (‘no, not’) that is used in \isi{declarative} and \isi{interrogative} sentences. The \isi{prohibitive} marker \textit{wana} ‘\textsc{proh}’ has the \isi{allomorph} \textit{wanap} ‘\textsc{proh}’.

\is{negation|)}
\is{negator|)}

\ea%51
    \label{ex:overview:51}
    \textit{(U)} \textbf{\textit{wana}} \textit{nuwalinda!}\\
\gll    (u)    \textbf{wana}  nï=wali-nda\\
    (\textsc{2sg)}  \textsc{proh}  \textsc{1sg=}hit-\textsc{irr}\\
\glt    ‘Don’t hit me!’
\z

\section{Interrogative words}\label{sec:overview:2.11}

\is{interrogative word|(}
\is{question word|(}
\is{question|(}

\is{yes/no question}

No special words are needed for \isi{polar question}s, which are \isi{syntactic}ally identical to \isi{declarative} statements. These ‘yes/no’ questions are generally differentiated from statements through \isi{intonation} alone. \isi{Content question}s, on the other hand, use \isi{interrogative} words, such as \textit{kwa} ‘who?’ or \textit{angos} ‘what?’. The position of these \isi{question word}s is determined by their grammatical role in the clause. For example, a \isi{question word} occurs clause-initially if it is the grammatical subject of the \isi{question} \REF{ex:overview:52}, but clause-medially if it is the grammatical object \REF{ex:overview:53}.

\ea%52
    \label{ex:overview:52}
    \textbf{\textit{Kwa}} \textit{utam mamap?}\\
\gll    \textbf{kwa}  utam  ma=ama-p\\
    one    yam  3\textsc{sg.obj}=eat-\textsc{pfv}\\
\glt    ‘Who ate the yam?’
\z

\ea%53
    \label{ex:overview:53}
    \textit{Itom} \textbf{\textit{angos}} \textit{mamap?}\\
\gll    itom  \textbf{angos}  ma=ama-p\\
    father  what  3\textsc{sg.obj}=eat-\textsc{pfv}\\
\glt    ‘What did father eat?’
\z

Although most \isi{pronoun}s and \isi{determiner}s maintain a three-way \isi{number} distinction (\isi{singular} ‘\textsc{sg}’ vs. \isi{dual} ‘\textsc{du}’  vs. \isi{plural} ‘\textsc{pl}’), the \isi{interrogative pronoun} ‘who?’ maintains a two-way \isi{number} distinction: \isi{singular} \textit{kwa} ‘who? [\textsc{sg}]’ versus \isi{non-singular} \textit{kuma} ‘who? [\textsc{nsg}]’.

\is{question|)}
\is{question word|)}
\is{interrogative word|)}


\section{Interjections}\label{sec:overview:2.12}

\is{interjection|(}

\isi{Interjection}s are sometimes glossed with approximations in \ili{English} (e.g., ‘ah’, ‘hey’) and are other times glossed as ‘\textsc{interj}’. This category also includes the  \isi{tag question} \isi{interjection} \textit{a} {\textasciitilde} \textit{e} ‘eh?’, which may be used utterance-finally in \is{polar question} polar \isi{interrogative}s, typically in \isi{leading question}s \REF{ex:overview:54}.

\ea%54
    \label{ex:overview:54}
    \textit{Ngun mundu ngunas \textbf{a}?}\\
\gll    ngun  mundu    ngun=asa  \textbf{a}\\
    2\textsc{du}  hunger    2\textsc{du}=hit  \textsc{interj}\\
\glt    ‘You two are hungry, yeah?’
\z

The \isi{vocative} \isi{interjection} \textit{=o} ‘\textsc{voc}’ may \isi{clitic}ize to various forms when the referents are being emphasized or are being directly called to \REF{ex:overview:55}.

\is{interjection|)}

\ea%55
    \label{ex:overview:55}
    \textit{Supamo!}\\
\gll    Supam=\textbf{o}\\
    [name]=\textsc{voc}\\
\glt    ‘Hey, Supam!’
\z

\section{Clause structure}\label{sec:overview:2.13}

The \is{basic constituent order} order of constituents within a clause is generally rather fixed in Ulwa. \isi{Intransitive} and \isi{transitive} clauses are both verb-final and subject-initial. The object follows the subject and precedes the verb in \isi{transitive} clauses \REF{ex:overview:56}.

\ea%56
    \label{ex:overview:56}
      Basic \isi{constituent order}\\
\begin{tabbing}
{(Intransitive clauses:)} \= {(SV)}\kill
{Intransitive clauses:} \> {SV}\\
{Transitive clauses:} \> {SOV}
\end{tabbing}
\z

This order is maintained in all \isi{active-voice} clauses, whether \isi{indicative}, \isi{interrogative}, or \isi{imperative}, although it is common for arguments to be left unexpressed when recoverable from context (i.e., “\isi{pro-drop}”).

\isi{Oblique} NPs most commonly follow the subject and precede the \isi{verb phrase} (i.e., SXV or SXOV) \REF{ex:overview:57}.

\ea%57
    \label{ex:overview:57}
    \textit{Ankam mï} \textbf{\textit{mananï}} \textit{lamndu masap.}\\
\gll    ankam    mï      \textbf{mana=nï}    lamndu  ma=asa-p\\
    person    3\textsc{sg.subj}  spear=\textsc{obl}    pig      3\textsc{sg.obj}=hit-\textsc{pfv}\\
\glt    ‘The person hit the pig with a spear.’
\z

In \is{passive voice} passive-voice clauses, the \is{inverted word order} \isi{word order} is inverted. \isi{Passive} sentences have the order VS. If the \isi{agent} argument is expressed, then it occurs as a clause-initial \isi{oblique} \isi{phrase}.

\section{Non-verbal clauses}\label{sec:overview:2.14}

\is{non-verbal clause|(}

\isi{Non-verbal predication} can be accomplished without any overt verbal marking. In such constructions, the \isi{predicate} (whether a noun, \isi{noun phrase}, or \isi{adjective}) simply follows the subject \REF{ex:overview:58}.

\ea%58
    \label{ex:overview:58}
  \textit{Nïnji itom mï} \textbf{\textit{anma}}.\\
\gll    nï-nji     itom  mï      \textbf{anma}\\
    1\textsc{sg-poss}  father  3\textsc{sg.subj}  good\\
\glt    ‘My father is good.’
\z

As an alternative to these \is{zero copula} zero-copula constructions, it is also possible to add a \isi{copular enclitic} \textit{=p} ‘\textsc{cop}’ to a noun, \isi{adjective}, or other \isi{parts of  speech} to create a \isi{predicate} \REF{ex:overview:59}.

\ea%59
    \label{ex:overview:59}
    \textit{Nïnji itom mï} \textbf{\textit{anmap}}.\\
\gll    nï-nji     itom  mï      anma=\textbf{p}\\
    1\textsc{sg-poss}  father  3\textsc{sg.subj}  good=\textsc{cop}\\
\glt    ‘My father is good.’
\z

The \isi{copula} \textit{=p} ‘\textsc{cop}’ is clearly related to the \isi{locative verb} \textit{p-} ‘be, be at (be located at)’, which is used in \isi{locative predication} \REF{ex:overview:60}.

\ea%60
    \label{ex:overview:60}
    \textit{Ndï amun Mosombla} \textbf{\textit{pe}}.\\
\gll    ndï    amun  Mosombla    \textbf{p}{}-e\\
    3\textsc{pl}    now  Yaul      be-\textsc{ipfv}\\
\glt    ‘They are now in Yaul [village].’
\z

This \isi{locative verb} \textit{p-} ‘be, be at’ has a \isi{suppletive} form that is used for reference to \isi{past} \isi{time}: \textit{wap} ‘be.\textsc{pst}’ \REF{ex:overview:61}.

\is{non-verbal clause|)}

\ea%61
    \label{ex:overview:61}
    \textit{Ndï Mosombla} \textbf{\textit{wap}}.\\
\gll    ndï    Mosombla    \textbf{wap}\\
    3\textsc{pl}    Yaul      be.\textsc{pst}\\
\glt    ‘They were in Yaul [village].’
\z

\section{Complex sentences}\label{sec:overview:2.15}

\is{complex sentence|(}

In\isi{dependent clause}s may be \isi{coordinate}d \isi{paratactic}ally. \isi{Dependent clause}s, however, are linked to following clauses by means of the \isi{dependent marker} \textit{{}-e} ‘\textsc{dep}’, which occurs as the final \isi{suffix} on the verb in the \isi{dependent clause} \REF{ex:overview:62}.

\ea%62
    \label{ex:overview:62}
    \textbf{\textit{Nï inim lopope}} \textit{nï mana.}\\
\gll    nï    inim  lopo-p-\textbf{e}    nï    ma-na\\
    1\textsc{sg}  water  wash-\textsc{pfv-dep}  \textsc{1sg}  go-\textsc{irr}\\
\glt    ‘After I have bathed, I will go.’
\z

In \REF{ex:overview:62} the \isi{dependent clause} has a \is{temporal subordinate clause} temporal relationship with the following \isi{main clause}. Other \isi{semantic} relationships, such as \is{causal subordinate clause} causal relationships, are also possible. Furthermore, the subject of the \isi{main clause} may be different from the subject of the \isi{dependent clause} \REF{ex:overview:63}.

\ea%63
    \label{ex:overview:63}
    \textbf{\textit{Lamndu mï Anam manji utam amape}} \textit{Anam mï masap.}\\
\gll    lamndu  mï      Anam  ma-nji      utam  ama-p-\textbf{e} Anam    mï      ma=asa-p\\
    pig      3\textsc{sg.subj}  [name]  3\textsc{sg.obj-poss}  yam  eat-\textsc{pfv-dep}    [name]    3\textsc{sg.subj}  3\textsc{sg.obj}=hit-\textsc{pfv}\\
\glt    ‘Anam killed the pig, because it ate his yam.’ (Literally ‘[Because] the pig ate Anam’s yam, Anam killed it.’)
\z

The \isi{dependent marker} \textit{{}-e} ‘\textsc{dep}’ is pervasive, occurring not only in clauses that are clearly being subordinated but also in clauses that would appear to be \isi{main clause}s. This may in part be due to the morpheme’s function as a \isi{floor-holding} device, signaling that the speaker still has more to say.

The \isi{dependent marker} is also used in \isi{passive} sentences, which have VS \isi{word order} despite functioning as \isi{independent clause}s \REF{ex:overview:64}.

\ea%64
    \label{ex:overview:64}
    \textbf{\textit{Asape}} \textit{lamndu mï.}\\
\gll    asa-p-\textbf{e}      lamndu  mï\\
    hit-\textsc{pfv-dep}  pig      3\textsc{sg.subj}\\
\glt    ‘The pig was killed.’
\z

Multiple analyses of verb forms are at times possible given the \isi{homophony} between the \isi{dependent marker} \textit{{}-e} ‘\textsc{dep}’ and the \isi{imperfective} \isi{suffix} \textit{{}-e} ‘\textsc{ipfv}’.

\is{complex sentence|)}