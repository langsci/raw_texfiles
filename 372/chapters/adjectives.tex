\chapter{Adjectives}\label{sec:5}

\is{adjective|(}

In Ulwa, words that denote properties are not particularly  distinct as a class in terms of their \isi{morphosyntax}. The fundamental divide among grammatical categories in Ulwa falls rather between verbs and non-verbs. When viewed within this dichotomy, adjectives resemble nouns somewhat more than they resemble verbs. For example, adjectives may receive the \isi{copular enclitic}. They never receive any of the three basic \isi{TAM} \isi{suffix}es found on verbs. Like nouns, adjectives are not marked for \isi{person}, \isi{number}, \isi{gender}, or \isi{case}. One possible minor criterion by which adjectives may be distinguished from nouns is their placement within a \isi{noun phrase}. Within NPs, adjectives follow nouns.

\is{adjective|)}

\section{Attributive adjectives}\label{sec:5.1}

\is{attributive adjective|(}
\is{adjective|(}

When an \isi{adjective} is neither functioning as a substantive nor serving as a \isi{predicate complement}, it occurs within the limits of a \isi{noun phrase}. When inside an NP, adjectives occur after the noun (the \isi{head} of the NP) and before the \isi{subject marker}, \isi{object marker}, or any other \isi{determiner} that may be found in an NP. Such \isi{adnominal adjective}s never occur outside the NP (i.e., \isi{discontinuous}ly). Sentences \REF{ex:adj:1} through \REF{ex:adj:4} provide examples of adjectives (in \textbf{bold}) as they appear in NPs, illustrating their postnominal position, preceding \isi{determiner}s.

\ea%1
    \label{ex:adj:1}
            \textit{Ankam} \textbf{\textit{ambi}} \textit{mï tïn} \textbf{\textit{njukuta}} \textit{masap.}\\
\gll    ankam  \textbf{ambi}  mï      tïn    \textbf{njukuta}  ma=asa-p\\
    person  big    3\textsc{sg.subj}  dog  small    3\textsc{sg.obj=}hit\textsc{{}-pfv}\\
\glt `The big person hit the small dog.’ [elicited]
\z

\ea%2
    \label{ex:adj:2}
           \textit{Ndïnji i} \textbf{\textit{anma}} \textit{mï ndul i.}\\
\gll    ndï-nji    i    \textbf{anma}  mï      ndï=ul    i\\
    3\textsc{pl-poss}  way  good  3\textsc{sg.subj}  3\textsc{pl}=with  go.\textsc{pfv}\\
\glt `Their good behavior has gone with them.’ [ulwa014\_62:57]
\z

\is{adjective|)}
\is{attributive adjective|)}

\newpage

\ea%3
    \label{ex:adj:3}
\is{adjective}
            \textit{Nï nïnji wandam} \textbf{\textit{ambi}} \textit{ndalop.}\\
\gll    nï    nï-nji    wandam  \textbf{ambi}  anda=lo-p\\
    1\textsc{sg}  1\textsc{sg-poss}  jungle    big    \textsc{sg.dist}=cut-\textsc{pfv}\\ 
\glt `I cleared that big garden of mine.’ [ulwa042\_00:04]
\z

\ea%4
    \label{ex:adj:4}
\is{attributive adjective}
            \textit{Nïmal} \textbf{\textit{wapata}} \textit{men pe apa ite mawap.}\\
\gll    nïmal  \textbf{wapata}  ma=in      p-e    apa    ita-e     ma=wap\\
    river  old     3\textsc{sg.obj}=in  be\textsc{{}-dep} house  build-\textsc{ipfv} 3\textsc{sg.obj}=be.\textsc{pst}\\
\glt `[They] were building houses in the old river there.’ [ulwa032\_27:47]
\z

\section{Predicative adjectives}\label{sec:5.2}

\is{predicative adjective|(}
\is{adjective|(}

When adjectives function predicatively, they may receive \isi{copula}r \isi{morphology} (\sectref{sec:10.2}), although this is not obligatory (neither for adjectives nor for nouns). These predicative adjectives occur clause-finally (the position held prototypically by verbs). Examples of predicative adjectives are given in \REF{ex:adj:5}, \REF{ex:adj:6}, and \REF{ex:adj:7}.

\is{adjective|)}
\is{predicative adjective|)}

\ea%5
    \label{ex:adj:5}
            \textit{Mïnkïn ndï} \textbf{\textit{wutota}}.\\
\gll mïnkïn    ndï  \textbf{wutota}\\
    sago.species  3\textsc{pl}  tall\\
\glt `The sago palms are tall.’ [ulwa014\_71:18]
\z

\ea%6
    \label{ex:adj:6}
            \textit{Nïnji wutï ambatïm ngala} \textbf{\textit{tembipe}}.\\
\gll nï-nji    wutï  ambatïm  ngala    \textbf{tembi}=p-e\\
    1\textsc{sg-poss}  foot  joint    \textsc{pl.prox}  bad=\textsc{cop-dep}\\
\glt `My knees are bad.’ [ulwa032\_49:17]
\z

\ea%7
    \label{ex:adj:7}
            \textit{Mï} \textbf{\textit{anmapïna}}.\\
\gll mï      \textbf{anma}=p-na\\
    3\textsc{sg.subj}  good=\textsc{cop-irr}\\
\glt `It [= sago starch] will be good.’ [ulwa014\_60:25]
\z

\section{Substantive adjectives}\label{sec:5.3}

\is{substantive adjective|(}
\is{adjective|(}

Adjectives may also function as substantives -- that is, they may have the same formal properties as prototypical nouns. In such cases, these adjectives have the same distribution as nouns. Thus, adjectives may serve as the \isi{head}s of NPs, which may themselves serve as subject \REF{ex:adj:8}, \isi{direct object} \REF{ex:adj:9}, or \isi{object of a postposition} \REF{ex:adj:10} within a clause.

\is{adjective|)}
\is{substantive adjective|)}


\ea%8
    \label{ex:adj:8}
            \textbf{\textit{Ambi}} \textit{mï ngunanu ndïtïna.}\\
\gll     \textbf{ambi}  mï      ngunan=u      ndï=tï-na\\
    big    3\textsc{sg.subj}  1\textsc{du.incl}=from  \textsc{3pl}=take-\textsc{irr}\\
\glt `The big man will get them from us.’ (Literally ‘the big [one]’) [ulwa014\_33:46] \is{adjective}
\z

\ea%9
    \label{ex:adj:9}
            \textbf{\textit{Tembi}} \textit{ndinap.}\\
\gll    \textbf{tembi}  ndï=ina-p\\
    bad    3\textsc{pl}=get-\textsc{pfv}\\
\glt `[I] got the bad ones [= tobacco plants].’ [ulwa037\_55:49]
\z

\ea%10
    \label{ex:adj:10}
\is{substantive adjective}
          \textbf{\textit{Tembi}} \textit{ngalawl inde.}\\
\gll    \textbf{tembi}  ngala=ul    inda{}-e\\
    bad    \textsc{pl.prox}=with  walk-\textsc{ipfv}\\
\glt `[They] walk around with these sick ones [= children].’ [ulwa014\_47:09]
\z

\section{Relationship to other word classes}\label{sec:5.4}

\is{adjective|(}
\is{word class|(}

One factor complicating the task of assigning words in Ulwa to the grammatical category of \isi{adjective} is the fact that adjectives in NPs sometimes precede their \isi{head noun}s (instead of following them). While some speakers consider this order to be ungrammatical (perhaps an influence from the \isi{word order} of \ili{Tok Pisin}; see \chapref{sec:15}), it nevertheless occurs in speech, thereby making it difficult to rely on the distributional criterion that adjectives follow nouns. This adjective-noun \isi{word order} is exemplified in sentences \REF{ex:adj:11}, \REF{ex:adj:12}, and \REF{ex:adj:13}.

\ea%11
    \label{ex:adj:11}
          \textbf{\textit{Waembïl}} \textit{ankam anda i.}\\
\gll    \textbf{waembïl}  ankam  anda    i\\
    white    person  \textsc{sg.dist}  go.\textsc{pfv}\\
\glt `That white person came.’\footnote{The form \textit{waembïl ankam} ‘white person’ may be \isi{lexical}ized, perhaps \isi{calque}d from \ili{Tok Pisin} \textit{waitman} ‘white man, white person’.}  [ulwa013\_05:21]
\z

\ea%12
    \label{ex:adj:12}
          \textbf{\textit{Tembi}} \textit{ankam ala imba pe.}\\
\gll    \textbf{tembi}  ankam  ala      imba  p-e\\
    bad    person  \textsc{pl.dist}  night  be\textsc{{}-ipfv}\\
\glt `Those bad people are around at night.’ [ulwa032\_16:02]
\z

\ea%13
    \label{ex:adj:13}
          \textbf{\textit{Tembi}} \textit{nji ala ala ndït indana.}\\
\gll    \textbf{tembi}  nji    ala      ala      ndï=tï    inda-na\\
    bad    thing  \textsc{pl.dist}  \textsc{pl.dist}  3\textsc{pl}=take  walk-\textsc{irr}\\
\glt `Those bad things -- they will bring them [here].’ [ulwa037\_20:39]
\z

  The \isi{morphosyntactic} similarity between nouns and adjectives also makes it difficult at times to assign certain words to one class or the other. For example, the word \textit{kalam} ‘knowledge’ can have either the more nominal meaning ‘knowledge, wisdom’ or the more adjectival meaning ‘knowledgeable, knowing, wise’, and it is difficult to define one of these meanings as being the primary one. Whereas in \REF{ex:adj:14} this word carries a more noun-like meaning, in \REF{ex:adj:15} it functions more like a substantive \isi{adjective} (\sectref{sec:5.3}).

\ea%14
    \label{ex:adj:14}
          \textit{Ndawa ndïnji} \textbf{\textit{kalam}} \textit{andol le.}\\
\gll    ndï-awa  ndï-nji    \textbf{kalam}      anda=ul    lo-e\\
    3\textsc{pl-int}  3\textsc{pl-poss}  knowledge    \textsc{sg.dist}=with  go-\textsc{ipfv}\\
\glt `They went around with their knowledge.’ [ulwa014†]
\z

\ea%15
    \label{ex:adj:15}

          \textit{Yena ambi anda u} \textbf{\textit{kalam}} \textit{anda.}\\
\gll    yena  ambi  anda    u    \textbf{kalam}    anda\\
    woman  big    \textsc{sg.dist}  2\textsc{sg}  knowledge  \textsc{sg.dist}\\
\glt `You’re a grown woman; you know well.’ (Literally ‘[You] are that big woman; you are that knowledgeable [woman].’) [ulwa032\_09:04]
\z

Complicating matters even further is the fact that \textit{kalam} ‘knowledge, knowledgeable’ very often functions like a verb. Although it does not take \is{verbal morphology} verbal \isi{TAM} \isi{morphology} but rather the \isi{copular enclitic} (thus making it resemble nouns and adjectives rather than verbs), it seems capable of taking object arguments (thus making it resemble verbs). In each of examples \REF{ex:adj:16}, \REF{ex:adj:17}, and \REF{ex:adj:18}, \textit{kalam} ‘knowledge’ appears to have a \isi{direct object} argument, including \isi{proclitic} \isi{object marking} in \REF{ex:adj:16} and \REF{ex:adj:17}.

\ea%16
    \label{ex:adj:16}
          \textit{Mï} \textbf{\textit{ukalampe}}.\\
\gll mï      u=\textbf{kalam}=p-e\\
    3\textsc{sg.subj}  2\textsc{sg}=knowledge=\textsc{cop-dep}\\
\glt `She knows you.’ [ulwa032\_36:00]
\z

\ea%17
    \label{ex:adj:17}
          \textit{Nï ango} \textbf{\textit{ndïkalam}}.\\
\gll nï    ango  ndï=\textbf{kalam}\\
    1\textsc{sg}  \textsc{neg}   3\textsc{pl}=knowledge\\
\glt `I don’t know about them.’ [ulwa032\_49:16]
\z

\ea%18
    \label{ex:adj:18}
          \textit{Na ndï anjikakape i} \textbf{\textit{kalampïna}}.\\
\gll na    ndï  anjikaka=p-e  i    \textbf{kalam}=p-na\\
    and    3\textsc{pl}  how=\textsc{cop-dep}  way  knowledge=\textsc{cop-irr}\\
\glt `And how are they going to know [good] behavior?’\footnote{See \sectref{sec:13.1.2} for the internal \isi{morphology} of \textit{anjikaka} ‘how?’} (\textit{na} < TP \textit{na} ‘and’) [ulwa014\_41:07]
\z

The peculiar behavior of \textit{kalam} ‘knowledge’ may reflect the fact that it is a \isi{loanword} from \ili{Waran} (\sectref{sec:1.5.6}): in being \isi{borrow}ed, it may have come to be associated with additional \isi{lexical class}es (cf. the \ili{Tok Pisin} \isi{loan} \textit{lukautim} ‘look after’ in \sectref{sec:15.6}).

  Even fairly prototypical adjectives, such as \textit{tembi} ‘bad, sick, etc.’ can be employed nominally (i.e., to mean ‘badness, sickness, etc.’), as seen in \REF{ex:adj:19} and \REF{ex:adj:20}.

\ea%19
    \label{ex:adj:19}
          \textbf{\textit{Tembi}} \textit{mï makape tïlwa ndo unden.}\\
\gll    \textbf{tembi}  mï      maka=p-e    tïlwa  anda=u    unda-en\\
    bad    3\textsc{sg.subj}  thus=\textsc{cop-dep}  road  \textsc{sg.dist}=from  go-\textsc{nmlz}\\
\glt `The sickness is one that goes along this kind of road.’ [ulwa038\_02:16]
\z

\ea%20
    \label{ex:adj:20}
          \textbf{\textit{Tembi}} \textit{nji ala un mat} \textbf{\textit{tembi}} \textit{tï mananda.}\\
\gll    \textbf{tembi}  nji    ala      u=n    u    ma=tï      \textbf{tembi}  tï    ma=na-nda\\
    bad    thing  \textsc{pl.dist}  2\textsc{sg=obl}  from  3\textsc{sg.obj}=take  bad  take  3\textsc{sg.obj}=give-\textsc{irr}\\

\glt `Bad things will take her from you and give her sickness.’ [ulwa032\_17:34]
\z

In example \REF{ex:adj:20} this word \textit{tembi} ‘bad/badness’ functions both as an \isi{adjective} and as a noun in the same sentence (note the non-canonical order of noun and \isi{adjective} in the first NP). The sense of ‘sick’ (i.e., an adjectival meaning) is illustrated by example \REF{ex:adj:21}; here the word functions as a \isi{predicate adjective}, receiving the \isi{copular enclitic}.

\ea%21
    \label{ex:adj:21}
          \textit{U} \textbf{\textit{tembipïta}}.\\
\gll u    \textbf{tembi}=p{}-ta\\
    2\textsc{sg}  bad=\textsc{cop-cond}\\
\glt `You may be sick [someday].’ [ulwa014\_13:10]
\z

\is{word class|)}
\is{adjective|)}

\is{adjective|(}

Another notable characteristic of the grammatical class of adjectives is its rather small size. Taking the definition (based both on \isi{semantics} and on \isi{syntactic} distribution) that adjectives are words that denote properties and can occur within NPs after nouns and before \isi{determiner}s, then the class of adjectives is quite small, and is, perhaps, closed. The list in \REF{ex:adj:22} contains the best exemplars of this class of adjectives. They almost all refer to physical properties.

\ea%22
    \label{ex:adj:22}
          Property words / adjectives
\begin{tabbing}
{(\textit{mundotoma}}) \= {(‘hard’ (i.e., not soft))}\kill
\textit{anma}   \>   ‘good’\\
 \textit{tembi}  \>    ‘bad’\\
 \textit{ambi}  \>    ‘big’\\
 \textit{njukuta} \>   ‘small’\\
 \textit{nïpat}  \>    ‘giant’\\
 \textit{ilum}  \>    ‘little’\\
 \textit{wapata}  \>    ‘old, dry’\\
 \textit{akïnaka} \>   ‘new, young’\\
 \textit{wananum} \>   ‘hot’\\
 \textit{mïnoma}  \>  ‘cold’\\
 \textit{namli}  \>    ‘soft’\\
 \textit{nïpokonam} \> ‘hard’ (i.e., not soft)\\
 \textit{kenmbu}   \> ‘heavy’\\
 \textit{wiwila}   \>   ‘light’ (i.e., not heavy)\\
 \textit{wutota}  \>    ‘tall, long’\\
 \textit{mundotoma} \> ‘short’\\
 \textit{nu}   \>     ‘near’\\
 \textit{ngaya}   \>   ‘far’\\
 \textit{mbun}  \> ‘black’\\
 \textit{waembïl}  \>  ‘white’\\
 \textit{andïl} \>      ‘careful, slow, quiet’\\
 \textit{yangle} \>     ‘strong’\\
 \textit{yangïmot}  \>  ‘tasty, sweet’\\
 \textit{mïnwata}  \>  ‘wet, ripe, rotten’\\
 \textit{maw}   \>   ‘correct’\\
 \textit{monop}  \>    ‘full, sated’\\
    \textit{ngusuwa}  \>  ‘poor, pitiful’\\
    \textit{wopa}  \>    ‘whole’
\end{tabbing}
\z

The words in \REF{ex:adj:22} may come close to representing a complete list of true adjectives, at least those most commonly used in discourse. To denote most other properties that could be ascribed to nominals, Ulwa employs other grammatical strategies, such as using \isi{postpositional phrase}s or \isi{verb phrase}s. For example, the notion ‘fast’ may be expressed with a \isi{metaphor}ical \isi{postpositional phrase} \textit{apïn wat} ‘on fire’, as in \REF{ex:adj:23}.

\ea%23
    \label{ex:adj:23}
          \textit{Tïn \textbf{apïn wat} mï imbamka.}\\
\gll    tïn    \textbf{apïn}  \textbf{wat}  mï      imbam-ka\\
    dog  fire    atop  3\textsc{sg.subj}  run-let\\
\glt `The fast dog ran.’ [elicited]
\z

The notion ‘happy’ may be expressed with the \isi{compound verb} \textit{wana-ni-} ‘feel-act’ along with the \isi{adjective} \textit{anma} ‘good’ \REF{ex:adj:24}.

\ea%24
    \label{ex:adj:24}
          \textit{\textbf{Anma wanane} mol lope i.}\\
\gll    \textbf{anma}  \textbf{wana-ni-e}    ma=ul      lo-p-e      i\\
    good  feel-act-\textsc{ipfv}  3\textsc{sg.obj}=with  go-\textsc{pfv-dep}  yay\\
\glt `[They] were happy, went with him, [and said:] “yay!”.’ [ulwa035\_03:50]
\z

It is also not uncommon to use \ili{Tok Pisin} \isi{loanword}s, as in \REF{ex:adj:25}, which contains \ili{Tok Pisin} \textit{amamas} ‘happy’ to denote the same attribute that is expressed with a \isi{verb phrase} in \REF{ex:adj:24}.

\ea%25
    \label{ex:adj:25}
          \textit{Ndï wa} \textbf{\textit{amamaspe}} \textit{mol lopen.}\\
\gll    ndï    wa  \textbf{amamas}=p-e    ma=ul      lo-p-en\\
    3\textsc{pl}    just  happy=\textsc{cop-dep}  3\textsc{sg.obj}=with  go-\textsc{pfv-nmlz}\\
\glt `They were just happy and went with him.’ (\textit{amamas} = TP) [ulwa035\_03:48]
\z

One final feature of adjectives to be discussed here is their ability to function as \isi{adverb}s when placed immediately before the verb in the clause. This results in the \isi{direct object} being \isi{demote}d to an \isi{oblique}, as in \REF{ex:adj:26}.

\ea%26
    \label{ex:adj:26}
         \textit{Inim u kwa} \textbf{\textit{man}} \textit{anma lan!}\\
\gll    inim  u    kwa  \textbf{ma=n}      anma  la-n\\
    water  \textsc{2sg}  just    3\textsc{sg.obj=obl}  good  eat-\textsc{imp}\\
\glt `Water – just drink it well!’ [ulwa014†]
\z

For more on this phenomenon, including additional examples, see the sections on \isi{adverb}s (\sectref{sec:8.2}), the \isi{oblique marker} \textit{=n} ‘\textsc{obl}’ (\sectref{sec:11.4.1}), and \isi{valency}-changing operations (\sectref{sec:13.8.9}).

\is{adjective|)}