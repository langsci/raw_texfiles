\chapter{Determiners}\label{sec:7}

\is{determiner|(}

I include under the heading “determiners” a number of rather different word types (including \isi{clitic}s, as well as possibly \isi{affix}es), that in some way indicate the \isi{definiteness} or \isi{specificity} of a referent, provide information that situates it in space, or identify its function within a clause. There are some \isi{syntactic} commonalities among the various categories described in the following sections, although they do not necessarily constitute a single \isi{syntactic}ally definable \isi{word class}. One function of Ulwa determiners is to encode the \isi{number} of a referent NP, which is otherwise unmarked for \isi{number}. Thus, \isi{subject marker}s (\sectref{sec:7.1}), \isi{object marker}s (\sectref{sec:7.2}), and \isi{demonstrative}s (\sectref{sec:7.3}) may all be marked as \isi{singular}, \isi{dual}, or \isi{plural}. This chapter also describes quantifiers (\sectref{sec:7.4}) and numerals (\sectref{sec:7.5}).

\is{determiner|)}

\section{Subject markers}\label{sec:7.1}

\is{subject marker|(}
\is{determiner|(}

Ulwa makes frequent use of a class of what are termed here “subject markers”. These are a set of postnominal determiners that occur in subject \isi{noun phrase}s. Though never obligatory, they are very common. When present, they always occur as the final element of their NP. The three basic subject markers have the same form as, and are clearly related to, the third \isi{person} subject \isi{pronoun}s described in \sectref{sec:6.1}. The subject markers are given in \REF{ex:det:1}.\footnote{The 3\textsc{sg} subject-marker \is{subject marker} form /mï/ is glossed as ‘3\textsc{sg.subj}’, since it is formally distinct from the 3\textsc{sg} \isi{object-marker} form /ma=/, which is glossed as ‘3\textsc{sg.obj’}.}

\ea%1
    \label{ex:det:1}
            Subject markers\\
\begin{tabbing}
{(\textit{ndï})} \= {(‘3\textsc{sg.subj}’)}\kill
{\textit{mï}} \> {‘3\textsc{sg.subj}’}\\
{\textit{min}} \> {‘3\textsc{du}’}\\
{\textit{ndï}} \> {‘3\textsc{pl}’}
\end{tabbing}
\z

As described in \sectref{sec:3.1}, nouns in Ulwa are not marked in any way to reflect \isi{number}. Subject markers, however, can indicate whether the NP to which they belong is \isi{singular}, \isi{dual}, or \isi{plural}. Also, although they are \isi{phonological}ly mostly \isi{homophonous} with their equivalents in the paradigm of \isi{object marker}s (\sectref{sec:7.2}), there is at least one formal difference in the 3\textsc{sg} forms; as such, subject markers may be said to indicate \isi{case} as well, albeit in a rather marginal way.

The \isi{subject marker} can occur with either \isi{animate} or \isi{inanimate} referents, as seen in examples \REF{ex:det:2} through \REF{ex:det:7}. In these examples, each NP marked by a \isi{subject marker} is translated with the \ili{English} definite article ‘the’. This reflects the fact that subject markers (as determiners) may function to signal \isi{definiteness} (or \isi{specificity}) in a referent. That said, it is often possible to translate NPs with subject markers with the \ili{English} \isi{indefinite} article ‘a, an’ or with no article at all. Indeed, the \isi{semantic} range and discourse functions of these articles remain largely unknown to me.

\ea%2
    \label{ex:det:2}
            \textit{Mana} \textbf{\textit{mï}} \textit{liyu.}\\
\gll    mana  \textbf{mï}      li-u\\
    spear  3\textsc{sg.subj}  down-put\\
\glt `The spear fell.’ [elicited]
\z

\ea%3
    \label{ex:det:3}
            \textit{Wot yana \textbf{mï} lïmndï mala.}\\
\gll    wot    yana    \textbf{mï}      lïmndï  ma=ala\\
    younger  woman    3\textsc{sg.subj}  eye    \textsc{3sg.obj}=see\\
\glt `The younger sister saw her.’ [ulwa001\_05:49]
\z

\ea%4
    \label{ex:det:4}
            \textit{Sokoy} \textbf{\textit{mï}} \textit{ango anmap tembi.}\\
\gll    sokoy    \textbf{mï}      ango  anma=p  tembi\\
    tobacco  3\textsc{sg.subj}  \textsc{neg}  good=\textsc{cop}  bad\\
\glt `The tobacco isn’t good; [it’s] bad.’ [ulwa037\_53:09]
\z

\ea%5
    \label{ex:det:5}
            \textit{Itom} \textbf{\textit{ndï}} \textit{isin ndïwanap.}\\
\gll    itom  \textbf{ndï}  isi=n    ndï=wana-p\\
    father  \textsc{3pl}  soup=\textsc{obl}  3\textsc{pl}=cook-\textsc{pfv}\\
\glt `The men cooked them in soup.’ [ulwa014\_67:13]
\z

\ea%6
    \label{ex:det:6}
            \textit{Alum} \textbf{\textit{ndï}} \textit{se.}\\
\gll    alum  \textbf{ndï}   sa{}-e\\
    child  3\textsc{pl}    cry-\textsc{ipfv}\\
\glt `The children were crying.’  [ulwa032\_33:34]
\z

\ea%7
    \label{ex:det:7}
            \textit{Tïn} \textbf{\textit{min}} \textit{mo maka lamndu kon anmbas.}\\
\gll    tïn    \textbf{min}  ma=u      maka  lamndu  ko=n an-mbï-asa\\
    dog  3\textsc{du}  3\textsc{sg.obj}=from  thus  pig      \textsc{indf}=\textsc{obl}    out-here-hit\\
\glt `The two dogs thus chased a pig out from there.’ [ulwa037\_02:28]
\z

These examples illustrate how subject markers can indicate whether a \isi{common noun} subject is \isi{singular}, as in \REF{ex:det:2}, \REF{ex:det:3}, and \REF{ex:det:4}; \isi{plural}, as in \REF{ex:det:5} and \REF{ex:det:6}; or \isi{dual}, as in \REF{ex:det:7}. Subject markers can also be used with \isi{proper noun}s, as in \REF{ex:det:8} and \REF{ex:det:9}.

\ea%8
    \label{ex:det:8}
            \textit{Tarambi} \textbf{\textit{mï}} \textit{ita nï asilaka man makïna.}\\
\gll    Tarambi  \textbf{mï}       i-ta        nï    asi-la-ka  ma=n ma=kï-na\\
    [name]    3\textsc{sg.subj}  go.\textsc{pfv-cond}  1\textsc{sg}  sit-\textsc{irr}{}-let  3\textsc{sg.obj=obl}    3\textsc{sg.obj}=say-\textsc{irr}\\
\glt `When Tarambi comes, I will sit and tell him.’ [ulwa014†]
\z

\ea%9
    \label{ex:det:9}
            \textit{Kumba} \textbf{\textit{ndï}} \textit{wolka anul anmbi.}\\
\gll    Kumba  \textbf{ndï}  wolka  an=ul        an-mbï-i\\
    Bun  3\textsc{pl}  again  1\textsc{pl.excl}=with  out-here-go.\textsc{pfv}\\
\glt `Again the [people from] Bun [village] came out with us.’ [ulwa002\_02:27]
\z

Subject markers can also be used with recent \isi{loanword}s, as in \REF{ex:det:10}, which contains the \ili{Tok Pisin} word \textit{polis} ‘police’.

\ea%10
    \label{ex:det:10}
          \textit{Polis \textbf{ndï} ndiya ata ma keka namndu ndïwalinda.}\\
\gll    polis  \textbf{ndï}  ndï=iya    ata  ma  keka      namndu ndï=wali-nda\\
    police  3\textsc{pl}  3\textsc{pl}=toward  up  go  completely  pig    3\textsc{pl}=hit-\textsc{irr}\\
\glt `The police will go up to them and completely kill the pigs.’ (\textit{polis} = TP) [ulwa014\_27:29]
\z

Throughout the examples in this grammar, subject markers are glossed with a ‘3’ (for third \isi{person}).\footnote{This is done for two reasons: first, the forms of the subject markers are identical to those of the third \isi{person} \isi{personal pronoun}s; second, subject markers can only appear with third \isi{person} \isi{noun phrase}s. They never occur with first \isi{person} or second \isi{person} \isi{noun phrase}s. Thus, they do indeed index the third \isi{person}.} Subject markers never appear with \isi{personal pronoun}s.\footnote{Thus, the following sequences are ungrammatical: \textit{*nï mï}, \textit{*ngunan min}, \textit{*min min}, \textit{*un ndï}, \textit{*ndï ndï}, and so on. However, subject markers may follow certain other pronominal forms, such as \is{indefinite pronoun} indefinite or \isi{interrogative pronoun}s, although some usages may be only marginally acceptable (see \sectref{sec:6.4} and \sectref{sec:6.5}).} Although apparently not obligatory, subject markers can be useful for clarifying meaning in certain circumstances. First, since \isi{adjective}s (as well as \isi{possessive pronoun}s) may be used substantively, the presence of a \isi{subject marker} may clarify that an \isi{adjective} or other modifier is functioning as the subject of the sentence, as in \REF{ex:det:11} and \REF{ex:det:12}.

\ea%11
    \label{ex:det:11}
          \textit{\textbf{Ambi mï} keka mat nin ndïl.}\\
\gll    \textbf{ambi}  \textbf{mï}      keka      ma=tï      nin    ndï=lï\\
    big    3\textsc{sg.subj}  completely  3\textsc{sg.obj}=take  thorn  3\textsc{pl}=put\\
\glt `The big one [= a pig] completely got him and put [him] on thorns.’ [ulwa020\_01:37]
\z

\ea%12
    \label{ex:det:12}
          \textit{\textbf{Nïnji ndï} anma iye.}\\
\gll    \textbf{nï-nji}    \textbf{ndï}  anma  i-e\\
    1\textsc{sg-poss}  \textsc{3pl}  good  go.\textsc{pfv-dep}\\
\glt `My [comrades] came [home] well.’ [ulwa002\_06:41]
\z

Also, in \is{equative predication} equative or \is{attributive predication} attributive sentences that lack any overt verb form (i.e., sentences that contain no overt \isi{copula}, \sectref{sec:10.1}), the \isi{subject marker} helps to break the clause into two halves: everything up to and including the \isi{subject marker} is clearly the subject of the clause; everything following must be the \isi{predicate}, as illustrated by sentences \REF{ex:det:13} through \REF{ex:det:16}, in which square brackets enclose first the [subject] and then the [\isi{predicate}].

\ea%13
    \label{ex:det:13}
          \textit{Ulum} \textbf{\textit{ndï}} \textit{ndïnji alo.}\\
\gll    {[ulum}  {\textbf{ndï}]}  {[ndï-nji}  ala      {wo]}\\
    {[palm}  {3\textsc{pl}]}  {[3\textsc{pl-poss}}  \textsc{pl.dist}  {own]}\\
\glt    ‘The sago palms are their very own.’ [ulwa014\_06:21]
\z

\ea%14
    \label{ex:det:14}
          \textit{Inom} \textbf{\textit{ndï}} \textit{wandam itom ala.}\\
\gll    {[inom}    {\textbf{ndï}]}  {[wandam}  itom  {ala]}\\
    {[mother}  {3\textsc{pl}]}  {[jungle}    father  {\textsc{pl.dist]}}\\
\glt `The mothers are the land owners.’ [ulwa014\_62:41]
\z

\ea%15
    \label{ex:det:15}
          \textit{Ya} \textbf{\textit{ndï}} \textit{ambi nji ala.}\\
\gll    {[ya}      {\textbf{ndï}]}  {[ambi}   nji    {ala]}\\
    {[coconut}  {3\textsc{pl]}}  {[big}  thing  {\textsc{pl.dist]}}\\
\glt `Coconuts are big things.’ [ulwa014\_08:06]
\z

\ea%16
    \label{ex:det:16}
          \textit{Supam Sinanam} \textbf{\textit{min}} \textit{atana wot.}\\
\gll    {[Supam}  Sinanam  {\textbf{min}]}  {[atana}      {wot]}\\
    {[[name]}  {[name]}    {3\textsc{du]}}  {[older.sister}  {younger]}\\
\glt `Supam and Sinanam were sisters.’ [ulwa001\_00:14]
\z

Example \REF{ex:det:16} further illustrates how \isi{coordinate}d subjects that lack any overt \isi{coordinator} (\sectref{sec:12.1}) may be clarified as such by means of the \isi{subject marker}. The \isi{subject marker} is all the more valuable in this regard when one or more of the members of the conjoined subject is left unexpressed. In example \REF{ex:det:17}, the \isi{dual} marker indicates that there are two subjects, even though only one is expressed.

\is{determiner|)}
\is{subject marker|)}


\is{determiner|(}
\is{subject marker|(}

\ea%17
    \label{ex:det:17}
          \textit{\textbf{Carobim min} wa mape.}\\
\gll    \textbf{Carobim}  \textbf{min}  wa    ma=p-e\\
    [name]    3\textsc{du}  village  3\textsc{sg.obj}=be\textsc{{}-ipfv}\\
\glt `Carobim and he [= Danny] were in the village.’ [ulwa037\_37:41]
\z

Thus, in addition to serving as an indicator of \isi{number}, the \isi{subject marker} can function as an \isi{associative plural} marker (or \isi{associative dual} marker) -- that is, when following a noun, a \isi{subject marker} can be used to give the interpretation of “that noun plus others associated with it”. In this way, the \isi{dual} \isi{subject marker} indexes that exactly one associated referent is to be understood, whereas the \isi{plural} \isi{subject marker} indexes that two or more additional referents are to be understood. This \isi{associative plural} function of subject markers (or \isi{object marker}s) is most commonly used with \isi{personal noun}s. An example with a \isi{plural} \isi{subject marker} is given in \REF{ex:det:18}.

\ea%18
    \label{ex:det:18}
          \textit{\textbf{Dorothy ndï} molop.}\\
\gll    \textbf{Dorothy}  \textbf{ndï}  ma=lo-p\\
    [name]    3\textsc{pl}  3\textsc{sg.obj}=go-\textsc{pfv}\\
\glt `Dorothy and the others went there.’ [ulwa042\_04:10]
\z

An example of an \isi{object marker} (\sectref{sec:7.3}) functioning as an \isi{associative plural} marker is given in \REF{ex:det:19}.

\ea%19
    \label{ex:det:19}
          \textbf{\textit{Otto}} \textbf{\textit{ndïkïna}}.\\
\gll \textbf{Otto}  \textbf{ndï}=kï-na\\
    [name]  3\textsc{pl}=say-\textsc{irr}\\
\glt `[We] will tell Otto and the others.’ [ulwa038\_04:08]
\z

In other circumstances, the \isi{subject marker} can help prevent a subject from being misinterpreted as being an object or \isi{oblique}. Since it is common for subjects to be omitted, the absence of a \isi{subject marker} could lead to such a miscue. The elicited sentence \REF{ex:det:20} is ambiguous, since \textit{yeta nungol} ‘boys’ could be interpreted either as the subject (without a \isi{subject marker}) or as the object (with a \isi{pro-drop}ped subject); in sentence \REF{ex:det:21}, on the other hand, the subject is clearly defined because it contains the \isi{subject marker}.

\ea%20
    \label{ex:det:20}
          \textit{Yeta nungol ndïnap le.}\\
\gll    yeta  nungol  ndï=nap  lo-e\\
    man  child  3\textsc{pl}=for  go-\textsc{ipfv}\\
\glt    (a) ‘The boys would go around on account of them [the girls].’\\
    (b) ‘[The girls] would go around on account of the boys.’ [elicited]
\z

\ea%21
    \label{ex:det:21}
          \textit{\textbf{Yeta nungol ndï} ndïnap le.}\\
\gll    \textbf{yeta}  \textbf{nungol}  \textbf{ndï}  ndï=nap  lo-e\\
    man  child  3\textsc{pl}  \textsc{3pl}=for  go-\textsc{ipfv}\\
\glt `The boys would go around on account of them [= the girls].’ [ulwa032\_44:07]
\z

Likewise the subject of sentence \REF{ex:det:22} is clear because it contains the \isi{subject marker}. The subject of the elicited sentence \REF{ex:det:23}, however, is ambiguous.

\ea%22
    \label{ex:det:22}
          \textit{\textbf{Inom mï} manji ay mamap.}\\
\gll    \textbf{inom}  \textbf{mï}      ma-nji      ay    ma=ama-p\\
    mother  3\textsc{sg.subj}  3\textsc{sg.obj-poss}  sago  3\textsc{sg.obj}=eat-\textsc{pfv}\\
\glt `Mother ate her sago.’ [ulwa032\_28:39]
\z

\ea%23
    \label{ex:det:23}
          \textit{Inom manji ay mamap.}\\
\gll    inom  ma-nji      ay    ma=ama-p\\
    mother  3\textsc{sg.obj-poss}  sago  3\textsc{sg.obj}=eat-\textsc{pfv}\\
\glt    (a) ‘Mother ate her sago.’\\
    (b) ‘[Someone] ate the mother’s sago.’ [elicited]
\z

A similar situation can also be seen in the pair of sentences \REF{ex:det:24} and \REF{ex:det:25}.

\ea%24
    \label{ex:det:24}
          \textit{\textbf{Nungol mï} ndala aw ndïnep.}\\
\gll    \textbf{nungol}  \textbf{mï}      ndï=ala  aw      ndï=ne-p\\
    child  3\textsc{sg.subj}  3\textsc{pl}=for  betel.nut  3\textsc{pl}=harvest-\textsc{pfv}\\
\glt `The child harvested betel nut for them.’ [ulwa014\_15:50]
\z

\ea%25
    \label{ex:det:25}
          \textit{Nungol ndala aw ndïnep.}\\
\gll    nungol  ndï=ala  aw      ndï=ne-p\\
    child  3\textsc{pl}=for  betel.nut  3\textsc{pl}=harvest-\textsc{pfv}\\
\glt    (a) ‘The child harvested betel nut for them.’\\
    (b) ‘[Someone] harvested betel nut for the children.’ [elicited]
\z


Although subject markers are used with great frequency and may be useful for marking \isi{number} or clarifying the subject in ambiguous circumstances, they do not occur in every subject NP. Their absence may be a simple omission (the product of casual speech), or it may rather be that their use is optional. Often, no difference in meaning can be detected between when subject markers are present and when they are absent. That said, there do seem to be some patterns underlying their absence. For example, subject markers seem more likely to be omitted when the subject is a \isi{proper noun}, as in examples \REF{ex:det:23bis} through \REF{ex:det:26}.

\ea%23
    \label{ex:det:23bis}
          \textit{Alkumot yana minkotïp.}\\
\gll    Alkumot  yana    Ø      min=kot-p\\
    [name]    woman    (\textsc{3sg.subj)}  \textsc{3du=}break-\textsc{pfv}\\
\glt `The woman Alkumot bore them.’ [ulwa001\_00:22]
\z

\ea%24
    \label{ex:det:24bis}
          \textit{Biwat atay.}\\
\gll    Biwat  Ø    ata-i\\
    [place]  (\textsc{3pl)}  up-go.\textsc{pfv}\\
\glt `The Biwat [people] went up.’ [ulwa002\_01:07]
\z

\ea%25
    \label{ex:det:25bis}
          \textit{Ambawanam Ngata i unip.}\\
\gll    Ambawanam  Ngata  Ø      i    uni-p\\
    [name]      grand  (\textsc{3sg.subj)}  go.\textsc{pfv}  shout-\textsc{pfv}\\
\glt `Ambawanam Ngata came and shouted.’ [ulwa008\_01:17]
\z

\ea%26
    \label{ex:det:26}
          \textit{Elias tïnanga.}\\
\gll    Elias  Ø      tïnanga\\
    [name]  (\textsc{3sg.subj)}  arise\\
\glt `Elias got up.’ [ulwa037\_16:05]
\z

Subject markers are perhaps also more likely to be omitted when the verb is \isi{intransitive}. This is perhaps unsurprising, since the role of the single core NP (i.e., subject) is easily determined by default in an \isi{intransitive} clause without needing any special marking. In other words, Ulwa appears to exhibit \isi{differential subject marking} in this regard. Subject markers may also be omitted more frequently when the referent of the subject is less definite, although I have not found any strict rules for their omission. In sentences \REF{ex:det:27} and \REF{ex:det:28}, the subjects are \isi{indefinite}. The subject NPs do not have subject markers.

\newpage

\ea%27
    \label{ex:det:27}
          \textit{Lamndu keka ndamap ulwap.}\\
\gll lamndu  Ø      keka      ndï=ama-p    ulwa=p\\
    pig      (\textsc{3sg.subj)}  completely  3\textsc{pl}=eat-\textsc{pfv}  nothing=\textsc{cop}\\
\glt `A pig completely ate them.’ [ulwa032\_22:12]
\z

\ea%28
    \label{ex:det:28}
          \textit{Ankam ulwap.}\\
\gll    ankam  Ø    ulwa=p\\
    person  (\textsc{3pl)}  nothing=\textsc{cop}\\
\glt `No one is left.’ (Literally ‘People are nothing.’) [ulwa037\_28:11]
\z

It may be noteworthy that it is common not to include \isi{object marker}s in \is{existential predication} \isi{negative} existential constructions, such as in \REF{ex:det:28}. This may suggest something of the definiteness-indexing nature of these markers. However, although rare, there are examples of subject markers being used in such \isi{negative} existential constructions \REF{ex:det:29}.

\is{subject marker|)}
\is{determiner|)}

\ea%29
    \label{ex:det:29}
          \textit{Kumba mo iye na \textbf{ndï} nap.}\\
\gll    Kumba  ma=u       i-e        na    \textbf{ndï}  ulwa    na-p\\
    Bun  3\textsc{sg.obj}=from  go.\textsc{pfv-dep}  talk  3\textsc{pl}  nothing  \textsc{detr}{}-be\\
\glt `There are no [more] stories about [me] coming from Bun [village].’ [ulwa032\_10:52]
\z

\section{Object markers (non-subject markers)}\label{sec:7.2}

\is{object marker|(}
\is{non-subject marker|(}
\is{determiner|(}

Like the set of \isi{subject marker}s, the set of “object markers” consists of postnominal determiners that occur as the final element in their respective NPs. The forms of the object markers are identical to the forms in the set of third-\isi{singular} objective \isi{personal pronoun}s (\sectref{sec:6.1}) -- thus, they are mostly identical to the set of \isi{subject marker}s. The main exception, however, is found in the \textsc{3sg} forms, which differ between the subject and object paradigms. Also, in certain \isi{phonological} environments, the \textsc{3du} forms are likewise distinct. The object markers \mbox{are given in \REF{ex:det:30}.}

\ea%30
    \label{ex:det:30}
          Object markers (non-subject markers)\\
\begin{tabbing}
{(\textit{min=} {\textasciitilde} \textit{mini=})} \= {(‘\textsc{3du}’ {\textasciitilde} \textsc{‘3du.obj’})}\kill
{\textit{ma=} {\textasciitilde} \textit{mo=}} \> {‘3\textsc{sg.obj}’}\\
{\textit{min=} {\textasciitilde} \textit{mini=}} \> {\textsc{‘3du’} \textsc{{\textasciitilde} ‘3du.obj}’}\\
{\textit{ndï=}} \> {‘3\textsc{pl}’}
\end{tabbing}
\z

The \isi{syntactic} distinction between \isi{subject marker}s and object markers is that \isi{subject marker}s are restricted to NPs that are grammatical subjects, whereas object markers occur in NPs with all other grammatical roles. Indeed, it would be more proper to refer to them as “non-subject makers”, since they occur in NPs with \isi{oblique} roles as well as in NPs with object roles, whether the object of a verb or the \isi{object of a postposition}. 

\isi{Phonological}ly, object markers are dependent on an immediately following verb, \isi{postposition}, or \isi{oblique marker} -- that is, they are \isi{proclitic}s, and, as, such, are glossed with an equal sign ‘=’, as opposed to the \isi{subject marker}s, which are glossed as unbound forms.\footnote{This \isi{phonological} distinction may be a further way of differentiating the paradigm of \isi{subject marker}s from the paradigm of object markers, although the situation is hazier than the glossing might suggest. First, although \isi{subject marker}s need not \isi{clitic}ize to following words, it is very common for them in effect to do so. This is especially common for the forms ending with the high \isi{central vowel} /ï/. For example, \textit{nï amun} ‘I … now’ is most typically pronounced [namun]; \textit{mï ango} ‘he … not’ is most typically pronounced [mango], and so on. Second, although object markers are consistently \isi{clitic}s when serving as markers of object or \isi{oblique} NPs, they need not \isi{clitic}ize when serving as markers in \isi{possessor} (\isi{genitive}) NPs (\sectref{sec:9.1.5}). Although the \isi{phonological} distinction between \isi{subject marker}s and object markers is not categorical, I propose that the object markers, which were probably originally free pronominal forms, have later undergone (or are currently undergoing) a \is{morphological change} grammatical change such that they are becoming \isi{bound morpheme}s.}

Object markers may follow either \is{common noun} common or \isi{proper noun}s, and they may have either \isi{animate} or \isi{inanimate} referents. The object markers -- whether 3\textsc{sg} \REF{ex:det:31}, 3\textsc{du} \REF{ex:det:32}, or 3\textsc{pl} \REF{ex:det:33} -- \isi{clitic}ize to the following verb.

\ea%31
    \label{ex:det:31}
          \textit{Inom mï utam} \textbf{\textit{mawanap}}.\\
\gll inom  mï      utam  \textbf{ma}=wana-p\\
    mother  3\textsc{sg.subj}  yam  3\textsc{sg.obj}=cook-\textsc{pfv}\\
\glt `Mother cooked the yam.’ [elicited]
\z

\ea%32
    \label{ex:det:32}
          \textit{Inom mï utam} \textbf{\textit{minwanap}}.\\
\gll inom  mï      utam  \textbf{min}=wana-p\\
    mother  3\textsc{sg.subj}  yam  3\textsc{du}=cook-\textsc{pfv}\\
\glt `Mother cooked two yams.’ [elicited]
\z

\ea%33
    \label{ex:det:33}
         \textit{Inom mï utam} \textbf{\textit{nduwanap}}.\\
\gll inom  mï      utam  \textbf{ndï=}wana-p\\
    mother  3\textsc{sg.subj}  yam  3\textsc{pl}=cook-\textsc{pfv}\\
\glt `Mother cooked the yams.’ [elicited]
\z

Example \REF{ex:det:33} illustrates the (optional) change of /ï/ to [u] before /w/ in the 3\textsc{pl} \isi{object marker} (\sectref{sec:2.5.6}). When a \isi{verb stem} begins with a \isi{vowel}, it is possible to witness  \is{elision} \isi{vowel elision} in the \isi{object marker} (\sectref{sec:2.5.5}), when it is \textsc{3sg} \REF{ex:det:34} or 3\textsc{pl} \REF{ex:det:36}, but not when it is \textsc{3du} \REF{ex:det:35}.

\ea%34
    \label{ex:det:34}
          \textit{Tïn ndï lamndu} \textbf{\textit{masap}}.\\
\gll tïn    ndï  lamndu  \textbf{ma}=asa-p\\
    dog  3\textsc{pl}  pig      3\textsc{sg.obj}=hit-\textsc{pfv}\\
\glt `The dogs killed the pig.’ [elicited]
\z

\ea%35
    \label{ex:det:35}
          \textit{Tïn ndï lamndu} \textbf{\textit{minasap}}.\\
\gll tïn    ndï  lamndu  \textbf{min}=asa-p\\
    dog  3\textsc{pl}  pig      3\textsc{du}=hit-\textsc{pfv}\\
\glt `The dogs killed two pigs.’ [elicited]
\z

\ea%36
    \label{ex:det:36}
          \textit{Tïn ndï lamndu} \textbf{\textit{ndasap}}.\\
\gll tïn    ndï  lamndu  \textbf{ndï=}asa-p\\
    dog  3\textsc{pl}  pig      3\textsc{pl}=hit-\textsc{pfv}\\
\glt `The dogs killed the pigs.’ [elicited]
\z

In example \REF{ex:det:37}, it is possible to see how [mo=], an \isi{allomorph} of the 3\textsc{sg} \isi{object marker} /ma=/, appears before a following /o/ in the \isi{verb stem} (\sectref{sec:2.5.7}). No similar \isi{allomorphy} occurs with \textsc{3du} \REF{ex:det:38} or \textsc{3pl} \REF{ex:det:39} object markers when in the same environment.

\ea%37
    \label{ex:det:37}
          \textit{Nï nïpïl} \textbf{\textit{momoplïp}}.\\
\gll nï    nïpïl  \textbf{ma}=mop-lï-p\\
    1\textsc{sg}  vine  3\textsc{sg.obj}=tie-put-\textsc{pfv}\\
\glt `I tied the rope.’ [elicited]
\z

\ea%38
    \label{ex:det:38}
          \textit{Nï nïpïl} \textbf{\textit{minmoplïp}}.\\
\gll nï    nïpïl  \textbf{min}=mop-lï-p\\
    1\textsc{sg}  vine  3\textsc{du}=tie-put-\textsc{pfv}\\
\glt `I tied two ropes.’ [elicited]
\z

\ea%39
    \label{ex:det:39}
          \textit{Nï nïpïl} \textbf{\textit{ndïmoplïp}}.\\
\gll nï    nïpïl  \textbf{ndï}=mop-lï-p\\
    1\textsc{sg}  vine  3\textsc{pl}=tie-put-\textsc{pfv}\\
\glt `I tied the ropes.’ [elicited]
\z

This form [mo=] only occurs as an \isi{allomorph} of the \isi{object marker}, never as an \isi{allomorph} of the \isi{subject marker}. Thus, whereas the 3\textsc{sg} form is realized as [mo=] when serving as an object of a verb beginning with /Co/, the 3\textsc{sg} form is never realized as \textsuperscript{†}[mo] when serving as a subject that immediately precedes a verb beginning with /Co/. For example, whereas the \isi{verb phrase} /ma=kot-p/ ‘broke it’ is pronounced [\textbf{mo}kotïp], the clause /mï wo-p/ ‘she slept’ is pronounced [\textbf{mï} wop] (and not \textsuperscript{†}[mo wop]).\footnote{The difference in behavior between the object and subject forms could be explained by a \isi{phonological} rule that affects /a/ but not /ï/ in this environment, namely: \newline /a/ → [o] / C [+\isi{labial}] \_ C\textsubscript{0}V [-high, +back].}

  Object markers may occur with verbs marked for any \isi{TAM} category, and there are no distinctions in the markers based on such \isi{TAM} distinctions. Thus they occur in \isi{imperfective} \REF{ex:det:40}, \isi{perfective} \REF{ex:det:41}, and \isi{irrealis} \REF{ex:det:42} \isi{verb phrase}s. Any \isi{transitive} verb can be preceded by one of these \isi{object-marking} \isi{proclitic}s.

\ea%40
    \label{ex:det:40}
          \textit{Tïn mï mïnda} \textbf{\textit{mame}}.\\
\gll tïn    mï      mïnda  \textbf{ma}=ama-e\\
    dog  3\textsc{sg.subj}  banana  3\textsc{sg.obj}=eat-\textsc{ipfv}\\
\glt `The dog is eating the banana.’ [elicited]
\z

\ea%41
    \label{ex:det:41}
          \textit{Tïn mï mïnda} \textbf{\textit{mamap}}.\\
\gll tïn    mï      mïnda    \textbf{ma}=ama-p\\
    dog  3\textsc{sg.subj}  banana    3\textsc{sg.obj}=eat-\textsc{pfv}\\
\glt `The dog ate the banana.’ [elicited]
\z

\ea%42
    \label{ex:det:42}
          \textit{Tïn mï mïnda} \textbf{\textit{malanda}}.\\
\gll tïn    mï      mïnda  \textbf{ma}=la-nda\\
    dog  3\textsc{sg.subj}  banana  3\textsc{sg.obj}=eat-\textsc{irr}\\
\glt `The dog will eat the banana.’ [elicited]
\z

   The 3\textsc{du} \isi{object marker} /min=/ has the \isi{allomorph} [mini] when preceding a verb with \isi{stem} beginning in /n/.\footnote{This \isi{allomorphy} offers further support for the claim that the status of object markers is distinct from that of simple pronominal forms.
   Although it may be possible to explain the \isi{allomorph} [mo=] (for \textit{ma=} ‘3\textsc{sg.obj’)} in terms of simple \isi{phonological} conditioning (that is, without considering \isi{morphology}, \sectref{sec:2.5.7}), the form \textit{mini=} ‘3\textsc{du.obj’} is clearly a \isi{morphological}ly conditioned change, since, elsewhere, consecutive \isi{consonant}s are simply \isi{degeminate}d. Thus, one should expect the \isi{allomorph}, were it \isi{phonological}ly conditioned, to be \textsuperscript{†}[mi=]. Indeed, the form \textit{mini=} ‘3\textsc{du.obj’} only appears before verbs, not even before \isi{postposition}s, suggesting that \isi{object-marker} \isi{clitic}s for verbs are somewhat more closely affiliated with their hosts than are \isi{object-marker} \isi{clitic}s for \isi{postposition}s.} The \isi{allomorph} \textit{mini=} ‘3\textsc{du.obj’} can thus be observed when the morpheme immediately precedes an initial /n/ of a \isi{verb stem} \REF{ex:det:43}, but not when it precedes other \isi{consonant}s (e.g., /l/) \REF{ex:det:44}.

\is{determiner|)}
\is{non-subject marker|)}
\is{object marker|)}
\is{object marker|(}
\is{non-subject marker|(}
\is{determiner|(}

\ea%43
    \label{ex:det:43}
          \textit{Itom mï inmi} \textbf{\textit{mininkap}}.\\
\gll itom  mï      inmi  \textbf{mini}=nïkï-p\\
    father  3\textsc{sg.subj}  hole  3\textsc{du.obj}=dig=\textsc{pfv}\\
\glt `Father dug two holes.’ [elicited]
\z

\ea%44
    \label{ex:det:44}
          \textit{Itom mï num} \textbf{\textit{minlop}}.\\
\gll itom  mï      num  \textbf{min}=lo-p\\
    father  3\textsc{sg.subj}  canoe  3\textsc{du}=cut-\textsc{pfv}\\
\glt `Father carved two canoes.’ [elicited]
\z


The fact that \textsuperscript{†}[mini] is not produced from /min/ in contexts other than those in which it directly precedes a verb can be illustrated by comparing two examples taken from texts: example \REF{ex:det:45} shows [mini] before a verb beginning with /n/, whereas \REF{ex:det:46} shows the pronominal form as a free \isi{subject marker} [min] preceding a word beginning with /n/. Only example \REF{ex:det:45} exhibits the form [mini].

\ea%45
    \label{ex:det:45}
          \textit{Wondi inom min ndï} \textbf{\textit{mininke}} \textit{isi up.}\\
\gll    wondi    inom  min  ndï  \textbf{mini}=nïkï-e    isi    u-p\\
    bandicoot  mother  \textsc{3du}  3\textsc{pl}  3\textsc{du.obj}=dig{}-\textsc{dep}  soup  put-\textsc{pfv}\\
\glt `The two mother bandicoots – they cut them up into the soup.’ [ulwa032\_20:20]
\z

\ea%46
    \label{ex:det:46}
          \textit{\textbf{Min num} si nïn ata lïp.}\\
\gll    \textbf{min}  \textbf{num}  si    nï=n    ata  lï-p\\
    3\textsc{du}  canoe  push  1\textsc{sg=obl}  up  put-\textsc{pfv}\\
\glt `The two of them came ashore with me.’ (Literally ‘put the canoe up with me’) [ulwa032\_22:18]
\z

Similarly, the form \textsuperscript{†}[mini] does not occur before \isi{postposition}s that begin with /n/ \REF{ex:det:47}.

\ea%47
    \label{ex:det:47}
          \textit{Unji yenat ngin} \textbf{\textit{minap}} \textit{mana na.}\\
\gll    u-nji    yenat    ngin    \textbf{min}=nap  ma-na  na[-kï-p]\\
    2\textsc{sg-poss}  daughter  \textsc{du.prox}  \textsc{3du=}for  go-\textsc{irr}  \textsc{detr[}{}-say-\textsc{pfv]}\\
\glt `These two daughters of yours -- [I] wanted to go on account of them.’ [ulwa037\_49:28]
\z

There are, however, admittedly few examples of [mini] in the Ulwa corpus of texts. This is not surprising, given the rarity both of \isi{dual} referents and of \isi{verb stem}s beginning with /n/. Speakers do, however, consistently produce the form in elicitation.

  Although helpful in designating the \isi{number} of referents in an object NP, object markers, like \isi{subject marker}s (\sectref{sec:7.1}), are not always included in their respective NPs. Again, their absence may be a simple omission, the product of casual speech. Their omission does, however, seem to be more likely when the referent is less definite, but no clear correlation has been found in the corpus. In other words, Ulwa may exhibit a form of \is{object marking} \isi{differential object marking} in addition to exhibiting \isi{differential subject marking} (\sectref{sec:7.1}). The objects of examples \REF{ex:det:48} and \REF{ex:det:49} are both \isi{indefinite}; in \REF{ex:det:48} the object receives the \isi{object marker} (for each of two verbs), whereas in \REF{ex:det:49} it does not.

\ea%48
    \label{ex:det:48}
\textit{Yawa ndï anasa \textbf{maytape} \textbf{mat} mananda.}\\
\gll    yawa  ndï  anasa    \textbf{ma}=ita-p-e        \textbf{ma}=tï ma=na-nda\\
    uncle  3\textsc{pl}  pick.axe  3\textsc{sg.obj}=build-\textsc{pfv-dep}  3\textsc{sg.obj}=take    3\textsc{sg.obj}=give-\textsc{irr}\\
\glt `The uncles will make a pick-axe and give it to her.’ [ulwa022\_00:15]
\z

\ea%49
    \label{ex:det:49}
          \textit{Ndï tïmbïl itap.}\\
\gll    ndï  tïmbïl  ita-p\\
    3\textsc{pl}  fence  build-\textsc{pfv}\\
\glt `They built a fence.’ [ulwa014\_44:09]
\z

Another set of examples with \isi{indefinite} object NPs contrasts the absence \REF{ex:det:50} and presence \REF{ex:det:51} of the \isi{object marker}.

\ea%50
    \label{ex:det:50}
          \textit{Un ay nïkap?}\\
\gll    un  ay    nïkï-p\\
    2\textsc{pl}  sago  dig{}-\textsc{pfv}\\
\glt    ‘Did you make sago?’ [ulwa018\_04:21]
\z

\ea%51
    \label{ex:det:51}
          \textit{Imba nape ay} \textbf{\textit{ndïnkap}} \textbf{\textit{ndïn}} \textit{amblan up.}\\
\gll    imba  na-p-e      ay    \textbf{ndï}=nïkï-p    \textbf{ndï}=n    ambla=n u-p\\
    night  \textsc{detr-}be\textsc{{}-dep} sago  3\textsc{pl}=dig{}-\textsc{pfv}  3\textsc{pl=obl}  \textsc{pl.refl=obl}    put-\textsc{pfv}\\
\glt `At night [they] made sago [packets] and left them for themselves.’ [ulwa014\_49:20]
\z


In addition to appearing as the final element in verbal object NPs (that is, immediately preceding verbs), object markers occur as the final elements of NPs that are the objects of \isi{postposition}s, as in examples \REF{ex:det:52} through \REF{ex:det:56}.

\ea%52
    \label{ex:det:52}
          \textit{Kayngam i ya} \textbf{\textit{maya}} \textit{atay.}\\
\gll    Kayngam  i    ya      \textbf{ma}=iya      ata-i\\
    [name]    go.\textsc{pfv}  coconut  3\textsc{sg.obj}=toward  up-go.\textsc{pfv}\\
\glt `Kayngam went, climbed up a coconut tree.’ [ulwa018\_01:56]
\z

\ea%53
    \label{ex:det:53}
\textit{Imba pe nï wolka tawatïp} \textbf{\textit{ndiya}} \textit{i.}\\
\gll    imba  p-e    nï    wolka  tawatïp  \textbf{ndï=}iya    i\\
    night  be-\textsc{dep}  \textsc{1sg} again  child  3\textsc{pl=}toward  go.\textsc{pfv}\\
\glt `That night, I again went to the young folks.’ [ulwa037\_06:33]
\z

\ea%54
    \label{ex:det:54}
          \textit{Tïlwa} \textbf{\textit{mo}} \textit{i wa mbi.}\\
\gll    tïlwa  \textbf{ma}=u      i    wa    mbï-i\\
    road  3\textsc{sg.obj}=from  go.\textsc{pfv}  village  here-go.\textsc{pfv}\\
\glt `[We] came along the path here to the village.’ [ulwa032\_04:57]
\z

\ea%55
    \label{ex:det:55}
          \textit{Manji yawa} \textbf{\textit{minul}} \textit{i.}\\
\gll    ma-nji      yawa  \textbf{min}=ul    i\\
    3\textsc{sg.obj-poss}  uncle  \textsc{3du}=with   go.\textsc{pfv}\\
\glt `[He] went with his two uncles.’ [ulwa014\_49:50]
\z

\ea%56
    \label{ex:det:56}
          \textit{An wolka ngata} \textbf{\textit{ndul}} \textit{iye.}\\
\gll    an      wolka  ngata  \textbf{ndï=}ul    i-e\\
    1\textsc{pl.excl}  again  grand  3\textsc{pl}=with  go.\textsc{pfv-dep}\\
\glt `We again went with the ancestors.’ [ulwa002\_03:40]
\z

\is{determiner|)}
\is{non-subject marker|)}
\is{object marker|)}

\is{object marker|(}
\is{non-subject marker|(}
\is{determiner|(}

Object markers are also found in NPs marked with the \isi{oblique marker} \textit{=n} \textsc{‘obl’}, as in \REF{ex:det:57} and \REF{ex:det:58}.

\ea%57
    \label{ex:det:57}
          \textit{Ay} \textbf{\textit{man}} \textit{mïnanap.}\\
\gll    ay    \textbf{ma}=n      mï=na-na-p\\
    sago  3\textsc{sg.obj}=\textsc{obl}  3\textsc{sg.subj}=\textsc{detr}{}-feed-\textsc{pfv}\\
\glt `[They] fed him with the sago.’ [ulwa011\_01:23]
\z

\ea%58
    \label{ex:det:58}
          \textit{An mïnda} \textbf{\textit{ndïn}} \textit{malan up ndamap.}\\
\gll    an       mïnda  \textbf{ndï}=n    manal    u-p      ndï=ama-p\\
    1\textsc{pl.excl}  banana  3\textsc{pl=obl}  hot.water  put-\textsc{pfv}  3\textsc{pl}=eat-\textsc{pfv}\\
\glt `We boiled bananas and ate them.’\footnote{The speaker \is{metathesis} metathesizes the /n/ and /l/ in \textit{manal} ‘hot water’ (here: [malan]).} (Literally ‘put bananas in hot water’) [ulwa032\_07:44]
\z


In addition to the three object markers used for indexing what are usually definite referents (whether \isi{singular}, \isi{dual}, or \isi{plural}), there is a (third-\isi{singular}) \isi{indefinite} marker, \textit{ko=} ‘\textsc{indf}’, clearly derived from the \isi{numeral} \textit{kwa} {\textasciitilde} \textit{kwe} ‘one’. I consider this form to be a sort of \isi{object marker}, both because it tends to \isi{clitic}ize to the following verb, \isi{postposition}, or \isi{oblique marker} and because it never appears in subject NPs: only the forms /kwa/ or /kwe/ may appear in this position (and when they do they have definite reference, i.e., ‘one’). Given that it is not found with nouns in subject NPs, but is rather thus restricted in use, the \isi{indefinite} marker \textit{ko=} ‘\textsc{indf}’ is not considered to be an “\isi{indefinite} article”.\footnote{Note, however, that the form [ko] can occur as a free morpheme. When it does, it is to be interpreted as the \isi{modal adverb} \textit{ko} ‘just’, not as the \isi{indefinite} \isi{object marker} \textit{ko=} ‘\textsc{indf}’, with which it is probably etymologically related. Perhaps another reason for not considering \textit{ko=} ‘\textsc{indf}’ to be an article is that it is never obligatory.} The \isi{indefinite} \isi{object marker} is illustrated by examples \REF{ex:det:59} through \REF{ex:det:63}.


\ea%59
    \label{ex:det:59}
          \textit{Ala nï nji \textbf{kosap}!}\\
\gll    ala      nï    nji    \textbf{ko}=asa-p\\
    \textsc{pl.dist}  1\textsc{sg}  thing  \textsc{indf}=hit-\textsc{pfv}\\
\glt `Guys, I killed something!’ [ulwa035\_03:39]
\z

\ea%60
    \label{ex:det:60}
          \textit{Nï ango wolka nungolke} \textbf{\textit{kotïn}}.\\
\gll nï    ango  wolka  nungolke  \textbf{ko}=tï-n\\
    1\textsc{sg}  \textsc{neg}  again  child    \textsc{indf=}take-\textsc{pfv}\\
\glt `I didn’t have another child.’ [ulwa036\_00:24]
\z

\ea%61
    \label{ex:det:61}
          \textit{Kayngam Kayngam wam ngatï ma ya \textbf{koya} ma!}\\
\gll    Kayngam  Kayngam  wam  nga=tï      ma  ya \textbf{ko}=iya      ma\\
    [name]    [name]    strap  \textsc{sg.prox}=take  go  coconut    \textsc{indf=}toward  go\\
\glt `Kayngam, Kayngam, go get this tree-climbing strap and go up a coconut tree!’ [ulwa018\_01:30]
\z


\ea%62
    \label{ex:det:62}
          \textit{Plas mï ango ma in nji} \textbf{\textit{kon}} \textit{mbïlp.}\\
\gll    Plas  mï      ango  ma      i=n      nji    \textbf{ko}=n mbï-lï{}-p\\
    [name]  3\textsc{sg.subj}  \textsc{neg}  3\textsc{sg.obj}  hand=\textsc{obl}  thing  \textsc{indf}=\textsc{obl}  here-put-\textsc{pfv}\\
\glt `Plas didn’t plant anything here with his [own] hands.’ [ulwa014†]
\z

\ea%63
    \label{ex:det:63}
          \textit{Ndï ango wondi} \textbf{\textit{kotïn}}.\\
\gll ndï  ango  wondi    \textbf{ko}=tï-n\\
    3\textsc{pl}  \textsc{neg}  bandicoot  \textsc{indf}=take-\textsc{pfv}\\
\glt `They didn’t get a [single] bandicoot.’ [ulwa032\_25:28]
\z

This marker is commonly used in demands or \isi{request}s to be given something, such as the very common \isi{request} to be passed betel nut \REF{ex:det:64}.
\ea%64
    \label{ex:det:64}
          \textit{Aw} \textbf{\textit{kot}} \textit{nïnan!}\\
\gll    aw      \textbf{ko}=tï  nï=na-n\\
    betel.nut  \textsc{indf}=take  1\textsc{sg}=give-\textsc{imp}\\
\glt `Please pass the betel nut!’ [elicited]
\z


Sentence \REF{ex:det:65} likewise illustrates the use of the \isi{indefinite} marker \textit{ko=} ‘\textsc{indf}’ in a demand to be given something.

\ea%65
    \label{ex:det:65}
          \textit{Kaw nungol} \textbf{\textit{kot}} \textit{nïnata!}\\
\gll    kaw  nungol  \textbf{ko}=tï  nï=na-ta\\
    cow  child  \textsc{indf}=take  1\textsc{sg}=give-\textsc{cond}\\
\glt `Give me a calf!’ (\textit{kaw} = TP \textit{kau}) [ulwa014\_09:19]
\z

Although etymologically related to the form \textit{kwa} ‘one’, the definite numerical sense of ‘one’ is generally not felt in the \isi{object marker}. Rather, to give the sense of ‘(exactly) one’, the \isi{numeral} itself is used, followed by a 3\textsc{sg} marker, as in examples \REF{ex:det:66}, \REF{ex:det:67}, and \REF{ex:det:68}.

\ea%66
    \label{ex:det:66}
          \textit{Mï may ndimbam lop} \textbf{\textit{kwa}} \textbf{\textit{molop}} \textit{lïp malep.}\\
\gll    mï      ma=i        ndï=imbam  lo-p  \textbf{kwa} \textbf{ma}=lo-p      lï-p      ma=ale-p\\
    3\textsc{sg.subj}  3\textsc{sg.obj}=go.\textsc{pfv}  3\textsc{pl}=under    go-\textsc{pfv}  one    3\textsc{sg.obj}=cut-\textsc{pfv}  put-\textsc{pfv}  3\textsc{sg.obj}=scrape-\textsc{pfv}\\
\glt `She went there, went under them, cut one [= a palm] down, and scraped it.’ [ulwa032\_37:39]
\z

\ea%67
    \label{ex:det:67}
          \textit{Nï} \textbf{\textit{kwa}} \textbf{\textit{mol}} \textit{ne}\\
\gll    nï    \textbf{kwa}  \textbf{ma}=ul      ni-e\\
    1\textsc{sg}  one    3\textsc{sg.obj}=with  act-\textsc{ipfv}\\
\glt `I was making one [= an armband].’ [ulwa015\_01:39]
\z

\ea%68
    \label{ex:det:68}
          \textit{Nïnji wot yana} \textbf{\textit{kwa}} \textbf{\textit{mï}} \textit{nip.}\\
\gll    nï-nji    wot      yana  \textbf{kwa}  \textbf{mï}       ni-p\\
    1\textsc{sg-poss}  younger  woman  one    3\textsc{sg.subj}  die-\textsc{pfv}\\
\glt `One younger sister of mine has died.’ [ulwa028\_00:19]
\z

\is{determiner|)}
\is{non-subject marker|)}
\is{object marker|)}

\is{object marker|(}
\is{non-subject marker|(}
\is{determiner|(}

Sometimes the only expressed element in an object NP (whether the \isi{direct object} of a \isi{transitive} verb or the object preceding a \isi{postposition} or \isi{oblique marker}) is an \isi{object marker}. Since these are identical in form to third \isi{person} non-subject personal pronominal forms and since first \isi{person} and second \isi{person} \isi{pronoun}s may also occur in these positions, it is probably most parsimonious to view these all as \isi{pronoun}s. That is, when no nominal is expressed in an object NP consisting solely of the form \textit{ma=} ‘3\textsc{sg.obj}’, \textit{min=} ‘3\textsc{du}’, or \textit{ndï=} ‘3\textsc{pl}’, these may be treated simply as object \isi{pronoun}s, as in examples \REF{ex:det:69} through \REF{ex:det:80}.

\ea%69
    \label{ex:det:69}
          \textit{Ndï} \textbf{\textit{mayte}}.\\
\gll ndï  \textbf{ma}=ita-e\\
    3\textsc{pl}  3\textsc{sg.obj}=build-\textsc{ipfv}\\
\glt `They were building it.’ [ulwa032\_09:27]
\z

\ea%70
    \label{ex:det:70}
          \textit{Unan} \textbf{\textit{maya}} \textit{mbiye.}\\
\gll    unan    \textbf{ma}=iya      mbï-i-e\\
    1\textsc{pl.incl}  3\textsc{sg.obj}=toward  here-go.\textsc{pfv-dep}\\
\glt `We came here to him.’ [ulwa037\_05:26]
\z

\ea%71
    \label{ex:det:71}
          \textit{Ndï nokoplïp lïmndï} \textbf{\textit{mala}}.\\
\gll ndï  nokop-lï-p    lïmndï  \textbf{ma}=ala\\
    3\textsc{pl}  hide-put-\textsc{pfv}  eye    \textsc{3sg.obj}=see\\
\glt `They hid and saw her.’ [ulwa020\_00:16]
\z


\ea%72
    \label{ex:det:72}
          \textit{Nï man} \textbf{\textit{mint}}.\\
\gll nï    ma=n      \textbf{min}=ta\\
    1\textsc{sg}  3\textsc{sg.obj=obl}  \textsc{3du=}say\\
\glt `I told them.’ [ulwa014\_72:19]
\z

\ea%73
    \label{ex:det:73}
          \textit{Nï ango} \textbf{\textit{ndïtïn}}.\\
\gll nï    ango  \textbf{ndï}=tï-n\\
    1\textsc{sg}  \textsc{neg}  3\textsc{pl}=take-\textsc{pfv}\\
\glt `I didn’t get them.’ [ulwa037\_17:55]
\z

\ea%74
    \label{ex:det:74}
          \textit{Mï} \textbf{\textit{nasape}}.\\
\gll mï      \textbf{nï}=asa-p-e\\
    3\textsc{sg.subj}  1\textsc{sg}=hit-\textsc{pfv-dep}\\
\glt `He hit me.’ [ulwa014\_26:00]
\z

\ea%75
    \label{ex:det:75}
          \textit{Nga mïnjikan} \textbf{\textit{ngant}}.\\
\gll nga      mïnjika=n    \textbf{ngan}=ta\\
    \textsc{sg.prox}  speech=\textsc{obl}  1\textsc{du.excl}=say\\
\glt `This one spoke to us.’ [ulwa014\_12:31]
\z

\ea%76
    \label{ex:det:76}
          \textit{Wondi andat} \textbf{\textit{ngunanata}} \textit{ngunan matïm.}\\
\gll    wondi    anda=tï    \textbf{ngunan}=na-ta    ngunan ma=atï-m\\
    bandicoot  \textsc{sg.dist}=take  1\textsc{du.incl}=give-\textsc{cond}  1\textsc{du.incl}           3\textsc{sg.obj}=hit-\textsc{irr}\\
\glt       ‘When [he] gives us that bandicoot, we will kill it.’ [ulwa029\_05:33]
\z

\ea%77
    \label{ex:det:77}
          \textit{Yalum un yanat un ango kïkal} \textbf{\textit{anwana}}.\\
\gll yalum    un  yanat    un  ango  kïkal  \textbf{an}=wana\\
    grandchild  \textsc{2pl}  daughter  \textsc{2pl}  \textsc{neg}  ear    \textsc{1pl.excl}=feel\\
\glt `You granddaughters and you daughters don’t listen to us.’ [ulwa014\_37:01]
\z

\ea%78
    \label{ex:det:78}
          \textit{Ndï kïkal} \textbf{\textit{unanwana}} \textit{mïnja m!}\\
\gll    ndï  kïkal  \textbf{unan}=wana  mïnja  m\\
    3\textsc{pl}  ear    1\textsc{pl.incl}=feel  speech  hm\\
\glt `They will hear us and say: “Hm!”’ [ulwa037\_65:11]
\z

\ea%79
    \label{ex:det:79}
          \textit{Ngan} \textbf{\textit{nguniya}} \textit{men iye.}\\
\gll    ngan    \textbf{ngun}=iya    ma=in      i-e\\
    1\textsc{du.excl}  2\textsc{du}=toward  3\textsc{sg.obj}=in  go.\textsc{pfv-dep}\\
\glt `We came to you in it.’ [ulwa014\_15:31]
\z

\ea%80
    \label{ex:det:80}
          \textit{Nï} \textbf{\textit{unul}} \textit{wa mana.}\\
\gll    nï    \textbf{un}=ul    wa    ma-na\\
    1\textsc{sg}  2\textsc{pl}=with  village  go-\textsc{irr}\\
\glt `I will go with you to the village.’ [ulwa037\_40:28]
\z

Similarly, the set of \isi{reflexive} (or \isi{reciprocal}) forms, when \isi{clitic}izing to verbs or \isi{postposition}s (or when preceding \isi{oblique marker}s), may simply be considered to be \isi{pronoun}s (see examples in \sectref{sec:6.3}).

  \isi{Subject marker}s and object markers in Ulwa are discussed in \citet[4--7]{Barlow2019a}. Similar NP-final determiners that are found elsewhere in the \ili{Keram-Ramu} family are discussed in \citet[51--54]{KillianBarlow2022}, where they are referred to broadly as “articles”. In Ulwa at least, although these determiners may in some ways function to mark \isi{specificity} or \isi{definiteness}, they probably have more to do with indicating topic or focus, in addition to serving a \isi{number}-indexing function. Thus, Ulwa may be said to lack both definite and \isi{indefinite} articles.


\is{determiner|)}
\is{non-subject marker|)}
\is{object marker|)}

\newpage

\section{Demonstratives}\label{sec:7.3}

\is{demonstrative|(}
\is{deixis|(}
\is{determiner|(}

Ulwa makes a two-way \isi{deictic} distinction within its set of \isi{demonstrative} words: \isi{proximal} referents (near the speaker) versus \isi{distal} referents (not near the \linebreak speaker). The Ulwa \isi{deictic} system is thus \isi{egocentric}. In this relatively simple near-versus-far contrast, \isi{deictic} words do not encode other possible distinctions, such as those based on \isi{elevation} or \isi{visibility}. However, demonstratives in Ulwa also index \isi{number}: \isi{singular}, \isi{dual}, or \isi{plural}. There are thus six \isi{demonstrative} determiners, as shown in \tabref{tab:7.1}.

\begin{table}
\caption{Demonstratives}
\label{tab:7.1}
\begin{tabularx}{.7\textwidth}{lQQQ}
\lsptoprule
& {\scshape sg} & {\scshape du} & {\scshape pl}\\
\midrule
{\scshape prox} & {\itshape nga} & {\itshape ngin} & {\itshape ngala}\\
{\scshape dist} & {\itshape anda} & {\itshape andin} & {\itshape ala}\\
\lspbottomrule
\end{tabularx}
\end{table}
As seen in \tabref{tab:7.1}, the \isi{proximal} forms all contain the element /ng-/, which combines with /-a/ in the \isi{singular} (cf. \textit{ma}= ‘3\textsc{sg.obj’}), /-in/ in the \isi{dual} (cf. \textit{min=} ‘3\textsc{du’}), and /-ala/ in the \isi{plural}. This last form does not correspond to anything in the other sets of pronominal forms. The \isi{distal} forms, on the other hand, contain the element /and-/, which -- as in the \isi{proximal} forms -- combines with /-a/ in the \isi{singular} and /-in/ in the \isi{dual}. Demonstrative determiners do not co-occur with \isi{subject marker}s or with \isi{object marker}s.

These usually occur in the same spot that otherwise might contain \isi{subject marker}s (\sectref{sec:7.1}) or \isi{object marker}s (\sectref{sec:7.2}). The use of these \isi{demonstrative} determiners instead of other markers may signal that a specific (as opposed to a generic) referent is being identified. In addition to functioning as determiners (i.e., as elements of NPs), the Ulwa \isi{demonstrative} forms may also be used as \isi{pronoun}s. Of the six forms, the \isi{plural} \isi{distal} form \textit{ala} ‘\textsc{pl.dist}’ (‘those’) is most commonly used in this way, often functionally equivalent to ‘they’ or ‘them’. 

  In examples \REF{ex:det:81} through \REF{ex:det:86}, \isi{demonstrative} determiners occur as the final elements of subject NPs. They all have spatial \isi{deictic} force, pointing to referents either near or far. They may occur with either \is{common noun} common or \isi{proper noun}s. Although nouns are not marked in any way for \isi{number}, adnominal demonstratives agree with their nouns’ covert \isi{number}: \isi{singular}, \isi{dual}, or \isi{plural}.

\newpage

\ea%81
    \label{ex:det:81}
          \textit{Inom} \textbf{\textit{nga}} \textit{mawanape.}\\
\gll    inom  \textbf{nga}    ma=wana-p-e\\
    mother  \textsc{sg.prox}  3\textsc{sg.obj}=cook-\textsc{pfv-dep}\\
\glt `This woman cooked it.’ (This was said of the woman in the house next to where the speaker was sitting.) [ulwa014\_05:54]
\z

\ea%82
    \label{ex:det:82}

          \textit{Wandam} \textbf{\textit{nga}} \textit{ambi ngatap.}\\
\gll    wandam  \textbf{nga}    ambi  ngata=p\\
    jungle    \textsc{sg.prox}  big    grand=\textsc{cop}\\
\glt `This garden is very big.’ [ulwa042\_03:25]
\z

\ea%83
    \label{ex:det:83}
          \textit{itom} \textbf{\textit{ngin}} \textit{li ngapen}\\
\gll    itom  \textbf{ngin}    li    nga=p-en\\
    father  \textsc{du.prox}  down  \textsc{sg.prox}=be\textsc{{}-nmlz}\\
\glt `these two men who live downstream’ (spoken while downstream) [ulwa014†]
\z

\ea%84
    \label{ex:det:84}
          \textit{Wa mbï olsem nungolke} \textbf{\textit{ngala}} \textit{sikul pe.}\\
\gll    wa    mbï  olsem  nungolke  \textbf{ngala}    sikul  p-e\\
    village  here  thus  child    \textsc{pl.prox}  school  be\textsc{{}-ipfv}\\
\glt `Here in the village, like, these children are in school.’ (\textit{olsem} = TP; \textit{sikul} < TP \textit{skul} ‘school’) [ulwa027\_00:19]
\z

\ea%85
    \label{ex:det:85}
          \textit{Wusim} \textbf{\textit{anda}} \textit{nïwalinda i nï masap.}\\
\gll    wusim    \textbf{anda}    nï=wali-nda  i    nï    ma=asa-p\\
    crocodile  \textsc{sg.dist}  \textsc{1sg}=hit-\textsc{irr}  \textsc{pred}  \textsc{1sg}  \textsc{3sg.obj=}hit-\textsc{pfv}\\
\glt `That crocodile could have killed me, but I killed it.’ (\textit{i} = TP?) [ulwa035\_03:41]
\z

\ea%86
    \label{ex:det:86}
          \textit{Awngala} \textbf{\textit{la}} \textit{kuk ato im andawatawe.}\\
\gll    awngala  \textbf{ala}      kuk  ata-u    im anda=wat-aw-e\\
    bird.species    \textsc{pl.dist}  gather  up-from  tree  \textsc{sg.dist}=atop-put\textsc{{}-ipfv}\\
\glt `Those birds are gathering up into that tree.’ [ulwa037\_47:12]
\z

Demonstratives may function pronominally, as in \REF{ex:det:87} and \REF{ex:det:88},

\ea%87
    \label{ex:det:87}
          \textbf{\textit{Anda}} \textit{man ute: …}\\
\gll    \textbf{anda}    ma=n      u=ta-e\\
    \textsc{sg.dist}  \textsc{3sg.obj=obl}  \textsc{2sg=}say-\textsc{dep}\\
\glt `That one told you: …’ [ulwa014\_07:58]
\z

\ea%88
    \label{ex:det:88}
          \textbf{\textit{Andin}} \textit{wot kokot nangani nalïp.}\\
\gll    \textbf{andin}    wot    ko=kot      nï=angani    na-lï-p\\
    \textsc{du.dist}  younger  \textsc{indf}=break  1\textsc{sg}=behind  \textsc{detr-}put-\textsc{pfv}\\
\glt `Those two [= my parents] bore a younger sibling after me.’ [ulwa001\_10:19]
\z

As a spatial \isi{deictic} word, \textit{nga} ‘\textsc{sg.prox}’ (‘this’) can mean ‘here’ \REF{ex:det:89}.

\ea%89
    \label{ex:det:89}
          \textit{\textbf{Nga} unji ani ngala ata \textbf{ngap}.}\\
\gll    \textbf{nga}    u-nji    ani    ngala    ata  \textbf{nga}=p\\
    \textsc{sg.prox}  \textsc{2sg-poss}  bilum  \textsc{pl.prox}  up  \textsc{sg.prox}=be\\
\glt `Here, these \textit{bilum} [= string bags] of yours are up here.’ [ulwa001\_04:55]
\z

Example \REF{ex:det:89} also illustrates the common use of \isi{demonstrative} determiners in \is{possession} possessive \isi{phrase}s, further illustrated by examples \REF{ex:det:90} through \REF{ex:det:93}.

\ea%90
    \label{ex:det:90}
         \textit{\textbf{Nïnji nungol ngala} mbïpe.}\\
\gll    \textbf{nï-nji}    nungol  \textbf{ngala}    mbï-p-e\\
    1\textsc{sg-poss}  child  \textsc{pl.prox}  here-be-\textsc{ipfv}\\
\glt `My children live here.’ (Literally ‘these children of mine’) [ulwa014\_05:16]
\z

\ea%91
    \label{ex:det:91}
         \textit{\textbf{Nïnji inom anda} kïkal wopa.}\\
\gll    \textbf{nï-nji}    inom  \textbf{anda}    kïkal  wopa\\
    1\textsc{sg-poss}  mother  \textsc{sg.dist}  ear    all\\
\glt `That mother of mine was deaf.’ [ulwa014\_02:18]
\z

\ea%92
    \label{ex:det:92}
          \textit{\textbf{unji inom tembi nda}}\\
\gll    \textbf{u-nji}    inom  tembi  \textbf{anda}\\
    2\textsc{sg-poss}  mother  bad    \textsc{sg.dist}\\
\glt `that poor mother of yours’ [ulwa037\_54:57]
\z

\ea%93
    \label{ex:det:93}
         \textit{\textbf{Manji na ngala} mï ndïtana.}\\
\gll \textbf{ma-nji}      na    \textbf{ngala}    mï      ndï=ta-na\\
    3\textsc{sg.obj-poss}  talk  \textsc{pl.prox}  \textsc{3sg.subj}  3\textsc{pl}=say-\textsc{irr}\\
\glt `These stories of his -- he will tell them.’ [ulwa037\_05:28]
\z

Demonstrative determiners occur not only in subject NPs, but also in object or \isi{oblique} \isi{phrase}s, as in examples \REF{ex:det:94} through \REF{ex:det:97}.

\ea%94
    \label{ex:det:94}
          \textit{Ndïn numïne} \textbf{\textit{ndalumopta}} \textit{ndï mïnapïna.}\\
\gll    ndï=n    numïne  \textbf{anda}=lumo-p-ta      ndï mï=na-p-na\\
    3\textsc{pl=obl}  ditch    \textsc{sg.dist}=put-\textsc{pfv-cond}  \textsc{3pl}    \textsc{3sg.subj=detr}{}-be-\textsc{irr}\\
\glt `Once [I] have planted them in that ditch, they will be there.’ [ulwa014\_73:33]
\z

\is{determiner|)}
\is{deixis|)}
\is{demonstrative|)}

\is{demonstrative|(}
\is{deixis|(}
\is{determiner|(}

\ea%95
    \label{ex:det:95}
          \textit{An tïn} \textbf{\textit{andol}} \textit{iye tïn} \textbf{\textit{anda}} \textit{lamndu nungol kosape.}\\
\gll    an      tïn    \textbf{anda}=ul    i-e        tïn    \textbf{anda} lamndu  nungol  ko=asa-p-e\\
    1\textsc{pl.excl}  dog  \textsc{sg.dist=}with  go.\textsc{pfv-dep}  dog  \textsc{sg.dist}    pig      child  \textsc{indf}=hit-\textsc{pfv-dep}\\
\glt `When we went with that dog, that dog killed one small pig.’ [ulwa037\_61:53]
\z

\ea%96
    \label{ex:det:96}
          \textit{Mïkï itïm ambi ngata} \textbf{\textit{lamana}}.\\
\gll mïkï  itïm  ambi  ngata  \textbf{ala}=ma-na\\
    tree.species  trash  big    grand  \textsc{pl.dist}=go-\textsc{irr}\\
\glt `[We] will go to those great big swamps.’ [ulwa038\_02:59]
\z

\ea%97
    \label{ex:det:97}
          \textit{Una ngusuwa} \textbf{\textit{laya}} \textit{wonlakan!}\\
\gll    unan    ngusuwa  \textbf{ala}=iya      won-la-ka-n\\
    1\textsc{pl.incl}  poor    \textsc{pl.dist=}toward  cut-\textsc{irr-}let-\textsc{imp}\\
\glt `Let’s cross over [the river] to those poor folks [on the other side]!’ [ulwa037\_03:41]
\z

Although the basic function of \isi{demonstrative} determiners is taken to be a means of providing spatial \isi{deixis} from the reference point of the speaker, the actual range of uses of demonstratives is much greater. First, it is not uncommon for a speaker to project a \isi{deictic center} to a point other than the self. Thus, while \isi{demonstrative} words in Ulwa are taken generally to be \isi{egocentric}, a speaker may choose a reference point other than himself or herself in the moment of \isi{speech}. This is common in recounted narratives \REF{ex:det:98}.

\ea%98
    \label{ex:det:98}
          \textit{Nï amun iwa} \textbf{\textit{ngalan}} \textit{mop mo kundan nïpat} \textbf{\textit{ngatïn}}.\\
\gll nï    amun  iwa      \textbf{ngala}=n    ma=u-p ma=u      kundan  nïpat  \textbf{nga}=tï-n\\
    1\textsc{sg}  now  basket  \textsc{pl.prox=obl}  3\textsc{sg.obj}=put-\textsc{pfv}    3\textsc{sg.obj}=from  eel    huge  \textsc{sg.prox}=take-\textsc{pfv}\\
\glt `Now I put these fish trap baskets [down] there and got this huge eel from there.’ [ulwa014\_05:52]
\z

In example \REF{ex:det:98}, although the \isi{proximal} \isi{deictic} words (\textit{ngala=} ‘\textsc{pl.prox}’ and \textit{nga=} ‘\textsc{sg.prox’}) are indeed used with reference to the speaker, they are not used in reference to the speaker’s \isi{location} at the \isi{time} of speaking, but rather to her \isi{location} in the past, when the events being recounted occurred. Projected \isi{deixis} can occur in narratives even when the actor of the clause is different from the narrator of the events, as, for example, in \REF{ex:det:99} and \REF{ex:det:100}.

\ea%99
    \label{ex:det:99}
          \textit{Mï i wolka i manji anaw} \textbf{\textit{ngatïn}} \textit{…}\\
\gll    mï      i    wolka  i    ma-nji      anaw \textbf{nga}=tï-n\\
    3\textsc{sg.subj}  go.\textsc{pfv}  again  go.\textsc{pfv}  \textsc{3sg.obj-poss}  paddle    \textsc{sg.prox}=take-\textsc{pfv}\\
\glt `He went, went back, got his motorboat …’ [ulwa035\_04:06]
\z

\ea%100
    \label{ex:det:100}
          \textit{Anul men i wonmbi} \textbf{\textit{ngintï}} \textit{men i.}\\
\gll    anul    ma=in      i    wonmbi  \textbf{ngin}=tï ma=in      i\\
    grassland  3\textsc{sg.obj}=in  go.\textsc{pfv}  tusk    \textsc{du.prox}=take     3\textsc{sg.obj}=in  go.\textsc{pfv}\\
\glt `[He] went into the grass, got these two tusks, and went in.’ [ulwa001\_13:04]
\z

This phenomenon of projection can further be illustrated with the \isi{locative adverb} \textit{mbï} ‘here’ (\sectref{sec:8.2.2}), which can signify space near the referent of the clause, even when this is not near the speaker in his or her current \isi{location}, as in example \REF{ex:det:101}.

\ea%101
    \label{ex:det:101}
          \textit{Alum mokotïp an mol} \textbf{\textit{mbïwap}}.\\
\gll alum  ma=kot-p        an       ma=ul      \textbf{mbï}{}-wap\\
    child  3\textsc{sg.obj}=break-\textsc{pfv}  \textsc{1pl.excl}  3\textsc{sg.obj}=with  here-be.\textsc{pst}\\
\glt `She bore a child, and we were there with her.’ [ulwa014\_38:44]
\z

Demonstratives, although fundamentally spatial, may be extended in their use to have \isi{temporal} \isi{deixis}. Thus, \isi{proximal} forms may be used to refer to \isi{time}s (\isi{metaphor}ically) close to the \isi{present}, whereas \isi{distal} forms signal more (\isi{metaphor}ically) distant \isi{time}, as illustrated by examples \REF{ex:det:102} through \REF{ex:det:106}.

\ea%102
    \label{ex:det:102}
          \textit{Ipka} \textbf{\textit{ndan}} \textit{matmat mbu ulwape.}\\
\gll    ipka  \textbf{anda}=n    matmat  mbï-u    ulwa=p-e\\
    before  \textsc{sg.dist=obl}  cemetery  here-from  nothing=\textsc{cop-dep}\\
\glt `In the past, there was no cemetery here.’ (\textit{matmat} = TP) [ulwa028\_04:26]
\z

\ea%103
    \label{ex:det:103}
          \textit{Amun} \textbf{\textit{ngan}} \textit{olsem matmat anda mbïpe.}\\
\gll    amun  \textbf{nga}=n      olsem  matmat  anda    mbï-p-e\\
    now  \textsc{sg.prox=obl}  thus  cemetery  \textsc{sg.dist}  here-be\textsc{{}-ipfv}\\
\glt `But at this time, like, there is that cemetery here.’ (\textit{olsem}, \textit{matmat} = TP) [ulwa028\_04:36]
\z

\ea%104
    \label{ex:det:104}
          \textit{Inim} \textbf{\textit{ngan}} \textit{maytap mat atal wap ma inim} \textbf{\textit{andan}} \textit{nï makïke lunda.}\\
\gll    inim  \textbf{nga}=n      ma=ita-p        ma=tï      ata-lï     wap  ma  inim  \textbf{anda}=n    nï    ma=kïke      lo-nda\\
    water  \textsc{sg.prox=obl}  \textsc{3sg.obj}=build-\textsc{pfv}  \textsc{3sg.obj}=take  up-put    be.\textsc{pst}  go  water  \textsc{sg.dist=obl}  1\textsc{sg}  3\textsc{sg.obj}=throw  go-\textsc{irr}\\
\glt `Having built it this year, and put it up, I’m going to sell it next year.’ [ulwa042\_01:11]
\z

\ea%105
    \label{ex:det:105}
          \textit{Ilom} \textbf{\textit{andan}} \textit{nï ango mbïpïna.}\\
\gll    ilom  \textbf{anda}=n    nï    ango  mbï-p-na\\
    day    \textsc{sg.dist=obl}  1\textsc{sg}  \textsc{neg}  here-be-\textsc{irr}\\
\glt `On that day, I won’t stay here.’ [ulwa042\_04:39]
\z

\ea%106
    \label{ex:det:106}
          \textit{Iwïl} \textbf{\textit{andan}} \textit{ma mapta apa ndaytana.}\\
\gll    iwïl  \textbf{anda}=n    ma  ma=p-ta      apa anda=ita-na\\
    moon  \textsc{sg.dist=obl}  go  \textsc{3sg.obj}=be\textsc{{}-cond} house    \textsc{sg.dist}=build-\textsc{irr}\\
\glt `Next month [I] will go and build a house there.’ [ulwa037\_36:50]
\z

Example \REF{ex:det:106} also illustrates how words like \ili{English} ‘this’, ‘these’, ‘that’, and ‘those’ are often not ideal (or even possible) translations for the \isi{demonstrative} markers. This is because, even though the Ulwa \isi{demonstrative}s serve some \isi{deictic} function of pointing to a place or \isi{time}, they do not necessarily have a definite referent. Thus, in example \REF{ex:det:106}, the translation ‘a house’ is given, since this unbuilt house has no definite referent; the salient information, however, is that the house will be built ‘there’.\footnote{Of course, it is possible that the ‘house’ in this sentence does indeed have a definite referent, just not a real-world one, and that the speaker and hearer can both be thinking of a specific yet-to-be-built house.}

  In addition to spatial and \isi{temporal} \isi{deictic} functions, the \isi{demonstrative} words in Ulwa can serve discourse functions as well, pointing to \isi{speech} itself, whether already spoken or not yet uttered \REF{ex:det:107}.

\newpage

\ea%107
    \label{ex:det:107}
          \textit{Oke li ngata ngusuwa} \textbf{\textit{nga}}\textit{: Kayta Amombi Yokombla Yaruwa Kayngam.}\\
\gll    oke  li    ngata  ngusuwa  \textbf{nga}    Kayta     Amombi Yokombla  Yaruwa  Kayngam\\
    ok  down  grand  poor    \textsc{sg.prox}  [name]   [name]    [name]    [name]    [name]\\
\glt `OK, the downstream ancestors, the poor things, were as follows: Kayta, Amombi, Yokombla, Yaruwa, and Kayngam.’ (\textit{oke} < TP \textit{oke} ‘OK’) [ulwa013\_09:00]
\z

Demonstrative words may also be used, to similar effect, as determiners modifying the word \textit{na} ‘talk’, as in examples \REF{ex:det:108}, \REF{ex:det:109}, and \REF{ex:det:110}.

\ea%108
    \label{ex:det:108}
          \textit{Ini na} \textbf{\textit{nga}} \textit{mï ambip.}\\
\gll    ini    na    \textbf{nga}    mï      ambi=p\\
    ground  talk  \textsc{sg.prox}  3\textsc{sg.subj}  big=\textsc{cop}\\
\glt `This talk about land is big -- it [has gotten] big.’ [ulwa037\_39:04]
\z

\ea%109
    \label{ex:det:109}
          \textit{Mase na} \textbf{\textit{nda}} \textit{una asika matap.}\\
\gll    ma=asa-e      na    \textbf{anda}    unan    asi-ka ma=ta-p\\
    3\textsc{sg.obj}=hit\textsc{{}-dep} talk  \textsc{sg.dist}  \textsc{1pl.incl}  sit-let   3\textsc{sg.obj}=say-\textsc{pfv}\\
\glt `That talk of [them] killing her – we sat and discussed it.’ [ulwa037\_00:24]
\z

\ea%110
    \label{ex:det:110}
          \textit{Na anma} \textbf{\textit{nda}}.\\
\gll na    anma  \textbf{anda}\\
    talk  good  \textsc{sg.dist}\\
\glt `That’s good talk.’ (i.e., ‘I agree with you.’) [ulwa038\_04:03]
\z

\is{determiner|)}
\is{deixis|)}
\is{demonstrative|)}

\is{demonstrative|(}
\is{deixis|(}
\is{determiner|(}


Similarly, when a referent has been introduced, a speaker can refer again to this referent with a \isi{deictic} word. In the text from which examples \REF{ex:det:111} and \REF{ex:det:112} are taken, the speaker introduces a subject with the \isi{subject marker} \textit{mï} ‘3\textsc{sg.subj’} \REF{ex:det:111}, but shortly thereafter refers again to the same referent with the \isi{demonstrative} word \textit{nga} ‘\textsc{sg.prox}’ (‘this’) \REF{ex:det:112}.

\ea%111
    \label{ex:det:111}
          \textit{Inom} \textbf{\textit{mï}} \textit{anganika nganul i.}\\
\gll    inom  \textbf{mï}      anganika  ngan=ul      i\\
    mother  3\textsc{sg.subj}  after    \textsc{1du.excl}=with  go.\textsc{pfv}\\
\glt `Later, the mother came with the two of us.’ [ulwa032\_03:35]
\z

\ea%112
    \label{ex:det:112}
          \textit{Inom} \textbf{\textit{nga}} \textit{nan makïta …}\\
\gll    inom  \textbf{nga}    na=n    ma=kï-ta\\
    mother  \textsc{sg.prox}  talk=\textsc{obl}  3\textsc{sg.obj}=say-\textsc{cond}\\
\glt `If this mother tells him …’ [ulwa032\_03:45]
\z

Often, however, the \isi{deictic} function of \isi{demonstrative} words is not clear. The choice between \textit{nga} ‘\textsc{sg.prox}’ (‘this’) and \textit{anda} ‘\textsc{sg.dist}’ (‘that’), for example, does not always seem to reflect proximity or distance, whether spatial, \isi{temporal}, or narrative. Perhaps speakers make decisions based on desires to signal \isi{metaphor}ical proximity or distance to referents. This seems possible especially when referring to people in general terms -- that is, people who are physically neither close nor far in the event encoded in the clause. Other times, however, it is not at all clear to me why these words are being used, and there could be some degree of \isi{free variation} for speakers in certain circumstances. In sentences \REF{ex:det:113} through \REF{ex:det:118}, the \isi{deictic} function of the \isi{demonstrative} words is unclear.

\ea%113
    \label{ex:det:113}
          \textit{Anji ngata} \textbf{\textit{ngalol}} \textit{inde.}\\
\gll    an-nji    ngata  \textbf{ngala}=ul    inda-e\\
    1\textsc{pl-poss}  grand  \textsc{pl.prox}=with  walk-\textsc{ipfv}\\
\glt `[We] walked around with our grandparents.’ [ulwa013\_04:25]
\z

\ea%114
    \label{ex:det:114}
          \textit{Anji ngata} \textbf{\textit{la}} \textit{ndït inde.}\\
\gll    an-nji        ngata  \textbf{ala}      ndï=tï    inda-e\\
    1\textsc{pl.excl-poss}  grand  \textsc{pl.dist}  3\textsc{pl}=take  walk-\textsc{ipfv}\\
\glt `Our ancestors used to carry them around.’ [ulwa015\_01:28]
\z


\ea%116
    \label{ex:det:116}
          \textit{Maria} \textbf{\textit{nga}} \textit{nan ndït.}\\
\gll    Maria  \textbf{nga}    na=n    ndï=ta\\
    [name]  \textsc{sg.prox}  talk=\textsc{obl}  3\textsc{pl}=say\\
\glt `Maria told them.’ [ulwa014\_50:21]
\z

\ea%117
    \label{ex:det:117}
          \textit{A nïnji aweta} \textbf{\textit{anda}} \textit{ko matïna!}\\
\gll    a  nï-nji    aweta  \textbf{anda}    ko  ma=tï-na\\
    ah  1\textsc{sg-poss}  friend  \textsc{sg.dist}  just  3\textsc{sg.obj}=hit-\textsc{irr}\\
\glt `Ah, that friend of mine will really hit her!’ [ulwa020\_00:40]
\z

\ea%118
    \label{ex:det:118}
          \textit{Nïnji yawa} \textbf{\textit{nga}} \textit{itom ndïnji tana mat nen.}\\
\gll    nï-nji    yawa  \textbf{nga}    itom  ndï-nji    tana  ma=tï ni-en\\
    1\textsc{sg-poss}  uncle  \textsc{sg.prox}  father  3\textsc{pl-poss}  axe    3\textsc{sg.obj}=take    act\textsc{{}-nmlz}\\
\glt `This uncle of mine was one who got the forefathers’ axe.’ [ulwa037\_44:58]
\z

Reference to undoubtedly distant entities, such as the sun or the moon, for example, may be referred to with either \isi{proximal} determiners, as in \REF{ex:det:119} and \REF{ex:det:120}, or \isi{distal} determiners, as in \REF{ex:det:121} and \REF{ex:det:122}; they may alternatively be referred to with the \isi{subject marker} \textit{mï} ‘3\textsc{sg}’ \REF{ex:det:123} or with no marker at all \REF{ex:det:124}.

\ea%119
    \label{ex:det:119}
          \textit{Ane} \textbf{\textit{nga}} \textit{sita …}\\
\gll    ane  \textbf{nga}    si-ta\\
    sun  \textsc{sg.prox}  push-\textsc{cond}\\
\glt `Whenever it was dry season …’ [ulwa014†]
\z

\ea%120
    \label{ex:det:120}
          \textit{Ango li ma lïmndï ato ane} \textbf{\textit{ngandïna}}.\\
\gll ango  li    ma  lïmndï  ata-u    ane  \textbf{nga}=andï-na\\
    \textsc{neg}  down  go  eye    up-from  sun  \textsc{sg.prox}=see{}-\textsc{irr}\\
\glt `[She] won’t go down and look up at the sun. ’ [ulwa014\_36:02]
\z

\ea%121
    \label{ex:det:121}
          \textit{A ane} \textbf{\textit{nda}} \textit{li namane.}\\
\gll    a  ane  \textbf{anda}    li    na-ma-n-e\\
    ah  sun  \textsc{sg.dist}  down  \textsc{detr-}go-\textsc{ipfv-dep}\\
\glt `Ah, the sun is setting.’ [ulwa037\_58:25]
\z

\ea%122
    \label{ex:det:122}
          \textit{Iwïl} \textbf{\textit{anda}} \textit{liye wa imba pe.}\\
\gll    iwïl  \textbf{anda}    li-i-e        wa    imba  p-e\\
    moon  \textsc{sg.dist}  down-go.\textsc{pfv-dep}  village  night  be\textsc{{}-ipfv}\\
\glt `The moon had set; the village was dark.’ [ulwa032\_25:02]
\z

\ea%123
    \label{ex:det:123}
          \textit{Anwe iwïl} \textbf{\textit{mï}} \textit{ata ne ne.}\\
\gll    an-we          iwïl  \textbf{mï}      ata  na-i na-i\\
    1\textsc{pl.excl-part.int}  moon  3\textsc{sg.subj}  up  \textsc{detr-}go.\textsc{pfv}    \textsc{detr-}go.\textsc{pfv}\\
\glt `We were alone; the moon rose and rose.’ [ulwa037\_07:00]
\z


\ea%124
    \label{ex:det:124}
          \textit{Ane namane.}\\
\gll    ane  na-ma-n-e\\
    sun  \textsc{detr-}go-\textsc{ipfv-dep}\\
\glt `The sun is setting.’ [ulwa037\_47:10]
\z

Although the Christian god is usually referred to with the \isi{proximal} \isi{deictic} word \textit{nga} ‘\textsc{sg.prox}’ (‘this’) (i.e., \textit{ambi nga} ‘this big [man]’), it seems also possible to refer to him with the \isi{distal} \isi{deictic} word \textit{anda} ‘\textsc{sg.dist}’ (‘that’), as illustrated by examples \REF{ex:det:125} through \REF{ex:det:128}.

\ea%125
    \label{ex:det:125}
          \textit{Ambi} \textbf{\textit{nganji}} \textit{na nga unaniya mbi.}\\
\gll    ambi  \textbf{nga-nji}    na    nga      unan=iya mbï-i\\
    big    \textsc{sg.prox-poss}  talk  \textsc{sg.prox}  1\textsc{pl.incl=}toward   here-go.\textsc{pfv}\\
\glt `The word of God has come to us.’ [ulwa037\_25:12]
\z

\ea%126
    \label{ex:det:126}
          \textit{Ambi} \textbf{\textit{ngawe}} \textit{una ikali mas.}\\
\gll    ambi  \textbf{nga-we}      unan    i-kali    ma=si\\
    big    \textsc{sg.prox-part.int}  \textsc{1pl.excl}  hand-send  3\textsc{sg.obj}=push\\
\glt `God alone -- we [must] hold onto him.’ [ulwa037\_08:55]
\z

\ea%127
    \label{ex:det:127}
          \textit{Wolka ambi} \textbf{\textit{ngayinakawana}}.\\
\gll wolka  ambi  \textbf{nga=}ina-ka-wana\\
    again  big    \textsc{sg.prox=}liver-at-feel\\
\glt `[He] in turn was thinking of God.’ [ulwa035\_03:00]
\z

\ea%128
    \label{ex:det:128}
          \textit{Ambi} \textbf{\textit{anda}} \textit{mat anmbïnalp.}\\
\gll    ambi  \textbf{anda}    ma=tï      an-mbï    na-lï{}-p\\
    big    \textsc{sg.dist}  \textsc{3sg.obj}=take  out-here  \textsc{detr-}put-\textsc{pfv}\\
\glt `God has revealed him.’ (Literally ‘That big [man] has taken him and put [him] out.’) [ulwa037\_29:50]
\z

As these examples illustrate, there is often much freedom in the use of \isi{subject marker}s, \isi{object marker}s, and demonstratives.

  In addition to serving their prototypically \isi{deictic} function, demonstratives in Ulwa may be used to indicate that an introduced NP is going to play a key role in the discourse to follow -- that is, even though the NP presents \isi{new information} (i.e., non-\isi{given information}), a form like \textit{anda} ‘\textsc{sg.dist}’ (‘that’) may be used, as in \REF{ex:det:129}, \REF{ex:det:130}, and \REF{ex:det:131}.

\ea%129
    \label{ex:det:129}
          \textit{Awlu ato anmoka} \textbf{\textit{anda}} \textit{apïnal ando anmbi.}\\
\gll    awlu  ata-u    anmoka  \textbf{anda}    apïnal  anda=u an-mbï-i\\
    step  up-from  snake    \textsc{sg.dist}  swamp  \textsc{sg.dist}=from    out-here-go.\textsc{pfv}\\
\glt `When [the moon] appeared, a snake came out from the swamp.’ [ulwa034\_01:53]
\z

\ea%130
    \label{ex:det:130}
          \textit{Nambi wandam ambi} \textbf{\textit{nda}}.\\
\gll nï-ambi  wandam  ambi  \textbf{anda}\\
    1\textsc{sg-top}  jungle    big    \textsc{sg.dist}\\
\glt `As for me, I have a big garden.’ [ulwa042\_04:20]
\z

\ea%131
    \label{ex:det:131}
          \textit{Mbalus} \textbf{\textit{anda}} \textit{ina mane.}\\
\gll    mbalus    \textbf{anda}    i-na    ma-n-e\\
    airplane  \textsc{sg.dist}  come-\textsc{irr}  go-\textsc{ipfv-dep}\\
\glt `An airplane was going to come.’ (\textit{mbalus} < TP \textit{balus} ‘dove, airplane’) [ulwa014\_49:15]
\z

\is{determiner|)}
\is{deixis|)}
\is{demonstrative|)}

\is{demonstrative|(}
\is{deixis|(}
\is{determiner|(}

This use of demonstratives is especially common when recounting narratives in a vivid manner. This may be compared with similar uses of demonstratives in \ili{English} (e.g., \textit{so this guy came up to me at a party}). In Ulwa, however, the \isi{distal} \isi{deictic} word is used in such contexts, rather than the \isi{proximal}.

  Demonstratives may also occur in non-subject NPs -- that is, in NPs encoding the objects of verbs, \is{object of a postposition} objects of postpositions, or \isi{oblique} \isi{phrase}s. While often attaching \isi{phonological}ly to following words (especially \isi{verb stem}s), these \isi{demonstrative} forms seem somehow less \isi{clitic}-like than true \isi{object marker}s (\sectref{sec:7.2}). When the \isi{demonstrative} forms appear (\isi{phonological}ly) to \isi{clitic}ize to host verbs, they are treated similarly to \isi{object marker}s and are glossed with a \isi{clitic} boundary marker (=) following them. Generally, however, it may be said that there is no formal distinction between subject demonstratives and object demonstratives.

  Sentences \REF{ex:det:132} through \REF{ex:det:137} exemplify the use of the demonstratives as \isi{object marker}s.

\ea%132
    \label{ex:det:132}
          \textit{Lapun nga lamndu \textbf{ngas}!}\\
\gll    lapun    nga      lamndu  \textbf{nga=}asa\\
    old.person  \textsc{sg.prox}  pig      \textsc{sg.prox}=hit\\
\glt `This old man killed this pig!’ (\textit{lapun} = TP) [ulwa029\_01:38]
\z

\ea%133
    \label{ex:det:133}
          \textit{Mï ya uta} \textbf{\textit{nginanda}}.\\
\gll mï      ya      uta    \textbf{ngin}=a-nda\\
    3\textsc{sg.subj}  coconut  shell  \textsc{du.prox}=break-\textsc{irr}\\
\glt `He will break these two coconut shells.’ [elicited]
\z

\ea%134
    \label{ex:det:134}
          \textit{Wambana} \textbf{\textit{ngalamoke}}.\\
\gll wambana  \textbf{ngala}=moko-e\\
    fish    \textsc{pl.prox}=take-\textsc{ipfv}\\
\glt `[They] were catching fish.’ [ulwa032\_22:37]
\z

\ea%135
    \label{ex:det:135}
          \textit{Mota wulis} \textbf{\textit{andaytap}}.\\
\gll mota    wulis    \textbf{anda}=ita-p\\
    bamboo.species  platform  \textsc{sg.dist}=build-\textsc{pfv}\\
\glt `[They] built that bamboo raft.’ [ulwa002\_00:14]
\z

\ea%136
    \label{ex:det:136}
          \textit{Sokoy} \textbf{\textit{andin}} \textit{lapap.}\\
\gll    sokoy    \textbf{andin}=n    lapa-p\\
    tobacco  \textsc{du.dist=obl}  plant-\textsc{pfv}\\
\glt `[He] has planted those two tobacco plants.’ [ulwa037\_56:01]
\z

\ea%137
    \label{ex:det:137}
          \textit{Upan wambana} \textbf{\textit{lawtata}} \textit{ndul wa undana.}\\
\gll    upan  wambana  \textbf{ala}=uta-ta      ndï=ul    wa    unda-na\\
    fish.species   fish    \textsc{pl.dist}=grind\textsc{{}-cond} 3\textsc{pl}=with  village  go-\textsc{irr}\\
\glt `If [we] catch those fish, [we] will go home with them.’ [ulwa038\_00:16]
\z

As mentioned, \isi{demonstrative} forms can also function pronominally. They can have this pronominal function, whether serving as subjects, as in examples \REF{ex:det:138}, \REF{ex:det:139}, and \REF{ex:det:140}; or serving as objects, as in examples \REF{ex:det:141} and \REF{ex:det:142}. No distinction is made between \isi{animate} and \isi{inanimate} referents.

\ea%138
    \label{ex:det:138}
          \textbf{\textit{Anda}} \textit{nip.}\\
\gll    \textbf{anda}    ni-p\\
    \textsc{sg.dist}  die-\textsc{pfv}\\
\glt `That [one] died.’ [elicited]
\z

\ea%139
    \label{ex:det:139}
          \textbf{\textit{Ngin}} \textit{liyu.}\\
\gll    \textbf{ngin}    li-u\\
    \textsc{du.prox}  down-put\\
\glt `These [two] fell.’ [elicited]
\z

\ea%140
    \label{ex:det:140}
          \textbf{\textit{Ala}} \textit{lamndu masap.}\\
\gll    ala      lamndu  ma=asa-p\\
    \textsc{pl.dist}  pig      3\textsc{sg.obj}=hit-\textsc{pfv}\\
\glt `Those [ones] killed the pig.’ (often = ‘They killed the pig.’) [elicited]
\z

\ea%141
    \label{ex:det:141}
          \textit{Nï lïmndï} \textbf{\textit{ngala}}.\\
\gll nï    lïmndï  \textbf{nga=}ala\\
    \textsc{1sg}  eye    \textsc{sg.prox}=see\\
\glt `I saw this [one].’ [elicited]
\z

\ea%142
    \label{ex:det:142}
          \textit{Nï lïmndï} \textbf{\textit{andinala}}.\\
\gll nï    lïmndï  \textbf{andin}=ala\\
    \textsc{1sg}  eye    \textsc{du.dist}=see\\
\glt `I saw those [two].’ [elicited]
\z

These \isi{object-marker} \isi{demonstrative} \isi{pronoun}s may not always be \isi{clitic}s. Examples such as \REF{ex:det:143} and \REF{ex:det:144} illustrate a greater \isi{phonological} separation between \isi{pronoun} and verb -- that is, the sequence in example \REF{ex:det:143} is pronounced [nga.la.i.ta.na] and not \textsuperscript{†}[nga.lay.ta.na]; and the sequence in example \REF{ex:det:144} is pronounced [a.nda.i] and not \textsuperscript{†}[a.nday].

\ea%143
    \label{ex:det:143}
          \textit{Apa} \textbf{\textit{ngala}} \textbf{\textit{itana}} \textit{mane.}\\
\gll    apa    \textbf{ngala}    \textbf{ita-na}    ma-n-e\\
    house  \textsc{pl.prox}  build-\textsc{irr}  go-\textsc{ipfv-dep}\\
\glt `[They] were going to build these houses.’ [ulwa028\_02:57]
\z

\ea%144
    \label{ex:det:144}
          \textit{Ndamape nï anmap nï i \textbf{anda i}}.\\
\gll ndï=ama-p-e    nï    anma=p  nï    i    \textbf{anda} i\\
    3\textsc{pl}=eat-\textsc{pfv-dep}  1\textsc{sg}  good=\textsc{cop}  \textsc{1sg}  go.\textsc{pfv}  \textsc{sg.dist}    go.\textsc{pfv}\\
\glt `Having taken them, I got better, and I went, went there.’ [ulwa026\_00:22]
\z

Demonstrative \isi{pronoun}s can be the subject of a clause that has a noun or \isi{adjective} as the \isi{predicate complement}. Here, their \isi{deictic} function is quite clear, as seen in examples \REF{ex:det:145} through \REF{ex:det:150}.

\ea%145
    \label{ex:det:145}
          \textbf{\textit{Nga}} \textit{nïnji apa.}\\
\gll    \textbf{nga}    nï-nji    apa\\
    \textsc{sg.prox}  \textsc{1sg-poss}  house\\
\glt `This is my house.’ [elicited]
\z

\ea%146
    \label{ex:det:146}
          \textbf{\textit{Nga}} \textit{wa anma.}\\
\gll    \textbf{nga}    wa    anma\\
    \textsc{sg.prox}  village  good\\
\glt `This is a good village.’ [elicited]
\z

\ea%147
    \label{ex:det:147}
          \textbf{\textit{Anda}} \textit{ango nïnji apa.}\\
\gll    \textbf{anda}    ango  nï-nji    apa\\
    \textsc{sg.dist}  \textsc{neg}  \textsc{1sg-poss}  house\\
\glt `That is not my house.’ [elicited]
\z

\ea%148
    \label{ex:det:148}
          \textbf{\textit{Ngin}} \textit{manji itom inom.}\\
\gll    \textbf{ngin}    ma-nji      itom  inom\\
    \textsc{du.prox}  \textsc{3sg.obj-poss}  father  mother\\
\glt `These are his parents.’ [elicited]
\z

\ea%149
    \label{ex:det:149}
          \textbf{\textit{Ngala}} \textit{ango ambip.}\\
\gll    \textbf{ngala}    ango  ambi=p\\
    \textsc{pl.prox}  \textsc{neg}  big=\textsc{cop}\\
\glt `These are not big.’ [elicited]
\z

\ea%150
    \label{ex:det:150}
          \textbf{\textit{Ala}} \textit{anmap.}\\
\gll    \textbf{ala}      anma=p\\
    \textsc{pl.dist}  good=\textsc{cop}\\
\glt `Those are good.’ [elicited]
\z

Demonstrative \isi{pronoun}s are used to refer to human referents (as in ‘this [one]’, ‘those [ones]’, etc.) much more than is common in, say, \ili{English}. In particular, the \isi{plural} \isi{distal} \isi{demonstrative} \isi{pronoun} \textit{ala} ‘\textsc{pl.dist}’ (‘those’) is often best translated simple as ‘they’ or ‘people’ (or sometimes as ‘other people’), as in examples \REF{ex:det:151} through \REF{ex:det:154}.

\ea%151
    \label{ex:det:151}
          \textbf{\textit{Ala}} \textit{natana.}\\
\gll    \textbf{ala}      na-ta-na\\
    \textsc{pl.dist}  \textsc{detr-}say-\textsc{irr}\\
\glt `They were going to have a talk.’ [ulwa032\_32:36]
\z

\ea%152
    \label{ex:det:152}
          \textbf{\textit{Ala}} \textit{angop tane.}\\
\gll    \textbf{ala}      ango=p  ta-n-e\\
    \textsc{pl.dist}  \textsc{neg=cop}  say-\textsc{ipfv-dep}\\
\glt `They tell lies.’ [ulwa014\_48:32]
\z

\ea%153
    \label{ex:det:153}
          \textbf{\textit{Ala}} \textit{ta ando apïn tï lïp.}\\
\gll    \textbf{ala}      ta    anda=u    apïn  tï    lï-p\\
    \textsc{pl.dist}  already  \textsc{sg.dist}=from  fire    take  put-\textsc{pfv}\\
\glt `People have already set fire there.’ [ulwa038\_03:27]
\z

\ea%154
    \label{ex:det:154}
          \textbf{\textit{Ala}} \textit{ndute ndame mbïp.}\\
\gll    \textbf{ala}      ndï=uta-e      ndï=ama-e    mbï-p\\
    \textsc{pl.dist}  \textsc{3pl}=grind-\textsc{ipfv}  3\textsc{pl}=eat-\textsc{ipfv}  here-be\\
\glt `People catch them and eat them here.’ [ulwa041\_02:37]
\z

Furthermore the \isi{plural} \isi{distal} \isi{demonstrative} \textit{ala} ‘\textsc{pl.dist}’ (‘those’) may be used instead of the second \isi{plural} \isi{personal pronoun} \textit{un} ‘\textsc{2pl}’, when addressing groups of people, as in examples \REF{ex:det:155} through \REF{ex:det:158}.

\is{determiner|)}
\is{deixis|)}
\is{demonstrative|)}
\is{demonstrative|(}
\is{deixis|(}
\is{determiner|(}

\ea%155
    \label{ex:det:155}
          \textbf{\textit{Ala}} \textit{wokïn anda unalu mbiyen anda ango i?}\\
\gll    \textbf{ala}      wokïn    anda    unan=lu    mbï-i-en anda    ango  i\\
    \textsc{pl.dist}  big.man  \textsc{sg.dist}  1\textsc{pl.incl}=with  here-go-\textsc{nmlz}    \textsc{sg.dist}  which  go.\textsc{pfv}\\
\glt `You folks, that big man who came with us -- where did that [man] go?’ [ulwa001\_13:15]
\z

\ea%156
    \label{ex:det:156}
          \textbf{\textit{Ala}} \textit{una wandam ma mundu anglalunda mane.}\\
\gll    \textbf{ala}      unan    wandam  ma  mundu  angla-lo-nda ma-n-e\\
    \textsc{pl.dist}  1\textsc{pl.incl}  jungle    go  food  await-go-\textsc{irr}    go-\textsc{ipfv-dep}\\
\glt `Everyone, we’re going to go to the jungle and look for food.’ [ulwa030\_01:10]
\z

\ea%157
    \label{ex:det:157}
          \textbf{\textit{Ala}} \textit{ndï ta lop.}\\
\gll    \textbf{ala}      ndï  ta    lo-p\\
    \textsc{pl.dist}  3\textsc{pl}  already  go-\textsc{pfv}\\
\glt `You all, have they already left?’ [ulwa013\_06:20]
\z

\ea%158
    \label{ex:det:158}
          \textbf{\textit{Ala}} \textit{ndïn anjikake ndï se?}\\
\gll    \textbf{ala}      ndï=n    anjikaka-e    ndï  sa-e\\
    \textsc{pl.dist}  3\textsc{pl=obl}  how-\textsc{dep}    3\textsc{pl}  cry-\textsc{ipfv}\\
\glt `Folks, what have [you done] with them, such that they are crying?’ [ulwa032\_52:42]
\z

In fact, combined with the form \textit{-nji} ‘thing’, the \isi{demonstrative} \textit{ala} ‘\textsc{pl.dist}’ (‘those’) can even be used in \is{possession} possessive constructions -- that is, \textit{alanji} \linebreak ‘\textsc{pl.dist-poss}’ (‘those [people]’s’) in place of \textit{ndïnji}  ‘\textsc{3pl-poss}’ (‘their’), often with the sense of ‘other people’s. This is illustrated by examples \REF{ex:det:159}, \REF{ex:det:160}, and \REF{ex:det:161}.

\ea%159
    \label{ex:det:159}
          \textit{Ambwat} \textbf{\textit{alanji}} \textit{Monde.}\\
\gll    Ambwat  \textbf{ala-nji}      Monde\\
    Kambot  \textsc{pl.dist-poss}  [name]\\
\glt `The Kambot people’s [ancestor] was Monde.’ [ulwa002\_04:15]
\z

\ea%160
    \label{ex:det:160}
          \textbf{\textit{Alanji}} \textit{wo ndï makape.}\\
\gll    \textbf{ala-nji}      wa    ndï  maka=p-e\\
    \textsc{pl.dist-poss}  village  3\textsc{pl}  thus=\textsc{cop{}-dep}\\
\glt `Other people’s villages are like that.’ [ulwa037\_32:25]
\z

\ea%161
    \label{ex:det:161}
          \textbf{\textit{alanji}} \textit{amba nda}\\
\gll    \textbf{ala-nji}      amba        anda\\
    \textsc{pl.dist-poss}  mens.house    \textsc{sg.dist}\\
\glt `that magic of other people’ [ulwa037\_10:01]
\z

In casual speech, the forms \textit{anda} ‘\textsc{sg.dist}’ (‘that’)  and \textit{ala} ‘\textsc{pl.dist}’ (‘those’) are commonly shortened to [nda] and [la], respectively. This is especially common when following a \isi{vowel}, but can occur in any environment.


\is{determiner|)}
\is{deixis|)}
\is{demonstrative|)}

\section{Quantifiers}\label{sec:7.4}

\is{quantifier|(}
\is{determiner|(}

On \isi{semantic} grounds, words that express concepts such as ‘much’, ‘many’, ‘few’, ‘all’, ‘some’, and so on may be considered quantifiers. The words that express these concepts in Ulwa mostly pattern \isi{syntactic}ally with words in other classes, namely \isi{adjective}s. There is at least one word, however, that warrants placement in a separate \isi{quantifier} class, since it displays some unique \isi{syntactic} properties. This word is \textit{wopa} ‘all’. The major quantity-denoting words are given in \REF{ex:det:202}.

\ea%202
    \label{ex:det:202}
          Quantity-denoting words\\
\begin{tabbing}
{(\textit{kekaka})} \= {(‘one each, one by one, just a few’)}\kill
{\textit{wopa}} \> {‘all’}\\
{\textit{nunu}} \> {‘every’}\\
{\textit{kuma}} \> {‘some’}\\
{\textit{ilum}} \> {‘piece, little, few’}\\
{\textit{kekaka}} \> {‘one each, one by one, just a few’}\\
{\textit{ambi}} \> {‘big, much’}\\
{\textit{tïngïn}} \> {‘many}’
\end{tabbing}
 \z

  The word \textit{wopa} ‘all’ can function as an \isi{adjective}, meaning ‘whole’, ‘entire’, or ‘full’. Like all \isi{adjective}s, its canonical position is immediately following the noun that it modifies (\sectref{sec:5.1}). If there is a \isi{subject marker}, \isi{object marker}, or other \isi{determiner} present, then the \isi{adjective} \textit{wopa} ‘all’ precedes this word. In this usage, it has a \isi{singular} (as opposed to \isi{plural}) meaning -- that is, it means something like ‘all of something’ or ‘the whole’. Accordingly, as in examples \REF{ex:det:162}, \REF{ex:det:163}, and \REF{ex:det:164}, NPs containing the \isi{attributive adjective} \textit{wopa} ‘all’ are followed by \isi{singular} determiners (e.g., the \isi{subject marker} \textit{mï} ‘3\textsc{sg.subj}’ or the \isi{demonstrative} \isi{object marker} \textit{anda=} ‘\textsc{sg.dist’).}

\newpage

\ea%162
    \label{ex:det:162}
          \textit{Im} \textbf{\textit{wopa}} \textit{mï liyu.}\\
\gll    im    \textbf{wopa}  mï      li-u\\
    tree  all    \textsc{3sg.subj}  down-put\\
\glt `The whole tree fell.’ [elicited]
\z

\ea%163
    \label{ex:det:163}
          \textit{Utam} \textbf{\textit{wopa}} \textit{mï tembip.}\\
\gll    utam  \textbf{wopa}  mï      tembi=p\\
    yam  all    \textsc{3sg.subj}  bad=\textsc{cop}\\
\glt `The entire yam is rotten.’ [elicited]
\z

\ea%164
    \label{ex:det:164}
          \textit{Ndï unan wat u apïn} \textbf{\textit{wopa}} \textit{ndatïne …}\\
\gll    ndï  unan=n    wat    u    apïn  \textbf{wopa} anda=tï-n-e\\
    3\textsc{pl}  1\textsc{pl.incl=obl}  atop  from  fire    all    \textsc{sg.dist}=take-\textsc{pfv-dep}\\
\glt `And once they have gotten the full fire from above us …’ [ulwa037\_14:12]
\z

Like all \isi{adjective}s, \textit{wopa} ‘all’ may also function as a substantive (\sectref{sec:5.3}). In example \REF{ex:det:165}, \textit{wopa} ‘all’ is followed by the \isi{plural} \isi{subject marker} \textit{ndï} `3\textsc{pl}’, because it is referring to multiple whole things (in this sentence, fish).

\ea%165
    \label{ex:det:165}
          \textbf{\textit{Wopa}} \textit{ndï ngamana.}\\
\gll    \textbf{wopa}  ndï  nga=ma-na\\
    all    3\textsc{pl}  \textsc{sg.prox}=go-\textsc{irr}\\
\glt `The whole [ones] will go here.’ [ulwa014\_68:54]
\z

As a \isi{syntactic}ally distinct \isi{quantifier}, however, \textit{wopa} ‘all’ refers to all members of a group or set of things. Thus it fills the function of a \is{universal quantifier} \isi{collective universal quantifier}. Instead of preceding the \isi{subject marker} (or subject \isi{pronoun}), the \isi{quantifier} follows it. If the \isi{quantifier} \textit{wopa} ‘all’ can be analyzed as belonging to the NP, then it is the only NP-internal element allowed to follow a \isi{subject marker} (or \isi{object marker}); however, it may be better to analyze the \isi{quantifier} as sitting \isi{syntactic}ally outside the NP. In sentences \REF{ex:det:166}, \REF{ex:det:167}, and \REF{ex:det:168}, \textit{wopa} ‘all’ follows the \isi{plural} \isi{subject marker} \textit{ndï} `3\textsc{pl}’.

\is{determiner|)}
\is{quantifier|)}

\is{quantifier|(}
\is{determiner|(}

\ea%166
    \label{ex:det:166}
          \textit{Im ndï} \textbf{\textit{wopa}} \textit{liyu.}\\
\gll    im    ndï  \textbf{wopa}  li-u\\
    tree  3\textsc{pl}  all    down-put\\
\glt `All the trees fell.’ [elicited]
\z

\ea%167
    \label{ex:det:167}
          \textit{Utam ndï} \textbf{\textit{wopa}} \textit{tembip.}\\
\gll    utam  ndï  \textbf{wopa}  tembi=p\\
    yam  \textsc{3pl}  all    bad=\textsc{cop}\\
\glt `All the yams are rotten.’ [elicited]
\z

\ea%168
    \label{ex:det:168}
          \textit{Nji ndï} \textbf{\textit{wopa}} \textit{men pe.}\\
\gll    nji    ndï  \textbf{wopa}  ma=in      p-e\\
    thing  3\textsc{pl}  all    3\textsc{sg.obj}=in  be\textsc{{}-ipfv}\\
\glt `All [his] possessions are in it.’ [ulwa014\_31:41]
\z

Examples \REF{ex:det:169}, \REF{ex:det:170}, and \REF{ex:det:171} illustrate that the \isi{quantifier} \textit{wopa} ‘all’ can appear after \isi{pronoun}s as well as after \isi{subject marker}s. Such \isi{pronoun}s are always \isi{plural}.

\ea%169
    \label{ex:det:169}
          \textit{Una} \textbf{\textit{wopa}} \textit{map.}\\
\gll    unan    \textbf{wopa}  ma=p\\
    1\textsc{pl.incl}  all    3\textsc{sg.obj}=be\\
\glt `We all stay there.’ [ulwa014\_06:11]
\z

\ea%170
    \label{ex:det:170}
          \textit{Ndï} \textbf{\textit{wopa}} \textit{wombïn ne.}\\
\gll    ndï    \textbf{wopa}  wombïn=n  ni-e\\
    \textsc{3pl}    all    work=\textsc{obl}  act-\textsc{ipfv}\\
\glt    (a) ‘They are all working.’\\
    (b) ‘All of them are working.’ [elicited]
\z

\ea%171
    \label{ex:det:171}
          \textit{Ndambi} \textbf{\textit{wopa}} \textit{anala mbïp.}\\
\gll    ndï-ambi  \textbf{wopa}  an=ala      mbï-p\\
    3\textsc{pl-top}  all    1\textsc{pl.excl}=for  here-be\\
\glt `As for them, they all stayed for our sake.’ [ulwa032\_25:21]
\z

As a \isi{quantifier}, \textit{wopa} ‘all’ has a rigid post-NP position. Attempts to raise the \isi{quantifier} overtly to a position within the NP (that is, between the noun and \isi{subject marker}) result in an \is{adjective} adjectival interpretation of the word (that is, ‘whole’, ‘full’, ‘complete’, etc.), as shown in \REF{ex:det:172} and \REF{ex:det:173}.

\ea%172
    \label{ex:det:172}
          \textit{Ankam ndï} \textbf{\textit{wopa}} \textit{wandam i.}\\
\gll    ankam  ndï  \textbf{wopa}  wandam  i\\
    person  \textsc{3pl}  all    jungle    go.\textsc{pfv}\\
\glt `All the people went to the jungle.’ [elicited]
\z

\ea[?]{%173
    \label{ex:det:173}
         \textit{Ankam} \textbf{\textit{wopa}} \textit{ndï wandam i.}\\
\gll    ankam  \textbf{wopa}  ndï  wandam  i\\
    person  all    \textsc{3pl}  jungle    go.\textsc{pfv}\\
\glt    ‘The whole people went to the jungle.’ (i.e., not just parts of their bodies went) [elicited]}
\z

Although the \isi{quantifier} \textit{wopa} ‘all’ must always follow the entire NP (including the \isi{subject marker}), in \isi{negative} clauses it may either precede \REF{ex:det:174} or follow \REF{ex:det:175} the \isi{negative} marker \textit{ango} ‘\textsc{neg}’. The fact that it may follow the \isi{negator} is further indication that it does not properly belong within the NP \isi{syntactic}ally, since the \isi{negator} word does not belong \isi{syntactic}ally to the NP.

\ea%174
    \label{ex:det:174}
          \textit{Ankam ndï} \textbf{\textit{wopa}} \textit{ango wandam i.}\\
\gll    ankam  ndï  \textbf{wopa}  ango  wandam  i\\
    person  \textsc{3pl}  all    \textsc{neg}  jungle    go.\textsc{pfv}\\
\glt `All the people did not go to the jungle.’ [elicited]
\z

\ea%175
    \label{ex:det:175}
          \textit{Ankam ndï ango} \textbf{\textit{wopa}} \textit{wandam i.}\\
\gll ankam  ndï  ango  \textbf{wopa}  wandam  i\\
    person  \textsc{3pl}  \textsc{neg}  all    jungle    go.\textsc{pfv}\\
\glt `All the people did not go to the jungle.’ [elicited]
\z

Sentences \REF{ex:det:174} and \REF{ex:det:175} have the same meaning. Indeed, the scopal relationship between the \isi{negator} and the \isi{quantifier} is also the same -- and, in both cases, ambivalent. That is, either may have \isi{scope} over the other, producing either the possible interpretation that ‘not all (i.e., some) people went to the jungle’ or the other possible interpretation that ‘no people went to the jungle’. In example \REF{ex:det:176}, only context reveals that \textit{ango wopa} ‘not all’ implies ‘no one’ as opposed to implying ‘some’.

\ea%176
    \label{ex:det:176}
          \textit{Ndï \textbf{ango wopa} mol lop.}\\
\gll    ndï  \textbf{ango}  \textbf{wopa}  ma=ul      lo-p\\
    3\textsc{pl}  \textsc{neg}  all    3\textsc{sg.obj=}with  go-\textsc{pfv}\\
\glt `They all did not go with him.’ (i.e., ‘None of them went with him.’ In other contexts, however, this same sentence could imply: ‘Not all of them went with him.’) [ulwa035\_01:30]
\z

At times, \textit{wopa} ‘all’ may alternatively be translated as ‘everything’ or ‘everyone’. In these instances, \textit{wopa} ‘all’ also follows \isi{subject marker}s or \isi{pronoun}s (as when the word functions elsewhere as a \isi{quantifier}), as in \REF{ex:det:177} and \REF{ex:det:178}.

\ea%177
    \label{ex:det:177}
          \textit{Nji ndï} \textbf{\textit{wopa}} \textit{liyu.}\\
\gll    nji    ndï  \textbf{wopa}  li-u\\
    thing  3\textsc{pl}  all    down-put\\
\glt `Everything fell.’ (Literally ‘All the things fell.’) [elicited]
\z

\ea%178
    \label{ex:det:178}
          \textit{Ala} \textbf{\textit{wopa}} \textit{wandam i.}\\
\gll    ala      \textbf{wopa}  wandam  i\\
    \textsc{pl.dist}  all    jungle    go.\textsc{pfv}\\
\glt `Everyone went to the jungle.’ (Literally ‘Those all went to the jungle.’) [elicited]
\z

\is{determiner|)}
\is{quantifier|)}

\is{quantifier|(}
\is{determiner|(}

However, it may also function somewhat like a \isi{pronoun}, forming its own NP \REF{ex:det:179}.

\ea%179
    \label{ex:det:179}
          \textbf{\textit{Wopa}} \textit{malanda.}\\
\gll    \textbf{wopa}  ma=la-nda\\
    all    \textsc{3sg.obj}=eat-\textsc{irr}\\
\glt `All would eat it.’ [ulwa014\_65:08]
\z

One of the most interesting aspects of the \isi{syntactic} positioning of the \isi{quantifier} \textit{wopa} ‘all’, however, is the fact that it follows not only \isi{subject marker}s, but also \isi{object marker}s. It is thus the only element known to intercede between \isi{object-marker} \isi{clitic}s and their associated verbs. Sentences \REF{ex:det:180} through \REF{ex:det:183} illustrate this unusual placement of \textit{wopa} ‘all’.

\ea%180
    \label{ex:det:180}
          \textit{Inom mï mïnda} \textbf{\textit{nduwopa}} \textit{wananda.}\\
\gll    inom  mï      mïnda  ndï=\textbf{wopa}  wana-nda\\
    mother  \textsc{3sg.subj}  banana  \textsc{3pl}=all  cook-\textsc{irr}\\
\glt `Mother will cook all the bananas.’ [elicited]
\z

\ea%181
    \label{ex:det:181}
          \textit{Nï lamndu} \textbf{\textit{nduwopa}} \textit{asap.}\\
\gll    nï    lamndu  ndï=\textbf{wopa}  asa-p\\
    1\textsc{sg}  pig      \textsc{3pl}=all  hit-\textsc{pfv}\\
\glt `I killed all the pigs.’ [elicited]
\z

\ea%182
    \label{ex:det:182}
          \textit{Nï lïmndï nji} \textbf{\textit{nduwopa}} \textit{ala.}\\
\gll    nï    lïmndï  nji    ndï=\textbf{wopa}  ala\\
    \textsc{1sg}  eye    thing  \textsc{3pl}=all  see\\
\glt `I saw everything.’ [elicited]
\z

\ea%183
    \label{ex:det:183}
          \textit{Nï lïmndï} \textbf{\textit{alawopa}} \textit{ala.}\\
\gll    nï    lïmndï  ala=\textbf{wopa}    ala\\
    \textsc{1sg}  eye    \textsc{pl.dist}=all  see\\
\glt `I saw everyone.’ [elicited]
\z

When preceding the \isi{object marker}, however, \textit{wopa} ‘all’ can only have an \is{adjective} adjectival interpretation \REF{ex:det:184}.

\ea%184
    \label{ex:det:184}
          \textit{Inom mï mïnda} \textbf{\textit{wopa}} \textit{nduwananda.}\\
\gll    inom  mï      mïnda  \textbf{wopa}  ndï=wana-nda\\
    mother  \textsc{3sg.subj}  banana  all    \textsc{3pl}=cook-\textsc{irr}\\
\glt `Mother will cook the whole bananas.’ (i.e., the uncut bananas) [elicited]
\z

The unique positioning of \textit{wopa} ‘all’ between \isi{object marker}s and their associated verbs is suggestive more than anything else that this word belongs to a \isi{syntactic} class of its own. Caution is required here, however, since this evidence comes solely from elicitations; there are (perhaps surprisingly) no examples in the Ulwa corpus of texts of \textit{wopa} ‘all’ occurring in non-subject NPs.

  The \is{universal quantifier} \isi{collective universal quantifier} \textit{wopa} ‘all’ may be contrasted, both formally and \isi{syntactic}ally, with the \is{universal quantifier} \isi{distributive universal quantifier} \textit{nunu} ‘every’. Unlike other modifiers, such as \isi{adjective}s (\sectref{sec:5.1}; cf. \textit{tïngïn} ‘many’), and unlike the \isi{quantifier} \textit{wopa} ‘all’, the \isi{quantifier} \textit{nunu} ‘every’ occurs before the noun it modifies. Thus, like \textit{wopa} ‘all’, \textit{nunu} ‘every’ may exist in a class of its own. The use of \textit{nunu} ‘every’ is illustrated by examples \REF{ex:det:185}, \REF{ex:det:186}, and \REF{ex:det:187}.

\ea%185
    \label{ex:det:185}
          \textbf{\textit{Nunu}} \textit{njin molnda mane.}\\
\gll    \textbf{nunu}  nji=n    ma=lu{}-nda      ma-n-e\\
    every  thing=\textsc{obl}  \textsc{3sg.obj}=put-\textsc{irr}  go-\textsc{ipfv-dep}\\
\glt `[I] am going to plant everything there.’ [ulwa042\_04:23]
\z

\ea%186
    \label{ex:det:186}
          \textit{Wa} \textbf{\textit{nunu}} \textit{wa ule.}\\
\gll    wa    \textbf{nunu}  wa    u-lo-e\\
    just    every  village  from-go-\textsc{ipfv}\\
\glt `[They] just go around in every village.’ [ulwa037\_12:53]
\z

\ea%187
    \label{ex:det:187}
          \textit{A} \textbf{\textit{nunu}} \textit{wombïn tembi ndambilakan!}\\
\gll    a  \textbf{nunu}  wombïn  tembi  ndï-ambi=la-ka-n\\
    ah  every  work    bad    \textsc{3pl-top}=\textsc{irr-}let-\textsc{imp}\\
\glt `Ah, every bad job -- forget about them!’ [ulwa014\_55:20]
\z

Other \isi{semantic}ally quantifier-like words do not occur between \isi{object marker}s and verbs, as \textit{wopa} ‘all’ does; nor do they occur prenominally, as \textit{nunu} ‘every’ does. Rather, they behave more like prototypical property-denoting words -- that is, they follow the \isi{head noun} of an NP and precede any \isi{subject marker} or \isi{object marker}. For example, when \textit{kuma} ‘some’ modifies an object NP, it occurs before the \isi{object marker} (when present), as in \REF{ex:det:188} and \REF{ex:det:189}.

\is{determiner|)}
\is{quantifier|)}
\is{quantifier|(}
\is{determiner|(}

\ea%188
    \label{ex:det:188}
          \textit{Nï lamndu} \textbf{\textit{kuma}} \textit{ndasap.}\\
\gll    nï    lamndu  \textbf{kuma}  ndï=asa-p\\
    \textsc{1sg}  pig      some  \textsc{3pl}=hit-\textsc{pfv}\\
\glt `I killed some pigs.’ [elicited]
\z

\ea%189
    \label{ex:det:189}
          \textit{Nï lïmndï tïn} \textbf{\textit{kuma}} \textit{ndala.}\\
\gll    nï    lïmndï  tïn    \textbf{kuma}  ndï=ala\\
    \textsc{1sg}  eye    dog  some  \textsc{3pl}=see\\
\glt `I saw some dogs.’ [elicited]
\z

In examples \REF{ex:det:188} and \REF{ex:det:189}, \textit{kuma} ‘some’ could also have the reading ‘a few’ (that is, ‘some’ as opposed to ‘many’). For the sense ‘some of’ (that is, a \isi{partitive} quantity), the \isi{postposition} \textit{ul} ‘with’ is employed, as in \REF{ex:det:190}, \REF{ex:det:191}, and \REF{ex:det:192}.

\ea%190
    \label{ex:det:190}
          \textit{Nï lïmndï tïn} \textbf{\textit{ndul}} \textbf{\textit{kuma}} \textit{ndala.}\\
\gll    nï    lïmndï  tïn    ndï=\textbf{ul}    \textbf{kuma}  ndï=ala\\
    \textsc{1sg}  eye    dog  \textsc{3pl=}with  some  \textsc{3pl}=see\\
\glt `I saw some of the dogs.’ (Literally ‘I saw some with the dogs.’) [elicited]
\z

\ea%191
    \label{ex:det:191}
          \textit{Nï utam} \textbf{\textit{ndul}} \textbf{\textit{kuma}} \textit{amap.}\\
\gll    nï    utam  ndï=\textbf{ul}    \textbf{kuma}  ama-p\\
    \textsc{1sg}  yam  \textsc{3pl=}with  some  eat-\textsc{pfv}\\
\glt `I ate some of the yams.’ [elicited]
\z

\ea%192
    \label{ex:det:192}
          \textit{An lamndu} \textbf{\textit{ndul}} \textbf{\textit{kuma}} \textit{asap.}\\
\gll    an      lamndu  ndï=\textbf{ul}    \textbf{kuma}  asa-p\\
    1\textsc{pl.excl}  pig      \textsc{3pl=}with  some  hit-\textsc{pfv}\\
\glt `We killed some of the pigs.’ [elicited]
\z

Like other modifiers, \textit{kuma} ‘some’ can function as a substantive, whether in a subject NP, as in \REF{ex:det:193} and \REF{ex:det:194}; in a \isi{direct object} NP, as in \REF{ex:det:195} and \REF{ex:det:196}; or in an \isi{oblique} NP, as in \REF{ex:det:197}.

\ea%193
    \label{ex:det:193}
          \textbf{\textit{Kuma}} \textit{la woyambïn alanji wandam ala nakap.}\\
\gll    \textbf{kuma}  ala      woyambïn  ala-nji      wandam  ala na-kï-p\\
    some   \textsc{pl.dist}  pointlessly  \textsc{pl.dist{}-poss} jungle     \textsc{pl.dist}      \textsc{detr-}say-\textsc{pfv}\\
\glt `Some people claimed absurdly that those are their jungles.’ (Literally ‘those some’) [ulwa032\_50:23]
\z

\ea%194
    \label{ex:det:194}
          \textbf{\textit{Kuma}} \textit{mo ato anmbundata undana.}\\
\gll    \textbf{kuma}  ma=u      ata-u    an-mbï-unda-ta  unda-na\\
    some  3\textsc{sg.obj}=from  up-from  out-here-go\textsc{{}-cond} go-\textsc{irr}\\
\glt `If some go out from there, [they] will go.’ [ulwa032\_56:05]
\z

\ea%195
    \label{ex:det:195}
          \textit{Ndï} \textbf{\textit{kuma}} \textit{ndït nïnane nï wolka i.}\\
\gll    ndï  \textbf{kuma}  ndï=tï    nï-na-n-e        nï    wolka  i\\
    3\textsc{pl}  some  3\textsc{pl}=take  1\textsc{sg}=give-\textsc{pfv-dep}  1\textsc{sg}  again  go.\textsc{pfv}\\
\glt `They gave me some and I in turn went.’ [ulwa032\_56:34]
\z

\ea%196
    \label{ex:det:196}
          \textit{Mï} \textbf{\textit{kuma}} \textit{ndïnkap niya i.}\\
\gll    mï      \textbf{kuma}  ndï=nïkï-p    nï=iya      i\\
    3\textsc{sg.subj}  some  3\textsc{pl}=dig-\textsc{pfv}  1\textsc{sg=}toward  go.\textsc{pfv}\\
\glt `She dug some out and came to me.’ [ulwa042\_03:22]
\z

\ea%197
    \label{ex:det:197}
          \textit{Min mape} \textbf{\textit{kuman}} \textit{upe.}\\
\gll    min  ma=p-e      \textbf{kuma}=n  u-p-e\\
    3\textsc{du}  3\textsc{sg.obj}=be\textsc{{}-dep} some=\textsc{obl}  put-\textsc{pfv-dep}\\
\glt `The two are there and [they] planted some.’ [ulwa037\_56:11]
\z

Note the use of \isi{subject marker}s and \isi{object marker}s. While \textit{kuma} ‘some’ patterns mostly like other \isi{adjective}s, there is at least one quirk in its \isi{syntactic} patterning. To express a \isi{partitive} sense in the first \isi{person} or second \isi{person} (i.e., ‘some of us’, ‘some of you’, etc.), \textit{kuma} ‘some’ is placed after the relevant \isi{pronoun}, as in \REF{ex:det:198} and \REF{ex:det:199}.

\ea%198
    \label{ex:det:198}
          \textit{\textbf{Una kuma} apa mawnde isal monombam awe.}\\
\gll    \textbf{unan}    \textbf{kuma}  apa    ma=unda-e    i-si{}-al monombam  aw-e\\
    1\textsc{pl.incl}  some  house  3\textsc{sg.obj}=go-\textsc{ipfv}  hand-push-\textsc{pfv}    forehead    put-\textsc{ipfv}\\
\glt `Some of us go to church and pray.’ (Literally ‘We some go to the house and push hands on foreheads.’) [ulwa037\_09:17]
\z

\ea%199
    \label{ex:det:199}
          \textit{\textbf{Un kuma} anangani pe imot aye.}\\
\gll    \textbf{un}  \textbf{kuma}  an=angani      p-e    imot  a-e\\
    2\textsc{pl}  some  1\textsc{pl.excl=}behind  be\textsc{{}-dep} log    break-\textsc{ipfv}\\
\glt `Some of you are behind us, breaking firewood.’ [ulwa037\_58:47]
\z

Often, \textit{kuma} ‘some’ is used in contrastive statements, providing a \isi{correlative} structure (i.e., ‘some … others …’), as in \REF{ex:det:200} and \REF{ex:det:201}.

\ea%200
    \label{ex:det:200}
          \textbf{\textit{Kuma}} \textit{matïna} \textbf{\textit{kuma}} \textit{manakam.}\\
\gll    \textbf{kuma}  ma=tï-na      \textbf{kuma}  ma=na-kamb\\
    some  3\textsc{sg.obj}=take-\textsc{irr}  some  3\textsc{sg.obj}=\textsc{detr-}shun\\
\glt `Some wanted to get her; others didn’t want it.’ [ulwa032\_10:24]
\z

\ea%201
    \label{ex:det:201}
          \textit{An} \textbf{\textit{kuma}} \textit{matane} \textbf{\textit{kuma}} \textit{an mama u manke itïm awe.}\\
\gll    an      \textbf{kuma}  ma=ta-n-e          \textbf{kuma}  an mama  u    ma=nïkï-e      itïm  aw-e\\
    1\textsc{pl.excl}  some  3\textsc{sg.obj}=say-\textsc{ipfv-dep}  some  \textsc{1pl.excl}    mouth  from  3\textsc{sg.obj}=dig{}-\textsc{dep}  trash  put\textsc{{}-ipfv}\\
\glt `Some of us are saying it; but others of us are cutting it [= this good message] out of [our] mouths and putting [it] into the trash.’ [ulwa037\_13:53]
\z

In addition to \textit{wopa} ‘all’, \textit{nunu} ‘every’, and \textit{kuma} ‘some’ -- all of which, to varying degrees, behave unusually \isi{syntactic}ally -- there are several other words in Ulwa that have quantity-related meanings. First, the principal means of expressing a small amount or number is the word \textit{ilum} ‘piece’, which I consider primarily to be a noun, but which can also function as a modifier along with other nouns in an NP. Its various uses are illustrated by examples \REF{ex:det:203} through \REF{ex:det:206}.

\ea%203
    \label{ex:det:203}
          \textit{Nï ndïn u ma} \textbf{\textit{ilum}} \textit{kotïn.}\\
\gll    nï    ndï=n    u    ma      \textbf{ilum}  ko=tï-n\\
    1\textsc{sg}  3\textsc{pl=obl}  from  3\textsc{sg.obj}  piece  \textsc{indf}=take-\textsc{pfv}\\
\glt `I got a piece of it [= tobacco] from them.’ (Literally ‘its piece’) [ulwa037\_51:51]
\z

\ea%204
    \label{ex:det:204}
          \textit{An} \textbf{\textit{ilum}} \textit{mokop ndïnan.}\\
\gll    an      \textbf{ilum}  moko-p  ndï=na-n\\
    1\textsc{pl.excl}  piece  take-\textsc{pfv}  3\textsc{pl}=give-\textsc{pfv}\\
\glt `We gave them a little.’ [ulwa036\_03:57]
\z

\ea%205
    \label{ex:det:205}
          \textit{Inim} \textbf{\textit{ilum}} \textit{kuk nji up.}\\
\gll    inim  \textbf{ilum}  kuk  nji    u-p\\
    water  piece  gather  thing  put-\textsc{pfv}\\
\glt `[They] got a little water into something.’ [ulwa014\_68:18]
\z

\ea%206
    \label{ex:det:206}
          \textit{Nï nji} \textbf{\textit{ilumnï}} \textit{molnda.}\\
\gll    nï    nji    \textbf{ilum}=nï  ma=lu-nda\\
    1\textsc{sg}  thing  piece=\textsc{obl}  3\textsc{sg.obj}=put-\textsc{irr}\\
\glt `I will plant a few things there.’ [ulwa014\_07:51]
\z

The word \textit{kekaka} ‘one each’ may also be used to express a limited number.\footnote{This word has the alternative form \textit{kwekaka} ‘one each’.} The word seems to have derived as a \isi{calque} from \ili{Tok Pisin} \textit{wanwan} ‘one each’. It behaves primarily like an \isi{adverb}, as in \REF{ex:det:207} and \REF{ex:det:208}.

\is{determiner|)}
\is{quantifier|)}
\is{quantifier|(}
\is{determiner|(}

\ea%207
    \label{ex:det:207}
          \textit{An ango mïka} \textbf{\textit{kekaka}} \textit{inde.}\\
\gll    an      ango  maka  \textbf{kekaka}  inda-e\\
    1\textsc{pl.excl}  \textsc{neg}  thus  one.each  walk-\textsc{ipfv}\\
\glt `We wouldn’t walk one by one.’ (i.e., ‘We wouldn’t walk alone.’) [ulwa013\_04:11]
\z

\ea%208
    \label{ex:det:208}
          \textit{Ndï unanï} \textbf{\textit{kekaka}} \textit{inap.}\\
\gll    ndï  unan=nï    \textbf{kekaka}  ina-p\\
    3\textsc{pl}  1\textsc{pl.incl=obl}  one.each  get-\textsc{pfv}\\
\glt `They had just a few of us.’ (Literally ‘They got one-each with us.’ In other words, ‘Our parents didn’t have many children.’) [ulwa014\_46:36]
\z

To express large non-countable quantities, \isi{adjective}s such as \textit{ambi} ‘big’ are used, as in \REF{ex:det:209} and \REF{ex:det:210}.

\ea%209
    \label{ex:det:209}
          \textit{Inim} \textbf{\textit{ambi}} \textit{keka i.}\\
\gll    inim  \textbf{ambi}  keka      i\\
    water  big    completely  go.\textsc{pfv}\\
\glt `A lot of water has gone.’ [ulwa038\_04:39]
\z

\ea%210
    \label{ex:det:210}
          \textit{Ango ndïn wombasa anga} \textbf{\textit{ambi}} \textit{moke.}\\
\gll    ango  ndï=n    wombasa  anga  \textbf{ambi}  moko-e\\
    \textsc{neg}  \textsc{3pl=obl}  clay.pot  side  big    take-\textsc{ipfv}\\
\glt `[They] don’t get lots of money with them.’ [ulwa032\_58:18]
\z

For large countable quantities, the word \textit{tïngïn} ‘many’ is used. It patterns for the most part with other modifiers (i.e., \isi{adjective}s). Namely, it can appear after nouns and precede \isi{subject marker}s or \isi{object marker}s. Like other modifiers, it can also serve as a substantive (i.e., as the \isi{head} of a \isi{noun phrase}). That said, there does seem to be a tendency for \isi{object marker}s to be omitted from NPs containing (or consisting exclusively of) \textit{tïngïn} ‘many’. Perhaps this suggests that \textit{tïngïn} ‘many’ behaves differently from other modifiers. Alternatively it may simply support the idea that there is a correlation between lack of \isi{object marker}s and lack of \isi{definiteness} (\sectref{sec:7.2}). Sentences \REF{ex:det:211} through \REF{ex:det:214} exemplify the use of \textit{tïngïn} ‘many’.

\is{determiner|)}
\is{quantifier|)}

\ea%211
    \label{ex:det:211}
          \textit{Ulum ndï ankam} \textbf{\textit{tïngïn}} \textit{ndame.}\\
\gll    ulum  ndï  ankam  \textbf{tïngïn}  ndï=ama-e\\
    palm  3\textsc{pl}  person  many  3\textsc{pl}=eat-\textsc{ipfv}\\
\glt `The sago palms -- many people are eating them.’ [ulwa014\_10:36]
\z

\ea%212
    \label{ex:det:212}
          \textit{Apa ango} \textbf{\textit{tïngïn}} \textit{ndï mape.}\\
\gll    apa  ango    \textbf{tïngïn}  ndï  ma=p-e\\
    house  \textsc{neg}  many  3\textsc{pl}  \textsc{3sg.obj}=be\textsc{{}-ipfv}\\
\glt `There aren’t many houses there.’ [ulwa028\_04:5]
\z

\ea%213
    \label{ex:det:213}
          \textit{Unanji yalum ngala ndï} \textbf{\textit{tïngïnpe}}.\\
\gll unan-nji    yalum    ngala    ndï  \textbf{tïngïn}=p-e\\
    1\textsc{pl.incl-poss}  grandchild  \textsc{pl.prox}  3\textsc{pl}  many=\textsc{cop{}-dep}\\
\glt `We have many grandchildren.’ (Literally ‘These grandchildren of ours – they are many.’) [ulwa037\_42:14]
\z

\ea%214
    \label{ex:det:214}
          \textit{Anambi ango uta} \textbf{\textit{tïngïn}} \textit{asap.}\\
\gll    an-ambi    ango  uta    \textbf{tïngïn}  asa-p\\
    1\textsc{pl.excl-top}  \textsc{neg}  bird  many  hit-\textsc{pfv}\\
\glt `As for us, we didn’t kill many birds.’ [ulwa032\_54:11]
\z

\section{Numerals}\label{sec:7.5}

\is{numeral|(}
\is{determiner|(}
\is{cardinal numeral|(}

\isi{Cardinal numeral}s in Ulwa may be characterized as constituting a \is{base} \isi{quinary} (\isi{base-5}) \isi{numeral} system. There are four distinct words used to refer to the numbers one through four, none of which has been derived from another \isi{numeral}. The number 5 is used as a base for forming higher numerals: the numerals 6 through 9 are generally formed as [5(${\cdot}$1) + n], where “n” represents the numbers 1 through 4; multiples of 5 may be expressed as 5${\cdot}$2 (= 10), 5${\cdot}$3 (= 15), 5${\cdot}$4 (= 20), and 5${\cdot}$5 (=25). However, there does not seem to be strong conventionalization of numerals greater than 5 -- that is, there are multiple ways that speakers may refer to these numbers. In practice, Ulwa numerals (especially higher numerals) occur only rarely in discourse.

There is an alternative \is{base} \isi{decimal} (\isi{base-10}) \isi{numeral} system that may be used to refer to multiples of 10. Instead of being a complex term formed by multiplication (i.e., 5${\cdot}$2), the alternative word for 10 is simplex (i.e., monomorphemic). Multiples of 10 can thus be expressed as 10${\cdot}$2 (= 20), 10${\cdot}$3 (= 30), 10${\cdot}$4 (= 40), and 10${\cdot}$5 (= 50).

  There are no signs of any \is{base} \isi{vigesimal} (\isi{base-20}) \isi{numeral} system present in the Ulwa counting terms. Ulwa speakers also have no known cultural practice of \isi{body-part tallying}, as has been documented in parts of \isi{New Guinea} and \mbox{\isi{Australia}.}

  \isi{Cardinal numeral}s from ‘one’ through ‘ten’ are given in \tabref{tab:7.2}. While the forms of the numerals ‘one’ through ‘five’ are rather \isi{lexical}ized, the higher numerals may be expressed in a variety of ways. The forms in \tabref{tab:7.2} reflect the preferences of consultants who themselves may use other formulations as well.

\begin{table}
\caption{Cardinal numerals 1 through 10}
\label{tab:7.2}
\begin{tabularx}{\textwidth}{lQll}
\lsptoprule
& cardinal numeral & gloss & analysis\\
\midrule
1 & \textit{kwe} {\textasciitilde} \textit{kwa} & ‘one’ & 1\\
2 & {\itshape nini} & ‘two’ & 2\\
3 & {\itshape lele} & ‘three’ & 3\\
4 & {\itshape watangïnila} & ‘four’ & 4\\
5 & {\itshape angay {\textasciitilde} angay kwe} & ‘five’ & 5 {\textasciitilde} 5${\cdot}$1\\
6 & {\itshape angay kwe kwe mowon ndïwatlïp} & ‘six’ & 5${\cdot}$1+1\\
7 & {\itshape angay kwe nini minwon ndïwatlïp} & ‘seven’ & 5${\cdot}$1+2\\
8 & {\itshape angay kwe lele ndïwon ndïwatlïp} & ‘eight’ & 5${\cdot}$1+3\\
9 & {\itshape angay kwe watangïnila ndïwon ndïwatlïp} & ‘nine’ & 5${\cdot}$1+4\\
10 & {\itshape angay nini {{\textasciitilde}}} {\itshape nali {\textasciitilde} nali kwe} & ‘ten’ & 5${\cdot}$2 {\textasciitilde} 10 {\textasciitilde} 10${\cdot}$1\\
\lspbottomrule
\end{tabularx}
\end{table}

The word for ‘one’, which may be pronounced either [kwe] or [kwa], is undoubtedly related to the \isi{indefinite} \isi{object marker} \textit{ko=} ‘\textsc{indf}’, as well as to the \isi{modal adverb} \textit{ko} {\textasciitilde} \textit{kwa} ‘just’, the \isi{indefinite pronoun} \textit{kwa} ‘someone’, and the \isi{interrogative pronoun} \textit{kwa} ‘who?’.

  The word \textit{nini} ‘two’ bears some resemblance to the \isi{dual} forms \textit{min} \textsc{‘3du’}, \textit{ngin} \textsc{‘du.prox’}, and \textit{andin} \textsc{‘du.dist’}. The [-in] endings of these \isi{dual} forms likely reflect (a possibly \is{metathesis} metathesized form of) the word for ‘two’, which in \ili{Proto-Keram} was likely *ni. The Ulwa form \textit{nini} ‘two’ may represent a form of \is{iconicity} \isi{iconic reduplication} of \ili{Proto-Keram} *ni ‘two’.

  Similarly, the word \textit{lele} ‘three’ appears to contain \isi{reduplication}. Of course, there is less logical justification for calling this \is{iconicity} iconic, but perhaps the form was derived by \isi{analogy} to the preceding form in the series of numerals. This is, however, speculative.

  The word \textit{watangïnila} ‘four’ seems to be analyzable as: \textit{watangïn} ‘last, final’ plus \textit{ila} ‘sago palm frond’. In traditional timekeeping, days can be marked by the breaking of one \textit{ila} ‘sago palm frond’ each day. The word \textit{watangïnila} ‘four’, thus seems to mean something like ‘the last straw’.

\is{cardinal numeral|)}
\is{determiner|)}
\is{numeral|)}

\is{numeral|(}
\is{determiner|(}
\is{cardinal numeral|(}

  The word \textit{angay} ‘five’ is transparently derived from \textit{anga} ‘piece, side’ plus \textit{i} ‘hand, arm’. Optionally, the word \textit{kwe} ‘one’ may be added to this (i.e., \textit{angay kwe} ‘one five’). This reflects a \is{digit tallying} digit-based system of counting that underlies the \isi{quinary} numerical system -- that is, people start to count objects using the fingers of one hand. When all fingers have been extended (that is, when the number ‘five’ has been reached), they have created a single outstretched palm (that is, one ‘side’ of ‘hand’).

  The numerals ‘six’ through ‘nine’ usually contain verbal elements. Various \isi{periphrases} are possible, but typically they express the notion that numbers (probably in origin palm fronds or other counters) have been ‘cut’ and ‘added’ to (or ‘put atop’) the number five. Thus, one verbal expression of the number six is literally analyzable as in \REF{ex:det:215}.

\ea%215
    \label{ex:det:215}
          \textit{angay kwe kwe mowon ndïwatlïp}\\
    \gll anga-i    kwe  kwe  ma=won    ndï=wat-lï-p\\
    side-hand  one    one    3\textsc{sg.obj}=cut  3\textsc{pl}=atop-put-\textsc{pfv}\\
\glt `one side of hand [= five]; [someone] cut one and put [it] atop them’ (= six) [elicited]
\z

The expressions for the numbers seven \REF{ex:det:216}, eight \REF{ex:det:217}, and nine \REF{ex:det:218} may be analyzed similarly.

\ea%216
    \label{ex:det:216}
          \textit{angay kwe nini minwon ndïwatlïp}\\
\gll    anga-i    kwe  nini  min=won  ndï=wat-lï-p\\
    side-hand  one    two  \textsc{3du}=cut  3\textsc{pl}=atop-put-\textsc{pfv}\\
\glt `one side of hand [= five]; [someone] cut two and put [them] atop them’ (= seven) [elicited]
\z

\newpage

\ea%217
    \label{ex:det:217}
          \textit{angay kwe lele ndïwon ndïwatlïp}\\
\gll    anga-i    kwe  lele    ndï=won  ndï=wat-lï-p\\
    side-hand  one    three  3\textsc{pl}=cut  3\textsc{pl}=atop-put-\textsc{pfv}\\
\glt `one side of hand [= five]; [someone] cut three and put [them] atop them’ (= eight) [elicited]
\z

\ea%218
    \label{ex:det:218}
          \textit{angay kwe watangïnila ndïwon ndïwatlïp}\\
\gll    anga-i    kwe  watangïnila  ndï=won  ndï=wat-lï-p\\
    side-hand  one    four      \textsc{3pl}=cut  3\textsc{pl}=atop-put-\textsc{pfv}\\
\glt `one side of hand [= five]; [someone] cut four and put [them] atop them’ (= nine) [elicited]
\z

Other \isi{periphrases} are possible to express sums larger than five. In example \REF{ex:det:219} the speaker uses forms similar to those in \REF{ex:det:215}, but with the alternative form of the word for ‘one’; in example \REF{ex:det:220}, however, instead of using the \isi{metaphor} of ‘cutting’, the speaker uses the \isi{metaphor} of numbers being ‘thrown’ atop each other (i.e., ‘added’).

\ea%219
    \label{ex:det:219}
          \textit{Lucy mï manji \textbf{angay kwa kwe mowon ndïwatlïp}}.\\
\gll Lucy  mï      ma-nji      \textbf{angay}  \textbf{kwa}  \textbf{kwe}  \textbf{ma=won} \textbf{ndï=wat-lï-p}\\
    [name]  3\textsc{sg.subj}  \textsc{3sg.obj-poss}  five  one   one    \textsc{3sg.obj}=cut    3\textsc{pl}=atop-put-\textsc{pfv}\\
\glt `Lucy has six [children].’ [ulwa024\_01:10]
\z

\ea%220
    \label{ex:det:220}
          \textit{Nï nïnji tawatïp \textbf{angay kwe nini top ndïwatlïp}}.\\
\gll nï    nï-nji    tawatïp    \textbf{angay}  \textbf{kwe}   \textbf{nini}  \textbf{top} \textbf{ndï=wat-lï-p}\\
    1\textsc{sg}  1\textsc{sg-poss}  child    five  one    two  throw    \textsc{3pl}=atop-put-\textsc{pfv}\\
\glt ‘I have seven children.’ [ulwa013\_12:49]
\z

The \isi{numeral} 10 may be expressed as ‘five (times) two’. An alternate form, \textit{nali} ‘ten’, reflects the traditional system for counting larger numbers in Ulwa, as this word also refers to the spines of sago fronds, which were used to mark units of ten when counting larger sums.

The \isi{cardinal numeral}s from ‘ten’ through ‘twenty’ are given in \tabref{tab:7.3}. The \isi{numeral} 20 can be expressed either as ‘five (times) four’ or ‘ten (times) two’. It can also be denoted by the \isi{phrase} \textit{lamndu unduwan} ‘pig(’s) head’, a term reflecting modern Papua New Guinean currency, as the twenty-kina note has the picture of a pig’s head.

\begin{table}
\caption{Cardinal numerals 10 through 20}
\label{tab:7.3}
\begin{tabularx}{\textwidth}{lQll}
\lsptoprule
& cardinal numeral & gloss & analysis\\
\midrule
10 & {\itshape angay nini {{\textasciitilde}}} {\itshape nali {\textasciitilde} nali kwe} & ‘ten’ & 5${\cdot}$2 {\textasciitilde} 10 {\textasciitilde} 10${\cdot}$1\\
11 & {\itshape angay nini kwe mowon ndïwatlïp {{\textasciitilde}}}

{\itshape nali kwe kwe} & ‘eleven’ & 5${\cdot}$2+1 {\textasciitilde}

10${\cdot}$1+1\\
12 & {\itshape angay nini nini minwon ndïwatlïp {{\textasciitilde}}} {\itshape nali kwe nini} & ‘twelve’ & 5${\cdot}$2+2 {\textasciitilde}

10${\cdot}$1+2\\
13 & {\itshape angay nini lele ndïwon ndïwatlïp {{\textasciitilde}}}

{\itshape nali kwe lele} & ‘thirteen’ & 5${\cdot}$2+3 {\textasciitilde}

10${\cdot}$1+3\\
14 & {\itshape angay nini watangïnila ndïwon ndïwatlïp {{\textasciitilde}}}

{\itshape nali kwe watangïnila} & ‘fourteen’ & 5${\cdot}$2+4 {\textasciitilde}

10${\cdot}$1+4\\
15 & {\itshape angay lele} & ‘fifteen’ & 5${\cdot}$3\\
16 & {\itshape angay lele kwe mowon ndïwatlïp} & ‘sixteen’ & 5${\cdot}$3+1\\
17 & {\itshape angay lele nini minwon ndïwatlïp} & ‘seventeen’ & 5${\cdot}$3+2\\
18 & {\itshape angay lele lele ndïwon ndïwatlïp} & ‘eighteen’ & 5${\cdot}$3+3\\
19 & {\itshape angay lele watangïnila ndïwon ndïwatlïp} & ‘nineteen’ & 5${\cdot}$3+4\\
20 & {\itshape angay watangïnila {{\textasciitilde}}}

{\itshape nali nini {{\textasciitilde}}}

{\itshape lamndu unduwan} & ‘twenty’ & 5${\cdot}$4 {\textasciitilde}

10${\cdot}$2 {\textasciitilde}

‘pig head’\\
\lspbottomrule
\end{tabularx}
\end{table}

 Higher-number counting was probably not a common practice among Ulwa speakers before the introduction of a cash economy. Similarly, the \isi{numeral} 50 can be expressed either as ‘ten (times) five’ or as \textit{ankam unduwan} ‘person(’s) head’, this \isi{phrase} likewise reflecting the fact that the fifty-kina note contains the image of a man’s head (that of Prime Minister \name{Michael}{Somare}). Finally, the \isi{numeral} 100 is expressed as \textit{uta} (\textit{kwe}) ‘(one) bird’, similarly derived from the fact that the hundred-kina note contains the image of a “bird” (actually an airplane). Some higher \isi{cardinal numeral}s are given in \tabref{tab:7.4}.

\begin{table}
\caption{Some higher numerals}
\label{tab:7.4}
\begin{tabularx}{\textwidth}{lQll}
\lsptoprule
& cardinal numeral & gloss & analysis\\
\midrule
25 & {\itshape angay angay {{\textasciitilde}}} {\itshape nali nini angay} & ‘twenty-five’ & 5${\cdot}$5 {\textasciitilde} 10${\cdot}$2+5\\
30 & {\itshape nali lele} & ‘thirty’ & 10${\cdot}$3\\
40 & {\itshape nali watangïnila} & ‘forty’ & 10${\cdot}$4\\
50 & {\itshape nali angay {{\textasciitilde}}} {\itshape ankam unduwan} & ‘fifty’ & 10${\cdot}$5 {\textasciitilde}

‘person head’\\
60 & {\itshape ankam unduwan nali {\textasciitilde} ankam unduwan nali kwe} & ‘sixty’ & 50+10 {\textasciitilde} 50+10${\cdot}$1\\
70 & {\itshape ankam unduwan nali nini} & ‘seventy’ & 50+10${\cdot}$2\\
80 & {\itshape ankam unduwan nali lele} & ‘eighty’ & 50+10${\cdot}$3\\
90 & {\itshape ankam unduwan nali watangïnila} & ‘ninety’ & 50+10${\cdot}$4\\
100 & {\itshape uta {\textasciitilde} uta kwe} & ‘hundred’ & 100 (= ‘bird’) {\textasciitilde} 100${\cdot}$1\\
200 & {\itshape uta nini} & ‘two hundred’ & 100${\cdot}$2\\
300 & {\itshape uta lele} & ‘three hundred’ & 100${\cdot}$3\\
\lspbottomrule
\end{tabularx}
\end{table}

  When modifying \isi{noun phrase}s, \isi{cardinal numeral}s occur in the same position as (other) \isi{adjective}s -- that is, immediately following the \isi{noun phrase}. Numerals can modify either subjects or objects; in subject NPs, the \isi{subject marker} is somewhat unnecessary (at least in terms of it serving its common function of indexing \isi{number} -- \isi{singular}, \isi{dual}, or \isi{plural}), and it is thus often omitted, as in \REF{ex:det:221} and \REF{ex:det:222}.

\ea%221
    \label{ex:det:221}
          \textit{Tïn} \textbf{\textit{nini}} \textit{utam mamap.}\\
\gll    tïn    \textbf{nini}  utam  ma=ama-p\\
    dog  two  yam  3\textsc{sg.obj}=eat-\textsc{pfv}\\
\glt `Two dogs ate the yam.’ [elicited]
\z

\ea%222
    \label{ex:det:222}
          \textit{Tïn} \textbf{\textit{lele}} \textit{utam mamap.}\\
\gll    tïn    \textbf{lele}    utam  ma=ama-p\\
    dog  three  yam  3\textsc{sg.obj}=eat-\textsc{pfv}\\
\glt `Three dogs ate the yam.’ [elicited]
\z

Numerals are not often used to indicate the \isi{number} of referents in a subject, however. Indeed, the ubiquitous \isi{subject marker}s often offer clues to the quantity of multiple referents in a subject NP, especially when the \isi{number} of referents is exactly two, as in example \REF{ex:det:223}, which may be contrasted with example \REF{ex:det:224}, in which the \isi{number} of referents is more than two.

\ea%223
    \label{ex:det:223}
          \textit{Tïn} \textbf{\textit{min}} \textit{awal wandam i.}\\
\gll    tïn    \textbf{min}  awal    wandam  i\\
    dog  3\textsc{du}  yesterday  jungle    go.\textsc{pfv}\\
\glt `Two dogs went to the jungle yesterday.’ [elicited]
\z

\ea%224
    \label{ex:det:224}
          \textit{Tïn} \textbf{\textit{ndï}} \textit{awal wandam i.}\\
\gll    tïn    \textbf{ndï}  awal    wandam  i\\
    dog  \textsc{3pl}  yesterday  jungle    go\textsc{.pfv}\\
\glt `[Three or more] dogs went to the jungle yesterday.’ [elicited]
\z

\is{cardinal numeral|)}
\is{determiner|)}
\is{numeral|)}

\is{numeral|(}
\is{determiner|(}

Despite the redundancy, it is, however, possible for the \isi{dual} \isi{subject marker} to appear alongside the \isi{numeral} \textit{nini} ‘two’ \REF{ex:det:225}.

\ea%225
    \label{ex:det:225}
          \textit{Manji nungol \textbf{nini min} ndïlope.}\\
\gll    ma-nji      nungol  \textbf{nini}  \textbf{min}  ndï=lo-p-e\\
    3\textsc{sg.obj-poss}  child  two  3\textsc{du}  \textsc{3pl}=go-\textsc{pfv-dep}\\
\glt `His two sons went around in them [= jungle areas].’ [ulwa035\_01:27]
\z

When modifying object NPs, the \isi{numeral} (again, not commonly used in discourse), also appears immediately following the NP, as in \REF{ex:det:226} and \REF{ex:det:227}.

\ea%226
    \label{ex:det:226}
          \textit{Tïn mï mïnda (}\textbf{\textit{nini}}\textit{) minamap.}\\
\gll    tïn    mï      mïnda    (\textbf{nini})  min=ama-p\\
    dog  3\textsc{sg.subj}  banana    (two)  3\textsc{du}=eat-\textsc{pfv}\\
\glt `The dog ate two bananas.’ [elicited]
\z

\ea%227
    \label{ex:det:227}
          \textit{Tïn mï mïnda (}\textbf{\textit{lele}}\textit{) ndamap.}\\
\gll    tïn    mï      mïnda  (\textbf{lele})  ndï=ama-p\\
    dog  3\textsc{sg.subj}  banana  (three)  3\textsc{pl}=eat-\textsc{pfv}\\
\glt `The dog ate (three) bananas.’ [elicited]
\z

Thus \isi{object marker}s, which often identify the \isi{number} of direct-object referents, also frequently render the use of \isi{cardinal numeral}s redundant. Of course, for \isi{number}s greater than two, numerals are useful for specifying exact quantities, as in \REF{ex:det:228}.

\ea%228
    \label{ex:det:228}
          \textit{Maple mï apa mo mïnda} \textbf{\textit{lele}} \textit{ndïtïna.}\\
\gll    Maple  mï      apa    ma=u      mïnda    \textbf{lele} ndï=tï-na\\
    [name]  3\textsc{sg.subj}  house  3\textsc{sg.obj}=from  banana    three    3\textsc{pl}=take-\textsc{irr}\\
\glt `Maple will take three bananas from the house.’ [elicited]
\z

Similarly, the \isi{numeral} \textit{kwe} {\textasciitilde} \textit{kwa} ‘one’ may be used to modify the object of a verb along with the \isi{object marker} \textit{ma}= ‘3\textsc{sg.obj}’, as in \REF{ex:det:229}. In such instances, the \isi{indefinite} \isi{object marker} \textit{ko=} ‘\textsc{indf}’ is not used (\sectref{sec:7.2}).

\ea%229
    \label{ex:det:229}
          \textbf{\textit{Kwe}} \textit{mat manane.}\\
\gll    \textbf{kwe}  ma=tï    ma=na-n-e\\
    one    3\textsc{sg.obj}=take  3\textsc{sg.obj}=give-\textsc{pfv-dep}\\
\glt `[They] gave him one [= a fruit].’ [ulwa001\_11:09]
\z

Thus, \isi{subject marker}s and \isi{object marker}s typically reflect the same \isi{number} that is expressed by a \isi{numeral} in the same NP. However, when numerals express higher numbers, which are \isi{periphrastic}, a \isi{determiner} can actually agree with the final component number, as in \REF{ex:det:230}, where the \isi{dual} \isi{object marker} \textit{min=} ‘3\textsc{du}’ agrees with just the final ‘two’ component of the \isi{phrase} for ‘ten’.

\ea%230
    \label{ex:det:230}
          \textit{Angay} \textbf{\textit{nini}} \textbf{\textit{minat}}.\\
\gll angay  \textbf{nini}  \textbf{min}=atï\\
    five  two  3\textsc{du}=hit\\
\glt `Ten [days] passed.’ (Literally ‘[The days] hit ten.’) [ulwa001\_04:27]
\z

It could be argued that the \isi{object marker} in \REF{ex:det:230} should properly be \textit{ndï=} ‘3\textsc{pl}’ and not \textit{min=} ‘3\textsc{du}’, since the object is a \isi{number} greater than two (‘ten days’). However, assuming that this sentence is grammatical, the use of the \isi{dual} marker could be taken as evidence that numeral formulations such as \textit{angay nini} ‘ten’ are not \isi{lexical}ized, but rather are analyzable periphrases (e.g., this phrase has the literal meaning ‘two fives’).

  As modifiers, numerals can also be \isi{predicate complement}s to subjects, serving as the verbal element of a clause. They can thus host the \isi{copular enclitic} \textit{=p} ‘\textsc{cop}’ (\sectref{sec:10.2}) or be followed by the \isi{suppletive} \isi{past}-\isi{tense} \isi{locative verb} \textit{wap} ‘be.\textsc{pst’} (\sectref{sec:10.4}). \is{existential predication} Existential constructions specifying a particular \isi{number} of referents can take this form, as seen in \REF{ex:det:231}, \REF{ex:det:232}, and \REF{ex:det:233}.

\ea%231
    \label{ex:det:231}
          \textit{Tïn ndï} \textbf{\textit{lelep}}.\\
\gll tïn    ndï  \textbf{lele=p}\\
    dog  3\textsc{pl}  three=\textsc{cop}\\
\glt `There are three dogs.’ (Literally ‘The dogs are three.’) [elicited]
\z

\ea%232
    \label{ex:det:232}
          \textit{Tïn ndï} \textbf{\textit{lelepïna}}.\\
\gll tïn    ndï  \textbf{lele=p-na}\\
    dog  3\textsc{pl}  three=\textsc{cop}{}-\textsc{irr}\\
\glt `There will be three dogs.’ (Literally ‘The dogs will be three.’) [elicited]
\z

\newpage

\ea%233
    \label{ex:det:233}
          \textit{Tïn ndï ipka \textbf{lele wap}}.\\
\gll tïn    ndï  ipka  \textbf{lele}    \textbf{wap}\\
    dog  3\textsc{pl}  before  three  be\textsc{.pst}\\
\glt `There were three dogs before.’ (Literally ‘The dogs were three before.’) [elicited]
\z

However, it should be noted that these examples come from direct elicitation and I do not have any corpus data of natural speech illustrating this usage.

  There is no distinct set of \isi{ordinal numeral}s in Ulwa. The relative ordering of events must be accomplished with forms of the verbs \textit{ip ka-} ‘precede’ or \textit{angani ka-} ‘follow’ (\sectref{sec:9.2.3}). Forms of these verbs that are \isi{nominalize}d with the \isi{suffix} \textit{-en} ‘\textsc{nmlz}’ (\sectref{sec:3.2}) can be paired in \isi{apposition} with NPs, as in examples \REF{ex:det:234} and \REF{ex:det:235}.

\ea%234
    \label{ex:det:234}
          \textit{Nïnji} \textbf{\textit{ipken}} \textit{yana mï nip.}\\
\gll nï-nji    \textbf{ip-ka-en}    yana    mï      ni-p\\
    \textsc{1sg-poss}  nose-let-\textsc{nmlz}  woman    3\textsc{sg.subj}  die-\textsc{pfv}\\
\glt `My first wife died.’ (Literally ‘My wife, the one preceding, died.’) [elicited]
\z

\ea%235
    \label{ex:det:235}
          \textit{Nïnji} \textbf{\textit{anganiken}} \textit{yana}\textbf{ }\textit{mï nip.}\\
\gll    nï-nji    \textbf{angani-ka-en}    yana    mï      ni-p\\
    \textsc{1sg-poss}  behind-let-\textsc{nmlz}  woman    3\textsc{sg.subj}  die-\textsc{pfv}\\
\glt `My second wife died.’ (Literally ‘My wife, the one following, died.’) [elicited]
\z

Consider the contrast between the \isi{adverb}s \textit{ipka} ‘before, earlier, first’ \REF{ex:det:236} or \textit{anganika} ‘after, later, soon’ \REF{ex:det:238} with the related \isi{nominalize}d verb forms in examples \REF{ex:det:237} and \REF{ex:det:239}, respectively.

\ea%236
    \label{ex:det:236}
          \textit{Kapos mï} \textbf{\textit{ipka}} \textit{lamndu masap.}\\
\gll    Kapos  mï      \textbf{ipka}  lamndu  ma=asa-p\\
    [name]  3\textsc{sg.subj}  before  pig      3\textsc{sg.obj}=hit-\textsc{pfv}\\
\glt `Kapos killed the pig first.’ [elicited]
\z

\ea%237
    \label{ex:det:237}
          \textit{Kapos mï} \textbf{\textit{ipken}} \textit{lamndu masap.}\\
\gll    Kapos  mï      \textbf{ip-ka-en}    lamndu  ma=asa-p\\
    [name]  3\textsc{sg.subj}  nose{}-let{}-\textsc{nmlz}  pig      3\textsc{sg.obj}=hit-\textsc{pfv}\\
\glt `Kapos killed the first pig.’ [elicited]
\z


\ea%238
    \label{ex:det:238}
          \textit{Kapos mï} \textbf{\textit{anganika}} \textit{lamndu masap.}\\
\gll    Kapos  mï      \textbf{anganika}  lamndu  ma=asa-p\\
    [name]  3\textsc{sg.subj}  after    pig      3\textsc{sg.obj}=hit-\textsc{pfv}\\
\glt `Kapos killed the pig afterwards.’ [elicited]
\z

\ea%239
    \label{ex:det:239}
          \textit{Nomnga mï} \textbf{\textit{anganiken}} \textit{lamndu masap.}\\
\gll    Nomnga  mï      \textbf{angani-ka-en}    lamndu  ma=asa-p\\
    [name]    3\textsc{sg.subj}  behind-let-\textsc{nmlz}  pig      3\textsc{sg.obj}=hit-\textsc{pfv}\\
\glt `Nomnga killed the second pig.’ [elicited]
\z

The notion of ‘third’ may be denoted by the \isi{dual} \isi{object marker} \textit{min=} ‘\textsc{3du}’ \isi{clitic}izing to the verb \textit{angani ka-} ‘follow’ \REF{ex:det:240}, whereas ordinal notions greater than ‘third’ would be suggested by the \isi{plural} \isi{object marker} \textit{ndï=} ‘\textsc{3pl}’ \REF{ex:det:241}.

\ea%240
    \label{ex:det:240}
          \textit{Yokombla mï} \textbf{\textit{minanganiken}} \textit{lamndu masap.}\\
\gll    Yokombla  mï      \textbf{min=angani-ka-en}  lamndu ma=asa-p\\
    [name]    3\textsc{sg.subj}  3\textsc{du}=before-let-\textsc{nmlz}  pig    3\textsc{sg.obj}=hit-\textsc{pfv}\\
\glt `Yokombla killed the third pig.’ (Literally ‘Yokombla killed the pig, the one following   two.’) [elicited]
\z

\ea%241
    \label{ex:det:241}
          \textit{Amiwa mï} \textbf{\textit{ndanganiken}} \textit{lamndu masap.}\\
\gll    Amiwa  mï      \textbf{ndï=angani-ka-en}  lamndu  ma=asa-p\\
    [name]  3\textsc{sg.subj}  3\textsc{pl}=before-let-\textsc{nmlz}  pig      3\textsc{sg.obj}=hit-\textsc{pfv}\\
\glt `Amiwa killed the fourth pig.’ (Literally ‘Amiwa killed the pig, the one following them [more than two].’) [elicited]
\z

In \REF{ex:det:241}, \textit{ndanganiken} ‘the one following them’ could refer to any ordinal number fourth or greater (or third or greater, if \isi{plural} marking may be allowed for \isi{dual} referents, \sectref{sec:9.1.2}). Thus, there is no straightforward means of distinguishing \is{ordinal numeral} ordinals in Ulwa beyond first-second-third.

\is{determiner|)}
\is{numeral|)}