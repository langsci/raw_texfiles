\title{Social and structural aspects of language contact and change}
\author{Bettina Migge and Shelome Gooden}
\renewcommand{\lsSeries}{cam}
\renewcommand{\lsSeriesNumber}{5}

\renewcommand{\lsID}{302}
\lsCoverTitleSizes{45pt}{16mm}

\BackBody{This book brings together papers that discuss social and structural aspects of language contact and language change.

Several papers look at the relevance of historical documents to determine the linguistic nature of early contact varieties, while others investigate the specific processes of contact-induced change that were involved in the emergence and development of these languages. A third set of papers look at how new datasets and greater sensitivity to social issues can help to (re)assess persistent theoretical and empirical questions as well as help to open up new avenues of research. In particular they highlight the heterogeneity of contemporary language practices and attitudes often obscured in sociolinguistic research.

The contributions all focus on language variation and change but investigate it from a variety of disciplinary and empirical perspectives and cover a range of linguistic contexts.
}

\typesetter{Shelome Gooden, Sebastian Nordhoff}
\proofreader{Alexandra Fosså,
Amir Ghorbanpour,
Andreas Hölzl,
Cesar Perez Guarda,
Janina Rado,
Jeroen van de Weijer,
Jorina Fenner,
Laura Arnold,
Ludger Paschen,
Jean Nitzke,
Sandra Auderset,
Tihomir Rangelov
}

\renewcommand{\lsISBNdigital}{978-3-96110-347-8}
\renewcommand{\lsISBNhardcover}{978-3-98554-044-0}

\BookDOI{10.5281/zenodo.6602539}
