
\clearpage\setcounter{page}{1}\begin{stylelsAbstract}
\textbf{\textup{\textsuperscript{The expression of possibility in }}}\textbf{\textup{the Chabacano creoles and their adstrates }}
\end{stylelsAbstract}

\begin{stylelsAbstract}
\textup{Marivic Lesho}
\end{stylelsAbstract}

\begin{stylelsAbstract}
\textup{Franklin University}
\end{stylelsAbstract}

\begin{stylelsAbstract}
\textup{This study investigates how possibility is expressed in Zamboanga and Cavite Chabacano and two of their respective adstrates, Hiligaynon and Tagalog. Following }\textup{W}\textup{inford's (2000, 2018)}\textup{ call for creolists to use standard typological frameworks to describe creole modality, this study presents questionnaire data elicited for each language and classifies the modals according to categories proposed by }\textup{v}\textup{an der Auwera \& Plungian (1998}\textup{), }\textup{P}\textup{almer (2001}\textup{), and }\textup{M}\textup{atthewson et al. (2005}\textup{). The data demonstrate that all four languages have the same typological profile, with mixed Philippine and Spanish elements. }Pwede\textup{ ‘can’ ({\textless} Sp. }puede\textup{)} \textup{expresses deontic, dynamic, and epistemic possibility, as in Spanish, and in the creoles, it is also marks nonvolitional circumstances, parallel to Philippine }ma(ka)-\textup{. Epistemic possibility is marked primarily by adverbs in each language, however, with }siguro\textup{ ‘possibly/probably’ ({\textless} Sp. }seguro\textup{ ‘sure’) flexibly able to mark necessity. The data support recent proposals }\textup{(}\textup{Fernández 2006, 2012a;}\textup{ }\textup{S}\textup{ippola \& Lesho 2020)}\textup{ that the Chabacano varieties are highly similar not because they descended from a single ancestor but because their adstrates are so closely related. In fact, the adstrate modal systems are nearly identical.}
\end{stylelsAbstract}


\setcounter{listWWNumiiileveli}{0}
\begin{listWWNumiiileveli}
\item 
\begin{stylelsSectioni}
Introduction
\end{stylelsSectioni}
\end{listWWNumiiileveli}
\begin{styleStandard}
This paper describes how possibility is expressed in two Philippine-Spanish creoles, Zamboanga and Cavite Chabacano, and two of their respective substrates/adstrates, Hiligaynon and Tagalog. Studies focusing on the modal systems of these languages from a semantic or typological perspective have been scarce. As Winford (2000, 2018) has observed, modality has been a neglected area of creole studies, due in large part to reliance on the coarse-grained categorization assumed by Bickerton's (1984) concept of the ‘prototypical’ creole tense–mood–aspect (TMA) system, which he proposed as part of his language bioprogram hypothesis. This approach assumes Irrealis as a grammatical category shared across creoles, with none of the more fine-grained distinctions that are commonly assumed in other TMA literature (e.g. Dahl 1985, Bybee et al. 1994, and Palmer 2001). There are other reasons why creole modality is relatively under explored, such as the fact that some semantic distinctions do not always occur spontaneously during naturalistic field recordings, and thus do not make it into the initial descriptions of a language; however, there is no denying that 'Bickerton’s rather idiosyncratic terminology and framework left a profound mark on the way research on creole TMA was conducted' for decades (Winford 2018:1). 
\end{styleStandard}

\begin{styleStandard}
Following these tendencies, previous Chabacano research has often discussed the preverbal TMA markers (e.g., Forman 1972, Lipski \& Santoro 2007), but only Sippola's (2011) Ternate Chabacano grammar has provided any systematic description of modality. Focusing on two other Chabacano varieties and providing comparison with the adstrates, this paper describes how different subtypes of possibility are marked. It compares the overall modal typology of these languages from a crosslinguistic perspective and discusses how Spanish contact shaped not only the creoles but also their closely related adstrates in this complex contact setting. 
\end{styleStandard}

\begin{listWWNumiiileveli}
\item 
\begin{stylelsSectioni}
Background
\end{stylelsSectioni}

\begin{listWWNumiiilevelii}
\item 
\begin{stylelsSectionii}
Chabacano formation
\end{stylelsSectionii}
\end{listWWNumiiilevelii}
\end{listWWNumiiileveli}
\begin{styleStandard}
Chabacano is a collective name for several creoles spoken in the Philippines. In the Manila Bay region in the north, Chabacano is spoken in the towns of Ternate and Cavite City, and it was also once used in Manila and nearby provinces (Fernández 2011). Other Chabacano varieties are spoken in Mindanao in the south, mainly in Zamboanga City and the surrounding region, including nearby islands like Basilan and Jolo. It is also spoken to some extent in Cotabato, and was formerly used in Davao. 
\end{styleStandard}

\begin{styleStandard}
Tagalog (Central Philippine) is the substrate/adstrate for the Manila Bay varieties. For the Mindanao varieties, Cebuano and Hiligaynon are generally considered the main substrates/adstrates, though several local languages are spoken there (e.g., Yakan and Tausug). Hiligaynon and Cebuano are in the Visayan branch of Central Philippine but are still closely related to Tagalog. In addition, the country's official languages, English and Filipino (a standardized variety based mainly on Tagalog), are now widely used across all regions. Spanish no longer has any presence in daily life for most Filipinos.
\end{styleStandard}

\begin{styleStandard}
The Chabacano varieties are strikingly similar. They have 95\% similarity on the 100-word Swadesh list (Sippola 2011:27) and share 64\% of the value assignments for 107 features in the \textit{Atlas of Pidgin and Creole Studies} (APiCS) (Michaelis et al. 2013). Speakers of the different varieties usually say they can understand each other; however, they generally consider their varieties to be distinct, due to the geographical distance between them, lexical influence from the different adstrates, and accent differences (Lesho \& Sippola 2014). Historically, these three communities have not had much interaction with each other.
\end{styleStandard}

\begin{styleStandard}
The similarity among the varieties has led to the assumption that they are all directly related. Whinnom (1956) proposed that they all descend from Ternate Chabacano, which he believed originally came from a Portuguese-based contact language transplanted to the Philippines in 1659, when 200 families were transferred from the Moluccas to the Ermita district of Manila, and later resettled in Ternate. This variety would have been relexifed with Spanish and emulated by people in Manila and Cavite, and then later spread to Zamboanga after Tagalog soldiers were supposedly transferred there in 1719. 
\end{styleStandard}

\begin{styleStandard}
Several aspects of this theory have since been disputed, however, including the supposed Portuguese influence (Lipski 1988); the dates being too early, based on the timing of certain Spanish sound changes that are reflected in Chabacano (Fernández \& Sippola 2017); and the historical accuracy of the events Whinnom described (Fernández 2006, 2011, 2012a, 2012b, 2019). For example, Tagalog speakers in Ermita would have already had intensive contact with Spanish before 1659 and thus would not have needed to borrow an outside contact vernacular to communicate (Fernández 2011, 2012a). Historical records also do not support the idea that Tagalog soldiers played any significant role in transmitting a creole to Zamboanga (Fernández 2019). 
\end{styleStandard}

\begin{styleStandard}
Many Chabacano scholars now agree that the varieties developed for the most part independently of each other (Lipski 1992; Fernández 2011, 2012a, 2012b; Fernández \& Sippola 2017; Sippola \& Lesho 2020). Lipski (1992:12), for example, proposed that rather than being a simple transplant of Manila Bay Chabacano, the Zamboanga variety first formed locally in the mid-1700s 'as the natural intersection of Philippine languages which shared cognate grammatical systems, and which had already absorbed a significant quantity of Hispanisms'. He suggested that any input from Manila Bay would have been introduced only after this initial formation.
\end{styleStandard}

\begin{styleStandard}
More recent work by Fernández (2006, 2011, \ 2012a, 2012b, 2019) and Fernández \& Sippola (2017), based on meticulous archival research as well as linguistic evidence, has suggested later timelines and even more separate trajectories of formation. Historical accounts or literary texts featuring any mentions or examples of Chabacano were scarce before the late 1800s (see also Lipski 2013). The oldest account that seems to refer to a fully restructured variety comes from 1806, in reference to Cavite, and the earliest Chabacano texts that have been identified so far come from 1859 1860 in Manila (Fernández \& Sippola 2017). These facts suggest that Chabacano did not crystallize in Cavite and Manila until the late 18th or early 19th centuries, though Ternate Chabacano formed somewhat earlier (Fernández \& Sippola 2017). For Zamboanga, historical accounts indicate that Spanish was widely used by local people there during the early colonial period, and that Chabacano developed much later, only after the population expanded in the late 19th century (Fernández 2006, 2012b). 
\end{styleStandard}

\begin{styleStandard}
If the Chabacano varieties do not have a direct genetic relationship, then how is their similarity explained? The answer is that Tagalog and the Visayan languages (and the Philippine family overall) share remarkably similar syntactic structures, morphemes, and semantic features, which led to similar restructuring outcomes in each contact situation (Sippola \& Lesho 2020). It is also the case that the Chabacano varieties are not quite as homogenous as they have appeared to be at first glance. If all varieties really descended from one source, we might expect them to be even more similar than they are. Yet it is well known that each variety has its own unique set of pronouns; for example, the 1PL is \textit{mihotro} in Ternate, \textit{niso} in Cavite, and \textit{kame} (exclusive) and \textit{kita} (inclusive) in Zamboanga (Lesho \& Sippola 2014:14; Lipski 2013:457). By comparing previously published documentation of each variety, Lesho \& Sippola (2014:9–16) identified several other lexical, phonological, and morphosyntactic differences among the varieties that have previously been overlooked. 
\end{styleStandard}

\begin{styleStandard}
Sippola \& Lesho (2020) compared reciprocal marking, argument marking, and modality in Cavite, Ternate, and Zamboanga Chabacano. We demonstrated that these are areas where the varieties have often followed different grammaticalization paths, suggesting that they do not stem from one source. For example, Zamboanga and Cavite Chabacano use the Philippine circumfix \textit{man-V-han} as well as the Spanish-derived construction \textit{uno'y otro} or \textit{uno a otro} ('one to another') to mark reciprocal actions, whereas Ternate Chabacano grammaticalized \textit{hugá} {\textless} (Sp. jugar 'play'), as in \textit{hugá keré} 'love each other' (Sippola \& Lesho 2020:115–116). A brief overview of the modals also showed instances where each Chabacano variety functions similarly but grammaticalized different elements from the lexifier. For example, each variety has a necessity verb grammaticalized from different Spanish sources: \textit{ne(se)sita} in Zamboanga ({\textless} Sp. \textit{necesita} 'need'), \textit{debi} in Cavite ({\textless} Sp. \textit{debe} 'must, owe'), and \textit{dabli} in Ternate ({\textless} Sp. \textit{dable} 'possible, feasible') (Sippola \& Lesho 2020:112–113). Taken together with the sociohistorical evidence (e.g., Fernández \& Sippola 2017), the differences that have now been documented (Lesho \& Sippola 2014, Sippola \& Lesho 2020) provide support for the idea that these three Chabacano varieties each developed locally, though the replication of related adstrate features led to similar results in each case. This paper further reinforces these points, following up on Sippola \& Lesho (2020) by offering an even more detailed investigation of the Chabacano modal systems and comparing them directly to those of the adstrates.
\end{styleStandard}

\begin{listWWNumiiileveli}
\item 
\begin{listWWNumiiilevelii}
\item 
\begin{stylelsSectionii}
Possibility in Chabacano
\end{stylelsSectionii}
\end{listWWNumiiilevelii}
\end{listWWNumiiileveli}
\begin{styleStandard}
Previous Chabacano descriptions have mostly glossed over modality, focusing more on the aspect markers: perfective \textit{ya}, imperfective \textit{ta}, and the future \textit{ay} (in Mindanao) or \textit{di} (in Manila Bay). \textit{Ay} and \textit{di} have been described as marking ‘future or unreal events’ (Lipski \& Santoro 2007:380) or ‘both future and modality’ (Lorenzino 2000:58). It is clear, however, that the future markers do not cover all types of irrealis/modal events; in fact, no creole seems to be set up that way (Winford 2018). 
\end{styleStandard}

\begin{styleStandard}
Frake (1980:297–301) listed the possibility verb \textit{puede} ‘able’ ({\textless} Sp. \textit{puede} ‘able.3SG) among several other modals in Zamboanga Chabacano. He defined it as expressing ‘ability because of physical circumstances such as one’s strength or the lack of external strength’ (Frake 1980:298). However, there was no indication of whether it can be extended to permission or epistemic possibility, as Spanish \textit{puede} and Tagalog \textit{puwede} both can be. 
\end{styleStandard}

\begin{styleStandard}
Rubino (2008) examined how Zamboanga Chabacano \textit{puede} marks ‘potentive mode’, a category in Cebuano that includes not only ability but also ‘actions that are brought about accidentally, coincidentally, or without volition or instigation’ (Rubino 2008:279). For example, in Error: Reference source not found, the speaker is not discussing their ability or intention to die but rather an eventual death that will occur beyond their control.
\end{styleStandard}


\setcounter{listWWNumiileveli}{0}
\begin{listWWNumiileveli}
\item 
\begin{stylelsLanginfo}
\textit{Si yo }\textbf{\textit{puede}}\textit{ muri enterra kamo kumigo aki na presioso sitio.}
\end{stylelsLanginfo}
\end{listWWNumiileveli}
\begin{stylelsIMT}
If 1\textsc{sg} able die bury 2\textsc{pl} \textsc{1sg.acc} here \textsc{loc} precious site 
\end{stylelsIMT}

\begin{stylelsIMT}
\ \ {}'If I die, bury me here in this precious site.' (Rubino 2008:292)
\end{stylelsIMT}

\begin{styleStandard}
Sippola (2011), following Palmer's (2001) framework of modal description, categorized modal markers in Ternate Chabacano. She described \textit{pwede} ‘able’ as expressing deontic permission, epistemic possibility, or dynamic ability, and also marking accidental events, as in Error: Reference source not found. The verb \textit{mari} ‘able’ ({\textless} Tag. \textit{maaari} 'able') expresses the same range of notions, except for perhaps epistemic possibility (such uses did not occur in Sippola's dataset; if they do exist, they must have very low frequency).
\end{styleStandard}

\begin{listWWNumiileveli}
\item 
\begin{stylelsLanginfo}
\textit{A-}\textbf{\textit{mari/pwedi}}\textit{ miyá yo na mi panti.}
\end{stylelsLanginfo}
\end{listWWNumiileveli}
\begin{stylelsIMT}
\textsc{pfv}{}-able/able pee 1\textsc{sg} \textsc{loc} 1.\textsc{poss} underwear 
\end{stylelsIMT}

\begin{stylelsIMT}
\ \ {}'I (accidentally) peed on my underwear.' (Sippola 2011:163–164)
\end{stylelsIMT}

\begin{stylelsLanginfo}
In addition, she described two epistemic adverbs, \textit{baka} ‘possibly’ and \textit{sigúru} ‘possibly’, listing the latter as marking uncertainty and the former as marking conjecture (Sippola 2011:208–209). 
\end{stylelsLanginfo}

\begin{stylelsLanginfo}
These few studies show insight into an under-explored area of Chabacano grammar and creoles more generally. In particular, Sippola (2011) offers a point of comparison for work on other Chabacano varieties.
\end{stylelsLanginfo}


\setcounter{listWWNumiiileveli}{0}
\begin{listWWNumiiileveli}
\item 

\setcounter{listWWNumiiilevelii}{0}
\begin{listWWNumiiilevelii}
\item 
\begin{stylelsSectionii}
Possibility in the adstrates
\end{stylelsSectionii}
\end{listWWNumiiilevelii}
\end{listWWNumiiileveli}
\begin{styleStandard}
Tagalog has four ability/possibility markers: the ‘pseudoverbs’ \textit{maaari}, \textit{puwede} ({\textless} Sp. \textit{puede}), and \textit{kaya}, and the verbal prefix \textit{ma(ka)}{}-.\textstyleFootnoteSymbol{ }\footnote{\textit{Maka-} is used in actor voice and \textit{ma- }in other voices (e.g., object or locative voice). \textit{Ma- }can also be used in actor voice for stative verbs. The initial consonant becomes [n] when the action is [+begun] (i.e., in the perfective or imperfective).}\ The pseudoverbs (labeled as such because they are not inflected for aspect) exhibit some syntactic differences; unlike \textit{puwede} and \textit{maaari}, \textit{kaya} ‘ability, power’ functions as a ‘modal noun’ (Kroeger 1993). How they contrast in meaning, however, is not always clear. For \textit{maaari} and \textit{puwede}, Schachter \& Otanes (1972:261) indicate only that the latter is less formal. They described both as covering ability, permission, and possibility. Asarina \& Holt (2005:14–15) described them as taking deontic/dynamic readings, and \textit{kaya} as being ‘strongly preferred’ in dynamic contexts. 
\end{styleStandard}

\begin{styleStandard}
\textit{Ma(ka)-} marks ability in some contexts, but it can also mean that the action was not deliberate, as in (\ref{bkm:Ref52727157}).
\end{styleStandard}


\setcounter{listWWNumiileveli}{0}
\begin{listWWNumiileveli}
\item 
\begin{stylelsLanginfo}
\label{bkm:Ref52727157}\textbf{\textit{naka}}\textit{gamit siya ng manggang hilaw}
\end{stylelsLanginfo}
\end{listWWNumiileveli}
\begin{stylelsIMT}
\ \ \textsc{pfv.av}.able.use 3\textsc{sg} \textsc{gen} mango.\textsc{lnk} unripe 
\end{stylelsIMT}

\begin{stylelsIMT}
\ \ {}'he was able to use a green mango' or 'he happened to use a green mango' (Schachter \& Otanes 1972:330)
\end{stylelsIMT}

\begin{styleStandard}
\textit{Ma(ka)-} has been identified as a marker of ‘non-volitive mood’ (Kroeger 1993) or ‘ability/involuntary action’ (AIA) in Tagalog (Schachter \& Otanes 1972), and the ‘potentive mode’ in Cebuano (Rubino 2008). Dell (1983) noted that Tagalog AIA verbs have an actuality entailment in the perfective. For example, the neutral form \textit{itinulak} ‘pushed’ in (4) does not entail that the rock actually moved; in (5), however, the perfective form with \textit{na-} entails that it did.
\end{styleStandard}

\begin{listWWNumiileveli}
\item 
\begin{stylelsLanginfo}
\textit{Itinulak ni Ben ang bato.}
\end{stylelsLanginfo}
\end{listWWNumiileveli}
\begin{stylelsIMT}
\ \ \textsc{pfv.ov}.push \textsc{gen }Ben \textsc{nom }rock 
\end{stylelsIMT}

\begin{stylelsIMT}
\ \ {}'Ben pushed the rock.' (Dell 1983:179–180)
\end{stylelsIMT}

\begin{listWWNumiileveli}
\item 
\begin{stylelsLanginfo}
\textbf{\textit{Na}}\textit{itulak ni Ben ang bato.}
\end{stylelsLanginfo}
\end{listWWNumiileveli}
\begin{stylelsIMT}
\ \ \textsc{pfv.ov}.able.push \textsc{gen }Ben \textsc{nom }rock 
\end{stylelsIMT}

\begin{stylelsIMT}
\ \ {}'Ben managed to push the rock’ or ‘Ben accidentally pushed the rock.' (Dell 1983:179–180)
\end{stylelsIMT}

\begin{styleStandard}
AIA forms assert not only that a maneuver has been executed but that a result has also been achieved, whether intentionally or not (hence the two possible readings in example 5).
\end{styleStandard}

\begin{styleStandard}
While \textit{ma(ka)-} marks dynamic/potentive verbs, \textit{ma-} is also part of a separate but overlapping paradigm marking statives (Himmelmann 2006). Stative \textit{ma-} can mark bodily conditions/emotional states (\textit{magutom }‘be hungry’, \textit{matakot} ‘be scared’), positional predicates (\textit{maupo} ‘be seated’), and perception (\textit{makita} ‘see’; Himmelmann 2006:491–494). In addition, the related adjectival \textit{ma-} denotes qualities/properties associated with the root (\textit{maliit} ‘small’, \textit{mabato} ‘stony/having stones’).\footnote{\textit{Ma(ka)-} and adjectival \textit{ma-} also exist somewhat productively in Chabacano with Spanish or even English-origin roots, e.g. Cav. \textit{maka-irrita} ‘be irritated’ and \textit{ma-}quarantine 'be quarantined' (Escalante 2010:98–105), or Zam. \textit{mapuersa }‘strong’ (i.e. ‘having strength’). } These various functions derive from Proto-Malayo-Polynesian \textit{*ma-}, which marked involuntariness and states (Evans \& Ross 2001).
\end{styleStandard}

\begin{styleStandard}
There has been less detailed examination of Hiligaynon \textit{ma(ka)-}, but Spitz (2002:383–384) lists examples showing that it covers a similar range of meanings to those described by Himmelmann (2006). In addition to AIA uses, Spitz observed that \textit{ma-} marks experiences related to cognition/perception, body function, happenstance, or a lack of control over the circumstances (e.g., it occurs with roots like \textit{kita} ‘see’, \textit{uhaw} ‘thirst’, \textit{subo} ‘sad’, \textit{patay} ‘die’, etc.). Interestingly, Rubino (2008:292) showed that Chabacano pwede occurs with verbs like \textit{murí} ‘die’. Thus, \textit{pwede} seems to map onto not only potentive \textit{ma(ka)-} but also at least some of these stative uses of \textit{ma}{}-.
\end{styleStandard}

\begin{styleStandard}
Asarina \& Holt (2005) noted that epistemic modality is not expressed in Tagalog through modal verbs. While this claim is not strictly accurate, since \textit{puwede} and \textit{maaari} can be used epistemically (Schachter \& Otanes 1972), it is true that epistemicity is expressed primarily through adverbs. Bader et al. (1994) listed \textit{baka}, \textit{posible} ({\textless} Sp. \textit{posible} ‘possible’), and \textit{siguro} ({\textless} Sp. seguro ‘sure’) as the main epistemic adverbs; however, they glossed all three as ‘perhaps’, so it is unclear how they might be distinct. This ambiguity is also present in dictionaries for Tagalog, Hiligaynon, and Chabacano. For example, Tagalog \textit{baka}, \textit{siguro}, and even the verb \textit{maaari} have all been defined as both ‘maybe/possibly’ and ‘probably/probable’ (English 2008:785).
\end{styleStandard}


\setcounter{listWWNumiiileveli}{0}
\begin{listWWNumiiileveli}
\item 
\begin{stylelsSectioni}
Methods
\end{stylelsSectioni}


\setcounter{listWWNumiiilevelii}{0}
\begin{listWWNumiiilevelii}
\item 
\begin{stylelsSectionii}
Data collection
\end{stylelsSectionii}
\end{listWWNumiiilevelii}
\end{listWWNumiiileveli}
\begin{styleStandard}
The data come primarily from elicitation using Dahl's (1985:198–206) TMA questionnaire, modified to include a few additional examples of intentional vs. non-intentional actions. The questionnaire was completed through elicitation sessions with speakers of Zamboanga Chabacano, Cavite Chabacano, Hiligaynon, and Tagalog, as summarized in Table 1.
\end{styleStandard}

\begin{flushleft}
\tablefirsthead{}
\tablehead{}
\tabletail{}
\tablelasttail{}
\begin{supertabular}{|m{1.0038599in}|m{0.78165984in}|m{0.44835988in}|m{1.1976599in}|m{1.1622599in}|}
\hline
\textbf{Language} &
\textbf{Consultants} &
\textbf{Ages} &
\textbf{Interview location} &
\textbf{Other languages}\\\hline
Zamboanga Chabacano &
6 &
30s-70s &
Atlantic City, New Jersey (5 living there, 1 visiting from Zamboanga) &
At least 4 others, some combination of: Filipino, English, Cebuano, Hiligaynon, Yakan, Tausug, Spanish or limited Spanish\\\hline
Cavite Chabacano &
3 &
50s-60s &
San Diego, California (1)

Cavite City (2) &
Tagalog/Filipino, English, limited Spanish\\\hline
Tagalog/Filipino &
1 &
20s &
Columbus, Ohio &
English, limited Hiligaynon\\\hline
Hiligaynon &
1 &
20s &
Columbus, Ohio &
Filipino, English, Kinaray-a\\\hline
\end{supertabular}
\end{flushleft}
\begin{styleStandard}
\textit{Table 1: Consultant backgrounds.}
\end{styleStandard}

\begin{styleStandard}
Most of these consultants were interviewed in the US at different points during 2009–2010 and had been living there for several years, though all originally came from the Philippines and continued to use their various languages, since they maintained Filipino social networks in both countries. Additional elicitation took place in Cavite City with two Cavite Chabacano consultants in 2016. Since Tagalog and Hiligaynon are already well documented compared to the creoles, there was less focus on finding several speakers; one speaker of each was enough to obtain parallel examples to the Chabacano data.
\end{styleStandard}

\begin{styleStandard}
All of the consultants were multilingual. Zamboanga is a particularly linguistically diverse region, so the Zamboangueños all knew at least four languages. Only one person (from Zamboanga) reported being proficient in Spanish. All the other Chabacano speakers had some exposure to Spanish, either through college classes or from older relatives who used to speak it, but did not use it regularly or list it among the languages they speak.
\end{styleStandard}

\begin{styleStandard}
The consultants all came from socioeconomic backgrounds that could be considered middle class within their respective countries of residence (for example, most had occupations such as nursing or office work). Some of the Zamboangueños originally came from Zamboanga City and some from Basilan, but there were no noticeable differences in the grammatical patterns elicited for this study (although the participants sometimes commented on phonological features that vary by region). Similarly, the Cavite Chabacano speakers were uniform in their modal usage, though they came from neighborhoods with slight accent differences (Lesho 2018). The Hiligaynon and Tagalog consultants’ speech was representative of their original hometowns (Iloilo and Manila, respectively).
\end{styleStandard}

\begin{styleStandard}
The elicitation involved presenting the consultants with pragmatic contexts from the questionnaire and asking how they would respond to them. Additional follow-up questions were added spontaneously as needed (e.g., whether they could use another modal in the same context, or if they could think of any other situations where the modal might be used). Negative evidence was also obtained by presenting the consultants with alternative utterances to confirm that using a certain modal in that context would be infelicitous. \ 
\end{styleStandard}

\begin{styleStandard}
The Dahl (1985) questionnaire is useful for systematically eliciting parallel examples for several speakers/languages and obtaining examples that may not happen to arise over the course of a natural conversation. There can be pragmatic gaps, however, and it can be difficult to elicit an intended meaning unless enough context is invented for the speakers. For this reason, a few other speakers in the Philippines were consulted for occasional follow-up questions as needed (usually via email). Occasionally, data from other sources (e.g., blogs, news websites, or grammars) were also used for additional evidence. These few examples from alternative sources are clearly labeled when presented in §4.
\end{styleStandard}

\begin{listWWNumiiileveli}
\item 
\begin{listWWNumiiilevelii}
\item 
\begin{stylelsSectionii}
Framework
\end{stylelsSectionii}
\end{listWWNumiiilevelii}
\end{listWWNumiiileveli}
\begin{styleStandard}
As Winford (2000, 2018) has observed, until fairly recently, creole TMA systems have often been described in terms of the three ‘prototypical creole’ categories proposed by Bickerton (1984): Anterior tense, Irrealis mood, and Nonpunctual aspect. As he has shown, however, this classification is too broad to result in accurate grammatical descriptions; it ignores the richness and complexity of individual creole systems, and it obscures variation when comparing creoles. It also makes it difficult to conduct crosslinguistic comparison to the substrates, lexifiers, or other languages in general. Instead, Winford (2000) argued for examining creole TMA systems within standard typological and semantic frameworks (e.g., Bybee et al. 1994, Palmer 2001), demonstrating the value of this approach with his analysis of Sranan. While many creole TMA studies have since moved beyond the ‘first phase’ of research within Bickerton’s paradigm and into a ‘second phase’ relying on more standard terminology and frameworks of analysis, modality is still an often-neglected area (Winford 2018).
\end{styleStandard}

\begin{styleStandard}
Following Winford’s recommendations, this paper examines deontic, dynamic, and epistemic possibility in Chabacano and its adstrates within frameworks more commonly used in crosslinguistic typological studies, primarily those used by Palmer (2001) and van der Auwera \& Plungian (1998). These types of possibility are illustrated in (\ref{bkm:Ref52740224})\ for English.
\end{styleStandard}


\setcounter{listWWNumiileveli}{0}
\begin{listWWNumiileveli}
\item 
\begin{stylelsLanginfo}
\label{bkm:Ref52740224}Possibility:
\end{stylelsLanginfo}


\setcounter{listWWNumiilevelii}{0}
\begin{listWWNumiilevelii}
\item 
\begin{stylelsLanginfo}
Deontic: \textit{John }\textbf{\textit{can/may}}\textit{ swim} (e.g., his father gave him permission).
\end{stylelsLanginfo}
\item 
\begin{stylelsLanginfo}
Dynamic: \textit{John }\textbf{\textit{can}}\textit{ swim} (i.e., he is able/knows how).
\end{stylelsLanginfo}
\item 
\begin{stylelsLanginfo}
Epistemic: \textit{John }\textbf{\textit{could/might}}\textit{ be swimming} (e.g., you are speculating on his whereabouts).
\end{stylelsLanginfo}
\end{listWWNumiilevelii}
\end{listWWNumiileveli}
\begin{styleStandard}
Possibility contrasts with necessity for each of these modal categories, as shown in the following examples.
\end{styleStandard}

\begin{listWWNumiileveli}
\item 
\begin{stylelsLanginfo}
Necessity:
\end{stylelsLanginfo}

\begin{listWWNumiilevelii}
\item 
\begin{stylelsLanginfo}
Deontic: \textit{John }\textbf{\textit{must/has to}}\textit{ swim} (e.g., his coach told him to).
\end{stylelsLanginfo}
\item 
\begin{stylelsLanginfo}
Dynamic: \textit{John }\textbf{\textit{must}}\textit{ swim back to shore} (e.g., because his boat sank).
\end{stylelsLanginfo}
\item 
\begin{stylelsLanginfo}
Epistemic: \textit{John }\textbf{\textit{must}}\textit{ be swimming} (e.g., you know he usually does so every day at this hour).
\end{stylelsLanginfo}
\end{listWWNumiilevelii}
\end{listWWNumiileveli}
\begin{styleStandard}
According to Palmer (2001), the deontic and dynamic categories comprise the broader category of ‘event modality’. Deontic modality involves the imposition of some external authority (see 6a, 7a). In the dynamic examples, in contrast, the situations arise from general circumstances related to John’s own ability or the facts of the situation. Epistemic modality differs from event modality because it involves the speaker’s assessment of the truth value of a proposition, based on the available knowledge of the situation.
\end{styleStandard}

\begin{styleStandard}
Dynamic modality can include several notions, such as physical or mental ability, learned ability, or possibility arising from circumstances affecting the situation (Palmer 2001). The latter type is distinct from the first two in that, similar to deontic possibility, the possibility arises externally rather than internally to the participant (van der Auwera \& Plungian 1998). These different subtypes of dynamic possibility are illustrated for English in (\ref{bkm:Ref52740274}).
\end{styleStandard}

\begin{listWWNumiileveli}
\item 
\begin{stylelsLanginfo}
\label{bkm:Ref52740274}Dynamic possibility:
\end{stylelsLanginfo}

\begin{listWWNumiilevelii}
\item 
\begin{stylelsLanginfo}
Participant-internal:
\end{stylelsLanginfo}
\end{listWWNumiilevelii}
\end{listWWNumiileveli}

\setcounter{listWWNumiiileveli}{0}
\begin{listWWNumiiileveli}
\item 

\setcounter{listWWNumiiilevelii}{0}
\begin{listWWNumiiilevelii}
\item 

\setcounter{listWWNumiiileveliii}{0}
\begin{listWWNumiiileveliii}
\item 

\setcounter{listWWNumiiileveliv}{0}
\begin{listWWNumiiileveliv}
\item 
\begin{styleListParagraph}
Learned ability: \textit{John }\textbf{\textit{can}}\textit{ read} (i.e., he knows how).
\end{styleListParagraph}
\begin{stylelsLanginfo}
Capability: \textit{John }\textbf{\textit{can}}\textit{ run a marathon} (i.e., he has the physical ability and mental drive).
\end{stylelsLanginfo}
\end{listWWNumiiileveliv}
\end{listWWNumiiileveliii}
\end{listWWNumiiilevelii}
\end{listWWNumiiileveli}

\setcounter{listWWNumiileveli}{0}
\begin{listWWNumiileveli}
\item 

\setcounter{listWWNumiilevelii}{0}
\begin{listWWNumiilevelii}
\item 
\begin{stylelsLanginfo}
Participant-external:
\end{stylelsLanginfo}
\end{listWWNumiilevelii}
\end{listWWNumiileveli}

\setcounter{listWWNumiiileveli}{0}
\begin{listWWNumiiileveli}
\item 

\setcounter{listWWNumiiilevelii}{0}
\begin{listWWNumiiilevelii}
\item 

\setcounter{listWWNumiiileveliii}{0}
\begin{listWWNumiiileveliii}
\item 

\setcounter{listWWNumiiileveliv}{0}
\begin{listWWNumiiileveliv}
\item 
\begin{stylelsSourceline}
\textup{Circumstantial possibility: }John \textbf{can} go to the beach whenever he wants\textup{ (e.g., he has the time or lives near it).}
\end{stylelsSourceline}
\end{listWWNumiiileveliv}
\end{listWWNumiiileveliii}
\end{listWWNumiiilevelii}
\end{listWWNumiiileveli}
\begin{styleStandard}
Rather than dynamic modality, some semanticists (e.g., Kratzer 1991, Matthewson et al. 2005) refer to a category of circumstantial modality, which includes examples like that in (8b). I follow Palmer (2001) in using ‘dynamic’ as the overarching label because it encompasses the diverse range of notions illustrated in (\ref{bkm:Ref52740274}), and reserve the label ‘circumstantial’ as one of its subtypes. Recognizing the distinctions in (\ref{bkm:Ref52740274})\ is important in describing languages like Tagalog, which encodes each of those notions in a different way (see §4). \ 
\end{styleStandard}

\begin{styleStandard}
In addition, I propose that the various potentive/AIA uses of \textit{pwede} and \textit{ma(ka)-} all fall under the category of participant-external dynamic possibility (which I take to be synonymous with circumstantial possibility). This categorization is inspired by the work of Matthewson et al. (2005) and Davis et al. (2009) on St’át’imcets, a Salishan language that has intriguing similarities to Austronesian AIA marking. Like \textit{ma(ka)-}, St’át’imcets \textit{ka-…-a} encompasses not only ‘be able to’ meanings but also a range of ‘out-of-control’ or nonvolitional contexts (including accidents, sudden events, ‘manage to’ readings, etc.). While the semantic parallels to the Philippine and creole languages in this study are not exact (Davis et al. 2009:219), the similarities to St’át’imcets are close enough to justify grouping all these different semantic notions under the umbrella of participant-external dynamic or circumstantial modality. 
\end{styleStandard}

\begin{styleStandard}
I also rely on a framework proposed by Matthewson et al. (2005) to consider the overall typology of the modal systems of the Chabacano varieties and their adstrates. They observed that modals in different languages can be lexically specified along two dimensions: modal base (i.e., circumstantial/dynamic, deontic, or epistemic), and modal force (i.e., weak modals used in possibility contexts vs. strong modals used in necessity contexts). In St’át’imcets, modals are always specified for base but not force. For example, \textit{ka-…-a} is used only in circumstantial contexts and cannot take deontic or epistemic readings, but it has flexible force. The ability reading is the weaker one (e.g., in possibility contexts like ‘six people \textbf{can} fit in that car’), and the nonvolitional reading is the stronger one (e.g., in necessity contexts like ‘Gertie \textbf{must/had to} sneeze’, because she had a cold; Davis et al. 2009:228, 231). Similarly, St’át’imcets deontic and epistemic modals take on both strong and weak meanings depending on the discourse context, but they are fixed to one modal base. This type of modal system is the opposite of languages like English or Spanish, which have a tendency to specify modals for force but not base; for example, \textit{can/could }takes either deontic, epistemic, or circumstantial/dynamic readings depending on the discourse context, but it always indicates possibility and not necessity (which is covered by the stronger \textit{must}).
\end{styleStandard}

\begin{styleStandard}
Later research within this framework has found that some languages have ‘mixed’ modal systems compared to St’át’imcets and English. Gitskan, for example, has circumstantial modals that are specified for modal force (as in English), but the epistemic modals have flexible force (as in St’át’imcets; Matthewson 2013:350). Paciran Javanese also has a strictly epistemic marker that is used with flexible force (necessity or possibility), and a necessity modal with fixed force that can be used in various non-epistemic modal contexts (Vander Klok 2013). In this paper, I consider how the Chabacano varieties and their adstrates fit into this typology. I also discuss the overall similarities and differences in the Chabacano modal systems to address questions about the relationships between these varieties and how they developed.
\end{styleStandard}

\begin{listWWNumiiileveli}
\item 
\begin{stylelsSectioni}
Data
\end{stylelsSectioni}
\end{listWWNumiiileveli}
\begin{styleStandard}
This section presents examples of how each language in this study expresses possibility in deontic, dynamic, and epistemic contexts. A few examples of how necessity is expressed are also included to provide overall context and show how the modals contrast in meaning. The examples come from the questionnaire unless otherwise specified.
\end{styleStandard}

\begin{listWWNumiiileveli}
\item 
\begin{listWWNumiiilevelii}
\item 
\begin{stylelsSectionii}
Deontic possibility
\end{stylelsSectionii}
\end{listWWNumiiilevelii}
\end{listWWNumiiileveli}
\begin{styleStandard}
Each language uses the verb \textit{pwede} to express deontic possibility or permission, as illustrated in (\ref{bkm:Ref1510171})\ and (\ref{bkm:Ref52741551}). The examples here (and throughout §4) also highlight how structurally similar each language is overall. For example, the default word order is VSO (as in 10) unless a noun is topicalized (as in the Chabacano varieties in 9). The Chabacano tendency to place to place pronouns between the auxiliary and main verb also comes from both adstrates, as shown in (\ref{bkm:Ref52741551}). \ 
\end{styleStandard}


\setcounter{listWWNumiileveli}{0}
\begin{listWWNumiileveli}
\item 
\begin{stylelsLanginfo}
[Giving permission for a child to stay over at their house.] \label{bkm:Ref1510171}
\end{stylelsLanginfo}
\end{listWWNumiileveli}
\begin{stylelsLanginfo}
‘The child \textbf{can }stay/sleep here tonight.’
\end{stylelsLanginfo}

\begin{listWWNumiileveli}
\item 
\begin{listWWNumiilevelii}
\item 
\begin{stylelsLanginfo}
Zam.
\end{stylelsLanginfo}
\end{listWWNumiilevelii}
\end{listWWNumiileveli}
\begin{stylelsSourceline}
El bata \textbf{pwede }est\'{a} ak\'{i} esta noche.
\end{stylelsSourceline}

\begin{stylelsIMT}
\textsc{det }child able stay here \textsc{det }night
\end{stylelsIMT}

\begin{listWWNumiileveli}
\item 
\begin{listWWNumiilevelii}
\item 
\begin{stylelsLanginfo}
Cav.
\end{stylelsLanginfo}
\end{listWWNumiilevelii}
\end{listWWNumiileveli}
\begin{stylelsSourceline}
El kratura \textbf{pwede} estar ak\'{i} esta noche.
\end{stylelsSourceline}

\begin{stylelsIMT}
\textsc{det }child able stay here \textsc{det} night
\end{stylelsIMT}

\begin{listWWNumiileveli}
\item 
\begin{listWWNumiilevelii}
\item 
\begin{stylelsLanginfo}
Hil.
\end{stylelsLanginfo}
\end{listWWNumiilevelii}
\end{listWWNumiileveli}
\begin{stylelsSourceline}
\textbf{Pwede} magtener ang bata diri subong nga gab-i.
\end{stylelsSourceline}

\begin{stylelsSourceline}
\textup{able }\textsc{av}\textup{.stay}\textup{ }\textsc{nom}\textup{ child here now}\textup{ }\textsc{lnk}\textup{ night}
\end{stylelsSourceline}

\begin{listWWNumiileveli}
\item 
\begin{listWWNumiilevelii}
\item 
\begin{stylelsLanginfo}
Tag.
\end{stylelsLanginfo}
\end{listWWNumiilevelii}
\end{listWWNumiileveli}
\begin{stylelsSourceline}
\textbf{Puwede }siyang matulog dito ngayong gabi.
\end{stylelsSourceline}

\begin{stylelsIMT}
able 3\textsc{sg.nom.lnk}\textit{ }\textsc{stv}.sleep here today.\textsc{lnk}\textit{ }night
\end{stylelsIMT}

\begin{listWWNumiileveli}
\item 
\begin{stylelsLanginfo}
\label{bkm:Ref52741551}[Mother to child: ‘If you behave…’] 
\end{stylelsLanginfo}
\end{listWWNumiileveli}
\begin{stylelsLanginfo}
‘You \textbf{can} go to the beach/sea with your friends.’
\end{stylelsLanginfo}

\begin{listWWNumiileveli}
\item 
\begin{listWWNumiilevelii}
\item 
\begin{stylelsLanginfo}
Zam.
\end{stylelsLanginfo}
\end{listWWNumiilevelii}
\end{listWWNumiileveli}
\begin{stylelsSourceline}
\textbf{Pwede} tu andá na mar hunto kon tu amigo.
\end{stylelsSourceline}

\begin{stylelsIMT}
able 2\textsc{sg} go \textsc{loc} sea together with 2\textsc{sg}.\textsc{poss} friend
\end{stylelsIMT}

\begin{listWWNumiileveli}
\item 
\begin{listWWNumiilevelii}
\item 
\begin{stylelsLanginfo}
Cav.
\end{stylelsLanginfo}
\end{listWWNumiilevelii}
\end{listWWNumiileveli}
\begin{stylelsSourceline}
\textbf{Pwede/di pud\'{i}} tu andá na apláya kompañero mga amigo.
\end{stylelsSourceline}

\begin{stylelsIMT}
able/\textsc{fut} able 2\textsc{sg} go \textsc{loc} beach companion \textsc{pl} friend
\end{stylelsIMT}

\begin{listWWNumiileveli}
\item 
\begin{listWWNumiilevelii}
\item 
\begin{stylelsLanginfo}
Hil.
\end{stylelsLanginfo}
\end{listWWNumiilevelii}
\end{listWWNumiileveli}
\begin{stylelsSourceline}
\textbf{Pwede} ka magkadto sa dagat kaupod imo mga abyan\textup{.}
\end{stylelsSourceline}

\begin{stylelsIMT}
able 2\textsc{sg}.\textsc{nom av.}go \textsc{loc} sea companion 2\textsc{sg.gen pl} friend
\end{stylelsIMT}

\begin{listWWNumiileveli}
\item 
\begin{listWWNumiilevelii}
\item 
\begin{stylelsLanginfo}
Tag.
\end{stylelsLanginfo}
\end{listWWNumiilevelii}
\end{listWWNumiileveli}
\begin{stylelsSourceline}
\textbf{Puwede/maaari} kang pumunta sa beach kasama ng mga kaibigan mo.
\end{stylelsSourceline}

\begin{stylelsIMT}
able/able 2\textsc{sg.nom.lnk av}.go \textsc{loc} beach companion \textsc{gen pl} friend 2\textsc{sg.gen}
\end{stylelsIMT}

\begin{styleStandard}
While the Tagalog consultant accepted the verb \textit{maaari} in deontic contexts (see 10d), she found it old-fashioned. \textit{Kaya}, however, was not acceptable (see Error: Reference source not found. For Hiligaynon and the Chabacano varieties, no cognates to Tagalog \textit{maaari} or Ternate Chabacano \textit{mári} were found.
\end{styleStandard}

\begin{listWWNumiileveli}
\item 
\begin{stylelsLanginfo}
Tag.
\end{stylelsLanginfo}
\end{listWWNumiileveli}
\begin{stylelsLanginfo}
\textbf{\textit{\#Kaya}}\textit{ niyang matulog dito ngayong gabi.}
\end{stylelsLanginfo}

\begin{stylelsLanginfo}
\textbf{\ \ }capability 3\textsc{sg.gen.lnk stv}.sleep here today.\textsc{lnk} night
\end{stylelsLanginfo}

\begin{stylelsSourceline}
\textup{\ \ {}'The boy }\textbf{\textup{can}}\textup{ stay here tonight.'}
\end{stylelsSourceline}

\begin{styleStandard}
Tagalog and Hiligaynon can use \textit{ma(ka)-} in similar contexts (see 12). In Hiligaynon (12b), it can occur alongside \textit{pwede}.
\end{styleStandard}

\begin{listWWNumiileveli}
\item 
\begin{stylelsLanginfo}
‘You \textbf{can} go to the beach/river with your friends.’
\end{stylelsLanginfo}

\begin{listWWNumiilevelii}
\item 
\begin{stylelsLanginfo}
Tag.
\end{stylelsLanginfo}
\end{listWWNumiilevelii}
\end{listWWNumiileveli}
\begin{stylelsSourceline}
\textbf{Maka}kapunta ka sa beach kasama ng mga kaibigan\textup{.}
\end{stylelsSourceline}

\begin{stylelsIMT}
able.\textsc{fut}.go 2\textsc{sg} \textsc{loc} beach companion \textsc{gen pl }friend
\end{stylelsIMT}

\begin{listWWNumiileveli}
\item 
\begin{listWWNumiilevelii}
\item 
\begin{stylelsLanginfo}
Hil.
\end{stylelsLanginfo}
\end{listWWNumiilevelii}
\end{listWWNumiileveli}
\begin{stylelsSourceline}
\textbf{Pwede} ka \textbf{maka}kadto sa suba kaupod imo mga abyan.
\end{stylelsSourceline}

\begin{stylelsIMT}
able 2\textsc{sg.nom} able.\textsc{av}.go \textsc{loc} river companion 2\textsc{sg.gen pl} friend
\end{stylelsIMT}

\begin{stylelsLanginfo}
\textit{Pwede} and \textit{ma(ka)-} are not used to indicate deontic necessity. Instead, necessity is expressed with the pseudoverbs \textit{kinanlan} in Hiligaynon (a reduced form of \textit{kinahanglan}), and \textit{kailangan} in Tagalog (\ref{bkm:Ref51841803}).
\end{stylelsLanginfo}

\begin{listWWNumiileveli}
\item 
\begin{stylelsLanginfo}
\label{bkm:Ref51841803}[A mother is speaking to a child.]
\end{stylelsLanginfo}
\end{listWWNumiileveli}
\begin{stylelsLanginfo}
‘You \textbf{must} wash your hands before you eat.’
\end{stylelsLanginfo}

\begin{listWWNumiileveli}
\item 
\begin{listWWNumiilevelii}
\item 
\begin{stylelsLanginfo}
Zam.
\end{stylelsLanginfo}
\end{listWWNumiilevelii}
\end{listWWNumiileveli}
\begin{stylelsSourceline}
\textbf{Nesesita} tu labá tu mano antes de komé.
\end{stylelsSourceline}

\begin{stylelsIMT}
must 2\textsc{sg} wash 2\textsc{sg}.\textsc{poss} hand before of eat
\end{stylelsIMT}

\begin{listWWNumiileveli}
\item 
\begin{listWWNumiilevelii}
\item 
\begin{stylelsLanginfo}
Cav.
\end{stylelsLanginfo}
\end{listWWNumiilevelii}
\end{listWWNumiileveli}
\begin{stylelsSourceline}
\textbf{Debi} tu labá tu mano bago tu kumí.
\end{stylelsSourceline}

\begin{stylelsIMT}
must 2\textsc{sg} wash 2\textsc{sg}.\textsc{poss} hand before 2\textsc{sg} eat
\end{stylelsIMT}

\begin{listWWNumiileveli}
\item 
\begin{listWWNumiilevelii}
\item 
\begin{stylelsLanginfo}
Hil.
\end{stylelsLanginfo}
\end{listWWNumiilevelii}
\end{listWWNumiileveli}
\begin{stylelsSourceline}
\textbf{Kinanlan} maghugas ka sang imo kamot bago ka magkaon.
\end{stylelsSourceline}

\begin{stylelsIMT}
must \textsc{av}.wash 2\textsc{sg} \textsc{gen} 2\textsc{sg.gen} hand before 2\textsc{sg} \textsc{av}.eat
\end{stylelsIMT}

\begin{listWWNumiileveli}
\item 
\begin{listWWNumiilevelii}
\item 
\begin{stylelsLanginfo}
Tag.
\end{stylelsLanginfo}
\end{listWWNumiilevelii}
\end{listWWNumiileveli}
\begin{stylelsSourceline}
\textbf{Kailangan} maghugas ka ng kamay bago ka kumain.
\end{stylelsSourceline}

\begin{stylelsIMT}
must \textsc{av}.wash 2\textsc{sg} \textsc{gen} hand before 2\textsc{sg} \textsc{av}.eat
\end{stylelsIMT}

\begin{stylelsLanginfo}
In the adstrates, \textit{dapat} 'should' could replace \textit{kinanlan} and \textit{kailangan} in (\ref{bkm:Ref51841803})\ to indicate a suggestion or general social obligation (i.e., 'you should wash your hands before you eat'). This is the only sense in which Zamboanga Chabacano can use the modal \textit{debe}, as in (\ref{bkm:Ref52721750}).
\end{stylelsLanginfo}

\begin{listWWNumiileveli}
\item 
\begin{stylelsLanginfo}
\label{bkm:Ref52721750}Zam.
\end{stylelsLanginfo}
\end{listWWNumiileveli}
\begin{stylelsLanginfo}
\textbf{\textit{Debe}}\textit{ era yo está na kasa pero ya andá yo na party.}
\end{stylelsLanginfo}

\begin{stylelsLanginfo}
\textbf{\ \ }should \textsc{cf} 1\textsc{sg} stay \textsc{loc} house but \textsc{pfv} go 1\textsc{sg} \textsc{loc} party
\end{stylelsLanginfo}

\begin{stylelsSourceline}
\textup{\ \ {}'I }\textbf{\textup{should}}\textup{ have stayed home, but I went to a party.'}
\end{stylelsSourceline}


\setcounter{listWWNumiiileveli}{0}
\begin{listWWNumiiileveli}
\item 

\setcounter{listWWNumiiilevelii}{0}
\begin{listWWNumiiilevelii}
\item 
\begin{stylelsSectionii}
Dynamic possibility
\end{stylelsSectionii}
\end{listWWNumiiilevelii}
\end{listWWNumiiileveli}
\begin{styleStandard}
There are some differences in how these four languages mark participant-internal dynamic possibility (including learned ability and capability) and participant-external dynamic possibility (i.e., circumstantial possibility, including all contexts that could be covered by the AIA marker in the adstrates). In general, the creoles make fewer lexical distinctions than the adstrates. 
\end{styleStandard}

\begin{listWWNumiiileveli}
\item 
\begin{stylelsSectioniii}
Learned ability
\end{stylelsSectioniii}
\end{listWWNumiiileveli}
\begin{styleStandard}
Examples of contexts involving learned ability, including either knowledge or physical skill, are shown in (15) and (16). The Zamboanga Chabacano consultants accepted either \textit{sabe} or \textit{pwede} in these contexts, whereas the Cavite Chabacano consultants accepted only \textit{sabe}. Tagalog uses \textit{marunong} ‘knowledgeable.’ 
\end{styleStandard}


\setcounter{listWWNumiileveli}{0}
\begin{listWWNumiileveli}
\item 
\begin{stylelsLanginfo}
‘The child \textbf{can} read well’ (i.e., they know how).
\end{stylelsLanginfo}


\setcounter{listWWNumiilevelii}{0}
\begin{listWWNumiilevelii}
\item 
\begin{stylelsLanginfo}
Zam.
\end{stylelsLanginfo}
\end{listWWNumiilevelii}
\end{listWWNumiileveli}
\begin{stylelsSourceline}
El bata \textbf{sabe/pwede} le enbwenamente.
\end{stylelsSourceline}

\begin{stylelsIMT}
\textsc{det} child know/able read well
\end{stylelsIMT}

\begin{listWWNumiileveli}
\item 
\begin{listWWNumiilevelii}
\item 
\begin{stylelsLanginfo}
Cav.
\end{stylelsLanginfo}
\end{listWWNumiilevelii}
\end{listWWNumiileveli}
\begin{stylelsSourceline}
Akel kratura \textbf{sabe/\#pwede} le bweno.
\end{stylelsSourceline}

\begin{stylelsIMT}
\textsc{det} child know/can read good
\end{stylelsIMT}

\begin{listWWNumiileveli}
\item 
\begin{listWWNumiilevelii}
\item 
\begin{stylelsLanginfo}
Tag.
\end{stylelsLanginfo}
\end{listWWNumiilevelii}
\end{listWWNumiileveli}
\begin{stylelsSourceline}
\textbf{Marunong} siyang magbasa.
\end{stylelsSourceline}

\begin{stylelsIMT}
knowledgeable 3\textsc{sg.lnk} \textsc{av}.read
\end{stylelsIMT}

\begin{listWWNumiileveli}
\item 
\begin{stylelsLanginfo}
‘The child \textbf{can} swim well.’
\end{stylelsLanginfo}

\begin{listWWNumiilevelii}
\item 
\begin{stylelsLanginfo}
Zam.
\end{stylelsLanginfo}
\end{listWWNumiilevelii}
\end{listWWNumiileveli}
\begin{stylelsSourceline}
El bata \textbf{sabe/pwede} nadá enbwenamente.
\end{stylelsSourceline}

\begin{stylelsIMT}
\textsc{det} child know/able swim well
\end{stylelsIMT}

\begin{listWWNumiileveli}
\item 
\begin{listWWNumiilevelii}
\item 
\begin{stylelsLanginfo}
Cav.
\end{stylelsLanginfo}
\end{listWWNumiilevelii}
\end{listWWNumiileveli}
\begin{stylelsSourceline}
Akel kratura \textbf{sábe} nadá bweno.
\end{stylelsSourceline}

\begin{stylelsIMT}
\textsc{det} child know swim good
\end{stylelsIMT}

\begin{listWWNumiileveli}
\item 
\begin{listWWNumiilevelii}
\item 
\begin{stylelsLanginfo}
Tag.
\end{stylelsLanginfo}
\end{listWWNumiilevelii}
\end{listWWNumiileveli}
\begin{stylelsSourceline}
\textbf{Marunong} lumangoy ang bata.
\end{stylelsSourceline}

\begin{styleStandard}
knowledgeable \textsc{av}.swim \textsc{nom} child
\end{styleStandard}

\begin{styleStandard}
Because these examples involve actions that are performed well, the Hiligaynon consultant responded using \textit{maayo} ‘good,’ as in (\ref{bkm:Ref53237001}); however, (\ref{bkm:Ref53237112})\ shows that the form for ‘can/knows how’ is \textit{kahibalo}. 
\end{styleStandard}

\begin{listWWNumiileveli}
\item 
\begin{stylelsLanginfo}
\label{bkm:Ref53237001}Hil.
\end{stylelsLanginfo}
\end{listWWNumiileveli}
\begin{stylelsLanginfo}
\textbf{\textit{Maayo}}\textit{ maglamoy ang bata.}
\end{stylelsLanginfo}

\begin{stylelsIMT}
\ \ good \textsc{av}.swim \textsc{nom} child
\end{stylelsIMT}

\begin{stylelsLanginfo}
\ \ {}'The child is good at swimming.'
\end{stylelsLanginfo}

\begin{listWWNumiileveli}
\item 
\begin{stylelsLanginfo}
\label{bkm:Ref53237112}Hil.
\end{stylelsLanginfo}
\end{listWWNumiileveli}
\begin{stylelsLanginfo}
\textit{Ako man }\textbf{\textit{kahibalo}}\textit{ magluto ‘sina‘.}
\end{stylelsLanginfo}

\begin{stylelsIMT}
\ \ 1\textsc{sg} also know \textsc{av}.cook that
\end{stylelsIMT}

\begin{stylelsLanginfo}
\ \ {}'I also \textbf{know how} to cook that.' (Wolfenden 1971:53)
\end{stylelsLanginfo}


\setcounter{listWWNumiiileveli}{0}
\begin{listWWNumiiileveli}
\item 
\begin{stylelsSectioniii}
Capability
\end{stylelsSectioniii}
\end{listWWNumiiileveli}
\begin{styleStandard}
For capability, the Chabacano varieties use \textit{pwede}, as shown in (\ref{bkm:Ref52810347}). Hiligaynon can use either \textit{ma(ka)-} or \textit{pwede} (see (\ref{bkm:Ref52810347})c and (\ref{bkm:Ref53237707})). In contrast, Tagalog uses \textit{kaya}.
\end{styleStandard}


\setcounter{listWWNumiileveli}{0}
\begin{listWWNumiileveli}
\item 
\begin{stylelsLanginfo}
\label{bkm:Ref52810347}‘When I was a child, I \textbf{could} run (very) fast.’
\end{stylelsLanginfo}


\setcounter{listWWNumiilevelii}{0}
\begin{listWWNumiilevelii}
\item 
\begin{stylelsLanginfo}
Zam.
\end{stylelsLanginfo}
\end{listWWNumiilevelii}
\end{listWWNumiileveli}
\begin{stylelsSourceline}
Kwando\ \ bata yo, yo ta \textbf{pwede} korr\'{e} r\'{a}pido.
\end{stylelsSourceline}

\begin{stylelsIMT}
when child 1\textsc{sg} 1\textsc{sg} \textsc{ipfv} able run fast
\end{stylelsIMT}

\begin{listWWNumiileveli}
\item 
\begin{listWWNumiilevelii}
\item 
\begin{stylelsLanginfo}
Cav.
\end{stylelsLanginfo}
\end{listWWNumiilevelii}
\end{listWWNumiileveli}
\begin{stylelsSourceline}
Kwando yo chik\'{i}to, yo ta \textbf{pwede} kurrí muy r\'{a}pido
\end{stylelsSourceline}

\begin{stylelsIMT}
when child 1\textsc{sg} little 1\textsc{sg} \textsc{ipfv} able run very fast
\end{stylelsIMT}

\begin{listWWNumiileveli}
\item 
\begin{listWWNumiilevelii}
\item 
\begin{stylelsLanginfo}
Hil.
\end{stylelsLanginfo}
\end{listWWNumiilevelii}
\end{listWWNumiileveli}
\begin{stylelsIMT}
\textit{Sang bata ako, }\textbf{\textit{maka}}\textit{dalagan ako dasig.}
\end{stylelsIMT}

\begin{stylelsIMT}
when child 1\textsc{sg} \textsc{av}.able.run 1\textsc{sg} fast\ \ 
\end{stylelsIMT}

\begin{listWWNumiileveli}
\item 
\begin{listWWNumiilevelii}
\item 
\begin{stylelsLanginfo}
Tag.
\end{stylelsLanginfo}
\end{listWWNumiilevelii}
\end{listWWNumiileveli}
\begin{stylelsSourceline}
Noong bata ako, \textbf{kaya}/\#\textbf{puwede} kong tumakbo nang mabilis.
\end{stylelsSourceline}

\begin{styleStandard}
when.\textsc{lnk} child 1\textsc{sg} capability/able 1\textsc{sg.lnk av}.run\textsc{ lnk} fast
\end{styleStandard}

\begin{listWWNumiileveli}
\item 
\begin{stylelsLanginfo}
\label{bkm:Ref53237707}Hil.
\end{stylelsLanginfo}
\end{listWWNumiileveli}
\begin{stylelsLanginfo}
\textbf{\textit{Pwede}}\textit{ siya maglamoy isa ka milya.}
\end{stylelsLanginfo}

\begin{stylelsLanginfo}
able 3\textsc{sg.nom} \textsc{av}.swim one \textsc{num} mile
\end{stylelsLanginfo}

\begin{stylelsLanginfo}
\ \ {}'He can swim a mile.'
\end{stylelsLanginfo}

\begin{styleStandard}
Tagalog \textit{kaya} is not limited to contexts like (\ref{bkm:Ref52810347})\ that involve physical ability. It also covers other types of internal determination. For example, only \textit{kaya} works in translating the political slogan \textit{yes, we can} or \textit{sí, se puede}.
\end{styleStandard}

\begin{listWWNumiileveli}
\item 
\begin{stylelsLanginfo}
Tag.
\end{stylelsLanginfo}

\begin{listWWNumiilevelii}
\item 
\begin{stylelsLanginfo}
\textit{\#}\textbf{\textit{Pwede}}\textit{ tayo!}
\end{stylelsLanginfo}
\end{listWWNumiilevelii}
\end{listWWNumiileveli}
\begin{stylelsIMT}
can 1\textsc{pl.nom.incl}
\end{stylelsIMT}

\begin{stylelsTranslation}
‘(Yes,) we can!’
\end{stylelsTranslation}

\begin{listWWNumiileveli}
\item 
\begin{listWWNumiilevelii}
\item 
\begin{stylelsLanginfo}
\textbf{\textit{Kaya}}\textit{ natin!}
\end{stylelsLanginfo}
\end{listWWNumiilevelii}
\end{listWWNumiileveli}
\begin{stylelsIMT}
capability 1\textsc{pl.gen.incl}
\end{stylelsIMT}

\begin{stylelsTranslation}
‘(Yes,) we can!’
\end{stylelsTranslation}

\begin{styleStandard}
These examples show that \textit{kaya} is not preferred over \textit{puwede} in dynamic contexts overall (Asarina \& Holt 2005:14); rather, it encodes a distinction between internal and external dynamic possibility not present in Spanish, Chabacano, or even Hiligaynon.
\end{styleStandard}


\setcounter{listWWNumiiileveli}{0}
\begin{listWWNumiiileveli}
\item 
\begin{stylelsSectioniii}
Circumstantial possibility (ability)
\end{stylelsSectioniii}
\end{listWWNumiiileveli}
\begin{styleStandard}
For circumstantial possibility contexts related to ability, all four languages use \textit{pwede}. Tagalog and Hiligaynon also use \textit{ma(ka)-}. In (\ref{bkm:Ref52815116}), no one has granted the speaker permission to attend the party; they can go because circumstances allow it.
\end{styleStandard}


\setcounter{listWWNumiileveli}{0}
\begin{listWWNumiileveli}
\item 
\begin{stylelsLanginfo}
\label{bkm:Ref52815116}‘(I have money now, so now) I \textbf{can} go to the party.’
\end{stylelsLanginfo}


\setcounter{listWWNumiilevelii}{0}
\begin{listWWNumiilevelii}
\item 
\begin{stylelsLanginfo}
Zam.
\end{stylelsLanginfo}
\end{listWWNumiilevelii}
\end{listWWNumiileveli}
\begin{stylelsSourceline}
Tyene yo sen, \textbf{pwede} yo and\'{a} na party.
\end{stylelsSourceline}

\begin{stylelsIMT}
have \textsc{1sg }money able \textsc{1sg }go \textsc{loc }party
\end{stylelsIMT}

\begin{listWWNumiileveli}
\item 
\begin{listWWNumiilevelii}
\item 
\begin{stylelsLanginfo}
Cav.
\end{stylelsLanginfo}
\end{listWWNumiilevelii}
\end{listWWNumiileveli}
\begin{stylelsSourceline}
Tyene yo sen, \textbf{pwede }ya yo and\'{a} na party.
\end{stylelsSourceline}

\begin{stylelsIMT}
have \textsc{1sg }money able now \textsc{1sg }go \textsc{loc }party
\end{stylelsIMT}

\begin{listWWNumiileveli}
\item 
\begin{listWWNumiilevelii}
\item 
\begin{stylelsLanginfo}
Hil.
\end{stylelsLanginfo}
\end{listWWNumiilevelii}
\end{listWWNumiileveli}
\begin{stylelsIMT}
\textit{Pwede ako }\textbf{\textit{maka}}\textit{kadto sa party.}
\end{stylelsIMT}

\begin{stylelsIMT}
able \textsc{1sg.nom} \textsc{av}.able.go \textsc{loc} party\ \ 
\end{stylelsIMT}

\begin{listWWNumiileveli}
\item 
\begin{listWWNumiilevelii}
\item 
\begin{stylelsLanginfo}
Tag.
\end{stylelsLanginfo}
\end{listWWNumiilevelii}
\end{listWWNumiileveli}
\begin{stylelsSourceline}
\textbf{Puwede/maaari }akong pumunta sa party.
\end{stylelsSourceline}

\begin{styleStandard}
able/able \textsc{1sg.nom.lnk} \textsc{av}.go \textsc{loc} party
\end{styleStandard}

\begin{styleStandard}
These languages often encode circumstantial possibility in contexts where Spanish or English would not, as in (\ref{bkm:Ref52815371}), which could be replies to ‘have you ever been to Manila?’ Zamboanga Chabacano does so using \textit{pwede}, and Cavite Chabacano uses \textit{pudí} ({\textless} Sp. \textit{poder} ‘able.\textsc{inf}’), while the adstrates use the perfective form \textit{na(ka)-.} The creoles also use the perfective marker \textit{ya} in this case, even though \textit{pwede} often goes unmarked for aspect, much like the Tagalog pseudoverb \textit{puwede.} In fact, this aspect marking is likely why Cavite Chabacano tends to use \textit{pudí} in these types of examples, since that form functions like a typical full verb. Perfectivity is overtly marked here because, as with Tagalog \textit{ma(ka)}{}-, there is an actuality entailment in these types of examples (Dell 1983); perfective \textit{pwede/pudí} indicate that a result has been achieved.
\end{styleStandard}

\begin{listWWNumiileveli}
\item 
\begin{stylelsLanginfo}
\label{bkm:Ref52815371}‘I have been to Manila.’
\end{stylelsLanginfo}

\begin{listWWNumiilevelii}
\item 
\begin{stylelsLanginfo}
Zam.
\end{stylelsLanginfo}
\end{listWWNumiilevelii}
\end{listWWNumiileveli}
\begin{stylelsSourceline}
Ya \textbf{pwede} yo and\'{a} na Manila.
\end{stylelsSourceline}

\begin{stylelsIMT}
\textsc{pfv} able \textsc{1sg} go \textsc{loc} Manila
\end{stylelsIMT}

\begin{listWWNumiileveli}
\item 
\begin{listWWNumiilevelii}
\item 
\begin{stylelsLanginfo}
Cav.
\end{stylelsLanginfo}
\end{listWWNumiilevelii}
\end{listWWNumiileveli}
\begin{stylelsIMT}
\textit{Ya }\textbf{\textit{pudí}}\textit{ yo andá na Manila.} 
\end{stylelsIMT}

\begin{stylelsIMT}
\textsc{pfv} able \textsc{1sg} go \textsc{loc} Manila
\end{stylelsIMT}

\begin{listWWNumiileveli}
\item 
\begin{listWWNumiilevelii}
\item 
\begin{stylelsLanginfo}
Hil.
\end{stylelsLanginfo}
\end{listWWNumiilevelii}
\end{listWWNumiileveli}
\begin{stylelsIMT}
\textbf{\textit{Naka}}\textit{kadto ako sa Manila.}
\end{stylelsIMT}

\begin{stylelsIMT}
\textsc{pfv}.\textsc{av}.able.go \textsc{1sg} \textsc{loc} Manila
\end{stylelsIMT}

\begin{listWWNumiileveli}
\item 
\begin{listWWNumiilevelii}
\item 
\begin{stylelsLanginfo}
Tag.
\end{stylelsLanginfo}
\end{listWWNumiilevelii}
\end{listWWNumiileveli}
\begin{stylelsSourceline}
\textbf{Naka}punta ako sa Manila.
\end{stylelsSourceline}

\begin{stylelsIMT}
\textsc{pfv}.\textsc{av}.able.go \textsc{1sg} \textsc{loc} Manila
\end{stylelsIMT}

\begin{styleStandard}
The marking of circumstantial possibility in this type of context is related to the experiential uses of Philippine \textit{ma(ka)-}. Another typical Tagalog example, taken from a news report, is presented in (\ref{bkm:Ref52815746}).
\end{styleStandard}

\begin{listWWNumiileveli}
\item 
\begin{stylelsLanginfo}
\label{bkm:Ref52815746}Tag.
\end{stylelsLanginfo}
\end{listWWNumiileveli}
\begin{stylelsLanginfo}
\textbf{\textit{Na}}\textit{tikman mo na ba ang ice cream na salted egg flavor?}
\end{stylelsLanginfo}

\begin{stylelsLanginfo}
\textsc{pfv.ov}.able.taste 2\textsc{sg.gen} already \textsc{q nom} ice cream \textsc{lnk} salted egg flavor
\end{stylelsLanginfo}

\begin{stylelsLanginfo}
{}'Have you ever tasted salted egg flavored ice cream?' (i.e., have you ever experienced it/had the opportunity?) (Santos 2018)
\end{stylelsLanginfo}

\begin{styleStandard}
Similarly, (\ref{bkm:Ref52815732})\ shows a Caviteña writer using \textit{ya pudi} while recounting her travel experiences.
\end{styleStandard}

\begin{listWWNumiileveli}
\item 
\begin{stylelsLanginfo}
\label{bkm:Ref52815732}Cav.
\end{stylelsLanginfo}
\end{listWWNumiileveli}
\begin{stylelsLanginfo}
\textit{Otro dia, }\textbf{\textit{ya pudi}}\textit{ entra nisos na un cueva donde hay como cristal colgante.}
\end{stylelsLanginfo}

\begin{stylelsLanginfo}
other day \textsc{pfv} enter 1\textsc{pl loc det }cave where \textsc{exst }like crystal hanging
\end{stylelsLanginfo}

\begin{stylelsLanginfo}
{}'On another day, we were able to enter a cave where there were things like hanging crystals' (i.e., they had the chance to do so). (del Rosario 2007)
\end{stylelsLanginfo}


\setcounter{listWWNumiiileveli}{0}
\begin{listWWNumiiileveli}
\item 
\begin{stylelsSectioniii}
Circumstantial possibility (nonvolition)
\end{stylelsSectioniii}
\end{listWWNumiiileveli}
\begin{styleStandard}
As previously documented in Zamboanga and Ternate Chabacano (Rubino 2008, Sippola 2011), Cavite Chabacano follows the adstrates in marking nonvolitional events using an ability verb, again using \textit{pudí} rather than \textit{pwede}. Hiligaynon and Tagalog use the AIA marker \textit{ma(ka)-}, and although they both use \textit{pwede} for ability contexts, it is not used for nonvolition.
\end{styleStandard}

\begin{styleStandard}
Examples (\ref{bkm:Ref1510837})\ and (\ref{bkm:Ref52816675})\ show the same proposition, ‘I spilled paint on his shirt’, in two contexts. In (\ref{bkm:Ref1510837}), the spilling was accidental, while in (\ref{bkm:Ref52816675}), it was deliberate. 
\end{styleStandard}


\setcounter{listWWNumiileveli}{0}
\begin{listWWNumiileveli}
\item 
\begin{stylelsLanginfo}
\label{bkm:Ref1510837}[The speaker tripped while carrying a bucket of paint.] 
\end{stylelsLanginfo}
\end{listWWNumiileveli}
\begin{stylelsLanginfo}
‘I spilled paint on his shirt/clothes.’
\end{stylelsLanginfo}

\begin{listWWNumiileveli}
\item 
\begin{listWWNumiilevelii}
\item 
\begin{stylelsLanginfo}
Zam.
\end{stylelsLanginfo}
\end{listWWNumiilevelii}
\end{listWWNumiileveli}
\begin{stylelsSourceline}
Ya \textbf{pwede} yo derram\'{a} pintura na su kamiseta.
\end{stylelsSourceline}

\begin{stylelsIMT}
\textsc{pfv }able 1\textsc{sg }spill paint\textsc{ loc }3\textsc{sg.poss }shirt
\end{stylelsIMT}

\begin{listWWNumiileveli}
\item 
\begin{listWWNumiilevelii}
\item 
\begin{stylelsLanginfo}
Cav.
\end{stylelsLanginfo}
\end{listWWNumiilevelii}
\end{listWWNumiileveli}
\begin{stylelsIMT}
\textit{Ya }\textbf{\textit{pud\'{i}}}\textit{ yo but\'{a} pintura na su kamiseta.}
\end{stylelsIMT}

\begin{stylelsIMT}
\textsc{pfv} able 1\textsc{sg }spill paint\textsc{ loc }3\textsc{sg.poss }shirt
\end{stylelsIMT}

\begin{listWWNumiileveli}
\item 
\begin{listWWNumiilevelii}
\item 
\begin{stylelsLanginfo}
Hil.
\end{stylelsLanginfo}
\end{listWWNumiilevelii}
\end{listWWNumiileveli}
\begin{stylelsIMT}
\textbf{\textit{Na}}\textit{tuluan ko pinta ang iya bayo.}
\end{stylelsIMT}

\begin{stylelsIMT}
\textsc{pfv}.\textsc{lv}.able.spill \textsc{1sg.gen} paint \textsc{nom }3\textsc{sg.gen }shirt
\end{stylelsIMT}

\begin{listWWNumiileveli}
\item 
\begin{listWWNumiilevelii}
\item 
\begin{stylelsLanginfo}
Tag.
\end{stylelsLanginfo}
\end{listWWNumiilevelii}
\end{listWWNumiileveli}
\begin{stylelsSourceline}
\textbf{Na}tapunan ko ng pintura ang kanyang damit.
\end{stylelsSourceline}

\begin{stylelsIMT}
\textsc{pfv}.\textsc{lv}.able.spill \textsc{1sg.gen }paint\textsc{ nom }3\textsc{sg.gen.lnk} clothes
\end{stylelsIMT}

\begin{listWWNumiileveli}
\item 
\begin{stylelsLanginfo}
\label{bkm:Ref52816675}[The speaker saw someone they have a grudge against and decided to spill paint on them.]
\end{stylelsLanginfo}
\end{listWWNumiileveli}
\begin{stylelsLanginfo}
‘I spilled paint on his shirt/clothes.’
\end{stylelsLanginfo}

\begin{listWWNumiileveli}
\item 
\begin{listWWNumiilevelii}
\item 
\begin{stylelsLanginfo}
Zam.
\end{stylelsLanginfo}
\end{listWWNumiilevelii}
\end{listWWNumiileveli}
\begin{stylelsLanginfo}
\textit{Ya derram\'{a} yo el pintura na su kamiseta.}
\end{stylelsLanginfo}

\begin{stylelsIMT}
\textsc{pfv }spill 1\textsc{sg det }paint\textsc{ loc }3\textsc{sg.poss }shirt
\end{stylelsIMT}

\begin{listWWNumiileveli}
\item 
\begin{listWWNumiilevelii}
\item 
\begin{stylelsLanginfo}
Cav.
\end{stylelsLanginfo}
\end{listWWNumiilevelii}
\end{listWWNumiileveli}
\begin{stylelsIMT}
\textit{Ya but\'{a} yo pintura na su kamiseta.}
\end{stylelsIMT}

\begin{stylelsIMT}
\textsc{pfv} spill \textsc{1sg} paint \textsc{loc }3\textsc{sg.poss }shirt
\end{stylelsIMT}

\begin{listWWNumiileveli}
\item 
\begin{listWWNumiilevelii}
\item 
\begin{stylelsLanginfo}
Hil.
\end{stylelsLanginfo}
\end{listWWNumiilevelii}
\end{listWWNumiileveli}
\begin{stylelsIMT}
\textit{Gintuluan ko pinta ang iya bayo.}
\end{stylelsIMT}

\begin{stylelsIMT}
\textsc{pfv}.\textsc{lv}.spill \textsc{1sg.gen }paint\textsc{ nom }3\textsc{sg.gen} shirt
\end{stylelsIMT}

\begin{listWWNumiileveli}
\item 
\begin{listWWNumiilevelii}
\item 
\begin{stylelsLanginfo}
Tag.
\end{stylelsLanginfo}
\end{listWWNumiilevelii}
\end{listWWNumiileveli}
\begin{stylelsSourceline}
Tinapunan ko ng pintura ang kanyang damit.
\end{stylelsSourceline}

\begin{stylelsIMT}
\textsc{pfv}.\textsc{lv}.spill \textsc{1sg.gen }paint\textsc{ nom }3\textsc{sg.gen.lnk} clothes
\end{stylelsIMT}

\begin{styleStandard}
\textit{Pwede/pud\'{i}} and \textit{ma(ka)-} also mark other unplanned events, like the sudden occurrence in (\ref{bkm:Ref52817365}). 
\end{styleStandard}

\begin{listWWNumiileveli}
\item 
\begin{stylelsLanginfo}
\label{bkm:Ref52817365}[Do you know what happened to me yesterday when I was walking in the forest?]
\end{stylelsLanginfo}
\end{listWWNumiileveli}
\begin{stylelsLanginfo}
‘I suddenly stepped on a snake. It bit me on the leg.’
\end{stylelsLanginfo}

\begin{listWWNumiileveli}
\item 
\begin{listWWNumiilevelii}
\item 
\begin{stylelsLanginfo}
Zam.
\end{stylelsLanginfo}
\end{listWWNumiilevelii}
\end{listWWNumiileveli}
\begin{stylelsSourceline}
Ya \textbf{pwede} yo pisá na kulebra. Ya mordé le konmigo la pierna.
\end{stylelsSourceline}

\begin{stylelsIMT}
\textsc{pfv} able \textsc{1sg} step \textsc{loc} snake. \textsc{pfv} bite 3\textsc{sg} 1\textsc{sg.obj} \textsc{det} leg
\end{stylelsIMT}

\begin{listWWNumiileveli}
\item 
\begin{listWWNumiilevelii}
\item 
\begin{stylelsLanginfo}
Cav.
\end{stylelsLanginfo}
\end{listWWNumiilevelii}
\end{listWWNumiileveli}
\begin{stylelsSourceline}
Ya \textbf{pwede} yo pisá un kulebra. Ya mordé konmigo na mi pierna.
\end{stylelsSourceline}

\begin{stylelsIMT}
\textsc{pfv} able \textsc{1sg} step \textsc{det} snake. \textsc{pfv} bite 1\textsc{sg.obj loc} 1\textsc{sg}.\textsc{poss} leg
\end{stylelsIMT}

\begin{listWWNumiileveli}
\item 
\begin{listWWNumiilevelii}
\item 
\begin{stylelsLanginfo}
\ Hil.
\end{stylelsLanginfo}
\end{listWWNumiilevelii}
\end{listWWNumiileveli}
\begin{stylelsIMT}
\textbf{\textit{Naka}}\textit{tapak ako sang man-og. Kinagat niya ako sa siki.}
\end{stylelsIMT}

\begin{stylelsIMT}
\textsc{pfv.av.}able.step\textsc{ 1sg gen }snake\textsc{ pfv.ov.}bite\textsc{ }3\textsc{sg.gen }1\textsc{sg.nom loc }leg
\end{stylelsIMT}

\begin{listWWNumiileveli}
\item 
\begin{listWWNumiilevelii}
\item 
\begin{stylelsLanginfo}
Tag.
\end{stylelsLanginfo}
\end{listWWNumiilevelii}
\end{listWWNumiileveli}
\begin{stylelsSourceline}
Bigla akong \textbf{naka}tapak sa isang ahas. Kinagat niya ako sa paa.
\end{stylelsSourceline}

\begin{stylelsIMT}
suddenly \textsc{1sg.lnk pfv.av.}able.step \textsc{loc det.lnk} snake \textsc{pfv}.\textsc{ov.}bite 3\textsc{sg.gen }1\textsc{sg.nom loc} leg
\end{stylelsIMT}

\begin{styleStandard}
Note that the verbs for 'bite' in (\ref{bkm:Ref52817365})\ do not use the possibility markers, since only the initial event is surprising.
\end{styleStandard}

\begin{styleStandard}
An example of \textit{pwede/pud\'{i}} taking a coincidental reading is shown in (\ref{bkm:Ref52827899}). In this case, they are also mapped onto a stative verb, with adstrate \textit{ma-} marking perception/experience with the root for ‘see’.
\end{styleStandard}

\begin{listWWNumiileveli}
\item 
\begin{stylelsLanginfo}
\label{bkm:Ref52827899}[The speaker was shopping and unexpectedly ran into a friend.]
\end{stylelsLanginfo}
\end{listWWNumiileveli}
\begin{stylelsLanginfo}
‘I saw her at the market/store.’
\end{stylelsLanginfo}

\begin{listWWNumiileveli}
\item 
\begin{listWWNumiilevelii}
\item 
\begin{stylelsLanginfo}
Zam.
\end{stylelsLanginfo}
\end{listWWNumiilevelii}
\end{listWWNumiileveli}
\begin{stylelsSourceline}
Ya \textbf{pwede} yo mirá kon\ \  ele na \ tyangge.
\end{stylelsSourceline}

\begin{stylelsIMT}
\textsc{pfv} able \textsc{1sg }see \textsc{obj} 3\textsc{sg} \textsc{loc} market
\end{stylelsIMT}

\begin{listWWNumiileveli}
\item 
\begin{listWWNumiilevelii}
\item 
\begin{stylelsLanginfo}
Cav.
\end{stylelsLanginfo}
\end{listWWNumiilevelii}
\end{listWWNumiileveli}
\begin{stylelsIMT}
\textit{Ya }\textbf{\textit{pudí}}\textit{ yo mirá kon eli na plasa.}
\end{stylelsIMT}

\begin{stylelsIMT}
\textsc{pfv }able \textsc{1sg }see \textsc{obj} 3\textsc{sg} \textsc{loc} market
\end{stylelsIMT}

\begin{listWWNumiileveli}
\item 
\begin{listWWNumiilevelii}
\item 
\begin{stylelsLanginfo}
Hil.
\end{stylelsLanginfo}
\end{listWWNumiilevelii}
\end{listWWNumiileveli}
\begin{stylelsIMT}
\textbf{\textit{Na}}\textit{kita ko siya sa tienda.}
\end{stylelsIMT}

\begin{stylelsIMT}
\textsc{stv.pfv.}see\textsc{ }1\textsc{sg.gen }3\textsc{sg.nom loc }store
\end{stylelsIMT}

\begin{listWWNumiileveli}
\item 
\begin{listWWNumiilevelii}
\item 
\begin{stylelsLanginfo}
Tag.
\end{stylelsLanginfo}
\end{listWWNumiilevelii}
\end{listWWNumiileveli}
\begin{stylelsLanginfo}
\textbf{\textit{Na}}\textit{kita ko siya sa tienda.}
\end{stylelsLanginfo}

\begin{stylelsIMT}
\textsc{stv.pfv.}see\textsc{ }1\textsc{sg.gen }3\textsc{sg.nom loc }store
\end{stylelsIMT}

\begin{styleStandard}
In St’át’imcets, as previously discussed, the circumstantial marker has been analyzed as taking these types of nonvolitional readings in necessity contexts, whereas the ability readings arise from possibility contexts (Davis et al. 2009). This point is one where the Chabacano/Philippine markers differ; circumstantial necessity is expressed not with \textit{pwede}, \textit{pudí}, or \textit{ma(ka)- }but with Zam. \textit{ne(se)sita}, Cav. \textit{debi}, Hil. \textit{kinahanglan/kinanlan,} and Tag. \textit{kailangan}, as shown in (\ref{bkm:Ref52827925}). These same markers are used for deontic but not epistemic necessity.
\end{styleStandard}

\begin{listWWNumiileveli}
\item 
\begin{stylelsLanginfo}
\label{bkm:Ref52827925}‘We/everyone must eat in order to live.’
\end{stylelsLanginfo}

\begin{listWWNumiilevelii}
\item 
\begin{stylelsLanginfo}
Zam.
\end{stylelsLanginfo}
\end{listWWNumiilevelii}
\end{listWWNumiileveli}
\begin{stylelsSourceline}
\textbf{Nesita} kita komé para bibí.
\end{stylelsSourceline}

\begin{stylelsIMT}
must 1\textsc{pl.incl} eat for live
\end{stylelsIMT}

\begin{listWWNumiileveli}
\item 
\begin{listWWNumiilevelii}
\item 
\begin{stylelsLanginfo}
Cav.
\end{stylelsLanginfo}
\end{listWWNumiilevelii}
\end{listWWNumiileveli}
\begin{stylelsIMT}
\textit{Todo niso }\textbf{\textit{debi}}\textit{ komí para bibí.}
\end{stylelsIMT}

\begin{stylelsIMT}
all 1\textsc{pl} must eat for live
\end{stylelsIMT}

\begin{listWWNumiileveli}
\item 
\begin{listWWNumiilevelii}
\item 
\begin{stylelsLanginfo}
Hil.
\end{stylelsLanginfo}
\end{listWWNumiilevelii}
\end{listWWNumiileveli}
\begin{stylelsIMT}
\textbf{\textit{Kinanlan}}\textit{ magkaon kita para mabuhi.}
\end{stylelsIMT}

\begin{stylelsIMT}
must \textsc{av}.eat 1\textsc{incl} for \textsc{stv}.live
\end{stylelsIMT}

\begin{listWWNumiileveli}
\item 
\begin{listWWNumiilevelii}
\item 
\begin{stylelsLanginfo}
Tag.
\end{stylelsLanginfo}
\end{listWWNumiilevelii}
\end{listWWNumiileveli}
\begin{stylelsSourceline}
\textbf{Kailangan} nating kumain para mabuhay.
\end{stylelsSourceline}

\begin{stylelsIMT}
must 1\textsc{pl.incl.lnk} \textsc{av}.eat for \textsc{stv}.live
\end{stylelsIMT}


\setcounter{listWWNumiiileveli}{0}
\begin{listWWNumiiileveli}
\item 

\setcounter{listWWNumiiilevelii}{0}
\begin{listWWNumiiilevelii}
\item 
\begin{stylelsSectionii}
Epistemic possibility
\end{stylelsSectionii}
\end{listWWNumiiilevelii}
\end{listWWNumiiileveli}
\begin{styleStandard}
All four languages use \textit{pwede} for epistemic possibility, and Tagalog also uses \textit{maaari}. As previously shown for Tagalog, however, each language marks epistemic modality mainly through adverbs. Spanish \textit{posible} ‘possible’ is found in all four languages, along with the slightly stronger marker \textit{siguro} ‘maybe/possibly/probably’ ({\textless} Sp. \textit{seguro }‘sure’).\footnote{ The Chabacano varieties also have \textit{siguraw }and the adstrates have \textit{sigurado} ‘surely’ ({\textless} Sp. \textit{asegurado} ‘assured’), but these markers denote complete certainty rather than mere probability.} In addition, Tagalog has \textit{baka} ‘possibly’ (also used in Cavite Chabacano), and Hiligaynon has \textit{basi }‘possibly’. The Zamboanga Chabacano consultants use all of the markers above as well as \textit{gaha} ‘possibly’ ({\textless} Cebuano \textit{kaha}). Examples of some of these markers are shown in (\ref{bkm:Ref52829259}).
\end{styleStandard}


\setcounter{listWWNumiileveli}{0}
\begin{listWWNumiileveli}
\item 
\begin{stylelsLanginfo}
\label{bkm:Ref52829259}‘It \textbf{might} rain (later) tonight.’
\end{stylelsLanginfo}


\setcounter{listWWNumiilevelii}{0}
\begin{listWWNumiilevelii}
\item 
\begin{stylelsLanginfo}
Zam.
\end{stylelsLanginfo}
\end{listWWNumiilevelii}
\end{listWWNumiileveli}
\begin{stylelsSourceline}
\textbf{Posible/basi/baka} kay ulan esta noche\textup{.}
\end{stylelsSourceline}

\begin{stylelsIMT}
possibly/possibly/possibly \textsc{comp} rain \textsc{det} night
\end{stylelsIMT}

\begin{listWWNumiileveli}
\item 
\begin{listWWNumiilevelii}
\item 
\begin{stylelsLanginfo}
Zam.
\end{stylelsLanginfo}
\end{listWWNumiilevelii}
\end{listWWNumiileveli}
\begin{stylelsSourceline}
Man-ulan \textbf{gaha} ara de noche.
\end{stylelsSourceline}

\begin{stylelsIMT}
\textsc{verb}{}-rain possibly now of night
\end{stylelsIMT}

\begin{listWWNumiileveli}
\item 
\begin{listWWNumiilevelii}
\item 
\begin{stylelsLanginfo}
Cav.
\end{stylelsLanginfo}
\end{listWWNumiilevelii}
\end{listWWNumiileveli}
\begin{stylelsIMT}
\textbf{\textit{Baka}}\textit{ di llubí lwego di noche.}
\end{stylelsIMT}

\begin{stylelsIMT}
possibly \textsc{fut} rain later of night
\end{stylelsIMT}

\begin{listWWNumiileveli}
\item 
\begin{listWWNumiilevelii}
\item 
\begin{stylelsLanginfo}
Hil.
\end{stylelsLanginfo}
\end{listWWNumiilevelii}
\end{listWWNumiileveli}
\begin{stylelsIMT}
\textbf{\textit{Pwede/basi/posible}}\textit{ mag-ulan subong nga gab-i.}
\end{stylelsIMT}

\begin{stylelsIMT}
able/possibly/possibly\ \  \textsc{av}.rain now \textsc{lnk} night
\end{stylelsIMT}

\begin{listWWNumiileveli}
\item 
\begin{listWWNumiilevelii}
\item 
\begin{stylelsLanginfo}
Tag.
\end{stylelsLanginfo}
\end{listWWNumiilevelii}
\end{listWWNumiileveli}
\begin{stylelsSourceline}
\textbf{Baka/posibleng/\#puwedeng }umulan mamayang gabi.
\end{stylelsSourceline}

\begin{styleStandard}
possibly/possibly.\textsc{lnk}/able.\textsc{lnk} \textsc{av}.rain later.\textsc{lnk} night
\end{styleStandard}

\begin{styleStandard}
While \textit{pwede} was also acceptable for Hiligaynon in (\ref{bkm:Ref52829259}), the Tagalog consultant rejected it for this context. However, (\ref{bkm:Ref52830591})\ shows epistemic examples of \textit{puwede} in Tagalog and Chabacano.
\end{styleStandard}

\begin{listWWNumiileveli}
\item 
\begin{stylelsLanginfo}
\label{bkm:Ref52830591}[There was some money on the table, but now it is missing. John is a known thief.] 
\end{stylelsLanginfo}
\end{listWWNumiileveli}
\begin{stylelsLanginfo}
‘It \textbf{could} be John who took the money.’
\end{stylelsLanginfo}

\begin{listWWNumiileveli}
\item 
\begin{listWWNumiilevelii}
\item 
\begin{stylelsLanginfo}
Zam.
\end{stylelsLanginfo}
\end{listWWNumiilevelii}
\end{listWWNumiileveli}
\begin{stylelsSourceline}
\textbf{Pwede} le saká kon el sen.
\end{stylelsSourceline}

\begin{stylelsIMT}
able 3\textsc{sg} take \textsc{obj} \textsc{det} money\ \ 
\end{stylelsIMT}

\begin{listWWNumiileveli}
\item 
\begin{listWWNumiilevelii}
\item 
\begin{stylelsLanginfo}
Cav.
\end{stylelsLanginfo}
\end{listWWNumiilevelii}
\end{listWWNumiileveli}
\begin{stylelsSourceline}
\textbf{Baka} John ya saká el sen.
\end{stylelsSourceline}

\begin{stylelsIMT}
possibly\ \ John \textsc{pfv} take \textsc{det} money
\end{stylelsIMT}

\begin{listWWNumiileveli}
\item 
\begin{listWWNumiilevelii}
\item 
\begin{stylelsLanginfo}
Cav.
\end{stylelsLanginfo}
\end{listWWNumiilevelii}
\end{listWWNumiileveli}
\begin{stylelsIMT}
\textbf{\textit{Pwede}}\textit{ John saká el sen.}
\end{stylelsIMT}

\begin{stylelsIMT}
able John take \textsc{det} money
\end{stylelsIMT}

\begin{listWWNumiileveli}
\item 
\begin{listWWNumiilevelii}
\item 
\begin{stylelsLanginfo}
Hil.
\end{stylelsLanginfo}
\end{listWWNumiilevelii}
\end{listWWNumiileveli}
\begin{stylelsIMT}
\textbf{\textit{Siguro}}\textit{ si John ang nagkuha sang kwarta.}
\end{stylelsIMT}

\begin{stylelsIMT}
probably \textsc{nom} John \textsc{nom} \textsc{pfv.av}.take \textsc{gen} money
\end{stylelsIMT}

\begin{listWWNumiileveli}
\item 
\begin{listWWNumiilevelii}
\item 
\begin{stylelsLanginfo}
Tag.
\end{stylelsLanginfo}
\end{listWWNumiilevelii}
\end{listWWNumiileveli}
\begin{stylelsSourceline}
\textbf{Puwedeng/maaaring} si John ang kumuha ng pera.
\end{stylelsSourceline}

\begin{styleStandard}
able.\textsc{lnk}/able.\textsc{lnk} \textsc{nom} John \textsc{nom} \textsc{pfv.av}.take \textsc{gen} money
\end{styleStandard}

\begin{styleStandard}
In this case, the Hiligaynon consultant used \textit{siguro}, since the background about John’s past suggests that he would be likely to steal. Yet the other consultants still used weaker markers.
\end{styleStandard}

\begin{styleStandard}
In addition to producing \textit{puwede }and accepting \textit{maaari} in (\ref{bkm:Ref52830591}), the Tagalog consultant also offered (\ref{bkm:Ref52832655}).
\end{styleStandard}

\begin{listWWNumiileveli}
\item 
\begin{stylelsLanginfo}
\label{bkm:Ref52832655}Tag.
\end{stylelsLanginfo}
\end{listWWNumiileveli}
\begin{stylelsLanginfo}
\textbf{\textit{Kaya}}\textit{ ni John kunin yung pera.} 
\end{stylelsLanginfo}

\begin{stylelsLanginfo}
capability \textsc{gen} John \textsc{ov.}take \textsc{det} money
\end{stylelsLanginfo}

\begin{stylelsLanginfo}
{}'John could have taken the money.'
\end{stylelsLanginfo}

\begin{styleStandard}
\textit{Kaya} implicates that not only is it possible that John took the money but that, as the consultant explained, ‘he has it in him’ to do so. This example of \textit{kaya} in an epistemic context was the only one elicited in this study.
\end{styleStandard}

\begin{styleStandard}
The epistemic \textit{pwede} in Tagalog and the Chabacano varieties in (\ref{bkm:Ref52830591})\ contrasts with its dynamic/circumstantial usage in (\ref{bkm:Ref52832676}).
\end{styleStandard}

\begin{listWWNumiileveli}
\item 
\begin{stylelsLanginfo}
\label{bkm:Ref52832676}[John was seen near where the money was, but he’s an honest person.]
\end{stylelsLanginfo}
\end{listWWNumiileveli}
\begin{stylelsLanginfo}
‘John could have taken the money, but he didn’t (take/do it).’
\end{stylelsLanginfo}

\begin{listWWNumiileveli}
\item 
\begin{listWWNumiilevelii}
\item 
\begin{stylelsLanginfo}
Zam.
\end{stylelsLanginfo}
\end{listWWNumiilevelii}
\end{listWWNumiileveli}
\begin{stylelsIMT}
\textit{Ya}\textbf{\textit{ pwede}}\textit{ era saká si John el sen, pero nuay le saká.}
\end{stylelsIMT}

\begin{stylelsIMT}
\textsc{pfv} able \textsc{cf} take \textsc{nom} John \textsc{det} money but \textsc{neg}.\textsc{exst} 3\textsc{sg} take
\end{stylelsIMT}

\begin{listWWNumiileveli}
\item 
\begin{listWWNumiilevelii}
\item 
\begin{stylelsLanginfo}
Cav.
\end{stylelsLanginfo}
\end{listWWNumiilevelii}
\end{listWWNumiileveli}
\begin{stylelsIMT}
\textbf{\textit{Pwede}}\textit{ saká John el sen, pero no eli ya así.}
\end{stylelsIMT}

\begin{stylelsIMT}
able take John \textsc{det} money but \textsc{neg} 3\textsc{sg pfv} do
\end{stylelsIMT}

\begin{listWWNumiileveli}
\item 
\begin{listWWNumiilevelii}
\item 
\begin{stylelsLanginfo}
Hil.
\end{stylelsLanginfo}
\end{listWWNumiilevelii}
\end{listWWNumiileveli}
\begin{stylelsIMT}
\textbf{\textit{Pwede}}\textbf{/}\textbf{\textit{posible}}\textit{ nga nakuha ni John ang kwarta, pero indi.}
\end{stylelsIMT}

\begin{stylelsIMT}
able/possible \textsc{lnk pfv.ov}.able.take \textsc{gen} John \textsc{nom} money but \textsc{neg}
\end{stylelsIMT}

\begin{listWWNumiileveli}
\item 
\begin{listWWNumiilevelii}
\item 
\begin{stylelsLanginfo}
Tag.
\end{stylelsLanginfo}
\end{listWWNumiilevelii}
\end{listWWNumiileveli}
\begin{stylelsSourceline}
\textbf{\textup{Puwede}}\textbf{/}\textbf{\textup{\#kaya}}\textup{ niyang kunin, pero hindi niya kinuha.}
\end{stylelsSourceline}

\begin{stylelsSourceline}
\textup{able/capability 3}\textsc{sg.gen.lnk ov.}\textup{take but }\textsc{neg}\textup{ 3}\textsc{sg.gen pfv.ov}\textup{.take}
\end{stylelsSourceline}

\begin{styleStandard}
In this case, each language uses \textit{pwede}. This context does not involve a judgment of likelihood, as in (\ref{bkm:Ref52832655}), but rather is concerned with whether the opportunity was even available.
\end{styleStandard}

\begin{styleStandard}
It was often difficult to elicit distinctions among the epistemic markers in regard to possibility vs. necessity/probability. For example, Zamboanga Chabacano consultants used \textit{baka} in contexts intended to elicit weaker and stronger possibility in Error: Reference source not found\ and (\ref{bkm:Ref52833318}), and \textit{siguro} was used for possibility in Error: Reference source not found.
\end{styleStandard}

\begin{listWWNumiileveli}
\item 
\begin{stylelsLanginfo}
Zam.
\end{stylelsLanginfo}

\begin{listWWNumiilevelii}
\item 
\begin{stylelsLanginfo}
\textit{Na kasa ya }\textbf{\textit{gaha}}\textit{ si John.}
\end{stylelsLanginfo}
\end{listWWNumiilevelii}
\end{listWWNumiileveli}
\begin{stylelsIMT}
\textsc{loc} house already possibly \textsc{nom }John
\end{stylelsIMT}

\begin{stylelsTranslation}
‘John \textbf{might} be at home already.’
\end{stylelsTranslation}

\begin{listWWNumiileveli}
\item 
\begin{listWWNumiilevelii}
\item 
\begin{stylelsLanginfo}
\textbf{\textit{Baka/siguro}}\textbf{ }\textit{talla na kasa ya si John.}
\end{stylelsLanginfo}
\end{listWWNumiilevelii}
\end{listWWNumiileveli}
\begin{stylelsIMT}
possibly/possibly there \textsc{loc} house already \textsc{nom }John
\end{stylelsIMT}

\begin{stylelsTranslation}
‘John \textbf{might} be at home already.’
\end{stylelsTranslation}

\begin{listWWNumiileveli}
\item 
\begin{stylelsLanginfo}
\label{bkm:Ref52833318}Zam.
\end{stylelsLanginfo}
\end{listWWNumiileveli}
\begin{stylelsLanginfo}
\textbf{\textit{Baka}}\textit{/}\textbf{\textit{posible}}\textit{ talla ya si John na kasa.}
\end{stylelsLanginfo}

\begin{stylelsLanginfo}
possibly/possibly there already \textsc{nom} John \textsc{loc} house
\end{stylelsLanginfo}

\begin{stylelsLanginfo}
{}'John must be at home already.'
\end{stylelsLanginfo}

\begin{styleStandard}
Admittedly, there was probably not enough context provided to the consultants here, but more detailed follow-up with another Zamboanga Chabacano speaker indicated that he would have similar responses (e.g., in Error: Reference source not found, \textit{baka} or \textit{gaha }could be used if there is a possibility John might be somewhere else, like the bank or the store). This speaker found \textit{siguro} acceptable for the sentence in (\ref{bkm:Ref52833318})\ but preferred to use it in response to a more specific context (e.g., to answer a question like ‘How did Maria enter the house without a key?’).\footnote{ Thanks to Jerome Herrera for providing this example.}
\end{styleStandard}

\begin{styleStandard}
This difficulty in distinguishing epistemic meanings is partly an artifact of the limited pragmatic contexts of the questionnaire. The considerable overlap among these markers has already been documented in other grammatical descriptions and dictionaries, however, so these markers do in fact seem to be less clearly distinct from each other than English \textit{can} vs. \textit{must}, or \textit{possibly/maybe }vs.\textit{ probably}. A similarly blurry distinction between epistemic possibility and necessity has been found in another Austronesian language, Paciran Javanese, which has a possibility marker, \textit{paleng}, that can flexibly take on necessity force in certain contexts (Vander Klok 2013).
\end{styleStandard}

\begin{styleStandard}
With that said, it is safe to say that \textit{siguro} is slightly stronger than the other epistemic markers, and it is possible to elicit consistent differences when there is enough context. As the Hiligaynon example in (\ref{bkm:Ref53166482})\ shows, \textit{siguro} (here reduced to \textit{guro})\textit{ }is preferred when there is more evidence or there is no reason to think there could be another possibility.
\end{styleStandard}

\begin{listWWNumiileveli}
\item 
\begin{stylelsLanginfo}
\label{bkm:Ref53166482}Hil.
\end{stylelsLanginfo}
\end{listWWNumiileveli}
\begin{styleStandard}
\textit{\ \ }[There are no lights on in John’s house.]
\end{styleStandard}

\begin{stylelsLanginfo}
\textit{Nagtulog na }\textbf{\textit{(si)guro}}\textit{ si John.}
\end{stylelsLanginfo}

\begin{stylelsLanginfo}
\textsc{pfv.av.}sleep already probably \textsc{nom} John
\end{stylelsLanginfo}

\begin{stylelsLanginfo}
{}'John must have gone to sleep already.'
\end{stylelsLanginfo}

\begin{styleStandard}
According to the consultant, \textit{basi} in this sentence would mean the speaker is just guessing, or maybe John could be doing something else, like attending a party. 
\end{styleStandard}

\begin{styleStandard}
There is still the question of whether \textit{siguro} has the default interpretation of necessity or possibility. As Vander Klok (2013:345) demonstrated, one test is to see whether the epistemic marker can be used in mutually exclusive propositions; for example, a possibility marker makes sense in a context like ‘maybe she's taking a nap, maybe she's not taking a nap’, but a marker meaning ‘must’ or ‘certainly’ cannot appear in both clauses. No such examples were collected in the questionnaire, but a search online confirms that Tagalog \textit{siguro }does not get canceled in such constructions, as (\ref{bkm:Ref52833713})\ shows. 
\end{styleStandard}

\begin{listWWNumiileveli}
\item 
\begin{stylelsLanginfo}
\label{bkm:Ref52833713}Tag.
\end{stylelsLanginfo}
\end{listWWNumiileveli}
\begin{stylelsLanginfo}
\textit{Mararamdaman ba natin kapag may parating na pagbabago? Siguro. Siguro hindi. Siguro minsan.}
\end{stylelsLanginfo}

\begin{styleTextbodyindent}
\textsc{fut.ov}.able.feel \textsc{q} 1.\textsc{incl.gen} when \textsc{exst} upcoming \textsc{lnk} change \ \ maybe maybe \textsc{neg} maybe sometimes
\end{styleTextbodyindent}

\begin{styleTextbodyindent}
‘Can we feel when a change is coming? Maybe. Maybe not. Maybe sometimes.’ (Yatchi 2012)
\end{styleTextbodyindent}

\begin{styleStandard}
This example suggests that, at least in Tagalog, \textit{siguro} is a flexible possibility marker like Paciran Javanese \textit{paleng} rather than a pure necessity marker.
\end{styleStandard}


\setcounter{listWWNumiiileveli}{0}
\begin{listWWNumiiileveli}
\item 
\begin{stylelsSectioni}
Discussion
\end{stylelsSectioni}


\setcounter{listWWNumiiilevelii}{0}
\begin{listWWNumiiilevelii}
\item 
\begin{stylelsSectionii}
Comparison and typological classification
\end{stylelsSectionii}
\end{listWWNumiiilevelii}
\end{listWWNumiiileveli}
\begin{styleStandard}
Tables 2 and 3 summarize the modal systems of Zamboanga Chabacano, Cavite Chabacano, Hiligaynon, and Tagalog. In addition, Table 2 includes Ternate Chabacano, although questionnaire data were not obtained for that variety. The categorization of the Ternate Chabacano modals is based on descriptions and examples from Sippola (2011:156–166, 208–210). 
\end{styleStandard}

\begin{styleStandard}
The tables show that all five languages have similar modal systems. Spanish influence is evident in each language through the use of \textit{pwede} and \textit{siguro}. In addition, \textit{posible} is listed for each language except Ternate Chabacano (it did not occur in Sippola's 2011 dataset, though it is listed as a lexical item for all Chabacano varieties in Riego de Dios' 1989 comparative dictionary).
\end{styleStandard}

\begin{flushleft}
\tablefirsthead{}
\tablehead{}
\tabletail{}
\tablelasttail{}
\begin{supertabular}{|m{0.9705598in}|m{1.1531599in}|m{1.1531599in}|m{1.1538599in}|}
\hline
\textbf{Modal Type} &
\textbf{Zam.} &
\textbf{Cav.} &
\textbf{Ter.}\\\hline
\multicolumn{4}{|m{4.66696in}|}{\textbf{Deontic }}\\\hline
Necessity &
\textit{ne(se)sita }\ {}'must' ({\textless} Sp. \textit{necesita} ‘need’)

\textit{debe} 'should' &
\textit{debi }{}'must, should' ({\textless} Sp. \textit{debe} ‘owe, must’) &
\textit{dabli} 'must' ({\textless} Sp. \textit{dable} ‘feasible’)\\\hline
Possibility &
\textit{pwede} ({\textless} Sp. \textit{puede} 'can.3\textsc{sg}{}') &
\textit{pwede}  &
\textit{pwede} and reduced forms (\textit{pwe}, \textit{pe}, \textit{pey})

\textit{mari} ({\textless} Tag. \textit{maaari }{}'can')\\\hline
\multicolumn{4}{|m{4.66696in}|}{\textbf{Dynamic: Participant-internal}}\\\hline
Capability &
\textit{pwede} &
\textit{pwede} &
\textit{pwede} and reduced forms

\textit{mari}\\\hline
Learned ability &
\textit{sabe }({\textless} Sp. \textit{sabe} 'know.3\textsc{sg}{}')

\textit{pwede} &
\textit{sabe} &
\textit{sabe}

\textit{sabé} ({\textless} Sp. \textit{saber} 'know.\textsc{inf}{}')\\\hline
\multicolumn{4}{|m{4.66696in}|}{\textbf{Dynamic: Participant-external}}\\\hline
Necessity &
\textit{ne(se)sita} &
\textit{debi} &
\textit{dabli} \\\hline
Possibility (ability) &
\textit{pwede} &
\textit{pwede}

\textit{pudí} ({\textless} Sp. \textit{poder} 'can.\textsc{inf}{}') &
\textit{pwede} and reduced forms \textit{mari} \ \\\hline
Possibility (nonvolition) &
\textit{pwede} &
\textit{pwede}

\textit{pudí}  &
\textit{pwede} 

\textit{mari} \\\hline
\multicolumn{4}{|m{4.66696in}|}{\textbf{Epistemic}}\\\hline
Necessity (weak) &
\textit{siguro} ({\textless} Sp. \textit{seguro }‘sure’) &
\textit{siguro} &
\textit{siguru}\\\hline
Possibility &
\textit{siguro}

\textit{baka }({\textless} Tag.)

\textit{basi }({\textless} Hil.)

\textit{gaha }({\textless} Ceb. \textit{kaha})

\textit{posible} ({\textless} Sp. \textit{posible})

\textit{pwede} &
\textit{siguro}

\textit{baka}

\textit{posible}

\textit{pwede} &
\textit{siguru}

\textit{baka}

\textit{pwede} and reduced forms\\\hline
\end{supertabular}
\end{flushleft}
\begin{styleStandard}
\textit{Table 2: Zamboanga, Cavite, and Ternate Chabacano modality.}
\end{styleStandard}

\begin{flushleft}
\tablefirsthead{}
\tablehead{}
\tabletail{}
\tablelasttail{}
\begin{supertabular}{|m{0.97955984in}|m{1.7643598in}|m{1.7663599in}|}
\hline
\textbf{Modal Type} &
\textbf{Hil.} &
\textbf{Tag.}\\\hline
\multicolumn{3}{|m{4.66776in}|}{\textbf{Deontic}}\\\hline
Necessity &
\textit{kinahanglan/kinanlan} ‘must’

\textit{dapat} ‘should’ &
\textit{kailangan} ‘must’

\textit{dapat} ‘should’\\\hline
Possibility &
\textit{pwede}

\textit{ma(ka)-} &
\textit{puwede}

\textit{maaari}

\textit{ma(ka)-}\\\hline
\multicolumn{3}{|m{4.66776in}|}{\textbf{Dynamic: Participant-internal}}\\\hline
Capability &
\textit{pwede} &
\textit{kaya} 'capability, power'\\\hline
Learned ability &
\textit{kahibalo} 'know' &
\textit{marunong} 'knowledgeable'\\\hline
\multicolumn{3}{|m{4.66776in}|}{\textbf{Dynamic: Participant-external}}\\\hline
Necessity &
\textit{kinahanglan/kinanlan} &
\textit{kailangan}\\\hline
Possibility (ability) &
\textit{pwede}

\textit{ma(ka)-} &
\textit{pwede}

\textit{maaari}

\textit{ma(ka)-}\\\hline
Possibility (nonvolition) &
\textit{ma(ka)-} &
\textit{ma(ka)-}\\\hline
\multicolumn{3}{|m{4.66776in}|}{\textbf{Epistemic}}\\\hline
Necessity (weak) &
\textit{siguro}  &
\textit{siguro} \\\hline
Possibility &
\textit{siguro}

\textit{basi} 

\textit{posible}

\textit{pwede} &
\textit{siguro}

\textit{baka} 

\textit{posible}

\textit{pwede}

\textit{maaari}\\\hline
\end{supertabular}
\end{flushleft}
\begin{styleStandard}
\textit{Table 3: Hiligaynon and Tagalog modality.}
\end{styleStandard}

\begin{styleStandard}
Within the typological framework used by Matthewson et al. (2005) and Vander Klok (2013), each language can be described as having a mixed modal system. \textit{Pwede} has fixed possibility force in all five languages, and the base is left to context, as in many European languages; except for Tagalog \textit{maaari}, it is the only verb in any of the languages that can cover epistemic contexts. The various verbs that express necessity in each language also have fixed modal force and can flexibly express deontic or dynamic (but not epistemic) modality. Finally, most of the epistemic markers cannot be used for other types of modality and have a base fixed to possibility contexts, but \textit{siguro} appears to have somewhat flexible force, and is therefore listed in the tables for now under both possibility and weak necessity. 
\end{styleStandard}

\begin{styleStandard}
The facts about Chabacano align with Winford's (2018) observation that it is common for creoles to have at least two modal auxiliaries: one corresponding to \textit{must }(necessity) and one to \textit{can }(possibility), with both categories covering a range of deontic, dynamic, and epistemic contexts. This pattern is not surprising, given how common this type of polyfunctionality is in the European lexifiers (van der Auwera et al. 2005, Matthewson et al. 2005). The fact that \textit{pwede} works this way in all Chabacano varieties is a somewhat new finding, since the current descriptions of Zamboanga and Ternate Chabacano in APiCS state that the ability verb is not used to express epistemic possibility (feature 55, Maurer et al. 2013; but see also Sippola 2011, which shows that \textit{pwede} can be epistemic in Ternate). The extension of the ability verb to epistemic contexts is also found in Palenquero and Papiamentu (Maurer et al. 2013), so this feature appears to be common to all Spanish-lexified creoles. 
\end{styleStandard}

\begin{styleStandard}
The expression of possibility in the Chabacano varieties is significantly influenced by the adstrates in at least two ways. First, Chabacano epistemic modality is expressed primarily through adverbs, which are of both Philippine and Spanish origin. Tagalog \textit{baka }is used in every Chabacano variety, although in Zamboanga, this word is likely a fairly recent borrowing,\footnote{As Lipski (2013:461) observed, it is only recent decades that Tagalog/Filipino has become influential enough in Mindanao for people to start borrowing grammatical items from it.} and the Visayan equivalents \textit{basi }({\textless} Hil.) and \textit{gaha }({\textless} Ceb.) are still used. \textit{Siguro} was semantically weakened as it was grammaticalized from Spanish \textit{seguro }{}'surely' in Chabacano as well as the adstrates. It consistently marks a higher degree of necessity or probability than the markers of Philippine origin, under the right contexts. Yet it also seems to mark possibility or general uncertainty, with the native markers left to indicate a more remote possibility. 
\end{styleStandard}

\begin{styleStandard}
Second, each Chabacano variety uses \textit{pwede} or a related form to mark not only ability but also nonvolitional circumstances, mapping onto the functions of adstrate \textit{ma(ka)-}. While the nonvolitional function of \textit{pwede} was already documented in Zamboanga and Ternate Chabacano (Rubino 2008, Sippola 2011), this data shows that the same is true in Cavite, but using \textit{pudí}. Furthermore, this study demonstrates just how closely the Chabacano varieties track with the adstrates in overtly marking \textit{pwede/pudí} for perfective aspect in these contexts, in order to mark that a result has been achieved (whether intentionally or not). These circumstantial uses of \textit{pwede }or \textit{pudí} are also related to the experience-marking functions of \textit{ma(ka)-} and some uses of stative \textit{ma-} (e.g., for verbs like ‘see’ and ‘die’). 
\end{styleStandard}

\begin{styleStandard}
From a broader crosslinguistic perspective, another finding is that while Chabacano \textit{pwede/pudí} and Philippine \textit{ma(ka)-} have remarkably similar functions to the circumstantial marker in St’át’imcets, they differ in some crucial ways. They are not fixed to a dynamic/circumstantial base, and they do not have flexible force (the opposite of how the St’át’imcets marker is specified). Unlike in St’át’imcets, the nonvolitional uses of these markers fall under the realm of possibility and not necessity.
\end{styleStandard}

\begin{styleStandard}
The data also clearly demonstrate that the Hiligaynon and Tagalog systems are nearly identical to each other. They share some Philippine and Spanish forms that are exactly the same (\textit{dapat }{}'should', \textit{maka-} 'able', \textit{pwede} 'able', etc.) and others that are also close Philippine cognates (Hil. \textit{kinahanglan/kinanlan} 'must'\textit{, }Tag. \textit{kailangan} 'must'). \textit{Pwede }does not cover nonvolitional circumstances in either language. It overlaps with \textit{ma(ka)-} to some extent, but the prefix is not used in epistemic contexts; rather, it is restricted to participant-external event modality (i.e., deontic and circumstantial possibility, including nonvolitional events).\textit{ }The only major differences between the adstrates are that Tagalog marks more distinctions between subtypes of dynamic possibility (using \textit{kaya} for capability instead of \textit{pwede}), and it still sometimes uses \textit{maaari} as a more formal native equivalent to \textit{pwede}.
\end{styleStandard}

\begin{listWWNumiiileveli}
\item 
\begin{listWWNumiiilevelii}
\item 
\begin{stylelsSectionii}
Development of the Chabacano modal systems
\end{stylelsSectionii}
\end{listWWNumiiilevelii}
\end{listWWNumiiileveli}
\begin{styleStandard}
Given the similarity of the creole modal systems, it might be tempting to assume that came from a common ancestor. Apart from the modals, the examples in §4 are also remarkably similar overall in their syntax and lexicon. I argue, however, that these systems developed separately but congruently because there are few substantial differences in the modal systems and overall structure of the substrates. 
\end{styleStandard}

\begin{styleStandard}
The slight lexical differences in the modal systems should not be overlooked, as Sippola \& Lesho (2020) have already argued. For example, while it would not have been out of the realm of possibility for one Chabacano variety to have selected either Spanish \textit{necesita} or \textit{debe} as the main necessity verb during the process of creole formation, and then another variety to have later switched to the other through internal change, this does not seem to be what happened. The supposed parent variety in Ternate did not grammaticalize either of these forms but rather uses \textit{dabli} (Sippola \& Lesho 2020:112). The more plausible explanation is that each variety selected different forms because they were separated geographically and socially, and formed during slightly different time periods (as historical evidence also suggests). 
\end{styleStandard}

\begin{styleStandard}
The possibility modals also have slight differences in form that are suggestive of development along separate trajectories. \textit{Pwede} is found in all three Chabacano varieties, but in Ternate, it is often reduced to just the first syllable, possibly indicating a higher degree of grammaticalization (Sippola \& Lesho 2020). This reduction occasionally happens in Cavite but appears to be much less frequent, and it has not yet been documented for the Zamboanga variety. It is also notable that Cavite Chabacano is the only current variety that uses \textit{pudí} in addition to \textit{pwede}, and Ternate Chabacano is the only variety that uses \textit{mari}.
\end{styleStandard}

\begin{styleStandard}
Ideally, there would be more diachronic data to base this discussion on, but old Chabacano samples are scarce. There are a handful of illuminating examples in the oldest available texts written in the Manila Bay and Mindanao varieties, however. They come from a set of stories from 1859/1860 in Manila Chabacano (reproduced in Fernández \& Sippola 2017) and a set of dialogues from 1883 in Cotabato Chabacano (reproduced in Fernández 2012b). 
\end{styleStandard}

\begin{styleStandard}
The Manila texts feature six examples each of \textit{puede} (in bare form) and \textit{podé} (usually marked for aspect). The use of \textit{podé} is notable because it is similar in form and function to Cavite Chabacano \textit{pudí}. This similarity makes sense given that Manila and Cavite are so close together and, unlike Cavite and Ternate (let alone Cavite and Zamboanga), have always had a strong social link. 
\end{styleStandard}

\begin{styleStandard}
In the Cotabato dialogues, there is one token of \textit{puede} and two of \textit{puedé}, with a final accent mark, which may be evidence of yet another form that does not seem to be found in other Chabacano varieties.\textstyleFootnoteSymbol{ }\footnote{ The accents could be writing errors, of course, but the author clearly knew Spanish and Chabacano and took care in how accents were marked throughout the text.}\ Another notable feature of this text, shown in (\ref{bkm:Ref53168856}), is that \textit{puedé/puede} were used to mark an accident (with \textit{trompesá} 'trip') and an involuntary lack of experience/cognition (with \textit{mirá} 'see'). 
\end{styleStandard}


\setcounter{listWWNumiileveli}{0}
\begin{listWWNumiileveli}
\item 
\begin{stylelsLanginfo}
\label{bkm:Ref53168856}Cot.
\end{stylelsLanginfo}
\end{listWWNumiileveli}
\begin{stylelsLanginfo}
\textit{… yá }\textbf{\textit{puedé}}\textit{ lang yó trompesá su pié, ni no hay gane yó }\textbf{\textit{puede}}\textit{ mirá cay estaba yó tá apurá el modo de sacá aguja que tá pidí si ñor Quicon.}
\end{stylelsLanginfo}

\begin{styleListParagraph}
\textsc{pfv} able only 1\textsc{sg} trip 3\textsc{sg.poss} foot \textsc{neg}.even \textsc{neg exst emph} 1\textsc{sg }able see because was 1\textsc{sg} \textsc{ipfv} hurry \textsc{det} manner of get needle \textsc{comp ipfv }ask\textsc{ nom }Mr. Quicon
\end{styleListParagraph}

\begin{styleStandard}
{}'… I only accidentally tripped over his foot, I didn't even see (him) because I was in a hurry to get a needle that Mr. Quicon was asking for.' (Fernández 2012b:308)
\end{styleStandard}

\begin{styleStandard}
These uses of \textit{puedé/puede }can be attributed to influence from the Visayan languages, since they occur with other specifically Visayan features, like the emphatic \textit{gane} and the use of \textit{no hay} to negate a perfective verb (modeled after how \textit{wala} ‘\textsc{neg.exst}’ is used in Hiligaynon/Cebuano but not in Tagalog). Tagalog influence, of course, could result in constructions like \textit{yá }\textbf{\textit{puedé}}\textit{ lang yó trompesá}, and indeed, modern Ternate and Cavite Chabacano speakers would say it almost exactly the same way, with only slight differences in the form of the possibility verb. The similarity, however, is only because their ancestors grafted the same Spanish lexical items onto identical substrate structures. There is no historical evidence that speakers of any Manila Bay language were involved in the formation of Cotabato Chabacano; in fact, as Fernández (2012b) observed, the existence of this text from this time period even calls into question the common assumption in Chabacano studies that this variety is a direct offshoot of Zamboanga Chabacano. \ 
\end{styleStandard}


\setcounter{listWWNumiiileveli}{0}
\begin{listWWNumiiileveli}
\item 

\setcounter{listWWNumiiilevelii}{0}
\begin{listWWNumiiilevelii}
\item 
\begin{stylelsSectionii}
Spanish influence on the adstrates
\end{stylelsSectionii}
\end{listWWNumiiilevelii}
\end{listWWNumiiileveli}
\begin{styleStandard}
The data in this study also show just how deeply the adstrate modal systems were affected by Spanish contact, which is remarkable, given Stolz's (2002) observation that Spanish grammatical influence on Philippine languages has been mostly superficial. For example, borrowed grammatical features such as gender marking have low productivity, and many borrowed function words are doublets of still existing native equivalents. Stolz suggested, however, that investigating more ‘covert’ borrowing, including the semantics of borrowed words, might reveal other grammatical areas where a greater degree of hispanization has taken place. Indeed, this study shows that modality is one such area. Functional doublets are still present (e.g., Tagalog \textit{puwede} and \textit{maaari}), but the borrowing of Spanish modals led to changes in the overall structure of the modal system.
\end{styleStandard}

\begin{styleStandard}
The data raise a number of semantic and diachronic issues. Due to the nature of the Philippine contact setting, the directionality and timing of borrowed elements can be unclear. It is often difficult to tell if an item has been borrowed directly from Spanish into a Philippine language, or if it has filtered indirectly into one Philippine language from another. Diachronic analysis is needed to determine the grammaticalization paths of elements of both Spanish and Philippine origin. 
\end{styleStandard}

\begin{styleStandard}
One question is whether \textit{siguro }introduced a distinction between epistemic necessity and possibility into Philippine languages, or if it simply replaced some older lexical item(s) related to probability. \textit{Malamang }‘apt, likely’ (root: \textit{lamang }‘only’) and \textit{marahil }‘possibly, probably’ (root: \textit{dahil }‘reason/cause’) are two Tagalog candidates for older options. Bader et al. (1994) did not consider these items to have been as grammaticalized as \textit{baka} and \textit{siguro}, however, since unlike those markers, they occur with low frequency and still take the linking particle \textit{{}-ng/na}, as modifiers normally do. Dictionaries indicate that these other markers also have some overlap with weaker possibility modals, lending further support to the idea that epistemic markers have rather flexible force in this language (e.g., English 2010:765 lists \textit{maaari} ‘possibly’ as a synonym of \textit{malamang} 'likely'). 
\end{styleStandard}

\begin{styleStandard}
Another question is whether Tagalog \textit{maaari} was used epistemically before \textit{puwede }was borrowed, since the language otherwise maintains a distinction between epistemic and event modality. The root \textit{ari} is related to possession/ownership, suggesting that \textit{maaari} originally had a deontic or dynamic meaning. Another sign that the original meanings were more strictly deontic/dynamic is that the form used in Ternate Chabacano, \textit{mari}, does not seem to occur in epistemic contexts. Hiligaynon does not have modal counterparts to \textit{maaari} or \textit{kaya}, suggesting that, unlike the older \textit{ma(ka)-}, these items grammaticalized independently in Tagalog. 
\end{styleStandard}

\begin{styleStandard}
Finally, it seems likely that \textit{siguro }and \textit{puwede }entered Philippine languages from Spanish and/or Chabacano late into the colonial period, given the way they are still used alongside native counterparts, and the fact that Spanish influence was strongest in the late 19th century.\footnote{ Spanish was not widely accessible to most Filipinos until the mid/late 1800s, when it became more widespread due to changes to the education system and social class structure (Lesho 2018:5). } Stolz (2002:151–152) made a similar argument for the large number of function words borrowed into Philippine languages (e.g., \textit{pero} 'but', \textit{maskin} 'even though', \textit{para} 'for', etc.), pointing out that Filipinos of that era would have used such words as a way to mark their education level and participate in colonial power structures and discourse styles. A look into Philippine texts from different points of the colonial period could help to illuminate some of this speculation about the timing of the borrowings.
\end{styleStandard}

\begin{listWWNumiiileveli}
\item 
\begin{stylelsSectioni}
Conclusion
\end{stylelsSectioni}
\end{listWWNumiiileveli}
\begin{styleStandard}
In an attempt to follow Winford's (2000, 2018) footsteps, this paper has provided a detailed look into how possibility is expressed in the Zamboanga and Cavite Chabacano varieties and their adstrates. By using crosslinguistic frameworks of analysis (van der Auwera \& Plungian 1998, Palmer 2001, Matthewson et al. 2005), this paper was able to provide parallel descriptions of these four languages, describe their overall typology, and identify the similarities and subtle differences in how they mark fine-grained subcategories of possibility. The creoles were shown to have rich systems beyond what focusing on only the three 'prototypical' creole TMA categories would suggest. 
\end{styleStandard}

\begin{styleStandard}
The data demonstrate that the Chabacano creoles, Hiligaynon, and Tagalog each exhibit ‘mixed’ modal systems in two different senses. First, they combine Philippine and Spanish ways of expressing possibility. The data from these languages show more generally how modal systems can be shaped by both language contact and internal change. Second, these languages all have the same typology in how possibility markers are specified according to base and force, with \textit{pwede} being lexically unspecified for modal base but having fixed possibility force, and several markers having a fixed epistemic base and mostly fixed possibility force (except for \textit{siguro}, which can also express necessity). These languages still preserve a division between event modality and epistemic modality in most ways, but language contact somewhat blurred these lines through the introduction of \textit{pwede}. In the creoles, the use of \textit{pwede} was extended even further, as it came to be used to cover all the functions of Philippine \textit{ma(ka)-}.
\end{styleStandard}

\begin{styleStandard}
Finally, the findings also help to illuminate the grammatical and historical relationships among these languages. This detailed comparison between the creoles and their respective adstrates lends support to theories of Chabacano formation that posit separate but parallel development in each of the Chabacano-speaking communities during different time periods. While the varieties may appear to be homogenous, subtle differences in modality as well as in other aspects of the grammar and lexicon suggest that they do not actually stem from a single source (Lesho \& Sippola 2014, Sippola \& Lesho 2020). Chabacano is now in at least the ‘second phase’ of creole TMA research (Winford 2018), and these findings provide a foundation for further research into the usage of these modals and how these systems developed.
\end{styleStandard}

\begin{stylelsUnNumberedSection}
Acknowledgements
\end{stylelsUnNumberedSection}

\begin{styleStandard}
First and foremost, thanks go to Don Winford not only for inspiring this work and guiding me through its earliest stages, but for his mentorship as my advisor and his friendship ever since. There are few who have modeled such a generous and kind approach to scholarship. In addition, I also thank Judith Tonhauser, Scott Schwenter, and Michelle Dionisio for their feedback on earlier versions of this work. I am also grateful to the language consultants and other correspondents in New Jersey, California, Ohio, Zamboanga, and Cavite for their hospitality and patience in answering my endless questions. 
\end{styleStandard}

\begin{stylelsUnNumberedSection}
Abbreviations
\end{stylelsUnNumberedSection}

\begin{styleStandard}
\textsc{acc\ \ }accusative
\end{styleStandard}

\begin{styleStandard}
\textsc{av\ \ }actor voice
\end{styleStandard}

\begin{styleStandard}
\textsc{cf\ \ }counterfactual
\end{styleStandard}

\begin{styleStandard}
\textsc{comp\ \ }complementizer
\end{styleStandard}

\begin{styleStandard}
\textsc{det\ \ }determiner
\end{styleStandard}

\begin{styleStandard}
\textsc{emph\ \ }emphatic
\end{styleStandard}

\begin{styleStandard}
\textsc{exst\ \ }existential
\end{styleStandard}

\begin{styleStandard}
\textsc{fut\ \ }future/contemplative
\end{styleStandard}

\begin{styleStandard}
\textsc{gen\ \ }genitive
\end{styleStandard}

\begin{styleStandard}
\textsc{incl\ \ }inclusive
\end{styleStandard}

\begin{styleStandard}
\textsc{inf\ \ }infinitive
\end{styleStandard}

\begin{styleStandard}
\textsc{ipfv\ \ }imperfective
\end{styleStandard}

\begin{styleStandard}
\textsc{lnk\ \ }linker
\end{styleStandard}

\begin{styleStandard}
\textsc{loc\ \ }locative
\end{styleStandard}

\begin{styleStandard}
\textsc{lv\ \ }locative voice
\end{styleStandard}

\begin{styleStandard}
\textsc{neg\ \ }negative
\end{styleStandard}

\begin{styleStandard}
\textsc{nom\ \ }nominative
\end{styleStandard}

\begin{styleStandard}
\textsc{num\ \ }numeral
\end{styleStandard}

\begin{styleStandard}
\textsc{obj\ \ }object
\end{styleStandard}

\begin{styleStandard}
\textsc{ov\ \ }object voice
\end{styleStandard}

\begin{styleStandard}
\textsc{pfv\ \ }perfective
\end{styleStandard}

\begin{styleStandard}
\textsc{pl\ \ }plural
\end{styleStandard}

\begin{styleStandard}
\textsc{poss\ \ }possessive
\end{styleStandard}

\begin{styleStandard}
\textsc{q\ \ }question marker
\end{styleStandard}

\begin{styleStandard}
\textsc{sg\ \ }singular
\end{styleStandard}

\begin{styleStandard}
\textsc{stv\ \ }stative
\end{styleStandard}

\begin{styleStandard}
\textsc{verb\ \ }verb marker
\end{styleStandard}

\begin{stylelsUnNumberedSection}
References
\end{stylelsUnNumberedSection}

\begin{styleBibliography}
Asarina, Alya \& Anna Holt. 2005. Syntax and semantics of Tagalog modals. In Jeffrey Heinz \& Dimitris Ntelitheos (eds.), \textit{UCLA Working Papers in Linguistics, no. 12, Proceedings of AFLA XII}, 1–17. Los Angeles: University of California, Los Angeles.
\end{styleBibliography}

\begin{styleBibliography}
Bader, Thomas, Iwar Werlen \& Adrian Wymann. 1994. \textit{Towards a typology of modality: The encoding of modal attitudes in Korean, Japanese, and Tagalog}. Bern: Universität Bern Institut für Sprachwissenschaft.
\end{styleBibliography}

\begin{styleBibliography}
Bickerton, Derek. 1984. The language bioprogram hypothesis. \textit{Behavioral and Brain Sciences} 7(2). 173–188.
\end{styleBibliography}

\begin{styleBibliography}
Bybee, Joan L., Revere D. Perkins \& William Pagliuca. 1994. \textit{The evolution of grammar: Tense, aspect, and modality in the languages of the world}. Chicago: University of Chicago Press.
\end{styleBibliography}

\begin{styleBibliography}
Dahl, Östen. 1985. \textit{Tense and aspect systems}. Oxford: Basil Blackwell.
\end{styleBibliography}

\begin{styleBibliography}
Davis, Henry, Lisa Matthewson \& Hotze Rullmann. 2009. ‘Out of control’ marking as circumstantial modality in St’át’imcets. In Lotte Hogeweg, Helen de Hoop \& A.L. Malchukov (eds.), \textit{Cross-linguistic semantics of tense, aspect and modality}, 205–244. Amsterdam: John Benjamins.
\end{styleBibliography}

\begin{styleBibliography}
del Rosario, Flora. 2007. Memorable viaje. \textit{Aviso: Newsletter of Cavite City Library and Museum}. Cavite City, 5(7) edition.
\end{styleBibliography}

\begin{styleBibliography}
Dell, François. 1983. An aspectual distinction in Tagalog. \textit{Oceanic Linguistics} 22/23(1/2). 175–206.
\end{styleBibliography}

\begin{styleBibliography}
English, Leo James. 2008. \textit{English-Tagalog dictionary}. Mandaluyong City: Cacho Hermanos, Inc.
\end{styleBibliography}

\begin{styleBibliography}
English, Leo James. 2010. \textit{Tagalog-English dictionary}. Mandaluyong City: Cacho Hermanos, Inc.
\end{styleBibliography}

\begin{styleBibliography}
Escalante, Enrique R. 2010. \textit{Learning Chabacano: A handbook}. Manila: Baby Dragon Printing.
\end{styleBibliography}

\begin{styleBibliography}
Evans, Bethwyn \& Malcolm Ross. 2001. The history of Proto-Oceanic *ma-. \textit{Oceanic Linguistics} 40(2). 269–290.
\end{styleBibliography}

\begin{styleBibliography}
Fernández, Mauro. 2006. Las lenguas de Zamboanga según los jesuitas y otros observadores occidentales. \textit{Revista internacional de lingüística iberoamericana} 7. 9–26.
\end{styleBibliography}

\begin{styleBibliography}
Fernández, Mauro. 2011. Chabacano en Tayabas: implicaciones para la historia de los criollos hispano-filipinos. \textit{Revista internacional de lingüística iberoamericana} (17). 189–218.
\end{styleBibliography}

\begin{styleBibliography}
Fernández, Mauro. 2012a. Leyenda e historia del chabacano de Ermita (Manila). \textit{UniverSOS: revista de lenguas indígenas y universos culturales} 9. 9–46.
\end{styleBibliography}

\begin{styleBibliography}
Fernández, Mauro. 2012b. El chabacano de Cotabato: el documento que Schuchardt no pudo utilizar. In Tomás Eduardo Jiménez Juliá, Belén López Meirama, Victoria Vázquez Rozas \& Alexandre Veiga (eds.), \textit{Cum corde et in nova grammatica: estudios ofrecidos a Guillermo Rojo}, 295–313. Santiago de Compostela: Servicio de Publicaciones e Intercambio Científico.
\end{styleBibliography}

\begin{styleBibliography}
Fernández, Mauro. 2019. El escenario lingüístico de zamboanga (filipinas) a mediados del siglo XVIII. In Antonio Briz, M.\textsuperscript{a} José Martínez Alcalde, Nieves Mendizábal, Mara Fuertes Gutiérrez, José Luis Blas \& Margarita Porcar (eds.), \textit{Estudios lingüísticos en homenaje a Emilio Ridruejo}, 439–451. Valencia: Universitat de València.
\end{styleBibliography}

\begin{styleBibliography}
Fernández, Mauro \& Eeva Sippola. 2017. A new window into the history of Chabacano. \textit{Journal of Pidgin and Creole Languages} 32(2). 304–338.
\end{styleBibliography}

\begin{styleBibliography}
Forman, Michael Lawrence. 1972. \textit{Zamboangueño texts with grammatical analysis: A study of Philippine Creole Spanish}. Ithaca: Cornell University doctoral dissertation.
\end{styleBibliography}

\begin{styleBibliography}
Frake, Charles O. 1980. Zamboangueño verb expressions. In Anwar S. Dil (ed.), \textit{Language and cultural description: Essays by Charles O. Frake}, 277–310. Stanford: Stanford University Press.
\end{styleBibliography}

\begin{styleBibliography}
Himmelmann, Nikolaus P. 2006. How to miss a paradigm or two: Multifunctional ma- in Tagalog. In Felix K. Ameka, Alan Dench \& Nicholas Evans (eds.), \textit{Catching language: The standing challenge of grammar writing}, 487–526. Berlin: Mouton de Gruyter.
\end{styleBibliography}

\begin{styleBibliography}
Kratzer, Angelika. 1991. Modality. In Dieter Wunderlich \& Arnim von Stechow (eds.), \textit{Semantics: An international handbook of contemporary research}, 639–650. Berlin: de Gruyter.
\end{styleBibliography}

\begin{styleBibliography}
Kroeger, Paul. 1993. \textit{Phrase structure and grammatical relations in Tagalog}. Stanford: Center for the Study of Language and Information.
\end{styleBibliography}

\begin{styleBibliography}
Lesho, Marivic. 2018. Folk perception of variation in Cavite Chabacano. \textit{Journal of Pidgin and Creole Languages} 33(1). 1–47.
\end{styleBibliography}

\begin{styleBibliography}
Lesho, Marivic \& Eeva Sippola. 2014. Folk perceptions of variation among the Chabacano creoles. \textit{Revista de crioulos de base lexical portuguesa e espanhola} 5. 1–46.
\end{styleBibliography}

\begin{styleBibliography}
Lipski, John. 1988. Philippine Creole Spanish: Assessing the Portuguese element. \textit{Zeitschrift für Romanische Philologie} 104. 25–45.
\end{styleBibliography}

\begin{styleBibliography}
Lipski, John M. 1992. New thoughts on the origins of Zamboangueño (Philippine Creole Spanish). \textit{Language Sciences} 14(3). 197–231.
\end{styleBibliography}

\begin{styleBibliography}
Lipski, John M. 2013. Remixing a mixed language: The emergence of a new pronominal system in Chabacano (Philippine Creole Spanish). \textit{International Journal of Bilingualism} 17(4). 448–478.
\end{styleBibliography}

\begin{styleBibliography}
Lipski, John \& Maurizio Santoro. 2007. Zamboangueño Creole Spanish. In John Holm \& Peter Patrick (eds.), \textit{Comparative creole syntax}, 373–398. London: Battlebridge Press.
\end{styleBibliography}

\begin{styleBibliography}
Lorenzino, Gerardo. 2000. \textit{The morphosyntax of Spanish-lexified creoles}. München: LINCOM.
\end{styleBibliography}

\begin{styleBibliography}
Matthewson, Lisa. 2013. Gitksan modals. \textit{International Journal of American Linguistics} 79(3). 349–394.
\end{styleBibliography}

\begin{styleBibliography}
Matthewson, Lisa, Hotze Rullmann \& Henry Davis. 2005. Modality in St’át’imcets. In J.C. Brown, M. Kiyota \& T. Peterson (eds.), \textit{Papers for the 40th international conference on Salish and neighbouring languages}, 166–183. Vancouver: University of British Columbia Working Papers in Linguistics.
\end{styleBibliography}

\begin{styleBibliography}
Maurer, Philippe \& the APiCS Consortium. 2013. Ability verb and epistemic possibility. In \textit{Atlas of pidgin and creole language structures}. Leipzig: Max Planck Institute for Evolutionary Anthropology. https://apics-online.info/parameters/55\#2/16.5/10.0 (20 February, 2019).
\end{styleBibliography}

\begin{styleBibliography}
Michaelis, Susanne Maria, Philippe Maurer, Martin Haspelmath \& Magnus Huber (eds.). 2013. \textit{Atlas of pidgin and creole language structures}. Leipzig: Max Planck Institute for Evolutionary Anthropology. http://apics-online.info/ (20 February, 2019).
\end{styleBibliography}

\begin{styleBibliography}
Palmer, Frank R. 2001. \textit{Mood and modality}. Cambridge: Cambridge University Press.
\end{styleBibliography}

\begin{styleBibliography}
Riego de Dios, Maria Isabelita O. 1986. A composite dictionary of Philippine Creole Spanish (PCS). \textit{Studies in Philippine Linguistics} 7(2).
\end{styleBibliography}

\begin{styleBibliography}
Rubino, Carl. 2008. Zamboangueño Chavacano and the potentive mode. In Susanne Michaelis (ed.), \textit{Roots of creole structures: Weighing the contribution of substrates and superstrates}, 279–300. Amsterdam: John Benjamins.
\end{styleBibliography}

\begin{styleBibliography}
Santos, Jamil. 2018. Natikman mo na ba ang ice cream na salted egg flavor? \textit{GMA News}. https://www.gmanetwork.com/news/balitambayan/talakayan/674621/natikman-mo-na-ba-ang-ice-cream-na-salted-egg-flavor/story/ (12 September, 2020).
\end{styleBibliography}

\begin{styleBibliography}
Schachter, Paul \& Fe T. Otanes. 1972. \textit{Tagalog Reference Grammar}. Berkeley: University of California Press.
\end{styleBibliography}

\begin{styleBibliography}
Sippola, Eeva. 2011. \textit{Una gramática descriptiva del chabacano de Ternate}. Helsinki: University of Helsinki doctoral dissertation.
\end{styleBibliography}

\begin{styleBibliography}
Sippola, Eeva \& Marivic Lesho. 2020. Contact-induced grammatical change and independent development in the Chabacano creoles. \textit{Bulletin of Hispanic Studies} 97(1). 105–123.
\end{styleBibliography}

\begin{styleBibliography}
Spitz, Walter. 2002. Voice and role in two Philippine languages. In Malcolm Ross \& Fay Wouk (eds.), \textit{The history and typology of western Austronesian voice systems}, 379–404. Canberra: Pacific Linguistics.
\end{styleBibliography}

\begin{styleBibliography}
Stolz, Thomas. 2002. General linguistic aspects of Spanish-indigenous language contacts with special focus on Austronesia. \textit{Bulletin of Hispanic Studies} 79(2). 133–158.
\end{styleBibliography}

\begin{styleBibliography}
van der Auwera, Johan, Andreas Ammann \& Saskia Kindt. 2005. polyfunctionality and Standard Average European. In Alex Klinge \& Henrik Høeg Müller (eds.), \textit{Modality: Studies in form and function}, 247–272. London: Equinox Publishing.
\end{styleBibliography}

\begin{styleBibliography}
van der Auwera, Johan \& Vladimir A. Plungian. 1998. Modality’s semantic map. \textit{Linguistic Typology} 2(1). 79–124.
\end{styleBibliography}

\begin{styleBibliography}
Vander Klok, Jozina. 2013. Pure possibility and pure necessity modals in Paciran Javanese. \textit{Oceanic Linguistics} 52(2). 341–374.
\end{styleBibliography}

\begin{styleBibliography}
Whinnom, Keith. 1956. \textit{Spanish contact vernaculars in the Philippine Islands}. Hong Kong: Hong Kong University.
\end{styleBibliography}

\begin{styleBibliography}
Winford, Donald. 2000. Irrealis in Sranan: Mood and modality in a radical creole. \textit{Journal of Pidgin and Creole Languages} 15(1). 63–125.
\end{styleBibliography}

\begin{styleBibliography}
Winford, Donald. 2018. Creole Tense–Mood–Aspect systems. \textit{Annual Review of Linguistics} 4(1). 193–212.
\end{styleBibliography}

\begin{styleBibliography}
Wolfenden, Elmer P. 1971. \textit{Hiligaynon reference grammar}. Honolulu: University of Hawaii Press.
\end{styleBibliography}

\begin{styleBibliography}
Yatchi. 2012. Change. \textit{Definitely Filipino}. https://definitelyfilipino.com/blog/change-2/ (20 February, 2019).
\end{styleBibliography}

