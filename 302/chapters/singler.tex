\documentclass[output=paper,colorlinks,citecolor=brown]{langscibook}
\ChapterDOI{10.5281/zenodo.6979331}


\author{John Victor Singler\affiliation{New York University}}
\title[African American language and life in the antebellum North]{African American language and life in the antebellum North:  Philadelphia’s Mother Bethel Church}


\abstract{Philadelphia was the most important city for African Americans in the Early Republic. In the decades up to the 1830’s, neighborhoods emerged with a concentration of African American churches and other public institutions, yet remained majority-white with everyday interracial interaction. The dominant African American institution in Philadelphia was Bethel Church, the mother church of the African Methodist Episcopal Church. Linguistic data from trials in the years 1822 to 1831 in Bethel’s \textit{Minutes and Trial Book} (MTB), while limited in domain and size, suggests – tentatively – that the vernacular of church members was for the most part like the vernacular of other Philadelphians.
Bethel Church’s leader, Bishop Richard Allen, spearheaded the campaign in 1824--1825 for African American emigration from Philadelphia to Samaná now in the Dominican Republic. The provenance of those who went to Samaná establish a northern basis for Samaná English and hence not directly pertinent to AAVE diachrony. 

\begin{sloppypar}
\keywords{Philadelphia, Samaná (Dominican Republic), Bethel A.M.E. Church, Minute and Trial Book, Rev. Richard Allen, Joseph Cox, written records, a-prefixing, Bethel-Samaná connection}
\end{sloppypar}
}

\begin{document}
\maketitle 

\section{Introduction}
Prior to the Great Migration of the twentieth century, the African American population of the US lived overwhelmingly in the South (with the South defined as those states where chattel slavery was legal at the beginning of the American Civil War). In every census from 1790 through 1890, 91--95\% of all African Americans lived in (ex-)slave states. Consequently, there is a tendency -- among linguists and others -- to overlook African Americans in the North in earlier eras of American history. The reality is that, starting in the seventeenth century, slavery had been part of all the American colonies, particularly those on the Atlantic seaboard. Throughout the colonial era, in addition to southern ports like Charleston, slave ships brought their human cargo to Boston; Newport, Rhode Island; and Perth Amboy, New Jersey.\footnote{{For a forty-year period in the mid-eighteenth century, New York and Pennsylvania imposed a tariff on the delivery of enslaved people from Africa and the West Indies to their ports, and New Jersey did not. Thus, Perth Amboy -- 50 km south of New York -- became the port of entry for the enslaved for the region.}} Then, in the decades after the American Revolution, the northern states abolished slavery. “In 1790 the black population in New England and the Middle Atlantic [New York, New Jersey, and Pennsylvania] was largely enslaved, with approximately 40,000 enslaved people versus 27,000 free blacks. By 1830 the black population of these areas had become overwhelmingly free, with 122,000 free blacks versus 3,000 slaves (mostly in New Jersey)” \citep[5]{Newman2008}.

The language and life of African Americans in the North in the early years of American independence merit examination. The focal point of the present study is Philadelphia’s Bethel Church, founded in 1794 and then taking on greater significance in 1816 when it became the mother church of the first independent African American religious denomination. Section \ref{section2} presents information regarding free African Americans in the Early Republic, Philadelphia’s African American population, and the place of Bethel Church in Philadelphia life. The remainder of the study then explores three aspects of Bethel’s history, each with linguistic consequences:

\begin{itemize}
\item  
An early \textit{Minutes and Trial Book} from Bethel survives. Section \ref{sec:section3} examines the data to be found therein, principally in the transcripts of 33 trials conducted by elders in the years from 1822 to 1831. Part of the discussion in section \ref{sec:section3} concerns the limitations of using written data. Beyond those limitations, the amount of data is admittedly somewhat limited. These restrictions notwithstanding, the transcripts are instructive in what they have to say about the language of African Americans in Philadelphia at that time.
 
\item  
There is much more information about the social setting of free African American society in the Early Republic than we may have assumed. In fact, the importance of Philadelphia in the history of African American cultural institutions, Bethel Church above all others has given rise to extensive analysis by historians of the Cedar neighborhood, the area containing -- and grounded by -- Bethel. Section \ref{section4} draws on historians’ extensive scholarship to present the neighborhood’s social and economic character. While it was home to Philadelphia’s African American institutions and routinely identified as an African American neighborhood, its population was majority-white. The level of interaction among different ethnoracial groups “points out the pitfalls inherent in generalizing twentieth-century interpretations to nineteenth century situations” \citep[277]{Lapsansky1975}.
 
\item  
Bethel Church was the primary launching point for African American immigration to Samaná in what is today the Dominican Republic. The English of Samaná has been studied by \citealt{DeBose1983}, \citealt{Hannah1997}, and Shana Poplack and her colleagues, beginning with \citealt{PoplackSankoff1987}. For Poplack and her colleagues, the relative conservatism of the language of Samaná has played a central role in their argument for divergence, i.e. that the African American Vernacular English (AAVE) of the early nineteenth century was much more like white vernacular Englishes than is true of AAVE today. In section \ref{section5}, I look at the information presented throughout this study as well as recent historical scholarship pertaining to Samaná to assess the applicability of Samaná English to the divergence hypothesis.
 
\end{itemize}
 
 
\section{Background} \label{section2}
\subsection{Cities as gathering points for free African Americans in the Early Republic}

The United States in the late eighteenth and early nineteenth century was overwhelmingly rural. In the 1820 census, only eleven cities had a population of more than 10,000. While the country as a whole was rural, free African Americans tended to live in cities, as \tabref{tab:1 singler:1} shows. The three largest American cities -- New York, Philadelphia, and Baltimore -- all had free African American populations of 10,000 or more.

\begin{table}  
\caption{US cities with the largest free African American population, 1820 census. Source: For Philadelphia, \citet[137]{Nash1988}; for Baltimore and New York, \citet[247]{Curry1986}; for states, \textit{Census of 1820}. Comments: “New York” combines the cities of New York and Brooklyn. For Philadelphia Nash includes the “adjacent urbanized areas of Northern Liberties, Southwark, Moyamensing, and … Spring Garden and Kensington,” (137) all of which were later absorbed into Philadelphia by the 1854 Act of Consolidation.}
\begin{tabular}{l rrrr}
\lsptoprule
& \multicolumn{2}{c}{Afr. Am. (city)}\\\cmidrule(lr){2-3}
& \textbf{Free} & Total & City total & Free (city) as pct. of state\\\midrule
Philadelphia & \textbf{12,110} & 12,110 & c.  108,809 & 40\%\\
New York & \textbf{11,065} & 11,773 &  134,893 & 38\%\\
Baltimore & \textbf{10,326} & 14,683 &  62,738 & 26\%\\
\lspbottomrule
\end{tabular}
\label{tab:1 singler:1}
\end{table}


\subsection{The provenance of Philadelphia’s African American population in the early nineteenth century}

While the members of Bethel Church are the central focus of the present study, there is very little information about individual members of the church.\footnote{Exceptionally, there is a considerable amount of information about the Reverend Richard Allen, the founder of Bethel Church, and some information about Joseph Cox, the secretary of the church’s board. Both of these men are discussed below.} Consequently, I turn to what is known of the provenance of African Americans in Philadelphia more generally. Even here, the data is limited. \citet{Nash1988} draws on two sources pertaining to Black mariners that required them to state their place of birth. He then uses the tabulation to get a rough idea of where Philadelphia’s African American men as a whole had been born. Nash’s chart has appeared in previous work \parencites[]{Singler1998}[13]{PoplackTagliamonte2001}[315]{Singler2007part2} and is presented below in the adapted form in which it occurred in those works.\footnote{Nash’s complete table presents tabulations for 1803, 1811, and 1821. The distribution is comparable in each. The 1821 tabulation has the most relevance to the present discussion.}

\begin{table}
\caption{Birthplaces of Black Philadelphia mariners, 1821 (Source: \citealt[136]{Nash1988})\label{tab:2 singler:2}}

\begin{tabular}{lrr}
\lsptoprule
& $n$ & \%\\
\midrule
New England & 18 & 6.3\\
New York and New Jersey & 56 & 19.7\\
Philadelphia & 24 & 8.5\\
The rest of Pennsylvania & 55 & 19.4\\
Delaware and Maryland & 87 & 30.6\\
Virginia and North Carolina & 26 & 9.2\\
Lower South & 17 & 6.0\\
Africa & 1 & 0.4\\
\midrule
& 284 & \\
\lspbottomrule
\end{tabular}
\end{table}

As Nash observes:

\begin{quote}
If those who pursued a maritime calling were roughly representative of Philadelphia’s free black male population in this period, only about one-twelfth of the city’s black residents had been born there. Philadelphia, then, was a city of refuge, not the place of birth of most of its free black populace. Roughly two-thirds of them came from within one hundred miles of Philadelphia (136–137, quoted in \cites[]{Singler1998}[314]{Singler2007part2}).
\end{quote}

The hundred-mile radius includes New York City, all of New Jersey and Delaware, and Baltimore. Delaware was a slave state, but Nash observes: 
\begin{quote}
In Delaware, … the declining viability of slavery, especially in corn-growing New Castle and Kent counties [the two more northerly of Delaware’s three counties, hence closer to Philadelphia], together with strong Quaker and Methodist lobbying against slaveholding, reduced the slave population from 8,887 to 4,177 between 1790 and 1810 (137--138).
\end{quote}

In the 1820 census, African Americans comprised 24\% of Delaware’s total population. The enslaved population of Delaware was but a quarter of this, thus only 6\% of the state’s entire population. Further, the number of enslaved people on a given farm tended to be small.

With regard to Maryland and Virginia, Cassidy observes: “In Virginia and Maryland … slaves were never as separate from the work and general life of their owners as in the South. Plantations or households were much smaller, and the numerical proportion of Blacks to Whites was never as high” (1986, 35, quoted in \citealt[110]{Singler2015}). Within Maryland, the rural area closest to Philadelphia practiced diversified agriculture, as landholders did not find large plantations with a large enslaved work force to be economically viable. The state did have large tobacco plantations, but they were located in the southern counties, the area furthest from Philadelphia. 

People who were formerly enslaved in Delaware, Maryland, or Virginia had still been enslaved; I do not mean to suggest otherwise. Still, if Nash’s table is seen as reasonably representative of the provenance of the African American population of Philadelphia, Nash’s calculation regarding the 100-mile radius suggests that at the very least two-thirds of those coming from Delaware and Maryland came either from Baltimore or from an area of small-scale agriculture in Delaware and the northeastern part of Maryland (including the northern part of the Eastern Shore).

In sum, the African American population of Philadelphia in the early nineteenth century consisted primarily of people who had moved there from elsewhere, with half of them coming from elsewhere in Pennsylvania or from states to its north and east. Of the other half, most came from nearby areas in Delaware and Maryland, not from large plantations. 

\subsection{Philadelphia and Mother Bethel Church}
\begin{sloppypar}
The unofficial capital of free African American society at the time of the American Revolution and for decades thereafter was Philadelphia. African Americans in Philadelphia took the lead in establishing what Newman designates the “communal institutions that guided African Americans from slavery to freedom” (\citeyear[5]{prophetNewman2008}). 
\end{sloppypar}

\begin{quote}
Newly freed blacks entered a largely hostile world. Antislavery groups like the Pennsylvania Abolition Society and the New York Manumission Society were important in providing legal support and in establishing schools, but blacks realized they needed their own organizations to promote their own issues in their own language and fashion. Benevolent organizations, fraternal groups, schools and literary societies and churches emerged to provide blacks with their own space, beyond the control or tutelage of whites \citep{librarycompany2011}.%I assume that an indented quote is intended here SN
\end{quote}

As Gordon observes, “Beginning in the late 1780s and gathering speed during the early 1790s, black religious life became the central expression of African American institution building” (2015:394). The Black church stood as “the one impregnable corner of the world where consolation, solidarity and mutual aid could be found and from which the master and the bossman -- at least in the North -- could be effectively barred” (\citealt[106]{Wilmore1972}, quoted in \citealt[114]{Nash1988}).

In 1787, Absalom Jones and Richard Allen, two leaders of African Americans in Philadelphia, formed the Free African Society, “a mutual aid group composed of free blacks and former slaves dedicated to piety, benevolence and black solidarity” \citep[9]{Newman2008} and possibly “the first autonomous organization of free blacks in the United States” \citep[332]{Nash1989}. While not officially a religious organization, the Free African Society came out of African American participation in St. George’s Methodist Church, a racially mixed congregation whose African American members were being subjected to growing discrimination by the church’s white leadership. The Free African Society gave rise to two Black congregations in 1794. Jones established St. Thomas African Episcopal Church, which was affiliated with the Episcopal Church, and Allen formed the Bethel African Methodist Episcopal Church. Bethel was originally part of the Methodist Church, but the hostility of whites in the church to African American participation led to Allen’s establishing the African Methodist Episcopal (AME) church in 1816 as a separate denomination. It came to be known as Mother Bethel, an acknowledgment of its role as the founding church of the AME denomination, which was in turn the first independent African American denomination.

Below I discuss the Cedar neighborhood where Bethel Church was located. I mention it here because of \citegen{Lapsansky1975} assertion that Mother Bethel was “…a central force in the building of a viable black community … in the Cedar neighborhood… Bethel, under the leadership of its founder, Richard Allen, had built a membership of several hundred by 1807 and had attracted a number of black residents to settle in the neighborhood…” (180, 182).

The church’s meteoric growth continued beyond 1807. \citet[139, 142]{HortonHorton1998} place its membership at almost 1300 in 1810 and approximately 2000 in 1820 -- the latter figure representing one-sixth of Philadelphia’s African American population.

Bethel and -- to a much lesser extent -- smaller African American churches such as St. Thomas African Episcopal Church and First African Presbyterian Church (founded in 1807) served their members in many vital ways. The most important institutions in Black life during this period, “black churches provided opportunities for spiritual worship and guidance, political forums, social and family aid associations, and facilities for community meetings, cultural preservation, entertainment programs, and education and training” \citep[130]{HortonHorton1998}.

Horton \& Horton note that church leaders 
\begin{quote}
… often acted as a judicial body. An analysis of church records reveals an institution deeply involved in the daily life of the community, as much concerned with secular relations as with the spiritual well-being of its congregation… It tried to control everything from petty personal disputes between members to more serious criminal acts. The black church attempted to handle internal problems within the community, without recourse to the formal white authority that was seen as unsympathetic, not trusted to render fair judgments or give equitable treatment to African Americans. Trials were held regularly in Philadelphia's Mother Bethel AME Church and judgments were passed on a broad range of offenses, from lying in public, to breach of contract, and assault (\citeyear[147, 148]{HortonHorton1998}).
\end{quote}

\section{The \textit{Minute and Trial Book} of Bethel Church} \label{sec:section3}
\subsection{Assessing the \textit{Minute and Trial Book} }

The trials to which Horton and Horton refer are recorded in Bethel’s \textit{Minute and Trial Book} (MTB), which is housed in the Historical Society of Pennsylvania’s library in Philadelphia.\footnote{The MTB is part of the microfilmed Bethel AME Church collection at the Historical Society, which is where I saw it. Horton \& Horton report that a microfilm copy is “on file at the Afro-American Communities Project at the National Museum of America History of the Smithsonian Institution.” (299n).} As the book’s title suggests, the MTB contains minutes of the governing board’s meetings (often limited to resolutions and actions taken by the board) and the transcripts of trials conducted by elders. The present focus is on 33 trials in the years from 1822 to 1831, and a dramatic account near the beginning of the book of a heated conflict that occurred on a Sunday morning in August, 1822, between Bethel Church and its breakaway neighbor, Wesley Church (cf. \citealt{prophetNewman2008}, Chap. 8). In addition to the trials under study, the MTB contains one from 1838 and twelve from the second half of the nineteenth century, but they are not under consideration here.

There are necessarily concerns and caveats in trying to extrapolate the grammar of the spoken language from written data from an earlier era. \citet[57]{schneider2013investigating} notes that
\begin{quote}
 ...  the written record functions as a filter, as it were: it provides us with a representation of a speech act that we would have liked to have listened to and recorded acoustically and that without the written record would have been lost altogether; but at the same time the rendering of the speech event is only indirect and imperfect, affected by the nature of the recording context in certain ways.   
\end{quote}
 

\citet[1]{Montgomery1999} makes a case for the principled use of manuscript documents for reconstructing earlier stages of colloquial English (see also \citealt{WolframThomas2002}). In the present case, the most obvious limitations are that the amount of data is limited and the structure of trials is such that the data is constrained in style and range. Under such circumstances, one has to focus on what is present in the documents rather than assigning importance to the absence of individual features. The trials, as formal events, would seem likely to elicit formal speech. Nonetheless, because of the emotion-charged setting, speakers seem likely to have been concentrating more on what they said than on how they said it. At the same time, the secretary, in writing down testimony, would seem to be much more likely to move it closer to standard English rather than further away. 

As a source of information about the spoken English of African Americans in Philadelphia in the early nineteenth century, the MTB has key advantages. Both the speaker and the scribe (the secretary of the board) are African Americans, members of the same Philadelphia community. The written language under study has not been created for its own sake. Rather, recording the testimony “as spoken” has a raison d’être in that the model for the proceedings is the court trial and, as such, there is an underlying quasi-legal motivation in the scribe’s attempt to present what a speaker has said verbatim.\footnote{{“Verbatim” speech is constructed, even when the reporter’s goal is to represent exactly what was said.}}

\subsection{Examples of minutes and trials}

The MTB contains minutes as well as trials, and the minutes are -- apart from a considerable amount of nonstandard spelling -- fairly standard, as illustrated by the minutes of the meeting on 15 June 1824:

\ea\label{ex 1:singler:1}
At a special meeting of the Trustees of the corperation of Bethel Ch, at the usual place after prayers we proceeded to Buisness, the buiness of the meeting was stated by Brother Jas Wilson, wich was that the property in Marrets lane belonging to the Corperation was much out of repear, \& growing worse, \& opon examination the House could not be repard without takeing the chimeny down. we thought it moust advantages to the corperation for the House to undergo a thourreth repear, \& also, as there is a vacant space of 8 feet on the West side of the House between our property \& our nabours property wich is a recepticle of filth from his slaughter House he being a bucher by trade it was On motion Resolved that the house be moved to the northwes cornor of the lot + undergo a thourreth repear

on motion resolved that we acceed to the propersition of Brother Tho Roberson wich is as followes that he will move the house to the NW cornor as above discribed + do all the carpenters work + glazing for \$30. On motion resolved that the foundation wall of the house be stone.  

\z

One illustration of the standard character of the minutes in (\ref{ex 1:singler:1}) lies in the two occurrences of the subjunctive \textit{be} in the reporting of resolutions that the board passed. 

The trials in the discussion that follows consist of 18 from 1822 through 1825 and 15 from 1829 through 1831. They were conducted by ad-hoc committees of elders, usually five or six men, with certain men (notably Richard Allen and the secretary Joseph Cox) generally a part of the committee. The trials address church members’ domestic disputes, financial and other quarrels, and personal impropriety.\footnote{{The trials from the second half of the nineteenth century are different in nature. They are “church trials,” where the subject matter is an alleged irregularity in church governance or misconduct on the part of a member of the clergy.}} The range of trial subjects can be seen by the list of topics addressed in the first set of trials, the ones from 1822 through 1825:


\begin{description}\sloppy
\item [Domestic:] marital discord, non-support and physical abuse, a man cohabiting with a woman not his wife
 
\item [Financial:] non-payment of a debt, failure to repay a loan, failure to share business earnings with a partner, illegal sale of other people’s goods, petty theft
 
\item [Personal impropriety:] imprudent conduct, rumor-mongering, slander, assault, belligerence, flirting with another woman’s husband, dispute between a women’s church group and an individual woman      
\end{description}

The most severe penalty for someone judged guilty was expulsion “from society,” i.e. from Bethel Church.

The secretary for Bethel Church -- hence the equivalent of a court reporter for the trials during the period under study -- was Joseph Cox (1778?--1843).\footnote{It is not known where Cox was born, but biographers of his much younger sister Patricia Cox Jackson state that she was born in 1795 in Philadelphia to free parents \citep[60]{Schroeder2007}.} A tanner by trade, Cox became a local preacher and elder. The obituary published by the Philadelphia District Conference of the AME Church said of Cox, “Though not educated, in the popular sense of the term, yet he had greatly enriched his mind by various reading and patient reflection” (quoted in \citealt{Payne1968} [1891]:178). Payne estimates Cox’s formal schooling as “about as much as a primary school education,” adding “but he was well read.” He describes Cox as having been “endowed by nature with a powerful intellect as a natural orator and logician” (1968 [1891]:394).

As noted, the MTB also includes an 1822 account of a melee between Bethel Church and its breakaway neighbor Wesley Church. It is not clear who the author is of this account, but its heated style is distinct from Cox’s.

The handwriting for the minutes and trials that Cox recorded is not always the same. An October, 1822, resolution stated: “Resolved that we employ a clerk to record the minutes of Bethel Church in a book to be provided for that purpose.” The possibility exists, then, that the clerk introduced errors or “corrected” what he copied. There is no way to identify any changes that would have been introduced in this way. A further point is that the handwriting can present its own problems. One of the clerks places a flourish on word-final <m> that looks like an <e>, thus giving the appearance of <hime> rather than <him>, as in \textit{brother Jonathan Trusty + Shepperd Gibbs is a committe to informe hime of this resolution}. I have assumed that the clerk intended <inform him> and have transcribed it accordingly.

Below I identify individual noteworthy features from the trials. However, the most effective way to convey the character of the trials is by presenting examples of complete trials themselves. Accordingly, I present three trials, selected essentially at random.\footnote{{Two question marks appear on either side of a word or part of a word if I think I can identify the word -- or can attest to the presence of a part of the word -- but am not certain, e.g.} {\textit{??run??}} {and} {\textit{Break??s??}}{. When I am not at all sure of a word, I present it with underlining. e.g.} {\textit{came a running a cross the \_\_\_\_ to oppose}}{... When full names of litigants are used in the MTB, I have reduced them to abbreviations followed by “[ ].” In cases where I present initials without a following “[],” it is because initials were used in the MTB. The date (/dd/mm/yy) after an example refers to the entry in the MTB.}}

\ea%2
    \label{ex:2 singler:2}
     
  
                                         Phil. Augt 19th, 1823

[three cases on that date; this was the third]

A charge brought against A[] R[] by D[] T[] of threattning to beat her \_\_\_. she ??said?? that D[] was up stairs pretending to be Sick and Beckoning to her husband out the windows to come to her – up stairs. AR. ??run?? her fist in D.Ts face and abused her very much

A R[] says that D.T. has from time to time decoyed her husband into DT. house. I told her of it and that I did not like it. she Borrowed 12 ½ cents from T. R[] which I did not like, I told her if she parted M[] D[] \& wife, or other men and there wives that she should not part my husband and me. I also sd. that my husbd had not time to eat his victuals for DT. keeps sending for him to pray for her 2 or 3 times during one meals victuals. DT. told me that I was a Devil and that I would go to hell any how. But T. R[] was a nice man. 

  We the Committee are of opinion that A[] R[] is guilty of a Breach of the Discipline and is no more a member of Society\medskip

Rev Rich Allen \qquad   William Cornish

Jonathan Trusty \qquad    James Wilson

John F. Gibbs~    \qquad    Joseph Cox Secty %Let me know if the names on the right side should be aligned.
\z

\ea%3
    \label{ex:singler:3}
    
                                          October 6th 1823

[two cases on that date; this is the second]

A charge brought against I[] D[] and wife by William Cornish of disputeing [symbol]

J[] D[] states that she went to bed and while she was taking her cloathes off he told her to blow out the candle. I told him that I had just came up to bed 

he appeared to just awake out of a dose of sleep and I thought that he did not know that I had just come up. 

he told me the second time to blow out the candle. I told him that he was pinched, however I blowed out the candle. Some conversation ensued between us. he asked me what I was a doing there, And why I did not go where I slept for those two nights. I had a great deal of washing to do, and had to work at night to get them ready to take them home in time

Some more talk took place, he then put me out of the Bed. I returned to the Bed + took hold of some of the Bed cloathes, and he struck me several times and hurt me very much. I hallowed murder and the watchman came 

Ann Cleasy states she heard the noise and when the watchman came I ran out the back door and did not return untill the next day

I[] D[] states that his wife came up and awakened him late at night --  the candle was shining in my face. I told her to blow it out but she would not. I arose to blow it out then she blew it out. I then asked her where she slept last night. She \# that that was her bed and told me to go down stairs to my Bed, Said I suppose you want to have it to ??Say?? of me as C[] T[] Said of his wife that she left his Bed

I told her that we had nothing to do with them

I told her to hush and let me alone or I would put her out of the Bed. She continued to go on in an aggravating way, untill I put her out of the bed. She laid hold of the Bed Cloathes and then took a foul hold of me and she cryed Murder untill the watchman came. Such is her abuse that I cannot speak without there being a noise.

  We have taken the above Case into Consideration and we are of opinion that Both of them are guilty of a Breach of the Decipline. They may remain on trial 6 months , But within that time if either of them Break??s?? the decipline they shall be Exspelled from society
\z


\ea%4
    \label{ex:singler:4}
     
    

                                       Philadelphia April 8th 1829

At a meeting of a committee to here + try the case of H[] H[] + M[] B[]. A charge brought Against H. H[] of Pregency of murdering the child + of M[] H[] takeing the Chamber pot to the prevey House for her and also her going into the Room of the said M. H[] in Lumbard St. + the Door shut after her. + not opend for the space of 10 or 15 minutes althrough some person knocked at the Door. M. B[] say that she whent to H. H[] to Borrow a cop as a pattern to cut one out by + She was grinding coffee. She put Down the coffee + whent into another room Walking half bent + sent the cop out by a child but did not come out her self, + a pear of Sissors also by the child but H. H[] did not come out while I stayed at her house, + M.B. said that she herself had had 4 children + that if ever herself was with child that she belived H[] H[] was with child at the time she went to her for the cop. H H[] Defendent, Denyes murdering any child or being pregnent or with child and as to M. H[] emptying the chamber pot she had denyed it but he told her that he did do it about dusk one evening one time the 2 children being sick + she not at home. Necesaty required it + he done it. and as to her going into her uncles M. H[] Room in Lumbrd St. at Solamon Clarkson, Choolroom one evening the Female society meet of Wich she is a member, a Disterbence took place in the St. the Wachman sprung his Rattels. the members of the society rund to the Windo to see what was going on. the window bing croded she whent up one pare of stears into her uncles Room in order to have a better oppertunity from his Windows to See

he/M.H/ Was sitting at the tabel Writing, as to the Door being Shut or fasned she could not tell anything. She stayed there but a short time, before she came Down again. 

Bashely Chandler Witness Say that she is aquainted with H.H. that she lives up stears over her + she never saw any thing in or about her that looked like her being in a state of Pregency to her Knowledge. She aperd as to her shop[shape?] always alike She had nothing to say against -- 

R. Valentine, Witness say that she is aquainted with H. H[]. + knows no harm of her. She did not viset her House ??perticler?? at the time of this Report About her because the small pox was in her own family + she did not think it prudent to visit any of her Nabours at that time, on that account, she said she never saw any thing about her that look like her being with child to her Knowledge.

Sybella Oliver Witness say that she knows H. H[] but Knows nothing Disrespectfull of her, that she always aperd to be an upwright person as farr as ever she saw. Rosannah Johnson Witness say that she Knows H. H[] but knows nothin, nor never saw anything Disrespectfull of her. Ann Fisher Witness say that she Know H. H[] but that she knows nothing nor never saw anything Disrespectfull of her. Lucy Lewis Witness say that she knows nothing Disrespectfull of H.H. 

We the Committee have taken the above case into consideration + are of an oppinion that from the strenght of evedance H[] H[] is not guilty of Breach of the Law of God nor of our Decipline We therefore acquit her.

Names of the Committee

1. Walter Proctor, 3. James Wilson,

2. John Cornish, 4. John?? Tecis??

Joseph Cox Secty

Richard Allen, Preacher in Charge
\z

\subsection{Nonstandard features}
In the accounts of the trials in the MTB, certain nonstandard features emerge:

\subsubsection{Possession}
While the possessive morpheme \textit{{}-s} is usually present, there are some instances where it is not present and where possession is signaled by Possessor-Possessed word order alone.\footnote{An anonymous reviewer questions whether an orthographic convention might exist to signal possession overtly that I was unaware of (e.g. an extra mark at the end of the possessor) and that what I considered zero marking was not in fact the case. I certainly consider this to be a possibility in many instances but not all. For example, it seems clear that this is not the case in the examples that I present in the text.}

\ea%5
    \label{ex:singler:5}
     
    

          her present husband was not the father of Josh Wife  12/12/23
\z

\ea%6
    \label{ex:singler:6}
     
    

          at the end of 2 weeks Robt wife father told her that she was not marred [married] + obreaded [upbraided] Sarah + all the methodist  of leaving… 23/3/30\footnote{{The absence of overt number marking on} {\textit{Methodist} }{is exceptional. In the case of} {\textit{Methodist}}{, the singular and plural forms of the noun would be homophonous.}}
          
\z       
          
\ea%7
    \label{ex:singler:7}
  
          A[] T[] say that Anthony Davis never saw anything indeasent or unbecoming by him in Chas B[]s house in his life that he did frequent the house it was true but had no evil desing[design] in it + it was by Chas requst + he did not know that it was disagreabel to Chas 13/10/30
 \z
 
\ea%8
    \label{ex:singler:8}
     
    

          he saw a half Dollar laying on the carpet + pick et up + put it in his pocket with the intenson of giveing it to the gentleman at the Brakfast table as he was not in the room but he saw no more of him untill he met him in the alderman office at wich time he layed the charge against him 4/7/31
\z

\subsubsection{\textit{Be} leveling: the subject type constraint}
A phenomenon found in dialects of American English -- and English vernaculars more globally -- is \textit{be} leveling, whereby \textit{was/were} alternation for past-tense forms is realized as \textit{was} across the board and likewise, though less frequently, the \textit{am/is/are} alternation is realized as \textit{is}. This shows up in the MTB but with an added wrinkle, namely that the leveling occurs only with full NP subjects.  This distinction between \textit{they} and other 3pl subjects is what \citet{MontgomeryDeMarse1993} designate “Subject Type Constraint”. While there are only a limited number of tokens, the \textit{they}/NP distribution is borne out in the trial testimony.  There are two instances of \textit{they were} (as in 9) and six of plural-NP \textit{was} (as in 10):

\ea%9
    \label{ex:singler:9}
    until I hear that they were about parting 18/12/22\\
         
\z

\ea%10
    \label{ex:singler:10}
    
    there frocks was open behind 15/4/29 \\
 \z          
          
While my focus was on the trial testimony, I also examined the rest of the data in the MTB for the years from 1822 to 1831, i.e. the report of the Bethel-Wesley melee as well as the board’s minutes. Past-tense \textit{be} variability shows up here as well, both in the account of the fracas and in the board’s minutes. As with the trial testimony, the leveling can occur when a noun is semantically plural (\ref{ex:singler:11}) or the NP consists of co-ordinate singular NP’s (\ref{ex:singler:12}). 

\ea%11
    \label{ex:singler:11}
     After the objects of the meeting was stated by the chair / Rev\textsuperscript{d}  Rich\textsuperscript{d}  Allen / the following resolutions was enterd into ... 22/11/24\\
        
    \z

      
\ea%12
    \label{ex:singler:12}
    Revd John Boggs + John F. Cisko was apionted judges of the election 11/4/25 \\
         
    \z

The distribution of relevant forms is presented in \tabref{tab:3 singler:3}.

\begin {table}
\caption{Past-tense forms of \textit{be} by subject type.}
\begin{tabular}{lrrrr}
\lsptoprule
Genre & \multicolumn{2}{c}{\textit{They}} & \multicolumn{2}{c}{NP}\\\cmidrule(lr){2-3}\cmidrule(lr){4-5}
& \textit{were} & \textit{was} & \textit{were} & \textit{was}\\
\midrule
Testimony & 2 & 0 & 0 & 6\\
Fracas & 0 & 0 & 3 & 4\\
Board minutes & 2 & 0 & 0 & 4\\
  Total & 4 & 0 & 3 & 14\\
\lspbottomrule
\end{tabular}

\label{tab:3 singler:3}
\end{table}
 

The only occurrences of 3pl non-past \textit{be} occur in the minutes, not in the trial testimony nor in the account of the fracas. Here, too, leveling can occur \REF{ex:singler:13} when the subject is a full NP, but \tabref{tab:4 singler:4} shows that this happens much less frequently in non-past when compared with past.\footnote{{In addition, there is one occurrence of unleveling, i.e. hypercorrection: :} {we the cmte have taken the above case into consideration + are of apionion that} {\textbf{A[]} \textbf{G[]} \textbf{are}} {guilty of a breach of decipline by srikin her sister AH + after being parted threatening her again to fight her + is accordingly disowned. 15/4/29}}

\ea%13
\label{ex:singler:13}
 ... whereby the peopel is detained longer then they otherwise would be 29/11/25 \\
\z

\begin{table}
\caption{Nonpast-tense forms of \textit{be} by subject type.\label{tab:4 singler:4}}
\begin{tabular}{lrrrr}
\lsptoprule
Genre & \multicolumn{2}{c}{ \textit{They}} & \multicolumn{2}{c}{NP}\\\cmidrule(lr){2-3}\cmidrule(lr){4-5}
& \textit{are} & \textit{is} & \textit{are} & \textit{is}\\
\midrule
Board minutes & 3 & 0 & 8 & 4\\
\lspbottomrule
\end{tabular}
\end{table}

\begin{sloppypar}
On the basis of letters written in the 1860’s to the Freedmen’s Bureau, Montgomery et al. (1993) identify the Subject Type Constraint as a characteristic of mid-nineteenth-century African American English. Further, they identify its source as Northern British English. (Along with the Proximate Subject Constraint, it forms the Northern Subject Rule \citep{Murray1873}.) At the same time, they present data from the McCullough Family Letters, written by Scotch-Irish immigrants to the South Carolina Up County. “The McCullough letters, written by three families of Scotch-Irish immigrants from Ulster and their descendants, show the subject type constraint clearly” (1993: 348). A difference between the MTB and the two data sets in Montgomery et al. is that the Subject Type Constraint in the MTB is limited to forms of \textit{be}. 
\end{sloppypar}

The Scots Irish -- whom Montgomery \textit{et al.} generally designate Ulster Scots, because they were Scots who immigrated first to Ulster and then to America -- were a major presence in Philadelphia in the late eighteenth and early nineteenth century. They had a long period of immigration from 1710 to 1776, followed by an intense second period from 1780 to 1820. \citet{Ridner2017} identifies them as 25\% of Philadelphia’s population in 1790. Whether the members of Bethel Church displayed the Subject Type Constraint as an African American feature of their speech, as a Philadelphia feature owing to Scots Irish influence, or for both reasons is not readily determinable. 

\subsubsection{Irregular verb forms} \label{sec:singler 3.3.3}

While the irregular verb forms that obtain are ordinarily those found in standard English, there are exceptions, of various types:



\ea%14
  Past for participle\\
    \label{ex:singler:14}
He said that he had been spoke to about visiting the house so much 13/10/30 
    \z

 

\ea%15
  Participle for preterit\\
    \label{ex:singler:15}
Necesaty required it + he done it. 8/4/29
    \z

 

\ea%16
  Bare root past form\\
    \label{ex:singler:16}
  he told me that was his buisiness and give me [no] further satisfaction 18/12/22    
    \z

         


\ea%17
  Regularized past form\\
    \label{ex:singler:17}
I told him that he was pinched, however I blowed out the candle. 6/10/23
    \z


\subsubsection{\textit{A}{}-prefixing} \label{sec:singler 3.3.4}

There are six occurrences of a-prefixing in the MTB. In (\ref{ex:singler:18}), I present the three that occur in the account of the Bethel Wesley fracas.

\ea%18
    \label{ex:singler:18}
    
          he mentioned that he was a going around to the Wesley Church [i.e. was getting ready to go]… As Rev Rich Allen drew near the church. William Perkins, James Bird, Simon Murray \& Tobias Sipple came a running a cross the \_\_\_\_ to oppose the people going into the church… While Perkins and his party was a contending at the door, Rev Rich Allen was conveyed in by some of the trustees of the Wesley Church. 18/8/22 
          
\z

In the Yale Grammatical Diversity Project’s description of \textit{a}{}-prefixing in modern American vernaculars (most saliently Appalachian English) \citep{Matyiku2011}, the phenomenon is identified as occurring in narratives, arguably to convey "immediacy or dramatic vividness” \citep{Feagin1979}. While this characterization seems generally to hold for the occurrences in the MTB, the first of the examples given in (\ref{ex:singler:18}) would not seem to involve either immediacy or dramatic vividness.

In modern vernaculars, there are phonological restrictions in that \textit{a}{}-prefixing only occurs when the verb is consonant-initial and carries stress on the initial syllable (though \citealt{Montgomery2009} presents a limited number of exceptions to the latter constraint). As can be seen, the third example in (\ref{ex:singler:18}) involves a verb with non-initial stress.

\subsubsection{Orthographic variation}

As attested by the excerpts from the MTB presented above, there is a fair amount of nonstandard spelling. Much of this can simply be attributed to the scribe’s lack of formal education. A noticeable instance where nonstandard spelling suggests a pronunciation pattern involves the sequence of what I assume to be /Ɛ/ followed by /r/: \textit{a thourreth repear} (15/6/24), \textit{his shear} (21/5/24), \textit{a pear of Sissors} (8/4/29), \textit{up stears} (8/4/29). Less often, the vowel is spelled <a>, as in \textit{up stars} (n.d.) and \textit{repard} (15/6/24).  One frequent site of departure from standard spelling involves unstressed vowels in polysyllabic words, e.g. \textit{propersition} (15/6/24), \textit{simerler} (12/8/29). The Philadelphia dialect at the time, i.e. as spoken by whites, was rhotic. However, the instances introduced above of \textit{propersition} and \textit{simerler} suggest that African American speech in Philadelphia was like African American speech elsewhere in being non-rhotic.

\subsection{Summary}

A crucial question is whether the trial testimony data in the MTB can be considered vernacular. Given certain formal attributes in the minutes proper, e.g. the use of the subjunctive in the framing of motions, to assert that the trial testimony data displays vernacular features requires that there be distributional differences between the trial testimony data and the minutes.  Before considering these, I wish to call attention to unmistakably vernacular features that are all but absent. For example, there are no occurrences anywhere in the MTB of \textit{ain’t}. Indeed, there is but a lone instance of a negative contraction. It occurs in a question in the trial testimony data:

\ea%19
    \label{ex:singler:19}
  She said, why don’t you come and see? 12/12/23    
    \z

Similarly, there is a lone occurrence of negative concord \REF{ex:singler:20} in the trial testimony data even though there are several instances where it might have been expected, as exemplified in (21--23).\footnote{{In (\ref{ex:singler:4}) above, there are two examples of the construction “Witness say that she Know H. H[] but that she knows nothing nor never saw anything Disrespectfull of her.” I have not considered the use of} {\textit{never}} {to constitute negative concord.}}

\ea%20
\label{ex:singler:20}
He ... acknowlidged that he had acted improper in visiting the house so much + was sorry for it + whould not do so no more except on cases of real necesaty. 13/10/30   
\z

         

\ea%21
    \label{ex:singler:21}
  She says that A. M[] never offerd anything disrespectfull to her in his life. 8/9/25   
    \z

          

\ea%22
    \label{ex:singler:22}
  She never saw any thing in or about her that looked like her being in a state of Pregency to her Knowledge. 8/4/29   
    \z

          

\ea%23
    \label{ex:singler:23}
   he would come in the morning + go into her bedroom without asking anybody 13/10/30  
    \z

However, there are other features that show up and only do so in trial testimony data (or, in one case, also in the account of the Bethel-Wesley melee) Thus, the only instances where possession is ever conveyed by word order alone are in the trial testimony. The same is true of nonstandard instances of irregular verbs. Similarly, \textit{a-}prefixing only occurs in the trial testimony plus the account of the Bethel-Wesley fracas. The Subject Type Constraint for 3pl subjects of past-tense copulas is distinct from the other features. It applies to trial testimony data, but it also applies in the other data as well, suggesting that this delineation of the Subject Type Constraint is not just vernacular.

As Montgomery et al.’s examination of the McCullough Family Letters attests, attention to the Subject Type Constraint in nineteenth-century American English is not African American alone. Neither are the nonstandard irregular verb forms illustrated in \sectref{sec:singler 3.3.3} nor the use of \textit{a}{}-prefixing in \sectref{sec:singler 3.3.4}. On the surface at least, the lone feature that can be identified in the MTB as uniquely African American (among American dialects) is the expression of possession by word order alone.          

\section{The Cedar neighborhood} \label{section4}

Because Philadelphia in the early Republic played a central role in the formation of African American institutions, historians have examined it in depth. One such study is Emma Lapsansky’s 1975 dissertation, \textit{South Street Philadelphia, 1764--1854: “A haven for those low in the world.”} It is a study of the South Street Corridor, which Lapsansky also terms “the Cedar neighborhood,” Cedar Street being the original name for South Street.\footnote{{Cedar Street is the original name for South Street. Lapsansky delineates the Cedar neighborhood (or South Street Corridor) as “Lombard Street at the north, Bainbridge Street at the south, the Delaware River at the east and Broad Street at the west” (xxvi).} {\citet[2]{Dorman2009} reports that, “after the establishment of the Bethel and St. Thomas churches, families flocked to the southern neighborhoods of the city, creating a vital center for black life. By 1820, 75 percent of black households in Philadelphia lived in the city’s southern neighborhoods”. (It is not clear how much beyond the Cedar neighborhood is included with it in the phrase “southern neighborhoods.”) Philadelphia’s population was overwhelmingly white, and these neighborhoods were still majority-white.}} The neighborhood “developed for urban occupation between about 1760 and 1840” (\citealt[xxvi]{Lapsansky1975}). Lapsansky draws extensively on period directories, maps, and the like. She presents evidence in exhaustive detail, thereby providing a rich and thorough description of the neighborhood’s residents and institutions. Above I cited Lapsansky’s identification of Mother Bethel Church as a major presence in the neighborhood, one that attracted African Americans to it.

In a work based on her dissertation, Lapsansky comments:

\begin{quote}
The black community of Philadelphia was spatially stable… In the 1790s, Bethel African Methodist Episcopal Church had been established at 6th and Lombard Streets. Since then Afro-Americans had increasingly anchored their “turf,” setting up a number of institutions -- schools, insurance companies, masonic lodges, and several additional churches within a few blocks of Bethel. As early as 1811, a black neighborhood was identifiable at the southern edge of the city, near Bethel church… Though Afro-Americans and their institutions were to be found in all parts of the city and its suburbs, there was, then, an early, clearly defined intellectual, social, and economic focus for the Negro community at the southern edge of the city (1980: 58).
\end{quote}

In assessing the Cedar neighborhood in the period under study, Lapsansky cautions:

\begin{quote}
The development of the South Street Corridor points out the pitfalls inherent in generalizing twentieth-century interpretations to nineteenth century situations. Not only do the terms “ghetto“ and “slum“ have no meaning in the context of the 1830s Cedar neighborhood, but our modern view of what constitutes a “black neighborhood” -- an area in which no white faces are to be seen -- must be adjusted to reflect the nineteenth century realities (277–278).
\end{quote}

Lapsansky presents a picture of residential -- and, to a lesser extent, occupational -- mixing. She identifies Philadelphia in the late eighteenth and early nineteenth century as a “walking city.” The absence of public transportation required people, whether management or labor, to live near their workingplace, which meant living near each other. According to Lapsansky:

\begin{quote}
There was … for Philadelphians, a period of about forty years -- between 1790 and 1830 -- when racial violence was at a minimum, driven underground by acute labor shortages and the absorption of class issues into traditional politics. In this period, ethnic hostility took the more subtle forms of exclusion from churches, clubs, and cemeteries, and of greater numbers of arrests and convictions among the Irish and blacks… Moreover, as racial and economic segregation was impossible in the early Philadelphia walking city – employers and employees simply had to live near their work -- and hence near each other -- little could be done in the way of using physical distance as a deterrent to either crime or other forms of intergroup antagonism.\footnote{\citet[692]{Stewart1999} extends Lapsansky’s description of this period in Philadelphia to the North more generally: “[F]rom 1790 until around 1830, society in the North, though suffused with prejudice, nevertheless fostered a surprisingly open premodern struggle over claims of ‘respectability’ and citizenship put forward by many social groups, and particularly by free African Americans."}
\end{quote}

It was during this period when racial hostility was latent, that the foundations of the Cedar neighborhood were laid; foundations that as early as 1790 presaged its future directions. The Cedar neighborhood, as evidenced by the 1790 census, consisted of four distinctly identifiable elements: a substantial native\-born landlord class, and an Irish, Black and white English or other European working class.\footnote{{Anti-black riots beset Philadelphia in the 1830s and 1840s, subsequent to the current focus.}}

Members of the four “distinctly identifiable elements” were mixed together spatially. Describing a block of Gaskill Street in 1811 with attention to each householder, Lapsansky notes, “This block, not highly stratified either occupationally, racially or ethnically, far enough from the river front not to be overly influenced by its demands, is not atypical of the social mix of the early nineteenth century Cedar neighborhood” (131). Moving forward a decade, she writes, “It was as rare in 1820 as in 1811 to find more than three or four families of similar background as near-neighbors” (227).

The mix of ethnicities and race down to the level of the individual city block in the Cedar neighborhood also held -- to some extent -- for class, Lapsansky noting that there was “little social or spatial distance between the upper and lower classes here -- few of the resident landlords were more than moderately well-off” (227). (In this, the Cedar neighborhood differed from the city center of Philadelphia, where prosperous whites lived on the major thoroughfares while the working class and the marginalized lived on backstreets.)

\begin{quote}As with the Irish and native-American sectors of the Cedar neighborhood, the occupation of a black resident was a poor predictor of his wealth or location. The frugal black hairdresser or tailor whose comfortable brick house might stand next to the Irish butcher's frame house, the English carpenter's clapboard one, or the poor black laborer's rented shanty, was a frequent occurrence of this neighborhood for many years (124).\end{quote}

Moreover, “one striking aspect of early nineteenth-century Philadelphia was the rarity with which race, ethnicity -- or even occupation -- was an accurate predictor of wealth” (119--120).

Lapsansky makes the point that, even as the Cedar neighborhood came to be associated with the city’s two most stigmatized groups, the Irish and African Americans, many of the white, non-Irish homeowners remained. Others viewed the two groups with “distaste and fear,” “yet distaste did not result in the rejection of the Irish or of the blacks” (112, 111).

Lapsansky’s work establishes the Cedar neighborhood as “polyethnic” (xxvi), down to the level of individual city blocks. Still, when Lapsansky is interviewed for the 1998 PBS series on “Africans in America,” she refrains from describing Philadelphia neighborhoods at the time as integrated. She acknowledges that working-class Germans, Irish, and African Americans in early nineteenth-cen\-tu\-ry Philadelphia “live roughly on the same blocks,” but she declined to consider the neighborhoods “integrated” because the different groups 

\begin{quote}would have had different institutions, [and] they would not have necessarily spent their leisure time in each other’s company. They would have worked together, they would have been mostly polite to each other on the streets, but when it came time to celebrate, when it came time to go to church, to marry, et cetera, people went off to their own ethnic foxholes. 
\end{quote}

Except, she adds, 
 \begin{quote}for public gatherings. In the early 19th century, it's clear that public gatherings were what one might call democratic. [For particular holidays,] there's a great deal of action ~in the public space and that action takes place with lots of different kinds of people sharing the public space
(Lapsansky, quoted in \citealt{Jones1998}).\end{quote}

Above I presented evidence pertaining to the provenance of Philadelphia’s African Americans in the period. The one adjustment to be added in light of Lapsansky’s discussion of the Cedar neighborhood is that “by the early nineteenth century, native Pennsylvania blacks, freed by the gradual emancipation law adopted by the legislature in 1780, increased the numbers who called this neighborhood home” (xxvi–xxvii). In other words, the neighborhood that is identified most strongly with African Americans (even though they remained a minority there) had a locally born element, one might even say a locally born foundation.

\section{Bethel Church, {Samaná}, and the divergence hypothesis} \label{section5}
\subsection{Conflicting assessments of {Samaná’s} history}

In 1987 Shana Poplack and David Sankoff published “The Philadelphia story in the Spanish Caribbean,” reporting on fieldwork that they had carried out in the African American enclave on the Samaná peninsula of the Dominican Republic. The original settlers had gone there in 1824--1825 as part of the Haitian government’s campaign to attract African Americans to Hispaniola. (From 1822 to 1844, the Haitian government controlled Hispaniola in its entirety.) Poplack \& Sankoff’s designation of the language of the Samaná residents as a “Philadelphia story” reflects the provenance of the original settlers. As \citet{PoplackTagliamonte2001} state: “Philadelphia’s African Methodist [Episcopal] church was apparently the main driving force in the immigration movement \citep[243]{Nash1988}, and Philadelphia appears to have been the major supplier of immigrants” (\ref{ex:singler:12}). 

Poplack and her colleagues argue that the language of the modern descendants of the original Samaná settlers provides a window into the African American Vernacular English (AAVE) of the time of emigration from the United States. Because the Samaná data is closer to white American vernaculars than is true of AAVE today, Poplack and her colleagues hold that the Samaná data provides evidence that AAVE has diverged from other American vernaculars over time.\footnote{I am unable to assess the impact of other Englishes on Samaná English (and it takes me too far afield from the Philadelphia focus of this article), but I think it important to acknowledge their presence. Thus, Hoetink’s investigation of Samaná church records from the 1870's turns up such frequent intermarriage with Turks Islanders that he speculates that, for the Americans in Samaná “[i]n 1870 Samaná was -- together with the United States and the Turks Islands maybe -- the world” \citep[21]{Hoetink1962}. There were missionaries from England and also Jamaican ones \citep[14]{Hoetink1962}. \citet{Vigo1982} and \citet{Davis2007} report that “the Americans” hired teachers from Jamaica and Turks Islands for their schools. \citet[224]{Mann-Hamilton2013} speaks of a “constant influx of other English-speaking emigrants from neighboring Caribbean islands,” and Vigo of “an extensive immigration of laborers from the British West Indies” to work on the building of a railroad and on sugar plantations in the late nineteenth and early twentieth centuries (\citeyear[6]{Vigo1987}). Corroboration of the West Indian presence comes from \citet{Parsons1928} and \citet{MillerKrieger1928}.  In the words of Davis, “Samaná es un microcosmos y une mezcolanza de Afronortamérica, el Anglocaribe, el Francocaribe, y, desde luego, el Hispanocaribe” \citep[166]{Davis1980}.} As such, they see the Samaná data as speaking directly to the divergence/convergence debate that arose in the 1980’s (\citealt{LabovHarris1986, FasoldRickford1987}).

The received view regarding AAVE is that it developed on plantations in the South; it seems clear that this is a view Poplack and her colleagues accept (as do I). Thus, the Philadelphia, i.e. northern, tie proves problematic for them.\footnote{\citet[37]{PoplackTagliamonte2001} make reference to the “misleading title” of \citet{PoplackSankoff1987}, i.e. “the Philadelphia story.”} To justify their statement that Samaná English is ancestral to modern AAVE, \citet{PoplackTagliamonte2001} set out to establish that “the demographic profile of the remaining input settlers would have been essentially that of the general African American English-speaking population of the time” (p. 37). In contrast, in my own work, I argue that the early Samaná settlers came primarily from the North and consequently that their language cannot be considered antecedent to modern AAVE (cf. \citealt[341--342]{Hannah1997}).

The works in question (specifically \citealt{PoplackTagliamonte2001} and \citealt{Singler2007part2}) present two key areas of difference:

\begin{itemize}
\item The provenance of the African American population of Philadelphia, hence of those who would have been likely to emigrate from there.
 
\item Immigration to Samaná vis-à-vis immigration to elsewhere on Hispaniola, with special reference to which immigrants remained and which returned to the US.
\end{itemize}

My assessment of the first of these points is presented in 1.2 above. For a detailed critique of \citet{PoplackTagliamonte2001} in this regard, see \citet{Singler2007part2}. While the general immigration venture sought to bring 6000 African Americans to Hispaniola to form the nucleus of a middle class, there was a pressing strategic need to establish a presence in the contested location of the Samaná peninsula and the bay it commanded. \citet[224–225]{Mann-Hamilton2013} identifies the Bay of Samaná as the point of entry for French forces “intent on stifling the events of the Haitian Revolution”; cf. \citealt{Keim1870}:116–117; \citealt{Madiou1988} [1848]:357.

To attract African Americans to Hispaniola and to effect this immigration, President Jean-Pierre Boyer had dispatched the diplomat Jonathas Granville to the American North, equipping him with 50,000 pounds of coffee to be sold to defray the immigrants’ travel expenses. Granville’s arrival in the American North generated excitement, and organizations formed in northern cities in support of the campaign.  He quickly made his way to Philadelphia, where he met with Richard Allen, who took the reins as president of the Haitian Emigration Society there. “Allen led the emigration efforts, convening a meeting of black community leaders at his house in late June 1824 and then calling a mass meeting at Bethel” \citep[242]{Nash1988}. The early immigration activity in Philadelphia seemed poised to send people to Samaná in particular. This is reflected in the MTB.  When one of the church’s leaders, Thomas Robertson, made plans to immigrate to Samaná, Allen called a special meeting of the “ministers, preachers, exhorters, trustees, and leaders ... [and] ... after meture deliberation on the subject it was on motion Resolved that Brother Thomas Robersen be set apart by holy ??orders?? for Samanah in Hayti” (3/11/24).  

 
Further evidence of Allen’s role in immigration to Haiti is Nash’s report that  “sixty Philadelphia blacks ... gathered their possessions, bade farewell to friends and relatives, and sailed from the Delaware wharves aboard the \textit{Charlotte Corday}...{.} Hundreds more, encouraged by optimistic reports from the first emigrants and the enthusiasm of Richard Allen, emigrated later in 1824... According to Allen, most of [the emigrants to Haiti from Philadelphia] ...  were Methodists” \citep[244]{Nash1988}.
 

Despite the promising start to the grand immigration scheme, it was, Samaná excepted, by and large a failure. As word of the Haitian venture spread, people wishing to immigrate to Haiti organized in a larger range of cities. With the increase in the range of source cities came a marked deterioration in the selection process. There was a decline in the quality of applicants or – at least – a decline in the scrutiny to which applicants were subjected, and there was a decline in their level of preparedness. According to Jackson, problems arose “in part because Granville had so little control over the recruiters that assisted him. Furthermore, many of them desired only to rid the country of those blacks whom they deemed undesirable” (115). He continues: “Many well-meaning recruiters were not of the caliber of the Reverend Richard Allen … What had been intended to be selective emigration in reality became en masse emigration which Jonathas Granville was powerless to manage” (\citealt{Jackson1976}:115, 116). 
 

 
Thus, there was a distinction between the early Samaná-bound immigrants and those who followed who went elsewhere. As the nineteenth-century Haitian historian Thomas Madiou observes: “Quant aux convois qui furent envoyés à Samana, ils étaient mieux composés que les autres ...” (1988 [1848]:425). The distinction between those sent to Samaná and those sent elsewhere was intensified on Hispaniola itself. While the Haitian government had envisioned that most of the immigrants would engage in farming, apart from those who went to Samaná, most flocked instead to cities, where they competed with the existing labor force and did so at a linguistic disadvantage. They were absorbed into existing Haitian society poorly if at all.  A great number of immigrants returned to the US, perhaps as many as a third.  In contrast, Samaná was “...  una comunidad contenta en la cual hubo muy pocos de sus miembros que ambicionaron retornar a los Estados Unidos” (\citealt[107]{PenzoDevers1999}; see also \citealt[230]{Godbout1987}). The distinctive character of the Samaná settlement from the rest of Hispaniola is a point that recent scholarship reinforces, e.g. \citealt{Mann-Hamilton2013, Mann-Hamilton2016}, \citealt{Fanning2015}, \citealt{Mongey2019}. “As Samana was accessible only by boat from the rest of the island, the Americans were not assimilated, and retained the English language, African American cuisine ..., religious affiliation to Protestantism (the Wesleyans provided a minister when the AME church could not), and their identity as Americans” \citep[111--112]{Fanning2015}. Further, \citet{Mann-Hamilton2013, Mann-Hamilton2016} lays stress on the role of Bethel church members in the establishment of the Samaná enclave. Certainly, one characteristic of Bethel Christianity that manifested itself in Samaná was the emphasis on education, illustrated by the establishment of church-run English-language schools. 
 

To the extent that the significance of Philadelphia cannot be denied, Poplack \& Tagliamonte seek to downplay the degree of interracial interaction to be found there in the years leading up to the immigration to Samaná. They reproduce a map of the “Residential pattern of black households in Philadelphia, 1820” from \citet[168]{Nash1988} (p. 14) and comment, “Residential segregation (and the attendant decrease in language variety contact) had clearly already begun’ (13).  However, this statement is at direct odds with Nash’s own comment regarding the map that “no segregated black community emerged... . Neighborhoods remained mixed by race and occupation” (169). And, with specific reference to a street that was only three blocks long, Nash states: “The twenty-four black families on Gaskill Street in 1816 lived in a neighborhood that formed a nearly perfect cross-section of Philadelphia’s industrious middle and lower classes. On a daily basis, black families encountered white neighbors (who still outnumbered them by two to one)” (169--170).\footnote{{Nash’s comment directly parallels one by Lapskansky, cited in \sectref{sec:section3}, regarding Gaskill Street.}}
 

In terms of linguistic data, that which appears in \citet{PoplackSankoff1987} and subsequent works seems qualitatively comparable to the trial data from the MTB. Like the MTB, it displays the Subject Type Constraint regarding verbal \textit{–s} (\citealt{PoplackTagliamonte1989}). A crucial difference, however, is that the Samaná English data displays variable copula absence (\citealt{PoplackSankoff1987}). 
 

\subsection{The divergence hypothesis without {Samaná}}

If the membership of Bethel AME Church was similar to the African American population of Philadelphia at the time more generally and if \citegen{Nash1988} table provides a valid indication of the provenance of the larger population, then at the very least we can see that the population, beyond consisting of free African Americans, is largely northern in orientation, with most African Americans in Philadelphia having been born either elsewhere in the North itself or in the Upper South, specifically in the part of the Upper South where large plantations were not economically sustainable. Within Philadelphia, specifically in the area where Bethel Church stood and where many of the Church’s congregants lived, the early-nineteenth-century neighborhood was pervasively mixed in terms of race/ethnicity and social class. It is not surprising, then, that the vernacular elements of the language of the MTB seem more generic than specifically African American. This statement assumes that the features associated with modern AAVE were present in the speech of African Americans elsewhere in this time period, specifically in the South, especially the Lower South. In \citet{Singler2007part1, Singler2015}, I brought evidence from Liberian Settler English to bear on the question. In the nineteenth century, 16,000 African Americans immigrated to Liberia under the auspices of the African Colonization Society, primarily in the fifty-year period from 1822 to 1872. There is extensive documentation as to the provenance of the Liberian Settlers (see \citealt{Singler1989}), and more than 90\% came from slave states. Moreover, in the years before the American Civil War brought an end of slavery, a majority of those who went to Liberia were enslaved people for whom immigration to Liberia was a condition for their manumission. 

When it comes to the non-use of standard features such as the copula, Samaná English and Liberian Settler English both pattern with AAVE, i.e. in displaying the features’ non-use, but so too do other vernacular varieties unconnected to AAVE (\citealt{Chambers2004}, \citealt{SzmrecsanyiKortmann2009}).  Arguably more revealing, \tabref{tab4:singler:4} presents the five \textbf{overt} non-quantitative nonstandard features that \citet{Myhill1995} posits as potential innovations in AAVE (cf. \citealt{Singler2007part2,Singler2015} ). 


\begin{table}
\begin{tabular}{lcccc}
\lsptoprule
& MTB (Bethel) & Samaná English & LSE & AAVE\\
\midrule
use of \textit{ain’t} for \textit{didn’t} &\ding{55}& \ding{51} & \ding{51} & \ding{51}\\
\textit{be done} &\ding{55}&\ding{55}& \ding{51} & \ding{51}\\
semiauxiliary \textit{come} &\ding{55}&\ding{55}& \ding{51} & \ding{51}\\
\textit{steady}  &\ding{55}&\ding{55}& \ding{51} & \ding{51}\\
stressed \textit{been} &\ding{55}&\ding{55}& \ding{51} & \ding{51}\\
\lspbottomrule
\end{tabular}
\caption{Non-standard features: Overt features not found in standard English. Sources of the Samaná English information: \citet{HoweWalker2000} for the use of \textit{ain’t} for \textit{didn’t}, and Shana Poplack (p.c.) for the rest.\label{tab4:singler:4}}
\end{table}

As can be seen, while only one of the features shows up in Poplack \& Sankoff’s Samaná English corpus and none are present in the MTB, all five are used in Liberian Settler English. The presence of the features in Liberian Settler English (when taken in tandem with their absence in English-lexifier varieties elsewhere in West Africa) indicates that they are antebellum features of AAVE that the Liberian Settlers took with them when they left the American South. Thus, their absence from the MTB and from Poplack \& Sankoff’s Samaná English corpus is not evidence of their absence from nineteenth-century AAVE. Further, the presence or absence of particular features is simply more basic, hence more telling, than comparative constraint-ranking in quantitative assessments. This is not to say that the divergence hypothesis for AAVE has been falsified, only that the use of Samaná data has been crucially undermined. 

The non-applicability of Samaná data aside, there are still limitations as to how far one can take claims of divergence or convergence. To begin with, as Spears notes in the 1987 panel on divergence vs. convergence, 

\begin{quote}
It's clear that it [divergence] can be interpreted in several ways...{.} For one thing, are we talking about global divergence, that is, between language varieties as wholes, or simply divergence with respect to certain features of grammar? Surely, neither Labov nor anyone else is claiming that there is divergence affecting white and black grammars as wholes, that is, with respect to all of their features (Spears in \citealt[50]{FasoldRickford1987})\footnote{Similarly, neither the Neo-Anglicist Origins Hypothesis (\citealt{VanHerk2015}) nor the Creole Origins Hypothesis \citep{rickford2015creoleorigins} has proven to be a tenable account in its “pure,” i.e. strong, form as to the origin of AAVE. Rather, a consensus has emerged (including Van Herk and Rickford) that “AAVE ... has been composed of components drawn from standard English, dialectal English, innovative developments, and creole, to varying proportions, and it is best described by carefully weighing moderate assessments that recognize the complexity of its contexts and its linguistic evolution (\citealt{Winford1997originssocio, Winford1998originsfeatures, winford2015origins})" (\citealt[136]{schneider2015documenting})}.\end{quote}

Beyond this, there are reasons for caution. The distribution of the AUX \textit{be done} and stressed \textit{been} in \tabref{tab4:singler:4} provides Liberian evidence attesting to their stature as examples of features of long standing in AAVE. As such, the features stand against divergence. However, while the features may be of long duration, they have expanded their domain of usage in AAVE in ways that they have not in Liberian Settler English. Thus, while (\ref{ex:singler:24}) is grammatical in Liberian Settler English and AAVE alike, (\ref{ex:singler:25}) is grammatical in AAVE but not in Liberian Standard English.



\ea%24
Liberian Settler English\\
    \label{ex:singler:24}
     I BÍN receiving certificates. \\
    ‘I have been receiving certificates [for outstanding churchwork] for years now.’
\ex
    \label{ex:singler:25}
    Speaker A: You gonna quit?  \\
    Speaker B: *I BÍN quit. (\citealt[136]{Labov1998})    \\
     
\z

Strictly speaking, in that other American vernaculars lack \textit{be done} and stressed \textit{been}, the further elaboration of domains where these features obtain cannot be counted as divergence. However, it also shows the limits of invoking convergence as an explanation for the differences between AAVE and other vernaculars.

\section{After 1830} \label{section6}

Early in the nineteenth century, African Americans in Philadelphia were in daily contact with other Philadelphians. Indeed, the city’s “African American” neighborhoods were majority white. The evidence presented here suggests that the African American English of the day was, for the most part, similar to the vernacular of white Philadelphians. If speech of African Americans in Philadelphia was not distinct from that of other Philadelphians in the first three decades of the nineteenth century, the question arises as to when that distinctiveness emerged. Was it not until the Great Migration of the twentieth century? I think that this would be unlikely. To begin with, a succession of riots by whites and increasing employment discrimination against Blacks beginning in the 1830’s led to greater segregation, though not all at once. At first, this was because Black and Irish residents alike were targeted. With regard to a succession of riots from 1834 to 1849, Lapsansky observes: “The anti-black riot path and destruction were not randomly spread throughout the neighborhood which the blacks and Irish shared but were concentrated in the areas of greatest black- property-ownership" (1975:225). However, riots against the two groups gave way fully to conflict between the two groups. Clearly, the period from the 1830’s to the onset of the Great Migration in the early twentieth century requires further study.



\section*{Acknowledgements}

Throughout his distinguished career, Don Winford has continually examined and re-examined key questions in sociolinguistics, creole studies, and African American English studies. Part of his modus operandi in doing this has been by seeking and finding new sources of data. He then brings his findings from these fresh sources to bear on the questions at hand. \citet{winford2017} is an exemplary case in point. It is in the spirit of reporting on a new source of data and analyzing it that -- in frank admiration of Don’s ongoing scholarship and the signal contributions that he has made -- I present this study. 

The late Steve Lynch provided me with a sense of the special character of Philadelphia. I thank the librarians at the Historical Society of Pennsylvania for their assistance. I owe a profound debt to Bethel AME Church for their willingness to make their records available to scholars. One way that non-linguists might look at this article is to say that I am focusing on church members’ bad behavior, with special reference to the “bad English” being used to describe it. I don’t see it that way at all. It is because Bethel Church made its records available to scholars that we can appreciate the remarkably comprehensive range of services that the church provided for its members, not least being its role as arbiter and adjudicator. 

I am grateful to Mercedes Duff and Cara Shousterman for their assistance in transcribing the Mother Bethel Church data and for their observations about it.

\section*{Abbreviations}
\begin{tabularx}{\textwidth}{@{}lQ@{}}
\textsc{AAVE} & African American Vernacular English  \\
\textsc{AME} & African Methodist Episcopal\\
\textsc{MTB} & Minute and Trial Book \\
\end{tabularx}

{\sloppy\printbibliography[heading=subbibliography,notkeyword=this]}
\end{document}
