\documentclass[output=paper, colorlinks,citecolor=brown]{langscibook} 
\ChapterDOI{10.5281/zenodo.6979303}
\author{Bettina Migge\affiliation{University of Dublin} and Shelome Gooden\affiliation{University of Pittsburgh}
}

\title{Social and structural aspects of language contact and change} 

\abstract{Research on language contact, sociolinguistics and language description (of less widely studied languages) are often treated as independent lines of investigation that at best intersect peripherally. One of the people who have excelled in showing that there is a close link between these lines of research and that studying linguistic phenomena always involves paying attention to all of them is Donald Winford. His tireless and proactive research, both individual and collaborative, and mentoring of a broad range of scholars, have significantly contributed to current understandings about analytical frameworks, notions and types of data required in these areas and beyond. Being very humble about his work, the significance of Don's contribution and the breadth and depth of his knowledge is often at least in part underestimated and requires far more attention than it currently has. Don’s work charts important research agendas and avenues that remain to be further explored in more detail.
The first part of this introduction provides a broad overview of Don’s career and his contribution to research. The second part briefly summarizes the research papers in this volume that were written in his honor by his colleagues and students,  offer insights into Don’s work applied to new contexts. Part three is a reflection on his life, work, teaching and mentorship from colleagues and students. The final part contains a bibliography of his work.
 }

\IfFileExists{../localcommands.tex}{
  \addbibresource{../localbibliography.bib}
  % add all extra packages you need to load to this file

\usepackage{tabularx,multicol}
\usepackage{url}
\urlstyle{same}

\usepackage{listings}
\lstset{basicstyle=\ttfamily,tabsize=2,breaklines=true}

\usepackage{langsci-basic}
\usepackage{langsci-optional}
\usepackage{langsci-lgr}
\usepackage{langsci-osl}
% \usepackage{./langsci/styles/langsci-lgr}
% \usepackage{./langsci/styles/langsci-osl}
% \usepackage{langsci-gb4e}

\usepackage{tikz}
\usetikzlibrary{patterns,calc}
\pgfdeclarepatternformonly{south east lines}{\pgfqpoint{-0pt}{-0pt}}{\pgfqpoint{3pt}{3pt}}{\pgfqpoint{3pt}{3pt}}{
    \pgfsetlinewidth{0.6pt}
    \pgfpathmoveto{\pgfqpoint{0pt}{3pt}}
    \pgfpathlineto{\pgfqpoint{3pt}{0pt}}
    \pgfpathmoveto{\pgfqpoint{.2pt}{-.2pt}}
    \pgfpathlineto{\pgfqpoint{-.2pt}{.2pt}}
    \pgfpathmoveto{\pgfqpoint{3.2pt}{2.8pt}}
    \pgfpathlineto{\pgfqpoint{2.8pt}{3.2pt}}
    \pgfusepath{stroke}}
    
\usepackage{stmaryrd}
\usepackage{wasysym}
\usepackage{multirow}
\usepackage{caption}
\usepackage{subcaption}
\usepackage{mathrsfs}
\usepackage{qtree}

\usepackage{linguex}


  %pminos do not split footnotes
% \interfootnotelinepenalty=10000 %Footnote in Laporte chapters has to be split SN


%\DeclareIndexNameFormat{default}{%
%\nameparts{#1}%
%\usebibmacro{index:name}%
%{\index[names]}%
%{\namepartfamily}%
%{\namepartgiveni}%
% {}% L1
% {}% L2
%{\namepartprefix}% generates spurious space L3
%{\namepartsuffix}% generates spurious space L4
%}

%  {\DeclareIndexNameFormat{default}{%
%     \usebibmacro{index:name}{\index[names]}{#1}{#3}{#5}{#7}}}

%\DeclareIndexNameFormat{default}{%
%  \usebibmacro{index:name}{\sindex[nom]}{#1}{#3}{#5}{#7}}

%\DeclareIndexNameFormat{default}{%
%  \usebibmacro{index:name}{\sindex[person]}{#1}{#3}{#5}{#7}}
%\DeclareIndexNameFormat{default}{%
%\nameparts{#1} \usebibmacro{index:name}{\sindex[person]]}{\namepartfamily}{‌​\namepartgiven}{\nam‌​epartprefix}{\namepa‌​rtsuffix}}

%\newcommand{\smiley}{:)}

%\renewbibmacro*{index:name}[5]{%
%\usebibmacro{index:entry}{#1}%
%{\iffieldundef{usera}{}{\thefield{usera}\actualoperator}\mkbibindexname{#2}{#3}{#4}{#5}}}

% \newcommand{\noop}[1]{}

%remove for final
%\overfullrule=1mm

\newcommand{\tobi}[2]}}
\renewcommand{\S}[1]{\tobi{#1}{\textsc{*}}}

% this volume references
% puts: [this volume]
% already defined: \citetv
%\newcommand{\citepv}[1]{(\citeauthor{#1} \citeyear*{#1} [this volume])}
\newcommand{\citealtv}[1]{\citeauthor{#1} \citeyear*{#1} [this volume]}

%parentheses around example number
\newcommand{\pref}[1]{(\ref{#1})}

% in-text examples

\newcommand{\lnex}[1]{\textit{#1}} %target lang word
\newcommand{\lnlit}[1]{(lit.: `#1')} %literal reading
\newcommand{\lnlat}[1]{(#1)} % latinization
\newcommand{\lntrans}[1]{`#1'} %translation
\newcommand{\lnexl}[2]%
{\lnex{#1}{} \lnlat{#2}} % ex with latinization
\newcommand{\lnexlat}[3]{\lnex{#1}{} \lnlat{#2}{} \lntrans{#3}} % ex with latinization and tranl.

%ch01
\newcommand{\co}[1]{\mbox{\textbf{#1}}}

%ch09

\newcommand{\cyrbulg}[1]{\begin{otherlanguage*}{bulgarian}#1\end{otherlanguage*}}


%ch10
\newcommand{\nlp}{{\small NLP}}
\newcommand{\mwe}{{\small MWE}}
\newcommand{\rae}{{\small RAE}}
\newcommand{\lvc}{{\small LVC}}
\newcommand{\pos}{{\small P}o{\small S}}
%\newcommand{\todo}[1]{ \textcolor{red}{#1} }

%\renewcommand{\labelenumi}{\theenumi}
%\ainamefmt{{vv}{ll}{, ff}{, jj}} % fullname

\newcommand{\biberror}[1]{{\color{red}#1}}

\newcommand{\osenovaitem}{--~} 
  %% hyphenation points for line breaks
%% Normally, automatic hyphenation in LaTeX is very good
%% If a word is mis-hyphenated, add it to this file
%%
%% add information to TeX file before \begin{document} with:
%% %% hyphenation points for line breaks
%% Normally, automatic hyphenation in LaTeX is very good
%% If a word is mis-hyphenated, add it to this file
%%
%% add information to TeX file before \begin{document} with:
%% %% hyphenation points for line breaks
%% Normally, automatic hyphenation in LaTeX is very good
%% If a word is mis-hyphenated, add it to this file
%%
%% add information to TeX file before \begin{document} with:
%% \include{localhyphenation}
\hyphenation{
    Beck-man
    Ngu-yen
    back-chan-nel
    back-chan-nels
    mo-not-o-nous
    ste-reo-typ-i-cal
}

\hyphenation{
    Beck-man
    Ngu-yen
    back-chan-nel
    back-chan-nels
    mo-not-o-nous
    ste-reo-typ-i-cal
}

\hyphenation{
    Beck-man
    Ngu-yen
    back-chan-nel
    back-chan-nels
    mo-not-o-nous
    ste-reo-typ-i-cal
}
 
  \togglepaper[1]%%chapternumber
}{}

\begin{document}
\maketitle
 \section{Bio and research intertwined}
 
 Donald Winford or Don, as people generally call him, grew up in Trinidad in the town of San Fernando. While he has not shared much detail about his early life in Trinidad the fact that language in the Caribbean and English-lexified Creoles in particular have been a life-long research interest even after he came to live away from Trinidad suggests that the social context of language use in his home country must have left an important impression on him. Between 1956 and 1962, Don attended \textit{Presentation College} in San Fernando. Based on his excellent performance, he received the prestigious Island scholarship which brought him to London to study English at King's College, \textit{University of London} from 1965 to 1968. Following completion of his BA, he moved to \textit{York University} (UK) where he worked as a graduate assistant and completed a PhD in Linguistics under the supervision of Robert Le Page, one of the trailblazers of research of language in the Caribbean. It is at York that he honed his research interests in Caribbean Linguistics with a focus on sociolinguistics. During his four years at York University he carried out one of the first sociolinguistic studies within the Labovian quantitative sociolinguistic paradigm which charted the patterns of phonetic variation and change within two sociolinguistically distinct communities in Trinidad (Winford 1972). This work was a remarkable achievement and in a real sense is one of the first sociolinguistic analyses of vowel variation in the Caribbean. After completing his PhD, he took up a post as Lecturer at the \textit{University of the West Indies}, St Augustine campus, in Trinidad where he worked for sixteen years, becoming a Senior Lecturer in 1984. While in St. Augustine, he served on many university committees ranging from being the Chair of the Unit for Language and Linguistics, Chair of the inter-Faculty and inter-Campus committee, to being the Elected Faculty Representative on the Assessment and Promotions Committee and Chair of the Advisory Committee on Sports. 
 
During this time, his research and teaching focused on Caribbean language and sociolinguistics. For example, he contributed to discussions on the notion of the community in sociolinguistic research. He examined the social and linguistic makeup of English official Caribbean communities with a view to evaluating the then current models such as diglossia, the post-Creole continuum and the notion of community underlying quantitative sociolinguistic research (e.g. Winford 1974a, 1978a, 1983, 1985a, 1980a;b, 1985a, 1988b, 1994a). Although he also carried out research on language attitudes among teachers in Trinidad (Winford 1976) and has written on Creole culture (Winford 1980a), from early on, his main interests have always been the linguistic aspects of variation.  Focusing in particular on morphosyntactic variation, he contributed in large part to establishing research on morphsyntactic variation within the quantitative sociolinguistic paradigm (e.g. Winford 1983, 1984, 1990a, 1994b, 1996a) which until then, and even today, largely focuses on sound variation. This research also contributed to shedding light on various areas of grammar in Creole communities such as negation (Winford 1983), complementation (Winford 1985b)  and passives (Winford 1988b). His life-long passion for TMA systems was also already coming to the fore in this early work as was his interest in the development of post-colonial Englishes (Winford 1978b). 
 
At St. Augustine he also supervised two PhD students, Lise Winer and Valerie Youssef whose work focused on the development of Trinidadian Creole from a sociohistorical perspective and the acquisition of the verb phrase among Trinidadian children, respectively. Both went on to play an important role in the development of sociolinguistically-oriented research in the Caribbean. While working as a lecturer in the Caribbean, he also spent two semesters in Texas, at the \textit{University of Austin} and the \textit{University at El Paso}, respectively, where he taught linguistics and a course on English as a world language.

In 1988, Don left Trinidad for Columbus, Ohio where he has been teaching and researching at the Department of Linguistics of the Ohio State University (OSU) ever since, teaching courses on sociolinguistics and language contact. He has been returning to Trinidad on a regular basis for holidays, family visits and sabbaticals and, during his stays, has generously given his time to support the activities of the Linguistics Department in St. Augustine and help to train new generations of Caribbean linguistics scholars. Despite the cold winters in Columbus, his research expanded exponentially and within only 10 years at OSU he advanced from Assistant Professor to Professor of Linguistics! At OSU, he wrote two ground-breaking go-to monographs, his 1993 book \textit{Predication in Caribbean English Creoles} and his meticulously researched \textit{Introduction to Contact Linguistics} (Winford 2003a), edited volumes and authored a large number of important papers across a broad range of topics. The two monographs critically assess and systematize existing research and bring novel data from his own field research to bear on debates in the field; this is an important trademark of all of Don’s publications. 

While his early work is not only strongly rooted in the quantitative sociolinguistic paradigm but also aims to critically assess the issues that arise in relation to morphosyntactic research within this paradigm (Winford 1990a;b, 1994a;b, 1996a, 1997f), we start to see an incremental shift towards a focus on language contact phenomena and language description, with a focus on the verb phrase (Winford 1993b; Edwards \& Winford 1991). At the same time, Don’s work came to increasingly focus on language phenomena outside of Trinidad for which he carried out extensive field research in Belize, Guyana and Suriname. As students in OSU linguistics we greatly benefited from these databases in that we used them to learn how to do transcription work and for our first attempts at research on language use in the Caribbean within the quantitative paradigm. Work on these data helped many of us to define our research interests, navigate the nuances of good fieldwork practices and ultimately find the focus of our PhD projects.

In line with developments in the field, Don’s research saw a shift from sociolinguistics to matters of Creole genesis and Creole typology and from a focus on Caribbean Creoles proper to the Creoles of Suriname. Again, his astute evaluation of Bickerton’s Creole prototype (TMA), which was based on a large database of recordings of Sranan Tongo (Winford 1996a;d, 2000a;d, 2001a) contributed significantly to then current understandings of the structure of the contemporary Creoles of Suriname which had by then received very little concerted attention. His work was also instrumental in demystifying Creole TMA systems and brought terminology and research methods in line with those used for non-Creole languages. Since their publication, subsequent research on Creole TMA systems, and not just those in the Caribbean and Suriname, have largely adopted the framework charted in these publications, improving comparability across Creoles and with other languages. While his initial work focused on critically assessing Bickerton’s claims and bringing empirical evidence to bear on it (e.g. Winford 1990a), Don has also generously contributed data to larger databases (Winford 2013f, Winford \&Plag 2013a) and wrote comprehensive overview articles that consider TMA phenomena in Creoles from different regions and lexical affiliations (Winford 2018). Although Don’s work on TMA is most widely known, he has also made a significant contribution to a better understanding of other areas of creole grammar, namely complementation (e.g. Winford 1985b, Migge \& Winford 2013) and the copula domain, particularly property items (e.g. Winford 1990a, 1997e). 

\begin{sloppypar}
Don’s interest in Creole genesis directly emerged from his interest in the careful description of Creole grammars, since explanations were required for the many divergences that were found between Creole and lexifier grammars. The research on origins largely focused on African American English (AAE) in the 1990s and was mostly concerned with understanding the connections between AAE and Caribbean Creoles. Don again made crucial contributions to that research by comparing Caribbean copula data with AAE data (Winford 1992a;b, 1998) and by carefully assessing the sociohistorical context of the formation of African American English (Winford 1997d, 2020a). Although Don never discounted universal influences (e.g Winford 2009b) and invested much time in conceptualizing their properties and in understanding the types and nature of their impact, he never doubted that language contact and the influence of the languages of the creators of Creoles played an important role in their formation (Winford 1997a, 2003a). Together with Bettina Migge, he obtained National Science Foundation funding to research the impact of the Gbe languages on the formation of TMA systems (Winford \& Migge 2007, 2009) and \textit{fact-type} complements (Migge \& Winford 2013) in the Creoles of Suriname. This work (Winford 1972) on language contact and his careful descriptive work have led him to further advance existing theories and approaches to language contact (Winford 2003a;b;c, 2005, 2007b, 2009a;b, 2010a) and shed new light on the relationship between language contact and the emergence of Creoles (e.g. Winford 2008b). This research revealed close similarities between wider language contact phenomena and the emergence of Creoles, thus contributing to the view that creole genesis is not different in kind from other processes of contact-induced change. For over a decade now, Don has also been actively contributing to World Englishes research from a language contact perspective, comparing the mechanisms and processes of change involved in the creation of World Englishes with those that played a role in the formation of Creoles (e.g. Winford 1999a, 2008b, 2009b, 2013g, 2017b).
\end{sloppypar}

Arguably his most important service to research on Creoles has been his long term editorship of the \textit{Journal of Pidgin and Creole Languages} (JPCL), which he took over from Glenn Gilbert in 1999. In that role, he has worked tirelessly to bring diverse and innovative research to the wider linguistic community. As editor, he has played an instrumental role in fostering high quality research on Creoles and in supporting early career scholars. In recent years, in line with his own beliefs about the place of Creoles within processes of language contact, he has successively broadened the journal’s scope beyond English-based Creoles and topics on creole genesis to include other contact languages and Creoles from a wide range of lexical affiliations and contexts. During his tenure as editor, he has also not shied away from making  the journal the home for controversial and at times heated discussions among scholars. With his engagement JPCL has become a highly regarded journal within the linguistics landscape.  

Last but not least, Don has provided intellectual guidance and support for a wide range of PhD and MA theses either as main or secondary supervisor or as member of dissertation panels and examination committees at OSU, across the Caribbean and beyond. The topics cover quite a broad range and in a sense mirror the breadth of his own scholarship.  These include but are not limited to projects on vowel variation in Korean, Columbus OH English varieties and Chabacano; phonology, phonetics and prosodic aspects of reduplication in Jamaica Creole; language contact and dialect contact effects in rhotic variation in Loraine OH Spanish; sociolinguistic and grammatical variation in AAVE and Gullah, to language use among Mennonite communities and Afro-Spanish varieties; an in-depth analysis of substrate influence in the formation of the Surinamese plantation Creoles.  Almost all of these projects, in true Winford fashion, employed rigorously the qualitative and quantitative approaches of data analysis to address issues on under-researched language varieties or tackle long-standing issues from a new perspective. They also very often combined several methodological or theoretical approaches. Finally, some of these dissertation projects also challenged existing knowledge in the field, such as, the role of substrate influence in creole formation and contemporary prosodic systems, the importance of sociological consciousness in variationist analysis or the role of social factors and speakers' ideological stances in observed phonetic differences. The range of languages covered is equally impressive and highlights the fact that Don's work has a far reaching impact beyond Creoles to a wider range of contact phenomenon, i.e. AAVE, Bahamian Creole, Cantonese, Cavite Chabacano, German, Gbe, Modern Greek, Gullah, Jamaican Creole, Korean, Kotis, Mandarin, Navajo, Pennsylvania German, Spanish (Afro-Bolivian, Puerto Rico), Trinidadian Creole, Maya-Mam,  Ndyuka.


%AAVE-Gullah connection

\section {Papers in the current volume}

The planning for this volume started in 2017 when Don first announced that he was soon going to retire. Several of Don’s former PhD advisees felt that it would be timely to consider his intellectual legacy. The project had a bumpy start due to everyone’s many obligations and when things seemed to move well, Covid-19 struck making it increasingly difficult for everyone to pursue research due to the added pressures. Given restrictions of space and people’s availability, it was not possible to accommodate large numbers of papers that would be required to cover all of Don’s interests. The papers included here thus focus on Don’s core interests, the social and structural aspects of language contact and language variation and change. The papers investigate these broad themes from a variety of disciplinary and empirical perspectives, covering a range of linguistic contexts. 

One set of papers looks at the relevance of historical documents for determining the linguistic nature of early contact varieties. Peter Bakker’s chapter makes a much needed contribution to furthering current understandings about the nature of pidgins and the processes of their emergence. Drawing on novel historical language data from the French Caribbean context, he provides further evidence for his thesis that Pidgins are structurally and socially distinct from Creoles. Discussing instances of structural differences between them, he argues that socially-induced structural changes that affect Pidgins offer a good “explanation for the genesis of creoles from pidgins, even in the absence of a documented pidgin stage.” 

Singler’s analysis of the transcripts from the Minute and Trial Books of the Philadelphia Mother Bethel A.M. E. Church aims to unearth evidence about the pre-Great Migration and resettlement period varieties spoken by African Americans during the 18th century. After setting out the context and assessing their linguistic value, he demonstrates that the recorded speech contains widely-used non-standard features which are closely identified with African American English. The investigation suggests that northern African American English while distinct nevertheless shared many features with other varieties of English. 

Lise Winer’s chapter discusses the linguistic insights and issues that are posed by folk presentations of Creole speech in newspaper columns and related Creole writings found in the Caribbean region. Focusing on dialect columns from Trinidad \& Tobago, she shows that with careful analysis these give a relatively substantial picture of the nature of Trinidadian Creole from the early part of the 20th century and should be given much more attention in theorizing about the emergence and development of Creoles.

\begin{sloppypar}
A second set of contributions explores specific processes of contact-induced change that were involved in the emergence and development of these languages. Marivic Lesho investigates processes of contact-induced language change in the context of the emergence and development of varieties of Chabacano. Comparing Zamboanga and Cavite Chabacano with three of the languages that played a role in their emergence and still coexist with them – Hiligaynon, Tagalog and Spanish – she provides a first careful description and analysis of the under-researched grammatical category of \textit{possibility}. The paper demonstrates that some forms that express modality closely parallel substrate/adstrate forms. However, both Chabacano and substrate/adstrate varieties “combine Philippine and Spanish ways of expressing possibility” and possibility markers are based on the same typology which has been much influenced by Spanish. 
\end{sloppypar}

The process of paradigmatic restructuring and its relationship to jargonization is explored in Clancy Clements' chapter within the framework of \citet{KleinPerdue1992, KleinPerdue1997} on naturalistic second language acquisition. He first recasts \citegen{Good2012} concept of “jargonization bottleneck” as the stage where speakers start creating grammar using the material they have at their disposal in contexts of non-overlapping language backgrounds. These materials are usually easily detectable and frequent linguistic forms. Clements then applies this to the specific case of verbal paradigms in Indo-Portuguese (IP) Creoles. Clements shows that the model allows to systematically account for the selection and development of verbal forms and the loss of paradigmatic structure. However, it does not prevent the emergence of new structures. In the case of Korlai, he demonstrates the creation of a new verb class to accommodate Marathi verbs and the emergence of a mini-paradigm. It involves the conjunction for ‘when’ which is used to encode realis and irrealis mood. This strategy is in a way reminiscent of how Portuguese and Spanish highlight indicative and subjunctive verb forms. 

Lastly, Piero Visconte \& Sandro Sessarego examine the emergence of Afro-Puerto Rican Spanish on the basis of both sociohistorical and linguistic data in order to address current debates on the emergence of Black communities in Latin America and their language varieties. Their evidence challenges existing assumptions that posit that Afro-Puerto Rican Spanish underwent a process of radical restructuring and that some of its linguistic features hail from a Creole stage. They argue that its features are due to “non-radical processes of grammatical restructuring” that emerged in a context that remained relatively unaffected by standardizing pressures. 

\begin{sloppypar}
The final set of  papers examines how new datasets, methodological approaches and greater sensitivity to social issues can help to (re)assess persistent theoretical and empirical questions and help to open up new avenues of research. John Rickford’s chapter makes a strong case for the greater use of corpora and online corpora in particular for research on language variation and change and applied linguistics purposes. The chapter discusses six examples where the use of corpora proved instrumental for resolving issues in descriptive, theoretical and applied linguistics. The case studies range from the analysis of variation in Jamaican music and large-scale automatized analysis of vowel variation in Californian English to the detection of bias in automated voice recognition technology created by large multi-national companies. He is urging scholars of Creoles to work together to create such corpora by pooling resources. 

Durian, Reynard \& Schumacher investigate phonological variation among middle and working class African Americans and European Americans in Columbus Ohio. They explore whether African American language use is diverging from Caucasian language use as suggested by some studies in other US cities. Reviewing previous studies, they argue that accounts showing divergence have generally relied on working class data only. Their study suggests that class, race, and age are all robust factors impacting vowel variation among both groups.  There are greater differences between working-class African Americans and European Americans, particularly among younger people than middle-class populations. They suggest that greater levels of regular contact among middle-class populations is an important factor promoting convergence.
\end{sloppypar}

Changes in language attitudes are the focus of Susanne Mühleisen’s paper. Following up on previous survey-based research on language attitudes in the Caribbean that focused on eliciting views from speakers, Mühleisen argues that such attitudinal data can now be fruitfully combined with data from social media and other discussion spaces such as newspaper editorials. They provide a host of qualitative insights into how speakers and communities conceptualize their language practices. 

Michelle Kennedy \& Tracey Messam-Johnson explore processes of variation and change in the Tense and Aspect system of Jamaican Creole based on language acquisition data from 3-year-olds in Jamaica and language use data of Jamaicans living in Curaçao. They find interesting similarities across the two data sets. There are patterns of change observable in the production of progressive constructions in Jamaican Creole which are affected by the other languages in the contact setting. Creole strategies in both settings, however, persist and show little variation in the case of the realization of the past tense.

\section{Reflections about Don’s scholarship \& life from colleagues and students}

This section presents short squibbs written by Don's colleagues and students, discussing Don's work, teaching, mentorship and interaction with colleagues. 

\subsection{Ian Robertston (UWI, St. Augustine)}

I present a brief examination and evaluation of the influence of Don Winford on the discipline of linguistics at the university, national, regional, and international levels during his tenure at \textit{The University of the West Indies}. It also addresses contributions that had consequential impact in areas critical to the university, the discipline more broadly, and the Caribbean region. 

\textit{Sociopolitical context}. The sociopolitical and academic climate in the early post-independence period in Commonwealth Caribbean countries provides an appropriate context for the evaluation of his work and contributions to the development of proper awareness of the Caribbean self.
Between 1962 and 1966 four Caribbean countries, Jamaica, Trinidad \& Tobago, Guyana and Barbados were granted political independence from Britain. One of the major challenges facing these newly independent countries with a history of centuries of slavery and indenture, was the need to develop social and economic models that could realize the promise of genuine political and economic independence. Issues of self understanding and self-definition impacted expectations of the population and, within a decade of attaining political independence, led to growing levels of popular frustration. The academic community, especially the only regionally owned university, could reasonably be tasked with leadership roles in examining and evaluating the contexts and in charting the options with the highest potential for success. 

The university itself was only one decade away from the umbilical links with London University. It had the responsibility for providing the region with the necessary academic leadership to support political independence. The university community became deeply involved in the popular reaction to the failure. By the end of the decade of the sixties, broad based political tolerance had largely dissipated. The euphoria of political independence was being replaced by impatience and frustration at the failure to provide instant change and success to the broad groups in the societies.  Public impatience with the political processes manifested itself in a number of political protests and significant public unrest. The public concern was captured and expressed in popular Jamaican song through the use of the saying,
\begin{quote}
    “Every day carry bucket a di well\\ 
     wan day dii bucket bottom must drop out.\\ 
     Everything crash”.
\end{quote}
 

The two most consequential expressions for the region were the so called 1968 “Rodney Riots” in Jamaica and the 1970 “Black Power” challenge to the government in Trinidad \& Tobago. In the former case, the Jamaican government’s decision taken in October of 1968, to declare Dr. Walter Rodney, one of the region’s foremost young historians and a lecturer in the History Department at the Mona campus, “persona non grata” and to exclude him from reentering the island, after attending an academic conference overseas, became the catalyst for widespread expressions of disappointment and frustration. Rodney himself, the author of \textit{“How Europe Underdeveloped Africa”}, was accused by the government of a level of political activism which was considered inimical to the nation’s political stability. 

For many in the university community, this presented a challenge to the perception of academic freedom. Massive protest, started by the university community at the Mona campus, brought protesting students and academics into the streets of the Jamaican capital, Kingston. These protests soon spread to include a large section of the urban population who felt a sense of frustration at the continuation of many of the practices they expected to disappear at independence. The frustrations were vented through burning and general unrest in Kingston. A second manifestation of this popular impatience, fueled in part by the rise of Black Power movement in the northern hemisphere, was the uprising in Trinidad in 1970 (Black Power riots). It is to be noted that here again these significant protest movements originated among members of The University of the West Indies community at the St Augustine campus.

\textit{Academic context}. As a fledgling institution, The University of the West Indies emerged from its status as a college of the University of London in 1962. It, too, was seeking to define itself and to come to terms with its responsibilities to the several stakeholder countries. It was expected to help chart the course of development. Faculties of Arts and General Studies were focused on the staples of History, Languages (French, Spanish, Latin) and English Literature. Linguistics, as a discipline, had not yet emerged as part of the academic offerings. Ironically, it is Linguistics, and in particular the study of Creole languages, that became central to defining Caribbean self, and to providing one of the major growth points, variation studies, in the study of languages worldwide. 


Donald Winford joined the staff at the St. Augustine campus in 1972, at a time when new directions had to be set, values had to be established, and aspirations to self definition had to be addressed.  As the university itself was still searching, seeking to determine its academic bona fides, disciplines such as Linguistics were initiated through a Senate sub-committee set up to facilitate work in the discipline across its three campuses, and  was centered at the Mona campus. 

Don's initial appointment in the English Department placed him in the delivery of the \textit{Use of English} course, but Winford’s entry brought a new focus in Linguistics. This course had been designed to introduce young undergraduate students to writing, logic, and argument through the media of English. Winford was the first member of the Faculty of Arts and General Studies at the St. Augustine campus who had completed doctoral studies in Linguistics, and only the second such staff member on the entire campus, the other being in the then Faculty of Education. Prior to his appointment, Linguistics as a discipline on the St Augustine campus was serviced entirely by staff members who had been appointed to the English and Modern Languages Departments and, in one case, to the then Faculty of Education.


Winford’s tenure at St Augustine brought significant outcomes. Appointed as lecturer in English language and Linguistics, he managed to maximise the obvious links between the two academic remits as well as to address and raise awareness of the relevance of linguistics for education in general, and for a proper appreciation of the Caribbean linguistic profile. Winford’s doctoral study, by presenting a sociolinguistic description of language patterns in two communities in Trinidad, provided what was probably the first such study to be done. More especially, its focus on non-urban, rural communities brought a different level of awareness of the nature, the scope and the significance of language behaviours in such communities which, until then, were treated as being of little significance on the national or even regional scenes. 
 
His position as lecturer in Linguistics facilitated the expansion of the undergraduate offerings by structuring and teaching a new course, \textit{Introduction of Sociolinguistics}, for levels two and three undergraduate students. His presence also facilitated the expansion of the \textit{Caribbean Dialectology} course and the development of a separate course in Creole Linguistics. These offerings ultimately led to an undergraduate major in Linguistics as well as the emergence of a separate Department of Linguistics. This department was to address some aspects of the more fundamental remit of the institution to help Caribbean students to expand their awareness of, and confidence in, the languages which documented the region’s history, helped define the region and at the same time presented the most revealing insights into the history of the region.


At the commencement of the 1972--1973 academic year, his first at the institution, he undertook responsibility for the level one course, Introduction to Language Theory. This course provided the undergraduate students with the first exposure to linguistics and its significance. It was essentially descriptive, relying on the standard fare of Smalley, Pike, Nida and Gleason. Phonetics and Phonology, as was to be expected in the post-Bloomfieldian period, formed the basic point of departure and Winford took responsibility for managing the operations of the less than adequate language laboratory facilities to support the exposure of undergraduate students. Despite the challenges, the exposure increased the flow of students into the courses at the next level. His commitment led to the gradual upgrading of this laboratory facility which was itself replaced by a modern and up to date, Centre for Language Learning at the turn of the century. The course in Sociolinguistics opened up new possibilities for both undergraduate and graduate studies and its impact has continued on this campus to the present time. A further input was made at the undergraduate level through the delivery of the course Caribbean Dialectology. This course was pivotal in the development of awareness of the complex linguistic nature of Caribbean societies especially as it reflected the vast range of variation in language use. These factors presented challenges to the existing awareness of the nature of linguistic variation and encouraged research into the nature of such variation.

Don Winford’s academic presence also facilitated some level of exposure to teachers enrolled in the Diploma in Education programme at the then Faculty of Education. His inputs into this programme were intended to develop awareness of the relevance of language studies and linguistics to their classroom functioning.
This education link provided a significant entrée into one of Winford’s research foci, the educational application of sociolinguistics research in Creole-speaking communities. His 1976 \textit{International Journal of the Sociology of Language} article on teacher attitudes toward language variation in Creole communities had implications to explore the critical need for language awareness among teachers in the national and regional communities. This need has become more pronounced today with the facilitation of, and exposure to schooling, to the broad bases of Caribbean populations who are often assumed to have or are expected to have sufficient competence in the official language on which schooling is premised.  

A second area of research which Winford pioneered on the campus and one that became one of the most significant dimensions of his entire academic profile, is the study of linguistic variation. Here again, research interest in Creole languages of the Caribbean put a different focus on the ways in which variation in language was to be treated. His work on variation published during the period at St. Augustine included \textit{Phonological Variation and Change in Trinidadian English}, (Society for Caribbean Linguistics, 1979), \textit{The linguistic variable and syntactic variation in Creole continua} (Lingua, 1984). Of even greater significance for linguistics was his exploration of the concept of diglossia and its appropriateness as a descriptor of Caribbean linguistic contexts. (Language in Society, 1985). Beyond these articles, Winford addressed issues of phonetics, phonology and language in Caribbean society, and grammatical characteristics of Caribbean Creole languages. All these provide very significant pathways to the development of the study of language in the Caribbean as well as to language studies in general. At the post-graduate level, Winford had significant inputs into the doctoral programmes having been supervisor for at least two doctoral candidates. His presence at St Augustine was properly rewarded by appointment on indefinite tenure and his promotion to Senior Lecturer well within the normal time for such promotion and before he left the institution. 

Don Winford also served as Secretary/Treasurer of the Society for Caribbean Linguistics (SCL), then located at St. Augustine. This post was at the centre of the society’s functioning. It required the holder to establish and maintain links with members scattered across Europe, North America, The Caribbean and Latin America. All this in the pre-computer age when the pace of correspondence was both tedious and time-consuming. His contributions in this post helped to ensure the longevity of one Caribbean institution in a historical context of failure of many such regional initiatives. SCL is now into its fiftieth year and continues to be the seed and growth box for the development of the discipline in the region. There was a special collegial relationship with colleagues in the faculty and the campus community during the period he worked with the institution. Those of us who interacted with him learnt to appreciate his frank and generous support and valued his work ethic.

Don Winford’s contribution to academia, Caribbean and national understanding of self and the study of linguistics during his tenure at The University of the West Indies provides a significant model for emulation by emerging scholars.



\subsection{Reflections from current colleagues and advisees (The Ohio State University)}

Mary Beckman, Cynthia Clopper, Hope Dawson, Elizabeth (Beth) Hume, Brian Joseph, Robert (Bob) Levine, Nandi Sims, Luana Lamberti, Justin Pinta

\subsubsection {Overview}
Don Winford joined the Department of Linguistics at Ohio State University in 1988. Thus, for some of us, he has been a trusted friend, a valued colleague, and an inspiration for more than three decades.  And for others of us, the time has been less but the impact of his presence has been no less strong. We offer here some personal reminiscences and other observations.

\subsubsection{Beth Hume} Don was an established figure in the department when I joined in 1991 as an assistant professor fresh out of graduate school. I felt an immediate connection to Don for a number of reasons including our shared interests in sociolinguistics and his West Indian roots, having spent time myself in Trinidad as a teenager. 

While I could comment on his many academic accomplishments, I prefer to focus on his commitment to mentoring young men. I discovered this initially while playing a game with my young son at the old Larkins gym on campus. To my surprise, Don walked into the gym with a young boy, his “little brother.” Over the years I came to know a few of the “little brothers” that Don mentored, supported and loved. In fact, more than one came to live with him, even after they were no longer “little” brothers. And while Don didn’t have immediate family in Columbus, his house was always filled with family: the young men he’d mentored, including their partners and children. I wouldn’t be surprised if there were also some grandchildren running around now!

I share this to underscore the kindness and generosity that is at the very core of who Don Winford is. Thank you, Don, for all your contributions to linguistics and to society. 

\subsubsection{Bob Levine}
I first met Don in 1988, along with Craige Roberts, when the three of us were new hires in the Department. Linguistics was housed at that time in Cunz Hall -- a confusing vertical labyrinth none of whose floors were laid out in exactly the same way, with what I remember as somewhat bizarre layouts that included apparent dead-ends and blocked access between parts of the same floor -- and, due to a shortage of office space, Don, Craige and I were housed in a single large room with movable dividers separating our desks. Despite (or maybe in part as a result of) the less-than-optimal conditions, we early on developed a strong camaraderie that, as it turned out, served us well, especially during the first few years, which were often not particularly easy ones for us. I very much appreciated Don’s ironic (and often sardonic) humor and his refreshingly straightforward way of talking about both our everyday life at the University and the larger world of linguistics -- as well as our frequent exchange of complementary gripes about the Columbus weather (with him hating the winters and me the summers). 
 
And I was quite happy to discover that certain aspects of Don’s research on Caribbean English Creoles had a direct bearing on some of my own research on unbounded dependency constructions. In particular, his watershed monograph, \textit{Predication in Caribbean English Creoles (CEC)} (Winford 1993a), shed crucial light on a long-standing argument in syntactic theory about the status of, e.g., \textit{for us} in \textit{John is easy for us to please}, providing unequivocal evidence for treating the Guyanese Creole cognate of \textit{for} here as a complementizer rather than a preposition, with \textit{John} being the subject of an infinitival clause rather than part of a PP. These and other conclusions in Don’s work -- almost certainly the most formally explicit and empirically robust investigations of CEC to date -- informed much of my thinking about the syntax of English. 
 
I’m going to miss Don a lot, as a long-time friend in the Department and as a very distinguished colleague whose international reputation has burnished the Department’s own over many decades. 

\subsubsection{Brian Joseph}
I was department chair when the opportunity to hire Don came up.  I knew of his work and knew that he would be a valuable addition to the department, given the breadth of his knowledge of sociolinguistics in general and the specialist knowledge he could contribute on language contact.

I had never met him, and this was before the days of Google and the Worldwide Web, so I didn’t even know what he looked like.  Don arrived at the Columbus airport in early March for his on-campus interview and this came just as we were having a serious cold spell and were in the midst of a snowstorm.  Realizing that he was coming from the Caribbean and figuring that he did not have a lot of recent experience with real wintry weather (his time at York University in England having been some 15 years before and his time in the US being in Texas), I brought him a full-length winter coat when I picked him up at the airport.  However, I am only about 5’6” and I did not know that Don was over 6’ tall!  So the coat I brought him was anything but full-length for him, and barely covered him to the waist.

Somehow, Don managed to stay warm during his visit here, and somehow  --  fortunately  --  the Columbus wintry slap in the face did not dissuade him from accepting the offer to join that department that came his way a few weeks later.

On a more serious note, I have greatly valued the contribution Don made to the department in general, offering an important socially and empirically informed counterweight to the somewhat more formal and theoretical orientation of the department overall. Moreover, his integrity as a researcher and his high standards of scholarship  --  what he expects both from his students and from his faculty colleagues  --  have upped our individual and collective game considerably.  On a personal level, Don has added enormously to my own understanding of language contact and of the importance of social factors in language change and I thank him for that. We will definitely miss him; his scholarly rigor and his keen insights will be difficult to replace.

\subsubsection{Cynthia Clopper}
I joined the department in 2006 and immediately began working alongside Don on a faculty search committee for a sociolinguist and on a number of student qualifying paper and dissertation committees. From those first experiences to the present, I have been struck by Don’s approach to our collective work in the department to achieve excellence in research, teaching, and service. Don is incredibly thoughtful and deliberative when it comes to all aspects of his contributions to the department, including governance decisions, student feedback, and faculty promotions. He doesn’t always speak up in department meetings, but when he does, his comments cut straight to heart of the topic and serve to advance the conversation. Our collective decisions are better as a result of Don’s incisive contributions. In his feedback to our students on their research, Don identifies the critical missing connections between data and theory, encouraging them to push themselves to produce the best scholarship that they can. Our students’ work is better as a result of his advice and guidance. In the faculty annual review and promotion processes, Don is always the most prepared member of the faculty, having clearly reviewed all of the materials and taken detailed notes in advance of the discussion. On a more personal note, Don was the Procedural Oversight Designee (POD) for both of my faculty promotions and provided exceptionally detailed suggestions for improving my dossier materials. Our faculty are stronger as a result of Don’s commitment to his colleagues’ success and career advancement. Don’s contributions to encouraging, supporting, and advancing the careers of all members of the department will be greatly missed. 

\subsubsection{Hope Dawson}
I had the great privilege of taking several courses with Don as a PhD student at Ohio State, beginning with Introduction to Sociolinguistics in Spring term of 1998. With my intellectual appetite whetted, I proceeded to take every other course on sociolinguistics and contact linguistics that he offered during my years as a graduate student. He is an excellent instructor, supportive and challenging, and two of my earliest conference presentations were of research done for his courses. Since then I have had the pleasure of working with Don as a colleague at Ohio State, and I greatly value his insights and collegiality.

\subsubsection{Mary Beckman}
Don joined the department three years after I did, but he had been a linguist and an academic far longer than I had been at the time, so in all ways other than our affiliation with Ohio State, he has been my \textit{sempai} rather than the other way around. 

Over the decades since then, the department has gone through two external reviews, the faculty has more than doubled in size, the graduate program has been revised to shorten the time to degree and triple the graduation rate, and the undergraduate major has developed from a minor appendage to the graduate curriculum into a full-fledged curriculum in its own right. Throughout the course of these changes, Don has consistently and thoughtfully contributed his time and his wisdom. I have learned so much from him at department meetings, on search committees, in dissertation defenses, etc. 

And I am not the only one to have learned from him. Over the years, I have watched as our graduates, who were his advisees, be the same kind of thoughtful, committed mentor to their advisees (Don’s grandstudents).  I have watched them contribute in similar ways to the growth and transformation of the departments and programs that hired them.  

Finally, more recently, as events in the world around us have forced me to reevaluate my role in society outside of linguistics, I have come to appreciate anew Don’s quiet but steadfast commitment to social justice. Living where I do now in the southern half of the Gullah Geechee Cultural Heritage Corridor, I am blessed to be able to also apply insights from his and his students’ research as I strive to follow his example of involvement in my community. 

\subsubsection{Nandi Sims, Luana Lamberti, Justin Pinta}
Like many of Don’s past advisees, his influence on us, both as scholars and as people, is indelible. We have each experienced his teaching within our first two years of school at Ohio State, which prompted us to subsequently take each class he has offered. While we each work in different regions, his passion for teaching promoted in each of us a love of contact, an appreciation for the subtle differences between transfer and imposition, and an attention to detail that has helped each of us succeed in graduate school and feel prepared for careers in academia. 

As an advisor and a committee member, Don’s rigorous expectations and meticulous attention to detail have caused each of us to pull out our hair from time to time. In the end, however, these high expectations have translated into each of us having a critical eye in reading, research and writing. His close readings of our papers and grant proposals, careful and detailed feedback, willingness to meet whenever we needed him, and countless letters of recommendation have played a tremendous role in each of our successes in our careers thus far. 

Don is the type of professor that has the ability to make one think about language from a different perspective. His courses have inspired each of us to expand our fields. Future students of linguistics at The Ohio State University will surely be missing out without Don’s expertise in New World Black Englishes and other contact varieties.

\subsection{Reflections on Donald Winford and the Creolist Hypothesis (Tracey L. Weldon, The University of South Carolina)}

In the fall of 1991, I moved from Columbia, South Carolina to Columbus, Ohio to begin my graduate studies at The Ohio State University. I had just completed my BA and was interested in learning more about the ways in which what was then called “Black English” was studied by linguists. I had gotten a glimpse into this line of research in my undergraduate classes on “The History of English” and “Modern English Grammar”. What perhaps intrigued me most about this planned area of study, however, was a conversation that I’d had with my then future advisor, Professor Donald Winford, during a visit to the Ohio State campus earlier that summer. When I shared with him my plans to study Black English, I distinctly recall him saying, “But it’s not English, right? It’s a separate language”. What’s this you say? Black English as a separate language? How can this be? What does this mean? I would later come to understand that Don was endorsing an early version of the Creolist Hypothesis, which posited a separate and autonomous system for Black English resulting from a Creole origin that distinguished it from other varieties of American English. I also later learned that Gullah, an English-based Creole spoken along the coasts of South Carolina and Georgia, was believed to be a critical component to understanding this possible Creole past. And, thus, began my foray into variationist sociolinguistics and the study of Gullah and AAVE under the tutelage of Professor Donald Winford.

My doctoral dissertation, written under Don’s direction, ultimately focused on the Gullah-AAVE connection, with a focus on variability of the copula. This line of research dovetailed nicely with Don’s expertise in the areas of Creole linguistics and African American language varieties. My decision to focus on Gullah, a more “intermediate” Creole variety, as a point of comparison with AAVE, was particularly inspired by Don’s 1992 papers, “Back to the Past: The BEV/Creole Connection Revisited,” and “Another look at the copula in Black English and Caribbean Creoles” which demonstrated the utility of comparing AAVE to more intermediate varieties, such as Trinidadian Creole (TC), which as he argues...

\begin{quote}
... provide the soundest basis for comparison between BEV and CEC. Both BEV and TC are vernaculars which have long been in contact with more standard varieties of English, in quite similar social circumstances. Such contact has led to similar patterns of restructuring in both varieties in areas such as the copula system (Winford 1992a: 24) 
\end{quote}
My research was also significantly informed by Don’s comprehensive two-part overview “On the origins of African American Vernacular English – A creolist perspective,” in which he presented an alternative view of the Creolist Hypothesis, which positioned AAVE not as a separate and autonomous system resulting from decreolization of an earlier plantation Creole, but rather as an English variety that was shaped by language contact and language shift between speakers of early AAVE and Creole varieties. 

According to Winford (1997h), Gullah likely emerged in South Carolina during the period from 1720 to 1775, which was marked by institutionalized segregation and the massive importation of enslaved Africans to meet the growing needs of the plantation economy. Prior to this time, roughly 1670 to 1720, conditions in South Carolina would have favored Africans learning an approximation to the settler dialects. With the introduction of the Creole, the linguistic situation among Africans in South Carolina would have come to resemble a continuum. 


\begin{quote}
... Africans in closer contact with whites must have continued to learn closer approximations to their dialects. By the mid-18th century, the linguistic situation on the South Carolina coast would have been similar to that in other Caribbean colonies -- a creole continuum within the African population, complicated by continuing input from white dialects on the one hand, and the African languages of newly-arrived slaves on the other (Winford 1997d: 315). 
\end{quote}

Within this framework, the period from 1780--1860 would have been one of consolidation and leveling across both Creole and dialectal English varieties. And the processes of language contact and language shift between Africans speaking a Creole and those speaking approximations to the settler dialects would have resulted in a significant amount of Creole substratum influence in AAVE. 

We can assume that there was a sizeable body of Africans throughout the southern states in this period whose primary vernacular was a creole English, and many of whom shifted over the years to AAVE as their primary vernacular, ‘transferring' or preserving in the process certain elements of the creole grammar. I also assume that there was a sizeable body of Africans whose primary vernacular was an earlier form of AAVE which was fashioned after the settler dialects, and which provided the target of the shift. Contact between these groups of Africans on the plantations is likely to have contributed to the development of AAVE (Winford 1997d: 317).

Along these lines, Don argued that a shift scenario would better account for patterns of copula variability in AAVE than decreolization.

My present position is that the copula pattern of AAVE is best explained as the result of imperfect second language learning with transfer from creolized or restructured varieties playing a significant role. On the one hand, many Africans must have acquired a close approximation to the superstrate copula system from the earliest stages of contact. Other groups of Africans speaking African languages, and later, creole or other forms of restructured English in which copula absence was common, shifted toward these established forms of AAVE, introducing fu[r]ther changes due to imperfect learning (Winford 1998: 111).

From this perspective, copula absence in early forms of AAVE would have been reinforced by Creole varieties introduced from Barbados and other areas of the Caribbean, where the copula was most likely also absent in all but nominal environments. And the early copula system of Gullah would have had a significant impact on the development of the emerging AAVE system. Accordingly, the decreolization model proposed in Winford (1992a) was “reinterpreted as a model of shift, with the shift from a ‘mesolectal’ to an ‘acrolectal' copula system most relevant to the developments in AAVE” (1998: 112). Furthermore, the similarities that he observed between AAVE and TC were attributed to the fact that both varieties emerged out of situations involving shift between a system with zero copula (except before nominals in the case of TC) and one with forms of \textit{be} (1998: 112). And the fact that modern-day TC and other CECs require a copula in nominal environments while AAVE allows for copula absence was attributed to differences in substratum input, “with early AAVE affected by restructured varieties containing no copula in nominal environments ...” (Winford 1998: 112).

 This perspective on the Creolist Hypothesis is one that I ultimately endorsed in my own dissertation research \citep{Weldon1998} and subsequent publications on the AAVE-Gullah connection (see e.g., \citealt{Weldon2003copula}; \citealt{Weldon2003revisiting}) and one that other scholars have endorsed in various forms as well (see e.g., \citealt{Mufwene1997}; \citealt{Rickford1997}; \citealt{Rickford1998}). While the field owes much to Donald Winford for his meticulous contributions to Creole linguistics and the study of African American Vernacular English, I am indebted to him for taking a budding linguist from South Carolina under his wing and offering her an opportunity to interrogate the Creolist Hypothesis through an exploration of the AAVE-Gullah connection in its sociohistorical and linguistic contexts. 


\subsection{Don's contribution to Sociolinguistics (Robin Dodsworth, North Carolina State University)}

Don Winford is probably best known as a leading scholar of Creoles and of contact linguistics, but many of us have known him first as a foundational sociolinguist. Don's initial contribution to sociolinguistics, his 1972 dissertation, was not only the first sociolinguistic account of Trinidadian English, but also one of the earliest systematic analyses of sociolinguistic variation. His subsequent publications about Trinidadian English both expanded empirical knowledge and advanced sociolinguistic theory. For example, his 1978b article in the \textit{Journal of Linguistics} first shows that the variable (th) in Trinidadian English has the now-familiar ‘crossover' pattern, in which a central social class uses an overall lower rate of the stigmatized variant, relative to higher social classes, in a word list reading task designed to call attention to the variable. The article next takes issue with \citet{Labov1966}, arguing that crossover differs from true hypercorrection, insofar as crossover involves no “irregular" placement of a phonetic variant as in [θin] for ‘tin'. True hypercorrection, the article goes on to say, “provides evidence that two distinct systems, each with its own norms of usage, are in contact", which is especially valuable in cases of decreolization (284).

The insights about crossover and hypercorrection foreshadow a consistent theme in Don's subsequent work: the critical appraisal of the relationship between social patterns and linguistic structure. This is especially clear in Don's wide-ranging work in the area of contact linguistics. I think Stephanie Hackert (\citeyear{Hackert2005}) says it best in her review of Don's (2003a) book \textit{An Introduction to Contact Linguistics}:
\begin{quote}

“In its emphasis on uniting linguistic and sociolinguistic approaches to language contact and treating all language-contact phenomena within a single theoretical and methodological framework, Winford’s Introduction … will surely contribute to drawing the various foci of contact linguistics closer together, thus further consolidating the field." (page 115) 
\end{quote}

I would add that Don's close and systematic attention to social context in relation to linguistic structure helps to consolidate not just the various areas of contact linguistics, but of sociolinguistics more broadly. For example, his (1985b) article in \textit{Language in Society} observes that creole-speaking communities are diglossic in Ferguson's sense. Winford (1997d) argues on the basis of extremely detailed socio-historical and linguistic data that mesolectal Caribbean creoles result mainly from contact with Europeans, language shift, and shift-induced restructuring, and not from a continuous ‘decreolization' process. Both of these claims (about diglossia and about ‘decreolization') function not only to affirm the importance of social context in understanding linguistic structure, but also to bring together creole studies and variationist sociolinguistics; Creoles result from, among other things, the linguistic effects of social forces that sociolinguists have observed in non-creole settings.
The themes of uniting linguistic and social data, and of uniting Creole studies with other kinds of sociolinguistic studies, are also central to his work in discerning the structure and origins of AAE. Winford (1992a) argues that AAE resembles a mesolectal Creole with respect to both structure and history, and that its copula bears structural similarities to the copula in Trinidad English creole (see also Winford 1997d, 1998). In both varieties, the copula has been restructured over time in response to pressure from a superstrate. Winford (2000c) focuses on the importance of social context in determining the origin(s) of AAE, a notoriously difficult question that he approaches as both a creolist and sociolinguist:
\begin{quote}
[I]t is clear that we are dealing with multiple sources whose effects varied according to differences in the nature of the contact settings, their demographics, and the types of social interaction among the groups involved. It should also be clear that for many AAVE features, we are dealing with the effects of multiple causation, involving externally and internally motivated change, leveling, and processes of simplification and restructuring. An understanding of these can come only from a thorough investigation of the sociohistorical contexts of the emergence and development of AAVE. However, this aspect of the genesis of AAVE in all its forms remains relatively unexplored. This is unfortunate because, as every student of language history knows, the sociolinguistic history of a community, and not linguistic factors, is the primary influence on how languages originate, change, and develop. (409--410) 
\end{quote}

He goes on to lament the absence of several kinds of information about the social contexts in which AAE is used, offering that we do not yet understand how linguistic choices within Black communities convey group identities, group values, and communicative goals, nor do we have the data we need about AAVE in “situated language use, conversational norms, communicative strategies, genres of talk, and the like" (2000: 411). The dual concern with, first, the linguistic processes of variation and change such as leveling and restructuring, and second, the social forces that give rise to and shape the linguistic processes, thus characterizes Don's perspective on questions normally associated with contact linguistics and questions often considered the territory of variationist sociolinguistics. 

The primacy of sociohistorical factors, a striking claim in the extended quote above, reappears unattenuated in later work. For example, a chapter on the social factors in language contact, argues that we need more detailed information about the social settings that produced contact languages in order to understand how they arise and why they have particular structural characteristics, e.g. the syntactic frame from one language and some of the lexicon from another language (Winford 2013e). Drawing upon a remarkable knowledge of distinct contact settings around the world, the chapter goes so far as to say that “[t]he evidence now available to us strongly supports the view that social factors play a significant, and in some cases a more important role than linguistic factors, in shaping the consequences of language contact" (page 365). It is clear from this chapter, as well as (Winford 2003a) and many of his other texts about contact linguistics (which are discussed at greater length elsewhere in this volume), that Don considers the outstanding questions in contact linguistics to be mainly about the interaction of social context with linguistic processes. He has observed, for example, that 

\begin{quote}
    … [W]e cannot always establish clear and consistent correspondences between the social contexts and linguistic processes involved in the creation of intertwined or other mixed languages. Different social circumstances can lead to similar processes of mixture, while different types of mixture may arise in what seem to be similar social settings. We still need to investigate the reasons for this … (2013: 379).
%\textbackslash 
\end{quote}

I would say that this observation is equally true of other areas of sociolinguistics. In the area of sociolinguistic variation, we haven't identified a clear correspondence between social contexts and linguistic outcomes. We do know some things about this. We have some evidence, for example, that dense, multiplex community networks are associated with maintenance of local vernaculars; and that, relatedly, ethnic segregation promotes inter-group linguistic difference; and that in class-stratified urban contexts, certain kinds of linguistic variables will also be class-stratified; and that linguistic variables with strong social indexicality can show tremendous intra-speaker variation from one interactional context to the next; and that complex internal constraints are often weakened or revised during language change or inter-community transmission. But our predictive ability doesn't go very far beyond these generalities. The most productive way forward when it comes to this question and others is surely to answer Don's call for more consistent, rich social data about the communities in which we study sociolinguistic variation.

With respect to richer social data, one of Don's wishes for the field, and one that made a strong impression on me when I was his student almost twenty years ago, is for better data about the relationship between social network characteristics and language variation and change. He observed that early sociolinguistic work about speakers' participation in dense, multiplex local networks has not led to widespread, systematic investigation of specific ego network characteristics as possible influences on language production or perception. We also have very little network analysis about real social interaction between people. My own work with social networks evolved from these conversations, and even now Don is both an insightful critic and a great supporter of it, always encouraging and remarkably accepting of new approaches to social context and to network data.

All of Don's PhD students have done research that reflects and affirms his commitment to sociolinguistics through engagement with socially contextualized linguistic variation. The topical breadth of their work, and the many strands of sociolinguistic theory that their work contributes to, are natural consequences of working with an advisor who has been a foundational and forward-thinking sociolinguist, and whose work has united distinct areas of sociolinguistics.  

\subsection{A tribute and a few recollections (Arthur K. Spears, The City University of New York)}

As with many colleagues and friends, I cannot remember when I first met Don; but, it was probably either a few years before or after his move to Ohio state in 1988. I do remember our getting together for lunch or dinner at many, perhaps most, conferences we both attended, especially those of the Society for Pidgin and Creole Linguistics (SPCL). To the extent we talked about linguistics, it was about Creole languages, African American English (AAE), and/or tense-mood-aspect (TMA), the last of which, I could say, has been the thread that has tied most of my work on Creoles and African American English together -- not to mention Haitian Creole -- and which has often triggered my interest in other areas of grammar. We not infrequently talked about negotiating our academic careers and how to avoid the worst pitfalls on the promotion path to full professor. 
Considering all of the subfields and sub-subfields of linguistics that I do research in, Don shares more of them than any other linguist. My recent linguistic research has concentrated on AAE grammar and history, with a particular focus on how contemporary grammars of AAE can shed light on what is still the central question on AAE history: to what extent was a Creole language, or languages, involved in it? We could go on to ask how such involvement occurred and what Creole language or languages left traces in AAE. 

Don’s research has indeed been a significant source of inspiration for my own work; and, I have often started very near the beginning of some writing by rereading something of his, just to get my thinking started or to make sure that there was not anything of significance that I was not taking into consideration. Indeed, I frequently find Don’s work overstimulating, due to its richness in detail, superb argumentation, and meticulous analyses. There are few times that I return to a writing of his without getting a few new ideas for articles, some better than the central ideas of articles I am already committed to writing. As a result, the probably more interesting Don-inspired articles end up unwritten most of the time. 
Actually, at times tiger-related metaphors pop into my mind when I am reading Don. Often his critique, or destruction, of claims and hypotheses does bear the marks of tigerishness in its diligence. I do not want to say ferocity because I am sure that Don never intends to be ferocious. The manner in which he carries himself is always highly suitable for drawing rooms across the United States and certainly the parlors, “yards,” etc. all over the Western Hemisphere. This remark about tigerishness reminds me of a story, which I will leave for another occasion. 

But then again, what better occasion than this to reveal something deeper in Don’s character than what is usually talked about. No, this story is not about Don’s sometimes mordant sense of humor. No, it is not about his sometimes directness of expression. ‘How did you like your meal?’ asked of Don as we were all leaving a traditional wrap-up ‘banquet’ at an SPCL conference. ‘It was awful and way overpriced!’ (speaking of a precious rendering of Nouvelle Moroccan cuisine). No mincing of words here. No chopping or sautéing of them either. I broke out laughing at his reply. It was the shock of recognition -- of typically African American directness, which, I have come to learn, is simply one branch of an African Diasporic directness of expression that I have witnessed all over the Americas. 

No, this story is about something else. One evening, while attending a conference in Graz, Austria, we went out to dinner and talked about the conference and a range of matters. I was already becoming impressed with the food in Austria, and that impression became more prominent in my mind as we finished the meal and attendant matters, and then headed for the exit, passing beside a very long bar. Something was shouted from somewhere at the bar. I could not hear it clearly. Don did, and as if releasing a tiger within him that sprang into action when required, he loudly hurled back a response to whatever it was that had been shouted from the bowels of the bar. “What was that?” I asked, but Don was unwilling to repeat whatever was said, distracted as he was. Don had stopped and turned toward his interlocutors, who I gathered had thrown insults or slurs -- or both -- at us. I was aware of news coverage of the then extreme-for-Europe anti-immigrant (read anti-brown and black) sentiment in Austria. Don said something. I said, “Let’s go,” my upbringing during the U.S. Reign of Terror (Jim Crow) kicking in. I am only two years younger than Emmett Till, whose lynching was a conspicuous factor in the launch of the African American (later joined by other groups) Civil Rights Movement. Don caught up with me as I went through the door to the sidewalk, and he remained mum about what had been said. I could not help but admire his courage, his deep reserve of tigerishness, standing his ground, even reprimanding them. I thought about asking him if he had had such encounters in the U.S., but our talk had already turned to other matters, as I thought how helpful the incident was in reminding me to be on high alert since, after the conference, I was going to visit Vienna, a fascinating, old imperial capital -- high alert advisable.

\subsection{A Career in Linguistics (Anthony Grant, Edge Hill University)}

Don Winford chose well in deciding where to pursue his undergraduate degree and his  PhD.  Having first studied for his BA at King’s College London, he ventured north and  studied under R. B. (Bob) Le Page at the University of York, which was long one of the main centres for the study of pidgins and Creoles, a field of study which was  much endorsed by Bob, the founding Head of the Department who had spent several years at the University of the West Indies at Mona and who was one of the most renowned creolists of his generation.  In addition to Don, York produced creolists such as Mark Sebba, who did his doctorate there, Sir Colville Young, titan of Belizean politics,  and the late Philip Baker who pursued his BPhil at York before completing his PhD at SOAS University of London.  Don would doubtless have drawn intellectual refreshment from the extensive creolistic library which Bob had gathered at the J B Morrell Library, and which remains essentially intact to this day.  (On a personal note, it was Creole research which impelled me to matriculate at York as an undergraduate in 1981).

Coming as he did from an island where an English-lexified Creole had been the major medium of communication for over a century (thus largely supplanting an exogenous French-lexified Creole), it is not surprising that Don’s research has focused on English-lexified Creoles, though his studies have not been confined to it.  One of his early papers (Winford 1975) discusses, with ample illustration, the nature of ‘Creole’ culture and language in Trinidad.


His monograph \textit{Predication in  English-Based Creoles}, drawing on data regarding verb phrases from a number of Caribbean Creoles and beyond,  and presenting views on change in verb phrases  which were not solely reliant on previous ideas about decreolisation and the (post-)creole continuum, appeared in Winford (1993b).


He co-edited a  book on code-switching (Isurin, Winford \& Bot 2009), in which psycholinguistic approaches are to the fore, and in which the languages surveyed go well beyond creole languages.  In the year when his book on Creole predication appeared, he  also saw the publication of a book which he had edited with Frank Byrne on focus and grammatical relations in Creole languages  (Byrne \& Winford 1993), which goes well beyond its title as it covers a range of topics from predicate clefting to discourse processes. He also edited a well-received book of essays by diverse hands with Arthur K. Spears (Spears \& Winford 1997), and several of these papers have become classics of the field. 

Don has also written a series of papers on the evolution of  African-American Vernacular English (AAVE) in \textit{Diachronica} and elsewhere, culminating in his article on the topic in the Oxford Online Encyclopedia of Linguistics (Winford 2014) and in a collection of papers honouring the work of Professor John R. Rickford (Winford 2020a).  In his work on the subject Winford argues from both linguistic and sociohistorical data that AAVE did not undergo decreolization, as many of his predecessors had stated, but that it emerged from an interaction of English dialects, especially Southern English and Scotch-Irish varieties, which were influenced by West African languages and additionally by Creoles emerging in the West Indies, especially Barbados, whence many settlers in Virginia, Maryland  and the Carolinas – which Don sees as the likely birthplace of AAVE – had come in the pre-Revolutionary period. Don covered not only creoles but also pidgins and mixed languages and much else besides in his book \textit{An Introduction to Contact Linguistics}  (Winford 2003a).  In this and subsequent work he has been a champion of the approach to language contact pursued by Frans van Coetsem, in which the two transfer types in which contact-induced linguistic change is involved are borrowing and imposition, which means that elements are brought from the source language into the recipient language (see especially \citet{vancoetsem1988}).    This is not the traditional way of viewing contact-induced change, but it shakes up the linguistic kaleidoscope and allows us to view contact phenomena from a different but equally enlightening angle. An article by Don on the same topic in \textit{Diachronica} (Winford 2005), applying the van Coetsem approach to new scenarios,  continued this approach, while Don subsequently published a chapter in Insurin et al. (2009), and also in  Winford (2020c), following these theories.

In addition to editing JPCL for more than half its lifespan so far, Don Winford has been a staunch advocate of work by newer scholars, as a teacher, a researcher and an advocate. He has never been afraid to challenge the prevalent views in creolistics research. We salute him!




\section*{Abbreviations}

\begin{tabular}{@{}ll@{}}
AAVE & African American Vernacular English  \\
AAE  & African American English\\
ASR  & Automated Speech Recognition \\
CEC  & Caribbean English Creole \\
TC   & Trinidadian Creole\\
\end{tabular}

\begin{tabular}{@{}ll@{}}
TMA  & Tense Mood Aspect\\
JPCL & Journal of Pidgin and Creole Languages \\
SPCL & Society for Pidgin and Creole Languages \\
\end{tabular}

\section*{Acknowledgements}

This project has taken quite some time to pull together. We (Bettina \& Shelome) first shared it with Tracey Weldon and Steve Hartmann-Kaiser who both had a couple of very good suggestions. Sandro Sessarego, during a visit to Dublin, then helped shape the overall character of the book, which was further refined during our talks at the SPCL conference in Tempere. We are very grateful to the authors of the different papers. Many of them have, by now, had to wait for a very long time to see the publication of this book. We would also like to thank Stefano Manfredi and Isabelle Léglise for accepting to publish the book in their series and for their help with selecting reviewers. We are very grateful to all the reviewers who generously gave of their time to give us and the authors valuable feedback on the papers. Last but not least, we would like to thank Sebastian Nordhoff and his team for invaluable support with the technical aspects of editing and converting the papers into \LaTeX, and Emily Martin and Ariana Arrojado, students at the University of Pittsburgh for their help with the editorial work leading up to the final product.


\printbibliography[heading=subbibliography,notkeyword=Winford]

\section*{Don Winford’s publications: A bibliography}
%I like this option 2
\subsection*{Journal articles}
\begin{itemize}[label={},itemsep=\bibitemsep,leftmargin=\bibhang,itemindent=-\parindent,noitemsep]\sloppy

    \item \citeWinfordWork{Winford2018}
    \item \citeWinfordWork{Winford2013_unified}
    \item \citeWinfordWork{migge_winford2013fact}

    \item \citeWinfordWork{EssegbeyWinford2013}

    \item \citeWinfordWork{winford2007amazonia}
    \item \citeWinfordWork{Winford2007_framing}
    \item \citeWinfordWork{Winford2008_processes}

    \item \citeWinfordWork{winfordmigge2007substrate}
    \item \citeWinfordWork{winford2005contact}
    \item \citeWinfordWork{Winford2003_osuworkingpapers}
    \item \citeWinfordWork{Winford2001typology}

    \item \citeWinfordWork{Winford2000_irrealis}
    \item \citeWinfordWork{Winford2000_classification}
    \item \citeWinfordWork{winford2000plus}
    
    \item \citeWinfordWork{Winford1999_anglistica}
    \item \citeWinfordWork{Winford1999_cuadernos}

    \item \citeWinfordWork{Winford1998originsfeatures}
    \item \citeWinfordWork{Winford1997_formation}
    \item \citeWinfordWork{Winford1997_sociolinguistics}
    \item \citeWinfordWork{Winford1997originssocio}
    \item \citeWinfordWork{Winford1997_propertyitems}
    \item \citeWinfordWork{Winford1997re-exam}

    \item \citeWinfordWork{Winford1996_commonground}
    \item \citeWinfordWork{Winford1996_typology}
    \item \citeWinfordWork{Winford1993}
    \item \citeWinfordWork{Winford1992anotherlook}
    \item \citeWinfordWork{Winford1992back}

    \item \citeWinfordWork{Winford1990copula}

    \item \citeWinfordWork{Winford1988_stativity}
    \item \citeWinfordWork{Winford1988_notion}
    \item \citeWinfordWork{winford1985_diglossia}
    \item \citeWinfordWork{Winford1985}
    \item \citeWinfordWork{Winford1984}
    \item \citeWinfordWork{Winford1980culture}
    \item \citeWinfordWork{Winford1979phonological}
    \item \citeWinfordWork{Winford1978hypercorrection}
    \item \citeWinfordWork{winford1976teacher}
    \item \citeWinfordWork{Winford1974diff}
\end{itemize}

\subsection*{Papers in edited collections}
\begin{itemize}[label={},itemsep=\bibitemsep,leftmargin=\bibhang,itemindent=-\parindent,noitemsep]\sloppy
    \item \citeWinfordWork{winford2020another}
    \item \citeWinfordWork{Winford2020_spanishes}
    \item \citeWinfordWork{Winford2020_theories}


    \item \citeWinfordWork{Winford2017_observations}
    \item \citeWinfordWork{Winford2017_ecology}
    \item \citeWinfordWork{Winford2017_worldenglishes}

    \item \citeWinfordWork{Winford2015_creoleformation}
    \item \citeWinfordWork{winford2015origins}

    \item \citeWinfordWork{Winford2014}

    \item \citeWinfordWork{Winford2013_unity}

    \item \citeWinfordWork{Winford2013_afterword}
    \item \citeWinfordWork{Winford2013_challenging}
    \item \citeWinfordWork{Winford2013_socialfactors}

    \item \citeWinfordWork{Winford2013_sranan}
    \item \citeWinfordWork{Winford2013_influence}



    \item \citeWinfordWork{Winford_plag2013_dataset}
    \item \citeWinfordWork{Winford_plag2013}

    \item \citeWinfordWork{Winford2012_creole}
    \item \citeWinfordWork{Winford2012_history}

    \item \citeWinfordWork{Winford2010_contact}
    \item \citeWinfordWork{Winford2010_revisiting}

    \item \citeWinfordWork{Winford2009_unity}
    \item \citeWinfordWork{Winford2009_interplay}


    \item \citeWinfordWork{WinfordMigge2009}

    \item \citeWinfordWork{Winford2008_atlantic}
    \item \citeWinfordWork{Winford2008_english}
    

    \item \citeWinfordWork{WinfordMigge2008}
    \item \citeWinfordWork{Winford2006_reduced}
    \item \citeWinfordWork{Winford2006_revisiting}
    \item \citeWinfordWork{Winford2006_belize}
    \item \citeWinfordWork{Winford2006_restructuring}
    \item \citeWinfordWork{Winford2003_ideologies}
    \item \citeWinfordWork{Winford2002}

    \item \citeWinfordWork{Winford2001comparison}
    \item \citeWinfordWork{Winford2001intermediate}

    \item \citeWinfordWork{Winford2000tenseaspect}


    \item \citeWinfordWork{Winford1997intro}
    \item \citeWinfordWork{Winford1996_syntactic}
    \item \citeWinfordWork{Winford1996_cec}
    \item \citeWinfordWork{Winford1994_anglophone}
    \item \citeWinfordWork{Winford1994model}
    \item \citeWinfordWork{Winford1991_directional}
    \item \citeWinfordWork{Winford1991_caribbean}
    \item \citeWinfordWork{Winford1991_passive}
    \item \citeWinfordWork{Winford1990_motion}
    \item \citeWinfordWork{Winford1983_negation}
    \item \citeWinfordWork{Winford1980_situation}
    \item \citeWinfordWork{Winford1978_system}

    \item \citeWinfordWork{winford1976teacher}
    \item \citeWinfordWork{winford1975}
\end{itemize}


\subsection*{Monographs}
\begin{itemize}[label={},itemsep=\bibitemsep,leftmargin=\bibhang,itemindent=-\parindent,noitemsep]\sloppy
    \item \citeWinfordWork{winford2003book}
    \item \citeWinfordWork{winford1993predication}
    \item \citeWinfordWork{Winford1972}
\end{itemize}

\subsection*{Edited collections}
\begin{itemize}[label={},itemsep=\bibitemsep,leftmargin=\bibhang,itemindent=-\parindent,noitemsep]\sloppy
    \item \citeWinfordWork{EssegbeyWinford2013}
    \item \citeWinfordWork{isurin2009}
    \item \citeWinfordWork{Winford2003_osuworkingpapers}
    \item \citeWinfordWork{spearswinford1997}
    \item \citeWinfordWork{byrnewinford1993}
    \item \citeWinfordWork{EdwardsWinford1991}
    \item \citeWinfordWork{Winford1974_lgsociety}
\end{itemize}
\end{document}
