\documentclass[output=paper,colorlinks,citecolor=brown]{langscibook}
\ChapterDOI{10.5281/zenodo.6979325}
\author{Trecel Messam\affiliation{University of the West Indies, Mona} and Michele Kennedy\affiliation{University of the West Indies, Mona}}
\title{Jamaican Creole tense and aspect in contact: Insights from acquisition and loss}



\abstract{This paper presents an account of the use of the progressive aspect and the simple past tense in the speech of three-year-olds from Jamaican Creole (JC) speaking communities, as well as in the speech of JC-speaking migrants to Curaçao, who now function in a second language (L2) dominant environment. We compare the two data sets, and find that parallels may be drawn between the interlanguages of these speakers; there are patterns of the mixing of the L1 with the languages in contact in the progressive construction, but little mixing in the past tense, where Creole forms persist in the speech of both sets of speakers. Such parallels may be unsurprising, given Winford’s (\citeyear[256]{winford2003book}) assertion that “the phenomena involved in language attrition … are similar to those found in many other cases of contact …”. We conclude that there are fundamental differences between the expression of pastness in JC and in the target languages, and suggest that this lack of congruence may cause difficulty in learning the L2 (\citealt[252]{winford2003book}). We suggest further, that the promotion of language awareness in language arts classrooms will go a long way to overcoming this difficulty.

\keywords{language acquisition, language attrition, language change, language contact, Papiamentu, Jamaican Creole, tense, progressive aspect, language education, language transfer}


}

\IfFileExists{../localcommands.tex}{
  \addbibresource{../localbibliography.bib}
  % add all extra packages you need to load to this file

\usepackage{tabularx,multicol}
\usepackage{url}
\urlstyle{same}

\usepackage{listings}
\lstset{basicstyle=\ttfamily,tabsize=2,breaklines=true}

\usepackage{langsci-basic}
\usepackage{langsci-optional}
\usepackage{langsci-lgr}
\usepackage{langsci-osl}
% \usepackage{./langsci/styles/langsci-lgr}
% \usepackage{./langsci/styles/langsci-osl}
% \usepackage{langsci-gb4e}

\usepackage{tikz}
\usetikzlibrary{patterns,calc}
\pgfdeclarepatternformonly{south east lines}{\pgfqpoint{-0pt}{-0pt}}{\pgfqpoint{3pt}{3pt}}{\pgfqpoint{3pt}{3pt}}{
    \pgfsetlinewidth{0.6pt}
    \pgfpathmoveto{\pgfqpoint{0pt}{3pt}}
    \pgfpathlineto{\pgfqpoint{3pt}{0pt}}
    \pgfpathmoveto{\pgfqpoint{.2pt}{-.2pt}}
    \pgfpathlineto{\pgfqpoint{-.2pt}{.2pt}}
    \pgfpathmoveto{\pgfqpoint{3.2pt}{2.8pt}}
    \pgfpathlineto{\pgfqpoint{2.8pt}{3.2pt}}
    \pgfusepath{stroke}}
    
\usepackage{stmaryrd}
\usepackage{wasysym}
\usepackage{multirow}
\usepackage{caption}
\usepackage{subcaption}
\usepackage{mathrsfs}
\usepackage{qtree}

\usepackage{linguex}


  %pminos do not split footnotes
% \interfootnotelinepenalty=10000 %Footnote in Laporte chapters has to be split SN


%\DeclareIndexNameFormat{default}{%
%\nameparts{#1}%
%\usebibmacro{index:name}%
%{\index[names]}%
%{\namepartfamily}%
%{\namepartgiveni}%
% {}% L1
% {}% L2
%{\namepartprefix}% generates spurious space L3
%{\namepartsuffix}% generates spurious space L4
%}

%  {\DeclareIndexNameFormat{default}{%
%     \usebibmacro{index:name}{\index[names]}{#1}{#3}{#5}{#7}}}

%\DeclareIndexNameFormat{default}{%
%  \usebibmacro{index:name}{\sindex[nom]}{#1}{#3}{#5}{#7}}

%\DeclareIndexNameFormat{default}{%
%  \usebibmacro{index:name}{\sindex[person]}{#1}{#3}{#5}{#7}}
%\DeclareIndexNameFormat{default}{%
%\nameparts{#1} \usebibmacro{index:name}{\sindex[person]]}{\namepartfamily}{‌​\namepartgiven}{\nam‌​epartprefix}{\namepa‌​rtsuffix}}

%\newcommand{\smiley}{:)}

%\renewbibmacro*{index:name}[5]{%
%\usebibmacro{index:entry}{#1}%
%{\iffieldundef{usera}{}{\thefield{usera}\actualoperator}\mkbibindexname{#2}{#3}{#4}{#5}}}

% \newcommand{\noop}[1]{}

%remove for final
%\overfullrule=1mm

\newcommand{\tobi}[2]}}
\renewcommand{\S}[1]{\tobi{#1}{\textsc{*}}}

% this volume references
% puts: [this volume]
% already defined: \citetv
%\newcommand{\citepv}[1]{(\citeauthor{#1} \citeyear*{#1} [this volume])}
\newcommand{\citealtv}[1]{\citeauthor{#1} \citeyear*{#1} [this volume]}

%parentheses around example number
\newcommand{\pref}[1]{(\ref{#1})}

% in-text examples

\newcommand{\lnex}[1]{\textit{#1}} %target lang word
\newcommand{\lnlit}[1]{(lit.: `#1')} %literal reading
\newcommand{\lnlat}[1]{(#1)} % latinization
\newcommand{\lntrans}[1]{`#1'} %translation
\newcommand{\lnexl}[2]%
{\lnex{#1}{} \lnlat{#2}} % ex with latinization
\newcommand{\lnexlat}[3]{\lnex{#1}{} \lnlat{#2}{} \lntrans{#3}} % ex with latinization and tranl.

%ch01
\newcommand{\co}[1]{\mbox{\textbf{#1}}}

%ch09

\newcommand{\cyrbulg}[1]{\begin{otherlanguage*}{bulgarian}#1\end{otherlanguage*}}


%ch10
\newcommand{\nlp}{{\small NLP}}
\newcommand{\mwe}{{\small MWE}}
\newcommand{\rae}{{\small RAE}}
\newcommand{\lvc}{{\small LVC}}
\newcommand{\pos}{{\small P}o{\small S}}
%\newcommand{\todo}[1]{ \textcolor{red}{#1} }

%\renewcommand{\labelenumi}{\theenumi}
%\ainamefmt{{vv}{ll}{, ff}{, jj}} % fullname

\newcommand{\biberror}[1]{{\color{red}#1}}

\newcommand{\osenovaitem}{--~}
  %% hyphenation points for line breaks
%% Normally, automatic hyphenation in LaTeX is very good
%% If a word is mis-hyphenated, add it to this file
%%
%% add information to TeX file before \begin{document} with:
%% %% hyphenation points for line breaks
%% Normally, automatic hyphenation in LaTeX is very good
%% If a word is mis-hyphenated, add it to this file
%%
%% add information to TeX file before \begin{document} with:
%% %% hyphenation points for line breaks
%% Normally, automatic hyphenation in LaTeX is very good
%% If a word is mis-hyphenated, add it to this file
%%
%% add information to TeX file before \begin{document} with:
%% \include{localhyphenation}
\hyphenation{
    Beck-man
    Ngu-yen
    back-chan-nel
    back-chan-nels
    mo-not-o-nous
    ste-reo-typ-i-cal
}

\hyphenation{
    Beck-man
    Ngu-yen
    back-chan-nel
    back-chan-nels
    mo-not-o-nous
    ste-reo-typ-i-cal
}

\hyphenation{
    Beck-man
    Ngu-yen
    back-chan-nel
    back-chan-nels
    mo-not-o-nous
    ste-reo-typ-i-cal
}

  \togglepaper[10]%%chapternumber
}{}

\begin{document}
\maketitle


\shorttitlerunninghead{Jamaican Creole tense and aspect in contact}
%%use this for an abridged title in the page headers


\section{Introduction} \label{sec:messamk:1}
A normally developing child is expected by age 5 to have acquired all the basic constructions that allow for native speaker functionality in the target language. Chomsky, in \citet{cockburn1994goldenage}, speaks of this process of language development as involving inherent cognitive mechanisms that allow us to naturally develop language in much the same way as we grow arms and legs. Yet, the mother tongue has proven not to be impervious to language change and loss. Research in the 1980s saw the advent of language attrition as a new subfield, with one of the first impactful collections of papers being \citet{LambertFreed1982}.

A reduction in input from the L1, a reduction in the use of that L1 as well as influence from an L2 may result in language attrition, which  \citet[11]{DeBotSchrauf2009} define as “ …the loss in language proficiency in an individual over time.” The susceptibility of the L1 to deterioration is further compounded by the presence of two languages in the mind of the L2 user. These languages share a relationship that may be either total separation, interconnection or total integration that \citet{Cook2002,Cook2003,Cook2016} has described as an integration continuum, reflective of a multi-competence model. In this model, Cook postulates that neither total integration nor total separation is possible as, in the case of total integration, the user has the ability to “keep one language at a time”, whilst for total separation, the user is “belied by the use of the same mouth and ears for both languages” (\citeyear[12]{Cook2002}); in other words, their existence in the same mind renders autonomy impossible (\citeyear[7]{Cook2003}).

The Regression Hypothesis (RH), a framework introduced by \citet{Jakobson1941}, is a seminal theoretical model of first language attrition. It posits that the order of this process is the inverse of language acquisition \citep{Schmitt2019}. A perspective of this hypothesis also incorporates the notion of complexity and frequency of use being a determining factor in language loss. The idea here is that features of the language which are learnt best and are frequently used are least susceptible to loss.

With the RH, we can look for parallels that presumably exist between first language attrition and language acquisition. \citet{Keijzer2009} in a study of English (L2) and Dutch (L1) contact reports that the RH holds ground in accounting for loss in the morphological domain, but less so in the syntax. Keijzer makes a distinction between loss and L2 influence, and reports changes in the syntax being marked by L2 influence rather than loss.  Parallels in L2 attrition studies have provided strong evidence in favour of the RH \citep[150]{Hansen1999}. Further investigation in this area is required for L1 studies as the evidence is either sparse or conflicting. However, \citet{Keijzer2010} found that of 15 features investigated in the case of Dutch L2 speakers of English, 9 parallels were found, leading to the conclusion that “regression only on the basis of L1 remodeling does not occur” \citep[223--224]{Keijzer2010}. Further support of the credibility of the RH is found in the work of \citet{slobin1977}, whose findings were later substantiated in Yağmur’s (\citeyear{Yağmur1997}) investigation of L1 attrition in Turkish.

This paper brings together the findings of two studies within a creole environment: one on language acquisition and the other on language attrition. With a focus on the past tense and the progressive aspect, insights are sought on the parallels that may exist in the acquisition and attrition of these areas by L1 speakers of Jamaican Creole (JC). It presents an account of the acquisition of the simple past tense and the progressive aspect in the speech of three-year-olds from JC speaking communities as presented in \citet{Kennedy2017}, primarily. These data are presented alongside attrition data based on \citet{Messam-Johnson2017} as attested in the speech of JC migrants to Curaçao, after residence for between 1 and 21 years in the Papiamentu (Pp)-dominant environment.

We compare the two data sets and find that parallels may be drawn between the interlanguages of these speakers: there are patterns of the mixing of the L1 with the languages in contact in the progressive construction, but little mixing in the past tense, where Creole forms persist in the speech of both sets of speakers.

\sectref{sec:messamk:2} of the paper provides a brief background to the language situations in the relevant communities. \sectref{sec:messamk:3} presents the methodologies of the two research projects. \sectref{sec:messamk:4} provides the theoretical background on tense as well as its occurrence in the acquisition and attrition data. Findings are discussed in terms of whether the outcomes in attrition are in line with the predictions in the literature in terms of L1 acquisition. In cases where there is divergence from these expectations, possible explanations from L2 acquisition are sought. A similar approach is taken with regard to aspect in \sectref{sec:messamk:5}, then \sectref{sec:messamk:6} concludes with the implications of findings for the language arts classroom.

\section{The language situations in Jamaica and Curaçao} \label{sec:messamk:2}
The assumptions here are that JC is the native language of the majority of Jamaicans, and that therefore, the official language English, or more accurately an indigenized variety commonly known as Jamaican English (JE), is spoken as an L2 by that majority. Implicit in this is the further assumption that JC and JE are two different languages (see \citealt[10--13]{Kennedy2017}). They do not exist, however, as discrete languages. Instead, there are several overlapping varieties or codes, so finely articulated that they cannot be identified as discrete codes but have been characterized as a continuous spectrum of speech varieties \citep[350]{DeCamp1971toward}. The spectrum comprises JC and JE at the extremes, with individuals occupying different spans of the spectrum. Indeed, at best JC and JE themselves are idealizations, since it is unlikely that any one speaker would use only forms belonging exclusively to one code or the other. The main sources of knowledge of the idealized JE for most speakers in Jamaica are formal education and writing (\citealt[256]{DevonishHarry2008}), to which they are exposed to varying degrees.

Such a spectrum is possible since the vocabulary of JC is largely derived from English – it is “English lexified”, with words having originated on the slave plantation as English, the language of the colonizers, and incorporated into the JC phonological system. The result is that items of English origin make up the vast majority of the lexicon of JC (\citealt[256]{DevonishHarry2008}). Consequently, JC words can readily be seen to be related to their JE counterparts. In many cases, however, given the spectrum, there exist a variety of combinations of JC and JE forms. An example is the JC \textit{yeside} ‘yesterday’, pronounced variously as \textit{yestide, yestude, yestade, yestudie, yestadie,} \textit{yestudee}, \textit{yesterdie} and \textit{yesterdee}, the JE pronunciation \citep[60]{Kennedy2017}.

Similarities between JC and JE exist, then, at the lexical and phonological levels. These are considered to be superficial (\citealt{Craig1980}; \citealt{Kennedy2017}). A consequence of the superficial similarity between JC and JE, is that speakers often believe that they are using JE when in fact they are not. Speakers believe this, because noting the many similarities in vocabulary, they assume that the languages are similar at the deeper syntactic level as well \citep{Craig1980}. This is not the case, however. We see in \sectref{sec:messamk:4}, for instance, that the way tense operates in JC is fundamentally different from how it operates in JE.

The language situation in Jamaica as outlined above, exists also to varying degrees in other Creole communities, inspiring the term (post-)Creole continuum. In continuum theory, the forms of speech that closely resemble the language considered to be the official language is termed the \textit{acrolect}. The speech forms that maximally diverge from the former are referred to as the  \textit{basilect}. It has traditionally been assumed that communities in rural areas would be more likely to have speakers of more basilectal forms, since such areas are isolated in both geographic and socio-economic terms, remaining therefore relatively immune to more mainstream developments, including, it is assumed, linguistic behaviour.\footnote{See, for example \citet[49]{patrick1999urban} who speaks of isolated rural areas as being culturally and linguistically conservative, and who quotes \citet[23]{Rickford1987} as suggesting that rural variants are “characteristically Creole”.} As we would expect given the variation which exists, however, speakers do not fit neatly into these two categories: instead of consistently using only forms which are considered to be either basilectal or acrolectal, forms reflecting more or less “Creole-ness” are regularly used as well. People using such “midway” forms have been termed mesolectal speakers. \citet[236]{Winford1997re-exam} indicates that the \textit{mesolect} is used to refer to an area of interaction between a relatively basilectal Creole and the local standard, an area which appears to have no distinct status as a system, whose “existence has been and continues to be, dependent on the cross influences from the two extremes” \citep[372]{Craig1971}.

The assumption of this chapter is that the input which children hear as they acquire language is characterized by variation, and will therefore form part of their own speech. However, yet another aspect of language use in the Jamaican society exists. This is also attributable to its emergence on the plantation. The colonizers spoke varieties of English, the language associated therefore with power and wealth, but the Creole which the enslaved developed was generally perceived to be a malformed version of English. Over the centuries, English has maintained its unique position since it remains the only official language of Jamaica, and since it continues to be the expected language for the conduct of government business, in the law courts, in schools as the language of instruction, in the mass media, in religious worship and in all other contexts where written language is required. In contrast, JC, the vernacular, is the expected language for use in private and informal interactions involving family and friends. For these reasons, the language situation has been characterized by some linguists as being \textit{diglossic} (see \citealt{Winford1985} among others) where the language of high prestige used in the formal domain is known as the H(igh) variety, and where the L(ow) variety, the vernacular, is not generally used in that domain.

We do note, however, that as underlined by \citet[231]{DevonishWalters2015}, in recent years, and particularly since Jamaica’s independence in 1962, JC has become a symbol of national identity across all social groups, resulting in a growing acceptance of its use in the public arena in domains where in colonial Jamaica it would have been frowned on. The language situation in Jamaica, therefore, is very complex both in the forms and in the use of the languages.

Similar to many creole languages of the Caribbean, including JC, Pp is the product of European colonization. In Curaçao, this began in the 1500s. The Spanish occupied the island for over a century until Dutch colonisation in 1634--1654. Portuguese-speaking Jews then arrived on the island, resulting in the formation of Pp, a language with Spanish, but predominantly Dutch and Portuguese influence, and the creation of the multilingual situation that exists today.

The study of JC attrition was conducted in this situation (see \citealt{Messam-Johnson2017}), a multilingual country with immigrants from diverse backgrounds. The informants on whose speech the study is based were speakers, who in light of their lower socio-economic status and level of education, among other social factors, were assumed to represent speakers of a variety closer to basilectal JC. They were exposed to different degrees to multiple languages including English, Dutch, Spanish, and Pp the dominant language, the vernacular of Curaçao.

\citet{KouwenbergMurray1994} indicate that most of the Curaçaon population consider themselves to be polygots, having varied competencies in Pp, Dutch, Spanish and English. If one were to encounter a monolingual resident of this country, however, the expectation would be that that individual is a Pp speaker. Though Dutch is the official language of the island, its use is restricted to higher education and government domains, where Pp is also present (\citealt{KouwenbergMurray1994}). Pp, having an official status in Curaçao, is used in the education system and in the print and electronic media as well. Knowledge of this language was a job requirement for some informants, but all had to interact in Pp with employers, fellow employees, and members of the public whom they served at varying levels. Those who were currently in the education system would also have had sufficient exposure to Pp as it is a language of instruction as well as the main means of interaction with fellow classmates.

It is noteworthy, however, that upon arriving in Curaçao, to get around, the Jamaican immigrants would have initially tried to communicate using any proficiency in English they had attained whilst living in Jamaica, as some residents did speak English as an L2, and others as an L3 or even an L4. Upon arrival, then, in a country where speakers of JC do not form a community, the assumption is that the main mode of communication for the immigrants was a variety that assimilated features of the acrolect or the upper mesolect. It is not surprising then that these features, which would normally have been characteristic of the formal domain in the L1 country, would over time form part of the informal speech of the immigrants, often resulting in a replacement of the more basilectal features of the L1.

Though L2 speakers of English exist within the country, the immigrants would not have been able to rely on the use of English to live well within the country. A 2001 census showed that a vast majority of the Curaçaon population are primarily Pp speakers and a minority speak English (see \tabref{tab:messam:1}).

\begin{table}
\begin{tabular}{lrr}
\lsptoprule
& Curaçao, 1981 & Curaçao, 2001\\\midrule
Papiamento & 86.9 & 80.3\\
Dutch & 6.8 & 9.3\\
English & 3.3 & 3.5\\
Spanish & ? & 4.6\\
\lspbottomrule
\end{tabular}
\caption{Census of 1981 and 2001 depicting percentage of language use in Curaçao (adapted from \cite[143]{Maurer1998} and \cite[238]{kester2012language} in \cite{Jacobs2013}).}
\label{tab:messam:1}
\end{table}

With this shift in language dominance, given the Jamaican migrants’ new language situation, Pp was expected to have an influence on the L1 forms that they would produce. It was however anticipated that with a need to communicate with persons in the L2 environment, these respondents, at least initially, would have had to utilize any knowledge of English they had prior to immigration. With an increased use of this English, however, further influence on the L1 would be expected. It was therefore critical that informants who were included in the study spoke a variety of JC that was closest to the basilect prior to migration. This had to be controlled for by selecting informants at the lower strata of the Jamaican society with whom varieties closest to the basilect are said to be associated.

\section{Methodologies} \label{sec:messamk:3}
\subsection{The acquisition data} \label{sec:messamk:3.1}

The acquisition data are drawn from the Child Language Acquisition Research (CLAR) project.\footnote{The project was fully funded by a UWI Mona New Initiatives grant, as well as the award of a year’s sabbatical leave.} The aim of the project was to determine what language variety children speak as they enter the public school system.

A total of 80 children in their first month of Basic School participated in the study. We note that the Basic School system is said to have been created to cater mainly to the lower socioeconomic groups \citep{Miller2015}.\footnote{In 2012 74.2\% of children ages three--five years attending early childhood education institutions in the country were enrolled in Basic Schools (\citealt[15]{SABERCountryReport2013}, table 12).} In addition, though the language situation outlined above would suggest far more complexity than this, the assumption of the Ministry of Education is that the communities feeding such schools are mainly JC-speaking, and that the children are monolingual speakers of JC.\footnote{The \citeyear{MinistryofEducation2001} Language Education Policy, for instance, states explicitly that JC is the language most widely used in Jamaica (p. 23).} Thirteen Basic Schools were chosen from eight areas across the island, with care taken to have representation from schools in rural areas (six schools), cities (three schools) as well as towns (four schools). Though consideration of possible gender effects on the speech of children was not an aim of the study, an attempt was made to have equal numbers of boys and girls from each school, totalling 43 and 37 respectively. Children were not assessed or screened for participation. Instead, they were chosen from among classmates on the basis of age and gender as outlined above, as well as using guidance from their teachers as to suitability in terms of expected levels of participation in interviews.

At the start of data collection half of the children were age 3;0± and the other half 3;6±.\footnote{The age of children is recorded here in the format years;months. A more fine-grained representation includes the number of days as follows: years;months.days. This is in line with the convention in the field of first language acquisition, and will be used throughout.} Each set of children was interviewed once a month for six months, resulting in the collection of a virtual year of the speech of children falling within the age range of 2;9 to 4;2. This method was patterned after \citet{Meade2001}.

A total of 214 half-hour video-recorded interview sessions were conducted between  September 2009 and April 2010 in the school setting (but not in the classroom), by JC native speaking graduate students at the University of the West Indies, Mona campus.  In approximately half of the sessions (108/214), only one child was interviewed. For the remainder, interviews were of two children. The rationale for choosing to interview two children at a time was to allow us to analyse the children in interaction, and for linguistic as well as non-linguistic reasons such as assessing attitudes, fair-play and dominance.

The store of materials used for elicitation included laminated flash cards, story books, toys, colouring books, crayons, scrap books and markers. To allow for role-playing, there was a range of cooking utensils including a wooden stove, pots and food items, as well as telephones. The aim was for the children to interact naturally, and data were elicited primarily via conversation during play. Sessions were loosely structured, beginning with general discussion, encouraged as necessary, using books or pictures. This was intended to set the tone for the session and to put the children at ease; it was followed by guided conversations using the flash cards or objects chosen especially for the elicitation of a range of structures including tense and aspect constructions. The final segment involved the children in various activities such as free play, role-playing or colouring, intended to foster discussion and interaction guided by the interviewer. All told, 51,650 utterances were collected from the children.

Interviews were transcribed from the videos for manipulation in CHILDES, the Child Language Data Exchange System, which provides online tools for transcription and data analysis (\citealt{MacWhinneydatabase2000,MacWhinneyTools2000}). Transcriptions were orthographic, using the Cassidy-JLU writing system, modified to include JE vowels used by the children. Lexicon files were created for the purpose of tagging transcriptions, with coding for tagging adapted to suit the purposes of the analysis, and tagging then achieved using facilities provided by the CHILDES software.

The basis of the analyses is just over one hundred hours of data collected (107). This corpus linguistics approach greatly improved the empirical power of claims made here. As will become apparent, frequency of occurrence and the range and frequency of possible combinations of JE and JC forms play an important role in the interpretation of the data. This has allowed for trends to be detected, and for patterns which may not be immediately obvious to be easily confirmed by the flexible interrogation of the speech of children by gender, age, major region and parish, and by the rural\slash urban status of their communities.

\subsection{The attrition data} \label{sec:messamk:3.2}
The data on the attrition of the constructions presented in this paper are drawn from a broader study, which investigated the susceptibility of JC to attrition in an L2 dominant environment where there is a reduction in the use of the L1 and a reduction in input from that L1.

Data were elicited from 20 Jamaican immigrants to Curaçao, who had been residing in the L2 country for a period of between 1 and 21 years. Respondents had to have resided in Jamaica until they were at least five years old to ensure that the L1 had been acquired, and that any changes evident in the L1 post migration were not simply a result of imperfect acquisition. Informants were then categorised according to their years of exposure to the L2. This was determined by their length of residence in the L2 country: 1--5 yrs, 6--10 yrs and >10 yrs. This categorisation would allow for tracing of the stages of attrition in JC, from the point of initial contact with the L2 (1--5 ys) when mild effects would have been anticipated, to the point at which stability in the attrition process would be likely to be realised (>10 yrs). Having a representation of immigrants with varied lengths of residence in the L2 country would then have provided evidence of the points at which unconventional L1 features would be likely to appear in the repertoire of JC immigrants in an L2 contact situation; the time at which features would become susceptible to change would be revealed. With the inclusion of immigrants whose length of residence exceeded ten years, an overview of an attrited JC grammar would also be possible, as at ten years attrition is expected to be relatively stable (cf. \citealt{DeBotClyne1994}; \citealt{Waas1993}).

The investigation further included a verification group against whose linguistic competence the structures produced by the immigration group were measured. Members of the verification group had to have been native to the L1 country and had to have resided there from birth. These participants had never visited another country and had no immigrants from other countries in their circle. It was expected then that they could provide evidence of structures which would reveal the norms of the L1. They were further selected with an intention of matching the socio-economic profile of the immigration group.

Following \citet{Ellis1994}, five data elicitation methods were used: natural use,\footnote{Natural use data represent the spontaneous speech of research participants.} clinical elicitation, experimental elicitation, metalinguistic judgements and self-report. These data were collected from the 21 immigrants over a three-month period, totalling 128 recorded sessions over 60 hours. Informants were given the option to meet at a location of their choosing to ensure comfort and relaxation throughout the process. This would have assisted in encouraging the natural production of informal speech for which the use of JC is considered the norm. Interaction with the participants was done individually to avoid external influence from other L2 users.

In eliciting spontaneous speech (natural use data), attempts were made to ensure participants spoke as naturally as possible. Participants spoke of their personal lives and on topics which were of interest to them. Utterances produced through natural use are considered to be authentic \citep[671]{Ellis1994} and were thus expected to reveal reliable evidence of L1 change.  This task provided an opportunity to gather data which was a true representation of the participants’ speech. Unconventional L1 forms identified here, which suggested a change in the participants’ L1 repertoire, would be targeted in following elicitation tasks to test if they would be reproduced. This became the pattern with each task administered.

Clinical elicitation further provided informants with the opportunity to be creative in their output, thereby providing more spontaneous speech. This type of data allows for elicitation through mostly unguided language use. Methods used usually include film recalls, written composition, information gap tasks, oral interviews and role plays (cf. \citealt{Ellis1994}: 672). Instruments such as picture sequencing and picture description tasks allowed for the production of targeted structures through unguided language use. Participants were allowed to create their own stories and provide intuitive descriptions of images with which they were presented. With the selected clinical elicitation tasks, it was possible to maintain some amount of control over the types of structures produced as the images presented to the participants targeted particular constructions, including activities, for example, that would likely be described using progressive and past tense forms, among others.

The experimental elicitation method, through the use of a scenario consideration and a translation task, directly targeted structures that were suspected to be prone to attrition on the basis that they were either areas in which unconventional L1 forms were exhibited in the previous data collection tasks, or they were areas of morpho-syntactic or syntactic differences that were identified as existing between the two primary languages in contact -- Papiamentu and JC. Prior to entering the field, items included in instruments for testing were selected for inclusion by contrastive analysis on the premise that changes are likely to be evidenced in the areas in which the languages in contact differ.

Unlike the clinical elicitation method, there was greater control over the structures that informants produced. In the scenario consideration task, participants were presented with scenarios that were structured so as to prompt the use of the targeted structures. The translation task presented participants with Pp sentences that contained these targeted structures for which they were required to orally provide the JC equivalents. In doing so, participants would be less likely to use avoidance strategies in the production of structures that they may deem more difficult to produce. Participants also had to rely solely on their own knowledge of the L1 to produce the required translations.

In metalinguistic judgements, informants are asked to judge the grammaticality of sentences. Through the use of this task, the participants were able to evaluate structures, some of which had been their own productions in earlier tasks. This allowed for an indication of structures which were impermissible in JC and provided alternative structures that were acceptable in their variety. It further provided evidence of what the immigrants knew as opposed to what their speech reflected. This task was later administered to the verification group in a bid to determine the acceptability of the utterances the immigrants produced and accepted in the data collection sessions.

The final method elicited Self-report data which usually involved introspection, retrospection and think aloud tasks. These sessions allowed informants of the substantive group to provide, among other details, personal input on their experience in using the language and how it might have changed.

\begin{sloppypar}
The analysis section of this paper includes examples from the data collected (Natural Use Data, Metalinguistics Judgements, Picture Sequence, Translation Task), which are relevant to the past and the progressive markers and it further specifies the elicitation task in square brackets.\footnote{For clarity, tasks from which the attrition data are extracted are denoted as Natural Use Data (NUD), Metalinguistics judgements (MJ), Picture Sequence (PS) and Translation Task (TT).}
\end{sloppypar}

Transcription began after the completion of each recording session. This allowed for initial analyses to be carried out while data collection was still in progress. With this approach, forms in the natural use data that seemed to deviate from the L1 norms -- based on the literature and the researchers' own native speaker knowledge of the L1 -- were included in instruments used for the data collection methods that followed. This was the pattern followed for other tasks, wherein findings from previous tasks would inform subsequent tasks, which were then amended to test for the reproduction of these questionable structures.


Transcription conventions followed the Cassidy-JLU orthographic system for JC utterances, whereas English and Papiamentu orthographies were used for those languages. Errors in the data and the frequency of their occurrence were identified in the transcriptions manually. Forms were determined to be erroneous or unconventional if they were primarily rejected by the verification group and if those forms could not be accounted for in the available literature on the JC grammar, including \citet{Bailey1966}, \citet{Patrick2004,Patrick2007} and \citet{winford2003book}.

\section{The past tense} \label{sec:messamk:4}
\subsection{The past tense in JC and Pp} \label{sec:messamk:4.1}
Tense comprises categories where time reference is the primary dimension. We may say that it locates an eventuality in time. An eventuality is taken to refer not only to events but also to situations and states. In the following, we will see that JC and JE contrast with regard to Tense, both in the form of tense marking and in the ways the temporal systems operate. We then apply these concepts to the acquisition and attrition data with a view to finding parallels and contrasts between trends in the two phenomena.


The grammatical expression of Tense is different in JE and JC. Indeed, the TMA (Tense, Mood and Apect) category is said by \citet[12]{Hackert2004} to be one of the areas which sets creoles off most visibly from their lexifiers. Basilectal JC uses independent pre-verbal tense markers \textit{e}(\textit{h})\textit{n} and its geographically determined variants \textit{me}(\textit{h})\textit{n}, \textit{we}(\textit{h})\textit{n}, \textit{mi}(\textit{h})\textit{n} and \textit{be}(\textit{h})\textit{n}. The variants \textit{dii} and \textit{did} have traditionally been considered to be mesolectal, used particularly in urban areas, and presumed to have arisen from JE. Note, however, that, like so many other forms, although JC \textit{did} and its variants are similar in form to JE ‘did’, they function differently in the two languages. In JC, this is an independent past tense marker (\textit{im did nuo} ‘he knew’ vs. \textit{im did bai} ‘he had bought’), functioning like other pre-verbal markers in the language. Such functions are discussed below. This contrasts with JE ‘did’ which serves as an emphatic marker in a declarative sentence (‘he did go’ meaning “he definitely went”). Differences in stress accompany these functions: in JC it is unstressed, whereas as a marker of emphasis in JE, it is necessarily stressed.

In JE, every finite verb which represents an eventuality taking place prior to the moment of speaking must be marked for the past tense. Such a language is considered to be tense prominent. JC, like other Creole languages is not tense prominent, since temporal fixing may but need not be determined in the syntax by grammatical marking -- it may be set by other means such as by the inherent meaning of a verb, by the discourse or by adverbs. In his seminal \citeyear{Reichenbach1947} work, Reichenbach constructed a theory of tense structure which has informed the linguistic study of Tense. The components of the theory are the time of utterance or speech (S), the time of the event (E) and the Reference Point (R). Tense is explained in terms of how these time points are related. This model has been applied to the study of temporal interpretation in Creole languages by \citet{Winford2001comparison, Lefèbvre1996,muysken1981} and others.

In the Reichenbach model, absolute tense such as exists in English, locates E in the past with respect to S. In contrast, tense in Creole languages is said to be Relative, that is, E is relative to R, not to S; the typical use of the tense marker is to locate some situation as occurring prior to the point under focus in the discourse \citep[162]{Winford2001comparison}. A determinant of the location of R in Creole languages is the (semantic) class of the verb. This makes it possible for the time of action to be implied merely by the aspects of verbal forms \citep[123]{Bhat1999}. This is the domain of lexical or inherent aspectual properties, \textit{aktionsarten}, which results in default tense interpretations for different lexical classes of verb in the absence of any overt tense marking, and also contributes to the determination of when such marking is present, as we now see.

Stativity and nonstativity are the major classes of verb relevant to the discussion of lexical aspect.\footnote{We acknowledge that the stative\slash non-stative distinction is very broad, and is not adequate to provide a full account of the effects of lexical aspect in creole languages. More fine-grained sub-divisions such as telic\slash non-telic for non-statives, for instance, are explored and applied to creole data by \citet{Gooden2008,mcphee2003} and \citet{Hackert2004}. For the purposes of this article, however, we restrict our discussion to the major stativity divisions and their interactions with past time reference, since that will suffice for laying the foundations for what follows.} As will become apparent, the two aspectual classes of verbs have different temporal interpretations when modified by the past tense marker; in effect, aspect overrides tense. Stative verbs constitute a relatively small class, including verbs such as \textit{nuo} ‘know’, \textit{lov} ‘love’, \textit{waahn} ‘want’, \textit{a}(\textit{v}) ‘have’\footnote{We note that three of these, \textit{waahn} ‘want’, \textit{nuo} ‘know’ and \textit{av} ‘have’, are among the top eight JC stative verbs most used by the children.}, all of which convey (typically continuous) physical or internal states. Following \citet{Vendler1967}, and using semantic criteria, \citet[63]{andersen1990papiamentu} characterizes such verbs quite unsurprisingly as requiring no energy for them to continue once the state has been entered, although the point of entering or leaving the state may be conceived as nonstative, depending on the particular circumstances.

The default tense reading for bare or zero-marked (∅) statives (that is, stative verbs not accompanied by a tense marker) is the present \citep[33]{winford1993predication}.\footnote{The characterization of a non-overtly marked verb as bearing null tense is controversial. Here, without entering the controversy, we follow the generative approach, where every finite verb bears tense, but a tensed verb need not be overtly marked for tense. Even when not overtly marked for tense, tense must still be accounted for in the syntax, and is represented by a null marker (∅), to indicate that it is expressed, though not phonologically. We say that the verb is zero-marked or that it is bare.}

\ea \label{bkm:messamK:1}bare stative: present\\
\gll im ∅  nuo.\\
 3\textsc{s.subj} \textsc{tense} know  \\
\glt ‘He (or she) knows.’
\z


Nonstative verbs such as JC \textit{ron} ‘run’, \textit{push} ‘push’, \textit{jrap} ‘drop’, or \textit{jomp} ‘jump’ require energy for the action or event to take place and to continue (cf. \citealt[63]{andersen1990papiamentu} for reference to their English counterparts). For this reason, such verbs have also been termed \textit{dynamic}. The default tense interpretation for bare nonstatives is said to be the past.

\ea \label{bkm:messamK:2}bare nonstative: past \\
\gll di    man ∅  tiif    i.\\
\textsc{def} man \textsc{tense} steal 3\textsc{s}  \\
\glt ‘The man stole it.’
\z


The interpretational consequences of the use of the tense marker in interaction with stativity are illustrated in (\ref{bkm:messamK:3}--\ref{bkm:messamK:5}). What happens here is that R, the time of Reference, is established by the verb. It is established as present (S) by statives (\textit{si} ‘see’ in \REF{bkm:messamK:3}), and as past -- prior to S -- by nonstatives (\textit{bait} ‘bite’ and \textit{ron} ‘run’ in \REF{bkm:messamK:4} as well as \textit{kik} ‘kick’ in \REF{bkm:messamK:5}). The anterior tense marker serves to shift the event (E) to a point prior to R in each case, resulting in a past interpretation for statives (\textit{dii si} ‘saw’ in \REF{bkm:messamK:4}) and past before past for nonstatives (\textit{wehn bait} ‘had bitten’ in \REF{bkm:messamK:5}).

\ea \label{bkm:messamK:3}
\gll   DAN:   mi   si   wahn   kloud   ina   dis.\\
~ 1\textsc{s}  see  \textsc{indef}  cloud  in  this\\
\glt  \hphantom{DAN:} `I see a cloud in this.’             [V6-MOB-N:426-3;4.6]
\z




\ea \label{bkm:messamK:4}
  \gll ROD: Tishien dii si wahn … daag … de pan i varanda  an … iihn bait Tishien ahn Tishien ron im.\\
~ Tishien  \textsc{past} see \textsc{indef} ~ dog ~ \textsc{cop} on \textsc{def} verandah and ~ 3\textsc{s} bite Tishien and Tishien run 3\textsc{s}  \\
\glt \hphantom{ROD:} `Tishien saw a dog on the verandah and he bit Tishien and Tishien chased him away’   [V1-BAL:342-3;3.25]
\z



\ea \label {bkm:messamK:5}
\gll   ANG:   yie, im   wen  bait mi  an   mi  kik  af  a mi han.\\
 ~ yes 3\textsc{s}  \textsc{past} bite 1\textsc{s}  and 1\textsc{s}   kick 3\textsc{s} off 1\textsc{s} hand\\
\glt \hphantom{ANG:}  `Yes, it had bitten me and I kicked it off my hand.’       [V3-MOB:676-3;4.22]
\z




Other than the inherent aspectual properties of the predicate, R can take its reference from the discourse, so that once the discourse context establishes R to be past, unmarked statives can have past time reference \citep[396]{Winford2000tenseaspect}. In story-telling, for instance, once past time reference has been set, that can be assumed to persist until the speaker notifies otherwise \citep[159]{Winford2001comparison}, and the events that make up the main narrative are conveyed by unmarked verbs \citep[404]{Winford2000tenseaspect}.

Following \citet{ChungTimberlake1985}, \citet{Gooden2008} uses the term tense locus (TL) to refer to the point in time in relation to which an event or state is perceived as past. TL may or may not be the time of speech (S), but it is located relative to S. \citet[322ff.]{Gooden2008} provides Belizean Creole data which show that in discourse, once the reference point is set in the present, the past tense marker can be used to distance the situation described by both states and events from speech time, resulting in an absolute past reference.

The marking of tense in Pp follows a pattern similar to that of JC, insofar as an independent pre-verbal marker indicates the simple past. Much like JC also, the choice of preverbal marker in Pp is dependent on the stativity of the verb{. However, unlike JC,} a past marker is obligatorily present for past reference: the regular past is marked by perfective \textit{a} and is restricted to non-stative verbs \REF{bkm:messamK:6}, whilst \textit{tabata} marks past reference for the stative verb \REF{bkm:messamK:7}. A reduced form of \textit{tabata} is used with the Pp verb \textit{tin ‘have’,} resulting in the form \textit{tabatin,} exemplified in \REF{bkm:messamK:8}:

\ea \label{bkm:messamK:6}
\gll   Ayera      mi  a      come pan\\
 Yesterday \textsc{1s}  \textsc{past} eat    bread\\
\glt  `Yesterday, I ate bread.’
                [Adapted from \citealt{Goilo2000}: 54]
\z



\ea \label{bkm:messamK:7}
\gll      Mi tabata ke kuminda\\
1s \textsc{past} want food\\
\glt `I wanted food.’
            [\citealt{KouwenbergRamos2007}:309]
\z



\ea \label{bkm:messamK:8}
\gll   Mi  tabatin      cincu buki riba mi     mesa\\
 1\textsc{s} have.\textsc{past} five   book on   \textsc{poss} table\\
\glt  `I had five books on my table.’
                [Adapted from \citealt{Goilo2000}: 59]
\z



The marker \textit{tabata} may also occur with non-stative verbs, but only allows for a past imperfective interpretation, as will be shown in section \sectref{sec:messamk:5}, where the progressive aspect is of focus.


\subsection{L1 acquisition of the JC past tense}\label{sec:messamk:4.2}

All subjects in the attrition study were presumed to be native speakers of JC, the youngest having migrated to Curaçao at age five. In this section, we look to data from the CLAR children regarding trends in the first language acquisition of the past tense in JC and to provide a basis therefore for investigating similarities and contrasts with the attrition process.

\citet[95ff]{DeLisser2015} studied the early acquisition of JC by six children in the age range 1;6--3;4 from Creole-speaking communities in Western Jamaica. She makes the point (p. 96) that because the unmarked verb is used to express both a past and a present reading, it is not possible to determine the exact point at which children acquire the concept of tense. She reports that of 5,836 utterances with a past time interpretation, only 33 (or 0.6\%) were used with an overt tense marker to express the simple past tense.

The CLAR children were older (2;9--4;2) than those in the De Lisser study, but there is nonetheless a reported sparse use of the markers: pre-verbal \textit{wehn} (seventeen (17) instances) and \textit{did} (124) with its phonologically reduced variant \textit{dii} (31), totalling 172. There is variation in the use of these forms. All but one of the occurrences of \textit{wehn} were by CLAR children in the West, as might be expected, but these children were also responsible for 25.8\% of the uses of \textit{did} and 41.9\% of \textit{dii}. There are cases of the use of \textit{did} and its variant \textit{dii} with non-stative verbs conveying a past-before-past reading; \REF{bkm:messamK:9} is an example.

\ea \label{bkm:messamK:9}
\gll   ROD:   Kim dii     bai wahn   chriichip    fi   mi       aahn rat bait  i.\\
~ Kim \textsc{past} buy \textsc{indef} CheezTrix for 1\textsc{s.obj} and   rat  bite 3\textsc{s.obj}\\
\glt \hphantom{ROD:}  `Kim had bought a pack of CheezTrix for me and a rat chewed it.’
[V1-SER:l 145-3;3.25]
\z



The stative verb did appear without the marker to convey a past meaning, as in \REF{bkm:messamK:10} below. In adult JC, the marker would need to accompany \textit{av} for a past interpretation:

\ea \label{bkm:messamK:10}
\gll   ASH:   mi mada ∅  av wan a dem kyaar ya we ∅  mash op.\\
~ \textsc{poss} mother \textsc{past} have one of \textsc{dem} cars there which \textsc{past} mash up\\
\glt  \hphantom{ASH:}       ‘My mother had one of those cars which mashed up.’
[V3-KN2:l 63-3;9.2]
\z




Later in the same interview (l. 1033), ASH uses the past marker with the same stative \textit{av} ‘have’, as would be required in adult JC, and a non-stative \textit{mash op} ‘destroyed’ with no marker, both indicating the absolute past as in \REF{bkm:messamK:11} below:

\ea \label{bkm:messamK:11}
\gll ASH:   mi dii av wan a dem biebi … it ∅ mash op.\\
~ 1\textsc{s} \textsc{past} have one of \textsc{dem} baby-\textsc{plu} ~ 3\textsc{s} \textsc{past} mash up\\
\glt `I did have one of those babies … it got destroyed.’
\z




Variation is evidenced throughout. An example is \REF{bkm:messamK:12} below, where SHA, in explaining what took place on a TV show, first uses the tense marker as would be required, then corrected himself, and produced an unacceptable sentence with the marker omitted:


\ea \label {bkm:messamK:12}
\gll SHA:    <den im did> [/-] im ∅ av aan dis pan i an dehn dis ∅ krash dong dis pahn ort.\\
~ <then 3\textsc{s} \textsc{past} > he \textsc{past} have on this on \textsc{poss} hand then this \textsc{past} crash down this on earth\\
\glt  `He had this on his hand, then this crashed down on the earth.’
[V6-KN1:l 390-3;10.0]
\z


In the dialogue in \REF{bkm:messamK:13} below, the interviewer believes that GAB is speaking about a man currently in her life, until she reveals that he had been killed. The interviewer’s question \textit{so wier im liv}? is very clearly present tense, following another question also in the present tense. Nonetheless, the child begins a story about the man, using the same form of the verb (stative \textit{liv} with no marker) to express the past tense. Interestingly, GAB’s sentence would be acceptable in (adult) JC with the interpretation she intended. It may be that the past interpretation of stative \textit{liv} is coerced by the necessary interpretation of \textit{kil} as past, and allowed by the possibilities available in the discourse.

\ea \label{bkm:messamK:13}
\gll INV:   so im liv   wid   yu?\\
~ so 3\textsc{s} live with 2\textsc{s}\\
\glt \hphantom{INV:}   `So, does he live with you?’

 GAB:  \textit{uhnuhn}.
\glt  `No.’

\gll   INV:   so wier    im liv?\\
~ so where 3\textsc{s} live\\
\glt \hphantom{INV:}  `So, where does he live?’

\gll   GAB:   im ∅  liv de    a  im yaad an   poliis  kil  im.\\
    ~ 3\textsc{s} \textsc{past} liv \textsc{loc} at 3\textsc{S} yard  and police kill 3\textsc{s}\\
\glt \hphantom{GAB:}  `He lived at his home, and the police killed him.’ [V6-SER:l 283-4;0.0]
\z



This may now shed some light on the interpretation of \REF{bkm:messamK:10} and \REF{bkm:messamK:12} above, pointing once more to the possibility of TL being set by the discourse. The sparsity of use, and the variation in both the form and the usage of the past tense by these L1 acquirers of JC may suggest more reliance on strategies used in the discourse than that used by adult speakers and pointing therefore to late acquisition of how the temporal system works in JC. Because aspects of L2 acquisition are relevant to discussions of attrition (see \sectref{sec:messamk:4.4}), and because Jamaican migrants would have been exposed to English before as well as after migrating to Curaçao, we now look at features of the L2 acquisition of the English past tense by the CLAR children with a view to investigating whether trends found in these data might shed some light on the attrition phenomena.



\subsection{L2 acquisition of the English past tense} \label{sec:messamk:4.3}

As indicated in \sectref{sec:messamk:4.1} above, for the past tense to be expressed in JE there must be tense marking on the verb regardless of contextual grounding. Marking of the regular past is by way of an inflectional suffix (‘-ed’) which is attached to the verb stem, and realized differently across words (e.g. /kuk/+/t/ but /beg/+/d/). Irregular past tense forms are created in an unpredictable fashion (e.g. ‘know’ → ‘knew’).  As we have also seen, marking apart, Tense operates very differently in JE and in JC, and it is not the case that the child learning JE must learn simply that there are inflectional endings on the verb in that language. Instead, the child must learn that the temporal systems differ in important ways. This will impact the JC native speaker’s production and understanding of JE. This becomes relevant as the findings from language attrition are explored below, and as we consider the implications which language acquisition and language attrition have for the language arts classroom (see \sectref{sec:messamk:6}).

L2 acquisition of the regular English past tense is a late acquisition and is in fact among the last morphemes to be acquired. Much like in the L1 acquisition of English, the irregular forms are predictably acquired earlier than the regular forms.\footnote{For a discussion of L1 and L2 morpheme order studies, see \citet[67ff]{Meisel2011}.} From a processing point of view this has been accounted for by calling on a dual-system mechanism, which posits that regular past tenses involve the acquisition of rule-based mechanisms whereas irregular past tenses are directly retrieved from memory since they exist separately from their stems in the lexicon (\citealt{PliatsikasMarinis2013}: 4). This is supported by studies finding that frequency effects apply to irregular verbs, in that more frequent forms are understood and produced faster than less frequent ones, but are not common in regular verbs (ibid, 5).

In line with such findings, the CLAR children, the oldest of whom was 4;2, displayed sparse use of the English past tense, and particularly of the inflected forms. We might also look to the L1 itself to explain this late acquisition: \citet[698]{ParadisMarquis2008} speak to differing acquisition by children whose initial state grammar might be close to the target than those whose initial state grammar is not. Since JC does not have inflectional morphology for tense, this would presumably put L1 JC speakers at a greater disadvantage than speakers of other L1s with the morpheme.\footnote{Interestingly, for the L2 learning of English by child and adult speakers of Cantonese, a language where, much like JC, tense is not represented by an affix attached to the verb, \citet{YangHuang2004} report that as proficiency increased, learners did switch gradually to marking verbs for tense using inflections, relying less on pragmatic and lexical devices for expressing tense, and more on grammatical devices.}

The findings were as follows. Only two verbs inflected with the regular JE past inflectional ending ‘–d’ were attested in the CLAR data. In \REF{bkm:messamK:14}, NAT is speaking about going to the beach. He uses the stem \textit{waant} ‘want’, which must be considered a JE form since the JC counterpart \textit{waahn} has a final nasalized vowel, not a consonant cluster.

\ea \label{bkm:messamK:14}
\gll   NAT:   an   a            waant-ed     to   kach  shel. \\
~ and 1\textsc{s}.\textsc{subj} want-\textsc{past} \textsc{inf}  catch shell\\
\glt  \hphantom{NAT:} `And I wanted to get shells.’        [V4-STJ:l 443-3;8.23]
\z

\begin{sloppypar}
The second instance of JE past inflection is an instance of over-generalization, shown in \REF{bkm:messamK:15}. Over-generalization of the English past tense marker is well-attested universally in L1 acquisition, and reported to be “perhaps the most notorious error” in language development among English-speaking children \citep[81]{Marcus1996}.
\end{sloppypar}

\ea \label{bkm:messamK:15}
\gll   DAN:   di    boi  mada    kom-d. \\
  ~ \textsc{def} boy mother  come-\textsc{past}\\
\glt \hphantom{DAN:} `The boy’s mother came.’           [V6-MOB:l 627-3;4.16]
\z



In this case, the past inflection is attached to a shared form \textit{kom} ‘come’, a verb with irregular past tense formation in JE. Presumably the ending was considered by the child to be usable in JC and JE alike. \REF{bkm:messamK:14} and \REF{bkm:messamK:15} were the only occurrences of a regular past tense form. The CLAR children used irregular JE past forms, but rarely so: eight different verbs a total of twelve times: \textit{bawt} ‘bought’ (one instance), \textit{brawt} ‘brought’ (one), \textit{brook} ‘broke’ (two) , \textit{got}  (three), \textit{had} (one), \textit{jangk} ‘drank’ (one), \textit{keem} ‘came’ (one) and \textit{sed} ‘said’ (two), all used in past tense contexts.\footnote{\citet{GrabowskiMindt1995} provide a corpus-based list of the most frequently used irregular verbs in two major corpora of the English of adult speakers. The list includes all of these irregular past tense forms with the following rankings: ‘bought’ \#48, ‘brought’ \#18, ‘broke’ \#43, ‘got’ \#8, ‘drank’ \#70, ‘came’ \#5 and ‘said’ \#1; ‘had’ was not included in this ranking, since it had a far higher absolute frequency count (4.5 times higher) than ‘said’, the highest to be ranked. There are no similar studies for frequency of use in Jamaica.}

\subsection{The past tense in attrition} \label{sec:messamk:4.4}

Given the late acquisition of tense in the CLAR children’s L1, we would expect that for attrition, tense would be highly susceptible to change in the early stages of the onset of this process, if we were to adopt the postulation made by the RH that that which is acquired earlier is lost last, and vice versa (see \citealt[16]{KöpkeSchmid2004}). Further discussion on this hypothesis and the implication it has for acquisition and attrition ensues as the attrition of the past marker is examined. Contrary to expectations, the attrition data reveal that past marking is hardly likely to be subject to language attrition in the presence of influence from an L2 and in the face of reduction in both input and use of the L1. Tense remained intact, showing minimal evidence of attrition.

In the attrition data, deviation from the pattern of tense marking by JC native speakers was exhibited predominantly through the absence of a tense marker where one was required, and further through the rejection of the overt marking of tense in metalinguistic judgements. The first evidence of deviation was the production of a structure indicating past reference of a stative verb, without the inclusion of a pre-verbal tense marker. In these cases, the absence of overt tense marking rendered the past interpretation recoverable through an adjoining clause or the context of the discourse (illustrated in \REF{bkm:messamK:16} and \REF{bkm:messamK:17} below). Though evidenced in the speech of five informants, the structure was only used more than once by an informant with 16 years exposure to the L2. This is reminiscent of the early stages of L1 acquisition, where, as discussed in \sectref{sec:messamk:4.2}, there was a sparse use of the tense marker by the CLAR children, resulting in the interpretation of past reference for some of their utterances being understood through context.

\ea \label{bkm:messamK:16}
\gll  If yu ∅  waahn  nuo,   yu wuda           kom.\\
if 2\textsc{p}  \textsc{past} \textsc{perf/past} want      know 2\textsc{p}  \textsc{cond} \textsc{perf} come\\
\glt `If you had wanted to know, you would have come\slash If you wanted to know, you     would have come.                           [9yrs- Informant, SCT]
\z



\ea \label{bkm:messamK:17}
\gll  Im ∅        jronk? \\
3\textsc{p}  \textsc{past}  drunk\\
\glt `Was he drunk?’                   [13yrs- Informant M, NUD]
\z




In \REF{bkm:messamK:16}, for a past perfect or past interpretation to be allowed in the subordinate clause, a variant of the past marker is required. In the absence of the pre-verbal marker with stative \textit{waahn} ‘want’, a construction that conveys reference to a present state is produced for that clause. In considering the complete utterance as intended by the speaker, a past interpretation is recovered through the main clause. Despite the absence of a tense marker preceding non-stative \textit{kom}  ‘come’, the conditional perfective \textit{wuda} ‘would have’, enables the listener to place the preceding subordinate clause \textit{If yu waahn nuo…} ‘If you want to know…’ within the past for an ‘\textit{If you wanted to know/If you had wanted to know’} interpretation. In JC, a variant of the past marker such as \textit{wehn} would have preceded the verb, resulting in the construction \textit{wehn waant} ‘wanted/had wanted’.

A similar construction is realized in \REF{bkm:messamK:17} with the adjectival predicate \textit{jronk} ‘drunk’. Here, the speaker is enquiring about the mental state of the driver of a car who had been involved in an accident that had occurred 18 years earlier. However, without the context of the discourse, the past interpretation intended by the speaker is irrecoverable. Though context assists in this interpretation, the JC conventions in such instances are reliant on structure, not context.

Further evidence that this is a case of attrition in the use of the past tense marker is provided by the informants’ rejection of the tense marker in constructions that participants of the verification group deemed acceptable. Sentence \REF{bkm:messamK:18}, for example, is an item included in the metalinguistic task that was administered to both the immigrants and the verification group. Sentence \REF{bkm:messamK:19} represents a ‘correction’ of this acceptable task item. The informant does not reject the presence of a tense marker, when presented with the task sentence but rather rejects the particular variant included in the task item, with a request that \textit{wehn} be substituted for \textit{did}. The informant further modifies the placement of the marker relative to the copula \textit{a}. The rejection of acceptable JC variants was not evidenced in the verification group, regardless of the variant that existed in their variety of JC.

\ea \label{bkm:messamK:18}
\gll  Maas   Juo a    wen    wahn  gud   man.\\
Mister Joe \textsc{cop} \textsc{past}  \textsc{indef}  good man\\
\glt `Mr. Joe was a good man.’  [MJ, 135]
\ex \label{bkm:messamK:19}
\gll  Maas   Juo did   a      wahn  gud   man.\\
Mister Joe \textsc{past} \textsc{cop}  \textsc{indef} good man\\
\glt `Mr. Joe was a good man.’      [1yr 6mths, MJ]
\z



The informant in \REF{bkm:messamK:20}, however, in tandem with the covert tense marking previously discussed, rejects the presence of the tense marker, and in correcting the perceived erroneous construction, suggests a structure which would normally be interpreted as present.\footnote{Notable as well, but not relevant to this discussion, is the deletion of the indefinite article.}

\ea \label{bkm:messamK:20}
\gll  Maas   Juo ∅  a      gud   man. \\
Mister Joe \textsc{past}   \textsc{cop}  good man\\
\glt `Mr. Joe was a good man.’             [9yrs- Informant K, MJ]
\z





A loss of variation may be acknowledged as a feature of reduced proficiency or loss in one’s L1.  Instances of this are paradigm levelling and a loss of stylistic options. With the informants here, unlike those who formed the verification group, there is seemingly a rejection of variation in the past tense.


The RH postulates that the processes of language acquisition and language attrition are parallel, in that attrition mirrors acquisition. This hypothesis, being one of the earliest hypotheses accounting for attrition processes was later modified, theorizing that that which is learnt first is lost last and that which is learned best, used often and thus reinforced, is lost last (cf. \citealt[16]{KöpkeSchmid2004}). By this logic, it would have been expected that the past tense form would have fallen susceptible to attrition in the early-mid stages of L1 attrition given its late acquisition by the CLAR children.

The attrition of the tense maker in JC cannot be claimed to be representative of attrition that is to be expected at any stage of attrition between 1--21 years, however, as the attrition of this feature is not convincingly characteristic of any of the established groups in the JC L1 attrition study. For a feature to have been labelled as being characteristic of a group or stage, prevalence of that unconventional feature in the speech of three or more informants within any one category needed to be evidenced. No three participants of either category had recurring instances of the effects discussed here.

The resistance of the past tense marker to L1 attrition may not be surprising given the revelation made by \citet{Keijzer2010} in her study of the attrition of L1 Dutch. Though Keijzer’s study found 9 parallels in language acquisition and language attrition of 15 features investigated in her study of Dutch-English contact, the author states that regression solely on the basis of L1 remodeling does not occur in the simple past. She suggests that other variables must be considered including, notably, L2 influence.

It would stand to reason that if this feature is acquired late in the L2 (Pp), then its influence on the attriter’s L1 would be delayed or less likely to occur. The data available on the CLAR children have revealed that the past construction is also acquired late in L2 acquisition, with only two instances of an inflected verb indicating past reference occurring in the data. If applicable to adult second language acquisition, successful maintenance in this area by the attrition group may be explained. \citegen{DulayBurt1973} Accuracy Order of Grammatical Morphemes in L2 acquisition, places proficiency in tense at a late stage of the process, being ranked number 6 of 8 stages. The stability of the JC past tense form may then be a result of a number of considerations, including first, the imperfect acquisition or a lack thereof of the English tense marker. This is worthy of consideration, as attrited data covering other areas, such as the progressive below, reveal that in the case of JC attrition, effects of L2 influence may take the form of English markers being adopted.

As explained earlier in \sectref{sec:messamk:2}, the language situation within the Jamaican community is one where JC exists alongside English on a creole continuum. Jamaicans acquire JC as their native language and learn English as an L2; however, the marking of tense is a problematic area in adult L2 speech, especially for those whose language would span the range of the continuum that the informants’ speech represents. Though formal study on the L2 adult acquisition of JE past tense is lacking, as shown with the CLAR children as well as other research on the L2 acquisition of the English past, it is acquired late (cf. \citealt{DulayBurt1973}).  Upon arriving in Curaçao, the JC immigrants would have opted to use whatever English proficiency they possess to get around the island. If they had not acquired the English rule, however, an English form could not have been adopted. There were no instances of the inflected verb in the attrition group’s representation of JC. There were however instances of the irregular past included in the attrition data. This bears similarity to the acquisition process, which revealed more instances of the irregular past than the regular past. It is acquired earlier in L2 acquisition, thus may be more likely to make an impact in L1 attrition.

It is noteworthy, however, that as discussed in \sectref{sec:messamk:4.1}, the past tense marker in Pp is realised as an obligatory independent morpheme, not an inflection. This brings us to the second more plausible reason for the retention of the past tense marker, which may be a confounding factor to consider when investigating the parallels between language acquisition and language attrition. If tense is mandatorily marked in Pp, influence from this L2 would facilitate the retention of the marker, rather than its loss. It stands to reason, however, that for some attriters, whose number would be fewer than those who retain the marker, there would be a tendency to unconventionally omit it. A plausible alternative interpretation is that it could be due to the fact that they are unsure about the use of the marker and thus overgeneralise omission as a strategy.

This may be representative of attrition through omission that may be independent of L2 influence. \citet{SeligerVago1991} refer to changes such as these as internally induced language change, which may take the form of simplification, regularization, naturalness, intra-linguistic effects or conceptual/cognitive/innate strategies, for instance. These changes are stated to be motivated by universal principles or are ‘related to some fact in the particular grammar of the L1’. This may be the seemingly optional nature of the occurrence of the tense marker in JC.

In consideration of the role of Pp in the attrition of the simple past in JC then, it may be that the obligatory nature of pre-verbal tense marking in Pp may have impacted the retention of the JC tense marker. Instances of loss without replacement may have been a result of internal language change, where in attrition, there is a tendency of simplification of the grammar, which may be exhibited as loss without compensation.

\section{The progressive aspect} \label{sec:messamk:5}
\subsection{The progressive aspect in JC and Pp } \label{sec:messamk:5.1}

 Aspect concerns the temporal structure of the situation, its internal make-up, or as \citet[3]{Comrie1976} puts it, it refers to the different ways of viewing the internal temporal constituency of a situation. Unlike tense, aspect does not locate situations in time. It is not unconnected with time, but it does not relate the time of a situation to any other time-point.  In this section, the discussions of aspect in acquisition and attrition centre around the progressive aspect. Progressive aspect expresses duration over some period of time, however short. “John was reading” is an example in English provided by Comrie, where “reference is made to an internal portion of John’s reading, while there is no explicit reference to the beginning or to the end of this reading” (\citeyear[4]{Comrie1976}).

Progressive aspect may be expressed syntactically in JC using the pre-verbal marker \textit{a} or its (rural) variants \textit{de} or \textit{da} followed by the invariant verb, so that in \REF{bkm:messamK:21} below, the event (the twisting of the doll’s hair) is presented as one taking place (at the present time), an eventuality which would take the form ‘You are twisting …’ in JE.

\ea \label{bkm:messamK:21}
\gll   TAS:   yu     a       twis  i      dali   ier.\\
    ~ 2\textsc{s.subj} \textsc{prog} twist \textsc{def} dolly hair\\
\glt  \hphantom{TAS:} `You are twisting the dolly’s hair.’           [V1-SMR:l.704-3;3.11]
\z


The English progressive is expressed with a form of the auxiliary ‘to be’ agreeing with the subject in number and person, followed by the lexical verb inflected with the invariant participle ‘-in(g)’ as in ‘He is playing’. This contrasts with the JC form where, as we have seen, both the progressive marker and the lexical verb are invariant.


The marking of the progressive in Pp bears similarity to English whereby, in contrast to JC, non-past \textit{ta} and past \textit{tabata} may combine with suffixal \textit{{}-ndo} in progressive constructions as exemplified in \REF{bkm:messamK:22} and \REF{bkm:messamK:23} below \citep[310]{KouwenbergRamos2007}. The English equivalent would have been ‘is’ and ‘was’ in conjunction with suffixal ‘-ing’


\ea \label{bkm:messamK:22}
\gll  E     kos    ta      bay-endo  hopi leu. \\
\textsc{def} thing  \textsc{pres} go-\textsc{prog} very far\\
\glt `The problem is becoming insurmountable.’        [\citet[310]{KouwenbergRamos2007}]
\z




\ea \label{bkm:messamK:23}
\gll  … den ora  nan di marduga ora   nos  tabata patruyando  kaya.\\
      ~ in   hour \textsc{pl}   of dawn       hour 1\textsc{pl} \textsc{past}    patrol-\textsc{prog} street\\
\glt `… in the early hours when we were patrolling the streets.’
 [\citet[406]{Maurer1988} in \citet[311]{KouwenbergRamos2007}]
\z




{Unlike English, however, an inflected verb in such a construction is not obligatory:}

\ea \label{bkm:messamK:24}
\gll  Mi   tabata kana.\\
        1\textsc{sg}  \textsc{past}   walk\\
\glt `I was walking/I have walked.’
[\citealt[61]{KouwenbergLefebvre2007}]
\z


\subsection{Aspect in acquisition} \label{sec:messamk:5.2}

The progressive construction was used by children of all ages from all regions, over 3,000 times. Early acquisition of the JC progressive aspectual marker by Jamaican children is also reported by \citet{DeLisser2015}: children produce the construction as early as 1;9.5 (p. 106), and use it with five different predicates, indicating consistent use, as early as 1;11.12 (p. 115).

In addition to the regular JC progressive construction, the bare verb (the verb without the JC progressive marker) is used throughout by the CLAR children in subjectless answers to questions using the progressive, as illustrated in the responses below by the youngest, and the oldest children in \REF{bkm:messamK:25} and \REF{bkm:messamK:26}, respectively.

\ea \label{bkm:messamK:25}
\gll INV:   wa    im          a       du?\\
  ~ what 3\textsc{s.subj} \textsc{prog} do\\
\glt  \hphantom{INV:} `What’s he doing.’

\gll   ARI:   iit  xxx.\\
    ~ eat xxx \\
\glt  \hphantom{ARI:} `Eating xxx.’                   [V1-STT:l 141-2;9.0]
\z



\ea \label{bkm:messamK:26}
INV:  So what is this boy doing now?\\
\gll DEV:   brosh ihn         tiit.\\
    ~ brush 3\textsc{s.poss} teeth\\
\glt \hphantom{INV:} `Brushing his teeth.’                [V6-SMR;l 435-4;3.1]
\z

The absence of the aspectual marker cannot be said, however, to reflect necessarily either a lack of aspect marking as such or incomplete acquisition, since alongside the use of the bare verb, the youngest child in the CLAR study aged 2;9.0 made use of the (adult) JC form \textit{a} + verb. Its absence may simply illustrate a common conversational strategy used in both adult JC and JE, particularly where it is given as an answer to a question about what someone is doing, as in \REF{bkm:messamK:24} and \REF{bkm:messamK:25} above.

The progressive construction may be expressed in the past by combining the progressive marker with the past tense marker \textit{dii}, \textit{did}, (\textit{b})\textit{ehn} or \textit{wehn} \REF{bkm:messamK:27}, but the tense marker need not be used if contextual grounding in the past is clear, as seen in \REF{bkm:messamK:28}.

\ea \label{bkm:messamK:27}
\gll OSS:  …  mi         mada    wehn de     krai.\\
 ~ ~ 3\textsc{s.poss} mother \textsc{past}  \textsc{prog} cry\\
\glt \hphantom{OSS:} `My mother was crying.’              [V5-STJ:l.327-3;7.14]
\z



\ea \label{bkm:messamK:28}
\gll DON:   ye   kaa        mi          si   im       wen   taim im         a        sliip lef       di   biebi.\\
    ~ yes because 1\textsc{s.subj} see 3\textsc{s.obj} when time 3\textsc{s.subj} \textsc{prog} sleep leave \textsc{def} baby\\
\glt \hphantom{DON:} `Yes, because I saw him the time that he was sleeping, leaving the baby.’
                  [V6-KN1-J:l.493-3;11.12]
\z


A past tense marker was used in only 7 cases, or 1.1\% of all progressive \textit{a} constructions, and its use must therefore be considered to be rare overall; in all of these instances, the marker used was \textit{did} or \textit{dii}.

We now look at the acquisition of the JE progressive. With regard to the use of this form, \citet{Stewart2010} shows that the children began using structures with the morpheme \textit{`-ing’} at age 2;5. Certainly, as we have seen, the youngest children in the substantive CLAR study were already using the construction regularly.\footnote{\citet{Stewart2010} served as a pilot for the CLAR study, hence the use of ‘substantive’; Stewart began publishing as Kennedy in 2013.}

The CLAR children use the JE auxiliary \textit{iz} ‘is’ followed by an uninflected (bare) verb thirty-six times, with just under a half of these (41.7\%) by children from rural areas. This use of \textit{iz} + bare V is not associated necessarily with the age of the children. There is one child at age 3;5.22 who used JE \textit{iz} + bare V in a progressive construction for the first time in his last recording, but all others who use the construction make use also of \textit{iz} + V-in(g).\footnote{The choice of \textit{–in} or \textit{–ing} is not considered to characterize a speaker as being a JC or a JE speaker, and so is not factored in here. Both variants exist in both languages. Indeed \textit{–in(g)} does not appear as one of the ten load-bearing oppositions proposed by \citet{irvine2005} as signaling an (in)ability to speak JE. As might be expected, 75.7\% of the inflections used were \textit{–in}.} The youngest child to do so was 3;1.29 and the oldest, 4;2.1. In fact, instances of \textsc{V-}in(g) without the auxiliary are used alongside, and twice as often as JE \textsc{prog} + V-in(g), even by the oldest children. The variation appears, then, to be characteristic of the speech of the children, as indeed it is in the input.

\subsection{Aspect in language attrition} \label{sec:messamk:5.3}
In the case of JC attrition, stages are noticed in the changes that occur in the progressive construction. The JC progressive aspect construction stands in contrast to the simple past with attrition effects being evidenced in three ways. The most pervasive effect is the marking of the present progressive through an inflected verb, in the absence of an independent progressive marker. In (\ref{bkm:messamK:29}--\ref{bkm:messamK:31}), \textit{waakin} ‘walking’, \textit{livin} ‘living’ and \textit{avin} ‘having’ are examples, where the English suffixal \textit{-in} marks the progressive. This contrasts with the JC use of the progressive aspectual \textit{a} and the bare verb.


\ea \label{bkm:messamK:29}
\gll  Mi   no    waahn  waak-in        roun …\\
1\textsc{sg} \textsc{neg} want    walk-\textsc{prog} round\\
\glt `I do not want to be walking around …’                 [8 yrs, NUD] \\{}
[JC: Mi no  waahn fi a waak roun]
\z


\ea \label{bkm:messamK:30}
\gll  Di    pikni      dem a        mek   siks yier  liv-in      ina Kyuuraso.\\
\textsc{def} children \textsc{pl}     \textsc{prog} make six  year liv-\textsc{prog} in   Curaçao\\
\glt `It will be six years since the children have been living in Curaçao.’     [16 yrs, MJ]\\{}
[JC: A go siks yier nou sins di pikni dem a liv ina Kyuuraso]
\z



\ea \label{bkm:messamK:31}
\gll  Dem  fren    dem av-in       wahn   hapi    die.\\
\textsc{poss} friend  \textsc{pl}   have-\textsc{prog}  \textsc{indef}  happy day\\
\glt `Their friends are having a happy day.’                   [19 yrs, PS]\\{}
[JC: Den fren dem a av wahn gud die]
\z


Within the Jamaican situation, the use of \textit{{}-in} to mark the progressive is English-influenced and is therefore typical of mesolectal varieties closer to the acrolectal end of the continuum. The use of these varieties in everyday informal speech does not correlate with the linguistic profiles of the participants as also evidenced through the rejection of this form by the verification group, whose grammar is taken to span a similar range of the language continuum. In the formal domains of the Jamaican society, Jamaica’s diglossic situation would have rendered the prevalent use of English-influenced forms such as those exhibited with the progressive construction an expectation or norm. It is therefore no surprise that these forms exist in the linguistic repertoire of the immigrants, and that the CLAR children would have produced them, since these forms are indeed in the input. In the informal domains, however, where JC is used, forms that range closer to the basilect are typical in the speech of the representative group of speakers. In instances where English-influenced forms are prevalent and even dominated the more Creole-like forms (as with the past progressive below), the question of a change in L1 use becomes evident.

A second way in which the present progressive was expressed, was with the omission of the marker and the use of the uninflected verb. In \REF{bkm:messamK:25} and \REF{bkm:messamK:26} above, we saw such a construction being used by the CLAR children in subjectless answers to questions. In the attrition data, however, this was problematic, since it resulted in a construction that is typical of the simple past in JC. A progressive interpretation therefore only becomes recoverable through the discourse or through the presence of another verb in the utterance that has been overtly marked for this feature.

In \REF{bkm:messamK:32}, the speaker refers to the action of ‘drying’ (represented as \textit{jrai} ‘dry’) that a boy continues to do, describing an action that was taking place in a picture she was shown. The omission of aspectual \textit{a} results in a construction that is erroneous in JC; the correct interpretation could only be retrieved through context.

\ea \label{bkm:messamK:32}
\gll  Di   bwai av     di     towil ∅  jrai out iihn   ed.\\
\textsc{def} boy   have \textsc{def} towel  \textsc{prog} dry  out 3\textsc{s}     hair\\
 \glt `The boy is drying his hair with the towel’       [16 yrs, PS]
\z



In attrition, a progressive aspectual interpretation in other contexts is dependent on structure. In \REF{bkm:messamK:33}, one is able to derive a progressive meaning for the bare verb \textit{se} ‘say’ since the previous non-stative verb \textit{taak} ‘talk’ accompanied by aspectual \textit{a}, conveys the progressive aspect. \textit{Waahn} ‘want’ would have been unmarked, as stative verbs are not overtly marked for progressive aspect in JC. The speaker is \textit{talking} and the bystander would wish to hear what he or she is \textit{saying}.

\ea \label{bkm:messamK:33}
\gll  If yu   a  taak pan i fuon    an … dem  waahn ier   wa  yu  ∅ se.\\
if 2\textsc{s}   \textsc{prog}  talk  on    \textsc{def} phone  and ~  3\textsc{pl}    want   hear what 2\textsc{s}  \textsc{prog} say\\
\glt `If you’re on the phone and … they want to hear what ‘you are saying.’
                       [13 yrs- Informant B, NUD]
\z


The omission of the progressive aspectual marker in the \textsc{asp}+ bare verb structure is revealed in the attrition data to be a feature that is likely to be evidenced in late attrition, that is, attrition occurring post the ten-year period, since it was uncharacteristic of productions made by JC immigrants who had been residing in the L2 country for a period under ten years. Such a conclusion has been made on the basis that of the seven informants each in the 1--5Y and 6--10Y category, only one informant from each category had this feature present in their speech. This is contrary to the >10Y category, where this feature was evidenced in the speech of four of the six informants.

The third deviation in the construction was with the past progressive form, where constructions consisted of an independent past progressive marker accompanied by the inflected verb. If we consider that the past marker was used only seven times in the L1 acquisition of the progressive, the implication for the attrition of progressive constructions would be that the past progressive is affected early, since it is acquired late in the L1. It is however acquired late in the L2 as well, which may delay cross-linguistic influence at an early stage.

Deviation in the past progressive construction, illustrated in \REF{bkm:messamK:34} and \REF{bkm:messamK:35}, is in the attrition of JC presented in the form of past \textit{woz} co-occurring with the verb bearing suffixal \textit{{}-in}. This stands in contrast to the \textsc{past+asp+verb} structure conventionally used in varieties of JC that are closest to the basilect.

\ea \label{bkm:messamK:34}
\gll  Dem woz   mek-in       a        mes.\\
3\textsc{pl}    \textsc{past} make-\textsc{prog} \textsc{indef} mess\\
\glt `They were making a mess.’              [6yrs- Informant L, PS]
\z




\ea \label{bkm:messamK:35}
\gll  Mi   tel  ar di    chiljren dem woz  fait-in      in di    klaas.\\
1\textsc{s} tell her  \textsc{def} children \textsc{pl}    \textsc{past} fight-\textsc{prog} in \textsc{def} class\\
\glt `I told her that the children were fighting in the class.’ [13 yrs- Informant M, SCT]
\z




This feature is characteristic of attrition six years and beyond, and is therefore not a feature that is expected of the earliest stages of attrition.

Of note in attrition data is the correction of the conventional JC \textit{a liv} to the inflected verb by an informant who had been residing in the L2 country for 16 years at the time of data elicitation. This structure is evidenced in early attrition and is maintained throughout, even as stability is achieved.\footnote{A feature is considered to be stable if it was found to be characteristic of the three established year groups in the attrition data.}

If one were to consider the attrition facts related to the progressive aspect, the conclusion may be made that it is one of the earliest features to fall susceptible to language change in JC. Changes in this area do not take place at once but do so gradually. In the initial stages, the aspectual marker is likely to be omitted and the verb inflected. In the 6--10 year stage, it is likely for there to be an omission of the inflection as well as the independent marker, with the bare verb standing alone. This construction becomes a variant. At the final stage of the attrition of the JC progressive construction, the past progressive becomes susceptible to change with past ‘woz’ being combined with an inflected verb. This is evidence of the past progressive being fully formed. Its occurrence in the final stage of the attrition of the progressive is plausible given the evidence from the L2 acquisition data, which shows that the past auxiliary is acquired late, with its use lacking among the CLAR children.

It stands to reason, then, that in the attrition of the first language under influence from an L2, the progressive aspectual construction typical of the L2 would be among the first features transferred into the L1.


\section{Conclusion and implications for the language education classroom}
\label{sec:messamk:6}
The findings from acquisition show a likelihood for the marking of the past tense to be acquired late by JC speakers, but that the progressive will be acquired early. As for attrition, it has been revealed that the past tense shows some resilience, whilst the progressive (past and present) is prone to attrition in stages. Divergent forms occurring in the data were not expected in the variety investigated, especially in the informal speech contexts in which they were used. Use in the ways identified is more prevalent in the upper mesolectal varieties. If it were then to be argued that these changes reflect a minor strategy of the language, change within these areas can still not be denied. As discussed and illustrated by the data, the use of the V+ing form for the past progressive out-numbered that of the bare verb, a major strategy. In the CLAR acquisition data, there is sparse use of past tense forms, taken to suggest late acquisition. One would then expect that the attrition data would reveal early effects in this construction. The attrition data however reveals that the JC past tense form is resistant to attrition, having remained stable in the speech of the attriters.


We have claimed that this may not be surprising in light of the fact that Pp obligatorily marks the past tense. This would have aided the retention of the past marker, both in obligatory and non-obligatory contexts. Possible Pp influence could become apparent if data from the attrited group were to be compared with equivalent data from a control group to determine frequency of overt tense marking by the groups. A more prevalent use of tense by the attriters could be a sign of possible influence, whereby Pp, unlike JC, requires that a tense marker precede both stative and non-stative verbs

The resilience of the past tense marker to change may also not be surprising as research on the accuracy order of grammatical morphemes has shown that the past tense is also acquired late in L2 acquisition, as discussed in \sectref{sec:messamk:4.4}. In light of the similarity in marking the past tense in JC and Pp, this argument primarily becomes relevant to this situation where the influence of English as an L2 is considered.  In instances where some attriters omitted the past tense marker from obligatory contexts, an early acquisition of the English past tense could have enabled attrition by replacement with an L2 form. As discussed, it has been postulated that in other areas of attrition data, there has been a transfer of English forms, which are usually used in the attriters’ attempt to apply a Pp rule. English effects in the creole grammar without Pp influence was however also evinced in this contact situation \citep{Messam-Johnson2017}. This is not attested in the attrition of past tense forms, with the data being void of the English past marker. The absence of English forms in this area may be indicative of a difficulty in the mastery of English tense marking, which, from a pedagogical perspective, may require that primary focus be given to this feature in the development of a curriculum geared toward the learning of English as a second language.

For this, given the acquisition and attrition trends found, a strategic approach will be possible. Whilst for Pp, there is some similarity with JC in the marking of tense, in that both systems utilise an independent past marker, this may be perceived as a shared fundamental difference between these systems and English, for which the marking of tense is inflectional. This supports the possible difficulty L2 learners of English might be having with mastering tense, as such a lack of congruence between language systems has been said to cause difficulty in learning the L2 \citep[252]{winford2003book}. Learning English for these speakers will also need to involve more than an indication that the past is marked by an inflectional ending, rather than with an independent tense marker. Learners ought to be encouraged to notice that not all verbs in the L1 require a marker, but that once the past tense is expressed in English, it must be marked.

An analysis of the findings for the progressive constructions reveals that the present progressive aspect is acquired early, with an expectation of attrition effects being shown at a later stage. The past progressive construction on the other hand is acquired late, thereby resulting in an anticipation of early attrition. It has however been shown that cross-linguistic effects in the present progressive are evident from the early years, whilst in contrast to the expectation for the past progressive, deviations in structure were characteristic of the later stages of the attrition process, the point at which stability in the attrited grammar is expected. Similar to the past tense construction, it is clear that the attrition effects represented for the progressive do not support the RH in terms of the order of attrition being the inverse of acquisition. The findings, however, have implications for the Jamaican language arts classroom.

Progressive constructions for both the CLAR children and the attriters reflect features of their JE counterparts to varying degrees: the auxiliary may or may not have been omitted, and the lexical verb may or may not have been inflected, but acquisition was attested early, and the attrition of the JC form also had an early start.  This may be explained by the fact that the progressive constructions in the three languages (JC, JE and Pp) comprise an auxiliary followed by a lexical verb; in JE and Pp the verb also bears an inflectional ending, but the interpretations in all languages are parallel – they express a continuous reading. Possible issues with concord apart, this construction may then prove to be less challenging to the L2 learner, and unlike the past tense, may be less of a priority in teaching English as an L2.


\section*{Abbreviations}
\begin{tabularx}{.6\textwidth}{@{}ll}
\textsc{jc} & Jamaican Creoles\\
\textsc{je} & Jamaican English\\
\textsc{asp} & aspect\\
\textsc{indef} & indefinite\\
\textsc{prog} & progressive \\
\textsc{clar} & Child Language Acquisition Research\\
\end{tabularx}%
\begin{tabularx}{.4\textwidth}{ll@{}}
\textsc{rh} &  Regression Hypothesis\\
\textsc{def} & definite \\
\textsc{poss} & possessive\\
\textsc{neg} & negative \\
\textsc{pl} & plural\\
\\
\end{tabularx}


\printbibliography[heading=subbibliography,notkeyword=this]
\end{document}
