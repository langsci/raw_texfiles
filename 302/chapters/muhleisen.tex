\documentclass[output=paper,colorlinks,citecolor=brown]{langscibook}
\ChapterDOI{10.5281/zenodo.6979335}

\author{Susannne Mühleisen\affiliation{Universität Bayreuth}}

\title{Talking about Creole: Language attitudes and public discourse in the Caribbean}

\abstract{Language attitude studies form an important indicator for the acceptability of a variety in general or in particular domains like education or the media. Caribbean Creole languages have traditionally been stigmatised due to the fact that they arose in contact situations during plantation slavery. Especially when they are in continued contact with their lexifier, the Creole is often not seen as a legitimate variety of its own. Language attitudes studies in Trinidad by \citet{winford1976teacher} and \citet{mühleisen2001bad} show, however, how the acceptance of Creole versus English has changed within two generations of Trinidadian schoolteachers. Ever since the publication of these quantitative survey-based studies, public opinion on language has been overwhelmingly expressed in traditional and digital media. A qualitative and quantitative analysis of language attitudes expressed in a corpus of just over 100 letters to the editor and in online forums as part of a public language debate in Jamaica will therefore update and complement the research findings in earlier studies.

\keywords{Teacher attitudes, qualitative methods, media discourse, online forums, language petition, Patois/Patwa, Trinidadian English Creole, Jamaican Language Unit, \textit{The Gleaner}}
}


\IfFileExists{../localcommands.tex}{
  \addbibresource{../localbibliography.bib}
  \usepackage{langsci-optional}
\usepackage{langsci-gb4e}
\usepackage{langsci-lgr}

\usepackage{listings}
\lstset{basicstyle=\ttfamily,tabsize=2,breaklines=true}

%added by author
% \usepackage{tipa}
\usepackage{multirow}
\graphicspath{{figures/}}
\usepackage{langsci-branding}

  
\newcommand{\sent}{\enumsentence}
\newcommand{\sents}{\eenumsentence}
\let\citeasnoun\citet

\renewcommand{\lsCoverTitleFont}[1]{\sffamily\addfontfeatures{Scale=MatchUppercase}\fontsize{44pt}{16mm}\selectfont #1}
  
  %% hyphenation points for line breaks
%% Normally, automatic hyphenation in LaTeX is very good
%% If a word is mis-hyphenated, add it to this file
%%
%% add information to TeX file before \begin{document} with:
%% %% hyphenation points for line breaks
%% Normally, automatic hyphenation in LaTeX is very good
%% If a word is mis-hyphenated, add it to this file
%%
%% add information to TeX file before \begin{document} with:
%% %% hyphenation points for line breaks
%% Normally, automatic hyphenation in LaTeX is very good
%% If a word is mis-hyphenated, add it to this file
%%
%% add information to TeX file before \begin{document} with:
%% \include{localhyphenation}
\hyphenation{
affri-ca-te
affri-ca-tes
an-no-tated
com-ple-ments
com-po-si-tio-na-li-ty
non-com-po-si-tio-na-li-ty
Gon-zá-lez
out-side
Ri-chárd
se-man-tics
STREU-SLE
Tie-de-mann
}
\hyphenation{
affri-ca-te
affri-ca-tes
an-no-tated
com-ple-ments
com-po-si-tio-na-li-ty
non-com-po-si-tio-na-li-ty
Gon-zá-lez
out-side
Ri-chárd
se-man-tics
STREU-SLE
Tie-de-mann
}
\hyphenation{
affri-ca-te
affri-ca-tes
an-no-tated
com-ple-ments
com-po-si-tio-na-li-ty
non-com-po-si-tio-na-li-ty
Gon-zá-lez
out-side
Ri-chárd
se-man-tics
STREU-SLE
Tie-de-mann
}
  \togglepaper[9]%%chapternumber
}{}

\usepackage{lineno}

\shorttitlerunninghead{Talking about Creole}
\begin{document}
\maketitle

\section{Introduction: language attitudes and Caribbean Creoles}\label{sec:muehleisen:1}

Language attitude studies have a long tradition in Creole studies, starting with Haynes’ \citeyearpar{haynes1973language} PhD research on contrastive language attitudes in Barbados and Guyana, followed by investigations on teacher attitudes towards Creole versus English in Trinidad \citep{winford1976teacher}, Rickford's \citeyear{rickford1983standard} exploration of prestige and solidarity values of Guyanese Creole in a rural community in Guyana, a follow-up study on the Winford publication \citep{mühleisen1993attitudes,mühleisen2001bad} which focuses on potential change of language attitudes in the Caribbean, as well as \citegen{BeckfordWassink1999} investigation in a semi-rural community in Jamaica. Post-millenium research moved on to explore more the acceptability of accents of standard English rather than attitudes towards Creole versus English in Barbados \citep{belgrave2008attitudes}, in Trinidad (\citealt{deuber2013towards}, \citealt{deuber2013investigating}, \citealt{meer2019trinidadian}), in the Bahamas (\citealt{oenbring_fielding2014}) and in Jamaica \citep{westphal2015attitudes}. With this almost exclusive focus on accents of English in media and educational contexts in the last two decades, one might ask: is there still any more need for talking about Creole?\footnote{The term Creole will be used in this paper as a general term for the type of contact language found in the Caribbean (and elsewhere). It is always capitalized to make clear its status as a language on par with English, French, etc. It can be modified (e.g. Trinidadian English Creole, Jamaican Creole) to denote the specific national Creole. Patois is the Jamaican name for their Creole, hence, the terms (Jamaican Patois, Jamaican Creole) will alternate without any meaning change. Patwa is a spelling variant of Patois which is often employed by the users themselves. This will be used especially in quotations and in the specific Jamaican language debate which is discussed in this paper.}

In this article, I would like to revisit some of the early studies, notably \citet{winford1976teacher}, on how Creoles were perceived by their speakers in the post-independence period in the Caribbean, to then contrast the results with later research (e.g. \citealt{mühleisen1993attitudes,mühleisen2001bad,BeckfordWassink1999}) to show how some of the linguistic research on Creoles starting in the 1960/70s has also triggered a change in beliefs about these languages. The question of whether or not this conceptual conversion has also changed the affective stance and the readiness to act, e.g. to accept changes in language politics, will be posed. For an inclusion of the “public voice” on language political proposals in Jamaica, an analysis of a corpus of ca. 100 letters to the editor of the \textit{Jamaican Gleaner}, collected between 1999 and 2020, will shed light on potential changes in the acceptance of Jamaican Patwa in language domains previously reserved for Standard English. This focus on self-selected public opinions will be complemented with an inclusion of comments and posts on social media platforms on the language petition posed in 2019.

Language attitudes are important in sociolinguistic research in that they reveal social evaluations in connection with a particular group of speakers. Firstly, they are important in exposing associations between speech patterns and a speaker’s membership in a particular social or ethnic group or community of practice. Secondly, they also show perceived correlations between speech patterns and personal qualities of an individual speaker (e.g. friendliness, intelligence, reliability, etc.). As mental variables, attitudes are not per se observable but they help to explain, predict or are related to patterns of behaviour. For language attitude studies, this means that this “learned predisposition to respond in a consistently favourable or unfavourable manner” \citep[6]{fishbein1975belief} with regard to a particular language code has to be elicited indirectly: beliefs about the variety/speech pattern in question (cognitive component), feelings and emotions towards the variety/speech pattern (affective component) and disposition to act (behavioural component).

The practice of evaluating other people’s speech positively or negatively has probably existed as long as there has been some kind of social differentiation between speakers in a community. In English language contexts, public debates about linguistic correctness and arguments about authority in the English language go back to at least the 18\textsuperscript{th} century, when much of what is still seen as the standard conventions in English were established through works like Dr Johnson’s \textit{Dictionary} (1755) or the grammar books by Robert Lowth (1762) and Lindley Murray (1795). As \citet[5]{burridge2010linguistic} writes, they are often “complaints about the language of others; i.e. observations on what is viewed as bad grammar, sloppy pronunciations, new-fangled words, vulgar colloquialisms, unwanted jargon and, of course, foreign items.” It is notable that, for the diglossic situation in the anglophone Caribbean  \citep{Winford1985}, the question of attitudes toward Creole, however, takes on a different dimension from the issues pointed out above: it is the evaluation of the two linguistic codes which have been used in functional distribution since the time of their presence in the Caribbean, one historically laden with the colonial baggage and association with slavery, the other not only the traditional “high” language in the diglossic situation but also a language with ever-increasing global significance (cf. also \citealt{mühleisen2002creole}). There are not only functional divides but also symbolic ones: as \citet{reisman1970cultural} pointed out so succinctly, both Creole and English are implicitly not only linguistic but also cultural codes connected to values of British/European versus African/Jamaican heritage. It is therefore no surprise that in most Creole attitude studies (e.g. \citealt{rickford1983standard}, \citealt{BeckfordWassink1999}) this linguistic and cultural ambiguity is also reflected in an ambivalent stance toward Creole by its speakers in that it achieves positive evaluations on a solidarity and friendship scale but low ratings on the level of power and authority.

\section{50 years on: Language attitude studies as barometer of change in educational contexts}\label{sec:muehleisen:2}

It is now five decades since \citegen{reisman1970cultural} anthropological observation of the connection between situational language choice and alignment with cultural values in the Caribbean and since the research was conducted by Winford in 1970 for the first Caribbean language attitude publication \citep{winford1976teacher}. The agenda pointed out in his article is first of all one of improving language teaching techniques for in the complex Caribbean language situation. Language attitudes by teachers, as Winford notes, might be a factor in determining progress in language acquisition:


\begin{quote}
    [Another] problem which must be stressed is the likelihood that the child's progress in the acquisition of new language skills is determined to a large extent by the attitudes to language varieties which prevail in the community. Again, relatively little research has been done in this area, and the present article attempts to remedy this situation to some extent by reporting on the attitudes of teachers themselves to the linguistic situation in their community. \citep[48]{winford1976teacher}
\end{quote}

Winford’s choice of educators as informants, then, was not random but well chosen: the position of the teacher is that of an influential multiplier of beliefs about language and other subjects and can help to shape future dispositions to act. On the other hand, teachers are also prone to follow idealized models of correctness. One of the most striking results of his questionnaire survey with 112 respondents (68 of Indian, 44 of African descent) from two different teacher training colleges in Trinidad was their evaluation of characteristics of Trinidadian Creole as “bad”, “broken” or/and “incorrect”:

\begin{quote}
    \ldots most of the respondents had a very clear picture of the characteristics of Trinidadian English that could be labelled “bad", “broken", “incorrect", etc., by comparison with whatever model of correctness each had in mind. At the same time, it seems clear that most respondents were not at all conscious of the fact that the Creole variety of Trinidadian English has its own grammatical system which operates according to different rules from those of Standard English. Informants generally show a great willingness to interpret what are essentially grammatically correct Creole structures as ``corruptions'' of ``good English''.
    \citep[51]{winford1976teacher}
\end{quote}


One of the achievements of linguistics research and resolution is a greater consciousness of the nature and value of Creole languages. As a follow up study \citep{mühleisen1993attitudes,mühleisen2001bad} to \citet{winford1976teacher} shows, two decades after the initial research, the distinction between Standard English and Trinidadian English Creole (TEC) is more clear to the respondents in the 1990s. For the 90 primary and secondary school teachers in \citet{mühleisen1993attitudes,mühleisen2001bad} Creole is not seen any more as ``bad English''. As one respondents writes in an open question section:

\begin{quote}
  It is an integral part of our society and is rich in our linguistic tradition. It has a vibrant role to play and should not be described as `Bad English' since it is a language in its own rights. I think people have to be better educated about the richness and evolution of Trinidadian Creole and its importance in their everyday lives. The media especially has an important role to play. (Resp. 029: F/Age 1/Pr/TC/Semi-urban/Ind.) \citep[74]{mühleisen2001bad}
\end{quote}

However, along with the heightened awareness of the separateness of the codes, an attentiveness to their functional distribution in interpersonal and situational contexts has also increased (see \tabref{tab:muhleisen:1} from \citealt[66]{mühleisen2001bad})

\begin{table}
\begin{tabular}{l@{ }lrrr}
\lsptoprule
&& Creole  &  Standard & Both\footnote{This choice was not explicitly given in the questionnaire but some respondents ticked both Creole and Standard (English) in this section.}\\
\midrule
a. & {With Parents} & 76.1\% & 17.0\% & 6.8\%   \\
b. & {With Spouse} & 72.2\% &22.8\% & 5.1\%\\
c. & {With Children} & 28.2\% & 61.2\%  & 10.6\% \\
d. & {With Friends} & 65.5\% & 19.5\% &14.9\%\\
e. & {At Work, With Colleagues} & 22.7\% &  59.1\% & 18.2\% \\
f. & {At Work, in Classroom}& 3.4\% & 91.0\% & 5.6\%\\
g. & {In Church} & 9.5\% & 79.8\% & 10.7\% \\
h. & {When ``Liming''} & 84.9\% & 9.3\% & 5.8\% \\
i. & {When Quarrelling} & 79.5\% & 12.0\% & 8.4\%\\
j. & {Telling Jokes} & 92.1\% & 5.6\% & 2.2\%\\
k. &  {New Acquaintance} & 10.2\% & 84.1\% & 5.7\% \\
l. & {When Introduced to a Trinidadian Abroad} & 30.6\% & 62.4\% & 7.1\%\\
\lspbottomrule
\end{tabular}
\caption{Language choice on an interpersonal level\label{tab:muhleisen:1}}
\end{table}

Creole as a language choice was clearly limited in classic (high) “H” domains classroom, church and work (e, f, g) and was also made less frequently in interpersonal contacts where awareness about one’s self-presentation (k, l) is requested, or with children (c) where considerations of educational success and a desire for upward social mobility might be at stake. In contrast to the findings in \citet{winford1976teacher}, the limitation of Trinidadian English Creole is not seen in the code itself but rather in its restricted usefulness in particular language domains  or as a language of international communication.

The dynamic perspective in this diachronic-comparative study  on language attitudes and factors of change shows that language evaluation patterns within the anglophone Caribbean are not unalterable. The conceptual level, beliefs and knowledge about an object, are usually seen as vital for attitude formation. \citeauthor{ajzen1989attitude} goes as far as postulating that “attitudes are not merely related to beliefs, they are actually a function of beliefs, i.e, beliefs are assumed to have a causal  effect on attitudes” (\citeyear[247]{ajzen1989attitude}, italics in the original, S.M.). To put it differently, the concept of the nature of a Creole, what speakers know or think they know about Creole, may also influence their overall language attitudes. However, in surveys and interviews as well as in experimental research, respondents also often comply with what they think is expected by the researchers, especially in the educational context. While the cognitive part of their attitudes towards Creole may have been influenced by the learned knowledge that Creole is a legitimate language, the affective and behavioral part may not always follow in the same way. In a classroom report by \citet{youssef_deuber2012} this becomes evident when linguistics students had access to schools and classroom and teacher discourse and were able to observe language attitudes in a variety of classes including some where the Caribbean Advanced Proficiency Examinations (CAPE) syllabus in Communication Studies which explicates and educates on the local language situation was conveyed. As they report,


\begin{quote}
    Only two days ago a student, now working part-time on the larger project, recounted to us an interaction within which a student had declared ‘But the Creole sounds so ‘retarded’!’ and the teacher had replied ‘Yes, but we have to accept it anyway!’ Clearly, despite the syllabus’ best efforts, negative attitudes to the Creole still loom large! \citep[5]{youssef_deuber2012}
\end{quote}

In a different regional Caribbean context, the hierarchy of attitude components can also be seen in Beckford Wassink's study “Historic low prestige and seeds of change: Attitudes toward Jamaican Creole” \citeyearpar{BeckfordWassink1999} on speaker attitudes in Gordon Town, Jamaica. Her structured interview questions are sub-grouped into three different categories – “feel”, “use” and “hear” questions – in her investigation of patterns in respondents' attitudes towards Jamaican Creole. Questions designed to reflect the respondents' reported willingness to use Jamaican Patois in a number of contexts (“use”-questions such as “Would use JC to answer the telephone” or “Would address their employer/supervisor in JC”) clearly scored lowest among speakers of all age and both gender groups while questions about knowledge and feelings (such as “Feel knowledge of JC is an asset” or “Believe JC can be used to say anything one could say in English”) scored highest \citeyearpar[72]{BeckfordWassink1999}.


\section{The linguist and the public voice: language ideology and politics in interaction}\label{sec:muehleisen:3}

Linguistics research in the area of language attitudes towards Creoles and prestige change has not only sought to inform and increase the knowledge about the nature and value of the varieties but also to change people’s disposition to act. The anglophone Caribbean country where linguistic debates and language politics have been led most publicly and fervently in the last three decades is, arguably, Jamaica. In \citeyear{devonish1986_liberation}, Devonish states that “even when the language question is not raised in any explicit manner, social and political conflict in the area of language is nevertheless present." And he adds, “in fact, the absence of any open debate on the language question is itself and expression of the complete control which those who benefit from the linguistic status quo have over the minds of the population at large.” But diglossic situations are not easily resolved, and especially when the H and the L (low) language are lexically related and the H language is one of not only local but also of global significance. Thus, some of the “popular struggles” Devonish describes in his \textit{Language and Liberation} \citeyearpar[87ff]{devonish1986_liberation} are still being fought for the legal status and a function elevation of Creole in Jamaica and elsewhere, despite a number of macro-functional changes, for example in the inclusion of Creole in primary and, to some extent, secondary education (cf. \citealt{morren_morren2007}).

The Jamaican Language Unit (JLU) of the University of the West Indies, with \citeauthor{devonish1986_liberation} in charge as coordinator for much of its existence, has been influential in bringing language issues to the public. The adaptation of a consistent phonemic orthography (known as Cassidy orthography) to Jamaican Patwa did not remain in the university classrooms but were brought to the public by publications and user-friendly materials \citep{JLU2009,JLU2009_video}, collaborations with translations of the New Testament \citeyearpar{biblesociety2012} and popular books like Alice in Wonderland \citeyearpar{Alis_advencha2016} into Jamaican Patwa in the Cassidy spelling and, last but not least, by the publication of a regular column in the \textit{Jamaican Gleaner} by cultural studies professor Carolyn Cooper from the 1990s on (cf. also \citealt{mühleisen1999konkur}).

There have been also a number of language political proposals to the Jamaican government, Ministry of Education, and the public spearheaded by the JLU. In 2001, a presentation was made to the House of Parliament of Jamaica on the issue of language rights in the Jamaican constitution. The Ministry of Education began the Bilingual Education Program (BEP) which included assigning equal status to Jamaican Creole in three pilot schools and supporting teachers in both languages \citep{morren_morren2007}. In 2005, the \citeauthor{JLU2005_survey} conducted an island-wide Language Attitude Survey with 1,000 Jamaicans of various regional affiliations, age groups and social and occupational groups in order to generally assess the views of Jamaicans towards Jamaican Creole. Since one of the agendas of the survey was also to elicit whether or not the public would accept a proposal for making Patwa a co-official language in Jamaica, some selected results will be briefly discussed here.

There was large agreement (almost 79.5\%) that Patwa is a language and that it should be made an official language alongside English (68.4\%). There was an almost equal score on questions where the use of Creole in public written space (road signs, school books, medicine bottles, etc.) was concerned and a high number of respondents supported the use of both Patwa and English in school. But the results also showed a surprising number of persistent stereotypes in the affective section of the evaluations: the results for questions of what sounded more intelligent (Patwa: 7.3\%, English: 55.0\% and both: 32.9\%) or educated (Patwa: 5.9\%, English: 59.1\% and both: 30.8\%) seem to go along with the classic power associations from earlier studies like  \citet{rickford1983standard}. However, on the solidarity level Patwa also did not reach an unchallenged position: on the level of honesty (Patwa: 28.3\%, English: 27.8\%, both: 35.3\%) and helpfulness (Patwa: 30.0\%, English: 24.2\%, both: 34.8\%) Patwa and English were rated almost equally, but Patwa was scoring slightly better on the friendliness ratings (37.9\%, 24.0\%, 33.3\%).

While the overall results of the survey remained ambiguous, the large agreement of the respondents would welcome Patwa as a co-official language in Jamaica paved the way to further proposals in that direction up to, finally, the submission of a petition in October 2019 to the Office of the Prime Minister demanding the government to take the necessary steps towards recognizing Jamaican Creole/Patwa as an official language alongside English (see analysis of comments in social media in section 4.2). For this step, the Jamaican public was not unprepared: ever since the early 2000s, Hubert Devonish and other members of the JLU regularly commented in the Jamaican newspaper \textit{The Gleaner} (among others) on “language rights, justice and the constitution” \citep{devonish2002language}, demanding to “stop demonizing Patois” \citep{devonish2012stop}, “end prejudice against Patois” \citep{devonish2016end} or “end Jamaican language apartheid” \citep{devonish2018end}. The catchy titles of the linguist’s opinion articles and comments, no doubt chosen by the editor of the newspaper, relate some of the heatedness of the debate which was not only taken up by linguists and politicians but also by members of the wider public, newspaper readers, in Letters to the Editor (LTEs) such as the following example from my corpus:


\ea
\textbf{The patois debate continues, }

Published:  29 September 2001

%\linenumbers
%\begin{linenumbers}

\textbf{THE EDITOR, Sir:}

ONCE AGAIN the question of patois in schools is being debated. The statement by Minister Whiteman carried in the \textbf{`Gleaner'} of September 20 is very clear Jamaican Patois will not be taught in schools, and will not be elevated through the creolisation of the education system. This pronouncement should put to rest the fears (or hopes) of us all. As we all realise, English is the language of international diplomacy, trade and science. It is interesting to hear Macedonian townspeople speaking English to reporters, and even the Taliban leadership has at least one English-speaker. But we seem to have a problem here at home. As evidence, Minister Philip Paulwell told us on `Nationwide' that a large tele-sales company in Montego Bay could not find enough speakers of standard English to fill the available jobs. This, in an English-speaking country, is beyond a joke.

Patois is a legitimate part of our identity and culture; we need not banish it from our lives. However, proficiency in English is almost mandatory today.

Incompetence can be a serious handicap; and the world is in no hurry to learn Jamaican Creole. My suggestion is that we treble or at least double current time and effort put into the teaching of English. Minister Whiteman used the word ‘immersion'; which may be needed to make up for the lack of standard English in the daily lives of so many, especially our children.

(…)

So let us enjoy our patois; it is ours, it is astonishingly expressive and important to our identity. And what is to stop us earning the respect and admiration of the world for having made ourselves into the finest speakers of the English language on the planet.

\textbf{I am, etc.,}

\textbf{D. B.}
%\end{linenumbers}
\z

I will take these letters as expressions of language attitudes -- this time not elicited from selected groups of respondents as in questionnaire surveys and experiments but self-selected by readers with a strong opinion on language matters. In the following section (\sectref{sec:muehleisen:4.1}), some examples of my corpus of just over 100 such letters on the Patwa language debate, collected in the Jamaican Gleaner between 1999 and 2020, will be given and analyzed with regard to the language attitudes expressed in them.

\section{Letters to the editor and public language debates in Jamaica}\label{sec:muehleisen:4}

Letters to the editor have existed as a genre at least since the late 18th century (cf. \citealt{sturiale2016}). As text type, the Letter to the Editor is typically argumentative, with the main communicative purpose of defending, analyzing or refuting an issue. While some letter writers might initiate a topic, the main body of LTEs are intertextual in that they are a reaction to a previously published article or letter by another reader. The example (1) above is therefore typical in its intertextual reference to another newspaper article, the statement by Minister Whiteman some days previous to the publication of the letter. LTEs are overtly addressed to the newspaper editor (“dear Editor, Sir”, “dear Editor, Madam”) but the covert addressee is the wider public and, specifically, the readership of the newspaper to which it is addressed. Expressing an opinion or making a complaint about something are primary functions of the genre LTE, others have been found to be “resolving a conflict, or to convincing readership and arousing reactions (cf. also Morrison \& Love’s reference as “problem-solution discourse”, \citeyear[50]{morrison_love1996}).

The proponents of the debate, Hubert Devonish or Carolyn Cooper as prominent UWI activists, are often subject of agreement or disagreement themselves, as some examples of titles from my corpus suggest, e.g. \textit{“`God' to Cooper:} Tap di foolinish!” (03/05/2011), \textit{Opposed to views of Boyne, Cooper} (28/12/2011), \textit{The irony that is Carolyn Cooper} (08/08/2012), \textit{The miseducation of Carolyn Cooper} (24/12/2014), or \textit{Devonish's Patwa economics nonsensical} (04/02/2017). In contrast to the conventions to the genre, there is also one example in my corpus where the letter is directly addressed to Carolyn Cooper (as opposed to the editor):


\ea
\textbf{Patois is geographically limited }

Published: Tuesday | August 10, 2010 | 12:00 AM


\textbf{THE EDITOR, Sir:}

\textit{PLEASE PUBLISH as a letter to Carolyn Cooper.}

Dear Carolyn: Re your August 8, 2010, article `Reading and writhing':

Most Jamaicans are not as concerned with the legitimacy of Jamaican as they are with its viability in the ``flat world'' in which we live. Can our present students become the assertive, effective arbitrators and wielders of power as they represent our country among the other world leaders
with whom they must interact?

It is interesting how vehemently you speak for the rights of the Jamaican student to speak his or her language even as you wield with great facility and ease the words of the English language to cut down contemptuously those who would challenge you. Do you not recognise that that facility is
exactly what the defenders of English crave for their children?

…
\z

In this heated language debate, many people feel challenged in their notion of what is correct or what standards should be upheld. Any attempt to challenge the belief in such notions of unique and eternal standards of correctness will inevitably attract criticism, scorn, as some traitors to the creed -- usually linguists -- had to experience (cf. \citealt{burridge2010linguistic}: 6). She writes about the conflicting roles of language professionals versus users in language debates in Australia, “linguists are clearly also in a tricky position. In the eyes of the wider speech community, they are seen as supporters of a permissive ethos encouraging the supposed decline and continued abuse of Standard English (\citealt{burridge2010linguistic}: 8).


\subsection{Data collection and analysis}\label{sec:muehleisen:4.1}

My corpus consists of just over 100 LTEs to the \textit{Jamaican Gleaner} in two time periods: 40 samples of LTEs on the Patois language debate were collected between 1999 and 2002 and another 62 cases between 2010 and early 2020. Therefore, the collection makes no claim to completeness but intends to provide a relatively good overview of the state of the debate in a large time frame. The examples were coded for a number of characteristics (sex of author, intertextual reference, potential claim of authority of author, overall communicative function of the letter and attitude towards elevating the status of Patwa and using it in public functions, i.e. in education, as co-official language or as language of writing. \tabref{tab:muehleisen:dates} gives that information together with title and date for the year 2011 as an example.

\begin{table}[t]
\small
\begin{tabularx}{\textwidth}{lQlllll}
\lsptoprule
 \textbf{2011} & \textbf{Title} & \textbf{Sex} & \textbf{Intertext} & \textbf{Authority} & \textbf{Function} & \textbf{Attitude} \\
\midrule
 08/01&  \textit{Closing gaps in education}&  M&  + article&  -&  inform& {{ \textcolor[HTML]{00B0F0}{+/$-$}}} \\
 19/01&  \textit{Nationalising Creole? You've got to be kidding!}&  M&  + article&  -&  disagree& {{ \textcolor[HTML]{FF0000}{$-$}}} \\
 12/04&  \textit{Defending Patois}&  M&  + article&  linguist&  disagree& {{ \textcolor[HTML]{00B050}{+}}} \\
 14/04&  Fait a Yaad, Daans Abraad&  F&  -&  -&  disagree& {{ \textcolor[HTML]{00B0F0}{+/$-$}}} \\
 17/04&  \textit{Teach English as foreign language}&  M&  -&  teacher&  disagree& {{ \textcolor[HTML]{FF0000}{$-$}}} \\
 19/04&  \textit{Argue logically for Patois}&  M&  + article&  -&  inform& {{ \textcolor[HTML]{00B0F0}{+/$-$}}} \\
 20/04&  \textit{Patois is best route to English}&  M&  -&  -&  agree& {{ \textcolor[HTML]{00B050}{+}}} \\
 24/04&  \textit{Jamaica must stick to the rules of English}&  M&  -&  -&  disagree& {{ \textcolor[HTML]{FF0000}{$-$}}} \\
 27/04&  \textit{Blinded by self-importance}&  M&  + letter&  -&  disagree& {{ \textcolor[HTML]{FF0000}{$-$}}} \\
 03/05&  \textit{'God' to Cooper: }Tap di foolinish!&  ?&  - &  -&  disagree& {{ \textcolor[HTML]{FF0000}{$-$}}} \\
 07/05&  \textit{Ignorant stance on language}&  F&  + letter&  -&  disagree& {{ \textcolor[HTML]{00B050}{+}}} \\
 24/05&  \textit{Patois lessons quite realistic}&  ?&  -&  -&  inform& {{ \textcolor[HTML]{00B050}{+}}} \\
 03/10&  \textit{Dick's stance on Patois absurd}&  M&  + article&  -&  disagree& {{ \textcolor[HTML]{FF0000}{$-$}}} \\
 18/10&  \textit{Patois as a first language: nonsense}&  M&  -&  -&  disagree& {{ \textcolor[HTML]{FF0000}{$-$}}} \\
 28/12&  \textit{Opposed to views of Boyne, Cooper}&  M&  + article& {} &  disagree& {{ \textcolor[HTML]{FF0000}{$-$}}} \\
 \lspbottomrule
\end{tabularx}
\caption{Dates, title and features of LTEs in 2011}
\label{tab:muehleisen:dates}
\end{table}





For the purpose of the reflection on the usefulness of including  LTEs as material for language attitude studies, the results of the positive, negative or ambiguous evaluation of Patwa in public space will be given according to year of collection (for a more detailed analysis of other aspects of the data, cf. \citealt{mühleisenmeta}).

\begin{table}
\begin{tabular}{l *7{r}}
\lsptoprule
     &            & \multicolumn{2}{c}{{Pos.}} & \multicolumn{2}{c}{{Neg.}} & \multicolumn{2}{c}{{Ambig. $\#$}}\\\cmidrule(lr){3-4}\cmidrule(lr){5-6}\cmidrule(lr){7-8}
Year & Total $\#$ & \# & \%& \# & \%& \# & \%\\\midrule
1999 & 11 & 5 & 45.45 & 5 & 45.45 & 1 & 9.00 \\
2000 & 2  & 1 & 50.00 & 0 & 0.00  & 1 & 50.00 \\
2001 & 15 & 8 & 53.33 & 6 & 40.00 & 1 & 6.66 \\
2002 & 12 & 7 & 58.33 & 5 & 41.66 & 0 & 0.00 \\
2010 & 4  & 1 & 25.00 & 3 & 75.00 & 0 & 0.00 \\
2011 & 15 & 4 & 26.66 & 8 & 53.33 & 3 & 20.00 \\
2012 & 7  & 5 & 71.43 & 2 & 28.57 & 0 & 0.00 \\
2013 & 1  & 0 & 0.00  & 1 & 10.00 & 0 & 0.00 \\
2014 & 2  & 1 & 50.00 & 1 & 50.00 & 0 & 0.00 \\
2015 & 2  & 0 & 0.00  & 0 & 0.00  & 2 & 10.00 \\
2016 & 5  & 3 & 60.00 & 2 & 40.00 & 0 & 0.00 \\
2017 & 5  & 1 & 20.00 & 3 & 60.00 & 1 & 20.00 \\
2018 & 3  & 2 & 66.66 & 1 & 33.33 & 0 & 0.00 \\
2019 & 13 & 7 & 53.38 & 5 & 38.46 & 1 & 7.70 \\
2020 & 5  & 2 & 40.00 & 3 & 60.00 & 0 & 0.00 \\
\lspbottomrule
\end{tabular}
\caption{Language attitudes in LTEs per year}
\end{table}


\newpage
The results show no coherent picture with regard to percentage of positive or negative language attitudes expressed in the letters, nor does there seem to be a linear development. Rather, there appear to be recurring swings in the one or the other direction. The overall number of positive (47) versus negative (45) language attitudes is almost perfectly balanced, with ambiguous attitudes expressed in 10 letters altogether.  However, what is significant is the peak of letters in some periods of time, i.e. in 1999, 2001, 2002, 2011 and 2019. In the last case, the rise in participation in the language debate is clearly related to the petition submitted in October 2019 to recognize Jamaican Creole/Patwa as co-official language.

\subsection{Langwij Pitishan: Comments in social media}\label{sec:muehleisen:4.2}

The petition/pitishan was submitted in both English and Patwa to “make Jamaican an official language alongside English\slash Mek Jamiekan wahn ofishal langwij saida Ingglish.”\footnote[2]{\url{https://opm.gov.jm/participate/jamaica-house-petition/sign-petition/?pet=100}} Timed to celebrate the 100\textsuperscript{th} anniversary of Jamaican poet Louise Bennett, “one of the biggest champions of the Jamaican language”, the petition called “on the Government and Parliament of Jamaica to take all the steps necessary to grant official status to Jamaican, alongside English.” The petition was also published in the Jamaican newspaper \textit{Observer} on December 4\textsuperscript{th}, 2019 with signatures from Jamaican and international supporters.

\begin{figure}
\includegraphics[width=0.85\linewidth]{figures/JLU_pitishan.png}
 
\end{figure}
 

The appeal received a lot of attention but did not achieve the necessary number of votes -- despite the overt agreement of respondents of the survey in 2005 cited in section 3 of this article to make Jamaican Patwa a co-official language. Reactions to the petition can be seen in yet another public medium, the social media platform Facebook page of JLU member and proponent of the petition, Joseph Farquharson. The comments and contributions on this site on the issue of Patwa as official language, once again, may not be representative in that they do not include the voices of those who were not interested enough to sign the petition or reject it but, rather, is restricted to those who are happy to give their opinion about the issue in public.

As of January 24\textsuperscript{th} 2020, there were 97 entries altogether which can be grouped into 40 comments with multiple interactive threads. The gender distribution of the contributors -- deduced from profile name and/or picture of the person\footnote[3]{This could be staged, of course, so there is no absolute certainty on the gender identity of the participants.} -- was clearly male-dominated, with 28 male contributors, 17 female writers plus the (male) author\slash moderator of the website.

Comment 1 (with three contributors and the author interacting) takes issue with the perceived difficulty to read and write Patwa (and practical reasons to not use it):

\begin{quote} \noindent
\textbf{Comment 1} \noindent

\textbf{M1}\footnote[4]{The profile names of the Facebook users were anonymized and coded as Male 1 (M1), Female 1 (F1) in order of appearance. Author is abbreviated as A.} This might become a reality but not in this life time, cause it’s too difficult to read, spell and write, it would have to be a broad base approach where everyone wishes just write and spell the words his/her way, then everyone would go back to basic of spelling and pronouncing, honestly, I was reading a chapter of this language recently and it took me about an hour to do so, which then reminded me of my days at BASIC School learning to read.

\textit{\textcolor[HTML]{00B050}{4 likes, 1 love}}

\textbf{F1 to M1} I love our native and first language, but if English so hard for some; how easy will it be for them to read and understand this our native language.

\textit{\textcolor[HTML]{00B050}{1 like, 1 love}}

\textbf{M1 to F1} My conclusion is that the only aspect of this native language that most of us will ever understand is to speak it,but as it relates to the spelling and writing aspects it’s a nono.

\textit{\textcolor[HTML]{00B050}{1 like}}

\textbf{F2 to M1} learned it in a day, reading fluently from the patwa bible. (includes video)

\textbf{A to M1} writing always has to be taught, no matter what language it is. People in England are not born writing in English. They have to be taught it in school.
\end{quote}

The advantage of the contributions in the social media platform over the expression of opinions in the Letters to the Editor becomes immediately clear: the interactive format allows for immediate reactions to a post. Additional feedback (approval, humour) may also be seen in the distribution of a “like,” “love,” or “laugh” button by other participants. The author, in this and other comments, acts as a moderator to the discussion/debate. In comment 1, it becomes clear that the alleged necessity to write Patwa in a certain way forms an obstacle to M1 -- a worry which is refuted by the other interactants in the thread.
Other comments concern the affective stance of contributors and the belief that Patois is a hindrance to development (emotional and practical reasons), as can be seen in the following one:

\begin{quote}
    \textcolor[HTML]{FF0000}{\textbf{Comment 17:}}

    \textbf{F8} English, English must be spoken and taught. Why are they encouraging people to be backward? This so called patio [sic] is just a massive hindrance to development.

    \textit{\textcolor[HTML]{00B050}{1 laugh}}

     \textbf{A to F8} nowhere in the petition does it say English will not be spoken and taught. It says ALONGSIDE English.

    \textit{\textcolor[HTML]{00B050}{1 like}}

     \textbf{F8} Patio should never be taught in schools at all it sounds ugly and vulgar it’s total waste of time instead beautiful languages like Spanish French German should be that would be beneficial for job.opportunities and self improvement

     \textit{\textcolor[HTML]{00B050}{1 laugh}}

     \textbf{A to F8} my mother language is not ugly and vulgar. I once saw a quote that said what we say about a language is a reflection of how we view the people who speak it.

     \textbf{M3 to F8} as much as I am against teaching patois in schools as standard your comments reeks of ignorance

\textit{\textcolor[HTML]{00B050}{1 like}}
\end{quote}


The contribution by F8 shows a negative evaluation of Patwa on the level of emotive and behavioral stance. Again, we can see in the post of M3 to F8 how the interaction possibilities of the medium have an immediate counteractive effect. Not all comments are interactive, i.e. evoke a reaction by another participant. In the following, we have a number of relatively short and rather isolated comments, displaying both positive (comment 28, 31) and negative (29, 30) attitudes towards the petition. The rejection of the idea rests on practical reasons rather than aesthetic or emotive ones. Apart from the issue of the perceived difficulty of reading and writing the code, the matter of costs of the language political measure is addressed in comment 32:
%\vspace{1\baselineskip}
\begin{quote}
\noindent\textcolor[HTML]{FF0000}{\textbf{Comment 28:}}

\noindent\textbf{F15} A waa u sah man mi love strate patwa mi seh

\noindent\textcolor[HTML]{FF0000}{\textbf{Comment 29:}}

\noindent\textbf{M19} lol cant even read it

\textbf{A} You can learn. You are one of our brightest minds. This can’t be too hard for you.

\noindent\textcolor[HTML]{FF0000}{\textbf{Comment 30:}}

\noindent\textbf{M20} It is, already

\noindent\textcolor[HTML]{FF0000}{\textbf{Comment 31:}}

\noindent\textbf{F16} Same like how we learned to do everything else….

\noindent\textcolor[HTML]{FF0000}{\textbf{Comment 32:}}

\noindent\textbf{M21} The money for changing all government forms and other documents is waiting already!

\noindent\textcolor[HTML]{FF0000}{\textbf{Comment 33:}}

\noindent\textbf{M22} No way that is a dumb idea English or Spanish

\textbf{A} English is already official, and the proposal does not rule out other languages being taught/learned.
\end{quote}

Finally, the significance of English as a global language was also stressed as a reason for rejecting the petition:

\begin{quote}
\textcolor[HTML]{FF0000}{\textbf{Comment 34:}}

\textbf{M23} Patwa can’t be used as our official language because the rest of the world barely understand our dialect. That’s 17ridiculous. If you can’t communicate the English language fluently or at a satisfactory level then that will cause major problems for our society in terms of business and communications to the world market.\\
\noindent We are quite fine as an English speaking country and proud of it.\\
\noindent Dialect is a part of our language as well for our unique culture and heritage as Jamaicans and we will never refute that. But knowing proper English is what will make many young youths more marketable and for worldwide social interaction in every industry as well as world power foreign relations

\textbf{A to M23} the petition says alongside English not replacing English. \\
\end{quote}

As with the Letter to the Editor data, the reactions to proposing Patwa to step out of its L language status and become an official language were consistently mixed. Out of the 28 male contributors of the discussion, 17 rejected the petition implicitly or explicitly, 9 supported it and the stance of 2 participants remained unclear. The opinion of the female contributors was more balanced: 7 of the female participants were in favour of Patwa and the petition while another 7 rejected it. 3 of the female contributors could not be placed in terms of their attitude towards the subject.


\section{Conclusion}\label{sec:muehleisen:5}
It seems that, 50 years after the first language attitude studies toward  Creole in the anglophone Caribbean, talking about Creole has not become obsolete. Rather, Creole discourse in public outlets like newspapers and social media platforms appears to be as alive, fiery and divided as ever. Some of the issues that were prevalent in the early studies have almost disappeared: beliefs that Creole is not a language or merely a corruption of the lexifier English are hardly found or expressed in systematic attitude studies or in the public sphere. Most writers in Letters to the Editor or in the Facebook comments in the Jamaican context appear to value Patwa as their language but are hesitant to invest time or effort when it comes to elevating it to the status of official language. As one participant in the Facebook discussion writes in his post:

\begin{quote}
\textcolor[HTML]{FF0000}{\textbf{Comment 26:}}

\textbf{M18} Y'all want too much. No one can even spell the words \ldots isn't it enough that we all know and love it
\end{quote}

In our more globalized world the importance of having a national language as expression of national identity seems to have decreased in comparison to the more immediate post-independence period. One of the great achievements of linguists in the standardization of Creoles, the creation of a consistent orthography which highlights the autonomy of the language and makes the language more visible, is seen as an obstacle by quite a few respondents. This is unfortunate for linguists but could also be seen as an unconscious resistance to authority on the part of the language user who might want to keep (and use) the language the way they want it. Writing practices have also changed and the use of Creole in the digital space is highly prevalent but, at the same time, the spelling remains anarchically creative and expressive (cf. \citealt{moll2017diasporic}) and continues to contribute to shape and transform the public discourse about Creole.


\section*{Abbreviations}
\begin{tabularx}{.5\textwidth}{@{}lQ}
\textsc{LTE} & Letters to the Editor\\
\textsc {BEP} & Bilingual Education Program \\
\textsc{JLU} & Jamaican Language Unit\\
\end{tabularx}


\printbibliography[heading=subbibliography,notkeyword=this]


\newpage
\section*{Appendix: Facebook comments on the \textit{Pitishan}}
\small

\begin{longtable}{llllp{6cm}}
\lsptoprule
{{No.}} & \multicolumn{3}{c}{{Attitudes to:}} & {{Reason\slash Commentary}} \\\cmidrule(lr){2-4}
& {{spoken P}} & {{written P}} & {{Petition}} &\\
\midrule
 {M1} & {positive} & {negative} & {\textcolor[HTML]{FF0000}{reject}} & {too difficult to read and write} \\
 {M2} & {positive} & {positive} & {\textcolor[HTML]{00B050}{support}} & {\textit{unique language\slash love my language}} \\
 {M3} & {positive} & {unclear} & {\textcolor[HTML]{FF0000}{reject}} & {which version of Patwa will be used} \\
 {M4} & {positive} & {positive} & {\textcolor[HTML]{00B050}{support}} & {\textit{\textcolor[HTML]{1C1E21}{Mek patwa reconize as wi original langwij}}} \\
 {M5} & {unclear} & {unclear} & {\textcolor[HTML]{FF0000}{reject}} & {\textit{\textcolor[HTML]{1C1E21}{if you cannot spell a word in English how can you spell one in patois}}} \\
 {M6} & {negative} & {negative} & {\textcolor[HTML]{FF0000}{reject}} & {\textit{\textcolor[HTML]{1C1E21}{Patois cannot be a language :D we speak broken engllish/standard english}}} \\
 {M7} & {positive} & {positive} & {\textcolor[HTML]{00B050}{support}} & {\textit{$\ldots$\textcolor[HTML]{1C1E21}{ this generation writes and reads it well.}}} \\
 {M8} & {positive} & {negative} & {\textcolor[HTML]{FF0000}{reject}} & {\textit{\textcolor[HTML]{1C1E21}{di xtraness nah go work}}} \\
 {M9} & {positive} & {positive} & {\textcolor[HTML]{00B050}{support}} & {\textit{\textcolor[HTML]{1C1E21}{how many Jamaicans read and speak English well}}} \\
 {M10} & {positive} & {positive} & {\textcolor[HTML]{00B050}{support}} & {\textit{\textcolor[HTML]{1C1E21}{If u can't read something as simple as patwah which we speak how are we to learn Spanish or French which we don't speak}}} \\
 {M11} & {unclear} & {unclear} & {unclear} & {\textit{\textcolor[HTML]{1C1E21}{The statement extra n nobody nuh txt so}}} \\
 {M12} & {positive} & {negative} & {\textcolor[HTML]{FF0000}{reject}} & {\textit{petition waste of time, too hard to read}} \\
 {M13} & {unclear} & {unclear} & {unclear} & {Language examples instead of positioning} \\
 {M14} & {unclear} & {negative} & {\textcolor[HTML]{FF0000}{reject}} & {\textit{\textcolor[HTML]{1C1E21}{One more aim to keep the poor , poor and ignorant}}} \\
 {M15} & {unclear} & {negative} & {\textcolor[HTML]{FF0000}{reject}} & {\textit{\textcolor[HTML]{1C1E21}{Nah for me, don’t need to be taking English language exams when I wanna go college abroad}}} \\
 {M16} & {unclear} & {unclear} & {\textcolor[HTML]{00B050}{support?}} & {\textit{\textcolor[HTML]{1C1E21}{Ones have to be foolfool to think this a laafing mata!}}} \\
 {M17} & {positive} & {positive} & {\textcolor[HTML]{00B050}{support}} & {\textit{\textcolor[HTML]{1C1E21}{Our first language}}} \\
 {M18} & {positive} & {negative} & {\textcolor[HTML]{FF0000}{reject}} & {\textit{\textcolor[HTML]{1C1E21}{Y'all want too much. No one can even spell the words... isn't it enough that we all know and love it}}} \\
 {M19} & {unclear} & {negative} & {\textcolor[HTML]{FF0000}{reject}} & {\textit{\textcolor[HTML]{1C1E21}{lol cant even read it}}} \\
 {M20} & {unclear} & {negative} & {\textcolor[HTML]{FF0000}{reject}} & {\textit{\textcolor[HTML]{1C1E21}{It is, already (i.e. too hard to learn)}}} \\
 {M21} & {unclear} & {negative} & {\textcolor[HTML]{FF0000}{reject}} & {\textit{\textcolor[HTML]{1C1E21}{The money for changing all government forms and other documents is waiting already!}}} \\
 {M22} & {unclear} & {unclear} & {\textcolor[HTML]{FF0000}{reject}} & {\textit{\textcolor[HTML]{1C1E21}{No way that is a dumb idea English or Spanish}}} \\
 {M23} & {positive} & {negative} & {\textcolor[HTML]{FF0000}{reject}} & {\textit{\textcolor[HTML]{1C1E21}{Patwa can't be used as our official language because the rest of the world barely understand our dialect.}}} \\
 {M24} & {positive} & {positive} & {\textcolor[HTML]{00B050}{support}} & {\textit{\textcolor[HTML]{1C1E21}{The aim is to simply add structure and form to what is already there}}} \\
 {M25} & {negative} & {negative} & {\textcolor[HTML]{FF0000}{reject}} & {\textit{\textcolor[HTML]{1C1E21}{How can you preach what you dont teach. i cant even speak that language}}} \\
 {M26} & {positive?} & {positive?} & {\textcolor[HTML]{00B050}{support}} & {\textit{\textcolor[HTML]{1C1E21}{Praktis mek Perfek}}} \\
 {M27} & {unclear} & {unclear} & {\textcolor[HTML]{FF0000}{reject}} & {\textit{\textcolor[HTML]{1C1E21}{Find something better to do...smdh}}} \\
 {M28} & {unclear} & {unclear} & {\textcolor[HTML]{FF0000}{reject}} & {\textit{\textcolor[HTML]{1C1E21}{pussy go look a real work}}} \\
 {F1} & {positive} & {positive} & {\textcolor[HTML]{00B050}{support}} & {\textit{\textcolor[HTML]{1C1E21}{how easy will it be for them to read and understand this our native language}}} \\
 {F2} & {positive} & {positive} & {\textcolor[HTML]{00B050}{support}} & {\textit{\textcolor[HTML]{1C1E21}{learned it in a day, reading fluently from the patwa bible}}} \\
 {F3} & {positive} & {positive} & {\textcolor[HTML]{00B050}{support}} & {\textit{\textcolor[HTML]{1C1E21}{are there hard copy petitions over the island?}}} \\
 {F4} & {positive} & {positive} & {\textcolor[HTML]{00B050}{support}} & {\textit{\textcolor[HTML]{1C1E21}{Its a damn shame n pity ow some ppl a react to this}}} \\
 {F5} & {positive} & {positive} & {\textcolor[HTML]{FF0000}{reject}} & {\textit{\textcolor[HTML]{1C1E21}{trus mi mi write an read patoi and me is a proud Jamaican but mi seh NO}}} \\
 {F6} & {??} & {negative} & {\textcolor[HTML]{FF0000}{reject}} & {\textit{\textcolor[HTML]{1C1E21}{Not interested, let's push English more.}}} \\
 {F7} & {positive} & {positive} & {\textcolor[HTML]{FF0000}{reject}} & {\textit{\textcolor[HTML]{1C1E21}{How many Jamaicans can read Patois? Most of us can only ``talk'' it}}} \\
 {F8} & {negative} & {negative} & {\textcolor[HTML]{FF0000}{reject}} & {\textit{\textcolor[HTML]{1C1E21}{Patio should never be taught in schools at all it sounds ugly and vulgar it's total waste of time}}} \\
 {F9} & {unclear} & {negative} & {\textcolor[HTML]{FF0000}{reject}} & {\textit{\textcolor[HTML]{1C1E21}{This so called patio is just a massive hindrance to development}}} \\
 {F10} & {unclear} & {unclear} & {unclear} & {\textit{\textcolor[HTML]{1C1E21}{yu miin `ku'? As in `Ku yu tu?!'}}} \\
 {F11} & {unclear} & {unclear} & {unclear} & {\textit{\textcolor[HTML]{1C1E21}{It took me a while to read thr above}}} \\
 {F12} & {positive} & {positive} & {\textcolor[HTML]{00B050}{support}} & {\textit{\textcolor[HTML]{1C1E21}{Anyone who is proposing this is a nincompoop}}} \\
 {F13} & {unclear} & {unclear} & {unclear} & {\textit{\textcolor[HTML]{1C1E21}{I can talk it but can hardly read or write it}}} \\
 {F14} & {unclear} & {unclear} & {\textcolor[HTML]{FF0000}{reject}} & {\textit{\textcolor[HTML]{1C1E21}{Will never happen .too hard to read and write it}}} \\
 {F15} & {positive} & {positive} & {\textcolor[HTML]{00B050}{support}} & {\textit{\textcolor[HTML]{1C1E21}{A waa u sah man mi love strate patwa mi seh}}} \\
 {F16} & {positive} & {positive} & {\textcolor[HTML]{00B050}{support}} & {\textit{\textcolor[HTML]{1C1E21}{Same like how we learned to do everything else....}}} \\
 {F17} & {unclear} & {unclear} & {\textcolor[HTML]{FF0000}{reject}} & {\textit{yes }-- in response to\textit{ ``find something else to do''}} \\
 \lspbottomrule
 \end{longtable}

\end{document}
