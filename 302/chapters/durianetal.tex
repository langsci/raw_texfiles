\documentclass[output=paper,colorlinks,citecolor=brown]{langscibook}
\ChapterDOI{10.5281/zenodo.6979323}
\author{David Durian\affiliation{Pennsylvania State University} and
Melissa Reynard\affiliation{North Carolina State University} and
Jennifer Schumacher}

\title[Apart, and yet a part]{Apart, and yet a part: Social class, convergence, and the vowel systems of Columbus African American English and European American English}


\abstract{\sloppy \citet{Thomas1989vowel} and \citet{Durianschumacher2010} found that increased segregation among working-class African Americans and European Americans in Columbus, Ohio in the second half of the 20th century, led to conflicting patterns of divergence and convergence in speaker’s vowel systems rather than divergence as in other American cities \citep{thomas_yaegerdror2010}. However, this research did not investigate vowel systems of middle-class residents who had been less impacted by segregation during that time period. This paper attempts to fill this gap by comparing the vowel systems of African and European American Columbusites of both working- and middle-class backgrounds. Results reveal that both sets of African Americans show convergence with European Americans for fronting of /u:/, /ʊ/, and /oʊ/. However, middle-class African Americans display stronger similarities with middle-class European Americans for retraction of /æ/, /ɑ/, /ɛ/, and /ɪ/ suggesting greater levels of integration among middle-class than among working-class Columbusites.


\keywords{Columbus Vowel Systems, Columbus African American English, Columbus English, Language Variation and Change, Dialect Variation, Dialect Covergence, Sociolinguistics, Phonetic Variation, Sociophonetics, African American Vernacular English}

}

\IfFileExists{../localcommands.tex}{
  \addbibresource{../localbibliography.bib}
  % add all extra packages you need to load to this file

\usepackage{tabularx,multicol}
\usepackage{url}
\urlstyle{same}

\usepackage{listings}
\lstset{basicstyle=\ttfamily,tabsize=2,breaklines=true}

\usepackage{langsci-basic}
\usepackage{langsci-optional}
\usepackage{langsci-lgr}
\usepackage{langsci-osl}
% \usepackage{./langsci/styles/langsci-lgr}
% \usepackage{./langsci/styles/langsci-osl}
% \usepackage{langsci-gb4e}

\usepackage{tikz}
\usetikzlibrary{patterns,calc}
\pgfdeclarepatternformonly{south east lines}{\pgfqpoint{-0pt}{-0pt}}{\pgfqpoint{3pt}{3pt}}{\pgfqpoint{3pt}{3pt}}{
    \pgfsetlinewidth{0.6pt}
    \pgfpathmoveto{\pgfqpoint{0pt}{3pt}}
    \pgfpathlineto{\pgfqpoint{3pt}{0pt}}
    \pgfpathmoveto{\pgfqpoint{.2pt}{-.2pt}}
    \pgfpathlineto{\pgfqpoint{-.2pt}{.2pt}}
    \pgfpathmoveto{\pgfqpoint{3.2pt}{2.8pt}}
    \pgfpathlineto{\pgfqpoint{2.8pt}{3.2pt}}
    \pgfusepath{stroke}}
    
\usepackage{stmaryrd}
\usepackage{wasysym}
\usepackage{multirow}
\usepackage{caption}
\usepackage{subcaption}
\usepackage{mathrsfs}
\usepackage{qtree}

\usepackage{linguex}


  %pminos do not split footnotes
% \interfootnotelinepenalty=10000 %Footnote in Laporte chapters has to be split SN


%\DeclareIndexNameFormat{default}{%
%\nameparts{#1}%
%\usebibmacro{index:name}%
%{\index[names]}%
%{\namepartfamily}%
%{\namepartgiveni}%
% {}% L1
% {}% L2
%{\namepartprefix}% generates spurious space L3
%{\namepartsuffix}% generates spurious space L4
%}

%  {\DeclareIndexNameFormat{default}{%
%     \usebibmacro{index:name}{\index[names]}{#1}{#3}{#5}{#7}}}

%\DeclareIndexNameFormat{default}{%
%  \usebibmacro{index:name}{\sindex[nom]}{#1}{#3}{#5}{#7}}

%\DeclareIndexNameFormat{default}{%
%  \usebibmacro{index:name}{\sindex[person]}{#1}{#3}{#5}{#7}}
%\DeclareIndexNameFormat{default}{%
%\nameparts{#1} \usebibmacro{index:name}{\sindex[person]]}{\namepartfamily}{‌​\namepartgiven}{\nam‌​epartprefix}{\namepa‌​rtsuffix}}

%\newcommand{\smiley}{:)}

%\renewbibmacro*{index:name}[5]{%
%\usebibmacro{index:entry}{#1}%
%{\iffieldundef{usera}{}{\thefield{usera}\actualoperator}\mkbibindexname{#2}{#3}{#4}{#5}}}

% \newcommand{\noop}[1]{}

%remove for final
%\overfullrule=1mm

\newcommand{\tobi}[2]}}
\renewcommand{\S}[1]{\tobi{#1}{\textsc{*}}}

% this volume references
% puts: [this volume]
% already defined: \citetv
%\newcommand{\citepv}[1]{(\citeauthor{#1} \citeyear*{#1} [this volume])}
\newcommand{\citealtv}[1]{\citeauthor{#1} \citeyear*{#1} [this volume]}

%parentheses around example number
\newcommand{\pref}[1]{(\ref{#1})}

% in-text examples

\newcommand{\lnex}[1]{\textit{#1}} %target lang word
\newcommand{\lnlit}[1]{(lit.: `#1')} %literal reading
\newcommand{\lnlat}[1]{(#1)} % latinization
\newcommand{\lntrans}[1]{`#1'} %translation
\newcommand{\lnexl}[2]%
{\lnex{#1}{} \lnlat{#2}} % ex with latinization
\newcommand{\lnexlat}[3]{\lnex{#1}{} \lnlat{#2}{} \lntrans{#3}} % ex with latinization and tranl.

%ch01
\newcommand{\co}[1]{\mbox{\textbf{#1}}}

%ch09

\newcommand{\cyrbulg}[1]{\begin{otherlanguage*}{bulgarian}#1\end{otherlanguage*}}


%ch10
\newcommand{\nlp}{{\small NLP}}
\newcommand{\mwe}{{\small MWE}}
\newcommand{\rae}{{\small RAE}}
\newcommand{\lvc}{{\small LVC}}
\newcommand{\pos}{{\small P}o{\small S}}
%\newcommand{\todo}[1]{ \textcolor{red}{#1} }

%\renewcommand{\labelenumi}{\theenumi}
%\ainamefmt{{vv}{ll}{, ff}{, jj}} % fullname

\newcommand{\biberror}[1]{{\color{red}#1}}

\newcommand{\osenovaitem}{--~}
  %% hyphenation points for line breaks
%% Normally, automatic hyphenation in LaTeX is very good
%% If a word is mis-hyphenated, add it to this file
%%
%% add information to TeX file before \begin{document} with:
%% %% hyphenation points for line breaks
%% Normally, automatic hyphenation in LaTeX is very good
%% If a word is mis-hyphenated, add it to this file
%%
%% add information to TeX file before \begin{document} with:
%% %% hyphenation points for line breaks
%% Normally, automatic hyphenation in LaTeX is very good
%% If a word is mis-hyphenated, add it to this file
%%
%% add information to TeX file before \begin{document} with:
%% \include{localhyphenation}
\hyphenation{
    Beck-man
    Ngu-yen
    back-chan-nel
    back-chan-nels
    mo-not-o-nous
    ste-reo-typ-i-cal
}

\hyphenation{
    Beck-man
    Ngu-yen
    back-chan-nel
    back-chan-nels
    mo-not-o-nous
    ste-reo-typ-i-cal
}

\hyphenation{
    Beck-man
    Ngu-yen
    back-chan-nel
    back-chan-nels
    mo-not-o-nous
    ste-reo-typ-i-cal
}

  \togglepaper[9]%%chapternumber
}{}


\begin{document}
\AffiliationsWithIndexing{}
\maketitle



\section{Introduction}
Despite sociolinguists’ detailed knowledge of general patterns of variation in African American English (AAE) morphosyntax and phonology/phonetics, the types and extent of phonetic and phonological variation as distributed by socio-economic class among middle and working class speakers, and their similarities and differences to equivalent European American vowel systems in communities throughout the United States, has largely remained understudied (see \citet{britt2015african} for more details).  In particular, several questions remain underexplored: (1) How are the vowel systems of speakers of AAE from working class and middle class backgrounds living within the same community similar to and different from one another? (2) How do these vowel systems compare and contrast with those of working class and middle class European Americans living in the same community? Are they becoming more alike (convergence) or less alike (divergence) as time goes on? (3) What can a more refined understanding of the impact of social class on patterns of linguistic variation bring to our understanding of language variation in AAE?

 As our discussion will illustrate, each of these questions is significant in current studies of AAE, as different communities throughout the US have been found to show different results in previous studies. For instance, at a national level, \citet{labov2006atlas} found in their regional dialect survey of US English that African Americans generally do not appear to be participating in local patterns of vowel shift found among European American speakers in the areas surveyed throughout their study. Meanwhile, studies at the community level find a more complex set of results. For example, in Philadelphia and Chicago, research has shown that strong patterns of divergence can be found among speakers of AAE and European American English (EAE) (e.g., \citealt{LabovHarris1986,Gordon2000}). In contrast, in cities such as Memphis and Texana, NC, research has revealed that stronger patterns of convergence can be found among speakers (e.g, \citealt{fridland2003network,FridlandBartlett2006,childscarpenter2010}). Given this diversity at the community level, researchers have begun to explore this issue in some detail in communities throughout the US in recent years.

\hspace*{-1mm}Studies on the influence of social class on patterns of language variation among African Americans have been generally lacking, with the exception of two early sociolinguistic studies: \citet{wolfram1969} in Detroit and \citet{Pederson1965} in Chicago. Both of these studies found social stratification of linguistic features such as the use of more standard variants among middle class speakers and more non-standard features among working class speakers. In the years since, studies of African American communities have tended to leave out analysis of social class, either leaving out consideration of this social factor altogether, or dealing with it in only a cursory way. Thus, we know little as a field of how social class might influence patterns of vowel variation among African Americans, including how the influence of social class may be similar to, and different from, the influence of social class on other ethnic groups in the US, such as European Americans. Generally, this has led us to have a body of research on AAE in the US that, although rich in complex and meaningful findings on a variety of linguistic issues, has been incomplete in regard to exploring social class as a key factor influencing language variation.

The present study attempts to answer each of the research questions noted above via the instrumental exploration of conversational data obtained from speakers living in Columbus, Ohio. Columbus provides an informative context for exploring contrasts and similarities between both working class and middle class African American and European American vowel systems for several reasons. Columbus is a metropolis located in the heart of the North American Midland, as it has been defined on the basis of both lexical and phonological features by \citet{Carver1987} and \citet{labov2006atlas}. According to the \citeyear{CensusBureau2010} US Census, Columbus has a population of 787,000 residents. Among the population, roughly 28\% are African American and roughly 61.5\% are European American (U.S. Census Bureau 2010). In the urban core, there is frequent contact between working class European Americans and African Americans, resulting from migration patterns among both ethnic groups tracing back to the late 19th century and early-to mid-20th century. Furthermore, there is frequent contact between middle class European Americans and African Americans in areas at the periphery of the core and in surrounding suburban space, as a result of changes to Columbus’s socio-geographic landscape beginning in the 1970s.

\begin{sloppypar}
Taking these attributes into account, we present the results of a pilot study comparing middle class African American and European American speaker vowel systems, as well as looking at how these systems compare to previously documented systems of working class African American and European American speakers, via data collected in Columbus in two studies. In doing so, we focus first on describing and analyzing similarities and differences in the vowel systems of working class and middle class African Americans. This is something which has not been explored in previous studies of Columbus AAE vowel systems (\citealt{Durianschumacher2010,Thomas1989vowel}). Second, we focus on whether AAE and EAE are showing patterns of growing similarity (convergence) or growing dissimilarity (divergence) as varieties of English over time in Columbus. In addition, we consider the relationship of our findings in Columbus to recent studies of vowel variation in AAE and EAE that have investigated convergence and divergence in other communities through the United States. Finally, we consider the social motivations for the class-based patterns of convergence and divergence between middle class and working class African Americans and European Americans in our data.
\end{sloppypar}

Before turning to our analysis of vowel variation in Columbus, however, it will first be useful to understand relevant background on each of the issues we will explore in some detail. In section \sectref{sec:durian:2}, we discuss the key findings of previous comparative studies of vowel variation in AAE and EAE vowel systems in the US. While doing so, we will also explore what these studies have revealed about the importance of considering the influence of community level segregation and integration on the patterns of vowel system convergence and divergence observed. In section \sectref{sec:durian:3}, we turn to providing background on Columbus as a speech community, as well as \citegen{Thomas1989vowel} and \citegen{Durianschumacher2010} earlier studies of Columbus vowel systems.

\section{Previous comparative studies of AAE and EAE vowel systems in the US} \label{sec:durian:2}

Among studies of phonetic and phonological variation in US English, there has been a long history of variation involving the vowel systems of speakers living in communities throughout the United States. This tradition goes back to the earliest studies of vowel pronunciation conducted in the United States -- the regional dialect atlas surveys of Hans Kurath and his colleagues beginning in the 1930s with the \textit{Linguistic Atlas of New England} \citep{Kurath1939}. During the earliest era of study, dialectologists such as Kurath tended to focus on individual vowel pronunciations in individual words, analyzing how these individual pronunciations distinguished one regional dialect of US English from another. At the time, although some of the research conducted did include African Americans, the bulk of the research focused on patterns of variation found primarily among European Americans -- in particular, older rural males (e.g., \citealt{kurath1961pronunciation,wetmore1959low}, \citealt{atwood1951some}).

 Beginning with the work of \citet{labov1972quantitative}, linguists began to look at the vowel system holistically, rather than on a vowel-by-vowel basis. This led researchers to find patterns of vowel variation involving systematic shifts in the pronunciation of several vowels at one time. These vowel shifts can be either chain shifts or parallel shifts, as shown in \figref{fig:durian:1} and \figref{fig:durian:2}. Note that for all vowel classes referenced in this essay, we use Well's (\citeyear{wells1982accents}) keyword notation.



\begin{figure}
\includegraphics[width=\textwidth]{figures/durian_figure_1.1_new.png}
\caption{The Northern Cities Shift (NCS) as per \citet{labov2006atlas}}
\label{fig:durian:1}
 \end{figure}



\begin{figure}
\includegraphics[width=\textwidth]{figures/durian_figure_1.2.png}
\caption{Parallel fronting of the back vowels in GOOSE and GOAT (as per \citet{durian2012new}}
\label{fig:durian:2}
 \end{figure}

\figref{fig:durian:1} shows the Northern Cities Shift, a commonly described chain shift found in the vowel systems of many European Americans living in the US Inland North dialect area (as per \citealt{labov2006atlas}). The numbers in the diagram indicate the stages in which the vowels typically shift. \figref{fig:durian:2} shows a parallel shift involving the fronting of the nuclei of the back vowels GOAT and GOOSE. This vowel shift is found among many European Americans living in the US Midland dialect area (as per \citealt{durian2012new}).

Within this body of research, vowel variation among African American speakers was not looked at systematically with any regularity until the late 1980s with the work of Erik Thomas ([\citeyear{Thomas1989vowel}]/1993) in Columbus, OH. Since that time, a number of studies have been conducted, most prominently since the late 2000s, when \citet{thomas_yaegerdror2010} assembled a paper collection by scholars specifically focused on systematic vowel variation in AAE speech. This recent work typically adopts an instrumental approach to analyzing the patterns of vowel variation, including and analysis using F1 by F2 vowel plots.

 Also, during the 1980s, discussion of whether African American and European American speech patterns were becoming more or less alike as time goes on began with the work of \citet{LabovHarris1986} in Philadelphia. In that study, Labov and Harris looked at several grammatical and consonantal patterns of variation among African Americans and European American Philadelphians. Based on the overall patterns found, Labov and Harris concluded that AAE and EAE in Philadelphia appeared to be showing strong patterns of difference (divergence) between the language varieties as time went on, rather than more similarities (convergence).

  This finding has led to some controversy in the field in the years since, as researchers began to explore whether similar patterns of divergence existed in other communities. This exploration has led researchers to find similar patterns of divergence in cities like Cleveland Heights, OH \citep{Thomas2007phonological}, Detroit \citep{anderson2002dialect}, and the Calumet region of Northwestern IN \citep{Gordon2000}, while others have instead found increasing patterns of similarity among AAE and EAE speakers in cities like Memphis \citep{FridlandBartlett2006,fridland2003network}, Roswell, GA (\citealt{andres2009african}), and Texana, NC \citep{childscarpenter2010}.

  Among the studies of AAE vowel systems carried out since the late 1980s, most have focused on comparing the vowel systems of AAE speakers with those of EAE speakers living in the same community, with an emphasis on exploring patterns of convergence or divergence in those vowel systems. In addition, these studies have similarly focused on age and sex as important secondary characteristics impacting the patterns of vowel variation; none, however, have explicitly explored the impact of social class. This is an issue we will return to later in this section.

  Generally, these studies have tended to find that African and European speakers living in areas that are more Southern or rural tend to have vowel systems showing more similarity to one another as time goes on, while speakers living in areas that are more Northern or urban often show vowel systems that are growing more and more distinct. This is especially true of studies that have been conducted at the community level in these regions. For instance, in studies conducted in both Memphis, a Southern city, and Texana, NC, a more rural (and Southern) community, speakers were found to be showing increased fronting of the back vowels in GOAT and GOOSE, a trend most prevalent in the speech of younger speakers in both communities. Additionally, in Memphis, as well as some other Southern communities, such as Roswell, speakers of both ethnic groups also show growing similarities in showing the reversal of the nucleus of FACE and DRESS, two vowels involved in the US Southern Vowel Shift. This is a pattern that does not often extend to the GOOSE and KIT vowels, however, which are also involved in the general rotation of the long and front short vowels that typifies the Southern Vowel Shift. This pattern of shift is typically found among EAE speakers, but not AAE speakers, in studies of Southern communities.

  In contrast to these patterns, in the US North, studies such as \citet{Thomas2007phonological} and \citet{Gordon2000} have found increasing patterns of divergence in vowel systems of speakers, a trend most prevalent in the speech of younger speakers in both communities. In particular, \citet{Thomas2007phonological} found that speakers in Cleveland Heights seem to be engaging in two distinct vowel shifting trends, with the Northern Cities Shift (shown earlier in \figref{fig:durian:1}) found in the vowel systems of EAE speakers, and what he called the African American Vernacular English (AAVE) Shift (shown in \figref{fig:durian:3}) found in the vowel systems of AAE speakers. The Northern Cities Shift involves the shifting of 6 vowel classes: TRAP, LOT, THOUGHT, DRESS, STRUT, and KIT. In particular, the shift involves the raising and fronting of the nucleus of the TRAP vowel, the fronting of nucleus of the LOT vowel, the lowering and fronting of the nucleus of the THOUGHT vowel, the backing of the nucleus of the DRESS and STRUT vowels, and the lowering and backing of the nucleus of the KIT vowel.

\begin{figure}
\includegraphics[width=\textwidth]{figures/durian_figure_1.3.png}
\caption{The African American Vernacular English (AAVE) Shift, as per \citet{Thomas2007phonological}}
\label{fig:durian:3}
 \end{figure}


In contrast is the AAVE Shift, which involves the raising and fronting of the nucleus of the TRAP, DRESS, and KIT vowels, and the fronting of the nucleus of LOT. This makes the patterns of vowel variation for TRAP and LOT similar to some extent among EAE and AAE speakers in Cleveland, but different for KIT and DRESS. However, because the vowels are interlocked in two distinct patterns of chain shift for speakers of each ethnic group, \citet{Thomas2007phonological} argues that the shifts ultimately show an overall pattern of divergence from one another. This can be seen by comparing the stages over which the shifts occur. In the Northern Cities shift, TRAP fronting and raising is typically seen as the first stage of the shift, followed by LOT fronting. In the AAVE Shift, LOT fronting appears to be the first stage of the shift, followed by TRAP fronting and raising as the second stage.

 Regardless of the differences in vowel variation patterns noted between Southern and Northern cities in the discussion above, a common thread found in many of these previous studies is the influence that patterns of residential segregation appear to have had on the patterns of language use observed. In the North, residential segregation was often stronger historically than in the South, especially during the period of the Great Migration \citep{lemann1991promised,collins2020great}. There, we often see stronger patterns of divergence among speakers, as the stronger degree of residential segregation limited contact opportunities among African Americans with European Americans, as well as other ethnic groups, in many urban settings. This is the case in cities such as Cleveland, Detroit, and Philadelphia, where strong patterns of divergence have been found \citep{Thomas2007phonological,anderson2002dialect,LabovHarris1986}. In the South, residential segregation was often less robust historically, and speakers typically lived in communities where speakers of different ethnicities resided side by side. This situation would allow more contact between speakers, making for language use that would be more similar among speakers. This is the case in areas where stronger patterns of convergence have been reported, especially among younger speakers, such as in Memphis, Texana, or Roswell \citep{FridlandBartlett2006,childscarpenter2010,andres2009african}.

 However, increased integration in a community with speakers of different ethnicities does not by itself guarantee increased contact. We must also consider how the social networks of speakers are impacted by the different patterns of integration, and how much face to face interaction occurs between speakers. Studies have revealed that those with more extensive social contacts with speakers from a wide range of social and ethnic backgrounds are more likely to converge with the language use patterns of these other social contacts, either in overall daily speech or the use of more elaborate codeswitching between AAE and EAE, while those with less extensive social networks will be more likely to show less convergence \citep{LabovHarris1986,childscarpenter2010,britt2015african}.


\section{The social and linguistic context of AAE in Columbus} \label{sec:durian:3}

Having now laid the groundwork for our discussion of the analysis of vocalic variation as it has been studied in previous work on AAE throughout the US, we turn to a discussion of the important issues discussed throughout section \sectref{sec:durian:2} specifically in Columbus. As we do so, it is important to understand how several significant aspects of this city as a speech community are useful for exploring variation in the vowel systems of both working class and middle class AAE and how that variation is similar to, and different from, variation in the vowel systems of EAE speakers of similar social class backgrounds living in the community. First, we discuss the social history of African Americans in Columbus, including how historical patterns of segregation have impacted language contact patterns among European Americans and African Americans in the city. We then discuss the results of two earlier studies of vowel variation conducted in the community that provide the necessary background for the present study.


\subsection{A brief social history of African Americans in Columbus}

To understand the context of contact among European Americans and African Americans in Columbus, it is instructive to look briefly at the community’s social history, particularly the years directly following the Civil War. During this time, and again in the post-World War I and World War II periods, a significant number of African Americans moved to the southern and eastern parts of the urban core to pursue industrial jobs in factories. A portion of this population migrated directly from the South and Appalachia, while others moved first to eastern cities such as Philadelphia and Pittsburgh before later resettling in Columbus \citep{Bryant1983,murphy1970}. In most cases, African Americans migrated to areas in the urban core where they found themselves in daily contact with recent European American migrants of predominantly Upper Southern, Lower Northern, and Appalachian backgrounds, as well as long time Columbus residents, whose families had begun settling in Columbus since the late 1700s \citep{lentz2003columbus}.

Until the 1970s, most African Americans in Columbus were working class, as a result of Columbus having a predominately “separate but equal” community structure, which endured since the founding of Columbus in 1803 \citep{jacobs1994,james1972}. Following the end of the Civil War, this “separate but equal” structure led to decades of discrimination in hiring practices by local businesses, as well as housing segregation, due to \textit{de facto} segregation resulting from restrictive deed covenants and the displacement of members of the African American community during the 1960s due to the construction of Interstates 70 and 71  \citep{Oriedo1982,burgess1994planning}. This developed into a situation where the majority of African Americans in Columbus now live in an eastward arch surrounding the periphery of the original “central core” area \citep{jacobs1994,reecelindsjo2012}. As a result, these factors prevented African Americans from obtaining higher skilled labor positions, either physically, due to geographic distance, or socially, due to job accessibility limitations.

During the late 1960s, however, the situation began to change as a result of the Civil Rights Movement and the passage of the Civil Rights Act of 1968, which put an end to enforcement of overt housing and employment discrimination practices in the community \citep{jacobs1994}. In conjunction with these changes, the Columbus Public Schools underwent changes from a system that was strongly characterized as “separate but equal” to one that was, at first, voluntarily desegregated, as in the late 1960s, but later court ordered to desegregate via the use of busing in 1979 \citep{foster1997september}. Even with these changes in place, present-day Columbus continues to remain strongly socio-geographically stratified by social class, with race playing a significant role as a secondary factory in the process, given the community's history. We can see the effects of these patterns on the Columbus landscape over the period 1970--2010 via Map 1 (\figref{fig:durian:4}) and Map 2 (\figref{fig:durian:5}).  Map 1 shows the areas of Columbus that were predominately African American populated in 1970.  The darkest areas are those where African Americans were most heavily concentrated at the time. Map 2 (\figref{fig:durian:5}) shows the community in 2010. It uses the same conventions as Map 1 and shows the impact of displacement and housing segregation practices in the community. Based on the 2010 U.S. Census, Columbus has a Taueber Index Score of 61.0 (U.S. Census Bureau \citeyear{CensusBureau2010}).

In present day Columbus, a growing middle class African American population tends to reside in more socio-economically and racially mixed parts of Columbus -- either in areas at the periphery of the core to the West and North, to the Northeast of the core in the neighborhood known as North Linden, or in nearby dormitory suburbs. At the same time, a significant portion of the African American population remains working class and living in areas closely surrounding the urban core, due to the lack of economic opportunity to move elsewhere. As a result, contact among many working class European Americans and African Americans continues to occur in areas closest to the core, such as the southeast and east sides, while in areas further from the core, contact between middle class European Americans and African Americans now occurs. We will return to these trends later in our discussion.

\begin{figure}
\includegraphics[width=0.79\linewidth]{figures/durian_fig1.4-1970.png}%
\includegraphics[width=0.19\linewidth]{figures/durian_map_key.png}
\caption{Map 1. Franklin County Census Tract Map, Shaded by Percent Black, 1970 (\citealt{reecelindsjo2012})}
\label{fig:durian:4}
 \end{figure}


\begin{figure}
\includegraphics[width=0.79\textwidth]{figures/durian_fig1.5-2010.png}
\includegraphics[width=0.19\linewidth]{figures/durian_map_key.png}
\caption{Map 2. Franklin County Census Tract Map, Shaded by Percent Black, 2010 (\citealt{reecelindsjo2012})}
\label{fig:durian:5}
 \end{figure}

\subsection{A brief discussion of historical dialect patterns in Columbus speech }

The dialect features of Columbus speech that have emerged during this period are a complex mixture of Northern and Southern features and are strongly Midland in character. As \citet{Thomas2001acoustic} and \citet{durian2012new} have discussed, Columbus speech of the early 20th century included features typically associated with the Southern Shift, such as the frontward movement of the nuclei of MOUTH, GOAT, FOOT, GOOSE, and SHOES, and historically North Midland features, such as r-fullness, the backing of LOT, and the merger of the NORTH and FORCE classes. This was especially true of working-class European American English, but also true to a lesser extent in middle class speech \citep{durian2012new}.  Less diachronic information about Columbus AAE is available. Features traditionally assumed to be most strongly affiliated with more Southern or older supra-regional AAE were a strong element of AAE in Columbus in the early 20th century.  Some of these features include r-lessness; glide weakening of PRICE in open syllables and before voiced consonants (a similar pattern to Southern speech); and the tendency for GOOSE, SHOES, GOAT and MOUTH to remain back (\citealt{Thomas1989vowel}).

 During the second half of the 20th century, these patterns of difference between ethnic groups appear to have diminished. Among working class African Americans and European Americans, this can be seen via the results of two previous studies of vowel systems in Columbus. The first is Thomas’(\citeyear{Thomas1989vowel}) primary impressionistic study based on data collected during the 1980s from working class African Americans and European Americans born between 1968--1970. The second is \citegen{Durianschumacher2010} primarily instrumental study that analyzed speech among working class African American and European American speakers belonging to two age cohorts. One group was older speakers who were born between 1950--1960, while the other was a younger group of speakers who were born between 1969--1985. Both studies found not only the decreased presence of the historical AAE features we discussed earlier, but also evidence that Columbus African American speakers have begun to realize a partial merger of LOT/THOUGHT before /t/ and the frontward movement of the nuclei of MOUTH, GOAT, FOOT, GOOSE, and SHOES.

In addition to these results, Durian et al. (\citeyear{Durianschumacher2010}) also found some evidence of nucleus lowering for GOAT among some older female and younger male African American speakers, which is not found as pervasively among working class European American speakers. Furthermore, tendencies were found among African American speakers towards raised KIT, DRESS, and TRAP articulations, lowered FOOT realizations, and fronter realizations of LOT and THOUGHT. These patterns indicate evidence of divergence from European American patterns for these vowel classes. Working class European Americans, in comparison, showed tendencies toward backer articulations of LOT and THOUGHT, non-lowered FOOT realizations, and either non-raised or mildly lowered KIT and DRESS realizations. As well, the TRAP vowel among many working-class European Americans shows a strong backing and lowering trend over time, increasing as speakers get younger in age. In addition, African Americans tended to show evidence of more robustly raised STRUT-realizations when compared to European Americans. Across speakers, the majority of these patterns of convergence and divergence were found to have stronger tendencies among younger speakers. \tabref{tab:durian:1} summarizes the differences described here between working class AAE and EAE.

 In middle class speech, it is currently not known how the varieties are converging or diverging, since no previous studies of middle class African American vowel systems have been completed. These trends will be investigated in the following sections of this analysis. Before turning to this analysis, it is useful to present a brief recap of what is known about vowel variation trends in middle class European American speech. The most extensive study of this group to date is \citet{durian2012new}, which studied the vowel systems of four generations of Columbusites born between 1895 and 1990. Among middle class European Americans, Durian found increasing patterns of fronting of the nuclei of GOAT, STRUT, FOOT, GOOSE, and SHOES. These tendencies each increase as speaker age decreases. Durian also found fronting for the nucleus of MOUTH among many speakers born before the 1970s, but this trend has begun to reverse itself among younger speakers, who now show nucleus retraction. Furthermore, Durian found a strong tendency towards merger or near merger of LOT and THOUGHT, a trend also found extensively in Labov et al  (\citeyear{labov2006atlas}'s) mostly middle-class Columbus speakers, interviewed for the \textit{Atlas of North American English}.

\begin{table}
\begin{tabularx}{\textwidth}{X}
\lsptoprule

\textbf{Working class AAE vs. EAE}\\
\midrule
{\textbf{AAE and EAE show convergence for:}}

- Fronting of the nucleus of SHOES, GOOSE, FOOT, GOAT, and MOUTH\\
\tablevspace

\midrule
{\textbf{AAE and EAE show divergence for:}}

- The AAVE Shift (Raised articulations of KIT, DRESS, and BAT, and fronter articulation of LOT)\\
\tablevspace
- Fronter articulation of THOUGHT and raised articulation of STRUT\\
\lspbottomrule
\end{tabularx}


\caption{Comparison of results among working-class AAE and EAE speakers}
\label{tab:durian:1}
\end{table}

In addition, \citet{durian2012new} found strong trends for backing of the nucleus of LOT, and backing/lowering of the nuclei of TRAP, DRESS, and KIT, a characteristic of Columbus speech that resembles significantly the same vowel shifting trends for these vowel classes found previously in Canada (e.g, \citealt{ClarkeYoussef1995,boberg2005canadian,roeder2010canadian}) and California (e.g, \citealt{luthin1987story,kennedy2012chain,podesva2015country}). Given this similarity, \citet{durian2012new} coined the term ``The Third Dialect Shift'' to describe the shift in Columbus, since all three dialect areas have essentially the same shift as well as being unified previously as a dialect group by \citet{labov1991three} based on the overlapping occurrence of the LOT/THOUGHT merger as a dialect feature in each area. As with other features, usage of the Third Dialect Shift appears to be increasing in middle class Columbus speech as time goes on. The Third Dialect Shift is shown in \figref{fig:durian:4}, with numbers indicating stages of the chain shift.

Compared to working class EAE speakers, middle class EAE speakers show many of the same vowel shifting tendencies. However, the extent to which those features are used within each social class group differs. Among the features of EAE that are shared between both groups, fronting of the nucleus of the back vowels SHOES, GOOSE, FOOT, GOAT, and MOUTH is more robust among working class speakers, and the merger of the LOT and THOUGHT classes is often more complete as a process among working class EAE speakers than middle class EAE speakers. In contrast to this, use of the Third Dialect features (especially backing and lowering of the nucleus of TRAP, DRESS, and KIT) are usually stronger among middle class EAE speakers than working class EAE speakers. \tabref{tab:durian:2} presents a summary of these trends as they are found in working and middle class EAE, while also showing the contrast in use with the patterns discussed earlier in this section for working class AAE.

\begin{figure}
\includegraphics[width=\textwidth]{figures/durian_figure_1.6.png}
\caption{ The Third Dialect Shift (as described in \citealt{durian2012new})}
\label{fig:durian:6}
 \end{figure}

As a final note in closing, although not labeled as such in their studies, aspects of the Third Dialect Shift, in particular the backing and lowering of the nucleus of TRAP, have been found in several studies of other US Midland cities in recent years. This includes Johnstown, OH  (\citealt{thomas1996sixthgrade}); Indianapolis, IN \citep{Fogle2008}; and several cities located in Southern, IL \citep{bigham2010correlation}. It should be noted that although \citet{labov2006atlas} did not find the Third Dialect Shift in their analysis of the US Midland, the occurrence of the Third Dialect Shift in a variety of these other, individual Midland cities, demonstrates this chain shift is a more general US Midland dialect feature, and it will be discussed as such throughout section \sectref{sec :durian:4}.

\begin{table}
\small
\begin{tabularx}{\textwidth}{lX}

\lsptoprule

Working class AAE & - Fronting of the nucleus of SHOES, GOOSE, FOOT, GOAT, and MOUTH \\
\tablevspace

& - Fronter Articulation of THOUGHT and raised articulations of STRUT\\
\tablevspace

& - The AAVE Shift (Raised Articulations of KIT, DRESS, and BAT, and Fronter Articulation of LOT) \\
\tablevspace

\midrule
Working class EAE & \textbf{- Fronting of the nucleus of SHOES, GOOSE, FOOT, GOAT, and MOUTH}\\
\tablevspace

&\textbf{- (Near) merger of the LOT and THOUGHT vowels}\\
\tablevspace

& - The Third Dialect Shift (Lowering and backing of the nucleus of LOT, TRAP, DRESS, and KIT)\\
\tablevspace

\midrule
Middle class EAE & - Fronting of the nucleus of SHOES, GOOSE, FOOT, GOAT, and MOUTH\\

& - (Near) merger of the LOT and THOUGHT vowels\\
\tablevspace

& \textbf{- The Third Dialect Shift (lowering and backing of the nucleus of LOT, TRAP, DRESS, and KIT)}\\
\lspbottomrule
\end{tabularx}
\caption{Vowel variation features of Columbus AAE and EAE found in previous studies. (Features used most strongly by a given EAE group are shown in bold).}
\label{tab:durian:2}
\end{table}



\section{A comparative analysis of present-day Columbus AAE and EAE vowel systems} \label{sec :durian:4}

Now that we have established an understanding of previous patterns of vowel variation found in the speech of working class African American and European Americans, as well as middle class Europeans Americans, we move to presenting the results of our comparative analysis of middle class African American vowel variation trends. The analysis follows the model provided in \citet{Durianschumacher2010}, by presenting a side-by-side analysis of representative speaker vowel plots for African and European American Columbusites, with commentary on the significant trends found with the plots. This approach allows patterns of vowel variation among middle class speakers to be understood within the wider context of the previous studies of Columbus speech discussed above, while also allowing us to showcase the new information on middle class Columbus African American vowel systems available from our data. Following the comparative analysis of middle-class speech trends in section \sectref{sec:durian:4.2}, we move in section \sectref{sec:durian:4.3} to a comparative analysis of our middle-class data set with the working-class data set analyzed by the first and third authors in \citet{Durianschumacher2010}.

\subsection{Methodology}
In order to make a meaningful comparison between the \citet{Durianschumacher2010} data set and the middle-class data we analyze here, two age cohorts were analyzed and plotted. Older speakers were born from 1955--1963, while younger speakers were born from 1976--1985. Speakers were chosen from a larger sample of speakers interviewed for two sociolinguistic studies of Columbus speakers conducted by Don Winford and his associates in 2007--2008, and by David Durian in 2008--2009. All subjects were speakers who were born and raised in the greater Columbus metropolitan area and all have continued to live in the Columbus metropolitan area as adults.

In both surveys, sociolinguistic interviews were conducted by the field workers involved, and large samples of conversational speech were collected during the interview process. African American field workers elicited data from African American speakers via sociolinguistic interviews, while data for European American speakers were elicited via interviews by a European American field worker. In the case of the African American interviews, all data collected was conversational. In the case of the European American interviews, data was collected from several tasks, including conversational speech, word lists, and dialect term elicitation. Through the analysis presented here, all data analyzed is drawn from conversational speech.

The European American speakers’ vowel systems were all originally analyzed in \citet{durian2012new}, and the speakers selected here for comparison thus represent only a small sample of the larger available pool of 62 speakers included in that study. As such, the reader is directed to the vowel plots included in Appendix B of \citet{durian2012new} to see the data for all 62 European American speakers. Socioeconomic status for all speakers in both studies was determined using available information from the interview recordings. Not all informants discussed this information to the same degree, and so our assignment of class is limited to  occupation level of adult informants,  and mean household income of the area in which informants were raised (if known) during the time of their childhood.

The sociolinguistic interviews were recorded at 44.1 KHz to a Sony DAT recorder. These files were then digitized as \texttt{.wav} files for acoustic analysis in PRAAT \citep{BoersmaWeenink2020}, using a variable window of 10--14 LPC coefficients depending on the quality of the token. Initial formant measurements were taken by all three authors, aided by a custom-made formant extraction script in PRAAT, and adjustments were made by hand to correct problematic measurements, when needed. The data were checked for inter-rater reliability across measurements following the initial coding. For all vowel classes analyzed here, no tokens with a preceding or following /r/ or /l/ were used, and tokens with a following nasal were also excluded. In addition, following velars and nasals were excluded for the BAT class.  Each of these segment types was avoided as they can lead to irregularities in formant patterns that make instrumental vowel analysis more difficult. No more than three instances of a particular lexeme were extracted for inclusion in our mean measurements. Following formant measurement, the data were normalized using the \citet{Lobanov1971} $z$-score formula. Data were normalized using measurements extracted across an entire speaker’s vowel system from the conversational speech portions of the interviews. F1 and F2 were measured at three points in the vowel’s duration -- 20\%, 50\%, and 80\%. Measurements of F1 were taken to represent vowel height, while measurements of F2 were taken to represent vowel frontness/backness.

Vowel plots were then created using R \citep{Rprogram}, and the data compared across speakers. For all vowel plots used in this analysis, ten tokens per vowel class were measured, with the value of each class plotted representing the mean across those ten tokens. These values are plotted using Wells' \citeyear{wells1982accents} notation system for all vowel classes save two: BAT and SHOES. The SHOES vowel class is a special subclass of /u/ that separates out pre-vocalic coronals, as these segments have been found to condition significant fronting versus non-coronals in previous studies of Columbus vowel systems (e.g., \citealt{durian2012new,Durianschumacher2010,Thomas2001acoustic}). Meanwhile BAT is a combined vowel class that includes short /ae/ tokens belonging to both Wells's BATH and TRAP classes.  Note that the PRICE class has been left out of our vowel plots to allow a clearer picture of how BAT, LOT, THOUGHT, and MOUTH occupy the low vowel space for each speaker.

For analysis and vowel plotting purposes, the traditional monophthongs (KIT, DRESS, BAT, LOT, THOUGHT, STRUT, and FOOT) use a measurement of the steady state taken at the 50\% point of the vowel’s duration. Vowels that are commonly treated as diphthongs (FLEECE, FACE, MOUTH, GOAT, GOOSE, and SHOES) use measurements taken at 20\% and 80\% to represent the nucleus and offglide, with arrowheads marking the offglide.

\subsection{Middle class comparative analysis}\label{sec:durian:4.2}

Turning first to the older female speakers, shown in \figref{fig:durian:oldwomen}, we can see that the older African American woman (S001) has a more conservative vowel system than the older European American woman (S002). This is most clearly seen by the position of GOOSE, GOAT, BAT, and DRESS in S001's vowel system. BAT is fronter and higher, while GOOSE and GOAT are somewhat backer than these same vowel classes in S002's system, as most easily noted by comparing the relative position of these vowel classes to THOUGHT noted in both women's systems. As discussed in more detail in \citet{durian2012new}, S002's BAT and DRESS show somewhat backed and lowered realizations, trends indicating the early progression of the Third Dialect Shift. This trend is typical for women born around the same time as S002 in Columbus. In terms of other vowel trends found in previous studies of Columbus, S001 also has a LOT and THOUGHT that are more distinctive than S002, who shows closer means, which, as also discussed in more detail in \citet{durian2012new}, are indicative of a partially merged set of classes. Compared to other speakers in Columbus, S001's FLEECE is also more diphthongal, as indicted by the somewhat pronounced offglide. Meanwhile, S001's FACE, KIT, and MOUTH classes are in the ``standard position'' for a speaker from Columbus.

\begin{figure}
  %\subfigure %[caption 1 goes here]
  {
\includegraphics[width=.45\textwidth]{figures/durian_S001_NORM.png}
}
  %\subfigure %[caption 2 goes here]
  {
 \includegraphics[width=.45\textwidth]{figures/durian_S002_NORM.png}
 }
      \caption{Vowel system of older women}
      \label{fig:durian:oldwomen}
\end{figure}

  In comparison to S001 and S002, as shown in \figref{fig:durian:youngwomen},  S003 and S004 show more pronounced patterns of vowel shift for many of the vowel classes indicated to be undergoing notable variation in middle class European American English in \citegen{durian2012new} analysis. These include the fronting of nuclei of SHOES, GOOSE, FOOT, and GOAT, and the backing/lowering of the nucleus of BAT, DRESS, and KIT. These changes together represent three of the four that \citet{durian2012new} coupled together as the Third Dialect Shift. In addition, both women show some retraction of the nucleus of MOUTH, and both speakers show at least partial merger of LOT and THOUGHT. Here the overlap between classes is more pronounced in S004's vowel system, a trend that is more indicative of the larger community, where younger African American women are also beginning to show the LOT/THOUGHT merger, but less frequently than European American women.

\begin{figure}
  \subfigure %[caption 1 goes here]
  {
 \includegraphics[width=.45\textwidth]{figures/durian_S003_NORM.png}
 }
  \subfigure%[caption 2 goes here]
  {
 \includegraphics[width=.45\textwidth]{figures/durian_S004_NORM.png}
 }
    \caption{Vowel systems of younger women}
     \label{fig:durian:youngwomen}
\end{figure}

As with the older speakers, the European American speaker (S004) shows each of these vowel change tendencies more robustly than the African American speaker (S003). However, both women show change tendencies in the same direction, and both have changes in that direction that show an intensification of these patterns versus the S001 and S002. This leads us to conclude that: a) each of these vocalic change trends appear to be a change in progress for both middle class European and African American women in Columbus; b) middle class European American women are leading these changes versus their African American counterparts; and c) by showing changes in the same direction, European American and African American females show strong patterns of convergence for these vowel classes.

Turning to our male speakers, we can see that, as with our female speakers, older male speakers in both groups have more conservative patterns of vowel variation than younger speakers. This is especially true for the nuclei of the back vowels GOOSE, SHOES, GOAT, and FOOT, as well as the short front vowels BAT, DRESS, and KIT. Generally, between groups, European American men show more robust lowering of the nucleus of BAT and DRESS versus African American men, and less robust, but still notable lowering of the nucleus of KIT. This is shown clearly in the vowel plots for S005 and S006, both shown in \figref{fig:durian:oldermen}. For the back vowels, European American men also show somewhat stronger fronting trends for the back vowel classes GOOSE, SHOES, and GOAT, than African American men, as also shown in the plots. For FOOT, the African American male (S005) shows a stronger pattern of fronting than the European American male. This is a trend that shows less consistency among men in this age group than the other trends discussed here. In other words, sometimes African American men show more fronting for this class, as in the plots here, while other times, European American men show more fronting.

\begin{figure}
  \subfigure%[caption 1 goes here]
  {
 \includegraphics[width=.45\textwidth]{figures/durian_S005_NORM.png}
 }
  \subfigure%[caption 2 goes here]
  {
 \includegraphics[width=.45\textwidth]{figures/durian_S006_NORM.png}
 }
    \caption{Vowel system of older men}
    \label{fig:durian:oldermen}
\end{figure}

\begin{sloppypar}
For the LOT and THOUGHT merger, the African American male (S005) does not show signs of extensive merger, whereas the European American male does. In addition, the African American male (S006) shows strong retraction of MOUTH, whereas the European American male does not. Given that similar differences are found between our younger men for these classes, we argue these differences may represent the residual influence of Southern speech on male speech in Columbus, while the retraction of MOUTH may represent a secondary influence of Pittsburgh speech. This is an idea we will return to in the next section.
\end{sloppypar}

For the younger men (S007 and S008) and shown in \figref{fig:durian:youngmen}, like younger women, we see a continuation of the vowel variation trends found in the vowel systems of the older men, with a stronger increase in each of those trends in the younger men's vowel systems. For SHOES, GOOSE, and GOAT, in particular, the younger men show more fronting than the older men. Between the younger men, the European American male S007 shows stronger fronting for SHOES, GOOSE and GOAT than S008, the African American male. We also see further lowering and backing of BAT and DRESS for the younger men versus the older men, with the European American male showing stronger lowering trends than the African American male. KIT also continues to show backing, although the data here does not suggest a strong generational difference between the older men's groups and the younger men's groups, nor a strong difference between men based on race. As with the back vowels, the vowel lowering/backing trends for BAT, DRESS, and KIT show a continuation and intensification of the vowel variation patterns found in older men's speech.

%THESE SHOULD HAVE A SINGLE FIGURE CAPTION
\begin{figure}
  \subfigure%[caption 2 goes here]
  {
 \includegraphics[width=.45\textwidth]{figures/durian_S007_NORM.png}
 }
  \subfigure%[caption 2 goes here]
  {
 \includegraphics[width=.45\textwidth]{figures/durian_S008_NORM.png}
 }
\caption{Vowel system of younger men}
\label{fig:durian:youngmen}
\end{figure}

Overall, the male trends shown here lead us to reach similar conclusions to those we drew for women's speech. Namely: a) each of these vocalic change trends appear to be a change in progress for both middle class European and African American men in Columbus; b) middle class European American men are leading these changes versus their African American counterparts; and c) by showing changes in the same direction, European American and African American males show strong patterns of convergence for these vowel classes. As with women's speech, these findings suggest middle class African American males also appear to be making use of the Third Dialect Shift, albeit to a lesser extent than either European American males or African American females. Related to this trend is also the noticeable difference in LOT and THOUGHT realization for African American men in our study -- the lack of LOT/THOUGHT merger. Since LOT is not as back as it would be due to the merger in these men's vowel spaces, this allows less space for BAT to retract, and by analogy, also less space for DRESS to retract.

In sum, the analysis in this section reveals that middle class African Americans and European Americans are showing increasing convergence in their vowel systems. This trend can be seen most clearly in two areas of the vowel system: a) the back vowels SHOES, GOOSE, FOOT, and GOAT; and b) the short front vowels BAT, DRESS, and KIT. More generally, b) suggests that both ethnic groups are converging by showing increasing use of the Third Dialect Shift. At the same time, these groups also show some increasing divergence from one another, with LOT and THOUGHT (near-)merger found more heavily in middle class EAE than in middle class AAE.  These realization patterns are summarized in \tabref{tab:durian:3}.

\begin{table}
\begin{tabularx}{\textwidth}{X}
\lsptoprule
\textbf{Middle class AAE vs. EAE}\\
\midrule
{\textbf{AAE and EAE Show Convergence for:}}\\
- Fronting of the nucleus of SHOES, GOOSE, FOOT, and GOAT\\
\tablevspace
- The Third Dialect Shift (in particular, lowered and backed articles of BAT, DRESS, and KIT)\\
\tablevspace
\midrule
{\textbf{AAE and EAE Show Divergence for:}}\\
- Lack of LOT/THOUGHT merger (among male speakers)\\
\lspbottomrule
\end{tabularx}
\caption{Comparison of results among working-class AAE and EAE speakers}
\label{tab:durian:3}
\end{table}



\subsection{Summary: A comparison of middle class\slash working class AAE and EAE vowel systems} \label{sec:durian:4.3}

Taken together, the trends revealed in section \sectref{sec:durian:4.2} suggest a general pattern of convergence is found among middle class African Americans and European Americans not only for the back vowels SHOES, GOOSE, FOOT, and GOAT, but also the short front vowels BAT, DRESS, and KIT. This contrasts with the working-class data analyzed for Columbus in  \citet{Durianschumacher2010}, where we find convergence between African Americans and European Americans for fronting of the nuclei of the non-low back vowels SHOES, BOOT, FOOT, and GOAT, but then divergence for the nuclei of KIT, DRESS, BAT, THOUGHT, and LOT. As mentioned earlier, working class African Americans show the use of the AAE Shift in their speech. That is, KIT, DRESS, and BAT raising, and THOUGHT and LOT fronting, while working class European Americans show backer articulations of THOUGHT and LOT and non-raising or mild lowering of KIT, DRESS, and BAT.

 For the working-class speakers, divergence of the KIT, DRESS, and BAT classes appears to be a result of working-class African Americans participating in the AAVE Shift (as per \citealt{Thomas2007phonological}), while for middle class speakers, the convergence of these classes appears to be the result of middle-class African Americans participating in the Third Dialect Shift (as per \citealt{durian2012new}).


\begin{table}
\small
\begin{tabularx}{\textwidth}{QQ}
\lsptoprule

\textbf{Working class AAE vs. EAE} & \textbf{Middle class AAE vs. EAE}\\
\midrule
{\textbf{AAE and EAE show convergence for:}} & {\textbf{AAE and EAE show convergence for:}}\\
- Fronting of the nucleus of SHOES, GOOSE, FOOT, GOAT, and MOUTH & - Fronting of the nucleus of SHOES, GOOSE, FOOT, and GOAT\medskip


- The Third Dialect Shift (lowered and backed articulations of LOT, BAT, DRESS, and  KIT)\\
\tablevspace
\midrule
{\textbf{AAE and EAE Show Divergence for:}} & {\textbf{AAE and EAE Show Divergence for:}}\\
- Fronter articulation of THOUGHT and raised articulation of STRUT\medskip

- The AAVE Shift (Raised articulations of KIT, DRESS, and BAT, and fronter articulation of LOT)  & - Lack of LOT/THOUGHT merger (among male speakers)\\
\lspbottomrule
\end{tabularx}

\caption{Comparison of results among middle- and working-class AAE and EAE speakers}
\label{tab:durian:4}
\end{table}

\begin{sloppypar}
More generally, considering not only the number of vowels showing stronger similarities, but also the degree of similarity found between middle class speakers versus working class speakers, our middle-class African Americans show stronger convergence with the vowel systems of the middle-class European Americans than our working-class African Americans show with working class European Americans. The overall differences between these patterns of convergence and divergence are summarized in \tabref{tab:durian:4}. Vowel plots of the working class African Americans showing the patterns referenced here can be found in \citet{Durianschumacher2010} for comparison.
\end{sloppypar}

As also shown in \tabref{tab:durian:1}, although overall there are stronger patterns of convergence in middle class speech, one important difference remains between the middle class African Americans and European Americans in our data set. African Americans show a notable difference in the use of merged LOT and THOUGHT realizations. This difference is most notable in the speech of men, who do not appear to be engaging in the merger. However, women in our study also show less extensive participation than their European American counterparts. As younger women appear to be beginning to make some use of the merger, however, it remains to be seen whether this may eventually impact men's speech, as well. This question will make for an interesting area of study in future research.


\section{The cross-regional and social implications of vocalic variation in Columbus}
With regard to the relationship of Columbus African American speech to African American speech elsewhere, it would appear that, over time, the non-low back vowels of both working class and middle-class Columbus African Americans are becoming more like those described recently for certain other communities, namely Hyde County, NC (\citealt{WolframThomas2002}), Texana, NC (\citealt{childscarpenter2010}), and Memphis, TN \citep{fridland2003network,FridlandBartlett2006}. In these communities, similar tendencies towards back vowel fronting among African Americans and European Americans have been found. This is perhaps unsurprising, since historically, these vowels have typically shown evidence of this Southern-Shift-like-tendency in each of these areas, although the influence in Columbus may actually be from western Pennsylvania instead of the South proper.

The trends found among working class African American speakers for the raising of BAT, DRESS, and KIT, and the fronting of LOT, also resemble those found in Memphis by  \citet{FridlandBartlett2006}, as well as Brooklyn, NY, and Cleveland Heights, OH by \citet{Thomas2007phonological}.  This suggests that working class African American speech in Columbus may be showing stronger alignment with more recent supra-regional African American English norms than middle class speech for these vowel classes. In particular, the supra-regional norm in play here appears to be participation in the AAVE Shift.  On the other hand, the tendency towards lowering and backing of KIT, DRESS and BAT (the Third Dialect Shift) among middle class African Americans suggests middle class African American English shows stronger alignment with local Columbus norms, given their occurrence among at least some middle class European Americans, as well.

Turning to an exploration of potential social motivations for the patterns of convergence and divergence by social class in our study, it is important to consider the context in which contact between African Americans and European Americans in Columbus occurs. This situation may be leading to a complex situation of “home” vs. “school” language influence impacting the patterns shown in our study. For instance, the pattern of back diphthong convergence, especially among our young speakers, may be best explained by considering the impact of Columbus’s school desegregation policies in the late 1960s to mid 1990s. Our older speakers went to school either before desegregation occurred or during the period when desegregation was purely voluntary. This led to a situation where only small groups of students began to attend more desegregated schools. On the other hand, following the implementation of busing in 1979, schools typically became strongly desegregated, such that schools that may have been 80\% African American previously were now roughly 50\% African American \citep{foster1997september}. Thus, there was much higher face-to-face daily contact among black and white speakers during the “busing era” as a result, which have led to this pattern of shift among realizations.

Although students may now have been attending more racially mixed schools during the day, after school, in their home community, many working class students returned to areas that were majority African American. This would continue to facilitate strong daily face-to-face interaction among African Americans, which might also lead to an increase in usage of more variables that may be somehow more ethnically marked. Hence, we see a simultaneous increase in the use of variables marked by divergence, such as raised articulations of KIT, DRESS, and BAT, as well as fronting of LOT and THOUGHT among our working class younger speakers. Considering that both African American class groups show strong patterns of convergence for the back vowel diphthongs, this suggests these variables may not be ethnically marked, perhaps due to their having less perceptual saliency as markers of ethnic identity among community members. Such a contrast would explain why working class speakers show contrasting patterns of convergence and divergence, dependent on the vowel subsystem under discussion.

The general pattern of convergence for both the front and low back vowels, as well as the back vowel diphthongs, among middle class speakers, are more straightforward to explain. The areas in which middle class African Americans live are more strongly integrated than the areas in which most working class African Americans reside. It is plausible that the frequency of daily face-to-face interactions between ethnic groups would be increased in this setting. In addition, among our speakers included in this study, the social networks of middle class speakers are often more diverse and expansive than those of their working class colleagues. These expanded networks include regular interactions with speakers of different ethnic groups -- in particular, European Americans -- a fact discussed by all of the middle class African Americans surveyed for this study during their sociolinguistic interviews. As a result, this stronger integration among speakers, including extended interaction with European American speakers, seems to be encouraging stronger patterns of convergence, a fact reflected by the patterns of vowel system convergence shown by the speakers including here in our analysis.

Clearly, given the complexities of the social situation in Columbus, these are issues that require a more detailed study for confirmation. For now, we find the results of our pilot study have provided us with some possible explanations for these patterns. These issues, as well as the more robust documentation of the comparative patterns of vowel variation noted, are matters we hope to explore in a future study.

\section*{Acknowledgments}
We wish to thank Yolanda Holt and Tinisha Tolbert for conducting interviews with middle class African American informants, as well as Rick Jones and Tammy Snow for conducting the fieldwork with working class African American informants. We also thank Don Winford for providing us with access to this data.

\section*{Abbreviations}
\begin{tabularx}{.5\textwidth}{@{}lQ}

\textsc{aae} & African American English  \\
\textsc{eae} & European American English \\
\end{tabularx}

\printbibliography[heading=subbibliography,notkeyword=this]
\end{document}
