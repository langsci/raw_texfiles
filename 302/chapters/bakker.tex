\documentclass[output=paper,colorlinks,citecolor=brown]{langscibook}
\ChapterDOI{10.5281/zenodo.6979305}
\author{Peter Bakker\affiliation{Aarhus University}}
\title[The pith of pidginization]{The pith of pidginization: How Francophones facilitated the simplification of French  through Foreigner Talk  in the Lesser Antilles}
\abstract{In the early 17\textsuperscript{th} century, French speakers simplified their language when communicating with indigenous populations and later Africans in the Lesser Antilles. Through a process of mutual adjustment, a pidgin developed. French missionaries were the only ones to leave linguistic observations in the 17\textsuperscript{th} century. The type of reduction as observed in their Foreigner Talk appears also in the speech of Africans and Amerindians. The documentation leads to the conclusion that the resulting French pidgin was a consequence of interaction and a common creation of the three groups.

\keywords{pidgins, pidginization, Caribs, Foreigner Talk, Blacks, French, creoles, Africans, Amerindians, Lesser Antilles}
}


\begin{document}

\maketitle
\section{Introduction} \label{sec:bakker:1}
 
This paper deals with a topic that received attention in creole studies until the mid 1980s, but not much after. Several researchers compared the simplification witnessed in pidgins with foreigner talk. If we compare, for instance, English with English pidgins, it is undisputable that the pidgins display considerably less grammatical complexity than the source language. Likewise, simplification can be observed, at least in some cultures, in the way that people address persons who do not (yet) speak the language, such as the speech to foreigners (“foreigner talk”, “alien talk”), or to young children (“baby talk”, “motherese” or rather “caretaker talk”).

Several studies comparing foreigner talk and pidgins were published until the mid-1980s, e.g. (\cite{Ferguson1971}; \cite{Meisel1975,Meisel1977}; \cite{harding1984} and \cite{Hinnenkamp1984}). After this, the topic received almost no attention, but there seems to be a revival in recent years, e.g. (\citet{Fedorova2006};  \citet{Versteegh2014} and \citet{Avram2017}).
 
Foreigner talk is the term used for speech by native speakers of a language who deliberately adjust their language through simplification when speaking to non-native speakers and language learners. Only one of the parties uses it, and it may be more or less conventionalized. A pidgin is a compromise form of speech used by two or more groups who do not share a fully-fledged language. A pidgin is typically lexically and grammatically reduced.

In this paper I continue this tradition of comparing pidgin and foreigner talk data from the 17\textsuperscript{th} century. I will compare the simplified – foreigner talk – French as spoken by native French speakers in addressing indigenous people and people of African descent, with the simplified – or pidginized – French recorded from the mouth of Amerindians and Blacks. Not only do they appear quite similar, in fact the metalinguistic comments by eyewitnesses at the time suggest that the three varieties were considered identical.These data make it clear that native French speakers adjusted their speech to the learners, and that the learners did not have a desire to acquire the full language, but also that they did not have the opportunity to do so. A pidginized version of French was a compromise between the two groups.
 
I will list all known simplified French utterances by French speakers written down by contemporary observers of the language ecology of the Lesser Antilles, uttered by French speakers in addressing non-French speakers as they were documented. The structures of these utterances will be compared with those of non-native speakers in the same environment in the same period. This study thus contributes to the theory of the genesis of pidgins, and it sheds light on the contribution of foreigner talk to pidginization. This is a topic taken up by Winford in his textbook on language contact (2003).

\section{Background} \label{sec:bakker:2}
At the Mona Conference on Pidgin and Creole languages in 1968 in Jamaica, Charles Ferguson presented a paper in which he linked the simplified speech spoken to babies and to foreigners, which was published as \citet{Ferguson1971}. Ferguson named the phenomenon of people adjusting and simplifying their own speech to conversation partners who are not speakers, Foreigner Talk. This article started a discussion about the role of what came to be known as Foreigner Talk and the role it could have played in the development of pidgins. Ferguson continued to play a major role in the discussion in the following decade. In connection with pidginization, \citet{FergusonDeBose1977} distinguished between Foreigner Talk (FT), a speech form by native speakers who simplify their own speech in order to be more effective in communication, and \textit{Broken Language} (BL), the non-native’s version of another language. As for the latter, it should be made clear that the non-natives did not have the language of the others as a target, only successful communication; see \citet{Baker1990} for an eloquent rebuttal of the idea of a target language in pidgin and creole studies.  \citet[105]{Whinnom1971} had already written earlier with regards to pidginization that “the target language is removed from consideration”. These two types of reduced languages, FT and BL, being similar in form and content, would converge in the process of pidginization, according to Ferguson and DeBose, and thus result in a pidgin. A pidgin is a more or less stable form of speech, developed by groups who had no language in common. A pidgin is always simplified in comparison with its lexifier or, in rare cases, lexifiers.

This paper sheds light on the influence of FT on pidgins, based on early data covering French FT and approximations of French by Amerindians and Africans, which the observers at the time considered a “jargon” or a “baragouin”, two terms synonymous with a pidgin in this context. The quotes presented below confirm the role of French speakers in the pidginization process through their FT and their imitation of the BL of the nonnative speakers.

The idea that a pidgin is not a missed target of language learners, but a common creation by native speakers and their non-native communication partners, through mutual adjustments, is at least as old as \citet[67]{Coelho1881} and \citet[7ff]{Schuchardt1888}; see also \citet[443]{Schuchardt1909}; or p. 69 in Gilbert’s 1981 English translation. This idea was also mentioned by \citet[216]{Jespersen1922} and is often attributed to \citet[472]{Bloomfield1933}. These authors referred to a variety of cases: the Mediterranean Lingua Franca, Bislama, Chinese Pidgin English, Portuguese creoles/pidgins in the Portuguese colonial empire, Pidgin and Creole French in Vietnam and Mauritius, etc., attesting to the attractiveness of the idea and its empirical grounding.

In the revival of creole studies from the 1960s, the role of FT in pidginization was picked up again by \citet{Naro1978} and others also adopted the idea in their studies of specific pidgins, or pidgins in general. Clements \citeyearpar{Clements1992} discussed the speech adjustments by Portuguese priests to creole Portuguese speakers in Korlai, India, in the framework of the origin of Pidgin Portuguese, solidly embedded in work on FT in second language acquisition in general. After this, there were no publications of significance for almost fifteen years.

The idea of mutual adjustment by native and non-native speakers as a factor in the genesis of pidgins is intuitively appealing, and this theoretical claim was backed up by the empirical study of cases in different continents. Somehow, the idea was hardly discussed by creolists/pidginists after the 1970s, even though it continued to play a significant role in the study of second language acquisition (see \cite{DelaRosa2016} for an overview).
It was in the mid-2010s that the role of Foreigner Talk was taken up again as a factor relevant for pidginization. \citet{Versteegh2014} was the first in his study of past-tense reference in pidgins, in which he took the possible role of what he calls “foreigner-directed speech” into account. \citet{Avram2017} makes an excellent comparison of grammatical features of Gulf Pidgin Arabic, linking them with examples from Foreigner Talk (see below). His data leave no doubt on the influence of FT on Arabic pidgins of the Gulf states, but he also emphasizes that FT is not the only factor in the genesis of pidgins. More details are given in \citet{Avram2018}, in which he compares structural and lexical features of four Arabic-lexifier pidgins (Pidgin Madame spoken by Sri Lankan maids in Lebanon, Jordanian Pidgin Arabic, Romanian Pidgin Arabic, and Gulf Pidgin Arabic, all spoken by foreign workers and native Arabic speakers in the Middle East). He indicates that FT played a role in the genesis of these pidgins.
These data on foreigner talk and its relevance for pidgin genesis contrast remarkably with the theory of pidginization advocated by \citet{Mufwene2015,Mufwene2020}, which seems to be virtually uncontested. In his view, pidgins cannot be the precursors of creole languages, because they developed later than creoles, and pidgins and creoles are in complementary distribution. Creole languages, according to him, develop in plantation settings and through a gradual process of adjustments to approximations by successive generations of non-native speakers. Pidgins, in contrast, according to him, develop in trade situations, where the presence of interpreters prevented the development of reduced forms of speech, except in later periods when interpreters approximated earlier attempts. Thus, both pidgins and creoles developed gradually: “creoles and pidgins have evolved in ways similar to the Romance languages, by gradual divergence away from their lexifiers” \citep[302]{Mufwene2020}. One of his claims is:

\begin{quote}
Although speakers of pidgins as L2 varieties must have started with interlanguages cum individual transitional varieties toward closer approximations of the target language, there is no historical documentation of communal ‘interlanguages’ as incipient pidgins (…) nor of ‘jargons,’
 (…) \citep[301]{Mufwene2020}.
\end{quote}

All of these statements seem far removed from reality.
In this paper, I investigate all the sources of French FT in the Lesser Antilles of the 17\textsuperscript{th} century, and compare them with documented utterances of non-native speakers from a structural perspective. In the conclusions, I will relate them to work on pidgins by \citet{Avram2017} and \citet{Parkvall2017}, and contrast that with the claims by Mufwene (2020 and elsewhere).
Pidgins and creoles are distinct language types, not only from a social point of view, but also from a structural point of view (pace Mufwene). Pidgins are, roughly, second languages for all, and severely reduced in lexicon and structure compared to their source languages. Most conspicuously, verbal and nominal morphology is almost completely eliminated, and generally even more in creoles, than in pidgins \citep{Bakker2003}. Creoles are first languages, either native tongues or main languages for a community. The fact that there are also pidgincreoles \citep{Bakker2008} indicates that pidgins can become creoles: pidgincreoles share several social characteristics with pidgins, but most structural properties with creoles. A language like Tok Pisin is an example of a pidgincreole: its name refers to its pidgin past, but it has become a creole. \citet[138]{Mufwene2015} does not consider these expanded pidgins a separate category: 

\begin{quote}
Confusing expanded pidgins such as Tok Pisin with creoles (\citealt{Thomason2001}; cf \citealt{Siegel2008}) depends conceptually on whether one subscribes to the position that creoles are nativized pidgins. Discussions that lump them together are informative in showing the extent to which different evolutionary trajectories can nonetheless produce similar structural outcomes.
\end{quote}

In this quote, Mufwene does recognize a structural typological profile for creoles, a position which he otherwise does not subscribe to. The wording “subscribes to” suggest that this is a point of view, whereas it is in reality a historical fact for these expanded pidgins.
I will use data from historical sources of the pre-creolization stage from the Lesser Antilles. These predate the structural transition from pidgin to creole. This is visible in the absence of innovated articles, innovated plural markers, innovated TMA systems, all universally present in creoles, including Lesser Antilles Creole. For more detailed data and argumentation, see \citet{Bakker2022}.
I will study pidgin material and the contribution of French speakers to the structure of the pidgin. The Lesser Antilles constitute, by the way, one of the areas (the Caribbean) where simplification or pidginization according to Mufwene never happened. Again, the facts contradict his point of view. I focus here on an aspect that has not received much attention, that is the use of reduced language by native speakers.

\section{French speakers using reduced French} \label{sec:bakker:3}
We have gathered pidgin-like data from several dozens of sources from the Lesser Antilles from the 17\textsuperscript{th} century (see \citealt{Bakker2022}), comprising more than 50 quotes and brief texts from different islands, all preceding creolization, i.e. before conspicuous creole grammatical traits were introduced, such as preverbal TMA, innovated plurals, new definite and indefinite articles (or something resembling them) and the like. Some of these simplified phrases were uttered by native speakers of French, and written down by native speakers of French, from their own mouth or from the mouth of their fellow countrymen. We will analyze them by comparing them with utterances by non-native speakers of French from the same region and period.
I do not suggest that the French, by simplifying their native language, were solely responsible for the structures and lexical peculiarities of the creole or pidgin. The reduced structures are much more likely the result of mutual accommodation from all sides: simplified forms were strengthened in the contacts between the groups because of their optimal functionality in this situation. In earlier days, students of creoles have suggested that properties of creoles could be explained by an assumption that Europeans spoke in a simplified way to non-Europeans, and that this simplified speech was learned by the first generation of creole speakers. This theory, sometimes called “baby-talk theory” (\citealt{Velupillai2015}; \citealt[158--159]{Stein2017}), is no longer in vogue among creolists. One author invoking this was \citet[472--473]{Bloomfield1933}, with English as the exemplified lexifier:

\begin{quote}
    This ‘baby-talk’ is the masters’ imitation of the subjects’ incorrect speech. [...] The basis is the foreigner’s desperate attempt at English. Then comes the English-speaker’s contemptuous imitation of this, which he tries in the hope of making himself understood. [...] The third layer of alteration is due to the foreigner’s imperfect reproduction of the English-speaker’s simplified talk, and will differ according to the phonetic and grammatical habit of the foreigner’s language.
\end{quote}

Even though adjustment by native speakers of their own speech in contact situations is no longer as popular as an explanation for language data in creole studies as in the 1970s, there is no doubt that it did play a role in pidginization. It was certainly the case in the Lesser Antilles. Metalinguistic comments on reduced forms of French as used by Amerindians, Africans and Europeans alike can be found in the sources. For a fuller discussion of these metalinguistic comments, see e.g.( \cite{Goodman1964,Hazaël-Massieux1996,Jennings1998,Prudent1999,Hazaël-Massieux2008} and \cite{Thibault2018}), some of which will be discussed below.

There are not only metalinguistic remarks, there is also a remarkable quantity of linguistic material. There are pidginized French data and French foreigner talk data from 1619 to 1696, where European examples cover the period from 1619 to 1682, Amerindians from 1635 to 1696 and Africans from the 1640s to 1698. These dates indicate the period of documentation of simplified but non-creolized forms of French. A source from 1671 from Martinique is the first that shows traits associated with creoles, such as preverbal TMA, indefinite articles, definite articles from demonstratives that are not found in the pidgin stage, or in pidgin varieties around the world \citep[11]{Baker1995}.
I will now present quotes from French speakers who simplify their French in communication with non-French speakers, in other words, French foreigner talk, and compare the linguistic phenomena in these sentences with quotes from non-native speakers from the Lesser Antilles.

\subsection{Martinique 1619 (anonymous filibuster)} \label{sec:bakker:3.1}
The first phrase in contact French of the Lesser Antilles can be found in an anonymous manuscript about the travels of captain Fleury to the Caribbean in 1619--1620, preceding the French colonization efforts in the Caribbean by a number of years. Fleury and his men spent a considerable amount of time on the island of Martinique. French colonization in the region first started on the island of Saint Kitts in the mid 1620s. The island of Martinique was settled later, from Saint Kitts in 1635, and shortly after 1658 the indigenous population was killed or deported. Enslaved Africans were imported as a labor force from 1635 in Saint Kitts and in the following decades the French settled the Lesser Antilles islands with slaves.
A phrase in simplified French (foreigner talk) is quoted (translation maintained) in the account by the anonymous filibuster who authored the manuscript.

\ea \label{ex: bakker:1}
\gll France bon, France bon \\
     France/French good France/French good \\
\glt `We are French, we (France/the French) are good'  (Anonyme de Carpentras; Grunberg et al. 2013)\\
    (French: \textit{La France est bonne} `France is good' )\\
    (or \textit{Les français sont bons} `The French are good' )
\z


The phrase is used by French soldiers when approaching Indians who feel threatened. The document is anonymous, and the author describes the experiences of the group of filibusters with ample information about the customs of the local Amerindians on Martinique and elsewhere during 1619--1620. The phrase quoted above is the only one in French. Further communication took place in the language(s) of the Amerindians; around 200 words and phrases are quoted in Island Carib.
The name of the country in example (\ref{ex: bakker:1}) probably refers to the inhabitants, i.e. the French. In later sources as well, the term ‘France’ is used with reference to the people rather than the country in the pidgin. Thus, the speakers hope that their interlocutors understand that they express their peaceful intentions.
In this very simple phrase, we can observe the following deviations from French, all of them changes vis-à-vis the original, or intended, French text.

\ea%2 
\begin{itemize}
    \item[A.] Replacement of the population by the name of the country;
    \item[B.] The copula, obligatory in French, is omitted;
    \item[C.] The adjectival predicate meaning ‘good’ is not inflected for number or for gender, which is obligatory in French (as ‘France’ has the feminine gender, one would have expected \textit{bonne} rather than \textit{bon}; or /bɔn/ rather than /b\~{ɔ}/);
    \item[D.] An article is lacking, which would have been \textit{la}, \textit{les} or \textit{des}, depending on the intended meaning;
    \item[E.] Probably the French speakers also assumed that pronouns would not be understood, and they used a full noun `France'  instead of a pronoun \textit{nous} “we”.
\end{itemize}
\z


We will now survey which of these phenomena are found in contemporary quotes from the 17\textsuperscript{th} century by Amerindians and Blacks. We do find the same phenomena in these three utterances by Amerindians some fifteen years later. I am emphatically not suggesting that there was a historical transmission from one manuscript or printed source to the next. I do suggest that the same strategies were used to obtain the same goals of communication. We find the same phenomena in these and other quotes from Amerindians:

\ea%3
\label{ex:bakker:3}
\citet{Tronchoy1709}: 162; Amerindian speaker, Guadeloupe 1635\\
 France non point fache \\
        France/French \textsc{neg} \textsc{neg} angry
\glt `The French are not angry'
    (French: `les Français ne sont pas fâchés' )
\z


\ea%4 
\citet[109]{Bouton1640}; Amerindian speaker, Martinique ca. 1635
\label{ex:bakker:4}\\
\gll Non ça [ca] bon pour France, bon pour Caraibe... \\
    \textsc{NEG} that good for France good for Carib\\
\glt `This [i.e. walking around naked] is not good for France/the French, good for the Caribs'
(French: “C'est n'est pas bon pour la France/les Français, c’est bon pour les Caraïbes”)
\z

\ea %5
    \citet{Tronchoy1709}: 165; Amerindian speake, Guadeloupe 1635\\
\label{ex:bakker:5}
    \gll Ô Jaques France mouche fache l’y matte Karaibes\\
O Jacques French/France very angry 3\textsc{sg} kill Carib(s)\\
    \glt `Jacques, the French are very angry, they have killed the Caribs' \\
(In Tronchoy’s French: \textit{Jacque les François sont extrêmement fâchez, ils ont tué les Sauvages})
\z



We find the following properties of the speech of the Frenchmen also in the speech of Amerindians. Note that I use capitals (here A–E) to refer to the same phenomenon throughout this section.

\ea %6
\begin{itemize}
\label{ex:bakker:6}
    \item[A.] The name of a country (France) is used for a population group (French) in \REF{ex:bakker:3}, \REF{ex:bakker:4} and \REF{ex:bakker:5}.
    \item[B.] Lacking copula: Examples \REF{ex:bakker:3} \REF{ex:bakker:4} and \REF{ex:bakker:5} likewise display a lack of copula.
    \item[C.] Agreement: The spelling of the word for `angry'
in \REF{ex:bakker:3} indicates that the missionaries who wrote down the phrases did not consider the adjectival predicates as reflecting a plural (`the French') or feminine meaning `La France' , otherwise it would have been spelled something like \textit{fachée} (feminine) or \textit{fachés} (masculine plural). The plural \textit{-s} is written in French, but not pronounced.
\item[D.] Articles: in all three Amerindian utterances, articles are lacking.
\item[E.] Personal pronouns: In example \REF{ex:bakker:5} there is probably a third person pronoun \textit{l’y}, representing the emphatic third person pronoun \textit{lui}. There is no evidence for pronoun avoidance in this sentence.
\end{itemize}
\z

\subsection{Saint Kitts, 1650s (Pelleprat)} \label{sec:bakker:3.2}
 

French and British colonization of the Caribbean started in the mid-1620s on the island of Saint Kitts, then called Saint Christopher\slash Saint Christophe \citep{Parkvall1995, Jennings1995}. Therefore the island is very important for the settlement history of the Caribbean \citep{BakerBruyn1999}, and of its linguistic landscape. Pidgins and creole languages may have diffused from there, together with populations moving to other islands that were settled later. The native population was killed off or deported after their revolt in 1638.

Pierre Pelleprat was a French Jesuit missionary who worked both in French Guiana and on the Caribbean Islands in the 1650s. His Carib language material appears sometimes pidginized, but not always. In addition, he quotes a few phrases in a reduced form of French, uttered by Blacks. He also indicates that not only the Blacks (perhaps also Amerindians) use this `baragouin', but the missionaries use it as well in addressing non-Europeans. The following utterances by a native French speaker in “broken French” are examples provided by Pelleprat in order to show how he and other missionaries speak to other populations, notably the enslaved people. Here are his two exemplifying sentences of missionaries addressing Black people on Saint Kitts.

 
\ea \label{ex:bakker:7}
\citet{Pelleprat1655}: 53\\
\gll moy prier Dieu, moy aller à l’Eglise, moy point manger\\
    1\textsc{sg:emph} pray God 1\textsc{sg:emph} go to the-church 1\textsc{sg:emph} \textsc{neg} eat\\
\glt `I have prayed to God, I have gone to the church, I have not eaten'
 (French: `J’ai prié à Dieu, je suis allé à l’Église, je n’ai pas mangé' )
\ex \label{ex:bakker:8}
 \citealt{Pelleprat1655}: 53\\
\gll demain moy manger, hier moy prier Dieu\\
tomorrow 1\textsc{sg:emph} eat, yesterday 1\textsc{sg:emph} pray God\\
\glt `I will eat tomorrow, yesterday I was praying/prayed to God'
\glt (French: `Je mangerai demain, hier je priais Dieu')
\z 

With regard to these phrases, we can make the following observations that coincide with the ones from Martinique (1619) discussed above (1–4).

\begin{itemize}
\item [D.] Articles: Both inherited and newly formed articles are usually absent in pidgins. There may be a separate article in the word for `church'  in \REF{ex:bakker:7}, but it could be part of the word as well.
\item [E.] Personal pronouns: the use of emphatic pronouns rather than the otherwise obligatory clitics of French: \textit{moy} (modern spelling \textit{moi}) rather than \textit{je} `I' . These emphatic pronouns are used in isolation and in emphatic contexts in French. This can be seen in both \REF{ex:bakker:7} and \REF{ex:bakker:8}.
In addition, we can add a number of new observations on deviations from French in these lines.
\item [F.] Infinitives: infinitives are used rather than inflected forms of verbs. That is the case in most pidgins. It has to be said that, in most cases in French, the spoken form of the infinitive is identical to the past participle. That the forms are infinitives, can be seen in French irregular forms (e.g. \textit{savoir }`to know'  rather than \textit{su} `known') and a few generalizations of infinitival endings.
\item [G.] Prepositions: inconsistent use of prepositions, which are sometimes given and sometimes omitted. A preposition can be seen in \REF{ex:bakker:7}, as it would be used in French. It is omitted in \REF{ex:bakker:8} where it would be expected with the noun \textit{Dieu}.
\item [H.] Negation: negation is indicated in \REF{ex:bakker:7} with a preverbal particle \textit{point} (modern French \textit{pas}) rather than with a circumverbal locution \textit{ne…. pas/point}. Thus, it differs in two ways from French at the time: only the emphatic part is preserved, and that part is preverbal in the missionary’s utterance rather than postverbal as required in French.
\item [I.] There is no indication of tense in the verb; instead, time reference is optionally indicated with an adverb to indicate the future. We can see that in \REF{ex:bakker:8} with \textit{demain} `tomorrow'  and \textit{hier }`yesterday', and \REF{ex:bakker:7} where there is no indication of time. This is a recurrent structure in pidgins \citep{ParkvallBakker2013, Parkvall2017}, but not typically found in creoles.
\end{itemize}
Do we have examples, or counterexamples, of these phenomena in Amerindian and African speech in the same region and period? Here are some sentences where we can observe the same phenomena.

\ea\label{ex:bakker:9}(Amerindian; Guadeloupe, 1650s)\\
\gll Pere moy non plus pour le Mabohia\\
Father 1\textsc{sg neg} more for the Maboya\\
\glt `I am no longer for [=a believer in] the Maboya [Carib deity]' (French: ‘Père, je ne suis plus pour le Maboya’)
\ex\label{ex:bakker:10}
(Amerindian; Martinique 1694; \citealt[29]{Labat1724_tome1})\\
\gll Bon jour compere, toi tenir taffia \\
Good day friend 2\textsc{sg} have tafia\\
\glt `Good day, friend, do you have sugarcane liquor' (French: ‘Bonjour, mon ami, avez-vous de la liqueur de canne à sucre’)
\ex\label{ex:bakker:11}
 (Black woman; \citet{Chatillon1984}: 104; St. Kitts 1682)\\
\gll Le Bon Dieu apprendre \`{a} moi cela\\
\textsc{det} good God teach \textsc{to} 1\textsc{sg} that\\
\glt `God has taught me that' (French: ‘Dieu m’a appris cela’)
\ex\label{ex:bakker:12}
 (Blacks; \citet{Chatillon1984}: 134; St. Kitts 1682)\\
\gll moi prier Dieu demain\\
1\textsc{sg}pray God tomorrow\\
\glt `I want to pray to God tomorrow' (French: `Je veux prier Dieu demain')
\ex\label{ex:bakker:13}
(Blacks; \citet[134]{Chatillon1984}; St. Kitts 1682)\\
\gll moi manger hier, toi donner manger à moi \\
1\textsc{sg} eat yesterday, 2\textsc{sg} give food to 1\textsc{sg}\\
\glt `I ate yesterday, you gave me something to eat' (French: ‘J'ai mangé hier, tu m'as donné à manger’)\\
\z

These sentences lead to the following observations:


\begin{itemize}
\item [B.] Copula: the copula is lacking in \REF{ex:bakker:9};
\item [C.] Agreement: agreement, including clitics, is lacking in the verbs in \REF{ex:bakker:10}, \REF{ex:bakker:11}, \REF{ex:bakker:12} and \REF{ex:bakker:13};
\item [D.] Articles. We find both articles where they belong, as in \REF{ex:bakker:9}, and lacking articles where they should have been present in French, as in \REF{ex:bakker:10};
\item [E.] Personal pronouns: only emphatic pronouns\textit{ toi} and \textit{moi} are used in \REF{ex:bakker:9} through \REF{ex:bakker:13}.
\item [F.] Infinitives: used in \REF{ex:bakker:9} through \REF{ex:bakker:13}; no inflected verbs;
\item [G.] Prepositions: two different ones (\textit{à, pour}) used consistently in \REF{ex:bakker:9}, \REF{ex:bakker:11} and \REF{ex:bakker:13};
\item [H.] Negation: is preverbal, and it combines \textit{non} `no'  and \textit{plus }`more'  in \REF{ex:bakker:9};
\item [I.] Time: there is no tense on the verb in \REF{ex:bakker:12}, but time is indicated by \textit{hier} `yesterday'  and \textit{demain} `tomorrow'  in \REF{ex:bakker:13}.
\end{itemize}
Thus, again we find striking similarities between foreigner talk by the French and non-native speech.

\subsection{Guadeloupe, 1650s (Chevillard)} \label{sec:bakker:3.3}

 
The longest text in French adjusted to non-native speakers by native speakers is the text below as quoted in \citet[145--146]{Chevillard1659}. He praises the openness and flexibility of the enslaved Black people. He says that this is ordinarily the way they are taught,\footnote{“On les enseigne pour l’ordinaire selon la matiere, en cette maniere.”  \citep[145]{Chevillard1659}} and then this text follows:

\ea \label{ex:bakker:14}
\ea \label{ex:bakker:14a}
\gll Toy sçavoir qu'il y a \textsc{vn} \textsc{diev}\\
    2\textsc{sg} know that.there.is  one God\\

\ex \label{ex:bakker:14b}
\gll luy grand Capitou\footnotemark{}\\
3\textsc{sg} big Captain\\
\footnotetext{\textit{capitou} is also the word which the Amerindians used for their chiefs.}

\ex \label{ex:bakker:14c}
\gll luy sçavoir tout faire sans autre pour l'ayder:\\
3\textsc{sg} know all make without other for help.him\\

\ex \label{ex:bakker:14d}
\gll luy donner à tous patates:\\
3\textsc{sg} give \textsc{to} all potatoes (bread)\\

\ex \label{ex:bakker:14e}
\gll luy mouche manigat pour tout faire,\\
3\textsc{sg} much(\textsc{sp}) skilfull for all do/make\\

\ex \label{ex:bakker:14f}
\gll non point autre comme luy.\\
\textsc{neg neg} other like 3\textsc{sg}\\

\ex \label{ex:bakker:14g}
\gll Vouloir faire maison, non point faire comme homme,\\
Want make house, \textsc{neg} \textsc{neg} make like man/human\\

\ex \label{ex:bakker:14h}
\gll car toy aller chercher hache pour bois,\\
because 2\textsc{sg} go search axe for wood\\

\ex \label{ex:bakker:14i}
\gll puis coupper roseaux, prendre mahoc\footnotemark{} \& lienes, \& ainsi pequino faire case.\\
Then cut reed take rope and creepers and thus people make house\\
\footnotetext{Mahot is a tree of which the bark can be stripped and used as ropes to attach coverings to the roof.}

\ex \label{ex:bakker:14j}
\gll Or Dieu mouche manigat, luy dire en son esprit,\\
but God much(\textsc{sp}) skilfull 3\textsc{sg} say in his spirit\\

\ex \label{ex:bakker:14k}
\gll moy vouloir monde luy preste miré monde:\\
1\textsc{sg} want world 3\textsc{sg} ready look world\\

\ex \label{ex:bakker:14l}
\gll Luy dire en son esprit,\\
3\textsc{sg} say in 3\textsc{sg.m.poss.} spirit\\

\ex \label{ex:bakker:14m}
\gll Moy vouloir homme luy preste mire homme.\\
1\textsc{sg} want man 3\textsc{sg} ready look man\\

\ex \label{ex:bakker:14n}
\gll Enfin luy enuoye meschant en bas en enfer,\\
Finally 3\textsc{sg} send bad \textsc{in} \textsc{down} \textsc{in} hell\\

\ex \label{ex:bakker:14o}
\gll au feu auec Mabohia \& autres Sauuages\\
\textsc{to} fire with Devil and other savages\\

\ex \label{ex:bakker:14p}
\gll qui n' ont point vouloir viure en bons Chrestiens.\\
\textsc{rel} \textsc{neg} have \textsc{neg}  want live in good Christian\\

\ex \label{ex:bakker:14q}
\gll Mais pour bon Chrestien,\\
But all good Christian\\

\ex \label{ex:bakker:14r}
\gll luy bon pour mettre en son Paradis\\
3\textsc{sg} good for put IN his paradise\\

\ex \label{ex:bakker:14s}
\gll où se trouve tout contentement,\\
\textsc{where.rel} \textsc{refl} find all    happiness\\

\ex \label{ex:bakker:14t}
\gll nul mal, nul trauail, \& nulle seruitude ou esclavage,\\
zero pain zero work and zero serfdom or slavery\\

\ex \label{ex:bakker:14u}
\gll mais une entière joye et parfaite liberté.\\
But \textsc{indef} entire glad and perfect freedom\\


\glt `You know that there is a God. He is the big boss. He knew how to create everything without others to help him. He gives food (potatoes, bread) to everybody. He is very skillful in making everything, there is nobody else like him. If he wants to make a house, he does not make it like people, for you have to go and look for an axe for the wood, then cut reeds, take ropes and creepers and thus make a dwelling. But God is very skillful, he says in his spirit: I want the world, he is ready to look at the world. He said in his spirit. I want people ready to look at people. In the end he sends bad people down into hell, to the fire with the Maboya (Island Carib God/Devil) and the other Indians who have not wanted to live like good Christians. But for a good Christian, he is so good to put him in his paradise where one finds complete happiness. No pain, no work and no serfdom or slavery, but full pleasure and perfect freedom.'
\z
\z


We find the same phenomena as previously observed: the absence of a copula (B), the lack of articles (D), everywhere except in the last sentence, the use of emphatic pronouns (E), the use of infinitives (F) and the lack of tense marking (I). In addition, we find a number of non-French words that are known from other sources of Lesser Antilles baragouin (see \citealt{Jansen2012} and \citealt{Bakker2022}), such as \textit{capitou} \REF{ex:bakker:14b}, \textit{mouche }(\ref{ex:bakker:14}e,j), \textit{manigat} (\ref{ex:bakker:14}e,j), \textit{pequino} (if it means `child') \REF{ex:bakker:14i}, \textit{mire} (\ref{ex:bakker:14}e, k, m), \textit{Mabohia} \REF{ex:bakker:14o}.


On the other hand, especially towards the end, the text becomes more French-like (14s-14u) and less pidgin-like with, for instance, adjectival agreement in \REF{ex:bakker:14u}, a reflexive \textit{se}, and the use of words with several derivational suffixes (\textit{{}-ment, -age, -té}). In addition, we find unexpected preverbal clitics \REF{ex:bakker:14c}, an indefinite article in \REF{ex:bakker:14u}, “correctly” used prepositions (14f, 14h, 14j, 14l, 14n-r), and \textit{non… point} negation in \REF{ex:bakker:14p}, which point to more orientation towards French. Still, the overall impression is of a starkly pidginized text rather than French or Creole French.

\subsection{Saint Kitts, 1682 (Mongin)}\label{sec:bakker:3.4}



\ea \label{ex:bakker:15}
Jesuit missionary Jean Mongin stayed on the island of Saint Kitts in the 1680s. In his letters that have been preserved, he writes that he spoke to the Blacks “in their own jargon”.
 He gives the same example twice, in slightly different wording.
\ea\label{ex:bakker:15a}
\gll  toi de même que nègres anglais, sans bapteme, sans eglise, sans sepulture\\
2\textsc{sg} of same as Blacks English without baptism without church without burial\\\
\glt `You are just like the English blacks, without baptism without church without burial'

(\citealt{Chatillon1984}: 76; St. Kitts 1682)

\ex\label{ex:bakker:15b}
\gll toi seras traité de même que nègre anglais, sans baptême, sans église, sans sépulture\\
2\textsc{sg} be. \textsc{fut.}2\textsc{sg} treat-\textsc{partic} same as Black English without baptism without church without burial\\
\glt `You will be treated just like the English blacks, without baptism without church without burial'
 (\citealt{Chatillon1984}: 135; St. Kitts 1682)
\z
\z

\noindent We can observe the following traits that deviate from French:
\begin{itemize}
\item [B.] Copula: the copula is lacking in \REF{ex:bakker:15a} but not in \REF{ex:bakker:15b};
\item[C.] Agreement: apparent agreement is present in \textit{seras} and \textit{traité} in \REF{ex:bakker:15b} but both verbs are omitted in \REF{ex:bakker:15a}; a non-infinitive form like \textit{seras} is normally not used in pidgins;
\item[D.] Articles. We find no articles where they would be expected, before \textit{nègres/nègre} in \REF{ex:bakker:15}.
\item [E.] Personal pronouns: only emphatic pronoun \textit{toi} is used in \REF{ex:bakker:15}.
\item[F.] Infinitives: not used in \REF{ex:bakker:15}. There is one exception, in \REF{ex:bakker:15b} where two inflected (non-infinitival) verbs are found: \textit{seras} and \textit{traité}, the latter written as a past participle.
\item[G.] Prepositions: \textit{sans} is used consistently in \REF{ex:bakker:15}, perhaps also \textit{que} in \REF{ex:bakker:15a}
\item[I.] Tense marking as in pidgins: there is no tense on the verb, but time is indicated by \textit{hier} `yesterday'  and \textit{demain} `tomorrow'.
\end{itemize}


The missionary Mongin also reports on a conversation he had with a Black Christian woman who used self-flegallation as an expression of her faith. Mongin spoke to her as follows:

\ea
(\citealt{Chatillon1984}: 104; St. Kitts 1682)\\ 
\gll  Mais qui celui la apprendre \`{a} toi cela?\\
 But who the.one there teach \textsc{to} 2\text{sg} that\\
\glt  `But who is the one who taught you that?'
 (French: `Mais qui est celui qui vous a appris cela?' )
\z

Here again we can observe the use of infinitives (F) and emphatic pronouns (E), and the presence of a preposition \textit{à }(G)\textit{.}

\section{Linguistic observations} \label{sec:bakker:4}
When we compare the modified forms of French used by French speakers and the quotes in the form of approximations of French from the mouth of non-French speakers, we can observe close parallels. Eight out of nine of the deviations from standard French are found in the speech of Amerindians and/or Africans as well as Europeans. This is a clear indication that the speech of the three groups in interethnic contacts was quite similar, and that mutual adjustments to each other’s speech took place. In the next sections, we will consider metalinguistic observations, and these appear to corroborate the linguistic materials. But first let me present some observations on shared lexicon.

The reductions we have observed (elimination of verbal morphology, elimination of the distinction between clitics and emphatic pronouns, dropping of selected prepositions) as well as the uniform solutions in the form of the use of emphatic personal pronouns and infinitives point to a common form of speech of the two groups. However, if these were “universal” processes of simplification, it would still not prove that a common language had developed in the interaction between Indians, Blacks and Europeans. We would also need unexpected lexical items shared across ethnic boundaries and grammatical idiosyncrasies. These are indeed present in the material.

There are for instance idiosyncrasies like the Spanish word \textit{mucho} `much'  and the Amerindian or dialectal French word \textit{manigat} which are found in simplified French utterances by Indigenous people, Blacks and Europeans alike. Also the Amerindian word \textit{Maboya} (see \citealt{Bakker2022}) is used by Blacks and Indians as well as the French. Obviously, these are quoted by Europeans, which suggests that they used them as well, and that they were perhaps part of the stereotype of non-native French. We know that these terms became part of the “French of the Islands” \citep{Jansen2012}. Thus, the presence of shared lexicon corroborates that reduced French was shared by French, Amerindians and Blacks.
\section{Metalinguistic observations: reduced French}
As we have seen above, native speakers of French used reduced forms of French in their communication with the Indigenous people and people of African descent. Thus, the Europeans played an obvious role in the reduction, and they would have addressed Blacks and Indians in the same way.
Two kinds of motivations for these adjustments are given by the missionaries: in that way, they are better understood (especially important when spreading God’s word), and on the other hand, the French speakers say that they adjust themselves to the way that the Amerindians and Black people speak. It thus seems to be a case of mutual adjustment. This is confirmed in a number of contemporary metalinguistic comments.
Pelleprat made the following metalinguistic remarks on the language, clearly indicating that the missionaries adjust to the speech of accommodating speakers:

\begin{quote}
\textit{We accommodate ourselves nevertheless to the way they [the Black slaves] }\textit{speak}, which is normally by means of the infinitive of the verb. Like, for example, \textit{me pray God, me go to church, me no eat }if they wish to say 'I have prayed to God', ‘I have gone to the church’, ‘I have not eaten’. And adding a word that indicates the time to come, or the past, they say 'tomorrow me eat’, yesterday me pray God', which means I will eat tomorrow, yesterday I prayed to God (\citealt{Pelleprat1655}: 53; my emphasis and translation).\footnote{“{Nous
   nous accommodons cependant à leur façon [des esclaves noirs] de parler, qui est ordinairement par l’infinitif du verbe; comme par exemple, }{\textit{moy prier Dieu, moy aller à l’Eglise, moy point manger, }}{pour dire }{\textit{i’ay prié Dieu, ie suis allé à l’Eglise, ie n’ay point mangé}}{;}{\textit{ }}{Et y adioustant vn mot qui marque le temps à venir, ou le passé, ils disent }{\textit{demain moy manger, hier moy prier Dieu, \& }}{cela signifie le mangeray demain, hier ie priay Dieu” (\citealt{Pelleprat1655}: 53).}}
\end{quote}
 

Mongin remarked almost the same: 
\begin{quote}
The Blacks have learned within a short time a certain \textit{French jargon that the missionaries know and which they use to instruct}, which is through the infinitive of the verb, without ever conjugating it, by adding a few words which indicate the time and the person about whom we are speaking. For example, if they want to say: “I want to pray to God tomorrow”, they will say “me pray God tomorrow”, “me eat yesterday”, “You give food to me”, and like this everywhere. This jargon is very easy to teach to the Blacks and to the missionaries also to instruct them, and thus they give it to get an understanding for all things” (my translation and emphasis; Mongin in \citealt[134--135]{Chatillon1984}).\footnote{“Les nègres ont appris en peu de temps un certain jargon français que les missionnaires savent et avec lequel ils les instruisent, qui est par l'infititif [SIC] du verbe, sans jamais le conjuguer, en y ajoutant quelques mots qui font connaître le temps et la personne de qui l'on parle. Par exemple s'ils veulent dire: Je veux prier Dieu demain, ils diront moi prier Dieu demain, moi manger hier, toi donner manger à moi, et ainsi en toutes choses. Ce jargon est fort aisé à apprendre aux nègres et aux missionnaires aussi pour les instruire, et ainsi ils le donnent à entendre pour toutes choses.”}

\end{quote}


The remarks that Pelleprat and Mongin make on the grammar (use of infinitives, no conjugation, no person marking, no tense except through adverbs) fit the data, and these concur with observations on pidgins in general \citep{Parkvall2017,Parkvall2020}.

As we saw, the differences between the accommodated speech as used by Europeans, Indigenous people and Blacks were minimal. In the linguistic material, the same modifications recur across the different groups. The fact that they were all written down by French speakers may have contributed to the homogeneity, but these are all the data that are available. 

The metalinguistic remarks by the missionaries confirm that they perceived the forms of speech used by all three groups to be the same. Du Tertre (1667: II, 510) confirmed that he considered it the same language, for instance:

\largerpage
\begin{quote}
(…) most of the young Blacks don’t know any language other than the French language, and (…) they understand nothing of the native language of their parents; except for only \textit{the pidgin [barago}\textit{ü}\textit{in], which they use on the Islands, and which we also use with the Indians}, which is a jargon composed of French, Spanish, English, and Dutch words.\footnote{“De là vient que la pluspart des petits Négres ne sçavent point d’autre langue que la langue Françoise, \& qu’ils n’entendent rien à la langue naturelle de leurs parens; excepté seulement le baragoüin, dont ils usent commuuément [sic] dans les Isles, \& don’t nous nous servons aussi avec les Sauvages, qui est un jargon composé de mots François, Espagnols, Anglois, \& Holandois”.}
 (\citealt{DuTertre1667}: II, 510; my emphasis and translation).\footnote{Observations on the nature of such languages by local observers are rarely accurate. Also in this case the representation of the lexical sources of the pidgin should be taken with a grain of salt. There is, however, a high degree of correspondence between the observers.}
\end{quote}
\clearpage

Here Du Tertre not only observes that the parents of Black children use pidgin, and the missionaries use it with the Amerindians, and that they fall with the range of the same pidginized variety of French, he also adds, just like Pelleprat and Mongin, that the French use it themselves as well in communication with the Indians. He further makes remarks about the lexical composition of the pidgin, and that appears generally correct based on the available documentation (\citealt{Bakker2022}), except that the material does not contain obvious words from English at all. His words on the pidgin and nativization sound surprisingly Bickertonian. Derek Bickerton (\citeyear{Bickerton1981,Bickerton1984}) took the influence of children to be the main factor in the appearance of a set of linguistic features he associated with an innate “bioprogram”. Observers mention that young Black people only understand and speak the \textit{baragouin} of the islands, which suggests incipient creolization. Labat (\citeyear[98]{Labat1724_tome1}) also observed that the Bozals (slaves born in Africa, and hence second language learners) spoke differently than the locally-born, when he mentions “new Negroes who spoke only a corrupt language, which I hardly understood, but to which, however, one is soon accustomed”.\footnote{“des Negres nouveaux qui ne parloient qu’un langage corrompu, que je n’entendois presque point, auquel cependant on est bien-t\^{o}t accoûtum\'e”.}

The beginning shift to French pidgin/creole by that time is confirmed by a comment through an earwitness from Québec, who wrote in 1670: “If you go to Martinique, you will have a great advantage that we have not had here [in Québec], in that one has no other language to study than the baragouin of the Blacks that you know as soon as you have heard it spoken” (my translation; \citealt{L’Incarnation1681}: 196).\footnote{“Si vous allez à la Martinique, ce vous sera un grand avantage que nous n'avons pas eu ici, de n'avoir point d'autre langue à étudier que le baragouin des Negres que l'on sçait dés qu'on la entendu parler.”}
Another quote suggests that not only Europeans provided French, or French pidgin, input to the congregation, but also that Black people transmitted it to other Blacks: “if need be, \textit{we use Blacks who understand French to teach} those of their nation the elements of our religion” (\citealt{Pelleprat1655}: 53; also in \citealt{Jennings1995}: 71; my translation and emphasis). \footnote{“Dans la nécessité nous nous servons des Nègres qui entendent le Français pour enseigner à ceux de leur nation les points de notre créance.”}
This is confirmed by another quote, following immediately the quote on simplified French (\citealt{Pelleprat1655}: 54–55): “We [missionaries] make them [Blacks] understand by this way of speaking what they are taught: And that is the method we keep in the initial stages of their instruction.”\footnote{“On leur fait comprendre par cette maniere de parler ce qu’on leur enseigne: Et c’est la methode que nous gardons au commencement de leurs instructions” (\citealt{Pelleprat1655}: 54–55).} 
Thus, the French pidgin seems to have been an important means of instruction. The missionaries were obviously aware that they did not speak French as they would speak it with fellow Frenchmen, as is clear from this observation (\citealt{Labat1724_tome2}: 87): “They know almost all, especially those of Dominica, enough bad French to make themselves understood, and to understand what they are told.”\footnote{Ils sçavent presque tous, particulierement ceux de la Dominique, assez de mauvais François pour se faire entendre, \& pour comprendre ce qu’on leur dit. (\citealt{Labat1724_tome2}: 87)} The missionaries were clearly aware of the practical advantages of the pidgin, which they generally had a high opinion of, as witnessed in this remark: “they never learn French well, and have only a pidgin, the most pleasant and natural of the world”\footnote{“Lorsqu'ils viennent un peu âgez dans le Païs, ils n'apprennent jamais bien le François, \& n'ont qu'un baragouin le plus plaisant \& le plus naturel du monde.”} (\citealt[57]{Labat1724_tome2}), referring to older Blacks born in Africa.

It is clear that the speech form that had developed was a common creation of French speakers from Europe and the Africans. The simplified speech was developed with the purpose of making communication easier, i.e. for practical reasons.
 
\section{Conclusions}
The observed similarities between the three types of reduced and simplified languages (FT and BL non-native approximations and pidgins), suggest strongly that pidgins emerge through mutual accommodations by native speakers and non-native speakers, as is the case with French in the Lesser Antilles. Both the linguistic and metalinguistic documentation point in that direction. Indirect evidence from other parts of the world, especially contact situations involving Russian \citep{Fedorova2006}, Arabic \citep{Avram2018}, Mediterranean Romance \citep{Schuchardt1909} and Portuguese \citep{Clements1992}, corroborate the genesis of pidgins involving reciprocal adjustments. Nothing in the documentation appears compatible with Mufwene’s claims on interpreters, trade and the assumed late development of pidgins. It also shows that there is documentation of pidgins preceding creoles in the Caribbean, the possibility of which was also excluded by Mufwene's allegedly historical approach (e.g. 2015), even though \citet{Baker2001} and many others showed the links between pidgin and creole properties.
In a more historical and data-oriented approach, \citet{Avram2017} observed around a dozen structural similarities between Arabic FT and Gulf Pidgin Arabic. Almost all of these are found in most pidgins in general, and most of them also in the Lesser Antilles materials. Many of them are also common in creoles, as summarized in Tables~\ref{tab:1 bakker:1a} and~\ref{tab:2 bakker:1b}.

\begin{sidewaystable} %supercool :-)
\caption{Structural comparisons between Arabic pidgins, pidgins in general and creoles in general. In this table, Avram’s number refer to paragraph numbers \citep{Avram2017}, Parkvall ’s  numbers  to section numbers, the Lesser Antilles letters to feature numbers in this paper and APiCs numbers to feature numbers.}
\label{tab:1 bakker:1a} 
\scriptsize
\begin{tabularx}{\textwidth}{QQp{2.5cm}Q}
\lsptoprule
{\citet{Avram2017} on Arabic pidgins and FT} &
{\citet{Parkvall2017} on pidgins general} &
{Lesser Antilles data} &
{Creoles in general (APiCS)}\\
\midrule
Loss of dual, numeral `two'  instead (3.1). Loss of plural markers, use of `all'  instead (3.2) &
No plural markers (10), no number marking &
No plural markers [C] &
Innovated and optional plural marking (22, 25)\\
\midrule
Loss of definite article, no alternative (3.3) &
No definite or indefinite articles (5) &
No articles [D] &
Innovated definite articles derived from demonstrative (28)\\
\midrule
Loss of gender in adjectives and demonstratives, masculine form instead (3.4)  &
No gender distinctions at all (6, 7) and reduced set of demonstratives &
No gender [C], no sufficient data on demonstratives &
No gender anywhere, except rarely in third person pronouns (13, 40)\\
\midrule
Loss of pronominal suffixes, use of independent pronouns instead (3.5, 3.6) &
Personal pronouns instead of verbal inflection (section 2) &
Emphatic pronouns [E] &
Personal pronouns derived from emphatic pronouns in lexifier (13, 59)\\
\midrule
Loss of verbal inflection, use of invariant verb forms instead (3.7) &
Not mentioned, but almost universal in pidgins ( \citealt{ParkvallBakker2013}) &
Invariant verb forms [F] &
Invariant verb forms (49, 50, 51, etc.) \\
\lspbottomrule
\end{tabularx}
\end{sidewaystable}

\begin{sidewaystable} %supercool :-)
\caption{Structural comparisons between Arabic pidgins, pidgins in general and creoles in general, cont. In this table, Avram’s number refer to paragraph numbers \citep{Avram2017}, Parkvall ’s  numbers  to section numbers, the Lesser Antilles letters to feature numbers in this paper and APiCs numbers to feature numbers.}
\label{tab:2 bakker:1b} 
\scriptsize
\begin{tabularx}{\textwidth}{QQp{2.5cm}Q}
\lsptoprule
{\citet{Avram2017} on Arabic pidgins and FT} &
{\citet{Parkvall2017} on pidgins general} &
{Lesser Antilles data} &
{Creoles in general (APiCS)}\\
\midrule
Loss of verbal inflection, use of `make'  light verb instead (3.8) &
Light verb `to make, do'  used in some pidgins, but not common (11)&
Not attested &
Close to nonexistent in creoles. (not in APiCS)\\
\midrule
Loss of verbal inflection, use of time adverbs to indicate time (3.9) &
Adverbs rather than inflection to indicate time (3) &
Adverbs to indicate time (optional) [I] &
Preverbal tense marker (43, 49, 50)\\
\midrule
Loss of verbal inflection, use of a frequent copula-like form instead (3.10) &
Usually no copulas, but regularly a verbal marker (8)&
No copula [B] &
Not a single form, but a handful, with distinct tense-mood-aspect meanings (49, 50, 51, etc.)\\
\midrule
Loss of inflection, use of a multi-purpose verbal marker instead (3.11) &
Some pidgins have a multi-purpose verbal marker, but most have not &
No multi-purpose verbal marker attested &
Not a single form, but a handful, with distinct tense-mood-aspect meanings (49, 50, 51, etc.)\\
\midrule
Loss of most preposition, use of one multi-purpose preposition instead (3.12, 3.13) &
Only few prepositions, one frequently used with broad meaning (4) &
Several prepositions [G] &
Small set of prepositions (not included; see 4)\\
\midrule
Constituent order partly lost, innovation of orders (3.14) &
Word order often not like lexifier &
Word order not always as in French (negation) [H] &
Word order almost always SVO (1)\\
%\midrule

\lspbottomrule
\end{tabularx}
\end{sidewaystable}



The fact that many of the structural properties observed in pidgins recur in creoles should be taken as a strong indication that the same processes have played a role in creolization, via the social and structural expansion of pidgins into pidgincreoles and creoles.

We have seen above that the simplified French as used by native speakers is basically indistinguishable from the simplified French as spoken by non-native speakers. We have to keep in mind that French speakers wrote down all of the quotes. This simplified French has the structural characteristics of a pidgin. 

This rare case study corroborates Winford’s view on the role of foreigner talk in pidginization and creolization \citet[279, 287, 290, 298]{winford2003book}. He remarks that “it is widely claimed that the primary input to pidgins came from foreigner talk versions of the major source or lexifier language. Thus, it might be argued that this deliberately reduced and simplified model was the source of many characteristic pidgin features such as absence of morphology and syntactic complexity” \citep[279]{winford2003book}. As for the two types of speech, he suggested that it “seems reasonable to assume that they are all subject to the same universal principles” \citep[287]{winford2003book}. Winford refers to the same principles of simplification in foreigner talk and pidginization. The empirical material from the French Caribbean indicates that he is right.

\section*{Acknowledgements}
 
The comments by the editors and the anonymous referees have contributed to a significant improvement of this paper and I am grateful for that.

\section*{Abbreviations}
\begin{tabularx}{.5\textwidth}{@{}lQ}
\textsc{tma}&  Tense Mood Aspect \\
\textsc{svo} &  Subject Verb Object\\
\end{tabularx}\begin{tabularx}{.5\textwidth}{lQ@{}}
\textsc{sg} & Singular\\
\textsc{det} & Determiner\\
\end{tabularx}

\printbibliography[heading=subbibliography,notkeyword=this]
\end{document}
