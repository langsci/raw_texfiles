\documentclass[output=paper,colorlinks,citecolor=brown]{langscibook}
\ChapterDOI{10.5281/zenodo.6979319}
\author{Piero Visconte \affiliation{University of Texas, Austin} and {Sandro Sessarego} \affiliation{University of Texas, Austin}}
\title[Loíza Spanish and the Spanish Creole debate]{Loíza Spanish and the Spanish Creole debate: A linguistic and sociohistorical account}

\abstract{This study combines sociohistorical and linguistic insights to cast light on the nature and origin of Loíza Spanish (LS), an Afro-Hispanic vernacular spoken in Loíza, Puerto Rico by the descendants of the Africans brought to this region to work as slaves during the colonial period. The present work assesses the evolution of this variety and its implications for creole studies. In so doing, it challenges the posture that would picture certain contemporary features of LS and other Afro-Hispanic dialects as the traces of a previous creole stage (\citealt{Grandade1968} et seq.). Thus, this article contributes to the long-lasting Spanish Creole Debate \citep{Lipski2005} by providing new information on a so-far little-studied Afro-Puerto Rican vernacular.

\begin{sloppypar}
\keywords{Afro-Puerto Rican Spanish, Loíza Spanish, Spanish Creole Debate, decreolization, creolization, language contact, Caribbean Spanish, slavery, sugarcane plantation, small-scale farms, Haitian Revolution, US territory}
\end{sloppypar}
}

\IfFileExists{../localcommands.tex}{
  \addbibresource{../localbibliography.bib}
  \usepackage{langsci-optional}
\usepackage{langsci-gb4e}
\usepackage{langsci-lgr}

\usepackage{listings}
\lstset{basicstyle=\ttfamily,tabsize=2,breaklines=true}

%added by author
% \usepackage{tipa}
\usepackage{multirow}
\graphicspath{{figures/}}
\usepackage{langsci-branding}

  
\newcommand{\sent}{\enumsentence}
\newcommand{\sents}{\eenumsentence}
\let\citeasnoun\citet

\renewcommand{\lsCoverTitleFont}[1]{\sffamily\addfontfeatures{Scale=MatchUppercase}\fontsize{44pt}{16mm}\selectfont #1}
   
  %% hyphenation points for line breaks
%% Normally, automatic hyphenation in LaTeX is very good
%% If a word is mis-hyphenated, add it to this file
%%
%% add information to TeX file before \begin{document} with:
%% %% hyphenation points for line breaks
%% Normally, automatic hyphenation in LaTeX is very good
%% If a word is mis-hyphenated, add it to this file
%%
%% add information to TeX file before \begin{document} with:
%% %% hyphenation points for line breaks
%% Normally, automatic hyphenation in LaTeX is very good
%% If a word is mis-hyphenated, add it to this file
%%
%% add information to TeX file before \begin{document} with:
%% \include{localhyphenation}
\hyphenation{
affri-ca-te
affri-ca-tes
an-no-tated
com-ple-ments
com-po-si-tio-na-li-ty
non-com-po-si-tio-na-li-ty
Gon-zá-lez
out-side
Ri-chárd
se-man-tics
STREU-SLE
Tie-de-mann
}
\hyphenation{
affri-ca-te
affri-ca-tes
an-no-tated
com-ple-ments
com-po-si-tio-na-li-ty
non-com-po-si-tio-na-li-ty
Gon-zá-lez
out-side
Ri-chárd
se-man-tics
STREU-SLE
Tie-de-mann
}
\hyphenation{
affri-ca-te
affri-ca-tes
an-no-tated
com-ple-ments
com-po-si-tio-na-li-ty
non-com-po-si-tio-na-li-ty
Gon-zá-lez
out-side
Ri-chárd
se-man-tics
STREU-SLE
Tie-de-mann
} 
  \togglepaper[6]%%chapternumber
}{}

\begin{document}
\maketitle

\section{Introduction}
For the past five decades, the field of Spanish contact linguistics has been characterized by a heated debate on the nature and origins of the vernaculars that formed in the Americas from the contact of African languages and Spanish in colonial times (\citealt{Grandade1968} et seq). A central aspect of this academic discussion, which \citet[Ch. 9]{Lipski2005} has labeled the “Spanish Creole Debate”, has to do with the paucity of attested Spanish-based creoles in this part of the world. Indeed, it is well known that there are only two contact varieties in the Americas that have traditionally been classified as Spanish creoles -- Papiamentu (spoken in the Netherlands Antilles) and Palenquero (used in a former maroon community, the village of Palenque, Colombia) -- a situation in sharp contrast with the relative abundance of their English- and French-based counterparts (\citealt{HolmPatrick2007}). In addition, some linguists would even question the status of Papiamentu and Palenquero as “Spanish-based”. In fact, they would claim that these languages may be classified as Spanish creoles only in a synchronic sense, since diachronically they would have developed out of Portuguese-based contact varieties, which only in a second phase of their evolution were relexified with Spanish words (\citealt{Schwegler1996,Martinus1996,McWhorter2000,Jacobs2012}). This proposal would, then, imply that Spanish never really creolized in the Americas.

A number of authors have tried to account for the paucity (or lack) of Spanish-based creoles by providing a variety of diverging hypotheses on pretty much every single Afro-Hispanic dialect spoken in the Americas (see \citealt{Sessarego2021}: Ch.1 for an overview). One of these hypotheses came to be known as the Decreolization Hypothesis \citep{Grandade1978}, which suggests that a Spanish creole once existed in Latin America, and that, after the abolition of slavery by the end of the 19th century, it quickly decreolized, and thus came to resemble the standard norm, leaving behind only a few grammatical traces of its former creole stage (e.g., “creole-like” features). Several scholars have embraced the Decreolization Hypothesis (\citealt{Perlschwegler1998,Schwegler1999,Otheguy1973,Megenney1993,Guy2017}), while others have questioned its validity by providing linguistic, sociohistorical, and even legal data to offer alternative explanations to the scarcity of attested Spanish creoles in the Americas (\citealt{Mintz1971,laurence1974caribbean,Lipski1993,McWhorter2000,Sessarego2017}).

To describe the heterogeneity of opinions on this topic and the lack of common agreement among the scholars involved in this animated discussion, \citet[304]{Lipski2005} stressed that “the last word [...] has yet to be written” on the Spanish Creole Debate. Lipski’s statement is sixteen years old now, but given the current range of diverging views on this issue (see, for example, the contrastive perspectives collected in \citealt{Sessarego2018_lingua}), it is certainly still valid today. Far from providing the “last word” on this topic, this paper will add more fuel to the debate by casting light on the nature and origin of a so-far little-studied Afro-Hispanic dialect from the Caribbean, Loíza Spanish (LS), a black vernacular spoken in Loíza, Puerto Rico by the descendants of the Africans taken to this region to work as slaves on the sugarcane plantations that developed on the island during the sugar boom in the 19th century. In particular, besides providing a linguistic account of the features encountered in this dialect that have traditionally been identified as indicators of a potential creole past for other Spanish Caribbean varieties (\citealt{Grandade1968,Otheguy1973}), this article adds a sociohistorical dimension to the analysis of slavery in the region, which will offer a more well-rounded perspective on the evolution of this Afro-Hispanic vernacular, and thus contribute in a broader sense to the Spanish Creole Debate.

\begin{figure}
% \includegraphics[width=\textwidth]{figures/loiza.png}
\includegraphics[width=\textwidth]{figures/PuertoRicoLoiza.png}

\caption{Loíza, Puerto Rico. (Map data ©2021 OpenStreetMap contributors)}
\label{fig:visconti:1}
\end{figure}

This paper is organized as follows: \sectref{sec:visconti:2} offers a brief analysis of the Spanish Creole Debate. \sectref{sec:visconti:3} provides an outline of some grammatical traits belonging to LS, which have commonly been classified in the literature as “creole-like” features, and that were detected in this dialect during linguistic fieldwork carried out in summer 2020. \sectref{sec:visconti:4} offers a sociohistorical account of black slavery in Puerto Rico, with a particular focus on Loíza, to understand whether a process of (de)creol\-i\-za\-tion could be at the root of the attested grammatical configurations. Finally, \sectref{sec:visconti:5} summarizes the main findings, elaborates on the origin and nature of LS, and provides the concluding remarks.

\section{The Spanish Creole Debate} \label{sec:visconti:2}

\Textcite{Grandade1968, Grandade1970, Grandade1978} was the first Hispanist who formulated a hypothesis to account for the scarcity of Spanish-based creoles in the Americas. He suggested that, since slavery in the Spanish colonies was presumably not that different from the forced-labor systems implemented by other European powers, Spanish creoles must have also existed in Spanish America, and that the contemporary lack of such contact varieties in these territories could only be explained as the result of a decreolization process (or approximation to the standard norm), which would have been driven by contact with the standard variety after the abolition of slavery. Such a decreolization process, in de Granda’s view, would not yet be completely over. Indeed, it would still be possible to detect some grammatical traces of this previous creole phase in the speech of a number of black communities across the Americas. According to de Granda, therefore, the presence of “creole-like” features in these varieties should be taken as linguistic evidence corroborating his model, which came to be known in the literature as the “Decreolization Hypothesis”.

A number of proposals have embraced the Decreolization Hypothesis, which was originally adopted to account for a set of grammatical features common to Caribbean Spanish varieties (\citealt{Otheguy1973, Schwegler1991,Schwegler1996}, \citealt{Megenney1993}). In this regard, \citet[334--335]{Otheguy1973} could not be more explicit when, after analyzing several grammatical traits encountered across these dialects (i.e., reduced number agreement across the Determiner Phrase, high rates of overt subject pronouns, presence of non-inverted questions, etc.), he stated:

\begin{quote}
In summary, the data presented here strongly suggest that the \textit{habla bozal}\footnote{The expression \textit{habla bozal} indicates the speech of \textit{bozal} slaves, African-born captives, who acquired a limited version of Spanish in the Americas.}  spoken in the Spanish Antilles (and possibly throughout the Caribbean) during colonial times was a Creole… Given this, the sample points of coincidence presented here between features which are shared by most Creoles but which are peculiar to Caribbean Spanish cannot be discarded as coincidence and must be taken into account in any explanation of the historical genesis of this major dialect type.
\end{quote}

A group of scholars, on the other hand, began to question such claims, and pointed out that the living and working conditions on the Spanish haciendas in Cuba and Santo Domingo were not as harsh as those found in other European plantations -- especially before the sugar boom of the 19th century -- and thus a creole origin for the Cuban and Dominican varieties would be quite doubtful (\citealt{Mintz1971,laurence1974caribbean,Chaudenson1992,Lipski1993,Ortiz-López1998,Clements2009}). Some linguists came to suggest that, even though Cuba and Santo Domingo might not have been the ideal places for Spanish creolization, other Spanish mainland territories, such as coastal Venezuela, Chota Valley (Ecuador), coastal Peru, Veracruz (Mexico), Los Yungas (Bolivia) and Chocó (Colombia), may have presented the proper sociodemographic conditions for creole formation (\citealt{Schwegler1999,Schwegler2018,Lipski2008,Perez2015}; \citealt{alvarez_obediente1998,McWhorter2000}). In contrast to these claims, a set of specific case studies on the aforementioned Afro-Hispanic communities have provided sociohistorical and linguistic data that reject a potential (de)creolization model for such vernaculars (\citealt{Díaz-Campos_clements2008,Sessarego2013_iberia,Sessarego2013_chotavalley,Sessarego2014,Sessarego2019,Sessarego_inpress}).
In addition, some proposals have added a legal dimension to the Spanish Creole Debate, by claiming that one of the main factors preventing the creolization of Spanish in the Americas had to do with the peculiarities of the Spanish legal system in matters of slavery, which would have facilitated the integration of slaves into free society and thus their acquisition of the colonial language (\citealt{Sessarego2015,Sessarego2017,Sessarego2018_enhancing}).

In recent years, a revival of the Decreolization Hypothesis has been proposed by well-known sociolinguists, who have indicated that Caribbean Spanish, as well as a few other Mainland Afro-Hispanic and Afro-Lusophone varieties, may indeed have creole roots (\citealt{Guy2017,Schwegler2014,Schwegler2018}). A clear example of this more recent trend is \citet[72]{Guy2017}, who highlighted how Afro-Bolivian Spanish (ABS) could be taken as an example of a variety that would have gone half-way through this supposed (de)creolization process, which, in his view, affected many other Afro-Latino vernaculars, including Popular Brazilian Portuguese and Caribbean Spanish (\citealt{guy1981linguistic}). He stated:

\begin{quote}
Its history of linguistic isolation implies that ABS must be more basilectal, closer to the speech of the earliest generations of Africans in the Americas, than Brazilian Portuguese and Caribbean Spanish. This in turn implies a historical trajectory by which all of these varieties started out as creoles, or at least restructured varieties tending toward the creole end …, and then acquired their present form through differing degrees of standardization.
\end{quote}

While most of the studies on the supposed (de)creolization of Caribbean Spanish have been primarily concerned with the Afro-Hispanic dialects of Cuba and the Dominican Republic, with the exception of a few works, such as those by \citet{Alvarez-Nazario1959,AlvarezNazario1974} and \citet{MauleónBenítez1974}, not much attention has ever been paid to Afro-Puerto Rican Spanish. This article aims at filling this gap by focusing on LS. In so doing, it will not only document a so-far little-studied Afro-Puerto Rican variety; it will also add a new piece to the Spanish Creole Debate puzzle and thus contribute to explaining the paucity of Spanish creoles in the Americas.

\section{An account of Loíza Spanish “creole-like” features} \label{sec:visconti:3}

Linguistic data were collected during summer 2020 in Loíza, the municipality of Puerto Rico with the highest percentage of inhabitants who self-identify as “black” -- more than 64\% of the total local population (\citealt{Moya2003}; US Census \citeyear{UScensus2010}). Sociolinguistic interviews and grammaticality judgments were carried out with 53 informants of different ages (ranging from 19 to 92) and levels of education (ranging from illiterate people to speakers holding a college degree). Findings indicate -- as expected -- that the older and lesser-educated informants tend to present the highest rates of vernacular feature use, while the language use of younger and more educated members of the community more closely resembles the standard variety, even though the use of non-standard forms is still quite noticeable in their speech.

Even if full light has yet to be cast on the genesis and evolution of LS, from the analysis of the speech of this heterogeneous sample of \textit{loiceños}, it is evident that, as it is currently spoken, this variety does not show the radical grammatical restructuring that characterizes Spanish creoles such as Papiamentu \citep{Jacobs2012} or Palenquero \citep{Schwegler1996}. Present-day LS displays several phonological and morphological reductions, some African lexical borrowings, and other minor traces of contact-induced language change, but it definitively lacks the more intense traits of grammatical restructuring that are typically encountered in creole languages \citep{winford2003book}. For this reason, LS is -- for the most part -- easily understandable to any speaker of standard Spanish and, in broad terms, it could be said that it classifies more as a “Spanish dialect” than a “Spanish creole” \citep[10]{McWhorter2000}. 

Even though LS is not that divergent from standard Spanish, it, nevertheless, presents some grammatical phenomena that have repeatedly been reported in the literature as potential indicators of  a previous creole stage (\citealt{Grandade1968,Otheguy1973,alvarez_obediente1998}): \REF{ex:visconte:1a} high use of overt subject pronouns; \REF{ex:visconte:1b} instances of lack of subject-verb agreement; \REF{ex:visconte:1c} variable gender agreement in the Determiner Phrase (DP); \REF{ex:visconte:1d} reduced number agreement across the DP; \REF{ex:visconte:1e} sporadic presence of bare nouns; \REF{ex:visconte:1f} cases of agglutination of the article with the following noun; \REF{ex:visconte:1g} copula reduction by aspiration or apheresis; \REF{ex:visconte:1h} lack of subject-verb inversion in questions; (1i) /ɾ/ reduction on infinitive verb forms.

\ea \label{ex:visconte:1}
\ea \label{ex:visconte:1a}
{Si} \textit{{tú}} {tiene[s] el pecho apreta[d]o,} \textit{{tú}} {te toma[s] un   guarapillo [d]e curía.}
\glt `If you feel that your chest is tight, you should take a \textit{curía   guarapillo} (infusion).’

\ex \label{ex:visconte:1b} 
{La lluvia y el viento} \textit{{estaba[n]}} {fuerte fuerte en la playa [d]e   Villa Pesquera.}
\glt `The rain and the wind were very strong at Villa Pesquera   beach.’\\

\ex \label{ex:visconte:1c}
{Teníamo[s]} \textit{{todo} \textit{[toda]}} {la tierra pa[ra] nojotros[s].}
\glt `All the land was for us.’

\ex \label{ex:visconte:1d}
{Había} \textit{{tres} \textit{gato[s]} \textit{negro[s]}} {en la calle.}
\glt `There were three black cats in the street.’

\ex \label{ex:visconte:1e}
{[El huracán] María se llevó} \textit{{[el]}} {techo.}
\glt `Hurricane Maria tore off the roof.’

\ex \label{ex:visconte:1f}
{Vamo[s] [a] comprá[r]}\textit{{lalmueso}} {[el almuerzo].}
\glt `Let's buy lunch.’

\ex \label{ex:visconte:1g}  
\textit{{[Es]tá}} {tó[do] bien}.
\glt `Everything is fine.’

\ex \label{ex:visconte:1h}
{¿Qué} \textit{{tú} \textit{hace[s]}}?
\glt `What are you doing?’

\ex \label{ex:visconte:1i}
{Oye lo que te va [a]} \textit{{decí[r]}} {tu mai.}
\glt `Listen to what your mother is going to tell you.’
\z
\z

It is relevant to point out that the features reported in \REF{ex:visconte:1} parallel those found in a number of Puerto Rican literary texts representing colonial \textit{habla bozal} (\ref{ex:visconte:2}), such as \textit{La juega de gallos o el negro bozal} \citep{Caballero1852}, \textit{Décimas de 1898} \citep{Mason1918}, \textit{Tío Fele} \citep{Derkes1883} – all of which have been recompiled by \citet{AlvarezNazario1974} – as well as \textit{Flor de una noche} by \citet{Escalona1883} and \textit{Dinga y Mandinga} by \citet{Vizcarrondo1983}.

\ea\label{ex:visconte:2}
\ea \label{ex:visconte:2a}
  \textit{{Yo} }{tiene uno becero en casa [se]ño[r] Juan de Dio,} \textit{{yo} }{tiene   dinero juntado y niña Federica ba a da a mí pa comprá uno   llegua.}
\glt `I have a calf in the house of Mr. Juan de Dios, I have money   saved and the girl Federica is going to give me to buy a mare.’   (\citealt{Caballero1852}, in \citealt[384]{AlvarezNazario1974})

\ex \label{ex:visconte:2b}
{¡[La] niña Fererica} \textit{{son} \textit{[es]}} {buen amo!}
\glt`The girl Federica is a good master.’ (\citealt{Caballero1852}, in            \citealt[384]{AlvarezNazario1974})

\ex \label{ex:visconte:2c}
{Tú siempre tá jablando a mí con} \textit{{grandísima} \textit{[grandísimo]}} {rigó[r].}
\glt `You always speak to me with great rigor.’ (\citealt{Caballero1852},   in \citealt[384]{AlvarezNazario1974})

\ex \label{ex:visconte:2d}
{Un día en e[l] trabajo co[n]tándole} \textit{{las} \textit{pata[s]}} {a un gusano.}
\glt `One day at work counting the legs of a worm.’ \citep[77]{Escalona1883}

\ex \label{ex:visconte:2e}
 {Llava llevá} \textit{{[la]}} {señora.}
\glt `The lady is going to bring it.’ (\citealt{Derkes1883}, in \citealt[390]{AlvarezNazario1974})

\ex \label{ex:visconte:2f}
 \textit{{Lamo} \textit{[el} \textit{amo]}} {Pantaleón ta bravo.}
\glt `Master Pantaleon is angry.’ (\citealt{Caballero1852}, in \citealt[385]{AlvarezNazario1974})

\ex \label{ex:visconte:2g} 
{Nanquí} \textit{{toy}} {ma Mákinley.}
\glt `Here I am, my Mákinley.’ (\citealt{Mason1918}, in \citealt[396]{AlvarezNazario1974})

\ex \label{ex:visconte:2h}
{¿Po qué} \textit{{tú} \textit{no} \textit{ta} \textit{queré}} {a mí?}
\glt `Why don’t you like me?’ (\citealt{Caballero1852}, in \citealt[384]{AlvarezNazario1974})

\ex \label{ex:visconte:2i}
{Ayé[r] me diji[s]te negro y hoy te boy a} \textit{{contejtá[r]}}.
\glt `Yesterday you called me black and today I am going to            answer you.’ \citep[77]{Vizcarrondo1983}
\z
\z

The examples given in \REF{ex:visconte:1} and \REF{ex:visconte:2} confirm that LS is an Afro-Hispanic vernacular presenting a set of features that diverge quite significantly from standard Spanish and that are rooted in the traditional colonial speech of black \textit{bozales}. All these grammatical elements have been commonly attested across a number of Spanish contact varieties (\citealt{Klee_lynch2009}) and vernacular dialects (\citealt{Zamora-Vicente1989,lipski1994latin}). They are not only commonly found in creole languages; rather, some of these features also systematically occur in advanced L2 varieties of Spanish, as well as attritional L1 and heritage varieties (e.g., instances of lack of subject-verb agreement, variable gender agreement in the DP, sporadic presence of bare nouns, high use of overt subject pronouns) (\citealt{Montrul2008,Montrul2016,Geeslin2013,Romerosessarego2018}). In addition, a subgroup of these grammatical traits -- such as reduced number agreement across the DP, cases of agglutination of the article with the following noun, copula reduction by aspiration or apheresis, /ɾ/ reduction on infinitive verb forms -- appear to be quite widespread across several rural dialects of Spanish (e.g., rural varieties of Canary Island Spanish, Andalusian Spanish, and Murcian Spanish, to mention a few) \citep{Alvar1996}, for which a (de)creolization trajectory is certainly not plausible. Thus, as explained elsewhere in greater detail \citep{Sessarego2019}, all of these phenomena may be understood as either quite common vernacular features (e.g., various types of phonological agglutination and reduction as well as morphological simplifications; see \citealt{Sessarego2011}), or as byproducts of processing constraints applying at the interface between different linguistic modules (i.e., pronominal use → syntax/pragmatic interface; agreement reductions → morphology/semantics interface, bare nouns → syntax/semantics interface; see \citealt{Sessarego2021}), which are naturally found in all cases of contact-driven restructuring, and thus should not necessarily be taken as indicators of a previous creole stage.

The linguistic data collected for LS, therefore, should not be taken as tangible evidence in support of the Decreolization Hypothesis for this Afro-Caribbean vernacular, as some authors seem to suggest (\citealt{Grandade1970,Otheguy1973,Guy2017}). Rather, the attested grammatical configurations may be understood as the result of advanced L2 processes and vernacular rural traits, which were nativized and conventionalized at the community level in this Afro-Puerto Rican community, a scenario that has similarly been reported for a number of other Afro-Hispanic varieties in the Americas (\citealt{Lipski2005,Sessarego2013_iberia}). This being said, in order to get a more precise picture of the origin and evolution of this vernacular, a sociohistorical account of the nature of slavery in the region will be provided in the following section.

\section{A socio-historical account of black slavery in Puerto Rico} \label{sec:visconti:4}

Black slavery lasted in Puerto Rico for almost four centuries, from the early phases of the Spanish colonization of the island by the end of the 15th century to its abolition in 1873. Nevertheless, as in the rest of Latin America, the formal elimination of slavery, in practice, did not automatically imply for the Afro-descendant population of Puerto Rico the same degree of freedom and wealth enjoyed by the white and mestizo citizens living on the island \citep{Bas-García2009}. 

Indeed, the post-abolitionist system in place was designed in a way that would force former slaves to pay for their own freedom. Essentially, they were turned into debtors, who had to repay their value to their former masters. Thus, they became \textit{peones} ‘peons’, who fundamentally had to work for free for their former owners \citep{Bas-García2009}. Even after the end of \textit{peonaje} in the 20th century, the living and working conditions of most Afro-Puerto Ricans was far from being optimal, and the effects of such a situation can still be observed in the present. Indeed, the municipalities of Puerto Rico with the highest concentrations of black inhabitants tend to be characterized by poor infrastructure and few higher educational centers. More than two thirds of the inhabitants of Loíza, for example, are currently living below the poverty line (US Census \citeyear{UScensus2010}), while government studies have repeatedly highlighted how a significant number of the citizens of this community feel marginalized and experience some degree of social exclusion \citep{Rivera-Quintero2014}.

Since the black presence in Puerto Rico spans a period of almost four hundred years of slavery followed by more than a century of freedom, it is impossible to approach this phenomenon in a homogeneous way. For this reason, in order to better appreciate the position of Afro-descendants in the Puerto Rican society over time and the parallel evolution of their language, three main historical phases will be analyzed. This will help us understand whether, at any point in the history of Puerto Rico, Spanish could (de)creolize on the island. 

The first phase (1510--1791) goes from the first documented introduction of black slaves into the island in 1510 to the Haitian Revolution of 1791, which triggered the sugar boom in the Spanish Caribbean. This period consists of a relatively reduced percentage of African-descendants in Puerto Rico, who worked on small-scale farms and in mines. The second phase (1791--1873) includes the sugar boom, which led to a more significant introduction of an African workforce and the development of bigger sugarcane haciendas. This phase ends with the decline of the sugar industry and the abolition of slavery. The third and last phase (1873-present) concerns the post-abolition period, characterized by a progressive acquisition of civil rights by Afro-Puerto Ricans up to the present day.

Before getting into the details of these phases, we wish to provide some demographic data, which serve the purpose of understanding the dimensions of the slave trade to Spanish America in general, and to Puerto Rico in particular, in comparison with the volumes of African captives that were introduced in the rest of the European colonies overseas. 

Indeed, as \citet[115]{Lucena2000} indicates (see \tabref{tab:visconti:1}, from \citealt[88]{Curtin1969}), the African slaves taken to Spanish America over four centuries represent only a small fraction (less than 15\%) of the total number of black captives introduced into the Americas. The reason for this, as explained by a number of historians (\citealt{Andrés-Gallego2005,Cushner1980,Brockington2006}), has to do with the fact that Spanish colonies, at least until the sugar boom of the 19th century, had never relied on a black workforce on a massive scale. Indeed, to adopt \citegen{Berlin1998} famous dichotomy, Spanish colonies were not “slave societies”, but rather “societies with slaves”, in which black captives were certainly present and performed a variety of jobs, though the institution of slavery was not the main drive of local economies.

\begin{table}
\begin{tabular}{lrrrrr}
\lsptoprule
 {Colonies} & {16th c.} & {17th c.} & {18th c.} & {19th c.} & \multicolumn{1}{c}{Total}\\
 \midrule
 {Spanish} & 75,000 & 292,500 & 578,600 & 606,000 & 1,552,100\\
 {Portuguese} & 50,000 & 500,000 & 1,891,400 & 1,145,400 & 3,586,800\\
 {English} &  & 527,400 & 2,802,600 &  & 3,330,000\\
 {French} &  & 311,600 & 2,696,800 & 155,000 & 3,163,400\\
 {Dutch} &  & 44,000 & 484,000 &  & 528,000\\
 \midrule
 {Total} & 125,000 & 1,675,500 & 8,453,400 & 1,906,400 & 12,160,300\\
\lspbottomrule
\end{tabular}
\caption{African slave importations to European colonies in the Americas\label{tab:visconti:1}}
\end{table}


In relation to this point, it is worth looking at the breakdown of imports across Spanish colonies as estimated by \citet[89]{Curtin1969} and reported here in \tabref{tab:visconti:2}. As the table shows, Curtin’s calculations indicate that almost 50\% of all slaves taken to Spanish America arrived via Cuba. As \citep[70]{Clements2009} correctly highlights, it makes sense to expect that it would be there where we would “find the necessary conditions for the formation of a Spanish-lexified creole language”, which, among other sociodemographic factors, would imply significant disproportions between African-born slaves and Europeans on the island. Nevertheless, historical data indicate that, except for a short period around 1532, blacks never outnumbered whites until 1811, when more Africans were introduced in the island as a result of the sugarcane boom, at which point Afro-descendants came to represent 54.5\% of the population (cf. \citealt{Masó1976}: 115; \citealt{Clements2009}: 77).

\begin{table}
\begin{tabular}{lr}
\lsptoprule
 {Country} & {Number}\\
 \midrule
 Cuba & 702,000\\
 Ecuador, Panama, Colombia & 200,000\\
 Mexico & 200,000\\
 Venezuela & 121,000\\
 Argentina, Uruguay, Paraguay, Bolivia & 100,000\\
 Peru & 95,000\\
 Puerto Rico & 77,000\\
 Dominican Republic & 30,000\\
 Central America & 21,000\\
 Chile & 6,000\\
 \midrule
 Total & 1,552,000\\
\lspbottomrule
\end{tabular}
\caption{Distribution of the estimated slaves in Spanish America\label{tab:visconti:2}}
\end{table}

\citet[78--79]{Clements2009} also compares the Cuban figures with the demographic data for colonial Haiti to highlight how even Cuba, the most sugar-oriented economy of the Spanish Caribbean, was far from presenting the demographic disproportions between Africans and Europeans that led to language creolization in other Antillean regions. He states:

\begin{quote}
Comparing the population distributions of different Caribbean islands, we see that the distribution of Cuba’s population was more balanced than that of the other islands. For example, at the end of the eighteenth century (1792), Cuba had 54,152 (20 per cent) free colored, 84,590 (31 per cent) slaves, and 133,559 (49 per cent) whites. By contrast, around that time Haiti had 452,000 (98 per cent) slaves and 11,000 (2 per cent) whites.
\end{quote}

When we turn our attention to Puerto Rico, which received almost ten times fewer African slaves than Cuba (estimated to be 77,000; see \tabref{tab:visconti:2}), it stands to reason to think that the chances of a Spanish creole forming on this island appear to be quite slim. \citet[72]{AlvarezNazario1974} provides a rough breakdown of the importation of African slaves to Puerto Rico from the 15th to the 19th century. His numbers align -- to a good extent -- with Curtin’s, since he estimates anywhere between 54,000 and 75,000 captives (\tabref{tab:visconti:3}). His records also indicate a significant increase during the 18th and 19th centuries, which parallel the years of the sugar boom in the rest of the Spanish Caribbean.

\begin{table}
\begin{tabular}{l>{from }r<{ to}@{ }r}
\lsptoprule
 {16th c.} & 6,000  & 8,000\\
 {17th c.} & 8,000  & 12,000\\
 {18th c.} & 20,000 & 30,000\\
 {19th c.} & 20,000 & 25,000\\
 {Total}   & 54,000 & 75,000\\
\lspbottomrule
\end{tabular}
\caption{African slave importations to Puerto Rico (15th–19th centuries)\label{tab:visconti:3}}
\end{table}

Given these numbers, it is already possible to observe how a (de)creolization model to account for LS and the rest of the Spanish Caribbean varieties appears quite unlikely. Nevertheless, to better understand the potential dynamics of contact and Afro-Hispanic language evolution, in the following sections a more detailed sociohistorical analysis will be provided for the three aforementioned historical phases.

\subsection{First phase (1510--1791)} \label{sec:visconti:5}

Upon the arrival of the first Spanish colonizers in Puerto Rico by the end of the 15th century, the local indigenous population, the \textit{taíno} people, rapidly began to decline, as a result of warfare, European diseases, and the harsh working conditions imposed by the Spaniards (\citealt{Rouse1994,Brinton1997,Bernárdez2009}). As a way of supplying the island with more laborers to work in the mines, in the local mint, and on farms, black slaves were gradually introduced. The first documented arrival of blacks to the island dates from 1510; it concerns two captives who were sent from Spain to help in the minting of gold coins. \citet[30]{Díaz-Soler1974} describes the event with the following words:

\begin{quote}
A la isla de Puerto Rico arriban los primeros esclavos africanos en el año de 1510 cuando su Majestad autorizó a Jerónimo de Bruselas para traer a dos esclavos negros que habrían de ayudarle en el desempeño de su oficio de fundidor real.\\
(`The first African slaves arrived on the island of Puerto Rico in the year 1510 when his Majesty authorized Jerónimo de Bruselas to bring two Black slaves who were to help him in the performance of his office as royal foundryman').
\end{quote}

During the 16th and 17th centuries, as the numbers reported in Tables 4--5 suggest, the local economy did not rely much on an enslaved workforce. In fact, contrary to the policies implemented by other European powers in the Americas, Spain, for a concomitance of reasons, did not support a massive introduction of enslaved Africans into its overseas colonies. Among other factors that compressed the importation of black slaves into Spanish America was the Crown’s monopoly on slave trading. The Crown, in fact, assigned a limited number of import licenses to individual traders, who would be responsible for supplying Spanish America with black captives. Being a monopoly, the market was much constrained and the Crown would charge import taxes (\textit{alcabalas}) and sales taxes (\textit{almojarifazgos}) on each slave introduced into the colonies \citep{Palmer1976}. 

As for Puerto Rico, besides the aforementioned taxes, in 1513 it was established that for each slave entering the island, an additional fee of two ducats would have to be paid to the local authorities (\citealt{AlvarezNazario1974}: 29). The effects of these policies generated complaints and frustration among the Spanish settlers, who often saw these restrictions as barriers to the agricultural exploitation of the island and its economic development. Even the first Governor of Puerto Rico to be born on the island, Juan Ponce de León y Troche, in the second half of the 16th century commented on the wish for a more significant enslaved workforce to exploit the natural resources of the colony. \citet[487]{Tió1961} quotes the Governor’s words stating:

\begin{quote}
...no hay otro remedio tan ymportante para la conservación desta tierra y no haziéndose creo durar[á] su población no más de cuanto duren los pocos esclavos que hay en ella ay.\\`there is no other equally important solution for the conservation of this land [than by exploiting the labor of enslaved people], and I believe that, without doing it in that way, the survival of its population will last no longer than that of the few captives who currently live there.'
\end{quote}

The local economy, at that time, consisted of small farms, where blacks and whites worked side by side. Demographic data from the end of the 16th century show that the white population was 55\%, blacks were 28\%, while mixed race people represented 17\% (see \tabref{tab:visconti:4}, \citealt[74]{AlvarezNazario1974}). Race mixing, in fact, was common practice in Spanish colonies, and children born from the union of white masters with their black slaves tended to be freed at birth, thus generating a rapidly growing free group of mixed-race people (\citealt{Mintz1971,laurence1974caribbean}).

\begin{table}
\begin{tabular}{lr}
\lsptoprule
 {White} & 2,000 (55\%)\\
 {Black} & 1,000 (28\%)\\
 {Mixed race} & 600 (17\%)\\
\lspbottomrule
\end{tabular}
\caption{Population of Puerto Rico by the end of the 16th century\label{tab:visconti:4}}
\end{table}

It must also be pointed out that not all black people at this time were brought directly from Africa and thus only spoke African languages. On the contrary, as has been shown on a number of occasions (\citealt{Palmer1976,Restall2000,Brockington2006}), during the early phases of the Spanish colonization of the Americas, a good number of slaves were not shipped directly from Africa to the Americas. On the contrary, on many occasions, these were people who had lived in Spain with their masters for a long time before crossing the Atlantic. For this reason, they spoke Spanish (either natively or as an L2), were Christians, and knew the Spanish way of life. The word used in Spanish to describe these black servants was \textit{ladinos}, which distinguished them from those directly proceeding from Africa, \textit{bozales.} Given the aforementioned data, it is likely that the \textit{bozal} population, which was intuitively less than the total percentage of blacks (28\%, see \tabref{tab:visconti:4}), managed to acquire Spanish from the whites, the \textit{ladinos} and the members of the mixed-race group, and thus did not develop a creole variety during the 16th century.

As for the 17th century, \citet[74]{AlvarezNazario1974} states that there is no abundance of data that would allow us to closely follow the numerical evolution of the Afro-descendant population in Puerto Rico. Nevertheless, given the economic stagnation and diffused poverty that characterized the island, especially after gold mining was exhausted and the conquest of some Mainland territories transferred the colonizing enthusiasm of the Spaniards outside the Antillean region, not many incentives were found in Puerto Rico to import expensive black slaves. \citet[36]{AlvarezNazario1974} describes with the following words the reduced participation of this region in the transatlantic slave trade:

\begin{quote}
...el país [Puerto Rico] y, por lo que parece, los puertos antillanos en general, no participaron de los embarques de bozales que se trajeron al Nuevo Mundo.\\
`the country [Puerto Rico] and, apparently, the Antillean ports in general, did not participate in the shipments of \textit{bozales} that were brought to the New World'.
\end{quote}

Over time, especially since the second half of the 17th century, the local economy began to slowly grow thanks to the gradual development of the sugar industry. In an attempt to supply local sugarcane planters with a cheap labor force, in 1664 the then Governor of Puerto Rico, Miguel de La Torre, offered freedom to marooned slaves escaping from English, French and Dutch colonies, if they decided to settle on the island, convert to Catholicism and work in the agricultural sector. De La Torre’s policies resulted in a significant increase in the black population on the island, which is partially reflected in the San Juan 1673 census (\tabref{tab:visconti:5}), in which captives and free \textit{pardos} (mulattos) constitute more than 54\% of the total population (\citealt[75]{AlvarezNazario1974}).

\begin{table}
\begin{tabular}{lr}
\lsptoprule
 {Whites}                                                     & 820 \phantom{1}(45.78\%)\\
 {Slaves}                                                     & 667 \phantom{1}(37.24\%)\\
 {Free \textit{pardos}}\footnote{\textit{Pardos:} mulattoes.} & 304 \phantom{1}(16.98\%)\\
 \midrule
 {Total} & 1,791 (100.00\%)\\
\lspbottomrule
\end{tabular}
\caption{San Juan population by 1673}
    \label{tab:visconti:5}
\end{table}

\begin{sloppypar}
In the following decades, throughout the 18th century, the population of Puerto Rico showed a constant growth, mainly due to the numerous arrivals of Spaniards from the Canary Islands and the more significant introduction of African slaves to be employed in different economic sectors (see \tabref{tab:visconti:6}, \citealt[75]{AlvarezNazario1974}). During this phase, the Puerto Rican sugarcane industry was still composed, for the most part, of small and middle-sized haciendas. Thus, despite the exploitation of a more consistent black workforce, census data indicate that black captives represented a minority (12\%), significantly less than the white population (36\%) on the island (see \tabref{tab:visconti:6}, \citealt[76]{AlvarezNazario1974}).
\end{sloppypar}

\begin{table}
\begin{tabular}{lr}
\lsptoprule
{Whites, \textit{pardos} and free \textit{morenos}}\footnote{\textit{Morenos:} blacks.} & 39,846 \phantom{1}(88\%)\\
~~~ {[Whites]}&  {[14,344 \phantom{1}(36\%)]}\\
~~~ {[\textit{Pardos} and free \textit{morenos}]} &  [20,719 \phantom{1}(52\%)]\\
 {Slaves} & 5,037 \phantom{1}(12\%)\\
 \midrule
 {Total} & 44,883 (100\%)\\
\lspbottomrule
\end{tabular}
\caption{Puerto Rican population by 1765\label{tab:visconti:6}}
\end{table}


It is of interest to see that the majority group at this point consists of \textit{pardos} and free \textit{morenos} (52\% of the population), a factor that highlights two important aspects of Puerto Rican society: 1) racial mixing was highly common; 2) most Afro-descendants were free people. If we contrast this situation with the one reported by \citet[78--79]{Clements2009} when comparing the Cuban and Haitian demographics by the time of the Haitian revolution, we can immediately realize that the Puerto Rican economy, which relied on only 12\% of slaves, was even less slave-dependent than the Cuban one, which had 31\% of them, and was significantly different from Haiti’s, which had 98\% of enslaved people. All these considerations strongly suggest that 18th-century Puerto Rico was nothing like the ideal place for the creolization of Spanish. Rather, it was, in all likelihood, a colony in which black slaves -- even those proceeding directly from Africa (\textit{bozales}) -- would be able to acquire the colonial language over time.

\subsection{Second phase (1791--1873)}

The crisis caused by the Haitian Revolution (1791--1803) created the economic environment for the demand in sugar on the international market, to be supplied by the colonies of the Spanish Caribbean \citep{Villagómez2005}, especially by Cuba and Puerto Rico. The same did not happen in Santo Domingo (now the Dominican Republic), since Spanish planters feared that introducing more Africans into the region to develop the sugarcane industry could cause the Haitian uprisings to spread to the Spanish-controlled side of Hispaniola (\citealt{Ott1973,gibson_peterhouse2010}).

Puerto Rico, on the other hand, tried to develop this agricultural sector to supply the international market with sugarcane products. Incentives were provided for planters to move to the island so that in less than 30 years its population almost tripled \citep{Dietz2018}. In this period, the highest number of captives in the entire history of Puerto Rico is recorded (see \tabref{tab:visconti:7}, \citealt[76]{AlvarezNazario1974}). Nevertheless, as indicated by several historians (\citealt{morales1978,Martínez-Fernández1993,Stark2009}), its percentage never achieved the dimensions observed elsewhere in the Caribbean.
 
\begin{table}
\begin{tabular}{cc}
\lsptoprule
 {Number of captives} & {Total population}\\\midrule
 17,500 (13,76\%) & 127,133 (100,00\%)\\
\lspbottomrule
\end{tabular}
\caption{Population of Puerto Rico by 1794\label{tab:visconti:7}}
\end{table}

In relation to this, \citet[74]{Blanco1948} highlights how the Afro-descendant population in Puerto Rico was significantly more reduced than that of other Caribbean colonies. On this point, he states:
\begin{quote}
No obstante el gran crecimiento que experimenta la población de color en Puerto Rico para fines de XVIII, su número sigue siendo insignificante visto sobre el conjunto general de los habitantes de otras islas del Caribe por la misma época.\\
`Despite the great growth experienced by the people of color in Puerto Rico at the end of the 18th century, their number continues to be insignificant   compared to the overall population in other Caribbean islands at the same time'. 
\end {quote}

In an effort to further attract planters to Puerto Rico, in 1815 Spain decreed the \textit{Cédula de Gracias}, which established favorable conditions for the production and sale of sugar. In particular, it eliminated some taxes -- including those on the importation of slaves -- and encouraged the immigration of white settlers willing to develop the agricultural business by providing them with six acres of land per family member plus three acres per slave they could bring (\citealt{Baralt1981,Dorsey2003}).

As \citet[77]{AlvarezNazario1974} pointed out (see \tabref{tab:visconti:8}), despite the significant increase in the Black population due to the development of the sugar industry, it must be kept in mind that slavery in Puerto Rico never achieved the dimensions observed in the English and French Antilles, not even during the 19th-century sugar boom.

\begin{table}
\fittable{\begin{tabular}{l * 4{r@{ }r} r}
\lsptoprule
 {Year} & \multicolumn{2}{c}{Whites} & \multicolumn{2}{c}{Free Mulattos} & \multicolumn{2}{c}{Free Blacks} & \multicolumn{2}{c}{Captives\footnote{Blacks and Mulattos}} & \multicolumn{1}{c}{Total}\\\cmidrule(lr){2-3}\cmidrule(lr){4-5}\cmidrule(lr){6-7}\cmidrule(lr){8-9}
        & \multicolumn{1}{c}{$n$} & \multicolumn{1}{c}{\%} & \multicolumn{1}{c}{$n$} & \multicolumn{1}{c}{\%}& \multicolumn{1}{c}{$n$} & \multicolumn{1}{c}{\%}& \multicolumn{1}{c}{$n$} & \multicolumn{1}{c}{\%}\\\midrule
 1802 & 78,281  & (48.0) & 55,164  & (33.8) & 16,414 & (10.1)& 13,333 & (8.1)  &  163,192\\
 1812 & 85,662  & (46.8) & 63,983  & (35.0) & 15,833 & (8.6) & 17,536 & (9.6)  &  183,014\\
 1820 & 102,432 & (44.4) & 86,269  & (37.4) & 20,191 & (8.8) & 21,730 & (9.4)  &  230,622\\
 1827 & 150,311 & (49.7) & 95,430  & (31.5) & 25,057 & (8.3) & 31,874 & (10.5) &  302,672\\
 1830 & 162,311 & (50.4) & 100,430 & (31.2) & 26,857 & (8.4) & 32,240 & (10.0)  &  321,838\\
 1836 & 188,869 & (52.9) & 101,275 & (28.4) & 25,124 & (7.0) & 41,818 & (11.7) &  357,086\\
\lspbottomrule
\end{tabular}}
\caption{Population of Puerto Rico in the first third of the 19th century\label{tab:visconti:8}}
\end{table}

In the first decades of the 19th century, mulattos and blacks taken together tend to slightly outnumber the whites on the island, a pattern that is reversed after 1830, due to the more intense arrival of Spaniards from both the Iberian Peninsula and the Canary Islands, as well as whites proceeding from Latin America and other European colonies (from Venezuela, the French Antilles, Santo Domingo, Louisiana, and Florida) (\citealt[77--78]{AlvarezNazario1974}). Overall, whites, free mulattos, and free blacks, who were probably Spanish speakers, make up more than 88\% of the population at any point in time. Captives, who were made up by blacks and mixed-race people, are never more than 12\%. Thus, also in this case, even in the middle of the sugar boom, demographic data do not appear to support a potential creolization process for Afro-Puerto Rican Spanish, at least at the national level.

Similar demographic figures are reported by \citet[329]{Moya2003}, who indicates that despite the significant increase in the black labor force in the period that spans from 1750 to 1850, Puerto Rico continued to be a “society with slaves” rather than a “slave society”. According to his account, in fact, during this period the percentage of black slaves ranged between 7\% and 11\% of the population, precisely the reverse of the Caribbean slave societies, where only 3\% to 10\% of the population was free. 

Even though Puerto Rico never achieved the demographic composition of a prototypical “slave society”, the sugar boom caused a shift from small haciendas to middle-sized and large plantations. The more significant employment of an enslaved workforce required a new regulation, the \textit{Reglamento sobre la educación, trato y ocupaciones que deben dar a sus esclavos los dueños y mayordomos en esta Isla} (Regulation on the education, treatment and employment that the owners and overseers on this island must give their slaves), which was approved by Governor De La Torre in 1826 \citep{Zavala-Trías2003}.

The \textit{Reglamento} outlined a series of rights and obligations for both the enslaved Africans and their masters. It consisted of sixteen chapters that touched on topics of a varied nature: Catholic education, captives’ entertainment, food and clothing, tools of labor, marriage and family rights, manumission, correctional punishment, fines for the masters who did not follow the rules, etc. A close analysis of this legal text reveals that, in line with what observed in a number of works on comparative colonial slave law (\citealt{Tannenbaum1946,Watson1989}; \citealt{delaFuente2004}), and more recently also in some linguistic studies (\citealt{Sessarego2015,Sessarego2017,Sessarego2019}), Spanish slaves, in sharp contrast with any other European slave, had legal personality, and thus benefited from a variety of rights that derived from their status of legal persons: the right to own property, family preservation, (Christian) education, manumission, etc. \citep{Sessarego2018_laschiavitu}. All of these legal peculiarities, which differentiated the Spanish slave from any other black captive in the Americas, can certainly help us cast additional light on the paucity of Spanish-based creoles in this region (\citealt{Sessarego2018_enhancing,Visconte_forthcoming}).

As for the Puerto Rican case, it appears of interest to highlight the importance given by the government to the religious education of black captives, which, as has been stressed elsewhere (\citealt{Sessarego2013_chotavalley,Sessarego2015,Sessarego2019}), certainly contributed to language transmission. Indeed, an entire chapter was dedicated to slaves’ education (\textit{De la educación cristiana y civil que deben dar los amos a sus esclavos} ‘On the Christian and civil education that masters must give their slaves’). According to this chapter (see \citealt{Zavala-Trías2003}), masters had to instruct slaves in the principles of the Catholic religion and baptize them within a year of residence (article 1). Christian education had to be provided on a regular basis. Article 2, in fact, states as follows:

\begin{quote}
Esta instrucción será todas las noches después del toque de oraciones, haciendo se rece en seguida del rosario de María Santísima con la mayor compostura y devoción, la cual está generalizada en toda la Isla.\\
`This instruction will take place “every evening after the call to prayer”, by “having them pray,” once the Holy Mary's rosary is recited, with the greatest composure and devotion, which is the common practice throughout the Island'.
\end{quote}

In addition, article 3 established that:

\begin{quote}
...en los domingos y fiestas […] deberán los dueños de hacienda hacer que los esclavos ya bautizados oigan misa y la explicación de la doctrina cristiana.\\
`On Sundays and holidays […], the hacienda owners must make the baptized slaves hear mass and the explanation of Christian doctrine'.
\end{quote}

This piece of information on "the systematic Christian education of captives (which implicitly implies  Spanish language transmission), in addition to the data so far provided in relation to racial mixing", manumission and overall demographic composition of the Puerto Rican population, strongly suggests that a Spanish creole was not likely to form in 19th-century Puerto Rico. 

\subsection{Third phase (1873–present)}

The years preceding and following the abolition of slavery in Puerto Rico (on March 22nd of 1873) record a constant growth of whites, parallel to a decrease in blacks, which further reduced the possibilities of the creolization of Spanish (\citealt[78--79]{AlvarezNazario1974}).

\begin{table}
\begin{tabular}{c *4{r@{ }l}}
\lsptoprule
 {Year} & \multicolumn{2}{c}{Whites} & \multicolumn{2}{c}{Blacks} & \multicolumn{2}{c}{Mulattos}\\\cmidrule(lr){2-3}\cmidrule(lr){4-5}\cmidrule(lr){6-7}
        & \multicolumn{1}{c}{$n$} & \multicolumn{1}{c}{\%}& \multicolumn{1}{c}{$n$} & \multicolumn{1}{c}{\%}& \multicolumn{1}{c}{$n$} & \multicolumn{1}{c}{\%}\\\midrule
 1872 & 328,806 & (53.19) & 31,635 & (5.12) & 257,709 & (41.69)\\
 1877 & 411,712 & (59.48) & 39,781 & (5.75) & 240,701 & (34.77)\\
 1887 & 471,933 & (62.46) & 36,985 &  (4.9) & 246,647 & (32.64)\\
 1897 & 573,187 & (63.77) & 35,824 & (3.99) & 289,808 & (32.24)\\
 1899 & 589,426 & (61.84) & 59,390 & (6.23) & 304,352 & (31.93)\\
\lspbottomrule
\end{tabular}
\caption{Population of Puerto Rico before and after the abolition of slavery in 1873\label{tab:visconti:9}}
\end{table}

The post-abolition phase in Puerto Rico coincides with the US intervention on the island. Indeed, in 1898 Puerto Rico ceased to be a Spanish colony and came under the control of the United States with the signing of the Treaty of Paris, at the end of the Spanish-American War \citep{Orama-López2012}.

\begin{sloppypar}
The constant growth in the number of whites registered in the last three decades of the 19th century continued through the 20th century. \citet[79]{AlvarezNazario1974} emphasized how the racial categories used in the United States Census diverge from those adopted during the Spanish colonial period, so that mulatto people would now be classified as blacks (see \tabref{tab:visconti:10}, \citealt[79]{AlvarezNazario1974}).
\end{sloppypar}

\begin{table}
\begin{tabular}{cc *2{r@{ }r}}
\lsptoprule
 {Year} & {Total} & \multicolumn{2}{c}{Whites} & \multicolumn{2}{c}{Blacks}\\\cmidrule(lr){3-4}\cmidrule(lr){5-6}
        &         & \multicolumn{1}{c}{$n$} & \multicolumn{1}{c}{\%} & \multicolumn{1}{c}{$n$} & \multicolumn{1}{c}{\%}\\\midrule
 1910 & 1,118,012 &   732,555 & (65.52) & 385,437 & (34.48)\\
 1920 & 1,299,809 &   948,709 & (72.99) & 351,062 & (27.01)\\
 1930 & 1,543,913 & 1,146,719 & (74.28) & 397,156 & (25.72)\\
 1940 & 1,869,255 & 1,430,744 & (76.54) & 438,458 & (23.46)\\
 1950 & 2,210,703 & 1,762,411 & (79.76) & 446,948 & (20.24)\\
\lspbottomrule
\end{tabular}
\caption{US Census 1910--1950 in Puerto Rico\label{tab:visconti:10}}
\end{table}

The US invasion of the island had significant effects on the lives of its inhabitants and the local economy \citep{Shekitka2017}. As for the agricultural sector, in the first decades of the 20th century the Puerto Rican sugar industry, despite experiencing a great boom thanks to modern American machinery (its production increased by 331\%), did not provide much employment to manual workers, and thus accelerated the Puerto Rican migration to the United States (\citealt{ayala_bernabe2007}). On the other hand, the second postwar period (1945--1970) registered waves of bilateral migration, with many Puerto Ricans returning to the island, tempted by the modernization and industrialization in progress at the time, but also due to the difficulties of adaptation to living in the US \citep{Thomas2010}. This constant travel to the United States, still very much in vogue today, has given life to the notion of “Puerto Rican Nation on the move” \citep{Duany2000}, which highlights the migratory patterns of the Puerto Rican diaspora to and from the United States.

Nowadays Puerto Ricans who self-identify as “blacks” for US census purposes correspond to 12,39\% of the total population (US Census \citeyear{UScensus2010}). They primarily reside in the municipalities of Loíza, Arroyo and Maunabo, where Afro-de\-scen\-dants make up 64,25\%, 32,46\% and 30,43\% of the population, respectively. Loíza is, therefore, the region with the highest concentration of black people on the island and, for this reason, the following section will be focused on this community and its development in relation to the sugarcane industry. 

\section{Focus on Loíza}

During the Spanish colonial period, the perceived need for employing captives of African origin in the sugar industry concentrated the highest percentages of blacks and mulattos in regions like Loíza, located on the coastal plains, that were particularly well-suited for sugarcane production (\citealt{Mayo-Santana_Negrón-Portillo2007}). The first sugar mills appeared in the region by the end of the 16th century \citep{Ungerleider2000}. At that time, people of both European and African ancestry worked side-by-side in small haciendas dedicated to the production of sugarcane products \citep{MauleónBenítez1974}.

Over the 17th and 18th centuries the sugar industry in Loíza gradually expanded. The region began to develop, to the point that in 1690 the Governor of Puerto Rico, Gaspar Arrendondo, requested permission from the Spanish King to establish the \textit{Villa de Loíza}, officially declaring it in 1719 as one of the six existing municipalities on the island (\citealt{MauleónBenítez1974,Ungerleider2000}).

Upon the Haitian sugar crisis caused by the Haitian revolution and the consequent sugar boom in the Spanish Caribbean, the region of Loíza experienced an additional period of economic expansion, which implied the introduction of a significant black labor force. Nevertheless, even in the middle of the growth of the sugarcane business, the demographic figures reported for this region are nothing similar to what could be observed in the English and French Caribbean. Indeed, by 1828, the population of Loíza numbered a total of 4,198 inhabitants, the vast majority of which were free (82\%); only 18\% were slaves \citep[39--40]{Ungerleider2000}. 

\begin{table}

\begin{tabular}{cccccc}
\lsptoprule
          & \multicolumn{3}{c}{{Free Afro-descendants}} & \\\cmidrule(lr){2-4}
   {Whites}  & {Mulattos} & {Blacks} & {{\textit{Otras castas}}\footnote{\textit{Otras castas}: other mixed racial categories. It was common custom in Spanish colonies to classify mixed-race individuals according to their racial background: \textit{mulato} was the result of white and black mixing, \textit{morisco} was the result of black and \textit{mulato} mixing,  \textit{chino} was the result of  \textit{morisco} and white mixing, and so on.}} & Slaves & {Total} \\\midrule
556 (13\%) & 1,133 (27\%) & 714 (17\%) & 1,053 (25\%) & 724 (18\%) & {4,198}\\
\lspbottomrule
\end{tabular}
\caption{Population of Loíza between 1779 and 1828\label{tab:visconti:11}}
\end{table}

While the sugar boom provided economic development for the region during a good part of the 19th century, by the 1860s the sugar industry began to lose traction \citep{Picó1986}. The abolition of slavery in 1873 certainly did not help to reinvigorate the sector. Most black slaves living in the region became peons. Thus, the new \textit{libertos} (freedmen) remained living in the haciendas as \textit{agregados}, economically dependent individuals who sold their labor in exchange for lodging and food (\citealt{Meriño_perera2009}). Over the 20th century, thanks to the Land Reform of the 1940s, some of these peons received small parcels of land and thus became landowners \citep{Stahl1966}. Nevertheless, the living and working conditions of most \textit{afroloiceños} remained quite precarious, and even today are far from being optimal.

Loíza is, nowadays, an economically and socially depressed region. US census data indicate that some 67\% of the population lives below the poverty line, 8\% is still illiterate, and the region has one of the highest unemployment rates in the island \citep[47--49]{Ungerleider2000}. Some locals see military service as an opportunity to migrate to the United States \citep{Aramburu2012}. Those who decide to stay in the community tend to work in the production or sale of coconut, shellfish, fish, cassava, \textit{pitorro} (a black-market rum), and street food \citep[44--45]{Ungerleider2000}. Loíza’s economic and social isolation is, in part, exacerbated by its geographic location, surrounded by rivers, canals and the ocean, and connected to the nearby capital city of San Juan only by the \textit{ancón}, a transporter boat, which crosses the \textit{Río Grande de Loí}\textit{za} \citep[44]{Ungerleider2000}. 

Given the particular socioeconomic segregation that has been affecting Loíza over the past century and half, after the abolition of slavery in 1873, it is difficult to imagine that, if a Spanish creole ever existed in the region, it would have disappeared so completely due to processes of standardization, schooling and normative pressure.

\section{Conclusions}
 
This study has combined linguistic and sociohistorical data to cast light on the nature and origin of LS, an Afro-Hispanic dialect spoken in Loíza, Puerto Rico. In so doing, it has also added new fuel to the Spanish Creole Debate \citep{Lipski2005}. In particular, this study has evaluated the feasibility of the Decreolization Hypothesis (\citealt{Grandade1968} et seq.) for the Puerto Rican context. 

Linguistic data indicate that LS presents a set of morphological and phonological reductions, as well as a number of other vernacular features, which should not necessarily be linked to a creole origin, since they tend to appear in most Spanish contact varieties and in some rural dialects (\citealt{Klee_lynch2009,Zamora-Vicente1989,lipski1994latin}). As for the sociohistorical analysis, our findings suggest that at no point in the history of Puerto Rico were the conditions in place for a creole language to develop. A concomitance of sociohistorical factors (demographic, religious, economic, legal, etc.) favored the non-creolization of Spanish and the acquisition of the colonial language by the enslaved population. In particular, contrary to what could be observed across the English and French Caribbean (\citealt{Mintz1971,laurence1974caribbean,Clements2009}), in Puerto Rico the black population was never the majority group, racial mixing was highly common, and free Afro-descendants outnumbered enslaved blacks.
 
In conclusion, both linguistic and sociohistorical findings strongly indicate that the Decreolization Hypothesis is not a feasible option to account for the nature and origin of LS. On the other hand, in line with recent studies on other Afro-Hispanic vernaculars (\citealt{Díaz-Campos_clements2008,Sessarego2014, Sessarego2019}), LS appears to be a dialect presenting the linguistic traces of moderate grammatical restructuring, which do not imply any previous (de)creolization stage.

\section*{Abbreviations}
\begin{tabular}{@{}ll@{}}
{LS} & Loíza Spanish\\
{L2} & Second language\\
\end{tabular}

\printbibliography[heading=subbibliography,notkeyword=this]
\end{document}
