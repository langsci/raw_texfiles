\documentclass[output=paper,colorlinks,citecolor=brown]{langscibook}
\ChapterDOI{10.5281/zenodo.6979313}
\author{Lise Winer\orcid{}\affiliation{McGill University}}

\title{``Suzie \& Sambo'' (1937–1956): What can they tell us today?}

\abstract{The recovery, discovery and “uncovery” of historical texts are particularly relevant to societies which have been traditionally undervalued and stigmatized.  In recent decades, more attention has been paid to texts in various Caribbean English/Creoles; the social and linguistic data therefrom have proved invaluable in reconstructing the historical development of these languages \citep{Winford1997re-exam}. The “Suzie and Sambo” columns are an excellent, hitherto neglected, source of “dialect” texts from Trinidad \& Tobago, published in local pro-labour newspapers between 1937 and 1956.  The columns purport to be conversations between “Suzie” and “Sambo”, two working-class Tobagonians.  They comment on local conditions, politics, events, and scandals, as well as carrying on their own fraught relationship. Winford’s approach to Caribbean English Creole language variation \citep{Winford1997re-exam} argues that “continua existed in these situations from the earliest period of contact”, and he challenges the claim “that they evolve solely via ‘decreolization’ of basilects under influence from acrolects.” He argues for a co-existent systems approach to the contemporary structure of these continua. The evidence of sociolinguistic studies supports the idea that they result from interaction between relatively stable grammars, conditioned by social and situational factors. The variation produced by this interaction provides insight into the kinds of shift and contact-induced change that have always operated in these situations. The Suzie \& Sambo texts help to fill in data for a particular and important time period, and support the contention that “varilingualism” is a long-standing characteristic of this (and other) creole societies.  

\keywords{Creole continua, varilingualism, language variation, Trinidad and Tobago language, working class, grammatical particles, lexicon, newspaper dialect columns}
}
    




\IfFileExists{../localcommands.tex}{
  \addbibresource{../localbibliography.bib}
  
\newcommand{\sent}{\enumsentence}
\newcommand{\sents}{\eenumsentence}
\let\citeasnoun\citet

\renewcommand{\lsCoverTitleFont}[1]{\sffamily\addfontfeatures{Scale=MatchUppercase}\fontsize{44pt}{16mm}\selectfont #1}
   
  \usepackage{langsci-optional}
\usepackage{langsci-gb4e}
\usepackage{langsci-lgr}

\usepackage{listings}
\lstset{basicstyle=\ttfamily,tabsize=2,breaklines=true}

%added by author
% \usepackage{tipa}
\usepackage{multirow}
\graphicspath{{figures/}}
\usepackage{langsci-branding}

  %% hyphenation points for line breaks
%% Normally, automatic hyphenation in LaTeX is very good
%% If a word is mis-hyphenated, add it to this file
%%
%% add information to TeX file before \begin{document} with:
%% %% hyphenation points for line breaks
%% Normally, automatic hyphenation in LaTeX is very good
%% If a word is mis-hyphenated, add it to this file
%%
%% add information to TeX file before \begin{document} with:
%% %% hyphenation points for line breaks
%% Normally, automatic hyphenation in LaTeX is very good
%% If a word is mis-hyphenated, add it to this file
%%
%% add information to TeX file before \begin{document} with:
%% \include{localhyphenation}
\hyphenation{
affri-ca-te
affri-ca-tes
an-no-tated
com-ple-ments
com-po-si-tio-na-li-ty
non-com-po-si-tio-na-li-ty
Gon-zá-lez
out-side
Ri-chárd
se-man-tics
STREU-SLE
Tie-de-mann
}
\hyphenation{
affri-ca-te
affri-ca-tes
an-no-tated
com-ple-ments
com-po-si-tio-na-li-ty
non-com-po-si-tio-na-li-ty
Gon-zá-lez
out-side
Ri-chárd
se-man-tics
STREU-SLE
Tie-de-mann
}
\hyphenation{
affri-ca-te
affri-ca-tes
an-no-tated
com-ple-ments
com-po-si-tio-na-li-ty
non-com-po-si-tio-na-li-ty
Gon-zá-lez
out-side
Ri-chárd
se-man-tics
STREU-SLE
Tie-de-mann
} 
  \togglepaper[3]%%chapternumber
}{}

\begin{document}
\maketitle 

%\shorttitlerunninghead{}%%use this for an abridged title in the page headers



\section{Introduction}

Support and investigation of linguistic heritage are important in the study of Caribbean languages.  The recovery, discovery and  “uncovery" of historical texts are particularly relevant to societies which have been traditionally – and to some extent still are – undervalued and stigmatized.\footnote{An earlier version of a paper on this topic – “‘Suzie and Sambo’: Linguistic archaism or working class solidarity? Creole columns from \textit{The People}, 1938--1940” – was presented at the joint Society for Caribbean Linguistics\slash Society for Pidgin and Creole Linguistics Conference, Guyana, August 1994.} Despite creole languages being traditionally generally considered ``non-lit\-er\-ate" or even  “unwritable" due to their supposedly incomplete and inferior nature, there exists quite a larger body of written texts in such languages than might be expected. Considerable attention has been paid to such texts in various Caribbean English creoles (see \citealt{BakerBruyn1999,lalla_dacosta1990,migge_mühleisen2010,winer1991new,winer1993,winer1995penny, winer1997six,winer2003,winerbuzelin2008,winer_gilbert1987,winer_rimmer1994}). These texts, and the social and linguistic – lexical, grammatical and phonological – data therefrom, have proved invaluable in (re-)constructing the historical development of these languages (\citealt{Winford1997re-exam}, discussed below).

For several decades, creolists have been finding and republishing historical texts – newspaper columns, plays, novels \citep{Brereton_winer2021}.  A number of recovered ``lost" literary works have been republished by Broadview Press, Macmillan Caribbean Press, and the University of the West Indies Press.  In addition to old vinyl records from Cook, Folkways and the Library of Congress, and tape recordings from archived private collections such as J.D. Elder (University of Indiana Archives of Traditional Folk Music), there are reissued collections of recordings, from, for example, Rounder Records and Bear Family Records.  Still, few Caribbean newspapers have been microfilmed, and even fewer digitized.  Several university and public libraries and archives in the region have been building collections of documents, photographs, and personal papers.  Many such valuable linguistic records are neither safe nor accessible, and it is hoped that in this survey of Suzie \& Sambo we can see an urgent need to make them both.

The current work seeks to introduce researchers in linguistics and cultural history to the ``Suzie and Sambo" columns, an excellent but hitherto neglected source of ``dialect"\footnote{ The term ``dialect" is the most widely used designation in the English Caribbean, replacing ``broken English" or ``Negro English" and giving way to ``vernacular," ``local" and ``Creole".} (i.e. creole) texts from Trinidad, published in local newspapers between 1937 and 1956. Although it is not possible at this time to provide a complete linguistic analysis, either qualitative or quantitative, a number of practical applications of such analysis are exemplified and described.


\section{Context of publication}
\largerpage

The known run of all ``Suzie \& Sambo" columns was published in \textit{The People} and \textit{The Chronicle} newspapers between 1937 and 1956, which are held in the National Archives of Trinidad \& Tobago.  Retrieved texts comprise 195 columns (a few incomplete) containing over 200,000 words.\footnote{The original quality of the printing is poor, even compared to contemporary local newspapers. Although the archival staff has skilfully repaired many pages, some issues have illegible areas indicated in transcripts with [... ], whether from missing pieces of the paper itself, broken or misplaced type, or faded ink. In some cases, an indicated continuation was not found in the same or subsequent issues.  Obvious errors (e.g. an upside-down letter) have been silently corrected; larger inferences are indicated by [ ].  One warning to future researchers: the texts were entered into an  MS Word file; unfortunately, the Spell-Check feature was activated, and ``corrected" much of the spelling, e.g. \textit{fo} became \textit{for} or \textit{foe}.  It is now taking a very long time to recheck and copy-read as many of the pieces as possible.} It appears to have run almost every week from 1937 through the beginning of 1947, and then continued from 1953--1956; appearing also in \textit{The Clarion} in 1955--1956.\footnote{These texts are being prepared for open-access deposition in the Digital Library of the Caribbean (dLOC), University of Florida-Gainesville. I have deposited there a number of other texts, most from Trinidad \& Tobago, including all of \textit{Penny Cuts} \citep{winer1995penny}, short stories, newspaper articles, and an 1826 novel set in Dutch Guiana (Suriname). \url{https://dloc.com/results/brief/2/?t$=$lise+winer}.} \textit{The People} was a Trinidad-based newspaper whose views were strongly pro-labour, progressive and anti-colonial; it was taken over after WW II by Tubal Uriah ``Buzz" Butler, a well-known political and labour activist, and founder of the Oilfields Workers Trade Union. \textit{The People} was especially popular during the pre-World War II period – a time of important labour unionization, particularly amongst workers in the oilfields and on sugar estates.  The newspaper, published by L.F. Walcott in Port of Spain, was especially vigorous in its coverage of and support for both local labour movements and foreign struggles, such as that of Abyssinia (now Ethiopia) against the Italian Fascists.  Given the political orientation of the paper, most of the readers (either primary or secondary readers or secondarily read-to),\footnote{According to the \textit{Trinidad \& Tobago Annual Statistical Digest 1991}, the literacy rate of persons over 5 years of age in 1931 was 64$\%$; in 1941 it was 74$\%$.} were probably working and lower middle class people of African descent.

The author of almost all of the ``Susie \& Sambo" columns was the Afro-Creole Tobagonian L.A. Peters (ca. 1880--ca. 1942) a school-teacher and civil servant known from his initials as ``The Lappe" (referring to the \textit{lap} or \textit{lappe}, a prized game animal) (see \citealt[83--84]{craigjames2008}).\footnote{There are other articles of opinion and reportage in the newspaper signed by LAP or occasionally L.A. Peters.  It is not known for certain who took over writing the column after Peters’ death, sometime after 1953.  Rumour holds that this was an unidentified woman working at \textit{The Chronicle} who had helped in the previous editing of the column when the LAP fell ill.} The columns purport to be conversations between ``Suzie" and ``Sambo", two working-class Tobagonians, usually speaking while in Tobago and occasionally in Trinidad.  They comment mostly on local conditions, politics, events, and scandals, as well as carry on their own fraught relationship.  For example, in the issue of April 23,  1938,  Suzie and Sambo discuss a schoolmaster’s immoral behaviour and the refusal of the Director of Education to remove the teacher.

 
\section{Implications of the material for creole linguistic models}
\begin{sloppypar}
In its simplest iteration, the question of creole continua has boiled down to wheth\-er there are two distinct linguistic systems (basilect and acrolect) ``mediated" or scaffolded with approximations of the mesolects, or a whole body of language belonging to one large system with a single grammar and no clear boundaries between varieties (a ``seamless whole").  Change was often assumed to be unidimensional and unidirectional, from basilect to acrolect \citep[233]{Winford1997re-exam}.
\end{sloppypar}

Winford’s approach to Caribbean English Creole language variation is that ``continua existed in these situations from the earliest period of contact", and he challenges the claim ``that they evolve solely via ‘decreolization’ of basilects under influence from acrolects." He argues for a co-existent systems approach to the contemporary structure of these continua. The evidence of sociolinguistic studies supports the idea that they result from interaction between relatively stable grammars, conditioned by social and situational factors. The variation produced by this interaction provides insight into the kinds of shift and contact-induced change that have always operated in these situations (\citeyear{Winford1997re-exam}: 233).

Winford positions his view to some extent on both types of explanations, in which: 

\begin{quote}
a range of variation existed in the early period of settlement... followed later by a period in which the rapidly growing population of field slaves developed a relatively homogeneous creole during the plantation era. It was this creole that was subject to later decreolization under the influence of varieties closer to the superstrate, in the period after emancipation... The character of the intermediate varieties was also shaped by contact with both the basilect and the acrolect, leading to a more complex pattern of variation in the developing continuum. (\cite{Winford1997re-exam}: 233) 
\end{quote}

The current paper is not designed to address the validity of various approaches to the structural evolution of linguistic variety in CEC situations.  Both the mechanisms and the results of linguistic development in the Caribbean are different according to territory.  Examining Jamaica, Guyana, Belize and Trinidad, \citet{Winford1997re-exam} notes that concepts such as ``decreolization" and ``mesolect" can vary greatly between specific territories. However, there are some significant points that should be noted about the particular relevance of the Suzie \& Sambo texts for the exploration of these models and hypotheses.

First, the series of linguistic, cultural and social inputs in terms of people that took place in Trinidad was among the most complex in the region (see discussions in  \citet{winer1993,winer1995penny,winer2009} and \citet[249--252]{Winford1997re-exam}).  The most important aspects of this development were:
1) the relative lateness of extensive outside settlement of Europeans and Africans from a wide range of locations;
2) the concomitant extensive variety of English Creoles spoken by incomers; and
3) that French Creole was in fact the lingua franca of many, if not most, Trinidad residents well into the twentieth century.\footnote{While \citet[251]{Winford1997re-exam} correctly points out the great influence of Barbadians on providing  models of English and mesolectal CEC forms such as \textit{doz} and \textit{did} for Trinidad, in the \textit{Penny Cuts} texts \citep{winer1995penny} from 1904--1906, ``born Trinidadians" aligned themselves even with ``small islanders" such as Grenadians in preference to not sounding like Bajans.}

Newspapers in Trinidad \& Tobago have a long tradition of writing in dialect, from one-offs such as ``The Sorrows of Kitty" in the \textit{Trinidad Standard} of 1839 \citep{winer1993} to long-term regular features such as the ``Letters to the Editor" from \textit{Penny Cuts} 1904--06 \citep{winer_rimmer1994}, the \textit{Trinidad Guardian}’s 1950s ``Macaw" \citep{Hannays1960}  and the uncollected 1970s ``Letter from Port of Spain" by ``Mamits".  All these columns focus on both personal and wider socio-political topics. The authors are anonymous or pseudonymous, although their identity is often an open secret known to contemporary and later audiences.
Second, these texts were published between 1937--1956, hence in not only a ``post-emancipation phase" \citep[238]{Winford1997re-exam} but, in the case of Trinidad, a significant period of time including the very influential influx of U.S. military personnel during and just after WWII. Change and developments within, as well as between, territories have been an uneven process affected by socio-economic factors (see \citealt{jamesyoussef2002}: 206--210).

\largerpage%long distance
Third, these texts support a varilingual model of linguistic competence, in which a child does not learn to produce the language varieties to which s/he is exposed to discretely but rather mixes them systematically (code-switching) according to external factors such as setting and topic. The child does not necessarily acquire \textit{full} competence in the two or three codes of his/her exposure but may rather have full competence in one and partial competence in others, or even partial competence in all three. Full varilingualism suggests full competence in two or more codes where code-mixing is a regular feature of use \citep[17--18]{jamesyoussef2002}. The Suzie \& Sambo texts help to fill in data for a particular and important time period, and support the contention that varilingualism is a long-standing characteristic of this (and other) creole societies.

Fourth, the texts are purported to be uttered by Tobagonians, not Trinidadians. There can be an unwelcome tendency to ``romanticize" Tobagonian as an archaic relic, and to exaggerate the current differences to the extent that many Trinidadians, especially, believe that Tobagonian is often unintelligible and more like Jamaican.  There have been and are differences between the two overall patterns of usage in which most features overlap, but with general differences in phonology, grammar and lexicon.  \citet{jamesyoussef2002} describe much of this in detail in Chs. 4 (sound system), 5 (verb phrase), 6 (noun and pronoun system) and 7 (other grammatical differences).  However, some of the features now considered Tobagonian and not Trinidadian, such as imperfective pre-verbal marker \textit{a}, e.g. \textit{me a go} ‘I am going/I go’, and remote (anterior) past marker \textit{bin}, e.g. \textit{me bin go} ‘I went [some time ago]’ as well as prepositions such as \textit{a} ‘to’, e.g. \textit{mi a go a market} ‘I go/am going to market’ were in fact liberally used in earlier texts from the 19\textsuperscript{th} and early 20\textsuperscript{th} centuries set in Trinidad.  That is, the language varieties of the two islands land used to overlap a lot \textit{more}.


\section{Language of the columns}
\largerpage
Linguistically, these columns are a treasure trove, and there are enough of them to enable quantitative as well as qualitative analysis.  Typical TT Creole features are plentiful in the texts. They are variable, that is, for example, both \textit{A/Ah} and \textit{me} are used for first person subject pronoun. Grammatical features of interest include those in \ref{ex:winer:1}:

\ea%1
	\label{ex:winer:1}
\ea object marker \textit{am}\\
    \textit{Gi}\textit{me de other $\$$10.00 le me put am up.}\\
    \textit{Look de money. No spend am pan dress.}\\
    \textit{Dat sarb am right.}

\ex future marker \textit{go}\\
    \textit{Wey me go get money fo buy frock[?]}\\
    \textit{... hot water go bun dem.}

\ex anterior\\
    \textit{bin Any body bun[?] Joe bin dey?}\\
    \textit{He say ‘e no bin go tell me t’all.}

\ex serial verbs\\
    \textit{come out go meet}

\ex verb +\textit{say}\\
    \textit{me ya’re say}

\ex initial copula \textit{a}\\
    \textit{O yes gal, a’ dat mek me no go lef yo t’all.}\\
    \textit{A so me coward}.

\ex directional preposition\\
    \textit{a} ... \textit{fo put am a jail}.

\ex pronouns\\
    \textit{O me gad.}\\
    \textit{Me no able tell you. }\\
    \textit{Go on play de fool, see if more Horn no grow a’ yo head.}\\
    \textit{Divide de money between a’ yo.}
    \z
\z



There are many examples of the variation that existed between and within speakers in both Trinidad and Tobago during the 19th and 20th centuries, and continues today. For example, plural is marked by ∅, \textit{dem} and \textit{–s}; more analysis might be able to discover patterns in when each form is favoured.  The non-completive marker \textit{a} serves as both habitual and progressive markers; however the habitual is marked by ∅, \textit{a} and \textit{does}, suggesting that the habitual diverged from the non-completive \textit{a} before the progressive did (eventually V\,+\,\textit{in}) (see discussion in \citealt[177--178]{winer_gilbert1987}).

Creole lexicon in the columns includes \textit{bush got ayze} ‘bush has ears’ (i.e., be careful, people may be listening), \textit{make mouth fast} ‘talk too much’, and \textit{go high, go low} ‘no matter what you do’.  \textit{Nuh} is both a negator and an emphatic tag soliciting agreement (now usually \textit{na} or  \textit{nah}).  Vocabulary now associated with (older) Tobagonian Creole rather than Trinidadian includes \textit{la-ka} ‘like’, \textit{suh tay} (\textit{so till}) ‘until’, \textit{ya’re} ‘hear’, \textit{yey-water} ‘tears’, \textit{bless eye on} [someone] ‘see (with pleasure)’, \textit{ratta} ‘rat’ and \textit{hi} (expression of surprise or disbelief).\footnote{ Lexical differences between the two islands have not yet been systematically studied, but a quick search of the Dictionary of the English/Creole of Trinidad \& Tobago (Winer 2009) reveals fewer than 300 words (out of 12,000+) tagged specifically for Tobago; of which more than half are flora and fauna. }

\largerpage
Typical phonological features of interest include consonant cluster reduction \textit{tan} ‘stand’, \textit{tory} ‘story’, final post-vocalic v > b, \textit{lib} ‘live’.
Associated only with Tobago are the h-shifting, i.e. variable /h/ (insertion/absence) as in \textit{hempty} ‘empty’, \textit{ome} ‘home’ and \textit{am an hegg} ‘ham and eggs’, but they do not constitute an overwhelming difference between the two linguistic areas. In many ``dialect" columns, the Creole (or nonstandard) features are perhaps ``thicker", ``deeper" or more frequent than they would be normally (i.e. in authentic speech), either because the writer wishes to make fun of the speakers, or because the writer wishes to emphasize language solidarity with the readers.  It should be pointed out that in these texts the Creole features are also not as thick as they \textit{could} be.  For example, in many cases in these columns, the author could have used \textit{dem} rather than \textit{-s} for plural marker.  Did he forget to be consistent?  Or was he trying to represent a typically occurring variation, more faithful to actual speech patterns?  The same question may be raised for more recent writing that includes creole language varieties (see \citealt{Bakerwiner1999}).

It is not easy to determine \textit{now} to what extent such language may have seemed exaggeration \textit{then}, particularly in view of the fact that this type of text is a highly polished rendition of a style of speech admired for its linguistic cleverness.  Although no one person may actually have spoken or speak exactly as these texts are written, they are in fact no more misrepresentative of ordinary speech than any dramatic play or monologue.  That is, while stylistically and statistically they may be of higher ``quality" than ordinary speech, they have counterparts in the oral domain not only of the practiced formal Creole styles such as \textit{robber talk}, \textit{wedding speech} and \textit{speech band}, but the everyday expert use of, for example, \textit{boof}, \textit{sweet talk} and \textit{old talk} \citep[57--58]{winer1993}.  The authors were, in any case, choosing language deliberately and from within a total repertoire that included standard formal written English, oral Creole, and some written Creole.

The use of Creole in these writings could indicate, on the one hand, solidarity and sympathy with the group of people thus portrayed, or it could indicate profound contempt and disrespect.  In some cases, the author seems clearly to be laughing \textit{at} people – because of their stupid actions, characteristics, or foibles, or the type of language they speak (accented, affected, or stigmatized).  How can one tell if the language in a particular text is an attempt to represent (and perhaps make fun of) linguistic archaisms or to indicate solidarity with the people (i.e., working class) who speak this way?  

One interesting area of investigation is the accompanying illustrations.  Each column has at the top a drawing of Suzie and Sambo.  At first and for a long while, this was as in \figref{fig:winer:1}, wherein the two are represented as rather ungainly peasants; in a number of later cases this was replaced by \figref{fig:winer:2}, portraying slim and fashionably dressed figures.

 
\begin{figure}
\includegraphics[width=8cm]{figures/winer1_cropped.png}
\caption{Illustration of Suzie and Sambo: Ungainly peasants\label{fig:winer:1}}
\end{figure}
 
\largerpage
\begin{sloppypar}
The representation of Suzie and Sambo’s speech is easily understood today by older Tobagonians.\footnote{The question of intelligibility is an interesting one, and potentially subject to empirical testing.  In a non-scientific experiment, presenting this as a ``Tobago" text to Trinidadians, even older ones, seems to have an immediate ``squashing" effect on Trinidadians who assume and expect Tobagonian to be unintelligible.} However, many features now considered (rightly or wrongly) as only Tobagonian, were also found – even common – in Trinidad at this time.  The fact that this column was printed in a newspaper primarily sold to working class readers in Trinidad – presumably not just to Tobagonians living there – is another indication that many of the linguistic features here were still widely understood (if not used) in both islands at this time (see discussion for earlier texts in \citealt{winer1995penny}).
\end{sloppypar}  



\begin{figure}
\includegraphics[width=6cm]{figures/winer2_cropped.jpg}
\caption{Later illustration of Suzie and Sambo\label{fig:winer:2}}
\end{figure}

\largerpage
There is plenty of material available in the texts to consider differences between Tobagonian and Trinidadian, although allowances must be made for more rapid loss of older words in many (but not all) Trinidadian contexts.  This is complicated by the fact that the demographically much higher population percentage of people of East Indian descent in Trinidad means that opportunity for the use of Bhojpuri-derived lexicon would concomitantly be greater.  Despite efforts to emphasize – and perhaps even exaggerate – linguistic differences between Tobago and (some speakers in) Trinidad, the two varieties share more and are more similar to each other than to any other CEC variety. In older texts, even well into the twentieth century, the differences are very small; ``older" Trinidadian is almost indistinguishable from older Tobagonian.  This is a controversial position, and it is important to distinguish contemporary speech patterns from those of the past.

 
\section{Sociolinguistic perspectives}

Opinions and attitudes expressed about topics indicate the author’s sympathies on social, economic, and political topics.  Nonetheless, it can be difficult to decide from the text alone – especially from small or isolated pieces – whether the author is being straightforwardly contemptuous or sarcastic, bitter or rueful, poignant or disgusted.  As discussed above, it is likely  that virtually all the readers knew – understood and spoke – varieties of Creole similar to those represented in the columns, and were being invited to identify with the social class and opinions – and language – of Suzie and Sambo, while at the same time they could appreciate and laugh/shake their heads at their weaknesses and exaggerations.  

Sambo seems to be an inept schemer and generally ineffectual braggart, depending on Suzie but wanting to keep his options free.  Suzie also schemes and frequently uses deception, her eventual aim being legal marriage. In the sample column given here (Appendix A), Sambo severely criticizes Suzie (and all women) for only caring about money, and for not providing adequate support (food).  Suzie retaliates by reminding him that this is because \textit{he} is not providing adequate support (money).  When this problem is temporarily resolved, Suzie professes her love for her Sambo.  So, there are plenty of examples of quarrelling and criticism, as well as \textit{sweet talk}.  In later columns, Suzie soothes her man not only with food and loving, but with liquor (``Oh-be-joyful", an old American expression).  Their relationship is characterized by professions of love, as well as criticism, on her part, and complaints on his.  


\section{Conclusion}


Altogether, these texts provide a coherent corpus that would repay scholarly attention.   That such a source has not been explored is almost certainly only the tip of a iceberg; there are many materials in archives and libraries (and private collections) – newspapers and memoirs, song lyrics and recipes – few catalogued, that could shine useful light on the domains of language, gender relations, and socio-cultural history.


\section*{Acknowledgements}
The author gratefully acknowledges Susan Craig-James for pointing out this source and for her immensely helpful historical work on Tobago (\citeyear{craigjames2008}); staff at the National Archives of Trinidad \& Tobago; and Jo-Anne Ferreira and Lawrence Carrington for thoughtful discussions.


\printbibliography[heading=subbibliography,notkeyword=this]

\section*{Appendix A: Sample “Suzie \& Sambo” column}

The People, 1 May 1937\\
\textsc{a little nonsense now and then}\\
Is Relished by the Wisest Men.

\subsection*{Original Version}
\begin{enumerate}[nosep]
    \item Sambo:  O’man, to tell you de truth, ef no bin fo de Law, a’ would a’ mash you skin an’ dreb yo way now, now. Dis a’ bittle fo man la’ka me to eat?
    \item Suzie:	Me ready to go now. No yo one ah man. Yo mus’ be mad Nigga Man. Yo wan’ me to tief fo feed yo?
    \item Sambo:  Well a’ right. Pack yo grip, an’ clare out at once. When man got money any o’man go glad to get am.
    \item Suzie:	A’ dem la’ka yo so a’ ha money? Look pan am. Nigga Mans go hide yo’ self.
    \item Sambo:  Oh ho. Yo tink me no ha money. Look. (He pulls out 3 five-dollar bills from his pocket).
    \item Suzie:	Sambo me honey, a’ fun me bin a’ mek wid you. Dear Sambo boy, yo tink me go leff you. A’ you a’ de only man in dis world me lub. Come kiss me an’ le me hug yo an’ squeeze yo.
    \item Sambo:   Yo see how ah yo ’omans tan. When man no ha’ money, dag better den he.\footnote{\textit{When man no ha’ money, dag better den he}. Perhaps a reference to Growling Tiger’s 1935 calypso, “Money is King”: “If you haven’t money, dog is better than you.”}
    \item Suzie:	No me Sugar, me Honey, me Sambo-Tambo.  Gi me de money le me run go ah shop come back. Yo go see how me lub yo me Sambo-Lambo, me own, own, own, own Sambo. Da Nigga man Joe bin a’ watch me but –
    \item Sambo:  Shut up. Yo tink me a’ fool no? Me only gi you one gad \$5.00.
    \item  Suzie:	A’right, gi me a’ me han’. Tell me how yo get so much money dis week.
    \item  Sambo:  Mark yo no tell nobody. Dis a’ ded secret. Dem a’ bill jetty an’ repair market hus dis week. Su when dem see de Boss a’ come, them mek stevedo mans liff up one piece a’ bode la-ka ef e a’ wuk. Mr. Full Stop e’ self come down pan de Wafe fo ketch dem but e so no barn yet.
    \item  Suzie:	An how you get su much?
    \item  Sambo:  ’Oman no hax me too much question. Me tell yo, me join Big Fish Club, an’ ef dem no run Bo-bul, how dem go mek out.
    \item  Suzie:	A’right me Sweetie. Gi me de other \$10.00 le me put am up. (He hands her the money). Tell me wha’ ’bout Jumbie Edman case and de Syrian case wha’ happen?
    \item  Sambo:  Gal oh… me no know wha fo say. After Jumbie Edmans done bruk the pillice mans ’ed wid stick, de Magistrate say e cant sen’ am a’ jail becarse ’e ’ed no good. He want fo sen am whe dem a’ sen mad people.
    \item  Suzie:	He right. Jumbie Edman out ob ’e ’ed fo true. An’ wha’ ’bout de Syrian Case.
    \item  Sambo:  A’ gal. Yo bin fo ya’re de Inspector talk fo Charles. He only a’ say ‘Council for de Defence.’ He no say ‘My Friend’ dis time.
    \item  Suzie:	Some body musse tell am. Or ’e does read ‘The People.’ But how de case pass?
    \item  Sambo:  De two side pay. All de money come up to ’bout \$70.00.
    \item  Suzie:	So much. Ah yo Bobul Club get any a’ dat?
    \item  Sambo:  ’Oman yo chupit eh? How dem go get dat? Look ya, me wan’ fo sleep. Shet yo mout now. When yo see money yo mout a’ a fly laka patch-corn.\footnote{\textit{yo mout’ [a] fly laka patch-corn}. i.e. salivates. \textit{Chilibibi} or \textit{samsam}, is a delicacy made
of sweetened pounded parched corn; when eaten, it can spray out of the mouth.}
    \item  Suzie:	A’right Sambo deer, go sleep love, and dream ’bout yo deer Suzie. To-marrow marning, me go gi yo Coffee-tea with ham and heggs an’ ting and ting.\footnote{ham and heggs. This is probably meant to be ’am and heggs.} Yo know how. The King self no go eat food la-ka yo to-marrow. Sleep good dear Sambo boy.
\end{enumerate}
\subsection*{English Version}
\begin{enumerate}[nosep]
    \item Sambo:  Woman, to tell you the truth, if it wasn’t for the Law, I would have mashed your skin and driven you away right now.  This is vittles for a man like me to eat?  
    \item Suzie:	I’m ready to go now.  Not you alone is man [‘You are not the only man’].  You must be mad, nigger man.  You want me to steal to feed you?
    \item Sambo:  Well all right. Pack your grip, and clear out at once. When a man has money any woman will be glad to get him. 
    \item Suzie:	A man like you has money? Look at you.  Nigger man, go hide yourself.
    \item Sambo:  Oh ho. You think I have no money.  Look.  (He pulls out 3 five-dollar bills from his pocket).
    \item Suzie:	Sambo, my honey, I was making fun with you.  Dear Sambo boy, you think I will leave you?  You are the only man in this world I love.  Come kiss me and let me hug you and squeeze you.  
    \item Sambo:  You see how all of you women are. When a man has no money, dog is better than him. 
    \item Suzie:	No, my Sugar, my Honey, my Sambo-Tambo.  Give me the money, and let me run to the shop and come back. You will see how I love you, my Sambo-Lambo, my own, own, own, own Sambo.  That nigga man Joe was looking at me but –
    \item Sambo:  Shut up. You think I’m a fool?  I’m only giving you one good \$5.00.
    \item  Suzie:	All right, give it to me in my hand. Tell me how you got so much money this week.
    \item  Sambo:  Mark you, don’t tell anybody. This is a dead secret. They were building a jetty and repairing the market house this week.  So when they see the Boss coming, they make the stevedores lift up a piece of board as if they are working.  Mr. Full Stop himself came down onto the Wharf to catch them but the man to catch them isn’t born yet.
    \item  Suzie:	And how did you get so much?
    \item  Sambo:  Woman, don’t ask me too many questions.  I told you, I joined the Big Fish Club, and if they don’t do something crooked, how will they make out?
    \item  Suzie:	All right my Sweetie. Give me the other \$10.00 and let me put it aside. (He hands her the money). Tell me what about the Jumbie Edmund case and the Syrian case, what happened?
    \item  Sambo:  Gal oh… I don’t know what to say. After Jumbie Edmans broke the police man’s head with a stick, the Magistrate said he couldn’t sent him to jail because his head is no good.  He wanted to send him where they send mad people.
    \item  Suzie:	He’s right. Jumbie Edmund is out of his mind in truth. And what about the Syrian Case?
    \item  Sambo:  Ah, gal. You should have heard the Inspector talk for Charles. He only said ‘Council for the Defence.’ He didn’t say ‘My Friend’ this time.
    \item  Suzie:	Somebody must have told him.  Or he reads ‘The People.’ But how did the case end up?
    \item  Sambo:  The two sides paid. All the money came up to about \$70.00.
    \item  Suzie:	So much! Did your Bobol Club get any of that?
    \item  Sambo:  Woman, you’re stupid, eh? How would they get that? Look here, I want to sleep. Shut your mouth now. When you see money your mouth flies like parched corn.
    \item  Suzie:	All right, Sambo dear, go to sleep, love, and dream about your dear Suzie.  Tomorrow morning, I will give you coffee-tea with ham and eggs and so on.  You know  how.  The King himself won’t eat food like you tomorrow.  Sleep well, dear Sambo boy.
\end{enumerate}
\end{document}
