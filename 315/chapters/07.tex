\chapter{Diachrony of voice syncretism} \label{sec:diachrony}
Voice syncretism has been described mainly in synchronic terms in the previous chapters, though a few brief diachronic comments have been provided sporadically. This chapter investigates the \isi{diachronic development} of voice syncretism in a more systematic manner. If it is assumed that voice marking in each of the seven voices of interest in this book (i.e. passive\is{passive voice}, reflexive\is{reflexive voice}, reciprocal\is{reciprocal voice}, anticausative\is{anticausative voice}, antipassive\is{antipassive voice}, causative\is{causative voice}, applicative\is{applicative voice}) can hypothetically develop one of the other six voice functions, 42 directional paths of voice development can logically be posited. Nevertheless, it has only been possible to find plausible evidence for the twenty paths that are not shown within parentheses in \tabref{tab:ch7:developmental-paths}. These paths serve as the focus of this chapter. In turn, the three paths within parentheses are also briefly discussed, yet evidence for these paths is tentative or lacks altogether. As evident in the table, antipassive\is{antipassive voice} voice marking has so far not been found to develop other voice functions in any language.

\begin{table} 
	%\setlength{\tabcolsep}{5pt}
	\begin{tabularx}{.88\textwidth}{rclcrclcrcl}
		\lsptoprule
		\multicolumn{3}{c}{Reflexive origin} & & \multicolumn{3}{c}{Reciprocal origin} & & \multicolumn{3}{c}{Anticausative origin} \\
		\midrule
		\textsc{refl} & → & \textsc{recp} & & \textsc{recp} & → & \textsc{refl} & & \textsc{antc} & → & \textsc{refl} \\
		\textsc{refl} & → & \textsc{antc} & & \textsc{recp} & → & \textsc{antc} & & \textsc{antc} & → & \textsc{recp} \\
		\textsc{refl} & → & \textsc{pass} & & (\textsc{recp} & → & \textsc{pass}) & & \textsc{antc} & → & \textsc{pass} \\
		\textsc{refl} & → & \textsc{antp} & & \textsc{recp} & → & \textsc{antp} & & (\textsc{antc} & → & \textsc{antp}) \\
		& & & & \textsc{recp} & → & \textsc{caus} & & & & \\
		& & & & \textsc{recp} & → & \textsc{appl} & & & & \\
		\midrule\midrule
		\multicolumn{3}{c}{Passive origin} & & \multicolumn{3}{c}{Causative origin} & & \multicolumn{3}{c}{Applicative origin} \\
		\midrule
		\textsc{pass} & → & \textsc{refl} & & \textsc{caus} & → & \textsc{antc} & & (\textsc{appl} & → & \textsc{recp}) \\
		\textsc{pass} & → & \textsc{recp} & & \textsc{caus} & → & \textsc{pass} & & \textsc{appl} & → & \textsc{antp} \\
		\textsc{pass} & → & \textsc{antc} & & \textsc{caus} & → & \textsc{appl} & & \textsc{appl} & → & \textsc{caus} \\
		\lspbottomrule
	\end{tabularx}
	\caption{Developmental paths of voice syncretism}
	\label{tab:ch7:developmental-paths}
\end{table}

The developmental paths listed in \tabref{tab:ch7:developmental-paths} represent somewhat simplified scenarios of how voice syncretism develops, as voice marking -- whether it is syncretic or not -- often has various additional semantic functions not qualifying as voice. For instance, in some languages reciprocal\is{reciprocal voice} voice marking can also be used to express \isi{sociativity} (\sectref{diachrony:reciprocal}). Consequently, it must be assumed that each of the developments of voice syncretism shown in \tabref{tab:ch7:developmental-paths} might potentially involve other semantic functions than merely voice. In other words, the rise of syncretic voice marking is not necessarily facilitated by a single voice function alone but jointly by the voice function in question and any other semantic functions that the marking might have or have had. Such additional functions of voice marking are duly acknowledged throughout this chapter and can serve as important bridging contexts\is{bridging context} in the rise of voice syncretism \citep{heine:kuteva:2007}. However, given the focus on voice in this book, the main interest of this chapter is syncretic voice marking for which it can plausibly be demonstrated that one voice function evolved \textit{prior} to other voice functions. For example, if synchronic reflexive-reciprocal voice marking was originally used for reflexivity\is{reflexive voice}\is{reflexive origin} (and other potential non-voice functions) or for reciprocity\is{reciprocal voice}\is{reciprocal origin} (and other potential non-voice functions like \isi{sociativity} mentioned above). In other words, if the reflexive-reciprocal syncretism in question has a reflexive\is{reflexive origin} or \isi{reciprocal origin}. By contrast, diachronic cases and scenarios in which it is unclear what voice function(s) evolved first are largely ignored in this chapter, and the chapter does not cover coincidental \isi{convergence} either. Moreover, observe that descriptions of diachronic developments\is{diachronic development} “from” a voice “to” another voice in this chapter do \textit{not} imply that voice marking loses one voice function in favour of another voice function. On the contrary, descriptions of this sort simply serve as a convenient way of expressing that the marking found in one voice comes to be used as marking in another voice as well -- or, in other words, marking with one voice function develops an additional voice function. 

Certain developmental paths of voice syncretism have received considerable attention in the literature, notably paths associated with \isi{middle syncretism}. Interestingly, however, there is not necessarily more diachronic evidence available for such paths than for other paths. In fact, as this chapter shows, the general lack of historical and comparative data for the vast majority of the world’s languages and genera\is{genus} makes it difficult to find concrete evidence for any given path of development. Consequently, data from historically well-attested languages, in particular from Indo-European languages, tend to get recycled in the literature, and it is not uncommon that diachronic developments\is{diachronic development} in attested in these languages are presupposed in languages with little historical and/or comparative data available. For the sake of linguistic diversity, Indo-European languages receive only little attention in this chapter while discussions of other language families and genera\is{genus} predominate. Furthermore, a strict distinction is maintained between synchronic observation and \isi{diachronic development}, meaning that the synchronic attestation of a pattern of syncretism is not automatically linked to any diachronic process. It is only deemed plausible that a given voice development has taken place in a particular language if genus-\is{genus} or family-internal comparative reconstructions\is{reconstruction} and/or historical data provide evidence for the development in question. The \isi{diachronic development} of voice syncretism in a given language is otherwise considered unresolved for the time being, and the ultimate origin of the syncretism regarded as uncertain. Evidence based solely on the language-specific synchronic distributional frequency or \isi{productivity} of voice functions is accordingly approached with caution, because such evidence cannot necessarily account for the loss of semantic functions. For example, as described in the next section, the passive-reciprocal-anticausative suffix \example{-s} in the Germanic language \ili{Danish} (\lang{ea}) has a reflexive\is{reflexive voice} diachronic origin\is{reflexive origin} but retains no reflexive\is{reflexive voice} function synchronically. 

\section{Reflexive origin} \label{diachrony:reflexive}
Voice syncretism of \isi{reflexive origin} is probably the best known and most extensively discussed voice diachrony in the literature (\citealt{haspelmath:1990, haspelmath:2003, kemmer:1993, heine:2000, heine:kuteva:2002, heine:miyashita:2008, maslova:2008, janic:2010, sanso:2017, sanso:2018}). Most notably, due to the long written tradition of Indo-European languages and centuries of comparative research, it is well known that the \ili{Proto-Indo-European} reflexive pronoun \example{*s(u)e} (\citealt[397]{kulikov:2010}; \citeyear[276]{kulikov:2013}) has grammaticalised\is{grammaticalisation} into a reflexive\is{reflexive voice} affix in many descendant languages which has later developed reciprocal\is{reciprocal voice}, anticausative\is{anticausative voice}, and passive\is{passive voice} functions. This development is illustrated in \tabref{tab:ch7:syncretism-ie} by examples from the Germanic language \ili{Danish}, the Romance language \ili{Spanish}, and the Slavic language \ili{Russian} (all \lang{ea}). In certain Indo-European languages the affix has even developed an antipassive\is{antipassive voice} function, notably in \ili{Russian} (\sectref{sec:complex-syncretism:multiplex}), but also in, for example, certain dialects of \ili{Danish} (e.g. \example{bid-s} ‘to bite [sb.]’, \citealt[62]{berkov:1985} via \citealt[297]{nedjalkov:2007d}) and standard \ili{Swedish} (cf. \example{bit-s} with the same meaning as the \ili{Danish} cognate). In \ili{Russian} this function is almost as common as the reciprocal\is{reciprocal voice} function \citep[681]{knjazev:2007} while it is marginal and/or fossilised\is{fossilisation} in \ili{Danish} and \ili{Swedish}. Moreover, note that the suffix \example{-s} in Danish has lost its reflexive\is{reflexive voice} function and its anticausative\is{anticausative voice} function has become almost obsolete, like in other Scandinavian languages, but both functions were present in earlier stages of the languages (cf. Old \ili{Norse} \example{verja-sk} ‘to protect self’, \citealt[203]{ottosson:2008}). Note also that the \ili{Spanish} voice marker \example{-se} only is used with infinitive, gerundive and imperative verbal forms (e.g. \example{alegrándo-se} ‘rejoicing’, \example{alégre-se} ‘rejoice’), while the particle \example{se} is used elsewhere (e.g. \example{se alegra} ‘s/he rejoices’).

\begin{table}
	\setlength{\tabcolsep}{3pt}
	\begin{tabularx}{\textwidth}{rclllll}
		\lsptoprule
		P.-I.-E.\il{Proto-Indo-European} & \example{*s(u)e} & \textsc{refl} & \multicolumn{1}{r}{→} & \textsc{recp} & \textsc{antc} & \textsc{pass} \\
		\midrule 
		\ili{Danish} & \example{-s} & \multicolumn{2}{l}{--} & \example{se-s} & \example{glæde-s} & \example{bygge-s} \\
		\ili{Spanish} & \example{-se} & \multicolumn{2}{l}{\example{defender-se}} & \example{ver-se} & \example{alegrar-se} & \example{construir-se} \\
		\ili{Russian} & \example{-sja} & \multicolumn{2}{l}{\example{zaščiščat’-sja}} & \example{videt’-sja} & \example{radovat’-sja} & \example{stroit’-sja} \\
		\midrule 
		& & \multicolumn{2}{l}{‘to protect self’} & ‘to see e.o.’ & ‘to rejoice’ & ‘to be built [by sb.]’ \\
		\lspbottomrule
	\end{tabularx}
	\caption{Voice syncretism of reflexive origin in Indo-European}
	\label{tab:ch7:syncretism-ie}
\end{table}

Owing in large part to observations from Indo-European languages, the development of passive\is{passive voice}, reciprocal\is{reciprocal voice}, and anticausative\is{anticausative voice} functions from a reflexive\is{reflexive voice} function is widely believed to be unidirectional\is{diachronic development, unidirectional} and follow certain paths of development. For instance, \citet[216]{heine:miyashita:2008} argue that “reflexives may grammaticalize\is{grammaticalisation} into reciprocals\is{reciprocal voice}, whereas reciprocals\is{reciprocal voice} do not seem to grammaticalize\is{grammaticalisation} into reflexives”, and \citet[921]{kazenin:2001b} states that “[i]t has been shown that the development always goes unidirectionally\is{diachronic development, unidirectional} from reflexive\is{reflexive voice} to passive\is{passive voice} (via anticausative\is{anticausative voice} […])”. The latter development has famously been elaborated and visualised by \citet{haspelmath:1990, haspelmath:2003}, who argues that “grammatical morphemes can only acquire new meanings from left to right” in \figref{fig:ch7:unidirectional}. There is undoubtedly good evidence for these diachronic scenarios, and additional evidence will be provided in the following sections, yet the unidirectionality\is{diachronic development, unidirectional} is not acknowledged in this book. As demonstrated later in the chapter, there is growing evidence for reverse or opposite developments as well.

\begin{figure}
	\caption{Unidirectional voice development \citep{haspelmath:2003}}
	\label{fig:ch7:unidirectional}
	\setlength{\tabcolsep}{2pt}
	\begin{tabularx}{0.88\textwidth}{lllllllll}
		Reflexive & → & Grooming\is{grooming} & → & Anticausative & → & Potential & → & Passive \\
		& & and motion & & & & passive & & \\
	\end{tabularx}
\end{figure}

As already mentioned in the beginning of this chapter, there is a general lack of historical and comparative data available for most of the world’s languages and genera\is{genus}, and the chronological order in which different functions of voice marking evolve consequently remains obscure in many languages. Moreover, as remarked by \citet[197]{kemmer:1993}, voice markers can be “so grammaticalized\is{grammaticalisation} in all their occurrences across a particular family that no diachronically prior function can be stated with confidence”. In fact, clear examples of voice syncretism of \isi{reflexive origin} in languages outside the Indo-European language family discussed above can be rather difficult to find, but various potential candidates are presented and discussed in the following sections.

\subsection{From reflexive to reciprocal} \label{diachrony:refl2recp}
Voice development from reflexive\is{reflexive voice} to reciprocal\is{reciprocal voice} has been discussed extensively in the literature, particularly in relation to non-affixal periphrastic reflexive\is{reflexive voice} and reciprocal\is{reciprocal voice} constructions (\citealt{heine:2000}; \citealt{heine:2000}; \citealt[254]{heine:kuteva:2002}; \citealt{heine:miyashita:2008}; \citealt{maslova:2008}). By contrast, evidence for verbal re\-flex\-ive-re\-ci\-pro\-cal syncretism of \isi{reflexive origin} is surprisingly scarce, though some clear cases of such syncretism have been reported in the literature. Most notably, as already described in the previous section, the \ili{Proto-Indo-European} reflexive pronoun \example{*s(u)e} (\citealt[397]{kulikov:2010}; \citeyear[276]{kulikov:2013}) has grammaticalised\is{grammaticalisation} into a reflexive\is{reflexive voice} affix and developed a reciprocal\is{reciprocal voice} function in languages of several Indo-European genera\is{genus}. Reflexive-reciprocal syncretism of \isi{reflexive origin} has also been noted occasionally for the Nilotic \isi{genus} of Africa in which the \ili{Proto-Nilotic} noun \example{*ri} ‘body’ has grammaticalised\is{grammaticalisation} into a reflexive\is{reflexive voice} suffix and developed a reciprocal\is{reciprocal voice} function in some descendant languages (\citealt[193ff.]{kemmer:1993};; \citealt[191f.]{heine:miyashita:2008}). This development is illustrated in \tabref{tab:ch7:refl-recp-nilotic} (Luo = \citealt[159]{tucker:1994}; Lango = \citealt[101]{noonan:1992}). \citet[44]{haspelmath:1990} observes a very similar development in the Biu-Mandara language \ili{Bura} (\lang{af}) in which the reflexive-reciprocal-anticausative suffix \example{-dzî} is related to the noun \example{dzá} ‘body’.

\begin{table}
	\setlength{\tabcolsep}{6.6pt}
	\begin{tabularx}{\textwidth}{rcrlll}
		\lsptoprule
		\ili{Proto-Nilotic} & \example{*ri} ‘body’ & & \textsc{refl} & → & \textsc{recp} \\
		\midrule 
		\ili{Luo} & \example{-rê} & \example{lwóko-rê} & ‘to wash self’ & & ‘to wash e.o.’ \\
		\ili{Lango} & \example{-(ɛ́r)ɛ̂} & \example{câŋ-ɛ́rɛ̂} & ‘to heal self’ & & ‘to heal e.o.’ \\
		\lspbottomrule
	\end{tabularx}
	\caption{\textsc{refl-recp} syncretism of \textsc{refl} origin in Nilotic}
	\label{tab:ch7:refl-recp-nilotic}
\end{table}

Reflexive-reciprocal syncretism in languages of another African \isi{genus}, Dogon, might have a \isi{reflexive origin} as well. Three languages of this \isi{genus} appear to possess cognates of the same suffix: Donno So\il{So, Donno} \example{-e/-i/u}, Tommo So\il{So, Tommo} \example{-i/-e}, and Toro So\il{So, Toro} \example{-ie}. \citet{culy:fagan:2001} reconstruct\is{reconstruction} the suffix \example{*-ie} for their ancestral language (here called \ili{Proto-So} for the sake of convenience) and argue that its original function likely was reflexive\is{reflexive origin} based on the fact that this function is attested in each of the three languages. Reflexes of this suffix in Donno So\il{So, Donno} and Tommo So\il{So, Tommo} -- but not in Toro So\il{So, Toro} -- can also serve as voice marking in the reciprocal\is{reciprocal voice} voice, a function which \citeauthor{culy:fagan:2001} consider a later development. This development is illustrated in \tabref{tab:ch7:refl-recp-dogon} \citep[181f., 188]{culy:fagan:2001}. Nevertheless, it can alternatively be hypothesised that the marker \example{*-ie} was characterised by reflexive-reciprocal syncretism already in \ili{Proto-So} and that the reciprocal\is{reciprocal voice} function eventually was lost in Toro So\il{So, Toro}.

\begin{table}
	\setlength{\tabcolsep}{6.6pt}
	\begin{tabularx}{\textwidth}{rcrllll}
		\lsptoprule
		\ili{Proto-So} & \example{*-ie} & \textsc{refl} & & → & \textsc{recp} & \\
		\midrule 
		Donno So\il{So, Donno} & \example{-e/-i/-u} & \example{yab-ɛ} & ‘to save self’ & & \example{tamb-ɛ} & ‘to kick e.o.’ \\
		Tommo So\il{So, Tommo} & \example{-i/-e} & \example{jɔŋ-i} & ‘to heal self’ & & \example{bɛ-i} & ‘to hit e.o.’ \\
		\lspbottomrule
	\end{tabularx}
	\caption{\textsc{refl-recp} syncretism of \textsc{refl} origin in Dogon}
	\label{tab:ch7:refl-recp-dogon}
\end{table}

\citet{heine:2000} argues for a general unidirectional development\is{diachronic development, unidirectional} from reflexive\is{reflexive voice} to reciprocal\is{reciprocal voice} among languages in Africa based on a survey of 62 languages spoken on the continent. However, while there are good grounds for postulating such development for non-affixal periphrastic reflexive\is{reflexive voice} and reciprocal\is{reciprocal voice} constructions on the basis of his survey, the scenario cannot automatically be extended to affixal reflexive-reciprocal syncretism. \citeauthor{heine:2000}’s (\citeyear[20ff.]{heine:2000}) sample includes only six languages featuring an affix serving as voice marking in both the reflexive\is{reflexive voice} and reciprocal\is{reciprocal voice} voices, three of which belong to the Nilotic \isi{genus} already discussed above (\ili{Luo}, \ili{Acholi}, \ili{Kalenjin}). The remaining three languages are the Bantu language \ili{Kisi}, and the Central Sudanic languages \ili{Lese} and \ili{Mangbetu}, yet the authors of the sources cited by \citeauthor{heine:2000} for these languages do not mention nor provide any evidence for a voice development from reflexive\is{reflexive voice} to reciprocal\is{reciprocal voice} (see \citealt{childs:1995} on \ili{Kisi}, \citealt{vorbichler:1965} on \ili{Lese}, and \citealt{larochette:1958} on \ili{Mangbetu}). Nevertheless, reflexive-reciprocal syncretism of \isi{reflexive origin} is attested in a Bantu language included the language sample of this book, Namibian Fwe\il{Fwe, Namibian}. In this language the reflexive\is{reflexive voice} prefix \example{rí-} has developed a reciprocal\is{reciprocal voice} function, whereas the historical reciprocal\is{reciprocal voice} suffix \example{-an} (cf. \ili{Proto-Bantu} \example{*-an}) has become almost obsolete \citep[257ff., 270f.]{gunnink:2018}. Moreover, it can be mentioned here that \citet{heine:miyashita:2008} briefly discuss the reflexive-reciprocal syncretism of the suffix \example{-ven̄ine} in the Edoid language \ili{Degema}, albeit not directly in relation to its diachrony. In any case, there does not seem to be any diachronic data on the precise origin of this suffix and its functions (cf. \citealt{kari:2004}).

It seems that there are few attested cases of reflexive-reciprocal syncretism of \isi{reflexive origin} outside of Eurasia and Africa in the literature, and examples from Indo-European genera tend to be recycled. For instance, oft-cited \citet[254]{heine:kuteva:2002} and \citet[233ff.]{maslova:2008} only provide examples from Russian. \citet{heine:miyashita:2008} do not provide any concrete examples of reflexive-reciprocal syncretism outside of Eurasia and Africa either, although they mention reflexive-reciprocal syncretism characterised by the suffix \example{-v} in the Yuman language \ili{Hualapai} (\lang{na}) and by the suffix \example{-inydji} in the Western Pama-Nyungan language \ili{Djinang} (\lang{au}) in their general discussion of the relationship between reflexivity\is{reflexive voice} and reciprocity\is{reciprocal voice}. However, there does not seem to be any evidence for the \isi{diachronic development} of the functions of the \ili{Hualapai} prefix \citep{watahomigie:al:1982, sohn:j-s:1995, ichihashi-nakayama:al:1997}, and \citet[149]{waters:1989} argues that the reciprocal\is{reciprocal voice} -- not the reflexive\is{reflexive voice} -- function “was probably the proto-function” of the \ili{Djinang} suffix (see also \sectref{diachrony:recp2refl} and \citealt[199f.]{heine:miyashita:2008}). \citet[375]{thompson:1996} argues that the reciprocal\is{reciprocal voice} function of the so-called \example{d}-classifier found throughout the Na-Dene language family (cf. \ili{Proto-Na-Dene} \example{*də-}) has evolved from a reflexive\is{reflexive voice} function. Nevertheless, the reflexive\is{reflexive voice} and reciprocal\is{reciprocal voice} functions are both attested throughout the language family, for which reason an alternative origin for the reflexive-reciprocal syncretism in these languages cannot be automatically rejected.

Despite reflexive-reciprocal syncretism being the most common kind of voice syncretism among the languages in the language sample (\sectref{dist:minimal}), a \isi{reflexive origin} can only be established with some certainty for a small number of the languages. For instance, as described and illustrated in \sectref{sec:complex-syncretism:middle}, the prefix \example{hup-} in the Nadahup language \ili{Hup} (\lang{sa}) serves as voice marking in the passive\is{passive voice}, reflexive\is{reflexive voice}, and reciprocal\is{reciprocal voice} voices. \citet[474, 486]{epps:2008} argues that the prefix ultimately derives from the noun \example{hup} ‘human, person’ which has the cognate \example{xup} ‘body’ in the related language \ili{Dâw}. Likewise, in the Yuman language \ili{Jamul Tiipay} (\lang{na}) the prefix \example{mat-} found in the reflexive\is{reflexive voice} and reciprocal\is{reciprocal voice} voices is derived from the noun \example{maat} ‘body’ \citep[167]{miller:a:2001}. This prefix also has an anticausative\is{anticausative voice} function described in the next section. \ili{Hup} and \ili{Jamul Tiipay} are likely to have undergone a development similar to that described for the Nilotic languages in the beginning of this section (see \tabref{tab:ch7:refl-recp-nilotic} on page \pageref{tab:ch7:refl-recp-nilotic}). Furthermore, as also noted in \sectref{sec:complex-syncretism:middle}, the prefix \example{mo-} in the Uto-Aztecan language Huasteca Nahuatl\il{Nahuatl, Huasteca} (\lang{na}) serves as voice marking in the passive\is{passive voice}, reflexive\is{reflexive voice}, reciprocal\is{reciprocal voice}, and anticausative\is{anticausative voice} voices. This prefix can be traced to \ili{Proto-Uto-Aztecan} \example{*mo-} for which \citet{langacker:1976} reconstructs\is{reconstruction} a reflexive\is{reflexive voice} function (see also \citealt[16]{anderson:al:1976}), suggesting that the reciprocal\is{reciprocal voice} and other functions represent later developments. Similarly, in the Tupi-Guaraní language \ili{Emerillon} (\lang{sa}) the reflexive-reciprocal prefix \example{ze-} \citep[348ff.]{rose:2003} is a reflex of the \ili{Proto-Tupi-Guaraní} reflexive\is{reflexive voice} prefix \example{*je-} which historically contrasted with reciprocal\is{reciprocal voice} \example{*jo-} \citep[534f.]{jensen:1998}. The developments in these four languages are illustrated in \tabref{tab:ch7:refl-recp-americas} (Hup = \citealt[479, 486]{epps:2008}; Jamul Tiipay = \citealt[166f.]{miller:a:2001};; Huasteca Nahuatl = \citealt[90]{llanes:al:2017}; Emerillon = \citealt[349f.]{rose:2003}).

\begin{table}
	\setlength{\tabcolsep}{3.1pt}
	\begin{tabularx}{\textwidth}{rllll}
		\lsptoprule
		& \textsc{refl} & \multicolumn{1}{r}{→} & \textsc{recp} & \\
		\midrule 
		\ili{Hup} & \example{hup-kɨ́t-} & ‘to cut self’ & \example{hup-nɔʔ-} & ‘to give e.o. sth.’ \\
		\ili{Jamul Tiipay} & \example{mat-aaxway} & ‘to kill self’ & \example{mat-tetekyuut} & ‘to greet e.o.’ \\
		Huast. Nahuatl\il{Nahuatl, Huasteca} & \example{mo-ilpi-} & ‘to tie self’ & \example{mo-ita-} & ‘to see e.o.’ \\
		\ili{Emerillon} & \example{-ze-kusug} & ‘to wash self’ & \example{-ze-potal} & ‘to love e.o.’ \\
		\lspbottomrule
	\end{tabularx}
	\caption{\textsc{refl-recp} syncretism of \textsc{refl} origin in the Americas}
	\label{tab:ch7:refl-recp-americas}
\end{table}

As discussed in more detail in \sectref{diachrony:recp2refl}, it has often been noted in the literature that several Australian genera\is{genus} feature what seem to be cognates of an ancestral reflexive\is{reflexive voice} proto-suffix \example{*-yi}. If this \isi{reconstruction} is accepted, the suffix appears to have developed a reciprocal\is{reciprocal voice} function among Worrorran and Mangrida languages (\citealt[341ff.]{alpher:al:2003};; \citealt[388]{green:2003}). The only potential evidence for reflexive-reciprocal syncretism of \isi{reflexive origin} among Papunesian languages in the sample can be found in the North Halmaheran language \ili{Ternate} (\lang{pn}) in which the reflexive\is{reflexive voice} prefix \example{ma-} and the reciprocal\is{reciprocal voice} prefix \example{maku-} bear some resemblance (i.e. type 2 syncretism\is{voice syncretism, partial resemblance -- type 2}). The related language \ili{Tidore} features the same marking as \ili{Ternate} \citep[244]{nedjalkov:2007d} while another related language, \ili{Sahu}, features very similar marking (cf. reflexive\is{reflexive voice} \example{ma-}, reciprocal\is{reciprocal voice} \example{ma’u-}, \citealt[199]{heine:miyashita:2008}). However, although the reflexive\is{reflexive voice} prefixes are less complex than the reciprocal\is{reciprocal voice} prefixes in these languages, the diachrony of the prefixes and their functions remain obscure. \citet[198f.]{heine:miyashita:2008} provide three other examples of similar syncretism from the Highland East Cushitic language \ili{Alaaba} (\lang{af}; cf. passive\is{passive voice} \example{-am} and reciprocal\is{reciprocal voice} \example{-akk’-am}, see \sectref{diachrony:pass2recp}), the Semitic language \ili{Amharic} (\lang{af}; cf. reflexive\is{reflexive voice} \example{tä-} and reciprocal\is{reciprocal voice} \example{tä-} plus \isi{reduplication}), and the Uto-Aztecan language Oklahoma Comanche\il{Comanche, Oklahoma} (cf. passive-reflexive \example{na-} and reciprocal\is{reciprocal voice} \example{nanah-}). Additional examples can be found in \sectref{resemblance-type2}. By contrast, compare the reflexive\is{reflexive voice} suffix \example{-l’at} and the reciprocal\is{reciprocal voice} suffix \example{-’at} in the South Guaicuruan language \ili{Pilagá} (\lang{sa}; \citealt[171f., 201ff.]{vidal:2001}). 

With regard to a functional diachronic explanation for reflexive-reciprocal syncretism of \isi{reflexive origin}, \citet[194]{heine:miyashita:2008} propose three plausible “[s]tages in the transition from reflexive\is{reflexive voice} to reciprocal\is{reciprocal voice}” presented in \figref{fig:ch7:refl-recp}. \citet[194]{heine:miyashita:2008} further specify that “[v]erbs used in Stage-III contexts tend to be referred to by labels such as inherently reciprocal\is{reciprocal voice} verbs, symmetric predicates, etc., typically including items such as ‘chat’, ‘follow’, ‘greet’, ‘kiss’, ‘marry’, ‘meet’, ‘shake hands’, etc.” As shown in \figref{fig:ch7:unidirectional} on page \pageref{fig:ch7:unidirectional}, Stage II can involve some kind of \isi{grooming} or body motion as an intermediary step towards becoming a full-fledged reciprocal\is{reciprocal voice}, e.g. ‘s/he washes self’ → ‘they wash themselves’ → ‘they wash each other’. \citet[194]{heine:miyashita:2008} regard the development in \figref{fig:ch7:refl-recp} as unidirectional\is{diachronic development, unidirectional}, yet it is worth observing that the opposite development appears to have taken place in several geographically diverse languages, as further discussed in \sectref{diachrony:recp2refl}. Thus, in this book reflexivity\is{reflexive voice} is considered but one possible origin of reflexive-reciprocal syncretism.

\begin{figure}
	\caption{Reflexive-reciprocal syncretism of reflexive origin}
	\label{fig:ch7:refl-recp}
	\begin{tabularx}{.90\textwidth}{lX}
		Stage-I & “There is a grammatical marker (and an associated construction) having a reflexive meaning when used with singular antecedent referents. \\
		Stage-II & When used with multiple antecedents, the marker may receive a reciprocal meaning in addition – the result being ambiguity. \\
		Stage-III & When used with multiple antecedents in specific contexts (e.g., with symmetric predicates), reciprocal is the only meaning”. \hfill \citep[194]{heine:miyashita:2008}
	\end{tabularx}
\end{figure}

\subsection{From reflexive to anticausative} \label{diachrony:refl2antc}
Voice development from reflexive\is{reflexive voice} to anticausative\is{anticausative voice} is commonly discussed in relation to its role as an intermediary stage in the development from reflexive\is{reflexive voice} to passive\is{passive voice}, as already shown in \sectref{diachrony:reflexive} (see \figref{fig:ch7:unidirectional} on page \pageref{fig:ch7:unidirectional}) and further discussed in the next section. Such development is often exemplified by data from Indo-European languages, yet examples of the phenomenon can be found sporadically in other genera\is{genus} as well. For instance, in the language isolate \ili{Nivkh} (\lang{ea}) the reflexive-anticausative marker \example{pʰ-} is derived from the reflexive pronoun \example{pʰi} (\citealt[191f.]{nedjalkov:otaina:1981};; \citeyear[108f.]{nedjalkov:otaina:2013};; \citealt[44]{haspelmath:1990}; \citealt{nedjalkov:al:1995}), and in the Central Arawakan language \ili{Paresi-Haliti} (\lang{sa}) the reflexive-anticausative suffix \example{-oa} can be traced back to the \ili{Proto-Arawakan} reflexive\is{reflexive voice} suffix \example{*-wa} \citep[109f.]{wise:1990}. In the Gunwinyguan language \ili{Nunggubuyu} (\lang{au}) the suffix \example{-i} serving as voice marking in the reflexive\is{reflexive voice}, anticausative\is{anticausative voice} and antipassive\is{antipassive voice} voices descends from the \ili{Proto-Gunwinyguan} reflexive\is{reflexive voice} suffix \example{*-yi} (\sectref{diachrony:recp2refl}). Likewise, as mentioned in the previous section, the prefix \example{mo-} found in the reflexive\is{reflexive voice} and anticausative\is{anticausative voice} voices in the Uto-Aztecan language Huasteca Nahuatl\il{Nahuatl, Huasteca} (\lang{na}) probably evolved from an original reflexive\is{reflexive voice}\is{reflexive origin} function (cf. \ili{Proto-Uto-Aztecan} reflexive\is{reflexive voice} \example{*mo-}, \citealt{langacker:1976}). As also briefly mentioned in the previous section, the prefix \example{mat-} derived from the noun \example{maat} ‘body’ and characterising the reflexive\is{reflexive voice} and reciprocal\is{reciprocal voice} voices in the Yuman language \ili{Jamul Tiipay} (\lang{na}) also has a marginal anticausative\is{anticausative voice} function. \ili{Jamul Tiipay} thus appears to have undergone a development similar to that discussed for \ili{Nivkh} above. The development from reflexive\is{reflexive voice} to anticausative\is{anticausative voice} in these languages is illustrated in \tabref{tab:ch7:refl-antc-world} (Nivkh = \citealt[69]{nedjalkov:al:1995}; Paresi-Haliti = \citealt[248f., 255]{brandao:2014}; Nunggubuyu = \citealt[390]{heath:1984}; Huasteca Nahuatl = \citealt[90ff.]{llanes:al:2017};; Jamul Tiipay = \citealt[166f.]{miller:a:2001}).

\begin{table}
	\setlength{\tabcolsep}{5pt}
	\begin{tabularx}{\textwidth}{rllll}
		\lsptoprule
		& \textsc{refl} & \multicolumn{1}{r}{→} & \textsc{antc} & \\
		\midrule 
		\ili{Nivkh} & \example{pʰ-χa-} & ‘to shoot self’ & \example{pʰ-χav-} & ‘to get hot’ \\
		\ili{Paresi-Haliti} & \example{airikoty-oa} & ‘to cut self’ & \example{txiholaty-oa} & ‘to open’ \\
		\ili{Nunggubuyu} & \example{balh-i-} & ‘to cut self up’ & \example{nᵍaṉḏ-i-} & ‘to sink’ \\
		Huasteca Nahuatl\il{Nahuatl, Huasteca} & \example{mo-ilpi-} & ‘to tie self’ & \example{mo-kweso-} & ‘to get sad’ \\
		\ili{Jamul Tiipay} & \example{mat-sxwan} & ‘to scratch self’ & \example{mat-uunall} & ‘to get lost’ \\
		\lspbottomrule
	\end{tabularx}
	\caption{\textsc{refl-antc} syncretism of \textsc{refl} origin across the world}
	\label{tab:ch7:refl-antc-world}
\end{table}

In addition to Eurasia, Australia, and the Americas, reflexive-anticausative syncretism of \isi{reflexive origin} has also been attested in Africa. For instance, as briefly mentioned in the previous section, in the Biu-Mandara language \ili{Bura} the suffix \example{-dzî} related to the noun \example{dzá} ‘body’ is not only used in the reflexive\is{reflexive voice} and reciprocal\is{reciprocal voice} voices, but also in the anticausative\is{anticausative voice} voice \citep[44]{haspelmath:1990}. In contrast, it has not been possible to find any examples of reflexive-anticausative syncretism of \isi{reflexive origin} among Papunesian languages. In the language sample only three Papunesian languages feature identical voice marking in both the reflexive\is{reflexive voice} and anticausative\is{anticausative voice} voices: the North Halmaheran language \ili{Ternate} (\example{ma-}), the Torricelli language \ili{Yeri} (\example{d-}), and the language isolate \ili{Oksapmin} (\example{t-}). However, there is currently little historical and comparative data available to shed light on the chronology of the different functions of the voice marking in the languages, though \citet[100]{loughnane:2009} very tentatively suggests that the \ili{Oksapmin} prefix \example{t-} may be related to reciprocity\is{reciprocal voice} (which is synchronically marked by the prefix \example{gos-}). As noted in the previous section, the \ili{Ternate} prefix \example{ma-} and the \ili{Yeri} prefix \example{d-} also serve as voice marking in the reciprocal\is{reciprocal voice} voice, while the \ili{Oksapmin} prefix \example{t-} also has an antipassive\is{antipassive voice} function (\sectref{sec:complex-syncretism:antp-refl}).



Voice development from reflexive\is{reflexive voice} to anticausative\is{anticausative voice} has been explained in terms of \isi{semantic bleaching} by \citet[45]{haspelmath:1990} who states that “[t]he anti\-cau\-sa\-tive\is{anticausative voice} use is more general than the reflexive\is{reflexive voice} use in that it is not restricted to clauses with an agentive \isi{subject}, and it is bleached in that the element of self-affecting\is{affectedness} action is absent”. It can further be argued that the \isi{semantic bleaching} probably takes place initially among verbs for which an animate\is{animacy} \isi{semantic participant} is conceivable, as reflexivity\is{reflexive voice} requires a \isi{semantic participant} acting upon itself, e.g. ‘to stretch (oneself)’ and ‘to stand (oneself) up’. Verbs of this kind are commonly called \isi{autocausative} in the literature, yet qualify as anticausative\is{anticausative voice} in this book (\sectref{def:causatives-anticausatives}). Subsequently, the anticausative\is{anticausative voice} function extends to verbs for which an animate\is{animacy} \isi{semantic participant} is inconceivable, e.g. ‘to shatter’ and ‘to split’. Although this diachronic development is generally considered unidirectional\is{diachronic development, unidirectional} (\sectref{diachrony:reflexive}), \citet{inglese:2020} has recently argued that the opposite development might have taken place in the extinct Indo-European language \ili{Hittite} (\sectref{diachrony:antc2refl}).

\subsection{From reflexive to passive} \label{diachrony:refl2pass}
Voice development from reflexive\is{reflexive voice} to passive\is{passive voice} has received much attention in the literature and is widely believed to involve an intermediary anticausative\is{anticausative voice} stage as already noted in \sectref{diachrony:reflexive} (see \tabref{fig:ch7:unidirectional} on page \pageref{fig:ch7:unidirectional}; see also \citealt[44f.]{haspelmath:1990};; \citealt[197f.]{kemmer:1993};; \citealt[253]{heine:kuteva:2002}; \citealt[225f.]{zuniga:kittila:2019}). In fact, \citet[205]{heine:miyashita:2008} argue that “[i]t would seem that there is in fact a universally well-attested evolution from reflexive\is{reflexive voice} (via anticausative\is{anticausative voice} and related functions) to passive\is{passive voice} markers”. This belief essentially entails two developments: from reflexive\is{reflexive voice} to anticausative\is{anticausative voice} and from anticausative\is{anticausative voice} to passive\is{passive voice}. The former development has been discussed in the previous section, while the latter is discussed in \sectref{diachrony:antc2pass}. However, note that such two-step development may give the false impression that the passive\is{passive voice} function evolves only from the anticausative\is{anticausative voice} function separately from the reflexive\is{reflexive voice} function. In fact, voice marking known to have undergone such development generally retains both a reflexive\is{reflexive voice} function and an anticausative\is{anticausative voice} function at the dawn of the passive\is{passive voice} function. Thus, it may be more accurate to describe the voice development under discussion in terms of syncretic reflexive-anticausative voice marking developing a passive\is{passive voice} function. This kind of development has been described most notably for Indo-European languages (see \tabref{tab:ch7:syncretism-ie} on page \pageref{tab:ch7:syncretism-ie}). Another oft-cited case is provided by \citeauthor{heine:kuteva:2002} (\citeyear[44]{heine:kuteva:2002}; \citeyear[110ff.]{heine:kuteva:2007}) from the Ju-Kung language Western !Xun\il{!Xun, Western} (\lang{af}) in which the noun \example{ǀʼé} ‘body’ has undergone a development similar to that attested for Indo-European languages, yet the noun in question has not evolved into an affix for which reason the language is not discussed further here. The same is true for other African languages, including the Central Sudanic language \ili{Ma’di} (cf. \example{rū} ‘body’, \citealt[203f.]{heine:miyashita:2008}) and the Biu-Mandara language \ili{Margi} (cf. \example{kə́r} ‘head’, \citealt[44]{haspelmath:1990}).

In fact, clear examples of voice development from reflexive-anticausative to passive\is{passive voice} involving verbal voice marking in non-Indo-European languages are difficult to obtain, as a lack of diachronic data for most languages blurs the chronological order in which the different functions evolve. For instance, \citet[102]{llanes:al:2017} suggest that the prefix \example{mo-} in the Uto-Aztecan language Huasteca Nahuatl\il{Nahuatl, Huasteca} (\lang{na}) already encountered in the previous two sections “has undergone two fairly widespread pathways of grammaticalization\is{grammaticalisation} from the original reflexive\is{reflexive origin} use: reflexive\is{reflexive voice} > reciprocal\is{reciprocal voice}, and reflexive\is{reflexive voice} > middle > impersonal/passive”.\is{impersonalisation}\is{middle syncretism} The reflexive\is{reflexive voice} function of the prefix does indeed seem to be the oldest (cf. \ili{Proto-Uto-Aztecan} reflexive\is{reflexive voice} \example{*mo-}, \citealt{langacker:1976}), but \citet{llanes:al:2017} provide no evidence for the latter pathway (see \sectref{diachrony:refl2recp} for a discussion of the former). In fact, \citet[102]{llanes:al:2017} admit that “none anticausative\is{anticausative voice} use has been documented in the corpus for the prefix \example{mo-}” (sic), though at least two verbs do actually seem to have an anticausative use (\sectref{sec:complex-syncretism:middle}), yet this function is evidently rare. Likewise, as shown in \sectref{sec:complex-syncretism:middle}, the suffix \example{-yii/-V} in the Tangkic language \ili{Kayardild} (\lang{au}) serves as voice marking in the passive\is{passive voice}, reflexive\is{reflexive voice}, and anticausative\is{anticausative voice} voices, and the reflexive\is{reflexive voice} function of the suffix is likely to be the oldest (\sectref{diachrony:recp2refl}). However, the more precise \isi{diachronic development} of its other functions remains obscure. A few additional languages in the language sample feature voice marking shared by the passive\is{passive voice}, reflexive\is{reflexive voice}, and anticausative\is{anticausative voice} voices for which there are even less historical and comparative data available, e.g. the Tibeto-Burman language \ili{Dhimal} (\lang{ea}) and the language isolate \ili{Sandawe} (\lang{af}). Although a development from reflexive\is{reflexive voice} to passive\is{passive voice} via an anticausative\is{anticausative voice} intermediary stage is plausible for all these languages, alternative development scenarios cannot automatically be ruled out.

The possibility of a development directly from reflexive\is{reflexive voice} to passive\is{passive voice} without an intermediary anticausative\is{anticausative voice} stage has largely been ignored in the literature. Nevertheless, it is worth observing that there are languages in which the reflexive\is{reflexive voice} and passive\is{passive voice} voices are characterised by voice marking for which there appears to be no evidence for (nor traces of) an anticausative\is{anticausative voice} function. For example, \citet[188]{mcfarland:2009} argues that “all verb forms in \example{-kan}” in the Totonacan language Filomeno Mata Totonac\il{Totonac, Filomeno Mata} (\lang{na}) can represent a reflexive\is{reflexive voice} or passive\is{passive voice} voice depending on context, while the suffix in question has no attested anticausative\is{anticausative voice} function. When a passive\is{passive voice} reading is intended, the \isi{agent} cannot be expressed syntactically and the passive\is{passive voice} function is thus more precisely absolute passive\is{passive voice} \citep[188]{mcfarland:2009}. The related languages Upper Necaxa Totonac\il{Totonac, Upper Necaxa}, Coatepec Totonac\il{Totonac, Coatepec}, Tlachichilco Tepehua\il{Tepehua, Tlachichilco}, Huehuetla Tepehua\il{Tepehua, Huehuetla}, and Pisaflores Tepehua\il{Tepehua, Pisaflores} are more or less similar to Filomeno Mata Totonac\il{Totonac, Filomeno Mata} in this respect \citep[22ff.]{beck:nd}. Another very similar example of such reflexive-passive syncretism comes from the Huitotoan language \ili{Bora} (\lang{sa}) in which the suffix \example{-meí} has reflexive\is{reflexive voice} and passive\is{passive voice} functions, but no anticausative\is{anticausative voice} function (\citealt[147f.]{thiesen:weber:2012};; \citealt[1499f.]{seifart:2015}). Both the passive\is{passive voice} and reflexive\is{reflexive voice} functions of the affixes \example{-kan} and \example{-meí} are illustrated in \tabref{tab:ch7:pass-refl-totonacan-bora} (Totonacan = \citealt[22ff.]{beck:nd};; Bora = \citealt[148]{thiesen:weber:2012}).

\begin{table}
	\setlength{\tabcolsep}{3pt}
	\begin{tabularx}{\textwidth}{rlll}
		\lsptoprule
		& & \textsc{refl} & \textsc{pass} \\
		\midrule 
		Fil. Mata Totonac\il{Totonac, Filomeno Mata} & \example{laaqtsin-kan} & ‘to see self’ & ‘to be seen [by sb.]’ \\
		Upper Necaxa Totonac\il{Totonac, Upper Necaxa} & \example{la̰ʔtsín-kan} & ‘to see self’ & ‘to be seen [by sb.]’ \\
		Coatepec Totonac\il{Totonac, Coatepec} & \example{paːškiː-kan} & ‘to love self’ & ‘to be loved [by sb.]’ \\
		Pisaflores Tepehua\il{Tepehua, Pisaflores} & \example{mispaa-kan} & ‘to know self’ & ‘to be known [by sb.]’ \\
		\midrule
		\multirow{2}{*}{\ili{Bora}} & \example{wáhdáhɨ́nú-meí} & ‘to cut self’ & ‘to be cut [by sb.]’ \\
		& \example{dsɨ́jɨvétsá-meí} & ‘to kill self’ & ‘to be killed [by sb.]’ \\
		\lspbottomrule
	\end{tabularx}
	\caption{\textsc{pass}-\textsc{refl} syncretism in Totonacan and Bora}
	\label{tab:ch7:pass-refl-totonacan-bora}
\end{table}

The ultimate origins of the Totonacan suffix \example{-kan} and the \ili{Bora} suffix \example{-meí} remain unknown for the time being, but considering the currently available data on the suffixes it is clear that there is no indication nor evidence for any anticausative\is{anticausative voice} involvement. Observe that the suffix \example{-kan} also indicates a plural possessor on nouns in Filomeno Mata Totonac\il{Totonac, Filomeno Mata}, Coatepec Totonac\il{Totonac, Coatepec}, and Huehuetla Tepehua\il{Tepehua, Huehuetla} \citep[32]{beck:nd}. In Upper Necaxa Totonac\il{Totonac, Upper Necaxa} and Tlachichilco Tepehua\il{Tepehua, Tlachichilco} similar but distinct suffixes are employed for this particular function, \example{-ka̰n} and \example{-kʼan}, respectively. Considering the plural possessive function of the nominal suffix \example{-kan} and the lack of an identifiable \isi{agent} associated with the verbal suffix \example{-kan}, the passive\is{passive voice} function may have developed from a “\isi{generalized-subject construction}” (\citealt[49f.]{haspelmath:1990};; called “\isi{indefinite subject construction}” by \citealt[224f.]{zuniga:kittila:2019}). This is mere speculation, however, and does not readily explain the reflexive\is{reflexive voice} function of the verbal suffix \example{-kan}. The only non-reflexive and non-passive function of the \ili{Bora} suffix \example{-meí} is characterised by an attempt to do something (e.g. \example{tsájtyé-meí} ‘to try to carry sth.’, \example{éjéhtsó-meí} ‘to try to run’, \citealt[1500]{seifart:2015}) which does not shed much additional light on the origin of its passive\is{passive voice} and reflexive\is{reflexive voice} functions.



Additionally, as illustrated in \sectref{sec:complex-syncretism:middle} and briefly discussed in \sectref{diachrony:refl2recp}, in the Nadahup language \ili{Hup} (\lang{sa}) the prefix \example{hup-} derived from the noun \example{hup} ‘human, person’ (cf. the cognate \example{xup} ‘body’ in the related language \ili{Dâw}, \citealt[486]{epps:2008}) serves as voice marking in the passive\is{passive voice}, reflexive\is{reflexive voice}, and reciprocal\is{reciprocal voice} voices. By contrast, there is currently no good evidence for any anticausative\is{anticausative voice} function of the prefix in question. \citet[314, 476]{epps:2008} mentions only one “semi-lexicalized\is{lexicalisation} and/or semi-idiomatic” use of the prefix which bears weak resemblance to an anticausative\is{anticausative voice} function with a single verb, \example{hup-kə́d} ‘to turn’ or ‘to be turned [by sb.]’ (cf. \example{kə́d} ‘to pass sth.’, \example{dʼoʔ-kə́d} or \example{dʼoʔ-hup-kə́d} ‘to turn sth.’). The prefix \example{dʼoʔ-} is a causative\is{causative voice} marker, lit. ‘take’ \citep[518]{epps:2008}. The available data suggest that the reflexive-passive syncretism in \ili{Hup} is of \isi{reflexive origin}, though “further study will shed more light on the processes of grammaticalization\is{grammaticalisation} that led to the present system” \citep[487]{epps:2008}. Alternatively, the passive\is{passive voice} function of the prefix \example{hup-} may have developed through a \isi{generalized-subject construction} (cf. the discussion of the Totonacan languages above) directly from the noun \example{hup} which can have the generic meaning ‘someone’ in some contexts \citep[479]{epps:2008}.

Finally, \citet[253]{heine:kuteva:2002} argue that the singular reflexive\is{reflexive voice} suffix \example{-o/-a} and the plural reflexive\is{reflexive voice} suffix \example{-os/-as} in the Nilotic language \ili{Ateso} (\lang{af}) have developed a passive\is{passive voice} function without mentioning any intermediary anticausative\is{anticausative voice} stage. Nevertheless, the authors of the source which \citeauthor{heine:kuteva:2002} cite, \citet{hilders:lawrance:1956}, provide little evidence for the diachrony of the suffixes, only stating that sometimes the form which they choose to call reflexive\is{reflexive voice} “is preferred” to express passivity\is{passive voice} \citep[57]{hilders:lawrance:1956}. In a more recent grammar of the language, \citet[175ff.]{barasa:2017} demonstrates that the suffix \example{-o/-a} is reciprocal\is{reciprocal voice}, but does not mention any reflexive\is{reflexive voice} function nor the suffix \example{-os/-as}. The passive\is{passive voice} voice marking \example{-oi/-aɪ} \citep[171ff.]{barasa:2017} in the language bears resemblance to the aforementioned reciprocal\is{reciprocal voice} suffix though. 

\subsection{From reflexive to antipassive} \label{diachrony:refl2antp}
Although antipassive-reflexive syncretism is not as well-attested as the patterns of syncretism discussed in the previous sections, the diachrony of such syncretism has attracted increasing attention during the last decades. A \isi{reflexive origin} has repeatedly been proposed for antipassive-reflexive syncretism (e.g. \citealt{terrill:1997, janic:2010, sanso:2017, sanso:2018}), while there is currently no evidence for an opposite development from antipassive\is{antipassive voice} to reflexive\is{reflexive voice}.\is{antipassive origin} \citet{terrill:1997} argues for a \isi{reflexive origin} of antipassive-reflexive syncretism among languages of Australia based on a survey of twelve languages spoken on the continent: the Northern Pama-Nyungan languages \ili{Guugu Yimidhirr}, \ili{Kuku-Yalanji}, \ili{Djabugay}, \ili{Yidiny}, \ili{Dyirbal}, \ili{Nyawaygi}, \ili{Warrungu}, and \ili{Kalkatungu}; the Central Pama-Nyungan language \ili{Diyari}; the Southeastern Pama-Nyungan language \ili{Bandjalang}; and the Gunwinyguan languages \ili{Ngandi} and \ili{Nunggubuyu}. Nevertheless, the purported antipassive\is{antipassive voice} voices mentioned by \citeauthor{terrill:1997} for \ili{Kuku-Yalanji} and \ili{Diyari} are not recognised here due to uncertainty about whether or not they comply with the antipassive\is{antipassive voice} definitions employed in this book (\sectref{sec:simple-syncretism:pass-antp}). This uncertainty also extends to \ili{Guugu Yimidhirr} and \ili{Nyawaygi}. As defined in \sectref{def:passives-antipassives}, an antipassive\is{antipassive voice} voice entails one \isi{semantic participant} that is less likely to be expressed syntactically than other semantic participants\is{semantic participant} (or cannot be syntactically expressed at all) and this \isi{semantic participant} is not an \isi{agent}. However, the antipassive\is{antipassive voice} voice cited by \citeauthor{terrill:1997} for \ili{Guugu Yimidhirr} is defined by \citet[128]{haviland:1979} according to \isi{case} marking alone (by “putting the A NP into S function with the derived verb” and “putting the original O NP into some oblique case”\is{case, oblique}). A similar definition is provided by \citet[496]{dixon:1983} for \ili{Nyawaygi} (“the underlying A NP of the verb now goes into S function, and the underlying O NP now takes dative\is{case, dative} or ergative-instrumental\is{case, ergative}\is{case, instrumental} inflection”). Moreover, the purported antipassivity\is{antipassive voice} in \ili{Ngandi} is also not acknowledged in this book due to unproductivity.\is{productivity} \citet{heath:1978} argues that the suffix \example{-i} discussed by \citet{terrill:1997} only can have a function which “indicates indefinite or unspecified object” with a single verb in \ili{Ngandi} (\example{ḍaː-bu-} ‘to test, taste, try sth.’ ↔ \example{ḍaː-b-i-} ‘to try [sth.], make an effort’, \citealt[92]{heath:1978}).

The antipassive\is{antipassive voice} voices cited by \citet{terrill:1997} for the remaining seven languages are acknowledged here and antipassive-reflexive syncretism is acknowledged for six of these languages (\ili{Yidiny} \example{-:dji}, \ili{Djabugay} \example{-yi}, \ili{Dyirbal} \example{-yi} or \example{-yirri}, \ili{Warrungu} \example{-li} or \example{-gali}, \ili{Bandjalang} \example{-li}, and \ili{Nunggubuyu} \example{-i}). In Kalkatungu the antipassive\is{antipassive voice} suffix \example{-yi} differs from the reflexive\is{reflexive voice} suffix \example{-ti} (at least synchronically), for which reason this language is not discussed further here. \citet[78]{terrill:1997} ultimately argues that the various suffixes are cognates derived from some ancestral proto-suffix \example{*-dhirri-yi} \citep{dixon:1980} or \example{*-dharri} \citep{dixon:2002}. This \isi{reconstruction} is highly tentative, however, and the precise development of its functions is no more certain than the reconstructed\is{reconstruction} form itself \citep[119f.]{mcgregor:2013}. By contrast, it appears that a reflexive\is{reflexive voice} suffix \example{*-yi} can be reconstructed\is{reconstruction} rather reliably for \ili{Proto-Gunwinyguan} (\sectref{diachrony:recp2refl}), which points to a \isi{reflexive origin} for the syncretism in Nunggubuyu. This presumed development in \ili{Nunggubuyu} is illustrated in \tabref{tab:ch7:refl-antp-nunggubuyu} \citep[390]{heath:1984}. It has so far not been possible to find examples of similar antipassive-reflexive syncretism in other Gunwinyguan languages. Note that that the suffix \example{-i} in \ili{Nunggubuyu} also has an anticausative\is{anticausative voice} function (\sectref{sec:complex-syncretism:antp-refl}) and the order in which this and the antipassive\is{antipassive voice} function evolved is uncertain. The meaning of the verb \example{yimunydharm-i-} is more precisely ‘to track [sth.] by smell’.

\begin{table}
	\setlength{\tabcolsep}{3pt}
	\begin{tabularx}{\textwidth}{rcllll}
		\lsptoprule
		P.-Gunwinyg.\il{Proto-Gunwinyguan} & \example{*-yi} & \textsc{refl} & \multicolumn{1}{r}{→} & \textsc{antp} & \\
		\midrule 
		\multirow{2}{*}{\ili{Nunggubuyu}} & \multirow{2}{*}{\example{-i}} & \example{n-i-} & ‘to see self’ & \example{yaḻgiw-i-} & ‘to pass [sth.]’ \\
		& & \example{balh-i-} & ‘to cut self up’ & \example{yimunydharm-i-} & ‘to track [sth.]’ \\
		\lspbottomrule
	\end{tabularx}
	\caption{\textsc{antp-refl} syncretism of \textsc{refl} origin in Nunggubuyu}
	\label{tab:ch7:refl-antp-nunggubuyu}
\end{table}

\citet[159]{janic:2010} argues that antipassive-reflexive syncretism “developed from reflexivity\is{reflexive voice} through functional extension” on the basis of data from ten geographically diverse languages: the Slavic languages \ili{Bulgarian} and \ili{Polish}, the Kartvelian language \ili{Laz}, the Northern Chukotko-Kamchatkan language \ili{Chukchi} (all four \lang{ea}), the Cariban language \ili{Ye’kwana}, the Tacanan languages \ili{Cavineña} and \ili{Ese Ejja} (all three \lang{sa}), the Western Mande language \ili{Bambara} (\lang{af}), and the Northern Pama-Nyungan languages \ili{Warrungu} and \ili{Yidiny} (both \lang{au}). The latter two languages have already been discussed above, while it was noted in \sectref{sec:simple-syncretism:appl-refl} that the purported antipassive-reflexive syncretism in \ili{Laz}, \ili{Bulgarian}, \ili{Polish} and \ili{Bambara} is not acknowledged in this book due to lack of verbal voice marking. However, as mentioned in the same section, antipassive-reflexive syncretism has been observed by \citet{creissels:2012, creissels:2015} in another Western Mande language, \ili{Soninke}, characterised by the suffix \example{-i}. \citet[13]{creissels:2015} also notes that a similar suffix can be found in the closely related \ili{Bobo} and \ili{Bozo} languages and that \example{í} is “attested in several West Mande languages as a reflexive pronoun”. Based on these observations, \citet[13]{creissels:2015} goes on to reconstruct\is{reconstruction} a reflexive\is{reflexive voice} suffix \example{*-i} for \ili{Proto-West-Mande}, yet admits that there is “a serious problem with this hypothesis”: a \isi{grammaticalisation} of \example{í} as a suffix would seem to entail an original \textsc{svo(x)} \isi{word order}, but “all Mande languages invariably show a rigid \textsc{sovx} constituent order, which consequently must be reconstructed\is{reconstruction} at Proto-Mande level”. While the Western Mande \isi{genus} remains a potential candidate for antipassive-reflexive syncretism of \isi{reflexive origin}, \citeauthor{creissels:2015} prefers to leave the question open and the same goes for this book.

The origin of antipassive-reflexive syncretism is also uncertain for the Tacanan and Northern Chukotko-Kamchatkan languages. \citet[525]{vuillermet:2012} explicitly discusses the origin of the syncretism among Tacanan languages, remarking that the antipassive-reflexive-reciprocal-anticausative circumfix \example{xa-…-ki} in \ili{Ese Ejja} and the antipassive-reflexive-reciprocal circumfix \example{k(a)-…-ti} in \ili{Cavineña} perhaps come “from a primary reflexive\is{reflexive voice} function” (cf. the reflexive-reciprocal suffix \example{-ti} in the closely related language \ili{Araona}, \citealt[555ff.]{emkow:2006}). However, more comparative research is needed to clarify and determine the more precise chronology. The antipassive-reflexive syncretism in the Northern Chukotko-Kamchatkan language \ili{Chukchi} is characterised by the suffix \example{-tku/-tko} conditioned by vowel harmony, and \citet[167]{janic:2010} admits that the suffix is “not related to reflexivity\is{reflexive voice} but to reciprocity\is{reciprocal voice}”. Interestingly, \citet[423]{fortescue:2005} proposes that the suffix in question descends from \ili{Proto-Chukotko-Kamchatkan} \example{*-tku} denoting “frequent or protracted action” (cf. \isi{frequentative} \ili{Alutor} \example{-tku}, \ili{Koryak} \example{-tku}, and \ili{Kerek} \example{-ttu}). Thus, there is little evidence for a \isi{reflexive origin}. By contrast, it is well-known that antipassivity\is{antipassive voice} is commonly related to reciprocity\is{reciprocal voice} (\sectref{sec:complex-syncretism:antp-refl}) and \isi{aspect} (\citealt{polinsky:2017}).

Cariban languages seem to be better candidates for antipassive-reflexive syncretism of \isi{reflexive origin}. \citet[217ff.]{meira:2000} states that a “detransitivizing prefix”\is{detransitivisation} is “found in every Cariban language” and can have a wide range of uses, including passive\is{passive voice}, antipassive\is{antipassive voice}, reflexive\is{reflexive voice}, reciprocal\is{reciprocal voice}, and anticausative\is{anticausative voice} functions. The prefix mentioned by \citeauthor{meira:2000} has subsequently been reconstructed\is{reconstruction} for \ili{Proto-Carib} as two distinct prefixes, \example{*(w)e-} and \example{*(w)ôte-}, which \citet[512]{meira:al:2010} and \citet[9]{gildea:2015} regard as reflexive\is{reflexive voice} and reciprocal\is{reciprocal voice}, respectively. While seemingly homogeneous in \ili{Proto-Carib}, at least 25 reflexes of the prefixes are attested among descendant languages many of which feature four or more different variants \citep[506]{meira:al:2010}. Consider, for instance, \ili{Tiriyó} \example{ə-}, \example{əəs-}, \example{e-}, \example{əl-}, \example{ət-}, \example{et-}; \ili{Wayna} \example{ət-}, \example{əh-}, \example{ə-}, \example{e-}; \ili{Kari’ña} \example{(w)ot-}, \example{os-}, \example{o(ʔ)-}, \example{e-}; and \ili{Apalaí} \example{ot-}, \example{os-}, \example{at-}, \example{o-}, \example{e-} \citep[217f.]{meira:2000}. While such variant forms are “mostly phonologically conditioned”, they sometimes involve \isi{suppletion} \citep[2]{gildea:al:2016} or “appear to be lexically conditioned” \citep[217]{meira:2000}. Despite the “complicated and idiosyncratic allomorphic patterns”\is{allomorphy} \citep[217]{meira:2000}, the variant forms are commonly treated as a single prefix synchronically which can make it difficult to determine the precise patterns of voice syncretism in the individual languages. Nevertheless, some prefixes do indeed have both antipassive\is{antipassive voice} and reflexive\is{reflexive voice} functions, including the prefix \example{e-} derived from \ili{Proto-Carib} \example{*(w)e-} \citep[511]{meira:al:2010}, which points to a \isi{reflexive origin} of the syncretism. This presumed development is illustrated in \tabref{tab:ch7:refl-antp-cariban} \citep{meira:2000}. It is worth observing, however, that the antipassive\is{antipassive voice} function of the prefix \example{e-} seems to be rather widespread among the Cariban languages, and the \isi{diachronic development} of the various functions of the \ili{Proto-Carib} prefixes \example{*(w)e-} and \example{*(w)ôte-} remains understudied.

\begin{table}
	\setlength{\tabcolsep}{3.5pt}
	\begin{tabularx}{\textwidth}{rclllll}
		\lsptoprule
		\ili{Proto-Carib} & \example{*(w)e-} & \textsc{refl} & &  → & \textsc{antp} & \\
		\midrule 
		\ili{Kari’ña} & \example{e-} & \example{e-kuupi} & ‘to bathe self’ & & \example{e-sapima} & ‘to play [sth.]’ \\
		\ili{Tiriyó} & \example{e-} & \example{e-suka} & ‘to wash self’ & & \example{e-puuka} & ‘to bewitch [sb.]’ \\
		\ili{Makushi} & \example{e-} & \example{e-roma} & ‘to wash self’ & & \example{e-name} & ‘to fear [sth.]’ \\
		\lspbottomrule
	\end{tabularx}
	\caption{\textsc{antp}-\textsc{refl} syncretism of \textsc{refl} origin in Cariban languages}
	\label{tab:ch7:refl-antp-cariban}
\end{table}

Recently, \citet{sanso:2017} has argued for antipassive-reflexive syncretism of reflexive\is{reflexive origin} or \isi{reciprocal origin} in 20 languages representing sixteen different genera\is{genus}. \citeauthor{sanso:2017} also mentions three additional languages, Gumuz and the Bantu language Eton (both \lang{af}) as well as the Oceanic language Chamorro (\lang{pn}). However, it is unclear if Gumuz features productive affixal antipassive voice marking (only a single example is provided by \citealt[194f.]{ahland:2012} in her grammar of the language), and the latter two languages do not feature affixal verbal antipassive-reflexive syncretism. Among the other 20 genera, five have already been mentioned above: Slavic, Northern Pama-Nyungan, Central Pama-Nyungan, Kartvelian, and Ta\-ca\-nan. \citet[193ff., 203]{sanso:2017} explicitly describes the syncretism in five of the remaining eleven genera\is{genus}, while he lists the last six genera\is{genus} in a table without further comments. In any case, there is only relatively clear evidence for a \isi{reflexive origin} in one of the five genera\is{genus} explicitly discussed by \citeauthor{sanso:2017}, Turkic. \citeauthor{sanso:2017} explicitly mentions antipassive-reflexive syncretism characterised by the suffix \example{-š} in \ili{Tatar} of \isi{reciprocal origin} (\sectref{sec:complex-syncretism:antp-refl}) and by the suffix \example{-n} in \ili{Tuvan} of \isi{reflexive origin}. As already briefly mentioned in \sectref{sec:complex-syncretism:pass-antp}, the suffix \example{-n} is probably diachronically “connected to the possessive form \example{an} of the [third person] pronoun \example{ol}” \citep[225]{salo:2013} and plausibly grammaticalised\is{grammaticalisation} into a reflexive\is{reflexive voice} suffix which developed an antipassive\is{antipassive voice} function. This development is illustrated in \tabref{tab:ch7:refl-antp-turkic} (Tuvan = \citealt[1173]{kuular:2007}; Tatar = \citealt[484f.]{burbiel:2018}). The other four genera\is{genus} explicitly addressed by \citet{sanso:2017} are discussed below.

\begin{table}
	\setlength{\tabcolsep}{3pt}
	\begin{tabularx}{\textwidth}{rclllll}
		\lsptoprule
		Common Turkic\il{Turkic, Common} & \example{*-n} & \textsc{refl} & & → & \textsc{antp} & \\
		\midrule 
		\ili{Tuvan} & \example{-n} & \example{savaηna-n-} & ‘to soap self’ & & \example{daara-n-} & ‘to sew [sth.]’ \\
		\ili{Tatar} & \example{-n} & \example{sört-en-} & ‘to dry self’ & & \example{teg-en} & ‘to sew [sth.]’ \\
		\lspbottomrule
	\end{tabularx}
	\caption{\textsc{antp}-\textsc{refl} syncretism of \textsc{refl} origin in Turkic languages}
	\label{tab:ch7:refl-antp-turkic}
\end{table}

In contrast to the Turkic case, \citet[100]{loughnane:2009} tentatively speculates that the antipassive-reflexive-anticausative prefix \example{t-} in the language isolate \ili{Oksapmin} (\lang{pn}) may historically be related to reciprocity\is{reciprocal voice}, not reflexivity\is{reflexive voice}, as already briefly mentioned in \sectref{diachrony:refl2antc}. In turn, \citet{bryant:1999} does not seem to address the origin of antipassive-reflexive syncretism at all in his grammar of the Surmic language \ili{Tirmaga} (\lang{af}). Furthermore, the so-called \example{d}-classifier characterising anti\-pas\-sive-re\-flex\-ive syncretism in the Na-Dene language \ili{Tlingit} (\lang{na}) described by \citet[193f.]{sanso:2017} seems to have neither a reflexive\is{reflexive origin} nor \isi{reciprocal origin}, and the same is true for the related language Eyak also included in \citeauthor{sanso:2017}’s study. \citet[374f.]{thompson:1996} argues that both the antipassive\is{antipassive voice} and reflexive\is{reflexive voice} functions of the classifier have evolved independently from a generalised function denoting a “suppressed \isi{patient}”. Finally, the purported \isi{reflexive origin} mentioned by \citet{sanso:2017} for the antipassive-reflexive suffix \example{-m} in the Central Salish language Chilliwack \ili{Halkomelem} (\lang{na}) seems to be supported by \citet[75ff.]{zahir:2018}. Nevertheless, observe that the suffix has an antipassive\is{antipassive voice} function in all the Salishan languages surveyed by \citeauthor{zahir:2018}, while its reflexive\is{reflexive voice} function is “not prototypical”\is{prototype} \citep[77]{zahir:2018}. The more precise chronology of the functions is uncertain. It is clear from \citeauthor{zahir:2018}’s discussions of the suffix \example{-m} that he presupposes a \isi{reflexive origin} and \isi{diachronic development} similar to that famously described by \citet{kemmer:1993} for Indo-European languages (\sectref{diachrony:reflexive}).

The remaining six languages and accompanying genera\is{genus} included but not explicitly discussed in \citeauthor{sanso:2017}’s (\citeyear{sanso:2017}) study are the Oceanic language \ili{Neverver} (\lang{pn}), the Nilotic language \ili{Luwo} (\lang{af}), the Mangarrayi-Maran language \ili{Mangarrayi} (\lang{au}), the Oto-Manguean language San Ildefonso Tultepec Otomí\il{Otomí, San Ildefonso Tultepec}, the Southern Iroquoian language \ili{Cherokee}, and the Northern Iroquoian language \ili{Seneca} (all three \lang{na}). As in the case of \ili{Tirmaga} mentioned further above, \citet{barbor:2012} does not address the diachrony of antipassive-reflexive syncretism in \ili{Neverver}, and neither does \citet{storch:2014} with regard to \ili{Luwo}. By contrast, \citet[157ff.]{palancar:2009} explicitly argues against a \isi{reflexive origin} for the prefix \example{n-} associated with antipassive\is{antipassive voice}, reflexive\is{reflexive voice} and other voices in San Ildefonso Tultepec Otomí\il{Otomí, San Ildefonso Tultepec} (compare the cognate prefix in Acazulco Otomí\il{Otomí, Acazulco} illustrated in \sectref{sec:complex-syncretism:antp-refl}). By contrast, there is some vague evidence indicating that the suffix \example{-(ñ)jiy(i)} in the Mangarrayi-Maran language \ili{Mangarrayi} (\lang{au}) with reflexive\is{reflexive voice}, reciprocal\is{reciprocal voice} and marginal antipassive\is{antipassive voice} functions perhaps is historically composed of a reciprocal\is{reciprocal voice} suffix \example{*-nci} and a reflexive\is{reflexive voice} suffix \example{*-yi} (\sectref{diachrony:recp2refl}). \citet{julian:2010} reconstructs\is{reconstruction} a reflexive\is{reflexive voice} prefix \example{*ataːt-} for \ili{Proto-Iroquoian} which points to a \isi{reflexive origin} for antipassive-reflexive syncretism characterised by cognates of the prefix in \ili{Cherokee} and \ili{Seneca}. This presumed development is illustrated for the former language in \tabref{tab:ch7:refl-antp-cherokee} \citep[343ff., 366, 371]{montgomery-anderson:2008}, but observe that the development is somewhat tentative. \citet{julian:2010} does not once consider antipassivity\is{antipassive voice} nor similar functions of the prefix in descendant languages, and it is therefore not entirely clear if this function has been overlooked in the \isi{reconstruction} of the \ili{Proto-Iroquoian} prefix or not.

\begin{table}
	\setlength{\tabcolsep}{1.8pt}
	\begin{tabularx}{\textwidth}{rcllll}
		\lsptoprule
		P.-Ir.\il{Proto-Iroquoian} & \example{*ataːt-} & \textsc{refl} & \multicolumn{1}{r}{→} & \textsc{antp} & \\
		\midrule 
		\multirow{2}{*}{Cher.}\il{Cherokee} & \multirow{2}{*}{\example{ataa(t)-}} & \example{ataa-kohwthíha-} & ‘to see self’ & \example{ataa-stehlt-} & ‘to help [sb.]’ \\
		& & \multicolumn{2}{l}{\example{ataat-olihka-} ‘to recognise self’} & \example{ataat-olihka-} & ‘to recognise [sth.]’ \\
		\lspbottomrule
	\end{tabularx}
	\caption{\textsc{antp}-\textsc{refl} syncretism of \textsc{refl} origin in Cherokee}
	\label{tab:ch7:refl-antp-cherokee}
\end{table}



In terms of functional explanations for antipassive-reflexive syncretism of \isi{reflexive origin}, \citet[79]{terrill:1997} argues that “[i]t seems possible that the antipassive\is{antipassive voice} constructions developed from reflexive\is{reflexive voice} constructions, by extending the pragmatic function of reflexives\is{reflexive voice}” because “reflexives\is{reflexive voice} and antipassives\is{antipassive voice} have very similar semantic/pragmatic functions”. Consequently, “it is a short functional step from a canonical\is{canonicity} reflexive\is{reflexive voice} function to a canonical\is{canonicity} antipassive\is{antipassive voice} function” \citep[79]{terrill:1997}. According to \citet[80ff.]{terrill:1997}, both reflexives\is{reflexive voice} and antipassives\is{antipassive voice} are more specifically characterised by i) low \isi{agency}, ii) low \isi{transitivity}, and iii) ‘non-distinct’ objects. This explanation is largely adopted by \citet{janic:2010, janic:2016}, while \citet[206]{sanso:2017} argues against it on the grounds that functional similarity “is an elusive concept if we are not able to figure out a hypothetical context in which there may be ambiguity between the source and the target constructions”. In the spirit of \citet{creissels:nouguier-voisin:2008} and \citet{bostoen:al:2015}, \citet[12]{sanso:2018} instead hypothesises that “the reinterpretation path leading to the extension of reflexive/reciprocal/middle\is{middle syncretism} markers to antipassive\is{antipassive voice} situations starts from a very specific bridgehead,\is{bridging context} namely, reciprocally marked comitative/sociative\is{comitativity}\is{sociativity} constructions”. This scenario is visualised in \figref{fig:ch7:antp-refl}. A development from reflexive\is{reflexive voice} to antipassive\is{antipassive voice} would thus entail a development from reflexive\is{reflexive voice} to reciprocal\is{reciprocal voice} (\sectref{diachrony:refl2recp}) and from reciprocal\is{reciprocal voice} to antipassive\is{antipassive voice} (\sectref{diachrony:recp2antp}).

\begin{figure}
	\caption{Antipassive-reflexive syncretism of reflexive origin}
	\label{fig:ch7:antp-refl}
	\begin{tabularx}{.90\textwidth}{llll}
		\multicolumn{4}{l}{\textsc{a} \& \textsc{b} hit each other (pure reciprocal)} \\
		→ & \multicolumn{3}{l}{\textsc{a} \& \textsc{b} cooperate in hitting / hit together (sociative/comitative)} \\
		& → & \multicolumn{2}{l}{\textsc{a} \& \textsc{b} hit [sb.] (antipassive, plural \isi{agent})} \\
		& & → & \textsc{a} hits [sb.] (antipassive, singular \isi{agent}) \hfill \citep{sanso:2018} \\
	\end{tabularx}
\end{figure}

\citeauthor{sanso:2017}’s (\citeyear{sanso:2017}, \citeyear{sanso:2018}) scenario represents a plausible explanation for the rise of antipassive-reflexive syncretism in languages in which the antipassive-reflexive marking also has a reciprocal\is{reciprocal voice} function. However, it does not explain the development of the syncretism in languages in which the antipassive-reflexive marking does not have a reciprocal\is{reciprocal voice} function like in Nunggubuyu and Tatar. Unless the reciprocal\is{reciprocal voice} function has simply fallen out of use, a more general explanation like the one proposed by \citet{terrill:1997} and \citet{janic:2010, janic:2016} might be a better alternative for such languages, if a hypothetical context or scenario in which the development might have taken place can be found. \citet[83]{terrill:1997} mentions in passing that the verb ‘to cover’ in Yidiny (\lang{au}) can be found in both the antipassive\is{antipassive voice} and reflexive\is{reflexive voice} voices with the same voice marking. If one focuses on the non-distinctiveness characterising reflexives\is{reflexive voice} and antipassives\is{antipassive voice} mentioned further above, it could be hypothesised that a reflexive\is{reflexive voice} meaning of a verb like ‘to cover’ could come to be used first in relation to a distinct part of the body and later more vaguely with regard to some non-distinct part of the body whence an antipassive\is{antipassive voice} function could evolve, e.g. ‘to cover (all of) oneself’ → ‘to cover distinct part of one’s own body’ → ‘to cover non-distinct part of one’s own body’ → ‘to cover [something non-distinct]’.

\section{Reciprocal origin} \label{diachrony:reciprocal}
As demonstrated in the previous sections, voice syncretism of \isi{reflexive origin} is well-known and rather well-attested among the languages of the world. In comparison, the prospect of a \isi{reciprocal origin} for both individual voices and voice syncretism has received relatively little attention in the literature, although the possibility of such development has been acknowledged sporadically for decades (e.g. \citealt[200]{kemmer:1993}). Nevertheless, growing evidence indicates that a \isi{reciprocal origin} (or at least a partially \isi{reciprocal origin}) may be more widespread than previously thought. Plausible cases of such development are discussed and illustrated in the following sections. It is important to stress here that it can be difficult to discern a purely \isi{reciprocal origin} for voice syncretism in many languages, for which reason a partially \isi{reciprocal origin} is mentioned in parentheses above. Indeed, in many of the languages treated in the following sections the purported original reciprocal\is{reciprocal origin} function of a given voice marker likely existed alongside various more or less semantically similar functions related to \isi{sociativity} (‘to \textsc{verb} together’), iterativity\is{iterative} (‘to \textsc{verb} iteratively’),\is{iterative} intensity (‘to \textsc{verb} intensely’), and/or habituality\is{habitual} (‘to \textsc{verb} habitually’).\is{habitual} In the spirit of \citet{lichtenberk:1985, lichtenberk:2000}, these functions are subsumed under the notion \textsc{plurality of relations}\is{plurality of relations} (\textsc{por}). This notion can further be divided into \textsc{plurality of participants}\is{plurality of participants} underlying functions in which semantic participants\is{semantic participant} act plurally in one way or another (e.g. \isi{sociativity}), and \textsc{plurality of actions}\is{plurality of actions} underlying functions in which an action is done plurally (e.g. iterativity).\is{iterative} The notion \isi{plurality of participants} and thereby \isi{plurality of relations} also underlie reciprocity\is{reciprocal voice}. Thus, the patterns of voice syncretism mentioned in the next sections do not necessarily all have an exclusively \isi{reciprocal origin} in all languages, but the voice marking in these patterns has a documented or reconstructible\is{reconstruction} reciprocal\is{reciprocal voice} function which likely evolved before additional voice functions of interest. For the sake of convenience, the voice syncretism will accordingly be described as having a \isi{reciprocal origin}.

\subsection{From reciprocal to reflexive} \label{diachrony:recp2refl}
As noted in \sectref{diachrony:reflexive}, it is widely believed that reflexive\is{reflexive voice} marking can develop a reciprocal\is{reciprocal voice} function and that reciprocal\is{reciprocal voice} marking cannot develop a reflexive\is{reflexive voice} function. As argued very explicitly by \citet[216]{heine:miyashita:2008}, “reciprocals\is{reciprocal voice} do not seem to grammaticalize\is{grammaticalisation} into reflexives\is{reflexive voice}”. A \isi{diachronic development} from reflexive\is{reflexive voice} to reciprocal\is{reflexive origin} is indeed well-attested cross-linguistically (\sectref{diachrony:refl2recp}), yet the opposite development\is{reciprocal origin} does seem to have taken place in a number of geographically diverse languages and genera\is{genus}. For instance, reflexes of the \ili{Proto-Oceanic} prefix \example{*pa\textsc{r}i-} \citep[150ff.]{pawley:1973} in descendant languages (\lang{pn}) have a wide range of functions related to the notion of \isi{plurality of relations} discussed in the previous section, including reciprocity\is{reciprocal voice} \citep{lichtenberk:2000}. For an overview of the various functions, see \citet[28]{bril:2005}. In contrast, a reflexive\is{reflexive voice} function is rare among the reflexes, and \citet[32]{lichtenberk:2000} argues that “there are no grounds for postulating a reflexive-marking function” for the prefix \example{*pa\textsc{r}i-} in the proto-language (see also \citeyear[181]{lichtenberk:1991}). This opinion is shared by \citet[32]{bril:2005} and \citeauthor{moyse-faurie:2008} (\citeyear[106]{moyse-faurie:2008}; \citeyear[108]{moyse-faurie:2017}). Interestingly, however, a reflexive\is{reflexive voice}\is{reciprocal origin} function has evolved as an innovation in a few descendant languages, most notably in “[l]anguages spoken in the Hienghene area (\ili{Nemi}, \ili{Fwâi}, \ili{Pije}, \ili{Jawe}) of the New Caledonian Mainland, as well as \ili{Cèmuhî} and at least some of the Voh-Koné dialects (Centre of the Mainland, such as \ili{Hmwaveke})” \citep[122]{moyse-faurie:2017}. For example, in the \ili{Hmwaveke} language mentioned by \citeauthor{moyse-faurie:2017} the prefix \example{ve-} derived from \ili{Proto-Oceanic} \example{*pa\textsc{r}i-} has an unambiguous reflexive\is{reflexive voice} function in the singular, while both reflexive\is{reflexive voice} and reciprocal\is{reciprocal voice} interpretations are possible in the dual and plural. 

\citet[110]{moyse-faurie:2017} argues that the phenomenon described above is otherwise “very rare in Oceanic languages”, though it can here be added that the Loyalty Islands language \ili{Drehu} and the Polynesian language East Futunan\il{Futunan, East} appear to have undergone a similar development, albeit on a much smaller scale. In \ili{Drehu} the prefix \example{i-} has reciprocal\is{reciprocal voice} and other functions related to \isi{plurality of relations} as well as an antipassive\is{antipassive voice} function (\citealt[35ff.]{bril:2005}; \sectref{diachrony:recp2antc}) in addition to a reflexive\is{reflexive voice} use “with a few verbs of \isi{grooming}” (\citealt[34]{bril:2005}. In East Futunan\il{Futunan, East} the prefix \example{fe-} has functions related to \isi{plurality of relations} and can also “mark a reciprocal\is{reciprocal voice} involving no more than two participants” with “a dozen of verbs”, while a reflexive\is{reflexive voice} function is “limited to a few verbs designating actions performed on one’s own body” \citep[1520ff.]{moyse-faurie:2007}. The prefixes in these languages also represent reflexes of \ili{Proto-Oceanic} \example{*pa\textsc{r}i-}. The presumed \isi{diachronic development} is shown in \tabref{tab:ch7:recp-recp-oceanic} (Hmwaveke = \citealt[123]{moyse-faurie:2008}; Drehu = \citealt[35, 38]{bril:2005}; East Futunan = \citealt[1520ff.]{moyse-faurie:2007}). As explicitly indicated in the table, the reflexive\is{reflexive voice} function of prefixes did not necessarily evolve directly nor exclusively from a reciprocal\is{reciprocal voice} function because the prefix likely had other functions related to \isi{plurality of relations} as well. However, the evidence presented here clearly shows that the reciprocal\is{reciprocal voice} function developed prior to the reflexive\is{reflexive voice} function. 

\begin{table}
	\setlength{\tabcolsep}{3.7pt}
	\begin{tabularx}{\textwidth}{rclllll}
		\lsptoprule
		\ili{Proto-Oceanic} & \example{*pa\textsc{r}i-} & \multicolumn{2}{l}{\textsc{recp} (+ \textsc{por})} & → & \textsc{refl} & \\
		\midrule 
		\ili{Hmwaveke} & \example{ve-} & \example{ve-caina} & ‘to know e.o.’ & & \example{ve-ibi} & ‘to pinch self’ \\
		\ili{Drehu} & \example{i-} & \example{i-atre} & ‘to know e.o.’ & & \example{i-sej} & ‘to comb self’ \\
		East Futunan\il{Futunan, East} & \example{fe-} & \example{fe-tuli} & ‘to chase e.o.’ & & \example{fe-ʼumo} & ‘to pinch self’ \\
		\lspbottomrule
	\end{tabularx}
	\caption{\textsc{refl}-\textsc{recp} syncretism of \textsc{recp} origin in Oceanic languages}
	\label{tab:ch7:recp-recp-oceanic}
\end{table}

There are vague hints of similar reflexive-reciprocal syncretism in some languages of South New Caledonia. For instance, \citet[39]{bril:2005} argues that in \ili{Ajië} “the middle\is{middle syncretism} prefix \example{vi-} has reciprocal\is{reciprocal voice}, reflexive\is{reflexive voice}, or collective meanings” but only provides one example of the reflexive\is{reflexive voice} use (\example{na vi-jiwé} ‘he kills himself’) and it has not been possible to obtain additional data on the language. \citet[1047]{moyse-faurie:2015} notes that in \ili{Xârâcùù} the prefix \example{-ù} “only derives a dozen verbs”,\is{derivation} two of which appear to qualify as reflexive\is{reflexive voice} (\example{cù} ‘to comb sth.’ ↔ \example{ù-cù} ‘to comb self’ and \example{mwé} ‘to put sth. into water’ ↔ \example{ù-mwé} ‘to take a bath’, i.e. ‘to put self into water’) whereas two other verbs may be regarded as reciprocal\is{reciprocal voice} with a little good will (\example{xâpârî} ‘to see sb.’ ↔ \example{ù-xâpârî} ‘to meet’ and \example{juu} ‘to agree to sth.’ ↔ \example{ù-juu} ‘to come to an agreement’, i.e. \textsuperscript{?}‘to agree to e.o.’). Both \ili{Ajië} \example{vi-} and \ili{Xârâcùù} \example{ù-} are derived from \ili{Proto-Oceanic} \example{*pa\textsc{r}i-} as well.

An affix associated with reciprocity\is{reciprocal voice} has also been reconstructed\is{reconstruction} for \ili{Proto-Arawakan}. According to \citet[109f.]{wise:1990}, the suffix \example{*-kʰakʰ} in this proto-language likely had a reciprocal\is{reciprocal voice} function because “that is its meaning in a wide range of [descendant Arawakan] languages” while “[i]n others the meaning is ‘comitative’\is{comitativity} which is clearly semantically related to ‘reciprocal\is{reciprocal voice}’”. \citeauthor{wise:1990}’s description suggests that the functions of the \ili{Proto-Arawakan} suffix \example{*-kʰakʰ} perhaps relate to \isi{plurality of relations} more generally, like in the case of the \ili{Proto-Oceanic} prefix \example{*pa\textsc{r}i-} discussed above. In the Inland Northern Arawakan language \ili{Tariana} the reflex \example{-kaka} has retained its reciprocal\is{reciprocal voice} use, but the comitative\is{comitativity} function mentioned by \citeauthor{wise:1990} has become almost obsolete and is retained only in “older people’s speech” \citep[264]{aikhenvald:2003}. In addition to its reciprocal\is{reciprocal voice} function, the suffix in questions appears to have developed a marginal reflexive\is{reflexive voice}\is{reciprocal origin} function found with three verbs: \example{pisu} ‘to cut sb.’, \example{inu} ‘to kill sb.’, and \example{ña} ‘to hit sb.’ (\citealt[266f.]{aikhenvald:2003}; \citeyear[1357]{aikhenvald:2007b}). The presumed development of the reflexive\is{reflexive voice} function from the reciprocal\is{reciprocal voice} function is illustrated in \tabref{tab:ch7:recp-refl-tariana}. In her discussion of the \ili{Tariana} suffix \example{-kaka}, \citet[1357]{aikhenvald:2007b} states that “[a]ll North-Arawak languages of the Upper Rio Negro use the same verbal suffix for reciprocals\is{reciprocal voice} and reflexives\is{reflexive voice}”. However, here \citet[847]{aikhenvald:2007b} does not refer to the same verbal suffix as in \ili{Tariana} (nor cognate suffixes) but to the fact that each of the languages of the Upper Rio Negro possesses a suffix which is used in both the reflexive\is{reflexive voice} and reciprocal\is{reciprocal voice} voices: \example{-na} in \ili{Warekena}, \example{-tini} in \ili{Bare}, and \example{-wa} in \ili{Baniwa}. \citet[104]{wise:1990} observes that the \ili{Yucuna} suffix \example{-čaka} seemingly reflecting \ili{Proto-Arawakan} \example{*-kʰakʰ} has both reflexive\is{reflexive voice} and reciprocal\is{reciprocal voice} functions, but it has not been possible to confirm this claim due to lack of data on the language.

\begin{table}
	\setlength{\tabcolsep}{6.3pt}
	\begin{tabularx}{\textwidth}{rcllll}
		\lsptoprule
		\ili{Proto-Arawakan} & \example{*-kʰakʰ} & & \textsc{recp} (+ \textsc{por}?) & → & \textsc{refl} \\
		\midrule 
		\ili{Tariana} & \example{-kaka} & \example{inu-kaka} & ‘to kill e.o.’ & & ‘to kill self’ \\
		\lspbottomrule
	\end{tabularx}
	\caption{\textsc{refl}-\textsc{recp} syncretism of \textsc{recp} origin in Tariana}
	\label{tab:ch7:recp-refl-tariana}
\end{table}

\citet[535]{jensen:1998} reconstructs\is{reconstruction} both a reflexive\is{reflexive voice} prefix \example{*je-} and a reciprocal\is{reciprocal voice} prefix \example{*jo-} for another South American \isi{genus}, Tupi-Guaraní. \citeauthor{jensen:1998} does not mention any additional functions of the latter prefix, but it is not unlikely that it may have had functions related to \isi{plurality of relations} in light of the discussions above. In any case, \citet[535]{jensen:1998} argues that the \ili{Proto-Tupi-Guaraní} prefix \example{*jo-} is reflected by the prefix \example{ju-} in the descendant language \ili{Urubú-Ka’apor}, whereas \ili{Proto-Tupi-Guaraní} \example{*je-} was lost in the language. The prefix \example{ju-} serves as voice marking in both the reflexive\is{reflexive voice} and reciprocal\is{reciprocal voice} voices, in the latter accompanied by \isi{reduplication} \citep[339f.]{kakumasu:1986}.\is{reciprocal origin} It seems that once the reflexive\is{reflexive voice} prefix \example{*je-} was lost in (an earlier stage of) the language, the reflexive\is{reflexive voice} function was acquired by the reciprocal\is{reciprocal voice} prefix \example{*jo-} (later \example{ju-}) to the extent that additional marking (i.e. \isi{reduplication}) eventually became necessary to express the original reciprocal\is{reciprocal voice}\is{reciprocal origin} meaning. Thus, it is worth noting that the synchronic reflexive-reciprocal syncretism in \ili{Urubú-Ka’apor} qualifies as type 2 syncretism\is{voice syncretism, partial resemblance -- type 2}, unlike the synchronic reflexive-reciprocal syncretism of type 1\is{voice syncretism, full resemblance -- type 1} described for \ili{Hmwaveke}, \ili{Drehu}, East Futunan\il{Futunan, East}, and \ili{Tariana} above. The development of the syncretism in \ili{Urubú-Ka’apor} is illustrated in \tabref{tab:ch7:recp-refl-urubu} \citep[340]{kakumasu:1986}. Reflexive-reciprocal syncretism of type 1b can be found in the related Tupi-Guaraní language \ili{Wayampi} (e.g. \example{o-j-awyky} ‘they do each other’s hair’ or ‘they do their own respective hair’, \citealt[334]{copin:2012}). However, in this language neither \ili{Proto-Tupi-Guaraní} \example{*je-} nor \example{*jo-} has been lost, and the prefix \example{j-} is simply an allomorph\is{allomorphy} of both the synchronic prefixes \example{je-} and \example{jo-} which have been retained in \ili{Wayampi} alongside their original reflexive\is{reflexive voice} and reciprocal\is{reciprocal voice} functions. Note that \citeauthor{copin:2012}’s account of the reflexive\is{reflexive voice} and reciprocal\is{reciprocal voice} voice marking in \ili{Wayampi} contrasts with that of \citet{jensen:1998} who argues that only Proto-Tupi-Guaraní \example{*je-} has been retained in the language (in the form \example{ji-}) while \example{*jo-} has been lost. The authors probably describe different varieties of the language.

\begin{table}
	\setlength{\tabcolsep}{2.9pt}
	\begin{tabularx}{\textwidth}{rclll}
		\lsptoprule
		\ili{Proto-Tupi-Guaraní} & \example{jo-} & \textsc{recp} & → & \textsc{refl} \\
		\midrule 
		\ili{Urubú-Ka’apor} & \example{ju-} & \example{ju-tuka\~{}tuka} ‘to bump e.o.’ & & \example{ju-pukwar} ‘to tie self’ \\
		\lspbottomrule
	\end{tabularx}
	\caption{\textsc{refl}-\textsc{recp} syncretism of \textsc{recp} origin in Urubú-Ka’apor}
	\label{tab:ch7:recp-refl-urubu}
\end{table}

\citet[341]{alpher:al:2003} argue for a distinction between reflexive\is{reflexive voice} \example{*-yi} and reciprocal\is{reciprocal voice} \example{*-nci} in \ili{Proto-Gunwinyguan} based on observations from languages of the Gunwinyguan language family (\lang{au}) and beyond the family, including Worrorran, Tangkic, Nyulnyulan, and Mangarrayi-Marran languages. Some of these observations are summarised in \tabref{tab:ch7:recp-refl-australian} (for information on the diachronic sound changes leading to the synchronic reciprocal\is{reciprocal voice} suffixes, see \citealt[343]{alpher:al:2003}). In Tangkic languages, in three Gunwinyguan languages (\ili{Waray}, \ili{Ngandi}, and \ili{Nunggubuyu}), and in the Mangarrayi-Maran language \ili{Warndarang} the reflexive\is{reflexive voice} and reciprocal\is{reciprocal voice} suffixes are distinct, while the remaining languages feature reflexive-reciprocal syncretism. As discussed further below, it seems that the reciprocal\is{reciprocal voice} voice marking in several Gunwinyguan languages has developed a reflexive\is{reflexive voice} function,\is{reciprocal origin} and the same might even be true for Nyulnyulan languages and the Mangarrayi-Maran language \ili{Alawa}. By contrast, the reflexive\is{reflexive voice} voice marking in Worrorran languages may have developed a reciprocal\is{reciprocal voice} function, while it seems that a reflexive\is{reflexive voice} suffix and a reciprocal\is{reciprocal voice} suffix have merged to form the suffix \example{-(ñ)jiyi} in \ili{Mangarrayi}. The suffix \example{-yi} in this language is retained with “[o]nly five verbs” \citep[154]{merlan:1989}. 

\begin{table}
	\setlength{\tabcolsep}{7pt}
	\begin{tabularx}{\textwidth}{llccc}
		\lsptoprule
		& & \textsc{refl} & & \textsc{recp} \\
		\midrule 
		Worrorran & \ili{Ungarinyin} & \example{-yi} & \textsuperscript{?} → & \example{-yi} \\
		& \ili{Worrorra} & \example{-ye} & \textsuperscript{?} → & \example{-ye} \\
		\midrule
		Tangkic & \ili{Kayardild} & \example{-yi} & & \example{-nycu} \\
		& \ili{Lardil} & \example{-yi} & & \example{-nyci} \\
		\midrule
		Gunwinyguan & \ili{Waray} & \example{-yi} & & \example{-tji} \\
		& \ili{Ngandi} & \example{-i} &  & \example{-yd̪i} \\
		& \ili{Nunggubuyu} & \example{-i} &  & \example{-nʸji} \\
		& \ili{Rembarrnga} & \example{-tti} & ← & \example{-tti} \\
		& \ili{Jawoyn} & \example{-ci} & ← & \example{-ci} \\
		& \ili{Ngalakan} & \example{-či} & ← & \example{-či} \\
		& \ili{Bininj Gun-Wok} & \example{-rri} & ← & \example{-rri} \\
		& \ili{Dalabon} & \example{-rri} & ← & \example{-rri} \\
		\midrule
		Nyulnyulan & \ili{Warrwa} & \example{-nyci} & ← \textsuperscript{?} & \example{-nyci} \\
		& \ili{Bardi} & \example{-inyci} & ← \textsuperscript{?} & \example{-inyci} \\
		& \ili{Nyigina} & \example{-nyci} & ← \textsuperscript{?} & \example{-nyci} \\
		& \ili{Yawurru} & \example{-nyci} & ← \textsuperscript{?} & \example{-nyci} \\
		\midrule
		Mangarrayi-Maran & \ili{Warndarang} & \example{-i} & & \example{-yi}, (\example{-ji}) \\
		& \ili{Alawa} & \example{-nyci} & ← \textsuperscript{?} & \example{-nyci} \\
		& \ili{Mangarrayi} & \example{-yi/-(ñ)jiyi} & & \example{-yi/-(ñ)jiyi} \\
		\lspbottomrule
	\end{tabularx}
	\caption{\textsc{refl}-\textsc{recp} syncretism in Australia}
	\label{tab:ch7:recp-refl-australian}
\end{table}



The data from the Gunwinyguan languages (and the observations from the Tangkic languages) in \tabref{tab:ch7:recp-refl-australian} evidently suggest that “the original reciprocal\is{reciprocal voice}\is{reciprocal origin} suffix has extended its range to replace the original reflexive\is{reflexive voice}” in \ili{Rembarrnga}, \ili{Jawoyn}, \ili{Ngalakan}, \ili{Bininj Gun-Wok}, and \ili{Dalabon} \citep[343]{alpher:al:2003}. Further evidence for this claim can be found in \ili{Nunggubuyu}. Although this language retains separate marking for the reflexive\is{reflexive voice} and reciprocal\is{reciprocal voice} voices, “occasionally a morphological Recip[rocal] is used in reflexive\is{reflexive voice} sense” (e.g. \example{wanᵍi-nʸji} ‘to bite self’ or ‘to bite e.o.’ and \example{ṟi-nʸji} ‘to spear self’ or ‘to spear e.o.’, \citealt[392]{heath:1984}). The development of reflexive-reciprocal syncretism in \ili{Nunggubuyu} and a few other Gunwinyguan languages is illustrated in \tabref{tab:ch7:recp-refl-gunwinyguan} (Nunggubuyu = \citealt[392]{heath:1984}; Rembarrnga = \citealt[278, 282]{mckay:1975}; Ngalakan = \citealt[193, 215]{merlan:1983}; Bininj Gun-Wok = \citealt[444]{evans:2003}; Dalabon = \citealt{evans:2017}). It has not been possible to obtain any examples for \ili{Jawoyn}. Observe that the suffixes in the table differ slightly in their cited and realised forms due to various morphophonological\is{morphophonology} and morphological conditions (for an overview of these differences, see \citealt[342]{alpher:al:2003}). As indicated by the question marks in \tabref{tab:ch7:recp-refl-australian}, the diachrony of reflexive-reciprocal syncretism in the Worrorran, Nyulnyulan, and Mangarrayi-Maran languages is more uncertain. Only if it is assumed that the suffixes \example{*-yi} and \example{*-nci} reconstructed\is{reconstruction} for \ili{Proto-Gunwinyguan} can be traced further back to a Northern Australian ancestral language (or represent an ancient areal feature), the reflexive-reciprocal syncretism in Worrorran languages can be considered to be of \isi{reflexive origin}, and the reflexive-reciprocal syncretism in Nyulnyulan languages and \ili{Alawa} of \isi{reciprocal origin}.

\begin{table}
	\setlength{\tabcolsep}{2.8pt}
	\begin{tabularx}{\textwidth}{rcllll}
		\lsptoprule
		P.-Gunwinyg.\il{Proto-Gunwinyguan} & \example{*-nci} & \textsc{recp} & \multicolumn{2}{r}{→} & \textsc{refl} \\
		\midrule 
		\ili{Nunggubuyu} & \example{-nʸji} & \example{ṟi-nʸji} & ‘to spear sth.’ & \example{ṟi-nʸji} & ‘to spear self’ \\
		\ili{Rembarrnga} & \example{-tti} & \example{ṛokna-ttə-} & ‘to meet e.o.’ & \example{ṭeţmə-ttə-} & ‘to cut self’ \\
		\ili{Ngalakan} & \example{-či} & \example{woymi-či-} & ‘to kill e.o.’ & \example{dačmi-či-} & ‘to cut self’ \\
		B. Gun-Wok\il{Bininj Gun-Wok} & \example{-rri} & \example{djobge-rre-} & ‘to cut e.o.’ & \example{djobge-rre-} & ‘to cut self’ \\
		\ili{Dalabon} & \example{-rri} & \example{na-rrû-} & ‘to look at e.o.’ & \example{na-rrû-} & ‘to look at self’ \\
		\lspbottomrule
	\end{tabularx}
	\caption{\textsc{refl}-\textsc{recp} syncretism of \textsc{recp} origin in Gunwinyguan}
	\label{tab:ch7:recp-refl-gunwinyguan}
\end{table}

Finally, reflexive-reciprocal syncretism of \isi{reciprocal origin} has been attested in at least one Turkic language, \ili{Tuvan} (\lang{ea}). In this language the suffix \example{-š} serves as productive\is{productivity} voice marking in the reciprocal\is{reciprocal voice} voice, but can also have a reflexive\is{reflexive voice} function with verbs with the very specific meaning ‘to make sth. dirty’ or ‘to smear sth.’ even though there is another “specialized and highly productive\is{productivity} marker of reflexivity\is{reflexive voice}” in the language, \example{-n} \citep[1213]{kuular:2007}. \citeauthor{kuular:2007} lists five such verbs, each with the meaning ‘to make sth. dirty’ or ‘to smear sth.’ without the suffix \example{-š} (\example{bəlča-}, \example{bəlčakta-}, \example{bəlga-}, \example{bora-}, \example{öge-}) and with the meaning ‘to make self dirty’ or ‘to smear self’ with the suffix (\example{bəlča-š-}, \example{bəlčakta-š-}, \example{bəlga-š-}, \example{bora-š-}, \example{öge-š-}). According to \citet[1154f.]{nedjalkov:nedjalkov:2007}, “[t]here is no generally accepted etymology of the reciprocal\is{reciprocal voice} suffix” but it is known that “[r]eciprocity was marked by the suffix \example{-š} as early as in Common Turkic\il{Turkic, Common} (approximately in the last centuries BCE)”. As argued by \citet{gandon:2018}, other common uses of the reflexes of Common Turkic\il{Turkic, Common} \example{*-š} in descendant languages can be subsumed under the notion of \isi{plurality of relations}. In any case, the reflexive\is{reflexive voice} use of the suffix is a much more recent innovation in \ili{Tuvan}, and the development is illustrated in \tabref{tab:ch7:recp-refl-tuvan} \citep[1177, 1213]{kuular:2007}. \citet[243]{salo:2013} argues that a similar development has taken place in “\ili{Bashkir} dialects in particular” and that “[t]his has been attested in some eastern and southern dialects”. Unfortunately, \citeauthor{salo:2013} provides no examples, and it has not been possible to obtain data on these \ili{Bashkir} varieties to confirm the claim.

\begin{table}
	\setlength{\tabcolsep}{3.7pt}
	\begin{tabularx}{\textwidth}{rclll}
		\lsptoprule
		Common Turkic\il{Turkic, Common} & \example{*-š} & \textsc{recp} (+ \textsc{por}) & → & \textsc{refl} \\
		\midrule 
		\ili{Tuvan} & \example{-š} & \example{sögle-š-} ‘to offend e.o.’ & & \example{öge-š-} ‘to make self dirty’ \\
		\lspbottomrule
	\end{tabularx}
	\caption{\textsc{refl}-\textsc{recp} syncretism of \textsc{recp} origin in Tuvan}
	\label{tab:ch7:recp-refl-tuvan}
\end{table}

Although the development from reciprocal\is{reciprocal voice} to reflexive\is{reflexive voice}\is{reciprocal origin} has been explicitly noted in the literature, albeit sporadically (and mostly in relation to the Oceanic languages; e.g. \citealt{moyse-faurie:2008, moyse-faurie:2017}), a possible explanation for the phenomenon has seldom been considered. In a rare explicit discussion of the diachrony, \citet[46f.]{lichtenberk:2000} briefly considers reflexive-reciprocal syncretism of \isi{reciprocal origin} in East Futunan\il{Futunan, East} (see \tabref{tab:ch7:recp-recp-oceanic} on page \pageref{tab:ch7:recp-recp-oceanic}) and notes that the reflexive\is{reflexive voice} function of the prefix \example{fe-} only is found with “body action” verbs, for which reason “these verbs must be distinguished from reflexives\is{reflexive voice} proper”. Following \citet{kemmer:1993}, \citet[47]{lichtenberk:2000} instead considers “such verbal constructions to be middle\is{middle syncretism} rather than reflexive\is{reflexive voice}”. In turn, \citet[48]{lichtenberk:2000} argues that “middles are particularly close to reciprocals\is{reciprocal voice} among the plurality-of-relations meanings”\is{plurality of relations} in terms of “Initiator-Endpoint unity”\is{initiator-endpoint unity} meaning that all participants are both initiator and endpoint (cf. \citealt[207ff.]{kemmer:1993}). Thus, \citet{lichtenberk:2000} essentially proposes a reverse development from the reciprocal stage\is{reciprocal origin} to the \isi{grooming}/motion stage in \citeauthor{haspelmath:2003}’s (\citeyear{haspelmath:2003}) semantic map of voice development presented in \figref{fig:ch7:unidirectional} on page \pageref{fig:ch7:unidirectional}, but argues that the East Futunan\il{Futunan, East} prefix \example{fe-} has not developed a reflexive\is{reflexive voice} function that goes beyond body actions. However, it is clear from several of the examples in this section that the reflexive\is{reflexive voice} stage has been reached in other languages and a more general explanation is therefore needed.

As demonstrated here and in \sectref{diachrony:refl2recp}, the reflexive\is{reflexive voice} and reciprocal\is{reciprocal voice} voices are functionally similar enough to converge\is{convergence} in terms of voice marking in languages worldwide. Considering the close ties between the two voices, there is really no reason to assume that a voice development from reciprocal\is{reciprocal voice} to reflexive\is{reflexive voice}\is{reciprocal origin} cannot be explained in the same terms as voice development from reflexive\is{reflexive voice} to reciprocal\is{reciprocal voice} discussed in \sectref{diachrony:refl2recp}, only in a reverse manner. Thus, it seems that the development of reflexive-reciprocal syncretism can potentially follow a reverse version of the developmental path from reflexive\is{reflexive voice} to reciprocal\is{reciprocal voice} formulated by \citet{heine:miyashita:2008} in \figref{fig:ch7:refl-recp} on \pageref{fig:ch7:refl-recp}, e.g. ‘they wash each other’ → ‘they wash themselves’ → ‘s/he washes self’.

\subsection{From reciprocal to anticausative} \label{diachrony:recp2antc}
Reciprocal-anticausative syncretism of \isi{reciprocal origin} has received minimal attention in the literature, and evidence for the phenomenon is accordingly scarce, though not entirely absent. For instance, reflexes of the \ili{Proto-Bantu} suffix \example{*-an} are known to be “notoriously polysemic”\is{polysemy} in descendant Bantu languages \citep[732]{bostoen:al:2015} and it is generally believed that the suffix originally pertained to reciprocity\is{reciprocal voice}\is{reciprocal origin} and other functions of \isi{plurality of relations}, notably \isi{sociativity} (\citealt[76]{schadeberg:2003}; \citealt[137ff.]{dom:al:2016}). As discussed at length by \citet{maslova:2000}, the proto-suffix seems to be related to the preposition \example{na} ‘with’ in many Bantu languages which would suggest that the proto-suffix \example{*-an} likely had a sociative\is{sociativity} function when it first arose (cf. \ili{Kirundi} \example{-tamb-an-} ‘to dance together’, \citealt[273, 277]{ndayiragije:2006}) from which the reciprocal\is{reciprocal voice} function subsequently evolved (cf. \ili{Kirundi} \example{-kúbit-an-} ‘to hit e.o.’ In any case, \citet[345]{maslova:2007} observes that reflexes of the suffix \example{*-an} can even be used as “non-reciprocal\is{reciprocal origin} detransitivizer[s],\is{detransitivisation} although this phenomenon is very rare and highly lexically constrained”. Likewise, \citet[139]{dom:al:2016} briefly mention that the reflexes in question can indicate “spontaneous events” in some descendant languages. Such development from reciprocal\is{reciprocal voice} (and \isi{plurality of relations} in general) to anticausative\is{anticausative voice} can, for instance, be seen in \ili{Babungo} \citep[209f.]{schaub:1985} and \ili{Orungu} \citep[191]{ambouroue:2007}, as illustrated in \tabref{tab:ch7:recp-antc-bantu}. The reflexes of the \ili{Proto-Oceanic} prefix \example{*pa\textsc{r}i-} described in the previous section are known to have a “spontaneous” use in some Oceanic languages as well (\citealt[48]{lichtenberk:2000}; \citealt[32, 51]{bril:2005}; \citealt[109]{moyse-faurie:2008}; \citeyear[109]{moyse-faurie:2017}). However, unlike the “spontaneous events” noted by \citet{dom:al:2016} among Bantu languages, the “spontaneous” uses in the Oceanic languages are generally to be understood in the literal sense ‘to happen spontaneously’. It has not been possible to find a proper anticausative\is{anticausative voice} function for any reflex of \example{*pa\textsc{r}i-} among Oceanic languages.

\begin{table}
	\setlength{\tabcolsep}{5.3pt}
	\begin{tabularx}{\textwidth}{rclllll}
		\lsptoprule
		\ili{Proto-Bantu} & \example{*-an} & \multicolumn{2}{l}{\textsc{recp} (+ \textsc{por})} & → & \textsc{antc} & \\
		\midrule 
		\ili{Babungo} & \example{-ne} & \example{yé-né} & ‘to see e.o.’ & & \example{ngà’-nè} & ‘to open’ \\
		\ili{Orungu} & \example{-àn} & \example{βòn-àn-} & ‘to look at e.o.’ & & \example{βùɾ-àn-} & ‘to fold/bend’ \\
		\lspbottomrule
	\end{tabularx}
	\caption{\textsc{recp}-\textsc{antc} syncretism of \textsc{recp} origin in Bantu languages}
	\label{tab:ch7:recp-antc-bantu}
\end{table}

Additionally, observe that the Common Turkic\il{Turkic, Common} suffix \example{*-š} with functions related to reciprocity\is{reciprocal voice} and \isi{plurality of relations} discussed in the previous section has possibly developed an anticausative\is{anticausative voice} function\is{reciprocal origin} in some descendant languages, including \ili{Tuvan}. While the reflexive\is{reflexive voice} use of the suffix \example{-š} in this language is very restricted in this language (see \tabref{tab:ch7:recp-refl-tuvan} on page \pageref{tab:ch7:recp-refl-tuvan}), its anticausative\is{anticausative voice} use is more productive\is{productivity}, although not as productive\is{productivity} as its reciprocal\is{reciprocal voice} use \citep[1176ff., 1221f.]{kuular:2007}. An anticausative\is{anticausative voice} use\is{reciprocal origin} of the suffix is also attested in a handful of other related languages, but in these languages the use is considerably more marginal. For instance, \citet[295]{nedjalkov:2007d} and \citet[1142]{nedjalkov:nedjalkov:2007} observe a “non-productive”\is{productivity} anticausative\is{anticausative voice} use of the suffix \example{-s} in \ili{Yakut}, and \citeauthor{gandon:2013} (\citeyear[16f.]{gandon:2013};; \citeyear{gandon:2018}) notes that the suffix \example{-ş} in \ili{Turkish} has an anticausative\is{anticausative voice} function with twelve verbs. \citet[57ff.]{gandon:2013} also provides a list of other Turkic languages in which hints of an anticausative\is{anticausative voice} use of the suffix can be found, including \ili{Khakas}, \ili{Uzbek}, \ili{Tatar}, and \ili{Karachay-Balkar}. \citet[58]{gandon:2013} even provides two examples of what seems to be an anticausative\is{anticausative voice} use of the suffix \example{-ş} in 11th--13th century Old Turkic\il{Turkic, Old}, \example{kar-} ‘to mix sth.’ ↔ \example{kar-ış-} ‘to mix’, \example{kat-} ‘to join sth.’ ↔ \example{kat-ış-} ‘to join (up)’. The presumed \isi{diachronic development} is illustrated by examples from three of these languages in \tabref{tab:ch7:recp-antc-turkic} (Tuvan = \citealt[1177, 1222]{kuular:2007}; Yakut = \citealt[295]{nedjalkov:2007d}; \citealt[1112]{nedjalkov:nedjalkov:2007}; Turkish = \citealt[12, 17]{gandon:2013}). Considering the age of the Old Turkic\il{Turkic, Old} examples and the wide distribution of the (barely productive)\is{productivity} anticausative\is{anticausative voice} use among modern Turkic languages in general, it can alternatively be hypothesised that Common Turkic\il{Turkic, Common} \example{*-š} had a marginal anticausative\is{anticausative voice} function, traces of which have simply been retained in some descendant languages. In any case, functions related to reciprocity\is{reciprocal voice}\is{reciprocal origin} and \isi{plurality of relations} would have been considerably more common than an anticausative\is{anticausative voice} function in Common Turkic, and \citet{gandon:2018} ultimately favours a \isi{diachronic development} from reciprocal\is{reciprocal voice} to anticausative\is{anticausative voice}.

\begin{table}
	\setlength{\tabcolsep}{3.2pt}
	\begin{tabularx}{\textwidth}{rclllll}
		\lsptoprule
		Common Turkic\il{Turkic, Common} & \example{*-š} & \multicolumn{2}{l}{\textsc{recp} (+ \textsc{por})} & → & \textsc{antc} & \\
		\midrule 
		\ili{Tuvan} & \example{-š} & \example{tanə-š-} & ‘to know e.o.’ & & \example{mööŋŋe-š-} & ‘to accumulate’ \\
		\ili{Yakut} & \example{-s} & \example{bul-us-} & ‘to find e.o.’ & & \example{tüm-üs-} & ‘to gather’ \\
		\ili{Turkish} & \example{-ş} & \example{bul-uş-} & ‘to find e.o.’ & & \example{yığ-ış-} & ‘to pile up’ \\
		\lspbottomrule
	\end{tabularx}
	\caption{\textsc{recp}-\textsc{antc} syncretism of \textsc{recp} origin in Turkic languages}
	\label{tab:ch7:recp-antc-turkic}
\end{table} 

Finally, in the previous section it was discussed at length that the Proto-Gun\-winy\-guan\il{Proto-Gunwinyguan} reciprocal\is{reciprocal voice} suffix \example{*-nci} appears to have developed a reflexive\is{reflexive voice} function\is{reciprocal origin} in several descendant languages (see \tabref{tab:ch7:recp-refl-gunwinyguan} on page \pageref{tab:ch7:recp-refl-gunwinyguan}). In one of these languages, \ili{Ngalakan}, the reflex \example{-či} has even developed a marginal anticausative\is{anticausative voice} function. \citet[133]{merlan:1983} explicitly argues that “[o]ften the reflexive-reciprocal is used with a kind of ‘middle’\is{middle syncretism} meaning, and represents a process as taking place only within and affecting the crossreferenced NP,\is{affectedness} not occurring through outside \isi{agency}”. \citet[133, 203]{merlan:1983} provides the verb \example{jurmi-či-} ‘to spill’ as an example (cf. \example{jurmi-} ‘to pour sth.’, i.e. ‘to make sth. spill’), and additional examples can be located elsewhere in her descriptive grammar of the language: \example{ḷerʔmi-} ‘to set sth. alight’ ↔ \example{ḷerʔmi-či-} ‘to come alight’, \example{jorŋmi-} ‘to stretch sth.’ ↔ \example{jorŋmi-či-} ‘to stretch’ \citep[7, 87, 202f.]{merlan:1983}. These verbs are here presented with the thematic auxiliary \example{-mi} “to which tense-aspect and reflexive-reciprocal suffixes are added” in thematic verbs like \example{jur-}, \example{ḷerʔ-}, and \example{jorŋ-} \citep[93]{merlan:1983}. Moreover, it should be noted that the suffix \example{-či} in \ili{Ngalakan} also has a reflexive\is{reflexive voice} function, and the more precise chronological order of this and the anticausative\is{anticausative voice} function is uncertain. Thus, as illustrated in \tabref{tab:ch7:recp-antc-ngalakan}, it is possible that the anticausative\is{anticausative voice} function has evolved from reflexive-reciprocal syncretism. It has hitherto not been possible to find similar syncretism in other Gunwinyguan languages.

\begin{table}
	\setlength{\tabcolsep}{2pt}
	\begin{tabularx}{\textwidth}{rclll}
		\lsptoprule
		\ili{Proto-Gunwinyguan} & \example{*-nci} & \textsc{recp} (+ \textsc{refl}?) & → & \textsc{antc} \\
		\midrule 
		\ili{Ngalakan} & \example{-či} & \example{woymi-či-} ‘to kill e.o.’ & & \example{ḷerʔmi-či-} ‘to come alight’ \\
		\lspbottomrule
	\end{tabularx}
	\caption{\textsc{recp}-\textsc{antc} syncretism of \textsc{recp} origin in Ngalakan}
	\label{tab:ch7:recp-antc-ngalakan}
\end{table}

In terms of diachrony, it is plausible that the rise of reciprocal-anticausative\is{reciprocal origin} syncretism is facilitated by lexically reciprocal\is{lexical reciprocal} verbs which do not necessarily involve conscious mutual action by the involved semantic participants\is{semantic participant}. As observed by \citet{nedjalkov:nedjalkov:2007}, the anticausative\is{anticausative voice} function of the suffix \example{-s} in \ili{Yakut} is restricted to precisely such verbs (e.g. \example{tüm-} ‘to gather sth.’ → \example{tüm-üs-} ‘to gather each other’ → ‘to gather’). As evident by the Bantu and \ili{Ngalakan} examples presented in this section, the anticausative\is{anticausative voice} function appears to be less restricted semantically in these languages, but it may very well have evolved in relation to lexically reciprocal\is{lexical reciprocal} verbs as well. Consider, for instance, the verb \example{-kuvhang-an-} ‘to gather’ in the Bantu language \ili{Venda}  \citep[341]{maslova:2007} and the verb \example{-mala-maŋi-či-} ‘to gather’ in \ili{Ngalakan} (\example{mala-} is a collective ‘group’ suffix, i.e. ‘to all gather’, \citealt[94]{merlan:1983}). As discussed later in \sectref{diachrony:antc2recp}, an opposite development from anticausative\is{anticausative voice} to reciprocal\is{reciprocal voice} might have taken place in the extinct Indo-European language \ili{Hittite}.

\subsection{From reciprocal to passive} \label{diachrony:recp2pass}
There is currently no good evidence for \isi{diachronic development} from reciprocal\is{reciprocal voice} to passive\is{passive voice}.\is{reciprocal origin} \citet[206]{heine:miyashita:2008} briefly consider such diachrony for the prefix \example{mə-} in the Berber language \ili{Tuareg} (\lang{af}) which has passive\is{passive voice} and reciprocal\is{reciprocal voice} functions, but no reflexive\is{reflexive voice} function. However, they conclude that “this case provides no convincing evidence for a reciprocal\is{reciprocal voice} > passive\is{passive voice} evolution” because a cognate prefix in the related language \ili{Tamasheq} features a reflexive\is{reflexive voice} function. In the language sample of this book the Highland East Cushitic language \ili{Sidaama} also features a suffix (\example{-am}) with passive\is{passive voice} and reciprocal\is{reciprocal voice} functions (in addition to an anticausative\is{anticausative voice} function) but no reflexive\is{reflexive voice} function, yet the original function of this suffix appears to have been passive\is{passive origin} (\sectref{diachrony:pass2recp}). 

\subsection{From reciprocal to antipassive} \label{diachrony:recp2antp}
Diachronic development\is{diachronic development} from reciprocal\is{reciprocal voice} to antipassive\is{antipassive voice}\is{reciprocal origin} has received slightly more attention in the literature than the diachronic scenarios discussed in the previous three sections, and has notably been discussed in relation to Bantu languages (e.g. \citealt{bostoen:al:2015}) and Oceanic languages (e.g. \citealt{janic:2016}). With regard to the former languages, the \ili{Proto-Bantu} suffix \example{*-an} associated with reciprocity\is{reciprocal voice} and \isi{plurality of relations} (\sectref{diachrony:recp2antc})\is{reciprocal origin} has developed an antipassive\is{antipassive voice} function in a number of descendant languages, including \ili{Kirundi} \citep[272ff.]{ndayiragije:2006}, \ili{Swazi}, \ili{Ndonga} \citep[297f.]{nedjalkov:2007d}, and \ili{Tswana} \citep[755]{creissels:2018}. \citet[731f., 738ff.]{bostoen:al:2015} argue that the antipassive-reciprocal syncretism in question has largely been overlooked among the Bantu languages in the past, suggesting that it might be even more widespread, and also attest the syncretism in \ili{Kinyarwanda}, \ili{Gikuyu}, \ili{Kikamba}, and \ili{Kilega}. \citet[742ff.]{bostoen:al:2015} even mention a few Bantu languages “where an unproductive\is{productivity} antipassive\is{antipassive voice} marker is likely to exist”. The diachrony of the antipassive-reciprocal syncretism is illustrated by examples from a few of these languages in \tabref{tab:ch7:recp-antp-bantu} (Kirundi = \citealt[275]{ndayiragije:2006}; Gikuyu = \citealt[163f.]{mugane:1999}; Kikamba = \citealt[39]{kioko:2005}; Kilega = \citealt[136f.]{botne:2003}). \citet[741]{bostoen:al:2015} also provide an interesting account of \ili{Kisongye} in which the suffix \example{-an} “is no longer polysemic”\is{polysemy} but “has become a dedicated antipassive\is{antipassive voice} marker”, while reciprocity\is{reciprocal voice} is “currently expressed through a combination of reflexive\is{reflexive voice} prefix \example{-i} and the suffix \example{-een-}, which is analyzed as a representation of \example{-an-} fused with the applicative\is{applicative voice} suffix \example{-il-}”.

\begin{table}
	\setlength{\tabcolsep}{5.5pt}
	\begin{tabularx}{\textwidth}{rcllll}
		\lsptoprule
		\ili{Proto-Bantu} & \example{*-an} & & \textsc{recp} (+ \textsc{por}) & \multicolumn{1}{r}{→} & \textsc{antp} \\
		\midrule 
		\ili{Kirundi} & \example{-an} & \example{-tuk-an-} & ‘to insult e.o.’ & & ‘to insult [sb.]’ \\
		\ili{Gikuyu} & \example{-an} & \example{-ingat-an-} & ‘to chase e.o.’ & & ‘to chase [sb.]’ \\
		\ili{Kikamba} & \example{-an} & \example{-m-an-} & ‘to bite e.o.’ & & ‘to bite [sb.]’ \\
		\ili{Kilega} & \example{-an} & \example{-kugamb-an-} & ‘to slander e.o.’ & & ‘to slander [sb.]’ \\
		\lspbottomrule
	\end{tabularx}
	\caption{\textsc{antp}-\textsc{recp} syncretism of \textsc{recp} origin in Bantu languages}
	\label{tab:ch7:recp-antp-bantu}
\end{table}

Like the \ili{Proto-Bantu} suffix \example{*-an}, the \ili{Proto-Oceanic} prefix \example{*pa\textsc{r}i-} is also generally associated with functions pertaining to \isi{plurality of relations}, including reciprocity\is{reciprocal voice} (\sectref{diachrony:recp2refl}). It seems that this prefix has developed an antipassive\is{antipassive voice}\is{reciprocal origin} function in some descendant languages, though the chronological order in which the antipassive\is{antipassive voice} function evolved in relation to the reciprocal\is{reciprocal voice} function remains somewhat uncertain. \citet[178]{janic:2016} speculates that the prefix probably had a general function in the proto-language “where the assignment of the semantic roles to the participants of the event was motivated by the general knowledge of the world, lexical meaning of a verb and/or by the external factors such as discourse context”, before it later “started to categorize the events characterized by the \isi{plurality of relations} into more specific types such as reciprocal\is{reciprocal voice}, antipassive\is{antipassive voice}, collective and chaining etc.” This scenario suggests that antipassivity\is{antipassive voice} did not necessarily evolve from reciprocity\is{reciprocal voice},\is{reciprocal origin} but concurrently alongside it. However, \citet[178f.]{janic:2016} admits that “[d]ue to the lack of historical data, the proposed hypothesis is highly speculative and by no means categorical and absolute in nature” and “a later development of the antipassive\is{antipassive voice} in the Oceanic languages cannot be entirely excluded”. It is, for instance, worth observing that attestations of the antipassive\is{antipassive voice} function are rather sporadic among the Oceanic languages, while the reciprocal\is{reciprocal voice} function is widespread (as also mentioned by \citealt[160]{janic:2016}). Furthermore, \citet[151]{pawley:1973} argues that the prefix \example{*pa\textsc{r}i-} is likely to have had a reciprocal\is{reciprocal voice} function in \ili{Proto-Oceanic}, albeit “restricted to a subclass of verbs”. Consequently, the possibility of a \isi{reciprocal origin} for the antipassive-reciprocal syncretism is here kept open, and illustrated in \tabref{tab:ch7:recp-antp-oceanic} (To’aba’ita = \citealt[1552, 1560]{lichtenberk:2007}; Tolai = \citealt[146f.]{mosel:1984};; Hoava = \citealt[136f.]{davis:2003}; Drehu, Iaai, Fijian = \citealt[35ff., 47, 57]{bril:2005}).

Antipassive-reciprocal syncretism among Oceanic languages has notably been discussed repeatedly in relation to the prefix \example{kwai-} in \ili{To’aba’ita} by \citet{lichtenberk:1991, lichtenberk:2000, lichtenberk:2007}. Additionally, \citet[147, 156]{mosel:1984} explicitly argues that the prefix \example{var-} in \ili{Tolai} “does not exclusively mean reciprocity\is{reciprocal voice}, but is also used to derive\is{derivation} non-reciprocal \isi{intransitive} verbs from \isi{transitive} verbs”. An antipassive\is{antipassive voice} function is also observed by \citet[137f.]{davis:2003} for the prefix \example{vari-} in \ili{Hoava}, by \citet[37f.]{bril:2005} for the prefixes \example{i-} and \example{ü-} in \ili{Drehu} and \ili{Iaai} (and possibly in \ili{Nengone}), and by \citet[164]{janic:2016} for the prefix \example{vei-} in Standard \ili{Fijian}. \citet[33, 39]{bril:2005} also mentions a marginal and lexicalised\is{lexicalisation} function of the \ili{Xârâcùù} prefix \example{ù-} which is reminiscent of antipassivity\is{antipassive voice} (e.g. \example{bë} ‘to move to sth.’ ↔ \example{ù-bë} ‘to be jittery’ and \example{xù} ‘to give sb. sth.’ ↔ \example{ù-xù} ‘to be contagious’). \citet[1047]{moyse-faurie:2015} contributes additional examples (e.g. \example{da} ‘to eat sth.’ ↔ \example{ù-da} ‘to bite [sb.]’ and \example{sö} ‘to pride oneself on sth.’ ↔ \example{ù-sö} ‘to be haughty, be a boaster”). The various prefixes mentioned here are all derived from \ili{Proto-Oceanic} \example{pa\textsc{r}i-}, even the \ili{To’aba’ita} prefix with its somewhat peculiar form. \citet[1566f.]{lichtenberk:2007} argues that “the expected reflex in To’aba’ita is \example{*fai-}” but “[f]or some reason, in the proto-language from which To’aba’ita and a few other very closely related languages are descended the prefix underwent an irregular change of earlier \example{**f} to \example{**w}” and “[l]ater on in the history of these languages, \example{**w} changed to \example{kw}”.

\begin{table}
	\setlength{\tabcolsep}{2pt}
	\begin{tabularx}{\textwidth}{rcllll}
		\lsptoprule
		\ili{Proto-Oceanic} & \example{*pa\textsc{r}i-} & \textsc{recp} (+ \textsc{por}) & \multicolumn{1}{r}{→} & \textsc{antp} & \\
		\midrule 
		\ili{To’aba’ita} & \example{kwai-} & \example{kwai-ngalufi} & ‘to berate e.o.’ & \example{kwai-labata’i} & ‘to harm [sb.]’ \\
		\ili{Tolai} & \example{var-} & \example{var-ubu} & ‘to hit e.o.’ & \example{var-karat} & ‘to bite [sth.]’ \\
		\ili{Hoava} & \example{vari-} & \example{vari-ome} & ‘to see e.o.’ & \example{vari-poni} & ‘to give [sth.]’ \\
		\ili{Drehu} & \example{i-} & \example{i-aja} & ‘to desire e.o.’ & \example{i-hej} & ‘to bite [sth.]’ \\
		\ili{Iaai} & \example{ü-} & \example{ü-hlingöö} & ‘to kill e.o.’ & \example{ü-hülü} & ‘to bite [sth.]’ \\
		\ili{Fijian} & \example{vei-} & \example{vei-dree} & ‘to pull e.o.’ & \example{vei-vuke} & ‘to bite [sth.]’ \\
		\lspbottomrule
	\end{tabularx}
	\caption{\textsc{antp}-\textsc{recp} syncretism of \textsc{recp} origin in Oceanic languages}
	\label{tab:ch7:recp-antp-oceanic}
\end{table}

The uncertainty of the \isi{diachronic development} of antipassive-reciprocal syncretism described for the Oceanic languages also extends to certain Turkic languages. The suffix \example{*-š} in Common Turkic\il{Turkic, Common} is generally believed to have had functions related to reciprocity\is{reciprocal voice}\is{reciprocal origin} and \isi{plurality of relations} (see \sectref{diachrony:recp2refl} and \sectref{diachrony:recp2antc}), and in at least two descendant languages the suffix in question has developed an antipassive\is{antipassive voice} function. In \ili{Tatar} the antipassive\is{antipassive voice} function of the reflex \example{-š} is rather productive\is{productivity} and has already been exemplified in \sectref{sec:simple-syncretism:antp-recp} (see \tabref{tab:ch4:antp-recp} on page \pageref{tab:ch4:antp-recp}), while the antipassive\is{antipassive voice} function of the reflex \example{-s} in \ili{Yakut} is considerably more restricted \citep[238]{nedjalkov:2007d}. In the spirit of \citet{janic:2016}, \citet{gandon:2018} argues that the reciprocal\is{reciprocal voice} and antipassive\is{antipassive voice} functions in these languages evolved independently of each other from a general function pertaining to \isi{plurality of relations}. However, considering the very limited distribution of the antipassive\is{antipassive voice} function among the Turkic languages, the reciprocal\is{reciprocal voice} function\is{reciprocal origin} most likely developed prior to the antipassive\is{antipassive voice} function, which is probably an innovation in Tatar and \ili{Yakut}. This development is illustrated in \tabref{tab:ch7:recp-antp-turkic}. \citet[1214]{kuular:2007} briefly describes a “detransitive”\is{detransitivisation} use of the suffix \example{-š} in \ili{Tuvan} whereby “[a] direct object\is{object, direct} is transformed into a non-direct object”.\is{object, indirect} However, it is unclear if the suffix simply entails a change in \isi{language-specific} \isi{argument marking} or if it indicates that the “non-direct object”\is{object, indirect} is less likely to be expressed syntactically and thereby qualifies as antipassive\is{antipassive voice}.

\begin{table}
	\setlength{\tabcolsep}{3.1pt}
	\begin{tabularx}{\textwidth}{rclllll}
		\lsptoprule
		Common Turkic\il{Turkic, Common} & \example{*-š} & \multicolumn{2}{l}{\textsc{recp} (+ \textsc{por})} & \multicolumn{1}{r}{→} & \textsc{antp} & \\
		\midrule 
		\ili{Tatar} & \example{-š} & \example{sug-əš-} & ‘to hit e.o.’ & & \example{jaz-əš-} & ‘to write [sth.]’ \\
		\ili{Yakut} & \example{-s} & \example{kuot-us-} & ‘to outrun e.o.’ & & \example{kuot-us-} & ‘to outrun [sb.]’ \\
		\lspbottomrule
	\end{tabularx}
	\caption{\textsc{antp}-\textsc{recp} syncretism of \textsc{recp} origin in Turkic languages}
	\label{tab:ch7:recp-antp-turkic}
\end{table}

Additionally, as argued in \sectref{diachrony:recp2refl}, a reciprocal\is{reciprocal voice} suffix \example{*-nci} can be reconstructed\is{reconstruction} rather reliably for \ili{Proto-Gunwinyguan}. In the descendant Gunwinyguan language \ili{Nunggubuyu} (\lang{au}) the reflex \example{-nʸji}\is{reciprocal origin} seems to have developed an antipassive\is{antipassive voice} function, though it is worth noting that this function is very restricted in the language. The only two examples of the phenomenon in the language provided by \citet[391ff.]{heath:1984} are those presented in \tabref{tab:ch7:recp-antp-nunggubuyu}. 

\begin{table}
	\setlength{\tabcolsep}{2.8pt}
	\begin{tabularx}{\textwidth}{rcllll}
		\lsptoprule
		P.-Gunwinyg.\il{Proto-Gunwinyguan} & \example{*-nci} & \textsc{recp} & \multicolumn{1}{r}{→} & \textsc{antp} & \\
		\midrule 
		\ili{Nunggubuyu} & \example{-nʸji} & \example{na-nʸji-} & ‘to see e.o.’ & \example{warguri-nʸji-} & ‘to carry [sth.]’ \\
		& & \example{yal̲giwa-nʸji-} & ‘to pass e.o.’ & \example{lharma-nʸji-} & ‘to chase [sth.]’ \\
		\lspbottomrule
	\end{tabularx}
	\caption{\textsc{antp}-\textsc{recp} syncretism of \textsc{recp} origin in Nunggubuyu}
	\label{tab:ch7:recp-antp-nunggubuyu}
\end{table}

Furthermore, as already discussed in \sectref{sec:complex-syncretism:appl-antp-recp}, the suffix \example{-ut} in the Eskimo language Central Alaskan Yupik\il{Yupik, Central Alaskan} (\lang{na}) can serve as voice marking in not only the antipassive\is{antipassive voice} and reciprocal\is{reciprocal voice} voices, but also in the applicative\is{applicative voice} voice, commonly with a comitative\is{comitativity} function. In fact, \citet[841]{fortescue:2007} argues that the suffix is “an original applicative\is{applicative voice} formant”,\is{applicative origin} a use retained throughout the Eskimo-Aleut language family. \citeauthor{fortescue:2007} reconstructs\is{reconstruction} the applicative-reciprocal suffix \example{*-utə} for \ili{Proto-Eskimo}, as both functions can be found in all descendant languages (see also \citealt[431]{fortescue:al:1994}). In contrast, the antipassive\is{antipassive voice} use of the suffix does not appear to be widespread and is, for instance, absent in the Inuit languages West Greenlandic\il{Greenlandic, West} \citep{schmidt:2003} and \ili{Inuktitut} \citep{spreng:2006}. Moreover, in Central Alaskan Yupik\il{Yupik, Central Alaskan} the antipassive\is{antipassive voice} use of \example{-ut} is restricted to a “rather limited number of stems”, unlike the applicative\is{applicative voice} and reciprocal\is{reciprocal voice} uses \citep[1109]{miyaoka:2012}. Evidently, the antipassive\is{antipassive voice} function of the suffix \example{-ut} represents an innovation that has evolved from applicative-reciprocal syncretism,\is{reciprocal origin} as illustrated in \tabref{tab:ch7:recp-antp-yupik} \citep[1092f.]{miyaoka:2012}. Thus, the evolution of antipassive-reciprocal syncretism in Central Alaskan Yupik\il{Yupik, Central Alaskan} is slightly different from that discussed above for Bantu and Turkic languages as well as \ili{Nunggubuyu}. 

\begin{table}
	\setlength{\tabcolsep}{5.7pt}
	\begin{tabularx}{\textwidth}{rcllll}
		\lsptoprule
		\ili{Proto-Eskimo} & \example{*-utə} & & \textsc{recp} (+ \textsc{appl}) & \multicolumn{1}{r}{→} & \textsc{antp} \\
		\midrule 
		C. A. Yupik\il{Yupik, Central Alaskan} & \example{-ut} &  \example{ikayu-ut-} & ‘to help e.o.’ & & ‘to help [sb.]’ \\
		& & \multicolumn{2}{l}{(cf. \example{an-ut-} ‘to go out with sb.’)} & & \\
		\lspbottomrule
	\end{tabularx}
	\caption{\textsc{antp}-\textsc{recp} syncretism of \textsc{recp} origin in C. A. Yupik}
	\label{tab:ch7:recp-antp-yupik}
\end{table}

As described in \sectref{sec:simple-syncretism:antp-recp}, the Kordofanian language \ili{Lumun} (\lang{af}) possesses the affixes \example{-(a)rɔ} (with the allomorphs\is{allomorphy} \example{<ar>}, \example{<rɔ>} and \example{-rɔ}) and \example{-ttɔ} (with the allomorph\is{allomorphy} \example{<ttɔ>}) which can both serve as voice marking in the reciprocal\is{reciprocal voice} and antipassive\is{antipassive voice} voices. Cognates of these affixes can be found in the related language \ili{Dagik}, in which \example{<(ə)r>} indicates \isi{sociativity} and reciprocity\is{reciprocal voice}, and \example{<-(ə)tː>} indicates \isi{pluractionality}, iterativity,\is{iterative} habituality,\is{habitual} durativity\is{durative} and also reciprocity\is{reciprocal voice} in combination with the former affix \citep[98ff., 128ff.]{vanderelst:2016}. Neither affix in \ili{Dagik} seems to have an antipassive\is{antipassive voice} function. In light of this (rather limited) data and the other various descriptions of antipassive-reciprocal syncretism of \isi{reciprocal origin} in other genera\is{genus}, it is possible that the reciprocal\is{reciprocal voice} function of the affixes \example{-(a)rɔ} and \example{-ttɔ} in \ili{Lumun} evolved prior to the antipassive\is{antipassive voice} function, though the exact chronology of the functions remains highly tentative for the time being.

In terms of functional explanations for antipassive-reciprocal syncretism,\is{reciprocal origin} both \citet{janic:2016} and \citet{gandon:2018} argue that the antipassive\is{antipassive voice} and reciprocal\is{reciprocal voice} functions evolved independently from a general function pertaining to \isi{plurality of relations}, at least in the Oceanic and Turkic languages. While \citet{janic:2016} does not address the diachrony in detail, \citet{gandon:2018} specifically argues that the antipassive\is{antipassive voice} function of the Common Turkic\il{Turkic, Common} suffix \example{*-š} evolved from \isi{plurality of actions} due to its close relationship to iterativity,\is{iterative} unlike reciprocity\is{reciprocal voice} associated with \isi{plurality of participants}. Such association between antipassivity\is{antipassive voice} and \isi{aspect} is typologically well known (\citealt{polinsky:2017}). By contrast, \citet[759]{bostoen:al:2015} acknowledge similarities between antipassivity\is{antipassive voice} and \isi{plurality of actions}, but ultimately argue that “it is the progressive destitution of the second participant of the coordinated plural \isi{subject} in reciprocal\is{reciprocal voice} constructions that ultimately leads to the antipassive\is{antipassive voice}”, at least among the Bantu languages, and they thus link the rise of antipassivity\is{antipassive voice} to \isi{plurality of participants} like reciprocity\is{reciprocal voice}. In other words, reciprocal referents\is{semantic referent} go from being equally prominent to being differentiated according to prominence (for instance, by \isi{word order} or a comitative\is{comitativity} phrase language-specifically\is{language-specific}) before the least prominent referents\is{semantic referent} are eventually omitted due to lack of prominence leading to antipassivity\is{antipassive voice}. Such scenario is perhaps best conceivable with lexically reciprocal\is{lexical reciprocal} verbs, e.g. ‘the man and his friends meet each other’ → ‘the man meets with his friends’ → ‘the man meets his friends’ → ‘the man meets [his friends]’. \citet{sanso:2017, sanso:2018} adopts a somewhat similar approach (\sectref{diachrony:refl2antc}), highlighting \isi{sociativity} and \isi{comitativity} as facilitating factors in the development from reciprocal\is{reciprocal voice} to antipassive\is{antipassive voice} (see \tabref{fig:ch7:antp-refl} on page \pageref{fig:ch7:antp-refl}). In any case, it can be difficult to effectively distinguish the explanations proposed by \citet{janic:2016}, \citet{gandon:2018}, \citet{bostoen:al:2015}, and \citet{sanso:2017, sanso:2018} from each other in practice due to the close relationship between reciprocity\is{reciprocal voice}\is{reciprocal origin} and \isi{plurality of relations}, and the explanations do not necessarily exclude one another. This section importantly shows that the reciprocal\is{reciprocal voice} function of the voice marking discussed for the various languages above most likely evolved prior to the antipassive\is{antipassive voice} function. The exact chronology of the functions pertaining to \isi{plurality of participants} (including reciprocity\is{reciprocal voice}) remains a topic of future research.

\subsection{From reciprocal to causative} \label{diachrony:recp2caus}
As demonstrated in \sectref{sec:complex-syncretism:caus-pass}, the suffix \example{-kaka} in the Arawakan language \ili{Yine} (\lang{sa}) can serve as voice marking in both the causative\is{causative voice} and reciprocal\is{reciprocal voice} voices (see \tabref{tab:ch5:caus-recp-pass} on page \pageref{tab:ch5:caus-recp-pass}). Moreover, as mentioned in \sectref{diachrony:recp2refl}, \citet{wise:1990} reconstructs\is{reconstruction} a reciprocal\is{reciprocal voice} function for the \ili{Proto-Arawakan} suffix \example{*-kʰakʰ} whence the Yine suffix derives which indicates a \isi{reciprocal origin} for the causative-reciprocal syncretism in the language. This presumed development is illustrated in \tabref{tab:ch7:recp-caus-yine} \citep[269, 271]{hanson:2010}. However, it is worth observing that \citet{wise:1990} and \citet{payne:2002} both suggest that the development has been facilitated by comitative\is{comitativity} applicativity\is{applicative voice}, at least among Pre-Andine Arawakan languages (\sectref{diachrony:appl2caus}). While \ili{Yine} does not belong to this Arawakan grouping, the possibility of an applicative\is{applicative voice} stage is presented in parentheses in \tabref{tab:ch7:recp-caus-yine}.

\begin{table}
	\setlength{\tabcolsep}{2.6pt}
	\begin{tabularx}{\textwidth}{rcll}
		\lsptoprule
		P.-Arawakan\il{Proto-Arawakan} & \example{*-kʰakʰ} & \textsc{recp} (→ \textsc{appl}?) → & \textsc{caus} \\
		\midrule 
		\ili{Yine} & \example{-kaka} & \example{-hiylaka-kaka} ‘to hit e.o.’ & \example{-halna-kaka} ‘to make sth. fly’ \\
		\lspbottomrule
	\end{tabularx}
	\caption{\textsc{caus}-\textsc{recp} syncretism of \textsc{recp} origin in Yine}
	\label{tab:ch7:recp-caus-yine}
\end{table}



So far it has only been possible to find potential evidence for \isi{diachronic development} from reciprocal\is{reciprocal voice} to causative\is{causative voice}\is{reciprocal origin} in two languages other than \ili{Yine}, the Atlantic language \ili{Wolof} (\lang{af}) and the Turkic language \ili{Khakas} (\lang{ea}). As briefly noted in \sectref{sec:simple-syncretism:caus-antp}, the former language features the suffix \example{-e} with causative\is{causative voice} and reciprocal\is{reciprocal voice} functions (in addition to applicative\is{applicative voice} and antipassive\is{antipassive voice} functions). \citet[304]{creissels:nouguier-voisin:2008} argue that “reciprocal\is{reciprocal voice} \example{-e} may be the reflex of an ancient suffix \example{*-e} whose possible uses included several varieties of \isi{co-participation}”. This diachronic\is{reciprocal origin} scenario would be very similar to that mentioned for the Pre-Andine Arawakan languages above, though \citeauthor{creissels:nouguier-voisin:2008} admit that more comparative research is needed to confirm their proposal. The \ili{Khakas} case is analogous to the \ili{Yine} and \ili{Wolof} cases. In this language the suffix \example{-s} has been observed to have a causative\is{causative voice} function with the two verbs in \tabref{tab:ch4:recp-caus-Khakas}. As discussed in the previous section as well as in \sectref{diachrony:recp2refl} and \sectref{diachrony:recp2antc}, the Common Turkic\il{Turkic, Common} suffix \example{*-š} whence \ili{Khakas} \example{-s} descends is generally believed to have had functions pertaining to reciprocity\is{reciprocal voice}\is{reciprocal origin} and \isi{plurality of relations} (cf. \example{hucahta-s-} ‘to embrace e.o.’, \citealt[1100]{arikoglu:2007}). \citet[71]{gandon:2013} briefly notes that a similar phenomenon is exemplified by \citet[707]{oner:2007} for \ili{Tatar} (cf. \example{kal-} ‘to stay’ ↔ \example{kal-ış-} ‘to leave sth.’) but goes on to argue that the translation of the latter verb here seems to be incorrect as \citet{oner:2009} translates it ‘to stay behind’ elsewhere.

\begin{table}
	\setlength{\tabcolsep}{3.8pt}
	\begin{tabularx}{\textwidth}{llllll}
		\lsptoprule
		\multicolumn{6}{l}{\ili{Khakas} (\citealt[1101]{arikoglu:2007}; \citealt[71]{gandon:2013})} \\
		\midrule
		\textsc{caus} & \example{art-} & ‘to stay’ & ↔ & \example{art-ıs-} & ‘to leave sth.’ (i.e. ‘to make sth. stay’) \\
		\textsc{caus} & \example{em-} & ‘to suckle’ & ↔ & \example{em-ĭs-} & ‘to breastfeed sb.’ (i.e. ‘to make sb. suckle’) \\
		\lspbottomrule
	\end{tabularx}
	\caption{Causative use of the suffix \example{-s} in Khakas}
	\label{tab:ch4:recp-caus-Khakas}
\end{table}

In light of the evidence presented above, it would seem that some sense of \isi{comitativity} or \isi{co-participation} is central to the \isi{diachronic development} from reciprocal\is{reciprocal voice} to causative\is{causative voice}, and this matter is discussed in more detail in \sectref{diachrony:appl2caus}.

\subsection{From reciprocal to applicative} \label{diachrony:recp2appl}
Evidence for a \isi{diachronic development} from reciprocal\is{reciprocal voice} to applicative\is{applicative voice}\is{reciprocal origin} is scant and the phenomenon has received little attention in the literature, yet the development does appear to have taken place in at least two genera\is{genus} in the language sample. For instance, as already discussed in \sectref{diachrony:recp2antc}, the \ili{Proto-Bantu} suffix \example{*-an} is widely associated with reciprocity\is{reciprocal voice}, \isi{sociativity}, and other functions related to \isi{plurality of relations}. While these functions are attested for reflexes of the suffix in a wide range of descendant Bantu languages, it seems that reflexes of the suffix have developed a proper comitative\is{comitativity} and/or instrumental applicative\is{applicative voice} function only sporadically\is{reciprocal origin} (\citealt[753ff.]{bostoen:al:2015};; \citealt[138f.]{dom:al:2016}). This development is illustrated in \tabref{tab:ch7:recp-appl-bantu} (Duala = \citealt[140f.]{ittmann:1939} via \citealt[341]{maslova:2007}; Kinyarwanda = \citealt[160, 177]{aksenova:1994} via \citealt[275]{nedjalkov:2007d}).

\begin{table}
	\setlength{\tabcolsep}{2.2pt}
	\begin{tabularx}{\textwidth}{rcllll}
		\lsptoprule
		\ili{Proto-Bantu} & \example{*-an} & \textsc{recp} (+ \textsc{por}) & \multicolumn{1}{r}{→} & \textsc{appl} & \\
		\midrule 
		\ili{Duala} & \example{-ne} & \example{énè-ne } & ‘to see e.o.’ & \example{dípà-ne } & ‘to beat sb. with sth.’ \\
		Kinyarw.\il{Kinyarwanda} & \example{-an} & \example{-kurèb-an-} & ‘to look at e.o.’ & \example{-kôr-an-} & ‘to work with sth./sb.’ \\
		\lspbottomrule
	\end{tabularx}
	\caption{\textsc{appl-recp} syncretism of \textsc{recp} origin in Bantu languages}
	\label{tab:ch7:recp-appl-bantu}
\end{table}

Likewise, as discussed in the previous sections, reflexes of the Common Turkic\il{Turkic, Common} suffix \example{*-š} are in descendant languages widely associated with functions pertaining to \isi{plurality of relations} like the \ili{Proto-Bantu} suffix \example{*-an} discussed above, including reciprocity\is{reciprocal voice}. In some Turkic languages reflexes of the proto-suffix \example{*-š} appear to have developed a proper comitative\is{comitativity} applicative\is{applicative voice} function,\is{reciprocal origin} for instance in \ili{Yakut} \citep[107]{nedjalkov:2007b} and \ili{Tuvan} \citep[1201]{kuular:2007}, as illustrated in \tabref{tab:ch7:recp-appl-turkic}. By comparison, in \ili{Karachay-Balkar} the suffix \example{-š} has a sociative\is{sociativity} function (e.g. \example{oηsun-uš-} ‘to be pleased together’, \citealt[1001]{nedjalkov:nedjalkov:2007}) but no comitative\is{comitativity} applicative\is{applicative voice} function, and in \ili{Kirghiz} the suffix has neither function \citep[1233]{nedjalkov:2007e}.

\begin{table}
	\setlength{\tabcolsep}{6.6pt}
	\begin{tabularx}{\textwidth}{rcllll}
		\lsptoprule
		Common Turkic\il{Turkic, Common} & \example{-š} & & \textsc{recp} (+ \textsc{por}) & → & \textsc{appl} \\
		\midrule 
		\ili{Yakut} & \example{-s} & \example{ölör-üs-} & ‘to kill e.o.’ & & ‘to kill sb. with sb.’ \\
		\ili{Tuvan} & \example{-š} & \example{üpte-š-} & ‘to rob e.o.’ & & ‘to rob sb. with sb.’ \\
		\lspbottomrule
	\end{tabularx}
	\caption{\textsc{appl-recp} syncretism of \textsc{recp} origin in Turkic languages}
	\label{tab:ch7:recp-appl-turkic}
\end{table}

Observe that the \ili{Yakut} and \ili{Tuvan} verbs in \tabref{tab:ch7:recp-appl-turkic} also can have a sociative\is{sociativity} meaning (‘to kill sb. together’ and ‘to rob sb. together’, respectively), and so can the suffix \example{-an} in \ili{Kinyarwanda} (cf. \example{-guhîng-an-} ‘to cultivate sth. together’, \citealt[15]{coupez:1985}), while it is unclear to which extent this function is productive\is{productivity} for the suffix \example{-ne} in \ili{Duala}. This syncretism clearly illustrates the close semantic relation between reciprocity\is{reciprocal voice} and \isi{sociativity} (i.e. \isi{plurality of participants}) on the one hand and comitative\is{comitativity} applicativity\is{applicative voice} on the other hand. In turn, comitative\is{comitativity} applicativity\is{applicative voice} is closely related to instrumental applicativity\is{applicative voice}, as further discussed in \sectref{diachrony:caus2appl} and \sectref{diachrony:appl2caus} (see also, e.g., \citealt{creissels:nouguier-voisin:2008} on \isi{co-participation}). These semantic links provide a plausible explanation for the rise of applicative-reciprocal syncretism in the languages discussed in this section. 

\section{Anticausative origin} \label{diachrony:anticausative}
Prospects of an \isi{anticausative origin} for voice syncretism are generally associated specifically with passive-anticausative syncretism, as \isi{diachronic development} from anticausative\is{anticausative voice} to passive\is{passive voice} is often regarded as an intermediary step in the evolution from reflexive\is{reflexive voice} to passive\is{passive voice}, notably among Indo-European languages (\sectref{diachrony:refl2pass}). However, as shown in \sectref{diachrony:antc2pass}, passive-anticausative syncretism can also have an \isi{anticausative origin} not associated with reflexivity\is{reflexive voice}. Furthermore, \citet{inglese:2020} has argued for an \isi{anticausative origin} for reflexive-anticausative and reciprocal-anticausative syncretism in the extinct Indo-European language \ili{Hittite}, as discussed in the next two sections. 

\subsection{From anticausative to reflexive} \label{diachrony:antc2refl}
While development from reflexive\is{reflexive voice} to anticausative\is{anticausative voice} is well-attested (\sectref{diachrony:refl2antc}), evidence for the opposite development\is{anticausative origin} is scant and seemingly restricted to the extinct Indo-European language \ili{Hittite}. As described by \citet{inglese:2020}, this language features a middle voice characterised by suffixation which fuses person \isi{agreement} and various other functions, including passive\is{passive voice}, reflexive\is{reflexive voice}, reciprocal\is{reciprocal voice}, and anticausative\is{anticausative voice} functions. These four voice functions are illustrated in \tabref{tab:ch4:antc-refl-hittite}. The translations on the left side of the bidirectional arrows in the table denote the meanings of the respective verbs when used without a middle suffix,\is{middle syncretism} while the translations on the right side of the arrows denote the meanings of the respective verbs when used with a middle suffix.\is{middle syncretism} \citet[240]{inglese:2020} ultimately argues that the anticausative\is{anticausative voice} function\is{anticausative origin} has given rise to the other three functions “through independent semantic extensions”. In turn, the anticausative\is{anticausative voice} function itself is believed to have evolved from \isi{media tantum} (i.e. deponent verbs) which always feature a middle suffix\is{middle syncretism} and cannot be used without one (\citealt[241ff.]{inglese:2020};; see also \citealt{luraghi:2012}).

\begin{table}
	\setlength{\tabcolsep}{5pt}
	\begin{tabularx}{\textwidth}{lllll}
		\lsptoprule
		\multicolumn{5}{l}{\ili{Hittite} \citep[133, 142, 148ff., 155f., 209]{inglese:2020}} \\
		\midrule
		\textsc{pass} & \example{istāp-} & ‘to close sth.’ & ↔ & [\textsc{mid}] ‘to be closed [by sb.]’ \\
		\textsc{pass} & \example{tamāss-} & ‘to oppress sb.’ & ↔ & [\textsc{mid}] ‘to be oppressed [by sb.]’ \\
		\textsc{refl} & \example{suppiyahh-} & ‘to purify sb.’ & ↔ & [\textsc{mid}] ‘to purify self’ \\
		\textsc{refl} & \example{das(sa)nu-} & ‘to strengthen sb.’ & ↔ & [\textsc{mid}] ‘to strengthen self’ \\
		\textsc{recp} & \example{zahh-} & ‘to hit sth.’ & ↔ & [\textsc{mid}] ‘to hit e.o.’ \\
		\textsc{recp} & \example{epp-} & ‘to take sth.’ & ↔ & [\textsc{mid}] ‘to take e.o.’ \\
		\textsc{antc} & \example{zinni-} & ‘to end sth.’ & ↔ & [\textsc{mid}] ‘to end’ \\
		\textsc{antc} & \example{istāp-} & ‘to close sth.’ & ↔ & [\textsc{mid}] ‘to close’ \\
		\lspbottomrule
	\end{tabularx}
	\caption{Middle syncretism in Hittite}
	\label{tab:ch4:antc-refl-hittite}
\end{table}

\citet{inglese:2020} favours an \isi{anticausative origin} for the passive\is{passive voice}, reciprocal\is{reciprocal voice}, and reflexive\is{reflexive voice} functions of the middle suffixes\is{middle syncretism} in \ili{Hittite} for a number of reasons, the most important ones of which are here summarised in brief. Firstly, \citeauthor{inglese:2020}’s (\citeyear[231]{inglese:2020}) data from different diachronic stages of the \ili{Hittite} language “clearly shows that the passive\is{passive voice} function is on the rise in the history of \ili{Hittite}, so that it appears to be a relatively younger development, hence unlikely to be the original function of the \isi{middle voice}”. Secondly, the reciprocal\is{reciprocal voice} function is also “an unlikely candidate” for the original function of the middle\is{middle syncretism} because it is “among the least frequent functions associated with the middle voice” \citep[230]{inglese:2020}. Moreover, the middle\is{middle syncretism} suffixes in \ili{Hittite} are not associated with \isi{plurality of relations} which alongside reciprocity\is{reciprocal voice} is known to serve as an origin for other voices (\sectref{diachrony:reciprocal}). Thirdly, following \citet{luraghi:2010, luraghi:2012}, \citet[230]{inglese:2020} argues that “reflexivity\is{reflexive voice} can hardly lie at the core of the \ili{Hittite} \isi{middle voice} system” because it “remains a quantitatively marginal function throughout the history of the language” and “middle\is{middle syncretism} forms with reflexive\is{reflexive voice} reading are reinforced by the particle \example{=za} since their earliest attestation”. In fact, \citet[83, 147]{inglese:2020} attests only two verbs that can have a reflexive\is{reflexive voice} meaning when used with a middle suffix in his corpus of original \ili{Hittite} texts, and in both cases the verbs are accompanied by the particle \example{=za} which also can be used on its own without a middle\is{middle syncretism} suffix to denote reflexivity\is{reflexive voice}. In his corpus of copies of \ili{Hittite} texts, \citet[148]{inglese:2020} only attests six additional verbs of the same kind which “are also quite systematically associated with the particle \example{=za}”. 

A probable developmental scenario from anticausative to reflexive\is{reflexive voice}\is{anticausative origin} is discussed here, while plausible scenarios of development from anticausative\is{anticausative voice} to reciprocal\is{reciprocal voice} and from anticausative\is{anticausative voice} to passive\is{passive voice} are described in the following two sections. \citet[235]{inglese:2020} suggests that the reflexive\is{reflexive voice} function of the middle\is{middle syncretism} suffixes in \ili{Hittite} has evolved from the anticausative\is{anticausative voice} function facilitated by autocausativity,\is{autocausative} e.g. ‘if some enemy mobilizes [\example{niniktari}.\textsc{prs.3sg.mid}]’ (i.e. ‘to rise’, the verb \example{ninik-} has the meaning ‘to raise sth.’ without a middle\is{middle syncretism} suffix). According to \citet[236]{inglese:2020}, “[o]ne can speculate that the possibility of animate\is{animacy} subjects\is{subject} to occur with otherwise [anticausative]\is{anticausative voice} verbs led to the expansion of the \isi{autocausative} use, hence providing the natural \isi{bridging context} to reflexive\is{reflexive voice} situations proper, in which the \isi{subject} not only initiates the event, but is also fully affected by it”\is{affectedness} (cf. the reflexive\is{reflexive voice} examples in \tabref{tab:ch4:antc-refl-hittite}). This diachronic scenario is essentially the exact opposite of that attested for reflexive-anticausative syncretism of reflexive origin\isi{reflexive origin} discussed in \sectref{diachrony:refl2antc}, and boils down to a shift in \isi{animacy} and thereby a shift in the capability to act upon oneself.

\subsection{From anticausative to reciprocal} \label{diachrony:antc2recp}
Both the anticausative\is{anticausative voice} and reciprocal\is{reciprocal voice} voices are known to commonly evolve from a reflexive\is{reflexive voice} voice\is{reflexive origin} (\sectref{diachrony:refl2recp} and \sectref{diachrony:refl2antc}), yet there is some evidence for a \isi{reciprocal origin} of reciprocal-anticausative syncretism (\sectref{diachrony:recp2antc}) and in this section potential evidence for an \isi{anticausative origin} for the same kind of voice syncretism is considered. As already discussed in the previous section, \citet{inglese:2017, inglese:2020} argues that the reciprocal\is{reciprocal voice} function of middle\is{middle syncretism} suffixes in the extinct Indo-European language \ili{Hittite} has evolved from an earlier anticausative\is{anticausative origin} function and not vice versa. \citeauthor{inglese:2020} considers two potential scenarios for this development. In one scenario the reciprocal\is{reciprocal voice} voice has evolved from a reflexive\is{reflexive voice} voice which in turn has evolved from an anticausative\is{anticausative voice} voice, as described in the previous section. However, \citet[238]{inglese:2020} considers this scenario unlikely as the reflexive\is{reflexive voice} function of the middle\is{middle syncretism} suffixes is “extremely limited in O[ld] H[ittite]” and restricted largely to the two verbs in \tabref{tab:ch4:antc-refl-hittite} in the previous section. Instead, \citet[238]{inglese:2020} prefers a scenario in which the reciprocal\is{reciprocal voice} function of the middle\is{middle syncretism} suffixes in \ili{Hittite} evolved directly from the anticausative\is{anticausative voice} function\is{anticausative origin} initially among lexically reciprocal\is{lexical reciprocal} verbs, e.g. ‘the gods gathered [\example{taruppantat}.\textsc{pst.3.pl.mid}] all together’. \citet[239]{inglese:2020} suggests that “[d]ue to the specific interplay of the verb’s inherent reciprocal\is{reciprocal voice} meaning, the \isi{middle voice}’s \isi{autocausative} meaning, and the plurality of the subjects\is{subject} involved […] can be conceived as describing a situation in which multiple entities bring about a change in spatial configuration with respect to one another” and “[f]rom such contexts, a reciprocal\is{reciprocal voice} non-spatial meaning can be easily inferred as primary, and the reciprocal\is{reciprocal voice} meaning can eventually be extended to non-spatial situations”. Thus, the scenario hypothesised by \citeauthor{inglese:2020} basically represents a reverse development in comparison to the development from reciprocal\is{reciprocal voice} to anticausative\is{anticausative voice} described in \sectref{diachrony:recp2antc}: \example{tarupp-} ‘to gather sth.’ → \example{tarupp-} [\textsc{mid}] ‘to gather’ and by extension → \example{zahh-} [\textsc{mid}] ‘to hit e.o.’

\subsection{From anticausative to passive} \label{diachrony:antc2pass}
Voice development from anticausative\is{anticausative voice} to passive\is{anticausative origin} is perhaps best known as an intermediary step in the evolution from reflexive\is{reflexive voice} to passive\is{passive voice}, notably among Indo-European (\sectref{diachrony:refl2pass}). The voice development in these languages can also be characterised as syncretic reflexive-anticausative voice marking developing a passive\is{passive voice} function, because the marking in question generally had both reflexive\is{reflexive voice} and anticausative\is{anticausative voice} functions when the passive\is{passive voice} function first evolved. A similar development seems to have taken place in the Tibeto-Burman language \ili{Dhimal} (\lang{ea}; \sectref{sec:simple-syncretism:pass-antc}). In contrast, voice development from anticausative\is{anticausative voice} to passive\is{anticausative origin} with no involvement of reflexivity\is{reflexive voice} has received little attention in the literature and examples of the phenomenon are rare. A clear case of such development can, however, be found in \ili{Korean} (\lang{ea}). As described by \citet{ahn:yap:2017}, the suffix \example{-aci/-eci} in this language has a number of functions, most notably “spontaneous middle” (anticausative\is{anticausative voice}), \isi{inchoative}, passive\is{passive voice}, and “\isi{facilitative}” (\isi{potential passive}). According to \citet[444ff.]{ahn:yap:2017}, the suffix ultimately derives from the verb \example{ti-} ‘to fall, sink’ which underwent a process of \isi{grammaticalisation} starting in the 15th century and developed an anticausative\is{anticausative voice} function\is{anticausative origin} when preceded by the infinitival suffix \example{-a/-e}. During the 17th century the initial consonant of the grammaticalised\is{grammaticalisation} suffix \example{-ti} underwent palatalisation and the innovative suffix \example{-aci/-eci} developed an \isi{inchoative} function \citep[446ff.]{ahn:yap:2017}. In the following century the suffix went on to develop a passive\is{passive voice} function as well \citep[451ff.]{ahn:yap:2017}. This development is illustrated in \tabref{tab:ch7:antc2pass-korean}. 

\begin{table}
	\setlength{\tabcolsep}{4pt}
	\begin{tabularx}{\textwidth}{lcllcc}
		\lsptoprule
		\ili{Korean} & & & & & \\
		\midrule
		15th c. & \multicolumn{2}{l}{\example{ti-} ‘to fall, sink’} & & & \\
		& ↓ & & & & \\
		& \example{-e/-a + -ti} & \example{sot-a-ti-} & ‘to pour away’ & \textsc{antc} & \\
		& ↓ & & (cf. \example{sot-} ‘to pour sth. out’) & ↓ & \\
		17th c. & \example{-aci/-eci} & \example{palk-acy-} & ‘to become bright’ & ↓ & \textsc{inch} \\
		& ⋮ & & (cf. \example{palk-} ‘to be bright’ & ↓ & ↓ \\
		18th c. & ⋮ & \example{mwunh-ecy-} & ‘to be destroyed [by sb.]’ & \multicolumn{2}{c}{\textsc{pass}} \\
		& & & (cf. \example{mwunh-} ‘to destroy sth.’) & & \\
		\lspbottomrule
	\end{tabularx}
	\caption{\textsc{pass-antc} syncretism of \textsc{antc} origin in Korean}
	\label{tab:ch7:antc2pass-korean}
\end{table}



Observe that \example{-acy/-ecy} are simply phonologically conditioned allomorphs\is{allomorphy} of \example{-aci/-eci}. Moreover, note that the 15th century represents Middle Korean and the 17th and 18th centuries represent Early Modern Korean, yet each of the three functions remain productive\is{productivity} in contemporary Korean as well \citep[459]{ahn:yap:2017}. A \isi{potential passive} function mentioned above did not evolve until the 20th century and is not covered by \tabref{tab:ch7:antc2pass-korean}. \citet[451]{ahn:yap:2017} argue that “[e]s\-sen\-tial\-ly, extended uses of \example{-eci} from \isi{intransitive} verb contexts to \isi{transitive} ones gave rise to passive\is{passive voice} voice usage”, and highlight inchoativity\is{inchoative} as a facilitating factor in the process: “[t]he semantic property that links the \isi{inchoative} middle with the passive\is{passive voice} is the complete lack of volitional\is{volition(ality)} initiation by the \isi{subject}, which in both \isi{inchoative} and passive\is{passive voice} constructions is the Patient\is{patient} of the event”. In more general terms, \citet[45]{haspelmath:1990} notes that the passive\is{passive voice} essentially is a “generalization of the anticausative\is{anticausative voice} in that it is not restricted to spontaneously occurring processes” but comes to feature an additional \isi{semantic participant}. 

\subsection{From anticausative to antipassive} \label{diachrony:antc2antp}
There is currently no clear evidence for a development from anticausative\is{anticausative voice} to antipassive\is{antipassive voice}\is{anticausative origin} in any language. \citet[225]{haspelmath:2003} tentatively links the two voices to each other but also explicitly states that “diachronic data are insufficient”. Nevertheless, it might be worth mentioning that the Surmic language \ili{Majang} (\lang{af}) features antipassive-anticausative voice marking with no other apparent voice functions (see examples \ref{ex:Majang:bite:a}--\ref{ex:Majang:break:b} on page \pageref{ex:Majang:bite:a}). Unfortunately, however, there are currently not enough diachronic data available to establish the exact development of antipassive-anticausative syncretism in this language. 

\section{Passive origin} \label{diachrony:passive}
Evidence for voice syncretism of passive\is{passive voice} origin is sparse,\is{passive origin} and the literature on such diachrony equally so. However, the following sections demonstrate that there is some evidence suggesting that passive\is{passive voice} voice marking can potentially develop a reflexive\is{reflexive voice}, reciprocal\is{reciprocal voice}, or anticausative\is{anticausative voice} function.

\subsection{From passive to reflexive} \label{diachrony:pass2refl}
Discussions of a \isi{passive origin} for passive-reflexive syncretism in the literature seem to be restricted to a single language, the Uto-Aztecan language \ili{Tarahumara} (\lang{na}), in which the “passive-impersonal”\is{impersonalisation} suffix \example{-ru} “has extended to reflexive\is{reflexive voice} use”\is{passive origin} (\citealt[803]{langacker:munro:1975}; see also, e.g., \citealt[18]{anderson:al:1976} and \citealt[252]{dik:1983}). This suffix derives from the \ili{Proto-Uto-Aztecan} \isi{copula} \example{*-tu} ‘to become’, and \citet[798]{langacker:munro:1975} remark that this original use is also retained in \ili{Tarahumara} but provide no examples. The purported \isi{diachronic development} of the suffix in \ili{Tarahumara} is illustrated in \tabref{tab:ch7:pass-refl-tarahumara}. Nevertheless, it should be noted that \citeauthor{langacker:munro:1975} only provide three verbs as evidence for their claim (the two verbs in \tabref{tab:ch7:pass-refl-tarahumara} in addition to the impersonal\is{impersonalisation} verb \example{goči-ru} ‘one sleeps’), and it is unclear how widespread and productive\is{productivity} the reflexive\is{reflexive voice} function of the suffix \example{-ru} is. For comparison, the passive\is{passive voice} and impersonal\is{impersonalisation} functions of the suffix \example{-ru} are covered by \citet{caballero:2008} on Choguita Tarahumara\il{Tarahumara, Choguita} and by \citet{jara:2013} on Urique Tarahumara\il{Tarahumara, Urique} but neither author mentions any reflexive\is{reflexive voice} use. In the closely related language River Warihio\il{Warihio, River} the suffix \example{-tu} (also reflecting Proto-Uto-Aztecan \example{*-tu}) does not appear to have any reflexive\is{reflexive voice} use either \citep{armendariz:2006}. \citet[32]{burgess:1984} characterises the suffix \example{-ru} in Western Tarahumara\il{Tarahumara, Western} as “\textsc{pass/impers}[onal]\textsc{/stat}[ive]\textsc{/refl/[appl]}” but provides no reflexive\is{reflexive voice} example and does not discuss the functionality of the suffix in any more detail. Consequently, although \ili{Tarahumara} remains a candidate for passive-reflexive syncretism of \isi{passive origin}, the matter remains unresolved for the time being until more data become available.

\begin{table}
	\setlength{\tabcolsep}{2pt}
	\begin{tabularx}{\textwidth}{lclll}
		\lsptoprule
		Proto- & ‘to become’ & & & \\
		Uto-Azt.\il{Proto-Uto-Aztecan} & \example{*-tu} & \textsc{pass} & → & \textsc{refl} \\
		\midrule 
		Tarah.\il{Tarahumara} & \example{-ru} & \example{ʔa-ru} ‘to be given sth. [by sb.]’ & & \example{pago-ru} ‘to wash self’ \\
		\lspbottomrule
	\end{tabularx}
	\caption{\textsc{pass-refl} syncretism of \textsc{pass} origin in Tarahumara}
	\label{tab:ch7:pass-refl-tarahumara}
\end{table}

Another and perhaps better candidate for passive-reflexive syncretism of \isi{passive origin} is the Lowland East Cushitic language \ili{Ts’amakko} (\lang{af}) in which the suffix \example{-am} can serve as voice marking in both the passive\is{passive voice} and reflexive\is{reflexive voice} voices \citep[207ff.]{sava:2005}. As discussed in more detail in the next section, this suffix can be traced back to \ili{Proto-East-Cushitic} \example{*-am} for which an original passive\is{passive voice} function\is{passive origin} has been reconstructed\is{reconstruction} \citep{hayward:1984}. The presumed development from passive\is{passive voice} to reflexive\is{reflexive voice} is illustrated in \tabref{tab:ch7:pass-refl-tsamakko}. The suffix \example{-om} is “probably historically” composed of the inceptive suffix \example{-aw} and the passive\is{passive voice} suffix \example{-am} \citep[198]{sava:2005}. It has hitherto not been possible to find a similar reflexive\is{reflexive voice} function for reflexes of the \ili{Proto-East-Cushitic} suffix \example{*-am} in other East Cushitic languages. It is worth noting that \citet[208, 242f., 257]{sava:2005} also provides two examples of the \ili{Ts’amakko} suffix \example{-am} which seemingly qualify as anticausative\is{anticausative voice}: \example{bul-am-} ‘to separate’ in the sense ‘to go separate ways’ (cf. \example{bul-} ‘to separate sth.’) and \example{ɠonɗ-am-} ‘to break’ (cf. \example{ɠonɗ-} \textsuperscript{?}‘to break sth.’). Thus, it is possible that the development from passive\is{passive voice} to reflexive\is{reflexive voice}\is{passive origin} has been facilitated in part by anticausativity\is{anticausative voice}. In that case, the diachrony of passive-reflexive syncretism in \ili{Ts’amakko} would present a reverse version of the \isi{diachronic development} from reflexive\is{reflexive voice} to passive\is{passive voice} facilitated by anticausativity\is{anticausative voice} (\sectref{diachrony:refl2pass}) generally assumed to have taken place among Indo-European languages (\sectref{diachrony:reflexive}).

\begin{table}
	\setlength{\tabcolsep}{3.1pt}
	\begin{tabularx}{\textwidth}{rcllll}
		\lsptoprule
		\ili{Proto-East-Cushitic} & \example{*-am} & \textsc{pass} & → & \textsc{refl} & \\
		\midrule 
		\ili{Ts’amakko} & \example{-am} & \example{q’aq’-am} & & \example{šiin-am-} & ‘to smear self’ \\
		& & ‘to be cut [by sb.]’ & & \example{šuɗ-am-} & ‘to dress self’ \\
		& & & & \multicolumn{2}{l}{(cf. \example{šooh-om-} ‘to wash self’)} \\
		\lspbottomrule
	\end{tabularx}
	\caption{\textsc{pass-refl} syncretism of \textsc{pass} origin in Ts’amakko}
	\label{tab:ch7:pass-refl-tsamakko}
\end{table} 

\subsection{From passive to reciprocal} \label{diachrony:pass2recp}
Diachronic development\is{diachronic development} from passive\is{passive voice} to reciprocal\is{reciprocal voice}\is{passive origin} does not seem to have received any prior treatment in the literature. However, as briefly mentioned in \sectref{diachrony:recp2pass}, such development appears to have taken place in the Highland East Cushitic language \ili{Sidaama} (\lang{af}). In this language the suffix \example{-am} serves as voice marking in the passive\is{passive voice}, reciprocal\is{reciprocal voice}, and anticausative\is{anticausative voice} voices (see \tabref{tab:ch5:middle} on page \pageref{tab:ch5:middle}). \citet[97]{hayward:1984} observes that cognates of this suffix can be found “in nearly every Eastern Cushitic language”, mainly with a passive\is{passive voice} function, and goes on to reconstruct\is{reconstruction} a “passive neuter extension” suffix \example{*-am} for \ili{Proto-East-Cushitic}. In some descendant languages reflexes of the suffix have a marginal anticausative\is{anticausative voice} function,\is{passive origin} as also noted implicitly by \citet[98]{hayward:1984} -- and even a reflexive\is{reflexive voice} function in one language, \ili{Ts’amakko}, as described in the previous section. By contrast, passive-reciprocal type 1 syncretism\is{voice syncretism, full resemblance -- type 1} is apparently only attested in \ili{Sidaama}, although passive-reciprocal type 2 syncretism\is{voice syncretism, partial resemblance -- type 2} can be found in the related languages \ili{Hadiyya}, \ili{Alaaba}, and \ili{K’abeena}. For comparative purposes, the expression of passivity\is{passive voice} and reciprocity\is{reciprocal voice} is illustrated in these five languages and seven other East Cushitic languages in \tabref{tab:ch7:pass-recp-cushitic}. The suffix \example{-akk’} is an innovative “middle” suffix \citep[90]{hayward:1984} with mainly \isi{autobenefactive} and reflexive\is{reflexive voice} uses when used on its own (see \citealt[312ff.]{schneider-blum:2007} on \ili{Alaaba} and \citealt[141ff.]{crass:2005} on \ili{K’abeena}).

\begin{table}
	\setlength{\tabcolsep}{8pt}
	\begin{tabularx}{\textwidth}{llccl}
		\lsptoprule
		& & \textsc{pass} & \textsc{recp} & \\
		\midrule 
		Highland & \ili{Sidaama} & \example{-am} & \example{-am} & \citep{kawachi:2007} \\
		& \ili{Hadiyya} & \example{-am} & \example{-am-am} & \citep{sibamo:2015} \\
		& \ili{Alaaba} & \example{-am} & \example{-akk’-am} & \citep{schneider-blum:2007} \\
		& \ili{K’abeena} & \example{-am} & \example{-akk’-am} & \citep{crass:2005} \\
		& \ili{Burji} & \example{-am} & [\textsc{periph.}] & \citep{assefa:2015} \\
		\midrule 
		Lowland & \ili{Ts’amakko} & \example{-am} & ? & \citep{sava:2005} \\
		& \ili{Konso} & \example{-am} & [\textsc{periph.}] & \citep{orkaydo:2013} \\
		& \ili{Bayso} & \example{-am} & [\textsc{periph.}] & \citep{kebebew:2018} \\
		& \ili{Girirra} & \example{-am} & \example{isi-} & \citep{mekonnen:2015} \\
		& \ili{Oromo} & \example{-am} & \example{wal-} & \citep{teferi:2019} \\
		& \ili{Saaho} & \example{-(V)m} & [\textsc{periph.}] & \citep{tajebe:2015} \\
		& \ili{Afar} & \example{-(V)m} & [\textsc{periph.}] & \citep{kamil:2015} \\
		\lspbottomrule
	\end{tabularx}
	\caption{Passive and reciprocal marking in E. Cushitic languages}
	\label{tab:ch7:pass-recp-cushitic}
\end{table}



Considering the distribution of the passive\is{passive voice} and reciprocal\is{reciprocal voice} functions of \ili{Proto-East-Cushitic} \example{*-am} in descendant languages, the reciprocal\is{reciprocal voice} function of the Si\-daa\-ma\il{Sidaama} suffix \example{-am} likely evolved from the passive\is{passive voice} function.\is{passive origin} This \isi{diachronic development} is illustrated in \tabref{tab:ch7:pass-recp-cushitic-2} \citep[334, 342]{kawachi:2007}, which also includes the passive-reciprocal type 2 syncretism\is{voice syncretism, partial resemblance -- type 2} found in \ili{Hadiyya} \citep[75]{sibamo:2015}, \ili{Alaaba} \citep[310, 321]{schneider-blum:2007}, and \ili{K’abeena} \citep[143, 145]{crass:2005}.

\begin{table}
	\setlength{\tabcolsep}{3pt}
	\begin{tabularx}{\textwidth}{rcllll}
		\lsptoprule
		P.-E.-Cush.\il{Proto-East-Cushitic} & \example{*-am} & \textsc{pass} & \multicolumn{1}{r}{→} & \textsc{recp} & \\
		\midrule 
		\ili{Sidaama} & \example{-am} & \example{gan-am-} & ‘to be hit’ & \example{gan-am-} & ‘to hit e.o.’ \\
		\ili{Hadiyya} & [\example{-am}]\example{-am} & \example{gan-am-} & ‘to be hit’ & \example{gan-am-am-} & ‘to hit e.o.’ \\
		\ili{Alaaba} & [\example{-akk’}]\example{-am} & \example{hog-am-} & ‘to be cleaned’ & \example{ʔiitt-akk’-am-} & ‘to love e.o.’ \\
		\ili{K’abeena} & [\example{-akk’}]\example{-am} & \example{mur-am-} & ‘to be cut’ & \example{leʾ-akk’-am-} & ‘to see e.o.’ \\
		\lspbottomrule
	\end{tabularx}
	\caption{\textsc{pass-recp} syncretism of \textsc{pass} origin in E. Cushitic languages}
	\label{tab:ch7:pass-recp-cushitic-2}
\end{table}

The manner in which the passive\is{passive voice} voice marking in \ili{Sidaama} developed its reciprocal\is{reciprocal voice} functions is not entirely clear,\is{passive origin} and information from related languages does not seem to shed much light on the issue either. However, it is worth noting that the passive\is{passive voice} and reciprocal\is{reciprocal voice} voices both involve semantic referents\is{semantic referent} being acted upon by others, but in the reciprocal\is{reciprocal voice} voice the referents\is{semantic referent} themselves also act upon others unlike in the passive\is{passive voice} voice. In \ili{Sidaama} it seems that the referents\is{semantic referent} in the passive\is{passive voice} voice apparently gained the capability to act upon others. Moreover, it might be noted here that the suffix \example{-am} in \ili{Sidaama} also has a “very limited” \isi{iterative} meaning \citep[344]{kawachi:2007}, which might link it to \isi{plurality of relations} and thereby maybe to reciprocity\is{reciprocal voice} (\sectref{diachrony:reciprocal}). Finally, note that the language possesses a lexicalised\is{lexicalisation} verb \example{šarr-am-} ‘to wrestle’, which cannot be used without the suffix \example{-am} \citep[344]{kawachi:2007}, and one can hypothesise that the reciprocal\is{reciprocal voice} function of the suffix might have first evolved with lexically reciprocal\is{lexical reciprocal} verbs: \example{šarr-am-} ‘to be wrestled by sb.’ → ‘to be wrestled by sb. and thereby wrestle that person’ → ‘to wrestle e.o.’ and by extension → \example{gan-am-} ‘to hit e.o.’

\subsection{From passive to anticausative} \label{diachrony:pass2antc}
Voice development from passive\is{passive voice} to anticausative\is{anticausative voice}\is{passive origin} has received slightly more attention than the diachronic developments\is{diachronic development} described in the previous two sections. For instance, \citet{malchukov:nedjalkov:2015} have argued for such development among certain Tungusic languages. As discussed in more detail later in \sectref{diachrony:caus2pass}, it is well-known that the \ili{Proto-Tungusic} causative\is{causative voice} suffix \example{*-bu} has developed a passive\is{passive voice} function in many descendant languages. Additionally, some reflexes of the suffix have even developed an anticausative\is{anticausative voice} function,\is{passive origin} albeit a marginal one, for instance the reflex \example{-v} in \ili{Evenki}. Given the distribution of the various functions among Tungusic languages,  \citet[611]{malchukov:nedjalkov:2015} suggest that the anticausative\is{anticausative voice} function developed via the passive\is{passive voice}. This development is illustrated by examples from \ili{Evenki} in \tabref{tab:ch7:pass-antc-evenki} \citep[608f.]{malchukov:nedjalkov:2015}. It is worth noting, however, that the causative\is{causative voice} function has been retained alongside the passive\is{passive voice} function, for which reason the development might be described more precisely in terms of causative-passive voice marking developing an anticausative\is{anticausative voice} function.  

\begin{table}
	\setlength{\tabcolsep}{3.2pt}
	\begin{tabularx}{\textwidth}{rclrl}
		\lsptoprule
		\ili{Proto-Tungusic} & \example{*-bu} & \textsc{pass} (+ \textsc{caus}) & → & \textsc{antc} \\
		\midrule 
		\ili{Evenki} & \example{-v} & \multicolumn{2}{l}{\example{oo-v-} ‘to be built [by sb.]’} & \example{sukča-v-} ‘to break’ \\
		& & \multicolumn{2}{l}{(cf. \example{suru-v-} ‘to lead sb. away’)} & \\
		\lspbottomrule
	\end{tabularx}
	\caption{\textsc{pass-antc} syncretism of \textsc{pass} origin in Evenki}
	\label{tab:ch7:pass-antc-evenki}
\end{table}

Furthermore, \citet[232]{kulikov:2011b} has hypothesised that a development from “passive\is{passive voice} to anticausative\is{anticausative voice} through impersonalization”\is{impersonalisation}\is{passive origin} is “not infrequent -- in particular, in a number of Indo-European languages”, but that “the passive\is{passive voice} to anticausative\is{anticausative voice} transition is only rarely explicitly mentioned in grammars and has not received due attention in the literature”. However, \citet[246ff.]{kulikov:2011b} only explicitly discusses such development in relation to \ili{Old Church Slavonic}, \ili{Greek}, \ili{Latin}, and Vedic \ili{Sanskrit}). \citet[232]{kulikov:2011b} concentrates on the latter language for which he describes a “clear instance of such development” in which the suffix \example{-yá} with a supposedly original passive\is{passive origin} function has developed an anticausative\is{anticausative voice} function with some verbs. In his discussion of this development, \citeauthor{kulikov:2011b} focuses primarily on verbs of perception, including those listed in \tabref{tab:ch4:pass-antc-vedic}. In these examples the original (a) meanings of the respective verbs are passive\is{passive voice}, while the later (b) meanings are anticausative\is{anticausative voice} according to \citet[234, 249]{kulikov:2011b}: “[t]he non-passive usages of the passives\is{passive voice} derived from verbs of perception of the type ‘is seen’ → ‘is visible; appears’ represent the commonest instance of passive\is{passive voice} to anticausative\is{anticausative voice} transition, and can probably be found in most languages with passives\is{passive voice}”. However, the purported anticausative function (b) is not acknowledged in this book. As described in \sectref{def:causatives-anticausatives}, in this book an anticausative\is{anticausative voice} voice is defined in contrast to a \isi{diathesis} in which an additional \isi{semantic participant} not found in the anticausative\is{anticausative voice} voice is a \isi{causer} -- but there is no such additional \isi{semantic participant} in \citeauthor{kulikov:2011b}’s examples (otherwise the contrasting meaning of the verbs \example{dr̥ś-}, \example{śrū-} and \example{vid-} would have been *‘to make sth. be visible’, *‘to make sth. be visible’, and *‘to make sth. be findable’, respectively).

\begin{table}
	\setlength{\tabcolsep}{4.6pt}
	\begin{tabularx}{\textwidth}{llllll}
		\lsptoprule
		\multicolumn{6}{l}{Vedic \ili{Sanskrit} \citep[234--241]{kulikov:2011b}} \\
		\midrule
		\example{dr̥ś-} & ‘to see sth.’ & ↔ & \example{dr̥ś-yá-} & a. ‘to be seen’ & b. ‘to be visible, appear’ \\
		\example{śrū-} & ‘to hear sth.’ & ↔ & \example{śru-yá-} & a. ‘to be heard’ & b. ‘to be audible, famous’ \\
		\example{vid-} & ‘to find sth.’ & ↔ & \example{vid-yá-} & a. ‘to be found’ & b. ‘to be findable, exist’ \\
		\lspbottomrule
	\end{tabularx}
	\caption{Verbs of perception in Vedic Sanskrit}
	\label{tab:ch4:pass-antc-vedic}
\end{table}

Nevertheless, \citet{kulikov:2011b} also discusses the verb of speech \example{vac-} ‘to pronounce sth.’ (i.e. ‘to sound sth.’) at some length, as well as a few verbs of “causation of motion” in brief, and the suffix \example{-yá} can indeed have an anticausative\is{anticausative voice} function with these verbs. \citet[245]{kulikov:2011b} remarks that the anticausative\is{anticausative voice} function of the verbs “could further be supported by the influence of the middle non-passive presents with the suffix \example{-ya-} and root accentuation […] derived from some verbs of motion”, qualifying as \isi{equipollent} causative-anticausative voice relations\is{voice relation}. Both kinds of voice relations\is{voice relation} are illustrated in \tabref{tab:ch4:pass-antc-vedic-2}.

\begin{table}
	\setlength{\tabcolsep}{7.3pt}
	\begin{tabularx}{\textwidth}{llllll}
		\lsptoprule
		\multicolumn{6}{l}{Vedic \ili{Sanskrit} (\citealt[202f.]{kulikov:2000};; \citeyear[713]{kulikov:2007}; \citeyear[241ff., 244ff.]{kulikov:2011b};; \citeyear[318]{kulikov:2011a};} \\
		\multicolumn{6}{r}{\citeyear[168]{kulikov:2012}; \citeyear[388]{kulikov:2017}; \citealt[302]{kulikov:lavidas:2017})} \\
		\midrule
		\textsc{antc} & \example{vac-} & ‘to sound sth.’ & ↔ & \example{uc-yá-} & ‘to sound’ \\
		\textsc{antc} & \example{sic-} & ‘to pour sth.’ & ↔ & \example{sic-yá-} & ‘to pour (out)’ \\
		\textsc{antc} & \example{kr̥̄-} & ‘to scatter sth.’ & ↔ & \example{kīr-yá-} & ‘to scatter’ \\
		\midrule
		\textsc{caus/antc} & \example{pād-áya-} & ‘to fell sth.’ & ↔ & \example{pád-ya-} & ‘to fall’ \\
		\textsc{caus/antc} & \example{ri-ṇā́-} & ‘to whirl sth.’ & ↔ & \example{rī́-ya-} & ‘to whirl’ \\
		\textsc{caus/antc} & \example{pr̥-ṇā́-} & ‘to fill sth.’ & ↔ & \example{pū́r-ya-} & ‘to fill’ \\
		\textsc{caus/antc} & \example{kṣi-ṇā́-} & ‘to perish sth.’ & ↔ & \example{kṣī́-ya-} & ‘to perish’ \\
		\lspbottomrule
	\end{tabularx}
	\caption{Anticausative voice in Vedic Sanskrit}
	\label{tab:ch4:pass-antc-vedic-2}
\end{table}



Observe that the difference in accentuation of the suffix (\example{-yá} vs. \example{-ya}) in \tabref{tab:ch4:pass-antc-vedic-2} has been the topic of much debate in its own right, but a detailed treatment of accentual differences goes beyond the scope of this discussion. \citet[246]{kulikov:2011b} focuses specifically on the development of \example{-yá} and its functions, but also briefly acknowledges a “passive\is{passive voice} to anticausative\is{anticausative voice} transition”\is{passive origin} for the suffix \example{-ya}. \citet[248]{kulikov:2011b} ultimately argues that the anticausative\is{anticausative voice} function arose from the passive\is{passive voice} function through four stages: (i) “canonical”\is{canonicity} passive\is{passive voice} → (ii) “agentless” passive\is{passive voice} → (iii) “impersonalized”\is{impersonalisation} passive\is{passive voice} → (iv) anticausative\is{anticausative voice}. According to \citeauthor{kulikov:2011b}, the difference between the second and third stages lies in the nature of the omitted agent: in the “agentless” passive\is{passive voice} it is non-generic, and in the “impersonalized”\is{impersonalisation} passive\is{passive voice} it is generic. In other words, non-absolute passive\is{passive voice} → absolute passive\is{passive voice} with non-generic \isi{agent} → absolute passive\is{passive voice} with generic \isi{agent} → anticausative\is{anticausative voice}. \citet[182]{hock:2019} has recently taken a more cautious stance on the matter, arguing that “with a few exceptions the Vedic [\ili{Sanskrit}] evidence makes it difficult to decide on the directionality” of the development due to the “systematic ambiguity between passive\is{passive voice} and anticausative\is{anticausative voice} interpretation” of the suffix \example{-yá} (and \example{-ya}). \citet[188]{hock:2019} instead speculates that “the distinction between passive\is{passive voice} and anticausative\is{anticausative voice} is secondary”. More specifically, \citet[188f.]{hock:2019} argues that “no distinctly passive\is{passive voice} or anticausative\is{anticausative voice} functions can be reconstructed\is{reconstruction} for the [\ili{Proto-Indo-European}] verbs in \example{*-ye/o-}” from which Vedic \ili{Sanskrit} \example{-yá} (and \example{-ya}) descend. Consequently, “the ancestors of our passive\is{passive voice}/anticausative\is{anticausative voice} verbs originally only had undifferentiated \isi{intransitive} function” \citep[189]{hock:2019}. In other words, “passive\is{passive voice} or anticausative\is{anticausative voice} readings would have been a matter of pragmatics” and “[o]nly in later Vedic would some forms of this type acquire unambiguous anticausative\is{anticausative voice} (or passive\is{passive voice}) functions” \citep[189]{hock:2019}. Finally, \citet[190]{hock:2019} comments that “[u]nder such near-systematic conditions of structural ambiguity, it is possible that different speakers preferred different accounts, whether for all relevant verbs, for subsets […] or even individual verbs, in individual contexts”. Nevertheless, \citeauthor{hock:2019} does not reject the possibility of a passive\is{passive voice} to anticausative\is{anticausative voice} development\is{passive origin} altogether, noting that at least in relation to late Vedic \ili{Sanskrit} such voice change “seems to be more appropriate” than an anticausative\is{anticausative voice} to passive\is{passive voice} development.

Finally, in \ili{Latin} some verbs marked by one of several suffixes generally associated with passivity\is{passive voice} can indeed have an anticausative\is{anticausative voice} function (for a list of such verbs, see \citealt[227]{miller:d:g:1993}), yet it remains unresolved whether or not this function is a vestige of the \ili{Proto-Indo-European} middle suffixes\is{middle syncretism} from which the Latin suffixes derive, as \citet[247]{kulikov:2011b} also notes. 

\section{Causative origin} \label{diachrony:causative}
Prospects of a \isi{causative origin} for voice syncretism are normally associated with causative-passive (\sectref{diachrony:caus2pass}) as well as causative-applicative syncretism (\sectref{diachrony:caus2appl}). Interestingly, as demonstrated in the next section, some evidence indicates that causative\is{causative voice} voice marking can even develop an anticausative\is{anticausative voice} function.

\subsection{From causative to anticausative} \label{diachrony:caus2antc}
Diachronic development\is{diachronic development} from causative\is{causative voice} to anticausative\is{anticausative voice} has been the focus of little research, yet sporadic evidence for the phenomenon can be found in a few Eurasian languages. For instance, the \ili{Proto-Tungusic} causative\is{causative voice} suffix \example{*-bu} has developed an anticausative\is{anticausative voice} function in some descendant languages (cf. \ili{Evenki} \example{-v}), likely facilitated by passivity\is{passive voice} (\sectref{diachrony:pass2antc}). Moreover, there seems to be some evidence pointing toward a \isi{causative origin} for causative-anticausative syncretism characterised by the suffix \example{-ke} in the language isolate \ili{Ainu} (see \tabref{tab:ch4:caus-antc} on page \pageref{tab:ch4:caus-antc}). As briefly noted in \sectref{resemblance-type3}, \citet{nonno:2015} suggests that this suffix can be traced back to the verb \example{*ki} ‘to do, act’ which suggests a causative\is{causative voice} rather than an \isi{anticausative origin}. Finally, \citet{yap:ahn:2019} have argued for a \isi{causative origin} for causative-anticausative syncretism characterised by the suffix \example{-(C)i} in \ili{Korean} (see \tabref{tab:ch5:caus-pass-antc} on page \pageref{tab:ch5:caus-pass-antc}). According to \citet[3ff., 9f.]{yap:ahn:2019}, the \ili{Korean} suffix \example{-(C)i} has an attested causative\is{causative voice} function dating back at least to the 10th century whence an anticausative\is{anticausative voice} function evolved around the 15th century. This \isi{diachronic development} is illustrated in \tabref{tab:ch7:caus-antc-korean}. Note that the same suffix also developed a passive\is{passive voice} function around the same time as the anticausative\is{anticausative voice} function (see the next section), but \citet[16ff.]{yap:ahn:2019} believe that both functions evolved concurrently from the causative\is{causative voice} function through a “causative-to-passive pathway” and “causative-to-middle pathway”, respectively. The origin of the suffix \example{-(C)i} itself is “largely unknown” though it may be related to the “proximal demonstrative \example{i} (‘this’) and the defective noun \example{i} (‘person’)” \citep[20]{yap:ahn:2019}. Both the causative\is{causative voice} and the anticausative\is{anticausative voice} functions remain productive\is{productivity} in contemporary Korean. 

\begin{table}
	\setlength{\tabcolsep}{6.9pt}
	\begin{tabularx}{\textwidth}{lclllc}
		\lsptoprule
		\ili{Korean} & & & & & \\
		\midrule
		10th c. & \example{-(C)i} & \example{nep-hi-} & ‘to widen sth.’ & (cf. \example{nep-} ‘to be wide’) & \textsc{caus} \\
		& ⋮ & & & & ↓ \\
		15th c. & ⋮ & \example{tat-hi-} & ‘to close’ & (cf. \example{tat-} ‘to close sth.’) & \textsc{antc} \\
		\lspbottomrule
	\end{tabularx}
	\caption{\textsc{caus-antc} syncretism of \textsc{caus} origin in Korean}
	\label{tab:ch7:caus-antc-korean}
\end{table}

\citet[8, 17f.]{yap:ahn:2019} argue that the development from causative\is{causative voice} to anticausative\is{anticausative voice}\is{causative origin} in \ili{Korean} “boil[s] down to shifts in perspective-taking” and hypothesise that so-called “reflexive\is{reflexive voice} causative\is{causative voice} \example{-i} constructions in \ili{Korean} that involve bodily actions such as ‘scratching oneself’ […] provide a \isi{bridging context} for causative\is{causative voice} \example{-i} constructions to develop into middle [incl. anticausative\is{anticausative voice}] \example{-i} constructions”, e.g. \example{kulk-} ‘to scratch sth.’ → \example{kulk-hi-} ‘to make sb. scratch a body part’ → ‘to make sb. scratch self’. \citet[17]{yap:ahn:2019} suggest that the last stage came about through the elision of the body part being scratched “for reasons of politeness or discretion”. Nevertheless, this scenario does not explain the absence of causation in the anticausative\is{anticausative voice} voice (cf. \example{tat-hi-} ‘to close’, not *‘to make sth. close itself’). Elsewhere \citet[10]{yap:ahn:2019} also hint at \isi{causer} elision in passing which itself can serve as an alternative explanation for the development from causative\is{causative voice} to anticausative\is{anticausative voice}: ‘the porter closed the gate’ → ‘(someone or something) closed the gate’ → ‘the gate closed’. 

\subsection{From causative to passive} \label{diachrony:caus2pass}
Alongside voice syncretism of \isi{reflexive origin}, causative-passive syncretism of \isi{causative origin} is one of the most discussed diachronic developments\is{diachronic development} of voice syncretism in the literature (e.g. \citealt{haspelmath:1990}; \citealt{washio:1993}; \citealt{knott:1995}; \citealt{yap:iwasaki:1998}; \citeyear{yap:iwasaki:2003}; \citealt{robbeets:2007}; \citeyear{robbeets:2015}; \citealt{ahn:yap:2017}; \citealt{yap:ahn:2019}; \citealt{zuniga:kittila:2019}). By contrast, it has hitherto not been possible to find any attestation of passive\is{passive voice} voice marking developing a causative\is{causative voice} function. Causative-passive syncretism of \isi{causative origin} has most notably been proposed for several Eurasian languages which will be described in this section. For instance, as mentioned in the previous section, the \ili{Korean} suffix \example{-(C)i} which historically had a causative\is{causative voice} function developed a passive\is{passive voice} function around the 15th century \citep[11f.]{yap:ahn:2019}. This development is illustrated in \tabref{tab:ch7:caus-pass-korean}.

\begin{table}
	\setlength{\tabcolsep}{4.8pt}
	\begin{tabularx}{\textwidth}{lclllc}
		\lsptoprule
		\ili{Korean} & & & & & \\
		\midrule
		10th c. & \example{-(C)i} & \example{nep-hi-} & ‘to widen sth.’ & (cf. \example{nep-} ‘to be wide’) & \textsc{caus} \\
		& ⋮ & & & & ↓ \\
		15th c. & ⋮ & \example{cap-hi-} & ‘to be caught [by sb.]’ & (cf. \example{cap-} ‘to catch sb.’) & \textsc{pass} \\
		\lspbottomrule
	\end{tabularx}
	\caption{\textsc{caus-pass} syncretism of \textsc{caus} origin in Korean}
	\label{tab:ch7:caus-pass-korean}
\end{table}

Causative-passive syncretism of \isi{causative origin} has also famously been described for the Tungusic languages mentioned in the previous sections. More specifically, the \ili{Proto-Tungusic} verb \example{*böö-} ‘to give’ is generally believed to have grammaticalised\is{grammaticalisation} into the suffix \example{*-bu} with a causative\is{causative voice} function which later developed a passive\is{passive voice} function (\citealt[518]{von-der-gabelentz:1861}; \citealt[48]{haspelmath:1990}; \citealt{nedyalkov:1993}; \citealt[194ff.]{yap:iwasaki:1998};; \citealt[608ff.]{malchukov:nedjalkov:2015}). This development is illustrated in two Tungusic languages in \tabref{tab:ch7:caus-pass-tungusic} (Manchu = \citeauthor{nedyalkov:1991} \citeyear[5]{nedyalkov:1991}; \citeyear[194]{nedyalkov:1993}; Kilen = \citealt[117, 188f.]{paiyu:2013}).

\begin{table}
	\setlength{\tabcolsep}{4pt}
	\begin{tabularx}{\textwidth}{rcllll}
		\lsptoprule
		\ili{Proto-Tungusic} & \example{*-bu} & & \textsc{caus} & → & \textsc{pass} \\
		\midrule 
		\ili{Manchu} & \example{-bu} & \example{va-bu-} & ‘to kill sb.’ & & ‘to be killed [by sb.]’ \\
		\ili{Kilen} & \example{-wu} & \example{tanta-wu-} & ‘to hit sb.’ & & ‘to be hit [by sb.]’ \\
		\lspbottomrule
	\end{tabularx}
	\caption{\textsc{caus-pass} syncretism of \textsc{caus} origin in Tungusic languages}
	\label{tab:ch7:caus-pass-tungusic}
\end{table}

Another rather clear example of voice development from causative\is{causative voice} to passive\is{passive voice}\is{causative origin} comes from Mongolic languages. \citet[11]{janhunen:2003b} reconstructs\is{reconstruction} a passive\is{passive voice} suffix (\example{*-dA/-tA/-gdA}) and three causative\is{causative voice} suffixes (\example{*-gA/-kA/-xA}, \example{*-lgA}, and \example{*-xUl}) for \ili{Proto-Mongolic} that have largely been retained alongside their original functions in descendant languages (see \citealt{janhunen:2003a}), though the passive\is{passive voice} function has been lost in many Southern Mongolic languages (see \citealt{field:1997} on Santa Mongolian\il{Mongolian, Santa}, \citealt{slater:2003} on \ili{Mangghuer}, and \citealt{fried:2010} on \ili{Bao’an Tu}). In a few Mongolic languages causative\is{causative voice} voice marking has developed a passive\is{passive voice} function, e.g. Mongolian causative-passive \example{-UUl} reflecting \ili{Proto-Mongolic} \example{*-xUl} (\citealt[172]{svantesson:2003}; see \tabref{tab:ch4:caus-pass} on page \pageref{tab:ch4:caus-pass}). This development in \ili{Mongolian} is illustrated in \tabref{tab:ch7:caus-pass-mongolian} \citep[250]{janhunen:2012}.

\begin{table}
	\setlength{\tabcolsep}{3.3pt}
	\begin{tabularx}{\textwidth}{rcllrl}
		\lsptoprule
		\ili{Proto-Mongolic} & \example{*-xUl} & & \textsc{caus} & → & \textsc{pass} \\
		\midrule 
		\ili{Mongolian} & \example{-UUl} & \example{id-uul} & \multicolumn{2}{l}{‘to make/let sb. eat sth.’} & ‘to be eaten [by sb.]’ \\
		\lspbottomrule
	\end{tabularx}
	\caption{\textsc{caus-pass} syncretism of \textsc{caus} origin in Mongolian}
	\label{tab:ch7:caus-pass-mongolian}
\end{table}

Causative-passive syncretism can also be found in the Uralic language family in which the \ili{Proto-Uralic} causative\is{causative voice} suffix \example{*-t} \citep[278f.]{collinder:1969} or \example{-tä/-tå} \citep[23]{janhunen:1982} has developed a passive\is{passive voice} function\is{causative origin} in at least two Finno-Ugric languages, the Ugric language \ili{Hungarian} (\citealt[48]{haspelmath:1990}; \citealt{tanko:2016, tanko:2017}) and the Finnic language \ili{Finnish}. In these languages the reflexes of the proto-suffix are \example{-(t)et/-(t)at} and \example{-ta/-tä}, respectively, and the development is illustrated in \tabref{tab:ch7:caus-pass-uralic}. For the sake of convenience, the \ili{Proto-Uralic}, \ili{Hungarian}, and \ili{Finnish} suffixes are here given as \example{*-tV}, \example{-(t)Vt}, and \example{-tV}, respectively. Moreover, note that the passive\is{passive voice} function of \ili{Hungarian} \example{-(t)Vt} is obsolete in the modern language, and the passive\is{passive voice} example of the suffix in \tabref{tab:ch7:caus-pass-uralic} thus represents archaic use. Also note that the \ili{Finnish} suffix is obligatorily accompanied by the suffix \example{-an/-än} in the passive\is{passive voice} voice (i.e. causative-passive type 2 syncretism\is{voice syncretism, partial resemblance -- type 2}).

\begin{table}
	\setlength{\tabcolsep}{2.8pt}
	\begin{tabularx}{\textwidth}{rclrlll}
		\lsptoprule
		\ili{Proto-Uralic} & \example{*-tV} & \textsc{caus} & & \multicolumn{1}{r}{→} & \textsc{pass} & \\
		\midrule 
		\ili{Hungarian} & \example{-(t)Vt} & \example{vár-at-} & \multicolumn{2}{l}{‘to make sb. wait’} & \example{ad-at-} & ‘to be given [by sb.]’ \\
		\ili{Finnish} & \example{-tV} & \example{alen-ta-} & \multicolumn{2}{l}{‘to lower sth.’} & \example{lue-ta-an} & ‘to be read [by sb.]’ \\
		\lspbottomrule
	\end{tabularx}
	\caption{\textsc{caus-pass} syncretism of \textsc{caus} origin in F.-Ugric languages}
	\label{tab:ch7:caus-pass-uralic}
\end{table}

\citet[48]{haspelmath:1990} observes that a similar development may have taken place in the Indo-Aryan language \ili{Gujarati} where the passive\is{passive voice} suffix \example{-ā} perhaps descends from the suffix \example{-āya} \citep[317]{masica:1991} which is believed to have had a causative\is{causative voice} function \citep[84]{kulikov:2009}. A \isi{causative origin} for causative-passive syncretism has also often been proposed for Turkic languages in some of which cognates of the suffix \example{-t} can serve as voice marking in both the causative\is{causative voice} and passive\is{passive voice} voices (\citealt[48]{haspelmath:1990}; \citealt[178f.]{robbeets:2007};; \citeyear[290ff.]{robbeets:2015}). However, \citet[290]{robbeets:2015} reconstructs\is{reconstruction} an “original causative-passive suffix” \example{*-ti} for \ili{Proto-Turkic}, suggesting that the syncretism was already present in the proto-language, and the further diachrony of the suffix therefore remains obscure. Outside of Eurasia it has only been possible to find one case of causative-passive syncretism for which a \isi{causative origin} can be established with some certainty. It has been repeatedly observed that the causative\is{causative voice} suffix \example{-tit} in the Eskimo language West Greenlandic\il{Greenlandic, West} (\lang{na}) seems to have developed a passive\is{passive voice} suffix rather recently (\citealt[265]{fortescue:1984}; \citealt[48]{haspelmath:1990}; \citealt[7]{schikowski:2009}). This development is shown in \tabref{tab:ch7:caus-pass-greenlandic} \citep[475f.]{underhill:1980}.

\begin{table}
	\setlength{\tabcolsep}{7.5pt}
	\begin{tabularx}{\textwidth}{rlll}
		\lsptoprule
		West Greenlandic\il{Greenlandic, West} & \textsc{caus} & → & \textsc{pass} \\
		\midrule 
		\example{neri-tit-} & ‘to make sb. eat sth.’ & & ‘to be eaten [by sb.]’ \\
		\lspbottomrule
	\end{tabularx}
	\caption{\textsc{caus-pass} syncretism of \textsc{caus} origin in West Greenlandic}
	\label{tab:ch7:caus-pass-greenlandic}
\end{table}

Voice development from causative\is{causative voice} to passive\is{passive voice}\is{causative origin} is generally hypothesised to involve a “causative-reflexive” or a “reflexive permissive-causative” intermediary stage whereby a \isi{causer} lets itself be acted upon by another \isi{semantic participant}, and subsequently loses its focus of attention until it eventually does not cause anymore (\citealt[476f.]{underhill:1980};; \citealt[840]{shibatani:1985}; \citealt[46f.]{haspelmath:1990};; \citealt{yap:iwasaki:1998}; \citealt{yap:ahn:2019}; \citealt[226]{zuniga:kittila:2019}). In broader terms, the causative\is{causative voice} voice can be said to share “the feature of A-\isi{demotion} with passives\is{passive voice}” \citep[24]{malchukov:2017}.

\subsection{From causative to applicative} \label{diachrony:caus2appl}
Like the diachrony discussed in the previous section, the origin of causative-applicative syncretism has received considerable attention in the literature (e.g. \citealt[166ff.]{shibatani:pardeshi:2001};; \citeyear[116ff.]{shibatani:pardeshi:2002};; \citealt[64ff.]{peterson:2007};; \citealt[403ff.]{malchukov:2016};; \citeyear[13ff.]{malchukov:2017}). However, as noted by \citet[236]{zuniga:kittila:2019}, “the border between causativization\is{causativisation} and applicativization\is{applicativisation} is porous” and it can therefore be difficult to determine the origin of causative-applicative syncretism. Indeed, it can sometimes be difficult to distinguish between a causative\is{causative voice} and an applicative\is{applicative voice} function in the first place, as certain situations can be conceptualised in different manners. For the sake of illustration, \citet[14, 17]{austin:2005} treats the verb \example{iti-nti} ‘to bring sth. back’ (cf. \example{iti} ‘to return’) in the Northern Pama-Nyungan language \ili{Kalkatungu} as causative\is{causative voice}, but the verb \example{gambira-ma-} ‘to bring sth. back’ (cf. \example{gambira-} ‘to return’) in the related language \ili{Margany} (both \lang{au}) as applicative\is{applicative voice}. Here it seems that \citeauthor{austin:2005} conceptualises the verbs ‘to make sth. return’ and ‘to return with sth.’, respectively. In any case, causative-applicative syncretism is often believed to generally have a \isi{causative origin} (especially following \citealt{shibatani:pardeshi:2001, shibatani:pardeshi:2002}), although the possibility of an \isi{applicative origin} is sometimes acknowledged as well (\citealt{wise:1990, payne:2002, guillaume:rose:2010, malchukov:2017}). Causative-applicative syncretism of \isi{causative origin} is discussed in this section, while causative-applicative syncretism of \isi{applicative origin} is described in \sectref{diachrony:appl2caus}.

\citeauthor{shibatani:pardeshi:2001} (\citeyear{shibatani:pardeshi:2001}; \citeyear[118]{shibatani:pardeshi:2002}) have famously argued for a \isi{causative origin} of causative-applicative syncretism suggesting that “the applicative\is{applicative voice} meanings of comitative,\is{comitativity} instrumental, and \isi{benefactive} forms be connected to sociative\is{sociativity} causatives\is{causative voice}”. For instance, “[t]he comitative\is{comitativity} meanings of ‘I walk with him’ and ‘I play with her’ are derivable from ‘I make him walk by walking with him’ and ‘I make her play by playing with her’” \citep[118]{shibatani:pardeshi:2002}. Likewise, “[i]f someone causes a knife to cut the meat, he/she is in effect cutting the meat with a knife, because a knife cannot cut meat independently from the \isi{causer} \isi{agent} who actually uses it” \citep[119]{shibatani:pardeshi:2002}. In support of their argument, \citeauthor{shibatani:pardeshi:2002} cite examples of causative-applicative syncretism from sixteen geographically diverse languages (representing sixteen different genera\is{genus}). The simple explanation proposed by \citeauthor{shibatani:pardeshi:2002} is certainly plausible in many languages (and will be considered again at the end of this section), yet it is important to note that there is actually little historical and comparative data available for most of the languages they discuss. Indeed, some of the authors of the sources cited by \citeauthor{shibatani:pardeshi:2002} do not address the issue of diachrony at all, including \citet{saunders:davis:1982} on the Salishan language \ili{Bella Coola} (\lang{na}), \citet[392]{plungian:1993} on the Dogon language Tommo So\il{So, Tommo} (\lang{af}), and \citet{ichihashi-nakayama:1996} on the Yuman language \ili{Hualapai} (\lang{na}). Consequently, in many cases it cannot be confirmed with certainty how the causative-applicative syncretism in the languages arose diachronically, and alternative origins cannot automatically be dismissed. As already mentioned above and further discussed in the next section, the opposite development seems to have taken place in some languages, even in cases involving \isi{sociativity}. Consequently, the diachrony of the causative-applicative syncretism in each of the remaining thirteen languages mentioned by \citet{shibatani:pardeshi:2002} is revisited here.

Some authors of the sources cited by \citet{shibatani:pardeshi:2002} explicitly state that the origin of causative-applicative syncretism in a given language may not necessarily be causative\is{causative voice}.\is{causative origin} \citet[396]{fleck:2002} argues that “we must conclude that [the causative-applicative suffix] \example{ua} was not specifically a causativizer,\is{causativisation} but a more general transitivizer”\is{transitivisation} in the Panoan language \ili{Matsés} (\lang{sa}). Likewise, \citet[344]{stefanowitsch:2002} calls the causative-applicative suffix \example{-ba} in the Cariban language \ili{Akawaio} (\lang{sa}) a “general transitivizer”.\is{transitivisation} In turn, \citet{queixalos:2002} suggests that the causative-applicative prefix \example{ka-} in the Guahiban language \ili{Sikuani} (\lang{sa}) has an \isi{applicative origin} (see \tabref{tab:ch7:appl-caus-sikuani} on page \pageref{tab:ch7:appl-caus-sikuani}). \citet[228]{vazquez-soto:2002} does not provide any concrete diachronic evidence for the origin of causative-applicative syncretism in the Corachol language \ili{Cora} (\lang{na}) but presupposes a \isi{causative origin} in the spirit of \citet{shibatani:pardeshi:2002} themselves (the studies are published in the same volume). The origins of causative-applicative syncretism in the Kartvelian language \ili{Svan} (\lang{ea}) and the Pama-Nyungan language \ili{Yidiny} (\lang{au}) also remain obscure (see \citealt{kulikov:1993} and \citealt{austin:2005}, respectively). Furthermore, the origin of the causative-applicative suffix \example{-kan} mentioned by \citet{shibatani:pardeshi:2002} in relation to the Malayo-Sumbawan language \ili{Malay} (\lang{pn}) has been the topic of much debate. \citet[438]{kikusawa:2012} believes it to be descended from an “oblique preposition \example{*kən}, which introduced adjunct (or, peripheral) elements of the event described in a sentence”. \citet[439]{kikusawa:2012} proposes that the preposition has grammaticalised\is{grammaticalisation} in Proto-Malay(ic)\il{Proto-Malayic}, in which the suffix \example{*-kən} appears to have had both applicative\is{applicative voice} and causative\is{causative voice} uses. The chronology of the individual functions remains unclear.

There are stronger indications of a \isi{causative origin} for causative-applicative syncretism in the remaining six languages discussed by \citet{shibatani:pardeshi:2002}. For instance, the suffix \example{-aw} in the Indo-Aryan language \ili{Marathi} and the suffix \example{-(sa)se} in \ili{Japanese} (both \lang{ea}) generally have a causative\is{causative voice} function, but also sociative\is{sociativity} applicative\is{applicative voice} functions in certain restricted contexts \citep[96ff.]{shibatani:pardeshi:2002}. The more restricted applicative\is{applicative voice} function of these suffixes seems to indicate a later development from the causative\is{causative voice} function. The same can be said for the Muskogean language \ili{Creek} (\lang{na}), for Huallaga Quechua\il{Quechua, Huallaga} (\lang{sa}) and for Kolyma Yukaghir\il{Yukaghir, Kolyma} (\lang{ea}), in which the applicative\is{applicative voice} function of the otherwise causative\is{causative voice} suffixes \example{-ic} \citep[225]{martin:2011}, \example{-chi} \citep[163]{weber:1989}, and \example{-š} \citep[215]{maslova:2003}, respectively, is barely productive.\is{productivity} It is, however, worth keeping in mind that the high synchronic \isi{productivity} of a certain function does not necessarily entail that it represents a diachronic origin, as already noted in the beginning of this chapter. The best evidence for causative-applicative syncretism of \isi{causative origin} mentioned by \citet{shibatani:pardeshi:2002} comes from the Bantu language \ili{Kinyarwanda} (\lang{af}) in which the causative-applicative suffix \example{-ish} can be traced back to the \ili{Proto-Bantu} causative\is{causative voice} suffix \example{*-ici} which contrasted with a general applicative\is{applicative voice} suffix \example{*-ɪd} \citep{meeussen:1967, bastin:1986, schadeberg:2003}. For an extensive investigation of the syncretism in \ili{Kinyarwanda}, see \citet{jerro:2017}. A similar development has also taken place in the related Namibian Fwe\il{Fwe, Namibian} language and “other Bantu Botatwe languages” (\citealt[216ff.]{gunnink:2018};; see also \citealt[66]{peterson:2007} on \ili{Shona} and \citealt[90]{creissels:2016} on \ili{Tswana}). The development from causative\is{causative voice} to applicative\is{applicative voice}\is{causative origin} in \ili{Kinyarwanda} \citep[6f.]{jerro:2017} and Namibian Fwe\il{Fwe, Namibian} \citep[216f.]{gunnink:2018} is illustrated in \tabref{tab:ch7:appl-caus-bantu}. These languages retain reflexes of the \ili{Proto-Bantu} suffix \example{*-ɪd} that continue to be used for expressing applicativity more broadly.

\begin{table}
	\setlength{\tabcolsep}{2.8pt}
	\begin{tabularx}{\textwidth}{rcllll}
		\lsptoprule
		P.-B.\il{Proto-Bantu} & \example{*-ici} & \textsc{caus} & \multicolumn{1}{r}{→} & \textsc{appl} & \\
		\midrule 
		Kiny.\il{Kinyarwanda} & \example{-ish} & \example{ndik-ish-} & ‘to make sb. write sth.’ & \example{kat-ish-} & ‘to cut sth. with sth.’ \\
		Fwe\il{Fwe, Namibian} & \example{-is} & \example{kur-is-} & ‘to make sb. sweep sth.’ & \example{fund-is-} & ‘to cut sth. with sth.’ \\
		\lspbottomrule
	\end{tabularx}
	\caption{\textsc{caus-appl} syncretism of \textsc{appl} origin in Bantu languages}
	\label{tab:ch7:appl-caus-bantu}
\end{table}

\citet[391]{guillaume:rose:2010} argue that six languages from four South American genera\is{genus} not covered by \citet{shibatani:pardeshi:2002} also feature causative\is{causative voice} affixes which in some contexts can have a sociative\is{sociativity} applicative\is{applicative voice} function:\is{causative origin} the prefix \example{mo-} in the Tupi-Guaraní language \ili{Guaraní}, the prefix \example{im-} in the Bolivia-Parana Arawakan language \ili{Trinitario}, the suffix \example{-aka(g)} in the Pre-Andine Arawakan languages \ili{Asheninka} and \ili{Caquinte}, and the suffix \example{-nopï} in the Cariban language \ili{Kari’ña} and the suffixes \example{-nîpî} and \example{-pa} in \ili{Makushi} of the same \isi{genus}. However, it is unclear how common the applicative\is{applicative voice} function is in \ili{Guaraní} and \ili{Trinitario} -- only one example is provided by \citet[522]{velazquez-castillo:2002} for the former language and by \citet[98]{wise:1990} for the latter language. Note also that \ili{Proto-Tupi-Guaraní} seems to have had a separate “comitative\is{comitativity} causative\is{causative voice}” prefix \example{*(e)ro-} \citep[593]{jensen:1998} which is, for example, retained (\example{elo-}) and characteristic for causative-applicative syncretism in \ili{Emerillon} \citep{rose:2003}. The \ili{Asheninka} and \ili{Caquinte} suffixes can be traced to the \ili{Proto-Arawakan} suffix \example{*-kʰakʰ} for which \citet[109]{wise:1990} reconstructs\is{reconstruction} an original reciprocal\is{reciprocal voice} function (\sectref{diachrony:recp2refl}). \citet[104, 110]{wise:1990} additionally shows that the suffix also has developed causative\is{causative voice} and comitative\is{comitativity} applicative\is{applicative voice} functions in a few other neighbouring languages, and ultimately argues that the causative\is{causative voice} function evolved from the comitative\is{comitativity} applicative\is{applicative voice} function and not vice versa (\sectref{diachrony:appl2caus}).\is{applicative origin} The Cariban languages appear to be better candidates for causative-applicative syncretism of \isi{causative origin} in light of \citeauthor{gildea:2015}’s (\citeyear[6ff.]{gildea:2015}) \isi{reconstruction} of three causative\is{causative voice} suffixes with no apparent applicative\is{applicative voice} functions for \ili{Proto-Carib}: \example{*-po} (cf. \ili{Makushi} \example{-pa}), \example{*-nɨpɨ} (cf. \ili{Makushi} \example{-nîpî}), and \example{*-nôpɨ} (cf. \ili{Kari’ña} \example{-nopï}). The presumed development from causative\is{causative voice} to applicative\is{applicative voice} in these languages is illustrated by examples from \ili{Makushi} in \tabref{tab:ch7:caus-appl-makushi} \citep[41, 125f.]{abbott:1991}. 

\begin{table}
	\setlength{\tabcolsep}{1.8pt}
	\begin{tabularx}{\textwidth}{rcllll}
		\lsptoprule
		P.-Carib\il{Proto-Carib} & \example{*-po} & \textsc{caus} & \multicolumn{1}{r}{→} & \textsc{appl} & \\
		\midrule 
		\multirow{2}{*}{\ili{Makushi}} & \example{-pa} & \example{we’nun-pa} & ‘to make sb. sleep’ & \example{manun-pa} & ‘to dance with sb.’ \\
		& \example{-nîpî} & \example{ereuta-nîpî} & ‘to sit sth. down’ & \example{erepan-nîpî} & ‘to arrive with sb.’ \\
		\midrule 
		P.-Carib\il{Proto-Carib} & \example{*-nɨpɨ} & & & & \\
		\lspbottomrule
	\end{tabularx}
	\caption{\textsc{caus-appl} syncretism of \textsc{appl} origin in Makushi}
	\label{tab:ch7:caus-appl-makushi}
\end{table}

\citeauthor{austin:2005}’s (\citeyear{austin:2005}) investigation of causative-applicative syncretism among Australian languages is often cited in discussions on the diachrony of syncretism, yet is worth observing that \citet[29]{austin:2005} strives to provide a “theoretical analysis of the observed patterns of transitivisation\is{transitivisation} in Australia, couched in terms of the framework of lexical mapping theory in Lexical Functional Grammar” based on synchronic data. The diachronic developments\is{diachronic development} of the causative-applicative syncretism in the individual languages discussed by \citeauthor{austin:2005} remain largely understudied. Thus, the origins of the syncretism in the languages are considered unresolved for the time being. Nevertheless, see \tabref{tab:ch7:caus-appl-pama-nyungan} for examples of the syncretism in some of the languages mentioned by \citeauthor{austin:2005}.

\begin{table}
	\setlength{\tabcolsep}{2pt}
	\begin{tabularx}{\textwidth}{rllll}
		\lsptoprule
		& \textsc{caus} & & \textsc{appl} & \\
		\midrule 
		\ili{Diyari} & \example{tharka-ipa-} & ‘to stand sth. up’ & \example{nandra-ipa-} & ‘to hit sb. for sb.’ \\
		\ili{Pitta-Pitta} & \example{yanthi-la-} & ‘to burn sth.’ & \example{wiya-la-} & ‘to laugh at sb.’ \\
		Arabana-W.\il{Arabana-Wangkangurru} & \example{kaji-la-} & ‘to turn sth.’ & \example{wiya-la-} & ‘to laugh at sb.’ \\
		M. Arrernte\il{Arrernte, Mparntwe} & \example{pwernke-lhile-} & ‘to split sth.’ & \example{therre-lhile-} & ‘to laugh at sb.’ \\
		\midrule
		\ili{Kalkatungu} & \example{ara-nti-} & ‘to insert sth.’ & \example{wani-nti-} & ‘to play with sb.’ \\
		\ili{Wik-Mungkan} & \example{ika-tha-} & ‘to split sth.’ & \example{kee’a-tha-} & ‘to play with sb.’ \\
		\ili{Margany} & \example{dhanggi-ma-} & ‘to drop sth.’ & \example{ngandhi-ma-} & ‘to talk to sb.’ \\
		\ili{Gunggari} & \example{banbu-ma-} & ‘to fell sth.’ & \example{ngalga-ma-} & ‘to talk to sb.’ \\
		\lspbottomrule
	\end{tabularx}
	\caption{\textsc{caus-appl} syncretism in Pama-Nyungan languages}
	\label{tab:ch7:caus-appl-pama-nyungan}
\end{table}

Note that some of the suffixes illustrated in \tabref{tab:ch7:caus-appl-pama-nyungan} barely have an applicative\is{applicative voice} function which may point towards a \isi{causative origin}. For instance, the applicative\is{applicative voice} function of the suffix \example{-lhile} in Mparntwe Arrernte\il{Arrernte, Mparntwe} is only attested with two verbs (\sectref{sec:simple-syncretism:caus-appl}) and the applicative\is{applicative voice} function of the suffix \example{-la} in \ili{Arabana-Wangkangurru} is only attested with five verbs \citep[11]{austin:2005}. By contrast, the applicative\is{applicative voice} function of the suffix \example{-la} in \ili{Pitta-Pitta} appears to be rather productive\is{productivity}, and the same is true for the \ili{Kalkatungu} suffix \example{-nti} \citep[12ff.]{austin:2005}.

\citet[12]{malchukov:2017} suggests that a “reanalysis from a causative\is{causative voice} to a \isi{benefactive} applicative\is{applicative voice} construction is under way” facilitated by \isi{sociativity} in the language isolate \ili{Seri} (\lang{na}) characterised by various prefixes, including \example{a(h)-} and \example{ac(o)-}. While this development is certainly probable, it is difficult to confirm with certainty due to the little historical and comparative data currently available for the language. The same is true for the causative-applicative suffix \example{-l} in the Araucanian language \ili{Mapuche} (or Mapudungun; \lang{sa}) also mentioned by \citet[9]{malchukov:2017}. Additionally, \citet{van-gysel:2018} has recently argued for cau\-sa\-tive-ap\-pli\-ca\-tive of \isi{causative origin} in the Chibchan language \ili{Pech} (\lang{na}) characterised by the prefix \example{ũː-}, in the Madang language \ili{Bongu} (\lang{pn}) characterised by the suffix \example{-t(e)}, and in the Edoid language \ili{Engenni} (\lang{af}) characterised by the suffix \example{-(e)se}. Unfortunately, there are very little data available on the former two languages, and it is difficult to determine not only the extent of the syncretism but also the chronology of the functions involved. In turn, \citeauthor{van-gysel:2018} tentatively speculates that the \ili{Engenni} prefix may be diachronically related to the \ili{Proto-Bantu} \example{*-is} discussed further above in which case the causative-applicative syncretism in the language would appear to be of \isi{causative origin} \citep{hyman:2007}.

As many of the languages discussed above show, there is little doubt that applicativity\is{applicative voice} has a close relationship to sociative\is{sociativity} causativity\is{causative voice}, prompting \citet[121]{shibatani:pardeshi:2002} to conclude that i) “the causative\is{causative voice}/applicative\is{applicative voice} syncretism is seen when there is a sociative\is{sociativity} reading associated with the causative\is{causative voice} constriction” and that ii) the split occurs at an advanced stage of grammaticalization/lexicalization”\is{grammaticalisation}\is{lexicalisation}. The split in question represents “a strong tendency […] to avoid the morphological causativization\is{causativisation} of active verbs [e.g., ‘to run’, ‘to play’, ‘to sit’], and to assign an applicative\is{applicative voice} function to the causative\is{causative voice} morphemes found with active verbs” \citep[118]{shibatani:pardeshi:2002}. It is not entirely clear what verbs qualify as ”active” though; for instance, they treat the verb ‘to stand’ variously as inactive and active \citep[116, 119]{shibatani:pardeshi:2002}. In any case, the tendency is essentially a logical consequence of the fact that a \isi{causer} can actively engage in such actions alongside the \isi{causee}, and the explanation thus seems plausible, especially for the rise of comitative\is{comitativity} and instrumental applicativity\is{applicative voice} as already briefly illustrated in the beginning of this section (e.g. ‘to make someone walk by walking with the person’ or ‘to cut something by using an instrument’). With regard to \isi{benefactive} applicativity\is{applicative voice}, \citet[11f.]{malchukov:2017} emphasises the assistive nature of sociative\is{sociativity} causativity\is{causative voice}, e.g. ‘to help someone sew a skirt’ → ‘to sew a skirt for someone’. These explanations apply primarily to the rise of syncretism between causativity\is{causative voice} and comitative/instrumental/\isi{benefactive}\is{comitativity} applicativity\is{applicative voice} but not necessarily to other types of applicativity\is{applicative voice}, e.g. locative. However, this does not pose a problem for the time being, because it currently appears that no language features causative-applicative syncretism of \isi{causative origin} involving applicativity\is{applicative voice} which is not comitative,\is{comitativity} instrumental, or \isi{benefactive}. Indeed, it has only been possible to find two languages featuring voice marking with both a causative\is{causative voice} function and a locative applicative\is{applicative voice} function, the Atlantic language \ili{Temne} (\lang{af}) and the Mixe-Zoque language Ayutla Mixe\il{Mixe, Ayutla} (\lang{na}), but in both languages this syncretism appears to be of \isi{applicative origin} (or, perhaps, the result of coincidental phonological \isi{convergence} in the latter case), as further discussed in \sectref{diachrony:appl2caus}.

\section{Applicative origin} \label{diachrony:applicative}
Voice syncretism of \isi{applicative origin} has received minimal attention in the literature, yet there appears to be some evidence for causative-applicative syncretism of \isi{applicative origin} (\sectref{diachrony:appl2caus}). By contrast, there are currently only weak indications of an \isi{applicative origin} for applicative-reciprocal and applicative-antipassive syncretism, as discussed in the next two sections.

\subsection{From applicative to reciprocal} \label{diachrony:appl2recp}
Diachronic development\is{diachronic development} from reciprocal\is{reciprocal voice} to applicative\is{applicative voice}\is{reciprocal origin} has been attested in a few languages (\sectref{diachrony:recp2appl}), whereas there is little solid evidence for the opposite development,\is{applicative origin} though vague hints of such development can be found among Eskimo-Aleut languages (\lang{na}). \citet[841]{fortescue:2007} argues that the \ili{Proto-Eskimo} suffix \example{*-utə} has applicative\is{applicative voice} and reciprocal\is{reciprocal voice} functions in all Eskimo languages, but cognates thereof only have the former function in the more distantly related \ili{Aleut} languages (\sectref{diachrony:recp2antp}). Furthermore, reflexes of the suffix also have a sociative\is{sociativity} function in Eskimo languages, for instance in West Greenlandic\il{Greenlandic, West} (e.g. \example{kavvisur-} ‘to drink coffee’ ↔ \example{kavvisu-up-} ‘to drink coffee together’, \citealt[827]{fortescue:2007}). The diachrony of the sociative\is{sociativity} function is not clear, but the distribution of reciprocal\is{reciprocal voice} and applicative\is{applicative voice} functions among the Eskimo-Aleut languages suggests that the reciprocal\is{reciprocal voice} function evolved following the applicative\is{applicative voice} function. As noted in \sectref{diachrony:recp2appl}, comitative\is{comitativity} applicativity\is{applicative voice} and \isi{sociativity} are rather similar in terms of semantics while reciprocity\is{reciprocal voice} is related to \isi{sociativity} in terms of \isi{plurality of participants}. Thus, it can tentatively be hypothesised that the reciprocal\is{reciprocal voice} function of \ili{Proto-Eskimo} \example{*-utə} evolved from the applicative\is{applicative voice} function facilitated by \isi{sociativity}, though more research is needed to confirm this scenario.

\subsection{From applicative to antipassive} \label{diachrony:appl2antp}
The applicative-reciprocal suffix \example{*-utə} in \ili{Proto-Eskimo} mentioned in the previous section is known to have developed an antipassive\is{antipassive voice} function in at least one descendant language, Central Alaskan Yupik\il{Yupik, Central Alaskan} (\lang{na}). The origin of applicative-antipassive syncretism in this language can thus be considered applicative\is{applicative voice},\is{applicative origin} at least partially (\sectref{diachrony:recp2antp}). It has hitherto not been possible to find evidence for a similar development in any other language.

\subsection{From applicative to causative} \label{diachrony:appl2caus}
Causative-applicative syncretism is generally believed to evolve from (sociative)\is{sociativity} causativity\is{causative voice} (\sectref{diachrony:caus2appl}), although the possibility of an opposite development is sporadically acknowledged in the literature. An early discussion of causative-applicative syncretism of \isi{applicative origin} is provided by \citet[110]{wise:1990} who argues that the suffix \example{-akag} (or cognate variants thereof) found in all Pre-Andine Arawakan languages derives from the \ili{Proto-Arawakan} reciprocal\is{reciprocal voice} suffix \example{*-kʰakʰ} and that “the meaning changed from reciprocal\is{reciprocal voice} to comitative\is{comitativity} to causative\is{causative voice}”. This view is adopted by \citet[501ff.]{payne:2002} who further explains that the suffix seems to have replaced the causative\is{causative voice} suffix \example{*-tʰa} among the languages. While this causative\is{causative voice} suffix and its original function is retained in a large number of modern Arawakan languages \citep[103]{wise:1990}, sporadic remnants of the suffix are “now devoid of a syntactic function” in the Pre-Andine Arawakan languages \citep[501]{payne:2002}. The presumed development among the Pre-Andine Arawakan languages is illustrated by examples from \ili{Asheninka} in \tabref{tab:ch7:appl-caus-asheninka}\is{applicative origin} \citep[491f., 501]{payne:2002}. Note that the suffix \example{-aka(g)} in \ili{Asheninka} has retained a reciprocal\is{reciprocal voice} function when preceded by the suffix \example{-aw} (e.g. \example{chek-aw-aka} ‘to cut e.o.’) which itself reflects the \ili{Proto-Arawakan} reflexive\is{reflexive voice} suffix \example{*-wa} \citep[109f.]{wise:1990}. In \ili{Ashéninka Perené} the latter suffix (cf. \example{-av}) seems to express reciprocity\is{reciprocal voice} on its own \citep[130]{mihas:2010}. Interestingly, \citet[488, 504]{payne:2002} even suggests that another causative\is{causative voice} suffix in \ili{Asheninka} with the variant forms \example{omin-/ogi-/ow-/o-} (e.g. \example{tyag-} ‘to fall over’ ↔ \example{o-tyag-} ‘to fell sth.’) also has a comitative\is{comitativity} \isi{applicative origin} derived from the verb \example{omintha} (the \example{-tha} element is an incorporated classifier for ‘word, language’) which is used for “deciding or encouraging someone to \textsc{accompany} the speaker somewhere”, e.g. \ili{Nomatsiguenga} \example{ominiC-} ‘to take along with, cause to accompany’. Nevertheless, it seems that this prefix does not retain a synchronic applicative\is{applicative voice} function in \ili{Asheninka} and it is therefore not discussed further here. 

\begin{table}
	\setlength{\tabcolsep}{2.2pt}
	\begin{tabularx}{\textwidth}{rcllll}
		\lsptoprule
		\ili{Proto-Arawakan} & \example{*-kʰakʰ} & & \textsc{appl} & → & \textsc{caus} \\
		\midrule 
		\ili{Asheninka} & \example{*-aka(g)} & \example{atait-aka-} & ‘to climb with sb.’ & → & ‘to make sb. climb’ \\
		\lspbottomrule
	\end{tabularx}
	\caption{\textsc{caus-appl} syncretism of \textsc{appl} origin in Asheninka}
	\label{tab:ch7:appl-caus-asheninka}
\end{table}



\citet{guillaume:rose:2010} argue that the prefix \example{him-} in the Arawakan language \ili{Yine} (\lang{sa}) -- which may be related to the \example{omin}-like prefixes in \ili{Asheninka} and \ili{Nomatsiguenga} \citep[195]{hanson:2010} -- also represents causative-applicative syncretism of \isi{applicative origin}. The comitative applicative function of the Yine prefix is very productive. However, it is not clear if it has yet developed a proper causative function, as \citet[276]{hanson:2010} only provides two examples with “cau\-sa\-tive overtones”: \example{him-hapoka-} ‘to arrive with sth.’ (‘to make sth. arrive’) and \example{him-satoka-} ‘to return with sth.’ (‘to make sth. return’). These examples illustrate the occasional problem of distinguishing between causativity\is{causative voice} and applicativity\is{applicative voice} discussed in \sectref{diachrony:caus2appl}. Similar cases can be found in other languages mentioned by \citet{guillaume:rose:2010}, including the language isolate \ili{Movima}, the Arauan languages \ili{Jarawara} and \ili{Paumarí} (all three \lang{sa}), as well as \ili{Yukatek Maya} (\lang{na}) also mentioned by \citet[12f.]{malchukov:2017} and \citet[236]{zuniga:kittila:2019}. In these languages there is clear applicative\is{applicative voice} voice marking (\example{-ɬe}, \example{ka-/wa-}, \example{va-/vi-}, and \example{t-}, respectively), which in some instances has an ambiguous causative\is{causative voice} reading. Consider, for example, \ili{Yukatek Maya} \example{áalkab-t-} ‘to run behind sb.’ or ‘to make sb. run’ (cf. \example{áalkab-} ‘to run’) in relation to causative\is{causative voice} \example{áalkab-ans-} ‘to make sb. run’ and applicative\is{applicative voice} \example{háakchek’-t-} ‘to slip on sth.’ (cf. \example{háakchek’-} ‘to slip’, \citealt[1452, 1457f.]{lehmann:2015}). Further research is needed to determine the extent and \isi{productivity} of such causative\is{causative voice} functions in these languages, but it is possible that they represent an early stage in the development of causative-applicative syncretism.

\citet{queixalos:2002} favours an \isi{applicative origin} for the causative-applicative syncretism characterised by the prefix \example{ka-} in the Guahiban language \ili{Sikuani} (\lang{sa}), as already briefly mentioned in \sectref{diachrony:caus2appl}. More specifically, \citet[320]{queixalos:2002} speculates that the prefix “could be etymologically related to the word for ‘hand’” and that “[o]ne of its possible senses -- presumably the most basic one -- is instrumental applicative\is{applicative voice}”. As described by \citet[392]{guillaume:rose:2010}, synchronically the prefix \example{ka-} in Sikuani “can have, on the one hand, a plain applicative\is{applicative voice} function, with no hint of causation, promoting\is{promotion} for instance an instrument into O function” and “[o]n the other hand, it can convey both comitative\is{comitativity} and causative\is{causative voice} meaning”. If the etymology proposed by \citet{queixalos:2002} can be confirmed, the diachronic scenario illustrated in \tabref{tab:ch7:appl-caus-sikuani} seems probable. 

\begin{table}
	\setlength{\tabcolsep}{3.3pt}
	\begin{tabularx}{\textwidth}{cllll}
		\lsptoprule
		‘hand’ & \textsc{appl} & → & \textsc{caus} & \\
		\midrule 
		\example{ka-} & \example{ka-nawiata} ‘to go back with sb.’ & & \example{ka-pitsapa} & ‘to make sb. go out’ \\
		& \multicolumn{1}{r}{(or ‘to make sb. go back’)} & & & \\
		\lspbottomrule
	\end{tabularx}
	\caption{\textsc{caus-appl} syncretism of \textsc{appl} origin in Sikuani}
	\label{tab:ch7:appl-caus-sikuani}
\end{table}

\citet{van-gysel:2018} has argued for an \isi{applicative origin} of causative-applicative syncretism in three languages spoken outside the Americas unlike the other languages covered so far in this section: the Northern Luzon language \ili{Pangasinan}, the Oceanic language \ili{Trukese} (both \lang{pn}), and the Atlantic language \ili{Temne} (\lang{af}). The purported causative-applicative syncretism in \ili{Pangasinan} is characterised by the prefix \example{pañgi-}, but unfortunately the data available for this prefix are very scant and seemingly restricted to a single example: \example{pañgi-tilák} ‘[I’ll] have [Juan] leave [the rice]’. \citet[140]{benton:1971} argues that the prefix in question is “[p]robably the least frequently encountered instrumental affix”. Consequently, it is difficult to determine the nature and \isi{productivity} of its causative\is{causative voice} and applicative\is{applicative voice} functions. By contrast, in \ili{Trukese} the suffix \example{-geni} has a clear applicative\is{applicative voice} function as well as a permissive causative\is{causative voice} function with at least two verbs \citep[52f.]{dyen:1965}. The suffix derives from the verb \example{(n)geni} ‘to give’ (see also \citealt[268]{goodenough:sugita:1980}) which \citet{van-gysel:2018} considers an indication for a \isi{applicative origin}. This presumed development is illustrated in \tabref{tab:ch7:appl-caus-trukese} \citep[53]{dyen:1965}. Nevertheless, it is worth noting that the verb ‘to give’ also is known to grammaticalise\is{grammaticalisation} a causative\is{causative voice} function, as in some Tungusic languages (\sectref{sec:simple-syncretism:caus-pass}). Thus, the voice development proposed here for the \ili{Trukese} suffix \example{-geni} is somewhat tentative, and more research into the chronology of its functions is needed to confirm the scenario. 

\begin{table}
	\setlength{\tabcolsep}{2.9pt}
	\begin{tabularx}{\textwidth}{cllll}
		\lsptoprule
		‘to give’ & & & & \\
		\example{(n)geni} & \textsc{appl} & \multicolumn{1}{r}{→} & \textsc{caus} & \\
		\midrule 
		\multirow{2}{*}{\example{-geni}} & \example{kupii-geni} & ‘to break sth. for sb.’ & \example{kkëwyy-geni} & ‘to let sb. stop’ \\
		& \example{jeniwin-geni} & ‘to return sth. to sb.’ & \example{jejiwen-geni} & ‘to let sb. lie down’ \\
		\lspbottomrule
	\end{tabularx}
	\caption{\textsc{caus-appl} syncretism of \textsc{appl} origin in Trukese}
	\label{tab:ch7:appl-caus-trukese}
\end{table}



\ili{Temne} is also a candidate for causative-applicative syncretism of \isi{applicative origin}, although the syncretism remains very limited in the language. As described by \citet[122ff., 167ff.]{kanu:2012}, \ili{Temne} features two productive\is{productivity} applicative\is{applicative voice} suffixes, \example{-(ə̀)r} and \example{-ʌ̀} that predominantly have a locative and \isi{benefactive} function, respectively. However, both suffixes can also have certain “idiosyncratic meanings” with some verbs, one of which appears to be causative\is{causative voice} \citep[136, 184]{kanu:2012}, though \citeauthor{kanu:2012} only provides one causative\is{causative voice} example  of the suffix \example{-(ə̀)r} and two causative\is{causative voice} examples of the suffix \example{-ʌ̀}. In any case, the suffixes appear to be related to the synchronic prepositions \example{rò} ‘to, from, in, on’ and \example{tà} ‘for’, respectively (\citealt[83]{kanu:2012}; cf. \citealt[156]{hyman:2007}) which -- together with the prominent applicative\is{applicative voice} use of the suffixes -- is a strong indicator of an \isi{applicative origin}. These developments are illustrated in \tabref{tab:ch7:appl-caus-temne} \citep[122, 135f., 176, 184]{kanu:2012}. The other applicative\is{applicative voice} example of the suffix \example{-ʌ̀} provided by \citeauthor{kanu:2012} but not shown in this table is \example{sə́kə̀th-ʌ̀} ‘to make sth. shift to sth.’ Moreover, note that the last vowel of the verb \example{tə́mʌ̀} ‘to stand up’ is replaced by \example{-ə̀r} in the causative\is{causative voice}. This phenomenon can also be seen among some verbs in which the suffix has an applicative\is{applicative voice} function (cf. \example{bánsʌ̀} ‘to be angry’ ↔ \example{bans-ə̀r} ‘to be angry at sb.’) but not all (cf. \example{yírʌ̀} ‘to sit’, \citealt[122, 132]{kanu:2012}). 

\begin{table}
	\begin{tabularx}{\textwidth}{clllll}
		\lsptoprule
		\example{rò} ‘on’ & \textsc{appl} & & → & \textsc{caus} & \\
		\midrule 
		\example{-(ə̀)r} & \example{yírʌ̀-ə̀r} & ‘to sit on sth.’ & & \example{tə́m-ə̀r} & ‘to make sb. stand up’ \\
		\example{-ʌ̀} & \example{wáy-ʌ̀} & ‘to buy sth. for sb.’ & & \example{bék-ʌ̀} & ‘to make sb. arrive’ \\
		\midrule
		\example{tà} ‘for’ & & & & & \\
		\lspbottomrule
	\end{tabularx}
	\caption{\textsc{caus-appl} syncretism of \textsc{appl} origin in Temne}
	\label{tab:ch7:appl-caus-temne}
\end{table}

The Mixe-Zoque language Ayutla Mixe\il{Mixe, Ayutla} (\lang{na}) features causative-applicative syncretism similar to that in \ili{Temne}, but the \isi{diachronic development} of the syncretism in this language is more uncertain. In Ayutla Mixe\il{Mixe, Ayutla} the syncretism in question is characterised by the prefix \example{a-}, yet \citet{romero-mendez:2009} appears to treat the prefix as two separate prefixes and does not address the similarity between them. On the one hand, \citet[97, 401f.]{romero-mendez:2009} states that one prefix \example{a-} is a “derivational\is{derivation} prefix that very often has a causative\is{causative voice} meaning” which generally derives\is{derivation} verbs from adjectives indicating change of state, but also “prefixes to verbs” (e.g. \example{tsë’ëk} ‘to be scared’ ↔ \example{a-tsë’ëk} ‘to scare sb.’). On the other hand, \citet[381ff., 602]{romero-mendez:2009} argues that another prefix \example{a-} diachronically derives from the word \example{ää} ‘mouth’ and has “a rather abstract meaning, indicating the trajectory of the action”, mostly ‘in’, ‘into’, or ‘inside’ a location (e.g. \example{tem-} ‘to roll’ ↔ \example{a-tem} ‘to roll into sth.’). It is unclear if the resemblance between the two prefixes \example{a-} is the result of coincidental phonological resemblance or if the causative\is{causative voice} function evolved from the applicative\is{applicative voice} function.

The boundaries between (sociative)\is{sociativity} causativity\is{causative voice} and applicativity\is{applicative voice} can be rather fluid and this helps explaining voice development from causative\is{causative voice} to applicative\is{applicative voice} (\sectref{diachrony:caus2appl}). There is no reason to assume that a voice development in the opposite direction cannot be explained in the same terms, only in a reverse manner. Indeed, the applicative\is{applicative voice} voices described for most of the languages in this section are similar to those discussed in \sectref{diachrony:caus2appl}, being instrumental, comitative,\is{comitativity} and/or \isi{benefactive} in nature. In fact, it seems that even the locative applicative\is{applicative voice} suffix \example{-(ə̀)r} in \ili{Temne} occasionally has \isi{benefactive} or \isi{benefactive}-like functions (e.g. \example{léŋ} ‘to sing’ ↔ \example{léŋ-ə̀r} ‘to sing to sb.’ and \example{bóyà} ‘to donate sth.’ ↔ \example{bóyà-r} ‘to donate sth. to sb.’), not to mention a \isi{malefactive} function with quite a few verbs \citep[131ff.]{kanu:2012}. Furthermore, the same suffix often indicates that an action is done ‘in the presence’ of someone which is reminiscent of a sociative\is{sociativity} function \citep[130]{kanu:2012}. Thus, the evolution of causative-applicative syncretism of \isi{applicative origin} essentially follows a reverse version of the developmental path from causative\is{causative voice} to applicative\is{applicative voice}, e.g. (instrumental) ‘to chop sth. with sth.’ → ‘to make sth. chop sth.’ → ‘to make sb. chop sth.’, (comitative)\is{comitativity} ‘to run with sb.’ → ‘to make sb. run by running with the person’ → ‘to make sb. run,’ (\isi{benefactive}) ‘to bake sth. for sb.’ → ‘to make sb. bake sth. by assisting the person’ → ‘to make sb. bake sth.’

\section{Overview} \label{diachrony:overview}
As demonstrated in this chapter, the diachrony of voice syncretism is an intricate and often unpredictable phenomenon which can seemingly follow a multitude of developmental paths. The various paths discussed in this chapter and their interrelationships are visualised in \figref{fig:ch07:syncretism-diachrony}. Dotted arrows indicate \isi{diachronic development} for which evidence remains cross-linguistically scarce and/or is deemed tentative, while solid arrows indicate development for which there is more evidence available. Showing only the seven voices of focus in this book, the figure represents a somewhat simplified diachronic overview of voice syncretism. As mentioned in the beginning of this chapter and suggested sporadically in the previous sections, many diachronic developments\is{diachronic development} of voice syncretism might be associated with various phenomena semantically related to one voice or another (e.g. \isi{plurality of relations}; \sectref{diachrony:reciprocal}) and specific bridging contexts\is{bridging context} (cf. \citealt{heine:kuteva:2007}) which are excluded in \figref{fig:ch07:syncretism-diachrony}. The reciprocal\is{reciprocal voice} voice is placed at the centre of the figure because it appears to be the only voice which can be linked diachronically to each of the six other voices in one way or another. These links are discussed further at the end of this section. By contrast, the reflexive\is{reflexive voice}, anticausative\is{anticausative voice}, passive\is{passive voice}, and causative\is{causative voice} voices are linked to four other voices each, and the applicative\is{applicative voice} and antipassive\is{antipassive voice} voices only to three other voices each.

\begin{figure}
	\centering
	\def\svgwidth{\textwidth}
	\input{figures/syncretism-diachrony.pdf_tex}
	\caption{Overview of the diachrony of voice syncretism}
	\label{fig:ch07:syncretism-diachrony}
\end{figure}

\figref{fig:ch07:syncretism-diachrony} indicates that several developmental paths are potentially bidirectional,\is{diachronic development, bidirectional} including paths that have traditionally been considered unidirectional\is{diachronic development, unidirectional} in the literature. For instance, although there is undoubtedly clear evidence for a \isi{diachronic development} from reflexive\is{reflexive origin} to anticausative\is{anticausative voice}\is{anticausative origin} to passive\is{passive voice} in some languages (\sectref{diachrony:reflexive}), voice syncretism in other languages has seemingly developed in the opposite direction: from anticausative\is{anticausative voice} to reflexive\is{reflexive voice} in \ili{Hittite} (\sectref{diachrony:antc2refl}), from passive\is{passive voice}\is{passive origin} to anticausative\is{anticausative voice} in Vedic \ili{Sanskrit} and the Tungusic language \ili{Evenki} (\sectref{diachrony:pass2antc}), and from passive\is{passive voice} directly to reflexive\is{reflexive voice} in the Lowland East Cushitic language \ili{Ts’amakko} (\sectref{diachrony:pass2refl}). Admittedly, evidence for these alternative scenarios is currently limited to a few isolated languages, yet the possibility of bidirectional\is{diachronic development, bidirectional} development on a larger scale is here kept open to encourage more research into the matter. Bidirectional\is{diachronic development, bidirectional} development between the causative\is{causative voice} and applicative\is{applicative voice} voices has previously been discussed notably by \citeauthor{malchukov:2015} (\citeyear{malchukov:2015, malchukov:2016}; \citeyear[24]{malchukov:2017}) who has designed a semantic map of “voice categories capturing selective similarities between individual categories” reproduced in \figref{fig:ch07:malchukov-map}. Not only does this semantic map show a bidirectional\is{diachronic development, bidirectional} connection between the causative\is{causative voice} and applicative\is{applicative voice} voices, it also connects the causative\is{causative voice} voice unidirectionally\is{diachronic development, unidirectional} to the passive\is{passive voice} voice and the applicative\is{applicative voice} voice unidirectionally\is{diachronic development, unidirectional} to the antipassive\is{antipassive voice} voice. There are thus clear similarities between \citeauthor{malchukov:2015}’s semantic map of voice similarities on the one hand and \figref{fig:ch07:syncretism-diachrony} showing the diachronic relations between different voices on the other hand. Observe also that neither the semantic map nor \figref{fig:ch07:syncretism-diachrony} propose any directionality between the passive\is{passive voice} and antipassive\is{antipassive voice} voices. Indeed, there is currently no good evidence for neither a development from passive\is{passive voice} to antipassive\is{antipassive voice}\is{antipassive origin} nor vice versa in any language. 

\begin{figure}
	\centering
	\def\svgwidth{\textwidth}
	\input{figures/syncretism-malchukov.pdf_tex}
	\caption{\citeauthor{malchukov:2017}’s (\citeyear{malchukov:2017}) semantic map of voice categories}
	\label{fig:ch07:malchukov-map}
\end{figure} 

\begin{sidewaystable}
	\begin{tabularx}{0.93\textwidth}{lcll}
		\lsptoprule
		\multicolumn{3}{c}{Path} & Languoids \\
		\midrule
		\textsc{refl} & → & \textsc{antp} & Indo-European, \ili{Nunggubuyu}, Cariban, Turkic, Iroquoian\\
		\textsc{recp} & → & \textsc{antp} & Bantu, Oceanic, Turkic, \ili{Nunggubuyu}, Central Alaskan Yupik\il{Yupik, Central Alaskan} \\
		\textsc{recp} & → & \textsc{caus} & \ili{Yine}, \ili{Khakas} \\
		\textsc{recp} & → & \textsc{appl} & Bantu, Turkic \\
		\textsc{pass} & → & \textsc{recp} & Highland East Cushitic \\
		\textsc{caus} & → & \textsc{antc} & \ili{Korean}, \ili{Evenki} \\
		\textsc{caus} & → & \textsc{pass} & \ili{Korean}, Tungusic, \ili{Mongolian}, Finno-Ugric, West Greenlandic\il{Greenlandic, West} \\
		\textsc{appl} & → & \textsc{antp} & Central Alaskan Yupik\il{Yupik, Central Alaskan} \\
		\midrule
		\textsc{refl} & → & \textsc{recp} & Indo-European, Nilotic, Dogon, \ili{Hup}, \ili{Jamul Tiipay}, Huasteca Nahuatl\il{Nahuatl, Huasteca}, \ili{Emerillon} \\
		\textsc{recp} & → & \textsc{refl} & Oceanic, \ili{Tariana}, \ili{Urubú-Ka’apor}, Gunwinyguan, \ili{Tuvan} \\
		\textsc{refl} & → & \textsc{antc} & Indo-European, \ili{Nivkh}, \ili{Nunggubuyu}, \ili{Jamul Tiipay}, Huasteca Nahuatl\il{Nahuatl, Huasteca}, \ili{Paresi-Haliti} \\
		\textsc{antc} & → & \textsc{refl} & \ili{Hittite} \\
		\textsc{refl} & → & \textsc{pass} & (Indo-European) \\
		\textsc{pass} & → & \textsc{refl} & \ili{Ts’amakko} \\
		\textsc{recp} & → & \textsc{antc} & Bantu, Turkic, \ili{Ngalakan} \\
		\textsc{antc} & → & \textsc{recp} & \ili{Hittite} \\
		\textsc{antc} & → & \textsc{pass} & Indo-European, \ili{Korean} \\
		\textsc{pass} & → & \textsc{antc} & \ili{Evenki}, Vedic \ili{Sanskrit} \\
		\textsc{caus} & → & \textsc{appl} & Bantu, Cariban \\
		\textsc{appl} & → & \textsc{caus} & \ili{Asheninka}, \ili{Sikuani}, \ili{Trukese}, \ili{Temne} \\
		\lspbottomrule
	\end{tabularx}
	\caption{Evidence for the diachrony of voice syncretism}
	\label{tab:ch7:evidence}
\end{sidewaystable}
 
The twenty developmental paths underlying \figref{fig:ch07:syncretism-diachrony} are listed in \tabref{tab:ch7:evidence}. The upper part of the table shows unidirectional\is{diachronic development, unidirectional} paths, while the lower part of the table shows bidirectional\is{diachronic development, bidirectional} paths. The table also provides an overview of the various evidence discussed for the paths in this chapter. The language families, genera\is{genus}, and languages (“languoids”) included in the table are intended to represent good candidates for the respective diachronic developments\is{diachronic development} in the light of currently available data. Consequently, languoids for which only highly tentative evidence for a given development has been discussed are not featured in \tabref{tab:ch7:evidence}. It is hoped that future research and additional data will lead to an expansion or reduction of languoids in the table. The functional explanations for the various developments in the table are diverse, and it is hardly feasible to subsume all the explanations discussed in this chapter under one notion. It is, however, worth noting that eight -- or almost half -- of the twenty developmental paths in the table involve reciprocity\is{reciprocal voice} in one way or another, and in some of these cases the diachronic developments\is{diachronic development} in question are jointly facilitated by functions closely associated with reciprocity\is{reciprocal voice}, including \isi{sociativity} and/or \isi{comitativity} \citep{nedjalkov:2007a} and/or \isi{co-participation} \citep{creissels:nouguier-voisin:2008} which can be subsumed under the notion of \isi{plurality of relations} (\sectref{diachrony:reciprocal}). This is notably the case for developments from reciprocal\is{reciprocal voice} to antipassive\is{antipassive voice} (\sectref{diachrony:recp2antp}), causative\is{causative voice} (\sectref{diachrony:recp2caus}), and applicative\is{applicative voice} (\sectref{diachrony:recp2appl}) in some languages. Nevertheless, in other cases, the semantics of reciprocity\is{reciprocal voice} itself are sufficiently similar to those of other voices to allow for voice syncretism to evolve, as in the bidirectional\is{diachronic development, bidirectional} developments of reflexive-reciprocal (\sectref{diachrony:refl2recp}, \sectref{diachrony:recp2refl}) and reciprocal-anticausative syncretism (\sectref{diachrony:recp2antc}, \sectref{diachrony:antc2recp}). Thus, no attempt is here made to unify the various explanations for voice syncretism of \isi{reciprocal origin} -- nor of any other voice origin for that matter.

