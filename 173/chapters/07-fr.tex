\documentclass[output=paper]{LSP/langsci}
\ChapterDOI{10.5281/zenodo.1228255} 
\author{Bjarke Frellesvig\affiliation{University of Oxford}\and 
Stephen Horn\affiliation{University of Oxford}\lastand 
Yuko Yanagida\affiliation{University of Tsukuba}}
\title{A diachronic perspective on differential object marking in pre-modern Japanese: Old Japanese and Early Middle Japanese}
\shorttitlerunninghead{A diachronic perspective on DOM in pre-modern Japanese}

\abstract{An exhaustive search of Old Japanese object NPs associated with weak floating quantifiers and question-focussed object NPs containing interrogative words confirms the suggestion made in \citet{Yanagidaetal2009Word}, confirmed by \citet{Frellesvigetal2015Differential}, that Old Japanese had differential object marking (DOM) with specificity (defined by \citealt{Frellesvigetal2015Differential} as D-linking) as a necessary condition.  Testing the same hypothesis on Early Middle Japanese, however, shows that this condition no longer obtained by the Heian Period. The resources for the expression of specificity and the set of conditions for differential object marking clearly changed over this span of the history of the Japanese language. 
}

\maketitle

\begin{document}

\section{Introduction}
\label{07-sec:1} 

Throughout its attested history \ili{Japanese} has exhibited variable object
marking: Some object NPs are marked by the \isi{accusative case} particle
\textit{wo},\footnote{The \isi{accusative case} particle \textit{wo} has been in
 use through the history of the language. Its phonemic shape changed
 to /o/ around the year 1000 AD due to regular sound change; we will
 refer to the \ili{Japanese} accusative particle as \textit{‘wo’} throughout
 the paper, except when citing examples, which have \textit{‘wo’} or
\textit{‘o’} depending on the age of the source. Like accusatives in
 many other languages, the \ili{Japanese} accusative has functions in
 addition to marking direct objects, mainly: marking adjuncts (path
 and source) and marking subjects raised to object and subjects in a
 few absolute constructions.} others are not. 
 We give two simple
examples from \ili{Old Japanese}
in~\REF{07-fr-ex:1}–\REF{07-fr-ex:2}.\footnote{Like
 modern \ili{Japanese}, \ili{Old Japanese} is head-final, has postposed
 particles, verbal suffixes in derivational and inflectional
 morphology, and pervasive pro-drop (whence many of the examples we
 cite have no overt subject). \ili{Old Japanese} has an extensive inventory
 of inflecting verbal suffixes, which are not found in modern
 \ili{Japanese}, expressing aspect, tense and mood. \ili{Old Japanese} does not
 have a \isi{nominative case} particle; subjects are sometimes bare and
 sometimes marked by one of the two \isi{genitive case} particles \textit{no}
 and \textit{ga}. In modern \ili{Japanese} \textit{ga} has become a nominative
 case particle, whereas \textit{no} remains a genitive in modern
 \ili{Japanese}. See further \citet{Frellesvig2010Japanese} about premodern
 \ili{Japanese}.}\textsuperscript{,}\footnote{Examples are transcribed in a time-appropriate
 phonemic transcription (see \citealt[33, 176]{Frellesvig2010Japanese}
 for simple transcription guidelines). 
 }

\ea \label{07-fr-ex:1}
\gll {\ob}\textsubscript{NP} kwomatu ga	sita no	kaya wo{\cb}	kara-sane\\
~ small.pine \textsc{gen}	under \textsc{gen}	grass \textsc{acc}	cut-\textsc{resp}.\textsc{opt}\\
\glt ‘(I) want (you) to cut some of the grass under the small pine.’   (MYS 1.11)
\z

\begin{exe}
\ex \label{07-fr-ex:2}%2
\gll akami-yama	{\ob}\textsubscript{NP} kusane Ø{\cb}	kari-soke\\
Akami-mountain	~ grass	 ~ cut-remove\\
\glt ‘cutting and removing grasses at Mount Akami...’   (MYS 14.3479)
\end{exe}

Phenomena suggesting the existence of differential object marking
(DOM) in \ili{Old Japanese} (OJ) have long been noted, and hypotheses about
the trigger for DOM in OJ have been developed and refined in recent
decades \citep{Motohashi1989Case, Yanagida2006Word,
 Yanagidaetal2009Word} to the point where a robust formulation of a
condition on DOM in OJ has now been proposed and tested in a survey of
OJ object noun phrases of a few selected types
\citet{Frellesvigetal2015Differential}.

In the present study we present an expanded and exhaustive survey of
OJ along the same lines as in \citet{Frellesvigetal2015Differential},
and then proceed to extend these same techniques to a body of texts
representative of the immediately following historical variant of
\ili{Japanese}, Early Middle \ili{Japanese} (EMJ), in order to ascertain whether
the DOM system of the \ili{Japanese} of the Asuka and Nara periods (as
represented by texts from 712 CE to 797 CE) persists into the Heian
period (as represented by texts from 900 CE to 1110 CE).

First we define the necessary condition for triggering DOM in OJ,
viz. specificity defined as D-linking (see~\sectref{07-sec:2}
below). Next we describe how we used the Oxford Corpus of \ili{Old Japanese}
\citet{Frellesvigetal2014Corpus} to determine that reference to this
condition contributes to an observationally adequate description of
DOM in OJ. Next we present the methods and results of a similar survey
of EMJ using the Historical Corpus of \ili{Japanese} 
\citet{NINJAL}, which
show clear and significant differences in object marking between Old
\ili{Japanese} and the immediately following period of Early Middle
\ili{Japanese}. In the discussion in~\sectref{07-sec:4}, we summarize and
discuss these findings and identify areas for further research.

\section{The conditions for DOM in OJ}
\label{07-sec:2}

In the data set used for the study on DOM in OJ (described in more
detail below) there are in total 4094 \isi{direct object} noun phrases
(NPs). Of these object NPs, 1946 (47.5\%) are marked with the
\isi{accusative case} particle \textit{wo}. It is evident that there is
variation in object marking
(\cf also~\REF{07-fr-ex:1},~\REF{07-fr-ex:2} above), and
the initial question is whether that is dependent on some factor or
combination of factors. Following
\citet{Frellesvigetal2015Differential}, we find that the alternation
found in \ili{Old Japanese} is related to a non-inherent discourse-based
argument property. In this respect the distribution of OJ accusative
case marker \textit{wo} is similar in many respects to that of the
\ili{Turkish} \isi{accusative case} suffix \textit{-i} for direct objects
\citep{Enc1991Semantics}. Also note that object marking alternation
in OJ is found in wh-NPs under question focus (\eg \textit{idure wo ka
 wakite sinwopamu} ‘Which (of them) shall I praise, separating it
out?’), which means that \emph{wo}-marking does not imply topichood. Rather,
we find that a necessary condition for DOM in OJ is a weak form of
\textit{specificity} which we define in terms of D-linking, the
working definition of which we set out as follows:

\begin{exe}
\ex%3
\label{07-fr-ex:3}
\textit{D-linking:} A relationship between an NP and a \isi{definite} discourse referent, whereby the possible reference of that NP is restricted. 
\end{exe}

\citet{Pesetsky1987Binding} used the term D(iscourse)-linking to
characterize \textit{wh-}NPs such as ‘which student’ as having a
special property due to their membership in a \isi{definite} superset, this
being, moreover, a property with consequences for syntax. Generally,
\textit{X} in \textit{which X} is linked with a \isi{definite} discourse
entity insofar as ‘which’ is uninterpretable without a presupposed
superset, thus such D-linked \textit{wh-}NPs are weakly ‘specific’
\citep{Cinque1990Dependencies,Kiss1993WH-movement}. As an example
with an overt superset rather than a merely presupposed one, consider
the expression \textit{Among the students in this year’s cohort, which
 is the best?}

Extending this idea, it is clear that this weak specificity can accrue
to \textit{wh-}NPs other than those containing ‘which’: For example,
the contextual material accompanying the \textit{wh-}NP ‘whom’ in the
expression \textit{Among the students in this year’s cohort, whom
 should we trust?} is sufficient to render that \textit{wh-}NP
D-linked. The phrase ‘what else’ in the expression \textit{What else
 do you want?} is D-linked to a \isi{definite} discourse entity by the
relation of exclusion, as that narrows the possible reference of the
NP ‘what else’.\footnote{It follows that DOM conditioned by D-linking
 can trigger an interpretation of weak specificity in a
\textit{wh-}NP that would otherwise be construed as non-specific (as
\citealt[210–211]{Dalrympleetal2011Objects} observe for \ili{Persian}).
 While all of the examples of \textit{wo-}marked \textit{wh-}NPs in
 our OJ data are accompanied by contextual material for D-linking
 (see \sectref{07-fr-sec:2-2}), this is valuable new information for the
 interpretation of \textit{wo-}marked objects in general in \ili{Old Japanese} texts.}

Further extending the idea, we see that the same kind of weak
specificity can be a significant property of \isi{indefinite} NPs that do not
contain \textit{wh-}words at all, established through the same kinds
of D-linking relations, a typical example being that of a \isi{definite}
possessive NP complement, as in \textit{the farm's products}, but
potentially established in a variety of ways (\eg \textit{a man on
 the bus}, \textit{a limb off the tree, another glass of beer},
etc.). We also stipulate that D-linking is not an irreflexive
relation. Thus a \isi{definite} NP is D-linked through the relation of
identity it has with itself. By this we also include co-indexing
through previous mention and pronominal reference as a way to
establish D-linking. Thus we account for the distribution of
accusative \isi{case marking} on both \isi{definite} objects and \isi{indefinite}
specific ones by reference to one principle.

Needless to say, there are also many ways for the \isi{definite} discourse
referents upon which these various relations depend to find their way
into the common ground: previous mention, ostension, presupposition
accommodation, uniqueness, etc. In OJ, the effect of weak specificity
can be seen in the near-minimal pair \REF{07-fr-ex:1} and \REF{07-fr-ex:2} given above. The object NP
in \REF{07-fr-ex:1} is modified by an NP complement containing the NP
\textit{kwomatu} ‘small pine’ and is marked by \isi{accusative case}
particle \textit{wo}. Because \textit{kwomatu} ('the small pine') is
\isi{definite}, as the context in the poem shows, the reference of the
object NP \textit{kaya wo} ‘grass’ is at least weakly specific due to
the D-linking relation which maps the whole object NP to the \isi{definite}
discourse entity denoted by the NP complement. Accordingly, we
translate the object NP here as a having at least a partitive
relation: \textit{some of the grass under the small pine,} but the NP
also could potentially refer to \textit{all the grass under the small
 pine}. The marking of the object NP conforms to the fact that it has
at least weak specificity, which property satisfies a necessary
condition for object marking. By contrast, the object NP in \REF{07-fr-ex:2} 
\textit{kusane} ‘grass’ is unmodified and unmarked, consistent with
non-specific reference, which we translate here with an English plural
common noun ‘grasses’. When we look at the wider context of the
expression in \REF{07-fr-ex:2} we see that a non-specific amount of grasses are cut
in order to open a space for lying down. The presence and absence of
object \isi{case marking} seen respectively in \REF{07-fr-ex:1} and \REF{07-fr-ex:2} corresponds to the presence and absence of specificity in the reference of the NPs.

This analysis is supported by \citegen{Yanagida2006Word} observation that a
great preponderance of unmarked object noun phrases in OJ \il{Old Japanese} are composed
of unmodified common nouns, while \textit{wo-}marked object NPs are
frequently modified by NP complements as in \REF{07-fr-ex:1} or by relative
clauses. Overt modification, while restricting the possible reference
of the NP so modified, does not by itself ensure a D-linking
relationship, but reference to a \isi{definite} discourse entity within the
modifying material is one way to establish D-linking, which in turn
licenses object marking. Needless to say, stronger types of
specificity, including epistemic specificity and \isi{definiteness}, also
license object marking in OJ.

Furthermore, as observed by many (\citealt{Matsuo1944Kyakugo};
\citealt[48]{Matsunaga1983Japanese}; \citealt{Miyagawa1989Structure};
\citealt{Yanagida2006Word}), object-marking in OJ \il{Old Japanese} is associated with
leftward movement, so that, for example, in this SOV language,
\textit{wo-}marked object NPs co-occurring with genitive-marked
subject NPs appear to the left of those subject NPs (\eg example
\REF{07-fr-ex:16} below), with extremely few
exceptions. \citet{Yanagidaetal2009Word} identify this as a movement
to a position outside of the domain of existential closure in the verb
phrase. This is a phenomenon common to specific object NPs, as
described by \citet{Diesing1992Indefinites}, \textit{inter alia}. In
contrast, bare, unmodified, common noun-headed object NPs in OJ  \il{Old Japanese}
commonly appear adjacent to the verb
\citep{Yanagida2006Word,Yanagidaetal2009Word}. These distributions
conform very well with what we observe here.\footnote{Note, however,
 that there are rare examples of unmodified \textit{wo-}marked NPs
 that appear adjacent to the selecting verb (possibly cases of
 vacuous movement). Conversely leftward movement does not imply
 specificity, as \textit{wh-}items in question focus are regularly
 left-shifted, and many of these are non-specific.}

 \largerpage
We also note that object NPs composed of personal pronouns
(\citealt{Wronaetal2010Japanese}, \textit{inter alia}) and NPs
modified by demonstratives are also fairly regularly object
marked. However, we also find clearly specific object NPs that are
unmarked. For example, we found 47 object NPs containing demonstrative
\textit{ko} ‘this’ at some structural level. All of these NPs are
specific, and indeed many of them are \isi{definite}, but while 25 are
\isi{accusative case} marked as predicted, \eg \REF{07-fr-ex:4}, 22 of them are bare,
\eg \REF{07-fr-ex:5}.

\begin{exe}
\ex \label{07-fr-ex:4}%4
\gll \textbf{ko} \textbf{no} \textbf{miki} \textbf{wo} kami-kyemu pito pa\\
this \textsc{gen} wine \textsc{acc} chew-must.have person \textsc{top}\\
\glt ‘as for the person who must have brewed \textbf{this wine}’  (KK 40)
\end{exe}

\begin{exe}
\ex \label{07-fr-ex:5}%5
\gll Yamato no womura no take ni sisi pusu to tare ka \textbf{ko} \textbf{no} \textbf{koto} opomapye ni mawosu?\\
 Yamato \textsc{gen} Womura \textsc{gen} peak \textsc{dat} game lie \textsc{comp} who \textsc{foc} this \textsc{gen} content Emperor \textsc{dat} say\\
\glt ‘That deer lie on the peaks in Womura in Yamato ---who is it that says \textbf{this thing} to His Majesty?’ (NSK 75a, b)
\end{exe}

The pattern for object marking in OJ  \il{Old Japanese} outlined above may be summarized as follows:

\begin{exe}
\ex%6
\label{07-fr-ex:6}
\renewcommand{\labelenumi}{\alph{enumi}}
\begin{enumerate}
\item \label{07-fr-ex:6a} Accusative case marked objects are specific;
\item \label{07-fr-ex:6b} non-specific objects are not accusative case marked;
\item \label{07-fr-ex:6c} not all specific objects are accusative case marked.
\end{enumerate}
\end{exe}

This leads us to form the following hypothesis: 

\begin{exe}
\ex%7
\label{07-fr-ex:7}
\textit{Condition on DOM in OJ:} Specificity is a necessary
condition for object marking in OJ, the weakest form of specificity
being D-linking. However, specificity is not a sufficient condition
for object marking in OJ.
\end{exe}

In this paper we focus on this condition and its falsifiability, but
do not to any significant extent discuss the – important – issue of
when specific objects in OJ  \il{Old Japanese} are not \isi{accusative case} marked. See,
however, \sectref{07-sec:4} for some remarks on this.

The hypothesis that some kind of specificity is a necessary but not
sufficient condition for DOM is falsifiable by finding an
unambiguously non-specific NP which is also accusatively
marked. Unfortunately, there is no linguistic pattern in OJ  \il{Old Japanese} that can
be said to be an unambiguous and categorical marker of
non-specificity, making it difficult to search for counterevidence to
our hypothesis on the basis of linguistic forms in an electronic
corpus. But there are at least two classes of object NPs which, other
things being equal, have reference that is normally non-specific: (i)
object NPs associated with weak Floating Quantifiers (FQs) of the form
[numeral + classifier],\footnote{For example, in the Modern \ili{Japanese}
 (NJ) expression \textit{Dooro de sika o rop-piki mita} ‘On the road
 I saw six deer’, a non-specific object NP \textit{sika} ‘deer’ is
 associated with the FQ \textit{rop-piki} ‘six-animal’. For this
 first class of object NPs, in special cases the reference can be
 specific, and indeed even \isi{definite}, but the function of the FQ
 ceases to be weakly quantifying in such cases (discussed in more
 detail below) (\citealt{Kim1995Quantifier}, \textit{inter alia}).}
and (ii) object NPs containing \textit{wh-}words (except for
\textit{idure} ‘which’, discussed in more detail below) and having
question focus.\footnote{For example, in the NJ expression
\textit{Mado kara dare o mimasita ka}  ‘From the window, whom did you
 see?’ the object NP \textit{dare o} ‘whom’ is under question
 focus. For this second class of object NPs, the reference is at most
 only weakly specific, and that only under special conditions
 (discussed in more detail below).} As it is, variable object
marking is attested among both of these classes of NPs, suggesting
that under marked conditions both types of NP can have specific
reference. In order to establish a systematic and exhaustive method
which can also be applied to following stages of the language, with
much larger volumes of material available, we therefore examined all
attestations of these two classes of object NPs using the \textit{Oxford
Corpus of Old Japanese} (OCOJ, \citealt{Frellesvigetal2014Corpus}), with the aim of demonstrating whether a
D-linking relation would be retrievable for the \textit{wo-}marked
object NPs. If not, such examples could constitute counterevidence to
the hypothesis about DOM.


The data set we used for the OJ  \il{Old Japanese} survey was extracted from the
September 2014 version of the OCOJ \citep{Frellesvigetal2014Corpus}, 
which primarily uses sources from the \textit{Nihon koten bungaku
 taikei} (Iwanami shoten, 1957–1962) as critical editions. We used
a sub-corpus comprising all extant poetic texts from 712 CE to 797 CE,
drawing material from the following sources: \textit{Kojiki kayō,
 Nihon shoki kayō, Fudoki kayō, Bussokuseki-ka, Shoku nihongi
 kayō, Manyōshū}. It is thought that some of the poetic texts
in these works are considerably older than the earliest date of
compilation. The volume of the corpus is 4,979 poems, comprising
89,419 words.

We looked at the two types of NPs which would under normal, unmarked
conditions be non-specific. As predicted, the exhaustive examination
of these object NPs showed that:

\begin{exe}
\ex%8
\label{07-fr-ex:8}
\renewcommand{\labelenumi}{\alph{enumi}}
\begin{enumerate}
\item There is a correspondence between accusative \isi{case marking} and specific interpretations for these two types of NPs (corroborated by the presence of contextual clues); and 
\item NPs of these two types receiving unambiguously non-specific interpretations (again corroborated by contextual clues) are bare.
\end{enumerate}
\end{exe}

Details for both types of NP are presented together with examples in
the sections that follow. In the remarks that follow we only discuss
the reference of marked object NPs, as only these serve as potential
counterevidence to the hypothesis for a condition on DOM in OJ  \il{Old Japanese} \REF{07-fr-ex:7}.

\subsection{Specificity of object NPs associated with weak floating quantifiers in OJ}
\label{07-fr-sec:2-1}

Out of the attested 100 expressions with the form of weak FQs in the
data set, we found 4 attestations of FQs both indexed with
\textit{wo-}marked object NPs and functioning as adverbial
modifiers of the predicate selecting their respective host NPs (see
examples \REF{07-fr-ex:9}--\REF{07-fr-ex:11}). In all cases the reference of the host NP was in fact
\isi{definite}. When an FQ that, other things being equal, would be
interpreted as weakly quantifying (\eg a cardinal FQ) is paired with
a \isi{definite} host NP, the resulting expression is construable in two
ways; either as meaning ‘\textit{n-}members of a \isi{definite} superset,’
(\ie ‘\textit{n} of them', where the FQ behaves as what we might
call a partitive quantifier), or as a cardinally specified universal
quantifier (\eg ‘both', \ie ‘all of them, with a cardinality of
2’). The interpretations presented in the examples below are derived
accordingly. We present all four examples in the following.

The \isi{definiteness} of the host NP in example \REF{07-fr-ex:9} derives from the fact that the relative clause modifying the head noun \textit{kamwi} ‘god’ serves to define a \isi{definite} superset: ‘those gods known as Chinese Tigers’.\footnote{It is well-known that NPs of the form \textit{X to iu Y} ‘Y which is called X’ regularly form \isi{definite} descriptions.} The FQ \textit{ya-tu} ‘eight-thing’ functions as a partitive quantifier ‘eight members of the superset’. 

\begin{exe}
\ex%9
\label{07-fr-ex:9}
\gll \textbf{karakuni} \textbf{no} \textbf{twora} \textbf{to} \textbf{ipu} \textbf{kamwi} \textbf{wo} ikedorini \textbf{ya-tu} tori-moti-ki\\
China \textsc{gen} tiger COMP say god \textsc{acc} live.take.as eight-thing take-hold-come\\
\glt ‘...taking and bringing by capturing live \textbf{eight of those gods called Chinese Tigers}...’   (MYS 16.3885)
\end{exe}

In \REF{07-fr-ex:10} the \isi{definiteness} of the host NP \textit{sinokipa} ‘arrow’
derives from a combination of metaphor and previous mention, explained
in detail in \citet{Frellesvigetal2015Differential}. The FQ functions
as a cardinally specified universal quantifier ‘both’.


\begin{exe}
\ex%10
\label{07-fr-ex:10}
\gll …adusa-yumi yu-bara puri-okosi \textbf{sinokipa} \textbf{wo} \textbf{puta-tu} ta-basami panati-kyemu pito si kuti-wosi\\
catalpa-bow bow-belly swing-raise arrow \textsc{acc} two-thing hand-pinch loose-must.have person \textsc{res} mouth-regrettable\\
\glt ‘Deplorable, the person who (…) must have raised a bow, pinched \textbf{both} \textbf{those} \textbf{arrows}, and shot them away!’   (MYS 13.3302)
\end{exe}

The \isi{definiteness} of the two host NPs in example \REF{07-fr-ex:11} \textit{u} ‘cormorant’ is inferred from the method of fishing referred to in this poem, which involves using exactly eight cormorants carried four to a basket, two baskets to a pole (see \citealt[202]{Frellesvigetal2015Differential}). 
Thus, the two FQs function as cardinally specified universal quantifiers ‘all eight’. 

\begin{exe}
\ex%11
\label{07-fr-ex:11}
\gll kami tu se ni \textbf{u} \textbf{wo} \textbf{ya-tu} kaduke simo tu se ni \textbf{u} \textbf{wo} \textbf{ya-tu} kaduke\\
upper \textsc{gen} stream \textsc{dat} cormorant \textsc{acc} eight-thing make.dive lower \textsc{gen} stream \textsc{dat} cormorant \textsc{acc}  eight-thing make.dive\\
\glt ‘...making \textbf{all} \textbf{eight} \textbf{of} \textbf{[my]} \textbf{cormorants} dive in the upper reaches, making \textbf{all} \textbf{eight} \textbf{of} \textbf{[my]} \textbf{cormorants} dive in the lower reaches...’   (MYS 13.3330) 
\end{exe}

Again, the reference for every \textit{wo-}marked host NP of FQ is \isi{definite}. Given our definition of D-linking in \REF{07-fr-ex:3} and the stipulation that \isi{definite} NPs are D-linked by reflexive identity, we determine that all \textit{wo-}marked object NPs associated with FQs in OJ  \il{Old Japanese} are at least weakly specific in reference. Accordingly, for this class of NPs, no counter-evidence to the hypothesis is found. 

\newpage 
\subsection{Specificity of object NPs containing \textit{WH}-words with question focus}
\label{07-fr-sec:2-2}

The set of \textit{wh-}words in OJ  \il{Old Japanese} is as follows: 

\begin{exe}
\ex%12
\label{07-fr-ex:12}
\textit{WH-words in OJ:}\textit{ ta, tare} ‘who’;
\textit{idu} ‘where’; \textit{iduku} ‘where’; \textit{idura} ‘where
(abouts)’; \textit{idupye} ‘which direction’; \textit{idure} ‘which’;
\textit{idusi} ‘which side’; \textit{iduti} ‘which direction’;
\textit{ika} ‘how’; \textit{iku} ‘how many’; \textit{ikubaku}
‘how much’; \textit{ikuda} ‘how much’; \textit{ikupisa(sa)} ‘how long
ago’; \textit{ikura} ‘how much’; \textit{ikutu} ‘how much’;
\textit{itu} ‘when’; \textit{nado, ado} ‘why’; \textit{na, nani}
‘what’; \textit{ani} ‘how’; \textit{uremuso} ‘why’
\end{exe}

The OCOJ has 469 occurrences of \textit{wh-}words. Out of these, we identified 70 that are contained in object NPs, of which 21 are \textit{wo-}marked. Of these \textit{wo-}marked NPs containing \textit{wh-}words, there are 18 which have question focus (\ie are themselves \textit{wh-}NP objects). As for the remaining 3 object NPs, they do not have question focus, either due to the focus being discharged within a complement clause embedded in a relative clause, or due to the \textit{wh-}word functioning as a quantifier, or both. For example, in \REF{07-fr-ex:13} below, the \textit{wh-}word (\textit{itu} ‘when’) is contained in an adverb NP (\textit{itu si ka mo}) of a complement clause (\textit{itu … mimu to}) embedded in a relative clause (\textit{itu … omopisi}) modifying the head (\textit{apasima}) of an object NP. The force of the \textit{wh-}word is discharged at the level of the complement clause. The whole utterance forms a yes/no question. 

\begin{exe}
\ex%13
\label{07-fr-ex:13}
%\langinfo{Old Japanese}{}{(MYS 15.3631)}\\
\gll {\ob}\textsubscript{NP} \textbf{itu} \textbf{si} \textbf{ka} \textbf{mo} \textbf{mi-mu} \textbf{to} \textbf{omopi-si} \textbf{apa-sima} \textbf{wo}{\cb} yoso ni ya kwopwi-mu\\
{ } when \textsc{res} \textsc{foc} \textsc{etop} see-will \textsc{comp}  think-\textsc{spst} Awa-island \textsc{acc}  afar \textsc{dat} \textsc{foc} yearn-will\\
\glt ‘Shall [I] have to yearn from afar for \textbf{Awa Island, about which [I] thought, “When will [I] see it?”}?’   (MYS 15.3631)
\end{exe}

Non-question focus object NPs such as these are excluded from
consideration, because they can easily have \isi{definite} reference, as
does in fact the example in \REF{07-fr-ex:13}.

Out of the \textit{wh-}words in OJ, listed above, only the following
appear in the formation of \textit{wo-}marked object \textit{wh-}NPs:
\textit{ika} ‘how’; \textit{ta}, \textit{tare} ‘who’; \textit{nani}
‘what’; \textit{idure} ‘which’. Under normal, unmarked conditions, NPs
containing such \textit{wh-}words (with the exception of
\textit{idure} ‘which’) would be non-specific. However, significantly,
in these 18 examples, the NPs containing them are \textit{wo-}marked
and in fact weakly specific in reference. For example, in \REF{07-fr-ex:14} 
immediately below, the reference of the \textit{wh-}NP headed by
\textit{yosi} ‘opportunity’ is associated with a \isi{definite} event that
occurred by chance, the D-linking established through the relationship
of exclusion: ‘what manner of opportunity other than by chance’. In
all 18 examples \REF{07-fr-ex:14}--\REF{07-fr-ex:31}, the \textit{wo-}marked object
\textit{wh-}NP is accompanied by contextual material by which that NP
is construable as related to a \isi{definite} discourse entity. While we
cannot include here all the considerations by which the judgments on
reference status were made due to lack of space, we reflect as much as
possible in the translations.
 
\subsubsection{\textit{ika} ‘how’}
\label{07-subsubsec:2-2-1}

\begin{exe}
\ex%14
\label{07-fr-ex:14}
\gll tamasakani wa ga mi-si pito wo \textbf{ika} \textbf{nara-mu} \textbf{yosi} \textbf{wo} motite ka mata pito-me mi-mu\\
by.chance I \textsc{gen} see-\textsc{spst} person \textsc{acc} how  \textsc{cop}-will  opportunity \textsc{acc}  holding \textsc{foc} again one-glimpse see-will\\
\glt ‘The person whom I met by chance –having \textbf{what other manner of opportunity is it that [I] will see a glimpse of her again?’}  (MYS 11.2396)
\end{exe} 

\largerpage[2]
\subsubsection{\textit{ta/tare} ‘who’}
\label{07-subsubsec:2-2-2}

\begin{exe}
\ex%15
\label{07-fr-ex:15}
\gll yamato no takasazinwo wo nana yuku wotomye-domo \textbf{tare} \textbf{wo} \textbf{si} maka-mu\\
Yamato \textsc{gen} Takasazino \textsc{acc} seven go girl-\textsc{pl}  who   \textsc{acc}  \textsc{res} wrap-will\\
\glt ‘As for the seven maidens walking along the plain Takasazi in Yamato -- \textbf{whom (of them)} will [you] wed?’  (KK 15)
\end{exe}

\begin{exe}
\ex%16
\label{07-fr-ex:16}
\gll nagatukwi no sigure no ame no yama-gwiri no asibuseki a ga mune \textbf{ta} \textbf{wo} miba yama-mu\\
9th.month \textsc{gen} shower \textsc{gen} rain \textsc{gen} mountain-mist as fretful I \textsc{gen} breast who \textsc{acc} see.if stop-will\\
\glt ‘As for my breast which is fretting like the mountain mist of the rain showers of the 9\textsuperscript{th} month, if [I] see \textbf{whom (other than you)} shall it quieten?’  (MYS 10.2263a)
\end{exe}

\begin{exe}
\ex%17
\label{07-fr-ex:17}
\gll maywone kaki \textbf{tare} \textbf{wo} \textbf{ka} mi-mu to omopitutu ke-nagaku kwopwi-si imo ni ap-yeru kamo\\
eyebrow scratch who \textsc{acc}  \textsc{foc}  see-will \textsc{comp} think.while days-long yearn-\textsc{spst} beloved \textsc{dat} meet-\textsc{stat} \textsc{sfp}\\
\glt ‘Scratching [my] eyebrow, thinking, “\textbf{Whom (other than you)} am [I] about to see?” here [I] am meeting my beloved (\ie you) whom [I] have longed for day in and day out!’  (MYS 11.2614b)\\
\end{exe}

\begin{exe}
\ex%18
\label{07-fr-ex:18}
\gll kapyeru beku toki pa nari-kyeri miyakwo nite \textbf{ta} \textbf{ga} \textbf{tamoto} \textbf{wo} \textbf{ka} wa ga makuraka-mu\\
return ought time \textsc{top} become-\textsc{mpst} Capital \textsc{cop}  who \textsc{gen} sleeve \textsc{acc} \textsc{foc} I \textsc{gen} lie.upon-will\\
\glt ‘The time has come for [us] to return. In the capital, \textbf{the sleeve of whom (other than my departed wife)} shall I use as my pillow?’   (MYS 3.439)\\
\end{exe}

\begin{exe}
\ex%19
\label{07-fr-ex:19}
\gll asigara no ya-pye-yama kwoyete imasi-naba \textbf{tare} \textbf{wo} \textbf{ka} kimi to mitutu sinwopa-mu\\
Ashigara \textsc{gen} eight-fold-mountain Crossing come-\textsc{pfv}.if  who  \textsc{acc}  \textsc{foc} lord \textsc{comp} seeing praise-will\\
\glt ‘If [you] cross the eight-fold mountains of Ashigara, then \textbf{whom (else)} shall [I], thinking [it] to be my lord, admire?’  (MYS 20.4440)\\
\end{exe}

\subsubsection{\textit{nani} ‘what’}
\label{07-subsubsec:2-2-3}

\begin{exe}
\ex%20
\label{07-fr-ex:20}
\gll kasuga.nwo no pudi pa tiri-nite \textbf{nani} \textbf{wo} \textbf{ka} \textbf{mo} mi-kari no pito no worite kazasa-mu\\
Kasuga.field \textsc{gen} wisteria \textsc{top} scatter-\textsc{pfv}.\textsc{ger}  what \textsc{acc} \textsc{foc} \textsc{etop} \textsc{pfx}-hunt \textsc{gen} person \textsc{gen} breaking.off don-will\\
\glt ‘The wisteria flowers on Kasuga fields having scattered, \textbf{what else} shall the hunters break off and wear on their heads?’  (MYS 10.1974)\\
\end{exe}

\begin{exe}
\ex%21
\label{07-fr-ex:21}
\gll kokoro sape matur-eru kimi ni \textbf{nani} \textbf{wo} \textbf{ka} \textbf{mo} ipa-zu ipi-si to wa ga nusumapa-mu\\
heart even offer.up-\textsc{stat} lord \textsc{dat} what \textsc{acc}  \textsc{foc}  \textsc{etop} say-\textsc{neg} say-\textsc{spst} \textsc{comp} I \textsc{gen} steal-will\\
\glt ‘To you whom [I] have given the very meaning (my very heart), \textbf{what (else)} would I steal from you by saying, “[It] is a thing which was said without speaking”?’  (MYS 11.2573)\\
\end{exe}

\begin{exe}
\ex%22
\label{07-fr-ex:22}
\gll moti no pi ni sasi-iduru tukwi no takatakani kimi wo imasete \textbf{nani} \textbf{wo} \textbf{ka} omopa-mu\\
mid.month \textsc{gen} day \textsc{dat} direct-come.out moon as refinedly lord \textsc{acc} making.come what \textsc{acc} \textsc{foc} think-will\\
\glt ‘Having you come resplendently like the moon that comes out on the 15th of the month, \textbf{what} \textbf{(}\textbf{else}\textbf{)} could [I] wish for?’  (MYS 12.3005)\\
\end{exe}

\begin{exe}
\ex%23
\label{07-fr-ex:23}
\gll yama-gapi ni sak-yeru sakura wo tada pito-me kimi ni mise-teba \textbf{nani} \textbf{wo} \textbf{ka} omopa-mu\\
mountain-saddle \textsc{dat} bloom-\textsc{stat} cherry.blossom \textsc{acc} just one-glimpse lord \textsc{dat} show-\textsc{pfv}.if  what \textsc{acc} \textsc{foc} think-will\\
\glt ‘If [I] managed to show my lord just once the cherry blossoms that bloom in the saddle of the mountain, \textbf{what (else)} could [I] wish for?’ (MYS 17.3967)
\end{exe}

\begin{exe}
\ex%24
\label{07-fr-ex:24}
\gll ipye ni yukite \textbf{nani} \textbf{wo} katara-mu asipikwino yama-pototogisu pito-kowe mo nakye\\
home \textsc{dat} going  what \textsc{acc} recount-will (pillow.word) mountain-cuckoo one-chirp \textsc{etop} cry.\textsc{imp}\\
\glt ‘Mountain cuckoo, sing even one note! Going home, \textbf{what (other than that)} shall [I] recount?’   (MYS 19.4203)
\end{exe}

\begin{exe}
\ex%25
\label{07-fr-ex:25}
\gll ima-sarani \textbf{nani} \textbf{wo} \textbf{ka} omopa-mu uti-nabiki kokoro pa kimi ni yori-ni-si monowo\\
now-newly what \textsc{acc} \textsc{foc}  think-will \textsc{pfx}-lie.down heart \textsc{top} lord \textsc{dat} depend-\textsc{pfv}-\textsc{spst} given.that\\
\glt ‘At this late date, \textbf{what more} could [one] ask for, given that [my] heart, lying down, has given itself over to you?’   (MYS 4.505)
\end{exe}

\begin{exe}
\ex%26
\label{07-fr-ex:26}
\gll ame-tuti wo terasu pi-tukwi no kipami naku aru beki monowo \textbf{nani} \textbf{wo} \textbf{ka} omopa-mu\\
heaven-earth \textsc{acc} illuminate sun-moon as limit lacking be ought given.that  what \textsc{acc} \textsc{foc}  think-will\\
\glt ‘Given that [it] must have no limit, just as the sun and moon which illuminate heaven and earth, \textbf{what else} could [one] wish for?’  (MYS 20.4486)
\end{exe}

\begin{exe}
\ex%27
\label{07-fr-ex:27}
\gll sipo pwi-naba tama-mo kari-tume ipye no imo ga pama-dutwo kopaba \textbf{nani} \textbf{wo} simyesa-mu\\
tide ebb-\textsc{pfv}.if jewel-weed cut-gather.\textsc{imp} home \textsc{gen} beloved \textsc{gen} beach-souvenir beg.if  what \textsc{acc} proffer-will\\
\glt ‘When the tide goes out, cut and gather some jewel-seaweed. If my darling at home asks for a beach souvenir, \textbf{what (other than that)} would [we] proffer?’  (MYS 3.360)
\end{exe}

\subsubsection{\textit{idure} ‘which’}
\label{07-subsubsec:2-2-4}

The \textit{wh-}word \textit{idure} ‘which’ is inherently specific, and NPs headed by \textit{idure} (\eg in \REF{07-fr-ex:29} below) or with \textit{idure} as a direct NP complement to the head (\eg in \REF{07-fr-ex:28}, \REF{07-fr-ex:30}, \REF{07-fr-ex:31} below) are specific. There are 4 examples of an object \textit{wh-}NP formed with \textit{idure} as a head or as a direct NP complement, and as expected all are \textit{wo-}marked.


\begin{exe}
\ex%28
\label{07-fr-ex:28}
\gll asipikwino tama-kadura no kwo kyepu no goto \textbf{idure} \textbf{no} \textbf{kuma} \textbf{wo} mitutu ki-ni-kyemu\\
(pillow.word) jewel-vine \textsc{gen} child today \textsc{gen} like which \textsc{gen}  bend \textsc{acc}  seeing come-\textsc{pfv}-must.have\\
\glt ‘Oh child of the jewel-vine, seeing \textbf{which bends in the mountain road} must [you] have come here, as [I come] today?’ (MYS 16.3790)
\end{exe}

\begin{exe}
\ex%29
\label{07-fr-ex:29}
\gll \textbf{idure} \textbf{wo} \textbf{ka} wakite sinwopa-mu\\
 which \textsc{acc} \textsc{foc}  separating praise-will\\
\glt ‘...\textbf{Which} shall [I] praise, separating [it] out? ...’   (MYS 18.4089)
\end{exe}

\protectedex{
\begin{exe}
\ex%30
\label{07-fr-ex:30}
\gll \textbf{watatumi} \textbf{no} \textbf{idure} \textbf{no} \textbf{kamwi} \textbf{wo} inoraba ka yuku sa mo ku sa mo pune no paya-kye-mu\\
 sea.god \textsc{gen} which \textsc{gen} god \textsc{acc}  supplicate.if \textsc{foc} go way \textsc{etop} come way \textsc{etop} boat \textsc{gen} fast-be-will\\
\glt ‘\textbf{Which god of the sea} is it that, if [I] beseech it, the boat will be fast both on the way out and the way back?’  (MYS 9.1784)\
\end{exe}
}

\begin{exe}
\ex%31
\label{07-fr-ex:31}
\gll \textbf{ame-tusi} \textbf{no} \textbf{idure} \textbf{no} \textbf{kami} \textbf{wo} inoraba ka utukusi papa ni mata koto-twopa-mu\\
heaven-earth \textsc{gen} which \textsc{gen} god \textsc{acc} beseech.if \textsc{foc} dear mother \textsc{dat} again word-ask-will\\
\glt ‘\textbf{Which of the gods of heaven and earth} is it that, if [I] beseech it, [I] will speak again to my dear mother?’   (MYS 20.4392)
\end{exe}

Thus, for object NPs containing \textit{wh-}words and having question focus, which under normal, unmarked conditions would be expected to have non-specific reference, all \textit{wo-}marked examples are demonstrably D-linked and thereby specific, so that no counter-evidence to the hypothesis about the condition on DOM in OJ  \il{Old Japanese} \REF{07-fr-ex:7} is found. 

\section{Does the DOM system of OJ persist into EMJ?}
\label{07-sec:3}

In this section we will address the question of whether Early Middle
\ili{Japanese} (EMJ, 900 CE to 1110 CE) exhibits the same system of DOM as
OJ, concluding that it does not. We will show that in EMJ both
specific and nonspecific objects may be \textit{wo-}marked or bare,
unlike OJ  \il{Old Japanese} which disallows non-specific \textit{wo-}marked objects. We
will first show three examples (all taken from \textit{Makura no
 Sōshi} %TODO add source
) which show that EMJ, like OJ, had \textit{wo-}marked
specific objects \REF{07-fr-ex:32}, bare specific objects \REF{07-fr-ex:33}, and bare non-specific objects \REF{07-fr-ex:34}. Following that we will present the results of an investigation of whether EMJ had non-specific \textit{wo-}marked
objects.

In \REF{07-fr-ex:32} the object denotes particular body parts of previously
mentioned people, and as such the reference is D-linked and
specific. The object NP is \-\textit{wo-}marked.

\begin{exe}
\ex
\label{07-fr-ex:32}
\textbf{Specific, \textit{wo}-marked object NP}\\
%\langinfo{Early Modern Japanese}{}{(Makura no sōshi, 3, Shinpen Zenshū, vol. 18, p.\,28)}\\ %TODO add source
\gll uta-rezi to youi site tuneni \textbf{usiro} \textbf{o} kokoro-dukawi si-taru kesiki\\
hit-\textsc{pass}.will.not \textsc{comp} preparation doing constantly  behind \textsc{acc}  heart-dispatch do-\textsc{stat} sight\\
\glt ‘the sight of [them] constantly guarding \textbf{[their] behinds} taking care lest [they] be struck’  (Makura no sōshi, 3, Shinpen Zenshū, vol.\,18, p.\,28)
\end{exe}

In \REF{07-fr-ex:33} previous mention of \textit{augi} ‘fan’ and \textit{putokorogami} ‘pocket paper’ is seen in the immediately preceding context, establishing D-linking through the relation of previous mention, and yet both object NPs are bare. 

\begin{exe}
\ex
\label{07-fr-ex:33}
\textbf{Specific, bare object NP}\\
%\langinfo{Early Modern Japanese}{}{(Makura no sōshi, 61, Shinpen Zenshū, vol. 18, p.\,117)}\\
\gll \textbf{augi} \textbf{tatau-gami} \textbf{nado} yobe makura-gami ni oki-sikado onodukara pika-re tiri-ni-keru o motomuru ni kurakereba ikade ka wa mi-mu idura idura tataki-watasi mi-idete \textbf{augi} putaputato tukawi \textbf{putokoro-gami} sasi-irete makari-na-mu to bakari koso iu rame\\
fan folding-paper etc.  last.night pillow-head \textsc{dat} put-\textsc{spst}.although naturally pull-\textsc{pass} scatter-\textsc{pfv}-\textsc{mpst} \textsc{acc} search \textsc{dat} dark.because how \textsc{foc} \textsc{top} see-will where where pat-cross see-putting.out  fan (mimetic) use  pocket-paper stick-put.in go.home-\textsc{pfv}-will \textsc{comp} \textsc{res} \textsc{foc} say \textsc{ext}\\
\glt ‘Although [he] had put \textbf{[his] fan and folded paper and such} at the head of his pillow the night before, when [he] searches [for them] among the things that naturally became disturbed and scattered, it being dark, how shall [he] ever find them?— saying, “Where? Where?” patting the whole area, and finding them, [he] uses \textbf{[his] fan}, “woosh-woosh,” and sticking \textbf{[his] pocket-paper} in, what [he] would surely say is something like, “[I'll] be going now”.’  (Makura no sōshi, 61, Shinpen Zenshū, vol.\,18, p.\,117)
\end{exe}

The example in \REF{07-fr-ex:34} is the first entry in a list entitled ``Despicable things''. There is no previous context other than the title of the list, and given the public nature of the list and the negative evaluations levied on the items therein, anything more than a non-specific reference would be unthinkable. The object NP is bare. 

\begin{exe}
\ex
\label{07-fr-ex:34}
\textbf{Non-specific, bare object NP}\\
%\langinfo{Early Modern Japanese}{}{(Makura no sōshi, 26, Shinpen Zenshū, vol. 18, p.\,65)}\\
\gll nadeu koto naki pito no we-gati nite \textbf{mono} itau iwi-taru\\
any.in.particular thing lacking person \textsc{gen} laugh-tending \textsc{cop}  thing extremely say-\textsc{stat}\\
\glt ‘A person who has nothing to commend himself, smugly saying \textbf{things} volubly.’  (Makura no sōshi, 26, Shinpen Zenshū, vol.\,18, p.\,65)
\end{exe}

As mentioned, \REF{07-fr-ex:32}--\REF{07-fr-ex:34} conform to the observed OJ  \il{Old Japanese}
distribution of specificity and \isi{case marking}. In order to examine
whether the DOM system of OJ  \il{Old Japanese} is also found in EMJ we investigated
systematically the existence of \textit{wo-}marked non-specific
objects, which were disallowed in OJ. We used the methodology outlined
above and examined object NPs associated with weak FQs and object
\textit{wh-}NPs with question focus, using the Heian Japanese
sub-corpus of the Historical Corpus of Japanese (NINJAL \citeyear{NINJAL}) in
conjunction with the Chūnagon search application available from the
National Institute of Japanese Language and Linguistics. The Heian
Japanese sub-corpus of the Historical Corpus of Japanese represents
prose and poetry in texts produced between 900 CE to 1110 CE, using
texts from the \textit{Shinpen Nihon koten bungaku zenshū}
(Shogakkan, 1994) %TODO add source
as critical editions. The Heian sub-corpus
is composed of the following texts: \textit{Kokin wakashū, Tosa
 nikki, Taketori monogatari, Ise monogatari, Ochikubo monogatari,
 Yamato monogatari, Makura no sōshi, Genji monogatari, Murasaki
 Shikibu monogatari, Izumi Shikibu monogatari, Sarashina nikki,
 Sanuki no suke nikki, Heichū monogatari, Kagerofu nikki, Tutumi
 Chūnagon monogatari.} The texts are primarily prose, with some
poetry. The sub-corpus contains 738,153 words.

Exhaustively examining NPs in EMJ fitting the same description as that
for OJ  \il{Old Japanese} outlined above, we found that the Condition on DOM in OJ  \il{Old Japanese} in
\REF{07-fr-ex:7} does not hold for EMJ. The evidence for this conclusion
comes in the form of \textit{wh-}marked non-specific object NPs. In a
situation where there are no overt forms that can be used to
unambiguously mark NPs as having non-specific reference, a
demonstration of this evidence relies on close examination of the
previous context, and various considerations about the most plausible
interpretations of NPs that appear in the text.

\subsection{Specificity of object NPs associated with weak floating quantifiers in EMJ}
\label{07-fr-sec:3-1}

A search of the sub-corpus for object NPs associated with weak FQs in
EMJ yielded results from texts produced between 900 CE
(\textit{Taketori monogatari}) and 1010 CE (\textit{Genji
 monogatari}). We found 512 expressions of the form
[numeral+classifier] in Heian texts. Among these we found 80 examples
associated with object NPs. Of these 80 object NPs, 8 are accusative
case marked, and if the OJ  \il{Old Japanese} system of DOM were to persist in EMJ, we
would expect all 8 \textit{wo-}marked object NP hosts of FQs
to be specific in reference. However, of the 8 \textit{wo-}marked
objects, 3 are arguably non-specific. We give all three examples
below. For example, in \REF{07-fr-ex:35} below, a simile is drawn to a
hypothetical situation in which two plums are stuck in the place where
eyes should be. There is no mention of these plums in previous
context, and they have no links to \isi{definite} discourse referents.


\begin{exe}
\ex%35
\label{07-fr-ex:35}
%\langinfo{Early Modern Japanese}{}{(Taketori monogatari, Shinpen Zenshū, vol. 12, p.\,48)}\\
\gll karouzite oki-agari-tamap-eru wo mireba kaze ito omoki pito nite para ito pukure konata kanata no me ni pa \textbf{sumomo} \textbf{wo} \textbf{puta-tu} tukeru yau nari\\
barely sit.up-rise-\textsc{resp}-\textsc{stat} \textsc{acc} see.when illness very heavy person \textsc{cop} belly very swell this.side that.side \textsc{gen} eye \textsc{dat} \textsc{top}  plum \textsc{acc} two-thing  attach appearance \textsc{cop}\\
\glt ‘... looking at [him] as [he] barely managed to raise himself, [he] was like someone with a terrible cold, [his] belly swelled up and it was as if [someone] had stuck \textbf{two plums} to [his] eyes on the one side and the other.’  (Taketori monogatari, Shinpen Zenshū, vol.\,12, p.\,48)
\end{exe}

In \REF{07-fr-ex:36} below, there is no previous mention of bridges in relation to the place called Yatsuhashi. They are newly introduced and are unlinked to any \isi{definite} discourse referent. 
 
\begin{exe}
\ex%36
\label{07-fr-ex:36}
%\langinfo{Early Modern Japanese}{}{(Ise monogatari, Shinpen Zenshū, vol.12, p.\,120)}\\
\gll Mikapa no kuni yatupasi to ipu tokoro ni itari-nu soko wo yatupasi to ipi-keru pa midu yuku kapa no kumo-de nareba \textbf{pasi} \textbf{wo} \textbf{ya-tu} watas-eru ni yorite namu yatupasi to ipi-keru.\\
Mikawa \textsc{gen} country Yatsuhashi \textsc{comp} say place \textsc{dat} arrive-\textsc{pfv} this.place \textsc{acc} Yatsuhashi \textsc{comp} say-\textsc{mpst} \textsc{top} water go river \textsc{gen} spider-hand \textsc{cop}.because  bridge \textsc{acc} eight-thing cross-\textsc{stat} \textsc{dat} depending \textsc{foc} Yatsuhashi \textsc{comp} say-\textsc{mpst}\\ 
\glt ‘[They] came to a place called Yatsuhasi. As for its being called Yatsuhashi, it was due to the fact that [they] spanned \textbf{eight bridges} over it, because the river of water divided into spider legs, that [they] called it “Yatuhashi”.’  (Ise monogatari, Shinpen Zenshū, vol.\,12, p.\,120)\\
\end{exe} 

In \REF{07-fr-ex:37} below, the main character is depicted as doing something unexpected and marvelous: releasing fireflies into a woman’s bedchamber. Both the fireflies and the cloth panel he used to conceal them are newly introduced into the scene and have no links to a \isi{definite} discourse referent. 


\begin{exe}
\ex%37
\label{07-fr-ex:37}
%\langinfo{Early Modern Japanese}{}{(Genji monogatari: ‘Hotaru’, Shinpen Zenshū, vol. 22, p.\,200) }\\
\gll yori-tamawite \textbf{mikityau} \textbf{no} \textbf{katabira} \textbf{wo} \textbf{pito-pe} uti-kake-tamau ni awasete sa.to pikaru mono ga\\
depend-\textsc{resp}.\textsc{ger} standing.blind \textsc{gen} panel \textsc{acc} one-layer \textsc{pfx}-hang-\textsc{resp} \textsc{dat} matching suddenly glow thing \textsc{gen}\\
\glt ‘…and just as [Otodo], drawing near, draped \textbf{a panel from a standing blind} (over the crossbeam), suddenly something glowing …’  (Genji monogatari: ‘Hotaru’, Shinpen Zenshū, vol.\,22, p.\,200) 
\end{exe}

Non-specific expressions of this form are unattested in OJ  \il{Old Japanese} and violate the condition on DOM \REF{07-fr-ex:7}, indicating that the OJ  \il{Old Japanese} system of DOM is no longer operative in EMJ. 

\subsection{Specificity of object NPs containing WH-words in EMJ}
\label{07-fr-sec:3-2}

Given the fact that the EMJ sub-corpus does not include mark-up of constituents larger than the unit ‘word’, and has made no provision for the annotation of grammatical role, it is impossible to mechanically identify object NPs in general, including those associated with FQs (as above) and those containing \textit{wh-}words (as below). Rather, attestations of the distinguishing part of speech have to be examined individually to determine their syntactic position and to determine the grammatical roles of the constituents which they distinguish. Given the difficulties of working with the EMJ sub-corpus, for this study we restricted our search to just variants of two \textit{wh-}words for comparison with OJ: \textit{ta, tare} ‘who’ and \textit{na}, \textit{nani} ‘what’. It will be recalled that the \textit{wh-}words in OJ  \il{Old Japanese} which figure in the formation of \textit{wo-}marked object \textit{wh-}NPs are the following: \textit{ika} ‘how’; \textit{ta, tare} ‘who’; \textit{na,} \textit{nani} ‘what’; \textit{idure} ‘which’. 

\subsubsection{WH-word \textit{tare} ‘who’}
\label{07-subsubsec:3-2-1}

A search of the sub-corpus for object NPs containing \textit{wh-}word \textit{ta, tare} ‘who’ yielded results from texts produced between 900 CE (\textit{Taketori monogatari}) and 1110 CE (\textit{Sanuki no suke nikki}). We found 553 NPs containing the \textit{wh-}word \textit{tare, ta} ‘who’. Of those, 21 are grammatical objects. Of the 21 grammatical objects, 18 are accusative marked. Again, if the OJ  \il{Old Japanese} system of DOM were to persist in EMJ, we would expect all 18 accusative marked examples to be specific. However, of these 18, 7 have question focus, and thus would have a non-specific interpretation under normal, unmarked circumstances. Upon inspection, we find no evidence to indicate that the reference for these is indeed anything but non-specific. For example, in the question in \REF{07-fr-ex:38}, there is a background assumption that no one is supposed to know the things that the addressee speaks about as a matter of course. Accordingly it is extremely unlikely that there is assumed in the question a \isi{definite} set of people from whom the addressee might learn such things. 

\begin{exe}
\ex%38
\label{07-fr-ex:38}
%\langinfo{Early Modern Japanese}{}{(Makura no sōshi, 131, Shinpen Zenshū, vol. 18, p.\,248)}\\
\gll \textbf{tare} \textbf{ga} \textbf{osiwe} \textbf{o} kikite pito no nabete siru beu mo ara-nu koto o-ba iu zo\\
who \textsc{gen} teaching \textsc{acc} hearing people \textsc{gen} lining.up know should \textsc{etop} exist-\textsc{neg} word \textsc{acc}-\textsc{top} say \textsc{foc}\\
\glt ‘Having heard \textbf{whose teachings} is it that [you] say these things which people invariably aren’t supposed to know?’ (Makura no sōshi, 131, Shinpen Zenshū, vol.\,18, p.\,248)
\end{exe}

In \REF{07-fr-ex:39} below, the combination of question particle \textit{ka} and topic particle \textit{wa} form a rhetorical question: there is no expectation of a concrete answer, so the reference of \textit{tare} is arguably non-specific. 


\begin{exe}
\ex%39
\label{07-fr-ex:39}
%\langinfo{Early Modern Japanese}{}{(Genji monogatari, ‘Yūgiri’, Shinpen Zenshū, vol. 23, p.\,451)}\\
\gll ima wa katazikenaku mo \textbf{tare} \textbf{o} \textbf{ka} \textbf{wa} yoru-be ni omowi-kikoe-tamawa-n\\
now \textsc{top} regrettably \textsc{etop}  who \textsc{acc} \textsc{foc} \textsc{top}  depend-place \textsc{cop} think-\textsc{resp}-\textsc{resp}-will\\
\glt ‘From here on — and [I] am terribly sorry to be saying this, but — \textbf{whom(ever)} might [you] consider as a benefactor?’  (Genji monogatari, ‘Yūgiri’, Shinpen Zenshū, vol.\,23, p.\,451)
\end{exe}

In \REF{07-fr-ex:40}--\REF{07-fr-ex:42}, the questions focus on previously unintroduced third-person entities. There is no obvious source of any basis for D-linking. 

\begin{exe}
\ex%40
\label{07-fr-ex:40}
%\langinfo{Early Modern japanese}{}{(Kokin wakashū, Shinpen Zenshū, vol. 11, p.\,101)}\\
\gll aki.kaze ni patu.kari ga ne zo kikoyu naru \textbf{tare} \textbf{ga} \textbf{tamaadusa} \textbf{wo} kakete ki-tu ramu\\
autumn.wind \textsc{dat} first.goose \textsc{gen} cry \textsc{foc} be.audible \textsc{ext}  who \textsc{gen} missive \textsc{acc}   hanging come-\textsc{pfv} \textsc{ext}\\
\glt ‘The voices of the first geese can be heard on the autumn wind. \textbf{Whose missives} do [they] come bearing?’ (Kokin wakashū, Shinpen Zenshū, vol.\,11, p.\,101)
\end{exe}

\begin{exe}
\ex%41
\label{07-fr-ex:41}
%\langinfo{Early Modern Japanese}{}{(Kokin wakashū,, Shinpen Zenshū, vol. 11, p.\,100)}\\
\gll momidiba no tirite tumor-eru wa ga yado ni \textbf{tare} \textbf{wo} matu.musi kokora naku ramu\\
red.leaves \textsc{gen} scatter pile.up-\textsc{stat} I \textsc{gen} dwelling \textsc{dat}  who \textsc{acc} await.insect around.here cry \textsc{ext}\\
\glt ‘In my dwelling on which autumn leaves, falling, have piled up — \textbf{whom} must the matsumushi be awaiting? — the matsumushi cries around here’  (Kokin wakashū, Shinpen Zenshū, vol.\,11, p.\,100)
\end{exe}

\begin{exe}
\ex%42
\label{07-fr-ex:42}
%\langinfo{Early Modern Japanese}{}{(Genji monogatari, ‘Tamakazura’, Shinpen Zenshū, vol. 22, p.\,90)}\\
\gll puna.ko-domo no araarasiki kowe nite uraganasiku mo tooku kana to ki-ni-keru utau o kiku mama ni puta-ri sasi-mukawite naki-keri puna.bito mo \textbf{tare} \textbf{o} kou to ka oo{-}sima no ura kanasi-geni kowe no kikoyuru\\
boat.man-\textsc{pl} \textsc{gen} rough voice \textsc{cop} mournfully \textsc{etop} from.afar \textsc{sfp} \textsc{comp} come-\textsc{pfv}-\textsc{mpst} singing \textsc{acc} listen thus \textsc{cop} two-people direct-face cry-\textsc{mpst} boat.man \textsc{etop}  who \textsc{acc} yearn.for \textsc{comp} \textsc{foc} \=O{}-island \textsc{gen} bay sad-appearing voice \textsc{gen} be.audible\\
\glt ‘Even as [they] heard the boatmen in their rough voices singing, “Heartlorn, [we]’ve come so far!” the two faced each other and cried. So \textbf{whom} do the boatmen long for? Voices from \=O Island Bay sound so heartsick.’  (Genji monogatari, ‘Tamakazura’, Shinpen Zenshū, vol.\,22, p.\,90)
\end{exe}

Finally in \REF{07-fr-ex:43}--\REF{07-fr-ex:44}, there is no mention in the previous context of a \isi{definite} superset of suitors out of which one specific suitor might be picked. It may be argued that the social context might delimit a \isi{definite} set of candidates, so the claim of non-specificity for these two examples is not as strong as that for the previous five. 

\begin{exe}
\ex%43
\label{07-fr-ex:43}
%\langinfo{Early Modern japanese}{}{(Ochikubo monogatari, Shinpen Zenshū, vol. 17, p.\,89)}\\
\gll medetaki ya \textbf{tare} \textbf{wo} \textbf{ka} tori-tamau to notamaweba sa.daisyau.dono no sakon.no.seusyau to ka\\
fortunate \textsc{sfp} who \textsc{acc} \textsc{foc} take-\textsc{resp} \textsc{comp} say.when Left.Major.Captain \textsc{gen} Minor.Captain \textsc{comp} \textsc{foc}\\
\glt ‘As [he] said, “That’s fortunate. \textbf{Whom} is [she] receiving (as a groom)?” [she] replied, “(I am given to understand) that it is the son of the Major Captain of the Left, the Minor Captain,” …’  (Ochikubo monogatari, Shinpen Zenshū, vol.\,17, p.\,89) 
\end{exe}

\begin{exe}
\ex%44
\label{07-fr-ex:44}
%\langinfo{Early Modern Japanese}{}{(Ochikubo monogatari, Shinpen Zenshū, vol. 17, p.\,147)}\\
\gll omuko no seusyau \textbf{tare} \textbf{wo} tori-tamau zo to towi-kereba sa.daisyau no sakon.no.seusyau.dono to\\
groom \textsc{gen} Minor.Captain who \textsc{acc} take-\textsc{resp} \textsc{foc} \textsc{comp} say-\textsc{mpst}.when Left.Major.Captain \textsc{gen} Left.Minor.Captain \textsc{comp}\\
\glt ‘As the husband, Minor Captain Kurauto, asked, “\textbf{Whom} will [she] take (as a groom)?” [she] replied, “(Mother says) [it] is the son of the Major Captain of the Left, the Minor Captain of the Left,”...’ (Ochikubo monogatari, Shinpen Zenshū, vol.\,17, p.\,147)
\end{exe}

Again, non-specific expressions of this form are unattested in OJ  \il{Old Japanese} and violate the condition on the OJ system of DOM, providing further evidence that the OJ system of DOM is no longer operative in EMJ. 

\subsubsection{WH-word \textit{nani} ‘what’}
\label{07-subsubsec:3-2-2}

A search of the sub-corpus for object NPs containing \textit{wh-}word \textit{na, nani} ‘what’ yielded results from texts produced between 900 CE (\textit{Taketori monogatari}) and 1110 CE (\textit{Sanuki no suke nikki}). We found 825 NPs containing the \textit{wh-}word \textit{na, nani} ‘what’. Of those, 113 are grammatical objects. Of the 113 grammatical objects, 39 are accusative marked. Of the 39 \textit{wo-}marked grammatical objects, 13 have question focus and are arguably non-specific in reference. For example, in \REF{07-fr-ex:45} below the speaker is expressing dismay at not being summoned in time for a funeral. The underlying assumption in the question is that there could only have been some unknown sort of prohibition preventing the addressee from sending an invitation. There is no mention of prohibitions in the previous context, nor does the speaker actually wait for an answer to the question, suggesting the absence of any presupposed superset related to \textit{nani no monoimi o} ‘what manner of prohibition?’. Similarly, in the remaining examples, open-ended questions are asked: ‘what in heaven’s name?’; ‘whatever?’ 


\begin{exe}
\ex%45
\label{07-fr-ex:45}
%\langinfo{Early Modern Japanese}{}{(Sanuki no suke nikki, Shinpen Zenshū, vol. 26, p.\,420-1)}\\
\gll ana kokoro u ya rei-sama ni mi-pirake-tamai-tu ran o ima pito-tabi mi-maira-se-zu nari-nuru kokoro usa o \textbf{nani} \textbf{no} \textbf{monoimi} \textbf{o} site yobi-tamawa-zari-turu zo\\
Ah heart despondent \textsc{sfp} usual-way \textsc{cop} see-open-\textsc{resp}-\textsc{pfv} \textsc{ext} \textsc{acc} yet one-time see-\textsc{hum}-\textsc{caus}-\textsc{neg} become-\textsc{pfv} heart despondency \textsc{acc}  what  \textsc{cop} prohibition \textsc{acc}  doing call-\textsc{resp}-\textsc{neg}-\textsc{pfv} \textsc{foc}\\
\glt ‘ ... “Oh, how sad! In the face of the sadness of the fact that [we] will never again be able to see his honourable face with his eyes open, observing \textbf{what} \textbf{prohibi-} \textbf{tions} was it that [you] didn't call [me]?” ...’  (Sanuki no suke nikki, Shinpen Zenshū, vol.\,26, p.\,420–421)
\end{exe}


\begin{exe}
\ex%46
\label{07-fr-ex:46}
%\langinfo{Early Modern Japanese}{}{(Genji monogatari, ‘Hahakigi’, Shinpen Zenshū, vol. 20, p.\,60)}}\\
\gll moto no sina toki yo no oboe uti-awi yamu-goto naki atari no uti-uti no motenasi kewawi okure-tara-mu wa sarani mo iwa-zu \textbf{nani} \textbf{o} site iki-owi-ide-kyemu to iu kawi naku oboyu besi\\
original \textsc{gen} class time age \textsc{gen} lesson \textsc{pfx}{-meet} stop-fact lacking spot \textsc{gen} inside-inside \textsc{gen} demeanour bearing be.late-\textsc{stat}-will \textsc{top} newly \textsc{etop} say-\textsc{neg}  what \textsc{acc}  doing live-grow-come.out-must.have \textsc{comp} say point lacking be.thought.of ought\\
\glt ‘ ... “There is nothing more to be said about those who, while coming from a venerable home where the original class and the repute of the world at large are in accord, nonetheless are lacking in the demeanor and bearing appropriate thereto. Doing \textbf{what} must it have been that [they] were raised, (I wonder)? [They] should be thought of as not worth mention.” …’  (Genji monogatari, ‘Hahakigi’, Shinpen Zenshū, vol.\,20, p.\,60)
\end{exe}

\begin{exe}
\ex%47
\label{07-fr-ex:47}
%\langinfo{Early Modern Japanese}{}{(Kokin wakashū, Shinpen Zenshū, vol. 11, p.\,404)}\\
\gll \textbf{nani} \textbf{wo} site mi no itadura ni oi-nu ramu tosi no omopa-mu koto zo yasasiki\\
 what \textsc{acc}  doing body \textsc{gen} in.vain \textsc{cop} grow.old-\textsc{pfv} \textsc{ext} year \textsc{gen} think-will content \textsc{foc} embarrassing\\
\glt ‘Doing \textbf{what} must it be that [my] body has grown old in vain? How shameful [to me], what the years must be thinking!’  (Kokin wakashū, Shinpen Zenshū, vol.\,11, p.\,404)\\
\end{exe}

\begin{exe}
\ex%48
\label{07-fr-ex:48}
%\langinfo{Early Modern Japanese}{}{(Genji monogatari, ‘Yūgao’, Shinpen Zenshū, vol. 20, p.\,158)}\\
\gll tati-wi no kewawi tawe-gata-geni okonau ito aware ni asa no kiri ni koto nara-nu yo o \textbf{nani} \textbf{o} musaboru mi no inori ni ka to kiki-tamau\\
stand-sit \textsc{gen} bearing withstand-hard-appearing undertake very pitiful \textsc{cop} morning \textsc{gen} mist \textsc{dat} otherwise be-\textsc{neg} world \textsc{acc} what \textsc{acc}  gobble.up body \textsc{gen} prayer \textsc{cop} \textsc{foc} \textsc{comp} listen-\textsc{resp}\\
\glt ‘Standing up and sitting down in a manner that appeared unbearable, (the old man) carried out the rites in a way that was so truly pitiful, [he] listened (to the old man), thinking, “Given that this world is no different than morning mist, these are the prayers of an earthly body hoarding up \textbf{what}, (I wonder)?” …’ (Genji monogatari, ‘Yūgao’, Shinpen Zenshū, vol.\,20, p.\,158)
\end{exe}

\begin{exe}
\ex%49
\label{07-fr-ex:49}
%\langinfo{Early Modern Japanese}{}{(Taketori monogatari, Shinpen Zenshū, vol. 12, p.\,29)}\\
\gll kono miko ni mawosi-tamapi-si pourai no tama no eda wo pito-tu no tokoro ayamata-zu motite opasimas-eri \textbf{nani} \textbf{wo} motite to.kaku mawosu beki\\
this aristocrat \textsc{dat} say-\textsc{resp}-\textsc{spst} Hōrai \textsc{gen} jewel \textsc{gen} branch \textsc{acc} one-thing \textsc{gen} place differ-\textsc{neg} having come-\textsc{stat} what \textsc{acc} having that.this say ought\\
\glt ‘[He] has brought the branch with the jewels of Hōrai that [you] spoke to this lord about, with not a point of difference [in it]. Having \textbf{what} (as grounds) am [I] supposed to tell [him] this and that (as excuses)?’ (Taketori monogatari, Shinpen Zenshū, vol.\,12, p.\,29)\\
\end{exe}

\begin{exe}
\ex%50
\label{07-fr-ex:50}
%\langinfo{Early Modern Japanese}{}{(Makura no sōshi, 131, Shinpen Zenshū, vol. 18, p.\,248)}\\
\gll saru koto ni wa \textbf{nani} \textbf{no} \textbf{irawe} \textbf{o} \textbf{ka} se-mu nakanaka nara-mu\\
such thing \textsc{dat} \textsc{top}  what \textsc{gen} reply \textsc{acc} \textsc{foc}  do-will awkward be-will\\
\glt ‘With respect to such a thing, \textbf{what reply} am [I] to make? [It] will be awkward.’ (Makura no sōshi, 131, Shinpen Zenshū, vol.\,18, p.\,248)
\end{exe}

\protectedex{
\begin{exe}
\ex%51
\label{07-fr-ex:51}
%\langinfo{Early Modern Japanese}{}{(Kokin wakashū, Shinpen Zenshū, vol. 11, p.\,418)}\\
\gll kakerite mo \textbf{nani} \textbf{wo} \textbf{ka} tama no kite mo mi-mu kara pa ponopo to nari-ni-si monowo\\
flying \textsc{etop}  what \textsc{acc} \textsc{foc}  soul \textsc{gen} coming \textsc{etop} see-will shell \textsc{top} ember as become-\textsc{pfv}-\textsc{spst} given.that\\
\glt ‘Even flying, \textbf{what} would [my] soul, coming here, see? Given that [her] remains are already turned to embers.’ (Kokin wakashū, Shinpen Zenshū, vol.\,11, p.\,418)
\end{exe}
}

\begin{exe}
\ex%52
\label{07-fr-ex:52}
%\langinfo{Early Modern Japanese}{}{(Heichū monogatari, Shinpen Zenshū, vol. 12, p.\,459)}\\
\gll ausaka no seki pa yoru koso mori-masare kurureba \textbf{nani} \textbf{wo} ware tanomu ramu\\
\=Osaka \textsc{gen} checkpoint \textsc{top} night \textsc{foc} guard-excel grow.dark.when  what \textsc{acc} I rely \textsc{ext}\\
\glt ‘It is at night that [they] guard the Osaka checkpoint more strongly. When the day ends, \textbf{what} shall I rely on?’  (Heichū monogatari, Shinpen Zenshū, vol.\,12, p.\,459)
\end{exe}

\begin{exe}
\ex%53
\label{07-fr-ex:53}
%\langinfo{Early Modern Japanese}{}{(Makura no sōshi, 327, Shinpen Zenshū, vol. 18, p.\,467)}\\
\gll miya.no.omawe ni uti.no.otodo no maturi-tamaw-eri-keru o kore ni \textbf{nani} \textbf{o} kaka-masi uwe.no.omawe ni wa siki to iu pumi o namu kaka-se-tamaw-eru\\
empress \textsc{dat} Minister.of.the.Centre \textsc{gen} give-\textsc{resp}-\textsc{stat}-\textsc{mpst} \textsc{acc} this \textsc{dat}  what \textsc{acc} write-\textsc{sbjv} emperor \textsc{dat} \textsc{top} chronicle \textsc{comp} say text \textsc{acc} \textsc{foc} write-\textsc{caus}-\textsc{resp}-\textsc{stat}\\
\glt ‘On the occasion of the Minister of the Centre giving [them] to the Empress, (she said), “\textbf{What} shall [I] write on these? On the Emperor's part, [he] is writing texts called ‘Chronicles’.” …’  (Makura no sōshi, 327, Shinpen Zenshū, vol.\,18, p.\,467)
\end{exe}

\begin{exe}
\ex%54
\label{07-fr-ex:54}
%\langinfo{Early Modern Japanese}{}{(Taketori monogatari, Shinpen Zenshū, vol. 12, p.\,23)}\\
\gll ka bakari kokoro.zasi oroka nara-nu pito.bito ni koso a mere kaguya.pime no ipaku \textbf{nani} \textbf{bakari} \textbf{no} \textbf{pukaki} \textbf{wo} \textbf{ka} mi-mu to ipa-mu isasaka no koto nari\\
this.way \textsc{res} resolve negligent be-\textsc{neg} person.person \textsc{cop} \textsc{foc} exist \textsc{ext} Shining.Princess \textsc{gen} saying  what   \textsc{res} \textsc{gen} depth \textsc{acc} \textsc{foc}  see-will \textsc{comp} say-will trifling \textsc{gen} thing be\\
\glt ‘... “It seems that [they] are people not lacking feeling to this degree.” The Shining Princess's reply: “[I] shall tell [you]: \textbf{What degree of depth} do [I] want to see? [It] is a mere trifling.”...’  (Taketori monogatari, Shinpen Zenshū, vol.\,12, p.\,23)
\end{exe}

\begin{exe}
\ex%55
\label{07-fr-ex:55}
%\langinfo{Early Modern Japanese}{}{(Sarashina nikki, Shinpen Zenshū, vol. 26, p.\,298)}\\
\gll aware-gari medurasi-garite kaweru ni \textbf{nani} \textbf{o} \textbf{ka} tatematura-mu mamemamesiki mono wa masa nakari namu\\
impression-exhibit rareness-exhibiting return \textsc{dat}  what \textsc{acc}  \textsc{foc} give-will practical thing \textsc{top} appropriateness lacking \textsc{sfp}\\
\glt ‘... making many signs of delight and interest (in me), when it was time [for me] to go home (she said), “\textbf{What} shall [I] give to you? Something practical just won't do.” ...’ (Sarashina nikki, Shinpen Zenshū, vol.\,26, p.\,298)
\end{exe}

\begin{exe}
\ex%56
\label{07-fr-ex:56}
%\langinfo{Early Modern Japanese}{}{(Genji monogatari, ‘Hatsune’, Shinpen Zenshū, vol. 22, p.\,144)}\\
\gll ware wa to omowi.agareru tiuzyau.no.kimi zo kanete zo miyuru nado koso kagami no kage ni mo katarawi-paberi-ture watakusi no inori wa \textbf{nani} \textbf{bakari} \textbf{no} \textbf{koto} \textbf{o} \textbf{ka} nado kikoyu\\
I \textsc{top} \textsc{comp} presuming Chūjō \textsc{foc} from.before \textsc{foc} be.visible and.the.like \textsc{foc} mirror.cake \textsc{gen} image \textsc{dat} \textsc{etop} talk-\textsc{hum}-\textsc{pfv} private \textsc{cop} prayer \textsc{top} what \textsc{res} \textsc{gen} word \textsc{acc} \textsc{foc}  and.the.like say\\
\glt ‘(The one to speak was) Chūjō, who presumed (to herself that if anyone has something to wish for, then) surely myself! “[I] was saying to [your] image in the mirror-cake, ‘(your thousand-year image) appeared from earlier,’ and so on. As for prayers for myself, \textbf{how much of a boon} (could I possibly ask)?” [she] continued in this vein.’  (Genji monogatari, ‘Hatsune’, Shinpen Zenshū, vol.\,22, p.\,144)
\end{exe}

\begin{exe}
\ex%57
\label{07-fr-ex:57}
%\langinfo{Early Modern Japanese}{}{(Genji monogatari, ‘Yadorigi’, Shinpen Zenshū, vol. 24, p.\,378)}\\
\gll yoki kakemono wa ari-nu bekeredo karugarusiku wa e-watasu maziki o \textbf{nani} \textbf{o} \textbf{ka} \textbf{pa} nado notamawa-suru mi-kesiki ikaga miyu ranee\\
good wager \textsc{top} exist-\textsc{pfv} ought.however lightly \textsc{top} can-hand.over impossible given.that what \textsc{acc} \textsc{foc} \textsc{top}  and.the.like say-\textsc{resp} \textsc{pfx}{}-visage how be.visible \textsc{ext}\\
\glt ‘...However must the sight [of him] saying such things as “Though there ought to be a good wager, [I] can't be handing anything over too lightly, so \textbf{what} (shall I wager)?” have appeared (to others)?’ (Genji monogatari, ‘Yadorigi’, Shinpen Zenshū, vol.\,24, p.\,378)
\end{exe}

Our evidence for the non-specificity of these items is perforce negative in nature: there is no positive way to rule out the possibility of a D-linking relationship for any of the \textit{wh-}NP objects in the examples above, and the strength of the grounds for our judgments of reference status varies for some of the examples we present here.\footnote{For example, there are conceivably exclusion relationships available to the object \textit{wh-}NPs in \REF{07-fr-ex:51}--\REF{07-fr-ex:52}.} However, most of our judgments carry a high degree of confidence. Given that non-specific expressions of this form are unattested in OJ  \il{Old Japanese} and violate the condition on DOM \REF{07-fr-ex:7}, the evidence shows that the OJ system of DOM is no longer operative in EMJ. 

\section{Discussion and conclusion}
\label{07-sec:4}

Like all other attested stages of \ili{Japanese}, both OJ  \il{Old Japanese} and EMJ have variable object marking. However, the results reported in this paper show clearly that EMJ does not share the OJ  \il{Old Japanese} system of DOM in which a correlation between accusative \isi{case marking} of objects and specificity is observed. As described through \sectref{07-sec:2}, we examined NPs in OJ  \il{Old Japanese} which under normal (unmarked) conditions were predicted to be non-specific in reference, namely object NP hosts of FQs and\textit{ wh-}object NPs with Q-focus. The distribution of object NP hosts of FQs in OJ  \il{Old Japanese} (\tabref{07-fr-tab:1}) gives a good reflection of the more general situation with regard to specificity and \textit{wo-}marking: 

\begin{table}
	\caption{Object NP hosts of FQs in OJ.}
	\begin{tabularx}{.75\textwidth}{ Q r  r  }
\lsptoprule
 & \textit{wo-}marked & zero-marked\\\midrule
 specific & 10 & 1\\
 non-specific & 0 & 4\\
\lspbottomrule
	\end{tabularx}
	\label{07-fr-tab:1}
\end{table}

In general, OJ  \il{Old Japanese} \textit{wo-}marked objects are specific (\eg \REF{07-fr-ex:1}), unspecific objects are bare (\eg \REF{07-fr-ex:2}), and some specific objects are bare (\eg \REF{07-fr-ex:5}), but there are no \textit{wo-}marked objects which are non-specific. This distribution is summarized in \tabref{07-fr-tab:4} further below. 

In EMJ, by contrast, the distribution of object NP hosts of FQs (see \sectref{07-fr-sec:3-1}) is as summarized in \tabref{07-fr-tab:2}.\footnote{Note that \tabref{07-fr-tab:2} does not break down the bare objects into specific and non-specific. As the point of interest for the comparison with OJ  \il{Old Japanese} was the reference of \textit{wo-}marked objects, we did not classify and quantify the reference of the bare objects. But as we already demonstrated by examples \REF{07-fr-ex:33}  and \REF{07-fr-ex:34}, the category of bare objects in EMJ contains both specific and non-specific NPs.} This reflects the general situation in EMJ, where both specific and nonspecific objects may be \textit{wo-}marked or bare, as shown in \sectref{07-sec:3}, where we demonstrated that EMJ has ample attestation of non-specific \textit{wo-}marked object NPs.

\begin{table}
	\caption{Object NP hosts of FQs in EMJ.}
	\begin{tabularx}{.75\textwidth}{ Q r r }
\lsptoprule
 & \textit{wo-}marked & zero-marked\\\midrule
 specific & 5 & \multirow{2}{*}{72}\\
 non-specific & 3 &\\
\lspbottomrule
	\end{tabularx}
	\label{07-fr-tab:2}
\end{table}

In general, the values for specificity and those for \textit{wo-} or zero-marking on objects are seen to cross-classify in EMJ. This distribution is summarized in \tabref{07-fr-tab:5} below. That pattern is not found in OJ  \il{Old Japanese} and is in direct contrast to the system seen in OJ, which disallows \textit{wo-}marked non-specific objects.

Thus, this paper identifies a major grammatical difference between OJ  \il{Old Japanese} and EMJ shown by the absence of non-specific \textit{wo-}marked objects in OJ, but their presence in EMJ. We observe a change from the OJ system with morphological expression (accusative marking on direct objects) of specificity in some contexts, to the EMJ system with no morphological expression (through \isi{case marking}) of specificity, that is, to a system where specificity is determined exclusively by context or NP modification or by the semantics of the head noun (\eg proper noun, relational noun, etc.). This is an important descriptive finding.

This does not mean, of course, that EMJ does not have some form of rule governed DOM, but it does show that the OJ  \il{Old Japanese} system of DOM, which takes part in expressing specificity, is not found in EMJ. For EMJ, the variability in \isi{case marking} must be investigated throughout the large amount of available data in order to identify a system which governs the observable variable \isi{case marking} of objects.

Now, in this paper we have not addressed the – important – issue of specific object NPs in OJ  \il{Old Japanese} which are not \textit{wo-}marked \REF{07-fr-ex:6c}, and which therefore show that there is no simple one-to-one correlation between specificity and \textit{wo-}marking on objects in OJ. In \citet{Frellesvigetal2015Differential} we discuss this briefly and outline some of the hypotheses which have been or may be proposed for absence of accusative \isi{case marking} on some specific objects, including conditions which may be formulated in terms of clause types (\eg main (disfavoring \textit{wo-}marking), embedded, relative, nominalized (favoring \textit{wo-}marking)), or other factors which may play a role, such as phonological form, or lexical idiosyncrasy (of both verbs and nouns). While a number of tendencies and individual factors may be identified, it remains clear that no strong condition or set of conditions for the absence of accusative \isi{case marking} on some specific objects in OJ  \il{Old Japanese} has been established yet.

Much work remains to be done on this for OJ, empirically involving careful scrutiny of the more than 2,000 bare objects in the OJ corpus. An important part of the interpretation of the data will be to consider whether the distribution observed in OJ, summarized in \tabref{07-fr-tab:4}, represents a stable system with (combinations of) conditions for absence of accusative \isi{case marking} on specific objects, which so far has proven too complex to be described; or whether in fact the distributional facts of OJ in \tabref{07-fr-tab:4} reflect a system in transition, from a stable, simple pre-OJ DOM system with straightforward rules for expression of the specificity of direct objects, such as that hypothesized in \tabref{07-fr-tab:3}, to the system of variable object marking we see in EMJ, summarized in \tabref{07-fr-tab:5}, which takes no part in the expression of specificity.

\begin{table}
	\caption{Possible system of case marking and specificity of objects in pre-OJ.}
	\begin{tabular}{ccc}
\lsptoprule
 & \textit{wo-}marked & zero-marked\\\midrule
 specific & $+$ & $-$\\
 non-specific & $-$ & $+$\\
\lspbottomrule
	\end{tabular}
	\label{07-fr-tab:3}
\end{table}

\begin{table}
	\caption{Accusative case marking and specificity of objects in OJ.}
	\begin{tabular}{ ccc }
\lsptoprule
 & \textit{wo-}marked & zero-marked\\\midrule
 specific & $+$ & $+$\\
 non-specific & $-$ & $+$\\
\lspbottomrule
	\end{tabular}
	\label{07-fr-tab:4}
\end{table}

\begin{table}
	\caption{Accusative case marking and specificity of objects in EMJ.}
	\begin{tabular}{ ccc }
\lsptoprule
 & \textit{wo-}marked & zero-marked\\\midrule
 specific & $+$ & $+$\\
 non-specific & $+$ & $+$\\
\lspbottomrule
	\end{tabular}
	\label{07-fr-tab:5}
\end{table}

This would mean that OJ represents a stage in the actualization of the change from a system like that in \tabref{07-fr-tab:3} (pre-OJ) to that in \tabref{07-fr-tab:5} (EMJ) and that in itself would provide a ready explanation for the fact that we observe variability in \isi{case marking} of specific objects in OJ. Much further research will be needed to determine whether that is the case, and if so, what governed the progression of the actualization of this change. A clearer understanding of the factors bearing on variable object marking in post-OJ stages of \ili{Japanese} would be of enormous help, but this too needs much further research. Determination and interpretation of \isi{markedness} values in a wide range of contexts will undoubtedly play an important role in investigating these questions (\cf \citealt{Andersen2001Actualization, Andersen2001Markedness}).

\section*{Abbreviations}
%\todo[inline]{pfx/prefix is a type of morpheme not a gloss, please replace accordingly}
\begin{tabularx}{.45\textwidth}{lQ}
\textsc{acc} & accusative\\
\textsc{com} & comitative\\
\textsc{comp} & complementizer\\
\textsc{cop} & copula\\
\textsc{dat} & dative\\
\textsc{etop} & emphatic topic\\
\textsc{ext} &  extension\\
\textsc{foc} & focus particle\\
\textsc{gen} & genitive\\
\textsc{ger} & gerund\\
\textsc{hum} & humble\\
 \textsc{imp} & imperative\\
\textsc{mpst} & modal past\\
\textsc{neg} & negative\\

\end{tabularx}
\begin{tabularx}{.45\textwidth}{lQ}
\textsc{opt} & optative\\
\textsc{pass} & passive\\
\textsc{pfv} & perfective\\
\textsc{pfx} & prefix\\
\textsc{pl} & plural\\
\textsc{res} & restrictive  particle\\
\textsc{resp} & respect\\
\textsc{sbjv} & subjunctive\\
\textsc{sfp} & sentence final particle\\
\textsc{spst} & simple  past\\
\textsc{stat} & stative\\
\textsc{subj} & subject\\
\textsc{top} & topic\\
\\
\end{tabularx}
\bigskip

\noindent
The following abbreviations indicate sources:\\
 KK \textit{Kojiki Kayō}; MYS \textit{Man'yōshū}; NSK \textit{Nihon Shoki}.

{\sloppy
\printbibliography[heading=subbibliography,notkeyword=this] }
\end{document}
