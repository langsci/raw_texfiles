\documentclass[output=paper]{LSP/langsci}
\ChapterDOI{10.5281/zenodo.1228263} 
\author{Klaus von Heusinger \affiliation{Universität zu Köln}}
\title{The diachronic development of Differential Object Marking in Spanish ditransitive constructions}
\shorttitlerunninghead{The diachronic development of DOM in Spanish ditransitive constructions}
\abstract{Differential Object Marking (DOM) in Spanish synchronically depends on the referential features of the direct object, such as animacy and referentiality, and on the semantics of the verb. Recent corpus studies suggest that the diachronic development proceeds along the same features, which are ranked in scales, namely the Animacy Scale, the Referentiality Scale and the Affectedness Scale. The present paper investigates this development in ditransitive constructions from the 17th to the 20th century. Ditransitive constructions in Spanish are of particular interest since the literature assumes that the differential object marker \textit{a} is often blocked by the co-occurrence of the case marker \textit{a} for the indirect object. The paper focuses on the conditions that enhance or weaken this blocking effect. It investigates three types of constructions with a ditransitive verb: (i) constructions with indirect objects realized as \textit{a}-marked full noun phrases, (ii) constructions with indirect objects as clitic pronouns, and (iii) constructions with non-overt indirect objects. The results clearly show that DOM is more frequent with (iii) and less frequent with (i). Thus the results support the observation that the co-occurrence of an \textit{a}-marked indirect object (partly) blocks \textit{a}-marking of the direct object to a certain extent. Furthermore, the results show for the first time that indirect objects realized as clitic pronouns without the marker \textit{a} have a weaker blocking effect, but still a stronger one than constructions without overt indirect objects. In summary, the paper presents new and original evidence of the competition between arguments in a diachronic perspective.}
\maketitle

\begin{document}

\section{Introduction} %1
\label{11-he-sec:1}

Differential Object Marking (DOM) in \ili{Spanish} is realized by the marker \textit{a}, which is derived from the preposition \textit{a} ‘to’ and which is also used to mark the \isi{indirect object}. DOM in \ili{Spanish} depends on \isi{referentiality}, \isi{animacy} and \isi{affectedness} (see \citealt{Pensado1995Complemento,Brugeetal1996Accusative,Leonetti2004Specificity,vonHeusingeretal2007Differential}). The \textit{a}-marking of the \isi{direct object} can easily co-occur with the prepositional \textit{a}, but in \isi{ditransitive} constructions with \textit{a}-marked indirect objects, the \textit{a}-marking of the \isi{direct object} can or must be dropped. 
In this paper I focus on the development of DOM in \ili{Spanish} \isi{ditransitive} constructions. While the development of DOM in transitive constructions is well-investigated (see \citealt{Melis1995Objetodirecto,Laca2002Gramaticalizacion,Laca2006Objeto,vonHeusinger2008Verbal}), there are very few studies that investigate competition of the marker \textit{a} between the \isi{direct object} and the \isi{indirect object} (but see \citealt{Company1998Interplay,Company2002Avance,Ortiz2005Objetos,Ortiz2011Construcciones,Mondonedo2007Syntax}). I will provide a qualitative corpus search, complementing the investigation of Ortiz Ciscomani  and providing new material to discuss the relation between the development of \textit{a}-marking in transitive sentences with the one in \isi{ditransitive} sentences. I take the result to support the view that DOM in \isi{ditransitive} constructions has developed similarly to DOM in transitive constructions, but that both, an indirect pronoun and an indirect full noun phrase, reduce the number of DOM for direct objects. 

In contemporary \ili{Spanish}, a human \isi{definite} \isi{direct object} in a transitive construction must be marked by the differential \isi{object marker} \textit{a} as illustrated in \REF{11-he-ex:1}. The \textit{a}-marked \isi{definite} \isi{direct object} can co-occur with a prepositional object marked by \textit{a}, as in \REF{11-he-ex:2}, but is generally blocked or disfavored by the occurrence of an \textit{a}-marked \isi{indirect object} realized by an \textit{a}-marked full noun phrase in a \isi{ditransitive} construction, as in \REF{11-he-ex:3}. The co-occurrence of an \textit{a}-marked \isi{direct object} and an \textit{a}-marked \isi{indirect object} is subject to controversial grammaticality judgments, \cf \REF{11-he-ex:4} – judgments according to \citet[20]{CompanyCompany2001Multiple}. 

\ea \label{11-he-ex:1}
\gll Busco al /{ } *el médico.\\
seek.1\textsc{sg} \textsc{dom}.the /{ } the doctor\\
\glt ‘I am seeking the doctor.’
\z

\ea
\label{11-he-ex:2}
\gll Envié a mi hermana a Caracas.\\
sent.1\textsc{sg} \textsc{dom} my sister to Caracas\\
\glt ‘I sent my sister to Caracas.’
\z

\ea
\label{11-he-ex:3}
\gll El maestro presentó Ø su mujer a los alumnos.\\
the teacher introduced.3\textsc{sg} {} his wife to the students\\
\glt ‘The teacher introduced his wife to the students.’
\z

\ea%4
\label{11-he-ex:4}
\gll ??/*El maestro presentó a su mujer a los alumnos.\\
??/*the teacher introduced.3\textsc{sg} \textsc{dom} his wife to the students\\
\glt ‘The teacher introduced his wife to the students.’
\z


There is a controversy about the effect of \isi{clitic doubling} of the \isi{indirect object}. According to certain grammatical conditions, indirect objects can or must be doubled by a clitic (pronoun) form that agrees in case and number with the \isi{indirect object} \citep{Campos1999Transitividad,Gabrieletal2010Information}. There are at least three positions on the effect of \isi{clitic doubling} in \isi{ditransitive} constructions: it facilitates \textit{a}-marking of the \isi{direct object}, it favors blocking of \textit{a}-marking, or it makes \textit{a}-marking ungrammatical. (i) \citet{Company1998Interplay,Company2002Avance} claims that the clitic \textit{le} in \REF{11-he-ex:5} facilitates the \textit{a}-marking of the \isi{direct object}. (ii) \citet[216]{Mondonedo2007Syntax} claims that “[...] clitic-doubled IOs seem to allow the dropping [of the \textit{a} marker] more easily than their non-doubled counterparts, at least for some speakers […].” (iii) \citet[31]{Fabregas2013Differential} reports that \textit{a}-marking of the \isi{direct object} is more grammatical without clitic than with clitic, as in \REF{11-he-ex:6}. \citet[224]{Ormazaletal2013Differential} also assume that \isi{clitic doubling} bans \textit{a}-marking of the \isi{direct object}.

\ea
%5
\label{11-he-ex:5}
\gll El maestro le presentó a su mujer a Juan.\\
the teacher \textsc{dat}.3\textsc{sg} introduced.\textsc{3sg} \textsc{dom} his wife to Juan\\
\glt ‘The teacher introduced his wife to Juan.’ (judgement according to \citealt[20]{CompanyCompany2001Multiple})
\z

\ea
\label{11-he-ex:6}
\gll *Le enviaron a todos los heridos a la doctora.\\
\textsc{dat}.\textsc{3sg} sent.\textsc{3pl} \textsc{dom} all the injured to the doctor\\
\glt ‘They sent all the injured to the doctor.’ (judgement according to \citealt[31]{Fabregas2013Differential})
\z

The \isi{diachronic development} of DOM in \ili{Spanish} is fairly well documented and investigated primarily in transitive construction (see \citealt{Melis1995Objetodirecto,Melisetal2009Interplay,Laca2002Gramaticalizacion,Laca2006Objeto,vonHeusingeretal2007Differential,vonHeusinger2008Verbal}). Diachronic data of \isi{ditransitive} constructions with two full noun phrases are rare and therefore difficult to collect, but the examples below provide some interesting observations. Already in \isi{ditransitive} constructions in the 13th century, an alternation between \textit{a}-marked direct objects \REF{11-he-ex:7} and unmarked direct objects \REF{11-he-ex:8} can be seen.
% in \isi{ditransitive} constructions in the 13th century. 

\ea
%7
\label{11-he-ex:7}
\gll E dio Ercules a Manilop a la reyna Anthipa, su Hermana.\\
and gave.\textsc{3sg} Hercules to Manilop \textsc{dom} the queen Anthipa his sister\\
\glt ‘And Hercules gave his sister, Queen Anthipa, to Manilop.’ (GEII (General Estoria, segunda parte), 21, 13th century, quoted after \citealt[167]{Ortiz2011Construcciones})
\z

\ea
%8
\label{11-he-ex:8}
\gll El dio Ø sus fijas a aquellos dos infantes ante todos sus ricos omnes.\\
he gave.\textsc{3sg} {} his daughters to those two infants in.front.of all his rich men\\
\glt ‘He gave his daughters to those two princes in front of all his rich men.’ (GE (General Estoria), 344, 13th century, quoted after \citealt[168]{Ortiz2011Construcciones})
\z

\largerpage
One also finds this alternation in sentences with clitic-doubled indirect objects: The \isi{direct object} \textit{Leonor} is \textit{a}-marked in \REF{11-he-ex:9}, while the \isi{direct object} \textit{media mujer} (‘half woman’) is unmarked in \REF{11-he-ex:10} (examples from the 17th century):

\ea
%9
\label{11-he-ex:9}
\gll A Mendo, hijo de hermana menor, le quiero dar a Leonor.\\
to Mendo son of sister younger \textsc{dat}.\textsc{3sg} want.\textsc{1sg} give.\textsc{inf} \textsc{dom} Leonor\\
\glt ‘To Mendo, son of (my) younger sister, I want to give Leonor.’ 

(Moreto, Agustín.\,(1618–1669), El lindo Don Diego)
\z 
\ea
%10
\label{11-he-ex:10}

\gll Aun si les dieran Ø media mujer a cada uno, fuera menor el daño.\\
even if \textsc{dat.3pl} gave.\textsc{3pl} {} half woman to each one.\textsc{masc} would.be.\textsc{3sg} less the damage\\
\glt ‘Even if they gave half a woman to each one (of them), the damage would be less.’ (Castro, Guillén de.\,(1569–1631), El conde de Irlos.)
\z

We can summarize the observations regarding DOM in transitive and \isi{ditransitive} constructions. DOM in transitive constructions in \ili{Spanish} is well-investigated: Synchronically specific \isi{indefinite} human direct objects are obligatorily marked, non-specific ones are optionally marked, and non-human direct objects are nearly never marked \citep{Brugeetal1996Accusative,Leonetti2004Specificity, vonHeusingeretal2007Differential,GarciaGarcia2014Objektmarkierung}. DOM is blocked or less often used in \isi{ditransitive} constructions with the \isi{indirect object} realized by a full noun phrase with the \isi{dative case} marker \textit{a}. There is variation in diachronic data, but so far the relevant parameters for this variation, if any, cannot be identified. 

There are various theories of DOM with different emphasis on syntactic, semantic or functional properties of the \textit{a}-marker. For the sake of the argument (and broadly simplifying), I assume four positions, which do not necessarily exclude each other: (a) DOM as a case marker, (b) DOM in competition with \isi{indirect object} \isi{case marking}, (c) DOM indicates the syntactic status of a noun phrase as an argument, (d) DOM as a means to disambiguate between subject and object. (a) It is often assumed that DOM is the case marker of the \isi{direct object}, which is shown by the dependency on certain syntactic constructions, such as small clauses  \citep{Brugeetal1996Accusative,Mondonedo2007Syntax,Ormazaletal2013Differential}. Such a syntactic perspective predicts a certain stability of the phenomenon and a clear prediction following general case principles (only one case assignment in a clause). (b) \citet{Company1998Interplay,Company2002Avance} argues that direct objects are marked by DOM, while there are different means in addition to the \textit{a}-marking to mark an \isi{indirect object} (such as \isi{clitic doubling}) – see also \citet{Delbecque1998Frames,Delbecque2002Construction} for a construction grammar approach. In the history of \ili{Spanish}, there has been continuous competition between these two strategies. DOM is a strategy for marking direct objects, and becomes unavailable when it creates an ambiguity with indirect objects. If, however, there are no other means available, \textit{a}-marking is reserved for the \isi{indirect object} and cannot be simultaneously used for the \isi{direct object}. This picture provides an account of some of the diachronic data, but does not always seem to be confirmed by synchronic data (see \citealt{Melisetal2009Interplay} for discussion). (c) Synchronically, it is assumed that DOM signals that the \isi{direct object} is a proper argument that saturates the verbal frame, while unmarked direct objects are more like bare nouns that modify the verb \citep{Chungetal2004Restriction,Lopez2012Indefinite}. This view predicts a certain stability in similar semantic contexts. It is, however, not clear how this view can account for the diachronic data, in particular the observation that in earlier stages of \ili{Spanish}, DOM was only obligatory for pronouns and proper names, but not for \isi{definite} noun phrases. Still, \isi{definite} noun phrases are arguments in \citegen{Chungetal2004Restriction} account and should be \textit{a}-marked according to \citet{Lopez2012Indefinite}. (d) Functional theories assume that one of the main functions of DOM is to identify a \isi{direct object}, if it is too similar to the subject, \ie if it has too many properties of prototypical subjects. Besides this main function, DOM can additionally express other semantic or pragmatic features, such as \isi{topicality}, \isi{referentiality} or specificity \citep{Comrie1975Definite,Bossong1985Differentielle,Aissen2003Differential} or \isi{telicity} \citep{Torrego1999Gramatica} or \isi{affectedness} \citep{vonHeusingeretal2011Affectedness}. DOM is often overextended and conventionalized (grammaticalized), \ie used in contexts where a distinction between subject and object is already given by other means (\eg  verbal agreement; see \citealt{Aissen2003Differential} for discussion). The functional view seems to be flexible enough to model diachronic change, and it predicts a certain variability in the actual realization of DOM. In this paper I cannot answer the question which of the four positions is the most appropriate one. I rather provide additional observations that might support one or the other account.

The main focus of this paper is to compare the development of DOM in transitive constructions with the development in \isi{ditransitive} constructions. I have restricted the data to direct objects realized by human noun phrases, \ie \isi{definite} NPs and \isi{indefinite} NPs. For transitive constructions, I will use the material presented in the literature \citep{Melis1995Objetodirecto, Laca2006Objeto, vonHeusingeretal2007Differential,vonHeusingeretal2011Affectedness,vonHeusinger2008Verbal} and compare this with the data of \citet{Ortiz2005Objetos,Ortiz2011Construcciones}. I have also created my own corpus, including three realizations of \isi{ditransitive} constructions, which all have a \isi{direct object} realized as a human \isi{definite} or \isi{indefinite} noun phrase (but not all subjects are realized or realized as full noun phrases): In type (i), the \isi{indirect object} is not realized – either because the \isi{indirect object} is inferred from the context or because it is left unspecified. Type (ii) realizes the \isi{indirect object} as a clitic pronoun – generally before the finite verb. Type (iii) realizes the \isi{indirect object} as full noun phrase that is obligatorily marked by \textit{a} (see \tabref{11-he-tab:1a}).

\begin{table}
\caption{Types of constructions and argument realizations}\label{11-he-tab:1a}
\begin{tabularx}{\textwidth}{lQl}
\lsptoprule
& Example & IO\\
\midrule 
(i) & \emph{El maestro presentó (a) su hijo} & not realized\\
(ii) &\emph{El maestro le presentó (a) su hijo} & clitic pronoun\\
(iii) & \emph{El maestro presentó (a) su hijo al alumno} & full NP\\
& ‘The teacher introduces his son (to him, to the student).’ &\\
\lspbottomrule
\end{tabularx}
\end{table}

 
I put forward the following hypotheses, which will be tested using data extracted from diachronic corpora:

\begin{itemize}
\item H1: The type of the \isi{ditransitive} construction determines the \isi{blocking effect}:
\begin{enumerate}[label=\roman*]
 \item constructions with indirect objects realized as \textit{a}-marked full noun phrases (\isi{definite} NPs, \isi{indefinite} NPs) show a high \isi{blocking effect}
 \item constructions with indirect objects as clitic pronouns show a low \isi{blocking effect}, and
 \item constructions with non-overt indirect objects do not show any \isi{blocking effect} 
\end{enumerate}
\item H2: DOM in \isi{ditransitive} constructions has a comparable development to DOM in transitive constructions.
\item H3: Verb classes differ with respect to the way they influence DOM and DOM-blocking.
\end{itemize}

In \sectref{11-he-sec:2} I summarize the synchronic and diachronic conditions for DOM in \ili{Spanish}. \sectref{11-he-sec:3} presents the synchronic restrictions on DOM in \isi{ditransitive} constructions. \sectref{11-he-sec:4} summarizes earlier research on ditransitives in \ili{Spanish} (Company Company and Ortiz Ciscomani), introduces the corpus created for this paper, and discusses the results of the corpus search. \sectref{11-he-sec:5} provides the evaluation of the results with respect to the three hypotheses and a general discussion of DOM in \isi{ditransitive} construction.

\section{DOM in transitive constructions} %2. /
\label{11-he-sec:2}
\subsection{Synchrony of nominal and verbal parameters related to DOM} %2.1 /
\label{11-subsec:2-1}

I will limit the investigation to European \ili{Spanish} throughout this paper, but see \citet{Company2002Avance} for Mexican \ili{Spanish}. It is commonly assumed that there are at least four main factors for DOM in the languages of the world: (i) \isi{animacy} properties of the \isi{direct object}; and (ii) referential properties, such as indexicality (deixis), \isi{definiteness} and specificity, of the \isi{direct object}. The \isi{referentiality} status is clearly indicated by the morphological form of the noun phrase and ordered on the Referentiality Scale (see below \REF{11-he-ex:15}). (iii) Information structure might determine DOM, in particular topical direct objects tend rather to be marked than not. (iv) Finally, \isi{transitivity} properties of the verb also influence DOM (see \citealt{Comrie1975Definite,Bossong1985Differentielle,Aissen2003Differential,deSwart2007Cross-linguistic,Iemmolo2010Topicality,Iemmoloetal2014Introduction,Witzlacketal2017Differential}). DOM or \textit{a}-marking in \ili{Spanish} is determined by all four main parameters: 

\textit{(i)} Only human direct objects can be marked, while non-human (\isi{animate}) and \isi{inanimate} direct objects are obligatorily unmarked. However, there is small class of verbs, such as verbs of substitution, that allow DOM for \isi{inanimate} direct objects (see \citealt{GarciaGarcia2014Objektmarkierung}, \citeyear{Garcia2017Nominal} [this volume] for an extensive discussion), \cf \REF{11-he-ex:14}. In the remainder, I will exclude \isi{inanimate} direct objects as I am not aware of \isi{ditransitive} constructions that allow DOM for inanimates.

\ea
%12
\label{11-he-ex:12}

\gll Conozco *(a) este actor.\\
know.1.\textsc{sg} \textsc{dom} this actor\\
\glt ‘I know this actor.’
\z

\ea
%13
\label{11-he-ex:13}

\gll Conozco (*a) esta película.\\
know.1.\textsc{sg} {} this film\\
\glt ‘I know this film.’
\z

\ea
%14
\label{11-he-ex:14}

\gll En esta receta la leche puede sustituir *el/al huevo.\\
in this recipe the milk can substitute the/\textsc{dom}.the egg\\
\glt ‘In this recipe the milk can substitute the egg.’
\z

\textit{(ii)} Specific \isi{indefinite} human direct objects and all direct objects that are higher on the Referentiality Scale \REF{11-he-ex:15} must be \textit{a}-marked, \cf \REF{11-he-ex:16}. Even non-specific indefinites can optionally be \textit{a}-marked, \cf \REF{11-he-ex:17}, where the subjunctive \textit{sepa} (`might know') of the relative clause indicates that the head noun \textit{un ayudante} (‘an assistant’) is non-specific. Determinerless noun phrases (‘bare nouns’ in their ‘non-argumental’ function) as \textit{camarero} (‘waiter’) in \REF{11-he-ex:18} must not be \textit{a}-marked.

\ea
%15
\label{11-he-ex:15}
 \textit{Referentiality Scale:}\\ 

 personal pronoun > proper noun > \isi{definite} NP > specific \isi{indefinite} NP\\ 

 > non-specific \isi{indefinite} NP > non-argumental
\z

\ea
%16
\label{11-he-ex:16}

\gll Vi *(a) la/una mujer.\\
saw.\textsc{1sg} \textsc{dom} the/a woman\\
\glt ‘I saw the / a woman.’
\z

\ea
%17
\label{11-he-ex:17}

\gll Necesitan (a) un ayudante que sepa inglés.\\
need.\textsc{3pl} \textsc{dom} an assistant that know.\textsc{3sg} English\\
\glt ‘They need an assistant who knows English.’
\z

\ea
%18
\label{11-he-ex:18}

\gll Necesitan (*a) camarero.\\
they.need {} waiter\\
\glt ‘They need a waiter.’
\z

\textit{(iii)} Topicality is also often said to be a parameter of DOM in \ili{Spanish}. Like in many other DOM languages, leftwards-moved direct objects are obligatorily \textit{a}-marked, \cf \REF{11-he-ex:19}, see \citet[86]{Leonetti2004Specificity}. It is, however, much harder to argue that non-moved \textit{a}-marked noun phrases are topical. \citet{Iemmolo2010Topicality} argues that such noun phrases show certain properties of topics and links DOM to topichood, while  \citet{Dalrympleetal2011Objects} assume that DOM indicates a secondary topic, as a \isi{direct object} is rarely the primary sentence topic.\footnote{See also \citet{Chiriacescu2014Discourse} and \citet{Guntsetseg2016} for the function of DOM as a secondary topic in \ili{Romanian} and Mongolian, respectively.}

\ea
%19
\label{11-he-ex:19}

\gll *(A) muchos estudiantes, ya los conocía.\\
*(\textsc{dom}) many students, already them knew.1\textsc{sg}\\
\glt ‘Many students I already knew.’
\z

\textit{(iv)} Verbal categories are also decisive for DOM in \ili{Spanish}. \citet[567–570]{Bello1847Gramatica} and \citet[151–190]{Fernandez1951Gramatica} present rich material on the variation according to different verb types in \ili{Spanish}. \citet[87]{Pottier1968Emploi} proposes the scale in \REF{11-he-ex:20} for \textit{a}-marking in \ili{Spanish}, which is slightly modified by von  \citet[94]{vonHeusingeretal2007Differential} to the Scale of Affectedness and Expected Animacy, \cf \tabref{11-he-tab:ex21} (see also \citealt{vonHeusingeretal2011Affectedness} for a different \isi{affectedness} categorization, based on \citealt{Tsunoda1985Remarks}).

\ea
%20
\label{11-he-ex:20}
Verbal Scale (\citealt[87]{Pottier1968Emploi}: ‘‘un axe sémantique verbal’’)\\
\textit{ matar} ‘kill’ > \textit{ver} ‘see’ > \textit{considerar} ‘consider’ > \textit{tener} ‘have’
\z

\begin{table}
\caption{Scale of Affectedness and Expected Animacy (\citealt[94]{vonHeusingeretal2007Differential})}
\label{11-he-tab:ex21}
\begin{tabularx}{\textwidth}{lll} 
\lsptoprule
Class 1 [+ human] > & Class 2 [± human]  & > Class 3 [(±)/– \isi{animate}]\\
\midrule
\textit{matar} ‘kill’, \textit{ herir} ‘hurt’ & \textit{ver} ‘see’, \textit{ hallar} ‘find’ & \textit{tomar} ‘take’, \textit{ poner} ‘put’\\
\lspbottomrule
\end{tabularx}
\end{table}


The scale in \tabref{11-he-tab:ex21} predicts that verbs like \textit{matar} (‘to kill’), which clearly prefer a human \isi{direct object}, are much more likely to mark the \isi{direct object} than verbs that do not show such a preference, such as \textit{ver} (‘to see’). Verbs that prefer an \isi{inanimate} \isi{direct object} show synchronically the lowest rate of \textit{a}-marking of their human direct objects.

\subsection{Diachrony of NP-related properties}\label{11-subsec:2-2}

Like Modern \ili{Spanish}, Old \ili{Spanish} exhibits DOM. However, as shown in several diachronic studies (\citealt{Melis1995Objetodirecto}, \citealt{Laca2006Objeto}), DOM in Old \ili{Spanish} is less frequent than in Modern \ili{Spanish} and used under different conditions. Human \isi{definite} direct objects are optionally \textit{a}-marked, as the two examples in \REF{11-he-ex:22}--\REF{11-he-ex:23} illustrate. Non-human \isi{animate} \isi{indefinite} direct objects are generally not \textit{a}-marked, as in \REF{11-he-ex:24}.

\ea \langinfo{Old Spanish}{}{Cid, 2637}\\
\label{11-he-ex:22}
\gll Reçiba a \textbf{mios} \textbf{yernos} commo elle pudier mejor.\\ %(Cid, 2637)
receive.\textsc{imp}.\textsc{2sg} \textsc{dom} my sons.in.law as he could.\textsc{3sg} better\\
\glt ‘Have him welcome my sons-in-law as best he can.’ 
\z

\ea \langinfo{Old Spanish}{}{Cid, 2956}\\
\label{11-he-ex:23}
\gll Ca yo case \textbf{sus} \textbf{fijas} con yfantes de Carrion.\\ %(Cid, 2956) 
for I married.\textsc{1sg} his daughters with Infantes of Carrion\\
\glt ‘for I married his daughters to the Infantes of Carrion.’
\z

\ea \label{11-he-ex:24}
\langinfo{Old Spanish}{}{Cid, 480--481}\\
\gll Tanto traen las grandes ganançias, \textbf{muchos} \textbf{gañados} \textbf{de} \textbf{ovejas} \textbf{e} \textbf{de} \textbf{vacas}.\\
very brought.\textsc{3pl} the big wealths many herds of sheep and of cows\\
\glt ‘They brought such great wealth, many herds of sheep and cows.’  %(Cid, 480-481)
\z

\tabref{11-he-tab:1} summarizes the findings of \citet{Laca2006Objeto}, which is based on the manual collection of utterances in her corpus of documents from the 12th to the 19th century. Proper names are \textit{a}-marked from the time of Old \ili{Spanish}, while \isi{definite} and \isi{indefinite} NPs show a clear development. Non-human direct objects are rarely marked.

\begin{table}
\fittable{
\begin{tabular}{lrrrrrrr} 
\lsptoprule
& \multicolumn{7}{c}{century} \\
& 12th & 14th & 15th & 16th & 17th & 18th & 19th\\
\midrule
 proper name & 96\% (26) & 100\% (8) & 100\% (35) & 95\% (44) & 100\% (65) & 79\% (29) & 89\% (27)\\
\tablevspace 
 \isi{definite} NP & 36\% (36) & 55\% (66) & 58\% (65) & 70\% (122) & 86\% (136) & 85\% (53) & 96\% (76)\\
\tablevspace 
 \isi{indefinite} NP & 0\% (6) & 6\% (31) & 0\% (11) & 12\% (59) & 39\% (53) & 62\% (32) & 41\% (29)\\
\tablevspace 
 bare noun & 0\% (12) & 0\% (7) & 16\% (12) & 5\% (40) & 2\% (39) & 9\% (22) & 6\% (17)\\
\lspbottomrule
\end{tabular}
}
\caption{Diachronic development of a-marking in Spanish according to the Referentiality Scale (selection from Table~3 of \citealt{Laca2006Objeto}: 442). I replaced the original abbreviations in the following way: NPrHum: human proper name, HumDef–Pro: human definite NP, HumInd–Pro: human indefinite NP, Hum0: human bare noun}\label{11-he-tab:1}
\end{table}

\figref{11-he-fig:1} presents Laca’s data in a graphic that illustrates that the rate of \textit{a}-marking has increased over time and along the Referentiality Scale.

\begin{figure}
	\centering
	\includegraphics[width=80ex]{figures/11-he-fig1.pdf}
\caption{Diachronic development of \textit{a}-marking in Spanish according to the Referentiality Scale (based on \citealt[442]{Laca2006Objeto}, \tabref{11-he-tab:3}; from \citealt{vonHeusingeretal2011Affectedness}, Fig.\,3)} \label{11-he-fig:1}
\end{figure}


\subsection{Diachrony and affectedness}\label{11-subsec:2-3}

Von Heusinger \& Kaiser (\citeyear{vonHeusingeretal2007Differential}) apply the Scale of Affectedness, \cf \tabref{11-he-tab:ex21}, to a small corpus from the Bible to show the \isi{diachronic development} along this scale. The corpus consists of the two books of Samuel and the two Books of Kings in three Bible translations, abbreviated as A–C: translation A is from the 14th century and was only available as a printed version. All other translations were electronically available at Biblegate: B, \textit{Reina Valera Antigua} from 16th/17th century, its contemporary version C from 1995 (\textit{Reina Valera}). \REF{11-he-ex:25} nicely illustrates the development and its interaction with topicalization. The verb \textit{tomar} (‘take’) is from Class 3, \ie from those verbs that strongly prefer an \isi{inanimate} \isi{direct object}. In the translation from the 14th century, the \isi{direct object} \textit{a} \textit{vuestra} \textit{fijas} (‘your daughter’) is \textit{a}-marked, since it is left-moved, while the \isi{direct object} in the translation from the 16th century is not moved and unmarked. However, the translation from the 20th century provides DOM for the \isi{direct object} in the base position, as expected for \isi{definite} human noun phrases.

\protectedex{
\ea 1 Samuel 8, 13\\\label{11-he-ex:25}
\begin{tabularx}{.95\textwidth}{lQ}
A (14th) & E \textbf{a} \textbf{vuestras} \textbf{fijas} \textbf{\textit{tomará}} por espeçieras e cosineras e panaderas.\\
B (16th) &  \textbf{\textit{Tomará}} también Ø \textbf{vuestras} \textbf{hijas} para que sean perfumadoras, cocineras, y amasadoras.\\
C (20th) &  \textbf{\textit{Tomará}} también \textbf{a} \textbf{vuestras} \textbf{hijas} para perfumistas, cocineras y  amasadoras.\\
English &   `He will \textbf{\textit{take}} (A: \textsc{dom}, B: Ø, C: \textsc{dom}) \textbf{your} \textbf{daughters} to be perfumers, cooks and bakers.'\\

\end{tabularx}
\z
}


In a detailed analysis, \citet{vonHeusingeretal2007Differential} searched the small corpus for all instances of \isi{definite} and \isi{indefinite} noun phrases that filled the \isi{direct object} of the following six verbs categorized in three classes: 
Class 1: \textit{matar} ‘kill’, \textit{herir} ‘hurt’, Class 2: \textit{ver} ‘see’, \textit{hallar} ‘find’, and Class 3: \textit{tomar} ‘take’, \textit{poner} ‘put’, \cf \tabref{11-he-tab:ex21}. These classes differ not so much in \isi{affectedness} of the \isi{direct object}, but rather in the expectedness of \isi{animacy} of the \isi{direct object}. Class 1 has a very high expectation that the object is human, while class 2 is rather neutral, and class 3 has an expectation of an \isi{inanimate} \isi{direct object}. \tabref{11-he-tab:2} provides the figures for human \isi{definite} direct objects and \tabref{11-he-tab:3} for human \isi{indefinite} direct objects.\footnote{An alternative view is that not the \isi{animacy}, but the \isi{agentivity} of the \isi{direct object} is the relevant parameter for DOM (see \citealt{GarciaGarcia2014Objektmarkierung} for this view).} 

\begin{table}
\caption{Percentage of \textit{a}-marking of human definite direct objects. (Bible translations of 1+2 Samuel and 1+2 Kings, from \citealt[606]{vonHeusingeretal2011Affectedness})}
\label{11-he-tab:2}


\begin{tabularx}{\textwidth}{Qlll}
\lsptoprule

class & 14th cent. & 16th/17th cent. & 20th cent.\\
\midrule 
1. \textit{matar} ‘kill’, \textit{herir} ‘hurt’ & 60\% (24/40) & 66\% (37/56) & 92\% (36/39)\\
2. \textit{ver} ‘see’, \textit{hallar} ‘find’ & 38\% (9/24) & 48\% (13/27) & 81\% (26/32)\\
3. \textit{tomar} ‘take’, \textit{poner} ‘put’ & 30\% (7/23) & 30\% (7/23) & 67\% (20/30)\\
\lspbottomrule
\end{tabularx}
\end{table}

\begin{table}
\caption{Percentage of \textit{a}-marking of human indefinite direct objects. (Bible translations of 1+2 Samuel and 1+2 Kings, from t\citealt[607]{vonHeusingeretal2011Affectedness})}

\label{11-he-tab:3}
\begin{tabularx}{\textwidth}{Qlll}
\lsptoprule
class & 14th cent. & 16th/17th cent. & 20th cent.\\
\midrule
1. \textit{matar} ‘kill’, \textit{herir} ‘hurt’ & 7\% (1/14) & 7\% (1/14) & 91\% (10/11)\\
2. \textit{ver} ‘see’, \textit{hallar} ‘find’ & 0\% (0/11) & 15\% (2/13) & 45\% (5/11)\\
3. \textit{tomar} ‘take’,\textit{ poner} ‘put’ & 0\% (0/15) & 0\% (0/28) & 17\% (2/12)\\
\lspbottomrule
\end{tabularx}
\end{table}

\figref{11-he-fig:2} summarizes the two tables and clearly shows that \isi{referentiality} is the main parameter for DOM: Definite direct objects are more often \textit{a}-marked than \isi{indefinite} direct objects. Furthermore, the verb class is a crucial parameter for DOM. Both parameters add up (there is no interaction).

\begin{figure}%[!h]
	\centering
	\includegraphics[width=80ex]{figures/11-he-fig2.pdf} %\vspace{-2ex}
\caption{Percentage of \textit{a}-marking depending on verb class, definiteness and time; Class 1: \textit{matar} ‘kill’,\textit{ herir} ‘hurt’, Class 2: \textit{ver} ‘see’, \textit{hallar} ‘find’, and Class 3: \textit{tomar} ‘take’, \textit{poner} ‘put’ (Three Bible translations of 1+2 Samuel and 1+2 Kings, from \citealt[607]{vonHeusingeretal2011Affectedness})} \label{11-he-fig:2}
\end{figure}

Von Heusinger (\citeyear{vonHeusinger2008Verbal}) provides a corpus search to more precise historical periods, using Mark Davies’ \textit{Corpus del Español}. The corpus comprises 100 million words of \ili{Spanish} texts from the 12th to the 19th century. 
The corpus interface allows one to search for lemmas, rather than for word forms (as in simple text files of the Bible texts). 
However, such searches are still very time-consuming since one has to select the \isi{definite} or \isi{indefinite} human direct objects by hand. In the case of \textit{tomar} only about 1–7\% of all hits were human \isi{definite} or \isi{indefinite} full NPs. 
The others were either \isi{inanimate}, or human and of a different type on the Referentiality Scale, such as clitics, personal pronouns, proper names or different types of quantifiers. 
The study originally differentiates between eight time periods from the 12th to the 19th century, which have been reduced to four time periods. 
Furthermore, the search was restricted to two verb classes, and one verb for each class: \textit{matar} ‘to kill’ for class 1 and \textit{tomar} ‘to take’ for class 3 (see \citealt{vonHeusinger2008Verbal} for the details, and \citealt{vonHeusingeretal2011Affectedness} for a compact presentation). \tabref{11-he-tab:4} shows that in the 12th and 13th century, 50\% of human \isi{definite} direct objects of \textit{matar} are marked with \textit{a}. 
This number continually increases and reaches about 90 percent by the 18th and 19th century. 
The marking of the \isi{definite} \isi{direct object} of \textit{tomar} is less preferred. 
Only about 40\% in the 12th and 13th century are marked, a number that continuously increases to about 80\% in the 18th and 19th century. \tabref{11-he-tab:5} provides the numbers for human \isi{indefinite} direct objects. As expected, \textit{a}-marking is less preferred, but there is a clear increase over time and some difference between the two verb classes.

\begin{table}
\caption{Percentage of \textit{a}-marking of human definite direct objects. (Corpus del Español, from \citealt[608]{vonHeusingeretal2011Affectedness})}
\label{11-he-tab:4}
\fittable{
\begin{tabular}{l rrrr}
\lsptoprule
class & 12th + 13th cent. & 14th + 15th cent. & 16th + 17th cent. & 18th + 19th cent\\
\midrule
1. \textit{matar} ‘kill‘ & 50\% (25/50) & 63\% (27/43) & 78\% (32/41) & 91\% (39/43)\\
3. \textit{tomar} ‘take’ & 40\% (38/95) & 55\% (30/55) & 70\% (7/10) & 83\% (20/24)\\
\lspbottomrule
\end{tabular}
}
\end{table}

\begin{table}
\caption{Percentage of \textit{a}-marking of human indefinite direct objects (Corpus del Español; from \citealt{vonHeusingeretal2011Affectedness}: 608) }
\label{11-he-tab:5}
\fittable{
\begin{tabular}{l rrrr}
\lsptoprule
class & 12th + 13th cent. & 14th + 15th cent. & 16th + 17th cent. & 18th + 19th cent\\
\midrule
1. \textit{matar} ‘kill‘ & 5\% (2/42) & 8\% (3/40) & 15\% (6/40) & 37\% (16/43)\\
3. \textit{tomar} ‘take’ & 3\% (1/34) & 4\% (2/47) & 11\% (1/9) & 23\% (7/31)\\
\lspbottomrule
\end{tabular}
}
\end{table}

\figref{11-he-fig:3} compares the development of \textit{a}-marking for \isi{definite} and \isi{indefinite} human direct objects for the two verbs. It shows three points: (i) \textit{a}-marking in \ili{Spanish} increases over time; (ii) it depends on the Referentiality Scale as human \isi{indefinite} direct objects show less preference for DOM than \isi{definite} ones; (iii) there is a tendency for \textit{a}-marking to depend on the verb class, \ie on the preference of the verb for the \isi{animacy} of the \isi{direct object}. Note that only human direct objects were counted, which means that there are two independent parameters: first the actual \isi{animacy} of the \isi{direct object} and second the preference of the verb for the \isi{animacy} of the \isi{direct object}. 

\begin{figure}%[!h]
	\centering
	\includegraphics[width=80ex]{figures/11-he-fig3.pdf} %\vspace{-2ex}
\caption{Percentage of\textit{ a}{}-marking depending on verb class, definiteness and time; Class 1: \textit{matar} ‘kill’ and Class 3: \textit{tomar} ‘take’ (Corpus del Español; from \citealt[606]{vonHeusingeretal2011Affectedness})} \label{11-he-fig:3}
\end{figure}


\section{Blocking of DOM in ditransitive constructions}
\label{11-he-sec:3}

As mentioned above, DOM in \ili{Spanish} is realized by the marker \textit{a}, which is also used for marking the \isi{indirect object} and this marker also represents the main preposition for direction – the marker derives from \ili{Latin} \textit{ad} ‘to’, which can clearly be seen in its prepositional use. 
The marker \textit{a} is differentially used for the \isi{direct object} and obligatorily for the \isi{indirect object}. I assume that \textit{a}-marked direct objects are not datives, but accusatives as shown by the following criteria: a) passivation, \cf \REF{11-he-ex:26}; b) the replacement by the pronoun \textit{lo} for masculine and \textit{la} for feminine, \cf \REF{11-he-ex:27}; and c) the doubling of a leftwards-moved \isi{direct object} by a clitic pronoun \textit{lo} or \textit{la}, \cf \REF{11-he-ex:28} \citep{Campos1999Transitividad}.

\ea %26
\label{11-he-ex:26}

\ea \label{11-he-ex:26a}
\gll Ema y Tito observaron a Ana.\\
Ema and Tito observe.\textsc{pst}.\textsc{3sg} \textsc{dom} Ana\\
\glt ‘Ema and Tito observed Ana.’ 

\ex \label{11-he-ex:26b}
\gll Ana fue observada por Ema y Tito.\\
Ana was observed by Ema and Tito\\
\glt ‘Ana was observed by Ema and Tito.’
\z
\z

\ea %27
\label{11-he-ex:27}
\ea \label{11-he-ex:27a}
\gll  A: ¿Viste a Kiko?\\
{} see.\textsc{pst}.\textsc{2sg} \textsc{dom} Kiko\\
\glt ‘Did you see Kiko?’ 

\ex \label{11-he-ex:27b}
\gll  B: Sí, lo vi.\\
{} Yes, \textsc{acc}.\textsc{3sg} see.\textsc{pst}.\textsc{1sg}\\
\glt ‘Yes, I saw him.’
\z
\z

\ea %28
\label{11-he-ex:28}
\gll A Claudito lo vi por primera vez en diciembre.\\
\textsc{dom} Claudito \textsc{acc}.\textsc{3sg} see.\textsc{pst}.\textsc{1sg} for first time in December\\
\glt ‘Claudito, I saw him for the first time in December.’
\z

The \isi{indirect object} in the dative is defined by the impossibility to form a passive, \cf \REF{11-he-ex:29}--\REF{11-he-ex:30} and the replacement by a clitic pronoun \textit{le} in the singular and \textit{les} in the plural, (\cf \REF{11-he-ex:31}).\footnote{The following assumes non-leísta varieties of \ili{Spanish}. \ili{Spanish} grammars describe as ‘leísta’ varieties the use of \ili{Spanish} where the form \textit{le} stands for the \isi{direct object} (instead of \textit{lo, la}) as in \REF{11-he-ex:i} (under certain conditions – depending on the leísta-type). The verb \textit{conocer} ‘to know’ takes a \isi{direct object}, in the first sentence the \textit{a}-marked \isi{direct object} \textit{a Juan}. In the second sentence, non-leista varieties would use the accusative pronoun \textit{lo}, while leísta varieties take \textit{le} (for the accusative). 

\ea
\label{11-he-ex:i}
%\langinfo{\ili{Spanish}, leísta variety}{}{src}\\
\gll ¿Conoces a Juan? Sí, le conozco hace tiempo.\\ %(leísta variety) 
know.\textsc{2sg} \textsc{dom} Juan. Yes, \textsc{acc}.\textsc{3sg} know.\textsc{1sg} since time\\
 \glt ‘Do you know Juan? Yes, I know him since some time.’
\z

In general, the question of leísta-varieties should not interfere with the question of \textit{a}-marking of the \isi{direct object} since only \isi{definite} indirect objects are clitic doubled, but not direct objects (in most varieties of \ili{Spanish}). Thus the clitic (pronoun) \textit{le} in \REF{11-he-ex:ii} can only double the \isi{indirect object} \textit{al alumno} (the student), but not the \isi{direct object} \textit{su hijo} (‘his son’), which has optionally DOM, see \citet{Fernandez-Ordonez1999Leismo}.

\ea
\label{11-he-ex:ii}
%\langinfo{\ili{Spanish}, leísta variety}{}{src}\\
\gll El maestro le presentó (a) su hijo al alumno.\\
the teacher \textsc{dat}.\textsc{3sg} present.\textsc{pst.3sg} (\textsc{dom}) his son to.the student\\
\glt ‘The teacher presented his son to the student.’
\z
}


\ea%29
\label{11-he-ex:29}

\gll Juan (le) dio una limosna a nuestro vecino ayer.\\
Juan (\textsc{dat.3sg}) give.\textsc{pst.3sg} a charity \textsc{dat} our neighbor yesterday\\
\glt ‘Juan gave our neighbor a charity yesterday.’
\z

\ea%30
\label{11-he-ex:30}
\textit{*Nuestro vecino fue dado una limosna.}\\
intended reading: Our neighbor was given a charity.
\z

\ea%31
\label{11-he-ex:31}
\gll Juan regaló un libro a María, y Pablo le regaló flores.\\
Juan present.\textsc{pst.3sg} a book to Maria, and Pablo \textsc{dat.3sg} present.\textsc{pst.3sg} flowers\\
\glt ‘Juan presented a book to Maria and Pablo presented her flowers.’
\z

Finally, a preposition introduced by \textit{a}, as in \REF{11-he-ex:32} can never be replaced by a clitic pronoun \textit{le}. Rather it must be picked up by a locative expression.

\ea%32
\label{11-he-ex:32}
\gll María viaja a París, y Ana *le / allá viaja, también.\\
María travel.\textsc{3sg} to Paris, and Ana *\textsc{dat.3sg} / there travel-\textsc{3sg}, too\\
\glt ‘María travels to Paris and Ana travels there, too.’
\z

To summarize, the form \textit{a} is used for marking the \isi{direct object} (and then glossed as \textsc{dom}), for marking the \isi{indirect object} (optionally glossed as ‘to’ or ‘\textsc{dat}’) and as a regular preposition ‘to’. 
One can clearly distinguish between the different functions.

\subsection{DOM and clitic doubling of indirect objects}
\label{11-subsec:3-1}

According to \citet[1548]{Campos1999Transitividad}, there are two classes of indirect objects, goals and benefactive: goals stand with predicates of movement or transferring, while benefactives cover indirect objects that are included in the event described by predicates of creation, destruction, ingestion or preparation. For goal datives, \isi{clitic doubling} is optional, \cf \REF{11-he-ex:33}; for benefactives, \isi{clitic doubling} is obligatory, \cf \REF{11-he-ex:34}.

\ea%33
\label{11-he-ex:33}
\gll Lola (le) dio el jugete a Pablo.\\ %(CInd\textsuperscript{1})
Lola (\textsc{dat.3sg}) give.\textsc{pst.3sg} the toy to Pablo\\
\glt ‘Lola gave the toy to Pablo.’ (CInd)
\z

\ea%34
\label{11-he-ex:34}
\gll Lola *(le) rompió el jugete a Pablo.\\ %(CInd\textsuperscript{2})
Lola \textsc{} break.\textsc{pst.3sg} the toy \textsc{dat} Pablo\\
\glt ‘Lola broke Pablo’s toy.’ (CInd)
\z

\citet[1554]{Campos1999Transitividad} also quotes the grammar of the Real Academia Espanola (RAE 1973: §3.4.6), which states that DOM may be dropped in order to disambiguate.

\ea%35
\label{11-he-ex:35}

\gll Presentaron Ø la hija a los invitados.\\
introduce.\textsc{pst.3sg} {} the daughter to the guests\\
\glt ‘They introduced the daughter to the guests.’
\z

According to Campos, the simultaneous use of the marker \textit{a} for the DO and IO becomes ungrammatical when a dative clitic doubles the \isi{indirect object} \REF{11-he-ex:36} (\citealt{Campos1999Transitividad}: 1554, fn.\,79):

\ea%36
\label{11-he-ex:36}
\gll *Les presentaron a la hija a los invitados.\\
\textsc{dat.3pl} introduce.\textsc{pst.3pl} \textsc{dom} the daughter to the guests\\
\glt 'They introduced the daughter to the guests.’ (\citealt{Campos1999Transitividad}: 1554, fn.\,79)
\z

There is extensive literature on \isi{clitic doubling} in \ili{Spanish} (or more generally in Romance languages). 
There are also studies on the development of \isi{clitic doubling} in \ili{Spanish},  I cannot do justice to all of them, but see \citet{Fontana1993Phrase,Fischeretal2003Explaining,Gabrieletal2010Information,vonHeusinger2017Case}.

\subsection{Causative constructions} %3.2 /
\label{11-subsec:3-2}

\citet[24]{Lopez2012Indefinite} observes that in causative constructions the human \isi{indefinite} causee of an \isi{intransitive} verb, as in \REF{11-he-ex:37}, is accusative and \textit{a}-marked according to its \isi{referentiality} status as specific. It is accusative since it cannot be doubled by a clitic in the dative and if it were \isi{inanimate} it would not be marked by \textit{a}. If the complement of a causative predicate is a transitive verb, the causee is obligatorily \textit{a}-marked, but this time it is dative, as can be observed from the \isi{clitic doubling} in \REF{11-he-ex:38}, which is plural, agreeing with \textit{a unas empleadas}. DOM is now optional for the \isi{direct object} of the embedded verb.

\ea%37
\label{11-he-ex:37}
\gll María hizo trabajar los domingos a/*Ø un empleado.\\
María made work the Sundays \textsc{dom} an employee\\
\glt ‘María made an employee work on Sundays.’
\z

\ea%38
\label{11-he-ex:38}
\gll María les hizo visitar a/Ø un enfermo a/*Ø unas empleadas.\\
María \textsc{pl}.\textsc{dat} made visit \textsc{dom} a sick \textsc{dat} some employees\\
\glt ‘María made some employees visit a sick person.’
\z

\citeauthor{Lopez2012Indefinite} also observes that the same facts hold of perception verbs. The \isi{direct object} of perception verbs are obligatorily \textit{a}-marked if human and at least specific, as in \REF{11-he-ex:39}. While the subjects of the embedded clause are indirect objects and thus obligatorily \textit{a}-marked, the \isi{direct object} of the embedded clause in \REF{11-he-ex:40} is optionally \textit{a}-marked.

\ea%39
\label{11-he-ex:39}
\gll María vio caer a/*Ø un niño.\\
María saw fall \textsc{dom} a child\\
\glt ‘María saw a child falling.’
\z

\ea%40
\label{11-he-ex:40}
\gll María vio a/*Ø una empleada visitar a/Ø un enfermo.\\
María saw \textsc{dat} an employee visit \textsc{dom} a sick\\
\glt ‘María saw an employee visiting a sick person.’
\z

Thus, alternating or blocking DOM by a second \textit{a}-marked NP can not only be found in \isi{ditransitive} constructions with direct and indirect objects, but also in causative constructions or constructions with perceptual verbs.

\subsection{Semantic and pragmatic effects}
\label{11-subsec:3-3}

\textit{A}{}-marking of \isi{indefinite} direct objects can signal wide-scope readings, while the lack of \textit{a}-marking often signals narrow scope readings (I leave it open whether the following examples are instances of scope or of a referential \vs non-referential reading of the \isi{indefinite}). \citet[77]{Lopez2012Indefinite} argues that the unmarked \isi{direct object} \textit{un niño} ‘a child’ cannot take scope over the operator expressed by \textit{la mayoria} ‘the most’, while the \textit{a}-marked \textit{a} \textit{un niño} can. This contrast is also found in \isi{ditransitive} constructions, as in \REF{11-he-ex:42}: the \textit{a}-marked version \textit{a un niño} expresses wide scope (a pragmatically not very prominent reading). 

\ea%41
\label{11-he-ex:41}
\gll Ayer vieron la mayoría de los hombres a/Ø un niño.\\
yesterday saw the most of the men \textsc{dom} a child\\
\glt ‘Yesterday most of the men saw a child.’\\
∃>MOST only with \textsc{dom} %TODO@LSP square brackets not displayed: should be "[∃ MOST only with \textsc{dom}] "
\z

\ea%42
\label{11-he-ex:42}
\gll Ayer entregaron a/Ø un niño a la mayoría de las madres.\\
yesterday delivered \textsc{dom} a child \textsc{dat} the majority of the mothers\\
\glt ‘Yesterday they delivered a child to most of the mothers.’\\
∃>MOST only with \textsc{dom} %TODO@LSP square brackets not displayed, change to [∃ > MOST only with \textsc{dom}]
\z
 
\citet[102]{Leonetti2004Specificity} argues that the \textit{a}-marked \textit{un prisionero} in \REF{11-he-ex:43} is a more prominent binder than the unmarked \textit{un prisionero}, and therefore can bind the possessive \textit{su} in the \isi{indirect object}. In the version with \textit{un prisionero}, the possessive \textit{su} is most probably bound by another antecedent.

\ea%43
\label{11-he-ex:43}
\gll Devolvieron a/Ø un prisionero a su tribu.\\
They-returned \textsc{dom} a prisoner to his tribe\\
\glt ‘They returned a prisoner to his tribe.’
\z


\subsection{Summary of the observation for DOM in ditransitive constructions}\label{11-subsec:3-4}

DOM in \isi{ditransitive} constructions is restricted by the co-occurrence of the indirect \isi{object marker} \textit{a}. The very short review above provides the following picture: in most constructions that require DOM in transitive contexts, DOM in \isi{ditransitive} or causative contexts can be blocked by an \isi{indirect object} realized by a full descriptive noun phrase with the marker \textit{a}. The characteristics of this blocking are still not well-investigated.

\section{A diachronic account of DOM in ditransitive constructions} %4. /
\label{11-he-sec:4}

In this section, I present the results of an intensive corpus search on three types of constructions of \isi{ditransitive} verbs: (i) constructions with indirect objects realized as \textit{a}-marked full noun phrases (\isi{definite} NPs and \isi{indefinite} NPs), (ii) constructions with indirect objects as clitic pronouns, and (iii) constructions with non-overt indirect objects. In \sectref{11-subsec:4-1} I give a short summary of a similar study of \citet{Ortiz2005Objetos,Ortiz2011Construcciones}, in \sectref{11-subsec:4-2} I provide information on how I collected the material and composed the corpus, and \sectref{11-subsec:4-3} the results and discussion of the three hypotheses formulated in \sectref{11-he-sec:1} are presented.


\subsection{Earlier studies in ditransitive constructions} %4.1 /
\label{11-subsec:4-1}

\citet{Ortiz2005Objetos,Ortiz2011Construcciones} has analyzed a \isi{diachronic corpus} of \ili{Spanish} with respect to \isi{ditransitive} construction from the 13th to the 20th century. In her corpus  \citet[20]{Ortiz2011Construcciones} identified 3,061 \isi{ditransitive} constructions, of which 2,269 occur with finite and 792 with nonfinite verbs. For \isi{ditransitive} constructions with full noun phrases, she restricts her analysis to the finite contexts. In her study \citep{Ortiz2005Objetos}, she investigates the 13th, 14th, 16th, 19th and 20th century with 1,661 \isi{ditransitive} constructions with 141 instances of full human noun phrases for the \isi{direct object} and for the \isi{indirect object}:\footnote{\citet[162]{Ortiz2011Construcciones} notes that languages resist a construction with full noun phrases for a human direct and a human \isi{indirect object}. Only 8.5\% of all investigated cases show this configuration. See also \citet{vonHeusingeretal2011Affectedness}, who report from similar low percentages of full noun phrases for human direct objects in transitive constructions.}\textsuperscript{,}\footnote{Note that Ortiz Ciscomani uses two different tables. In her dissertation \citep{Ortiz2011Construcciones} she presents the table as in \tabref{11-he-tab:8} with all centuries from 13th to 20th, while in her article \citep{Ortiz2005Objetos} she only selects 13th, 14th, 16th, 19th and 20th – hence the different numbers.}

\begin{table}
\caption{Percentage of human direct object with DOM and without DOM with respect to all instances of ditransitive constructions (\citealt[162]{Ortiz2011Construcciones}; \citealt[198]{Ortiz2005Objetos})}
\label{11-he-tab:6}

\begin{tabularx}{\textwidth}{Xrrr}
\lsptoprule

 century & \% DO with DOM & \% DO without DOM & \% total\\
\midrule
 13th & 2.2\% (7/316) & 8.2\% (26/316) & 10.4\% (33/316)\\
 14th & 5.2 \% (6/115) & 30.4\% (35/115) & 35.7\% (41/115)\\
 16th & 1.1\% (6/567) & 8.3\% (47/567) & 9.3\% (53/567)\\
 19th & 0.8\% (3/381) & 1.6\% (6/381) & 2.4\% (9/381)\\
 20th & 1.4\% (4/282) & 0.4\% (1/282) & 1.8\% (5/282)\\
\midrule
 total & 1.6\% (26/1661) & 7\% (115/1661) & 8.5 (141/1661)\\
\lspbottomrule
\end{tabularx}
\end{table}

%\begin{table}
%\caption{Percentage of human direct object with DOM and without DOM with respect to all instances of ditransitive constructions (\citealt[162]{Ortiz2011Construcciones}; \citealt[198]{Ortiz2005Objetos})}
%\label{11-he-tab:6}
%
%\begin{tabularx}{\textwidth}{Xrrr}
%\lsptoprule
%
% century & \% DO with DOM & \% DO without DOM & \% total\\
%\midrule
% 13th & 2\% (7/316) & 8\% (26/316) & 10\% (33/316)\\
% 14th & 5 \% (6/115) & 30\% (35/115) & 36\% (41/115)\\
% 16th & 1\% (6/567) & 8\% (47/567) & 9\% (53/567)\\
% 19th & 0.8\% (3/381) & 1.6\% (6/381) & 2.4\% (9/381)\\
% 20th & 1.4\% (4/282) & 0.4\% (1/282) & 1.8\% (5/282)\\
%\midrule
% total & 1.6\% (26/1661) & 7\% (115/1661) & 8.5 (141/1661)\\
%\lspbottomrule
%\end{tabularx}
%\end{table}



\citet{Ortiz2005Objetos} observes that (i) the percentage of this construction (with two full human noun phrases) with respect to all constructions decreases from 10\% and 36\% in the 13th and 14th century to about 2\% in the 19th and 20th century; (ii) that the contrast between DOM and the lack of DOM persists through time. She does not calculate the percentages of DOM \vs non-DOM constructions for full noun phrases (both \isi{direct object} and \isi{indirect object}), but see \citet[47]{Comrie2013Human} and \tabref{11-he-tab:7} for a different presentation of the same material such that one can compare the relation between DOM \vs non-DOM at each century. It becomes obvious that DOM increases through time even though the 19th and 20th centuries provide very few data. \tabref{11-he-tab:7} compares the figures for \isi{ditransitive} constructions with the figures of \citet{Laca2006Objeto}, see \tabref{11-he-tab:1} above) for transitive constructions. One can assume that the stark contrast between \isi{definite} and \isi{indefinite} direct objects with respect to DOM observed for transitive construction also holds for \isi{ditransitive} construction.

\begin{table}
\caption{Percentages of DOM based on number of ditransitive constructions with human direct objects and human indirect objects (reanalysis of Table 6 of \citealt[198]{Ortiz2005Objetos}) – compared to the data of transitive constructions (see \citealt{Laca2006Objeto}: 442 and \tabref{11-he-tab:1} above)}
\label{11-he-tab:7}

\begin{tabularx}{\textwidth}{Qp{4.2cm}Qll}
\lsptoprule
& \% of DOM with ditr.\,verbs for \isi{definite} and \isi{indefinite} NPs \citep{Ortiz2005Objetos} && \multicolumn{2}{p{4cm}}{ \% of DOM with tr.\,verbs
 \citep{Laca2006Objeto}
}\\
cent.& & cent.  & \isi{definite} NPs & \isi{indefinite} NPs\\
\midrule 
 13th & 21\% (7/33) & 12th & 36\% (13/36) & 0\% (0/6)\\
 14th & 15\% (6/41) & 14th & 55\% (36/66) & 6\% (2/31)\\
 16th & 11\% (6/53) & 16th & 70\% (85/122) & 12\% (7/59)\\
 19th & 33\% (3/9) &19th & 96\% (73/76) & 41\% (12/29)\\
20th & 80\% (4/5) & 20th & {}- & {}-\\
 total & 18\% (26/141) & total & 69\% (207/300) & 17\% (21/125)\\
\lspbottomrule
\end{tabularx}
\end{table}

\citet[166]{Ortiz2011Construcciones} also observes that only certain \isi{ditransitive} verbs are constructed with DOM, as \tabref{11-he-tab:8} shows.



\begin{table}
\caption{Verbs with DOM in ditransitive constructions with human direct objects and human indirect objects (\citealt[166]{Ortiz2011Construcciones}; my own translation, KvH)}
\label{11-he-tab:8}
\fittable{
\begin{tabularx}{\textwidth}{l@{\,}c@{\,}c@{\,}c@{\,}c@{\,}c@{\,}c@{\,}c@{\,}rS}

\lsptoprule
& \multicolumn{8}{c}{century}\\
Verbo & 13th & 14th & 15th & 16th& 17th & 18th & 19th & 20th & Total\\
\midrule
\textit{dar} ‘to give’ & 2 & 1 & & 1 & & & 1 & & 5\\
\textit{enviar} ‘to send’ & 2 & 5 & 4 & 4 & & & & 1 & 16\\
\textit{encomendar} ‘to entrust’ & 1 & & & & 1 & 2 & 1 & & 5\\
\textit{toller} ‘to take away’ & 1 & & & & & & & & 1\\
\textit{echar} ‘to throw’ & 1 & & 1 & & & & & & 2\\
\textit{llevar} ‘to carry’ & & & & 1 & & & & & 1\\
\textit{entregar} ‘to submit’ & & & & & & & 1 & & 1\\
\textit{mandar} ‘to order, to send’ & & & & & & & & 1 & 1\\
\textit{mostrar} ‘to show’ & & & & & & & & 1 & 1\\
\textit{presenter} ‘to present’ & & & & & & & & 1 & 1\\
\midrule
total & 7 & 6 & 5 & 6 & 1 & 2 & 3 & 4 & (34/2269) 1.5\%\\
\lspbottomrule
\end{tabularx}
}
\end{table}


To summarize, \citet{Ortiz2011Construcciones} provides the first quantitative approach to the \isi{diachrony} of \isi{ditransitive} constructions. She has analyzed more than 3,000 sentences with \isi{ditransitive} constructions, of which less than 10\% are with a human full NP as \isi{indirect object} and a human full NP as a \isi{direct object}. There are less than 20\% of instances with \textit{a}-marking for both arguments and the data suggest a development towards this kind of marking (and less blocking). However, data are very scarce and therefore quantitative conclusions cannot be drawn from her analysis. She has also identified certain verb classes that allow DOM in this construction. While this study is very instructive, it needs complementary studies in larger corpora.

\subsection{Data collection}\label{11-subsec:4-2}
 
\subsubsection{Method} \label{11-subsubsec:4-2-1}

In order to complement the corpus study of \citet{Ortiz2005Objetos,Ortiz2011Construcciones}, I started an extensive corpus search focused on particular verbs. I used Mark Davies’ Corpus del Español, which comprises 100 million words of \ili{Spanish} texts from the 12th to the 19th century. The corpus interface allows one to search for lemmas, rather than for word forms. In a first step I identified the verbs to be analyzed. I started from \citeauthor{Ortiz2011Construcciones}'s  (\citeyear{Ortiz2005Objetos,Ortiz2011Construcciones}) list of verbs and modified it according to assumed verb properties and their behavior in contemporary \ili{Spanish}. I identified two verb classes with two verbs each: A: verbs of caused perception (\textit{presentar} ‘to present’, \textit{recomendar} ‘to recommend’; and B: verbs of caused motion (\textit{enviar} ‘to send’, \textit{poner} ‘to put’). 

In the Corpus del Español, I searched for the corresponding lemmata for \textit{presentar} for four different centuries: 17th, 18th, 19th, and 20th, for \textit{recomendar} I collected data from the 18th and 20th century and for \textit{enviar} and \textit{poner} from the 17th and 20th century. When the search resulted in more than 1,000 hits per century, the search was restricted to the first 1,000 hits and filtered to cases with human full noun phrases as direct objects (\isi{definite} NPs and \isi{indefinite} NPs), since only those cases qualify for DOM. I distinguished three types of constructions: (i) The \isi{indirect object} is realized as a human full noun phrase. (ii) The \isi{indirect object} is realized by a clitic pronoun, and (iii) the \isi{indirect object} is not overtly realized, \ie the construction looks like a transitive construction. E.g. the search for the lemma \textit{presentar} resulted in 1,031 hits from the 17th century. I analyzed the first 1,000 hits; there were 47 instances with a human full noun phrase as \isi{direct object}. Out of these 47 cases, there were 8 (2+6) with a human full noun phrase as \isi{indirect object}; 18 (2+16) instances of the \isi{indirect object} realized as a clitic pronoun, 18 (9+9) instances of no overt \isi{indirect object}, and 3 cases I could either not analyze or not categorize into one of the three categories. For the first three categories I distinguished between DOM or the lack of it, as summarized in \tabref{11-he-tab:9}.

\begin{table}
\caption{Sample analysis for \textit{presenter} ‘to present’ for the 17th century in the Corpus del Español}
\label{11-he-tab:9}
\fittable{
\begin{tabular}{l rr rr rr rr rrr} 
\lsptoprule
& 
cent &
\multicolumn{2}{c}{full human IO} &
\multicolumn{2}{c}{ IO as clitic only} &
\multicolumn{2}{c}{ no overt IO} & &
\multicolumn{3}{c}{ hits}\\
\midrule 
& &  \textsc{dom} &  no \textsc{dom} &  \textsc{dom} &  no \textsc{dom} &  \textsc{dom} &  no \textsc{dom} &  else &  analyzed &  searched &  all\\
\textit{presentar} &  17 &  2 &  6 &  2 &  16 &  9 &  9 &  3 &  47 &  1000 &  1031\\
\lspbottomrule
\end{tabular}
}
\end{table}

\newpage 
About 13,000 entries in total were analyzed out of which about 600 had a human \isi{direct object}, \ie a \isi{direct object} that can be optionally \textit{a}-marked. Some verbs and constructions had to be eliminated such that eventually 322, \ie about 2.5\% of the analyzed hits, could be used for the final analysis, as presented in \tabref{11-he-tab:10}.\footnote{Four other verbs had to be excluded from further analysis: the search for the lemmata \textit{acusar} (‘to accuse’) and \textit{denunciar} (‘to denounce’) resulted in only transitive constructions, but not in \isi{ditransitive} constructions. I also excluded the verb \textit{encomendar} (‘to entrust’, ‘to (re)commend’) as it seems to be conventionalized in using it with an \isi{indirect object} either \textit{a Dios} (‘God’) or \textit{a la Madre del cielo} (‘the mother of heaven’). The great majority of these examples have \textit{a}-marking for the \isi{direct object}. I speculate that the meaning is conventionalized and understood as an opaque idiomatic expression. I also excluded the 16 instances of \textit{dar} ‘to give’, since they were difficult to categorize and often close to idiomatic or light verb constructions, as well as all bare nouns and proper names since their \isi{referentiality} status obligatorily determines DOM or no DOM, respectively, see \sectref{11-subsubsec:4-3-1} and \tabref{11-he-tab:11} below.}

\begin{table}
\caption{Overview of the distribution of hits to verb classes and DOM \vs no DOM instances in the Corpus del Español (17th to 20th century).}
\label{11-he-tab:10}
\begin{tabularx}{\textwidth}{Qrrr}
\lsptoprule
Verb class &  \textsc{dom} &  no \textsc{dom} &  Sum\\
\midrule
\textbf{A (caused perception)} &  \textbf{64}  &  \textbf{61} &  \textbf{125}\\
\midrule
\textit{presentar} ‘to present’ &  54 &  50 &  104\\
\textit{recomendar} ‘to recommend’ &  10 &  11 &  21\\
    \tablevspace
\textbf{B (caused motion)} &  \textbf{92} &  \textbf{105} &  \textbf{197}\\
\midrule
\textit{enviar} ‘to send’ &  73 &  90 &  163\\
\textit{poner} ‘to put’ &  19 &  15 &  34\\
\midrule
\textbf{total} &  \textbf{156} &  \textbf{166} &  \textbf{322}\\
\lspbottomrule
\end{tabularx}
\end{table}

 
\subsubsection{Analyzing particular examples}\label{11-subsubsec.4-2-2}

Before I discuss the overall results, I will present some particular examples in detail. This will provide more information about the structure of the examples, but also show that in each particular case, additional parameters might have contributed to the \textit{a}-marking of the \isi{direct object}, its blocking or its lack of \textit{a}-marking (one cannot always clearly distinguish between a \isi{blocking effect} and a case in which \textit{a}-marking is not licensed due to other parameters). In order to facilitate the reading of the examples, I annotated the subject (Sub), the \isi{direct object} (DO), the \isi{indirect object} (IO) and highlighted the verb, the \isi{direct object} and the \isi{indirect object}. In some cases I mark long noun phrases by brackets for the ease of parsing. In \REF{11-he-ex:44} the \isi{direct object} \textit{el celebrado don Diego de Covarrubias y Leiva} ‘the celebrated don Diego de Covarrubias y Leiva’ is \textit{a}-marked besides the \textit{a}-marked \isi{indirect object}\textit{ a nuestro obispado} ‘our bishopric’. In \REF{11-he-ex:45} the \isi{direct object} \textit{los enfermos} is not \textit{a}-marked, even though the construction and \isi{word order} are very similar. There are clear differences between the two direct objects: the \isi{direct object} in \REF{11-he-ex:44} is a proper name, is singular and has much more descriptive content – all parameters known to contribute to DOM.

\ea (ia): \textsc{dom} and full indirect object\\\label{11-he-ex:44}
\emph{Promovido a Valencia don Martín Pérez Ayala, \textbf{presentó} el rey\textsubscript{Sub} \textbf{a nuestro obispado}\textbf{\textsubscript{’IO}} [\textbf{al celebrado don Diego de Covarrubias y Leiva]}\textbf{\textsubscript{DO}}, que al presente era obispo de Ciudad Rodrigo.} (Colmenares, Diego de. (1586–1651), Historia de la insigne ciudad de Segovia y compendio de las historias de Castilla)\\

\glt ‘After the promotion of don Martín Pérez Ayala to Valencia, the king \textbf{introduced} \textbf{to our bishopric the celebrated don Diego de Covarrubias y Leiva} who currently was the bishop of the city (of) Rodrigo.’
\z

\ea(ib): no DOM and full indirect object\\\label{11-he-ex:45}
\emph{Los Médicos\textsubscript{Sub} son los\textsubscript{Sub} que \textbf{presentan} \textbf{al Rey}\textbf{\textsubscript{IO}}\textbf{ los enfermos}\textbf{\textsubscript{DO}}.} (Feijoo, Benito Jerónimo (1676–1764), Cartas eruditas y curiosas, vol. 1)\\
\glt ‘The doctors are the ones who \textbf{present the sick to the king}.’
\z

In \REF{11-he-ex:46} the \isi{indirect object} is realized as the postclitic pronoun \textit{os}, and the \isi{direct object} \textit{al señor conde del Verde Saúco} is \textit{a}-marked. In \REF{11-he-ex:47}, however, the \isi{direct object} \textit{profetas y doctores} is unmarked. Again, there are further differences between these two examples: the \isi{direct object} in \REF{11-he-ex:46} is a proper name, while it is a plural \isi{indefinite} in \REF{11-he-ex:47}. According to the Referentiality Scale a proper name obligatorily takes DOM, while a plural \isi{indefinite} can take it optionally.

\ea (iia): DOM and indirect object realized as clitic pronoun\\\label{11-he-ex:46}
\emph{Tengo el honor de \textbf{presentar-os}\textbf{\textsubscript{IO}} \textbf{[al señor conde del Verde Saúco]}\textbf{\textsubscript{DO}}, de quien acabamos de recibir esa carta pidiéndonos nuestra hija en matrimonio.} (Larra, Mariano José de. (1809–1837), No más mostrador)\\
\glt ‘I have the honor of \textbf{introducing to you the count of Verde Sauco} […].’
\z

\ea (iib): no DOM and indirect object realized as clitic pronoun\\\label{11-he-ex:47}
\emph{Con estas dos causas, que una bastara ante vos, parezco, \textbf{y [profetas y doctores]}\textbf{\textsubscript{DO}}\textbf{ por testigos os}\textbf{\textsubscript{IO}}\textbf{ presento.}} (Calderón de la Barca, Pedro. (1600–1681), El pleito matrimonial del cuerpo y el alma)\\
\glt ‘With these two cases I appear, so that one should suffice before you, and I \textbf{present to you prophets and doctors} as witnesses.’
\z

In the following two instances, the \isi{indirect object} is not overtly expressed. In \REF{11-he-ex:48} the descriptively rich proper name is \textit{a}-marked, while the \isi{indefinite} plural noun phrases in \REF{11-he-ex:49} are not. This seems to replicate the effect of \REF{11-he-ex:46} \vs \REF{11-he-ex:47} in that the position on the Referentiality Scale determines DOM. 

\ea(iiia): DOM and indirect object not overtly realized\\\label{11-he-ex:48}
\emph{Tuvo el emperador\textsubscript{Sub} aviso en Alemania de la muerte de nuestro obispo don Antonio Ramírez; y \textbf{presentó} para obispo \textbf{[a nuestro gran segoviano fray Domingo de Soto]}\textbf{\textsubscript{DO}}, que interpolado el santo concilio, fue llamado del césar para su confesor.} (Colmenares, Diego de. (1586–1651), Historia de la insigne ciudad de Segovia y compendio de las historias de Castilla)\\
\glt ‘While in Germany the emperor was informed about the death of our bishop don Antonio Ramírez; and he \textbf{proposed} as bishop \textbf{our great Brother Domingo de Soto from Segovia} who was called by the emperor as his confessor after the interpolation of the holy council.’
\z

\ea(iiib): no DOM and indirect object not overtly realized\\\label{11-he-ex:49}
\emph{Luego veintiocho hermanos conducidos de Juan de Dios; de la Victoria, ochenta, por su ministro provincial regidos. \textbf{Ochenta y seis}\textbf{\textsubscript{DO}}\textbf{ San Augustín}\textbf{\textsubscript{Sub}}\textbf{ presenta, ciento}\textbf{\textsubscript{DO}}\textbf{ da San Francisco}\textbf{\textsubscript{Sub}}\textbf{, y otros ciento}\textbf{\textsubscript{DO}} \textbf{santo Domingo}\textbf{\textsubscript{Sub}}\textbf{ da} con igual cuenta.}  (Espinosa, Pedro. (1578–1650), Poesía)\\
\glt ‘Afterwards twenty eight brothers brought from Juan de Dios; from Victoria eighty, controlled by the provincial minister. San Augustin \textbf{presents eighty six}, San Francisco \textbf{gives hundred}, Santo Domingo \textbf{gives hundred} \textbf{more} with identical bill.’
\z

\subsection{Main results}\label{11-subsec:4-3}

\subsubsection{Referentiality}\label{11-subsubsec:4-3-1}

Referentiality of the \isi{direct object} is one of the main factors in determining \textit{a}-marking in transitive constructions. This also holds for \isi{ditransitive} verbs. As can be seen in \tabref{11-he-tab:11}, nearly all direct objects realized as bare nouns are unmarked and all except one realized as proper names are marked. This means that the variation only affects \isi{definite} and \isi{indefinite} noun phrases.

\begin{table}
\caption{Referentiality or types of direct objects (bare, indefinite, definite, proper name) 
and DOM in the Corpus del Español (17th to 20th century).
}\label{11-he-tab:11}
\begin{tabularx}{\textwidth}{lQQQ}
\lsptoprule
Type of noun phrase &  DOM &  No DOM &  Sum\\
\midrule
bare noun &  10\% (1) &  90\% (9) &  100\% (10)\\
\isi{definite} NPs &  67\% 116) &  33\% (56) &  100\% (172)\\
\isi{indefinite} NPs &  27\% (40) &  73\% (110) &  100\% (150)\\
proper name &  99\% (66) &  1\% (1) &  100\% (67)\\
total &  56\% (223) &  44\% (176) &  100\% (399)\\
\lspbottomrule
\end{tabularx}
\end{table}

Therefore, the remaining discussion has been limited to \isi{definite} and \isi{indefinite} noun phrases, 322 hits in total, with nearly as much DOM direct objects as no DOM direct objects, as listed in \tabref{11-he-tab:12}.\footnote{Note that the 322 hits are the number that has already been presented in \tabref{11-he-tab:10}, where only \isi{definite} and \isi{indefinite} direct objects were listed.} There is the expected difference between these two groups of referential expressions: one third of \isi{indefinite} noun phrases are marked, while two thirds of definites are marked.

\begin{table}
\caption{Distribution of definite and indefinite direct objects and DOM in the Corpus del Español (17th to 20th century).}\label{11-he-tab:12}
\begin{tabularx}{\textwidth}{lQQQ}
\lsptoprule
Type of noun phrase &  DOM &  No DOM &  Sum\\
\midrule
\isi{definite} NPs &  67\% (116) &  33\% (56) &  100\% (172)\\
\isi{indefinite} NPs &  27\% (40) &  73\% (110) &  100\% (150)\\
total &  48\% (156) &  52\% (166) &  100\% (322)\\
\lspbottomrule
\end{tabularx}
\end{table}

\subsubsection{Type of ditransitive construction} %4.3.2 /
\label{11-subsubsec:4-3-2}

Hypothesis 1 said that the type of \isi{ditransitive} construction determines the \isi{blocking effect}. One distinguished between (i) constructions with indirect objects realized as \textit{a}-marked full noun phrases (\isi{definite} NPs, \isi{indefinite} NPs), (ii) constructions with indirect objects as clitic pronouns, and (iii) constructions with non-overt indirect objects do not show any \isi{blocking effect}.

The data show that (i) construction with a full \isi{indirect object} blocks \textit{a}-marking of the \isi{direct object} blocks DOM: only 24\% of the direct objects are \textit{a}-marked in this construction. On the other side, if the \isi{indirect object} is not realized, 54\% of the direct objects are \textit{a}-marked. This very much corresponds to the percentage of DOM with transitive verbs, see \tabref{11-he-tab:1} above. (ii) The construction with an \isi{indirect object} realized as clitic pronoun shows less blocking than the full noun and more blocking than the case without overt \isi{indirect object}.\footnote{The contrast between these three constructions is not an effect of an uneven distribution of \isi{definite} \vs \isi{indefinite} direct objects (see \tabref{11-he-tab:12}). In all three construction types, the number of \isi{definite} and \isi{indefinite} direct objects is more or less equal.}

\begin{table}
\caption{Distribution of types of indirect objects in percentage of \textit{a}-marking (absolute values) in the Corpus del Español}
\label{11-he-tab:13}

\begin{tabularx}{\textwidth}{lllll}
\lsptoprule
 realization of IO &  full human IO &  clitic pronoun IO &  no overt IO& sum\\
 \midrule
 DOM &  24\% (8/34) &  44\% (27/64) &  54\% (121/224)& 48\% (156/322)\\
\lspbottomrule
\end{tabularx}
\end{table}

\subsubsection{Diachronic development} \label{11-subsubsec:4-3-3}

\tabref{11-he-tab:14} summarizes the \isi{diachronic development} from the 17th/18th century to the 19th/\linebreak 20th century %TODO fix box overflow
– two centuries have been collapsed in order to have a larger number of instances. For a zero realization and the realization by a clitic pronoun of the \isi{indirect object}, no \isi{blocking effect} is observable. In both construction types the \textit{a}-marking increases over time, such as in the cases of the transitive verbs (see \citealt{Melis1995Objetodirecto}, \citealt{Laca2006Objeto}, \citealt{vonHeusingeretal2007Differential,vonHeusinger2008Verbal}, see \tabref{11-he-tab:1}--\tabref{11-he-tab:5} in \sectref{11-he-sec:2} above). What is surprising, though, is that for full indirect objects the \textit{a}-marking of the \isi{direct object} is blocked by 70\% and 100\%, respectively. 
This would suggest that only the overt \textit{a} for the \isi{indirect object} blocks the \textit{a}-marking of the \isi{direct object}. Note, however, that there were only 7 instances of this construction.

\begin{table}
\caption{DOM for human full direct objects and 17th/18th \vs 19th/20th century in percentage (absolute values) in the Corpus del Español }\label{11-he-tab:14}
\begin{tabularx}{\textwidth}{lQQQ}
\lsptoprule
cent & full human IO & pronominal clitic IO & no overt IO\\
\midrule
17th/18th & 30\% (8/27) & 22\% (7/32) & 45\% (46/102)\\
19th/20th & 0\% (0/7) & 67\% (20/30) & 60\% (75/124)\\
\lspbottomrule
\end{tabularx}
\end{table}

\subsubsection{Verb class}
\label{11-subsubsec:4-3-4}

The second hypothesis is that verb class differences are mirrored in the blocking of the \textit{a}-marking of the \isi{direct object} (or in the strength with which the \textit{a}-marking of the \isi{direct object} has to be obtained). In earlier work it was shown that there is a clear difference for different transitive verb classes. According to the study discussed in \sectref{11-subsec:2-2} above, transitive verbs that require an \isi{animate} \isi{direct object} (such as \textit{matar} ‘to kill’) more often take DOM than verbs like \textit{tomar} (‘to take’) that prefer an \isi{inanimate} \isi{direct object} (see \tabref{11-he-tab:2}--\tabref{11-he-tab:5} in \sectref{11-subsec:2-3} above). In a forced choice experiment conducted by \citet{vonHeusinger2017Case}, verbs of caused perception (\textit{presentar} ‘to present’, \textit{proponer} ‘to propose’ received DOM in 54\% (98/182) of the cases, while verbs of caused motion (\textit{enviar} ‘to send’, \textit{mandar} ‘to send’) received DOM in 65\% (119/182) of the cases. Therefore, I predict that in the \isi{diachronic corpus} there will be more verbs of caused motion with \textit{a}-marking, than verbs of caused perception, even in typical blocking contexts. However, as can be seen in \tabref{11-he-tab:15}, there are more \textit{a}-marked direct objects with verbs of caused perception (68\%) than \textit{a}-marked verbs of caused motion (49\%) if the \isi{indirect object} is not realized. And there is a slight preference for \textit{a}-marking for verbs of caused perception over verbs of caused motion in the other conditions as well.

\begin{table}
\caption{DOM for human animate full direct objects and verb class in percentage (absolute values) in the Corpus del Español}
\label{11-he-tab:15}


\begin{tabularx}{\textwidth}{Q rrr}
\lsptoprule
verb class & full human IO & IO as clitic only & no overt IO\\
\midrule
A: \textit{presentar, recomendar} & 26\% (5/19) & 45\% (25/56) & 68\% (34/50)\\
B: \textit{enviar, poner} & 20\% (3/15) & 33\% (2/6) & 49\% (87/176)\\
\lspbottomrule
\end{tabularx}
\end{table}


\section{General discussion and conclusion} %5. /
\label{11-he-sec:5}

In \sectref{11-he-sec:1} I put forward three hypotheses, which are repeated below and which were tested by the extended corpus search and the analysis in the last section. Due to the scarce data I cannot make any statistically significant claims, but the figures show certain tendencies for the hypothesis.
\begin{itemize}
\item H1: The type of the \isi{ditransitive} construction determines the \isi{blocking effect}:
\begin{enumerate}[label=\roman*]
 \item constructions with indirect objects realized as \textit{a}-marked full noun phrases (\isi{definite} NPs, \isi{indefinite} NPs) show a high \isi{blocking effect}
 \item constructions with indirect objects as pronominal clitics show a low \isi{blocking effect}, and 
 \item constructions with non-overt indirect objects do not show any \isi{blocking effect}
\end{enumerate}

\item H2: DOM in \isi{ditransitive} constructions has a comparable development as DOM in transitive constructions.
\item H3 Verb classes differ with respect to the way they influence DOM and DOM-blocking.
\end{itemize}

The analysis of the corpus data suggests that Hypothesis 1 is correct: Type (i) realizes the \isi{indirect object} as a full noun phrase that is obligatorily marked by \textit{a.} Here \textit{a}-marking of the \isi{direct object} is very low. In type (iii), the \isi{indirect object} is not realized – either because the \isi{indirect object} is inferred from the context or left unspecified. Here, \textit{a}-marking of the \isi{direct object} is high and similar to pure transitive constructions. In type (ii), the \isi{indirect object} is realized as a clitic pronoun. Here the rate of \textit{a}-marking lies between construction (i) and (iii) – if correct, this is surprising since no overt \textit{a} for the \isi{direct object} is available.

The \isi{diachronic development} of DOM in \isi{ditransitive} constructions follows the dia\-chro\-nic development of DOM in transitive constructions. However, the \isi{blocking effect} for construction (i) is becoming stronger over the years. Due to the very low figures I cannot estimate whether this is a stable tendency or not. There is no clear evidence for Hypothesis 3, as the contrast between the two verb classes are minor, except for the transitive construal (iii), where a tendency towards more marking of verbs of caused perception can be seen.

The investigation of a corpus of diachronic data of \isi{ditransitive} constructions in \ili{Spanish} has revealed that DOM in \isi{ditransitive} constructions has developed similarly to DOM in transitive constructions – along the Referentiality Scale and the Affectedness Scale. However, DOM in \isi{ditransitive} constructions occurs with a lower frequency than in transitive constructions. This effect is generally assumed to be the result of some blocking between the \textit{a}-marking of the \isi{indirect object} and the \textit{a}-marking (\ie DOM) of the \isi{direct object}. I have investigated three types of \isi{ditransitive} constructions: (i) with indirect objects realized as \textit{a}-marked full noun phrases (\isi{definite} NPs, \isi{indefinite} NPs), (ii) with indirect objects as clitic pronouns, and (iii) non-overt indirect objects. There is a clear difference between these three types: DOM is more frequent with (iii) and less frequent with (i). The data revealed an interesting interaction with the \isi{diachronic development}: for construal (i) I found more DOM in the 17th and 18th century than in the 19th and 20th. The data did not support a strong interaction between verb class and DOM. Nevertheless, they show the importance of an analysis that allows to distinguish nominal from verbal parameters.

\section*{Acknowledgements}

I would like to thank one anonymous reviewer for very helpful and constructive comments, and the editors of this volume, Ilja A\,Seržant and Alena Witzlack-Makarevich, for editing this volume and providing very helpful comments on an earlier draft. I also thank Marco García García, Georg A.\,Kaiser and Manuel Leonetti for constructive comments. 
Alexandra Wolf and Diego Romero helped me in searching the corpus and annotating the examples. All remaining mistakes are of course my own. 
The research for this paper has been funded by the German Research Foundation (DFG) as part of the SFB 1252 “Prominence in Language” in the project B04 “Interaction of nominal and verbal features for Differential Object Marking” at the University 
of Cologne.

\section*{Abbreviations}
\begin{tabularx}{.45\textwidth}{lQ}
\textsc{1} & first person\\
\textsc{2} & second person\\
\textsc{3} & third person\\
\textsc{acc} & accusative\\
\textsc{dat} & dative\\
\textsc{do} & direct object\\
\textsc{dom} & differential object marking \\
\textsc{gen} & genitive\\ 
\textsc{imp} & imperative\\
\end{tabularx}
\begin{tabularx}{.45\textwidth}{lQ}
\textsc{inf} & infinitive\\
\textsc{io} & indirect object\\
\textsc{masc} & masculine\\
\textsc{nom} & nominative\\
\textsc{pl} & plural\\
\textsc{pst} & past\\
\textsc{sg} & singular\\
\textsc{subj} & subject\\
\end{tabularx}

{\sloppy
\printbibliography[heading=subbibliography,notkeyword=this] }
\end{document}