\documentclass[output=paper]{LSP/langsci}
\ChapterDOI{10.5281/zenodo.1228269} 
\author{Yuko Yanagida\affiliation{University of Tsukuba}} 
\title{Differential subject marking and its demise in the history of Japanese}
%\epigram{Change epigram in chapters/03.tex or remove it there }
\abstract{The subject of various types of subordinate or nominalized clauses in Old Japanese (700--800) is marked in one of three different ways: with the postpositional particle \textit{ga}, \textit{no} or zero. This paper argues that the opposition between case marked and unmarked subjects fit into cross-linguistically well attested patterns of differential subject marking (DSM). Following \citet{Woolford2008Differential}, it shows that the syntactic and semantic characteristics of these case marking patterns reveal thatOJ \il{Old Japanese} displays two kinds of DSM effects which are associated with distinct grammatical levels. This paper also examines three possible scenarios for the loss of DSM, which occurred in Early Middle Japanese (EMJ 800–1200). TheOJ \il{Old Japanese} \il{Old Japanese} and EMJ data suggest that case systems do not simply shift from one alignment pattern to another, as sometimes assumed (\cf \citealt[258]{Harrisetal1995Historical}). Instead, the morphological features of individual case markers change incrementally over time, ultimately giving rise to global changes in the overall system.

% \keywords{nominalization, alignment change, nominal hierarchy, psych-predicate, object movement}
}
\maketitle

\begin{document}

\section{Introduction}
\label{14-ya-sec:1}

Modern \ili{Japanese} (ModJ) displays a straightforward nominative-accusative system. Transitivity does not affect the \isi{case marking} on the subject \REF{14-ya-ex:1}.

\begin{exe}
\ex Modern \ili{Japanese}%1
\label{14-ya-ex:1}
\begin{xlist}
\ex
\label{14-ya-ex:1a}
\gll Taroo \textbf{ga} sake \textbf{o} non-da koto \upshape(transitive)\\
Taroo \textsc{nom} sake \textsc{acc} drink-\textsc{pst} that\\
\glt ‘that Taroo drank sake’

\ex
\label{14-ya-ex:1b}
\gll sakura \textbf{ga} sai-ta koto \upshape(\isi{intransitive})\\
cherry.blossom \textsc{nom} bloom-\textsc{pst} that\\
\glt ‘that Cherry blossoms bloomed’
\end{xlist}
\end{exe}


In ModJ the case markers \textit{ga} and \textit{o} mark the subject and object respectively as grammatical case markers; these particles display no semantic effects. 

In \ili{Old Japanese} (OJ; 8th century), \textit{ga} is a \isi{genitive case} marker. \textit{Ga} marks the possessor of noun phrases \REF{14-ya-ex:2} and the subject of various types of subordinate or nominalized clauses \REF{14-ya-ex:3}. Personal pronouns and human nouns intimate to the speaker as in \textit{seko} ‘lover’ and \textit{kimi} ‘lord’ are obligatorily marked by \textit{ga}, while non-human \isi{animate} and \isi{inanimate} NPs are predominantly marked by the other genitive \textit{no} or by zero.\footnote{OJ data in this study are taken from the \textit{Man’y\=osh\=u} (MYS, compiled in mid-8th century), the earliest written record of OJ, comprising 4516 long (\textit{ch\=oka}) and short (\textit{tanka}) poems. The data is taken from electronic text “\textit{Man’y\=osh\=u Search System}” (Yamaguchi University, Japan) as well as the Oxford Corpus of \ili{Old Japanese} (University of Oxford). For periodization, I follow \citet{Frellesvig2010Japanese}. \ili{Old Japanese} (abbreviated ‘OJ,’ approximately 700–800), Early Middle \ili{Japanese} (‘EMJ’ 800–1200), Late Middle \ili{Japanese} (‘LMJ’ 1200–1600), Early Modern \ili{Japanese} (‘EModJ’ 1600–1800).}


\begin{exe}
\ex%2
\label{14-ya-ex:2}
\langinfo{Old Japanese}{}{MYS 4303; MYS 4191}
\begin{xlist}
\ex
\label{14-ya-ex:2a}
\gll [wa \textbf{ga} sekwo \textbf{ga} yadwo]\\ %(MYS 4303)
I \textsc{gen} lover \textsc{gen} house\\
\glt ‘my lover’s house’

\ex
\label{14-ya-ex:2b}
\gll [\textbf{ayu} \textbf{no} si ga pata]\\ %(MYS 4191)
sweetfish \textsc{gen} it \textsc{gen} fin\\
\glt ‘sweetfish’s fins’
\end{xlist}
\end{exe}


\begin{exe}
\ex%3
\label{14-ya-ex:3}
\langinfo{Old Japanese}{}{MYS 2926; MYS 3837; MYS 925}
\begin{xlist}
\ex
\label{14-ya-ex:3a}
\gll [wa ga sekwo \textbf{ga} motomu\textbf{-ru}] omo ni ika-masi mono wo\\ %(MYS 2926) 
I \textsc{gen} lord \textsc{agt} ask-\textsc{adn} nurse \textsc{dat} go-\textsc{aux} thing \textsc{excl}\\
\glt ‘I would go as the wet nurse that my lord asks for.’

\ex
\label{14-ya-ex:3b}
\gll [mizu \textbf{no} tama ni nita-\textbf{ru}] mimu\\ %(MYS 3837) 
water \textsc{gen} pearl \textsc{dat} resemble-\textsc{adn} see\\
\glt ‘(I) see water resembles a pearl.’

\ex
\label{14-ya-ex:3c}
\gll [pisaki Ø\textsubscript{S} opu-\textbf{ru}] kiyoki kapara-ni\\ %(MYS 925) 
catalpa {} grow-\textsc{adn} clear riverbank-on\\
\glt ‘on the banks of the clear river where catalpas grow’
\end{xlist}
\end{exe}


A number of researchers argue that adnominal verb ending \textit{-ru} (with a different set of endings on adjectives and auxiliaries) as in (\ref{14-ya-ex:3a}--\ref{14-ya-ex:3c}) had nominalizing functions (see \citealt{Miyagawa1989Structure,Yanagidaetal2009Word,Robbeets2015Diachrony}).\footnote{\citet{Robbeets2015Diachrony} suggests that the adnominal form \textit{-ru} has undergone a \isi{grammaticalization} from deverbal noun suffix to clausal nominalizer to relativizer and, finally, to finite form.} The subject of a nominalized verb is marked in one of three ways. The semantic difference between \textit{ga} and \textit{no} has been treated in the literature (\cf \citealt{Ohno1977Development}; \citealt{Nomura1993Particles}), but bare subjects as in~\REF{14-ya-ex:3c} have not been integrated into this discussion; they are generally set aside as instances of stylistic case drop. Below I show that the alternation between case-marked and unmarked arguments inOJ \il{Old Japanese} fits into cross-linguistically attested patterns of differential subject marking (DSM). Under this approach, unmarked arguments cannot be viewed as mere stylistic case drop, but they have both syntactic and semantic significance. 

The paper is organized as follows. \sectref{14-ya-sec:2} briefly discusses the general approach to DSM which I adopt: DSM is realized through the interaction of three distinct levels: (i) \isi{argument structure}, (ii) syntax and (iii) PF (morphological spell-out), as proposed by \citet{Woolford2008Differential}. In \sectref{14-ya-sec:3}, I argue that \textit{ga} and \textit{no}, – each functioning in opposition to the zero form – are associated with different levels of DSM: \textit{ga} is a morphological realization of \isi{active} case assigned to an external argument within the \textit{v}P phase. It follows independently motivated PF constraints relatable to \citegen{Silverstein1976Hierarchy} nominal hierarchy. Genitive \textit{no} is assigned to any NPs in the CP phase, where they receive specific interpretations. \sectref{14-ya-sec:4} examines three possible scenarios for the loss of DSM, which occurred in Early Middle \ili{Japanese} (EMJ; 800–1200). I argue that the development of nominative \textit{ga} results from the reanalysis of psych transitive predicates as \isi{intransitive} taking a single theme argument. The present study suggests that the loss of DSM cannot be interpreted as a simple, one-step shift in alignment or \isi{case marking}, as such changes are sometimes presented in work on diachronic syntax (\cf \citealt{Harrisetal1995Historical}). Instead, the morphological features of individual case markers change incrementally over time, only after time giving rise to global changes in the overall system.


\section{Differential Subject Marking (DSM)}\label{14-ya-sec:2}

I assume with \citet{Woolford2008Differential} that DSM effects are associated with three distinct grammatical levels. The first level of DSM is closely linked to \textit{${\theta}$} role assignment (canonically, Agent) to subjects, and to contexts where inherent (or non-structural) Case is assigned to external arguments. This level of DSM is identified as the \isi{argument structure} (or \textit{v}P phase), which corresponds to the representational level of D-structure in the government-binding theory of \citet{Chomsky1981Lectures}. The second level of DSM is associated with syntax above \textit{v}P. It behaves in parallel to differential object marking (DOM) in that case alternation depends on the syntactic position of the subject: often, subject or object arguments which move outside \textit{v}P are morphologically marked (by an affix or by triggering agreement) and assigned language particular interpretative properties, such as specificity, \isi{definiteness}, \isi{animacy}, etc. (\cf \citealt{Diesing1992Indefinites}, \citealt{Chomsky2001Minimalist}). The third level of DSM involves post-syntactic PF constraints; this is the level at which abstract case features are spelled out morphologically. According to \citet{Woolford2008Differential}, DSM at this level involves \isi{markedness}, which she defines in relation to \citegen{Silverstein1976Hierarchy} \citeyear{Silverstein1976Hierarchy} nominal hierarchy. Cases at the more marked end of the hierarchy are more likely to be morphologically marked. 

In both the typological and theoretical literature, \isi{active} alignment is often classified as a subtype of ergative (\cf \citealt{Comrie1973Ergative,Comrie1978Ergativity,Silverstein1976Hierarchy,Bittneretal1996Structural}). Active, however, differs crucially from ergative alignment in that \isi{transitivity} plays no role. In \ili{Hindi}, for example, the case marker \textit{-ne} appears on the agent subject of both transitive \REF{14-ya-ex:4a} and unergative \isi{intransitive} verbs \REF{14-ya-ex:4b}, while the theme subject of unaccusatives \REF{14-ya-ex:4c} is unmarked:

\newpage 
\begin{exe}
\ex%4
\label{14-ya-ex:4}
\langinfo{Hindi}{Indo-Aryan}{\citealt[71, 107]{Mohanan1994Argument}}
\begin{xlist}
\ex
\label{14-ya-ex:4a}
\gll raam-ne lakdii kaatii\\
Ram-\textsc{erg} wood.\textsc{nom} cut.\textsc{perf}.\textsc{f}\\
\glt ‘Ram cut wood.’

\ex
\label{14-ya-ex:4b}
\gll raam-ne nahaayaa\\
Ram-\textsc{erg} bath.\textsc{perf}\\
\glt ‘Ram bathed.’

\ex
\label{14-ya-ex:4c}
\gll raam (*-ne) giraa\\
Ram (*-\textsc{erg}) fall.\textsc{perf}\\
\glt ‘Ram fell.’
\end{xlist}
\end{exe}

According to \citet{Woolford1997Four,Woolford2008Differential}, DSM effects in \ili{Hindi} are determined at \isi{argument structure}. The external argument (\textsc{agent}) is \textit{$\theta $}-marked and at the same time inherently case-assigned by \textit{v} in a \textit{v}P projection above VP, as represented in~\REF{14-ya-ex:5}.\footnote{The descriptive generalization that supports the view that ergative is an inherent case comes from the fact that ergative subjects in some instances occur in non-finite clauses while structural nominative subjects cannot (\cf \citealt{Legate2002Warlpiri,Aldridge2004Ergativity}). Derived subjects are never ergative; that is, no language promotes objects to ergative through operations such as raising or passive. A reviewer points out that this fact may have a functional explanation, but the structural consequence remains the same: ergatives are assigned inherent case.} 

\begin{exe}
\ex%5
\label{14-ya-ex:5}
\emph{DSM at argument structure}\\
\begin{forest} nice empty nodes
	[\textit{v}P
	  [external\\argument] [\textit{v}'
		 [\textit{v}\\{[}+Agt{]}] [VP [] []]
	  ]
	]
\end{forest}	
\end{exe}

The analysis of ergative (or \isi{active}) as inherent case assigned to the external argument in the specifier position of \textit{v}P originates with \citet{Woolford1997Four} and is shared by many researchers such as \citet{Legate2002Warlpiri,Legate2008Morphological,Aldridge2004Ergativity,Aldridge2008Generative} and \citet{Anandetal2006Locus}. I maintain that while ergative is assigned to the external argument in the specifier position of [+transitive] \textit{v}, \isi{active} is assigned to the external argument in the specifier of [+Agent]\textit{v} \citep{Yanagidaetal2009Word}. 


\section{Two Types of DSM in OJ}\label{14-ya-sec:3}

\subsection{DSM: \textit{ga \vs zero}}
\label{14-sec:3.1}
%Yuko: I changed the subtitle 3.1 from Active Alignment to DSM.


\citet{Yanagida2007Jdaigo} and  \citet{Yanagidaetal2009Word} argue that while inOJ \il{Old Japanese} main declarative clauses have a nominative-accusative pattern: the subjects of both transitive and \isi{intransitive} verbs are morphologically unmarked. Various types of embedded or nominalized clauses, exemplified by the adnominal clauses \REF{14-ya-ex:3} and \REF{14-ya-ex:6}, show \isi{active} alignment.\footnote{Main declarative clauses and embedded clauses selected by the cognitive/speech verb such as \textit{ip-} ‘say’ or \textit{omop-} ‘think’, appear with the verb in the \textit{sh\=usikei} ‘conclusive form’ V\textit{-u}, with a different set of endings on adjectives and auxiliaries. In conclusive clauses, both subject and object are morphologically unmarked. The subject is never marked by \textit{no} or \textit{ga}.} 

\begin{exe}
\ex Adnominal clauses: \label{14-ya-ex:6}
\langinfo{Old Japanese}{}{MYS 868; MYS 3443; MYS 925} 
\begin{xlist}
\ex
\label{14-ya-ex:6a}
\gll [Saywopimye no kwo \textbf{ga} pire Ø puri-si] yama\\ %(MYS 868) 
Sayohime \textsc{gen} child \textsc{agt} scarf {} wave-\textsc{pst}.\textsc{adn} mountain\\
\glt ‘the mountain where the child Sayohime waved her scarf’

\ex
\label{14-ya-ex:6b}
\gll [wa \textbf{ga} yuku] miti ni\\ %(MYS 3443)
\textsc{1p} \textsc{agt} go.\textsc{adn} road \textsc{loc}\\
\glt ‘…on the road I travel.’

\ex
\label{14-ya-ex:6c}
\gll [pisakwi Ø opu-ru] kiywoki kapara\\ %(MYS 925)
catalpa {} grow-\textsc{adn} clear riverbank\\
\glt ‘the banks of the clear river where catalpas grow’
\end{xlist}
\end{exe}

As we see in~\REF{14-ya-ex:6}, the subjects of \isi{intransitive} verbs display two distinct patterns; the agent subjects of the transitive and \isi{active} \isi{intransitive} verbs \REF{14-ya-ex:6a}–\REF{14-ya-ex:6b} are marked by \textit{ga}, but the patient subject of the inactive \isi{intransitive} \REF{14-ya-ex:6c} is morphologically unmarked in the same way as the transitive object in~\REF{14-ya-ex:6a}. 

OJ behaves in parallel to \ili{Hindi} in that morphological case appears on agent subjects, but theme subjects of unaccusatives are zero marked. OJ, however, differs crucially from \ili{Hindi} in that it displays a nominal-based split. Nominal based split ergative languages show an ergative pattern with some NPs, and a nominative pattern with others. This interacts with \citegen{Silverstein1976Hierarchy} nominal hierarchy \REF{14-ya-ex:7}. Silverstein’s nominal hierarchy, as is well known, references the feature specification of noun phrases and makes crucial use of featural \isi{markedness}. Pronouns are specified for [person (+ego, 1P)/(+tu, 2P)], [±number], [±gender], etc. Noun phrases are specified for [±proper] [±human][±\isi{animacy}] etc.

\begin{exe}
\ex%7
\label{14-ya-ex:7}
\textbf{The Nominal Hierarchy} \citep{Silverstein1976Hierarchy}\\
pronouns > proper nouns > human > \isi{animate} > \isi{inanimate}\\
1st >2nd >3rd person\\
\end{exe}

Nominative in a nominative-accusative system and absolutive in an ergative-absolu\-tive system are unmarked (in terms of \textsc{markedness}), typically phonologically zero. The accusative in the one system and ergative in the other are marked. Silverstein observes that “if the noun phrases of a language have accusative case-marking at a certain plus-value of a feature [Fi], and ergative case-marking for [-Fi], then noun phrases are accusative for all features above [Fi] in the hierarchy and ergative for all feature below [Fi] in the hierarchy” \citep[123]{Silverstein1976Hierarchy}. \citet[86--87]{Dixon1979Ergativity} interprets the hierarchy to “roughly indicate the overall \textit{agency potential} of any given NP”, and observes that a number of languages have split \isi{case marking} exactly on this principle. 

\citet{Woolford2008Differential}, whom I follow in the discussion below, argues that \textsc{markedness} as expressed in Silverstein’s nominal hierarchy is a PF constraint (to be exact, a constraint on morphological spell-out). PF is the level where “decisions are made concerning the overt realization of (abstract) features from syntax” \citep[29]{Woolford2008Differential}. On this view, nominals lower on the hierarchy are atypical subjects; thus they are marked ergative at PF, while those higher on the hierarchy are atypical objects, and thus they are marked accusative. Nominals that realize typical subject and object grammatical functions are unmarked morphologically. In other words, \isi{ergative case} is assigned to all transitive subjects, but in nominal based split ergative languages, the more marked subjects are those that lie lower on the hierarchy. Accusative, on the other hand, is the mirror image of ergative. The more marked categories for the object are those that lie higher in the hierarchy. 

A split based on the nominal hierarchy is also typical of \isi{active} alignment, but crucially, the nominal hierarchy applies to the argument NPs in the opposite direction as first suggested by \citet{Dahlstrom1983Agent-patient}. As \citet{Mithun1991Active} points out, case markers based on \textit{agency} are frequently restricted to nominals referring to human beings. \citeauthor{Mithun1991Active} identifies the semantic basis of the \isi{active} marking of various non-accusative languages, both synchronically and diachronically. The \isi{active} system in \ili{Batsbi} (\ili{Tsova-Tush}) is limited to first and second persons. Central Pomo has an \isi{active} system in nominals referring to humans only. The Georgian \isi{active} system is restricted to human beings. The \ili{Yuki} system is restricted to animates. From these cross-linguistic observations, the implication follows that \isi{active} marking is exactly the opposite of the right-to-left application of the hierarchy proposed by Silverstein for ergative languages. The relationship between \isi{active} marking and the nominal hierarchy is as stated in~\REF{14-ya-ex:8} (\cf \citealt{Yanagidaetal2009Word}):

\begin{exe}
\ex%8
\label{14-ya-ex:8}
\textbf{The \isi{active} marking hierarchy} (AMH)

In \isi{active} languages, if \isi{active} marking applies to an NP type ${\alpha}$, it applies to
every NP type to the left of ${\alpha}$ on the nominal hierarchy.
\end{exe}

Assignment of \isi{active} case is dependent not just on the \isi{thematic role} assigned by the verb, but on the place of S on the nominal hierarchy. \citet{Klimov1974Character,Klimov1977Tipologija} emphasizes this point, stressing that in \isi{active} languages both the semantics of the predicate and the subject NP govern the distribution of \isi{active} case.

InOJ \il{Old Japanese} the \isi{active} marking appears when the S argument has control over the activity and the inactive pattern appears when control is lacking. Consider \REF{14-ya-ex:9}--\REF{14-ya-ex:10}:

\clearpage
\begin{exe}
\ex%9
\label{14-ya-ex:9}
\langinfo{Old Japanese}{}{MYS 3724; MYS 177; MYS 2991}
\begin{xlist}
\ex
\label{14-ya-ex:9a}
\gll [kimi \textbf{ga} yuk-u] miti no nagate\\ %(MYS 3724)
Lord \textsc{agt} go-\textsc{adn} road \textsc{gen} length\\
\glt ‘the length of the road my lord travels’

\ex
\label{14-ya-ex:9b}
\gll [\textbf{wa} \textbf{ga} naku] namita\\ %(MYS 177)
\textsc{1p} \textsc{agt} cry.\textsc{adn} tear\\
\glt ‘the tears that I cry’

\ex
\label{14-ya-ex:9c}
\gll [\textbf{papa} \textbf{ga} kap-u] kwo\\ %(MYS 2991)
mother \textsc{agt} breed-\textsc{adn} silkworm\\
\glt ‘the silkworms bred by my mother’
\end{xlist}
\end{exe}


\begin{exe}
\ex%10
\label{14-ya-ex:10}
\langinfo{Old Japanese}{}{MYS 2713; MYS 3352; MYS 4318}

\begin{xlist}
\ex
\label{14-ya-ex:10a}
\gll [\textbf{asuka}-\textbf{gapa} Ø yuku] se wo paya-mi\\ %(MYS 2713)
Asuka-river {} go.\textsc{adn} shallows \textsc{obj} fast-\textsc{conj}\\
\glt ‘since the shallows where the Asuka River flows are fast’

\ex
\label{14-ya-ex:10b}
\gll [\textbf{pototogisu} Ø naku] kope\\ %(MYS 3352)
cuckoo (\textsc{agt}) cry.\textsc{adn} call\\
\glt ‘the call of the cuckoo crying’

\ex
\label{14-ya-ex:10c}
\gll [aki no nwo ni \textbf{tuyu} Ø opye-ru pagwi] wo ta-wora-zu\\ %(MYS 4318)
fall \textsc{gen} field \textsc{loc} dew {} cover-\textsc{adn} bush.clover \textsc{obj} hand-break-not\\
\glt ‘without breaking off the dew-laden bush clover in the fall meadow’
\end{xlist}
\end{exe}


The verbs \textit{yuku} `go’ and \textit{naku} ‘cry’ are classified as \isi{active}, more specifically, unergative verbs, and hence the subject NPs are case assigned by \textit{v}[+Agent] (see \REF{14-ya-ex:5} above), but whether the subject NP is morphologically realized depends on the semantic features of the nominals. 
The use of \textit{ga} is obligatory for personal pronouns such as \textit{wa} ‘I’ and \textit{kimi} ‘you/lord’. 
The human NPs higher on the hierarchy are associated with prototypical agents, which express volition and control, whereas the non-human or \isi{inanimate} NPs lower on the hierarchy do not correspond to the \isi{transitivity} prototype. 
This correlates with the fact that transitive subjects are marked by \textit{ga}, but never marked by zero in embedded nominalized clauses in OJ.

The most crucial syntactic property of transitive clauses inOJ \il{Old Japanese} is that \textit{wo}-marked objects necessarily move over the \textit{ga}{}-marked subject, resulting in OSV \isi{word order} \REF{14-ya-ex:11}. 
When objects are unmarked, they have canonical SOV \isi{word order} \REF{14-ya-ex:12} \citep{Yanagida2006Word,Yanagidaetal2009Word}. 
\textit{Wo}{}-marked objects are specific, while zero marked objects are non-specific.\footnote{In \citet{Yanagidaetal2009Word} and \citet{ Frellesvigetal2015Differential,Frellesvigetal2017Diachronic} we argue thatOJ \il{Old Japanese} displays DOM effects associated with specificity (\cf \citealt{Aissen2003Differential}).}


\begin{exe}
\ex \langinfo{Old Japanese}{}{MYS 3669; MYS 3960; MYS 3459}\\ \label{14-ya-ex:11}%11
[Object \textit{wo} Subject \textit{ga} V]:

\begin{xlist}
\ex
\label{14-ya-ex:11a}
\gll \textbf{ware} \textbf{wo} yami ni ya \textbf{imo} \textbf{ga} kwopi-tutu aru ram-u?\\ %(MYS 3669)
I \textsc{obj} dark \textsc{loc} \textsc{q} wife \textsc{agt} long.for-\textsc{cont} be \textsc{aux}-\textsc{adn}\\
\glt ‘Would my wife be longing for me in the dark?

\ex
\label{14-ya-ex:11b}
\gll \textbf{kimi} \textbf{wo} \textbf{aga} mat-an-akuni\\ %(MYS 3960)
lord \textsc{obj} I.\textsc{agt} wait-not-\textsc{nmlz}\\
\glt ‘without me waiting for you’ 

\ex
\label{14-ya-ex:11c}
\gll \textbf{aga} \textbf{te} \textbf{wo} \textbf{tono} \textbf{no} \textbf{wakugwo} \textbf{ga} torite nageka-mu\\ %(MYS 3459)
my hand \textsc{obj} lord \textsc{gen} child \textsc{agt} take weep-\textsc{aux}.\textsc{adn}\\
\glt ‘Will my lord’s child take my hand and weep again tonight?’
\end{xlist}
\end{exe}


\begin{exe}
\ex \langinfo{Old Japanese}{}{MYS 868; MYS 3351}\\%12
\label{14-ya-ex:12}
[Subject \textit{ga} Object Ø V]:
\begin{xlist}
\ex
\label{14-ya-ex:12a}
\gll Saywopimye no kwo ga pire Ø puri-si yama\\ %(MYS 868)
Sayohime \textsc{gen} child \textsc{agt} scarf {} wave-\textsc{pst}.\textsc{adn} mount\\
\glt ‘the mountain where the child Sayohime waved a scarf/did scarf-waving’

\ex
\label{14-ya-ex:12b}
%\langinfo{(Eastern \ili{Old Japanese})}{}{}\\
\gll kanasiki kwo-ro ga ninwo Ø pos-aru kamo\\ %(MYS 3351)
sad child-\textsc{dim} \textsc{agt} cloth {} hang.out-\textsc{adn} \textsc{q}\\
\glt ‘The sad child has hung out a piece of cloth.’ 
\end{xlist}
\end{exe}

Given our assumption that ergative/\isi{active} is assigned by \textit{v} in a \textit{v}P projection \REF{14-ya-ex:5}, the accusative is not licensed inside \textit{v}P; the OSV dominant \isi{word order} is derived by movement of the object to the left peripheral topic position; namely, the specifier of CP, as represented in~\REF{14-ya-ex:13}.

As discussed extensively in \citet{Yanagidaetal2009Word}, when the subject is marked by \textit{ga}, the objects that follow the subject are without exception non-branching noun heads, as in \textit{pire} ‘scarf’ and \textit{ninwo} ‘cloth’ \REF{14-ya-ex:12a}--\REF{14-ya-ex:12b}. These noun heads are syntactically incorporated into the verb.\footnote{ModJ does not have noun incorporation in a strict sense. Noun incorporation discussed by \citet{Kageyama1980Goi} such as \textit{kosi o kakeru} ‘sit a seat’ \vs \textit{kosi-kakeru}, \textit{tema o toru} ‘take time’ \vs \textit{tema-doru} are not productive. These expressions are possibly analyzable as lexical compounds.} Noun incorporation, which is widely observed in ergative languages, is a detransitiving process on a par with antipassives, in that both involve a shift in valency, creating a derived \isi{intransitive} (see \citealt{Baker1988Incorporation}). In other words, the transitive verbs with the object in~\REF{14-ya-ex:12} pattern like unergative intransitives; the subject is marked by \textit{ga}, but the incorporated object is not assigned structural \isi{accusative case} by the verb. 

\ea \label{14-ya-ex:13}
\begin{forest}
 [CP
  [Obj{=}\textit{wo},name=Obj]
  [C'
    [TP
      [\textit{v}P,name=vP
	[Subj{=}\textit{ga}] [\textit{v}'
	  [VP
	    [Obj{=}∅,name=Obj2] [V]
	  ] [\textit{v}\textsubscript{{[}+Agt{]}},name=agt]
	]
      ] [T]
    ]
    [C]
  ]
 ]
\draw[-{Stealth[]}] (Obj2) to [bend left=45] (Obj);
\node [draw, fit={(vP) (Obj2) (agt)},label=right:{DSM at Argument Structure}]  {};
\end{forest}
\z
%TODO@LSP: The author says "please move v to the right and place it right under the tree." (see the word file)

In this section, I have proposed that the alternation between \textit{ga} and zero, as illustrated in \tabref{14-ya-tab:1},\footnote{As noted above, \isi{active} marking is sensitive not only to the semantics of NPs but also to the semantics of predicates. The subjects of transitive verbs and \isi{active} \isi{intransitive} verbs are necessarily marked by \textit{ga} (or \textit{no}), but never by zero. (See \sectref{14-sec:3.3} for \textit{no}.)} arises within a smaller domain of a nominalized clause, namely \textit{v}P \REF{14-ya-ex:13}.\footnote{In \sectref{14-sec:3.3}, we discuss the other type of DSM which arises in a higher domain of nominalized clauses; namely CP phase.}

\begin{table}
\caption{DSM \textit{ga} \vs zero in OJ}\label{14-ya-tab:1}

\begin{tabularx}{\textwidth}{XXX} 
\lsptoprule
& Active & Inactive\\
\midrule
Subject & ga & Ø\\
Object & Ø &\\
\lspbottomrule
\end{tabularx}
\end{table}

The external argument is assigned \isi{active} case by \textit{v}\textsubscript{[+Agt]}, in the same way as \ili{Hindi}. OJ, however, displays \citegen{Woolford2008Differential} third level of DSM effects. The actual exponence or marking of the feature [+Agent] is independently determined by language particular PF constraints, relatable to \citegen{Silverstein1976Hierarchy} nominal hierarchy. Subject NPs higher on the nominal hierarchy appear with \isi{active} predicates, and NPs lower on the hierarchy appear with inactive predicates.\footnote{\citet[95–96]{Klimov1977Tipologija} discusses a similar correlation between subject NPs and their predicates in \isi{active} languages.} 


\subsection{Experiencer Predicates}\label{14-sec:3.2}

Ergative (or \isi{active}) languages often mark the subject of an \isi{experiencer} verb with ergative (or \isi{active}) case, treating them like an external argument. This is illustrated by Basque and \ili{Hindi}, respectively in~\REF{14-ya-ex:14}--\REF{14-ya-ex:15}.

\begin{exe}
\ex%14
\label{14-ya-ex:14}
\langinfo{Basque}{isolate}{\citealt[24]{Woolford2008Differential}}\\
\gll Mikel-ek ni haserretu nau.\\
Michael-\textsc{erg} 1\textsc{sg}.\textsc{nom} angry.\textsc{perf} \textsc{aux}\\
\glt ‘Michael angered me.’
\end{exe}

\begin{exe}
\ex%15
\label{14-ya-ex:15}
\langinfo{Hindi}{Indo-Aryan}{\citealt[142]{Mohanan1994Argument}}\\
\gll tusaar-ne vah kahaanii yaad kii\\
Tushar-\textsc{erg} that story.\textsc{nom} memory.\textsc{nom} do.\textsc{perf}\\
\glt ‘Tushar remembered that story.’
\end{exe}

In Basque, the theme argument is marked by \isi{ergative case} \REF{14-ya-ex:14}, while in \ili{Hindi}, the \isi{experiencer} is marked by \isi{ergative case} \REF{14-ya-ex:15}.

\citet{Kikuta2012Jodai} points out thatOJ \il{Old Japanese} \textit{ga} appears on the non-agentive theme subject of \isi{experiencer} verbs, such as \textit{wasur-} ‘forget’ \textit{omop-} ‘think’, \textit{mi-} ‘see’ etc., and that this raises a problem for \citegen{Yanagidaetal2009Word} hypothesis that \textit{ga} is an \isi{active} case. However, all of Kikuta’s examples of these psych verbs with \textit{ga}-marked theme subjects appear with an unspecified first person \isi{experiencer} and a form of the auxiliary \textit{yu} (stem \textit{ye}-), which derives middles, passives, and potentials.\footnote{The
  productive passive auxiliary \textit{-yu} inOJ \il{Old Japanese} appears after the irrealis (\textit{mizenkei}) \textit{a}{}-stem of the verb, as in~\REF{14-ya-ex:16a}. With a small number of verbs such as \textit{omopoyu} in~\REF{14-ya-ex:16b} \textit{-yu} appears after a different stem vowel, probably reflecting an older fossilized pattern. The reviewer pointed out to me that current linguistic scholarship (\cf \citealt{Whitman2008Source,Frellesvig2010Japanese,Robbeets2015Diachrony}) has mostly agreed with \citet{Ohno1953Nihongo} that the \textit{a}-stem of consonant verbs is nothing but a surface stem that diachronically reflects re-segmentation of suffixes in initial *\textit{a}-. With a polysyllabic vowel final stem followed by a polysyllabic vowel initial suffix, we would expect the first vowel to drop, thus *\textit{omop-ayu}. However, the productive medialOJ \il{Old Japanese} \textit{-(a)yu} may have been derived from the copula *\textit{a-} ‘to be’ followed by the original causative/medial *\textit{-yu} \citep{Robbeets2015Diachrony}. Adding \textit{omopo-} and \textit{-yu} would give the expected result.}

\begin{exe}
\ex%16
\label{14-ya-ex:16}
\langinfo{Old Japanese}{}{MYS 4407; MYS 3191}
\begin{xlist}
\ex
\label{14-ya-ex:16a}
\gll \textbf{imo} \textbf{ga} kopisiku wasura-\textbf{ye}-nu-kamo\\ %(MYS 4407)
my.lover \textsc{agt} miss forget-\textsc{mid}-\textsc{neg.adn}-\textsc{q}\\
\glt ‘Did I miss my dear and cannot forget her?’

\ex
\label{14-ya-ex:16b}
\gll yama kopeni-si, \textbf{kimi} \textbf{ga} omop-\textbf{yu}-raku-ni\\ %(MYS 3191)
mountain cross-\textsc{pst} you/lord \textsc{agt} think-\textsc{mid}-\textsc{nmlz}-\textsc{loc}\\
\glt ‘when you came to my mind as I was crossing the mountains’
\end{xlist}
\end{exe}

\textit{-Yu} is arguably related to the acquisitive light verb \textit{u} (stem \textit{E-}) ‘get’, which \citet{Whitman2008Source} proposes as the source of the well-known \isi{transitivity} alterations in \textit{-e-} inOJ \il{Old Japanese} and later stages of the language. \textit{-E} derives both transitives and intransitives, a property of acquisitives such as English auxiliary \textit{get}. If this analysis is correct, \isi{experiencer} middles such as \REF{14-ya-ex:16} may have an original transitive source, \ie ‘my dear got me to forget her’, ‘you got me to think of him’. That is, \REF{14-ya-ex:16} can be analyzed as a causative middle construction; the theme subject serves as the causer argument of the verb +\textit{yu}. A parallel construction can be seen, for example, in \ili{Assamese}, cited by \citet{Woolford2008Differential}, where the theme subject of an \isi{experiencer} verb is marked ergative when the light verb \textit{make/do} is added to the verb: 


\begin{exe}
\ex%17
\label{14-ya-ex:17}
\langinfo{Assamese}{Indo-Aryan}{\citealt{Woolford2008Differential}}
\begin{xlist}
\ex
\label{14-ya-ex:17a}
\gll gan-tu-e xap-tu-k khogal korile\\
song-class-\textsc{erg} snake-\textsc{class}{}-\textsc{dat} anger made/did\\
\glt ‘The song angered the snake.’

\ex
\label{14-ya-ex:17b}
\gll boroxun-e Ram-ok xant korile\\
rain-\textsc{erg} Ram-\textsc{dat} calm made/did\\
\glt ‘The rain calmed Ram.’
\end{xlist}
\end{exe}


The subject is the external argument of the light verb \textit{korile} ‘make/do’ and is assigned ergative in \ili{Assamese}. Facts like these show that languages may differ as to which argument is mapped to the external argument position. The agent subject is invariably an external argument, but in some languages the causer argument of a psych-verb can be an external argument, and thus an agent, marked with ergative. 

In OJ, there are also some instances in which \textit{ga} marks clausal complements of psychological adjectives (or \isi{experiencer} adjectives) that end with \textit{si}, such as \textit{po-si} ‘want’ or \textit{kana-si} ‘sad’ (\textit{-si} may be historically related to the verb \textit{si} ‘do’), as shown in~\REF{14-ya-ex:18}. Importantly, these clausal complements are always marked by \textit{ga} but never marked by \textit{no} or zero. 


\begin{exe}
\ex%18
\label{14-ya-ex:18}
\langinfo{Old Japanese}{}{MYS 4338; MYS 1007}
\begin{xlist}
\ex
\label{14-ya-ex:18a}
\gll [papa wo panarete yuku] \textbf{ga} kana-si sa\\ %(MYS 4338)
mother \textsc{obj} part go.\textsc{adn} \textsc{agt} sad-do \textsc{nmlz}\\
\glt ‘I am sad about parting from mother.’

\ex
\label{14-ya-ex:18b}
\gll [tada pitorigo ni aru] \textbf{ga} kuru-si sa\\ %(MYS 1007)
only one.child \textsc{dat} be.\textsc{adn} \textsc{agt} painful-do \textsc{nmlz}\\
\glt ‘I am pained that I am the only child…’ 
\end{xlist}
\end{exe}

Although the two types of \textit{ga} – the genitive \textit{ga} and \textit{ga} marking the clausal complement of psych adjectives – have been widely recognized, the historical relation between the two has not been examined. 
In~\REF{14-ya-ex:18a}--\REF{14-ya-ex:18b} the theme argument of psych verbs appears in external argument position marked by \textit{ga}, whereas an unspecified (or implicit) \isi{experiencer} is an internal argument identified as first person singular (\ie the speaker).
\REF{14-ya-ex:16a}--\REF{14-ya-ex:16b} are apparently related to \REF{14-ya-ex:18a}--\REF{14-ya-ex:18b} in that they originate from a psych-transitive predicate with an unspecified first person \isi{experiencer} object. Thus, \REF{14-ya-ex:18a} literally means that ‘parting from my mother made me sad’, as represented in~\REF{14-ya-ex:19}.

\begin{exe}
\ex
\label{14-ya-ex:19}
 [ … V.\textsc{adn}] \textbf{ga} [\textsubscript{VP} pro\textsubscript{ [+1\textsc{sg}]} [\textsubscript{AP}…] \textit{si} ‘do’ ]
\end{exe}

The clausal subject in~\REF{14-ya-ex:18}, as in the case of \REF{14-ya-ex:16}, serves as the causer, thus agentive, of the matrix predicate \textit{po-si} ‘do-wanting’, \textit{kana-si} ‘doing sad’. Below in \sectref{14-ya-sec:4}, I will argue that after OJ, this psych transitive construction was reanalyzed as \isi{intransitive}, taking a single theme argument; this was the historical source of nominative \textit{ga}.


\subsection{DSM in syntax}\label{14-sec:3.3}


In \sectref{14-sec:3.1}, I show that DSM effects identified at the \isi{argument structure} within \textit{v}P constitute semantically motivated case alternations between \textit{ga} and zero. In this section, we discuss the other type of DSM associated with the alternation between \textit{no} and zero. The latter type of DSM occurs when the subject NP is located in the position lower on the nominal hierarchy. A primary question to be addressed is: What is the difference between \textit{no}{}-marked NP and z\textit{ero}-marked NPs, given that both appear on the nominals whose semantic features are lower in the hierarchy? Examples \REF{14-ya-ex:20a}--\REF{14-ya-ex:20b} indicate thatOJ \il{Old Japanese} has DSM associated with a specific/non-specific distinction on a par with DSM in \ili{Turkish} and other languages with genitive subjects in nominalized clauses:

\begin{exe}
\ex%20
\label{14-ya-ex:20}
\langinfo{Old Japanese}{}{MYS 4066}
\begin{xlist}
\ex
\label{14-ya-ex:20a}
\gll [u no \textbf{pana} \textbf{no} saku] tukwi tati-nu\\ %(MYS 4066)
deutzia \textsc{gen} flower \textsc{gen} bloom month pass-\textsc{perf}\\
\glt ‘it was the month when the deutzia flower blooms’

\ex
\label{14-ya-ex:20b}
\gll [okitu mo no \textbf{pana} Ø saki-tara]-ba ware ni tuge koso\\
offing seaweed \textsc{gen} flower {} bloom-\textsc{perf}-if I \textsc{dat} tell \textsc{foc}\\
\glt ‘If seaweed flowers were to bloom in the offing, tell me. (But they would not bloom.)’
\end{xlist}
\end{exe}

In~\REF{14-ya-ex:20a} the author composes the song at the sight of the deutzia flower in the garden where the banquet was held, thus referring to a specific entity. In~\REF{14-ya-ex:20b}, on the other hand, the flower in the subjunctive conditional \textit{ba} ‘if’-clause is unambiguously non-specific, since it is not at the sight of the author, nor previously mentioned in the preceding context. 

In \ili{Turkish}, as is well known, subjects of subordinate clauses marked by genitive are always specific, but when the subordinate subject is nominative, that is, zero-marked, its referent is interpreted as non-specific. \citet{Woolford2008Differential} argues that DSM in \ili{Turkish} is determined at the level of syntax. Consider \REF{14-ya-ex:21a}--\REF{14-ya-ex:21c}.

\begin{exe}
\ex%21
\label{14-ya-ex:21}
\langinfo{Turkish}{Turkic}{\citealt{Kornfilt2003Subject}}
\begin{xlist}
\ex \label{14-ya-ex:21a}
\gll [(\textbf{bir)ari}-\textbf{nin} bugün cocug-u sok-tug-un]-u duy-du-m\\
bee-\textsc{gen} today child-\textsc{acc} sting-\textsc{f}.\textsc{nom}-\textsc{3sg}-\textsc{acc} hear-\textsc{pst}-\textsc{1sg}\\
\glt ‘I heard that the bee/a bee (+specific) stung the child today.’ 

\ex
\label{14-ya-ex:21b}
\gll [cocug-u bugün (\textbf{bir)ari} sok-tug-un]-u duy-du-m\\
child-\textsc{acc} today bee sting-\textsc{f}.\textsc{nom}-3\textsc{sg}-\textsc{acc} hear-\textsc{pst}-\textsc{1sg}\\
\glt ‘I heard that today bees/a bee [-specific] stung the child.’

\ex
\label{14-ya-ex:21c}
\gll *[(\textbf{bir)ari} Ø \textbf{cocug}-\textbf{u} bugün sok-tug-un]-u duy-du-m\\
bee {} child-\textsc{acc} today sting-\textsc{f}.\textsc{nom}-3\textsc{sg}-\textsc{acc} hear-\textsc{pst}-1\textsc{sg}\\
\glt ‘I heard that today bees/a bee [-specific] stung the child.’
\end{xlist}
\end{exe}

As originally observed by \citet{Kornfilt2003Subject,Kornfilt2008DOM}, genitive subjects move outside \textit{v}P, thus, appearing before the object \REF{14-ya-ex:21a}. Unmarked nominative subjects in subordination must appear adjacent to the verb, resulting in OSV order \REF{14-ya-ex:21b}--\REF{14-ya-ex:21c}.OJ \il{Old Japanese} \textit{no}-marked \vs zero marked subjects behave exactly like \ili{Turkish}, as evidenced by \REF{14-ya-ex:22a}--\REF{14-ya-ex:22b}. 

\begin{exe}
\ex%22
\label{14-ya-ex:22}
\langinfo{Old Japanese}{}{MYS 3689; MYS 2665}
\begin{xlist}
\ex
\label{14-ya-ex:22a}
\gll ipe \textbf{pito} \textbf{no} idura-to ware \textbf{wo} topa-ba ikani ipa-mu\\ %(MYS 3689)
home someone \textsc{gen} where-that I \textsc{obj} ask-if how say-\textsc{aux}\\
\glt ‘How should (I) say if someone in your family asks me where (you) are? 

\ex
\label{14-ya-ex:22b}
\gll waga kosi \textbf{wo} \textbf{pito} Ø mike-mu kamo\\ %(MYS 2665)
I\textsc{p}.\textsc{agt} coming \textsc{obj} someone {} see-\textsc{fut}.\textsc{adn} Q\\
\glt ‘Would someone see me coming?’
\end{xlist}
\end{exe}


In~\REF{14-ya-ex:22a}, the \textit{no}-marked subject \textit{pito} ‘person’ has a \textsc{specific} reading; it picks out someone in the family member.\footnote{I assume that \textsc{specific} entities presuppose the existence of a set of individuals; the set of individuals is discourse-linked and refers to a previously mentioned set (\cf\citealt{Enc1991Semantics}).} Example \REF{14-ya-ex:22b}, in contrast, has a \textsc{non-specific} reading: the existence of a set of individuals is completely undefined in previous discourse. Subjects marked by \textit{no}, unlike \textit{ga}-marked subjects, can appear preceding the \textit{wo}-marked object. Unmarked subjects, in contrast, appear strictly adjacent to the verb. \citet{Yanagida2007Jdaigo} provides quantitative data for zero-marked subjects in the \textit{Man’y\=osh\=u}. For a total of 667 zero-marked subjects found in \textit{Man’y\=osh\=u}, 580 occur immediately adjacent to the verb and 9 instances of non-conclusive transitive clauses have the pattern [Object wo Subject Ø V], given in~\REF{14-ya-ex:22b}. These examples, however, without exception, appear in main clauses \citep[183]{Yanagida2007Jdaigo}. Transitive subjects are never marked zero in embedded clauses.\footnote{As noted above,OJ \il{Old Japanese} displays main/embedded split case systems. In main clauses, the subjectYanagida2007Jdaigos of both transitive and \isi{intransitive} verbs are marked by zero.}

The \isi{word order} facts indicate thatOJ \il{Old Japanese} nominalized clauses employ DSM in parallel to DOM associated with a specific/non-specific distinction. They are configurationally determined in the syntax. While the zero-marked subject of transitive verbs remains in the external argument position, namely the specifier of \textit{v}P, the subject marked by genitive moves to the specifier of TP. This is represented in~\REF{14-ya-ex:23}.

\ea \label{14-ya-ex:23}
\begin{forest}
 [CP,name=cp
  [TP
    [Subj {[}\textsc{gen}{]},name=gen] [T'
      [\textit{v}P
	[Subj{[}∅{]},name=subj] [\textit{v}'
	  [VP
	    [Obj] [V]
	  ] [\textit{v}\textsubscript{{[}\textsc{+acc}{]}}]
	 ]
      ] [T]
    ]
  ] [C\textsubscript{{[}+N\textsc{mlz}{]}},name=c]
 ]
\draw[-{Stealth[]}] (subj) to [bend left=30] (gen);
\node [draw, fit={(cp) (subj) (gen) (c)},label=right:{DSM in Syntax}]  {};
\end{forest}
\z

The genitive subject construction \REF{14-ya-ex:23} has a nominative-accusative pattern; the genitive subject is case-licensed by C\textsubscript{[+}\textsc{\textsubscript{nmlz}}\textsubscript{]}, and the accusative object is case-licensed by~\textit{v}. 

\section{The historical development of nominative \textit{ga}} \label{14-ya-sec:4}

It is well known that \textit{ga} in both possessor and subject/agent marking functions drastically decreased after OJ. The ratios between \textit{ga} and \textit{no} in the \textit{Man’y\=osh\=u} (OJ; 8th century) and in \textit{Genji monogatari} (EMJ; 11th century) taken from the Corpus of Historical \ili{Japanese} (CHJ) produced by the National Institute of \ili{Japanese} Language and Linguistics (NINJAL) are given below:\footnote{In \tabref{14-ya-tab:3}, the quantitative data taken from the corpus is limited to the sequence of Noun+\textit{ga/no} Verb (Subject), Noun+\textit{ga/no}+Noun (Possessor), and Adnominal Clause+\textit{ga/no} +Verb respectively, due to the design of the corpus. It is therefore not precisely the total occurrence of \textit{ga/no} in the subject/possessor/clausal patterns.} 

\begin{table}
\caption{The ratios between \textit{ga} and \textit{no} in the \textit{Man’y\=osh\=u} \citep{Koji1988Many}}\label{14-ya-tab:2}
\begin{tabularx}{\textwidth}{XXX} 
\lsptoprule
& \textit{=ga} & \textit{=no}\\
\midrule
Subject & 372 (48\%) & 411 (52\%)\\
Possessor & 606 (10\%) & 5711 (90\%)\\
Clausal subject & 19 (100\%) & 0\\
\lspbottomrule
\end{tabularx}
\end{table}

\begin{table}
\caption{The ratios between \textit{ga} and \textit{no} in (\textit{Genji}, ca. 1010, CHJ)}\label{14-ya-tab:3}
\begin{tabularx}{\textwidth}{XXX} 
\lsptoprule
& \textit{=ga} & \textit{=no}\\
\midrule
Subject & 57 (4\%) & 1358 (96\%)\\
Possessor & 78 (0.7\%) & 11302 (99.3\%)\\
Clausal subject & 261 (98\%) & 4 (2\%)\\
\lspbottomrule
\end{tabularx}
\end{table}

These two tables indicate that \textit{ga} in both subject and possessor functions was significantly reduced in \textit{Genji monogatari}, written in the EMJ period. In \textit{Genji}, 39 out of 57 tokens of \textit{ga}-marked subjects are personal pronouns, of which 24 are first person \textit{waga}, which was already the lexicalized first person pronominal form for both possessor and subject. In contrast, instances of \textit{ga} marking clausal subjects which select psych-predicates, as illustrated in~\REF{14-ya-ex:18}, drastically increases after OJ.\footnote{The CHJ corpus is not designed to make distinctions between clause types. However, it is well known among traditional \ili{Japanese} grammarians that the subject marker \textit{ga/no} is restricted to what \citet{Yanagidaetal2009Word} identified as nominalized clauses inOJ \il{Old Japanese} and EMJ. While \textit{no} remains genitive marker throughout the history, \textit{ga} started to mark the subject in main clauses in Late Middle \ili{Japanese} (see \tabref{14-ya-tab:5} cited from \citealt{Yamada2000Expansion}). By this period, the adnominal endings have been reanalyzed as matrix clause endings.} 

A further significant change in EMJ is that the OSV dominant order associated with \textit{ga} was completely lost. This change directly results from the fact that transitive subjects came to be either zero-marked or marked by genitive \textit{no} as in~\REF{14-ya-ex:24}, resulting in [S (\textit{no}) O \textit{wo} V] basic \isi{word order}, as represented in~\REF{14-ya-ex:23}:

\begin{exe}
\ex%24
\label{14-ya-ex:24}
\langinfo{Early Middle Japanese}{}{Papakigi; Genji}\\
\gll [ki no miti no takumi] \textbf{no} yorodu no mono \textbf{wo} tukuri idasu mo\\
wood \textsc{gen} tool \textsc{gen} craftsman \textsc{gen} various \textsc{gen} thing \textsc{obj} make out \textsc{excl}\\
\glt ‘The craftsman invents various things.’ 
\end{exe}

These observations suggest that EMJ is characterized as displaying the transition from an \isi{active} system to an accusative system. In the following (\sectref{14-ya-sec:4.1}--\sectref{14-ya-sec:4.3}), I will discuss three possible scenarios for this shift in alignment in the history of \ili{Japanese}.

\subsection{Scenario 1: Antipassive > Accusative}
\label{14-ya-sec:4.1}

A number of researchers propose that alignment change from ergative/\isi{active} to accusative arises as a result of reanalysis of antipassives (\cf \citealt{Harrisetal1995Historical,Bittneretal1996Structural,Aldridge2012Antipassive}).\footnote{In antipassives, the external argument has absolutive status rather than ergative, while the notional object is either dropped or marked as an oblique.} The transition from ergative to accusative begins when the oblique object in antipassives is reanalyzed as accusative. This explanation for alignment change may be applicable to ergative languages that have antipassive constructions. Not all languages do, of course: \citet{Polinsky2013Antipassive} and \citet{Comrie2013Alignment} identify 14 ergative and 2 \isi{active} languages with no antipassives.OJ \il{Old Japanese} had no antipassives. Thus the reanalysis of antipassives is not a possible diachronic pathway from non-accusative to accusative for \ili{Japanese}.

\subsection{Scenario 2: Active > Nominative}\label{14-ya-sec:4.2}

\citet[258]{Harrisetal1995Historical} describe as a possible but hypothetical change a shift from \isi{active} to \isi{accusative alignment} caused by reanalysis of an \isi{active} case marker as nominative.\footnote{\citet{Klimov1974Character,Klimov1977Tipologija} also suggests that the development from \isi{active} into nominative is a widespread development.} \citet{King1988Korean} suggests a somewhat similar hypothesis on the basis of the view that the \ili{Korean} nominative marker \textit{-i} was originally an ergative marker that underwent a shift to nominative, as shown in \tabref{14-ya-tab:4}. King hypothesizes that \textit{-i} originates as an \isi{ergative case} and the nominative function of \textit{-i} arises as a result of ergative \textit{-i} coming to mark \isi{intransitive} subjects.


\begin{table}
\caption{Alignment change in Korean \citep{King1988Korean}}
\label{14-ya-tab:4}

\begin{tabularx}{\textwidth}{lQQQ}
\lsptoprule
& Direct Object & Subject Intransitive & Subject
Transitive\\
\midrule
Before change: Ergative & Ø & Ø & \textit{-i}\\
After change: Accusative & Ø/\textit{-l} & Ø/\textit{-i} & Ø/\textit{-i}\\
\lspbottomrule
\end{tabularx}
\end{table}

\citet{Whitmanetal2015Korean} show that King’s hypothesis is not supported by the \ili{Korean} data. In the case of \ili{Japanese}, ModJ nominative \textit{ga} does not directly descend fromOJ \il{Old Japanese} genitive \textit{ga} used to mark \isi{active} subjects. \textit{Ga} became highly infrequent as an NP subject marker in EMJ around the 9–10th centuries. 

\citet{Yamada2000Expansion} examines the reappearance of \textit{ga} as nominative in the text known as the \textit{Amakusa Heike}, which was published in the late 16th century.\footnote{The \textit{Amakusa Heike} is a romanized version of the \textit{Heike} \textit{Monogatari}. It was composed as a textbook to teach \ili{Japanese} to foreign missionaries.} 
\tabref{14-ya-tab:5}, cited from \citet{Yamada2000Expansion}, shows that while subject marker \textit{ga} was restricted to embedded clauses inOJ \il{Old Japanese} and EMJ, it started to reappear in main clauses in Late Middle \ili{Japanese} (LMJ). 

\begin{table}
\caption{\textit{Ga} in main clauses (Amakusa Heike 1592, \citealt{Yamada2000Expansion}).}\label{14-ya-tab:5} 
\begin{tabularx}{\textwidth}{lXXXXlX}
\lsptoprule
& Genitive & transitive & unergative & adjective & unaccusative & total\\
\midrule
\textit{ga} & 0 (0\%) & 2 (2\%) & 13  (16\%) & 15 (18\%) & 54 (64\%) & 84 (100\%)\\
\lspbottomrule
\end{tabularx}
\end{table}

According to Yamada, nominative \textit{ga} in LMJ starts out as a marker for the subject of \isi{intransitive} verbs, in particular, unaccusative verbs, and rarely marks the subject of transitive verbs. \textit{Ga} appears on transitive subjects after the mid 17th century. \tabref{14-ya-tab:6} presents the ratios between \textit{ga} and \textit{no} in the \textit{Toraakira-bon Kyogen} published in 1642. 

\begin{table}
\caption{ the ratios between \textit{ga} and \textit{no} (\textit{Toraakira bon}, 1642, CHJ)} \label{14-ya-tab:6}
\begin{tabularx}{\textwidth}{XXX} 
\lsptoprule
& \textit{=ga} & \textit{=no}\\
\midrule
Subject & 1622 (76\%) & 503 (24\%)\\
Possessor & 353 (7\%) & 5267 (93\%)\\
clausal subject & 20 (100\%) & 0 (0\%)\\
\lspbottomrule
\end{tabularx}
\end{table}

The data in the \textit{Toraakira bon} reveal that transitive clauses came to appear in the canonical [S \textit{ga} O \textit{o} V] pattern in EModJ (1600–1800), as shown by the data in~\REF{14-ya-ex:25}:

\begin{exe}
\ex%25
\label{14-ya-ex:25}
\langinfo{Early Modern Japanese}{}{\textit{Toraakira bon}, 1642}\\
\gll ano mono \textbf{ga} orusu \textbf{o} itase-ba\\
that person \textsc{nom} watch.house \textsc{acc} do-if\\
\glt ‘if that person watches over the house…’ 
\end{exe}

These facts raise a basic question concerning the assumption that case systems shift from \isi{active} to accusative: IfOJ \il{Old Japanese} \isi{active} \textit{ga} is the ancestor of ModJ nominative \textit{ga}, why did \textit{ga} decrease drastically in frequency in EMJ only to reappear in unaccusative rather than transitive verbs.

To account for these facts, I propose a third scenario; that is, a global shift from \isi{active} to nominative never took place in \ili{Japanese}. Instead, change in the semantic features of individual case markers, \textit{ga} and \textit{wo}, reorganized the overall grammatical structure of the language. 


\subsection{Scenario 3: Impersonal psych transitive > Intransitive}
\label{14-ya-sec:4.3}

\ili{Japanese} is a so-called pro-drop language throughout its history; sentences often contain no overt subject. This means that learners ofOJ \il{Old Japanese} were presented with scant evidence that the object moved to the left of the subject, since direct evidence for OSV would be available only in sentences with overt subjects. As a result, object movement was eventually lost. 
The loss of object movement then results in a reanalysis of \textit{wo} as a pure structural \isi{accusative case}.\footnote{\citetv{Frellesvigetal2017Diachronic} 
argue that DOM is no longer operative in EMJ. In EMJ, \textit{wo} was established as the structural \isi{accusative case}. Its range of use was expanded to mark direct objects even with non-specific reading. Because of this change, the division between \textit{wo} marked objects and unmarked objects became semantically opaque.} The reanalysis of \textit{wo} subsequently led to another change. That is, \textit{ga}-marked subjects were unable to remain in the specifier of \textit{v}P. \citet{YanigadaDifferential} proposes that this is attributable to the subject \textit{in-situ} generalization (SSG), originally proposed by \citet{Alexiadouetal2001Subject-in-situ}. The SSG is analyzed as the general condition on structural case, which states that if two DP arguments are merged in the \textit{v}P domain, at least one of them must externalize. \citet{Alexiadouetal2001Subject-in-situ} argue that the SSG applies synchronically in a variety of constructions across languages. I suggest that the SSG provides a diachronic explanation for the loss of \textit{ga} marked subjects of transitive verbs. That is, once \textit{wo} was reanalyzed as structural accusative and the object remained inside \textit{v}P domain, the subject was no longer able to stay in the specifier of \textit{v}P; it had to move outside \textit{v}P. This results in the dramatic increase in tokens of the [DP \textit{no} DP \textit{wo} V] construction \REF{14-ya-ex:23} in EMJ.

Recall that \REF{14-ya-ex:26} is the impersonal psych transitive construction that involves an implicit first person \isi{experiencer} object.

\begin{exe}
\ex%26
\label{14-ya-ex:26}
\langinfo{Old Japanese}{}{MYS 4338}\\
\gll [papa wo panarete yuku] \textbf{ga} kana-si sa\\ %(MYS 4338)
mother \textsc{obj} part go.\textsc{adn} \textsc{agt} sad-do \textsc{nmlz}\\
\glt ‘I am sad about parting from Mother.’
\end{exe}

As shown in \tabref{14-ya-tab:3} above, examples like \REF{14-ya-ex:26} significantly increased in frequency after OJ. Some examples are given in~\REF{14-ya-ex:27} cited by \citet[142]{Ohno1977Development}. \citet{Ohno1977Development,Ohno1987Bunpo} observes that in EMJ, adnominal clauses marked by \textit{ga} are used predominantly with psych predicates with a first person \isi{experiencer} \REF{14-ya-ex:27a}, as is the case in OJ, but that they began to appear with non-psych \isi{intransitive} verbs \REF{14-ya-ex:27b}. 

\begin{exe}
\ex%27
\label{14-ya-ex:27}
\langinfo{Early Middle Japanese}{}{Kocho/Genji, Usugumo/Genji}\\ 
\begin{xlist}
\ex
\label{14-ya-ex:27a}
\gll [kokorobape wo mi-ru] \textbf{ga} wokasi-u mo\\ %(Kocho, Genji)
kindness \textsc{acc} see-\textsc{adn} \textsc{agt} thankful-\textsc{concl} \textsc{excl}\\
\glt ‘Seeing (someone’s) kindness makes (me) thankful.’

\ex
\label{14-ya-ex:27b}
\gll [kumo no usuku watare-ru] \textbf{ga} nibi iro na-ru wo\\ %(Usugumo, Genji)
cloud \textsc{gen} shallow.pass away-\textsc{adn} \textsc{agt} red color become-\textsc{adn} \textsc{excl}\\
\glt ‘the clouds passing thinly away become red’
\end{xlist}
\end{exe}


In~\REF{14-ya-ex:27b} the adnominal clause marked by \textit{ga} is the subject of a non-psych \isi{intransitive} verb, and it involves no implicit first person \isi{experiencer}. A further change in EMJ is that while this psych predicate construction was used only in nominalized clauses in OJ, it came to appear in non-nominalized main clauses as in~\REF{14-ya-ex:27a}. Based on MJ (800–1600) data, I hypothesize that ModJ nominative \textit{ga} is descended from \textit{ga} marking the clausal complements of psychological predicates. Following Ohno's (\citeyear{Ohno1977Development,Ohno1987Bunpo}) observations and data collected from the corpus, nominative \textit{ga} developed as a result of a reanalysis of impersonal psych-transitive as unaccusative \isi{intransitive} where the \textit{ga} marked argument came to be the sole argument of the predicate, that is, nominative. \textit{Ga} reappeared in LMJ as a nominative postposition, marking the theme argument of intransitives, and it was extended to mark the subjects of transitive verbs in EModJ. This scenario gives a straightforward explanation for why nominative \textit{ga} started to mark the subject of \isi{intransitive} verbs, as observed by \citet{Yamada2000Expansion}.

\section{Summary}
\label{14-ya-sec:5}

I have argued that the semantic opposition between case marked \vs zero marked subjects inOJ \il{Old Japanese} nominalized clauses show two types of DSM effects which fit with well-established cross-linguistic patterns. I have also argued that the reanalysis of \textit{wo} as structural accusative is a direct cause of the loss of \isi{active} \textit{ga} marking the subject of transitive verbs. The quantitative data in EMJ and LMJ suggest that nominative \textit{ga} emerges as a result of a reanalysis of psych-transitive predicates as \isi{intransitive} where the \textit{ga} marked argument is the sole argument of the predicate. It has been widely believed that case systems change from non-accusative to accusative or accusative to non-\isi{accusative alignment}. TheOJ \il{Old Japanese} data support the view that case systems do not merely shift from one alignment to another due to a single change. Instead, a cascade of changes in the morphological/semantic features of individual case markers, as exemplified byOJ \il{Old Japanese} and EMJ \textit{ga} and \textit{wo}, occur over time, eventually leading to overall change of \isi{case marking} systems in a given language. 


\section*{Digitized texts}
\label{14-ya-sec:Dig}
\begin{itemize}
\item The Japanese Historical Corpus, the National Institute of Japanese Language and Linguistics, \url{https://maro.ninjal.ac.jp/}
\item The Oxford Corpus of Old Japanese, \url{http://vsarpj.orinst.ox.ac.uk/corpus/}
\item \textit{Man’yôshû} Kensaku, Yamaguchi University\\ \url{http://infux03.inf.edu.yamaguchi-u.ac.jp/~manyou/ver2_2/manyou.php}
\end{itemize}

\largerpage[2]
\section*{Abbreviations}
\begin{tabularx}{.45\textwidth}{lQ}
\textsc{abs} &  absolutive \\%
\textsc{acc} &  accusative \\%
\textsc{adn} &  adnominal \\%
\textsc{agt} &  agent \\
\textsc{asp} &  aspect \\
\textsc{aux} &  auxiliary verb \\
\textsc{conc} &  concessive \\
\textsc{concl} &  conclusive \\
\textsc{conj} &  conjunctive \\
\textsc{cont} & continuative\\
\textsc{dat} &  dative \\
\textsc{dim} &  diminutive \\
\textsc{erg} &  ergative \\
\textsc{excl} &  exclamative \\
\textsc{f} &  female \\
\textsc{foc} & focus marker\\
\textsc{fut} &  future \\
\textsc{gen} &  genitive \\
\end{tabularx}
\begin{tabularx}{.5\textwidth}{lQ}
\textsc{hon} &  honorific\\
\textsc{imperf} &  imperfective\\
\textsc{loc} &  locative\\
\textsc{mid} &  middle\\
\textsc{mod} &  modal\\
\textsc{neg} &  negative\\
\textsc{nmlz} &  nominalizer\\
\textsc{nom} &  nominative\\
\textsc{nonfut} &  non-future\\
\textsc{obj} &  object marker\\
\textsc{pst} &  past\\
\textsc{pl} &  plural\\
\textsc{prt} &  second position particle (an evidential)\\
\textsc{perf} &  perfective\\
\textsc{1p} &  first person\\
\textsc{2p} &  second person\\
\textsc{q}& question particle\\ 
\end{tabularx}


{\sloppy
\printbibliography[heading=subbibliography,notkeyword=this] }

\end{document}
