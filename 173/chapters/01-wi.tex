\documentclass[output=paper]{LSP/langsci}
\ChapterDOI{10.5281/zenodo.1228243} 
\author{Alena Witzlack-Makarevich\affiliation{University of Kiel}\lastand Ilja A. Seržant\affiliation{Leipzig University}}
\title{Differential argument marking: Patterns of variation}
%\epigram{Change epigram in chapters/03.tex or remove it there}

%\maketitle

\abstract{In this introductory article we provide an overview of the range of the phenomena that can be referred to as differential argument marking (DAM). 
We begin with an overview of the existing terminology and give a broad definition of the DAM to cover the phenomena discussed in the present volume and in the literature under this heading. 
We then consider various types of the phenomenon which have figured prominently in studies of DAM in various traditions. 
First, we differentiate between arguments of the same predicate form and arguments of different predicate forms. 
Within the first type we discuss DAM systems triggered by inherent lexical argument properties and the ones triggered by non-inherent, discourse-based argument properties, as well as some minor types. 
It is this first type that traditionally constitutes the core of the phenomenon and falls under our narrow definition of DAM. 
The second type of DAM is conditioned by the larger syntactic environment, such as clause properties (\eg main \vs embedded) or properties of the predicate (\eg its TAM characteristics). 
Then, we also discuss the restrictions that may constrain the occurrence of DAM cross-linguistically, other typical features of DAM systems pertaining to the morphological realization (symmetric \vs asymmetric) or to the degree of optionality of DAM. 
Finally, we provide a brief overview over functional explanations of DAM.}

\maketitle

\begin{document} 
\section{Introduction}
\label{01-wi-sec:Introduction}
In this introductory article we provide an overview of the range of phenomena that can be referred to as \textit{differential argument marking} (DAM).\footnote{Both authors contributed equally to the writing of this paper.} 
We begin this introduction with a survey of the existing terminology (this section). 
We then proceed to consider individual aspects of the phenomenon which have figured prominently in studies of DAM in various traditions (\sectref{01-wi-sec:2-Synchronic} and~\sectref{01-wi-sec:3-Morphological}).

The term \textit{differential marking} – or to be historically precise, \textit{differential object marking} (abbreviated as DOM) – was first used by \citet{Bossong1982Praepositionale, Bossong1985Differentielle} in his investigations of the phenomenon in \ili{Sardinian} and New Iranian languages. 
Somewhat older than this term is the term \textit{split} (as in \textit{split ergativity}) used in the line of research focusing primarily on the differential marking of the agent argument. 
It has been in use since \citet{Silverstein1976Hierarchy} and was popularized by \citet{Dixon1979Ergativity, Dixon1994Ergativity}. 

Recent years have been marked by a growing interest in differential marking, and as a result numerous related terms have been coined to refer to individual roles marked differentially and particular patterns of differential marking. For example, 
\citet{deHoopetal2008Differential} were the first to systematically discuss \textit{differential subject marking} (DSM). 
Here, the syntactic term \textit{subject} was understood rather broadly including different kinds of less canonical, subject-like arguments. 
Later, notions covering more specific argument roles were introduced: \citet{Fauconnier2011Differential} studies \textit{differential agent marking}, whereas \citet{Haspelmath2007Ditransitive} and \citet{Kittila2008Animacy} explore \textit{differential recipient marking} or \textit{differential goal marking}, as well as \textit{differential theme marking}. 
Another notion that is subsumed under DAM is \textit{optional ergative marking} (\cf among others \citealt{McGregor1992Semantics, McGregor1998Optional, McGregor2006Focal, McGregor2010Optional}; \citealt{Meakins2009Case}; \citealt{Gaby2010Discourse}). 
As these and other authors show, in addition to the semantic function of encoding agents, \isi{ergative case} is sometimes also employed to mark focal, unexpected or contrastive agent arguments. 
Finally, \citet{Sinnemki2014Typological} – observing that the term DOM sometimes implies an assumption as to which factors trigger differential marking – introduced the term \textit{restricted case marking (of the object)} to cover all cases of differential marking no matter what the respective factors are. 
Finally, in the traditions of the DAM research in individual language families and languages, many more language-, role- or marking-specific labels have been used, for instance, \textit{prepositional accusative} in Romance linguistics (\eg \citealt{Torrego1999Gramatica}) or \textit{bi-absolutive construction} in the Nakh-Daghestanian languages (\eg \citealt{Forker2012Bi-absolutive}).

The list of terms provided above makes it clear that research on differential marking has focused primarily on arguments. 
However, \isi{differential argument marking} can be viewed as a subtype of a larger phenomenon which manifests itself in a complex interaction between the meaning and function of a particular marking pattern, on the one hand, and some properties of the constituents involved – both arguments and adjuncts –, on the other. 
For instance, the \ili{Persian} marker \textit{-rā} is not only used with direct object NPs but can follow nearly all kinds of constituents except for subject NPs: one finds it marking time-adverbial NPs, objects of prepositions, \etc (\cf various examples in \citealt{Dabir-Moghaddam1992Independence}; for a different example see the discussion of \textit{differential time adverbial marking} in Baltic in \citealt[141--154]{Serzant2016Nominative}). 
Besides, \isi{case marking} needs not be fully paradigmatic and different cases/adpositions impose different selectional restrictions on the type of nominals they can mark. 
These restrictions may potentially create paradigmatic gaps and differential marking with both arguments and adjuncts. 
The main condition for this is the semantic compatibility between the meaning of a particular case/adposition and the nominal \citep{Comrie1986Markedness, Aristar1997Marking, Creisselsetal2011Animacy}. 
For example, \citet{Aristar1997Marking} shows that locational cases/adpositions are often less or zero marked with place names but require a dedicated suffix with other nouns which are less expected to occur in expressions denoting location. 
Similarly, \isi{animacy} is an important factor that decreases the likelihood of such cases as instrumental, ablative or locative to occur. 
Hence, highly \isi{animate} nominals may either not form the locative cases at all or require additional marking. 
In turn, cases/adpositions such as dative or comitative typically require \isi{animate} participants. Having said this, in what follows we will focus on differential marking of arguments primarily for reasons of space.

As is obvious from the plethora of terms listed above, differential marking is a very broad notion that covers a wide range of different phenomena. 
Given that the investigations in the present volume are aimed at diachronic processes we cannot \textit{a priori} focus on a subset of cases for something that we treat here as being in flux, thereby leaving out phenomena that have the potential to develop into DAM in a more accepted sense (or in fact have been attested to undergo this development), as well as those phenomena that arguably originate from DAM but exhibit somewhat deviating properties due to later developments. 
For this reason, we keep the definition of DAM fairly broad. 
We will use the term DAM as defined in~(\ref{01-wi-ex:1:DefinitionsDAM}) (drawing on \citealt{Woolford2008Differential, Iemmoloetal2012Differential}):\footnote{Some authors go even further and consider inverse systems and voice alternations as instances of DAM (\eg \citealt[1]{deHoopetal2008Cross-linguistic}).}
 
\ea\label{01-wi-ex:1:DefinitionsDAM}
Broad definition of DAM:\\
Any kind of situation where an argument of a predicate bearing the same generalized semantic argument role may be coded in different ways, depending on factors other than the argument role itself, and which is not licensed by diathesis alternations.
\z

\noindent It follows from this definition that DAM is not restricted to \isi{case marking} in the broad sense (also called dependent marking or flagging) and subsuming both morphological case and adposition marking (\cf \citealt{Haspelmath2005Argument}), but also includes differential agreement (or head marking or indexing). 
For example, \citet{Iemmolo2011Towards} has introduced the term \textit{differential object indexing} (DOI) to refer to cases of \isi{differential argument marking} on the verb in contrast to differential \isi{case marking} on the noun phrase. 
Whereas some linguists think that the two types of differential marking share commonalities (\eg \citealt[1--2]{Dalrympleetal2011Objects}), others claim that they are different in terms of their functions and triggers and may emerge from different diachronic processes (\citealt[5]{deHoopetal2008Cross-linguistic}; \citealt{Iemmoloetal2012Differential}). 
While we agree with this second view, we are open to the possibility that there might nevertheless be considerable overlap in both \isi{diachrony} and synchrony. 

To capture the different kinds of DAM systems, we put forward a coordinate system in which we highlight the aspects that we consider central for the understanding of DAM and give a narrower definition of DAM in~(\ref{01-wi-ex:16:NarrowDAM}). 
Both definitions will be used in the present volume and, in fact, there is often a diachronic relationship between them.
In what follows we will first provide an overview of the properties staking out the phenomenon of DAM. 

We begin with an overview of the synchronic variation of the phenomenon and first consider the argument-triggered DAM systems (\sectref{01-wi-sec:2.1-Argument-triggered}). In particular, we discuss both inherent lexical argument properties (\sectref{01-wi-sec:2.1.1-Inherent}; \sectref{01-wi-sec:2.1.2-Morphological}) and non-inherent discourse-based argument properties (\sectref{01-wi-sec:2.1.3-Non-inherent-discourse}) and proceed with the properties of the larger syntactic environment (\sectref{01-wi-sec:2.1.5-Scenario-global-local}). 
\sectref{01-wi-sec:2.2-Predicate-triggered} covers DAM cases triggered by various predicate properties. 
\sectref{01-wi-sec:2.3-Summary-triggers} provides a brief summary of the various triggers for DAM. 
in~\sectref{01-wi-sec:2.4-DAM-scope}, we introduce various restrictions that constrain the occurrence of DAM cross-linguistically. 
\sectref{01-wi-sec:3-Morphological} is devoted to realization properties of DAM. 
\sectref{01-wi-sec:3.1-Symmetric} discusses the morphological distinction between symmetric \vs asymmetric DAM types. 
We then contrast different loci of realization of DAM: head-marking and dependent-marking (\sectref{01-wi-sec:3.2-Differential}). 
\sectref{01-wi-sec:3.3-Syntactic} highlights differences in syntactic (behavioral) properties found with DAM. 
The distinction between obligatory \vs optional is introduced in~\sectref{01-wi-sec:3.4-Obligatory}. 
\sectref{01-wi-sec:3.5-Summary} provides a brief summary of the factors involved in variation. 
Finally, we discuss a few functional explanations (\sectref{01-wi-sec:4-Functional}) and conclusions (\sectref{01-wi-sec:5-Conclusions}).


\section{Synchronic variation of DAM}
\label{01-wi-sec:2-Synchronic}
As defined above, DAM encompasses a range of phenomena sharing the trait of encoding the same argument role in different ways. 
However, apart from this shared property DAM systems vary from language to language. 
To allow for the comparison of DAM systems and their diachronic development paths, we decompose the phenomenon into a number of characteristics which build upon the attested \isi{synchronic variation} and suggestions made in the literature on the topic. 

In what follows we introduce two orthogonal distinctions of DAM systems: \textit{argument-triggered DAM} (\sectref{01-wi-sec:2.1-Argument-triggered}) \vs \textit{predicate-triggered DAM} (\sectref{01-wi-sec:2.2-Predicate-triggered}) and \textit{restricted DAM} \vs \textit{unrestricted DAM} (\sectref{01-wi-sec:2.4-DAM-scope}). 
We begin by considering those DAM systems where the \isi{differential argument marking} may be found with one and the same form of the predicate (henceforth: \textit{argument-triggered DAM}). 
For this type of DAM a number of variables are needed to account for the attested variation. 
These are various properties of arguments (\sectref{01-wi-sec:2.1.1-Inherent}–\sectref{01-wi-sec:2.1.3-Non-inherent-discourse}) and event semantics (\sectref{01-wi-sec:2.1.5-Scenario-global-local}).
In~\sectref{01-wi-sec:2.2-Predicate-triggered}, we will turn to predicate-triggered DAM types, all of which have in common that the \isi{differential argument marking} depends on the actual form of the predicate involved. 


\subsection{Argument-triggered DAM}
\label{01-wi-sec:2.1-Argument-triggered}

The properties of arguments can determine DAM in two ways. 
First, the properties of the differentially marked argument alone can be responsible for a particular marking. Second, the properties of more than one argument in a clause, \ie the whole constellation of arguments, also referred to as scenario, can determine a particular marking. 
The first type is discussed in~\sectref{01-wi-sec:2.1.1-Inherent}–\sectref{01-wi-sec:2.1.3-Non-inherent-discourse} and summarized in~\sectref{01-wi-sec:2.1.4-Argument-triggered-Summary}, whereas the second type is considered in~\sectref{01-wi-sec:2.1.5-Scenario-global-local}. 
In both cases, the relevant argument properties include a wide range of inherent lexical (semantic and formal), as well as non-inherent, first of all pragmatic characteristics of arguments. These subtypes are considered in individual subsections. 
We thus follow \citet[159]{Bossong1991Differential} who first made the distinction between inherent and non-inherent properties of the NP in the context of DOM (\cf \citealt[282]{Sinnemki2014Typological}, who distinguishes between referential and discourse properties). 
Inherent properties of arguments (semantic and formal) are considered in~\sectref{01-wi-sec:2.1.1-Inherent}–\sectref{01-wi-sec:2.1.2-Morphological}, non-inherent discourse-based properties are discussed in~\sectref{01-wi-sec:2.1.3-Non-inherent-discourse}. 
Finally, we isolate as a subtype of DAM triggers cases, where argument properties closely linked to the semantics of the respective event determine the type of marking (\sectref{01-wi-sec:2.1.6-Properties-dependent}).

\subsubsection{Inherent lexical argument properties}
\label{01-wi-sec:2.1.1-Inherent}

Many of the properties we cover in this and the following section are often represented as integrated into various implicational hierarchies or scales. 
One of the most cited versions of such hierarchies is given in (\ref{01-wi-ex:2:Hierarchy1}). 
It was introduced by \citet{Dixon1979Ergativity} as \textit{potentiality of agency scale} and was based on \citegen{Silverstein1976Hierarchy} \textit{hierarchy of inherent lexical content}. 
A similar hierarchy was independently introduced by \citet{Moravcsik1978Distribution} as \textit{activity scale}.\footnote{For a more extensive overview of the history of research on the effects of referential hierarchies on differential marking, see \citet{Filimonova2005Noun}.} 
The hierarchy was widely popularized by \citet[130]{Croft2003Typology} as the \textit{extended animacy hierarchy}. 
Other common versions of the hierarchy include \citegen{DeLancey1981Interpretation} \textit{empathy hierarchy} in (\ref{01-wi-ex:3:Hierarchy2}), \citegen{Aissen1999Markedness} \textit{prominence hierarchy} given in~(\ref{01-wi-ex:4:Hierarchy3}), and \textit{indexability hierarchy} in \citet{Bickeletal2007Inflectional}.

\ea\label{01-wi-ex:2:Hierarchy1}	
first person pronoun > second person pronoun > third person pronoun > proper nouns > human common noun > animate common noun > inanimate common noun \citep[85]{Dixon1979Ergativity}
\z

\ea\label{01-wi-ex:3:Hierarchy2}
speech-act-participant (SAP) > 3rd person human > 3rd person > non-human animate > inanimate (adapted from \citealt[627--628]{DeLancey1981Interpretation})
\z

\ea\label{01-wi-ex:4:Hierarchy3}
local person > pronoun 3rd > proper noun 3rd > human 3rd > animate 3rd > inanimate 3rd \citep[674]{Aissen1999Markedness}
\z

\noindent These and similar complex hierarchies involve a range of distinct dimensions, such as \eg person or \isi{animacy} (\cf \citealt[130]{Croft2003Typology}). 
These dimensions may be more or less relevant in shaping DAM systems in individual languages (see \citealt{Aissen1999Markedness} for examples). 
The major reason for the suggestion of extended versions of hierarchies, as in (\ref{01-wi-ex:2:Hierarchy1}) or (\ref{01-wi-ex:3:Hierarchy2}), is the fact that individual dimensions are not entirely orthogonal. 
Personal pronouns are not only inherently animate (except for the third person, \cf English \textit{it}), they are also inherently \isi{definite} and highly accessible referents. Therefore, they are highest ranked also on hierarchies based on \isi{definiteness} (see \sectref{01-wi-sec:2.1.2-Morphological}) and on the accessibility hierarchy (\cf \citealt{Ariel1988Referring, Ariel2001Accessibility}) or in terms of topic-worthiness \citep{Wierzbicka1981Case}. 
On the other hand, some authors (\eg \citealt{Dahl2008Animacy}) argue that complex hierarchies are problematic in many respects and should rather be viewed in terms of a combination of different factors operating simultaneously and not as one, unidimensional factor. 
Thus, though first and second person referents are always \isi{animate}, whereas the third person referents can be both animate and inanimate, there is no reason to regard \isi{animate} third person referents as less \isi{animate} than first and second person referents (\cf \citealt[195]{Comrie1989Language}). 
Analogically, personal pronouns, proper names or \isi{definite} NPs are not distinct in terms of \isi{definiteness} – these NP types are equally \isi{definite} (\cf \citealt[45]{vonHeusingeretal2003Interaction}). 
Several researchers have proposed to decompose the single complex hierarchy into several layers or sub-hierarchies (\cf \citealt[130]{Croft2003Typology}; \citealt[149]{Siewierska2004Person}). 
The advantage of such multi-layered hierarchies is that their sub-hierarchies are logically independent, and each hierarchy may have more or less influence on shaping the grammatical system of an individual language \citep{Haudeetal2016Referential}. 

In what follows we first provide an overview of individual dimensions contributing to the complex hierarchies discussed above and relevant for DAM and then present a few examples. 
We begin this overview with the inherent lexical argument properties which have a semantic component. The relevant dimensions and their levels are listed in \tabref{01-wi-tab:1:semantic}.\footnote{Some authors rank the first and the second persons, \eg \citet[85]{Dixon1979Ergativity} ranks the first person over the second person.} 
These are probably the most frequently discussed factors behind DAM and examples of their effects on case marking or agreement can be easily found in the literature (\eg \citealt{Silverstein1976Hierarchy, Aissen1999Markedness, Dixon1994Ergativity}). 
Note that these dimensions are still inherently complex in the sense that they can be further decomposed into a range of binary features as in \citegen{Silverstein1976Hierarchy} original proposal (\eg [±animate], [±human], [±ego]) or in \citealt[159]{Bossong1991Differential}).

\begin{table}
\begin{tabularx}{\textwidth}{>{\hsize=.4\hsize}X>{\hsize=1.6\hsize}X}
\lsptoprule
 Dimension & Example\\ 
\midrule 
 Person & First \& Second person > Third person > (Obviative / Fourth person) (\cf \citealt[85]{Dixon1979Ergativity}; \citealt[130]{Croft2003Typology})\\ 
 
 Animacy & Humans > Animate non-humans (animals) > Inanimate (\cf \citealt[159]{Bossong1991Differential}; \citealt{Silverstein1976Hierarchy, Aissen2003Differential})\\ 
 Uniqueness & Proper nouns > Common nouns (\eg as part of \citealt[130]{Croft2003Typology})\\ 
 Discreteness & Count nouns > Mass nouns (\cf \citealt[159]{Bossong1991Differential})\\ 
 Number & Singular \vs Plural \vs Dual\\ 
\lspbottomrule
\end{tabularx}
\caption{Inherent semantic argument properties.}
\label{01-wi-tab:1:semantic}
\end{table}

The individual levels in \tabref{01-wi-tab:1:semantic} are ordered – where possible – in an implicational hierarchy. 
With respect to argument marking these hierarchies are meant to reflect either universal constraints on possible splits in alignment of case and agreement and/or the cross-linguistic frequency of actual language types (\cf \citealt[123]{Croft2003Typology}). 
For instance, according to one reading, the types at the top of the hierarchies tend to show \isi{accusative alignment}, whereas the ones at the bottom of the hierarchy tend to align ergatively (\cf \citealt{Silverstein1976Hierarchy}, see also \citealt{Bickeletal2015Typological} for the testing of the effects of various hierarchies on alignment against a large sample of over 370 case systems worldwide).

By listing the dimensions individually in \tabref{01-wi-tab:1:semantic} we do not imply that for each of then there exists a DAM system in which a particular property is the only trigger of DAM. 
Rather, in the vast majority of languages these and further dimensions to be introduced later interact in an intricate fashion. 
For instance, we do not know of any language in which number is the only relevant dimension, but there are many synchronic cases in which a combination of person and number provides an exact characterization of the split in marking, which is particularly common within pronouns (see \citealt{Bickeletal2015Typological} for examples). 
Number is also known to play a role in the diachrony of DAM. 
For instance, in \ili{Old Russian} primarily animacy-driven DOM has started out in singulars and spread further to plurals. 
In this language, DOM (genitive \vs zero accusative) is attested with singular masculine proper names and human nouns from the earliest original \ili{Old Russian} sources on, \ie from the 11th c., representing the Common Slavic inheritance. 
At the same time, animacy-driven DOM spread onto plurals during the 13–15th centuries and to nouns referring to animals in the 16th c.\, (\textit{inter alia}, \citealt[61]{Krysko1994Razvitie}). 
The dual forms developed animacy-driven DOM from the 12–14th c.\ \citep[98]{Krysko1994Razvitie}. 
There is evidence that the plural forms acquired DOM approximately during the same time period as the dual in \ili{Old Russian}. 

Not all of the properties listed in \tabref{01-wi-tab:1:semantic} apply to both DSM and DOM to the same extent. 
For instance, \isi{animacy} is sometimes claimed to be a relevant parameter for DOM, while DSM/Differential Agent Marking systems that are organized exclusively along the \isi{animacy scale} are rare \citep{Fauconnier2011Differential}. 
\citet{Fauconnier2011Differential} demonstrates that independently acting inanimates may pattern with animates with respect to Differential Agent Marking, while being distinct from inanimates acting non-independently (via human instigation). 
(See also \citealt{Sinnemki2014Typological} on the frequency of \isi{animacy} as a factor conditioning DOM.)

Finally, \isi{animacy} may have an effect on the DAM in a less straightforward way. 
Thus, \citeauthor{vonHeusingeretal2007Differential} (\citeyear{vonHeusingeretal2007Differential}; \citeyear{vonHeusingeretal2011Affectedness}) and \citet{vonHeusinger2008Verbal} investigate the impact of \isi{animacy} on the \isi{diachronic development} of DOM in \ili{Spanish}. 
They show that for a particular subset of objects, namely for both \isi{definite} and \isi{indefinite} human direct objects, the preference for \textit{a}-marking depends among other things on the verb class. 
If the respective verb regularly takes human or \isi{animate} objects, it tends to use the \textit{a}-marking on its human objects more frequently than the verbs which regularly take \isi{inanimate} objects. 
This trend is stable across different periods irrespective of the overall preference for the \textit{a}-marking of objects.


\subsubsection{Morphological argument properties}
\label{01-wi-sec:2.1.2-Morphological}

Apart from the inherent semantic properties of arguments discussed in \sectref{01-wi-sec:2.1.1-Inherent}, differences in argument marking may often be better captured in terms of inherent morphological properties of the relevant arguments. 
The latter include the part-of-speech distinction (pronoun \vs noun) and – much less frequently discussed – gender/inflectional-class distinctions. 
These two types of DAM will be discussed in what follows.

The pronoun \vs noun distinction is one of the most common lines of split in case marking worldwide (\cf \citealt{Bickeletal2015Typological}). 
For instance, in Jingulu all pronominal patient-like arguments are marked with the accusative suffix \textit{-u}, as in (\ref{01-wi-ex:5:Jingulu}), whereas all nominal patients are in the unmarked nominative case, no matter whether they are animate, as in (\ref{01-wi-ex:6c:Jingulu}) and (\ref{01-wi-ex:6d:Jingulu}), human, as in (\ref{01-wi-ex:6d:Jingulu}) or definite, as in (\ref{01-wi-ex:6b:Jingulu} -- \ref{01-wi-ex:6d:Jingulu}):

\ea\label{01-wi-ex:5:Jingulu}
\langinfo{Jingulu}{Mirndi}{\citealt[102, 160, 247]{Pensalfini1997Jingulu}}\\
\begin{xlist}

\ex\label{01-wi-ex:5a:Jingulu}
	\gll	Angkurla	larrinka-nga-ju	ngank-u.\\
	\textsc{neg}		understand-\textsc{1sg}-do	\textsc{2sg}-\textsc{acc}\\
	\glt	‘I didn't understand you.’ % (Pensalfini 1997: 247)

\ex\label{01-wi-ex:5b:Jingulu}
	\gll	Ngiji-ngirri-nyu-nu		kunyaku.\\
	see-\textsc{1pl.excl}-\textsc{2obj}-did	\textsc{2du.acc}\\
	\glt	‘We saw you two.’ % (Pensalfini 1997: 102)

\ex\label{01-wi-ex:5c:Jingulu}
	\gll	Jaja-mi	ngarr-u!	\\
	wait-\textsc{irr}	\textsc{1sg}-\textsc{acc}\\
	\glt	‘Wait for me!’ %(Pensalfini 1997: 160)\\
	
\end{xlist}
\z

\ea\label{01-wi-ex:6:Jingulu}
\langinfo{Jingulu}{Mirndi}{\citealt[100, 198, 249, 275]{Pensalfini1997Jingulu}}\\
\begin{xlist}

\ex\label{01-wi-ex:6a:Jingulu}
	\gll	Ngangarra	ngaja-nga-ju.\\
	wild.rice	see-\textsc{1sg}-do\\
	\glt	‘I can see wild rice.’%(Pensalfini 1997: 100)

\ex\label{01-wi-ex:6b:Jingulu}
	\gll	Jani	madayi-rni		ngaja-nya-ju?\\
	\textsc{q}	cloud.\textsc{nom}-\textsc{foc}	see-\textsc{2sg}-do\\
	\glt	‘Can you see the cloud?’ %(Pensalfini 1997: 198)
	
\ex\label{01-wi-ex:6c:Jingulu}
	\gll	Wiwimi-darra-rni	warlaku	ngaja-ju.\\
	girl-\textsc{pl}-\textsc{erg}		dog.\textsc{nom} 	see-do\\
	\glt	‘The girls see the dog.’ %(Pensalfini 1997: 275)
	
\ex\label{01-wi-ex:6d:Jingulu}
	\gll	Ngaja-nga-ju	niyi-rnini	nayurni.\\
	see-\textsc{1sg}-do	\textsc{3sg.gen}-\textsc{f}	woman.\textsc{nom}\\
	\glt	‘I can see his wife.’ %(Pensalfini 1997: 249)\\
	
\end{xlist}
\z

\noindent Differential case marking here is the consequence of a larger phenomenon that consists in pronouns patterning differently from nouns when it comes to argument marking. 
First, pronominal case-markers are often phonologically (and etymologically) distinct from the nominal ones. 
As \citet{Filimonova2005Noun} points out, pronouns belong to the most archaic parts of the lexicon and might be more stable and resistant to morphological and phonological changes than nouns and, hence, preserve the older case markers longer than nouns. 
On the other hand, pronouns often are subject to stronger syntactic constraints. 
This might also be part of the explanation for why pronouns – especially those referring to the speech act participants – represent the most notorious hierarchy offenders (see examples in \citealt{Bickeletal2015Typological}).

Finally, inherent properties can only be viewed as triggers of DAM but not as its function or result since these properties (such as pronouns \vs nouns or animate \vs inanimate distinctions) are already coded lexically \citep[4--5]{Kleinetal2011Case}.

The second group of inherent morphological argument properties which can trigger DAM are gender and inflectional classes. 
For example, in \ili{Icelandic}, certain noun classes distinguish between nominative and accusative while others do not \citep[153]{Thrinsson2002Germanic}, compare the two examples: 

\ea\label{01-wi-ex:7:Icelandic}
\langinfo{Icelandic}{Indo-European}{\citealt[153]{Thrinsson2002Germanic}}\\
\begin{xlist}
\ex \textit{tím-i} ‘time-\textsc{nom.sg}’ \vs \textit{tím-a} ‘time-\textsc{acc.sg}’ (masculine weak I)

\ex \textit{nál} ‘needle-\textsc{nom.sg}’ and ‘needle-\textsc{acc.sg}’ (feminine strong I)

\end{xlist}
\z

\noindent In other languages, different inflectional classes have different but always overt allomorphs of a marker, as \eg in Kuuk Thaayorre (Pama-Nyungan, Australia), in which there are three ergative alomorphs depending on the conjugation class plus minor patterns: the ergative is marked either with the suffix \textit{-(n)thurr}, or with a lexically specified suffixed vowel plus the segment /l/ \citep[158--163]{Gaby2006Grammar}.

This type of differences in argument marking is only rarely discussed in the context of DAM, probably due to the fact that \isi{inflectional class} assignments in many languages are only partly semantically conditioned (\eg by the sex of their extensions) and are otherwise idiosyncratic and thus do not yield any obvious functional explanations. 
An exception in the case of typological studies is \citet{Bickeletal2015Typological} and a few discussions of DAM in individual languages, \eg \citet{Karatsareas2011Study} on \ili{Cappadocian Greek}. 
Another reason for the neglect of this type of DAM probably results from the fact that many studies on DAM, starting with \citet{Silverstein1976Hierarchy}, were interested in different alignment patterns resulting from DAM and not in DAM yielding identical alignment patterns, as is the case in languages which use different overt allomorphs of a marker, such as in Kuuk Thaayorre, where the overall alignment pattern does not change despite the difference in marking.

Sometimes differences between inflectional classes might be viewed as a diachronic effect of “morphologization” of a previously semantically constrained DAM. 
\ili{Russian} seems to undergo this process whereby the animacy-driven DOM by the opposition of the former accusative case (zero) (\textit{stol-ø} ‘table-\textsc{acc/nom}’) \vs genitive case (\textit{čelovek-a} ‘human-\textsc{acc/gen}’) is now becoming just one heterogeneous \isi{accusative case} with two allomorphs depending on the particular noun and, hence, on its \isi{inflectional class}. 
The allomorphy can be argued for by applying various syntactic and substitution tests. 
For example, \citet[165--167]{Corbett1991Gender} treats \isi{animacy} in \ili{Russian} as a sub-gender.

\subsubsection{Non-inherent, discourse-based argument properties}
\label{01-wi-sec:2.1.3-Non-inherent-discourse}

Apart from the inherent semantic and morphological lexical argument properties discussed in~\sectref{01-wi-sec:2.1.1-Inherent}–\sectref{01-wi-sec:2.1.2-Morphological} above, a range of further characteristics related to how referents are used in discourse are known to interact with DAM. 
On the one hand, these properties include such semantic dimensions as \isi{definiteness} and specificity; %(\sectref{01-wi-sec:2.1.3.1-Definiteness}),
 on the other hand, they include other categories considered under the umbrella term of \textsc{information structure}. %(\sectref{01-wi-sec:2.1.3.2-Information}).
%Sections 2.1.3.1 and 2.1.3.2 are as of now unnumbered to avoid too many subsections (as recommended in General style rules for linguistics) 

\paragraph*{Definiteness and specificity}
\label{01-wi-sec:2.1.3.1-Definiteness}

As the examples of the effect of \isi{definiteness} and specificity on argument marking, in particular, on DOM, are abundant and easy to find, in this section we only briefly introduce this type of DAM. 
Definiteness and \isi{specificity} are notoriously difficult to define. 
A common proxy for \isi{definiteness} is the semantic-pragmatic notion of identifiability. 
Thus, a \isi{definite} argument is one for which the hearer can identify the referent \citep[2--5]{Lyons1999Definiteness}. 
In a similar way, \citet{Lambrecht1994Information} defines identifiability as reflecting “a speaker’s assessment of whether a discourse representation of a particular referent is already stored in the hearer’s mind or not” \citep[76]{Lambrecht1994Information}.
 In contrast to \isi{definiteness}, which depends both on the speaker and the hearer, specificity only depends on the speaker; a nominal is specific whenever the speaker has a “particular referent in mind” \citep[35]{Lyons1999Definiteness}.\footnote{For an overview of the history of research on \isi{specificity} and other approaches to specificity, see \citet{vonHeusinger2011Specificity}.} 
As the two phenomena of \isi{definiteness} and specificity interact closely, they are frequently integrated into one hierarchy, as in~(\ref{01-wi-ex:8:Hierarchy4}) (see \eg \citealt[94]{Comrie1986Markedness}; \citealt[132]{Croft2003Typology}):

\ea\label{01-wi-ex:8:Hierarchy4}
\isi{definite} > (\isi{indefinite}) specific > (\isi{indefinite}) non-specific
\z

A recent study by \citet{Sinnemki2014Typological} investigates the effect of \isi{definiteness} and \isi{specificity} on DOM and finds that in 71 of 178 languages with DOM in his sample (and in 43 out of 83 genealogical units) \isi{definiteness} and/or specificity play a role, though the respective geographic distribution is somewhat biased: DOM of the languages in the Old World (Africa, Europe, and Asia) are more prone to be affected by this feature than the languages in Australia, New Guinea and the Americas.

\paragraph*{Information structure}
\label{01-wi-sec:2.1.3.2-Information}

The effects of another type of discourse-based properties of arguments on DAM viz. \isi{information structure} properties have been noticed already in early studies of DAM (\eg \citealt{Laca1987Sobre} on \ili{Spanish}; \citealt{Bossong1985Differentielle}) and has become particularly prominent in some recent studies on DAM, including \citet{McGregor1998Optional, McGregor1998Optional, McGregor2006Focal} on differential agent marking, as well as \citet{Iemmolo2010Topicality}; \citet{vonHeusingeretal2007Differential, vonHeusingeretal2011Affectedness}; \citet{Escandell-Vidal2009Differential} and \citet{Dalrympleetal2011Objects} on DOM. 
In what follows we provide an outline of some of the claims.

\citet[14]{Dalrympleetal2011Objects} claim that many seemingly unpredictable cases of variation in DOM can be accounted for by considering \isi{information structure}, understood as that level of sentence grammar where propositions (\ie conceptual states of affairs) are structured in accordance with the information-structure role of sentence elements. 
Specifically, \textit{topicality} plays a critical role in many cases of DOM, such that the distribution of the differential marking depends on whether the object is a \textsc{secondary topic} or (part of) the focus constituent (\citealt{Nikolaeva2001Secondary, Dalrympleetal2011Objects}).
 In this line of research, secondary topic is understood as “an element under the scope of the pragmatic presupposition such that the utterance is construed to be about the relation that holds between it and the primary topic” \citep[2]{Nikolaeva2001Secondary}. 
\citet{Iemmolo2010Topicality} argues against \citegen{Dalrympleetal2011Objects} suggestion and claims that DOM is primarily related to primary topics and special marking is reserved for pragmatically atypical objects, which are primary (or aboutness) topics. 

Apart from \isi{topicality}, focality also figures as a demarcation line for DAM, particularly in cases of a variant of differential agent marking called optional ergativity.
 For instance, in Central (\ili{Lhasa}) Tibetan (\ili{Sino-Tibetan}) unmarked agent arguments are associated with unmarked information distribution, whereas the use of the ergative marker yields a reading with emphasis (focus) on either the identity or the agency of the agent (\cf \citealt{Tournadre1991Rhetorical}). 
While it is somewhat difficult to define and operationalize the notion of emphasis or focality, related notions of \isi{unexpectedness}, surprise or unpredictability of the referent might be better terms in describing individual DAM systems.
For instance, \citet{Schikowski2013Object-conditioned} uses the term \textit{unexpectedness} in addition to various other inherent (\isi{animacy}) and context-dependent (specificity) properties to explain DOM in \ili{Nepali}. 
In \ili{Warrwa} (Nyulnyulan, Western Australia), NPs are marked with the focal ergative marker \textit{-nma}, as in~(\ref{01-wi-ex:9b:Warrwa}), when they are “unexpected, unpredictable, or surprising in terms of their identity and \isi{agentivity}” \citep[399]{McGregor2006Focal}, otherwise they are marked with a different ergative exponent, viz. \textit{-na}, as in in~(\ref{01-wi-ex:9a:Warrwa}).
To account for the distribution of the two markers in continuous stretches of discourse, \citet[516]{McGregor1998Optional} postulates the Expected Actor Principle: “The episode protagonist is — once it has been established — the expected (and unmarked) Actor of each foregrounded narrative clause of the episode; any other Actor is unexpected”.

\ea\label{01-wi-ex:9:Warrwa}
\langinfo{Warrwa}{Nyulnyulan, Western Australia}{\citealt[402]{McGregor2006Focal}}
\begin{xlist}
\ex\label{01-wi-ex:9a:Warrwa}
	\gll nyinka	jurrb	ø-ji-na-yina			kinya	wanyji	kwiina	iri ka-na-ngka-ndi-ø			ø-ji-na,		kinya-\textbf{na}	wuba,\\
		this		jump	3min\textsc{nom}-say-\textsc{pst}-3min\textsc{obl}	this	later	big	woman 1min\textsc{nom}-\textsc{tr}-\textsc{fut}-get-3min\textsc{acc}	3min\textsc{nom}-say-\textsc{pst}	this-\textsc{\textbf{erg}}	small\\
	\glt ‘The little one jumped at her then, at the big woman, and tried to get her.’
	
\ex\label{01-wi-ex:9b:Warrwa}
	\gll kinya	kwiina-\textbf{nma}	iri		marlu	laj	ø-ji-na-ø kinya	wuba,	laj, 	marlu	laj 	ø-ji-na-ø,\\
	this big-\textbf{f\textsc{erg}} woman	not throw 3min\textsc{nom}-say-\textsc{pst}-3min\textsc{acc} this	little 	throw 	not 	throw	3min\textsc{nom}-say-\textsc{pst}-3min\textsc{acc}\\
	\glt ‘But no, the big woman threw the little man away.’

\end{xlist}
\z

To summarize, the information-structure roles that are typically coded by DAM are foci with S and A arguments and topics with P arguments. 
Rarely also the status of P arguments as focal or non-focal triggers DOM (\eg in \ili{Yukaghir}, isolate; \citealt{Maslova2003Information, Maslova2008Case}), while topicality-triggered differential A marking seems unattested. 
This asymmetry may be explained by the findings of \citet{Maslova2003Information} and \citet{Dalrympleetal2011Objects}, who show that in the languages they considered P is common both as focus and topic, while A’s predominantly occur as topics. 
For instance, P’s are 65\% topics in Tundra \ili{Yukaghir} and 60\% topics in \ili{Ostyak} while they are respectively 35\% foci in Tundra \ili{Yukaghir} and 40\% foci in \ili{Ostyak} (\citealt[182]{Maslova2003Information}; \citealt[167]{Dalrympleetal2011Objects}). 
In turn, of all nominal foci of Maslova’s \ili{Yukaghir} corpus 97\% are P foci and less than 1\% are A foci (\citealt[182]{Maslova2003Information}; \citeyear[796]{Maslova2008Case}).


\subsubsection{Argument-triggered DAM: a summary}
\label{01-wi-sec:2.1.4-Argument-triggered-Summary}

The clean typology of argument effects on DAM presented above is an idealization: In many languages argument-triggered DAM systems are conditioned by an intricate combination of both inherent and non-inherent properties. 
For example, the DOM in \ili{Spanish} is primarily conditioned by \isi{animacy} (an inherent property) but inanimates allow for variation depending on factors such as \isi{definiteness} and specificity. 
Moreover, while definites are always marked, indefinites again allow for variation of marking where \isi{topicality}, semantic verb class, preverbal position may favor the marking \citep{vonHeusingeretal2007Differential, vonHeusingeretal2011Affectedness}. 
According to \citet{Escandell-Vidal2009Differential}, pronominal objects in Balearic \ili{Catalan} are always case-marked by accusative, \ie an inherent part-of-speech characteristic of the argument is at work, whereas with non-pronominal objects \isi{case marking} is partly determined by \isi{topicality}.
The DOM of Biblical \ili{Hebrew} is conditioned by a highly complex set of factors from different domains of grammar, including alongside \isi{animacy} and \isi{definiteness}, modality (volitionals) and polarity (under negation) of the verb, preverbal position of the object NP, presence of the reflexive possessor, \etc \citep[173]{Bekins2012Information}.


\subsubsection{Properties of scenario and global \vs local DAM systems}
\label{01-wi-sec:2.1.5-Scenario-global-local}

In \sectref{01-wi-sec:2.1.1-Inherent}–\sectref{01-wi-sec:2.1.4-Argument-triggered-Summary} we discussed how various inherent and discourse-based properties of arguments affect argument marking. 
This type of DAM conditioned by argument-internal properties is sometimes referred to as \textsc{local} (\citealt[178]{Silverstein1976Hierarchy}; \citealt[213]{Malchukov2008Animacy}, \textit{passim}). 
However, not only the properties of differentially marked arguments themselves might be relevant: In some languages, argument marking is sensitive to the properties of other arguments of the same clause, \ie to the nature of the co-arguments. 
In other words, not only one argument on its own, but the whole configuration of who is acting on whom can shape DAM systems. 
This type of DAM is labeled \textsc{global} by \citet[178]{Silverstein1976Hierarchy}, because the assignment of case-marking is regulated on the global level of the event involving all arguments. 
Following \citet{Bickel1995Vestibule, Bickel2011Grammatical} and \citet{Zuniga2006Deixis}, such argument configurations will be referred to as \textit{scenarios} in what follows. 
Within flagging the effects of scenarios are not common, but they are well known in the domain of indexing under the notion of \textsc{hierarchical agreement} (\cf \citealt{Siewierska2003Person}; \citeyear[51--56]{Siewierska2004Person}). 

Effects of scenarios on \isi{case marking} can be illustrated with object marking in \ili{Aguaruna}. 
In this language, the object argument is marked in one of two ways. 
First, it can be in the unmarked nominative, such as the nominal argument \textit{yawaã} ‘dog.\textsc{nom}’ in~(\ref{01-wi-ex:10a:Aguaruna}) and the pronominal arguments \textit{nĩ} ‘3s\textsc{nom}’ in \REF{01-wi-ex:10b:Aguaruna} or \textit{hutii} ‘1p\textsc{nom}’ in \REF{01-wi-ex:10c:Aguaruna}:

\ea\label{01-wi-ex:10:Aguaruna}
\langinfo{Aguaruna}{Jivaroan, Peru}{\citealt[155, 443, 444]{Overall2007Grammar}}
\begin{xlist}

\ex\label{01-wi-ex:10a:Aguaruna}
	\gll Yawaã		ii-nau		maa-tʃa-ma-ka-umɨ?\\
	dog.\textsc{nom}	\textsc{1pl}-\textsc{poss}	kill.\textsc{hiaf}-\textsc{neg}-\textsc{rec.pst}-\textsc{int}-2sg\textsc{pst}\\
	\glt ‘Have you killed our dog?’ %(Overall 2007: 155) 

\ex\label{01-wi-ex:10b:Aguaruna}
	\gll Nĩ		ɨɨma-ta.\\
	\textsc{3sg.nom}	carry.\textsc{pfv}-\textsc{imp}\\
	\glt ‘You(sg.) carry him!’ %(Overall 2007: 443)

\ex\label{01-wi-ex:10c:Aguaruna}
	\gll Hutii		ainau-ti	atumɨ		wai-hatu-ina-humɨ-i.\\
	\textsc{1pl.nom}	\textsc{pl}-\textsc{sap}	2\textsc{pl.nom}	see-\textsc{1pl.obj}-\textsc{pl.ipfv}-\textsc{2pl}-\textsc{decl}\\
	\glt ‘You(pl.) see us.’ %(Overall 2007: 444)
\end{xlist}
\z

\noindent Second, objects can be marked with the \isi{accusative case} suffix \textit{-na}, such as \textit{biika-na} ‘beans-\textsc{acc}’ in~(\ref{01-wi-ex:11a:Aguaruna}), \textit{ii-na} ‘\textsc{1pl-acc}’ in~(\ref{01-wi-ex:11b:Aguaruna}) or \textit{ami-na} ‘\textsc{2sg-acc}’ in~(\ref{01-wi-ex:11c:Aguaruna}):

\ea\label{01-wi-ex:11:Aguaruna}
\langinfo{Aguaruna}{Jivaroan, Peru}{\citealt[146, 326, 444]{Overall2007Grammar}}
\begin{xlist}

\ex\label{01-wi-ex:11a:Aguaruna}
	\gll Ima		biika-na-kɨ		yu-a-ma-ha-i.\\
	\textsc{intens}	bean-\textsc{acc}-\textsc{restr}	eat-\textsc{hiaf}-\textsc{rec.pst}-\textsc{1sg}-\textsc{decl}\\
	\glt ‘I only ate beans.’ %(Overall 2007: 146)

\ex\label{01-wi-ex:11b:Aguaruna}
	\gll Nĩ		ii-na		antu-hu-tama-ka-aha-tata-wa-i.\\
	\textsc{3sg.nom}	\textsc{1pl}-\textsc{acc}	listen-\textsc{appl}-\textsc{1pl.obj}-\textsc{intens}-\textsc{pl}-\textsc{fut}-\textsc{3}-\textsc{decl}\\
	\glt ‘He will listen to us.’ %(Overall 2007: 326)

\ex\label{01-wi-ex:11c:Aguaruna}
	\gll Hutii		a-ina-u-ti			daka-sa-tata-hamɨ-i 	ami-na.\\
	\textsc{1pl.nom}	\textsc{cop}-\textsc{pl.ippfv}-\textsc{rel}-\textsc{sap}	wait-\textsc{att}-\textsc{fut}-\textsc{1sg}>\textsc{2sg.obj}-\textsc{decl} \textsc{2sg}-\textsc{acc}\\
	\glt ‘We will wait for you.’ %(Overall 2007: 444)
\end{xlist}
\z

\noindent As (\ref{01-wi-ex:10c:Aguaruna}) and~(\ref{01-wi-ex:11b:Aguaruna}) demonstrate, an object with identical referential properties (first person plural pronoun) can be either in the nominative or in the \isi{accusative case}. 
Thus, the internal properties of arguments cannot be the trigger of DOM in \ili{Aguaruna}. 
The information-structural properties are not relevant either. 
Instead, the distribution of the two types of object marking is determined by the configuration of the referential properties of both transitive arguments – the A and the P – and is summarized as follows:

\begin{quote}
Object NPs are marked with the accusative suffix \textit{-na}, with some exceptions, that are conditioned by the relative positions of subject and object on the following person hierarchy:

1sg > 2sg > 1pl/2pl > 3

First person singular and third person subjects trigger accusative \isi{case marking} on any object NP, but second person singular, second person plural, and first person plural only trigger marking on higher-ranked object NPs. \citep[168--169]{Overall2007Grammar}
\end{quote}

Similar cases have been reported from other languages. 
Thus, \citet[213]{Malchukov2008Animacy} states that differently from \ili{Hindi}, where DOM is purely locally constrained, the related language \ili{Kashmiri} has globally conditioned DOM: “P takes an object \textsc{(acc/dat)} case if A is lower than P on the Animacy/Person Hierarchy” (\citealt[213]{Malchukov2008Animacy} relying on \citealt[155]{Walietal1997Kashmiri}). 
Thus, as \citet{Malchukov2008Animacy} points out, the global \vs local distinction may be observed even with DAM systems that have the same origin.
Not only inherent argument properties of more than one argument involved in a scenario can trigger DAM, as in the examples above, but also non-inherent discourse-related argument properties of the whole scenario are known to trigger DAM.
The well-known examples include proximate \vs obviative \isi{case marking} in the Algonquian languages (see, for instance, \citealt{Dahlstrom1986Plains} on Plains Cree).

\subsubsection{Properties dependent on event semantics}
\label{01-wi-sec:2.1.6-Properties-dependent}

In some languages DAM is not directly triggered by the inherent or discourse-related properties of arguments or a constellation of several arguments, as discussed in~\sectref{01-wi-sec:2.1.1-Inherent}–\sectref{01-wi-sec:2.1.5-Scenario-global-local}, but rather by the way these arguments are involved in an event. 
The relevant aspects include – among others – volitionality/control or \isi{agentivity} and \isi{affectedness} (for discussions, see \citealt{Naess2004What, McGregor2006Focal}; \citealt[4]{Fauconnier2012Constructional}). 
DAM is used in this context to differentiate between various degrees of \isi{transitivity} in several ways. 
While manipulating the degrees of \isi{agentivity}/control/volitionality is typically done by means of differential agent (or subject) marking, various degrees of \isi{affectedness} (pertaining to P arguments) and resultativity (pertaining to the verbal domain) may be expressed via DOM. 
This division of labor is, of course, expected, because such semantic entailments as volitionality/\isi{agentivity} %(\cf Section 2.1.5.1) 
or \isi{affectedness} %(\cf Section 2.1.5.2) %check: sections at fourth sublevel as of now unnumbered, comp. above
are associated with the A and the P arguments, respectively. 
In what follows, we provide an overview of these two subtypes.

%\paragraph*{Agentivity-related DAM/DSM}

\ili{Tsova-Tush} provides an example of differential S marking triggered by volitionality: according to \citet{Holisky1987Case}, when the argument is volitionally involved and/or in control of the event the S argument appears in the ergative, as in~(\ref{01-wi-ex:12a:Tsova-Tush}), whereas when the involvement of the argument lacks volition or control, it appears in the \isi{nominative case}, as in~(\ref{01-wi-ex:12b:Tsova-Tush}):

\ea\label{01-wi-ex:12:Tsova-Tush}
\langinfo{Tsova-Tush}{Nakh-Daghestanian; Georgia}{\citealt[105]{Holisky1987Case}}
\begin{xlist}

\ex\label{01-wi-ex:12a:Tsova-Tush}
	\gll (As) 		vuiž-n-as.\\
	1s\textsc{erg}	fall-\textsc{aor}-1s\textsc{erg}\\
	\glt ‘I fell. (It was my own fault that I fell down.)’

\ex\label{01-wi-ex:12b:Tsova-Tush}
	\gll (So)		vož-en-sO.\\
	1s\textsc{nom}	fall-\textsc{aor}-1s\textsc{nom}\\
	\glt ‘I fell down, by accident.’
\end{xlist}
\z

\noindent The difference between (\ref{01-wi-ex:12a:Tsova-Tush}) and~(\ref{01-wi-ex:12b:Tsova-Tush}) may also be approached in slightly different terms. 
Discussing the data from \ili{Latvian} and \ili{Lithuanian}, illustrated in~(\ref{01-wi-ex:13:Lithuanian}), \citet{Serzant2013Rise} suggests that some cases of DAM might be better explained by operating with the property of the \textit{control over the pre-stage} of an event. 
This account is somewhat different from \textit{volitionality} and \textit{control}, because the subject referent does not have control over the very event of falling in~(\ref{01-wi-ex:12:Tsova-Tush}) or getting cold in~(\ref{01-wi-ex:13:Lithuanian}) below. 
At the same time, the more agentive marking implies that the subject referent had the opportunity to prevent the situation from coming about, but failed to exercise control at the stage before the event took place. 
Thus, in \ili{Lithuanian}, both (\ref{01-wi-ex:13a:Lithuanian}) and~(\ref{01-wi-ex:13b:Lithuanian}) are grammatical in isolation, but given the context provided by the sentence with the doctor, only (\ref{01-wi-ex:13a:Lithuanian}) is allowed:

\ea\label{01-wi-ex:13:Lithuanian}
\langinfo{Lithuanian}{Baltic, Indo-European}{\citealt[289]{Serzant2013Rise}}\\

	\gll Gydytojas	ant	skaudančio	piršto	uždėjo	ledų, ir		po	dešimties	minučių\\
	doctor		on	aching		finger	put	ice and	after	ten		minute\\

\begin{xlist}

\ex\label{01-wi-ex:13a:Lithuanian}
	\gll man	piršt-as	visai	atšal-o\\
	I.\textsc{dat}	finger-\textsc{nom}	fully	get.cold-\textsc{3pst}\\

\ex\label{01-wi-ex:13b:Lithuanian}
	\gll *aš	piršt-ą		visai	atšal-a-u\\
	I.\textsc{nom}	finger-\textsc{acc}	fully	get.cold-\textsc{pst}-\textsc{1sg}\\
	\glt ‘The doctor put ice on [my] aching finger and after 10 minutes my finger got cold (lit. to me the finger got cold).’ [Elicited]
\end{xlist}
\z

\noindent In both examples (\ref{01-wi-ex:13a:Lithuanian}) and~(\ref{01-wi-ex:13b:Lithuanian}), there is no direct control over the event itself on the part of the \isi{experiencer} (to denote full control, the respective causative form of the verb ‘to get cold’ has to be used in \ili{Lithuanian}).

%\paragraph*{Affectedness and resultativity-related DAM}

The other subtype of DAM conditioned by event semantics, viz.\,\isi{affectedness} and re\-sulta\-ti\-\-vity-related DAM, has often been discussed in relation to particular areas and families, most prominently with respect to the \textit{total} \vs \textit{partitive} alternation in the Finnic and some neighboring \ili{Indo-European} languages. 
Languages of the eastern Circum-Baltic area \citep{Dahletal2001Circum-Baltic} show a remarkable degree of productivity of this type of DAM \citep{Serzant2015Independent}:

\ea\label{01-wi-ex:14:Lithuanian}
\langinfo{Lithuanian}{Baltic, Indo-European}{own knowledge}
\begin{xlist}

\ex\label{01-wi-ex:14a:Lithuanian}
	\gll Jis iš-gėrė	vanden-į.\\
	he \textsc{telic}-drink.\textsc{3pst}	water-\textsc{acc.sg}\\
	\glt ‘He drank (up) (the/some) water.’

\ex\label{01-wi-ex:14b:Lithuanian}
	\gll Jis	iš-gėrė 		vanden-s.\\
	he	\textsc{telic}-drink.\textsc{3pst}	water-\textsc{gen.sg}\\
	\glt ‘He drank (*the/some) water.’
\end{xlist}
\z

\noindent The verb ‘to drink’ subcategorizes for an accusative object in \ili{Lithuanian}, as in~(\ref{01-wi-ex:14a:Lithuanian}), which is the default option in this language and may have both \isi{definite} and \isi{indefinite} (weak/‘some’) interpretation, since this language does not have grammaticalized articles and bare NPs are generally ambiguous regarding \isi{definiteness}. 
However, the regular accusative marking may be overridden by the \isi{genitive case}, as in~(\ref{01-wi-ex:14b:Lithuanian}), where the exhaustive or \isi{definite} reading is no longer available \citep{Serzant2014Denotational}. 
The genitive option induces the indefinite-quantification reading in~(\ref{01-wi-ex:14b:Lithuanian}) which, in turn, is related to non-specificity. 
Furthermore, the indefinite-quantity reading renders the verbal phrase in~(\ref{01-wi-ex:14b:Lithuanian}) \isi{atelic} (non-resultative in the \ili{Finnish} tradition, \cf \citealt{Huumo2010Nominal}), the whole event of ‘drinking water’ becomes an activity predicate in contrast to the accomplishment interpretation in~(\ref{01-wi-ex:14a:Lithuanian}). 
While this effect is found mostly with verbs taking the incremental theme \citep{Dowty1991Thematic} in \ili{Lithuanian} \citep{Serzant2014Denotational}, Finnic languages allow basically any accomplishment verb to acquire an activity interpretation by means of this type of DOM, \cf the verb `to open’ in~(\ref{01-wi-ex:15:Finnish}) taking a non-incremental theme (\cf \citealt{Kiparsky1998Partitive, Huumo2010Nominal}):

\ea\label{01-wi-ex:15:Finnish}
\langinfo{Finnish}{Finnic, Finno-Ugric}{\citealt[273]{Kiparsky1998Partitive}}
\begin{xlist}

\ex\label{01-wi-ex:15a:Finnish}
	\gll Hän	avasi		ikkunan.\\
	he	open.\textsc{3sg.pst}	window.\textsc{acc.sg}\\
	\glt ‘He opened the window.’

\ex\label{01-wi-ex:15b:Finnish}
	\gll Hän	avasi		ikkunaa.\\
	he	open.\textsc{3sg.pst}	window.\textsc{part.sg}\\
	\glt 	(i) ‘He was opening the window.’\\
		(ii) ‘He opened the window (partly).’\\
		(iii) ‘He opened the window for a while.’\\
		(iv) ‘He opened the window again and again.’\\
\end{xlist}
\z

\noindent Crucially, all four readings in~(\ref{01-wi-ex:15b:Finnish}) imply a construal of an event in the past that is not committal as to the achievement of an inherent end point (the door is closed). 
In turn, only (\ref{01-wi-ex:15a:Finnish}) with accusative marking\footnote{The \ili{Finnish} \isi{accusative case} is highly syncretic: it is homonymous with the genitive in the singular and with the nominative in the plural and has dedicated morphology only with personal pronouns \citep[100--101]{Karlsson1999Essential}.
 This is why it is sometimes (somewhat misleadingly) referred to as the genitive in the traditional linguistic literature on \ili{Finnish}.} 
of the object indicates that the inherent end point of the process of ‘window opening’ has been achieved. 
At the same time, in contrast to (\ref{01-wi-ex:14b:Lithuanian}), there is no weak quantification of the object referent – only the verbal action is quantified while the object is affected holistically. 
Note that there is no relation to viewpoint (or even progressive) aspect here, as is sometimes assumed in the literature (see the discussion in \citealt{Serzant2015Independent}). 
The non-resultativity (or only partial result) of the event in~(\ref{01-wi-ex:15b:Finnish}), of course, entails that the object referent has not been affected to the extent that it has been in~(\ref{01-wi-ex:15a:Finnish}).

\subsubsection{Argument-triggered DAM: a summary}
\label{01-wi-sec:2.1.7-DAMsummary2}

\sectref{01-wi-sec:2.1-Argument-triggered} considers only those cases of DAM where argument properties function as trigger, while the form of the predicate remains the same. 
This type has been in the focus of the study of DAM since its very beginning and arguably represents the consensus examples of DAM (\cf \citealt{Bossong1985Differentielle, Bossong1991Differential}). 
We follow this tradition and consider this type of DAM as a more central one. The following is thus our narrow definition of DAM:

\protectedex{
\ea\label{01-wi-ex:16:NarrowDAM}
Narrow definition of DAM:\\
Any kind of situation where an argument of a predicate bearing the same generalized \isi{semantic role} may be coded in different ways, depending on factors other than the argument role itself and/or the clausal properties of the predicate such as polarity, TAM, embeddedness, etc.
\z
}

\subsection{Predicate-triggered DAM}
\label{01-wi-sec:2.2-Predicate-triggered}
We now turn to the discussion of the other major type of DAM, namely, \textsc{predicate-triggered} DAM. 
The cases of DAM to be discussed in this section involve a broader understanding of the phenomenon according to the definition in~(\ref{01-wi-ex:1:DefinitionsDAM}) but not according to the definition in~(\ref{01-wi-ex:16:NarrowDAM}), which requires one and the same form of the predicate. 
In this type of DAM, different – though paradigmatically related – forms of the predicate require differential marking of its argument and neither inherent nor discourse-related properties of arguments play any role. 
Nevertheless, we think that such DAM systems are of no lesser interest than the systems discussed in~\sectref{01-wi-sec:2.1-Argument-triggered} and may be related to them diachronically. 

\subsubsection{Clause-type-based differential marking}
\label{01-wi-sec:2.2.1-Clause-type-based}

A very common, but not very frequently discussed kind of DAM is the one in which a particular kind of argument marking is found in one type of clause, whereas in some other type of clause the relevant argument is marked differently (\cf “main” versus “subordinate” clause split in \citealt[101]{Dixon1994Ergativity} or “split according to construction” in \citealt[492]{McGregor2009Typology}). 
This type of DAM can be illustrated by the comparison of the main clause with different types of dependent clauses in \ili{Maithili}. 
In the main clause, the sole argument of one-argument clauses and the more agent-like arguments of two-argument clauses are in the nominative, as in~(\ref{01-wi-ex:17a:Maithili}) and~(\ref{01-wi-ex:17b:Maithili}) respectively:

\ea\label{01-wi-ex:17:Maithili}
\langinfo{Maithili}{\ili{Indo-European}; India, Nepal}{\citealt[346, 347]{Bickeletal2000Fresh}}
\begin{xlist}

\ex\label{01-wi-ex:17a:Maithili}
	\gll O		hãs-l-aith.\\
	3h\textsc{rem.nom}	laugh-\textsc{pst}-3h\textsc{nom}\\
	\glt ‘He(h\textsc{rem}) laughed.’ %(Bickel & Yādava 2000: 346)
	
\ex\label{01-wi-ex:17b:Maithili}
	\gll O		okra			cāh-ait			ch-aith.\\
	3h\textsc{rem}	3nh\textsc{rem.dat}	like-\textsc{ipfv.ptcp}	\textsc{aux}-3h\textsc{nom}\\
	\glt ‘S/he(h\textsc{rem}) likes him/her(nh.\textsc{rem}).’ %(Bickel & Yādava 2000: 347)
\end{xlist}
\z

\noindent However, in various types of dependent clauses, for instance in converbial clauses, as in~(\ref{01-wi-ex:18a:Maithili}), and infinitival clauses as in~(\ref{01-wi-ex:18b:Maithili}), these arguments are in the \isi{dative case}:

\protectedex{
\ea\label{01-wi-ex:18:Maithili}
\langinfo{Maithili}{\ili{Indo-European}; India, Nepal}{\citealt[353, 358]{Bickeletal2000Fresh}}
\begin{xlist}

\ex\label{01-wi-ex:18a:Maithili}
	\gll [Hamrā	(*ham)	 ghar	āib-kẽ]	pitā-jī		khuśī	he-t-āh.\\
	\textsc{1dat}		\textsc{1nom}	 home	come-\textsc{cvb}	father-h\textsc{nom}	happy	be(come)-\textsc{fut}-3h\textsc{nom}\\
	\glt ‘When I come home, father will be happy.’ %(Bickel & Yādava 2000: 353)
	
\ex\label{01-wi-ex:18b:Maithili}
	\gll [Rām-kẽ	(*Rām)	sut-b-āk		lel]	ham	yahī̃	ṭhām-sã	uṭhī-ge-l-aũh.\\	
	Ram-\textsc{dat}	Ram.\textsc{nom}	sleep-\textsc{inf:obl}-\textsc{gen}	for	\textsc{1nom}	here	place-\textsc{abl}	rise-\textsc{tel}-\textsc{pst}-1\textsc{nom}\\
	\glt ‘I got up from this place in order for Ram to (be able to) sleep.’ %(Bickel & Yādava 2000: 358)
\end{xlist}
\z
}

\noindent Note that differential marking is never possible with one and the same form of the predicate. 
Instead, the two types of marking are in complementary distribution as determined by the matrix \vs embedded status of the predicate.

\subsubsection{TAM-based differential marking}
\label{01-wi-sec:2.2.2-TAM-based}

Tense, aspect, and mood of the clause present an often discussed trigger of DAM, in particular in case of differential agent marking, when discussing so-called split ergativity (\cf \citealt{Comrie1978Ergativity}; \citealt[97--101]{Dixon1994Ergativity}; \citealt{deHoopetal2007Fluid}). 
The distribution of case markers in Georgian illustrates this type of DAM. 
In the present, the agent argument appears in the \isi{nominative case}, \eg \textit{deda} ‘mother.\textsc{nom}’ in~(\ref{01-wi-ex:19a:Georgian}). 
In the aorist, the agent argument appears in the narrative case (sometimes also called ergative), \eg \textit{deda-m} ‘mother-\textsc{narr}’ in~(\ref{01-wi-ex:19b:Georgian}):

\ea\label{01-wi-ex:19:Georgian}
\langinfo{Georgian}{Kartvelian; Georgia}{\citealt[42]{Harris1981Georgian}}
\begin{xlist}

\ex\label{01-wi-ex:19a:Georgian}
	\gll Deda		bans			tavis		švil-s.\\
	mother.\textsc{nom}	she.bathes.him.\textsc{prs}	self.\textsc{gen}	child-\textsc{dat}\\	
	\glt ‘The mother is bathing her child.’
	
\ex\label{01-wi-ex:19b:Georgian}
	\gll Deda-m		dabana		tavis-i			švil-i.\\
	mother-\textsc{narr}	she.bathed.him.\textsc{aor}	self.\textsc{gen}-\textsc{nom}	child-\textsc{nom}\\
	\glt ‘The mother bathed her child.’
\end{xlist}
\z

A number of functional explanations and predictions about possible systems of marking have been proposed with respect to the effects of tense and aspect properties of the clause (see \citealt[97--101]{Dixon1994Ergativity}; \citealt{DeLancey1981Interpretation, DeLancey1982Aspect}). 
For instance, \citet[99]{Dixon1994Ergativity} predicts that if a language shows differential agent marking conditioned by tense or aspect, the ergative marking pattern is always found either in the past tense or in the perfective aspect. 
Such functional explanations of alleged correlations of marking and TAM are sometimes presented as textbook knowledge (\cf \citealt[174]{Song2001Linguistic}). 
However, they are not unproblematic, as discussed in \citet{Creissels2008Direct} and \citet[143--144]{Witzlack-Makarevich2011Typological}. 
One of the problems lies in the following: The languages frequently used to illustrate effects of the tense-aspect properties of the clause on DAM include a number of \ili{Indo-Aryan} and Iranian languages (\eg \citealt[100]{Dixon1994Ergativity}; \citealt{deHoopetal2007Fluid}). 
However, although tense-aspect values of the clause might superficially seem to condition a particular argument marking in these languages, the distribution of case markers is actually determined by certain morphological verb forms (for instance, a special participle or a converb) – and not by TAM as such – and this distribution has an etymological motivation (for examples, see \citealt[144]{Witzlack-Makarevich2011Typological}).

\subsubsection{Polarity-based differential marking}
\label{01-wi-sec:2.2.3-Polarity-based}

Polarity of the clause is another predicate-related feature that has long been known to interact with argument marking (\cf \citealt[101]{Dixon1994Ergativity}). 
Its effects can be illustrated with the \ili{Finnish} examples in~(\ref{01-wi-ex:20:Finnish}). 
Whereas in affirmative clauses the P argument can appear either in the accusative or \isi{partitive case}, as in~(\ref{01-wi-ex:20a:Finnish}), in negative clauses only the \isi{partitive case} marking of the P argument is grammatical, as in~(\ref{01-wi-ex:20b:Finnish}):

\ea\label{01-wi-ex:20:Finnish}
\langinfo{Finnish}{\ili{Uralic}; Finland}{\citealt[115]{Sulkalaetal1992Finnish}}
\begin{xlist}

\ex\label{01-wi-ex:20a:Finnish}
	\gll Söin		omena-n	/	omena-a.\\
	eat.1s\textsc{ipfv}	apple-\textsc{acc}	/	apple-\textsc{part}\\
	\glt ‘I ate/was eating an apple.’

\ex\label{01-wi-ex:20b:Finnish}	
	\gll En		syönyt		omena-a.\\
	\textsc{neg}-1s	eat-\textsc{2ptcp}	apple-\textsc{part}\\
	\glt ‘I didn’t eat/was not eating an apple.’
\end{xlist}
\z

\subsubsection{Differential marking and marking of information structure with verbal morphology}
\label{01-wi-sec:2.2.4-Differential}

While information-structure-driven DAM systems mostly represent cases of DAM in the narrow sense, as defined in~(\ref{01-wi-ex:16:NarrowDAM}), individual information-structural configurations may also require different forms of the predicate, \eg in \ili{Somali} \citep{Saeed1987Somali}. 
Similarly, in Arbor, the form of the predicate in~(\ref{01-wi-ex:21a:Arbor}) is different from the one in~(\ref{01-wi-ex:21b:Arbor}): the topical, nominative subject (\ref{01-wi-ex:21a:Arbor}) takes the predicate with the auxiliary \textit{Ɂíy} while the focal subject (\ref{01-wi-ex:21b:Arbor}) does not allow the auxiliary:

\ea\label{01-wi-ex:21:Arbor}
\langinfo{Arbore}{Cushitic, Ethiopia}{\citealt[113]{Hayward1984Arbore}}
\begin{xlist}

\ex\label{01-wi-ex:21a:Arbor}
	\gll Farawé 		Ɂí-y 		zaɦate\\
	horse.\textsc{f.nom} 	\textsc{pvs}-\textsc{3sg}	die.\textsc{3sg.f}\\
	\glt ‘(A) horse died.’

\ex\label{01-wi-ex:21b:Arbor}
	\gll Farawa 		zéɦe\\
	horse.\textsc{f.pred} 	died.\textsc{3sg.m}\\
	\glt ‘(A) \textsc{horse} died.’ (Capitals signify the narrow focus)
\end{xlist}
\z

\subsection{Summary of DAM triggers}
\label{01-wi-sec:2.3-Summary-triggers}

Sections \sectref{01-wi-sec:2.1-Argument-triggered}–\sectref{01-wi-sec:2.2-Predicate-triggered} cover the entire range of DAM triggers. 
We identify two major types of DAM systems. 
On the one hand, we distinguish argument-triggered DAM systems with no direct dependency on the predicate form. 
Such systems can be triggered by various argument properties and event semantics and are in accordance with both our narrow definition in~\REF{01-wi-ex:16:NarrowDAM} and broad definition in~\REF{01-wi-ex:1:DefinitionsDAM}. 
On the other hand, there are a whole range of DAM systems where the same argument role is marked differently in different subparadigms of the predicate. 
\tabref{01-wi-tab:2:Triggers} summarizes this typology and provides references to the respective examples. 

\begin{table}
{\small \begin{tabularx}{\textwidth}{LLLLL}
\lsptoprule
	 & \multicolumn{2}{l}{DAM trigger type} & DAM trigger & Examples\\ 
\midrule
& properties of the argument (local DAM) & inherent properties & \isi{animacy}, person, discreteness, part of speech, inflection class & Jingulu \REF{01-wi-ex:5:Jingulu}, \REF{01-wi-ex:6:Jingulu}\\ 
\cmidrule{3-5}
& 	& non-inherent properties & \isi{definiteness}, specificity, \isi{topicality}, focality & \ili{Warrwa} \REF{01-wi-ex:9:Warrwa}\\ 

\cmidrule{2-5}

same predicate form		& properties of the whole scenario (global DAM) & inherent properties	& animacy, person, discreteness, part of speech, inflection class & \ili{Aguaruna} \REF{01-wi-ex:10:Aguaruna}, \REF{01-wi-ex:11:Aguaruna}\\
 
\cmidrule{3-5}	
	&		& non-inherent properties	& definiteness, specificity, topicality, focality	& \textsc{3prox} > 3\textsc{obv} (not in the text)\\ 

\cmidrule{2-5}	

& event semantics			&		& \isi{affectedness}, control over the event & \ili{Tsova-Tush} \REF{01-wi-ex:12:Tsova-Tush}\\ 

\midrule
different predicate forms 	&				&				& TAM, polarity, clause type, etc.	& \ili{Maithili} \REF{01-wi-ex:17:Maithili}, \REF{01-wi-ex:18:Maithili}; Georgian \REF{01-wi-ex:19:Georgian}; \ili{Finnish} \REF{01-wi-ex:20:Finnish}\\ 

\lspbottomrule
\end{tabularx}}
\caption{DAM systems according to the trigger}
\label{01-wi-tab:2:Triggers}
\end{table}


\subsection{The scope of DAM: restricted and unrestricted DAM systems}
\label{01-wi-sec:2.4-DAM-scope}

Whereas in some languages DAM seems to apply throughout the whole language system, in many languages its range is restricted in various ways, \eg to particular predicates or individual clause types or to particular inflectional classes. 
Thus, one can distinguish between restricted DAM systems (to be illustrated in this section) and apparently unrestricted systems (the examples given in~\sectref{01-wi-sec:2.1-Argument-triggered}–\sectref{01-wi-sec:2.2-Predicate-triggered}, though admittedly we are not always certain whether DAM indeed applies without any restrictions in these languages). 

In \ili{Latvian}, the nominative-accusative split in patient marking is restricted to a very limited domain, namely, to the debitive construction denoting necessity. 
The construction is marked by an auxiliary (optional in the present tense) and the prefix \textit{j\=a-} on the verb, as in~\REF{01-wi-ex:22:Latvian}:

\protectedex{
\ea\label{01-wi-ex:22:Latvian}
\langinfo{Latvian}{Baltic, Indo-European}{personal knowledge}
\begin{xlist}

\ex\label{01-wi-ex:22a:Latvian}
	\gll Tev		(ir)		jā-ciena	mani/*es.\\
	you.\textsc{dat}	(\textsc{aux.prs.3})	\textsc{deb}-respect 	\textsc{i.acc/*i.nom}\\

\ex\label{01-wi-ex:22b:Latvian}
	\gll Tev		(ir)		jā-ciena 	viņš/māte/valsts.\\ 
	you.\textsc{dat}	(\textsc{aux.prs.3})	\textsc{deb}-respect 	he.\textsc{nom}/mother.\textsc{nom}/state.\textsc{nom}\\
\end{xlist}

a. ‘You have to be respectful towards me (\textsc {acc}).’\\
b. ‘You have to be respectful towards him (\textsc{nom}) / [your] mother (\textsc{nom}) / [the] country (\textsc{nom}).’ [Constructed example]\\

\z}

\noindent In this construction, the \isi{patient argument} realized with speech-act-participant personal and reflexive pronouns is obligatorily marked with the accusative case, while other NP types are marked with the nominative case in the standard language. 
Elsewhere, \ili{Latvian} does not show any DAM. 
The debitive construction in~(\ref{01-wi-ex:22:Latvian}) is thus the only domain in \ili{Latvian} within which one finds DAM.

Another type of a cross-linguistically recurrent domain for DAM is subordinate clauses. 
For instance, in \ili{Turkish}, the domain for the differential subject marking is the nominalized subordinate clause in which the subject must either bear the nominative case – which is a morphological zero – or be marked overtly by the \isi{genitive case}. 
In the former case the subject has a generic, non-specific interpretation, as in~(\ref{01-wi-ex:23b:Turkish}), in the latter case, it has a specific \isi{indefinite} interpretation, as in~(\ref{01-wi-ex:23a:Turkish}) (\citealt[95]{Comrie1986Markedness}; \citealt[83--84]{Kornfilt2008DOM}):

\ea\label{01-wi-ex:23:Turkish}
\langinfo{Turkish}{Turkic}{\citealt[83--84]{Kornfilt2008DOM}}
\begin{xlist}

\ex\label{01-wi-ex:23a:Turkish}
	\gll [Köy-ü 		bir	haydut-un	bas-tığ-ın]-ı		duy-du-m.\\
	village-\textsc{acc}	a	robber-\textsc{gen}	raid-\textsc{fn}-\textsc{3sg}-\textsc{acc}	hear-\textsc{pst}-\textsc{1sg}\\
	\glt ‘I heard that a (certain) robber raided the village.’ (specific) %(Kornfilt 2008: 84)

\ex\label{01-wi-ex:23b:Turkish}
	\gll [Köy-ü		haydut		bas-tığ-ın]-ı		duy-du-m.\\
	village-\textsc{acc}	robber		raid-\textsc{fn}-\textsc{3sg}-\textsc{acc}	hear-\textsc{pst}-\textsc{1sg}\\
	\glt ‘I heard that robbers raided the village.’ (non-specific, generic) %(Kornfilt 2008: 84)
\end{xlist}
\z

\noindent Crucially, the nominative \vs genitive differential subject marking is found only in the subordinate clauses, while the main clauses in \ili{Turkish} do not allow this type of DAM. 
Note that the distinction between the subordinated \vs main clause is not the trigger for the DAM here, in contrast to the cases discussed in~\sectref{01-wi-sec:2.1.1-Inherent}. 
In this case, the DAM is triggered by the properties of the respective argument – specific \vs non-specific, as discussed in~\sectref{01-wi-sec:2.1.3-Non-inherent-discourse}.%Section \sectref{01-wi-sec:2.1.3.1-Definiteness} 
%added reference for section 2.1.3 instead of 2.1.3.1 to avoid four sublevels, check if this is ok
The only difference to the other similar examples is that the distribution of DAM is restricted to subordinate clauses.

In addition to syntactically restricted domains, as in~(\ref{01-wi-ex:22:Latvian}) and~(\ref{01-wi-ex:23:Turkish}), DAM systems may also be restricted lexically. 
Thus, the range of DAM may be limited by a particular class of verbs – motivated semantically or otherwise. 
For instance, a small number of one-argument predicates in \ili{Hindi}/Urdu allow for differential marking of its sole argument conditioned by volitionality, \eg \textit{bhõk-} ‘bark’, \textit{khãs-} ‘cough’, \textit{chĩk-} ‘sneeze’, \textit{hãs-} ‘laugh’, \etc (see \citealt{Davison1999Ergativity} for an exhaustive list). 
This is illustrated in~(\ref{01-wi-ex:24:Hindi-Urdu}): whereas in~(\ref{01-wi-ex:24a:Hindi-Urdu}) the sole argument is in the unmarked nominative case and the event of coughing is understood as being unintentional, in~(\ref{01-wi-ex:24b:Hindi-Urdu}) the sole argument is in the \isi{ergative case} to reflect the intentional nature of the coughing event:

\ea\label{01-wi-ex:24:Hindi-Urdu}
\langinfo{Hindi-Urdu}{\ili{Indo-Aryan}; India, Pakistan}{\citealt[264]{Tuiteetal1985Agentivity}}
\begin{xlist}

\ex\label{01-wi-ex:24a:Hindi-Urdu}
	\gll Ram		khãs-a.\\
	Ram.\textsc{nom}	cough-\textsc{prf.m}\\
	\glt ‘Ram coughed.’

\ex\label{01-wi-ex:24b:Hindi-Urdu}
	\gll Ram=ne	khãs-a.\\
	Ram=\textsc{erg}	cough-\textsc{prf.m}\\
	\glt ‘Ram coughed (purposefully).’\\
\end{xlist}
\z

We discussed similar cases in~\sectref{01-wi-sec:2.1.6-Properties-dependent} under properties dependent on event semantics. 
The major difference between these examples and the examples in~\sectref{01-wi-sec:2.1.6-Properties-dependent} lies in the fact that the intentionality-based DAM in \ili{Hindi}/Urdu does not apply to every sole argument, but, its domain is limited to a very small set of verbs.

To summarize, the range of DAM can be restricted in various ways by the properties of the predicate: by various verbal grammatical categories (such as tense, aspect or mood), by the syntactic position (\eg embedded \vs matrix) or by lexical restrictions (particular verb classes only). 
The categories which restrict the range of DAM are often similar to those discussed in~\sectref{01-wi-sec:2.2-Predicate-triggered}, but their effect on DAM is different: whereas in restricted systems discussed in this section we find DAM triggered mostly by the familiar inherent or discourse-based properties of arguments but \textit{limited} to particular contexts, \eg to particular types of clauses, the predicate-based DAM systems in~\sectref{01-wi-sec:2.2-Predicate-triggered} are directly triggered by a particular form of the predicate.
Note that the restricted argument-triggered DAM systems still adhere to the narrow definition of DAM in~(\ref{01-wi-ex:16:NarrowDAM}) alongside the unrestricted argument-triggered ones. 
Another way to put it is as follows: if one knows that the DAM system is restricted, one can identify the domain where one finds alternating argument marking. 
However, to predict what kind of marking an argument takes, one still has to consider the triggers of DAM. 
The cross-tabulation of the scope variable of DAM system and the familiar trigger variable yields the four subtypes of DAM systems summarized in~\tabref{01-wi-tab:3:Typological-variation}:

\begin{table}
\begin{tabularx}{\textwidth}{XXLL}

\lsptoprule
	 & 	 & \multicolumn{2}{ c}{\centering Trigger}	\\ 
\midrule
	 & 	 & argument properties & predicate properties\\ 
 \midrule
\multirow{8}{*}{\centering Scope} & unrestricted & unrestricted argument-triggered DAM & unrestricted predicate-/ clause-triggered DAM\\ 
\cmidrule{2-4}
	 & restricted & restricted argument-triggered DAM & restricted predicate-/ clause-triggered DAM\\ 
\lspbottomrule
\end{tabularx}
\caption{Typological variation of DAM systems}
\label{01-wi-tab:3:Typological-variation}
\end{table}


\section{Morphological and syntactic properties of DAM}\label{01-wi-sec:3-Morphological}

In this section we provide a survey of the variation in DAM related to its morphological and syntactic properties. 
We first discuss the morphological dichotomy between symmetric and asymmetric DAM systems (\sectref{01-wi-sec:3.1-Symmetric}) and then proceed to the locus of marking and give a short overview of the research on differential flagging in contrast to differential indexing (\sectref{01-wi-sec:3.2-Differential}). 
In~\sectref{01-wi-sec:3.3-Syntactic} we briefly consider the syntactic properties of DAM. 
Finally, \sectref{01-wi-sec:3.4-Obligatory} touches upon the issues of obligatoriness of DAM.

\subsection{Symmetric \vs asymmetric DAM}\label{01-wi-sec:3.1-Symmetric}

From the beginning of the research on DOM it has generally been assumed that DOM yields a binary opposition based on \isi{markedness}: certain NP types are marked in terms of both prominence (\isi{animacy}, \isi{definiteness}, etc.) and morphological encoding while others are unmarked, i.e., are non-prominent and morphologically unmarked (\textit{inter alia}, \citealt{Bossong1985Differentielle, Bossong1991Differential}; but also \citealt{Aissen2003Differential}). 
In other words, semantic \isi{markedness} is mirrored by the morphological markedness or \textsc{asymmetric} encoding: X \vs zero. 
Many DOM systems are of this type, \eg the DOM of \ili{Spanish} or \ili{Persian}. 
For example, \ili{Spanish} contrasts animate specific objects to all others by marking the former but not the latter with the preposition \textit{a}. 

Recently, however, also \textsc{symmetric} DAM systems – \ie systems where both alternatives receive overt morphological marking – have become the focus of attention in several studies (\eg \citealt{Hoopetal2008Case-marking, Iemmolo2013Symmetric}). 
Some researchers have argued that symmetric and asymmetric DAM systems are regulated by different principles (\citealt[19]{Dalrympleetal2011Objects}; \citealt{Abrahametal2012Case, Iemmolo2013Symmetric}). 
For instance, \citegen{Iemmolo2013Symmetric} study shows that symmetric DOM systems respond to parameters related to the overall semantics of the event, \eg polarity and quantification, \isi{affectedness} or boundedness (aspectuality), whereas asymmetric systems reflect various participant properties, most prominently its information-structure role, \isi{animacy}, \isi{referentiality}, \etc (similarly \citealt[320]{Abrahametal2012Case}). 

While functional correlations between prominence and morphological realization of DOM like those put forward by \citet{Iemmolo2013Symmetric} do indeed find some cross-linguistic support, there are a number of counterexamples. 
For instance, the DOM found in Kolyma \ili{Yukaghir} (\ili{Yukaghir}, isolate) is symmetric: it requires accusative marking \textit{-gele}/\textit{-kele} for \isi{definite} nouns and the instrumental case ending \textit{-le} for \isi{indefinite} nouns with third person A arguments \citep[93]{Maslova2003Information}. 
Functionally, this type of DOM is very much reminiscent of the asymmetric DOM in Biblical (and modern) \ili{Hebrew}. 
The latter is also conditioned by \isi{definiteness} but, in contrast to Kolyma \ili{Yukaghir}, is morphologically asymmetric as it requires the preposition \textit{’et} with \isi{definite} NPs and disallows it with \isi{indefinite} NPs. 
Counterexamples are found with differential agent marking as well: for instance, \ili{Warrwa} (Kimberley, Western Australia; \citealt{McGregor2006Focal}) has alternations between two different ergatives and is thus an instance of symmetric DAM by definition. 
However, in contrast to the claims \eg in \citet{Iemmolo2013Symmetric}, this system is solely conditioned by the properties of the A argument itself (such as expectedness) and is not related to verbal semantics.

The aforementioned claim about the correlations of symmetrically realized DAMs with event interpretation, on the one hand, and asymmetrically realized DAMs correlating with participant interpretation, on the other, is too strong also for the following reason. 
The opposition between an overt \vs zero marker is only possible if there is no general ban on zeros in the particular domain of a language. 
For example, the opposition between accusative and nominative object marking in the \ili{Latvian} debitive construction is functionally dependent – somewhat similarly to the \ili{Spanish} DOM – on factors such as \isi{animacy} and accessibility but the morphological realization here is the one between one overt marking (nominative, \eg \textit{-s}) \vs another overt marking (accusative, \eg \textit{-u}) simply because \ili{Latvian} disallows zero markers for any case. 
For this \ili{Latvian} system it is difficult to determine which option is morphologically (more) marked and which one is unmarked or less marked\footnote{But see, for instance, \citet{Keineetal2008Differential} for using not only the length of markers, but also their phonological properties, such as sonority, to determine phonological \isi{markedness}.} and, crucially, whether more prominent participants (animates and more accessible referents) or the less prominent participants (inanimates and less accessible) are more coded.

While \ili{Latvian} disallows zeros in all its declensional paradigms, other languages preclude zeros only in a particular (sub)paradigm: typically, the plural and pronominal paradigms in fusional declensions do not contain zeros. 
For example, in \ili{Russian}, all DOM types are symmetric in the plural (but not in the singular) because there is a dedicated plural marker \textit{-y/-i} for the nominative. 
Even the textbook example of \ili{Spanish} does not fully fit the pattern X \vs zero when it comes to pronouns, \cf \textit{a mí} `\textsc{acc 1sg.acc}' \vs \textit{me} `\textsc{1sg.acc}'. 
Pronouns are often morphologically (suppletive) portmanteau words combining both the referential and case-marking morphemes. 
It is therefore often difficult to distinguish between symmetric \vs asymmetric DAM in these cases.

Rarely are there DOM systems which are asymmetric but where their asymmetry is reverse to what is expected because it is the morphologically marked member that is less prominent while the zero-marked one is more prominent. 
For example, the DOM based on the opposition between the partitive use of the genitive in \ili{Russian} is a case in point. 
Here, the less prominent NP is always marked by the partitive genitive with dedicated morphological coding. 
In turn, the \isi{accusative case} has no dedicated marking for a large number of \isi{inanimate} (and some \isi{animate}) NPs: 

\ea\label{01-wi-ex:25:Russian}
\langinfo{Russian}{Slavic, Indo-European}{personal knowledge}
\begin{xlist}

\ex\label{01-wi-ex:25a:Russian}
	\gll Ja 	vypil 		konjak-ø.\\
	\textsc{I.nom}	drink.\textsc{sg.pst}	cognac-\textsc{sg.acc}\\
	\glt ‘I drank up the cognac.’

\ex\label{01-wi-ex:25b:Russian}
	\gll Ja 	vypil 		konjak-a.\\
	\textsc{I.nom} drink.\textsc{sg.pst}	cognac-\textsc{sg.gen}\\
	\glt ‘I drank some/*the cognac.’
\end{xlist}
\z

\noindent The DOM found in \REF{01-wi-ex:25a:Russian}–\REF{01-wi-ex:25b:Russian} is asymmetric by definition. 
However, it is the semantically more prominent NP in \REF{01-wi-ex:25a:Russian}  that is unmarked as opposed to \REF{01-wi-ex:25b:Russian}.

To conclude, there are three ways of how prominence correlates with morphological \isi{markedness}: (i) the prominent meaning is coded with more material than the non-prominent (\eg the \ili{Spanish} DOM), (ii) both the prominent and the non-prominent meanings are similarly coded (\eg the \ili{Latvian} debitive’s DOM, \citealt{Serzantetal2016Differential}), and (iii) the less prominent meaning is coded with more material than the more prominent (\cf \ref{01-wi-ex:25:Russian} above). 
However, these types are not distributed equally cross-linguistically. 
Type (iii) is rarer than type (i). 
According to \citet[304]{Sinnemki2014Typological}, in the asymmetric DOM systems conditioned by \isi{topicality}, it is the topical object that receives overt marking in all cases. 
In turn, when it comes to the symmetric type (ii), the correlations mentioned in \citet{Iemmolo2013Symmetric} do not seem to represent a strong bias.


\subsection{Differential flagging \vs differential indexing}
\label{01-wi-sec:3.2-Differential}

Differential marking of arguments may be realized as head- or as dependent-marking – a difference that is largely constrained by the strategy the language uses to mark core arguments (\ie indexing only, indexing and flagging or flagging only, \citealt{Nichols1986Head-marking}). 
Thus, among others, \citet{Dalrympleetal2011Objects} treat both as different aspects of the same phenomenon. 
At the same time, indexing and flagging are often claimed to have different functions not only synchronically but also at earlier historical stages (\cf \citealt[167--168]{Croft1988Agreement}). 
While agreement or indexing is “a topic related phenomenon” as \citet[185]{Givon1976Topic} puts it (\cf also \citealt{Kibrik2011Reference}), flagging is not related to topichood or information-structure in general, but rather to semantic argument roles and various dependency relations between a head and its dependent (\cf \citealt{IemmoloPolysemy}). 
Semantic roles and various dependency relations constitute the most frequent function of cases (\cf \citealt{Blake1994Case}). 
At the same time, dependent marking can and does sometimes end up being employed for pragmatic rather than semantic purposes, as with the optional ergative marking illustrated in~\sectref{01-wi-sec:2.1.3-Non-inherent-discourse}, where one of the ergative markers is associated with continuous topichood, as in~(\ref{01-wi-ex:9a:Warrwa}), while the other occurs with a certain degree of contrast, as in~(\ref{01-wi-ex:9b:Warrwa}). 

\citet{IemmoloDifferential} is an important attempt to delineate the distinction between differential object marking (DOM) or rather differential \isi{case marking}, on the one hand, and differential object indexing (DOI), on the other. Iemmolo claims that the main distinction between the two is that DOI is related to topic continuity whereas DOM is employed to encode topic discontinuity. 
This also naturally follows from the fact that independent argument expressions (such as full NPs) are more related to topic discontinuities while verb affixes or bound pronouns are typically employed for expected referents such as continuous topics. 
It might thus be the case that the effects found in \citet{IemmoloDifferential} are due to the distinction between different referential expressions, namely, independent versus bound expressions. 

\subsection{Syntactic properties of DAM}
\label{01-wi-sec:3.3-Syntactic}

In the previous sections we have discussed morphological properties of DAM systems. 
Yet, the syntactic or behavioral properties (to use the term from \citealt{Keenan1976Towards}) of arguments in general may be heavily constrained by the morphological marking involved – an issue that has been notoriously neglected in the discussion of various DAM systems, as emphasized by \citet[17, 140--141]{Dalrympleetal2011Objects}. 
It is tacitly assumed – and perhaps correctly for many but not all instances of DAM – that concomitant to a shift in marking of an argument, the syntactic properties of that argument do not change. 
However, there are many instances in which this is not the case and differential function leads not only to differential marking but also to different syntactic properties, as \citet[140--168]{Dalrympleetal2011Objects} extensively argue for languages such as \ili{Ostyak}, \ili{Mongolian}, \ili{Chatino} and \ili{Hindi}. 
For example, marked and unmarked objects in the DOI of \ili{Ostyak} exhibit asymmetries in syntactic behavioral properties related to reference control in nominalized dependent clauses, ability to topicalize the possessor, \etc where the marked object is more of a \isi{direct object} than the unmarked \citep[17]{Dalrympleetal2011Objects}. 
To account for the differences in the syntactic properties \citet[141]{Dalrympleetal2011Objects} suggest two cross-linguistic categories (within the LFG framework, drawing on \citealt{Buttetal1996Structural}, but see already \citealt[158]{Bossong1991Differential}): the grammatically marked, topical object OBJ and the non-topical, unmarked object OBJ\textsubscript{θ} – a distinction that was originally introduced for objects of \isi{ditransitive} verbs but was extended to monotransitive objects in \citet{Buttetal1996Structural}.\footnote{Note that \citet{Buttetal1996Structural} use the labels OBJ and OBJ\textsubscript{θ} in exactly reverse functions than adopted by \citet{Dalrympleetal2011Objects}.} 
Consider Table~\ref{01-wi-tab:4:Marking} from \citet[141]{Dalrympleetal2011Objects}:

\begin{table}
\begin{tabularx}{\textwidth}{>{\hsize=.8\hsize}X>{\hsize=.2\hsize}X>{\hsize=.2\hsize}X}
\lsptoprule
	 & OBJ & OBJ\textsubscript{θ}\\ 
\midrule
Marking & Yes & No\\ 
Information-structure role & Topic & Non-topic\\ 
Properties of core grammatical functions & Yes & No\\ 
\lspbottomrule
\end{tabularx}
\caption{Marked and unmarked patient/theme objects (according to \citealt[141]{Dalrympleetal2011Objects})}
\label{01-wi-tab:4:Marking}
\end{table}

\noindent While OBJ represents the morphologically marked, discursively salient, topical objects, the extreme of the opposite case of OBJ\textsubscript{θ} would be incorporated objects, \eg in some Eastern Cushitic languages (as discussed in \citealt{Sasse1984Case}).
For example, the accusative-marked objects but not the unmarked objects in \ili{Khalkha Mongolian} (all \isi{definite} NPs and some \isi{indefinite} NPs) may be combined with the topical particle \textit{ni} (whose distribution is syntactically governed) and be fronted \citep[153–154]{Dalrympleetal2011Objects}.
Another example of differences in the syntactic properties is the \ili{Russian} partitive-accusative DOM: while (\ref{01-wi-ex:26a:Russian}) can easily be passivized, as in~(\ref{01-wi-ex:26c:Russian}), there is no passive counterpart in Standard \ili{Russian} like (\ref{01-wi-ex:26d:Russian}) that would match the meaning in~(\ref{01-wi-ex:26b:Russian}) inducing weak quantification of the object referent:

\ea\label{01-wi-ex:26:Russian}
\langinfo{Russian}{Slavic, Indo-European}{personal knowledge}
\begin{xlist}

\ex\label{01-wi-ex:26a:Russian}
	\gll Ja	vy-pil 			sok.\\
	\textsc{i.nom}	drink.\textsc{pst.m.sg}	juice.\textsc{acc.sg}/\textsc{nom.sg}\\
	\glt ‘I drank (up) the/some juice.’ [Elicited]

\ex\label{01-wi-ex:26b:Russian}
	\gll Ja	vypil			sok-a.\\
	\textsc{i.nom}	drink.\textsc{pst.m.sg}	juice.\textsc{gen.sg}\\
	\glt ‘I drank some juice.’ [Elicited]

\ex\label{01-wi-ex:26c:Russian}
	\gll Sok 			byl 		vypit.\\
	juice.\textsc{nom.sg.m} 	\textsc{aux.pst.sg.m}	drink.\textsc{pst.pass.m.sg}\\	
	\glt ‘The juice was drunk.’ [Elicited]

\ex\label{01-wi-ex:26d:Russian}
	\gll *Sok-a			byl-o 		vypito.\\
	juice-\textsc{gen.sg.m} 	\textsc{aux.pst.sg.n}	drink.\textsc{pst.pass.n.sg}\\	
	\glt [Intended meaning] ‘Some juice was drunk.’
\end{xlist}
\z

Ideally, according to the definition of DAM in~(\ref{01-wi-ex:1:DefinitionsDAM}) and~(\ref{01-wi-ex:16:NarrowDAM}), there should be no change in the syntactic behavior for an alternation to qualify as DAM. 
In case of a former dislocation, there should be no resumptive pronoun and, more generally, no other factors that would rather suggest extra-clausal status of the marked option.


\subsection{Obligatory \vs optional DAM}
\label{01-wi-sec:3.4-Obligatory}
\Citet{deHoopetal2007Fluid} distinguish between fluid DAM and split DAM. 
The former refers to constellations in which an argument in one and the same proposition may take both marking options depending on pragmatics and context. 
In turn, the latter is found when the differential marking is conditioned by inherent properties of an NP. 
Indeed, systems of DAM vary in terms of the degree of obligatoriness of a particular marking. 
Whereas in some DAM systems a particular marking applies in predictable and consistent fashion with certain types of NPs or in certain grammatical contexts, other systems seem to be more flexible (\cf \citegen{McGregor2009Typology} “split” \isi{case marking} on the one hand, and “optional” \isi{case marking} on the other). 
Thus, \citet{deSwart2006Case} reports that \isi{definiteness} may but need not be marked on objects in \ili{Hindi}. 
It is only if the speaker commits himself to the \isi{definite} interpretation that it is marked by case. 
Obligatoriness also implies that the alternative option is equally committal. 
To summarize, the principles conditioning DAM may be fully (i) obligatory (splits), (ii) obligatory-optional (split-fluid) (similar to Type 3/mixed type in \citeauthor{Dalrympleetal2011Objects}'s \citeyear{Dalrympleetal2011Objects} typology) and fully (iii) optional (fluid). 
Note that – in contrast to \citet{deHoopetal2007Fluid} and \citet{Kleinetal2011Case} – we do not attribute particular semantic domains such as \isi{definiteness} or specificity to the fluid type since there are DAM systems in which the distinction between \isi{definite} and everything else or specific and everything else is rigid. 
For example, the \isi{definite} NPs must be marked in Modern \ili{Hebrew} in terms of a fairly rigid rule, thus yielding a split. 
The three types are summarized and illustrated below.

\renewcommand{\labelenumi}{\roman{enumi}}
\begin{enumerate}
\item \textit{Splits} (obligatory complementary distribution) are common both with argu\-ment-triggered DAM, \eg in the case of differential marking of nouns \vs pronouns, as in Jingulu in~(\ref{01-wi-ex:5:Jingulu}), and with predicate-triggered DAM, such as cases of split ergativity where the form of the predicate determines the marking of the argument, as in the Georgian examples in~(\ref{01-wi-ex:19:Georgian}).
\item \textit{Fluid} DAM works solely according to probabilistic rules, as \eg the DSM restricted to negated predicates in \ili{Russian} (see \eg \citealt[300--311]{Timberlake2004Reference} and the references therein).
\item Finally, \textit{split-fluid} is a DAM system which has a combination of both splitting and fluid contexts, \ie contexts that obligatorily require a particular marking (splits) and contexts that allow for some \isi{optionality}. 
In most of the cases, \isi{optionality} is subordinate to splits. 
For example, the DOM in \ili{Persian} has rigid rule for \isi{definite} NPs which must be marked, hence, a definite-\isi{indefinite} split. 
In turn, the realm of indefinites is conditioned by various degrees of individuation \citep[183--185]{Lazard1992Grammar}, not exclusively by \isi{topicality} (pace \citealt[107--113]{Dalrympleetal2011Objects}). 
Finally, Kannada (\ili{Dravidian}) has an \isi{animate} \vs \isi{inanimate} split where animates must be marked while inanimates are either marked or unmarked depending on various additional factors \citep{Lidz2006Grammar}.
\end{enumerate}

\noindent While splits are defined in terms of rigid and simple rules, \isi{optionality} is highly complex and involves a number of often competing motivations. 
For example, in an argument-triggered DAM such as \ili{Spanish} DOM, different lexical verbs may considerably alter the preferences for DOM \citep{vonHeusingeretal2007Differential}. 
In the argument-triggered DOM of the \ili{Latvian} debitive, the preferences for \textsc{acc} \vs \textsc{nom} marking of non-pronominal NPs are also dependent on the lexical verb but not exclusively so and other factors such as the linear position (preverbal \vs postverbal) also play an important role. 
In the argument-driven DOM of \ili{Khalkha Mongolian}, \isi{definite} NPs (nouns, pronouns, proper names) are obligatorily marked, while weak \isi{indefinite} (semantically incorporated) bare NPs are obligatorily unmarked (yielding a split). 
In turn, the \isi{indefinite} NPs modified by the \isi{indefinite} article \textit{neg} are optional and tendentiously constrained by factors such as discourse persistence (whether or not the referent will be talked about in the following discourse), \isi{animacy}, \isi{affectedness}, incremental relation with the verb, specificity, \etc \citep[67]{Guntsetseg2008Differential}. 

While splits typically revolve around inherent properties, this need not always be the case. Non-inherent properties may also – albeit more rarely – yield splits. 
For example, Modern \ili{Hebrew} requires all \isi{definite} objects to carry the DOM marker \textit{’et} \citep{Danon2001Syntactic}, thus splitting all NP types of \ili{Hebrew} into \isi{definite} and \isi{indefinite} ones.\footnote{\citet[5]{Kleinetal2011Case} assume that fluid \vs split is “always” correlated with function (“result”) \vs triggers.}

\subsection{Summary}
\label{01-wi-sec:3.5-Summary}

So far we have outlined various DAM systems and their properties. 
In Section \sectref{01-wi-sec:Introduction}, we gave a broad definition of DAM (\ref{01-wi-ex:1:DefinitionsDAM}) which we recapitulate here for convenience: the term DAM broadly refers to any kind of the situation where an argument of a predicate bearing the same generalized \isi{semantic role} may be coded in different ways, depending on factors other than the argument role itself, and which is not licensed by diathesis alternations (similarly to the way it is defined in \citealt{Woolford2008Differential}, \citealt{Iemmoloetal2012Differential}). 
This definition encompasses both argument-triggered and predicate-triggered DAM systems.
 
However, it has to be acknowledged that the consensus examples are all argument-triggered DAM, \eg the DOM in \ili{Spanish}, for which we have provided the narrower definition. 
In turn, predicate-triggered DAM systems are quite different in many respects, as is summarized in~\tabref{01-wi-tab:3:Typological-variation}.
Here, DAM alternations are complementarily licensed by two distinct forms of the predicate (\eg past \vs present) and/or by two distinct syntactic positions of the predicate (embedded \vs main) – both situations do not immediately concern NP-internal properties, scenario or event semantics. 
The latter are crucial for the argument-triggered DAM. 
To capture these differences, we have provided also the narrow definition of DAM in~(\ref{01-wi-ex:16:NarrowDAM}) above, recapitulated here for convenience:

\begin{exe}
\exr{01-wi-ex:16:NarrowDAM}
Narrow definition of DAM:\\
Any kind of situation where an argument of a predicate bearing the same generalized \isi{semantic role} may be coded in different ways, depending on factors other than the argument role itself and/or the clausal properties of the predicate such as polarity, TAM, embeddedness, etc.\\
\end{exe}

\noindent Having said this, different predicate forms expressing, for example, different aspectual properties (such as perfective \vs imperfective) are indeed interrelated with such factors as event semantics, but, crucially, only indirectly (\eg in terms of \citealt{Hopperetal1980Transitivity}). 
In diachronic terms, predicate-triggered DAM systems may develop into argument-triggered ones, which suggests that these two types are not totally distinct. 
To capture potential diachronic and synchronic relations, we have introduced the distinction between the broad definition of DAM and the narrow definition. 

\section{Functional explanations for DAM}
\label{01-wi-sec:4-Functional}

In this section we will briefly survey a few common explanations of DAM. 
These explanations are directly linked to the understanding of what functions morphological marking generally serves, in particular, to the functions of \isi{case marking}. 
Note, however, that these explanations primarily concern the NP-triggered and not for the predicate-triggered DAM type, \ie only the DAM systems that satisfy the narrow definition in~(\ref{01-wi-ex:16:NarrowDAM}). 
The two most frequently mentioned functions of \isi{case marking} here are the distinguishing (also called discriminatory or disambiguating) function and the identifying (also called highlighting, indexing or coding) function (\cf \citealt{Dixon1979Ergativity, Dixon1994Ergativity, Mallinsonetal1981Language, Comrie1989Language, Song2001Linguistic, Hoopetal2008Case-marking, Siewierskaetal2009Case}; \citealt[3--8]{Dalrympleetal2011Objects}). 
The distinguishing function of \isi{case marking} serves the purposes of disambiguation of the argument roles in clauses with two or more arguments. 
Case marking fulfills the identifying function in that it codes the semantic relationship that the argument bears to its verb. 
In what follows, these two functions are presented in further details and linked to particular configurations of argument marking.

In the identifying approach to the function of DAM, the presence of a marker on an argument is independent of the relationship between the arguments of a clause. 
Instead, a particular marker is viewed as a device to highlight more fine-grained distinctions of the same \isi{semantic role} (volitional \vs non-volitional agents, affected \vs non-affected patients, controlling \vs non-controlling experiencers, etc.) or various properties of the argument itself (\eg in \citealt{Hopperetal1980Transitivity, Dalrympleetal2011Objects}). 
For instance, for \citealt[1206]{Naess2004What}, the relevant property triggering overt object marking is \isi{affectedness} of the argument. 
Affectedness is, in turn, defined by employing two other concepts: the concept of part-whole relations and of salience. 
In terms of part-whole relations, an entity of which only a subpart is affected is generally less affected than one affected as a whole. 
The concept of salience relies on the assumption that some types of effects are more easily perceptible and of greater interest to humans than others \citep[1202]{Naess2004What}.

Recently, \citet{Sinnemki2014Typological} has claimed that neither \isi{animacy} nor \isi{definiteness} are the universal factors conditioning DOM. 
Thus, \citet[295]{Sinnemki2014Typological} argues that “there is a crosslinguistic dispreference for object \isi{case marking} to be driven by \isi{animacy}.” 
His study shows that only 47 (= 39\%) of genealogical units in his sample had an animacy-effect as opposed to 61\% of genealogical units in which \isi{animacy} was not the conditioning factor. 
Analogically, there were 34\% (43 genealogical units) affected by \isi{definiteness} (with an areal bias for the Old World) as opposed to 66\% (83 genealogical units) which were not. 
Both factors are found to condition DOM in 58\% (70 genealogical units) as opposed to 42\% (51 genealogical units) which are conditioned by some other factors \citep[296]{Sinnemki2014Typological}. 
However, problematic in \citegen{Sinnemki2014Typological} account is that he did not only consider argument-triggered DOM systems for which the predictions mentioned above were designed but also predicate-triggered DOM systems such as those conditioned by split ergativity. 
Moreover, crucially, the DOM systems in 42\% of genealogical units are not conditioned by one single factor but instead by a variety of factors, including tense/aspect, singular \vs plural, gender, \etc (\cf the ones listed in \citealt[284--285]{Sinnemki2014Typological}). 
Notably, the strengths of each of these are not even remotely similar to either \isi{animacy} (39\%) or \isi{definiteness} (34\%), let alone \isi{animacy} and \isi{definiteness} together.

In turn, \citet{Dalrympleetal2011Objects} claim that DOM is primarily motivated by the \isi{information structure}. 
According to them, DOM is used to highlight “similarities between subjects and topical objects” \citep[3--8]{Dalrympleetal2011Objects} and to delineate topical objects and (generally topical) subjects as primary arguments from other, less canonical arguments. 
\citet{Dalrympleetal2011Objects} make an important claim based on corpus frequencies that objects are as likely to be topics as foci or parts of foci and that focus, therefore, is not the most typical information-structural role of an object, as in previous accounts.

The distinguishing function of \isi{case marking} always operates together with the two more general principles responsible for coding asymmetries: economy and \isi{markedness} (\cf \citealt{deSwart2006Case, deSwart2007Cross-linguistic, Hoopetal2008Case-marking}). 
In particular, the principle of economy requires arguments to be unmarked to reduce the speaker’s efforts. 
In turn, the distinguishing function forces the speaker to mark at least one of the arguments to achieve their distinguishability from each other, although the choice of the argument to be marked is not arbitrary and is predicted by \isi{markedness}: the most marked combination of the filler and the syntactic slot of the verb’s arguments will have a longer morphological marking. 
Markedness here is based on the intuition of what represents the most natural monotransitive clause (\eg the most frequent clause type in actual discourse) in terms of its arguments. 
\citet{Comrie1989Language} summarizes this intuition as follows:

\begin{quote}
"[…], the most natural kind of transitive construction is one where the A is high in \isi{animacy} and \isi{definiteness}, and the P is lower in \isi{animacy} and \isi{definiteness}; and any deviation from this pattern leads to a more marked construction." \citep[128]{Comrie1989Language}
\end{quote}

This account thus predicts that \isi{animate} and/or \isi{definite} objects, which represent a less natural (\ie more marked) combination of role and semantic features, should be marked formally, \eg with an overt case marker (or by some other means, \eg a passive or inverse construction), while \isi{inanimate} and/or \isi{indefinite} objects, which manifest a natural combination, need not be marked overtly (\cf \citealt[128]{Comrie1989Language}; \citealt[162--163]{Bossong1991Differential}; \citealt{Malchukov2008Animacy}). 

There are several operational definitions for functional \isi{markedness}. 
% \citet[10]{Bickeletal2015Typological} 
\citet[10]{Bickeletal2015Typological}, for example, 
adopt the interpretation of \isi{markedness} in the context of DSM following \citegen{Silverstein1976Hierarchy} lead. 
They speak of \isi{markedness} relations and operationalize them in terms of the alignment of argument roles: the sets that also include the S argument role (\ie \{S, P\}, \{S, A\} and \{S, A, P\}) are all less specific and thus less marked in comparison to the sets \{A\} and \{P\}. 
They test the often claimed effects of various referential hierarchies, such as the ones in~\tabref{01-wi-tab:1:semantic}. High-ranking A and low-ranking P arguments are then expected to be associated with the more general sets, \ie \{S, A, P\} or \{S, A\} for high-ranking A arguments, and \{S, A, P\} or \{S, P\} for low-ranking P arguments.

A more radical view is \citet{Haspelmath2006Against} who discards \isi{markedness} altogether, replacing it with frequency-based expectations. 
This approach can be straightforwardly related to DAM because it provides a falsifiable account of asymmetries mentioned above. 
For example, \isi{animate} direct objects are much less frequent and, hence, less expected to occur, while objects are typically \isi{inanimate}. 
For example, \citet[51]{Dahetal1996Animacy} and \citet{Dahl2000Egophoricity} demonstrate that the proportion of \isi{animate} \vs \isi{inanimate} direct objects in corpus of written \ili{Swedish} is 87\% \isi{inanimate} while in spoken \ili{Swedish} 89\% \isi{inanimate} \vs 13\% and 11\% \isi{animate} NPs, respectively (analogical proportions are found in English and \ili{Portuguese}, \cf \citealt[6, 12]{Everett2009Reconsideration}). 
This means that \isi{animate} objects are less expected to occur. 
This is claimed to be the reason why they require more marking than \isi{inanimate} objects which are expected anyway. In turn, the A position seems to be less biased. 
There are 56\% human A’s \vs 44\% non-humans in the same corpus \citep[51]{Dahetal1996Animacy}.

Systems of \isi{case marking} fulfilling a purely distinguishing function are infrequent synchronically (\citealt{deHoopetal2005Differential}; \citealt[569]{Hoopetal2008Case-marking}). 
These are the systems of the kind described in~\sectref{01-wi-sec:2.1.5-Scenario-global-local} under scenario; apart from \ili{Aguaruna}, other known examples are \ili{Awtuw} \citep{Feldman1986Grammar} or \ili{Hua} \citep{Haimann1979Hua}. 
Contrary to what one would expect from the perspective of the distinguishing function, in the majority of DAM systems a particular argument marking applies mechanically across the board and is not restricted to marking arguments only in contexts of actual ambiguity (\cf \citealt[213]{Malchukov2008Animacy}). 
However, even in these cases, the distinguishing function does seem to be operative in the background because DAM rarely leads to syntactic ambiguities here. 

More generally, DAM provides a means for speakers to differentiate between various additional factors that are themselves secondary to the event and do not considerably alter the state of affairs. 
The exact semantic and/or pragmatic component that a particular DAM system contributes is sometimes difficult to discern precisely because differential marking does not significantly change the interpretation of the event. 
In turn, the versatility of DAM systems is smoothened by the simple, mostly binary opposition between two case-marking strategies which may be either complementarily distributed or one marking may be the semantic default that may be used in all contexts.

\section{Conclusions}
\label{01-wi-sec:5-Conclusions}

Differential marking is a pervasive phenomenon cross-linguistically. 
Thus, \citet[297]{Sinnemki2014Typological} shows that, independently of genealogical and areal factors, the asymmetric DOM (“restricted marking” in \citealt{Sinnemki2014Typological}) is found in the overwhelming majority of languages that employ flagging of objects: 74\% of all genealogical units in his large-scale study on DOM involving 744 languages attest splits in the object marking where only a subset of objects is overtly marked. 

Moreover, this phenomenon is highly versatile. 
We have suggested that the two main types of DAM systems are the \textsc{argument-triggered dam} and the \textsc{predicate-triggered dam} with various subtypes: while the former is primarily sensitive to the interpretation of the respective participant (its semantic and pragmatic properties), the latter responds to the properties of the event: \eg whether the event is seen as perfective or imperfective, whether it takes place in the past or in the present, whether it is referential or modal, construed as independent (and hence coded by the main predicate) or as in some way dependent on another event (and hence coded by an embedded predication), etc. 
It is only the argument-triggered type that falls under the \textsc{narrow definition} of DAM (in~(\ref{01-wi-ex:16:NarrowDAM}) above) and has been at heart of research on DAM.

Orthogonally to this distinction we made the distinction between \textsc{restricted} \vs \textsc{unrestricted dam} systems. 
The former ones are found if the DAM system does not apply across the board but is limited to specific contexts such as particular constructions or particular verbs; the latter, in turn, have no such restrictions. 
Crucially, most of the functional explanations of DAM revolved around the argument-triggered DAM systems and are not applicable to the predicate-driven type.

Furthermore, DAM systems may be classified into \textsc{split, fluid and split-fluid} systems, depending on the degree of obligatoriness and complementarity of the markers.

\section*{Acknowledgements}

We thank Marco García García, Martin Haspelmath, John Peterson, Karsten Schmidtke-Bode and Björn Wiemer for useful comments and remarks. 
All disclaimers apply.









\section*{Abbreviations}
\begin{tabularx}{.45\textwidth}{lQ}
\textsc{aor}		aorist\\
\textsc{att}		attenuative \textit{Aktionsart}\\
\textsc{deb}		debitive (necessity) mood\\
h		honorific\\
\textsc{hiaf}		high affectedness \textit{Aktionsart}\\
\textsc{int}		interrogative\\
\textsc{intens}	intensifier\\
f\textsc{erg}		focal ergative (as opposed to the non-focal ergative)\\
\end{tabularx}
\begin{tabularx}{.45\textwidth}{lQ}
min		minimal number\\
\textsc{narr}		narrative case\\
nh		non-honorific\\
\textsc{rel}		subject relativizer\\
\textsc{rem}		remote\\
\textsc{restr}	restrictive\\
\textsc{sap}		speech act participant\\
\textsc{tr}		transitive conjugation marker\\
\end{tabularx}

{\sloppy
\printbibliography[heading=subbibliography,notkeyword=this] }
\end{document}
