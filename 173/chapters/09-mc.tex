\documentclass[output=paper]{LSP/langsci}
\ChapterDOI{10.5281/zenodo.1228259} 
\author{William B. McGregor\affiliation{Aarhus University}
} 
\title{Emergence of optional accusative case marking in Khoe languages}

\abstract{A number of languages of the Khoe family – one of three genetic lineages comprising southern African Khoisan – show an accusative marker, typically a postposition which in its elsewhere form has the shape \textit{(-)(ʔ)à}. In all languages for which adequate data is available, this postposition is optional on object NPs, at least in some circumstances. A few proposals have been made for the grammaticalisation of this marker, notably by Kilian-Hatz (\citeyear[55]{Kilian-Hatz2008Grammar}; \citeyear[376–378]{Kilian-Hatz2013Kxoe}). However, not only are these proposals specific to the Khwe language, but also they fail to account for the fact that \textit{(-)(ʔ)à} marks the accusative and that it is optional. In this paper I widen the net to the Khoe family as a whole, and consider the synchronic situations for the usage of the marker \textit{(-)(ʔ)à} and its putative cognates in those languages for which pertinent data is available. This is used to motivate a diachronic proposal concerning the grammaticalisation of \textit{(-)(ʔ)à} in the modern languages. Specifically, it is proposed that the accusative marker began life as a presentative copula; this served to index an item, drawing the addressee’s attention to it. It later became an optional accusative marker via grammaticalisation processes akin to those outlined in \citet{McGregor2008Indexicals,McGregor2010Optional,McGregor2013Optionality,McGregor2017Grammaticalization} for the development of optional ergative case markers in some Australian languages. Thus the grammaticalisation scenario proposed is consistent with pathways of development of other optional case-markers.

%  Keywords
%  
%  \keywords{accusative case-marking; grammaticalisation; Khoe family; optional case-marking; attention; prominence; backgrounding}
}
\maketitle

\begin{document}

\section{Introduction}
\label{09-mc-sec:1}
\subsection{Aims}
\label{09-mc-sec:1-1}


This paper is concerned with the grammaticalisation of optional accusative marking in the Khoe languages of southern Africa. I argue that the accusative marker, which generally takes the elsewhere form \textit{(-)(ʔ)à} in Khoe languages, began as a copula in presentative clauses (\citealt[90–91, 307–310]{McGregor1997Semiotic}); see also Kilian-Hatz (\citeyear[55]{Kilian-Hatz2008Grammar}, \citeyear[376–377]{Kilian-Hatz2013Kxoe}); \citet[276]{Koenig2008Case} for a similar suggestion. This was also employed to draw attention to certain NPs in verbal clauses, especially unexpected or atypical objects of transitive clauses. It subsequently developed into an optional accusative marker in most of the languages. This scenario is supported on the one hand by an examination of the range of synchronic uses of the accusative marker and possible cognates, and on the other by evidence from the grammaticalisation of other optional case markers, in particular optional ergative markers, as discussed in \citet{McGregor2008Indexicals,McGregor2010Optional,McGregor2013Optionality}.

The paper is organised as follows. After providing an outline of the lineage, and sources of information on the languages in~\sectref{09-mc-sec:1-2}, the subsequent two sections set the scene for the grammaticalisation scenario proposed in~\sectref{09-mc-sec:4}. \sectref{09-mc-sec:2} presents a detailed overview of the uses of the marker \textit{(-)(ʔ)à}, including possibly homophonous and/or cognate morphemes in those Khoe languages for which information is available. Following this, \sectref{09-mc-sec:3} presents a discussion of the motivations that have been proposed for the choice between using and not using the accusative marker on object NPs in a small selection of Khoe languages – for the majority of the languages information on this issue is not available. The paper is wound up in~\sectref{09-mc-sec:5} with a brief conclusion.

\subsection{The languages and sources of information} \label{09-mc-sec:1-2}

Khoe is a branch of the putative \ili{Khoe-Kwadi} family (Greenberg’s Central \ili{Khoisan}) (\citealt{Gldemann2004Reconstruction,Guldemannetal2010External,Vossen1997Khoe}; \citealt[10--11]{Vossen2013Khoisan}), one of three distinct \ili{Khoisan} lineages found in southern Africa. A tentative tree for \ili{Khoe-Kwadi} is shown in \figref{09-mc-fig:1}.

\begin{figure}
\resizebox{\textwidth}{!}{%
\begin{forest}
[\ili{Khoe-Kwadi}
  [Kwadi]
  [Khoe
    [\ili{Khoekhoe}
      [Nama-Damara{,} Haiǁ'om{,}\\ ǂAakhoe{,} Eini{,} ǃOra{,}\\ Cape varieties,roof]
    ]
    [Kalahari Khoe
      [East
        [\ili{Shua}
          [Cara{,} ǀXaise{,}\\Deti{,} \ili{Danisi}{,}\\Nata \ili{Shua}{,}\\Ts'ixa,roof]
        ]
        [Tshwa
          [Kua{,} Cua{,}\\Tsua{,}\\ Tyiretyire,roof]
        ]
      ]
      [West
        [Kxoe
          [\ili{Khwe}{,} ǁAni{,}\\Buga{,} Gǀanda {,}\\ǁXom{,} ǁXo,roof]
        ]
        [Gǁana
          [Gǁana{,} Gǀui{,}\\Tshila,roof]
        ]
        [Naro
          [Naro{,} ǂHaba{,}\\Ts'ao,roof]
        ]
      ]
    ]
  ]
]
\end{forest}
}
\caption{A possible tree for the Khoe-Kwadi lineage (based on \citealt[27]{Gueldemann2014Khoisan})}\label{09-mc-fig:1}
 \end{figure}
 

It should be noted, however, that there are a number of uncertainties: not all specialists agree that the evidence convincingly supports Kwadi as a sister of Khoe; the referents of the terms for many varieties are uncertain (\eg Tyiretyire (Cirecire) and its relation to Cua); the placement of some varieties is tentative (\eg of Ts’ixa as a \ili{Shua} variety); and \ili{Khoekhoe} is sometimes divided into north and south (\eg \citealt[102]{Guldemannetal2000Khoisan}, \cf \citealt[28]{Gueldemann2014Khoisan}).

The main languages dealt with in this paper, along with the primary sources of information on them, are listed in \tabref{09-mc-tab:1}.\footnote{In the remainder of the paper the term \textit{Shua} will be used in reference to the variety spoken in Nata. Where reference is made to the set of varieties in the subgroup I will speak of \textit{\ili{Shua} varieties}.} There is insufficient information on Kwadi to permit its inclusion in this study. Otherwise, all of the Khoe groups and subgroups are represented by at least one language; unfortunately, however, the data available for some subgroups is seriously inadequate.

\begin{table}
\caption{Languages and main sources of data\label{09-mc-tab:1}}
\begin{tabular}{lll}
% \begin{tabularx}{\textwidth}{llX}
%\begin{tabularx}{>{\hsize=.4\hsize}X>{\hsize=1.6\hsize}X}
\lsptoprule
Group & Language & Main sources\\
\midrule
\ili{Khoekhoe} & Nama-Damara & \citet{Hagman1973Nama}\\
& ǃOra & \citet{Haacke2013Gora}\\
\midrule
East Kalahari Khoe & Nata \ili{Shua} & Own fieldnotes\\
& \ili{Danisi} & Fieldnotes Fehn \& McGregor; \citet{Vossen2013Introduction}\\
& Ts’ixa & \citet{Fehn2014Grammar}\\
& Tyiretyire & Own fieldnotes\\
\midrule 
West Kalahari Khoe & \ili{Khwe} & \citet{Kilian-Hatz2008Grammar,Kilian-Hatz2013Kxoe}\\
& ǁAni & \citet{Heine1999Ani}\\
& Gǀui & \citet{Ono2011Gui}\\
& Naro & \citet{Visser2013Naro}\\
\lspbottomrule
% \end{tabularx}
\end{tabular}
\end{table}


\section{The marker(s) \textit{(-)(ʔ)à} in Khoe languages} \label{09-mc-sec:2} 

One or more grammatical markers showing the shape \textit{(-)(ʔ)à} are attested in all Khoe languages that have been sufficiently well described;\footnote{Sources are inconsistent in representing an initial glottal stop. In some languages two distinct allomorphs exist, one with and one without an initial glottal stop. Various other allomorphs are found in particular languages, including allomorphs with different vowel shapes (usually conditioned by preceding segments) and fused allomorphs (often morphologically conditioned by a preceding person-gender-number marker or pronoun). Discussion of the allomorphy is beyond the scope of the present paper, although it is clearly crucial to a complete and convincing grammaticalisation story.} these are found in languages of all three branches. There are a number of differences in the range of uses of these markers across the languages, as shown in \tabref{09-mc-tab:2}–\tabref{09-mc-tab:4}.  Note that these tables identify grammatical uses of morphemes with the shape \textit{(-)(ʔ)à}, regardless of whether or not they represent different uses of a single morpheme or distinct morphemes – which in many cases is not known for certain.\footnote{For this reason, I adopt the convention of glossing \textit{(-)(ʔ)à} according to its putative function, rather than with a single gloss, except where the evidence indicates that a single morpheme is involved. It should not, of course, be presumed that each gloss corresponds to a different, homophonous morpheme, although it may.}\textsuperscript{,}\footnote{Kilian-Hatz states explicitly that there is a single morpheme \textit{(ʔ)à} in \ili{Khwe} with the range of senses indicated in \tabref{09-mc-tab:2}–\tabref{09-mc-tab:4} (\citeauthor{Kilian-Hatz2008Grammar}  \citeyear[52–53]{Kilian-Hatz2008Grammar}, \citeyear[368]{Kilian-Hatz2013Kxoe}). Whether or not this proposal is viable remains unclear to me.} Nor has it yet been established that the morphemes are all cognate. Moreover, the listing is incomplete. For expository purposes I have been selective, and excluded those functions (and possibly morphemes) that are irrelevant to the grammaticalisation scenario proposed in this paper. For instance, most Khoe languages have a verbal juncture morpheme, one allomorph of which is -\textit{a} \citep{Vossen2010Verbal}. Whether or not this morpheme is cognate with \textit{(-)(ʔ)à}, it plays no role in the grammaticalisation scenario proposed in~\sectref{09-mc-sec:4}.

\begin{table}[t]
\caption{NP role marking functions of morpheme or morphemes \textit{(-)(ʔ)à}\\
(1 (almost) certainly a use of the form in the language: either attested or implied by the description; 0 evidence suggests not a use of the form in the language; — unattested use in the language, though information is insufficient to determine whether it is a possible use.)}\label{09-mc-tab:2}

{\small \begin{tabularx}{\textwidth}{XXXXXXXX}
\lsptoprule
{\bfseries Language} & {\bfseries Object} & {\bfseries Indirect Object} & {\bfseries Subject} & {\bfseries Topic} & {\bfseries Focus} & {\bfseries Locative} & {\bfseries Genitive}\\
\midrule
\ili{Shua} & 1 & 1 & 0 & 0 & 0 & 1 & —\\
Ts’ixa & 1 & 1 & 0 & 0 & 0 & 1 & —\\
Tyiretyire & 1 & 1 & 0 & 0 & — & — & —\\
\ili{Khwe} & 1 & 1 & 1 & 0 & 1 & 1 & 1\\
ǁAni & 1 & 1 & 1 & 1 & — & 0 & 0\\
Buga & 1 & 1 & — & 0 & — & 0 & 0\\
Naro & 1 & 1 & — & 0 & — & 0 & 0\\
Gǀui & 1 & 1 & — & 0 & 0 & 0 & 0\\
Nama-Damara & 1 & 1 & 1 & 0 & 0 & 1 (temp) & 0\\
!Ora & 1 & 1 & 1 & — & — & — & —\\
Haiǁom & 1 & 1 & 1 & — & — & — & —\\
\lspbottomrule
\end{tabularx}}
\end{table}

\begin{table}[t]
\caption{Other types of NP marking by morpheme(s) \textit{(-)(ʔ)à}}\label{09-mc-tab:3}
{\small \begin{tabularx}{\textwidth}{Xccc}
\lsptoprule
{\bfseries Language} & {\bfseries Dislocated NPs} & {\bfseries Appositive NP Os} & {\bfseries Governed by postposition}\\
\midrule
\ili{Shua} & 0 & — & 0\\
Ts’ixa & 0 & — & 0\\
Tyiretyire & 0 & — & 0\\
\ili{Khwe} & 0 & 1 (attributing) & 1\\
ǁAni & 0 & — & 0\\
Buga & — & — & —\\
Naro & — & — & 0\\
Gǀui & 1 & 1 (identifying) & 0\\
Nama-Damara & — & — & 1\\
!Ora & — & — & —\\
Haiǁom & — & — & —\\
\lspbottomrule
\end{tabularx}}
\end{table}


\begin{table}[t]
\caption{Other uses of morpheme(s) \textit{(-)(ʔ)à}}\label{09-mc-tab:4}
{\small \begin{tabularx}{\textwidth}{XCCCC}
\lsptoprule
{\bfseries Language} & {\bfseries Relational copula} & {\bfseries Presentative copula} & {\bfseries Clausal connector} & {\bfseries Extraposed elements}\\
\midrule
\ili{Shua} & 0 & 0 & — & 0\\
Ts’ixa & 0 & 0 & — & 0\\
Tyiretyire & 0 & 0 & — & 0\\
\ili{Khwe} & 1 & 1 & — & 0\\
ǁAni & 1 & — & — & 0\\
Buga & — & — & — & —\\
Naro & 0 & 0 & 1 & —\\
Gǀui & 1 & — & — & 1\\
Nama-Damara & 1 & — & 1 & —\\
!Ora & 1 & 1 & — & —\\
Haiǁom & 1 & — & — & —\\
\lspbottomrule
\end{tabularx}}
\end{table}

\largerpage[-1]
Two functions are universally associated with \textit{(-)(ʔ)à} in Khoe languages. First, in every language \textit{(-)(ʔ)à} is attested as a marker of both direct objects and indirect objects. This is illustrated in the \ili{Khwe} example~\REF{09-mc-ex:1}, where the marker is a free postposition, as in other Kalahari Khoe languages. In at least some of these languages there is an allomorph that fuses with a preceding pronoun or person-gender-number (PGN) marker, a portmanteau morph attached to a nominal and encoding its person, grammatical gender and number. For instance, in \ili{Shua} one finds \textit{Pitama:}\textit{ˍ} {\textasciitilde} \textit{Pita-maˍ-ʔaˍ} (Peter-\textsc{M}-\textsc{acc}) ‘Peter’ and \textit{ta}\textit{ˉ}\textit{a}\textit{ˍ} {\textasciitilde} \textit{ta:}\textit{ˉ{}-ʔ}\textit{a}\textit{ˍ} (1\textsc{sg}-\textsc{acc}) ‘me’. In \ili{Khoekhoe} the corresponding marker is a suffix, as shown by the Nama-Damara example~\REF{09-mc-ex:2}.\footnote{\citegen{Haacke2013Gora} construal of the morpheme \emph{-à} as an oblique marker seems preferable to \citegen{Hagman1973Nama} construal as a subordinate case marker, and I adopt it in this paper.}

  
\begin{exe}
\ex\label{09-mc-ex:1}
\langinfo{Khwe}{West Kalahari Khoe}{\citealt[374]{Kilian-Hatz2013Kxoe}}\\
\gll màtìaci-\`{m} à ǀ’áò à tí xàró-ná-tà \\
Matthew-3\textsc{sg.m} \textsc{acc} money \textsc{acc} 1\textsc{sg} give-\textsc{j}-\textsc{pst}\\
\glt ‘I gave money to Matthew.’
\end{exe}

\begin{exe}
\ex\label{09-mc-ex:2}
\langinfo{Nama-Damara}{Khoekhoe}{\citealt{Hagman1973Nama}}\\
\gll ʔáo-p ke tará-s-à péré-p-a kè màa\\
man-3\textsc{sg.m} \textsc{decl} woman-3\textsc{sg.f}-\textsc{obl} bread-3\textsc{sg.m}-\textsc{obl} \textsc{pst} give\\
\glt ‘The man gave the woman bread.’ 
\end{exe}

As example~\REF{09-mc-ex:3} shows, in \ili{Shua} the erstwhile beneficiary (an \isi{indirect object}) in an applicative construction is marked by the \textsc{acc} \textit{(ʔ)à}; this marking may be retained under passivisation, as shown by~\REF{09-mc-ex:4}.

\ea
\langinfo{Shua}{East Kalahari Khoe}{own fieldnotes}\\

\ea\label{09-mc-ex:3}
\gll taa ʔa pii tyana-ma\\
1\textsc{sg} \textsc{acc} milk bring-\textsc{appl}\\
\glt ‘Bring me some milk.’

\ex\label{09-mc-ex:4}
%\langinfo{Shua}{East Kalahari Khoe}{own fieldnotes}\\
\gll tse: ʔa aka k’ohu ngǀo:-a-ma-e-ha\\
1\textsc{pl.c} \textsc{acc} \textsc{pst} meat cook-\textsc{j}-\textsc{appl}-\textsc{pass}-\textsc{pst}\\
\glt ‘The meat was cooked for us.’
\z
\z

Second, in both \ili{Khoekhoe} and West Kalahari Khoe \textit{(-)(ʔ)à} is widely attested as a relational copula, that is, as a copula in attributing and/or identifying clauses. Example~\REF{09-mc-ex:5} illustrates this usage in the West Kalahari Khoe language ǁAni.\footnote{\citet{Heine1999Ani} equivocates on the status of this marker as a morpheme distinct from the \textsc{acc} marker.} For Nama-Damara \citet[114–116, 164]{Hagman1973Nama} identifies a present tense copula \textit{ʔa} that is used in attributing clauses, as shown by example~\REF{09-mc-ex:6}. He also identifies a suffix \textit{-à} that is used as a marker of the “predicate” in identifying clauses \citep[110]{Hagman1973Nama}, as in example~\REF{09-mc-ex:7}; this also appears to exemplify a copula function (see also \citealt[174]{Vossen1997Khoe} and \citealt[342]{Haacke2013Gora} for brief remarks on ǃOra.)

\begin{exe}
\ex\label{09-mc-ex:5}
\langinfo{ǁAni}{West Kalahari Khoe}{\citealt[24]{Heine1999Ani}}\\
\gll kx’oxu tshaa-kx’oxu-dzi ʔa \\
animal water-animal-3\textsc{pl.f} \textsc{cop}\\
\glt ‘Water animals are edible [are meat].’ 
\end{exe}

\ea \langinfo{Nama-Damara}{Khoekhoe}{\citealt[116]{Hagman1973Nama}}\\
\ea\label{09-mc-ex:6}
\gll saá-ts ke ʔa káí \\
2-2\textsc{sg.m} \textsc{decl} \textsc{cop} big\\
\glt ‘You are big.’ 

\ex\label{09-mc-ex:7}
%\langinfo{Nama-Damara}{Khoekhoe}{\citealt[116]{Hagman1973Nama}}\\
\gll saá-ts ke káí-ts-a \\
2-2\textsc{sg.m} \textsc{decl} big-2\textsc{sg.m}-\textsc{cop}\\
\glt ‘You are the big one.’ 
\z
\z

Copula usage may also be available in Gǀui. \citet[400]{Nakagawa2013Gana} speaks of a linking use of \textit{-à} that incorporates a body part nominal into a complex adjective, as in~\REF{09-mc-ex:8}.

\ea \langinfo{Gǀui}{West Kalahari Khoe}{\citealt{Ono2011Gui}}\\ \label{09-mc-ex:8}
\gll ʔàbì !ái͂ ja ǂáó\\
3\textsc{sg.m} good \textsc{cop} heart\\
\glt ‘He is good in the heart.’ \ie ‘He is happy.’
\z
An alternative and more plausible analysis is that \textit{ja} (an allomorph of \textit{à}) does not link the adjective ‘good’ with the nominal ‘heart’ into a complex adjective, but rather functions as a clausal copula in an external possession construction of the double subject type. That is to say, in~\REF{09-mc-ex:8} goodness is attributed of the person, and the following body part nominal indicates a restriction of the attribute to the person’s heart – they are good in, or with respect to, the heart.

In addition to its use as a relational copula, in \ili{Khwe} and !Ora \textit{(ʔ)à} can be used as a presentative or existential copula (Tom Güldemann p.c.; \citealt[52]{Kilian-Hatz2008Grammar}), as shown by examples~\REF{09-mc-ex:9} and~\REF{09-mc-ex:10}.\footnote{Strictly speaking, this is of course not a copular function in that it does not connect linguistic forms; indeed, it better resembles the index \textit{there} of the English presentative/existential than the copula \textit{be}. However, I follow usual convention and use the term \textit{copula} loosely in this fashion; it is not unreasonable in the sense that what is linked is the addressee’s attention and the referent item.}

\ea \langinfo{Khwe}{West Kalahari Khoe}{\citealt[251]{Kilian-Hatz2013Kxoe}}\\
\ea\label{09-mc-ex:9}  
\gll thíyà goava à \\
many Mbukushu \textsc{cop}\\
\glt ‘There are many Mbukushu.’
 
\ex\label{09-mc-ex:10} 
%\langinfo{Khwe}{West Kalahari Khoe}{\citealt[251]{Kilian-Hatz2013Kxoe}}\\
\gll ǁhaḿ! teá ǁeí-coava à! \\
come.near 2\textsc{sg.m} skin-be.rotten \textsc{cop}\\
\glt ‘Come near! Here is your rotten skin [\ie your food]!’
\z
\z

The copula function is not attested in any East Kalahari Khoe language to the best of my knowledge.

Other uses of \textit{(-)(ʔ)à} are rather sporadically distributed across the Khoe languages, at least given the existing evidence. I briefly overview these additional senses.

In addition to marking direct and indirect objects, in \ili{Khwe}, ǁAni, Nama-Damara and !Ora \textit{(-)(ʔ)à} occurs on subject NPs as well, albeit rarely. In the latter two languages \textit{-à} occurs on what is referred to as a “deposed” subject, that is, a subject that does not occur in its usual clause-initial position, as in example~\REF{09-mc-ex:11}. Unfortunately, the descriptions do not make entirely clear either the formal properties of this construction or its meaning and uses (\citealt[203]{Hagman1973Nama}; \citealt[341]{Haacke2013Gora}, \citeyear[328]{Haacke2013Namibian}).

\begin{exe}
\ex\label{09-mc-ex:11} 
\langinfo{Nama-Damara}{Khoekhoe}{\citealt[328]{Haacke2013Namibian}}\\
\gll tsî-b ge axa-b-a ǀôa-s-a tsaurase go ǂai\\
and-3\textsc{sg.m} \textsc{ind} boy-3\textsc{sg.m}-\textsc{obl} girl-3\textsc{sg.f}-\textsc{obl} gently \textsc{pst} call\\
\glt ‘And then he, the boy, gently called the girl.’ 
\end{exe}

In Nama-Damara \textit{-à} also regularly occurs on subjects of some clauses in marked moods, the interrogative and imperative/hortative (\citealt[260, 270–271]{Hagman1973Nama}).

In \ili{Khwe} subjects of both relational and verbal clauses can be followed by \textit{(ʔ)à}, though only when \isi{indefinite}; for subjects of transitive clauses this marking is extremely rare (\citealt[51–52]{Kilian-Hatz2008Grammar}, \citeyear[369–371]{Kilian-Hatz2013Kxoe}), slightly more common for \isi{intransitive} subjects. \citet[370]{Kilian-Hatz2013Kxoe} considers that in these contexts \textit{(ʔ)à} serves as a \isi{focus marker} rather than as a subject marker, as in~\REF{09-mc-ex:12}.

\ea\langinfo{Khwe}{West Kalahari Khoe}{\citealt[370]{Kilian-Hatz2013Kxoe}}\\\label{09-mc-ex:12} 
\gll kúcugucugu à ǁgèvùu-à-tè \\
eagle \textsc{foc} fly-\textsc{j}-\textsc{pst}\\
\glt ‘An eagle is flying.’ 
\z

More generally, Kilian-Hatz considers that the primary NP marking function of \textit{(ʔ)à} in \ili{Khwe} is to mark focus, regardless of what other grammatical role is simultaneously borne by the NP, whether it be a core grammatical relation or the locative – see especially Kilian-Hatz (\citeyear[54]{Kilian-Hatz2008Grammar}, \citeyear[370, 377]{Kilian-Hatz2013Kxoe}). A secondary function is to mark the object; this happens only (in Kilian-Hatz’s view) in those circumstances in which the object is obligatorily marked by \textit{(ʔ)à}, namely on proper noun objects and on indirect objects other than specific nouns (see further \sectref{09-mc-sec:3-3} below). Consistent with this, only one NP in a clause is normally marked by \textit{(ʔ)à}. In no other Khoe language has it been suggested that \textit{(-)(ʔ)à} is a general \isi{focus marker}. By contrast, \citet[31, 67, 68]{Heine1999Ani} suggests that \textit{(ʔ)à} in ǁAni serves as a topic marker, perhaps primarily. He does not, however, explain what he means by “topic” and the examples given could as well be interpreted as invoking focus on the marked NP.

In both \ili{Shua} and Ts’ixa \textit{(ʔ)à} marks a general \isi{locative case}, as shown in~\REF{09-mc-ex:13}. However, in both languages this form represents a different postposition to the \textsc{acc} marker: it shows different allomorphy and occurs with a different case form of the PGN markers (\citealt[202]{Fehn2014Grammar}; \citealt{McGregor2015Optional}).

\ea \langinfo{Ts’ixa}{East Kalahari Khoe}{\citealt[202]{Fehn2014Grammar}}\\\label{09-mc-ex:13} 
\gll kolóí=sí ǁʔáé=\`{m} ʔà téè \\
car=\textsc{sg.f:i} village=\textsc{sg.m:i} \textsc{loc} be.standing\\
\glt ‘The car stands in the village.’
\z

In Nama-Damara \textit{-à} occurs on NPs indicating time units, marking temporal duration (\citealt[112, 199]{Hagman1973Nama}).

In \ili{Khwe} there is a \isi{genitive case} suffix \textit{-à} that is used in the expression of part-whole relations within NPs when the whole (modifying) nominal is \isi{indefinite} (\citealt[77]{Kilian-Hatz2008Grammar}). Examples are \textit{gù-á ǀ’ṹũ} (sheep-\textsc{gen} hair) ‘sheep’s wool’, \textit{xúni-a khòó} (crocodile-\textsc{gen} skin) ‘crocodile’s skin’, and \textit{hémpè-à píì} (shirt-\textsc{gen} pocket) ‘pocket of a shirt’. This is cognate with the postposition \textit{(ʔ)à} according to \citet[55]{Kilian-Hatz2008Grammar}.

\newpage 
The features identified in \tabref{09-mc-tab:3} are mostly language specific, and most are poorly exemplified and described in the sources. The one in the final column, ‘Governed by a postposition’ is something of an exception, and is attested in both \ili{Khwe} and Nama-Damara. In \ili{Khwe}, PPs with postpositions other than \textit{à} – \ie with local postpositions – pronouns and PGN-marked (\ie \isi{definite}) NPs take the postpositions directly while non-PGN-marked (\ie \isi{indefinite} or non-specific) NPs are marked by \textit{-à} \textsc{gen} (\citealt[64]{Kilian-Hatz2008Grammar}), as in example~\REF{09-mc-ex:14}.

\begin{exe}
\ex\label{09-mc-ex:14} 
\langinfo{Khwe}{West Kalahari Khoe}{\citealt[66]{Kilian-Hatz2008Grammar}}\\
\gll tíì kóánácì ki tcá cà-á ki ǁóé-è-ǁòè nò cé tɛ́-ɛ́-ǁòè ǁxó dì xóm̀-à ki\\
then because \textsc{loc} 2\textsc{sg.m} water-\textsc{gen} \textsc{loc} lie-\textsc{atv}-\textsc{hab} \textsc{con} 1\textsc{pl.f} stay-\textsc{atv}-\textsc{hab} dry \textsc{poss} sand-\textsc{gen} \textsc{loc}\\
\glt ‘Since you are used to lying in the water, and we are used to staying in the dry sand [it is not good to come with us].’ 
\end{exe}

Similarly in Nama-Damara an NP marked by one of the three local postpositions \textit{!’oá} \textsc{all} (optionally), \textit{xuú} \textsc{abl}, or \textit{’úú} \textsc{per} selects the oblique suffix \textit{-à} following its PGN marker (\citealt[112, 192–193]{Hagman1973Nama}).

\begin{exe}
\ex\label{09-mc-ex:15} 
\langinfo{Nama-Damara}{Khoekhoe}{\citealt[192–193]{Hagman1973Nama}}\\
\gll ʔàríp ke ʔom-s-à xuú kè pèé\\
dog \textsc{decl} house-3\textsc{sg.f}-\textsc{obl} \textsc{abl} \textsc{rpst} go:away\\
\glt ‘The dog went away from the house.’ 
\end{exe}

In both \ili{Khwe} and Gǀui \textit{(ʔ)à} can occur on an NP in apposition with the object of a clause, as in~\REF{09-mc-ex:16}. The examples in \ili{Khwe} all involve an attributive relation between the second NP and the first; by contrast, in Gǀui they involve identification. These restrictions may be simply an artifact of the small number of tokens given in the sources, and none of the sources mention the restriction on the grammatical role of the NP attributed on or identified, although all of the illustrative examples satisfy this condition. I suspect that this usage is more widespread in Khoe languages.

\ea \langinfo{Gǀui}{West Kalahari Khoe}{\citealt[2]{Ono2011Gui}}\\\label{09-mc-ex:16} 
\gll da ci nǁoori-xa=na ǁ’ao n{\textbar}e=na (ʔa)\\
1\textsc{sg}.\textsc{irr} 1\textsc{sg.gen} grand:junior-with=\textsc{pl}.\textsc{c}.\textsc{acc} insult \textsc{dem}=\textsc{pl}.\textsc{c}.\textsc{acc} \textsc{acc}\\
\glt ‘Let me insult my grandchildren, these ones.’
\z

In Gǀui according to \citet{Ono2011Gui}, as shown in \tabref{09-mc-tab:3}, \textit{(ʔ)à} occurs on dislocated NPs, by which she apparently means NPs set off on their own intonation contour and either preceding or following the remainder of the clause. The free translation for example~\REF{09-mc-ex:17} suggests that these are a type of cleft construction. This is the only example given, and it is not known whether the dislocated NP can bear any role other than object.

\ea \langinfo{Gǀui}{West Kalahari Khoe}{\citealt[2]{Ono2011Gui}}\\\label{09-mc-ex:17} 
\gll ʔa ja k{\textbar}oã-ki=sa (ʔa) tsa gǁae=si aaku ʔaba-ø-nǂoẽ\\
\textsc{dem} \textsc{conj} child-\textsc{foc}=\textsc{sg}.\textsc{f}.\textsc{acc} \textsc{idtf} 2\textsc{sg}.\textsc{m}.\textsc{gen} woman=\textsc{sg}.\textsc{f}.\textsc{nom} come strap-\textsc{j}-sit\\
\glt ‘It is the child who your wife is strapping to her back.’
\z

\citet[2]{Ono2011Gui} also says that \textit{(ʔ)à} can be used to mark dislocated clauses. However, just one example is given, and in this example \textit{(ʔ)à} might be interpreted as marking a complement clause.


There are a few attestations of \textit{(-)(ʔ)à} as a clausal connector. In Naro \textit{a} can be used to connect a subordinate clause to a preceding main clause, according to Visser (\citeyear[1]{Visser2001Naro}, \citeyear[180]{Visser2010Verbal}). In Nama-Damara \textit{-à} can mark an indirect speech report, as in example~\REF{09-mc-ex:18}. According to \citet[256]{Hagman1973Nama}, the subordinator \textit{-à} is attached to the indirectly quoted clause plus complementiser\textit{ ǃhai}, which forms a single NP syntagm; the analysis provided in \citet[345]{Haacke2013Gora}, although inexplicit, is consistent with this parsing. Note that the \textit{-à} is usually attached to an instance of \textit{ǃhai-s} (that-3\textsc{sg.f}) in final position in the complement clause; occasionally, however, the connector \textit{ǃhai} is omitted and the PGN marker is directly connected to the final word of the indirect quote (it is possible that this function is also served by \textit{(ʔ)à} in Gǀui.)

\begin{exe}
\ex\label{09-mc-ex:18}
\langinfo{Nama-Damara}{Khoekhoe}{\citealt[345]{Haacke2013Gora}}\\
\gll ots kara mû-ba-sen ǃgam-he khom ra ǃhai-s-a\\
2sg \textsc{pot}.\textsc{prog} see-\textsc{appl}-\textsc{refl} kill-\textsc{pass} 1\textsc{du} \textsc{prs}.\textsc{prog} that-3\textsc{sg.f}-\textsc{comp}\\
\glt ‘Then you may see for yourself that we are killed.’ 
\end{exe}

A use not specifically indicated in the tables above is found in \ili{Khwe} alone. This use is in possessive NPs, where \textit{(ʔ)à} marks an \isi{indefinite} possessum – \ie one that is not marked by a PGN marker (\citealt[70, 73]{Kilian-Hatz2008Grammar}). Kilian-Hatz treats this as an instance of the copular function of \textit{(ʔ)à}. An example is given in~\REF{09-mc-ex:19}.

\ea\langinfo{Khwe}{West Kalahari Khoe}{\citealt[70]{Kilian-Hatz2013Kxoe}}\\
\label{09-mc-ex:19} 
\gll tá-khò-\`{m} dì nǂgóá à\\
old-\textsc{ag}-3\textsc{sg.m} \textsc{poss} walking.stick \textsc{cop}\\
\glt ‘the old man’s walking stick’
\z

Finally, it should be remarked that in \ili{Khoekhoe}, with one exception, the morphemes discussed above are suffixes that invariably follow a PGN marker (\citealt[33–34]{Hagman1973Nama}; \citealt[341]{Haacke2013Gora}). Probable cognates of these suffixes are found in the final \textit{à} vowel in one series of PGN markers in most Kalahari Khoe languages, including \ili{Shua} (\citealt[49]{McGregor2014Numerals}), Tyiretyire (my fieldnotes), Ts’ixa \citep[62–64]{Fehn2014Grammar}, ǀGui (\citealt{Nakagawa1993Preliminary} cited in \citealt[315]{Fehn2014Grammar}) and possibly in ǁAni \citep[26–28]{Heine1999Ani} and Eastern ǁAni \citep[315]{Fehn2014Grammar}.\footnote{The situation in \ili{Khwe} seems to be somewhat different, and \citet[40–41]{Kilian-Hatz2008Grammar} does not distinguish a distinct PGN series in \textit{à}; she treats the different forms of the PGN markers in the third person as allomorphs.} The \textit{à}-series of PGN markers serve a different range of functions in each of the languages, but in all languages it is this series that is used on object NPs (\citealt[40–41]{Kilian-Hatz2008Grammar}; \citealt[49]{McGregor2014Numerals}; \citealt[228, 315]{Fehn2014Grammar}). It seems likely that the suffix \textit{-à} and final \textit{à} vowel are both cognate with the free form \textit{(ʔ)à} of Kalahari Khoe languages, the latter having been added via the diachronic process sometimes referred to as “doubling” or “reinforcement”. The fact that similar environments of use and patterns of \isi{optionality} are found in Kalahari Khoe and \ili{Khoekhoe} languages lends some support to this hypothesis.


\section{Optional accusative marking in Khoe languages} \label{09-mc-sec:3} 
\largerpage
As has been shown, in all Khoe languages for which there is sufficient data \textit{(-)(ʔ)à} can mark both direct and indirect objects; Haiǁom is the only language where this use is not mentioned or exemplified in a basic syntactic description \citep{Widlok2013Syntax}.\footnote{It seems likely that the uses of \textit{-à} in Haiǁom are comparable with those of the cognate morpheme in Nama-Damara and ǃOra. \citet[158]{Widlok2013Morphology} indicates that there is an oblique suffix -\textit{a} that attaches to the PGN marker of an NP. Although its usage is not discussed in \citet{Widlok2013Morphology}, it presumably marks objects (both direct and indirect) and subjects as in the other two \ili{Khoekhoe} languages.} In Kalahari Khoe languages it is a phrase-level marker that occurs in NP final position, normally as a separate word or clitic, though sometimes fused with the final word. In \ili{Khoekhoe} it appears to be an inflectional suffix.

In almost all Khoe languages \textit{(-)(ʔ)à} is optional as a direct \isi{object marker} in the sense of McGregor (\citeyear{McGregor2010Optional}: 1610–1613, \citeyear{McGregor2013Optionality}: 1152).\footnote{This phenomenon has also been referred to as “differential object marking” (DOM). I have suggested, however, that this term as generally used covers a disparate range of phenomena which need to be distinguished (\eg \citealt[1613]{McGregor2010Optional}). In particular, the situation in which a single morpheme may be present or absent on an object NP must be distinguished from the situation in which an object NP can be marked by two different morphemes.} First, it may be present or absent on a \isi{direct object} NP without affecting the grammatical role borne by that phrase. There is no reason to believe that the NP serves a different grammatical role when \textit{(-)(ʔ)à} is present/absent, and that in one instance it is not an object; nor (as far as I am aware) has any investigator suggested that it has. Second, the presence or absence of \textit{(-)(ʔ)à} is not predictable from grammatical characteristics of the clause in which it occurs. Both conditions appear to obtain in all East and West Kalahari Khoe languages, and in \ili{Khoekhoe} at least in ǃOra \citep[341]{Haacke2013Gora}. Nama-Damara is a probable exception. According to \citet[341]{Haacke2013Gora} \textit{-à} is consistently used on object NPs, in contrast with !Ora. \citet{Hagman1973Nama} makes no reference to the \isi{optionality} of this marker, and, given that he discusses \isi{optionality} of a range of other morphemes, his description implicates that it is obligatorily used. One context in which \textit{-à} does not occur on object NPs in Nama-Damara is in relative clauses, where the object NP occurs in final position and effectively serves as a relative clause relator \citep[230–231]{Hagman1973Nama}. Being a grammatically conditioned absence, this does not count as an instance of \isi{optionality}.\footnote{It does however fall within the range of phenomena commonly dubbed DOM (see previous footnote), what the editors refer to in the introductory chapter as “clause-type-based differential marking”.} However, in certain other environments the marker is perhaps optional, including when preceding the allative postposition and on indirect speech complement clauses (see discussion of example~\REF{09-mc-ex:18} above).

The situation for the marking of indirect objects in Khoe languages seems rather different, at least in those languages for which information is available. On \isi{indirect object} NPs – which are prototypically human – \textit{(-)(ʔ)à} usually appears. This is the case in \ili{Khwe} (\citealt[51, 56, 63]{Kilian-Hatz2008Grammar}), as in example~\REF{09-mc-ex:20}, \ili{Shua} (my own fieldnotes), and Ts’ixa (where in the majority of examples cited in \citealt{Fehn2014Grammar} are marked either by the \textsc{acc} or the \textsc{dat} postposition; few are unmarked). In the remainder of this section I focus on the marking of direct objects, excluding indirect objects from the exposition.

\protectedex{
\ea \langinfo{Khwe}{West Kalahari Khoe}{\citealt[63]{Kilian-Hatz2008Grammar}}\\ \label{09-mc-ex:20}
\gll màtìaci-\`{m} à ǀ’áò à tí xàró-á-tà\\
Matthew-3\textsc{sg.m} \textsc{acc} money \textsc{acc} 1\textsc{sg} give-\textsc{j}-\textsc{npst}\\
\glt ‘I gave money to Matthew.’ 
\z
}

Existing accounts say little about the motivations for use vs. non-use of \textit{(-)(ʔ)à} on \isi{direct object} NPs. Indeed, a number are silent on the issue, as in the brief treatments of \ili{Danisi}, Deti, Cara and Kua morphology and syntax in \citet{Vossen2013SyntaxShua}. One has to examine the examples given in the papers to discover that the marker is not always present on object NPs.

I would argue that the usage-based theory of optional \isi{case marking} elaborated in \eg \citet{McGregor2010Optional,McGregor2013Optionality} accounts for the optional accusative in Khoe languages. In what follows I provide a brief overview of the main features of this theory; see \citet{McGregor2006Focal,McGregor2010Optional,McGregor2013Optionality} for more detailed discussion.

Two fundamental assumptions of the theory are: (a) within particular constructions case-markers index specific grammatical relations; and (b) either use and/or non-use of an optional case-marker can potentially encode a meaning (again within the specified construction). The first assumption would seem to be uncontroversial, and is assumed by most grammarians: for instance, in a transitive construction (with only the inherent grammatical roles), a case-marker such as the accusative will mark a particular role, namely the object. The second assumption is perhaps more controversial. It applies specifically to optional case-markers and asserts that a meaning may be coded by using and/or not using the case-marker in the environment of its \isi{optionality}. These assumptions imply that there are two possible loci of meaning: the case-marking morphemes themselves and their usage or non-usage.

A case-marker indexes the grammatical role(s) that it marks; this is its meaning. As a consequence, it indirectly and symbolically conveys the meaning associated with that grammatical role – where I presume, along with various functionally oriented theories, that grammatical categories, including roles such as subject and object, are meaningful (\eg \citealt{Haas1954Defining}; \citealt[30–32]{Halliday1985Introduction}; \citealt[275, 316]{Langacker1987Foundations}, \citeyear[289]{Langacker1991Cognitive}; \citealt[27]{Shaumyan1987Semiotic}; \citealt[2]{McGregor1997Semiotic}).

The meanings associated with use or non-use of an optional case-marker do not, by definition, concern grammatical relations; rather, they relate to the domain of joint attention, to the integration of information into the joint-attentional frame \citep{Tomaselloetal1986Joint,Tomasello2003Constructing}. My proposal is that use of an optional case-marker can serve to accord particular attention to the marked grammatical role or its filler, singling it out as the centre of attention – in other words, highlighting it (\citealt[1157]{McGregor2013Optionality}). By contrast, non-use of an optional case-marker may serve a backgrounding function, shifting the role or its filler outside of the domain of the joint-attentional frame, assigning it to the domain of what is presumed by the speech interactants, to the common ground at that point in the speech interaction. The point of the modal qualifications of the previous two sentences is that if a meaning is conveyed – \ie coded – by use or non-use of a marker it will be of the type specified; it is also possible that no meaning is conveyed by either or both.

It is convenient for descriptive and comparative purposes to assign feature labels to the two possible meanings, [prominent] and [backgrounded], and to allow them to take values + (specifying that the feature is coded and thus marked), and – (the unmarked value of the feature, where it is not coded and no meaning of the specified type is conveyed). There are thus four coding possibilities for use or non-use of an optional case marker, as shown in \tabref{09-mc-tab:5}.

\begin{table}
\caption{Meanings potentially coded by presence and absence of an optional marker}\label{09-mc-tab:5}
\small
\begin{tabularx}{\textwidth}{p{1cm}LCCC}
\lsptoprule
Use & No meaning\newline [–prominent] & Meaning\newline [+prominent] & No meaning\newline [–prominent] & Meaning \newline [+prominent]\\
\tablevspace
Non-use & No meaning\newline \mbox{[–backgrounded]} & No meaning\newline \mbox{[–backgrounded]} & Meaning\newline [+backgrounded] & Meaning\newline [+backgrounded]\\
\lspbottomrule
\end{tabularx}
\end{table}

The two features [prominent] and [backgrounded] correlate with expectedness in the grammatical role: prominence is naturally assigned to something that is unexpected, while it is natural for something completely expected to be backgrounded. These features are intended to capture the commonality in the cross-linguistic diversity in the actual meanings associated with usage and/or non-usage. They contextualise in a range of different ways in different languages and constructions, depending in part on how the notion of expectedness is construed, on what sense of (un)expectedness is associated with prominence and/or backgrounding. For instance, among other possibilities, it may concern the prototypical likelihood of the referent of a particular type in that role; it may concern the likelihood of the particular referent in the role in the specific token; it may concern the identity of the filler of the object role.

For just three Khoe languages is some discussion of motivations for optional accusative case-marking available: \ili{Shua} \citep{McGregor2015Optional}, Ts’ixa \citep{Fehn2014Grammar} and \ili{Khwe} \citep{Kilian-Hatz2008Grammar,Kilian-Hatz2013Kxoe}. These are overviewed in the following three subsections, respectively. For the other languages little can be said given the absence of discussion in the sources and the paucity of examples – though the ǁAni texts in \citet{Heine1999Ani} are probably quantitatively sufficient to warrant examination.


\subsection{Shua (East Kalahari Khoe)} \label{09-mc-sec:3-1} 

In Nata \ili{Shua} the frequency of use of the \textsc{acc} marker \textit{(ʔ)à} differs according to the position of the object NP on an \isi{animacy scale} \citep{McGregor2015Optional}: on personal pronouns (\ie pronouns other than the 3\textsc{sg.c} ‘it’) and personal names the \textsc{acc} is (almost) obligatory; on PGN-marked lexical NPs it is quite common, though not obligatory; on ordinary lexical human NPs unmarked by a PGN marker it is relatively infrequent; on lower-order \isi{animate} NPs and inanimates it is rare; on mass inanimates the \textsc{acc} marker is not attested. Frequency of use of the optional marker in specified environments is indicative of semantic \isi{markedness} in the respective contexts, as per Levinson’s I and M heuristics \citep{Levinson2000Presumptive} and the observation that the two types of \isi{markedness} often correlate. \citet{McGregor2015Optional} proposes that in \ili{Shua} use and/or non-use of the \textsc{acc} serves to either make the \isi{direct object} prominent or to background it, depending on the \isi{animacy} of the \isi{direct object} NP, as shown in \tabref{09-mc-tab:6}, where grey background indicates uncertainty due to paucity of examples.


\begin{table}
\caption{Meanings of presence vs. absence of \textsc{acc} on different NP types in Shua}\label{09-mc-tab:6}
\begin{tabularx}{\textwidth}{LLL}
%\begin{tabularx}{\textwidth}{QQp{2.5cm}}
\lsptoprule
{\bfseries NP type} & {\bfseries \textsc{acc} marker present} & {\bfseries \textsc{acc} marker absent}\\
\midrule
Personal pronouns & No meaning & \cellcolor[gray]{0.8}Direct object\newline backgrounded\\
\tablevspace
Personal names & No meaning &\cellcolor[gray]{0.8} Direct object\newline backgrounded\\
\tablevspace
PGN-marked NP & \cellcolor[gray]{0.8} Direct object prominent or no meaning & Direct object\newline backgrounded\\
\tablevspace
Other human NP & Direct object prominent & \cellcolor[gray]{0.8}No meaning\\
\tablevspace
Non-human \isi{animate} \&
\isi{inanimate} & Direct object prominent & No meaning\\
\lspbottomrule
\end{tabularx}
\end{table}

For personal pronouns and names \textsc{acc} marking is (almost) always present, and thus is unlikely to convey a meaning; by contrast the absence of the \textsc{acc} marker (if permissible) can be expected to background the object. For PGN-marked NPs the \textsc{acc} marker is also very frequent and can be expected to convey no meaning; however, if it is absent the \isi{direct object} is backgrounded. For human NPs of other types the \textsc{acc} marker is normally absent, and this most likely conveys no meaning; its presence, by contrast, is rather rare, and marks the \isi{direct object} as prominent. Similarly for non-human animates and inanimates the presence of the \textsc{acc} marker assigns prominence to the \isi{direct object}, whilst its absence, the usual situation, conveys no specific meaning – the \isi{direct object} is neither made prominent nor is it backgrounded.

\newpage 
The present paper is not the place to present arguments for these claims, which are discussed in \citet{McGregor2015Optional}. I illustrate here just the claim for NPs of the lowest \isi{animacy}. For these NPs \citet{McGregor2015Optional} shows that two types of consideration are relevant to the choice of making a \isi{direct object} prominent. One set of considerations concerns identity, in particular whether or not the referent is expected as the filler of the \isi{direct object} role in the particular discourse circumstances. This is illustrated by example~\REF{09-mc-ex:21}, which comes from a description of a drawing in the wordless picture book \textit{A boy, a dog, and a frog} \citep{Mayer1967Boy} in which the boy has netted his dog in a fishing net. The preceding drawings construct a story in which it is expected that the boy will net the frog; the identity of the object referent is thus unexpected. Usually in \ili{Shua}, as in this example, prominence is assigned by use of the \textsc{acc} marker when the \isi{direct object} referent is selected from an already established set of referents, from an established space of potential referents. Less often, it is assigned when the \isi{direct object} referent contrasts with another potential filler of the role, as in~\REF{09-mc-ex:22}.

\ea \langinfo{Shua}{East Kalahari Khoe}{own fieldnotes}\\ 
\ea \label{09-mc-ex:21}
\gll aba: ʔa ema ǁam-rekareka\\
dog \textsc{acc} 3\textsc{sg.m} hit-maybe\\
\glt ‘Maybe he is hitting the dog.’

\ex \label{09-mc-ex:22}
\gll ta: aka ke lori ʔa mũ: ta: aka sekuskara ʔa mũ:-ta \\
1\textsc{sg} \textsc{pst} \textsc{ipfv} truck \textsc{acc} see 1\textsc{sg} \textsc{pst} donkey.cart \textsc{acc} see-\textsc{neg}\\
\glt ‘I saw the truck, not the donkey cart.’
\z
\z

The other set of considerations concerns the degree of patientivity of the \isi{direct object} referent. Consider~\REF{09-mc-ex:23}, which describes an event in which a tent is completely destroyed by hail; it is affected to a higher degree than might be expected – it might be expected that a tent is knocked down, though not necessarily completely wrecked. In this instance the unexpectedly high degree of effect on the tent motivates making the \isi{direct object} prominent by marking it with the \textsc{acc} postposition (identity considerations are irrelevant in this instance.) Other descriptions of similar events in which the tent is not so heavily affected by the event (not torn to shreds), or in which it is just rain that did the job, did not employ \textsc{acc} marking on the object NP.\footnote{Note that it is not suggested that the absence of the \textsc{acc} marker indicates a lower degree of effect on the \isi{direct object}, only that it is consistent with lower \isi{affectedness} of that entity. Absence of the marker on \isi{inanimate} NPs, as indicated in \tabref{09-mc-tab:6}, conveys no meaning.}

\ea \langinfo{Shua}{East Kalahari Khoe}{own fieldnotes}\\ \label{09-mc-ex:23}
\gll he͂:xo: ʔa tu:-a-ta tu: ka:ro ka tante ʔa bo:ru-hu-a-ha\\
this \textsc{loc} rain-\textsc{j}-\textsc{pst} rain hail \textsc{ins} tent \textsc{acc} hole-\textsc{caus}-\textsc{j}-\textsc{pst}\\
\glt ‘The rain that rained here with hail tore the tent to shreds.’
\z


\subsection{Ts’ixa (East Kalahari Khoe)} \label{09-mc-sec:3-2} 

\citet[231]{Fehn2014Grammar} sums up the meanings associated with use of the \textsc{acc} marker \textit{(ʔ)à} in Ts’ixa as shown in \tabref{09-mc-tab:7} (slightly modified), where the \isi{definiteness} of a lexical NP is dependent on the presence of a PGN marker. In contrast with the situation for \ili{Shua}, in Ts’ixa \isi{word order} is relevant to the choice of using or not using the \textsc{acc} marker.


\begin{table}
\caption{Meaning of \textsc{acc} marking on definite and indefinite NPs in Ts’ixa}\label{09-mc-tab:7}

\begin{tabularx}{\textwidth}{lQQ}
\lsptoprule
{{\bfseries Word order}} & {{\bfseries Definite NP: personal pronoun or PGN-marked lexical NP}} & {\bfseries Indefinite lexical NP (not PGN-marked)}\\
\midrule
{\bfseries S O \textit{ʔà} V} & No meaning (\textsc{acc} obligatory) & Contrastive focus\\
{\bfseries S V O \textit{ʔà}} & No meaning (\textsc{acc} obligatory) & No meaning (\textsc{acc} precluded)\\
{\bfseries O \textit{ʔà} S V} & Contrastive focus & Contrastive focus\\
\lspbottomrule
\end{tabularx}
\end{table}

It will be observed that no information is provided in \tabref{09-mc-tab:7} about meanings associated with non-use of the \textsc{acc} marker in Ts’ixa. However, it is likely that non-use never conveys meaning. For \isi{indefinite} direct objects the \textsc{acc} marker occurs rarely \citep[229]{Fehn2014Grammar}; its absence is the norm, and may reasonably be presumed to convey no meaning. For OSV clauses the fact that accusative marking on both \isi{definite} and \isi{indefinite} direct objects has the same meaning leads one to expect that its omission is similarly motivated in each case, and thus has no meaning.

It is worth observing in passing that the facts of optional accusative \isi{case marking} in Ts’ixa are not well accounted for by the disambiguation theory, according to which the case-marker is used when there is a possibility of confusion between which roles are borne by the NPs of a transitive clause, and not used when there is no likelihood of confusion. First, the most common word orders in Ts’ixa are SOV and SVO, which are about equally frequent \citep[214]{Fehn2014Grammar}. In these unmarked word orders \isi{definite} object NPs are obligatorily marked by the \textsc{acc}, whilst in the more marked OSV \isi{word order} the marking of the \isi{direct object} is optional. Second, the \textsc{acc} occurs either obligatorily or optionally (in OSV clauses) where it is not required to disambiguate of the fillers of the subject and object roles: on PGN-marked direct objects, where the form of the PGN marker indicates the grammatical role of the NP. By contrast, for direct objects that are not PGN-marked – and thus where the NPs denoting them are not morphologically distinct from non-PGN-marked subject NPs – the \textsc{acc} marker is optional or precluded.

\citet{Fehn2014Grammar} suggests that in all circumstances where the \textsc{acc} is optional that its presence assigns \isi{contrastive focus} on the object. This is illustrated in the following exchange, invented by a native speaker to illustrate the meaning difference between the presence and absence of the \textsc{acc} marker.~\REF{09-mc-ex:27} in particular illustrates \isi{contrastive focus} on the object.

\newpage 
\ea \langinfo{Ts’ixa}{East Kalahari Khoe}{\citealt[230]{Fehn2014Grammar}}\\%\24–27
\ea  \label{09-mc-ex:24}
\gll maá ʔà tsá ǁ’áḿ-nà-tà\\
who \textsc{acc} 2\textsc{sg.m} beat-\textsc{j}-\textsc{sdpst}\\
\glt ‘Who did you beat?’

\ex  \label{09-mc-ex:25}
\gll k’aro=mà tí ǁ’áḿ-nà-tà\\
boy=3\textsc{sg}.\textsc{m.ii} 1\textsc{sg} beat-\textsc{j}-\textsc{sdpst}\\
\glt ‘I beat the boy.’

\ex  \label{09-mc-ex:26}
\gll ʔabá=mà tsá ǁ’áḿ-nà-tà\\
dog=3\textsc{sg}.\textsc{m.ii} 2\textsc{sg.m} beat-\textsc{j}-\textsc{sdpst}\\
\glt ‘Did you beat the dog?’

\ex  \label{09-mc-ex:27}
\gll ʔ\~\i\~\i, ʔabá=mà tí ǁ’áḿ-nà-tà ʔíté k’aro=mà ʔà tí ǁ’áḿ-nà-tà\\
no dog=3\textsc{sg}.\textsc{m.ii} 1\textsc{sg} beat-\textsc{j}-\textsc{sdpst} \textsc{neg} boy=3\textsc{sg}.\textsc{m.ii} \textsc{acc} 1\textsc{sg} beat-\textsc{j}-\textsc{sdpst}\\
\glt ‘No, I did not beat the dog, I beat the boy.’
\z
\z

However, a number of other examples provided in \citet{Fehn2014Grammar} appear not to exemplify \isi{contrastive focus}, but rather, as in \ili{Shua}, the presence of the \textsc{acc} functions to select a referent (type) from a set of presumed entities or entity types. This is illustrated by~\REF{09-mc-ex:29} – compare this example with the neutral~\REF{09-mc-ex:28}, which does not involve an instance of the \textsc{acc} postposition.

\ea \langinfo{Ts’ixa}{East Kalahari Khoe}{\citealt[229]{Fehn2014Grammar}}\\
\ea \label{09-mc-ex:28}
\gll xaḿ=mà ʔé.ǁù ǀ’\~uṹ-á-tá\\
lion=3\textsc{sg}.\textsc{m.ii} 3\textsc{pl.m} kill-\textsc{j}-\textsc{sdpst}\\
\glt ‘They killed the lion.’

\ex\label{09-mc-ex:29}
\gll xaḿ=mà ʔà ʔé.ǁù ǀ’\~uṹ-á-tá\\
lion=3\textsc{sg}.\textsc{m.ii} \textsc{acc} 3\textsc{pl.m} kill-\textsc{j}-\textsc{sdpst}\\
\glt ‘They killed the lion (and not something else).’
\z
\z

Indeed, there are other environments in which use of the \textsc{acc} does not assign \isi{contrastive focus} to a \isi{direct object}. One such situation is when the object is human and \isi{indefinite} \citep[232]{Fehn2014Grammar}, as in example~\REF{09-mc-ex:30}. Here again it is possible that prominence is assigned to the object by use of the \textsc{acc} in view of selecting the relevant entities from the class of available ones: a possible interpretation of this example is that it narrows down to Khoe people as possible speakers, from the set of all persons who might speak at Gǀoxa-Hill (there may well be other possible explanations for this are consistent with assigning prominence to the object NP ‘people’, but more detailed knowledge of the discourse environment would be required to permit evaluation).

\newpage 
\ea \langinfo{Ts’ixa}{East Kalahari Khoe}{\citealt[232]{Fehn2014Grammar}}\\\label{09-mc-ex:30}
\gll gǀóóxà=\`{m} ngùà ǀú.xùà tsá kò khoe ʔà kúḿ k’uí  kò=sè\\
\textsc{gn}=3\textsc{sg}.\textsc{m.i} \textsc{loc} sometimes 2\textsc{sg.m} \textsc{ipfv} person \textsc{acc} hear speak \textsc{ipfv}=\textsc{adv}\\
\glt ‘At Gǀoxa-Hill, you can sometimes hear people speak.’
\z

To sum up, the facts as outlined above for Ts’ixa are not inconsistent with my proposals for the motivations for using and/or not using the \textsc{acc} postposition. In particular, use of the \textsc{acc} in the environments in which it is optional assigns a high degree of prominence to the object. What is significantly different from \ili{Shua} is that only identity of the object referent seems to be a relevant consideration in making the object prominent; considerations of the degree of patientivity of the object appear not to be pertinent in Ts’ixa (\cf example ~\REF{09-mc-ex:23} above). To make the case watertight requires more evidence, in particular, more information on frequencies – especially on the frequency of marking vs. non-marking of \isi{definite} object NPs in clause-initial position, and on the effects of \isi{animacy} on frequencies.

\subsection{Khwe (West Kalahari Khoe)} \label{09-mc-sec:3-3}

The situation in \ili{Khwe} is somewhat murky, despite the extensive treatment in \citet{Kilian-Hatz2008Grammar,Kilian-Hatz2013Kxoe}. The basic facts concerning the distribution of the marker on NPs in core clausal roles appear to be as shown in \tabref{09-mc-tab:8}, which excerpts relevant information from Table~12 of \citet[47]{Kilian-Hatz2008Grammar} and Table~15 of \citet[56]{Kilian-Hatz2008Grammar}.

\begin{table}
\caption{Accusative marking of core NPs in Khwe}\label{09-mc-tab:8}
\begin{tabularx}{\textwidth}{XCCCCCC @{\hspace*{1cm}} CCCC} 
\lsptoprule
& \multicolumn{6}{c}{\bfseries Definite NP}
& \multicolumn{4}{c}{{\bfseries Indefinite NP}}\\
& \multicolumn{2}{c}{Pronoun} 
& \multicolumn{2}{c}{Proper noun} 
& \multicolumn{2}{c}{Specific} 
& \multicolumn{2}{c}{Generic}
& \multicolumn{2}{c}{Unspecific}\\
& PGN & \textit{(ʔ)à} & PGN & \textit{(ʔ)à} & PGN & \textit{(ʔ)à} & PGN & \textit{(ʔ)à} & PGN & \textit{(ʔ)à}\\
\midrule
\multicolumn{11}{l}{Intransitive clauses}\\
\midrule
S & +\footnote{+ ‘with marker’; – ‘without marker’; ± ‘optional marker’; () – rare.} & – & – & – & + & – & – & ± & – & ±\\



\tablevspace
\midrule
\multicolumn{11}{l}{Transitive and \isi{ditransitive} clauses}\\
\midrule
S & + & – & – & – & + & – & – & (+)– & – & (+)–\\
O & + & ± & + & + & + & +(–) & – & ± & – & ±\\
IO & + & + & + & + & + & +(–) & – & + & – & +\\
\lspbottomrule
\end{tabularx}
%* + ‘with marker’; – ‘without marker’; ± ‘optional marker’; () – rare.
\end{table}


As already mentioned, Kilian-Hatz considers \textit{(ʔ)à} to be an \isi{object marker} only in those contexts in which it is obligatory on NPs in that role – \ie proper noun objects and indirect objects other than specific nouns; elsewhere she takes it to be a focal marker. About two thirds of \isi{direct object} NPs overall are \textit{(ʔ)à}-marked (\citealt[376]{Kilian-Hatz2013Kxoe}). Kilian-Hatz suggests the following generalisations for the marking of direct objects in those circumstances in which it is not obligatory (\citealt[59–61]{Kilian-Hatz2008Grammar}, \citeyear{Kilian-Hatz2013Kxoe}: 371–372):

\textit{Specific/\isi{definite} objects}
\begin{itemize}

\item Use of \textit{(ʔ)à} is motivated by:
    \begin{itemize}
    \item Possibility of confusion as to who is acting on who
    \item Contrastive contexts, including selective contexts
    \end{itemize}

\protectedex{
\item Non-use of \textit{(ʔ)à} is motivated by:
\begin{itemize}
    \item Presence of another NP marked by \textit{(ʔ)à} – \eg on the subject or \isi{indirect object}\\ (see \tabref{09-mc-tab:8})
    \item No possibility of confusion as to who is acting on who
\end{itemize}
}
\end{itemize}


\protectedex{
\textit{Non-specific/\isi{indefinite} objects}
\begin{itemize}

\item Non-use of \textit{(ʔ)à} is motivated by:
\begin{itemize}
\item No possibility of confusion as to who is acting on who
\item Presence of another NP marked by \textit{(ʔ)à}
\end{itemize}
\end{itemize}
}

As indicated, for both specific/\isi{definite} objects and non-specific/\isi{indefinite} objects non-use of the marker \textit{(ʔ)à} is motivated by the same considerations. \citet{Kilian-Hatz2008Grammar} does not make it clear why this should be the case; in fact, it is doubtful for two reasons. First, for specific nouns absence of the marker is rare according to \tabref{09-mc-tab:8}, suggesting that use is the norm (as stated specifically in \citealt[372]{Kilian-Hatz2013Kxoe}). Being so predominant, use is unlikely to be associated with specific contexts or meanings. Second, for non-specific/\isi{indefinite} nouns presence and absence appear less skewed in distribution – although \citet[371]{Kilian-Hatz2013Kxoe} says that the marker is used in most cases – and one would expect both to be motivated and meaningful. Nonetheless, no motivation is suggested for the use of the marker on such objects.


To begin, the non-use of \textit{(ʔ)à} will be examined. Examples cited in \citet{Kilian-Hatz2008Grammar,Kilian-Hatz2013Kxoe} reveal that the two circumstances of non-use cited in the above generalisations are at best statistically correlated with non-use. There are examples in which objects are marked by \textit{(ʔ)à} alongside of other NPs that are also marked by \textit{(ʔ)à} – see examples (34) and (35) in \citet[52]{Kilian-Hatz2008Grammar}. According to Kilian-Hatz such examples tend to be found in elicitation rather than in actual discourse; this is, however, a tendency and not a rule.

Similarly, the association of non-use of \textit{(ʔ)à} with contexts in which there is no doubt as to who is acting on who is not consistently borne out:~\REF{09-mc-ex:31} is an example in which an object is marked by \textit{(ʔ)à} and there is clearly no real possibility of confusion as to who/what is acting on who/what.

\ea \langinfo{Khwe}{West Kalahari Khoe}{\citealt[372]{Kilian-Hatz2013Kxoe}}\\ \label{09-mc-ex:31}
\gll ǀóɛ-hɛ átaxa 	ǂ’ṹ-à-tè  cɛ-djì  à\\
child-3\textsc{sg.f} thus eat-\textsc{atv}-\textsc{prs} bush:species-3\textsc{pl.f} \textsc{acc}\\
\glt ‘Thus the child eats the cɛ fruits.’
\z

Nor does the possibility of confusion as to who is acting on who consistently engender the presence of \textit{(ʔ)à} on specific object NPs, as shown by~\REF{09-mc-ex:32} and~\REF{09-mc-ex:33}. In both of these examples it would seem that there is a genuine possibility of confusion as to who is acting on who, though no instance of the marker \textit{(ʔ)à} is present.

\protectedex{
\ea\langinfo{Khwe}{West Kalahari Khoe}{\citealt[372]{Kilian-Hatz2013Kxoe}}\\
\ea\label{09-mc-ex:32}
\gll ó-ǀ’ṹũ-khòè-tè té khóé-tè xà-má kx’ó\\
\textsc{prv}-hair-person-1\textsc{pl.c} 1\textsc{pl.c} person-1\textsc{pl.c} \textsc{dem}-3\textsc{sg.m} eat:meat\\
\glt ‘He has to eat us, the ones without fur.’
\ex \label{09-mc-ex:33}
%\langinfo{Khwe}{West Kalahari Khoe}{\citealt[373]{Kilian-Hatz2013Kxoe}}\\
\gll tàn tíì tcá tí ú\\
stand:up then 2\textsc{sg.m} 1\textsc{sg} bring\\
\glt ‘Stand up, then I may take you.’ 
\z
\z
}

Just a few examples cited in \citet{Kilian-Hatz2008Grammar,Kilian-Hatz2013Kxoe} – one of which is~\REF{09-mc-ex:34} – illustrate the contrastive function of \textit{(ʔ)à} on \isi{direct object} NPs. In~\REF{09-mc-ex:34} the rock monitor and genet have already been mentioned and are represented by \isi{definite} NPs (marked by a PGN marker). However, only the former is marked by \textit{(ʔ)à}: \citet[372]{Kilian-Hatz2013Kxoe} comments that the genet is the overall \isi{discourse topic} of the narrative, presumably accounting for the absence of the marker \textit{(ʔ)à} on this NP in the second clause.

\begin{exe}
\ex\label{09-mc-ex:34}
\langinfo{Khwe}{West Kalahari Khoe}{\citealt[372]{Kilian-Hatz2008Grammar}}\\
\gll tínù córò-mà-à ǁgàa-khòè-djì nǀgóá-à-tè kx’á-khòè-ǁè tcámba-mà nǀgóá-à-tè \\
then rock:monitor-3\textsc{sg.m}-\textsc{acc} female-person-3\textsc{pl.f} cook-\textsc{atv}-\textsc{prs} male-person-1\textsc{pl.m} genet-3\textsc{sg.m} cook-\textsc{atv}-\textsc{prs}\\
\glt ‘Then the women are cooking the rock monitor, and we men are cooking the genet.’ 
\end{exe}

A better explanation of the situation in \ili{Khwe} is possible within McGregor’s (\citeyear{McGregor2010Optional,McGregor2013Optionality}) theory of optional \isi{case marking}. First, as mentioned above the \textsc{acc} marker is almost always present on specific direct objects, and is unlikely to convey meaning. In this context only non-use of the \textsc{acc} marker conveys meaning; this must be to background the \isi{direct object}. In examples such as~\REF{09-mc-ex:31} and~\REF{09-mc-ex:34}, then, the presence of the \textit{(ʔ)à} marker on the \isi{direct object} NP does not serve to foreground it, or to disambiguate it from the subject, but rather indicates nothing particular: the object is simply an object. It is the absence of the marker on the \isi{direct object} of the second clause in~\REF{09-mc-ex:34} that is meaningful, and serves to background it – consistent with the fact that it is the primary \isi{discourse topic}, and thus a good candidate for something presumed, for something that is understood as a component of the common ground at that point in the interaction. Certainly contrasts are apparent in this example: between the women and men as subjects and the rock monitor and genet as direct objects. But it is not the use of \textit{(ʔ)à} that signals them; it is presumably some other features of the utterance (assuming that the contrasts are actually marked rather than inferred).

Second, for pronominal objects and \isi{indefinite} objects, as suggested above, it seems that use and non-use of \textit{(ʔ)à} are more evenly distributed. However, in the absence of statistical data and a comprehensive examination of examples in the sources it is impossible to be certain whether both are meaningful. Indeed, it may be that further distinctions need to be made according to \eg inherent features of the NP (\isi{animacy}) or \isi{word order}.

Third, if (as \citealt[371]{Kilian-Hatz2013Kxoe} avers) the marker is used on most \isi{indefinite} direct objects, it is likely that non-use on them is meaningful, and serves a backgrounding function. This is consistent with the absence of \textit{(ʔ)à} on an object comprising a very long list of types of things and an object with an abstract type referent (respectively, examples (471a) and (471b), \citealt[371--372]{Kilian-Hatz2013Kxoe}). Use of the marker could also be meaningful, though this is less certain.

\subsection{Concluding remark} \label{09-mc-sec:3-4}

In most Khoe languages accusative marking of \isi{direct object} NPs is optional, at least in certain environments. Little is known for certain concerning the factors that motivate use vs. non-use of the accusative marker in its environments of \isi{optionality}, or the meanings that are expressed. Nonetheless, it is likely that the theory of \citet{McGregor2010Optional,McGregor2013Optionality} can account for the facts in the various languages. Admittedly, much more research on each Khoe language is necessary to make a convincing case. For present purposes, it is sufficient to observe that the theory suggests that presence and/or absence of the postposition concern joint attention and grounding.


\section{The emergence and development of optional accusative marking} \label{09-mc-sec:4}
Before beginning the exposition of my proposals for the development of optional accusative marking in Khoe languages some cautions are in order. First, as has already been mentioned, descriptions of most Khoe languages lack comprehensiveness in their treatment of \textit{(-)(ʔ)à}, including motivations for its use and non-use. Second, diachronic data of any significant depth is non-existent: there is no long tradition of writing in any Khoe language or of linguistic investigation going back very far in the past. Third, as \citet[319--320]{Fehn2014Grammar} rightly observes, serious problems lie in the low distinctiveness of the form \textit{(-)(ʔ)à} – and hence the probability of spurious cognates and look-alikes – to say nothing of the irregular presence of the initial /ʔ/ in the sources. These problems bedevil grammaticalisation investigations of “exotic” languages, for which descriptions are frequently partial, and historical depth is lacking in the data; moreover, grammemes are often phonologically reduced and/or show phonologically unmarked shapes.

These concerns do not mean that one should avoid the domain of grammaticalisation. But they do imply that one needs to constrain these hypotheses. One way is to invoke attested pathways as far as possible, especially pathways that have empirical evidence in actual diachronic data. Another is to compensate for lack of time depth with synchronic diversity. Thus the need for information on the relevant forms and their functions in a diverse sample of languages in the family or in the geographical region. Even with these constraints, the proposals ultimately remain speculative, albeit hopefully plausible.

Before outlining my proposals for the grammaticalisation of accusative marking in Khoe languages in~\sectref{09-mc-sec:4-2} I briefly overview the few existing suggested scenarios.

\subsection{Overview of existing proposals} \label{09-mc-sec:4-1}
The Khoe literature contains just a few suggestions for the grammaticalisation of the accusative \textit{(-)(ʔ)à}. None of these explain how \textit{(-)(ʔ)à} became an accusative marker, or why it is optional in \ili{Shua}, Ts’ixa, \ili{Khwe} and ǃOra. It is useful to overview these proposals and draw out their inadequacies so that the proposals I outline in~\sectref{09-mc-sec:4-2} can be tested against these weaknesses.

The only detailed scenario for the grammaticalisation of \textit{(-)(ʔ)à} is that proposed in Kilian-Hatz (\citeyear[55]{Kilian-Hatz2008Grammar}, \citeyear[376--378]{Kilian-Hatz2013Kxoe}) for \ili{Khwe}.\footnote{\citet[276--278]{Koenig2008Case} briefly outlines a scenario similar to Kilian-Hatz’s, involving a change from copula to \isi{focus marker} to accusative marker, which she says is applicable to both \ili{Khwe} and \ili{Khoekhoe} (see~\REF{09-mc-ex:35} and~\REF{09-mc-ex:36} below).  On the other hand, \citet[342]{Haacke2013Gora} suggests in a parenthetical aside that \textit{-à} in !Ora is “derived from the stative aspect marker”. He provides no discussion or evidence for this suggestion, which is presumably based on formal identity or similarity of the oblique suffix with a stative aspect marker -\textit{a}.}  
These two sources describe effectively the same diachronic scenario, albeit with some differences in detail and in the degree of elaboration of certain points. Both proposals are based squarely on the situation in \ili{Khwe}, and although they might be extended to other Khoe languages, emendations would be necessary to account for the different endpoints in the various modern languages.

The scheme in \REF{09-mc-ex:35} shows the grammaticalisation scenario proposed in \citet[55]{Kilian-Hatz2008Grammar}, while~\REF{09-mc-ex:36} shows the version presented in \citet[376--377]{Kilian-Hatz2013Kxoe}. According to~\REF{09-mc-ex:35}, the development of the genitive function of \textit{(ʔ)à} is independent of the development of the object marking functions, and this is left out of~\REF{09-mc-ex:36}. This part of the story is of no concern here.

\protectedex{
\ea \label{09-mc-ex:35}

genitive (\isi{indefinite}/unspecific)\\
{       }     ↑\\
copula/presentative (\isi{indefinite}/unspecific)\\

{       } ↓

subject focus (\isi{indefinite}/unspecific)

 ↓

\isi{direct object} focus

 ↓

\isi{indirect object} focus

 ↓

adverbial focus
\z
}
\protectedex{
\ea 
\label{09-mc-ex:36} 

{COP} {copulative/presentative use of \textit{à}, which is restricted to \isi{indefinite} subjects in verbless clauses}

 ↓

{FOC “becomes a \isi{focus marker} that introduces new information”: \isi{indefinite} subjects, \isi{indefinite} objects}

 ↓

{O extends to new but \isi{definite} objects (a) to indicate \isi{contrastive focus}; (b) to disambiguate syntactic roles}

↓

{OBL extends to focus on local and temporal adverbials (apparently NPs)}
\z
}

According to both~\REF{09-mc-ex:35} and~\REF{09-mc-ex:36} \textit{(ʔ)à} was initially a copula/presentative marker that was restricted to verbless clauses with \isi{indefinite} or unspecified subjects, in which case it followed the subject NP. Reflexes of this putative initial state are found in the modern language, as shown by~\REF{09-mc-ex:37}–\REF{09-mc-ex:38} (see also~\REF{09-mc-ex:9} and~\REF{09-mc-ex:10} above).\footnote{Kilian-Hatz usually glosses \textit{à} in example~\REF{09-mc-ex:37} as \textsc{foc}, not \textsc{cop} (she is inconsistent in example~\REF{09-mc-ex:38}, glossing it as \textsc{cop} on p.\, 199, but as \textsc{foc} on p.\, 208). What she normally glosses as \textsc{cop} is \textit{à} in final position in the relational clause, as in the case of the second instance of \textit{à} in~\REF{09-mc-ex:38}. In keeping with the remarks of footnote 2 above, I employ the gloss \textsc{cop} for those cases in which the marker appears to be serving as a copula.}

\begin{exe}
\ex\label{09-mc-ex:37} 
\langinfo{Khwe}{West Kalahari Khoe}{\citealt[135]{Kilian-Hatz2008Grammar}}\\
\gll yì á ǀéú\\
tree \textsc{cop} big\\
\glt ‘The tree is big.’ 
\end{exe}

\begin{exe}
\ex\label{09-mc-ex:38} 
\langinfo{Khwe}{West Kalahari Khoe}{\citealt[99, 208]{Kilian-Hatz2008Grammar}}\\
\gll nǀĩĩ́ dáó à ǂ’ó dáó à \\
\textsc{dem}{\rmfnm} path \textsc{foc} small path \textsc{cop}\\
\glt ‘This path is a small path.’ 
\end{exe}

\footnotetext{Note that in this example \citet[208]{Kilian-Hatz2008Grammar} treats the subject as having an \isi{indefinite} head noun (since it is not marked by a PGN marker), even though there is a demonstrative in the NP.}

From marking \isi{indefinite}/unspecified subjects of copula relational clauses (see \REF{09-mc-ex:35}) \textit{(ʔ)à} extended to marking \isi{indefinite}/unspecified subjects of ordinary verbal clauses, then to marking direct objects, and ultimately to marking indirect objects. One difficulty with this scenario is that it predicts that the frequency of \textit{(ʔ)à} should be highest on the oldest usage, subjects, and lowest on the newest, indirect objects. In fact, the newer uses of the \isi{focus marker} are more frequent than the most established one. Indeed, as has been remarked, \textit{(ʔ)à} occurs on almost all \isi{indirect object} NPs, making it improbable that it assigns focus to indirect objects.

\REF{09-mc-ex:36} gets around this difficulty by presuming that in its first stage of development \textit{(ʔ)à} became a general \isi{focus marker} that was not specifically associated with any grammatical role, but marked NPs that presented new information. From this, it seems that the focal value of \textit{(ʔ)à} began to change somewhat, so that on object NPs it came to be associated with \isi{contrastive focus} and disambiguation. There are a couple of things that are not accounted for in this scenario.

First, the change from \isi{focus marker} to contrastive \isi{focus marker} is a restriction and strengthening of the focal value of the marker, and seems an unlikely change for a \isi{focus marker} – one expects a \isi{focus marker} to weaken over time, not to strengthen. How this relates to the disambiguation function is not made clear, though in this instance it is plausible that the focal value has weakened. This scenario thus invokes both strengthening and weakening of the focal value of the marker.

Second, it is not explained in this story why this change happened with \isi{direct object} NPs but not with subject NPs. Indeed, neither~\REF{09-mc-ex:35} nor~\REF{09-mc-ex:36} accounts for the strong association of \textit{(ʔ)à} with object NPs in \ili{Khwe} (or any other Khoe language) – recall that about two thirds of object NPs in \ili{Khwe} are marked by this postposition; NPs in other grammatical relations are far less frequently marked.\footnote{\citet[356–357]{Kilian-Hatz2013Kxoe} provides some relevant figures showing the strength of the association of \textit{(ʔ)à} with the object role, based on a corpus of some 1,500 sentences from a set of 30 texts. In this small corpus 29 object NPs are marked by \textit{(ʔ)à} (\ie almost 80\%), 8 are not marked; no transitive subject NPs are marked. (No figures are given for \isi{intransitive} subject NPs.)}

Third,~\REF{09-mc-ex:36} presumes an initial association of \textit{(ʔ)à} with \isi{indefinite} NPs and new information, as expected for a \isi{focus marker}. And indeed as can be seen from \tabref{09-mc-tab:8}, this association is manifest in \ili{Khwe} for subject NPs. However, for \isi{direct object} NPs the situation is inconsistent with the initial association: \isi{definite} proper and specific ones are almost always marked by \textit{(ʔ)à}, whereas for \isi{indefinite} ones \textit{(ʔ)à} remains optional. It remains unexplained why the marking of \isi{indefinite} \isi{direct object} NPs did not become more entrenched and frequent than the marking of \isi{definite} \isi{direct object} NPs, given that the former represents the older and less marked situation. One expects under scenario~\REF{09-mc-ex:36} that on \isi{definite} NPs occurrence of the postposition \textit{(ʔ)à} would have been more restricted and infrequent.

Finally, it is unclear why (in both~\REF{09-mc-ex:35} and~\REF{09-mc-ex:36}) it is only in the final stage that \textit{(ʔ)à} comes to be used to mark spatial and temporal locatives. To be sure, in this instance the low frequency of use of the postposition is consistent with the late development of the function. But it remains unclear why – if it had really begun as an unrestricted focal marker – \textit{(ʔ)à} would not have been used generally on clausal units, regardless of their grammatical role.

\citet[378]{Kilian-Hatz2013Kxoe} attempts to flesh out details of the development of \textit{(ʔ)à} from a pragmatic marker of focus to a “more grammatical” marker of object, as per~\REF{09-mc-ex:39}. She does not indicate, however, how precisely this sequence of steps fits with that proposed in~\REF{09-mc-ex:36}.

\protectedex{
\ea  \label{09-mc-ex:39} 
focussed referent precedes main clause in copulative periphrasis with \textit{à}, and object referent identical with that of main clause – O \textit{à}, SOV

↓

reinterpretation of copulative periphrasis as focussed and topicalised object of main clause – O \textit{à} SV

↓

reinterpretation of \isi{focus marker} \textit{à} as combined focus-\isi{object marker}

↓

expansion of focus-\isi{object marker} to objects in SOV and SVO clauses (SO \textit{à} V, SVO \textit{à})
\z
}
Again, a number of stages proposed here lack motivation. To begin with, why didn’t the story begin at the second stage? An obvious motivation for this might be to mark an object that occurred in a marked order with respect to the subject. Second, given that at the same time \textit{(ʔ)à} marked \isi{indefinite} subjects, which also typically occurred preverbally, why should \textit{(ʔ)à} have come to be strongly associated with object NPs? Third, this additional scenario still does not account for the most serious difficulty of~\REF{09-mc-ex:35} and~\REF{09-mc-ex:36}, namely how and why \textit{(ʔ)à} extended to \isi{definite} direct objects, and why it is so frequent on them.

\subsection{An alternative proposal} \label{09-mc-sec:4-2}
I begin by reconstructing in broad brush a set of probable diachronic changes leading to the core case-markers of modern Khoe languages; these will be discussed and elaborated further below. It should be noted that this is intended to capture only a small part of the possible diachronic developments involving \textit{(-)(ʔ)à}; developments that do not pertain to the accusative marker as endpoint are largely ignored. For instance, it seems reasonable to presume that the locative postposition \textit{(ʔ)à} of many \ili{Shua} varieties developed from the same source as the accusative, albeit via independent processes. Also explicitly left out of the proposed diachronic scenario are those changes specific to \textit{(-)(ʔ)à} on indirect objects.

It is reasonable to presume that the marker \textit{(ʔ)à} was initially a separate word. At some point in time it lost its freedom of occurrence when following the PGN marker, becoming a suffix to that marker. In the \ili{Khoekhoe} lineage PGN markers became effectively obligatory on NPs, and the suffix was restricted to this environment. However, in this lineage the association with objects never became exclusive, and a small fraction of subject NPs retained the suffix – \eg “deposed” subjects, and in Nama-Damara subjects in some marked illocutionary moods; moreover, in ǃOra at least on object NPs the suffix never became entirely obligatory.

In the Kalahari Khoe lineage different diachronic developments occurred. In contrast with \ili{Khoekhoe} languages the PGN markers were used only on some NPs (perhaps \isi{definite} ones as in many of the modern languages), and it was only in this environment that the \textit{(ʔ)à} lost its freedom of occurrence. Ultimately the suffix was reinterpreted as a part of the PGN marker, leading to the development of two (sometimes three) series of PGN markers in the majority of languages. The \textit{à} series was strongly associated with objects, and effectively became an accusative series in some languages (Ts’ixa, ǀGui, and Eastern ǁAni); in other languages (\eg Nata \ili{Shua} and \ili{Khwe}) the association of the \textit{à} series with subjects strengthened rather than weakened, and this series was ultimately used for NPs in both subject and object roles.

Elsewhere, \ie with NPs that were not marked by PGN markers, the \textit{(ʔ)à} remained a free word. In most of the Kalahari Khoe languages this free morpheme became exclusively but optionally associated with non-PGN-marked objects, and is not employed on subjects at all. The next development that occurred in the Kalahari Khoe lineage was the extension of the marker \textit{(ʔ)à} from non-PGN-marked object to PGN-marked objects, presumably via reinforcement. In \ili{Khwe}, where \textit{(ʔ)à} does occasionally mark subjects, it is restricted to non-PGN-marked ones; no extension to PGN-marked subjects occurred.

The above scenario focuses on the formal aspects of the grammaticalisation of the accusative marker \textit{(-)(ʔ)à}. The schematic representations provided in Figure~\ref{09-mc-fig:2-Scheme} and~\ref{09-mc-fig:2-Scheme2} show the main concomitant developments in usage across the two major lineages. The arrows show diachronic developments. However, not all of these changes can be located in a single chronological sequence with respect to one another, and hence two parallel pathways are indicated for each representation. It should be noted that not all of the diachronic developments occurred in all languages.


\begin{figure} %ex40
	 
	\includegraphics[angle=270,width=\textwidth]{figures/09-mc-fig2.pdf} %\vspace{-2ex}
	\caption{Diachronic developments of \textit{(-)(ʔ)à} in the Khoekhoe lineage}\label{09-mc-fig:2-Scheme}
\end{figure}


\begin{figure} %ex41
	 
	\includegraphics[width=\textwidth]{figures/09-mc-fig3.pdf} %\vspace{-2ex}
	\caption{Diachronic developments of \textit{(-)(ʔ)à} in Kalahari Khoe lineages}\label{09-mc-fig:2-Scheme2}
\end{figure}

The initial stages of Figure~\ref{09-mc-fig:2-Scheme} and~\ref{09-mc-fig:2-Scheme2} are the same: \textit{(ʔ)à} is a free copula in presentational clauses that occurred after the NP presented to the addressee’s attention, as illustrated in the \ili{Khwe} examples~\REF{09-mc-ex:9} and~\REF{09-mc-ex:10} above. In these environments that free copula served an indexing function in the Peircean sense – \cf footnote 7 on the term \textit{copula}. It is not unreasonable to presume that the same marker could also be used in ordinary verbal clauses to draw attention to an NP, indexing its presence and drawing the addressee’s attention to it. \ili{Khwe} example~\REF{09-mc-ex:42} illustrates this usage in one modern language. There is no reason to suppose either of these uses predates the other. In other words, the proposal is that \textit{(ʔ)à} began as an indexical word that served to draw attention to a referent entity, regardless of whether it occurred in a dedicated presentational clause or in an ordinary verbal clause.

\protectedex{
\ea\langinfo{Khwe}{West Kalahari Khoe}{\citealt[220]{Kilian-Hatz2008Grammar}}\\
\label{09-mc-ex:42} 
\gll ndée! tóm̀tom-xò à ŋ́kà tí à tóm̀-a-tè!{\rmfnm}\\
mum swallow-\textsc{nmlz} \textsc{cop} there 1\textsc{sg} \textsc{acc} swallow-\textsc{atv}-\textsc{prs}\\
\glt ‘Mum, there is a swallowing thing [\ie a python] that (wants to) swallow me!’
\z}
\footnotetext{Elsewhere, the same example is given a different free translation, “Mom, there is a swallowing thing here; it swallows me!” (\citealt[250]{Kilian-Hatz2008Grammar}). Given the discussion of the previous page (\citealt[249]{Kilian-Hatz2008Grammar}), this is inappropriate, and the monoclausal free translation given in~\REF{09-mc-ex:42} is preferable.}


My proposed initial stage is somewhat reminiscent of the initial stage suggested in \citet[376–378]{Kilian-Hatz2013Kxoe} and \citet[276–277]{Koenig2008Case}. In all of the scenarios \textit{(-)(ʔ)à}  began as a type of copula, though in Kilian-Hatz’s and König’s accounts it was not specifically a presentative one. It was, however, restricted to verbless clauses with \isi{indefinite} subjects. This became a \isi{focus marker} that introduced new information, in particular \isi{indefinite} subjects and \isi{indefinite} objects. By contrast, I take the presentative use – an attentional resource that permits the speaker to direct attention to something so that it comes to occupy the centre of the joint attentional frame \citep{Tomasello2003Constructing} – across both verbal and verbless clause types to be the original source stage for the diachronic changes; how that relates temporally with the use of \textit{(ʔ)à} as an attributive or identifying copula is not clear to me, and is irrelevant to my scenarios for the development of the accusative marker.

\newpage
A crucial feature of Figure~\ref{09-mc-fig:2-Scheme} and~\ref{09-mc-fig:2-Scheme2} is that the indexing function of \textit{(-)(ʔ)à} was not restricted to NPs in any particular grammatical relation. Nonetheless, from early on, at least from the second stage, it was strongly associated statistically with certain types of object, specifically \isi{definite} objects – not \isi{indefinite} ones, as per \citet[376–378]{Kilian-Hatz2013Kxoe} – less strongly with \isi{indefinite} objects and \isi{intransitive} subjects, and perhaps even less with transitive subjects and locatives. What motivated these initial associations?

In essence, my answer is that NPs were indexed and drawn to the addressee’s attention when they were unexpected for some reason. One circumstance in which \isi{unexpectedness} emerges is when the NP in the grammatical role does not fit with the prototype for the role. For the two core grammatical relations in transitive clauses the prototypes may be assumed to be something like the following:

\begin{itemize}
 \item Transitive subjects (Agents) are prototypically given (presuming my reformulation of \citeauthor{DuBois1987Discourse}’s (\citeyear{DuBois1987Discourse}) given A constraint, \citealt{McGregor1998Optional}), \isi{animate}, and \isi{definite};
\item Objects (Undergoers) are prototypically new, \isi{inanimate}, and \isi{indefinite} (\eg \citealt{Comrie1979Definite}).
\end{itemize}

For \isi{intransitive} subjects I presume no corresponding prototypical features: NPs in this role are not strongly associated with any particular givenness, \isi{animacy}, or \isi{definiteness} values. Thus different \isi{intransitive} clause types are associated with different norms on these dimensions. This means that for \isi{intransitive} subjects \isi{unexpectedness} must be based on considerations other than not matching a prototype. For NPs in this role either only the local discourse consideration that it is informationally new or \isi{indefinite} is relevant to the evaluation as unexpected, or (if given and/or \isi{definite}) the \isi{unexpectedness} relates to the identity of the filler of the role – some other entity being expected in the role.

What may have happened in the early stages of the scenarios in Figure~\ref{09-mc-fig:2-Scheme} and~\ref{09-mc-fig:2-Scheme2} is that object NPs were marked as unexpected primarily when \isi{definite}, when they failed to match this component of the role prototype. Ultimately, all or the majority of \isi{definite} objects came to be marked by \textit{(ʔ)à}.\footnote{I presume that this was a gradual process, beginning with only some \isi{definite} objects being marked, and that the frequency of marking increased over time. However, a rapid, virtually instantaneous event cannot be ruled out. Both are consistent with the proposed scenarios.} If, in these early stages, PGN markers were markers of \isi{definite}, \ie identifiable, NPs (as in some modern Kalahari Khoe languages, \eg \ili{Khwe} – \citealt[43]{Kilian-Hatz2008Grammar}, Ts’ixa – \citealt[63, 74]{Fehn2014Grammar}), the strong affinity of \textit{(ʔ)à} with PGN markers can be accounted for. For these NPs marking by \textit{(ʔ)à} became obligatory or almost obligatory, and this fed into the development of the marker into a suffix, and ultimately to loss of its separate status as a morpheme and its incorporation into the PGN forms of one of the series in Kalahari Khoe languages. This series is the one that is in all languages associated with NPs in object roles. In \ili{Khoekhoe} something different happened: the PGN markers generalised to all NPs regardless of \isi{definiteness}, and the \textit{-à} suffix went with it on all object NPs by extension. As already remarked, in Nama-Damara it seems that \textit{-à} is obligatory on objects; it may be optional in !Ora, but no information is available on the conditions of its use and non-use, and without relevant data it is pointless to speculate on its development.

The situation for \isi{indefinite} NPs was initially quite different, and remained different in Kalahari Khoe languages where PGN markers did not generalise to all NPs. Indefinite NPs satisfy the relevant component of the prototype, and their \isi{unexpectedness} – and thus marking by the optional \textit{(ʔ)à} presentative index – could only be based on local considerations relating to the discourse context. These local considerations concern information status (\eg whether new or contrastive) on the one hand and the extent to which it satisfies the patientivity profile prototypically associated with the role (whether it is more patientive than normal) on the other. The result was that in Kalahari Khoe languages the \textit{(ʔ)à} continued to be a free word with non-PGN-marked NPs, where it remained optional on object NPs. Reflection of these considerations remains in the modern languages, where, as seen in~\sectref{09-mc-sec:3}, the motivations for use or non-use of the accusative marker differ across the languages: information status is in all languages a relevant variable; patientivity profile is documented as a consideration only for \ili{Shua}.

The overall preference of \textit{(-)(ʔ)à} on objects was further skewed by its infrequent occurrence on subject NPs. In \ili{Khoekhoe}, marking of subject NPs became restricted to certain marked syntactic environments, such as on ``deposed" subjects. 
As a result, \textit{-à} was ultimately interpreted as an oblique suffix (as per the analysis of \citealt[341]{Haacke2013Gora}). In Kalahari Khoe languages the \textit{-à} series of PGN markers became the one that was consistently associated with objects; in addition, in some languages it was associated with subjects, presumably through extension from the occasional uses of the marker on PGN-marked subjects. When the free \textit{(ʔ)à} was extended to \isi{definite} NPs it was only to those in object roles. In those languages like \ili{Khwe} where the free \textit{(ʔ)à} also occurred on \isi{indefinite} subjects (almost invariably \isi{intransitive}), there was no corresponding extension to \isi{definite} subjects. Thus the occurrence of \textit{(ʔ)à} on subjects in \ili{Khwe} is a relic of the original indexical-presentative function; it is not a later extension of the marker to subjects. In East Kalahari Khoe languages this use either never arose or completely disappeared. The strength of the association with objects resulted in reinterpretation of \textit{(ʔ)à} as an accusative marker in Kalahari Khoe languages.

It is important to observe that it was \isi{definite} NPs that overwhelmingly tended to be marked by \textit{(-)(ʔ)à}; indeed, in many Kalahari Khoe languages, they are etymologically double marked. Two staged sequences of grammaticalisation of \textit{(-)(ʔ)à} were involved with \isi{definite} NPs, one resulting in the fusion of \textit{à} with PGN markers, the other involving the expansion in usage of the free reflex of \textit{(-)(ʔ)à} (indicated by the greyed boxes in Figure~\ref{09-mc-fig:2-Scheme2}). Both were motivated by the fact that \isi{definite} NPs failed to match the prototype for objects. At some point in the first sequence only \isi{direct object} NPs that were \isi{indefinite} (\ie non-PGN-marked) were marked by the free \textit{(ʔ)à}. This situation is highly marked and unusual in that more prototypical objects are morphologically more marked than less prototypical ones. The extension of the marker to \isi{definite} \isi{direct object} NPs may have been driven by this disparity. It is likely that at the beginning of the second sequence, as of the first, the marker was a presentative index, and that it was only subsequently reanalysed as an accusative marker. Once established as the norm for \isi{definite} NPs, no longer would the \isi{presentative meaning} be associated with \textit{(ʔ)à}. The loss of its \isi{presentative meaning} may have been what ultimately led to the reanalysis of \textit{(ʔ)à} as an accusative marker on \isi{definite} NPs: without its \isi{presentative meaning} and with the \isi{definite} meaning being marked by the presence of the PGN marker, the only meaning available in this circumstance for the morpheme \textit{(ʔ)à} was accusative. Subsequently this reanalysis extended to \isi{indefinite} NPs as well, where the free \textit{(ʔ)à} was reanalysed as an optional accusative marker, as shown in the final stage of Figure~\ref{09-mc-fig:2-Scheme2}.

The reinterpretation of \textit{(ʔ)à} as an accusative marker was concomitant with its loss of its inherent presentative value. What happened at this stage was that what was strongly associated with \textit{(ʔ)à} came to be interpreted as its coded meaning; correspondingly, the coded presentative sense was lost. Simultaneously with this, meanings became associated with the use and/or non-use of the accusative marker, which was optional at least on \isi{indefinite} NPs. As per my theory of optional grammatical marking (\citealt{McGregor2013Optionality}), the meanings that could be associated with usage and/or non-usage of the marker were restricted to different values of the features [prominent] and [backgrounded] and their combinations. This process of reinterpretation involved no significant meaning change, as \textit{(ʔ)à}-marking of object NPs already presented the NP to the addressee’s attention. In short, the processes involved at this point are well known processes in grammaticalisation, the replacement of coded meaning with habitually associated meaning. In the present case both presentative and case meanings remained, albeit in somewhat modified forms. The meanings also changed their loci of expression: \isi{presentative meaning} – in the revised form [+prominent] – became associated with usage of the morpheme, while the case meaning habitually associated with the morpheme took the place of its coded meaning.

An important difference between the two sequences involved in Figure~\ref{09-mc-fig:2-Scheme2} is that in the first one marking by \textit{(ʔ)à} was strongly associated with \isi{definite} NPs from the beginning, whereas in the second sequence the marking was initially most strongly associated with \isi{indefinite} NPs. The former situation is contrary to the scenario of \citet[376]{Kilian-Hatz2013Kxoe}. It might seem that Kilian-Hatz’s initial stage is more in keeping with the use of a presentative marker, which presumably generally serves to introduce new items into the discourse. Two observations attest to the plausibility of my interpretation. First, I would agree that introduced items typically present new information, information that is not retrievable from the previous discourse. However, in that it indexes the entity, the marker presents the item as identifiable by the addressee, namely the target of the index. The situation in the second stage is as assumed by \citet{Kilian-Hatz2013Kxoe}. But its foundation is quite different from that assumed in \citet[376]{Kilian-Hatz2013Kxoe}. It is not because of an association of a \isi{focus marker} with \isi{indefinite} NPs, but rather a consequence of reanalysis of the PGN morphology that was associated with \isi{definite} NPs. Second, the presentative marker was used to draw attention to something, to single it out as noteworthy, and not necessarily to introduce it. In general one can expect that a speaker will draw attention to something when there is something unusual or unpredictable about it. This may be that it is assumed to be unknown to the hearer, and needs to be presented to them; but there are other reasons that concern not the identity of the thing, but \eg whether it matches the prototype for the grammatical role. In other words, what the addressee’s attention is drawn to need not necessarily be the presence of the entity in the context.

In Kalahari Khoe languages the strengths of the associations of \textit{(ʔ)à} with object NPs vary across the languages and within them according to the circumstances, as seen in~\sectref{09-mc-sec:3}. In \ili{Shua} the strongest associations are with NPs very high on the \isi{animacy hierarchy}, including pronouns and personal names, where \textit{(ʔ)à} is (almost) obligatory. A similar thing happened in \ili{Khwe}, although somewhat unexpectedly \textit{(ʔ)à} is obligatory on personal names but not on pronouns. In Ts’ixa \isi{definiteness} and \isi{word order} seem to have been the major factors (see \tabref{09-mc-tab:7} above). In Kalahari Khoe languages obligatorification of \textit{(ʔ)à} on object NPs remained local and restricted, unlike the situation for the \textit{à} series of PGN markers (obligatorily chosen if a PGN was used on the object NP) and \textit{-à} in Nama-Damara. Elsewhere \textit{(ʔ)à} remained optional. Nonetheless, there were evidently statistical differences in the frequency of usage of \textit{(ʔ)à} depending on these factors, and corresponding differences in motivations for presence vs. absence of \textit{(ʔ)à}.

Let us see how the modern situations might have arisen historically. 
Here I outline the three relevant scenarios, linking them to actual situations in the modern languages. I do not attempt to account for the situations in the modern languages – impossible given the present state of knowledge.

In contexts in which the marker was infrequently used on object NPs use of \textit{(ʔ)à} simply took the value of the morpheme as an attention-director, while no meaning was associated with its absence, the normal and unmarked condition. 
The expression-locus of the \isi{presentative meaning} shifted from the morpheme itself to its use. This is the situation for \isi{inanimate} and lower order \isi{animate} object NPs in \ili{Shua}.

Where neither use nor non-use of \textit{(ʔ)à} was strongly dominant, the same shift in the expression-locus of meaning could have occurred, and use could still be associated with prominence as an attention director (although this may have been a somewhat reduced type of prominence vis-à-vis the initial state where the marker was rarely used). At the same time, as non-use of the marker became a less frequent choice, and this choice became more restricted, non-use could have begun to acquire a meaning. When use and non-use had become roughly equal in frequency the contrast between them was liable to be reinterpreted as an equipollent one, in which neither is marked with respect to the other. In this circumstance, rather than carrying a complementary meaning to use, non-use conveyed a qualitatively different meaning. According to my theory of \isi{optionality}, there are restrictions on what this new meaning can be: it must be [backgrounded] (\citealt{McGregor2013Optionality}). Thus one arrives at the situation represented in the final column of \tabref{09-mc-tab:5}, which may be the situation for PGN-marked NPs in \ili{Shua} (see \tabref{09-mc-tab:6}).\footnote{At this stage then for NPs of the specified type the speaker is forced to choose between foregrounding and backgrounding the object NP. There is no option of conveying a neutral meaning or a (strongly) focal meaning. If such meanings are desired, then other means of expression might be chosen by the speaker, \eg expression by a pronominal rather than a lexical NP or use of another focal strategy such as \isi{word order} or intonation.}

Where the frequency of use of \textit{(ʔ)à} was or became high, as on non-prototypical object NPs such as pronouns in \ili{Shua} and PGN-marked \isi{definite} NPs in \ili{Khwe}, the original attention-directing value of the marker would be completely lost with the high degree of usage. Assuming the increase in frequency of use was a gradual process, the meaning associated with non-use as per the previous paragraph would have been retained. In such circumstances only non-use of the marker would be meaningful, as in the case of personal name objects in \ili{Shua}. The association of meaning with non-use of \textit{(ʔ)à} is not however dependent on gradual increase in the use of the marker. The same process as invoked in the previous paragraph could account for a meaningful non-use even if this arose virtually instantaneously.

To wind up this discussion, it is worth drawing a brief comparison with the grammaticalisation of optional ergative case-markers in Australian languages; this lends some credibility to the proposed grammaticalisation scenario for Khoe languages. A number of Australian languages exhibit, as suggested by \eg \citet{McGregor2010Optional,McGregor2013Optionality,McGregor2017Grammaticalization}, the association of a type of focal marker with transitive subject NPs, where the focal marker was originally an indexical element. This is a plausible source for the optional ergative marker in some Australian languages. The critical grammaticalisation processes here are essentially the same as involved in the development of the optional accusative in the Khoe family: highlight and draw attention the unexpected and/or non-prototypical. The differences concern on the one hand which of the two roles of transitive clauses was selected for this special attention, and on the other the nature of the erstwhile indexical element – a presentative copula in Khoe languages, often a determiner or pronominal element in Australian languages. Furthermore, it is noteworthy that in both Australia and southern Africa evidence of the earlier attention-directing meaning remains in some languages in the otherwise inexplicable occasional use of the marker on subjects of \isi{intransitive} clauses.

\section{Conclusions}\label{09-mc-sec:5}
I have suggested that – despite the cautions rightly voiced by \citet[319–320]{Fehn2014Grammar} – it is possible to propose a viable scenario for the emergence and development of the marker \textit{(-)(ʔ)à} as an accusative marker in Khoe languages. This scenario is preferable to the proposals of Kilian-Hatz (\citeyear[55]{Kilian-Hatz2008Grammar}, \citeyear[376–378]{Kilian-Hatz2013Kxoe}). It postulates an initial state in which \textit{(ʔ)à} was a presentative copula, and traces its development into the final vowel of a set of PGN markers that are consistently associated with NPs in the object role and an optional accusative marker in most Kalahari Khoe languages, and into an oblique suffix in \ili{Khoekhoe}.

I have discussed the ranges of uses of \textit{(-)(ʔ)à} across the Khoe family in as much detail as possible given present knowledge and limitations of space, in the belief that – in circumstances such as those that Khoe languages find themselves where time depth is seriously lacking – a motivated diachronic scenario requires a broad spectrum of \isi{synchronic variation}. I have also as far as possible attempted to motivate stages and developments among them through reference to other documented processes of grammaticalisation – in the present instance, primarily to development of optional \isi{ergative case} marking (\eg \citealt{McGregor2010Optional,McGregor2013Optionality,McGregor2017Grammaticalization}).

\newpage 
Much more work needs to be done aside from the above-mentioned need for careful synchronic investigations of the motivations of optional accusative marking in Khoe languages. First, I have ignored the dimension of \isi{word order}, which is likely to also be a significant factor in the grammaticalisation of \textit{(-)(ʔ)à}. This awaits more detailed investigations of \isi{word order} in most modern Khoe languages. Second, my scenario focuses on grammaticalisations of \textit{(-)(ʔ)à} to a marker of \isi{direct object} NPs. I have not included in the diachronic story its role as a marker of indirect objects. Contrary to the assertions of \citet[373]{Kilian-Hatz2013Kxoe}, \textit{(ʔ)à} behaves in a very different way on indirect objects to direct objects, and it is not obvious how the account of the grammaticalisation of \textit{(ʔ)à} as an accusative marker on direct objects should be extended to account for its use on indirect objects. Nor have I addressed the development of the genitive, attributive and identifying copula, and other functions of \textit{(-)(ʔ)à} found in Khoe languages.

\section*{Acknowledgements}
The fieldwork on which this paper is based was supported financially by the Danish Agency for Science, Technology and Innovation through the EuroBABEL Program of the European Science Foundation. I am grateful to the speakers of Shua who generously shared their language with me, most particularly to: Blesswell Kure, Jerryson, †Pono Zuka, Ketebualemang Zuka, Ketshepilemang Kure, Elder Masoko, Margaret Rock, Kakabe Rasethora, Reggie Jerryson, Kegakolegetswe Reggie, Boithaganelo Riitsana, and Dikanelang Hashe. Without their assistance and involvement in the project, this paper would have been impossible. The initial research on this project was carried out in July 2015 under a Visiting Fellowship from the School of Literature, Languages and Linguistics, at the Australian National University. I am grateful to Jane Simpson for the invitation and for hosting me, and to the School of Literature, Languages and Linguistics for the Fellowship. A previous version of this paper was presented at the Centre for Research on Language Change on 5\textsuperscript{th} August 2015. Thanks to the audience for very useful discussion. Thanks also to the editors, an anonymous referee, Anne-Maria Fehn and Tom Güldemann for comments, overwhelmingly negative, on an earlier draft; these have caused me to significantly reformulate my proposals, hopefully for the better. Doubtless none of them would want to take responsibility for any of the reformulations.


\section*{Abbreviations}
\begin{tabularx}{.45\textwidth}{lQ}
\textsc{abl} & ablative\\
\textsc{acc} & accusative\\
\textsc{adv} & adverbial\\
\textsc{ag} & agentive nominalisation\\
\textsc{all} & allative\\
\textsc{appl} & applicative\\
\textsc{atv} & non-past \isi{active}\\
\textsc{c} & common gender\\
\textsc{caus} & causative\\
\textsc{comp} & complementiser\\
\textsc{conj} & conjunction\\
\textsc{cop} & copula\\
\textsc{decl} & declarative particle\\
\textsc{dem} & demonstrative\\
\textsc{dom} & differential object marking\\
\textsc{du} & dual\\
\textsc{f} & feminine\\
\textsc{foc} & focus\\
\textsc{gen} & genitive\\
\textsc{gn}  & geographical name\\
\textsc{hab} & habitual\\
\textsc{idtf}  & identified\\
\textsc{ipfv} & imperfective\\
\textsc{ind} & indicative\\
\textsc{ins} & instrumental\\
\textsc{io}  & {indirect object}\\ 
\end{tabularx}
\begin{tabularx}{.45\textwidth}{lQ}
\textsc{irr} & irrealis\\
\textsc{j} & juncture morpheme\\
\textsc{loc} & locative\\
\textsc{m} & masculine\\
\textsc{neg} & negative marker\\
\textsc{nmlz} & nominaliser\\
\textsc{np} & noun phrase\\
\textsc{npst} & near past tense\\
\textsc{o} & {direct object}\\
\textsc{obl} & oblique\\
\textsc{pass} & passive\\
\textsc{per} & perlative\\
\textsc{pgn} & person-gender-number (marker)\\
\textsc{pl} & plural\\
\textsc{poss} & possessive\\
\textsc{pot} & potential\\
\textsc{prog} & progressive\\
\textsc{prs} & present\\
\textsc{prv} & privative\\
\textsc{pst} & past\\
\textsc{refl} & reflexive\\
\textsc{rpst} & remote past tense\\
\textsc{s} & subject\\
\textsc{sdpst} & same day past\\
\textsc{sg} & singular\\
%TODO
% V -{}- verb. The first three numerals indicate the persons, first, second, and third respectively\\
%\textsc{i} & paradigm\\
%\textsc{i} of the PGN markers\\
%\textsc{ii} &  paradigm\\
%\textsc{ii} of the PGN markers
\end{tabularx}

{\sloppy
\printbibliography[heading=subbibliography,notkeyword=this] }
\end{document}
