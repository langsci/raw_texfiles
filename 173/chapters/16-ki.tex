\documentclass[output=paper]{LSP/langsci}
\ChapterDOI{10.5281/zenodo.1228273} 
\author{Seppo Kittilä\affiliation{University of Helsinki}\lastand Jussi Ylikoski\affiliation{University of Oulu \& Sámi University of Applied Sciences}}
\title{Some like it transitive: Remarks on verbs of liking and the like in the Saami languages}
\shorttitlerunninghead{Some like it transitive: Remarks on verbs of liking in the Saami languages}

\abstract{Canonical DOM is rather uncommon in the Saami languages (Uralic), and the only clear instances of this are attested in South Saami where definiteness does determine the coding of objects in the plural. On the other hand, the coding of experiencer verbs (e.g., ‘like’, ‘care’ and ‘fear’) displays variation in this regard across Saami languages. With the North Saami verb \textit{liikot} ‘like’, for example, the stimulus may appear in the illative, genitive-accusative and locative cases without any major difference in meaning. This has usually been viewed as unwelcome influence from the majority languages (Norwegian, Swedish and Finnish). In this paper, however, we will argue that it is no coincidence that the variation concerns mainly experiencer verbs, and more specifically, we will show that the attested variation can be seen as an uncanonical instance of DOM. First of all, the variation in the coding is not semantically determined in the sense that it does not affect the semantic roles of the relevant arguments, which is typical of canonical DOM as well. Second, differently from canonical instances of DOM, the variation concerns semantic cases instead of structural cases, and the variation is between two non-zero cases, while canonical DOM is between zero and non-zero case. Third, the conditioning factors are different from canonical DOM, since animacy and definiteness do not contribute to the discussed variation in any direct way. Language contact does play an important role in this process, but the pursuit of coherence, the semantic emptiness of the cases, and features of semantic transitivity also make a significant contribution to the variation.

% \keywords{Saami languages, Differential Object Marking, language contacts, case marking, experiencer verbs}
}
\maketitle

\begin{document}

\section{Introduction}
\label{16-sec:1}

As is typical of all languages discussed in this volume, instances of canonical differential object marking are attested also in the \ili{Saami} languages, as illustrated below:

\ea\langinfo{South Saami}{Uralic}{\citealt[60]{Bergsland1994Sydsamisk}}\label{16-ki-ex:1}
\ea\label{16-ki-ex:1a}
\gll Laara treavkah dorjeme.\\
 L. ski.\textsc{pl}(.\textsc{nom}) make.\textsc{pst}.\textsc{ptcp}\\
\glt  ‘Laara has made (a pair of) skis.’ 

\ex\label{16-ki-ex:1b}
\gll Dejtie treavkide vööjnim.\\
 it.\textsc{pl}.\textsc{acc} ski.\textsc{pl}.\textsc{acc} see.\textsc{pst}.1\textsc{sg}\\
\glt  ‘I saw the skis.’ 
\z
\z

As the above examples show, \isi{indefinite} objects in South \ili{Saami} bear (zero) nominative coding \REF{16-ki-ex:1a}, while \isi{definite} objects appear in the accusative \REF{16-ki-ex:1b}. The examples in \REF{16-ki-ex:1} thus constitute a typical instance of DOM, as the notion is typically understood, even though it should be noted that the variation illustrated in \REF{16-ki-ex:1} is limited to the plural while canonical DOM concerns also objects in the singular. Moreover, differently from many other languages with DOM, \isi{animacy} appears to play no role for DOM in South \ili{Saami}. Instead, the variation in object coding in \REF{16-ki-ex:1} is determined solely by \isi{definiteness}.\footnote{As pointed out by \citet[208]{Siegl2012Strovtag}, however, the nominative/accusative DOM in South \ili{Saami} has not been studied thoroughly. Furthermore, the contemporary object marking seems to differ from that of the language system depicted by earlier grammarians.} This is in line with the object coding in other \ili{Uralic} languages (see, e.g., \citealt{Virtanen2015Transitivity} for a discussion of Eastern Mansi).

 Even though the examples above can be viewed as a canonical instance of DOM, they do not constitute the most widespread type of variation in the object coding in the \ili{Saami} languages. Quite the opposite, canonical DOM in the form illustrated in \REF{16-ki-ex:1} seems to be limited to South \ili{Saami} only. Much more common across the \ili{Saami} languages is the kind of variation illustrated in \REF{16-ki-ex:2} from North \ili{Saami}:

\begin{exe}
\ex%2
\label{16-ki-ex:2}
\langinfo{North Saami}{Uralic}{personal knowledge}\\
\gll Vástit gažaldaga {\textasciitilde} gažaldahkii!\\
answer.\textsc{imp}.2\textsc{sg} question.\textsc{genacc} {\textasciitilde} question.\textsc{ill}\\
\glt  ‘Answer the question!’
\end{exe}


\begin{exe}
\ex%3
\label{16-ki-ex:3}
\langinfo{North Saami}{Uralic}{personal knowledge}\\
\gll Itgo liiko gažaldaga {\textasciitilde} gažaldahkii {\textasciitilde} gažaldagas?\\
\textsc{neg}.\textsc{2sg}.\textsc{q} like.\textsc{cng} question.\textsc{genacc} {\textasciitilde} question.\textsc{ill} {\textasciitilde} question.\textsc{loc}\\
\glt  ‘Don’t you like the question?’
\end{exe}


As shown above, the object may appear in the (genitive-)accusative, illative and also \isi{locative case} without any major changes in semantics.\footnote{In the \ili{Saami} grammatical tradition, only the (genitive-)accusative (and South \ili{Saami} nominative plural) are regarded as \textit{object} cases. Semantic cases such as the illative and the locative in analogous functions are usually characterized as \textit{adverbials}. For the purposes of the present paper, all non-nominative arguments of the type seen in \REF{16-ki-ex:2}--\REF{16-ki-ex:3} are regarded as \textit{objects} in the sense that they are not subjects and they are parts of the valence of the verbs in question, as it would be somewhat awkward to label freely alternating arguments as either (accusative) \textit{objects} or (illative, locative or elative) \textit{adverbials} on the basis of their external appearance only.} In other words, both constructions in \REF{16-ki-ex:2} mean ‘Answer the question!’, and the three alternatives in \REF{16-ki-ex:3} pose the same question as to whether the hearer likes the question or not. It is also noteworthy that neither \isi{definiteness} nor \isi{animacy} contribute to the attested variation. As for \REF{16-ki-ex:3}, it should be noted that the kind of variation exemplified here is one of the favorite eyesores of \ili{Saami} language purists, because usually only one of the three alternatives is deemed good North \ili{Saami}. The variation illustrated above is not limited to North \ili{Saami}: as the discussion in this paper will show, it is attested in other \ili{Saami} languages as well. The variation seems to be most common for \isi{experiencer} verbs \REF{16-ki-ex:3}, which will be the focus of our study. It is, however, important to note that similar phenomena can to some extent be observed for other verbs, such as \textit{vástidit} ‘answer’ in \REF{16-ki-ex:2}.

There are numerous studies dealing with DOM from different perspectives, as the chapters of this volume also very well show. Most of these studies are characterized by two important common features. First, the variation is between two structural cases, usually a zero-marked nominative (or absolutive) and an explicitly marked accusative (or accusative-dative) case. Second, the great majority of DOM studies restrict the notion to cases where variation in object coding is determined by \isi{animacy}, \isi{definiteness} or \isi{topicality} on the coding of objects. This paper also adds an entry to the already long list of DOM studies, but the type of DOM examined here is clearly different from that usually discussed. First, the typical DOM triggers, namely \isi{animacy} and \isi{definiteness} (or \isi{topicality}), play no role in object coding. Second, the variation is often between semantic cases (e.g., locative and illative), although the accusative partakes in the variation as well. Third, we are dealing with a clear instance of lexically restricted predicate-triggered DOM, in \ili{Saami} languages attested primarily (yet not exclusively) for \isi{experiencer} verbs.

Despite the evident differences from typical DOM studies, the variation examined in this paper also resembles canonical DOM in certain respects. The semantic roles of the differently coded objects do not vary in the cases discussed, which can be claimed to be true of canonical DOM as well; in the cases discussed in this paper, the role of the differently coded objects, regardless of their coding, is that of a stimulus. We hope that our study will broaden our perspectives on DOM and help to identify similar phenomena in other languages as well. The number of examples discussed in this paper and attested in \ili{Saami} languages is not very high, but they nevertheless provide us with clear clues as to what kind of variation we are dealing with.

As noted above, the instances of DOM discussed in this paper differ from the type typically discussed under DOM. The topic is also rather novel to \ili{Saami} linguistics, where the type of variation illustrated in \REF{16-ki-ex:2}--\REF{16-ki-ex:3} is usually understood as unwanted interference from majority languages or language decay (cf., e.g., \citealt[425]{Vuolab-Lohi2007Mailmmi}; \citealt{Olthuis2009Mii}: 86–87). In this paper, the problem is approached from a more general perspective. The main features considered here are the effects of language contact, emptiness of semantic cases and tendency towards coherence in marking. In other words, we will show that the variation is not random and not necessarily a result of language decay following from language contact, as is often the view of language purists, but that it can also be given a valid language-internal explanation. Moreover, it is not a coincidence that the variation concerns \isi{experiencer} verbs and not, for example, highly transitive verbs. It is, however, important to note that DOM is still a rather limited phenomenon in the \ili{Saami} languages. It is attested mainly with \isi{experiencer} verbs, and moreover, it applies only to a small set of these verbs. Despite this, we hope that our paper provides new insights into DOM.

The discussion in this paper is based on the six most widely spoken \ili{Saami} languages, which are described in the following section. We illustrate and discuss examples of many different \isi{experiencer} verbs. However, it is not our goal to give a systematic overview of the variation between different verbs; instead, the variety of verbs serves only the purpose of illustrating the nature and limits of the variation under discussion.

 The organization of the paper is as follows. \sectref{16-sec:2} discusses the examined \ili{Saami} languages and their basic argument marking patterns as they are relevant to the discussion in this paper. \sectref{16-sec:3} presents a set of concrete examples of the differential coding of \isi{experiencer} verbs in \ili{Saami} languages. In \sectref{16-sec:4}, the main theoretical implications of the paper are discussed.\footnote{We wish to thank the editors, an anonymous reviewer and Nils Øivind Helander for a number of valuable comments on earlier versions of this paper. We also thank Tiina Sanila-Aikio and Eino Koponen for help and discussions on Skolt Saami and Elisabeth Scheller for corresponding help with Kildin Saami.}

\section{The Saami languages and argument marking}
\label{16-sec:2}

The \ili{Saami} branch of the \ili{Uralic} language family consists of a chain of closely related languages whose territory extends from the central parts of Norway and Sweden up to the Kola Peninsula of northwest Russia. Of the nine or ten living \ili{Saami} languages, seven have official literary standards and six of them have several hundred or even thousands of speakers each. The discussion in this paper focuses on data from the following six languages with the most speakers and widest literary use: South \ili{Saami} (Norway, Sweden), Lule \ili{Saami} (Norway, Sweden), North \ili{Saami} (Norway, Sweden, Finland), \ili{Aanaar} (Inari) \ili{Saami} (Finland), Skolt \ili{Saami} (Finland, Russia), and \ili{Kildin Saami} (Russia). Our data is either drawn from or otherwise based on the literary use of the present-day languages. As the total number of speakers of the \ili{Saami} languages is less than thirty thousand, all \ili{Saami} languages are minority languages except in two Norwegian municipalities, where North \ili{Saami} is the majority language of the local communities. As a consequence, virtually all present-day speakers of \ili{Saami} languages are bi- or trilingual to some extent, and this naturally affects the minority languages in many ways – argument marking not being an exception.

 Not unlike in the other \ili{Uralic} languages, the morphosyntax of the \ili{Saami} languages is largely based on the interplay of morphological cases. All \ili{Saami} languages are explicitly nominative–accusative languages, with the zero-marked nominative for subject arguments and descendants of the \ili{Proto-Saami} (and ultimately Proto-\ili{Uralic}) accusative for direct objects. However, the picture is partly blurred by the fact that in the \ili{Saami} languages east of Lule \ili{Saami} (including North, \ili{Aanaar}, Skolt and \ili{Kildin Saami}), both the original accusative (*-\textit{m}) and genitive (*-\textit{n}) singular case suffixes have been lost, and the two cases have merged into one, the genitive-\isi{accusative case}, which for most nouns differs from the nominative only by stem-internal differences. Furthermore, the same language border – between Lule \ili{Saami} in the west and North \ili{Saami} in the east – witnesses the merger of two local cases in the west: the descendants of the \ili{Proto-Saami} inessive (‘at’) and elative (‘from’) survive in the so-called \isi{locative case} of the easternmost languages.\footnote{The \ili{Proto-Saami} inessive (*-\textit{sna}) and elative (*-\textit{sta}) are cognate with their namesakes in Finnic languages such as \ili{Finnish} and \ili{Estonian} (see, e.g., \citealt{Sammallahti1998Saami}: 66–71, 203). However, in the absence of the so-called external local cases characteristic of Finnic, the \ili{Saami} cases are also used in most of the functions of the Finnic cases adessive and ablative. As a consequence, the single locative cases in languages like North, \ili{Aanaar} and Skolt \ili{Saami} (all spoken in Finland) as well as \ili{Kildin Saami} roughly correspond to as many as four local cases in \ili{Finnish}. In the same vein, the \ili{Saami} illative (*-\textit{se̮n}) is cognate with the Finnic illative (*-\textit{sen}), but is also a functional equivalent of the Finnic allative.} On the other hand, the genitive–accusative merger is total – affecting both singular and plural forms – in North \ili{Saami} only, but not in the easternmost languages (including \ili{Aanaar}, Skolt and \ili{Kildin Saami}), which have retained the distinction in the plural.\footnote{When speaking of the “easternmost \ili{Saami} languages”, we are not taking a stance on whether or not the \ili{Saami} branch must be strictly divided to two – Western \ili{Saami} and Eastern \ili{Saami} with capital letters – along the phonologically significant, but lexically less decisive border between North \ili{Saami} and \ili{Aanaar} \ili{Saami}. For a comprehensive discussion of these issues, see \citet{Rydving2013Words}.}

\begin{table}
\caption{The South, Lule and North Saami case systems exemplified with the words for ‘fish’}
\label{16-ki-tab:1}
\fittable{
\begin{tabular}{p{27mm}llllllp{15mm}} 
\lsptoprule
& \multicolumn{2}{c}{South Saami~~~~~~~~~~~~} & \multicolumn{2}{c}{Lule Saami} & \multicolumn{2}{c}{North Saami} & \\
& Singular & Plural & Singular & Plural & Singular & Plural & \\
\midrule

Nominative & \textit{guelie} & \textit{guelieh} & \textit{guolle} & \textit{guole} & \textit{guolli} & \textit{guolit} & Nominative\\
\midrule

Accusative & \textit{gueliem} & \textit{guelide} & \textit{guolev} & \textit{guolijt} & \multirow{2}{*}{\centering\textit{guoli}} & \multirow{2}{*}{\centering\textit{guliid}} & Genitive-\\
\cmidrule{1-5}

Genitive (‘of’) & \textit{guelien} & \textit{gueliej} & \textit{guole} & \textit{guolij} & & &accusative \\
\midrule

Illative (‘to’) & \textit{gualan} & \textit{guelide} & \textit{guolláj} & \textit{guolijda} & \textit{guollái} & \textit{guliide} & Illative\\
\midrule

Inessive (‘at’) & \textit{guelesne} & \textit{gueline} & \textit{guolen} & \textit{guolijn} & \multirow{2}{*}{\centering\textit{guolis}} & \multirow{2}{*}{\centering\textit{guliin}} & Locative\\

\cmidrule{1-5}
Elative (‘from’) & \textit{gueleste} & \textit{guelijste} & \textit{guoles} & \textit{guolijs} & & & (‘at;~from’)\\
\midrule

Comitative (‘with’) & \textit{gueline} & \textit{gueliejgujmie} & \textit{guolijn} & \textit{guolij} & \textit{guliin} & \textit{guliiguin} & Comitative\\
\midrule

Essive (‘as’) & \multicolumn{2}{c}{ \textit{gueline}} & \multicolumn{2}{c}{ \textit{guollen}} & \multicolumn{2}{c}{ \textit{guollin}} & Essive\\
\lspbottomrule
\end{tabular}
}
\end{table}

Individual \ili{Saami} languages also exhibit various degrees of syncretism within plural case forms and between the plural inessive/locative and singular comitative, for example. In addition, some of the languages make use of additional cases or regressing case-like adverbs labeled as abessives and partitives, but as their functions fall outside the immediate scope of the present paper, they will be omitted in the following description of argument marking in \ili{Saami}. (For a more comprehensive description of the \ili{Saami} case markers and their syncretism, see, e.g., \citealt{Sammallahti1998Saami}: 65–71; \citealt{Hansson2007Productive}.) For the purposes of the present paper, the common core of the \ili{Saami} case morphology is presented in \tabref{16-ki-tab:1}, which exemplifies the case systems in South, Lule and North \ili{Saami}.

As regards phenomena that can be labeled as \isi{differential argument marking}, as many as five of the six to eight cases in \tabref{16-ki-tab:1} are involved: nominative, (genitive-)accusative, illative, locative (elative) and comitative.\footnote{In addition to the genitival functions of the genitive(-accusative) and the spatial (‘at’) semantics of the inessive/locative, the functions of the essive case are not directly relevant for the present discussion, although the interplay between nominative- and essive-marked arguments and secondary predicates could also be regarded as \isi{differential argument marking} in the broad sense (\cf \citealt{Siegl2017Essive,YlikoskiEssive}, and \citealtv{Witzlacketal2017Differential}).} 
Of special interest here is the use of the local cases illative and locative as argument markers whose functions can hardly be distinguished from the direct objects marked with the (genitive-)accusative (and the South \ili{Saami} plural nominative). Although this paper pays special attention to the internal and mutual variation in \ili{Saami} argument marking, it is worth emphasizing that in spite of considerable phonological, morphological and lexical variation that makes even the closest \ili{Saami} languages mutually unintelligible, their morphosyntactic structures are essentially quite similar. In a nutshell, the variation between individual \ili{Saami} languages is comparable to the variation within the Germanic languages, for example.

 The basics of \ili{Saami} argument marking can be seen in the following examples from South \ili{Saami}:

\begin{exe}
\ex%4
\label{16-ki-ex:4}
\langinfo{South Saami}{Uralic}{SIKOR}\\
\gll Gosse aktem gåmmam gaavnedigan akte dejstie guaktijste laejpieh öösti jih dejtie varki byöpmedi.\\
 when one.\textsc{acc} woman.\textsc{acc} find.\textsc{pst}.\textsc{3du} one it.\textsc{pl}.\textsc{ela} couple.\textsc{pl}.\textsc{ela} bread.\textsc{pl}(.\textsc{nom}) buy.\textsc{pst}.\textsc{3sg} and it.\textsc{pl}.\textsc{acc} quickly eat.\textsc{pst}.\textsc{3sg}\\
\glt ‘When they (two) found a woman, one of them bought some loaves of bread and ate them quickly.’
\end{exe}

\begin{exe}
\ex%5
\label{16-ki-ex:5}
\langinfo{South Saami}{Uralic}{SIKOR}\\
\gll Daelie die rïektes aaksjoem bøøkti jïh ålmese vedti.\\
now then real axe.\textsc{acc} bring.\textsc{pst}.\textsc{3sg} and man.\textsc{ill} give.\textsc{pst}.\textsc{3sg}\\
\glt ‘Then he brought a real axe and gave it to the man.’
\end{exe}

\begin{exe}
\ex%6
\label{16-ki-ex:6}
\langinfo{South Saami}{Uralic}{SIKOR}\\
\gll Månnoeh dutnjien jeehkimen, Læjsa.\\
\textsc{1du} \textsc{2sg}.\textsc{ill} trust.\textsc{1du} Læjsa\\
\glt  ‘We (two) trust you, Læjsa.’
\end{exe}

\begin{exe}
\ex%7
\label{16-ki-ex:7}
\langinfo{South Saami}{Uralic}{SIKOR}\\
\gll Edtjem manne datneste bïlledh juktie im datnem lyjhkh?\\
 shall.\textsc{1sg} \textsc{1sg} \textsc{2sg}.\textsc{ela} fear.\textsc{inf} because \textsc{neg}.\textsc{1sg} \textsc{2sg}.\textsc{acc} like.\textsc{cng}\\
\glt  ‘Am I supposed to fear you if I don’t like you?’ 
\end{exe}

Explicit subject NPs (present in \REF{16-ki-ex:6} and \REF{16-ki-ex:7} are frequently omitted, as the subject participant can often be inferred from the context and the form of the finite verb. As for patients, themes or stimuli of various actions and events such as finding \REF{16-ki-ex:4}, buying \REF{16-ki-ex:4}, eating \REF{16-ki-ex:4}, bringing \REF{16-ki-ex:5}, and liking \REF{16-ki-ex:7}, the object is most often in the accusative. However, as was already briefly mentioned in the introduction, for plural objects such as loaves of bread, a somewhat classical example of differential object marking is available: accusative plural is used to refer to \isi{definite} objects, whereas less \isi{definite} objects may be expressed by the nominative plural. Thus, in \REF{16-ki-ex:4} ‘buying loaves of bread’ and ‘eating them (= the loaves)’ are expressed by the nominative and accusative, respectively. In singular, such objects are always marked by the accusative only. Although this underdescribed phenomenon seen in \REF{16-ki-ex:1} and \REF{16-ki-ex:4} would merit a separate study (see, e.g., \citealt[30–36]{Wickman1955Form}; \citealt[184–186]{Maggaetal2012Sorsamisk}), it is enough to state here that South \ili{Saami} seems to be the only \ili{Saami} language exhibiting differential nominative/accusative object marking (see also example \REF{16-ki-ex:1} above).

 For the purposes of the present paper, however, it is important to note that some verbs take their arguments in other cases, too, such as the local cases illative (‘to’) and elative (‘from’). To begin with, the illatives of all \ili{Saami} languages could actually also be labeled as \textit{datives}, as in addition to their spatial meaning (‘to’), the illatives are the default case for marking recipients such as the man to whom an axe is given in \REF{16-ki-ex:5}. Moreover, the illative-marked noun phrase of \REF{16-ki-ex:6} as an argument of the verb \textit{jaehkedh} ‘believe, trust’ is also reminiscent of so-called dative objects in cases such as \textit{ich glaube/vetraue dir} ‘I believe/trust you’ in \ili{German}, as the illatives of \ili{Saami} languages share many functions with the dative in \ili{German}. Example \REF{16-ki-ex:7} presents the second person singular pronoun \textit{datne} in two case forms: \textit{datnem} as the accusative object of the verb \textit{lyjhkedh} ‘like’, but \textit{datneste} as the elative (‘from’) complement of the verb \textit{bïlledh} ‘fear’. While the choice of cases like the ones seen here may have historical and metaphorical connections to the concrete spatial meanings of local cases, from a strictly synchronic perspective we are often dealing with verbs whose argument structures seem to require the use of the elative instead of the default \isi{accusative case} used for most verbs (\textit{bïlledh} ‘fear’) or vice versa (\textit{lyjhkedh} ‘like’). 

 Many \ili{Saami} languages show considerable variation as to which cases are used for marking arguments of verbs such as the \isi{experiencer} verbs for ‘trust’, ‘fear’ and ‘like’ as seen in \REF{16-ki-ex:6}--\REF{16-ki-ex:7}. While the nominative–accusative differential object marking seen in \REF{16-ki-ex:4} has a clear semantic function, it is more difficult to recognize possible semantic differences behind what seems to be more arbitrary variation in \ili{Saami} argument marking. As it turns out, however, an important source of the variation to be described in the following section seems to be the sociopolitical environment of the \ili{Saami} languages: in spite of relatively uniform morphosyntax, the present-day \ili{Saami} languages are mostly used by bilinguals whose other languages include nation-state languages as divergent as Norwegian, \ili{Swedish}, \ili{Finnish} and \ili{Russian}. While the Scandinavian (Norwegian and \ili{Swedish}) influence on \ili{Saami} syntax is rather uniform, \ili{Russian} is quite different, and \ili{Finnish} belongs to the altogether different stock of \ili{Uralic} languages.

\section{Data: experiencer verbs and their coding in Saami (with a special focus on ‘like’)}
\label{16-sec:3}

In this section, the linguistic coding of \isi{experiencer} verbs across languages and in \ili{Saami} languages will be discussed. After briefly commenting on the coding of \isi{experiencer} verbs from a cross-linguistic perspective, we present some of the semantic features that explain their less transitive coding. The section is devoted to the examination of \ili{Saami} data, especially focusing on the argument marking of verbs that denote positive emotions such as liking, loving and caring (see also \sectref{16-ki-sec:4-1} further below).

 It is received wisdom in linguistics that the coding of \isi{experiencer} verbs often deviates from the basic transitive pattern of a given language; for example, dative coding of the subject is common with \isi{experiencer} verbs (see, e.g., \citealt{Vermaetal1990Experiencer} and \citealt{Aikhenvaldetal2001Noncanonical}). These formal differences from basic transitive constructions of a given language are not random, instead following from the different \isi{semantic role} assignment of experiencing; \isi{experiencer} verbs do not involve an agent and a patient, but an \isi{experiencer} and a stimulus instead. Neither the \isi{experiencer} nor the stimulus is necessarily affected, while in typical transitive events, the patient must be affected in order to constitute a true patient. It is, however, important to note that \isi{experiencer} verbs do not constitute a semantically coherent verb class, but there are clear differences in their nature, which is also reflected in their coding. First, for example, in \ili{Finnish}, the partitive (with verbs like ‘love’ and ‘hate’), elative (‘like’), illative (‘get bored with’), allative (‘get mad at’) and also accusative (‘see’, ‘hear’) can appear with \isi{experiencer} verbs. Second, different classes of \isi{experiencer} verbs differ according to whether the stimulus or the \isi{experiencer} surfaces as the subject.

 With the \ili{Finnish} verbs noted above, the subject refers to the \isi{experiencer}, while the differently coded second argument codes the stimulus. However, there are other verbs, such as \textit{miellyttää} ‘please’, or \textit{pelottaa} ‘scare’, where the stimulus surfaces as the subject, and the partitively coded object refers to the \isi{experiencer}. In the same vein, in \ili{Saami} languages such as North \ili{Saami}, verbs like \textit{balddihit} ‘scare’ code the stimulus as a nominative subject and the \isi{experiencer} as a genitive-accusative object. Finally, there are also verbs such as \ili{Finnish} \textit{iloita} ‘rejoice’, where the (elatively coded) stimulus can be seen as a kind of optional oblique that can be left out if the reason for rejoicing is not contextually relevant. Again, the same can be said about \ili{Saami} verbs like North \ili{Saami} \textit{illudit} ‘rejoice’ (cognate of \ili{Finnish} \textit{iloita}), which will be discussed further below. Consequently, it is rather hard to make any cross-linguistic generalizations about the coding of \isi{experiencer} verbs apart from the fact they typically somehow deviate from basic transitive constructions. In this paper, the focus is exclusively on \isi{experiencer} verbs that code the stimulus as the (direct) object. This is very well in line with the goals of the paper, which is to show that there is a kind of differential marking for the objects of \isi{experiencer} verbs. Taking other types of \isi{experiencer} verbs into account may distort the results, because the attested variation follows from features that are not relevant to the discussion in this paper.

 \newpage 
 Argument marking of \isi{experiencer} verbs has received almost no attention in \ili{Saami} linguistics \textit{per se}. Except for North \ili{Saami}, the major \ili{Saami} language that is spoken by about 90\% of all speakers of the \ili{Saami} languages, grammatical descriptions of most \ili{Saami} languages contain only little information about argument marking. The general pattern of the existing school grammars (e.g., \citealt{Spiik1989Lulesamisk,Olthuis2000Kielaoppa,Moshnikoffetal2009Koltansaamen,Maggaetal2012Sorsamisk}) is to state that the object is marked by the \isi{accusative case}, whereas most other cases function as adverbials. The latter functions are described quite sporadically and impressionistically, though. For example, descriptions of South \ili{Saami} characterize the use of elative in clauses like \REF{16-ki-ex:7} as adverbials of cause, whereas some other complement-like elatives have been labeled as partial objects (\citealt[60–61, 72]{Bergsland1994Sydsamisk}; \citealt[186]{Maggaetal2012Sorsamisk}). On the other hand, the identical behavior of the Lule \ili{Saami} elative with verbs like \textit{ballat} ‘fear’ (cognate of South \ili{Saami} \textit{bïlledh} seen in \REF{16-ki-ex:7}) is explained as part of a larger whole, wherein verbs of fearing are said to co-occur with the object of fear marked by the elative \citep[98]{Spiik1989Lulesamisk}. Further still, Nickel and \citet[233, 236, 529–530]{Nickeletal2011Nordsamisk} describe the analogous use of the North \ili{Saami} \textit{ballat} ‘fear’ as an example of verbs that come close to being transitive but govern the \isi{locative case} instead. However, none of the grammars or other descriptions of \ili{Saami} syntax have paid significant attention to possible semantic reasons for not using the accusative for all object-like arguments. Additionally, little attention has been paid to the fact that in actual use, many verbs show variation in how the non-subject arguments are coded. The most remarkable exception in this respect is \citegen[134–143]{Helander2001Ii} study of the North \ili{Saami} illative in which he briefly examines the \isi{argument structure} of the emotion verbs \textit{áibbašit} ‘miss, yearn’, \textit{dorvvastit} ‘count on, rely on’, \textit{duhtat} ‘settle for’, \textit{jáhkkit} ‘believe’, \textit{liikot} ‘like’, \textit{luohttit} ‘trust’, \textit{oskut} ‘believe, have faith’ and \textit{suhttat} ‘get angry’, some of which also show variation between illative arguments and other cases as well as postpositions in their coding of the stimuli. The list could be continued with verbs like \textit{dolkat} ‘get fed up’, which takes either the illative or the locative, or \textit{illudit} ‘rejoice; celebrate’ and \textit{heahpanit} ‘be ashamed of’ with even more variation to be discussed further below. Any comparative studies that would focus on these issues and cover more than one \ili{Saami} language do not exist, however.

 In the following, such variation in argument marking will be described and discussed by examining the use of \isi{experiencer} verbs denoting liking in six \ili{Saami} languages. This particular group of verbs shows both language-internal and cross-\ili{Saami} variation, which makes it suitable for providing novel contributions to our understanding of less typical instances of DOM. Due to the deficiencies and often prescriptive attitudes of existing grammatical descriptions most of the data is drawn from authentic (in part translated) texts made available by the SIKOR corpus at UiT The Arctic University of Norway. Although much of our understanding of South, Lule, North and \ili{Aanaar} \ili{Saami} is backed up by comparatively large corpora, this study is predominantly qualitative.\footnote{With respect to the size of the language communities, the available corpora are quite large. With 21.1 million words for North \ili{Saami}, 0.8M for Lule \ili{Saami} and 0.7M for South \ili{Saami}, they contain approximately one thousand words per one speaker of the languages. As for the 1.3M words for \ili{Aanaar} \ili{Saami}, with about 400 speakers, the ratio is even higher.} As for our understanding of the severely endangered Skolt \ili{Saami} and \ili{Kildin Saami}, our observations are more dependent on second-hand sources and elicited information from native and second-language speakers.

 The first verb to be examined is the South \ili{Saami} \textit{lyjhkedh} ‘like’, an apparently recent loan from Scandinavian languages where especially the Norwegian \textit{like} (and to lesser extent \ili{Swedish} \textit{lika}) has approximately the same meaning and exhibits similar syntactic behavior. It was already seen in example \REF{16-ki-ex:7} above that \textit{lyjhkedh} is a transitive verb that takes an accusative object instead of elative or any other local case, for example. Yet fully in line with the general object marking pattern discussed in \sectref{16-sec:2}, the plural object is marked with either accusative or nominative, depending on whether its referent is \isi{definite} \REF{16-ki-ex:8} or \isi{indefinite} \REF{16-ki-ex:9}, respectively:

\begin{exe}
\ex%8
\label{16-ki-ex:8}
\langinfo{South Saami}{Uralic}{SIKOR}\\
\gll Im lyjhkh niejtide mah desnie.\\
\textsc{neg}.\textsc{1sg} like.\textsc{cng} girl.\textsc{pl}.\textsc{acc} \textsc{rel}.\textsc{pl} here\\
\glt  ‘I don’t like the girls here.’ 
\end{exe}

\begin{exe}
\ex%9
\label{16-ki-ex:9}
\langinfo{South Saami}{Uralic}{SIKOR}\\
\gll Dihte lyjhkoe åenehks mirhke ålmah, guktie månnoeh, Ajloe føørhkedi.\\
\textsc{3sg} like.\textsc{3sg} short dark man.\textsc{pl}.\textsc{nom} like \textsc{1du} Ajloe laugh.\textsc{pst}.\textsc{3sg}\\
\glt ‘She likes short dark men, like the two of us, Ajloe laughed.’
\end{exe}


In a word, \textit{lyjhkedh} behaves just like any normal transitive verb of South \ili{Saami}. By contrast, in Lule \ili{Saami} the analogous loan verb \textit{lijkkut} ‘like’ usually governs the \isi{illative case} instead. As a matter of fact, grammars and dictionaries present the illative as the only option (\citealt[97]{Spiik1989Lulesamisk}; \citealt{Kintel2012Julevsame} s.v.), but accusative objects also exist. Both alternatives are present simultaneously in \REF{16-ki-ex:10} where the illative NPs \textit{guolláj} and accusative \textit{dáv gåvåv} could apparently be exchanged with the accusative \textit{guolev} and illative \textit{dán gåvvåj} without a change in meaning:\footnote{In the Lule Saami corpus of approximately 800,000 words (SIKOR), nearly half of the 160 instances of the verb \textit{lijkkut} take an infinitive complement. Of the 88 instances with an NP complement, 80 are in the illative and 8 in the accusative, with no visible differences in meaning or distribution. Both cases are used to refer to singular and plural, \isi{animate} and \isi{inanimate}, \isi{definite} and \isi{indefinite} referents, for example.}

\begin{exe}
\ex%10
\label{16-ki-ex:10}
\langinfo{Lule Saami}{Uralic}{NuorajTV}\\
\gll Lijkku guolláj? De ham de lijkku dáv gåvåv aj?\\
like.\textsc{2sg} fish.\textsc{ill} then \textsc{dpt} then like.\textsc{2sg} this.\textsc{acc} picture.\textsc{acc} also\\
\glt  ‘You like fish? Then you must like this picture too, right?’
\end{exe}


However, unlike the nominative/accusative alternation in South \ili{Saami}, the choice of illative or accusative does not seem to be motivated by either semantic or syntactic factors. Instead, the most plausible explanation for the variation seems to align with the received view on similar variation in North \ili{Saami}:

\ea\langinfo{North Saami}{Uralic}{personal knowledge}
\label{16-ki-ex:11}
\ea\label{16-ki-ex:11a}
\gll Liikot guollái? De han de liikot dán govvii maid?\\
 like.\textsc{2sg} fish.\textsc{ill} then \textsc{dpt} then like.\textsc{2sg} this.\textsc{genacc} picture.\textsc{ill} also\\

\ex\label{16-ki-ex:11b}
\gll Liikot guoli? De han de liikot dán gova maid?\\
 like.\textsc{2sg} fish.\textsc{genacc} then \textsc{dpt} then like.\textsc{2sg} this.\textsc{genacc} picture.\textsc{genacc} also\\

\ex\label{16-ki-ex:11c}
\gll Liikot guolis? De han de liikot dán govas maid?\\
 like.\textsc{2sg} fish.\textsc{loc} then \textsc{dpt} then like.\textsc{2sg} this.\textsc{genacc} picture.\textsc{loc} also \\
\glt  ‘You like fish? Then you must like this picture too, right?’
\z
\z

For the North \ili{Saami} \textit{liikot} ‘like’, as many as three different cases are available.\footnote{In accordance with the general patterns of NP morphosyntax (see, e.g., \citealt{Sammallahti1998Saami}: 100–101), the determiner \textit{dán} \REF{16-ki-ex:11a}--\REF{16-ki-ex:11c} remains in the genitive-accusative even when headed by a noun in the illative or locative.} North \ili{Saami} is the \ili{Saami} language with not only the most speakers, but also the most grammatical research and language planning. As a consequence, the variation seen in \REF{16-ki-ex:11a}--\REF{16-ki-ex:11c} has attracted the attention of both descriptive and prescriptive grammarians. To put it briefly, the use of the illative \REF{16-ki-ex:11a} is unanimously regarded as the most original North \ili{Saami}, whereas the use of the genitive-accusative and locative are considered interference from Scandinavian \REF{16-ki-ex:11b} and \ili{Finnish} \REF{16-ki-ex:11c}, respectively:

\begin{exe}
\ex%12
\label{16-ki-ex:12}
\langinfo{Norwegian}{Germanic}{personal knowledge}\\
\gll Liker du fisk?\\
like.\textsc{prs} \textsc{2sg} fish\\
\glt
\end{exe}
 

\begin{exe}
\ex%13
\label{16-ki-ex:13}
 \langinfo{Finnish}{Uralic}{personal knowledge}\\
\gll Pidätkö kalasta?\\
 like.\textsc{2sg}.\textsc{q} fish.\textsc{ela}\\
\glt  ‘Do you like fish?’
\end{exe}

The data in \REF{16-ki-ex:12} and \REF{16-ki-ex:13} corresponds to the variation in the \ili{Saami} languages rather directly. However, although continuously rejected by language purists (e.g., \citealt[127]{Magga1987Samas}; \citealt[87]{Callinravagirji2003}; \citealt[425]{Vuolab-Lohi2007Mailmmi}), both the genitive-accusative and the locative have accompanied the verb \textit{liikot} for decades if not centuries. It seems that the authenticity of the use of the illative has been taken granted due to the fact that the illative was the most common alternative, and nearly the only alternative in earlier periods. The most detailed discussion on this issue is presented by \citet[139]{Helander2001Ii} whose earliest examples of the “wrong” cases stem from the beginning of the 20th century, and some instances of the genitive-accusative can actually be found already in the folklore recorded and authentic texts composed in the 19th century (see, e.g., \citealt[134, 190]{Qvigstad1927Lappiske}; \citealt{Ylikoski2016}). From the non-prescriptivist point of view adopted by Helander, it is easy to agree that all of the sentences \REF{16-ki-ex:11a}--\REF{16-ki-ex:11c} are grammatical North \ili{Saami}. The difference is that only \REF{16-ki-ex:11a} seems to be shared by the entire speech community, whereas \REF{16-ki-ex:11b} is mainly used by \ili{Saami}-Scandinavian bilinguals and \REF{16-ki-ex:11c} by \ili{Saami}-\ili{Finnish} bilinguals.\footnote{It is noteworthy that the variation exemplified in \REF{16-ki-ex:11a}--\REF{16-ki-ex:11c} has never been regarded as anything but full synonymy (\citealt[127]{Magga1987Samas}; \citealt{Helander2001Ii}: 139, 141; \citealt{Callinravagirji2003}: 87; \citealt{Sammallahti2005Laidehus}: 205; \citealt[425]{Vuolab-Lohi2007Mailmmi}). As seen in example triplets such as \REF{16-ki-ex:i} and \REF{16-ki-ex:ii}, the illative, genitive-accusative and locative are used with both \isi{animate} and \isi{inanimate}, and both \isi{definite} and \isi{indefinite} referents, for example.

\ea\langinfo{North Saami}{Uralic}{\citealt[87]{Callinravagirji2003}}

\label{16-ki-ex:i}
\gll Mun liikon dutnje {\textasciitilde} du {\textasciitilde} dus.\\
 \textsc{1sg} like.\textsc{1sg} \textsc{2sg}.\textsc{ill} {\textasciitilde} \textsc{2sg}.\textsc{genacc} {\textasciitilde} \textsc{2sg}.\textsc{loc}\\
\glt  ‘I like you.’ 
\z

\ea  
\label{16-ki-ex:ii}
\langinfo{North Saami}{Uralic}{\citealt[205]{Sammallahti2005Laidehus}}\\
\gll Mun in liiko guollái {\textasciitilde} guoli {\textasciitilde} guolis.\\
 \textsc{1sg} \textsc{neg}.\textsc{1sg} like.\textsc{cng} fish.\textsc{ill} {\textasciitilde} fish.\textsc{genacc} {\textasciitilde} fish.\textsc{loc}\\
\glt ‘I don’t like fish.’ 
\z
}

 The above-mentioned verbs \textit{lyjhkedh} (South \ili{Saami}), \textit{lijkkut} (Lule \ili{Saami}) and \textit{liikot} (North \ili{Saami}) have not been compared with each other earlier, but when this is done, the comparison can be extended up to \ili{Aanaar} \ili{Saami} where the etymological and semantic equivalent of these verbs is \textit{lijkkuđ}:

\begin{exe}
\ex%14
\label{16-ki-ex:14}
\langinfo{Aanaar Saami}{Uralic}{SIKOR} \\
\gll Amahân te mij puoh vissâsávt lijkkup kuálán já rähistep kyele, ko tom jyehi peeivi šiev puurrâmlustoin puurrâp (...) \\
I.guess \textsc{dpt} \textsc{1pl} all surely like.\textsc{1pl} fish.\textsc{ill} and love.\textsc{1pl} fish.\textsc{acc}  as it.\textsc{acc} every day.\textsc{gen} good appetite.\textsc{com} eat.\textsc{1pl}\\
\glt ‘I guess we all really like fish and love fish, as we eat it every day with great pleasure (...)’

\end{exe}

\protectedex{
\begin{exe}
\ex%15
\label{16-ki-ex:15}
\langinfo{Aanaar Saami}{Uralic}{SIKOR}\\ 
\gll Kreikkaliih iä lijkkum ennuv=gin syemmilijn.\\
 \ili{Greek}.\textsc{pl} \textsc{neg}.\textsc{3pl} like.\textsc{pst}.\textsc{ptcp} much=\textsc{dpt} Finn.\textsc{pl}.\textsc{loc}\\
\glt ‘The Greeks did not like Finns that much.’  
\end{exe}
}

To begin with, \REF{16-ki-ex:14} contains two accusative objects: one for the \isi{experiencer} verb \textit{rähistiđ} ‘love’ and one for a more concrete transitive verb \textit{puurrâđ} ‘eat’, and for their part \ili{Aanaar} \ili{Saami} does not differ from the languages discussed thus far. However, \textit{lijkkuđ} apparently never takes accusative objects, but it does not remain without variation either: the verb governs the illative \textit{kuálán} ‘fish’ in \REF{16-ki-ex:14}, but the locative \textit{syemmilijn} ‘Finns’ in \REF{16-ki-ex:15}. Again, the two variants are in free variation, as it would be equally possible to replace the illative \textit{kuálán} with the locative \textit{kyeleest}, or, vice versa, the locative \textit{syemmilijn} with the illative \textit{syemmiláid}. Furthermore, quite like with North \ili{Saami} \REF{16-ki-ex:11a}--\REF{16-ki-ex:11c}, the \ili{Aanaar} \ili{Saami} language planners have until recently regarded the use of locative as unwelcome \ili{Finnish} interference, but according to a recent decision of an \ili{Aanaar} \ili{Saami} language planning organ, both alternatives are now acceptable \citep[86–87]{Olthuis2009Mii}.

 The easternmost \ili{Saami} languages such as Skolt \ili{Saami} and \ili{Kildin Saami} do not share the Scandinavian loan verb discussed above, nor do we have large corpora for these languages. However, the existing dictionaries and texts support the information provided by our colleagues with intimate knowledge of these languages. In Skolt \ili{Saami}, the verb \textit{tu´ǩǩeed} ‘like’ behaves like North \ili{Saami} \textit{liikot} and \ili{Aanaar} \ili{Saami} \textit{lijkkuđ} in governing the \isi{locative case} as seen in \REF{16-ki-ex:16} and \REF{16-ki-ex:17a}; neither the accusative, illative nor other cases actually occur in the present-day language, although data from traditional dialects also include examples of accusative objects, as in \REF{16-ki-ex:17b}, which is deemed ungrammatical in today’s language:

\begin{exe}
\ex%16
\label{16-ki-ex:16}
\langinfo{Skolt Saami}{Uralic}{\citealt[97]{Kopononenetal2010Saamki}}\\
\gll Mon jiõm tõʹst tuʹǩǩääm ni vooʹps, dõõʹst.\\
\textsc{1sg} \textsc{neg}.\textsc{1sg} it.\textsc{loc} like.\textsc{pst}.\textsc{ptcp} not at.all it.\textsc{loc}\\
\glt ‘I didn’t like that [work] at all.’
\end{exe}

\begin{exe}
\ex%17
\label{16-ki-ex:17}
\langinfo{Skolt Saami}{Uralic}{personal knowledge; confirmed by Tiina Sanila-Aikio \REF{16-ki-ex:17b} from \citealt[612]{Itkonen1958Koltan}; not accepted by present-day speakers}\footnote{This claim is based on the data from and judgments by speakers of Skolt \ili{Saami} in Finland, but the language also has some elderly speakers in Russia.}
\begin{xlist}

\ex\label{16-ki-ex:17a}
\gll Tõst jie tuʹǩǩed.\\
 it.\textsc{loc} \textsc{neg}.\textsc{3pl} like.\textsc{cng}\\ 

\ex\label{16-ki-ex:17b}
\gll (*)Tõʹn jie tuʹǩǩed.\\
 it.\textsc{acc} \textsc{neg}.\textsc{3pl} like.\textsc{cng}\\
\glt  ‘They don’t like it.’ 
\end{xlist}
\end{exe}
 

Our last example comes from \ili{Kildin Saami}, a language that in a way lacks a verb for ‘like’. Instead, sentences denoting liking are centered around the verb \textit{miillte} ‘please’, and the word referring to the stimulus of liking (\textit{tedt} \textit{lańń} ‘this country’ in example \REF{16-ki-ex:18}) functions as the grammatical subject of pleasing, whereas the \isi{experiencer} is marked with the illative. Alternatively, it would be possible to resort to the transitive verb \textit{šoabše} ‘love’, which takes the accusative just like the corresponding verbs in apparently all \ili{Saami} languages (compare example \REF{16-ki-ex:14} from \ili{Aanaar} \ili{Saami}).

\begin{exe}
\ex%18
\label{16-ki-ex:18}
\langinfo{Kildin Saami}{Uralic}{\citealt[240]{Lindgren2013Pippi}}\\
\glll {Я,} {мунн} {надҍеда} {тэдт} {ланнҍ} {меллт} {тонн\"э}.\\
\textit{ja} \textit{munn} \textit{naadʹeda} \textit{tedt} \textit{lańń} \textit{meellt} \textit{tońńe.}\\
and \textsc{1sg} believe.\textsc{1sg} this country please.\textsc{3sg} \textsc{2sg}.\textsc{ill}\\
\glt ‘and I believe that you will like this country.’
\end{exe}


What is most interesting in \ili{Kildin Saami} is that the \isi{argument structure} of \textit{miillte} ‘please’ \REF{16-ki-ex:18} is fully the opposite of the most common pattern of the Lule, North and \ili{Aanaar} \ili{Saami} verbs \textit{lijkkut} \REF{16-ki-ex:10}, \textit{liikot} \REF{16-ki-ex:11a} and \textit{lijkkuđ} \REF{16-ki-ex:14} with which the \isi{illative case} is used to code the stimulus, not the \isi{experiencer} of pleasure (liking). On the other hand, as the illative is also the case of recipients and thus in a way the “dative” case of all \ili{Saami} languages (see, \eg \REF{16-ki-ex:5}), the \ili{Kildin Saami} sentence \REF{16-ki-ex:18} is conceptually and structurally an instance of a well-known type of dative \isi{experiencer} sentences.

To summarize, the variation in the coding of liking verbs in the six \ili{Saami} languages described above can be condensed in \tabref{16-ki-tab:2}.\footnote{Personal knowledge; Skolt \ili{Saami} and \ili{Kildin Saami} examples provided and confirmed by Tiina Sanila-Aikio and Elisabeth Scheller, respectively. For the purpose of visualization, the South \ili{Saami} example is presented in a slightly marked \isi{word order} (SVO) instead of the most unmarked SOV order typical of the language (\cf \REF{16-ki-ex:1}, \REF{16-ki-ex:4}, \REF{16-ki-ex:5}, \REF{16-ki-ex:7}). In cases of variation, the boldface indicates the variants officially acknowledged by language authorities. \ili{Kildin Saami} \textit{šoabašt} in \tabref{16-ki-tab:2} means primarily ‘loves’; for the use of the verb \textit{miillte} ‘please’, see \REF{16-ki-ex:18} above and \REF{16-ki-ex:iii} below:

\ea
\label{16-ki-ex:iii}
\langinfo{Kildin Saami}{Uralic}{personal knowledge; confirmed by Elisabeth Scheller}\\
\glll {Пеннгэ} {меллт} {к{\=у}лль.}\\
\textit{Peennge} \textit{meellt} \textit{kuullʹ.}\\
 dog.\textsc{ill} please.\textsc{3sg} fish\\
\glt ‘The dog likes fish.’ (Lit. ‘Fish pleases the dog.’)
\z
} 
For the purposes of the present discussion, the focus is on the types of DOM related to the verbs of liking in particular, and the more canonical instances of DOM as seen in the plural object marking of South \ili{Saami} in general (examples \REF{16-ki-ex:1a}--\REF{16-ki-ex:1b} and \REF{16-ki-ex:4}) are not repeated here.

\begin{table}
{  \begin{tabularx}{\textwidth}{Qllllll}
\lsptoprule
South \ili{Saami}\newline (Norway, Sweden) &  \textit{Bïenje} &  \textit{lyjhkoe} &  \textit{gueliem.} &  & \\
Lule \ili{Saami}\newline (Norway, Sweden) & \textit{Bena} &  \textit{lijkku} &  \textit{guolev} &$\sim$ \textit{guolláj.} & \\
North \ili{Saami} \newline (Norway, Sweden, Finland) &  \textit{Beana} &  \textit{liiko} & \textit{guoli}  &$\sim$ \textit{guollái} &$\sim$ \textit{guolis.}\\
{Aanaar} \ili{Saami} \newline(Finland) &   \textit{Peenâ} &  \textit{lijkkoo} &  &  \textit{kuálán} &$\sim$  \textit{kyeleest.}\\
Skolt \ili{Saami} \newline(Finland, Russia) &  \textit{Piânnai} &  \textit{tuʹǩǩad} &  \textit{(*)kueʹl} &  &  $\sim$ \textit{kueʹlest.}\\
{Kildin Saami}   &   \textit{Пе̄ннэ} &   \textit{шоабашт} &  \textit{кӯль.} & &\\
 (Russia) &   \textit{Peenne} &   \textit{šoabašt} &  \textit{kuulʹ.} & &\\
  &   dog(.\textsc{nom}) & like.3\textsc{sg} & fish.\textsc{acc} & fish.\textsc{ill}& fish.\textsc{loc}\\
  & \multicolumn{4}{l}{ ‘The dog likes (the) fish.’}\\			
\lspbottomrule
\end{tabularx}
}\caption{Argument marking of ‘liking’ in six Saami languages.}\label{16-ki-tab:2} 
\end{table}

\tabref{16-ki-tab:2} also lists the states in which the examined languages – presented in geographical order from southwest to northeast of the Saami territory – are spoken as minority languages. In this connection, a number of facts are worth noting: As for the variation seen in Lule, North and \ili{Aanaar} \ili{Saami}, the use of the \isi{illative case} is considered the most original. Even though public prescriptivist statements about the unwanted influence of majority languages have been presented for \ili{Aanaar} and North \ili{Saami} verbs only (e.g., \citealt[33]{Morottaja2008Anaraskiela}; \citealt[425]{Vuolab-Lohi2007Mailmmi}), it is also quite likely that the use of the accusative in Lule \ili{Saami} and that of the locative in Skolt \ili{Saami} are influenced by their respective majority languages. When speaking of verbs of liking, two types of foreign influences are available. As seen in \REF{16-ki-ex:12}, the Norwegian verb \textit{like} follows a nominative–accusative pattern, but so does its closest \ili{Swedish} equivalent \textit{gilla} ‘like’, as well as the verb \textit{ljubitʹ} ‘love, like’ in \ili{Russian}, which has long had a considerable influence on \ili{Kildin Saami}. On the other hand, the use of the \ili{Finnish} elative – the cognate of the \ili{Saami} elative/locative – in \REF{16-ki-ex:13} easily explains the established use of the locative for the liking verbs of all three \ili{Saami} languages spoken in Finland. To make the role of language contact even more explicit, it can be pointed out that the use of the \ili{Kildin Saami} \textit{miillte} ‘please’ in \REF{16-ki-ex:18} is analogous to that of the \ili{Russian} \textit{nravitʹsja} ‘please’ \REF{16-ki-ex:19}. However, this verb type falls outside the main scope of the present paper.

\begin{exe}
\ex%19
\label{16-ki-ex:19}
\langinfo{Russian}{Slavic}{personal knowledge}\\
\glll {Я} {надеюсь,} {что} {тебе} {нравится} {эта} {страна.}\\
\textit{Ja} \textit{nadejusʹ,} \textit{čto} \textit{tebe} \textit{nravitʹsja} \textit{eta} \textit{strana.}\\
\textsc{1sg} hope.\textsc{1sg} \textsc{comp} \textsc{2sg}.\textsc{dat} please.\textsc{3sg} this.\textsc{f} country\\
\glt ‘I hope that you will like this country.’
\end{exe}

The influence of language contact will be discussed in more detail and with additional examples in \sectref{16-ki-sec:4-2} below. However, it must be noted that the \ili{Saami} languages also exhibit DOM that cannot be easily explained away by referring only to interference from majority languages. As pointed out by \citet[140–141]{Helander2001Ii}, the North \ili{Saami} \textit{suhttat} ‘get angry’ may take not only the illative and locative cases, but also a postpositional phrase headed by \textit{ala} ‘on(to)’, and only the latter alternative can be explained by the influence of the Scandinavian preposition \textit{på} ‘on(to)’. Some verbs such as the North \ili{Saami} \textit{illudit} ‘rejoice; celebrate’ take not only the illative, locative and genitive-accusative, but also the comitative case. Furthermore, the more than two thousand occurrences of \textit{illudit} ‘rejoice; celebrate’ in the available North \ili{Saami} corpus (SIKOR) also include many sentences in which the stimulus of rejoicing is not marked by any of these four cases, but by the postpositions \textit{alde} ‘on’, \textit{badjel} ‘over’, \textit{badjelii} ‘onto’ and \textit{dihte} ‘because of’. What is more, occurrences of the verb \textit{heahpanit} ‘be ashamed of’ are accompanied, in addition to the four above-mentioned cases, by yet another set of postpositions (\textit{alde} ‘on’, \textit{badjel} ‘over’, \textit{dihte} ‘because of’, \textit{beales} ‘for, on behalf’, \textit{geažil} ‘for, on account of’ and \textit{ovddas} ‘for, in front of’) (see also \citealt{Ylikoski2016}).

To our knowledge, however, language contacts are not the whole story: there are other factors at play here as well. It might also be possible to analyze the rich variation in some verbs such as the North \ili{Saami} \textit{illudit} ‘rejoice; celebrate’ and \textit{heahpanit} ‘be ashamed of’ as combinations of \isi{intransitive} predicates and optional obliques denoting the cause or stimulus of the experience. However, multiple patterns of coding the stimulus are generally verb-specific and therefore seem to belong primarily to the realm of argument marking. Needless to say, details and possible preconditions of such phenomena in the syntactic patterns of individual verbs in North \ili{Saami} and other \ili{Saami} languages call for further research. The present discussion of a small sample of \ili{Saami} \isi{experiencer} verbs is the first attempt to outline some possibilities and perspectives on such endeavors.

\section{Discussion}
\label{16-sec:4}

\subsection{Preliminaries}
\label{16-ki-sec:4-1}

In the previous section, we have presented some of the variation in the coding of objects with \isi{experiencer} verbs in the \ili{Saami} languages. The variation is best seen as manifestations of DOM, because the marking is not semantically determined in the sense that the semantic roles borne by the affected arguments are maintained (the affected argument retains its role as a stimulus) and the alternation in the marking is not directly determined, but only made possible by the verb (i.e., we are not dealing with variation determined by the inherent semantics of verbs, as we are in the case of \isi{experiencer} vs. prototypical transitive verbs). The instances discussed here represent restricted predicate-triggered DOM, because the described variation is attested mainly for \isi{experiencer} verbs. Moreover, the discussed instances of DOM can be claimed to be connected only loosely with \isi{definiteness}, as there are only a few signs that suggest that the variation may be affected by habitual vs. concrete reading of the constructions in question. The rationale behind the variation differs from that of typical canonical DOM in that the typical triggers of DOM, \isi{animacy} or \isi{definiteness}, seem to play no role in the cases discussed in this paper (the possible contribution of \isi{definiteness} is best seen as a by-product). Finally, the variation is not between two structural cases, but rather concerns semantic cases (and in some instances also postpositions, as mentioned above). In this section, we will discuss the most important contribution of the \ili{Saami} languages to our understanding of DOM in more detail. Basically, three partly competing factors can be seen that add to our understanding of DOM: language contact, the semantic emptiness of the cases (or other case-like categories) involved in the variation, and the pursuit of coherence.

\subsection{Language contact}
\label{16-ki-sec:4-2}

As noted above, the \ili{Saami} languages are all minority languages spoken in the northern parts of Finland, Sweden and Norway, as well as the northwesternmost part of Russia. This has the very natural consequence that language contact has influenced and continues to influence the structure of \ili{Saami} languages in many ways, and argument marking is no exception in this regard. The major results of this contact were illustrated in \tabref{16-ki-tab:2} above. \tabref{16-ki-tab:2} and the following discussion clearly show how the majority languages have affected the coding of liking verbs in \ili{Saami}, given that the most original pattern in Lule, North and \ili{Aanaar} \ili{Saami} has been the one in which the stimulus of liking is coded with the \isi{illative case}, whereas the accusative and locative marking are both new and analogous to the patterns of the majority languages at the same time. It is also important to note that we are not dealing with a transfer of DOM in a language contact situation, but rather contact with different languages has produced DOM for a group of predicates in the minority languages.

To give another example of DOM among the \isi{experiencer} verbs in \ili{Saami} languages, \tabref{16-ki-tab:3} presents a likewise condensed collection of the major patterns of expressing ‘caring’ and its participants in five \ili{Saami} languages. The South \ili{Saami} verb \textit{pryjjedh~}is a relatively recent loan from Norwegian and \ili{Swedish} (\textit{bry seg/sig}), whereas the Lule \ili{Saami} \textit{berustit}, North \ili{Saami} \textit{beroštit}, \ili{Aanaar} \ili{Saami} \textit{perustiđ} and Skolt \ili{Saami} \textit{peersted} all go back to \ili{Finnish} (\textit{perustaa}).


\begin{table}
{\small  \begin{tabularx}{\textwidth}{Qll@{~}l@{~}l@{~}l@{~}l@{~}l}
\lsptoprule
South \ili{Saami} &  \textit{Bïenje} &  \textit{ij} &  \textit{pryjjh} &  \textit{gueleste} & $\sim$  \textit{gueliem} & ($\sim$  \textit{guelien} & \textit{bïjre}).\\
Lule \ili{Saami} & \textit{Bena} & \textit{ij} & \textit{berusta} & \textit{guoles} & $\sim$ \textit{guolev} & ($\sim$ \textit{guole} & \textit{birra}).\\
North \ili{Saami} & \textit{Beana} & \textit{ii} & \textit{beroš} & \textit{guolis} & $\sim$ \textit{guol}i & ($\sim$ \textit{guoli} & \textit{birra}).\\
 {Aanaar} \ili{Saami} & \textit{Peenâ} & \textit{ii} & \textit{peerust} & \textit{kyeleest}.\\
Skolt \ili{Saami} & \textit{Piânnai} & \textit{ij} & \textit{peerst} & \textit{kueʹlest}.\\
 & dog & \textsc{neg.3sg} & care.\textsc{cng} & fish.\textsc{ela/loc} & fish.\textsc{(gen)acc} & fish.\textsc{gen(acc)} & about\\
 & \multicolumn{4}{l}{‘The dog doesn’t care about fish.’}\\

\lspbottomrule
\end{tabularx}}
\caption{Argument marking of ‘caring’ in five Saami languages}
\label{16-ki-tab:3} 
\end{table}
 
 

In the Scandinavian languages, the stimulus of ‘caring’ is coded with the preposition \textit{om} ‘about’, whereas the \ili{Finnish} verb governs the elative. It is understandable that \ili{Saami} languages most commonly use the elative/\isi{locative case} for caring verbs, too, because this is probably inherited from the \ili{Finnish} loan original. On the other hand, it is also understandable that the westernmost \ili{Saami} languages (under Scandinavian influence) occasionally resort to the postposition \textit{bïjre}/\textit{birra} ‘about’, which largely corresponds to the most abstract functions of the Scandinavian \textit{om}. However, at the same time, the same languages – South, Lule and North \ili{Saami} – also witness accusative coding that seems likewise absent in \ili{Aanaar} and Skolt \ili{Saami}.

 It is probably no coincidence that \isi{experiencer} verbs are the foremost playground of DOM in \ili{Saami} languages. As noted above, the coding of \isi{experiencer} verbs often deviates from the basic transitive pattern of a given language in addition to which there is variation in their coding within languages (see examples \REF{16-ki-ex:11a}--\REF{16-ki-ex:11c} from \ili{Finnish}). What makes the coding of \isi{experiencer} verbs in \ili{Saami} languages interesting is the fact that contact with structurally different source languages (governing different cases and adpositions) has created a situation where the coding patterns of the source languages mirror the cross-linguistic variation attested within verbs in other languages (\eg in \ili{German} ‘be cold’ governs a dative subject, while ‘see’ appears in a transitive construction). In contrast to typical cross-linguistic variation in \isi{experiencer} verbs, yet due to contact with structurally different source languages, similar variation is reflected within one language and even more so in the group of closely related \ili{Saami} languages. What is also noteworthy here is that the variation seems most evident and productive for \isi{experiencer} verbs; other verbs allow it only to a limited degree, if at all. For example, the coding of basic transitive clauses is consistent in the contact languages, because all of them are nominative-accusative languages, even though Norwegian and \ili{Swedish} do not code A and O\footnote{A and O are here understood in the spirit of \citet{Comrie1978Ergativity} and \citet{Dixon1979Ergativity}.} using cases like \ili{Finnish} and \ili{Russian} do. Consequently, there is no contact-induced variation in the coding of A and O in prototypical transitive clauses, and language contact aids in explaining why obliquely coded arguments have been affected. However, borrowing does not follow automatically nor can it be considered random, since there are many areas of grammar that have remained largely unaffected in the described language contact situations (see, for example, \citealt{Riessler2007Grammatical} for \ili{Kildin Saami} and \ili{Russian}). An illustrative example is represented by the \ili{Finnish} variation between nominative, accusative and partitive in subject and object coding, which has – in spite of occasional translators’ and semi-speakers’ errors (\citealt[131]{Magga1987Samas}; \citealt[78–79]{Lansman2009Oahppiid}) – not gained a significant foothold in any of the three \ili{Saami} languages spoken in Finland. Moreover, the lack of morphological cases for coding core arguments (characteristic of Scandinavian languages) is not found in any \ili{Saami} language.

 As shown above, contact with the surrounding majority languages provides a rather good explanation for the variation attested in \isi{experiencer} verbs in the \ili{Saami} languages, but it is important to distinguish the results of recent language contacts and interference from changes that are due to language contact that has become an established part of the grammar of the modern languages. Although the data presented above may give the impression that the DOM examined here is a recent phenomenon, it has existed in at least North and \ili{Aanaar} \ili{Saami} for more than a century (see \sectref{16-sec:3}), and thus the variation cannot be seen as random, but rather an entrenched feature of the languages. Somewhat paradoxically, this also underlines the fact that even a seemingly superfluous DOM can be a somewhat stable phenomenon that in itself can be resistant to language change. In other words, this observation is interesting in light of the fact that DOM can be viewed as disturbing the consistency in object coding, but \ili{Saami} data shows that it can nevertheless be retained through generations.

\subsection{Emptiness of semantic cases}
\label{16-ki-sec:4-3}

As the data discussed in \sectref{16-sec:3} shows, the variation in the O coding (referring to the stimulus) concerns a variety of semantic cases (in addition to the accusative also employed for this function). Semantic cases, as the label implies, differ from syntactic or structural cases (such as the nominative and the accusative) in that they are more directly related to a certain semantic function. Across languages, a variety of semantic cases, such as the dative and different local cases, are used for marking the arguments of \isi{experiencer} constructions. From the nature of semantic cases it follows that variation between them usually has semantic consequences as well; for example, replacing the allative (‘to’) with the ablative (‘from’) typically results in a change in the direction of the denoted instance of motion (see also \citealt{Vasti2012Verbittomat} for a somewhat different discussion of the allative and ablative in \ili{Finnish}). However, as the discussion in this paper has shown, \ili{Saami} languages provide us with numerous examples of rather free variation between semantic cases. For example, in the North \ili{Saami} and \ili{Aanaar} \ili{Saami} examples in \REF{16-ki-ex:11c} and \REF{16-ki-ex:15}, replacing the illative (core meaning ‘to’) with the locative (‘at; from’) does not yield any major semantic differences in the reading of the clauses. This means that semantic cases are deprived of their semantic content when they appear with \isi{experiencer} verbs. These differences reflect the cross-linguistic and also cross-verbal variation in the coding of \isi{experiencer} verbs rather well, but the variation is manifested within one language and one verb.

 One of the main reasons for the loss of semantic content is that with \isi{experiencer} verbs semantic cases are used for coding arguments that are parts of the verb’s valency. In these cases, the arguments are accorded a \isi{semantic role} directly by the verb, which has the consequence that the exact mechanism used for argument coding becomes less relevant, which renders the attested variation understandable. In the \ili{Saami} languages, this has led to the loss of semantic contrast between certain semantic cases when they are used for coding objects (and stimuli) of \isi{experiencer} verbs. The semantic differences are, however, relevant in other contexts, especially when the given cases are used for coding adverbials. From a synchronic point of view, a given language may select the case its contact language employs for coding \isi{experiencer} verbs without this having any consequences for the reading of the construction. On the other hand, the choice of acceptable cases is determined – or at least allowed – by the \isi{argument structure} of an individual verb, after all. For example, even though the North \ili{Saami} verb \textit{illudit} ‘rejoice; celebrate’ (mentioned at the end of \sectref{16-sec:3}) can also code the stimulus using the comitative, such an alternative seems entirely impossible with \textit{liikot} ‘like’.

The discussion above also underlines the fact that DOM seems to emerge only if the attested changes do not have any major consequences for the \isi{semantic role} assignment of the affected argument. In typical cases, the variation is between two structural cases that are inherently void of semantic content, but the data from the \ili{Saami} languages shows that similar variation is possible also with semantic cases. As the semantic differences between the cases have been neutralized, however, the variation has no semantic consequences. The important feature of \isi{experiencer} verbs seems to be their differences from the basic transitive construction, \ie the events (or rather states) denoted by \isi{experiencer} verbs rank lower for \isi{transitivity}, which makes it possible for other cases than the (default) accusative cases to be used for their coding. In other words, the exact mechanism or case form used for argument coding appears to be less relevant due to decreased \isi{transitivity}, which gives rise to DOM for \isi{experiencer} verbs in \ili{Saami} languages. Moreover, typical features of \isi{transitivity}, such as agency and \isi{affectedness}, are rather irrelevant to \isi{experiencer} verbs in that the stimuli are usually not affected at all and even though agency does play a role in cases such as ‘see’ vs. ‘look’, experiencing is always less agentive and affective than typical transitive actions. This has the consequence that changes in these features cannot account for the attested differences in \isi{case marking}. These features also make \isi{experiencer} verbs easy targets for semantically rather void DOM.

\newpage 
 Above, the reasons for the rather free variation between semantic cases in the function of coding the O were discussed. However, this might not be the whole story, as there are also cases where the variation is not completely semantically free, but it may have resulted in a slight change in meaning. When asked about a possible semantic difference between accusative and illative in cases such as \REF{16-ki-ex:11a} and \REF{16-ki-ex:11b}, speakers of North \ili{Saami} may suggest that the accusative is used for concrete liking of fish (i.e., when eating), while the illative is supposed to refer to a habit of liking fish in general.\footnote{Even though native speaker students of North \ili{Saami} are often aware of the prescriptive grammarians’ view of the “impurity” of accusative objects with \textit{liikot} `like’, \ili{Saami}-Norwegian bilinguals at the Sámi University of Applied Sciences (Guovdageaidnu) and UiT The Arctic University of Norway (Tromsø) have often, when asked, suggested this kind of semantic nuance between the use of accusative and illative.
} In other words, the difference between accusative and illative may at least to some extent have a semantic basis, and it is also related to semantic \isi{transitivity} (habituals rank lower for \isi{transitivity}, see, e.g., \citealt{Gerstner-Link1998transitive}), and, as was noted above, \isi{definiteness} might also play a potential role here, although it is best viewed only as a by-product of the attested variation whose ultimate origins seem to lie in language contacts. It must, however, be noted that authentic text materials do not obviously support the elicited judgments on possible semantic differences, and more research is thus needed on this issue. In this context, it is also relevant to note that some verbs that describe more intense feelings, such as ‘love’ and ‘hate’ (e.g., South \ili{Saami} \textit{iehtsedh}, Lule \ili{Saami} \textit{iehttset}, North \ili{Saami} \textit{ráhkistit}, \ili{Aanaar} \ili{Saami} \textit{rähistiđ} (\cf \REF{16-ki-ex:14}), Skolt \ili{Saami} \textit{rä´ǩsted} and \ili{Kildin Saami} \textit{šoabpše}, all meaning ‘love’), only govern the accusative (and nominative in South \ili{Saami}), which may lend further support to the higher \isi{transitivity} associated with the accusative in \REF{16-ki-ex:11b}.

\subsection{Coherence in marking}
\label{16-ki-sec:4-4}

The variation between accusative and semantic cases can also be approached from another perspective. As suggested above, the variation may be related to a slight semantic change in certain cases, but examples like \REF{16-ki-ex:20} below suggest another reason for this:

\begin{exe}
\ex%20
\label{16-ki-ex:20}
\langinfo{North Saami}{Uralic}{SIKOR}\\
\gll Nuorran diggejin Beatles joavkku.\\
young.\textsc{ess} dig.\textsc{pst}.\textsc{1sg} Beatles group.\textsc{genacc}\\
\glt ‘When I was young, I dug the Beatles.’ 
\end{exe}


The North \ili{Saami} verb \textit{digget} ‘dig’ is a new internationalism whose O argument bears accusative coding. In \REF{16-ki-ex:20}, the (genitive-)accusative coding does not necessarily reflect a higher degree of \isi{transitivity} of “digging” (in comparison to liking, for example). This can be explained in two ways. First, the occurrence of the accusative can be explained by the fact that new loan verbs govern the most common case for O coding, namely the accusative, which is used in typical transitive clauses, and, as has been shown, also appears with certain \isi{experiencer} verbs. Second, this may be interference from Norwegian, or even English, the ultimate source language for the loan. This, as opposed to the cases discussed above, can be taken as a tendency towards coherence in marking; functionally superfluous variation is usually avoided in favor of a more coherent marking system. This argument is in line with, for example, Barðdal’s (see, e.g., \citealt{Barddal2008Productivity,Barddal2009Germanic}) findings on \ili{Icelandic} and other Germanic languages: in many Germanic languages, the less frequent argument marking patterns have disappeared, since the default nominative–accusative has replaced them.

 On the other hand, while the accusative coding of \textit{digget} \REF{16-ki-ex:20} can nevertheless be also regarded as inheritance of the transitive originals such as Norwegian \textit{digge} and ultimately English \textit{dig}, the accusative objects of the South, Lule and North \ili{Saami} verbs for ‘care’ seen in \tabref{16-ki-tab:3} seem to be best explained by a language-internal pursuit of coherence in marking – even when neither the etymological background of the verbs nor the predominant patterns of the majority languages seem to promote the use of the accusative. It is notable that the accusative coding of caring verbs coincides with the westernmost \ili{Saami} languages, in which the accusative coding is at least one of the alternatives for the liking verbs as well (\tabref{16-ki-tab:2}). In other words, while the accusative objects of Lule \ili{Saami} \textit{lijkkut} \REF{16-ki-ex:10} and North \ili{Saami} \textit{liikot} \REF{16-ki-ex:11} can be explained as foreign influence, Norwegian \textit{like} in turn may be interpreted as the subsequent model for extending the accusative coding to caring verbs as well.

 It is also notable that for some verbs, the multiple outside pressures on minority languages may pull the \ili{Saami} languages in a new but single direction: A case in point are verbs for ‘fear’, which, as illustrated in \REF{16-ki-ex:7} for South \ili{Saami}, traditionally govern the elative/\isi{locative case} in the \ili{Saami} languages. However, it appears that not only the transitive pattern of Norwegian \textit{frykte} and \ili{Swedish} \textit{frykta} both meaning ‘fear’, but also the partitive coding of \ili{Finnish} \textit{pelätä} ‘fear’ have given the impetus for the emergence of accusative coding in the \ili{Saami} languages as well (\cf \citealt[425]{Vuolab-Lohi2007Mailmmi}; \citealt[86–87]{Olthuis2009Mii}):

\begin{exe}
\ex%21
\label{16-ki-ex:21}
\langinfo{North Saami}{Uralic}{personal knowledge}\\
\gll Sii ballet guliin {\textasciitilde} guliid!\\
 \textsc{3pl} fear.\textsc{3pl} fish.\textsc{pl}.\textsc{loc} {\textasciitilde} fish.\textsc{pl}.\textsc{genacc}\\
\glt ‘They are afraid of fish!’
\end{exe}
 


\begin{exe}
\ex%22
\label{16-ki-ex:22}
\langinfo{Norwegian}{Germanic}{personal knowledge}\\
\gll De frykter fisk!\\
\textsc{3pl} fear.\textsc{3pl} fish\\
\glt ‘They are afraid of fish!’
\end{exe}
 


\begin{exe}
\ex%23
\label{16-ki-ex:23}
\langinfo{Finnish}{Uralic}{personal knowledge}\\
\gll He pelkäävät kaloja!\\
 \textsc{3pl} fear.\textsc{3pl} fish.\textsc{pl}.\textsc{ptv}\\
\glt ‘They are afraid of fish!’
\end{exe}

Again, the accusative coding for verbs of fearing may be seen as strengthening the tendency towards coherence in marking. As noted above, the coding of the verbs for ‘fear’ differs from the contact languages in that in \ili{Finnish} the verb does not govern the accusative, but rather the partitive \REF{16-ki-ex:23}, which is common for many \isi{experiencer} verbs in \ili{Finnish}. However, this has resulted in the accusative coding in North \ili{Saami}, because the language lacks a partitive, and the \ili{Finnish} partitive can also be seen as a grammatically determined structural case.\footnote{The \ili{Saami} accusative is also historically directly connected to the \ili{Finnish} partitive, as the \ili{Saami} plural accusative ending is cognate to the Finnic plural partitive, and both North \ili{Saami} \textit{guliid} [fish.\textsc{pl}.\textsc{genacc}] \REF{16-ki-ex:21} and \ili{Finnish} \textit{kaloja} [fish.\textsc{pl}.\textsc{ptv}] \REF{16-ki-ex:23} thus go back to a common proto-form *\textit{kala-j-ta} [fish.\textsc{pl}.\textsc{ptv}] \citep[68, 203--206]{Sammallahti1998Saami}. This is possibly further reflected in the fact that Saami-Finnish bilinguals and Finnish learners of \ili{Saami} languages often tend to equate the \ili{Saami} genitive-accusative with the \ili{Finnish} partitive (\citealt[131]{Magga2002North}; \citealt[78–79]{Lansman2009Oahppiid}).}

\subsection{Theoretical implications}
\label{16-ki-sec:4-5}

In the preceding sections, we have briefly discussed the motivation for the occurrence of DOM in \ili{Saami} languages. We have suggested that the variation in O coding follows primarily from three different factors, namely language contact, emptiness of semantic cases and tendency towards coherence. In addition, \isi{transitivity} may play a role in cases such as \REF{16-ki-ex:11a}, where the accusative (instead of the illative) coding may underline the concreteness of the denoted event, which makes the event in question more dynamic and thus more transitive (see, e.g., \citealt[76]{Givon1995Functionalism}). In other words, the occurrence of DOM constitutes a rather canonical instance of competing motivations. On one hand, contact with different languages and the semantic emptiness of the cases used for coding \isi{experiencer} constructions produces variation in the marking, while on the other hand, the dominance of accusative coding especially with new loan verbs may create coherence in marking. Experiencer verbs lend themselves naturally to this kind of variation, because their lower degree of \isi{transitivity} favors the use of semantic cases for their coding. It is easy for a language to adopt the coding pattern of a surrounding majority language in this kind of case, and in many of the discussed instances, the coding pattern of the majority language is mirrored in the given \ili{Saami} language. The future will show which of the motivations will be stronger.

 Another question related to the data discussed in this paper concerns the emergence of DOM. Recently, \citet{Iemmolo2011Towards} has argued that the occurrence (and emergence) of DOM is best explained by \isi{topicality}. In other words, topical objects gradually start receiving explicit (non-zero) marking, which eventually results in a fully grammaticalized DOM system. What is interesting from a cross-linguistic perspective is that \isi{animacy} and \isi{definiteness}, typically seen as the hallmark features of DOM, are not in any direct way related to the cases discussed in this paper (see also \citealt{Iemmolo2011Towards} for a recent discussion based on \isi{topicality}); the possible effects of \isi{definiteness} are only indirect. This means that DOM cannot be exhaustively explained by \isi{animacy} and \isi{definiteness} (or \isi{topicality}), but the data from \ili{Saami} languages provides another kind of view to the development of DOM instead. First of all, the type of DOM examined here appears to be most common within a certain verb class only, namely \isi{experiencer} verbs. This means that semantics makes an important contribution to its occurrence. As noted numerous times in the paper, the coding of \isi{experiencer} verbs varies both within and across languages. This might be the main reason for the fact that they are so prone to external influences. In principle, the language has no reason to resist the emerging variation, because it is not connected to any major semantic differences. For example, the differences between accusative and illative are not related to any semantic differences in the case of \isi{experiencer} verbs, because, as noted also above, the \isi{affectedness} of stimuli is not a relevant feature with them. This is in line with more common manifestations of DOM, where the main consequences of DOM are pragmatic in nature, \ie they do not affect the semantic roles of arguments.

 The data from the \ili{Saami} languages does not provide us with a clear answer to the question of how and why DOM emerges in more general terms, but it aids us in understanding the circumstances under which it may arise. Favorable conditions are present if the variation is between two structural (such as nominative and accusative) or two semantic cases (such as illative and locative), and the variation is thus not related to any major semantic differences. The differential coding of topical objects also lacks an obvious semantic motivation (see \citealt{Iemmolo2011Towards}), but with time, the seemingly arbitrary variation in object coding acquires pragmatic functions. On the other hand, \isi{animacy} effects on the coding of goals, for example, are more dramatic in nature, because we are also dealing with differences in roles of the affected arguments (see \citealt{Kittila2008Animacy} for a more detailed discussion). It remains to be seen whether the kind of DOM attested in \ili{Saami} languages will become more functionally triggered in the future. In any case, it is clear that at this point, the DOM in the \ili{Saami} languages is predicate-triggered and only time will tell whether it will extend to objects in more general terms, and whether it will give rise to more evident semantic differences between the alternatives that are now best seen as free variation.

 Another thing that the data discussed in this paper may shed more light on is the semantic nature of cases used for coding arguments that belong to the valence of a given verb. The typical structural cases, most notably nominative, absolutive, accusative and ergative, are semantically rather void of any specific meaning and usually get their \isi{semantic role} from the verb. Their use is more directly related to distinguishing between A and O. The DOM discussed in this paper provides us with a somewhat different kind of evidence for the semantic emptiness of these cases, because cases that are prototypically best regarded as semantic behave as structural cases instead. In other words, in the data discussed in this paper, the employed case forms receive their meaning from the verb instead of having independent semantics of their own, even though we are dealing with semantic cases. The object slot is inherently related to a certain kind of \isi{semantic role}, and the formal requirements outrank the inherent semantics of the employed case forms.


\section*{Abbreviations}
\begin{tabularx}{.45\textwidth}{lQ}
1 & first  person\\
2 & second  person\\
3 & third  person\\
\textsc{acc} & accusative\\
\textsc{cng} & connegative\\
\textsc{comp} & complementizer\\
\textsc{dpt} & discourse  particle\\
\textsc{du} & dual\\
\textsc{ela} & elative\\
\textsc{ess} & essive\\
\textsc{f} & feminine\\
\textsc{gen} & genitive\\
\textsc{genacc} & genitive-accusative\\
\textsc{ill} & illative\\
\end{tabularx}
\begin{tabularx}{.45\textwidth}{lQ}
\textsc{imp} & imperative\\
\textsc{inf} & infinitive\\
\textsc{loc} & locative\\
\textsc{neg} & negation\\
\textsc{nom} & nominative\\
\textsc{pl} & plural\\
\textsc{prs} & present\\
\textsc{pst} & past\\
\textsc{ptcp} & participle\\
\textsc{ptv} & partitive\\
\textsc{q} & question  marker\\
\textsc{rel} & relative\\
\textsc{sg} & singular\\
\\
\end{tabularx}


{\sloppy
\printbibliography[heading=subbibliography,notkeyword=this] 
}
\end{document}
