\chapter{Space}
\hypertarget{Toc75352703}{}
In addition to tense and aspect, which situate an event in time, Fwe verbs may be inflected for space, situating the event in the physical space. The distal marker indicates that the event takes place away from the deictic center, e.g. in a place other than where the utterance is spoken (\sectref{bkm:Ref489965878}). Fwe also has a locative pluractional, which indicates that an event takes place in multiple locations (\sectref{bkm:Ref494442567}).

\section{Distal}
\label{bkm:Ref489965878}\hypertarget{Toc75352704}{}
Fwe has a post-initial distal prefix \textit{ka}-, not to be confused with the pre-initial prefix \textit{ka-}, which marks the remote past imperfective (see \sectref{bkm:Ref492377456}), or negation (see \sectref{bkm:Ref490843739}). The prefix \textit{ka}- as a distal marker is well-attested in Bantu languages, especially in south-central Bantu \citep{Botne1999}.

The distal is used to indicate that an action takes place away from the deictic center, usually the place where the utterance is spoken. In the utterance in (\ref{bkm:Ref431991588}), the speaker uses the distal because it is spoken at a place other than his house, hence the action referred to and the place where the utterance is spoken are not the same. The use of the distal in (\ref{bkm:Ref431991716}) is necessary because this utterance describes an action taking place in Namibia, and the utterance was spoken at the speaker’s home village in Zambia.

\ea
\label{bkm:Ref431991588}
kùnjûò ndìkàzwâ\\
\gll ku-N-júo    ndi-ka-zw-á̲\\
\textsc{np}\textsubscript{17}-\textsc{np}\textsubscript{9}-house  \textsc{sm}\textsubscript{1SG}-\textsc{dist}-come\_out-\textsc{fv}\\
\glt ‘I came from home.’ (NF\_Elic15)
\z

\ea
\label{bkm:Ref431991716}
mwákàrí kwànàmíbyá kàndìkàsèbèzâ\\
\gll mu-ákarí  kwa-namibyá  ka-ndi-ka-sebez-á̲\\
\textsc{np}\textsubscript{3}-last\_year  \textsc{np}\textsubscript{17}-Namibia  \textsc{pst}.\textsc{ipfv}-\textsc{sm}\textsubscript{1SG}-\textsc{dist}-work-\textsc{fv}\\
\glt ‘Last year I worked in Namibia.’ (ZF\_Elic14)
\z

Bantu languages with distal \textit{ka-} may differ in terms of which moods the distal \textit{ka-} can combine with \citep{Botne1999}. In Fwe, the distal \textit{ka-} can be used in all moods. Examples of the distal marker used in the indicative were given in (\ref{bkm:Ref431991588}) and (\ref{bkm:Ref431991716}). The distal marker can also combine with an infinitive, as in (\ref{bkm:Ref416186425}). When the distal combines with an infinitive, the infinitive prefix \textit{ku-} is replaced by the distal prefix \textit{ka-}.\footnote{The change from the infinitive prefix \textit{ku-} to \textit{ka-} when used with a distal is one of the main diagnostics that can be used to identify infinitives, both synchronically and diachronically, in verbal constructions that derive from earlier infinitive forms. The other main diagnostic is lack of melodic tone.}

\ea
\label{bkm:Ref416186425}
nàndámànà kàtémà èmìsùmò\\
\gll na-ndí̲-a-man-a    ka-tém-a    e-mi-sumo\\
\textsc{rem}-\textsc{sm}\textsubscript{1SG}-\textsc{pst}-finish-\textsc{fv}  \textsc{inf}.\textsc{dist}-chop-\textsc{fv}  \textsc{aug}-\textsc{np}\textsubscript{4}-pole\\
\glt ‘I finished chopping poles there.’ (ZF\_Elic14)
\z

The distal can also be used with verbs in the imperative, as in (\ref{bkm:Ref451417267}--\ref{bkm:Ref494982220}), and in the subjunctive, as in (\ref{bkm:Ref451417268}). Note that the imperative and the subjunctive take the same form, but are distinguished by the use of the subject marker (see Chapter \ref{bkm:Ref99112450}).

\ea
\label{bkm:Ref451417267}
yêndè kàtêkè mênjì\\
\gll é̲nd-e    ka-té̲k-e    ma-ínji\\
go-\textsc{pfv}.\textsc{sbjv}  \textsc{dist}-fetch-\textsc{pfv}.\textsc{sbjv}  \textsc{np}\textsubscript{6}-water\\
\glt ‘Go and fetch water.’ (ZF\_Elic14)
\z

\ea
\label{bkm:Ref494982220}
kàsúmwínè bànyòkò\\
\gll ka-sumwin-é̲  ba-nyoko\\
\textsc{dist}-tell-\textsc{pfv}.\textsc{sbjv}  \textsc{np}\textsubscript{2}-mother\\
\glt ‘Go tell your mother.’ (NF\_Elic17)
\z

\ea
\label{bkm:Ref451417268}
kùtêyè ndìkàkùmbùré rùkùmbà\\
\gll kuteye  ndi-ka-kumbur-é̲    ru-kumba\\
that    \textsc{sm}\textsubscript{1SG}-\textsc{dist}-strip-\textsc{pfv}.\textsc{sbjv}  \textsc{np}\textsubscript{11}-fibre\\
\glt ‘… in order to cut strips of fibre there.’ (ZF\_Narr14)
\z

In many Bantu languages, the distal \textit{ka}- is interpreted as ‘to go and X’. This itive semantics is possibly the result of a grammaticalization of a verb ‘to go’, for which evidence can be found in southern Bantoid and northwestern Narrow Bantu languages \citep{Botne1999}. The development of distal markers from verbs of motion is a well-attested grammaticalization path (\citealt{HeineEtAl1993}: 103-104), and is also seen in two Tanzanian Bantu languages \citep{Nicolle2003}. The link between the distal marker and an itive interpretation is not seen in all languages, however; in Yeyi, a Bantu language geographically but not genealogically close to Fwe, the distal marker \textit{ka}- is not interpreted as itive \citep{Seidel2007}. In Fwe, itive semantics do appear to form a central part of the interpretation of the distal marker \textit{ka}-. This is seen in the use of the distal with imperative verbs, as in example (\ref{bkm:Ref494982220}) above, where the itive semantics ‘go and’ is contributed by the distal marker alone. Another example showing that motion is a necessary component for the use of distal \textit{ka-} is illustrated in (\ref{bkm:Ref416188091}--\ref{bkm:Ref74909097}), drawn from a narrative. In (\ref{bkm:Ref416188091}), the speaker narrates that he moves away from the deictic center, as attested by his use of the distal marker \textit{ka-} on the verb. Having reached this place, a second event takes place; he hears Claudia calling him. His hearing of Claudia takes place away from the deictic center, but no movement is involved; therefore, the distal marker is not used in (\ref{bkm:Ref74909097}).

\ea
\label{bkm:Ref416188091}
àhá ndíkàhùrá kùrwâmbà\\
\gll a-ha    ndí̲-ka-hur-á̲      ku-ru-ámba\\
\textsc{dem}.\textsc{i}\textsubscript{16} \textsc{sm}\textsubscript{1SG}.\textsc{rel}-\textsc{dist}-arrive-\textsc{fv}  \textsc{np}\textsubscript{17}-\textsc{np}\textsubscript{11-}middle\_of\_field\\
\glt ‘(…) when I reached the middle of the field…’
\z

\ea
\label{bkm:Ref74909097}
ndìshùwîrè bàklàùdìyà bàndìkûwà\\
\gll ndi-shu\textsubscript{H}-í̲re    ba-klaudia  ba-ndi-kú̲-a\\
\textsc{sm}\textsubscript{1SG}-hear-\textsc{stat}  \textsc{np}\textsubscript{2}-Claudia  \textsc{sm}\textsubscript{2}-\textsc{om}\textsubscript{1SG}-call\\
\glt ‘I heard Mrs. Claudia calling me.’ (ZF\_Narr13)
\z

These examples suggest that motion is a necessary component of the interpretation of the distal prefix \textit{ka-}. More specifically, it encodes motion away from the deictic center, and is not used for motion towards the deictic center. In (\ref{bkm:Ref431992358}), the verb \textit{bàhúrè} ‘he will arrive’ is used without the distal because the place of the expected arrival is the same place as the place of speaking. In (\ref{bkm:Ref431992424}), the verb \textit{kàndíkêːzyà} ‘I was coming’ is used without the distal because it describes a journey that ends at the place of speaking.

\ea
\label{bkm:Ref431992358}
ênì òbùrótù mbòkúꜝté bàhúrè tùrâːrè\\
\gll éni  o-bu-rótu    N-bo-kúteyé    ba-hur-é̲ tu-rá̲ːr-e \\
yes  \textsc{aug}-\textsc{np}\textsubscript{14}-good  \textsc{cop}-\textsc{np}\textsubscript{14}-that  \textsc{sm}\textsubscript{2}-arrive-\textsc{pfv}.\textsc{sbjv}
\textsc{sm}\textsubscript{1PL}-sleep-\textsc{pfv}.\textsc{sbjv}\\
\glt ‘Yes, it’s good that he comes back and we spend the night here.’ (NF\_Narr15)
\z

\ea
\label{bkm:Ref431992424}
àhá kàndíkêːzyà ndàhîtì òcècì\\
\gll a-ha    ka-ndí̲-ké̲ːzy-a ndi-a-hít-i      o-∅-ceci\\
\textsc{aug}-\textsc{dem}.\textsc{i}\textsubscript{16} \textsc{pst}.\textsc{ipfv}-\textsc{sm}\textsubscript{1SG}-come-\textsc{fv}
\textsc{sm}\textsubscript{{\textasciigrave}1SG}-\textsc{pst}-pass-\textsc{npst}.\textsc{pfv}  \textsc{aug}-\textsc{np}\textsubscript{1a}-church\\
\glt ‘When I came here, I passed by the church.’ (ZF\_Elic14)
\z
\section{Locative pluractional}
\label{bkm:Ref494442567}\hypertarget{Toc75352705}{}
The post-initial prefixes \textit{yabú-} and \textit{kabú-} both express a locative pluractional, an event that is carried out in different places. \textit{kabú}- and \textit{yabú}- are interchangeable, and no difference in meaning could be observed. Which form is used appears to depend on the individual speaker’s preference. Both locative pluractional prefixes are illustrated in (\ref{bkm:Ref99103614}).

\ea
\label{bkm:Ref99103614}
cìkàbúkùkà {\textasciitilde} cìyàbúkùkà\\
\gll ci-kabú/yabú-kuk-a\\
\textsc{sm}\textsubscript{7}-\textsc{loc}.\textsc{pl}-float-\textsc{fv}\\
\glt ‘It floats, it goes by floating.’ (NF\_Elic17)
\z

The locative pluractional indicates an event taking place in different places: in (\ref{bkm:Ref489874030}), without locative pluractional, the verb \textit{rí}ː\textit{zy} indicates climbing in one place, and in (\ref{bkm:Ref74909162}), with a locative pluractional, the verb \textit{rí}ː\textit{zy} indicates climbing in several places.

\ea
\label{bkm:Ref489874030}
ndìkwèsì ndìrî̲ːzyà\\
\gll ndi-kwesi  ndi-ríːzy-a\\
\textsc{sm}\textsubscript{1SG}-\textsc{prog}  \textsc{sm}\textsubscript{1SG}-climb-\textsc{fv}\\
\glt ‘I am climbing.’
\z

\ea
\label{bkm:Ref74909162}
\glll ndìkàbúrìːzyà\\
ndi-kabú-riːzy-a\\
\textsc{sm}\textsubscript{1SG}- \textsc{loc}.\textsc{pl}-climb-\textsc{fv}\\
\glt ‘I am going around climbing, I am climbing in different places.’ (NF\_Elic17)
\z

The locative pluractional differs from the two other pluractional strategies used in Fwe, which are not strictly locative. As discussed in \sectref{bkm:Ref489866362}, these pluractional strategies may express that an event is repeated, or involves multiple participants. The locative pluractional suffix \textit{yabú}-/\textit{kabú-} only expresses that an event is repeated in different locations. It may combine with either or both of the other pluractional strategies, as in (\ref{bkm:Ref99103664}--\ref{bkm:Ref99103665}), combining the interpretation of event repetition of pluractional I or II with the locative pluractional’s interpretation of spatial distribution.

\ea
\label{bkm:Ref99103664}
Locative pluractional + Pluractional I (suffix -a)\\
\glll ndìkàbúbàsùndàíkà\\
ndi-kabú-ba-sund-a-ik-á̲\\
\textsc{sm}\textsubscript{1SG}-\textsc{loc}.\textsc{pl}-\textsc{om}\textsubscript{2}-point-\textsc{pl}1-\textsc{imp}.\textsc{tr}-\textsc{fv}\\
\glt ‘I am going around pointing at them.’
\z

\ea
Locative pluractional + Pluractional II (stem reduplication)\\
àkàbúkàbìràkàbìrà múmàràpá ꜝábàntù\\
\gll a-kabú-kabira-kabir-a  mú-ma-rapá    a-á=ba-ntu\\
\textsc{sm}\textsubscript{1}-\textsc{loc}.\textsc{pl}-\textsc{pl}2-enter-\textsc{fv}  \textsc{np}\textsubscript{18}-\textsc{np}\textsubscript{6}-courtyard  \textsc{pp}\textsubscript{6}-\textsc{con}=\textsc{np}\textsubscript{2}-person\\
\glt ‘S/he keeps going round entering people's courtyards.’ (NF\_Elic17)
\z

\ea
\label{bkm:Ref99103665}
Locative pluractional + Pluractional I + Pluractional II\\
nàkàyâ ìyé àkábúyèndàùràyèndàùrà òkábúbônà\\
\gll na=ka-y-á      iyé  a-kabú-endaura-end-a-ur-a    o-kabú-bón-a \\
\textsc{com}=\textsc{inf}.\textsc{dist}-go-\textsc{fv}  that
\textsc{sm}\textsubscript{1}-\textsc{loc}.\textsc{pl}-\textsc{pl}2-go-\textsc{pl}1-\textsc{sep}.\textsc{tr}-\textsc{fv}  \textsc{aug}-\textsc{loc}.\textsc{pl}-see-\textsc{fv}\\
\glt ‘And he went out to walk around, and look around.’ (NF\_Narr17)
\z

The exact interpretation of the locative pluractional depends on the lexical aspect of the verb, as well as the wider linguistic context. Two main interpretations are possible: associated motion, where the event and motion co-occur (‘go while X-ing’), and distributive, where the event alternates with motion (‘go and X, go and X’). The associated motion interpretation of the locative pluractional is available with verbs that have a long nucleus, such as dynamic verbs. This is illustrated with the verb \textit{shíb} ‘whistle’ in (\ref{bkm:Ref99103717}), which expresses whistling while moving when combined with the locative pluractional.

\ea
\label{bkm:Ref99103717}
\glll àkábúꜝshíbà\\
a-kabú-shib-á̲\\
\textsc{sm}\textsubscript{1}-\textsc{loc}.\textsc{pl}-whistle-\textsc{fv}\\
\glt ‘S/he whistles while walking.’ (NF\_Elic17)
\z

Stative verbs also have a long nucleus, and therefore the locative pluractional is interpreted as associated motion with these verbs, as shown for the stative verb \textit{tíy} ‘be afraid’ in (\ref{bkm:Ref497245742}).

\ea
\label{bkm:Ref497245742}
\glll àkàbútìyà\\
a-kabú-tiy-a\\
\textsc{sm}\textsubscript{1}-\textsc{loc}.\textsc{pl}-be\_afraid-\textsc{fv}\\
\glt ‘S/he is afraid on the way/while going.’ (NF\_Elic17)
\z

The locative pluractional may also take a distributive interpretation with dynamic verbs, marking that an event takes place in different places, as in (\ref{bkm:Ref99103790}).

\ea
\label{bkm:Ref99103790}
mbùryàhó kàbákàbúpàngà bùryáhò\\
\gll N-bu-ryahó    ka-bá̲-kabú-pang-a      bu-ryahó\\
\textsc{cop}-\textsc{np}\textsubscript{14}-like\_that  \textsc{pst}.\textsc{ipfv}-\textsc{sm}\textsubscript{2}-\textsc{loc}.\textsc{pl}-do-\textsc{fv}  \textsc{np}\textsubscript{14}-like\_that\\
\glt ‘That is how he used to do in different places.’ (NF\_Narr17)
\z

Whether the locative pluractional with dynamic verbs is interpreted as associated motion or distributive depends on the lexical semantics of the verb, as well as the wider context. The associated motion interpretation is typically limited to events that may logically co-occur with motion, such as motion verbs, as in (\ref{bkm:Ref99103820}--\ref{bkm:Ref99103822}).

\ea
\label{bkm:Ref99103820}
ndìyàbúyèndà bùryáhò ndókùryàt’ énjôkà\\
\gll ndi-yabú-end-a    bu-ryahó ndí-o-ku-ryat-á    e-N-jóka\\
\textsc{sm}\textsubscript{1SG}-\textsc{loc}.\textsc{pl}-walk-\textsc{fv}  \textsc{np}\textsubscript{14}-like\_that
\textsc{con}\textsubscript{1SG}-\textsc{aug}-\textsc{inf}-step-\textsc{fv}  \textsc{aug}-\textsc{np}\textsubscript{9}-snake\\
\glt ‘I was walking like that, then I stepped on a snake.’ (ZF\_Narr13)
\z

\ea
kùshàmbà ndíꜝkábúꜝshámbà\\
\gll ku-shamb-a  ndí̲-kabú-shá̲mb-a\\
\textsc{inf}-swim-\textsc{fv}  \textsc{sm}\textsubscript{1SG}.\textsc{rel}-\textsc{loc}.\textsc{pl}-swim-\textsc{fv}\\
\glt ‘I am swimming (across a distance, or to somewhere).’ (NF\_Elic15)
\z

\ea
\label{bkm:Ref99103822}
\glll àkàyàbúcòbà\\
a-ka-yabú-cob-a\\
\textsc{sm}\textsubscript{1}-\textsc{dist}-\textsc{loc}.\textsc{pl}-cycle-\textsc{fv}\\
\glt ‘She goes riding the bicycle.’ (NF\_Narr17)
\z

The locative pluractional has a distributive interpretation with change-of-state verbs that lack an onset phase, such as the verb \textit{w} ‘fall’ in (\ref{bkm:Ref497245841}); when combined with the locative pluractional, it expresses something that repeatedly falls in different places.

\ea
\label{bkm:Ref497245841}
\glll cìkàbúwà\\
ci-kabú-w-a\\
\textsc{sm}\textsubscript{7}-\textsc{loc}.\textsc{pl}-fall-\textsc{fv}\\
\glt ‘It keeps falling. (while traveling; the item keeps falling out of your pocket in different places)’ (NF\_Elic17)
\z

Change-of-state verbs without an onset phase also take \textit{yabú}-/\textit{kabú-}, but in this case it is interpreted as a gradual change through time, as in (\ref{bkm:Ref99103865}--\ref{bkm:Ref99103868}).

\ea
\label{bkm:Ref99103865}
kànínì kànînì kùfúmà bákàbúfùmà bénà\\
\gll ka-níni  ka-níni ku-fúm-a    bá-kabú-fum-a      bená \\
\textsc{adv}-small  \textsc{adv}-small
\textsc{inf}-get\_rich-\textsc{fv}  \textsc{sm}\textsubscript{2}.\textsc{rel}-\textsc{loc}.\textsc{pl}-get\_rich-\textsc{fv}  \textsc{dem}.\textsc{iv}\textsubscript{2}\\
\glt ‘S/he is slowly getting more and more rich.’
\z

\ea
\glll cìkàbúrèmà\\
ci-kabú-rem-a\\
\textsc{sm}\textsubscript{7}-\textsc{loc}.\textsc{pl}-become\_heavy-\textsc{fv}\\
\glt ‘It is becoming heavy.’ (of something that you have been carrying for a long time) (NF\_Elic17)
\z

\ea
\label{bkm:Ref99103868}
\glll shèkùkàbúhìsà\\
she-ku-kabú-his-a\\
\textsc{inc}-\textsc{sm}\textsubscript{17}-\textsc{loc}.\textsc{pl}-become\_hot-\textsc{fv}\\
\glt ‘It is becoming hot.’ (NF\_Elic15)
\z

The markers \textit{kabú}- and \textit{yabú-} are historically derived from an inflected verb followed by a verb with the adverbial prefix \textit{bú-} (see \sectref{bkm:Ref489870394} on adverbs). The syllable \textit{ya} is derived from the lexical verb \textit{ya} ‘go’, which is still used in Fwe with this meaning. \textit{kabú-} is the result of the contraction of distal \textit{ka-} with the locative pluractional \textit{yabú-}. In modern Fwe, \textit{ka-yabú-} is considered to be interchangeable with \textit{kabú-}, as shown in (\ref{bkm:Ref99103952}). The original deictic semantics of distal \textit{ka-} have been lost in \textit{kabú-}, which does not mark motion away from the deictic center.

\ea
\label{bkm:Ref99103952}
ùkàyàbútùmbúkà {\textasciitilde} ùkàbútùmbúkà\\
\gll u-ka-yabú-tumbuk-á     {\textasciitilde}  u-kabú-tumbuk-á\\
\textsc{sm}\textsubscript{3}-\textsc{dist}-\textsc{loc}.\textsc{pl}-burn-\textsc{fv}    {\textasciitilde}  \textsc{sm}\textsubscript{3}-\textsc{loc}.\textsc{pl}-burn-\textsc{fv}\\
\glt ‘It [fire] comes while burning.’ (NF\_Elic17)
\z

When the prefix \textit{yabú}-/\textit{kabú-} grammaticalized, the earlier inflected verb lost its status as an independent lexical verb. This can be seen by the lack of melodic tone in the \textit{ya}/\textit{ka} element, and by optional high tone spread from \textit{bú} to the preceding syllable, e.g. \textit{yábú-} and \textit{kábú-}. High tone spread does not cross word boundaries (see \sectref{bkm:Ref430865664}), so its occurrence shows that the formerly independent verb has become part of the prefix.

A similar marker \textit{yabo-} is found in Subiya, as in \textit{ch’o ya bo sibila} ‘he goes while whistling’, which is also analyzed as a combination of the prefix \-\textit{bo} and the lexical verb \textit{ya} ‘go’ \citep[61]{Jacottet1896}.

