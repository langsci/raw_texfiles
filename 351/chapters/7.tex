\chapter{Subject, object, and locative marking}
\label{bkm:Ref100140368}\label{bkm:Ref99105339}\label{bkm:Ref98512234}\hypertarget{Toc75352672}{}
Subjects and objects are marked on the verb with a prefix, and locatives with a clitic. Subject marking is obligatory, independent of whether a subject noun is used in the clause. Object marking only occurs when no object noun is used in the same clause. Locative marking may also only refer to a locative complement that is introduced in an earlier clause, or is otherwise understood from the discourse or physical environment.

\section{Subject marking}
\label{bkm:Ref451511047}\hypertarget{Toc75352673}{}
Subjects are marked on the verb by a prefix. \tabref{tab:7:1} gives an overview of the subject markers for each speech act participant and noun class, which will be glossed as “\textsc{sm}” with the number of the noun class in subscript. Subject markers are all toneless, and surface as low-toned unless a melodic tone is assigned, which is the case in certain TAM constructions and most relative clauses (see \sectref{bkm:Ref71539267} on melodic tone).

\begin{table}
\label{bkm:Ref506391424}\caption{\label{tab:7:1}Subject markers}
\begin{tabularx}{\textwidth}{QlQl}
\lsptoprule
Noun class/first or second person (singular) & Subject marker & Noun class/first or second person (plural) & Subject marker\\
\midrule
1\textsc{sg} & {\itshape ndi-} & 1\textsc{pl} & {\itshape tu-}\\
2\textsc{sg} & {\itshape u-} & 2\textsc{pl} & {\itshape mu-}\\
1/1a & {\itshape a-} & 2 & {\itshape ba-}\\
3 & {\itshape u-} & 4 & {\itshape i-}\\
5 & {\itshape ri-} & 6 & {\itshape a-}\\
7 & {\itshape ci-} & 8 & {\itshape zi-}\\
9 & {\itshape i-} & 10 & {\itshape zi-}\\
11 & {\itshape ru-} &  & \\
12 & {\itshape ka-} & 13 & {\itshape tu-}\\
14 & {\itshape bu-} &  & \\
15 & {\itshape ku-} &  & \\
16 & {\itshape ha-} &  & \\
17 & {\itshape ku-} &  & \\
18 & {\itshape mu-} &  & \\
\lspbottomrule
\end{tabularx}
\end{table}

The subject marker is obligatory, whether the subject noun is used in the same clause as the verb, as in (\ref{bkm:Ref452540561}), or is absent from the clause, as in (\ref{bkm:Ref74906677}).\largerpage

\ea
\label{bkm:Ref452540561}
ècí cìpùrà càcôːkì\\
\gll e-cí    ci-pura  ci-a-cóːk-i\\
\textsc{aug}-\textsc{dem}.\textsc{i}\textsubscript{7}  \textsc{np}\textsubscript{7}-chair  \textsc{sm}\textsubscript{7}-\textsc{pst}-break-\textsc{npst}.\textsc{pfv}\\
\glt ‘This chair is broken.’
\z

\ea
\label{bkm:Ref74906677}
\glll càcôːkì \\
ci-a-cóːk-i\\
\textsc{sm}\textsubscript{7}-\textsc{pst}-break-\textsc{npst}.\textsc{pfv}\\
\glt ‘It is broken.’ (ZF\_Elic14)
\z

When the subject noun is not used in the same clause, the subject marker still agrees in noun class with the intended subject noun. In (\ref{bkm:Ref452543253}), the class 7 subject marker \textit{ci-} in the verb \textit{cìbònàhàrá} ‘it looked’ refers back to the noun phrase \textit{cìmbòtwé cìnênè} ‘a big frog’, that was introduced in the previous sentence. In (\ref{bkm:Ref452543245}), the people that the speaker describes are standing close by and can therefore be inferred from the physical surroundings.

\ea
\label{bkm:Ref452543253}
àkàbônà ècìbwângà \textbf{cìmbòtwé} \textbf{cìnênè} \textbf{cì}bònàhàrá òbùnénènênè\\
\gll a-ka-bón-a    e-ci-bwánga    ci-mbotwé  ci-néne ci-bo\textsubscript{H}n-ahar-á̲  o-bu-néne-néne\\
\textsc{sm}\textsubscript{1}-\textsc{dist}-see-\textsc{fv}  \textsc{aug}-\textsc{np}\textsubscript{7}-frog  \textsc{np}\textsubscript{7}-frog  \textsc{np}\textsubscript{7}-big
\textsc{sm}\textsubscript{7}-see-\textsc{neut}-\textsc{fv}  \textsc{aug}-\textsc{np}\textsubscript{14}-big-big\\
\glt ‘He saw a frog there, \textbf{a} \textbf{big} \textbf{frog}. \textbf{It} looked very, very big.’ (NF\_Narr15)
\z

\ea
\label{bkm:Ref452543245}
bàkwèsì bàkòndòr’ óbùjwàrà\\
\gll ba-kwesi  ba-kondor-á̲  o-bu-jwara\\
\textsc{sm}\textsubscript{2}-have  \textsc{sm}\textsubscript{2}-brew-\textsc{fv}  \textsc{aug}-\textsc{np}\textsubscript{14}-beer\\
\glt ‘They’re brewing beer.’ (NF\_Elic15)
\z

Subject agreement becomes more complex when the subject consists of coordinated nouns of different noun classes. Different languages employ different gender resolution rules, i.e. the strategies which determine agreement with coordinated noun phrases, which may be based on (a combination of) syntactic and semantic criteria \citep{Corbett1991}. In Fwe, class 8 agreement is used, at least if both nouns are non-human, as in (\ref{bkm:Ref451258925}) and (\ref{bkm:Ref451258926}). No distinction is made between animate and inanimate non-human nouns. Data on the agreement patterns of coordinated nouns referring to humans are limited. Although this requires further research, it may suggest that Fwe tends to avoid such constructions.

\ea
\label{bkm:Ref451258925}
zìzyùnì nàbànkûkù zìzárà màyîː\\
\gll zi-zyuni  na=ba-nkúku    zi-zá̲r-a    ma-yíː\\
\textsc{np}\textsubscript{8}-bird  \textsc{com}=\textsc{np}\textsubscript{2}-chicken  \textsc{sm}\textsubscript{8}-give.birth-\textsc{fv}  \textsc{np}\textsubscript{6}-egg\\
\glt ‘Birds and chickens lay eggs.’
\z

\ea
\label{bkm:Ref451258926}
mwêzì nèzyûbà mùwírú ꜝzínà\\
\gll mu-ézi  ne=∅-zyúba    mu-∅-wirú  zi-iná\\
\textsc{np}\textsubscript{3}-moon  \textsc{com}=\textsc{np}\textsubscript{5}-sun    \textsc{np}\textsubscript{18}-\textsc{np}\textsubscript{5}-sky  \textsc{sm}\textsubscript{8}-be\_at\\
\glt ‘The moon and the sun are in the sky.’ (NF\_Elic15)
\z
\section{Object marking}
\label{bkm:Ref451511050}\hypertarget{Toc75352674}{}\label{bkm:Ref75169159}
Objects can be marked on the verb through use of an object marker, a prefix that appears directly before the verb stem. \tabref{tab:7:2} gives an overview of the object markers per noun class and speech act participant. Fwe lacks object markers for the locative classes 16, 17 and 18. All object markers are high-toned, except those of the first and second person singular and of class 1, which are underlyingly toneless. When used in TAM constructions that take melodic tone 4, the deletion of underlying tones, high-toned object markers lose their high tone (see \sectref{bkm:Ref71539267} on melodic tone).

\begin{table}
\label{bkm:Ref505870543}\caption{\label{tab:7:2}Object markers}
\fittable{\begin{tabular}{llll}
\lsptoprule
Noun class/person & Object marker & Noun class/person & Object marker\\
\midrule
1\textsc{sg} & \textit{ndi-}\footnote{Traces of an older first person singular object prefix \textit{N-}, rather than the prefix \textit{ndi-}, are seen in proper names and in what speakers consider ‘archaic Fwe’; see \sectref{bkm:Ref451507060} for examples.} & 1\textsc{pl} & {\itshape tú-}\\
2\textsc{sg} & {\itshape ku-} & 2\textsc{pl} & {\itshape mí-}\\
1/1a & {\itshape mu-} & 2 & {\itshape bá-}\\
3 & {\itshape ú-} & 4 & {\itshape yí-}\\
5 & {\itshape rí-} & 6 & {\itshape á-}\\
7 & {\itshape cí-} & 8 & {\itshape zí-}\\
9 & {\itshape yí-} & 10 & {\itshape zí-}\\
11 & {\itshape rú-} &  & \\
12 & {\itshape ká-} & 13 & {\itshape tú-}\\
14 & {\itshape bú-} &  & \\
15 & {\itshape kú-} &  & \\
\lspbottomrule
\end{tabular}
}
\end{table}

Object markers can only be used when no object noun is used in the same clause. The noun class of the object marker corresponds to that of the intended noun. (\ref{bkm:Ref448142515}) is the answer to a question about \textit{ngùbò} ‘blankets’; as this is a noun of class 10, the class 10 object marker is used.

\ea
\label{bkm:Ref448142515}
ndàzíhîndì ndìkàzìsânzà\\
\gll ndi-a-zí-hind-i      ndi-ka-zi\textsubscript{H}-sá̲nz-a\\
\textsc{sm}\textsubscript{1SG}-\textsc{pst}-\textsc{om}\textsubscript{10}-take-\textsc{npst}.\textsc{pfv}  \textsc{sm}\textsubscript{1SG}-\textsc{dist}-\textsc{om}\textsubscript{10}-wash-\textsc{fv}\\
\glt ‘I took them to wash them.’ (NF\_Elic15)
\z

An object marker is obligatory when the intended noun is not in the same clause as the verb. This is the case, for instance, with dislocated objects, as in (\ref{bkm:Ref452554600}), where a constituent is moved to the left periphery of a sentence to function as a topic.

\ea
\label{bkm:Ref452554600}
òrú rùzyîmbò kàndìrúꜝshákì\\
\gll o-rú    ru-zyímbo  ka-ndi-rú-shak-í̲\\
\textsc{aug}-\textsc{dem}.\textsc{i}\textsubscript{11}  \textsc{np}\textsubscript{11}-song  \textsc{neg}-\textsc{sm}\textsubscript{1SG}-\textsc{om}\textsubscript{11}-like-\textsc{neg}\\
\glt ‘This song, I don’t like it.’ (NF\_Elic15)
\z

Constituents can also be moved out of a clause to the right periphery as a way of definiteness marking. When right dislocation targets object constituents, they retain their canonical post-verbal position, but require the use of an object marker of the verb, as in (\ref{bkm:Ref445977500}--\ref{bkm:Ref99093552}).

\ea
\label{bkm:Ref445977500}
ndìzìsháká ꜝzí nswì\\
\gll ndi-zi\textsubscript{H}-shak-á̲  zí    N-swi\\
\textsc{sm}\textsubscript{1SG}-\textsc{om}\textsubscript{10}-like-\textsc{fv}  \textsc{dem}.\textsc{i}\textsubscript{10}  \textsc{np}\textsubscript{10}-fish\\
\glt ‘I like these fish.’
\z

\ea
\label{bkm:Ref99093552}
ndàyíbàrì èyí mbùkà\\
\gll ndi-a-í-bar-i        e-í    N-buka\\
\textsc{sm}\textsubscript{1SG}-\textsc{pst}-\textsc{om}\textsubscript{9}-read-\textsc{npst}.\textsc{pfv}  \textsc{aug}-\textsc{dem}.\textsc{i}\textsubscript{9}  \textsc{np}\textsubscript{9}-book\\
\glt ‘I’ve read this book.’ (NF\_Elic15)
\z

For a discussion of left and right dislocation, and a more detailed analysis of post-verbal objects with an object marker as a case of right dislocation, see chapter \ref{bkm:Ref99093567}.

A ditransitive verb can have multiple object markers, which appear in a fixed order: the object marker for the benefactive object appears closer to the stem than the object marker for the theme object. This is shown in (\ref{bkm:Ref491427919}), where the class 2 object marker referring to the benefactive object (‘for her’) appears closer to the stem than the class 13 object marker referring to the theme object (‘them’; in this case, the speaker is referring to dishes).

\ea
\label{bkm:Ref491427919}
\ea
\glll àtùbàsànzírà\\
a-tu\textsubscript{H}-ba\textsubscript{H}-sanz-ir-á̲\\
\textsc{sm}\textsubscript{1}-\textsc{om}\textsubscript{13}-\textsc{om}\textsubscript{2}-wash-\textsc{appl}-\textsc{fv}\\
\glt ‘I wash them for her.’

\ex
*àbàtùsànzírà
\z\z

Verbs can take up to three object markers, as in (\ref{bkm:Ref491427907}). I was unable to come up with a suitable context in which four or more object markers might be warranted; possibly, given the right context, such constructions might be acceptable.

\ea
\label{bkm:Ref491427907}
\glll cìmùndìsúndîrè\\
ci\textsubscript{H}-mu-ndi-su\textsubscript{H}nd-í̲r-e\\
\textsc{om}\textsubscript{7}-\textsc{om}\textsubscript{1}-\textsc{om}\textsubscript{1SG}-show-\textsc{appl}-\textsc{pfv}.\textsc{sbjv}\\
\glt ‘Show it to her/him for me.’ (NF\_Elic17)
\z

Multiple object markers are not allowed when two or more object markers refer to an inanimate object. This is illustrated with the sentence in (\ref{bkm:Ref488927201}), containing two inanimate objects. It is possible to express either of these objects with an object marker, as in (\ref{bkm:Ref74907372}) and (\ref{bkm:Ref74907373}), but not both, as the ungrammaticality of (\ref{bkm:Ref74907374}) shows.

\ea
\label{bkm:Ref488927201}
ndìzyàːkìr’ ómùndáré ꜝwángù cìòngò\\
\gll ndi-zyaː\textsubscript{H}k-ir-á̲    o-mu-ndaré    u-angú  ci-ongo\\
\textsc{sm}\textsubscript{1SG}-build-\textsc{appl}-\textsc{fv}  \textsc{aug}-\textsc{np}\textsubscript{3}-maize  \textsc{pp}\textsubscript{3}-\textsc{poss}\textsubscript{1SG}  \textsc{np}\textsubscript{7}-storage\\
\glt ‘I am building a storage for my maize.’
\z

\ea
\label{bkm:Ref74907372}
ndìcìzyàːkìr’ ómùndárè\\
\gll ndi-ci\textsubscript{H}-zyaː\textsubscript{H}k-ir-á̲    o-mu-ndaré\\
\textsc{sm}\textsubscript{1SG}-\textsc{om}\textsubscript{7}-build-\textsc{appl}-\textsc{fv}  \textsc{aug}-\textsc{np}\textsubscript{3}-maize\\
\glt ‘I am building it for the maize.’
\z

\ea
\label{bkm:Ref74907373}
ndìùzyàːkìr’ écìòngò\\
\gll ndi-u\textsubscript{H}-zyaː\textsubscript{H}k-ir-á̲    e-ci-ongo\\
\textsc{sm}\textsubscript{1SG}-\textsc{om}\textsubscript{3}-build-\textsc{appl}-\textsc{fv}  \textsc{aug}-\textsc{np}\textsubscript{7}-storage\\
\glt ‘I am building a storage for it.’
\z

\ea
\label{bkm:Ref74907374}
\glll *ndìùcìzyàːkírà\\
ndi-u\textsubscript{H}-ci\textsubscript{H}-zyaː\textsubscript{H}k-ir-á̲\\
\textsc{sm}\textsubscript{1SG}-\textsc{om}\textsubscript{3}-\textsc{om}\textsubscript{7}-build-\textsc{appl}-\textsc{fv}\\
\glt Intended: ‘I am building it for it.’ (NF\_Elic17)
\z
\section{Reflexive}
\label{bkm:Ref451256199}\hypertarget{Toc75352675}{}
In addition to object markers for noun classes and first and second person, Fwe has a reflexive prefix \textit{kí-} (Zambian Fwe) / \textit{rí-} (Namibian Fwe) which is used in the same position as the object marker. Examples of the use of the reflexive are given in (\ref{bkm:Ref451865865}--\ref{bkm:Ref445986266}).

\ea
\label{bkm:Ref451865865}
\glll ndàkírèmèkì\\
ndi-a-kí-remek-i\\
\textsc{sm}\textsubscript{1SG}-\textsc{pst}-\textsc{refl}-hurt-\textsc{npst}.\textsc{pfv}\\
\glt ‘I’ve hurt myself.’ (ZF\_Elic13)
\z

\ea
\label{bkm:Ref445986266}
àtàtìk’ ókùrínyàyà kùrínyàyà\\
\gll a-tatik-á  o-ku-rí-nyay-a    ku-rí-nyay-a\\
\textsc{sm}\textsubscript{1}-start-\textsc{fv}  \textsc{aug}-\textsc{inf}-\textsc{refl}-scratch-\textsc{fv}  \textsc{inf}-\textsc{refl}-scratch-\textsc{fv}\\
\glt ‘She starts to scratch herself, scratch herself.’ (NF\_Narr15)
\z

The reflexive prefix can be combined with an emphatic reflexive, consisting of the nominal root \textit{íni}, with the lexical meaning ‘owner’, and an agreement prefix. \textit{íni} is inflected for number, e.g. class 1 \textit{mw-înì} for singular and class 2 \textit{b-ênì} for plural. In addition, an appositive prefix is used that is co-referential with the verb’s subject (see \sectref{bkm:Ref492039371} on appositives). Examples of emphatic reflexives are given in (\ref{bkm:Ref492039232}--\ref{bkm:Ref492039233}).

\ea
\label{bkm:Ref492039232}
ndìrìbwènè ndémwìnì\\
\gll ndi-ri\textsubscript{H}-bwe\textsubscript{H}ne    nde-mw-ini\\
\textsc{sm}\textsubscript{1SG}-\textsc{refl}-see.\textsc{stat}    \textsc{app}\textsubscript{1SG}-\textsc{np}\textsubscript{1}-owner\\
\glt ‘I see myself.’ (NF\_Elic15)
\z

\ea
nòkíbònì wèmwînì\\
\gll no-kí-bon-i        we-mu-íni\\
\textsc{sm}\textsubscript{2SG}.\textsc{pst}-\textsc{refl}-see-\textsc{npst}.\textsc{pfv}  \textsc{app}\textsubscript{2SG}-\textsc{np}\textsubscript{1}-owner\\
\glt ‘You see yourself.’
\z

\ea
\label{bkm:Ref492039233}
twàkíbònì tùbênì\\
\gll tu-a-kí-bon-i      tu-ba-íni\\
\textsc{sm}\textsubscript{1PL}-\textsc{pst}-\textsc{refl}-see-\textsc{npst}.\textsc{pfv}  \textsc{app}\textsubscript{1PL}-\textsc{np}\textsubscript{2}-self\\
\glt ‘We see ourselves.’ (ZF\_Elic13)
\z

When the subject is not a first or second person, the nominal root \textit{íni} is marked for noun class agreement with the subject, and an anaphoric demonstrative is used, as in (\ref{bkm:Ref99093781}--\ref{bkm:Ref99093782}).

\ea
\label{bkm:Ref99093781}
sìbàrìkùnkùmúnà kùrícènès’ ábò bênì\\
\gll si-ba-ri\textsubscript{H}-kunkumun-á̲  ku-rí-cen-es-a a-bó    ba-íni\\
\textsc{inc}-\textsc{sm}\textsubscript{2}-\textsc{refl}-brush-\textsc{fv}  \textsc{inf}-\textsc{refl}-be\_clean-\textsc{caus}-\textsc{fv}
\textsc{aug}-\textsc{dem}.\textsc{iii}\textsubscript{2}  \textsc{np}\textsubscript{2}-self\\
\glt ‘He now starts brushing himself off to clean himself.’
\z

\ea
ímùnyà ìkwèsì ìwá èyó yînì\\
\gll í-munya  i-kwesi  i-w-á̲    e-yó    i-íni\\
\textsc{pp}\textsubscript{4}-other  \textsc{sm}\textsubscript{4}\--\textsc{prog}  \textsc{sm}\textsubscript{4}-fall-\textsc{fv}  \textsc{aug}-\textsc{dem}.\textsc{iii}\textsubscript{4}  \textsc{pp}\textsubscript{4}-self\\
\glt ‘Others are falling off their own accord.’ (NF\_Narr17)
\z

\ea
\label{bkm:Ref99093782}
màkwátìrò ànàcôːkì kònó nkòmòkí èyó ꜝyínì kàyâfwì\\
\gll ma-kwátiro  a-na-có̲ːk-i konó  N-komokí  e-yó    i-íni    ka-i-á̲-fw-i\\
\textsc{np}\textsubscript{6}-handle  \textsc{sm}\textsubscript{6}-\textsc{pst}-break-\textsc{npst}.\textsc{pfv}
but  \textsc{np}\textsubscript{9}-cup  \textsc{aug}-\textsc{dem}.\textsc{iii}\textsubscript{9}  \textsc{pp}\textsubscript{9}-self  \textsc{neg}-\textsc{sm}\textsubscript{9}-\textsc{pst}-break-\textsc{npst}.\textsc{pfv}\\
\glt ‘The handle broke, but the cup itself did not break.’ (NF\_Elic17)
\z

The prefix \textit{kí-/rí-} is also used with a reciprocal meaning, as in (\ref{bkm:Ref99093796}--\ref{bkm:Ref99093797}).

\ea
\label{bkm:Ref99093796}
\glll tùrìshákà\\
tu-ri\textsubscript{H}-shak-á̲\\
\textsc{sm}\textsubscript{1PL}-\textsc{refl}-love-\textsc{fv}\\
\glt ‘We love each other.’ (NF\_Elic15)
\z

\ea
\glll tùkìshúwîrè\\
tu-ki\textsubscript{H}-shu\textsubscript{H}-í̲re\\
\textsc{sm}\textsubscript{1PL}-\textsc{refl}-hear-\textsc{stat}\\
\glt ‘We hear each other.’ (ZF\_Elic14)
\z

\ea
\label{bkm:Ref99093797}
màmésàjì bákìŋòrérà\\
\gll N-ma-mésaji  bá̲-ki\textsubscript{H}-ŋo\textsubscript{H}r-er-á̲\\
\textsc{cop}-\textsc{np}\textsubscript{6}-message  \textsc{sm}\textsubscript{2}.\textsc{rel}-\textsc{refl}-write-\textsc{appl}-\textsc{fv}\\
\glt ‘It’s messages that they write to each other.’ (ZF\_Conv13)
\z

Reflexive/reciprocal polysemy is not uncommon in languages, as both express that the agent of the action is simultaneously the patient. In the Bantu languages of zones H, K and R reciprocal and reflexive are expressed by the same pre-stem morpheme (\citealt{SchadebergBostoen2019}: 183). Outside these zones, many Bantu languages use a reflex of the reciprocal *-an to express reciprocal meaning. In Fwe, this suffix is all but gone, though speakers can still produce forms with \textit{\nobreakdash-an} when prompted (see \sectref{bkm:Ref492040534}). When necessary, speakers can differentiate the reciprocal and reflexive meanings of the prefix \textit{rí-/kí-} by adding the emphatic reflexive \textit{\-íni} (see (\ref{bkm:Ref99093781}--\ref{bkm:Ref99093782})).

The reflexive prefix \textit{kí-/rí-} is similar to object markers in a number of ways. The reflexive and object markers make use of the same slot in the verb, directly before the verb root. Like most object markers, the reflexive prefix has a high tone, which is deleted in the same TAM constructions (see \sectref{bkm:Ref71539267} on melodic tone). This is illustrated in (\ref{bkm:Ref445989915}--\ref{bkm:Ref445989917}), which show that the high tone of the object marker and the high tone of the reflexive prefix are maintained in the infinitive, but deleted in the present, a construction which deletes underlying high tones.

\ea
\label{bkm:Ref445989915}
\glll kùbáshàkà\\
ku-bá-shak-a\\
\textsc{inf}-\textsc{om}\textsubscript{2}-love-\textsc{fv}\\
\glt ‘to love them’
\z

\ea
\glll ndìbàshákà\\
ndi-ba\textsubscript{H}-shak-á̲\\
\textsc{sm}\textsubscript{1SG}-\textsc{om}\textsubscript{2}-love-\textsc{fv}\\
\glt ‘I love them.’
\z

\ea
\glll kùríshàkà\\
ku-rí-shak-a\\
\textsc{inf}-\textsc{refl}-love-\textsc{fv}\\
\glt ‘to love each other’
\z

\ea
\label{bkm:Ref445989917}
\glll tùrìshákà\\
tu-ri\textsubscript{H}-shak-á̲\\
\textsc{sm}\textsubscript{1PL}-\textsc{refl}-love-\textsc{fv}\\
\glt ‘We love each other.’
\z

Like object markers, the reflexive can co-occur with another object marker in ditransitive verbs, as in (\ref{bkm:Ref99094122}).

\ea
\label{bkm:Ref99094122}
bàcìrìshúmínìnìtè mwívùmò\\
\gll ba-ci\textsubscript{H}-ri\textsubscript{H}-shumí̲n-in-ite    mú-e-∅-vumo\\
\textsc{sm}\textsubscript{2}-\textsc{om}\textsubscript{7}-\textsc{refl}-tie-\textsc{appl}-\textsc{stat}  \textsc{np}\textsubscript{18}-\textsc{aug}-\textsc{np}\textsubscript{5}-stomach\\
\glt ‘He has tied it around his waist.’ (NF\_Narr17)
\z
\section{Locative marking}
\label{bkm:Ref451257563}\hypertarget{Toc75352676}{}
Reference to a location can be marked on the verb through locative clitics, which correspond to the three locative noun classes: =\textit{ho} for class 16, =\textit{ko} for class 17, and =\textit{mo} for class 18. All three locative clitics are underlyingly toneless; they surface as low-toned, unless a high melodic tone is assigned by the TAM construction. A detailed study of locative clitics in Fwe is presented in {\citet{Gunnink2017}}.

The locative clitic is the last morpheme in the verb: it appears after derivational suffixes, such as the applicative suffix \textit{-ir} in (\ref{bkm:Ref448143628}), and after inflectional suffixes, such as the habitual \textit{-ang} and the final vowel suffix \textit{-a} in (\ref{bkm:Ref451261148}).

\ea
\label{bkm:Ref448143628}
\glll ndìfùtàtìrákò\\
ndi-fu\textsubscript{H}tat-ir-a=kó̲\\
\textsc{sm}\textsubscript{1SG}-turn\_back-\textsc{appl}-\textsc{fv}=\textsc{loc}\textsubscript{17}\\
\glt ‘I turn my back towards it.’
\z

\ea
\label{bkm:Ref451261148}
\glll kàtùnákùzíbìkàngàkò\\
ka-tu-náku-zí-bik-ang-a=ko\\
\textsc{pst}.\textsc{ipfv}-\textsc{sm}\textsubscript{1PL}-\textsc{hab}-\textsc{om}\textsubscript{1PL}put-\textsc{hab}-\textsc{fv}=\textsc{loc}\textsubscript{17}\\
\glt ‘We usually put them there.’ (NF\_Elic15)
\z

When used with a reduplicated verb stem, as in (\ref{bkm:Ref99094143}), the locative clitic is not reduplicated, even though the verb stem is reduplicated together with its inflectional suffixes, providing further evidence for its clitic status.

\ea
\label{bkm:Ref99094143}
\glll ndàyèndíyèndìkò\\
ndi-a-endí̲-end-i=ko\\
\textsc{sm}\textsubscript{1SG}-\textsc{pst}-\textsc{pl}2-go-\textsc{pst}=\textsc{loc}\textsubscript{17}\\
\glt ‘I kept going there.’ (NF\_Elic15)
\z

Phonologically, the locative clitic is fully integrated into the verb to which it attaches. Locative clitics influence the placement of melodic tone and penultimate lengthening. In the present construction, for instance, a melodic tone is assigned to the final mora of the verb, which retracts to the preceding mora in phrase-final position. The examples in (\ref{bkm:Ref505879591}) and (\ref{exampleiknockonit}) show that in determining the penultimate syllable, the locative clitic is also counted.

\ea
\label{bkm:Ref505879591}
\glll ndìngòngótà\\
ndi-ngo\textsubscript{H}ngot-á̲\\
\textsc{sm}\textsubscript{1SG}-knock-\textsc{fv}\\
\glt ‘I knock.’
\z

\ea
\label{exampleiknockonit}
\glll ndìngòngòtáhò\\
ndi-ngo\textsubscript{H}ngot-a=hó̲\\
\textsc{sm}\textsubscript{1SG}-knock-\textsc{fv}=\textsc{loc}\textsubscript{16}\\
\glt ‘I knock on it.’ (NF\_Elic15)
\z

Locative clitics are never used for referring to a locative noun phrase in the same clause, but only to locations that are introduced in the earlier discourse. An example is given in (\ref{bkm:Ref448144029}), an utterance consisting of two clauses, each with their own inflected verb. The noun \textit{cì-pùrà} ‘chair’ is introduced in the first clause, and the verb of the second clause uses a locative clitic =\textit{ho} to refer back to it.

\ea
\label{bkm:Ref448144029}
mùbàhé cìpùrà bàkáréhò\\
\gll mu-ba\textsubscript{H}-ha-é̲    ci-pura  ba-kar-e=hó̲\\
\textsc{sm}\textsubscript{2PL}-\textsc{om}\textsubscript{2}-give-\textsc{pfv}.\textsc{sbjv}  \textsc{np}\textsubscript{7}-chair  \textsc{sm}\textsubscript{2}-sit-\textsc{pfv}.\textsc{sbjv}=\textsc{loc}\textsubscript{16}\\
\glt ‘Give her a chair, so she may sit on it.’ (NF\_Elic15)
\z

The three locative clitics each have their own semantics. The class 16 locative clitic =\textit{ho} is used to refer to movement away from, as in (\ref{bkm:Ref446065313}), a location on, as in (\ref{bkm:Ref446065314}), or a more general location, as in (\ref{bkm:Ref446065854}).

\ea
\label{bkm:Ref446065313}
ènzâsì zàkùrí kùǀásàùkàhò\\
\gll e-N-zási    zi-aku-rí    ku-ǀás-a-uk-a=ho\\
\textsc{aug}-\textsc{np}\textsubscript{10}-spark  \textsc{sm}\textsubscript{10}-\textsc{npst}.\textsc{ipfv}-be  \textsc{inf}-sparkle-\textsc{pl}1-\textsc{sep}.\textsc{intr}-\textsc{fv}=\textsc{loc}\textsubscript{16}\\
\glt ‘Sparks were flying from it.’
\z

\ea
\label{bkm:Ref446065314}
\glll ndàngóngòtìhò\\
ndi-a-ngóngot-i=ho\\
\textsc{sm}\textsubscript{1SG}-\textsc{pst}-knock-\textsc{pst}=\textsc{loc}\textsubscript{16}\\
\glt ‘I knocked on it.’
\z

\ea
\label{bkm:Ref446065854}
\glll tàbènáhò\\
ta-ba-ina=hó̲\\
\textsc{neg}-\textsc{sm}\textsubscript{2}-be\_at=\textsc{loc}\textsubscript{16}\\
\glt ‘She is not here.’ (NF\_Elic15)
\z

The class 17 locative clitic =\textit{ko}, is used to refer to a direction, as in (\ref{bkm:Ref446065865}), or to a general location, as in (\ref{bkm:Ref446065874}).

\ea
\label{bkm:Ref446065865}
\glll kàtóndìkò\\
ka-a-tónd-i=ko\\
\textsc{neg}-\textsc{sm}\textsubscript{1}-look-\textsc{neg}=\textsc{loc}\textsubscript{17}\\
\glt ‘She doesn’t look that way.’ (NF\_Narr15)
\z

\ea
\label{bkm:Ref446065874}
\glll kàndíhàràngákò\\
ka-ndí̲-ha\textsubscript{H}r-ang-a=kó̲\\
\textsc{pst}.\textsc{ipfv}-\textsc{sm}\textsubscript{1SG}-live-\textsc{hab}-\textsc{fv}=\textsc{loc}\textsubscript{17}\\
\glt ‘I used to live there.’ (NF\_Elic15)
\z

The class 18 locative clitic =\textit{mo}, is used to refer to a location inside, as in (\ref{bkm:Ref99094232}), or to a movement away from inside, as in (\ref{bkm:Ref99094242}).

\ea
\label{bkm:Ref99094232}
yènkéː náàkàráːràmò\\
\gll ye-nkéː  ná̲-a-a-ka-ráːr-a=mo\\
\textsc{np}\textsubscript{1}-one  \textsc{pst}-\textsc{sm}\textsubscript{1}-\textsc{dist}-sleep-\textsc{fv}=\textsc{loc}\textsubscript{18}\\
\glt ‘He slept alone in there.’
\z

\ea
\label{bkm:Ref99094242}
\glll àkùbútùkàmò\\
a-aku-bútuk-a=mo\\
\textsc{sm}\textsubscript{1}-\textsc{npst}.\textsc{ipfv}-run-\textsc{fv}=\textsc{loc}\textsubscript{18}\\
\glt ‘He ran out of it.’ (NF\_Narr15)
\z

In addition to their locative function, locative clitics can also be used with a partitive function. This has also been noted for a number of other Bantu languages, including Bemba (\citealt{MartenKula2014}), Kanincin \citep{DevosEtAl2010}, and others (\citealt{PersohnDevos2017}). In Fwe, all three locative clitics can have a partitive interpretation. The partitive use of the class 16 clitic =\textit{ho} is illustrated in (\ref{bkm:Ref494450771}), indicating that the speaker did not sell all the cattle, but only some of them. In (\ref{bkm:Ref494450803}), the class 17 clitic \textit{=ko} is used to indicate that only a part of the intended salary is given, not the whole amount. In (\ref{bkm:Ref494450773}), the class 18 clitic \textit{=mo} is used to stress that the addressee should take some, not everything.

\ea
\label{bkm:Ref494450771}
zòbírè bùryó nìndáùrìsáhò\\
\gll ∅-zi-o=bíre    bu-ryó  ni-ndí̲-a-ur-is-a=hó̲\\
\textsc{cop}-\textsc{pp}\textsubscript{10}-\textsc{con}=two  \textsc{np}\textsubscript{14}-only  \textsc{rem}-\textsc{sm}\textsubscript{1SG}-\textsc{pst}-buy-\textsc{caus}-\textsc{fv}=\textsc{loc}\textsubscript{16}\\
\glt ‘It is only two of them that I sold.’ (Answer to: ‘Did you sell all the cattle?’) (NF\_Elic15)
\z

\ea
\label{bkm:Ref494450803}
bàshìkùhàkó àkàháfù\\
\gll ba-shi\textsubscript{H}-ku-ha\textsubscript{H}-a=kó̲    a-ka-hafú\\
\textsc{sm}\textsubscript{2}-\textsc{per}-\textsc{om}\textsubscript{2SG}-give-\textsc{fv}=\textsc{loc}\textsubscript{17}  \textsc{aug}-\textsc{np}\textsubscript{12}-half\\
\glt ‘They still only give you half of it.’ (ZF\_Conv13)
\z

\ea
\label{bkm:Ref494450773}
\glll hìndèmó kànînì òsìyìrèmó bámwì\\
hind-e=mó̲      ka-níni   o-si\textsubscript{H}-ir-e=mó̲        ba-mwí\\
take-\textsc{pfv}.\textsc{sbjv}=\textsc{loc}\textsubscript{18}  \textsc{adv}-little
 \textsc{sm}\textsubscript{2SG}-leave-\textsc{appl}-\textsc{pfv}.\textsc{sbjv}=\textsc{loc}\textsubscript{18} \textsc{pp}\textsubscript{2}-other\\
\glt ‘Take a little bit from it, leave some for the others.’ (NF\_Elic17)
\z

The class 17 locative clitic has an additional function of marking a polite request, as in (\ref{bkm:Ref492041509}). This function is also seen with the class 17 nominal prefix (see \sectref{bkm:Ref452049189} for examples).

\ea
\label{bkm:Ref492041509}
ndìshàká kùkàrìmàkò ècìŋórìsó ꜝcákò\\
\gll ndi-shak-á̲    ku-karim-a=ko    e-ci-ŋórisó    ci-akó\\
\textsc{sm}\textsubscript{1SG}-want-\textsc{fv}  \textsc{inf}-borrow-\textsc{fv}=\textsc{loc}\textsubscript{17}  \textsc{aug}-\textsc{np}\textsubscript{7}-pen  \textsc{pp}\textsubscript{7}-\textsc{poss}\textsubscript{2SG}\\
\glt ‘I want to borrow your pen, please.’ (NF\_Elic15)
\z

The locative clitic of class 17 may also be used on the progressive auxiliary \textit{kwesi}, or the locative clitic of class 16 on the progressive auxiliary \textit{ina,} to express focus on the progressive aspect; examples are given in \sectref{bkm:Ref431917333} on the progressive.

