\chapter{Useful phrases}
\hypertarget{Toc75352721}{}
This appendix contains a number of phrases that can be useful when communicating with Fwe speakers. A learner’s grammar or handbook of Fwe has, to my knowledge, never been made. Although the purpose of the current grammar is not the instruction of those who intend to learn Fwe as a second language, it is nonetheless hoped that the remarks made here can be of use.


When greeting Fwe speakers, non-verbal communication is as important as verbal communication. A practice that is widely spread across Western Zambia and the Zambezi region involves repeatedly clapping the hands, as a sign of respect. A typical greeting consists of clapping the hands once or twice, shaking the other person’s hand, and clapping the hands again. This process is repeated, depending on the relative importance of the participants, and the degree of respect that is due. Even more respect is expressed by bending the knees.

The morning greeting is \textit{mbùtí mwàbûːkì}, literally ‘how did you wake up?’, comparable to English ‘good morning’. It can be shortened to \textit{mwàbûːkì}.


\ea
mbùtí mwàbûːkì\\
\gll N-bu-tí    mu-a-búːk-i\\
\textsc{cop}{}-\textsc{np}\textsubscript{14}{}-how  \textsc{sm}\textsubscript{2PL}{}-\textsc{pst}{}-wake-\textsc{npst}.\textsc{pfv}\\
\glt ‘Good morning.’ (Lit. ‘How did you wake up?’)
\z

\ea
mwàbûːkì\\
\gll mu-a-búːk-i\\
\textsc{sm}\textsubscript{2PL}{}-\textsc{pst}{}-wake-\textsc{npst}.\textsc{pfv} \\
\glt ‘Good morning.’ (Lit. ‘Did you wake up?’)
\z



The answer to the morning greeting is \textit{twàbúːkì nênjà}, literally ‘we woke up well’, comparable to English good morning. It can be shortened to \textit{twàbûːkì}, or to \textit{nênjà}.


\ea
twàbúːkì nênjà\\
\gll tu-a-búːk-i      nénja\\
\textsc{sm}\textsubscript{1PL}{}-\textsc{pst}{}-wake-\textsc{npst}.\textsc{pfv}  well\\
\glt ‘Good morning.’ (Lit. ‘We woke up well.’)
\z


\ea
\glll twàbûːkì  \\
tu-a-búːk-i\\
\textsc{sm}\textsubscript{1PL}{}-\textsc{pst}{}-wake-\textsc{npst}.\textsc{pfv}\\
\glt ‘Good morning.’ (Lit. ‘We woke up.’)
\z


\ea
\glll nênjà\\
nénja\\
well\\
\glt ‘[We woke up] well.’
\z


Morning greetings are appropriate to about midday. From midday onwards, a different greeting is used, \textit{mbùtí mwàríꜝshárì}, comparable to English ‘good afternoon’, though with a literal meaning ‘how have you stayed?’. As with the morning greeting, \textit{mbùtí} can be left out.


\ea
mbùtí mwàríꜝshárì\\
\gll N-bu-tí    mu-a-rí-shar-í̲\\
\textsc{cop}{}-\textsc{np}\textsubscript{14}{}-how  \textsc{sm}\textsubscript{2PL}{}-\textsc{pst}{}-stay-\textsc{npst}.\textsc{pfv}\\
\glt ‘Good afternoon.’ (Lit. ‘How have you stayed?’)
\z


\ea
\glll mwàríꜝshárì\\
mu-a-rí-shar-í̲\\
\textsc{sm}\textsubscript{2PL}{}-\textsc{pst}{}-stay-\textsc{npst}.\textsc{pfv}\\
\glt ‘Good afternoon.’ (Lit. ‘Have you stayed?’)
\z


The answer to the afternoon greeting is \textit{twàríshàrí nênjà}, which can be shortened to \textit{twàríꜝ}\textit{shárì}. A correct response to the afternoon greeting is also \textit{nênjà}.


\ea
twàríshàrí nênjà\\
\gll tu-a-rí-shar-í̲    nénja\\
\textsc{sm}\textsubscript{1PL}{}-\textsc{pst}{}-stay-\textsc{npst}.\textsc{pfv}  well\\
\glt ‘Good afternoon.’ (Lit. ‘We’ve stayed well.’)
\z


\ea
\glll twàríꜝshárì\\
tu-a-rí-shar-í̲\\
\textsc{sm}\textsubscript{1PL}{}-\textsc{pst}{}-stay-\textsc{npst}.\textsc{pfv}\\
\glt ‘Good afternoon.’ (Lit. ‘We’ve stayed.’)
\z


\ea
\glll nênjà\\
nénja\\
well\\
\glt ‘[We’ve stayed] well.’
\z


Afternoon greetings are appropriate from midday until the end of the day. All greetings are reciprocal; after the first participants has asked after the well-being of the second, the second inquires after the well-being of the first.

Like greeting, thanking involves non-verbal expressions of respect such as (repeated) clapping, handshaking, and bowing, depending on the level of respect and gratitude one wishes to express. There is a Namibian and a Zambian variant, one with \textit{kí-} using the form of the reflexive prefix as it is used in Zambina Fwe, and one with \textit{rí-} using the form of the reflexive prefix as it is used in Namibian Fwe.


\ea
Namibian Fwe\\
\glll twàrítùmêrì\\
tu-a-rí-tumé̲r-i\\
\textsc{sm}\textsubscript{1PL}{}-\textsc{pst}{}-\textsc{refl}{}-thank-\textsc{npst}.\textsc{pfv}\\
\glt ‘Thank you.’
\z

\ea
Zambian Fwe\\
\glll twàkítùmêrì\\
tu-a-kí-tumé̲r-i\\
\textsc{sm}\textsubscript{1PL}{}-\textsc{pst}{}-\textsc{refl}{}-thank-\textsc{npst}.\textsc{pfv}\\
\glt ‘Thank you.’
\z



The expression for thanking can take a first person plural subject marker, or, less commonly, a first person singular subject marker, \textit{ndàrítùmêrì / ndàkítùmêrì}.

The verb \textit{tùmèlà} is not of Fwe origin, as the lack of vowel and nasal harmony in the putative applicative suffix \textit{{}-el} show. It is evidently borrowed from the Lozi verb \textit{ku itumela} ‘be thankful’, which is inflected as \textit{ni itumezi} to mean ‘thank you’ \citep{Burger1960}.

As in many African/Bantu languages, the expressions for goodbye depend on who stays and who goes. To bid farewell to someone who leaves, the person who stays says \textit{mùyéndè nênjà}, literally ‘go well’. The person who leaves bids farewell to the person who stays with \textit{mùsìyàré nênjà} ‘stay well’.


\ea
mùyéndè nênjà\\
\gll mu-é̲nd-e    nénja\\
\textsc{sm}\textsubscript{2PL}{}-go-\textsc{pfv}.\textsc{sbjv}  well\\
\glt ‘Goodbye (said to someone who leaves).’
\z

\largerpage[2]
\ea
mùsìyàré nênjà\\
\gll mu-siar-é̲    nénja\\
\textsc{sm}\textsubscript{2PL}{}-stay-\textsc{pfv}.\textsc{sbjv}  well\\
\glt ‘Goodbye (said to someone who stays).’
\z



