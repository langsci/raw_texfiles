\chapter{Syntax and information structure}
\label{bkm:Ref99093567}\hypertarget{Toc75352711}{}
Various issues in the syntax of Fwe have already been discussed in previous chapters: the marking of subjects and (multiple) objects in Chapter \ref{bkm:Ref99105339}, the syntactic behavior of arguments introduced by the causative or applicative derivation in Chapter \ref{bkm:Ref99105347}, the use of copulative prefixes to mark non-verbal predication in \sectref{bkm:Ref489963307}, to name a few. This chapter discusses remaining issues in the syntax of Fwe. \sectref{bkm:Ref492315922} discusses the canonical word order in Fwe, and Sections \ref{bkm:Ref403656711} and \ref{bkm:Ref452042958} discuss pragmatically motivated derivations from this order. In \sectref{bkm:Ref491361275} locative inversion is discussed, which involves the use of a locative constituent as a syntactic subject. \sectref{bkm:Ref75346027} discusses a number of dependent clause types, including relative clauses. \sectref{bkm:Ref491333435} discusses cleft constructions, which combine nominal predication with a relative clause to mark constituent focus.

\section{Canonical word order}
\label{bkm:Ref492315922}\label{bkm:Ref492308543}\hypertarget{Toc75352712}{}\label{bkm:Ref492317959}
Constituent order in Fwe depends on three factors; the syntactic function of the constituent, that is if it functions as a subject, object, (inflected) verb, or a locative adjunct or adverb; the information structural properties of the constituent, whether it is in focus, topicalized, or marked for definiteness; and the clause type, either main or subordinate. The canonical, unmarked order of constituents in a main clause in Fwe is SVO, as illustrated in (\ref{bkm:Ref449629903}); note that, while such clauses can easily by elicited, in actual discourse it is likely for the subject, the object, or both to be expressed pronominally rather than as as nominal constituents.

\ea
\label{bkm:Ref449629903}
òmùsá nàhíbí ènjìngà yángù\\
\glll o-mu-sá    na-hib-í̲      e-N-jinga    i-angú\\
\textsc{aug}-\textsc{np}\textsubscript{1}-thief  \textsc{sm}\textsubscript{1}.\textsc{pst}-steal-\textsc{npst}.\textsc{pfv}  \textsc{aug}-\textsc{np}\textsubscript{9}-bicycle  \textsc{pp}\textsubscript{9}-\textsc{poss}\textsubscript{1SG}\\
{}[Subject]    [Verb]        {[    Object  ]}\\
\glt ‘A thief has stolen my bicycle.’ (NF\_Elic15)
\z

SVO order is used for sentences that are unmarked with respect to information structure; neither of the constituents in a sentence with SVO order is overtly marked for either topic or focus. Constituents may move out of their canonical position to the left periphery of the sentence, in order to be marked as topic, or the right periphery of the sentence, in order to be marked for definiteness. These processes of left dislocation and right dislocation are discussed in the following sections.

\section{Left dislocation}
\label{bkm:Ref403656711}\hypertarget{Toc75352713}{}
Constituents can be moved out of their canonical position to the beginning of the clause, in which case they are morphologically and prosodically marked as a separate phrase. The prosodic marking of left dislocation is most clearly seen by the application of phrase-final tonal processes, namely the realization of underlying high tones as falling and the shift of final high tones to the penultimate mora (see \sectref{bkm:Ref445214894} on tonal processes), for instance, the final falling tone in the dislocated subject constituent in (\ref{bkm:Ref505778244}). The morphological marking of left dislocation is only seen on dislocated constituents that function as an object or locative adjunct, in which case the dislocated constituent needs to be cross-referenced by an object marker, as in (\ref{bkm:Ref505778246}), or locative clitic, as in (\ref{bkm:Ref505778247}).

\ea
\label{bkm:Ref505778244}
\textbf{àá} \textbf{màyîː} àbórêtè\\
\gll a-á    ma-yíː    a-bor-é̲te\\
\textsc{aug}-\textsc{dem}.\textsc{i}\textsubscript{6}  \textsc{np}\textsubscript{6}-egg  \textsc{sm}\textsubscript{6}-rot-\textsc{stat}\\
\glt ‘These eggs, they’re rotten.’
\z

\ea
\label{bkm:Ref505778246}
\textbf{òzyú} \textbf{múꜝ}\textbf{kwámè} kàndìmùzyîː\\
\gll o-zyú    mú-kwamé  ka-ndi-mu-zyi\textsubscript{H}-í̲\\
\textsc{aug}-\textsc{dem}.\textsc{i}\textsubscript{1}  \textsc{np}\textsubscript{1}-man  \textsc{neg}-\textsc{sm}\textsubscript{1SG}-\textsc{om}\textsubscript{1}-know.\textsc{stat}-\textsc{neg}\\
\glt ‘This man, I don’t know him.’
\z

\ea
\label{bkm:Ref505778247}
\textbf{mòwíꜝ}\textbf{n’} \textbf{ómùnzì} ndáyꜝámò\\
\gll mu-o-winá    o-mu-nzi    ndi-á̲-y-a=mó̲\\
\textsc{np}\textsubscript{18}-\textsc{aug}-\textsc{dem}.\textsc{iv}\textsubscript{3}  \textsc{aug}-\textsc{np}\textsubscript{3}-village  \textsc{sm}\textsubscript{1SG}-\textsc{pst}-go-\textsc{fv}=\textsc{loc}\textsubscript{18}\\
\glt ‘That village, I’ve been there.’ (NF\_Elic15)
\z

As the canonical position for the subject can be the preverbal position, not all subjects appearing before a verb are dislocated. This is only the case when a subject constituent at the left edge of a sentence is affected by phrase-final tone rules. Pre-verbal subjects that are not affected by these phrase-final processes are not left-dislocated, but remain in situ; this is illustrated in (\ref{bkm:Ref449624390}), where the subject constituent \textit{bàmùrútí} ‘teachers’ is not affected by the phrase-final tone process of H retraction, showing that it is not dislocated. Compare with (\ref{bkm:Ref505778244}) above, where phrase-final processes do affect the left-dislocated subject constituent \textit{àá màyîː} ‘these eggs’.

\ea
\label{bkm:Ref449624390}
bàmùrútí bàbùtúkà\\
\gll ba-mu-rutí    ba-bu\textsubscript{H}tuk-á̲\\
\textsc{np}\textsubscript{2}-\textsc{np}\textsubscript{1}-teacher  \textsc{sm}\textsubscript{2}-run-\textsc{fv}\\
\glt ‘The teachers run.’ (NF\_Elic15)
\z

Constituents are dislocated to the left periphery of the sentence in order to function as a topic, the referent that a sentence is “about” \citep[114]{Lambrecht1994}, the old information, given through physical or linguistic context, to which the speaker intends to add new information. In (\ref{bkm:Ref449625150}), the left-dislocated constituent \textit{òzyú mwâncè} ‘this child’ functions as the topic; as it refers to a child who is present at the time, it is known to the discourse through the immediate physical surrounding and as such functions as a topic for the rest of the utterance.

\ea
\label{bkm:Ref449625150}
òzyú mwâncè mùmùtwárè kùcìpátêrà\\
\gll o-zyú    mu-ánce  mu-mu-twá̲r-e    ku-ci-patéra\\
\textsc{aug}-\textsc{dem}.\textsc{i}\textsubscript{1}  \textsc{np}\textsubscript{1}-child  \textsc{sm}\textsubscript{2PL}-\textsc{om}\textsubscript{1}-carry-\textsc{pfv}.\textsc{sbjv}  \textsc{np}\textsubscript{17}-\textsc{np}\textsubscript{7}-hospital\\
\glt ‘This child, take her/him to the hospital.’ (ZF\_Elic14)
\z

Another example of the use of left dislocation for topicalization is given in (\ref{bkm:Ref449625470}), which is the beginning of a story. In the first sentence, the referent \textit{òmfûmù} ‘a rich man’ is introduced. In the second, this same referent is marked as a topic by left-dislocation; it serves as the old information to which the sentence contributes new facts.

\ea
\label{bkm:Ref449625470}
kàrê kàkwín’ ꜝómfûmù\\
\gll ka-réː    ka-ku-iná    o-∅-mfúmu\\
\textsc{adv}-long  \textsc{pst}.\textsc{ipfv}-be\_at  \textsc{aug}-\textsc{np}\textsubscript{1a}-rich\_man\\
\glt ‘Long ago, there was a rich man.’
\z

\ea
òmfûmù bàmùkúwè mùrènà\\
\gll o-∅-mfúmu    ba-mu-kú̲-e      mu-rena\\
\textsc{aug}-\textsc{np}\textsubscript{1a}-rich\_man  \textsc{sm}\textsubscript{2}-\textsc{om}\textsubscript{1}-call-\textsc{pfv}.\textsc{sbjv}  \textsc{np}\textsubscript{1}-chief\\
\glt ‘The rich man, they would call him chief.’ (NF\_Narr15)
\z

Left-dislocation can be used to mark a contrastive topic; when various referents are accessible, the speaker can choose to pick out a single referent to the exclusion of others. (\ref{bkm:Ref449689647}) and (\ref{bkm:Ref449689652}) are taken from a conversation in which speakers discuss their views on marriage; in (\ref{bkm:Ref449689647}), the first speaker gives his view, and in (\ref{bkm:Ref449689652}), the second speakers gives his own, contrastive view, using the personal pronoun \textit{me} ‘I’, in the left-dislocated position to mark a contrastive topic.

\ea
\label{bkm:Ref449689647}
ndìbwènè mbóbùmángò òkùshéshà òmùkéntù òzyú tàkìtùtìtêː\\
\gll ndi-bwene    mbó-bu-mángo  o-ku-shésh-a    o-mu-kéntu o-zyú    ta-a-kitut-ite-í̲\\
\textsc{sm}\textsubscript{1SG}-see.\textsc{stat}  \textsc{cop}.\textsc{def}\textsubscript{14}-\textsc{np}\textsubscript{14}-bad  \textsc{aug}-\textsc{inf}-marry-\textsc{fv}  \textsc{aug}-\textsc{np}\textsubscript{1}-woman
\textsc{aug}-\textsc{dem}.\textsc{i}\textsubscript{1}  \textsc{neg}.\textsc{sm}\textsubscript{1}-be\_educated-\textsc{stat}.\textsc{neg}\\
\glt ‘I think that it is bad to marry an uneducated woman.’
\z

\ea
\label{bkm:Ref449689652}
kònó mè òbùrótù òbò ndíbwènè òkùshéshà òmùkéntù zyù tàkìtùtìtêː\\
\gll konó   mè    o-bu-rótu    o-bo      ndí-bwene o-ku-shésh-a    o-mu-kéntu    zyu  ta-a-kitut-ite-í̲\\
but  \textsc{pers}\textsubscript{1SG}  \textsc{aug}-\textsc{np}\textsubscript{14}-good  \textsc{aug}-\textsc{dem}.\textsc{iii}\textsubscript{14} \textsc{sm}\textsubscript{1SG}.\textsc{rel}-see.\textsc{stat}
\textsc{aug}-\textsc{inf}-marry-\textsc{fv}  \textsc{aug}-\textsc{np}\textsubscript{1}-woman  \textsc{dem}.\textsc{i}\textsubscript{1}  \textsc{neg}.\textsc{sm}\textsubscript{1}-be\_educated-\textsc{stat}-\textsc{neg}\\
\glt ‘But me, I think that it is good to marry an uneducated woman.’ (ZF\_Conv13)
\z
\section{Right dislocation}
\label{bkm:Ref452042958}\hypertarget{Toc75352714}{}
Constituents can also be moved out of their canonical position to the right edge of the clause. Right dislocation resembles left dislocation in that dislocated objects and locative adjuncts require cross-referencing on the main clause verb, as in (\ref{bkm:Ref491438233}--\ref{bkm:Ref449630026}), where the dislocated constituent is marked in bold. Right-dislocation may also target subjects, as in (\ref{bkm:Ref505783997}).

\ea
\label{bkm:Ref491438233}
ndìrùshákà \textbf{òrú} \textbf{rùzyîmbò}\\
\gll ndi-ru\textsubscript{H}-shak-á̲  o-rú    ru-zyímbo\\
\textsc{sm}\textsubscript{1SG}-\textsc{om}\textsubscript{11}-like-\textsc{fv}  \textsc{aug}-\textsc{dem}.\textsc{i}\textsubscript{11}  \textsc{np}\textsubscript{11}-song\\
\glt ‘I like this song.’
\z

\ea
\label{bkm:Ref449630026}
ndáꜝyámò \textbf{mòwín’} \textbf{ꜝ}\textbf{ómùnzì}\\
\gll ndí̲-a-ya=mó̲      mo-winá  o-mu-nzi\\
\textsc{sm}\textsubscript{1SG}-\textsc{pst}-go-\textsc{fv}=\textsc{loc}\textsubscript{18} \textsc{np}\textsubscript{18}-\textsc{dem}.\textsc{iv}\textsubscript{3}  \textsc{aug}-\textsc{np}\textsubscript{3}-village\\
\glt ‘I’ve been to that village.’ (NF\_Elic15)
\z

\ea
\label{bkm:Ref505783997}
shìbáꜝnázyìbì \textbf{báꜝ}\textbf{múꜝ}\textbf{kwáꜝ}\textbf{mé} \textbf{ꜝ}\textbf{wénù}\\
\gll shi-bá-ná-zyib-i      bá-mú-kwámé  u-enú\\
\textsc{inc}-\textsc{sm}\textsubscript{2}-\textsc{pst}-know-\textsc{npst}.\textsc{pfv}  \textsc{np}\textsubscript{2}-\textsc{np}\textsubscript{1}-man    \textsc{pp}\textsubscript{1}-\textsc{poss}\textsubscript{2PL}\\
\glt ‘Your husband has now become aware.’ (NF\_Narr15)
\z

Right dislocation differs from left dislocation, however, in the phonological phrasing of the dislocated constituent. Whereas left-dislocated constituents are always followed by a prosodic boundary, a prosodic boundary preceding the right-dislocated constituent is optional. Examples of right-dislocated constituents that do function as a separate phrase are given in (\ref{bkm:Ref491438233}--\ref{bkm:Ref449630026}), as seen from the application of phrase-final tonal processes on the verb preceding the dislocated constituent. An example of a right-dislocated constituent which is not preceded by a prosodic boundary is given in (\ref{bkm:Ref492136669}), as seen from the lack of high tone retraction on the verb preceding the dislocated constituent.

\ea
\label{bkm:Ref492136669}
mùrùsháká \textbf{òrú} \textbf{rùzyîmbò}\\
\gll mu-ru\textsubscript{H}-shak-á̲  o-rú    ru-zyímbo\\
\textsc{sm}\textsubscript{2PL}-\textsc{om}\textsubscript{11}-like-\textsc{fv}  \textsc{aug}-\textsc{dem}.\textsc{i}\textsubscript{11}  \textsc{np}\textsubscript{11}-song\\
\glt ‘Do you like this song?’ (NF\_Elic15)
\z

The possible lack of a prosodic boundary between the verb and the right-dislocated object might suggest that the object is not dislocated, but occurs in situ, and that the use of the object marker in this context, which is otherwise obligatory only when objects are dislocated, indicates that Fwe allows object marking for agreement, e.g. object marking when an overt object noun is present in the clause. However, right dislocation may target subject and locative constituents as well as objects; for subjects and locatives, right-dislocation clearly involves movement out of the constituent’s canonical position, suggesting that objects are moved out of their canonical position as well, and that this explains the occurrence of the object marker.

Right dislocation marks constituents as definite. The notion of definiteness shows some overlap with the notion of topic, because both definite constituents and topic constituents are referents that are known to both the speaker and the hearer. They differ, however, in that a topic constituent is not only known, but also the constituent that the rest of the sentence is about, to which the sentence aims to contribute new information. A definite constituent, however, does not (necessarily) play this pivotal role. An example of a definite constituent that does not function as a topic is given in (\ref{bkm:Ref449629532}). The topic is the locative adjunct \textit{mùnjìrà kwécì cìkúnì} ‘along the path, at the tree’, which occurs in the sentence-initial topic position. The object noun \textit{ménò énù} ‘your teeth’, which occurs in the right-dislocated position as seen from the use of the object marker on the verb, is definite but does not function as a topic.

\ea
\label{bkm:Ref449629532}
mùnjìrà kwécì cìkúnì kókò ndàázìkì ménò énù\\
\gll mu-N-jira    kú-e-ci    ci-kuní  kó-ko ndi-a-á-zik-i        ma-íno  a-enú  \\
\textsc{np}\textsubscript{18}-\textsc{np}\textsubscript{9}-path  \textsc{np}\textsubscript{17}-\textsc{aug}-\textsc{dem}.\textsc{i}\textsubscript{7}  \textsc{np}\textsubscript{7}-tree  \textsc{cop}.\textsc{def}\textsubscript{17}-\textsc{dem}.\textsc{iii}\textsubscript{17}
\textsc{sm}\textsubscript{1SG}-\textsc{pst}-\textsc{om}\textsubscript{6}-hide-\textsc{npst}.\textsc{pfv}  \textsc{np}\textsubscript{6}-tooth  \textsc{pp}\textsubscript{6}-\textsc{poss}\textsubscript{2PL}\\
\glt ‘Along the path, at the tree, that’s where I’ve hidden your teeth.’ (NF\_Narr15)
\z

Subjects can be moved to the post-verbal position to be marked for definiteness. In (\ref{bkm:Ref449629756}), taken from a narrative about a lion, the lion has been mentioned frequently in the previous discourse and is therefore construed as definite.

\ea
shànàkàkárìhì òndávù\\
\gll sha-na-ka-kárih-i        o-∅-ndavú\\
\textsc{inc}-\textsc{sm}\textsubscript{1}.\textsc{pst}-\textsc{dist}-be\_angry-\textsc{npst}.\textsc{pfv}  \textsc{aug}-\textsc{np}\textsubscript{1a}-lion\\
\glt ‘The lion was now very angry.’ (NF\_Narr15)\label{bkm:Ref449629756}
\z

Right-dislocation can also affect inherently definite constituents, such as personal pronouns, as in (\ref{bkm:Ref449690536}--\ref{bkm:Ref449690539}), nouns modified by a demonstrative, as in (\ref{bkm:Ref449690548}), and proper names, as in (\ref{bkm:Ref449690555}).

\ea
\label{bkm:Ref449690536}
rímwì zyûbà kàrì nèmúbûːk’ ꜝénwè\\
\gll rí-mwi  ∅-zyúba  ka-ri    ne-mú̲-bú̲ːk-e    enwé\\
\textsc{pp}\textsubscript{5}-other  \textsc{np}\textsubscript{5}-day  \textsc{neg}-be  \textsc{rem}-\textsc{sm}\textsubscript{2PL}-wake-\textsc{pfv}.\textsc{sbjv}  \textsc{pers}\textsubscript{2PL}\\
\glt ‘One day you are not going to wake up.’ (NF\_Narr15)
\z

\ea
\label{bkm:Ref449690539}
èyí nyàmà kàtwíyírí swè\\
\gll e-í    N-nyama  ka-tu-í-ri-i      eswé\\
\textsc{aug}-\textsc{dem}.\textsc{i}\textsubscript{9}  \textsc{np}\textsubscript{9}-meat  \textsc{neg}-\textsc{sm}\textsubscript{1PL}-\textsc{om}\textsubscript{9}-eat-\textsc{neg}  \textsc{pers}\textsubscript{1PL}\\
\glt ‘This meat, we don’t eat it.’ (NF\_Elic15)
\z

\ea
\label{bkm:Ref449690548}
ndókùrídàmà èryó zyôkà\\
\gll ndi-ó=ku-rí-dam-a      e-ryó    ∅-zyóka\\
\textsc{pp}\textsubscript{1SG}-\textsc{con}=\textsc{inf}-\textsc{om}\textsubscript{5}-beat-\textsc{fv}  \textsc{aug}-\textsc{dem}.\textsc{iii}\textsubscript{5}  \textsc{np}\textsubscript{5}-snake\\
\glt ‘Then I beat that snake.’ (ZF\_Narr13)
\z

\ea
\label{bkm:Ref449690555}
mbàndíbànánúnè bàhènì\\
\gll mba-ndí̲-ba\textsubscript{H}-nanú̲n-e      ba-heni\\
\textsc{near}.\textsc{fut}-\textsc{sm}\textsubscript{1SG}-\textsc{om}\textsubscript{2}-lift-\textsc{pfv}.\textsc{sbjv}  \textsc{np}\textsubscript{2}-Hennie\\
\glt ‘I will lift up Mr. Hennie.’ (ZF\_Elic14)
\z

Although right-dislocated constituents are always definite, a constituent that is not right-dislocated is not necessarily indefinite. An example of a definite noun phrase used in the pre-verbal position is given in (\ref{bkm:Ref449950453}), and an example of a definite noun phrase (describing a hoe that was mentioned earlier in the discourse) that is post-verbal but not dislocated, as seen from the lack of object marker, is given in (\ref{bkm:Ref449950683}).

\ea
\label{bkm:Ref449950453}
ècí cìkùnì cìrìbórérá bùryô\\
\gll e-cí    ci-kuni  ci-ri\textsubscript{H}-bor-er-á̲    bu-ryó\\
\textsc{aug}-\textsc{dem}.\textsc{i}\textsubscript{7}  \textsc{np}\textsubscript{7}-tree  \textsc{sm}\textsubscript{7}-\textsc{refl}-rot-\textsc{appl}-\textsc{fv}  \textsc{np}\textsubscript{14}-only\\
\glt ‘This wood rots easily.’ (NF\_Elic15)
\z

\ea
\label{bkm:Ref449950683}
kàshùrwè ákùdánsìká èhàmbà\\
\gll ka-shurwe  a-ó=ku-dánsik-á    e-∅-amba\\
\textsc{np}\textsubscript{12}-rabbit  \textsc{pp}\textsubscript{1}-\textsc{con}=\textsc{inf}-drop-\textsc{fv}  \textsc{aug}-\textsc{np}\textsubscript{5}-hoe\\
\glt ‘The rabbit drops the hoe.’ (NF\_Narr15)
\z

Human or humanized referents that are definite are more likely to be overtly marked for definiteness by right-dislocation than non-human and inanimate referents. This is a tendency that is also observed in many other Bantu languages \citep{Riedel2009}.

\section{Locative inversion}
\label{bkm:Ref491361275}\label{bkm:Ref451503992}\hypertarget{Toc75352715}{}
Locative inversion is a type of clause where a locative noun phrase functions as the grammatical subject of the clause, and the logical subject is expressed as a post-verbal constituent. Similar constructions are widespread in Bantu, and may involve locatives, e.g. locative inversion, but also other constituents, such as patient or instrument inversion (\citealt{Wal2014}). In Fwe, the only attested inversion construction is locative inversion.

Locative inversion in Fwe is illustrated in (\ref{bkm:Ref74909960}). In the basic construction in (\ref{bkm:Ref491350600}), the grammatical subject \textit{rùkúngwè} ‘snake’ is also the logical subject. In the locative inversion construction in (\ref{bkm:Ref74909960}), the noun phrase \textit{mwìnjúò} ‘in the house’ is the grammatical subject, and the logical subject \textit{rùkúngwè} ‘snake’ is expressed postverbally.

\ea
\label{bkm:Ref491350600}
rùkúngwè nàkàbírì mwínjûò\\
\gll ∅-rukúngwe  na-kabí̲r-i      mú-e-N-júo\\
\textsc{np}\textsubscript{1a}-snake  \textsc{sm}\textsubscript{1}.\textsc{pst}-enter-\textsc{npst}.\textsc{pfv}  \textsc{np}\textsubscript{18}-\textsc{aug}-\textsc{np}\textsubscript{9}-house\\
\glt ‘The/a snake entered the house.’
\z

\ea
\label{bkm:Ref74909960}
mwìnjúò mwàkàbírì rùkûngwè\\
\gll mu-e-N-júo      mu-a-kabí̲r-i      ∅-rukúngwe\\
\textsc{np}\textsubscript{18}-\textsc{aug}-\textsc{np}\textsubscript{9}-house  \textsc{sm}\textsubscript{18}-\textsc{pst}-enter-\textsc{npst}.\textsc{pfv}  \textsc{np}\textsubscript{1a}-snake\\
\glt ‘A snake entered the house.’ (NF\_Elic17)
\z

In locative inversion, the locative subject triggers subject marking on the verb; in (\ref{bkm:Ref74909960}), the subject marker on the verb is that of class 18, agreeing with the locative noun phrase \textit{mwìnjúò} ‘in the house’, which is marked with a nominal prefix of class 18. The pre-verbal locative constituent may not be cross-referenced on the verb with a locative clitic, as shown by the ungrammaticality of (\ref{bkm:Ref491353170}).

\ea
\label{bkm:Ref491353170}
*mùnjúò mwàkàbírìmò mùsâ\\
\gll mu-N-júo    mu-a-kabí̲r-i=mo    mu-sá\\
\textsc{np}\textsubscript{18}-\textsc{np}\textsubscript{9}-house  \textsc{sm}\textsubscript{18}-\textsc{pst}-enter-\textsc{pst}=\textsc{loc}\textsubscript{18}  \textsc{np}\textsubscript{1}-thief\\
\glt Intended: ‘Into the house entered a thief.’ (NF\_Elic17)
\z

As is typical for Bantu languages, there is no prosodic boundary between the verb and the post-verbal constituent in locative inversion constructions. This is seen in the locative inversion construction in (\ref{bkm:Ref491351352}), where the verb \textit{kwàhúrí} does not undergo high tone retraction, showing that there is no prosodic boundary between the verb and the post-verbal constituent, and both are phrased together.

\ea
\label{bkm:Ref491351352}
kùmùnzì kwàhúrí bàbàrà\\
\gll ku-mu-nzi    ku-a-hur-í̲      ba-bara\\
\textsc{np}\textsubscript{17}-\textsc{np}\textsubscript{3}-village  \textsc{sm}\textsubscript{17}-\textsc{pst}-arrive-\textsc{npst}.\textsc{pfv}  \textsc{np}\textsubscript{2}-visitor\\
\glt ‘Some visitors arrived in the village.’ (NF\_Elic17)
\z

Locative inversion focuses the post-verbal constituent, and presents the pre-verbal locative constituent as discourse-old. This is illustrated in (\ref{bkm:Ref491353206}), where the location ‘this courtyard’ is discourse-old, and the post-verbal constituent, ‘a snake’, is new information. Note that in this locative inversion construction, the pre-verbal locative constituent is left out, as it is made clear by context, but the use of locative subject morphology still identifies it as locative inversion.

\ea
\label{bkm:Ref491353206}
mùbwènè èrí ꜝrápà mwàkàbírì rùkûngwè\\
\gll mu-bwene    e-rí    ∅-rapá mu-a-kabí̲r-i      ∅-rukúngwe \\
\textsc{sm}\textsubscript{2PL}-see.\textsc{stat}  \textsc{aug}-\textsc{dem}.\textsc{i}\textsubscript{5}  \textsc{np}\textsubscript{5}-courtyard
\textsc{sm}\textsubscript{18}-\textsc{pst}-enter-\textsc{npst}.\textsc{pfv}  \textsc{np}\textsubscript{1a}-snake\\
\glt ‘Do you see this courtyard? A snake entered in it.’ (NF\_Elic17)
\z

As the post-verbal constituent is discourse-new, it cannot be combined with an anaphoric demonstrative, as shown by the ungrammaticality of (\ref{bkm:Ref491352363}).

\ea
\label{bkm:Ref491352363}
*mwìrápá mwàkàbírì òzyó rùkúngwè\\
  \gll mu-e-∅-rapá mu-a-kabí̲r-i o-zyó  ∅-rukúngwe\\
  \textsc{np}\textsubscript{18}-\textsc{aug}-\textsc{np}\textsubscript{5}-courtyard \textsc{sm}\textsubscript{18}-\textsc{pst}-enter-\textsc{npst}.\textsc{pfv} \textsc{aug}-\textsc{dem}.\textsc{iii}\textsubscript{1} \textsc{np}\textsubscript{1a}-snake\\
\glt Int.: ‘This (aforementioned) snake entered into the courtyard.’ (NF\_Elic17)
\z

Locative inversion may also be interpreted as thetic focus, e.g. all the information is presented as new, as in (\ref{bkm:Ref491352905}), repeated from (\ref{bkm:Ref491351352}), which invites questions about who these visitors are, and what they want, e.g. the information is presented as all new.

\ea
\label{bkm:Ref491352905}
kùmùnzì kwàhúrí bàbàrà\\
\gll ku-mu-nzi    ku-a-hur-í̲      ba-bara\\
\textsc{np}\textsubscript{17}-\textsc{np}\textsubscript{3}-village  \textsc{sm}\textsubscript{17}-\textsc{pst}-arrive-\textsc{npst}.\textsc{pfv}  \textsc{np}\textsubscript{2}-visitor\\
\glt ‘Some visitors arrived in the village.’ (NF\_Elic17)
\z
\section{Dependent clauses}
\hypertarget{Toc75352716}{}\label{bkm:Ref75346027}
This section discusses types of dependent clauses that are used in Fwe. Relative clauses are dependent clauses that modify one of the constituents in the main clause; these are discussed in \sectref{bkm:Ref491095705}. There are various other ways of creating a dependent clause, mostly introduced by a specific free morpheme; these are discussed in \sectref{bkm:Ref491770136}.

\subsection{Relative clauses}
\label{bkm:Ref490828195}\hypertarget{Toc75352717}{}\label{bkm:Ref491095705}
A relative clause is syntactically embedded in the matrix clause, and describes one of the arguments of the matrix clause. The main clause contains an antecedent, the noun that the relative clause modifies.

A relative clause differs from a main clause in four respects: the verb is always the first element of the relative clause; the verb has a special form; the relative clause is optionally headed by a demonstrative functioning as a relativizer; and the antecedent noun optionally undergoes tonal changes.

The verb of a relative clause has a different tonal pattern than the verb of the same TAM construction in a main clause. For the present, near past imperfective, stative, and perfective subjunctive, the use of a high tone on the subject marker (melodic tone 2) changes a main clause verb into a relative clause verb, as in (\ref{bkm:Ref71272287}--\ref{bkm:Ref71272288}).

\ea
\label{bkm:Ref71272287}
\ea
mùrìrò ùtùmbúkà\\
\gll mu-riro  u-tu\textsubscript{H}mbuk-á̲\\
\textsc{np}\textsubscript{3}-fire  \textsc{sm}\textsubscript{3}-burn-\textsc{fv}\\
\glt ‘The fire burns.’

\ex
mùrìró òwò útùmbúkà\\
\gll mu-riró  o-o    ú̲-tu\textsubscript{H}mbuk-á̲\\
\textsc{np}\textsubscript{3}-fire  \textsc{aug}-\textsc{dem}.\textsc{iii}\textsubscript{3}  \textsc{sm}\textsubscript{3}.\textsc{rel}-burn-\textsc{fv}\\
\glt ‘the fire that burns’
\z\z

\ea
\ea
bànjòvù bàkùjwêngà\\
\gll ba-njovu    ba-aku-jwéng-a\\
\textsc{np}\textsubscript{2}-elephant    \textsc{sm}\textsubscript{2}-\textsc{npst}.\textsc{ipfv}-shout-\textsc{fv}\\
\glt ‘The elephants were shouting.’

\ex
bànjòvù bákùjwêngà\\
\gll ba-njovu    bá̲-aku-jwéng-a\\
\textsc{np}\textsubscript{2}-elephant    \textsc{sm}\textsubscript{2}.\textsc{rel}-\textsc{npst}.\textsc{ipfv}-shout-\textsc{fv}\\
\glt ‘the elephants who were shouting’
\z\z

\ea
\ea
ènyàmà ìbórêtè\\
\gll e-nyama  i-bor-é̲te\\
\textsc{aug}-meat  \textsc{sm}\textsubscript{9}-rot-\textsc{stat}\\
\glt ‘The meat is rotten.’

\ex
ènyàm’ êyò íbòrêtè\\
\gll e-nyamá  e-yo    í̲-bor-é̲te\\
\textsc{aug}-meat  \textsc{aug}-\textsc{dem}.\textsc{iii}\textsubscript{9}  \textsc{sm}\textsubscript{9}.\textsc{rel}-rot-\textsc{stat}\\
\glt ‘meat that is rotten’
\z\z

\newpage
\ea
\label{bkm:Ref71272288}
\ea
àbàntù bàhùpúrè\\
\gll a-ba-ntu    ba-hupur-é̲\\
\textsc{aug}-\textsc{np}\textsubscript{2}-person  \textsc{sm}\textsubscript{2}-think-\textsc{pfv}.\textsc{sbjv}\\
\glt ‘People should think.’

\ex
àbàntw’ ábò báhùpúrè\\
\gll a-ba-ntú    a-bo    bá̲-hupur-é̲\\
\textsc{aug}-\textsc{np}\textsubscript{2}-person  \textsc{aug}-\textsc{dem}.\textsc{iii}\textsubscript{2}  \textsc{sm}\textsubscript{2}.\textsc{rel}-think-\textsc{pfv}.\textsc{sbjv}\\
\glt ‘people who should think’ (NF\_Elic17)
\z\z

The remote past perfective (RPP) uses melodic tone 2 in its main clause form, which is maintained in the relative clause form. In addition, the relative clause form of the RPP makes use of melodic tone 4 (the loss of underlying tones), which is not seen in the main clause form of the RPP (see also \sectref{bkm:Ref489260766} on the remote past perfective). The tonal differences between main and relative clause forms of the RPP are illustrated in (\ref{bkm:Ref494561990}).

\ea
\label{bkm:Ref494561990}
\ea
\glll nìndádàmà\\
ni-ndí̲-a-dam-a\\
\textsc{rem}-\textsc{sm}\textsubscript{1SG}-\textsc{pst}-beat-\textsc{fv}\\
\glt ‘I beat.’

\ex
òmùntú zyò nìndáꜝdámà\\
\gll o-mu-ntú    zyo    ni-ndí̲-a-dam-á̲\\
\textsc{aug}-\textsc{np}\textsubscript{1}-person  \textsc{dem}.\textsc{iii}\textsubscript{1}  \textsc{rem}-\textsc{sm}\textsubscript{1SG}-\textsc{pst}-beat-\textsc{fv}<\textsc{rel}>\\
\glt ‘the person that I beat’ (NF\_Elic17)
\z\z

The remote past imperfective has a high tone on the subject marker in the main clause, as in (\ref{bkm:Ref99105705}). When used in a relative clause, as in (\ref{bkm:Ref99992493}), this high tone is retained and the verb does not undergo any tonal changes.

\ea
\label{bkm:Ref99105705}
\glll kàndíꜝshákà\\
ka-ndí̲-shak-á̲\\
\textsc{pst}.\textsc{ipfv}-\textsc{sm}\textsubscript{1SG}-want-\textsc{fv}\\
\glt ‘I used to like/want.’
\z

\ea
\label{bkm:Ref99992493}
cìntw’ ícò kàndíꜝshákà\\
\gll ci-ntú    e-co    ka-ndí̲-shak-á̲\\
\textsc{np}\textsubscript{7}-thing  \textsc{aug}-\textsc{dem}.\textsc{iii}\textsubscript{7}  \textsc{pst}.\textsc{ipfv}-\textsc{sm}\textsubscript{1SG}-want-\textsc{fv}\\
\glt ‘the thing that I used to like/want’ (NF\_Elic17)
\z

In the relative clause form of the near past perfective, a high tone on the subject marker also appears to play a role, but some variation is observed that can so far not be explained. There are cases where the relative near past perfective has a high tone on the subject marker, as in (\ref{bkm:Ref487812904}), or where the high tone is absent and the relative clause form is identical to the main clause form, as in (\ref{bkm:Ref487822000}). More data are needed to study the tonal behavior of the near past perfective in relative clauses, and what, if anything, conditions the use of the high tone on the subject marker.

\ea
\label{bkm:Ref487812904}
\ea
bànjòvù bànàjwêngì\\
\gll ba-njovu    ba-na-jwéng-i\\
\textsc{np}\textsubscript{2}-elephant    \textsc{sm}\textsubscript{2}-\textsc{pst}-shout-\textsc{npst}.\textsc{pfv}\\
\glt ‘The elephants shouted.’

\ex
bànjòvù bánàjwêngì\\
\gll ba-njovu    bá̲-na-jwéng-i\\
\textsc{np}\textsubscript{2}-elephant    \textsc{sm}\textsubscript{2}.\textsc{rel}-\textsc{pst}-shout-\textsc{npst}.\textsc{pfv}\\
\glt ‘the elephants who shouted’
\z\z

\ea
\label{bkm:Ref487822000}
\ea
ècìntù càhíkìwà\\
\gll e-ci-ntu    ci-a-hík-iw-a\\
\textsc{aug}-\textsc{np}\textsubscript{7}-thing  \textsc{sm}\textsubscript{7}-\textsc{pst}-cook-\textsc{pass}-\textsc{fv}\\
\glt ‘The thing is cooked.’

\ex
ècìntú cò càhíkìwà\\
\gll e-ci-ntú    co    ci-a-hík-iw-a\\
\textsc{aug}-\textsc{np}\textsubscript{7}-thing  \textsc{dem}.\textsc{iii}\textsubscript{7}  \textsc{sm}\textsubscript{7}-\textsc{pst}-cook-\textsc{pass}-\textsc{fv}\\
\glt ‘the thing that is cooked’
\z
\z

Future constructions cannot be used in relative clauses. Various strategies exist to express future temporal reference in a relative clause. A subjunctive verb can be used; either marked with a remoteness prefix \textit{na-/ne-} to express a remote future, as in (\ref{bkm:Ref492105913}), or preceded by the subordinator \textit{sàké}, as in (\ref{bkm:Ref492105914}), or both, as in (\ref{bkm:Ref492105916}). The present construction can also be used to express future reference in relative clauses, as in (\ref{bkm:Ref492105917}); as discussed in \sectref{bkm:Ref72233436}, the present construction can have a futurate use in main clauses was well.

\ea
\label{bkm:Ref492105913}
èŋòmbé zò nèndíꜝúrè\\
\gll e-N-ŋombé    zo    ne-ndí̲-ur-é̲\\
\textsc{aug}-\textsc{np}\textsubscript{10}-cow  \textsc{dem}.\textsc{iii}\textsubscript{10}  \textsc{rem}-\textsc{sm}\textsubscript{1SG}.\textsc{rel}-buy-\textsc{pfv}.\textsc{sbjv}\\
\glt ‘the cattle that I will buy’ (NF\_Elic17)
\z

\ea
\label{bkm:Ref492105914}
ècò shàké ꜝcípàngàhárè hânù\\
\gll e-co    shaké  cí̲-pang-ahar-é̲    hánu\\
\textsc{aug}-\textsc{dem}.\textsc{iii}\textsubscript{7}  when  \textsc{sm}\textsubscript{7}-do-\textsc{neut}-\textsc{pfv}.\textsc{sbjv}  \textsc{dem}\textsubscript{16}\\
\glt ‘That which will happen now…’ (NF\_Narr17)
\z

\ea
\label{bkm:Ref492105916}
címùnyà ècìntù ècò sàké nókàwânè kwàzyúmùnyà\\
\gll cí-munya  e-ci-ntu  e-co    saké  na-ó̲-ka-wá̲n-e      kwa-zyú-munya\\
\textsc{pp}\textsubscript{7}-other  \textsc{aug}-\textsc{np}\textsubscript{7}-thing
\textsc{aug}-\textsc{dem}.\textsc{iii}\textsubscript{7}  when  \textsc{rem}-\textsc{sm}\textsubscript{2SG}-\textsc{dist}-find-\textsc{pfv}.\textsc{sbjv}  \textsc{np}\textsubscript{17}-\textsc{pp}\textsubscript{1}-other\\
\glt ‘The other thing that you will get from the other one…’ (NF\_Song17)
\z

\ea
\label{bkm:Ref492105917}
òzyw’ ásèbèzá zyônà\\
\gll o-zyu    á̲-sebez-á̲    zyóna\\
\textsc{aug}-\textsc{dem}.\textsc{iii}\textsubscript{1}  \textsc{sm}\textsubscript{1}.\textsc{rel}-work-\textsc{fv}  tomorrow\\
\glt ‘the one who will work tomorrow…’ (NF\_Elic15)
\z

\tabref{tab:13:1} gives an overview of the changes that affect relative clause verbs in different TAM constructions.

\begin{table}
\label{bkm:Ref490576656}\caption{\label{tab:13:1}Tonal patterns of relative clause verbs}
\begin{tabularx}{\textwidth}{lQ}
\lsptoprule
Inflection & Relative clause form\\
\midrule
Present & high tone on the subject marker\\
Subjunctive & high tone on the subject marker\\
Stative & high tone on the subject marker\\
Remote Past Perfective & high tone on the subject marker + different melodic tone \\
Near Past Perfective & optional (?) high tone on the subject marker\\
Remote Past Imperfective & high tone on the subject marker\\
Near Past Imperfective & high tone on the subject marker\\
Near Future & -\\
Remote Future & -\\
\lspbottomrule
\end{tabularx}
\end{table}

Relative clauses are also distinguished from main clauses in the position of the verb. In a relative clause, the verb is always the first constituent. Any other constituent that appears in the relative clause appears after the verb, regardless of its syntactic or pragmatic properties. This distinguishes relative clauses from main clauses, where information structure influences word order, and where, in pragmatically neutral contexts, the subject precedes the verb (see \sectref{bkm:Ref492315922}). This is illustrated in (\ref{bkm:Ref490752278}), where the relative clause contains both a nominal subject, \textit{kàshùrwè} ‘the rabbit’, and a nominal object, \textit{òzyú mùkázànà} ‘this girl’; both constituents occur after the relative clause verb.

\ea
\label{bkm:Ref490752278}
mbóbùryàhó nàáshèshá kàshùrwè òzyú mùkázànà\\
\gll mbó-bu-ryahó    na-á̲-shesh-á̲        ka-shurwe o-zyú    mu-kázana \\
\textsc{cop}.\textsc{def}\textsubscript{14}-\textsc{np}\textsubscript{14}-like\_that  \textsc{pst}-\textsc{sm}\textsubscript{1}-\textsc{pst}-marry-\textsc{fv}<\textsc{rel}>  \textsc{np}\textsubscript{12}-rabbit
\textsc{aug}-\textsc{dem}.\textsc{i}\textsubscript{1}  \textsc{np}\textsubscript{1}-girl\\
\glt ‘That is how the rabbit married this girl.’ (NF\_Narr15)
\z

Relative clauses may be headed by a demonstrative that functions as a relativizer. With subject relatives, where the antecedent is the subject of the relative clause, the demonstrative as a relativizer is optional. This is illustrated in (\ref{bkm:Ref448834760}--\ref{bkm:Ref74910015}), where the demonstrative \textit{abo} can be used, as in (\ref{bkm:Ref448834760}), or left out, as in (\ref{bkm:Ref74910015}).

\ea
\label{bkm:Ref448834760}
bànjòvw’ ábò bájwêngà\\
\gll ba-njovú  a-bo    bá̲-jwé̲ng-a\\
\textsc{np}\textsubscript{2}-elephant  \textsc{aug}-\textsc{dem}.\textsc{iii}\textsubscript{2}  \textsc{sm}\textsubscript{2}.\textsc{rel}-shout-\textsc{fv}\\
\glt ‘The elephants who shout…’
\z

\ea
\label{bkm:Ref74910015}
bànjòvù bájwêngà\\
\gll ba-njovu    bá̲-jwé̲ng-a\\
\textsc{np}\textsubscript{2}-elephant    \textsc{sm}\textsubscript{2}.\textsc{rel}-shout-\textsc{fv}\\
\glt ‘The elephants who shout…’ (NF\_Elic17)
\z

In object relatives, where the object functions as the antecedent of the relative clause, the demonstrative functioning as a relativizer is obligatory, as in (\ref{bkm:Ref494562073}), and leaving out the demonstrative is ungrammatical, as in (\ref{bkm:Ref74910063}).

\ea
\label{bkm:Ref494562073}
bàntw’ ábò ndíbwènè\\
\gll ba-ntú  a-bo    ndí̲-bwe\textsubscript{H}ne\\
\textsc{np}\textsubscript{2}-person  \textsc{aug}-\textsc{dem}.\textsc{iii}\textsubscript{2} \textsc{sm}\textsubscript{1SG}.\textsc{rel}-see.\textsc{stat}\\
\glt ‘The people that I see…’
\z

\ea
\label{bkm:Ref74910063}
*bàntù ndíbwènè\\
\gll ba-ntu  ndí̲-bwe\textsubscript{H}ne\\
\textsc{np}\textsubscript{2}-person  \textsc{sm}\textsubscript{1SG}.\textsc{rel}-see.\textsc{stat}\\
\glt Intended: ‘The people that I see…’ (NF\_Elic17)
\z

When the antecedent is a locative, a demonstrative functioning as a relativizer is obligatory, as in (\ref{bkm:Ref490752430}), which uses the class 17 demonstrative \textit{oko} as a relativizer. Cross-referencing the locative antecedent on the relative clause verb through the use of a locative clitic, is not possible, as in (\ref{bkm:Ref490752431}).

\ea
\label{bkm:Ref490752430}
kùmùnzí òkò ndíyà kwámàkângà\\
\gll ku-mu-nzí    o-ko      ndí̲-i-a    ∅-kwá-makánga\\
\textsc{np}\textsubscript{17}-\textsc{np}\textsubscript{3}-village  \textsc{aug}-\textsc{dem}.\textsc{iii}\textsubscript{17}  \textsc{sm}\textsubscript{1SG}.\textsc{rel}-go-\textsc{fv}  \textsc{cop}-\textsc{np}\textsubscript{17}-Makanga\\
\glt ‘The village that I go to is Makanga.’
\z

\ea
\label{bkm:Ref490752431}
*kùmùnzí òkò ndíyàkò kwámàkângà\\
\gll ku-mu-nzí    o-ko      ndí̲-i-a=ko ∅-kwá-makánga \\
\textsc{np}\textsubscript{17}-\textsc{np}\textsubscript{3}-village  \textsc{aug}-\textsc{dem}.\textsc{iii}\textsubscript{17}  \textsc{sm}\textsubscript{1SG}.\textsc{rel}-go-\textsc{fv}=\textsc{loc}\textsubscript{17}
\textsc{cop}-\textsc{np}\textsubscript{17}-Makanga\\
\glt Intended: ‘The village that I go to is Makanga.’
\z

The demonstratives of the locative classes are also used with non-locative antecedents which only have a locative use in the relative clause, as in (\ref{bkm:Ref448833975}): the antecedent \textit{mùsébézì} ‘a job’ is not locative, but has a locative use in the following relative clause, which is headed by the class 17 demonstrative \textit{òkò}.

\ea
\label{bkm:Ref448833975}
kùbònàhárá yé òkwésí \textbf{mùsébézì} \textbf{òkò} kòshákí nòkùàmbà nàbàntù\\
\gll ku-bon-ahar-á̲  yé  o-kwesí  mu-sebézi  o-ko ka-o-shak-í̲      no=ku-amb-a  na=ba-ntu\\
\textsc{inf}-see-\textsc{neut}-\textsc{fv}  that  \textsc{sm}\textsubscript{2SG}-have  \textsc{np}\textsubscript{3}-job  \textsc{aug}-\textsc{dem}.\textsc{iii}\textsubscript{17}
\textsc{neg}-\textsc{sm}\textsubscript{2SG}-want-\textsc{neg}  \textsc{com}=\textsc{inf}-talk-\textsc{fv}  \textsc{com}=\textsc{np}\textsubscript{2}-person\\
\glt ‘It seems you have a job where you don’t want to talk to people.’ (NF\_Narr15)
\z

In cleft constructions, the demonstrative is never used as a relativizer, even when the antecedent, which is the clefted element, has the role of object (see also \ref{bkm:Ref491333435} on cleft constructions), as in (\ref{bkm:Ref99107609}).

\ea
\label{bkm:Ref99107609}
mbàntù ndíꜝdámà\\
\gll N-ba-ntu    ndí̲-dam-á̲\\
\textsc{cop}-\textsc{np}\textsubscript{2}-person  \textsc{sm}\textsubscript{1SG}.\textsc{rel}-beat-\textsc{fv}\\
\glt ‘It’s people that I beat.’ (NF\_Elic15)
\z

Of the four demonstrative series used in Fwe (see \sectref{bkm:Ref492026896}), most can be used as relativizer. In Namibian Fwe, a series III demonstrative is always used. In Zambian Fwe, a series I demonstrative is preferred, but other demonstratives are also allowed, as illustrated in (\ref{bkm:Ref490576744}).

\ea
\label{bkm:Ref490576744}
àkàfùró àkà / àkànò / àkò /àkènà ndíbèrèkìsâ\\
\gll a-ka-furó    {a-ka /} {a-kano /} {a-ko /} a-kena  ndí̲-berek-is-á̲ \\
\textsc{aug}-\textsc{np}\textsubscript{12}-knife  \textsc{aug}-\textsc{dem}.\textsc{i}\textsubscript{12} /\textsc{aug}-\textsc{dem}.\textsc{ii}\textsubscript{12} /\textsc{aug}-\textsc{dem}.\textsc{iii}\textsubscript{12} /\textsc{aug}-\textsc{dem}.\textsc{iv}\textsubscript{12}
\textsc{sm}\textsubscript{1SG}.\textsc{rel}-work-\textsc{caus}-\textsc{fv}\\
\glt ‘The knife that I am using…’ (ZF\_Elic13)
\z

As discussed in \sectref{bkm:Ref492026896}, the tonal realization of demonstratives varies depending on their syntactic function. When used as a relativizer, the demonstrative does not have a high tone on the demonstrative stem. The demonstrative does, however, have an underlying high tone on the augment which attaches to the last syllable of the preceding word, namely the antecedent. This is illustrated in (\ref{bkm:Ref74910110}) with the noun \textit{bànjòvù} ‘elephants’, which is realized without high tones in isolation, but is assigned a final high tone when followed by the demonstrative functioning as a relativizer.

\ea
\label{bkm:Ref74910110}
bànjòvú àbò bánùnîtè\\
\gll ba-njovú  a-bo    bá̲-nun-í̲te\\
\textsc{np}\textsubscript{2}-elephant  \textsc{aug}-\textsc{dem}.\textsc{iii}\textsubscript{2}  \textsc{sm}\textsubscript{2}.\textsc{rel}-become\_fat-\textsc{stat}\\
\glt ‘Elephants who are fat…’ (NF\_Elic17)
\z

This high tone only occurs on the antecedent noun when a demonstrative used as relativizer is present. When the demonstrative is absent, as it may be in subject relatives, no high tone is assigned to the last syllable of the antecedent, as in (\ref{bkm:Ref99107703}).

\ea
\label{bkm:Ref99107703}
bànjòvù bánùnîtè\\
\gll ba-njovu  bá̲-nun-í̲te\\
\textsc{np}\textsubscript{2}-elephant  \textsc{sm}\textsubscript{2}.\textsc{rel}-become\_fat-\textsc{stat}\\
\glt ‘Elephants who are fat…’ (NF\_Elic17)
\z

The high tone of the demonstrative’s augment does appear, however, when the vowel of the augment is not realized. This is illustrated in (\ref{bkm:Ref71273387}), where the demonstrative \textit{zyo} lacks the augment \textit{o-}, but still assigns a high tone to the antecedent \textit{ònjòvú} ‘elephant’.

\ea
\label{bkm:Ref71273387}
ònjòvú zyò ndíbwènè\\
\gll o-∅-njovú    zyo    ndí̲-bwe\textsubscript{H}ne\\
\textsc{aug}-\textsc{np}\textsubscript{1a}-elephant  \textsc{dem}.\textsc{iii}\textsubscript{1}  \textsc{sm}\textsubscript{1SG}.\textsc{rel}-see.\textsc{stat}\\
\glt ‘The elephant that I see…’ (NF\_Elic17)
\z

The behavior of the augment on demonstratives in relative clauses is similar to the behavior of augments in other contexts, where the tonal and segmental form of the augment are also separated and one may occur without the other (see \sectref{bkm:Ref444175456}).

All the previous examples contain relative clauses with an overt antecedent. Fwe also allows headless relative clauses, where the antecedent is a demonstrative that functions as both antecedent and relativizer, as in (\ref{bkm:Ref490753325}).

\ea
\label{bkm:Ref490753325}
òzyw’ ázìzyìː òzyw’ ázìshúwîrè òzyw’ ázìbwènè\\
\gll o-zyu    á̲-zi\textsubscript{H}-zyiː\textsubscript{H}      o-zyu    á̲-zi\textsubscript{H}-shu\textsubscript{H}-í̲re o-zyu    á̲-zi\textsubscript{H}-bwe\textsubscript{H}ne\\
\textsc{aug}-\textsc{dem}.\textsc{i}\textsubscript{1}  \textsc{sm}\textsubscript{1}.\textsc{rel}-\textsc{om}\textsubscript{8}-know.\textsc{stat}  \textsc{aug}-\textsc{dem}.\textsc{i}\textsubscript{1}  \textsc{sm}\textsubscript{1}.\textsc{rel}-\textsc{om}\textsubscript{8}-hear-\textsc{stat}
\textsc{aug}-\textsc{dem}.\textsc{i}\textsubscript{1}  \textsc{sm}\textsubscript{1}.\textsc{rel}-\textsc{om}\textsubscript{8}-see.\textsc{stat}\\
\glt ‘The one who knows them, the one who hears them, the one who sees them.’ (NF\_Song17)
\z

Headless relative clauses introduced by a class 16 demonstrative, \textit{àhà}, express a temporal clause, translated as ‘when’, as in (\ref{bkm:Ref99107783}--\ref{bkm:Ref99107784}). Noun class 16 is primarily a locative class, but is also used for expressing location in time rather than in space, as discussed in \sectref{bkm:Ref452049189}. Fwe also has various other ways of expressing temporal clauses, which are discussed in \sectref{bkm:Ref491770136}.

\ea
\label{bkm:Ref99107783}
àhà bákèːzyà kùkúw’ òbwâtò\\
\gll a-ha    bá̲-ké̲ːzy-a    ku-kú-a  o-bu-áto\\
\textsc{aug}-\textsc{dem}.\textsc{i}\textsubscript{16}  \textsc{sm}\textsubscript{2}.\textsc{rel}-come-\textsc{fv}  \textsc{inf}-call-\textsc{fv}  \textsc{aug}-\textsc{np}\textsubscript{14}-canoe\\
\glt ‘When they came to call the canoe…’ (NF\_Narr15)
\z

\ea
\label{bkm:Ref99107784}
àhà kàndírwârà nàndákàtà\\
\gll a-ha    ka-ndí̲-rwá̲r-a    na-ndí̲-a-kat-a\\
\textsc{aug}-\textsc{dem}.\textsc{i}\textsubscript{16} \textsc{pst}.\textsc{ipfv}-\textsc{sm}\textsubscript{1SG}-be\_sick-\textsc{fv}  \textsc{pst}-\textsc{sm}\textsubscript{1SG}-\textsc{pst}-become\_thin-\textsc{fv}\\
\glt ‘When I was sick, I was very thin.’ (ZF\_Elic14)
\z

\largerpage
\subsection{Other types of dependent clauses}
\label{bkm:Ref491770136}\hypertarget{Toc75352718}{}
There are various other types of dependent clauses, marked by a free morpheme, or by a verbal affix. \tabref{tab:13:2} gives an overview of the different dependent clause markers.

\begin{table}
\label{bkm:Ref490824259}\caption{\label{tab:13:2}Markers of dependent clauses}

\begin{tabular}{ll}
\lsptoprule
\textit{kùtí / kùtêyè / ìyé} & - complement ‘that’\\
                             & - quotative ‘that’\\
                             & - purpose ‘(so) that’\\\
                             & - conditional ‘if’\\
\textit{háìbà} & - conditional \\
\textit{shàké} & - conditional ‘if’\\
               & - temporal ‘when’\\
\textit{nârì} & - counterfactual ‘if, if not for’\\
{\itshape shi-}  & - conditional ‘if’\\
\lspbottomrule
\end{tabular}
\end{table}

The free morpheme \textit{kùtí / kùtêyè / ìyé} ‘that, so that, if’ is realized as \textit{kùtí} in Zambian Fwe, as \textit{ìyé} in Namibian Fwe, and \textit{kùtêyè} can be used in both varieties. The forms \textit{kùtí} and \textit{kùtêyè} are contractions of the verb \textit{kùtá} ‘to say’, with the complementizer \textit{ìyé} ‘that’.

\begin{sloppypar}
The forms \textit{kùtí / kùtêyè / ìyé} can introduce various types of dependent clauses. It can be used to introduce a complement clause, as in (\ref{bkm:Ref490816023}), where \textit{ìyé} marks a complement clause that functions as the object of the main clause verb \textit{shòshùwírè} ‘you hear’. A complement clause marked by \textit{kùtí} is illustrated in (\ref{bkm:Ref494562355}), and a complement clause introduced by \textit{kùtêyè} in (\ref{bkm:Ref494562359}).
\end{sloppypar}

\ea
\label{bkm:Ref490816023}
kàpá shòshùwírè \textbf{ìyé} shàkwèsí òmúkwàmé ꜝkwímbari\\
\gll kapá  sha-o-shu\textsubscript{H}-í̲re    iyé  sha-a-kwesí    o-mú-kwamé  kú-e-N-bari \\
or  \textsc{inc}-\textsc{sm}\textsubscript{2SG}-hear-\textsc{stat}  \textsc{comp}
\textsc{inc}-\textsc{sm}\textsubscript{1}-have  \textsc{aug}-\textsc{np}\textsubscript{1}-man  \textsc{np}\textsubscript{17}-\textsc{aug}-\textsc{np}\textsubscript{9}-side\\
\glt ‘Or you hear that she now has a man on the side.’ (ZF\_Conv13)
\z

\ea
\label{bkm:Ref494562355}
mbábòné \textbf{kùtí} cìpèpà bùryó cìbámùdàrà\\
\gll mbo-á̲-bo\textsubscript{H}n-é̲      kutí ∅-ci-pepa    bu-ryó  ci-bá-mu-dara \\
\textsc{near}.\textsc{fut}-\textsc{sm}\textsubscript{1}-see-\textsc{pfv}.\textsc{sbjv}  \textsc{comp}
\textsc{cop}-\textsc{np}\textsubscript{7}-paper  \textsc{np}\textsubscript{14}-only  \textsc{pp}\textsubscript{7}-\textsc{np}\textsubscript{2}-\textsc{np}\textsubscript{1}-old\_man\\
\glt ‘She will see that it is just a paper of her husband.’ (ZF\_Conv13)
\z

\ea
\label{bkm:Ref494562359}
ndìkéːzyà kùtóndà \textbf{kùtêyè} ndùngwè\\
\gll ndi-ké̲ːzy-a    ku-tónd-a  kutêye    ndu-∅-ngwe\\
\textsc{sm}\textsubscript{1SG}-come-\textsc{fv}  \textsc{inf}-see-\textsc{fv}  \textsc{comp}    \textsc{cop}\textsubscript{1a}-\textsc{np}\textsubscript{1a}-leopard\\
\glt ‘I came and saw that it is a leopard.’ (ZF\_Narr14)
\z

Complement clauses are often introduced by a verb of saying in the main clause, where the complement clause represents that which is said. This can be direct speech, where the complement clause literally quotes what is said, as in (\ref{bkm:Ref490817605}), or indirect speech, where the complement clause paraphrases what is said from the perspective of the speaker, as in (\ref{bkm:Ref490817606}).

\ea
\label{bkm:Ref490817605}
rùkúngwè àkéːzyà kùmùtóròkèrà ìyé mùyéꜝnzángù ndìkùfwírà ènshêː\\
\gll ∅-rukúngwe  a-ké̲ːzy-a  ku-mu-tórok-er-a    iyé mu-énz-angú  ndi-ku-fw-í̲r-a    e-nshéː\\
\textsc{np}\textsubscript{1a}-snake  \textsc{sm}\textsubscript{1}-come-\textsc{fv}  \textsc{inf}-\textsc{om}\textsubscript{1}-explain-\textsc{appl}-\textsc{fv}  \textsc{comp}
\textsc{np}\textsubscript{1}-friend-\textsc{poss}\textsubscript{1SG}  \textsc{sm}\textsubscript{1SG}-\textsc{om}\textsubscript{2SG}-die-\textsc{appl}-\textsc{fv}  \textsc{aug}-pity\\
\glt ‘Snake came to tell him: my friend, I feel pity for you.’ (NF\_Narr17)
\z

\ea
\label{bkm:Ref490817606}
nàndìsúmwìnì ìyé ndákùménèkàngà\\
\gll na-ndi-súmwin-i      iyé  ndi-áku-mének-ang-a\\
\textsc{sm}\textsubscript{1}.\textsc{pst}-\textsc{om}\textsubscript{1SG}-tell-\textsc{npst}.\textsc{pfv}  \textsc{comp}  \textsc{sm}\textsubscript{1SG}-\textsc{sbjv}.\textsc{ipfv}-wake\_early-\textsc{hab}-\textsc{fv}\\
\glt ‘S/he told me that I should regularly wake up early.’ (NF\_Elic17)
\z

\textit{ìyé} can also be used as a quotative without an overt speech verb in the main clause, as in (\ref{bkm:Ref71274254}--\ref{bkm:Ref71274256}), where the quotative \textit{ìyé} is directly followed by the quoted speech.

\ea
\label{bkm:Ref71274254}
òmbwá ꜝákùshwáhùrà ìyé hmm òzyú mùntù kàndíhì ècí cìfûhà\\
\gll o-mbwá  á-ku-shwáhur-a    iyé  hmm  o-zyú    mu-ntu ka-ndí̲-ha-i      e-cí    ci-fúha\\
\textsc{aug}-dog  \textsc{con}\textsubscript{1}-\textsc{inf}-give\_up-\textsc{fv}  \textsc{comp}  hmm  \textsc{aug}-\textsc{dem}.\textsc{i}\textsubscript{1}  \textsc{np}\textsubscript{1}-person
\textsc{neg}-\textsc{sm}\textsubscript{1SG}-give-\textsc{neg}  \textsc{aug}-\textsc{dem}.\textsc{i}\textsubscript{7}  \textsc{np}\textsubscript{7}-bone\\
\glt ‘The dog then gave up. [He said] that, hmm, this person, he will not give me this bone.’ (NF\_Narr17)
\z

\ea
\label{bkm:Ref71274256}
ìyé njìnyàmà njìnyàmà índìrwáríkà\\
\gll iyé  nji-N-nyama nji-N-nyama  í̲-ndi-rwa\textsubscript{H}r-ik-á̲ \\
\textsc{comp}  \textsc{cop}\textsubscript{9}-\textsc{np}\textsubscript{9}-meat
\textsc{cop}\textsubscript{9}-\textsc{np}\textsubscript{9}-meat  \textsc{sm}\textsubscript{9}.\textsc{rel}-\textsc{om}\textsubscript{1SG}-be\_sick-\textsc{imp}.\textsc{tr}-\textsc{fv}\\
\glt ‘[She said] that, it’s meat. It’s meat that makes me sick.’ (NF\_Narr17)
\z

\textit{kùtí / kùtêyè / ìyé} may also introduce a dependent clause with a subjunctive verb, that expresses the (intended) goal of the main clause, as in (\ref{bkm:Ref99108031}--\ref{bkm:Ref99108032}).

\ea
\label{bkm:Ref99108031}
ákùhá òmòyà kwíŋwàrárà \textbf{ìyé} àyéndè kózywìnà òmùntù\\
\gll á-ku-há-a    o-mu-oya    kú-e-∅-ŋwarará iyé  a-é̲nd-e    kú-o-zywina    o-mu-ntu\\
\textsc{con}\textsubscript{1}-\textsc{inf}-give-\textsc{fv}  \textsc{aug}-\textsc{np}\textsubscript{3}-soul  \textsc{np}\textsubscript{17}-\textsc{aug}-\textsc{np}\textsubscript{5}-crow
\textsc{comp}  \textsc{sm}\textsubscript{1}-go-\textsc{pfv}.\textsc{sbjv}  \textsc{np}\textsubscript{17}-\textsc{aug}-\textsc{dem}.\textsc{iv}\textsubscript{1}  \textsc{aug}-\textsc{np}\textsubscript{1}-person\\
\glt ‘Then he gave a soul to the crow, \textbf{so} \textbf{that} he can go to that person.’ (NF\_Narr17)
\z

\ea
\label{bkm:Ref99108032}
mbùtí nàyíꜝwánè èyí shérêŋì \textbf{òkùtêyè} àyé ndìbòózèrè\\
\gll N-bu-tí    na-í̲-wan-é̲        e-í    ∅-sheréŋi okutéye  a-y-é̲      ndi-boó̲z-er-e\\
\textsc{cop}-\textsc{np}\textsubscript{14}-how  \textsc{rem}.\textsc{sm}\textsubscript{1}-\textsc{om}\textsubscript{9}-find-\textsc{pfv}.\textsc{sbjv}  \textsc{aug}-\textsc{dem}.\textsc{i}\textsubscript{9}  \textsc{np}\textsubscript{9}-money
\textsc{comp}    \textsc{sm}\textsubscript{1}-go-\textsc{pfv}.\textsc{sbjv}  \textsc{om}\textsubscript{1SG}-return-\textsc{appl}-\textsc{pfv}.\textsc{sbjv}\\
\glt ‘How will he get this money, \textbf{so} \textbf{that} he brings it back to me?’ (ZF\_Conv13)
\z

\textit{kùtí / kùtêyè / ìyé} may also introduce a dependent clause that functions as a conditional, as in (\ref{bkm:Ref99108051}--\ref{bkm:Ref99108052}).

\ea
\label{bkm:Ref99108051}
mùzyìː òmfúmù \textbf{kùtèè} àkwèsí bânà bèná bânà bàsépáhárá ꜝcáhà\\
\gll mu-zyiː    o-∅-mfúmu    kuteye    a-kwesí  ba-ána bená    ba-ána  ba-sep-ahar-á̲  cáha\\
\textsc{sm}\textsubscript{2PL}-know.\textsc{stat}  \textsc{aug}-\textsc{np}\textsubscript{1a}-chief  \textsc{comp}    \textsc{sm}\textsubscript{1}-have  \textsc{np}\textsubscript{2}-child
\textsc{dem}.\textsc{iv}\textsubscript{2}  \textsc{np}\textsubscript{2}-child  \textsc{sm}\textsubscript{2}-trust-\textsc{neut}-\textsc{fv}   very\\
\glt ‘You know, a chief, \textbf{if} he has children, those children are highly respected.’ (NF\_Narr15)
\z

\ea
\label{bkm:Ref99108052}
èswé tùbáꜝkwámè \textbf{òkùtêyè} tùshúwé bùryáhò ryètú èfùfá rìhítírìzè\\
\gll eswé    tu-bá-kwamé  o-kutéye  tu-shu\textsubscript{H}-é̲     bu-ryahó   ri-etú    e-∅-fufá    ri-hi\textsubscript{H}t-í̲riz-e\\
\textsc{pers}\textsubscript{1PL}  \textsc{app}\textsubscript{1PL}-\textsc{np}\textsubscript{2}-man  \textsc{aug}-\textsc{comp}  \textsc{sm}\textsubscript{1PL}-hear-\textsc{pfv}.\textsc{sbjv} 
\textsc{np}\textsubscript{14}-like\_that \textsc{pp}\textsubscript{5}-\textsc{poss}\textsubscript{1PL}  \textsc{aug}-\textsc{np}\textsubscript{5}-jealousy  \textsc{sm}\textsubscript{5}-pass-\textsc{int}.\textsc{caus}-\textsc{pfv}.\textsc{sbjv}\\
\glt ‘Us men, if we hear like that, our jealousy is very big.’ (ZF\_Conv13)
\z

The free morpheme \textit{háìbà} ‘if, when’ can be used to introduce a conditional clause (‘if…’), as in (\ref{bkm:Ref448844586}--\ref{bkm:Ref100139794}), or a temporal clause (‘when…’), as in (\ref{bkm:Ref490820378}).

\ea
\label{bkm:Ref448844586}
\label{bkm:Ref71274435}
\textbf{háìbà} mbwáshòk’ ómvûrà kàndìyêndì\\
\gll háiba  mbo-á̲-sho\textsubscript{H}k-é̲      o-∅-mvúra    ka-ndi-é̲nd-i\\
if  \textsc{near}.\textsc{fut}-\textsc{sm}\textsubscript{1a}-rain-\textsc{pfv}.\textsc{sbjv}  \textsc{aug}-\textsc{np}\textsubscript{1a}-rain  \textsc{neg}-\textsc{sm}\textsubscript{1SG}-go-\textsc{neg}\\
\glt ‘If it rains, I will not go.’ (NF\_Elic15)
\z

\ea
\label{bkm:Ref100139794}
\textbf{háìbà} ènyázì yàkàkùŋórèrì ŋórò\\
\gll háiba  e-N-nyázi    i-a-ka-ku-ŋór-er-i ∅-ŋoró\\
if  \textsc{aug}-\textsc{np}\textsubscript{9}-mistress  \textsc{sm}\textsubscript{9}-\textsc{pst}-\textsc{dist}-\textsc{om}\textsubscript{2SG}-write-\textsc{appl}-\textsc{npst}.\textsc{pfv}
\textsc{np}\textsubscript{5}-letter\\
\glt ‘If your mistress has written you a letter…’ (ZF\_Conv13)
\z

\ea
\label{bkm:Ref490820378}
èfoni \textbf{háìbà} mbòíꜝrírè òìtábè\\
\gll e-∅-foni    háiba  mbo-í̲-rir-é̲ o-i\textsubscript{H}-tab-é̲ \\
\textsc{aug}-\textsc{np}\textsubscript{9}-phone  if  \textsc{near}.\textsc{fut}-\textsc{sm}\textsubscript{9}-cry-\textsc{pfv}.\textsc{sbjv}
\textsc{sm}\textsubscript{2SG}-\textsc{om}\textsubscript{9}-answer-\textsc{pfv}.\textsc{sbjv}\\
\glt ‘The phone, when it rings, you must answer it.’ (NF\_Elic17)
\z

\textit{háìbà} is a borrowing from Lozi \textit{haiba} ‘if’ \citep[78]{Burger1960}. In Fwe, it may occur on its own, as in (\ref{bkm:Ref71274435}--\ref{bkm:Ref490820378}), or it may combine with the native complementizer \textit{kùtí} (and variations thereof), as in (\ref{bkm:Ref71274461}).

\ea
\label{bkm:Ref71274461}
\textbf{háìbà} \textbf{kùtéyè} sìànàmání mênjì kàzíꜝyángà kúmìrâkà\\
\gll háiba  kutéye    si-a-na-man-í̲      ma-ínji ka-zí̲-ya-á̲ng-a    kú-mi-ráka\\
when  \textsc{comp}    \textsc{inc}-\textsc{sm}\textsubscript{6}-\textsc{pst}-finish-\textsc{npst}.\textsc{pfv}  \textsc{np}\textsubscript{6}-water
\textsc{pst}.\textsc{ipfv}-\textsc{sm}\textsubscript{10}-go-\textsc{hab}-\textsc{fv}  \textsc{np}\textsubscript{17}-\textsc{np}\textsubscript{4}-kraal\\
\glt ‘When the water is finished, they would go to the kraals.’ (NF\_Narr17)
\z

The free morpheme \textit{shàké} ‘when, if’ is used to introduce a dependent clause that is either conditional, as in (\ref{bkm:Ref490824801}--\ref{bkm:Ref490824802}), or temporal, as in (\ref{bkm:Ref490824803}--\ref{bkm:Ref490824804}). The verb in the dependent clause is in the subjunctive mood. The morpheme itself is realized as \textit{shàká} in Zambian Fwe, and as either \textit{shàké} or \textit{sàké} in Namibian Fwe. The interchangeability of /s/ and /sh/ is also seen in other grammatical morphemes (see \sectref{bkm:Ref70695065}). \textit{shàké} is derived from the lexical verb \textit{shàkà} ‘want’.

\newpage
\ea
\label{bkm:Ref490824801}
òzyú mùntù \textbf{shàká} ndìmùshêshè ndìmùkwànìsá kàpá kàndìmùkwánîsì\\
\gll o-zyú    mu-ntu  shaká  ndi-mu-shé̲sh-e ndi-mu-kwan-is-á̲    kapá  ka-ndi-mu-kwan-í̲s-i\\
\textsc{aug}-\textsc{dem}.\textsc{i}\textsubscript{1} \textsc{np}\textsubscript{1}-person  if  \textsc{sm}\textsubscript{1SG}-\textsc{om}\textsubscript{1}-marry-\textsc{pfv}.\textsc{sbjv}
\textsc{sm}\textsubscript{1SG}-\textsc{om}\textsubscript{1}-fit-\textsc{caus}-\textsc{fv}  or  \textsc{neg}-\textsc{sm}\textsubscript{1SG}-\textsc{om}\textsubscript{1}-fit-\textsc{caus}-\textsc{fv}\\
\glt ‘This person, if I marry her, will I manage her, or will I not manage her?’ (ZF\_Conv13)
\z

\ea
\label{bkm:Ref490824802}
\textbf{shàké} bàkéːzyè bàtùbùrè hànò mbòbátùcìrírè\\
\gll shaké  ba-ké̲ːzy-e    ba-tu\textsubscript{H}-bur-é̲ hano     mbo-bá̲-tu\textsubscript{H}-cirir-é̲ \\
if  \textsc{sm}\textsubscript{2}-come-\textsc{pfv}.\textsc{sbjv}  \textsc{sm}\textsubscript{2}-\textsc{om}\textsubscript{1PL\-}-miss-\textsc{pfv}.\textsc{sbjv}
\textsc{dem}.\textsc{ii}\textsubscript{16} \textsc{near}.\textsc{fut}-\textsc{sm}\textsubscript{2}-\textsc{om}\textsubscript{1PL}-follow-\textsc{pfv}.\textsc{sbjv}\\
\glt ‘If he comes and does not find us here, he will follow us.’ (NF\_Narr15)
\z

\ea
\label{bkm:Ref490824803}
\textbf{shàké} ndíkàhùré ꜝkúnjûò ndìkàráːrà bùryô\\
\gll shaké  ndí̲-ka-hur-é̲        kú-N-júo ndi-ka-rá̲ːr-a      bu-ryó \\
when  \textsc{sm}\textsubscript{1SG}.\textsc{rel}-\textsc{dist}-arrive-\textsc{pfv}.\textsc{sbjv}  \textsc{np}\textsubscript{17}-\textsc{np}\textsubscript{9}-house
\textsc{sm}\textsubscript{1SG}-\textsc{dist}-sleep-\textsc{fv}  \textsc{np}\textsubscript{14}-just\\
\glt ‘When I arrive home, I will just sleep.’ (NF\_Elic17)
\z

\ea
\label{bkm:Ref490824804}
wìná òmùndárè \textbf{sàké} mùwânè mùkàcìncìsá èŋòmbè\\
\gll winá    o-mu-ndaré    saké  mu-wá̲n-e mu-ka-cinc-is-á̲      e-N-ŋombe \\
\textsc{dem}.\textsc{iv}\textsubscript{3}  \textsc{aug}-\textsc{np}\textsubscript{3}-maize  when  \textsc{sm}\textsubscript{2PL}-find-\textsc{pfv}.\textsc{sbjv}
\textsc{sm}\textsubscript{2PL}-\textsc{dist}-change-\textsc{caus}-\textsc{fv}  \textsc{aug}-\textsc{np}\textsubscript{10}-cattle\\
\glt ‘That maize, when you get it, you exchange for cattle.’ (ZF\_Conv13)
\z

The verbal post-initial prefix \textit{shi}- marks a dependent clause with a conditional interpretation, as in (\ref{bkm:Ref99108175}--\ref{bkm:Ref99108176}). This prefix is glossed as ‘conditional’ \textsc{cond}.

\ea
\label{bkm:Ref99108175}
òshìshónj’ ónjòvù òkwàtìwâ\\
\gll o-shi-sho\textsubscript{H}nj-á̲    o-∅-njovu    o-kwat-iw-á̲\\
\textsc{sm}\textsubscript{2SG}-\textsc{cond}-shoot-\textsc{fv}  \textsc{aug}-\textsc{np}\textsubscript{1a}-elephant  \textsc{sm}\textsubscript{1}-catch-\textsc{pass}-\textsc{fv}\\
\glt ‘If you shoot an elephant, you will be caught.’ (NF\_Elic15)
\z

\ea
\label{bkm:Ref99108176}
òshìpángà bútì tùzwírà hábùsò\\
\gll o-shi-pá̲ng-a    bu-tí    tu-zw-í̲r-a      há-bu-so\\
\textsc{sm}\textsubscript{2SG}-\textsc{cond}-do-\textsc{fv}  \textsc{np}\textsubscript{14}-so  \textsc{sm}\textsubscript{1PL}-come\_out-\textsc{appl}-\textsc{fv}  \textsc{np}\textsubscript{16}-\textsc{np}\textsubscript{14}-front\\
\glt ‘If you do like this, we will make a profit.’ (ZF\_Conv13)
\z

\hspace*{-6pt}The conditional prefix \textit{shi-} resembles the post-initial persistive prefix \textit{shí-}, which marks persistive aspect, i.e. a subtype of imperfective aspect that presents an event as still ongoing (see \sectref{bkm:Ref445905502}). It is unclear if conditional \textit{shi-} and persistive \textit{shí-} are two functions of the same morpheme, or accidentally homophonous. According to \citet[148]{Nurse2008}, there are two separate morphemes common in Bantu that are a reflex of *ki-; one expressing persistive, and one expressing a situative, possibly both with a different tone. Persistive \textit{shí-} in Fwe is underlyingly high-toned, but the underlying tones of conditional \textit{shi-} cannot be established, because it is only ever used with verbs in the present construction, and therefore always combines with melodic tone pattern 4, the deletion of underlying tones. It can therefore not be established if the low-toned realization of conditional \textit{shi-} is a reflex of an underlyingly toneless morpheme, or the result of the tonal pattern imposed by the present construction.

There are two strategies for marking counterfactuals, a type of conditional dependent clause in which the condition is presented as not met. The first is to introduce the conditional clause with the marker \textit{nárì}, while the main clause verb is marked with the remoteness prefix \textit{na-/ne-/ni-}, as in (\ref{bkm:Ref99108433}--\ref{bkm:Ref99108434}).

\ea
\label{bkm:Ref99108433}
nárì nóndìtúsì nìndàkùríhì\\
\gll nári  nó̲-ndi-tus-i        ni-ndi-a-ku-rih-í̲\\
if  \textsc{sm}\textsubscript{2SG}.\textsc{pst}-\textsc{om}\textsubscript{1SG}-help-\textsc{npst}.\textsc{pfv}  \textsc{rem}-\textsc{sm}\textsubscript{1SG}-\textsc{pst}-\textsc{om}\textsubscript{2SG}-pay-\textsc{npst}.\textsc{pfv}\\
\glt ‘If you had helped me [but you did not], I would have paid you.’ (NF\_Elic17)
\z

\ea
\label{bkm:Ref99108434}
nárì nómùtúkì nánàkùkùtì\\
\gll nári  nó̲-mu-tuk-í̲        ná̲-na-ku-kut-i\\
if  \textsc{sm}\textsubscript{2SG}.\textsc{pst}-\textsc{om}\textsubscript{1}-insult-\textsc{npst}.\textsc{pfv}  \textsc{rem}-\textsc{sm}\textsubscript{1}.\textsc{pst}-\textsc{om}\textsubscript{2SG}-curse-\textsc{npst}.\textsc{pfv}\\
\glt ‘If you had insulted her/him, s/he would have cursed you.’ (NF\_Elic17)
\z

The remoteness prefix used in a counterfactual is the same remoteness prefix used in, for instance, the remote past perfective. When a counterfactual contains a remote past perfective verb, the remoteness prefix is stacked onto the prefix marking remote past, as in (\ref{bkm:Ref489892745}).

\newpage
\ea
\label{bkm:Ref489892745}
nárì nìmwákêːzyà zyônà \textbf{nìnìmwákêːzyà} kùshàngànà mùyéꜝnzángù\\
\gll nári  ni-mú̲-a-ké̲ːzy-a    zyóna ni-ni-mú-a-ké̲ːzy-a      ku-shangan-a  mu-yénz-angú\\
if  \textsc{pst}-\textsc{sm}\textsubscript{2PL}-\textsc{pst}-come-\textsc{fv}  yesterday
\textsc{rem}-\textsc{pst}-\textsc{sm}\textsubscript{2PL}-\textsc{pst}-come-\textsc{fv}  \textsc{inf}-meet-\textsc{fv}    \textsc{np}\textsubscript{1}-friend-\textsc{poss}\textsubscript{1SG}\\
\glt ‘If you had come yesterday [but you did not], you would have met my friend.’ (NF\_Elic15)
\z

The use of the remoteness prefix to mark temporal remoteness as well as counterfactual meaning can be united in the model developed by \citet{BotneKershner2008}. They conceptualize tense not as a linear timeline, but as a number of separate cognitive “worlds” or domains, which can be associated, i.e. close to the here and now, or dissociated. The remoteness prefix \textit{na-/ne-/ni-} in Fwe could be analyzed as a marker of the dissociated domain, marking temporal remoteness in the case of the remote past perfective or remote future, and marking irrealis in the case of the counterfactual.

Counterfactuals may also contain a conditional clause that lacks a verb, in which case they are introduced by the marker \textit{shárì}, as in (\ref{bkm:Ref99108453}--\ref{bkm:Ref99108454}).

\ea
\label{bkm:Ref99108453}
shárì òmwêzì nèkùsíhà\\
\gll shári  o-mu-ézi    ne-ku-sih-á̲\\
if  \textsc{aug}-\textsc{np}\textsubscript{3}-moon  \textsc{rem}-\textsc{sm}\textsubscript{17}-be\_dark-\textsc{fv}\\
\glt ‘If not for the moon, it would be dark.’ (NF\_Elic17)
\z

\ea
\label{bkm:Ref99108454}
ákùbáꜝtéyè \textbf{shárì} zyùzyú mwâncè nìndáꜝyéndà néyè nìnìndámàn’ óꜝkáfwà\\
\gll á-ku-bá-téye      shári  zyu-zyú  mu-ánce ni-ndí̲-a-é̲nd-a    ne=ye  ni-ni-ndí̲-a-man-á̲      o-ka-fw-á \\
\textsc{con}\textsubscript{1}-\textsc{inf}-\textsc{om}\textsubscript{2}-say\_that  if  \textsc{emph}\textsubscript{1}-\textsc{dem}.\textsc{i}\textsubscript{1}  \textsc{np}\textsubscript{1}-child
\textsc{rem}-\textsc{sm}\textsubscript{1SG}-\textsc{pst}-go-\textsc{fv}  \textsc{com}=\textsc{pers}\textsubscript{3SG}
\textsc{rem}-\textsc{rem}-\textsc{sm}\textsubscript{1SG}-\textsc{pst}-finish-\textsc{fv}  \textsc{aug}-\textsc{inf}.\textsc{dist}-die-\textsc{fv}\\
\glt ‘She told them: if not for this very child, that I went with, I would have died there.’ (NF\_Narr15)
\z
\section{Cleft constructions}
\hypertarget{Toc75352719}{}\label{bkm:Ref491333435}
Cleft constructions are used to mark that a constituent is in focus, meaning that it contains new information, not recoverable from the pragmatic context. However, the use of a cleft construction is not obligatory for presenting new information in Fwe; information can be new or unrecoverable from the pragmatic context even when it is not presented in a cleft construction, as in (\ref{bkm:Ref451515859}), which answers the question ‘what did you buy?’. Although the bicycle is new information and the fact that the speaker bought something is old information, no cleft construction is used to present the new information.

\ea
\label{bkm:Ref451515859}
nìndákàùr’ énjìngà\\
\gll ni-ndí̲-a-ka-ur-á      e-N-jinga\\
\textsc{pst}-\textsc{sm}\textsubscript{1SG}-\textsc{pst}-\textsc{dist}-buy-\textsc{fv}  \textsc{aug}-\textsc{np}\textsubscript{9}-bicycle\\
\glt ‘I bought a bicycle.’ (NF\_Elic15)
\z

Even though a focus interpretation is available outside a cleft construction, clefts are extremely common in Fwe, especially in Zambian Fwe. A cleft construction consists of two clauses, a main clause and a relative clause. The main clause consists of a copulative prefix and a nominal, and the relative clause, which modifies the constituent in the main clause. An example of a cleft construction is given in (\ref{bkm:Ref449691735}), consisting of the clefted element \textit{ndìŋòmbè} ‘it’s a cow’ and the relative clause \textit{ndíꜝ}\textit{shákà} ‘that I want’.

\ea
\label{bkm:Ref449691735}
ndìŋòmbè ndíꜝshákà\\
\glll ndi-N-ŋombe  ndí̲-shak-á̲\\
\textsc{cop}-\textsc{np}\textsubscript{9}-cow  \textsc{sm}\textsubscript{1SG}.\textsc{rel}-want-\textsc{fv}\\
[{clefted element}] [{relative clause}]\\
\glt ‘It’s a cow that I want.’ (NF\_Elic15)
\z

The copulative prefix on the clefted element can be the basic or the definite copulative prefix (which differs in form according to the noun class, see \sectref{bkm:Ref450747606} on the copula), but as clefts are mainly used to present new information, the copulative forms expressing definiteness are rarely used.

The clefted element is always a nominal, but rarely a complex noun phrase. If the noun that is clefted is modified by a connective, only the head noun is clefted, and the connective modifying it is expressed in the relative clause. This is illustrated in (\ref{bkm:Ref505786207}), where the noun \textit{mbóbùrótù} ‘it is good’ is clefted, and the connective \textit{bókùshéshà} modifying it is expressed in the relative clause modifying the clefted element.

\newpage
\ea
\label{bkm:Ref505786207}
kònò mbóbùrótù ndíbwènè bókùshéshà zywìn’ ákìtùtîtè\\
\gll konó  mbó-bu-rótu      ndí̲-bwe\textsubscript{H}ne bu-ó=ku-shésh-a    zywina  á̲-kitut-í̲te\\
but  \textsc{cop}.\textsc{def}\textsubscript{14}-\textsc{np}\textsubscript{14}-good  \textsc{sm}\textsubscript{1SG}.\textsc{rel}-see.\textsc{stat}
\textsc{pp}\textsubscript{14}-\textsc{con}=\textsc{inf}-marry-\textsc{fv}  \textsc{dem}.\textsc{iv}\textsubscript{1}  \textsc{sm}\textsubscript{1}.\textsc{rel}-be\_educated-\textsc{stat}\\
\glt ‘But I think that it is good to marry one who is educated.’ (Literally: ‘It is goodness that I see in marrying an educated one.’) (ZF\_Conv13)
\z

Less complex nominal modifiers, such as a possessive or a numeral, are allowed in the clefted element, as in (\ref{bkm:Ref98513181}--\ref{bkm:Ref98513182}); the clefted element is marked in bold.

\ea
\label{bkm:Ref98513181}
\textbf{ndìwá} \textbf{ꜝ}\textbf{ryángù} kàndíkèkérà\\
\gll ndi-∅-wá    ri-angú  ka-ndí̲-ke\textsubscript{H}ker-á̲\\
\textsc{cop}\textsubscript{5}-\textsc{np}\textsubscript{5}-field  \textsc{pp}\textsubscript{5}-\textsc{poss}\textsubscript{1SG}  \textsc{pst}.\textsc{ipfv}-\textsc{sm}\textsubscript{1SG}-plough-\textsc{fv}\\
\glt ‘It was my field that I was ploughing.’ (ZF\_Elic14)
\z

\ea
\label{bkm:Ref98513182}
\textbf{njìcécì} \textbf{yònkéː} túkàbírà\\
\gll nji-∅-céci    i-onké    tú̲-kabir-á̲\\
\textsc{cop}\textsubscript{9}-\textsc{np}\textsubscript{9}-church  \textsc{pp}\textsubscript{5}-one  \textsc{sm}\textsubscript{1PL\-}.\textsc{rel}-enter-\textsc{fv}\\
\glt ‘It’s the same church that we go to.’ (ZF\_Narr15)
\z

The clefted element does not need to consist of a full noun, but can also consist of a demonstrative, as in (\ref{bkm:Ref449953081}), or a personal pronoun, as in (\ref{bkm:Ref449953090}).

\ea
\label{bkm:Ref449953081}
\textbf{mómò} nìbákìtòbòhérà\\
\gll N-o-mó    ni-bá̲-ki\textsubscript{H}-to\textsubscript{H}boh-er-á̲\\
\textsc{cop}-\textsc{aug}-\textsc{dem}.\textsc{iii}\textsubscript{18}  \textsc{pst}-\textsc{sm}\textsubscript{2}-\textsc{refl}-console-\textsc{appl}-\textsc{fv}<\textsc{rel}>\\
\glt ‘That’s how they consoled themselves.’ (ZF\_Narr15)
\z

\ea
\label{bkm:Ref449953090}
\textbf{ndíw’} ózyâːkà\\
\gll ndi-wé  ó̲-zyá̲ːk-a\\
\textsc{cop}-\textsc{pers}\textsubscript{2SG}  \textsc{sm}\textsubscript{2SG}.\textsc{rel}-build-\textsc{fv}\\
\glt ‘It is you who builds.’ (NF\_Elic15)
\z

The clefted element is modified by a relative clause, which takes the same shape as relative clauses used outside cleft constructions (see \sectref{bkm:Ref491095705}), except that a demonstrative functioning as a relativizer never occurs in a cleft construction.

Any kind of constituent can be clefted; examples are given where the clefted element is a subject in (\ref{bkm:Ref75336656}), an object in (\ref{bkm:Ref75336657}), a locative in (\ref{bkm:Ref75336658}), an adverb in (\ref{bkm:Ref75336660}), and a temporal adverb in (\ref{bkm:Ref75336661}).

\ea
\label{bkm:Ref75336656}
\textbf{ndúmbwá} ábbòzâ\\
\gll ndu-∅-mbwá  á̲-bbo\textsubscript{H}z-á̲\\
\textsc{cop}\textsubscript{1a}-\textsc{np}\textsubscript{1a}-dog  \textsc{sm}\-\textsubscript{1}.\textsc{rel}-bark-\textsc{fv}\\
\glt ‘It’s a dog who barks.’ (ZF\_Elic14)
\z

\ea
\label{bkm:Ref75336657}
hàpé \textbf{ndìgámbùtì} ndízyàbèrè\\
\gll hapé  ndi-∅-gámbuti  ndí̲-zyabere\\
again  \textsc{cop}\textsubscript{5}-\textsc{np}\textsubscript{5}-boot  \textsc{sm}\textsubscript{1SG}.\textsc{rel}-wear.\textsc{stat}\\
\glt ‘Again, it’s boots that I am wearing.’ (ZF\_Narr13)
\z

\ea
\label{bkm:Ref75336658}
shùnù \textbf{kùmùnzì} ndíyà\\
\gll shunu  ∅-ku-mu-nzi    ndí̲-y-a\\
today  \textsc{cop}-\textsc{np}\textsubscript{17}-\textsc{np}\textsubscript{3}-village  \textsc{sm}\textsubscript{1SG}.\textsc{rel}-go-\textsc{fv}\\
\glt ‘Today, it is to home that I go.’ (ZF\_Elic14)
\z

\ea
\label{bkm:Ref75336660}
\textbf{mbóbùryáhò} nìyápàngàhàrírà\\
\gll mbó-bu-ryáho    ni-í̲-a-pang-ahar-ir-á̲\\
\textsc{cop}.\textsc{def}\textsubscript{14}-\textsc{np}\textsubscript{14}-like\_that  \textsc{pst}-\textsc{sm}\textsubscript{9}-\textsc{pst}-do-\textsc{neut}-\textsc{appl}-\textsc{fv}<\textsc{rel}>\\
\glt ‘That is how it happened.’ (ZF\_Narr15)
\z

\ea
\label{bkm:Ref75336661}
\textbf{ndìshúnù} ndàtátìkì kèːzyà kùnù\\
\gll ndi-shúnu  ndi-a-tátik-i      keːzy-a  kunu\\
\textsc{cop}-today  \textsc{sm}\textsubscript{1SG}-\textsc{pst}-start-\textsc{npst}.\textsc{pfv}  come-\textsc{fv}  \textsc{dem}.\textsc{ii}\textsubscript{17}\\
\glt ‘It’s today that I started to come here.’ (ZF\_Elic14)
\z

\hspace*{-2.7pt}Cleft constructions can be embedded into longer sentences, where a constituent can be moved to the position before the clefted element (see also \sectref{bkm:Ref403656711} on left dislocation). This left-dislocated constituent behaves like other left-dislocated constituents in that it functions as a topic, and that it is prosodically marked as extraclausal, i.e. it is affected by phrase-final tonal processes such as high tones realized as falling, as in the left-dislocated constituent \textit{òbûcì} in (\ref{bkm:Ref492285101}).

\ea
\label{bkm:Ref492285101}
òbûːcì ndìmpùká názàbúpàngà\\
\gll o-búː-ci    ndi-N-puká    ná̲-zi-a-bú-pang-a\\
\textsc{aug}-\textsc{np}\textsubscript{14}-honey  \textsc{cop}-\textsc{np}\textsubscript{10}-bee  \textsc{pst}-\textsc{sm}\textsubscript{10}-\textsc{pst}-\textsc{om}\textsubscript{14}-make-\textsc{fv}<\textsc{rel}>\\
\glt ‘Honey, it’s bees who make it.’ (ZF\_Elic14)
\z

Cleft constructions are used to mark focus on the clefted element, as in (\ref{bkm:Ref449953409}), which answers the question ‘when did you become ill?’. The speaker becoming ill is old information, but the time at which this happens is not. To mark this as new information, the speaker uses a cleft construction.

\ea
\label{bkm:Ref449953409}
ndìzyónà nàndárwârà\\
\gll ndi-zyóna    na-ndí̲-a-rwá̲r-a\\
\textsc{cop}-yesterday  \textsc{pst}-\textsc{sm}\textsubscript{1SG}-\textsc{pst}-become\_sick-\textsc{fv}<\textsc{rel}>\\
\glt ‘It was yesterday that I became sick.’ (ZF\_Elic14)
\z

Cleft constructions are not only used to mark information as new, but also to mark information as contradicting the beliefs of the hearer (or rather, the beliefs that the speaker assumes the hearer has), called ‘counter-presuppositional focus’ by \citet[332]{Dik1997}. This is illustrated in (\ref{bkm:Ref449954057}), which contains direct speech taken from a narrative in which a girl becomes angry at a rabbit who is weeding in her field, pulling out crops instead of weeds. The girl corrects the rabbit by explaining that it is not maize that people usually weed, but grass, using a cleft construction.

\ea
\label{bkm:Ref449954057}
ndìsózú ꜝbárìmângà\\
\gll ndi-∅-sozú    bá̲-rim-á̲ng-a\\
\textsc{cop}\textsubscript{5}-\textsc{np}\textsubscript{5}-grass  \textsc{sm}\textsubscript{2}.\textsc{rel}-weed-\textsc{hab}-\textsc{fv}\\
\glt ‘It’s grass that people usually weed.’ (NF\_Narr15)
\z

Another example where a cleft construction marks counter-presuppositional focus is given in (\ref{bkm:Ref492285208}), from a conversation between two sisters which is part of a narrative. Previously, the older sister did not believe her younger sister; now that the younger sister has provided proof, the older sister concedes that she was in fact right.

\ea
\label{bkm:Ref492285208}
njíꜝnítì wákùàmbà\\
\gll njí-N-níti    ó̲-aku-amb-a\\
\textsc{cop}\textsubscript{9}-\textsc{np}\textsubscript{9}-truth  \textsc{sm}\textsubscript{2SG}.\textsc{rel}-\textsc{npst}.\textsc{ipfv}-speak-\textsc{fv}\\
\glt ‘It’s the truth that you were speaking.’ (NF\_Narr15)
\z

Another type of focus for which cleft constructions are used is exclusive or restrictive focus; the speaker uses a cleft construction to indicate that only the referent in focus, and no other, is meant, combined with the adverb \textit{bùryò} ‘only’, as in (\ref{bkm:Ref99109456}).

\ea
\label{bkm:Ref99109456}
màbéré bùryò ndíbyârà\\
\gll N-ma-beré    bu-ryo  ndí̲-byá̲r-a\\
\textsc{cop}-\textsc{np}\textsubscript{6}-millet  \textsc{np}\textsubscript{14}-only  \textsc{sm}\textsubscript{1SG}.\textsc{rel}-plant-\textsc{fv}\\
\glt ‘It’s only millet that I plant.’ (ZF\_Elic14)
\z

Cleft constructions can also mark thetic focus, where all the information is new and therefore the entire utterance is in focus, and not just one constituent. Though only one element (either the subject or the object) is clefted, the entire construction is interpreted as being in focus. This is illustrated in (\ref{bkm:Ref449701908}); the context for this utterance is that a noise was heard, and the speaker was asked what happened. Neither the breaking nor the fact that it was a cup that broke are known to the hearer, yet only the cup is marked as the clefted element, and the verb expressing the breaking, though equally focal, is expressed in the relative clause.

\ea
\label{bkm:Ref449701908}
njìnkómókí yàpwàcûkì\\
\gll nji-N-komokí  i-a-pwacú̲k-i\\
\textsc{cop}\textsubscript{9}-\textsc{np}\textsubscript{9}-cup  \textsc{sm}\textsubscript{9}-\textsc{pst}-break-\textsc{npst}.\textsc{pfv}\\
\glt ‘A cup broke.’ (NF\_Elic15)
\z

Another example of thetic focus using a cleft is given in (\ref{bkm:Ref449953497}). In this context, the speaker was asked if his wife is at home. Although the hearer does not know that the wife is fetching something, nor what she is fetching, only the constituent \textit{ménjì} ‘water’ is expressed as the clefted element, and the verb \textit{báꜝ}\textit{tékà} ‘she fetches’ is expressed in the relative clause.

\ea
\label{bkm:Ref449953497}
tàbènáhò ménjì báꜝtékà\\
\gll ta-ba-ina=hó̲    N-ma-ínji    bá̲-te\textsubscript{H}k-á̲\\
\textsc{neg}-\textsc{sm}\textsubscript{2}-be=\textsc{loc}\textsubscript{16}  \textsc{cop}-\textsc{np}\textsubscript{6}-water  \textsc{sm}\textsubscript{2}.\textsc{rel}-fetch-\textsc{fv}\\
\glt ‘She’s not here, she’s fetching water.’ (ZF\_Elic14)
\z

In order to focus a verb, a fronted-infinitive construction (FIC) is used, which is essentially a cleft construction in which the inflected verb is copied as an infinitive and clefted. The infinitive form which forms the clefted element is an infinitive, which behaves like a noun of class 15. As the infinitive functions as a clefted element, it is marked with a copulative prefix, which is realized as zero before a voiceless consonant (see \sectref{bkm:Ref489963307} on the form of copulatives), as in (\ref{bkm:Ref99109592}). The copula also has a form which is used on definite constituents, and for class 15, this form of the copula is (\textit{n)kó-}. This definite copula can also be used to mark the infinitive in a FIC, as in (\ref{bkm:Ref99109635}).

\ea
\label{bkm:Ref99109592}
kùshèkà báꜝshékà\\
\gll ∅-ku-shek-a    bá̲-shek-á̲\\
\textsc{cop}-\textsc{np}\textsubscript{15}-laugh-\textsc{fv}  \textsc{sm}\textsubscript{2}.\textsc{rel}-laugh-\textsc{fv}\\
\glt ‘They laugh.’
\z

\ea
\label{bkm:Ref99109635}
kókùmànà ndíꜝmánà\\
\gll kó-ku-man-a    ndí̲-man-á̲\\
\textsc{cop}.\textsc{def}-\textsc{np}\textsubscript{15}-finish-\textsc{fv}  \textsc{sm}\textsubscript{1SG}.\textsc{rel}-finish-\textsc{fv}\\
\glt ‘I just finished.’ (ZF\_Elic14)
\z

The FIC is also used to mark progressive aspect. This use, as well as other formal aspects of the construction, are discussed in \sectref{bkm:Ref431917326}. The focus use of the FIC is illustrated in (\ref{bkm:Ref451515872}), in which the speaker warns someone not to drink the tea yet, as it is still cooling down.

\ea
\label{bkm:Ref451515872}
èntîː kùhórà íꜝhórà\\
\gll e-N-tíː  ku-hór-a  í̲-ho\textsubscript{H}r-á̲\\
\textsc{aug}-\textsc{np}\textsubscript{9}-tea  \textsc{inf}-cool-\textsc{fv}  \textsc{sm}\textsubscript{9}.\textsc{rel}-cool-\textsc{fv}\\
\glt ‘The tea is cooling down.’ (ZF\_Elic14)
\z

Another example of the use of the FIC to express focus on the verb is given in (\ref{bkm:Ref449958275}), which is the answer to the question ‘what did you do today?’.

\ea
\label{bkm:Ref449958275}
kùkékèrà kàndíkèkérà\\
\gll ku-kéker-a    ka-ndí̲-ke\textsubscript{H}ker-á̲\\
\textsc{inf}-plough-\textsc{fv}  \textsc{pst}.\textsc{ipfv}-\textsc{sm}\textsubscript{1SG}-plough-\textsc{fv}\\
\glt ‘I was ploughing.’ (ZF\_Elic14)
\z

In many cases where the FIC marks verb focus, the verb is also interpretable as progressive. There are, however, examples of the fronted-infinitive construction where the verb is in focus, but not progressive. This is the case in (\ref{bkm:Ref449959063}), where the inflected verb of the FIC is in the near past perfective, which is incompatible with a progressive interpretation (see \sectref{bkm:Ref488767759} on the near past perfective). This sentence is uttered in a context where an injured child is brought to the clinic, and the clinic personnel asks how the injury came about.

\ea
\label{bkm:Ref449959063}
òmwâncè kùgwà nâgwì\\
\gll o-mu-ánce    ku-gw-a  ná̲-gw-i\\
\textsc{aug}-\textsc{np}\textsubscript{1}-child  \textsc{inf}-fall-\textsc{fv}  \textsc{sm}\textsubscript{1}.\textsc{pst}-fall-\textsc{npst}.\textsc{pfv}\\
\glt ‘The child has fallen down.’ (ZF\_Elic14)
\z

The use of the FIC differs between Namibian and Zambian Fwe. In Zambian Fwe, a simple present verb may not occur on its own, as in (\ref{bkm:Ref99109710}), but only in a FIC, as in (\ref{bkm:Ref99109711}).

\ea[*]{
\label{bkm:Ref99109710}
ndìshékà\\
\gll ndi-shek-á̲\\
\textsc{sm}\textsubscript{1SG}-laugh-\textsc{fv}\\
\glt Intended: 'I am laughing/I laugh.'
}
\z

\ea
\label{bkm:Ref99109711}
kùshèkà ndíꜝshékà\\
\gll ku-shek-a  ndí̲-shek-á̲\\
\textsc{inf}-live-\textsc{fv}  \textsc{sm}\textsubscript{1SG}.\textsc{rel}-laugh-\textsc{fv}\\
\glt ‘I am laughing/I laugh.’ (ZF\_Elic14)
\z

A verb may occur without the FIC if it is combined with an object, an adverb or a subject, though in the latter case the use of the FIC is still preferred. In Namibian Fwe, however, an inflected verb is allowed outside the FIC, even if no other constituent is present. The use of the FIC in Zambian Fwe whenever the verb is the only element in the sentence is related to the focal meaning of the FIC; when no other constituent is present, focus must be marked on the verb.

Cleft constructions are also used in questions, where the question word functions as the clefted element. This is illustrated with the question words \textit{ni} ‘who’, \textit{nji} ‘what’, \textit{kwí} ‘where’, and \textit{bu-ti} ‘how’ in (\ref{bkm:Ref98833583}--\ref{bkm:Ref98833585}).

\ea
\label{bkm:Ref98833583}
ndìní náàŋánk’ òndôngò\\
\gll ndi-ní    ná̲-a-ŋá̲nk-a      o-∅-ndóngo\\
\textsc{cop}-who  \textsc{sm}\textsubscript{1}-\textsc{pst}-peel-\textsc{fv}<\textsc{rel}>  \textsc{aug}-\textsc{np}\textsubscript{1a}-groundnut\\
\glt ‘Who has peeled the groundnuts?’ (ZF\_Elic14)
\z

\ea
cìnjí bátêndà\\
\gll ∅-ci-njí    bá̲-té̲nd-a\\
\textsc{cop}-\textsc{np}\textsubscript{7}-what  \textsc{sm}\textsubscript{2}.\textsc{rel}-do-\textsc{fv}\\
\glt ‘What are they doing?’
\z

\ea
nkòkwí ꜝmúyà\\
\gll N-kokwí  mú̲-y-a\\
\textsc{cop}-where  \textsc{sm}\textsubscript{2PL}-go-\textsc{fv}\\
\glt ‘Where are you going?’ (NF\_Elic15)
\z

\ea
\label{bkm:Ref98833585}
mbùtí mwàbûːkì\\
\gll N-bu-tí    mu-a-búːk-i\\
\textsc{cop}-\textsc{np}\textsubscript{14}-how  \textsc{sm}\textsubscript{2PL}-\textsc{pst}-wake-\textsc{npst}.\textsc{pfv}\\
\glt ‘How did you wake up?’ (morning greeting)
\z
