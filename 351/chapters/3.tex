\chapter{Tone}
\label{bkm:Ref451507583}\hypertarget{Toc75352616}{}
Like most Bantu languages, Fwe is a tone language: the relative pitch at which a vowel is articu\-lated is phonologically contrastive. This is illustrated by tonal minimal pairs in (\ref{bkm:Ref478999474}--\ref{bkm:Ref445299219}), words that are identical on the segmental level, but have different tones and a different meaning.
\NumTabs{6}
\ea
\label{bkm:Ref478999474}
kùhârà \tab  - \tab kùhàrà\\
ku-hár-a \tab\tab    ku-har-a\\
\textsc{inf}-live-\textsc{fv}   \tab\tab \textsc{inf}-scrape-\textsc{fv}\\
\glt ‘to live’  \tab\tab  ‘to scrape’
\z

\ea
évù \tab    - \tab èvù\\
e-∅-vú  \tab\tab  e-∅-vu\\
\textsc{aug}-\textsc{np}\textsubscript{5}-sand \tab \textsc{aug}-\textsc{np}\textsubscript{5}-wasp\\
\glt ‘sand, soil, land’ \tab ‘wasp’
\z

\ea
\label{bkm:Ref445299219}
màsírà  \tab  - \tab màsìrà\\
ma-sirá  \tab\tab  ma-sira\\
\textsc{np}\textsubscript{6}-cloth   \tab\tab \textsc{np}\textsubscript{6}-dirt\\
\glt ‘pieces of cloth’ \tab ‘dirt’
\z

Tone also plays an important role in the grammar of Fwe. A tonal distinction is used, for in\-stance, in distinguishing main clause verbs from relative clause verbs. A main clause verb has a low-toned subject marker, as in (\ref{bkm:Ref98929708}), and a relative clause verb has a high-toned subject marker, as in (\ref{bkm:Ref98929728}); other than these tonal differences, main clause verbs and relative clause verbs are identical in terms of segmen\-tal material (for most TAM constructions; a detailed overview of the tonal and other differ\-ences between relative clause verbs and main clause verbs is given in \sectref{bkm:Ref491095705}).

\ea
\label{bkm:Ref98929708}
báncè bàzânà\\
\gll ba-ánce  ba-zá̲n-a\\
\textsc{np}\textsubscript{2}-child  \textsc{sm}\textsubscript{2}-play-\textsc{fv}\\
\glt ‘The children play.’
\z

\ea
\label{bkm:Ref98929728}
báncè bázânà\\
\gll ba-ánce  bá̲-zá̲n-a\\
\textsc{np}\textsubscript{2}-child  \textsc{sm}\textsubscript{2}.\textsc{rel}-play-\textsc{fv}\\
\glt ‘The children who play…’ (NF\_Elic15)
\z

Underlyingly, Fwe has a two-tone system. Through various tonal processes, tones may be realized as high (H), low (L), falling (F) and downstepped high (ꜝH). These tonal pro\-cesses, discussed in \sectref{bkm:Ref445214894}, only affect high tones, showing that Fwe can be analyzed as hav\-ing a privative system, where only high tones are represented underlyingly (\citealt{Hyman2001}; \citealt{OddenMarlo2019}). Toneless moras (symbolized as ∅) surface as low-toned, unless a melodic high tone is assigned, or the mora is targeted by a specific tonal process. Furthermore, the system of melodic tones, which are assigned by a specific tense/aspect/mood construction to a specific syllable or mora of the verb, only makes use of high tones (melodic tone is discussed in \sectref{bkm:Ref71539267}). Fwe has floating high tones (discussed in \sectref{bkm:Ref71539878}), but no floating low tones. In the analysis of tone languages, the presence of a floating low tone is sometimes evoked to account for the occurrence of downstep. Although downstep occurs in Fwe, \sectref{bkm:Ref445296627} shows that it is a purely phonetic process, and is not influenced by putative underlying low tones.

The relevant unit for tonal analysis in Fwe is the mora, not the syllable. Long vowels and automatically lengthened vowels consist of two moras, all short vowels, or vowels targeted by penultimate lengthening, consist of one mora (see \sectref{bkm:Ref489974139} on vowels). These non-contrastive types of lengthening are not indicated in the orthography in this book, to distinguish them from phonemic vowel length. In this chapter, and when necessary, bimoraic vowels are written with two vowel signs in between periods marking syllable boundaries, e.g. /.ee./, as opposed to two vowels separated by a period, which mark two separate short vowels in two separate syllables, e.g. /e.e/.

The following tonal transcriptions are used, both in this chapter and throughout this book. In the phonetic transcription (the first line of examples), high tones are marked with acute accent, low tones are marked with grave accent, falling tones are marked with a circumflex, down-stepped high tones are marked with ꜝ preceding the high-toned vowel. In the phonological transcription (the second line of examples), underlying high tones are marked with acute accent, melodic high tones are marked by acute accent combined with underlining of the vowel, and underlying high tones that are deleted as the result of a specific melodic tone pattern are represented by \textsubscript{H} (see also \sectref{bkm:Ref71539267} on melodic tones).

\section{Tonal processes}
\label{bkm:Ref445214894}\hypertarget{Toc75352617}{}
This section discusses the tonal processes that play a role in Fwe. These processes determine where and how an underlying high tone is maintained, deleted, shifted, copied or modified. To\-nal processes are conditioned by their phonological, morphological and syntactic environments. Phonological criteria that influence tonal processes are vowel length and phonetic vowel lengthen\-ing; the latter is in turn is conditioned by the nature of the consonants following or pre\-ceding a vowel. Tonal processes are also influenced by penultimate lengthening, which in turn is conditioned by syntactic criteria. Morphological criteria that can play a role in the applica\-tion of tone rules are the morphological structure of the word and the position of morpheme boundaries; high tone spread (see \sectref{bkm:Ref430865664}), for instance, is blocked by certain morpheme boundaries. The syntactic environment plays a role in the application of tonal rules, because some rules only apply at the end of a phrase.

Tonal processes also interact with each other. Certain tone rules only affect tones that are the result of an earlier rule, whereas others only apply to tones that are not the result of an earlier rule. This suggests that the application of tonal processes follows a set order, which is set out in \sectref{bkm:Ref432073870}. A schematic overview of tone rules is given in \tabref{tab:3:1}.

\begin{table}
\label{bkm:Ref489974335}\caption{\label{tab:3:1}Tone rules}
\begin{tabular}{l>{\raggedright}p{\widthof{∅∅∅H > [HHHH]}}l}

\lsptoprule

Rule & Schematization & Section\\
\midrule
Meeussen’s Rule & HH > /H∅/ & \ref{bkm:Ref440987952}\\
Downstep & HH > [HꜝH]

HLH > [HLꜝH] & \ref{bkm:Ref445296627}\\
Bimoraic doubling & H∅. > HH.

∅H. > HH. & \ref{bkm:Ref486506295}\\
H retraction & ∅H\# > [HL]\# & \ref{bkm:Ref486506264}\\
H > F & H\# > [F]\#

H.∅ > [F.L]\# & \ref{bkm:Ref432074291}\\
H tone spread & ∅H > [HH]

∅∅H > [HHH]

∅∅∅H > [HHHH]

etc. & \ref{bkm:Ref486506344}\\
\lspbottomrule
\end{tabular}
\end{table}
\subsection{Meeussen’s Rule}
\label{bkm:Ref440987952}\hypertarget{Toc75352618}{}
Fwe makes use of Meeussen’s Rule, a tone rule that affects sequences of two adjacent high tones by deleting the second high tone, as schematized in (\ref{bkm:Ref479082622}).

\ea
\label{bkm:Ref479082622}
  Meeussen’s Rule: /HH/ > [HL]
\z

This tone rule is found in various Bantu languages (\citealt{KisseberthOdden2003}), and has come to be known as Meeussen’s Rule after {\citet{Goldsmith1984}}, who refers to the descrip\-tion of the rule in Tonga by \citet{Meeussen1963}. Meeussen’s Rule is one of two tone rules in Fwe which follow the Oblig\-atory Contour Principle, or OCP, a general tendency to avoid successive high tones (\citealt{KisseberthOdden2003}; \citealt{OddenMarlo2019}). The other tone rule that follows the OCP is downstep (see \sectref{bkm:Ref445296627}), which affects two successive high tones by lowering the second high tone to a mid tone. Although both these tonal processes affect sequences of successive high tones, only Meeussen’s Rule deletes high tones, whereas downstep lowers the pitch of high tones but keeps them recog\-nizable as high. There are a number of differences in the ways Meeussen’s Rule and downstep are conditioned. First, Meeussen’s Rule only affects high tones on adjacent moras, whereas downstep also affects high tones which are on adjacent syllables but are separated by a toneless mora. Second, Meeussen’s Rule does not occur across word boundaries, whereas downstep does. Third, Meeussen’s Rule does not target high tones that are the result of H retraction, whereas downstep does. Meeussen’s Rule is applied before downstep: in situations where both may apply, Meeussen’s Rule is applied instead of downstep. The diachronic application of Meeussen’s Rule in Fwe was already noted by {\citet[122]{Bostoen2009}}. This section shows that Meeussen’s Rule is still active synchronically in Fwe.

The application of Meeussen’s Rule is illustrated in (\ref{bkm:Ref71530778}): the high tone of the syllable \textit{bbá{} } is deleted when immediately preceded by a high-toned object marker \textit{zí{} \-}-.

\ea
\label{bkm:Ref71530778}
\ea
\glll kùbbátùrà\\
ku-bbát-ur-a\\
\textsc{inf}-separate-\textsc{sep}.\textsc{tr}-\textsc{fv}\\
\glt ‘to separate’

\ex
kùzíbbàtùrà\\
ku-zí-bbát-ur-a > ku-zí-bbat-ur-a\\
\textsc{inf}-\textsc{om}\textsubscript{8}-separate-\textsc{sep}.\textsc{tr}-\textsc{fv}\\
\glt ‘to separate them’
\z\z

Meeussen’s Rule is applied repeatedly from right to left: in a sequence of more than two high tones, all high tones are deleted except for the first, as schematized and illustrated in (\ref{bkm:Ref488661088}).

\ea
\label{bkm:Ref488661088}
  Repeated application of Meeussen’s Rule
  \ea
  /HHH/ > [HLL]

\ex
cázyùrì \\
ci-á-zyúr-í̲ > ci-á-zyur-i\\
\textsc{sm}\textsubscript{7}-\textsc{pst}-become\_full-\textsc{npst}.\textsc{pfv}\\
\glt ‘It has become full.’
\z\z

Meeussen’s Rule only affects high tones on adjacent moras. When a high tone is followed by an\-other high tone that is on an adjacent syllable, but not an adjacent mora, Meeussen’s Rule does not apply, as schematized in (\ref{bkm:Ref471290207}), and illustrated in (\ref{bkm:Ref69985943}): the high tone in the bimoraic syllable /tée/ does not trigger the application of Meeussen’s Rule to the high tone in the subsequent syllable /ndé/, because of the intervening toneless mora.

\ea
\label{bkm:Ref471290207}
No Meeussen’s Rule on H∅.H sequences:\\
/H∅.H/\\
HH.H    (bimoraic doubling: H is copied from the first to the second mora)\\
{}[Hː.H]\\
*[Hː.L]
\z

\ea
\label{bkm:Ref69985943}
/ma.tée.ndé a.ngú/ > màtéːꜝndé ꜝáːngù\\
\gll ma-téndé  a-angú\\
\textsc{np}\textsubscript{6}-foot  \textsc{pp}\textsubscript{6}-\textsc{poss}\textsubscript{1SG}\\
\glt ‘my feet’
\z

Meeusen’s Rule only applies within the word, and adjacent high tones separated by a word boundary are not subject to Meeussen’s Rule. In (\ref{bkm:Ref98510009}), the high tone of the syllable /njí/ does not cause the high tone of the following syllable /ndí-/ to be deleted, as the two high tones are separated by a word boundary.

\ea
\label{bkm:Ref98510009}
cìnjí ꜝndímìtàhwîːrà?\\
\gll ∅-ci-njí    ndí̲-mi\textsubscript{H}-ta\textsubscript{H}hw-í̲r-a\\
\textsc{cop}-\textsc{np}\textsubscript{7}-what  \textsc{sm}\textsubscript{1SG}.\textsc{rel}-\textsc{om}\textsubscript{2PL}-divide-\textsc{appl}-\textsc{fv}\\
\glt ‘What can I give you?’ (NF\_Elic15)
\z

Meeussen’s Rule precedes all other tone rules, as seen from the fact that high tones which have been influenced by other, phonetic tone rules are not subject to Meeussen’s Rule. This is the case for sequences of high tones that were created by H retraction (see \sectref{bkm:Ref486506264}). In (\ref{bkm:Ref494187697}), the high tone of the syllable \textit{rú} and the high tone of the syllable \textit{há} are only adjacent as the result of H retraction, and therefore are not affected by Meeussen’s Rule.

\ea
\label{bkm:Ref494187697}
\glll rúꜝhátì\\
rú-hatí\\
\textsc{np}\textsubscript{11}-rib\\
\glt ‘a rib’
\z

Adjacent high tones that are not subject to Meeussen’s Rule, either because they are separated by a toneless mora, because they are separated by a word boundary, or because they are the result of other tone rules, are subject to downstep. The use of downstep is discussed in the following section.

\subsection{Downstep}
\label{bkm:Ref445296627}\hypertarget{Toc75352619}{}
Another manifestation of the Obligatory Contour Principle in Fwe is the rule of downstep, which lowers a high tone to a mid tone. Downstep applies to every high tone that is preceded by another high tone somewhere in the phrase. Downstep affects adjacent high tones, as schematized in (\ref{bkm:Ref479087691}), but also high tones that are not in adjacent syllables, but are separated by one or more low-toned syllables, as schematized in (\ref{bkm:Ref479089993}).

\ea
\label{bkm:Ref479087691}
  Downstep on adjacent high tones: HH > [HꜝH]
\z

\ea
\label{bkm:Ref479089993}
  Downstep on non-adjacent high tones: HLH > [HLꜝH]
\z

Downstep across overt low-toned syllables is usually referred to as downdrift, or automatic down\-step \citep{Connell2011}. In Fwe, both downstep on adjacent high tones and downstep on non-adjacent high tones are manifestations of the same process, and downstep can be accurately ana\-lyzed as targeting any high tone but the first in a phrase.

The occurrence of downstep in Fwe differs from the occurrence of downstep and downdrift in many other African languages, where they are analyzed as the result of an interven\-ing low tone; a surface low tone in the case of downdrift, and an underlying low tone in the case of phonemic downstep \citep[148]{Yip2002}. In Fwe, however, intervening low tones are not required to trigger downstep, because downstep also occurs on adjacent high tones where there is no overt intervening low tone. This is shown in (\ref{bkm:Ref479153737}), where the high tone of the syllable /ká/ is directly followed by that of the syllable /bá/, causing the second to be downstepped.

\ea
\label{bkm:Ref479153737}
\gll /bu-kábabú/ > bu-kábábu (H retraction) > [bù-káꜝbábù]\\
\textsc{np}\textsubscript{14}-problem\\
\glt ‘problem’
\z

It is not possible to analyze examples such as (\ref{bkm:Ref479153737}) by attributing downstep to the toneless mora that intervenes between the two high tones. Such a reanalysis would involve analyzing toneless moras as underlyingly low-toned, rather than underlyingly toneless, and there is no evidence for the existence of underlying low tones elsewhere in the tonal system. Furthermore, downstep across word boundaries also gives clear examples of downstep not triggered by intervening toneless (or low-toned) moras, as in (\ref{bkm:Ref69985972}).

\ea
\label{bkm:Ref69985972}
\gll /ndi-y-á̲  kú-mu-nzi/ > [ndìyá ꜝkúmùːnzì]\\
\textsc{sm}\textsubscript{1SG}-go-\textsc{fv}  \textsc{np}\textsubscript{17}-\textsc{np}\textsubscript{3}-village\\
\glt ‘I go home.’
\z

Downstep between any two high tones, without an intervening low tone, is also described for the Bantu language Shambaa \citep{Odden1982}. See {\citet{Odden1986}} for a theoretical account of downstep not introduced by low tones.

Throughout this book, only downstep triggered by an immediately preceding high tone will be marked, in order to distinguish it from two adjacent surface high tones that are the result of high tone spread (see \sectref{bkm:Ref430865664}). Downstep triggered by a high tone across one or more low tones (i.e. what is more commonly referred to as downdrift) will not be marked, except in the current section.

Downstep, like Meeussen’s Rule, is a manifestation of the Obligatory Contour Principle: both processes reduce the number of high tones that are realized on the same pitch. The main differences between the two processes are summed up in \tabref{tab:3:2}, and will subse\-quently be discussed and illustrated.

\begin{table}
\label{bkm:Ref496783734}\caption{\label{tab:3:2}Differences between Meeussen’s Rule and Downstep}
\begin{tabularx}{\textwidth}{Xl}
\lsptoprule
Meeussen’s Rule & Downstep\\
\midrule
deletes high tones & lowers high tones\\
only affects adjacent moras & affects adjacent and non-adjacent moras\\
only word-internally & word-internally and across word bounda\-ries\\
before H retraction & after H retraction\\
\lspbottomrule
\end{tabularx}
\end{table}

\begin{sloppypar}
Meeussen’s Rule only applies word-internally, but downstep applies both word-internally, as in (\ref{bkm:Ref459733972}), and across word boundaries, as in (\ref{bkm:Ref71531956}).
\end{sloppypar}

\ea
\label{bkm:Ref459733972}
\gll /bu-kábabú/ > bukábábú > [bùkáꜝbábù]\\
\textsc{np}\textsubscript{14}-problem\\
\glt ‘problem’
\z

\ea
\label{bkm:Ref71531956}
\gll /N-shukí  zi-ó=mu-kéntu/ > [nshùkí ꜝzómùkêːntù]\\
\textsc{np}\textsubscript{10}-hair  \textsc{pp}\textsubscript{10}-\textsc{con}=\textsc{np}\textsubscript{1}-woman\\
\glt ‘the hair of the woman’ (ZF\_Elic14)
\z

Word-internally, downstep and Meeussen’s Rule are conditioned differently. Meeussen’s Rule only applies to high tones on adjacent moras, whereas downstep applies to all high tones, including those separated by one or more toneless moras, as in (\ref{bkm:Ref99552163}--\ref{bkm:Ref99552164}).

\ea
\label{bkm:Ref99552163}
H-toned moras separated by one toneless mora: Downstep\\
\gll /ku-táand-á    ba-ntu/ > [kùtáːꜝndá bàntù]\\
\textsc{inf}-chase-\textsc{fv}    \textsc{np}\textsubscript{2}-person\\
\glt ‘to chase people’
\z

\ea
\label{bkm:Ref99552164}
  H-toned moras separated by more than one toneless mora: Downstep\\

\gll mbo-ndí̲-ci\textsubscript{H}-to\textsubscript{H}rok-é̲ > [mbòːndícìtòꜝrókè]\\
\textsc{near}.\textsc{fut}-\textsc{sm}\textsubscript{1SG}-\textsc{om}\textsubscript{7}-explain-\textsc{pfv}.\textsc{sbjv}\\
\glt ‘I will explain it.’
\z

Furthermore, Meeussen’s Rule precedes the application of H retraction, but downstep follows H retraction, as can be seen from the fact that retracted high tones are subject to downstep, but not to Meeussen’s Rule.

\ea
 /bu-kábabú/\\
bu-kábábu  high tone retraction\\
bu-káꜝbábu  downstep\\
{}[bùkáꜝbábù]\\
\glt ‘problem’
\z

Falling tones, occurring in the last or penultimate syllable of a phrase (see \sectref{bkm:Ref432074291}), may also be subject to downstep, in which they case the starting pitch of the falling contour tone is lower than in a non-downstepped falling tone.

Downstep is progressive: for each subsequent high tone, the pitch is lowered. Examples of succes\-sive downsteps are given in (\ref{bkm:Ref445218491}--\ref{bkm:Ref98833717}): in each case, the downstep indicates an additional pitch lowering.

\ea
\label{bkm:Ref445218491}
\gll /N-mwa-Imushó  ndí̲-ha\textsubscript{H}r-á̲/ >[mwàìmúshó ꜝndíꜝhárà]\\
\textsc{cop}-\textsc{np}\textsubscript{18}-Imusho  \textsc{sm}\textsubscript{1SG}.\textsc{rel}-live-\textsc{fv}\\
\glt ‘I live in Imusho.’
\z

\ea
\label{bkm:Ref98833717}
\gll /zi-ryó  zí-cenyá / > [zìryó ꜝzíꜝcényà]\\
\textsc{np}\textsubscript{8}-food  \textsc{np}\textsubscript{8}-lion\\
\glt ‘the ears of the lion’ (ZF\_Elic\_2014)
\z

Although sequences of up to three successive downsteps have been attested, pitch cannot be low\-ered indefinitely, and at a certain point in speech, the pitch is reset to its original quality and a new series of downsteps may be initiated. More research is needed to determine at which point in speech the pitch is restored. One possibility is that the pitch ceiling is reset after the prosodic bound\-ary that is marked by the processes of penultimate lengthening, high tone retraction, and the realization of high tones as falling. Another possibility is that the pitch is reset when the speaker has reached his or her bottom reach and/or stops for breath, in which case the limits of downstepping may be related to the number of downsteps. More research is needed to clarify these issues.

\subsection{Bimoraic doubling}
\label{bkm:Ref432074325}\hypertarget{Toc75352620}{}\label{bkm:Ref486506295}
As discussed in the introduction, the mora is the relevant tone-bearing unit, and syllables can have two moras, in the case of a phonemically long or automatically lengthened vowel, or one mora. The two moras of a bimoraic syllable behave independently when it comes to high tone assignment, and tone rules such as high tone retraction, Meeussen’s Rule and downstep. After the assignment of high tones and the application of tone rules, however, a high tone associ\-ated with one mora of a bimoraic syllable will automatically be copied onto the other mora of that syllable. This is illustrated in (\ref{bkm:Ref69986003}), where the high tone associated with the last syllable will retract to the second mora of the penultimate syllable in phrase-final context, and is subsequently copied to the first mora of the penultimate syllable in order to avoid a rising tone. For the sake of clarity, the two moras are transcribed with separate vowel sysmbols, rather than with the lengthening symbol ː, and a dot . marking syllable boundaries is added to indicate that the two moras together form a single syllable.

\ea
\label{bkm:Ref69986003}
/ka.roo.ngó/\\
ka.roó.ngo\#  after H retraction\\
{[kà.róó.ngò]}  after bimoraic doubling
\z

Bimoraic doubling serves to avoid all contour tones, both rising and falling. An example of bi\-moraic doubling to avoid a falling contour tone is given in (\ref{bkm:Ref479242278}), where a high tone assigned to the second mora of the bimoraic syllable \textit{yií} is copied to the first mora to create a level high tone.

\ea
\label{bkm:Ref479242278}
\gll /N-ma-yií.    ndí̲-hi\textsubscript{H}b-á̲/ > [màyíː ꜝndíꜝhíbà]\\
\textsc{cop}-\textsc{np}\textsubscript{6}-egg    \textsc{sm}\textsubscript{1SG}.\textsc{rel}-steal-\textsc{fv}\\
\glt ‘It’s eggs that I steal.’
\z

Although bimoraic doubling is obligatory, contour tones do occur in Fwe, namely falling tones and optional rising tones in the penultimate or final syllable. Contour tones are not re\-stricted to bimoraic syllables, however, and can therefore not be analyzed as the realization of an underlying /H∅/ or /∅H/ respectively. Instead, it appears that after bimoraic doubling has taken place, both monomoraic and bimoraic syllables display the same behavior, and are subject to the same tone rules. The rules that create contour tones only apply in the last or penultimate syllable of a phrase-final verb, and will be discussed in the following two sections.

\subsection{H retraction}
\label{bkm:Ref486506264}\hypertarget{Toc75352621}{}
There are two tonal processes in Fwe that only apply at the end of a phrase: high tone retraction, which is an instance of what {\citet[9-10]{OddenMarlo2019}} call ‘nonfinality’, and the realization of high tones in the final or penultimate syllable as falling.

The process of high tone retraction causes a high tone on the last mora of a phrase-final word to move to the preceding mora, as schematized in (\ref{bkm:Ref479236682}).

\ea
\label{bkm:Ref479236682}
H retraction: /∅H/\# > [HL]\#
\z

H retraction can, for instance, be seen in disyllabic nominal stems with an un\-derlying /∅H/ pattern, which surfaces as [LH] in non-final contexts, as in (\ref{bkm:Ref449966908}). If the same noun is used phrase-finally, the high tone of the last syllable shifts to the preceding syllable, result\-ing in a [HL] surface pattern, as in (\ref{bkm:Ref449966919}).

\ea
\label{bkm:Ref449966908}
\gll /N-shukí   zi-angú/ > [nshùkí ꜝzáːngù]\\
\textsc{np}\textsubscript{10}-hair  \textsc{pp}\textsubscript{10}-\textsc{poss}\textsubscript{1SG}\\
\glt ‘my hair’
\z

\ea
\label{bkm:Ref449966919}
\gll N-shukí > [nshúkì]\\
\textsc{np}\textsubscript{10}-hair\\
\glt ‘hair’
\z

H retraction targets moras, not syllables. If a high tone is assigned to the last mora of a bimoraic syllable, H retraction causes it to move to the preceding mora, but not the preceding syllable. The retracted high tone then undergoes bimoraic doubling, and is subsequently subject to the rule that creates falling tones in the last or penultimate syllable of a phrase. This is schematized and illustrated in (\ref{bkm:Ref479238834}). Compare (\ref{bkm:Ref97890340}), where the same verb is used in a non-final context.

\ea
\label{bkm:Ref479238834}
  H retraction in phrase-final /∅H/ syllables:

\ea
/∅.∅H/\# > ∅.H∅ \# (H retraction) \\
> ∅.HH \# (bimoraic doubling) \\
> [L.F] (H > F)

\ex
/ndi-tw-.aá̲./ > [ndìtwâː]\\
\textsc{sm}\textsubscript{1SG}-pound-\textsc{fv}\\
\glt ‘I pound.’
\z\z

\ea
\label{bkm:Ref97890340}
  No H retraction in medial /∅H/ syllables:\\
\gll /ndi-tw-aá̲    mu-ndaré/ >{ [ndìtwáː mùndárè}]\\
\textsc{sm}\textsubscript{1SG}-pound-\textsc{fv}  \textsc{np}\textsubscript{3}-maize\\
\glt ‘I pound maize.’
\z

If a high tone is assigned to the first mora of a bimoraic syllable, H retraction causes the high tone to move to the preceding mora, which is also the preceding syllable. This is schematized and illustrated in (\ref{bkm:Ref479238879}).

\ea
\label{bkm:Ref479238879}
  H retraction in phrase-final /H∅/ syllables

\ea
/∅.H∅/\# > [H.LL]

\ex
\gll /mu-.twíi./ > [mútwìː]\\
\textsc{np}\textsubscript{3}-head\\
\glt ‘a head’
\z\z

Retracted high tones are never realized as falling (see \sectref{bkm:Ref432074291}); instead, they may be realized with a slight rising contour. Non-retracted high tones, however, are realized as falling. This is schematized and illustrated in (\ref{bkm:Ref479241396}--\ref{bkm:Ref74910939}).\footnote{Retracted high tones in the final, rather than the penultimate, syllable do become falling, see (\ref{bkm:Ref479238834}). There is some inter-speaker variation in the application of H > F to retracted high tones in the final syllable; some speakers apply H > F to retracted high tones in the final syllable, others never apply H > F to retracted high tones, either in the final or the penultimate syllable.}

\ea
\label{bkm:Ref479241396}
  /∅.H./\# > [HL]\#  retracted high tones: realized as level high\\
/ku-s-áa/ > [kúsàː]\\
\textsc{inf}-dig-\textsc{fv}\\
\glt ‘to dig
\z

\ea
\label{bkm:Ref74910939}
  /H.∅./\# > [FL]\#  non-retracted high tones: realized as falling\\
/ku-sí-w-a/ > [kùsîːwà]\\
\textsc{inf}-dig-\textsc{pass}-\textsc{fv}\\
\glt ‘to be dug’
\z

High tones can only be realized as rising if they have been retracted to the penultimate syllable, and can only be realized as falling if they are the manifestation of an underlying high tone in the final or penultimate syllable. In all other cases, high tones have to be realized as level high. There is thus a clear restriction of the occurrence of contour tones to the final and penultimate syllable, which can be ex\-plained as the result of the penultimate lengthening of this sylalble. Note that neither phonemic length\-ening, nor automatic lengthening conditioned by the factors discussed in \sectref{bkm:Ref459132421} (i.e. a following prenasalized consonant, a preceding glide, and several others), sanction the occurrence of contour tones.

\subsection{H > F}
\label{bkm:Ref432074291}\hypertarget{Toc75352622}{}
Another phrase-final tone rule in Fwe is the realization of high tones as falling, or H > F for short. This rule causes an underlying high tone in the last or penultimate mora to be realized as falling in a phrase-final word. Examples are given in (\ref{bkm:Ref98833757}--\ref{bkm:Ref98833758}), where the high tone of the verb stem is realized as falling if it occurs in the penultimate syllable, but is realized as high when the high tone is not on the penultimate syllable because of the addition of derivational suffixes.

\ea
\label{bkm:Ref98833757}
/ku-kwáng-a/ > [kùkwâːngà]\\
\textsc{inf}-become\_tired-\textsc{fv}\\
\glt ‘to become tired’
\z

\ea
/ku-kwáng-is-a/ > [kùkwáːngìsà]\\
\textsc{inf}-become\_tired-\textsc{caus}-\textsc{fv}\\
\glt ‘to make [someone] tired’
\z

\ea
\label{bkm:Ref98833758}
/ku-gáb-a/ > [kùgâbà]\\
\textsc{inf}-block-\textsc{fv}\\
\glt ‘to block’
\z

\ea
/ku-gáb-urur-a/ > [kùgábùrùrà]\\
\textsc{inf}-block-\textsc{sep}.\textsc{tr}-\textsc{fv}\\
\glt ‘to unblock’
\z

High tones are rarely found in the final syllable of a phrase-final word, as such high tones are sub\-ject to H retraction (see \sectref{bkm:Ref486506264}). High tones may only occur in a phrase-final syllable if this syllable is bimoraic, in which case this high tone is realized as falling.

\ea
\gll N-mu-.saá.    ndí̲-bwe\textsubscript{H}ne    >[mùsáː ꜝndíbwèːnè]\\
\textsc{cop}-\textsc{np}\textsubscript{1}-thief  \textsc{sm}\textsubscript{1SG}-see.\textsc{stat}\\
\glt ‘I see a thief.’
\z

\ea
\gll ndi-bwe\textsubscript{H}né    mu-.saá.    > [ndìbwèːné mùsâː]\\
\textsc{sm}\textsubscript{1SG}-see.\textsc{stat}  \textsc{np}\textsubscript{1}-thief\\
\glt ‘I see a thief.’
\z

Retracted high tones are never realized as falling (see \sectref{bkm:Ref486506264}). Another context in which final or pre-final high tones are not realized as falling is in questions. Questions have a rising intonation on the final syllable. If the final syllable is low-toned, question intonation will create a rising tone. If the final syllable is high-toned, question intonation will create a level high tone, rather than a falling tone. In (\ref{bkm:Ref69986020}), the high-toned sylla\-ble \textit{kwí} at the end of the phrase is realized as high, rather than falling, as a result of question intona\-tion.

\ea
\label{bkm:Ref69986020}
bànyòkò kòkwíː\\
\gll ba-nyo-ko      kokwí\\
\textsc{np}\textsubscript{2}-mother-\textsc{poss}\textsubscript{2SG}    where\\
\glt ‘Where is your mother?’ (NF\_Elic15)
\z

So far, both H retraction and H > F are described as occurring phrase-finally. Copulative constructions display some ambiguity with respect to phrase-final tonal processes. The noun \textit{njúò} ‘house’ is treated as being at the end of a phrase in (\ref{bkm:Ref445222129}), where the high tone becomes falling, but not in (\ref{bkm:Ref99614146}), where the high tone re\-mains high.

\ea
\label{bkm:Ref445222129}
èyí njûò njétù\\
\gll e-í    N-júo    N-i-etú\\
\textsc{aug}-\textsc{dem}.\textsc{i}\textsubscript{9}  \textsc{np}\textsubscript{9}-house  \textsc{cop}-\textsc{pp}\textsubscript{9}-\textsc{poss}\textsubscript{1PL}\\
\glt ‘This house is ours.’
\z

\ea
\label{bkm:Ref99614146}
yìn’ énjúò njìrôtù\\
\gll yiná    e-N-júo    nji-rótu\\
\textsc{dem}\-.\textsc{iv}\textsubscript{9}  \textsc{aug}-\textsc{np}\textsubscript{9}-house  \textsc{cop}\textsubscript{9}-beautiful\\
\glt ‘That house is beautiful.’ (ZF\_Elic14)
\z

Since H retraction and falling tones only occur at the end of a phrase, they can be used to detect syntactic boundaries. This is relevant for left dislocation, a topicalisation process which consists of moving a constituent to the sentence-initial position where it is phrased separately. This interac\-tion between tone and syntax is discussed in \sectref{bkm:Ref403656711} on left dislocation.

\subsection{High tone spread}
\label{bkm:Ref430865664}\hypertarget{Toc75352623}{}\label{bkm:Ref486506344}
High tones in Fwe may spread to the left onto underlyingly toneless syllables. This is illustrated in (\ref{bkm:Ref430769183}), where the high tone of the final syllable \textit{sá} spreads onto the two preceding, tone\-less syllables. This spread is optional: the realization without high tone spread is also heard.

\ea
\label{bkm:Ref430769183}
\gll /ndi-ur-is-á̲     ma-.yií./ >[ndìúrísáː màyîː {\textasciitilde} ndìùrìsáː màyîː]\\
\textsc{sm}\textsubscript{1SG}-buy-\textsc{caus}-\textsc{fv}  \textsc{np}\textsubscript{6}-egg\\
\glt ‘I sell eggs.’ (NF\_Elic15)
\z

H spread, when it does occur, may result in a sequence of tones with equally high pitch; most commonly, however, the final high tone (from which the spread originates) has the highest pitch, and the preceding high tone(s) are lower. In this way the high tone spread conforms to the obligatory contour principle, which is also served by the processes of Meeus\-sen’s Rule and downstep (see \sectref{bkm:Ref440987952}-\ref{bkm:Ref445296627}), as high tone spread does not create high tones that are preceded by high tones of equally high pitch.

Leftward spread of high tones is an unbounded spread within its domain, not limited to a fixed number of sylla\-bles. In (\ref{bkm:Ref445223737}), the high tone of the final syllable \textit{ri} of the noun \textit{mumusipirí} ‘on a journey’ spreads to the two preceding syllables. In (\ref{bkm:Ref445223852}), the high tone associated with the final vowel suffix \textit{-á} spreads three syllables.

\ea
\label{bkm:Ref445223737}
\gll N-mu-mu-sipirí    ba-iná > [mùmùsípírí ꜝbénà]\\
\textsc{cop}-\textsc{np}\textsubscript{18}-\textsc{np}\textsubscript{3}-journey  \textsc{sm}\textsubscript{2}-be\_at\\
\glt ‘She is on a journey.’
\z

\ea
\label{bkm:Ref445223852}
\gll ba-sep-ahar-á̲  cáha > [bàsépáhárá ꜝcáhà]\\
\textsc{sm}\textsubscript{2}-trust-\textsc{neut}-\textsc{fv}  very\\
\glt ‘They are highly respected.’ (NF\_Elic15)
\z

H spread stops at certain morpheme boundaries. Within verbs, high tones may spread across derivational suffixes, but not onto any pre-stem affixes, such as the object marker \textit{mu-} in (\ref{bkm:Ref489977261}), or the distal marker \textit{ka-} in (\ref{bkm:Ref445224602}).

\ea
\label{bkm:Ref489977261}
\glll ndàmùrémêkì\\
ndi-a-mu-remé̲k-i \\
\textsc{sm}\textsubscript{1SG}-\textsc{pst}-\textsc{om}\textsubscript{1}-hurt-\textsc{npst}.\textsc{pfv}\\
\glt ‘I’ve hurt her/him.’
\z

\ea
\label{bkm:Ref445224602}
àkàpótérá Kàmwìː\\
\gll a-ka-pot-er-á̲    Kamwi \\
\textsc{sm}\textsubscript{1}-\textsc{dist}-visit-\textsc{appl}-\textsc{fv}  Kamwi\\
\glt ‘S/he visits Kamwi.’ (NF\_Elic15)
\z

Within nouns, high tones may spread up to the first root syllable, but not onto the nominal pre\-fix, augment, or any other grammatical prefix. This is illustrated in (\ref{bkm:Ref445224861}), where the high tone of the final syllable /zí/ spreads to the two preceding root syllables, but not to the nominal prefix /mu-/.

\ea
\label{bkm:Ref445224861}
mùsébézí ꜝwábò\\
\gll mu-sebezí  u-abó\\
\textsc{np}\textsubscript{3}-work  \textsc{pp}\textsubscript{3}-\textsc{dem}.\textsc{iii}\textsubscript{2}\\
\glt ‘his job’
\z

H spread may affect the first high tone in an utterance, but also a subsequent high tone, which by default is downstepped. This is illustrated in (\ref{bkm:Ref494903100}): the first high tone of the utterance, on the syllable \textit{cí}, is not downstepped, but the following high tone, which originates on the sylla\-ble \textit{ngí}, is subject to downstep. Subsequently, the second high tone spreads onto the syllable \textit{nyú}. Note that there is a pitch drop between the initial high tone on the syllable \textit{cí} and the spread, downstepped high tone on the following syllable ꜝ\textit{nyú}, as illustrated in the pitch trace.

\ea
\label{bkm:Ref494903100}
ndàcíꜝnyúngínyùngì\\
{[\_  ¯  - - \_ \_ ]}\\
ndi-a-cí-nyungí̲-nyung-i\\
\textsc{sm}\textsubscript{1}-\textsc{pst}-\textsc{om}\textsubscript{7}-\textsc{pl}2-shake-\textsc{npst}.\textsc{pfv}\\
\glt ‘I have shaken it.’ (NF\_Elic15)
\z

Leftward high tone spread in Fwe bears some resemblance to high tone anticipation, or leftward high tone shift, which causes a high tone to surface on one mora to the left. This system has been described for eastern Bantu Botatwe languages, including Tonga (\citealt{Goldsmith1984}; \citealt{Meeussen1963}), Ila and Lenje \citep{Bostoen2009}, but also for the Zambian variety of Totela, which, like Fwe, is part of the western branch of Bantu Botatwe (\citealt{Crane2014}; \citealt{Crane2011})\footnote{According to \citet[55]{Crane2011} however, Zambian Totela should be considered as part of the eastern branch of Bantu Botatwe, rather than the western branch, based, among other criteria, on its use of HTA. Descriptions of the tone systems of other Western Bantu Botatwe languages, such as Subiya and Shanjo, will have to point out whether the occurrence of HTA is an innovation that defines the Eastern branch of Bantu Botatwe with respect to the Western branch. The study of lexical tone in Shanjo by \citet{Bostoen2009} indicates no trace of HTA in this language.}. As already observed by {\citet[123]{Bostoen2009}}, Fwe does not make use of HTA, as illustrated with the reflexes of the reconstructed root *kúpà ‘bone’ in (\ref{bkm:Ref99614338}). In Totela, Tonga and Lenje, the high tone of the first root syllable shifts to the preceding syllable, whereas in Fwe, this high tone does not shift.

\ea
\label{bkm:Ref99614338}
Totela   èchí-fùwà  ‘bone’ \citep[65]{Crane2014}\\
Tonga  ící-fùwà  ‘bone’ \citep[65]{Carter1962}\\
Lenje    cí-fùwà  ‘bone’ \citep[49]{Kagaya1987}\\
Fwe    è-cì-fûhà  ‘bone’
\z
\subsection{The order of tonal processes}
\label{bkm:Ref432073870}\hypertarget{Toc75352624}{}
The way in which tonal processes influence each other suggests that the application of tonal rules follows a set order, with each rule only being applied once; once the rule is applied, it cannot be applied again, even though a different rule may create the conditions for the rule to apply. The following order of tone rules is proposed: Meeussen’s Rule > H retraction > bimoraic doubling > H realized as F > downstep > optional high tone spread. This ordering explains why Meeussen’s Rule and downstep, both rules targeting successive high tones, both play a role, as the intervening rule of H retraction creates new sequences of high tones. The position of optional high tone spread as the last tonal processes explains why successive high tones created by H spread are not subject to Meeussen’s Rule or downstep. The position of H retrac\-tion before H > F explains why certain retracted high tones are realized as falling. Finally, it needs to be noted that the addition of melodic high tones precedes all these tonal processes; tonal pro\-cesses, therefore, treat lexical and melodic tones in an equal fashion.

\section{Lexical tone}
\label{bkm:Ref71539878}\hypertarget{Toc75352625}{}
This section discusses the tonal patterns found on nominal and verbal stems. A first inventory of tonal patterns has been given by {\citet{Bostoen2009}}. This section mostly confirms his findings, but also adds a number of less frequently occurring tonal pat\-terns which were not yet discussed before.

\subsection{Tone on noun stems}
\label{bkm:Ref71540184}\hypertarget{Toc75352626}{}
Disyllabic noun stems can have five different surface tonal patterns in isolation: LL, HL, FL H-ꜝHL, and H-LL. For the latter two patterns, the initial high tone is a floating tone that attaches to any preceding syllable, usually the noun’s nominal prefix or augment. Examples of each of the surface patterns are given in \REF{ex:3:tonalpatterndisyllabic}.

\NumTabs{3}
 \ea\label{ex:3:tonalpatterndisyllabic}
  Tonal patterns on nouns with disyllabic stems


% \begin{tabularx}{\textwidth}{XXX}
%
% \lsptoprule

\ea
/∅∅/ \tab [LL] \tab \\
/vumo/ \tab vùmò \tab ‘stomach’\\
/ma-ira/ \tab mà-hìrà \tab ‘sorghum’\\
/mu-riro/ \tab mù-rìrò \tab ‘fire’\\

\ex
/H∅/ \tab [FL] \tab \\
/n-júo/ \tab njûò \tab ‘house’\\
/zyúba/ \tab zyûbà \tab ‘sun, day’\\
/ku-bóko/ \tab kù-bôkò \tab ‘arm’\\

\ex
/∅H/ \tab [HL] \tab \\
/mbufú/ \tab mbúfù \tab ‘bream’\\
/ndavú/ \tab ndávù \tab ‘lion’\\
/ci-shamú/ \tab cì-shámù \tab ‘tree’\\

\newpage
\ex
/H-∅H/ \tab [H-ꜝHL ] \tab \\
/bú-cenyá/ \tab bú-ꜝcényà \tab ‘smallness’\\
/cí-monshó/ \tab cí-ꜝmóːnshò \tab ‘left’\\
/ká-nensá/ \tab ká-ꜝnéːnsà \tab ‘pink, little toe’\\

\ex
/H-∅∅/ \tab [H-LL ] \tab \\
/mú-ngorwe/ \tab mú-ngòrwèː \tab ‘tree sp. (used to \tab\tab\tab cure a curse)’\\
/ká-nsikwe/ \tab ká-nsìkwèː \tab ‘darkness’\\
/mí-raːra/ \tab mí-ràːrà \tab ‘leftovers’\label{bkm:Ref413224726}\\
% \lspbottomrule
% \end{tabularx}
\z
\z

Given the productive use of Meeussen’s Rule in Fwe (see \sectref{bkm:Ref440987952}), turning a /HH/ se\-quence into /H∅/, nouns surfacing with a [FL] pattern could have an underlying /H∅/ or /HH/ pattern. Historically, Fwe nouns with a [FL] surface pattern are reflexes of nouns reconstructed as either *HH or *HL, for example \textit{mà-fûtà} ‘oil’, from *kútà ‘oil, fat’, and \textit{n-sîngò} ‘neck’, from *kíngó ‘neck’ \citep[121]{Bostoen2009}. There is evidence, however, that [FL] nouns all have an underlying /HH/ tonal pattern synchronically. When these nouns are combined with the diminutive suffix \textit{-ána}, as in (\ref{bkm:Ref98929804}), they lose all but the first high tone, which is indica\-tive of an underlying /HH/ pattern affected by repeated Meeussen’s Rule.

\ea
\label{bkm:Ref98929804}
\gll /ka-zyúrú-ána/ > /ka-zyúru-ana/ > [kàzyúrùànà]\\
\textsc{np}\textsubscript{12}-nose-\textsc{dim}\\
\glt ‘small nose’
\z

All nouns with a [FL] tonal pattern have the same tonal pattern when combined with the dimin\-utive \textit{-ána}. No distinction is made between reflexes of a historical *HL pat\-tern and reflexes of a historical *HH pattern, as shown in \tabref{tab:3:3}.

\begin{table}
\label{bkm:Ref98512572}\caption{\label{tab:3:3}Tonal patterns of disyllabic /HH/ nouns with the diminutive \textit{-ána}}

\begin{tabular}{lll}

\lsptoprule

Underived noun & Noun with diminutive /-ána/ & Reconstruction\\
\midrule 
\textit{n-jôkà} ‘snake’ & \textit{n-jókàànà} ‘small snake’ & *-jókà ‘snake’\\
\textit{rù-rîmì} ‘tongue’ & \textit{kà-rímìànà} ‘small tongue’ & *-dímì ‘tongue’\\
\textit{mù-zîò} ‘load’ & \textit{mù-zíòànà} ‘small load’ & *-dígò ‘load’\\
\textit{mù-kûrù} ‘adult’ & \textit{mù-kúrùànà} ‘young adult’ & *-kʊdʊ ‘adult’\\
\lspbottomrule
\end{tabular}
\end{table}

Four different patterns are found in nouns with a monosyllabic stem in isolation; L-L, H-L, F-L and L-F, as in (\ref{bkm:Ref71539917}). As these stems are monosyllabic, only the second tone is realized on the noun root, and the first tone is realized either on the nominal prefix, or, when the nominal prefix lacks a vowel, on the augment prefix.

\NumTabs{3}
\ea
\label{bkm:Ref71539917}
  Tonal patterns on nouns with monosyllabic stems

\ea
/∅-∅/ \tab L-L \tab \\
/mu-ntu/ \tab mù-ntù \tab ‘person’\\
/e-wa/ \tab è-wà \tab ‘field’\\
/ci-zo/ \tab cì-zò \tab ‘tradition’\\
\ex
/∅-H/ \tab H-L \tab \\
/ku-twí/ \tab kú-twì \tab ‘ear’\\
/e-vú/ \tab é-vù \tab ‘sand’\\
/e-zwí/ \tab é-zwì \tab ‘knee’\\
\ex
/Hː- ∅/ \tab F-L \tab \\
/rúː-ho/ \tab rûː-hò \tab ‘wind’\\
/búː-ci/ \tab bûː-cì \tab ‘honey’\\
\ex
/∅-Hː/ \tab L-F \tab \\
/mu-sáː/ \tab mù-sâː \tab ‘thief’\\
/e-gíː/ \tab è-gîː \tab ‘egg’\\
\z
\z

The [H-L] and [L-L] patterns are the most frequently occurring patterns. The tonal pattern [L-F] only occurs with nominal stems with a bimoraic vowel, which can be phonemically long, as in (\ref{bkm:Ref450122717}--\ref{bkm:Ref489979431}), or automatically lengthened, as in (\ref{bkm:Ref487190859}--\ref{bkm:Ref487190860}) (see \sectref{bkm:Ref459132421} for the conditions of automatic lengthening).

\ea
\label{bkm:Ref450122717}
\glll bùǀôː\\
bu-ǀóː\\
\textsc{np}\textsubscript{14}-tasteless\\
\glt ‘tastelessness’
\z

\ea
\label{bkm:Ref489979431}
\glll bùrêː\\
bu-réː\\
\textsc{np}\textsubscript{14}-long\\
\glt ‘length’
\z

\ea
\label{bkm:Ref487190859}
\glll rùkwêː\\
ru-kwé\\
\textsc{np}\textsubscript{11}-grass\\
\glt ‘grass (\textit{Schoenoplectus brachyceras})’
\z

\ea
\label{bkm:Ref487190860}
\glll mùsâː\\
mu-sá\\
\textsc{np}\textsubscript{1}-thief\\
\glt ‘thief’
\z

Monosyllabic nouns with a long vowel may also occur with a [H-L] pattern, reflecting underlying /∅-H/, as in in (\ref{bkm:Ref450122596}--\ref{bkm:Ref450122597}), or as [L-L], reflecting no underlying high tones, as in (\ref{bkm:Ref450122958}).

\ea
\label{bkm:Ref450122596}
/o-∅-mbwáa/ > [ómbwàː]\\
\textsc{aug}-\textsc{np}\textsubscript{1a}-dog\\
\glt ‘dog’
\z

\ea
\label{bkm:Ref450122597}
/e-N-shwaá/ > [ènshwâː]\\
\textsc{aug}-\textsc{np}\textsubscript{10}-termite\\
\glt ‘termites’
\z

\ea
\label{bkm:Ref450122958}
/mu-nwee/ > [mùnwèː]\\
\textsc{np}\textsubscript{3}-finger\\
\glt ‘finger’
\z

Monosyllabic noun stems with the tonal pattern [F-L] have an extra mora before the first (and only) root consonant, causing the vowel of the nominal prefix to be lengthened. Monosyllabic noun stems taking the [F-L] pattern historically derive from disyllabic noun stems. The noun \textit{cî-rì} ‘adder’ derives from a disyllabic noun root *-p\'{ɪ} d\`{ɪ}  ‘puff adder’ \citep{BastinEtAl2002}; the initial consonant /p/ is systemati\-cally lost in Fwe, and the vowel of the nominal prefix \textit{ci-} and the initial vowel of the stem \textit{-iri} have subsequently contracted. Only three other examples with this tonal pattern are found, which are presented in (\ref{bkm:Ref451964135}--\ref{bkm:Ref451964136}).

\ea
\label{bkm:Ref451964135}
\glll bûːcì\\
búː-ci\\
\textsc{np}\textsubscript{14}-honey\\
\glt ‘honey’
\z

\ea
\glll rûːhò\\
rúː-ho\\
\textsc{np}\textsubscript{11}-wind\\
\glt ‘wind’
\z

\ea
\label{bkm:Ref451964136}
\glll bûːsì\\
búː-si\\
\textsc{np}\textsubscript{14}-smoke\\
\glt ‘smoke’
\z

Noun stems with three or more syllables attest a number of different tone patterns. Among polysyl\-labic nominal stems are a number of deverbal nouns, reduplicated nouns, compounds, and animal names that contain a prefix \textit{na}- or \textit{shi}- followed by a former nominal prefix. The most common tonal patterns for trisyllabic noun stems, as laid out in (\ref{bkm:Ref98930032}), are [HLL], corresponding to an under\-lying /H∅∅/ pattern, and [LLL], corresponding to an underlying tone pattern without high tones.
\NumTabs{5}
\ea
\label{bkm:Ref98930032}
  Trisyllabic noun stems with a /H∅∅/ or /∅∅∅/ pattern

\ea
/H∅∅/ \tab [HLL] \\
o-nkúmbizi \tab ò-nkúmbìzì \tab ‘beggar’\\
mu-kázana \tab mù-kázànà \tab ‘girl’\\
mu-gwégwesi \tab mù-gwégwèsì \tab ‘joint’\\
mpúbira \tab mpúbìrà \tab ‘papaya’\\
\ex
/∅∅∅/ \tab [LLL] \tab \\
/o-ntimbira/ \tab ò-ntìmbìrà \tab ‘dung beetle’\\
/mu-cembere/ \tab mù-cèmbèrè \tab ‘old lady’\\
/e-n-daano/ \tab è-n-dàànò \tab ‘message’\\
/ci-wakaka/ \tab cì-wàkàkà \tab ‘horned melon (\textit{Cucumis metuliferus})’\\

\z
\z

The tonal pattern [HꜝHL], as in (\ref{bkm:Ref98930057}), is also fairly common in trisyllabic noun stems. It represents an under\-lying /H∅H/ pattern where the second H is retracted and subsequently downstepped (see \sectref{bkm:Ref445296627} on downstep and \sectref{bkm:Ref432074325} on H retraction).
\NumTabs{3}
\ea
\label{bkm:Ref98930057}
  Trisyllabic noun stems with a /H∅H/ pattern


/H∅H/ \tab [HꜝHL] \tab \\
/bu-shómaní/ \tab bù-shóꜝmánì \tab ‘bad luck’\\
/ru-vútamó/ \tab rù-vúꜝtámò \tab ‘lower stomach’\\
/bu-kábabú/ \tab bù-káꜝbábù \tab ‘problem’\\
/mu-túkutá/ \tab mù-túꜝkútà \tab ‘heat’\\
\z\largerpage

Other tonal patterns found with trisyllabic noun stems, as presented in (\ref{bkm:Ref431293122}), have a more restricted distribution and mainly occur with borrowings: a /∅∅H/ pattern, which may surface as [HHL] or [LHL] in isolation; a /∅H∅/ pattern, which may surface as [HFL] or [LFL] in isolation, and which occurs with borrowings and nouns derived with the deverbal suffix \textit{-ntu} (see \sectref{bkm:Ref444251366}).

\NumTabs{4}
\TabPositions{.19\textwidth, .36\textwidth, .5\textwidth, .4\textwidth}
\ea
\label{bkm:Ref431293122}
  Trisyllabic noun stems with a /∅∅H/ or /∅H∅/ pattern

\ea /∅∅H/ \tab {[LHL] {\textasciitilde} [HHL]} \tab source\\
/ka-pikirí/ \tab kà-píkírì \tab ‘nail’ \tab Afrikaans spyker  ‘nail’\\
/mu-sebezí/ \tab mù-sébézì \tab ‘work’ \tab Lozi musebezi ‘work’\\
/mu-sipirí/ \tab mù-sípírì \tab ‘journey’ \tab Lozi musipili 
‘journey’\\
/n-tauró/ \tab n-táúrò \tab ‘headveil’ \tab English towel\\
/ci-fatehó/ \tab cì-fàtéhò \tab ‘face’ \tab Lozi sifateho ‘face’\\
/n-komokí/ \tab n-kòmókì \tab ‘cup’ \tab Lozi komoki ‘cup’\\
/n-kereké/ \tab n-kèrékè \tab ‘church’ \tab Afrikaans kerk 
‘church’\\
\ex /∅H∅/ \tab [LFL] {\textasciitilde} [HFL]\\
/ci-munántu/ \tab cì-múnântù \tab ‘domesticated animal’ \tab -muna ‘own’ +-ntu \\
/ci-tendántu/ \tab cì-téndântù \tab ‘action’ \tab cf. -tenda ‘do’ +-ntu \\
/ma-hondéro/ \tab mà-hóndêrò \tab ‘kitchen’ \tab cf. -honda ‘cook’\\
/hemére/ \tab hèmêrè \tab ‘bucket’ \tab Afrikaans emmer ‘bucket’\\
/mu-kotána/ \tab mù-kòtânà \tab ‘bag’ \tab Lozi mukotana ‘bag’\\
\z\z

Nominal stems of four syllables are also attested. Many of these are reduplicated, though they are usually not attested in their unreduplicated form. The tonal patterns attested with nominal stems of four syllables are given in (\ref{bkm:Ref98833361}). Longer nominal stems are usually regularly derived from verbs, or compounds.
\NumTabs{4}
\ea
\label{bkm:Ref98833361}
  Tonal patterns of nominal stems with four syllables

\ea
/H∅∅H/ \tab [HLHL]\\
/ma-síkusikú/ \tab mà-síkùsíkù \tab ‘morning’\\
/njóvenjové/ \tab njóvènjóvè \tab ‘tree (\textit{Abrus precatorius})’\\
\ex 
/H∅H∅/ \tab [HLFL]\\
/ka-ríkuríku/ \tab kà-ríkùrîkù \tab ‘hiccup’\\
/mu-rárambínda/ \tab mù-ráràmbîndà \tab ‘milky way’\\
\ex 
/∅H∅H/ \tab {[HH!HL] {\textasciitilde} [LH!HL]}\\
/ka-cióció/ \tab kà-cíyóꜝcíyò \tab ‘chick’\\
/maíwué/ \tab màyíꜝwúyè \tab ‘duck sp.’\\
\ex 
/∅H∅∅/ \tab [LHLL]\\
/ka-rurérure/ \tab kà-rùrérùrè \tab ‘plant sp.’\\
/kacípembe/ \tab kàcípèmbè \tab ‘mongongo beer’\\
\ex 
/∅∅∅H/ \tab [LLHL\\
/bbimbiriró/ \tab bbìmbìrírò \tab ‘rubbish heap’\\
/harantené/ \tab hàrànténè \tab ‘cockroach’\\
\ex /∅∅∅∅/ \tab [LLLL] \tab \\
/ci-tukutuku/ \tab cì-tùkùtùkù \tab ‘hiccup’\\
/ci-tepwerere/ \tab cì-tèpwèrèrè \tab ‘thin porridge’\\
\z\z

Although nominal prefixes are underlyingly toneless, and as such are realized with a low tone with the majority of nouns (see \sectref{bkm:Ref489364754} on nominal prefixes), there are a number of nouns that have a high-toned nominal prefix. Nouns with a high tone on the prefix can have stems of two, three or more syllables, as in (\ref{bkm:Ref98930270}). (In monosyllabic nouns, a high-toned nominal prefix is the result of H retraction; see (\ref{bkm:Ref71539917}).)

\NumTabs{3}
\ea
\label{bkm:Ref98930270}
/H-∅H/ \tab [H-ꜝHL] \\
/mú-kwamé/ \tab mú-ꜝkwáːmè \tab ‘man’\\
/cí-nsozí/ \tab cí-ꜝnsózì \tab ‘tear’\\
/cí-ariso/ \tab cí-àrìsò \tab ‘latch’\\
/má-nshawánshawa/ \tab má-ꜝnsháwánshàwà \tab ‘berries sp.’\\
\z

These nouns have a floating high tone that precedes the nominal root, which is realized on the nominal prefix. When the nominal root is not preceded by a (syllabic) nominal prefix, the floating high tone is realized on the noun’s augment prefix, as in (\ref{bkm:Ref71539076}--\ref{bkm:Ref71539074}). The augment prefix itself is realized with a low tone in all other cases (see \sectref{bkm:Ref444175456}).

\ea
\label{bkm:Ref71539076}
\glll éꜝnkórì\\
é-N-korí\\
\textsc{aug}-\textsc{np}\textsubscript{9}-walking\_stick\\
\glt ‘walking stick’
\z

\ea
\label{bkm:Ref71539074}
\glll éꜝmpúndù\\
é-N-pundú\\
\textsc{aug}-\textsc{np}\textsubscript{10}-berry\\
\glt ‘berries’
\z

A number of nouns with a floating high tone are derived from verbs that also have a floating high tone (see \sectref{bkm:Ref71539940}), as illustrated in (\ref{bkm:Ref71535793}--\ref{bkm:Ref71535794}).

\ea
\label{bkm:Ref71535793}
cíàzò \tab     cf. kúàrà\\
cí-azo \tab kú-ar-a\\
\textsc{np}\textsubscript{7}-door \tab \textsc{inf}-close-\textsc{fv}\\
\glt ‘door’\tab     ‘to close’
\z

\ea
\label{bkm:Ref71535794}
cíyàzì  \tab     cf.  kúyàà\\
cí-yazi   \tab  kú-ya-a\\
\textsc{np}\-\textsubscript{7}-traitor   \tab \textsc{inf}-kill-\textsc{fv}\\
\glt ‘traitor’ \tab   ‘to kill’
\z

For other nouns, the origin of the floating tone is unclear. Out of about 1100 nominal stems, 33 nominal stems have a floating high tone, of which 7 are transparently derived from verbs that have a floating tone. The remaining 26 nouns are listed in (\ref{bkm:Ref489982300}).

\NumTabs{2}
\TabPositions{.3\textwidth, .5\textwidth}
\ea
\label{bkm:Ref489982300}
é-ꜝtángányàmbè  \tab    ‘calabash’\\
mú-ngòrwè   \tab     ‘tree sp. (used to cure a curse)’\\
ká-nkàfwà     \tab   ‘bat’\\
ká-nsìkwè  \tab      ‘darkness’\\
ká-nshèrèrè    \tab     ‘small mushroom sp.’\\
rú-ngàmàzyòbà   \tab   ‘plant sp.’\\
mú-ⁿǀùryà {\textasciitilde} mú-ꜝⁿǀúryà   \tab ‘lizard’  \\
bú-ꜝcényà   \tab     ‘smallness’\\
mú-ꜝkwámè       \tab ‘man’\\
cí-ꜝmónshò \tab       ‘left-hand side’\\
é-ꜝmpúndù      \tab  ‘berries of the sandpaper raisin bush’\\
rú-ꜝⁿǀáⁿǀà \tab       ‘sedge leaf’\\
ká-ꜝnénsà      \tab  ‘pink, little toe’\\
é-n-ꜝkórì \tab       ‘walking stick’\\
bú-ꜝŋómbà  \tab      ‘plant (\textit{Lannea edulis})’\\
ká-ꜝnsínsì     \tab   ‘small blue bird sp.’\\
cí-ꜝnsózì   \tab     ‘tear’\\
mú-ꜝnzúrè    \tab    ‘shadow; malaria’\\
rú-ꜝshíkà     \tab   ‘African mangosteen (\textit{Garcinia livingstonei})’\\
é-ꜝsímà    \tab    ‘well’\\
má-ꜝsínzà   \tab     ‘snot’\\
rú-ꜝsúmà     \tab   ‘jackalberry’\\
ká-ꜝmpáfwà   \tab  ‘bat sp.’\\
{\textasciitilde} ká-mpàfwà\\
ká-ꜝnyángwényàngwè  \tab  ‘shrub (\textit{Mundulea sericea})’\\
má-ꜝnsháwánshàwà  \tab ‘shrub (\textit{Grewia} sp.)’\\
{\textasciitilde} má-ꜝnsháwà
\z\pagebreak

Nouns with a floating high tone before the nominal stem can have various tonal patterns on the nominal stem, e.g. an underlying /∅H/ pattern which is realized as [H!HL] in isolation, as in (\ref{bkm:Ref71535921}), or an underlying /H-H∅/ tonal pattern, which corresponds to a [H-LL] surface pattern, as in (\ref{bkm:Ref71535930}).

\ea
\label{bkm:Ref71535921}
\gll /mú-kwamé/ > mú-kwámè (H retraction) > [mú-ꜝkwáːmè] (downstep)\\
\textsc{np}\textsubscript{1}-man\\
\glt ‘man’
\z

\ea
\label{bkm:Ref71535930}
cí-áriso > /cí-ariso/ > cí-àrìsò\\
\textsc{np}\textsubscript{7}-latch\\
\glt ‘latch’
\z

Floating high tones are also found with a number of verb stems (see \sectref{bkm:Ref71539970}), and with cer\-tain grammatical forms, such as the augment (see \sectref{bkm:Ref444175456}) and possessives (see chapter \ref{bkm:Ref451868043}). In all cases, floating tones are realized on the first available mora to the left of the morpheme with which the floating tone is associated; no floating tones have been found that associate to the right edge of a morpheme.

\subsection{Tone on verb stems}
\label{bkm:Ref71539940}\hypertarget{Toc75352627}{}\label{bkm:Ref71540163}\label{bkm:Ref71539970}
This section discusses the tonal patterns found on verb stems, as used in the infinitive form. An infinitive consists of an infinitive prefix \textit{ku-}, followed by the verb stem (which may contain derivational suffixes), followed by a final vowel suffix \textit{-a.} For the purpose of the to\-nal analysis, this suffix, which is underlyingly toneless and appears on all infinitives (as well as a variety of verbal inflections), is taken as part of the verb stem; verbs may never appear without a final vowel suffix, and \textit{-a} is the most common, morphologically and semantically unmarked final vowel suffix.

Verbs have a lexical tone contrast in their first stem syllable, which can have a high tone or no tone, and/or assign a floating high tone to the preceding sylable. Inflected verbs may or may not maintain lexi\-cal tone, and may assign additional high tones to specific moras or syllables of the verb. Tonal patterns on inflected verbs are discussed in \sectref{bkm:Ref71539267}.

Disyllabic verb stems have three possible tone patterns in the infinitive in isolation, as in (\ref{bkm:Ref98930522}): FL, LL and the fairly marginal pattern H-LL, with a floating high tone that is realized on the infini\-tive prefix (see (\ref{bkm:Ref489973399}) for more examples of this floating high tone).\pagebreak

\NumTabs{3}
\TabPositions{.2\textwidth, .2\textwidth, .4\textwidth}
\ea
\label{bkm:Ref98930522}
  Tonal patterns on disyllabic verb stems

\ea /H∅/ \tab FL \tab \\
ku-hár-a \tab kù-hâr-à \tab ‘to live’\\
ku-zyímb-a \tab kù-zyîːmb-à \tab ‘to sing’\\
ku-shésh-a \tab  kù-shêsh-à \tab ‘to marry’\\
ku-ráːr-a \tab kù-râːr-à \tab ‘to sleep’\\
\ex /∅∅/ \tab LL \tab \\
ku-har-a \tab kù-hàr-à \tab ‘to scrape’\\
ku-end-a \tab kù-yèːnd-à \tab ‘to walk’\\
ku-shek-a \tab kù-shèk-à \tab ‘to laugh’\\
ku-coːk-a \tab kù-còːk-à \tab ‘to break’\\
\ex /H-∅∅/ \tab H-LL \tab \\
kú-pak-a \tab kú-pàk-à \tab ‘to carry on one’s back (of a child)’\\
kú-zyus-a \tab kú-zyùs-à \tab ‘to fill’\\
kú-zyib-a \tab kú-zyìb-à \tab ‘to get to know’\\
\z\z

Verb stems surfacing as LL have no underlying high tones. Verb stems surfacing as FL have an underlying high tone on the first syllable of the root; the pre-final high tone in disyl\-labic verb stems is realized as falling phrase-finally and in isolation (see \sectref{bkm:Ref432074291}).

Monosyllabic verb stems consist of a root of either a single consonant, or a single vowel, or a consonant and a vowel, where the last vowel is glided or elided under influence of the final vowel suffix \textit{-a.} Two surface patterns are found on monosyllabic verb stems, H-L and L-L, as in (\ref{bkm:Ref98930683}). The first tone of the pattern verbs is realized on the infinitive prefix \textit{ku-}.

\ea
\label{bkm:Ref98930683}
  Tone patterns on monosyllabic verb stems

\ea /∅-H/ \tab [H-L] \tab \\
ku-w-á \tab kú-w-à \tab ‘to give’\\
ku-s-á \tab kú-s-àː \tab ‘to dig’\\
ku-nyw-á \tab kú-nyw-àː \tab ‘to drink’\\
\ex /∅-∅/ \tab [L-L] \tab \\
ku-gw-a \tab kù-gw-àː \tab ‘to fall’\\
ku-rw-a \tab kù-rw-àː \tab ‘to fight’\\
ku-zw-a \tab kù-zw-àː \tab ‘to leave’\\
\z\z

The high tone of a monosyllabic high-toned verb stem is realized on the infinitive prefix rather than the verb stem because of H retraction (see \sectref{bkm:Ref432074325}). If a monosyllabic verb with a [H-L] pattern in isolation is extended with a suffix, as in (\ref{bkm:Ref98934683}), the high tone is realized on the verb stem itself.

\ea
\label{bkm:Ref98934683}
\glll kútwàː\\
ku-tw-á\\
\textsc{inf}-pound-\textsc{fv}\\
\glt ‘to pound’
\z

\ea
\glll kùtwîːwà\\
ku-tw-íw-a\\
\textsc{inf}-pound-\textsc{pass}-\textsc{fv}\\
\glt ‘to be pounded’
\z

Verb stems with three or more syllables can also be divided into those with and without a high tone, as in (\ref{bkm:Ref98934750}). The high tone, if present, is always real\-ized on the first syllable of the stem. This is related to the fact that trisyllabic and longer verb stems consist of a root followed by derivational suffixes (though many of these are fossilized and no longer analyzable as such), and derivational suffixes in Fwe are invariably toneless (see chapter \ref{bkm:Ref97890592}). Verb stems with more than four syllables follow the same patterns as verb stems with three or four syllables.

\ea
\NumTabs{3}
\TabPositions{.25\textwidth, .5\textwidth, .4\textwidth}
\label{bkm:Ref98934750}
  Tone patterns on polysyllabic verb stems

\ea
/∅∅∅/ \tab [LLL]\\
ku-dokor-a \tab kù-dòkòr-à \tab ‘to belch’\\
ku-hompwer-a \tab kù-hòːmpwèːr-à \tab ‘to hammer’\\
ku-kabir-a \tab kù-kàbìr-à \tab ‘to enter’\\
\ex /H∅∅/ \tab [HLL] \tab \\
ku-cécent-a \tab kù-cécèːnt-à \tab ‘to winnow’\\
ku-círuk-a \tab kù-círùk-à \tab ‘to jump’\\
ku-kárih-a \tab kù-kárìh-à \tab ‘to shout’\\
\ex /∅∅∅∅/ \tab [LLLL] \tab \\
ku-barakat-a \tab kù-bàràkàt-à \tab ‘to flap (as a fish on dry land)’\\
ku-fufurerw-a \tab kù-fùfùrèrw-àː \tab ‘to sweat’\\
/H∅∅∅/ \tab [HLLL] \tab \\
ku-káwuhany-a \tab kù-káwùhàny-à \tab ‘to separate’\\
ku-súrumuk-a \tab kù-súrùmùk-à \tab ‘to descend’\\
\z\z

A number of verb stems have a floating high tone that is realized on any syllable that directly precedes the verb stem. In the infinitive form, the floating high tone is realized on the underly\-ingly toneless infinitive prefix \textit{ku-}, as in (\ref{bkm:Ref489973399}).

\ea
\label{bkm:Ref489973399}
/H-∅∅/ \tab [H-LL] \tab \\
/kú-ar-a/ \tab kú-àr-à \tab ‘to close’\\
/kú-kar-a/ \tab kú-kàr-à \tab ‘to sit’\\
/kú-kut-a/ \tab kú-kùt-à \tab ‘to become full’\\
/kú-min-a/ \tab kú-mìn-à \tab ‘to set (of the sun)’\\
/kú-pak-a/ \tab kú-pàk-à \tab ‘to carry on one’s back (of a child)’\\
/kú-swaner-a/ \tab kú-swànèr-à \tab ‘to be obliged to’\\
/kú-tab-a/ \tab kú-tàb-à \tab ‘to answer’\\
/kú-ya-a/ \tab kú-yà-à \tab ‘to kill’\\
/kú-zyib-a/ \tab kú-zyìb-à \tab ‘to get to know’\\
/kú-zyur-a/ \tab kú-zyùr-à \tab ‘to become full’\\
\z

The floating high tone of these verb stems is realized on whatever syllable precedes the verb stem. In (\ref{bkm:Ref430695761}), the floating high tone of \textit{taba} ‘answer’ is realized on the underlyingly toneless past prefix \textit{a}-. In (\ref{bkm:Ref488767943}), the verb’s floating high tone is realized on the underlyingly toneless object marker \textit{mu}-.

\ea
\label{bkm:Ref430695761}
/ndi-á-tab-i/ > [ndátàbì]\\
\textsc{sm}\textsubscript{1SG}-\textsc{pst}-answer-\textsc{npst}.\textsc{pfv}\\
\glt ‘I answered.’
\z

\ea
\label{bkm:Ref488767943}
/ku-mú-tab-a/ > [kùmútàbà]\\
\textsc{inf}-\textsc{om}\textsubscript{1}-answer-\textsc{fv}\\
\glt ‘to answer him’
\z

\begin{sloppypar}
The surface realization of infinitives with a floating high tone may correspond either to an un\-derlying tone pattern of /H-H∅/ or /H-∅∅/, because through Meeus\-sen’s Rule, both would surface as [H-LL]. Looking at verbs with floating high tones in certain verbal tense/aspect/mood constructions, however, makes it clear that these verbs have a /H-H∅/ pattern, as the melodic high tone assigned to the second stem syllable is deleted, which can only be the result of the repeated application of Meeussen’s Rule. This is illustrated with the near past perfective in (\ref{bkm:Ref98934963}--\ref{bkm:Ref98934956}). No differences between different lexical verbs were observed, showing that all verbs with a floating high tone have a /H-H∅/ pattern.
\end{sloppypar}\largerpage

\ea
\label{bkm:Ref98934963}
ndi-á-táb-í̲ > ndi-á-tab-i > [ndátàbì]\\
\textsc{sm}\textsubscript{1SG}-\textsc{pst}-answer-\textsc{npst}.\textsc{pfv}\\
\glt ‘I answered.’
\z

\ea
ndi-á-kút-í̲ > ndi-á-kut-i > [ndákùtì]\\
\textsc{sm}\textsubscript{1SG}-\textsc{pst}-become\_full-\textsc{npst}.\textsc{pfv}\\
\glt ‘I am full.’
\z

\ea
\label{bkm:Ref98934956}
ci-á-zyúr-í̲ > ci-á-zyur-i > [cázyùrì]\\
\textsc{sm}\textsubscript{7}-\textsc{pst}-become\_full-\textsc{pst} \\
\glt ‘It is full.’ (NF\_Elic15)
\z

All verb stems with a floating high tone attested in Fwe are listed in (\ref{bkm:Ref489973399}). Three more verbs are attested that occur both with and without a floating high tone; for two of them, which form is used appears to depend on the individual speaker’s preference, and no semantic differences where observed. For one verb, there is a semantic difference between the two forms. All these verbs are listed in (\ref{bkm:Ref98935024}).

\ea
\NumTabs{4}
\TabPositions{.34\textwidth, .6\textwidth}
\label{bkm:Ref98935024}
/kú-cirir-a/ {\textasciitilde} /ku-círir-a/ \tab kúcìrìrà {\textasciitilde} kùcírìrà \tab ‘to follow’\\
/kú-hik-a/ {\textasciitilde} /ku-hík-a/ \tab kúhìkà {\textasciitilde} kùhîkà \tab ‘to cook’\\
/kú-min-a/ \tab kúmìnà \tab ‘to set (of the sun)’\\
/ku-min-a/ \tab kùmìnà \tab ‘to swallow’\\
\z

Floating high tones mostly behave like lexical tones: in tense/aspect/mood constructions that delete underly\-ing lexical tones, floating high tones are usually also deleted, though there are also some exceptions, suggesting that floating high tones have a status that differs from both lexical and melodic tones. This is discussed in \sectref{bkm:Ref71540134}.

The floating high tone with certain verb stems derives from an earlier high-toned vowel occurring at the stem-initial position, preceding the modern verb stem. This is evidenced by the Totela cognates of Fwe verb stems with floating high tones, which have a high-toned vowel \textit{í} as the first syllable of the verb stem, and by the corresponding Bantu reconstructions, which include an initial high-toned syllable. These comparisons are shown in \tabref{tab:3:4}.

\begin{table}
\label{bkm:Ref98935090}\caption{\label{tab:3:4}The origin of floating high tones in Fwe verbs}

\begin{tabularx}{\textwidth}{Xll}
\lsptoprule
Fwe & Totela \citep{Crane2011} & Bantu reconstruction (BLR3)\\
\midrule
\textit{kú-àr-ùr-à} ‘open’ & \textit{òkwíjàlùlà} ‘open’ & \\
\textit{kú-kàr-à} ‘sit, stay’ & \textit{òkwíkàlà} ‘stay’ & \\
\textit{kú-yà-à} ‘kill’ & \textit{òkwíjàyà} ‘kill’ & \\
\textit{kú-kùt-à} ‘become full’ &  & *-jíkut- ‘be satiated’\\
\textit{kú-tàb-à} ‘answer’ &  & *-jítab- ‘answer call’\\
\textit{kú-zyìb-à} ‘know’ &  & *-jíjib- ‘know’\\
\textit{kú-zyùr-à} ‘become full’ &  & *-jíjʊd- ‘become full’\\
\lspbottomrule
\end{tabularx}
\end{table}

The loss of the high-toned vowel in Fwe but the maintenance of its high tone resulted in a float\-ing high tone that is realized on any pre-stem morpheme. In some cases, the earlier vowel /i/ still surfaces. In the verb \textit{kú-yàà} ‘to kill’, devocalization of /i/ may explain the occurrence of the root-initial glide /y/.

\section{Melodic tone}
\label{bkm:Ref71539267}\hypertarget{Toc75352628}{}
The tone pattern of most inflected verbs is determined by the tense/aspect/mood (TAM) construction, which may assign high tones to a particular position in an inflected verb. This use of tone is seen in many Bantu languages, and is referred to as “melodic tone” (\citealt{OddenBickmore2014}). Fwe has four melodic tone patterns: a high tone assigned to the last mora of the word (melodic tone 1), to the subject marker (melodic tone 2), and to the second stem syllable (melodic tone 3). Melodic tone pattern 4 refers to the process of deleting underlying tones, which occurs in specific TAM constructions. \tabref{tab:3:5} gives an overview of melodic tones that are used in Fwe.

\begin{table}
\label{bkm:Ref467232079}\caption{\label{tab:3:5}Melodic tone in Fwe}

\begin{tabularx}{\textwidth}{lQQ}
\lsptoprule
Melodic tone & Realization & TAM construction\\
\midrule
Melodic tone 1 & H on the last mora

\textit{or} H on the penultimate syllable if it is bimoraic & present

remote past imperfective

near future perfective

subjunctive perfective

negative stative

relative remote past perfective\\
\midrule
Melodic tone 2 & H on the subject marker & remote past imperfective

remote future

near future

remote past perfective

most relative clause verbs\\
\midrule
Melodic tone 3 & H on the second stem syllable & near past perfective

negative present

stative

subjunctive perfective with object marker\\
\midrule
Melodic tone 4 & deletes all underlying H & present

remote past imperfective

stative

subjunctive perfective\\
\midrule
no melodic tone & no H is assigned; underlying H are maintained & near past imperfective

habitual \textit{náku-}

subjunctive imperfective\\
\lspbottomrule
\end{tabularx}
\end{table}

As \tabref{tab:3:5} shows, each melodic tone is used by more than one TAM construction, and there is no obvious semantic link between TAM constructions using the same melodic tone pattern. It is therefore not possible to assign a meaning to melodic tones. TAM constructions may combine several melodic tones, and only three TAM constructions do not use melodic tone at all: these are all recent grammaticalizations derived from an infinitive verb, a verb form that does also not use melodic tone.

Melodic tones are marked in the phonological transcription (the second line of the examples) with acute accent combined with underlining, to distinguish them from underlying high tones, which are marked with an acute accent without underlining. Underlying high tones that are deleted as the result of melodic tone pattern 4 will be marked with a following \textsubscript{H}. These conventions are summarized in \tabref{tab:3:6}. As no single function can be linked to melodic tones, they are not represented with a gloss in the third line.

\begin{table}
\label{bkm:Ref507424979}\caption{\label{tab:3:6}Melodic tone marking conventions}

\begin{tabularx}{\textwidth}{XX}

\lsptoprule

Underlying (lexical) tone & /c\'{v}/, e.g. /ku-kám-a/ ‘to milk’\\
Melodic tone & /cv\textsubscript{H}/, e.g. /ndí̲-ra-kám-a/ ‘I will milk.’\\
Tones deleted as the result of MT4 & /cv̲/, e.g. /ndi-ka\textsubscript{H}m-á̲/ ‘I am milking.’\\
\lspbottomrule
\end{tabularx}
\end{table}

Melodic tones and underlying tones are treated the same in the phonology of Fwe, with one exception: melodic tone pattern 4 only deletes underlying tones, not melodic tones. The rone rules set out in \sectref{bkm:Ref445214894} apply to melodic and underlying tones in the same way.

The following sections give a discussion and examples of the realization of melodic tone patterns in Fwe.

\subsection{Melodic Tone 1: H on the last mora}
\label{bkm:Ref71540310}\hypertarget{Toc75352629}{}\label{bkm:Ref71540433}
Melodic Tone 1 (MT 1) is assigned to the last mora of the inflected verb. Examples are given with verbs in the present in (\ref{bkm:Ref70516255}), the subjunctive in (\ref{bkm:Ref70516266}), and the near future perfective in (\ref{bkm:Ref70516275}): the vowel carrying the melodic tone is underlined in the phonological transcription.

\ea
\label{bkm:Ref70516255}
bàhùrá ꜝshûnù\\
\gll ba-hur-á̲    shúnu\\
\textsc{sm}\textsubscript{2}-arrive-\textsc{fv}  today\\
\glt ‘They arrive today.’
\z

\ea
\label{bkm:Ref70516266}
mbòbáhùré ꜝshûnù\\
\gll mbo-bá̲-hur-é̲      shúnu\\
\textsc{near}.\textsc{fut}-\textsc{sm}\textsubscript{2}-arrive-\textsc{pfv}.\textsc{sbjv}  today\\
\glt ‘They will arrive today.’ (NF\_Elic15)
\z

\ea
\label{bkm:Ref70516275}
òshòtòké òmùkwàkwà\\
\gll o-sho\textsubscript{H}tok-é̲      o-mu-kwakwa\\
\textsc{sm}\textsubscript{2SG}-jump-\textsc{pfv}.\textsc{sbjv}  \textsc{aug}-\textsc{np}\textsubscript{3}-road\\
\glt ‘You should cross the road.’ (NF\_Elic17)
\z

In many cases, the last mora of the verb is the final vowel suffix. However, MT1 cannot be analyzed as underlyingly belonging to the final vowel suffix, as the final vowel suffixes on which it occurs, \textsc{fv} \textit{-a} and subjunctive \textit{-e}, occur without a high tone in other TAM inflections. Furthermore, when verbs that take MT1 include a post-verbal locative clitic, MT 1 is assigned to this clitic, as illustrated with the clitic \textit{=mo} in (\ref{bkm:Ref70516336}).

\ea
\label{bkm:Ref70516336}
…ndìhìkìrèmó bùjwàːrà\\
\gll ndi-hi\textsubscript{H}k-ir-e=mó̲      bu-jwara\\
\textsc{sm}\textsubscript{1SG}-cook-\textsc{appl}-\textsc{pfv}.\textsc{sbjv}=\textsc{loc}\textsubscript{18}  \textsc{np}\textsubscript{14}-beer\\
\glt ‘…so that I cook beer in it.’ (NF\_Elic15)
\z

MT 1 targets the mora, not the syllable. When a verb has a bimoraic final syllable, as in (\ref{bkm:Ref98935316}), the melodic tone is assigned to the second mora, which can be seen from the lack of high tone retraction in phrase-final contexts, as in (\ref{bkm:Ref98935317}).

\ea
\label{bkm:Ref98935316}
\gll /ba-nyw-.aá̲.    o-bu-jwara/ > bànywáː òbùjwàrà\\
\textsc{sm}\textsubscript{2}-drink-\textsc{fv}    \textsc{aug}-\textsc{np}\textsubscript{14}-beer\\
\glt ‘They drink beer.’
\z

\ea
\label{bkm:Ref98935317}
\gll /ba-nyw-.aá̲./ > bànywâː\\
\textsc{sm}\textsubscript{2}-drink-\textsc{fv}\\
\glt ‘They drink.’ (NF\_Elic15)
\z

MT 1 has two different realizations, based on the segmental shape of the verb stem. If the penultimate syllable has a long vowel, the H tone is not assigned to the last mora but to the penultimate syllable. This is illustrated in (\ref{bkm:Ref98935376}) with the verb stem \textit{rìmà} ‘farm’, which has no long vowels and therefore MT 1 is assigned to the last mora of the word, compared to the verb stem \textit{tòmbwèrà} ‘weed’ in (\ref{bkm:Ref98935417}), which has a lengthened penultimate vowel (on account of the preced\-ing glide), and here MT 1 is assigned to the penultimate syllable.

\ea
\label{bkm:Ref98935376}
tùrìmá shûnù\\
\gll tu-rim-á̲    shúnu\\
\textsc{sm}\textsubscript{1PL}-farm-\textsc{fv}  today\\
\glt ‘We farm today.’
\z

\ea
\label{bkm:Ref98935417}
tùtòmbwérà shûnù\\
\gll tu-tombwér-a  shúnu\\
\textsc{sm}1PL-weed-\textsc{fv}  today\\
\glt ‘We weed today.’ (NF\_Elic15)
\z

When MT 1 is used with a verb stem that has two moras both in the last and in the penultimate syllable, the melodic tone is assigned to the last verb mora, as in (\ref{bkm:Ref98935437}).

\ea
\label{bkm:Ref98935437}
ndi-nyans-á > ndìnyàːnsâː\\
*ndi-nyáns-a > ndìnyâːnsàː\\
\textsc{sm}\textsubscript{1SG}-accuse-\textsc{fv}\\
\glt ‘I accuse.’ (NF\_Elic15)
\z

The alternation between final and penultimate assignment of this melodic tone cannot be inter\-preted in terms of the tone rules that are used in Fwe, but should nonetheless be analyzed as exponents of the same melodic tone: the final and penultimate assignment are in complete complementary distribution, and are found in all TAM constructions that use MT1. The assignment of a penultimate high tone can thus be seen as an allophonic variant of the assignment of a final high tone, conditioned by the phonologi\-cal shape of the penultimate syllable. \tabref{tab:3:7} summarizes the realization of melodic tone 1 on different stem shapes.

\begin{table}
\label{bkm:Ref489535613}\caption{\label{tab:3:7}The realization of melodic tone 1}

\begin{tabular}{llll}
\lsptoprule
\multicolumn{2}{c}{Last mora} & \multicolumn{2}{c}{Penultimate syllable}\\
\cmidrule(r){1-2}\cmidrule(l){3-4}
CVCV & {\itshape shèká} & CVːCV & {\itshape zwáːtà}\\
CVCVː & {\itshape bùzáː} &  & \\
CVːCVː & \textit{nyàːnsáː} &  & \\
\lspbottomrule
\end{tabular}
\end{table}

Melodic tone 1 is used in six different TAM constructions: the present; the remote past perfec\-tive; the near future perfective; the negative stative; the subjunctive; and the relative clause form of the remote past perfective. As the near future perfective is based on the subjunctive, and the remote past perfective is historically based on the present, it is likely that the present and subjunctive were the first to use this melodic tone, and it was subsequently maintained in new constructions that grammaticalized from them.

All TAM constructions that use melodic tone 1 also use melodic tone pattern 4, the deletion of underlying tones (see \tabref{tab:3:5}). Melodic tone 4 is not an inherent characteristic of MT 1 alone, but is also used in combination with other melodic tones.

\subsection{Melodic Tone 2: H on the subject marker}\largerpage[-1]
\hypertarget{Toc75352630}{}
Melodic tone pattern 2 (MT 2) assigns a high tone to the verb’s subject marker. An example is given with the remote future construction as used in Zambian Fwe in (\ref{bkm:Ref98935540}).

\ea
\label{bkm:Ref98935540}
\glll nàːndínàshòshòtà\\
na-ndí̲-na-shoshot-a\\
\textsc{rem}-\textsc{sm}\textsubscript{1SG}-\textsc{rem}.\textsc{fut}-whisper-\textsc{fv}\\
\glt ‘I will whisper.’ (ZF\_Elic14)
\z

MT 2 is used in the remote past imperfective, the remote fu\-ture, the near future perfective, the remote past perfective, and in all relative clause verbs except the near past perfective. For the remote future, the high tone on the sub\-ject marker is the result of an earlier high-toned prefix \textit{á-} which can still be realized as such in Namibian Fwe (see \sectref{bkm:Ref489268446}). Some of the other TAM constructions using MT2 appear to be grammaticalizations from an earlier relative clause verb; this is clearest for the remote past imperfective (see \sectref{bkm:Ref488767517}), and possibly also the near future based on the perfective subjunctive (see \sectref{bkm:Ref489268312}). The almost ubiquitous use of MT2 in relative clause verbs suggests that it started out in this context, and spread to other inflections as they grammaticalized from earlier relative clause verbs.

\subsection{Melodic Tone 3: H on second stem syllable}
\label{bkm:Ref71540337}\hypertarget{Toc75352631}{}\label{bkm:Ref71540417}
Melodic tone pattern 3 (MT 3) assigns a high tone to the second syllable of the verb stem. This is illustrated with the negative present in (\ref{bkm:Ref505959496}).

\ea
\label{bkm:Ref505959496}
\glll kàyìòːrésèkì\\
ka-i-oːr-é̲sek-i\\
\textsc{neg}-\textsc{sm}\textsubscript{9}-can-\textsc{neut}-\textsc{neg}\\
\glt ‘It is not possible.’ (ZF\_Conv13)
\z

In some Bantu languages, object markers are counted as part of the verb stem for tone assignment \citep{Marlo2013}. This is not the case in Fwe; melodic tone 3 is invariably assigned to the second syllable of the verb stem, counting from the first syllable of the stem and disregarding object markers, as seen in (\ref{bkm:Ref98935671}--\ref{bkm:Ref98935672}).\largerpage[-1]

\ea
\label{bkm:Ref98935671}
Melodic tone 3: without an object marker\\
ndàrìndîrì\\
ndi-a-rind-í̲r-i\\
\textsc{sm}\textsubscript{1SG}-\textsc{pst}-wait-\textsc{appl}-\textsc{npst}.\textsc{pfv}\\
\glt ‘I’ve waited for.’
\z

\ea
\label{bkm:Ref98935672}
Melodic tone 3: with an object marker\\
ndàkùrìndîrì\\
ndi-a-ku-rind-í̲r-i\\
\textsc{sm}\textsubscript{1SG}-\textsc{pst}-\textsc{om}\textsubscript{2SG}-wait-\textsc{appl}-\textsc{npst}.\textsc{pfv}\\
\glt ‘I’ve waited for you.’ (NF\_Elic15)
\z

Melodic tone 3 is realized on the penultimate syllable, rather than the second stem syllable, under two conditions. The first is when this melodic tone pattern is used with monosyllabic verb stems, as in (\ref{bkm:Ref99615756}). As these lack a second stem syllable, MT3 is assigned to the verb’s penultimate syllable, which may contain markers with various functions, including subject markers, object markers, tense markers, or the distal marker.\pagebreak

\ea
\label{bkm:Ref99615756}
  Melodic tone 3 with monosyllabic verbs: H on the penultimate syllable

\ea
\glll tàːndînywìː\\
ta-ndí̲-nyw-i\\
\textsc{neg}-\textsc{sm}\textsubscript{1SG}-drink-\textsc{neg}\\
\glt ‘I don’t drink.’

\ex
\glll ndìnânywìː\\
ndi-ná̲-nyw-i\\
\textsc{sm}\textsubscript{1SG}-\textsc{pst}-drink-\textsc{npst}.\textsc{pfv}\\
\glt ‘I drank.’

\ex
\glll ndìnàkûwì\\
ndi-na-kú̲-w-i\\
\textsc{sm}\textsubscript{1SG}-\textsc{pst}-\textsc{om}\textsubscript{2SG}-give-\textsc{npst}.\textsc{pfv}\\
\glt ‘I have given you.’ (ZF\_Elic14)

\ex
\glll kàːndìkârì\\
ka-ndi-ká̲-r-i\\
\textsc{neg}-\textsc{sm}\textsubscript{1SG}-\textsc{dist}-eat-\textsc{neg}\\
\glt ‘I don’t eat there.’ (NF\_Elic15)
\z\z

Melodic tone 3 also surfaces on the penultimate syllable when this syllable contains a long vowel, as in (\ref{bkm:Ref506216090}), where the penultimate syllable is lengthened on account of the following nasal consonant cluster. This conditioning is similar to that of MT 1, which also surfaces on the penultimate syllable if it contains a long vowel.

\ea
\label{bkm:Ref506216090}
\glll ndìnàyêndì\\
ndi-na-é̲nd-i\\
\textsc{sm}\textsubscript{1SG}-\textsc{pst}-go-\textsc{npst}.\textsc{pfv}\\
\glt ‘I went.’ (ZF\_Elic14)
\z

Melodic tone 3 is used with four TAM constructions: the negative present; the near past perfec\-tive; the stative (with the exception of negated statives and statives with a disyllabic verb stem, see \ref{bkm:Ref431984198} for details); and the perfective subjunctive with object marker. The stative combines MT 3 with the deletion of lexical high tones (melodic tone 4), the other three constructions maintain lexical high tones.

\subsection{Melodic Tone 4: deletion of underlying high tones}
\label{bkm:Ref71540134}\hypertarget{Toc75352632}{}
Melodic tone pattern 4 (MT 4) does not add a high tone, but rather deletes the lexical high tones of the verb. This is illustrated in (\ref{bkm:Ref507575156}) with the high-toned verb root \textit{bútuk} ‘run’, which loses its high tone when used in the present, one of the TAM construc\-tions that use MT 4. Deleted high tones are marked by subscript \textsubscript{H} after the syllable originally bearing the high tone.

\ea
\label{bkm:Ref507575156}
\glll ndìbùtúkà\\
ndi-bu\textsubscript{H}tuk-á̲\\
\textsc{sm}\textsubscript{1SG}-run-\textsc{fv}\\
\glt ‘I run.’ (NF\_Elic15)
\z

MT 4 also deletes high tones that are associated with affixes, such as object markers, as in (\ref{bkm:Ref70517452}), where the underlyingly high-toned object marker of class 2 \textit{bá}- is realized as low-toned \textit{bà-} when used with a present tense verb. MT 4 also affects other grammatical affixes, such as the high-toned persistive prefix \textit{shí-}, as in (\ref{bkm:Ref505960227}).

\ea
\label{bkm:Ref70517452}
\glll ndìbàshákà\\
ndi-ba\textsubscript{H}-shak-á̲\\
\textsc{sm}\textsubscript{1SG}-\textsc{om}\textsubscript{2}-like-\textsc{fv}\\
\glt ‘I like them.’ (ZF\_Elic14)
\z

\ea
\label{bkm:Ref505960227}
\glll ndìshìhôːndà\\
ndi-shi\textsubscript{H}-hó̲nd-a\\
\textsc{sm}\textsubscript{1SG}-\textsc{per}-cook-\textsc{fv}\\
\glt ‘I am still cooking.’ (NF\_Elic15)
\z

MT 4 always co-occurs with another melodic tone, and the deletion of high tones does not af\-fect the high tones assigned by this pattern. The present construction combines MT 4 with MT 1, which is assigned to the verb’s last mora, and this melodic tone is not affected by the deletion of underlying tones, as in (\ref{bkm:Ref489354952}).

\ea
\label{bkm:Ref489354952}
\glll bàzyìbàhárà\\
ba-zyi\textsubscript{H}b-ahar-á̲\\
\textsc{sm}\textsubscript{2}-know-\textsc{neut}-\textsc{fv}\\
\glt ‘S/he is famous.’ (NF\_Elic15)
\z

The floating high tone that is part of the lexical tone pattern of certain verb stems (see \sectref{bkm:Ref71540163}) poses a challenge for this analysis. As it is part of the verb’s lexical tone, it is usually deleted when a verb with a floating high tone is used in a TAM construction that makes use of MT 4. (\ref{bkm:Ref489354995}) shows the deletion of the floating high tone of the verb \textit{ˊtab} ‘answer’, used in the pre\-sent construction.

\ea
\label{bkm:Ref489354995}
\glll ndìtábà\\
ndi-tab-á̲\\
\textsc{sm}\textsubscript{1SG}-answer-\textsc{fv}\\
\glt ‘I answer.’ (NF\_Elic15)
\z

In one environment, however, MT 4 fails to affect floating tones. This is the case when the pre\-fix before the verb root, normally the syllable the floating tone attaches to, is a toneless prefix. In (\ref{bkm:Ref489276204}), the verb \textit{ˊtab} ‘answer’ is used in the present, with the toneless class 1 object marker \textit{mu-}. Although the present uses MT 4, the floating high tone of this verb is not deleted but realized on the object marker \textit{mu-}.

\ea
\label{bkm:Ref489276204}
\glll ndìmúꜝtábà\\
ndi-mú-tab-á̲\\
\textsc{sm}\textsubscript{1SG}-\textsc{om}\textsubscript{1}-answer-\textsc{fv}\\
\glt ‘I answer her/him.’
\z

The realization of floating tones in the present construction is also seen with other toneless pre\-fixes, such as the distal \textit{ka-} in (\ref{bkm:Ref99616216}), used with the verb \textit{ˊkar} ‘sit’.

\ea
\label{bkm:Ref99616216}
\glll ndìkáꜝkárà\\
ndi-ká-kar-á̲\\
\textsc{sm}\textsubscript{1SG}-\textsc{dist}-sit-\textsc{fv}\\
\glt ‘I sit there.’ (NF\_Elic17)
\z

Floating tones may not be realized on an underlyingly high-toned prefix, even though the use of melodic tone 4 deletes their high tones. This is shown with the high-toned ob\-ject marker \textit{bá-} in (\ref{bkm:Ref99616238}) and the high-toned persistive prefix \textit{shí-} in (\ref{bkm:Ref99616247}).

\ea
\label{bkm:Ref99616238}
\glll ndìbàtábà\\
ndi-ba\textsubscript{H}-tab-á̲\\
\textsc{sm}\textsubscript{1SG}-\textsc{om}\textsubscript{2}-answer-\textsc{fv}\\
\glt ‘I answer them.’
\z

\ea
\label{bkm:Ref99616247}
\glll ndìshìtábà\\
ndi-shi\textsubscript{H}-tab-á̲\\
\textsc{sm}\textsubscript{1SG}-\textsc{per}-answer-\textsc{fv}\\
\glt ‘I still answer.’ (NF\_Elic17)
\z

Although subject markers are underlyingly toneless, floating tones never attach to them in TAM constructions that use MT 4, such as the present in (\ref{bkm:Ref99616268}).

\ea
\label{bkm:Ref99616268}
\glll ndìtábà\\
ndi-tab-á̲\\
\textsc{sm}\textsubscript{1SG}-answer-\textsc{fv}\\
\glt ‘I answer.’ (NF\_Elic15)
\z

More research is needed to explain the complex interaction between floating tones and melodic tones, and to explain why these specific phonological and morphological environments allow for the realization of floating tones, where other lexical tones cannot be realized.

\subsection{No melodic high tones}
\label{bkm:Ref71540393}\hypertarget{Toc75352633}{}
As summarized in \tabref{tab:3:5}, there are three TAM constructions in Fwe that do not use melodic tones: the near past imperfective, one of the two habituals, and the subjunctive imperfective. The lack of melodic tone with these constructions is similar to the lack of melodic tone on infinitive verbs. These constructions also resemble the infinitive segmentally, as they all contain a syllable \textit{ku}, homophonous with the infinitive prefix. A more detailed account of the similarities between these constructions and the infinitive are given in \sectref{bkm:Ref451515182} on the near past imperfective, \sectref{bkm:Ref489363965} on the habitual, and \sectref{bkm:Ref489363941} on the subjunctive imperfective. These sections also argue in detail that these TAM constructions are the result of relatively recent grammaticalizations involving an inflected verb and an infinitive verb.

