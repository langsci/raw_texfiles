\chapter{Minor word categories}\largerpage
\hypertarget{Toc75352651}{}
This chapter discusses a number of minor syntactic categories: personal pronouns in \sectref{bkm:Ref488767962}, comitative clitics in \sectref{bkm:Ref486270340}, copulative prefixes in \sectref{bkm:Ref489963307}, appositive prefixes in \sectref{bkm:Ref492039371}, and adverbs in \sectref{bkm:Ref489870394}.

\section{Personal pronouns}
\label{bkm:Ref488767962}\hypertarget{Toc75352652}{}
Fwe has a set of personal pronouns that are used to refer to the first, second and third person singular and plural. The forms of these personal pronouns are given in \tabref{tab:5:1} The personal pronoun for the third person plural is identical to the demonstrative form \textit{àbó}, which is also used as a third person plural possessive (see \sectref{bkm:Ref491333327}). In Namibian Fwe, the initial vowel of the personal pronouns can be either \textit{e-} or \textit{i-}. Personal pronouns typically have a high tone on the last syllable, but this high tone may be intonational; personal pronouns are frequently used in contexts where they are directly followed by a pause, which seems to condition a rising intonation. Although intonation in Fwe has not been studied systematically, it is possible that the frequently attested final high tone on personal pronouns is intonational.

\begin{table}
\label{bkm:Ref463366758}\caption{\label{tab:5:1}Personal pronouns}
\begin{tabular}{lll}
\lsptoprule
& Singular & Plural\\
\midrule
first person & \textit{emé / imé} (‘I’) & \textit{eswé / iswé} (‘we’)\\
second person & \textit{ewé / iwé} (‘you’) & \textit{enwé / inwé} (‘you’)\\
third person & \textit{eyé / iyé} (‘he/she’) & (\textit{abó}) (‘they’)\\
\lspbottomrule
\end{tabular}
\end{table}

Personal pronouns are only used for human referents; to refer to non-human referents, demonstratives are used (see \sectref{bkm:Ref450747952}).

The involvement of a first, second or third person as a subject or object is usually marked with subject and object markers on the verb, except when it is in focus or topicalized. To mark a first, second or third person as topic, a personal pronoun is used in the left-dislocated position (see also \sectref{bkm:Ref403656711} on left dislocation), as in (\ref{bkm:Ref99021532}--\ref{bkm:Ref99021534}).

\ea
\label{bkm:Ref99021532}
cwárè \textbf{éyè} kàzyíː kùŋôrà\\
\gll cwaré  eyé    ka-a-zyí̲ː      ku-ŋór-a\\
then  \textsc{pers}\textsubscript{3SG}  \textsc{neg}-\textsc{sm}\textsubscript{1}-know.\textsc{stat}  \textsc{inf}-write-\textsc{fv}\\
\glt ‘But she, she doesn’t know how to write.’ (ZF\_Conv13)
\z

\ea
\label{bkm:Ref99021534}
\textbf{émè} kwààzy’ ómò sàké ndìmùpángîrè\\
\gll émè    ku-aazyá o-mo      saké  ndi-mu-pang-í̲r-e\\
\textsc{pers}\textsubscript{1SG}  \textsc{sm}\textsubscript{17}-be\_not 
\textsc{aug}-\textsc{dem}.\textsc{iii}\textsubscript{18}  if  \textsc{sm}\textsubscript{1SG}-\textsc{om}\textsubscript{1}-do-\textsc{appl}-\textsc{pfv}.\textsc{sbjv}\\
\glt ‘Me, there is nothing I can do for her.’ (NF\_Narr17)
\z

To express focus on the first, second or third person, a personal pronoun is used as the clefted element of a cleft construction (see also \sectref{bkm:Ref491333435} on cleft constructions). A clefted pronoun marking exclusive focus (‘only she, no one else’) is shown in (\ref{bkm:Ref496869630}), and a clefted pronoun marking information focus is shown in (\ref{bkm:Ref496869707}).

\ea
\label{bkm:Ref496869630}
\textbf{ndéyè} bùryó ꜝárèːtà èzìbyà mwíꜝrápà\\
\gll ndi-eyé  bu-ryó  á̲-reː\textsubscript{H}t-a e-zi-bya    mú-e-∅-rapá \\
\textsc{cop}-\textsc{pers}\textsubscript{3SG}  \textsc{np}\textsubscript{14}-only  \textsc{sm}\textsubscript{1}.\textsc{rel}-bring-\textsc{fv}
\textsc{aug}-\textsc{np}\textsubscript{8}-item  \textsc{np}\textsubscript{18}-\textsc{aug}-\textsc{np}\textsubscript{5}-courtyard\\
\glt ‘She is the only one who can bring items into the courtyard.’ (ZF\_Conv13)
\z

\ea
\label{bkm:Ref496869707}
èyí ènjûò \textbf{ndìmé} nìbáyìzyàːkírà\\
\gll e-í    e-N-júo ndi-mé  ni-bá̲-a-i\textsubscript{H}-zyaː\textsubscript{H}k-ir-á̲\\
\textsc{aug}-\textsc{dem}.\textsc{i}\textsubscript{9}  \textsc{aug}-\textsc{np}\textsubscript{9}-house \textsc{cop}-\textsc{pers}\textsubscript{1SG}  \textsc{rem}-\textsc{sm}\textsubscript{2}-\textsc{pst}-\textsc{om}\textsubscript{9}-build-\textsc{appl}-\textsc{fv}<\textsc{rel}>\\
\glt ‘This house, it is me that it is was built for.’ (NF\_Elic15)
\z

Personal pronouns are also required when the first, second or third person is used with a comitative or a copula, as in (\ref{bkm:Ref99021431}--\ref{bkm:Ref99021434}).

\ea
\label{bkm:Ref99021431}
mbùryó ꜝndízànà \textbf{néwè}\\
\gll mbu-ryó  ndí̲-zan-a    ne=wé\\
only    \textsc{sm}\textsubscript{1SG}-play-\textsc{fv}  \textsc{com}=\textsc{pers}\textsubscript{2SG}\\
\glt ‘I’m just joking with you.’ (NF\_Elic15)
\z

\ea
\label{bkm:Ref99021434}
\textbf{ndìmé} ꜝSánètì Cábòrà\\
\gll ndi-mé  sáneti  cábora\\
\textsc{cop}-\textsc{pers}\textsubscript{1SG}  Saneti  Chabola\\
\glt ‘I am Saneti Chabola.’ (NF\_Narr17)
\z

Personal pronouns for the second person are frequently used as a term of address, as in (\ref{bkm:Ref99021603}--\ref{bkm:Ref99021607}).

\ea
\label{bkm:Ref99021603}
íwè cìnjí àhò kórâːrì\\
\gll iwé    ∅-ci-njí    a-ho      ka-ó̲-rá̲ːr-i\\
\textsc{pers}\textsubscript{2SG} \textsc{cop}-\textsc{np}\textsubscript{7}-what  \textsc{aug}-\textsc{dem}.\textsc{iii}\textsubscript{16} \textsc{neg}-\textsc{sm}\textsubscript{2SG}-sleep-\textsc{neg}\\
\glt ‘You! Why are you not sleeping!’ (NF\_Narr15)
\z

\ea
\label{bkm:Ref99021607}
íwè òtèèzé ꜝkúnù\\
\gll iwé    o-te\textsubscript{H}ez-é̲      kunú\\
\textsc{pers}\textsubscript{2SG}  \textsc{sm}\textsubscript{2SG}-listen-\textsc{pfv}.\textsc{sbjv}  \textsc{dem}.\textsc{ii}\textsubscript{17}\\
\glt ‘You, listen here.’ (NF\_Narr17)
\z
\section{Comitatives}
\label{bkm:Ref486270340}\hypertarget{Toc75352653}{}
The comitative expresses a variety of meanings, some of which are captured by the English translation ‘and’. It is expressed by a clitic with the form \textit{nV=}, where \textit{V} stands for a vowel /a/, /e/, /o/, or /i/. When used with nouns that can take an augment, the vowel of the augment determines the vowel of the comitative, as in (\ref{bkm:Ref99021991}--\ref{bkm:Ref99021994}).

\ea
\label{bkm:Ref99021991}
nòngwènà (< òngwènà ‘crocodile’)\\
no=∅-ngwena\\
\textsc{com}=\textsc{np}\textsubscript{1a}-crocodile\\
\glt ‘and a crocodile’
\z

\ea
nénswì (< énswì ‘fish’)\\
ne=N-swí\\
\textsc{com}=\textsc{np}\textsubscript{9}-fish\\
\glt ‘and a fish’
\z

\ea
\label{bkm:Ref99021994}
nàkàfùrò (< àkàfùrò ‘knife’)\\
na=ka-furo\\
\textsc{com}=\textsc{np}\textsubscript{12}-knife\\
\glt ‘and a knife’
\z

The comitative \textit{nV=} in Fwe is the reflex of a marker *na reconstructed for Bantu as an “associative index” by {\citet{Meeussen1967}}. Traces of the original vowel /a/ in this marker are no longer found in Fwe; the vowel of the comitative fully assimilates to the augment of the noun to which it prefixes. There are also cases where the comitative in Fwe does not copy the vowel of the augment, as discussed below, but even in these cases, the original vowel /a/ never surfaces.

When the comitative is cliticized to a word that cannot take an augment, it is realized as \textit{na=}, \textit{ne=}, or \textit{ni=}. This is the case with inflected verbs, where the comitative is realized as \textit{na-} in Zambian Fwe, as in (\ref{bkm:Ref476926475}), and as \textit{ni}=, as in (\ref{bkm:Ref476926486}) or \textit{ne}=, as in (\ref{bkm:Ref476926495}), in Namibian Fwe.

\ea
\label{bkm:Ref476926475}
kàndípàkíté mwâncè \textbf{nàndìkwèsì} ndìtòmbwêrà\\
\gll ka-ndí̲-pak-í̲te      mw-ánce na=ndi-kwesi  ndi-tombwé̲r-a\\
\textsc{rpp}-\textsc{sm}\textsubscript{1SG}-carry\_on\_back-\textsc{stat}  \textsc{np}\textsubscript{1}-child 
\textsc{com}=\textsc{sm}\textsubscript{1SG}-\textsc{prog}  \textsc{sm}\textsubscript{1SG}-weed-\textsc{fv}\\
\glt ‘I was carrying my child on my back while I was weeding.’ (ZF\_Elic14)
\z

\ea
\label{bkm:Ref476926486}
ndàkùrí kùbútùkà \textbf{nìndìzìmbùrùká} ègrâùndì\\
\gll ndi-aku-rí    ku-bútuk-a  ni=ndi-zi\textsubscript{H}mburuk-á̲     e-∅-gráundi \\
\textsc{sm}\textsubscript{1SG}-\textsc{npst}.\textsc{ipfv}-be  \textsc{inf}-run-\textsc{fv}
\textsc{com}=\textsc{sm}\textsubscript{1SG}-surround-\textsc{fv}  \textsc{aug}-\textsc{np}\textsubscript{9}-sports\_field\\
\glt ‘I was running around the sports field.’ [lit.: ‘I was running while surrounding the sportsfield.’] (NF\_Elic15)
\z

\ea
\label{bkm:Ref476926495}
ndìzyìmáná \textbf{nèndìtóntwêrè}\\
\gll ndi-zyi\textsubscript{H}man-á̲  ne=ndi-to\textsubscript{H}ntwé̲re\\
\textsc{sm}\textsubscript{1SG}-stand-\textsc{fv}  \textsc{com}=\textsc{sm}\textsubscript{1SG}-be\_cold.\textsc{stat}\\
\glt ‘I stand up quietly.’ [lit. ‘I stand up while I am quiet’] (NF\_Elic15)
\z

With nouns that never take an augment, the form of the comitative is \textit{ni=, ne=} or \textit{na=}. For instance, with nouns with a secondary prefix \textit{ba-} (used to mark respect; see \sectref{bkm:Ref499035982}), the form of the comitative may be \textit{na=} or \textit{ne=} in Zambian Fwe, as in (\ref{bkm:Ref99022089}--\ref{bkm:Ref99022091}), and \textit{ni=} in Namibian Fwe, as in (\ref{bkm:Ref99022109}).

\ea
\label{bkm:Ref99022089}
kàtúrèrè kúrùwà \textbf{nèbàmùkéntù} wángù\\
\gll ka-tú̲-re\textsubscript{H}re      kú-ru-wa ne=ba-mu-kéntu    u-angú \\
\textsc{pst}.\textsc{ipfv}-\textsc{sm}\textsubscript{1PL}-sleep.\textsc{stat}  \textsc{np}\textsubscript{17}-\textsc{np}\textsubscript{11}-field
\textsc{com}=\textsc{np}\textsubscript{2}-\textsc{np}\textsubscript{1}-woman  \textsc{pp}\textsubscript{1}-\textsc{poss}\textsubscript{1SG}\\
\glt ‘My wife and I were sleeping at the field.’ (ZF\_Elic13)
\z

\ea
\label{bkm:Ref99022091}
\textbf{nàbàmùkéntù} wángù\\
\gll na=ba-mu-kéntu    u-angú\\
\textsc{com}=\textsc{np}\textsubscript{2}-\textsc{np}\textsubscript{1}-woman  \textsc{pp}\textsubscript{1}-\textsc{poss}\textsubscript{1SG}\\
\glt ‘And also my wife.’ (ZF\_Narr15)
\z

\ea
\label{bkm:Ref99022109}
ndìhárá \textbf{nìbàmùkéntù} wángù nàbánàngù\\
\gll ndi-ha\textsubscript{H}r-á̲    ni=ba-mu-kéntu    u-angú  na=ba-ána-angu\\
\textsc{sm}\textsubscript{1SG}-live-\textsc{fv}  \textsc{com}=\textsc{np}\textsubscript{2}-\textsc{np}\textsubscript{1}-woman  \textsc{pp}\textsubscript{1}-\textsc{poss}\textsubscript{1SG}  \textsc{com}=\textsc{np}\textsubscript{2}-child-\textsc{poss}\textsubscript{1SG}\\
\glt ‘I live with my wife and children.’ (NF\_Elic15)
\z

The same variation in the realization of the comitative is seen with other nouns referring to kinship terms or social relations, even though these nouns do take an augment, such as the noun phrase \textit{mùkéntù wàkwé} ‘his wife’ in (\ref{bkm:Ref71201279}) and the noun \textit{mwânè} ‘her child’ in (\ref{bkm:Ref71201293}).

\ea
\label{bkm:Ref71201279}
káhùpúrà ìyé témà \textbf{nèmùkéntù} \textbf{wàkwé} mómò àkàráːrè\\
\gll ka-á̲-hupur-á̲    iyé  téma  ne=mu-kéntu  u-akwé N-ó-mo    a-ka-raː\textsubscript{H}r-é̲\\
\textsc{pst}.\textsc{ipfv}-\textsc{sm}\textsubscript{1}-think-\textsc{fv}  that  maybe  \textsc{com}=\textsc{np}\textsubscript{1}-woman  \textsc{pp}\textsubscript{1}-\textsc{poss}\textsubscript{1SG}
\textsc{cop}-\textsc{aug}-\textsc{dem}.\textsc{iii}\textsubscript{18} \textsc{sm}\textsubscript{1}-\textsc{dist}-sleep-\textsc{pfv}.\textsc{sbjv}\\
\glt ‘He thought that maybe his wife would also be sleeping in there.’ (NF\_Narr15)
\z

\ea
\label{bkm:Ref71201293}
òmùbèrékì kàswànéré kùkèːzyà kúnò \textbf{nèmwânè}\\
\gll o-mu-beréki    ka-a-swaneré̲ ku-keːzy-a  kúno    ne=mu-án-e\\
\textsc{aug}-\textsc{np}\textsubscript{1}-worker  \textsc{neg}-\textsc{sm}\textsubscript{1}-must
\textsc{inf}-come-\textsc{fv}  \textsc{dem}.\textsc{ii}\textsubscript{17} \textsc{com}=\textsc{np}\textsubscript{1}-child-\textsc{poss}\textsubscript{3SG}\\
\glt ‘A worker must not come here with her child.’ (ZF\_Conv13)
\z

In Namibian Fwe, the use of the \textit{ne=} form with nouns that do not have an \textit{e-} augment is restricted to a handful of nouns referring to kinship relations. In Zambian Fwe, the \textit{ne=} form is also frequently found with nouns of class 6 or 12. These nouns take an augment \textit{a-}, and therefore the expected comitative form would be \textit{na=}, as in the Namibian Fwe example in (\ref{bkm:Ref436820039}); in Zambian Fwe, the comitative with these nouns is often realized as \textit{ne=}, as in (\ref{bkm:Ref437511600}).

\ea
\label{bkm:Ref436820039}
ndìbyârà òmùndárè \textbf{nàmàbérè}\\
\gll ndi-byá̲r-a    o-mu-ndaré    na=ma-beré\\
\textsc{sm}\textsubscript{1SG}-plant-\textsc{fv}  \textsc{aug}-\textsc{np}\textsubscript{3}-maize  \textsc{com}=\textsc{np}\textsubscript{6}-millet\\
\glt ‘I grow maize and millet.’ (NF\_Elic15)
\z

\ea
\label{bkm:Ref437511600}
mùndáré \textbf{nèmàhìrà}\\
\gll mu-ndaré  ne=ma-ir-a\\
\textsc{np}\textsubscript{3}-maize  \textsc{com}=\textsc{np}\textsubscript{6}-sorghum\\
\glt ‘maize and sorghum’ (ZF\_Elic14)
\z

The \textit{ne=} form of the comitative with nouns with an \textit{a}- augment is not obligatory in Zambian Fwe, though. Both the \textit{ne=} and \textit{na=} forms of the comitative are found with nouns with an \textit{a-} augment, as seen in (\ref{bkm:Ref98512623}).

\ea
\label{bkm:Ref98512623}
òmbwá nàkásè {\textasciitilde} òmbwá nèkásè\\
\gll o-∅-mbwá    na/ne=ka-sé\\
\textsc{aug}-\textsc{np}\textsubscript{1a}-dog  \textsc{com}=\textsc{np}\textsubscript{12}-cat\\
\glt ‘a dog and a cat’\footnotemark{} (ZF\_Elic14)
\z
\footnotetext{There is even an example of a comitative \textit{no=} used with a noun that takes an augment \textit{e}\nobreakdash-. The only occurrence of this is with the noun \textit{eminwe} ‘fingers’ used in counting; in this case the comitative is always realized as \textit{no=}.\textit{zònéː nòmìnwè yòbírè}
\ea
\gll zi-o=néː    no-mi-nwe    i-o=biré\\
\textsc{pp}\textsubscript{10}-\textsc{con}=four  \textsc{com}-\textsc{np}\textsubscript{4}-finger  \textsc{pp}\textsubscript{4}-\textsc{con}=two\\
\glt ‘six (lit. four and two fingers)’
\zlast
}

The comitative clitic is phonologically dependent on the word to which it is attached, as seen from its interaction with the augment, which determines the quality of the vowel. Morphosyntactically, the comitative clitic is relatively free. The comitative precedes all prefixes: when added to a noun, the comitative precedes the noun’s (primary) nominal prefix, but also its secondary nominal prefix, such as those of the locative classes 16-18, as shown in (\ref{bkm:Ref437423816}), or the class 2 prefix used as secondary prefix, as seen in (\ref{bkm:Ref437423825}).

\ea
\label{bkm:Ref437423816}
ndìkwèsí njûò mwàìmûshò \textbf{nòkwásìnjèmbèrà}\\
\gll ndi-kwesí  N-júo    mwa-imúsho  no=kwá-sinjembera\\
\textsc{sm}\textsubscript{1SG}-have  \textsc{np}\textsubscript{9}-house  \textsc{np}\textsubscript{18}-Imusho  \textsc{com}=\textsc{np}\textsubscript{17}-Sinjembela\\
\glt ‘I have a house in Imusho and in Sinjembela.’ (ZF\_Elic14)
\z

\ea
\label{bkm:Ref437423825}
\textbf{nàbàmùkéntù} wángù\\
\gll na=ba-mu-kéntu    u-angú\\
\textsc{com}=\textsc{np}\textsubscript{2}-\textsc{np}\textsubscript{1}-woman  \textsc{pp}\textsubscript{1}-\textsc{poss}\textsubscript{1SG}\\
\glt ‘And also my wife.’ (ZF\_Narr15)
\z

Furthermore, the comitative clitic may attach to any word: nouns, pronouns, infinitives, and inflected verbs. There are some similarities between the comitative and the connective clitic (see \sectref{bkm:Ref492133177}), which is also phrase-initial and interacts with the augment. However, whereas the connective may be targeted by H spread, a tone process that never crosses word boundaries, H spread never targets the comitative clitic. Furthermore, when the comitative and connective are combined, the comitative precedes the connective clitic, as seen in (\ref{bkm:Ref505886208}).

\ea
\label{bkm:Ref505886208}
nàkíhùrìrì mùròrà wàkwê \textbf{nòwámùkéntù} \textbf{wàkwê}\\
\gll na-kí-ur-ir-i          mu-rora  u-akwé no=u-á=mu-kéntu      u-akwé\\
\textsc{sm}\textsubscript{1}-\textsc{pst}-\textsc{refl}-buy-\textsc{appl}-\textsc{npst}.\textsc{pfv}  \textsc{np}\textsubscript{3}-soap  \textsc{pp}\textsubscript{3}-\textsc{poss}\textsubscript{3SG}
\textsc{com}=\textsc{pp}\textsubscript{3}-\textsc{con}=\textsc{np}\textsubscript{1}-woman  \textsc{pp}\textsubscript{1}-\textsc{poss}\textsubscript{3SG}\\
\glt ‘He has bought soap for himself and his wife.’ (ZF\_Elic14)
\z

Finally, whereas the comitative may attach to any word, including inflected verbs, the connective is limited to nominal elements. These facts suggest that the connective clitic is more closely integrated into the word it attaches to than the comitative, though both can be considered clitics.

One of the main functions of the comitative is to express conjunctive coordination, for instance, of two nouns, as in (\ref{bkm:Ref437512254}), or of two pronouns, as in (\ref{bkm:Ref437512262}).

\ea
\label{bkm:Ref437512254}
ndávú nònjòvù\\
\gll ∅-ndavú   no=∅-njovu\\
\textsc{np}\textsubscript{1a}-lion   \textsc{com}=\textsc{np}\textsubscript{1a}-elephant\\
\glt ‘a lion and an elephant’
\z

\ea
\label{bkm:Ref437512262}
èmé nêwè\\
\gll emé    né=we\\
\textsc{pers}\textsubscript{1SG}  \textsc{com}=\textsc{pers}\textsubscript{2SG}\\
\glt ‘you and me’ (ZF\_Elic14)
\z

When the comitative is used with a conjunctive function, the comitative usually appears on the second conjunct only. The comitative may also be repeated on both conjuncts to express emphatic coordination, as in (\ref{bkm:Ref99022262}).

\ea
\label{bkm:Ref99022262}
\textbf{nò}mwáncè \textbf{nò}mùkêntù kwàázy’ écò kàbàzyîː\\
\gll no=mu-ánce    no=mu-kéntu ku-aazyá  e-có    ka-ba-zyi-í̲\\
\textsc{com}=\textsc{np}\textsubscript{1}-child  \textsc{com}=\textsc{np}\textsubscript{1}-woman
\textsc{sm}\textsubscript{17}-be\_not  \textsc{aug}-\textsc{dem}.\textsc{iii}\textsubscript{7}  \textsc{pst}.\textsc{ipfv}-\textsc{sm}\textsubscript{2}-know.\textsc{stat}-\textsc{neg}\\
\glt ‘Both the child and the wife, they knew nothing.’ (NF\_Narr15)
\z

The comitative is used with an infinitive to create a consecutive verb form, which expresses subsequent action, as illustrated in (\ref{bkm:Ref71201810}--\ref{bkm:Ref71201812}) (see \sectref{bkm:Ref494204746} on the consecutive).

\ea
\label{bkm:Ref71201810}
àpàpàúrà nòkùhìnd’ òmùzîò\\
\gll a-papaur-á̲    no=ku-hind-a  o-mu-zío\\
\textsc{sm}\textsubscript{1}-divide-\textsc{fv}  \textsc{com}=\textsc{inf}-take-\textsc{fv}  \textsc{aug}-\textsc{np}\textsubscript{3}-load\\
\glt ‘He divides the animal into pieces and takes it as a load.’ (NF\_Narr15)
\z

\ea
\label{bkm:Ref71201812}
ndàtóːꜝrí cìshámù nòkùdàmá zyôkà\\
\gll ndi-a-tóːr-í        ci-shamú no=ku-dam-á  ∅-zyóka \\
\textsc{sm}\textsubscript{1SG}-\textsc{pst}-pick\_up-\textsc{npst}.\textsc{pfv}  \textsc{np}\textsubscript{7}-stick \textsc{com}=\textsc{inf}-beat-\textsc{fv}  \textsc{np}\textsubscript{5}-snake\\
\glt ‘I took a stick and beat the snake.’ (ZF\_Narr13)
\z

The comitative can also be used with inflected verbs, which are then interpreted as simultaneous with the previous inflected verb. The comitative may only be used on a verb in the present tense construction; its temporal implications are then determined by the inflection of the preceding inflected verb: both events are interpreted as present if the preceding verb is in the present construction, as in (\ref{bkm:Ref476929407}), or past, if the preceding verb is inflected for past tense, as in (\ref{bkm:Ref476929416}).

\ea
\label{bkm:Ref476929407}
ndìshúwírè ònjòvù nàjwêngà\\
\gll ndi-shu\textsubscript{H}-í̲re    o-∅-njovu    na=a-jwé̲ng-a\\
\textsc{sm}\textsubscript{1SG}-hear-\textsc{stat}  \textsc{aug}-\textsc{np}\textsubscript{1a}-elephant  \textsc{com}=\textsc{sm}\textsubscript{1}-shout-\textsc{fv}\\
\glt ‘I hear an elephant shouting.’
\z

\ea
\label{bkm:Ref476929416}
ndàbónì bâncè nìbàbùtúkà\\
\gll ndi-a-bón-i      ba-ánce  ni=ba-bu\textsubscript{H}tuk-á̲\\
\textsc{sm}\textsubscript{1SG}-\textsc{pst}-see-\textsc{npst}.\textsc{pfv}  \textsc{np}\textsubscript{2}-child  \textsc{com}=\textsc{sm}\textsubscript{2}-run-\textsc{fv}\\
\glt ‘I saw children running.’ (NF\_Elic15)
\z

A second major function of the comitative clitic in Fwe is to express comitative meaning, roughly translatable as ‘(together) with’, as in (\ref{bkm:Ref99022295}--\ref{bkm:Ref99022299}).\largerpage

\ea
\label{bkm:Ref99022295}
kàbáyêndà nàbàmbwá ꜝbábò\\
\gll ka-bá̲-é̲nd-a    na-ba-mbwá    ba-abó\\
\textsc{pst}.\textsc{ipfv}-\textsc{sm}\textsubscript{2}-go-\textsc{fv}  \textsc{com}=\textsc{np}\textsubscript{2}-dog  \textsc{pp}\textsubscript{2}-\textsc{dem}.\textsc{iii}\textsubscript{2}\\
\glt ‘She was walking with her dogs.’ (ZF\_Narr15)
\z

\ea
nènyàzì yákw’ ꜝákèrè\\
\gll ne=N-nyazi    i-akwé  á̲-ke\textsubscript{H}re\\
\textsc{com}=\textsc{np}\textsubscript{9}-lover  \textsc{pp}\-\textsubscript{9}-\textsc{poss}\textsubscript{3SG}  \textsc{sm}\textsubscript{1}.\textsc{rel}-sit.\textsc{stat}\\
\glt ‘She is with her lover.’ (ZF\_Conv13)
\z

\ea
\label{bkm:Ref99022299}
ndìsháká èntí nòmùzírìrì\\
\gll ndi-shak-á̲    e-N-tí    no=mu-zíriri\\
\textsc{sm}\textsubscript{1SG}-want-\textsc{fv}  \textsc{aug}-\textsc{np}\textsubscript{9}-tea  \textsc{com}=\textsc{np}\textsubscript{3}-fresh\_milk\\
\glt ‘I want tea with fresh milk.’ (ZF\_Elic14)
\z

Fwe can also use the comitative for a type of conjunction called ‘inclusory conjunction’ \citep{Haspelmath2007}. This involves a comitative-marked nominal which refers to a participant that is already implied by a plural pronoun or subject marker. In (\ref{bkm:Ref69997095}), the subjects ‘you and I’ are both covered by the first person plural subject marker \textit{tu}\nobreakdash- ‘we’ on the verb. The second person singular is expressed again through a comitative-marked personal pronoun \textit{ewe} ‘you (SG)’.

\ea
\label{bkm:Ref436905818}
\label{bkm:Ref69997095}
mbòtúyèndèrérè \textbf{néwè} kwíꜝtáwúnì\\
\gll mbo-tú̲-end-er-er-é      ne=wé    ku-é-∅-tawuní\\
\textsc{near}.\textsc{fut}-\textsc{sm}\textsubscript{1PL}-go-\textsc{int}-\textsc{pfv}.\textsc{sbjv}  \textsc{com}=\textsc{pers}\textsubscript{2SG}  \textsc{np}\textsubscript{17}-\textsc{aug}-\textsc{np}\textsubscript{9}-town\\
\glt ‘I will walk with you to town.’ (NF\_Elic15)
\z

Inclusory conjunction involving a full noun rather than a pronoun is illustrated in (\ref{bkm:Ref436905937}), which describes the speaker and his wife; although \textit{bàmùkéntù wángù} ‘my wife’, is expressed as a comitative, the agreement on the verb is plural ‘we’, indicating that both ‘I’ and ‘my wife’ are subjects of the verb.

\ea
\label{bkm:Ref436905937}
\label{bkm:Ref437356554}
kàtúrèrè kúrùwà \textbf{nèbàmùkéntù} \textbf{wángù}\\
\gll ka-tú̲-re\textsubscript{H}re      kú-ru-wa ne=ba-mu-kéntu    u-angú \\
\textsc{pst}.\textsc{ipfv}-\textsc{sm}\textsubscript{1PL}-sleep.\textsc{stat}  \textsc{np}\textsubscript{17}-\textsc{np}\textsubscript{11}-field
\textsc{com}=\textsc{np}\textsubscript{2}-\textsc{np}\textsubscript{1}-woman  \textsc{pp}\textsubscript{1}-\textsc{poss}\textsubscript{1SG}\\
\glt ‘My wife and I were sleeping at the field.’ (ZF\_Narr13)
\z

Inclusory conjunction is also possible when both the conjuncts are full noun phrases. In (\ref{bkm:Ref489369117}), the noun \textit{bàntù} ‘people’ is in the plural, and is supplemented by an inclusory conjunct \textit{nòmùshêrè} ‘and [his] friend’.

\ea
\label{bkm:Ref489369117}
ònkómbwè nèŋwárárà kàbárí bàntù \textbf{nòmùshêrè}\\
\gll o-∅-nkombwe  ne=∅-ŋwarará ka-bá̲-rí    ba-ntu  no=mu-shére \\
\textsc{aug}-\textsc{np}\textsubscript{1a}-tortoise  \textsc{com}=\textsc{np}\textsubscript{5}-crow
\textsc{pst}.\textsc{ipfv}-\textsc{sm}\textsubscript{2}-be  \textsc{np}\textsubscript{2}-person  \textsc{com}=\textsc{np}\textsubscript{1}-friend\\
\glt ‘Tortoise and crow, they were friends.’ (lit.: ‘They were people and [his] friend.’) (NF\_Narr17)
\z

Inclusory conjunction is not obligatory. In (\ref{bkm:Ref436906615}), the subjects of the verb are the speaker and his dog, but the verb shows first person singular agreement, rather than first person plural.

\ea
\label{bkm:Ref436906615}
hàcìtûngù ndàyèndérèrì \textbf{nòmbwá} \textbf{ꜝ}\textbf{wángù} \\
\gll ha-ci-túngu  ndi-a-end-é̲r-er-i      no=∅-mbwá    u-angú\\
\textsc{np}\textsubscript{16}-\textsc{np}\textsubscript{7}-hut  \textsc{sm}\textsubscript{1SG}-\textsc{pst}-go-\textsc{int}-\textsc{npst}.\textsc{pfv}  \textsc{com}=\textsc{np}\textsubscript{1a}-dog  \textsc{pp}\textsubscript{1}-\textsc{poss}\textsubscript{1SG}\\
\glt ‘From the hut, I left with my dog.’ (ZF\_Narr13)
\z

Crucial in determining whether a nominal marked with the comitative marker \textit{nV}= is treated as an inclusory conjunct is the degree of control by the comitative-marked subject over the action. In the examples of inclusory conjunction in (\ref{bkm:Ref436905818}) and (\ref{bkm:Ref437356554}), the subjects expressed by a comitative are human (\textit{ewe} ‘you’, (\ref{bkm:Ref436905818}), and \textit{bàmùkéntù wángù} ‘my wife’, (\ref{bkm:Ref436905937})), and therefore equally in control of the action as the speaker. In the examples without inclusory conjunction, such as (\ref{bkm:Ref436906615}), the speaker (‘I’), as a human, is more in control of the action than the comitative subject \textit{nòmbwá ꜝ}\textit{wángù} ‘my dog’.

The comitative can also be used to express an instrumental, as in (\ref{bkm:Ref99022408}--\ref{bkm:Ref99022411}).

\ea
\label{bkm:Ref99022408}
shìbànàkàsírì \textbf{nòbwátò}\\
\gll shi-ba-na-ka-sír-i      no=bu-ató\\
\textsc{inc}-\textsc{sm}\textsubscript{2}-\textsc{pst}-\textsc{dist}-sail-\textsc{npst}.\textsc{pfv}  \textsc{com}=\textsc{np}\textsubscript{14}-canoe\\
\glt ‘He has sailed with the canoe.’ (NF\_Narr15)
\z

\ea
\label{bkm:Ref99022411}
kùkànkà ndíꜝkánkà ècìkúnì \textbf{nàkàtêmù}\\
\gll ku-kank-a  ndí̲-ká̲nk-a  e-ci-kuní    na=ka-tému\\
\textsc{inf}-cut-\textsc{fv}  \textsc{sm}\textsubscript{1SG}-cut-\textsc{fv}  \textsc{aug}-\textsc{np}\textsubscript{7}-tree  \textsc{com}=\textsc{np}\textsubscript{12}-axe\\
\glt ‘I chop the tree with an axe.’ (NF\_Elic15)
\z

Another strategy Fwe uses to express an instrumental is the verbal causative suffix (see \sectref{bkm:Ref451514620} on the causative), which may combine to express focus on the instrument; see (\ref{bkm:Ref489369936}) in \sectref{bkm:Ref451514620}.

The comitative can also be used to express additive focus, translatable as ‘also’, ‘too’ or ‘as well’, as in (\ref{bkm:Ref99022459}--\ref{bkm:Ref99022462}).

\ea
\label{bkm:Ref99022459}
\textbf{nèmùkêntù} wángù nàshwénì wâwà\\
\gll ne=mu-kéntu  u-angú  na-shwén-i        wáwa\\
\textsc{com}=\textsc{np}\textsubscript{1}-woman  \textsc{pp}\textsubscript{1}-\textsc{poss}\textsubscript{1SG}  \textsc{sm}\textsubscript{1}.\textsc{pst}-be\_tired-\textsc{npst}.\textsc{pfv}  very\\
\glt ‘My wife has also become very tired.’ (ZF\_Elic14)
\z

\ea
\label{bkm:Ref99022462}
\textbf{nèshúnù} hánù ndìshíní múꜝcécì yáꜝpéntékòsítì\\
\gll ne=shunú  hanú    ndi-shi\textsubscript{H}-ní  mú-∅-céci  i-á-pentékosití \\
\textsc{com}=today  \textsc{dem}.\textsc{ii}\textsubscript{16}  \textsc{sm}\textsubscript{1SG}-\textsc{per}-be  \textsc{np}\textsubscript{18}-\textsc{np}\textsubscript{9}-church \textsc{pp}\textsubscript{9}-\textsc{con}=Pentecoste\\
\glt ‘Even today/up to this very day, I am still in the Pentecost church.’ (ZF\_Narr15)
\z

Rather than marking the focused noun with a comitative, additive focus can also be expressed by adding a co-referential personal pronoun marked with the comitative, as in (\ref{bkm:Ref99022509}--\ref{bkm:Ref99022513}).

\ea
\label{bkm:Ref99022509}
\textbf{néyè} mùkéntù ákùbúːkà\\
\gll ne=yé    mu-kéntu  á-o-ku-búːk-a\\
\textsc{com}=\textsc{pers}\textsubscript{3SG}  \textsc{np}\textsubscript{1}-woman  \textsc{con}\textsubscript{1}-\textsc{aug}-\textsc{inf}-wake-I\textsc{tr}-\textsc{fv}\\
\glt ‘The wife also wakes up.’ (NF\_Narr15)
\z

\ea
\label{bkm:Ref99022513}
òmúꜝkwámè \textbf{nêyè} zàkwé zézìzì\\
\gll o-mú-kwamé  ne=yé    zi-akwé  zé-zi-zi\\
\textsc{aug}-\textsc{np}\textsubscript{1}-man  \textsc{com}=\textsc{pers}\textsubscript{3SG}  \textsc{pp}\textsubscript{10}-\textsc{poss}\textsubscript{3SG}  \textsc{cop}.\textsc{def}\textsubscript{8}-\textsc{emph}-\textsc{dem}.\textsc{i}\textsubscript{8}\\
\glt ‘The husband, too, his things are this and that.’ (ZF\_Conv13)
\z

Another function of the comitative is as a marker of direct speech. It is attached to a personal pronoun indicating the speaker of the quotation, as in (\ref{bkm:Ref99022529}--\ref{bkm:Ref99022534}).

\ea
\label{bkm:Ref99022529}
òmúꜝkwámé \textbf{nêyè} shìbànàrâːrì\\
\gll o-mú-kwamé  ne=yé    shi-ba-na-ráːr-i\\
\textsc{aug}-\textsc{np}\textsubscript{1}-man  \textsc{com}=\textsc{pers}\textsubscript{3SG}  \textsc{inc}-\textsc{sm}\textsubscript{2}-\textsc{pst}-sleep-\textsc{npst}.\textsc{pfv}\\
\glt ‘The man said: they are asleep now.’ (NF\_Narr15)
\z

\ea
\textbf{némè} ndùngwè\\
\gll ne=mé  ndu-∅-ngwe\\
\textsc{com}=\textsc{pers}\textsubscript{3SG}  \textsc{cop}-\textsc{np}\-\textsubscript{1a}-leopard\\
\glt ‘I said: it was a leopard.’ (ZF\_Narr14)
\z

\ea
\label{bkm:Ref99022534}
mwáncè \textbf{néyè} máyè máyè màshènè\\
\gll mu-ánce  ne=yé    ∅-máye  ∅-máye  N-ma-shene\\
\textsc{np}\textsubscript{1}-child  \textsc{com}=\textsc{pers}\textsubscript{3SG}  \textsc{np}\-\textsubscript{1a}-mother  \textsc{np}\-\textsubscript{1a}-mother  \textsc{cop}-\textsc{np}\textsubscript{6}-worm\\
\glt ‘The child said: mother, mother, there are worms.’ (NF\_Narr15)
\z

The comitative can be used to coordinate two identical nouns, giving the interpretation ‘every’, as in (\ref{bkm:Ref99022564}--\ref{bkm:Ref99022567}).

\ea
\label{bkm:Ref99022564}
òmùntù nòmùntù\\
\gll o-mu-ntu    no=mu-ntu\\
\textsc{aug}-\textsc{np}\textsubscript{1}-person  \textsc{com}=\textsc{np}\textsubscript{1}-person\\
\glt ‘everyone’ (ZF\_Elic13)
\z

\ea
\textbf{èzyúbà} \textbf{nèzyûbà} káyàngà kùrùwà\\
\gll e-∅-zyúba    ne=∅-zyúba    ka-á̲-y-ang-a    ku-ru-wa\\
\textsc{aug}-\textsc{np}\textsubscript{5}-day  \textsc{com}=\textsc{np}\textsubscript{5}-day  \textsc{pst}.\textsc{ipfv}-\textsc{sm}\textsubscript{1}-go-\textsc{hab}-\textsc{fv}  \textsc{np}\textsubscript{17}-\textsc{np}\textsubscript{11}-field\\
\glt ‘Every day she would go to the field.’ (NF\_Narr15)
\z

\ea
\label{bkm:Ref99022567}
\textbf{ècìntù} \textbf{nècìntù} cìkwèsì òbùrôtù nòbúbbì\\
\gll e-ci-ntu    ne=ci-ntu ci-kwesi  o-bu-rótu     no=bu-bbí\\
\textsc{aug}-\textsc{np}\textsubscript{7}-thing  \textsc{com}=\textsc{np}\textsubscript{7}-thing
\textsc{sm}\textsubscript{7}-have  \textsc{aug}-\textsc{np}\textsubscript{14}-good  \textsc{com}=\textsc{np}\textsubscript{14}-bad\\
\glt ‘Everything has advantages and disadvantages.’ (ZF\_Conv13)
\z
\section{Copulatives}
\label{bkm:Ref489963307}\hypertarget{Toc75352654}{}\label{bkm:Ref450747606}
A copulative prefix is used in non\nobreakdash-verbal sentences to link the subject to a predicate. The copulative prefix has a basic and a definite form. The basic form consists of a homorganic nasal prefix \textit{N-}, which interacts with the noun’s nominal prefix in ways that only partially follow established morphophonological rules in Fwe. The definite form consists of a separate form for each noun class. The full paradigm of copulative prefixes is shown in \tabref{tab:5:2}.

\begin{table}
\label{bkm:Ref463366717}\caption{\label{tab:5:2}Copulative prefixes}

\begin{tabular}{llll}
\lsptoprule
& Nominal prefix & Basic copulative & Definite copulative\\
\midrule
1/2/3 SG &  & ndi- & ndé-\\
1 & mu- & N- & ndó-\\
2 & ba- & N- & mbá-\\
1a & ∅- & ndu- & ndó-\\
3 & mu- & N- & ngó-\\
4 & mi- & N- & njé-\\
5 & ∅- & ndi- & ndé-\\
6 & ma- & N- & ngá-\\
7 & ci & ∅- & cé-\\
8 & zi- & ∅- & zé-\\
9 & N- & nji- & njé-\\
10 & N- & ∅- & zé-\\
11 & ru- & N- & ndó-\\
12 & ka- & ∅- & ká-\\
13 & tu- & ∅- & (n)tó-\\
14 & bu- & N- & mbó-\\
15 & ku- & ∅- & kó-\\
16 & ha- & N- & mpá-\\
17 & ku- & ∅- & kó-\\
18 & mu- & N- & mó-\\
\lspbottomrule
\end{tabular}
\end{table}

When the homorganic nasal of the basic copula is added to a nominal prefix that begins with a nasal consonant, the homorganic nasal is absorbed by the nasal consonant, leading to homophony between the nominal prefix and nominal prefix combined with a copulative. This is the case for the nominal prefixes of class 1 \textit{mu}\nobreakdash-, class 3 \textit{mu}\nobreakdash-, class 4 \textit{mi}\nobreakdash-, class 6 \textit{ma}\nobreakdash-, and class 18 \textit{mu-}. For these classes, a simple noun can be interpreted as either with or without the copulative, as shown in (\ref{bkm:Ref450640706}) with the class 1 noun \textit{mu-ntu} ‘person’, which is ambiguous between ‘a person’ and ‘it is a person’. The only formal distinction between nouns with and without a basic copulative prefix is that nouns with a copula may not take a vocalic augment, whereas nouns without a copula do, as shown in (\ref{bkm:Ref69997533}).

\ea
\label{bkm:Ref450640706}
\glll mùntù\\
mu-ntu\\
\textsc{np}\textsubscript{1}-person\\
\glt ‘a person’
\z

\ea
\glll mùntù\\
N-mu-ntu\\
\textsc{cop}-\textsc{np}\textsubscript{1}-person\\
\glt ‘It is a person.’
\z

\ea
\label{bkm:Ref69997533}
\glll òmùntù\\
o-mu-ntu\\
\textsc{aug}-\textsc{np}\textsubscript{1}-person\\
\glt ‘a person’ (* ‘It’s a person.’)
\z

When the nominal prefix begins with a voiceless stop, the basic copula is zero, i.e. no homorganic nasal is used. This is the case for the prefixes of class 7 \textit{ci}\nobreakdash-, class 12 \textit{ka}\nobreakdash-, class 13 \textit{tu}\nobreakdash-, class 15 \textit{ku}\nobreakdash-, and class 17 \textit{ku-}. The homorganic nasal of the copula is also not realized with the prefix of class 8 \textit{zi}\nobreakdash-, which begins with a voiced fricative. In Namibian Fwe, the nasal prefix can occasionally be heard in these cases. The loss of a nasal before a voiceless stop is not a regular morphophonological rule in Fwe; as discussed in \ref{bkm:Ref451507060}, homorganic nasals that mark noun classes 9/10 are maintained on voiceless stops, and as shown in \tabref{tab:2:1}, prenasalized voiceless stops are regular phonemes in Fwe. Therefore the loss of the homorganic nasal of the copula before voiceless stops is specific to the copulative prefix.

Nominal prefixes with the bilabial fricative /b/, the alveolar tap /r/ or the glottal fricative /h/, change their initial consonant to a stop when combined with the copulative prefix \textit{N-}. This is the case for the prefixes of class 2 \textit{ba}\nobreakdash-, class 11 \textit{ru}\nobreakdash-, class 14 \textit{bu}\nobreakdash-, and class 16 \textit{ha-}, but also for class 5, where the regular prefix is zero, but the allomorph \textit{ri-} is used when combined with the homorganic nasal of the copulative, creating \textit{ndi}\nobreakdash-.

The nominal prefix of class 1a is zero, and the prefixes of class 9 and 10 are a homorganic nasal only. When used with the basic copula, the nominal prefix of class 1a is realized as \textit{ndu}\nobreakdash-, the nominal prefix of class 9 is realized as \textit{nji}\nobreakdash-, and the nominal prefix of class 10 is realized as \textit{zi-}. The forms \textit{nji}\nobreakdash- and \textit{zi}\nobreakdash- for class 9/10 resemble the historical form of the augment, reconstructed as *jɪ\nobreakdash- for class 9 and *ji\nobreakdash- for class 10 \citep[99]{Meeussen1967}. Many Bantu languages have lost or reduced the earlier CV augment, but traces of it can still be seen in certain contexts, such as the copulative (\citealt{Blois1970}). The form of the basic copulative prefixes for class 9 and 10 in Fwe have been created by combining a homorganic nasal with the historical augment of these classes, resulting in the modern \textit{nji}\nobreakdash- and \textit{zi}\nobreakdash- forms.

The copulative form \textit{ndi-} of class 5 shows signs of being extended to other classes. In certain cases, it is used on nouns of class 1, as in (\ref{bkm:Ref492232307}), 1a, as in (\ref{bkm:Ref492232309}), or 9, as in (\ref{bkm:Ref492232313}). This is not an indication that class 9 nouns are reassigned to class 5; as the agreement on the adjective in (\ref{bkm:Ref492236057}) shows, the noun \textit{nako} ‘time’ functions as a class 9 noun, even though it takes the copulative prefix \textit{ndi}\nobreakdash-.

\ea
\label{bkm:Ref492232307}
ênì ndìmwáncù wángú ꜝndírìndîrè\\
\gll éni  ndi-mu-áncu    u-angú  ndí̲-rind-í̲r-e\\
yes  \textsc{cop}-\textsc{np}\textsubscript{1}-younger\_sibling  \textsc{pp}\textsubscript{1}-\textsc{poss}\textsubscript{1SG}  \textsc{sm}\textsubscript{1SG}.\textsc{rel}-wait-\textsc{appl}-\textsc{stat}\\
\glt ‘Yes, I am waiting for my younger brother.’
\z

\ea
\label{bkm:Ref492232309}
zywìn’ ómúꜝkwámè ndìbbâbbà\\
\gll zywiná  o-mú-kwamé  ndi-∅-bbábba\\
\textsc{dem}.\textsc{iv}\textsubscript{1}  \textsc{aug}-\textsc{np}\textsubscript{1}-man  \textsc{cop}-\textsc{np}\textsubscript{1a}-grandfather\\
\glt ‘That man is my grandfather.’ (ZF\_Elic14)
\z

\ea
\label{bkm:Ref492232313}
ndìnyàmà {\textasciitilde} njìnyàmà\\
\gll ndi-N-nyama  {\textasciitilde}  nji-N-nyama\\
\textsc{cop}\textsubscript{5}-\textsc{np}\textsubscript{9}-meat  {\textasciitilde}   \textsc{cop}\textsubscript{9}-\textsc{np}\textsubscript{9}-meat\\
\glt ‘It is meat.’ (ZF\_Elic14)
\z

\ea
\label{bkm:Ref492236057}
ndìnàkw’ éꜝncényà bùryò\\
\gll ndi-N-nakó    e-N-cenyá    bu-ryo\\
\textsc{cop}-\textsc{np}\textsubscript{9}-time  \textsc{aug}-\textsc{np}\textsubscript{9}-small  \textsc{np}\textsubscript{14}-only\\
\glt ‘Just a short time…’ (ZF\_Narr13)
\z

The basic copula \textit{N-} can also be used with nouns or pronouns that are marked with a pronominal prefix, which causes the same phonological changes as the combination of the homorganic nasal with nominal prefixes. With vowel-initial pronominal prefixes, the use of the homorganic nasal causes a velar stop /g/ to surface in the case of class 1, 1a, 3, and 6, resulting in the forms \textit{ngu-} for class 1/1a and 3, and \textit{nga-} for class 6. With the vowel-initial pronominal prefix of class 9, the addition of the homorganic nasal creates an additional /j/, resulting in the form \textit{nji-.}

In addition to the basic copula consisting of a homorganic nasal, Fwe also has a paradigm of definite copulative prefixes. These have a CV shape and are added to the nominal prefix without phonological interaction. This is illustrated with the class 11 noun \textit{ru-tángo} ‘story’, with a basic copula \textit{N-} in (\ref{bkm:Ref505700047}) and a definite copula in (\ref{bkm:Ref438543117}).

\ea
\label{bkm:Ref505700047}
\glll ndùtângò\\
N-ru-tángo\\
\textsc{cop}-\textsc{np}\textsubscript{11}-story\\
\glt ‘It’s a story.’
\z

\ea
\label{bkm:Ref438543117}
\glll ndórùtângò\\
ndó-ru-tángo\\
\textsc{cop}.\textsc{def}\textsubscript{11}-\textsc{np}\textsubscript{11}-story\\
\glt ‘It is the story.’
\z

Historically, the paradigm of definite copulative prefixes is the result of the combination of the copula \textit{N-} with a historical CV form of the augment. The initial consonant of these earlier augments has disappeared in Fwe, but has been maintained in these copulative forms. This is the case, for instance, for the class 3 definite copulative \textit{ngó-}, which results from the combination of the homorganic nasal with the earlier augment *gu\nobreakdash-.

The form of definite copulas has also been influenced by the modern vocalic augment, as seen by the use of mid vowels /e/ and /o/ rather than high vowels /i/ and /u/; these are the result of influence of the modern vocalic augment, which consists of a mid (or low) vowel. The high tone used in definite copulas may also be attributed to the high tone of the (modern) augment (see \sectref{bkm:Ref444175456}).

The influence of the augment on the definite forms may also be the reason for their definite interpretation; there are Bantu languages in which the augment plays a role in expressing definiteness, such as Dzamba \citep{Bokamba1971}. In modern Fwe, the function of the augment is unclear (see \sectref{bkm:Ref444175456}), but unrelated to definiteness, as augmented nouns are frequently found both with definite and indefinite interpretations.

The copula is used to combine a nominal subject with a nominal predicate, by marking the latter with the copulative prefix. The subject can be a noun, such as \textit{bàwáyìsì} ‘the vice (leader)’ in (\ref{bkm:Ref450648492}), followed by the predicate \textit{mbàmùkéntù ꜝ}\textit{wángù} ‘is my wife’. The subject can also be an infinitive verb functioning as a noun, as in (\ref{bkm:Ref506906388}); or a pronoun, such as a demonstrative pronoun in (\ref{bkm:Ref433200018}), or a personal pronoun, as in (\ref{bkm:Ref444602246}).

\ea
\label{bkm:Ref450648492}
bàwáyìsì mbàmùkéntù ꜝwángù\\
\gll ba-wáyisi  N-ba-mu-kéntu    u-angú\\
\textsc{np}\textsubscript{2}-vice  \textsc{cop}-\textsc{np}\textsubscript{2}-\textsc{np}\textsubscript{1}-woman  \textsc{pp}\textsubscript{1}-\textsc{poss}\textsubscript{1SG}\\
\glt ‘The vice leader is my wife.’ (ZF\_Narr15)
\z

\ea
\label{bkm:Ref506906388}
òkùhíbà nkúbbì\\
\gll o-ku-híb-a    N-ku-bbí\\
\textsc{aug}-\textsc{np}\textsubscript{15}-steal-\textsc{fv}  \textsc{cop}-\textsc{np}\textsubscript{15}-bad\\
\glt ‘Stealing is bad.’
\z

\ea
\label{bkm:Ref433200018}
àbá mbàrìmì\\
\gll a-bá    N-ba-rimi\\
\textsc{aug}-\textsc{dem}.\textsc{i}\textsubscript{2}  \textsc{cop}\textsubscript{2}-farmer\\
\glt ‘They are farmers.’ (NF\_Elic15)
\z

\ea
\label{bkm:Ref444602246}
èmé ꜝndónjòvù\\
\gll emé    ndó-∅-njovu\\
\textsc{pers}\textsubscript{1SG}  \textsc{cop}.\textsc{def}\textsubscript{1a}-\textsc{np}\textsubscript{1a}-elephant\\
\glt ‘I am the elephant.’ (NF\_Narr15)
\z

A copulative predicate can also be used without a subject. Compare (\ref{bkm:Ref433202231}), where the copulative predicate \textit{njínswì} ‘is a fish’ is preceded by a subject \textit{mbúfù} ‘a bream’, with (\ref{bkm:Ref433202262}), where the subject is absent.

\ea
\label{bkm:Ref433202231}
mbúfù njínswì\\
\gll N-bufú  nji-N-swí\\
\textsc{np}\textsubscript{9}-bream  \textsc{cop}\textsubscript{9}-\textsc{np}\textsubscript{9}-fish\\
\glt ‘A bream is a fish.’
\z

\ea
\label{bkm:Ref433202262}
\glll \label{bkm:Ref98775427}njínswì\\
nji-N-swí\\
\textsc{cop}\textsubscript{9}-\textsc{np}\textsubscript{9}-fish\\
\glt ‘It’s a fish.’ (ZF\_Elic14)
\z

When a copulative construction lacks an overt, nominal subject, the intended subject is often inferable from the discourse, as in (\ref{bkm:Ref98512641}). The intended subject of \textit{njìnênè} ‘(it) is big’ is the speaker’s house, a topic which has been brought into the discussion by the previous speaker.

\ea
\label{bkm:Ref98512641}
\ea
ènjúò yákò njìnénè kàpá ndíꜝncényà\\
\gll e-N-júo    i-akó    nji-N-néne  kapá  ndí-N-cenyá\\
\textsc{aug}-\textsc{np}\textsubscript{9}-house  \textsc{pp}\textsubscript{9}-\textsc{poss}\textsubscript{2SG}  \textsc{cop}\textsubscript{9}-\textsc{np}\textsubscript{9}-big  or  \textsc{cop}\textsubscript{5}-\textsc{np}\textsubscript{9}-small\\
\glt ‘Is your house big or small?’

\ex
\glll njìnênè\\
nji-N-néne\\
\textsc{cop}\textsubscript{9}-\textsc{np}\textsubscript{9}-big\\
\glt ‘It [=my house] is big.’ (ZF\_Elic13)
\z\z

The predicate consists of the copulative prefix followed by a noun, as in (\ref{bkm:Ref98775427}), or an infinitive verb used as a noun, as in (\ref{bkm:Ref494182359}), or an adjective, in which case the copulative agrees in noun class with the subject, as in (\ref{bkm:Ref444602411}). Other nominal elements that may be marked by a copulative prefix are demonstratives, as in (\ref{bkm:Ref433201568}), possessives, as in (\ref{bkm:Ref433201646}), or personal pronouns, as in (\ref{bkm:Ref433201795}).

\ea
\label{bkm:Ref494182359}
òmùsèbèzí ꜝwángù nkùùrìsà\\
\gll o-mu-sebezí    u-angú  N-ku-urisa\\
\textsc{aug}-\textsc{np}\textsubscript{3}-work  \textsc{pp}\textsubscript{3}-\textsc{poss}\textsubscript{1SG}  \textsc{cop}-\textsc{np}\textsubscript{15}-sell\\
\glt ‘My job is selling.’ (NF\_Elic15)
\z

\ea
\label{bkm:Ref444602411}
èyî nswî njì-nênè\\
\gll e-í    e-N-swí  nji-N-néne\\
\textsc{aug}-\textsc{dem}.\textsc{i}\textsubscript{9}   \textsc{aug}-\textsc{np}\textsubscript{9}-fish  \textsc{cop}\textsubscript{9}-\textsc{np}\textsubscript{9}-big\\
\glt ‘This fish is big.’ (ZF\_Elic14)
\z

\ea
\label{bkm:Ref433201568}
òbùkáꜝbábù mbóꜝbúbù mbóꜝbúbù\\
\gll o-bu-kábabú    mbó-bu-bú      mbó-bu-bú\\
\textsc{aug}-\textsc{np}\textsubscript{14}-problem  \textsc{cop}.\textsc{def}\textsubscript{14}-\textsc{emph}-\textsc{dem}.\textsc{i}\textsubscript{14} \textsc{cop}.\textsc{def}\textsubscript{14}-\textsc{emph}-\textsc{dem}.\textsc{i}\textsubscript{14}\\
\glt ‘The problem is such and such.’ (ZF\_Conv13)
\z

\ea
\label{bkm:Ref433201646}
àbá ꜝbámbwà mbángù\\
\gll a-bá    ba-mbwá  N-ba-angú\\
\textsc{aug}-\textsc{dem}.\textsc{i}\textsubscript{2}  \textsc{np}\textsubscript{2}-dog  \textsc{cop}-\textsc{pp}\textsubscript{2}-\textsc{poss}\textsubscript{1SG}\\
\glt ‘These dogs are mine.’ (ZF\_Elic14)
\z

\ea
\label{bkm:Ref433201795}
ndínwè éè ndímè\\
\gll ndi-nwé  ée  ndi-mé\\
\textsc{cop}-\textsc{pers}\textsubscript{2PL}  yes  \textsc{cop}-\textsc{pers}\textsubscript{1SG}\\
\glt ‘Are you the one?’ ‘Yes, I’m the one!’ (ZF\_Narr13)
\z

Phrase-final tonal processes affect both the subject and the predicate of the copulative construction. This is illustrated in (\ref{bkm:Ref450649453}), where the tonal process of high tone retraction, which only affects the last syllable of a phrase-final word, affects both the head \textit{mbufu} ‘bream’, and the predicate \textit{njinswi} ‘is a fish’.

\ea
\label{bkm:Ref450649453}
mbúfù njínswì\\
\gll N-bufú  nji-N-swí\\
\textsc{np}\textsubscript{9}-bream  \textsc{cop}\textsubscript{9}-\textsc{np}\textsubscript{9}-fish\\
\glt ‘A bream is a fish.’ (ZF\_Elic14)
\z

To negate a copulative construction, the auxiliary verb \textit{ri} ‘be’ is required in addition to the copulative prefix. This construction is discussed in Chapter \ref{bkm:Ref98752806} on negation.

\section{Appositives}
\label{bkm:Ref492039371}\hypertarget{Toc75352655}{}
This section discusses apposition, a construction combining a first or second person referent with a co-referential, full noun through the use of an appositive prefix. \tabref{tab:5:3} gives an overview of appositive prefixes.

\begin{table}
\label{bkm:Ref492039311}\caption{\label{tab:5:3}Appositive prefixes}
\begin{tabular}{lll}
\lsptoprule
& Singular & Plural\\
\midrule
First person & {\itshape nde-} & {\itshape tu-}\\
Second person & {\itshape we-} & {\itshape mu-}\\
\lspbottomrule
\end{tabular}
\end{table}

Appositive prefixes are used on a noun, to mark the identity between the referent of the noun and the intended person, as in (\ref{bkm:Ref98835584}--\ref{bkm:Ref98835590}).

\ea
\label{bkm:Ref98835584}
èmé ndènyòkò\\
\gll emé    nde-∅-nyoko\\
\textsc{pers}\textsubscript{1PL\-}  \textsc{app}\textsubscript{1SG}-\textsc{np}\textsubscript{1a}-your.mother\\
\glt ‘I, your mother…’
\z

\ea
èwé wèmwáꜝnángù\\
\gll ewé    we-mu-án-angú\\
\textsc{pers}\textsubscript{2SG}  \textsc{app}\textsubscript{2SG}-\textsc{np}\textsubscript{1}-child-\textsc{poss}\textsubscript{1SG}\\
\glt ‘You, my child…’
\z

\ea
èswé tùbàntù\\
\gll eswé    tu-ba-ntu\\
\textsc{pers}\textsubscript{1PL}  \textsc{app}\textsubscript{1PL}-\textsc{np}\textsubscript{2}-person\\
\glt ‘Us, people…’
\z

\ea
\label{bkm:Ref98835590}
ènwé mùbáꜝnángù\\
\gll enwé    mu-ba-án-angú\\
\textsc{pers}\textsubscript{2PL}  \textsc{app}\textsubscript{2PL}-\textsc{np}\textsubscript{2}-child-\textsc{poss}\textsubscript{1SG}\\
\glt ‘You, my children…’ (NF\_Elic17)
\z

Appositive prefixes may be combined with a co-referential personal pronoun, as in (\ref{bkm:Ref98835584}--\ref{bkm:Ref98835590}), or without a personal pronoun, as in (\ref{bkm:Ref71202998}--\ref{bkm:Ref71202999}).

\ea
\label{bkm:Ref71202998}
\textbf{tùbakêntù} kàtùnákùtíyàngà cáhà\\
\gll tu-ba-kéntu    ka-tu-náku-tí-ang-a  cahá\\
\textsc{app}\textsubscript{1PL}-\textsc{np}\textsubscript{2}-woman  \textsc{neg}-\textsc{sm}\textsubscript{1PL}-\textsc{hab}-\textsc{fv}    very\\
\glt ‘\textbf{Us} \textbf{women}, we did not used to be afraid often.’ (NF\_Narr17)
\z

\ea
\label{bkm:Ref71202999}
néwè òshùmékò \textbf{wènkômbwè}\\
\gll né=we    o-shu\textsubscript{H}m-e=kó̲    we-∅-nkómbwe\\
\textsc{com}=\textsc{pers}\textsubscript{2SG}  \textsc{sm}\textsubscript{2SG}-bite-\textsc{pfv}.\textsc{sbjv}=\textsc{loc}\textsubscript{17}  \textsc{app}\textsubscript{2SG}-\textsc{np}\textsubscript{1a}-tortoise\\
\glt ‘And you must also bite, \textbf{you} \textbf{tortoise}.’ (NF\_Narr17)
\z

The appositive prefixes are also used on the stem \textit{íni} ‘self’, used as an emphatic reflexive; see (\ref{bkm:Ref492039232}--\ref{bkm:Ref492039233}) in \sectref{bkm:Ref451256199} on the reflexive.

\section{Adverbs}
\label{bkm:Ref489870394}\hypertarget{Toc75352656}{}
Adverbs in Fwe can be simplex or derived from other parts of speech with a derivational prefix \textit{ka-}, \textit{bú-} or \textit{mbó-}. Adverbs can modify a verb, an adjective or another adverb, as illustrated with the adverb \textit{wâwà} ‘very’ in (\ref{bkm:Ref98512678}--\ref{bkm:Ref498357438}).

\ea
\label{bkm:Ref98512678}
èzí zìshámù zìgórétè wâwà\\
\gll e-zí    zi-shamú  zi-gor-é̲te      wáwa\\
\textsc{aug}-\textsc{dem}.\textsc{i}\textsubscript{8}  \textsc{np}\textsubscript{8}-tree  \textsc{sm}\textsubscript{8}-become\_strong-\textsc{stat}  very\\
\glt ‘These trees are very strong.’
\z

\ea
èyí njûò njìndótù wâwà\\
\gll e-í    N-júo    nji-N-dótu    wáwa\\
\textsc{aug}-\textsc{dem}.\textsc{i}\textsubscript{9}  \textsc{np}\textsubscript{9}-house  \textsc{cop}\textsubscript{9}-\textsc{np}\textsubscript{9}-nice  very\\
\glt ‘This house is very nice.’ (ZF\_Elic14)
\z

\ea
\label{bkm:Ref498357438}
kàréː wâwà ndìnàmánì\\
\gll ka-réː    wáwa  ndi-na-man-í̲\\
\textsc{adv}-long  very  \textsc{sm}\textsubscript{1SG}-\textsc{pst}-finish-\textsc{npst}.\textsc{pfv}\\
\glt ‘I finished very long ago.’ (ZF\_Elic13)
\z

Fwe has a small, closed set of words that typically function as adverbs, listed in (\ref{bkm:Ref71204521}).

\ea
\label{bkm:Ref71204521}
shùnù   \tab     ‘today’\\
zyônà        \tab ‘yesterday/tomorrow’\footnote{The interpretation of this adverb as either yesterday or tomorrow is dependent on the tense of the verb.}\\
câhà (Namibian Fwe)  \tab  ‘very’\\
wâwà (Zambian Fwe)  \tab‘very’ \\
cwárè    \tab    ‘then’\\
hápè\tab        ‘again’\\
nênjà   \tab     ‘well’\\
nàngá       \tab ‘even’\\
témà (Namibian Fwe) \tab  ‘maybe’\\
mwèndí (Zambian Fwe) \tab‘maybe’
\z

The adverb \textit{câhà} and its Zambian Fwe counterpart \textit{wâwà} function as adverbs expressing general intensity, translatable as ‘very’, but can receive various more specific interpretations based on context, as in (\ref{bkm:Ref99025388}--\ref{bkm:Ref99025390}).

\ea
\label{bkm:Ref99025388}
àbùtùká \textbf{câhà}\\
\gll a-bu\textsubscript{H}tuk-á̲  cáha\\
\textsc{sm}\textsubscript{1}-run-\textsc{fv}  very\\
\glt ‘S/he runs \textbf{fast}.’
\z

\ea
\glll àkóːrà \textbf{câhà}\\
a-kó̲ːr-a    cáha\\
\textsc{sm}\textsubscript{1}-cough-\textsc{fv}  very\\
\glt ‘S/he coughs \textbf{loudly}.’ (NF\_Elic15)
\z

\ea
àbèná bàkéntù bàámbà \textbf{wâwà}\\
\gll a-bená  ba-kéntu  ba-á̲mb-a  wáwa\\
\textsc{aug}-\textsc{dem}.\textsc{iv}\textsubscript{2}  \textsc{np}\textsubscript{2}-woman  \textsc{sm}\textsubscript{2}-talk-\textsc{fv}  very\\
\glt ‘Those women talk \textbf{a} \textbf{lot}.’
\z

\ea
kòkwí  ꜝ\textbf{wáwà} nòmùbônì\\
\gll kokwí  wáwa  no-mu-bón-i\\
where  very  \textsc{sm}\textsubscript{2SG}.\textsc{pst}-\textsc{om}\textsubscript{1}-see-\textsc{npst}.\textsc{pfv}\\
\glt ‘Where \textbf{exactly} did you see it?’
\z

\ea
\label{bkm:Ref99025390}
cìcíná cìrìmò ndìnàshînjì \textbf{wâwà}\\
\gll ci-ciná    ci-rimo  ndi-na-shínj-i      wáwa\\
\textsc{emph}\textsubscript{7}-\textsc{dem}.\textsc{iv}\textsubscript{7}  \textsc{np}\textsubscript{7}-year  \textsc{sm}\textsubscript{1SG}-\textsc{pst}-harvest-\textsc{npst}.\textsc{pfv}  very\\
\glt ‘This year I had a \textbf{good} harvest.’ (ZF\_Elic14)
\z

The prefix \textit{ka-} derives an adverb from other words. Although this prefix resembles the class 12 nominal prefix \textit{ka-} (see \sectref{bkm:Ref489005545} on nominal prefixes), this homophony is likely accidental: whereas the class 12 nominal prefix \textit{ka-} replaces the noun’s original nominal prefix (see the examples in (\ref{bkm:Ref498357099}) in \sectref{bkm:Ref498357105}), the use of the adverb-deriving prefix \textit{ka-} causes the noun’s original nominal prefix and augment to be maintained, as in (\ref{bkm:Ref71184200}--\ref{bkm:Ref71184201}).

\ea
\label{bkm:Ref71184200}
njékàndé ꜝryángù \textbf{kóbùfwîhì}\\
\gll njé-kandé    rí-angú  ká-o-bu-fwíi\\
\textsc{cop}.\textsc{def}\textsubscript{9}-story  \textsc{pp}\textsubscript{5}-\textsc{poss}\textsubscript{1SG}  \textsc{adv}-\textsc{aug}-\textsc{np}\textsubscript{14}-short\\
\glt ‘This is my story, \textbf{in} \textbf{short}.’ (NF\_Narr17)
\z

\ea
\label{bkm:Ref71184201}
\textbf{kómùtàrà} kwìná àbákwàmé sò mwànàmìbìà\\
\gll ká-o-mu-tara    ku-iná    a-bá-kwamé    so mwa-Namibia \\
\textsc{adv}-\textsc{aug}-\textsc{np}\textsubscript{3}-example  \textsc{sm}\textsubscript{17}-be\_at  \textsc{aug}-\textsc{np}\textsubscript{2}-man  thus
\textsc{np}\textsubscript{18}-Namibia\\
\glt ‘\textbf{For} \textbf{example}, there is a man like that in Namibia.’ (ZF\_Conv13)
\z

The adverbial prefix \textit{ka-} can be used to derive adverbs from nouns, as in (\ref{bkm:Ref71184200}--\ref{bkm:Ref71184201}), or from adjectives (\ref{bkm:Ref498357503}), infinitive verbs (\ref{bkm:Ref498417129}), or numerals (\ref{bkm:Ref498417486}).

\ea
\label{bkm:Ref498357503}
ndìfwìrè \textbf{kànínì} ènjàrà\\
\gll ndi-fw\textsubscript{H}-ire    ka-níni  e-N-jara\\
\textsc{sm}\textsubscript{1SG}-die-\textsc{stat}  \textsc{adv}-small  \textsc{aug}-\textsc{np}\textsubscript{9}-hunger\\
\glt ‘I’m \textbf{a} \textbf{bit} hungry.’ (NF\_Elic15)
\z

\ea
\label{bkm:Ref498417129}
àkóːrà \textbf{kòkùóngòzà}\\
\gll a-kó̲ːr-a    ka-o-ku-óngoz-a\\
\textsc{sm}\textsubscript{1}-cough-\textsc{fv}  \textsc{adv}-\textsc{aug}-\textsc{inf}-shout-\textsc{fv}\\
\glt ‘S/he coughs \textbf{loudly}.’ (NF\_Elic15)
\z

\ea
\label{bkm:Ref498417486}
náàkóːrà \textbf{kòbírè}\\
\gll ná̲-a-a-kóːr-a    ka-o=biré\\
\textsc{rem}-\textsc{sm}\textsubscript{1}-\textsc{pst}-cough-\textsc{fv}  \textsc{adv}-\textsc{con}=two\\
\glt ‘He coughed \textbf{twice}.’ (ZF\_Elic14)
\z

The adverbial prefix \textit{ka-} can be used to derive adverbs of manner, as in (\ref{bkm:Ref498357503}--\ref{bkm:Ref498417486}), but also temporal adverbs, as in (\ref{bkm:Ref98512719}--\ref{bkm:Ref98512721}).

\ea
\label{bkm:Ref98512719}
zyônà nàndínàbúːkà \textbf{kàfôrù}\\
\gll zyóna    na-ndí̲-na-búːk-a      ka-fóru\\
tomorrow  \textsc{rem}-\textsc{sm}\textsubscript{1SG}-\textsc{rem}.\textsc{fut}-wake-\textsc{fv}  \textsc{adv}-four\\
\glt ‘Tomorrow I will wake up \textbf{at} \textbf{four}.’ (ZF\_Elic13)
\z

\ea
\label{bkm:Ref98512721}
émè nándàréːtìwà \textbf{káꜝ}\textbf{náìntínsíkìsìtì}\\
\gll emé    ná̲-ndi-a-réːt-iw-a      ká-náintinsíkisiti\\
\textsc{pers}\textsubscript{1SG}  \textsc{rem}-\textsc{sm}\textsubscript{1SG}-\textsc{pst}-bear-\textsc{pass}-\textsc{fv}  \textsc{adv}-1960\\
\glt ‘Me, I was born \textbf{in} \textbf{1960}.’ (ZF\_Narr15)
\z

The prefix \textit{bú-} derives manner adverbs. This prefix is similar to the nominal prefix of class 14 \textit{bu-}, but the adverbial prefix has a high tone whereas the nominal prefix is toneless. The adverbial prefix \textit{bú-} is productive, and can be used with adjectival roots, as in (\ref{bkm:Ref451866497}), and with verbs, as in (\ref{bkm:Ref451866458}).

\ea
\label{bkm:Ref451866497}
àsèbèzá ꜝbúcènyà búcènyà\\
\gll a-sebez-á̲  bú-cenya  bú-cenya\\
\textsc{sm}\textsubscript{1}-work-\textsc{fv}  \textsc{adv}-small  \textsc{adv}-small\\
\glt ‘S/he works slowly.’ (NF\_Elic15)
\z

\ea
\label{bkm:Ref451866458}
\label{bkm:Ref489957276}náàráːrà búꜝtútúmà\\
\gll ná̲-a-a-ráːr-a      bú-tutum-á\\
\textsc{rem}-\textsc{sm}\textsubscript{1}-\textsc{pst}-sleep-\textsc{fv}  \textsc{adv}-shiver-\textsc{fv}\\
\glt ‘She slept shivering.’ (NF\_Narr15)
\z

Adverbs derived from verbs maintain certain verbal characteristics: melodic tone (for instance, the final high tone on /bú-tutum-á/ in (\ref{bkm:Ref489957276})), and certain verbal affixes, such as the reflexive \textit{rí-} and the stative suffix, as in (\ref{bkm:Ref451866568}). Adverbs derived from verbs can even take their own object, as in (\ref{bkm:Ref451866585}).

\ea
\label{bkm:Ref451866568}
tùkèrè búrìgùmbênè\\
\gll tu-ke\textsubscript{H}re    bú-ri\textsubscript{H}-gumbé̲ne\\
\textsc{sm}\textsubscript{1PL}-sit.\textsc{stat}  \textsc{adv}-\textsc{refl}-sit\_close\_to.\textsc{stat}\\
\glt ‘We sit next to each other.’
\z

\ea
\label{bkm:Ref451866585}
ndìkèrè búrìyàngítè màkárà\\
\gll ndi-ke\textsubscript{H}re    bú-ri\textsubscript{H}-ang-í̲te    ma-kará\\
\textsc{sm}\textsubscript{1SG}-sit.\textsc{stat}  \textsc{adv}-\textsc{refl}-cross-\textsc{stat}  \textsc{np}\textsubscript{6}-leg\\
\glt ‘I sit cross-legged.’ (NF\_Elic15)
\z

There are also three underived adverbs that have the nominal prefix of class 14 \textit{bu-}: \textit{bu-tí} ‘how, so/like this’, as in (\ref{bkm:Ref99025443}) \textit{bu-ryó} ‘only, just’, as in (\ref{bkm:Ref99025459}), and \textit{bu-ryahó} ‘like that’, as in (\ref{bkm:Ref99025470}).

\ea
\label{bkm:Ref99025443}
mbòndítêndè bútì kántì\\
\gll mbo-ndí̲-té̲nd-e      bu-tí      kantí\\
\textsc{near}.\textsc{fut}-\textsc{sm}\textsubscript{1SG}-do-\textsc{pfv}.\textsc{sbjv}  \textsc{np}\textsubscript{14}-like\_this  then\\
\glt ‘I will do like this then.’ (NF\_Narr15)
\z

\ea
\label{bkm:Ref99025459}
ndìyéndè bùryó ꜝkúmùnzì\\
\gll ndi-é̲nd-e    bu-ryó  kú-mu-nzi\\
\textsc{sm}\textsubscript{1SG}-go-\textsc{pfv}.\textsc{sbjv}  \textsc{np}\textsubscript{14}-just  \textsc{np}\textsubscript{17}-\textsc{np}\textsubscript{3}-village\\
\glt ‘Let me just go home.’ (ZF\_Narr14)
\z

\ea
\label{bkm:Ref99025470}
àhà bárèrè bùryáhò\\
\gll a-ha    bá̲-re\textsubscript{H}re    bu-ryahó\\
\textsc{aug}-\textsc{dem}.\textsc{i}\textsubscript{16}  \textsc{sm}\textsubscript{2}.\textsc{rel}-sleep.\textsc{stat}  \textsc{np}\textsubscript{14}-like\_that\\
\glt ‘When they were sleeping like that…’ (NF\_Narr17)
\z

The prefix \textit{bu-} in these adverbs is not the same as the productive adverbializer prefix \textit{bú-}: it lacks a high tone, and functions as a nominal prefix, as seen from the fact that it may take a copulative prefix, either the homorganic nasal, as in (\ref{bkm:Ref490751209}), or the definite copulative prefix \textit{mbó-} of class 14, as in (\ref{bkm:Ref490751210}) (see also \sectref{bkm:Ref489963307} on copulatives).

\ea
\label{bkm:Ref490751209}
mbùryó ꜝndíꜝzánà\\
\gll N-bu-ryó    ndí̲-zá̲n-a\\
\textsc{cop}-\textsc{np}\textsubscript{14}-only  \textsc{sm}\textsubscript{1SG}.\textsc{rel}-joke-\textsc{fv}\\
\glt ‘I am only joking.’ (NF\_Elic15)
\z

\ea
\label{bkm:Ref490751210}
mbóbùryàhó ꜝtúꜝkéːzyà\\
\gll mbó-bu-ryahó    tú̲-ké̲ːzy-a\\
\textsc{cop}.\textsc{def}\textsubscript{14}-\textsc{np}\textsubscript{14}-like\_that  \textsc{sm}\textsubscript{1PL}.\textsc{rel}-come-\textsc{fv}\\
\glt ‘It is like that that we are coming.’ (NF\_Elic17)
\z

Adverbs can also be derived with the prefix \textit{mbó-}, to express a comparison, translatable as ‘like’, as in (\ref{bkm:Ref99025559}--\ref{bkm:Ref99025561}).

\ea
\label{bkm:Ref99025559}
àrírà mbómùcècè\\
\gll a-rir-á̲  mbó-mu-cece\\
\textsc{sm}\textsubscript{1}-cry-\textsc{fv}  \textsc{adv}-\textsc{np}\textsubscript{1}-baby\\
\glt ‘She cries like a baby.’ (NF\_Elic15)
\z

\ea
\label{bkm:Ref99025561}
èzí zìkúnì zìfwánà mbómùshòbò wònkéː\\
\gll e-zí    zi-kúni  zi-fwá̲n-a mbó-mu-shobo  u-o=nkéː \\
\textsc{aug}-\textsc{dem}.\textsc{i}\textsubscript{8}  \textsc{np}\textsubscript{8}-tree  \textsc{sm}\textsubscript{8}-resemble-\textsc{fv}
\textsc{adv}-\textsc{np}\textsubscript{3}-type  \textsc{pp}\textsubscript{3}-\textsc{con}=one\\
\glt ‘These trees look like the same type.’ (ZF\_Elic14)
\z

