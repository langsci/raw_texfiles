

\chapter{Introduction}
\hypertarget{Toc75352593}{}
This book describes the grammar of Fwe, a Bantu language spoken in Zambia and Namibia. In this chapter, background will be given about the language, its classification (\sectref{bkm:Ref450562312}) and its sociolinguistic situation (\sectref{bkm:Ref450562320}), an estimate of the language’s vitality (\sectref{bkm:Ref450562376}), and a brief overview of regional variation in Fwe (\sectref{bkm:Ref450562330}). \sectref{bkm:Ref492223019} discusses the small body of earlier research that mentions Fwe, and \sectref{bkm:Ref70697316} discusses the purpose of the current study and the data on which it is based.

The Fwe language is called \textit{cìfwè} by its speakers; the initial syllable \textit{ci-} is a prefix of noun class 7 indicating a language. As is common when referring to Bantu language names in English, the nominal prefix is omitted and the language is referred to as Fwe in this work. Another name that many speakers, as well as speakers of surrounding languages, use for the language is \textit{sifwe}, where \textit{si-} is the class 7 prefix in the regional lingua franca Lozi.

\section{Classification}
\label{bkm:Ref450562312}\hypertarget{Toc75352594}{}
Fwe belongs to the Bantu language family, which is part of the Niger-Congo phylum, Africa’s largest language family. Although Bantu languages clearly form a genealogical unit, its subclassification is notoriously difficult because of extensive horizontal contact between Bantu languages. An influential attempt at subgrouping Bantu languages, not as genealogical subgroups but mainly for referential purposes, was made by {\citet{Guthrie1948}}, though this work did not include Fwe. In the most recent referential classification of Bantu languages, by {\citet{Hammarström2019}}, Fwe is classified as K402, sharing the K40 group with Ikuhane (Subiya) and Zambian and Namibian Totela.

Genealogical classification has placed Fwe in a subgroup called Bantu Botatwe (\citealt{Bostoen2009,Luna2010}). Bantu Botatwe consists of an eastern branch, made up of Toka, Leya, Ila, Tonga, Sala, Lenje, Lundwe and Soli, and a western branch, made up of Shanjo, Fwe, Mbalangwe, Subiya and Totela (\citealt{Luna2010}: 69).\footnote{According to \citet[54-55]{Crane2011}, only Namibian Totela is part of the western branch, and Zambian Totela should be grouped with the eastern branch.} Within western Bantu Botatwe, Fwe is most closely related to Shanjo. {\citet{Seidel2005}} also found a slight similarity between Fwe and Yeyi, al\-though he contends, together with many others (\citealt{Andersson1997,Elderkin1998,Sommer1995}), that Yeyi is an isolate within Bantu, and that its closest genealogical relative, if any, is yet to be determined.

\section{Sociolinguistic profile}
\label{bkm:Ref450562320}\hypertarget{Toc75352595}{}
Fwe is spoken on both sides of the Zambian-Namibian border. In Zambia, the Fwe-speaking area is concentrated in the southwestern tip of the Western Prov\-ince, in the Imusho and Sinjembela areas, and parts of the Mutomena area. The western boundary of the Fwe-speaking area is the Kwando river, which is also the national border between Zambia and Angola. In Namibia, Fwe is spoken in the area formerly known as the Caprivi strip, which was officially renamed “Zambezi region” in 2013. Fwe is mainly spoken in the area surrounding the village of Kongola, stretching north to Singalamwe and into Zambia, east up to Sibbinda, and south to Lizauli. For a detailed overview of the areal distribution of the languages in the Zambezi region, see {\citet{Seidel2005}}. The maps in \figref{bkm:Ref98513207}\footnote{I am grateful to Jan Gunnink from TNO Geomodelling for designing these maps.} give an approximation of the area in which Fwe is spoken.

\begin{figure} 
\includegraphics[width=.95\textwidth]{figures/fwe-img001.jpg}
\includegraphics[width=.95\textwidth]{figures/fwe-img002.jpg}
\caption{\label{bkm:Ref98513207}The distribution of Fwe}
\end{figure}

The area where Fwe is spoken is an area of high linguistic diversity. The Zambian Fwe-speaking area is bordered by a Kwamashi-speaking area in the north, and a Shanjo-speaking area in the north-east. In Namibia, Fwe speakers are surrounded by Yeyi speakers in the south and Totela speakers in the east. To the west lies the sparsely inhabited Caprivi Game Park. In both Zambia and Namibia, Fwe-speaking villages are interspersed with Mbukushu-speaking villages, though Fwe speakers form a clear majority; Mbukushu is a Bantu language that is not closely related to Fwe, but instead to Kwamashi, and to Manyo and Kwangali spoken further to the west in Namibia \citep{Möhlig1997}. Larger numbers of Mbukushu speakers are found further east in Namibia and further south in Botswana. Small pockets of Khwe-speakers are also found living close to the Fwe-speaking area (\citealt{Brenzinger1998,JonesDieckmann2014}); Khwe is a Khoisan language of the Khoe family, formerly called Central Khoisan (see \citealt{Güldemann2014} for an overview of Khoisan linguistic classification).

In all of the Zambezi region and most of the Western province of Zambia, Lozi is the most important contact language. Lozi is recognized as one of Zambia’s seven national languages, and is among the country’s largest languages, in terms of both first and second language speakers (\citealt{MartenKula2008}). Lozi is a Bantu language that came into being when speakers of Kololo, a southern Sotho variety, fled South Africa in the nineteenth century and settled in western Zambia, where they came into contact with the local elite speaking Luyi, a Zambian Bantu language. The resulting Lozi language maintains a mostly Sotho grammar and lexicon, but with a clear Luyi phonology \citep{Gowlett1989}. Because of its South African origin, Lozi is not mutually intelligible with any of the Bantu languages of the Western Province or the Zambezi region \citep{Seidel2005}. Lozi plays an important role as language of wider communication, especially in Zambia, and virtually all Fwe speakers speak it fluently as a second language. In the Zambezi region in Namibia, English is also widely used as a language of wider communication, and among older generations, Afrikaans. In addition to these languages of wider communication, many Zambian Fwe speakers also speak Mbukushu as a second language, especially those who live in mixed Fwe-Mbukushu villages. In Namibia, Yeyi, Totela and Subiya are common as second languages among Fwe speakers, especially for people in mixed marriages and their offspring. In general, multilingualism among Fwe speakers appears to be extremely common, and I interviewed several speakers who spoke up to eight different (Bantu) languages.

The number of native Fwe speakers is difficult to determine. National census data are too broad-meshed: the Population and Housing Census of Namibia from 2011 counts 22,484 households whose main languages were “Caprivian languages”. Ethnologue mentions 13,700 Fwe speakers in Namibia \citep{EberhardEtAl2021}. A preliminary report compiled as a preparation for a Bible translation project mentions an estimate of 12,000 to 14,000 Fwe speakers in Zambia, and a total of more than 20,000 \citep{SakuhukaEtAl2011}. Estimates of second language speakers of Fwe are even more difficult, though I observed during my fieldwork numerous cases where adults moving to the Fwe-speaking areas for work or family reasons learned Fwe as a second language. Second language acquisition of Fwe is also motivated by intermarriage.

Speakers of Fwe call themselves \textit{màfwè}, where \textit{ma-} is a prefix of noun class 6, indicating an ethnic group. In Namibia, the connection between the ethnic designation Mafwe and the use of the language Fwe is very complex. The German colonial administration, which had little active interest in the Caprivi strip, subsumed all but the Subiya under the label “Mafwe”: Totela, Mbukushu, Mbalangwe, Yeyi, and speakers of Khoisan languages, presumably Khwe. The use of Mafwe as an ethnic label covering a linguistically diverse group has since been accepted, and was taken over when the South African government took control of Namibia (then South-West Africa). This broad, non-linguistic use of the term “Mafwe” persisted after independence, and in Namibia the term “Mafwe” usually designates those inhabitants of the Zambezi region living between the town of Katima Mulilo up to the western boundary of the Zambezi region, and therefore includes speakers of Fwe as well as Yeyi, Totela, Mbukushu and Khwe. For a detailed history of the Zambezi region, see {\citet{Kangumu2011}}.

\section{Language vitality}
\label{bkm:Ref450562376}\hypertarget{Toc75352596}{}
Some linguists estimate that within the next hundred years, half of the world’s languages will disappear (\citealt{AustinSallabank2011}). Although speaker numbers are not a failsafe predictor of language endangerment, it is clear that languages with smaller numbers of speakers are more likely to become endangered. The number of Fwe speakers is small, and the Fwe speech community is further hindered by the national border that cuts across it. In neither Zambia nor Namibia does Fwe have any institutional support or recognition. In Zambia, Fwe is under pressure from Lozi, one of the national languages of Zambia that is used in education and other formal domains. In Namibia, Fwe is also under pressure from Lozi, as well as from Subiya, which at approximately 30,000 speakers (Ethnologue) is a larger language than Fwe. Many Fwe speakers have at least a passive knowledge of Subiya, whereas few Subiya speakers speak or even understand Fwe. Both Fwe and Subiya speakers contend that Fwe is a “more difficult” language than Subiya.

All these factors indicate that the vitality of Fwe is threatened, both in Zambia and Namibia. However, data on its actual usage contradict this. Children in Fwe-speaking areas typically begin life with Fwe as their first and only language, and only start learning Lozi when they enter school. This also appears to be the case with children of Fwe-speaking parents who grow up in urban areas, where Fwe is not the dominant language. Migrants moving to Fwe-speaking areas mostly learn Fwe as a second language. Fwe speakers use their language online, on Facebook and WhatsApp, and in text messages. There is popular music in Fwe, and in Zambia, a Bible translation in Fwe is being prepared. The findings of {\citet{SakuhukaEtAl2011}}, who surveyed Fwe in Zambia, also underscore the stable use of Fwe across all social domains, with the exception of formal education, where both Fwe and Lozi are used, and church settings, where Lozi is preferred.

Speakers tend to have a positive attitude towards Fwe, and speaking Fwe is often considered an important part of one’s identity. Illustrative in this regard is an affair in 2008 where Fwe-speaking chiefs fined Yeyi-speaking chiefs for speaking Yeyi. They reasoned that Yeyi speakers are part of the Mafwe ethnic group, and as such should speak Fwe rather than Yeyi (Lieneke de Visser, personal communication). This incident is part of a long-standing and complex power struggle between various ethnic groups in the Zambezi region. It shows that speaking Fwe is considered a relevant component of identity and ethnic identification, and thus underscores the vitality of the language.

In conclusion, it appears that despite the strong functions of Subiya, Lozi, and English, and widespread bi- and multilingualism, Fwe does not appear to be endangered, and Fwe speakers opt for stable multilingualism instead.

\section{Regional variation}
\hypertarget{Toc75352597}{}\label{bkm:Ref450562330}
Though I have not undertaken a dedicated study focusing on regional variation in Fwe, some observations can be made. An obvious divide, both offered by speakers and seen in the data, is that between Zambian Fwe and Namibian Fwe. The main phonological differences between Zambian and Namibian Fwe are summarized in \tabref{tab:1:1}.

\begin{table}
\label{bkm:Ref505349490}\caption{\label{tab:1:1}Main phonological differences between Zambian and Namibian Fwe}

\begin{tabularx}{\textwidth}{Xl}
\lsptoprule
Zambian Fwe & Namibian Fwe\\
\midrule 
loss of clicks & maintenance of clicks\\
overgeneralization of /l/ & [l] only as conditioned allophone of /r/\\
epenthetic [h] frequently used & epenthetic [h] rarely used\\
\lspbottomrule
\end{tabularx}
\end{table}
Morphological differences between the two varieties are more salient than phonological or lexical differences. \tabref{tab:1:2} presents an overview of grammatical morphemes that differ between Zambian and Namibian Fwe. The two main tendencies are the interchangeability of /s/ and /sh/ in Namibian Fwe, which is not seen in Zambian Fwe, and the correspondence between /a/ in Zambian Fwe with /i/ in Namibian Fwe. This correspondence is seen only in the remote past and inceptive prefixes, both verbal prefixes that occur at the very beginning of the verb.

\begin{table}
\label{bkm:Ref491781507}\caption{\label{tab:1:2}Morphological differences between Zambian and Namibian Fwe}

\begin{tabularx}{\textwidth}{XXl}
\lsptoprule
& Zambian Fwe & Namibian Fwe\\
\midrule 
past & na- & a-\\
reflexive & kí- & rí-\\
remote past & na- & ni-\\
remote future & na- & (á)ra-\\
inceptive & sha- & shi-\\
connective & \textsc{pp} - \textit{o} & \textsc{pp} - \textit{a}\\
persistive & shí- & shí-/sí-\\
negative imperative & ásha- & ásha- / ása-\\
negative infinitive & shá- & shá-/sá-\\
negative subjunctive & sha- & sha-/sa-\\
near future & mbo-/mba- & mbo-\\
\lspbottomrule
\end{tabularx}
\end{table}
The linguistic border between Namibian and Zambian Fwe does not directly follow the national border; the Imusho area in Zambia, directly north of the border, displays many features also found in Namibian Fwe. Furthermore, not all regional differences follow the same geographical distribution.

\section{Earlier research}
\label{bkm:Ref492223019}\hypertarget{Toc75352598}{}
Earlier research on the Fwe language is very limited, and mostly dates from after 2000. The earliest mention of Fwe in the scientific literature is in publications by {\citet{Fortune1970}}, which is limited to listing languages and their approximate geographic locations. {\citet{Baumbach1997}} gives a grammar sketch of five languages of the (then) Eastern Caprivi, including an 18-page grammar sketch of Fwe. This is based, as he states in the introduction, “on very sketchy data” \citep[308]{Baumbach1997}, which undoubtedly explains the many differences between his findings and those presented in this work, such as the omission of noun class 18, the analysis of three rather than four paradigms of demonstratives, or the analysis of stative verbs as present tense verbs and present tense verbs as future tense verbs, to name a few.

{\citet{Seidel2005}} presents a dialectometrical classification of Caprivian languages, including Fwe, which he groups with Subiya, Mbalangwe and Totela, though disregarding Shanjo, which is not spoken in the Caprivi. As the focus of this article is on classification, it presents little in the way of analysis, though the appendix contains a small word list and a list of modern reflexes for reconstructed Bantu phonemes. {\citet{Bostoen2009}} describes the synchronic phoneme inventory and its diachronic development of both Fwe and Shanjo; as shown in chapters \ref{bkm:Ref451511011} and \ref{bkm:Ref451507583}, his findings on the phonology of Fwe mostly tally with mine. A discussion of the history of western Zambian peoples, including Fwe speakers, is presented by de Luna (\citealt{Luna2008,Luna2010,Luna2016}), though, as it is focused on historical analysis, it contains very little linguistic data. {\citet{BostoenSands2012}} discuss the use of clicks in Fwe as well as three other Bantu languages of northern Namibia; as discussed in \sectref{bkm:Ref70695065}, the click inventory that they present for Fwe differs slightly from the findings presented in this work. {\citet{Crane2012}} discusses the use of the verbal suffix \textit{-ite} in various Bantu Botatwe languages, including a brief discussion of its use in Fwe; her analysis of this suffix in Fwe is taken over in the current study (see \sectref{bkm:Ref431984198}).

\section{Data collection and transcription}
\label{bkm:Ref70697316}\hypertarget{Toc75352599}{}
The data on which this study is based were all collected by me over a total of seven months, on four separate occasions. The first field trip took place between April and June 2013 and was mainly spent in the town of Sesheke, Zambia, as well as a week in the village of Imusho, Zambia. The second trip was undertaken in May and June 2014 and took place in the villages of Imusho and Sinjembela in Zambia. The third field trip, from July to September 2015, was mainly spent in the town of Katima Mulilo, Namibia, as well as a week in the village of Imusho, Zambia. A fourth field trip was undertaken in May 2017, and was spent in its entirety in Katima Mulilo, Na\-mibia, combined with a one-day visit to Makanga village, about 70 kilometers east of Katima Mulilo. Although the towns of Sesheke and Katima Mulilo are not predominantly Fwe-speaking, many Fwe speakers can be found there, especially in Katima Mulilo, who have moved there re\-cently from more rural areas.

As Fwe is a virtually undescribed language, data collection consisted mainly of elicitation, espe\-cially at the beginning stages. In elicitation, speakers were presented with as much detail and con\-text as possible to ensure that the data were as close to natural speech as possible. With this method, a total of about 10,000 elicited phrases and sentences were collected, transcribed and translated, as well as about 2,200 lexemes.

In addition to elicitation, natural speech data were collected in the form of stories and conversa\-tional data. A total of seventeen stories were collected: eleven fictional tales, five personal (true) narra\-tives, and a Fwe version of the pear story, a small video clip without spoken text, frequently used in linguistic elicitation \citep{Chafe1980}, amounting to about two hours of narrative. A 45-minute conversation between two speakers was recorded, which was almost completely transcribed and translated. I also acquired songs from the pop artist Tuzizyi, who performs in Lozi, Fwe, and Totela, and tran\-scribed eight of his Fwe songs. Transcription and translation of all data was done by replaying the recording to a native speaker, who slowly repeated the recording sentence by sentence in Fwe (allowing me to transcribe it), and supplied an English translation.

For all examples used in this work (except isolated words and short phra\-ses), the source is indi\-cated with a code: NF for Namibian Fwe and ZF for Zambian Fwe, followed by Elic for elicited data, Narr for narrative data, Conv for conversational data, and Song for pop music. The number at the end of each code indicates the year the data were collected. For example, ZF\_Elic13 repre\-sents elicited data from Zambian Fwe collected in 2013.

Fwe is mainly an oral language, but the increased use of cell phones has created the need for speakers to reduce it to writing. Fwe is usually written with an orthography inspired by the Lozi orthography, which is fairly suitable for this purpose thanks to the overlap between the phoneme inventories of the two languages. An official orthography for Fwe is currently being developed as part of a Bible translation project \citep{Bow2013}. The practical orthography used in this work devi\-ates from this orthography in a number of respects. There are a number of reasons for not adopt\-ing the official orthography wholesale: firstly, it was developed in Zambia and for Zambian Fwe, and makes use of certain orthographical conventions that are common in Zambia but are not well-known in Namibia, such as <zh> for [ʒ]. It also makes use of certain orthographical conven\-tions that are not commonly used in Bantu languages, such as <n\~{} > for [ŋ], and in certain cases the orthography is not the most faithful representation of the spoken form, such as the use of <l> for /r/; although [l] is a conditioned allophone of /r/ in Fwe, it occurs in more restricted contexts than /r/, and therefore /r/ is clearly the underlying form. All these considerations are, of course, of mi\-nor importance for speakers, who will be able to deal equally well with either the official orthography or with the practical orthography used in this work. The practical orthography used in this work is therefore for the benefit of linguists, who lack prior knowledge of the language, and therefore need a more detailed and cross-linguistically common ortho\-graphy, which is not necessary for Fwe speakers.

The symbols used in this practical orthography will be explained in chapter \ref{bkm:Ref451511011} on segmental phonology. Each Fwe example in this work consists of four lines. The first line represents the phonetic realization of the entire sentence, phrase, or word, in which the surface realization of tones are marked. Phonetic and penultimate vowel lengthening are not marked, in order to distinguish them from phonemic vowel length, which is marked. No punctuation is used, as punctuation presumes an understanding of the syntactic structure, which is not available for every example. Periods to indicate the end of sentences are not used, because it is often unclear to me where a sentence ends, and what criteria can be used to establish sentence boundaries. Capitalization is not used, as tone marking is difficult to read on capitalized vowels, and because capitalized words may have grammatical prefixes or clitics. In order to avoid the question of which letter should be capitalized, capitalization is left out altogether. The second line of each example gives the underlying form, in which underlying tones are marked, and in which hyphens indicate morpheme boundaries. The third line gives a morpheme by mor\-pheme gloss, and the last line gives a free translation into English. These orthographical conventions only apply to the Fwe data. Whenever data on other languages are cited, the orthography of the original source is maintained.

