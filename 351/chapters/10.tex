\chapter{Mood}
\label{bkm:Ref99112450}\label{bkm:Ref99112433}\label{bkm:Ref99112414}\hypertarget{Toc75352699}{}
In this chapter the three morphologically marked moods of Fwe are discussed: the imperative in \sectref{bkm:Ref489881434}, the perfective subjunctive in \sectref{bkm:Ref492309168}, and the imperfective subjunctive in \sectref{bkm:Ref489363941}.

\section{Imperative}
\label{bkm:Ref489881434}\hypertarget{Toc75352700}{}
An imperative form in Fwe is formed with a suffix \textit{-e}, but without the subject marker, as in (\ref{bkm:Ref99102804}--\ref{bkm:Ref99102805}). The imperative form ending in \textit{-a}, as commonly found in Bantu languages, does not exist in Fwe.

\ea
\label{bkm:Ref99102804}
\glll yêndè\\
é̲nd-e\\
go-\textsc{pfv}.\textsc{sbjv}\\
\glt ‘Go!’
\z

\ea
\label{bkm:Ref99102805}
zwé hànò\\
\gll zw-é̲      hano\\
come\_out-\textsc{pfv}.\textsc{sbjv}  \textsc{dem}.\textsc{ii}\textsubscript{16}\\
\glt ‘Get out of here!’ (ZF\_Elic14)
\z

The suffix \textit{-e} is also used in the perfective subjunctive, which is only distinguished from the imperative form by the presence of the subject marker. The imperative and the perfective subjunctive also take the same melodic tones. When used without an object marker, the imperative takes melodic tone 1, combined with melodic tone 4, the deletion of underlying high tones, as in (\ref{bkm:Ref71903613}--\ref{bkm:Ref74908194}). (See \sectref{bkm:Ref71539267} for an overview of melodic tones.) With an object marker, the imperative combines melodic tone 4 with melodic tone 3 instead of melodic tone 1, as in (\ref{bkm:Ref71903625}--\ref{bkm:Ref74908216}).

\ea
\label{bkm:Ref71903613}
hùwé ꜝcáhà\\
\gll huw-é̲    cáha\\
shout-\textsc{pfv}.\textsc{sbjv}  very\\
\glt ‘Shout loudly.’
\z

\ea
kàbìré mùnjûò\\
\gll kabir-é̲    mu-N-júo\\
enter-\textsc{pfv}.\textsc{sbjv}  \textsc{np}\textsubscript{18}-\textsc{np}\textsubscript{9}-house\\
\glt ‘Enter the house.’ (NF\_Elic15)
\z

\ea
\label{bkm:Ref74908194}
fùrùmìké kàsúbà kò\\
\gll fu\textsubscript{H}rumik-é̲        ka-súba  ko\\
turn\_upside\_down-\textsc{pfv}.\textsc{sbjv}  \textsc{np}\textsubscript{12}-dish  \textsc{dem}.\textsc{iii}\textsubscript{12}\\
\glt ‘Turn that dish upside down.’ (NF\_Elic17)
\z

\ea
\label{bkm:Ref71903625}
\glll bàtúsè\\
ba\textsubscript{H}-tus-é̲\\
\textsc{om}\textsubscript{2}-help-\textsc{pfv}.\textsc{sbjv}\\
\glt ‘Help them.’
\z

\ea
\glll ndìàmbîsè\\
ndi-amb-í̲s-e\\
\textsc{om}\textsubscript{1SG}-talk-\textsc{caus}-\textsc{pfv}.\textsc{sbjv}\\
\glt ‘Talk to me.’ (NF\_Elic17)
\z

\ea
\label{bkm:Ref74908216}
\glll ndìbèrékèrè\\
ndi-beré̲k-er-e\\
\textsc{om}\textsubscript{1SG}-work-\textsc{appl}-\textsc{pfv}.\textsc{sbjv}\\
\glt ‘Work for me.’ (NF\_Elic15)
\z

The imperative is used to express a command or order. An order expressed with the imperative is considered less polite and more direct than an order expressed with the subjunctive. The imperative can only be used for orders directed at a singular addressee, as in (\ref{bkm:Ref98771149}--\ref{bkm:Ref98771152}). Orders directed at plural addressees are expressed by subjunctives (see Sections \ref{bkm:Ref492309168}-\ref{bkm:Ref489363941}).

\ea
\label{bkm:Ref98771149}
íwè tóndè kúnò\\
\gll iwé    tó̲nd-e    kunó\\
\textsc{pers}\textsubscript{2SG}  watch-\textsc{pfv}.\textsc{sbjv}  \textsc{dem}.\textsc{ii}\textsubscript{17}\\
\glt ‘You! Look here!’ (NF\_Narr15)
\z

\ea
\label{bkm:Ref98771152}
\glll tòntórè\\
to\textsubscript{H}ntor-é̲\\
be\_quiet-\textsc{pfv}.\textsc{sbjv}\\
\glt ‘Be quiet!’ (NF\_Elic17)
\z

The negation of both the imperative and subjunctive form takes a post-initial prefix \textit{ásha-}, and a final vowel suffix \textit{-i}, as well as a different tonal pattern. The negation of imperatives and subjunctives is discussed in \sectref{bkm:Ref499029715}.

\section{Perfective subjunctive}
\label{bkm:Ref492309168}\hypertarget{Toc75352701}{}
The perfective subjunctive form is formed with the suffix \textit{-e} on the verb, and, unlike the imperative, takes a subject marker. Other than the presence of the subject marker, the perfective subjunctive is identical to the imperative, and also takes the same melodic tones: melodic tone 1 and 4 when the verb lacks an object marker, as in (\ref{bkm:Ref74908272}--\ref{bkm:Ref74908273}), or 3 and 4 when the verb includes an object marker, as in (\ref{bkm:Ref74908306}--\ref{bkm:Ref74908307}).

\ea
\label{bkm:Ref74908272}
òtùmbùsé mùrìrò\\
\gll o-tu\textsubscript{H}mbus-é̲      mu-riro\\
\textsc{sm}\textsubscript{2SG}-light-\textsc{pfv}.\textsc{sbjv}  \textsc{np}\textsubscript{3}-fire\\
\glt ‘You should light a fire.’ (ZF\_Elic14)
\z

\ea
\label{bkm:Ref74908273}
mùbíːkè òtúꜝcényà\\
\gll mu-bí̲ːk-e    o-tú-cenyá\\
\textsc{sm}\textsubscript{2PL}-put-\textsc{pfv}.\textsc{sbjv}  \textsc{aug}-\textsc{np}\textsubscript{13}-small\\
\glt ‘You should put a little bit.’ (NF\_Elic15)
\z

\ea
\label{bkm:Ref74908306}
tùmùbóózèrè ècìntú ꜝcákwè\\
\gll tu-mu-boó̲z-er-e      e-ci-ntú    cí-akwé\\
\textsc{sm}\textsubscript{1PL}-\textsc{om}\textsubscript{1SG}-return-\textsc{appl}-\textsc{sbjv}   \textsc{aug}-\textsc{np}\textsubscript{7}-thing  \textsc{pp}\textsubscript{7}-\textsc{poss}\textsubscript{3SG}\\
\glt ‘We should bring his thing back to him.’ (ZF\_Conv13)
\z

\ea
\label{bkm:Ref74908307}
tùzìbbátúrè èzí zìkûnì\\
\gll tu-zi\textsubscript{H}-bba\textsubscript{H}t-ú̲r-e        e-zí    zi-kúni\\
\textsc{sm}\-\textsubscript{1PL}-\textsc{om}\textsubscript{8}-separate-\textsc{sep}.\textsc{tr}-\textsc{pfv}.\textsc{sbjv}  \textsc{aug}-\textsc{dem}.\textsc{i}\textsubscript{8}  \textsc{np}\textsubscript{8}-tree\\
\glt ‘Can we separate these trees?’ (NF\_Elic15)
\z

The perfective subjunctive describes a one-time event, as in (\ref{bkm:Ref99102952}), and contrasts with the imperfective subjunctive, which describes habitual or ongoing events, as in (\ref{bkm:Ref99102971}) (see also \sectref{bkm:Ref489884199}).

\ea
\label{bkm:Ref99102952}
\glll òndìtúsè\\
o-ndi-tus-é̲\\
\textsc{sm}\textsubscript{2SG}-\textsc{om}\textsubscript{1SG}-help-\textsc{pfv}.\textsc{sbjv}\\
\glt ‘You should help me (one time only).’
\z

\ea
\label{bkm:Ref99102971}
\glll wákùndìtùsà\\
o-áku-ndi-tus-a\\
\textsc{sm}\textsubscript{2SG}-\textsc{sbjv}.\textsc{ipfv}-\textsc{om}\textsubscript{1SG}-help-\textsc{fv}\\
\glt ‘You should help me regularly/be helping me.’ (NF\_Elic17)
\z

A near future can be derived from the perfective subjunctive by addition of a future prefix \textit{mbo-}, and an additional high tone on the subject marker (see \sectref{bkm:Ref489880945}).

The perfective subjunctive has various functions. It can express a plan or intention, as in (\ref{bkm:Ref489953824}), where the speaker discusses what he plans to do to escape a fire.

\ea
\label{bkm:Ref489953824}
\textbf{tùpìcùké} mùrìrò \textbf{tùyé} òkò úkàzwîrà\\
\gll tu-pi\textsubscript{H}cuk-é̲      mu-riro tu-y-é̲      o-ko    ú̲-ka-zw-í̲r-a\\
\textsc{sm}\textsubscript{1PL}-escape-\textsc{pfv}.\textsc{sbjv}  \textsc{np}\textsubscript{3}-fire
\textsc{sm}\textsubscript{1PL}-go-\textsc{pfv}.\textsc{sbjv}  \textsc{aug}-\textsc{dem}.\textsc{iii}\textsubscript{17}  \textsc{sm}\textsubscript{3}.\textsc{rel}-\textsc{dist}-come\_out-\textsc{appl}-\textsc{fv}\\
\glt ‘We will dodge the fire, we will go to where it comes from.’ (NF\_Narr17)
\z

The perfective subjunctive can be used to express volition or desire, as in (\ref{bkm:Ref449440622}--\ref{bkm:Ref449440623}).

\ea
\label{bkm:Ref449440622}
nêyè àyéndè nêyè\\
\gll né=ye    a-é̲nd-e    né=ye\\
\textsc{com}=\textsc{pers}\textsubscript{3SG}  \textsc{sm}\textsubscript{1}-go-\textsc{pfv}.\textsc{sbjv}  \textsc{com}=\textsc{pers}\textsubscript{3SG}\\
\glt ‘She too wanted to go with her.’ (NF\_Narr15)
\z

\ea
\label{bkm:Ref449440623}
\glll ndìpátámè\\
ndi-patam-é̲\\
\textsc{sm}\textsubscript{1SG}-lie\_on\_stomach-\textsc{pfv}.\textsc{sbjv}\\
\glt ‘I want to lie down a bit.’ (ZF\_Elic14)
\z

When combined with the adverb \textit{nanga}, the perfective subjunctive expresses uncertainty, as in (\ref{bkm:Ref99103032}--\ref{bkm:Ref99103033}). Note that the adverb \textit{nanga} with the imperfective subjunctive does not express uncertainty, but immediate future (see \sectref{bkm:Ref489884199}).

\ea
\label{bkm:Ref99103032}
nàngà bàkéːzyè bàtùpángé cìmwî\\
\gll nanga  ba-ké̲ːzy-e    ba-tu\textsubscript{H}-pang-é̲    ci-mwí\\
even  \textsc{sm}\textsubscript{2}-come-\textsc{pfv}.\textsc{sbjv}  \textsc{sm}\textsubscript{2}-\textsc{om}\textsubscript{1PL}-do-\textsc{pfv}.\textsc{sbjv}  \textsc{pp}\textsubscript{7}-other\\
\glt ‘He might come and do something else to us.’ (NF\_Narr15)
\z

\ea
wáshàívùkùmì nàngà ìfwê\\
\gll o-ásha-í-vukum-i        nanga  i-fw-é̲\\
\textsc{sm}\textsubscript{2SG}-\textsc{neg}.\textsc{sbjv}-\textsc{om}\textsubscript{9}-throw-\textsc{neg}    even  \textsc{sm}\textsubscript{9}-die-\textsc{pfv}.\textsc{sbjv}\\
\glt ‘Don’t throw it, it might break.’
\z

\ea
\label{bkm:Ref99103033}
àndìzìmísìkìzè màláìtì ángù nàngà àndìhìsíkìzè ènjûò\\
\gll a-ndi-zim-í̲sikiz-e        ma-láiti nanga  a-ndi-his-í̲kiz-e      e-N-júo\\
\textsc{sm}\textsubscript{1}-\textsc{om}\textsubscript{1SG}-go\_out-\textsc{caus}.\textsc{appl}-\textsc{pfv}.\textsc{sbjv}  \textsc{np}\textsubscript{6}-light
even  \textsc{sm}\textsubscript{6}-\textsc{om}\textsubscript{1SG}-\textsc{caus}-\textsc{appl}-\textsc{pfv}.\textsc{sbjv}  \textsc{aug}-\textsc{np}\textsubscript{9}-house\\
\glt ‘S/he must turn off the lights for me, they might burn down my house.’ (NF\_Elic17)
\z

With a first person subject, the perfective subjunctive may express a hortative, as in (\ref{bkm:Ref99103056}--\ref{bkm:Ref99103057}).

\ea
\label{bkm:Ref99103056}
\glll tùràpérè\\
tu-raper-é̲\\
\textsc{sm}\textsubscript{1PL}-pray-\textsc{pfv}.\textsc{sbjv}\\
\glt ‘Let’s pray.’ (ZF\_Elic14)
\z

\ea
ndìrìkòshórèkó bùryô\\
\gll ndi-ri\textsubscript{H}-ko\textsubscript{H}sh-ó̲r-e=ko      bu-ryó\\
\textsc{sm}\textsubscript{1SG}-\textsc{om}\textsubscript{5}-cut-\textsc{sep}.\textsc{tr}-\textsc{pfv}.\textsc{sbjv}=\textsc{loc}\textsubscript{17}  \textsc{np}\textsubscript{14}-just\\
\glt ‘Let me just cut it.’ (ZF\_Narr14)
\z

\ea
\label{bkm:Ref99103057}
kàntí ndìkùtòmbwérìsè\\
\gll kantí  ndi-ku-tombwé̲r-is-e\\
well  \textsc{sm}\textsubscript{1SG}-\textsc{om}\textsubscript{2SG}-weed-\textsc{caus}-\textsc{pfv}.\textsc{sbjv}\\
\glt ‘Then let me help you weed.’ (NF\_Narr15)
\z

With a second person subject, the subjunctive may express a command, as in (\ref{bkm:Ref99103078}--\ref{bkm:Ref99103080}).

\ea
\label{bkm:Ref99103078}
òkêːzyè òndìtúsè\\
\gll o-ké̲ːzy-e      o-ndi-tus-é̲\\
\textsc{sm}\textsubscript{2SG}-come-\textsc{pfv}.\textsc{sbjv}  \textsc{sm}\textsubscript{2SG}-\textsc{om}\textsubscript{1SG}-help-\textsc{pfv}.\textsc{sbjv}\\
\glt ‘Come and help me.’
\z

\ea
\label{bkm:Ref99103080}
mùtòntórè mùyéndè mùkàráːrè\\
\gll mu-to\textsubscript{H}ntor-é̲    mu-é̲nd-e    mu-ka-raː\textsubscript{H}r-é̲\\
\textsc{sm}\textsubscript{2PL}-be\_quiet-\textsc{pfv}.\textsc{sbjv}  \textsc{sm}\textsubscript{2PL}-go-\textsc{pfv}.\textsc{sbjv}  \textsc{sm}\textsubscript{2PL}-\textsc{dist}-sleep-\textsc{pfv}.\textsc{sbjv}\\
\glt ‘Be quiet and go to sleep.’ (NF\_Elic15)
\z

A command expressed with the subjunctive form is usually interpreted as more polite than a command expressed with the imperative form (see \sectref{bkm:Ref489881434}). To express even more politeness, the prefix \textit{ngá-} ‘can’ can be added, as in (\ref{bkm:Ref99103146}).

\ea
\label{bkm:Ref99103146}
ngóndìtúsè kùndìkwátìrà ècí cìpùpè\\
\gll ngá-o-ndi-tus-é̲      ku-ndi-kwát-ir-a e-cí    ci-pupe \\
can-\textsc{sm}\textsubscript{2SG}-\textsc{om}\textsubscript{1SG}-help-\textsc{pfv}.\textsc{sbjv}  \textsc{inf}-\textsc{om}\textsubscript{1SG}-grab-\textsc{appl}-\textsc{fv}
\textsc{aug}-\textsc{dem}.\textsc{i}\textsubscript{7}  \textsc{np}\textsubscript{7}-container\\
\glt ‘Can you please carry that container for me?’ (ZF\_Elic14)
\z

Subjunctives are also used in subordinate clauses, where they can carry the same functions as subjunctives in main clauses, or can be used to express the desired or intended consequence of the event expressed in the main clause, as in (\ref{bkm:Ref99103166}--\ref{bkm:Ref99103168}).

\ea
\label{bkm:Ref99103166}
bàmùbérékérà òkùtéyè \textbf{àfúmè}\\
\gll ba-mu-berek-er-á̲  okuteye  a-fum-é̲\\
\textsc{sm}\textsubscript{2}-\textsc{om}\textsubscript{1}-work-\textsc{fv}  that    \textsc{sm}\textsubscript{1}-become\_rich-\textsc{pfv}.\textsc{sbjv}\\
\glt ‘They work for him, \textbf{so} \textbf{that} \textbf{he} \textbf{becomes} \textbf{rich}.’ (NF\_Elic17)
\z

\ea
\label{bkm:Ref99103168}
mbóshàkèsháké àkàshérêŋì \textbf{òpàngé} àkà-business\\
\gll mbo-ó̲-shake-shak-é̲      a-ka-sheréŋi o-pang-é̲      a-ka-business \\
\textsc{near}.\textsc{fut}-\textsc{sm}\textsubscript{2SG}-\textsc{pl}2-find-\textsc{pfv}.\textsc{sbjv}  \textsc{aug}-\textsc{np}\textsubscript{12}-money
\textsc{sm}\textsubscript{2SG}-make-\textsc{pfv}.\textsc{sbjv}  \textsc{aug}-\textsc{np}\textsubscript{12}-business\\
\glt ‘You will find a little money \textbf{so} \textbf{that} \textbf{you} \textbf{make} \textbf{a} \textbf{small} \textbf{business}.’ (ZF\_Conv13)
\z

The perfective subjunctive can combine with the remoteness prefix \textit{na-}; in subordinate clauses, this indicates a remote future, as in (\ref{bkm:Ref74908458}--\ref{bkm:Ref74908459}). In main clauses, the perfective subjunctive with \textit{na-} expresses the same functions as the perfective subjunctive without \textit{na-}, only set in the remote future, such as a command to be followed up tomorrow, not today. This use is illustrated in (\ref{bkm:Ref74908501}--\ref{bkm:Ref74908502}). Remoteness is usually considered as at least one day removed from the day of speaking, as it is throughout the tense/aspect system of Fwe (see, for instance, the remote past perfective, \sectref{bkm:Ref489260766}).

\ea
\label{bkm:Ref74908458}
mbùtí náyìwánè èyí shérêŋì\\
\gll N-bu-tí    na-á̲-i\textsubscript{H}-wan-é̲      e-í    ∅-sheréŋi\\
\textsc{cop}-\textsc{np}\textsubscript{14}-how  \textsc{rem}-\textsc{sm}\textsubscript{1}-\textsc{om}\textsubscript{9}-find-\textsc{pfv}.\textsc{sbjv}  \textsc{aug}-\textsc{dem}.\textsc{i}\textsubscript{9}  \textsc{np}\textsubscript{9}-money\\
\glt ‘How will he get this money?’ (Lit.: ‘It is how that he will get this money?’) (ZF\_Conv13)
\z

\ea
\label{bkm:Ref74908459}
éwè zyúmùnyà ndíwè nóbè háꜝkátì\\
\gll éwe    zyú-munya ndí-we  na-ó̲-b-e      há-ka-tí\\
\textsc{pers}\textsubscript{2SG}  \textsc{pp}\textsubscript{1}-other
\textsc{cop}-\textsc{pers}\textsubscript{2SG}  \textsc{rem}-\textsc{sm}\textsubscript{2SG}-be-\textsc{pfv}.\textsc{sbjv}  \textsc{np}\textsubscript{16}-\textsc{np}\textsubscript{12}-middle\\
\glt ‘You, the other one, it is you who will be in the middle.’ (ZF\_Narr13)
\z

\ea
\label{bkm:Ref74908501}
nóyêndè zyônà\\
\gll na-ó̲-é̲nd-e      zyóna\\
\textsc{rem}-\textsc{sm}\textsubscript{2SG}-go-\textsc{pfv}.\textsc{sbjv}  tomorrow\\
\glt ‘Go tomorrow.’
\z

\ea
\label{bkm:Ref74908502}
nìbézyè bàkùbónè\\
\gll ni-bá̲-izy-e      ba-ku-bo\textsubscript{H}n-é̲\\
\textsc{rem}-\textsc{sm}\textsubscript{2SG}-come-\textsc{pfv}.\textsc{sbjv}  \textsc{sm}\textsubscript{2}-\textsc{om}\textsubscript{2SG}-see-\textsc{pfv}.\textsc{sbjv}\\
\glt ‘She has to come and take care of you.’ (NF\_Narr17)
\z

The remoteness prefix \textit{na-} is used with the verb \textit{ta} ‘say’ in the subjunctive, followed by a subjunctive main verb, to express an event that almost, but not quite, took place, as in (\ref{bkm:Ref99103237}--\ref{bkm:Ref99103238}).

\ea
\label{bkm:Ref99103237}
nàté ndìmùcáîsè zywínà\\
\gll na-ta-é̲    ndi-mu-caí̲s-e      zwiná\\
\textsc{rem}-say-\textsc{pfv}.\textsc{sbjv}  \textsc{sm}\textsubscript{1SG}-\textsc{om}\textsubscript{1}-bump\_into-\textsc{pfv}.\textsc{sbjv}  \textsc{dem}.\textsc{iv}\textsubscript{1}\\
\glt ‘I almost bumped into her/him, that one.’ (NF\_Elic17)
\z

\ea
\label{bkm:Ref99103238}
nòbónì cwárè rìn’ éòndè nàté òírè\\
\gll no-bón-i      cwaré  riná    e-∅-onde na-ta-é    o-ir-é̲ \\
\textsc{sm}\textsubscript{2SG}.\textsc{pst}-see-\textsc{npst}.\textsc{pfv}  then  \textsc{dem}.\textsc{iv}\textsubscript{5}  \textsc{aug}-\textsc{np}\textsubscript{5}-waterlily
\textsc{rem}-say-\textsc{pfv}.\textsc{sbjv}  \textsc{sm}\textsubscript{2SG}-go.\textsc{appl}-\textsc{pfv}.\textsc{sbjv}\\
\glt ‘Did you see that flower that you wanted to go to?’ (Context: a boy wanted to pick a waterlily. A bird warns him not to, picks up the waterlily and reveals a snake underneath it. The bird returns to the boy and discusses what would have happened if he went to pick the waterlily as he planned.) (NF\_Narr17)
\z
\section{Imperfective subjunctive}
\label{bkm:Ref489363941}\hypertarget{Toc75352702}{}\label{bkm:Ref489884199}
An imperfective subjunctive is formed with the post-initial prefix \textit{áku-}, as in (\ref{bkm:Ref99103273}). Verbs in the imperfective subjunctive maintain their underlying tones, and aside from the high tone associated with the prefix \textit{áku-} itself, no melodic high tones are added.

\ea
\label{bkm:Ref99103273}
ènwé ꜝbáꜝnángù mwákùkàrà\\
\gll enwé    bá-na-angú    mu-áku-kar-a\\
\textsc{pers}\textsubscript{2PL} \textsc{np}\textsubscript{2}-child-\textsc{poss}\textsubscript{1SG}  \textsc{sm}\textsubscript{2PL}-\textsc{sbjv}.\textsc{ipfv}-stay-\textsc{fv}\\
\glt ‘You, my children, must stay here.’ (NF\_Elic17)
\z

The second syllable \textit{ku} of the prefix \textit{áku-} is derived from the infinitive prefix \textit{ku-}. Two of the characteristics of the imperfective subjunctive point to its origin in an infinitive: the fact that the syllable \textit{ku} may change to \textit{ka} when used with the distal marker (see (\ref{bkm:Ref490842863})), and the lack of melodic tones, which is typical of infinitives and rarely seen in inflected verbs (see also \sectref{bkm:Ref71540393}).

Habitual is a subtype of imperfective aspect, and the imperfective subjunctive is therefore often used with a habitual meaning, combined with the habitual suffix \textit{-ang}, as in (\ref{bkm:Ref99103335}) (see also \sectref{bkm:Ref451268511}).

\ea
\label{bkm:Ref99103335}
\glll wákùmùtùsàngà\\
o-áku-mu-tus-ang-a\\
\textsc{sm}\textsubscript{2SG}-\textsc{sbjv}.\textsc{ipfv}-\textsc{om}\textsubscript{1}-help-\textsc{hab}-\textsc{fv}\\
\glt ‘You should help her/him regularly.’ (NF\_Elic17)
\z

Without the habitual suffix \textit{-ang}, both a habitual and a progressive reading are possible, as in (\ref{bkm:Ref99103358}). The imperfective subjunctive does not combine with overt progressive markers, and in most cases, such as in (\ref{bkm:Ref99103360}), the habitual reading appears to be preferred.

\ea
\label{bkm:Ref99103358}
\glll wákùmùtùsà\\
o-áku-mu-tus-a\\
\textsc{sm}\textsubscript{2SG}-\textsc{sbjv}.\textsc{ipfv}-\textsc{om}\textsubscript{1}-help-\textsc{fv}\\
\glt ‘You should be helping her/him.’ / ‘You should help her/him regularly.’
\z

\ea
\label{bkm:Ref99103360}
\glll wákùmùtùsàngà\\
o-áku-mu-tus-ang-a\\
\textsc{sm}\textsubscript{2SG}-\textsc{sbjv}.\textsc{ipfv}-\textsc{om}\textsubscript{1}-help-\textsc{hab}-\textsc{fv}\\
\glt ‘You should help her/him regularly.’ (NF\_Elic17)
\z

From the imperfective subjunctive, a near future imperfective is derived by addition of the prefix \textit{mbo-}, see \sectref{bkm:Ref490749852}.

More data are needed to study the range of meanings of the imperfective subjunctive, though it appears to be similar to that of the perfective subjunctive, e.g. a command, as in (\ref{bkm:Ref489953949}), or a hortative, as in (\ref{bkm:Ref489953950}).

\ea
\label{bkm:Ref489953949}
\glll mwákùrítèèzà\\
mu-áku-rí-teez-a\\
\textsc{sm}\textsubscript{2}-\textsc{sbjv}.\textsc{ipfv}-\textsc{refl}-listen-\textsc{fv}\\
\glt ‘You have to listen to each other.’
\z

\ea
\label{bkm:Ref489953950}
\glll ndákùmènèkàngà\\
ndi-áku-menek-ang-a\\
\textsc{sm}\textsubscript{1SG}-\textsc{sbjv}.\textsc{ipfv}-wake\_early-\textsc{hab}-\textsc{fv}\\
\glt ‘I should regularly wake up early.’ (NF\_Elic17)
\z

Like the perfective subjunctive, the imperfective subjunctive may combine with the adverb \textit{nanga} ‘even’, not to express uncertainty, as is the case for the perfective subjunctive, but to express immediate future, as in (\ref{bkm:Ref490842863}--\ref{bkm:Ref99103420}).

\ea
\label{bkm:Ref490842863}
nàngà ndákàyà\\
\gll nanga  ndi-áka-y-a\\
even  \textsc{sm}\textsubscript{1SG}-\textsc{sbjv}.\textsc{ipfv}.\textsc{dist}-go-\textsc{fv}\\
\glt ‘I am about to leave.’ (NF\_Elic15)
\z

\ea
òmùndáré nàngà wákùbîzwà\\
\gll o-mu-ndaré    nanga  u-áku-bízw-a\\
\textsc{aug}-\textsc{np}\textsubscript{3}-maize  even  \textsc{sm}\textsubscript{3}-\textsc{sbjv}.\textsc{ipfv}-ripen-\textsc{fv}\\
\glt ‘The maize is almost ripe/is about to ripen.’
\z

\ea
\label{bkm:Ref99103420}
nàngà bákùhùrà ndìkàréː ꜝbákànànúkà\\
\gll nanga  ba-áku-hur-a    ndi-ka-réː    bá̲-ka-nanuk-á̲\\
even  \textsc{sm}\textsubscript{2}-\textsc{sbjv}.\textsc{ipfv}-arrive-\textsc{fv}  \textsc{cop}-\textsc{adv}-long  \textsc{sm}\textsubscript{2}.\textsc{rel}-\textsc{dist}-leave-\textsc{fv}\\
\glt ‘S/he is about to arrive, s/he left a long time ago.’ (NF\_Elic17)
\z

