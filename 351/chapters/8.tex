\chapter{Tense}
\hypertarget{Toc75352677}{}
The following chapters describe the expression of the inter-related categories of tense, aspect, mood, space, and negation, which mostly make use of verbal affixes and auxiliaries. The interpretation of tense, aspect and mood (TAM) constructions also depends on lexical aspect, the inherent or contextually constructed phase structure of a verb. \sectref{bkm:Ref74925053} discusses some basic theoretical concepts that are required for understanding the Fwe TAM system, as well as a basic introduction to the lexical aspectual categories that are relevant in Fwe. The remainder of this chapter is dedicated to discussing the various tense constructions used in Fwe.

\section{TAM constructions in Fwe}
\label{bkm:Ref74925053}\hypertarget{Toc75352678}{}
Tense situates an event before, after or overlapping with a certain reference point. The reference point is often the time of speaking, e.g. “utterance time” \citep{Klein1994}. Other reference points are also possible, in subordinate clauses, for instance, which may require the use of a more flexible reference point, which {\citet{CoverTonhauser2015}} call “evaluation time”. The interpretation of TAM constructions that are not evaluated with respect to the utterance time, but with respect to some other “evaluation time”, will be left for future research.

Tense, aspect and mood are closely related in Fwe. This is most clearly seen in the system of past tense and subjunctive constructions, which are all divided into perfective and imperfective constructions. There is an extensive (theoretical) literature on aspect and (im)perfectivity, but recurrent definitions include a distinction between complete (perfective) and incomplete (imperfective), and a distinction between an event-external viewpoint (perfective) and an event-internal viewpoint (imperfective) \citep[27]{Klein1994}. No attempt at a detailed and comprehensive definition of aspect in Fwe is made here, but it seems that especially the difference in viewpoint is relevant in distinguishing perfective and imperfective aspect in Fwe. The near and remote past perfective constructions present the event as a single, completed whole, and do not allow reference to its internal structure; the event is viewed “from the outside”. As such the past perfective constructions can give a present (result) state or a past interpretation, depending on the lexical aspect of the verb, as discussed below. The near and remote past imperfective constructions, on the other hand, focus on the internal structure of the event, viewing it “from the inside”. As such the past imperfective constructions can give readings such as past progressive or habitual, as discussed in more detail below.

The distinction between perfective and imperfective constructions also determines their co-occurrence with aspectual markers. Fwe has specific markers for progressive, habitual, stative, and persistive aspect, which are subtypes of imperfective aspect (e.g. \citealt{Comrie1976}, among others), and can therefore not be used in perfective constructions. Subjunctives also have a perfective and an imperfective construction, and this also affects the near future, which derives from the subjunctive by addition of a near future prefix. These show the same co-occurrence restrictions as past tense constructions, with only the imperfective subjunctive allowing co-occurrence with markers of a subtype of imperfective aspect.

The fact that the past and future constructions are all have a perfective and an imperfective form raises the question whether these constructions should be considered tenses/moods or aspects. There are a number of reasons not to consider them primarily aspectual. Firstly, their formal properties are more similar to those of other tense constructions than those of aspect markers. Constructions that unambiguously express aspect consist of a single affix or auxiliary and generally lack their own melodic tones (with the exception of the stative, see \sectref{bkm:Ref431984198}). Constructions that express tense without an aspectual distinction (and are therefore unambiguously temporal), such as the present or the remote future, make use of a complex construction with various affixes, and do use melodic tone. Furthermore, for past constructions their temporal semantics is more detailed than their aspectual semantics. Aspectually, past forms only distinguish perfective or imperfective, whereas temporally, they distinguish not only past tense but also a degree of remoteness, namely near past versus remote past.

The interpretation of TAM constructions is influenced by the inherent structure of the event, its lexical aspect. Three main lexical aspectual classes are relevant: dynamic, change-of-state, and true stative, as summarized in \tabref{tab:8:1}.

\begin{table}
\label{bkm:Ref492291890}\caption{\label{tab:8:1}Lexical aspect}
\begin{tabularx}{\textwidth}{Xll}
\lsptoprule
Dynamic & long nucleus & \textit{bútùkà} ‘run’\\
Change-of-state & short nucleus & without an onset: \textit{ŋàtùkà} ‘break’\\
&  & with an onset: \textit{nùnà} ‘become fat’\\
True stative & unbounded nucleus & \textit{shàkà} ‘want’\\
\lspbottomrule
\end{tabularx}
\end{table}

Different models exist for the analysis of lexical aspect, and languages differ with respect to the number and kinds of subdivisions that they make, and the way lexical verbs are distributed across them. A model originally developed by {\citet{Freed1979}} for English, and since then applied to various Bantu languages by {\citet{Botne1983}}, {\citet{Kershner2002}}, {\citet{Seidel2008}}, {\citet{Crane2011}}, {\citet{Persohn2017}} and others, divides events into three phases, an onset, nucleus, and coda. The nucleus is the characteristic, most prominent phase of the event. The onset describes the phase leading up the nucleus, and the coda the phase following the nucleus. For instance, the Fwe verb \textit{nun} ‘become fat’ includes an onset phase of becoming fat, a pivotal nucleus in which the processes of becoming fat is completed and the state of being fat commences, and an ensuing coda phase of being fat. Every event has a nucleus, but the presence of an onset and a coda phase is optional, depending on the lexical verb as well as its wider context. Verb stems can be divided into different lexical-aspectual classes based on the duration of the nuclear phase of the event, which can be short (almost instantaneous), in the case of change-of-state verbs such as Fwe \textit{bomb} ‘become wet’, or \textit{coːk} ‘break’, or more drawn out in duration, in the case of dynamic verbs such as Fwe \textit{zyáːk} ‘build’ or \textit{bútuk} ‘run’.

The distinction between change-of-state verbs and dynamic verbs is central in many Bantu languages (\citealt{CranePersohn2019}), including Fwe: change-of-state verbs and dynamic verbs have a different interpretation in a number of constructions, most notably the present, the near past perfective and the stative. In addition to these two main categories, Fwe also has a category of verbs encoding events that completely lack internal phasic structure, which I refer to as “true statives” (following \citealt{Crane2011}). Examples of true stative verbs in Fwe are \textit{shak} ‘want, like’, \textit{tíiz} ‘be fearsome/dangerous’, though in general true stative verbs in Fwe are rare. Certain verbs can be used in different lexical aspectual classes, which may involve a change in interpretation: the verb \textit{shak} can have a true stative use with the interpretation ‘want, like, love’, but also a dynamic use with the interpretation ‘look for’.

Verbs can be further subdivided depending on the presence of a coda phase. Change-of-state verbs typically have a coda phase, which is the resultant state of the change in state denoted by the nucleus, e.g. for \textit{bomb} ‘become wet’, the coda phase would include ‘being wet’. Dynamic verbs may also have a coda phase, though this is heavily dependent on context.

Events also differ in whether they conceptualize an onset phase, the phase leading up to the nucleus. Events with an onset phase are, for instance, \textit{nun} ‘become fat’, where the nucleus consists of the pivotal transition into a state of being fat, and the onset phase consists of the drawn out process of becoming more and more fat, until the pivotal nucleus is reached. Events without an onset phase are, for instance, \textit{ŋatuk} ‘break’, where there is no phase that leads up to the nuclear change of breaking. The presence of an onset phase is mainly relevant to the interpretation of the progressive and inceptive aspects, discussed in \sectref{bkm:Ref431917333} and \ref{bkm:Ref445905588}.

Lexical aspect can be influenced by derivational suffixes. The passive, for instance, derives a change-of-state verb, so that when the passive suffix is used with a dynamic verb, the verb’s lexical aspect changes from dynamic to change-of-state. Verbs with the intransitive forms of the separative and impositive suffixes also function as change-of-state verbs. Verbs with the neuter suffix tend to function as stative verbs, though they can also be used as change-of-state verbs.

Lexical aspect can be further influenced by the context of the utterance as a whole, for instance, by the presence and nature of the object (see e.g. \citealt{Verkuyl1972}). A Fwe example where the presence of an object influences lexical aspect is with the dynamic verb \textit{bar} ‘read’. Without an object, it is considered to lack a coda state, and as such use with the stative suffix \textit{-ite} is generally considered ungrammatical. The verb phrase \textit{bàrà mbúkà} ‘read a book’, however, does have an associated coda state (namely ‘knowing the content of the book’), and therefore use with stative -\textit{ite} was accepted.\footnote{The conceptualization of a coda state with dynamic verbs is dependent on more than the presence and nature of the object, but depends on the general context as well. For instance, \textit{nywá} ‘drink’ essentially lacks a coda phase, but can still be used with the stative \textit{-ite} to express ‘being drunk’. In this case, the (non-linguistic) context is used to construct a state associated with this verb.}

Finally, it should be noted that the lexical aspectual classes that are distinguished here have been established based on their interaction with TAM constructions. No other tests have been conducted, such as acceptability and interpretation with certain time adverbials. However, the lexical aspectual classes that are proposed here do account for the interpretation of verbs in a wide variety of constructions.

Having introduced the theoretical concepts and lexical aspectual distinctions that are relevant for the analysis of tense, aspect and mood in Fwe, I will now turn to the analysis of TAM constructions in Fwe. Considering their formal properties, most TAM constructions make use of one or more affixes with or without one or more melodic tone patterns (see \sectref{bkm:Ref71539267} on melodic tone). For some TAM constructions, it is not possible to determine the exact meaning of all the different (segmental and tonal) morphemes that make up a construction, and the semantics of a TAM construction is often not a simple product of the semantic import of its composite morphemes. This poses some challenges in glossing these TAM constructions; the glossing conventions chosen will be justified in the relevant subsections. TAM constructions will be presented in a template form (as commonly used in the study of Bantu tense and aspect), e.g. [pre-initial]-\textsc{sm}-[post-initial]-B-[final vowel], where \textsc{sm} stands for the subject marker, and B for the verb base, the verb root with optional derivational suffixes. An overview of the templates and melodic tone patterns of TAM constructions is given in \tabref{tab:8:2}.

\begin{table}
\label{bkm:Ref506388187}\caption{\label{tab:8:2}TAM constructions}
\begin{tabularx}{\textwidth}{QQl}
\lsptoprule
Construction & Segmental form & Melodic tone \\
\midrule
Present & \textsc{sm}-B-\textit{a} & MT 1, 4\\
Near past perfective & \textsc{sm}-\textit{a/na}-B-\textit{i} & MT 3\\
Remote past perfective & \textit{na}-\textsc{sm}-\textit{a}-B-\textit{a} & MT 2\\
Near past imperfective & \textsc{sm}-\textit{aku}-B-\textit{a} & -\\
Remote past imperfective & \textit{ka}-\textsc{sm}-B-\textit{a} & MT 1, 2, 4\\
Remote future\newline (Zambian Fwe) & \textit{na}-\textsc{sm}-\textit{na}-B-\textit{a} & MT 2\\
Remote future\newline (Namibian Fwe) & (\textit{ni}-)\textsc{sm}-(\textit{á})\textit{ra}-B-\textit{a} & MT 2\\
Subjunctive perfective & \textsc{sm}-B-\textit{e} & MT 1, 4 / MT 3\\
Subjunctive imperfective & \textsc{sm}-\textit{áku}-B-\textit{a} & -\\
Near future & pre-initial \textit{mbo-/mba-} + subjunctive & -\\
Progressive & auxiliary \textit{kwesi}/\textit{iná} & -\\
Stative & final vowel -\textit{ite} & MT 3\\
Habitual 1 & suffix \textit{-ang} & -\\
Habitual 2 & \textsc{sm}-\textit{náku}-B-a & -\\
Persistive & post-initial \textit{shí-} & -\\
Inceptive & pre-initial \textit{shi-/she-/sha-} & -\\
\lspbottomrule
\end{tabularx}
\end{table}

The following sections discuss the different tense constructions used in Fwe. Tense constructions situate events before, after, or during utterance time. They differ in whether they target the nucleus of the event, or the entire event, which gives rise to different interpretations based on the verb’s lexical aspect. The present construction situates the event’s nucleus at least partially after the utterance time; if the event structure allows, the nucleus may overlap with UT, but the continuation of the nucleus after UT is the present’s basic meaning. Past contsructions are divided into near and remote pasts, which each have a perfective and imperfective form. The remote and near past perfective, too, target the nucleus of the event, situating the event’s nucleus completely before utterance time. These constructions do not specify if the event’s coda phase (if present) overlaps with utterance time; both an interpretation where the entire coda phase is situated before UT, and one where the coda phase overlaps with UT, are possible. The remote and near past imperfective, on the other hand, do not target the event nucleus, but the entire event, situating the event completely before utterance time, meaning that the event’s coda phase (if present) cannot overlap with UT. The near and remote future constructions situate the event’s nucleus in the future, that is after UT, and do not allow overlap between the nucleus and UT. \tabref{tab:8:3} gives an overview of tense constructions, their segmental and suprasegmental realization, their basic meaning, and their main uses.

\begin{table}[t]
\label{bkm:Ref492292016}\caption{\label{tab:8:3}Tense constructions}
\begin{tabularx}{\textwidth}{lQQ}
\lsptoprule
Label & Basic meaning & Main uses\\
\midrule
Present & nucleus (partially/ completely) in the future & gnomic; generic; futurate; modal; present\\
Near Past Perfective (NPP) & nucleus in the recent past; external viewpoint & recent past; present state\\
Remote Past Perfective (RPP) & nucleus in the remote past; external viewpoint & remote past; present state\\
Past imperfective (PI) & nucleus in the (remote) past; internal viewpoint & past imperfective\\
Near Past Imperfective (NPI) & nucleus in the near past; internal viewpoint & past progressive\\
Near Future Perfective & nucleus in the near future; external viewpoint & near future\\
Near Future Imperfective & nucleus in the near future; internal viewpoint & near future habitual, progressive\\
Remote Future & nucleus in the remote future & remote future\\
\lspbottomrule
\end{tabularx}
\end{table}
\section{Present}
\label{bkm:Ref72233436}\hypertarget{Toc75352679}{}
The present tense construction has the form \textsc{sm}-B-a, that is the verb base is used with the default final vowel -\textit{a}. The present takes two melodic tones (MT), MT 1 (assigned to the verb’s last mora), and MT 4 (deletion of lexical tones). An example of a verb in the present is given in (\ref{bkm:Ref469479628}).

\ea
\label{bkm:Ref469479628}
\glll bàbùtúkà\\
ba-bu\textsubscript{H}tuk-á̲\\
\textsc{sm}\textsubscript{2}-run-\textsc{fv}\\
\glt ‘They are running.’ (NF\_Elic15)
\z

One of the characteristics of melodic tone 1 (see \sectref{bkm:Ref71540310}) is that the high tone is not assigned to the last verb mora, but to the penultimate syllable, if this syllable contains a long vowel. This is illustrated in (\ref{bkm:Ref469479857}), where the melodic high tone is assigned to the penultimate syllable /zyi/, because its vowel is lengthened by the following nasal-consonant cluster, and in (\ref{bkm:Ref469479859}), where the high tone is assigned to the penultimate syllable /mbwe/, because the vowel is lengthened due to the preceding glide (see also \sectref{bkm:Ref459132421}; note that automatic vowel lengthening is not marked in the practical orthography used here).

\ea
\label{bkm:Ref469479857}
àzyímbà nênjà \\
\gll a-zyí̲mb-a    nénja\\
\textsc{sm}\textsubscript{1SG}-sing-\textsc{fv}  well\\
\glt ‘She sings well.’
\z

\ea
\label{bkm:Ref469479859}
tùtòmbwérà shûnù\\
\gll tu-tombwé̲r-a  shúnu\\
\textsc{sm}\textsubscript{1PL}-weed-\textsc{fv}  today\\
\glt ‘We are weeding today.’ (NF\_Elic15)
\z

None of the formal characteristics of the present construction can be analyzed as marking present tense: the suffix \textit{-a} is the default final vowel suffix, used in the majority of TAM constructions, including many that are incompatible with a present meaning. The same is true of the two melodic tones, MT 1 and MT 4: they are the two most common melodic tones, used in a variety of constructions (see \tabref{tab:3:5}). Comparison with other tense constructions might suggest a zero post-initial morpheme marking present tense; as seen in \tabref{tab:8:3}, most tense constructions use a post-initial marker. The remote past imperfective (with a template ka-\textsc{sm}-B-a), however, also does not use a post-initial morpheme, nor does the near future perfective (with a template mbo-\textsc{sm}-B-e), so there is no one-to-one correspondence between a post-initial zero marking and a present interpretation.\footnote{A historical analysis of a post-initial zero morpheme marking the present is more likely. The remote past imperfective has grammaticalized from the present construction, as discussed in \sectref{bkm:Ref492377456}. Furthermore, the near future is synchronically based on a subjunctive construction (see \sectref{bkm:Ref490749852}).} Rather, the present construction is a morphological “null form”, commonly used to indicate present tense in Bantu languages \citep[117]{Nurse2008}. As will be shown in the discussion of the interpretations of the present construction, its lack of morphological marking corresponds to a relative lack of semantic specification.

The syntactic use of the present construction differs between Namibian and Zambian Fwe. In Namibian Fwe, a present verb may occur on its own as a full and grammatical utterance. In Zambian Fwe, a present verb is only a grammatical utterance when supplemented by another word, such as a subject, object, locative or adverb. Otherwise, a fronted infinitive construction is used. This is discussed in \sectref{bkm:Ref431917326}.

Semantically, the present construction has a wide variety of different interpretations, depending on lexical and grammatical aspect, linguistic and non-lin\-guis\-tic context. The basic meaning of the present construction is that the event’s nucleus is situated, at least partially, after the time of speaking (utterance time, UT). Whether the nucleus also overlaps with UT is not specified; it is possible, but not obligatory. The present construction only references the nuclear phase; an onset phase leading up to the nucleus cannot be targeted by the present construction. This results in a number of different possibilities, partially dependant on lexical aspect. (\ref{bkm:Ref75247169}) illustrates the interpretations of the present with dynamic verbs, using the verb \textit{bútuk} ‘run’. It is possible for the entire nucleus of the verb to be situated after UT; this results in a futurate, modal or hypothetical interpretation ‘I will/would/can run’. It is also possible for the nucleus to overlap with UT, as long as it extends beyond UT, giving a progressive interpretation, ‘I am running’. It is also possible for the nucleus to be situated intermittently before and after UT, giving a habitual or generic/gnomic interpretation, ‘I (usually) run’. It is not possible, however, for the nucleus to end at UT, because this does not satisfy the present construction’s basic criterion of extending beyond UT.

\ea
\label{bkm:Ref75247169}
ndìbùtúkà\\
ndi-bu\textsubscript{H}tuk-á̲\\
\textsc{sm}\textsubscript{1SG}-run-\textsc{fv}\\
Future/modal/hypothetical: ‘I will/would/can run.’\\
Progressive: ‘I am running.’\\
Habitual/generic/gnomic: ‘I (usually) run.’
\z

(\ref{bkm:Ref75248605}) illustrates the interpretation of the present with change of state verbs, using the change-of-state verb \textit{beng} ‘become angry’. The nucleus of this verb describes the pivotal moment when the state of being angry is reached. For such verbs, it is not possible for the nucleus of the event to extend beyond UT as well as overlap with UT. A progressive interpretation is therefore excluded: the only way in which events with a short nucleus can satisfy the present construction’s criterion that the nucleus extends beyond UT is by situating the entire nucleus after UT. Therefore the only possible interpretation of the present construction with verbs with a short nucleus is futurate/modal/hypothetical, i.e. essentially non-present.

\ea
\label{bkm:Ref75248605}
ndìbêngà\\
ndi-bé̲ng-a\\
\textsc{sm}\textsubscript{1SG}-become\_angry-\textsc{fv}\\
Future/modal/hypothetical: ‘I will/would/can become angry.’
\z

That the present construction only specifies that the event nucleus extends beyond UT, and does not specify if it overlaps with UT, may suggest that the label “present” is incorrect, and that an analysis of this construction as future is more suitable. There are, however, a number of reasons why a present analysis is preferred. Fwe has two future constructions (see \sectref{bkm:Ref463007186}), whose basic criteria are that the nucleus is situated in its entirety after UT: their only possible interpretation is future. This contrasts with the present construction, where overlap with UT is optional, and both future and present interpretations are possible. This difference is illustrated in (\ref{bkm:Ref468114262}--\ref{bkm:Ref74908034}): the present construction in (\ref{bkm:Ref468114262}) can either be interpreted as indicating that the speaker already started working, or that he will start working. The near future construction in (\ref{bkm:Ref74908034}), however, can only indicate that the speaker has not yet started working, but will start working later the same day.

\ea
\label{bkm:Ref468114262}
shùnù \textbf{ndìsèbèzâ}\\
\gll shunu  ndi-sebez-á̲\\
today  \textsc{sm}\textsubscript{1SG}-work-\textsc{fv}\\
\glt ‘Today, I am working. / Today, I will work.’
\z

\ea
\label{bkm:Ref74908034}
shùnù \textbf{mbòndísèbèzê}\\
\gll shunu  mbo-ndí̲-sebez-é̲\\
today  \textsc{near}.\textsc{fut}-\textsc{sm}\textsubscript{1SG}-work-\textsc{pfv}.\textsc{sbjv}\\
\glt ‘Today, I will work.’ (NF\_Elic15)
\z

Another reason to analyze the present construction as present, even though it can also carry futurate meaning, is that overlap with UT, although optional, does appear to be implied. In contexts where different interpretations are possible, speakers usually interpret the use of dynamic verbs in the present construction as present, rather than future. A third argument for the analysis of the present construction as present is economy; if this construction were analyzed as future, Fwe would have three futures, and no present.

\tabref{tab:8:4} gives an overview of the different interpretations of the present construction, and the lexical aspectual classes with which they are available.

\begin{table}
\label{bkm:Ref467673480}\caption{\label{tab:8:4}Interpretations of the present construction with different lexical aspects}

\begin{tabularx}{\textwidth}{llQ}
\lsptoprule
Interpretation & Lexical aspect & Example\\
\midrule
present progressive & dynamic, stative & \textit{tùryâ} ‘we are eating’\\
futurate & all lexical aspects & \textit{ndìsèbèzâ} ‘I will work’\\
modal & dynamic, change-of-state & \textit{ndìtwâ} ‘I can pound’

ndìbêngà ‘I would become angry’\\
conditional & change-of-state & \textit{ònúnà} ‘(If X), you’d become fat’\\
generic & all lexical aspects & \textit{zìtììzâ} ‘they are dangerous’\\
\lspbottomrule
\end{tabularx}
\end{table}

I will now discuss and illustrate the different interpretations of the present construction in more detail. The present progressive interpretation, where the event nucleus overlaps with an extends beyond utterance time, is illustrated with the dynamic verbs \textit{rí} ‘eat’ in (\ref{bkm:Ref98834031}), and \textit{kánan} ‘argue’ in (\ref{bkm:Ref98834032}).

\ea
\label{bkm:Ref98834031}
\textbf{tùry’} ónkûkù òzyò ndáꜝyáyì\\
\gll tu-ri-á̲  o-∅-nkúku    o-zyo    ndí̲-a-ya-í̲\\
\textsc{sm}\textsubscript{1PL}-eat-\textsc{fv}  \textsc{aug}-\textsc{np}\textsubscript{1a}-chicken  \textsc{aug}-\textsc{dem}.\textsc{iii}\textsubscript{1}  \textsc{sm}\textsubscript{1SG}-\textsc{pst}-kill-\textsc{npst}.\textsc{pfv}\\
\glt ‘We are eating the chicken that I killed.’ (ZF\_Elic14)
\z

\ea
\label{bkm:Ref98834032}
zìnjí ꜝ\textbf{múkànàná}\\
\gll ∅-zì-njí    mú̲-ka\textsubscript{H}nan-á̲\\
\textsc{cop}-\textsc{np}\textsubscript{8}-what  \textsc{sm}\textsubscript{2PL}.\textsc{rel}-argue-\textsc{fv}\\
\glt ‘What are you arguing about?’ (asked of a group of people who are currently having an argument) (NF\_Elic15)
\z

The futurate interpretation of dynamic verbs in the present is illustrated in (\ref{bkm:Ref477186128}--\ref{bkm:Ref477186129}).

\ea
\label{bkm:Ref477186128}
\textbf{ndìùtwá} shùnù\\
\gll ndi-u\textsubscript{H}-tw-á̲      shunu\\
\textsc{sm}\textsubscript{1SG}-\textsc{om}\textsubscript{3}-pound-\textsc{fv}  today\\
\glt ‘I’ll pound it today.’ (speaking about maize, the speaker is asked if she plans to pound it today) (NF\_Elic15)
\z

\ea
\label{bkm:Ref477186129}
èmwíkí ꜝíkêːzyà \textbf{ndìsèbèzâ}\\
\gll e-N-mwikí    í̲-ké̲ːzy-a    ndi-sebez-á̲\\
\textsc{aug}-\textsc{np}\textsubscript{9}-week  \textsc{sm}\textsubscript{9}.\textsc{rel}-come-\textsc{fv}  \textsc{sm}\textsubscript{1SG}-work-\textsc{fv}\\
\glt ‘Next week, I’ll work.’ (NF\_Elic15)
\z

The present construction can be used interchangeably with the remote future construction, as in (\ref{bkm:Ref468117989}--\ref{bkm:Ref75248495}): the present form and the remote future form were considered equivalent to express future reference (see \sectref{bkm:Ref443303356}). This interchangeability is not reversible, however: whereas present constructions can have remote future reference, remote future constructions were not accepted with present reference.

\ea
\label{bkm:Ref468117989}
tùkàbòòrá zyônà\\
\gll tu-ka-boor-á̲      zyóna\\
\textsc{sm}\textsubscript{1PL}-\textsc{dist}-return-\textsc{fv}  tomorrow\\
\glt ‘We will return tomorrow.’
\z

\ea
\label{bkm:Ref75248495}
twáràkàbòòrà zyônà\\
\gll tu-ára-ka-boor-a      zyóna\\
\textsc{sm}\textsubscript{1PL}-\textsc{rem}.\textsc{fut}-\textsc{dist}-return-\textsc{fv}  tomorrow\\
\glt ‘We will return tomorrow.’ (NF\_Elic15)
\z

The use of the present construction for remote future (tomorrow and later) events is also possible without an overt time adverbial, as in (\ref{bkm:Ref494976898}), which is a speaker’s response to the question why he cannot come to work tomorrow; his statement therefore refers to his plans for the next day, a\-l\-though he does not use \textit{zyônà} ‘tomorrow’.

\ea
\label{bkm:Ref494976898}
ndìyá kùrùwà\\
\gll ndi-y-á̲  ku-ru-wa\\
\textsc{sm}\textsubscript{1SG}-go-\textsc{fv}  \textsc{np}\textsubscript{17}-\textsc{np}\textsubscript{11}-field\\
\glt ‘[Because] I will go to the field.’ (NF\_Elic15)
\z

Interestingly, interchangability between the present and near future was not observed. In elicitation contexts, present constructions were frequently offered as alternatives to remote future constructions, but never as alternatives to near future constructions. When asked, most speakers considered them acceptable, though they preferred near future constructions. Present constructions with near future reference were only encountered in natural texts, and even there near future reference is more commonly expressed by near future constructions.

That the present construction is more easily interchanged with the remote future construction, rather than the near future construction, may seem counterintuitive, as near future describes event situated closer to the time of speaking than remote future. A possible explanation for the interchangeability of the present and remote future constructions is that the remote future derives from an earlier present construction. The Namibian Fwe remote future is marked by a post-initial prefix \textit{(á)ra-}. In two Bantu Botatwe languages, Zambian Totela and Tonga, a prefix \textit{la-} is used as a marker of present tense (\citealt{Carter2002}: 45; \citealt{Crane2011}: 173-176). The present tense can also be marked with a zero prefix: \textit{la-} marks a disjunct, which is used for predicate focus, and zero marks a conjunct, which is used for argument focus (see {\citet{WalHyman2017}}, and other chapters in the same volume on the conjoint/disjoint distinction in Bantu). If this is the older situation - as suggested by the fact that *da- is reconstructed as a disjunct present for Proto-Bantu (\citealt{Güldemann2003}: 344; \citealt{Meeussen1967}: 109) - Fwe would have reanalyzed the former disjunct present as a remote future, and the former conjunct present as a present. The interchangeability of the remote future marked with \textit{ára-}, presumably cognate with the marker \textit{la-} as used in Totela and Tonga, with the present construction may be a relic of this older system.

Similar to their future interpretation, dynamic verbs in the present construction may also receive a modal interpretation, as in (\ref{bkm:Ref75249269}--\ref{bkm:Ref75249272}).

\ea
\label{bkm:Ref75249269}
èzí zìzwâtò zìcípîtè kònó \textbf{zìrìfwírà} búryò\\
\gll e-zí    zi-zwáto  zi-cip-í̲te konó  zi-ri\textsubscript{H}-fw-í̲r-a    bu-ryó\\
\textsc{aug}-\textsc{dem}.\textsc{i}\textsubscript{8}  \textsc{np}\textsubscript{8}-cloth  \textsc{sm}\textsubscript{8}-be\_cheap-\textsc{stat}
but  \textsc{sm}\textsubscript{8}-\textsc{refl}-die-\textsc{appl}-\textsc{fv}  \textsc{np}\textsubscript{14}-only\\
\glt ‘These clothes are cheap, but they won’t last long (lit. ‘they will just die’).’
\z

\ea
kùfwèbà \textbf{kùrèːtèrá} màrwáꜝrírà\\
\gll ku-fweba  ku-re\textsubscript{H}ːt-er-á̲  ma-rwárirá\\
\textsc{np}\textsubscript{15}-smoke  \textsc{sm}\textsubscript{15}-carry-\textsc{appl}-\textsc{fv}  \textsc{np}\textsubscript{6}-disease\\
\glt ‘Smoking can cause disease.’
\z

\ea
mùndárè \textbf{ndìùtwâ}\\
\gll mu-ndaré  ndi-u\textsubscript{H}-tw-á̲\\
\textsc{np}\textsubscript{3}-maize  \textsc{sm}\textsubscript{1SG}-\textsc{om}\textsubscript{3}-pound-\textsc{fv}\\
\glt ‘Maize, I can pound it.’
\z

\ea
\label{bkm:Ref75249272}
\textbf{ndìmùná} èŋòmbè zíngîː\\
\gll ndi-mun-á̲    e-N-ŋombe    zí-ngíː\\
\textsc{sm}\textsubscript{1SG}-own-\textsc{fv}  \textsc{aug}-\textsc{np}\textsubscript{10}-cow  \textsc{pp}\textsubscript{10}-many\\
\glt ‘I want to own many cattle.’ (NF\_Elic15)
\z

I now turn to the interpretation of change-of-state verbs in the present construction. As shown in (\ref{bkm:Ref75248605}), the only possible interpretation of change-of-state verbs in the present is one that situates the nucleus after the time of speaking, i.e. a futurate or modal interpretation. More examples of this use of the present are given in (\ref{bkm:Ref441844899}--\ref{bkm:Ref75248624}).

\ea
\label{bkm:Ref441844899}
\glll ndìbêngà\\
ndi-bé̲ng-a\\
\textsc{sm}\textsubscript{1SG}-become\_angry-\textsc{fv}\\
\glt ‘I would/will become angry.’ *‘I am becoming angry.’
\z

\ea
\glll ndìrèmánà\\
ndi-reman-á̲\\
\textsc{sm}\textsubscript{1SG}-become\_injured-\textsc{fv}\\
\glt ‘I would/will become injured.’
\z

\ea
mwínì \textbf{ùkwàtìwâ}\\
\gll mw-íni  u-kwa\textsubscript{H}t-iw-á̲\\
\textsc{np}\textsubscript{3}-handle  \textsc{sm}\textsubscript{3}-grab-\textsc{pass}-\textsc{fv}\\
\glt ‘A handle can be grabbed.’
\z

\ea
\label{bkm:Ref75248624}
èmpótó \textbf{ìbbámúkà}\\
\gll e-N-potó  i-bbam-uk-á̲\\
\textsc{aug}-\textsc{np}\textsubscript{9}-pot  \textsc{sm}\textsubscript{9}-break-\textsc{sep}.\textsc{intr}-\textsc{fv}\\
\glt ‘A pot can/might break.’ (uttered as a warning to someone who is handling a pot carelessly) (NF\_Elic15)
\z

Linked to their modal interpretation in main clauses, change-of-state verbs in the present construction are also often used in the apodosis of a factual conditional, expressing an event that will come to pass if certain conditions are met, as in (\ref{bkm:Ref99094967}--\ref{bkm:Ref99094968}).

\ea
\label{bkm:Ref99094967}
òshìryá câhà \textbf{ònúnà}\\
\gll o-shi\textsubscript{H}-ri-á̲    cáha  o-nun-á̲\\
\textsc{sm}\textsubscript{2SG}-\textsc{cond}-eat-\textsc{fv}  very  \textsc{sm}\textsubscript{2SG}-become\_fat-\textsc{fv}\\
\glt ‘When you eat too much, you become fat.’
\z

\ea
òwú mùndárè kùté tùùhíkè \textbf{ùbìzwâ}\\
\gll o-ú    mu-ndaré  kuté  tu-u\textsubscript{H}-hi\textsubscript{H}k-é̲      u-bizw-á̲\\
\textsc{aug}-\textsc{dem}.\textsc{i}\textsubscript{3}  \textsc{np}\textsubscript{3}-maize  if  \textsc{sm}\textsubscript{1PL}-\textsc{om}\textsubscript{3}-cook-\textsc{pfv}.\textsc{sbjv}  \textsc{sm}\textsubscript{3}-ripen-\textsc{fv}\\
\glt ‘This maize, if we cook it, will it be done?’ (NF\_Elic15)
\z

\ea
\label{bkm:Ref99094968}
òshìpángá bùtì \textbf{tùzwírà} hábùsò\\
\gll o-shi\textsubscript{H}-pang-á̲  bu-ti tu-zw-í̲r-a      há-bu-so\\
\textsc{sm}\textsubscript{2SG}-\textsc{cond}-do-\textsc{fv}  \textsc{np}\textsubscript{14}-like\_this
\textsc{sm}\textsubscript{1PL}-come\_out-\textsc{appl}-\textsc{fv}  \textsc{np}\textsubscript{16}-\textsc{np}\textsubscript{14}-front\\
\glt ‘If you do it like this, we will make a profit.’ (ZF\_Conv13)
\z

Change-of-state verbs can be divided into those with and without an onset phase. This distinction is relevant in, for instance, the interpretation of the progressive (see \sectref{bkm:Ref431917333}), the inceptive (see \sectref{bkm:Ref445905588}), and the locative pluractional (see \sectref{bkm:Ref494442567}). In the present construction, however, the future, modal or hypothetical interpretation is the only possible reading for change-of-state verbs, both with an onset phase, such as \textit{bomb} ‘become wet’ in (\ref{bkm:Ref75248965}), and without an onset phase, such as \textit{aruk} ‘open’ in (\ref{bkm:Ref75248967}). This shows that the present construction specifically targets the nucleus, and not the onset phase.

\ea
\label{bkm:Ref75248965}
òmvúrà àshìshókà èvú rìbômbà\\
\gll o-∅-mvúra    a-shi\textsubscript{H}-sho\textsubscript{H}k-á̲ e-∅-vú    ri-bó̲mb-a \\
\textsc{aug}-\textsc{np}\textsubscript{1a}-rain  \textsc{sm}\textsubscript{1}-\textsc{cond}-fall-\textsc{fv}
\textsc{aug}-\textsc{np}\textsubscript{5}-ground  \textsc{sm}\textsubscript{5}-become.wet-\textsc{fv}\\
\glt ‘If it rains, the ground becomes wet.’
\z

\ea
\label{bkm:Ref75248967}
cíàzò cìàrúkà \\
\gll cí-azo    ci-ar-uk-á̲\\
\textsc{np}\textsubscript{7}-door  \textsc{sm}\textsubscript{7}-close-\textsc{sep}.\textsc{intr}-\textsc{fv}\\
\glt ‘A door can open.’ *A door is opening. (NF\_Elic15)
\z

\begin{sloppypar}
Perception verbs, such as \textit{bón} ‘see’ and \textit{shuw} ‘hear, feel’, also function as change-of-state verbs; the use of the present construction gives them a modal, future, or conditional interpretation, not a present ongoing interpretation, as in (\ref{bkm:Ref99095031}--\ref{bkm:Ref99095033}); a present interpretion can only be achieved with the stative (see \sectref{bkm:Ref431984198}).
\end{sloppypar}

\ea
\label{bkm:Ref99095031}
\glll ndìbónà\\
ndi-bo\textsubscript{H}n-á̲\\
\textsc{sm}\textsubscript{1SG}-see-\textsc{fv}\\
\glt ‘I can see.’ *I see.
\z

\ea
\label{bkm:Ref99095033}
\glll ndìshùwâ\\
ndi-shu\textsubscript{H}-á̲\\
\textsc{sm}\textsubscript{1SG}-hear-\textsc{fv}\\
\glt ‘I can hear.’ *I hear. (NF\_Elic17)
\z

Stative verbs, which refer to a single, unbounded and lasting state, are used in the present construction to express a state that holds at the time of speaking, as in (\ref{bkm:Ref99095058}--\ref{bkm:Ref99095059}). Because the state referred to by a stative verb is unbounded, it automatically precedes, follows and overlaps with UT.

\ea
\label{bkm:Ref99095058}
\glll kùshàkàhárà\\
ku-shak-ahar-á̲\\
\textsc{sm}\textsubscript{15}-need-\textsc{neut}-\textsc{fv}\\
\glt ‘It is necessary.’ (NF\_Elic15)
\z

\ea
\label{bkm:Ref99095059}
\glll zìtìyìzâ\\
zi-ti\textsubscript{H}iz-á̲\\
\textsc{sm}\textsubscript{8}-be\_dangerous-\textsc{fv}\\
\glt ‘They are dangerous.’ (NF\_Elic15)
\z

The present construction can also be used with a generic/gnomic interpretation, e.g. a statement that is generally true, independent of whether the action is happening at the time of speaking. This interpretation is available with all lexical aspectual classes, as illustrated for change-of-state verbs in (\ref{bkm:Ref477191357}), for stative verbs in (\ref{bkm:Ref477191359}), and for dynamic verbs in (\ref{bkm:Ref477191361}--\ref{bkm:Ref477191362}).

\ea
\label{bkm:Ref477191357}
bàkêntù bàzwátà zìkócì\\
\gll ba-kéntu  ba-zwá̲t-a  zi-kocí\\
\textsc{np}\textsubscript{2}-woman  \textsc{sm}\textsubscript{2}-wear-\textsc{fv}  \textsc{np}\-\textsubscript{8}-skirt\\
\glt ‘Women wear skirts.’
\z

\ea
\label{bkm:Ref477191359}
òngwè cìbàtànà cítììzâ\\
\gll o-∅-ngwe    ∅-ci-batana    ci-ti\textsubscript{H}iz-á̲\\
\textsc{aug}-\textsc{np}\textsubscript{1a}-leopard  \textsc{cop}-\textsc{np}\textsubscript{7}-predator  \textsc{sm}\textsubscript{7}.\textsc{rel}-be\_fearsome-\textsc{fv}\\
\glt ‘A leopard is a fearsome predator.’ (ZF\_Elic\_13)
\z

\ea
\label{bkm:Ref477191361}
cìzyùnì cìntù cíùrúkà\\
\gll ci-zyuni  ∅-ci-ntu    cí̲-uruk-á̲\\
\textsc{np}\textsubscript{7}-bird  \textsc{cop}-\textsc{np}\-\textsubscript{7}-thing  \textsc{sm}\textsubscript{7}.\textsc{rel}-fly-\textsc{fv}\\
\glt ‘A bird is something that flies.’ (NF\_Elic15)
\z

\ea
\label{bkm:Ref477191362}
ècíkwꜝámè cámꜝárì cìyéndà mbómwêzì\\
\gll e-cí-kwáme    ci-á=mári    ci-é̲nd-a  mbó-mu-ézi\\
\textsc{aug}-\textsc{np}\textsubscript{7}-man  \textsc{pp}\textsubscript{7}-\textsc{con}=polygamy  \textsc{sm}\textsubscript{7}-go-\textsc{fv}  \textsc{adv}-\textsc{np}\textsubscript{3}-moon\\
\glt ‘A polygamous man walks like the moon.’ (saying)\footnote{This saying compares the behavior of a man with two wives to that of the moon. Like the moon travels across the sky each month, from one star to the other, so does the polygamous man regularly travel from one wife to the other.} (NF\_Elic15)
\z

The wide variety of possible interpretations of the present construction can be narrowed by combining it with overt aspectual markers, such as those marking progressive aspect (see \sectref{bkm:Ref445905308}). Present progressive constructions can only be interpreted as an action currently in progress; the modal or futurate interpretation is not seen with the present progressive. Compare the aspectually unmarked present in (\ref{bkm:Ref468110497}) with the present progressive in (\ref{bkm:Ref72239027}--\ref{bkm:Ref72239029}). The bare present leaves uncertainty as to whether they are currently busy milking; as explained by one speaker, it triggers the question: ‘Are they milking now, or will they do it later?’ The present progressive forms in (\ref{bkm:Ref72239027}--\ref{bkm:Ref72239029}) leave no such uncertainty; the only interpretation is that they are currently busy milking.

\ea
\label{bkm:Ref468110497}
\glll bàkámà\\
ba-ka\textsubscript{H}m-á̲\\
\textsc{sm}\textsubscript{2}-milk-\textsc{fv}\\
\glt ‘They are milking. / They will milk.’
\z

\ea
\label{bkm:Ref72239027}
kùkámà ꜝbákámà\\
\gll ku-kám-a  bá̲-ka\textsubscript{H}m-á̲\\
\textsc{inf}-milk-\textsc{fv}  \textsc{sm}\textsubscript{2}.\textsc{rel}-milk-\textsc{fv}\\
\glt ‘They are milking.’
\z

\ea
\label{bkm:Ref72239029}
bàkwèsì bàkámà\\
\gll ba-kwesi  ba-ka\textsubscript{H}m-á̲\\
\textsc{sm}\textsubscript{2}-\textsc{prog}  \textsc{sm}\textsubscript{2}-milk-\textsc{fv}\\
\glt ‘They are milking.’ (NF\_Elic15)
\z

Present progressives are interpreted as having a certain duration, whereas bare present verbs have no implications about duration. This difference is illustrated in (\ref{bkm:Ref463005140}--\ref{bkm:Ref72239109}): unlike the bare present in (\ref{bkm:Ref463005140}), the present progressive in (\ref{bkm:Ref72239109}) suggests that s/he has been knocking for a long time.

\ea
\label{bkm:Ref463005140}
\textbf{àngòngòtá} hàcíàzò mbítà mùntù shàkàmúꜝtábè\\
\gll a-ngo\textsubscript{H}ngot-á̲  ha-cí-azo mbíta  mu-ntu  shaká  a-mú-tab-é̲ \\
\textsc{sm}\textsubscript{1}-knock-\textsc{fv}  \textsc{np}\textsubscript{16}-\textsc{np}\textsubscript{7}-door
until  \textsc{np}\textsubscript{1}-person  if  \textsc{sm}\textsubscript{1}-\textsc{om}\textsubscript{1}-answer-\textsc{pfv}.\textsc{sbjv}\\
\glt ‘S/he is knocking on the door until someone answers.’
\z

\ea
\label{bkm:Ref72239109}
\textbf{àkwèsì} \textbf{àngòngòtá} hàcíàzò mbítà mùntù shàk’ ámúꜝtábè\\
\gll a-kwesi  a-ngo\textsubscript{H}ngot-á̲  ha-cí-azo mbíta  mu-ntu  shaká  a-mú-tab-é̲ \\
\textsc{sm}\textsubscript{1}-\textsc{prog}  \textsc{sm}\textsubscript{1}-knock-\textsc{fv}  \textsc{np}\textsubscript{16}-\textsc{np}\textsubscript{7}-door
until  \textsc{np}\textsubscript{1}-person  if  \textsc{sm}\textsubscript{1}-\textsc{om}\textsubscript{1}-answer-\textsc{pfv}.\textsc{sbjv}\\
\glt ‘S/he is knocking on the door until someone answers.’ (implies that s/he has been knocking for a long time) (NF\_Elic15)
\z

The difference between the present progressive and aspectually unmarked present also relates to modality. With the present progressive, the speaker expresses certainty that the event is taking place at UT, but the aspectually unmarked present may leave more doubt about whether the action fully overlaps with UT. This contrast is illustrated in (\ref{bkm:Ref467676720}--\ref{bkm:Ref72239269}), which both answer the question: ‘Where is that person?’. In (\ref{bkm:Ref72239269}), the aspectually unmarked present is used to imply that the person is supposed to wash dishes, but may at this very moment be busy with something else. In (\ref{bkm:Ref467676720}), the use of a present progressive implies that the person referred to is currently, without a doubt, busy washing dishes.

\ea
\label{bkm:Ref72239269}
mùnjúù wèná \textbf{àsànz’} ótùsûbà\\
\gll mu-N-júo    a-in-á    a-sanz-á̲  o-tu-súba\\
\textsc{np}\textsubscript{18}-\textsc{np}\textsubscript{9}-house  \textsc{sm}\textsubscript{1}-be\_at-\textsc{fv}  \textsc{sm}\textsubscript{1}-wash-\textsc{fv}  \textsc{aug}-\textsc{np}\textsubscript{13}-dish\\
\glt ‘S/he is in the house, s/he is washing dishes.’ (it is not certain that s/he is washing dishes; s/he is supposed to wash dishes but maybe s/he is currently doing something else)
\z

\ea
\label{bkm:Ref467676720}
mùnjúù wèná \textbf{àkwès’} \textbf{àsànz’} ótùsûbà\\
\gll mu-N-júo    a-iná    a-kwesi  a-sanz-á̲  o-tu-súba\\
\textsc{np}\textsubscript{18}-\textsc{np}\textsubscript{9}-house  \textsc{sm}\textsubscript{1}-be\_at  \textsc{sm}\textsubscript{1}-\textsc{prog}  \textsc{sm}\textsubscript{1}-wash-\textsc{fv}  \textsc{aug}-\textsc{np}\textsubscript{13}-dish\\
\glt ‘S/he is in the house, s/he is washing dishes.’ (NF\_Elic15)
\z

Another aspectual marker that may combine with the present is the post-initial persistive prefix \textit{shí-} (see \sectref{bkm:Ref445905502}). The persistive usually expresses an event that started before, and is still ongoing at utterance time, but combined with the present construction, may also express an event that started before, and will continue later, but has been paused at the exact time of speaking. In (\ref{bkm:Ref431315122}), the present is used with a persistive prefix \textit{shí-} to indicate that the task of pounding is currently interrupted, to be returned to later.

\ea
\label{bkm:Ref431315122}
\glll ndìshìtwâ\\
ndi-shi\textsubscript{H}-tw-á̲\\
\textsc{sm}\textsubscript{1SG}-\textsc{per}-pound-\textsc{fv}\\
\glt ‘I’m still pounding.’ (the speaker is currently taking a break, but intends to resume the task shortly) (NF\_Elic15)
\z

A present persistive can also indicate an action that has not yet started before utterance time, but will start after UT. (\ref{bkm:Ref494964767}) is uttered by a speaker who is the last to enter a room, and is urged to hurry, to which he responds that he still needs to close the door, that is, his closing of the door has not yet started as he utters these words.

\ea
\label{bkm:Ref494964767}
\glll ndìshìcìárà\\
ndi-shi\textsubscript{H}-ci\textsubscript{H}-ar-á̲\\
\textsc{sm}\textsubscript{1SG}-\textsc{per}-\textsc{om}\textsubscript{7}-close-\textsc{fv}\\
\glt ‘I still need to close it.’ (NF\_Elic17)
\z
\section{Past}
\hypertarget{Toc75352680}{}
Fwe has four past constructions, distinguished by degree of remoteness (near/remote) and aspect (perfective/imperfective), as schematized in \tabref{tab:8:5}.

\begin{table}
\label{bkm:Ref486847003}\caption{\label{tab:8:5}Past constructions}
\begin{tabularx}{\textwidth}{lQQ}
\lsptoprule
& Perfective & Imperfective\\
\midrule
near & \textsc{sm}-a/na-B-i & \textsc{sm}-aku-B-a \\
\midrule
& ndàbérêkì & ndàkùbèrèkà \\
& ndi-a-beré̲k-i & ndi-aku-berek-a\\
& \textsc{sm}\textsubscript{1SG}-\textsc{pst}-work-\textsc{npst}.\textsc{pfv} & \textsc{sm}\textsubscript{1SG}-\textsc{pst}.\textsc{ipfv}-work-\textsc{fv}\\
& ‘I worked (earlier today).’ & ‘I was working (earlier today).’\\
\midrule
remote & na/ni-\textsc{sm}-a-B-a & ka-\textsc{sm}-B-a \\
\midrule
& nàndábèrèkà & kàndíbèrékà \\
& na-ndí̲-a-berek-a & ka-ndí̲-berek-á̲\\
& \textsc{rem}-\textsc{sm}\textsubscript{1SG}-\textsc{pst}-work-\textsc{fv} & \textsc{pst}.\textsc{ipfv}-\textsc{sm}\textsubscript{1SG}-work-\textsc{fv}\\
& ‘I worked (before today).’ & ‘I was working/used to work (before today).’\\
\lspbottomrule
\end{tabularx}
\end{table}

All four past constructions situate the event’s nucleus in the past, i.e. before the utterance time. In out-of-the-blue and elicitation contexts, the relevant time domain is the day of speaking, e.g. near pasts are treated as hodiernal (for events that took place earlier the same day) and remote pasts as pre-hodiernal (for events that took place before the day of speaking). With sufficient context, more flexible interpretations are possible.

The four past constructions are also distinguished by aspect: the remote{\slash}near past perfective constructions present an event as a single, completed whole, and do not allow reference to the internal structure of the nucleus. The remote/near past imperfective constructions present the event’s nucleus as more drawn out, and make specific reference to the internal structure of the event’s nucleus. These imperfective past constructions may be combined with affixes or constructions that express a specific subtype of imperfective aspect, such as progressive, habitual, stative, or persistive.

A third variable in the interpretation of past constructions in Fwe is the relevance or continuance of the event’s coda phase at utterance time. Verbs that typically include a coda phase are change-of-state verbs, where the coda phase is the state that is entered into. In the near past perfective, the use of a change-of-state verb typically implies that the resultant coda state still applies at UT. The remote past perfective, in contrast, has no such implicature, and the coda state may persist or not, depending on context. Both imperfective pasts, however, only allow an interpretation where both the nucleus and the coda state are located in the past.

The following four sections discuss each past construction in turn, discussing their temporal, aspectual and pragmatic interpretations.

\subsection{Near past perfective}
\label{bkm:Ref488767483}\hypertarget{Toc75352681}{}\label{bkm:Ref488767759}\label{bkm:Ref488767671}
The near past perfective (NPP) construction has the form \textsc{sm}-a/na-B-i, i.e. making use of a post-initial prefix \textit{a}-/\textit{na-}, and a final vowel suffix -\textit{i}, as illustrated in (\ref{bkm:Ref99963837})

\ea
\label{bkm:Ref99963837}
\glll ndìnàyêndì\\
ndi-na-é̲nd-i\\
\textsc{sm}\textsubscript{1SG}-\textsc{pst}-walk-\textsc{npst}.\textsc{pfv}\\
\glt ‘I walked.’ (ZF\_Elic14)
\z

The prefix \textit{a-}/\textit{na-} is subject to geographical variation and phonological conditioning. In the northernmost varieties of Fwe, the prefix \textit{na-} is strongly preferred, as in (\ref{bkm:Ref99963857}). In central Fwe, \textit{a-} and \textit{na-} are used interchangeably, as in (\ref{bkm:Ref99963868}). In Namibian Fwe, geographically the southernmost variety, \textit{a-} and \textit{na-} are conditioned phonologically. When the vowel preceding the post-initial prefix is /a/, the allomorph \textit{na-} is used, as in (\ref{bkm:Ref99963894}). In all other cases, the form \textit{a-} is used, as in (\ref{bkm:Ref99963896}), and vowel hiatus resolution affects the vowel of the subject marker (see \sectref{bkm:Ref491962181} on vowel hiatus resolution).

\ea
\label{bkm:Ref99963857}
Northern Zambian Fwe\\
ndìnàyêndì\\
\gll ndi-na-é̲nd-i\\
\textsc{sm}\textsubscript{1SG}-\textsc{pst}-walk-\textsc{npst}.\textsc{pfv}\\
\glt ‘I walked.’ (ZF\_Elic14)
\z

\ea
\label{bkm:Ref99963868}
Central Zambian Fwe\\
ndìnàyêndì {\textasciitilde} ndàyêndì\\
\gll ndi-(n)a-é̲nd-i\\
\textsc{sm}\textsubscript{1SG}-\textsc{pst}-walk-\textsc{npst}.\textsc{pfv}\\
\glt ‘I walked.’ (ZF\_Elic13)
\z

\ea
\label{bkm:Ref99963894}
\ea
Namibian Fwe: na- after /a/\\
bànàhúrì\\
\gll ba-na-hur-í̲\\
\textsc{sm}\textsubscript{2}-\textsc{pst}-arrive-\textsc{npst}.\textsc{pfv}\\
\glt ‘They arrived.’

\ex
\glll ànàcôːkì\\
a-na-có̲ːk-i\\
\textsc{sm}\textsubscript{6}-\textsc{pst}-break-\textsc{npst}.\textsc{pfv}\\
\glt ‘They broke.’
\z\z

\ea
\label{bkm:Ref99963896}
\ea
Namibian Fwe: a- elsewhere\\
ndàhúrì\\
\gll ndi-a-hur-í̢\\
\textsc{sm}\textsubscript{1SG}-\textsc{pst}-arrive-\textsc{npst}.\textsc{pfv}\\
\glt ‘I arrived.’

\ex
\glll mwàhúrì\\
mu-a-hur-í̲\\
\textsc{sm}\textsubscript{2PL}-\textsc{pst}-arrive-\textsc{npst}.\textsc{pfv}\\
\glt ‘You arrived.’ (NF\_Elic15)
\z\z

The only exceptions are the second person singular subject marker \textit{o-}, which merges with the past prefix to become \textit{no-}, as in (\ref{bkm:Ref99963922}), and the class 1/1a subject marker 1/1a \textit{a-}, which merges with the past prefix to become \textit{na-}, as in (\ref{bkm:Ref99963923}). This applies to all varieties of Fwe.

\ea
\label{bkm:Ref99963922}
\glll nòhúrì\\
no-hur-í̲\\
\textsc{sm}\textsubscript{2SG}.\textsc{pst}-arrive-\textsc{npst}.\textsc{pfv}\\
\glt ‘You arrived.’
\z

\ea
\label{bkm:Ref99963923}
\glll nàhúrì\\
na-hur-í̲\\
\textsc{sm}\textsubscript{1}.\textsc{pst}-arrive-\textsc{npst}.\textsc{pfv}\\
\glt ‘S/he arrived.’ (NF\_Elic15)
\z

The post-initial prefix \textit{a-} is a past marker, also used in the remote past perfective (see \sectref{bkm:Ref489260766}) and the near past imperfective (see \sectref{bkm:Ref494480139}). The variation between \textit{a-} and \textit{na-} is specific to its use in the near past perfective, however, and is not seen with the remote past perfective and near past imperfective constructions.

The final vowel suffix \textit{-i} is only used in the NPP, not in any other past constructions (its occurrence in the negative present is likely due to accidental homophony), and is therefore glossed as such, using the abbreviation \textsc{npst}.\textsc{pfv}.

The near past perfective suffix cannot be used after a passive suffix -(\textit{i})\textit{w} (see \sectref{bkm:Ref452972446} on the passive); instead, the final vowel suffix \textit{-a} is used, as in (\ref{bkm:Ref99963963}--\ref{bkm:Ref99963964}).

\ea
\label{bkm:Ref99963963}
cìshámú cìnàtémìwà\\
\gll ci-shamú  ci-na-tém-iw-a\\
\textsc{np}\textsubscript{7}-tree  \textsc{sm}\textsubscript{7}-\textsc{pst}-chop-\textsc{pass}-\textsc{fv}\\
\glt ‘The tree was chopped.’
\z

\ea
\label{bkm:Ref99963964}
zònshéː zìzyùnì zàzwísìwà\\
\gll z-onshéː  zi-zyuni  zi-a-zw-í̲s-iw-a\\
\textsc{pp}\textsubscript{8}-all    \textsc{np}\textsubscript{8}-bird  \textsc{sm}\textsubscript{8}-\textsc{pst}-leave-\textsc{caus}-\textsc{pass}-\textsc{fv}\\
\glt ‘All the birds have been removed.’ (ZF\_Elic14)
\z

The past suffix \textit{-i} never causes spirantization of the preceding consonant, as opposed to the agentive suffix \textit{-i}, which causes spirantization in a number of cases (see \sectref{bkm:Ref444251366}), and the stative suffix \textit{\--ite}, where spirantization occurs with a number of allomorphs of the suffix (see \sectref{bkm:Ref431984198}).

Verbs in the NPP take melodic tone 3, a high tone on the second stem syllable, and retain their lexical tones, as illustrated with the toneless verb \textit{yendaur} ‘walk around’ in (\ref{bkm:Ref506216673}).

\ea
\label{bkm:Ref506216673}
\glll ndàyèndáùrì\\
ndi-a-end-á̲-ur-i\\
\textsc{sm}\textsubscript{1SG}-\textsc{pst}-walk-\textsc{pl}1-\textsc{sep}.\textsc{tr}-\textsc{npst}.\textsc{pfv}\\
\glt ‘I walked around.’ (NF\_Elic15)
\z

The NPP situates the nucleus of the event in the recent past with respect to the utterance time. In most contexts, recent past is interpreted as earlier the same day, as in (\ref{bkm:Ref98512885}--\ref{bkm:Ref98512886}).

\ea
\label{bkm:Ref98512885}
shùnù ndàhúrùrì màpùrù\\
\gll shunu  ndi-a-húrur-i        ma-puru\\
today  \textsc{sm}\textsubscript{1SG}-\textsc{pst}-take\_off\_yoke-\textsc{npst}.\textsc{pfv}  \textsc{np}\textsubscript{6}-ox\\
\glt ‘Today I took the yoke off the oxen.’
\z

\ea
\label{bkm:Ref98512886}
àmênjì àyìsâ kàkúrì \textbf{ndàábìrìsì}\\
\gll a-ma-ínji    a-i\textsubscript{H}s-á̲ kakúri    ndi-a-á-bir-is-i \\
\textsc{aug}-\textsc{np}\textsubscript{6}-water  \textsc{sm}\textsubscript{6}-burn-\textsc{fv}
because  \textsc{sm}\textsubscript{1SG}-\textsc{pst}-\textsc{om}\textsubscript{6}-boil-\textsc{caus}-\textsc{npst}.\textsc{pfv}\\
\glt ‘The water is hot, because I (just) boiled it.’ (ZF\_Elic14)
\z

{\citet[93]{Nurse2008}} notes that Bantu languages may differ with respect to the interpretation of time reference as fixed or flexible. In Fwe, flexible interpretations seem possible; events that are perceived to be in the same time cycle can be conceived as hodiernal, and events that are perceived to be in a previous time cycle can be conceived as prehodiernal. The “same time cycle” can be construed as larger than the day of speaking, for instance, as the year (which includes the day of speaking), as in (\ref{bkm:Ref402270667}), where the NPP is used for an event that took place earlier the same year, although it took place before the day of speaking.

\ea
\label{bkm:Ref402270667}
cìnó cìrìmò ndìnàshínjì wâwà\\
\gll cinó    ci-rimo  ndi-na-shínj-i      wáwa\\
\textsc{dem}.\textsc{ii}\textsubscript{7}  \textsc{np}\textsubscript{7}-year  \textsc{sm}\textsubscript{1SG}-\textsc{pst}-harvest-\textsc{npst}.\textsc{pfv}  very\\
\glt ‘This year, I had a good harvest.’ (ZF\_Elic14)
\z

The NPP can also be used to express surprise. The use of the NPP in (\ref{bkm:Ref486863328}) does not imply that the event of becoming rich happened earlier the same day, but that the event of becoming rich was unexpected and sudden, for instance, someone won a jackpot, or was given 50 heads of cattle.

\ea
\label{bkm:Ref486863328}
\glll nàfúmì\\
na-fum-í̲\\
\textsc{sm}\textsubscript{1}.\textsc{pst}-become\_rich-\textsc{npst}.\textsc{pfv}\\
\glt ‘S/he has become rich (suddenly/unexpectedly).’ (NF\_Elic17)
\z

Similarly, the use of the NPP in (\ref{bkm:Ref486863331}) has two possible interpretations: either that the subject got married earlier the same day, or that the subject got married before the day of speaking, but that his marriage was secret and has been recently revealed.

\ea
\label{bkm:Ref486863331}
nàshêshì\\
na-shésh-i\\
\textsc{sm}\textsubscript{1}.\textsc{pst}-marry-\textsc{npst}.\textsc{pfv}\\
1. ‘He got married (earlier today).’\\
2. ‘He got married (before today, but I discovered it recently).’ (NF\_Elic17)
\z

The use of the near past perfective to express that an event is sudden, surprising, or unexpected, may be a pragmatic extension of its recent past semantics: by situating an event closer to the utterance time, the speaker is highlighting its unexpectedness.

Aspectually, the NPP presents the nucleus of the event as a single, complete whole, without reference to its internal structure. That the internal structure of the nucleus cannot be referenced is seen when an NPP verb is combined with a verb in the consecutive form (cf. \sectref{bkm:Ref494204746}), as in (\ref{bkm:Ref494881934}), where the NPP verb \textit{nàréngì} ‘[lightning] struck’ is followed by a consecutive verb \textit{cóꜝ}\textit{kúyà} ‘and it burnt’. As the NPP presents the event of the lightning striking as perfective, without reference to its internal constituency, the event presented by the consecutive form cannot co-occur with the lightning striking, but is interpreted as occurring after it.

\newpage
\ea
\label{bkm:Ref494881934}
òmvúrà nàréngì cìkúnì cóꜝkúyà\\
\gll o-∅-mvúra    na-réng-i      ci-kuní ci-ó=ku-y-á \\
\textsc{aug}-\textsc{np}\textsubscript{1a}-rain  \textsc{sm}\textsubscript{1}.\textsc{pst}-strike-\textsc{npst}.\textsc{pfv}  \textsc{np}\textsubscript{7}-tree
\textsc{pp}\textsubscript{7}-\textsc{con}=\textsc{inf}-burn-\textsc{fv}\\
\glt ‘The lightning struck the tree, and it burnt.’ (NF\_Elic17)
\z

The perfective nature of the near past perfective is also seen in its interaction with aspectual markers; the NPP does not co-occur with imperfective aspectual forms such as progressives, habituals, and the persistive, nor with the locative pluractional marker, which indicates an event taking place in different locations (see \sectref{bkm:Ref494442567}); as the NPP does not allow reference to the internal structure of the event’s nucleus, it cannot be used with a marker that describes the spatial distribution of the event, as illustrated in (\ref{bkm:Ref494974356}).

\ea
\label{bkm:Ref494974356}
\glll *ndàkàbúyêndì\\
ndi-a-kabú-é̲nd-i\\
\textsc{sm}\textsubscript{1SG}-\textsc{pst}-\textsc{loc}.\textsc{pl}-walk-\textsc{npst}.\textsc{pfv}\\
\glt Intended: ‘I walked around/ in different places.’ (NF\_Elic17)
\z

When the NPP is used with a verb that includes a coda phase, there is a strong implication that this coda phase still holds at UT. The examples in (\ref{bkm:Ref98834712}--\ref{bkm:Ref98834714}) show that, when used without further clarifying context, the default interpretation of the NPP is that the coda phase is still ongoing at UT.

\ea
\label{bkm:Ref98834712}
\glll ndànjômbì\\
nd-a-njó̲mb-i\\
\textsc{sm}\textsubscript{1SG}-\textsc{pst}-get\_stuck-\textsc{npst}.\textsc{pfv}\\
\glt ‘I got stuck (and am still stuck).’ (NF\_Elic17)
\z

\ea
ècí cìpúrà càcôːkì\\
\gll e-cí    ci-purá  ci-a-có̲ːk-i\\
\textsc{aug}-\textsc{dem}.\textsc{i}\textsubscript{7}  \textsc{np}\textsubscript{7}-chair  \textsc{sm}\textsubscript{7}-\textsc{pst}-break-\textsc{npst}.\textsc{pfv}\\
\glt ‘This chair broke (and is still broken).’ (ZF\_Elic13)
\z

\ea
\label{bkm:Ref98834714}
\glll ndàzísânzì\\
ndi-a-zí-sá̲nz-i\\
\textsc{sm}\textsubscript{1SG}-\textsc{pst}-\textsc{om}\textsubscript{8}-wash-\textsc{npst}.\textsc{pfv}\\
\glt ‘I washed them (and they are clean now).’ (NF\_Elic15)
\z

This is also true of the use of the NPP with a change-of-state verb, where it is usually interpreted as a present state, as in (\ref{bkm:Ref99965412}--\ref{bkm:Ref99965413}).

\ea
\label{bkm:Ref99965412}
\glll ndàshwênì\\
ndi-a-shwé̲n-i\\
\textsc{sm}\textsubscript{1SG}-\textsc{pst}-become\_tired-\textsc{npst}.\textsc{pfv}\\
\glt ‘I am tired.’ (ZF\_Elic14)
\z

\ea
\glll ndàǀôsì\\
ndi-a-ǀós-i\\
\textsc{sm}\textsubscript{1SG}-\textsc{pst}-become\_bored-\textsc{npst}.\textsc{pfv}\\
\glt ‘I am bored.’ (NF\_Elic15)
\z

\ea
\label{bkm:Ref99965413}
\glll cànyóngâmì\\
ci-a-nyong-á̲m-i\\
\textsc{sm}\textsubscript{7}-\textsc{pst}-bend-\textsc{imp}.\textsc{intr}-\textsc{npst}.\textsc{pfv}\\
\glt ‘It is bent.’ (NF\_Elic15)
\z

Even though the NPP implies a lasting coda phase, the nuclear phase is also part of the conceptualization: in (\ref{bkm:Ref431466152}), the NPP not only expresses that the handle is broken at the time of speaking, but the earlier breaking of the handle is also conceptualized, as it invites the question: who broke it?

\ea
\label{bkm:Ref431466152}
\ea
mwínì wéhàmbà wàcôːkì\\
\gll mu-íni  u-é=amba    u-a-có̲ːk-i\\
\textsc{np}\textsubscript{3}-handle  \textsc{pp}\textsubscript{3}-\textsc{con}=hoe  \textsc{sm}\textsubscript{3}-\textsc{pst}-break-\textsc{npst}.\textsc{pfv}\\
\glt ‘The handle of the hoe is broken.’

\ex
ndíní nàúcôːrì\\
\gll ndi-ní    na-ú-có̲ːr-i\\
\textsc{cop}-who  \textsc{sm}\textsubscript{1}.\textsc{pst}-\textsc{om}\textsubscript{3}-break-\textsc{npst}.\textsc{pfv}\\
\glt ‘Who broke it?’ (NF\_Elic15)
\z\z

That the earlier change of state is part of the conceptualization of the verb is further supported by the fact that an agent phrase is allowed; this agent phrase provides information about how the earlier change of state came about. In (\ref{bkm:Ref494459378}), the change-of-state verb \textit{bomb} ‘become wet’ is used in the NPP, implying that the clothes are still wet. The earlier change in state, however, namely the moment the clothes became wet, is also conceptualized, and the agent phrase \textit{kúmvûrà} ‘by the rain’ refers to this nuclear phase.

\ea
\label{bkm:Ref494459378}
èzìzwátò zìnàbómbì kúmvûrà\\
\gll e-zi-zwáto    zi-na-bó̲mb-i      kú-∅-mvúra\\
\textsc{aug}-\textsc{np}\textsubscript{8}-cloth  \textsc{sm}\textsubscript{8}-\textsc{pst}-become\_wet-\textsc{npst}.\textsc{pfv}  \textsc{np}\textsubscript{17}-\textsc{np}\textsubscript{1a}-rain\\
\glt ‘The clothes have become wet because of the rain.’ (ZF\_Elic14)
\z

The remote past perfective may also be used to imply a coda state that still holds at UT, but situates the nuclear phase in the remote past, rather than the recent past. Both (\ref{bkm:Ref486931968}) and (\ref{bkm:Ref486931970}) indicate that the speaker is still sick at the time of speaking, but the remote past perfective in (\ref{bkm:Ref486931968}) indicates that the speaker became sick in the remote past, whereas the near past perfective in (\ref{bkm:Ref486931970}) indicates that the speaker became sick in the recent past.

\ea
\label{bkm:Ref486931968}
níndàrwárà zyônà nèshúnù ndìshìrwárîtè\\
\gll ní̲-ndi-a-rwár-a    zyóna    ne=shúnu  ndi-shi\textsubscript{H}-rwa\textsubscript{H}r-í̲te\\
\textsc{pst}-\textsc{sm}\textsubscript{1SG}-\textsc{pst}-be\_sick-\textsc{fv}  yesterday  \textsc{com}=today  \textsc{sm}\textsubscript{1SG}-\textsc{per}-be\_sick-\textsc{stat}\\
\glt ‘I got sick yesterday, and I am still sick today.’ (NF\_Elic17)
\z

\ea
\label{bkm:Ref486931970}
ndàrwârì mwívùmò\\
\gll ndi-a-rwár-i        mú-e-∅-vumo\\
\textsc{sm}\textsubscript{1SG}-\textsc{pst}-be\_sick-\textsc{npst}.\textsc{pfv}  \textsc{np}\textsubscript{18}-\textsc{aug}-\textsc{np}\textsubscript{5}-stomach\\
\glt ‘I got sick to my stomach [this evening].’ (NF\_Narr17)
\z

The NPP also shows some similarities to the stative construction. The stative construction expresses a state that holds at utterance time, but makes no reference to if or when the state has come about (see \sectref{bkm:Ref431984198}). This contrasts with the NPP, where the entering of the state is conceptualized. As such, the NPP, may be used with temporal adverbs referring to the change in state, as in (\ref{bkm:Ref99965496}), but not the stative, as in (\ref{bkm:Ref99965498}).

\ea
\label{bkm:Ref99965496}
Near past perfective\\
\glll èténdè ryángù ryàcóːkì shûnù\\
e-∅-ténde    ri-angú  ri-a-có̲ːk-i      shúnu\\
\textsc{aug}-\textsc{np}\textsubscript{5}-foot  \textsc{pp}\textsubscript{5}-\textsc{poss}\textsubscript{1SG}  \textsc{sm}\textsubscript{5}-\textsc{pst}-break-\textsc{npst}.\textsc{pfv}  today\\
\glt ‘My leg broke today.’
\z

\ea
\label{bkm:Ref99965498}
  Stative\\
*èténdè ryángù rìcókêtè shûnù\\
Intended: ‘My leg broke today.’\footnote{An interpretation where the adverb modifies the current state, e.g. ‘my leg is broken today’, was also not accepted.}
\z

Although the default interpretation of the NPP is that any resulting coda phase still holds at UT, this implication can be canceled. In (\ref{bkm:Ref486936352}), the NPP verb \textit{ndàrwárì} ‘I got sick’ has an implied coda state of being sick, but in this context, the coda state is canceled. Similarly in (\ref{bkm:Ref486936353}), the implied coda state of \textit{ndàzísànzì} ‘I washed them’, namely that the clothes are clean, does not hold at UT.

\ea
\label{bkm:Ref486936352}
ndàrwárì màsíkùsîkù hànú màntêngù shèndìrìshúwírè nênjà\\
\gll ndi-a-rwár-i        ma-síkusíku hanú    ma-nténgu  she-ndi-ri\textsubscript{H}-shu\textsubscript{H}-í̲re  nénja\\
\textsc{sm}\textsubscript{1SG}-\textsc{pst}-be\_sick-\textsc{npst}.\textsc{pfv}  \textsc{np}\textsubscript{6}-morning
\textsc{dem}.\textsc{ii}\textsubscript{6}  \textsc{np}\textsubscript{6}-evening  \textsc{inc}-\textsc{sm}\textsubscript{1SG}-\textsc{refl}-feel-\textsc{stat}  well\\
\glt ‘I got sick this morning, but now in the evening I feel well.’
\z

\ea
\label{bkm:Ref486936353}
ndàzísànzì èzí zìzwátò shûnù hàpé hánù shìzázyùrì túꜝkútà\\
\gll ndi-a-zí-sanz-i      e-zí    zi-zwáto  shúnu hapé  hánu    shi-zi-á-zyur-i        ∅-túkutá\\
\textsc{sm}\textsubscript{1SG}-\textsc{pst}-\textsc{om}\textsubscript{8}-wash-\textsc{npst}.\textsc{pfv}  \textsc{aug}-\textsc{dem}.\textsc{i}\textsubscript{8}  \textsc{np}\textsubscript{8}-cloth  today
again  \textsc{dem}.\textsc{ii}\textsubscript{16}  \textsc{inc}-\textsc{sm}\textsubscript{8}-\textsc{pst}-become\_full-\textsc{npst}.\textsc{pfv}  \textsc{np}\textsubscript{5}-dirt\\
\glt ‘I washed these clothes today, but now they are dirty again.’ (NF\_Elic17)
\z

Other verbs do not include a possible coda phase, but it is possible that the nuclear phase continues to be relevant in some other way. For example, the use of the NPP with the verb \textit{hur} ‘arrive’ in (\ref{bkm:Ref506460571}) implies the continued relevance of the event’s nucleus, namely ‘being in a certain place’.\footnote{That this particular verb lacks a coda phase is seen from its incompatibility with the stative ending \textit{-ite}. The stative ending regularly derives a coda state from verbs where a coda is part of their lexical event structure.}

\ea
\label{bkm:Ref506460571}
òmfûmù kwênà nàhúrì\\
\gll o-mfúmu  kú-a-ina    na-hur-í̲\\
\textsc{aug}-king  \textsc{sm}\textsubscript{17}-\textsc{sm}\textsubscript{1}-be\_at  \textsc{sm}\textsubscript{1}.\textsc{pst}-arrive-\textsc{npst}.\textsc{pfv}\\
\glt ‘The king, he’s here, he has arrived.’ (NF\_Elic17)
\z

The relevant consequences of an event in the NPP are treated in the same way as the post-nuclear coda phase: their relevance is implied, but this implication can be canceled. This is illustrated in (\ref{bkm:Ref506460700}), where the consequences of buying salt, namely having salt, are no longer valid at UT, e.g. the salt is already finished.

\ea
\label{bkm:Ref506460700}
ndàùrí zwâyì kònó shìryàmánì\\
\gll ndi-a-ur-í̲      ∅-zwái  konó  shi-ri-a-man-í̲\\
\textsc{sm}\textsubscript{1SG}-\textsc{pst}-buy-\textsc{npst}.\textsc{pfv}  \textsc{np}\textsubscript{5}-salt  but  \textsc{inc}-\textsc{sm}\textsubscript{5}-\textsc{pst}-finished-\textsc{npst}.\textsc{pfv}\\
\glt ‘I bought salt [earlier today], but [now] it’s already finished.’ (NF\_Elic17)
\z

The implication of the NPP, that the verb’s coda phase or relevance lasts up to the time of speaking, cannot be canceled when the verb is combined with the inceptive prefix: in this case, the verb’s coda phase or relevance are always interpreted as valid at UT. This is illustrated in (\ref{bkm:Ref490743344}), which shows that the NPP with the inceptive implies that the rain is still falling. (\ref{bkm:Ref72242545}) shows that this implication cannot be canceled, and (\ref{bkm:Ref72242546}) shows that it can be canceled when the NPP is used without the inceptive.

\ea
\label{bkm:Ref490743344}
sìnàtángì òkùshôkà\\
\gll si-na-táng-i        o-ku-shók-a\\
\textsc{inc}-\textsc{sm}\textsubscript{1}.\textsc{pst}-start-\textsc{npst}.\textsc{pfv}  \textsc{aug}-\textsc{inf}-rain-\textsc{fv}\\
\glt ‘It has started to rain.’ (and is raining now)
\z

\ea
\label{bkm:Ref72242545}
*sìnàtángì òkùshôkà cwàré sànàkàbûkì\\
\gll si-na-táng-i        o-ku-shók-a cwaré  sa-na-kabú̲k-i \\
\textsc{inc}-\textsc{sm}\textsubscript{1}.\textsc{pst}-start-\textsc{npst}.\textsc{pfv}  \textsc{aug}-\textsc{inf}-rain-\textsc{fv} 
then  \textsc{inc}-\textsc{sm}\textsubscript{1}.\textsc{pst}-stop\_rain-\textsc{npst}.\textsc{pfv}\\
\glt Intended: ‘It started to rain [earlier today], but now it stopped.’
\z

\ea
\label{bkm:Ref72242546}
nàtángì òkùshôkà cwàré sànàkàbûkì\\
\gll na-táng-i      o-ku-shók-a cwaré  sa-na-kabú̲k-i \\
\textsc{sm}\textsubscript{1}.\textsc{pst}-start-\textsc{npst}.\textsc{pfv}  \textsc{aug}-\textsc{inf}-rain-\textsc{fv}
then  \textsc{inc}-\textsc{sm}\textsubscript{1}.\textsc{pst}-stop\_rain-\textsc{npst}.\textsc{pfv}\\
\glt ‘It started to rain [earlier today], but now it stopped.’ (NF\_Elic17)
\z
\subsection{Remote past perfective}
\label{bkm:Ref468174257}\label{bkm:Ref451515182}\hypertarget{Toc75352682}{}\label{bkm:Ref489260766}
The remote past perfective (RPP) construction has the form na/ni-\textsc{sm}-a-B-a, with a pre-initial remoteness prefix \textit{na-}/\textit{ni-}, a post-initial past prefix \textit{a-}, and the default final vowel suffix \textit{-a}. An example of a remote past perfective construction is given in (\ref{bkm:Ref99965736}).

\ea
\label{bkm:Ref99965736}
\glll nàndáshâmbà\\
na-ndí̲-a-shámb-a\\
\textsc{rem}-\textsc{sm}\textsubscript{1SG}-\textsc{pst}-swim-\textsc{fv}\\
\glt ‘I swam.’ (ZF\_Elic14)
\z

The pre-initial prefix exhibits a certain degree of geographical variation. It is realized as \textit{ni-} in Namibian Fwe, as in (\ref{bkm:Ref468885505}). In Zambian Fwe, it is mostly realized as \textit{na-}, as in (\ref{bkm:Ref72242950}), but can also be realized as \textit{ne-}, especially in subordinate clauses, as in (\ref{bkm:Ref72242951}).

\ea
\label{bkm:Ref468885505}
\glll níndàtêmà\\
ní̲-ndi-a-tém-a\\
\textsc{rem}-\textsc{sm}\textsubscript{1SG}-\textsc{pst}-chop-\textsc{fv}\\
\glt ‘I chopped.’ (NF\_Elic15)
\z

\ea
\label{bkm:Ref72242950}
\glll nándàtêkà\\
ná̲-ndi-a-ték-a\\
\textsc{rem}-\textsc{sm}\textsubscript{1SG}-\textsc{pst}-fetch-\textsc{fv}\\
\glt ‘I fetched.’ (ZF\_Elic14)
\z

\ea
\label{bkm:Ref72242951}
kàrí ndìmé nèndáꜝyáyà\\
\gll ka-rí    ndi-mé  ne-ndí̲-a-ya-á̲\\
\textsc{neg}-be  \textsc{cop}-\textsc{pers}\textsubscript{1SG}  \textsc{rem}-\textsc{sm}\textsubscript{1SG}-\textsc{pst}-kill-\textsc{fv}<\textsc{rel}>\\
\glt ‘It wasn’t me who broke it.’ (ZF\_Elic14)
\z

The prefix \textit{na-/ni-/ne-} marks remoteness, selecting a time period that is considered to be far away from the time of speaking. In the case of the remote past perfective, it selects a domain long \textit{before} the time of speaking. The same remoteness prefix is used with the remote future construction, which combines the remoteness prefix with a post-initial prefix \textit{na-} (Zambian Fwe) or \textit{ára-} (Namibian Fwe) (see \sectref{bkm:Ref443303356}): here it selects a domain long \textit{after} the time of speaking. The remoteness prefix is also used with a subjunctive to express a remote future in a subordinate clause (see Chapter \ref{bkm:Ref99112414}), and with any verb in the apodosis of a counterfactual (see \sectref{bkm:Ref491770136}).

The remoteness prefix is left out when the RPP has an experiential reading, expressing an event that has occurred at least once in the indeterminate past, as in (\ref{bkm:Ref99965779}--\ref{bkm:Ref99965780}). This construction differs from the RPP only in the absence of remoteness prefix; it takes the same segmental morphemes and melodic tones as the RPP, suggesting that it functions as a subtype of the RPP.

\ea
\label{bkm:Ref99965779}
ênì ècó ꜝcíryò ndácìryà\\
\gll éni  e-có    ci-ryó    ndí̲-a-ci-ry-a\\
yes  \textsc{aug}-\textsc{dem}.\textsc{iii}\textsubscript{7}  \textsc{np}\textsubscript{7}-food  \textsc{sm}\textsubscript{1SG}-\textsc{pst}-\textsc{om}\textsubscript{7}-eat-\textsc{fv}\\
\glt ‘Yes, this food, I have eaten it.' (Answer to: ‘Have you ever eaten this food?’) (NF\_Elic17)
\z

\ea
nóshàngànà mùkúrù wángù\\
\gll nó̲-shangan-a  mu-kúru  u-angú\\
\textsc{sm}\textsubscript{2SG}.\textsc{pst}-meet-\textsc{fv}  \textsc{np}\textsubscript{1}-brother  \textsc{pp}\textsubscript{1}-\textsc{poss}\textsubscript{1SG}\\
\glt ‘Have you ever met my brother?’ (ZF\_Elic13)
\z

\ea
\label{bkm:Ref99965780}
kàrí ndáyà mòwín’ ómùnzì\\
\gll ka-ri    ndí̲-a-y-a    mo-winá  o-mu-nzi\\
\textsc{neg}-be  \textsc{sm}\textsubscript{1SG}-\textsc{pst}-go-\textsc{fv}  \textsc{np}\textsubscript{18}-\textsc{dem}.\textsc{iv}\textsubscript{3}  \textsc{aug}-\textsc{np}\textsubscript{3}-village\\
\glt ‘I’ve never been to that village.’ (NF\_Elic15)
\z

The post-initial prefix \textit{a-} used in the RPP is a past marker; it is also seen in the near past imperfective, as part of the post-initial prefix \textit{aku-} (see \sectref{bkm:Ref451515182}), and in the near past perfective, where it combines with a suffix \textit{-i} (see \sectref{bkm:Ref488767483}). The post-initial prefix \textit{a-} of the remote past perfective is not completely identical to the post-initial prefix \textit{a-} of the near past perfective, however, because near past perfective \textit{a-} has an allomorph \textit{na-}, which is not seen with remote past perfective \textit{a-}.

Verbs in the RPP retain their underlying tones, combined with melodic tone 2, which is assigned to the subject marker. When the verb root has a lexical high tone, such as the verb \textit{shótok} ‘jump’ in (\ref{exampleihavejumped}), the prefix \textit{ni-/ne-/na-} is also realized with a high tone. The adjacency of the high tone of \textit{ni-/ne-/na-} to the high tone on the subject marker causes the second high tone to be deleted as a result of Meeussen’s Rule (see \sectref{bkm:Ref440987952}). When used with a toneless verb root, such as \textit{zibar} ‘forget’ in (\ref{bkm:Ref100140406}), the prefix \textit{ni-/ne-/na-} is not realized with a high tone, in which case the high tone of the subject marker is also not deleted.

\ea
\label{exampleihavejumped}
\glll nándàshótòkà\\
ná̲-ndí̲-a-shótok-a > ná-ndi-a-shótok-a \\
\textsc{rem}-\textsc{sm}\textsubscript{1SG}-\textsc{pst}-jump-\textsc{fv}\\
\glt ‘I have jumped.’ (ZF\_Elic14)
\z

\ea
\label{bkm:Ref100140406}
\glll nàndázìbàrà\\
na-ndí̲-a-zibar-a\\
\textsc{rem}-\textsc{sm}\textsubscript{1SG}-\textsc{pst}-forget-\textsc{fv}\\
\glt ‘I have forgotten.’ (ZF\_Elic14)\label{bkm:Ref466391002}
\z

Temporally, the RPP situates the nucleus of the event in the remote past with respect to utterance time. In most cases, remote past is interpreted as any time before the day of speaking, such as yesterday in (\ref{bkm:Ref431382059}); more than fifty years ago in (\ref{bkm:Ref431459411}); a few months ago in (\ref{bkm:Ref431382075}), which is the conclusion of a story about an elephant attack that happened a few months before.

\ea
\label{bkm:Ref431382059}
nìbáhùrà zyônà\\
\gll ni-bá̲-a-hur-a    zyóna\\
\textsc{rem}-\textsc{sm}\textsubscript{2}-\textsc{pst}-arrive-\textsc{fv}  yesterday\\
\glt ‘They arrived yesterday.’ (NF\_Elic15)
\z

\ea
\label{bkm:Ref431459411}
êmè nándàréːtìwà kánàìntìnsíkìsitì\\
\gll eme    ná̲-ndi-a-réːt-iw-a        ká-naintinsíkisiti\\
\textsc{pers}\textsubscript{1SG}  \textsc{rem}-\textsc{sm}\textsubscript{1SG}-\textsc{pst}-give\_birth-\textsc{pass}-\textsc{fv}  at-1960\\
\glt ‘I was born in 1960.’ (ZF\_Narr15)
\z

\ea
\label{bkm:Ref431382075}
mbóbùryâhò nìyápàngàhàrírà\\
\gll mbó-bu-riáho    ni-í̲-a-pang-ahar-ir-á̲\\
\textsc{cop}.\textsc{def}\textsubscript{14}-\textsc{np}\textsubscript{14}-like\_that  \textsc{rem}-\textsc{sm}\textsubscript{9}-\textsc{pst}-do-\textsc{neut}-\textsc{appl}-\textsc{fv}<\textsc{rel}>\\
\glt ‘That is how it happened.’ (ZF\_Narr15)
\z

The RPP may also contrast time units larger than the day of speaking, such as the year; in (\ref{bkm:Ref506366838}), the speaker is contrasting this year’s farming activities with those of the previous year.

\ea
\label{bkm:Ref506366838}
cìrìmò cíkêːzyà nàndínàkúná màyìrà cìrìmò nàcámànà mùndáré \textbf{nàndáꜝ}\textbf{kúnà}\\
\gll ci-rimo  cí-kéːzy-a    na-ndí-na-kun-á    ma-ira ci-rimo  na-cí-a-man-a N-mu-ndaré    na-ndí-a-kun-á \\
\textsc{np}\textsubscript{7}-year  \textsc{sm}\textsubscript{7}.\textsc{rel}-come-\textsc{fv}  \textsc{rem}-\textsc{sm}\textsubscript{1SG}-\textsc{fut}-plant-\textsc{fv}  \textsc{np}\textsubscript{6}-sorghum
\textsc{np}\textsubscript{7}-year  \textsc{rem}-\textsc{sm}\textsubscript{7}-\textsc{pst}-finish-\textsc{fv}
\textsc{cop}-\textsc{np}\textsubscript{3}-maize  \textsc{rem}-\textsc{sm}\textsubscript{1SG}-\textsc{pst}-plant-\textsc{fv}<\textsc{rel}>\\
\glt ‘Next year, I will plant sorghum. \textbf{Last} \textbf{year} \textbf{I} \textbf{planted} \textbf{maize}.’ (ZF\_Elic14)
\z

The RPP expresses perfective aspect; it presents the event’s nucleus as a single event and does not allow reference to its internal structure. (\ref{bkm:Ref467683809}) illustrates the use of the RPP in contrast with its imperfective counterpart (see \sectref{bkm:Ref492377456}): the remote past imperfective verb \textit{kàndírwârà} ‘I was sick’ provides the background for the RPP verb \textit{nàndákàtà} ‘I became thin’.

\ea
\label{bkm:Ref467683809}
àhà kàndírwârà nàndákàtà\\
\gll a-ha    ka-ndí̲-rwá̲r-a      na-ndí̲-a-kat-a\\
\textsc{aug}-\textsc{dem}.\textsc{i}\textsubscript{16}  \textsc{pst}.\textsc{ipfv}-\textsc{sm}\textsubscript{1SG}-become\_sick-\textsc{fv}  \textsc{rem}-\textsc{sm}\textsubscript{1SG}-\textsc{pst}-become\_thin-\textsc{fv}\\
\glt ‘When I was sick, I became thin.’ (ZF\_Elic14)
\z

Because the RPP is perfective, it does not co-occur with imperfective markers such as persistive \textit{shí-}, habitual \textit{náku-} or \textit{-ang}, or a progressive construction (see chapter \ref{bkm:Ref99112358} on aspect). As seen in (\ref{bkm:Ref99965996}), the RPP may also not co-occur with the locative pluractional marker, which indicates that an event takes place in multiple locations (see \sectref{bkm:Ref494442567}); because the RPP does not allow reference to the event’s internal structure, co-occurrence with a marker that describes the event’s spatial distribution is disallowed. Incompatibility with the locative pluractional is also seen for the near past perfective (see (\ref{bkm:Ref494974356}) in \sectref{bkm:Ref488767483},). The near and remote past perfective constructions do occur with the locative pluractional (see Sections \ref{bkm:Ref494480139} and \ref{bkm:Ref488767517}).

\ea
\label{bkm:Ref99965996}
\glll *nìndákàbúyèndà\\
ni-ndí̲-a-kabú-end-a\\
\textsc{pst}-\textsc{sm}\textsubscript{1SG}-\textsc{pst}-\textsc{loc}.\textsc{pl}-walk-\textsc{fv}\\
\glt Intended: ‘I walked around/walked in different places.’ (NF\_Elic17)
\z

If the RPP is used with an event that includes a coda phase, such as the result state of a change-of-state verb, it is possible that the coda phase no longer holds at UT, as in (\ref{bkm:Ref486858336}), or that the coda phase continues at UT, as in (\ref{bkm:Ref486858338}).

\ea
\label{bkm:Ref486858336}
níndàrwárà zyônà kònó shûnù ndìrìshùwírè nênjà\\
\gll ní̲-ndi-a-rwár-a      zyóna konó  shúnu  ndi-ri\textsubscript{H}-shu\textsubscript{H}-í̲re     nénja\\
\textsc{pst}-\textsc{sm}\textsubscript{1SG}-\textsc{pst}-become\_sick-\textsc{fv}  yesterday
but  today  \textsc{sm}\textsubscript{1SG}-\textsc{refl}-feel-\textsc{stat}  well\\
\glt ‘I got sick yesterday, but today I feel well.’
\z

\ea
\label{bkm:Ref486858338}
níndàrwárà zyônà nèshûnù ndìshìrwàrîtè\\
\gll ní̲-ndi-a-rwár-a    zyóna ne=shúnu  ndi-shi\textsubscript{H}-rwa\textsubscript{H}r-í̲te \\
\textsc{pst}-\textsc{sm}\textsubscript{1SG}-\textsc{pst}-be\_sick-\textsc{fv}  yesterday
\textsc{com}=today  \textsc{sm}\textsubscript{1SG}-\textsc{per}-be\_sick-\textsc{stat}\\
\glt ‘I got sick yesterday, and today I am still sick.’ (NF\_Elic17)
\z

Certain dynamic verbs may also have a coda phase, such as \textit{zyáka enjúo} ‘to build a house’, whose coda phase is the existence of the house. Again, the RPP can be used in a context where the coda phase no longer holds, as in (\ref{bkm:Ref486859565}), and in a context where the coda phase still holds, as in (\ref{bkm:Ref486859566}).

\ea
\label{bkm:Ref486859565}
níndàzyáːk’ ènjûò ndókùyíǀàpàùrà hápè\\
\gll ní̲-ndi-a-zyáːk-a    e-N-júo ndi-ó=ku-í-ǀap-a-ur-a        hapé \\
\textsc{rem}-\textsc{sm}\textsubscript{1SG}-\textsc{pst}-build-\textsc{fv}  \textsc{aug}-\textsc{np}\textsubscript{9}-house
\textsc{pp}\textsubscript{1SG}-\textsc{con}=\textsc{inf}-\textsc{om}\textsubscript{9}-tear-\textsc{pl}1-\textsc{sep}.\textsc{tr}-\textsc{fv}  again\\
\glt ‘I built a house, then I destroyed it again.’ (NF\_Elic15)
\z

\ea
\label{bkm:Ref486859566}
ndímè níndàyízyàːkà èyí njûò òmò áꜝkárà\\
\gll ndí-me  ní̲-ndi-a-yí-zyaːk-a        e-í    N-júo o-mo      á̲-kar-á̲ \\
\textsc{cop}-\textsc{pers}\textsubscript{3SG}  \textsc{rem}-\textsc{sm}\textsubscript{1SG}-\textsc{pst}-\textsc{om}\textsubscript{9}-build-\textsc{fv}<\textsc{rel}>  \textsc{aug}-\textsc{dem}.\textsc{i}\textsubscript{9}  \textsc{np}\textsubscript{9}-house
\textsc{aug}-\textsc{dem}.\textsc{iii}\textsubscript{18} \textsc{sm}\textsubscript{1}.\textsc{rel}-stay-\textsc{fv}\\
\glt ‘It is me who built the house in which s/he stays.’ (NF\_Elic17)
\z
\subsection{Near past imperfective}
\label{bkm:Ref494480139}\hypertarget{Toc75352683}{}
The near past imperfective (NPI) only occurs in Namibian Fwe. It has the form \textsc{sm}-aku-B-a, with a post-initial prefix \textit{aku-} that is glossed as \textsc{npst}.\textsc{ipfv} ‘near past imperfective’. An example of a near past imperfective is given in (\ref{bkm:Ref99966038}).

\ea
\label{bkm:Ref99966038}
\glll ndàkùtòmbwèrà\\
ndi-aku-tombwer-a\\
\textsc{sm}\textsubscript{1SG}-\textsc{npst}.\textsc{ipfv}-weed-\textsc{fv}\\
\glt ‘I was weeding.’ (NF\_Elic15)
\z

The syllable \textit{ku} that occurs in the NPI prefix resembles the infinitive prefix \textit{ku-}. The NPI construction also shares certain other characteristics with the infinitive: like the infinitive prefix \textit{ku-}, the syllable \textit{ku} of the NPI can be dropped when the distal marker \textit{ka-} is used (see \sectref{bkm:Ref489965878}), as in (\ref{bkm:Ref71101958}). However, maintenance of both \textit{ku} and the distal prefix \textit{ka-} is also possible, as in (\ref{bkm:Ref72243736}).

\ea
\label{bkm:Ref71101958}
\glll ndàkàbèrèkà\\
ndi-a-ka-berek-a\\
\textsc{sm}\textsubscript{1SG}-\textsc{npst}.\textsc{ipfv}-\textsc{dist}-work-\textsc{fv}\\
\glt ‘I was working there.’
\z

\ea
\label{bkm:Ref72243736}
\glll ndàkùkàbèrèkà\\
ndi-aku-ka-berek-a\\
\textsc{sm}\textsubscript{1SG}-\textsc{npst}.\textsc{ipfv}-\textsc{dist}-work-\textsc{fv}\\
\glt ‘I was working there.’ (NF\_Elic17)
\z

The NPI also resembles the infinitive in its maintenance of lexical tones, without melodic tone, as illustrated in (\ref{bkm:Ref71102087}--\ref{bkm:Ref71102085}).

\newpage
\ea
\label{bkm:Ref71102087}
\ea
hîkà ‘cook’

\ex
\glll ndàkùhîkà\\
ndi-aku-hík-a\\
\textsc{sm}\textsubscript{1SG}-\textsc{npst}.\textsc{ipfv}-cook-\textsc{fv}\\
\glt ‘I was cooking.’ (NF\_Elic17)
\z\z

\ea
\label{bkm:Ref71102085}
\ea
rìmà ‘cultivate’

\ex
\glll ndàkùrìmà\\
ndi-aku-rim-a\\
\textsc{sm}\textsubscript{1SG}-\textsc{npst}.\textsc{ipfv}-cultivate-\textsc{fv}\\
\glt ‘I was cultivating.’ (NF\_Elic15)
\z\z

The NPI prefix \textit{aku-} can be used on the lexical verb, as in (\ref{bkm:Ref98834889}), or on an auxiliary verb \textit{ri} ‘be’, as in (\ref{bkm:Ref98834890}). The constructions are interchangeable, and no difference in meaning was observed.

\ea
\label{bkm:Ref98834889}
\glll bàkùbèrèkà\\
ba-aku-berek-a\\
\textsc{sm}\textsubscript{2}-\textsc{npst}.\textsc{ipfv}-work-\textsc{fv}\\
\glt ‘They were working.’
\z

\ea
\label{bkm:Ref98834890}
bàkùrí kùbèrèkà\\
\gll ba-aku-rí    ku-berek-a\\
\textsc{sm}\textsubscript{2}-\textsc{npst}.\textsc{ipfv}-be  \textsc{inf}-work-\textsc{fv}\\
\glt ‘They were working.’ (NF\_Elic15)
\z

The NPI situates an event in the near past, which is usually interpreted as earlier on the day of speaking, and aspectually, it references the internal structure of the event. In (\ref{bkm:Ref492453030}), the NPI is used to describe an event that was ongoing earlier the same day.

\ea
\label{bkm:Ref492453030}
ndàkùtòmbwèrà shûnù\\
\gll ndi-aku-tombwer-a  shúnu\\
\textsc{sm}\textsubscript{1SG}-\textsc{npst}.\textsc{ipfv}-weed-\textsc{fv}  today\\
\glt ‘I was weeding today.’ (NF\_Elic17)
\z

As the NPI expresses imperfectivity, it may express a longer, backgrounded event during which a shorter event is situated. In (\ref{bkm:Ref494720254}), the NPI verb \textit{ndákùbútùkà} ‘I was running’ describes the ongoing event which subsumes the shorter event described with the near past perfective verb \textit{ndàdóntì} ‘I got blisters’.

\ea
\label{bkm:Ref494720254}
ndàdóntì múmàténdè ángù àhà ndákùbútùkà\\
\gll ndi-a-dó̲nt-i          mú-ma-ténde  a-angú a-ha    ndí̲-aku-bútuk-a\\
\textsc{sm}\textsubscript{1}-\textsc{pst}-develop\_blister-\textsc{npst}.\textsc{pfv}  \textsc{np}\textsubscript{18}-\textsc{np}\textsubscript{6}-foot  \textsc{pp}\textsubscript{6}-\textsc{poss}\textsubscript{1SG}
\textsc{aug}-\textsc{dem}.\textsc{i}\textsubscript{16}  \textsc{sm}\textsubscript{2}.\textsc{rel}-\textsc{npst}.\textsc{ipfv}-run-\textsc{fv}\\
\glt ‘I got blisters on my feet when I was running.’ (NF\_Elic15)
\z

As an imperfective construction, the NPI can co-occur with other markers of imperfectivity, such as persistive in (\ref{bkm:Ref99966194}) and stative in (\ref{bkm:Ref99966196}).

\ea
\label{bkm:Ref99966194}
\glll àkùshíŋòrà\\
a-aku-shí-ŋor-a\\
\textsc{sm}\textsubscript{1}-\textsc{npst}.\textsc{ipfv}-\textsc{per}-write-\textsc{fv}\\
\glt ‘S/he was still writing.’
\z

\ea
\label{bkm:Ref99966196}
\glll ndàkùrwárîtè\\
ndi-aku-rwa\textsubscript{H}r-í̲te\\
\textsc{sm}\textsubscript{1SG}-\textsc{npst}.\textsc{ipfv}-become\_sick-\textsc{stat}\\
\glt ‘I was sick.’ (NF\_Elic17)
\z

The NPI cannot be combined with an overt progressive construction, such as the progressive auxiliary \textit{kwesi}, as shown by the ungrammaticality of (\ref{bkm:Ref98851674}). When used without other overt imperfective markers, the NPI has a progressive interpretation, as in (\ref{bkm:Ref98851677}).

\ea
\label{bkm:Ref98851674}
  *bàkwèsì bàkùsèbèzà\\
Intended: ‘They were working.’
\z

\ea
\label{bkm:Ref98851677}
\glll bàkùsèbèzà\\
ba-aku-sebez-a\\
\textsc{sm}\textsubscript{2}-\textsc{npst}.\textsc{ipfv}-work-\textsc{fv}\\
\glt ‘They were working.’ (NF\_Elic17)
\z

The NPI also does not co-occur with habituals, as shown for the habitual suffix \textit{-ang} in (\ref{bkm:Ref99966234}).

\ea
\label{bkm:Ref99966234}
\glll *ndàkùtòmbwèràngà\\
ndi-aku-tombwer-ang-a\\
\textsc{sm}\textsubscript{1SG}-\textsc{npst}.\textsc{ipfv}-weed-\textsc{hab}-\textsc{fv}\\
\glt Intended: ‘I used to weed.’ (NF\_Elic17)
\z

Even when used without habitual markers, the NPI is never used with a habitual interpretation. This may be a result of its restriction to the near past: this time frame may be too short for any event to be considered habitual. The remote past imperfective does combine with \textit{-ang} to express a past habitual (see \sectref{bkm:Ref492377456}).

The NPI may be combined with the locative pluractional, which marks that an event takes place across different locations (see \sectref{bkm:Ref494442567}), as in (\ref{bkm:Ref99966261}). The remote past imperfective, too, can co-occur with the locative pluractional, but not the near and remote past perfective. Because the locative pluractional describes the internal structure of the event, namely its spatial distribution, it is restricted to imperfective constructions, that allow reference to the event’s internal structure.

\ea
\label{bkm:Ref99966261}
\glll ndàkùrí kàbúyèndà\\
ndi-aku-rí    kabú-end-a\\
\textsc{sm}\textsubscript{2}-\textsc{npst}.\textsc{ipfv}-be  \textsc{loc}.\textsc{pl}-work-\textsc{fv}\\
\glt ‘I was walking around.’ (NF\_Elic17)
\z

The NPI situates the entire event in the recent past; the event’s nucleus or coda is no longer ongoing at the time of speaking. The NPI construction in (\ref{bkm:Ref490745119}) situates the verb’s nucleus (‘working’) in the near past, and simultaneously expresses that the nuclear phase no longer holds at UT.

\ea
\label{bkm:Ref490745119}
\glll bàkùsèbèzà\\
ba-aku-sebez-a\\
\textsc{sm}\textsubscript{2}-\textsc{npst}.\textsc{ipfv}-work-\textsc{fv}\\
\glt ‘They were working (but they’re not working anymore).’ (NF\_Elic17)
\z

The NPI also does not allow overlap between the event’s coda and utterance time. This is illustrated in (\ref{bkm:Ref490745118}), where the NPI situates both the nuclear phase of becoming sick and the coda phase of being sick in the near past; an interpretation where the coda phase of being sick is still ongoing at the time of speaking is not possible. In this sense the NPI differs from the near and remote past perfective constructions; although both the NPI and the perfective past constructions situate the nucleus before UT, the perfective past constructions do allow overlap between the event’s coda and the nucleus.

\ea
\label{bkm:Ref490745118}
\glll ndàkùrwárîtè\\
ndi-aku-rwa\textsubscript{H}r-í̲te\\
\textsc{sm}\textsubscript{1SG}-\textsc{npst}.\textsc{ipfv}-become\_sick-\textsc{stat}\\
\glt ‘I was sick (but I am not anymore).’ (NF\_Elic17)
\z
\subsection{Remote past imperfective}
\label{bkm:Ref488767517}\hypertarget{Toc75352684}{}\label{bkm:Ref492377456}
The (remote) past imperfective construction has the form ka-\textsc{sm}-B-a, with a pre-initial prefix \textit{ka-} that specifically marks (remote) past imperfective. Because the near past imperfective marked with \textit{aku-} does not exist in Zambian Fwe, Zambian Fwe uses this construction for both near and remote past imperfective meanings, and only in Namibian Fwe is it dedicated to remote past imperfective. Because of this ambiguity, the construction will be referred to as either past imperfective (PI) or remote past imperfective (RPI), and its marker \textit{ka-} will be glossed as ‘past imperfective’ \textsc{pst}.\textsc{ipfv}.

The past imperfective has a high tone on the subject marker (melodic tone 2) and a high tone on the last syllable, or on the penultimate syllable if this syllable is bimoraic (melodic tone 1), and underlying tones are deleted (melodic tone 4). Examples of the tonal realizations of verbs in the past imperfective are given in (\ref{bkm:Ref99969598}--\ref{bkm:Ref99969669}).

\ea
\label{bkm:Ref99969598}
ménjì kàátòntórà\\
\gll ma-ínji  ka-á̲-to\textsubscript{H}ntor-á̲\\
\textsc{np}\textsubscript{6}-water  \textsc{pst}.\textsc{ipfv}-\textsc{sm}\textsubscript{6}-be\_cold-\textsc{fv}\\
\glt ‘The water was cold.’ (NF\_Elic15)
\z

\ea
kàbáyêndà nàbàmbwá ꜝbábò\\
\gll ka-bá̲-é̲nd-a    na=ba-mbwá    ba-a=bó\\
\textsc{pst}.\textsc{ipfv}-\textsc{sm}\textsubscript{2}-go-\textsc{fv}  \textsc{com}=\textsc{np}\textsubscript{2}-dog  \textsc{pp}\textsubscript{2}-\textsc{con}=\textsc{dem}.\textsc{iii}\textsubscript{2}\\
\glt ‘She was walking with her dogs.’ (ZF\_Narr15)
\z

\ea
\label{bkm:Ref99969669}
àhá kàbádàmàdàmá bùryàhò\\
\gll a-há    ka-bá̲-dama-dam-á̲    bu-ryaho\\
\textsc{aug}-\textsc{dem}.\textsc{i}\textsubscript{16}  \textsc{pst}.\textsc{ipfv}-\textsc{sm}\textsubscript{2}-\textsc{pl}2-beat-\textsc{fv}  \textsc{np}\textsubscript{14}-like\_that\\
\glt ‘When they were beating [the drum] like that…’ (ZF\_Narr13)
\z

The PI construction seems to have developed from an auxiliary followed by a subordinate present verb. The PI construction resembles the present construction because both make use of melodic tones 1 and 4, and both lack post-initial and suffixal tense/aspect markers (see \sectref{bkm:Ref72233436} on the present). The high tone of the subject marker, seen in the PI construction, is also used in subordinate verbs (see \sectref{bkm:Ref492308543} on clause types). The earlier auxiliary grammaticalized into a prefix \textit{ka-} on the lexical verb.

In Namibian Fwe, the remote past imperfective has the same temporal domain as the remote past perfective: it canonically refers to events that took place before the day of speaking, as in (\ref{bkm:Ref99969709}--\ref{bkm:Ref99969710}). To refer to events that took place earlier on the day of speaking, Namibian Fwe uses the near past imperfective (see \sectref{bkm:Ref451515182}).

\ea
\label{bkm:Ref99969709}
kàndírwàrítè zyônà\\
\gll ka-ndí̲-rwa\textsubscript{H}r-í̲te        zyóna\\
\textsc{pst}.\textsc{ipfv}-\textsc{sm}\textsubscript{1SG}-become\_sick-\textsc{stat}  yesterday\\
\glt ‘I was sick yesterday.’ (NF\_Elic17)
\z

\ea
\label{bkm:Ref99969710}
èzìryó kàzíꜝryóhà\\
\gll e-zi-ryó    ka-zí̲-ryo\textsubscript{H}-á̲\\
\textsc{aug}-\textsc{np}\textsubscript{8}-food  \textsc{pst}.\textsc{ipfv}-\textsc{sm}\textsubscript{8}-be\_tasty-\textsc{fv}\\
\glt Describing yesterday’s party: ‘The food was tasty.’ (NF\_Elic15)
\z

As the near past imperfective does not exist in Zambian Fwe, Zambian Fwe uses the PI construction as a general past imperfective form, for both events situated in the recent past, as in (\ref{bkm:Ref99969730}), and the remote past, as in (\ref{bkm:Ref99969732}).

\ea
\label{bkm:Ref99969730}
mùndáré kàndíꜝtwá shùnù\\
\gll N-mu-ndaré    ka-ndí̲-tw-á̲      shunu\\
\textsc{cop}-\textsc{np}\textsubscript{3}-maize  \textsc{pst}.\textsc{ipfv}-\textsc{sm}\textsubscript{1SG}-pound-\textsc{fv}  today\\
\glt ‘I was pounding maize today.’
\z

\ea
\label{bkm:Ref99969732}
mùndáré kàndíꜝtwá zyônà\\
\gll N-mu-ndaré    ka-ndí̲-tw-á̲      zyóna\\
\textsc{cop}-\textsc{np}\textsubscript{3}-maize  \textsc{pst}.\textsc{ipfv}-\textsc{sm}\textsubscript{1SG}-pound-\textsc{fv}  yesterday\\
\glt ‘I was pounding maize yesterday.’ (ZF\_Elic14)
\z

The RPI presents an event as ongoing, with explicit reference to the internal constituency of the event’s nucleus. This becomes clear when combining a verb in the RPI with a consecutive verb, which lacks explicit tense marking but derives its temporal interpretation from a preceding inflected verb. In (\ref{bkm:Ref431806450}), the RPI verb \textit{kàndìtèká} ‘I was fetching’ is followed by the consecutive verb \textit{ndókùsúsà} ‘I dropped’, indicating that the event of dropping the container is situated during the fetching of water.

\ea
\label{bkm:Ref431806450}
àhà kàndìtèká mênjì ndókùsús’ ècìbìyà cángù\\
\gll a-ha    ka-ndí̲-te\textsubscript{H}k-á̲    ma-ínji  ndi-ó=ku-sús-a e-ci-biya    ci-angú\\
\textsc{aug}-\textsc{dem}.\textsc{i}\textsubscript{16}  \textsc{pst}.\textsc{ipfv}-\textsc{sm}\textsubscript{1SG}-fetch-\textsc{fv}  \textsc{np}\textsubscript{6}-water  \textsc{sm}\textsubscript{1SG}-\textsc{con}=\textsc{inf}-drop-\textsc{fv}
\textsc{aug}-\textsc{np}\textsubscript{7}-container  \textsc{pp}\textsubscript{7}-\textsc{poss}\textsubscript{1SG}\\
\glt ‘While I was fetching water, I dropped my container.’ (ZF\_Elic14)
\z

The RPI may co-occur with markers that indicate a type of imperfective aspect, such as the stative in (\ref{bkm:Ref99970756}), the habitual \textit{-ang} in (\ref{bkm:Ref99970757}), the progressive-marking fronted-infinitive construction in (\ref{bkm:Ref99970759}), the progressive auxiliary \textit{kwesi} in (\ref{bkm:Ref99970760}), and the persistive \textit{shí}- in (\ref{bkm:Ref99970761}).

\ea
\label{bkm:Ref99970756}
zyônà \textstyleunderlinedChar{\textbf{kàndìshwénêtè}}\\
\gll zyóna    ka-ndi-shwen-é̲te\\
yesterday  \textsc{pst}.\textsc{ipfv}-\textsc{sm}\textsubscript{1SG}-become\_tired-\textsc{stat}\\
\glt ‘Yesterday, \textstyleunderlinedChar{\textbf{I} \textbf{was} \textbf{tired}}.’ (ZF\_Elic14)
\z

\ea
\label{bkm:Ref99970757}
\textstyleunderlinedChar{\textbf{kárìzòːrángà}} òndávù kùyà kúkùcâːnà\\
\gll ka-á̲-ri\textsubscript{H}-zoːr-á̲ng-a      o-∅-ndavú ku-i-a    kú-ku-cáːn-a \\
\textsc{pst}.\textsc{ipfv}-\textsc{sm}\textsubscript{1}-\textsc{refl}-turn-\textsc{hab}-\textsc{fv}  \textsc{aug}-\textsc{np}\textsubscript{1a}-lion
\textsc{inf}-go-\textsc{fv}  \textsc{np}\textsubscript{17}-\textsc{inf}-hunt-\textsc{fv}\\
\glt ‘\textstyleunderlinedChar{\textbf{He} \textbf{used} \textbf{to} \textbf{turn} \textbf{himself}} into a lion to go hunting.’ (NF\_Narr15)
\z

\ea
\label{bkm:Ref99970759}
kùshókà káꜝshókà\\
\gll ku-shók-a  ka-\textstyleunderlinedChar{á}̲-sho\textsubscript{H}k-á̲\\
\textsc{inf}-rain  \textsc{pst}.\textsc{ipfv}-\textsc{sm}\textsubscript{1a}-rain-\textsc{fv}\\
\glt ‘It has been raining.’ (ZF\_Elic14)
\z

\ea
\label{bkm:Ref99970760}
cìntù císhàkàhárà ècí \textstyleunderlinedChar{\textbf{kàtúkwèsì} \textbf{tùàmbàúrà}}\\
\gll ∅-ci-ntu    cí̲-shakahar-á̲    e-cí ka-tú̲-kwesi      tu-ambaur-á̲ \\
\textsc{cop}-\textsc{np}\textsubscript{7}-thing  \textsc{sm}\textsubscript{7}.\textsc{rel}-be\_important-\textsc{fv}  \textsc{aug}-\textsc{dem}.\textsc{i}\textsubscript{7}
\textsc{pst}.\textsc{ipfv}-\textsc{sm}\textsubscript{1PL}-\textsc{prog}  \textsc{sm}\textsubscript{1PL}-discuss-\textsc{fv}\\
\glt ‘It’s an important thing that \textbf{we} \textbf{were} \textbf{discussing}.’ (ZF\_Elic14)
\z

\ea
\label{bkm:Ref99970761}
\textbf{kàshìkéːzyà} mùrùshàrá ꜝrwángù\\
\gll ka-á̲-shi\textsubscript{H}-ké̲ːzy-a      mu-ru-shará    ru-angú\\
\textsc{pst}.\textsc{ipfv}-\textsc{sm}\textsubscript{1}-\textsc{per}-come-\textsc{fv}  \textsc{np}\textsubscript{18}-\textsc{np}\textsubscript{11}-back  \textsc{pp}\textsubscript{11}-\textsc{poss}\textsubscript{1SG}\\
\glt ‘\textbf{He} \textbf{was} \textbf{still} \textbf{coming} behind me.’ (ZF\_Narr13)
\z

When not used with markers indicating a specific subtype of imperfective aspect, the PI is usually interpreted as a progressive, as in (\ref{bkm:Ref506223331}), or less commonly, habitual, as in (\ref{bkm:Ref506223333}).

\newpage
\ea
\label{bkm:Ref506223331}
kàtúyêndà nòzyú mùyéꜝnzángù\\
\gll ka-tú̲-é̲nd-a      no=zyú  mu-énz-angú\\
\textsc{pst}.\textsc{ipfv}-\textsc{sm}\textsubscript{1PL}-go-\textsc{fv}  \textsc{com}=\textsc{dem}.\textsc{i}\textsubscript{1}  \textsc{np}\textsubscript{1}-friend-\textsc{poss}\textsubscript{1SG}\\
\glt ‘I was traveling with this friend of mine.’ (NF\_Narr17)
\z

\ea
\label{bkm:Ref506223333}
\glll kàndízyîmbà\\
ka-ndí̲-zyí̲mb-a\\
\textsc{pst}.\textsc{ipfv}-\textsc{sm}\textsubscript{1SG}-sing-\textsc{fv}\\
\glt ‘I used to sing/be a singer.’ (NF\_Elic15)
\z

The PI may also co-occur with the locative pluractional marker, as in (\ref{bkm:Ref99970977}), which describes that an event takes place in different locations; although not strictly aspectual, the locative pluractional does describe the internal structure of the event (namely its spatial distribution), and therefore may only occur with imperfective constructions.

\ea
\label{bkm:Ref99970977}
\glll kàndíkàbúyêndà\\
ka-ndí̲-kabú-é̲nd-a\\
\textsc{pst}.\textsc{ipfv}-\textsc{sm}\textsubscript{1SG}-\textsc{loc}.\textsc{pl}-walk-\textsc{fv}\\
\glt ‘I was walking around/walking in different places.’ (NF\_Elic17)
\z

Unlike perfective past forms, the past imperfective can be used with the verbs \textit{ri} ‘be’, as in (\ref{bkm:Ref99970996}--\ref{bkm:Ref498095669}), and \textit{ina} ‘be (somewhere)’ in (\ref{bkm:Ref99971016}).

\ea
\label{bkm:Ref99970996}
èzíryó kàzîrì zìrôtù\\
\gll e-zi-río    ka-zí̲-ri    zi-rótu\\
\textsc{aug}-\textsc{np}\textsubscript{8}-food  \textsc{pst}.\textsc{ipfv}-\textsc{sm}\textsubscript{8}-be  \textsc{np}\textsubscript{8}-good\\
\glt ‘The food was good.’ (NF\_Elic15)
\z

\ea
\label{bkm:Ref498095669}
kàbárì bànînì\\
\gll ka-bá̲-ri    ba-níni\\
\textsc{pst}.\textsc{ipfv}-\textsc{sm}\textsubscript{2}-be  \textsc{np}\textsubscript{2}-small\\
\glt ‘They were small.’ (NF\_Elic15)
\z

\ea
\label{bkm:Ref99971016}
kàkwín’ ꜝómùnzì òmù kàmwíꜝná bàntù\\
\gll ka-kú̲-iná    o-mu-nzi o-mu    ka-mú̲-iná      ba-ntu \\
\textsc{pst}.\textsc{ipfv}-\textsc{sm}\textsubscript{17}-be\_at  \textsc{aug}-\textsc{np}\textsubscript{3}-village
\textsc{aug}-\textsc{dem}.\textsc{i}\textsubscript{18}  \textsc{pst}.\textsc{ipfv}-\textsc{sm}\textsubscript{18}-be\_at    \textsc{np}\textsubscript{2}-person\\
\glt ‘There was a village, where people were living.’ (NF\_Narr15)
\z

The remote past imperfective situates the entire event in the past, including an optional coda phase. The event cannot overlap with UT, as in (\ref{bkm:Ref494820165}), which indicates that it is no longer raining at utterance time. When the PI expresses a past habitual, overlap with UT is also not possible, as in (\ref{bkm:Ref494820164}), where all instances of weeding (which together constitute the speaker’s habit of weeding) are situated before UT.

\ea
\label{bkm:Ref494820165}
kùshókà káꜝshókà\\
\gll ku-shók-a  ka-á̲-sho\textsubscript{H}k-á̲\\
\textsc{inf}-rain-\textsc{fv}  \textsc{pst}.\textsc{ipfv}-\textsc{sm}\textsubscript{1}-rain-\textsc{fv}\\
\glt ‘It has been raining (but it’s not raining now).’ (ZF\_Elic14)
\z

\ea
\label{bkm:Ref494820164}
\glll kàndítòmbwèrângà\\
ka-ndí̲-tombwer-á̲ng-a\\
\textsc{pst}.\textsc{ipfv}-\textsc{sm}\textsubscript{1SG}-weed-\textsc{hab}-\textsc{fv}\\
\glt ‘I used to weed.’ (but not anymore) (NF\_Elic15)
\z

When the PI is used with stativized verbs, it describes an ongoing state (e.g. the coda state that follows the nuclear change in state), which cannot overlap with UT. For instance, in (\ref{bkm:Ref494820264}), the coda phase of being tired does not hold at the time of speaking, and in (\ref{bkm:Ref494820265}), the coda phase of knowing them does not hold at the time of speaking, because the people described have now passed away.

\ea
\label{bkm:Ref494820264}
zyônà kàndìshwénêtè shùnù tàndìshwènètêː\\
\gll zyóna    ka-ndi-shwen-é̲te shunu  ta-ndi-shwen-ete-í̲ \\
yesterday  \textsc{pst}.\textsc{ipfv}-\textsc{sm}\textsubscript{1SG}-become\_tired-\textsc{stat}
today  \textsc{neg}-\textsc{sm}\textsubscript{1SG}-become\_tired-\textsc{stat}-\textsc{neg}\\
\glt ‘Yesterday I was tired, today I’m not tired.’ (ZF\_Elic14)
\z

\ea
\label{bkm:Ref494820265}
\glll kàndíbàzyìː\\
ka-ndí̲-ba-zyiː\\
\textsc{pst}.\textsc{ipfv}-\textsc{sm}\textsubscript{1SG}-\textsc{om}\textsubscript{2}-get\_to\_know.\textsc{stat}\\
\glt ‘I used to know them.’ (but they passed away) (NF\_Elic15)
\z

Note that the use of the past imperfective with a change-of-state verb that is not stativized is interpreted as dynamic, i.e. an incipient change of state, that is no longer ongoing at the time of speaking, as in (\ref{bkm:Ref99971124}).

\ea
\label{bkm:Ref99971124}
káꜝnúnà kònó hànó shàkábúkàtà\\
\gll ka-á̲-nun-á̲        konó  hanó shi-a-kabú-kat-a \\
\textsc{pst}.\textsc{ipfv}-\textsc{sm}\textsubscript{1}-become\_fat-\textsc{fv}  but  \textsc{dem}.\textsc{ii}\textsubscript{16}
\textsc{inc}-\textsc{sm}\textsubscript{1}-\textsc{loc}.\textsc{pl}-become\_thin-\textsc{fv}\\
\glt ‘She was getting fat, but now she’s getting thin again.’ (NF\_Elic15)
\z
\section{Future}
\label{bkm:Ref463007186}\hypertarget{Toc75352685}{}
Like the past, the future is divided into two domains based on their perceived distance from the utterance time: the near future construction situates the event after utterance time but within the current temporal domain (most commonly, the day of speaking), and the remote future construction situates the event after the current temporal domain, i.e. typically tomorrow or later.

\subsection{Near future}
\label{bkm:Ref490749852}\label{bkm:Ref489880945}\label{bkm:Ref489268312}\hypertarget{Toc75352686}{}
The near future construction consists of a prefix \textit{mbo-}, glossed as \textsc{near}.\textsc{fut}, added to the verb in the subjunctive mood. The subjunctive has an imperfective and a perfective form (see chapter \ref{bkm:Ref99112433}), and both can be made into near future forms, as in (\ref{bkm:Ref99971271}--\ref{bkm:Ref99971273}).

\ea
\label{bkm:Ref99971271}
\ea
Subjunctive perfective \\
\glll ndìbèrékè\\
ndi-berek-é\\
\textsc{sm}\textsubscript{1SG}-work-\textsc{pfv}.\textsc{sbjv}\\  
\glt ‘I should work.’

\ex
Near future perfective\\
mbòndíbèrékè\\
\gll mbo-ndí̲-berek-é̲̲\\
\textsc{near}.\textsc{fut}-\textsc{sm}\textsubscript{1SG}-work-\textsc{pfv}.\textsc{sbjv}\\    
\glt ‘I will work.’          
\z\z

\ea
\label{bkm:Ref99971273}
\ea
Subjunctive imperfective\\
\glll mbòndákùbèrèkà        \\
mbo-nd-áku-berek-a \\
\textsc{near}.\textsc{fut}-\textsc{sm}\textsubscript{1SG}-\textsc{sbjv}.\textsc{ipfv}-work-\textsc{fv} \\
\glt ‘I will be working.’

\ex
Near future imperfective\\
\glll ndákùbèrèkà\\
ndi-áku-berek-a\\
\textsc{sm}\textsubscript{1SG}-\textsc{sbjv}.\textsc{ipfv}-work-\textsc{fv}  \\
\glt ‘I should be working.’ (NF\_Elic17)
\z\z

Subjunctive forms maintain their tonal patterns when turned into near future forms with the prefix \textit{mbo-}, but a high tone is added to the subject marker (melodic tone 2), which is absent in the corresponding subjunctive form (see \sectref{bkm:Ref492309168}). The perfective near future form shares another tonal peculiarity with the perfective subjunctive on which it is based, namely a change in melodic tone conditioned by the presence of object markers. The perfective subjunctive takes MT 1 when the verb does not include an object marker, but MT 3, a high tone on the second stem syllable, if the verb includes an object marker. The perfective near future takes MT 3 only when the verb includes two object markers, as in (\ref{bkm:Ref99971546}); MT 1 is used when there is no object marker, as in (\ref{bkm:Ref99971558}), or only one object marker, as in (\ref{bkm:Ref99971559}).

\ea
\label{bkm:Ref99971546}
\glll mbòndícìkùtòrókèrè\\
mbo-ndí̲-ci\textsubscript{H}-ku-to\textsubscript{H}ró̲k-er-e\\
\textsc{near}.\textsc{fut}-\textsc{sm}\textsubscript{1SG}-\textsc{om}\textsubscript{7}-\textsc{om}\textsubscript{2SG}-exlain-\textsc{appl}-\textsc{pfv}.\textsc{sbjv}\\
\glt ‘I will explain it to you.’ (NF\_Elic15)
\z

\ea
\label{bkm:Ref99971558}
\glll mbòndítòrókè\\
mbo-ndí̲-to\textsubscript{H}rok-é̲\\
\textsc{near}.\textsc{fut}-\textsc{sm}\textsubscript{1SG}-explain-\textsc{pfv}.\textsc{sbjv}\\
\glt ‘I will explain.’
\z

\ea
\label{bkm:Ref99971559}
\glll mbòndícìtòrókè\\
mbo-ndí̲-ci\textsubscript{H}-to\textsubscript{H}rok-é̲\\
\textsc{near}.\textsc{fut}-\textsc{sm}\textsubscript{1SG}-\textsc{om}\textsubscript{7}-explain-\textsc{pfv}.\textsc{sbjv}\\
\glt ‘I will explain it.’
\z

In Zambian Fwe, the near future prefix has an alternative form \textit{mba-}, as in (\ref{bkm:Ref99971684}), which is used interchangeably with the prefix \textit{mbo-.} Namibian Fwe only uses the prefix \textit{mbo-}, as in (\ref{bkm:Ref99971699}).

\ea
\label{bkm:Ref99971684}
\glll mbàndíyêndè\\
mba-ndí̲-é̲nd-e\\
\textsc{near}.\textsc{fut}-\textsc{sm}\textsubscript{1SG}-go-\textsc{pfv}.\textsc{sbjv}\\
\glt ‘I will go.’ (Zambian Fwe)
\z

\ea
\label{bkm:Ref99971699}
\glll mbòndíyêndè\\
mbo-ndí̲-é̲nd-e\\
\textsc{near}.\textsc{fut}-\textsc{sm}\textsubscript{1SG}-go-\textsc{pfv}.\textsc{sbjv}\\
\glt ‘I will go.’ (Zambian and Namibian Fwe)
\z

The near future is used to situate an event after utterance time, but within the same temporal domain, usually interpreted as the day of speaking. As such, it can be used with time adverbials such as \textit{màsíkù} ‘tonight’ in (\ref{bkm:Ref431476794}), or \textit{shùnù} ‘today’ in (\ref{bkm:Ref431476804}).

\ea
\label{bkm:Ref431476794}
mbàndíꜝrárè màsíkù\\
\gll mba-ndí̲-rá̲ːr-e      ma-sikú\\
\textsc{near}.\textsc{fut}-\textsc{sm}\textsubscript{1SG}-sleep-\textsc{pfv}.\textsc{sbjv}  \textsc{np}\textsubscript{6}-evening\\
\glt ‘I will sleep tonight.’ (ZF\_Elic14)
\z

\ea
\label{bkm:Ref431476804}
àbàbàrà mbòbáhùré shùnù\\
\gll a-ba-bara    mbo-bá̲-hur-é̲      shunu\\
\textsc{aug}-\textsc{np}\textsubscript{2}-visitor  \textsc{near}.\textsc{fut}-\textsc{sm}\textsubscript{2}-arrive-\textsc{pfv}.\textsc{sbjv}  today\\
\glt ‘The visitors will arrive today.’ (NF\_Elic15)
\z

The near future can also be based on larger temporal domains, such as the current year in (\ref{bkm:Ref494466153}).

\ea
\label{bkm:Ref494466153}
mwánàngú ómweri mbwámàné cìkòró ùnó mwâkà\\
\gll mu-án-angú    u-ó=mu-eri  mbo-á̲-man-é̲      unó    mu-áka \\
\textsc{np}\textsubscript{1}-child-\textsc{poss}\textsubscript{1SG}  \textsc{pp}\textsubscript{1}-\textsc{con}=\textsc{np}\textsubscript{1}-firstborn
\textsc{near}.\textsc{fut}-\textsc{sm}\textsubscript{1}-finish-\textsc{pfv}.\textsc{sbjv}  \textsc{dem}.\textsc{ii}\textsubscript{3}  \textsc{np}\textsubscript{3}-year\\
\glt ‘My eldest child will finish school this year.’ (NF\_Elic17)
\z

The near future can also be used to refer to events that are imminent. The example in (\ref{bkm:Ref431817599}) is taken from a narrative in which the two main characters are trying to hide from a lion who is pursuing them. They ask help from a frog, and he devises a plan to help them, which will be put into action immediately. This imminence is expressed with the use of the near future.

\ea
\label{bkm:Ref431817599}
ècìmbòtwè cókùbáꜝtéyé mbòndímìtúsè\\
\gll e-ci-mbotwe    ci-ó=ku-bá-ta-a     iyé  mbo-ndí̲-mi\textsubscript{H}-tus-é̲ \\
\textsc{aug}-\textsc{np}\textsubscript{7}-frog  \textsc{pp}\textsubscript{7}-\textsc{con}=\textsc{inf}-\textsc{om}\textsubscript{2}-say-\textsc{fv}  that 
\textsc{near}.\textsc{fut}-\textsc{sm}\textsubscript{1SG}-\textsc{om}\textsubscript{2PL}-help-\textsc{pfv}.\textsc{sbjv}\\
\glt ‘The frog told them, I will help you.’ (NF\_Narr15)
\z

The near future form can only be used for events that have not yet started at the time of speaking, as in (\ref{bkm:Ref468178599}), which can only be said by someone who has not yet started to work. In (\ref{bkm:Ref468178611}), from a narrative, the speaker is considering removing his injured eye, because he cannot focus with his remaining good eye. This shows that the event expressed by the near future verb, seeing with this remaining eye, does not hold at the time of speaking.

\ea
\label{bkm:Ref468178599}
shùnù mbòndísèbèzê\\
\gll shunu  mbo-ndí̲-sebez-é̲\\
today  \textsc{near}.\textsc{fut}-\textsc{sm}\textsubscript{1SG}-work-\textsc{pfv}.\textsc{sbjv}\\
\glt ‘Today, I will work.’ (said by someone who has not yet started) (NF\_Elic15)
\z

\ea
\label{bkm:Ref468178611}
mwèndì mbòndíbòné nèrí rìnàsìyárìrì\\
\gll mwendi  mbo-ndí̲-bo\textsubscript{H}n-é̲  ne=rí    ri-na-siá̲rir-ir-i \\
maybe  \textsc{near}.\textsc{fut}-\textsc{sm}\textsubscript{1SG}-see-\textsc{pfv}.\textsc{sbjv}
\textsc{com}=\textsc{dem}.\textsc{i}\textsubscript{5}  \textsc{sm}\textsubscript{5}-\textsc{pst}-leave-\textsc{appl}-\textsc{npst}.\textsc{pfv}\\
\glt ‘Maybe I will see with the other one.’ (ZF\_Narr14)
\z

The near future perfective is used to refer to single events, as in (\ref{bkm:Ref72246547}), and the near future imperfective to extended or recurring events, as in (\ref{bkm:Ref98852252}).

\ea
\label{bkm:Ref72246547}
mbòndísèbèzé shûnù\\
\gll mbo-ndí̲-sebez-é̲      shúnu\\
\textsc{near}.\textsc{fut}-\textsc{sm}\textsubscript{1SG}-work-\textsc{pfv}.\textsc{sbjv}  today\\
\glt ‘I will work today.’ (NF\_Elic17)
\z

\ea
\label{bkm:Ref98852252}
mbòndákùbèrèkà èzyúbà nèzyûbà\\
\gll mbo-ndi-áku-berek-a      e-∅-zyúba  ne=∅-zyúba\\
\textsc{near}.\textsc{fut}-\textsc{sm}\textsubscript{1SG}-\textsc{sbjv}.\textsc{ipfv}-work-\textsc{fv}  \textsc{aug}-\textsc{np}\textsubscript{5}-day  \textsc{com}=\textsc{np}\textsubscript{5}-day\\
\glt ‘I will work every day.’
\z

The near future imperfective can have a progressive interpretation, or more commonly a habitual interpretation. The near future imperfective may combine with the habitual suffix \textit{-ang} (see also \sectref{bkm:Ref451268511}), as in (\ref{bkm:Ref492309981}), but a habitual interpretation is also available without habitual markers, as in (\ref{bkm:Ref492309983}).

\ea
\label{bkm:Ref492309981}
mbòndákùshàmbàngà\\
\gll mbo-ndi-áku-shamb-ang-a\\
\textsc{near}.\textsc{fut}-\textsc{sm}\textsubscript{1SG}-\textsc{sbjv}.\textsc{ipfv}-wash-\textsc{hab}-\textsc{fv}\\
\glt ‘I will wash regularly.’
\z

\ea
\label{bkm:Ref492309983}
\glll mbòndákùbèrèkà\\
mbo-ndi-áku-berek-a\\
\textsc{near}.\textsc{fut}-\textsc{sm}\textsubscript{1SG}-\textsc{sbjv}.\textsc{ipfv}-work-\textsc{fv}\\
\glt ‘I will work regularly.’
\z

In Zambian Fwe, a near future habitual can be expressed by combining the near future perfective with the habitual suffix \textit{-ang}, as in (\ref{bkm:Ref492310019}). In Namibian Fwe the expression of a near future habitual always requires the near future prefix \textit{áku-}, as in (\ref{bkm:Ref492309983}).

\ea
\label{bkm:Ref492310019}
èyìnó nsûndà \textbf{mbòndíbùːkángè} kàêtì\\
\gll e-inó    N-súnda  mbo-ndí̲-buː\textsubscript{H}k-á̲ng-e      ka-éti\\
\textsc{aug}-\textsc{dem}.\textsc{ii}\textsubscript{9}  \textsc{np}\textsubscript{9}-week  \textsc{near}.\textsc{fut}-\textsc{sm}\textsubscript{1SG}-wake-\textsc{hab}-\textsc{pfv}.\textsc{sbjv}  \textsc{adv}-eight\\
\glt ‘This week, I will wake up at eight.’ (ZF\_Elic14)
\z

The near future construction cannot be used in subordinate clauses, as shown in (\ref{bkm:Ref467504624}). Instead, near future can be expressed in subordinate clauses with a present verb, as in (\ref{bkm:Ref72247218}) (note that the present construction may also have a future interpretation in main clauses; see \sectref{bkm:Ref72233436}). This is in line with the origin of this construction from in an earlier subordinated verb, which is is further supported by the use of melodic tone 2, which is also used in subordinated verbs (see \sectref{bkm:Ref491095705} for details).

\ea
 *àbàbàrà àbó mbòbáhùré shùnù\\
\gll a-ba-bara    a-bó    mbo-bá̲-hur-é̲      shunu\\
\textsc{aug}-\textsc{np}\textsubscript{2}-visitor  \textsc{aug}-\textsc{dem}.\textsc{iii}\textsubscript{2} \textsc{near}.\textsc{fut}-\textsc{sm}\textsubscript{2}-arrive-\textsc{pfv}.\textsc{sbjv}  today\\
Intended: ‘The visitors who will arrive today…’
\z

\ea
\label{bkm:Ref72247218}
àbàbàrà àbó ꜝbáhùrá shùnù\\
\gll a-ba-bara    a-bó    bá̲-hur-á̲    shunu\\
\textsc{aug}-\textsc{np}\textsubscript{2}-visitor  \textsc{aug}-\textsc{dem}.\textsc{iii}\textsubscript{2}  \textsc{sm}\textsubscript{2}.\textsc{rel}-arrive-\textsc{fv}  today\\
\glt ‘The visitors who will arrive today…’ (NF\_Elic15)\label{bkm:Ref467504624}
\z

The near future is also incompatible with negation. In order to negate a near future event, the near future prefix \textit{mbo-} is left out and the subjunctive form of the verb is used, which is preceded by a negated auxiliary \textit{ri} ‘be’ (see also \sectref{bkm:Ref494466580} on negation).

\ea
kàrì ndíkàâmbè\\
\gll ka-ri    ndí̲-ka-á̲mb-e\\
\textsc{neg}-be  \textsc{sm}\textsubscript{1SG}.\textsc{rel}-\textsc{dist}-speak-\textsc{pfv}.\textsc{sbjv}\\
\glt ‘I will not speak there.’ (NF\_Elic17)
\z

The incompatibility with subordinate clauses and with negation is also seen with the remote future construction: in this case, it relates to the origin of the remote future prefix as a marker of verb focus (see \sectref{bkm:Ref443303356}).

\subsection{Remote future}
\label{bkm:Ref443303356}\label{bkm:Ref489268446}\hypertarget{Toc75352687}{}
The form of the remote future construction differs between Zambian and Namibian Fwe. In Zambian Fwe, the remote future has the form na-\textsc{sm}-na-B-a, that is with a prefix \textit{na-} both in the pre-initial and the post-initial morpheme slot, as in (\ref{bkm:Ref494876959}--\ref{bkm:Ref494876960}).

\ea
\label{bkm:Ref494876959}
zyônà nàndínàménèkà\\
\gll zyóna    na-ndí̲-na-mének-a\\
tomorrow  \textsc{rem}-\textsc{sm}\textsubscript{1SG}-\textsc{rem}.\textsc{fut}-go\_early-\textsc{fv}\\
\glt ‘Tomorrow I will go very early.’ (ZF\_Elic14)
\z

\ea
\label{bkm:Ref494876960}
zyônà nàndínàbûːkà kàfôrù\\
\gll zyóna    na-ndí̲-na-búːk-a      ka-fóru\\
tomorrow  \textsc{rem}-\textsc{sm}\textsubscript{1SG}-\textsc{rem}.\textsc{fut}-wake-\textsc{fv}  at-four\\
\glt ‘Tomorrow I will wake up at four.’ (ZF\_Elic14)
\z

The pre-initial prefix \textit{na-} is the same remoteness marker that is used in the remote past perfective (see \sectref{bkm:Ref468174257}) and remote subjunctive (see \sectref{bkm:Ref492309168}), and is therefore glossed as ‘remote’ \textsc{rem}. The post-initial prefix \textit{na-} resembles the post-initial prefix \textit{na-} used in the near past perfective (see \sectref{bkm:Ref488767483}), though the near past perfective prefix \textit{na-} has an alternative realization \textit{a-}, whereas the remote future prefix \textit{na-} is consistently realized as \textit{na-}. Due to this difference in allomorphy, as well as the lack of (obvious) semantic connection between the near past perfective and remote future meanings, remote future \textit{na-} and near past perfective \textit{na-} are analyzed as distinct morphemes, and remote future \textit{na-} will be glossed as ‘remote future’ \textsc{rem}.\textsc{fut}.

The Zambian Fwe remote future construction takes melodic tone 2, a high tone on the subject marker, and maintains the verb’s underlying tones, as in (\ref{bkm:Ref75354891}--\ref{bkm:Ref75354892}).

\ea
\label{bkm:Ref75354891}
nàndínàóngòzà (cf. óngòzà ‘shout’)\\
\gll na-ndí̲-na-óngoz-a\\
\textsc{rem}-\textsc{sm}\textsubscript{1SG}-\textsc{rem}.\textsc{fut}-shout-\textsc{fv}\\
\glt ‘I will shout.’
\z

\ea
\label{bkm:Ref75354892}
nàndínàshòshòtà (cf. shòshòtà ‘whisper’)\\
\gll na-ndí̲-na-shoshot-a\\
\textsc{rem}-\textsc{sm}\textsubscript{1SG}-\textsc{rem}.\textsc{fut}-whisper -\textsc{fv}\\
\glt ‘I will whisper.’ (ZF\_Elic14)
\z

The Namibian Fwe remote future has a form (\textit{na}-)\textsc{sm}-\textit{ára}-B-\textit{a}, that is with a post-initial prefix \textit{ára-} rather than \textit{na-}, as seen in (\ref{bkm:Ref97909376}). The remoteness prefix \textit{na-} is optional in Namibian Fwe, and most often left out, as in (\ref{bkm:Ref97909381}).

\ea
\label{bkm:Ref97909376}
nàndíràcípàngà zyônà\\
\gll na-ndí̲-ra-cí-pang-a    zyóna\\
\textsc{rem}-\textsc{sm}\textsubscript{1SG}-\textsc{rem}.\textsc{fut}-\textsc{om}\textsubscript{7}-do-\textsc{fv}  tomorrow\\
\glt ‘I will do it tomorrow.’ (NF\_Elic17)
\z

\ea
\label{bkm:Ref97909381}
ndáràyèndà zyônà\\
\gll ndi-ára-end-a    zyóna\\
\textsc{sm}\textsubscript{1SG}-\textsc{rem}.\textsc{fut}-go-\textsc{fv}  tomorrow\\
\glt ‘I will go tomorrow.’ (NF\_Elic15)
\z

The prefix \textit{ára-} may also surface as \textit{ra-}, without the initial vowel \textit{á}, as in (\ref{bkm:Ref99972934}). The high tone of this vowel is maintained, though, and surfaces on the subject marker.

\ea
\label{bkm:Ref99972934}
ndáràtèndà {\textasciitilde} ndíràtèndà\\
\gll ndi-ára-tend-a\\
\textsc{sm}\textsubscript{1SG}-\textsc{rem}.\textsc{fut}-do-\textsc{fv}\\
\glt ‘I will do.’ (NF\_Elic15)
\z

Like the Zambian form, the Namibian Fwe form of the remote future maintains the lexical tone of the verb stem, as in (\ref{bkm:Ref99972951}--\ref{bkm:Ref99972953}).

\ea
\label{bkm:Ref99972951}
ndáràzyîmbà (cf. zyîmbà  ‘sing’)\\
\gll ndi-ára-zyímb-a\\
\textsc{sm}\textsubscript{1SG}-\textsc{rem}.\textsc{fut}-sing-\textsc{fv}\\
\glt ‘I will sing.’
\z

\ea
\label{bkm:Ref99972953}
ndáràtèndà (cf. tèndà ‘do’)\\
\gll ndi-ára-tend-a\\
\textsc{sm}\textsubscript{1SG}-\textsc{rem}.\textsc{fut}-do-\textsc{fv}\\
\glt ‘I will do.’ (NF\_Elic15)
\z

The loss of the vowel \textit{á} of the prefix \textit{ára-}, and the subsequent use of the high tone on the subject marker, may also explain why the subject marker of the remote future construction in Zambian Fwe is high-toned, if the Zambian prefix \textit{na-} derives from an earlier *\textit{ána-} or *\textit{ára\--}, with subsequent vowel loss.

The interpretation of the remote future construction is the same for Zambian and Namibian Fwe: it situates the entire event in the remote future with respect to the utterance time. Remote future is usually interpreted as at least one day after UT, for instance, ‘tomorrow’, in (\ref{bkm:Ref492310429}), or ‘next week’, in (\ref{bkm:Ref505766596}).

\ea
\label{bkm:Ref492310429}
mùrâːrè twáràzíkàndèkà zyônà\\
\gll mu-rá̲ːr-e      tu-ára-zí-kandek-a      zyóna\\
\textsc{sm}\textsubscript{2PL}-sleep-\textsc{pfv}.\textsc{sbjv}  \textsc{sm}\textsubscript{1PL}-\textsc{rem}.\textsc{fut}-\textsc{om}\textsubscript{8}-tell-\textsc{fv}  tomorrow\\
\glt ‘Go to sleep, we’ll discuss it tomorrow.’ (NF\_Narr15)
\z

\ea
\label{bkm:Ref505766596}
ènsúndá yìkêːzyà nàndínàyà kùbàmàtè\\
\gll e-N-sundá    i-ké̲ːzy-a    na-ndí̲-na-i-a ku-ba-mate \\
\textsc{aug}-\textsc{np}\textsubscript{9}-week  \textsc{sm}\textsubscript{9}-come-\textsc{fv}  \textsc{rem}-\textsc{sm}\textsubscript{1SG}-\textsc{rem}.\textsc{fut}-go-\textsc{fv}
\textsc{np}\textsubscript{17}-\textsc{np}\textsubscript{2}-Mate\\
\glt ‘Next week I will go to Mate.’ (ZF\_Elic14)
\z

Like the remote past, the remote future can be used for any time frame that the speaker considers to be far in the future. In (\ref{bkm:Ref506461735}), the speaker is discussing a house that is currently being built, but has not been completed yet, and therefore the statement about the house is set in the remote future.

\ea
\label{bkm:Ref506461735}
yáràdùrà cáhà\\
\gll i-ára-dur-a        cahá\\
\textsc{sm}\textsubscript{9}-\textsc{rem}.\textsc{fut}-be\_expensive-\textsc{fv}  very\\
\glt ‘It will be very expensive.’ (about a house that is currently being built) (NF\_Elic15)
\z

As discussed in \sectref{bkm:Ref72233436}, remote future meaning can also be expressed by the present construction, without a difference in meaning, as in (\ref{bkm:Ref469492825}--\ref{bkm:Ref488771786}).

\ea
\label{bkm:Ref469492825}
ndìtwá zyônà\\
\gll ndi-tw-á̲    zyóna\\
\textsc{sm}\textsubscript{1SG}-pound-\textsc{fv}  tomorrow\\
\glt ‘I will pound tomorrow.’
\z

\ea
ndáràtwá zyônà\\
\gll ndi-ára-tw-á      zyóna\\
\textsc{sm}\textsubscript{1SG}-\textsc{rem}.\textsc{fut}-pound-\textsc{fv}  tomorrow\\
\glt ‘I will pound tomorrow.’ (NF\_Elic15)
\z

\ea
\label{bkm:Ref488771786}
ndìyêndà zyônà\\
\gll ndi-é̲nd-a  zyóna\\
\textsc{sm}\textsubscript{1SG}-go-\textsc{fv}  tomorrow\\
\glt ‘I will go tomorrow.’ (ZF\_Elic14)
\z

The remote future form cannot be used in subordinate clauses. To indicate a remote future event in a subordinate clause, Fwe uses either the present construction, as in (\ref{bkm:Ref488771878}), or a subjunctive construction with the remoteness prefix \textit{na-}, as in (\ref{bkm:Ref72247790}).

\ea
\label{bkm:Ref488771878}
ndìzyónà ndíyêndà\\
\gll ndi-zyóna    ndí̲-é̲nd-a\\
\textsc{cop}-tomorrow  \textsc{sm}\textsubscript{1SG}.\textsc{rel}-go-\textsc{fv}\\
\glt ‘It’s tomorrow that I will go.’
\z

\ea
\label{bkm:Ref72247790}
ndìzyónà nàndíyêndè\\
\gll ndi-zyóna    na-ndí̲-é̲nd-e\\
\textsc{cop}-tomorrow  \textsc{rem}-\textsc{sm}\textsubscript{1SG}.\textsc{rel}-go-\textsc{pfv}.\textsc{sbjv}\\
\glt ‘It’s tomorrow that I will go.’ (NF\_Elic15)
\z

The remote future form is also incompatible with negation. Instead, a negated auxiliary \textit{ri} ‘be’ is used followed by a subjunctive verb with the remoteness prefix \textit{na-}, as in (\ref{bkm:Ref99976515}).

\ea
\label{bkm:Ref99976515}
kàrì nèndícìpángè zyônà\\
\gll ka-ri    ne-ndí̲-ci\textsubscript{H}-pá̲ng-e      zyóna\\
\textsc{neg}-be  \textsc{rem}-\textsc{sm}\textsubscript{1SG}-\textsc{om}\textsubscript{7}-do-\textsc{pfv}.\textsc{sbjv}  tomorrow\\
\glt ‘I will not do it tomorrow.’ (NF\_Elic17)
\z

That the remote future form is not allowed in subordinate clauses, and cannot be negated, is related to its origin as a former marker of verb focus. As already discussed in \sectref{bkm:Ref72233436}, the remote future prefix \textit{ára}- is cognate with a marker of verb focus in other Bantu Botatwe languages; in Fwe, it has become a marker of remote future, but its incompatibility with negation and subordination is a relic of its earlier function as a marker of verb focus. The reanalysis of the earlier focused present as remote future is related to the development of a new strategy of verb focus, the fronted-infinitive construction (see \sectref{bkm:Ref492314081}).

\section{Consecutive}
\label{bkm:Ref494204746}\hypertarget{Toc75352688}{}
Fwe has a consecutive verb form, which is, both in form and function, intermediate between an inflected and an infinitive verb form. Temporally, the consecutive situates the event relative to an event encoded with an inflected verb that occurs earlier in the same discourse. Despite this relative lack of underspecification for tense, the consecutive displays interesting interactions with preceding verbs that are inflected for tense, and therefore the consecutive construction will be discussed in this chapter.

Formally, the consecutive consists of an infinitive verb preced by a connective or a comitative clitic. The connective clitic consists of a connective stem and a pronominal prefix (see \sectref{bkm:Ref492133189} on connectives), which in the consecutive verb marks agreement with the intended subject. An example is given in (\ref{bkm:Ref469317924}), where the consecutive verb \textit{yókúfwà} ‘and then it died’ is marked with a class 9 pronominal prefix referring back to its intended subject \textit{ènjókà} ‘the snake’.

\ea
\label{bkm:Ref469317924}
ndàmání kùyídàmá ènjókà yókúfwà\\
\gll ndi-a-man-í̲      ku-í-dam-á    e-N-jóka                                                                                                                                              í-o=ku-fw-á \\
\textsc{sm}\textsubscript{1SG}-\textsc{pst}-finish-\textsc{npst}.\textsc{pfv}  \textsc{inf}-\textsc{om}\textsubscript{9}-beat-\textsc{fv}  \textsc{aug}-\textsc{np}\textsubscript{9}-snake \textsc{pp}\textsubscript{9}-\textsc{con}=\textsc{inf}-die-\textsc{fv}\\
\glt ‘I finished beating the snake, and it died.’ (ZF\_Narr13)
\z

Instead of the connective clitic, consecutives may also take a comitative clitic \textit{no-} (see also \sectref{bkm:Ref486270340} on comitatives), as in (\ref{bkm:Ref492133792}).

\ea
\label{bkm:Ref492133792}
nàháshàmì \textbf{nòkùkárìsà} kùzyîmbà\\
\gll na-ásham-i        no=ku-káris-a  ku-zyímb-a\\
\textsc{sm}\textsubscript{1}.\textsc{pst}-open\_mouth-\textsc{npst}.\textsc{pfv}  \textsc{com}=\textsc{inf}-start-\textsc{fv}  \textsc{inf}-sing-\textsc{fv}\\
\glt ‘She opens her mouth \textbf{and} \textbf{starts} to sing.’ (ZF\_Elic14)
\z

As the base of the consecutive verb form is an infinitive verb, it displays the typical properties of infinitive verbs, namely lack of melodic tone (see also \sectref{bkm:Ref71539267} on melodic tone in TAM constructions), and the replacement of the infinitive prefix \textit{ku-} with the distal prefix \textit{ka-} to expresses an event taking place away from the place of speaking (see \sectref{bkm:Ref489965878} on the distal). An example of a consecutive using the distal infinitive \textit{ka-} is given in (\ref{bkm:Ref469318954}).

\newpage
\ea
\label{bkm:Ref469318954}
àhà bákàsúkꜝáhò \textbf{bókàyèndà} kàhùrà kúmùnzì\\
\gll a-ha    bá̲-ka-sú̲k-a=hó ba-ó=ka-end-a    ka-hur-a    kú-mu-nzi\\
\textsc{aug}-\textsc{dem}.\textsc{i}\textsubscript{16}  \textsc{sm}\textsubscript{2}.\textsc{rel}-\textsc{dist}-disembark-\textsc{fv}=\textsc{loc}\textsubscript{16}
\textsc{pp}\textsubscript{2}-\textsc{con}=\textsc{inf}.\textsc{dist}-go-\textsc{fv}  \textsc{inf}.\textsc{dist}-arrive-\textsc{fv}  \textsc{np}\textsubscript{17}-\textsc{np}\textsubscript{3}-village\\
\glt ‘When they climbed out of the canoe, \textbf{then} \textbf{they} \textbf{walked} and arrived home.’ (NF\_Narr15)
\z

A consecutive verb can only be used when preceded by another, tense-in\-flect\-ed verb, and the consecutive verb is interpreted as occuring more or less directly after the event encoded by the inflected verb. In (\ref{bkm:Ref492135151}), the remote past perfective verb \textit{níndàzyáːkà} ‘I built’ describes an event immediately followed by that of the consecutive \textit{ndókùyíǀàpàùrà} ‘I took it apart’.

\ea
\label{bkm:Ref492135151}
\label{bkm:Ref507580705}
níndàzyáːk’ ènjûò ndókùyíǀàpàùrà hápè\\
\gll ni-ndí̲-a-zyáːk-a    e-N-júo ndi-ó=ku-í-ǀap-a-ur-a \\
\textsc{rem}-\textsc{sm}\textsubscript{1}-\textsc{pst}-build-\textsc{fv}  \textsc{aug}-\textsc{np}\textsubscript{9}-house
\textsc{pp}\textsubscript{1SG}-\textsc{con}=\textsc{inf}-\textsc{om}\textsubscript{9}-destroy-\textsc{pl}1-\textsc{sep}.\textsc{tr}-\textsc{fv}\\
\glt ‘I built a house, then I took it apart again.’ (NF\_Elic15)
\z

When the consecutive is preceded by a perfective verb, such as the remote past perfective in (\ref{bkm:Ref507580705}), the event expressed by the consecutive directly follows the event expressed by the inflected verb. When preceded by an imperfective verb, on the other hand, the event encoded by the consecutive is interpreted as co-occurring with it. This is illustrated with a stative verb \textit{kàndíyèndètè} ‘I was on a walk’, in (\ref{bkm:Ref452642494}), and an imperfective past verb \textit{kàndíshâmbà} ‘I was swimming’, in (\ref{bkm:Ref431990473}).

\ea
\label{bkm:Ref452642494}
zyônà \textbf{kàndíyèndètè} mùtêmwà ndókùshótòkà zyôkà\\
\gll zyóna    ka-ndí̲-end-ete    mu-témwa ndí-o=ku-shótok-a    ∅-zyóka \\
yesterday  \textsc{pst}.\textsc{ipfv}-\textsc{sm}\textsubscript{1SG}-go-\textsc{stat}  \textsc{np}\textsubscript{3}-bush
\textsc{pp}\textsubscript{1SG}-\textsc{con}=\textsc{inf}-jump-\textsc{fv}  \textsc{np}\textsubscript{5}-snake\\
\glt ‘Yesterday I was on a walk in the bush, and I stepped on a snake.’ (ZF\_Narr14)
\z

\newpage
\ea
\label{bkm:Ref431990473}
àhà \textbf{kàndíshâmbà} ndókùbón’ òngwènà\\
\gll a-ha    ka-ndí̲-shá̲mb-a ndi-ó=ku-bón-a    o-∅-ngwena \\
\textsc{aug}-\textsc{dem}.\textsc{i}\textsubscript{16}  \textsc{pst}.\textsc{ipfv}-\textsc{sm}\textsubscript{1SG}-swim-\textsc{fv}
\textsc{pp}\textsubscript{1SG}-\textsc{con}=\textsc{inf}-see-\textsc{fv}  \textsc{aug}-\textsc{np}\textsubscript{1a}-crocodile\\
\glt ‘While I was swimming, I saw a crocodile.’ (ZF\_Elic14)
\z

Multiple consecutive verbs can be used in succession, as in (\ref{bkm:Ref98512958}), which is taken from the start of a narrative and describes the various steps of a marriage contract, using a tense-inflected verb followed by three consecutive verbs.

\ea
\label{bkm:Ref98512958}
àkéːzyà kùmùshàkà bókùmùtòmènà ákùmànà kùróbòrà nòkútéyè àhíndè mùkéntù wàkwé cwárè àyêndè\\
\gll a-kéːzy-a    ku-mu-shak-a    ba-ó=ku-mu-tomen-a a-ó=ku-man-a    ku-róbor-a    no=kú-t-a    íye a-hínd-e    mu-kéntu  u-akwé  cwáre  a-énd-e\\
\textsc{sm}\textsubscript{1}-come-\textsc{fv}  \textsc{inf}-\textsc{om}\textsubscript{1}-propose-\textsc{fv}  \textsc{pp}\textsubscript{2}-\textsc{con}=\textsc{inf}-\textsc{om}\textsubscript{1}-charge\_dowry-\textsc{fv}
\textsc{pp}\textsubscript{1}-\textsc{con}=\textsc{inf}-finish-\textsc{fv}  \textsc{inf}-pay\_dowry-\textsc{fv}  \textsc{com}=\textsc{inf}-say-\textsc{fv}  that
\textsc{sm}\textsubscript{1}-take-\textsc{pfv}.\textsc{sbjv}  \textsc{np}\textsubscript{1}-woman  \textsc{pp}\textsubscript{1}-\textsc{poss}\textsubscript{3SG}  then  \textsc{sm}\textsubscript{1}-go-\textsc{pfv}.\textsc{sbjv}\\
\glt ‘He came to propose to her, then they charged him dowry, then he finished paying the dowry, then they said he can take his wife and go.’ (NF\_Narr15)
\z

Since subject marking is not possible on the comitative-marked consecutive, it is usually interpreted as having the same subject as the preceding, inflected verb, as in (\ref{bkm:Ref451515777}), or even the same subject and object as the preceding inflected verb, as in (\ref{bkm:Ref451515778}).

\ea
\label{bkm:Ref451515777}
àkàrôngò kànâgwì nòkúfwà\\
\gll a-ka-róngo    ka-ná̲-gw-i      no=ku-fú-a\\
\textsc{aug}-\textsc{np}\textsubscript{12}-pot  \textsc{sm}\textsubscript{12}-\textsc{pst}-fall-\textsc{npst}.\textsc{pfv}  \textsc{com}=\textsc{inf}-die-\textsc{fv}\\
\glt ‘The pot fell, and it broke.’ (ZF\_Elic14)
\z

\ea
\label{bkm:Ref451515778}
ndìnàhîndì nsânzù nòkùbíːkà hàzìkù\\
\gll ndi-na-hí̲nd-i    N-sánzu  no=ku-bíːk-a    ha-∅-ziku\\
\textsc{sm}\textsubscript{1SG}-\textsc{pst}-take-\textsc{npst}.\textsc{pfv}  \textsc{np}\textsubscript{9}-wood  \textsc{com}=\textsc{inf}-put-\textsc{fv}  \textsc{np}\textsubscript{16}-\textsc{np}\textsubscript{5}-hearth\\
\glt ‘I took a piece of wood and put it on the fire.’ (ZF\_Elic14)
\z

Given appropriate context, the comitative-marked consecutive may also be used for verbs that have a different intended subject, as in (\ref{bkm:Ref70073741}), where the preceding two verbs (in the present and consecutive form respectively) are marked for a first person singular subject, but the last verb, a comitative-marked consecutive, has as its intended subject not the speaker himself, but a snake, whose encounter was the topic of the story.

\ea
\label{bkm:Ref70073741}
àhá ndíìbùkùmá bùrỳahò ndókùyídàmà nòkúfwà\\
\gll a-ha    ndí̲-i\textsubscript{H}-bu\textsubscript{H}kum-á̲    buryaho ndi-ó=ku-í-dam-a      no=ku-fú-a\\
\textsc{aug}-\textsc{dem}.\textsc{i}\textsubscript{16}  \textsc{sm}\textsubscript{1SG}-\textsc{om}\textsubscript{5}-throw-\textsc{fv}  \textsc{np}\textsubscript{14}-like\_that
\textsc{pp}\textsubscript{1SG}-\textsc{con}=\textsc{inf}-\textsc{om}\textsubscript{9}-hit-\textsc{fv}  \textsc{com}=\textsc{inf}-die-\textsc{fv}\\
\glt ‘When I threw it, I hit the snake and it [=the snake] died.’ (ZF\_Narr13)
\z

The comitative-marked consecutive is only allowed when context is sufficient to establish the intended subject, either through the preceding inflected verb, or through the wider (discourse-internal or external) context. (\ref{bkm:Ref451515804}) was considered ungrammatical, because the lack of context does not provide enough clues to correctly identify the buffalo as the intended subject of the verb.

\ea
\label{bkm:Ref451515804}
*ndàshónjì ònyátì nòkúfwà\\
\gll ndi-a-shónj-i    o-∅-nyáti    no=ku-fú-a\\
\textsc{sm}\textsubscript{1SG}-\textsc{pst}-shoot-\textsc{npst}.\textsc{pfv}  \textsc{aug}-\textsc{np}\textsubscript{1a}-buffalo  \textsc{com}=\textsc{inf}-die-\textsc{fv}\\
\glt Intended: ‘I shot a buffalo and it [not I] died.’ (ZF\_Elic14)
\z

