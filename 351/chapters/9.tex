\chapter{Aspect}
\label{bkm:Ref99112358}\hypertarget{Toc75352689}{}
In this chapter, I discuss different ways in which Fwe verbs can be inflected for aspect, specifying the internal temporal structure of the verb. In Fwe, aspect can be expressed morphologically, with pre- and post-initial verbal prefixes, or with verbal suffixes, and periphrastically with an auxiliary combined with an inflected or infinitive main verb. Melodic tone, which plays an important role in the expression of tense constructions, is only seen in the aspectual construction expressing a stative. \tabref{tab:9:1} summarizes the aspect constructions used in Fwe, which will be discussed in this chapter.

\begin{table}
\label{bkm:Ref494478974}\caption{\label{tab:9:1}Aspect constructions}
\begin{tabularx}{\textwidth}{lllQ}
\lsptoprule
Label & Segmental form & Melodic tone & Interpretation\\
\midrule
Progressive & auxiliary \textit{kwesi} & - & progressive; inchoative; repetitive\\
Fronted-infinitive & \textit{ku}-B-\textit{a} \textsc{sm}-B-\textit{a} & - & progressive; verb focus\\
Habitual & \textit{-ang} & - & habitual\\
Habitual & {\itshape náku-} & - & habitual  \\
Stative & {\itshape -ite} & 3,4 & stative; progressive\\
Persistive & {\itshape shí-} & - & persistive\\
Inceptive & {\itshape sha-/she-/shi-} & - & inchoative; proximative; contrastive; completive\\
\lspbottomrule
\end{tabularx}
\end{table}
\section{Progressive}
\label{bkm:Ref445905308}\hypertarget{Toc75352690}{}
Fwe has two constructions that express progressive aspect, indicating an ongoing event; a construction with an auxiliary \textit{kwesi} followed by an inflected main verb, and a fronted infinitive construction involving a finite verb preceded by an infinitive verb of the same stem. Progressive aspect is a subtype of imperfective aspect, and as such progressive constructions may not be used with tense and mood constructions that also express perfectivity.

\subsection{Progressive auxiliary}
\label{bkm:Ref431917333}\label{bkm:Ref431917326}\hypertarget{Toc75352691}{}
Progressive aspect can be expressed with the auxiliary \textit{kwesi} followed by an inflected lexical verb, as in (\ref{bkm:Ref99095165}--\ref{bkm:Ref99095167}). Both the auxiliary and main verb are inflected for subject, indicated by coreferential subject markers. Neither verb is subordinate to the other, as both verbs have the tonal marking of a main clause verb, and not that of a relative clause verb, e.g. they lack a high tone on the subject marker (see \sectref{bkm:Ref491095705} on relative clauses).

\ea
\label{bkm:Ref99095165}
òmvúrà àkwèsì àshôkà\\
\gll o-∅-mvúra    a-kwesi  a-shó̲k-a\\
\textsc{aug}-\textsc{np}\textsubscript{1a}-rain  \textsc{sm}\textsubscript{1a}-\textsc{prog}  \textsc{sm}\textsubscript{1a}-rain-\textsc{fv}\\
\glt ‘It is raining.’ (ZF\_Elic14)
\z

\ea
\label{bkm:Ref99095167}
ndìkwèsì ndìrìkúkà\\
\gll ndi-kwesi  ndi-ri\textsubscript{H}kuk-á̲\\
\textsc{sm}\textsubscript{1SG}-\textsc{prog}  \textsc{sm}\textsubscript{1SG}-have\_hiccups-\textsc{fv}\\
\glt ‘I have the hiccups.’ (NF\_Elic15)
\z

The progressive auxiliary \textit{kwesi} is also used in Fwe as a lexical verb with the meaning ‘have’. It derives from the verb \textit{kwát} ‘grasp’, with an imbricated stative suffix -\textit{ite} (see \sectref{bkm:Ref431984198} on the stative). Similar forms are seen in Totela, which uses \textit{kwesi} (as the stative of \textit{kwata}) \citep[674]{Crane2019} as a progressive auxiliary, and in Subiya, which uses an auxiliary \textit{kwete,} derived from \textit{ku kwata} ‘to grab’ \citep[64]{Jacottet1896}.

An object marker cannot be used on the progressive auxiliary, only on the lexical verb, as shown with the object marker \textit{ndi-} in (\ref{bkm:Ref492316635}).

\ea
\label{bkm:Ref492316635}
àkwèsì àndìàmbìsâ\\
\gll a-kwesi  a-ndi-amb-is-á̲\\
\textsc{sm}\textsubscript{1}-\textsc{prog}  \textsc{sm}\textsubscript{1}-\textsc{om}\textsubscript{1SG}-talk-\textsc{caus}-\textsc{fv}\\
\glt ‘S/he is talking to me.’ (NF\_Elic15)
\z

The same is true for the locative clitic, which may only be used on the second, lexical verb when it has locative reference, as in (\ref{bkm:Ref492316654}--\ref{bkm:Ref72249434}). A locative clitic of class 17 \textit{=ko}, however, may be used on the auxiliary \textit{kwesi} to focus the progressive aspect, as in (\ref{bkm:Ref492316669}--\ref{bkm:Ref72249558}).\footnote{Though the locative clitic is synchronically only used with the progressive to express aspect focus, it is likely that it was obligatory in an earlier form of the construction, as progressive constructions very often develop out of earlier locative constructions (cf. \citealt{BybeeEtAl1994}: 127-133).}

\ea
\label{bkm:Ref492316654}
ndìkwèsì ndìngòngòtáhò\\
\gll ndi-kwesi  ndi-ngo\textsubscript{H}ngot-a=hó̲\\
\textsc{sm}\textsubscript{1SG}-\textsc{prog}  \textsc{sm}\textsubscript{1SG}-knock-\textsc{fv}=\textsc{loc}\textsubscript{16}\\
\glt ‘I am knocking on it.’
\z

\ea
ndìkwèsì ndìngòngòtákò\\
\gll ndi-kwesi  ndi-ngo\textsubscript{H}ngot-a=kó̲\\
\textsc{sm}\textsubscript{1SG}-\textsc{prog}  \textsc{sm}\textsubscript{1SG}-knock-\textsc{fv}=\textsc{loc}\textsubscript{17}\\
\glt ‘I am knocking there.
\z

\ea
\label{bkm:Ref72249434}
bàkwèsì bàràːrámò\\
\gll ba-kwesi  ba-raː\textsubscript{H}r-a=mó̲\\
\textsc{sm}\textsubscript{2}-\textsc{prog}  \textsc{sm}\textsubscript{2}-sleep-\textsc{fv}=\textsc{loc}\textsubscript{18}\\
\glt ‘S/he is sleeping in there
\z

\ea
\label{bkm:Ref492316669}
ndìkwèsìkó ndìngòngótà\\
\gll ndi-kwesi=kó̲  ndi-ngo\textsubscript{H}ngot-a=kó̲\\
\textsc{sm}\textsubscript{1SG}-\textsc{prog}=\textsc{loc}\textsubscript{17}  \textsc{sm}\textsubscript{1SG}-knock-\textsc{fv}=\textsc{loc}\textsubscript{17}\\
\glt ‘I am knocking there (for a long time).’
\z

\ea
bàkwèsìkó bàhíkà\\
\gll ba-kwesi=kó̲    ba-hi\textsubscript{H}k-á̲\\
\textsc{sm}\textsubscript{2}-\textsc{prog}=\textsc{loc}\textsubscript{17}  \textsc{sm}\textsubscript{2}-cook-\textsc{fv}\\
\glt ‘They are busy cooking.’ (stresses that they have already started)
\z

\ea
\label{bkm:Ref72249558}
ndìshìní òkùmànà ndìshìkwèsìkó ndìhíkà\\
\gll ndi-shi\textsubscript{H}-ní  o-ku-man-a    ndi-shi\textsubscript{H}-kwesi=kó̲    ndi-hi\textsubscript{H}k-á̲\\
\textsc{sm}\textsubscript{1SG}-\textsc{per}-be  \textsc{aug}-\textsc{inf}-finish-\textsc{fv}  \textsc{sm}\textsubscript{1SG}-\textsc{per}-\textsc{prog}=\textsc{loc}\textsubscript{17}  \textsc{sm}\textsubscript{1SG}-cook-\textsc{fv}\\
\glt ‘I have not yet finished, I am still cooking.’ (Answer to: ‘Did you finish cooking?’) (NF\_Elic17
\z

Fwe has another progressive auxiliary \textit{iná}, which also functions as a lexical verb ‘be at’. The progressive auxiliary \textit{iná} is used in much the same way as \textit{kwesi}, i.e. it is followed by a non-subordinate inflected lexical verb. There appears to be no difference in meaning between the two auxiliaries. (\ref{bkm:Ref494878236}--\ref{bkm:Ref72249618}) illustrate the use of both progressive auxiliaries.

\ea
\label{bkm:Ref494878236}
ndìkwèsì ndìfwêbà\\
\gll ndi-kwesi  ndi-fwé̲b-a\\
\textsc{sm}\textsubscript{1SG}-\textsc{prog}  \textsc{sm}\textsubscript{1SG}-smoke-\textsc{fv}\\
\glt ‘I am smoking.’
\z

\ea
\label{bkm:Ref72249618}
ndìná ndìfwêbà\\
\gll ndi-iná  ndi-fwé̲b-a\\
\textsc{sm}\textsubscript{1SG}-\textsc{prog}  \textsc{sm}\textsubscript{1SG}-smoke-\textsc{fv}\\
\glt ‘I am smoking.’ (NF\_Elic17)
\z

The only established difference between the progressive auxiliaries \textit{kwesi} and \textit{iná} is that where \textit{kwesi} combines with the class 17 locative clitic =\textit{ko} to focus the progressive aspect (see (\ref{bkm:Ref492316669})), \textit{iná} takes the locative clitic of class 16 =\textit{ho} to focus the progressive aspect, as in (\ref{bkm:Ref494878422}).

\ea
\label{bkm:Ref494878422}
ndìná ndìfwêbà\\
\gll ndi-ina=hó̲    ndi-fwé̲b-a\\
\textsc{sm}\textsubscript{1SG}-\textsc{prog}=\textsc{loc}\textsubscript{16}  \textsc{sm}\textsubscript{1SG}-smoke-\textsc{fv}\\
\glt ‘I am smoking.’ (NF\_Elic17)
\z

The use of progressive \textit{iná} appears to be restricted. I have not found this construction with any Zambian speakers, and with only one of the Namibian speakers that were interviewed. Other Namibian Fwe speakers accepted the construction but would only use \textit{kwesi} in their own speech. More research is needed to establish if the auxiliary \textit{iná} is really functionally equivalent to the auxiliary \textit{kwesi} (as it appears to be), and, if there is a geographic dimension to the use of these two progressive auxiliaries, what their distribution is.

The progressive auxiliary \textit{kwesi} marks an ongoing and durative event, meaning that it cannot be instantaneous, but has to cover a certain time span. With dynamic verbs, it typically presents the nuclear phase as ongoing, as in (\ref{bkm:Ref72249792}--\ref{bkm:Ref72249793}).

\ea
\label{bkm:Ref72249792}
òmvúrà àkwèsì àshókà\\
\gll o-∅-mvúra    a-kwesi  a-sho\textsubscript{H}k-á̲\\
\textsc{aug}-\textsc{np}\textsubscript{1a}-rain  \textsc{sm}\textsubscript{1}-\textsc{prog}  \textsc{sm}\textsubscript{1}-fall-\textsc{fv}\\
\glt ‘It’s raining (right now).’ (ZF\_Elic14)
\z

\ea
\label{bkm:Ref72249793}
èfónì yòzyûmwì ìkwès’ ìrírà\\
\gll e-∅-fóni    i-o=zyú-mwi  i-kwesi  i-rir-á̲\\
\textsc{aug}-\textsc{np}\textsubscript{9}-phone  \textsc{pp}\textsubscript{9}-\textsc{con}=\textsc{pp}\textsubscript{1}-other  \textsc{sm}\textsubscript{9}-\textsc{prog}  \textsc{sm}\textsubscript{9}-cry-\textsc{fv}\\
\glt ‘Someone’s phone is ringing.’ (in a room, you hear a phone ringing) (NF\_Elic15)
\z

Progressive aspect is most typically used with dynamic verbs \citep[35]{Comrie1976}, but Fwe also allows the use of progressives with change-of-state verbs. The use of \textit{kwesi} with change-of-state verbs that have an onset gives an inchoative interpretation: it presents the onset phase, which describes the phase leading up to the change in state, as ongoing, as in (\ref{bkm:Ref98513008}--\ref{bkm:Ref497242247}).

\ea
\label{bkm:Ref98513008}
bàkwèsì bàsèpàhárà\\
\gll ba-kwesi  ba-sep-ahar-á̲\\
\textsc{sm}\textsubscript{2}-\textsc{prog}  \textsc{sm}\textsubscript{2}-trust-\textsc{neut}-\textsc{fv}\\
\glt ‘S/he is becoming important.’
\z

\ea
\label{bkm:Ref497242247}
cìkwèsì cìcénà\\
\gll ci-kwesi  ci-cen-á̲\\
\textsc{sm}\textsubscript{7}-\textsc{prog}  \textsc{sm}\textsubscript{7}-become\_clean-\textsc{fv}\\
\glt ‘It is becoming clean.’ (while you are washing it, you see it getting cleaner) (NF\_Elic17)
\z

With change-of-state verbs that do not have an onset phase, the progressive gives a repetitive interpretation, as illustrated with the change-of-state verb \textit{aruk} ‘open’ in (\ref{bkm:Ref497240795}), and the change-of-state verb \textit{ráːr} ‘sleep/fall asleep’ in (\ref{bkm:Ref497240808}).

\ea
\label{bkm:Ref497240795}
cìkwèsì cìàrúkà\\
\gll ci-kwesi  ci-ar-uk-á̲\\
\textsc{sm}\textsubscript{7}-\textsc{prog}  \textsc{sm}\textsubscript{7}-close -\textsc{sep}.\textsc{intr}-\textsc{fv}\\
\glt ‘It keeps opening.’ (of a door that doesn’t close properly)
\z

\ea
\label{bkm:Ref497240808}
bàkwèsì bàràːrámò\\
\gll ba-kwesi  ba-raː\textsubscript{H}r-a-mó̲\\
\textsc{sm}\textsubscript{2}-\textsc{prog}  \textsc{sm}\textsubscript{2}-sleep-\textsc{fv}\\
\glt ‘S/he is sleeping in in there [for the duration of his/her stay].’ (of someone who is a temporary guest) (NF\_Elic17)
\z

The repetitive interpretation of progressives with change-of-state verbs can also mean that the event has multiple subjects. This is shown with the change-of-state verb \textit{fw} ‘die’ in (\ref{bkm:Ref99095435}), which can be used with the progressive when it has a plural subject.

\ea
\label{bkm:Ref99095435}
bàkwèsì bàfwâ\\
\gll ba-kwesi  ba-fw-á̲\\
\textsc{sm}\textsubscript{2}-\textsc{prog}  \textsc{sm}\textsubscript{2}-die-\textsc{fv}\\
\glt ‘They are dying.’ (NF\_Elic17)
\z

The progressive examples seen so far involved present progressives, which present ongoing actions set at or around the time of speaking. \textit{kwesi} can also be combined with a past construction, in which case the auxiliary takes the (remote) past imperfective prefix \textit{ka-}, as in (\ref{bkm:Ref99095463}). The auxiliary also takes the melodic tone of the RPI, with a high tone on the subject marker and a high tone on the last mora. \textit{kwesi} is not used with the near past imperfective.

\ea
\label{bkm:Ref99095463}
àhà \textbf{kàtúkwèsí} tùkàndèká èzìntù nòkùkárìsà kùkákànà\\
\gll a-ha    ka-tú̲-kwesí̲    tu-kandek-á̲  e-zi-ntu no=ku-káris-a    ku-kákan-a\\
\textsc{aug}-\textsc{dem}\textsubscript{16}  \textsc{pst}.\textsc{ipfv}-\textsc{sm}\textsubscript{1PL}-\textsc{prog}  \textsc{sm}\textsubscript{1PL}-tell-\textsc{fv}  \textsc{aug}-\textsc{np}\textsubscript{8}-thing
\textsc{com}=\textsc{aug}-\textsc{inf}-start-\textsc{fv}  \textsc{inf}-argue-\textsc{fv}\\
\glt ‘When we were discussing things, we started arguing.’ (ZF\_Elic14)
\z

The auxiliary \textit{kwesi} is mainly used for events that have a relatively short duration, such as smoking a cigarette, as in (\ref{bkm:Ref468437797}), or getting dressed, as in (\ref{bkm:Ref492317059}). Progressive events with a longer duration tend to be expressed with the fronted-infinitive construction (see \sectref{bkm:Ref492317075}).

\ea
\label{bkm:Ref468437797}
bàkwèsì bàfwébà mùtômbwè\\
\gll ba-kwesi  ba-fwé̲b-a    mu-tómbwe\\
\textsc{sm}\textsubscript{2}-\textsc{prog}  \textsc{sm}\textsubscript{2}-smoke-\textsc{fv}  \textsc{np}\textsubscript{3}-cigarette\\
\glt ‘S/he is smoking a cigarette.’
\z

\ea
\label{bkm:Ref492317059}
wáshàkàbìrì múnjûò kwìn’ ózyò ákwèsì àzwâtà\\
\gll o-ásha-kabir-i      mú-N-júo ku-iná  o-zyo    á̲-kwesi    a-zwá̲t-a \\
\textsc{sm}\textsubscript{2SG}-\textsc{neg}.\textsc{sbjv}-enter-\textsc{neg}  \textsc{np}\textsubscript{18}-\textsc{np}\textsubscript{9}-house
\textsc{sm}\textsubscript{17}-be\_at  \textsc{aug}-\textsc{dem}.\textsc{iii}\textsubscript{1}  \textsc{sm}\textsubscript{1}.\textsc{rel}-\textsc{prog}  \textsc{sm}\textsubscript{1}-dress-\textsc{fv}\\
\glt ‘Don’t go in the house, there is someone getting dressed.’ (NF\_Elic17)
\z
\subsection{Fronted infinitive construction}
\label{bkm:Ref492314081}\hypertarget{Toc75352692}{}\label{bkm:Ref71533571}\label{bkm:Ref492317075}
The fronted-infinitive construction (FIC) is used to mark progressive aspect or verb focus. This construction consists of an inflected lexical verb immediately preceded by an infinitive copy of the same verb stem. For a detailed analysis of the fronted-infinitive construction in Fwe, see {\citet{Gunnink2019}}. Examples of the FIC are given in (\ref{bkm:Ref71280638}--\ref{bkm:Ref71280639}).

\ea
\label{bkm:Ref71280638}
shùnù kùsèbèzà ndísèbèzâ\\
\gll shunu  ku-sebez-a  ndí̲-sebez-á̲\\
today  \textsc{inf}-work-\textsc{fv}  \textsc{sm}\textsubscript{1SG}.\textsc{rel}-work-\textsc{fv}\\
\glt ‘Today I am working.’ (ZF\_Elic14)
\z

\ea
\label{bkm:Ref71280639}
kùshèkà báꜝshékà\\
\gll ku-shek-a    bá̲-shek-á̲\\
\textsc{inf}-laugh-\textsc{fv}  \textsc{sm}\textsubscript{2}.\textsc{rel}-laugh-\textsc{fv}\\
\glt ‘They are laughing.’ (NF\_Elic15)
\z

The FIC is a type of cleft construction (see also \sectref{bkm:Ref491333435} on cleft constructions): the infinitive functions as a clefted element, and the inflected verb as (the beginning of) a relative clause. Example (\ref{bkm:Ref445908644}) presents the analysis of a FIC as a cleft construction.

\ea
\label{bkm:Ref445908644}
kùyèndà ndíyêndà\\
\glll ∅-ku-end-a    ndí̲-é̲nd-a\\
{[clefted element]}  {[relative clause]}\\
\textsc{cop}-\textsc{np}\textsubscript{15}-walk-\textsc{fv}  \textsc{sm}\textsubscript{1SG}.\textsc{rel}-walk-\textsc{fv}\\
\glt ‘I am walking.’ (ZF\_Elic14)
\z

In a cleft construction, the clefted element is marked by a copula. Although the copulative prefix is zero with nouns of class 15 (such as the infinitive), its presence can still be detected. In Namibian Fwe the copulative prefix on class 15 nouns can be realized as \textit{nku-}, and this form can also be seen with the infinitive used in the FIC, as in (\ref{bkm:Ref492317772}).

\ea
\label{bkm:Ref492317772}
nkùhóꜝm’ áꜝhómà\\
\gll N-ku-hóm-a  á̲-ho\textsubscript{H}m-á̲\\
\textsc{cop}-\textsc{np}\textsubscript{15}-lie-\textsc{fv}  \textsc{sm}\textsubscript{1}.\textsc{rel}-lie-\textsc{fv}\\
\glt ‘He’s lying.’ (NF\_Elic15)
\z

The class 15 copula also has a definite form \textit{kó-}, which can also be used on the infinitive in the FIC, as in (\ref{bkm:Ref492317885}).

\ea
\label{bkm:Ref492317885}
kókùmànà ndíꜝmánà\\
\gll kó-ku-man-a    ndí̲-man-á̲\\
\textsc{cop}.\textsc{def}\textsubscript{15}-\textsc{inf}-finish-\textsc{fv}  \textsc{sm}\textsubscript{1SG}.\textsc{rel}-finish-\textsc{fv}\\
\glt ‘I’ve just finished.’ (ZF\_Elic14)
\z

Furthermore, the copula can never be preceded by a vocalic augment. In infinitives, the prefix \textit{ku-} can optionally be preceded by an augment \textit{o-}, as in (\ref{bkm:Ref467591299}), but in the FIC, the augment \textit{o-} is not allowed, as shown in (\ref{bkm:Ref99095885}--\ref{bkm:Ref72309754}).

\ea
\label{bkm:Ref467591299}
ndìpàtéhìtè (ò)kùnywá ètìyì\\
\gll ndi-paté̲h-ite      (o-)ku-nyw-á  e-∅-tiyi\\
\textsc{sm}\textsubscript{1SG}-be\_busy-\textsc{stat}  (\textsc{aug}-)\textsc{inf}-drink  \textsc{aug}-\textsc{np}\textsubscript{9}-tea\\
\glt ‘I’m busy drinking tea.’
\z

\ea
\label{bkm:Ref99095885}
kùnywá ꜝndínywà\\
\gll N-ku-nyú-a    ndí̲-nyw-á̲\\
\textsc{cop}-\textsc{inf}-drink-\textsc{fv}  \textsc{sm}\textsubscript{1SG}.\textsc{rel}-drink-\textsc{fv}\\
\glt ‘I am drinking.’
\z

\ea
\label{bkm:Ref72309754}
*òkùnywá ꜝndínywà (ZF\_Elic14)
\z

The inflected verb of a FIC has a relative clause tone pattern. For most TAM constructions, the relative clause verb form is distinguished from its main clause counterpart by the addition of a high tone on the subject marker (melodic tone 2), as is the case for the present construction (see \sectref{bkm:Ref491095705} on relative clauses). The relative clause form of the present construction is given in (\ref{bkm:Ref71281593}), and (\ref{bkm:Ref71281611}) shows that this same form is used in the FIC.

\ea
\label{bkm:Ref71281593}
màyìrà ndíꜝhíbà\\
\gll ma-ira    ndí̲-hib-á̲\\
\textsc{np}\textsubscript{6}-sorghum  \textsc{sm}\textsubscript{1SG}.\textsc{rel}-steal-\textsc{fv}\\
\glt ‘the sorghum that I steal’
\z

\ea
\label{bkm:Ref71281611}
kùhíbà ndíꜝhíbà\\
\gll N-ku-híb-a    ndí̲-hib-á̲\\
\textsc{cop}-\textsc{inf}-steal-\textsc{fv}  \textsc{sm}\textsubscript{1SG}.\textsc{rel}-steal-\textsc{fv}\\
\glt ‘I am stealing.’ (NF\_Elic15)
\z

The word order used with the FIC is also typical of relative clauses. In a canonical main clause, subjects tend to precede the verb, and objects and locatives tend to follow the verb (see also \sectref{bkm:Ref492317959} on word order). With a FIC, however, subjects, objects, and locatives all follow the verb, as in (\ref{bkm:Ref98513028}--\ref{bkm:Ref98513032}).

\ea
\label{bkm:Ref98513028}
Verb - Object\\
kùhòndà ndíꜝhóndà bùhòbè\\
\gll ku-hond-a  ndí̲-hó̲nd-a    bu-hobe\\
\textsc{inf}-cook-\textsc{fv}  \textsc{sm}\textsubscript{1SG}.\textsc{rel}-cook-\textsc{fv}  \textsc{np}\textsubscript{14}-porridge\\
\glt ‘I am cooking porridge.’ (ZF\_Elic14)
\z

\ea
Verb - Locative\\
kùyèndà ndíꜝyéndà mùmùtêmwà\\
\gll ku-end-a  ndí̲-é̲nd-a    mu-mu-témwa\\
\textsc{inf}-walk-\textsc{fv}  \textsc{sm}\textsubscript{1SG}.\textsc{rel}-walk-\textsc{fv}  \textsc{np}\textsubscript{18}-\textsc{np}\textsubscript{3}-forest\\
\glt ‘I am walking through the forest.’ (ZF\_Elic13)
\z

\ea
\label{bkm:Ref98513032}
Verb - Subject\\
kùshóká ꜝshókò mvûrà\\
\gll ku-shók-a  á̲-shó̲k-a    o-∅-mvúra\\
\textsc{inf}-fall-\textsc{fv}  \textsc{sm}\textsubscript{1}.\textsc{rel}-fall-\textsc{fv}  \textsc{aug}-\textsc{np}\textsubscript{1a}-rain\\
\glt ‘It is raining.’ (ZF\_Elic13)
\z

Even when used with a FIC, a subject may be placed before the verb, as in (\ref{bkm:Ref442188158}). In that case, however, it precedes both the infinitive and inflected verb; subjects (or any other constituents) never occur between the infinitive and the inflected verb. This is consistent with the structure of relative clauses, where no constituent is allowed between the antecedent and the relative clause verb. The movement of the subject constituent to the beginning of the clause is the result of left dislocation, a frequently used change in word order that functions to mark the left-dislocated constituent as a topic (see \sectref{bkm:Ref403656711} on left dislocation).

\ea
\label{bkm:Ref442188158}
zywìn ómùntù kùkúrá ꜝkúrà\\
\gll zwiná    o-mu-ntu    ku-kúr-a    á̲-ku\textsubscript{H}r-á̲\\
\textsc{dem}.\textsc{iv}\textsubscript{1}  \textsc{aug}-\textsc{np}\textsubscript{1}-person  \textsc{inf}-sweep-\textsc{fv}  \textsc{sm}\textsubscript{1}.\textsc{rel}-sweep-\textsc{fv}\\
\glt ‘That person is sweeping.’ (ZF\_Elic13)
\z

Only the progressive auxiliary \textit{kwesi} can be used between the infinitive and inflected verb, as in (\ref{bkm:Ref99095992}). The high tone on the subject marker of \textit{túkwèsì} shows that in this case, it is the auxiliary verb that functions as the relative clause verb in the cleft construction.

\ea
\label{bkm:Ref99095992}
kùnèngà túkwèsì tùnêngà\\
\gll ku-neng-a    tú̲-kwesi    tu-né̲ng-a\\
\textsc{inf}-dance-\textsc{fv}  \textsc{sm}\textsubscript{1PL}.\textsc{rel}-\textsc{prog}  \textsc{sm}\textsubscript{1PL}-dance-\textsc{fv}\\
\glt ‘We are dancing.’ (ZF\_Elic14)
\z

A final argument that shows that the FIC can be analyzed as a cleft construction is that it cannot be combined with another cleft: (\ref{bkm:Ref469495204}) shows the clefting of the infinitive verb, and (\ref{bkm:Ref72310342}) the clefting of a locative adjunct, but as shown by the ungrammaticality of (\ref{bkm:Ref72310343}), clefting both constituents is not possible.

\ea
kùkízìkìtè ndíkìzíkîtè\\
\gll ku-kí-zik-ite      ndí̲-ki\textsubscript{H}-zik-í̲te\\
\textsc{inf}-\textsc{refl}-hide-\textsc{stat}  \textsc{sm}\textsubscript{1SG}.\textsc{rel}-\textsc{refl}-hide-\textsc{stat}\\
\glt ‘I am hidden.’
\z

\ea
\label{bkm:Ref72310342}
mùmùtémwà ndíkìzìkîte\\
\gll N-mu-mu-témwa    ndí̲-ki\textsubscript{H}-zik-í̲te\\
\textsc{cop}-\textsc{np}\textsubscript{18}-\textsc{np}\textsubscript{3}-forest  \textsc{sm}\textsubscript{1SG\-\-}.\textsc{rel}-\textsc{refl}-hide-\textsc{stat}\\
\glt ‘It’s in the forest that I’m hidden.’
\z

\ea
\label{bkm:Ref72310343}
*mùmùtémwà kùkízìkìtè ndíkìzìkîtè (ZF\_Elic13\label{bkm:Ref469495204}
\z

The analysis of the FIC as a cleft also explains its focus function, as clefts are the most common focus structure used in Fwe. The progressive-marking use of the FIC is likely to have developed out of its focus-marking use, as also argued for Kikongo (De \citealt{KindEtAl2015}). The focus use of the FIC is discussed in \sectref{bkm:Ref491333435} on cleft constructions.

The FIC can be used to express progressive aspect, although the duration of the event referred to by the FIC can vary considerably. In (\ref{bkm:Ref467593745}) and (\ref{bkm:Ref467593746}), the FIC describes a progressive action that takes up most of the day. The FIC in (\ref{bkm:Ref467594302}) describes an event that takes place over several months, and the FIC in (\ref{bkm:Ref467594303}) describes an event that takes place over several years. This use of the FIC contrasts with the use of the progressive \textit{kwesi}, which typically describes events with a relatively short duration.

\ea
\label{bkm:Ref467593745}
zyônà kùsébèzà kàndìsèbèzâ\\
\gll zyóna    ku-sébez-a  ka-ndi-sebez-á̲\\
yesterday  \textsc{inf}-work-\textsc{fv}  \textsc{pst}.\textsc{ipfv}-\textsc{sm}\textsubscript{1SG}-work-\textsc{fv}\\
\glt ‘Yesterday, I was working.’
\z

\ea
\label{bkm:Ref467593746}
kùkékèrà kàndíkèkérá shùnù\\
\gll ku-kéker-a    ka-ndí̲-ke\textsubscript{H}ker-á̲    shunu\\
\textsc{inf}-plough-\textsc{fv}  \textsc{pst}.\textsc{ipfv}-\textsc{sm}\textsubscript{1SG}-plough-\textsc{fv}  today\\
\glt ‘I was ploughing today.’
\z

\ea
\label{bkm:Ref467594302}
kùpòtà ákàpòtà bàkwâkwè mwànàmìbìà\\
\gll ku-pot-a  á̲-ka-pot-a    ba-kwákwe  mwa-namibia\\
\textsc{inf}-visit-\textsc{fv}  \textsc{sm}\textsubscript{1}-\textsc{dist}-visit-\textsc{fv}  \textsc{np}\textsubscript{2}-relative  \textsc{np}\textsubscript{18}-Namibia\\
\glt ‘She’s visiting her relatives in Namibia.’ (ZF\_Elic14)
\z

\ea
\label{bkm:Ref467594303}
òzyú mwâncè kùkúrà áꜝkúrà\\
\gll o-zyú  mu-ánce  ku-kú̲r-a  á̲-ku\textsubscript{H}r-á̲\\
\textsc{dem}.\textsc{i}\textsubscript{1}  \textsc{np}\textsubscript{1}-child  \textsc{inf}-grow-\textsc{fv}  \textsc{sm}\textsubscript{1}-grow-\textsc{fv}\\
\glt ‘The child is growing.’ (ZF\_Elic13)
\z

The FIC can even be used when the speaker is not certain, or does not assert strongly, that the event is actually ongoing. In (\ref{bkm:Ref432672376}), the FIC is used to describe people who are away for months at a time doing construction work in Angola. Here, the speaker does not assert that the people described are actually doing work at the time, yet he still uses the FIC.

\ea
\label{bkm:Ref432672376}
\label{bkm:Ref506483368}
àbàntù kùbèrèkà bákàbèrèkà mwààngòrà\\
\gll a-ba-ntu    ku-berek-a  bá̲-ka-berek-á̲    mwa-angora\\
\textsc{aug}-\textsc{np}\-\textsubscript{2}-person  \textsc{inf}-work-\textsc{fv}  \textsc{sm}\textsubscript{2}.\textsc{rel}-\textsc{dist}-work-\textsc{fv}  \textsc{np}\textsubscript{18}-Angola\\
\glt ‘The people are working in Angola.’ (ZF\_Elic14)
\z

The FIC may combine with the progressive auxiliary \textit{kwesi} to expresses both progressive aspect and verb focus. This is illustrated in (\ref{bkm:Ref416184516}), which is uttered to alert a passer-by to the fact that the container she is carrying on her head is leaking. The event is presented as progressive through use of the auxiliary \textit{kwesi}, and the focus on the verb is expressed with the fronted infinitive construction.

\ea
\label{bkm:Ref416184516}
\label{bkm:Ref506483372}
ècìpùpé ꜝcákò kùzywìzyà cíkwèsì cìzywîzyà\\
\gll e-ci-pupé    cí-akó    ku-zywizy-a  cí̲-kwesi    ci-zywí̲z-a\\
\textsc{aug}-\textsc{np}\textsubscript{7}-container  \textsc{pp}\textsubscript{7}-\textsc{poss}\textsubscript{2SG}  \textsc{inf}-leak-\textsc{fv}  \textsc{sm}\textsubscript{7}.\textsc{rel}-\textsc{prog}  \textsc{sm}\textsubscript{7}-leak-\textsc{fv}\\
\glt ‘Your container is leaking!’ (ZF\_Elic14)
\z

The FIC can combine with different TAM constructions, such as the present in (\ref{bkm:Ref506483368}--\ref{bkm:Ref506483372}) above. When used to mark progressive aspect, the FIC may only combine with imperfective constructions, such as the remote past imperfective in (\ref{bkm:Ref440616266}) or the near past imperfective in (\ref{bkm:Ref506305666}). When used to express verb focus, the FIC may also combine with perfective past constructions, such as the near past perfective in (\ref{bkm:Ref506300540}).

\ea
\label{bkm:Ref440616266}
zywìn’ ómùntù kùnywá kànywâ\\
\gll zywiná  o-mu-ntu    ku-nyú-a    ka-a-nyu-á̲\\
\textsc{dem}.\textsc{iv}\textsubscript{1}  \textsc{aug}-\textsc{np}\textsubscript{1}-person  \textsc{inf}-drink-\textsc{fv}    \textsc{pst}.\textsc{ipfv}-\textsc{sm}\textsubscript{1}-drink-\textsc{fv}\\
\glt ‘That person has been drinking.’ (ZF\_Elic14)
\z

\ea
\label{bkm:Ref506305666}
kùshèkà ndákùshèkà\\
\gll ku-shek-a    ndí̲-aku-shek-a\\
\textsc{inf}-laugh-\textsc{fv}  \textsc{sm}\textsubscript{1}.\textsc{rel}-\textsc{npst}.\textsc{ipfv}-laugh-\textsc{fv}\\
\glt ‘I was laughing.’ (NF\_Elic15)
\z

\ea
\label{bkm:Ref506300540}
kùshúmà nàmùshûmì kònó kànâfwì\\
\gll ku-shúm-a  na-mu-shúm-i      konó  ka-ná̲-fw-i  \\
\textsc{inf}-bite-\textsc{fv}  \textsc{sm}\textsubscript{1}.\textsc{pst}-\textsc{om}\textsubscript{1}-bite-\textsc{npst}.\textsc{pfv}  but  \textsc{neg}-\textsc{sm}\textsubscript{1}.\textsc{pst}-die-\textsc{npst}.\textsc{pfv}\\
\glt ‘He bit him, but he didn’t die.’ (NF\_Elic17)
\z

The FIC cannot be used with future constructions, as these only occur in main clauses (see \sectref{bkm:Ref463007186}). Instead, to express a progressive action the FIC combines with a verb in the subjunctive mood, as in (\ref{bkm:Ref99096078}) (see also \sectref{bkm:Ref492309168} on the subjunctive). This is one of the default strategies for expressing future temporal reference in subordinate clauses.

\ea
\label{bkm:Ref99096078}
shûnù àbáncè kùzànà bázânè\\
\gll shúnu  a-ba-ánce    ku-zan-a  bá̲-zá̲n-e\\
today  \textsc{aug}-\textsc{np}\textsubscript{2}-child  \textsc{inf}-play-\textsc{fv}  \textsc{sm}\textsubscript{2}.\textsc{rel}-play-\textsc{pfv}.\textsc{sbjv}\\
\glt ‘Today the children will be playing.’ (ZF\_Elic14)
\z

The infinitive verb does not retain all the inflectional and derivational affixes of the inflected verb. Suffixes occur on both the inflected verb and the infinitive: this is the case for derivational suffixes, such as the pluractional suffix \textit{-a} and the transitive separative suffix \textit{-ur} in (\ref{bkm:Ref415747170}), or the causative suffix \textit{-is} in (\ref{bkm:Ref445907210}), as well inflectional suffixes, such as the aspectual suffix -\textit{ite} in (\ref{bkm:Ref445907286}).

\ea
\label{bkm:Ref415747170}
kùàmbàùrà túàmbàúrà kwàmànà nòmfûmù\\
\gll ku-amb-a-ur-a    tú̲-amb-a-ur-á̲ kwamana  no=∅-mfúmu \\
\textsc{inf}-talk-\textsc{pl}1-\textsc{sep}.\textsc{tr}-\textsc{fv}  \textsc{sm}\textsubscript{1PL}.\textsc{rel}-talk-\textsc{pl}1-\textsc{sep}.\textsc{tr}-\textsc{fv}
about    \textsc{com}=\textsc{np}\textsubscript{1a}-chief\\
\glt ‘We are talking about the chief.’ (ZF\_Elic13)
\z

\ea
\label{bkm:Ref445907210}
kùrísꜝá  rìsó mùcècè\\
\gll ku-rí-is-a    á̲-ri\textsubscript{H}-is-á̲      o-mu-cece\\
\textsc{inf}-eat-\textsc{caus}-\textsc{fv}  \textsc{sm}\textsubscript{1}.\textsc{rel}-eat-\textsc{caus}-\textsc{fv}  \textsc{aug}-\textsc{np}\textsubscript{1}-child\\
\glt ‘She is feeding the child.’ (ZF\_Elic14)
\z

\ea
\label{bkm:Ref445907286}
kùzíkìtè ndìkìzíkîtè\\
\gll ku-zík-ite    ndi-ki\textsubscript{H}-zik-í̲te\\
\textsc{inf}-hide-\textsc{stat}  \textsc{sm}\textsubscript{1SG}.\textsc{rel}-\textsc{refl}-hide-\textsc{stat}\\
\glt ‘I am hiding.’ (ZF\_Elic13)
\z

Prefixes of the inflected verb are never copied onto the infinitive verb. This is the case for the object marker in (\ref{bkm:Ref445907388}); the reflexive prefix in (\ref{bkm:Ref452976657}); the persistive prefix in (\ref{bkm:Ref452976658}), and the distal in (\ref{bkm:Ref452976659}).

\ea
\label{bkm:Ref445907388}
kùtwírà ndímùtwîrà\\
\gll ku-tw-ír-a      ndí̲-mu-tw-í̲r-a\\
\textsc{inf}-pound-\textsc{appl}-\textsc{fv}  \textsc{sm}\textsubscript{1SG}.\textsc{rel}-\textsc{om}\textsubscript{1}-pound-\textsc{appl}-\textsc{fv}\\
\glt ‘I am pounding for someone.’ (ZF\_Elic14)
\z

\ea
\label{bkm:Ref452976657}
kùzíkìtè ndìkìzìkîtè\\
\gll ku-zík-ite    ndi-ki\textsubscript{H}-zik-í̲te \\
\textsc{inf}-hide-\textsc{stat}  \textsc{sm}\textsubscript{1SG}.\textsc{rel}-\textsc{refl}-hide-\textsc{stat}\\
\glt ‘I am hiding.’ (ZF\_Elic13)
\z

\ea
\label{bkm:Ref452976658}
éntì kùhórà íshìhórà\\
\gll e-N-tí  ku-hór-a  í̲-shi\textsubscript{H}-ho\textsubscript{H}r-á̲\\
\textsc{aug}-\textsc{np}\textsubscript{9}-tea  \textsc{inf}-cool-\textsc{fv}  \textsc{sm}\textsubscript{9}.\textsc{rel}-\textsc{per}-cool-\textsc{fv}\\
\glt ‘The tea is still cooling down.’ (ZF\_Elic14)
\z

\ea
\label{bkm:Ref452976659}
kùsèbèzà kàndíkàsèbèzâ\\
\gll ku-sebez-a  ka-ndí̲-ka-sebez-á̲\\
\textsc{inf}-work-\textsc{fv}  \textsc{pst}.\textsc{ipfv}-\textsc{sm}\textsubscript{1SG}-\textsc{dist}-work-\textsc{fv}\\
\glt ‘I worked there.’ (ZF\_Elic13)
\z
\section{Habitual}
\hypertarget{Toc75352693}{}
Habitual is a subtype of imperfective aspect (see, for instance, {\citet[25]{Comrie1976}}). Habitual expresses a repeated event that is considered characteristic of the subject (\citealt{BertinettoLenci2012}). Fwe expresses the habitual with the suffix -\textit{ang} or the prefix \textit{náku}-, which may be combined on the same verb. The following two sections describe the form and function of both habitual markers.

\subsection{Habitual 1}
\label{bkm:Ref451268511}\hypertarget{Toc75352694}{}\label{bkm:Ref70945764}
The habitual suffix \textit{-ang} follows the verb base, and precedes the final vowel suffix, as in (\ref{bkm:Ref99096161}).

\ea
\label{bkm:Ref99096161}
\glll ndìshámbângà\\
ndi-shamb-á̲ng-a\\
\textsc{sm}\textsubscript{1SG}-swim-\textsc{hab}-\textsc{fv}\\
\glt ‘I swim.’ (NF\_Elic15)
\z

The suffix \textit{-ang} is underlyingly toneless, and surfaces as low-toned unless a melodic high tone is assigned or the syllable is affected by H retraction or spread. The suffix formally resembles a derivational suffix (see Chapter \ref{bkm:Ref98758856}), most of which also have a VC shape, follow the verb root and lack underlying tone. The habitual suffix \textit{-ang}, however, is inflectional rather than derivational, and as such, derivational suffixes stand closer to the verb root than the habitual suffix. This order is shown with the passive in (\ref{bkm:Ref490749889}), and the applicative in (\ref{bkm:Ref452393178}).

\ea
\label{bkm:Ref490749889}
ècí cìntù kàcìrìwângà\\
\gll e-cí    ci-ntu    ka-ci-ri\textsubscript{H}-iw-á̲ng-a\\
\textsc{aug}-\textsc{dem}.\textsc{i}\textsubscript{7}  \textsc{np}\textsubscript{7}-thing  \textsc{neg}-\textsc{sm}\textsubscript{7}-eat-\textsc{pass}-\textsc{hab}-\textsc{fv}\\
\glt ‘This thing, it is not eaten.’ (NF\_Elic17)
\z

\ea
\label{bkm:Ref452393178}
tùkìŋòrèrâːngà àmàŋórò\\
\gll tu-ki\textsubscript{H}-ŋo\textsubscript{H}r-er-á̲ng-a    a-ma-ŋoró\\
\textsc{sm}\textsubscript{1PL}-\textsc{refl}-write-\textsc{appl}-\textsc{hab}-\textsc{fv}  \textsc{aug}-\textsc{np}\textsubscript{6}-letter\\
\glt ‘We write each other letters.’ (ZF\_Elic13)
\z

The habitual suffix -\textit{ang} is common in Bantu, reconstructed as *ag or *ang \citep{Meeussen1967}, and its cognates are often used with a habitual meaning \citep[98]{Nurse2008}. The habitual \textit{-ang} in Fwe describes a recurrent event that is considered a characteristic of the situation or its participiants, as in (\ref{bkm:Ref497996631}), where the habitual \textit{-ang} indicates that making the speaker sleepy is a typical property of this medicine.

\ea
\label{bkm:Ref497996631}
òwú mùshámù ùnákùndìsùkùrìsàngà\\
\gll o-ú    mu-shámu    u-náku-ndi-sukur-is-ang-a\\
\textsc{aug}-\textsc{dem}.\textsc{i}\textsubscript{3}  \textsc{np}\textsubscript{3}-medicine  \textsc{sm}\textsubscript{3}-\textsc{hab}-\textsc{om}\textsubscript{1SG}-become\_dozy-\textsc{caus}-\textsc{hab}-\textsc{fv}\\
\glt ‘This medicine makes me sleepy.’ (NF\_Elic17)
\z

The habitual suffix \textit{-ang} is used to describe an event that is repeated, for instance, every day, as in (\ref{bkm:Ref99096230}), or every morning, as in (\ref{bkm:Ref99096243}).

\ea
\label{bkm:Ref99096230}
èzyúbà nèzyûbà káyàngà kúrùwà\\
\gll e-∅-zyúba    ne=∅-zyúba    ka-á̲-i-ang-a      kú-ru-wa\\
\textsc{aug}-\textsc{np}\textsubscript{5}-day  \textsc{com}=\textsc{aug}-\textsc{np}\textsubscript{5}-day  \textsc{pst}.\textsc{ipfv}-\textsc{sm}\textsubscript{1}-go-\textsc{hab}-\textsc{fv}  \textsc{np}\textsubscript{17}-\textsc{np}\textsubscript{11}-field\\
\glt ‘Every day, she went to the field.’ (NF\_Narr15)
\z

\ea
\label{bkm:Ref99096243}
mùzyûbà màsíkùsîkù ndìnywângà màsàmbà\\
\gll mu-∅-zyúba  ma-síkusíku    ndi-nyw-á̲ng-a    ma-samba\\
\textsc{np}\textsubscript{18}-\textsc{np}\textsubscript{5}-day  \textsc{np}\textsubscript{6}-morning    \textsc{sm}\textsubscript{1SG}-drink-\textsc{hab}-\textsc{fv}  \textsc{np}\textsubscript{6}-tea\\
\glt ‘Every morning I drink tea.’ (ZF\_Elic14)
\z

In present habituals, at least some of the intervals that make up a habitual event are situated before the utterance time. In (\ref{bkm:Ref467749716}), the use of the habitual suffix \textit{-ang} indicates that a number of the occasions of waking up at six are in the past, and that some are planned for the future as well.

\ea
\label{bkm:Ref467749716}
kásìkìsì ndíbùːkângà\\
\gll ∅-ká-sikisi    ndí̲-buː\textsubscript{H}k-á̲ng-a\\
\textsc{cop}-\textsc{adv}-six    \textsc{sm}\textsubscript{1SG}.\textsc{rel}-wake-\textsc{hab}-\textsc{fv}\\
\glt ‘It’s at six that I normally wake up.’ (ZF\_Elic14)
\z

The habitual suffix \textit{-ang} may also have a gnomic meaning, as in (\ref{bkm:Ref494980694}), where it describes the general behavior of all dogs, and in (\ref{bkm:Ref494980695}), where it describes the general characteristics of old people’s hair.

\ea
\label{bkm:Ref494980694}
àbámbwà bàbbózângà\\
\gll a-ba-mbwá    ba-bbo\textsubscript{H}z-á̲ng-a\\
\textsc{aug}-\textsc{np}\textsubscript{2}-dog  \textsc{sm}\textsubscript{2}-bark-\textsc{hab}-\textsc{fv}\\
\glt ‘Dogs bark.’ (ZF\_Elic13)
\z

\ea
\label{bkm:Ref494980695}
ènshúkí ꜝzábànkàrâmbà zìtùbângà\\
\gll e-N-shukí    zi-á=ba-nkarámba    zi-tub-á̲ng-a\\
\textsc{aug}-\textsc{np}\textsubscript{10}-hair  \textsc{pp}\textsubscript{10}-\textsc{con}=\textsc{np}\textsubscript{2}-old\_person  \textsc{sm}\textsubscript{2}-be\_white-\textsc{hab}-\textsc{fv}\\
\glt ‘Old people’s hair is white.’ (NF\_Elic17)
\z

Habitual \textit{-ang} can combine with the imperfective past, as habitual is a subtype of imperfective aspect. As discussed in \sectref{bkm:Ref494480139}, this is only possible for the remote past imperfective, not the near past imperfective. When used with the remote past imperfective, the habitual indicates that all repetitions of the action take place in the past; the action habitually took place, but no longer holds in the present, as in (\ref{bkm:Ref99097044}).

\ea
\label{bkm:Ref99097044}
\glll kàndítòmbwèrângà\\
ka-ndí̲-tombwer-á̲ng-a\\
\textsc{pst}.\textsc{ipfv}-\textsc{sm}\textsubscript{1SG}-weed-\textsc{hab}-\textsc{fv}\\
\glt ‘I used to weed (but not anymore).’ (NF\_Elic15)
\z

In Zambian Fwe, the habitual suffix \textit{-ang} may be used with a subjunctive, as in (\ref{bkm:Ref494480249}), or a near future based on the subjunctive, as in (\ref{bkm:Ref494480250}).

\ea
\label{bkm:Ref494480249}
òràpèrángè múzyûbà\\
\gll o-raper-á̲ng-e    mú-∅-zyúba\\
\textsc{sm}\textsubscript{2SG}-pray-\textsc{hab}-\textsc{pfv}.\textsc{sbjv}  \textsc{np}\textsubscript{18}-\textsc{np}\textsubscript{5}-day\\
\glt ‘You should pray every day.’ (ZF\_Elic14)
\z

\ea
\label{bkm:Ref494480250}
èyìnó nsûndà mbòndíbùːkángè kàêtì\\
\gll e-inó    N-súnda  mbo-ndí̲-buː\textsubscript{H}k-á̲ng-e      ka-éti\\
\textsc{aug}-\textsc{dem}.\textsc{ii}\textsubscript{9}  \textsc{np}\textsubscript{9}-week  \textsc{near}.\textsc{fut}-\textsc{sm}\textsubscript{1SG}-wake-\textsc{hab}-\textsc{pfv}.\textsc{sbjv}  \textsc{adv}-eight\\
\glt ‘This week, I will wake up at eight.’
\z

In Namibian Fwe, the habitual suffix \textit{-ang} can only co-occur with the imperfective subjunctive, as in (\ref{bkm:Ref494482340}), and the near future based on the imperfective subjunctive, as in (\ref{bkm:Ref494482341}). The imperfective subjunctive may also express habitual without the suffix \textit{-ang}, as in (\ref{bkm:Ref72310976}) (see also \sectref{bkm:Ref489363941} on the imperfective subjunctive).

\ea
\label{bkm:Ref494482340}
ìnú èmvîkì wákùménèkàngà éwè\\
\gll inú    e-N-víki    o-áku-mének-ang-a    éwe\\
\textsc{dem}.\textsc{ii}\textsubscript{9}  \textsc{aug}-\textsc{np}\textsubscript{9}-week  \textsc{sm}\textsubscript{2SG}-\textsc{sbjv}.\textsc{ipfv}-wake\_early-\textsc{hab}-\textsc{fv} \textsc{pers}\textsubscript{2SG}\\
\glt ‘This week, you should wake up early every day.’
\z

\ea
\label{bkm:Ref494482341}
\glll mbòndákùbèrèkàngà\\
mbo-ndi-áku-berek-ang-a\\
\textsc{near}.\textsc{fut}-\textsc{sm}\textsubscript{1SG}-\textsc{sbjv}.\textsc{ipfv}-work-\textsc{hab}-\textsc{fv}\\
\glt ‘I will work every day.’
\z

\ea
\label{bkm:Ref72310976}
\glll mbòndákùbèrèkà\\
mbo-ndi-áku-berek-a\\
\textsc{near}.\textsc{fut}-\textsc{sm}\textsubscript{1SG}-\textsc{sbjv}.\textsc{ipfv}-work-\textsc{fv}\\
\glt ‘I will work every day.’ (NF\_Elic17)
\z
\subsection{Habitual 2}
\label{bkm:Ref489363965}\hypertarget{Toc75352695}{}
Another form of the habitual uses the post-initial prefix \textit{náku}-, as in (\ref{bkm:Ref99097076}). Aside from the high tone on the habitual prefix \textit{náku}-, no melodic high tones are assigned, and the underlying tones of the verb surface.

\ea
\label{bkm:Ref99097076}
bàntù bànákùrìm’ òmùndárè\\
\gll ba-ntu  ba-náku-rim-a  o-mu-ndaré\\
\textsc{np}\textsubscript{2}-person  \textsc{sm}\textsubscript{2}-\textsc{hab}-farm-\textsc{fv}  \textsc{aug}-\textsc{np}\textsubscript{3}-maize\\
\glt ‘People usually farm maize.’ (NF\_Elic15)
\z

The prefix \textit{náku-} grammaticalized from the verb \textit{iná} ‘be (at)’ and an infinitive verb, beginning with \textit{ku-}.\footnote{I am indebted to Sebastian Dom for suggesting this etymology.} The lack of melodic tone in verbs with \textit{náku-} is consistent with its origin in an infinitive, which also lacks melodic tone. \textit{náku-} changes to \textit{náka-} when combined with the distal prefix \textit{ka-}, indicating a location away from the place of speaking. This, too, is typical of the infinitive prefix \textit{ku-} (see \sectref{bkm:Ref489965878} on the distal). It is also possible, however, for the distal not to merge with the prefix \textit{náku-}, but to be added after it, as in (\ref{bkm:Ref492369367}). This is part of the grammaticalization process of this construction, and shows that it no longer functions as an infinitive.

\ea
\label{bkm:Ref492369367}
ànákàtòngàùkà {\textasciitilde} ànákùkàtòngàùkà\\
\gll a-ná(ku)-ka-tongauk-a\\
\textsc{sm}\textsubscript{1}-\textsc{hab}-\textsc{dist}-complain-\textsc{fv}\\
\glt ‘She always complains there.’ (NF\_Elic17)
\z

The habitual marked with \textit{náku-} is similar in meaning to the habitual marked with the suffix \textit{-ang} (see \sectref{bkm:Ref70945764}), both expressing an action characteristic of a certain time period. Similar to the suffix \textit{-ang}, verbs with \textit{náku-} may express an event repeated periodically, as in (\ref{bkm:Ref492369425}), or may have a gnomic use, as in (\ref{bkm:Ref492369426}).

\ea
\label{bkm:Ref492369425}
nákùríhìndàwìrà zìntù zábàntù\\
\gll náku-rí-hind-a-u-ir-a      zi-ntu    zi-á=ba-ntu\\
\textsc{sm}\textsubscript{1}.\textsc{hab}-\textsc{refl}-take-\textsc{pl}1-\textsc{sep}-\textsc{appl}-\textsc{fv}  \textsc{np}\textsubscript{8}-thing  \textsc{pp}\textsubscript{8}-\textsc{con}=\textsc{np}\textsubscript{2}-person\\
\glt ‘S/he is always taking people’s things for him/herself.’
\z

\ea
\label{bkm:Ref492369426}
\glll zìnákùtíyìzà \\
zi-náku-tíiz-a\\
\textsc{sm}\textsubscript{8}-\textsc{hab}-be\_dangerous-\textsc{fv}\\
\glt ‘They are dangerous.’ (NF\_Elic17)
\z

The prefix \textit{náku-} may co-occur on the same verb with the habitual suffix \textit{-ang}, as in (\ref{bkm:Ref99097106}--\ref{bkm:Ref99097108}).

\ea
\label{bkm:Ref99097106}
hàhéná \textbf{ndìnákùbúːkàngà} ìyé màshènè màshènè\\
\gll ha-hená    ndi-náku-búːk-ang-a iyé  N-ma-shene    N-ma-shene \\
\textsc{emph}-\textsc{dem}.\textsc{iv}\textsubscript{16}  \textsc{sm}\textsubscript{1SG}-\textsc{hab}-wake-\textsc{hab}-\textsc{fv}
that  \textsc{cop}-\textsc{np}\textsubscript{6}\--worm  \textsc{cop}-\textsc{np}\textsubscript{6}-worm\\
\glt ‘Every time I wake up and say: there are worms, there are worms.’ (NF\_Narr15)
\z

\ea
\label{bkm:Ref99097108}
\textbf{tùnákùzìbònângà} kàrì mbùryó túhâmbà kònó zìntù túbwènè zìténdéhèrè\\
\gll tu-náku-zi\textsubscript{H}-bo\textsubscript{H}n-á̲ng-a ka-ri    N-bu-ryó    tú̲-á̲mb-a konó  ∅-zi-ntu    tú̲-bwe\textsubscript{H}ne    zi-tend-é̲here\\
\textsc{sm}\textsubscript{1PL}-\textsc{hab}-\textsc{om}\textsubscript{8}-see-\textsc{hab}-\textsc{fv}
\textsc{neg}-be  \textsc{cop}-\textsc{np}\textsubscript{14}-only  \textsc{sm}\textsubscript{1PL}-speak-\textsc{fv}
but  \textsc{cop}-\textsc{np}\textsubscript{8}-thing  \textsc{sm}\textsubscript{1PL}.\textsc{rel}-see.\textsc{stat}   \textsc{sm}\textsubscript{8}-do-\textsc{neut}.\textsc{stat}\\
\glt ‘We usually see these things, we’re not just talking, they’re things that we see happening.’ (ZF\_Conv13)
\z

No difference in meaning has yet been observed between habitual \textit{náku-} and habitual \textit{-ang}, although there is a difference in distribution, namely that only \textit{-ang}, but not \textit{\-náku-} can be combined with a past tense. Historically, \textit{náku-} is clearly a newer form, as it still shows signs of recent grammaticalization.

\section{Stative}
\label{bkm:Ref431984198}\hypertarget{Toc75352696}{}
Fwe has a stative suffix which displays complex allomorphy. Its regular form is the final vowel suffix \--\textit{ite}, which displays vowel harmony with the stem of the verb: it is realized as \textit{-ete} after verb stems with a mid vowel, and as \textit{-ite} in all other cases, as in (\ref{bkm:Ref99097243}--\ref{bkm:Ref99097245}) (see also \sectref{bkm:Ref451863900} on vowel harmony).

\ea
\label{bkm:Ref99097243}
\glll ndìfúmîtè\\
ndi-fum-í̲te\\
\textsc{sm}\textsubscript{1SG}-become\_rich-\textsc{stat}\\
\glt ‘I am rich.’
\z

\ea
\glll zìbómbêtè\\
zi-bomb-é̲te\\
\textsc{sm}\textsubscript{8}-become\_wet-\textsc{stat}\\
\glt ‘They are wet.’
\z

\ea
\glll ndìkátîtè\\
ndi-kat-í̲te\\
\textsc{sm}\textsubscript{1SG}-become\_thin-\textsc{stat}\\
\glt ‘I am thin.’
\z

\ea
\glll ndìshéshêtè\\
ndi-she\textsubscript{H}sh-é̲te\\
\textsc{sm}\textsubscript{1SG}-marry-\textsc{stat}\\
\glt ‘I am married.’
\z

\ea
\label{bkm:Ref99097245}
\glll ndìtíyîtè\\
ndi-ti\textsubscript{H}-í̲te\\
\textsc{sm}\textsubscript{1SG}-fear-\textsc{stat}\\
\glt ‘I am afraid.’ (ZF\_Elic14)
\z

The stative uses melodic tone pattern 4, e.g. the deletion of underlying high tones, and melodic tone 3, which adds a high tone to the second stem syllable (see \sectref{bkm:Ref71540337}). The suffix \textit{-ite} is counted as part of the stem, so that with CVC verb roots MT 3 is assigned to the first syllable of the suffix \textit{-ite}, as in (\ref{bkm:Ref99097243}--\ref{bkm:Ref99097245}). This tone may spread to the left up until the first syllable of the verb stem, as in (\ref{bkm:Ref99097308}--\ref{bkm:Ref99097309}) (see also \sectref{bkm:Ref430865664} on optional high tone spread).\footnote{Although leftward spread is an optional process in most words (see \sectref{bkm:Ref430865664}), the high tone of the stative is virtually always subject to leftward spread. Very few examples have been found where stative verbs do not display high tone spread, though when asked, speakers concede that the pronunciation without high tone spread is allowed.}

\ea
\label{bkm:Ref99097308}
\glll cìtúrúkìtè\\
ci-tu\textsubscript{H}rú̲k-ite\\
\textsc{sm}\textsubscript{7}-burst-\textsc{stat}\\
\glt ‘It is burst.’ (ZF\_Elic14)
\z

\ea
\label{bkm:Ref99097309}
\glll ndìpátéhètè\\
ndi-paté̲h-ete\\
\textsc{sm}\textsubscript{1SG}-be\_busy-\textsc{stat}\\
\glt ‘I am busy.’ (NF\_Elic15)
\z

When the verb stem, that is the verb root together with the stative suffix, has no more than two syllables, melodic tone 3 is not assigned. This is the case with monosyllabic roots that take the regular stative suffix \textit{-ite}, but also with disyllabic roots that take an irregular stative suffix that does not add an extra syllable. For the assignment of MT 3, only the number of syllables is relevant, not the number of moras: no melodic tone is assigned to disyllabic stems with three moras, as in (\ref{bkm:Ref72311760}), or to disyllabic stems with two moras, as in (\ref{bkm:Ref72311784}), but melodic tone is assigned to trisyllabic stems with three moras, as in (\ref{bkm:Ref72311806}). This contrasts with melodic tone 1, which does take moras into account (see \sectref{bkm:Ref71539267} on melodic tone).

\ea
\label{bkm:Ref72311760}
\glll cìfwìtè\\
ci-fw\textsubscript{H}-ite\\
\textsc{sm}\textsubscript{7}-die-\textsc{stat}\\
\glt ‘It has died.’ (ZF\_Elic14)
\z

\ea
\label{bkm:Ref72311784}
\glll ndìkèrè\\
ndi-kere\\
\textsc{sm}\textsubscript{1SG}-sit.\textsc{stat}\\
\glt ‘I sit.’
\z

\ea
\label{bkm:Ref72311806}
\glll ndìtábîtè\\
ndi-tab-í̲te\\
\textsc{sm}\textsubscript{1SG}-become\_happy-\textsc{stat}\\
\glt ‘I am happy.’ (ZF\_Elic14)
\z

Aside from the regular application of vowel harmony, the segmental form of the stative suffix can vary in other, more unpredictable ways. If the last stem consonant is a continuant, imbrication may take place, causing the vowel(s) of the stative suffix to merge with the last vowel(s) of the verb stem. If the last stem consonant is a stop, spirantization may take place, changing the stop to a fricative. Spirantization is partly lexically determined, i.e. not all verb stems ending in a stop are subject to spirantization. There is also some regional and inter-speaker variation in the occurrence of these processes; irregular forms of the stative (i.e. those not using \textit{-ite}) appear to be less common in Zambian Fwe than in Namibian Fwe. Verbs with the intransitive impositive \textit{-am} use a stative suffix \textit{-i} and drop the suffix \textit{-am}. The passive suffix -(\textit{i})\textit{w} also requires a non-canonical form of the stative; when combined with a stative, it is realized as \textit{-itwe} or \nobreakdash-\textit{itwa}, that is the passive suffix merges with the stative suffix. Finally, there is a handful of lexical exceptions taking a suffix \textit{-ire/-ere} rather than \textit{-ite/-ete}. These allomorphs are summarized in \tabref{tab:9:2}.

\begin{table}
\label{bkm:Ref72312294}\caption{\label{tab:9:2}Forms of the stative suffix}

\begin{tabular}{ll}
\lsptoprule
Allomorph & Conditioning\\
\midrule
{\itshape -ite} & regular\\
{\itshape -ete} & vowel harmony: after mid vowels\\
{\itshape -i} & with intransitive impositive verbs\\
{\itshape -ire} & lexical exceptions\\
imbrication & verbs ending in a continuant\\
spirantization & lexical exceptions\\
\lspbottomrule
\end{tabular}
\end{table}

The process of imbrication is common in Bantu languages and usually affects cognates of the suffix -\textit{ide} \citep{Bastin1983}. Whether Fwe \textit{-ite} is cognate with this suffix is not clear: although there are formal similarities between Fwe \textit{-ite} and reconstructed *-ide, the regular reflex of *-ide would be \textit{-ire}, because reconstructed *d corresponds to /r/ in Fwe \citep[114-115]{Bostoen2009}. For a discussion of the historical relationship between *-ite and *-ile in Bantu Botatwe, see Crane (2012: Appendix). At least in Fwe, -\textit{\-ite} and \textit{-ire} are allomorphs of the same suffix, as will become clear in this secdtion.

Imbricated forms of the stative suffix are used with verbs where the last stem consonant is a continuant, i.e. a nasal or /r/. The vowel /i/ of the stative suffix moves before the last stem consonant and merges with the last vowel of the verb stem. The second vowel /e/ of the stative suffix is used after the last consonant of the verb stem. The last stem consonant of the verb stem is not affected by imbrication. This is illustrated in (\ref{bkm:Ref468262620}) with the verb \textit{rind-ir} ‘wait for’, where the verb stem ends in a continuant /r/, thus allowing imbrication.

\ea
\label{bkm:Ref468262620}
\ea
rind-ir ‘wait for’

\ex
\glll ndìríndîrè\\
ndi-rind-í̲r-e\\
\textsc{sm}\textsubscript{1SG}-wait-\textsc{appl}-\textsc{stat}\\
\glt ‘I am waiting.’ (NF\_Elic15)
\z\z

If the last stem vowel is /i/, imbrication of /i/ does not result in a change of the vowel, as in (\ref{bkm:Ref468262620}). If the last stem vowel is /e/ or /a/, the imbricated vowel /i/ lowers to /e/, as in (\ref{bkm:Ref468262605}--\ref{bkm:Ref506287172}).

\ea
\label{bkm:Ref468262605}
\ea
deber ‘dangle’

\ex
\glll cìdébêrè\\
ci-debé̲r-e\\
\textsc{sm}\textsubscript{7}-dangle-\textsc{stat}\\
\glt ‘It is dangling.’ (NF\_Elic15)
\z\z

\ea
\label{bkm:Ref506287172}
\ea
sumbar ‘become pregnant’

\ex
\glll àsúmbêrè\\
a-su\textsubscript{H}mbé̲r-e\\
\textsc{sm}\textsubscript{1}-become\_pregnant-\textsc{stat}\\
\glt ‘She is pregnant.’ (NF\_Elic15)
\z\z

When the last vowel of the verb stem is a back vowel, imbrication with the vowel /i/ of the stative changes the back vowel to a glide [w], as in (\ref{bkm:Ref99097454}). In the case of a mid back vowel /o/, the imbricated vowel /i/ is lowered to a mid vowel /e/, as in (\ref{bkm:Ref99097466}).

\ea
\label{bkm:Ref99097454}
\ea
zyur ‘become full’

\ex
\glll cìzywìrè\\
ci-zywir-e\\
\textsc{sm}\textsubscript{7}-become\_full-\textsc{stat}\\
\glt ‘It is full.’ (NF\_Elic15)
\z\z

\ea
\label{bkm:Ref99097466}
\ea
tontor ‘be cold’

\ex
\glll kùtòntwêrè\\
ku-to\textsubscript{H}ntwé̲r-e\\
\textsc{sm}\textsubscript{15}-be\_cold-\textsc{stat}\\
\glt ‘It is quiet.’ (NF\_Elic15)
\z\z

Imbrication of the stative suffix is most common with verb stems where the last syllable is either a productive derivational suffix, such as the applicative, or formally resembles a derivational suffix, without functioning as such. There are also a number of other verb stems that require imbrication of the stative suffix, listed in \tabref{tab:9:3}; these include mainly verbs that are more commonly used with the stative suffix than in a different construction.

\begin{table}
\label{bkm:Ref99097503}\caption{\label{tab:9:3}Imbrication}

\begin{tabular}{lll}
\lsptoprule
Verb root & English translation & Stative form\\
\midrule
\textit{bón} & ‘see’ & \textit{bwènè}\\
\textit{kar} & ‘sit down’ & \textit{kèrè}\\
\textit{ráːr} & ‘lie down; go to sleep’ & \textit{rèːrè}\\
\textit{rwar} & ‘become sick’ & \textit{rwèrè}\\
\textit{zyur} & ‘become full’ & \textit{zywìrè}\\
\lspbottomrule
\end{tabular}
\end{table}

In verb stems with the neuter suffix \textit{-ahar}, imbrication may target both the vowels of the suffix, which are raised to /e/ when combined with the stative. This double imbrication is not obligatory, however, and forms where only the last stem vowel are subject to imbrication are also allowed, as in (\ref{bkm:Ref490209104}). The verb \textit{bonahar} ‘appear’, even displays imbrication up to the first stem vowel, as in (\ref{bkm:Ref490209052}). Note that the underived verb \textit{bón} ‘see’ also has an imbricated form \textit{bwene}.\largerpage

\ea
\label{bkm:Ref490209104}
\ea
sep-ahar ‘be trustworthy’

\ex
\glll bàsépéhèrè {\textasciitilde} bàsépáhèrè\\
ba-sep-é̲her-e {\textasciitilde} ba-sep-á̲her-e\\
\textsc{sm}\textsubscript{2}-promise-\textsc{neut}-\textsc{stat}\\
\glt ‘S/he is trustworthy.’
\z\z

\ea
\label{bkm:Ref490209052}
\ea
bón-ahar ‘appear, be visible’

\ex
\glll kùbwénéhèrè\\
ku-bwe\textsubscript{H}n-é̲her-e\\
\textsc{sm}\textsubscript{15}-see-\textsc{neut}-\textsc{stat}\\
\glt ‘It is visible.’ (NF\_Elic15)
\z\z

Many verbs with an imbricated stative form also have an unimbricated stative form, as in (\ref{bkm:Ref99097564}--\ref{bkm:Ref99097566}). Both forms are used interchangeably, without a discernable change in meaning.

\ea
\label{bkm:Ref99097564}
\ea
gumb-am ‘be next to’

\ex
\glll bàrìgùmbêmè\\
ba-ri\textsubscript{H}-gumb-é̲me\\
\textsc{sm}\textsubscript{2}-\textsc{refl}-be\_next\_to-\textsc{imp}.\textsc{intr}.\textsc{stat}\\

\ex
\glll bàrìgùmbámìtè\\
ba-ri\textsubscript{H}-gumb-á̲m-ite\\
\textsc{sm}\textsubscript{2}-\textsc{refl}-be\_next\_to-\textsc{imp}.\textsc{intr}-\textsc{stat}\\
\glt ‘They are next to each other.’ (NF\_Elic15)
\z\z

\ea
\label{bkm:Ref99097566}
\ea
rwár ‘become sick’

\ex
\glll àrwèrè\\
a-rwe\textsubscript{H}re\\
\textsc{sm}\textsubscript{1}-become\_sick.\textsc{stat}\\

\ex
\glll àrwárîtè\\
a-rwa\textsubscript{H}r-í̲te\\
\textsc{sm}\textsubscript{1}-become\_sick-\textsc{stat}\\
\glt ‘S/he is sick.’ (ZF\_Elic14)
\z\z

In certain cases, the stative suffix causes spirantization; this is a formerly productive sound change in Fwe, where stops followed by a high vowel became fricatives \citep[117-118]{Bostoen2009}. Spirantization is no longer active in Fwe, but forms that were created as the result of spirantization are still seen in the stative forms of certain verbs. Spirantization is combined with imbrication, but differs from other cases of imbrication because the last vowel is /i/ rather than /e/. \tabref{tab:9:4} lists all attested verbs that have a spirantized stative form. Three of these have an alternative form without spirantization, but with the regular stative suffix \textit{-ite}. There appears to be a geographic distribution, where irregular, spirantized forms are more common in Namibian Fwe, and forms with the regular suffix and no spirantization are more common in Zambian Fwe.

\begin{table}
\label{bkm:Ref468263413}\caption{\label{tab:9:4}Stative verbs with spirantization}
\begin{tabular}{lll}
\lsptoprule
Verb root & Translation & Stative form\\
\midrule
\textit{kwát} & ‘grab, grasp’ & \textit{kwèsì} {\textasciitilde} \textit{kwátîtè}\\
\textit{pak} & ‘carry on one’s back’ & \textit{pèsì} {\textasciitilde} \textit{pákîtè}\\
\textit{vúrumat} & ‘close one’s eyes’ & \textit{vúrúmèsì}\\
\textit{zwát} & ‘get dressed’ & \textit{zwèsì} {\textasciitilde} \textit{zwátîtè}\\
\lspbottomrule
\end{tabular}
\end{table}

Spirantization is also seen in the stative form of a number verbs with the intransitive impositional suffix \textit{-am}, listed in \tabref{tab:9:5}. Verbs with this suffix drop the impositional suffix \textit{-am} and take a stative suffix \textit{-i}, which causes spirantization of the preceding consonant in some cases. This form of the stative is productively used with all intransitive impositive verbs, but spirantization only occurs in some of these verbs.

\begin{table}
\label{bkm:Ref506903429}\caption{\label{tab:9:5}Intransitive impositive verbs in the stative}
\begin{tabularx}{\textwidth}{Xll}
\lsptoprule
Verb stem & Translation & Stative form\\
\midrule
\textit{bémbàmà} & ‘stand next to’ & \textit{bémbì}\\
\textit{bòmbàmà} & ‘soak’ & \textit{bómbì}\\
\textit{cànkàmà} & ‘be put on a fire (of a pot)’ & \textit{cánsì}\\
\textit{céngèkà} & ‘be close to’ & \textit{cénzì}\\
\textit{còkàmà} & ‘spy (from a hidden position)’ & \textit{cósì}\\
\textit{gábàmà} & ‘hang (on a hook)’ & \textit{gábì}\\
\textit{gùmbàmà} & ‘be next to’ & \textit{gúmbì}\\
\textit{hángàmà} & ‘hang (intr.)’ & \textit{hánzì}\\
\textit{jánàmà} & ‘open one’s mouth wide’ & \textit{jánì}\\
\textit{kòtàmà} & ‘bend forward’ & \textit{kósì}\\
\textit{kúnàmà} & ‘be smoked (of food stuff, i.e. fish)’ & \textit{kúnì}\\
\textit{nyòngàmà} & ‘bend (intr.)’ & \textit{nyónzì}\\
\textit{ⁿǀùmpàmà} & ‘plant’ & \textit{ⁿǀúmpì}\\
\textit{shèndàmà} & ‘lean’ & \textit{shéndì}\\
\textit{súngàmà} & ‘bow the head’ & \textit{súnzì}\\
\textit{téngàmà} & ‘bend (intr)’ & \textit{ténzì}\\
\textit{tùmpwàmà} & ‘be thrown in water’ (of an inanimate object) & \textit{túmpwì}\\
\textit{zyánàmà} & ‘hang’ & \textit{zyánì}\\
\textit{zyáshàmà} & ‘open one’s mouth’ & \textit{zyáshì}\\
\textit{zyíàmà} & ‘lean’ & \textit{zyéndì}\\
\lspbottomrule
\end{tabularx}
\end{table}

These stative forms also have a different tonal realization. Regular stative verbs are realized without high tones when they have a disyllabic stem, but stative impositive verbs all take a high tone on the last stem syllable (which retracts to the penultimate syllabe in phrase-final position), as in (\ref{bkm:Ref99097698}--\ref{bkm:Ref99097699}). That these stative forms are derived from impositive verbs is clear from the fact that they retain their impositive semantics, and that most of these verb roots do not occur without the impositive suffix (see \sectref{bkm:Ref450835510}).

\ea
\label{bkm:Ref99097698}
\ea
\glll kùkúnàmà\\
ku-kún-am-a\\
\textsc{inf}-smoke-\textsc{imp}.\textsc{intr}-\textsc{fv}\\
\glt ‘to be put on a smoking shelve’

\ex
\glll zìkúnì\\
zi-ku\textsubscript{H}n-í̲\\
\textsc{sm}\textsubscript{8}-smoke-\textsc{imp}.\textsc{intr}.\textsc{stat}\\
\glt ‘They (fish) are lying on a smoking shelve.’

\ex
*kùkûnà (NF\_Elic15)
\z\z

\ea
\label{bkm:Ref99097699}
\ea
\glll kùzyánàmà\\
ku-zyán-am-a\\
\textsc{inf}-spread-\textsc{imp}.\textsc{intr}-\textsc{fv}\\
\glt ‘to be spread out to dry’

\ex
\glll zìzyánì\\
zi-zya\textsubscript{H}n-í̲\\
\textsc{sm}\textsubscript{8}-spread-\textsc{imp}.\textsc{intr}.\textsc{stat}\\
\glt ‘They (clothes) are spread out to dry.’

\ex
*kùzyânà (NF\_Elic15)
\z\z

Intransitive impositive verbs can also take a more regular form of the stative suffix, either with imbrication, resulting in a form \textit{-eme}, or with a regular stative suffix \textit{-ite} added after the impositive suffix \textit{-am}, resulting in the form \textit{-amite}. All three forms are illustrated with the impositive intransitive verb \textit{nyong-am} ‘bend’ in . All three stative forms are available for all intransitive impositive verbs. Again, regular forms with \textit{\--ite} are more common in Zambian Fwe, and irregular forms either with imbrication or with \textit{\--i} and spirantization are more common in Namibian Fwe.

\ea
\ea
\gll ci-nyónz-ì\\
\textsc{sm}\textsubscript{7\-}-bend-\textsc{imp}.\textsc{intr}.\textsc{stat}\\

\ex
\gll cì-nyóng-émè\\
\textsc{sm}\textsubscript{7}-bend-\textsc{imp}.\textsc{intr}.\textsc{stat}\\

\ex
\gll cì-nyóng-ám-ìtè\\
\textsc{sm}\textsubscript{7}-bend-\textsc{imp}.\textsc{intr}-\textsc{stat}\\

\glt ‘It is bent.’ (NF\_Elic15)\label{bkm:Ref431911100}
\z\z

Only verbs with the intransitive impositive suffix -\textit{am} take the stative suffix -\textit{i}. Verbs with the transitive impositive suffix -\textit{ik} may also be used in the stative (with the passive), in which case the regular stative suffix is used, as in (\ref{bkm:Ref99097817}).

\ea
\label{bkm:Ref99097817}
\glll zìkúníkìtwà\\
zi-kun-í̲k-itwa\\
\textsc{sm}\textsubscript{10}-smoke-\textsc{imp}.\textsc{tr}-\textsc{stat}-\textsc{pass}-\textsc{fv}\\
\glt ‘They are being smoked.’ (ie lying on the smoking shelve) (NF\_Elic15)
\z

Combined with the passive suffix -(\textit{i})\textit{w}, the stative suffix is realized as -\textit{itwe} in Zambian Fwe, as in (\ref{bkm:Ref99097872}), and -\textit{itwa} in Namibian Fwe, as in (\ref{bkm:Ref99097887}) (see also \sectref{bkm:Ref452972446} on the passive).

\ea
\label{bkm:Ref99097872}
\glll ndìshéshêtwè\\
ndi-she\textsubscript{H}sh-é̲twe\\
\textsc{sm}\textsubscript{1SG}-marry-\textsc{stat}.\textsc{pass}\\
\glt ‘I am married (said by a woman).’ (ZF\_Elic14)
\z

\ea
\label{bkm:Ref99097887}
\glll cìhàrîtwà\\
ci-ar-í̲twa\\
\textsc{sm}\textsubscript{7}-close-\textsc{stat}.\textsc{pass}\\
\glt ‘It is closed.’ (NF\_Elic15)
\z

Finally, the stative has an allomorph \textit{-ire} that is used with only four verbs, listed in \tabref{tab:9:6}.

\begin{table}
\label{bkm:Ref506287274}\caption{\label{tab:9:6}Stative verbs with -\textit{ire}}

\begin{tabular}{lll}
\lsptoprule
Verb root & Translation & Stative form\\
\midrule
\textit{shúw} & ‘hear, feel, perceive’ & \textit{shùwîrè}\\
\textit{fú} & ‘die; break’ & \textit{fwìrè} {\textasciitilde} \textit{fwìtè}\\
\textit{fwíìmp} & ‘become short’ & \textit{fwíímpèrè}\\
\textit{bbíh} & ‘become bad’ & \textit{bbíhîrè} {\textasciitilde} \textit{bbíhîtè}\\
\lspbottomrule
\end{tabular}
\end{table}
The interpretation of the stative depends on lexical aspect. With change-of-state verbs, the stative gives a present state interpretation, as in (\ref{bkm:Ref71287557}--\ref{bkm:Ref71287558}).

\ea
\label{bkm:Ref71287557}
hànshí kùbómbêtè\\
\gll ha-N-shí    ku-bomb-é̲te\\
\textsc{np}\textsubscript{16}-\textsc{np}\textsubscript{9}-ground  \textsc{sm}\textsubscript{17}-become\_wet-\textsc{stat}\\
\glt ‘The ground is wet.’ (ZF\_Elic14)
\z

\ea
\label{bkm:Ref71287558}
òpótó àzywìré bùsù\\
\gll o-∅-potó    a-zywir-é̲      bu-su\\
\textsc{aug}-\textsc{np}\textsubscript{1a}-pot  \textsc{sm}\textsubscript{1}-become\_full-\textsc{stat}  \textsc{np}\textsubscript{14}-flour\\
\glt ‘The pot is full of flour.’ (ZF\_Elic14)
\z

The experiencer verbs \textit{bón} ‘see’ and \textit{shúw} ‘hear, feel, smell’ also function as change-of-state verbs; in the present construction, they take a modal, futurate, or conditional interpretation. With the stative, they are interpreted as ongoing at the time of speaking, as in (\ref{bkm:Ref99097995}--\ref{bkm:Ref99097996}).

\ea
\label{bkm:Ref99097995}
\glll ndìbwènè\\
ndi-bwe\textsubscript{H}ne\\
\textsc{sm}\textsubscript{1SG}-see.\textsc{stat}\\
\glt ‘I see.’
\z

\ea
\label{bkm:Ref99097996}
\glll ndìshúwîrè\\
ndi-shu\textsubscript{H}-í̲re\\
\textsc{sm}\textsubscript{1SG}-hear-\textsc{stat}\\
\glt ‘I hear.’ (ZF\_Elic14)
\z

True stative verbs, which express a continuing, unbounded state, cannot be used in the stative construction, as in (\ref{bkm:Ref71287631}). A present state interpretation is achieved when a true stative verb is used in the present, as in (\ref{bkm:Ref72313519}).

\ea
\label{bkm:Ref71287631}
\glll *zìtìyìzîtè \\
zi-ti\textsubscript{H}iz-í̲te\\
\textsc{sm}\textsubscript{8}-be\_busy-\textsc{stat}\\
\glt Intended: ‘They are dangerous.’
\z

\ea
\label{bkm:Ref72313519}
\glll zìtìyìzâ\\
zi-ti\textsubscript{H}iz-á̲\\
\textsc{sm}\textsubscript{8}-be\_busy-\textsc{fv}\\
\glt ‘They are dangerous.’ (NF\_Elic15)
\z

Some verbs\footnote{More research into the lexical aspectual properties of these verbs is needed, including their interpretation in various tense/aspect construction, and which lexical verbs exhibit this behaviour. Further data collection might also reveal that the differences in interpretation of this subset of lexical verbs is not (only) due to a difference in lexical aspect but possibly (also) lexical semantics.} can be used either as change-of-state verbs or as true stative. This is the case, for instance, with the verb \textit{cen} ‘be/become clean’, which is interpreted as a present stative when used in the present tense, as in (\ref{bkm:Ref99464453}), as is typical of true stative verbs, but also as present state when used with the stative construction, as is typical of change-of-state verbs.

\ea
\label{bkm:Ref99464453}
èzí zìzwâtò zìcénà\\
\gll e-zí    zi-zwáto  zi-cen-á̲\\
\textsc{aug}-\textsc{dem}.\textsc{i}\textsubscript{8}  \textsc{np}\textsubscript{8}-cloth  \textsc{sm}\textsubscript{8}-be\_clean-\textsc{fv}\\
\glt ‘Are these clothes clean?’
\z

\ea
èzí zìzwâtò zìcénêtè\\
\gll e-zí    zi-zwáto  zi-cen-é̲te\\
\textsc{aug}-\textsc{dem}.\textsc{i}\textsubscript{8}  \textsc{np}\textsubscript{8}-cloth  \textsc{sm}\textsubscript{8}-become\_clean-\textsc{stat}\\
\glt ‘Are these clothes clean?’ (ZF\_Elic14)
\z

With verbs that are ambivalent between change-of-state and stative, the use of the stative suffix can give a different interpretation than the use of the present tense form. As discussed in \sectref{bkm:Ref72233436}, the present construction indicates that the event nucleus is situated at least partly after the utterance time; overlap with UT is possible (for certain lexical aspects), but not obligatory. The stative form, however, necessarily refers to a state that is ongoing at utterance time. These different interpretations of the present and stative are illustrated with the verb \textit{rwár} ‘be/become sick’: in the present construction in (\ref{bkm:Ref99465046}), it is interpreted as referring to a chronic illness, such as diabetes, from which a person can suffer without actually feeling ill all the time. In the stative construction in (\ref{bkm:Ref99465063}), it can only be interpreted as the speaker feeling ill right now.

\ea
\label{bkm:Ref99465046}
\glll ndìrwârà\\
ndi-rwá̲r-a\\
\textsc{sm}\textsubscript{1SG}-be\_sick-\textsc{fv}\\
\glt ‘I am sick/have an illness.’
\z

\ea
\label{bkm:Ref99465063}
\glll ndìrwárîtè\\
ndi-rwa\textsubscript{H}r-í̲te\\
\textsc{sm}\textsubscript{1SG}-be\_sick-\textsc{stat}\\
\glt ‘I am (feeling) sick.’ (NF\_Elic15)
\z

The stative construction presents an event as a currently ongoing state, and does not include reference to if (or when) the state has come about. In (\ref{bkm:Ref468282229}--\ref{bkm:Ref468282230}), the stative is used to indicate a currently ongoing state, which is not the result of an earlier change of state.

\ea
\label{bkm:Ref468282229}
èzí zìshámù zìgórêtè wâwà\\
\gll e-zi    zi-shamú  zi-gor-é̲te      wáwa\\
\textsc{aug}-\textsc{dem}.\textsc{i}\textsubscript{8}  \textsc{np}\textsubscript{8}-tree  \textsc{sm}\textsubscript{8}-become\_strong-\textsc{stat}  very\\
\glt ‘These trees are very strong.’ (ZF\_Elic14)
\z

\ea
\label{bkm:Ref468282230}
èzí zìntù zìkìkózêtè\\
\gll e-zí    zi-ntu    zi-ki\textsubscript{H}-koz-é̲te\\
\textsc{aug}-\textsc{dem}.\textsc{i}\textsubscript{8}  \textsc{sm}\textsubscript{8}-thing  \textsc{sm}\textsubscript{8}-\textsc{refl}-resemble-\textsc{stat}\\
\glt ‘These things are similar.’ (ZF\_Elic13)
\z

States that have not always held, but have come into being at some point in the past, can also be expressed with the stative, but the change in state is not part of their conceptualization. The use of the stative merely presents a state as currently ongoing, and backgrounds the earlier change of state that has given rise to it. In (\ref{bkm:Ref469498587}), a stative form is used to describe that eggs are rotten; although these eggs were once fresh, and the fact that they are now rotten is the result of a change in their state, this change is not referenced by the stative form, and only their current state is described.

\ea
\label{bkm:Ref469498587}
àá màyîː àbórêtè\\
\gll a-á    ma-yíː    a-bor-é̲te\\
\textsc{aug}-\textsc{dem}.\textsc{i}\textsubscript{6}  \textsc{np}\textsubscript{6}-egg  \textsc{sm}\textsubscript{6}-rot-\textsc{stat} \\
\glt ‘These eggs, they’re rotten.’ (NF\_Elic15)
\z

The fact that the stative focuses on a current state of affairs, and backgrounds its cause, also means that verbs in the stative cannot co-occur with an agent phrase; because the original action that led to the current state is not conceptualized, the agent that instigated this original action can also not be referenced. Without an agent, the stative can be used, as in (\ref{bkm:Ref498352844}), but the addition of an agent phrase is ungrammatical, as in (\ref{bkm:Ref72313820}). An agent phrase can only be used with a verb in the near past perfective construction, as in (\ref{bkm:Ref72313833}).

\ea
\label{bkm:Ref498352844}
cíàzò cìàrúkìtè\\
\gll cí-azo    ci-ar-ú̲k-ite\\
\textsc{np}\textsubscript{7}-door  \textsc{sm}\textsubscript{7}-close-\textsc{sep}.\textsc{intr}-\textsc{stat}\\
\glt ‘The door is open.’
\z

\ea
\label{bkm:Ref72313820}
  *cíàzò cìàrúkìtè kúꜝrúːho\\
Intended: ‘The door is opened by the wind.’
\z

\ea
\label{bkm:Ref72313833}
cíàzò cáàrùkì kúꜝrúːhò\\
\gll cí-azo    ci-á-ar-uk-i        kú-rúː-ho\\
\textsc{np}\textsubscript{7}-door  \textsc{sm}\textsubscript{7}-\textsc{pst}-close-\textsc{sep}.\textsc{intr}-\textsc{npst}.\textsc{pfv}  \textsc{np}\textsubscript{17}-\textsc{np}\textsubscript{11}-wind\\
\glt ‘The door is opened by the wind.’ (NF\_Elic15)
\z

As the stative does not refer to when or how the current state has come about, temporal adverbs may only describe the time at which the current state holds, as in (\ref{bkm:Ref490233470}), not the time of the preceding change in state, as the ungrammaticality of (\ref{bkm:Ref490233472}) shows.

\ea
\label{bkm:Ref490233470}
\glll \textstyleungrammatical{ndìrwárítè shûnù}\\
ndi-rwa\textsubscript{H}\textstyleungrammatical{r-í̲te  shúnu}\\
\textsc{sm}\textsubscript{1SG}-be\_sick-\textsc{stat}  today\\
\glt ‘I am sick today.’ (NF\_Elic17)
\z

\ea
\label{bkm:Ref490233472}
*èténdè ryómbwà wángù rìcóːkétè zyônà \\
\gll e-tènde  rí-o-∅-mbwá  u-angú  ri-coːk-é̲te    zyóna\\
\textsc{aug}-leg  \textsc{pp}\textsubscript{5}-\textsc{aug}-\textsc{np}\textsubscript{1a}-dog  \textsc{pp}\textsubscript{1}-\textsc{poss}\textsubscript{1SG}  \textsc{sm}\textsubscript{5}-break-\textsc{stat}  yesterday\\
Intended: ‘The leg of my dog broke yesterday.’ (ZF\_Elic14)
\z

The near past perfective may also give a present state reading with change-of-state verbs (see \sectref{bkm:Ref488767483}), but conceptualizes both the preceding change of state situated in the near past, and the resultant state which holds in the present. This difference is illustrated with the verb \textit{nyongam} ‘bend (intr.), become bent’: in the near past perfective construction in (\ref{bkm:Ref468284605}), it expresses something that has become bent recently, and both the earlier bending and the current bent state are referenced, whereas in the stative construction in (\ref{bkm:Ref72313899}), it expresses something that is currently bent, without implying anything about if or how this has come about.

\ea
\label{bkm:Ref468284605}
\glll cànyóngâmì\\
ci-a-nyong-á̲m-i\\
\textsc{sm}\textsubscript{7}-\textsc{pst}-bend-\textsc{imp}.\textsc{intr}-\textsc{npst}.\textsc{pfv}\\
\glt ‘It is bent (has become bent).’
\z

\ea
\label{bkm:Ref72313899}
\glll cìnyòngámìtè\\
ci-nyong-á̲m-ite\\
\textsc{sm}\textsubscript{7}-bend-\textsc{imp}.\textsc{intr}-\textsc{stat}\\
\glt ‘It is bent.’ (NF\_Elic15)
\z

The focus of the stative on the current state and the backgrounding of the previous change of state has a number of effects. For one, it is related to evidentiality (see also \citealt{Crane2012}): the backgrounding of the previous change of state can be used to indicate that the speaker is unaware of when or how the change of state took place. The contetxt for (\ref{bkm:Ref71288039}) is that the speaker has found a dog lying on the road while traveling. He checks up on the dog and concludes that it is dead. As the speaker has no knowledge of when or how the dog died, he uses the stative rather than the near past perfective.

\ea
\label{bkm:Ref71288039}
òzyû mbwà àfwìtè\\
\gll o-zyú    o-∅-mbwá    a-fw\textsubscript{H}-ite\\
\textsc{aug}-\textsc{dem}.\textsc{i}\textsubscript{1}  \textsc{aug}-\textsc{np}\textsubscript{1a}-dog  \textsc{sm}\textsubscript{1}-die-\textsc{stat}\\
\glt ‘This dog is dead.’ (ZF\_Elic14)
\z

For the sake of comparison, (\ref{bkm:Ref469498825}) gives an example of the same verb in the near past perfective. In this context, the speaker himself has just killed the snake: because the speaker was involved in the killing of the snake, which resulted in its current state of being dead, he uses the recent past, rather than the stative.

\ea
\label{bkm:Ref469498825}
èzyôkà rìnáfwì\\
\gll e-∅-zyóka    ri-na-fw-í̲\\
\textsc{aug}-\textsc{np}\textsubscript{5}-snake  \textsc{sm}\textsubscript{5}-\textsc{pst}-die-\textsc{npst}.\textsc{pfv}\\
\glt ‘The snake is dead.’ (ZF\_Elic14)
\z

Another example of the evidential use of \textit{-ite} is given in (\ref{bkm:Ref431916453}). The context for this utterance is seeing a person staggering and talking incoherently, upon which the speaker concludes that he is drunk. The speaker is not aware of the previous actions that have led to the current state, but only bases his statement on the current state of the person he describes.

\ea
\label{bkm:Ref431916453}
\glll ànywìtè\\
a-nyw\textsubscript{H}-ite\\
\textsc{sm}\textsubscript{1}-drink-\textsc{stat}\\
\glt ‘S/he is drunk.’ (NF\_Elic15)
\z

The focus of the stative on the current state of affairs, rather than the previous actions that have caused it, also relates to information structure. In the context of (\ref{bkm:Ref431912048}) the speaker has two buckets of clothes; one with dry clothes, and one with wet clothes. The contrastive focus stresses the difference between the current states of the two sets of clothes, not when or how this state occurred. To express the irrelevance of the change in state, and the focus on the current state, the stative is used.

\ea
\label{bkm:Ref431912048}
èzìzwátò zìbómbêtè èzí zìzyúmîtè\\
\gll e-zi-zwáto    zi-bomb-é̲te      e-zí    zi-zyu\textsubscript{H}m-í̲te\\
\textsc{aug}-\textsc{np}\textsubscript{8}-cloth  \textsc{sm}\textsubscript{8}-become\_wet-\textsc{stat}  \textsc{aug}-\textsc{dem}.\textsc{i}\textsubscript{8} \textsc{sm}\textsubscript{8}-dry-\textsc{stat}\\
\glt ‘These clothes are wet, these are dry.’ (ZF\_Elic14)
\z

The interpretation of -\textit{ite} as a focus on a current state rather than its origin also has temporal implications. The stative tends to refer to states that have a longer duration than states expressed by the near past perfective. This difference is illustrated in (\ref{bkm:Ref431916645}) and (\ref{bkm:Ref431916647}) with the verb \textit{búːk} ‘wake up’, where the use of the near past perfective expresses a state which has come about recently and is of a fleeting nature, whereas the use of the stative form expresses a state that is relatively more permanent.

\ea
\label{bkm:Ref431916645}
àbâncè bànàbûːkì\\
\gll a-ba-ánce    ba-na-búːk-i\\
\textsc{aug}-\textsc{np}\textsubscript{2}-child  \textsc{sm}\textsubscript{2}-\textsc{pst}-wake-\textsc{npst}.\textsc{pfv}\\
\glt ‘The children are awake (have woken up).’
\z

\ea
\label{bkm:Ref431916647}
àbâncè bàbúːkîtè\\
\gll a-ba-ánce     ba-buː\textsubscript{H}k-í̲te\\
\textsc{aug}-\textsc{np}\textsubscript{2}-child  \textsc{sm}\textsubscript{2}-wake-\textsc{stat}\\
\glt ‘The children are healthy.’ (ZF\_Elic14)
\z

With dynamic verbs, the interpretation of the stative depends on the presence of a result state. If present, the result state is targeted by the stative, similar to the use of the stative with change-of-state verbs. In (\ref{bkm:Ref443317582}), the dynamic verb \textit{zímburuk} ‘surround’ is used in the stative construction, and is interpreted as a currently valid state. In (\ref{bkm:Ref468268713}), the speaker uses the verb \textit{bar} ‘read’ with a stative suffix in order to stress that he has knowledge of the laws, since he has read, and is thus familiar with, a law book.

\ea
\label{bkm:Ref443317582}
èrápà rìzìmbúrùkìté njûò\\
\gll e-∅-rapá    ri-zi\textsubscript{H}mbú̲ruk-ite  N-júo\\
\textsc{aug}-\textsc{np}\textsubscript{5}-courtyard  \textsc{sm}\textsubscript{5}-surround-\textsc{stat}\textsubscript{} \textsc{np}\textsubscript{9}-house\\
\glt ‘The courtyard surrounds the house.’
\z

\ea
\label{bkm:Ref468268713}
ndìbárítè èmbúká ꜝyémìràhò\\
\gll ndi-bar-í̲te    e-N-buká    i-é=mi-raho\\
\textsc{sm}\textsubscript{1SG}-read-\textsc{stat}  \textsc{aug}-\textsc{np}\textsubscript{9}-book  \textsc{pp}\textsubscript{9}-\textsc{con}=\textsc{np}\textsubscript{4}-law\\
\glt ‘I’ve read a law book.’ (i.e., I know the law) (NF\_Elic15)
\z

Dynamic verbs without an associated result state, however, receive a progressive interpretation when used with the stative, i.e. the state expressed by the stative is a state of dancing, as in (\ref{bkm:Ref468269495}), a state of walking, as in (\ref{bkm:Ref416336349}), or a state of shouting, as in (\ref{bkm:Ref468269498}).

\ea
\label{bkm:Ref468269495}
\glll ndìzánîtè\\
ndi-zan-í̲te\\
\textsc{sm}\textsubscript{1SG}-dance-\textsc{stat}\\
\glt ‘I am busy dancing.’ (NF\_Elic15)
\z

\ea
\label{bkm:Ref416336349}
zyônà kàndíyèndêtè mùmùtêmwà\\
\gll zyóna    ka-ndí̲-end-é̲te    mu-mu-témwa\\
yesterday  \textsc{pst}.\textsc{ipfv}-\textsc{sm}\textsubscript{1SG}-go-\textsc{stat}  \textsc{np}\textsubscript{18}-\textsc{np}\textsubscript{3}-bush\\
\glt ‘Yesterday I was walking in the bush.’ (ZF\_Elic14)
\z

\ea
\label{bkm:Ref468269498}
kwìná òzyù ákàríhìtè\\
\gll ku-iná  o-zyu    á̲-ka\textsubscript{H}rí̲h-ite\\
\textsc{sm}\textsubscript{17}-be\_at  \textsc{aug}-\textsc{dem}.\textsc{i}\textsubscript{1}  \textsc{sm}\textsubscript{1}.\textsc{rel}-shout-\textsc{stat}\\
\glt ‘There’s someone who is shouting.’ (NF\_Elic15)
\z

The relevance of a result state can be seen with the verb \textit{beːzy} ‘carve’. In (\ref{bkm:Ref494983723}), the verb \textit{beːzy} ‘carve’ has a progressive reading with the stative construction, and a resultant state reading is not allowed. In (\ref{bkm:Ref494983787}), the verb \textit{beːzy} ‘carve’ is used with an object, giving the event a natural endpoint, and therefore the stative construction gives a result state reading (the context construed by the speaker was one where you describe a storage full of the carver’s handiwork). In this case, a progressive reading was not allowed.

\ea
\label{bkm:Ref494983723}
mùbèzyì àbéːzyêtè\\
\gll mu-bezyi  a-beːzy-é̲te\\
\textsc{np}\textsubscript{1}-carver  \textsc{sm}\textsubscript{1}-carve-\textsc{stat}\\
\glt ‘The carver is carving.’ *The carver has carved.
\z

\ea
\label{bkm:Ref494983787}
mùbèzyì àbéːzyêtè zìntù zìngîː\\
\gll mu-bezyi  a-beːzy-é̲te    zi-ntu    zi-ngíː\\
\textsc{np}\textsubscript{1}-carver  \textsc{sm}\textsubscript{1}-carve-\textsc{stat}  \textsc{np}\textsubscript{8}-thing  \textsc{pp}\textsubscript{8}-many\\
\glt ‘The carver has carved many things.’ *The carver is carving many things. (NF\_Elic17)
\z

The progressive use of \textit{-ite} with a dynamic verb usually describes an action with an extended duration, which sets the background for other events. The action described by the stative verb holds for a longer time span, during which several other, shorter actions take place. This is illustrated in (\ref{bkm:Ref416336349}) above, which is the first sentence of a short narrative about events that transpired during the narrators walk in the bush. All subsequent events take place during this walk in the bush, which is described by the stative verb \textit{kàndíyèndêtè} ‘I was walking’.

Except when describing a background state, the stative is rarely used with dynamic verbs, and progressive aspect is mostly expressed with the fronted infinitive construction or the auxiliary \textit{kwesi} (see \sectref{bkm:Ref445905308}).

\tabref{tab:9:7} summarizes the interpretations of the stative with different lexical aspectual classes.

\begin{table}
\label{bkm:Ref492378493}\caption{\label{tab:9:7}Interpretation of the stative construction}
\begin{tabularx}{\textwidth}{Xl}
\lsptoprule
Lexical aspect & Interpretation with the stative construction\\
\midrule
Change-of-state & Present (resultant) state \\
Dynamic: telic & Present (resultant) state\\
Dynamic: atelic & Progressive (long duration, background to other events)\\
Stative & ungrammatical\\
\lspbottomrule
\end{tabularx}
\end{table}

Although the interpretation of the stative construction can be quite different between change-of-state and dynamic verbs, its function can be best subsumed under the term stative, following Crane (2011, 2012, 2013). In the case of change-of-state verbs, the state expressed in the stative construction is the coda state that results from the nuclear change in state. In the case of dynamic verbs, the stative is interpreted as ‘to be in the state of doing something’; this may be interpreted as a progressive, but is usually interpreted as a background state, during which other actions take place. The past action that has led to the state described by the stative construction is never conceptualized.

The stative may be combined with other morphologically and periphrastically marked TAM constructions, such as the fronted infinitive, as illustrated in \sectref{bkm:Ref71533571}, or the persistive \textit{shí-} (see also \sectref{bkm:Ref445905502}), as in (\ref{bkm:Ref416342791}--\ref{bkm:Ref431917653}).

\ea
\label{bkm:Ref416342791}
\glll òshìrwárîtè\\
o-shi\textsubscript{H}-rwa\textsubscript{H}r-í̲te\\
\textsc{sm}\textsubscript{2SG}-\textsc{per}-be\_sick-\textsc{stat}\\
\glt ‘Are you still sick?’ (ZF\_Elic14)
\z

\ea
\label{bkm:Ref431917653}
\glll ndìshìbàzyìː\\
ndi-shi\textsubscript{H}-ba\textsubscript{H}-zyiː\textsubscript{H}\\
\textsc{sm}\textsubscript{1SG}-\textsc{per}-\textsc{om}\textsubscript{2}-know.\textsc{stat}\\
\glt ‘I still know them.’ (NF\_Elic15)
\z

To express a past state, the stative can co-occur with a remote or near past imperfective, as in (\ref{bkm:Ref71288826}--\ref{bkm:Ref71288827}). Both refer to a state that held in the past, but that no longer holds at the time of speaking. A state that held in the past and still holds in the present is expressed by the stative construction without past marking, as in (\ref{bkm:Ref494986964}).

\ea
\label{bkm:Ref71288826}
òzyú mùkêntù kànúnítè kònò hànó shànàkátì\\
\gll o-zyú    mu-kéntu  ka-á̲-nun-í̲te kono  hanó    sha-na-kat-í̲\\
\textsc{aug}-\textsc{dem}.\textsc{i}\textsubscript{1}  \textsc{np}\textsubscript{1}-woman  \textsc{pst}.\textsc{ipfv}-\textsc{sm}\textsubscript{1}-become\_fat-\textsc{stat}
but  \textsc{dem}.\textsc{ii}\textsubscript{16}  \textsc{inc}-\textsc{sm}\textsubscript{1}.\textsc{pst}-become\_thin-\textsc{npst}.\textsc{pfv}\\
\glt ‘This woman used to be fat, but now she’s thin.’ (NF\_Elic15)
\z

\ea
\label{bkm:Ref71288827}
\glll ndàkùrwárîtè\\
ndi-aku-rwa\textsubscript{H}r-í̲te\\
\textsc{sm}\textsubscript{1SG}-\textsc{npst}.\textsc{ipfv}-become\_sick-\textsc{stat}\\
\glt ‘I was sick (but I am not anymore).’
\z

\ea
\label{bkm:Ref494986964}
kùzwà zyônà àrwárîtè\\
\gll ku-zw-a    zyóna    a-rwa\textsubscript{H}r-í̲te\\
\textsc{inf}-come\_out-\textsc{fv}  yesterday  \textsc{sm}\textsubscript{1}-become\_sick-\textsc{stat}\\
\glt ‘S/he has been sick since yesterday.’ (NF\_Elic17)
\z
\section{Persistive}
\label{bkm:Ref445905502}\hypertarget{Toc75352697}{}
Persistive aspect is marked with a post-initial prefix \textit{shí-}. Its high tone does not surface when combined with a construction that uses melodic tone 4 (the deletion of underlying high tones), such as the present construction, as in (\ref{bkm:Ref495050822}). In constructions that do not use MT 4, such as the near past imperfective, the high tone of the prefix \textit{shí-} can be observed, as in (\ref{bkm:Ref495050836}).

\ea
\label{bkm:Ref495050822}
èntî ìshìhôrà\\
\gll e-n-tí    i-shi\textsubscript{H}-hó̲r-a\\
\textsc{aug}-\textsc{np}\textsubscript{9}-tea  \textsc{sm}\textsubscript{9}-\textsc{per}-cool-\textsc{fv}\\
\glt ‘The tea is still cooling down.’ (ZF\_Elic14)
\z

\ea
\label{bkm:Ref495050836}
\glll ndàkùshíbèrèkà\\
ndi-aku-shí-berek-a\\
\textsc{sm}\textsubscript{1SG}-\textsc{npst}.\textsc{ipfv}-\textsc{per}-work-\textsc{fv}\\
\glt ‘I was still working.’ (NF\_Elic17)
\z

A grammatical persistive marker is common in Bantu, where it is usually a reflex of *kɪ- \citep{Nurse2008}. This is also the case for the Fwe persistive marker \textit{shí-}.

The persistive expresses that an action started before, and is still ongoing at, the time period under discussion. When combined with a present construction, as in (\ref{bkm:Ref498095651}), the persistive indicates an event that started before, and is still ongoing at utterance time.

\ea
\label{bkm:Ref498095651}
\glll àshìŋórà\\
a-shi\textsubscript{H}-ŋo\textsubscript{H}r-á̲\\
\textsc{sm}\textsubscript{1}-\textsc{per}-write-\textsc{fv}\\
\glt ‘He is still writing.’ (NF\_Elic17)
\z

The persistive may also be interpreted as a temporarily interrupted event, as in (\ref{bkm:Ref99099628}), which indicates that the speaker has run before, and will run again later, but is currently not running.

\ea
\label{bkm:Ref99099628}
\glll ndìshìbùtúkà\\
ndi-shi\textsubscript{H}-bu\textsubscript{H}tuk-á̲\\
\textsc{sm}\textsubscript{1SG}-\textsc{per}-run-\textsc{fv}\\
\glt ‘I’ll run again.’ (NF\_Elic15)
\z

The persistive may even be used to indicate an event that has not yet started at or before utterance time, but will take place after utterance time, as in (\ref{bkm:Ref506302349}).

\ea
\label{bkm:Ref506302349}
ndìshìkàzyámbírá ꜝzóꜝkúryà\\
\gll ndi-shi\textsubscript{H}-ka-zyambir-á̲  zi-ó-ku-ry-á\\
\textsc{sm}\textsubscript{1SG}-\textsc{per}-\textsc{dist}-gather-\textsc{fv}  \textsc{pp}\textsubscript{8}-\textsc{con}-\textsc{inf}-eat\--\textsc{fv}\\
\glt ‘I still need to go and gather something to eat.’ (NF\_Elic17)
\z

The persistive may also occur with past constructions, indicating that an event started before, and is still ongoing at the past time interval that is currently discussed. As persistive is a subtype of imperfective aspect, specifying the internal structure of the event, it may only co-occur with the remote past imperfective, in (\ref{bkm:Ref492375625}), or the near past imperfective, in (\ref{bkm:Ref492375626}). It may not co-occur with the near past perfective, as the ungrammaticality of (\ref{bkm:Ref492375627}) shows.

\ea
\label{bkm:Ref492375625}
káshìkéːzyà mùrùshàrá ꜝrwángù\\
\gll ka-á̲-shi\textsubscript{H}-ké̲ːzy-a      mu-ru-shará    ru-angú\\
\textsc{pst}.\textsc{ipfv}-\textsc{sm}\textsubscript{1}-\textsc{per}-come-\textsc{fv}  \textsc{np}\textsubscript{18}-\textsc{np}\textsubscript{11}-back  \textsc{pp}\textsubscript{11}-\textsc{poss}\textsubscript{1SG}\\
\glt ‘It (the elephant) was still coming behind me.’ (ZF\_Narr13)
\z

\ea
\label{bkm:Ref492375626}
\glll àkùshíŋòrà\\
a-aku-shí-ŋor-a\\
\textsc{sm}\textsubscript{1}-\textsc{npst}.\textsc{ipfv}-\textsc{per}-write-\textsc{fv}\\
\glt ‘S/he was still writing.’ (NF\_Elic17)
\z

\ea
\label{bkm:Ref492375627}
\glll *ndàshívùrùmàtì\\
ndi-a-shí-vurumat-i\\
\textsc{sm}\textsubscript{1}-\textsc{pst}-\textsc{per}-close\_eyes-\textsc{npst}.\textsc{pfv}\\
\glt Intended: ‘My eyes are still closed.’
\z

The persistive can co-occur with other subtypes of imperfective aspect, such as the stative \textit{-ite} (see \sectref{bkm:Ref431984198}, examples (\ref{bkm:Ref416342791}) and (\ref{bkm:Ref431917653})), the progressive-marking fronted infinitive construction (see \sectref{bkm:Ref431917326}, example (\ref{bkm:Ref452976658})), and the progressive auxiliary \textit{kwesi} in (\ref{bkm:Ref506297904}).

\ea
\label{bkm:Ref506297904}
àshìkwèsì àfwêbà\\
\gll a-shi\textsubscript{H}-kwesi    a-fwé̲b-a\\
\textsc{sm}\textsubscript{1}-\textsc{per}-\textsc{prog}  \textsc{sm}\textsubscript{1}-smoke-\textsc{fv}\\
\glt ‘He is still smoking.’
\z

The persistive can be negated in two ways, giving different interpretations. With a negative prefix \textit{ka-/ta-} and a negative suffix \textit{-i}, the persistive expresses discontinuity: the situation used to hold, but does not hold anymore, as in (\ref{bkm:Ref72316521}--\ref{bkm:Ref72316522}).

\ea
\label{bkm:Ref72316521}
kàndíshìkwàngìtêː\\
\gll ka-ndí̲-shi\textsubscript{H}-kwa\textsubscript{H}ng-ite-í̲\\
\textsc{neg}-\textsc{sm}\textsubscript{1SG}-\textsc{per}-tired-\textsc{stat}-\textsc{neg}\\
\glt ‘I am no longer tired.’
\z

\ea
àbá bàntù kàbáshìkìzyîː\\
\gll a-bá    ba-ntu  ka-bá̲-shi\textsubscript{H}-ki\textsubscript{H}-zyi\textsubscript{H}-í̲\\
\textsc{aug}-\textsc{dem}.\textsc{i}\textsubscript{2}  \textsc{np}\textsubscript{2}-person  \textsc{neg}-\textsc{sm}\textsubscript{2}-\textsc{per}-\textsc{refl}-know.\textsc{stat}-\textsc{neg}\\
\glt ‘The people do not know each other anymore.’ (ZF\_Elic13)
\z

\ea
\label{bkm:Ref72316522}
àbàmbwá tàbáshìbbózì\\
\gll a-ba-mbwá    ta-bá-shi\textsubscript{H}-bbo\textsubscript{H}z-í̲\\
\textsc{aug}-\textsc{np}\textsubscript{2}-dog    \textsc{neg}-\textsc{sm}\textsubscript{2}-\textsc{per}-bark-\textsc{neg}\\
\glt ‘The dogs are no longer barking.’ (ZF\_Narr14)
\z

The persistive can also be negated with an auxiliary \textit{ni}\footnote{This auxiliary, which is not used in any other constructions, formally resembles the verb \textit{ina} ‘be at’ with a negative suffix \textit{-i}. While this may represent the historical origin of this auxiliary, it cannot be synchronically analyzed as such, as \textit{ina} does not take the negative suffix \textit{-i}; instead, Fwe uses a different lexical verb \textit{aazya}.}, followed by the main verb in the infinitive, to express negative continuity: the situation did not hold in the past, and still does not hold at the time of speaking, as in (\ref{bkm:Ref72316540}--\ref{bkm:Ref72316541}).

\ea
\label{bkm:Ref72316540}
kàndìshìní kùshéshìwà\\
\gll ka-ndi-shi\textsubscript{H}-ní  ku-shésh-iw-a\\
\textsc{neg}-\textsc{sm}\textsubscript{1SG}-\textsc{per}-be  \textsc{inf}-marry-\textsc{pass}-\textsc{fv}\\
\glt ‘I am not yet married.’ (ZF\_Elic14)
\z

\ea
\label{bkm:Ref72316541}
kàtùshíní kùríbònà\\
\gll ka-tu-shi\textsubscript{H}-ní  ku-rí-bon-a\\
\textsc{neg}-\textsc{sm}\textsubscript{1PL}-\textsc{per}-be  \textsc{inf}-\textsc{refl}-marry-\textsc{fv}\\
\glt ‘We have not yet seen each other.’ (NF\_Elic17)
\z
\section{Inceptive}
\label{bkm:Ref445905588}\hypertarget{Toc75352698}{}
The inceptive indicates that an action is starting or is about to happen, and is marked by a pre-initial prefix that can be realized as \textit{shi\nobreakdash-}, as in (\ref{bkm:Ref99099672}), \textit{she}-, as in (\ref{bkm:Ref99100067}), or \textit{sha}-, as in (\ref{bkm:Ref99100091}).

\ea
\label{bkm:Ref99099672}
\glll shìrìŋátùrà\\
shi-ri-ŋát-ur-a\\
\textsc{inc}-\textsc{sm}\textsubscript{5}-tear-\textsc{sep}.\textsc{tr}-\textsc{fv}\\
\glt ‘It [the sun] is starting to come up.’ (NF\_Elic15)
\z

\ea
\label{bkm:Ref99100067}
èzyúbà   shèrìmínà\\
\gll e-∅-zyúba  she-ri-min-á̲\\
\textsc{aug}-\textsc{np}\textsubscript{5}-sun  \textsc{inc}-\textsc{sm}\textsubscript{5}-set-\textsc{fv}\\
\glt ‘The sun is starting to set.’ (NF\_Narr15)
\z

\ea
\label{bkm:Ref99100091}
\glll shàndìkwângà\\
sha-ndi-kwá̲ng-a\\
\textsc{inc}-\textsc{sm}\textsubscript{1SG}-become\_tired-\textsc{fv}\\
\glt ‘I am getting tired.’ (ZF\_Elic14)
\z

The allomorphs of the inceptive prefix are subject to regional and free variation. The main form used in Namibian Fwe is \textit{shi-}, and the main form in Zambian Fwe is \textit{sha-}, but both varieties have a free allomorph \textit{she-}\footnote{A similar kind of variation is seen in the realization of another pre-initial prefix, the remoteness prefix, which is realized as \textit{na-} in Zambian Fwe, as \textit{ni-} in Namibian Fwe, and has a free allomorph \textit{ne-} in both varieties (see \sectref{bkm:Ref489260766} on the use of the remoteness prefix in the remote past perfective construction).}. In Namibian Fwe, the inceptive prefix can be realized with an alveolar fricative /s/ instead of a post-alveolar fricative /sh/. This variation, as all /s {\textasciitilde} sh/ variation in grammatical prefixes, is mainly speaker-dependent, but it is not observed in Zambian Fwe (cf. \sectref{bkm:Ref70695065}). \tabref{tab:9:8} summarizes the forms of the inceptive prefix. In addition to these base forms, vowel hiatus resolution between vowel-initial subject markers and the inceptive may result in the surface forms \textit{sha-}, analyzable as /shi-a/, and \textit{sho-}, analyzable as /shi-o/.

\begin{table}
\label{bkm:Ref490241419}\caption{\label{tab:9:8}Allomorphs and regional variation in the inceptive prefix}
\begin{tabular}{lll}
\lsptoprule
Form & Zambian Fwe & Namibian Fwe\\
\midrule
{\itshape shi-}  & not attested & default form\\
{\itshape she-} & free allomorph & free allomorph\\
{\itshape sha-} & default form & not attested\\
{\itshape se-} & not attested & inter-speaker variation\\
{\itshape si-} & not attested & inter-speaker variation\\
\lspbottomrule
\end{tabular}
\end{table}

The inceptive highlights the initial phases of an event, resulting in different interpretations depending on lexical aspect: inchoative (‘starting to’), proximative (‘be about to’), contrastive (‘now’, as opposed to earlier), completive (‘already’). The inchoative interpretation, highlighting the initial stages of the event, is available with dynamic verbs, as shown with \textit{kwesi tutuma} ‘shiver’ in (\ref{bkm:Ref497244502}) and \textit{hík} ‘cook’ in (\ref{bkm:Ref497244503}).

\ea
\label{bkm:Ref497244502}
shàkwèsì kwátùtúmà\\
\gll sha-a-kwesi    kwá-tutumá\\
\textsc{inc}-\textsc{sm}\textsubscript{1}-have  \textsc{np}\textsubscript{17}-shiver\\
\glt ‘She started shivering.’
\z

\largerpage
\ea
\label{bkm:Ref497244503}
àbó shìbàhíkà\\
\gll a-bó    shi-ba-hi\textsubscript{H}k-á̲\\
\textsc{aug}-\textsc{dem}.\textsc{iii}\textsubscript{2}  \textsc{inc}-\textsc{sm}\textsubscript{2}-cook-\textsc{fv}\\
\glt ‘They start cooking.’ (NF\_Narr15)
\z

The inchoative interpretation also occurs with change-of-state verbs, where it highlights the onset phase. This is illustrated with the change-of-state verb \textit{nun} ‘become fat’ in (\ref{bkm:Ref497244745}), where the use of the inceptive is interpreted as ‘starting to get fat’.

\ea
\label{bkm:Ref497244745}
hànó màzyûbà ndìryá nênjà kòbwéné \textbf{shèndìnúnà}\\
\gll hanó    ma-zyúba  ndi-ri-á̲  nénja ka-o-bwe\textsubscript{H}né̲  she-ndi-nun-á̲ \\
\textsc{dem}.\textsc{ii}\textsubscript{6}  \textsc{np}\textsubscript{6}-day  \textsc{sm}\textsubscript{1SG}-eat-\textsc{fv}  well
\textsc{neg}-\textsc{sm}\textsubscript{2SG}-see.\textsc{stat}  \textsc{inc}-\textsc{sm}\textsubscript{1SG}-become\_fat-\textsc{fv}\\
\glt ‘These days I’m eating well, don’t you see \textbf{I’m} \textbf{starting} \textbf{to} \textbf{get} \textbf{fat}?’ (NF\_Elic15)
\z

With change-of-state verbs without an onset, the inceptive cannot highlight the initial stages of the nuclear phase, as the nucleus is too short, nor the onset phase, as the event lacks an onset. Instead, the inceptive highlights the phase just before the event, giving a proximative interpretation, as in (\ref{bkm:Ref507158243}--\ref{bkm:Ref507158244}).

\ea
\label{bkm:Ref507158243}
èsáká shàrìŋàtúkà\\
\gll e-∅-saká    sha-ri-ŋatuk-á̲\\
\textsc{aug}-\textsc{np}\textsubscript{5}-bag  \textsc{inc}-\textsc{sm}\textsubscript{5}-break-\textsc{fv}\\
\glt ‘The bag is about to break.’ (ZF\_Elic14)
\z

\ea
\label{bkm:Ref507158244}
énswí shàyìfwâ\\
\gll e-N-swí    sha-i-fw-á̲\\
\textsc{aug}-\textsc{np}\textsubscript{9}-fish  \textsc{inc}-\textsc{sm}\textsubscript{9}-die-\textsc{fv}\\
\glt ‘The fish is about to die.’ (i.e., the fish is out of the water, flapping about, and clearly almost, but not quite, dead) (ZF\_Elic14)
\z

This use of the inceptive prefix is also seen with dynamic verbs that have a short nucleus, such as \textit{nanuk} ‘leave’, \textit{zu} ‘go out’, and \textit{u} ‘fall’. Again, the lack of onset and the short nucleus means that the phase highlighted by the inceptive is the phase right before the event, as in (\ref{bkm:Ref99100600}--\ref{bkm:Ref497244747}).

\ea
\label{bkm:Ref99100600}
kàtùàmbáhùrì kàkúrì shàndìnànúkà\\
\gll ka-tu-amb-á̲-ur-i      kakúri    sha-ndi-nanuk-á̲\\
\textsc{neg}-\textsc{sm}\textsubscript{1PL}-talk-\textsc{pl}1-\textsc{sep}.\textsc{tr}-\textsc{neg}  because  \textsc{inc}-\textsc{sm}\textsubscript{1SG}-leave-\textsc{fv}\\
\glt ‘We cannot talk, I am about to leave.’ (ZF\_Elic14)
\z

\ea
shìbàkàzwá ꜝhánjè hàhánò\\
\gll shi-ba-ka-zu-á̲    ha-njé    ha-hanó\\
\textsc{inc}-\textsc{sm}\textsubscript{2}-\textsc{dist}-go\_out-\textsc{fv}  \textsc{np}\textsubscript{16}-outside  now\\
\glt ‘S/he is about to walk out right now.’
\z

\ea
\label{bkm:Ref497244747}
ìn’ énjûò shèyìwá ꜝyínà\\
\gll iná    e-N-júo    she-i-w-á̲    iná\\
\textsc{dem}.\textsc{iv}\textsubscript{9}  \textsc{aug}-\textsc{np}\textsubscript{9}-house  \textsc{inc}-\textsc{sm}\textsubscript{9}-fall-\textsc{fv}  \textsc{dem}.\textsc{iv}\textsubscript{9}\\
\glt ‘That house is falling apart/about to fall apart (i.e. in a very bad state).’ (NF\_Elic15)
\z

A contrastive interpretation of the inceptive is obtained with verbs that are conceptualized as unbounded, as without a clear starting point. Example (\ref{bkm:Ref506303056}) is cited from a conversation, in which the speaker describes marriage customs in modern times. The modern times that he describes do not have a clear starting point (though logic dictates that they must have started at some point), and as such the verbs used to describe them are conceptualized as lacking a clear onset. In these cases, the use of the inceptive causes an interpretation of ‘now (in contrast to earlier/ elsewhere)’.

\ea
\label{bkm:Ref506303056}
mwáìnò ènàkò \textbf{shìtúꜝ}\textbf{hárà} mbàmúwânè màfòní \textbf{shàbábèrèkìsâ}\\
\gll mwá-ino  e-N-nako    shi-tú̲-ha\textsubscript{H}r-á̲ mba-mú̲-wá̲n-e N-ma-foní    sha-bá̲-berek-is-á̲ \\
\textsc{np}\textsubscript{18}-\textsc{dem}.\textsc{ii}\textsubscript{9}  \textsc{aug}-\textsc{np}\textsubscript{9}-time  \textsc{inc}-\textsc{sm}\textsubscript{1PL}.\textsc{rel}\textsubscript{\-}-live-\textsc{fv}
\textsc{near}.\textsc{fut}-\textsc{sm}\textsubscript{2PL}-find-\textsc{pfv}.\textsc{sbjv}
\textsc{cop}-\textsc{np}\textsubscript{6}-phone  \textsc{inc}-\textsc{sm}\textsubscript{2}.\textsc{rel}-work-\textsc{caus}-\textsc{fv}\\
\glt ‘In this time that \textbf{we} \textbf{now} \textbf{live} in, you will find that \textbf{they} \textbf{are} \textbf{now} \textbf{using} \textbf{phones}.’ (ZF\_Conv13)
\z

This contrastive interpretation is also used with change-of-state verbs in a stative construction, as in (\ref{bkm:Ref99100651}).

\ea
\label{bkm:Ref99100651}
màsíkùsîkù kàndíshùwìrè njârà hànó \textbf{shàndìkútîtè}\\
\gll ma-síkusíku  ka-ndí̲-shu\textsubscript{H}-ire    N-jára hanó    sha-ndi-kut-í̲te\\
\textsc{np}\textsubscript{6}-morning  \textsc{pst}.\textsc{ipfv}-\textsc{sm}\textsubscript{1SG}-feel-\textsc{stat}  \textsc{np}\textsubscript{9}-hunger
\textsc{dem}.\textsc{ii}\textsubscript{16}  \textsc{inc}-\textsc{sm}\textsubscript{1SG}-become\_full-\textsc{stat}\\
\glt ‘This morning I was hungry, but \textbf{now} \textbf{I} \textbf{am} \textbf{full}.’ (ZF\_Elic14)
\z

The inceptive may also give a contrastive ‘now’ interpretation with verbs in the near past perfective (NPP), as in (\ref{bkm:Ref98835411}--\ref{bkm:Ref98835412}). As discussed in \sectref{bkm:Ref488767483}, the NPP usually gives a present state reading with change-of-state verbs. Because this construction is perfective, presenting an event as lacking internal structure, the inceptive cannot be interpreted as highlighting the initial phases of the event, and is rather used to contrast the current situation with a different, previous situation.

\ea
\label{bkm:Ref98835411}
cwàré bùryénà \textbf{shìbáꜝ}\textbf{názyìbì} báꜝmúꜝkwáꜝmé ꜝwénù\\
\gll cwaré  bu-ryená    shi-bá-ná-zyib-i  bá-mú-kwámé  u-enú \\
then  \textsc{np}\textsubscript{14}-like\_that  \textsc{inc}-\textsc{sm}\textsubscript{2}-\textsc{pst}-know-\textsc{npst}.\textsc{pfv}
\textsc{np}\textsubscript{2}-\textsc{np}\textsubscript{1}-man  \textsc{pp}\textsubscript{1}-\textsc{poss}\textsubscript{2PL}\\
\glt ‘Then as you see, your husband \textbf{has} \textbf{now} \textbf{become} \textbf{aware}.’
\z

\ea
\glll shàbànàbûːkì\\
sha-ba-na-búːk-i\\
\textsc{inc}-\textsc{sm}\textsubscript{2}-\textsc{pst}-wake-\textsc{npst}.\textsc{pfv}\\
\glt ‘They are now awake.’ (NF\_Narr15)
\z

\ea
\label{bkm:Ref98835412}
òzyú mùkêntù kànúnítè kònò hànó \textbf{shànàkátì}\\
\gll o-zyú    mu-kéntu  ka-a-nun-í̲te kono  hanó    sha-na-kat-í̲\\
\textsc{aug}-\textsc{dem}.\textsc{i}\textsubscript{1}  \textsc{np}\textsubscript{1}-woman  \textsc{pst}.\textsc{ipfv}-\textsc{sm}\textsubscript{1}-become\_fat-\textsc{stat}
but  \textsc{dem}.\textsc{ii}\textsubscript{16}  \textsc{inc}-\textsc{sm}\textsubscript{1}.\textsc{pst}-become\_thin-\textsc{npst}.\textsc{pfv}\\
\glt ‘This woman used to be fat, but \textbf{now} \textbf{she’s} \textbf{thin}.’ (NF\_Elic15)
\z

The inceptive with verbs in the near past perfective may also be interpreted as completive, e.g. it adds a sense of ‘already’, as in (\ref{bkm:Ref443663081}) and (\ref{bkm:Ref443910289}), or ‘yet’, as in (\ref{bkm:Ref443910521}). Again, the inceptive is used to contrast a current situation with an earlier one, similar to the contrastive interpretation seen in (\ref{bkm:Ref98835411}--\ref{bkm:Ref98835412}).

\ea
\label{bkm:Ref443663081}
shìryámìnì zyûbà\\
\gll shi-ri-á-min-i      ∅-zyúba\\
\textsc{inc}-\textsc{sm}\textsubscript{5}\--\textsc{pst}-set-\textsc{npst}.\textsc{pfv}    \textsc{np}\textsubscript{5}-sun\\
\glt ‘The sun had already set.’ (ZF\_Narr15)
\z

\ea
\label{bkm:Ref443910289}
shètwàtángì kàré kúryà\\
\gll she-tu-a-táng-i      karé    ku-rí-a\\
\textsc{inc}-\textsc{sm}\textsubscript{1PL}-\textsc{pst}-start-\textsc{npst}.\textsc{pfv}  already  \textsc{inf}-eat-\textsc{fv}\\
\glt ‘They’ve already started to eat.’ (ZF\_Elic14)
\z

\ea
\label{bkm:Ref443910521}
bèshò shàbànàhúrì\\
\gll ba-esh-o    sha-ba-na-hur-í̲\\
\textsc{np}\textsubscript{2}-father-\textsc{poss}\textsubscript{2SG}  \textsc{inc}-\textsc{sm}\textsubscript{2}-\textsc{pst}-arrive-\textsc{npst}.\textsc{pfv}\\
\glt ‘Has your father arrived yet?’ (ZF\_Elic13)
\z

The inceptive can also be prefixed to nouns, interpreted as inchoative, as in (\ref{bkm:Ref72317032}--\ref{bkm:Ref72317128}), contrastive, as in (\ref{bkm:Ref72317137}--\ref{bkm:Ref72317138}), or completive, as in (\ref{bkm:Ref72317219}--\ref{bkm:Ref72317220}).

\ea
\label{bkm:Ref72317032}
shórùmwî kàrè\\
\gll sha-ó-ru-mwí  kare\\
\textsc{inc}-\textsc{aug}-\textsc{np}\textsubscript{11}-heat  already\\
\glt ‘It’s becoming summer.’ (NF\_Elic15)
\z

\ea
\label{bkm:Ref72317128}
kàréː kàréː àbàcèmbèrè shóꜝndávù\\
\gll karé  karé  a-ba-cembere    shí-o-ndavú\\
now  now  \textsc{aug}-\textsc{np}\textsubscript{2}-old\_woman  \textsc{inc}-\textsc{aug}-lion\\
\glt ‘The old woman immediately turned into a lion.’ (NF\_Narr17)
\z

\ea
\label{bkm:Ref72317137}
òmùndáré ꜝsómùbîzù\\
\gll o-mu-ndaré    sí-o-mu-bízu\\
\textsc{aug}-\textsc{np}\textsubscript{3}-maize  \textsc{inc}-\textsc{aug}-\textsc{np}\textsubscript{3}-something\_ripe\\
\glt ‘The maize is now ripe.’ (NF\_Elic17)
\z

\ea
\label{bkm:Ref72317138}
sóbùhùbà cáhà òkàhùràkò\\
\gll sí-o-bu-huba  cáha  o-ka-hur-a=ko\\
\textsc{inc}-\textsc{aug}-\textsc{np}\textsubscript{14}-easy  very  \textsc{aug}-\textsc{inf}.\textsc{dist}-arrive-\textsc{fv}=\textsc{loc}\textsubscript{17}\\
\glt ‘It is now very easy to reach there.’ (discussing a place where cattle are watered; in earlier times, it could only be reached with ox carts and sledges, but now, the road is tarred and accessible to cars.) (NF\_Narr17
\z

\ea
\label{bkm:Ref72317219}
shémàsíkù kàrêː\\
\gll shé-N-ma-síku  karéː\\
\textsc{inc}-\textsc{cop}-\textsc{np}\textsubscript{6}-night  already\\
\glt ‘It’s already night.’ (NF\_Elic15)
\z

\ea
\label{bkm:Ref72317220}
àh’ átôndà shécìbàkà shìcàhítìhò\\
\gll a-ha    á̲-tó̲nd-a shé-ci-baka    shi-ci-a-hít-i=ho\\
\textsc{aug}-\textsc{dem}.\textsc{i}\textsubscript{16}  \textsc{sm}\textsubscript{1SG}.\textsc{rel}-watch-\textsc{fv}
\textsc{inc}-\textsc{np}\textsubscript{7}-place  \textsc{inc}-\textsc{sm}\textsubscript{7}-\textsc{pst}-pass-\textsc{pst}=\textsc{loc}\textsubscript{16}\\
\glt ‘When she looked, he had already covered a large place.’ (Lit: ‘a place had already passed.’) (NF\_Narr15)
\z

The nominal use of the inceptive has most likely developed out of its verbal use, if the prefix was originally used on a verb \textit{ri} ‘be’, followed by the loss of the verbal base \textit{ri} and the reanalysis of the inceptive as a nominal prefix, as schematized in (\ref{bkm:Ref490244755}).\footnote{This grammaticalization also involves a tonal change, from a low-toned inceptive on verbs to a high-toned inceptive prefix as it is usually realized on nouns. This is the result of the high tone of the nominal augment; as discussed in \sectref{bkm:Ref444175456}, augments have a floating high tone that is never realized on the augment prefix itself, but always on the immediately preceding syllable.}

\ea
\label{bkm:Ref490244755}
\ea
Putative source construction\\
shàrì mwâncè\\
\gll shi-a-ri  o-mu-ánce\\
\textsc{inc}-\textsc{sm}\textsubscript{1}-be  \textsc{aug}-\textsc{np}\textsubscript{1}-child\\
\glt ‘S/he is starting to be/is becoming a child.’

\ex
Loss of ri ‘be’ \\
shì mwâncè\\
\gll shi  o-mu-ánce\\
\textsc{inc}  \textsc{aug}-\textsc{np}\textsubscript{1}-child\\

\ex
Reanalysis of inceptive as a nominal prefix\\
\glll shómwâncè\\
shí-o-mu-ánce\\
\textsc{inc}-\textsc{aug}-\textsc{np}\textsubscript{1}-child\\
\glt ‘S/he is starting to be/becoming a child.’
\z
\z

The inceptive prefix may have developed from a lexical verb \textit{shak} ‘want, like, love, need, look for’. Grammaticalization of earlier lexical verbs of volition into markers of proximative aspect (‘be about to’) is well-attested in African languages \citep{Heine1994}. The volitional element of the original lexical verb can still be seen in some uses of the inceptive \textit{sha}-. For instance, the utterance in (\ref{bkm:Ref468374722}) was considered dubious, because it could be interpreted as the speaker wanting to become sick.

\ea
\label{bkm:Ref468374722}
?shèndìrwârà\\
\gll she-ndi-rwá̲r-a\\
\textsc{inc}-\textsc{sm}\textsubscript{1SG}-be\_sick-\textsc{fv}\\
\glt ‘I am getting sick/I want to get sick.’ (NF\_Elic15)
\z

Furthermore, the lexical verb \textit{shak} ‘want’ is also used to express meanings similar to the inceptive: in (\ref{bkm:Ref507160488}), the verb \textit{shak} is not used to express volition, but to express an event about to happen.

\ea
\label{bkm:Ref507160488}
òmvúrà shàshàk’ ókùshôkà\\
\gll o-∅-rain    shi-a-shak-á̲    o-ku-shók-a\\
\textsc{aug}-\textsc{np}\textsubscript{1a}-rain  \textsc{inc}-\textsc{sm}\textsubscript{1}-want-\textsc{fv}  \textsc{aug}-\textsc{inf}-fall-\textsc{fv}\\
\glt ‘The rain is about to fall.’
\z

These traces of volitional semantics in the inceptive prefix also argue against an alternative analysis, which is that the inceptive prefix in Fwe is a borrowing from Lozi. Lozi makes use of a prefix \textit{sè}-, which “expresses ‘already’, ‘and then’, ‘now’, or ‘soon’” \citep[199]{Gowlett1967}. Similar verbal prefixes are attested in other languages of the Sotho group \citep[143]{Doke1954}. However, as the Lozi suffix lacks the implication of volition, a Fwe-internal grammaticalization scenario from the verb \textit{shak} ‘want’ is a more plausible explanation.

