\chapter{Negation}
\label{bkm:Ref100140375}\label{bkm:Ref98752806}\label{bkm:Ref97905708}\hypertarget{Toc75352706}{}
Negation in Fwe is marked through verbal affixes, auxiliaries, and combinations thereof, depending on the TAM construction. The pre-initial prefix \textit{ka-} (Namibian Fwe) \textit{/ta-} (Zambian Fwe) is used to negate indicative verbs. Fwe also has two post-initial negative suffixes, \textit{ásha-}, used with subjunctive verb forms, and \textit{shá-}, used with infinitive verb forms. A negative final vowel suffix \textit{-i} is seen in certain constructions, but it is never the only marker of negation. Tone also plays a role in negation: the present and stative constructions have different tonal patterns for affirmative and negative forms. \tabref{tab:12:1} gives an overview of the different negative strategies used in Fwe.

\begin{table}
\label{bkm:Ref489278666}\caption{\label{tab:12:1}Negation}

\begin{tabularx}{\textwidth}{llQ}
\lsptoprule
Position & Form & Inflections in which it is used\\
\midrule
pre-initial & \textit{ka-} (Namibian Fwe) & present, near past \\
            & \textit{ta-} (Zambian Fwe) & perfective, stative\\
post-initial & \textit{(á)sha- / (á)sa-} & subjunctive/imperative\\
             & \textit{shá- / sá-}  & infinitive\\
final vowel suffix & {\itshape -i} & present, subjunctive\\
auxiliary & {\itshape aazyá} & stative, fronted-infinitive construction\\
auxiliary & {\itshape ka-/ta-ri} & remote past, future, past progressive, past imperfective, nominal predicates\\
\lspbottomrule
\end{tabularx}
\end{table}
\section{Negation of indicative verb forms}
\label{bkm:Ref490843739}\hypertarget{Toc75352707}{}
Indicative verb forms are negated with a pre-initial prefix \textit{ka-} or \textit{ta-}, and the final vowel suffix \textit{-i}. This is illustrated with the present indicative in (\ref{bkm:Ref99103972}--\ref{bkm:Ref99103973}).

\ea
\label{bkm:Ref99103972}
\glll ndìúrà\\
ndi-ur-á̲\\
\textsc{sm}\textsubscript{1SG}-buy-\textsc{fv}\\
\glt ‘I buy.’
\z

\ea
\glll kàndìúrì\\
ka-ndi-ur-í̲\\
\textsc{neg}-\textsc{sm}\textsubscript{1SG}-buy-\textsc{neg}\\
\glt ‘I don’t buy.’ (NF\_Elic15)
\z

\ea
\label{bkm:Ref99103973}
\glll tàndìúrì\\
ta-ndi-ur-í̲\\
\textsc{neg}-\textsc{sm}\textsubscript{1SG}-buy-\textsc{neg}\\
\glt ‘I don’t buy.’ (ZF\_Elic14)
\z

Present tense verbs also change their tone pattern when negated. Affirmative present verbs take MT 1 and 4 (see \sectref{bkm:Ref71539267}), but negated present verbs take only MT 3. The tonal difference between the affirmative and negative present is illustrated in (\ref{bkm:Ref74909280}).

\ea
\label{bkm:Ref74909280}
kàndìzíbârì (cf. ndìzìbárà ‘I forget’)\\
\gll ka-ndi-zibá̲r-i\\
\textsc{neg}-\textsc{sm}\textsubscript{1SG}-forget-\textsc{neg}\\
\glt ‘I don’t forget.’ (NF\_Elic15)
\z

The negative suffix \textit{-i} cannot be directly preceded by a passive suffix \textit{-(i)w}. When a passive verb is negated, the negative suffix \textit{-i} is not used, but rather the default final vowel suffix \textit{-a}, as in (\ref{bkm:Ref505773692}). However, when the passive suffix -\textit{(i)w} is separated from the final vowel by the occurrence of the habitual suffix \textit{-ang}, the negative suffix \textit{-i} is used, as in (\ref{bkm:Ref505773694}). Incompatibility with the passive suffix is also observed for the near past perfective suffix \textit{\--i} (see \sectref{bkm:Ref488767671}).\footnote{There are also other cases of overlap between the near past perfective and the negative present tense form. Both forms use a suffix \textit{\-–i}, neither of which ever causes spirantization (as opposed to certain other suffixes with /i/, where spirantization is attested in lexicalized cases). Both forms use melodic tone 3, which is assigned to the second stem syllable. In spite of these formal similarities, however, there is little semantic overlap between the negative and near past perfective meanings.}

\ea
\label{bkm:Ref505773692}
\glll kàcìhîkwà\\
ka-ci-hík-w-a\\
\textsc{neg}-\textsc{sm}\textsubscript{7}-cook-\textsc{pass}-\textsc{fv}\\
\glt ‘It cannot be cooked.’ (NF\_Elic15)
\z

\ea
\label{bkm:Ref505773694}
\glll báshàshéshíwàngì\\
ba-ásha-shesh-í̲w-ang-i\\
\textsc{sm}\textsubscript{2}-\textsc{neg}-marry-\textsc{pass}-\textsc{hab}-\textsc{neg}\\
\glt ‘They should not be married.’ (ZF\_Conv13)
\z

Of the two forms of the negative prefix, \textit{ka-} is mainly used in Namibian Fwe, and \textit{ta-} in Zambian Fwe. This areal distribution is also seen in several other Bantu languages of the region, including those of the Bantu Botatwe subgroup, such as Totela and Subiya, but also Yeyi, not part of Bantu Botatwe. Totela, which, like Fwe, has a Zambian and a Namibian variety, exhibits the same distribution as Fwe; \textit{ta-} is used in the Zambian variety \citep[82]{Crane2011}, and \textit{ka-} in the Namibian variety. Subiya and Yeyi, only spoken in Namibia, both only use \textit{ka-} (\citealt{Jacottet1896}: 57-58; \citealt{Seidel2008}: 405-408). The distribution of the \textit{ka-} and \textit{ta-} forms of the negative prefix thus more or less follows the national border between Zambia and Namibia.

The negative prefix \textit{ta}-/\textit{ka}- is placed directly before the subject marker of the verb. When the subject marker consists of a vowel only, vowel hiatus resolution takes place between the vowel of the negative prefix and the vowel of the subject marker. Aside from subject markers affected by predictable rules of vowel hiatus resolution, there are no special forms of subject markers used exclusively with negative verbs, as opposed to a tendency often observed in Bantu languages for subject markers of the first person singular to have a special negated form: the negated form of the first person singular is a morphologically regular combination of the negative prefix with the first person singular subject marker \textit{ndi-}, as in (\ref{bkm:Ref74909320}).

\ea
\label{bkm:Ref74909320}
tàndìbútùkì (cf. ndìbùtúkà ‘I run’)\\
\gll ta-ndi-bú̲tuk-i\\
\textsc{neg}-\textsc{sm}\textsubscript{1SG}-run-\textsc{neg}\\
\glt ‘I don’t run.’ (ZF\_Elic14)
\z

The prefix \textit{ka-/ta-} is also used to negate the near past perfective. This tense uses a past suffix \textit{-i} which is homophonous with the negative suffix \textit{-i}. Negated verbs in the near past perfective have the same tonal pattern as their affirmative counterparts, as illustrated in (\ref{bkm:Ref99104398}).

\ea
\label{bkm:Ref99104398}
kàndàzíbònì (cf. ndàzíbònì ‘I’ve seen them’)\\
\gll ka-ndi-a-zí-bon-i\\
\textsc{neg}-\textsc{sm}\textsubscript{1SG}-\textsc{pst}-\textsc{om}\textsubscript{10}-see-\textsc{npst}.\textsc{pfv}\\
\glt ‘I haven’t seen them.’ (NF\_Elic15)
\z

Verbs in the stative construction are also negated with the prefix \textit{ka-/ta-}, combined with lengthening of the last vowel of the verb, which is not seen in the affirmative stative. This can be seen as influence from the negative suffix \textit{-i}, which contributes an extra mora to the last vowel of the verb, but its vowel quality merges with the last vowel of the verb (/e/ or /i/, depending on the allomorph of the stative suffix, see \sectref{bkm:Ref431984198}). The length difference in the last vowel of affirmative and negative stative verbs is illustrated in (\ref{bkm:Ref75180218}--\ref{bkm:Ref75180219}).

\ea
\label{bkm:Ref75180218}
kàìbòrètêː (cf. ìbórêtè ‘it is rotten’)\\
\gll ka-i-bor-ete-í̲\\
\textsc{neg}-\textsc{sm}\textsubscript{1SG}-rot-\textsc{stat}-\textsc{neg}\\
\glt ‘It is not rotten.’ (ZF\_Elic14)
\z

\ea
\label{bkm:Ref75180219}
kàndìyìzyîː (cf. ndìyìzyì ‘I know it’)\\
\gll ka-ndi-i\textsubscript{H}-zyi-í̲\\
\textsc{neg}-\textsc{sm}\textsubscript{1SG}-\textsc{om}\textsubscript{9}-know.\textsc{stat}-\textsc{neg}\\
\glt ‘I don’t know it.’ (NF\_Elic15)
\z

The negation of stative verbs also involves a change in tone pattern. Affirmative stative verbs take a high tone on the second stem syllable (MT 3, see \sectref{bkm:Ref71540417}). Negated stative verbs take a high tone on the last mora of the verb (MT 1, see \sectref{bkm:Ref71540433}). The deletion of the lexical tone of the root, as seen in the affirmative stative, also affects the negated stative. Optional high tone spread, i.e. the copying of high tones up to the first syllable of the verb stem, is never seen in negated stative verbs, though it is very common in affirmative stative verbs. The different tone patterns of affirmative and negated stative verbs are illustrated in (\ref{bkm:Ref98833526}--\ref{bkm:Ref98833527}).

\ea
\label{bkm:Ref98833526}
tàndìshèshètêː (cf. ndìshéshêtè ‘I am married’)\\
\gll ta-ndi-she\textsubscript{H}sh-ete-í̲\\
\textsc{neg}-\textsc{sm}\textsubscript{1SG}-marry-\textsc{stat}-\textsc{neg}\\
\glt ‘I am not married.’
\z

\ea
\label{bkm:Ref98833527}
tàtùkàtìtêː (cf. tùkátîtè ‘we are thin’)\\
\gll ta-tu-kat-ite-í̲\\
\textsc{neg}-\textsc{sm}\textsubscript{1PL\-}-become\_thin-\textsc{stat}-\textsc{neg}\\
\glt ‘We are not thin.’ (ZF\_Elic14)
\z
\section{Negation of imperative and subjunctive verb forms}
\label{bkm:Ref499029715}\hypertarget{Toc75352708}{}
Imperative and subjunctive verb forms are negated with a post-initial prefix \mbox{\textit{ásha-},} combined with the negative suffix \textit{-i}, as in (\ref{bkm:Ref99104532}--\ref{bkm:Ref99104535}). In Namibian Fwe, the prefix has a free variant \textit{ása-}, as in (\ref{bkm:Ref99104547}) (see \sectref{bkm:Ref70695065} on the free variation between /s/ and /sh/ in grammatical prefixes).

\ea
\label{bkm:Ref99104532}
wáshàyáshàmì òkìmúmé bùryò\\
\gll o-ásha-yásham-i        o-ki\textsubscript{H}-mum-é̲    bu-ryo\\
\textsc{sm}\textsubscript{2SG}-\textsc{neg}.\textsc{sbjv}-open\_mouth-\textsc{neg}  \textsc{sm}\textsubscript{2SG}-\textsc{refl}-close-\textsc{pfv}.\textsc{sbjv}  \textsc{np}\textsubscript{14}-only\\
\glt ‘Don’t open your mouth, just close it like that.’ (ZF\_Narr13)
\z

\ea
mwáshàbútùkì câhà\\
\gll mu-ásha-bútuk-i    cáha\\
\textsc{sm}\textsubscript{2PL}-\textsc{neg}.\textsc{sbjv}-run-\textsc{neg}  very\\
\glt ‘Don’t go too fast.’ (NF\_Elic17)
\z

\ea
\label{bkm:Ref99104535}
ndìryá bùryó kànînì òkùtêyè ndáshànúnì\\
\gll ndi-ry-á  bu-ryó  ka-níni okutéye  ndi-ásha-nun-í̲ \\
\textsc{sm}\textsubscript{1SG}-eat-\textsc{fv}  \textsc{np}\textsubscript{14}-only  \textsc{adv}-little
that    \textsc{sm}\textsubscript{1SG}-\textsc{neg}.\textsc{sbjv}-become\_fat-\textsc{neg}\\
\glt ‘I only eat a little, so that I do not become fat.’ (NF\_Elic17)
\z

\ea
\label{bkm:Ref99104547}
kònó náàryá òkùtêyè ásàrémùhì\\
\gll konó  ná̲-a-a-ry-á      okutéye  á-sa-rémuh-i\\
but  \textsc{rem}-\textsc{sm}\textsubscript{1}-\textsc{pst}-eat-\textsc{fv}  that    \textsc{sm}\textsubscript{1}-\textsc{neg}.\textsc{sbjv}-find\_out-\textsc{neg}\\
\glt ‘But she ate, so that he wouldn’t find out.’ (NF\_Narr17)
\z

The negative subjunctive/imperative prefix may be realized as \textit{ásha-/ása-} or \textit{sha}-/\textit{sa-}. When the first vowel /a/ is dropped, the high tone of the suffix is realized on the subject marker, as in (\ref{bkm:Ref99104565}).

\ea
\label{bkm:Ref99104565}
\glll músàndìtáfùnì\\
mú-sa-ndi-táfun-i\\
\textsc{sm}\textsubscript{2PL}-\textsc{neg}.\textsc{sbjv}-\textsc{om}\textsubscript{1SG}-chew-\textsc{neg}\\
\glt ‘Don’t eat me!’ (NF\_Narr17)
\z
\section{Negation of infinitive verb forms}
\hypertarget{Toc75352709}{}
Infinitive verb forms are negated with a post-initial prefix \textit{shá-}, as in (\ref{bkm:Ref99104688}--\ref{bkm:Ref99104690}). In Namibian Fwe, the prefix \textit{shá-} has a free variant \textit{sá-}, as in (\ref{bkm:Ref99104701}) (/s/ and /sh/ are interchangeable in grammatical prefixes; see \sectref{bkm:Ref70695065}).

\ea
\label{bkm:Ref99104688}
kùshátèèzà mbùkáꜝbábù\\
\gll ku-shá-teez-a    N-bu-kábabú\\
\textsc{inf}-\textsc{neg}.\textsc{inf}-listen-\textsc{fv}  \textsc{cop}-\textsc{np}\textsubscript{14}-problem\\
\glt ‘Not listening is a problem.’ (NF\_Elic17)
\z

\ea
\label{bkm:Ref99104690}
nàngá mwínàkò yóbùkòbà mbàngíː bànàdàmwá kókùsházyìbà òkùbàrà ècìpùrá nècìŋòrétwà ìyé cámàkúwà èwé mpàhó àkèːzyà kúkàrà nòrì mùntù ókùsìhà\\
\gll nangá  mú-e-N-nako  i-ó=bu-koba      N-ba-ngíː ba-na-dam-w-á̲    \textbf{kó-ku-shá-zyib-a}      o-ku-bar-a  e-ci-purá    ne-ci-ŋo\textsubscript{H}r-é̲twa    iyé  ci-á-ma-kuwá ewe    N-pa-hó    a-keːzy-a    kú-kar-a na=o-ri    mu-ntu  u-ó=ku-sih-a  \\
even  \textsc{np}\textsubscript{18}-\textsc{aug}-\textsc{np}\textsubscript{9}-time  \textsc{pp}\textsubscript{9}-\textsc{con}=\textsc{np}\textsubscript{14}-apartheid  \textsc{cop}-\textsc{pp}\textsubscript{2}-many
\textsc{sm}\textsubscript{2}-\textsc{pst}-beat-\textsc{pass}-\textsc{fv}  \textsc{adv}-\textsc{inf}-\textsc{neg}.\textsc{inf}-know-\textsc{fv}  \textsc{aug}-\textsc{inf}-read-\textsc{fv}
\textsc{aug}-\textsc{np}\textsubscript{7}-chair  \textsc{rem}-\textsc{sm}\textsubscript{7}-write-\textsc{stat}-\textsc{pass}  that  \textsc{pp}\textsubscript{7}-\textsc{con}=\textsc{np}\textsubscript{6}-white
\textsc{pers}\textsubscript{2SG}  \textsc{cop}-\textsc{np}\textsubscript{16}-\textsc{dem}.\textsc{iii}\textsubscript{16}  \textsc{sm}\textsubscript{1}-come-\textsc{fv}    \textsc{inf}-sit-\textsc{fv}
\textsc{com}=\textsc{sm}\textsubscript{2SG}-be  \textsc{np}\textsubscript{1}-person  \textsc{pp}\textsubscript{1}-\textsc{con}=\textsc{inf}-be\_black-\textsc{fv}\\
\glt ‘Even in the time of apartheid, many were beaten because of \textbf{not} \textbf{knowing} how to read. On a bench, it is written, whites only. You, that is where you sit, when you are a black person.’ (NF\_Song17)
\z

\ea
\label{bkm:Ref99104701}
\glll kùshábònà {\textasciitilde} kùsábònà\\
ku-shá-bon-a\\
\textsc{inf}-\textsc{neg}.\textsc{inf}-see-\textsc{fv}\\
\glt ‘to not see’
\z
\section{Negation with auxiliaries}
\label{bkm:Ref494466580}\hypertarget{Toc75352710}{}
All other verbal constructions are negated with the use of an auxiliary \textit{ri} ‘be’ or \textit{aazyá} ‘be not’, or a lexical verb \textit{síy} ‘stop, leave’. Negation with \textit{ri} ‘be’ involves the negative prefix \textit{ka}-/\textit{ta-} marked on the auxiliary, followed by the inflected lexical verb, which takes a high tone on the subject marker, showing that it is a relative verb (see \sectref{bkm:Ref490828195} on the formal properties of relative clause verbs). This construction is used to negate the remote past perfective, as in (\ref{bkm:Ref99104743}), the remote past imperfective, as in (\ref{bkm:Ref99104763}), and the near past imperfective, as in (\ref{bkm:Ref489284413}).

\newpage
\ea
\label{bkm:Ref99104743}
kàrì ndáyìbònà\\
\gll ka-ri    ndi-á̲-i-bon-a\\
\textsc{neg}-be  \textsc{sm}\textsubscript{1SG}-\textsc{pst}-\textsc{om}\textsubscript{9}-see-\textsc{fv}\\
\glt ‘I did not see it.’ (NF\_Elic15)
\z

\ea
\label{bkm:Ref99104763}
kàrì kátòmbwèr’ éꜝsózù\\
\gll ka-ri    ka-á̲-tombwer-á̲    e-∅-sozú\\
\textsc{neg}-be  \textsc{pst}.\textsc{ipfv}-\textsc{sm}\textsubscript{1}-weed-\textsc{fv}  \textsc{aug}-\textsc{np}\textsubscript{5}-grass\\
\glt ‘He was not weeding grass.’ (NF\_Narr15)
\z

\ea
\label{bkm:Ref489284413}
kàrì ndákùhîkà\\
\gll ka-ri    ndí̲-aku-hík-a\\
\textsc{neg}-be  \textsc{sm}\textsubscript{1SG}-\textsc{npst}.\textsc{ipfv}-cook-\textsc{fv}\\
\glt ‘I was not cooking.’ (NF\_Elic17)
\z

The auxiliary \textit{ri} ‘be’ with a negative prefix is also used to negate nominal predicates. Affirmative nominal predicates are marked by a copulative prefix only (see \sectref{bkm:Ref489963307}). When negated with the auxiliary \textit{ri}, the copulative prefix is maintained, as in (\ref{bkm:Ref99104807}--\ref{bkm:Ref99104808}).

\ea
\label{bkm:Ref99104807}
mbùrôtù kònó \textbf{kàrí} \textbf{mbùrótù} nênjà\\
\gll N-bu-rótu    konó  ka-rí    N-bu-rótu    nénja\\
\textsc{cop}-\textsc{np}\textsubscript{14}-good  but  \textsc{neg}-be  \textsc{cop}-\textsc{np}\textsubscript{14}-good  well\\
\glt ‘It is good, but it is not very good.’ (ZF\_Conv13)
\z

\ea
\label{bkm:Ref99104808}
òwú \textbf{kàrí} \textbf{ꜝ}\textbf{ngómùnzí} ꜝwángù\\
\gll o-ú    ka-rí    ngó-mu-nzí      u-angú\\
\textsc{aug}-\textsc{dem}.\textsc{i}\textsubscript{3}  \textsc{neg}-be  \textsc{cop}.\textsc{def}\textsubscript{3}-\textsc{np}\textsubscript{3}-village  \textsc{pp}\textsubscript{3}-\textsc{poss}\textsubscript{1SG}\\
\glt ‘This is not my village.’ (ZF\_Elic13)
\z

To express a negative future, the auxiliary \textit{ri} ‘be’ is used, marked with the negative prefix \textit{ka-/ta-}, followed by a subjunctive verb. To indicate a more remote future, the subjunctive verb takes a remoteness prefix \textit{na-/ne-}, as used in (\ref{bkm:Ref471225116}--\ref{bkm:Ref471225117}). To express a near future, the remoteness prefix is omitted, as in (\ref{bkm:Ref471225384}--\ref{bkm:Ref489284739}).

\ea
\label{bkm:Ref471225116}
rímwì zyûbà kàrì nèmúbûːꜝké nwè\\
\gll rí-mwi  ∅-zyúba ka-ri    ne-mú̲-bú̲ːk-e      enwé \\
\textsc{pp}\textsubscript{5}-other  \textsc{np}\textsubscript{5}-day
\textsc{neg}-be  \textsc{rem}-\textsc{sm}\textsubscript{2PL}-wake.\textsc{intr}-\textsc{pfv}.\textsc{sbjv}  \textsc{pers}\textsubscript{2PL}\\
\glt ‘One day, you are not going to wake up.’ (NF\_Narr15)
\z

\ea
\label{bkm:Ref471225117}
kàrì nándìsépè\\
\gll ka-ri    na-á̲-ndi-sep-é̲\\
\textsc{neg}-be  \textsc{rem}-\textsc{sm}\textsubscript{1}-\textsc{om}\textsubscript{1SG}-trust-\textsc{pfv}.\textsc{sbjv}\\
\glt ‘He will not trust me.’ (ZF\_Conv13)
\z

\ea
\label{bkm:Ref471225384}
kàrì ndífìyérè\\
\gll ka-ri    ndí̲-fi\textsubscript{H}yer-é̲\\
\textsc{neg}-be  \textsc{sm}\textsubscript{1SG}-sweep-\textsc{pfv}.\textsc{sbjv}\\
\glt ‘I will not sweep.’ (ZF\_Elic13)
\z

\ea
\label{bkm:Ref489284739}
kàrì ndícìpángè shûnù\\
\gll ka-ri    ndí̲-ci\textsubscript{H}-pá̲ng-e    shúnu\\
\textsc{neg}-be  \textsc{sm}\textsubscript{1SG}-\textsc{om}\textsubscript{7}-do-\textsc{pfv}.\textsc{sbjv}  today\\
\glt ‘I will not do it today.’ (NF\_Elic17)
\z

\begin{sloppypar}
The auxiliary \textit{aazyá} ‘be/have not’ is also used to negate the verb \textit{iná} ‘be at/have’, as in (\ref{bkm:Ref489285426}--\ref{bkm:Ref99104890}).
\end{sloppypar}

\ea
\label{bkm:Ref489285426}
kwìn’ écò ndíbwènè\\
\gll ku-iná    e-co    ndí̲-bwe\textsubscript{H}ne\\
\textsc{sm}\textsubscript{17}-be\_at  \textsc{aug}-\textsc{dem}.\textsc{iii}\textsubscript{7}  \textsc{sm}\textsubscript{1SG}.\textsc{rel}-see.\textsc{stat}\\
\glt ‘There is something that I see. / I see something.’
\z

\ea
\label{bkm:Ref99104890}
kùààzy’ écò ndíbwènè\\
\gll ku-aazyá  e-co    ndí̲-bwe\textsubscript{H}ne\\
\textsc{sm}\textsubscript{17}-be\_not  \textsc{aug}-\textsc{dem}.\textsc{iii}\textsubscript{7}  \textsc{sm}\textsubscript{1SG}.\textsc{rel}-see.\textsc{stat}\\
\glt ‘There is not something that I see. / I see nothing.’ (NF\_Elic15)
\z

Where the auxiliary \textit{iná} with a locative subject marker is used to express ‘something’, ‘someone’, or ‘somewhere’, its negated counterpart \textit{aazyá} is used to express ‘nothing’, ‘no one’, or ‘nowhere’. Subject markers of all three locative classes can be used with the verb \textit{aazyá}, e.g. class 16, as in (\ref{bkm:Ref494561456}), class 17, as in (\ref{bkm:Ref494561459}--\ref{bkm:Ref471307062}), and class 18, as in (\ref{bkm:Ref494561466}).

\ea
\label{bkm:Ref494561456}
ákèːzyà kùwànà ìyé hààzyá bàntù\\
\gll á̲-keːzy-a    ku-wan-a  iyé  ha-aazyá  ba-ntu\\
\textsc{sm}\textsubscript{1}.\textsc{rel}-come-\textsc{fv}  \textsc{inf}-find-\textsc{fv}  that  \textsc{sm}\textsubscript{16}-be\_not  \textsc{np}\textsubscript{2}-person\\
\glt ‘When he came to find that there were no people there…’ (NF\_Narr15)
\z

\ea
\label{bkm:Ref494561459}
kwààzyá mùntù\\
\gll ku-aazyá  mu-ntu\\
\textsc{sm}\textsubscript{17}-be\_not  \textsc{np}\textsubscript{1}-person\\
\glt ‘There is no one.’ (ZF\_Elic13)
\z

\ea
\label{bkm:Ref471307062}
kwàázyó kò nìbáwânè ménò\\
\gll ku-aazyá  o-kó      ni-bá̲-wá̲n-e      ma-inó\\
\textsc{sm}\textsubscript{17}-be\_not  \textsc{aug}-\textsc{dem}.\textsc{iii}\textsubscript{17}  \textsc{rem}-\textsc{sm}\textsubscript{2}-find-\textsc{pfv}.\textsc{sbjv}  \textsc{np}\textsubscript{6}-tooth\\
\glt ‘There’s nowhere where he can get the teeth.’ (NF\_Narr15)
\z

\ea
\label{bkm:Ref494561466}
òbú bùsùnsò mwáázyé zwàyì\\
\gll o-bú    bu-sunso  mu-aazyá  e-zwai\\
\textsc{aug}-\textsc{dem}.\textsc{i}\textsubscript{14}  \textsc{np}\textsubscript{14}-relish  \textsc{sm}\textsubscript{18}-be\_not  \textsc{aug}-salt\\
\glt ‘This relish, there is no salt in it.’ (ZF\_Elic14)
\z

The auxiliary \textit{aazyá} can also be used to negate a fronted infinitive construction. The fronted infinitive construction, which consists of an inflected verb preceded by an infinitive copy of the same verb stem (see \sectref{bkm:Ref431917326}), is illustrated in (\ref{bkm:Ref99105077}). It cannot be negated through the prefix \textit{ta}-/\textit{ka}- and the suffix -\textit{i}, as shown by the ungrammaticality of (\ref{bkm:Ref99105098}). Instead a construction is used with the negative \textit{aazyá} inflected for subject agreement, followed by the lexical verb in the infinitive, as in (\ref{bkm:Ref99105118}).

\ea
\label{bkm:Ref99105077}
kùhòndà ndíꜝhóndà\\
\gll ku-hond-a  ndí̲-hó̲nd-a\\
\textsc{inf}-cook-\textsc{fv}  \textsc{sm}\textsubscript{1SG}.\textsc{rel}-cook-\textsc{fv}\\
\glt ‘I am cooking.’
\z

\ea
\label{bkm:Ref99105098}
*kùhòndà tàndíꜝhóndì (ZF\_Elic14)
\z

\ea
\label{bkm:Ref99105118}
ndààzyá kùhòndà\\
\gll ndi-aazyá  ku-hond-a\\
\textsc{sm}\textsubscript{1SG}-be\_not  \textsc{inf}-cook-\textsc{fv}\\
\glt ‘I am not cooking.’
\z

\textit{aazyá} is also occasionally used to negate verbs that may also be negated with a prefix \textit{ka-/ta-} or an auxiliary \textit{ri} ‘be’. This is the case for verbs with a reduplicated stem, as in (\ref{bkm:Ref99105150}), which may be negated with a prefix \textit{ka-/ta-} and a suffix \textit{-i} in the present tense, as in (\ref{bkm:Ref99105160}), but most speakers prefer to use the auxiliary \textit{aazyá} followed by the reduplicated verb in the infinitive form, as in (\ref{bkm:Ref99105170}).

\ea
\label{bkm:Ref99105150}
\glll ndìtóːrátôːrà\\
ndi-toːra-tó̲ːr-a\\
\textsc{sm}\textsubscript{1SG}-\textsc{pl}2-pick-\textsc{fv}\\
\glt ‘I pick.’
\z

\ea
\label{bkm:Ref99105160}
\glll kàndìtóːrìtòːrì\\
ka-ndi-tó̲ːri-toːr-i\\
\textsc{neg}-\textsc{sm}\textsubscript{1SG}-\textsc{pl}2-pick-\textsc{neg}\\
\glt ‘I don’t pick.’
\z

\ea
\label{bkm:Ref99105170}
ndààzy’ ókùtóːràtòːrà\\
\gll ndi-aazyá  o-ku-tóːra-toːr-a\\
\textsc{sm}\textsubscript{1SG\-}-be\_not  \textsc{aug}-\textsc{inf}-\textsc{pl}2-pick-\textsc{fv}\\
\glt ‘I don’t pick.’ (NF\_Elic15)
\z

\textit{aazyá} is also used to negate verbs expressing states, either verbs in the stative construction, as in (\ref{bkm:Ref99105195}--\ref{bkm:Ref99105196}), or true stative verbs, as in (\ref{bkm:Ref99105197}). As shown in \sectref{bkm:Ref490843739}, stative verbs can also be negated with affixes on the verb. A meaning difference between periphrastic and morphological negation of stative verbs has not been observed.

\ea
\label{bkm:Ref99105195}
ècìyângò cààzyá kùbórêtè\\
\gll e-ci-ángo    ci-aazyá  ku-bor-é̲te\\
\textsc{aug}-\textsc{np}\textsubscript{7}-fruit  \textsc{sm}\textsubscript{7}-be\_not  \textsc{inf}-rot-\textsc{stat}\\
\glt ‘The fruit is not rotten.’ (ZF\_Elic14)
\z

\ea
\label{bkm:Ref99105196}
cààzy’ ókùhárîtwà\\
\gll ci-aazyá  o-ku-ar-í̲t-w-a\\
\textsc{sm}\textsubscript{7}-be\_not  \textsc{aug}-\textsc{inf}-close-\textsc{stat}-\textsc{pass}-\textsc{fv}\\
\glt ‘It is not closed.’ (NF\_Elic15)
\z

\ea
\label{bkm:Ref99105197}
ndàázyá kùshàkà kùrìhà òmùrándù\\
\gll ndi-aazyá  ku-shak-a  ku-rih-a  o-mu-randú\\
\textsc{sm}\textsubscript{1SG}-be\_not  \textsc{inf}-want-\textsc{fv}  \textsc{inf}-pay-\textsc{fv}  \textsc{aug}-\textsc{np}\textsubscript{3}-fine\\
\glt ‘I don’t want to pay a fine.’ (NF\_Elic15)
\z

The lexical verb \textit{síy} ‘leave, let go, stop’, is used in the imperative form and followed by an infinitive to express a prohibitive, as in (\ref{bkm:Ref471982528}--\ref{bkm:Ref492060835}).

\ea
\label{bkm:Ref471982528}
sìy’ ókùndìkwâtà\\
\gll si\textsubscript{H}-é̲      o-ku-ndi-kwát-a\\
stop-\textsc{pfv}.\textsc{sbjv}  \textsc{aug}-\textsc{inf}-\textsc{om}\textsubscript{1SG}-grab-\textsc{fv}\\
\glt ‘Don’t touch me.’ (NF\_Elic15)
\z

\ea
\label{bkm:Ref492060835}
òsìyé kùyángà kwìnà\\
\gll o-si\textsubscript{H}-é̲      ku-yá-ang-a    kwina\\
\textsc{sm}\textsubscript{2SG}-leave-\textsc{pfv}.\textsc{sbjv}  \textsc{inf}-go-\textsc{hab}-\textsc{fv}  \textsc{dem}.\textsc{iv}\textsubscript{17}\\
\glt ‘Never go there.’ (NF\_Elic17)
\z

