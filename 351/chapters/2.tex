\chapter{Segmental phonology}
\label{bkm:Ref451511011}\hypertarget{Toc75352600}{}\section{Introduction}
\hypertarget{Toc75352601}{}
This chapter discusses the segmental phonology of Fwe. Tone is discussed in Chapter \ref{bkm:Ref451507583}, which also explains transcription conventions of tones used throughout this book. Earlier sketches of the phonology of Fwe can be found in {\citet{Baumbach1997}} and {\citet{Seidel2005}}, who describe the Namibian variety of Fwe, and {\citet{Bostoen2009}}, who describes the Zambian variety. The analysis presented here largely confirms their findings, but also adds many details on previously undescribed patterns.

\section{Consonants}
\label{bkm:Ref435187046}\hypertarget{Toc75352602}{}\label{bkm:Ref70695065}
\tabref{tab:2:1} gives an overview of the contrastive consonants of Fwe, in the practical or\-thography that is used in this book. Wherever this deviates from the conventions of the International Phonetic Alphabet, the corresponding IPA symbol is given in brackets. The practical orthography is partly based on widespread Africanist or Bantuist conventions, such as the use of <y> for the palatal glide [j], and partly on orthographical conventions that are commonly used in Zambia, such as <bb> for the voiced bilabial stop [b].

\begin{table}[t]
\label{bkm:Ref485997713}\caption{\label{tab:2:1}Contrastive consonants}
\small
\begin{tabularx}{\textwidth}{p{1.7cm}C@{}C@{}C@{}C@{}C@{}C@{}C@{}C@{}C@{}C@{}l@{}}
\lsptoprule
& \multicolumn{2}{c}{bilabial} & \multicolumn{2}{c}{dental} & \multicolumn{2}{c}{alveolar} & \multicolumn{2}{c}{palatal} & \multicolumn{2}{c}{velar} & glottal\\
\midrule
stop & p & bb\newline  [b] & & & t & d & & & k & g & \\
\tablevspace
affricate & & & & & & &  {c\newline [tʃ]} & j\newline [dʒ] & & & \\
\tablevspace
fricative &  & b\newline [ß] & f & v & s & z & sh\newline [ʃ] & {zy\newline [ʒ]} & &  & h\\
\tablevspace
nasal &  & m & & &  & n &  & {ny\newline [ɲ]} &  & ŋ & \\
\tablevspace
tap & & & & &  & r\newline [ɾ] & & & & & \\
\tablevspace
glide & & & & & & &  & {y\newline [j]} &  & w & \\
\tablevspace
click & & & ǀ & ᵍǀ & & & & & & & \\
\tablevspace
prenasalized stop & mp\newline [ᵐp] & mb\newline [ᵐb] & & & nt\newline [ⁿt] & nd\newline [ⁿd] & & & nk\newline [ᵑk] & ng\newline [ᵑg] & \\
\tablevspace
prenasalized fricative & & & mf [{\ᶬ}f] & mv\newline [{\ᶬ}v] & ns\newline [ⁿs] & nz\newline [ⁿz] & {nsh\newline [{\ᶮ}ʃ]} & & & \\
\tablevspace
prenasalized affricate & & & & & & & nc\newline [{\ᶮ}tʃ] & {nj\newline [{\ᶮ}dʒ]} & & & \\
\tablevspace
prenasalized click & & & ⁿǀ̥ & ⁿǀ & & & & & & & \\
\lspbottomrule
\end{tabularx}
\end{table}
\subsection{Stops}

Of the six simple (non-prenasalized) stops in Fwe, only the voiceless alveolar stop /t/ and the voice\-less velar stop /k/ are frequently attested. /t/ and /k/ are contrastive phonemes, as illustrated by the minimal pair in (\ref{bkm:Ref98513252}).

\NumTabs{5}
\ea
\label{bkm:Ref98513252}
kùtôːrà \tab   - \tab kùkôːrà\\
ku-tóːr-a \tab \tab     ku-kóːr-a\\
\textsc{inf}-pick\_up-\textsc{fv}  \tab \textsc{inf}-cough-\textsc{fv}\\
\glt ‘to pick up’ \tab \tab     ‘to cough’
\z

The voiceless bilabial stop /p/ as well as the voiced stops /b/ (written <bb>), /d/ and /g/ are less frequent. The (near-)minimal pairs in (\ref{bkm:Ref98513263}--\ref{bkm:Ref98513264}) show that they are contrastive phonemes.

\ea
\label{bkm:Ref98513263}
kùpàrà \tab    - \tab  kùgàrà\\
ku-par-a \tab \tab    ku-gar-a\\
\textsc{inf}-fail-\textsc{fv} \tab \tab    \textsc{inf}-search-\textsc{fv}\\
\glt ‘to fail, refuse’ \tab \tab   ‘to search/dig around’
\z

\newpage
\ea
kùdùnkà  \tab - \tab  kùgùnkà\\
ku-dunk-a \tab \tab   ku-gunk-a\\
\textsc{inf}-swim-\textsc{fv} \tab \tab   \textsc{inf}-bump-\textsc{fv}\\
\glt ‘to swim’ \tab \tab    ‘to bump into; lean against’
\z

\ea
\label{bkm:Ref98513264}
cìbbákù  \tab -  \tab cìbàkà\\
ci-bbakú \tab \tab   ci-baka\\
\textsc{np}\textsubscript{7}-snake \tab \tab   \textsc{np}\textsubscript{7}-place\\
\glt ‘snake sp.’ \tab \tab   ‘place’
\z

/p, bb, d, g/ are relatively infrequent in the lexicon: out of a 2200 word database, /bb/, /d/, and /g/ each occur in about 20 lexemes, and /p/ in about 80 lexemes. The plosives /p/, /bb/, /d/ and /g/ are not reflexes of *p, *b, *d and *g as recon\-structed for Proto-Bantu \citep{Bostoen2009}, but mainly appear in loanwords\footnote{Some words with /g/ may be borrowings from Shanjo, because unlike Fwe, Shanjo has maintained proto-Bantu *g. However, the available documentation on Shanjo is too limited to trace Fwe borrowings to this language. Some Fwe speakers consider the Fwe verb \textit{gùnkàmà} ‘kneel’ to be of Shanjo origin.}, as in (\ref{bkm:Ref71645400}--\ref{bkm:Ref97911219}), or sound-symbolic words and ideophones, as in (\ref{bkm:Ref71645446}--\ref{bkm:Ref71645447}).

\ea
\label{bkm:Ref71645400}
  cìpúrà ‘chair’ < Lozi sipula ‘chair’ \citep[27]{Burger1960}
\z

\ea
  kùpàpàùrà ‘divide a dead animal into pieces’ < Mbukushu papaghura ‘dismember (animal after skinning)’ \citep[175]{Wynne1980}
\z

\ea
  kúpàkà ‘carry (a child) on one’s back’ < Yeyi paka ‘carry in a cradle on the back as a baby’ \citep[140-141]{Lukusa2009}
\z

\ea
  kàpíkírì ‘nail’ < Afrikaans spyker ‘nail’
\z

\ea
  kàpêrù ‘pail’ < English pail
\z

\ea
  kùdàbbàmà ‘jump into water’ < Mbukushu dabwama ‘throw oneself, jump into water, dive’ \citep[393]{Wynne1980}
\z

\ea
  kùdùrà ‘be expensive’ < Afrikaans duur ‘expensive’
\z

\ea
\label{bkm:Ref97911219}
 màgrázì ‘glasses’ < English glasses
\z

\ea
\label{bkm:Ref71645446}
 bbùndù bbúndù\\
\glt ‘ideophone expressing sudden appearance’
\z

\ea
cìsùbírà cò bbûkù\\
\gll ci-subir-á̲    co    bbúku\\
\textsc{sm}\textsubscript{7}-be\_red-\textsc{fv}  \textsc{dem}.\textsc{iii}\textsubscript{7}  very\_red\\
\glt ‘It is very red.’ (NF\_Elic17)
\z

\ea
\glll kùbbôzà\\
ku-bbóz-a\\
\textsc{inf}-bark-\textsc{fv}\\
\glt ‘to bark’
\z

\ea
\label{bkm:Ref71645447}
\glll kùdòkòrà\\
ku-dokor-a\\
\textsc{inf}-belch-\textsc{fv}\\
\glt ‘to belch; to clear one’s throat’
\z

In Namibian Fwe, /p, bb, d, g/ also occur when prenasalized consonants lose the homorganic nasal as the result of a change in noun class; \textit{ò-ndávù} ‘lion’, \textit{kà-dávù} ‘small lion’. This is explained in detail in \sectref{bkm:Ref489005545} on nominal prefixes. There are also a number of lexemes, listed in (\ref{bkm:Ref97911315}--\ref{bkm:Ref97911319}), where /g/ appears as an apparently unconditioned allophone of /k/. This variation is limited to Namibian Fwe, Zambian Fwe only uses the variant with /g/.

\ea
\label{bkm:Ref97911315}
cìkùrùbè (NF) {\textasciitilde} cìgùrùbè (ZF/NF)\\
\gll ci-kurube\\
\textsc{np}\textsubscript{7}-pig\\
\glt ‘pig’
\z

\ea
cìkébéngà (NF) {\textasciitilde} cìgébéngà (ZF/NF)\\
\gll ci-kebengá\\
\textsc{np}\textsubscript{7}-criminal\\
\glt ‘criminal’
\z

\ea
\label{bkm:Ref97911319}
mùkwàkwà (NF) {\textasciitilde} mùgwàgwà (ZF/NF)\\
\gll mu-kwakwa\\
\textsc{np}\textsubscript{3}-road\\
\glt ‘road’
\z

The voiced velar plosive /g/ also appears as an unconditioned allophone of the voiced oral click /ᵍǀ/, as in (\ref{bkm:Ref98923623}).

\ea
\label{bkm:Ref98923623}
\glll mùᵍǀênè {\textasciitilde} mù-gênè\\
mu-ᵍǀéne\\
\textsc{np}\textsubscript{1}-thin\\
\glt ‘thin person’
\z

/g/ is also found in words that do not have an alternative pronunciation with a click, but whose etymology suggests that they originally contained a click, as in (\ref{bkm:Ref98923638}).

\ea
\label{bkm:Ref98923638}
\glll mùgwégwèsì\\
mu-gwégwesi\\
\textsc{np}\textsubscript{3}-ankle\_bone\\
\glt ‘ankle bone’ (from Neitsas/Nurugas !Xung gwé: ‘ankle’ \citep{Doke1925}, or Juǀ’hoan ǂˈhòèǂˈhòrè ‘enkelknop [ankle bone]’) \citep[107]{Snyman1975}\footnote{I am indebted to Bonny Sands for suggesting these possible etymologies.}
\z

One word with /ᵍǀ/ has an alternative pronunciation with either /g/ or /d/, as in (\ref{bkm:Ref98923659}); possibly, other words with /d/ used to have an alternative pronunciation with /ᵍǀ/ as well.

\ea
\label{bkm:Ref98923659}
ᵍǀúkùmù {\textasciitilde} gúkùmù {\textasciitilde} dúkùmù\\
\gll ∅-ᵍǀúkumu\\
\textsc{np}\textsubscript{5}-fruit\\
\glt ‘fruit sp.’
\z
\subsection{Affricates}

Fwe has two postalveolar affricates, voiceless /tʃ/, written as <c>, and voiced /dʒ/, written as <j>. Minimal pairs contrasting /c/ with /ʃ/ (written as <sh>), and /k/ are given in (\ref{bkm:Ref75331680}--\ref{bkm:Ref75331681}), and (near-)minimal pairs contrasting /j/ with /ʒ/ (written as <zy>) and /g/ are given in (\ref{bkm:Ref75331694}--\ref{bkm:Ref75331696}).

\NumTabs{4}
\ea
\label{bkm:Ref75331680}
kùcírìrà                        \tab  - \tab kùshírìrà\\
ku-círir-a                        \tab    \tab   ku-shírir-a\\
\textsc{inf}-follow-\textsc{fv}      \tab \tab  \textsc{inf}-desire-\textsc{fv}\\
‘to follow’                       \tab \tab       ‘to desire’
\z

\ea
\label{bkm:Ref75331681}
kùcâːnà    \tab  - \tab  kùkâːnà\\
ku-cáːn-a     \tab    \tab   ku-káːn-a\\
\textsc{inf}-hunt-\textsc{fv}    \tab    \tab    \textsc{inf}-reject-\textsc{fv}\\
‘to hunt’   \tab    \tab     ‘to refuse, reject, divorce’
\z

\ea
\label{bkm:Ref75331694}
kùjánàmà  \tab  - \tab  kùzyánàmà\\
ku-jánam-a \tab \tab     ku-zyánam-a\\
\textsc{inf}-gape-\textsc{fv} \tab \tab     \textsc{inf}-hang-\textsc{fv}\\
\glt ‘to gape’ \tab \tab     ‘to hang to dry
\z

\ea
\label{bkm:Ref75331696}
kùjùmbà \tab    - \tab kùgùmbàmà\\
ku-jumb-a \tab\tab     ku-gumbam-a\\
\textsc{inf}-leave-\textsc{fv} \tab\tab     \textsc{inf}-stand\_next\_to-\textsc{fv}\\
\glt ‘to leave in protest’  \tab\tab  ‘to stand next to each other’
\z

The near-minimal pairs in (\ref{bkm:Ref71882646}--\ref{bkm:Ref75331768}) show the contrast between /c/ and /j/. However, /j/ also occurs as a free variant of /c/, as in (\ref{bkm:Ref75331777}--\ref{bkm:Ref71883077}). Like the voiced stops, the voiced affricate /j/ is less frequently attested than its voiceless counterpart /c/.

\ea
\label{bkm:Ref71882646}
kùcérùkà \tab    - \tab kùjérùmùkà\\
ku-céruk-a \tab\tab     ku-jérumuk-a\\
\textsc{inf}-tear-\textsc{fv} \tab\tab     \textsc{inf}-be\_sour-\textsc{fv}\\
\glt ‘to be torn’  \tab\tab   ‘to be sour’
\z

\ea
\label{bkm:Ref75331768}
kùcùkùnsà \tab   - \tab kùjùkùtà\\
ku-cukuns-a  \tab\tab    ku-jukut-a\\
\textsc{inf}-shake-\textsc{fv}   \tab\tab   \textsc{inf}-rinse-\textsc{fv}\\
\glt ‘to shake’ \tab\tab    ‘to rinse out clothes’
\z

\ea
\label{bkm:Ref75331777}
kùjânà {\textasciitilde} kùcânà\\
ku-ján-a\\
\textsc{inf}-gape-\textsc{fv}\\
\glt ‘to gape’
\z

\ea
\label{bkm:Ref71883077}
bù-cwàrà {\textasciitilde} bù-jwàrà\\
bu-cwara\\
\textsc{np}\textsubscript{14}-beer\\
\glt ‘beer’
\z
\subsection{Fricatives}

As shown in \tabref{tab:2:1}, Fwe has eight fricative phonemes: /β/, written as <b>, /f/, /v/, /s/, /z/, /ʃ/, written as <sh>, /ʒ/, written as <zy>, and /h/. The labiodental, alveolar and post-alveo\-lar fricatives occur as both voiceless and voiced; (near-)minimal pairs are given in (\ref{bkm:Ref448220681}) and (\ref{bkm:Ref75331889}).

\ea
\label{bkm:Ref448220681}
kùvùrùrà    \tab  - \tab  kùfùrà\\
ku-vur-ur-a   \tab\tab     ku-fur-a\\
\textsc{inf}-winnow-\textsc{sep}.\textsc{tr}-\textsc{fv}   \tab \textsc{inf}-pick-\textsc{fv}\\
‘to winnow’   \tab \tab    ‘to pick (fruit)’\label{bkm:Ref448220682}
\z

\ea
\label{bkm:Ref75331889}
\ea
\glll kùfûmà\\
ku-fúm-a\\
\textsc{inf}-become\_rich-\textsc{fv}\\
\glt ‘to become rich’

\ex
\glll kùsûmà\\
ku-súm-a\\
\textsc{inf}-sew-\textsc{fv}\\
\glt ‘to sew’

\ex
\glll kùshûmà\\
ku-shúm-a\\
\textsc{inf}-bite-\textsc{fv}\\
\glt ‘to bite’  

\ex
\glll kùzûmà\\
ku-zúm-a\\
\textsc{inf}-hum-\textsc{fv}\\
\glt ‘to hum’

\ex
\glll kùzyûmà\\
ku-zyúm-a\\
\textsc{inf}-dry-\textsc{fv}\\
\glt ‘to dry’
\z
\z

The bilabial fricative /b/ has no voiceless counterpart. Its phonemic status is shown by the (near-)minimal pairs in (\ref{bkm:Ref98513314}) and (\ref{bkm:Ref98513316}).

\ea
\label{bkm:Ref98513314}
kùbûrà \tab    - \tab  kùfûrà\\
ku-búr-a \tab \tab     ku-fúr-a\\
\textsc{inf}-miss-\textsc{fv}  \tab\tab    \textsc{inf}-sharpen-\textsc{fv}\\
‘to miss’ \tab\tab     ‘to sharpen’
\z

\ea
\label{bkm:Ref98513316}
cìràbò \tab     - \tab ràmbò\\
ci-rabo   \tab\tab   ∅-rambo\\
\textsc{np}\textsubscript{7}-paddle  \tab\tab    \textsc{np}\textsubscript{5}-pit\\
‘paddle’ \tab\tab      ‘pit’
\z

Many speakers realize /v/ as a bilabial fricative /b/, as in (\ref{bkm:Ref98923763}--\ref{bkm:Ref98923767}). Comparative data and reconstructions suggest that /v/ is the older realization: /v/ in Fwe is the result of spirantization of *b or *g before a high back vowel (\citealt{Bostoen2009}: 118, see also \sectref{bkm:Ref451940391}). The change of /v/ to /b/ could be the result of the higher fre\-quency of the latter; whereas /v/ only occurs before /u/, /b/ occurs in all environments, and is therefore much more common.

\ea
\label{bkm:Ref98923763}
kùvwângà {\textasciitilde} kùbwângà\\
\gll ku-vwáng-a\\
\textsc{inf}-wrap-\textsc{fv}\\
\glt ‘to wrap’\\
cf. *búang ‘mix’ \citep{BastinEtAl2002}\\
\z

\ea
\label{bkm:Ref98923767}
cìvwângà {\textasciitilde} cìbwângà\\
\gll ci-vwánga\\
\textsc{np}\textsubscript{7}-frog\\
\glt ‘frog’
\z

The bilabial fricative tends to be more open than a canonical fricative, and is pronounced with a minimal amount of friction, in between a fricative and an approximant. Previous descrip\-tions of the phonology of Fwe also differ in describing this phoneme as an approximant \citep[228]{Seidel2005} or a fricative (\citealt{Baumbach1997}: 398; \citealt{Bostoen2009}: 113).

/s/ and /sh/ are contrastive in lexical roots, as seen in the minimal pairs in (\ref{bkm:Ref448220682}), as well as the minimal pair in (\ref{bkm:Ref486268530}).

\ea
\label{bkm:Ref486268530}
kùsèkà \tab    - \tab  kùshèkà\\
ku-sek-a\tab\tab      ku-shek-a\\
\textsc{inf}-insert-\textsc{fv} \tab\tab   \textsc{inf}-laugh-\textsc{fv}\\
‘to insert’   \tab\tab   ‘to laugh’
\z

In grammatical prefixes in Namibian Fwe, /s/ and /sh/ are allo\-phones in free variation, as illustrated in (\ref{bkm:Ref98496273}) with the inceptive \textit{she-}, which can be realized as \textit{se-} or \textit{she-}.

\ea
\label{bkm:Ref98496273}
shèndìrère {\textasciitilde} sèndìrèrè\\
\gll she-ndi-re\textsubscript{H}re\\
\textsc{inc}-\textsc{sm}\textsubscript{1SG}-sleep.\textsc{stat}\\
\glt ‘I am now sleeping.’ (NF\_Elic17)
\z

The alternation between /s/ and /sh/ affects all grammatical prefixes in which the phoneme oc\-curs. The only grammatical suffix with /s/ is the causative \textit{\nobreakdash-is/\nobreakdash-es}, which is invariably realized with /s/, never with /sh/. As this suffix is derivational, it may be conceptualized as part of the lexical verb, and as such not be subject to [s {\textasciitilde} sh] variation, as this does not occur in lexemes\footnote{I am indebted to an anonymous reviewer for this analysis.}. A complete list of grammatical prefixes in which [s] and [sh] alter\-nate is given in (\ref{bkm:Ref486268783}).

\ea
\label{bkm:Ref486268783}
ásha- {\textasciitilde} ása-     negative imperative\\
sha- {\textasciitilde} sa-    negative subjunctive\\
shá- {\textasciitilde} sá-    negative infinitive\\
shí- {\textasciitilde} sí-    persistive\\
shi- {\textasciitilde} si-    inceptive\\
shi- {\textasciitilde} si-     conditional\\
shí- {\textasciitilde} sí-    associative\\
shaké {\textasciitilde} saké    conditional
\z

In Zambian Fwe, only the realization with [sh] is used. In Namibian Fwe, the alternation between [s] and [sh] mostly concerns inter-speaker variation, with each speaker consistently using his or her preferred pronunciation. A possible explanation for this variation and its geographic distribution is contact between Fwe and the closely-related languages Subiya and Totela; Fwe /sh/ corresponds to Subiya and Totela /s/ \citep[116]{Bostoen2009}, and given the high mutual intelligibility between Fwe, Subiya and Totela, and wide-spread multilingualism, this may have led Fwe speakers in Namibia to interchange /sh/ with /s/. This may also explain why this free variation is not seen in Zambian Fwe, as this variety of Fwe is not in active contact with Totela and Subiya. It fails to explain, however, why [s {\textasciitilde} sh] variation in Fwe only targets grammatical prefixes, and not lexical stems.

The phonemic status of the glottal fricative /h/ is shown by the minimal pair in (\ref{bkm:Ref486256581}), which shows the contrast between /h/ and /t/, and in (\ref{bkm:Ref459737663}), which shows the contrast between /h/ and zero.

\ea
\label{bkm:Ref486256581}
mùhàrà \tab   - \tab  mùtàrà\\
mu-hara   \tab\tab   mu-tara\\
\textsc{np}\textsubscript{3}-rope   \tab\tab   \textsc{np}\textsubscript{3}-footprint\\
‘rope’ \tab\tab       ‘footprint’
\z

\newpage
\ea
\label{bkm:Ref459737663}
kùhùrà \tab    - \tab  kùùrà\\
ku-hur-a  \tab\tab    ku-ur-a\\
\textsc{inf}-arrive-\textsc{fv}   \tab\tab   \textsc{inf}-buy-\textsc{fv}\\
\glt ‘to arrive’ \tab\tab     ‘to buy’
\z

Though there are numerous cases where /h/ contrasts with zero, i.e. where /h/ cannot be omit\-ted, [h] is also often used as an epenthetic consonant, in which case it can be freely interchanged with [w], [y] and zero (see \sectref{bkm:Ref491962181}). Phonemic /h/, on the other hand, cannot be interchanged with a glide nor can it be dropped. Furthermore, phonemic /h/ can be accompanied by slight nasalization of the following vowel. These differences between phonemic /h/ and epenthetic [h] are shown in (\ref{bkm:Ref71645923}--\ref{bkm:Ref71645924}).

\ea
\label{bkm:Ref71645923}
  Phonemic /h/\\
\ea
{}[rùh\'{\~{a}}tì {\textasciitilde} rùhátì  ]\\
*rùwáti {\textasciitilde} rùátì\\
ru-hatí\\
\textsc{np}\textsubscript{11}-rib\\
\glt ‘rib’

\ex
rûhò {\textasciitilde} rûhò̰\\
*rûwò {\textasciitilde} rûò\\
ru-úho\\
\textsc{np}\textsubscript{11}-wind\\
\glt ‘wind’
\z
\z

\ea
\label{bkm:Ref71645924}
  Epenthetic [h]  \\  
    \ea
    kùròhà {\textasciitilde} kùròwà {\textasciitilde} kùròà\\
    *kùròh\`{\~{a}}\\
    ku-ro-a\\
    \textsc{inf}-bewitch-\textsc{fv}\\
    \glt ‘to bewitch’
    
    \ex
    rùsíꜝhízà {\textasciitilde} rùsíꜝyízà {\textasciitilde} rùsíꜝízà\\
    *rùsíꜝh\'{ı}̰zà\\ % this is currently a Turkish ı
    ru-síizá\\
    \textsc{np}\textsubscript{11}-darkness\\
    \glt ‘darkness before rain’
    \z
\z
% \todo{problem with i with acute and tilde}

\subsection{Prenasalization}

Fwe also makes use of contrastive prenasalization on stops, fricatives and affricates. With stops, Fwe distinguishes bilabial, alveolar and velar prenasalized stops. The (near-)minimal pairs in (\ref{bkm:Ref98513390}--\ref{bkm:Ref98513391}) show the phonemic status of prenasalized stops.

\ea
\label{bkm:Ref98513390}
mántà \tab     -\tab  mátà\\
ma-ntá   \tab\tab       ma-tá\\
\textsc{np}\textsubscript{6}-power  \tab\tab    \textsc{np}\textsubscript{6}-bow\\
\glt ‘power’ \tab\tab     ‘bows’
\z

\ea
kùdùnà \tab   - \tab ìndúnà\\
ku-dun-a  \tab\tab    ∅-induná\\
\textsc{inf}-stare-\textsc{fv} \tab\tab     \textsc{np}\textsubscript{1a}-induna\\
\glt ‘to stare’ \tab\tab     ‘induna (political figure)’
\z

\ea
mùnêː \tab     - \tab  mùnkêː\\
mu-néː  \tab\tab    mu-nkéː\\
\textsc{np}\textsubscript{1}-four \tab\tab     \textsc{np}\textsubscript{1}-one\\
\glt ‘four’  \tab\tab      ‘one’
\z

\ea
\label{bkm:Ref98513391}
bûːngìː \tab   - \tab è-gîː\\
búː-ngiː  \tab\tab    e-∅-gíː\\
\textsc{np}\textsubscript{14}-many \tab\tab     \textsc{aug}-\textsc{np}\textsubscript{5}-egg\\
\glt ‘many’ \tab\tab      ‘egg’
\z

Fwe contrasts voiceless and voiced prenasalized stops, as shown by the minimal pairs in (\ref{bkm:Ref98513412}--\ref{bkm:Ref98513413}).

\ea
\label{bkm:Ref98513412}
mpùndù \tab   - \tab mbùndù\\
N-pundu  \tab\tab    N-bundu\\
\textsc{np}\textsubscript{9}-bush  \tab\tab    \textsc{np}\textsubscript{9}-dew\\
\glt ‘sandpaper raisin bush’ \tab ‘dew’
\z

\ea
ndìtântà  \tab  - \tab ndìtândà\\
ndi-ta\textsubscript{H}nt-a   \tab\tab   ndi-ta\textsubscript{H}nd-a\\
\textsc{sm}\textsubscript{1SG}-overtake-\textsc{fv} \tab\tab   \textsc{sm}\-\textsubscript{1SG}-chase-\textsc{fv}\\
\glt ‘I overtake.’  \tab\tab    ‘I chase.’
\z

\ea
\label{bkm:Ref98513413}
kùsìnkà   \tab - \tab kùsìngà\\
ku-sink-a  \tab\tab    ku-sing-a\\
\textsc{inf}-patch-\textsc{fv}  \tab\tab    \textsc{inf}-paint-\textsc{fv}\\
\glt ‘to patch’  \tab\tab    ‘to paint’
\z

Fwe has two prenasalized post-alveolar affricates, voiceless /nc/ and voiced /nj/. The near-minimal pairs in (\ref{bkm:Ref98513430}) and (\ref{bkm:Ref98513433}) show that these two phonemes are contrastive, even though the voiceless and voiced affricate without prenasalization are not.

\ea
\label{bkm:Ref98513430}
bâncè   \tab   - \tab rùbánjè\\
ba-ánce  \tab\tab    ru-banjé\\
\textsc{np}\-\textsubscript{2}-child  \tab\tab    \textsc{np}\textsubscript{11}-cannabis\\
\glt ‘children’  \tab\tab    ‘cannabis’
\z

\ea
\label{bkm:Ref98513433}
ncèrè  \tab    -  \tab njèwè\\
∅-ncere \tab\tab     ∅-njewe\\
\textsc{np}\textsubscript{1a}-snake   \tab\tab   \textsc{np}\-\textsubscript{1a}-poor\\
\glt ‘snake sp.’  \tab\tab    ‘poor person’
\z

It is more difficult to prove that prenasalization is also contrastive in affricates. The sound /j/, the non-prenasalized counterpart of the voiced prenasalized affricate /nj/, does occur, but it has a low frequency and mainly occurs in loanwords. When prenasali\-zation is involved in a morphophonological process, /nj/ commutes with /zy/ (see \sectref{bkm:Ref451507060} on prenasalization as a result of a morphophonological process). The voiceless affricate /nc/ does have a non-prenasalized counterpart /c/ as a regular phoneme. There are, however, no minimal or near-mini\-mal pairs to prove that /c/ and /nc/ are contrastive phonemes, though there is also no clear condi\-tioning for the distribution of /c/ and /nc/, should they be analyzed as allophones.

Fwe also has prenasalized fricatives: labiodental /mf/ and /mv/, alveolar /ns/ and /nz/ and postalveolar /nsh/. Prenasalized fricatives contrast with non-pre\-na\-sal\-ized fricatives, as shown for the alveolar fricatives in (\ref{bkm:Ref459307763}).

\ea
\label{bkm:Ref459307763}
bànsâ  \tab    - \tab  bàsâ\\
ba-nsá    \tab\tab    ba-sá\\
\textsc{np}\textsubscript{2}-duiker  \tab\tab  \textsc{np}\textsubscript{2}-thief\\
\glt ‘duikers’  \tab\tab  ‘thieves’
\z

Prenasalized labiodental fricatives occur, though they are infrequent; only four exam\-ples of /mf/ and five examples of /mv/ are found in the data. Examples of both voiceless and voiced prenasal\-ized labiodental fricatives are given in (\ref{bkm:Ref459307296}).

\ea
\label{bkm:Ref459307296}
\glll mfùmò\\
∅-mfumo\\
\textsc{np}\textsubscript{1a}-rhinoceros\\
\glt ‘rhinoceros’
\z

\ea
\glll mvûrà\\
∅-mvúra\\
\textsc{np}\textsubscript{1a}-rain\\
\glt ‘rain’
\z

The contrast between prenasalized and non-prenasalized fricatives appears to be diminishing: /ns/, /nz/ and /nsh/ are occasionally pronounced without prenasalization, without apparent conditioning, as shown in (\ref{bkm:Ref494184788}--\ref{bkm:Ref494184790}).

\ea
\label{bkm:Ref494184788}
mpásì {\textasciitilde} mpánsì\\
N-pansí\\
\textsc{np}\textsubscript{9}-grasshopper\\
\glt ‘grasshopper’
\z

\ea
kùbîzwà {\textasciitilde} kùbînzwà\\
ku-bínzw-a\\
\textsc{inf}-ripen-\textsc{fv}\\
\glt ‘to ripen’
\z

\ea
\label{bkm:Ref494184790}
rùshònshò {\textasciitilde} rùshòshò\\
ru-shonsho\\
\textsc{np}\textsubscript{11}-tibia\\
\glt ‘tibia’
\z

/sh/ is occasionally realized as prenasalized /nsh/ in words where comparative data and reconstruction suggest that the sound was never prenasalized, as in (\ref{bkm:Ref98923934}--\ref{bkm:Ref98923939}). The prenasalization may be related to the preceding /m/, though as seen in (\ref{bkm:Ref494184788}--\ref{bkm:Ref494184790}), variation between prenasalized and non-prenasalized fricatives also occurs outside this context.

\ea
\label{bkm:Ref98923934}
mùshêmpù {\textasciitilde} mùnshêmpù (< shémpèkà ‘shoulder a load’)\\
mu-shémpu\\
\textsc{np}\textsubscript{3}-load\\
\glt ‘load’
\z

\ea
\label{bkm:Ref98923939}
mùshûː {\textasciitilde} mùnshûː (< shûbà ‘urinate’, *cʊ ‘urine’ \citep{BastinEtAl2002}\\
mu-shúː\\
\textsc{np}\textsubscript{3}-urine\\
\glt ‘urine’
\z
\subsection{Nasals}

Fwe has four contrastive nasal consonants: bilabial /m/, alveolar /n/, palatal /ɲ/ (written as <ny>) and velar /ŋ/. Their phonemic status is shown by the near-minimal pairs in (\ref{bkm:Ref75345975}--\ref{bkm:Ref75345979}).

\ea
\label{bkm:Ref75345975}
ŋàngà  \tab    - \tab  nángà\\
∅-ŋanga   \tab\tab   nangá\\
\textsc{np}\textsubscript{1a}-doctor\\
\glt ‘doctor’  \tab\tab    ‘even, even though’
\z

\ea
ŋórò   \tab   - \tab cìnyôrò\\
∅-ŋoró   \tab\tab   ci-nyóro\\
\textsc{np}\textsubscript{5}-letter   \tab\tab   \textsc{np}\textsubscript{7}-plant\_remains\\
\glt ‘letter’   \tab\tab     ‘plant remains in the field’
\z

\ea
kùnyènsà  \tab  - \tab káꜝnénsà\\
ku-nyens-a   \tab\tab   ká-nensá\\
\textsc{inf}-defeat-\textsc{fv}  \tab\tab  \textsc{np}\textsubscript{12}-pinkie\\
\glt ‘to defeat’   \tab\tab   ‘pinkie, little toe’
\z

\ea
\label{bkm:Ref75345979}
nyôtà   \tab   - \tab mótà\\
N-nyóta  \tab\tab    ∅-motá\\
\textsc{np}\textsubscript{9}-thirst   \tab\tab   \textsc{np}\textsubscript{5}-car\\
\glt ‘thirst’  \tab\tab      ‘car’
\z
\subsection{Taps}

The alveolar tap /r/ is phonemic, as seen from its contrast with /d/ in (\ref{bkm:Ref98924016}) and /t/ in (\ref{bkm:Ref98924030}).

\ea
\label{bkm:Ref98924016}
kùrùrà  \tab  - \tab kùdùrà\\
ku-rur-a  \tab\tab    ku-dur-a\\
\textsc{inf}-be\_bitter-\textsc{fv} \tab\tab   \textsc{inf}-be\_expensive-\textsc{fv}\\
\glt ‘to be bitter’  \tab\tab    ‘to be expensive’
\z

\ea
\label{bkm:Ref98924030}
kùrâmbà  \tab  - \tab kùtâmbà\\
ku-rámb-a  \tab\tab    ku-támb-a\\
\textsc{inf}-plaster-\textsc{fv} \tab\tab   \textsc{inf}-give\_herbs-\textsc{fv}\\
\glt ‘to plaster’  \tab\tab    ‘to give herbs (as medicine)’
\z

The alveolar tap /r/ has an allophone [l]. /r/ is realized as [l] before a high front vowel /i/ and as [r] elsewhere, as illustrated in (\ref{bkm:Ref98924062}) and (\ref{bkm:Ref98924065}).

\ea
\label{bkm:Ref98924062}
\glll [mùlìrò]\\
mu-riro\\
\textsc{np}\textsubscript{3}-fire\\
\glt ‘fire’
\z

\ea
\label{bkm:Ref98924065}
\glll [kùkûrà]\\
ku-kúr-a\\
\textsc{inf}-grow-\textsc{fv}\\
\glt ‘to grow’
\z

\ea
\glll [rùlímà]\\
ru-rimá\\
\textsc{np}\textsubscript{11}-bat\\
\glt ‘bat’
\z

Before the palatal glide /y/, /r/ is always realized as [l], as in (\ref{bkm:Ref98924221}), because /y/ is often (but not always) an allophonic realization of /i/. Before the labial glide /w/, /r/ is always realized as [r], as in (\ref{bkm:Ref98924237}), because /w/ is often (but not always) an allophonic realization of /u/.

\ea
\label{bkm:Ref98924221}
\glll [èzílyò]\\
e-zi-ryó\\
\textsc{aug}-\textsc{np}\textsubscript{8}-food\\
\glt ‘food’
\z

\ea
\label{bkm:Ref98924237}
\glll [kùrwârà]\\
ku-rwár-a\\
\textsc{inf}-be\_sick-\textsc{fv}\\
\glt ‘to be sick’
\z

In Zambian Fwe, /r/ is occasionally realized as [l] even when it is not followed by /i/. The proliferation of [l] in Zambian Fwe may be the result of the growing influence of Lozi in this area. Lozi resembles Fwe in that [l] and [ɾ] are allophones of the same phoneme, a\-l\-though their distribution is reversed with respect to Fwe; /l/ is realized as [ɾ] before the high front vowel, and as [l] elsewhere \citep[129]{Gowlett1989}.

\subsection{Glides}

Fwe has two glides, labial /w/ and palatal /y/. These occur as allophones of the vowels /u/ and /i/, or as epenthetic consonants (see \sectref{bkm:Ref491962181}), but also in envi\-ronments where their occurrence cannot be explained allophonically, and therefore /w/ and /y/ must be considered phonemes.

[w] can be in\-serted when the first of two vowels is a back vowel /u/ or /o/ (see \sectref{bkm:Ref491962181}). When /w/ is pre\-ceded by a vowel other than /u/ or /o/, its occurrence is phonemic, as in (\ref{bkm:Ref486264204}--\ref{bkm:Ref486264205}).

\ea
\label{bkm:Ref486264204}
\glll mbwâwà\\
∅-mbwáwa\\
\textsc{np}\textsubscript{1a}-jackal\\
\glt ‘jackal’
\z

\ea
\glll máꜝnshwáwánshàwà\\
má-nshawánshawa\\
\textsc{np}\textsubscript{6}-berry\\
\glt ‘berries of Grewia sp.’
\z

\ea
\glll bùnjèwè\\
bu-njewe\\
\textsc{np}\textsubscript{14}-poor\\
\glt ‘poverty’
\z

\ea
\label{bkm:Ref486264205}
\glll cìwàkàkà\\
ci-wakaka\\
\textsc{np}\textsubscript{7}-horned\_melon\\
\glt ‘horned melon (\textit{Cucumis metuliferus})’
\z

[y] may be used as an epenthetic consonant when one of two adjacent vowels is a front vowel, or when both vowels are /a/ (see \sectref{bkm:Ref491962181}). /y/ also occurs in other contexts, as illustrated in (\ref{bkm:Ref98513510}--\ref{bkm:Ref98513511}), motivating its analysis as a phoneme.

\ea
\label{bkm:Ref98513510}
\glll mòyà\\
mu-oya\\
\textsc{np}\textsubscript{3}-wind\\
\glt ‘wind’
\z

\ea
\glll ngûyà\\
∅-ngúya\\
\textsc{np}\textsubscript{1a}-baboon\\
\glt ‘baboon’
\z

\ea
\label{bkm:Ref98513511}
\glll kùyòcà\\
ku-bake-a\\
\textsc{inf}-bake-\textsc{fv}\\
\glt ‘to bake’
\z

The palatal glide may occur as an allophonic realization of /i/ before another vowel, but only when the preceding consonant is /r/ (in its allophonic realization [l], conditioned by the vowel /i/). There are also, however, sequences of /ri/ that are realized as /ri/, and not as /ry/, showing that /i/ is not automatically changed to a glide when preceded by /r/, and therefore the glide /y/ must be analyzed as a contrastive phoneme. An example is given in (\ref{bkm:Ref505356582}), where the root \textit{ríya} contains a sequence /ri/ that is not changed to /ry/. The following glide is an epenthetic consonant inserted to separate the vowel /i/ from the vowel /a/ in the following syllable (see \sectref{bkm:Ref491962181}).

\ea
\label{bkm:Ref505356582}
\glll rùrîyà\\
ru-ríya\\
\textsc{np}\textsubscript{11}-taro\\
\glt ‘taro’
\z

Glides may be preceded by another consonant, in which case they are subject to certain co-occurence restrictions, as discussed in \sectref{bkm:Ref451940391}.

\subsection{Clicks}

As shown in \tabref{tab:2:1}, Fwe has four click phonemes. Their functional load is fairly low, with only 84 words (out of a 2200 word database) with a click attested. Clicks are used in the variety of Fwe spoken in Namibia, and the variety of Zambian Fwe that is spoken close to the Namibian border, which forms a transition zone between Zambian and Namibian Fwe. In the northernmost variety of Fwe spoken in Zambia, clicks are not used. A more detailed discussion of clicks in Fwe can be found in {\citet{Gunnink2020}}.

Fwe uses different click types, the dental, lateral and post-alveolar, but click type is not contrastive; instead, the same word may be realized with a dental, lateral or palatal click without change in meaning, as in (\ref{bkm:Ref413681861}).

\ea
\label{bkm:Ref413681861}
kùǀàpùrà {\textasciitilde} kùǂàpùrà {\textasciitilde} kùǁàpùrà\\
ku-ǀapur-a\\
\textsc{inf}-tear-\textsc{fv}\\
\glt ‘to tear’
\z

Which click type is used depends mainly on the speaker, with the dental click being the most common. Of the thirteen speakers inter\-viewed for a contrastive study, the majority used only the dental click, and those who used a click type other than the dental, would also use the dental click.

Voicing and nasality, on the other hand, are used contrastively on clicks, and Fwe distinguishes four click phonemes on the basis of a combination of these features: a voiceless oral click /ǀ/, as in (\ref{bkm:Ref98833153}) a voiced oral click /ᵍǀ/, as in (\ref{bkm:Ref98833164}), a prenasalized voiceless click /ⁿǀ˳/,\textsuperscript{} as in (\ref{bkm:Ref98833192}), and a voiced nasal click /ⁿǀ/, as in (\ref{bkm:Ref98833195}).

\ea
\label{bkm:Ref98833153}
\glll rùǀómà\\
ru-ǀomá\\
\textsc{np}\textsubscript{11}-papyrus\\
\glt ‘papyrus’
\z

\ea
\label{bkm:Ref98833164}
\glll kù\textsuperscript{ ɡ}ǀárùmùkà\\
ku-\textsuperscript{ ɡ}ǀárumuk-a\\
\textsc{inf}-shout-\textsc{fv}\\
\glt ‘to shout loudly’
\z

\ea
\label{bkm:Ref98833192}
\glll mùⁿǀ̥ápì\\
mu-ⁿ{ǀ}̥apí\\
\textsc{np}\textsubscript{3}-frog\\
\glt ‘small frog sp.’
\z

\ea
\label{bkm:Ref98833195}
\glll kùⁿǀàmbùrà\\
ku-ⁿǀambur-a\\
\textsc{inf}-strip-\textsc{fv}\\
\glt ‘to strip (a tree)’
\z

Due to the small number of click words, the phonemic status of these four clicks is difficult to prove with minimal pairs. Two minimal pairs proving the contrast between the voiceless and voiced oral click are given in (\ref{bkm:Ref97911844}) and (\ref{bkm:Ref97911847}).

\ea
\label{bkm:Ref97911844}
kùǀàpùrà \tab   - \tab kùᵍǀàpùrà\\
ku-ǀapur-a \tab\tab     ku-ᵍǀapur-a\\
\textsc{inf}-tear-\textsc{fv} \tab\tab      \textsc{inf}-stand-\textsc{fv}\\
\glt ‘to tear’\tab\tab      ‘to stand with legs apart’
\z

\ea
\label{bkm:Ref97911847}
kùǀòpòrà  \tab  - \tab kùᵍǀòpòrà\\
ku-ǀopor-a    \tab\tab   ku-ᵍǀopor-a\\
\textsc{inf}-run-\textsc{fv}   \tab\tab   \textsc{inf}-remove\_flesh-\textsc{fv}\\
\glt ‘to run fast’   \tab\tab   ‘to remove flesh, an eye’
\z

Minimal pairs to prove the contrastive use of nasality in clicks are not attested, but nasality does seem to be a contrastive feature. When comparing the pronunciation of clicks of thirteen differ\-ent Fwe speakers, no variation was found in the realization of nasality: the same click words were consistently realized with a nasal click by all speakers. The near-minimal pairs in (\ref{bkm:Ref98513542}--\ref{bkm:Ref98513545}) provide further support for the analysis of nasality as a contrastive feature in clicks.

\NumTabs{8}
\ea
\label{bkm:Ref98513542}
ᵍǀúmù    \tab  - \tab  kùⁿǀûmà\\
∅-ᵍǀumú  \tab\tab    ku-ⁿǀúm-a\\
\textsc{np}\textsubscript{5}-reed   \tab\tab   \textsc{inf}-suck-\textsc{fv}\\
\glt ‘edible reed’  \tab    ‘to suck out blood (to treat disease, injury or curse)’
\z

\NumTabs{2}
\ea
\label{bkm:Ref98513545}
kùǀáꜝpwízà        - \tab  kù\textsuperscript{ŋ}ǀâmpà\\
ku-ǀámpwíz-a   \tab     ku-\textsuperscript{ŋ}ǀámp-a\\
\textsc{inf}-click-\textsc{fv}     \tab     \textsc{inf}-be\_flat-\textsc{fv}\\
\glt ‘to click in anger or resentment’ \tab  ‘to be flat (of stomach)’
\z

Although click type is not used contrastively, and click types can be interchanged by speakers, there do seem to be a few words where there is a preference for a click type, even for speakers who consistently use dental clicks elsewhere. This is the case for various interjection-like words, such as \textit{ǃakuroko} ‘it’s not true!’, which always takes a post-alveolar click, and \textit{ndi-ǁose} ‘it’s true’, which always takes a lateral click. A preference for the lateral click is also seen in \textit{nǁáꜝ}\textit{mpwízà} ‘to click in anger or resentment’; although the pronunciation with the dental click can also be heard, the pronunciation with the lateral click was preferred. This most likely relates to the meaning of the word, which is to produce a lateral click as a sign of anger or resentment. The same word occurs in Yeyi as \textit{kùnǁàpìzá} ‘disapprove by making a lateral click’ \citep[43]{Seidel2008}, which also has a lateral click, even though lateral clicks are otherwise marginal in the language.

In addition to the free variation between click types, speakers of Fwe in some areas also alternate clicks with non-click consonants. These non-click consonants share the voicing and nasality contrasts of their click counterparts, and are always velar, even though clicks are usually dental. The alternation between clicks and non-click consonants is the result of the loss of the front closure of the click, which is usually dental, so that only the back closure, which is always velar, remains. The voiceless click may alternate with [k], as in (\ref{bkm:Ref98513561}).

\ea
\label{bkm:Ref98513561}
rùǀómà {\textasciitilde} rùkómà\\
ru-ǀomá\\
\textsc{np}\textsubscript{11}-papyrus\\
\glt ‘papyrus’
\z

The voiced click may alternate with [g], as in (\ref{bkm:Ref98513577}). There is also one example, given in (\ref{bkm:Ref98513578}), of a voiced click alternating with either [g] or [d].

\ea
\label{bkm:Ref98513577}
èᵍǀìmà {\textasciitilde} ègìmà\\
e-∅-ᵍǀima\\
\textsc{aug}-\textsc{np}\textsubscript{5}-fish\\
\glt ‘small fish sp.’
\z

\ea
\label{bkm:Ref98513578}
ᵍǀúkùmù {\textasciitilde} gúkùmù {\textasciitilde} dúkùmù\\
∅-ᵍǀúkumu\\
\textsc{np}\textsubscript{5}-fruit\\
\glt ‘fruit sp.’
\z

The prenasalized voiceless click may alternate with [ᵑk], as in (\ref{bkm:Ref98513598}).

\ea
\label{bkm:Ref98513598}
mùⁿǀ̥ápì {\textasciitilde} mùᵑkápì\\
mu-ⁿ{ǀ}̥apí\\
\textsc{np}\textsubscript{3}-frog \\
\glt ‘frog sp.’
\z

The voiced nasal click may alternate with [ŋ], as in (\ref{bkm:Ref98513599}).

\ea
\label{bkm:Ref98513599}
kùⁿǀúmèntà {\textasciitilde} kùŋúmèntà\\
ku-ⁿǀúment-a\\
\textsc{inf}-kiss-\textsc{fv}\\
\glt ‘to kiss’
\z

Free variation between clicks and non-click velars is mainly seen in the central region of the Fwe-speaking area, close to the Namibian/Zambian border, where the Zambian clickless vari\-ety and the Namibian click-using variety come into contact with each other. {\citet{Gunnink2020}} therefore analyzes this free variation as the result of contact between these two varieties.

\section{Vowels}
\label{bkm:Ref489974139}\hypertarget{Toc75352603}{}
Fwe has five contrastive vowel phonemes, which are discussed in \sectref{bkm:Ref451524795} together with evidence for their phonemic status. Vowel length plays a role in Fwe in three different ways. Firstly, there is a phonemic distinction between long and short vowels, even though long vowels are quite rare (\sectref{bkm:Ref459132406}). Secondly, there are two environments in which Fwe automatically lengthens vowels: before and after certain consonants (\sectref{bkm:Ref459132421}), and in the penultimate mora of a phrase-final word (\sectref{bkm:Ref451507715}). Although vowel length and the two processes of automatic lengthening differ in their conditioning, they are very similar in their phonetic properties: phonemically long vowels, automatically lengthened vowels and vowels affected by penultimate lengthening are equally long, and the distinction between short vowels and long or length\-ened vowels is very minimal and possibly diminishing, though their importance in the tonal sys\-tem remains. Furthermore, both long vowels and automatically lengthened vowels contain two tone-bearing units, rather than one. Penultimate lengthening, however, does not affect the number of moras.

\subsection{Phonemic vowels}
\label{bkm:Ref451524795}\hypertarget{Toc75352604}{}
Fwe has five contrastive vowel phonemes, /i, ɛ, a, ɔ, u/, as attested by the minimal pairs in (\ref{bkm:Ref98513649}--\ref{bkm:Ref98513665}). Throughout this book, /ɛ/ will be written as <e> and /ɔ/ will be written as <o>.

\NumTabs{9}
\ea
\label{bkm:Ref98513649}
kùkûmbà \tab - \tab   kùkômbà \tab - \tab kùkâmbà\\
ku-kúmb-a  \tab\tab  ku-kómb-a  \tab\tab  ku-kámb-a\\
\textsc{inf}-howl-\textsc{fv} \tab\tab   \textsc{inf}-lick-\textsc{fv} \tab\tab   \textsc{inf}-clap-\textsc{fv}\\
\glt ‘to howl’  \tab\tab  ‘to lick’  \tab\tab\tab  ‘to clap’
\z

\ea
kùmìnà \tab   - \tab\tab  kùmènà\\
ku-min-a   \tab\tab   ku-men-a\\
\textsc{inf}-swallow-\textsc{fv}  \tab\tab  \textsc{inf}-sprout-\textsc{fv}\\
\glt ‘to swallow’   \tab\tab   ‘to sprout (of wild plants)’
\z

\ea
kùsîkà  \tab\tab    -\tab  kùsûkà\\
ku-sík-a  \tab \tab\tab   ku-súk-a\\
\textsc{inf}-light-\textsc{fv}   \tab\tab   \textsc{inf}-descend-\textsc{fv}\\
\glt ‘to light’  \tab\tab    ‘to descend’
\z

\ea
\label{bkm:Ref98513665}
kùrêːtà   \tab\tab -\tab  kùrôːtà\\
ku-réːt-a   \tab\tab   ku-róːt-a\\
\textsc{inf}-bring-\textsc{fv}   \tab\tab   \textsc{inf}-dream-\textsc{fv}\\
\glt ‘to bring’  \tab\tab    ‘to dream’
\z
\subsection{Phonemic vowel length}
\label{bkm:Ref459132406}\label{bkm:Ref451506997}\hypertarget{Toc75352605}{}
Fwe has a phonemic opposition between short and long vowels, as shown by the mini\-mal pairs in (\ref{bkm:Ref459125515}) and (\ref{bkm:Ref459125516}). Phonemic vowel length is marked in the orthography used in this book with the sym\-bol /ː/.

\ea
\label{bkm:Ref459125515}
kùkûrà   \tab\tab -\tab  kùkûːrà\\
ku-kúr-a   \tab\tab   ku-kúːr-a\\
\textsc{inf}-grow-\textsc{fv}   \tab\tab   \textsc{inf}-shift-\textsc{fv}\\
\glt ‘to grow’   \tab\tab   ‘to shift, move house’
\z

\ea
\label{bkm:Ref459125516}
kùkôrà \tab\tab   - \tab kùkôːrà\\
ku-kór-a   \tab\tab   ku-kóːr-a\\
\textsc{inf}-irritate-\textsc{fv}  \tab\tab  \textsc{inf}-cough-\textsc{fv}\\
\glt ‘to irritate’   \tab\tab   ‘to cough’
\z

All five vowel qualities occur as either short or long; examples of /oː/ and /uː/ are given in (\ref{bkm:Ref459125515}--\ref{bkm:Ref459125516}). Examples of /aː/, /eː/ and /iː/ are given in (\ref{bkm:Ref459125517}--\ref{bkm:Ref459125519}). Long vowels can occur in any position of the word, and word-final long vowels are not shortened, as seen in (\ref{bkm:Ref459125519}--\ref{bkm:Ref497906688}).

\ea
\label{bkm:Ref459125517}
\glll kùrâːrà\\
ku-ráːr-a\\
\textsc{inf}-sleep-\textsc{fv}\\
\glt ‘to sleep’
\z

\ea
\glll kùkèːzyà\\
ku-keːzy-a\\
\textsc{inf}-come-\textsc{fv}\\
\glt ‘to come’
\z

\ea
\label{bkm:Ref459125519}
\glll ègîː\\
e-∅-gíː\\
\textsc{aug}-\textsc{np}\textsubscript{5}-egg\\
\glt ‘egg’
\z

\ea
\label{bkm:Ref497906688}
\glll yènkêː\\
ye-nkéː\\
\textsc{pp}\textsubscript{1}-one\\
\glt ‘alone’
\z

In some cases, a long vowel in Fwe is a reflex of a reconstructed long vowel or vowel sequence for Proto-Bantu, as in (\ref{bkm:Ref459626466}--\ref{bkm:Ref459626468}).

\ea
\label{bkm:Ref459626466}
kùrôːtà (from *dóot ‘dream’ \citep{BastinEtAl2002})\\
ku-róːt-a\\
\textsc{inf}-dream-\textsc{fv}\\
\glt ‘to dream’
\z

\ea
kùkâːnà (from *káan ‘deny, refuse’ \citep{BastinEtAl2002})\\
ku-káːn-a\\
\textsc{inf}-reject-\textsc{fv}\\
\glt ‘to reject, divorce’
\z

\ea
\label{bkm:Ref459626468}
bùrêː (from *dàì ‘long’ \citep{BastinEtAl2002})\\
bu-réː\\
\textsc{np}\textsubscript{14}-long\\
\glt ‘length’
\z

Long vowels may also be the result of the historical merger of two vowels across a morpheme boundary. Example (\ref{bkm:Ref98513732}) shows that the verb root \textit{coːr} historically consisted of a root \textit{cò} and a separa\-tive suffix -\textit{or}, because the transitive separative suffix -\textit{or} can be replaced by an intransitive separative suffix -\textit{ok}. (For more on the separative derivation, see \sectref{bkm:Ref486270391}.) The underived root \textit{co} is not attested in Fwe.

\ea
\label{bkm:Ref98513732}
\ea
\glll kùcòːrà\\
ku-coːr-a\\
\textsc{inf}-break-\textsc{fv}\\
\glt ‘to break’

\ex
\glll kùcòːkà\\
ku-co-ok-a\\
\textsc{inf}-break-\textsc{sep}.\textsc{intr}-\textsc{fv}\\
\glt ‘to break’
\z
\z

In other verb roots where the long vowel appears to result from a historical merger of two short vowels, the modern form of the verb can no longer take differ\-ent suffixes. Nonetheless, formal similarities between the verb root and attested derivational suffixes in Fwe do show that the long vowels in these verbs go back to a historical merger of the vowel of the root with the vowel of a derivational suffix, which has subsequently become unanalyza\-ble. This is in line with the fact that many derivational suffixes in Fwe are lexicalized. Examples include the verb root \textit{ziːk} ‘hide’, which appears to contain the transitive impositional suffix -\textit{ik} (for more on the impositional, see \sectref{bkm:Ref450835510}), and the verb root \textit{zúːr} ‘undress’, which appears to contain the transitive separative suffix \textit{-ur} (see \sectref{bkm:Ref486270693} for the various allo\-morphs of this suffix).

Long vowels only arise from historical processes of vowel juxtaposition; synchronic vowel juxta\-position does not always lead to vowel lengthening. This is discussed in more detail in \sectref{bkm:Ref491962181}.

Vowel length plays an important role in the tonal system of Fwe. Long vowels are bimoraic, and a high tone can be assigned to either of the two moras. Subsequently, however, the high tone is copied onto the other mora of the vowel, so that the surface realizations of tones on bimoraic vowels are identical to the surface realizations of tones on monomoraic vowels. This is discussed in more detail in chapter \ref{bkm:Ref451507583} on tone.

Long vowels are not common in Fwe: only 30 words (out of a 2,200-word list) with a long vowel have been identified. Furthermore, the phonetic realization of phonemic vowel length is fairly subtle, and its effects are mainly found in the tonal system. It seems then that phonemic vowel length is becoming increasingly marginal in Fwe.

\subsection{Automatic vowel lengthening}
\label{bkm:Ref459132421}\hypertarget{Toc75352606}{}
In addition to phonemic vowel length, Fwe has automatic, non-contrastive vowel lengthening, which is conditioned by the nature of the consonants following and preceding the vowel. In order to distinguish it from phonemic lengthening, automatic lengthening is not marked in the orthography used in this book, with the exception of the examples given in this section, where lengthening is marked with the symbol [ː].

There are a number of different phonological environments that condition vowel lengthening. Firstly, vowels are lengthened when preceded by the a consonant cluster involving a glide /w/ or /y/. Lengthening can target vowels in word-medial position, as in (\ref{bkm:Ref458605409}), but also in word-final position, as in (\ref{bkm:Ref458605425}--\ref{bkm:Ref459192501}).

\ea
\label{bkm:Ref458605409}
\glll kùtwâːrà\\
ku-twár-a\\
\textsc{inf}-bring-\textsc{fv}\\
\glt ‘to bring’
\z

\ea
\label{bkm:Ref458605425}
\glll kúryàː\\
ku-rí-a\\
\textsc{inf}-eat-\textsc{fv}\\
\glt ‘to eat’
\z

\ea
\label{bkm:Ref459192501}
\glll kàmwîː\\
ka-mwí\\
\textsc{np}\textsubscript{12}-heat\\
\glt ‘heat; afternoon’
\z

Vowels are also lengthened if immediately followed by a prenasalized consonant, as illustrated in (\ref{bkm:Ref98513796}) and (\ref{bkm:Ref98513798}).

\ea
\label{bkm:Ref98513796}
\glll kùrâːmbà\\
ku-rámb-a\\
\textsc{inf}-plaster-\textsc{fv}\\
\glt ‘to plaster’
\z

\ea
\label{bkm:Ref98513798}
\glll kùtùːmpà\\
ku-tump-a\\
\textsc{inf}-sprout-\textsc{fv}\\
\glt ‘to sprout (of wild plants)’
\z

Vowel lengthening also occurs when the vowel /a/ is preceded by an alveolar fricative. Both the prenasalized fricatives /ns/ and /nz/ and the non-prenasalized fricatives /s/ and /z/ cause the follow\-ing /a/ to lengthen, as shown in (\ref{bkm:Ref459627036}--\ref{bkm:Ref459627038}). The post-alveolar fricatives /sh/ and /zy/, however, do not cause the following vowels to lengthen, as shown in (\ref{bkm:Ref486271209}--\ref{bkm:Ref486271211}).

\ea
\label{bkm:Ref459627036}
\glll kùyáshìmìsàː\\
ku-yáshimis-a\\
\textsc{inf}-sneeze-\textsc{fv}\\
\glt ‘to sneeze’
\z

\ea
\glll òːnsâː\\
o-∅-nsá\\
\textsc{aug}-\textsc{np}\textsubscript{1a}-duiker\\
\glt ‘duiker (antelope sp.)’
\z

\ea
\label{bkm:Ref459627038}
\glll kùzàːnà\\
ku-zan-a\\
\textsc{inf}-play-\textsc{fv}\\
\glt ‘to play’
\z

\ea
\label{bkm:Ref486271209}
\glll kùshàkà\\
ku-shak-a\\
\textsc{inf}-want-\textsc{fv}\\
\glt ‘to want, like, love’
\z

\ea
\label{bkm:Ref486271211}
\glll kùzyàbàrà\\
ku-zyabar-a\\
\textsc{inf}-dress-\textsc{fv}\\
\glt ‘to get dressed’
\z

Lengthening of /a/ before alveolar fricatives is the last step in a process of sound change and ana\-logical extension very similar to what is described for Ganda \citep{Hyman2003a}. In Ganda, a causa\-tive suffix -\textit{i} caused spirantization of the last consonant of the root of the verb to /s/. The vowel /i/ of the causative was subsequently absorbed into the preceding consonant, combined with compensatory lengthening of the final vowel \textit{-a} of the verb. In other verbs ending in /sa/, where no causative morphology is present, the lengthening was added in analogy with the lengthen\-ing in causative verbs. A similar process appears to have taken place in Fwe, where an earlier causa\-tive suffix *i also triggered spirantization of the previous consonant to /s/ or /z/, leading to the loss of /i/ and compensatory lengthening.\footnote{In Ganda, this process involved glide formation from /i/ to /y/ \citep{Hyman2003a}. In Fwe, there is no clear evidence for glide formation, e.g. no causative verbs are attested where /s/ is followed by /y/. It is possible that glide formation historically took place, and that the glide was subsequently lost, as Fwe does not allow (or no longer allows) combinations of /s/ and /y/ (see \sectref{bkm:Ref451940391} on co-occurrence restrictions).} Although this process is no longer productive in Fwe, examples such as (\ref{bkm:Ref98514649}) and (\ref{bkm:Ref98514652}) show that the change of a final stem consonant to /s/ or /z/ was part of causative formation (see \sectref{bkm:Ref451514620} for more examples).

\ea
\label{bkm:Ref98514649}
\ea
\glll kùtùkùtà\\
ku-tukut-a\\
\textsc{inf}-become\_warm-\textsc{fv}\\
\glt ‘to become warm’

\ex
\glll kùtùkùsàː\\
ku-tukus-a\\
\textsc{inf}-become\_warm.\textsc{caus}-\textsc{fv}\\
\glt ‘to warm (something) up’

\ex
  from ku-tukut-i-a > ku-tukus-i-a > ku-tukus-aː
\z\z

\ea
\label{bkm:Ref98514652}
\ea
\glll kùhârà\\
ku-hár-a\\
\textsc{inf}-live-\textsc{fv}\\
\glt ‘to live’

\ex
\glll kùhâzàː\\
ku-ház-a\\
\textsc{inf}-save.\textsc{caus}-\textsc{fv}\\
\glt ‘to save (lit. ‘make someone live’)’

\ex
  from ku-har-i-a > ku-haz-i-a > ku-haz-aː
\z\z

The lengthening of the final vowel /a/ in causative verbs is the result of compensatory length\-ening triggered by the loss of the earlier vowel /i/. Subsequently, all instances of /a/ after an alveolar fricative where lengthened, not only those that were the result of causative formation. Whereas in Ganda, this analogical extension was limited to /sa/ sequences at the end of a verb, in Fwe the analogical extension includes all instances of /a/ before an alveolar fricative, also when such a sequence is not the last syllable of a verb stem, as in (\ref{bkm:Ref459283655}--\ref{bkm:Ref459283656}), and even in nouns, as in (\ref{bkm:Ref459283674}--\ref{bkm:Ref459283678}).

\ea
\label{bkm:Ref459283655}
\glll kùzàːnà\\
ku-zan-a\\
\textsc{inf}-dance-\textsc{fv}\\
\glt ‘to dance, play’
\z

\ea
\label{bkm:Ref459283656}
\glll kùzâːrà\\
ku-zár-a\\
\textsc{inf}-give\_birth-\textsc{fv}\\
\glt ‘to give birth (of animals)’
\z

\ea
\label{bkm:Ref459283674}
\glll èsàːbúrè\\
e-∅-saburé\\
\textsc{aug}-\textsc{np}\textsubscript{5}-machete\\
\glt ‘machete’
\z

\ea
\glll káꜝnéːnsàː\\
ká-nensá\\
\textsc{np}\textsubscript{12}-pinkie\\
\glt ‘pinkie, little toe’
\z

\ea
\glll ⁿǀórꜝézàː\\
N-ⁿǀórezá\\
\textsc{np}\textsubscript{9}-resin\\
\glt ‘resin’
\z

\ea
\label{bkm:Ref459283678}
\glll nzâːsì\\
N-zási\\
\textsc{np}\textsubscript{10}-spark\\
\glt ‘sparks’
\z

That the lengthening of /a/ before /s/ and /z/ is the result of analogical extension, and not of individual cases of spirantization in each of the words that contain a /sa/ or /za/ sequence, can be seen from the fact that many words with /sa/ and /za/ sequences are borrowings, such as \textit{mù-sâː} ‘thief’ from Khwe \textit{tc’á̰à̰} ‘to steal’ \citep[355]{Kilian-Hatz2003}\footnote{In this case, however, the source word also has a long vowel.}, \textit{kù-sèbèz-àː} ‘to work’, from Lozi \textit{ku sebeza} ‘to work’ \citep[168]{Burger1960}.\footnote{An alternative explanation for the origin of lengthening of /a/ before /s/ and /z/ would be a more general rule of spirantization followed by glide absorption and compensatory lengthening, not only in causative verbs. This would fail to explain, however, why only the alveolar fricatives are affected, and not the labiodental fricatives, which are also the result of spirantization.}

Although phonemically long vowels and automatically lengthened vowels differ in their condi\-tioning, their behavior is otherwise parallel. Both long vowels and lengthened vowels con\-tain two tone-bearing units rather than one, an important distinction in the tonal system of Fwe (see Chapter \ref{bkm:Ref451507583}). Furthermore, the difference between both long vowels and lengthened vowels, and short vowels, is very minimal, and the actual length or lengthening is barely perceptible. This is a trait Fwe shares with closely related Totela, which also lengthens vowels under conditions comparable to those in Fwe, but barely so. As {\citet[71]{Crane2011}} states, “I found vowel length some\-what hard to perceive, especially in nouns, and speakers did not correct my productions for it as they corrected for tone and other segmental errors”. Precise phonetic measurements of short and long vowels in Fwe should be done in order to understand the degree of vowel lengthening in Fwe.

\subsection{Penultimate lengthening}
\label{bkm:Ref451507715}\hypertarget{Toc75352607}{}
Fwe also makes use of a second type of predictable vowel lengthening, which targets the penultimate mora of a phrase-final word. The automatic lengthening of phrase-final penultimate vowels is common in Bantu languages, and had already been noted for Fwe by {\citet[111]{Bostoen2009}}. As penultimate lengthening is predictable, it is not marked in the orthography used in this book, with the exception of the exam\-ples in this section, where lengthening is marked with [ː].

Lengthening targets the penultimate mora of an utterance-final word, as seen in (\ref{bkm:Ref98853405}--\ref{bkm:Ref98853407}).

\ea
\label{bkm:Ref98853405}
\glll cìbàːkà\\
ci-baka\\
\textsc{np}\textsubscript{7}-place\\
\glt ‘place
\z

\ea
\label{bkm:Ref98853407}
\glll kùbábàrèːrà\\
ku-bábarer-a\\
\textsc{inf}-guard-\textsc{fv}\\
\glt ‘to guard’
\z

Penultimate lengthening targets the mora, and not the syllable; if the last syllable of a phrase-final word is bimoraic, such as the bimoraic last syllable \textit{kwaː} in (\ref{bkm:Ref459628182}), lengthening does not target the penultimate syllable \textit{ro}, but the penultimate mora of the last syllable. As such penultimate lengthening is realized on the last syllable rather than the penultimate syllable.

\ea
\label{bkm:Ref459628182}
kùkósòròkwàː\\
*kùkósòròːkwà\\
ku-kósorokw-a\\
\textsc{inf}-sleep-\textsc{fv}\\
\glt ‘to sleep until rested’
\z

\hspace*{-.3pt}Penultimate lengthening can target automatically lengthened vowels, in which case both types of length are cumulative; an automatically lengthened vowel in the penultimate position is pro\-nounced with more length than an automatically lengthened vowel in other positions.

Penultimate lengthening can also target phonemically long vowels. In this case too, both types of length are cumulative, and long vowels in the penultimate position are audibly longer than long vowels in other positions. This is illustrated in (\ref{bkm:Ref459289155}--\ref{bkm:Ref459289173}) with the verbal root \textit{coːk}, which contains a long vowel /oː/. If the vowel /oː/ occurs in the penultimate syllable of an utterance, as in (\ref{bkm:Ref459289155}), it is pronounced with more length (indicated by a double ː symbol) than when the same vowel is used in a position other than the penultimate, as seen in (\ref{bkm:Ref459289173}).

\ea
\label{bkm:Ref459289155}
\glll càcôːːkì\\
ci-a-có̲ːk-i\\
\textsc{sm}\textsubscript{7}-\textsc{pst}-break-\textsc{npst}.\textsc{pfv}\\
\glt ‘It broke.’
\z

\ea
\label{bkm:Ref459289173}
\glll cìcóːkêtè\\
ci-coːk-é̲te\\
\textsc{sm}\textsubscript{7}-break-\textsc{stat}\\
\glt ‘It is broken.’
\z

This shows that phonetically, there is a three-way length distinction in Fwe. Short vowels are pronounced with the least length; intermediate lengthening is found with phonemi\-cally long vowels, and automatically lengthened vowels or vowels in the penultimate position; and vow\-els where penultimate lengthening combines with contrastive vowel length or automatic lengthening are pronounced with the most length. This three-way distinction is not phonemic, however, because the difference between intermediate and long is determined by at least one non-contrastive factor, penultimate lengthening.

Impressionistically, penultimate lengthening is quite subtle, with only a very small difference between vowels with and without penultimate lengthening. Its phonetic realization is comparable to both phonemic vowel length and phonetic vowel lengthening, with the difference between short vo\-wels on the one hand and either long vowels, automatically lengthened vowels or penultimate lengthened vowels on the other hand being quite small.

Whereas automatically lengthened vowels are counted as bimoraic in the tonal system of Fwe (cf. \sectref{bkm:Ref451506997}), vowels targeted by penultimate lengthening are not counted as bimoraic, but as monomoraic. Penultimate lengthening does influence the tonal system, however, the realization of high tones as falling is only possible on vowels that are targeted by penultimate lengthening (see \sectref{bkm:Ref432074291} of Chapter \ref{bkm:Ref451507583} on tone).

\section{Syllable structure}
\hypertarget{Toc75352608}{}
Fwe has a strictly open syllable structure, which is discussed in \sectref{bkm:Ref451875134}. Certain consonants are subject to co-occurrence restrictions, as shown in \sectref{bkm:Ref451940391}.

\subsection{Syllable types}
\label{bkm:Ref451875134}\hypertarget{Toc75352609}{}
Fwe has a strictly open syllable structure, where coda consonants are never allowed. Fwe allows for three different syllable types: CV, where the onset is a consonant and the nucleus a vowel, CGV, where the onset is a consonant followed by a glide, and V, which lacks an onset and con\-sists of a vowel only. All three syllable types can be seen to occur in (\ref{bkm:Ref451873278}).

\ea
\label{bkm:Ref451873278}
  [ò.kù.rwà]\\
\glt ‘to fight’
\z

A syllable onset may also consist of a nasal followed by another consonant. These nasal-conso\-nant combinations are analyzed as a single prenasalized pho\-neme rather than a combination of two phonemes, and have been discussed in \sectref{bkm:Ref70695065}.

V syllables may occur word-initially or word-medially. In the latter case, the resultant VV se\-quence is often broken up by an epenthetic consonant [h], [y] or [w] (see \sectref{bkm:Ref491962181}). Consonant epenthesis is not obligatory, however, and word-medial VV sequences are allowed, as shown in the following examples. VV sequences may contain two different vowels, as in (\ref{bkm:Ref489365098}), or two identical vowels, as in (\ref{bkm:Ref489365100}).

\newpage
\ea
\label{bkm:Ref489365098}
  V.V sequences of two different vowels

\ea
\glll mà.rì.â.njò\\
∅-mariánjo\\
\textsc{np}\textsubscript{1a}-virgin\\
\glt ‘virgin’

\ex
\glll mbó.ꜝé.rà\\
∅-mbóerá\\
\textsc{np}\textsubscript{1a}-wild\_dog\\
\glt ‘wild dog’

\ex
\glll kù.fú.à.mà\\
ku-fú-am-a\\
\textsc{inf}-approach-\textsc{imp}.\textsc{intr}-\textsc{fv}\\
\glt ‘to approach’
\z\z

\ea
\label{bkm:Ref489365100}
  V.V sequences of two identical vowels

\ea
\glll kù.bò.ò.rà\\
ku-boor-a\\
\textsc{inf}-return-\textsc{fv}\\
\glt ‘to return’

\ex
\glll ndà.à.nò\\
N-daano\\
\textsc{np}\textsubscript{9}-message\\
\glt ‘message’

\ex
\glll kù.cù.ù.nà\\
ku-cuun-a\\
\textsc{inf}-limp-\textsc{fv}\\
\glt ‘to limp’
\z\z

Vowel sequences are distinct from long vowels or lengthened vo\-wels (see Sections \ref{bkm:Ref459132406}-\ref{bkm:Ref451507715}). Vowel sequences are longer than long or lengthened vowels, and also have different possible tonal realizations, as shown in \tabref{tab:2:2}. Vowels in sequences can each take a different tone; the patterns L-L, H-H, H-L, L-H and F-L are all attested. Long and lengthened vowels only take one of the following three tonal melodies: L, H, and F.

\begin{table}
\label{bkm:Ref98925329}\caption{\label{tab:2:2}Tonal patterns on vowel sequences and long vowels}

\begin{tabular}{llll}
\lsptoprule
& Vowel sequences &  & Lengthened vowels\\
\midrule
LL & kù.nyè.è.zà ‘to annoy’ & L & kù.nèː.ngà ‘to dance, play’\\
HH & mvú.ú ‘hippopotamus’ & H & kù.túː.mbù.kà ‘to burn’\\
HL & mvú.ù ‘hippopotamus’ & F & kù.bûː.mbà ‘to create, mould’\\
LH & ndì.rà.á.nà ‘I say goodbye.’ &  & \\
FL & ntû.ù ‘hyena’ &  & \\
\lspbottomrule
\end{tabular}
\end{table}

The fact that both vowels can take a different tone shows that these vowels are sequences of two separate vowels of identical vowel quality, and that each vowel functions as its own tone-bearing unit. Further\-more, vowel sequences can be broken up by an epenthetic consonant [h], [y] or [w], as shown with the vowel sequence /o.o/ in (\ref{bkm:Ref459634073}) (see also \sectref{bkm:Ref491962181}), but lengthened or long vowels can never be separated by an epenthetic consonant, as shown with the long vowel [oː] in (\ref{bkm:Ref459634108}).

\ea
\label{bkm:Ref459634073}
[kù.bò.ò.rà] {\textasciitilde} [kù.bò.hò.rà]\\
/kù-bòòr-à/\\
\textsc{inf}-return-\textsc{fv}\\
\glt ‘to return’
\z

\ea
\label{bkm:Ref459634108}
[kù.còː.kà]\\
*[kù.cò.hò.kà]\\
/ku-coːk-a/\\
\textsc{inf}-break-\textsc{fv}\\
\glt ‘to break’
\z

Vowel sequences and lengthened vowels are also distinct from a historical point of view; vowel sequences (of either identical or different vowels) mostly derive from original CV.CV sequences, from which the second consonant was lost through regular diachronic sound changes. This has affected *p and *g, which were both lost before non-high vowels \citep{Bostoen2009}. Examples of such vowel sequences and their etymology are given in (\ref{bkm:Ref98514755}--\ref{bkm:Ref98514756}).

\ea
\label{bkm:Ref98514755}
fwî.ì (from *kúpɪ ‘short’ \citep{BastinEtAl2002})\\
\glt ‘short’
\z

\ea
njú.ù (from *jʊgʊ ‘groundnut’ \citep{BastinEtAl2002})\\
N-juú\\
\textsc{np}\textsubscript{10}-pea\\
\glt ‘peas’
\z

\ea
\label{bkm:Ref98514756}
njûò (from *jʊgò ‘house’ \citep{BastinEtAl2002})\\
N-júo\\
\textsc{np}\textsubscript{9}-house\\
\glt ‘house’
\z

Long vowels, on the other hand, derive from earlier long vowels or vowel sequences, as discussed in \sectref{bkm:Ref459132406}, and length\-ened vowels are the result of predictable synchronic processes as discussed in Sections \ref{bkm:Ref459132421} and \ref{bkm:Ref451507715}. Based on both synchronic and diachronic evidence, it is clear that vowel sequences of either identical or different vowels are distinct from long or lengthened vowels.

\subsection{Co-occurrence restrictions}
\hypertarget{Toc75352610}{}\label{bkm:Ref451940391}
There are a number of restrictions on which vowel can be preceded by which conso\-nant. Labiodental and alveolar frica\-tives are mainly followed by high vowels or glides. This is the result of the diachronic sound change of Bantu Spirantization, whereby stops followed by a high vowel changed into a fricative, followed by a merger of high vowels and near-high vowels. No restrictions apply to the postalveolar fricatives /sh/ and /zy/, the bilabial fricative /b/ and the glottal fricative /h/ because they are not the result of Bantu Spirantization, but of a change of the reconstructed stops to fricatives before non-high vowels \citep{Bostoen2009}.

The alveolar fricatives /s/ and /z/, as well as their prenasalized counterparts, can only be followed by a high vowel /i/ or /u/, or a glide /w/. Examples are given for /s/ and /ns/ in (\ref{bkm:Ref488828866}), and for /z/ and /nz/ in (\ref{bkm:Ref488828867}).

\ea
\label{bkm:Ref488828866}
\glll kùsîkà\\
ku-sík-a\\
\textsc{inf}-light-\textsc{fv}\\
\glt ‘to light’
\z

\ea
\glll mùsùmò\\
mu-sumo\\
\textsc{np}\textsubscript{3}-pole\\
\glt ‘pole’
\z

\ea
\glll múswà\\
mu-swá\\
\textsc{np}\textsubscript{3}-rope\\
\glt ‘small rope’
\z

\ea
\glll mùsûnsù\\
mu-súnsu\\
\textsc{np}\textsubscript{3}-lower\_leg\\
\glt ‘front part of lower leg’
\z

\ea
\label{bkm:Ref488828867}
\glll zìbà\\
∅-ziba\\
\textsc{np}\textsubscript{5}-lake\\
\glt ‘lake’
\z

\ea
\glll cìzùmà\\
ci-zuma\\
\textsc{np}\textsubscript{7}-basket\\
\glt ‘basket’
\z

\ea
\glll rùbênzwà\\
ru-bénzwa\\
\textsc{np}\textsubscript{11}-pancreas\\
\glt ‘pancreas’
\z

A number of exceptions are found, which are mostly borrowings; some examples are given in (\ref{bkm:Ref98514789}--\ref{bkm:Ref98514790}).

\ea
\label{bkm:Ref98514789}
sákà (from Afrikaans sak ‘bag’)\\
∅-saká\\
\textsc{np}\textsubscript{5}-bag\\
\glt ‘bag’
\z

\ea
kùsèpà (from Lozi ku sepa ‘to trust’ \citep{Burger1960})\\
ku-sep-a\\
\textsc{inf}-trust-\textsc{fv}\\
\glt ‘to trust, hope’
\z

\ea
\label{bkm:Ref98514790}
mùsâ (from Khwe tc’á̰à̰ ‘to steal’ \citep[355]{Kilian-Hatz2003})\\
mu-sá\\
\textsc{np}\textsubscript{1}-thief\\
\glt ‘thief’
\z

Another exception occurs when the alveolar fricative is part of a causative. The synchronically productive causa\-tive suffix \textit{-is} can be followed by the vowels /a/, as in (\ref{bkm:Ref98515057}), or /e/, as in (\ref{bkm:Ref98515059}), functioning as inflectional suffixes, or the vowel /o/, as in (\ref{bkm:Ref98515091}), functioning as a nominalizing suffix.

\ea
\label{bkm:Ref98515057}
\glll kùùrìsà\\
ku-ur-is-a\\
\textsc{inf}-buy-\textsc{caus}-\textsc{fv}\\
\glt ‘to sell’
\z

\ea
\label{bkm:Ref98515059}
\glll òndìtúsè\\
o-ndi-tus-é̲\\
\textsc{sm}\textsubscript{2SG}-\textsc{om}\textsubscript{1SG}-help-\textsc{pfv}.\textsc{sbjv}\\
\glt ‘You should help me.’
\z

\ea
\label{bkm:Ref98515091}
\glll cìkùrìsò\\
ci-kur-is-o\\
\textsc{np}\textsubscript{7}-sweep-\textsc{caus}-\textsc{ins}\\
\glt ‘broom’
\z

Some instances of /s/ or /z/ are the result of an earlier causative suffix \nobreakdash-\textit{i}, which caused spirantization of the preceding consonant. In these lexicalized causative forms, alveolar fricatives may also combine with vowels other than /i/ or /u/, as in (\ref{bkm:Ref98515129}--\ref{bkm:Ref98515130}).

\ea
\label{bkm:Ref98515129}
\glll kùbûsà\\
ku-bús-a\\
\textsc{inf}-wake-\textsc{fv}\\
\glt ‘to wake (someone) up’
\z

\ea
\glll mbòndímùbúsè\\
mbo-ndí̲-mu-bu\textsubscript{H}s-é̲\\
\textsc{near}.\textsc{fut}-\textsc{sm}\textsubscript{1SG}-\textsc{om}\textsubscript{1}-wake-\textsc{pfv}.\textsc{sbjv}\\
\glt ‘I will wake her/him up.’
\z

\ea
\glll kùfwìnsà\\
ku-fwins-a\\
\textsc{inf}-seal-\textsc{fv}\\
\glt ‘to seal’
\z

\ea
\label{bkm:Ref98515130}
\glll cìfwìnsò\\
ci-fwins-o\\
\textsc{np}\textsubscript{7}-seal-\textsc{ins}\\
\glt ‘stopper’
\z

Alveolar fricatives followed by non-high vowels are also seen in the alternative pronunciation of grammatical prefixes with a post-alveolar fricative; some speakers of Namibian Fwe realize these as alveolar fricatives (see \sectref{bkm:Ref70695065} for examples).

\hspace*{-2.65pt}The labio-dental fricatives /f/ and /v/ are subject to even stronger co-occurrence restrictions; these phonemes can only be followed by a high back vowel /u/ or by the glide /w/, see (\ref{bkm:Ref98515162}--\ref{bkm:Ref98515164}).

\ea
\label{bkm:Ref98515162}
\glll màfútà\\
ma-futá\\
\textsc{np}\textsubscript{6}-oil\\
\glt ‘oil, lotion’
\z

\ea
\glll kùfwèbà\\
ku-fweb-a\\
\textsc{inf}-smoke-\textsc{fv}\\
\glt ‘to smoke’
\z

\ea
\glll vùmò\\
∅-vumo\\
\textsc{np}\textsubscript{5}-stomach\\
\glt ‘stomach’
\z

\ea
\label{bkm:Ref98515164}
\glll kùrívwàngà\\
ku-rí-vwang-a\\
\textsc{inf}-\textsc{refl}-wrap-\textsc{fv}\\
\glt ‘to put on a chitenge’
\z

For the labiodental fricatives too, a few exceptions are found where a labiodental fricative is fol\-lowed by a vowel other than /u/, which are mostly loanwords, as in (\ref{bkm:Ref98925572}--\ref{bkm:Ref98925574}).

\ea
\label{bkm:Ref98925572}
fônì (borrowed from English phone)\\
∅-fóni\\
\textsc{np}\textsubscript{5}-phone\\
\glt ‘phone’
\z

\ea
\label{bkm:Ref98925574}
cìfàtéhò (borrowed from Lozi \-sifateho ‘face’ \citep[54]{Burger1960})\\
ci-fatehó\\
\textsc{np}\textsubscript{7}-face\\
\glt ‘face’
\z

Another co-occurrence restriction concerns the velar stop /k/, which is not found with the high front vowel /i/. This is the result of the diachronic shift from *k to /c/ before /i/ \citep[118-119]{Bostoen2009}. One of the main exceptions is the reflexive prefix \textit{kí\nobreakdash-} (see \sectref{bkm:Ref451256199}), used in Zambian Fwe. Namibian Fwe uses a different reflexive prefix \textit{rí-}; in combination with the unexpected maintenance of /k/ before /i/, this suggests that the reflexive form \textit{kí-} in northern Fwe may be a borrowing from another Bantu language.

Clicks also appear to be subject to certain co-occurrence restrictions. Although click words are not common in Fwe, in 78 out of 84 click words collected the click is followed by a vowel /a/, /o/ or /u/. The only six click words in which clicks are followed by a front vowel /i/ or /e/ are listed in (\ref{bkm:Ref98925692})- (\ref{bkm:Ref448238192}); the two words in (\ref{bkm:Ref488832709}) and (\ref{bkm:Ref488832710}) may be borrowings from Yeyi, and the words listed in (\ref{bkm:Ref448238192}) appear to contain the same (ideophonic) root. With the exception of the Yeyi borrowing in (\ref{bkm:Ref488832709}), all cases of clicks followed by front vowels involve a voiced click.

\ea
\label{bkm:Ref98925692}
\glll ᵍǀìmà\\
∅-ᵍǀima\\
\textsc{np}\textsubscript{5}-fish\\
\glt ‘small fish sp.’
\z

\ea
\glll cìᵍǀìnjò\\
ci-ᵍǀinjo\\
\textsc{np}\textsubscript{7}-tree\\
\glt ‘tree sp.’
\z

\ea
\glll kùᵍǀìntùrà\\
ku-ᵍǀintur-a\\
\textsc{inf}-lie-\textsc{fv}\\
\glt ‘to lie with bent knees’
\z

\ea
\label{bkm:Ref488832709}
ⁿǀ̥íⁿǀ̥à (from Yeyi zĩǀĩǀa ‘fruits of the date palm tree’ \citep[28]{Seidel2008})\\
N-ⁿ{ǀ}̥iⁿ{ǀ}̥á \\
\textsc{np}\textsubscript{10}-date\\
\glt ‘dates’
\z

\ea
\label{bkm:Ref488832710}
 ᵍǀênè (from Yeyi nc’ene ‘thin’)\\
‘thin’
\z

\ea
\label{bkm:Ref448238192}
  ᵍǀí\\
  ‘sound of landing’
\z

\ea
\glll kùᵍǀínkìtà\\
ku-ᵍǀínkit-a\\
\textsc{inf}-pound-\textsc{fv}\\
\glt ‘to pound with short, sharp movements’
\z

\ea
\glll kùᵍǀìntà\\
ku-ᵍǀint-a\\
\textsc{inf}-hop-\textsc{fv}\\
\glt ‘to crash/fall down noisily; to hop up and down’
\z

Despite the low number of click words and the handful of counterexamples, there is thus a clear tendency for clicks to be followed by non-front vowels. Similar tendencies are observed in various Khoisan languages, where a Back Vowel Constraint (BVC) assimilates front vowels to back vowels when preceded by certain clicks \citep{Miller2011}. This only affects vowels preceded by labial, alveolar and lateral clicks, however, not vowels preceded by dental and palatal clicks. It is therefore surprising that Fwe shows such a strong preference for back vowels after clicks, as Fwe clicks are most commonly realized as dental. The preference of back vowels after clicks in Fwe could be the result of borrowing from languages such as Juǀ’hoan, where the BVC is active \citep{Miller2013}. Another possible explanation is that the modern variation in click type, with a preference for the dental, has not always existed, but that Fwe at an earlier stage had a preference for alveolar or lateral clicks, thus explaining the prevalence of back vowels after clicks, or even used alveolar and/or lateral clicks phonemically.

Co-occurrence restrictions on glides are also attested. Glides may be preceded by another conso\-nant; for the glide /w/, virtually all logically possible consonant-glide combinations are at\-tested. There are a few possible combinations that are not attested, such as /dw/, /ᵍǀw/, /ⁿǀw/ and /ⁿ{ǀ}̥w/. The absence of these combinations is probably the result of the low frequency of /d/, /ᵍǀ/, /ⁿǀ/ and /ⁿ{ǀ}̥/, and is unlikely to represent some underlying constraint on their co-occurrence with /w/, as /w/ does co-occur with other voiced stops, affricates and clicks, as shown in (\ref{bkm:Ref451515593}--\ref{bkm:Ref451515595}).

\ea
\label{bkm:Ref451515593}
\glll sìbbwê\\
∅-sibbwé\\
\textsc{np}\textsubscript{1a}-jackal\\
\glt ‘jackal’
\z

\ea
\glll kùgwà\\
ku-gw-a\\
\textsc{inf}-fall-\textsc{fv}\\
\glt ‘to fall’
\z

\ea
\glll bùcwàrà\\
bu-cwara\\
\textsc{np}\textsubscript{14}-beer\\
\glt ‘beer’
\z

\ea
\label{bkm:Ref451515595}
kùǀwámpìzà (variant of -ǀámpwìzà)\\
ku-ǀwámpiz-a\\
\textsc{inf}-click-\textsc{fv}\\
\glt ‘to click in anger’
\z

A consonant followed by /w/ is never followed by a back vowel /o/ or /u/. This constraint is likely to be related to the historical development of /w/, which derives from an earlier vowel /u/ or /o/, as in (\ref{bkm:Ref98925809}--\ref{bkm:Ref98925813}).

\ea
\label{bkm:Ref98925809}
èbwè (from *bʊè ‘stone’ \citep{BastinEtAl2002})\\
e-∅-bwe\\
\textsc{aug}-\textsc{np}\textsubscript{5}-stone\\
\glt ‘stone’
\z

\ea
kùkwâtà (from *kʊat ‘seize, grasp’ \citep{BastinEtAl2002})\\
ku-kwát-a\\
\textsc{inf}-touch-\textsc{fv}\\
\glt ‘to touch’
\z

\ea
\label{bkm:Ref98925813}
    kútwà (from *tó ‘stamp, pound, bite’ \citep{BastinEtAl2002})\\
ku-tw-á\\
\textsc{inf}-pound-\textsc{fv}\\
\glt ‘to pound’
\z

The vocalic origin of glides still has its effects on modern Fwe. As discussed in \sectref{bkm:Ref451506997}, vowels preceded by a consonant-glide combination are lengthened. This length\-ening may be interpreted as the effect of the length of the earlier vowel.

Combinations of a consonant with the glide /y/ also exist, though they may only involve the consonant /r/, in which case /r/ is realized as [l]. This is part of the same allophony that causes /r/ to be realized as [l] before the high front vowel /i/ (see also \sectref{bkm:Ref70695065}), because the palatal glide derives from an earlier vowel /i/. Exam\-ples of syllables with an onset /ry/ are given in (\ref{bkm:Ref98833257}--\ref{bkm:Ref98833259}).

\ea
\label{bkm:Ref98833257}
\glll kùryénkwètà\\
ku-ryénkwet-a\\
\textsc{inf}-bribe-\textsc{fv}\\
\glt ‘to bribe’
\z

\ea
\label{bkm:Ref98833259}
\glll shíryà\\
∅-shiryá\\
\textsc{np}\textsubscript{5}-other\_side\\
\glt ‘other/opposite side’
\z
\section{Morphophonology}
\hypertarget{Toc75352611}{}
In this section, I discuss a number of morphophonological processes: prenasalization, which mainly plays a role as a noun class prefix of class 9/10; vowel hiatus resolution, which affects juxtaposed vowels across morpheme boundaries, but also occasionally across word boundaries or within morphemes; and vowel and nasal harmony, which affect certain verbal suffixes.

\largerpage
\subsection{Prenasalization}
\label{bkm:Ref451507060}\hypertarget{Toc75352612}{}
As shown in \sectref{bkm:Ref435187046}, prenasalized consonants are part of the phoneme inventory of Fwe. In some cases, the homorganic nasal is a separate grammatical morpheme, which is discussed in this section.

A homorganic nasal functions as the nominal prefix of class 9/10 (see also \sectref{bkm:Ref489005545} on nominal prefixes). A comparison between nouns in class 9/10 and the same root in a different construction, such as a verb, or a noun in another noun class, allows for the identification of the underlying consonant and therefore also of the phonological effect of prenasalization.

When the class 9/10 nominal prefix \textit{N-} combines with a root where the initial consonant is a stop, the stop is prenasal\-ized, as illustrated in (\ref{bkm:Ref98925901}--\ref{bkm:Ref98925903}). This is the case for the voiceless stops /p/, /t/ and /k/, and probably also for the more peripheral voiced stop phonemes /bb, d, g/, though the number of examples is too limited to fully describe the behavior of voiced stops when prenasalized.

\ea
\label{bkm:Ref98925901}
ntòrókò (cf. kù-tóròk-à ‘to translate, explain’)\\
N-torokó\\
\textsc{np}\textsubscript{9}-meaning\\
\glt ‘meaning’
\z

\ea
nkáꜝmbámò (cf. kù-kámbàm-à ‘to ascend’)\\
N-kámbamó\\
\textsc{np}\textsubscript{9}-slope\\
\glt ‘upward slope’
\z

\ea
mpâkwà (cf. kú-pàk-à ‘carry on one’s back (of a child)’ + -w passive)\\
N-pákwa\\
\textsc{np}\textsubscript{9}-sling\\
\glt ‘sling’
\z

\ea
\label{bkm:Ref98925903}
\ea
\glll mbórà\\
N-bborá\\
\textsc{np}\textsubscript{9}-ball\\
\glt ‘ball’

\ex  cf.
èbbórà\\
e-∅-bborá\\
\textsc{aug}-\textsc{np}\textsubscript{5}-ball\\
\glt ‘ball’
\z\z

The effect of the prefix \textit{N-} on fricatives is more varied. The alveolar fricatives /s/ and /z/ be\-come /ns/ and /nz/, as in (\ref{bkm:Ref492028235}) and (\ref{bkm:Ref498095623}).

\ea
\label{bkm:Ref492028235}
nsúrùmùkò (cf. kù-súrùmùk-à ‘to descend’)\\
N-súrumuko\\
\textsc{np}\textsubscript{9}-slope\\
\glt ‘downward slope’
\z

\ea
\label{bkm:Ref498095623}
nzâsì (cf. class 11 rù-zâsì ‘spark’)\\
N-zási\\
\textsc{np}\-\textsubscript{10}-spark\\
\glt ‘sparks’
\z

The post-alveolar fricative /sh/ becomes /nsh/, but its voiced counterpart /zy/ changes from a fric\-ative to an affricate /j/ when combined with \textit{N-}.

\ea
nshíkà (cf. class 11 rú-ꜝshíkà ‘African mangosteen’)\\
N-shiká\\
\textsc{np}\textsubscript{10}-mangosteen\\
\glt ‘African mangosteens’
\z

\ea
njîmbò (cf. kù-zyîmb-à ‘to sing’)\\
N-jímbo\\
\textsc{np}\textsubscript{10}-song \\
\glt ‘songs’
\z

The bilabial fricative /b/ and the glottal fricative /h/ change to stops before \textit{N-}: the fricative /b/ be\-comes a prenasalized stop /mb/\footnote{As the bilabial fricative /b/ always changes to a stop before /m/, the prenasalized fricative is written as /mb/ in the practical orthography.} , as in (\ref{bkm:Ref98926071}), and fricative /h/ becomes a prenasalized stop /mp/, as in (\ref{bkm:Ref98926081}).

\ea
\label{bkm:Ref98926071}
mbèzyò (cf. kù-bèːzy-à ‘to carve’)\\
N-bezyo\\
NP\textsubscript{9}-axe\\
\glt ‘small axe (for carving)’
\z

\ea
\label{bkm:Ref98926081}
mpátì (cf. class 11 rù-hátì ‘rib’)\\
N-patí\\
\textsc{np}\textsubscript{10}-rib\\
\glt ‘ribs’
\z

The tap /r/ changes to a plosive /d/ before \textit{N-}, as in (\ref{bkm:Ref98926095}).

\ea
\label{bkm:Ref98926095}
ndúngàtì (cf. kù-rûngà ‘make noise’)\\
N-dúngati\\
\textsc{np}\textsubscript{9}-noise\\
\glt ‘noise’
\z

The combination of \textit{N-} with a vowel-initial root results in a prenasalized velar stop /ng/, as in (\ref{bkm:Ref98926175}--\ref{bkm:Ref98926178}). This mostly concerns stems that had an initial consonant /g/ originally, which is regularly lost in Fwe \citep[115]{Bostoen2009}. In one case, presented in (\ref{bkm:Ref98926206}), a vowel-initial stem takes /ny/ when used with a prefix \textit{N-}, even though this stem, too, is a reflex of a stem reconstructed with *g.

\ea
\label{bkm:Ref98926175}
\ea
ngômà (from *gòmà ‘drum’ \citep{BastinEtAl2002})\\
N-góma\\
\textsc{np}\textsubscript{9}-drum\\
\glt ‘drum’

\ex
màômà\\
ma-óma\\
\textsc{np}\textsubscript{6}-drum\\
\glt ‘drums’

\z\z

\ea
\label{bkm:Ref98926178}
\ea
ngùrìsò (from *g\`{ʊ}d ‘buy’ \citep{BastinEtAl2002})\\
N-guriso      \\
\textsc{np}\textsubscript{9}-profit\\  
\glt ‘profit’      

\ex
kùùrìsà\\
ku-uris-a\\
\textsc{inf}-sell-\textsc{fv}\\
\glt ‘to sell’

\z\z

\ea
\label{bkm:Ref98926206}
\ea
nyózì (from *-gòdí ‘string’ \citep{BastinEtAl2002})\\
ny-ozí\\
\textsc{np}\textsubscript{10}-plant \\
\glt ‘plants (used for making ropes)’

\ex
rózì\\
ru-ozí\\
\textsc{np}\textsubscript{11}-plant \\
\glt ‘plant (used for making ropes)’
  
\z\z

The rules in (\ref{bkm:Ref98926278}) summarize the changes to root-initial phonemes caused by the prefix \textit{N-}.

\ea
\label{bkm:Ref98926278}
  zy $\to$ nj / N{\longrule}\\
  b $\to$ bb / N{\longrule}\\
  h $\to$ p / N{\longrule}\\
  r $\to$ d / N{\longrule}\\
  ∅ $\to$ g / N{\longrule}
\z

Interestingly, while nouns that shift from class 9/10 to another class for derivational purposes lose their nasal prefix, the realization of the initial conso\-nant remains plosive, and does not change back to a fricative or tap. This is shown in (\ref{bkm:Ref506827044}) with the noun \textit{mpúzò} ‘ques\-tion’, which shifts to class 12 to derive a diminutive: the homorganic nasal prefix of class 9 is lost, but the consonant /p/ does not change back to /h/.

\ea
\label{bkm:Ref506827044}
\ea
\glll mpúzò\\
N-puzó\\
\textsc{np}\textsubscript{9}-question\\
\glt ‘question’

\ex
\glll kàpúzò\\
ka-puzó\\
\textsc{np}\textsubscript{12}-question\\
\glt ‘small question’
\z\z

Although the modern form of the first person singular subject and object marker is a syllabic prefix \textit{ndi\nobreakdash-}, there are also traces of an earlier first person singular object \textit{N-} prefix. The form with the homorganic nasal in (\ref{bkm:Ref459893930}) was offered by speakers as “archaic Fwe”, in contrast to the modern form with \textit{ndi-}. Another petrified trace of a first person singular object marker \textit{N-} is seen in the per\-sonal name \textit{Mùngúríkè} in (\ref{bkm:Ref459893837}).

\ea
\label{bkm:Ref459893930}
\ea
Archaic form\\
ntámbìkè\\
N-támbik-e\\
\textsc{om}\textsubscript{1SG}-give-\textsc{pfv}.\textsc{sbjv}\\
\glt ‘Give me.’

\ex
Modern form\\
ndìtámbìkè\\
ndi-támbik-e\\
\textsc{om}\textsubscript{1SG}-give-\textsc{pfv}.\textsc{sbjv}\\
\glt ‘Give me.’
\z\z

\ea
\label{bkm:Ref459893837}
\ea
  Mùngúríkè\\
  ‘Mungurike (boy’s name)’

\ex
Putative historic source\\
mùngúríkè\\
mu-ng-urí̲k-e\\
\textsc{sm}\textsubscript{2PL}-\textsc{om}\textsubscript{1SG}-name-\textsc{pfv}.\textsc{sbjv}\\
\glt ‘Name me.’

\ex
Corresponding modern verb\\
kùùrìkà \\
ku-urik-a\\
\textsc{inf}-name-\textsc{fv}\\
\glt ‘to name’
\z\z

One of the two forms of the copulative prefix also consists of a homorganic nasal prefix; its phonological interaction with the pho\-nemes it attaches to is different from what is described in this section, and is treated in detail in \sectref{bkm:Ref489963307} on copulas.

\subsection{Vowel hiatus resolution}
\label{bkm:Ref491962181}\hypertarget{Toc75352613}{}
Sequences of two adjacent vowels are found within morphemes, across morpheme boundaries, and across word boundaries. Fwe often, but not always, applies vowel hiatus resolution strategies to resolve such sequences. Which strategy, if any, is used, depends on the morpheme in question, and is also partly lexically determined. This section discusses the various ways Fwe deals with vowel juxtaposition.

\subsubsection{Maintenance of both vowels}

As Fwe allows for syllables without a consonantal onset, one of the strategies applied to juxta\-posed vowels is to maintain both vowels without any changes. This occurs, for instance, when a verbal prefix of CV shape is added to a vowel-initial verb root, in which case both vowels are maintained unchanged. Exam\-ples with different verbal prefixes are given in (\ref{bkm:Ref451591068}), using the vowel-initial verb root \textit{ur} ‘buy’.

\ea
\label{bkm:Ref451591068}
Infinitive\\
/ku-ur-a/ > kùùrà\\
\textsc{inf}-buy-\textsc{fv}\\
\glt ‘to buy’
\z

\ea
Subject marker\\
/ndi-ur-á̲/ > ndìúrà\\
\textsc{sm}\textsubscript{1SG}-buy-\textsc{fv}\\
\glt ‘I buy.’
\z

\ea
Object marker\\
/ku-í-ur-a/ > kùyíùrà\\
\textsc{inf}-\textsc{om}\textsubscript{9}-buy-\textsc{fv}\\
\glt ‘to buy it’
\z

\newpage
\ea
TA marker\\
/ndi-na-ur-í̲/ > ndìnàúrì\\
\textsc{sm}\textsubscript{1SG}-\textsc{pst}-buy-\textsc{npst}.\textsc{pfv}\\
\glt ‘I bought.’
\z

\ea
Distal marker\\
/ndi-a-ka-ur-í̲/ > ndàkàúrì\\
\textsc{sm}\textsubscript{1SG}-\textsc{pst}-\textsc{dist}-buy-\textsc{npst}.\textsc{pfv}\\
\glt ‘I bought there.’
\z

Maintenance of both juxtaposed vowels may also occur when a nominal pre\-fix of CV shape directly precedes a vowel-initial nominal stem, as in (\ref{bkm:Ref459222489}--\ref{bkm:Ref98926448}). Changes to one of the two vowels is also common in this case however, as discussed in the following sections.

\ea
\label{bkm:Ref459222489}
/mu-ono/ > mùònò\\
\textsc{np}\textsubscript{3}-snoring\\
\glt ‘snoring’
\z

\ea
/ka-inga/ > kàìngà\\
\textsc{np}\textsubscript{12}-bowl\\
\glt ‘bowl made out of clay’
\z

\ea
\label{bkm:Ref98926448}
/mi-ézi/ > mìêzì\\
\textsc{np}\textsubscript{4}-month\\
\glt ‘months’
\z

Maintenance of two juxtaposed vowels can also occur in other environments, such as a verb root ending in a vowel that is followed by a vowel-initial suffix, as in (\ref{bkm:Ref98926480}).

\ea
\label{bkm:Ref98926480}
/ku-bbu-a/ > kùbbùà\\
\textsc{inf}-swim-\textsc{fv}\\
\glt ‘to swim, splash about’
\z

Two adjacent vowels can also be maintained unchanged when they occur within a single lexical root, as in (\ref{bkm:Ref98927353}--\ref{bkm:Ref98927355}).

\ea
\label{bkm:Ref98927353}
/N-daano/ > ndàànò\\
\textsc{np}\textsubscript{9}-message\\
\glt ‘message’
\z

\ea
/N-júo/ > njûò\\
\textsc{np}\textsubscript{9}-house\\
\glt ‘house’
\z

\ea
\label{bkm:Ref98927355}
/N-bao/ > mbàò\\
\textsc{np}\textsubscript{9}-bird\\
\glt ‘bird sp.’
\z

In many cases where maintenance of two juxtaposed vowels is possible, an alternative strategy for maintenance of both vowels is consonant epenthesis (discussed below). Maintenance of both vowels without any changes is particularly common when the two juxtaposed vowels are identical, as in (\ref{bkm:Ref98927416}--\ref{bkm:Ref98927420}).

\ea
\label{bkm:Ref98927416}
/ma-amba/ > mààmbà\\
\textsc{np}\textsubscript{6}-scale\\
\glt ‘scales (of a fish)’
\z

\ea
/ku-zíiz-a/ > kùzíìzà\\
\textsc{inf}-imitate-\textsc{fv}\\
\glt ‘to imitate’
\z

\ea
/ku-teen-a/ > kùtèènà\\
\textsc{inf}-limp-\textsc{fv}\\
\glt ‘to limp’
\z

\ea
\label{bkm:Ref98927420}
/ku-uru/ > kùùrù\\
\textsc{np}\textsubscript{15}-leg\\
\glt ‘leg’
\z

Another possible realization of two juxtaposed vowels is deletion of the first vowel. This often takes place when vowel-initial nominal roots are combined with a nominal prefix ending in a vowel (for an overview of nominal prefixes, see \sectref{bkm:Ref489005545}). Nominal prefixes con\-sist of a consonant followed by a vowel /i/, /a/ or /u/. When a nominal prefix with /i/ or /a/ is combined with a vowel-initial nominal root, the vowel of the nominal prefix can be deleted, as in (\ref{bkm:Ref506807248}--\ref{bkm:Ref451591190}).

\ea
\label{bkm:Ref506807248}
/ci-úngu/ > cûngù \\
\textsc{np}\textsubscript{7}-bird\\
\glt ‘bird sp. (with a red tail)’
\z

\ea
\label{bkm:Ref451591190}
/ma-ató/ > mátò\\
\textsc{np}\textsubscript{6}-canoe\\
\glt ‘canoes’
\z

Not all vowel-initial roots cause the vowel of the preceding nominal prefix to be deleted; mainte\-nance of the vowel is also possible, and which strategy applies is lexically determined, al\-though maintenance is more common than deletion. Deletion of one of two juxtaposed vowels does not lead to compensatory lengthening of the remaining vowel.

Deletion of the first of the two vowels also ocurs when a subject prefix, which is always of (C)V shape (see \sectref{bkm:Ref451511047} on subject agreement), is combined with a vowel-initial verbal prefix, such as the past prefix \textit{a-} in (\ref{bkm:Ref98927493}) and (\ref{bkm:Ref98927527}), or the remote future prefix \textit{ára-} in (\ref{bkm:Ref98927513}).

\ea
\label{bkm:Ref98927493}
/ndi-a-có̲ːr-i > ndàcôːrì\\
\textsc{sm}\textsubscript{1SG}-\textsc{pst}-break-\textsc{npst}.\textsc{pfv}\\
\glt ‘I broke.’
\z

\ea
\label{bkm:Ref98927527}
/ní̲-ba-a-ráːr-a/ >níbàráːrà\\
\textsc{rem}-\textsc{sm}\textsubscript{2}-\textsc{pst}-sleep-\textsc{fv}\\
\glt ‘They went to sleep.’
\z

\ea
\label{bkm:Ref98927513}
/ndi-ára-end-a/ > ndáràyèndà\\
\textsc{sm}\textsubscript{1SG}-\textsc{rem}.\textsc{fut}-go-\textsc{fv}\\
\glt ‘I will go.’
\z
\subsubsection{Glide formation}

Glide formation to [w] can take place when the first of two juxtaposed vowels is a back vowel /u/ or /o/, but never when the second vowel is also a back vowel; in this case, the first vowel is deleted, or both vowels are maintained. Glide formation to [y] occurs when the first of two juxtaposed vowels is a front vowel /i/ or /e/. Glide formation is always accompanied by lengthening of the following vowel (see \sectref{bkm:Ref459132421} on phonetic vowel lengthening).

Glide formation to [w] occurs in subject markers with /u/ or /o/, as in (\ref{bkm:Ref98833333}).

\ea
\label{bkm:Ref98833333}
  Glide formation to [w] in subject markers

\ea
ni-tú̲-a-rim-a > nìtwárìmà\\
\textsc{rem}-\textsc{sm}\textsubscript{1PL}-\textsc{pst}-farm-\textsc{fv}\\
\glt ‘We farmed.’

\ex
o-ásha-ndi-dam-í̲ > wáshàndìdámì\\
\textsc{sm}\textsubscript{2SG}-\textsc{neg}.\textsc{sbjv}-\textsc{om}\textsubscript{1SG}-beat-\textsc{npst}.\textsc{pfv}\\
\glt ‘Don’t beat me!’

\ex
bu-a-hík-iw-a > bwàhíkìwà\\
\textsc{sm}\textsubscript{14}-\textsc{pst}-cook-\textsc{pass}-\textsc{fv}\\
\glt ‘It [relish] is cooked.’ (NF\_Elic15)
\z\z

Glide formation to [y] affects subject markers that contain a vowel /i/, but only those of class 4 (\textit{i-}), 5 (\textit{ri-}), and 9 (\textit{i-}).

\ea
  Glide formation to [y] in subject markers i-, ri-

\ea
/i-a-có̲ːk-i/ > yàcôːkì\\
\textsc{sm}\textsubscript{4}-\textsc{pst}-break-\textsc{npst}.\textsc{pfv}\\
\glt ‘They (pot legs) are broken.’ (NF\_Elic17)

\ex
/ri-a-zyón-a-uk-i/ > ryàzyónàùkì\\
\textsc{sm}\textsubscript{5}-\textsc{pst}-destroy-\textsc{pl}1-\textsc{sep}.\textsc{intr}-\textsc{npst}.\textsc{pfv}\\
\glt ‘It (field) is destroyed.’ (ZF\_Elic13)

\ex
/i-ára-dur-a/ > yáràdùrà\\
\textsc{sm}\textsubscript{9}-\textsc{rem}.\textsc{fut}-be\_expensive-\textsc{fv}\\
\glt ‘It will be expensive.’ (NF\_Elic15)
\z\z

Other subject markers with /i/, namely \textit{ndi-} (first person singular), \textit{ci-} (class 7), and \textit{zi-} (class 8/10), never undergo glide formation, as illustrated in (\ref{bkm:Ref98927902}).

\ea
\label{bkm:Ref98927902}
  No glide formation to [y] in subject markers ndi-, ci-, zi-

\ea
/ndi-a-pwac-ú̲r-i/ > ndàpwàcûrì\\
\textsc{sm}\textsubscript{1SG}-\textsc{pst}-break-\textsc{sep}.\textsc{tr}-\textsc{npst}.\textsc{pfv}\\
\glt ‘I broke.’

\ex
/ci-á-zyur-i/ > cázyùrì\\
\textsc{sm}\textsubscript{7}-\textsc{pst}-become\_full-\textsc{npst}.\textsc{pfv}\\
\glt ‘It is full.’

\ex
/zi-a-ndi-bús-i/ > zàndìbûsì\\
\textsc{sm}\textsubscript{8}-\textsc{pst}-\textsc{om}\textsubscript{1SG}-wake-\textsc{npst}.\textsc{pfv}\\
\glt ‘They woke me up.’
\z\z

The reason for this conditioning of y-formation is that only /ry/ and /y/ occur phonemically in Fwe, and sequences such as /ndy/, /cy/ and /zy/ (not to be confused with <zy>, representing the voiced postalveolar fricative [ʒ]), are not found in the phonology.

Glide formation to [w] occurs when a nominal prefix with /u/ is combined with a vowel-initial root, as in (\ref{bkm:Ref98928065}). Glide formation to [y] does not affect nominal prefixes with /i/, even when combined with a vowel-initial root, as in (\ref{bkm:Ref98928088}).

\ea
\label{bkm:Ref98928065}
  Glide formation to [w] in nominal prefixes with /u/

\ea
/mu-ánce/ > mwâncè\\
\textsc{np}\textsubscript{1}-child\\
\glt ‘a child’

\ex
/mu-iní/ > mwínì\\
\textsc{np}\textsubscript{3}-handle\\
\glt ‘handle’

\ex
/bu-eké/ > bwékè\\
\textsc{np}\textsubscript{14}-grain\\
\glt ‘grains’

\ex
/ru-áta/ > rwâtà\\
\textsc{np}\textsubscript{11}-crack\\
\glt ‘crack’
\z\z

\ea
\label{bkm:Ref98928088}
  No glide formation to [y] in nominal prefixes with /i/

\ea
/mi-áka/ >mìâkà \\
\textsc{np}\textsubscript{4}-year\\
\glt ‘years’

\ex
/ci-ánda/ > cândà\\
\textsc{np}\textsubscript{7}-pole\\
\glt ‘pole’

\ex
/zi-ongo/ > zìòngò\\
\textsc{np}\textsubscript{8}-storage\\
\glt ‘storage huts’
\z\z

When a high-toned vowel is changed to a glide, the high tone is maintained and realized on the adjacent vowel. This is shown with the high-toned subject markers \textit{ú-} in (\ref{bkm:Ref98928123}) and \textit{í-} in (\ref{bkm:Ref98928138}); when these vowels are changed to glides, their high tones are realized on the following vowels.

\ea
\label{bkm:Ref98928123}
ni-ú̲-a-rih-iw-a > nìwárìhìwà\\
\textsc{rem}-\textsc{sm}\textsubscript{3}-\textsc{pst}-pay-\textsc{pass}-\textsc{fv}\\
\glt ‘It has been paid.’ (NF\_Elic15)
\z

\ea
\label{bkm:Ref98928138}
ni-í̲-a-hond-iw-a > nìyáhòndìwà\\
\textsc{rem}-\textsc{sm}\textsubscript{9}-\textsc{pst}-cook-\textsc{pass}-\textsc{fv}\\
\glt ‘It has been cooked.’
\z

Glide formation also occurs across word boundaries, as in (\ref{bkm:Ref459220701}), where the final vowel /u/ of \textit{ndùndávú} is changed to a glide under influence of the initial vowel of the following word.

\ea
\label{bkm:Ref459220701}
\gll ndu-∅-ndavú  á̲-shá̲mb-a > [ndùndávw’ áshâmbà]\\
\textsc{cop}\textsubscript{1a}-\textsc{np}\textsubscript{1a}-lion  \textsc{sm}\textsubscript{1}.\textsc{rel}-swim-\textsc{fv}\\
\glt ‘It’s a lion who swims.’
\z

Glide formation across word boundaries is transcribed in the phonetic transcription with an apos\-trophe after the glide. In the phonological transcription, the underlying vowel is transcribed.

\subsubsection{Vowel coalescence}

Another vowel hiatus resolution strategy is vowel coalescence, the merger of the two juxtaposed vowels into a third vowel that combines properties of both. It often combines with glide formation if the first vowel is a back vowel /u/ or /o/. It does not lead to lengthening, except when vowel coalescence com\-bines with glide formation.

Word-internally, vowel coalescence is rare, found only in Namibian Fwe in certain construc\-tions where a prefix with a vowel /u/, such as the class 17 prefix \textit{ku-}, is used with a noun that has an augment prefix \textit{e-}, as in (\ref{bkm:Ref98928371}). The resultant sequence /ku + e/ is realized as /kwi/, where the high back vowel /u/ changes to a glide, and the vowel /i/ combines the height property of /u/ with the front property of /e/.

\ea
\label{bkm:Ref98928371}
kú-e-∅-ténde > [kwítêndè]\\
\textsc{np}\textsubscript{17}-\textsc{aug}-\textsc{np}\textsubscript{5}-leg\\
\glt ‘on the leg’
\z

Vowel coalescence is more common across word boundaries, when a vowel-initial word is pre\-ceded by another word which, due to the strictly open syllable structure of Fwe, invariably ends in a vowel. In this context, /i/ can coalesce with /o/ to become the vowel /u/, which carries the height feature of /i/ combined with the back fea\-ture of /o/, as in (\ref{bkm:Ref451515613}). Vowel coalescence is represented in the phonetic transcription with an apostrophe in place of the lost vowel, similar to the representation of vowel deletion.

\ea
\label{bkm:Ref451515613}
\label{bkm:Ref97731851}
\gll ndi-kwesí   o-∅-mbwá > [ndìkwès’ ûmbwà]\\
\textsc{sm}\textsubscript{1SG}-have  \textsc{aug}-\textsc{np}\textsubscript{1a}-dog\\
\glt ‘I have a dog.’ (ZF\_Elic14)
\z

When /u/ coalesces with /e/, both vowel coalescence and glide formation take place: /u/ is changed to a glide [w], and the vowel /e/ is raised to /i/, combined the height feature of /u/ with the front feature of /e/.

\ea
\gll e-zi-ntú     e-zo > [èzìntw’ ízò]\\
\textsc{aug}-\textsc{np}\textsubscript{7}-things  \textsc{aug}-\textsc{dem}.\textsc{iii}\textsubscript{8}\\
\glt ‘the things, that…’
\z

Vowel coalescence is not observed in all cases of vowel juxtaposition across word boundaries. Compare (\ref{bkm:Ref97731812}), where there is no vowel coalescence between the final vowel of \textit{kwesi} ‘have’ and the initial vowel of \textit{oburotu} ‘something good’, with (\ref{bkm:Ref97731851}), where vowel coalescence between the final vowel of \textit{kwesi} ‘have’ and the intial vowel of \textit{ombwa} ‘dog’ does take place.

\ea
\label{bkm:Ref97731812}
ècìntù nècìntù cìkwèsì òbùrótù nòbúbbì\\
\gll e-ci-ntu    ne=ci-ntu    ci-kwesi  o-bu-rótu no=bu-bbí   \\
\textsc{aug}-\textsc{np}\textsubscript{7}-thing  \textsc{com}=\textsc{np}\textsubscript{7}-thing  \textsc{sm}\textsubscript{7}-have  \textsc{aug}-\textsc{np}\textsubscript{14}-good \textsc{com}=\textsc{aug}-\textsc{np}\textsubscript{14}-bad\\
\glt ‘Everything has an advantage and a disadvantage.’ (ZF\_Conv13)
\z
\subsubsection{Consonant epenthesis}

Finally, vowel hiatus may be resolved by an epenthetic consonant, [h], [y] or [w]. This process only occurs word-internally. Consonant epenthesis is optional; in any context where epenthetic consonants may occur, they may also be left out, as in (\ref{bkm:Ref506809953}), which shows that epenthetic [h] is optional.

\ea
\label{bkm:Ref506809953}
kùàmbàhàmbà {\textasciitilde} kùàmbààmbà\\
ku-amba-amb-a\\
\textsc{inf}-\textsc{pl}2-talk-\textsc{fv}\\
\glt ‘to talk a lot’
\z

The palatal glide [y] can be inserted when the first or the second juxtaposed vowel is the front vowel /i/, as in (\ref{bkm:Ref459209872}), or /e/, as in (\ref{bkm:Ref459209875}). It is also occasionally used as an epenthetic consonant between /a/ and /a/, espe\-cially in Zambian Fwe, as seen in (\ref{bkm:Ref506810289}).

\ea
\label{bkm:Ref459209872}
/mi-áni/ > mìyânì\\
\textsc{np}\textsubscript{4}-mopane\\
\glt ‘mopane trees’
\z

\ea
\label{bkm:Ref459209875}
/ku-bíraer-a/ > kùbíràyèrà\\
\textsc{inf}-complain-\textsc{fv}\\
\glt ‘to complain’
\z

\ea
\label{bkm:Ref506810289}
/kú-ya-a/ > kúyàyà\\
\textsc{inf}-kill-\textsc{fv}\\
\glt ‘to kill’
\z

The labial glide [w] can be inserted when the first of the juxtaposed vowels is a back vowel /o/, as in (\ref{bkm:Ref98928593}), or /u/, as in (\ref{bkm:Ref98928595}).

\ea
\label{bkm:Ref98928593}
/ku-ko-a/ > kùkòwà\\
\textsc{inf}-blink-\textsc{fv}\\
\glt ‘to blink’
\z

\ea
\label{bkm:Ref98928595}
/N-kúa/ > nkûwà\\
\textsc{np}\textsubscript{9}-tick\\
\glt ‘tick’
\z

[h] can be used as an epenthetic consonant between any two vowels. As such it is often used as a substitute for either [w], as in (\ref{bkm:Ref98928623}), or [y], as in (\ref{bkm:Ref98928625}), and is also often inserted in contexts where [w] or [y] usually do not occur, such as between /a/ and /a/ in (\ref{bkm:Ref506810235}).

\ea
\label{bkm:Ref98928623}
/ku-ko-a/ > kùkòwà {\textasciitilde} kùkòhà\\
\textsc{inf}-blink-\textsc{fv}\\
\glt ‘to blink’
\z

\ea
\label{bkm:Ref98928625}
/N-peó/ > mpéyò {\textasciitilde} mpéhò\\
\textsc{np}\textsubscript{9}-cold\\
\glt ‘cold, malaria’
\z

\ea
\label{bkm:Ref506810235}
/a-a\textsubscript{H}mb-a/ > àhâmbà\\
\textsc{sm}\textsubscript{1}-speak-\textsc{fv}\\
\glt ‘S/He\footnote{As agreement markers of class 1 refer to a singular human being and do not express biological sex, examples such as this can be translated to English with ‘he’ or ‘she’. I use ‘s/he’ or ‘her/him’ in the translation of elicited examples. In natural text examples, and elicited examples where the referent is known through the context, ‘he’ and ‘she’ will be used as appropriate.}  is speaking.’
\z

Epenthetic [h] should not be confused with phonemic /h/ (see also \sectref{bkm:Ref70695065}), which can never be dropped nor realized as a glide [y] or [w]. Fur\-thermore, phonemic /h/ can be pronounced with slight nasalization, which is never the case with epenthetic [h]. In (\ref{bkm:Ref459213206}), examples of epenthetic [h] are given, which are contrasted with examples of phonemic /h/ in (\ref{bkm:Ref459213207}).

\ea
\label{bkm:Ref459213206}
 Epenthetic [h]

\ea
/ci-uru/ > cìùrù {\textasciitilde} cìwùrù {\textasciitilde} cìhùrù\\
\textsc{np}\textsubscript{7}-hill\\
\glt ‘hill’\\
*cìhṵrù

\ex
/bu-fwíi/ > bùfwîì {\textasciitilde} bùfwîyì {\textasciitilde} bùfwîhì\\
\textsc{np}\textsubscript{14}-short\\
\glt ‘shortness’\\
*bùfwîhḭ̀
\z\z

\ea
\label{bkm:Ref459213207}
  Phonemic /h/

\ea
/bu-háro/ > bùhârò {\textasciitilde} bùhâ̰rò\\
\textsc{np}\textsubscript{14}-life\\
\glt ‘life’\\
*bùwârò \\
*bùârò

\ex
\gll /ku-hík-a/ > kùhîkà {\textasciitilde} kùhḭ̂kà\\
\textsc{inf}-cook-\textsc{fv}\\
\glt ‘to cook’\\
*kùîkà\\
*kùyîkà
\z\z

Consonant epenthesis occurs in a variety of contexts. It can occur mor\-pheme-internally, for instance, in a lexical root as in (\ref{bkm:Ref459215554}). It can also occur across a morpheme boundary, where vowel juxtaposition is the result of the addition of a prefix or suffix, as seen in (\ref{bkm:Ref70695202}--\ref{bkm:Ref70695203}).

\ea
\label{bkm:Ref459215554}
/ma-roa/ > màròhà {\textasciitilde} màròwà\\
\textsc{np}\textsubscript{6}-blood\\
\glt ‘blood’
\z

\ea
\label{bkm:Ref70695202}
\gll /ma-ira/ > màyìrà {\textasciitilde} màhìrà\\
\textsc{np}\textsubscript{6}-sorghum\\
\glt ‘sorghum’
\z

\ea
\label{bkm:Ref70695203}
\gll /e-N-swí-ana/ > ènswíyànà\\
\textsc{aug}-\textsc{np}\textsubscript{10}-fish-\textsc{dim}\\
\glt ‘small fish’
\z
\subsection{Vowel harmony}
\label{bkm:Ref451863900}\hypertarget{Toc75352614}{}
Fwe has two related processes of vowel height harmony that affect a number of verbal deriva\-tional suffixes, as well as one inflectional suffix, the stative -\textit{ite}. Front vowel harmony lowers /i/ in verbal suffixes to /e/ when preceded by the mid vowel /e/ or /o/; in all other cases, the vowel remains /i/. This affects causative -\textit{is}, as in (\ref{bkm:Ref98928885}), applicative -\textit{ir}, as in (\ref{bkm:Ref98928900}), transitive impositive \textit{-ik}, as in (\ref{bkm:Ref98928913}), epenthetic causative/applicative \textit{-ik}, as in (\ref{bkm:Ref98928937}), and stative \textit{-ite}, as in (\ref{bkm:Ref98928980}).

\ea
\label{bkm:Ref98928885}
  Vowel harmony affecting the causative \textit{-is}

{\itshape kù-fúm-ìs-à} \tab  ‘to make rich’\\
{\itshape kù-bìr-ìs-à} \tab  ‘to bring to a boil’\\
{\itshape kú-kàr-ìs-à} \tab  ‘to sit with someone’\\
{\itshape kù-shèk-ès-à} \tab  ‘to make laugh’\\
{\itshape kù-gòr-ès-à} \tab  ‘to make strong, insist’\\
\z


\ea
\label{bkm:Ref98928900}
  Vowel harmony affecting the applicative \textit{-ir}

{\itshape kù-bútùk-ìr-à} \tab  ‘to run to’\\
{\itshape kù-zyímb-ìr-à} \tab  ‘to sing for’\\
{\itshape kù-kwát-ìr-à} \tab  ‘to hold for’\\
{\itshape kù-tènd-èr-à} \tab  ‘to do for’\\
{\itshape kù-shótòk-èr-à} \tab  ‘to jump into’\\
\z

\ea
\label{bkm:Ref98928913}
  Vowel harmony affecting the transitive impositive \textit{-ik}

{\itshape kù-fúrùm-ìk-à} \tab  ‘to place upside down’\\
{\itshape kù-fwí-ìk-à} \tab  ‘to approach’\\
{\itshape kù-cànk-ìk-à} \tab  ‘to put a pot on the fire’\\
{\itshape kù-nyòng-èk-à} \tab  ‘to bend’\\
{\itshape kù-kór-èk-à} \tab  ‘to carry on the shoulder’\\
\z

\ea
\label{bkm:Ref98928937}
  Vowel harmony affecting the epenthetic causative/applicative \textit{-ik}

{\itshape kù-búːs-ìk-ìz-à} \tab  ‘to wake up for’\\
{\itshape kù-zìm-ìs-ìk-ìz-à} \tab  ‘to extinguish for’\\
{\itshape kù-kác-ìk-ìz-à} \tab  ‘to interrupt’\\
{\itshape kù-cèn-ès-èk-èz-à} \tab  ‘to clean for’\\
{\itshape kù-nyòns-èk-èz-à} \tab  ‘to nurse for’\\
\z

\ea
\label{bkm:Ref98928980}
  Vowel harmony affecting the stative \textit{-ite}

{\itshape ndì-fúm-îtè} \tab  ‘I am rich.’\\
{\itshape ò-bízw-îtè} \tab  ‘It is ripe.’\\
{\itshape ndì-kwáng-îtè} \tab  ‘I am tired.’\\
{\itshape ndì-shésh-êtè} \tab  ‘I am married.’\\
{\itshape cì-bór-êtè} \tab  ‘It is rotten.’\\
\z

Vowel height harmony does not affect the passive suffix \textit{-(i)w}, as seen in (\ref{bkm:Ref70695533})m even though, like other derivational suffixes affected by vowel height harmony, it also contains a high front vowel /i/.

\ea
\label{bkm:Ref70695533}
  No vowel harmony affecting the passive \textit{-iw}

{\itshape kù-shúm-ìw-à} \tab  ‘to be bitten’\\
{\itshape kù-rìh-ìw-à} \tab  ‘to be paid’\\
{\itshape kù-sànz-ìw-à} \tab  ‘to be washed’\\
{\itshape kù-tém-ìw-à} \tab  ‘to be chopped’\\
{\itshape kù-hònd-ìw-à} \tab  ‘to be cooked’\\
\z

Vowel harmony is only triggered by the vowel of the syllable immediately preceding the suf\-fix, which can be part of the verb root or of a different derivational suffix. This means that a mid vowel in the verb root does not trigger vowel harmony a suffix with a low or high vowel intervenes, such as the transitive separative suffix \textit{-uk} in (\ref{bkm:Ref448247755}).

\ea
\label{bkm:Ref448247755}
\glll zìcèrúkìtè\\
zi-cer-ú̲k-ite\\
\textsc{sm}\textsubscript{8}-tear-\textsc{sep}.\textsc{intr}-\textsc{stat}\\
\glt ‘They are torn.’
\z

Although vowel harmony is blocked by intervening low or high vowels, in a sequence of adjacent suf\-fixes that are susceptible to vowel harmony, vowel harmony applies up to the last suffix, as shown by the combination of applicative and causative in (\ref{bkm:Ref70695812}).

\ea
\label{bkm:Ref70695812}
\glll kùcènèsèrà\\
kù-cèn-ès-èr-à\\
\textsc{inf}-clean-\textsc{caus}-\textsc{appl}-\textsc{fv}\\
\glt ‘to clean for’
\z

Fwe has borrowed verbs from Lozi, a neighboring Bantu language that lacks vowel harmony, and where the causative is invariably realized as \textit{-is} and the applicative as \textit{-el}. In some of these Lozi borrowings, such as those in (\ref{bkm:Ref98929082}) and (\ref{bkm:Ref98929085}), the rules of vowel harmony do not apply as they do to native Fwe verbs, suggesting these were borrowed from Lozi as complex verbs which include a derivational suffix. This is supported by the fact that many borrowed Lozi verbs only occur with the derivational suffix, and never without it.

\ea
\label{bkm:Ref98929082}
kùràtèrèrà (borrowed from Lozi ku latelela ‘to follow’)\\
ku-rat-er-er-a\\
\textsc{inf}-follow-\textsc{int}-\textsc{fv}\\
\glt ‘to follow’\\
*kùràtà
\z

\ea
\label{bkm:Ref98929085}
kùsèpìsà (borrowed from Lozi ku sepisa ‘to promise’)\\
ku-sep-is-a\\
\textsc{inf}-trust-\textsc{caus}-\textsc{fv}\\
\glt ‘to promise’
\z

Some borrowed Lozi verbs occur either with or without a derivational suffix in Fwe. In these cases, the Fwe rules of vowel harmony do apply to the suffix, as in (\ref{bkm:Ref69985878}).

\ea
\label{bkm:Ref69985878}
\ea
kùpângà (borrowed from Lozi ku panga ‘construct (a wooden frame)’)\\
ku-páng-a\\
\textsc{inf}-do-\textsc{fv}\\
\glt ‘to do, make’

\ex
\glll kùpángìrà\\
ku-páng-ir-a\\
\textsc{inf}-do-\textsc{appl}-\textsc{fv}\\
\glt ‘to do for (someone)’

\ex
  *kùpángèrà
\z\z

The form of suffixes displaying vowel harmony is slightly different in verbs with a monosyllabic root. As \tabref{tab:2:3} shows, monosyllabic verb roots that consist of a consonant-glide combination always take the \textit{i-} form of the suffix.

\begin{table}
\label{bkm:Ref98929229}\caption{\label{tab:2:3}Vowel height harmony in CG verb roots}
\begin{tabular}{*4{l}}
\lsptoprule
{\itshape kú-tw-à} & ‘to pound’ & {\itshape kù-tw-îr-à} & ‘to be pounded’\\
{\itshape kù-gw-à} & ‘to fall’ & {\itshape kù-gw-ìs-à} & ‘to drop’\\
{\itshape kú-nyw-à} & ‘to drink’ & {\itshape à-nyw-ìtè} & ‘S/he is drunk.’\\
{\itshape kù-rw-à} & ‘to fight’ & {\itshape kù-rw-ìs-à} & ‘to fight someone’\\
{\itshape kú-ry-à} & ‘to eat’ & {\itshape kù-r-îs-à} & ‘to feed’\\
\lspbottomrule
\end{tabular}
\end{table}

There are two monosyllabic verb roots that consist of a consonant and a vowel, \textit{tá} ‘say’ and \textit{há} ‘give’. \tabref{tab:2:4} shows that when used with a causative, applicative or passive suffix, the vowel /i/ of the suffix coalesces with the vowel /a/ of the root to become /e/ (see also \sectref{bkm:Ref491962181} on vowel hiatus resolution).

\begin{table}
\label{bkm:Ref98929304}\caption{\label{tab:2:4}Vowel height harmony in CV verb roots}
\fittable{
\begin{tabular}{ll@{}ll}
\lsptoprule
/ku-tá-a/ > \textit{kútà} & ‘to say’ & /ku-tá-is-a/ > \textit{kùtêsà}  & ‘to accuse’             \\
/ku-tá-ir-a/ > \textit{kùtêrà}                                         & ‘to tell on behalf of’  \\
/ku-tá-iw-a/ > \textit{kùtêwà}                                         & ‘to be said’\\

/ku-há-a-/ > \textit{kúhà} & ‘to give’ & /ku-há-is-a/ > \textit{kùhêsà}  & ‘to give with’            \\
/ku-há-ir-a/ > \textit{kùhêrà}                                           &    ‘to give on behalf of’ \\
/ku-há-iw-a/ > \textit{kùhêwà}                                           &   ‘to be given’\\
\lspbottomrule
\end{tabular}
}
\end{table}

The second type of vowel harmony, back vowel harmony, affects derivational suffixes with a back vowel /u/, the separative suffixes \textit{-ur} (transitive) -\textit{uk} (intransitive). These suffixes are realized with a mid vowel /o/ when used with a verb stem with a mid back vowel /o/, but not when used with a verb stem with a front mid vowel /e/, as in (\ref{bkm:Ref448244057}).

\newpage
\ea
\label{bkm:Ref448244057}
kù-ᵍǀòp-òr-à    ‘to widen (a hole)’\\
kù-cénk-ùr-à  ‘to cut off half’\\
kù-àr-ùr-à    ‘to open’\\
kù-nyùk-ùr-à  ‘to uproot’\\
kù-vwìk-ùr-à  ‘to uncover’
\z
\subsection{Nasal harmony}
\label{bkm:Ref70697295}\hypertarget{Toc75352615}{}\label{bkm:Ref70947851}\label{bkm:Ref70697565}
In addition to vowel harmony, certain derivational suffixes in Fwe are also subject to nasal harmony. Nasal harmony affects all derivational suffixes with a consonant /r/: the applica\-tive -\textit{ir}, as in (\ref{bkm:Ref98929486}), the transitive separative -\textit{ur}, as in (\ref{bkm:Ref98929502}), and the (highly lexicalized) extensive \textit{\-\nobreakdash-ar}, as in (\ref{bkm:Ref98929515}). The consonant /r/ of the suffix is changed to /n/ when preceded by a verb stem ending in a nasal consonant. Like vowel harmony, this type of nasal harmony is a common Bantu phenomenon \citep{Greenberg1951}.

\ea
\label{bkm:Ref98929486}
  Nasal harmony in the applicative\\
kù-rìm-ìn-à    ‘to farm for’\\
kù-tòm-èn-à    ‘to charge dowry’\\
kù-zyúm-ìn-ìn-à  ‘to become unconscious; to dry’
\z

\ea
\label{bkm:Ref98929502}
Nasal harmony in the transitive separative\\
kù-bbám-ùn-à  ‘to break’\\
kù-fúrùm-ùn-à  ‘to put upright’\\
kù-ⁿǀòngòm-òn-à  ‘to hollow out’
\z

\ea
\label{bkm:Ref98929515}
Nasal harmony in the extensive\\
kù-fúrùm-àn-à  ‘to become adult (of girls)’\\
kù-rém-àn-à    ‘to become injured’\\
kù-zyím-àn-à  ‘to stop, stand up’
\z

Nasal harmony is not trigger by prenasalized consonants, as shown in (\ref{bkm:Ref506546004}).

\ea
\label{bkm:Ref506546004}
kù-rìnd-ìr-à    ‘to wait for’\\
kù-kámb-ùr-à  ‘to remove (from on top of each other)’\\
kù-súmb-àr-à  ‘to be pregnant’
\z

Like vowel harmony, nasal harmony is only triggerd by the syllable immediately preceding the target. No nasal harmony takes place when nasal roots consonants are separated from the derivational suffix by a non-nasal consonant, as in (\ref{bkm:Ref489631913}), where the causative separating the root-final nasal /m/ from the applicative suffix \textit{-ir} prevents the application of nasal harmony.

\ea
\label{bkm:Ref489631913}
\glll kùzìmìsìrà\\
ku-zim-is-ir-a\\
\textsc{inf}-be\_extinguished-\textsc{caus}-\textsc{appl}-\textsc{fv}\\
\glt ‘to extinguish for’
\z

Nasal harmony is also triggered by nasal conso\-nants in derivational suffixes, namely the intransitive impositive suffix \textit{-am}. When com\-bined with an applicative suffix, the applicative suffix follows the impositive, and as such is real\-ized as \nobreakdash-\textit{in}, as in (\ref{bkm:Ref98929572}).

\ea
\label{bkm:Ref98929572}
\glll kùrísùngàmìnà\\
ku-rí-sung-am-in-a\\
\textsc{inf}-\textsc{refl}-bow-\textsc{imp}.\textsc{intr}-\textsc{appl}-\textsc{fv}\\
\glt ‘to bow one’s head’
\z

Similar to vowel harmony, nasal harmony fails to apply in a number of borrowed verbs, as in (\ref{bkm:Ref98929624}) and (\ref{bkm:Ref98929626}). Such verbs are likely to have been borrowed from or through Lozi, as Lozi does not regularly apply nasal harmony \citep[141]{Gowlett1989}.

\ea
\label{bkm:Ref98929624}
\glll kùfónèrà\\
ku-fón-er-a\\
\textsc{inf}-phone-\textsc{appl}-\textsc{fv}\\
\glt ‘to phone’
\z

\ea
\label{bkm:Ref98929626}
kùkòpànèrà (< Lozi kopana ‘meet’)\\
ku-kopan-er-a\\
\textsc{inf}-meet-\textsc{appl}-\textsc{fv}\\
\glt ‘to meet at’
\z

