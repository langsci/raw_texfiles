\chapter{Nominal morphology}
\hypertarget{Toc75352634}{}
A pervasive feature of Fwe nominal morphology is its use of noun classes, nominal genders which are marked through a prefix on the noun and agreement on modifiers. This noun class system, which is typical for Bantu languages, is discussed in \sectref{bkm:Ref70948951}. Nominal morphology is also used to create nouns from verbs or from other nouns, through affixation, compounding and reduplication, as discussed in \sectref{bkm:Ref70948964}. Nominal modifiers, which include adjectives, demonstratives, connectives, quantifiers and possessives, are discussed in \sectref{bkm:Ref74305906}.

\section{Noun classes}
\label{bkm:Ref70948951}\hypertarget{Toc75352635}{}
Fwe nouns are divided into genders, which are commonly referred to as noun classes in Bantu linguistics. Fwe uses 19 noun classes, which are numbered 1-18 (including 1a) according to the Bantu tradition. Noun class agreement is marked on modifiers, as discussed in \sectref{bkm:Ref74305906}, and on verbs, as discussed in Chapter \ref{bkm:Ref98512234}. Noun class membership is also marked on the noun itself by nominal prefixes. The nominal and pronominal prefixes for each noun class are presented in \tabref{tab:4:1}.

\begin{table}
\label{bkm:Ref492029881}\caption{\label{tab:4:1}Nominal agreement}

\begin{tabular}{lll}
\lsptoprule
Noun class & Nominal prefix (\textsc{np}) & Pronominal prefix (\textsc{pp})\\
\midrule
1 & \textit{mu}- & \textit{u-/zyu-}\\
1a & ∅-/\textit{mu}- & \textit{u}-/\textit{zyu}-\\
2 & \textit{ba}- & \textit{ba}-\\
3 & \textit{mu}- & \textit{u}-\\
4 & \textit{mi}- & \textit{i}-\\
5 & ∅-/\textit{ri}- & \textit{ri}-\\
6 & \textit{ma}- & \textit{a}-\\
7 & \textit{ci}- & \textit{ci}-\\
8 & \textit{zi}- & \textit{zi}-\\
9 & \textit{N}-/∅- & \textit{i}-\\
10 & \textit{N}-/∅- & \textit{zi}-\\
11 & \textit{ru}- & \textit{ru}-\\
12 & \textit{ka}- & \textit{ka}-\\
13 & \textit{tu}- & \textit{tu}-\\
14 & \textit{bu}- & \textit{bu}-\\
15 & \textit{ku}- & \textit{ku}-\\
16 & \textit{ha}- & \textit{ha}-\\
17 & \textit{ku}- & \textit{ku}-\\
18 & \textit{mu}- & \textit{mu}-\\
\lspbottomrule
\end{tabular}
\end{table}

Nominal prefixes are glossed as \textsc{np} with a subscript number indicating the noun class. They are used on nouns, as discussed in \sectref{bkm:Ref489005545}, and to mark agreement on adjectives, as discussed in \sectref{bkm:Ref491277755}. Pronominal prefixes are glossed as \textsc{pp} with a subscript number indicating the noun class. Pronominal prefixes are usually toneless, though their tonal behavior is quite variable. They are used to mark agreement on connectives, possessives and quantifiers, and are also used to create demonstratives; these modifiers are discussed in \sectref{bkm:Ref74305906}.

The following sections discuss morphological marking of noun class on nouns. In addition to the obligatory nominal prefix, nouns can take an augment; its form and possible functions are discussed in \sectref{bkm:Ref444175456}. Noun class is used to express number, with certain classes used for singular nouns, and others for their corresponding plural. The pairing of singular and plural noun classes is discussed in \sectref{bkm:Ref498941936}. Noun class membership is partially governed by semantic criteria, and these can be exploited to shift nominal roots to another noun class to derive a different meaning. The semantic basis of noun classes and the derivational processes that are motivated by it are discussed in \sectref{bkm:Ref444190057}. The locative noun classes 16, 17 and 18 have a different syntax than the other noun classes, and are therefore treated separately in \sectref{bkm:Ref452049189}. Finally, in \sectref{bkm:Ref452049199} some observations will be noted about noun class assignment of borrowed nouns.

\subsection{Nominal prefixes}
\label{bkm:Ref489005545}\hypertarget{Toc75352636}{}\label{bkm:Ref489364754}
Nouns are marked for noun class with a nominal prefix, which directly precedes the nominal stem. Most nominal prefixes have a CV-shape, with the exception of the prefixes of class 1a and 5, which have a zero prefix, and the prefixes of class 9 and 10, which consist of a homorganic nasal. The only vowels occurring in nominal prefixes are /a/, /i/ and /u/, never the mid vowels /e/ and /o/. In addition to the nominal prefix, nouns may be marked by an augment prefix, which is discussed in \sectref{bkm:Ref444175456}.

\tabref{tab:4:2} gives an overview of the nominal prefixes, their possible allomorphs and the form of the augment. It should be noted that, whenever a noun is presented as belonging to a certain class, this is backed up by its agreement pattern, e.g. it triggers agreement of that class on its dependents, such as demonstratives, adjectives, connectives, etc. For reasons of space, the relevant agreement patterns will not always be given.

\begin{table}
\label{bkm:Ref488751330}\caption{\label{tab:4:2}Nominal prefixes}
\begin{tabular}{lllll}
\lsptoprule
& Nominal prefix & Augment & Example & Translation\\
\midrule
1 & \textit{mu}- / \textit{mw}- / \textit{m}- & \textit{o}- & \textit{mù-ntù} & ‘person’\\
1a & ∅- / \textit{N}- & \textit{o}- & ∅-\textit{ŋàngà} & ‘doctor’\\
2 & \textit{ba}- / \textit{b}- & \textit{a}- & \textit{bà-ntù} & ‘people’\\
3 & \textit{mu}- / \textit{mw}- / \textit{m}- & \textit{o}- & \textit{mù-bìrì} & ‘body’\\
4 & \textit{mi}- & \textit{e}- & \textit{mì-bìrì} & ‘bodies’\\
5 & ∅- / \textit{r(i)}- & \textit{e}- & \textit{ànjà} & ‘hand’\\
6 & \textit{ma}- / \textit{m}- & \textit{a}- & \textit{mà-ànjà} & ‘hands’\\
7 & \textit{ci}- / \textit{c}- & \textit{e}- & \textit{cì-púrà} & ‘chair’\\
8 & \textit{zi}- / \textit{z}- / \textit{bi}- & \textit{e}- & \textit{zì-púrà} / \textit{bì-púrà} & ‘chairs’\\
9 & \textit{N}- / ∅- & \textit{e}- & \textit{n-gìnà} & ‘louse’\\
10 & \textit{N}- / ∅- & \textit{e}- & \textit{n-gìnà} & ‘lice’\\
11 & \textit{ru}- / \textit{rw}- / \textit{r}- & \textit{o}- & \textit{rù-rîmì} & ‘tongue’\\
12 & \textit{ka}- & \textit{a}- & \textit{kà-shùtò} & ‘fish hook’\\
13 & \textit{tu}- & \textit{o}- & \textit{tù-shùtò} & ‘fish hooks’\\
14 & \textit{bu}- / \textit{bw}- / \textit{b}- & \textit{o}- & \textit{bù-zyûmì} & ‘life’\\
15 & \textit{ku}- /\textit{kw}- & \textit{o}- & \textit{kù-bôkò} & ‘arm’\\
16 & \textit{ha}- & - & \textit{hà-mù-shânà} & ‘on the back’\\
17 & \textit{ku}- & - & \textit{kù-rù-wà} & ‘at the field’\\
18 & \textit{mu}- & - & \textit{mù-mù-nzì} & ‘in the village’\\
\lspbottomrule
\end{tabular}
\end{table}

Class 1a nouns mostly use the agreement pattern of class 1. The only differences between class 1 and class 1a is the nominal prefix, which is \textit{mu-} for class 1 and zero (or N-) for class 1a, and the copulative prefix, which is \textit{ndi-} for class 1 and \textit{ndu-} for class 1a (see \sectref{bkm:Ref450747606} on copulas). The latter is an especially convincing argument to treat class 1a as a separate noun class, but it should be noted that with the exception of the copula, agreement patterns of class 1a are identical to those of class 1, and will be glossed as such.

The nominal prefix and corresponding agreement morphology of class 8 have a variant \textit{bi-} in Zambian Fwe. This could be due to contact with either Lozi or Shanjo, as the class 8 prefix in both languages is \textit{bi-} (\citealt{Bostoen2009}: 120; \citealt{Fortune1977}: 10).

There is a tendency to merge classes 5 and 9, which manifests itself in different ways. Nouns in class 9 often take the class 5 copulative prefix \textit{ndi-} rather than the class 9 copulative prefix \textit{nji-}, and class 9 nouns often do not take their plural in the expected plural class 10, but in class 6, which is the canonical plural class for class 5 nouns. This is discussed in more detail in \sectref{bkm:Ref498941936} on singular and plural pairings.

As seen in \tabref{tab:4:2}, some nominal prefixes have one or two allomorphs. One of these is lexically conditioned: the allomorph \textit{r(i)-} of class 5 only appears on two nouns, given in (\ref{bkm:Ref492314485}). As the prefix \textit{r(i)-} is lost when the noun is used in class 6 to mark a plural, the initial segment \textit{r(i)-} can be analyzed as a prefix of class 5. The presence of /i/ in this allomorph cannot be proven, as the combination of the putative /i/ of the nominal prefix and the following /i/ of the nominal stem may account for the deletion of the initial /i/. Comparison with the paradigm of pronominal prefixes, where the class 5 prefix is \textit{ri-} (see \tabref{tab:4:1}), suggests an underlying vowel /i/ is likely.
\NumTabs{5}
\ea
\label{bkm:Ref492314485}
rínò   \tab\tab   ménò\\
ri-inó  \tab\tab    ma-inó\\
\textsc{np}\textsubscript{5}-tooth   \tab\tab \textsc{np}\textsubscript{6}-tooth\\
\glt ‘tooth’   \tab\tab   ‘teeth’
\z

\ea
rîshò  \tab\tab    mêshò\\
ri-ísho   \tab\tab   ma-ísho\\
\textsc{np}\textsubscript{5}-eye   \tab\tab \textsc{np}\textsubscript{6}-eye\\
\glt ‘eye’   \tab\tab   ‘eyes’
\z

The other allomorphs of nominal prefixes are the result of two morphophonological processes that play a role when combining the prefix with the nominal root: vowel hiatus resolution and prenasalization. As discussed in \sectref{bkm:Ref491962181}, vowel hiatus resolution may take place when a nominal prefix with a CV-shape combines with a vowel-initial noun stem. Nominal prefixes of class 1, 3, 11, and 14 have two allomorphs that are used with vowel-initial stems. One of these allomorphs is created by deleting the vowel /u/ of the prefix and replacing it with a glide /w/. This allomorph is used when the stem of the noun begins with a vowel /a/, /i/ or /e/; examples are given in (\ref{bkm:Ref492314403}).

\ea
\label{bkm:Ref492314403}
\ea
class 1    mw-âncè    ‘child’\\
\ex
class 3    mw-îndì    ‘leg of a pot’\\
\ex
class 11  rw-âtà      ‘crack’\\
\ex
class 14  bw-ékè    ‘grain’\\
\ex
class 15  kw-àhà    ‘armpit’
\z\z

Nominal prefixes with /u/ have a second allomorph used with vowel-initial stems with a back vowel /o/ or /u/. This allomorph is created by deleting the vowel /u/ of the nominal prefix without glide formation. Examples of these allomorphs are given in (\ref{bkm:Ref492314369}).

\ea
\label{bkm:Ref492314369}
\ea 
class 1    m-ôfù      ‘blind person’\\
\ex class 3    m-ûzyà    ‘character’\\
\ex class 11  r-ózì      ‘rope’\\
\ex class 14  b-ôzyà    ‘feathers’
\z\z

The nominal prefixes that have a vowel /i/ or /a/ are usually not changed when combined with a vowel-initial root, as in (\ref{bkm:Ref492314322}).

\ea
\label{bkm:Ref492314322}
\ea
class 4    mì-âkà    ‘years’\\
\ex class 6    mà-ànjà    ‘hands’\\
\ex class 7    cì-òngò    ‘storage’\\
\ex class 8    zì-òngò    ‘storages’\\
\ex class 12  kà-ìngà    ‘spot on the skin’
\z\z

There are a few exceptions to this rule, which are lexically determined. With the two vowel-initial noun stems listed in (\ref{bkm:Ref98512328}), the vowel /i/ of the nominal prefix is deleted.

\ea
\label{bkm:Ref98512328}
\ea 
class 7/8  c-ândà/ z-ândà  ‘pole(s)’\\
\ex class 7/8  c-ûngù/ z-ûngù  ‘bird(s) sp.’
\z\z

There are also vowel-initial stems where the vowel of the nominal prefix is not deleted, but merges with the vowel of the nominal root, as in (\ref{bkm:Ref99539773}), where the vowel /i/ of the root is maintained in the singular, but merges with the vowel /a/ of the nominal prefix in the plural form.

\newpage
\ea
\label{bkm:Ref99539773}
\ea class 1    mw-ìkà    ‘slave’\\
\ex  class 2    /ba-ika/ > bèkà  ‘slaves’
\z\z

A second set of nominal prefix allomorphs are those of class 9 and 10. The basic form of the prefixes of both class 9 and class 10 is a homorganic nasal, segmented in the phonological transcription as N-, that combines with the initial consonant of the nominal root. Morphophonological changes that accompany this prefix have been discussed in \sectref{bkm:Ref451507060}. That the homorganic nasal functions as a nominal prefix can be seen from the loss of the nasal when a nominal root shifts from class 9/10 to another noun class which does not have a homorganic nasal as its nominal prefix, as in (\ref{bkm:Ref491959291}).

\ea
\label{bkm:Ref491959291}
\ea class 9    m-pòhò    ‘bull’\\
\ex class 6    mà-pòhò    ‘bulls’
\z\z

There are also indications that the homorganic nasal is losing its function as a nominal prefix of class 9/10. Most nouns with an apparent \textit{N\nobreakdash-} prefix in class 9/10 do not lose the homorganic nasal when used in a different class, as in (\ref{bkm:Ref99541502}), showing that in these nouns, the homorganic nasal has been reanalyzed as part of the nominal root. There seems to be no conditioning on where the homorganic nasal loses its status as a separate morpheme, and there is also inter-speaker variation in its realization.

\ea
\label{bkm:Ref99541502}

\ea
class 9    m-pòndà    ‘spear’\\
classs 6  mà-mpòndà    ‘spears’

\ex
class 9    n-kúnjù    ‘mortar’\\
class 6    mà-nkúnjù    ‘mortars’

\ex
class 9    m-bútò    ‘seed’\\
class 6    mà-mbútò    ‘seeds’
\z\z

Some borrowed stems that are assigned to class 9 take the \textit{N-} prefix, as in (\ref{bkm:Ref99541539}a). and (\ref{bkm:Ref99541539}b)., others take a zero prefix, as in (\ref{bkm:Ref99541539}c). and (\ref{bkm:Ref99541539}d). Note that in all cases, these nouns function as class 9 nouns, that is, they trigger class 9 agreement on their dependents.

\ea
\label{bkm:Ref99541539}
\ea class 9    n-díshì    ‘dish’\\
\ex class 9    n-kèrékè    ‘church’ (from Afrikaans kerk)\\
\ex class 9    ∅-ràyîsì    ‘rice’\\
\ex class 9    ∅-fúrâyì    ‘airplane’
\z\z

A number of class 9 nouns can also occur in class 5, as seen from the nominal prefix and agreement pattern, as illustrated in (\ref{bkm:Ref506907229}). The choice of noun class differs from speaker to speaker, and there appears to be no difference in interpretation.

\ea
\label{bkm:Ref506907229}
èyí njôkà {\textasciitilde} èrí zyôkà\\
\gll e-í    N-jóka  {\textasciitilde}   e-rí    ∅-zyóka\\
\textsc{aug}-\textsc{dem}.\textsc{i}\textsubscript{9}  \textsc{np}\textsubscript{9}-snake   {} \textsc{aug}-\textsc{dem}.\textsc{i}\textsubscript{5}  \textsc{np}\textsubscript{5}-snake\\
\glt ‘snake’
\z

Many nouns that were originally in class 9 are shifting to class 1a; this is especially (but not exclusively) the case for animal names. When a noun shifts to class 1a, the homorganic nasal prefix is reanalyzed as part of the nominal stem, as in (\ref{bkm:Ref98750663}). This initial nasal suggests that the noun originally belonged to class 9, and its use in class 1a is a recent innovation.

\ea
\label{bkm:Ref98750663}
\ea class 9    è-n-gwè    ‘leopard’\\
\ex class 1a  ò-ngwè    ‘leopard’
\z\z

Variation between class 9 and 1a, such as in (\ref{bkm:Ref98750663}), is uncommon, and most class 1a nouns do not retain any trace of class 9 membership; they take agreement markers of class 1a, and a plural in class 2 rather than class 10, as illustrated with the noun \textit{ò-njòvù} ‘elephant’ in (\ref{bkm:Ref74912549}); the prenasalization of the initial root consonant suggests that it was originally in class 9, but in modern Fwe, this nasal has been reanalyzed as part of the root, and \textit{ò-njòvù} functions as a class 1a noun only, as shown by its class 1 agreement pattern.

\ea
\label{bkm:Ref74912549}
ònjòvù àryâ\\
\gll o-∅-njovu    a-ry\textsubscript{H}-á̲\\
\textsc{aug}-\textsc{np}\textsubscript{1a}-elephant  \textsc{sm}\textsubscript{1}-eat-\textsc{fv}\\
\glt ‘The elephant eats.’
\z

In Zambian Fwe, the \textit{N-} prefix becomes part of the nominal root when the noun shifts to class 1a, and no longer functions as a nominal prefix in any way. In Namibian Fwe, however, the homorganic nasal prefix in class 1a nouns partly functions as a prefix: while a shift to class 2 to express a plural does not involve loss of the nasal, a shift to class 12 to express a diminutive causes the homorganic nasal to be dropped. This is illustrated with the class 1a noun \textit{nshókò} ‘monkey’, which occurs in class 1a, as seen in (\ref{bkm:Ref98750807}), and takes its plural in class 2, as seen in (\ref{bkm:Ref98750810}). In Namibian Fwe, shift to class 12 involves the loss of the nasal, as seen in (\ref{bkm:Ref98750811}), but in Zambian Fwe, even in this case the nasal is maintained, as seen in (\ref{bkm:Ref98750813}).

\ea
\label{bkm:Ref98750807}
òzyú ꜝnshókò\\
\gll o-zyú    ∅-nshokó\\
\textsc{aug}-\textsc{dem}.\textsc{i}\textsubscript{1}  \textsc{np}\textsubscript{1a}-monkey\\
\glt ‘this monkey’
\z

\ea
\label{bkm:Ref98750810}
bàshókò \\
\gll ba-shokó\\
NP\textsubscript{2}-monkey\\
\glt ‘monkeys’
\z

\ea
\label{bkm:Ref98750811}
kàshókóànà\\
\gll ka-shokó-ana\\
\textsc{np}\textsubscript{12}-monkey-\textsc{dim}\\
\glt ‘baby monkey’ (Namibian Fwe)
\z

\ea
\label{bkm:Ref98750813}
kànshókóànà\\
\gll ka-nshokó-ana\\
\textsc{np}\textsubscript{12}-monkey-\textsc{dim}\\
\glt ‘baby monkey’ (Zambian Fwe)
\z

Any class 1a noun loses its homorganic nasal when shifted to class 12. The corresponding unprenasalized consonant has the same manner and place of articulation as the original prenasalized consonant, as well as the same voicing. Surprisingly, though, the morphophonological principles governing the changes that take place when a consonant is prenasalized do not apply here. These determine, for instance, that continuants turn into stops before \textit{N-} (see \sectref{bkm:Ref451507060}). The loss of prenasalization that is observed here, however, does not turn stops back into continuants. This means that /mb/, when it loses its homorganic nasal, changes to the bilabial stop /b/ (written here as <bb>), and not to the fricative /β/: class 1a \textit{ómbwà} ‘dog’ becomes class 12 \textit{ká-bbwà} ‘small dog’. Similarly, when /nd/ loses its homorganic nasal it changes to /d/, and not to /r/, e.g. class 1a \textit{ndávù} ‘lion’ becomes class 12 \textit{kà-dávù} ‘small lion’. /nj/ turns into /j/ rather than /ʒ/, as seen in the class 1a noun \textit{njòvù} ‘elephant’, that becomes class 12 \textit{kàjòvù} ‘small elephant’; and /ng/ turns into /g/ rather than being lost, as in the class 1a noun \textit{ngìrì} ‘warthog’, that becomes class 12 \textit{kà-gìrì} ‘small warthog’.

Not only does this go against the general rules that govern the correspondence between consonants with and without a homorganic nasal, it also results in a proliferation of otherwise uncommon phonemes. Voiced stops are phonemic in Fwe, but their use is limited and they are mainly found in loanwords. Their prenasalized counterparts, however, are very common phonemes found in native words as well. Therefore this surprising morphophonological alternation cannot be the result of nativization, because it makes the form of these words less, rather than more, native.

\subsection{The augment}
\label{bkm:Ref444175456}\hypertarget{Toc75352637}{}
Nouns, as well as certain other nominal elements, can take an augment, a vocalic prefix with a floating tone that precedes the nominal prefix. A similar prefix occurs in different Bantu languages with different forms, where it is sometimes called pre-prefix (\citealt{Gambarage2013}; \citealt{Visser2008}, among others). In this book, following { \citet{Blois1970}}, {\citet{Katamba2003}}, {\citet{Maho1999}} and others, the term “augment” will be used. There is extensive variation in the conditioning of the use of the augment in Bantu languages; mostly, the use of the augment is conditioned by syntactic, semantic, pragmatic or stylistic factors (\citealt{Blois1970}), or an intricate combination thereof, such as in Luganda (\citealt{HymanKatamba1993}). There are also Bantu languages where the use of the augment is optional without apparent conditioning \citep[62]{Maho1998}, or where the use of the augment is becoming more and more optional, such as Kagulu \citep{Petzell2003}, and Namibian Totela \citep{Crane2019}. This section describes the form of the augment in Fwe, showing that it consists of both a vowel and a floating high tone, which can occur independently of each other. Whether the augment has a grammatical function in Fwe is unclear: in most cases there seems to be free variation between absence and presence of the augment.

The nominal augment in Fwe consists of a single prefixed vowel \textit{e-, a-} or \mbox{\textit{o-},} combined with a floating high tone that is realized on the syllable preceding the vowel of the augment. The augment displays vowel harmony with the vowel of the nominal prefix: \textit{e}- is used with nominal prefixes with a vowel /i/, which includes the prefixes of class 4 \textit{mi}-, class 7 \textit{ci}-, class 8 \textit{zi-,} as well as classes 5, 9 and 10, which lack a syllabic nominal prefix; \textit{o}- is used with nominal prefixes with a vowel /u/, which includes the prefixes of class 1 \textit{mu}-, class 3 \textit{mu}-, class 11 \textit{ru\-}-, class 13 \textit{tu}-, class 14 \textit{bu}-, class 15 \textit{ku}-, as well as the prefixless class 1a; and \textit{a}- is used with nominal prefixes with a vowel /a/, which includes the prefixes of class 2 \textit{ba}-, class 6 \textit{ma}-, and class 12 \textit{ka}-. The locative classes 16, 17 and 18 do not have a nominal augment.

\hspace*{-2.2pt}Nouns, adjectives, demonstratives, and infinitive verbs (which behave like nominals) can all be used with or without the augment vowel, as illustrated in (\ref{bkm:Ref98833387}--\ref{bkm:Ref98833389}).

\ea
\label{bkm:Ref98833387}
òmùndárè {\textasciitilde} mùndárè\\
(o-)mu-ndaré\\
\textsc{aug}-\textsc{np}\textsubscript{3}-maize\\
\glt ‘maize’
\z

\ea
mùndárè òmùgênè {\textasciitilde} mùndárè mùgênè\\
\gll mu-ndaré  (o-)mu-géne\\
\textsc{np}\textsubscript{3}-maize  (\textsc{aug}-)\textsc{np}\textsubscript{3}-thin\\
\glt ‘small maize’
\z

\ea
òwìná mùndárè {\textasciitilde} wìná mùndárè\\
\gll (o-)winá    mu-ndaré\\
(\textsc{aug}-)\textsc{dem}.\textsc{iv}\textsubscript{3}  \textsc{np}\textsubscript{3}-maize\\
\glt ‘this maize’
\z

\ea
\label{bkm:Ref98833389}
òkùshàkà {\textasciitilde} kùshàkà\\
(o-)ku-shak-a\\
(\textsc{aug}-)\textsc{inf}-love-\textsc{fv}\\
\glt ‘to love’
\z

Not all nouns can take the augment; the augment is never used with personal names, as in (\ref{bkm:Ref99541892}), or with nouns that are marked with a secondary nominal prefix, such as that of class 2 to mark a honorific, as in (\ref{bkm:Ref99541909}), or those of class 16, 17 or 18 to mark a location, as in (\ref{bkm:Ref99541911}).

\ea
\label{bkm:Ref99541892}
  (*ò)Mwèzì\\
\glt ‘Mwezi’ (girl’s name)
\z

\ea
\label{bkm:Ref99541909}
(*à)bàmùkéntù wángù\\
\gll ba-mu-kéntu  u-angú\\
\textsc{np}\textsubscript{2}-\textsc{np}\textsubscript{1}-woman  \textsc{pp}\textsubscript{1}-\textsc{poss}\textsubscript{1SG}\\
\glt ‘my wife’
\z

\ea
\label{bkm:Ref99541911}
\glll (*ò)kùrùwà\\
ku-ru-wa\\
\textsc{np}\textsubscript{17}-\textsc{np}\textsubscript{11}-field\\
\glt ‘at the field’
\z

With these exceptions, there appears to be no conditioning on the use of the augment vowel on nouns. Nouns may be used with or without the augment vowel, and no change in meaning is observed, as illustrated with the noun \textit{njìngà} ‘bicycle’ in (\ref{bkm:Ref444250297}).

\ea
\label{bkm:Ref444250297}
\ea
nìndákàùrá njìngà\\
\gll ni-ndí̲-a-ka-ur-á      N-jinga\\
\textsc{pst}-\textsc{sm}\textsubscript{1SG}-\textsc{pst}-\textsc{dist}-buy-\textsc{fv}  \textsc{np}\textsubscript{9}-bicycle\\
\glt ‘I bought a bicycle.’

\ex
nìndákàùr’ énjìngà\\
\gll ni-ndí̲-a-ka-ur-á    e-N-jinga\\
\textsc{pst}-\textsc{sm}\textsubscript{1SG}-\textsc{pst}-\textsc{dist}-buy  \textsc{aug}-\textsc{np}\textsubscript{9}-bicycle\\
\glt ‘I bought a bicycle.’ (NF\_Elic15)
\z\z

For demonstratives, the augment vowel is optional but its presence is often governed by phonological well-formedness: monosyllabic demonstrative stems strongly prefer the use of the augment; disyllabic demonstrative stems strongly disprefer the use of the augment (see \sectref{bkm:Ref492026896} on demonstratives).

As Fwe does not allow closed syllables, the vowel-initial syllable of the augment is usually preceded by a word ending in a vowel. The ensuing sequence of two vowels is frequently subject to vowel hiatus resolution, by deleting the vowel of the augment, as in (\ref{bkm:Ref444247824}); by deleting the final vowel of the preceding word, as in (\ref{bkm:Ref444247864}); or by merging the two vowels as in (\ref{bkm:Ref444247873}--\ref{bkm:Ref444247874}) (see also \ref{bkm:Ref491962181} on vowel hiatus resolution).

\ea
\label{bkm:Ref444247824}
ndìkwèsí bámbwà\\
\gll ndi-kwesí  a-ba-mbwá\\
\textsc{sm}\textsubscript{1SG}-have  \textsc{aug}-\textsc{np}\textsubscript{2}-dog\\
\glt ‘I have dogs.’
\z

\ea
\label{bkm:Ref444247864}
ndìshák’ ènyàmà\\
\gll ndi-shak-á̲    e-N-nyama\\
\textsc{sm}\textsubscript{1SG}-want-\textsc{fv}  \textsc{aug}-\textsc{np}\textsubscript{9}-meat\\
\glt ‘I want meat.’ (NF\_Elic15)
\z

\ea
\label{bkm:Ref444247873}
kànt’ úꜝndávù\\
\gll kantí  o-n-davú\\
then  \textsc{aug}-\textsc{np}\textsubscript{1a}-lion\\
\glt ‘Well, the lion…’ (NF\_Narr15)
\z

\ea
\label{bkm:Ref444247874}
vùmw’ énênè\\
\gll ∅-vumó  e-∅-néne\\
\textsc{np}\textsubscript{5}-stomach  \textsc{aug}-\textsc{np}\textsubscript{5}-big\\
\glt ‘a big stomach’ (ZF\_Elic14)
\z

The augment has a floating high tone, which is realized on the vowel directly preceding the augment vowel. The augment vowel itself is normally realized as low-toned (unless a floating high tone is assigned by the nominal stem, see \sectref{bkm:Ref71540184}). In (\ref{bkm:Ref75180241}), the floating high tone of the augment is realized on the preceding syllable, the final vowel suffix \textit{-a} of the infinitive verb, which is underlyingly toneless.

\ea
\label{bkm:Ref75180241}
kùkànká èŋòmbè (cf. kùkànkà ‘to slaughter’)\\
\gll ku-kank-á    e-N-ŋombe\\
\textsc{inf}-slaughter-\textsc{fv}  \textsc{aug}-\textsc{np}\textsubscript{9}-cow\\
\glt ‘to slaughter a cow’
\z

However, because vowel hiatus resolution rules frequently reduce sequences of adjacent vowels to a single vowel, the floating high tone of the augment may revert to the vowel of the augment, when the preceding vowel is deleted. This is illustrated in (\ref{bkm:Ref441658395}), where the floating high tone of the augment \textit{e-} attaches to the preceding syllable \textit{nka}, but when \textit{-a} merges with the vowel of the augment, the floating high tone returns to the vowel of the augment.

\ea
\label{bkm:Ref441658395}
kùkànk’ éŋòmbè\\
\gll ku-kank-á    e-N-ŋombe\\
\textsc{inf}-slaughter-\textsc{fv}  \textsc{aug}-\textsc{np}\textsubscript{9}-cow\\
\glt ‘to slaughter a cow’
\z

The vowel and the floating high tone of the augment can occur independently of each other. In (\ref{bkm:Ref441756770}), the augment’s high tone is used, but its vowel is not. In (\ref{bkm:Ref441756760}), the augment vowel is used, but without the high tone of the augment. It is also possible for a noun to be used without either the vocalic or the tonal augment, as in (\ref{bkm:Ref444249432}).

\ea
\label{bkm:Ref441756770}
kùshàyìká ꜝzíryò\\
\gll ku-sháik-á  zi-ryó\\
\textsc{inf}-cook-\textsc{fv}  \textsc{np}\textsubscript{8}-food\\
\glt ‘to cook food’ (NF\_Elic15)
\z

\ea
\label{bkm:Ref441756760}
kùkùmbìrà èzwáyì\\
\gll ku-kumbir-a    e-∅-zwái\\
\textsc{inf}-request-\textsc{fv}  \textsc{aug}-\textsc{np}\textsubscript{5}-salt\\
\glt ‘to ask for salt’ (ZF\_Narr13)
\z

\ea
\label{bkm:Ref444249432}
kùzímìsà mùrìrò\\
\gll ku-zím-is-a      mu-riro\\
\textsc{inf}-extinguish-\textsc{caus}-\textsc{fv}  \textsc{np}\textsubscript{3}-fire\\
\glt ‘to extinguish fire’ (NF\_Elic15)
\z

Even though the vowel and the high tone of the augment can occur independently of each other, they are clearly related to each other. This can be seen from the form of nouns that can never take a vocalic augment, such as personal names or nouns with a secondary, honorific class 2 prefix. When an augmentless noun follows a word with a low-toned final syllable, no high tone can be assigned to this syllable, and no vocalic augment can be used on the noun, as in (\ref{bkm:Ref99542082}--\ref{bkm:Ref99542083}).

\ea
\label{bkm:Ref99542082}
\ea
ndìzyìː   nyàmbè\\
\gll ndi-zyiː\textsubscript{H}  nyambe\\
\textsc{sm}\textsubscript{1SG}-know  Nyambe\\
\glt ‘I know Nyambe.’

\ex
  *ndìzyíː nyàmbè 
\z
\z

\ea
\label{bkm:Ref99542083}
\ea
ndìsháká kùhòndèrà bámà\\
\gll ndi-shak-á̲    ku-hond-er-a    ba-∅-má\\
\textsc{sm}\textsubscript{1SG}-want-\textsc{fv}  \textsc{aug}-\textsc{inf}-cook-\textsc{appl}-\textsc{fv}  \textsc{np}\textsubscript{2}-\textsc{np}\textsubscript{1a}-mother\\
\glt ‘I want to cook for my mother.’

\ex
  *ndìsháká kùhòndèrá ꜝbámà (NF\_Elic15)
\z\z

Like its vowel, the use of the augment’s high tone is also optional, as shown with the noun \textit{mà-shérêŋì} ‘money’. This noun assigns a high tone to the preceding syllable in (\ref{bkm:Ref498435892}), which may also be absent, as in (\ref{bkm:Ref74912770}). No difference in meaning was observed between the two different realizations.

\ea
\ea    \label{bkm:Ref498435892}
ndìsháká òkùkòròtá màshérêŋì\\
\gll ndi-shak-á̲    o-ku-korot-á    ma-sheréŋi\\
\textsc{sm}\textsubscript{1SG}-want-\textsc{fv}  \textsc{aug}-\textsc{inf}-borrow-\textsc{fv}  \textsc{np}\textsubscript{6}-money\\
\glt ‘I want to borrow some money.’

\ex
\label{bkm:Ref74912770}
ndìsháká òkùkòròtà màshérêŋì\\
\gll ndi-shak-á̲    o-ku-korot-a    ma-sheréŋi\\
\textsc{sm}\textsubscript{1SG}-want-\textsc{fv}  \textsc{aug}-\textsc{inf}-borrow-\textsc{fv}  \textsc{np}\textsubscript{6}-money\\
\glt ‘I want to borrow some money.’ (NF\_Elic17)
\z\z

A question that requires further investigation is whether the augment is completely optional, or whether the presence or absence of the augment correlates with a certain change in meaning. One of the factors that may condition the use of the augment in Bantu languages is referentiality, where the augment is absent on non-referential nouns (\citealt{Velde2019}). This does not appear to be the case in Fwe: on non-referential nouns, the augment may be present, as in (\ref{bkm:Ref498446933}), where the augment’s high tone is discernable on the final vowel of the preceding infinitive verb, or absent, as in (\ref{bkm:Ref74912815}), where the final vowel of the preceding verb does not bear a high tone.

\ea
\label{bkm:Ref498446933}
ndìsháká kùhònd’ énkôkò\\
\gll ndi-shak-á̲    ku-hond-á    e-N-kóko\\
\textsc{sm}\textsubscript{1SG}-want-\textsc{fv}  \textsc{aug}-\textsc{inf}-cook-\textsc{fv}  \textsc{aug}-\textsc{np}\textsubscript{9}-porridge\\
\glt ‘I want to cook some porridge.’
\z

\ea
\label{bkm:Ref74912815}
ndìsháká kùhònd’ ènkôkò\\
\gll ndi-shak-á̲    ku-hond-a    e-N-kóko\\
\textsc{sm}\textsubscript{1SG}-want-\textsc{fv}  \textsc{aug}-\textsc{inf}-cook-\textsc{fv}  \textsc{aug}-\textsc{np}\textsubscript{9}-porridge\\
\glt ‘I want to cook some porridge.’ (NF\_Elic17)
\z

Another factor that can play a role in the conditioning of the augment in Bantu languages is focus, where the absence of the augment correlates with focus (as in, for instance, Luganda, \citealt{HymanKatamba1993}). This, too, does not appear to be the case in Fwe. The main strategy for expressing focus is the use of a cleft construction, which is incompatible with the use of the augment (see \sectref{bkm:Ref491333435} on cleft constructions). Nouns that are not clefted are rarely in focus, but when they are, both absence and presence of the augment is attested, as in (\ref{bkm:Ref498443060}), which is the answer to the question: ‘What did you buy?’, so the noun \textit{njìngà} ‘bicycle’ in the answer is in focus.

\ea
\label{bkm:Ref498443060}
\ea
nìndákàùr’ énjìngà\\
\gll ni-ndí̲-a-ka-ur-á      e-N-jinga\\
\textsc{pst}-\textsc{sm}\textsubscript{1SG}-\textsc{pst}-\textsc{dist}-buy-\textsc{fv}  \textsc{aug}-\textsc{np}\textsubscript{9}-bicycle\\
\glt ‘I bought a bicycle.’

\ex
nìndákàùrá njìngà\\
\gll ni-ndí̲-a-ka-ur-á      N-jinga\\
\textsc{pst}-\textsc{sm}\textsubscript{1SG}-\textsc{pst}-\textsc{dist}-buy-\textsc{fv}  \textsc{np}\textsubscript{9}-bicycle\\
\glt ‘I bought a bicycle.’ (NF\_Elic15)
\z\z

Examples where the presence of the tonal augment on a noun that is in focus can be discerned, are currently not attested. The fact that the tone and vowel of the augment can appear independently of each other complicates the analysis of the possible functions of the augment in Fwe, leaving the possibility that the augment’s tone and vowel are not conditioned by the same factors. Furthermore, the presence of the augment vowel cannot always be discerned, in cases where it may have undergone coalescence with the final vowel of a preceding word. The presence of the high tone of the augment is even more difficult to establish, as it may only surface when the noun is preceded by another word ending in a toneless syllable. A future analysis of the functions of the augment in Fwe needs to take all these factors into account.

\subsection{Singular and plural pairings}
\label{bkm:Ref498941936}\hypertarget{Toc75352638}{}
Noun classes are paired; singular nouns are found in classes 1, 1a, 3, 5, 7, 9, 11, 12, 14 and 15, and their corresponding plurals in classes 2, 4, 6, 8, 10 and 13. The majority of nominal roots can occur in both singular or plural form, some only occur in a singular or only in a plural form. An overview of the combinations of singular and plural classes that are attested is given in (\ref{bkm:Ref505879590}).

\ea
\label{bkm:Ref505879590}
Singular \tab   Plural\\
1 \tab      2, 6\\
1a  \tab    2\\
3  \tab    4\\
5   \tab   6\\
7   \tab   8\\
9   \tab   10, 6\\
11 \tab     10, 6, 13, 14, 1a\\
12   \tab   13, 5\\
14  \tab    6\\
15  \tab    6
\z

The majority of nouns that occur in class 1 in the singular occur in class 2 in the plural form, as in (\ref{bkm:Ref99542247}).

\ea
\label{bkm:Ref99542247}
\ea
class 1     mù-ntù   ‘person’\\
class 2    bà-ntù  ‘people’

\ex
class 1    mù-àmbì    ‘speaker’\\
  class 2    bà-àmbì    ‘speakers’
\z\z

Exceptions, where the plural form is in class 6 rather than class 2, are names for ethnic groups, as in (\ref{bkm:Ref99542320}), and borrowings from Lozi, as in (\ref{bkm:Ref99542321}).

\ea
\label{bkm:Ref99542320}
class 1    mù-búrù    ‘Afrikaner’\\
class 6    mà-búrù    ‘Afrikaners’
\z

\ea
\label{bkm:Ref99542321}
class 1    mù-rútì    ‘teacher’\\
class 6    mà-rútì    ‘teachers’
\z

As discussed in \sectref{bkm:Ref489005545}, class 1a nouns often follow the behavior of class 1 nouns. They also take the corresponding plural of class 1 nouns, which is class 2, as in (\ref{bkm:Ref98836467}--\ref{bkm:Ref99542392}).

\ea
\label{bkm:Ref98836467}
class 1a  ∅-nzìkè    ‘single person’\\
class 2    bà-nzìkè    ‘single people’
\z

\ea
\label{bkm:Ref99542392}
class 1a  ∅-nyâtì    ‘buffalo’\\
class 2    bà-nyâtì    ‘buffaloes’
\z

Nouns that have their singular in class 3 have their plural in class 4, as in (\ref{bkm:Ref98836481}--\ref{bkm:Ref99542429}).

\ea
\label{bkm:Ref98836481}
class 3    mù-bìrì    ‘body’\\
class 4    mì-bìrì    ‘bodies’
\z

\ea
\label{bkm:Ref99542429}
class 3    mw-îngà    ‘thorn’\\
class 4    mì-îngà    ‘thorns’
\z

For a small number of nouns, use in class 4 does not represent the plural of its use in class 3, but a different meaning, which is not as predictable as a change from singular to plural but nonetheless clearly semantically related; some examples are given in (\ref{bkm:Ref491965358}--\ref{bkm:Ref99542534}).

\ea
\label{bkm:Ref491965358}
class 3    mù-rèzù    ‘chin’\\
class 4    mì-rèzù    ‘beard’ (* ‘chins’)
\z

\ea
\label{bkm:Ref99542534}
class 3    mù-ròmò    ‘mouth’\\
class 4    mì-ròmò    ‘lips’ (* ‘mouths’)
\z

Nouns that have their singular in class 5 have their plural in class 6, as in (\ref{bkm:Ref505613159}), and nouns that have their singular in class 7 have their plural in class 8, as in (\ref{bkm:Ref505613160}).

\ea
\label{bkm:Ref505613159}
\ea
class 5    ∅-sèsì       ‘bullfrog’\\
class 6    mà-sèsì    ‘bullfrogs’

\ex
class 5    ∅-nôkà    ‘hip’\\
class 6    mà-nôkà    ‘hips’
\z\z

\ea
\label{bkm:Ref505613160}
\ea
class 7    cì-bâtà    ‘scar’\\
class 8    zì-bâtà    ‘scars’

\ex
class 7    cì-fwìnsò    ‘stopper’\\
class 8    zì-fwìnsò    ‘stoppers’
\z\z

Nouns that have their singular in class 9 have their plural in class 10, as in (\ref{bkm:Ref491965793}--\ref{bkm:Ref99543071}), or in class 6, as in (\ref{bkm:Ref491965795}--\ref{bkm:Ref99543100}).

\ea
\label{bkm:Ref491965793}
class 9    m-búfù    ‘bream’\\
class 10  m-búfù    ‘breams’
\z

\ea
class 9    m-pâmpà    ‘forked stick’\\
class 10  m-pâmpà    ‘forked sticks’
\z

\ea
\label{bkm:Ref99543071}
class 9    n-cùpà    ‘whip’\\
class 10  n-cùpà    ‘whips’
\z

\ea
\label{bkm:Ref491965795}
class 9    n-jûò      ‘house’\\
class 6    mà-zyûò    ‘houses’
\z

\ea
class 9    n-gômà    ‘drum’\\
class 6    mà-ômà    ‘drums’
\z

\ea
\label{bkm:Ref99543100}
class 9    n-káꜝmbámò    ‘slope’\\
class 6    mà-nkáꜝmbámò  ‘slopes’
\z

Nouns that have their singular in class 11 have their corresponding plural in class 10, as in (\ref{bkm:Ref491966094}--\ref{bkm:Ref99544028}), or in class 6, as in (\ref{bkm:Ref491966095}--\ref{bkm:Ref99544029}). Class 11 is also used as a singulative; examples are given in \sectref{bkm:Ref450747656}.

\ea
\label{bkm:Ref491966094}
class 11  rù-kânì    ‘jaw’\\
class 10  n-kânì      ‘jaws’
\z

\ea
\label{bkm:Ref99544028}
class 11  rù-shôshò    ‘shinbone’\\
class 10  n-shôshò    ‘shinbones’
\z

\ea
\label{bkm:Ref491966095}
class 11  rù-nâkà    ‘horn’\\
class 6    mà-nâkà    ‘horns’
\z

\ea
\label{bkm:Ref99544029}
class 11  rù-tângò    ‘story, proverb’\\
class 6    mà-tângò    ‘stories, proverbs’
\z

Nouns that have their singular in class 12 have their plural in class 13, as in (\ref{bkm:Ref98836547}--\ref{bkm:Ref99544063}).

\ea
\label{bkm:Ref98836547}
class 12  kà-cíyóꜝcíyò    ‘chick’\\
class 13  tù-cíyóꜝcíyò    ‘chicks’
\z

\ea
\label{bkm:Ref99544063}
class 12  kà-nyàndì    ‘fishing net’\\
class 13  tù-nyàndì     ‘fishing nets’
\z

Class 14 contains mostly nouns that occur only in the singular. Nouns with their singular in class 14 that do have a plural have their plural in class 6, as in (\ref{bkm:Ref99544079}--\ref{bkm:Ref99544081}).

\ea
\label{bkm:Ref99544079}
class 14  bú-tà      ‘bow’\\
class 6    má-tà      ‘bows’
\z

\ea
\label{bkm:Ref99544081}
class 14  bù-kwízyù    ‘fig tree’\\
class 6    mà-kwízyù    ‘fig trees’
\z

Only four nouns are attested that have their singular in class 15, listed in (\ref{bkm:Ref97891564}). These have their plural in class 6. Other class 15 nouns are infinitives, which do not have a plural form.

\ea
\label{bkm:Ref444168488}
\label{bkm:Ref97891564}
class 15  kú-twì  ‘ear’    \tab class 6    má-twì  ‘ears’\label{bkm:Ref492314890}\\
class 15  kw-àhà  ‘armpit’ \tab class 6    m-àhà    ‘armpits’\\
class 15  kù-ùrù  ‘leg’ \tab   class 6    mà-ùrù  ‘legs’\\
class 15  kù-bôkò  ‘arm’ \tab   class 6    mà-bôkò  ‘arms’
\z

Some nouns occur only in a singular class, and have no corresponding plural. These are found in most singular classes, except class 1, which is restricted to human referents. Many refer to abstract concepts, uncountable objects or mass nouns, i.e. objects where counting is irrelevant or impossible, as in (\ref{bkm:Ref99544127}).

\ea
\label{bkm:Ref99544127}
class 1a  shómbò    ‘cassava leaves’\\
class 1a  mvûrà      ‘rain’\\
class 3    mù-mè    ‘dew’\\
class 3    mù-rízìngè    ‘ivy’\\
class 5    dùdùsâ    ‘dust’\\
class 5    hûzyà      ‘breath’\\
class 7    cì-fwè      ‘Fwe (language)’\\
class 7    cì-nyùngèrà    ‘type of dish’\\
class 9    m-bùndù    ‘mist’\\
class 9    nyôtà      ‘thirst’\\
class 11  rû-hò      ‘wind’\\
class 11  rù-nèmbwè    ‘cannabis’\\
class 12  kà-mwî    ‘heat; mid-day’\\
class 12  ká-nsìkwè    ‘darkness’
\z

Fwe has also a number of nouns that occur only in a plural noun class, without a corresponding singular form, as in (\ref{bkm:Ref99544169}). These are found in class 6, 8, and 10, and include mass nouns and certain abstract concepts.

\ea
\label{bkm:Ref99544169}
class 6    m-ênjì    ‘water’\\
class 6    mà-shêshwà    ‘marriage’\\
class 6    mà-síkù    ‘night’\\
class 8    zí-ryò      ‘food’\\
class 8    zì-zyàmbìrò    ‘gathered foods’\\
class 10  n-shúkì    ‘hair’\\
class 10  n-kûnì    ‘firewood’\\
class 10  n-têtè      ‘berries sp.’
\z
\subsection{The semantics of noun classes}
\hypertarget{Toc75352639}{}\label{bkm:Ref499035982}\label{bkm:Ref498357105}\label{bkm:Ref450750293}\label{bkm:Ref450747845}\label{bkm:Ref450747827}\label{bkm:Ref450747802}\label{bkm:Ref450747656}\label{bkm:Ref444190057}
Some noun classes have a clear semantic core, others are used for a variety of different nouns with no clear semantic coherence. An overview of the semantics of each noun class is given in (\ref{bkm:Ref492314267}).

\NumTabs{2}
\TabPositions{.05\textwidth, .6\textwidth}
\ea
\label{bkm:Ref492314267}

  1 \tab humans\\
2 \tab plural of class 1, 1a\\
1a \tab  mainly animates\\
3 \tab nature, tree and plant names; single body parts; tools; miscellaneous\\
4 \tab plural of class 3\\
5 \tab miscellaneous\\
6 \tab plural of class 5; mass nouns, liquids; deverbal nouns; miscellaneous\\
7 \tab miscellaneous\\
8 \tab plural of class 7\\
9 \tab miscellaneous\\
10 \tab plural of class 9, 11\\
11 \tab elongated objects; singulative; miscellaneous\\
12 \tab diminutives, miscellaneous\\
13 \tab plural of class 12\\
14 \tab abstract nouns, mass nouns, miscellaneous\\
15 \tab body parts, verbs\\
16 \tab location: on, at or near\\
17 \tab location, direction\\
18 \tab location: inside
\z

The semantic principles underlying the noun class system are also used for derivation. Nouns may shift from their inherent noun class to a different noun class, involving a change in semantics. These derivational functions will also be illustrated in this section.

Class 1 is exclusively used for nouns referring to humans, as in (\ref{bkm:Ref98836656}).

\ea
\label{bkm:Ref98836656}
mù-ntù  ‘person’\\
mù-sâ    ‘thief’\\
mù-râmù  ‘brother-in-law’\\
mù-shêrè  ‘friend’\\
mù-sûmbà  ‘pregnant woman’
\z

Class 1a is mainly used for animate nouns, some human, including personal names, some non-human, although it also contains a few inanimates, mainly edible plants. Examples are given in (\ref{bkm:Ref98836715}).

\ea
\label{bkm:Ref98836715}
\ea Humans  \\
kàpàsò      ‘policeman’\\
    màrìânjò    ‘virgin’\\
    ŋàngà      ‘doctor’\\
    mfûzì      ‘blacksmith’

\ex
  Names  \\
  nyàmbè    ‘Nyambe (boy’s name)’\\
    nèzyûbà    ‘Nezyuba (girl’s name)’

\ex
  Animals  \\
  mvwì      ‘kudu’\\
    ŋárò      ‘chameleon’\\
    ngwènà    ‘crocodile’\\
    nkângà      ‘guinea fowl’

\ex
  Plants  \\
  (kà)ngùrù    ‘sweet potato’\\
    mbwîtì    ‘horned melon’\\
    shómbò    ‘cassava leaves’\\
    ndôngò    ‘groundnuts’

\ex
  Inanimates  \\
  mvûrà      ‘rain’\\
    (m)pótò    ‘pot’
\z\z

Class 1a nouns referring to humans are mainly restricted to borrowings, e.g. the English or Afrikaans borrowing \textit{dòkótà} ‘doctor’, and the Lozi borrowing \textit{kàpà\-sò} ‘policeman’. Other human nouns in class 1a are kinship terms, e.g. \textit{mâmà} ‘grandmother’, \textit{mâyè} ‘mother’, \textit{bbâbbà} ‘grandfather’.

The majority of nouns in class 1a are words for animals, although animal names are also found in other classes. There seems to be no semantic coherence as to which animal names are found in class 1a.

A group of nouns in class 1a that cuts across semantic groupings is nouns with a derivational prefix \textit{shi-/si-} or \textit{na-}. These nouns, which can refer to humans, animals or plants, are invariably assigned to class 1a. For more on this derivational strategy, see \sectref{bkm:Ref450750897}.

Class 2 is used to form the plural of nouns in class 1 or 1a, but the class 2 nominal prefix can also be added to refer to a single person in a respectful way. In this case the class 2 nominal prefix is used a secondary prefix; it precedes, rather than replaces, the original nominal prefix. The resulting noun takes the class 2 agreement pattern, as in (\ref{bkm:Ref450134818}), where the noun \textit{bàmùrútí}, derived with the class 2 prefix, triggers the use of a pronominal prefix of class 2.

\ea
\label{bkm:Ref450134818}
bàmùrútí bóꜝngánà\\
\gll ba-mu-rutí    ba-ó=nganá\\
\textsc{np}\textsubscript{2}-\textsc{np}\textsubscript{1}-teacher  \textsc{pp}\textsubscript{2}-\textsc{con}=smart\\
\glt ‘a smart teacher’
\z

This differs from the use of the locative classes 16, 17 and 18, whose prefixes are also used in addition to the noun’s original prefix, but who keep the agreement pattern of the original noun class (see \sectref{bkm:Ref452049189}). Even more complicated agreement patterns are seen with the nouns \textit{mùkêntù} ‘wife’ and \textit{múꜝ}\textit{kwámè} ‘husband’; when used with a possessive, the possessive is marked with class 1 agreement even when the head noun is marked with a class 2 honorific prefix, as in (\ref{bkm:Ref99544595}). All other modifiers, however, do take class 2 agreement, as is the case with the demonstrative in (\ref{bkm:Ref450135386}), and the subject and object marker referring to \textit{bàmùkéntù wángù} ‘my wife’, as in (\ref{bkm:Ref450135406}).

\ea
\label{bkm:Ref99544595}
bàmùkéntù wángù\\
\gll ba-mu-kéntu  u-angú\\
\textsc{np}\textsubscript{2}-\textsc{np}\textsubscript{1}-woman  \textsc{pp}\textsubscript{1}-\textsc{poss}\textsubscript{1SG}\\
\glt ‘my wife’
\z

\ea
\label{bkm:Ref450135386}
àbá bàmúꜝkwámè wénù\\
\gll a-bá    ba-mú-kwamé  u-enú\\
\textsc{aug}-\textsc{dem}.\textsc{i}\textsubscript{2}  \textsc{np}\textsubscript{2}-\textsc{np}\textsubscript{1}-husband  \textsc{pp}\textsubscript{1}-\textsc{poss}\textsubscript{2PL}\\
\glt ‘this husband of yours’ (NF\_Narr15)
\z

\ea
\label{bkm:Ref450135406}
háìbà bàmùkéntù wángù bàkwèsì nyàzì mbòndíbàkâːnè \\
\gll háiba  ba-mu-kéntu  u-angú  ba-kwesi  N-nyazi\\
when  \textsc{np}\textsubscript{2}-\textsc{np}\textsubscript{1}-wife    \textsc{pp}\textsubscript{1}-\textsc{poss}\textsubscript{1SG}  \textsc{sm}\textsubscript{2}-have  \textsc{np}\textsubscript{9}-lover\\
mbo-ndí̲-ba\textsubscript{H}-ká̲ː\textsubscript{H}n-e\\
\textsc{near}.\textsc{fut}-\textsc{sm}\textsubscript{1SG}-\textsc{om}\textsubscript{2}-refuse-\textsc{pfv}.\textsc{sbjv}\\
\glt ‘If my wife has a lover, I will divorce her.’ (ZF\_Conv13)
\z

The honorific use of \textit{ba-} is required when the speaker wants to refer to anyone older than himself, as well as to anyone who generally commands respect, such as teachers, policemen, chiefs and other figures of authority. The honorific prefix can also be used with personal names, as in (\ref{bkm:Ref99544665}--\ref{bkm:Ref99544666}).

\ea
\label{bkm:Ref99544665}
  bá-nyàmbè\\
\glt ‘Mr. Nyambe’
\z

\ea
\label{bkm:Ref99544666}
  bà-klàwùdìà\\
\glt ‘Mrs. Claudia’
\z

When no noun is used, class 2 agreement can be used to refer to a single person in a respectful way, such as the use of the class 2 subject marker in (\ref{bkm:Ref450137252}), or the class 2 object marker in (\ref{bkm:Ref450137268}).

\ea
\label{bkm:Ref450137252}
\glll bàzyíbéhèrè\\
ba-zyi\textsubscript{H}b-é̲here\\
\textsc{sm}\textsubscript{2}-know-\textsc{neut}.\textsc{stat}\\
\glt ‘S/he is well-known.’
\z

\ea
\label{bkm:Ref450137268}
mùbàhé cìpùrà bàkáréhò\\
\gll mu-ba\textsubscript{H}-ha\textsubscript{H}-é̲    ci-pura  ba-ka\textsubscript{H}r-e=hó̲\\
\textsc{sm}\textsubscript{2PL}-\textsc{om}\textsubscript{2}-give-\textsc{pfv}.\textsc{sbjv}  \textsc{np}\textsubscript{7}-chair  \textsc{sm}\textsubscript{2}-sit-\textsc{pfv}.\textsc{sbjv}=\textsc{loc}\textsubscript{16}\\
\glt ‘Give her a chair to sit on.’ (NF\_Elic15)
\z

The use of plural forms as a marker of respect is also used for the second person; this use is discussed for subject and object markers in Sections \ref{bkm:Ref451511047}-\ref{bkm:Ref451511050}, and for personal pronouns in \sectref{bkm:Ref488767962}.

As seen in (\ref{bkm:Ref99544734}), class 3 contains nouns from various semantic fields: trees, plants, or other natural phenomena in the broad sense of the word; body parts, mainly those which do not occur in pairs; tools, used in cooking, hunting, medical procedures, or for general chores. Many other nouns in class 3 do not fall into either of these categories.

\ea
\label{bkm:Ref99544734}

\ea
  Trees      \\
  mù-swîtì    ‘magic guarri (\textit{Euclea divinorum})’\\
      mù-táfùnànjòvù  ‘acacia’\\
      mù-kûsì    ‘Zambezi teak (\textit{Baikiaea plurijuga})’

\ex
  Plants  \\
  mù-nshàrè    ‘sugar cane’\\
      mù-shwátì    ‘sugar cane’\\
      mù-tébè    ‘reed (\textit{Typha capensis})’

\ex
  Natural  \\
  mw-êzì    ‘moon, month’\\
phenomena    mù-fwè    ‘stone’\\
      mù-nùnkò    ‘(bad) smell’\\
      mú-ꜝnzúrè    ‘shadow; malaria’\\
      m-òyà      ‘wind’

\ex
  Unpaired body parts   \\
  mù-cîrà    ‘tail’\\
    m-òzyò    ‘heart’\\
      mù-rívù    ‘windpipe’\\
      mù-shânà    ‘back’

\ex
  Tools \\
  mù-shûwì    ‘horn for sucking blood from a               wound’ \\
      mù-sókwânì    ‘stirring stick’\\
      mù-nséfà    ‘sieve’\\
      mw-ìnshì    ‘pestle’\\
      mù-wàyò    ‘arrow’

\ex
  Miscellaneous \\
  mù-zîò    ‘load’\\
      mù-zwákêrà    ‘poison’\\
      mù-sûngà    ‘belt\\
      mù-sébézì    ‘work’
\z\z

Class 5 contains nouns with varying semantics: nouns referring to paired body parts; other paired items; mass nouns. Class 5 also contains many loanwords from non-Bantu languages; their incorporation into class 5 is facilitated by the zero nominal prefix of this class. An overview is given in (\ref{bkm:Ref99544779}).

\ea
\label{bkm:Ref99544779}
\ea
 Paired body parts \\
 háfù    ‘lung’\\
      nshwê    ‘breast’\\
      rákàtà    ‘gill’\\
      r-îshò    ‘eye’

\ex
  Other paired items \\
  nyàtérà  ‘sandal’\\
      nyìnyánì  ‘earring’\\
      sìkíò    ‘earring’\\
      kàmbà    ‘river bank’

\ex
  Mass nouns  \\
  shékèshêkè  ‘sand’\\
      tàpà    ‘mud’\\
      túꜝkútà  ‘dirt’\\
      é-twè    ‘ash’\\
      sûtù    ‘chaff’

\ex
Loanwords \\
fônì ‘phone’\\
        jókwè    ‘yoke’\\
        sákà    ‘bag’; from Afrikaans \textit{sak} ‘bag’\\
        hèmêrè  ‘bucket’; from Afrikaans \textit{emmer} ‘bucket’ \\        
        ᵍǀúmù ‘edible reed’; from Ju \textit{gǂkò’m} ‘milky sap’ (\citealt{GunninkEtAl2015}: 227)
        \z\z

As discussed in \sectref{bkm:Ref498941936}, many nouns that occur only in the plural form are found in class 6. These include non-count nouns, especially those referring to liquids; paired items that are always referred to with a plural form, or only occur in the plural; abstract concepts, and deverbal nouns. These semantic categories are illustrated in (\ref{bkm:Ref99544963}).

\ea
\label{bkm:Ref99544963}
 \ea
Non-count nouns\\  mà-hìrà    ‘sorghum’\\
      mà-shérêŋì    ‘money’\\
      mà-bérè    ‘millet’

\ex
  Liquids  \\
  mà-bísì    ‘sour milk’\\
      mà-ròhà    ‘blood’\\
      m-ênjì    ‘water’

\ex
  Paired items  \\
  mà-gìrázì    ‘(eye-)glasses’\\
      mà-shángànjìrà  ‘crossroads’\\
      mà-zyòvù    ‘twins’

\ex
  Abstract concepts \\
  mà-ntà    ‘power’\\
      mà-rwêzyà    ‘taboo’

\ex
  Deverbal nouns \\
  mà-hóndêrò    ‘kitchen’; cf. hònd-à ‘cook’ \\
      mà-kwátìrò    ‘handle’ cf. kwât-à ‘grab’ \\
      mà-rârò    ‘room’ cf. râːr-à ‘sleep’
\z\z

Nouns in class 7 mostly refer to inanimate objects, including those derived from verbs, or to the names of languages, as in (\ref{bkm:Ref99544979}).

\ea
\label{bkm:Ref99544979}

\ea
  Miscellaneous \\
  cì-zùmà  ‘basket with lid’\\
inanimate     cì-byà    ‘household item’\\
      cì-mátè  ‘wall’

\ex
  Deverbal nouns \\
  cì-fwìnsò  ‘stopper, seal’, cf. fwìns-à ‘seal’\\
      cí-fò    ‘poison used in hunting’, cf. fw-à ‘die’\\
      cí-àzò    ‘door’ cf. àr-à ‘close’\\
      cì-bónàntù  ‘something visible’, cf. bôn-à ‘see’\\
      cì-téndântù  ‘action’ cf. tènd-à ‘do’

\ex
  Language names \\  cì-fwè    ‘Fwe’\\
      cì-búrù  ‘Afrikaans’\\
      cì-kúwà  ‘English’\\
      cì-rwîzyì  ‘Lozi’
\z\z

Some nouns in class 7 have a derogatory meaning, or express something that is useless, bad, or broken. This derogatory meaning may be seen in underived nouns, as illustrated in (\ref{bkm:Ref99545006}); class 7 contains the names of diseases, of disfunctional or undesirable body parts, of animals that are useless or harmful to humans, and of humans of low social status, or with physical disabilities; the latter, however, may also occur in class 1.
\NumTabs{3}
\ea
\label{bkm:Ref99545006}
  Class 7 nouns with a derogatory meaning

\ea Diseases \\
cì-kâzì \tab ‘women’s disease’\\
cì-sháꜝmátwà \tab ‘kind of illness (involving nausea)’\\
cì-sóngò \tab ‘kind of illness’\\
cì-rwârù \tab ‘disease (generic)’\\
\ex Disfunctional/ undesirable body parts \\ cì-tùkùtùkù \tab ‘sweat’\\
cì-bâtà \tab ‘scar’\\
cì-ⁿǀûshù \tab ‘sore’\\
cì-rábì \tab ‘wound’\\
\ex Useless or harmful animals \\
cì-mbòtwè \tab ‘frog’\\
cì-sînzì \tab ‘termite’\\
cì-shûmì \tab ‘biting insect’\\
cîː-rì \tab ‘puff-adder’\\
cì-bàtànà \tab ‘predator, wild animal’\\
\ex Humans with physical disabilities or low social status \\
cì-nkómbwà \tab ‘slave’\\
cì-púrùpúrù \tab ‘deaf and dumb person’\\
cì-dàkwà \tab ‘heavy drinker, alcoholic’\\
cì-kébéngà \tab ‘criminal’\\
cì-hórè \tab ‘disabled person’\\
cí-yàzì \tab‘traitor’\\
\z\z

A derogatory meaning can also be derived by shifting a noun to class 7, such as \textit{mbwà} ‘dog’, inherently in class 1a, which can be shifted to class 7 \textit{cí-bbwà} ‘stupid/ugly dog’ to derive a derogative. Class 7 agreement may also be used to express a derogative meaning, as illustrated in (\ref{bkm:Ref98512351}--\ref{bkm:Ref99545055}), an excerpt from a story. The speaker relays how he cuts off his own eye that has been wounded. In (\ref{bkm:Ref98512351}), the word for ‘eye’, \textit{rínshò}, is used in its inherent class 5, because it is still attached to his body; once cut off, he refers to the eye with agreement concords of class 7 in (\ref{bkm:Ref99545055}). This is in line with the tendency for class 7 to contain disfunctional body parts.

\ea
\label{bkm:Ref98512351}
àhà ndíkèːzyà kùtêyè èrí rînshò ndìzèràzérà ndìrìkóshórèkò búryò\\
\gll a-ha    ndí̲-keːzy-a    kutéye    e-rí     ri-ínsho\\
\textsc{aug}-\textsc{dem}.\textsc{i}\textsubscript{16} \textsc{sm}\textsubscript{1SG}.\textsc{rel}-come-\textsc{fv}  that    \textsc{aug}-\textsc{dem}.\textsc{i}\textsubscript{5} \textsc{np}\textsubscript{5}-eye\\
ndi-zera-zer-á̲    ndi-ri\textsubscript{H}-ko\textsubscript{H}shó̲r-e=ko    bu-ryó\\
\textsc{sm}\textsubscript{1SG}-\textsc{pl}2-dangle-\textsc{fv}  \textsc{sm}\textsubscript{1SG}-\textsc{om}\textsubscript{5}-cut-\textsc{pfv}.\textsc{sbjv}=\textsc{loc}\textsubscript{17}  \textsc{np}\textsubscript{14}-just\\
\glt ‘Then, when I saw that the eye was dangling, let me just cut it.’
\z

\ea
\label{bkm:Ref99545055}
àhà    ndákùcíkòshòrà\\
\gll a-ha    ndí̲-aku-cí-koshor-a\\
\textsc{aug}-\textsc{dem}.\textsc{i}\textsubscript{16}  \textsc{sm}\textsubscript{1SG}.\textsc{rel}-\textsc{npst}.\textsc{ipfv}-\textsc{om}\textsubscript{7}-cut-\textsc{fv}\\
\glt ‘When I had cut it…’ (ZF\_Narr14)
\z

As seen in (\ref{bkm:Ref99545117}), the semantics of nouns in class 9/10 is very varied; it contains words for manufactured objects, for a wide variety of mental and physical sensations, abstract concepts, especially those derived from verbs, and animals, especially those that are useful for humans, which includes but is not limited to domesticated animals. This is not an exhaustive list of categories; many nouns in class 9/10 do not fit these semantic criteria.

\ea
\label{bkm:Ref99545117}
  Semantics of class 9/10 nouns
\ea
Manufactured objects \\
ŋòmézò \tab ‘button’\\
zândò \tab ‘fishing trap (made out of reed)’\\
n-gômà \tab ‘drum (musical instrument)’\\
n-kwánà \tab ‘pot for beer or water’\\
\ex Mental and physical sensations \\
fúfà \tab ‘jealousy’\\
nyôtà \tab ‘thirst’\\
m-péhò \tab ‘cold; malaria’\\
n-zózì\footnotemark{} \tab ‘dreaming’\\
n-sépò \tab ‘hope’\\
ŋônzì \tab ‘sleep, drowsiness’\\
\ex Abstract concepts \\
n-tùkèrò \tab ‘responsibility, right’\\
n-gàzyàrò \tab ‘plan’\\
n-kàwùhânò \tab ‘divorce’\\
n-gùrìsò \tab ‘profit’\\
\ex Useful animals \\
n-gù \tab ‘sheep’\\
ŋòmbè \tab ‘cow’\\
m-pênè \tab ‘goat’\\
m-bòmà \tab ‘python’\footnotemark{}\\
n-swì \tab ‘fish’\\
m-púkà \tab ‘bee’\\
\z\z

\addtocounter{footnote}{-2}
\stepcounter{footnote}\footnotetext{Fwe distinguishes \textit{nzózì}\textrm, the process of dreaming, from \textit{cì-rôːtò}\textit, the content of the dream.}
\stepcounter{footnote}\footnotetext{As I was told by my informants, the python is the only snake that is eaten.}

Class 11 contains many nouns referring to elongated objects, including grass and reed species, as in (\ref{bkm:Ref99545138}).
\NumTabs{3}
\ea
\label{bkm:Ref99545138}
  Semantics of class 11 nouns

\ea 
Reed species\\ 
rù-tàkà \tab ‘reed’\\
rú-ⁿǀáⁿǀà \tab ‘sedge-leaf (\textit{Kylinga alba})’\\
rù-ǀómà \tab ‘papyrus’\\
rù-kwê \tab ‘reed (\textit{Schoenoplectus corymbosus})’\\
\ex Grass species \\
rù-gwáràrà \tab ‘grass (\textit{Juncus krausii})’\\
rù-sîwù \tab ‘grass (\textit{Cyperus fulgens})’\\
rù-fíyêrò \tab ‘grass (\textit{Stipagrostis uniplumis})’\\
\ex Other elongated objects \\
rù-kwákwà \tab ‘fence’\\
rw-îzyì \tab ‘river’\\
rù-hátì \tab ‘rib’\\
rù-shòshò \tab ‘tibia’\\
rù-òngòrà \tab ‘backbone’\\
\z\z

Class 11 is also used as to derive a singulative; a noun stem can be shifted to class 11 to express a singular entity of something that usually does not occur by itself, as in (\ref{bkm:Ref99545164}).

\ea
\label{bkm:Ref99545164}

\ea
  class 3      mù-tàkà    ‘reeds’\\
class 11    rù-tàkà    ‘a single reed’

\ex
  class 1a    ndôngò    ‘groundnuts’\\
class 11    rù-ndôngò    ‘a single groundnut’

\ex
  class 10    m-bàrè    ‘seeds, pips’\\
class 11    rù-bàrè    ‘a single seed, pip’

\ex
  class 14    bw-ékè    ‘grains’\\
class 11    rw-ékè    ‘a single grain’
\z\z

Class 12/13 is the diminutive class; it contains a number of nouns that only occur in class 12/13, mostly nouns referring to small things, including small or young animals, and also a number of utensils and tools used in food preparation. These are illustrated in (\ref{bkm:Ref99545195}).

\ea
\label{bkm:Ref99545195}
  Semantics of class 12 nouns
\ea
\NumTabs{2}
\TabPositions{.25\textwidth, .6\textwidth}
Small items \\
kà-shòtò \tab ‘fish hook’\\
ká-nshèrèrè \tab ‘small mushroom sp.’\\
kà-nyùndwè \tab ‘pebble’\\
kà-shùtò \tab ‘fishing hook’\\
\ex Small animals \\
kà-nàmánì \tab ‘calf’\\
kà-cíyóꜝcíyò \tab ‘chick’\\
kà-bérèbèrè \tab ‘centipede’\\
kà-mbàryàmbàryà \tab ‘lizard sp.’\\
\ex Small body parts \\
kà-téntèrè \tab ‘xiphoid bone’\\
ká-ꜝnénsà \tab ‘pink, little toe’\\
kà-sîyè \tab ‘forehead wrinkle’\\
\ex Utensils \\
kà-tûò \tab ‘spoon’\\
kà-sûbà \tab ‘dish’\\
kà-róngò \tab ‘pot’\\
kà-nkúnè \tab ‘smoking shelf’ (for smoking foods, such as fish)\\
kà-fùrò \tab ‘knife’\\
kà-ìngà \tab ‘bowl’\\
\z\z


Class 12/13 is productively used to derive a diminutive from nouns that occur in other classes, as illustrated in (\ref{bkm:Ref498357099}).

\ea
\label{bkm:Ref498357099}

\ea
  class 1      mw-âncè    ‘child’\\
class 12    k-âncè    ‘small child’

\ex
  class 5      hànjà      ‘hand’\\
class 12    kà-hànjà    ‘small hand

\ex
  class 7      cì-púrà    ‘chair’\\
class 12    kà-púrà    ‘stool’

\ex
  class 9      n-jûò      ‘house’\\
class 12    kà-jûò      ‘small house’
\z\z

Nouns in this class may also be combined with the diminutive suffix \textit{-ána} (see \sectref{bkm:Ref450750897}).

Class 14 contains mainly words for abstract concepts, but also a few mass nouns, and a few words for types of trees, especially large trees. Examples are given in (\ref{bkm:Ref99545246}).

\ea
\label{bkm:Ref99545246}
  Semantics of class 14 nouns

\ea Abstract concepts \\
\NumTabs{2}
\TabPositions{.3\textwidth, .6\textwidth}
bú-sò \tab ‘front’\\
bù-hârò \tab ‘life’\\
bù-zûnzù \tab ‘loneliness’\\
bù-sîrù \tab ‘stupidity’\\
bù-shèbè \tab ‘gossip’\\
\ex Mass nouns \\
bûː-cì \tab ‘honey’\\
bw-ékè \tab ‘grains’\\
bù-sùnsò \tab ‘relish’\\
\ex Trees \\
bù-kwízyù \tab ‘fig tree’\\
bù-hómà \tab ‘mongongo tree (\textit{Schinziophyton rautanenii})’\\
bù-zyíyì \tab ‘tree (\textit{Berchemia discolor})’\\
\z\z

Class 14 is also used to derive abstract nouns from other nouns or from adjectives, as in (\ref{bkm:Ref99545300}).

\ea
\label{bkm:Ref99545300}

\ea
class 1    mù-ntù    ‘person’\\
class 14  bù-ntù    ‘humanity’

\ex
class 1    mù-ròzì    ‘witch’\\
class 14  bù-ròzì    ‘witchcraft’

\ex
class 1    mù-kúwà    ‘white person’\\
class 14  bù-kúwà    ‘town; any area dominated by white people’

\ex
  adjective  kûrù      ‘old’\\
class 14  bù-kûrù    ‘old age’

\ex
  adjective  rêː      ‘long’\\
class 14  bù-rêː      ‘length’
\z\z

Aside from infinitives, class 15 contains only four nouns, all referring to parts of the body (see (\ref{bkm:Ref444168488})) in \sectref{bkm:Ref498941936}). Some of these are being reassigned to class 5, e.g. \textit{kú-twì} ‘ear’ and \textit{kù-bôkò} ‘arm’ can also function as class 5 nouns, losing their class 15 prefix \textit{ku-}. The remainder of this class consists of infinitives, which can function as nouns: an infinitive can function as a subject, for instance, triggering class 15 subject agreement on the verb, as in (\ref{bkm:Ref99545334}).

\ea
\label{bkm:Ref99545334}
òkùhísà kwàndìkwángìsì\\
\gll o-ku-ís-a    ku-a-ndi-kwáng-is-i\\
\textsc{aug}-\textsc{inf}-burn-\textsc{fv}  \textsc{sm}\textsubscript{15}-\textsc{pst}-\textsc{om}\textsubscript{1SG}-tire-\textsc{caus}-\textsc{npst}.\textsc{pfv}\\
\glt ‘The heat has made me tired.’ (NF\_Elic15)
\z

Classes 16, 17 and 18 are locative classes. Very few nouns have inherent class 16, 17 or 18 membership, and these classes are mainly used derivationally; their semantics are discussed in \sectref{bkm:Ref452049189}.

\subsection{The locative noun classes}
\label{bkm:Ref452049189}\hypertarget{Toc75352640}{}
Class 16, 17 and 18 are locative classes; they indicate a location on (class 16), at (class 17) or in (class 18) an object. Only the root \textit{ntu} can take a locative prefix as its only nominal prefix, occuring as class 16 \textit{ha-ntu}, class 17 \textit{ku-ntu}, and class 18 \textit{mu-ntu}. This same nominal root also occurs in other, non-locative noun classes, e.g. class 1 \textit{mu-ntu} ‘person’, class 7 \textit{ci-ntu} ‘thing’, class 11 \textit{ru-ntu} ‘pupil (of the eye)’, and class 14 \textit{bu-ntu} ‘humanity’. To express a locative meaning with other nouns, the locative prefix is added before the noun’s own nominal prefix as a secondary prefix, as in (\ref{bkm:Ref99545419}--\ref{bkm:Ref99545420}).

\ea
\label{bkm:Ref99545419}
\glll hàmùkwàkwà\\
ha-mu-kwakwa\\
\textsc{np}\textsubscript{16}-\textsc{np}\textsubscript{3}-road\\
\glt ‘on the road’
\z

\ea
\glll kùrùwà\\
ku-ru-wa\\
\textsc{np}\textsubscript{17}-\textsc{np}\textsubscript{11}-field\\
\glt ‘at the field’
\z

\ea
\label{bkm:Ref99545420}
\glll mùmùnzì\\
mu-mu-nzi\\
\textsc{np}\textsubscript{18}-\textsc{np}\textsubscript{3}-village\\
\glt ‘in the village’
\z

The nouns \textit{ha-ntu / ku-ntu /mu-ntu} take the agreement pattern of the locative classes, as illustrated for the class 16 noun \textit{hàntù} ‘place’, in (\ref{bkm:Ref69986183}). Nouns that are marked with a secondary locative prefix, however, keep the agreement pattern of their original noun class, as illustrated with derived class 16 noun \textit{hàmùtwí} ‘on the head’ in (\ref{bkm:Ref450140939}), which triggers class 3 agreement on the following possessive pronoun.

\ea
\label{bkm:Ref69986183}
hàntù hònkêː\\
\gll ha-ntu  ha-o=nkéː\\
\textsc{np}\textsubscript{16}-place  \textsc{pp}\textsubscript{16}-\textsc{con}=one\\
\glt ‘one place, the same place’
\z

\ea
\label{bkm:Ref450140939}
hàmùtwí ꜝwángù\\
\gll ha-mu-twí    u-angú\\
\textsc{np}\textsubscript{16}-\textsc{np}\textsubscript{3}-head  \textsc{pp}\textsubscript{3}-\textsc{poss}\textsubscript{1SG}\\
\glt ‘on my head’
\z

When a noun has a prenominal modifier, the locative prefix is prefixed to this modifier, rather than to the noun itself, as illustrated in (\ref{bkm:Ref492026950}) with the possessive, which is pre-nominal when used contrastively (see \sectref{bkm:Ref491333327} on possessives), and in (\ref{bkm:Ref492026948}) with the demonstrative, whose canonical position is before the noun it modifies (see \sectref{bkm:Ref492026896} on demonstratives).

\ea
\label{bkm:Ref492026950}
mùwètú mùshòbò\\
\gll mu-u-etú    mu-shobo\\
\textsc{np}\textsubscript{18}-\textsc{pp}\textsubscript{3}-\textsc{poss}\textsubscript{1PL}  \textsc{np}\textsubscript{3}-language\\
\glt ‘in our language’
\z

\ea
\label{bkm:Ref492026948}
mòwíná mùnzì\\
\gll mu-o-winá  mu-nzi\\
\textsc{np}\textsubscript{18}-\textsc{dem}.\textsc{iv}\textsubscript{3}  \textsc{np}\textsubscript{3}-village\\
\glt ‘in that village’
\z

Locative prefixes are usually attached to augmentless forms, with two exceptions. Firstly, demonstratives retain their augment when marked with a locative prefix, as in (\ref{bkm:Ref97891916}--\ref{bkm:Ref71011804}).

\ea
\label{bkm:Ref97891916}
hèrìn’ éshâshà\\
\gll ha-e-riná    e-∅-shásha\\
\textsc{np}\textsubscript{16}-\textsc{aug}-\textsc{dem}.\textsc{iv}\textsubscript{5}  \textsc{aug}-\textsc{np}\textsubscript{5}-mat\\
\glt ‘on that mat’ (NF\_Elic17)
\z

\ea
\label{bkm:Ref71011804}
rìyá kwábà bàkázànà básìshèshìwâ\\
\gll ri-y-á̲    kú-a-ba    ba-kázana  bá̲-si\textsubscript{H}-she\textsubscript{H}sh-iw-á̲\\
\textsc{sm}\textsubscript{5}-go-\textsc{fv}  \textsc{np}\textsubscript{17}-\textsc{aug}-\textsc{dem}.\textsc{i}\textsubscript{2}  \textsc{np}\textsubscript{2}-lady  \textsc{sm}\textsubscript{2\-}.\textsc{rel}-\textsc{prs}-marry-\textsc{pass}-\textsc{fv}\\
\glt ‘It [the story] goes to these ladies who are not yet married.’ (NF\_Narr17)
\z

Secondly, in Namibian Fwe, nouns that take an augment \textit{e-}, and that lack a syllabic noun class prefix, e.g. those of class 5, 9 or 10, may retain the augment when combined with a locative prefix. The regular rules of vowel hiatus resolution apply (see \sectref{bkm:Ref491962181}), resulting in the forms \textit{ha- e- > he-} for class 16, as in (\ref{bkm:Ref98418736}) \textit{ku- e- > kwi-} for class 17, as in (\ref{bkm:Ref98418737}), and \textit{mu- e- > mwi-} for class 18, as in (\ref{bkm:Ref98418739}).

\ea
\label{bkm:Ref98418736}
ndìráːrà héshâshà\\
\gll ndi-rá̲ː\textsubscript{H}r-a    há-e-∅-shásha\\
\textsc{sm}\textsubscript{1SG}-sleep-\textsc{fv}  \textsc{np}\textsubscript{16}-\textsc{aug}-\textsc{np}\textsubscript{5}-mat\\
\glt ‘I sleep on a mat.’ (NF\_Elic15)
\z

\ea
\label{bkm:Ref98418737}
mbòndíshùmìn’ ómùhàrà kwítêndè\\
\gll mbo-ndí̲-shu\textsubscript{H}min-é̲    o-mu-hara    kú-e-∅-ténde\\
\textsc{near}.\textsc{fut}-\textsc{sm}\textsubscript{1SG}-tie-\textsc{pfv}.\textsc{sbjv}  \textsc{aug}-\textsc{np}\textsubscript{3}-rope  \textsc{np}\textsubscript{17}-\textsc{aug}-\textsc{np}\textsubscript{5}-foot\\
\glt ‘I will tie the rope to my foot.’ (NF\_Narr15)
\z

\ea
\label{bkm:Ref98418739}
kùshàmbà mwízìbà\\
\gll ku-shamb-a    mú-e-∅-ziba\\
\textsc{inf}-swim-\textsc{fv}   \textsc{np}\textsubscript{18}-\textsc{aug}-\textsc{np}\textsubscript{5}-lake\\
\glt ‘to swim in the lake’
\z

These forms are not found in Zambian Fwe, and even in Namibian Fwe, the change of \textit{ku-} and \textit{mu}- to \textit{kwi-} and \textit{mwi-} before \textit{e-} is optional; this could be related to the optional status of the augment vowel (see \sectref{bkm:Ref444175456}), where the \textit{ku-} and \textit{mu-} forms indicate that the noun is used without an augment.

The locative prefixes of class 17 and 18 have an allomorph that is used with names; \textit{kwa}- for class 17, as in (\ref{bkm:Ref98419180}), and \textit{mwa}- for class 18, as in (\ref{bkm:Ref98419181}). The locative prefix of class 16 \textit{ha-} remains unchanged when used with names, as in (\ref{bkm:Ref98419206}). Class 1a nouns other than names take the regular forms \textit{ha-, ku-} and \textit{mu}-, as shown for class 18 \textit{mu-} in (\ref{bkm:Ref98421091}).

\ea
\label{bkm:Ref98419180}
\glll hàmàkângà\\
ha-makánga\\
\textsc{np}\textsubscript{16}-Makanga\\
\glt ‘at Makanga’
\z

\ea
\label{bkm:Ref98419181}
\glll kwàmòngù\\
kwa-mongu\\
\textsc{np}\textsubscript{17}-Mongu\\
\glt ‘in Mongu’
\z

\ea
\label{bkm:Ref98419206}
\glll mwànàmìbìà\\
mwa-namibia\\
\textsc{np}\textsubscript{18}-Namibia\\
\glt ‘in Namibia’
\z

\ea
\label{bkm:Ref98421091}
\glll mùpótò\\
mu-∅-potó\\
\textsc{np}\textsubscript{18}-\textsc{np}\textsubscript{1a}-pot\\
\glt ‘in the pot’
\z

The three locative noun classes each have their own semantics. Class 16 is used to mark a location on something, as in (\ref{bkm:Ref71013022}--\ref{bkm:Ref71013023}), or a more general location at or near something, as in (\ref{bkm:Ref450663404}--\ref{bkm:Ref450663405}).

\ea
\label{bkm:Ref71013022}
kúkàrà hácìpúrà\\
\gll kú-kar-a  há-ci-purá\\
\textsc{inf}-sit-\textsc{fv}  \textsc{np}\textsubscript{16}-\textsc{np}\textsubscript{7}-chair\\
\glt ‘to sit on a chair’
\z

\ea
àrâːrà hámùmbétà\\
\gll a-rá̲ː\textsubscript{H}r-a  há-mu-mbetá\\
\textsc{sm}\textsubscript{1}-sleep-\textsc{fv}  \textsc{np}\textsubscript{16}-\textsc{np}\textsubscript{3}-bed\\
\glt ‘S/he sleeps on the bed.’ (NF\_Elic15)
\z

\ea
\label{bkm:Ref71013023}
àkéːzyà kùzyímànà hékàmbà\\
\gll a-ké̲ːzy-a    ku-zyíman-a    há-e-∅-kamba\\
\textsc{sm}\textsubscript{1}-come-\textsc{fv}  \textsc{inf}-stand-\textsc{fv}    \textsc{np}\textsubscript{16}-\textsc{aug}-\textsc{np}\textsubscript{5}-bank\\
\glt ‘He comes to stand on the river bank.’ (NF\_Narr15)
\z

\ea
\label{bkm:Ref450663404}
tùzânà hámùkítì\\
\gll tu-zá̲n-a    há-mu-kití\\
\textsc{sm}\textsubscript{1PL}-dance-\textsc{fv}  \textsc{np}\textsubscript{16}-\textsc{np}\textsubscript{3}-party\\
\glt ‘We dance at the party.’
\z

\ea
\label{bkm:Ref450663405}
àzyíménè hácìzyì\\
\gll a-zyi\textsubscript{H}mé̲ne    há-ci-zyi\\
\textsc{sm}\textsubscript{1}-stand.\textsc{stat}  \textsc{np}\textsubscript{16}-\textsc{np}\textsubscript{7}-door\\
\glt ‘S/he stands at the door.’ (NF\_Elic15)
\z

When combined with the verb \textit{zw} ‘come out’, the class 16 locative can be used to indicate a motion away from an original point, as in (\ref{bkm:Ref99545696}).

\ea
\label{bkm:Ref99545696}
àmàròhà àzwá hàcìrábì\\
\gll a-ma-roha    a-zw-á̲    ha-ci-rabí\\
\textsc{aug}-\textsc{np}\textsubscript{6}-blood  \textsc{sm}\textsubscript{6}-come\_out-\textsc{fv}  \textsc{np}\textsubscript{16}-\textsc{np}\textsubscript{7}-wound\\
\glt ‘Blood comes from the wound.’ (NF\_Elic15)
\z

The class 17 locative is mostly used to express a more general location at or near something, as in (\ref{bkm:Ref492028330}--\ref{bkm:Ref492028331}), or a direction, as in (\ref{bkm:Ref492028332}).

\ea
\label{bkm:Ref492028330}
àbâncè kùcìkóró kábàkénà shûnù\\
\gll a-ba-ánce    ku-ci-koró    ka-bá̲-kena    shúnu\\
\textsc{aug}-\textsc{np}\textsubscript{2}-child  \textsc{np}\textsubscript{17}-\textsc{np}\textsubscript{7}-school  \textsc{pst}.\textsc{ipfv}-\textsc{sm}\textsubscript{2}-be\_at  today\\
\glt ‘The children were at school today.’ (ZF\_Elic14)
\z

\ea
\label{bkm:Ref492028331}
ndàmùsíyì kù kùnjìrà\\
\gll ndi-a-mu-sí-i      ku    ku-N-jira\\
\textsc{sm}\textsubscript{1SG}-\textsc{pst}-\textsc{om}\textsubscript{1}-leave-\textsc{npst}.\textsc{pfv}  \textsc{dem}.\textsc{i}\textsubscript{17}  \textsc{np}\textsubscript{17}-\textsc{np}\textsubscript{9}-path\\
\glt ‘I’ve left him there, on the path.’ (ZF\_Narr13)
\z

\ea
\label{bkm:Ref492028332}
ndìyá ꜝkúmùnzì\\
\gll ndi-y-á̲  kú-mu-nzi\\
\textsc{sm}\textsubscript{1SG}-go-\textsc{fv}  \textsc{np}\textsubscript{17}-\textsc{np}\textsubscript{3}-village\\
\glt ‘I go home.’ (NF\_Elic15)
\z

The class 18 locative is used to express a location inside something, as in (\ref{bkm:Ref492028402}--\ref{bkm:Ref492028403}). With verbs of motion, the class 18 locative expresses a movement into, or out of, a location inside an object, as in (\ref{bkm:Ref492028406}--\ref{bkm:Ref492028408}).

\ea
\label{bkm:Ref492028402}
ndìkèrè múnjûò\\
\gll ndi-ke\textsubscript{H}re    mú-N-júo\\
\textsc{sm}\textsubscript{1SG}-sit.\textsc{stat}  \textsc{np}\textsubscript{18}-\textsc{np}\textsubscript{9}-house\\
\glt ‘I’m sitting in the house.’ (NF\_Elic17)
\z

\ea
\label{bkm:Ref492028403}
ècìkúnì càkùrí kùdánsì mùnjîrà\\
\gll e-ci-kúni    ci-aku-rí    ku-dá̲ns-i    mu-N-jíra\\
\textsc{aug}-\textsc{np}\textsubscript{7}-stick  \textsc{sm}\textsubscript{7}-\textsc{npst}.\textsc{ipfv}-be  \textsc{inf}-lie-\textsc{imp}.\textsc{stat}  \textsc{np}\textsubscript{18}-\textsc{np}\textsubscript{9}-path\\
\glt ‘The stick was lying on the path.’
\z

\ea
\label{bkm:Ref492028406}
àshòtòkérá mùmênjì\\
\gll a-sho\textsubscript{H}tok-er-á̲  mu-ma-ínji\\
\textsc{sm}\textsubscript{1}-jump-\textsc{appl}-\textsc{fv}  \textsc{np}\textsubscript{18}-\textsc{np}\textsubscript{6}-water\\
\glt ‘S/he jumps into the water.’
\z

\ea
\label{bkm:Ref492028408}
òzwé mùkàmwî\\
\gll o-zw-é̲      mu-ka-mwí\\
\textsc{sm}\textsubscript{2SG}-come\_out-\textsc{pfv}.\textsc{sbjv}  \textsc{np}\textsubscript{18}-\textsc{np}\textsubscript{12}-sun\\
\glt ‘Come out of the sun.’ (NF\_Elic15)
\z

The locative prefixes also have a number of non-locative uses. The class 16 and 18 locatives can be used to express a location in time, as in (\ref{bkm:Ref99545724}--\ref{bkm:Ref99545725}). The temporal use of class 16 is also seen in the demonstrative of class 16 (see \sectref{bkm:Ref492026896} on demonstratives).

\ea
\label{bkm:Ref99545724}
\glll hàrùmwî\\
ha-ru-mwí\\
\textsc{np}\textsubscript{16}-\textsc{np}\textsubscript{11}-summer\\
\glt ‘in summer’
\z

\ea
\label{bkm:Ref99545725}
mùnàkò yómvûrà\\
\gll mu-N-nako    i-ó=∅-mvúra\\
\textsc{np}\textsubscript{18}-\textsc{np}\textsubscript{9}-time  \textsc{pp}\textsubscript{9}-\textsc{con}=\textsc{np}\textsubscript{1a}-rain\\
\glt ‘in the rainy season’
\z

The class 17 locative can be used to express a partitive, as in (\ref{bkm:Ref485812455}). It can also be used to mark a polite request, as in (\ref{bkm:Ref485812447}); this use is related to its partitive use, e.g. the request for the phone is “softened” by asking for only part of the phone. The use of class 17 to express a partitive or polite request is also seen with the class 17 locative clitic \textit{-ko} (see \sectref{bkm:Ref451257563} on locative clitics).

\ea
\label{bkm:Ref485812455}
bàtòmá ꜝkwínyàmà\\
\gll ba-tom-á̲  kú-e-N-nyama\\
\textsc{sm}\textsubscript{2}-share-\textsc{fv}  \textsc{np}\textsubscript{17}-\textsc{aug}-\textsc{np}\textsubscript{9}-meat\\
\glt ‘S/he shares from the meat.’
\z

\ea
\label{bkm:Ref485812447}
ndìóːr’ òkùkárìmà kwífòní ꜝyénù\\
\gll ndi-ó̲ːr-a  o-ku-kárim-a    kú-e-∅-foní      i-enú\\
\textsc{sm}\textsubscript{1SG}-can-\textsc{fv}  \textsc{aug}-\textsc{inf}-borrow-\textsc{fv}  \textsc{np}\textsubscript{17}-\textsc{aug}-\textsc{np}\textsubscript{9}-phone  \textsc{pp}\textsubscript{9}-\textsc{poss}\textsubscript{2PL}\\
\glt ‘Can I borrow your phone?’ (NF\_Elic17)
\z

The class 17 locative \textit{ku-} can be used to mark an agent in a construction where an agent cannot be marked as a core argument. This is the case, for instance, for verbs with the passive derivation, as in (\ref{bkm:Ref450665409}), or nouns, as in (\ref{bkm:Ref450665567}). The class 17 prefix \textit{ku-} may also be used to express less canonical agents, as in (\ref{bkm:Ref450666359}), or even peripheral arguments functioning as a reason or circumstance, rather than an agent, as in (\ref{bkm:Ref498346503}). The agentive use of the class 17 prefix is also seen in various other Bantu languages \citep{Fleisch2005}.

\ea
\label{bkm:Ref450665409}
nàshúmìwà \textbf{kúmbwà}\\
\gll na-shúm-iw-a    kú-∅-mbwá\\
\textsc{sm}\textsubscript{1}.\textsc{pst}-bite-\textsc{pass}-\textsc{fv}  \textsc{np}\textsubscript{17}-\textsc{np}\textsubscript{1a}-dog\\
\glt ‘He was bitten \textbf{by} \textbf{a} \textbf{dog}.’ (NF\_Elic17)
\z

\ea
\label{bkm:Ref450665567}
ndóꜝrúfù rùbànyámùzàmbàràrà \textbf{kúnjòvù}\\
\gll ndó-ru-fú      ru-ba-nyá-muzambarara    kú-∅-njovu\\
\textsc{cop}.\textsc{def}\textsubscript{11}-\textsc{np}\textsubscript{11}-death  \textsc{pp}\textsubscript{11}-\textsc{np}\textsubscript{2}-mother-Muzambarara  \textsc{np}\textsubscript{17}-\textsc{np}\textsubscript{1a}-elephant\\
\glt ‘That is the death of Mrs. Muzambarara \textbf{by} \textbf{the} \textbf{elephant}.’ (ZF\_Narr15)
\z

\ea
\label{bkm:Ref450666359}
ècìzyábáró ꜝcángù càbúrûkì \textbf{kúꜝ}\textbf{rúːhò}\\
\gll e-ci-zyabaró    ci-angú ci-a-bur-ú̲k-i        kú-rúː-ho\\
\textsc{aug}-\textsc{np}\textsubscript{7}-shirt  \textsc{pp}\textsubscript{7}-\textsc{poss}\textsubscript{1SG} \textsc{sm}\textsubscript{7}-\textsc{pst}-blow-\textsc{sep}.\textsc{intr}-\textsc{npst}.\textsc{pfv}  \textsc{np}\textsubscript{17}-\textsc{np}\textsubscript{11}-wind\\
\glt ‘My shirt was blown away \textbf{by} \textbf{the} \textbf{wind}.’ (NF\_Elic15)
\z

\ea
\label{bkm:Ref498346503}
èzìzwátò zìnàbómbì \textbf{kúmvûrà}\\
\gll e-zi-zwáto    zi-na-bó̲mb-i      kú-∅-mvúra\\
\textsc{aug}-\textsc{np}\textsubscript{8}-cloth  \textsc{sm}\textsubscript{8}-\textsc{pst}-become\_wet-\textsc{npst}.\textsc{pfv}  \textsc{np}\textsubscript{17}-\textsc{np}\textsubscript{1a}-rain\\
\glt ‘The clothes have become wet \textbf{because} \textbf{of} \textbf{the} \textbf{rain}.’ (NF\_Elic15)
\z
\subsection{Noun class assignment of loanwords}\largerpage
\label{bkm:Ref452049199}\hypertarget{Toc75352641}{}
Because every noun in Fwe belongs to a noun class, new words that enter the language through borrowing also need to be assigned to a noun class. This section is about the principles that are used in noun class assignment of loanwords. Differences are observed between loanwords originating from other Bantu languages, which also have a noun class system often quite similar in form and function to that of Fwe, and loanwords originating from non-Bantu languages, which lack noun classes. Borrowings from Bantu languages are often assigned to the noun class whose prefix is formally most similar to the prefix of the borrowed word. Borrowings from non-Bantu languages use other processes, notably assignment to a default class, but also the more uncommon process of paralexification \citep{GunninkEtAl2015}.

Fwe has borrowed extensively from Lozi, and a small number of words can be identified as borrowings from Mbukushu and Yeyi. Loanwords from other Bantu languages, such as Totela, Subiya and Shanjo, are likely to exist but difficult to identify. This is due to the limited lexical documentation of these languages, but also their close genealogical relationship to Fwe, which makes such borrowings difficult to distinguish from native Fwe words.

As can be seen from \tabref{tab:4:3}, Lozi borrowings are usually incorporated into the same noun class in Fwe as in Lozi. For most classes, this may simply be the result of the similar forms of nominal prefixes, for instance, for class 1 and 3, where the prefix is \textit{mu-} in both Fwe and Lozi, or class 7, where the prefix is \textit{ci-} in Fwe and \textit{si-} in Lozi. However, borrowed nouns also retain their noun class when Fwe and Lozi do not have similar nominal prefixes. This is the case for nouns of class 5, where Fwe has a zero prefix but Lozi uses the prefix \textit{li-}. The assignment of nouns that are in class 5 in Lozi to class 5 in Fwe may be the result of their plural; in both Lozi and Fwe the plural corresponding to class 5 takes the class 6 prefix \textit{ma-}. The assignment of borrowings to corresponding noun classes, even in the absence of a similar nominal prefix, may be the result of the fairly extensive Fwe-Lozi bilingualism in Fwe-speaking communities.

\begin{table}
\label{bkm:Ref492118585}\caption{\label{tab:4:3}Lozi loanwords in Fwe}
\begin{tabular}{llllll}
\lsptoprule
& Fwe &  &  & Lozi & \\\cmidrule(lr){1-3}\cmidrule(lr){4-6}
1 & \textit{mù-rútì} & ‘teacher’ & 1 & \textit{mu-luti} & ‘teacher’\\
3 & \textit{mù-ràhò} & ‘law’ & 3 & \textit{mu-lao} & ‘law’\\
3 & \textit{mù-râkà} & ‘kraal’ & 3 & \textit{mu-laka} & ‘kraal’\\
5 & \textit{rápà} & ‘courtyard’ & 5 & \textit{li-lapa} & ‘courtyard’\\
5 & \textit{zúpà} & ‘wet clay’ & 5 & \textit{li-zupa} & ‘clay’\\
5 & \textit{kòndè} & ‘banana’ & 5 & \textit{li-konde} & ‘banana’\\
7 & \textit{cì-pátù} & ‘duck’ & 7 & \textit{si-pato} & ‘duck’\\
7 & \textit{cì-rìmò} & ‘season, year’ & 7 & \textit{si-limo} & ‘year’\\
9 & \textit{nyàzì} & ‘lover’ & 9 & \textit{nyazi} & ‘concubine’\\
\lspbottomrule
\end{tabular}
\end{table}

Fwe has also borrowed words from various Khoisan languages, notably the Khoe language (West-Caprivi) Khwe, and the Kx’a language Ju \citep{GunninkEtAl2015}. As the donor language is not a Bantu language, formal similarities between the noun class system of the donor language and that of Fwe cannot play a role in noun class assignment. Instead, many Khoisan borrowings in Fwe are assigned to a noun class on the basis of the noun class of a semantically similar or identical native Fwe word, such as Fwe \textit{mú-ⁿǀùryà} ‘type of lizard’, which is assigned to noun class 3 on the basis of its synonym \textit{mù-shúndùkìrè}, a native Fwe word with the same meaning which is also in class 3 (\citealt{GunninkEtAl2015}: 207). This process is referred to as ‘paralexification’ \citep{Mous2001}, and is not commonly used as a strategy for noun class assignment of borrowings by Bantu languages. The paralexification of Khoisan borrowings in Fwe and related languages, and the implications this has for the analysis of the contact situation, are discussed in {\citet{GunninkEtAl2015}}. Not all Khoisan borrowings are assigned to a noun class on the basis of the paralexification of an existing noun; examples where evidence for paralexification is lacking (though it may have taken place on the basis of a noun that has since been lost) are given in \tabref{tab:4:4}.

\begin{table}
\label{bkm:Ref492118612}\caption{\label{tab:4:4}Possible Khwe and Ju (!Xung/!Xun/!Xuun/Ju{\textbar}’hoan) loanwords in Fwe}
\begin{tabularx}{\textwidth}{lllQ}
\lsptoprule
noun class & Fwe word & translation & putative source word\\
\midrule
3 & \textit{mù-gwégwèsì} & ‘ankle bone’ & \textit{gwéː} ‘ankle’ (Neitsas/Nurugas !Xung, \citealt{Doke1925})\\
&  &  & \textit{ǂ’hòèǂ’hòrè} ‘ankle bone’ (Ju{\textbar}'hoan, \citealt{Snyman1975}: 107)\\
5 & \textit{shèngà} & ‘liver’ & \textit{c\'{ŋ}} ‘liver’ (Northwestern !Xun, \citealt{KönigHeine2008}: 18)\\
&  &  & \textit{tchín (ka)} ‘liver’ (Ju{\textbar}'hoan, \citealt{Dickens1994}: 108)\\
&  &  & \textit{ʃŋ}̱ ‘liver’ (Central !Xuun, \citealt{Doke1925})\\
11 & \textit{rù-kânì} & ‘jaw’ & \textit{gǁȁŋ} ‘chin’ (Northwestern !Xun, \citealt{KönigHeine2008}: 34) \\
&  &  & \textit{g!aihn} ‘chin’ (Ju{\textbar}'hoan, \citealt{Dickens1994}: 54)\\
&  &  & \textit{gyànìí} ‘chin’ (Khwe, \citealt{Kilian-Hatz2003}: 51)\\
\lspbottomrule
\end{tabularx}
\end{table}

Fwe has also borrowed from English and Afrikaans, as listed in \tabref{tab:4:5}. These borrowings are usually assigned to class 5 or 9, both noun classes with minimal morphological marking.

\begin{table}
\label{bkm:Ref99545807}\caption{\label{tab:4:5}English and Afrikaans loanwords in Fwe}
\begin{tabularx}{\textwidth}{lXXl}
\lsptoprule
noun class & Fwe word & translation & putative source word\\
\midrule
5 & \textit{bòtêrà} & ‘bottle’ & English \textit{bottle}\\
5 & \textit{bùkà} & ‘book’ & English \textit{book}\\
5 & \textit{fônì} & ‘phone’ & English \textit{phone}\\
9 & \textit{n-kèrékè} & ‘church’ & Afrikaans \textit{kerk}\\
9 & \textit{bbórà} & ‘ball’ & English \textit{ball}\\
9 & \textit{n-díshì} & ‘dish’ & English \textit{dish}\\
9 & \textit{n-súndà} & ‘week’ & Afrikaans \textit{sondag} ‘Sunday’\\
\lspbottomrule
\end{tabularx}
\end{table}

The only example of a borrowed noun assigned to class 1a is the English borrowing \textit{pótò} ‘pot’, which functions as a class 1a noun in Zambian Fwe, as in (\ref{bkm:Ref99545855}), but as a class 9 noun in Namibian Fwe, as in (\ref{bkm:Ref99545857}), as seen by their respective agreement patterns.

\ea
\label{bkm:Ref99545855}
òzyú ꜝpótò\\
\gll o-zyú    ∅-potó\\
\textsc{aug}-\textsc{dem}.\textsc{i}\textsubscript{1}  \textsc{np}\textsubscript{1a}-pot\\
\glt ‘this pot’ (Zambian Fwe)
\z\largerpage

\ea
\label{bkm:Ref99545857}
èyí ꜝmpótò\\
\gll e-í    N-potó\\
\textsc{aug}-\textsc{dem}.\textsc{i}\textsubscript{9}  \textsc{np}\textsubscript{9}-pot\\
\glt ‘this pot’ (Namibian Fwe)
\z

English or Afrikaans words are not necessarily direct borrowings in Fwe, but can also be borrowed via Lozi, as direct contact between Fwe and both English and Afrikaans is more limited than that between Fwe and Lozi. This also means that the way in which these borrowings are integrated into the Fwe noun class system may have followed the Lozi pattern rather than the Fwe pattern.

\section{Word formation}
\label{bkm:Ref70948964}\hypertarget{Toc75352642}{}\label{bkm:Ref451514817}
Fwe has a number of strategies to create new nouns from existing nominal or verbal stems. Verb-to-noun derivation makes use of various suffixes, as discussed in \sectref{bkm:Ref444251366}. Noun-to-noun derivation, discussed in \sectref{bkm:Ref450750897}, is done through various affixes. Noun class shift is also productively used to derive new meanings from nominal roots; this process has been discussed in \sectref{bkm:Ref450747802} on the semantics of noun classes. Nominal compounding and reduplication are also used as strategies for word formation, though both processes are unproductive.

\subsection{Verb-to-noun derivation}
\label{bkm:Ref444251366}\hypertarget{Toc75352643}{}
Nouns can be derived from verbs by the addition of the suffixes \textit{-i, -o, -u}, -\textit{e}, or \textit{-a}, which are common Bantu suffixes (\citealt{SchadebergBostoen2019}), or \textit{-ntu}, which is a Fwe innovation. These derivational suffixes differ in function and productivity, as summarized in \tabref{tab:4:6}, which gives an overview of the deverbal derivational suffixes, their functions and their productivity.

\begin{table}
\label{bkm:Ref463014744}\caption{\label{tab:4:6}Deverbal suffixes}
\fittable{
\begin{tabular}{lll}
\lsptoprule
Form & Function & Productivity\\
\midrule
-\textit{i} & agentive (human) & mostly productive\\
-\textit{o} & instrumental, patientive, action, result, place, time & mostly productive\\
{\itshape -ntu} & general nominalizer & mostly productive\\
{\itshape -u} & instrumental, patientive, abstract & unproductive\\
-\textit{a} & instrumental, patientive, agentive (non-human) & unproductive\\
-\textit{e} & instrumental, agentive (non-human) & unproductive\\
\lspbottomrule
\end{tabular}
}
\end{table}

Deverbal nouns typically retain the tonal profile of the corresponding verb, but there are also occasional tonal mismatches; these are especially common with the less productive deverbal suffixes. \tabref{tab:4:7} illustrates both patterns.

\begin{table}
\label{bkm:Ref74564279}\caption{\label{tab:4:7}Tone in derived nouns}
\begin{tabularx}{\textwidth}{XXXl}
\lsptoprule
\multicolumn{4}{l}{Maintenance of lexical tone}\\
\midrule
\textit{bûmbà} & ‘make pottery’ & \textit{mù-bûmbì} & ‘potter’\\
\textit{rôːtà} & ‘dream’ & \textit{cì-rôːtò} & ‘dream’\\
\textit{kú-fwà} & ‘die’ & \textit{rú-fù} & ‘death’\\
\textit{kákàtìrà} & ‘stick’ & \textit{rù-kákàtìrà} & ‘burdock’\\
\midrule
\multicolumn{4}{l}{Changes in lexical tone}\\
\midrule
\textit{tùsà} & ‘help’ & \textit{n-túsò} & ‘help’\\
\textit{fûrà} & ‘sharpen, weld’ & \textit{kà-fùrò} & ‘knife’\\
\textit{kòhà} & ‘blink’ & \textit{n-kôhè} (cl 10) & ‘eyelids’\\
\textit{tùkà} & ‘insult’ & \textit{mà-tûkà} & ‘insults’\\
\textit{tár-ùk-à} & ‘take a step’ & \textit{mù-tàrà} & ‘footprint’\\
\lspbottomrule
\end{tabularx}
\end{table}

Deverbal nouns may also incorporate verbal derivational suffixes, such as the causative or applicative. In some cases, the corresponding verb is also attested with the same derivational suffix, whereas in others, the verbal derivational suffix is only attested in the derived noun. Examples are given in \tabref{tab:4:8}.

\begin{table}
\label{bkm:Ref74564290}\caption{\label{tab:4:8}Deverbal nouns incorporating a verbal derivational suffix}
\begin{tabular}{llll}
\lsptoprule
\multicolumn{2}{l}{Base verb}  & \multicolumn{2}{l}{Derived noun}  \\
\midrule
\textit{rêːtà}  & ‘give birth’ & \textit{mù-réːt-ìs-ì} & ‘midwife’\\
\textit{ùr-ìs-à}  & ‘sell’ (cf. \textit{ùr-à} ‘buy’) & \textit{mù-ùr-ìs-ì} & ‘seller’\\
\textit{yènd-ès-à}  & ‘guide’ (cf. \textit{yènd-à} ‘walk’) & \textit{mù-yènd-ès-ì} & ‘supervisor’\\
\textit{tôngà} & ‘become ill’ & \textit{mà-tòng-êr-à} & ‘illness’\\
\textit{shèngà} & ‘sharpen’ & \textit{mù-shèng-èr-à} & ‘sharp tip’\\
\textit{tìmbà} & ‘push’ & \textit{n-tìmb-ìr-à} & ‘dung beetle’\\
\'{} \textit{àrà} & ‘close’ & \textit{cí-àr-ìs-ò} & ‘door’\\
\textit{bbùkùrà} & ‘blow on fire’ & \textit{cì-bbùkùr-ìs-ò} & ‘bellows’\\
\textit{fwìnkà} & ‘plug with a stopper’ & \textit{cì-fwìnk-ìs-ò} & ‘stopper’\\
\lspbottomrule
\end{tabular}
\end{table}

Occasionally, a verbal suffix that is obligatorily present in the verb is absent in the corresponding noun. This is especially the case with the less productive deverbal suffixes; examples are given in \tabref{tab:4:9}.

\begin{table}
\label{bkm:Ref74564302}\caption{\label{tab:4:9}Absence of verbal derivational suffixes in deverbal nouns}
\begin{tabularx}{\textwidth}{lQlQ}
\lsptoprule
Base verb &  & Derived noun & \\
\midrule
\textit{kùmb-ùr-à} & ‘cut strips (as building material)’ & \textit{mà-kùmb-à} & ‘strips (for building)’\\
\textit{kúzy-ùr-à} & ‘peel a mongongo nut’ & \textit{∅-kùzy-à} (cl 5) & ‘outer peel of a mongongo nut’\\
\textit{shèb-èk-à} & ‘gossip’ & \textit{bù-shèb-è} & ‘gossip’\\
\textit{shémp-èk-à} & ‘shoulder a load’ & \textit{mù-shêmp-ù} & ‘load’\\
\textit{súmb-àr-à} & ‘become pregnant’ & \textit{bù-sûmb-à} & ‘pregnancy’\\
\lspbottomrule
\end{tabularx}
\end{table}

The suffix \textit{-i} derives an agent noun from a verb, indicating ‘a person who does X’, as shown in \tabref{tab:4:10}. On account of the noun referring to a human being, the noun is usually assigned to noun class 1. Derivation with the suffix \textit{-i} is fairly productive: it can be used with most verbs, always deriving an agentive noun.

\begin{table}
\label{bkm:Ref488765104}\caption{\label{tab:4:10}Agent nouns derived with \textit{-i}}
\begin{tabular}{llll}
\lsptoprule
\multicolumn{2}{l}{Base verb}  & \multicolumn{2}{l}{Derived noun}\\\midrule
\textit{bàrà}  & ‘read’ & \textit{mù-bàrì} & ‘reader’\\
\textit{fùmà}  & ‘become rich’ & \textit{mù-fùmì} & ‘rich person’\\
\textit{fûrà}  & ‘sharpen, weld’ & \textit{mù-fûrì} & ‘blacksmith’\\
\textit{rwà}  & ‘fight’ & \textit{mù-rwì} & ‘fighter’\\
\textit{zyâːkà}  & ‘build’ & \textit{mù-zyâːkì} & ‘builder’\\
\lspbottomrule
\end{tabular}
\end{table}

There are five words where the agentive suffix \textit{-i} causes the preceding consonant to change to /z/, listed in \tabref{tab:4:11}.

\begin{table}
\label{bkm:Ref488764705}\caption{\label{tab:4:11}Agent nouns with spirantization}
\begin{tabular}{llll}
\lsptoprule
\multicolumn{2}{l}{Base verb} & \multicolumn{2}{l}{Derived noun} \\\midrule
\textit{fûrà}  & ‘forge’ & \textit{mù-fûzì} {\textasciitilde} \textit{mù-fûrì} & ‘blacksmith’\\
\textit{fwèbà}  & ‘smoke’ & \textit{mù-fwèzì} & ‘smoker’\\
\textit{kúmbìrà}  & ‘beg, request’ & \textit{∅-nkúmbìzì} & ‘beggar’\\
\textit{ròwà}  & ‘perform witchcraft’ & \textit{mù-ròzì} & ‘witch’\\
\textit{yàà}  & ‘kill’ & \textit{cí-yàzì} & ‘traitor’\\
\lspbottomrule
\end{tabular}
\end{table}

The change to /z/ in the agent noun is a lexicalized trace of the earlier sound change of Bantu Spirantization, the change from stops to fricatives before high vowels; in Fwe, this sound change has changed all voiced stops to /z/ before the reconstructed high vowel *i \citep[117-118]{Bostoen2009}. In words other than those listed in \tabref{tab:4:11}, the agentive suffix \textit{-i} does not cause spiranitzation of the final consonant of the verb root (see the examples in \tabref{tab:4:10}). Spirantization in agent nouns is not phonologically determined; the verb roots that undergo spirantization end in a different consonants, and other verb roots ending in the same consonant do not undergo spirantization. Instead, this is a case of what {\citet{Bostoen2008}} calls ‘limited agent noun spirantization’: spirantization is only attested in a handful of nouns derived with the agentive suffix \textit{-i,} and most nouns derived with this suffix do not undergo spirantization. Interestingly, in languages where only a handful of nouns undergo agent noun spirantization, the same nouns are often affected, especially reflexes of *-dògì ‘witch’ and *-jíbì ‘thief’. In Fwe the reflex of *-dògì ‘witch’, \textit{mù-ròzì} ‘witch’, is in fact one of the nouns undergoing spirantization. The reflex of *-jíbì ‘thief’ was lost in Fwe, probably as it was replaced by the borrowing \textit{mù-sâ} ‘thief’.

The suffix \textit{-o} derives instrumental nouns from verbs, as shown in \tabref{tab:4:12}. Nouns derived with this suffix are assigned to various noun classes, though never to class 1/2; class 7/8 seems to be the most common choice.

\begin{table}
\label{bkm:Ref98855579}\caption{\label{tab:4:12}Nouns derived with -\textit{o}}
\begin{tabular}{llll}
\lsptoprule
\multicolumn{2}{l}{Base verb} & \multicolumn{2}{l}{Derived noun in class 3/4}\\
\midrule
\textit{nùnkà} & ‘smell’ & \textit{mù-nùnkò} & ‘(bad) smell’\\
\tablevspace
\multicolumn{2}{l}{} & \multicolumn{2}{l}{Derived noun in class 5/6}\\
\midrule
\textit{shândà} & ‘suffer’ & \textit{shândò} & ‘suffering’\\
\tablevspace
\multicolumn{2}{l}{} & \multicolumn{2}{l}{Derived noun in class 7/8}\\
\midrule
\textit{tèndà} & ‘do, make’ & \textit{cì-tènd}ò & ‘action’\\
\textit{zànà} & ‘play’ & \textit{cì-zànò} & ‘game’\\
\textit{zwâtà} & ‘dress’ & \textit{cì-zwâtò} & ‘garment’\\
\textit{zyàbàrà} & ‘dress’ & \textit{ci-zyàbàrò} & ‘bottom garment’\\
\textit{zyàrà} & ‘spread a bed’ & \textit{cì-zyàrò} & ‘mat’\\
\tablevspace
\multicolumn{2}{l}{} & \multicolumn{2}{l}{Derived noun in class 9}\\
\midrule
\textit{tùsà} & ‘help’ & \textit{n-túsò} & ‘help’\\
\textit{súrùmùkà} & ‘descend’ & \textit{n-súrùmùkò} & ‘downward slope’\\
\tablevspace
\multicolumn{2}{l}{} & \multicolumn{2}{l}{Derived noun in class 11}\\
\midrule
\textit{zyîmbà} & ‘sing’ & \textit{rù-zyîmbò} & ‘song’\\
\tablevspace
\multicolumn{2}{l}{} & \multicolumn{2}{l}{Derived noun in class 12}\\
\midrule
\textit{fûrà} & ‘sharpen, weld’ & \textit{kà-fùrò} & ‘knife’\\
\tablevspace
\multicolumn{2}{l}{} & \multicolumn{2}{l}{Derived noun in class 14}\\
\midrule
\textit{hârà} & ‘live’ & \textit{bù-hârò} & ‘life’\\
\textit{sùnsà} & ‘dip porridge in relish’ & \textit{bù-sùnsò} & ‘relish’\\
\lspbottomrule
\end{tabular}
\end{table}

Semantically, most nouns derived with \textit{-o} refer either to the patient or the instrument of the verb. Less commonly, the derivational suffix \textit{-o} derives a noun referring to a place, a time or a result of the action described by the verb, or the action itself. \tabref{tab:4:13} gives an overview of the different meanings of nouns derived with \textit{-o}.

\begin{table}
\label{bkm:Ref490833448}\caption{\label{tab:4:13}Semantics of nouns derived with -o}

\begin{tabularx}{\textwidth}{lllQ}
\lsptoprule
\multicolumn{4}{c}{Patient nouns with \textit{-o}} \\
\midrule
\textit{zyîmbà}  & ‘sing’ & \textit{rù-zyîmbò} & ‘song’\\
\textit{ryà}  & ‘eat’ & \textit{zí-ryò} & ‘food, crops’\\
\tablevspace
\multicolumn{4}{c}{Instrumental nouns with \textit{-o}} \\
\midrule
\textit{bèːzyà} & ‘carve (wood)’ & \textit{m-bèzyò} & ‘small axe for making surfaces smooth’\\
\textit{shùtà}  & ‘fish (with line)’ & \textit{kà-shùtò} & ‘fish hook’\\
\tablevspace
\multicolumn{4}{c}{Action}\\
\midrule
\textit{èndà}  & ‘go, travel’ & \textit{rù-yèndò} & ‘journey’\\
\textit{tèndà}  & ‘do’ & \textit{cì-tèndò} & ‘action’\\
\tablevspace
\multicolumn{4}{c}{Result}\\
\midrule
\textit{ùrà}  & ‘buy’ & \textit{n-gùr-ìs-ò} & ‘profit’\\
\textit{zyàmbìrà}  & ‘gather’ & \textit{zì-zyàmbìrò} & ‘gathered fruits’\\
\tablevspace
\multicolumn{4}{c}{Place} \\
\midrule
\textit{hòndà}  & ‘cook’ & \textit{mà-hònd-èr-ò} & ‘kitchen’\\
\textit{rí-zìkà}  & ‘hide oneself’ & \textit{mà-rí-ꜝzíkò} & ‘hiding place’\\
\tablevspace
\multicolumn{4}{c}{Time}\\
\midrule
\textit{rìmà}  & ‘cultivate, farm’ & \textit{cì-rìmò} & ‘season, year’\\
\lspbottomrule
\end{tabularx}
\end{table}

In some cases, non-systematic formal differences can be observed between the derived noun and its verbal source, such as the change of the last stem consonant \textit{n} to \textit{ng} in the noun \textit{cì-shàmbàng-ò} ‘place to play in water’, from the verb \textit{shàmbàn-à} ‘play in water’. In the derived noun \textit{rù-tângò} ‘story, proverb’, the separative transitive suffix \textit{-ur} of the source verb \textit{táng-ùr-à} ‘tell a story’ is lost in the derived noun.

The derivational suffixes \textit{-u}, -\textit{e} and -\textit{a} are unproductive: some of the limited number of attested examples are presented in \tabref{tab:4:14}.

\begin{table}
\label{bkm:Ref75257806}\caption{\label{tab:4:14}Nominal derivation with \textit{-u}, \textit{-e} and \textit{-a}}

\begin{tabular}{llll}
\lsptoprule
Base verb &  & Derived noun & \\
\midrule
\textit{bòmb-à} & ‘become wet’ & \textit{mà-bòmb-à} & ‘blisters’\\
\textit{hùzy-à} & ‘breathe’ & \textit{∅-hûzy-à} (cl 5) & ‘breath’\\
\textit{bòr-à} & ‘rot’ & \textit{bù-bóz-ù} & ‘something rotten’\\
\textit{gòr-à} & ‘be strong’ & \textit{mù-kóz-ù} & ‘strength’\\
\textit{àndà} & ‘freeze’ & \textit{cì-ând-è} & ‘frost’\\
\lspbottomrule
\end{tabular}
\end{table}

For the suffix \textit{-u}, there are two cases where its use involves spirantization of the preceding consonant in a similar way as the agentive suffix \textit{-i} discussed above: \textit{bù-bóz-ù} ‘something rotten’, from \textit{bòrà} ‘rot’, and \textit{mù-kózù} ‘strength’, from \textit{gòrà} ‘be strong’. Aside from spirantization, these examples are also deviant in their tonal pattern and in the realization of the velar stop as voiceless in the noun \textit{mù-kózù} and as voiced in the verb \textit{gòrà}. The irregular spirantization suggests that there may have been two deverbal suffixes in Fwe, a high vowel *-u causing spirantization, and a lowered high vowel *-ʊ not causing spirantization, possibly also with a tonal difference. As *u and *ʊ merged (cf. \citealt{Bostoen2009}), the difference between the two suffixes was lost. {\citet[95]{Meeussen1967}} also reconstructs two different deverbal suffixes, *-ú and *-ʊ, though both with the same tone.

\tabref{tab:4:15} shows that the semantic functions of the suffixes \textit{-u}, -\textit{e} and -\textit{a} are very varied, including instrumental and patient, both also found with the more productive suffix \textit{-o}. The suffixes \textit{-e} and \textit{-a} are also used to indicate a non-human agent, in contrast with the suffix \textit{-i} which is exclusively used to derive human agents. The suffix \textit{-u}, on the other hand, can be used to derive an abstract concept.

\begin{table}
\label{bkm:Ref98855670}\caption{\label{tab:4:15}Semantics of nouns derived with \textit{-u}, \textit{-e}, and \textit{-a}}

\begin{tabular}{llll}
\lsptoprule
\tablevspace
\multicolumn{4}{c}{Instrumental}\\
\midrule
\textit{bùkùtà} & ‘sharpen’ & \textit{mà-bùkùt-à} & ‘skin used for sharpening’\\
\textit{têmà} & ‘chop’ & \textit{kà-têm-ù} & ‘axe’\\
\textit{kékèrà} & ‘plough’ & \textit{cì-kékêr-è} & ‘disc plough’\\
\tablevspace
\multicolumn{4}{c}{Patient}\\
\midrule
\textit{shémp-èk-à} & ‘shoulder a load’ & \textit{mù-shêmp-ù} & ‘load’\\
\textit{nyùngà} & ‘shake’ & \textit{cì-nyùng-èr-à} & ‘food prepared by shaking’\\
\tablevspace
\multicolumn{4}{c}{Non-human agent}\\
\midrule
\textit{tìmbà} & ‘push’ & \textit{n-tìmb-ìr-à} & ‘dung beetle’\\
\textit{rí-zìngà} & ‘twist oneself’ & \textit{mù-rí-zìng-è} & ‘vine’\\
\tablevspace
\multicolumn{4}{c}{Abstract concept}\\
\midrule
\textit{fwà} & ‘die’ & \textit{rú-fù} & ‘death’\\
\textit{rùrà} & ‘be bitter’ & \textit{bù-rùr-ù} & ‘bitterness’\\
\lspbottomrule
\end{tabular}
\end{table}

The suffix \textit{-ntu} is a general nominalizer, that can be added to a verb stem to derive a noun. The lexical tone of the verb stem is maintained, but unlike other derivational suffixes, the suffix \textit{-ntu} also adds its own high tone, which is assigned to the second syllable of the verb it combines with. These tones are subsequently subject to the tone rules that occur in Fwe, namely Meeussen’s Rule in the case of a disyllabic, high-toned verb stem, as shown in (\ref{bkm:Ref492028234}).

\ea
\label{bkm:Ref492028234}
ci-byár-á-ntu > [cìbyáràntù]\\
\textsc{np}\textsubscript{7}-plant-\textsc{fv}-\textsc{nmlz}\\
\glt ‘something that is planted’\\
cf. byârà ‘plant’
\z

When the verb has no lexical high tone, the high tone assigned to the second syllable of the verb usually spreads to the preceding syllable as the result of high tone spread, as in (\ref{bkm:Ref98855718}) (see also \sectref{bkm:Ref430865664} on optional high tone spread).

\ea
\label{bkm:Ref98855718}
ci-rim-á-ntu > [cìrímântù]\\
\textsc{np}\textsubscript{7}-plough-\textsc{fv}-\textsc{nmlz}\\
\glt ‘something that is ploughed’\\
 cf. rìmà ‘plough’
\z

The origin of the high tone that is added in compounds is unclear. There are no other nominalizing suffixes that have their own tonal profile, and melodic tones are otherwise only assigned by inflected verbs (see \sectref{bkm:Ref71539267}).

The use of \textit{-ntu} to derive nouns from verbs is highly productive, and may be interchanged with other strategies for deriving nouns from verbs, such as the nominalizing suffix \textit{-o}, as in (\ref{bkm:Ref99546069}).

\ea
\label{bkm:Ref99546069}
\ea
\glll cìtèndò\\
ci-tend-o\\
\textsc{np}\textsubscript{7}-do-\textsc{nmlz}\\
\glt ‘action’

\ex
\glll cìténdântù\\
ci-tend-á-ntu\\
\textsc{np}\textsubscript{7}-do-\textsc{fv}-thing\\
\glt ‘action’

\ex
 cf. tènd-à ‘do’
\z\z

When used with a transitive verb, the suffix \textit{-ntu} derives a noun that designates its object, as in (\ref{bkm:Ref98856312}). With an intransitive verb, the deverbal noun designates its subject, as in (\ref{bkm:Ref98856313}). In each case, human involvement is key to derivation with \textit{-ntu}; the derived noun \textit{cìbyáràntù} ‘plant’ specifically refers to a plant cultivated by humans, and the derived noun \textit{cìbúmbwàntù} ‘creature’ specifically refers to human beings.

\ea
\label{bkm:Ref98856312}
\glll cìbyáràntù\\
ci-byár-á-ntu\\
\textsc{np}\textsubscript{7}-plant-\textsc{fv}-\textsc{nmlz}\\
\glt ‘(domesticated) plant’
\z

\ea
\label{bkm:Ref98856313}
\glll cìbúmbwàntù\\
ci-búmb-w-á-ntu\\
\textsc{np}\textsubscript{7}-create-\textsc{pass}-\textsc{fv}-\textsc{nmlz}\\
\glt ‘creature’
\z

The derivation of deverbal nouns with \textit{-ntu} differs from other deverbal derivational processes: the suffix consists of an NCV syllable rather than a single vowel; it adds a high tone to the second stem syllable; and as a deverbal derivational strategy, it is neither a common Bantu strategy nor reconstructed for Proto-Bantu. Instead, derivation with \textit{-ntu} in Fwe has grammaticalized from a verb-noun compound with the nominal root \textit{-ntu} as the second element. This root is still used in the nouns \textit{mù-ntù} ‘person’, \textit{cì-ntù} ‘thing’, and \textit{bù-ntù} ‘humanity’. In grammaticalizing into a derivational suffix, the second member of the compound lost its nominal prefix. A similar grammaticalization has led to the creation of the diminutive suffix \textit{-ána} (see \sectref{bkm:Ref450750897}), which also lost its nominal prefix as it developed into a nominal suffix.

Verb-noun compounds are not common in Fwe, and the few compounds that do exist lack the systematic semantic correspondence between the simple verb and the verb-noun compound that is seen in nouns derived with \textit{-ntu}. Instead, the development of compounds with the root \textit{\-ntu} could be the result of contact with the Khoisan language Khwe. Khwe productively uses a suffix -\textit{khòè} deriving nouns from verbs; although it synchronically functions as a suffix, it has its origin in a compound in which the second member is the noun \textit{khòè} ‘person’ \citep[90-91]{Kilian-Hatz2008}. Possibly, the Fwe construction is a calque of this Khwe construction, similar to what has been proposed for the development of diminutive suffixes (see \sectref{bkm:Ref450750897} for discussion).

\subsection{Noun-to-noun derivation}
\label{bkm:Ref450750897}\hypertarget{Toc75352644}{}
Fwe has a number of strategies to create nouns based on existing nominal stems: a diminutive derivation with the suffix \textit{-ána}; two derivational prefixes \textit{shi-/si-} and \textit{na-}, used to derive personal names, association or ownership; nominal compounding; and reduplication. Changes in noun class membership are also used as a derivational mechanism; these are described in \sectref{bkm:Ref450747827}.

\subsubsection{Diminutive}

As discussed in \sectref{bkm:Ref450747845} on the semantics of noun classes, a diminutive can be created by shifting the relevant noun root to class 12/13. Another diminutive marking strategy uses the diminutive suffix \textit{-ána} after the nominal root. A diminutive can be expressed by a shift to class 12/13, as in (\ref{bkm:Ref98856380}), by a diminutive suffix, as in (\ref{bkm:Ref98856398}), or both, as in (\ref{bkm:Ref98856399}); no clear differences in semantics were observed.

\ea
\label{bkm:Ref98856380}
\glll kámbwà\\
ka-mbwá\\
\textsc{np}\textsubscript{12}-dog\\
\glt ‘small dog; puppy’
\z

\ea
\label{bkm:Ref98856398}
\glll mbwáànà\\
∅-mbwá-ana\\
\textsc{np}\textsubscript{1a}-dog-\textsc{dim}\\
\glt ‘small dog; puppy’
\z

\ea
\label{bkm:Ref98856399}
\glll kàmbwáànà\\
ka-mbwá-ana\\
\textsc{np}\textsubscript{12}-dog-\textsc{dim}\\
\glt ‘small dog; puppy’
\z

Vowel juxtaposition takes place when the vowel-initial suffix \textit{-ána} is added to a noun, which invariably ends in a vowel. In most cases, no changes affect either of the vowels, except when the last vowel of the noun is /a/, in which case it may merge with the vowel /a/ of the diminutive suffix, as in (\ref{bkm:Ref98856457}).

\ea
\label{bkm:Ref98856457}
\gll /ka-mbwá-ana/ > [kàmbwáànà] {\textasciitilde} [kàmbwânà]\\
\textsc{np}\textsubscript{12}-dog-\textsc{dim}\\
\glt ‘small dog; puppy’
\z

In certain more petrified forms with a diminutive suffix, however, the last vowel of the nominal stem has elided even though it was not a vowel /a/, but /i/ as in (\ref{bkm:Ref69986265}).

\ea
\label{bkm:Ref69986265}
\glll mùkázànà\\
mu-kázana\\
\textsc{np}\textsubscript{1}-girl\\
\glt ‘girl’\\
 cf. kázì ‘female’ + -ánà diminutive
\z

The suffix \textit{-ána} has a high tone on its first syllable, which may interact with the tone of the last syllable of the root to which it attaches according to the regular tone rules of Fwe. When the diminutive is added to a noun with a final high tone, the high tone of the diminutive suffix is deleted as the result of Meeussen’s Rule, which deletes the second of two adjacent high tones within a single word, as in (\ref{bkm:Ref98856492}--\ref{bkm:Ref98856494}) (see also \sectref{bkm:Ref440987952}).

\ea
\label{bkm:Ref98856492}
/ka-shokó-ána/ > kàshòkóànà\\
\textsc{np}\textsubscript{12}-monkey-\textsc{dim}\\
\glt ‘small monkey’
\z

\ea
\label{bkm:Ref98856494}
/ci-shamú-ána/ > cìshàmúànà\\
\textsc{np}\textsubscript{7}-tree-\textsc{dim}\\
\glt ‘small tree’
\z

A similar diminutive suffix -\textit{ána} (or cognate forms) also occurs in other Bantu languages, mainly of zones R and S \citep{GibsonEtAl2017}, but also in certain languages of the Kikongo Language Cluster (\citealt{GoesBostoen2021}). {\citet{Güldemann1999}} shows that these diminutive forms have grammaticalized from a head-final nominal compound involving reflexes of *jánà ‘child’. Although the grammaticalization of a diminutive from a noun with this meaning is highly common, its function as a suffix is not what would be expected as the result of language-internal grammaticalization, as Bantu languages have a strict head-initial noun phrase structure. Instead, the development of the suffix is the result of contact with Khoisan languages that have a head-final structure. This is also the case for Fwe, and other Bantu languages in the area in which this (and other) nominal suffixes occur. In addition to the use of the diminutive suffix in Fwe and other languages that have a history of contact with Khoisan, there are also a number of head-final compounds referring to plant names in Mbukushu, Manyo and Fwe, providing further evidence that Bantu-Khoisan contact has influenced, to a very limited extent, the nominal structure of the Bantu languages involved \citep{GunninkEtAl2015}. The same is true of the development of the nominalizing suffix \textit{-ntu}; as discussed in \sectref{bkm:Ref444251366}, this suffix goes back to an earlier head-final verb-noun compound, uncommon for Bantu languages but possibly calqued from the Khoe language Khwe.

\newpage
\subsubsection{Associative}

Fwe has two derivational prefixes \textit{shi}- (alternatively realized as \textit{si-}; see \sectref{bkm:Ref70695065} on the interchangability of /s/ and /sh/ in prefixes) and \textit{na-}, which can be prefixed to nouns to derive personal names, animal and plant names, and ownership of, or association with, a concept. The associative meaning appears to be the largest common denominator, and these prefixes are therefore glossed as associative ‘AS’.

The prefixes \textit{shi-/si-} or \textit{na-} occur before the nominal prefix of the underived noun. Nouns derived with \textit{shi-/si-} or \textit{na-} are invariably assigned to noun class 1a/2. The use of the associative prefix \textit{shi-} is illustrated in (\ref{bkm:Ref99546159}).

\ea
\label{bkm:Ref99546159}
\ea
\glll màndwâ\\
ma-ndwá   \\
\textsc{np}\textsubscript{6}-fight\\
\glt ‘fight’\\   
\ex
\glll shímàndwâ\\
∅-shí-ma-ndwá\\
\textsc{np}\textsubscript{1a}-\textsc{as}-\textsc{np}\textsubscript{6}-fight\\
\glt ‘fighter’
\z\z

The prefixes \textit{si-} and \textit{na-} are productively used to derive personal names from nouns. \textit{na-} is used to derive a woman’s name, as in (\ref{bkm:Ref99546177}), and \textit{si-} is used to derive a man’s name, as in (\ref{bkm:Ref99546196}). In this context, \textit{si-} is consistently realized as \textit{si-}, never as \textit{shi}-.

\ea
\label{bkm:Ref99546177}
\glll nàmàsíkù\\
na-ma-sikú\\
\textsc{as}\textsubscript{F}-\textsc{np}\textsubscript{6}-night\\
\glt ‘Namasiku (name given to a girl born at night)’
\z

\ea
\label{bkm:Ref99546196}
\glll sìmàsíkù\\
si-ma-sikú\\
\textsc{as}\textsubscript{M}-\textsc{np}\textsubscript{6}-night\\
\glt ‘Simasiku (name given to a boy born at night)’
\z

The prefix \textit{shi-/si-} can be used to derive association with, or ownership of, a certain concept, as in (\ref{bkm:Ref99546230}--\ref{bkm:Ref99546232}). This function is not available with the prefix \textit{na-}. In all attested cases, the derived noun refers to a human.

\ea
\label{bkm:Ref99546230}
bàshígêmù bàshíꜝzáwà, àbò bábòná èzìpâù\\
\gll ba-shí-gému  ba-shí-zawá    a-bo    bá̲-bo\textsubscript{H}n-á̲    e-zi-páu\\
\textsc{np}\textsubscript{2}-\textsc{as}-game  \textsc{np}\textsubscript{2}-\textsc{as}-Zawa  \textsc{aug}-\textsc{dem}.\textsc{iii}\textsubscript{2}  \textsc{sm}\textsubscript{2}.\textsc{rel}-see-\textsc{fv}  \textsc{aug}-\textsc{np}\-\textsubscript{8}-animal\\
\glt ‘The game people, the ZAWA\footnote{ZAWA refers to the Zambian Wildlife Authority, charged with managing and protecting Zambia’s wildlife.}  people, those who guard the wild animals…’ (ZF\_Narr15)
\z

\ea
nábò bàshíbwâtò ngá nìbàkànànúkà\\
\gll ná=bo    ba-shí-bu-áto  ngá    ni=ba-ka-nanuk-á̲\\
\textsc{com}=\textsc{dem}\textsubscript{2}  \textsc{np}\textsubscript{2}-\textsc{as}-\textsc{np}\textsubscript{14}-canoe  \textsc{cop}.\textsc{dem}.\textsc{i}\textsubscript{16}  \textsc{com}=\textsc{sm}\textsubscript{2}-\textsc{dist}-lift-\textsc{fv}\\
\glt ‘And those with the canoe [those who have the canoe/are sailing in the canoe], that’s when they started coming.’ (NF\_Narr15)
\z

\ea
\label{bkm:Ref99546232}
èswé tùbàsíꜝnkútà, mbòtúmìààtúrè\\
\gll eswé    tu-ba-sí-N-kutá    mbo-tú̲-mi\textsubscript{H}-a\textsubscript{H}atur-é̲\\
\textsc{pers}\textsubscript{1PL}  \textsc{app}\textsubscript{1PL}-\textsc{np}\textsubscript{2}-\textsc{as}-\textsc{np}\textsubscript{9}-court   \textsc{near}.\textsc{fut}-\textsc{sm}\-\textsubscript{1PL}-\textsc{om}\textsubscript{2PL}-judge-\textsc{pfv}.\textsc{sbjv}\\
\glt ‘Us, the people of the court, we will pass judgment on you.’ (NF\_Narr17)
\z

There are a number of lexicalized cases of derivation with \textit{si-/shi-} and \textit{na-}, listed in \tabref{tab:4:16}. In these nouns, the derivational prefix is followed by an apparent nominal prefix, such as \textit{ka-} of class 12, \textit{mu-} of class 1/3, \textit{ru-/rw-} of class 11 or a homorganic nasal of class 9, although in most cases, no underived noun is attested. However, it is likely that the resemblance to nominal prefixes is not accidental, because some of the nominal roots become analyzable once the presumed former nominal prefix is taken into account. \textit{na-ru-nkaramba} ‘praying mantis’ can be analyzed as a root \textit{nkaramba} ‘old person’ and two prefixes, derivational \textit{na-} and a former class 11 prefix \textit{ru-}, and \textit{shi-ru-bumbira} ‘mud wasp’ can be analyzed as a root \textit{bumbira} derived from the verb \textit{bumba}, ‘make pottery, create’.

\begin{table}
\label{bkm:Ref490838504}\caption{\label{tab:4:16}Lexicalized derivational prefix \textit{shi-/si-} and \textit{na-}}
\begin{tabularx}{\textwidth}{lQQ}
\lsptoprule
\multicolumn{1}{c}{Derived noun} & \multicolumn{1}{c}{Translation} & Putative source\\
\midrule
\textit{shíkáꜝnkózè} & ‘falcon’ & \\
\textit{shìkàrìmbírè} & ‘kite’ & \\
\textit{shímúꜝǀópwè} {\textasciitilde} \textit{múꜝǀópwè} & ‘fish sp.’ & cf. \textit{ǀôhà} ‘be tasteless’ (this fish species is considered edible but not tasty)\\
\textit{shínténgwè} {\textasciitilde}

\textit{sínténgwè} & ‘red-winged starling’ & \\
\textit{shírùbùmbìrà} & ‘mud wasp’ & cf \textit{bùmbà} ‘create, make pottery’\\
\textit{síbbwê} & ‘jackal’ & cf. \textit{mbwâwà} ‘jackal’ ?\\
\textit{síkùcèrà} & ‘mole’ & \\
\textit{síyàbàrìrà} & ‘black mamba’ & \\
\textit{nàmúntàbùrà} & ‘flower (\textit{Commelina subulata})’ & \\
\textit{nàmùróbáꜝróbà} & ‘wild hyacinth (\textit{Scilla natalensis})’ & \\
\textit{nákàrà} & ‘acacia’ & \\
\textit{nàrùnkàrámbà} & ‘praying mantis’ & cf. \textit{nkàrâmbà} ‘old person’\\
\textit{nàrwézáꜝézà} & ‘chameleon’ & \\
\lspbottomrule
\end{tabularx}
\end{table}

The prefix \textit{shi-/si-} is etymologically related to the lexical root \textit{sh(o)} ‘father’, as found in constructions such as \textit{bá-shw-}\textit{ꜝ}\textit{ábò} ‘his father’ and \textit{bá-shw-ꜝ}\textit{étù} ‘our father’. The prefix \textit{na-} relates to the lexical root \textit{ny} used in constructions such as \textit{bà-ny-òkò} ‘your mother’ and \textit{bà-ny-ìnà} ‘his mother’. The sex-specific semantics of \textit{shi-/si-} and \textit{na-} are still seen in the use of these prefixes to form personal names, but not in the formation of plant and animal names, nor in the formation of nouns expressing ownership or association.

\subsubsection{Nominal compounds}

Nouns can be created by compounding a noun with a verb stem or with another noun, though neither strategy is productive in Fwe. In compounds consisting of a noun and a verb, the verb is always the first element of the compound. The verb form used in these compounds includes the final vowel suffix \textit{-a}, and the following noun maintains its nominal prefix. Both elements of the compound retain their underlying tonal pattern, with the application of the usual tone rules that function in Fwe. Verb-noun compounds are rare, and the majority of the attested compounds are plant names, as in (\ref{bkm:Ref433186539}--\ref{bkm:Ref444255396}).

\ea
\label{bkm:Ref433186539}
\glll mùtáfùnànjòvù\\
mu-táfunanjovu\\
\textsc{np}\textsubscript{3}-acacia\\
\glt ‘acacia’\\
 cf. táfùnà ‘chew, graze’, njòvù ‘elephant’
\z

\ea
\glll kàryábàcânì\\
ka-ryábacáni\\
\textsc{np}\textsubscript{12}-geranium\\
\glt ‘geranium sp.’ \\
 cf. ryà ‘eat’, bàcânì ‘hunters’
\z

\ea
\label{bkm:Ref444255396}
\glll mùbèzyàmpâmpà\\
mu-bezyampámpa\\
\textsc{np}\textsubscript{3}-tree\\
\glt ‘tree sp.’\\
 cf. bèːzyà ‘carve (wood)’, mpâmpà ‘forked stick’
\z

Compounds consisting of two nouns are often kinship terms, combining existing kinship terms such as \textit{mwâncè} ‘child’ or \textit{máyè} ‘mother’ into new terms, as in (\ref{bkm:Ref99546405}--\ref{bkm:Ref99546406}).

\ea
\label{bkm:Ref99546405}
\glll bàmáyèmwàncè\\
ba-máyemwance\\
\textsc{np}\textsubscript{2}-maternal\_aunt\\
\glt ‘maternal aunt’\\
 cf. maye ‘mother’, mw-áncè ‘child’
\z

\ea
\glll bàtàtánkâzì\\
ba-tatankázi\\
\textsc{np}\textsubscript{2}-paternal\_aunt\\
\glt ‘paternal aunt’\\
 cf. tátà ‘father’, -kâzì ‘female’
\z

\ea
\label{bkm:Ref99546406}
\glll mùkwérùmè\\
mu-kwérume\\
\textsc{np}\textsubscript{1}-father\_in\_law\\
\glt ‘father-in-law’\\
 cf. mú-kwè ‘in-law’, -rùmè ‘male’
\z

Only two noun-noun compounds that are not kinships are found, listed in (\ref{bkm:Ref99546426}--\ref{bkm:Ref99546427}).

\ea
\label{bkm:Ref99546426}
\glll étángányámbè\\
é-tanganyambé\\
\textsc{aug}-\textsc{np}\textsubscript{5}-calabash\\
\glt ‘calabash’\\
 cf. tàngà ‘pumpkin’, nyámbè ‘god’
\z

\ea
\label{bkm:Ref99546427}
\glll òngwébùnà\\
o-ngwébuna\\
\textsc{aug}-\textsc{np}\textsubscript{1a}-plant\\
\glt ‘plant sp.’\\
 cf. ngwè ‘leopard’, bùnà ‘leaf’
\z

\subsubsection{Noun reduplication}
\begin{sloppypar}
Reduplication of nouns is not a productive derivational strategy (unlike verbal reduplication, which is a productive derivational process, see \sectref{bkm:Ref490839448}), but many noun stems exhibiting reduplication are attested; some examples are given in (\ref{bkm:Ref99546467}). An underived, non-reduplicated noun stem is not attested for any of these nouns, but some are apparently derived from or related to verbs, such as \textit{mùrímbùrîmbù} ‘ignorance’, related to \textit{rímbàùzà} ‘not pay attention’, or \textit{cìtùkùtùkù} ‘sweat’, related to \textit{tùkùtà} ‘be warm’. Reduplication targets both segmental and tonal material (as opposed to verbal reduplication, which targets segmental material only).
\end{sloppypar}

\ea
\label{bkm:Ref99546467}
kàcìyóꜝcíyò \tab ‘chick’\\
kàhàráꜝhárà \tab ‘African finger millet’\\
cìsìkíꜝsíkì \tab ‘tree stump’\\
kàríkùrîkù \tab ‘hiccup’\\
mùrímbùrîmbù \tab ‘ignorance’\\
cìgòrògòrò \tab ‘seasonal stream’\\
cìkùrùkùrù \tab ‘lock’\\
kàmbàryàmbàryà \tab ‘lizard’\\
mbìrìmbìrì \tab ‘pepper’\\
mfùrèmfùrè \tab ‘small insect that goes backward’\\
cìpàùpàù \tab ‘basket with lid; purse, briefcase’\\
\z

\section{Nominal modifiers}
\label{bkm:Ref74305906}\hypertarget{Toc75352645}{}
The following sections describe adjectives (\sectref{bkm:Ref491277755}), demonstratives (\sectref{bkm:Ref492026896}), connectives (\sectref{bkm:Ref492133177}), quantifiers (\sectref{bkm:Ref99546522}), and possessives (\sectref{bkm:Ref491333327}), which can all be used as nominal modifiers, or pronominally. Noun class agreement is marked on all modifiers, making use of nominal prefixes, in the case of adjectives, or pronominal prefixes, in all other cases.

\subsection{Adjectives}
\label{bkm:Ref491277755}\hypertarget{Toc75352646}{}
As is typical for Bantu languages \citep[105]{Maho1999}, Fwe has only a small class of adjectives. Adjectives are marked for agreement with the noun they modify through nominal prefixes. The form of nominal prefixes used on adjectives is identical to those used on nouns (see \tabref{tab:4:1}). One exception is class 1a; class 1a nouns follow the agreement pattern of class 1, and this is also the case for adjectives. Adjectives agreeing with a class 1a noun use the class 1 prefix \textit{mu-}, and not the class 1a nominal prefix, which is zero, as in (\ref{bkm:Ref496804533}). The difference in nominal prefix between class 1a nouns and class 1a adjectives is a first indication that adjectives are a category that is distinct from nouns.

\ea
\label{bkm:Ref496804533}
ndàvú mùcècè\\
\gll ∅-ndavú  mu-cece\\
\textsc{np}\textsubscript{1a}-lion  \textsc{np}\textsubscript{1}-small\\
\glt ‘a small lion’ (ZF\_Elic14)
\z

The obligatory nominal prefix on adjectives may be preceded by an optional augment prefix, as in (\ref{bkm:Ref99547023}--\ref{bkm:Ref99547025}). The augment is also found on other words, such as nouns and demonstratives (see \sectref{bkm:Ref444175456} for the form and function of the augment as it appears on nouns). The form of the augment on adjectives is identical to that on nouns though, like the use of the augment on nouns, its use is optional, and its function, if any, is not yet well understood.

\ea
\label{bkm:Ref99547023}
mùndárè mùgênè {\textasciitilde} mùndárè òmùgênè\\
\gll mu-ndaré  (o-)mu-géne\\
\textsc{np}\textsubscript{3}-maize  (\textsc{aug}-)\textsc{np}\textsubscript{3}-thin\\
\glt ‘small maize’
\z

\ea
\label{bkm:Ref99547025}
bàntú bàrôtù {\textasciitilde} bàntú àbàrôtù\\
\gll ba-ntú  (a-)ba-rótu\\
\textsc{np}\textsubscript{2}-person  (\textsc{aug}-)\textsc{np}\textsubscript{2}-beautiful\\
\glt ‘beautiful people’ (ZF\_Elic14)
\z

The vowel of the augment, if it is used on an adjective, is subject to vowel hiatus resolution rules, resulting in vowel coalescence and/or glide formation, as in (\ref{bkm:Ref99547062}--\ref{bkm:Ref99547063}) (see also \sectref{bkm:Ref491962181}).

\ea
\label{bkm:Ref99547062}
òmbw’ óꜝmúbbì\\
\gll o-∅-mbwá    o-mu-bbí\\
\textsc{aug}-\textsc{np}\textsubscript{1a}-dog  \textsc{aug}-\textsc{np}\textsubscript{1}-ugly\\
\glt ‘an ugly dog’ (NF\_Elic15)
\z

\ea
\label{bkm:Ref99547063}
vùmw’ énênè\\
\gll ∅-vumó  e-∅-néne\\
\textsc{np}\textsubscript{5}-stomach  \textsc{aug}-\textsc{np}\textsubscript{5}-big\\
\glt ‘a big stomach’ (ZF\_Elic14)
\z

Like the augment used with nouns, the adjectival augment may also take a floating high tone. This high tone is realized on the syllable preceding the vowel of the augment, though when the vowel of the augment merges with the preceding syllable, the high tone comes to be realized on the vowel of the augment itself, as in (\ref{bkm:Ref99547270}).

\ea
\label{bkm:Ref99547270}
rùtàká òrùrêː {\textasciitilde} rùtàk’ órùrêː (cf. rùtàkà ‘reed’)\\
\gll ru-taká  o-ru-réː\\
\textsc{np}\textsubscript{11}-reed  \textsc{aug}-\textsc{np}\textsubscript{11}-long\\
\glt ‘a long reed’ (ZF\_Elic14)
\z

The floating high tone of the adjectival augment can also be used when the vocalic augment is absent, as in (\ref{bkm:Ref98835500}). This, too, is a property the adjectival augment shares with the nominal augment (see \sectref{bkm:Ref444175456}).

\ea
\label{bkm:Ref98835500}
mùntú mùrêː (cf. mùntù ‘person’)\\
\gll mu-ntú  mu-réː\\
\textsc{np}\textsubscript{1}-person  \textsc{np}\textsubscript{1}-tall\\
\glt ‘a tall person’ (ZF\_Elic14)
\z

The adjective always follows the noun it modifies when used adnominally, as in (\ref{bkm:Ref496804533}--\ref{bkm:Ref98835500}). Adjectives can also be used predicatively, in which case the adjective is marked with a copulative prefix that agrees in noun class with the noun it describes, as in (\ref{bkm:Ref99547339}--\ref{bkm:Ref99547340}) (for more on the copula, see \sectref{bkm:Ref450747606}).

\ea
\label{bkm:Ref99547339}
èhámbà ndìnênè\\
\gll e-∅-ámba    N-ri-néne\\
\textsc{aug}-\textsc{np}\textsubscript{5}-hoe    \textsc{cop}-\textsc{np}\textsubscript{5}-big\\
\glt ‘The hoe is big.’ (NF\_Elic15)
\z

\ea
\label{bkm:Ref99547340}
yìn’ énjúò njìrôtù\\
\gll iná    e-N-júo    nji-rótu\\
\textsc{dem}.\textsc{iv}\textsubscript{9}  \textsc{aug}-\textsc{np}\textsubscript{9}-house  \textsc{cop}\textsubscript{9}-beautiful\\
\glt ‘That house is beautiful.’ (ZF\_Elic14)
\z

Adjectives can also be used nominally, in which case the adjective takes a prefix that agrees in noun class with the noun it describes. This is illustrated in (\ref{bkm:Ref450563669}), where the nominally used adjective \textit{ómùrê} ‘long’ takes the prefix of class 3, as it refers to a class 3 noun \textit{mù-hàrà} ‘rope’.

\ea
\label{bkm:Ref450563669}
ndìshàk’ ómùrêː\\
\gll ndi-shak-á̲    o-mu-réː\\
\textsc{sm}\textsubscript{1SG}-want-\textsc{fv}  \textsc{aug}-\textsc{np}\textsubscript{3}-long\\
\glt ‘I want the long one.’ (Answer to: ‘Which rope do you want?’) (ZF\_Elic14)
\z

The number of adjectival stems in Fwe is limited: an exhaustive list is given in (\ref{bkm:Ref440614059}).

\ea
\label{bkm:Ref440614059}
  Adjective stems in Fwe

bbí \tab ‘bad’ (Namibian Fwe only)\\
céː \tab ‘few’\\
cékù \tab ‘sharp’\\
cényà \tab ‘small’\\
dânà \tab ‘small’\\
fwîyì \tab ‘short’\\
ᵍǀênè \tab ‘thin’\\
hùbà \tab ‘light’\\
káꜝbábù \tab ‘difficult’\footnotemark{}\\
kâtà \tab ‘weak’\footnotemark{}\\
kûrù \tab ‘old’\\
mângò \tab ‘bad’ (Zambian Fwe only)\\
nênè \tab ‘big’\\
nînì \tab ‘small’\\
rêː \tab ‘tall, long, far’\\
rémù \tab ‘heavy’\footnotemark{}\\
rôtù \tab ‘good, beautiful’\\
tékè \tab ‘fresh’\\
tòrè \tab ‘soft, easy’\\
yá \tab ‘new’\\
{}ǀòː \tab ‘tasteless’\\
\z

\addtocounter{footnote}{-3}
\stepcounter{footnote}\footnotetext{Most speakers prefer to use the noun \textit{bù-káꜝ}\textit{bábù} ‘problem, something difficult’.}
\stepcounter{footnote}\footnotetext{Most speakers prefer the use of the verb \textit{kàtà} ‘become weak’.}
\stepcounter{footnote}\footnotetext{Most speakers prefer the use of the verb \textit{rèmà} ‘be heavy’.}

Three adjective stems appear to be derived from verbs by means of the suffix \textit{-u}, also used to derive nouns from verbs (see \sectref{bkm:Ref444251366}): \textit{kûrù} ‘old’, from \textit{kûrà} ‘grow’, \textit{rémù} ‘heavy’, from \textit{rèmà} ‘become heavy’, \textit{cékù} ‘sharp’, from \textit{cékùrà} ‘cut oneself’.

Adjectives may be reduplicated to give an intensifying or emphatic meaning, as in (\ref{bkm:Ref98512420}--\ref{bkm:Ref98512421}).

\ea
\label{bkm:Ref98512420}
ndákàkùrímìnà éwà ènénènênè\\
\gll ndi-á-ka-ku-rím-in-a e-∅-wá    e-∅-néne-néne \\
\textsc{sm}\textsubscript{1SG}-\textsc{sbjv}.\textsc{ipfv}-\textsc{dist}-\textsc{om}\textsubscript{2SG}-farm-\textsc{appl}-\textsc{fv} \textsc{aug}-\textsc{np}\textsubscript{5}-field  \textsc{aug}-\textsc{np}\textsubscript{5}-big-big\\
\glt ‘I could cultivate a very big farm for you.’ (NF\_Narr15)
\z

\ea
\label{bkm:Ref98512421}
kàcírì cìrótùrôtù ècí cìbàkà\\
\gll ka-cí̲-ri    ci-rótu-rótu    e-cí    ci-baka\\
\textsc{pst}.\textsc{ipfv}-\textsc{sm}\textsubscript{7}-be  \textsc{np}\textsubscript{7}-nice-nice  \textsc{aug}-\textsc{dem}.\textsc{i}\textsubscript{7}  \textsc{np}\textsubscript{7}-place\\
\glt ‘It was very nice, this place.’ (NF\_Narr17)
\z

In one case, the reduplicated meaning differs in an unexpected manner from the unreduplicated meaning: the adjective \textit{kûrù} ‘old’ is used to describe animates, and its reduplicated form \textit{kúrùkûrù} describes inanimates.

That adjectives are marked with almost the same prefixes as nouns (with the exception of class 1a) may suggest that adjectives should be treated as nouns as well. Furthermore, many adjective stems also occur as nouns, although many of these are abstract nouns, which are likely to be derived from adjectives, rather than vice versa. A non-exhaustive list is given in \tabref{tab:4:17}.

\begin{table}
\label{bkm:Ref491078565}\caption{\label{tab:4:17}Adjective stems that also occur as nouns}
\begin{tabular}{llll}
\lsptoprule
Adjective &  & Noun & \\
\midrule
{\itshape bbì} & ‘bad’ & {\itshape bú-bbì} & ‘evil’\\
{\itshape dânà} & ‘small’ & {\itshape mù-dânà} & ‘child’\\
{\itshape kûrù} & ‘old’ (of animates) & {\itshape mù-kûrù} & ‘elder, elder sibling/ cousin’\\
{\itshape rêː} & ‘tall, long, far’ & {\itshape bù-rêː} & ‘length’\\
{\itshape rémù} & ‘heavy’ & {\itshape bù-rémù} & ‘weight’\\
{\itshape rôtù} & ‘good, beautiful’ & {\itshape bù-rôtù} & ‘goodness’\\
\lspbottomrule
\end{tabular}
\end{table}

Despite these similarities, adjectives display syntactic behavior that is distinct from that of nouns, because they can modify nouns without the use of additional morphological material. Al\-though nouns can also modify other nouns, through the use of a connective clitic, for instance, nouns cannot modify other nouns through mere juxtaposition. Adjectives, however, do modify nouns through juxtaposition, as long as a nominal prefix is used that agrees in noun class with the modified noun. This syntactic construction is limited to the adjectival stems listed in (\ref{bkm:Ref440614059}), which shows that the category of adjective is a distinct part of speech in Fwe.

Another characteristic that sets adjectives apart from nouns is that adjective stems may occur in any noun class, as long as agreement with the head noun is maintained. Noun stems, however, belong to a single set of noun classes only. Although nominal stems may be shifted to a different noun class as a result of derivation (see \sectref{bkm:Ref450750293}), this influences the meaning of the noun, and the number of noun classes in which a single nominal stem can be used is limited; it is not possible to use a single nominal stem in any noun class.

Finally, adjectives differ from nouns because only adjectives can be combined with the suffix \textit{-h} to derive a verb. Some verbs are attested where the deadjectival suffix is realized as \textit{-mp} instead of \textit{-h}. The derivation of verbs from adjectives is illustrated in \tabref{tab:4:18}.

\begin{table}
\label{bkm:Ref491097959}\caption{\label{tab:4:18}Deadjectival verbs}
\begin{tabular}{llll}
\lsptoprule
\multicolumn{2}{l}{Derived verb} & \multicolumn{2}{l}{Underived adjective}\\
\midrule
{\itshape rêː-h-à} & ‘become tall’ & {\itshape rêː} & ‘tall, long’\\
{\itshape nénè-h-à} & ‘become big’ & {\itshape nênè} & ‘big’\\
{\itshape tòrè-h-à} & ‘become soft’ & {\itshape tòrè} & ‘soft’\\
{\itshape ǀôː-h-à} & ‘become tasteless’ & {\itshape ǀôː} & ‘tasteless’\\
{\itshape bbî-h-à} & ‘become bad’ & {\itshape bbí} & ‘bad’\\
{\itshape fwîì-mp-à} & ‘become short’ & {\itshape fwîì} & ‘short’\\
{\itshape kúrù-mp-à} & ‘become old’ & {\itshape kûrù} & ‘old’\\
\lspbottomrule
\end{tabular}
\end{table}

The deadjectival suffix \textit{-h} derives an intransitive change-of-state verb, where the state that is entered into is the state described by the underived adjective.

\subsection{Demonstratives}
\label{bkm:Ref492026896}\hypertarget{Toc75352647}{}\label{bkm:Ref450747952}
Fwe has four paradigms of demonstratives (also called “series” in Bantu linguistics, see e.g. \citet{Nicolle2012}; \citet{Wal2010}), which are distinguished by the relative distance between the referent and the speaker and hearer: general proximity (series I), proximity to the speaker (series II), proximity to the hearer (series III) and distance (series IV). \tabref{tab:4:19} gives the form for each noun class for each demonstrative series. Each demonstrative form has an optional augment prefix, formally identical to the augment used on nouns.

\begin{table}
\label{bkm:Ref491095359}\caption{\label{tab:4:19}Demonstratives}
\begin{tabular}{lllll}
\lsptoprule
& series I & series II & series III & series IV\\
\midrule
1 & o-zyu & o-zyuno / o-zyunu & o-zyo & o-zywina\\
2 & a-ba & a-bano / a-banu & a-bo & a-bena\\
1a & o-zyu & o-zyuno / o-zyunu & o-zyo & o-zywina\\
3 & o-u & o-uno / o-unu & o-o & o-wina\\
4 & e-i & e-ino / e-inu & e-yo & e-ina\\
5 & e-ri & e-rino / e-rinu & e-ryo & e-rina\\
6 & a-a & a-ano / a-anu & a-o & a-ena\\
7 & e-ci & e-cino / e-cinu & e-co & e-cina\\
8 & e-zi & e-zino / e-zinu & e-zo & e-zina\\
9 & e-i & e-ino / e-inu & e-yo & e-ina\\
10 & e-zi & e-zino / e-zinu & e-zo & e-zina\\
11 & o-ru & o-runo / o-runu & o-o & o-rwina\\
12 & a-ka & a-kano / a-kanu & a-ko & a-kena\\
13 & o-tu & o-tuno / o-tunu & o-to & o-twina\\
14 & o-bu & o-buno / o-bunu & o-bo & o-bwina\\
15 & o-ku & o-kuno / o-kunu & o-ko & o-kwina\\
16 & a-ha & a-hano / a-hanu & a-ho & a-hena\\
17 & o-ku & o-kuno / o-kunu & o-ko & o-kwina\\
18 & o-mu & o-muno / o-munu & o-mo & o-mwina {\textasciitilde} o-muna\\
\lspbottomrule
\end{tabular}
\end{table}

The series I demonstratives are formally identical to the paradigm of pronominal prefixes (see \tabref{tab:4:1}). For class 1 and 1a, which have two different pronominal prefixes, demonstratives are based on the form \textit{zyu} rather than the form \textit{u}. The other three demonstrative series are derived from series I by the addition of a suffix: \textit{-no} (Zambian Fwe) or \nobreakdash-\textit{nu} (Namibian Fwe) for series II,\footnote{This is in contrast with {\citet{Nicolle2012}}, who lists Fwe as a language that lacks a reflex of *\nobreakdash-no, based on \citet{Baumbach1997}. As noted in \sectref{bkm:Ref492223019}, Baumbach’s grammar sketch of Fwe is very limited and numerous differences between it and my data exist.} \textit{-o} for series III, and \textit{-ina} for series IV, the latter resulting in vowel hiatus resolution through vowel coalescence, vowel deletion, and glide formation (see \sectref{bkm:Ref491962181}).

The tonal realization of demonstratives depends on their syntactic position. Adnominal demonstratives have a high tone on the last mora of the stem, as in (\ref{bkm:Ref494890604}). Adverbial demonstratives have a high tone on the first stem mora, as in (\ref{bkm:Ref494890605}). Demonstratives used as relativizers are realized without any high tones, as in (\ref{bkm:Ref494890607}) (see also \sectref{bkm:Ref491095705} on relative clauses). The tonal behavior of pronominal demonstratives requires further study: various patterns are attested, as in (\ref{bkm:Ref75346009}--\ref{bkm:Ref75250854}), and it is unclear what, if anything, conditions their use.

\ea
\label{bkm:Ref494890604}
\textbf{èrí} hànjà\\
\gll e-rí    hanja\\
\textsc{aug}-\textsc{dem}.\textsc{i}\textsubscript{5}  hand\\
\glt ‘this hand’ (ZF\_Elic14)
\z

\newpage
\ea
\label{bkm:Ref494890605}
ndìkárángà \textbf{kûnù}\\
\gll ndi-kar-á̲ng-a  kúnu\\
\textsc{sm}\textsubscript{1SG}-sit-\textsc{hab}-\textsc{fv}  \textsc{dem}.\textsc{ii}\textsubscript{17}\\
\glt ‘I normally stay here.’ (NF\_Elic17)
\z

\ea
\label{bkm:Ref494890607}
àmàshéréŋì \textbf{àò} nìtwáshàngàúrà\\
\gll a-ma-sheréŋi    a-o    ni-tú̲-a-sha\textsubscript{H}ngaur-á̲\\
\textsc{aug}-\textsc{np}\textsubscript{6}-money  \textsc{aug}-\textsc{dem}.\textsc{iii}\textsubscript{6}  \textsc{rem}-\textsc{sm}\textsubscript{1PL}-\textsc{pst}-contribute-\textsc{fv}<\textsc{rel}>\\
\glt ‘the money that we contributed’ (NF\_Elic17)
\z

\ea
\label{bkm:Ref75346009}
ndìsháká kùùrà \textbf{cînà} {\textasciitilde} \textbf{cìnà}\\
\gll ndi-shak-á̲    ku-ur-a  cína {\textasciitilde} cina\\
\textsc{sm}\textsubscript{1SG}-want-\textsc{fv}  \textsc{inf}-buy-\textsc{fv}  \textsc{dem}.\textsc{iv}\textsubscript{7}\\
\glt ‘I want to buy that one.’
\z

\ea
\label{bkm:Ref75250854}
àkéːzyà \textbf{zywînà} {\textasciitilde} \textbf{zywínà}\\
\gll a-ké̲ːzy-a    zywína {\textasciitilde} zywiná\\
\textsc{sm}\textsubscript{1}-come-\textsc{fv}  \textsc{dem}.\textsc{iv}\textsubscript{1}\\
\glt ‘S/he is coming, that one.’ (NF\_Elic17)
\z

As seen in \tabref{tab:4:19}, demonstratives can take an augment prefix. Similar to the augment on nouns and adjectives,\footnote{Augments used on nouns also have a floating high tone, which surfaces on the syllable immediately preceding the vocalic augment. It is not clear if the augment on demonstratives has this same tonal realization, as the number of contexts in which it could be realized is very limited. This matter requires further investigation.} it consists of a single vowel that displays vowel harmony with the demonstrative stem: \textit{e-} is used with demonstrative stems with a front vowel \textit{i}, \textit{o-} is used with demonstrative stems with a back vowel \textit{u}, and \textit{a-} is used with demonstrative stems with the vowel \textit{a}. Demonstratives may be used with an augment, as in (\ref{bkm:Ref450566532}), or without an augment, as in (\ref{bkm:Ref450566525}).

\ea
\label{bkm:Ref450566532}
àbèná bàkéntù bàámbà wàwà\\
\gll a-bená  ba-kéntu  ba-á̲mb-a  wawa\\
\textsc{aug}-\textsc{dem}.\textsc{iv}\textsubscript{2}  \textsc{np}\textsubscript{2}-woman  \textsc{sm}\textsubscript{2}-talk-\textsc{fv}  very\\
\glt ‘Those women talk a lot.’ (ZF\_Elic14)
\z

\ea
\label{bkm:Ref450566525}
bèná bàntù\\
\gll bená    ba-ntu\\
\textsc{dem}.\textsc{iv}\textsubscript{2}  \textsc{np}\textsubscript{2}-person\\
\glt ‘those people’
\z

The use of the augment on demonstratives is influenced by a number of factors. Firstly, the augment is more commonly used with the monosyllabic series I and III demonstratives, and is more commonly dropped with the disyllabic series II and IV demonstratives. Secondly, demonstratives used to introduce a relative clause often occur without an augment vowel, even if they are monosyllabic (see \sectref{bkm:Ref491095705} on relative clauses).

In addition to the demonstrative forms listed in \tabref{tab:4:19}, an emphatic demonstrative can be created by prefixing the basic demonstrative stem of series I to the demonstrative, e.g. \textit{zyo} ‘that one’, \textit{zyu-zyo} ‘that very one’. This can be applied to demonstratives of all four series, as illustrated for series III in (\ref{bkm:Ref99547938}) and series I in (\ref{bkm:Ref99547939}); in each case, it is the basic demonstrative stem of series I that is prefixed to the demonstrative stem. The derived demonstrative indicates extra emphasis, translated as ‘this/that very (same)’.

\ea
\label{bkm:Ref99547938}
\textbf{ríryò} shènè óbwènè ndíwè\\
\gll rí-ryo      ∅-shene  ó̲-bwe\textsubscript{H}ne    ndi-wé\\
\textsc{emph}-\textsc{dem}.\textsc{iii}\textsubscript{5}  \textsc{np}\textsubscript{5}-worm  \textsc{sm}\textsubscript{2SG}.\textsc{rel}-see.\textsc{stat}  \textsc{cop}-\textsc{pers}\textsubscript{2SG}\\
\glt ‘\textbf{This} \textbf{very} \textbf{worm} that you see, it’s you.’ (NF\_Song17)
\z

\ea
\label{bkm:Ref99547939}
ákùbáꜝtéyè shárì \textbf{zyùzyú} mwâncè nìndáꜝyéndà néyè nìnìndámàn’ óꜝkáfwà\\
\gll á-ku-bá-téye      shári  zyu-zyú  mu-ánce ni-ndí̲-a-é̲nd-a      ne=ye
ni-ni-ndí̲-a-man-á̲      o-ka-fw-á\\
\textsc{con}\textsubscript{1}-\textsc{inf}-\textsc{om}\textsubscript{2}-say\_that  if  \textsc{emph}-\textsc{dem}.\textsc{i}\textsubscript{1}  \textsc{np}\textsubscript{1}-child
\textsc{rem}-\textsc{sm}\textsubscript{1SG}-\textsc{pst}-go-\textsc{fv}<\textsc{rel}>  \textsc{com}=\textsc{pers}\textsubscript{3SG}
\textsc{rem}-\textsc{rem}-\textsc{sm}\textsubscript{1SG}-\textsc{pst}-finish-\textsc{fv}  \textsc{aug}-\textsc{inf}.\textsc{dist}-die-\textsc{fv}\\
\glt ‘She told them: if not for \textbf{this} \textbf{very} \textbf{child}, that I went with, I would have died there.’ (NF\_Narr15)
\z

Demonstratives always show noun class agreement. Adnominal demonstratives agree with the noun they modify, as in (\ref{bkm:Ref437526099}--\ref{bkm:Ref496806945}). Pronominal demonstratives agree with the noun they replace or refer to, as in (\ref{bkm:Ref496806997}), taken from a narrative, where the class 1a demonstrative \textit{òzwyínà} ‘that one’ refers back to an earlier mentioned elephant, \textit{njòvù}, which is a class 1a noun.

\ea
\label{bkm:Ref437526099}
òzyú mùntù\\
\gll o-zyú    mu-ntu\\
\textsc{aug}-\textsc{dem}.\textsc{i}\textsubscript{1}  \textsc{np}\textsubscript{1}-person\\
\glt ‘this person’
\z

\ea
\label{bkm:Ref437526104}
\label{bkm:Ref496806945}òkú ꜝkútwì\\
\gll o-kú    ku-twí\\
\textsc{aug}-\textsc{dem}.\textsc{i}\textsubscript{15}  \textsc{np}\textsubscript{15}-ear\\
\glt ‘this ear’ (ZF\_Elic14)
\z

\ea
\label{bkm:Ref496806997}
bókùndìsùndà òzwyínà\\
\gll bá-o-ku-ndi-sund-a  o-zywiná\\
\textsc{con}\textsubscript{2}-\textsc{inf}-\textsc{om}\textsubscript{1SG}-show-\textsc{fv}  \textsc{aug}-\textsc{dem}.\textsc{iv}\textsubscript{1}\\
\glt ‘They showed him to me.’ (ZF\_Narr13)
\z

The unmarked position of adnominally used demonstratives is before the noun they modify, as in (\ref{bkm:Ref437526099}--\ref{bkm:Ref437526104}) above. Demonstratives do occur post-nominally when the noun is marked by a copulative prefix, as in (\ref{bkm:Ref452385401}--\ref{bkm:Ref452385402}). This is due to right dislocation: constituents can move to the right edge of the clause when they function as definite (see \sectref{bkm:Ref452042958} for discussion and examples). As demonstratives are frequently used anaphorically, referring to a referent that is identifiable to both speaker and hearer, they are frequently subject to right-dislocation.

\ea
\label{bkm:Ref452385401}
mùndár’ ôwù\\
\gll N-mu-ndaré    o-ú\\
\textsc{cop}-\textsc{np}\textsubscript{3}-maize  \textsc{aug}-\textsc{dem}.\textsc{i}\textsubscript{3}\\
\glt ‘It’s maize, this.’
\z

\ea
\label{bkm:Ref452385402}
ndùngúy’ òzyù\\
\gll ndu-∅-nguyá  o-zyú\\
\textsc{cop}\textsubscript{1a}-\textsc{np}\textsubscript{1a}-baboon  \textsc{aug}-\textsc{dem}.\textsc{i}\textsubscript{1}\\
\glt ‘It’s a baboon, this one.’ (ZF\_Elic14)
\z

Demonstratives can also be used postnominally when the noun phrase is the object of an imperative or subjunctive verb, as in (\ref{bkm:Ref452385509}) and (\ref{helpthatperson}), although prenominal demonstratives are also allowed, as in (\ref{bkm:Ref494891071}). Postnominal demonstratives are only possible with subjunctive or imperative verbs expressing an order, not with other functions of the subjunctive.

\ea
\label{bkm:Ref452385509}
òzìmìsé mùrìrò ówù\\
\gll o-zi\textsubscript{H}m-is-é̲        mu-riro  o-ú\\
\textsc{sm}\textsubscript{2SG}-extinguish-\textsc{caus}-\textsc{pfv}.\textsc{sbjv}  \textsc{np}\textsubscript{3}-fire  \textsc{aug}-\textsc{dem}.\textsc{i}\textsubscript{3}\\
\glt ‘Extinguish this fire.’ (NF\_Elic15)
\z

\ea
\label{helpthatperson}
òtùsé òmùntú zyò\\
\gll o-tus-é̲      o-mu-ntú    zyo\\
\textsc{sm}\textsubscript{2SG}-help-\textsc{pfv}.\textsc{sbjv}  \textsc{aug}-\textsc{np}\textsubscript{1}-person  \textsc{dem}.\textsc{iii}\textsubscript{1}\\
\glt ‘Help that person.’
\z

\ea
\label{bkm:Ref494891071}
òtùsé òzyó mùntù\\
\gll o-tus-é̲    o-zyó    mu-ntu\\
\textsc{sm}\textsubscript{2SG}-help-\textsc{pfv}.\textsc{sbjv}  \textsc{aug}-\textsc{dem}.\textsc{i}\textsubscript{1}  \textsc{np}\textsubscript{1}-person\\
\glt ‘Help that person.’ (NF\_Elic17)
\z

\hspace*{-1.73pt}All other adnominal demonstratives appear before the noun they modify. Other nominal modifiers in Fwe, however, canonically appear after the noun they modify. The preferred pre-nominal position of the demonstrative in Fwe is also uncommon for Bantu languages in general, which, like Fwe, have a strict head - dependent order which also determines the placement of the demonstrative. In a sample of 138 Bantu languages, {\citet{Velde2005}} found only five languages in which the demonstrative always precedes the noun it modifies. Languages in which the demonstrative may either follow or precede the noun are more common, including some of Fwe’s closest linguistic relatives such as the western Bantu Botatwe language Subiya \citep[33]{Jacottet1896}, and the eastern Bantu Botatwe languages Tonga (\citealt{Carter2002}: 40; \citealt{Collins1962}: 83) and Ila \citep[105]{Smith1964}. Even among Bantu Botatwe languages, however, Fwe appears to be the only language in which the pre-nominal demonstrative is much more common than the post-nominal demonstrative. More thorough documentation of Western Bantu Botatwe languages such as Shanjo and Totela is needed to understand the position of the demonstrative in these languages.

Demonstratives have a situational use, with which the demonstrative singles out a referent in the physical surroundings of the speaker, and a non-situational use, with which the demonstrative singles out a referent that is known through general knowledge or the earlier discourse. This distinction is known under different labels in the literature, such as exophoric/endophoric \citep{Diessel1999}; following {\citet{Himmelmann1996}}, I will use the terms situational/non-situational.

The situational use of the series 1 demonstratives is to indicate that a referent is generally close to both the hearer and the speaker, as illustrated in (\ref{bkm:Ref450568975}), referring to shoes that are in the immediate vicinity of both the speaker and the hearer.

\ea
\label{bkm:Ref450568975}
\textbf{èzí} nshângù zìcénà\\
\gll e-zí    N-shángu  zi-cen-á̲\\
\textsc{aug}-\textsc{dem}.\textsc{i}\textsubscript{10}  \textsc{np}\textsubscript{10}-shoe  \textsc{sm}\textsubscript{10}-be\_clean-\textsc{fv}\\
\glt ‘These shoes are clean.’ (ZF\_Elic14)
\z

Series II demonstratives are used to indicate that a referent is close to the speaker, but not to the hearer, as illustrated in (\ref{bkm:Ref450569148}), from an elicitation context in which a bag of beans was lying on the table next to the speaker.

\ea
\label{bkm:Ref450569148}
èzìnó nyàngù\\
\gll e-zinó    N-nyangu\\
\textsc{aug}-\textsc{dem}.\textsc{ii}\textsubscript{10}  \textsc{np}\textsubscript{10}-bean\\
\glt ‘these beans’ (ZF\_Elic13)
\z

Series III demonstratives are used to indicate a referent close the hearer, but not close to the speaker. In (\ref{bkm:Ref494891240}), the speaker warns the hearer of an approaching elephant, using a series III demonstrative as an indication of the elephant’s location close to the hearer.

\ea
\label{bkm:Ref494891240}
bbónàdì bbónàdì shá ònjòvú zyw’ ákèːzy’ \textbf{ókò}\\
\gll bbónadi   bbónadi  shá o-∅-njovú    zyú  á̲-keːzy-á̲  o-kó \\
Bonard  Bonard  sir \textsc{aug}-\textsc{np}\textsubscript{1a}-elephant  \textsc{dem}.\textsc{i}\textsubscript{1}  \textsc{sm}\textsubscript{1}-come-\textsc{fv} \textsc{aug}-\textsc{dem}.\textsc{iii}\textsubscript{17}\\
\glt ‘Mr Bonard, Mr Bonard! There is an elephant coming to you!’ (ZF\_Narr13)
\z

Series IV demonstratives are used to indicate a referent far from both the speaker and the hearer. In (\ref{bkm:Ref450569612}), taken from a narrative, the speaker uses a series IV demontrative \textit{énà} to refer to teeth that are hidden at a place far away from the speaker and the hearer.

\ea
\label{bkm:Ref450569612}
èmé ndìhíndè \textbf{énà} ménò\\
\gll emé    ndi-hí̲nd-e    ená    ma-inó\\
\textsc{pers}\textsubscript{1SG}  \textsc{sm}\textsubscript{1SG}-take-\textsc{pfv}.\textsc{sbjv}  \textsc{dem}.\textsc{iv}\textsubscript{6}  \textsc{np}\textsubscript{6}-tooth\\
\glt ‘I will take those teeth.’ (NF\_Narr15)
\z

Demonstratives also have various non-situational uses. One of these is the use of a demonstrative for discourse deixis, i.e. to refer to the general information referent of a larger, broader chunk of discourse. In (\ref{bkm:Ref437947332}), the series III demonstrative \textit{èryó} ‘that’ refers back to the topic of the preceding discourse in its entirety, which has described the attack on an old lady by elephants.

\ea
\label{bkm:Ref437947332}
kónàkùrì \textbf{èryó} kàndè ryábànjòvù\\
\gll kónakuri  e-ryó    ∅-kande  ri-á=ba-njovu\\
because  \textsc{aug}-\textsc{dem}.\textsc{iii}\textsubscript{5}  \textsc{np}\textsubscript{5}-story  \textsc{pp}\textsubscript{5\-}-\textsc{con}=\textsc{np}\textsubscript{2}-elephant\\
\glt ‘Because of this story of the elephants…’ (ZF\_Narr15)
\z

Within discourse, demonstratives can be used anaphorically, to refer back to earlier mentioned entities and participants. In the anaphoric use of demonstratives, Fwe uses different demonstrative series in a different way, depending on the salience of the referent in the discourse. A series III demonstrative is used to refer back to a referent that is still highly salient. In (\ref{bkm:Ref437948356}), a new referent, a village, is introduced by means of the noun \textit{mùnzì}, and when the aforementioned village is mentioned again, it is marked by the series III demonstrative \textit{òwó} ‘this’.


\newpage
\ea
\label{bkm:Ref437948356}
\ea
kàkwín’ ꜝómùnzì òmù kàmwíꜝná bàntù\\
\gll ka-kú̲-iná      o-mu-nzi o-mu    ka-mú̲-iná    ba-ntu\\
\textsc{pst}.\textsc{ipfv}-\textsc{sm}\textsubscript{17}-be\_at    \textsc{aug}-\textsc{np}\textsubscript{3}-village
\textsc{aug}-\textsc{dem}.\textsc{i}\textsubscript{18}  \textsc{pst}.\textsc{ipfv}-\textsc{sm}\textsubscript{18}-be\_at  \textsc{np}\textsubscript{2}-person\\
\glt ‘There was a village, where people were living.’

\ex
\textbf{òwó} mùnzì kàwínà shíryà yórwîzyì\\
o-ó    mu-nzi  ka-ú̲-ina ∅-shírya    i-ó=ru-ízyi\\
\textsc{aug}-\textsc{dem}.\textsc{iii}\textsubscript{3}  \textsc{np}\textsubscript{3}-village  \textsc{pst}.\textsc{ipfv}-\textsc{sm}\textsubscript{3}-be\_at \textsc{np}\textsubscript{9}-other\_side  \textsc{pp}\textsubscript{9}-\textsc{con}=\textsc{np}\textsubscript{11}-river\\
\glt ‘This village was at the other side of the river.’ (NF\_Narr15)
\z\z

Salience, or accessibility \citep{Ariel2001}, describes how easy it is for the listener to retrieve the intended referent from the discourse. Accessibility is influenced by various factors, such as the number of times the referent was mentioned, the time elapsed since the last mention and the number of potentially competing referents that were introduced since then. In (\ref{bkm:Ref437948356}), the recent use of the word \textit{mùnzì} ‘village’ has caused its referent to be highly salient, and therefore referred to with the series III demonstrative. An example where the frequent earlier mention of the referent has contributed to its salience is given in (\ref{bkm:Ref437948827}), taken from the middle section of a longer narrative in which a man, his wife and the wife’s younger sister are the main participants. All three main characters have been mentioned frequently in the previous discourse, hence allowing one of them, the man, to be referred to with the series III demonstrative.

\ea
\label{bkm:Ref437948827}
\textbf{òzyó} múꜝkwámè ákùhìndá kàtêmù\\
\gll o-zyó    mú-kwamé  á-ku-hind-á    ka-tému\\
\textsc{aug}-\textsc{dem}.\textsc{iii}\textsubscript{1}  \textsc{np}\textsubscript{1}-man  \textsc{pp}\textsubscript{1}-\textsc{inf}-take-\textsc{fv}  \textsc{np}\textsubscript{12}-axe\\
\glt ‘That man took an axe…’ (NF\_Narr15)
\z

To refer back to referents that are not salient in the discourse, the series IV demonstrative is used. Example (\ref{bkm:Ref450572179}) is taken from the beginning of the narrative about the man, his wife and the wife’s little sister. The wife’s sister has been introduced, but only briefly and since she was last discussed, the focus of the story has been on the man and his wife. Now the wife’s sister, referred to by means of \textit{kèná kâncè} ‘that small child’, is reintroduced into the story, but with a series IV rather than a series III demonstrative as the result of this participant’s low salience.

\ea
\label{bkm:Ref450572179}
kàntí kèná kâncè káꜝyéndà nâkò\\
\gll kantí  kená    ka-ánce  ka-á̲-é̲nd-a    ná=ko\\
then  \textsc{dem}.\textsc{iv}\textsubscript{12}  \textsc{np}\textsubscript{12}-child  \textsc{pst}.\textsc{ipfv}-\textsc{sm}\textsubscript{1}-go-\textsc{fv}  \textsc{com}=\textsc{dem}.\textsc{iii}\textsubscript{12}\\
\glt ‘Then that small child that she was coming with…’ (NF\_Narr15)
\z

The series I demonstrative can be used to introduce a new referent. This is illustrated in (\ref{bkm:Ref437950079}), where \textit{òzyú} introduces a participant which had not yet been part of the story.

\ea
\label{bkm:Ref437950079}
néy’ òzyú múꜝkwámè àkêzyà\\
\gll né=o-zyú    mú-kwamé  a-ké̲ːzy-a\\
\textsc{com}=\textsc{aug}-\textsc{dem}.\textsc{i}\textsubscript{1}  \textsc{np}\textsubscript{1}-man  \textsc{sm}\textsubscript{1}-come-\textsc{fv}\\
\glt ‘And another man came.’ (NF\_Narr15)
\z

Series II demonstratives can be used with expressions of time, to indicate the current time period, as in (\ref{bkm:Ref99548100}--\ref{bkm:Ref99548101}).

\ea
\label{bkm:Ref99548100}
èyìnó nsûndà\\
\gll e-inó    N-súnda\\
\textsc{aug}-\textsc{dem}.\textsc{ii}\textsubscript{9}  \textsc{np}\textsubscript{9}-week\\
\glt ‘this week’ (ZF\_Elic14)
\z

\ea
\label{bkm:Ref99548101}
mwáìnò ènàkò shìtúꜝhárà\\
\gll mwá-ino    e-N-nako    shi-tú̲-ha\textsubscript{H}r-á̲\\
\textsc{con}\textsubscript{18}-\textsc{dem}.\textsc{ii}\textsubscript{9}  \textsc{aug}-\textsc{np}\textsubscript{9}-time  \textsc{inc}-\textsc{sm}\textsubscript{1PL}.\textsc{rel}-live-\textsc{fv}\\
\glt ‘This time that we are now living in…’ (ZF\_Conv13)
\z

This temporal function of the series II demonstrative is also reflected in the use of the locative demonstrative of class 16, which can be used adverbially meaning ‘(right) now’, as in (\ref{bkm:Ref491101554}--\ref{bkm:Ref491101556}).

\ea
\label{bkm:Ref491101554}
àbàntù \textbf{hánù} sìbàyèndàngàkó nèmótà\\
\gll a-ba-ntu    hanú    si-ba-end-ang-a=kó̲      ne=N-motá\\
\textsc{aug}-\textsc{np}\textsubscript{2}-person  \textsc{dem}.\textsc{ii}\textsubscript{16}  \textsc{inc}-\textsc{sm}\textsubscript{2}-go-\textsc{hab}-\textsc{fv}=\textsc{loc}\textsubscript{17}  \textsc{com}=\textsc{np}\textsubscript{9}-car\\
\glt ‘People, \textbf{now}, they go there with cars.’ (as opposed to earlier, when they would go with oxcarts) (NF\_Narr17)
\z

\newpage
\ea
\label{bkm:Ref491101556}
òmwâncè kàrí kàákìshùwírè nênjà kònó \textbf{hànô} shààkìshùwírè nênjà\\
\gll o-mu-ánce    ka-rí    ka-á̲-ki\textsubscript{H}-shu\textsubscript{H}w-í̲re      nénja konó  hanó    sha-a-ki\textsubscript{H}-shu\textsubscript{H}w-í̲re  nénja\\
\textsc{aug}-\textsc{np}\textsubscript{1}-child  \textsc{neg}-be  \textsc{pst}.\textsc{ipfv}-\textsc{sm}\textsubscript{1}-\textsc{refl}-feel-\textsc{stat}  well but  \textsc{dem}.\textsc{ii}\textsubscript{16} \textsc{inc}-\textsc{sm}\textsubscript{1}-\textsc{refl}-feel-\textsc{stat}  well\\
\glt ‘The child was not feeling well (earlier), but \textbf{now} she is feeling well.’ (ZF\_Elic14)
\z

Aside from expressing a temporal adverb, which is restricted to the demonstratives of locative class 16, demonstratives of all three locative classes, viz. 16, 17 and 18, can be used as locative adverbs. These demonstratives can describe general locations for class 16, as in (\ref{bkm:Ref99548183}), and 17, as in (\ref{bkm:Ref99548184}), and a contained location, e.g. ‘in there/here’, for class 18, as in (\ref{bkm:Ref99548185}).

\ea
\label{bkm:Ref99548183}
bàzyíménè hênà\\
\gll ba-zyi\textsubscript{H}mé̲n-e    héna\\
\textsc{sm}\textsubscript{2}-stand-\textsc{stat}  \textsc{dem}.\textsc{iv}\textsubscript{16}\\
\glt ‘S/he stands there.’ (NF\_Elic17)
\z

\ea
\label{bkm:Ref99548184}
wáshàkêːzyì kûnò kùshàmbà ndíshâmbà\\
\gll o-ásha-ké̲ːzy-i      kúno    ku-shamb-a  ndí̲-shá̲mb-a\\
\textsc{sm}\textsubscript{2SG}-\textsc{neg}.\textsc{sbjv}-come-\textsc{neg}  \textsc{dem}.\textsc{ii}\textsubscript{17}  \textsc{inf}-bath-\textsc{fv}  \textsc{sm}\textsubscript{1SG}.\textsc{rel}-bath-\textsc{fv}\\
\glt ‘You cannot come here, I am bathing.’ (ZF\_Elic14)
\z

\ea
\label{bkm:Ref99548185}
bàrèrè mwínà\\
\gll ba-re\textsubscript{H}re    mwiná\\
\textsc{sm}\textsubscript{2}-sleep.\textsc{stat}  \textsc{dem}.\textsc{iv}\textsubscript{18}\\
\glt ‘They are asleep in there.’ (NF\_Elic17)
\z
\subsection{Connectives}
\label{bkm:Ref492133177}\hypertarget{Toc75352648}{}\label{bkm:Ref492133189}
Connective constructions are used to link two nouns or pronouns through use of a connective clitic. (\ref{bkm:Ref437444323}) gives an example of a connective construction in Fwe.

\ea
\label{bkm:Ref437444323}
mìnwè yómwânce\\
\glll mi-nwe  i-ó=    mu-ánce\\
head    connective  dependent\\
\textsc{np}\textsubscript{4}-finger  \textsc{pp}\textsubscript{4}-\textsc{con}=  \textsc{np}\textsubscript{1}-child\\
\glt ‘the fingers of the child’ (ZF\_Elic14)
\z

Similar markers are found in many Bantu languages, and referred to as connective, associative, genitive or connexive (see \citealt{Velde2013}). One of the points on which Bantu languages differ is the degree to which the connective is phonologically integrated into the noun. In Fwe the connective functions as a clitic, as it is phonologically integrated into the host noun, but displays the syntactic behavior of a free word.

The connective clitic consists of a connective stem and a pronominal prefix (see \tabref{tab:4:1}), which agrees in noun class with the head of the connective construction. The connective stem consists of a single vowel, which is determined by the noun class of the dependent of the connective construction, though in this case there are significant differences between Zambian and Namibian Fwe. In Namibian Fwe, the connective stem is identical to the vowel of the augment. This is illustrated in (\ref{bkm:Ref98512478}--\ref{bkm:Ref98512481}) with a connective clitic that has a pronominal prefix of class 3, which is realized as \textit{w-o}-, \textit{w-e}- or \textit{w-a}-, depending on the augment of the following noun.

\ea
\label{bkm:Ref98512478}
mùcírà wóꜝndávù (< òndávù ‘lion’)\\
\gll mu-círa  u-ó=∅-ndavú\\
\textsc{np}\textsubscript{3}-tail  \textsc{pp}\textsubscript{3}-\textsc{con}=\textsc{np}\textsubscript{1a}-lion\\
\glt ‘the tail of a lion’
\z

\ea
mùbárá ꜝwènjûò (< ènjûò ‘house’)\\
\gll mu-bará  u-é=N-júo\\
\textsc{np}\-\textsubscript{3}-color  \textsc{pp}\textsubscript{3}-\textsc{con}=\textsc{np}\textsubscript{9}-house\\
\glt ‘the color of the house’
\z

\ea
\label{bkm:Ref98512481}
mùbárá ꜝwámàbûnà (< àmàbûnà ‘leaves’)\\
\gll mu-bará  u-á=ma-búna\\
\textsc{np}\textsubscript{3}-color  \textsc{pp}\textsubscript{3}-\textsc{con}=\textsc{np}\textsubscript{6}-leaf\\
\glt ‘the color of the leaves’ (NF\_Elic15)
\z

In Zambian Fwe, the vowel of the connective stem is always /o/, regardless of the augment of the noun with which the connective is used, as in (\ref{bkm:Ref98512501}--\ref{bkm:Ref98512503}).

\ea
\label{bkm:Ref98512501}
téꜝndé ꜝryóꜝndávù (< òndávù ‘lion’)\\
\gll ∅-téndé  ri-ó=∅-ndavú\\
\textsc{np}\textsubscript{5}-leg  \textsc{pp}\textsubscript{5}-\textsc{con}=\textsc{np}\textsubscript{1a}-lion\\
\glt ‘the leg of the lion’
\z

\ea
cìtúwá cònjûò (< ènjûò ‘house’)\\
\gll ci-tuwá  ci-ó=N-júo\\
\textsc{np}\textsubscript{7}-roof  \textsc{pp}\textsubscript{7}-\textsc{con}=\textsc{np}\textsubscript{9}-house\\
\glt ‘the roof of the house’
\z

\ea
\label{bkm:Ref98512503}
téꜝndé ꜝryókàzyùnì (< àkàzyùnì ‘bird’)\\
\gll ∅-téndé  ri-ó=ka-zyuni\\
\textsc{np}\textsubscript{5}-leg  \textsc{pp}\textsubscript{5}-\textsc{con}=\textsc{np}\textsubscript{12}-bird\\
\glt ‘the leg of the bird’ (ZF\_Elic14)
\z

The form of the connective also changes depending on the nature of the dependent noun. When the dependent is a noun that cannot take an augment, the vowel of the connective stem is always /a/, in both Namibian and Zambian Fwe. This is the case with proper names, as in (\ref{bkm:Ref492119044}), and adverbs, as in (\ref{bkm:Ref491175346}). The vowel of the connective is also realized as \textit{a} when used with a demonstrative pronoun, as in (\ref{bkm:Ref71190780}--\ref{bkm:Ref71190782}), as opposed to when the connective is used with an adnominal demonstrative, in which case the vowel of the connective is determined by the augment of the demonstrative; see (\ref{bkm:Ref491276208}--\ref{bkm:Ref491276210}).

\ea
\label{bkm:Ref492119044}
hànjà \textbf{ryaRebecca}\\
\gll hanja  ri-a=Rebecca\\
hand  \textsc{pp}\textsubscript{5}-\textsc{con}=Rebecca\\
\glt ‘Rebecca’s hand’ (ZF\_Elic14)
\z

\ea
\label{bkm:Ref491175346}
èzìàmbò \textbf{zàshûnù} nzícìkóró ꜝcámàyùnì\\
\gll e-zi-ambo    zi-a=shúnu    N-zí-ci-koró    ci-á=mayuni\\
\textsc{aug}-\textsc{np}\textsubscript{8}-topic  \textsc{pp}\textsubscript{8}-\textsc{con}=today  \textsc{cop}-\textsc{pp}\textsubscript{8}-\textsc{np}\textsubscript{7}-school  \textsc{pp}\textsubscript{7}-\textsc{con}=Mayuni\\
\glt ‘Today’s topic is Mayuni school.’ (NF\_Song17)
\z

\ea
\label{bkm:Ref71190780}
bànyûmbù nèmìcírà \textbf{yábò}\\
\gll ba-nyúmbu    ne=mi-círa    i-á=bo\\
\textsc{np}\textsubscript{2}-wildebeest  \textsc{com}=\textsc{np}\textsubscript{4}-tail  \textsc{pp}\textsubscript{4}-\textsc{con}=\textsc{dem}.\textsc{iii}\textsubscript{2}\\
\glt ‘The wildebeests and their tails.’ (NF\_Song17)
\z

\ea
\label{bkm:Ref71190782}
ècìntù nècìntù cìkwèsì òbùrótù \textbf{bwácò} nòbùbbí ꜝ\textbf{bwácò}\\
\gll e-ci-ntu    ne=ci-ntu    ci-kwesi  o-bu-rótu                                                                                                                                    bu-a=có    no=bu-bbí      bu-a=có\\
\textsc{aug}-\textsc{np}\textsubscript{7}-thing  \textsc{com}=\textsc{np}\textsubscript{7}-thing  \textsc{sm}\textsubscript{7}-have  \textsc{aug}-\textsc{np}\textsubscript{14}-good \textsc{pp}\textsubscript{14}-\textsc{con}=\textsc{dem}.\textsc{iii}\textsubscript{7}  \textsc{com}=\textsc{aug}-\textsc{np}\textsubscript{14}-bad  \textsc{pp}\textsubscript{14}-\textsc{con}=\textsc{dem}.\textsc{iii}\textsubscript{7}\\
\glt ‘Everything has its advantage and its disadvantage.’ (ZF\_Conv13)
\z

Another group of nouns that never take an augment are nouns marked with a locative prefix of class 16, 17 or 18. With these nouns, however, the vowel of the connective is not consistently realized as \textit{a}-, but as \textit{o}- with class 17 and 18, as in (\ref{bkm:Ref99548320}--\ref{bkm:Ref75251506}), and as \textit{a}- with class 16, as in (\ref{bkm:Ref75251507}). These forms resemble the augment, which is determined by vowel harmony with the nominal prefix, and therefore the expected augment with class 16 would be \textit{a-}, and \textit{o-} with class 17 and 18, even though these nouns may never take an augment.

\ea
\label{bkm:Ref99548320}
bàntù \textbf{bòkúmùnzì}\\
\gll ba-ntu  ba-o=kú-mu-nzi\\
\textsc{np}\textsubscript{2}-person  \textsc{pp}\textsubscript{2}-\textsc{con}=\textsc{np}\textsubscript{17}-\textsc{np}\textsubscript{3}-village\\
\glt ‘the people of the village’
\z

\ea
\label{bkm:Ref75251506}
zíryó ꜝ\textbf{zómúrùwà}\\
\gll zi-ryó    zi-o=mú-ru-wa\\
\textsc{np}\textsubscript{8}-food  \textsc{pp}\textsubscript{8}-\textsc{con}=\textsc{np}\textsubscript{18}-\textsc{np}\textsubscript{11}-field\\
\glt ‘the crops of the field’
\z

\ea
\label{bkm:Ref75251507}
zíryó \textbf{zàhámùkítì}\\
\gll zi-ryó    zi-a=há-mu-kití\\
\textsc{np}\textsubscript{8}-food  \textsc{pp}\textsubscript{8}-\textsc{con}=\textsc{np}\textsubscript{16}-\textsc{np}\textsubscript{3}-party\\
\glt ‘the food at the party’
\z

Nouns that take a secondary class 2 prefix (used to mark respect; see \sectref{bkm:Ref489005545}) also never take an augment. When such a noun takes a connective clitic, the connective stem is reduced to zero, as in (\ref{bkm:Ref99548365}).

\ea
\label{bkm:Ref99548365}
ndóꜝrúfù rùbànyámùzàmbàràrà kúnjòvù\\
\gll ndó-ru-fú      ru-∅=ba-nyámuzambarara    kú-∅-njovu\\
\textsc{cop}.\textsc{def}\textsubscript{11}-\textsc{np}\textsubscript{11}-death  \textsc{pp}\textsubscript{11}-\textsc{con}=\textsc{np}\textsubscript{2}-Nyamuzambarara  \textsc{np}\textsubscript{17}-\textsc{np}\textsubscript{1a}-elephant\\
\glt ‘That is the death of Mrs. Nyamuzambarara by an elephant.’ (ZF\_Narr15)
\z

\tabref{tab:4:20} gives an overview of the different forms of the connective clitic found in Fwe.

\begin{table}
\small
\label{bkm:Ref98512528}\caption{\label{tab:4:20}Connective clitics (including pronominal prefix)}
\begin{tabularx}{\textwidth}{lQp{25mm}p{19mm}l}
\lsptoprule
& nouns with an augment /a/;\newline certain augmentless nouns;\newline demonstrative pronouns & in Zambian Fwe; nouns with an augment /o/ & nouns~with an augment /e/ & honorifics\\
\midrule
1 & wa & o/w & we & u\\
1a & wa & o/w & we & u\\
2 & ba & bo & be & ba\\
3 & wa & o/wo & we & u\\
4 & ya & yo & ye & i\\
5 & rya & ryo & rye & ri\\
6 & a & o & e & a\\
7 & ca & co & ce & ci\\
8 & za & zo & ze & zi\\
9 & ya & yo & ye & i\\
10 & za & zo & ze & zi\\
11 & rwa & ro & rw & ru\\
12 & ka & ko & ke & ka\\
13 & twa & to & twe & tu\\
14 & bwa & bo & bwe & bu\\
15 & kwa & ko & kwe & ku\\
\lspbottomrule
\end{tabularx}
\end{table}

The large number of allomorphs and regional variants of the connective can mostly be explained historically as the result of vowel hiatus resolution between a putative earlier connective stem *a and the vowel of the augment. That the original form of the connective was -\textit{a} is shown by its use with certain nouns that cannot take an augment. This is in line with the analysis of a canonical Bantu connective construction by {\citet{Velde2013}}, where the connective stem is \textit{a}, as well as with its reconstruction for Proto-Bantu by {\citet{Meeussen1967}}. The forms of the connective where the vowel has changed to \textit{e} or \textit{o} are the result of coalescence with the vowel of the augment. In Zambian Fwe, a further development has taken place where the connective stem with the vowel \textit{o}, as a result of coalescence with the augment \textit{o}- of class 1, 1a, 3, 11, 13, 14 and 15, was extended to nouns of all other classes, where the augment is \textit{a-} or \textit{e-}. This process of analogical leveling has not affected Namibian Fwe.

Synchronically, the different forms of the connective can no longer be explained as coalescence of a vowel \textit{a} of the connective stem with the augment of the dependent noun, especially not in Zambian Fwe, where the vowel \textit{o} is even used with nouns that do not take \textit{o-} as their augment. Even in Namibian Fwe, if the different forms of the connective were the result of coalescence with the augment, forms where coalescence does not take place would also be expected, because the augment vowel in Fwe is optional (see \sectref{bkm:Ref444175456}).

Both the connective stem and the pronominal prefix are underlyingly toneless. The connective clitic may be realized as high-toned only when the floating high tone of the nominal augment attaches to it (see \sectref{bkm:Ref444175456} on the formal properties of the nominal augment). Examples of high-toned connective clitics are given in (\ref{bkm:Ref491179934}--\ref{bkm:Ref491179935}).

\ea
\label{bkm:Ref491179934}
mùcírà wóꜝndávù\\
\gll mu-círa  u-ó=∅-ndavú\\
\textsc{np}\textsubscript{3}-tail  \textsc{pp}\textsubscript{3}-\textsc{con}=\textsc{np}\textsubscript{1a}-lion\\
\glt ‘tail of a lion’
\z

\ea
\label{bkm:Ref491179935}
ènshùkí ꜝzómùkêntù\\
\gll e-N-shukí    zi-ó=mu-kéntu\\
\textsc{aug}-\textsc{np}\textsubscript{10}-hair  \textsc{pp}\textsubscript{10}-\textsc{con}=\textsc{np}\textsubscript{1}-woman\\
\glt ‘the hair of the woman’ (ZF\_Elic14)
\z

When the connective is used with a dependent noun that can never take an augment, the connective stem is consistently realized as low-toned, as illustrated with locative-marked nouns in (\ref{bkm:Ref71188211}--\ref{bkm:Ref71188213}).

\ea
\label{bkm:Ref71188211}
mìnwè yòkúmàànjà\\
\gll mi-nwe  i-o=kú-ma-anja\\
\textsc{np}\textsubscript{4}-finger  \textsc{pp}\textsubscript{4}-\textsc{con}=\textsc{np}\textsubscript{17}-\textsc{np}\textsubscript{6}-hand\\
\glt ‘fingers of the hands’ (ZF\_Elic14)
\z

\ea
\label{bkm:Ref71188213}
bàntù bòmúnjûò\\
\gll ba-ntu  ba-o=mú-N-júo\\
\textsc{np}\textsubscript{2}-person  \textsc{pp}\textsubscript{2}-\textsc{con}=\textsc{np}\textsubscript{18}-\textsc{np}\textsubscript{9}-house\\
\glt ‘people of the house’ (NF\_Elic15)
\z

{\citet{Velde2013}} notes that the connective element in Bantu languages generally has an intermediate position between affix and word, and therefore analyzes it as a clitic. The same applies to the connective in Fwe. The phonological integration is seen from the fact that the vowel of the connective stem interacts with the augment of the noun it attaches to, and from its tonal behavior: the connective clitic may be the target for high tone shift, as in (\ref{bkm:Ref491181481}), where the high tone of the syllable \textit{mú} spreads onto the preceding connective \textit{ryó}. High tone spread never crosses word boundaries (see \sectref{bkm:Ref430865664}), thus proving the phonological integration of the connective clitic into the noun.

\ea
\label{bkm:Ref491181481}
èzwáyí ꜝryómúbùsùnsò\\
\gll e-∅-zwaí  ri-o=mú-bu-sunso\\
\textsc{aug}-\textsc{np}\textsubscript{5}-salt  \textsc{pp}\textsubscript{5}-\textsc{con}=\textsc{np}\textsubscript{18}-\textsc{np}\textsubscript{14}-relish\\
\glt ‘the salt of the relish’ (NF\_Elic15)
\z

Syntactically, the connective clitic behaves like a separate word. When combined with nouns that have a pre-nominal modifier, such as a demonstrative, as in (\ref{bkm:Ref491276208}--\ref{bkm:Ref491276210}), the connective clitic is marked on the demonstrative, not the noun itself. This shows that the connective behaves like a phrasal clitic, rather than a nominal affix.

\ea
\label{bkm:Ref491276208}
\textbf{òmùkìtí} \textbf{ꜝ}\textbf{wábèná} \textbf{bàntù} mànì wáràtèndàhàrà\\
\gll o-mu-kití    u-á=bená    ba-ntu mani  o-ára-tend-ahar-a \\
\textsc{aug}-\textsc{np}\textsubscript{3}-party  \textsc{pp}\textsubscript{3}-\textsc{con}=\textsc{dem}.\textsc{iv}\textsubscript{2}  \textsc{np}\textsubscript{2}-person when  \textsc{sm}\textsubscript{3}-\textsc{rem}.\textsc{fut}-do-\textsc{neut}-\textsc{fv}\\
\glt ‘Those people’s party, when will it take place?’ (NF\_Elic17)
\z

\ea
\label{bkm:Ref491276210}
\textbf{èmísì} \textbf{yècí} \textbf{cìshámù} mùshámù\\
\gll e-mi-ísi    i-e=cí      ci-shamú  N-mu-shamú\\
\textsc{aug}-\textsc{np}\textsubscript{4}-root  \textsc{pp}\textsubscript{4}-\textsc{con}=\textsc{dem}.\textsc{i}\textsubscript{7}  \textsc{np}\textsubscript{7}-tree  \textsc{cop}-\textsc{np}\textsubscript{3}-medicine\\
\glt ‘The roots of this tree are medicine.’ (ZF\_Elic14)
\z

The head of the connective construction can be left unexpressed, so the construction consists of a dependent only. In this case, the noun class of the pronominal prefix is determined by the intended or implied noun. In (\ref{bkm:Ref491184139}), a speaker asks where her \textit{citenge} (piece of fabric worn as wrap-around skirt) is; the response uses headless connectives to ask for a description of the citenge, marked for agreement with the class 7 noun \textit{citenge} with class 7 pronominal prefixes.

\ea
\label{bkm:Ref491184139}
\ea
nòndìbónènì ècìtèngé ꜝcángù\\
\gll no-ndi-bón-en-i        e-ci-tengé    ci-angú\\
\textsc{sm}\textsubscript{2SG}.\textsc{pst}-\textsc{om}\textsubscript{1SG}-see-\textsc{appl}-\textsc{npst}.\textsc{pfv}  \textsc{aug}-\textsc{np}\textsubscript{7}-citenge  \textsc{pp}\textsubscript{7}-\textsc{poss}\textsubscript{1SG}\\
\glt ‘Have you seen my citenge?’

\ex
cómùshòbònjí cókùsùbìrà cókùsìhà cókùtùbà\\
\gll ci-ó=mu-shobo-njí    ci-ó=ku-subir-a\\
\textsc{pp}\textsubscript{7}-\textsc{con}=\textsc{np}\textsubscript{3}-type-what  \textsc{pp}\textsubscript{7}-\textsc{con}=\textsc{inf}-be\_red-\textsc{fv}\\
ci-ó=ku-sih-a     ci-ó=ku-tub-a\\
\textsc{pp}\textsubscript{7}-\textsc{con}=\textsc{inf}-be\_black-\textsc{fv}  \textsc{pp}\textsubscript{7}-\textsc{con}=\textsc{inf}-be\_white-\textsc{fv}\\
\glt ‘What kind? A red one, a black one, a white one?’ (NF\_Elic15)
\z\z

Semantically, the relationship between the two nouns in a connective construction can be interpreted in different ways. A connective can be used to indicate possession, where the dependent is the possessor and the head the possessee, as in (\ref{bkm:Ref99548645}--\ref{bkm:Ref99548647}).

\ea
\label{bkm:Ref99548645}
mùndáré òbàmùrútì\\
\gll mu-ndaré  u-o=ba-mu-rutí\\
\textsc{np}\textsubscript{3}-maize  \textsc{pp}\textsubscript{3}-\textsc{con}=\textsc{np}\textsubscript{2}-\textsc{np}\textsubscript{1}-teacher\\
\glt ‘the maize of the teacher’
\z

\ea
\label{bkm:Ref99548647}
njûò yámùyéꜝnzángù\\
\gll N-júo    i-á=mu-énz-angú\\
\textsc{np}\textsubscript{9}-house  \textsc{pp}\textsubscript{9}-\textsc{con}=\textsc{np}\textsubscript{1}-friend-\textsc{poss}\textsubscript{1SG}\\
\glt ‘the house of my friend’ (ZF\_Elic14)
\z

The relationship expressed by a connective construction may be a relationship of qualification, where the dependent describes some property of the head, as in (\ref{bkm:Ref99548665}--\ref{bkm:Ref99548667}).

\ea
\label{bkm:Ref99548665}
mwánà wècìsìzánì\\
\gll mu-ána  u-e=ci-sizaní\\
\textsc{np}\textsubscript{1}-child  \textsc{pp}\textsubscript{1}-\textsc{con}=\textsc{np}\textsubscript{7}-female\\
\glt ‘a female child’
\z

\ea
cíꜝkwáꜝmé cáꜝmárì\\
\gll cí-kwamé  ci-á=marí\\
\textsc{np}\textsubscript{7}-man  \textsc{pp}\textsubscript{7}-\textsc{con}=polygamy\\
\glt ‘a polygamous man’ (NF\_Elic15)
\z

\ea
\label{bkm:Ref99548667}
ràpá ꜝryókùcènà\\
\gll rapá    ri-ó=ku-cen-a\\
courtyard  \textsc{pp}\textsubscript{5}-\textsc{con}=\textsc{inf}-be\_clean-\textsc{fv}\\
\glt ‘a clean courtyard’ (ZF\_Elic14)
\z

A connective may also express the location of the head with respect to the dependent, in which case the dependent is marked with a locative prefix of class 16, 17 or 18, as in (\ref{bkm:Ref99548933}--\ref{bkm:Ref99548934}).

\ea
\label{bkm:Ref99548933}
mìnwè yòkúmàànjà\\
\gll mi-nwe  i-o=kú-ma-anja\\
\textsc{np}\textsubscript{4}-finger  \textsc{pp}\textsubscript{4}-\textsc{con}=\textsc{np}\textsubscript{17}-\textsc{np}\textsubscript{6}-hand\\
\glt ‘fingers of the hands’ (ZF\_Elic14)
\z

\ea
\label{bkm:Ref99548934}
bàntù bòmúmùnzì\\
\gll ba-ntu  ba-o=mú-mu-nzi\\
\textsc{np}\textsubscript{2}-person  \textsc{pp}\textsubscript{2}-\textsc{con}=\textsc{np}\textsubscript{18}-\textsc{np}\textsubscript{3}-village\\
\glt ‘people from the village’ (NF\_Elic17)
\z

When the connective is used on an infinitive, it may take on some properties of a separate clause. The infinitive may, for instance, have its own object, either marked through a separate noun, as in (\ref{bkm:Ref448750425}), or with an object marker on the verb, as in (\ref{bkm:Ref448750435}).

\ea
\label{bkm:Ref448750425}
ndààzyá màshérêŋì ókùkwèrès’ éꜝmótà\\
\gll ndi-aazyá    ma-sheréŋi a-ó=ku-kwer-es-á      e-N-motá \\
\textsc{sm}\textsubscript{1SG}-have\_not  \textsc{np}\textsubscript{6}-money \textsc{pp}\textsubscript{6}-\textsc{con}=\textsc{inf}-board-\textsc{caus}-\textsc{fv}  \textsc{aug}-\textsc{np}\textsubscript{9}-car\\
\glt ‘I don’t have money for a taxi.’ (NF\_Elic15)
\z

\ea
\label{bkm:Ref448750435}
mùròrà ókùtúsànzìsà\\
\gll mu-rora  u-ó=ku-tú-sanz-is-a\\
\textsc{np}\textsubscript{3}-soap  \textsc{pp}\textsubscript{3}-\textsc{con}=\textsc{inf}-\textsc{om}\textsubscript{13}-wash-\textsc{caus}-\textsc{fv}\\
\glt ‘soap for washing them (dishes) with’ (NF\_Elic17)
\z
\subsection{Quantifiers}\label{bkm:Ref99546522}
\begin{sloppypar}
Fwe has the following quantifiers: \textit{onshéː} ‘all’, \textit{ngíː} ‘many’, \textit{mwi(nya)/munya} ‘some, other, a certain’. (Another quantifier, \textit{céː} ‘few’, functions as an adjective; see \sectref{bkm:Ref491277755}.) Quantifiers display agreement with the noun through use of pronominal prefixes (see \tabref{tab:4:1}). For class 1 and 1a, where two forms of the pronominal prefix are attested, the form \textit{zyu-} is used rather than the form \textit{u-}.
\end{sloppypar}

The quantifier \textit{onshéː} is used with the meaning ‘all, every, each, any’. It is typically used after the noun it modifies, as in (\ref{bkm:Ref99549023}), but may also be used before the noun, as in (\ref{bkm:Ref99549040}). The pronominal prefix used with this quantifier is realized as low-toned.

\ea
\label{bkm:Ref99549023}
\textbf{èŋòmbè} \textbf{zònshéː} nàzáùrìsìwà\\
\gll e-N-ŋombe    zi-onshéː  na-zí̲-a-ur-is-iw-a\\
\textsc{aug}-\textsc{np}\textsubscript{10}-cow  \textsc{pp}\textsubscript{10}-all  \textsc{rem}-\textsc{sm}\textsubscript{10}-\textsc{pst}-buy-\textsc{caus}-\textsc{pass}-\textsc{fv}\\
\glt ‘All the cattle have been sold.’ (ZF\_Elic14)
\z

\ea
\label{bkm:Ref99549040}
\textbf{yònshéː} \textbf{èntúsó} èyò ndíꜝóːrà òkùkùtùsà\\
\gll i-onshéː  e-N-tusó e-yo    ndí̲-ó̲ːr-a    o-ku-ku-tus-a \\
\textsc{pp}\textsubscript{9}-all    \textsc{aug}-\textsc{np}\textsubscript{9}-help \textsc{aug}-\textsc{dem}.\textsc{iii}\textsubscript{9}  \textsc{sm}\textsubscript{1SG}.\textsc{rel}-can-\textsc{fv}  \textsc{aug}-\textsc{inf}-\textsc{om}\textsubscript{2SG}-help-\textsc{fv}\\
\glt ‘Any help that I can provide to you…’ (NF\_Narr17)
\z

The quantifier \textit{onshéː} may also be used with pronominal prefixes of the first and second person, with an interpretation of ‘all of us/you; us/you together’, as in (\ref{bkm:Ref99549290}--\ref{bkm:Ref99549291}).

\ea
\label{bkm:Ref99549290}
kùààzyá òzyò áshàká òkúfwà \textbf{twènshéː} tùsháká ꜝbúmì\\
\gll ku-aazyá  o-zyo    á̲-shak-á̲    o-ku-fw-á tu-enshéː  tu-shak-á̲    bu-mí \\
\textsc{sm}\textsubscript{17}-be\_not  \textsc{aug}-\textsc{dem}.\textsc{iii}\textsubscript{1}  \textsc{sm}\textsubscript{1}.\textsc{rel}-want-\textsc{fv}  \textsc{aug}-\textsc{inf}-die-\textsc{fv} \textsc{pp}\textsubscript{1PL}-all  \textsc{sm}\textsubscript{1PL}-want-\textsc{fv}  \textsc{np}\textsubscript{14}-life\\
\glt ‘There is no one who wants to die, \textbf{we} \textbf{all} want to be alive.’ (NF\_Song17)
\z

\ea
tùyéndè \textbf{twènshêː}\\
\gll tu-é̲nd-e      tu-enshé\\
\textsc{sm}\textsubscript{1PL}-walk-\textsc{pfv}.\textsc{sbjv}  \textsc{pp}\textsubscript{1PL}-all\\
\glt ‘Shall we walk \textbf{together}?’ (NF\_Elic15)
\z

\ea
\label{bkm:Ref99549291}
háìbà \textbf{mwènshéː} mùbèrékà\\
\gll háiba  mu-enshéː  mu-berek-á̲\\
if  \textsc{pp}\textsubscript{2PL}-all  \textsc{sm}\textsubscript{2PL}-work-\textsc{fv}\\
\glt ‘If \textbf{you} \textbf{all} are working…’ (ZF\_Conv13)
\z

The quantifier \textit{ngíː} ‘many; other’ is typically used after the noun it modifies, as in (\ref{bkm:Ref99549339}--\ref{bkm:Ref99549360}), though a prenominal position is also possible, as in (\ref{bkm:Ref99549341}).

\ea
\label{bkm:Ref99549339}
zìzyùnì zîngîː\\
\gll zi-zyuni  zí-ngíː\\
\textsc{np}\textsubscript{8}-bird  \textsc{pp}\textsubscript{8}-many\\
\glt ‘many birds’ (ZF\_Elic14)
\z

\ea
\label{bkm:Ref99549360}
nàdàmwá ꜝ\textbf{kúbàntù} \textbf{bângîː}\\
\gll na-dam-w-á̲      kú-ba-ntu    bá-ngíː\\
\textsc{sm}\textsubscript{1}.\textsc{pst}-beat-\textsc{pass}-\textsc{fv}  \textsc{np}\textsubscript{17}-\textsc{np}\textsubscript{2}-person  \textsc{pp}\textsubscript{2}-many\\
\glt ‘S/he was beaten by many people.’ (NF\_Elic17)
\z

\ea
\label{bkm:Ref99549341}
\textbf{zíngìː} \textbf{èmbúkà} bábàrâ bò\\
\gll zí-ngiː  e-N-búka    bá̲-bar-á̲    bo\\
\textsc{pp}\textsubscript{10}-many  \textsc{aug}-\textsc{np}\textsubscript{10}-book  \textsc{sm}\textsubscript{2}.\textsc{rel}-read-\textsc{fv}  \textsc{dem}.\textsc{iii}\textsubscript{2}\\
\glt ‘S/he reads many books.’ (NF\_Elic15)
\z

The quantifier \textit{mwi} can be realized as \textit{mwi}, \textit{mwinya}, or \textit{munya}, without observable changes in meaning. This quantifier is used with the meaning ‘some, other, another, a certain’. It may be used before the noun, as in (\ref{bkm:Ref99549401}), or after it, as in (\ref{bkm:Ref99549411}).

\ea
\label{bkm:Ref99549401}
\textbf{zyúmwì} múꜝkwámè\\
\gll zyú-mwi  mú-kwamé\\
\textsc{pp}\textsubscript{1}-other  \textsc{np}\textsubscript{1}-man\\
\glt ‘a certain man’ (ZF\_Elic14)
\z

\ea
\label{bkm:Ref99549411}
kùààzyá kùmwí òkò nèmúkàwánè \textbf{òbùhárò} \textbf{búmùnyà}\\
\gll ku-aazyá  ku-mwí  o-ko      ne-mú̲-ka-wá̲n-e o-bu-háro    bú-munya\\
\textsc{sm}\textsubscript{17}-be\_not  \textsc{pp}\textsubscript{17}-other  \textsc{aug}-\textsc{dem}.\textsc{iii}\textsubscript{17}  \textsc{rem}-\textsc{sm}\textsubscript{2PL}-\textsc{dist}-find-\textsc{pfv}.\textsc{sbjv} \textsc{aug}-\textsc{np}\textsubscript{14}-life  \textsc{pp}\textsubscript{14}-other\\
\glt ‘There is nowhere where you can find another life.’ (ZF\_Conv13)
\z

Used with a pronominal prefix of class 16, as in (\ref{bkm:Ref99549499}), this quantifier may have a temporal interpretation, e.g. ‘sometimes’.

\ea
\label{bkm:Ref99549499}
\textbf{hámùnyà} kàzíꜝyángà kwàrìzáùrì \textbf{hámùnyà} kàtúꜝzwángà kwàmakanga tùyá kwàrìnyântì\\
\gll há-munya  ka-zí̲-y-á̲ng-a    kwa-rizáuli há-munya  ka-tú̲-zw-á̲ng-a        kwa-makanga tu-y-á̲  kwa-rinyánti\\
\textsc{pp}\textsubscript{16}-other  \textsc{pst}.\textsc{ipfv}-\textsc{sm}\textsubscript{10}-go-\textsc{hab}-\textsc{fv}  \textsc{np}\textsubscript{17}-Lizauli \textsc{pp}\textsubscript{16}-other  \textsc{pst}.\textsc{ipfv}-\textsc{sm}\textsubscript{1PL}-come\_out-\textsc{hab}-\textsc{fv}  \textsc{np}\textsubscript{17}-Makanga \textsc{sm}\textsubscript{1PL}-go-\textsc{fv}  \textsc{np}\textsubscript{17}-Linyanti\\
\glt ‘\textbf{Sometimes} they would go to Lizauli. \textbf{Sometimes}, we would go from Makanga to Linyanti.’ (NF\_Narr17)
\z

Like other nominal modifiers, quantifiers may also be used nominally, replacing instead of modifying a noun. In this case, the quantifier takes the pronominal prefix that agrees in noun class with the noun it replaces or refers to, e.g. class 2 in (\ref{bkm:Ref496809163}) to indicate plural human referents, and class 1 in (\ref{bkm:Ref496809191}) to indicate a single human referent.

\ea
\label{bkm:Ref496809163}
\textbf{bònshéː} bàrwárà kàmpòrwè\\
\gll ba-onshéː  ba-rwá̲\textsubscript{H}r-a  ka-mporwe\\
\textsc{pp}\textsubscript{2}-all    \textsc{sm}\textsubscript{2}-be\_ill-\textsc{fv}  \textsc{np}\textsubscript{12}-diarrhea\\
\glt ‘\textbf{They} \textbf{all} suffer from diarrhea.’ (NF\_Elic17)
\z

\ea
\label{bkm:Ref496809191}
bàkéntù bòbírè \textbf{zyúmwì} ákùzârà òmùntù \textbf{zyúmwì} ákùzârà èŋwárárà\\
\gll ba-kéntu  ba-o=biré  zyú-mwi  á-ku-zár-a o-mu-ntu     zyú-mwi  á-ku-zár-a      e-∅-ŋwarará\\
\textsc{np}\textsubscript{2}-woman  \textsc{pp}\textsubscript{2}-\textsc{con}=two  \textsc{pp}\textsubscript{1}-other  \textsc{pp}\textsubscript{1}-\textsc{inf}-give\_birth-\textsc{fv} \textsc{aug}-\textsc{np}\textsubscript{1}-person   \textsc{pp}\textsubscript{1}-other  \textsc{pp}\textsubscript{1}-\textsc{inf}-give\_birth-\textsc{fv}  \textsc{aug}-\textsc{np}\textsubscript{5}-crow\\
\glt ‘Two women. \textbf{One} gave birth to a human being, \textbf{the} \textbf{other} \textbf{one} gave birth to a crow.’ (NF\_Narr17)
\z
\subsection{Possessives}
\label{bkm:Ref491333327}\hypertarget{Toc75352650}{}\label{bkm:Ref451868043}
Fwe has a small set of possessives stems, listed, with their underlying tone patterns, in \tabref{tab:4:21}.

\begin{table}
\label{bkm:Ref463366816}\caption{\label{tab:4:21}Possessive stems}

\begin{tabular}{lll}
\lsptoprule
& singular & plural\\
\midrule
1 & \textit{angú} & {\itshape etú}\\
2 & \textit{akó} & {\itshape enú}\\
3 & {\itshape akwé} & (\textit{abó})\\
\lspbottomrule
\end{tabular}
\end{table}

The possessive stem is marked for agreement with the head noun with a pronominal prefix (see \tabref{tab:4:1}). An example is given in (\ref{bkm:Ref494202008}), where the possessive stem \textit{etú} is marked with a pronominal prefix \textit{u-} of class 3, agreeing with the head noun \textit{mùnzí} ‘village’.

\ea
\label{bkm:Ref494202008}
mùnzí ꜝwétù\\
\gll mu-nzí  u-etú\\
\textsc{np}\textsubscript{3}-village  \textsc{pp}\textsubscript{3}-\textsc{poss}\textsubscript{1PL\-}\\
\glt ‘our village’
\z

Fwe lacks a dedicated possessive stem for the third person plural. Instead, the demonstrative of class 2 (the class for plural human nouns) is used, \textit{abó}, as in (\ref{bkm:Ref99549542}).

\ea
\label{bkm:Ref99549542}
òmùndáré ꜝwábò\\
\gll o-mu-ndaré  u-abó\\
\textsc{aug}\--\textsc{np}\textsubscript{3}-maize  \textsc{pp}\textsubscript{3}-\textsc{dem}.\textsc{iii}\textsubscript{2}\\
\glt ‘their maize’
\z

All possessives have a floating high tone which surfaces on the mora preceding the possessive, usually the last mora of the noun it modifies. In (\ref{bkm:Ref450574530}), the low-toned noun \textit{vùmò} ‘stomach’, is realized as \textit{vùmó} when followed by the possessive \textit{ryángù} ‘my’.

\ea
\label{bkm:Ref98835526}
\label{bkm:Ref450574530}
vùmó ꜝryángù\\
\gll ∅-vumó  ri-angú\\
\textsc{np}\textsubscript{5}-stomach  \textsc{pp}\textsubscript{5}-\textsc{poss}\textsubscript{1SG}\\
\glt ‘my stomach’ (ZF\_Elic14
\z

Possessives may be used adnominally, modifying a noun, or nominally, replacing a noun. When used adnominally, the possessive may follow the noun it modifies, as in (\ref{bkm:Ref98835526}), or may precede the noun it modifies, in which case focus lies on the possessive, as in (\ref{bkm:Ref485747165}). In this setting, another speaker has just finished telling a short story. The speaker focuses the possessive ‘my’ here to indicate that his story, too, is short.

\ea
\label{bkm:Ref485747165}
rwàngú rùtângò ndùfwíhì nórò\\
\gll ru-angú  ru-tángo  N-ru-fwíi    no=ró\\
\textsc{pp}\textsubscript{11}-\textsc{poss}\textsubscript{1SG}  \textsc{np}\textsubscript{11}-story  \textsc{cop}-\textsc{np}\textsubscript{11}-short  \textsc{com}=\textsc{dem}.\textsc{iii}\textsubscript{11}\\
\glt ‘My story is also short.’ (NF\_Narr17)
\z

When a possessive is used to replace a noun, the entity referred to can be inferred from context, and also provides the agreement prefix used on the possessive. In (\ref{bkm:Ref491186062}), two speakers discuss a cow; in the response, the possessive \textit{yángù} is used to refer back to this cow, and agrees in noun class by taking the pronominal prefix of class 9.

\ea
\label{bkm:Ref491186062}
\ea
ndàbónì èŋòmbè kúrwîzyì njákò\\
\gll ndi-a-bón-i      e-N-ŋombe    kú-ru-ízyi N-i-akó \\
\textsc{sm}\textsubscript{1SG}-\textsc{pst}-see-\textsc{npst}.\textsc{pfv}  \textsc{aug}\--\textsc{np}\textsubscript{9}-cow  \textsc{np}\textsubscript{17}-\textsc{np}\textsubscript{11}-river \textsc{cop}-\textsc{pp}\textsubscript{9}-\textsc{poss}\textsubscript{2SG}\\
\glt ‘I saw a cow at the river. Is it yours?’

\ex
íngà yángù kùmùnzì íkèrè\\
\gll ínga  i-angú    ku-mu-nzi    í̲-ke\textsubscript{H}re\\
no  \textsc{pp}\textsubscript{9}-\textsc{poss}\textsubscript{1SG} \textsc{np}\textsubscript{17}-\textsc{np}\textsubscript{3}-village  \textsc{sm}\textsubscript{9}.\textsc{rel}-stay.\textsc{stat}\\
\glt ‘No, mine is at home.’ (ZF\_Elic13)
\z\z

In some cases, the possessive stem may fuse with the noun it modifies as a suffix. This is restricted to a closed set of nouns expressing social or family relations, such as \textit{yenz} ‘friend’, as in (\ref{bkm:Ref99549571}), \textit{ana} ‘child’, as in (\ref{bkm:Ref99549589}), or \textit{isho} ‘father’, as in (\ref{bkm:Ref99549600}).

\ea
\label{bkm:Ref99549571}
\glll mùyéꜝnzángù\\
mu-énz-angú\\
\textsc{np}\textsubscript{1}-friend-\textsc{poss}\textsubscript{1SG}\\
\glt ‘my friend’
\z

\ea
\label{bkm:Ref99549589}
\glll àbáꜝnénù\\
a-ba-án-enú\\
\textsc{aug}-\textsc{np}\textsubscript{2}-child-\textsc{poss}\textsubscript{2PL}\\
\glt ‘your (\textsc{pl}) children’
\z

\ea
\label{bkm:Ref99549600}
\glll béshwétù\\
ba-ísho-etú\\
\textsc{np}\textsubscript{2}-father-\textsc{poss}\textsubscript{1PL}\\
\glt ‘our father’
\z

Suffixed possessives do not show agreement, but are otherwise very similar to the forms of the independently used possessive stems, except those of the second and third person singular, which have been reduced from \textit{akó} and \textit{akwé} in their independent form to \textit{-ó} and \textit{-é} in the suffixed form. \tabref{tab:4:22} gives the forms of the suffixed possessive stems in Fwe.

\begin{table}
\label{bkm:Ref98512548}\caption{\label{tab:4:22}Suffixed possessive stems}
\begin{tabular}{lll}
\lsptoprule
& singular & plural\\
\midrule
1 & \textit{-àngú} & {\itshape -ètú}\\
2 & \textit{-ó} & {\itshape -ènú}\\
3 & \textit{-é} & {\itshape -àbó}\\
\lspbottomrule
\end{tabular}
\end{table}

Some nouns that take suffixed possessives cannot occur without a possessive. Other nouns take suffixed possessives for the second and third person singular, and suffixed possessives for other persons, such as the noun \textit{mu-kúru} ‘elder sibling’ in (\ref{bkm:Ref436225558}).

\ea
\label{bkm:Ref436225558}
\glll mùkúrwê\\
mu-kúrw-é\\
\textsc{np}\textsubscript{1}-elder\_sibling-\textsc{poss}\textsubscript{3SG}\\
\glt ‘his/her (elder) sister’
\z

\ea
mùkúrù wángù\\
\gll mu-kúru    u-angú\\
\textsc{np}\textsubscript{1}-elder\_sibling  \textsc{pp}\textsubscript{1}-\textsc{poss}\textsubscript{1SG}\\
\glt ‘my (elder) sister’ (ZF\_Elic14)
\z
