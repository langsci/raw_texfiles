\chapter{Verbal derivation}
\label{bkm:Ref99105347}\label{bkm:Ref98758856}\label{bkm:Ref97890592}\hypertarget{Toc75352657}{}
Verbs in Fwe are morphologically highly complex, taking multiple derivational suffixes, discussed in this chapter, as well as complex inflectional morphology, discussed in chapters \ref{bkm:Ref100140368}-\ref{bkm:Ref100140375}. Verbal derivation in Fwe mainly makes use of suffixes, in addition to full and partial stem reduplication. Verbal derivational suffixes appear directly after the verb stem, and before inflectional suffixes. The rich verbal derivational morphology of Fwe is typical of Bantu languages, and most derivational suffixes are clear reflexes of common Bantu morphemes.

Derivational strategies differ in productivity. Some strategies are highly productive: they can be freely used to derive new verbs from a wide variety of existing verbs, have clear and regular semantic and syntactic functions, and most lexical verbs that can occur in a derived form also have an attested underived form. This the case for the passive, causative, applicative, and pluractional 2 (marked by stem reduplication). Given their high productivity, these suffixes tend to occur after other, less productive suffixes. The passive is always the last derivational suffix, even when combined with an equally productive causative, as in (\ref{bkm:Ref462826094}), or applicative, as in (\ref{bkm:Ref462826104}).

\ea
\label{bkm:Ref462826094}
\glll àzwìsìwâ\\
a-zw-is-iw-á̲\\
\textsc{sm}\textsubscript{1}-leave-\textsc{caus}-\textsc{pass}-\textsc{fv}\\
\glt ‘S/he was fired.’ (Lit.: ‘S/he was made to leave.’) (NF\_Elic15)
\z

\ea
\label{bkm:Ref462826104}
ècí cìpùrà ndìmè nàcápàngìrwà\\
\gll e-cí    ci-pura  ndi-me  na-cí̲-a-pang-ir-w-a\\
\textsc{aug}-\textsc{dem}.\textsc{i}\textsubscript{7}  \textsc{np}\textsubscript{7}-chair  \textsc{cop}-\textsc{pers}\textsubscript{1SG}  \textsc{rem}-\textsc{sm}\textsubscript{7}-\textsc{pst}-make-\textsc{appl}-\textsc{pass}-\textsc{fv}<\textsc{rel}>\\
\glt ‘The chair, it’s me that it was made for.’ (ZF\_Elic14)
\z

Less productive derivational strategies are the neuter, separative, impositive, and pluractional 1 suffixes. These occur in a large number of verbs: some of these also occur in an underived form, some do not occur in an underived form but do occur with another derivational suffix, and some only occur in their derived form. These derivational strategies cannot be used to freely derive new verbs, and although they have a clear semantic core, they also occur in verbs which do not seem to fit their basic semantic characterization. The intensive, reciprocal, extensive, tentive, and partial reduplication strategies are completely unproductive: they only occur in a handful of lexicalized verbs, and their semantic function cannot always clearly be established.

Most derivational suffixes have a -VC or -V(C)VC shape, and are underlyingly toneless, so that they surface as low-toned unless a melodic high tone is assigned, or if the syllable is the target of high tone retraction or high tone spread. Various forms of vowel and nasal harmony affect derivational suffixes. Vowel height harmony affects suffixes with /i/ and /u/, as discussed in \sectref{bkm:Ref451863900}, and nasal harmony affects suffixes with /r/, as discussed in \sectref{bkm:Ref70697295}.

Most derivational strategies influence the valency of the verb. The passive and the neuter suffix decrease valency, the causative and the applicative increase valency. The separative and impositive have two forms, a transitive and an intransitive form.

\begin{sloppypar}
Derivational suffixes also influence the lexical aspect of the verb. Verbs that take the passive, or the intransitive separative or impositive, all function as change-of-state verbs. Verbs derived with the neuter are used either as change-of-state verbs or as true statives; for more on lexical aspect, see \sectref{bkm:Ref74925053}.
\end{sloppypar}

The following sections discuss the formal, syntactic and semantic properties of each verbal derivation: the passive in \sectref{bkm:Ref452972446}, the causative in \sectref{bkm:Ref451514620}, the applicative in \sectref{bkm:Ref451514914}, the neuter in \sectref{bkm:Ref489452510}, the separative in \sectref{bkm:Ref485823385}, the impositive in \sectref{bkm:Ref450835510}, the two pluractional strategies in \sectref{bkm:Ref489866362}, the intensive in \sectref{bkm:Ref485997587}, the reciprocal in \sectref{bkm:Ref98773050}, the extensive in \sectref{bkm:Ref486253151}, the tentive in \sectref{bkm:Ref485997273}, and partial redupcliation in \sectref{bkm:Ref98773057}.

\largerpage
\section{Passive}
\label{bkm:Ref452972446}\hypertarget{Toc75352658}{}
The passive\footnote{In Bantu languages, the passive is typically treated as a derivational strategy, and it also functions as such in Fwe: it makes use of the same type of formal marking, e.g. a toneless verbal suffix of the shape -V(C), and the same syntactic properties, influencing the valency of the verb.}  is marked by a suffix -(\textit{i})\textit{w}, which follows the verb stem and precedes the final vowel of the verb, as in (\ref{bkm:Ref99025604}--\ref{bkm:Ref99025607}).

\ea
\label{bkm:Ref99025604}
cìshámú cìnàtémìwà\\
\gll ci-shamú  ci-na-tém-\textbf{iw}-a\\
\textsc{np}\textsubscript{7}-tree  \textsc{sm}\textsubscript{7}-\textsc{pst}-chop-\textsc{pass}-\textsc{fv}\\
\glt ‘The tree has been chopped.’ (ZF\_Elic14)
\z

\ea
\label{bkm:Ref99025607}
nzézò zíbònwâ\\
\gll nzé-zo    zí̲-bo\textsubscript{H}n-\textbf{w}-á̲\\
\textsc{cop}.\textsc{def}\textsubscript{8}-\textsc{dem}.\textsc{iii}\textsubscript{8}  \textsc{sm}\textsubscript{8}.\textsc{rel}-see-\textsc{pass}-\textsc{fv}\\
\glt ‘These are the things that can be experienced.’ (NF\_Song17)
\z

Unlike other derivational suffixes with /i/, the passive suffix does not undergo vowel harmony: its vowel is always realized as /i/ and never as /e/ (see \sectref{bkm:Ref451863900} on vowel harmony). The passive suffix can be realized as \textit{-w} instead of \textit{-iw} in certain cases. In Zambian Fwe, the passive is realized as -\textit{w} when preceded by another derivational suffix, as in (\ref{bkm:Ref463269273}), where the passive -\textit{w} is preceded by the separative suffix \textit{-or}. When not preceded by another derivational suffix, the passive is always realized as \textit{-iw}, as in (\ref{bkm:Ref463269276}).

\ea
\label{bkm:Ref463269273}
\glll kùkòndòrwà\\
ku-kond-or-w-a\\
\textsc{inf}-brew\_beer-\textsc{sep}.\textsc{tr}-\textsc{pass}-\textsc{fv}\\
\glt ‘to be brewed (beer)’ (ZF)
\z

\ea
\label{bkm:Ref463269276}
\glll kùtémìwà\\
ku-tém-iw-a\\
\textsc{inf}-chop-\textsc{pass}-\textsc{fv}\\
\glt ‘to be chopped’ (ZF)
\z

In Namibian Fwe, the two forms of the passive suffix are in free variation: both derived and underived verbs can take the suffix \textit{-iw} or \textit{-w}, as in (\ref{bkm:Ref99025633}--\ref{bkm:Ref99025635}).

\ea
\label{bkm:Ref99025633}
kùréːtìwà {\textasciitilde} kùrêːtwà\\
ku-réːt-w-a\\
\textsc{inf}-give\_birth-\textsc{pass}-\textsc{fv}\\
\glt ‘to be born’ (NF)
\z

\ea
cìhìkwâ {\textasciitilde} cìhìkìwâ\\
ci-hi\textsubscript{H}k-w-á̲\\
\textsc{sm}\textsubscript{7}-cook-\textsc{pass}-\textsc{fv}\\
\glt ‘It can be cooked.’ (NF\_Elic15)
\z

\ea
\label{bkm:Ref99025635}
kùnànùnwà {\textasciitilde} kùnànùnìwà\\
ku-nan-un-w-a\\
\textsc{inf}-lift-\textsc{sep}.\textsc{tr}-\textsc{pass}-\textsc{fv}\\
\glt ‘to be lifted’ (NF)
\z

With monosyllabic verb roots, the passive suffix is always realized as \textit{-iw}, e.g. the vowel \textit{i} can never be dropped. When the monosyllabic verb root ends in the vowel /a/, vowel coalescence between the low vowel /a/ of the root and the high front vowel /i/ of the suffix results in a mid front vowel /e/, as in (\ref{bkm:Ref99025754}--\ref{bkm:Ref99025758}).

\ea
\label{bkm:Ref99025754}
\glll kùtêwà\\
ku-tá-iw-a\\
\textsc{inf}-tell-\textsc{pass}-\textsc{fv}\\
\glt ‘to be told’
\z

\ea
\label{bkm:Ref99025758}
\glll kùhêwà\\
ku-há-iw-a\\
\textsc{inf}-give-\textsc{pass}-\textsc{fv}\\
\glt ‘to be given’
\z

When combined with the stative suffix \textit{-ite}, the passive becomes \textit{-itwe/-etwe} in Zambian Fwe, as in (\ref{bkm:Ref99025779}), or \textit{-itwa/-etwa} in Namibian Fwe, as in (\ref{bkm:Ref99025793}). (See also \sectref{bkm:Ref431984198} on the stative.)

\ea
\label{bkm:Ref99025779}
\glll ndìshéshêtwè\\
ndi-she\textsubscript{H}sh-é̲twe\\
\textsc{sm}\textsubscript{1SG}-marry-\textsc{stat}.\textsc{pass}\\
\glt ‘I am married (said by a woman).’ (ZF\_Elic14)
\z

\ea
\label{bkm:Ref99025793}
\glll ndìkòmókètwà\\
ndi-komó̲k-etwa\\
\textsc{sm}\textsubscript{1SG}-be\_surprised-\textsc{stat}.\textsc{pass}\\
\glt ‘I am surprised.’ (NF\_Elic15)
\z

The passive decreases the valency of the verb, by expressing the patient in the subject position and leaving the agent unexpressed. Compare the active sentence in (\ref{bkm:Ref415152508}), where the patient of \textit{ndìùrìsá} ‘I sell’ is \textit{njûò}, ‘the house’, with its passive version in (\ref{bkm:Ref74913719}), where \textit{njúò} ‘the house’ has been promoted to subject position, and the first person singular agent, marked in the active version through agreement on the verb, is left unexpressed.

\ea
\label{bkm:Ref415152508}
ndìùrìsá njûò\\
\gll ndi-ur-is-á̲    N-júo\\
\textsc{sm}\textsubscript{1SG}-buy-\textsc{caus}-\textsc{fv}  \textsc{np}\textsubscript{9}-house\\
\glt ‘I sell the house.’
\z

\ea
\label{bkm:Ref74913719}
ènjúò ìhùrìsìwâ\\
\gll e-N-júo    i-ur-is-iw-á̲\\
\textsc{aug}-\textsc{np}\textsubscript{9}-house  \textsc{sm}\textsubscript{9}-buy-\textsc{caus}-\textsc{pass}-\textsc{fv}\\
\glt ‘The house is being sold.’ (ZF\_Elic13)
\z

As the passive decreases the valency of the verb, the use of the passive with a transitive verb, such as \textit{kwâtà} ‘grab’ in (\ref{bkm:Ref445307129}), results in an intransitive verb, as in (\ref{bkm:Ref75163980}).

\ea
\label{bkm:Ref445307129}
\glll ndàmùkwâtì\\
nd-a-mu-kwát-i\\
\textsc{sm}\textsubscript{1SG}-\textsc{pst}-\textsc{om}\textsubscript{1}-grab-\textsc{npst}.\textsc{pfv}\\
\glt ‘I caught her/him.’
\z

\ea
\label{bkm:Ref75163980}
\glll òkwàtìwâ\\
o-kwa\textsubscript{H}t-iw-á̲\\
\textsc{sm}\textsubscript{2SG}-grab-\textsc{pass}-\textsc{fv}\\
\glt ‘You’d be caught.’ (NF\_Elic15)
\z

When used with intransitive verbs, the passive decreases the valency of the verb to zero to create an impersonal passive. An impersonal passive takes a locative grammatical subject, which has the semantic function of location. The locative subject may be expressed (pro)nominally, as in (\ref{bkm:Ref71206617}--\ref{bkm:Ref71206618}), or only through subject marking on the verb, as in (\ref{bkm:Ref71206553}--\ref{bkm:Ref71206554}).

\ea
\label{bkm:Ref71206617}
hàmùkítí hàzànìwâ\\
\gll ha-mu-kití    ha-zan-iw-á\\
\textsc{np}\textsubscript{16}-\textsc{np}\textsubscript{3}-party  \textsc{sm}\textsubscript{16}-dance-\textsc{pass}-\textsc{fv}\\
\glt ‘Dancing may take place at the party.’
\z

\ea
\label{bkm:Ref71206618}
kwìná kùkwèsì kùtàkùmìwâ\\
\gll kwiná    ku-kwesi  ku-takum-iw-á\textsubscript{H}\\
\textsc{dem}.\textsc{iv}\textsubscript{17}  \textsc{sm}\textsubscript{17}-\textsc{prog}  \textsc{sm}\textsubscript{17}-shout-\textsc{pass}-\textsc{fv}\\
\glt ‘Shouting is taking place there.’ (NF\_Elic17)
\z

\ea
\label{bkm:Ref71206553}
kùkwèsì kùshìbìwâ\\
\gll ku-kwesi  ku-shi\textsubscript{H}b-iw-á̲\\
\textsc{sm}\textsubscript{17}-\textsc{prog}  \textsc{sm}\textsubscript{17}-whistle-\textsc{pass}-\textsc{fv}\\
\glt ‘There is whistling there.’
\z

\ea
\label{bkm:Ref71206554}
\glll kàmùrídàmînwà\\
ka-mu-rí-dam-í̲n-w-a\\
\textsc{neg}-\textsc{sm}\textsubscript{18}-\textsc{refl}-beat-\textsc{appl}-\textsc{fv}\\
\glt ‘Beating each other is not allowed in here.’ (NF\_Elic17)
\z

The use of the passive removes the agent as a core argument, but the agent can still be expressed as a peripheral participant by use of the class 17 nominal prefix \textit{ku-}, as in (\ref{bkm:Ref71206307}--\ref{bkm:Ref71206308}). If the agent marked with \textit{ku-} is a first or second person, the possessive stem is used, as shown with the first person singular possessive \textit{kwángù} in (\ref{bkm:Ref71206356}).

\ea
\label{bkm:Ref71206307}
nàdámwà \textbf{kúbàntù} \textbf{bângîː}\\
\gll na-dam-w-á̲      kú-ba-ntu    bá-ngíː\\
\textsc{sm}\textsubscript{1}.\textsc{pst}-beat-\textsc{pass}-\textsc{fv}  \textsc{np}\textsubscript{17}-\textsc{np}\textsubscript{2}-person  \textsc{pp}\textsubscript{2}-many\\
\glt ‘S/he was beaten \textbf{by} \textbf{many} \textbf{people}.’ (NF\_Elic17)
\z

\ea
\label{bkm:Ref71206308}
mùnàkó ímwìnyà ònkômbwè nàtéwà \textbf{kùzìzyùnì} \textbf{zòbírè} kùtè\\
\gll mu-N-nakó    í-mwinya  o-∅-nkómbwe  na-tá-iw-a  ku-zi-zyuni    zi-o=bíre  kute\\
\textsc{np}\textsubscript{18}-\textsc{np}\textsubscript{9}-time  \textsc{pp}\textsubscript{9}-certain  \textsc{aug}-\textsc{np}\textsubscript{1a}-tortoise  \textsc{sm}\textsubscript{1}-\textsc{pst}-say-\textsc{pst}.\textsc{pass}-\textsc{fv}
\textsc{np}\textsubscript{17}-\textsc{np}\textsubscript{8}-bird  \textsc{pp}\textsubscript{8}-\textsc{con}=two  that\\
\glt ‘Once upon a time, a tortoise was told \textbf{by} \textbf{two} \textbf{eagles} that…’ (ZF\_Narr13)
\z

\ea
\label{bkm:Ref71206356}
sìmátá nàdámíwà \textbf{kwángù}\\
\gll simatá    na-dam-í̲w-a      kw-angú\\
Simata  \textsc{sm}\textsubscript{1}.\textsc{pst}-beat-\textsc{pass}-\textsc{fv}  \textsc{np}\textsubscript{17}-\textsc{poss}\textsubscript{1SG}\\
\glt ‘Simata was beaten \textbf{by} \textbf{me.}’ (NF\_Elic17)
\z

The agent noun may also be used without the prefix \textit{ku-}: both possibilities are illustrated in (\ref{bkm:Ref494444592}--\ref{bkm:Ref74913797}).

\ea
\label{bkm:Ref494444592}
Sìmátá nàshúmìwà \textbf{kúmbwà}\\
\gll simatá    na-shúm-iw-a    ku-∅-mbwá\\
Simata  \textsc{sm}\textsubscript{1}.\textsc{pst}-bite-\textsc{pass}-\textsc{fv}  \textsc{np}\textsubscript{17}-\textsc{np}\textsubscript{1a}-dog\\
\glt ‘Simata was bitten by a dog.’
\z

\ea
\label{bkm:Ref74913797}
Sìmátá nàshúmìwà \textbf{ómbwà}\\
\gll simatá    na-shúm-iw-a    o-∅-mbwá\\
Simata  \textsc{sm}\textsubscript{1}.\textsc{pst}-bite-\textsc{pass}-\textsc{fv}  \textsc{aug}-\textsc{np}\textsubscript{1a}-dog\\
\glt ‘Simata was bitten by a dog.’ (NF\_Elic17)
\z

The prefix \textit{ku-} is obligatory when the agent noun is a proper name, as in (\ref{bkm:Ref485731762}), or when the agent is in focus, as in (\ref{bkm:Ref485731765}).

\ea
\label{bkm:Ref485731762}
\ea
sìmátá nàdàmíwà \textbf{kùbányàmbè}\\
\gll simatá    na-dam-í̲w-a      ku-bá-nyambe\\
Simata  \textsc{sm}\textsubscript{1}.\textsc{pst}-beat-\textsc{pass}-\textsc{fv}  \textsc{np}\textsubscript{17}-\textsc{np}\textsubscript{2}-Nyambe\\
\glt ‘Simata was beaten by Mr. Nyambe.’

\ex
*sìmátá nàdàmíwà \textbf{bányàmbè}
\z\z

\ea
\label{bkm:Ref485731765}
\ea
kùnjí nàshúmìwà sìmátà\\
\gll ku-njí    na-shúm-iw-a    simatá\\
\textsc{np}\textsubscript{17}-what  \textsc{sm}\textsubscript{1}.\textsc{pst}-bite-\textsc{pass}-\textsc{fv}  Simata\\
\glt ‘Who was Simata bitten by?’

\ex
nàshúmìwà \textbf{kúmbwà}\\
\gll na-shúm-iw-a    ku-∅-mbwá\\
\textsc{sm}\textsubscript{1}.\textsc{pst}-bite-\textsc{pass}-\textsc{fv}  \textsc{np}\textsubscript{17}-\textsc{np}\textsubscript{1a}-dog\\
\glt ‘He was bitten by a dog.’

\ex
  *nàshúmìwà \textbf{ómbwà} (NF\_Elic17)
\z\z

The agent-marking function of the class 17 prefix \textit{ku-} is not restricted to verbs overtly marked with a passive, but can occur in any construction where the agent cannot be expressed as a core argument (see \sectref{bkm:Ref452049189} on locative noun classes).

Verbs derived with a passive suffix display behavior that is typical for change-of-state verbs: they have a conditional/modal reading in the present construction, and do not allow a present continuous interpretation, as in (\ref{bkm:Ref445823991}), but a present stative reading when combined with the stative inflection, as in (\ref{bkm:Ref451331093}). (For more on the interpretation of the present inflection in relation to lexical aspect, see \sectref{bkm:Ref72233436}.)

\ea
\label{bkm:Ref445823991}
mwínì ùkwàtìwâ\\
\gll mu-íni  u-kwa\textsubscript{H}t-iw-á̲\\
\textsc{np}\textsubscript{3}-handle  \textsc{sm}\textsubscript{3}-grab-\textsc{pass}-\textsc{fv}\\
\glt ‘The handle can be touched.’ (*The handle is being touched.)
\z

\ea
\label{bkm:Ref451331093}
évú rìvwìkítwà kúmàbùnà\\
\gll e-vú    ri-vwik-í̲twa      kú-ma-buna\\
\textsc{aug}-ground  \textsc{sm}\textsubscript{5}-cover-\textsc{stat}.\textsc{pass}  \textsc{np}\textsubscript{17}-\textsc{np}\textsubscript{6}-leaf\\
\glt ‘The ground is covered with leaves.’ (NF\_Elic15)
\z
\section{Causative}
\label{bkm:Ref451514620}\hypertarget{Toc75352659}{}
The causative in Fwe has a productive long form with a suffix \textit{-is/-es}, and a less productive short form, which consists of a change of the last stem consonant to /s/ or /z/.

The productive causative suffix \textit{-is/-es} undergoes vowel height harmony with the stem (see \sectref{bkm:Ref451863900}). Examples of verbs with a long causative are given in \tabref{tab:6:1}.

\begin{table}
\label{bkm:Ref463339933}\caption{\label{tab:6:1}Verbs taking the long causative}
\begin{tabularx}{\textwidth}{XlXl}
\lsptoprule
\textit{bìrà} & ‘boil (intr.)’ & \textit{bìrìsà} & ‘boil (tr.), bring to a boil’\\
\textit{shèkà} & ‘laugh’ & \textit{shèkèsà} & ‘make (someone) laugh’\\
\textit{tàbà} & ‘become happy’ & \textit{tàbìsà} & ‘make happy’\\
\textit{bòmbà} & ‘become wet’ & \textit{bòmbèsà} & ‘make wet’\\
\textit{zyûmà} & ‘become dry’ & \textit{zyúmìsà} & ‘dry, make (something) dry’\\
\lspbottomrule
\end{tabularx}
\end{table}

The less productive short form of the causative suffix consists of the change of the last stem consonant to /s/ in the case of a voiceless consonant, or to /z/ in the case of a voiced consonant. This goes back to the causative derivation reconstructed for Proto-Bantu as *-i. The reconstructed high vowel caused spirantization of the preceding consonant, a diachronic sound change that changed stops into fricatives before high vowels (see \citealt{Bostoen2009} for an account of spirantization in Fwe). This resulted in the causative forms with /s/ and /z/ seen in Fwe today. This is illustrated in (\ref{bkm:Ref70067629}) with the verb \textit{donk} ‘drip’, which takes a short causative \textit{dons} ‘cause to drip’.

\ea
\label{bkm:Ref70067629}
\ea
Simple verb stem\\
kùdònkà\\
ku-donk-a\\
\textsc{inf}-drip-\textsc{fv}\\
\glt ‘to drip (intr.)’

\ex
\label{bkm:Ref70067630}
  Historical derivation of short causative\\
-donk - + *i > -dons-

\ex
\glll kùdònsà\\
ku-dons-a\\
\textsc{inf}-drip.\textsc{caus}-\textsc{fv}\\
\glt ‘to drip (tr.), to cause to drip’
\z\z

The short and the long causative in Fwe have the same function. The short form is used with a specific set of lexical verbs and with specific derivational suffixes. The long causative is used in all other cases, and many verbs that may take the short causative are also attested with the long causative. Lexical verbs that may take the short causative are listed in \tabref{tab:6:2}, including verbs that may take either the long or the short causative. In most cases, there is no semantic difference between the short and the long causative, with the exception of \textit{bûːkà} ‘wake up; consult spirits’ (see the first line of \tabref{tab:6:2}).

\begin{table}
\label{bkm:Ref462832163}\caption{\label{tab:6:2}Verbs that (may) take the short causative}
\begin{tabularx}{\textwidth}{lQlQ}
\lsptoprule
\multicolumn{2}{l}{Underived verb} & \multicolumn{2}{l}{Causative verb}\\
\midrule
\textit{bûːkà} & ‘wake up (intr.); consult spirits (as a witch doctor)’ & \textit{bûːsa} & ‘greet, wake up (tr.)’\\
&  & \textit{búkìsà} & ‘ask a witch doctor to consult spirits’\\
\textit{dònkà} & ‘drip (intr.)’ & \textit{dònsà} {\textasciitilde} \textit{dònkèsà} & ‘cause to drip’\\
\textit{fwìnkà} & ‘become sealed’ & \textit{fwìnsà} {\textasciitilde} \textit{fwìnkìsà} & ‘seal’\\
\textit{ᵍǀôntà} & ‘drip’ & \textit{ᵍǀônsà} & ‘cause to drip’\\
\textit{kwâtà} & ‘hold, grab’ & \textit{kwâsà} & ‘help’\\
\textit{nùnkà} & ‘smell (intr.)’ & \textit{nùnsà} & ‘make (someone) smell (something); imagine to smell (something)’\\
\textit{nyônkà} & ‘breastfeed (intr.)’ & \textit{nyônsà} {\textasciitilde} \textit{nyónkèsà} & ‘breastfeed (tr.)’\\
\textit{rûkà} & ‘vomit’ & \textit{rûsà} {\textasciitilde} \textit{rúkìsà} & ‘hold someone who is vomiting’\\
\textit{sûkà} & ‘disembark’ & \textit{sûsà} & ‘put down (when carrying)’\\
\textit{tùkùtà} & ‘be warm’ & \textit{tùkùsà} {\textasciitilde} \textit{tùkùtìsà} & ‘warm (something) up’\\
\textit{zwâtà} & ‘get dressed’ & \textit{zwâsà} & ‘dress (someone)’\\
\textit{bòòrà} & ‘come back’ & \textit{bòòzà} & ‘bring back’\\
\textit{hùrà} & ‘arrive’ & \textit{hùzà} & ‘cause to arrive’\\
\textit{hârà} & ‘live’ & \textit{hâzà} & ‘save’\\
\textit{kàbìrà} & ‘enter’ & \textit{kàbìzà} {\textasciitilde} \textit{kàbìrìsà} & ‘bring into’\\
\textit{nyèèrwà}\footnotemark{} & ‘become annoyed’ & \textit{nyèèzà} & ‘annoy (someone)’\\
\lspbottomrule
\end{tabularx}
\end{table}
\footnotetext{This verb appears to contain a passive suffix \textit{-w}, suggesting an original verb root \textit{nyeer}, but such a root is currently not attested.}

Some of the underived verbs in \tabref{tab:6:2} are historically bimorphemic. For instance, the verb \textit{búːk} ‘wake up’ appears to consist of the root \textit{bú} with the separative suffix \textit{\nobreakdash-uk} (see \sectref{bkm:Ref485823385}), which also explains the occurrence of the long vowel /uː/; and \textit{zwâtà} ‘get dressed’ appears to consist of the root \textit{zú} and the tentive suffix \textit{-at} (see \sectref{bkm:Ref485997273}).

The short causative is also used with certain derivational suffixes. Verbs with a separative suffix \textit{-ur/-uk} that may take the short causative are listed in \tabref{tab:6:3}; some of these may either take the short or the long causative. All other separative verbs only take the long causative.

\begin{table}
\label{bkm:Ref488847470}\caption{\label{tab:6:3}Separative verbs that (may) take the short causative}
\begin{tabularx}{\textwidth}{lQlQ}
\lsptoprule
\multicolumn{2}{l}{Separative verb} & \multicolumn{2}{l}{Separative verb with causative}\\
\midrule
\textit{fùndùkà} & ‘leave’ & \textit{fùndùsà} & ‘escort (someone who is leaving)’\\
\textit{kàntùkà} & ‘cross a river’ & \textit{kàntùsà} {\textasciitilde} \textit{kàntùkìsà} & ‘help (someone) cross a river’\\
\textit{ŋàtùrà} & ‘tear; come up (of the sun)’ & \textit{ŋàtùzà} & ‘stay up till sunrise’\\
\textit{ùrùkà} & ‘fly away’ & \textit{ùrùsà} & ‘blow away’\\
\textit{túmbùkà} & ‘burn (intr.)’ & \textit{túmbùsà} & ‘burn (tr.)’\\
\textit{zímbùkà} & ‘go around’ & \textit{zímbùsà} & ‘bring around’\\
\textit{zímbùrùkà} & ‘cross the border illegally, circumvent; spin (intr.)’ & \textit{zímbùrùsà} & ‘smuggle (tr.); spin (tr.)’\\
\lspbottomrule
\end{tabularx}
\end{table}

Short causatives are also used with other, unproductive derivational affixes, namely the neuter \textit{\nobreakdash-ahar}, as in (\ref{bkm:Ref99026161}), and the extensive suffix \textit{-ar}, as in (\ref{bkm:Ref99026173}).

\ea
\label{bkm:Ref99026161}
\ea
\glll kùbónàhàrà\\
ku-bón-ahar-a\\
\textsc{inf}-see-\textsc{neut}-\textsc{fv}\\
\glt ‘to be visible’

\ex
 cf. kùbónàhàzà\\
ku-bón-ahaz-a\\
\textsc{inf}-see-\textsc{neut}.\textsc{caus}-\textsc{fv}\\
\glt ‘to make visible’
\z\z

\ea
\label{bkm:Ref99026173}
\ea
\glll kùsúmbàzà \\
ku-súmb-az-a\\
\textsc{inf}-become\_pregnant-\textsc{ext}.\textsc{caus}-\textsc{fv}\\
\glt ‘to impregnate’

\ex
 cf. kùsúmbàrà\\
ku-súmb-ar-a\\
\textsc{inf}-become\_pregnant-\textsc{ext}-\textsc{fv}\\
\glt ‘to become pregnant’
\z\z

The intensive, which consists of the reduplicated applicative suffix (see \sectref{bkm:Ref485997587}), invariably takes the short causative, as in (\ref{bkm:Ref99026458}).

\ea
\label{bkm:Ref99026458}
\ea
\glll kùtúmìnìzà\\
ku-túm-iniz-a\\
\textsc{inf}-send-\textsc{int}.\textsc{caus}-\textsc{fv}\\
\glt ‘to send (someone) incessantly’

\ex
 cf. kùtúmìnìnà\\
ku-túm-inin-a\\
\textsc{inf}-send-\textsc{int}-\textsc{fv}\\
\glt ‘to send incessantly’
\z\z

Other derivational suffixes, namely the impositive and reciprocal, only take the long causative. The passive suffix, when it combines with the causative, does not influence the form of the causative suffix, as the passive always follows rather than precedes the causative (see also \sectref{bkm:Ref452972446}). The conditioning of the long and short causative forms is summarized in (\ref{bkm:Ref494444716}).

\ea
\label{bkm:Ref494444716}
Short causative: lexical exceptions, separative, neuter, extensive\\
Long causative: all remaining lexemes, impositive, reciprocal\\
\z

The causative derivation is highly productive; this derivation may combine with any verb, and its semantics are highly predictable. There are also a few lexicalized causatives, verbs with a causative suffix where the corresponding underived verb is not attested. Lexicalized causatives are seen with the long causative, such as the verbs \textit{mwénges} ‘greet’, and \textit{cáis} ‘collide, knock off’, and also with the short causative, such as the verbs \textit{nyens} ‘defeat’, and \textit{suns} ‘dip (porridge in relish)’. Lexicalized causatives are rare, though, and in most cases the causative derivation is used productively.

The causative increases the valency of the verb by adding an agent participant. For example, the intransitive verb \textit{túmbuk} ‘burn’ takes a single argument \textit{òmùrìrò} ‘fire’ expressed as a subject, as in (\ref{bkm:Ref488850724}). When derived with a causative in (\ref{bkm:Ref488850725}), the subject is demoted to object, and the newly added agent ‘I’ is expressed as a subject.

\ea
\label{bkm:Ref488850724}
òmùrìrò ùtùmbúkà\\
\gll o-mu-riro  u-tu\textsubscript{H}mbuk-á̲\\
\textsc{aug}-\textsc{np}\textsubscript{3}-fire  \textsc{sm}\textsubscript{3}-burn-\textsc{fv}\\
\glt ‘The fire burns.’
\z

\ea
\label{bkm:Ref488850725}
ndìtùmbùs’ ómùrìrò\\
\gll ndi-tu\textsubscript{H}mbus-á̲    o-mu-riro\\
\textsc{sm}\textsubscript{1SG}-burn.\textsc{caus}-\textsc{fv}  \textsc{aug}-\textsc{np}\textsubscript{3}-fire\\
\glt ‘I light the fire.’ (NF\_Elic15)
\z

With an intransitive verb, the causative derives a transitive verb, as in (\ref{bkm:Ref488850725}). With a transitive verb, such as \textit{rí} ‘eat’, the causative derives a ditransitive verb \textit{rí-is} ‘feed’, as in (\ref{bkm:Ref445824605}), where \textit{rí-is} ‘feed’ is used with two objects, a causer object, the child, and a causee object, the porridge.

\ea
\label{bkm:Ref445824605}
ndìrìs’ óꜝmwáncè nkôkò\\
\gll ndi-ri\textsubscript{H}-is-á̲    o-mu-ánce    N-kóko\\
\textsc{sm}\textsubscript{1SG}-eat-\textsc{caus}-\textsc{fv}  \textsc{aug}-\textsc{np}\textsubscript{1}-child  \textsc{np}\textsubscript{9}-porridge\\
\glt ‘I feed the child porridge.’ (NF\_Elic17)
\z

When a causative verb has two objects, both objects display the same syntactic behavior. The order of the objects is free, as in (\ref{bkm:Ref98833421}--\ref{bkm:Ref98833423}).

\ea
\label{bkm:Ref98833421}
ndàtésì òmúkwàmé òbùsâ\\
\gll ndi-a-tá-is-i        o-mú-kwamé  o-bu-sá\\
\textsc{sm}\textsubscript{1SG}-\textsc{pst}-say-\textsc{caus}-\textsc{npst}.\textsc{pfv}  \textsc{aug}-\textsc{np}\textsubscript{1}-man  \textsc{aug}-\textsc{np}\textsubscript{14}-thief\\
\glt ‘I accused the man of theft.’
\z

\ea
\label{bkm:Ref98833423}
ndàtésì òbùsá múꜝkwámè\\
\gll ndi-a-tá-is-i        o-bu-sá    o-mú-kwamé\\
\textsc{sm}\textsubscript{1SG}-\textsc{pst}-say-\textsc{caus}-\textsc{npst}.\textsc{pfv}  \textsc{aug}-\textsc{np}\textsubscript{14}-thief  \textsc{aug}-\textsc{np}\textsubscript{1}-man\\
\glt ‘I accused the man of theft.’ (NF\_Elic17)
\z

This is also the case when the causative introduces an instrumental object: as shown in (\ref{bkm:Ref75168782}--\ref{bkm:Ref75168784}), the instrument \textit{àkàfùrò} ‘knife’ introduced by the causative can appear before or after the patient \textit{ènyàmà} ‘meat’.

\ea
\label{bkm:Ref75168782}
ndìfùndìsá ènyàmà àkàfùrò\\
\gll ndi-fund-is-á̲  e-N-nyama    a-ka-furo\\
\textsc{sm}\textsubscript{1SG}-cut-\textsc{caus}-\textsc{fv}  \textsc{aug}-\textsc{np}\textsubscript{9}-meat  \textsc{aug}-\textsc{np}\textsubscript{12}-knife\\
\glt ‘I cut the meat with a knife.’
\z

\ea
\label{bkm:Ref75168784}
ndìfùndìsá àkàfùrò ènyàmà\\
\gll ndi-fund-is-á̲  a-ka-furo    e-N-nyama\\
\textsc{sm}\textsubscript{1SG}-cut-\textsc{caus}-\textsc{fv}  \textsc{aug}-\textsc{np}\textsubscript{12}-knife  \textsc{aug}-\textsc{np}\textsubscript{9}-meat\\
\glt ‘I cut the meat with a knife.’ (NF\_Elic17)
\z

Both objects of the causative verb may be pronominalized, as shown with the causative verb \textit{rí-is} ‘feed’: both objects can be expressed nominally, as in (\ref{bkm:Ref492044828}), or the causer can be pronominalized, as in (\ref{bkm:Ref75164990}), or the causee can be pronominalized, as in (\ref{bkm:Ref75164992}). It is also possible for both objects of a causative verb to be pronominalized, as in (\ref{bkm:Ref488916066}).

\ea
\label{bkm:Ref492044828}
ndìrìs’ óꜝmwáncè nkôkò\\
\gll ndi-ri\textsubscript{H}-is-á̲    o-mu-ánce    N-kóko\\
\textsc{sm}\textsubscript{1SG}-eat-\textsc{caus}-\textsc{fv}  \textsc{aug}-\textsc{np}\textsubscript{1}-child  \textsc{np}\-\textsubscript{9}-porridge\\
\glt ‘I feed the child porridge.’
\z

\ea
\label{bkm:Ref75164990}
ndìmùrìs’ énkôkò\\
\gll ndi-mu-ri\textsubscript{H}-is-á̲    e-N-kóko\\
\textsc{sm}\textsubscript{1SG}-\textsc{om}\textsubscript{1}-eat-\textsc{caus}-\textsc{fv}  \textsc{aug}-\textsc{np}\textsubscript{9}-porridge\\
\glt ‘I feed her/him porridge.’
\z

\ea
\label{bkm:Ref75164992}
ndàyírìsì mwâncè\\
\gll ndi-a-í-ri-is-i        mu-ánce\\
\textsc{sm}\textsubscript{1SG}-\textsc{pst}-\textsc{om}\textsubscript{9}-eat-\textsc{caus}-\textsc{npst}.\textsc{pfv}  \textsc{np}\textsubscript{1}-child\\
\glt ‘I fed it to the child.’ (NF\_Elic17)
\z

\ea
\label{bkm:Ref488916066}
\glll ndàbúmùtêsì\\
ndi-a-bú-mu-tá-is-i\\
\textsc{sm}\textsubscript{1SG}-\textsc{pst}-\textsc{om}\textsubscript{14}-\textsc{om}\textsubscript{1}-say-\textsc{caus}-\textsc{npst}.\textsc{pfv}\\
\glt ‘I accused her/him of it.’ (NF\_Elic17)
\z

Instrumental causatives also allow the pronominalization of either object, as in (\ref{bkm:Ref75169043}--\ref{bkm:Ref75169044}), but, as (\ref{bkm:Ref75169061}) shows, not both. This is not necessarily a property of the instrumental causative, however, but due to a wider generalization in Fwe that when multiple object markers are used, only one can have an inanimate referent (see \sectref{bkm:Ref75169159}).

\ea
ndìsùmìs’ éndòngà cìzyàbàrò\\
\gll ndi-su\textsubscript{H}m-is-á̲  e-N-donga    ci-zyabaro\\
\textsc{sm}\textsubscript{1SG}-sew-\textsc{caus}-\textsc{fv}  \textsc{aug}-\textsc{np}\textsubscript{9}-needle  \textsc{np}\textsubscript{7}-shirt\\
\glt ‘I sew the shirt with a needle.’
\z

\ea
\label{bkm:Ref75169043}
ndàcísùmìsì ndòngà\\
\gll ndi-a-cí-sum-is-i        N-donga\\
\textsc{sm}\textsubscript{1SG}-\textsc{pst}-\textsc{om}\textsubscript{7}-sew-\textsc{caus}-\textsc{npst}.\textsc{pfv}  \textsc{np}\textsubscript{9}-needle\\
\glt ‘I’ve sewn it with a needle.’
\z

\ea
\label{bkm:Ref75169044}
ndàyísùmìsì cìzyàbàrò\\
\gll ndi-a-í-sum-is-i        ci-zyabaro\\
\textsc{sm}\textsubscript{1SG}-\textsc{pst}-\textsc{om}\textsubscript{9}-sew-\textsc{caus}-\textsc{npst}.\textsc{pfv}  \textsc{np}\textsubscript{7}-shirt\\
\glt ‘I’ve sewn the shirt with it.’ (NF\_Elic17)
\z

\ea
\label{bkm:Ref75169061}
*ndàyícìsùmìsì\\
ndi-a-í-ci-sum-is-i\\
\textsc{sm}\textsubscript{1SG}-\textsc{pst}-\textsc{om}\textsubscript{9}-\textsc{om}\textsubscript{7}-sew-\textsc{caus}-\textsc{npst}.\textsc{pfv}\\
Intended: ‘I sew it with it.’ (NF\_Elic17)
\z

The causative in Fwe can be used to express different types of causation, which form part of a “causative continuum” (\citealt{ShibataniPardeshi2001}), ranging from direct causation to indirect causation through a number of different, intermediate causation types. Direct causation involves the direct, physical manipulation of the causee by the causer. Only the causer is an agent, and the action performed by the causer and that performed by the causee are (almost) simultaneous. This use of the causative in Fwe is shown in (\ref{bkm:Ref488851568}), which uses a causative verb \textit{cènèsà} to express that the agent ‘I’, causes the patient (the house) to become clean by physically cleaning it.

\ea
\label{bkm:Ref488851568}
ndìcènèsá ènjûò\\
\gll ndi-cen-es-á̲        e-N-júo\\
\textsc{sm}\textsubscript{1SG}-become\_clean-\textsc{caus}-\textsc{fv}  \textsc{aug}-\textsc{np}\textsubscript{9}-house\\
\glt ‘I clean the house.’ (NF\_Elic15)
\z

Moving along the causative continuum, direct causation is bordered by sociative causation, where the causer agent does not cause the causee patient to perform the action, but rather assists the patient in performing the action, by performing the action with her, for instance (\citealt{ShibataniPardeshi2001}). Sociative causation is similar to direct causation, because there is a spatio-temporal overlap between the action of the causer and the action of the causee, but differs from direct causation in that the causee is also an agentive, active participant in the action. The use of the causative for sociative causation in Fwe is illustrated in (\ref{bkm:Ref98512745}--\ref{bkm:Ref98512746}).

\ea
\label{bkm:Ref98512745}
kàntí ndìkùtòmbwérìsè\\
\gll kantí  ndi-ku-tombwé̲r-is-e\\
then  \textsc{sm}\textsubscript{1SG}-\textsc{om}\textsubscript{2SG}-weed-\textsc{caus}-\textsc{pfv}.\textsc{sbjv}\\
\glt ‘Let me help you weeding (by weeding with you).’ (NF\_Narr15)
\z

\ea
àkwèsì àndìàmbìsâ\\
\gll a-kwesi  a-ndi-amb-is-á̲\\
\textsc{sm}\textsubscript{1}-have  \textsc{sm}\textsubscript{1}-\textsc{om}\textsubscript{1SG}-talk-\textsc{caus}-\textsc{fv}\\
\glt ‘S/he is talking to/with me.’ (NF\_Elic15)
\z

\ea
\label{bkm:Ref98512746}
bàkwèsì bàndìzyàmbìrìsâ\\
\gll ba-kwesi  ba-ndi-zyambir-is-á̲\\
\textsc{sm}\textsubscript{2}-\textsc{prog}  \textsc{sm}\textsubscript{2}-\textsc{om}\textsubscript{1SG}-gather-\textsc{caus}-\textsc{fv}\\
\glt ‘They are helping me gather.’ (Explanation: we are all gathering, but the results will go to me.) (NF\_Elic17)
\z

The sociative use of the causative may also refer to keeping someone company, rather than actively helping them perform a certain action, as in (\ref{bkm:Ref99027235}--\ref{bkm:Ref99027240}).

\ea
\label{bkm:Ref99027235}
òyéndè òkàmúkàrìsè\\
\gll o-é̲nd-e    o-ka-mú-kar-is-e\\
\textsc{sm}\textsubscript{2SG}-go-\textsc{pfv}.\textsc{sbjv}  \textsc{sm}\textsubscript{2SG}-\textsc{dist}-\textsc{om}\textsubscript{1}-sit-\textsc{caus}-\textsc{pfv}.\textsc{sbjv}\\
\glt ‘Go and sit with him/keep her/him company.’ (NF\_Elic17)
\z

\ea
\label{bkm:Ref99027240}
mùbàfúndúsè bàêndè\\
\gll mu-ba\textsubscript{H}-fund-ú̲s-e        ba-é̲nd-e\\
\textsc{sm}\textsubscript{2PL}-\textsc{om}\textsubscript{2}-leave-\textsc{sep}.\textsc{caus}-\textsc{pfv}.\textsc{sbjv}  \textsc{sm}\textsubscript{2}-go-\textsc{pfv}.\textsc{sbjv}\\
\glt ‘Escort her/him as/so that s/he goes.’ (NF\_Elic15)
\z

It is also possible for the causative to express that the causer is present, but does not perform the same action as the causee, e.g. “assistive causative” (\citealt{ShibataniPardeshi2001}: 100), as in (\ref{bkm:Ref99027269}).

\ea
\label{bkm:Ref99027269}
kùrúkìsà {\textasciitilde} kùrûsa\\
ku-rúk-is-a\\
\textsc{inf}-vomit-\textsc{caus}-\textsc{fv}\\
\glt ‘to hold someone who is vomiting’
\z

The other end of the causative continuum is represented by indirect causation, where the causer and the causee are both agentive participants, and there is no spatio-temporal overlap between the actions that they perform. Rather, the causer may act upon the causee by verbal command, or through some other, indirect means. In Fwe, indirect causation is mostly expressed through periphrastic constructions using lexical verbs such as \textit{rêːtà} ‘bring’, as in (\ref{bkm:Ref488854415}--\ref{bkm:Ref488854449}), or \textit{sîyà} ‘leave’, as in (\ref{bkm:Ref488854450}).

\ea
\label{bkm:Ref488854415}
ècò \textbf{nìcáꜝ}\textbf{réːtà} kùtéyè ndìkàbíré mùcêcì kùrwáràrwàrà háꜝzíkò ryángù\\
\gll e-co    ni-cí̲-a-ré̲ːt-a        kutéye ndi-kabir-é̲      mu-∅-céci ∅-ku-rwára-rwar-a    há-zíko  ri-angú \\
\textsc{aug}-\textsc{dem}.\textsc{iii}\textsubscript{7}  \textsc{pst}-\textsc{sm}\textsubscript{7}-\textsc{pst}-bring-\textsc{fv}<\textsc{rel}>  that
\textsc{sm}\textsubscript{1SG}-enter-\textsc{pfv}.\textsc{sbjv}  \textsc{np}\textsubscript{18}-\textsc{np}\textsubscript{9}-church
\textsc{cop}-\textsc{np}\textsubscript{15}-\textsc{pl}2-be\_sick-\textsc{fv}  \textsc{np}\textsubscript{16}-hearth  \textsc{pp}\textsubscript{5}-\textsc{poss}\textsubscript{1SG}\\
\glt ‘What made me go to church, was sickness in my family.’ (ZF\_Narr15)
\z

\ea
\label{bkm:Ref488854449}
òzyú mùntù ndéyè \textbf{nàréːtì} bàndìzwîsè\\
\gll o-zyú    mu-ntu ndi-éye  na-réːt-i      ba-ndi-zwís-e \\
\textsc{aug}-\textsc{dem}.\textsc{i}\textsubscript{1}  \textsc{np}\textsubscript{1}-person
\textsc{cop}-\textsc{pers}\textsubscript{3SG}  \textsc{sm}\textsubscript{1}.\textsc{pst}-bring-\textsc{npst}.\textsc{pfv}  \textsc{sm}\textsubscript{2}-\textsc{om}\textsubscript{1SG}-fire-\textsc{pfv}.\textsc{sbjv}\\
\glt ‘This person, s/he is the one who got me fired.’ (NF\_Elic17)
\z

\ea
\label{bkm:Ref488854450}
\textbf{kàndìsîyì} ìyé ndìyàbùré zìfûhà\\
\gll ka-a-ndi-sí-i      iye  ndi-yabur-é̲    zi-fúha\\
\textsc{neg}-\textsc{sm}\textsubscript{1}-\textsc{om}\textsubscript{1SG}-leave-\textsc{neg}  that  \textsc{sm}\textsubscript{1SG}-pick-\textsc{pfv}.\textsc{sbjv}  \textsc{np}\textsubscript{8}-bone\\
\glt ‘He doesn’t let me pick the bones.’ (NF\_Narr17)
\z

The causative suffix can, however, also be used to express indirect causation, in which case it adds a sense of force or urgency. In (\ref{bkm:Ref488854522}), the speaker’s mother is directing her/him to sweep using a verbal command, but this is interpreted as being very forceful, for instance, as a punishment.

\ea
\label{bkm:Ref488854522}
bámà bànàndìkúrîsì\\
\gll ba-má    ba-na-ndi-kur-í̲s-i\\
\textsc{np}\textsubscript{2}-mother  \textsc{sm}\textsubscript{2SG}-\textsc{pst}-\textsc{om}\textsubscript{1SG}-sweep-\textsc{caus}-\textsc{npst}.\textsc{pfv}\\
\glt ‘My mother made/forced me to sweep.’ (NF\_Elic17)
\z

In other cases, examples that may be ambiguous between an indirect reading and a more direct or sociative reading never receive an indirect reading. In (\ref{bkm:Ref485900877}), the only correct interpretation of the causative is sociative, where both participants perform the action together. An interpretation of indirect causation, where the causer directs the causee to perform the action through verbal instruction, is not accepted.

\ea
\label{bkm:Ref485900877}
àndìkàbìrìsá ꜝmwíꜝrápà\\
\gll a-ndi-kabir-is-á̲    mú-e-∅-rapá\\
\textsc{sm}\textsubscript{1}-\textsc{om}\textsubscript{1SG}-enter-\textsc{caus}-\textsc{fv}  \textsc{np}\textsubscript{18}-\textsc{aug}-\textsc{np}\textsubscript{5}-courtyard\\
\glt ‘S/he enters the courtyard with me.’ \\
Not: ‘S/he tells me to enter/makes me enter the courtyard.’ (NF\_Elic17)
\z

The preference for an interpretation of direct causation, and the added notion of ‘force’ or ‘urgency’ in indirect causatives, show that the causative derivation in Fwe is mainly used for the expression of direct causation. Indirect causation is more accurately expressed with periphrastic constructions.

The causative also has other uses which are less closely related to its central causative meaning. One of these is to express an instrumental meaning, in which case the object of the causative verb is interpreted as an instrument. In this sense Fwe differs from most Bantu languages, where the applicative rather than the causative is used as an instrumental \citep{Jerro2017}. The instrumental use of the causative is also attested in other Bantu Botatwe languages, such as Tonga (\citealt{Carter2002}: 47; \citealt{Collins1962}: 58-59), Ila \citep[123-127]{Smith1964}, Lenje \citep[47]{Madan1908}, and Totela \citep[669]{Crane2019}, suggesting that this innovation may have occurred on the level of Proto-Bantu Botatwe. The instrumental use of the causative in Fwe is illustrated in (\ref{bkm:Ref488917150}--\ref{bkm:Ref488917152}).

\newpage
\ea
\label{bkm:Ref488917150}
ndìkùmbìrákò àkàfùrò \textbf{ndìkàfúndìsèkò} ènyàmá ꜝyángù\\
\gll ndi-ku\textsubscript{H}mbir-a=kó̲    ka-furo  ndi-ka\textsubscript{H}-fú̲nd-is-e=ko e-nyamá  i-angú\\
\textsc{sm}\textsubscript{1SG}-request-\textsc{fv}=\textsc{loc}\textsubscript{17}  \textsc{np}\textsubscript{12}-knife  \textsc{sm}\textsubscript{1SG}-\textsc{om}\textsubscript{12}-cut-\textsc{caus}-\textsc{pfv}.\textsc{sbjv}=\textsc{loc}\textsubscript{17}
\textsc{aug}-meat  \textsc{pp}\textsubscript{9}-\textsc{poss}\textsubscript{1SG}\\
\glt ‘I ask for a knife so that I can cut my meat with it.’ (ZF\_Elic13)
\z

\ea
\label{bkm:Ref488917152}
kwìn’ èsábúrè èryò bánàkùshàkà \textbf{kùmífùndìsàngà}\\
\gll ku-iná  e-∅-sabúre e-ryo    bá̲-naku-shak-a    ku-mí-fund-is-ang-a\\
\textsc{np}\textsubscript{17}-be\_at  \textsc{aug}-\textsc{np}\textsubscript{5}-machete
\textsc{aug}-\textsc{dem}.\textsc{iii}\textsubscript{5}  \textsc{sm}\textsubscript{2}.\textsc{rel}-\textsc{hab}-want-\textsc{fv}  \textsc{inf}-\textsc{om}\textsubscript{2PL}-cut-\textsc{caus}-\textsc{hab}-\textsc{fv}\\
\glt ‘There is a machete that he keeps wanting to cut you with.’ (NF\_Narr15)
\z

Another strategy for marking instruments is the use of the comitative clitic \textit{nV=} (see \sectref{bkm:Ref486270340}). This clitic may be used without the causative suffix on the verb, as in (\ref{bkm:Ref489369936}), or may combine with a verb with a causative, as in (\ref{bkm:Ref74914062}), which is interpreted as emphasizing the instrument.

\ea
\label{bkm:Ref489369936}
kùhòmpwèrà nènsàndò\\
\gll ku-hompwer-a  ne=N-sando\\
\textsc{inf}-hammer-\textsc{fv}  \textsc{com}=\textsc{np}\textsubscript{9}-hammer\\
\glt ‘to hit with a hammer’
\z

\ea
\label{bkm:Ref74914062}
kùhòmpwèrèsà nènsàndò\\
\gll ku-hompw-er-es-a    ne=N-sando\\
\textsc{inf}-hammer-\textsc{caus}-\textsc{fv}  \textsc{com}=\textsc{np}\textsubscript{9}-hammer\\
\glt ‘to hit with a hammer (not with something else)’ (NF\_Elic17)
\z

The instrumental meaning of the causative is also found in nouns derived from causative verbs with the suffix \textit{-o} (see also \sectref{bkm:Ref451514817} on nominal derivation).
\NumTabs{4}
\TabPositions{.2\textwidth, .4\textwidth, .6\textwidth, .8\textwidth}

\ea
cì-bbùkùrìsò \tab ‘bellows’ \tab kù-bbùkùr-à \tab ‘to stoke a fire’\\
cì-fwìnkìsò \tab ‘stopper, seal’ \tab kù-fwìnk-à \tab ‘to seal’\\
cì-kùrìsò \tab ‘broom’ \tab kù-kùr-à \tab ‘to sweep’\\
cí-àrìsò \tab ‘latch’ \tab kú-àr-à \tab ‘to close’\\
\z

The causative can also be used in combination with the reflexive prefix \textit{rí-/kí-} to indicate an action that someone is pretending to perform, as in (\ref{bkm:Ref99027514}--\ref{bkm:Ref99027516}).

\ea
\label{bkm:Ref99027514}
ákùríònèsà búryò \\
\gll a-óku-rí-on-es-a        bu-ryó\\
\textsc{sm}\textsubscript{1}-\textsc{npst}.\textsc{ipfv}-\textsc{refl}-snore-\textsc{caus}-\textsc{fv}   \textsc{np}\textsubscript{14}-just\\
\glt ‘She was just pretending to snore.’
\z

\ea
\label{bkm:Ref99027516}
kùrízyùmìnìzà (cf. kùzyúmìnìnà ‘be unconscious’)\\
ku-rí-zyúm-iniz-a\\
\textsc{inf}-\textsc{refl}-be\_hard-\textsc{int}.\textsc{caus}-\textsc{fv}\\
\glt ‘to pretend to be unconscious’
\z
\section{Applicative}
\label{bkm:Ref451514914}\hypertarget{Toc75352660}{}
The applicative is marked by a derivational suffix realized as \nobreakdash-\textit{ir/\nobreakdash-er/\nobreakdash-in/-en}, depending on vowel height harmony and nasal harmony (see Sections \ref{bkm:Ref451863900}-\ref{bkm:Ref70697295}). The four different forms are illustrated in (\ref{bkm:Ref70072714}).

\NumTabs{5}
\ea
\label{bkm:Ref70072714}
kùàmbà  ‘to speak’ \tab    >  kùàmbìrà ‘to tell (someone)’\\
kùnyènsà ‘to defend’ \tab   > kùnyènsèrà ‘to defend for’\\
kùkàːnà ‘to refuse’  \tab   > kùkáːnìnà ‘to refuse to/for’\\
kùtòmà ‘to charge dowry’ \tab  > kùtòmènà to charge dowry to’
\z

The applicative can be realized differently when preceded by a causative suffix. Three different realizations of the causative/applicative combination are possible (aside from allomorphs due to vowel harmony): \textit{-is-ir, -is-iz}, \textit{-is-ik-iz}. All three forms are illustrated in (\ref{bkm:Ref485978617}) with the verb \textit{zw} ‘come out’. Note that in all cases, the causative precedes the applicative, as is typical for many Bantu languages \citep{Hyman2003b}.

\ea
\label{bkm:Ref485978617}
kùzwìsìrà {\textasciitilde} kùzwìsìzà {\textasciitilde} kùzwìsìkìzà\\
ku-zw-is-ir/iz/ikiz-a\\
\textsc{inf}-come\_out-\textsc{caus}-\textsc{appl}-\textsc{fv}\\
\glt ‘to take out to/for’
\z

With verbs that take a short causative, the addition of the applicative suffix leads to similar forms, e.g. \textit{-s-ir}, \textit{-s-iz}, and \textit{-s-ik-iz}, as illustrated in (\ref{bkm:Ref74914203}) with the causative verb \textit{bûːs} ‘wake up (someone)’.

\ea
\label{bkm:Ref74914203}
\glll kùbúːsìrà\\
ku-búː-s-ir-a\\
\textsc{inf}-wake-\textsc{caus}-\textsc{appl}-\textsc{fv}\\
\glt ‘to wake up for/on behalf of’
\z

\ea
\glll kùbúːsìzà\\
ku-búː-s-iz-a\\
\textsc{inf}-wake-\textsc{caus}-\textsc{appl}-\textsc{fv}\\
\glt ‘to wake up for/on behalf of’
\z

\ea
\glll kùbúːsìkìzà\\
ku-búː-s-ik-iz-a\\
\textsc{inf}-wake-\textsc{caus}-?-\textsc{appl}-\textsc{fv}\\
\glt ‘to wake up for/on behalf of
\z

The form \textit{-(i)s-ir} is the regular combination of the causative \textit{-(i)s} and the applicative \textit{-ir}. The form \textit{-(i)s-iz} can be a analyzed as a combination of the causative \textit{-(i)s}, the applicative \textit{-ir}, and the short causative, which causes the consonant /r/ of the applicative to change to /z/. The form \textit{-(i)s-ik-iz} is similar to the form \textit{-(i)s-iz}, but contains an extra epenthetic sequence \textit{-ik}. Similar forms where the combination of causative and applicative contains an unexpected /k/ are seen in, for instance, Nyakyusa. {\citet{Hyman2003a}} shows that the appearance of /k/ is related to the spirantization of the root-final consonant caused by the addition of the causative suffix. When an additional applicative suffix is used, spirantization targets the final consonant of the applicative suffix instead, which spirantizes to /s/, but the original root-final consonant is reinterpreted as /k/ (rather than the original non-spirantized consonant). This subsequently led to the insertion of \textit{-ik} with applicativized causatives, even with those verb roots that were never subject to spirantization. A similar scenario may account for the use of \textit{-ik} in the combination of causative and applicative in Fwe. While in Fwe, applicativized causatives never show the reinterpretation of the verb’s last root consonant to /k/, it is possible that this took place in an earlier stage of the language and has since been undone through analogy.

The applicative is highly productive: it can be added to any verb stem, and its semantic and syntactic functions are very stable. There are also some verbs that appear to feature a lexicalized, unproductive applicative suffix, but that are not attested without the applicative suffix. Examples are given in (\ref{bkm:Ref492118866}).

\ea
\label{bkm:Ref492118866}
àrìrà \tab ‘follow (in order of birth)’\\
dékèshèrà \tab ‘move the shoulders in a dancing movement’\\
fúzìrà \tab ‘blow on/fan a fire’\\
gángìrà \tab ‘freeze’\\
kàbìrà \tab ‘enter’\\
kácìkìrà \tab ‘get interrupted’\\
kákàtìrà \tab ‘get stuck’\\
ròbèrà \tab ‘capsize; to eat fast’\\
sùbìrà \tab ‘be red’\\
tòmbwèrà \tab ‘weed’\\
zùmìnà \tab ‘believe, agree; accept a marriage proposal’\\
zyàmbìrà \tab ‘gather’\\
\z

Other verbs with a lexicalized applicative suffix do occur in their underived form, but there are unsystematic differences in meaning between the underived verb and the verb featuring the applicative, as in (\ref{bkm:Ref70697282}).

\NumTabs{3}
\ea
kúmbìrà ‘beg’  \tab -  kûmbà ‘shout, howl’\\
shúmìnà ‘tie’  \tab  -  shûmà ‘bite’\\
ráːrìrà ‘eat dinner’ \tab -  râːrà ‘sleep’\\
shèndèkèrà ‘mock’ \tab -  shèndèkà ‘put in a leaning position’\label{bkm:Ref70697282}
\z

A verb cannot take more than one applicative suffix. The intensive suffix, which formally consists of the reduplication of the applicative suffix, carries neither the syntactic nor the semantic functions of the applicative, and is therefore analyzed separately in \sectref{bkm:Ref485997587}. Verbs that have a lexicalized applicative suffix do take an applicative suffix in the appropriate syntactic and semantic contexts, providing further evidence that the apparent applicative suffix has been reanalyzed as part of the root. For instance, the verb \textit{zyambir} ‘gather’ contains an element \textit{-ir} that functions as part of the verb stem, and therefore allows the addition of the applicative suffix, as in (\ref{bkm:Ref506980992}).

\ea
\label{bkm:Ref506980992}
bàkwèsì bàndìzyàmbìrírà\\
\gll ba-kwesi  ba-ndi-zyambir-ir-á̲\\
\textsc{sm}\textsubscript{2}-\textsc{prog}  \textsc{sm}\textsubscript{2}-\textsc{om}\textsubscript{1SG}-gather-\textsc{appl}-\textsc{fv}\\
\glt ‘They are gathering for me.’ (NF\_Elic17)
\z

The applicative suffix increases the valency of the verb by allowing the expression of an extra, applied object. When the applicative derivation is used with an intransitive verb, such as the verb \textit{berek} ‘work’, it derives a transitive verb \textit{bereker} ‘work for’, as in (\ref{bkm:Ref99961934}).

\ea
\label{bkm:Ref99961934}
ndìbérékèrè\\
\gll ndi-beré̲k-er-e\\
\textsc{om}\textsubscript{1SG}-work-\textsc{appl}-\textsc{pfv}.\textsc{sbjv}\\
\glt ‘Work for me.’ (NF\_Elic15)
\z

When used with a transitive verb, the applicative derives a ditransitive verb taking two objects. The order of the two objects is free: the applied object can either be the first object, as in (\ref{bkm:Ref445887382}), or the second object, as in (\ref{bkm:Ref461182140}).

\ea
\label{bkm:Ref445887382}
tùzyáːkír’ ómwâncè njûò\\
\gll tu-zyaː\textsubscript{H}k-ir-á̲    o-mu-ánce    N-júo\\
\textsc{sm}\textsubscript{1PL}-build-\textsc{appl}-\textsc{fv}  \textsc{aug}-\textsc{np}\textsubscript{1}-child  \textsc{np}\textsubscript{9}-house\\
\glt ‘…so that we build a house for the child.’ (NF\_Narr15)
\z

\ea
\label{bkm:Ref461182140}
náàùrìrá èzíryò àbânè\\
\gll ná̲-a-ur-ir-á̲      e-zi-río    a-ba-án-e\\
\textsc{pst}.\textsc{sm}\textsubscript{1}-buy-\textsc{appl}-\textsc{fv}  \textsc{aug}-\textsc{np}\textsubscript{8}-food  \textsc{aug}-\textsc{np}\textsubscript{2}-child-\textsc{poss}\textsubscript{3SG}\\
\glt ‘S/he bought food for her/his children.’ (ZF\_Elic14)
\z

It is possible for either the applied object to be pronominalized with an object marker on the verb, as in (\ref{bkm:Ref485988650}), or the direct object, as in (\ref{bkm:Ref75169894}), or both, as in (\ref{bkm:Ref75169895}). When both objects are marked by object markers, the applied object is marked closest to the verb stem, and the reverse order is not possible, as shown by the ungrammaticality of (\ref{bkm:Ref75169896}). Example (\ref{bkm:Ref98836118}), which involves an animate applied object (‘you’) and an animate direct object (‘him’) shows that animacy does not play a role, as the applied object is still closest to the verb stem. Note that Fwe only allows multiple object markers if at least one has an animate referent (see \sectref{bkm:Ref75169159}).

\ea
\label{bkm:Ref485988650}
àbàsànzìrá òtùsûbà\\
\gll a-ba\textsubscript{H}-sanz-ir-á̲    o-tu-súba\\
\textsc{sm}\textsubscript{1}-\textsc{om}\textsubscript{2}-wash-\textsc{appl}-\textsc{fv}  \textsc{aug}-\textsc{np}\textsubscript{13}-dish\\
\glt ‘S/he washes the dishes for her.’
\z

\ea
\label{bkm:Ref75169894}
àtùsànzìrá bànyìnà\\
\gll a-tu\textsubscript{H}-sanz-ir-á̲    ba-nyina\\
\textsc{sm}\textsubscript{1}-\textsc{om}\textsubscript{13}-wash-\textsc{appl}-\textsc{fv}  \textsc{np}\textsubscript{2}-mother\\
\glt ‘S/he washes them for her/his mother.’
\z

\ea
\label{bkm:Ref75169895}
\glll àtùbàsànzírà\\
a-tu\textsubscript{H}-ba-sanz-ir-á̲\\
\textsc{sm}\textsubscript{1}-\textsc{om}\textsubscript{13}-\textsc{om}\textsubscript{2}-wash-\textsc{appl}-\textsc{fv}\\
\glt ‘S/he washes them for her.’
\z

\ea
\label{bkm:Ref75169896}
*àbàtùsànzírà\\
a-ba\textsubscript{H}-tu\textsubscript{H}-sanz-ir-á̲\\
\textsc{sm}\textsubscript{1}-\textsc{om}\textsubscript{2}-\textsc{om}\textsubscript{13}-wash-\textsc{appl}-\textsc{fv}\\
Intended: ‘S/he washes them for her.’ (NF\_Elic17)
\z

\ea
\label{bkm:Ref98836118}
\glll ndàmùkùdámînì\\
ndi-a-mu-ku-dam-ín-i\\
\textsc{sm}\textsubscript{1SG}-\textsc{pst}-\textsc{om}\textsubscript{1}-\textsc{om}\textsubscript{2SG}-beat-\textsc{appl}-\textsc{npst}.\textsc{pfv}\\
\glt ‘I’ve beaten him for you.’
\z

When an applicative verb is passivized, either object of the applicative can become the subject. Compare the active clause in (\ref{bkm:Ref492378049}) with the passive version in (\ref{bkm:Ref485988949}), where the direct object has become the subject, and in (\ref{bkm:Ref75170012}), where the applied object has become the subject.

\ea
\label{bkm:Ref492378049}
àzyàːkìrá mwáncè kàjûò\\
\gll a-zyaː\textsubscript{H}k-ir-á  ̲  mu-ánce  ka-júo\\
\textsc{sm}\textsubscript{1}-build-\textsc{appl}-\textsc{fv}  \textsc{np}\textsubscript{1}-child  \textsc{np}\textsubscript{12}-room\\
\glt ‘S/He builds a room for the child.’
\z

\ea
\label{bkm:Ref485988949}
kàjúò kàzyáːkìrwà mwâncè\\
\gll ka-júo    ka-zyáːk-ir-w-a    mw-ánce\\
\textsc{np}\textsubscript{12}-room  \textsc{sm}\textsubscript{12}-build-\textsc{appl}-\textsc{pass}-\textsc{fv}  \textsc{np}\textsubscript{1}-child\\
\glt ‘The room is built for the child.’
\z

\ea
\label{bkm:Ref75170012}
mwáncè àzyàːkìrwá kàjûò\\
\gll mu-ánce  a-zyaː\textsubscript{H}k-ir-w-á̲    ka-júo\\
\textsc{np}\textsubscript{1}-child  \textsc{sm}\textsubscript{1}-build-\textsc{appl}-\textsc{pass}-\textsc{fv}  \textsc{np}\textsubscript{12}-room\\
\glt ‘The child is built a room for.’ (NF\_Elic17)
\z

The applicative can be used to express an action performed for the benefit of someone, as in (\ref{bkm:Ref485989591}), where the beneficiary is \textit{òmùkéntù wàkwé} ‘his wife’, and in (\ref{bkm:Ref488922302}), where the beneficiary is \textit{àbânè} ‘her children’. The applicative can also be used with a malefactive meaning, i.e. an action performed to the detriment of the recipient, e.g. the first person singular in (\ref{bkm:Ref488921351}), or \textit{bàntù} ‘people’ in (\ref{bkm:Ref445887391}).

\ea
\label{bkm:Ref485989591}
nàhúrírì \textbf{òmùkéntù} \textbf{wàkw’} ómùròrà\\
\gll na-ur-í̲r-i        o-mu-kéntu    u-akwé o-mu-rora\\
\textsc{sm}\textsubscript{1}.\textsc{pst}-buy-\textsc{appl}-\textsc{npst}.\textsc{pfv}  \textsc{aug}-\textsc{np}\textsubscript{1}-woman  \textsc{pp}\textsubscript{1}-\textsc{poss}\textsubscript{3SG}
\textsc{aug}-\textsc{np}\textsubscript{3}-soap\\
\glt ‘He bought soap \textbf{for} \textbf{his} \textbf{wife}.’ (ZF\_Elic14)
\z

\ea
\label{bkm:Ref488922302}
èzìbyá èzò nàáꜝsíyà náàzísìyìrà \textbf{àbânè}\\
\gll e-zi-byá    e-zo    na-á̲-si\textsubscript{H}-á̲
ná̲-a-zí-si-ir-a       a-ba-án-e\\
\textsc{aug}-\textsc{np}\textsubscript{8}-item  \textsc{aug}-\textsc{dem}.\textsc{iii}\textsubscript{8}  \textsc{rem}-\textsc{sm}\textsubscript{1}-leave-\textsc{fv}<\textsc{rel}>
\textsc{rem}-\textsc{sm}\textsubscript{1}-\textsc{om}\textsubscript{8}-leave-\textsc{appl}-\textsc{fv}  \textsc{aug}-\textsc{np}\textsubscript{2}-child-\textsc{poss}\textsubscript{3SG}\\
\glt ‘The items that she left, she left them \textbf{for} \textbf{her} \textbf{children}.’ (ZF\_Conv13)
\z

\ea
\label{bkm:Ref488921351}
shòshák’ ókù\textbf{ndì}zyónàwìrà màshéshwà ángù\\
\gll sha-o-shak-á̲    o-ku-ndi-zyón-a-u-ir-a ma-shéshwa    a-angú\\
\textsc{inc}-\textsc{sm}\textsubscript{2SG}-want-\textsc{fv}  \textsc{aug}-\textsc{inf}-\textsc{om}\textsubscript{1SG}-destroy-\textsc{pl}1-\textsc{sep}-\textsc{appl}-\textsc{fv}
\textsc{np}\textsubscript{6}-marriage    \textsc{pp}\textsubscript{6}-\textsc{poss}\textsubscript{1SG}\\
\glt ‘You now want to destroy [\textbf{for} \textbf{me}] my marriage.’ (NF\_Narr15)
\z

\ea
\label{bkm:Ref445887391}
kùhíbìrà \textbf{bàntù} màshéréŋì mbúbbì\\
\gll ku-híb-ir-a    ba-ntu  ma-sheréŋi  N-bu-bbí\\
\textsc{inf}-steal-\textsc{appl}-\textsc{fv}  \textsc{np}\textsubscript{2}-person  \textsc{np}\textsubscript{6}-money  \textsc{cop}-\textsc{np}\textsubscript{14}-bad\\
\glt ‘Stealing money \textbf{from} \textbf{people} is bad.’ (NF\_Elic17)
\z

Applicatives can have a substitutive function, where the applied object refers to someone on whose behalf the action is performed, as in (\ref{bkm:Ref71208035}--\ref{bkm:Ref71208036}).

\ea
\label{bkm:Ref71208035}
\textbf{ndì}hítwìrè bùk’ éyì kwàòbèt\\
\gll ndi-hítur-ir-e      ∅-buká  e-í    kwa-obet\\
\textsc{om}\textsubscript{1SG}-carry-\textsc{appl}-\textsc{pfv}.\textsc{sbjv}  \textsc{np}\textsubscript{9}-book  \textsc{aug}-\textsc{dem}.\textsc{i}\textsubscript{9}  \textsc{np}\textsubscript{17}-Orbet\\
\glt ‘Carry this book \textbf{for} \textbf{me} to Orbet.’ (ZF\_Elic14)
\z

\ea
\label{bkm:Ref71208036}
\glll ndà\textbf{mù}káːnìnì\\
ndi-a-mu-káːn-in-i\\
\textsc{sm}\textsubscript{1SG}-\textsc{pst}-\textsc{om}\textsubscript{1}-refuse-\textsc{appl}-\textsc{npst}.\textsc{pfv}\\
\glt ‘I’ve refused \textbf{on} \textbf{his} \textbf{behalf}.’ (Context: someone wants to take the belongings of a third person, who is not present. The speaker refuses on behalf of this absent third person.) (NF\_Elic17)
\z

The applied object can also be interpreted as the reason of the action, as in (\ref{bkm:Ref71208055}--\ref{bkm:Ref75170222}).

\ea
\label{bkm:Ref71208055}
mbòndísànzìr’ \textbf{ómùráːrìrò} tùsûbà\\
\gll mbo-ndí̲-sanz-ir-é̲        o-mu-ráːriro    tu-súba\\
\textsc{near}.\textsc{fut}-\textsc{sm}\textsubscript{1SG}-wash-\textsc{appl}-\textsc{pfv}.\textsc{sbjv}  \textsc{aug}-\textsc{np}\textsubscript{3}-dinner  \textsc{np}\textsubscript{13}-dish\\
\glt ‘I will wash the dishes \textbf{for} \textbf{dinner}.’
\z

\ea
ndìzyàːkìr’ \textbf{ómùndáré} \textbf{ꜝ}\textbf{wángù} cìòngò\\
\gll ndi-zyaː\textsubscript{H}k-ir-á̲    o-mu-ndaré    u-angú  ci-ongo\\
\textsc{sm}\textsubscript{1SG}-build-\textsc{appl}-\textsc{fv}  \textsc{aug}-\textsc{np}\textsubscript{3}-maize  \textsc{pp}\textsubscript{3}-\textsc{poss}\textsubscript{1SG}  \textsc{np}\textsubscript{7}-storage\\
\glt ‘I am building a storage \textbf{for} \textbf{my} \textbf{maize}.’ (NF\_Elic17)
\z

\ea
\label{bkm:Ref75170222}
kòóːrì òkù\textbf{yí}bèrèkèrà múmwêzì mbó\textbf{yì}bèrèkèré èmyézì yòbírè yòtâtwè\\
\gll ka-o-ó̲ːr-i      o-ku-í-berek-er-a      mú-mu-ézi mbo-ó̲-i\textsubscript{H}-berek-er-é̲  e-mi-ézi     i-o=biré    i-o=tátwe \\
\textsc{neg}-\textsc{sm}\textsubscript{2SG}-can-\textsc{neg}  \textsc{aug}-\textsc{inf}-\textsc{om}\textsubscript{9}-work-\textsc{appl}-\textsc{fv}  \textsc{np}\textsubscript{18}-\textsc{np}\textsubscript{3}-month
\textsc{near}.\textsc{fut}-\textsc{sm}\textsubscript{2SG}-\textsc{om}\textsubscript{9}-work-\textsc{appl}-\textsc{pfv}.\textsc{sbjv}
\textsc{aug}-\textsc{np}\textsubscript{4}-month   \textsc{pp}\textsubscript{4}-\textsc{con}=two  \textsc{pp}\textsubscript{4}-\textsc{con}=three\\
\glt ‘You cannot work \textbf{for} \textbf{it} in a month, you will work \textbf{for} \textbf{it} for two or three months.’ (Context: discussing how long it takes to earn 2000 Namibian dollars.) (ZF\_Conv13)
\z

The applicative can also be used to add a locative noun phrase, with two possible functions: either to express a direction or goal, or to express focus on the locative (see  \citealt{GunninkPacchiarottiforthcoming} for a detailed discussion of Fwe applicatives when used with locative phrases). While locative phrases can also be added to underived verbs, the use of the applicative causes the locative phrase to be interpreted as a direction or goal. This is illustrated with the verb \textit{shotok} ‘jump’, where a locative with the underived verb is interpreted as that which is jumped on or over, as in (\ref{bkm:Ref463363470}--\ref{bkm:Ref463363472}), but used with an applicative, the locative expresses a direction, as in (\ref{bkm:Ref463363486}).

\ea
\label{bkm:Ref463363470}
nàshótòkì àkàyèzì\\
\gll na-shótok-i      a-ka-yezi\\
\textsc{sm}\textsubscript{1}.\textsc{pst}-jump-\textsc{npst}.\textsc{pfv}  \textsc{aug}-\textsc{np}\textsubscript{12}-stream\\
\glt ‘S/he jumped over the stream.’ (ZF\_Elic14)
\z

\ea
\label{bkm:Ref463363472}
\glll ndókùríshòtòkà\\
ndi-ó=ku-rí-shotok-a\\
\textsc{pp}\textsubscript{1SG}-\textsc{con}=\textsc{inf}-\textsc{om}\textsubscript{5}-jump-\textsc{fv}\\
\glt ‘Then I stepped on it.’ (ZF\_Narr13)
\z

\ea
\label{bkm:Ref463363486}
àshòtòkèrá mùmênjì\\
\gll a-sho\textsubscript{H}tok-er-á̲  mu-ma-ínji\\
\textsc{sm}\textsubscript{1}-jump-\textsc{appl}-\textsc{fv}  \textsc{np}\textsubscript{18}-\textsc{np}\textsubscript{6}-water\\
\glt ‘S/he jumps into the water.’ (NF\_Elic15)
\z

Whether the applicative is required to express a direction or goal depends on the lexical verb. For certain motion verbs, a location, such as a source or direction, is part of their lexical semantics, and as such these verbs can be combined with a locative phrase without the use of the applicative derivation. This is the case for, for instance, the verb \textit{zw} ‘leave’, which includes the source (the place from which one leaves) in its lexical semantics, and therefore the use of a locative noun phrase referring to the source does not require an applicative, as in (\ref{bkm:Ref463366428}). Verbs that include direction as inherent part of their lexical semantics also do not require the applicative to combine with a locative noun phrase expressing direction, such as the verb \textit{yend} ‘go, walk’ in (\ref{bkm:Ref99960395}), \textit{y} ‘go’ in (\ref{bkm:Ref99960396}), and \textit{keːzy} ‘come’ in (\ref{bkm:Ref99960398}).\largerpage

\ea
\label{bkm:Ref463366428}
àmàròhà àzwá hàcìrábì\\
\gll a-ma-roha    a-zw-á̲    ha-ci-rabí\\
\textsc{aug}-\textsc{np}\textsubscript{6}-blood  \textsc{sm}\textsubscript{6}-come\_out-\textsc{fv}  \textsc{np}\textsubscript{16}-\textsc{np}\textsubscript{7}-wound\\
\glt ‘Blood comes from the wound.’ (NF\_Elic15)
\z

\ea
\label{bkm:Ref99960395}
ndìyéndè bùryò kùmùnzì\\
\gll ndi-é̲nd-e    bu-ryo  ku-mu-nzi\\
\textsc{sm}\textsubscript{1SG}-go-\textsc{pfv}.\textsc{sbjv}  \textsc{np}\textsubscript{14}-just  \textsc{np}\textsubscript{17}-\textsc{np}\textsubscript{3}-village\\
\glt ‘Let me just go home.’ (ZF\_Narr14)
\z

\ea
\label{bkm:Ref99960396}
ndìyá kwàsèshèkè\\
\gll ndi-y-á̲  kwa-sesheke\\
\textsc{sm}\textsubscript{1SG}-go-\textsc{fv}  \textsc{np}\textsubscript{17}-Sesheke \\
\glt ‘I am going to Sesheke.’ (ZF\_Elic13)
\z

\ea
\label{bkm:Ref99960398}
nàbàkéːzyà kúmùnzí ꜝwábò\\
\gll na-ba-a-ké̲ːzy-a    kú-mu-nzí    u-abó\\
\textsc{rem}-\textsc{sm}\textsubscript{2}-\textsc{pst}-come-\textsc{fv}  \textsc{np}\textsubscript{17}\-\textsc{np}\textsubscript{3}-village  \textsc{pp}\textsubscript{3}\--\textsc{dem}.\textsc{iii}\textsubscript{2}\\
\glt ‘She was coming to her village.’ (ZF\_Narr15)
\z

In motion verbs where the direction is not part of the verb’s lexical semantics, the use of a locative noun phrase expressing a direction requires the use of the applicative. This is illustrated with the verb \textit{bútuk} ‘run’ in (\ref{bkm:Ref99960480}), \textit{shótok} ‘jump’ in (\ref{bkm:Ref99960482}), and \textit{hít} ‘pass’ in (\ref{bkm:Ref99960484}).

\ea
\label{bkm:Ref99960480}
kùnjúò yàkwé àbùtùkírà\\
\gll N-ku-N-júo      i-akwé  a-bu\textsubscript{H}tuk-ir-á̲\\
\textsc{cop}-\textsc{np}\textsubscript{17}-\textsc{np}\textsubscript{9}-house  \textsc{pp}\textsubscript{9}-\textsc{poss}\textsubscript{3SG}  \textsc{sm}\textsubscript{1}-run-\textsc{appl}-\textsc{fv}\\
\glt ‘S/He is running to his house.’
\z

\ea
\label{bkm:Ref99960482}
àshòtòkérá mùmênjì\\
\gll a-sho\textsubscript{H}tok-er-á̲  mu-ma-ínji\\
\textsc{sm}\textsubscript{1}-jump-\textsc{appl}-\textsc{fv}  \textsc{np}\textsubscript{18}-\textsc{np}\textsubscript{6}-water\\
\glt ‘S/He jumps into the water.’ (NF\_Elic15)
\z

\ea
\label{bkm:Ref99960484}
bókèːzyà kùhítìrà hámùnzì\\
\gll ba-ó=keːzy-a    ku-hít-ir-a    há-mu-nzi\\
\textsc{pp}\textsubscript{2}-\textsc{con}=come-\textsc{fv}  \textsc{inf}-pass-\textsc{appl}-\textsc{fv}  \textsc{np}\textsubscript{16}-\textsc{np}\textsubscript{3}-village\\
\glt ‘Then they passed over a village.’ (ZF\_Narr13)
\z

As seen in (\ref{bkm:Ref75174184}), the use of the applicative to add a locative argument does not necessarily involve (physical) movement.

\ea
\label{bkm:Ref75174184}
ècí cìntù kàbábbòzérá àbá ꜝbámbwà cìntúnjí\\
\gll e-cí    ci-ntu    ka-bá̲-bbo\textsubscript{H}z-er-á̲  a-bá    ba-mbwá  ∅-ci-ntu-njí\\
\textsc{aug}-\textsc{dem}.\textsc{i}\textsubscript{7}  \textsc{np}\textsubscript{7}-thing  \textsc{pst}.\textsc{ipfv}-\textsc{sm}\textsubscript{2}-bark-\textsc{appl}-\textsc{fv}
\textsc{aug}-\textsc{dem}.\textsc{i}\textsubscript{2}  \textsc{np}\textsubscript{2}-dog   \textsc{cop}-\textsc{np}\textsubscript{7}-thing-what\\
\glt ‘This thing that the dogs are barking at, what is it?’ (ZF\_Narr14)
\z

The applicative can also be used to express focus on the locative, a function also seen in various other Bantu languages (see \citealt{Pacchiarotti2020}: 145 for an overview). This use of the applicative often (but not necessarily) combines with a cleft construction, the most common construction in Fwe for expressing focus (see also \sectref{bkm:Ref451503992}). As seen in (\ref{bkm:Ref506371954}--\ref{bkm:Ref506371955}), the direction/goal semantics otherwise seen in applicatives combined with locative noun phrases is not part of the use of the applicative to focus a locative.\largerpage[-1]

\ea
\label{bkm:Ref506371954}
bàbbónádì kwàsìòmà bábèrèkérà\\
\gll ba-bbonádi  ∅-kwa-sioma    bá̲-berek-er-á̲\\
\textsc{np}\textsubscript{2}-Bonard  \textsc{cop}-\textsc{np}\textsubscript{17}-Sioma  \textsc{sm}\textsubscript{2}.\textsc{rel}-work-\textsc{appl}-\textsc{fv}\\
\glt ‘Mr. Bonard, it is in Sioma that he works.’
\z

\ea
ècìbàka òkù ásèbèzèrà mùkéntù wángù kùréː ècìbàkà òkù ndísèbèzérà\\
\gll e-ci-baka    o-ku    á̲-sebez-er-á̲ mu-kéntu  u-angú ∅-ku-réː    e-ci-baka    o-ku    ndí̲-sebez-er-á̲\\
\textsc{aug}-\textsc{np}\textsubscript{7}-place  \textsc{aug}-\textsc{dem}.\textsc{i}\textsubscript{17} \textsc{sm}\textsubscript{1}.\textsc{rel}-work-\textsc{appl}-\textsc{fv}
\textsc{np}\textsubscript{1}-woman  \textsc{pp}\textsubscript{1}-\textsc{poss}\textsubscript{1SG}
\textsc{cop}-\textsc{np}\textsubscript{17}-long  \textsc{aug}-\textsc{np}\textsubscript{7}-place  \textsc{aug}-\textsc{dem}.\textsc{i}\textsubscript{17}  \textsc{sm}\textsubscript{1SG}-work-\textsc{appl}-\textsc{fv}\\
\glt ‘The place where my wife works is far from the place where I work.’ (ZF\_Elic13)
\z

\ea
\label{bkm:Ref506371955}
páhà rímànìná èkàndé ꜝryángù\\
\gll p-áha      rí̲-man-in-á̲    e-∅-kandé    ri-angú\\
\textsc{cop}\textsubscript{16}-\textsc{dem}.\textsc{i}\textsubscript{16}  \textsc{sm}\textsubscript{5}-end-\textsc{appl}-\textsc{fv}  \textsc{aug}-\textsc{np}\textsubscript{5}-story  \textsc{pp}\textsubscript{5}-\textsc{poss}\textsubscript{1SG}\\
\glt ‘This is where my story ends.’ (NF\_Narr15)
\z

\hspace*{-3.7pt}The applicative can also be used to focus morphologically locative noun phrases that refer to a time rather than a place. Locative class 16 can be used in Fwe with both locative and temporal interpretations, and the applicative can also be used to express focus when the temporal interpretation is intended, as in (\ref{bkm:Ref99962138}).

\ea
\label{bkm:Ref99962138}
páhò náàbàhìndírà\\
\gll p-áho      na-á̲-a-ba\textsubscript{H}-hind-ir-á̲\\
\textsc{cop}\textsubscript{16}-\textsc{dem}.\textsc{iii}\textsubscript{16}  \textsc{rem}-\textsc{sm}\textsubscript{1}-\textsc{pst}-\textsc{om}\textsubscript{2}-take-\textsc{appl}-\textsc{fv}<\textsc{rel}>\\
\glt ‘That’s when he took her.’ (ZF\_Narr15)
\z

The argument added by the applicative derivation may also express manner. This interpretation is only available in relative clauses introduced by the class 18 demonstrative \textit{òmò} ‘(the way) how’, used as relativizer, as in (\ref{bkm:Ref99962159}--\ref{bkm:Ref99962160}).

\ea
\label{bkm:Ref99962159}
ndìsháká òmò ázyìmbírà\\
\gll ndi-shak-á̲    o-mo    á̲-zyi\textsubscript{H}mb-ir-á̲\\
\textsc{sm}\textsubscript{1SG}-like-\textsc{fv}  \textsc{aug}-\textsc{dem}.\textsc{iii}\textsubscript{18}  \textsc{sm}\textsubscript{1}.\textsc{rel}-sing-\textsc{appl}-\textsc{fv}\\
\glt ‘I like the way s/he sings.’ (NF\_Elic15)
\z

\ea
kàbásùmwìná òmò nìbákàhàrírà\\
\gll ka-bá̲-su\textsubscript{H}mwin-á̲    o-mo      ni-bá̲-a-ka-ha\textsubscript{H}r-ir-á̲\\
\textsc{pst}.\textsc{ipfv}-\textsc{sm}\textsubscript{2}-report-\textsc{fv}  \textsc{aug}-\textsc{dem}.\textsc{iii}\textsubscript{18}  \textsc{rem}-\textsc{sm}\textsubscript{2}-\textsc{pst}-\textsc{dist}-live-\textsc{appl}-\textsc{fv}<\textsc{rel}>\\
\glt ‘They were reporting how they had been living.’ (NF\_Narr15)
\z

\ea
\label{bkm:Ref99962160}
òmò nìbáfwîrà àbò bámùcémbérè\\
\gll o-mo    ni-bá̲-a-fw-í̲r-a a-bo    bá-mu-cémbere \\
\textsc{aug}-\textsc{dem}.\textsc{iii}\textsubscript{18}  \textsc{rem}-\textsc{sm}\textsubscript{2}-\textsc{pst}-die-\textsc{appl}-\textsc{fv}<\textsc{rel}>
\textsc{aug}-\textsc{dem}.\textsc{iii}\textsubscript{2}  \textsc{np}\textsubscript{2}-\textsc{np}\textsubscript{1}-old\_woman\\
\glt ‘the way that old lady died’ (ZF\_Narr15)
\z

Verbs that have an applicative suffix that carries a different function than manner, such as benefactive, may also be used in a relative clause headed by \textit{òmò,} as in (\ref{bkm:Ref99962192}). Only one applicative suffix is used, which carries both benefactive and manner functions simultaneously, as in (\ref{bkm:Ref99962194}); as the ungrammaticality of (\ref{bkm:Ref99962196}) shows, repeating the applicative suffix is not possible. This is in line with the general restriction on combining two applicative suffixes on the same verb.

\ea
\label{bkm:Ref99962192}
ndìsháká òmw’ áhìkírà\\
\gll ndi-shak-á̲    o-mo      á̲-hi\textsubscript{H}k-ir-á̲\\
\textsc{sm}\textsubscript{1SG}-like-\textsc{fv}  \textsc{aug}-\textsc{dem}.\textsc{iii}\textsubscript{18}  \textsc{sm}\textsubscript{1}.\textsc{rel}-cook-\textsc{appl}-\textsc{fv}\\
\glt ‘I like the way she cooks.’
\z

\ea
\label{bkm:Ref99962194}
ndìsháká òmw’ ábàhìkírà\\
\gll ndi-shak-á̲    o-mo      á̲-ba\textsubscript{H}-hi\textsubscript{H}k-ir-á̲\\
\textsc{sm}\textsubscript{1SG}-like-\textsc{fv}  \textsc{aug}-\textsc{dem}.\textsc{iii}\textsubscript{18}  \textsc{sm}\textsubscript{1}.\textsc{rel}-\textsc{om}\textsubscript{2}-cook-\textsc{appl}-\textsc{fv}\\
\glt ‘I like the way she cooks for them.’
\z

\ea
\label{bkm:Ref99962196}
  *ndìsháká òmw’ ábàhìkìrírà
\z

The applicative is combined with the reflexive prefix \textit{rí}-/\textit{kí}- and the adverb \textit{buryo} ‘just, only’, to express a useless or purposeless action, as in (\ref{bkm:Ref99962252}--\ref{bkm:Ref99962254}).

\ea
\label{bkm:Ref99962252}
èrí ꜝsózù \textbf{rìrìtùmbùkírá} bùryò\\
\gll e-rí    ∅-sozú  ri-ri\textsubscript{H}-tu\textsubscript{H}mbuk-ir-\textbf{á}    bu-ryo\\
\textsc{aug}-\textsc{dem}.\textsc{i}\textsubscript{5}  \textsc{np}\textsubscript{5}-grass  \textsc{sm}\textsubscript{5}-\textsc{refl}-burn-\textsc{appl}-\textsc{fv}  \textsc{np}\textsubscript{14}-only\\
\glt ‘This grass \textbf{burns} \textbf{easily}.’
\z

\ea
èzí zìzwâtò zìcípîtè kònó \textbf{zìrìfwírà} búryò\\
\gll e-zí    zi-zwáto  zi-cip-í̲te konó  zi-ri\textsubscript{H}-fw-í̲r-a    bu-ryó\\
\textsc{aug}-\textsc{dem}.\textsc{i}\textsubscript{8}  \textsc{np}\textsubscript{8}-cloth  \textsc{sm}\textsubscript{8}-become\_cheap-\textsc{stat}
but  \textsc{sm}\textsubscript{8}-\textsc{refl}-die-\textsc{appl}-\textsc{fv}  \textsc{np}\textsubscript{14}-only\\
\glt ‘These clothes are cheap, but they won’t last long (lit. ‘they \textbf{will} \textbf{just} \textbf{break’}).’ (NF\_Elic15)
\z

\ea
\label{bkm:Ref99962254}
òmùntù \textbf{árìàmbìrààmbírà} bùryô\\
\gll o-mu-ntu    á̲-ri\textsubscript{H}-ambira-amb-ir-á̲  bu-ryó\\
\textsc{aug}-\textsc{np}\textsubscript{1}\textsuperscript{\-\-}-person  \textsc{sm}\textsubscript{1}.\textsc{rel}-\textsc{pl}2-talk-\textsc{appl}-\textsc{fv}  \textsc{np}\textsubscript{14}-just\\
\glt ‘A person who \textbf{just} \textbf{talks}…’ (NF\_Elic17)
\z
\section{Neuter}
\label{bkm:Ref489452510}\hypertarget{Toc75352661}{}
The neuter is expressed with a suffix \textit{-ahar.} This suffix is unproductive: all the attested examples are listed in (\ref{bkm:Ref494446546}).

\NumTabs{3}
\ea
\label{bkm:Ref494446546}
bônà ‘see’ \tab bónàhàrà ‘be visible’\\
fòsà ‘sin, make a mistake’ \tab fòsàhàrà ‘be wrong, be a bad person’\\
pàngà ‘do, make’ \tab pàngàhàrà ‘happen, take place’\\
sèpà ‘trust, hope’ \tab sèpàhàrà ‘be honest, important’\\
shàkà ‘want, need’ \tab shàkàhàrà ‘be necessary’\\
tèndà ‘do, make’ \tab tèndàhàrà ‘happen, take place’\\
wànà ‘find’ \tab wànàhàrà ‘be found, occur’\\
zyìbà ‘get to know’ \tab zyíbàhàrà ‘be known, famous’\\
\z

The use of the neuter derivation causes the agent of the action to be deleted and the patient to be expressed as a subject. This is illustrated in (\ref{bkm:Ref99457991}) with the verb \textit{bón} ‘see’; underived, the patient (that which is seen) is expressed as the object, and derived with the neuter suffix \textit{-ahar}, the patient is expressed as the subject.

\ea
\label{bkm:Ref99457991}
òcìbwènè ênì cìbònàhárà\\
\gll o-ci\textsubscript{H}-bwe\textsubscript{H}ne    éni  ci-bo\textsubscript{H}n-ahar-á̲\\
\textsc{sm}\textsubscript{2SG}-\textsc{om}\textsubscript{7}-see.\textsc{stat}  yes  \textsc{sm}\textsubscript{7}-see-\textsc{neut}-\textsc{fv}\\
\glt ‘Do you see it?’ ‘Yes, it’s visible.’ (NF\_Elic15)
\z

Unlike the passive, the neuter does not allow the reintroduction of the agent as a peripheral participant, as shown by the ungrammaticality of (\ref{bkm:Ref99028733}).

\ea
\label{bkm:Ref99028733}
*nìbáwànàhàrà kwángù\\
ni-bá̲-a-wan-ahar-a    ku-angú\\
\textsc{rem}-\textsc{sm}\textsubscript{2}-\textsc{pst}-find-\textsc{neut}-\textsc{fv}  \textsc{np}\textsubscript{17}-\textsc{poss}\textsubscript{1SG}\\
Intended: ‘S/he was found by me.’ (NF\_Elic17)
\z

The neuter presents the event as having no agent. The neuter verb \textit{bónahar} is interpreted as ‘look, be visible’. It does not imply being looked at by an agent, merely that being looked at is a possibility, e.g. the subject is “potentially or factually affected” \citep[75]{Schadeberg2003}, and the agent is backgrounded. The complete backgrounding of the agent is seen with the neuter verb \textit{wanahar} in (\ref{bkm:Ref452477320}), which focuses on the assumption that the profit will exist, rather than who, if anyone, will be present to find it.

\ea
\label{bkm:Ref452477320}
èngùrìsó yàkwé mbòyíwànàhárè\\
\gll e-N-gurisó    i-akwé  mbo-í̲-wan-ahar-é̲\\
\textsc{aug}-\textsc{np}\textsubscript{9}-profit  \textsc{pp}\textsubscript{9}-\textsc{poss}\textsubscript{3SG}  \textsc{near}.\textsc{fut}-\textsc{sm}\textsubscript{9}-find-\textsc{neut}-\textsc{pfv}.\textsc{sbjv}\\
\glt ‘Her profit can/will be found.’ (ZF\_Conv13)
\z

The neuter suffix \textit{-ahar} in Fwe is a borrowing from Lozi. Lozi has a number of different neuter suffixes, including the suffix \textit{-ahal}, which is unproductive according to \citet[60-61]{Gowlett1967}, as it only occurs in a fixed set of verbs. The suffix \textit{-ahar} has acquired a productivity of its own in Fwe, as it is used in verbs that do not use it in Lozi, such as the Fwe verb \textit{wanahar} ‘be found, occur’, which does not have a Lozi counterpart with the suffix \textit{-ahal}. Other Bantu languages spoken in the same region have also acquired the neuter suffix \textit{-ahar} (or variants thereof). \citet[245]{Seidel2008} notes the use of \nobreakdash-\textit{ahar} as a neuter in Yeyi, also attributing it to influence from Lozi. The use of the suffix \textit{-hala} ‘neuter’ is described for Subiya by {\citet[77]{Jacottet1896}}. It is likely that all these languages borrowed the suffix from Lozi, as Lozi is the only language in which the suffix \textit{-ahar} is morphologically analyzable as a combination of the neuter suffixes \textit{-ah} and \textit{-al} \citep[60]{Gowlett1967}. Nonetheless, the wide-spread use of \textit{-ahar} as a neuter suffix in languages that have been in contact with Lozi is surprising, as \textit{-ahar} is only one of the neuter suffixes used in Lozi, and it is not the most frequent or the most productive form of the neuter.

Fwe also has another suffix that expresses neuter, viz. \textit{-isik}/-\textit{esek}, which can be analyzed as a combination of the productive causative suffix \textit{\nobreakdash-is}, and a suffix that may be the reflex of the suffix *-ɪk reconstructed with neuter meaning for Proto-Bantu\footnote{Traces of an earlier neuter(-like) suffix that may have been a reflex of *-ɪk are conspicuously absent; no verbs have been recorded which can be analyzed as a combination of a verb stem with a now-petrified neuter-like suffix.}  (\citealt{SchadebergBostoen2019}: 173). Neuter \textit{-isik/-esek} is found with only two verbs, \textit{wan} ‘find’, which may also take the neuter suffix \textit{-ahar} without a change in meaning, as in (\ref{bkm:Ref99029817}), and \textit{oːr} ‘can’, as in (\ref{bkm:Ref99029831}).

\ea
\label{bkm:Ref99029817}
\glll kùwànìsìkà    {\textasciitilde}   kùwànàhàrà\\
ku-wan-isik-a  {}  ku-wan-ahar-a\\
\textsc{inf}-find-\textsc{neut}-\textsc{fv}   {} \textsc{inf}-find-\textsc{neut}-\textsc{fv}\\
\glt ‘to be found’
\z

\ea
\label{bkm:Ref99029831}
\glll kùòːrèsèkà \\
ku-oːr-esek-a\\
\textsc{inf}-can-\textsc{neut}-\textsc{fv}\\
\glt ‘to be possible’
\z

Possibly, the suffix \textit{-isik/-esek} was the original, native neuter suffix in Fwe, and was gradually replaced by the Lozi neuter suffix \textit{-ahar}, a development also seen in various other languages that are in contact with Lozi.

\section{Separative}
\label{bkm:Ref485823385}\hypertarget{Toc75352662}{}\label{bkm:Ref486270693}\label{bkm:Ref486270391}
The separative derivation makes use of the suffixes \textit{-ur} (transitive) and \textit{-uk} (intransitive). { \citet[186]{SchadebergBostoen2019}} analyze the common core meaning of this derivation in Bantu to be “movement out of some original position”, and hence propose the term separative. This semantic characterization fits the use of the separative in Fwe as well.

\begin{sloppypar}
The transitive separative suffix has four allomorphs -\textit{ur/-or/-un/-on}, conditioned by vowel harmony (see \sectref{bkm:Ref451863900}) and nasal harmony (see \sectref{bkm:Ref70697295}). The intransitive separative has two allomorphs -\textit{uk}/\textit{-ok} conditioned by vowel harmony. An example of the intransitive and transitive separative derivation of the verb  \'{} \textit{ar} ‘close’ is given in (\ref{bkm:Ref98833879}--\ref{bkm:Ref98833880}).\end{sloppypar}

\ea
\label{bkm:Ref98833879}
\glll kúàrùrà\\
kú-ar-ur-a\\
\textsc{inf}-close-\textsc{sep}.\textsc{tr}-\textsc{fv}\\
\glt ‘to open (tr.)’
\z

\ea
\label{bkm:Ref98833880}
\glll kúàrùkà\\
kú-ar-uk-a\\
\textsc{inf}-close-\textsc{sep}.\textsc{intr}-\textsc{fv}\\
\glt ‘to open (intr.)’
\z

Verbs with the intransitive separative suffix \textit{-uk} function as change-of-state verbs; they receive a modal interpretation in the present tense (\ref{bkm:Ref479606609}), and a present reading when used with the stative suffix \textit{-ite} (\ref{bkm:Ref479606610}).

\ea
\label{bkm:Ref479606609}
èmpótó ìbbámúkà\\
\gll e-N-potó  i-bbam-uk-á̲\\
\textsc{aug}-\textsc{np}\textsubscript{9}-pot  \textsc{sm}\-\textsubscript{9}-break-\textsc{sep}.\textsc{intr}-\textsc{fv}\\
\glt ‘A pot can break.’ (a warning to someone who is handling a pot carelessly)
\z

\ea
\label{bkm:Ref479606610}
èzí zìzyàbàrò zìcèrúkìtè\\
\gll e-zí    zi-zyabaro  zi-ce\textsubscript{H}r-ú̲k-ite\\
\textsc{aug}-\textsc{dem}.\textsc{i}\textsubscript{8}  \textsc{np}\textsubscript{8}-cloth  \textsc{sm}\textsubscript{8}-tear-\textsc{sep}.\textsc{intr}-\textsc{stat}\\
\glt ‘These clothes are torn.’ (NF\_Elic15)
\z

The separative derivation may occur in a large number of verbs and its semantics is quite predictable, but there are also many verbs that may not take the separative, as well as verbs that take the separative that may not occur without it, and verbs where the semantic import of the separative is unclear. Most verbs that take the separative derivation may occur with either the transitive or the intransitive form, as in \tabref{tab:6:4}.

\begin{table}
\label{bkm:Ref99029905}\caption{\label{tab:6:4}Transitive and intransitive separative verbs}
\begin{tabularx}{\textwidth}{lllQ}
\lsptoprule
\multicolumn{2}{c}{Transitive separative} & \multicolumn{2}{c}{Intransitive separative}\\
\midrule
\textit{àrùmùnà} & ‘roll (tr.)’ & \textit{àrùmùkà} & ‘roll (intr.)’\\
\textit{bbátùrà} & ‘separate (tr.)’ & \textit{bbátùkà} & ‘separate (intr.), be separated’\\
\textit{kùmbùrà} & ‘peel, strip’ & \textit{kùmbùkà} & ‘come off in strips, be peeled/stripped off’\\
\textit{kúzyùrà} & ‘peel a \textit{mongongo} nut’ & \textit{kúzyùkà} & ‘be peeled (of a \textit{mongongo} nut)’\\
\textit{túrùrà} & ‘pierce’ & \textit{túrùkà} & ‘burst’\\
\lspbottomrule
\end{tabularx}
\end{table}

Some verbs that may take a separative suffix are also attested in an underived form, or are also attested with another derivational suffix, such as the impositive \textit{-ik/-am}, or the extensive \textit{-ar/-an}, as shown in \tabref{tab:6:5}.

\begin{table}
\label{bkm:Ref489002537}\caption{\label{tab:6:5}Separative verbs from underived verbs}
\begin{tabularx}{\textwidth}{lQQQ}
\lsptoprule
\multicolumn{2}{c}{Separative} & \multicolumn{2}{c}{Underived verb}\\
\midrule
\textit{ròngòrà} & ‘unload’ & \textit{ròngà} & ‘load’\\
\textit{rwárùkà} & ‘become better’ & \textit{rwârà} & ‘become sick’\\
\textit{vwìkùrà} & ‘uncover’ & \textit{vwìkà} & ‘cover’\\
\textit{zyàrùrà} & ‘take blankets off the bed’ & \textit{zyàrà} & ‘make the bed’\\
\tablevspace
\multicolumn{2}{c}{Separative} & \multicolumn{2}{c}{Other derivational suffix}\\
\midrule
\textit{cánkùrà} & ‘remove from the fire’ & \textit{cánkìkà} & ‘put on the fire’\\
\textit{fúrùmùnà} & ‘put upright’ & \textit{fúrùmìkà}

\textit{fúrùmànà} & ‘place upside down’

\glt ‘be initiated (of girls)’\\
\textit{hángùrà} & ‘remove from a high position’ & \textit{hánjìkà} & ‘put in a high position’\\
\textit{kámbùrà} & ‘remove (from on top of each other)’ & \textit{kámbìkà}

\textit{kámbàmà} & ‘put on top of each other’

\glt ‘be on top of each other’\\
\textit{zyàbùrà} & ‘undress’ & \textit{zyàbàrà} & ‘dress’\\
\textit{ǀàpùrùrà} & ‘take mud from a wall’ & \textit{ǀàpìkà} & ‘put mud on a wall’\\
\lspbottomrule
\end{tabularx}
\end{table}

Many separative verbs, however, are not attested in their underived form, and the separative cannot be freely used to derive new verbs from any existing verb stem. There are also many verbs apparently consisting of a separative suffix which lack separative semantics, as in (\ref{bkm:Ref98512767}).

\ea
\label{bkm:Ref98512767}
bbùkùrà  ‘stoke a fire’\\
cùncùrà  ‘stumble’\\
bárùkà  ‘taste a crop to test if it’s ripe’\\
bútùkà  ‘run’
\z

What further underscores the semi-productive status of the separative is that some verbs with the transitive separative suffix \textit{-ur} do not function as transitive verbs, such as \textit{ᵍ}\textit{ǀíntùrà} ‘lie with bent knees’, \textit{shwáhùrà} ‘be disappointed, give up’, \textit{sùkùrà} ‘doze’. There are also verbs with the intransitive separative \textit{-uk} that are not intransitive, such as \textit{cébùkà} ‘look behind at’, \textit{kàntùkà} ‘cross (a road, river)’, \textit{tóròkà} ‘translate, explain’.

The separative suffix also occurs in a reduplicated form. Like its unreduplicated counterpart, the reduplicated separative suffix undergoes both vowel and nasal harmony, surfacing as either \textit{-urur}, \textit{-oror}, -\textit{unun} or \textit{-onon}. The intransitive variant of the reduplicated separative is \textit{-uruk}, also subject to vowel and nasal harmony. The distribution of the reduplicated and unreduplicated separative appears to be lexical, with the reduplicated form mainly (but not exclusively) occurring with verbs that also occur as underived verb stems. Verbs with the reduplicated separative suffix and their underived counterpart, if attested, are given in \tabref{tab:6:6}.

\begin{table}
\label{bkm:Ref99030029}\caption{\label{tab:6:6}The reduplicated separative suffix}

\begin{tabularx}{\textwidth}{lllQ}
\lsptoprule
\multicolumn{2}{c}{Underived verb} & \multicolumn{2}{c}{Separative verb}\\
\midrule
{\itshape gâbà} & ‘close a kraal’ & {\itshape gábùrùrà} & ‘open a kraal’\\
{\itshape hôshà} & ‘plait hair’ & {\itshape hóshòròrà} & ‘take out plaits’\\
{\itshape kìyà} & ‘lock’ & {\itshape kìyùrùrà} & ‘unlock’\\
{\itshape -} & - & {\itshape ⁿǀónzòròkà} & ‘be thread-like, stretching (like okra)’\\
{\itshape ràmbà} & ‘plaster a wall’ & {\itshape ràmbùrùrà} & ‘smoothen a plastered wall’\\
{\itshape shúmìnà} & ‘tie’ & {\itshape shúmùnùnà} & ‘untie’\\
{\itshape shwènà} & ‘become tired’ & {\itshape shwènùnùkà} & ‘become rested’\\
\lspbottomrule
\end{tabularx}
\end{table}

When the separative suffix \textit{-ur} is used in combination with the applicative suffix \textit{\nobreakdash-ir}, the form of the combined suffix is \textit{-wir}, in which the vowel /u/ of the separative suffix has devocalized to a glide. This is illustrated with the separative verb \textit{bbukur} ‘blow on a fire’ in (\ref{bkm:Ref74915135}).

\ea
\label{bkm:Ref74915135}
\glll òndìbbúkwír’ ómùrìrò\\
o-ndi-bbuk-wir-é̲            o-mu-riro\\
\textsc{sm}\textsubscript{2SG}-\textsc{om}\textsubscript{1SG}-blow\_on\_fire-\textsc{sep}.\textsc{tr}.\textsc{appl}-\textsc{pfv}.\textsc{sbjv}  \textsc{aug}-\textsc{np}\textsubscript{3}-fire\\
\glt ‘Blow on the fire for me.’ (NF\_Elic17
\z

When the separative suffix combines with a more productive causative or passive suffix, the separative suffix is directly adjacent to the verb stem, as illustrated for the combination of the separative suffix and the passive suffix in (\ref{bkm:Ref445459072}). This ordering is consistent with the tendency for morphemes with a higher productivity, like the causative and the passive, to occur at the periphery of a word, and for less productive morphemes, such as the separative, to be closer to the verb stem.

\ea
\label{bkm:Ref445459072}
\glll zàzyángùrìwà\\
zi-a-zyáng-ur-iw-a\\
\textsc{sm}\textsubscript{8}-\textsc{pst}-harvest-\textsc{sep}.\textsc{tr}-\textsc{pass}-\textsc{fv}\\
\glt ‘Are they harvested?’ (NF\_Elic17)
\z

The separative expresses a movement out of an original position. This is illustrated in (\ref{bkm:Ref445819644}--\ref{bkm:Ref445725028}), taken from a narrative in which one of the main characters, a lion, has hidden his teeth. The hiding of the teeth is described in (\ref{bkm:Ref445819644}) using the verb \textit{ziːk} ‘hide’. Afterwards, the other main character, a girl, goes to retrieve the teeth from their hiding place. This is described in (\ref{bkm:Ref445725028}) using the same verb with the separative suffix, \textit{ziːkur} ‘retrieve from its hiding place’.

\ea
\label{bkm:Ref445819644}
òndávú nàkàzíːkì àménò ákwê hàcítwè\\
\gll o-∅-ndavú    na-ka-zíːk-i a-ma-íno    a-akwé   ha-ci-twé \\
\textsc{aug}-\textsc{np}\textsubscript{1a}-lion  \textsc{sm}\textsubscript{1}.\textsc{pst}-\textsc{dist}-hide-\textsc{npst}.\textsc{pfv}
\textsc{aug}-\textsc{np}\textsubscript{6}-tooth  \textsc{pp}\textsubscript{6}-\textsc{poss}\textsubscript{3SG} \textsc{np}\textsubscript{16}-\textsc{np}\textsubscript{7}-ash\\
\glt ‘The lion has hidden his teeth under the ash.’
\z

\ea
\label{bkm:Ref445725028}
ákàzìːkùrà áò mênò\\
\gll á-o-ka-ziːk-ur-a      a-o    ma-íno\\
\textsc{sm}\textsubscript{1}-\textsc{aug}-\textsc{dist}-hide-\textsc{sep}.\textsc{tr}-\textsc{fv}  \textsc{aug}-\textsc{dem}.\textsc{iii}\textsubscript{6}  \textsc{np}\textsubscript{6}-tooth\\
\glt ‘She then dug out those teeth there.’ (NF\_Narr15)
\z

Many verbs with the separative derivation describe various acts of destruction, such as cutting, tearing or breaking, as listed in (\ref{bkm:Ref99030204}). These verbs usually lack an underived counterpart.
\NumTabs{3}
\TabPositions{.2\textwidth, .6\textwidth}
\ea
\label{bkm:Ref99030204}
bbàmùkà \tab ‘break in half’\\
bútùrà \tab ‘clear a field (by removing small shrubs and weeds)’\\
càmùnà \tab ‘cut off a small piece’\\
cènkùrà \tab ‘cut off half’\\
cérùrà \tab ‘tear’\\
kóshòrà \tab ‘cut/pull off’\\
kúkùrà \tab ‘cut nails; cut off sides of a grass mat to make it even’\\
kùrùrà \tab ‘cut hair’\\
ŋàtùrà \tab ‘tear’\\
ngwénjùrà \tab ‘slash grass (in order to clear a piece of land)’\\
ⁿǀàmbùkà \tab ‘burst (of a mukusi pod)’\\
pwàcùrà \tab ‘break’\\
rùkùrùrà \tab ‘divorce’\\
tùmbùrà \tab ‘cut and gut a fish’\\
túrùrà \tab ‘pierce’\\
ǀàpùrà \tab ‘tear’\\
ǀàpùtùrà \tab ‘tear’\\
\z

Verbs referring to various acts of removing also often take a separative suffix, as in (\ref{bkm:Ref99030227}). These, too, often lack an underived counterpart.

\ea
\label{bkm:Ref99030227}
còkòrà \tab ‘remove skins of maize’\\
dùnkùrà \tab ‘thresh’\\
kúngùrà \tab ‘clean up after a meal’\\
nyùkùrà \tab ‘uproot’\\
ⁿǀòngòmònà \tab ‘hollow out’\\
ⁿǀòndòrà \tab ‘take out a fingerful of something’\\
shàrùrà \tab ‘pick out, e.g. rotten groundnuts’\\
tòmpòrà \tab ‘uproot’\\
tùmpùrà \tab ‘take a piece of meat from a boiling pot’\\
zùbùrà \tab ‘take a bit of food from a boiling pot’\\
zyángùrà \tab ‘harvest’\\
ǀòpòrà \tab ‘take out flesh, an eye’\\
\z

\section{Impositive}
\label{bkm:Ref450835510}\hypertarget{Toc75352663}{}
Fwe has an impositive suffix \textit{-am} (intransitive) and \textit{-ik} (transitive), which give the meaning of assuming or putting in a certain position. The transitive impositive \textit{-ik} displays vowel harmony, with an allomorph \textit{-ek} used after stems with a mid-vowel (see \sectref{bkm:Ref451863900} on vowel harmony). Examples of the use of the impositive derivation are given in (\ref{bkm:Ref492055690}).

\ea
\label{bkm:Ref492055690}
  cànkàmà  ‘stand on the fire (of a pot)’\\
cànkìkà  ‘put (a pot) on the fire’
\z

There are two verbs where the transitive impositive suffix \textit{-ik} influences the verb’s final root consonant: the verb \textit{háng-am} / \textit{hánj-ik} ‘hang (tr./intr.)’, where the root-final plosive /ng/ changes to an affricate /nj/, and the verb \textit{dank-am} / \textit{dans-ik} ‘be dropped/ drop’, where the root-final plosive /nk/ changes to a fricative /ns/. In all other cases, the suffix \textit{-ik} does not cause changes to the last consonant of the verb root, as in (\ref{bkm:Ref492055690}).

When the intransitive impositive \textit{-am} is combined with the separative \textit{-un/-uk}, the vowel /a/ of the suffix \textit{-am} changes to /u/ under influence of the following vowel /u/, as in (\ref{bkm:Ref99090924}). No other suffixes are attested whose vowel assimilates to that of the following separative suffix, nor are there any other cases where regressive vowel harmony takes place. As (\ref{bkm:Ref486334960}) shows, vowel harmony with the mid back vowel of the stem is maintained, showing that the assimilation of \textit{-am} to \textit{-um} precedes the rule of vowel harmony that lowers /u/ to /o/, e.g. /\textit{kot-am-un}/ > /\textit{kot-um-un}/ > /\textit{kot-om-on}/.

\ea
\label{bkm:Ref99090924}
\ea
kùhángàmà \\
ku-háng-am-a\\
\textsc{inf}-climb-\textsc{imp}.\textsc{intr}-\textsc{fv}\\
\glt ‘to climb’   

\ex
kùhángùmùkà\\
ku-hang-am-uk-a\\
\textsc{inf}-climb-\textsc{imp}.\textsc{intr}-\textsc{sep}.\textsc{intr}-\textsc{fv}\\
‘to fall down’
\z\z

\ea
\label{bkm:Ref486334960}
\ea
kùkòtàmà\\
ku-kot-am-a\\
\textsc{inf}-bend-\textsc{imp}.\textsc{intr}-\textsc{fv}\\
\glt ‘to bow the head’   

\ex
kùkòtòmònà\\
ku-kot-am-un-a\\
\textsc{inf}-bend-\textsc{imp}.\textsc{intr}-\textsc{sep}.\textsc{intr}-\textsc{fv}\\
‘to hold up someone’s head’
\z\z

As \tabref{tab:6:7} shows, any verb that can occur with either the transitive or the intransitive impositive suffix may also occur with the other suffix.

\begin{table}
\label{bkm:Ref98841609}\caption{\label{tab:6:7}Transitive and intransitive impositive verbs}
\begin{tabularx}{\textwidth}{lQlQ}
\lsptoprule
\multicolumn{2}{l}{Transitive impositive \textit{-ik/-ek}} & \multicolumn{2}{l}{Intransitive impositive \textit{-am}}\\
\midrule
{\itshape dàbbìkà} & ‘throw into water’ & {\itshape dàbbàmà} & ‘jump into water’\\
{\itshape hánjìkà} & ‘hang, put in a high position’ & {\itshape hángàmà} & ‘be put in a hanging/high position’\\
{\itshape kúnìkà} & ‘put on a smoking shelve’ & {\itshape kúnàmà} & ‘be put on a smoking shelve’\\
{\itshape nyòngèkà} & ‘bend (sideways)’ & {\itshape nyòngàmà} & ‘become bent (sideways)’\\
\lspbottomrule
\end{tabularx}
\end{table}

Some verb roots that take the impositive transitive suffix do not occur with the impositive intransitive suffix \textit{-am}, but rather with the extensive suffix \textit{-ar/-an} (see also \sectref{bkm:Ref486253151}), or with the separative suffix (see also \sectref{bkm:Ref485823385}), as in \tabref{tab:6:8}.

\begin{table}
\label{bkm:Ref99091008}\caption{\label{tab:6:8}Impositive verbs from extensive / separative verbs}

\begin{tabularx}{\textwidth}{lQlQ}
\lsptoprule
\multicolumn{2}{l}{Transitive impositive \textit{-ik/-ek}} & \multicolumn{2}{l}{Extensive / separative}\\
\midrule
{\itshape rémèkà} & ‘injure’ & {\itshape rémànà} & ‘get injured’\\
{\itshape súmbìkà} & ‘impregnate’ & {\itshape súmbàrà} & ‘become pregnant’\\
{\itshape tándàbìkà} & ‘stretch (someone’s) legs’ & {\itshape tándàbàrà} & ‘stretch (one’s own) legs’\\
{\itshape zỳabìkà} & ‘dress (someone)’ & {\itshape zyàbàrà} & ‘dress (oneself)’\\
{\itshape zyímìkà} & ‘put in a standing position’ & {\itshape zyímànà} & ‘stand up’\\
{\itshape cànkìkà} & ‘put on the fire’ & {\itshape cànkùrà} & ‘remove from the fire’\\
{\itshape fùrùmìkà} & ‘put upside down’ & {\itshape fúrùmùnà} & ‘put upright’\\
{\itshape hánjìkà} & ‘hang, put in a high position’ & {\itshape hángùrà} & ‘remove from a high/hanging position’\\
{\itshape kámbìkà} & ‘stack, put on top of each other’ & {\itshape kámbùrà} & ‘remove from on top of each other’\\
{\itshape shémpèkà} & ‘shoulder a load’ & {\itshape shémpùrà} & ‘go with a load on one’s shoulders’\\
\lspbottomrule
\end{tabularx}
\end{table}

The impositive suffix \textit{-am/-ik} may be used to derive an impositive verb from an adjective or an ideophone, as in (\ref{bkm:Ref492056311}).

\ea
\label{bkm:Ref492056311}
  fwîyì      ‘short; close (by)’\\
kù-fú-àm-à    ‘to approach’\\
kù-fwí-ìk-à    ‘to bring closer’
\z

\ea
  túmpwì    ideophone of falling in water\\
kù-tùmpw-àm-à  ‘to fall in water’\\
kù-tùmpw-ìk-à  ‘to throw into water’
\z

The impositive suffix \textit{-am/-ik} adds the meaning of putting or being put in a certain position. In (\ref{bkm:Ref452481814}), the verb \textit{bomb} ‘become wet’ is used with the transitive impositive to describe putting something in water.

\ea
\label{bkm:Ref452481814}
ndàbòmbékì zìzyàbàrò\\
\gll ndi-a-bomb-é̲k-i          zi-zyabaro\\
\textsc{sm}\textsubscript{1SG}-\textsc{pst}-become\_wet-\textsc{imp}.\textsc{tr}-\textsc{npst}.\textsc{pfv}  \textsc{np}\textsubscript{8}-cloth\\
\glt ‘I’ve put the clothes in water.’ (NF\_Elic15)
\z

In (\ref{bkm:Ref445470694}), the intransitive impositive verb \textit{hángam} ‘become high, be put in a high position’, is used metaphorically; the speaker is making the claim that life has become too high, referring to the increasing complexity of the modern world and the skills needed to succeed in it.

\newpage
\ea
\label{bkm:Ref445470694}
òbùhárò shàbùhángámìtè\\
\gll o-bu-háro    sha-bu-ha\textsubscript{H}ng-á̲m-ite\\
\textsc{aug}-\textsc{np}\textsubscript{14}-life  \textsc{inc}-\textsc{sm}\textsubscript{14}-become\_high-\textsc{imp}.\textsc{intr}-\textsc{stat}\\
\glt ‘Life has become too demanding (lit. ‘high’).’ (ZF\_Conv13)
\z

The intransitive impositive suffix -\textit{am} creates a change-of-state verb, e.g. to assume, or to be put, in a certain position. As is typical of change-of-state verbs, intransitive impositive verbs have a hypothetical interpretation in the present construction (\ref{bkm:Ref490575324}), and a present state interpretation when combined with a stative (\ref{bkm:Ref490575325}). The combination of the intransitive impositive suffix with the stative suffix results in a number of allomorphs, which are discussed in \sectref{bkm:Ref431984198} on the stative suffix.

\ea
\label{bkm:Ref490575324}
\glll mùkàmbámà\\
mu-ka\textsubscript{H}mb-am-á̲\\
\textsc{sm}\textsubscript{2PL}-ascend-\textsc{imp}.\textsc{intr}-\textsc{fv}\\
\glt ‘[if you do like that] You’d ascend.’ (NF\_Elic15)
\z

\ea
\label{bkm:Ref490575325}
\glll cìhàngámìtè\\
ci-ha\textsubscript{H}ng-ám-ite\\
\textsc{sm}\textsubscript{7}-hang-\textsc{imp}.\textsc{intr}-\textsc{stat}\\
\glt ‘It hangs.’ (NF\_Elic17)
\z

The intransitive impositive -\textit{am} refers to assuming a position without expressing an agent that caused this position, as in (\ref{bkm:Ref445469923}). The transitive impositive \textit{-ik/-ek}, however, requires the expression of both the agent and the patient, as in (\ref{bkm:Ref445469935}).

\ea
\label{bkm:Ref445469923}
\glll zìkúnì\\
zi-ku\textsubscript{H}n-í̲\\
\textsc{sm}\textsubscript{10}-smoke-\textsc{imp}.\textsc{intr}.\textsc{stat}\\
\glt ‘They [the fish] are on the smoking shelve.’
\z

\ea
\label{bkm:Ref445469935}
níndàkúnꜝík’ énswì\\
gll\ ní̲-ndi-a-kún-ik-á        e-N-swí\\
\textsc{rem}-\textsc{sm}\textsubscript{1SG}-\textsc{pst}-smoke-\textsc{imp}.\textsc{tr}-\textsc{fv}  \textsc{aug}-\textsc{np}\textsubscript{10}-fish\\
\glt ‘I’ve put the fish on a smoking shelve.’ (NF\_Elic15)
\z
\section{Pluractional}
\label{bkm:Ref489866362}\hypertarget{Toc75352664}{}
Fwe has two derivational strategies that express a pluractional, an event that is in some way repeated. Event repetition can be interpreted in many different ways; events may be interpreted as repeated on a single occasion, or on multiple occasions, or on different locations. Event repetition may also be interpreted as plurality of arguments.

\hspace*{-1.4pt}In Fwe, pluractionality is expressed by reduplication, a cross-linguistically common strategy for pluractional marking \citep[13-15]{Inkelas2014}, or by a derivational suffix \textit{-a}. Both pluractional markers display a similar range of pluractional meanings, and are therefore treated together in this section. They differ in their connotations of intensity: the pluractional suffix \textit{-a} is associated with a high degree of intensity or completeness, and the pluractional marked by stem reduplication marks a low degree of intensity, and may also express negative connotations.

\largerpage
\subsection{Pluractional 1: completeness}
\label{bkm:Ref70945678}\hypertarget{Toc75352665}{}
The derivational suffix \textit{-a} marks a pluractional with overtones of intensity or completeness. This pluractional is glossed as \textsc{pl}1. The pluractional suffix can be realized as \textit{-a} or \textit{-ah}; the epenthetic [h] is part of a regular process of [h] epenthesis to break up vowel clusters (see \sectref{bkm:Ref491962181}). Unlike other derivational suffixes, the pluractional suffix \textit{-a} is always followed by another derivational suffix. The only derivational suffixes with which the pluractional may be used are the separative \textit{-ur/-uk}, the transitive impositive \textit{-ik}, the applicative \nobreakdash-\textit{ir}, or a combination of the separative and the applicative, as in (\ref{bkm:Ref99091275}).

\ea
\label{bkm:Ref99091275}
Pluractional verbs
\ea
Pluractional -a with separative -ur/-uk\\
dàmàùrà \tab ‘beat up, beat to a pulp’\\
cènkàùkà \tab ‘look over both shoulders’\\
céràùrà \tab ‘keep on tearing’\\
ⁿǀùmàùnà \tab ‘uproot’\\
pwàcàùkà \tab ‘break (intr.) (of multiple objects)’\\
túkàùra \tab ‘insult (multiple people)’\\
\ex Pluractional \-a with transitive impositive -ik\\
dánsàìkà \tab ‘scatter’\\
hánjàìkà \tab ‘hang up (multiple objects)’\\
sóndàìkà \tab ‘point (at multiple objects)’\\
ùràìkà \tab ‘name (multiple people)’\\
\ex Pluractional -a with applicative -ir/-in\\
shónjàìrà \tab ‘throw (multiple times)’\\
sòsàìrà \tab ‘keep on poking a fire’\\
shúmàìnà \tab ‘tie (multiple knots)’\\
shwátàìrà \tab ‘keep on whipping’\\
\ex Pluractional -a with separative -ur and applicative -ir\\
ᵍǀánàwìnà \tab ‘divide (food) among’\\
shónjàwìrà \tab ‘throw (repeatedly) to’\\
hìndàwìrà \tab ‘keep taking for’\\
zyónàwìrà \tab ‘destroy for’\\
\z\z


Any verb that can be used with the separative derivation, may take the pluractional suffix \textit{-a}. Which impositive or applicative verbs can take the pluractional suffix is lexically determined. For the applicative, the pluractional \textit{-a} can be used with verbs that use the applicative as a productive suffix, as in (\ref{bkm:Ref75176004}), but also with many verbs that have a lexicalized applicative suffix, as in (\ref{bkm:Ref489450421}--\ref{bkm:Ref489450422}).

\ea
\label{bkm:Ref75176004}
kùzyáːkàìrà\\
ku-zyáːk-a-ir-a\\
\textsc{inf}-build-\textsc{pl}1-\textsc{appl}-\textsc{fv}\\
\glt ‘to build for (multiple people)’
cf. kùzyáːkìrà ‘to build for’, kùzyâːkà ‘to build’
\z

\ea
\label{bkm:Ref489450421}
kùshwátàìrà\\
ku-shwát-a-ir-a\\
\textsc{inf}-whip-\textsc{pl}1-\textsc{appl}-\textsc{fv}\\
\glt ‘to keep on whipping’
cf. kùshwátìrà ‘to whip’; *kùshwâtà
\z

\ea
\label{bkm:Ref489450422}
kùshúmàìnà\\
ku-shúm-a-in-a\\
\textsc{inf}-tie-\textsc{pl}1-\textsc{appl}-\textsc{fv}\\
\glt ‘to tie’
 cf. kùshúmìnà ‘to tie’; kùshûmà ‘to bite’
\z

Some verbs combining the pluractional with the separative also exist as separative verbs without a pluractional; some exist as underived verbs, but not as separative verbs; and some are only attested as pluractionals, not as separative or underived verbs. Examples of all three types are given in \tabref{tab:6:9}.

\begin{table}
\label{bkm:Ref489452423}\caption{\label{tab:6:9}The combination of the pluractional and separative suffixes}
\begin{tabularx}{\textwidth}{Qll}
\lsptoprule
Pluractional & Separative & Underived\\
\midrule
\textit{bbátàùrà} ‘divide (into more than two)’ & \textit{bbátùrà} ‘divide (into two)’ & -\\
\textit{ᵍǀàndàùkà} ‘disperse’ & \textit{ᵍǀàndùkà} ‘disperse’ & -\\
\textit{táràùkà} ‘go step by step’ & \textit{tárùkà} ‘take a step’ & -\\
\textit{shótàùkà} ‘jump up and down’ & \textit{shótòkà} ‘jump’ & -\\
\textit{dàmàùrà} ‘beat up’ & - & \textit{dàmà} ‘beat’\\
\textit{ᵍǀóntàùrà} ‘drip continuously’ & - & \textit{ᵍǀôntà} ‘drip’\\
\textit{hàràùkà} ‘be scratched all over’ & - & \textit{hàrà} ‘scratch’\\
\textit{yèndàùrà} ‘walk around’ & - & \textit{yèndà} ‘walk, go’\\
\textit{kózyàùrà} ‘pick (fruit)’ & - & -\\
\textit{shángàùrà} ‘contribute’ & - & -\\
\textit{tángàùrà} ‘provoke’ & - & -\\
\textit{zùkàùrà} ‘stir’ & - & -\\
\lspbottomrule
\end{tabularx}
\end{table}

Most pluractional separatives that do not occur without the pluractional lack separative semantics. The separative expresses “movement out of an original position”, and many separative verbs express destruction or removal (see \sectref{bkm:Ref485823385} on the separative). Pluractional separative verbs that have no separative form without a pluractional, do not fit this semantic characterization, as the examples in (\ref{bkm:Ref98512796}) show.

\newpage
\ea
\label{bkm:Ref98512796}
Pluractional/separative verbs that lack separative semantics
\ea
àmbàùrà    ‘discuss’\\
\ex
kwátàùrà    ‘touch all over’\\
\ex
yèndàùrà    ‘walk around’\\
\ex
zùkàùrà    ‘stir’
\z\z

The transitive separative is subject to nasal harmony (see \sectref{bkm:Ref485823385}), and in some verbs that combine the pluractional with the separative, nasal harmony is maintained, e.g. \textit{ⁿǀùmàùnà} ‘uproot’, \textit{ᵍ}\textit{ǀànàùnà} ‘divide (food)’, \textit{càmàùnà} ‘divide (food)’. In others, nasal harmony is not maintained and the separative is realized with /r/ instead, e.g. \textit{dàmàùrà} ‘beat up’, \textit{zyónàùrà} ‘destroy’.

The pluractional suffix \textit{-a} resembles the initial syllable of the neuter suffix \textit{-ahar}, but this is a chance resemblance, and the neuter is not a combination of a pluractional \textit{-a} plus a suffix \textit{-har}. The pluractional and the neuter are semantically very different, and the neuter suffix is likely to have been borrowed in its entirety from Lozi \textit{-ahal} (see \sectref{bkm:Ref489452510} on the neuter).

Unlike most other derivational suffixes, the pluractional \textit{-a} does not influence valency. Most pluractional verbs take their valency from the derivational suffix following the pluractional suffix, namely transitive with the transitive impositive \textit{-ik}, the applicative \textit{-ir}, and the transitive separative \textit{-ur}, and intransitive with the intransitive separative \nobreakdash-\textit{uk}. Some intransitive verbs, however, take the transitive separative \textit{-ur} rather than the intransitive separative \textit{-uk}, e.g. \textit{yàkàùrà} ‘writhe’, \textit{yèndàùrà} ‘walk around’.

The core function of pluractional \textit{-a} is to indicate that an action happens more than once, which can manifest itself in different ways: in a repetition of the action, or in an action involving multiple participants (either agents, patients, or recipients), as in (\ref{bkm:Ref492056909}).
\NumTabs{2}
\TabPositions{.45\textwidth, .5\textwidth}
\ea
\label{bkm:Ref492056909}
Without pluractional \tab With pluractional\\
bbátùrà ‘divide into two’ \tab bbátàùrà ‘divide into more than two’\\
cènkùkà ‘look over one’s shoulder’ \tab cènkàùkà ‘look over both shoulders’\\
jùntà ‘hop’ \tab jùntàùkà ‘hop repeatedly’\\
nyàkùrà ‘kick, stretch a limb’ \tab nyàkàùrà ‘writhe’\\
\z

When used to express multiple participants, intransitives express plurality of subjects, as in (\ref{bkm:Ref75176572}--\ref{bkm:Ref75176574}), transitives express plurality of patients, as in (\ref{bkm:Ref75176639}--\ref{bkm:Ref75176640}), and ditransitives express plurality of indirect objects, as in (\ref{bkm:Ref490492423}--\ref{bkm:Ref75176679}). This syntactic alginment is typical for pluractional verbs (see, e.g. \citealt{StorchColy2017} and other papers in the same volume).

\ea
\label{bkm:Ref75176572}
màténdè àcóːkáùkìtè\\
\gll ma-ténde  a-coːk-á̲-uk-ite\\
\textsc{np}\textsubscript{6}-leg  \textsc{sm}\textsubscript{6}-break-\textsc{pl}1-\textsc{sep}.\textsc{intr}-\textsc{stat}\\
\glt ‘His legs are broken.’
\z

\ea
\label{bkm:Ref75176574}
èmpótó zàpwácáùkì\\
\gll e-N-potó    zi-a-pwac-á̲-uk-i\\
\textsc{aug}-\textsc{np}\textsubscript{10}-pot  \textsc{sm}\-\textsubscript{10}-\textsc{pst}-break-\textsc{pl}1-\textsc{sep}.\textsc{intr}-\textsc{npst}.\textsc{pfv}\\
\glt ‘The pots are broken
\z

\ea
\label{bkm:Ref75176639}
mùbòné bèná bàntù bàkwèsì bàdàbbàìká bàntù múmênjì\\
\gll mu-bo\textsubscript{H}n-é̲    bená    ba-ntu ba-kwesi  ba-dabb-a-ik-á̲    ba-ntu  mú-ma-ínji\\
\textsc{sm}\textsubscript{2PL}-see-\textsc{pfv}.\textsc{sbjv}  \textsc{dem}.\textsc{iv}\textsubscript{2}  \textsc{np}\textsubscript{2}-person
\textsc{sm}\textsubscript{2}-\textsc{prog}  \textsc{sm}\textsubscript{2}-throw-\textsc{pl}1-\textsc{imp}.\textsc{tr}-\textsc{fv}  \textsc{np}\textsubscript{2}-person  \textsc{np}\textsubscript{18}-\textsc{np}\textsubscript{6}-water\\
\glt ‘Can you see those people? They are throwing people into the water.’ (NF\_Elic17)
\z

\ea
\label{bkm:Ref75176640}
òshùmàìné màkôtò\\
\gll o-shu\textsubscript{H}m-a-in-é̲      ma-kóto\\
\textsc{sm}\textsubscript{2SG}-tie-\textsc{pl}1-\textsc{appl}-\textsc{pfv}.\textsc{sbjv}  \textsc{np}\textsubscript{6}-knot\\
\glt ‘Tie knots.’ (NF\_Elic15)
\z

\ea
\label{bkm:Ref490492423}
àyàbúzyàːkàìrá bàntù\\
\gll a-yabú-zyaːk-a-ir-á      ba-ntu\\
\textsc{sm}\textsubscript{1}-\textsc{loc}.\textsc{pl}-build-\textsc{pl}1-\textsc{appl}-\textsc{fv}  \textsc{np}\textsubscript{2}-person\\
\glt ‘S/he is going around building for people.’
\z

\ea
\label{bkm:Ref75176679}
àkwèsì àbàhàmbàìká èntàbà\\
\gll a-kwesi  a-ba\textsubscript{H}-ha\textsubscript{H}mb-a-ik-á̲    e-N-taba\\
\textsc{sm}\textsubscript{1}-\textsc{prog}  \textsc{sm}\textsubscript{1}-\textsc{om}\textsubscript{2}-accuse-\textsc{pl}1-\textsc{imp}.\textsc{tr}-\textsc{fv}  \textsc{aug}-\textsc{np}\textsubscript{10}-case\\
\glt ‘S/he is accusing them of many things.’ (NF\_Elic17)
\z

Most pluractional verbs are ambiguous between a repeated event reading and a multiple participant reading. The pluractional verb \textit{pwàcàùkà} ‘break’, has a multiple participant reading when used with a plural subject in (\ref{bkm:Ref75176894}), and a repeated event reading with a singular subject in (\ref{bkm:Ref75176896}).

\ea
\label{bkm:Ref75176894}
èmpótó zàpwácáùkì\\
\gll e-N-potó    zi-a-pwac-á̲-uk-i\\
\textsc{aug}-\textsc{np}\textsubscript{10}-pot  \textsc{sm}\textsubscript{10}-\textsc{pst}-break-\textsc{pl}1-\textsc{sep}.\textsc{intr}-\textsc{npst}.\textsc{pfv}\\
\glt ‘The pots are broken.’
\z

\ea
\label{bkm:Ref75176896}
èmpótó yàpwácáùkì\\
\gll e-N-potó  i-a-pwac-á̲-uk-i\\
\textsc{aug}-\textsc{np}\textsubscript{9}-pot  \textsc{sm}\textsubscript{9}-\textsc{pst}-break-\textsc{pl}1-\textsc{sep}.\textsc{intr}-\textsc{npst}.\textsc{pfv}\\
\glt ‘The pot is broken in many places (after someone hit it repeatedly).’ (NF\_Elic17)
\z

Other verbs only allow a multiple participant reading, as shown in (\ref{bkm:Ref99091816}) with the transitive pluractional \textit{shúmàìnà} ‘tie (multiple objects)’, which requires a plural object, and is ungrammatical with a singular object.

\ea
\label{bkm:Ref99091816}
\ea
kùshúmàìnà màkôtò\\
\gll ku-shúm-a-in-a    ma-kóto\\
\textsc{inf}-tie-\textsc{pl}1-\textsc{appl}-\textsc{fv}    \textsc{np}\textsubscript{6}-knot\\
\glt ‘to tie knots’

\ex
  *kùshúmàìnà kôtò\\
Intended: ‘to tie a knot (repeatedly)’ (NF\_Elic17)
\z\z

The inverse is also possible, where a plural argument requires the use of the pluractional, and the absence of the pluractional suffix is ungrammatical, as in (\ref{bkm:Ref489624868}).

\ea
\label{bkm:Ref489624868}
\ea
màténdè àcóːkáùkìtè\\
\gll ma-ténde  a-coːk-á̲-uk-ite\\
\textsc{np}\textsubscript{6}-leg  \textsc{sm}\textsubscript{6}-break-\textsc{pl}1-\textsc{sep}.\textsc{intr}-\textsc{stat}\\
\glt ‘His legs are broken.’

\ex
  *màténdè àcóːkêtè\\
Intended: ‘His legs are broken.’ (NF\_Elic17)
\z\z

More research is needed to study what conditions the availability of the repeated event reading and the multiple participant reading, and under which conditions a plural participant requires a pluractional verb.

Pluractional \-\textit{-a} can combine with the locative pluractional marker \textit{kabú}-/\textit{yabú}- (see \sectref{bkm:Ref494442567}) to indicate an event that is repeated in different locations, as in (\ref{bkm:Ref99091944}--\ref{bkm:Ref490492398}).

\ea
\label{bkm:Ref99091944}
\glll ndìkàbúbàsùndàíkà\\
ndi-kabú-ba\textsubscript{H}-sund-a-ik-á̲\\
\textsc{sm}\textsubscript{1SG}-\textsc{loc}.\textsc{pl}-\textsc{om}\textsubscript{2}-point-\textsc{pl}1-\textsc{imp}.\textsc{tr}-\textsc{fv}\\
\glt ‘I am going around pointing at them.’
\z

\ea
\label{bkm:Ref490492398}
kùshúmàìnà áꜝkábúshùmàìnà màkôtò\\
\gll ku-shúm-a-in-a    á̲-kabú-shum-a-in-a    ma-kóto\\
\textsc{inf}-tie-\textsc{pl}1-\textsc{appl}-\textsc{fv}  \textsc{sm}\textsubscript{1}.\textsc{rel}-\textsc{loc}.\textsc{pl}-tie-\textsc{pl}1-\textsc{appl}-\textsc{fv}  \textsc{np}\textsubscript{6}-knot\\
\glt ‘S/he is going around tying knots./ S/he is tying knots in different places.’ (NF\_Elic17)
\z

The pluractional \textit{-a} often implies that an action is completed. This is an extension of its pluractional meaning, and not part of its basic meaning, as illustrated in (\ref{bkm:Ref489523341}--\ref{bkm:Ref74915410}), which discuss a three-legged cooking pot. When used without further qualifying information, the use of a pluractional implies that all the legs of the pot are broken, as in (\ref{bkm:Ref489523341}). This implicature can be canceled, however, as in (\ref{bkm:Ref74915410}), which uses the pluractional \textit{-a} describing that two of the pot’s legs are broken.

\ea
\label{bkm:Ref489523341}
míndì yéꜝmpótò yàcóːkáùkì\\
\gll mi-índi  i-é=N-potó    i-a-cóːk-a-uk-i\\
\textsc{np}\textsubscript{4}-leg  \textsc{pp}\textsubscript{4}-\textsc{con}=\textsc{np}\textsubscript{9}-pot  \textsc{sm}\textsubscript{4}-\textsc{pst}-break-\textsc{pl}1-\textsc{sep}.\textsc{intr}-\textsc{npst}.\textsc{pfv}\\
\glt ‘The legs of the pot are (all) broken.’
\z

\ea
\label{bkm:Ref74915410}
míndì yòbírè yéꜝmpótò yàcóːkáùkì\\
\gll mi-índi  i-o=biré    i-é=N-potó i-a-có̲ːk-a-uk-i\\
\textsc{np}\textsubscript{4}-leg  \textsc{pp}\textsubscript{4}-\textsc{con}=two  \textsc{pp}\textsubscript{4}-\textsc{con}=\textsc{np}\textsubscript{9}-pot
\textsc{sm}\textsubscript{4}-\textsc{pst}-break-\textsc{pl}1-\textsc{sep}.\textsc{intr}-\textsc{npst}.\textsc{pfv}\\
\glt ‘Two legs of the pot are broken.’ (NF\_Elic17)
\z

Furthermore, the pluractional may only imply completeness when repeated action is also involved, as in (\ref{bkm:Ref489524458})- (\ref{bkm:Ref74915858}), which discuss a window that was destroyed by a stone. (\ref{bkm:Ref489524458}) describes a single window pane that was destroyed by a stone; although the window is completely broken, the pluractional cannot be used as it only concerns a single window. In (\ref{bkm:Ref74915858}), the pluractional is allowed as it concerns a window consisting of multiple broken window panes.

\ea
\label{bkm:Ref489524458}
\glll ryàpwácûkì\\
ri-a-pwac-ú̲k-i\\
\textsc{sm}\textsubscript{5}-\textsc{pst}-break-\textsc{sep}.\textsc{intr}-\textsc{npst}.\textsc{pfv}\\
\glt ‘It broke.’
\z

\ea
\label{bkm:Ref74915858}
\glll ryàpwácáùkì\\
ri-a-pwac-á̲-uk-i\\
\textsc{sm}\textsubscript{5}-\textsc{pst}-break-\textsc{pl}1-\textsc{sep}.\textsc{intr}-\textsc{npst}.\textsc{pfv}\\
\glt ‘It broke (in different places).’
\z

The pluractional marked with \textit{-a} can combine with the pluractional marked with reduplication, as in (\ref{bkm:Ref99092026}--\ref{bkm:Ref99092027}). Although there are semantic differences between the two pluractional strategies, a difference in meaning between using either pluractional strategy and using both pluractional strategies on the same verb has not yet been observed.

\ea
\label{bkm:Ref99092026}
nàkàyâ ìyé àkábúyèndàùràyèndàùrà òkábúbônà\\
\gll na=ka-y-á̲ iyé  a-kabú-endaura-end-a-ur-a    o-kabú-bón-a \\
\textsc{com}=\textsc{inf}.\textsc{dist}-go-\textsc{fv}
that  \textsc{sm}\textsubscript{1}-\textsc{loc}.\textsc{pl}-\textsc{pl}2-go-\textsc{pl}1-\textsc{sep}.\textsc{tr}-\textsc{fv}  \textsc{aug}-\textsc{loc}.\textsc{pl}-see-\textsc{fv}\\
\glt ‘And he went out to walk around, and look around.’ (NF\_Narr17)
\z

\ea
\label{bkm:Ref99092027}
\glll àbàzìmbàùkàzìmbàúkà\\
a-ba\textsubscript{H}-zi\textsubscript{H}mbauka-zimb-a-uk-á̲\\
\textsc{sm}\textsubscript{1}-\textsc{om}\textsubscript{2}-\textsc{pl}2-go\_around-\textsc{pl}1-\textsc{sep}.\textsc{intr}-\textsc{fv} \\
\glt ‘She is avoiding them.’ (NF\_Narr15)
\z
\subsection{Pluractional 2: low intensity}
\hypertarget{Toc75352666}{}\label{bkm:Ref490839448}
The second pluractional strategy used in Fwe is reduplication of the verb stem, glossed as \textsc{pl}2. Examples are given in (\ref{bkm:Ref492118879}).

\ea
\label{bkm:Ref492118879}
àmbà ‘talk’ \tab àmbààmbà ‘talk a lot’\\
dàmà ‘beat’ \tab dàmàdàmà ‘beat repeatedly’\\
kwâtà ‘touch’ \tab kwátàkwàtà ‘touch everywhere’\\
shèkà ‘laugh’ \tab shèkàshèkà ‘laugh a lot’\\
\z

Reduplication is very productive, and appears to be accepted with any verb stem. Most reduplicated verbs also occur in their underived form; a number of exceptions are noted in \tabref{tab:6:10}. In other cases, reduplicated verbs are also attested in their underived form, but the reduplicated meaning appears to be lexicalized.

\begin{table}
\label{bkm:Ref488767412}\caption{\label{tab:6:10}Lexicalized reduplicated verbs}

\begin{tabularx}{\textwidth}{lQlQ}
\lsptoprule
\multicolumn{2}{l}{Reduplicated verb} & \multicolumn{2}{l}{Underived base verb}\\
\midrule
{\itshape gábàgàbà} & ‘talk nonsense’ & \multicolumn{2}{c}{-}\\
{\itshape rúngàrùngà} & ‘disturb (with noise)’ & \multicolumn{2}{c}{\textit{-}}\\
{\itshape shángàshàngà} & ‘contribute (money)’ & \multicolumn{2}{c}{\textit{-}}\\
{\itshape cábàcàbà} & ‘fish by scooping with a bucket (lexicalized meaning); collect (productive meaning)’ & \textit{câbà} & ‘fetch, collect (firewood)’\\
{\itshape shàkàshàkà} & ‘look for’ & {\itshape shàkà} & ‘want, need’\\
\lspbottomrule
\end{tabularx}
\end{table}

\begin{sloppypar}
Reduplication targets the entire verb stem, including derivational suffixes, such as the applicative \textit{-ir} in (\ref{bkm:Ref489613261}) and the causative \textit{-es} in (\ref{bkm:Ref75177525}), and inflectional suffixes, such as the subjunctive suffix -\textit{e} in (\ref{bkm:Ref489613302}) and the past suffix \textit{-i} in (\ref{bkm:Ref75177560}). Any inflectional prefixes, however, are not maintained when the verb stem is reduplicated. This is also the case for the object marker, which is not reduplicated, as seen in (\ref{bkm:Ref489613303}).
\end{sloppypar}

\ea
\label{bkm:Ref489613261}
\glll kùríhìndìràhìndìrà\\
ku-rí-hindira-hind-ir-a\\
\textsc{inf}-\textsc{refl}-\textsc{pl}2-take-\textsc{appl}-\textsc{fv}\\
\glt ‘to keep taking from’ (NF\_Elic15)
\z

\ea
\label{bkm:Ref75177525}
mùrìgórésègòrèsè bùryáhò\\
\gll mu-ri\textsubscript{H}-goré̲se-gor-es-e          bu-ryáho\\
\textsc{sm}\textsubscript{2PL}-\textsc{refl}-\textsc{pl}2-become\_strong-\textsc{caus}-\textsc{pfv}.\textsc{sbjv}  \textsc{np}\textsubscript{14}-like\_that\\
\glt ‘Just be strong.’ (NF\_Elic17)
\z

\ea
\label{bkm:Ref489613302}
\glll mbòndíshàkèshákè\\
mbo-ndí̲-shake-shak-é̲\\
\textsc{near}.\textsc{fut}-\textsc{sm}\textsubscript{1SG}-\textsc{pl}2-search-\textsc{pfv}.\textsc{sbjv}\\
\glt ‘I will search.’
\z

\ea
\label{bkm:Ref75177560}
\glll ndànyùngínyùngì\\
ndi-a-nyungí̲-nyung-i\\
\textsc{sm}\textsubscript{1SG}-\textsc{pst}-\textsc{pl}2-shake-\textsc{npst}.\textsc{pfv}\\
\glt ‘I have shaken.’
\z

\ea
\label{bkm:Ref489613303}
\glll ndàcíꜝnyúngínyùngì\\
ndi-a-cí-nyungí̲-nyung-i\\
\textsc{sm}\textsubscript{1SG}-\textsc{pst}-\textsc{om}\textsubscript{7}-\textsc{pl}2-shake-\textsc{npst}.\textsc{pfv}\\
\glt ‘I’ve shaken it.’ (NF\_Elic15)
\z

Although full stem reduplication, including derivational and inflectional suffixes, is the norm, there are certain exceptions. One concerns the negative suffix \textit{-i}. It is possible to negate reduplicated verbs with this suffix, as in (\ref{bkm:Ref490575423}), but many speakers are hesitant to produce such forms, and prefer to use an auxiliary \textit{aazyá} followed by the reduplicated verb in the infinitive form, as in (\ref{bkm:Ref97905746}). (See also chapter \ref{bkm:Ref97905708} on negation.)

\ea
\label{bkm:Ref490575423}
tàndìshàkíshàkì mwáꜝnángù\\
\gll ta-ndi-shakí̲-shak-i      mu-án-angú\\
\textsc{neg}-\textsc{sm}\textsubscript{1SG}-\textsc{pl}2-search-\textsc{neg}  \textsc{np}\textsubscript{1}-child-\textsc{poss}\textsubscript{1SG}\\
\glt ‘I am not looking for my child.’
\z

\ea
\label{bkm:Ref97905746}
ndààzyá kùshàkàshàkà mwáꜝnángù\\
\gll ndi-aazyá    ku-shaka-shak-a  mu-án-angú\\
\textsc{sm}\textsubscript{1SG}-be\_not    \textsc{inf}-\textsc{pl}2-search-\textsc{fv}  \textsc{np}\textsubscript{1}-child-\textsc{poss}\textsubscript{1SG}\\
\glt ‘I am not looking for my child.’ (ZF\_Elic14)
\z

The second exception to full stem reduplication is that suffixes are occasionally not reduplicated. An example where the applicative suffix may either be maintained or dropped in reduplication is given in (\ref{bkm:Ref489616860}). A similar example is given for the past suffix in (\ref{bkm:Ref489617670}): when the past suffix is dropped in the reduplication, the default final vowel \textit{-a} is used instead. Although these examples are limited, they show that the reduplicand is pre-posed, as the morphologically simplified form appears before the morphologically complete form. More research is needed to establish the behavior of suffixes in reduplication, and under what conditions suffixes can, must, or must not, be reduplicated.

\ea
\label{bkm:Ref489616860}
kùríhìndìràhìndìrà {\textasciitilde} kùríhìndàhìndìrà\\
\gll ku-rí-hindira-hind-ir-a {\textasciitilde} ku-rí-hinda-hind-ir-a\\
\textsc{inf}-\textsc{refl}-\textsc{pl}2-take-\textsc{appl}-\textsc{fv} \\
\glt ‘to keep taking for oneself’ (NF\_Elic17)
\z

\ea
\label{bkm:Ref489617670}
ndàyéndíyèndì {\textasciitilde} ndàyéndáyèndì\\
\gll ndi-a-endí̲-end-i {\textasciitilde} ndi-a-endá̲-end-i\\
\textsc{sm}\textsubscript{1SG}-\textsc{pst}-\textsc{pl}2-go-\textsc{npst}.\textsc{pfv}\\
\glt ‘I have traveled to many places.’ (NF\_Elic15)
\z

There are no limitations on the maximum number of syllables that can be reduplicated; (\ref{bkm:Ref489616139}) gives two examples of the reduplication of verb stems with four syllables.

\ea
\label{bkm:Ref489616139}
shàkùǀàrùmùnàǀàrùmùnà shòkùsónsònìsàsònsònìsà\\
\gll sha-ku-ǀarumuna-ǀarumun-a  sha-o-ku-sónsonisa-sonsonis-a\\
\textsc{inc}-\textsc{inf}-\textsc{pl}2-search-\textsc{fv}  \textsc{inc}-\textsc{aug}-\textsc{inf}-\textsc{pl}2-search-\textsc{fv}\\
\glt ‘They keep searching through my things, they keep searching carefully.’ (NF\_Song17))
\z

Tones are assigned after reduplication, and are not reduplicated themselves. This concerns both melodic tones, which are assigned by specific TAM constructions, and lexical tones, which are associated with the first syllable of the verb root\footnote{More research is needed to study the effect of reduplication on verbs with a floating high tone.}. That lexical tones are not reduplicated can be seen in the infinitive form in (\ref{bkm:Ref99092311}): the lexical high tone of the underived verb \textit{kwát} only surfaces on the root’s initial syllable, both in the simple and in the reduplicated form.

\ea
\label{bkm:Ref99092311}
\glll kùkwâtà\\
ku-kwát-a\\
\textsc{inf}-touch-\textsc{fv}\\
\glt ‘to touch’
\z

\ea
\glll kùkwátàkwàtà\\
ku-kwáta-kwat-a\\
\textsc{inf}-\textsc{pl}2-touch-\textsc{fv}\\
\glt ‘to touch everywhere’
\z

That melodic tones are not reduplicated can be seen in the near past perfective in (\ref{bkm:Ref99092338}), which has a melodic tone on the second syllable of the verb (melodic tone 3). When used with a reduplicated verb, the melodic tone is only assigned to the second syllable of the entire verb stem, not to the second syllable of both reduplicands.

\ea
\label{bkm:Ref99092338}
\glll ndànyùngínyùngì\\
ndi-a-nyungí̲-nyung-i\\
\textsc{sm}\textsubscript{1SG}-\textsc{pst}-\textsc{pl}2-shake-\textsc{npst}.\textsc{pfv}\\
\glt ‘I have shaken.’ (NF\_Elic15)
\z

Stem reduplication is used to express a pluractional, i.e. an action that takes place more than once. This may be an action repeated on a single occasion, as in (\ref{bkm:Ref75177959}--\ref{bkm:Ref75177960}), or on multiple occasions, as in (\ref{bkm:Ref75178010}--\ref{bkm:Ref75178011}).

\ea
\label{bkm:Ref75177959}
\glll ndàcíꜝnyúngínyùngì\\
ndi-a-cí-nyungí̲-nyung-i\\
\textsc{sm}\textsubscript{1SG}-\textsc{pst}-\textsc{om}\textsubscript{7}-\textsc{pl}2-shake-\textsc{pst} \\
\glt ‘I’ve shaken it.’
\z

\ea
\label{bkm:Ref75177960}
ndàkùrí kùyèndàyèndà há ndàkùàmbà héfònì\\
\gll ndi-aku-rí    ku-enda-end-a ha    ndí̲-aku-amb-a      ha-é-∅-foni\\
\textsc{sm}\textsubscript{1SG}-\textsc{npst}.\textsc{ipfv}-be  \textsc{inf}-\textsc{pl}2-go-\textsc{fv}
\textsc{dem}.\textsc{i}\textsubscript{16}  \textsc{sm}\textsubscript{1SG}.\textsc{rel}-\textsc{npst}.\textsc{ipfv}-talk-\textsc{fv}  \textsc{np}\textsubscript{16}-\textsc{aug}-\textsc{np}\textsubscript{5}-phone\\
\glt ‘I was walking back and forth while I was on the phone.’ (NF\_Elic15)
\z

\ea
\label{bkm:Ref75178010}
òsháká ꜝcáhà kùndìhùmpàhùmpà wè\\
\gll o-shak-á̲    cáha  ku-ndi-humpa-hump-a  we\\
\textsc{sm}\textsubscript{2SG}-like-\textsc{fv}  very  \textsc{inf}-\textsc{om}\textsubscript{1SG}-\textsc{pl}2-follow-\textsc{fv}  \textsc{pers}\textsubscript{2SG}\\
\glt ‘You really like following me.’ (said to someone who has followed the speaker on several occasions.)
\z

\ea
cìnjí ꜝáshèkàshékà\\
\gll ∅-ci-njí    á̲-sheka-shek-á̲\\
\textsc{cop}-\textsc{np}\textsubscript{7}-what  \textsc{sm}\textsubscript{1}.\textsc{rel}-\textsc{pl}2-laugh-\textsc{fv}\\
\glt ‘Why is s/he laughing all the time?’
\z

\ea
\glll ndàyèndáyèndì\\
ndi-a-endá̲-end-i\\
\textsc{sm}\textsubscript{1SG}-\textsc{pst}-\textsc{pl}2-go-\textsc{npst}.\textsc{pfv}\\
\glt ‘I’ve traveled to many places.’
\z

\ea
\label{bkm:Ref75178011}
àrìráːríráꜝráːrírá bùryô\\
\gll a-ri\textsubscript{H}-raː\textsubscript{H}rirá̲-raːr-ir-á̲    bu-ryó\\
\textsc{sm}\textsubscript{1}-\textsc{refl}-\textsc{pl}2-sleep-\textsc{appl}-\textsc{fv}  \textsc{np}\textsubscript{14}-just\\
\glt ‘S/he sleeps often.’
\z

To express an action repeated in different locations, reduplication combines with the locative pluractional marker \textit{\-kabú-/yabú-}, as in (\ref{bkm:Ref99092359}--\ref{bkm:Ref99092361}).

\ea
\label{bkm:Ref99092359}
kàbúrìhíndìràhìndìrà bùryô\\
\gll kabú-ri-híndira-hind-ir-a    bu-ryó\\
\textsc{loc}.\textsc{pl}-\textsc{refl}-\textsc{pl}2-take-\textsc{appl}-\textsc{fv}  \textsc{np}\textsubscript{14}-only\\
\glt ‘S/he is just going around taking for himself.’
\z

\ea
\label{bkm:Ref99092361}
mbùryó ꜝndíkàbúzìshùwàshùwà kúbàntù\\
\gll N-bu-ryó    ndí̲-kabú-zi\textsubscript{H}-shuwa-shuw-a    kú-ba-ntu\\
\textsc{cop}-\textsc{np}\textsubscript{14}-only  \textsc{sm}\textsubscript{1SG}.\textsc{rel}-\textsc{loc}.\textsc{pl}-\textsc{om}\textsubscript{8}-\textsc{pl}2-hear-\textsc{fv}  \textsc{np}\textsubscript{17}-\textsc{np}\textsubscript{2}-person\\
\glt ‘I’m just going around hearing things from people.’ (NF\_Elic15)
\z

Repeated action may also be interpreted as an action involving multiple participants: multiple subjects in the case of an intransitive verb, as in (\ref{bkm:Ref75178156}), and multiple objects in the case of a transitive verb, as in (\ref{bkm:Ref75178157}). This same pattern is also observed with pluractional 1 (see \sectref{bkm:Ref70945678}).

\ea
\label{bkm:Ref75178156}
bònshéː nìbáyèrèkàyèrèkà\\
\gll ba-onshéː  ni-bá̲-a-ereka-erek-a\\
\textsc{pp}\textsubscript{2}-all    \textsc{rem}-\textsc{sm}\textsubscript{2}-\textsc{pl}2-try-\textsc{fv}\\
\glt ‘They have all tried.’ (NF\_Narr15)
\z
\pagebreak
\ea
\label{bkm:Ref75178157}
ènwé sèmùkàcònkòmònàcònkòmónà tùmùtwárè kúcìpàtêrà ámùnyà màshérêŋì kúcìkórò\\
\gll enwé  se-mu-ka-conkomona-conkomon-á̲  tu-mu-twá̲r-e kú-ci-patéra    á-munya  ma-sheréŋi  kú-ci-kóro\\
\textsc{pers}\textsubscript{2PL} \textsc{inc}-\textsc{sm}\textsubscript{2PL}-\textsc{dist}-\textsc{pl}2-press-\textsc{fv}     \textsc{sm}\textsubscript{1PL}-\textsc{om}\textsubscript{1}-bring-\textsc{pfv}.\textsc{sbjv} \textsc{np}\textsubscript{17}-\textsc{np}\textsubscript{7}-hospital  \textsc{pp}\textsubscript{6}-other  \textsc{np}\textsubscript{6}-money  \textsc{np}\textsubscript{17}-\textsc{np}\textsubscript{7}-school\\
\glt ‘You just withdraw and withdraw [multiple amounts of money]. We can take him to the hospital [with one amount of money]. The other money, for the school.’ (ZF\_Conv13)
\z

The pluractional marked with \textit{-a} and the pluractional marked with stem reduplication are semantically similar. Many verbs may take either pluractional strategy, without a change in meaning, as illustrated in \tabref{tab:6:11}.

\begin{table}
\label{bkm:Ref98833464}\caption{\label{tab:6:11}Interchangability of pluractional 1 and 2}
\begin{tabular}{lll}
\lsptoprule
Pluractional \textit{-a} & Stem reduplication & \\
\midrule
{\itshape ᵍǀóntàùrà} & {\itshape \-ᵍǀóntàᵍǀòntà} & ‘drip continuously’\\
\textit{kwátàùrà} & {\itshape kwátàkwàtà} & ‘touch everwhere’\\
\textit{shángàùrà} & {\itshape shángàshàngà} & ‘contribute’\\
\textit{shótàùkà} & {\itshape shótòkàshòtòkà} & ‘jump up and down’\\
\textit{yèndàùrà} & {\itshape yèndàyèndà} & ‘walk around’\\
\lspbottomrule
\end{tabular}
\end{table}

The difference between these two pluractional strategies is the connotation of completeness or intensity. As discussed in \sectref{bkm:Ref70945678}, pluractional \textit{-a} implies completeness. Stem reduplication, on the other hand, implies low intensity: it is used to describe an action that is done only lightly, halfheartedly, or haphazardly. Examples of this use of the pluractional marked with reduplication are given in (\ref{bkm:Ref445902153}), which describes the first stages of light sleep; in (\ref{bkm:Ref445902239}), which describes walking a small distance; and in (\ref{bkm:Ref445902355}), which describes that the hoes were strewn about in a disorderly fashion.

\ea
\label{bkm:Ref445902153}
\glll shìbànàráːrìràːrì\\
shi-ba-na-ráːri-raːr-i\\
\textsc{inc}-\textsc{sm}\textsubscript{2}-\textsc{pst}-\textsc{pl}2-sleep-\textsc{pst} \\
\glt ‘They started to sleep a little bit.’
\z
\pagebreak
\ea
\label{bkm:Ref445902239}
mùyéndéyéndè bùryò kànínì\\
\gll mu-ende-é̲nd-e    bu-ryo  ka-niní\\
\textsc{sm}\textsubscript{2PL}-\textsc{pl}2-walk-\textsc{pfv}.\textsc{sbjv}  \textsc{np}\textsubscript{14}-just  \textsc{adv}-little\\
\glt ‘Just walk a little bit/small distance.’
\z

\ea
\label{bkm:Ref445902355}
màhámbà òkùtòmbwèrìsà mângìː àdànsídànsì\\
\gll ma-ámba  a-o=ku-tombwer-is-a    má-ngiː
a-dansí̲-dans-i \\
\textsc{np}\textsubscript{6}-hoe  \textsc{pp}\textsubscript{6}-\textsc{con}=\textsc{inf}-weed-\textsc{caus}-\textsc{fv}  \textsc{pp}\textsubscript{6}-many
\textsc{sm}\textsubscript{6}-\textsc{pl}2-lie-\textsc{imp}.\textsc{stat} \\
\glt ‘Many hoes for weeding were lying around.’ (NF\_Narr15)
\z

The pluractional expressed with stem reduplication can also express negative connotations, as in (\ref{bkm:Ref99092417}--\ref{bkm:Ref99092418}), which is not seen with the pluractional suffix \textit{-a}.

\ea
\label{bkm:Ref99092417}
mbùryó ꜝkágàbàgábà\\
\gll N-bu-ryó    ka-á̲-ga\textsubscript{H}ba-gab-á̲\\
\textsc{cop}-\textsc{np}\textsubscript{14}-only  \textsc{pst}.\textsc{ipfv}-\textsc{sm}\textsubscript{1}-\textsc{pl}2-talk\_nonsense-\textsc{fv} \\
\glt ‘S/he is just talking nonsense.’ (NF\_Elic17)
\z

\ea
àkwèsì ààmbàâmbà\\
\gll a-kwesi  a-amba-á̲mb-a\\
\textsc{sm}\textsubscript{1}-\textsc{prog}  \textsc{sm}\textsubscript{1}-\textsc{pl}2-talk-\textsc{fv}\\
\glt ‘S/he talks too much.’
\z

\ea
cìnj’ áh’ ꜝóshèkàshékà ꜝbúryò\\
\gll ∅-ci-njí    a-ha    ó̲-sheka-shek-á̲    bu-ryó\\
\textsc{cop}-\textsc{np}\textsubscript{7}-what  \textsc{aug}-\textsc{dem}.\textsc{i}\textsubscript{16}  \textsc{sm}\textsubscript{2SG}.\textsc{rel}-\textsc{pl}2-laugh-\textsc{fv}  \textsc{np}\textsubscript{14}-only\\
\glt ‘Why are you always just laughing (stupidly/annoyingly)?’ (NF\_Elic15)
\z

\ea
\label{bkm:Ref99092418}
kwàshíààzyà zòkùtêyè ndìkàbúzèbùzè\\
\gll kwa-shí-aayza  zi-o=kutéye    ndi-ka-búze-buz-e\\
\textsc{sm}\textsubscript{17}-\textsc{per}-be\_not  \textsc{pp}\textsubscript{10}-\textsc{con}=that  \textsc{sm}\textsubscript{1SG}-\textsc{dist}-\textsc{pl}2-ask-\textsc{pfv}.\textsc{sbjv}\\
\glt ‘Now there is no longer anything that I have to keep asking.’ (The speaker has repeatedly gone back and forth to ask his wife where she has hidden his teeth, and has grown very impatient and annoyed.) (NF\_Narr15)
\z

\begin{sloppypar}
Both pluractional strategies share some characteristics with the intensive derivation, which may also express a repeated action. As discussed in \sectref{bkm:Ref485997587}, repeated action is only an extension of the “intensive” basic meaning of the reduplicated applicative, and unlike the two pluractional strategies, marking repeated action is not a basic function of the intensive derivation.
\end{sloppypar}

\section{Intensive}
\label{bkm:Ref485997587}\label{bkm:Ref492040534}\hypertarget{Toc75352667}{}
The intensive suffix is formally identical to the reduplicated form of the applicative suffix, e.g. it is realized as \textit{\nobreakdash-irir, \nobreakdash-erer, \nobreakdash-inin} or \textit{\-\nobreakdash-enen} depending on vowel and nasal harmony (see \sectref{bkm:Ref451863900}-\ref{bkm:Ref70947851}). It does not, however, have the typical function of applicative, namely adding a participant, as seen when comparing the underived verb in (\ref{bkm:Ref506372689}) with the intensive verb in (\ref{bkm:Ref74916113}).

\ea
\label{bkm:Ref506372689}
\glll cìzyúmîtè\\
ci-zyu\textsubscript{H}m-í̲te\\
\textsc{sm}\textsubscript{7}-become\_dry-\textsc{stat}\\
\glt ‘It is dry.’
\z

\ea
\label{bkm:Ref74916113}
\glll cìzyúmínìnè\\
ci-zyu\textsubscript{H}m-í̲nine\\
\textsc{sm}\textsubscript{7}-become\_dry-\textsc{int}.\textsc{stat}\\
\glt ‘It is very dry/hard.’ (NF\_Elic15)
\z

The core meaning of the this suffix is intensity, as shown in(\ref{bkm:Ref71210866}--\ref{bkm:Ref71210867}), but it may also express a range of related meanings: completeness, as in (\ref{bkm:Ref71210899}--\ref{bkm:Ref71210902}); high frequency or habitual, as in (\ref{bkm:Ref71211104}--\ref{bkm:Ref71210990}); long duration, as in (\ref{bkm:Ref71211122}); or repetition, as in (\ref{bkm:Ref71211136}--\ref{bkm:Ref71211137}).

\ea
\label{bkm:Ref71210866}
\glll kùtóndèrèrà\\
ku-tónd-erer-a\\
\textsc{inf}-watch-\textsc{int}-\textsc{fv}\\
\glt ‘to stare at’
\z

\ea
\label{bkm:Ref71210867}
\glll kúmìnìnìzà\\
kú-min-iniz-a\\
\textsc{inf}-tuck\_in-\textsc{int}.\textsc{caus}-\textsc{fv}\\
\glt ‘to tuck in properly’
\z

\ea
\label{bkm:Ref71210899}
kùáázy’ ézwâyì kwìná àbó bànàkéːzyì \textbf{kùríùrìrìrà} ryònshêː\\
\gll ku-aazyá  e-∅-zwáyi  ku-iná    a-bó ba-na-ké̲ːzy-i    ku-rí-ur-irir-a    ry-onshéː\\
\textsc{sm}\textsubscript{17}-be\_not  \textsc{aug}-\textsc{np}\textsubscript{5}-salt  \textsc{sm}\textsubscript{17}-be\_at  \textsc{aug}-\textsc{dem}.\textsc{iii}\textsubscript{2}
\textsc{sm}\textsubscript{2}-\textsc{pst}-come-\textsc{npst}.\textsc{pfv}  \textsc{inf}-\textsc{om}\textsubscript{5}-buy-\textsc{int}-\textsc{fv}  \textsc{pp}\textsubscript{5}-all\\
\glt ‘There is no salt, someone has come and \textbf{bought} \textbf{it} \textbf{all}.’ (NF\_Elic15)
\z

\ea
\label{bkm:Ref71210902}
\textbf{àhíndírír’} émìsèbézì yònshêː àfíyérà àsánz’ ótùsûbà àténdà zònshéː ꜝzómùnjûò\\
\gll a-hind-irir-á̲    e-mi-sebézi  i-onshéː a-fi\textsubscript{H}er-á̲    a-sanz-á̲  o-tu-súba a-té̲nd-a  zi-onshéː  zi-ó=mu-N-júo\\
\textsc{sm}\textsubscript{1}-take-\textsc{int}-\textsc{fv}  \textsc{aug}-\textsc{np}\textsubscript{4}-job  \textsc{pp}\textsubscript{4}\--all
\textsc{sm}\textsubscript{1}-sweep-\textsc{fv}  \textsc{sm}\textsubscript{1}-wash-\textsc{fv}  \textsc{aug}-\textsc{np}\textsubscript{13}-dish
\textsc{sm}\textsubscript{1}-do-\textsc{fv}  \textsc{pp}\textsubscript{10}-all  \textsc{pp}\textsubscript{10}-\textsc{con}=\textsc{np}\textsubscript{18}-\textsc{np}\textsubscript{9}-house\\
\glt ‘She \textbf{takes} \textbf{all} \textbf{the} \textbf{jobs}. She sweeps, she washes dishes, she does all the things in the house.’ (NF\_Elic15
\z

\ea
\label{bkm:Ref71211104}
bâncè \textbf{bàtèkèrèrá} mênjì\\
\gll ba-ánce  ba-te\textsubscript{H}k-erer-á̲  ma-ínji\\
\textsc{np}\textsubscript{2}-child  \textsc{sm}\textsubscript{2}-fetch-\textsc{int}-\textsc{fv}  \textsc{np}\textsubscript{6}-water\\
\glt ‘Children [normally] fetch water.’ (explaining which tasks are usually performed by whom) (ZF\_Elic14)
\z

\ea
\label{bkm:Ref71210990}
\glll ndìshàmbírìrè\\
ndi-shamb-í̲rire\\
\textsc{sm}\textsubscript{1SG}-swim-\textsc{int}-\textsc{stat}\\
\glt ‘I always swim.’ (NF\_Elic17)
\z

\ea
\label{bkm:Ref71211122}
\glll àbèngérèrè\\
a-be\textsubscript{H}ng-é̲rere\\
\textsc{sm}\textsubscript{1}-become\_angry-\textsc{int}.\textsc{stat}\\
\glt ‘S/he is always angry.’
\z

\ea
\label{bkm:Ref71211136}
\glll kùfúzìrìrìrà\\
ku-fúzir-irir-a\\
\textsc{inf}-fan-\textsc{int}-\textsc{fv}\\
\glt ‘to keep on fanning [a fire]’
\z

\ea
\label{bkm:Ref71211137}
\glll kùkámbìrìrà\\
ku-kámb-irir-a\\
\textsc{inf}-clap-\textsc{int}-\textsc{fv}\\
\glt ‘to applaud, clap repeatedly
\z

As seen in (\ref{bkm:Ref71211136}--\ref{bkm:Ref71211137}), repeated action can be part of the interpretation of the intensive derivation. This is not its core meaning, but merely an extension of its intensity meaning, can be seen by comparing the intensive with the two pluractional constructions, the pluractional suffix \textit{-a} and stem reduplication, which both have repetition as their core meaning (see \sectref{bkm:Ref489866362}). This difference is illustrated with the verb \textit{kwát} ‘touch, grab’: used with the intensive in (\ref{bkm:Ref75178404}), it may refer to a single event of touching which has either a long duration or a high intensity; with stem reduplication in (\ref{bkm:Ref75178424}) or the pluractional \textit{-a} in (\ref{bkm:Ref75178434}), it is interpreted as multiple instances of touching.

\ea
\label{bkm:Ref75178404}
\glll ndìkwàtírìrè\\
ndi-kwa\textsubscript{H}t-í̲rire\\
\textsc{sm}\textsubscript{1SG}-touch-\textsc{int}.\textsc{stat}\\
\glt ‘I hold (for a long time/firmly).’
\z

\ea
\label{bkm:Ref75178424}
kàndìshàkí mùntù ándìkwàtàkwátà bùryáhò\\
\gll ka-ndi-shak-í̲    mu-ntu  á̲-ndi-kwata-kwá̲t-a    bu-ryahó \\
\textsc{neg}-\textsc{sm}\textsubscript{1SG}-like-\textsc{neg}  \textsc{np}\textsubscript{1}-person
\textsc{sm}\textsubscript{1}.\textsc{rel}-\textsc{om}\textsubscript{1SG}-\textsc{pl}2-touch-\textsc{fv}  \textsc{np}\textsubscript{14}-like\_that\\
\glt ‘I don’t like it when someone touches me all over like that.’
\z

\ea
\label{bkm:Ref75178434}
mùzwé kùkwátàùrà múzìpàùpàù zángù\\
\gll mu-zw-é̲      ku-kwát-a-ur-a mú-zi-paupua  zi-angú\\
\textsc{sm}\textsubscript{2PL}-leave-\textsc{pfv}.\textsc{sbjv}  \textsc{inf}-touch-\textsc{pl}1-\textsc{sep}.\textsc{tr}-\textsc{fv}
\textsc{np}\textsubscript{18}-\textsc{np}\textsubscript{8}-basket  \textsc{pp}\textsubscript{8}-\textsc{poss}\textsubscript{1SG}\\
\glt ‘Stop touching in my baskets/bags/purses.’ (NF\_Elic17
\z

Another difference between the intensive and the pluractional marked by stem reduplication specifically is that stem reduplication implies a repeated action with low intensity, i.e. only slightly or without strong consequences. This difference is illustrated with the verb \textit{sanz} ‘wash’: with the intensive in (\ref{bkm:Ref75178478}), it refers to washing something thoroughly and properly, but with stem reduplication in (\ref{bkm:Ref75178506}), it refers to washing something slightly, not thoroughly.

\ea
\label{bkm:Ref75178478}
ndìshàká \textbf{kùyísànzìrìrà} bùryô ìcénè\\
\gll ndi-shak-á̲    ku-í-sanz-irir-a    bu-ryó i-cen-é̲ \\
\textsc{sm}\textsubscript{1SG}-want-\textsc{fv}  \textsc{inf}-\textsc{om}\textsubscript{4}-wash-\textsc{int}-\textsc{fv}  \textsc{np}\textsubscript{14}-just
\textsc{sm}\textsubscript{4}-become\_clean-\textsc{pfv}.\textsc{sbjv}\\
\glt ‘I just want to wash them thoroughly, so that they become clean.’
\z

\newpage
\ea
\label{bkm:Ref75178506}
mbùryó ꜝ\textbf{ndíyìsànzàsànzá} bùryô yáràshàmbà nênjà\\
\gll N-bu-ryó    ndí̲-i\textsubscript{H}-sanza-sanz-á̲      bu-ryó i-ára-shamb-a      nénja\\
\textsc{cop}-\textsc{np}\textsubscript{14}-only  \textsc{sm}\textsubscript{1SG}.\textsc{rel}-\textsc{om}\textsubscript{4}-\textsc{pl}2-wash-\textsc{fv}  \textsc{np}\textsubscript{14}-only
\textsc{sm}\textsubscript{4}-\textsc{rem}.\textsc{fut}-be\_washed-\textsc{fv}  well\\
\glt ‘I’m only washing them a bit, they will become clean (properly) later.’ (NF\_Elic17)
\z
\section{Reciprocal}
\hypertarget{Toc75352668}{}\label{bkm:Ref98773050}
Many Bantu languages use a reflex of the reconstructed reciprocal suffix *-an to express a reciprocal. In Fwe, reciprocal semantics is productively expressed by the prefix \textit{kí-/rí-} which also expresses a reflexive (see \sectref{bkm:Ref451256199}). A reciprocal suffix \textit{-an}, however, occurs in a very small set of lexicalized verbs, in Lozi borrowings, and can still be readily elicited from speakers.

Three lexicalized verbs with a reciprocal suffix \textit{-an} exist. The verb \textit{shúwànà} is derived from the verb \textit{shûwà} ‘hear, understand’. The verb \textit{gumban} ‘stand next to each other’ has an alternative form \textit{gumbam}, where the reciprocal suffix \textit{-an} is replaced by the intransitive impositive suffix \textit{-am} (see \sectref{bkm:Ref450835510}). The meaning of the verb seems to fit well with both the reciprocal and the impositive, which may have facilitated the replacement of \textit{-an} with \textit{-am} (or vice versa). The verb \textit{kánan} is also not an unambiguously reciprocal verb: it can be used as a reciprocal, as in (\ref{bkm:Ref75178625}), which describes a group of people arguing with each other, but also without any reciprocal meaning, as in (\ref{bkm:Ref75178635}). Although \textit{múkànàná} takes a seconperson plural subject marker, a single person is referred to in this excerpt from a narrative, which describes a conversation between the speaker and her sister.

\ea
\label{bkm:Ref75178625}
zìnjí ꜝmúkànàná\\
\gll ∅-zi-njí    mú̲-ka\textsubscript{H}n-an-á̲\\
\textsc{cop}-\textsc{np}\textsubscript{8}-what  \textsc{sm}\textsubscript{2PL}.\textsc{rel}-argue-\textsc{rec}-\textsc{fv}\\
\glt ‘What are you (\textsc{pl}) arguing about?
\z

\ea
\label{bkm:Ref75178635}
háìbà mùkánánà\\
\gll háiba  mu-ka\textsubscript{H}n-an-á̲\\
if  \textsc{sm}\textsubscript{2PL}-refuse-\textsc{rec}-\textsc{fv}\\
\glt ‘If you (SG) disagree…’
\z

The reciprocal suffix is also seen in borrowings from Lozi, where the reciprocal suffix \textit{-an} is used productively \citep{Fortune1977}. Many of these borrowings do not occur without the reciprocal suffix in Fwe, as in (\ref{bkm:Ref99093226}--\ref{bkm:Ref99093227}).

\ea
\label{bkm:Ref99093226}
\ea
\glll kùkòpànà\\
ku-kop-an-a\\
\textsc{inf}-meet-\textsc{rec}-\textsc{fv}\\
\glt ‘to meet’

\ex
*kù-kòp-à

\ex
borrowed from Lozi ku kopana ‘to meet, assemble’ \citep[94]{Burger1960}
\z\z

\ea
\label{bkm:Ref99093227}
\ea
\glll kùkáwùhànà\\
ku-káwuh-an-a\\
\textsc{inf}-separate-\textsc{rec}-\textsc{fv}\\
\glt ‘to be separated’

\ex
*kù-káwùh-à

\ex
borrowed from Lozi ku kauhana ‘to turn apart’ \citep[133]{Burger1960}
\z\z

Surprisingly, verbs with reciprocal \textit{-an} can readily be elicited from speakers, as in in (\ref{bkm:Ref494449589}--\ref{bkm:Ref74909222}). Speakers consistently produce forms with reflexive \textit{rí-} / \textit{kí-} when asked to translate or describe reciprocal situations, but accepted forms with \textit{-an} when prompted.

\ea
\label{bkm:Ref494449589}
\glll kùbúzànà\\
ku-búz-an-a\\
\textsc{inf}-ask-\textsc{rec}-\textsc{fv}\\
\glt ‘to ask each other’
\z

\ea
\glll kùbbózànà\\
ku-bbóz-an-a\\
\textsc{inf}-bark-\textsc{rec}-\textsc{fv}\\
\glt ‘to bark at each other’
\z

\ea
\glll kùtùkànà\\
ku-tuk-an-a\\
\textsc{inf}-insult-\textsc{rec}-\textsc{fv}\\
\glt ‘to insult each other’ (NF\_Elic17)
\z

\ea
\label{bkm:Ref74909222}
\glll kùshótòkànà\\
ku-shótok-an-a\\
\textsc{inf}-jump-\textsc{rec}-\textsc{fv}\\
\glt ‘to cross each other’ (ZF\_Elic13)
\z

With the exception of lexicalized verbs and Lozi borrowings, verbs with reciprocal \textit{-an} were never encountered in spontaneous discourse. Even when asked to describe a situation that could be interpreted as either reflexive or reciprocal, speakers would use periphrastic strategies to disambiguate reflexive and reciprocal meanings, rather than the distinction between \textit{rí-/kí-} and \textit{-an}. Possibly, the ease with which reciprocal \textit{-an} could be elicited, even though it never occurred in spontaneous data, may be a result of extensive bilingualism with Lozi, where a reciprocal \textit{-an} is still highly productive. All speakers interviewed in this study (and presumably, the vast majority of Fwe-speaking adults) were also fluent in Lozi.

\section{Extensive}
\label{bkm:Ref486253151}\hypertarget{Toc75352669}{}
The extensive derivation \textit{-ar/-an} (subject to nasal harmony, see \sectref{bkm:Ref70697565}) is unproductive. The only attested examples are listed in \tabref{tab:6:12}. None of the verbs using the extensive suffix are attested without this suffix, but in some of these verbs the extensive can be replaced by the transitive impositive suffix \textit{-ik/-ek} (see also \sectref{bkm:Ref450835510}), or the transitive separative \textit{-ur/-un} (see also \sectref{bkm:Ref485823385}).

\begin{table}[b]
\label{bkm:Ref486254096}\caption{\label{tab:6:12}The extensive suffix \textit{-ar/-an}}
\begin{tabular}{ll}
\lsptoprule
{\itshape àzyàrà} & ‘think, plan’\\
{\itshape fúrùmànà} & ‘be initiated (of girls)’\\
cf. \textit{fúrùmìkà} & ‘place upside down’\\
cf. \textit{fúrùmùnà} & ‘place rightside up’\\
{\itshape òmbàrà} & ‘be quiet, calm’\\
{\itshape rémànà} & ‘become injured’\\
cf. \textit{rémèkà} &  ‘injure’\\
{\itshape shàràngàrà} & ‘scatter’\\
{\itshape súmbàrà} & ‘become pregnant’\\
cf. \textit{súmbìkà} &  ‘impregnate’ \\
{\itshape tándàbàrà} & ‘stretch one’s legs’\\
cf. \textit{tándàbìkà}  &  ‘cause to stretch (another person’s) legs’\\
{\itshape tàngàràrà} & ‘rejoice’\\
{\itshape zìbàrà} & ‘forget’\\
{\itshape zyàbàrà} &  ‘dress (oneself)’\\
cf. z\textit{yàbìkà} & ‘dress (someone else)’\\
cf. \textit{zyàbùrà} &  ‘undress’\\
{\itshape zyímànà} & ‘stand up, stop’\\
cf. \textit{zyímìkà} & ‘put in a standing position’\\
\lspbottomrule
\end{tabular}
\end{table}

Given the limited number of examples and the suffix’s lack of productivity, little can be said about its syntactic and semantic functions. Considering the verbs in \tabref{tab:6:12}, it is clear that verbs with the extensive suffix tend to be intransitive, and many are posture verbs, hence the tendency to derive impositive verbs. The label “extensive” is chosen for this derivational suffix on the basis of comparative data.  {\citet[184]{SchadebergBostoen2019}} describe the core semantics of reflexes of a reconstructed suffix *\nobreakdash-ad as ‘being in a spread-out position’, and as such uses the label extensive. In some of the attested Fwe verbs using the extensive suffix, such semantics also seem to play a role, such as \textit{sharangar} ‘scatter’, \textit{tándabar} ‘stretch one’s legs’, and \textit{zyíman} ‘stand up’.

\section{Tentive}
\label{bkm:Ref485997273}\hypertarget{Toc75352670}{}
There are a number of verb stems in which a suffix -\textit{at} is discernable. This is a reflex of a suffix reconstructed for Proto-Bantu as “contactive” \citep[92]{Meeussen1967}, or “tentive” (\citealt{SchadebergBostoen2019}: 184-185), and is completely unproductive in Fwe. All attested examples are listed in (\ref{bkm:Ref98841772}).

\ea
\label{bkm:Ref98841772}
bbábbàtà \tab ‘touch (with flat hands)’\\
bàràkàtà \tab ‘flap (as a fish on dry land)’\\
kámàtà \tab ‘scoop’\\
kwâtà (cf. kú-at-a) \tab ‘catch, grab’\\
kúmbàtà \tab ‘hug’\\
ràndàtà \tab ‘track’\\
ryàːtà (cf. ri-at-a) \tab ‘step on’\\
vúrùmàtà \tab ‘close one’s eyes’\\
\z


There is one example, given in (\ref{bkm:Ref98841730}), where the tentive suffix can be replaced with a different derivational suffix.

\ea
\label{bkm:Ref98841730}
\glll kùzwâtà\\
ku-zú-at-a\\
\textsc{inf}-dress-\textsc{tent}-\textsc{fv}\\
\glt ‘to dress’
\z

\ea
\glll kùzûrà\\
ku-zú-ur-a\\
\textsc{inf}-dress-\textsc{sep}.\textsc{tr}-\textsc{fv}\\
\glt ‘to undress’
\z

The semantics of the tentive derivation in Bantu is described as ‘actively making firm contact’ (\citealt{SchadebergBostoen2019}: 184-185). Although the number of attested examples in Fwe is limited, many of these seem to fit this semantic characterization.

\section{Partial reduplication}
\hypertarget{Toc75352671}{}\label{bkm:Ref98773057}
An apparent, but unproductive, verbal derivational process in Fwe is partial reduplication, which targets the first syllable of the verb root. The complete list of verbs attested that exhibit partial reduplication is given in \tabref{tab:6:13}.

\begin{table}
\label{bkm:Ref489606874}\caption{\label{tab:6:13}Partial reduplication}

\begin{tabular}{ll}
\lsptoprule
{\itshape bbábbàtà} & ‘touch with flat hands’\\
{\itshape càncàùsà} & ‘be fast’\\
{\itshape cécèntà} & ‘winnow’\\
{\itshape cúncùnà} & ‘kiss’\\
{\itshape cùncùrà} & ‘stumble’\\
{\itshape fùfùrèrwà} & ‘sweat’\\
{\itshape fwáfwàtìrà} & ‘get crushed, crumpled’\\
{\itshape kákàtìrà} & ‘stick (as a burdock)’\\
{\itshape mwémwètà} & ‘smile’\\
{\itshape ngóngòtà} & ‘knock’\\
{\itshape nyényèntèzà} & ‘warn’\\
{\itshape ⁿǀóⁿǀòwèzà} & ‘eat/drink slowly’\\
{\itshape pòpòkà} & ‘pop, explode with a popping sound’\\
{\itshape shòshòtà} & ‘whisper’\\
{\itshape sónsònìsà} & ‘search around’\\
{\itshape tùtùmà} & ‘shiver’\\
{\itshape zùzùnyà} & ‘doubt’\\
\lspbottomrule
\end{tabular}
\end{table}

Partial reduplication does not always reproduce the first root syllable perfectly. Prenasalization on the second element may be missing on the first, as in \textit{càncàùsà} ‘be fast’ and \textit{cùncùrà} ‘stumble’, possibly because prenasalization of an initial root consonant is dispreferred in Fwe.

Many verbs with partial reduplication use /t/ (or /nt/) directly after the reduplicand. This could be a trace of the unproductive tentive suffix \textit{-at} (see \sectref{bkm:Ref485997273}), where the vowel of the suffix would have merged with the vowel of the verb stem, as the vowel /a/ is prone to do (see \sectref{bkm:Ref491962181} on vowel hiatus resolution).

Partial reduplication is unproductive, and none of the verbs attested with partial reduplication are attested without it. Considering the attested examples, the iconic relation between reduplication and repeated movement seems to play a role in, for instance, \textit{cécent} ‘winnow’, \textit{tutum} ‘shiver’, and \textit{cuncur} ‘stumble’. Sound symbolism also plays a role, in forms such as \textit{shoshot} ‘whisper’, \textit{cúncun} ‘kiss’, and \textit{ngóngot} ‘knock’.

