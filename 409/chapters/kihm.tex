\documentclass[output=paper]{langscibook}
\ChapterDOI{10.5281/zenodo.10280614}

\author{Alain Kihm\orcid{}\affiliation{Université Paris Cité}}

\title[Plural marking in spoken French, Fa d’Ambô and Réunion Creole]{Another story of Z: Plural marking in spoken French, Fa d’Ambô and Réunion Creole}

\abstract{Convergent evolution is observed in languages as well as in living species. Thus the three languages spoken French, Fa d’Ambô, and Réunion Creole independently developed a way of signifying plurality in count nouns involving the presence of a /z/ phoneme of uncertain morphological status at the left end of the stem. Two of them are creole languages with different lexical bases, Portuguese for Fa d’Ambô, French for Réunion Creole. As for spoken French, its inflectional morphology markedly differs from Written French inflectional morphology. Such independent developments satisfy all requisites to qualify as convergent evolution. Yet, spoken French diverged from Written French only as far as \textit{exponence} is concerned, not in terms of the semantic-pragmatic rationale for number specification, whereas Fa d’Ambô and Réunion Creole departed from their respective lexifiers at both levels. The difference is bound to be related to the creole vs. noncreole statuses of Fa d’Ambô and Réunion Creole on the one hand, and spoken French on the other. The difference between Fa d’Ambô and Réunion Creole, on the other hand, is due to the different language ecologies in the midst of which the two languages emerged.
\keywords{convergent evolution, creolization, inflection, language change, number}}

\IfFileExists{../localcommands.tex}{
  \addbibresource{../localbibliography.bib}
  \usepackage{langsci-optional}
\usepackage{langsci-gb4e}
\usepackage{langsci-lgr}

\usepackage{listings}
\lstset{basicstyle=\ttfamily,tabsize=2,breaklines=true}

%added by author
% \usepackage{tipa}
\usepackage{multirow}
\graphicspath{{figures/}}
\usepackage{langsci-branding}

  
\newcommand{\sent}{\enumsentence}
\newcommand{\sents}{\eenumsentence}
\let\citeasnoun\citet

\renewcommand{\lsCoverTitleFont}[1]{\sffamily\addfontfeatures{Scale=MatchUppercase}\fontsize{44pt}{16mm}\selectfont #1}
   
  %% hyphenation points for line breaks
%% Normally, automatic hyphenation in LaTeX is very good
%% If a word is mis-hyphenated, add it to this file
%%
%% add information to TeX file before \begin{document} with:
%% %% hyphenation points for line breaks
%% Normally, automatic hyphenation in LaTeX is very good
%% If a word is mis-hyphenated, add it to this file
%%
%% add information to TeX file before \begin{document} with:
%% %% hyphenation points for line breaks
%% Normally, automatic hyphenation in LaTeX is very good
%% If a word is mis-hyphenated, add it to this file
%%
%% add information to TeX file before \begin{document} with:
%% \include{localhyphenation}
\hyphenation{
affri-ca-te
affri-ca-tes
an-no-tated
com-ple-ments
com-po-si-tio-na-li-ty
non-com-po-si-tio-na-li-ty
Gon-zá-lez
out-side
Ri-chárd
se-man-tics
STREU-SLE
Tie-de-mann
}
\hyphenation{
affri-ca-te
affri-ca-tes
an-no-tated
com-ple-ments
com-po-si-tio-na-li-ty
non-com-po-si-tio-na-li-ty
Gon-zá-lez
out-side
Ri-chárd
se-man-tics
STREU-SLE
Tie-de-mann
}
\hyphenation{
affri-ca-te
affri-ca-tes
an-no-tated
com-ple-ments
com-po-si-tio-na-li-ty
non-com-po-si-tio-na-li-ty
Gon-zá-lez
out-side
Ri-chárd
se-man-tics
STREU-SLE
Tie-de-mann
} 
  \togglepaper[1]%%chapternumber
}{}

\begin{document}
\AffiliationsWithoutIndexing{}
\maketitle 
%\shorttitlerunninghead{}%%use this for an abridged title in the page headers



\section{Introduction}\label{sec:kihm:1}
\largerpage[-2]
“Since adaptive themes are limited and animals so diverse, convergence of different evolutionary lineages to the same general solution (but not to detailed repetition) are common” \citep[412]{Gould1985}.\footnote{To quote a not so prestigious source, “Convergent evolution is the independent evolution of similar features in species of different periods or epochs in time. Convergent evolution creates analogous structures that have similar form or function but were not present in the last common ancestor of those groups. The cladistic term for the same phenomenon is homoplasy.” (Wikipedia)} In language as well, themes (understand ‘structures’) are limited in types and individual languages (\textit{langues}) are diverse. Convergent evolution should therefore be observed in language. That is, there ought to be cases of languages not directly in contact -- so that no areal effects are involved -- independently (i.e. not as a result of genetic kinship) developing similar solutions to the same universal grammatical requirements that every human language must somehow satisfy (adapt to). The present study is devoted to such a case involving three languages: spoken French, Fa d’Ambô, and Réunion Creole, two of which are creole languages with different lexical bases, Portuguese for Fa d’Ambô, French for Réunion Creole. All three developed a particular way of signifying plurality in count nouns through the presence of a /z/ phoneme of uncertain morphological status at the left end of the stem.

I will first describe the three situations. Then I will show that taken together they satisfy all requisites to qualify as convergent evolution. Finally I will try to achieve some understanding of what the phenomenon implies for our current views of creolization as compared with more “regular” language change. In particular, I will argue that while the changes leading to the present situation in spoken French did open a typological gap between it and Written French as far as exponence is concerned, they did not upset the semantic basis of the expression of number. Fa d’Ambô and Réunion Creole, in contrast, drifted further away from their respective lexifiers. That they did not drift quite along the same path, on the other hand, is due to the different language ecologies in the midst of which they emerged.

\section{Plural marking in spoken French}
\label{sec:kihm:2}

By spoken French I refer to European spoken French as distinct from Written French.\footnote{More precisely to the variety of French spoken in the northern half of “metropolitan” France (\textit{pays d’oïl}) and, to the best of my knowledge, francophone Belgium and Switzerland. No geographical boundaries are required for Written French.} The breadth of the typological gap between both instantiations of the language is often undervalued for sociolinguistic reasons, including school teaching. Nowhere is it more evident to the unprejudiced eye than in the domain of inflectional morphology, including the way nouns and adjectives (nominals) are marked for plurality.

\begin{sloppypar}
As I view it, spoken French ought not to be confused with so-called “\textit{français populaire}” (see e.g. \citealt{Gadet1992}) which, as acknowledged by the author just mentioned, is characterized by “\textit{instabilité}” and “\textit{hétérogénéité}” \citep[122]{Gadet1992}. With respect at least to inflection, in contrast, spoken French shows a rather stable and homogeneous system common to all registers, but for a few details. It is in fact the only way Written French can be transposed into the oral medium.\footnote{Things are different with syntax, where variation appears to be wider.}
\end{sloppypar}

Concerning plural marking, Written French indisputably belongs to the Western group of Romance languages. Most nominals (and determiners) pluralize by adding 〈s〉 or 〈x〉 to the word: cf. \textit{le chat noir} ‘the black cat’ vs. \textit{le\textbf{s} chat\textbf{s} noir\textbf{s} }‘the black cats’ like Languedocian Occitan \textit{lo gat negre} ‘the black cat’ vs. \textit{lo}\textbf{\textit{s}} \textit{gat}\textbf{\textit{s}} \textit{negre}\textbf{\textit{s}} ‘the black cats’.\footnote{For clarity’s sake, Written French forms are enclosed between angled brackets.} Only nominals ending in 〈al〉 or 〈ail〉 -- and by far not all of them -- behave somewhat differently: 〈(i)l〉 changes to 〈u〉 and the plural ending is 〈x〉, originally a medieval scribe’s shortening for 〈us〉 (\cites[95]{Cohen1987}[20]{Moignet1988}): cf. 〈un journal régional〉 ‘a regional newspaper’ vs. 〈des journa\textbf{ux} régiona\textbf{ux}〉 ‘regional newspapers’, 〈un travail〉 ‘a work’ vs. 〈des trava\textbf{ux}〉 ‘works’.\footnote{In Old French \textit{u} notates the velarization of /l/ before /s/ leading to a diphthong /aʊ/, later simplified to /o/.}

Things are very different in spoken French. Except in so-called “liaison” contexts -- to which I return presently -- no morph corresponds to orthographic 〈s〉 or 〈x〉, so that \textit{ʃa} ‘cat’ does not inflect for number.\footnote{I use italicized IPA characters to transcribe spoken French, adding a few operators such as “=” indicating clitichood. Phonemic transcriptions are enclosed between slashes.} Plurality can therefore only be marked on proclitic~determiners: \textit{lə=ʃa\_nwar} /lœʃanwaʁ/ ‘the black cat’ vs. \textit{le=ʃa\_nwar} /leʃanwaʁ/~‘the black cats’, \textit{ɛ=ʃa\_nwar} /ɛʃanwaʁ/ ‘a black cat’ vs. \textit{de=ʃa\_nwar} /deʃanwaʁ/ ‘some black cats’, \textit{mɔ=ʃa\_nwar} /mɔʃanwaʁ/ ‘my black cat’ vs. \textit{me=ʃa\_nwar} /meʃanwaʁ/ ‘my black cats’, and so forth. Note that none of the determiners include an -\textit{s} suffix, but they show suppletive forms according to number. In \textit{ʒurnal}\slash\textit{ʒurno} ‘newspaper(s)’, suppletion in the final syllable appears to be the relevant inflectional device as well, for in no reasonable analysis may /o/ be viewed as a plural morph. Moreover, ending in /al/ is not enough, the noun’s gender must be masculine: compare \textit{yn=kabal reʒjonal} ‘a regional cabal’ vs. \textit{de=kabal reʒjonal} ‘regional cabals’.\footnote{In Written French, 〈cabale〉 and 〈régionale〉 do not end in 〈al〉, but in 〈ale〉. Again final 〈e〉, a feminine marker in Written French, does not correspond to anything in spoken French – at least in the northern variety I am dealing with.} Isolated cases of total or partial suppletion are \textit{œj} (〈œil〉) ‘eye’ vs. \textit{jø} (〈yeux〉) ‘eyes’, \textit{bœf} (〈boeuf〉) ‘ox’ vs. \textit{bø} (〈boeufs〉) ‘oxen’, \textit{œf} (〈oeuf〉) ‘egg’ vs. \textit{ø} (〈oeufs〉) ‘eggs’, \textit{ɔs} (〈os〉) ‘bone’ vs. \textit{o} (〈os〉) ‘bones’, \textit{sɛrf} (〈cerf〉) ‘stag’ vs. \textit{sɛr} (〈cerfs〉) ‘stags’.\footnote{Many speakers regularize the partially suppletive items, especially the last one, using the singular form for both numbers. Note that \textit{œj} 〈œil〉 is regular in the compound \textit{œjdəbœf} 〈œil-de-bœuf〉 ‘bull’s eye’: \textit{dez=œjdəbœf} 〈des œils-de-bœuf〉.}

What about liaison? Much has been made of it in order to ‘rescue orthography’, that is to show that orthographic representations, for all their remoteness from actual pronunciation, nevertheless give a faithful image of French inflection (see in particular \citealt{Dell1973, Huot2005}). A liaison context occurs, for instance, whenever the plural word a proclitic determiner attaches to begins with a vowel or a vowel-like approximant. Such a context is supposed to reveal the materiality of the 〈s〉 graphic suffix of the determiner that shows up as /z/, voiced because intervocalic: cf. Written French 〈les amis〉 ‘the friends’, 〈les oies〉 ‘the geese’ always realized as /lezami/ and /lezwa/ (\citealt[134--136]{Huot2005}). Hence there is a distinction between ‘underlying’ plural forms involving s-marking (in Spoken as well as Written French) and ‘surface’ forms  (spoken French only) resulting from a morphophonological truncation rule deleting final /s/ (as well as most word-final obstruents) before consonants or pause \citep[181--182]{Dell1973}. Only before vowels and vowel-like approximants does /s/ escape truncation. At least two objections may be raised against such an analysis.

Firstly, liaison in spoken French is a complex phenomenon, not covered by the traditional prescriptive tripartite division between obligatory, optional and forbidden. According to \citegen{Mallet2008} corpus study, the only liaison that is realized in one hundred percent of the cases, or very close to it, is between a plural determiner and a following vowel or vowel-like approximant, as in the above examples. There is great variation otherwise, including those cases prescriptive grammar considers obligatory, e.g. with preposed adjectives: cf. 〈de savants aveugles〉 ‘learned blind persons’ supposedly pronounced /dəsavãzavœgl/ and contrasting with 〈des savants aveugles〉 ‘blind scientists’ where liaison is said to be excluded (/desavãavœgl/) (\citealt[51]{MilnerRegnault1987}). As a matter of fact, liaison in the second case (N < Adj) is absent from everyday, unmonitored speech, although recurrent in formal contexts, where it is likely to be a hypercorrection effect. It is no more than frequent in the first case (Adj < N), where the crucial factor seems to be that adjective-noun collocations often refer to what may be conceived of as semantic units, kinds of noncompositional compounds -- not the case of blind scientist(s), but obviously of 〈un/des petit(s) ami(s)〉 ‘(a) boyfriend(s)’ (not ‘(a) small friend(s)’) nearly always pronounced /ɛp(ə)titami/ and /dep(ə)tizami/.\footnote{Let it be said once and for all that schwas are unstable in spoken French, so that I will henceforth dispense with bracketing.} Other types of prescriptively possible liaisons are little more than grammatical ghosts, occasionally revived by literary and poetic diction, e.g. between subject and verb as in 〈les savants ont dit〉 ‘the scientists said’ pronounced /lesavãzɔdi/ instead of (usual) /lesavãɔdi/ (see \citealt[51]{MilnerRegnault1987}).\footnote{As hinted at in the text, hypercorrection, a rampant phenomenon in literate societies, often plays havoc with all these data.}

Two conclusions, I think, come out of these facts. The first is that spoken French liaison is too erratic and piecemeal a phenomenon to allow for any generalization about plural marking, especially considering that spoken French morphophonology is acquired by children before they enter school, thus precluding any interference from orthography such as may and does affect older French speakers. On the other hand, frequency data about effectively realized liaisons seem to support the assumption that determiners do include /z/ in their plural forms. Yet, given all other facts, this would suggest allomorphy -- e.g. /le/ ‘the\textsubscript{PL}’ before consonant vs. /lez/ ‘id.’ before vowel -- rather than suffixation as a minimal hypothesis. The same data do not support the notion that nouns and nonpreposable adjectives (plus a fair contingent of preposable ones) are pluralized via mostly ‘mute’ s-suffixing in spoken French. Hence the plausible inference is that plural -s is not part of the active competence of spoken French speakers.

My second objection rests on what I will call the ‘stray \textit{z}’ illustrated by such examples as \textit{dez=avjɔ-a-reaksjɔ z amerikɛ} (〈des avions à réaction z-américains〉) ‘American jet-planes’ (\cites{Morin2005}[321]{MorinKaye1982}), where no 〈s〉 grapheme corresponds to spoken French \textit{z}.\footnote{I leave aside the possible compound-internal liaison (/dezavjɔ(z)aʁeaksjɔ/), a domain where variation is rife.} In spoken French registers monitored by Written French knowledge -- what one may call ‘Spoken Written French’, another name for what is usually considered Standard French -- such stray \textit{z}’s are severely stigmatized and banned. In relaxed spoken French they are, if not frequent, at least not uncommon.

True, Morin and Kaye’s example suggests an analysis such that \textit{z} would be the plural marker attached to the compound 〈avion à réaction〉 ‘jet plane’ and suffixed to the second, dependent term. Yet, the few cases where number can be decided without recourse to spelling show that compounds of the [N PP] type do not pluralize in this way: cf. \textit{yn=fjɛvr\_də=ʃəval} 〈une fièvre de cheval〉 ‘a raging fever’ (lit. ‘a horse fever’) vs. \textit{de=fjɛvr\_də=ʃəval} 〈des fièvres de cheval〉, not *\textit{də=ʃəvo} 〈de chevaux〉, only possible with the literal, compositional meaning ‘fevers that horses have’. Moreover, stray \textit{z} occurs in contexts where plural suffixing -- assuming it for a moment to be a feature of spoken French -- is out of the question. Thus it is heard slipping between con\-so\-nant-fi\-nal numerals and vowel-initial nouns as in \textit{sɛk z ãfã} (〈cinq z-enfants〉) ‘five children’, \textit{x z ãfã} /iksœzãfã/ (〈x z-enfants〉) ‘x children’; or between prefixes and bases as in \textit{de=mini-z-ordinatœr} (〈des mini-z-ordinateurs〉) ‘minicomputers’, \textit{de=sibɛr-z-atak} (〈des cyber-z-attaques〉) ‘cyberattacks’; or following a plural quantifier as in \textit{tro\_də z ãfã} (〈trop de z-enfants〉) ‘too many children’ (\citealt[323]{MorinKaye1982}; personal observation for \textit{sibɛr-z-atak}).

On the other hand, several facts demonstrate that \textit{z} \textit{does} mark plurality. In \textit{lœr z ami} (〈leurs ami.e.s〉) ‘their friends’ it is the sole clue to the plurality of the phrase as compared with \textit{lœr ami} (〈leur ami.e〉) ‘their friend’. \citet{Gougenheim1938} points out that liaison is possible in 〈des nez aquilins〉 /dene(z)akilɛ/ ‘aquiline noses’ (despite prescriptive ban), but excluded in 〈un nez aquilin〉 /ɛne(*z)akilɛ/ ‘an aquiline nose’. This is in accordance with the rule that in [N Adj] phrases a final ‘latent’ consonant in the noun never links to a following vowel -- cf. 〈un savant aveugle〉 /ɛsavã(*t)avœgl/ ‘a blind scientist’ -- unless it corresponds to written plural 〈s〉. Interestingly, standard spelling does not notate plurality in nouns ending in 〈s〉, 〈x〉 or 〈z〉 in the singular, e.g. 〈radis〉 ‘radish(es)’, 〈croix〉 ‘cross(es)’, 〈nez〉, etc., in which graphic consonants are actually never given phonological substance.\footnote{Except in loanwords such as 〈fez〉 /fɛz/ ‘fez’ and 〈merguez〉 /mɛʁgɛz/ ‘spicy beef or mutton sausage’. This was already true in Old French, where final consonants were pronounced and words like 〈cors〉 /kɔrs/ ‘body’ did not inflect for plurality, just like Modern French 〈corps〉 /kɔr/. For some reason, Old French never had recourse to inserting a vowel as in Catalan \textit{mes} ‘month’ vs. \textit{mesos} ‘months’ and it balked at spellings like *〈corss〉.} This suggests that the /z/ that shows up in plural contexts such as \textit{lez=ãfã} in  spoken French must be the same /z/ as in \textit{sɛk z ãfã} ‘five children’, with the significant difference that it is then sanctioned by Written French grammar.

\citet{Gougenheim1938}, followed by \citet{Morin2005} and \citet{MorinKaye1982}, analyses \textit{z} as a plural prefix optionally attaching to vowel-initial semantically plural nominals when preceded by a plural item. Precedence may be immediate as in \textit{sɛk (z) ãfã}, or it may be at a distance as in \textit{dez=avjɔ-a-reaksjɔ (z) amerikɛ} or \textit{de=sibɛr-(z-)atak}, where \textit{de(z)} ‘some’ is the plural item separated from the plural noun by items unspecified for number (\textit{reaksjɔ}, \textit{sibɛr}). Because of such possible discontinuities, I find it more adequate to view \textit{z} as an inflectional phrasal proclitic, analogous to the English genitive phrasal enclitic \textit{’s}. Only between determiners and a following V-initial first NP element is \textit{z} insertion categorical: /dezavjɔ/ ‘airplanes’, not */deavjɔ/. Shall we conclude that with determiners \textit{z} is indeed a suffix corresponding to orthographic 〈s〉? That in other words, but for the invariability of the noun, /lezami/ ‘the friends’ is structurally identical to its Languedocian Occitan counterpart \textit{los amics} /lo-s=amik-s/ [luzaˈmits]? I do not think so. In Languedocian Occitan the suffix is realized no matter what the phonological rightward context may be -- cf. \textit{los companhs} [luskumˈpans] ‘the pals’ -- and it constitutes the sole exponent of plurality: compare \textit{lo companh} [lukumˈpan] ‘the pal’.\footnote{Before non-stops \textit{s} is realized as [j] : cf. \textit{las filhas} [lajˈfiʎas] ‘the girls’, \textit{las serps} [lajˈsɛrps] ‘the snakes’ (\citealt[39]{Alibèrt2000}).} In spoken French, plurality is expressed first and foremost by formal suppletion of the determiner itself (\textit{lə}/\textit{la} vs. \textit{le}, \textit{sə}/\textit{sɛt} vs. \textit{se}, \textit{mɔ}/\textit{ma} vs. \textit{me}, etc.) while \textit{z} only appears before vowels and vowel-like approximants. Parsimony therefore commands us, it seems, to view it as the same proclitic as in the other cases, except for its near-obligatoriness (see \citealt{Mallet2008}), hence my spoken French notation \textit{le=z=ami} \{le\textsubscript{DET.PL}=z\textsubscript{PL}=ami\} ‘the friends’. That the conditions on \textit{z} insertion are complex ought not to come as a surprise, I believe, for they result from a lengthy process involving several intersecting changes and starting at least as soon as the sixteenth century, as we shall see later on (\sectref{sec:kihm:5}).

To sum up, spoken French diverged radically from Written French and the other Western Romance languages in that most nouns and adjectives do not inflect for number, the singular/plural contrast being marked instead by the suppletive alternation of the determiners included in the NP (or DP), with the upshot that NPs are unspecified for the feature when no determiner is present, e.g. in locutions such as  \textit{fɛr\_kado} (〈faire cadeau〉) ‘to give as a present’ or enumerations such as \textit{vo, vaʃ, koʃɔ, kuve…} (〈veaux, vaches, cochons, couvées…〉) ‘calves, cows, pigs, clutches…’ (see \citealt{TasmowskiLaca2021}). The logic of privative oppositions indeed implies that \textit{kado}, \textit{vo}, etc. cannot be specified as singular, for there are no markedly plural forms to contrast them with. Morphosyntax being mute, only general semantics will tell us that \textit{kado} is to be understood as generic in the locution, whereas \textit{vo}, etc. refer to an indefinite plurality of exemplars of the concepts, a possible reading of bare nouns enumerations in French.\footnote{Unless the theoretical framework one is working in allows for ‘silent’ functional categories. I do not subscribe to this hypothesis. Neither do the authors cited in the text.} Also note that even in the presence of a determiner the number value of an NP may remain indeterminate ; this is the case with the portmanteau \textit{o} (〈au(x)〉) ‘to the’ when followed by a masculine NP whose first element begins with a consonant, as in \textit{o=profɛsœr} (〈au(x) professeur(s)〉) ‘to the teacher(s)’. Speakers often feel constrained to use some extralinguistic device to disambiguate the expression, such as ‘with an 〈s〉’ or ‘in the plural’.\footnote{This is reminiscent of the reported Chinese practice of virtually drawing ideograms with a finger in the palm of the hand, to show to the addressee when at risk of misunderstanding due to homophony. Parallels between French ortography and ideogrammatic writing have often been pointed out.}

As mentioned, the exception to number invariability consists of a set of masculine nouns and adjectives such as \textit{ʒurnal}\slash\textit{ʒurno} (〈journal〉\slash 〈journaux〉) ‘newspaper(s)’, \textit{reʒjonal}\slash\textit{reʒjono} (〈régional〉\slash 〈régionaux〉) ‘regional’, whose plural involves suppletion as well, since it proceeds via stem modification rather than suffixation. Moreover, since gender must be taken into account and exceptions to the exception (e.g. \textit{fɛstival} ‘festival’, \textit{tonal} ‘tonal’, \textit{pɔrtaj} ‘gate’, etc.) are numerous, no general rule such as ‘if singular \textit{Xal} or \textit{Xaj} then plural \textit{Xo}’ can be posited. The noun subset is closed, the adjective subset open (see recent \textit{manaʒerjal}\slash\textit{manaʒerjo} ‘managerial’).

Let me add that spoken French appears exceptional in this domain not only with respect to the other Romance languages, but also to Indo-European languages in general, none of which fail to mark plurality if only as the sole nominal inflection they exhibit (see English, Persian, etc.). Noninflectional number marking is by far not a rare feature, but one has to look outside Indo-European (and Afroasiatic, Uralic…) to see it regularly instantiated, which leads us to the next case.

\section{Plural marking in Fa d’Ambô}
\label{sec:kihm:3}

Fa d’Ambô is one of the four Portuguese-related Creoles spoken in the Gulf of Guinea, the other three being Santome and Angolar on the island of São Tomé, and Principense (aka Lung’Ie ‘Language of the Island’) on the island of Príncipe. Fa d’Ambô is spoken by about 6600 people on the island of Annobón -- hence its name meaning ‘Language of Annobón’ (\textit{fala de Anobom}) -- a province of the Republic of Equatorial Guinea 182km southwest of São Tomé. Unlike in the other two islands, the socially dominant language in Annobón is not Portuguese, but Spanish. This is owing to the fact that, after having been a Portuguese possession during three centuries from 1470 to 1778, Annobón passed into Spanish hands and became a Spanish colony. Yet, until the very end of the nineteenth century, Spain as a colonial power was little present in the island, in part because of fierce local opposition, so that the Spanish influence on Fa d’Ambô remained slight. Moreover, since 2010, Portuguese has been established as one of the official languages of Equatorial Guinea, next to Spanish and French.

European Portuguese is, like Written French, a Western Romance language where nominal plurality is marked by way of an -\textit{s} suffix that is always pronounced, [ʃ] or [ʒ] when implosive or before pause, [z] intervocalically: cf. \textit{as casas} [ɐʃˈkazɐʃ] ‘the houses’, \textit{as mulheres} [ɐʒˈmuˈʎɛr\textsuperscript{ə}ʃ] ‘the women’, \textit{os homens} [uzˈɔmɛjʃ] ‘the men’.\footnote{I specify “European” because Brazilian Portuguese is markedly different owing to the higher frequency of fully bare nouns (no determiner, no plural marking) (see \citealt{BritoLopes2016}).} Fa d’Ambô is another matter altogether (see \citealt[35--38]{HagemeijerEtAl2020}).

There is a basic divide in the nominal domain between nouns referring to animate (especially human) entities and other nouns, that plays no role in Portuguese.\footnote{Adjectives are invariable (\citealt[68]{HagemeijerEtAl2020}). Grammatical gender is not a property of Fa d’Ambô.} The former always show some plural marking, while the latter are mostly unspecified for number, except for a closed subset whose common property is beginning with a vowel. And there are two plural markers: preposed \textit{nan} /nã/ and prefixed \textit{z}{}-. \textit{Nan} is formally similar and historically related to the \textsc{3pl} pronoun \textit{ineyn} /inɛj/ ‘they’ and it combines with animate-referring nouns only: cf. \textit{nan mína } \{\textsc{pl} boy\} ‘the boys’, \textit{nan khasô} /nãxasó/ ‘the dogs’, \textit{zugwan nan mína} \{some \textsc{pl} boy\} ‘(some) boys’.\footnote{The acute accent notates high tone, usually corresponding to the etymon’s stressed syllable, here Portuguese \textit{cachorro} /kɐˈʃoʀu/ ‘puppy’ and \textit{menino} /məˈninu/ ‘boy’.} \textit{Z}{}-prefixing is not sensitive to animacy, only to whether the noun is vowel-initial or not; it is “far less frequent than \textit{nan}; native speakers of Fa d’Ambô may not even know this strategy” (\citealt{HagemeijerEtAl2020}: 35). Hagemeijer et al. give no more than seven examples of \textit{z}{}-plurals: \textit{aba}\slash\textit{zaba} ‘branch(es)’, \textit{alu}\slash\textit{zalu} ‘halo(s), \textit{ankha}\slash\textit{zankha} ‘buttock(s)’, \textit{atxi}\slash\textit{zatxi} ‘art(s)’, \textit{ome}\slash\textit{zome} ‘man\slash men’, \textit{ope}\slash\textit{zopi} ‘foot\slash feet’, \textit{ubêlê}\slash\textit{zubêlê} ‘udder(s)’.\footnote{〈tx〉 = /ʧ/, 〈ê〉 = /e/, 〈x〉 = /ʃ/.} They do not imply it is all there is, however.  \citet[90]{ZamoraSegorbe2007}, who includes the same list, explicitly mentions he is just giving “\textit{algunos ejemplos}”. At any rate, the sample he and Hagemeijer et al. provide is enough to show that, besides the phonological provision, \textit{z}{}-prefixing is arbitrary. In a few cases, as pointed out by \citet[144]{ZamoraSegorbe2007}, the \textit{z}{}-prefixed member of the pair has acquired a distinct meaning: according to him, \textit{zalu} does not mean ‘halos’, but ‘alcoholic drink(s)’ (Spanish \textit{trago(s)}).

As remarked by the authors, there is an obvious etymological relation between \textit{z}{}- and the voiced allomorph of the Portuguese plural suffix occurring between vowels: compare \textit{zome} ‘men’ with \textit{os homens} [uzˈɔmɛjʃ] ‘the men’. As shown by this example, \textit{z}’s precise origin is likely to be the -\textit{s} suffix of the Portuguese plural definite articles \textit{os}\textsubscript{M}/\textit{as}\textsubscript{F}, subsequently lost in creolization, for “Fa d’Ambô has no definite articles” (\citealt{HagemeijerEtAl2020}: 42). Interestingly Fa d’Ambô does not evidence the meaningless, merely phonetic /z/ agglutination that is so frequent in French-related Creoles, having the same origin: cf. Haitian \textit{zwazo} ‘bird’ < French \textit{les oiseaux} /lezwazo/ ‘the birds’, Mauritian \textit{zetwal} ‘star’ < French \textit{les étoiles} /lezetwal/ ‘the stars’, etc. There may be some correlation here, especially as agglutination to the stem of the Portuguese singular definite article is fairly common in contrast: cf. \textit{alê} ‘king’ < Portuguese \textit{o rei} ‘the king’, \textit{olemu} ‘paddle’ < Portuguese \textit{o remo} ‘the paddle’, \textit{ope} ‘foot’ < Portuguese \textit{o pé} ‘the foot’.\footnote{The pair \textit{ope}\slash\textit{zopi} betrays the action of analogy, since \textit{z}{}-affixation could only occur after the word had become vowel-initial owing to article agglutination.}

As a result of the animacy constraint and the constraints on \textit{z}{}-prefixing, con\-so\-nant-ini\-tial inanimate-referring nouns and a good number of vowel-initial ones as well are unspecified for number unless they are accompanied by a demonstrative determiner: cf. \textit{djendja pi nen xi} \{banana unripe \textsc{pl} \textsc{dem}\} ‘these unripe bananas’, \textit{nen} the dedicated plural marker for determiners.

\section{Plural marking in Réunion Creole}
\label{sec:kihm:4}

Réunion Creole is a French-related creole spoken on Reunion Island, a French \textit{département} part of the Mascarenes Archipelago in the Indian Ocean (see \citealt{Bollée2013}).\footnote{I am grateful to an anonymous reviewer whose comments led to substantial revision of this section.} It started forming by the end of the seventeenth century and has been (controversially) defined as a semi-creole owing to various features that bring it closer to its lexifier than are its geographical neighbours Mauritian and Seychelles Creole, also French-related (see \citealt{Holm2004}). For instance, Réunion Creole kept a gender contrast apparent in definite NPs, such that masculine singular nouns select the definite determiner \textit{lë} /lə/ or \textit{lo} -- e.g. \textit{lë {\textasciitilde} lo syin} ‘the dog’ (F \textit{le chien}) -- and feminine singular nouns select \textit{la} -- e.g. \textit{la kaz} ‘the house’ (F \textit{la case}) (\citealt[355, 2007]{Chaudenson1974}; and see \citet{Albers2020} for a prosodic constraint on \textit{lë} {\textasciitilde} \textit{lo} attachment which complexifies the picture to some extent). The indefinite determiner is \textit{in} /ɛ/ ‘a(n)’, and it does not discriminate gender (\textit{in syin} ‘a dog’, \textit{in kaz} ‘a house’).

Descriptions disagree as far as plural NPs are concerned. According to \citet{Bollée2013}, “[b]are nouns can have plural meaning, but plural is mostly marked with \textit{le} for inanimates and \textit{bann} for animates, e.g. \textit{le ros, le gale} ‘the stones’, \textit{bann zanfan, bann marmay} ‘the children’. The distinction is, however, not clear-cut, \textit{le} can also be used with animates: \textit{le zanfan} ‘the children’.” Plural \textit{le} /le/ is epicene. And she gives \textit{de} as the plural of \textit{in}: \textit{Na de personn ki…} ‘There are people who…’. Except for \textit{bann} /bãn/ this looks very much like spoken French: cf. \textit{le=rɔʃ} (〈les roches〉) ‘the rocks’, \textit{le=z=ãfã} (〈les enfants〉) ‘the children’, \textit{Ja de=persɔn ki…} (〈Il y a des personnes qui…〉) ‘There are people who…’. \citet[358]{Chaudenson1974}, in contrast, insists that definite masculine NPs such as \textit{lo gran syin} can be singular or plural, meaning ‘the big dog’ or ‘the big dogs’. Definite feminine NPs such as \textit{la ti fiy} ‘the little girl’, on the other hand, can only be singular. Plurality is then expressed through \textit{bann}, optionally with masculine NPs, should disambiguation be felt necessary, obligatorily with feminine NPs: \textit{lo (bann) gran syin} ‘the big dogs’, \textit{lo bann ti fiy} ‘the little girls’. For Chaudenson, the use of plural \textit{le} is severely limited, appearing almost exclusively in place names, and feminine \textit{la} is replaced in the plural by nonfeminine \textit{lo} {\textasciitilde} \textit{lë}. Finally, Chaudenson does not mention \textit{de}; he says that indefinite plurality is expressed by combining \textit{in} with \textit{bann}: \textit{in bann syin} ‘(some) dogs’ \citep[358]{Chaudenson1974}.

The complex Reunionese sociolinguistic situation explains these discrepancies. That is, Bollée’s and Chaudenson’s analyses are probably both right, but they address different stages and/or registers of the language. Providing the correct definitions is also an issue. For instance, \textit{bann syin} meaning ‘the dogs’ and \textit{in bann syin} ‘(some) dogs’ seems to imply that \textit{bann} actually realizes the feature set [\textsc{spec} + \textsc{num} \textit{pl}], and that it is a specific plural marker, hence compatible with definiteness (expressed or not) as well as indefiniteness.

Where both authors agree, albeit negatively, is in not mentioning another phenomenon related to the expression of plurality, namely the existence of a small and apparently closed group of /l/-initial nouns possibly showing a /z/-initial alternant when plural \citep[68--69]{Caron2011}. Whatever its morphological status -- an issue I take up presently -- this /z/ is obviously related to the spoken French \textit{z} proclitic examined above. Caron’s analysis brings a number of differences to light, however. Réunion Creole /z/ does not appear in nouns with a vowel-initial stem, but in nouns whose stem begins with an /l/ resulting from amalgamation of the French singular definite article preceding a vowel-initial noun, in such a way that /z/ replaces /l/: e.g. \textit{lékol} ‘school’ (\textit{l=ekɔl} 〈l’école〉 /lekɔl/ ‘the school’) vs. \textit{zékol} ‘schools’ (\textit{lez=ekɔl} 〈les écoles〉 /lezekɔl/) ‘the schools’), \textit{linfirmyé} ‘male nurse’ (\textit{l=ɛfirmje} 〈l’infirmier〉 /lɛfiʁmje/) vs. \textit{zinfirmyé} ‘male nurses’ (\textit{les infirmiers} /lezɛfiʁmje/ ‘the nurses’), \textit{linstititér} ‘teacher’ (\textit{l’instituteur} /lɛstitytœʁ/ ‘the teacher’) vs. \textit{zinstititér} ‘teachers’ (\textit{les instituteurs} /lezɛstitytœʁ/ ‘the teachers’).\footnote{The device was also noted by \citet[78]{Corne1999}: “Some nouns have different initial consonants (\textbf{zaṅfaṅ}, \textbf{laṅfaṅ}, \textbf{naṅfaṅ} ‘child’), \textbf{z}{}- allowing a plural reading when no article is present, as in \textbf{mweṅ} \textbf{na} \textbf{zaṅfaṅ} \textbf{i} \textbf{sar} \textbf{lekol} ‘I have children who go to school’.” For \citet{Caron2011} \textit{zanfan} does not belong to the same group as \textit{lékol}/ \textit{zékol}.} Nouns whose initial /l/ or /z/ belongs to the stem of the French etymon, e.g. \textit{lima} ‘slug(s)’ (\textit{limace}), \textit{langaz} ‘language(s)’ (\textit{langage}), \textit{ziz} ‘judge(s)’ (\textit{juge}), \textit{zol} ‘jail(s)’ (\textit{geôle}), etc. are not concerned. If /l/-initial, they pluralize (or not) like nouns with different initial onsets; if /z/-initial, they do not pair with /l/-initial singulars, being singular by default. Nor do all nouns whose initial /l/ or /z/ results from determiner amalgamation belong to the group: cf. \textit{loto} ‘car(s)’ (\textit{l’auto}), \textit{zabèy} ‘bee(s)’ (\textit{les abeilles}). In some cases, differential amalgamation of French \textit{l’} or \textit{(le)s} resulted in distinct lexemes: e.g. \textit{lavoka} ‘lawyer(s)’ (\textit{l’avocat}) vs. \textit{zavoka} ‘avocado(s)’ (\textit{les avocats}), \textit{listwar} ‘history/ies’ (\textit{l’histoire}) vs. \textit{zistwar} ‘story/ies, tale(s)’ (\textit{les histoires}), etc.

\citet{Caron2011} analyses alternating /l/ and /z/ as two prefixes, one singular, the other plural. The problem with this analysis is variation. For no speaker, it seems, does the /l/-initial form invariably express a singular meaning; nor does the /z/-initial form always mean plurality. It seems entirely to be a matter of implicit preference or wont, of varying strength depending on speakers and individual lexemes. In other words, Réunion Creole /z/ appears not to have grammaticalized as Fa d’Ambô /z/ has. A more adequate analysis of the /l/-/z/ alternation is therefore to view it as optional stem allomorphy related to number. Referring to one teacher you will use the /l/-initial stem: \textit{lë {\textasciitilde} lo linstititér} ‘the teacher’, \textit{in linstititér} ‘a teacher’; referring to several you have a ‘choice’ between \textit{lo {\textasciitilde} le bann zinstititér} and  \textit{lo {\textasciitilde} le bann linstititér} ‘the teachers’, \textit{in bann zinstititér} and \textit{in bann linstititér} ‘(some) teachers’ (following Chaudenson).

Diachronically, a necessary (but by no means sufficient) condition for the optional allomorphy to emerge was of course the availability in the spoken French input of singular/plural pairs of definite NPs including vowel-initial nouns such as \textit{l=ekɔl} vs. \textit{lez=ekɔl} (see \sectref{sec:kihm:5.1}). In a number of words the singular article \textit{l} then fused into the stem, while the plural article \textit{lez} (see \sectref{sec:kihm:2}) reduced to /z/, which also fused into the stem. Hence two stems for one lexeme, which could but need not be used differentially according to number. Words where /l/ and /z/ (or /ʒ/ > /z/) were already in the stem, such as \textit{la=limas} vs. \textit{le=limas} ‘the slug(s)’, \textit{lə=ʒyʒ} vs. \textit{le=ʒyʒ} ‘the judge(s)’, were of course exempt from the process. Why did only a rather small subset of eligible nouns undergo the process? I have no answer to this question, and I doubt any can be found.

In a synchronic grammar of Réunion Creole aiming to somehow modelize native speakers’ competence, on the other hand, no distinction can be made, it seems, between etymological initial /l/ or /z/ as in \textit{lima} and \textit{ziz} and initial /l/ or /z/ corresponding to (part of) the French definite article – unless, that is, it could be demonstrated that Réunion Creole speakers, being as a matter of fact all bilingual in French, do relate \textit{linstitité}\slash\textit{zinstitité} and \textit{l’instituteur}/\textit{les instituteurs} within a putative department of their internalized grammars where Réunion Creole and French grammars intersect. Pending this demonstration, the minimal assumption is that stem-allomorphic nouns like \textit{linstitité}/\textit{zinstitité} constitute a group -- a neutral term I prefer to ‘class’ given lack of grammaticalization -- whose membership seems to be unpredictable -- like for the \textit{ankha}/\textit{zankha} class in Fa d’Ambô -- so that its members have to be listed.

\section{\textit{Z}-plural as convergent evolution}
\label{sec:kihm:5}

Let me list the conditions the emergence of a given feature φ in languages L\textsubscript{1}, L\textsubscript{2},…L\textsubscript{n} must satisfy in order to qualify as the product of convergent evolution: (i) the languages share a known last common ancestor (LCA) or they do not; (ii) if they do, the LCA does not include the feature; (iii) the languages evolved independently as far as the feature is concerned; (iv) the instantiations of the feature in the different languages are analogous in terms of form and/or function (full identity is not required).\footnote{To take a standard example, the wings of birds and bats serve the same function (flying), they show roughly resembling shapes, but they are otherwise quite different.}

In the case at hand, the LCA of present-day spoken French and Réunion Creole is some variety of seventeenth century spoken French which did not yet include \textit{z}{}-plurals but was on its way to acquiring them as a consequence of changes taking place in the previous century (see below). However, the evolution that led from seventeenth century spoken French to Réunion Creole is clearly not a continuation of the evolution leading from Old French to seventeenth century, then modern spoken French. Starting from their LCA, spoken French and Réunion Creole evolved independently. With Fa d’Ambô, one has to go back much farther into history to find the LCA it shares with spoken French and Réunion Creole, which can be no other than Late Latin, the common ancestor of all Romance languages. It seems safe to assume that Late Latin had no \textit{z}{}-plurals. Neither does any variety of Portuguese to this day. Yet Portuguese does include the necessary condition for \textit{z}{}-plural to emerge, just as seventeenth century French did.

\subsection{From Middle French to present spoken French}
\label{sec:kihm:5.1}

Middle French (sixteenth to early seventeenth century) is characterized by an overall drift toward CV syllabification, which entailed widespread dropping of the codas that were pronounced in Old French (eleventh to fifteenth century): “les consonnes finales se prononcent devant un mot commençant par une voyelle… et à la pause…; elles ne se prononcent pas devant une consonne” \citep[31]{Gougenheim1951}. Plural -\textit{s} did not escape this process despite its morphosyntactic weight. In \textit{Nous sommes tous tenus de prier Dieu} ‘We all are required to pray to God’ \citep[32]{Gougenheim1951}, the final \textit{s} of \textit{tenus} ‘required’ was pronounced because it stands at the end of a syntactic group and before a possible pause, whereas every other final \textit{s} was mute, irrespective of whether it means plurality as in \textit{tous} ‘all’ or not as in \textit{nous} `we' and \textit{sommes} ‘are’, hence /nusɔmtutənys\#dəprijedjø/.\footnote{/nusɔmtustəny\#dəpʁijedjø/ in present spoken French, for the pause condition no longer holds and the final \textit{s} of \textit{tous} ‘all’ lexicalized.} \textit{Des savants anglais} ‘(some) English scientists’ was pronounced /desavãzãglɛ/ as it still can be, although in a now definitely stilted style as compared to usual /desavããglɛ/. That \textit{z}{}-plural as in present spoken French was not yet part of the language, however, is suggested by two facts.

First, one finds no trace of stray \textit{z} -- not surprisingly though, as it is an entirely oral phenomenon not liable to be registered in texts. More serious is the pause condition mentioned by Gougenheim, which must have endowed plural \textit{s} with a modicum of reality, although weakened by its frequent dropping following final /ə/, still pronounced in Middle French and well into the seventeenth century: cf. \textit{des prunes blanches et noires} ‘white and black plums’ /deprynəblãʃɛnwɛrə/ instead of /deprynəblãʃəzɛnwɛrəs/, a pronunciation ascribed to “\textit{les dames de Paris}”, but probably more widespread \citep[33]{Gougenheim1951}.\footnote{/depʁynblãʃɛnwaʁ/ in Present spoken French. The role of women in the spread of language innovations is well documented in sociolinguistic studies (see \citealt{Labov2001}).}

Be it as it may, one may hold the second condition for convergent evolution to be fulfilled: insofar as seventeenth century spoken French did not yet differ significantly from its sixteenth century predecessor as far as plural marking is concerned (see \citealt{Brunot1939}), one is warranted to assume that the LCA of Present spoken French and Réunion Creole did not include \textit{z}{}-plural, being thus close to the high register of Present spoken French. The crucial change leading to the current state of affairs occurred when the category of ‘optional’ (or possible) liaison for all intents and purpose disappeared from the colloquial registers of spoken French, so that \textit{z}{}-liaison only remained in those contexts where its absence means ungrammaticality (see \citealt{Encrevé1988, BonamiEtAl2005}). Now these contexts -- i.e. [\textsubscript{DP} D NP] where the NP’s first element begins with a vowel or vowel-like approximant as in \textit{le=z=ami} (\textit{les ami.e.s}) ‘the friends’ and \textit{le=z=wa} (\textit{les oies}) ‘the geese’ -- are precisely the contexts where \textit{z} is redundant as a plural marker, since plurality is also expressed suppletively by the determiners, as shown by \textit{lə}\textsubscript{M.SG}/\textit{la}\textsubscript{F.SG} vs. \textit{le}\textsubscript{M/F.PL}. At the same time, within the global signal for plurality, /z/ is clearly more salient than the vowel alternation. Such a combination of redundancy with salience was, I assume, the crucial factor that triggered \textit{z}’s ‘release’. It was helped by two other uses the phoneme /z/ has in the language.

One is as a paragogic consonant~as in \textit{Donne-moi-z-en !} \{give.\textsc{imp}{}-me-\textit{z}{}-of.it\} ‘Give me some!’, a stigmatized but common variant of standard \textit{Donne m’en} ‘id.’.\footnote{\citet{Morin1979,Morin2005} analyses /zã/ and /zi/, etc. as allomorphs of the clitic pronouns \textit{ã} (\textit{en}) and \textit{i} (\textit{y}).}  Wishing to form the imperative of \textit{Tu l’y emmènes} \{you it there take\} ‘You take it there’, one has little choice but \textit{Emmène-le-z-y !} \{take.\textsc{imp}{}-it-\textit{z}{}-there\} /ãmɛnləzi/ ‘Take it there !’, for */ãmɛnləi/ violates the constraint preventing schwa from appearing before vowels, while elided~/ãmɛnli/, although acceptable, sounds quaint (\citealt[87]{MilnerRegnault1987}).\footnote{The marginally possible variant \textit{Emmène-z-y-le} /ãmɛnzilø/ also requires /z/.} Paragogic /z/ is sometimes registered by orthography, as in \textit{Vas-y !} \{go.\textsc{imp}{}-there\} /vazi/ ‘Go there !’, where the 〈s〉 realized /z/ is deprived of functional or etymological reality (also see \citealt{Frei1929/2007}).\footnote{Paragogic /z/ was more widespread in seventeenth century French than it now is \citep{Brunot1939}. It is also common in other Oïl languages (often but improperly called ‘French dialects’) such as Poitevin as in \textit{Doune me z-ou!} \{give.\textsc{imp} me \textit{z}{}-it\} /dunməzu/ ‘Give me it!’ (Arantéle) or \textit{i sé pa s’i va o-z-aconsentir} \{I know not whether I go it-\textit{z}{}-accept\} /isepa siva ozakɔsãtir/ ‘I don’t know whether I’m going to accept it’ \citep[29]{Gautier1986}.}

Secondly, there is the frequent liaison triggered by nonplural /z/, obligatorily as in \textit{deux amis} /døzami/ ‘two friends’ (compare \textit{deux copains} /døkopɛ/ ‘two pals’) or optionally – and therefore excluded from ‘basilectal’ spoken French – as in \textit{J’avais un rêve} /ʒavɛ(z)ɛrɛv/ ‘I had a dream’ (compare \textit{J’avais des rêves} /ʒavɛderɛv/ ‘I had dreams’) \citep{Encrevé1988}.

Paragogic and nonplural liaison /z/ share meaninglessness. And they are likely to be connected in the sense that the frequency of the latter, also greater in the seventeenth century than now, may have been instrumental in the ‘choice’ of the former (next to less used /t/). Plural \textit{z} may thus be seen as an extension of this latent \textit{z} spoken French speakers had in their competence; all they had to do was provide it with a meaning, i.e. add the morphosyntactic feature [\textsc{num} \textit{pl}] to the phonological pattern [XV\#VY → XVzVY]. The process took place within a general drift toward noun inflectional invariability for number, itself related to the growing amount of cases where the plural suffix lacked a phonological counterpart owing to the disuse of optional liaison.

\subsection{From Middle Portuguese to Fa d’Ambô}
\label{sec:kihm:5.2}

The evolution from Middle Portuguese to Fa d’Ambô was just as catastrophic, but perhaps not so eventful. In the Portuguese input to Fa d’Ambô -- or Proto-Gulf-of-Guinea (see \cites[37--38]{HagemeijerOgie2011}{Bandeira2017}{Rougé2018}) -- there was no question of final consonant deletion as in  Middle French; stem-final plural -\textit{s} was well present. Nevertheless it was retained in none of the ensuing Gulf or Guinea Creoles (GGC).\footnote{\label{fn:kihm:29}This is in sharp contradistinction to the Upper Guinea Creoles (Guinea-Bissau-Casamance Kriyol, Cape Verdean) that retained -\textit{s} as a pluralizing suffix.} This is likely to be due to the interplay of two fundamental discrepancies, one phonotactic, the other morphosemantic, between the internalized grammars of the founders of GGC, that is Middle Portuguese speakers on the one hand and speakers of Volta-Niger and Bantu languages on the other hand (see \citealt{Mufwene1996} for the “founder” notion).\footnote{Volta-Niger, a subfamily of Volta-Congo, includes Edo, Fon, Yoruba, etc. (\citealt{WilliamsonBlench2000}). One has no choice but to assume that these languages as well as the Bantu languages involved were not significantly different in the fifteenth century from what they now are.} The phonotactic discrepancy has to do with the possible occurrence of word-final consonants. Middle Portuguese does not tolerate plosives in that position, but /s/, /z/, /r/, /l/ and the nasal archiphoneme are acceptable.\footnote{Owing to variable final schwa deletion, Modern European Portuguese does show word-final plosives as in \textit{leque} ‘fan’ pronounced [lɛk], whereas Middle Portuguese only allowed [ˈlɛke] \citep{Teyssier1980}. It is debatable, however, whether final schwa is ever ‘underlyingly’ absent in any register of Modern European Portuguese (see \citealt[364]{Mateus1989})} In Volta-Niger and Bantu languages, in contrast, CV, V possibly nasalized, is the only legitimate syllable type. Speakers of these languages in contact with Portuguese were therefore predisposed to discard word-final /s/ along with the other consonants in the same position -- as they consistently did in all GGCs -- in order to align Portuguese words with their own syllabic patterns.\footnote{CVC is a common syllabic type in the Atlantic languages, in contrast, with the correlate that Upper Guinea Creoles retained word-final consonants, including /s/ (see footnote~\ref{fn:kihm:29}).}

Concerning the morphosemantics of number marking, European Portuguese is, like English, a language such that the morphological pluralization of count nouns by and large follows the arithmetic principle $\{|n| > 1 \Rightarrow  \text{plural}\}$, with the consequences that (i) nonpluralized count nouns are interpreted as singular by default; (ii) plural marking often appears redundant or useless, because plurality either is externally indicated (e.g. \textit{dez maçãs} ‘ten apples’, \textit{muitas maçãs} ‘many apples’), or is obvious given neurotypical world knowledge, or is not really important in the current exchange and could easily be passed over -- but cannot be because of grammatical strictures; (iii) given the interplay of number and definiteness, the occurrence of bare singular count nouns is severely limited to a few contexts, for instance proverbs such as \textit{Cão que ladra não morde} \{dog that barks not bites\} ‘Barking dogs do not bite’.

Owing to their noun-class systems that regularly distinguish singular from plural in count nouns, Bantu languages seem to be like European Portuguese and English in this respect (see \citealt{NursePhilippson2006}). Not so Volta-Niger languages, however, the other founding contact for Fa d’Ambô. Yoruba, for instance, shows the same animacy constraint on plural marking as does Fa d’Ambô, only not so strict because it interacts with a discourse relevance constraint and an ontological hierarchy that places humans on top and things at the bottom. That is to say, count nouns denoting human or ‘high’ animate entities (e.g. livestock) are regularly marked for plurality when several individuals are being referred to \textit{qua} individuals, rather than collectively as a group, and mentioning it is considered important -- usually the case when discussing humans (see \citealt{Cartwright1979, Corbett2000, KwonZribi-Hertz2006}). The plurality marker is then preposed \textit{àwọn} /a\textsuperscript{L}wɔ\textsuperscript{M}/ which happens to be the emphatic \textsc{3pl} personal pronoun as well: \textit{àwọn ọmọ} ‘(the) children’. In contrast, count nouns denoting “lower” animates (e.g. ants) or things are ordinarily unspecified for number and therefore unmarked for the feature: cf. \textit{ìwé mi dà?} \{book my be.where\} ‘Where is/are my book(s)?’ \parencites[41]{Rowlands1969}{Bamgboṣe1966}.

Such formal and semantic matches leave little doubt about the Fa d’Ambô dominant plural strategy being due to the influence of one or several Volta-Niger languages present on the language stage where Fa d’Ambô emerged. This influence was given free rein once contact-induced phonotactic constraints had made Portuguese lexemes invariable. To quote \citet{Post2013}, “Plural is normally ∅-marked… It is expressed only if necessary”. So much is true of the other GGCs, except that they have no strategy besides the Volta-Niger one. What about the Fa d’Ambô \textit{z}-strategy then?

As mentioned in \sectref{sec:kihm:3}, prefixed \textit{z}{}- can only proceed from the voiced allomorph of the Portuguese plural suffix of determiners when preceding vowel-initial words: \textit{as ancas} /ɐzˈɐkɐʃ/ > \textit{zankha} /ˈzãxa/. Clearly, intervocalic position prevented /z/ from undergoing the phonotactic constraint, since it could be and was resyllabified with the following vowel onset. Here Fa d’Ambô and spoken French show convergent evolution, except that spoken French \textit{z} stopped en route to becoming a full prefix because number-inflected determiners remained: cf. \textit{les anches} /le=z=ãʃ/ ‘the reeds [musical]’. Now the most likely explanation, I believe, for the fact that this circumstance was exploited at all as a way of marking plurality is the presence in Annobón of a significant Bantu substrate, weaker or altogether lacking in the other islands \citep{Ferraz1984}.\footnote{Bantu influence is also noteworthy in Angolar, but it seems mainly to consist in numerous lexical borrowings with no structural consequences (see \citealt{Maurer2013}).} Thanks to this substrate, resyllabifiable /z/, orphaned from its vanished original bearer, could be reanalysed as a plural prefix analogous to the noun class plural prefixes of, e.g., Kikongo: cf. \textit{ngulu}\slash\textit{zingulu} ‘pig(s)’.\footnote{Noun classes 9/10. The reason should be obvious why I picked out this pair or gender out of the seven or so genders of ‘ethnic’ Kikongo.}  Why wasn’t the device exploited more than it actually was? There is at least one obvious cause: its only being serviceable with vowel-initial nouns. And Bantu probably was less influential than Volta-Niger.

\subsection{From seventeenth century spoken French to Réunion Creole}
\label{sec:kihm:5.3}

To some extent the evolution from seventeenth century spoken French to Réunion Creole runs parallel to that of Middle Portuguese to Fa d’Ambô. On the one hand, one sees the same change from plural marking according to the arithmetic principle to reference-driven pluralization. The change was much less pervasive in Réunion Creole than in Fa d’Ambô, however, owing to the maintenance in the former of the French number-inflected definite articles: \textit{le}\textsubscript{M.SG} > \textit{lë}, \textit{la}\textsubscript{F.SG} > \textit{la}, \textit{les}\textsubscript{PL} > \textit{le}, which, together with indefinite \textit{in} ‘a’ and \textit{in bann} ‘some’, account for the limited contexts allowing bare nouns possibly ambiguous as to number. Yet Réunion Creole bare nouns, if less widespread than in Fa d’Ambô, occur in contexts where all varieties of Modern French reject them. \citet{Chaudenson2007} gives a review of these contexts, e.g. genericity as in \textit{Koson i manz pa lswar} \{pig \textsc{pm} eat \textsc{neg} at.night\} ‘Pigs do not eat at night’, where French  requires plural definiteness: \textit{Les cochons (ne) mangent pas le soir} \citep[230]{Chaudenson2007}.\footnote{Alternatives are singular definite \textit{Le cochon…} and singular indefinite \textit{Un cochon…}, all ambiguous between being interpreted as generic or specific.\label{fn:kihm:35}} Other clearly contrastive contexts involve anaphoricity, as in \textit{Dokter la di ael…} \{doctor \textsc{prf} say \textsc{3sg.obl}\} ‘The doctor(s) told him…’ (\textit{Le(s) docteur(s) lui a/ont dit…}); and complements of nouns, prepositions or verbs: cf. \textit{su l bor semin} \{on \textsc{def.sg} side road\} ‘on the roadside(s)’ (\textit{sur le(s) bord(s) de la route}), \textit{dan bitasyon} ‘in the fields’ (\textit{dans les champs}, \textit{aux champs}), \textit{fé kwi manzé} \{make cook food\} ‘to prepare the food’ (\textit{faire cuire le manger}) \citep[231--234]{Chaudenson2007}.

\citet{Chaudenson2007}, quoting \citet{Valli1994}, views Réunion Creole as being closer in this respect to Middle than to Modern French. Yet, previously uninhabited Reunion Island, although appropriated by the French in 1640, was not permanently settled before 1665 \citep{Bollée2013}, that is a half-century after the end of the Middle French period. Moreover, already during that period, “D’une façon générale, l’usage de l’article défini se développe au cours de la seconde moitié du siècle” 'Generally speaking, the use of the definite article developed during the second half of the century.' \citep[64]{Gougenheim1951}. In fact, for all the examples above, seventeenth century French had become similar to Modern French \citep{Spillebout2000} -- unless the dialects spoken by the first French colonists had preserved archaic features not reflected in the texts, a possibility not to be dismissed.

As for the other component of the feature pool, finding a possible origin to the frequency of bare nouns and the consequent number underspecification proves more problematical than with Fa d’Ambô. Réunion Creole’s reputedly main substrate language, Malagasy, does not pluralize NPs at all unless they happen to include a number-inflected demonstrative \citep[41--44]{Dez1980}, whereas the African slaves massively transported after 1760 hailed from the Eastern coast and spoke Bantu languages. However that may be, with the regular use of \textit{le}, a fairly recent development, Réunion Creole had recourse to a pluralization device quite foreign to the lexifier: the preposed plural word \textit{bann} (French \textit{bande} ‘gang, troop’), an areal feature also present in Mauritian and Seychelles Creole (on plural words see \citealt{Dryer2013coding}). This is similar to Fa d’Ambô \textit{nan}, also analysable as a plural word, but for the fact that \textit{bann} pluralizes all semantic types of nouns.

Turning now to the \textit{linstitité}/\textit{zinstitité} noun group, the same initial conditions obtained in seventeenth century spoken French as in Middle Portuguese; that is, plural definite NPs involving a liaison /z/ before vowel-initial nouns, that can easily be resyllabified from the article’s coda to the noun’s onset. There is a difference, though. Whereas Fa d’Ambô does not tolerate codas, Réunion Creole is ill at ease with word-initial vowels, quite common in Fa d’Ambô in contrast. The discrepancy is bound to be due to a difference between the non-lexifier contributions to the founding stage. Concerning Fa d’Ambô, the anti-coda ban in Volta-Niger languages was already mentioned. Another characteristic of these languages is the frequency of vowel-initial nouns -- as contrasted with mostly con\-so\-nant-ini\-tial verbs -- owing to the amalgamation of former, degrammaticalized noun-class prefixes. The upshot is that there was no incentive in emerging Fa d’Ambô to repair vowel-initial Portuguese nouns -- themselves fairly frequent. On the contrary, amalgamation of the Portuguese definite article \textit{o} ‘the.\textsc{m.sg}’ or \textit{a} ‘the.\textsc{f.sg}’ as in \textit{ope} ‘foot’ (\textit{o pé} ‘the foot’) yielded new vowel-initial items not present in the lexifier, but more in keeping with Volta-Niger phonotactics. This explains why \textit{z}{}-attachment to the stem occurred only in cases where the phoneme was imbued with morphological meaning. Not so in Réunion Creole, where spoken French /z/, although functionally equivalent to Portuguese /z/, often fused into stems just so that they would not begin with a vowel, as in \textit{zavoka} ‘avocado’ (F \textit{avocat}), \textit{zéguiy} ‘needle’ (F \textit{aiguille}), \textit{zonyon} ‘onion’ (F \textit{oignon}), etc. Identifying a \textit{z}{}-initial noun as belonging to the \textit{linstitité}\slash\textit{zinstitité} group therefore entirely depends on the copresence of an /l/-initial noun otherwise phonologically identical and conveying the same meaning but for possibly being singular, whereas the /z/-initial partner preferably occurs in plural contexts, thus excluding \textit{zavoka} and \textit{lavoka} from the class since they do not share a reference (see above). In other words, they are distinct lexemes, whereas \textit{linfirmyé} and \textit{zinfirmyé} realize the same lexeme via stem allomorphy. The large number and discourse frequency of nouns like \textit{zonyon}, whose unique stem begins with a /z/ phoneme, probably was the main cause why /z/ never grammaticalized as a plural marker. Picking out \textit{z}{}-plural nouns in Fa d’Ambô is easier, since there are few of them and nonplural \textit{z}{}-initial nouns such as \textit{zinggantxi} ‘giant’ (\textit{gigante}) are also not very numerous.

The motive for Réunion Creole’s averseness to vocalic initials is not so easy to find out, however. French and Malagasy are quite at ease with vowel-initial words. This leaves us with the Bantu languages of East Africa. Here the relevant fact is that noun stems are preceded by class prefixes. Yet, whereas these prefixes are con\-so\-nant-ini\-tial in some languages, e.g. Kiswahili and Setswana, other languages include vocalic prefixes, e.g. Makhuwa \textit{epuri}\slash\textit{ipuri} ‘kid(s)’, while still others only have vowel-initial prefixes owing to the insertion of so-called ‘augments’ as in Zulu \textit{umfana}\slash\textit{abafana} ‘boy(s)’ (see \citealt{NursePhilippson2006}).\footnote{The augment may be omitted, but this only occurs in special contexts (see \citealt{Mbeje2005}).} No satisfying account comes out from such facts, so that Réunion Creole’s partiality to initial consonants remains something of a mystery.

\section{Bringing it all back home}\label{sec:kihm:6}

Let us look again at the diachrony of the languages under study. The branching that leads from Late Latin to seventeenth century spoken French on the one hand and to Middle Portuguese on the other is obviously not convergent evolution, but the regular change from the plural accusative ending -\textit{Vs} to the plural ending -\textit{s} that took place in all Western Romance languages, antedating their separation. Convergent evolution starts below, on the path to the present-day languages, excepting Modern European Portuguese which simply continues Middle Portuguese in the same way that Modern Written French continues seventeenth century written and probably oral French. In the other three languages a similar process independently occurred, namely the migration of a plural \textit{z} suffix out of the codas of determiners to the onsets of following nouns (NP’s first elements in spoken French).

The initial conditions at the start of the process were the same if, as seems likely, seventeenth century spoken French /lezami/ (〈les amis〉) ‘the friends’ and Middle Portuguese /ozaˈmigos/ (〈os amigos〉)  ‘id.’ shared near identical morphological structures: \{le\textsubscript{DEF.M.PL}{}-\textsc{s}\textsubscript{PL}=ami(-\textsc{s}\textsubscript{PL})\} and \{o\textsubscript{DEF.M}{}-\textsc{s}\textsubscript{PL}=aˈmigo-\textsc{s}\textsubscript{PL}\}.\footnote{I give the Middle Portuguese pronunciation (see \citealt{Teyssier1980}). Modern European Portuguese has /uzaˈmiguʃ/. S notates the archiphoneme realized /z/ intervocalically, /s/ elsewhere.} The main difference, as we saw, resides in the fact that, while Middle Portuguese \textsc{s}\textsubscript{PL} was always sounded, its seventeenth century spoken French counterpart was often mute. Yet, it was still pronounced often enough, more often than in Present spoken French, that its suffixhood was not, one may assume, entirely blurred. See, for instance, the following sentence in \citet[216]{Brunot1939} quoting Kohlhan's 1667 \textit{Grammatica Gallica}: \textit{Les ignorants et les présomptueux sont altiers, mais les sachants modestes et humbles} /lɛz=iɲorãz\_ɛ\_le=prezɔtyø sɔt altjɛ, mɛ le=saʃã modɛst(ə)z\_ɛ\_\~{œ}blə/ ‘The ignorant and the presumptuous are haughty, but they who know (are) modest and humble’.\footnote{I convert to IPA Kohlhan’s phonetic transcription as given by Brunot.} Of the three /z/'s only the first one would be (obligatorily) realized in Present spoken French, whereas the other two are excluded, hypercorrection apart. In other words, preschool children would pronounce the first one, only adults would the other two, and only in monitored speech. In fact, the crucial question is: when did /\textsc{s}\textsubscript{PL}/ cease to be pronounced before a pause (see above) or, to put it differently, when did /\~{œ}blə/ and /\~{œ}bləs/ in the above example cease to be sociolinguistic variants and /\~{œ}blə/ remain as the sole grammatical form? The texts will not give us such information. We can only surmise it happened at some point during the seventeenth century, probably sooner or later depending on distance from the prestige variety (Parisian educated middle class). Once this change was accomplished and the domain of liaison had decreased to an extent close to the present one, then one would be allowed to claim that the /\textsc{s}\textsubscript{PL}/ suffix had disappeared from spoken French.

From a formal viewpoint, we saw that the evolution from seventeenth century spoken French to Present spoken French changed the suffix into a floating phrasal proclitic whose only obligatory landing site is located between a plural determiner and a vowel-initial item. In Fa d’Ambô a similar evolution changed it into a prefix for a particular, semantically arbitrary and quantitatively limited noun class. In Réunion Creole, the outcome was even more limited. A small group of vowel-initial nouns underwent /z/ agglutination to the stem -- as in \textit{zonyon} from \textit{les onions} -- \textit{together with} agglutination of the elided singular definite article \textit{l’}. (I underline the preposition, for nothing special happened when either one of the processes occurred by itself: see \textit{lespadon} ‘swordfish’ < \textit{l’espadon} (no *\textit{zespadon}) or \textit{zonyon} (no *\textit{lonyon}).) Such a dual process resulted in a stem allomorphy that never settled as a morphological device, however, but only led to a variable, albeit significant penchant for using the /z/-initial allomorph in plural contexts. Now, the structural difference between spoken French clitic \textit{z}, on the one hand, and the Fa d’Ambô \textit{z} prefix and Réunion Creole stem allomorphy, on the other, is likely to be connected with another, more profound difference that divorces spoken French from the two creole languages.

\begin{sloppypar}
It is indeed remarkable that, despite the typological gap between spoken French and Written French as far as plural \textit{marking} is concerned, both instantiations of the abstract object ‘French’ share a common semantic basis for number assignment to count nouns, namely the arithmetic principle $\{|n| > 1 \Rightarrow  \text{plural}\}$ already identified in European Portuguese. Where they diverge is in those few contexts in which bare (i.e. determinerless) count nouns are admissible. In Written French \textit{faire cadeau} and \textit{veaux, vaches}… \textit{cadeau} ‘present’ is singular because it stands in opposition with plural \textit{cadeaux} ‘presents’, \textit{veaux} ‘veals’ and \textit{vaches} ‘cows’ are plural because they stand in opposition with singular \textit{veau} ‘veal’ and \textit{vache} ‘cow’. Spoken French \textit{kado} and \textit{vo, vaʃ…}, in contrast, can only be said to be unspecified for number (see \sectref{sec:kihm:2}). Yet, as soon as the NP includes a determiner, the default and most frequent configuration in all varieties of French, number must be marked on count nouns even when such marking appears redundant, useless or irrelevant in the given speech context: you have to say \textit{\MakeUppercase{ʒ}}\textit{=ɛ\_aʃəte *(de=)tomat o=marʃe} ‘\textit{J’ai acheté *(des) tomates au marché}’ ‘I bought tomatoes in the market’. By opposition, many languages would be content with a bare noun here on the shared understanding that one usually buys more than one tomato at a time, so that there is no need to be explicit about it (see for instance Haitian \textit{M achte tomat nan mache}, Wolof \textit{Dem-al jënd-al ko tamaate} \{go-\textsc{imp} buy-\textsc{appl} \textsc{3sg.obj} tomato\} ‘\textit{Va lui acheter des tomates}’ ‘Go and buy tomatoes for her/him’ (\citealt[212]{FalEtAl1990})).
\end{sloppypar}

According to a common and entirely plausible analysis, count nouns in such contexts are treated like mass nouns (e.g. \citealt{KwonZribi-Hertz2006}). Note however that even then, massified count nouns as well as inherently mass nouns must be specified for number and definiteness in Spoken as well as Written French: cf. \textit{də=la=farin} (\textit{de la farine}) ‘flour’, \textit{də=la=tomat} (\textit{de la tomate}) ‘tomato’. In Middle as well as Modern European Portuguese, on the other hand, definiteness is not required, but number is necessarily specified owing to the privative opposition between \textit{tomate} /tuˈmat\textsuperscript{ə}/ ‘tomato’, singular even if generically understood, and overtly plural \textit{tomates} /tuˈmat(ə)ʃ/ ‘tomatoes’. A crucial property is therefore the admissibility of \textit{fully} bare nouns, i.e. noun forms that convey no grammatical features in addition to their lexical meanings. It is nil in Written French and European Portuguese, for noun morphology does not allow for any escape from the singular/plural opposition; it is limited to a few designated contexts in spoken French. As far as the semantic-pragmatic rationale for number specification is concerned, therefore, no typological gap separates Present spoken French from seventeenth century spoken French and Written French. Not so, however, between seventeenth century spoken French and Réunion Creole, on the one hand, Middle Portuguese and Fa d’Ambô, on the other.

At first sight the major difference between Middle Portuguese and Fa d’Ambô appears to be the latter’s sensitivity to animacy (especially humanness) with respect to plural marking. As we saw in \sectref{sec:kihm:3}, only count nouns referring to animate (especially human) entities must be marked for plurality; count nouns denoting inanimate (or nonhuman animate) entities are unspecified for the feature unless they belong to the small \textit{z}{}-plural class and/or are modified by a demonstrative determiner. The distinction plays no role in Middle Portuguese, which strictly adheres to the arithmetic principle. Now I wish to argue that it is actually a consequence of a deeper difference, namely the substitution of the relevance principle for the arithmetic principle. As adumbrated above, number is specified according to the former only if it is contextually relevant to do so, actual relevance being in turn determined by a variety of cultural and pragmatic factors. Particularly effective is the position of the referent in an \textit{individualization} hierarchy such as the following (see \citealt{Corbett2000}):

\ea humans > commonly humanized non-human animates (cows, dogs, elephants, lions, rabbits, ravens, etc.) > seldom humanized non-human animates (ants, fish, frogs, termites, small birds, etc.) > things (including plants, artefacts, natural objects, abstractions).
\z

This hierarchy, whose influence is widely felt across languages, seems to reflect a spontaneous ontology: the lower on the scale the countable entity, the less individual substance it is endowed with, and the more cognitively adequate and culturally permissible it becomes not to be explicit about how many instances are being referred to -- to treat it as mass in other terms. Here is, I believe, the crucial typological divide. On the one hand, there are languages like European Portuguese and French (spoken as well as written) which do not base their number marking policy on the hierarchy -- which does not prevent it from possibly manifesting itself in various areas, an issue I cannot discuss further --\footnote{See, e.g., \citet{KrifkaEtAl1995} on the issue of why “filming the grizzly in Alaska” is fine in generic contexts.} and mark plurality only according to known or inferred quantity, with the upshot that count nouns cannot remain unspecified for number. On the other hand, one finds languages such as Fa d’Ambô and Yoruba, among others, where relevance is the criterion for specifying number, so that count nouns frequently appear bare, unmarked for plurality -- and definiteness, given the close connection between definiteness and individuality.

Things are not so clear-cut in Réunion Creole owing to the presence of number-inflected determiners that often seem to be used as in the lexifier. Yet, as we saw, Réunion Creole includes several contexts in which fully bare count nouns are perfectly acceptable, but have been excluded from all varieties of French for the last three hundred years. Generic contexts appear especially diagnostic of a change from quantity-based to relevance-based number specification. By definition, genericity implies reference to the totality of the exemplars of the kind, thus making cardinality irrelevant, hence the above quoted example \textit{Koson i manz pa lswar}, while arithmetic languages like French and English nevertheless insist on marking plurality: \textbf{\textit{Le}}\textit{=koʃɔ (i(l))=(nə=)mãʒ=pa lə=swar} (\textbf{\textit{Les}} \textit{cochon}\textbf{\textit{s}} \textit{ne mang}\textbf{\textit{ent}} \textit{pas le soir}) ‘Pig\textbf{s} do not eat at night’ (and see footnote~\ref{fn:kihm:35}).

\begin{sloppypar}
To summarize, the evolution from Middle Portuguese to Fa d’Ambô and from seventeenth century spoken French to Réunion Creole involves a typological change in the semantic-pragmatic basis of number specification from an arithmetic to a relevance principle. The change occurred across the board in Fa d’Ambô; although quite obvious, it seems to have been more limited in Réunion Creole -- but more research is required in this tangled area of Réunion Creole grammar. Seventeenth century and present-day spoken French, in contrast, share the same arithmetic principle. What changed drastically was the morphological marking device for expressing plurality, from synthetic marking via suffixation to analytic marking via procliticization of a plural determiner and the \textit{z} phrasal proclitic.
\end{sloppypar}

All three evolutions thus ran converging courses at the level of expression starting from a common morphophonological configuration /Xz ${\prec}$ VY/, where Xz is a plural determiner preceding a vowel-initial noun VY, ending up at a common pattern /(X) ${\prec}$ zVY/, where initial /z/, be it a proclitic, a prefix or the beginning of a stem allomorph, turns out to be associated, categorically or optionally, with plurality. At the level of the semantic-pragmatic underpinnings, on the other hand, only in Fa d’Ambô and Réunion Creole does plurality expression by any device result from convergent evolution away from the initial conditions. Present spoken French remained faithful to these conditions.

Could it be a coincidence that Fa d’Ambô and Réunion Creole are creole languages whereas Present spoken French is not? It strikes me as unlikely. What makes our story of \textit{z} interesting, I believe, is that it clearly brings to light the difference between ordinary (“Neogrammarian”) language change and the special type of change represented by creolization.\footnote{I avoid calling it “exceptional” because the term is known to have caused misunderstandings, as if the creoles were exceptional languages. Of course they are not \textit{qua} languages. Creole emergence, in contrast, is a rare event, for it requires a particular conjunction of causes seldom brought together in history, perhaps only once, at the time of the first globalization (also known as “Great Discoveries”).}  The loss of coda consonants that affected Middle French is a textbook example of the former. Changes of this kind are said to be “blind”. This may be taken in two senses. The change was blind insofar as it paid no attention to the possible morphological functions of the consonants: /s/ ceased to be pronounced in \textit{radis} /radi/ ‘radish’, where it belongs to the stem, as well as in \textit{cris} /kri/ ‘shouts’ where it marked plurality.\footnote{And object case in Old French. Case distinctions had been lost by the time of Middle French.} It was blind also and in a deeper sense because it entirely passed over the semantic property the consonant expressed: quantity-based plurality remained untouched, only its formal expression, its exponence was altered. Changing expression forms while sparing meanings necessarily expressed in a given language might be a defining characteristic of Neogrammarian language change. Quantity-based plurality was in fact already a semantic feature of Latin -- perhaps even of Proto-Indo-European -- that was inherited by all Romance languages in spite of the numerous accidents that attended such a lengthy process.

The change was more profound in Fa d’Ambô and Réunion Creole, as well as in most, if not all creole languages -- at least on the Atlantic side of the domain, including Juba Arabic and Kinubi. It affected not only the exponence, but also the components of the meaning to be expressed. It is not the purpose of the present study to try and evaluate the relative weights of the factors that contributed to such a deep-seated transformation, to decide for instance what may have been more important of the processes inherent in unguided second-language acquisition by adults conducive to creole emergence or of contact-induced restructurings (substrate). It suffices to show that such factors are, if not absent, at least much less influential in ordinary language change. There \textit{is} a difference between neogrammarian change and creolization.

\section*{Abbreviations}
\begin{tabularx}{.5\textwidth}{@{}lQ}
\textsc{adj}& adjective\\
 \textsc{appl}&applicative\\
 \textsc{dem} &demonstrative\\
 \textsc{det} &determiner\\
 \textsc{imp} &imperfective\\
 \textsc{n} &noun\\
 \end{tabularx}\begin{tabularx}{.5\textwidth}{lQ@{}}
 \textsc{num} &numeral\\
 \textsc{obj} &object\\
 \textsc{pl} &plural\\
 \textsc{pp} &prepositional phrase\\
 \textsc{sg} &singular\\
 \textsc{spec} &specifier\\
\end{tabularx}

\printbibliography[heading=subbibliography,notkeyword=this]
\end{document} 
