\documentclass[output=paper]{langscibook}
\ChapterDOI{10.5281/zenodo.10280598}
\author{Inès Saddour\affiliation{University of Toulouse – Jean Jaurès} and Pascale Leclercq\affiliation{University of Paul Valéry Montpellier 3}}
\title[Self-positioning in a host community]{Self-positioning in a host community: Longitudinal insights from study abroad and migrant students}
\abstract{This chapter aims to examine the evolution of self-positioning over a period of time abroad and to make links between socialisation dynamics and the use of pronominal reference in L2 discourse. In particular, it investigates the links between the evolving social networks of resident abroad students – based on Coleman’s model of social circles – and self-positioning regarding the different communities they refer to, as manifested in their pronoun use in L2. It intersects interview data from two longitudinal databases in French and English – the SOFRA and the PROLINGSA corpora, respectively – featuring student populations from different sociocultural profiles, migration contexts and expectations regarding the host society: nine Syrian refugees in France, and two French Erasmus students in Ireland and the UK. Our data reveal common patterns of socialisation in both student groups with a shift from initial socialisation with co-nationals to the circle of international students, which confirms Coleman’s observation that students abroad have limited interaction with locals even after a long time of residence. However, unlike the Erasmus students whose core social network remains strongly international, Syrian students evolve from a strong anchorage within the co-nationals circle to increasing affordances to meet the host society members as well as international community members. The observed differences between the two student profiles highlight the need to further study the factors of L2 socialisation and self-positioning development and to carry out research taking into account the diversity of student profiles in the university environment.}


\IfFileExists{../localcommands.tex}{
  \addbibresource{../localbibliography.bib}
  % add all extra packages you need to load to this file

\usepackage{tabularx,multicol}
\usepackage{url}
\urlstyle{same}

\usepackage{listings}
\lstset{basicstyle=\ttfamily,tabsize=2,breaklines=true}

\usepackage{langsci-basic}
\usepackage{langsci-optional}
\usepackage{langsci-lgr}
\usepackage{langsci-osl}
% \usepackage{./langsci/styles/langsci-lgr}
% \usepackage{./langsci/styles/langsci-osl}
% \usepackage{langsci-gb4e}

\usepackage{tikz}
\usetikzlibrary{patterns,calc}
\pgfdeclarepatternformonly{south east lines}{\pgfqpoint{-0pt}{-0pt}}{\pgfqpoint{3pt}{3pt}}{\pgfqpoint{3pt}{3pt}}{
    \pgfsetlinewidth{0.6pt}
    \pgfpathmoveto{\pgfqpoint{0pt}{3pt}}
    \pgfpathlineto{\pgfqpoint{3pt}{0pt}}
    \pgfpathmoveto{\pgfqpoint{.2pt}{-.2pt}}
    \pgfpathlineto{\pgfqpoint{-.2pt}{.2pt}}
    \pgfpathmoveto{\pgfqpoint{3.2pt}{2.8pt}}
    \pgfpathlineto{\pgfqpoint{2.8pt}{3.2pt}}
    \pgfusepath{stroke}}
    
\usepackage{stmaryrd}
\usepackage{wasysym}
\usepackage{multirow}
\usepackage{caption}
\usepackage{subcaption}
\usepackage{mathrsfs}
\usepackage{qtree}

\usepackage{linguex}


  %pminos do not split footnotes
% \interfootnotelinepenalty=10000 %Footnote in Laporte chapters has to be split SN


%\DeclareIndexNameFormat{default}{%
%\nameparts{#1}%
%\usebibmacro{index:name}%
%{\index[names]}%
%{\namepartfamily}%
%{\namepartgiveni}%
% {}% L1
% {}% L2
%{\namepartprefix}% generates spurious space L3
%{\namepartsuffix}% generates spurious space L4
%}

%  {\DeclareIndexNameFormat{default}{%
%     \usebibmacro{index:name}{\index[names]}{#1}{#3}{#5}{#7}}}

%\DeclareIndexNameFormat{default}{%
%  \usebibmacro{index:name}{\sindex[nom]}{#1}{#3}{#5}{#7}}

%\DeclareIndexNameFormat{default}{%
%  \usebibmacro{index:name}{\sindex[person]}{#1}{#3}{#5}{#7}}
%\DeclareIndexNameFormat{default}{%
%\nameparts{#1} \usebibmacro{index:name}{\sindex[person]]}{\namepartfamily}{‌​\namepartgiven}{\nam‌​epartprefix}{\namepa‌​rtsuffix}}

%\newcommand{\smiley}{:)}

%\renewbibmacro*{index:name}[5]{%
%\usebibmacro{index:entry}{#1}%
%{\iffieldundef{usera}{}{\thefield{usera}\actualoperator}\mkbibindexname{#2}{#3}{#4}{#5}}}

% \newcommand{\noop}[1]{}

%remove for final
%\overfullrule=1mm

\newcommand{\tobi}[2]}}
\renewcommand{\S}[1]{\tobi{#1}{\textsc{*}}}

% this volume references
% puts: [this volume]
% already defined: \citetv
%\newcommand{\citepv}[1]{(\citeauthor{#1} \citeyear*{#1} [this volume])}
\newcommand{\citealtv}[1]{\citeauthor{#1} \citeyear*{#1} [this volume]}

%parentheses around example number
\newcommand{\pref}[1]{(\ref{#1})}

% in-text examples

\newcommand{\lnex}[1]{\textit{#1}} %target lang word
\newcommand{\lnlit}[1]{(lit.: `#1')} %literal reading
\newcommand{\lnlat}[1]{(#1)} % latinization
\newcommand{\lntrans}[1]{`#1'} %translation
\newcommand{\lnexl}[2]%
{\lnex{#1}{} \lnlat{#2}} % ex with latinization
\newcommand{\lnexlat}[3]{\lnex{#1}{} \lnlat{#2}{} \lntrans{#3}} % ex with latinization and tranl.

%ch01
\newcommand{\co}[1]{\mbox{\textbf{#1}}}

%ch09

\newcommand{\cyrbulg}[1]{\begin{otherlanguage*}{bulgarian}#1\end{otherlanguage*}}


%ch10
\newcommand{\nlp}{{\small NLP}}
\newcommand{\mwe}{{\small MWE}}
\newcommand{\rae}{{\small RAE}}
\newcommand{\lvc}{{\small LVC}}
\newcommand{\pos}{{\small P}o{\small S}}
%\newcommand{\todo}[1]{ \textcolor{red}{#1} }

%\renewcommand{\labelenumi}{\theenumi}
%\ainamefmt{{vv}{ll}{, ff}{, jj}} % fullname

\newcommand{\biberror}[1]{{\color{red}#1}}

\newcommand{\osenovaitem}{--~}
  %% hyphenation points for line breaks
%% Normally, automatic hyphenation in LaTeX is very good
%% If a word is mis-hyphenated, add it to this file
%%
%% add information to TeX file before \begin{document} with:
%% %% hyphenation points for line breaks
%% Normally, automatic hyphenation in LaTeX is very good
%% If a word is mis-hyphenated, add it to this file
%%
%% add information to TeX file before \begin{document} with:
%% %% hyphenation points for line breaks
%% Normally, automatic hyphenation in LaTeX is very good
%% If a word is mis-hyphenated, add it to this file
%%
%% add information to TeX file before \begin{document} with:
%% \include{localhyphenation}
\hyphenation{
    Beck-man
    Ngu-yen
    back-chan-nel
    back-chan-nels
    mo-not-o-nous
    ste-reo-typ-i-cal
}

\hyphenation{
    Beck-man
    Ngu-yen
    back-chan-nel
    back-chan-nels
    mo-not-o-nous
    ste-reo-typ-i-cal
}

\hyphenation{
    Beck-man
    Ngu-yen
    back-chan-nel
    back-chan-nels
    mo-not-o-nous
    ste-reo-typ-i-cal
}

  \togglepaper[1]%%chapternumber
}{}

\begin{document}
\maketitle


\section{Introduction}\label{sec:saddour:1}

Research on second language acquisition (SLA) and development has long fallen into one of two paradigms: research adopting a cognitivist approach and interested in neurophysiological and mental mechanisms and processes, on the one hand; and on the other hand, socioculturally oriented research focusing on situational and contextual factors of language acquisition (\citealt{MondadaDoehler2000, Dewaele2013, Véronique2013Socialization}). These two strands have long remained separate in spite of \citet{FirthWagner1997}’s call to rebalance the methodological orientations of language acquisition research. Indeed, as \citet{Véronique2013Socialization} observed, it is difficult to establish direct links between the cognitive mechanisms of acquisition and sociocultural factors, as it is still unclear how sociocultural factors are involved in the representation and processing of L2 (second language) information \citep{Hulstijn2007}. Among the recommendations formulated in his overview, \citet[270]{Véronique2013Socialization} stressed the need for a comprehensive approach using both quantitative and qualitative data in order to establish a causal link between the social context and L2 development and use, a call also made by \citet{Coleman2013}.

In more recent SLA research, contextual factors have been brought to the fore and attempts to intersect data on learners’ cognitive and linguistic development and information about their socialisation processes in the host society have increasingly developed. A particular focus has been placed on L2 learners who are immersed in temporary study abroad (SA) programmes (\citealt{SchartnerWright2013, Tracy-Ventura2017, HuenschEtAl2019, McManus2019}). Research on this learner profile in different host societies and at different stages – before, during and after studying abroad (\citealt{Mitchell2015, MitchellHuensch2020}) – has contributed to establishing correlations between language contact and social networks in the L2 context with these factors leading to an enhancement of linguistic skills (\citealt{DewaeleDewaele2021}).

Links between language acquisition and identity (re)construction have been getting more and more attention in SLA research (\citealt{Block2007a, Grieve2013,Grieve2015, Compiegne2020}). According to these studies, the quality of the language learning experiences abroad is affected by how an individual integrates into a host community, and how their personal history and habitus may come into conflict with the history of the host society’s institutions and cultural structures (cf. \citealt{Coleman2013,Coleman2015}). This idea is summarized as follows: 

\begin{quote}
An individual emerges through processes of social interaction, not as a relatively fixed end-product but as one who is constituted and reconstituted through the various discursive practices in which they participate. \hbox{}\hfill\hbox{\citep[46]{DaviesHarré1990}} 

\end{quote}

Therefore, it is through discursive activity that social activity contributes to identifying (re)construction: 

\begin{quote}
In a sense, the ongoing push and pull and give and take of discursive activity translates into the constant positioning and repositioning and the constant definition and redefinition of who one is.\hbox{}\hfill\hbox{\citep[20]{Block2007b}}

\end{quote}

Speakers’ discursive positioning has also been referred to as \textit{stance}. According to \citet[99]{Biber2006}, “stance expressions can convey many different kinds of personal feelings and assessments, including attitudes that a speaker has about certain information, how certain they are about its veracity, how they obtained access to the information, and what perspective they are taking”. Stance is related to speaker style and identity insofar as it allows the speakers to express their subjectivity, their level of confidence towards a given propositional content (\citealt{BucholtzHall2005}), and to align with their interlocutors (\citealt{DuBois2007}).

Among the many linguistic tools that speakers may use to express the way they position themselves in discourse, pronouns appear as a means to identify or distance themselves from an entity or a propositional content (\citealt{Langhans1996, Hidalgo-DowningEtAl2014}). In her study on the use of pronouns in interview data by the inhabitants of a village in the south of France, \citet{Langhans1996} explains how pronouns provide an interesting window onto the speakers’ positioning relative to the different communities living in the village (original villagers and newcomers from the city):

\begin{quote}
The pronouns thus testify to the establishment of a whole system of references. It is in the discursive instances that the speaker's universe is constructed. This appeared to us to be particularly important to approach iden\-ti\-ty-re\-lat\-ed discourse, in which positioning strategies are constantly carried out (between interlocutors, in relation to a group to which they belong or to an outsider, whether this system of inclusion and exclusion is shared by the interactants or not). These positions are constructed and undone in the course of the interaction, the categorizations being shared, refused or accepted by the other.\footnote{Our translation. Original: “Les pronoms témoignent donc de la mise en place de tout un système de références. C’est dans les instances discursives que l'univers du locuteur se construit. Ce côté nous a semblé particulièrement important pour une approche des discours identitaires, discours où s’effectuent sans cesse des stratégies de positionnement (entre interlocuteurs, par rapport à un groupe d’appartenance ou à un extérieur, que ce système d’inclusion et d’exclusion soit partagé par les interactants ou non). Ces positionnements se construisent et se défont dans l’interaction, les catégorisations étant partagées, refusées ou acceptées par l’autre.” \citep[53]{Langhans1996}.}\hbox{}\hfill\hbox{\citep[53]{Langhans1996}}
\end{quote}

Nonetheless, the expression of stance-taking through pronouns has been relatively understudied, particularly as regards L2 speakers, who are in the process of constructing their second language identity and style (cf. \citealt{Kirkham2011, Liao2009}). For them, the choice of pronouns may reveal their evolving stance relative to the different communities they come into contact with, according to proficiency level and discursive situation.

While research bringing together data on individuals’ socialisation and linguistic development contributes to rebalancing investigation of SLA development, it focuses on a rather limited spectrum of adult learners, mainly middle class secondary school or university learners who embark on an immersion journey that is well planned and offers possibilities and affordances that are quite predictable. More work is needed to address self-positioning by individuals with different life trajectories, like for instance economic or political migrants, who have to learn the language of their new environment.

In an attempt to contribute to the debate on the stance-taking and self-po\-si\-tioning in L2 discourse, our article’s objectives are twofold: (i) to study longitudinally the use of personal reference in interview data, focusing in particular on how resident abroad students refer to themselves in relation to others, and (ii) to compare socialisation processes in students from different sociocultural profiles (Erasmus vs. refugee students).

We compare two different sets of student populations, with different migration contexts and expectations regarding the host society: on the one hand Syrian refugees in France, and on the other hand French Erasmus students in Ireland and the UK. In particular, we are interested in the links between participants’ evolving social networks, their language development and self-positioning regarding the host community as manifested in their L2 French (Syrian students in France) and English use (French students in Ireland and the UK).

Based on a subset of L2 interview data from 11 participants from two longitudinal databases -- French L2: SOFRA corpus\footnote{SOFRA: \textit{Approche SOcioculturelle et psychologique de l’acquisition du FRAnçais par des demandeurs d’asile syriens} (2019--2022). Under this project, 33 Syrian refugees and asylum seekers in France were recorded three times over a period of ten months.} and English L2: PROLINGSA corpus\footnote{\textit{Linguistic Progress during Study Abroad} is a longitudinal corpus that tracks the linguistic progress of five French-speaking students during Erasmus+ stays over the course of an academic year either in the UK or Ireland.} (\citealt{LeclercqEtAl2021}) -- we seek to provide different insights into: (a) the way personal reference (\textit{eux/ils/les gens}/\textit{they} vs \textit{nous/on}/\textit{we}) evolves in SA students’ discourse over a year, and (b) the learner’s stance relative to the different communities she/he is involved with.

\section{Theoretical background}
\subsection{Social networks} %1.1. /

Studying in a foreign university, whether through an exchange programme such as Erasmus, or as an individual migrant, involves numerous challenges: discovering a new academic system, adapting to the host academic culture, but also taking part in the host society, through the constitution of new social networks. \citet{Coleman2015} stresses that the latter constitutes a major part of the studying abroad experience, and one that profoundly shapes individual trajectories:

\begin{quote}
Meeting new people can nurture new activities and new attitudes. This is the fundamental basis of learning through mobility. The new perspectives of new acquaintances allow and prompt you to re-invent yourself. The problem with research which adopts a pre-and-post design is that the person whom you greet on return from extended study abroad is not the same person to whom you said goodbye several months earlier.\hbox{}\hfill\hbox{\citep[42]{Coleman2015}}
\end{quote}

The growing interest in social networks when studying language acquisition is indeed the result of the growing awareness of the crucial role of social interaction in language acquisition and learning. Speaking about the residence abroad experience, Coleman argues that language learning is only one part of what is at stake when people learn a second language:

\begin{quote}
The research community needs to treat study abroad as a broader ethnographic domain, with language learning just one of many spin-offs, in order to recognise the significance of social networks.\hbox{}\hfill\hbox{\citep[35]{Coleman2015}}
\end{quote}

Many researchers have therefore started to move away from considering discrete parts of language acquisition in isolation and to dive into the networks and the social groups with whom learners practice, and “eat, sleep, drink” the second language \citep[34]{Coleman2015}. Based on his experience as a study abroad administrator and a researcher, Coleman proposes the following model (\figref{fig:saddour:1}) representing typical patterns of social networking for study abroad students.


\begin{figure}
% \includegraphics[width=0.6\textwidth]{figures/saddourEtAl1.pdf}
\begin{tikzpicture}
    \node [draw, circle] (co) {Co-nationals};
    \node [above=.5cm of co,align=center] (other) {Other\\outsiders}; 
    \node [above=1cm of other] (locals) {Locals}; 
    \node [fit=(other) (co), draw, circle, inner sep=0pt] (circle) {};
    \node [fit=(circle) (locals), draw, circle, inner sep=0pt] {};
    \draw [-{Stealth[]}] (co.south) +(0,0.5) -- ++(2.5,0.5);
\end{tikzpicture}
\caption{Coleman’s representation of study abroad networks \citep[44]{Coleman2015}\label{fig:saddour:1}}
\end{figure}
% \todo{The figure 1 is a bit distorted and different from Coleman’s = the arrow and the inner circle are centered} 

The model also describes the socialisation dynamics in terms of progression from one circle to the other (even if there is often some temporal overlap in the constitution of these networks). As such, learners initially rely on co-nationals, interacting most of the time in the L1; then progressively move to the middle circle, where they exchange either in the L2 or in a \textit{lingua franca} with other outsiders; and then socialise with local community members. Coleman underlines the fact that the structure of a learners’ social network has implications in terms of language exposure and thus, in language development:

\begin{quote}
Greater contact with the local community leads to greater gains. Interacting with host nationals has been shown to be a key to successful adjustment (Chirkov et al., Safdar, de Guzman \& Playford, 2008), while interacting with co-nationals reduces contact with locals (Chapdelaine \& Alexich, 2004; Teichler, 1991).\hbox{}\hfill\hbox{\citep[43]{Coleman2015}}
\end{quote}

Social networking has thus been shown to be a key factor in language development, while fostering reconfigurations of identity (See \citealt{Grieve2013,Grieve2015} and \citealt{Compiegne2020} on these issues). In particular, duration of SA has been shown by \citet{Grieve2015} to impact attitudes relative to social networking as well as language development, with shorter stay students displaying less social involvement than longer stay students. However, what might be considered a “long” stay abroad remains to be defined, and very little is known about the social networking patterns of foreign residents with different migration histories. Some immigrants may have deliberately moved away from their countries of origin to settle in another one and seek study and/or work opportunities. Others were forced to leave for political, economic or climate-related reasons to seek shelter and a better-quality life. These L2 speakers have received relatively little attention in SLA research in the last four decades in spite of the many 1970s and 1980s projects focusing on “disadvantaged” learners, such as low-educated immigrant workers. Note that the more recent studies generally focus on middle-class learners in relatively short study-abroad periods \citep{Young-Scholten2013}. For the purpose of this study, we distinguish between limited duration stay students whose return home is scheduled – \textit{sojourners} (\citealt{Isabelli-GarciaIsabelli2020}) – and unlimited duration stay residents seeking a new life in the host country.

\subsection{Language socialisation in L2 and self-positioning} %1.2. /

Closely linked to social network studies \citep{Coleman2015}, L2 socialisation, as defined by \citet[564]{Duff2011}, is a process by which non-native or heritage speakers of a language “seek competence in the language and, typically, membership and the ability to participate in the practices of communities in which that language is spoken”. From a historical perspective, the social dynamics of learning, also referred to with the use of expressions such as \textit{social cognition}, or \textit{sociocultural theory} \citep[6]{Duff2019}, have been relatively recently integrated in SLA research. With the \textit{sociocultural turn} (\citealt{FirthWagner1997, Block2007b, Véronique2013Socialization}), more effort has been made to bridge the gap between the linguistic and the sociocultural development, by adopting a more ecological view of language acquisition (cf. \citealt{SteffensenKramsch2017}) and taking more into account the realities of social encounters (\citealt{DiaoMaa2019, Duff2019}). L2 socialisation as an empirical approach has also only recently developed as a field on its own (\citealt{Kinginger2017, DiaoMaa2019}). The concept of language socialisation encompasses the processes by which individuals socialise \textit{through language} and \textit{into language} (\citealt{OchsSchieffelin2017}: 6). Under this perspective, sociocultural knowledge is seen as a necessary component for speakers of a certain speech community to efficiently participate and use the language in socially appropriate ways \citep{Gumperz1964}. Such research is concerned with how language development is linked to the development of sociocultural knowledge and competencies (i.e., acting in socially appropriate ways, and conveying social meanings such as emotions, identities and ideologies). Furthermore, language socialisation research is interested in how language development may vary in different social contexts and groups (\citealt{DuffMay2017}). Indeed, as argued by \citet{Duff2011}, taking part in the communities in which the L2 is spoken is an important part of the L2 socialisation process.

However, if L1 socialisation necessarily implies the child’s membership in the language community, L2 learners are perceived as outsiders and their membership in the L2 language community has to be negotiated:

\begin{quote}
It is in the maelstrom resulting from this relative vacuum that individuals are forced to reconstruct and redefine themselves, both for their own sense of ontological security (\citealt{Guiddens1991}) and the positions ascribed to them by others in their new surroundings. Nevertheless, these processes of reconstruction and repositioning do not take place in predictable manners and it is certainly not the case that the naturalistic context guarantees sustained contact with longer-term inhabitants of the second language context.\hbox{}\hfill\hbox{\citep[75]{Block2007a}}
\end{quote}

\begin{sloppypar}
This redefinition is done through discursive activity, in particular stance-taking. In other words, L2 learners need to elaborate L2-mediated identities, as this may affect their access to meaningful sociocultural activities and also the availability of affordances (i.e., possibilities and opportunities for language learning and socialisation) in the L2 context (\citealt{Uju2011, Kinginger2013, Strömmer2016, Kinginger2017}).
\end{sloppypar}

The sense of belonging, membership and identity negotiation in L2 socialisation are all complex processes that are highly dependent on multiple interrelated factors. These include the role of the host community members and learner impetus -- \textit{grit}\footnote{Combination of perseverance and passion for long-term goals related to language achievement gains.}, in the terminology of \citet{TeimouriEtAl2022} -- and learners’ need to integrate into the host community. In fact, the L2 community members’ attitudes and perceptions about their status may shape the way they self-position within the social networks they develop. Additionally, learners’ own desire and motivation to be part of the new community and negotiate and co-construct hybrid identities are also determinant factors of their investment in L2 socialisation.

To sum up, the type of integration and identity negotiation in the new community may highly depend on the learners’ socio-economic profiles and the specificities of their learning contexts (namely the type of sojourn, its duration and their learning purposes and objectives). The two L2 profiles under consideration in our study (French Erasmus students and Syrian refugee students) may be faced with distinct types of expectations and challenges with respect to integration into the L2 society.

\subsection{Pronouns as markers of self-positioning} %1.3. /

Understanding how learners position themselves within their new societies encompasses understanding of how they orient to particular social membership categories and refer to themselves and others around them. Several studies have indeed focused on pronouns as stance markers, in oral and written data (\citealt{vanHellEtAl2005, Langhans1996, Hidalgo-DowningEtAl2014, Moskowich2017}), as the use of pronouns constitutes a window into how individuals consider their own self and their relations to others. Indeed, pronouns can “simultaneously provide insight into how a person is conceptualising the self; signal a particular stance to the addressee(s); and evoke different perspectives for both the speaker and addressee(s)” (\citealt[2]{OrvelEtAl2022}).

\begin{sloppypar}
We therefore analyse learners’ positioning and social circle membership, through their choice of personal reference (pronouns and lexical expressions) in discourse. To this end, we draw on the concept of positioning at the intergroup level, i.e., the way speakers relate to a given social group. Intergroup positioning can be discursively achieved through the use of pronouns to designate the different groups an individual relates to: 
\end{sloppypar}

\begin{quote}
Intergroup positioning is fundamentally achieved through the use of linguistic devices such as `we', `they', `us', `them', `I' (as a member of a certain group), `you' (as a member of a certain group), and specific group names. Group affiliation and disaffiliation are also achieved largely through linguistic devices, such as ‘\textit{I am} a true-blue patriotic American', `\textit{We} Irish are a resilient people', or `\textit{Most} lawyers seem to be interested in high-profile cases, \textit{but I} ...' When persons distance themselves from their groups (e.g., `\textit{I am not like} most wealthy folk ...'), they inevitably position their group even if they do not characterize it in explicit terms: `Not like most wealthy folk' entails a particular position for `most wealthy folk'. \hbox{}\hfill\hbox{\citep[183]{TanMoghaddam1999}}
\end{quote}

Lexical expressions (for example, ‘the Irish’ or \textit{ma famille}) provide us with some information as to the social groups the participants come into contact with, as well as some information on the speaker’s positioning (‘the Irish’ does not suggest any affiliation to this community, while the possessive pronoun in \textit{ma famille} indicates the speaker’s inclusion in this group). In particular, we mainly focus in this paper on the use of pronouns in interview data, as a reflection of the way in which participants signal their stance relative to a variety of communities (See list in \tabref{tab:saddour:1}), and how this evolves over the course of a residence-abroad period in an academic context.

\begin{table}
\begin{tabular}{lll}
\lsptoprule
        & French & English\\
\midrule
speaker’s inclusion in referent     & je & I\\
                                    & on\footnote{We excluded impersonal \textit{on} as in \textit{Comment dit-on?} from our analysis.}/nous & we/us\\
\midrule
speaker’s non-inclusion in referent & il/elle/lui & he/she/it\\
                                    & ils/elles/eux & they/them\\
\lspbottomrule
\end{tabular}
\caption{Speakers’ positioning and pronominal reference in French and English\label{tab:saddour:1}}
\end{table}

A number of studies on the development of pronoun use in learner interlanguage after spending time abroad have focused on the impact of immersion in the acquisition of sociostylistic pronominal variants and sociolinguistic competence: \citet{Dewaele2004} reviews a number of studies (e.g., \citealt{RehnerEtAl2003, Dewaele2002}, \citealt{Lemée2002}) on the use of pronominal variants (\textit{tu/vous, on/nous}), and reports a shift from \textit{nous} in early immersion to a more frequent use of \textit{on} in late immersion learners. The use of \textit{on} is also positively correlated to authentic interaction in the target language and is comparable to native speakers’ predominant use of this pronoun.

While in previous research on pronouns in an L2, the focus has been placed on the acquisition of sociostylistic and sociolinguistic competence during study abroad or immersion, we aim to investigate the development of the pronoun system as a linguistic tool that is strategically used to regulate social distance and proximity in interpersonal relationships and to manage ingroup and outgroup members. Social proximity is understood as familiarity, intimacy and solidarity with others.\largerpage[2]

Pronouns may also serve as markers of belonging: \textit{we}, \textit{us} reduce social distance whereas \textit{he, she, I} are found to increase it \citep{Semin2007}. The four studies conducted by \citet{FitzsimonsKay2004} have shown that the manipulation of pronouns’ use can lead people to change their perceptions of their own relationships. For example, while the first-person singular (\textit{I,} \textit{je}) indicates a self-focused perspective, shifting from the first-person singular to the first-person plural (\textit{we, nous}) indicates a shift in perspective to reflect a shared experience, identity or reality and connection to others \citep{OrvelEtAl2022}. The “\textit{we}{}-perspective” can express interdependence \citep{AgnewEtAl1998} and can at the same time contribute to reinforcing relationships with others. Similarly, the French pronoun \textit{nous} features a deictic value (it includes \textit{je}), as well as a generic value that can be interpreted through contextual information, and which indicates the speaker’s belonging to a certain social group while evoking an identity contrast\footnote{For example: \textit{Vous avez quelque chose contre} \textbf{\textit{nous}}\textit{, les footballeurs?, ‘Do you have anything against us, you all footballers?’}.} (\citealt[5]{HilgertPalma2014}). \citet{Timmis2015} in his study of the historical Bolton corpus, which contains transcriptions of conversations by working-class people from Bolton (UK) in the 1930s, highlights the fact that the pronoun \textit{they} is often homophoric, i.e. it can only be interpreted if both interlocutors share the same cultural references, and it “involves implicit agreement about the intended referent” \citep[131]{Timmis2015}, as illustrated in example \REF{ex:1}:

\ea%1
    \label{ex:1}
    “[What do you think about the war situation?]\\
    9.5.40: I couldn’t really tell you [about the war situation]. It’s not for us to say. We aren’t educated. If there is opinion in war, it don’t do...We have to need them because they need us.” \citep[126]{Timmis2015}
\z

\citet[126]{Timmis2015} interprets the speaker’s use of “us” and “they/them” as reinforcing “the idea of remoteness from the governing class and remoteness from the war”.

In spite of pronouns’ promising potential as linguistic markers of social belonging and self-positioning, few studies focus on these uses by second language learners. In one such study on L2 socialisation and learner agency, \citet{WrightFogle2012} analyses pronominal practices in the narratives of Dima, a 10-year-old boy of Ukrainian origin, adopted by an English-speaking father. As exemplified in the following extract, Dima shifts in the use of \textit{they} for a general statement about Ukrainian people, to the use of ‘\textit{we}’ to refer to himself and his family: “They have a lot of them in Ukraine because we leave – live right next to the fe – field” (\citealt[93]{WrightFogle2012}).

To sum up, investigating the use of pronouns is a promising avenue to examine self-positioning in relation to others. Therefore, drawing on the concept of \textit{stance} in the context of second language acquisition, we wish to compare two groups of learners with different durations of stay and expectations to examine developing patterns of socialisation through discourse.

\section{Research questions and methodology}\label{sec:saddour:2}\largerpage

In this paper, we aim to address the following research questions:

\begin{description}
 \item[Q1:] What does the use of pronouns by the two groups of students reveal as to their self-positioning relative to their socialisation circles?
 \item[Q2:] Does pronominal reference to the different communities evolve over time?
 \item[Q3:] To what extent does the use of pronouns provide insights into the participants’ dynamics of socialisation?
\end{description}


For the purpose of this paper, we sought to constitute two comparable subsets: we selected SOFRA participants who were university students in France, living on their own, often in students’ residences and with no family-related daily responsibilities, just like the PROLINGSA Erasmus students, who were based in Ireland or the UK. Since the SOFRA interviews were much shorter than PROLINGSA’s, we decided on a subset of nine SOFRA and two PROLINGSA participants, with similar duration of residence (DOR) between the two interviews that were selected, so as to get two samples of similar length (in number of utterances).

\subsection{Study 1}\label{sec:saddour:2.1}

This study uses a data subset of nine participants from the SOFRA project who were forced to leave their country and settle in France for an undetermined period of time. The data were collected over a period of ten months and each participant was seen three times (with an interval of 2.5 to 3 months between each data collection time). In this study, we examine interview data collected at the first (T1) and the third (T3)\footnote{At T1, Syrian students had already settled in France. T3 took place between 6 and 9 months after T1.}  time of data collection. The subset of nine participants includes two females and seven males aged between 19 and 41 years old. According to their placement test eLAO\footnote{Test based on the CEFRL (Common European Framework of Reference for Languages). The skills tested are the following: grammatical structures, active vocabulary (or productive), passive vocabulary (or receptive), listening comprehension, choice between general and/or professional evaluation of the language. The average duration is 20 to 35 minutes.} – administered prior to their enrolment at the university and prior to the data collection – their proficiency in French ranges from A2 to B2. Analyses of the SOFRA data subset are presented in such a way as to distinguish between the findings of A2 level interviews and those of B1–B2 level interviews.\footnote{With this distinction, we aim to provide a basis for discussion when comparing observations from Study 1 and Study 2.}

Interviews were conducted by a Syrian interviewer in French and in Syrian Arabic. The participants were asked the same questions\footnote{See \citet{Saddour2020} for a detailed presentation of the interview protocol.} at T1 and T3. Eight questions were asked in French, about:  daily routines \REF{ex:1} leisure activities \REF{ex:2}, life experience \REF{ex:3} and changes in France \REF{ex:4}, future plans \REF{ex:5}, administrative experience \REF{ex:6}, feelings while interacting in French \REF{ex:7}, and the main difficulties encountered in France \REF{ex:8}. In addition, immediately after responding to questions \REF{ex:3}, \REF{ex:5}, \REF{ex:6}, \REF{ex:7} and \REF{ex:8}, they were asked to give three words in Syrian Arabic to describe their life experience, feelings or difficulties. For this study, we focus on the French data only. The interview data were transcribed in CHAT format using the CLAN analysis program from CHILDES\footnote{Child Language Data Exchange System \textit{(https://childes.talkbank.org/)}.}. \tabref{tab:saddour:2} provides information about the participants and the used data subset.

\begin{table}
	\begin{tabular}{lcccccrrcc}
		\lsptoprule
		&     &     &  &  &  & \multicolumn{4}{c}{Corpus size}\\\cmidrule(lr){7-10}
		&     &     & \multicolumn{2}{c}{DOR\footnote{In months.}} & &  \multicolumn{2}{c}{\#\footnote{Length (in number of utterances or smallest clausal units at the discourse level).}} & \multicolumn{2}{c}{Duration}\\\cmidrule(lr){4-5}\cmidrule(lr){7-8}\cmidrule(lr){9-10}
		&  Age   & Sex    &  T1  & T3 &    L2 proficiency & T1 & T3 & T1 & T3\\
		\midrule
		{Arwa}   & 34 & F & 48 & 53 & A2 & 54 & 36 & 8:67 &  5:28\\
		{Salim}  & 31 & M & 11 & 17 & A2 & 36 & 21 &  9:03 & 6:44\\
		{Shady}  & 24 & M & 18 & 24 & A2 & 60 & 18 &  7:37 & 3:34\\
		{Waseem} & 24 & M & 19 & 25 & A2 & 35  & 35 &  5:51 & 4:40\\
		{Amjad}  & 24 & M & 22 & 29 & B1 & 42 & 25 &  6:48 & 5:04\\
		{Elian}  & 19 & M & 14 & 23 & B1 & 80  & 54 &  14:49 &  10:02\\
		{Wiam}   & 24 & F & 19 & 25 & B1 & 54  & 87 &  10:58 &  9:42\\
		{Chams}  & 24 & M & 18 & 27 & B2 & 170 & 143 &  25:52 &  15:36\\
		{Lounis} & 41 & M & 31 & 38 & B2 & 93 & 51 & 9:35 & 5:04\\
		\midrule
		Mean      & 27 &   & 22 & 29 &  Total   & 624 & 470 & &  \\
		\lspbottomrule
	\end{tabular}
	\caption{Participant information and data description (Study 1)}
	\label{tab:saddour:2}
\end{table}

\subsection{Study 2}\label{sec:saddour:2.2}

This study uses a data subset of two participants from the PROLINGSA project (\citealt{LeclercqEtAl2021}), a longitudinal database including semi-guided interview data for five Erasmus students, all from the South of France, who spent nine months on a UK or Irish campus. The five participants (two males, three females) were all francophone undergraduates studying modern languages (English and Spanish, Arabic or Chinese). They were interviewed before departure (T1, June 2018), then three times while abroad (T2, November 2018, T3, February 2019, T4, March 2019) and upon their return to France (May 2019). Their proficiency in English was assessed before departure and upon return by means of the Oxford Quick Placement Test (OQPT).\footnote{The version of the OQPT placement test which we used mostly taps into lexical and grammatical knowledge.} The two interviewers were the main project initiators. One of them is francophone, while the other is anglophone and speaks American English. Although the relationship between the participants and the interviewers was asymmetrical – the former being Erasmus students and the latter being researchers at their home institution – the participants felt comfortable enough to evoke some aspects of their personal life (e.g., accommodation, friends, experiences) and their academic and linguistic experience. The interview data were transcribed with CLAN. The full project protocol, as well as some of the data, is available on the Ortolang platform.\footnote{\url{https://www.ortolang.fr/market/corpora/prolingsa/v1}, visited on 2023-10-27.}

For this study, we focus on data from A and C\footnote{In the PROLINGSA project and the related publications, participants are referred to by a letter.}, two B1 participants, who present similar L2 proficiency at onset to the SOFRA participants in Study 1. We selected two data collection points (T2 and T4), which took place while participants were abroad, spanning a total of five months. \tabref{tab:saddour:3} summarises the characteristics of our database.

\begin{table}
\begin{tabular}{lccccccccc}
	\lsptoprule
	&  &  & & &  {} & \multicolumn{4}{c}{Corpus size} \\\cmidrule(lr){7-10}
	&  &  & \multicolumn{2}{c}{DOR\footnote{In months.}}& {} &  \multicolumn{2}{c}{Duration} & \multicolumn{2}{c}{\# utterances}\\\cmidrule(lr){4-5}\cmidrule(lr){7-8}\cmidrule(lr){9-10}
	&  Age    &  Sex   & T2  & T4  & L2 proficiency & T1 & T2 & T1 & T2\\
	\midrule
	{A} & 19 & F & 3 & 7 & B1 & 35:03 & 17:13 & 754 & 343\\
	{C} & 18 & F & 3 & 7 & B1 & 26:40 & 12:51 & 359 & 206\\
	\lspbottomrule
\end{tabular}
\caption{Participant information and data description (Study 2)}
\label{tab:saddour:3}
\end{table}

\subsection{Coding}\label{sec:saddour:2.3}

In this paper, we focus on expressions that inform us about how participants position themselves in relation to the different social communities they interact with in the host environment. These include:

\begin{itemize}
\item subject pronouns (e.g., \textit{on, tu, nous, vous, ils / we, you, they})
\item object pronouns (e.g., \textit{nous, eux / they, us})
\item other lexical means that refer to identified social groups (\textit{les gens/people, les français/French people, famille/family, les amis/friends}, etc).
\end{itemize}

All the identified social groups were classified into \textit{topical categories}: family members, home country community members, host country community members, people in the learning environment, students and friends. The different social groups were then classified into three broad categories, following Coleman’s model (\citeyear{Coleman2013, Coleman2015}), to which we added two categories (overlap and indeterminate):

\begin{description}
\item[Co-nationals:] includes the family circle and other groups of compatriots at the university and outside, as in \REF{ex:2}:

\ea%2
    \label{ex:2}
	AT2: there is [*] \textbf{a} \textbf{lot} \textbf{of} \textbf{french} \textbf{people} here actually. \textit{[Co-nationals, home country community]}
\z

\item[Other outsiders:] international students and other migrants.
\item[Locals:] French or Irish students nationals.
\item[Overlap:] communities including members of the three other circles. For example, “Chinese class” for participant C in the PROLINGSA project is a mixed group which includes members belonging to several of the three concentric circles, as illustrated in example \REF{ex:3}:

\TabPositions{1.5cm,4cm,6cm,8cm}
\ea%3
\label{ex:3}
CT2 – Chinese class\footnote{The data were transcribed in the CHAT format. The CHAT transcription conventions were kept in the examples: @u indicates a phonetic transcription. Filled and empty pauses are also coded using the conventions \& -* and (.) conventions respectively. Reformulations and repetitions were also transcribed using [/], [//] and [///] conventions.}\medskip\\
*INT1: \tab  ok and so do you mostly speak with them in english .\\
*INT1: \tab or do you use chinese ?\\
*C:    \tab yeah yeah \& euh in english because <they> [//] i think .\\
*C:    \tab there is only one <irish> [//] native irish in \textbf{the} \textbf{group} . \tab \tab  \textit{[Overlap, friends]}\\
*INT1:  \tab uhhuh@i .\\
*C:  \tab so there are italians americans french .
\z

\item[Indeterminate:] This label was used when it was impossible to determine the exact social circle of the referent, as in \REF{ex:4}:

\ea%4
    \label{ex:4}
	Lounis, T1\medskip\\
\textit{Je fais le sport soit la natation soit le football avec \textbf{mes amis.} [Indeterminate, friends]}\\
`I play sport, either swimming or football with my friends'
\z
\end{description}

Each reference to a social group in the two corpora has been examined based on the discursive context, and also on the analysts’ knowledge about the participants’ histories and circumstances. We distinguish two main types of markers: pronouns and lexical expressions.

In order to analyse pronominal reference, we first identified and classified lexical expressions referring to the speakers’ social circles, which we then classified into data-driven topical categories (i.e., family members, home country community members, host country community members, people in the learning environment, students and friends). For example, the lexical marker “French people” in \REF{ex:2} is coded as referring to co-nationals, and is used for a generic statement about French people (home country community).

Similarly, possessive NPs such as \textit{mes amis, ma famille} refer to co-nationals, but \textit{mes amis} might refer, depending on context, to any of the three circles of \figref{fig:saddour:1} (i.e., co-nationals, other outsiders, locals), as illustrated in \REF{ex:4}. When context does not permit reference identification, as in \REF{ex:5}, we excluded the token from our analysis.

\ea%5 \todo{what about the @ ?!}
\label{ex:5}
Arwa, T1\\
\textit{je konɛ@u \textbf{beaucoup de personnes}}\\
\glt `I know many people'
\z

\section{Results}\label{sec:saddour:3}

\subsection{Study 1: Syrian students (SOFRA data subset)}\label{sec:saddour:3.1}

In the SOFRA data subset, we identified all the personal reference markers used by A2 and B1–B2 students. We analysed reference to social communities through pronouns and we classified them using the above-mentioned categories (co-na\-tion\-als, other outsiders, locals and indeterminate. No overlap category was found in the SOFRA dataset). The identification of referents was based on the analysis of lexical expressions in discourse.

\subsubsection{Use of personal reference markers}\label{sec:saddour:3.1.1}

We identified 128 linguistic markers referring to social communities in the nine participants’ interviews: 55 occurrences by the less proficient A2 group, and 73 by the B1–B2 group. \tabref{tab:saddour:4} summarises all occurrences found in the data.


\begin{sidewaystable}
\small
\begin{tabular}{llccccccccccc}
\lsptoprule
& & \multicolumn{5}{c}{{A2}} & \multicolumn{5}{c}{{B1–B2}} & {Total}\\
\midrule
&  & \multicolumn{3}{c}{{Subject}} & \multicolumn{2}{c}{{Lexical}} & \multicolumn{3}{c}{{Subject}} & \multicolumn{2}{c}{{Lexical}} & \\
&  & \multicolumn{3}{c}{{and object}} & \multicolumn{2}{c}{{markers}} & \multicolumn{3}{c}{{and object}} & \multicolumn{2}{c}{{markers}} & \\
%&  & \multicolumn{3}{c}{{object}} & \multicolumn{2}{c}{} & \multicolumn{3}{c}{{object}} & \multicolumn{2}{c}{} & \\
&  & \multicolumn{3}{c}{{pronouns}} & \multicolumn{2}{c}{} & \multicolumn{3}{c}{{pronouns}} & \multicolumn{2}{c}{} & \\
\midrule
 &  & {3P} & \multicolumn{2}{c}{{1P}} &  &  & {3P} & \multicolumn{2}{c}{{1P}} &  &  & \\
\midrule
\multicolumn{2}{c}{{Social groups}} & {elle(s)} & {nous} & {on} & {Possessive} & {Other} & {elle(s)} & {nous} & {on} & {Possessive} & {Other} & \\
& & {ils} & & & {NPs} & {NPs} & {ils} & & & {NPs} & {NPs} & \\
& & {eux} & & & & & {eux} & & & & & \\
\midrule
{Co-} & {Family} & / & / & 7 & 4 & / & 2 &  & 2 & 8 & / & 23\\
{nationals} & {Home} & / & 1 &  & / & / & 1 & 2 & 6 & 2 & 3 & 15\\
& {community} &  &  &  &  &  &  &  &  &  &  & \\
& {country} &  &  &  &  &  &  &  &  &  &  & \\
\midrule
{Other}& {Student} & / & / & 13 & 6 & 2 &  & 1 & 6 & 4 & 4 & 36\\
{outsiders} & {community} &  &  & & & &  &  & & & & \\
\midrule
{Locals} & {Associations} & / & / & / & / & 1 & / & / & / & / & / & 1\\
& {Host} & 6 & / & / & 1 & 10 & 7 & / & / & 1 & 17 & 42\\
& {community} &  &  &  &  &  &  &  &  &  &  & \\
& {country} &  &  &  &  &  &  &  &  &  &  & \\
& {University/} & / & / & / & / & / & / & / & 2 & / & 1 & 3\\
& {learning} & & & & & & & & & & & \\
& {environment} & & & & & & & & & & & \\
\midrule
{Indeter-} &  & / & / & / & 1 & 3 & / & / & 1 & 1 & 2 & 8\\
{minate} &  & & & & & & & & & & & \\
\midrule
{Total} & & 6 & 1 & 20 & 12 & 16 & 10 & 3 & 17 & 16 & 27 & 128\\
\lspbottomrule
\end{tabular}
\caption{Description of reference to social communities in SOFRA students’ database}
\label{tab:saddour:4}
\end{sidewaystable}

Overall, there is a limited number of markers (pronouns and lexical markers) which refer to Coleman’s three circles. Relations of inclusion in a community are mainly expressed through the use of 1st person pronouns \textit{nous} and \textit{on} and possessive NPs. These markers, mainly used to refer to the family and to the student community, represent 60\% and 49\% of total occurrences by A2 and B1–B2 groups respectively. The use of 3rd person pronouns, which indicates the speaker’s non-inclusion in the referent, represents 11\% of total A2 group’s occurrences and 14\% of total mentions by B1–B2 group. They are mainly used to refer to locals in both groups. In addition, reference to this social circle is mostly made through lexical expressions (Other NPs). The locals circle encompasses members of the learning environment or social workers. Most mentions of this circle tend to be generic (“\textit{les gens}” (people), “\textit{les Français}” (the French), “\textit{les autres}” (Others)). For example, in \REF{ex:6} Arwa uses \textit{ils} and \textit{les Français,} therefore self-positioning outside the circle of locals, and establishing a contrast between them, her Syrian community (\textit{nous}) and other foreigners (\textit{l’autre}).

\TabPositions{1.5cm,3cm,5cm,7cm}
\ea%6
    \label{ex:6}
    Arwa, T3\medskip\\
INT: \tab qu'est-ce qui a retenu votre attention depuis que vous êtes\\
\tab ici en France?\\
\tab `What has caught your attention since you have been here \\ \tab in France?'\\
Arwa: \tab \textit{\& -euh gentillesse 	(.) \textbf{les Français} normalement}\\
\tab \& -euh kindness (.) the French normally\\
\tab \textit{parce que: \textbf{(.) tout le monde} \& -euh \textbf{ils} trop gentil.}\\
\tab `Because everybody is kind''\\
\tab \textit{quand \textbf{ils} paʁl@u avec \textbf{nous} avec \textbf{l'autre}.}\\
\tab `When they talk to us to the others'
\z

Through lexical expressions, Syrian students refer to a limited number of social communities belonging to the three circles. The co-nationals group includes family members and other compatriots referred to as friends or Syrians who stayed in Syria as in example \REF{ex:7} below:

\newpage
\ea%7
\label{ex:7}Elian, T1\medskip\\
\textit{\& -euh setɛ@u au début setɛ@u très très difficile.}\\
\glt `At the beginning it was so hard'\\
\textit{parce que ʒɛ@u lɛse@u \textbf{tous mes connaissances tous mes amis}. [Co-nationals, Syrians living in Syria]}\\
\glt `Because I had left all my acquaintances all my friends'\\
\textit{je me sɑ@u de temps en temps jusqu'à ce moment là je me sɑ@u nostalgique.}\\
\glt `I feel from time to time until now I feel nostalgic'
\z

The other outsiders circle contains exclusively students that the participants are in contact with in the university environment or at the student residences. However, analyses of the way personal reference to the different social communities evolves between T1 and T3 reveal a clear shift from reference to co-nationals to reference to other outsiders (the student community) and to locals in both A2 and B1–B2 groups, as shown in \tabref{tab:saddour:5}.


\begin{table}
\begin{tabular}{lcccc}
\lsptoprule
{Data-driven categories} & \\
{belonging to the three circles} & \multicolumn{2}{c}{{A2}} & \multicolumn{2}{c}{{B1--B2}}\\
\cmidrule(lr){2-3}\cmidrule(lr){4-5}
& {T1} & {T3} & {T1} & {T3}\\
\midrule
Co-nationals & 10 & 2 & 22 & 4\\
Family & 10 & 1 & 12 & /\\
Home country community & / & 1 & 10 & 4\\
\midrule
Other outsiders & 11 & 10 & 7 & 8\\
Student community & 11 & 10 & 7 & 8\\
\midrule
Locals & 10 & 8 & 14 & 14\\
Associations & 1 & / & / & /\\
Host country community & 9 & 8 & 11 & 14\\
University/learning environment & / & / & 3 & /\\
\midrule
Indeterminate & 2 & 2 & 1 & 3\\
Friends & 2 & 2 & 1 & 3\\
\midrule
{Total}  & {33} & {22} & {44} & {29}\\
\lspbottomrule
\end{tabular}
\caption{Number of utterances with reference to social communities for A2 and B1–B2 group}
\label{tab:saddour:5}
\end{table}

As \tabref{tab:saddour:5} illustrates, reference to family drops in A2 interviews (from 30\% to 5\% of total occurrences) and B1–B2 ones (from 27\% to 0\%). At T1, only B1–B2 students refer to co-nationals other than their families, but reference to the Syrian community also decreases in this group at T3. In parallel, reference to the other social circles increases between T1 and T3. For example, mentions of the student community by both proficiency groups increased by 12\%, and reference to locals also increased, especially in B1–B2 group data where the number of occurrences shifts from 32\% to 48\%. In the following section, we analyse the use of pronouns at T1 and T3 comparing A2 and B1–B2 groups.

\subsubsection{Use of pronouns}\label{sec:saddour:3.1.2}

Tables~\ref{tab:saddour:6} and \ref{tab:saddour:7} provide details on the use of pronouns by A2 and B1–B2 participants. As shown in \tabref{tab:saddour:6}, A2 learners mostly use \textit{on} at T1 and T3 (respectively 80\% and 66.7\% of all pronouns). At T1, \textit{on} equally refers to co-nationals and other outsiders, while at T3, it is used almost exclusively to refer to other outsiders. This might be interpreted as showing a shift in patterns of belonging, from a strong anchorage in the family environment at T1, to an increasing self-inclusion in the student community at T3.

\begin{table}
\tabcolsep=.66\tabcolsep
\begin{tabular}{lcccccccccccc}
\lsptoprule
 {A2} & \multicolumn{6}{c}{{T1}}  & \multicolumn{6}{c}{{T3}}\\\cmidrule(lr){2-7}\cmidrule(lr){8-13}
& {\textit{elle}} & {\textit{eux}} & {\textit{ils}} & {\textit{nous}} & {\textit{on}} & {Σ} & {\textit{elle}} & {\textit{eux}} & {\textit{ils}} & {\textit{nous}} & {\textit{on}} & {Σ}\\
\midrule
Co-nationals & / & / & / & / & 6 & 6 & / & / & / & 1 & 1 & 2\\
Other outsiders & / & / & / & / & 6 & 6 & / & / & / & / & 7 & 7\\
{Locals} & / & 2 & 1 & / & / & 3 & / & / & 3 & / & / & 3\\
{Indeterminate} & / & / & / & / & / & / & / & / & / & / & / & /\\
\midrule
{Total} & / & 2 & 1 & / & 12 & 15 & / & / & 3 & 1 & 8 & 12\\
\lspbottomrule
\end{tabular}
 \caption{Number of pronouns referring to the social circles in A2 interviews\label{tab:saddour:6}}
\end{table}


\begin{table}
\tabcolsep=.66\tabcolsep
\begin{tabular}{lcccccccccccc}
\lsptoprule
 {B1–B2} & \multicolumn{6}{c}{{T1}}  & \multicolumn{6}{c}{{T3}}\\\cmidrule(lr){2-7}\cmidrule(lr){8-13}
& {\textit{elle}} & {\textit{eux}} & {\textit{ils}} & {\textit{nous}} & {\textit{on}} & {T1} & {\textit{elle}} & {\textit{eux}} & {\textit{ils}} & {\textit{nous}} & {\textit{on}} & {T3}\\
\midrule
{Co-nationals} & 1 & 1 & 1 & 1 & 7 & 11 & / & / & / & 1 & 1 & 2\\
{Other outsiders} & / & / & / & 1 & 4 & 5 & / & / & / & / & 2 & 2\\
{Locals} & / & 4 & 2 &  & 2 & 8 & / & / & 1 & / & / & 1\\
{Indeterminate} & / & / & / & / & / & / & / & / & / & / & 1 & 1\\
\midrule
{Total} & 1 & 5 & 3 & 2 & 13 & 24 & / & / & 1 & 1 & 4 & 6\\
\lspbottomrule
\end{tabular}
\caption{Number of pronouns referring to the social circles in B1–B2 interviews\label{tab:saddour:7}}
\end{table}

At B1–B2 level (see \tabref{tab:saddour:7}), the use of pronouns paints more diverse patterns of self-positioning. Indeed, a larger variety of pronouns is used at T1 ($n=24$), with the use of \textit{on} and \textit{nous} pronouns suggesting that B1–B2 students self-position as included in a variety of social circles (co-nationals, $n=8/11$; other outsiders, $n=5/5$; locals, $n=2/8$), while sometimes distancing themselves from co-nationals ($n=3/11$) and locals ($n=6/8$) with \textit{elle}, \textit{ils} and \textit{eux} pronouns. Note that instances of ‘nous’ are extremely infrequent ($n=4$) and have been produced by one A2, one B1 and two B2 participants. Since the very low number of occurrences at T3 ($n=6$) makes it difficult to comment on a shift in self-positioning based on pronoun analysis, we use a case study (Chams) to get insights into a B1–B2 speaker’s evolving stance.

We now analyse the development of means of reference in the production of Chams, a 24-year-old man with a B2 level who has already spent 18 months in France at T1. At T3 he had started a new curriculum in a Paris university. The way he uses personal reference changes from T1 to T3, with many references to locals at T1 (55.5\% vs 12.5\% at T3), and more instances of reference to other outsiders at T3 (62.5\% vs 16.6\% at T1). This can be related to the changes in his personal circumstances. In fact, at T1, he mentions his positive experiences with French families which afforded him opportunities to develop his use of French \REF{ex:8}. The linguistic means used to refer to the host family are not very diversified.

\ea%8
\label{ex:8}Chams, T1\medskip\\
\textit{mais quand swi@u vwajazE@u à [/] à [///] ou quand j'ai voyagé à montpellier.}\\
\glt `But when I moved to Montpellier'\\
\textit{\& -euh j'ai rencontré \textbf{une famille française}}.\\
\glt `\& -euh I met a French family'\\
\textit{qui ne parle jamais l'anglais}.\\
\glt `Who never speaks English'\\
\textit{j'étais obligé de [/] de oser.}\\
\glt `I was obliged to dare (to speak)'\\
\textit{\& -euh donc je [x3] je je trouve à l' époque.}\\
\glt `\& -euh so I think that at that time'\\
\textit{que j'ai pu [/] j'ai pu j'ai \& -euh j'ai pu parler en français.}\\
\glt `I could practice French'\\
\textit{\& -euh (.) \& -euh un mois ou quelques semaines après j'ai [///] je suis allé à paris.}\\
\glt `\& -euh after one month or a few weeks, I went to Paris'\\
\textit{aussi j'étais dans \textbf{une famille} j'ai [/] j'ai dormi dans \textbf{une famille français} \& -euh dans [/] dans \textbf{une famille française}}.\\
\glt `I was also staying with a French family'\\
\textit{je savais pas.}\\
\glt `I didn’t know'\\
\textit{qu'\textbf{ils} parlent anglais [/] qu'\textbf{ils} parlent anglais. [Locals, une famille française]}\\
\glt `That they speak English that they speak English'
\z

At T3, however, most of the tokens refer to other outsiders as Chams actively engages in interaction with other flatmates. It is interesting to note that, in contrast to the T1 interview, he uses specific vocabulary to refer to his flatmates (\textit{résidents}) \REF{ex:9}. As he explains it, his habit to seek interaction with French nationals and international students allows him to develop his language skills (lexicon, syntactic complexity, and fluency).

\ea%9
    \label{ex:9}
	Chams, T3\medskip\\
\textit{maintenant bah vu que je suis à paris.}\\
\glt `Now that I am in Paris'\\
\textit{(il) y a beaucoup plus d'activités qu'avant.}\\
\glt `There are more activities than before'\\
\textit{parce que quand j'étais dans d'autres villes.}\\
\glt `Because when I was living in other cities'
\\
\textit{et \& -euh là comme (il) y a beaucoup plus d'activités . (...)}\\
\glt `And here as there are many things to do'\\
\textit{\& -euh vu que je suis à la cité universitaire.}\\
\glt `And since I am at the students’ residence'\\
\textit{j' habite dans une \textbf{résidence} \& -euh detid@u d'étudiants}.\\
\glt `I live in a student’s residence'\\
\textit{\& -euh bah c'est l' une des [//] de mes activités.}\\
\glt `\& -euh one of my activities'\\
\textit{bah je [/] je descends au salon collectif \& -euh.}\\
\glt `I go down to the collective lounge'\\
\textit{où \textbf{la plupart des résidents} \& -euh se trouvent \& -euh presque \& -euh régulièrement.}\\
\glt `Where most of the residents almost regularly meet'\\
\textit{\& -euh ça prend [/] ça prend assez de mon temps.}\\
\glt `It takes me a lot of time'\\
\textit{ça prend par exemple ça peut arriver une heure ou deux heures par journée}.\\
\glt `It takes for example an hour or two during the day'\\
\textit{(en)fin ça veut dire \& -euh.}\\
\glt `Well it means'\\
\textit{là je parle vraiment de [/] de rencontres avec \textbf{les gens} pas des études.}\\
\glt `I am really talking about meetings with people not about studies'
\z

While Chams still self-positions outside the community of “\textit{résidents}”, meeting people forms part of his strategies to develop affordances to talk to people, get acquainted and form new bonds, as well as practicing his French in a fairly intensive way.

\subsubsection{Discussion}

Overall, the Syrian learners show an interesting mastery of the pronominal system of French by the exclusive and efficient use of \textit{on} to include themselves in the different social groups with which they identify (the \textit{nous} variant is scarcely used). This is all the more interesting since Syrian Arabic is a pro-drop, subject suppressed language where the subject is indicated through verbal morphology and where pronouns are only used for emphasis (\citealt{Cowell1964, BassiouneyKatz2012}). In other studies on the acquisition of \textit{nous} and \textit{on} in French immersion contexts, learners have been found to be sensitive to the variation from early stages of acquisition \citep{Dewaele2002}, and their use of \textit{on} increases with L2 contact and exposure to the pronoun \textit{on}, supposedly a less formal variant of \textit{nous}, even in instructed settings \citep{RehnerEtAl2003}. In that regard, our results confirm previous findings.

Furthermore, the pronominal and lexical means used to refer to social groups in the SOFRA dataset can give insights into the individuals’ actual social dynamics and the development of their social networking over a period of time. Our analyses show that there are differences in personal reference between T1 and T3 in terms of the social groups that are mostly referred to as well as the degree to which speakers express inclusion in the French community. The Syrian participants evoke their family and other co-nationals more frequently in the first interview than in the last one. In parallel, reference to other outsiders and to locals slightly increases at T3. This evolution of personal reference over time seems to corroborate the dynamics described in Coleman’s model in which learners gradually move from the inner circle of co-nationals towards other social circles (i.e., other outsiders and locals). Self-positioning as part of the local community through the use of \textit{on} and \textit{nous} is scarce in the SOFRA data. This could partly be interpreted in relation to the “existential tension” that Syrian refugees may experience relative to their home community. In fact, the study of \citet[13–14]{McEnteeEtAl2022} on the discursive functions of multilingual varieties used on Facebook by six Syrian refugees in the United Kingdom illustrates how Syrians fear to be “othered” and distanced by the Syrian community and choose to distinguish themselves from the host society and self-position as Syrians so that they have the right to talk as part of the Syrian community.

Overall, our Syrian students in both proficiency groups report difficulties finding interaction opportunities with the local community. These difficulties are reflected in the types of pronouns and lexical expressions, which convey distance from the host community. Given that they are long-stay migrants who have moved to France to seek political asylum, these difficulties seem to be coherent with observations made by previous studies, which have also emphasised the self-positioning struggles of migrants in the host society. Norton's study (2000, 2013) on five immigrant women workers in Canada also points at the limited access to the host community and interaction opportunities in L2:

\begin{quote}
[…] the opportunity to practice speaking English outside the classroom is dependent largely on their access to anglophone social networks. Access to such networks was difficult to achieve for these immigrant women.\\\hbox{}\hfill\hbox{\citep[135]{Norton2000}}
\end{quote}

\citet[79]{Block2007a} argues that the lack of social interaction opportunities may be due to the fact that generally, long-stay migrants are judged from the lenses of dominant discourses and positioned as \textit{inadequate interlocutors}. However, our participants in the SOFRA data show their awareness of the importance of interaction with locals in order to augment their ‘capital’ and to fully function in the L2. As \citet{Block2007a} states it, feeling legitimate is an essential condition for effective personal and linguistic development:

\begin{quote}
Participation thus must always begin peripherally and if the individual is not deemed legitimate, or the individual chooses not to participate as a reflective form of resistance, then it might not begin at all. Thus, in order to participate in particular communities of practice, the individual needs to have acquired or accumulated sufficient and appropriate cultural capital (\citealt{Bourdieu1977, Bourdieu1984}), that is, the educational resources and assets, necessary to be a fully functioning participant in a particular community of practice.\hfill\hbox{\citep[25]{Block2007a}}
\end{quote}

Our data suggest that being a student in a French university fosters engagement in such communities of practice, as the use of \textit{on} by the SOFRA students implies.

\subsection{Study 2 French Erasmus students (PROLINGSA data subset)}

In Study 2, we analyse the use of personal reference markers by two francophone Erasmus students (A and C) spending a year abroad in Ireland, to find out about the various communities they mention in discourse; as well as the use of the pronouns \textit{we/us} and \textit{they/them} to reflect their self-positioning relative to four circles: co-nationals (French people), other outsiders (international students), locals (Irish people) and overlap.

\subsubsection{Use of personal reference markers}

Though both A’s and C’s pre-departure proficiency scores in English are similar (both are B1 at the beginning of the project according to the Oxford Quick Placement Test), they have very different communicative styles. A is a very fluent and fairly self-confident L2 speaker, while C often struggles to find the right words to express her thoughts. While both participants increased their fluency over their nine-months stay abroad, C’s gains are fewer than A’s (cf. \citealt[108]{GilyukGerman2021} for a detailed analysis of fluency development in the PROLINGSA project). Several authors suggest that fluency – and more generally speaking, linguistic development – might be linked to networking affordances with locals \citep{Coleman2015}.

While both A and C live in an international environment and build a close network of international friends, C’s core social kernel is composed of two French nationals, one of whom is her flatmate, as illustrated in \REF{ex:10}. She also frequently socialises with a mixed group (overlap category), made of her classmate from her Chinese class at university.

\TabPositions{1.7cm,3cm,5cm,7cm}

\newpage
\ea%10
    \label{ex:10}
    CT2 – Three people C interacts most with\medskip\\
*INT1: \tab ok so we 'd like to know .\\
*INT1: \tab if you could maybe name the three people .\\
*INT1: \tab that you spend the most time with . (…)\\
*INT1: \tab so the three people you think .\\
*INT1: \tab that you interact with the most .\\
*C: \tab ok so \textbf{oxanne my french roommate} .\\
*INT1: \tab uhhuh@i ok .\\
*C: \tab \& euh marie+astrid another french[*] .\\
*INT1: \tab ok .\\
*C: \tab <but> [//] and the other that 's \textbf{a group of people} after xxx .\\
*INT1: \tab ok .\\
*C: \tab people that i see the most is \textbf{the chinese class} . \\
\z

Contrary to C, A’s core network does not include any French speakers. It includes international friends (two Indian and one Italian nationals), with whom she communicates in English, as illustrated in \REF{ex:11}:

\ea%11
    \label{ex:11}
    AT2 – The three people A interacts most with\medskip\\
*INT2: \tab so who are your top three \& =laughs ?\\
*A: \tab but that 's \textbf{people in this house} actually .\\
*INT2: \tab but that 's fine yeah .\\
*A: \tab \& uh ok i would say \textbf{one italian girl one and two indian boys} .\\
\z

We will now turn to the analysis of the different social communities mentioned by A and C during the interviews, at T2 and T4. \tabref{tab:saddour:8} summarises the number of occurrences through which they refer to social communities, either through pronouns or lexical markers.

\begin{table}
\small
\begin{tabular}{llccccc}
\lsptoprule
\multicolumn{2}{c}{} & \multicolumn{2}{c}{{Subject}} & \multicolumn{2}{c}{{Lexical}} & {Total}\\
\multicolumn{2}{c}{} & \multicolumn{2}{c}{{and object}} & \multicolumn{2}{c}{{markers}} & \\
%\multicolumn{2}{c}{} & \multicolumn{2}{c}{{object}} & \multicolumn{2}{c}{} & \\
\multicolumn{2}{c}{} & \multicolumn{2}{c}{{pronouns}} & & \\
\midrule
\multicolumn{2}{c}{{Social}} & {3PP} & {1PP} & {Possessive} & {Other} & \\
\multicolumn{2}{c}{{groups}} &
{they/them} & {we/us} & {NPs} & {NPs} & \\
%\multicolumn{2}{c}{} & {them} & {us} & & & \\
\midrule
{Co-} & {Family} & 3 & 2 & 5 & 1 & 11\\
{nationals} & {Other}& 6 & 16 & / & 12 & 34\\
&  {co-nationals} & & & & & \\
\midrule
{Other} & {International} & 15 & 35 & 4 & 29 & 83\\
{outsiders} &{students} & & & & & \\
& {Other} & / & 14 & / & 8 & 22\\
&{international}& & & & & \\
&{communities} & & & & & \\
\midrule
{Locals} & {Host} & 18 & 3 & / & 28 & 49\\
 &{country} & & & & & \\
 & {community} & & & & & \\
& {University/} & 11 & 1 & 2 & 6 & 20\\
& {learning} & & & & & \\
& {environment} & & & & & \\
& {Associations} & 1 & 1 & / & 1 & 3\\
\midrule
\multicolumn{2}{c}{{Overlap}} & 8 & 26 & 1 & 24 & 59\\
\midrule
\multicolumn{2}{c}{{Total}} & 62 & 98 & 12 & 109 & 281\\
\lspbottomrule
\end{tabular}
\caption{Description of A and C database regarding reference to social communities\label{tab:saddour:8}}
\end{table}

Overall, the social groups listed in \tabref{tab:saddour:8} paint a broad-brush representation of the communities with which participants A and C come into contact during their stay abroad. Mentions of co-nationals represent 19.5\% of all occurrences, while other outsiders constitute 37.4\%, locals 25.6\% and overlap 21\% of occurrences. Relations of inclusion in a community – either through the use of the pronouns \textit{we/us} or possessive NPs – constitute 39.1\% of all mentions, while the use of \textit{them/they} indicating a distanciation from the referent constitutes 22\% of all mentions. Let now analyse the way A and C refer to co-nationals, locals, other outsiders and overlap evolves between T2 and T4, as illustrated in \tabref{tab:saddour:9}.

\begin{table}
\begin{tabular}{lcccc}
\lsptoprule
{Data-driven categories} & \multicolumn{2}{c}{ {A}} & \multicolumn{2}{c}{ {C}}\\
{belonging to the three circles} & \multicolumn{2}{c}{} & \multicolumn{2}{c}{}\\
\midrule
& { {T2}}{\#} & { {T4}}{\#} & { {T2}}{\#} & { {T4}}{\#}\\
\midrule
{Co-nationals} & {10} & {13} & {24} & {3}\\
Family & 1 & 8 & 7 & /\\
Flatmates & / & / & 7 & 2\\
French student community & 5 & / & 2 & /\\
Friends & 1 & / & 6 & 1\\
Home country community & 3 & 2 & 2 & /\\
Work & / & 3 & / & /\\
{Other} {outsiders} & {53} & {11} & {28} & {13}\\
Flatmates & 39 & 6 & 6 & 3\\
Friends & / & / & 13 & 8\\
International student community & 14 & 4 & 8 & 2\\
University/learning environment & / & / & 1 & /\\
Work & / & 1 & / & /\\
{Locals} & {18} & {34} & {11} & {6}\\
Flatmates & 1 & / & / & /\\
Friends & / & / & / & 1\\
Host country community & 8 & 32 & 2 & /\\
University/learning environment & 9 & 1 & 6 & 5\\
Swimming club & / & / & 3 & /\\
Work & / & 1 & / & / \\
{Overlap} & {27} & {11} & {14} & {3}\\
Friends & 1 & / & / & 1\\
Others & 10 & 1 & 3 & 2\\
University/learning environment & 16 & 6 & 11 & /\\
Work & / & 4 & / & /\\
\midrule
{Total} & {108} & {68} & {77} & {25}\\
\lspbottomrule
\end{tabular}
\caption{Number of utterances with reference to social communities for participants A and C}
\label{tab:saddour:9}
\end{table}

Co-nationals include family members, flatmates, friends, mostly within the French student community, but also general mentions of the home country community, as illustrated in \REF{ex:12}, where A compares Irish people’s commitment to the dance class, and their willingness to take part in competitions, to French people’s attitude towards this kind of leisure activity.

\ea%12
    \label{ex:12}
    AT2 – Home country community – dance class\medskip\\
*A: \tab yeah and in France no \textbf{they} ’re just like for an activity.  \\ \tab \textit{[co-nationals, home country community]}\\
\z

In \REF{ex:12}, \textit{they} refers to French people in France. The use of this pronoun suggests that A self-positions herself as distant from the French community.

Family mentions are fairly infrequent. A mostly refers to her family because of her mother’s visit in March, while C mentions them at T2, but not at T4.

As for the locals, they feature highly in A’s speech, with a strong progression from T2 (18 mentions) to T4 (34 mentions). For A, this category includes mostly reference to the host country communities, while for C it mainly includes reference to the university environment and the swimming club. Such reference was elicited by the interviewer asking them to comment on Brexit and Saint Patrick’s Day, as illustrated in \REF{ex:13}:

\ea%13
    \label{ex:13}
    AT4 – Brexit\medskip\\
*INT1:  \tab ok so \& uh now let's speak about brexit .\\
*INT1: \tab which is another hot topic .\\
*INT1: \tab do people around you speak about it ?\\
*A: \tab \& uh i don 't know many \textbf{irish peoples[*]} . \textit{[locals, host \\ \tab country community]}\\
*A: \tab but \& uh just one of my lecturer .\\
*INT1: \tab mmhm@i .\\
*A: \tab (be)cause i took class of irish cultural studies .\\
*INT1: \tab mmhm@i .\\
*A: \tab so \textit{we} talked about that actually . (…) \textit{[overlap, students]}\\
*A: \tab like the lecturer was really waiting for that .\\
*INT1: \tab  mmhm@i .\\
*A: \tab \& uh he told \textit{us} like . \textit{[overlap, students]}\\
*A: \tab \textbf{people} here are really scared like . \textit{[locals, host country \\ \tab community]}\\
*INT1: \tab  mmhm@i .\\
*A: \tab \textbf{irish people} like that it would start like again . \textit{[locals, host\\ \tab country community]}\\
*A: \tab a civil war with northern ireland .\\
*A: \tab \& uh and like \textbf{they} really don 't know . \textit{[locals, host country \\ \tab community]}\\
*A: \tab what to expect actually .
\z

In \REF{ex:13} A refers to Irish people through lexical markers (\textit{irish peoples}, \textit{people}, \textit{irish people}) and the pronoun \textit{they}. She clearly does not associate herself to the host country community, but uses \textit{us} and \textit{we} to include herself in the Irish cultural studies class community, which we coded in the overlap category. Although C was asked the same question, it did not lead to a comparable amount of reference to the Irish social communities. This might reflect A’s greater interest in Irish society, particularly at T4. As for the other outsiders circle, it mostly includes reference to flatmates, especially for A at T2; and to international friends, as in \REF{ex:14}.

\ea%14
    \label{ex:14}
    CT2 – International welcome activities\medskip\\
	*INT1: \tab so you know .\\
	*INT1: \tab often when you arrive on a new campus . \\
	*INT1: \tab there 's like an international welcome .\\
	*C: \tab yeah .\\
	*INT1: \tab or that sort of th(ing) did you participate in that .\\
	*INT1: \tab can you tell me about it ?\\
	*C: \tab yeah i participate[*] 0prep the city walking tour .\\
	*INT1: \tab ok .\\
	*C: \tab \& euh <cinema's[*] night> [//] cinema night \& euh parties\\ \tab obviously .\\
	*INT1: \tab yeah yeah .\\
    *C: \tab <and> [//] but I [x 2] didn 't go to the trips .\\
    *INT1: \tab ok .\\
    *C: \tab i prefer .\\
    *C: \tab to go on my own with \textbf{a little group of friends} . \textit{[other\\ \tab outsiders, friends]}\\
\z

Finally, the overlap category features mostly mentions of the university environment, and of indeterminate \textit{others}, as illustrated in \REF{ex:15} in which C describes her experience of Saint Patrick’s day:

\ea%15
\label{ex:15}CT4 – Saint Patrick’s Day\medskip\\
*C: \tab  and the atmosphere were[*] just totally crazy .\\
*C: \tab like \textbf{everyone} was dressed up in green . \textit{[overlap, others]}\\
*C: \tab the music was amazing too .
\z

In this example, she uses the indefinite lexical marker \textit{everyone} to refer to the crowd celebrating Saint Patrick’s Day, whom we assume included both Irish nationals and international participants. For A, the other outsiders category also includes reference to work: she found a job as a waitress in Dublin and her co-workers include Irish people along with international team members.

In a nutshell, \tabref{tab:saddour:9} gives us a glimpse into the way both participants refer to the various communities they come into contact with during their study abroad period: both mention members of Coleman’s three circles, but reference to locals remain fairly general (\textit{people} from Ireland), while reference to co-nationals, other outsiders and overlap mainly includes reference to identified groups of individuals (family, friends, flatmates, classmates). This is congruent with \citegen{Coleman2015} analysis of socialisation patterns during study abroad. Nevertheless, A and C display fairly different referent patterns when it comes to referring to locals, which hints at individual variation regarding reference and self-positioning relative to the host community.

Let us now analyse the way A and C situate themselves in their new social environment, as reflected by their use of pronouns to refer to social communities.

\subsubsection{Use of pronouns}

First, it is clear from Tables~\ref{tab:saddour:10}--\ref{tab:saddour:11} that participants A and C display different patterns of pronoun use and have their own preferences as regards reference to their social circles. As shown in Tables~\ref{tab:saddour:10}--\ref{tab:saddour:11}, A and C self-position themselves differently relative to co-nationals: while C mostly refers to them through the pronoun \textit{we}, at T2 and at T4, therefore indicating a relationship of inclusion into the French community, A refers to them mostly through \textit{they/them} (3 and 4 occurrences at T2 and T4, respectively), thereby marking her distanciation from French nationals.

\begin{table}\tabcolsep=.66\tabcolsep%
\begin{tabular}{lcccccccccccccc}
\lsptoprule
& \multicolumn{5}{c}{{T2}} & \multicolumn{5}{c}{{T4}}\\\cmidrule(lr){2-6}\cmidrule(lr){7-11}
& {them} & {they} & {us} & {we} & {Σ} & {them} & {they} & {us} & {we} & {Σ}\\\midrule
Co-nationals & / & 3 & 1 & 12 & 16 & / & / & / & 3 & 3\\
Other outsiders & / & 1 & / & 14 & 15 & / & / & / & 4 & 4\\
Locals & / & 5 & / & 1 & 6 & / & 1 & / & 1 & 2\\
Overlap & / & / & 5 & 9 & 14 & / & / & / & / & /\\\midrule
{Total} & / & 8 & 6 & 36 & 51 & / & 2 & / & 8 & 10\\
\lspbottomrule
\end{tabular}
\caption{C’s number of pronouns referring to the social circles\label{tab:saddour:10}}
\end{table}

\begin{table}\tabcolsep=.66\tabcolsep%
\begin{tabular}{lcccccccccc}
\lsptoprule
& \multicolumn{5}{c}{{T2}} & \multicolumn{5}{c}{{T4}}\\\cmidrule(lr){2-6}\cmidrule(lr){7-11}
& {them} & {they} & {us} & {we} & {Σ} & {them} & {they} & {us} & {we} & {Σ}\\\midrule
{Co-nationals}    & 1 & 2 & / & 1 & 4 & / & 3 & / & 1 & 4\\
{Other outsiders} & / & 8 & 1 & 26 & 35 & / & 3 & / & 5 & 8\\
{Locals} & 2 & 9 & / & 1 & 12 & 1 & 13 & / & 1 & 15\\
{Overlap} & / & 5 & / & 6 & 11 & / & 2 & 3 & 3 & 8\\\addlinespace
{Total} & 3 & 24 & 1 & 34 & 62 & 1 & 31 & 3 & 10 & 35\\
\lspbottomrule
\end{tabular}
\caption{A’s number of pronouns referring to the social circles\label{tab:saddour:11}}
\end{table}

As for locals, both A and C refer almost exclusively to them through \textit{they/them}, even at T4, suggesting their self-position as outside the Irish community. The only instances when they include themselves in the local community through the use of \textit{we} corresponds to A’s comment about the Irish weather, as in \REF{ex:16} and C’s comment on a social event with the swimming club, whose members are all Irish \REF{ex:17}:

\ea%16
    \label{ex:16}
    AT4 – Weather in Ireland\medskip\\
*A:\tab and sometimes <we have> [/] \textbf{we} have like a whole week\\ \tab like just blue . [locals, host country community]
\ex%17
    \label{ex:17}
   CT2 – Social event with the swimming club\medskip\\
*C: \tab and \textbf{we} had a social night in a pub . [locals, swimming club]
\z

Reference to other outsiders and overlap categories is predominantly accomplished by A and C through the use of \textit{we/us} (A T2 = 32/46, A T4 = 11/16; C T2 = 28/29, C T4 = 4/4), both at T2 and at T4. This suggests that both participants feel strongly involved in an international network, which might at times include an Irish member. However, we observe a drop from T2 to T4 for both A (T2 = 35, T4 = 8) and C (T2 = 15, T4 = 4), which might be ascribed to the different conversation topics at the two points of data collection.

\subsubsection{Discussion}

All in all, our analysis shows that participants A and C mention, through the use of personal pronouns and lexical markers, a large variety of social communities, either from their co-nationals circle, the locals circle, or the other outsiders circle. Flatmates and fellow students constitute unsurprisingly their closest network, but they also interact occasionally with French and Irish people. Their use of pronouns points to a strong grounding of both participants in the international student community, labelled ‘other outsiders’ circle by Coleman. Their use of pronouns shows a clear distanciation from their co-nationals, as well as from the local Irish communities, with whom they have limited contact. This pattern seems already well-established at T2, three months after their arrival in the host country, and does not change at T4. It is coherent with Coleman's observations, and is in line with the expectations which the participants themselves expressed prior to departure \REF{ex:18}:

\ea%18
    \label{ex:18}
    AT1 – Pre-departure expectations regarding linguistic development\medskip\\
	*INT1: \tab how do you plan .\\
	*INT1: \tab to achieve \& heu these goals .\\
	*INT1: \tab  let 's focus on \& heu the linguistic development . (…)\\
	*INT1: \tab   so how do you plan to achieve that ?\\
	*A: \tab \& euh because maybe i will have \textbf{a roomate} .\\
	*INT1: \tab yeah .\\
	*A: \tab      \& heu \textbf{that is not french} .\\
	*A: \tab  so <i would be> [//] i would speak english so with him or \\ \tab with her .\\
	*INT1: \tab mmhmmh@i .\\
	*A: \tab so maybe will improve my english .\\
	*A: \tab and just yeah ask [x 2] questions to teachers .\\
	*A: \tab or make <friends that> [/] \textbf{friends} \textbf{that} \textbf{are} \textbf{not} \textbf{french} .\\
	*A: \tab and so to ask them questions .\\
	*A: \tab and to speak to them fluently .
	\z

Before departure, A imagines herself to be living with friends or flatmates “that are not French”, but not necessarily Irish either. To some extent, this forms part of the Erasmus experience, as popularised through Erasmus students’ narratives, and Cédric Klapisch’s film \textit{L’auberge espagnole}, which captures remarkably well this aspect of the study abroad programme. In short, meeting international friends is part of the horizon for Erasmus students, while meeting locals is often at best a wishful intention.

\section{Contributions of Studies 1 \& 2 and conclusion} \label{sec:saddour:4}

In this paper, we addressed the following research questions:

\begin{description}
\item[Q1:] What does the use of pronouns by the two groups of students reveal as to their self-positioning relative to their socialisation circles?
\item[Q2:] Does pronominal reference to the different communities evolve over time?
\item[Q3:] To what extent does the use of pronouns provide insights into the participants’ dynamics of socialisation?
\end{description}

Regarding our first research question, our study reveals diverse approaches to self-positioning in the SOFRA and PROLINGSA data subsets. The Syrian learners’ personal reference evolves from self positioning as part of the home country community to more inclusion in the international community and a strong interest in the host community. The data indeed suggest more reference to locals at T3, even though markers of inclusion in this circle are rare. As for the Erasmus students, their use of pronouns clearly indicates the extent to which they associate with the international community, even at T2. Locals are almost exclusively referred through \textit{they/them}, which indicates that the Erasmus students self-position as distant from the Irish community, even after seven months abroad. Interest in the locals is keen but expressed diversely, with A mentioning locals much more frequently than C at T2 and T4.

Regarding Q2 and Q3, we wanted to find out whether linguistic reference to the different communities over the study period reflected the development of social networks. We observed that reference patterns in both groups partially evolved according to Coleman’s predictions. However, Coleman’s model was devised based on the observation of Erasmus students, and our findings highlight the importance of taking into consideration the context of residence of foreign students as well as individual differences. In particular, Syrian refugee students come to university with different expectations from other study abroad students. While the Erasmus students from the PROLINGSA study arrive on campus expecting to meet international and local students, Syrian students mostly expect to meet French people, but are faced with difficulties due to perceived linguistic and cultural barriers (Granget \& Saddour in press, \citealt{Alsadhan2022}). Nevertheless, over time, Syrian students seem to create affordances to meet French and international community members (e.g., Chams). Being enrolled at university gives them enough social confidence to develop friendships and feel part of the French student community. As for Erasmus students, they develop a variable interest in the Irish society, as exemplified by the number of topical categories referring to Irish communities produced by A and C (cf. Tables~\ref{tab:saddour:9} and~\ref{tab:saddour:10}); however, their core social network remains strongly international, even after seven months abroad, as predicted in \citet{Coleman2013, Coleman2015}.

Finally, the analyses of the two student groups’ productions (long-stay students and stay-abroad students) suggest different socialisation objectives and this seems to have an impact in the way they relate to the social groups about whom they talk. In the case of Syrian students, interactions with people from the local community are driven by the need to learn the language and efficiently use it (i.e., to improve pronunciation, learn appropriate expressions, etc.). In turn, Erasmus students seem to be both interested in the SA social experience and in the acquisition of target-like uses in L2 English. These motivations are closely linked to the circumstances of stay of each group of students. In particular, we observe more diversity in the topical categories used in the PROLINGSA data (which may be partly due to the participants’ cultural capital, and their motivation to meet new people in a variety of social networks), than in the Syrian data, where many of them struggle to meet their basic needs – finding accommodation and work – as illustrated in the personal narratives of Syrian refugees in France gathered by \citet{Alsadhan2022}. In this data subset, students describe more limited affordances except for individuals with active socialisation habits, such as Chams.

The study of pronouns and their reference enables us to get insights into the evolution of self-positioning over a period of time abroad and to make links between socialisation dynamics and actual linguistic means used in discourse. The comparison of two student groups with different residence conditions reveals common patterns of socialisation, with an increasing sense of belonging to the other outsiders circles. Our data confirm Coleman (2015)’s observation that interaction with locals is limited and that students perceive themselves as outsiders even after a long time of residence. The observed differences between the two student profiles highlights the need to further study the factors of L2 socialisation and stance-taking development~and to carry out research that takes into account the diversity of student profiles in the university environment.

\section*{Acknowledgements}
We are deeply grateful to the editors of the book for giving us the opportunity to pay homage to the work of Daniel Véronique. Daniel was involved in setting up the SOFRA project and motivated us to pursue our work on SLA by learners in exile. Furthermore, his work has been very inspiring for our reflection on the importance of linking linguistic and cognitive aspects with sociocultural aspects of language acquisition.

\printbibliography[heading=subbibliography]
\end{document}
