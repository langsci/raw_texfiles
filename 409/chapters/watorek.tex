\documentclass[output=paper]{langscibook}
\ChapterDOI{10.5281/zenodo.10280610}

\author{Marzena Watorek\orcid{}\affiliation{Université de Paris 8; UMR7023} and Rebekah Rast\orcid{}\affiliation{American University of Paris; UMR7023} and Pascale Trévisiol\orcid{}\affiliation{Université Sorbonne Nouvelle; UR DILTEC 2288}}

\title[Focalisation et acquisition de la morphologie flexionnelle]{Effet de la focalisation sur la forme enseignée dans l’acquisition de la morphologie nominale flexionnelle et implications didactiques}
      
\abstract{La question de savoir si un \textit{input} axé sur la forme enseignée est plus bénéfique à l’apprenant occupe une place importante dans la réflexion sur les implications didactiques des recherches en acquisition des langues secondes. Dans cette contribution, nous présentons une étude sur les toutes premières étapes d’acquisition qui est centrée sur l’acquisition de la morphologie nominale. Afin de cerner le rôle de l’explicitation grammaticale dans un cours de langue étrangère, nous avons effectué une étude permettant la comparaison du traitement de la morphologie nominale dans deux types de séances du polonais suivies par des apprenants francophones, qui se différencient par la focalisation plus ou moins explicite sur la forme enseignée. Les résultats montrent que les apprenants du groupe exposé à la focalisation sur la forme enseignée ont obtenu de meilleurs résultats au test de jugement de grammaticalité que ceux de l'autre groupe. En production, le premier groupe a montré plus de variation dans les flexions produites. Cependant, certains apprenants qui ont obtenu des scores élevés au test de jugement de grammaticalité, quel que soit leur groupe, ont peu utilisé la flexion casuelle. Cette étude exploratoire donne l’occasion de discuter les avantages de l'approche «~focus on form~» dans les pratiques didactiques en LE et de contribuer ainsi à la réflexion sur l’interface acquisition/didactique.

\keywords{traitement de l’\textit{input} en L2, focalisation sur la forme, interface acquisition/didactique, acquisition de la morphologie nominale, acquisition du polonais}}

\IfFileExists{../localcommands.tex}{
  \addbibresource{../localbibliography.bib}
  % add all extra packages you need to load to this file

\usepackage{tabularx,multicol}
\usepackage{url}
\urlstyle{same}

\usepackage{listings}
\lstset{basicstyle=\ttfamily,tabsize=2,breaklines=true}

\usepackage{langsci-basic}
\usepackage{langsci-optional}
\usepackage{langsci-lgr}
\usepackage{langsci-osl}
% \usepackage{./langsci/styles/langsci-lgr}
% \usepackage{./langsci/styles/langsci-osl}
% \usepackage{langsci-gb4e}

\usepackage{tikz}
\usetikzlibrary{patterns,calc}
\pgfdeclarepatternformonly{south east lines}{\pgfqpoint{-0pt}{-0pt}}{\pgfqpoint{3pt}{3pt}}{\pgfqpoint{3pt}{3pt}}{
    \pgfsetlinewidth{0.6pt}
    \pgfpathmoveto{\pgfqpoint{0pt}{3pt}}
    \pgfpathlineto{\pgfqpoint{3pt}{0pt}}
    \pgfpathmoveto{\pgfqpoint{.2pt}{-.2pt}}
    \pgfpathlineto{\pgfqpoint{-.2pt}{.2pt}}
    \pgfpathmoveto{\pgfqpoint{3.2pt}{2.8pt}}
    \pgfpathlineto{\pgfqpoint{2.8pt}{3.2pt}}
    \pgfusepath{stroke}}
    
\usepackage{stmaryrd}
\usepackage{wasysym}
\usepackage{multirow}
\usepackage{caption}
\usepackage{subcaption}
\usepackage{mathrsfs}
\usepackage{qtree}

\usepackage{linguex}


  %pminos do not split footnotes
% \interfootnotelinepenalty=10000 %Footnote in Laporte chapters has to be split SN


%\DeclareIndexNameFormat{default}{%
%\nameparts{#1}%
%\usebibmacro{index:name}%
%{\index[names]}%
%{\namepartfamily}%
%{\namepartgiveni}%
% {}% L1
% {}% L2
%{\namepartprefix}% generates spurious space L3
%{\namepartsuffix}% generates spurious space L4
%}

%  {\DeclareIndexNameFormat{default}{%
%     \usebibmacro{index:name}{\index[names]}{#1}{#3}{#5}{#7}}}

%\DeclareIndexNameFormat{default}{%
%  \usebibmacro{index:name}{\sindex[nom]}{#1}{#3}{#5}{#7}}

%\DeclareIndexNameFormat{default}{%
%  \usebibmacro{index:name}{\sindex[person]}{#1}{#3}{#5}{#7}}
%\DeclareIndexNameFormat{default}{%
%\nameparts{#1} \usebibmacro{index:name}{\sindex[person]]}{\namepartfamily}{‌​\namepartgiven}{\nam‌​epartprefix}{\namepa‌​rtsuffix}}

%\newcommand{\smiley}{:)}

%\renewbibmacro*{index:name}[5]{%
%\usebibmacro{index:entry}{#1}%
%{\iffieldundef{usera}{}{\thefield{usera}\actualoperator}\mkbibindexname{#2}{#3}{#4}{#5}}}

% \newcommand{\noop}[1]{}

%remove for final
%\overfullrule=1mm

\newcommand{\tobi}[2]}}
\renewcommand{\S}[1]{\tobi{#1}{\textsc{*}}}

% this volume references
% puts: [this volume]
% already defined: \citetv
%\newcommand{\citepv}[1]{(\citeauthor{#1} \citeyear*{#1} [this volume])}
\newcommand{\citealtv}[1]{\citeauthor{#1} \citeyear*{#1} [this volume]}

%parentheses around example number
\newcommand{\pref}[1]{(\ref{#1})}

% in-text examples

\newcommand{\lnex}[1]{\textit{#1}} %target lang word
\newcommand{\lnlit}[1]{(lit.: `#1')} %literal reading
\newcommand{\lnlat}[1]{(#1)} % latinization
\newcommand{\lntrans}[1]{`#1'} %translation
\newcommand{\lnexl}[2]%
{\lnex{#1}{} \lnlat{#2}} % ex with latinization
\newcommand{\lnexlat}[3]{\lnex{#1}{} \lnlat{#2}{} \lntrans{#3}} % ex with latinization and tranl.

%ch01
\newcommand{\co}[1]{\mbox{\textbf{#1}}}

%ch09

\newcommand{\cyrbulg}[1]{\begin{otherlanguage*}{bulgarian}#1\end{otherlanguage*}}


%ch10
\newcommand{\nlp}{{\small NLP}}
\newcommand{\mwe}{{\small MWE}}
\newcommand{\rae}{{\small RAE}}
\newcommand{\lvc}{{\small LVC}}
\newcommand{\pos}{{\small P}o{\small S}}
%\newcommand{\todo}[1]{ \textcolor{red}{#1} }

%\renewcommand{\labelenumi}{\theenumi}
%\ainamefmt{{vv}{ll}{, ff}{, jj}} % fullname

\newcommand{\biberror}[1]{{\color{red}#1}}

\newcommand{\osenovaitem}{--~} 
  %% hyphenation points for line breaks
%% Normally, automatic hyphenation in LaTeX is very good
%% If a word is mis-hyphenated, add it to this file
%%
%% add information to TeX file before \begin{document} with:
%% %% hyphenation points for line breaks
%% Normally, automatic hyphenation in LaTeX is very good
%% If a word is mis-hyphenated, add it to this file
%%
%% add information to TeX file before \begin{document} with:
%% %% hyphenation points for line breaks
%% Normally, automatic hyphenation in LaTeX is very good
%% If a word is mis-hyphenated, add it to this file
%%
%% add information to TeX file before \begin{document} with:
%% \include{localhyphenation}
\hyphenation{
    Beck-man
    Ngu-yen
    back-chan-nel
    back-chan-nels
    mo-not-o-nous
    ste-reo-typ-i-cal
}

\hyphenation{
    Beck-man
    Ngu-yen
    back-chan-nel
    back-chan-nels
    mo-not-o-nous
    ste-reo-typ-i-cal
}

\hyphenation{
    Beck-man
    Ngu-yen
    back-chan-nel
    back-chan-nels
    mo-not-o-nous
    ste-reo-typ-i-cal
}
 
  \togglepaper[1]%%chapternumber
}{}

\begin{document}
\begin{otherlanguage}{french}
\lsFrenchChapterSettings{}
\maketitle


\section{Introduction}\label{sec:watorek:intro}
Cet article s’inscrit dans le domaine de l’acquisition des langues secondes et de la didactique des langues étrangères.\footnote{Les travaux de Daniel Véronique en acquisition attestent de son intérêt pour les dimensions cognitive et interactionnelle du processus acquisitionnel. En tant qu’un des acteurs majeurs du projet ESF \citep{Perdue1993}, Daniel Véronique a contribué à l’élaboration de l’approche des lectes d’apprenants forgée suite aux recherches longitudinales sur les apprenants non guidés de ce projet. La richesse interdisciplinaire de sa recherche l’a conduit à poser comme point de débat le dialogue entre deux champs, la recherche en acquisition des langues et la didactique des langues. Nous faisons ici référence à une série d’articles dans lesquels Daniel Véronique propose une réflexion épistémologique sur la séparation qui existe dans les travaux de langue française entre ces deux champs disciplinaires, ce qui n’est pas nécessairement le cas des travaux de langue anglaise.} L’interface entre la recherche en acquisition des langues et la didactique des langues peut s’étudier à partir de différents phénomènes du processus d’acquisition/apprentissage.\footnote{Contrairement à \citet{Krashen1981}, nous ne faisons pas la différence entre le processus d’acquisition et celui d’apprentissage et traiterons l’un et l’autre comme synonymes.} Dans cette contribution, nous allons aborder l’impact de la focalisation explicite sur les formes enseignées, autrement dit l’impact d’un enseignement qui prône la mise en relief des morphèmes, sur leur acquisition par l’apprenant adulte débutant d’une nouvelle langue. Cette réflexion renvoie aux recherches connues sous le terme de Focus on form vs Focus on meaning (\citealt{DoughtyWilliams1998}). La question des effets de la focalisation sur la forme aux premières heures de l’acquisition d’une nouvelle langue en milieu guidé a été l’une des préoccupations du grand projet européen VILLA \citep{DimrothEtAl2013}. Notre objectif est ici de revenir sur le rapport entre l’exposition à la langue (\textit{input}) et sa saisie (\textit{intake}) en présentant une étude exploratoire de type qualitatif sur l’expérimentation de deux types d’enseignement de la morphologie nominale flexionnelle, l’un plus axé sur la forme que l’autre, qui a contribué à la préparation et l’opérationnalisation du projet VILLA.\footnote{Le projet VILLA (\textit{Varieties of Initial Learners in Language Acquisition: controlled classroom input and elementary forms of linguistic organisation}) a été financé par l’ANR ORA (Open Research Area) pour la période 2011--2014.} Aussi, nous cherchons à répondre aux questions suivantes : Quel est l’impact de la focalisation sur la forme sur l’acquisition de la morphologie nominale flexionnelle chez des débutants~du polonais ? Quel est l’impact d’un enseignement focalisé sur la forme comme la mise en relief des formes enseignées («~input enhancement~» selon \citealt{SharwoodSmith1993}) sur l’acquisition de la morphologie nominale flexionnelle chez des débutants~du polonais ? Dans cet article, nous résumons d’abord les positions principales de Daniel Véronique sur le dialogue entre la recherche en acquisition des langues et la didactique des langues autour de la question de la focalisation sur la forme. Nous présentons ensuite les cadres théorique et méthodologique dans lesquels s’inscrit notre étude empirique visant à examiner l’effet de la focalisation sur la forme dans un cours de polonais langue cible pour des apprenants débutants francophones. La présentation des résultats et des limites de cette étude nous permettra de discuter les effets de la mise en relief sur l’acquisition de la morphologie nominale flexionnelle, en ouvrant sur des implications d’ordre théorique et didactique.


\section{Les apports des études en acquisition des langues secondes/étrangères à la didactique des langues}\label{sec:watorek:1}

La réflexion sur la possibilité d’un dialogue entre deux champs de recherche, celui de la didactique des langues et celui de l’acquisition des langues, dans le contexte français (\citealt{Véronique2000,Véronique2005Interrelations,Véronique2019}), suit en cela la tradition anglo-saxone selon laquelle on ne sépare pas l’étude de la didactique des langues («~teaching~») de celle de leur apprentissage\slash acquisition («~learning~»). En effet, les travaux français en didactique des langues avancent l’expression «~enseignement/apprentissage~» avec un intérêt privilégié pour le premier terme.

\begin{sloppypar}
L’interface possible entre Recherche en Acquisition des Langues et Didactique des Langues (qui ne fait d’ailleurs pas l’unanimité parmi les chercheurs, \citealt{HousenPierrard2005, KangEtAl2019, Trévisiol-OkamuraKomur-Thilloy2011}) pose la question des apports des recherches en acquisition sur l’enseignement  entre autres et la didactique des langues (en termes d’analyse des méthodes, pratiques et contenus d’enseignement) malgré un dialogue que \citet{Véronique2005Interrelations} estime difficile, du moins dans le contexte français, et ce pour diverses raisons. D’une part, comme le souligne l’auteur, les deux champs de recherche ne partagent pas les mêmes objectifs~par rapport à l’apprentissage : la didactique des langues s’intéresse à l’intervention sur le processus d’appropriation dans un contexte d’enseignement donné, à travers des activités pédagogiques ciblées, tandis que les chercheurs en acquisition s’intéressent aux différentes étapes de ce processus et aux principes le régissant à partir des performances de l’apprenant. D’autre part, pour ce qui est du rapport entre enseignabilité et apprenabilité \citep{Pienemann1989}, Véronique relève la difficulté d'appliquer tels quels pour la classe de langue les résultats des travaux de recherche en acquisition L2 qui se sont intéressés aux séquences développementales. Comme le rappelle l’auteur, 
\begin{quote}
  on ne peut postuler que la progression pédagogique doive suivre les parcours d’acquisition. Cela, d’autant plus que ces parcours et itinéraires commencent à peine à être explorés. Il ne faudrait pas tomber, de plus, dans l’illusion de croire que l’on peut influencer directement et sans intermédiaire le procès d’appropriation linguistique.\hfill\hbox{(\citealt[18]{Véronique2005Interrelations})}
\end{quote} 
\end{sloppypar}

Le processus d’acquisition témoigne en effet d’une certaine forme d’indépendance vis-à-vis de l’ordre de présentation des structures de L2 en classe \citep{Krashen1981}, comme l’ont montré un grand nombre de travaux d’inspiration cognitive sur le développement de micro-systèmes grammaticaux tels que ceux de la morphologie verbale, du syntagme nominal et de la négation en français L2 (voir par ex. \citealt{MylesEtAl2002}, \citealt{Royer2005} pour l’acquisition de la négation, \citealt{Bergström1997, BartningSchlyter2004} pour l’acquisition de la morphologie verbale et du syntagme nominal).

Un numéro de la revue \textit{Etudes en Linguistique Appliquée} dirigé par \citet{Véronique2000} aborde les éclairages nouveaux apportés par les travaux de recherche en acquisition des langues (RAL) sur l’enseignement-apprentissage, notamment en termes des profils d’apprenants en tant que locuteurs pourvus de stratégies, de la structuration de la matière à enseigner et de l’évaluation des activités langagières. Véronique souligne aussi en quoi les travaux interactionnistes en acquisition sont les plus à même d’inspirer didacticiens et enseignants, et permettent de penser l’appropriation guidée d’une langue étrangère grâce à des notions comme celles de communication exolingue (constitutive des échanges en classe), de séquences potentiellement acquisitionnelles, d’’exposition à la langue (\textit{input}) et d’intégration d’une structure ou saisie (\textit{intake}). 

C’est sur ce rapport \textit{input/intake} que nous nous attarderons. La question de la mise en relation des connaissances explicites et implicites en langue étrangère, ou encore des savoirs déclaratifs et des savoir-faire procéduraux de l’apprenant, telle que présentée par \citet{Ellis2005}, peut en effet être discutée (\cite{LaurensVéronique2017}). Partant à la suite de \citet{Long1991} d’une typologie des discours instructionnels, ceux plus ou moins axés sur la forme (\textit{Focus on form}) ou les formes linguistiques (\textit{Focus on forms}), et ceux davantage axés sur la signification (\textit{Focus on meaning}),  on peut en effet distinguer les connaissances en jeu dans l’acquisition des langues selon leur statut et leur rôle dans le processus d’enseignement-apprentissage (\cite{Véronique2019}). Si la présentation des données de la langue cible dans le cadre de l’enseignement avec focalisation sur les formes est forcément explicite, à savoir renforçant les connaissances déclaratives des apprenants, elle peut se faire de manière explicite ou implicite dans l’enseignement de type ‘Focalisation sur la forme’ (cf. \tabref{tab:watorek:1}). \citet{Véronique2019} fait ici référence à la distinction établie par Ellis entre ces deux modalités d’enseignement des formes de la langue cible, reprise ci-dessous par \citet[10]{HousenPierrard2005}.



\begin{table}
\caption{\label{tab:watorek:1} Distinction entre les deux types de focalisation sur la forme, implicite et explicite, d’après \citet{Véronique2019}
}
\begin{tabularx}{\textwidth}{QQ}
\lsptoprule
{Instruction implicite de type «~Focus on Form~»} & {Instruction explicite de type «~Focus on Form~»}\\\midrule
Attire l’attention sur la forme linguistique cible & Dirige l’attention sur la forme linguistique cible\\
Surgit spontanément dans une activité communicative & Est prédéterminée et planifiée (en tant que point principal et but de l’activité d’enseignement)\\
Est non intrusive (interruption minimale de la communication de la signification) & Est intrusive (interruption de la communication de la signification)\\
Présente les formes cibles en contexte & Présente les formes cibles hors contexte\\
Ne fait pas recours au métalangage & Utilise une terminologie métalinguistique (ex : explication de la règle)\\
Encourage le réemploi libre de la forme cible & Implique la pratique contrôlée de la forme cible\\
\lspbottomrule
\end{tabularx}
\end{table}

Ces travaux de langue anglaise distinguant la focalisation sur la forme et celle sur le sens peuvent être mis en relation avec un courant de la didactique francophone et des travaux en analyse conversationnelle de \citet{BangeEtAl2005} sur la bifocalisation de l’attention sur le message et le code. D’autres didacticiens comme \citet{Cicurel2002}, \citet{GajoEtAl2004}, ou encore  \citet{Py1989,Py1993} se sont eux aussi intéressés à la façon dont l’enseignement de la forme interrompt la communication, en distinguant les séquences interactionnelles principales centrées sur la transmission de contenu informationnel et les séquences latérales centrées sur le code ou la négociation du sens des mots.

En se basant sur les travaux de \citet{Doughty2003} sur les discours instructionnels, \citet{Véronique2019} reprend cette distinction entre explicite et implicite en opposant un type d’instruction où les savoirs à acquérir sont formulés sous forme de règles (démarche explicite déductive), à un type d’instruction qui privilégie les exemples illustratifs relevant d’un apprentissage inductif et implicite. Pour ce qui est de la distinction déductif/inductif, nous reprenons celle d’\citet{Ellis2005} selon lequel l'enseignement explicite, qui requiert d'être délibérément attentif à la forme visée dans le but de la comprendre, peut se faire de deux manières : selon une démarche dite «~didactique~» (ou déductive) lorsqu'on fournit une explication à propos de la forme aux apprenants («~rule based~»), ou selon une démarche de découverte (ou inductive) lorsqu'on fournit les données de la L2 illustrant la forme et qu'on demande aux apprenants de se faire une idée par eux-mêmes de la façon dont elle fonctionne («~exemplar based~»). \citet{Véronique2019} s’appuie également sur une typologie des discours instructionnels formulée par \citet{Doughty2003} illustrant les types d’activités correspondantes en classe. Parmi les exemples de discours relevant de la focalisation sur la forme, Doughty cite des pratiques de classe telles que la mise en relief de l’\textit{input} et de l’\textit{output} verbal ou la négociation du sens dans l’interaction verbale qui font appel aux connaissances implicites des apprenants (instruction implicite), ou au contraire des activités métalinguistiques (jugements de grammaticalité, auto-correction des erreurs, etc) qui ont recours à leurs connaissances explicites (instruction explicite).


Dans l’étude exploratoire dont nous présentons les résultats dans cet article, nous adoptons la terminologie du projet VILLA, respectivement \textit{Form-based input} pour la démarche d’enseignement axée sur le code et \textit{Meaning-based input} pour celle axée sur le sens. Dans cette étude exploratoire, nous avons voulu observer les effets de deux méthodes d’enseignement opposées au regard des principes mentionnés auparavant. Nous synthétisons ci-dessous les principes didactiques sous-jacents aux deux approches d’enseignement dans le cadre de cette étude exploratoire au projet VILLA.



\begin{table}
\caption{\label{tab:watorek:2} Critères de différenciation des deux approches didactiques dans l’étude exploratoire}
\begin{tabularx}{\textwidth}{QQ}
\lsptoprule
{Approche «~Form-based~» (FB)} & {Approche «~Meaning-based~» (MB)}\\\midrule
\multicolumn{2}{c}{Focalisation sur la forme prédéterminée et planifiée}\\\midrule
Présentation des formes cibles en contexte et hors contexte & Présentation des formes cibles en contexte\\\midrule
Eléments informatifs sur les marques grammaticales & Pas d’éléments informatifs\\\midrule
Mise en relief graphique des marques grammaticales dans l’\textit{input} écrit & Aucune mise en relief de l’\textit{input} écrit\\\midrule
Présence de métalangage et explicitation des règles (synthèse grammaticale) & Absence de métalangage et d’explicitation des règles\\\midrule
Entrainement contrôlé des formes (activités de systématisation) & Activités à visée communicative\\\midrule
Feedback correctif sur les formes & Pas de feedback correctif sur les formes\\
\lspbottomrule
\end{tabularx}
\end{table}


\section{«~First exposure~study~» et acquisition de la morphologie nominale}\label{sec:watorek:2}

Notre étude exploratoire se situe dans une approche méthodologique que l’on peut qualifier avec \citet{Carroll2013} et \citet{ShoemakerRast2013} de «~première exposition~» (first exposure). Elle porte sur l’acquisition de la morphologie nominale en polonais langue cible (LC) par des débutants francophones et vise à évaluer le rôle de la focalisation sur la forme enseignée pour l’appropriation de la morphologie nominale dès les premières heures d’un cours de langue.

\subsection{Etudes de la première exposition à une nouvelle langue}\label{sec:watorek:2.1}

Des études de la première exposition comme celle que nous présentons ici ont pris de l’ampleur depuis les années 1990. Comme le souligne \citet[29]{Rast2008}:
\begin{quote}
    la notion de ‘première exposition’ renvoie aux études dont les données sont collectées au moment du tout premier contact avec la langue cible et durant les premières secondes, minutes et heures suivant l’exposition, et dont l'input de la langue cible est contrôlé.
\end{quote} 
Il s’ensuit que la «~première exposition~» dans l’acquisition d’une langue seconde fait référence davantage à une approche méthodologique qu’à une théorie, un modèle ou une hypothèse spécifique.

En effet, les «~études de la première exposition~» ont commencé à apparaître dans les années 1950 sur des langues artificielles créées par les chercheurs afin de pouvoir répondre à des questions pour lesquelles un contrôle de l’exposition à la langue était exigé. Ces premières «~langues~» ressemblent aux formules mathématiques, comme par exemple la «~grammaire artificielle~» construite par \citet{Reber1967} à la base d’un lexique composé de cinq lettres : P, S, T, V, X. Des règles précises concernant l’ordre des lettres génèrent la construction de phrases. Selon \citet[856]{Reber1967}, «~La langue en soi est tout simplement l’ensemble de toutes les phrases qui peuvent être produites/générées par la grammaire~». Une telle langue permet aux chercheurs d’élaborer des structures qui ressemblent aux structures grammaticales des langues, de les «~enseigner~» aux participants et, par la suite, d’observer l’apprentissage des participants en fonction de cette exposition. De cette manière, on peut examiner les processus d’apprentissage d’une sélection d’éléments linguistiques. Cependant, la question de la comparaison se pose: peut-on comparer l’apprentissage d’une langue artificielle avec l’apprentissage d’une langue naturelle?

Cette question a inspiré une série d’études (par ex. \citealt{YangGivón1997}) dont le but était de savoir si les résultats d’une étude expérimentale sur l’apprentissage d’une langue artificielle (appelée Keki) dans des conditions de laboratoire contrôlées pouvaient être généralisés à l’apprentissage de «~vraies~» langues dans un environnement plus naturel. Les auteurs en ont conclu que « Les résultats des diverses mesures du développement langagier fournissent des preuves solides que l'acquisition d'une langue seconde peut être étudiée avec profit dans des conditions de laboratoire contrôlées~» (\citealt{YangGivón1997} :187). Malgré des résultats fort probants, Yang et Givón restent prudents : « Nous tenons à souligner que même si nous aimerions généraliser nos résultats à l'apprentissage de langues réelles, nous pensons qu'il est trop tôt pour le faire. Le fait même que Keki soit une langue artificielle simplifiée et miniature signifie que nos résultats ne peuvent pas encore être généralisés à l'apprentissage de langues réelles~» (\citealt[190]{YangGivón1997}).

Ces conclusions ont mené à de nouvelles études sur les langues naturelles dans des contextes variés, allant des études expérimentales en laboratoire sur ordinateur (\citealt{KempeBrooks2008}) et en salle de classe (\citealt{Rast2008, DimrothEtAl2013}) à des études dites «~in the wild~» où les participants en contexte non-guidé sont exposés à des extraits de monologues ou dialogues dans une langue naturelle (\citealt{GullbergEtAl2012, HanLiu2013}). En général, les résultats ont montré une capacité impressionnante chez l’apprenant adulte d’entrer rapidement dans la nouvelle langue et de traiter toutes sortes d’éléments linguistiques aux niveaux phonologiques, morphologiques et syntaxiques. Ceci dit, le besoin s’est manifesté de disposer de méthodologies appropriées et cohérentes pour les études sur l’exposition à une nouvelle langue. Le projet VILLA a largement contribué au développement d’outils d’observation du traitement initial de l’\textit{input}, s’appuyant au préalable sur une série d’études exploratoires dont celle que nous présentons dans cette contribution. Les études exploratoires permettant la mise en place du projet VILLA se situent, tout comme celle-ci, dans l’approche théorique des lectes d’apprenants (Learner variety approach, \citealt{KleinPerdue1997}) selon laquelle le processus d’acquisition d’une nouvelle langue par l’apprenant adulte est contraint par deux ensembles de facteurs : des facteurs communicatifs qui constituent le moteur de l’acquisition (factors «~pushing” acquisition) et des facteurs structurels provenant de l’\textit{input} en LC (factors «~shaping” acquisition). L’exposition à l’\textit{input} permet à l’apprenant d’émettre des hypothèses sur le fonctionnement de la L2 qu’il doit vérifier en étant confronté à différentes tâches communicatives. Ainsi, l’apprenant cherche d’abord à saisir des séquences sonores qui correspondent à une signification lexicale. Les items de son lexique sont associés à des informations phonologiques et sémantiques. Les apprenants font plus ou moins abstraction des informations grammaticales, notamment de la morphologie flexionnelle en fonction de la parenté typologique entre la langue maternelle (langue source) et la langue à apprendre (langue cible). Les items lexicaux sont mis en relation dans des énoncés grâce à des principes d’organisation discursive que l’apprenant maîtrise en tant que locuteur natif de sa langue maternelle, ces principes étant neutres par rapport aux spécificités des langues.


\subsection{Acquisition de la morphologie flexionnelle – un défi pour l’apprenant L2}\label{sec:watorek:2.2}

De nombreuses études montrent que la morphologie flexionnelle nominale présente une difficulté majeure pour les apprenants (cf. \citealt{Larsen-Freeman2010}). Certaines recherches comme celles de \citet{Bardovi-Harlig1992}, \citet{KleinPerdue1997}, \citet{Starren2001}, montrent que les apprenants adultes encodent des concepts temporels avec des items lexicaux avant d’utiliser des marqueurs flexionnels. Dans d’autres recherches,  la question est de savoir si l’apprenant focalise son attention sur le sens plutôt que sur la forme (cf. \citealt{VanPatten2004Processing}) ou inversement (cf. \citealt{HanPeverly2007}, \citealt{Park2011}). D’après les résultats d’autres études encore, se situant dans l’approche de la première exposition et portant sur la morphologie flexionnelle, celle-ci constitue aussi une difficulté importante en L2 (cf. \citealt{CarrollWidjaja2013}, \citealt{HanLiu2013}). Néanmoins, malgré cette difficulté, il est possible d’observer que les apprenants développent très tôt (après quelques heures d’exposition à la L2) une sensibilité aux formes morphologiques de la langue cible. Par exemple, \citet{HinzEtAl2013} et \citet{RastEtAl2014} indiquent qu’après seulement quelques heures, les apprenants arrivent à juger correctement la morphologie nominale en polonais et à produire des énoncés simples en contexte avec un marquage casuel approprié.\footnote{ Afin de vérifier que la production des noms avec le marquage casuel ne soit pas liée à un apprentissage lexical non analysé du point de vue de la morphologie nominale, dans la tâche de production ciblée l’apprenant est amené à produire des noms qui sont soit présents, soit absents dans l’\textit{input}.} Ces résultats suggèrent un traitement et une production relativement précoces des formes en relation avec leurs fonctions, quelles que soient les activités dans lesquelles ont été impliqués les apprenants (cf. \citealt{WatorekEtAl2016}). La précocité du traitement et de la production de la morphologie flexionnelle pourrait être liée aux conditions d’exposition donc au type d’enseignement (\textit{focus on form} vs \textit{on meaning}), au type d’\textit{input} reçu et au profil des apprenants, ce qui expliquerait la différence dans l’utilisation de la morphologie flexionnelle entre les apprenants en milieu guidé (comme ceux de l’étude présentée ici) et en milieu non guidé comme ceux du projet ESF \citep{Perdue1993}.  

En effet, les apprenants de notre étude apprennent la LC dans un cours de langue intensif et sont des apprenants universitaires d’instruction élevée \citep{Bartning1997}. En revanche, les apprenants du projet ESF sont des migrants faiblement scolarisés qui apprennent la LC en milieu naturel à travers les contacts quotidiens. De plus, la différence entre ces deux populations d’apprenants pourrait résider dans le type de méthodologie de recueil des données. Les acquis des apprenants de notre étude sont mesurés juste après l’exposition aux données cibles et donc sur un empan temporel beaucoup plus court que dans le projet ESF. On peut imaginer que l’aspect intensif de l’apprentissage joue également un rôle. La différence entre les deux types d’apprenants (guidés vs non guidés) en termes de réussite relative dans la maîtrise de la morphologie flexionnelle est largement discutée dans la section 3.1 de \citet{WatorekEtAl2021}.


\section{Questions de recherche}\label{sec:watorek:3}

Nous exposons dans cette contribution les résultats d’une étude exploratoire se basant sur une méthodologie de recueil des données élaborée en vue de la mise en place du projet européen VILLA qui  s’inscrit dans l’approche méthodologique de «~first exposure~». Il s’agit de proposer un cours d’initiation à une nouvelle langue dont l’\textit{input} fourni par l’enseignant est entièrement contrôlé et donc enregistré pour une documentation complète de ce à quoi les apprenants ont eu l’accès. 

Voici quatre questions qui guident l’étude:

\begin{itemize}
\item Les apprenants ayant reçu l’\textit{input} axé sur la forme réussissent-ils mieux à traiter des formes grammaticales enseignées ?
\item Les apprenants ayant reçu l’\textit{input} axé sur la forme produisent-ils plus de formes appropriées dans des structures morpho-syntaxiques complexes ?
\item Les apprenants ayant reçu l’\textit{input} axé sur la forme réussissent-ils mieux à généraliser des règles sur de nouveaux items lexicaux ?
\item Les apprenants ayant reçu l’\textit{input} axé sur le sens produisent-ils des discours plus longs, composés d’un plus grand nombre d’énoncés mais avec plus d’erreurs dans le marquage casuel ?
\end{itemize}

\section{Méthodologie}\label{sec:watorek:4}

L’objectif de notre étude est de mener une réflexion sur les procédures permettant d’analyser le traitement de l’\textit{input} en tout début d’acquisition d’une nouvelle langue. Nous nous limitons dans cet article à la méthodologie de recueil des données pour examiner l’impact de la focalisation explicite sur la forme enseignée sur son appropriation à travers le procédé ponctuel et systématique de mise en relief de la forme pour attirer l’attention de l’apprenant. L’effet de ce procédé est testé par des tâches de traitement et de production.

Le dispositif méthodologique possède deux volets. D’une part, il s’agit d’élaborer le cours de polonais en deux versions avec une focalisation plus ou moins importante sur la forme enseignée. D’autre part, en relation avec le contenu de la séance et l’\textit{input} reçu, il s’agit de mettre en place des tâches permettant de tester les acquis des apprenants en traitement et en production


\subsection{Participants}\label{sec:watorek:4.1}

Un groupe de 14 apprenants, étudiants en licence de Sciences du langage, option FLE (Français Langue Etrangère), a été exposé à un cours d’initiation au polonais dans le cadre d’un cours d’auto-observation de l’apprentissage d’une langue nouvelle à l’Université Paris 8.\footnote{ {Le cours d’auto-observation est un cours obligatoire pour les étudiants en licence de Sciences du langage (SDL), option Français langue étrangère (FLE), futurs enseignants de langues. L’objectif est de les mettre en situation d’apprentissage d’une langue aussi éloignée que possible de leur langue maternelle et d’autres langues étrangères déjà apprises. En adoptant le point de vue de l’apprenant, les étudiants effectuent une auto-observation sur les différents facteurs en jeu dans l’enseignement/apprentissage d’une langue étrangère.}} Les étudiants participant à cette expérience sont âgés entre 22 et 35 ans et n’ont jamais été exposés auparavant ni au polonais ni à aucune autre langue slave. Nous n’avons  pas pu contrôler la variable langue(s) source(s) dans la mesure où cette étude exploratoire est menée dans une formation universitaire s’adressant à un ensemble d’étudiants de langues diverses. Ces étudiants ont suivi le cours de langue durant un semestre à raison d’une séance de 90 minutes (2 fois 45 minutes) par semaine, soit 8 séances, ainsi que la 9\textsuperscript{ème} et la dernière séance de 45 minutes. Toutes les séances ont été enregistrées sous format vidéo et audio afin de documenter la totalité de l’\textit{input} reçu en polonais.

Les 8 premières séances (12h au total) consistent en un cours faisant appel à l’approche communicative, axé sur le sens et sur la communication, entièrement en L2, sans recours au français et sans aucune explicitation des règles grammaticales enseignées. Sans que l’écrit ne fasse l’objet d’un enseignement, ce dernier est présent tout au long du cours via les diaporamas présentés lors des leçons et les polycopiés distribués aux étudiants. Cependant, les activités et les exercices proposés dans le cours ne sont réalisés qu’à l’oral.

Afin d’examiner l’effet de la focalisation sur la forme et de voir comment l’étudier, lors de la 9\textsuperscript{ème} séance, les apprenants ont été divisés en deux groupes (FB pour \textit{Form based input}, et MB pour \textit{Meaning based} \textit{input}) de 7 sujets chacun. Etant donné que tous les apprenants étaient des débutants complets n’ayant jamais été en contact avec la langue polonaise ni aucune autre langue slave (informations recueillies grâce à un questionnaire), nous avons considéré qu’ils avaient un niveau comparable et similaire au bout des 8 séances. Nous avons écarté de l’expérience les étudiants qui n’avaient pas été présents à chacune des séances.

Le groupe FB a été exposé à un cours axé sur la forme \footnote{ Dans cet article, nous employons la terminologie adoptée par le projet VILLA pour différencier les deux types de cours, \textit{Form based input}, cours axé sur la forme enseignée, et \textit{Meaning based input}, cours axé sur le sens (cf. \citealt{DimrothEtAl2013}).}, tandis que le groupe MB a eu un cours axé sur le sens sans explicitation des règles (,. Les deux variantes de la 9\textsuperscript{ème} et dernière séance de 45 minutes avaient le même contenu lexical, grammatical et communicatif avec une différence qui concernait la présentation des formes enseignées et le type d’activités proposées. 

L’enseignante a varié le type de feedback correctif dans les deux variantes de la séance 9. Celui du groupe FB étant axé sur la correction des formes contrairement à celui du groupe MB plus axé sur le contenu des messages produits par des apprenants. Malheureusement,  l’analyse du type de feedback correctif n’a pas pu être conduite, car les conditions de l’étude exploratoire n’ont pas permis d’enregistrer de façon systématique et fiable les échanges entre les apprenants et l’enseignante. En effet, seule la parole de l’enseignante a été enregistrée à l’aide du micro-cravate. Cet enregistrement n’a pas pu saisir les interventions des apprenants de façon audible. De plus, la caméra était placée de façon à filmer essentiellement l’enseignante.


\subsection{La morphologie nominale en polonais}\label{sec:watorek:4.2}

L’ensemble des cours de polonais visait en premier lieu l’appropriation de la morphologie nominale qui présente une différence majeure par rapport à celle du français.

Le polonais, membre du groupe occidental des langues slaves, est doté d’une morphologie verbale (4~conjugaisons) et d’une morphologie nominale (5~déclinaisons) très riche. A titre informatif, nous proposons le tableau ci-dessous qui résume la complexité du système casuel. 


\begin{table}
\caption{\label{tab:watorek:3} Le système casuel du polonais}
\begin{tabular}{l lll}
\lsptoprule
    & \multicolumn{3}{c}{Singulier}\\\cmidrule(lr){2-4}
Cas & Masculin & Neutre & Féminin\\\midrule
Nominatif & ${\emptyset}$, -a & -o & -a\\
Génitif & -a, -u & -a & -y -i\\
Datif & -owi, -u & -u & -e\\
Accusatif & -a & -o & -ę\\
Instrumental & -em & -em & -ą\\
Locatif & -e, -u & -e & -e\\
Vocatif & -e, -u & -o & -o\\
\lspbottomrule
\end{tabular}
\end{table}

L’accord est marqué massivement non seulement entre le sujet et le verbe (marquage systématique de la personne et du nombre, et, au passé, du genre), mais aussi entre le substantif et l’adjectif et certains numéraux (genre, nombre et cas). Cette riche morphologie va de pair non seulement avec une organisation pragmatique des constituants mais aussi avec l’absence du pronom sujet (sujet nul), excepté dans des contextes de contraste, et avec l’absence de l’article, à l’exception du démonstratif. 

Concernant la morphologie nominale du polonais, contrairement au français, elle permet de marquer non seulement le genre et le nombre, mais également la fonction syntaxique. Il convient de souligner que le polonais est une langue non configurationnelle, c’est-à-dire qu’elle se caractérise par un ordre des mots relativement libre. Ce sont les morphèmes casuels qui marquent les relations entre les différents constituants de la proposition. Le marquage casuel du nom varie également en fonction de la préposition. 

Dans la suite de cette section, nous présentons d’abord le contenu de la séance 9 et sa présentation dans les deux variantes de cette séance (\sectref{sec:watorek:4.3}) et les tâches de langues auxquelles les apprenants ont été exposés (\sectref{sec:watorek:4.4}).


\subsection{Contenu de la séance 9}\label{sec:watorek:4.3}

Les résultats présentés dans cet article portent sur la 9\textsuperscript{ème} et dernière séance du semestre, les apprenants étudiés ayant été exposés au préalable à douze heures d’\textit{input} en polonais. Cette séance a pour objectif de faire acquérir les moyens linguistiques permettant d’exprimer la description spatiale statique, type de discours où le locuteur localise des objets et des personnes qui ne sont pas en déplacement, les uns par rapport aux autres, et la description dynamique où le locuteur exprime le déplacement dans l’espace des objets/personnes. 

Pour ce faire, l’apprenant a besoin de maîtriser un certain nombre de verbes de localisation et de déplacement ainsi que des prépositions de lieu qui impliquent un marquage casuel sur le nom s’intégrant dans un syntagme prépositionnel (Sprép). Les objectifs fonctionnels de cette séance consistent donc à demander et indiquer le chemin (description dynamique) et à savoir localiser un endroit (description statique).

Les objectifs linguistiques renvoient au lexique verbal avec l’introduction des verbes de mouvement et déplacement \textit{iść} `aller', \textit{skręcić} `tourner' et des verbes statiques \textit{być} `être' et \textit{znajdowac się} `se trouver'. Pour réaliser la tâche d’indication d’itinéraire, la construction «~\textit{trzeba} + infinitif `il faut' + infinitif a été proposée dans l’input. Pour ce qui est du lexique nominal, les noms suivants ont été présentés dans cette leçon : \textit{muzeum} `musée', \textit{centrum} `centre', \textit{plac} `place', \textit{rynek} `place du marché', \textit{szkoła} `école', \textit{hotel} `hôtel', \textit{dom} `maison', \textit{turysta} `touriste', ulica `rue', \textit{kosciół} `église'.

Les prépositions et adverbes spatiaux \textit{do} `jusqu’à', \textit{na} `sur', \textit{na wprost} `en face', \textit{w prawo} `à droite', \textit{w lewo} `à gauche', \textit{prosto} `tout droit', introduits dans la séance 9 impliquent différents cas. Par ailleurs, le syntagme nominal (SN) qui suit le verbe de mouvement `aller' \textit{iść} est marqué par l’instrumental pour exprimer le déplacement.

L’\textit{input} en polonais présenté aux apprenants dans la séance 9 visait en particulier l’emploi de trois prépositions : \textit{w} `dans'~+~accusatif,  \textit{do} ~'jusqu’à'~+~génitif, \textit{na} `sur' + locatif (cf. \tabref{tab:watorek:4}). Le \tabref{tab:watorek:3} en \sectref{sec:watorek:4.2} résume la flexion nominale qui a fait l’objet d’un enseignement lors de cette séance.


\begin{sidewaystable}
\caption{\label{tab:watorek:4}La flexion nominale en polonais enseignée lors de la séance 9}
\begin{tabularx}{\textwidth}{Ql>{\raggedright}p{\widthof{\textit{Rynek} -${\emptyset}$/'marché'}}>{\raggedright\arraybackslash}p{\widthof{Centrum/centre}}} 
\lsptoprule
& Nom féminin & Nom masculin & Nom neutre \footnote{nom d’origine étrangère, invariable}\\\midrule
\textit{Iść do} `aller jusqu’à' + SN-Génitif & \textit{Ulic}\textbf{y} `rue'

\textit{Szkoł\textbf{y}}'école' & \textit{Rynk\textbf{u}} `marché'

\textit{Plac\textbf{u}} `place'

\textit{Dom\textbf{u}} `maison' & \textit{Centrum} `centre'

\textit{Muzeum} `musée'\\
\textit{Iść} `aller'

+ SN-Instrumental & \textit{Ulic\textbf{ą}} `rue' & \textit{Rynk\textbf{iem}} `marché' & \\
\textit{Skręcić w} `tourner dans' + SN-Accusatif & \textit{Ulic\textbf{ę}} `rue' & \textit{Rynek} -${\emptyset}$/'marché' & \\
\textit{Iść} `aller'/ \textit{skręcić} `tourner' \textit{na} `sur' + SN-Accusatif & \textit{Ulic\textbf{ę}} `rue' & \textit{Plac} \textbf{{}-${\emptyset}$} `place'

\textit{Rynek} `marché' & \\
\textit{Jest} `est'/3psg être / \textit{znajduje się} `se trouve'/3psg se trouver \textit{na} `sur' + SN-Locatif & \textit{Ulic\textbf{y}} `rue' & \textit{Plac\textbf{u}} `place'.

\textit{Rynk\textbf{u}} `marché' & \\
\textit{Jest} `est' /3psg être / \textit{znajduje się} `se trouve' /3psg se trouver \textit{w} `dans' + SN-Locatif & \textit{Ulic\textbf{y}} `rue' & \textit{Rynk\textbf{u}} `marché'

\textit{Dom\textbf{u}} `maison'

\textit{Hotel\textbf{u}} `hôtel' & \\
\lspbottomrule
\end{tabularx}
\end{sidewaystable}

Afin d’enseigner ce matériel linguistique, la séance 9 porte sur la visite de Cracovie, une ville historique de la Pologne, ce qui a été illustré par un dialogue fabriqué entre une touriste et un habitant de Cracovie. C’est à travers du discours instructionnel produit dans le cadre d’une tâche d’indications d’itinéraire que les apprenants ont été exposés à la morphologie nominale. A partir d’un plan de la ville, les apprenants devaient retrouver les informations pertinentes. La séance de 45 minutes a été surtout axée sur l’écoute, le repérage des informations, la compréhension et dans une moindre mesure sur la production.

La distinction que nous avons introduite dans cette étude pour différencier les deux séances concerne les supports de cours par la mise en relief graphique des formes enseignées et les activités/exercices proposés ainsi que le recours au métalangage, quoique dans une moindre mesure. En effet, l’enseignante n’utilise que la langue cible dans le cours, ce qui entrave l’emploi du métalangage, notamment parce qu’il s’agit d’un cours pour débutants. Le polonais et le français étant des langues relativement éloignées, l’enseignant ne peut pas compter sur la transparence des termes métalinguistiques, ce qui serait possible dans le cas de langues apparentées.

L’enseignante a fait en sorte également de corriger des erreurs de formes dans le groupe FB et de n’intervenir auprès des apprenants du groupe MB qu’en cas de problème de communication lié au sens de l’échange. Le groupe FB a été exposé à de l’\textit{input} axé sur la forme où les formes enseignées ont été surlignées sur les supports de cours (diaporamas). Des tableaux de synthèse grammaticale ont été proposés aux apprenants afin d’attirer leur attention sur des phénomènes linguistiques, notamment la flexion nominale, et leur permettre de systématiser le contenu grammatical. Par ailleurs, les apprenants ont dû effectuer des activités/exercices visant à mettre en œuvre les règles apprises. Il s’agit donc principalement d’une focalisation sur la forme non interactive \citep{Long2015}, puisque les apprenants sont amenés à traiter un \textit{input} oral et/ou écrit dans lequel des traits spécifiques de LC ont été mis en relief.

Le groupe MB a été exposé à de l’\textit{input} axé sur le sens où les formes enseignées n’ont pas été surlignées dans le document support du cours. Aucun tableau de synthèse grammaticale ne leur a été présenté et les activités proposées aux apprenants étaient essentiellement à visée communicative. Les activités en classe consistaient à retrouver le chemin décrit sur un plan de la ville en se basant sur une description d’itinéraire donnée à l’oral. 

\begin{sloppypar}
Les treize premières diapositives de cours sont les mêmes pour les deux groupes tout en se différenciant par le surlignement des désinences casuelles (hotel\textbf{u}) et des expressions de déplacement ciblées (verbes et prépositions, en gras). L’exemple ci-dessous montre ce type de mise en relief.
\end{sloppypar}

\ea\label{fig:watorek:1}Exemple de mise en relief dans le cours \textit{Form based input}
\ea \textit{Form based input} - Dialog\\
\begin{xlist}[Monika:]
\exi{Monika:}{Przepraszam \textbf{jak trzeba iść do} \underline{muzeum} Wyspiańskiego ?
             \glt `Excusez-moi comment il faut aller au musée de Wyspianski ?'}

\exi{Pan:}{\textbf{Do} \underline{muzeum} Wyspiańskiego … tak, wiem, \textbf{trzeba iść
            prosto} \textit{ulicą} Grodzką i potem \textbf{skręcić w prawo w} ulicę Poselską a potem \textbf{w lewo w} ulicę Kanoniczą, potem znowu prosto. Muzeum \textbf{znajduje się} mniej więcej \textbf{na wprost} hotelu Copernicus.
            \glt `Au musée de Wyspianski… oui, je sais, il faut aller tout droit rue Grodzka et après tourner à droite dans la rue Poselska, et après à gauche dans la rue Kanonicza, après de nouveau tout droit. Le musée se trouve plus ou moins en face de l’hôtel Copernicus.'}

\exi{Monika:}{dziekuję panu bardzo.
            \glt `merci beaucoup Monsieur'}

\exi{Pan:}{Ależ proszę. A pani nie jest z Krakowa ?
          \glt `mais je vous en prie. Et vous n’êtes pas de Cracovie ?'}

\exi{Monika:}{Nie, jestem z Warszawy, jestem w Krakowie przez trzy dni.
            \glt `Non, je suis de Varsovie, je suis à Cracovie pour trois jours'}
\end{xlist}
\ex \textit{Meaning based input} - Dialog:\\
\begin{xlist}[Monika:]
\exi{Monika:}{Przepraszam  {jak trzeba iść do} {muzeum} Wyspiańskiego ?
            \glt `Excusez-moi comment il faut aller au musée de Wyspianski ?'}

\exi{Pan:}{Do {muzeum} Wyspiańskiego … tak, wiem, {trzeba iść prosto} {ulicą} Grodzką i potem {skręcić w prawo w} ulicę Poselską a potem {w lewo w} ulicę Kanoniczą, potem znowu prosto. Muzeum {znajduje się} mniej więcej {na wprost} hotelu Copernicus.
            \glt `Au musée de Wyspianski… oui, je sais, il faut aller tout droit rue Grodzka et après tourner à droite dans la rue Poselska, et après à gauche dans la rue Kanonicza, après de nouveau tout droit. Le musée se trouve plus ou moins en face de l’hôtel Copernicus.'}

\exi{Monika:}{dziekuję panu bardzo.
            \glt `merci beaucoup Monsieur'}

\exi{Pan:}{Ależ proszę. A pani nie jest z Krakowa ?
            \glt `mais je vous en prie. Et vous n’êtes pas de Cracovie ?'}

\exi{Monika:}{Nie, jestem z Warszawy, jestem w Krakowie przez trzy dni.
            \glt `Non, je suis de Varsovie, je suis à Cracovie pour trois jours'}
\end{xlist}
\z
\z

Les diapositives suivantes (cf. annexe A) proposées au groupe MB constituent des supports pour les activités de repérage et de compréhension. Par exemple, à l’aide du plan de la ville, les apprenants doivent retrouver le chemin. Les diapositives de 14 à 21 proposées uniquement au groupe FB sont des tableaux de synthèse grammaticale permettant d’organiser les contenus grammaticaux vus dans le dialogue. La figure 2 montre un des tableaux de synthèse utilisés dans le cours FB.

  
\begin{figure}
% \begin{table}
% 
% \begin{tabular}{lll}
% & żeński &  męski\\
% Iść do & ulic-y &
% \end{tabular}
% \caption{Synteza gramatyczna\\Opis dynamiczny}
% \end{table}
\caption{\label{fig:watorek:2}Exemple de tableau de synthèse grammaticale dans le cours FB}
\includegraphics[width=.75\textwidth]{figures/Watorek-img001.jpg}
\end{figure}
 


En fin de la séance un exercice à trous de systématisation (voir ci-dessous) a été proposé. En suivant la consigne l’apprenant doit trouver la bonne forme à réemployer dans un discours similaire à celui du dialogue. 


\ea\label{fig:watorek:3}
Exemple d’un exercice à trou de systématisation dans le cours FB
\gll \textbf{Ćwiczenie:} \textbf{proszę} \textbf{użyć} \textbf{dobrej} \textbf{formy} \textbf{:} \textbf{\textit{ulica} \textbf{/} \textbf{plac} \textbf{/} \textbf{rynek}}\\
      Exercice: demander1p.sg. utiliser   bonne.Adj.ACC           forme.ACC : rue / place / place du marché\\
\glt (Exercice : SVP utilisez la bonne forme : rue / place / place du marché)\medskip\\

\gll \textbf{Trzeba}             \textbf{iść}   \textbf{prosto} {\longrule} \textbf{Szpitalną},\\
      falloir.3P.SG   aller   droit {\longrule} Szpitalna.Npropre.INSTR,\\
\gll  potem   skręcić  w lewo    w {\longrule} Solskiego\\
      après    tourner  à gauche dans {\longrule} Solskiego.Npropre.GEN\\
\glt (\textit{Il faut aller tout droit {\longrule} Szpitalna, après tourner à gauche dans {\longrule} Solskiego})\medskip\\
\gll i   iść   do {\longrule} Wiosny Ludów\\
      et   aller   jusqu’à {\longrule} Wiosny Ludow.Npropre.ACC\\
\glt (\textit{et aller jusqu’à {\longrule} Wiosny Ludow})\medskip\\
\gll \textbf{Na} \textbf{...} \textbf{znajduje} \textbf{się}     \textbf{kościół}   \textbf{Mariacki}.\\
      Sur {\longrule} se trouver.3P.SG   église.NOM   Mariacki.Npropre.GEN\\
\glt (\textit{Sur {\longrule} se trouve l’église Mariacki})\medskip\\
\gll \textbf{Na} \textbf{wprost} {\longrule} \textbf{Coprnicus} \textbf{jest} \textbf{...} \textbf{Wyspiańskiego}.\\
      En face {\longrule} Copernicus.Npropre.NOM      être.3P.SG {\longrule} Wyspianski.Npropre.GEN\\
\glt (\textit{En face de {\longrule} Copernicus est {\longrule} Wyspianski})\medskip\\
\gll \textbf{Trzeba}   \textbf{skręcić}   \textbf{w} \textbf{lewo}   \textbf{na} {\longrule} \textbf{Bracką}.\\
      faut     tourner   à gauche   dans {\longrule} Bracka.Npropre.ACC\\
\glt (\textit{Il faut tourner à gauche dans {\longrule} Bracka})
\z


\subsection{Les tâches}\label{sec:watorek:4.4}

Les apprenants ont été testés directement après la séance par deux tâches, une de jugements de grammaticalité et l’autre de production ciblée. Etant donné que la 9\textsuperscript{ème} séance était la dernière du semestre, il n’a pas été possible de continuer l’observation sur l’assimilation des acquis à plus long terme. Avec les deux tâches, nous avons donc procédé à une vérification immédiate des acquis de la séance. Les deux tâches ont permis de mesurer l’appropriation de la morphologie nominale, à la fois en traitement (tâche de jugements de grammaticalité) et en production (tâche d’indication d’itinéraire). L’analyse des réponses à ces tâches vise à répondre aux questions de recherche qui guident cette étude et sont formulées en \sectref{sec:watorek:3}.


\subsubsection{Tâche de jugement de grammaticalité}\label{sec:watorek:4.4.1}

Cette tâche consiste à écouter une série de phrases et à cocher dans une grille si la phrase entendue est correcte ou pas en polonais. La consigne a été formulée en français de la façon suivante : «~Vous allez entendre des phrases. Pour chaque phrase entendue, cochez la case, si la phrase est correcte ou pas correcte~».

Nous avons soumis au jugement des apprenants 64 phrases au total dont 10 sont des distracteurs, 39 correctes et 15 incorrectes. Les phrases cibles sont de quatre types : \REF{ex:watorek:1} phrases correctes dont les constituants ont été présentés dans l’\textit{input} du cours, \REF{ex:watorek:2} phrases correctes où seuls les syntagmes prépositionnels (Sprép) étaient dans l’\textit{input}, \REF{ex:watorek:3} phrases correctes où le nom du Sprép correspond à un item lexical nouveau et absent de l’\textit{input}, et \REF{ex:watorek:4} phrases dont les constituants ont été présentés dans l’\textit{input} et où le nom dans le Sprép contient une erreur de marquage casuel par rapport à la contrainte imposée par la préposition utilisée.

Toutes les phrases expriment un déplacement ou une localisation spatiale et contiennent des Sprép où le nom doit avoir un marquage casuel approprié par rapport à la préposition utilisée.

Ainsi, les jugements par rapport aux phrases de type 1 (phrases correctes) permettent de voir si l’apprenant reconnaît la morphologie nominale dans une structure présente dans l’\textit{input}.

\ea%1
    \label{ex:watorek:1}
\gll \textbf{\textit{Trzeba}}   \textbf{iść}   \textbf{prosto}     \textbf{do}   \textbf{szkoły}\footnotemark \\
faut     aller   {tout droit} à   {école.GEN}\\
\footnotetext{Nous utilisons les caractères gras pour indiquer les parties de la phrase présentes dans l’\textit{input}.}
\glt ‘Il faut aller tout droit à l’école.’
\z

Les jugements des phrases de type 2 (phrases correctes) montrent si l’apprenant est capable de reconnaître le marquage casuel à l’intérieur d’un Sprép présent dans le cours inséré dans une structure phrastique absente de l’\textit{input}.

\ea%2
    \label{ex:watorek:2}
\gll \textit{Należy}   \textit{wejść} \textbf{\textit{w}}   \textbf{ulicę}     \textbf{Bracką}.\\
  convient   entrer     dans   rue.ACC   Bracka.ACC\\
\glt ‘Il convient d’entrer dans la rue Bracka.’
\z

Avec les phrases de type 3 (phrases correctes), nous testons la capacité à généraliser la règle du marquage casuel lorsque le nom dans le Sprép (\textit{aleje}) est nouveau et n’a pas été vu en cours.

\ea%3
    \label{ex:watorek:3}
\gll \textbf{\textit{Trzeba}}   \textbf{skręcić}   \textbf{w}   \textbf{prawo}   \textbf{w} \textit{aleję}.\\
  faut     tourner   à   droite   dans   avenue.ACC\\
\glt ‘Il faut tourner à droite dans l’avenue.’
\z

Et finalement, les phrases de type 4 (phrases incorrectes) mesurent la capacité à reconnaître une erreur dans le marquage casuel sur le nom faisant partie du Sprép. Ces phrases sont construites avec des éléments présents dans l’\textit{input}.

\ea%4
    \label{ex:watorek:4}
\gll \textbf{\textit{Trzeba}}   \textbf{iść}   \textbf{prosto}   \textbf{do}   \textbf{*dom}  \textit{(attendu: domu)}\\
  faut     aller   {tout droit} à   *maison.NOM \\
\glt ‘Il faut aller tout droit à la maison.’
\z

Dans cette phrase en \REF{ex:watorek:4}, la forme correcte du nom serait \textit{domu}, ce qui correspond au cas Génitif. L’ensemble des phrases soumises aux jugements se trouve en annexe B.


\subsubsection{Tâche d’indications d’itinéraire}\label{sec:watorek:4.4.2}

La tâche d’indications d’itinéraire est une tâche semi-dirigée. L’apprenant, en se servant du plan de la ville (en annexe C), doit jouer le rôle de l’habitant de Cracovie et répondre à un touriste joué par l’enquêteur pour savoir comment se rendre d’un endroit à l’autre. Les points de départ et d’arrivée sont indiqués sur le plan. Le discours attendu a été travaillé pendant le cours à travers le dialogue, support de la séance. Le même dialogue a été utilisé pour les deux groupes, MB (\textit{Meanig based)} et FB (\textit{Form based)}.

\ea    \label{ex:watorek:5}
trzeba iść prosto \textit{ulicą} Grodzką i potem skręcić w prawo \textbf{w} \textbf{\textit{ulicę}} \textbf{Poselską} \textbf{(ACC}   \textbf{Fem)} a potem w lewo \textbf{w} \textbf{ulicę} \textbf{Kanoniczą} \textbf{(ACC} \textbf{Fem)}, potem znowu prosto.   Muzeum znajduje się mniej więcej \textbf{na} \textbf{wprost} \textbf{hotelu} \textbf{(GEN} \textbf{Masc)} Copernicus.
\glt `il faut aller tout droit et après tourner à droite dans la rue Poselska et après à   gauche dans la rue Kanonicza, ensuite de nouveau tout droit. Le musée se trouve   plus ou moins en face de l’hôtel Copernicus.’
\z

L’objectif de cette tâche est de solliciter un discours instructionnel à visée spatiale qui implique l’emploi des verbes de mouvement et des expressions locatives (Sprép) où le nom reçoit un marquage casuel variant en fonction de la préposition spatiale.


\section{Résultats}\label{sec:watorek:5}


\subsection{Jugements de grammaticalité}\label{sec:watorek:5.1}

Dans l’ensemble (cf. \figref{fig:watorek:4}, ci-dessous), le groupe FB (\textit{Form-Based}) présente de meilleurs résultats que le groupe MB (\textit{Meaning-Based}). Cependant, les scores corrects sont assez élevés et dépassent 50\% pour les deux groupes. La différence entre les deux groupes avec un léger avantage pour le groupe FB se maintient dans les scores des jugements des phrases de type 1, 2 et 4.

Le type 1 correspond au meilleur score pour les deux groupes, ce qui peut être expliqué par le fait que ces phrases correspondent à ce qui a été vu en cours. Le type 2 est plus difficile, il s’agit de juger l’acceptabilité du marquage casuel d’un nom dans une nouvelle structure phrastique, absente de l’\textit{input}. Dans le type 4 où il s’agit de reconnaître une forme erronée, tous les apprenants obtiennent les scores les moins élevés. 

Le résultat concernant le jugement des phrases de type 3 attire l’attention par des scores quasi identiques dans les deux groupes (environ 60\% des réponses correctes). Pour obtenir un score correct, l’apprenant doit ici généraliser la règle en l’appliquant sur un nom nouveau, absent de l’\textit{input}. Les noms en gras dans les exemples \ref{ex:watorek:6} ci-dessous correspondent à l’item absent de l’\textit{input} (phrases de type 3) et sont présentés avec le marquage casuel correct.

\ea%6
    \label{ex:watorek:6}
      \ea
         \gll Trzeba   skręcić   w \textbf{aleję}\\
            {il faut} tourner   dans   avenue.ACC\\
            \glt ‘il faut tourner dans l’avenue.’

      \ex
      \gll Trzeba   skręcić   na \textbf{łąkę}\\
            {il faut} tourner  sur  pré.ACC\\
      \glt ‘il faut   tourner dans le pré.’

      \ex
      \gll Muzeum   jest     na wprost \textbf{sądu}\\
            Musée    est    en face    tribunal.GEN\\
      \glt ‘le musée   est     en face du   tribunal.’
      \z
      \z

\begin{figure}
\begin{tikzpicture}
  \begin{axis}[
      ybar,
      width=\textwidth,
      axis lines*=left,
      xtick=data,
      height=5cm,
      ymin=0,
      ymax=100,
      symbolic x coords = {type 1,type 2,type 3,type 4,total},
      cycle list name=langsci-greys-2,
      bar width=.75cm,
      nodes near coords,
      enlarge x limits=0.1,
      legend cell align=left,
      legend pos={south west}
      ]
      \addplot coordinates {(type 1,75) (type 2,63) (type 3,60) (type 4,45) (total,64)};
      \addlegendentry{FB}
      \addplot coordinates {(type 1,65) (type 2,52) (type 3,62) (type 4,37) (total,56.3)};
      \addlegendentry{MB}
  \end{axis}
\end{tikzpicture}
\caption{\label{fig:watorek:4} Jugements de grammaticalité : scores corrects en fonction du type de phrase}
\end{figure}




\subsection{Tâche de production}\label{sec:watorek:5.2}

En ce qui concerne la production, la différence entre les deux groupes est bien plus nette que dans le cas de la tâche de jugements de grammaticalité. Le groupe MB a plus de difficultés à employer une préposition suivie d’une forme correctement fléchie du nom: seules 5\% des formes produites par ce groupe sont correctement marquées, comme le montre la \figref{fig:watorek:5}.

\begin{figure}
\begin{tikzpicture}
  \begin{axis}[
      ybar,
      width=.66\textwidth,
      axis lines*=left,
      xtick=data,
      height=5cm,
      ymin=0,
      ymax=100,
      bar width=1ex,
      symbolic x coords = {FB,MB},
      cycle list name=langsci-greys-2,
      bar width=.75cm,
      nodes near coords,
      enlarge x limits=0.5,
      legend pos={outer north east},
      legend cell align=left
      ]
      \addplot coordinates {(FB,60) (MB,5)};
      \addlegendentry{rép. correctes}
      \addplot coordinates {(FB,40) (MB,95)};
      \addlegendentry{rép. incorrectes}
  \end{axis}
\end{tikzpicture}
\caption{\label{fig:watorek:5} Formes nominales produites correctes/incorrectes}
\end{figure}

\begin{sloppypar}
Voici deux exemples, en \REF{ex:watorek:7} et \REF{ex:watorek:8}, provenant des productions des apprenants des deux groupes. Dans ces exemples, les «~++~» indiquent des pauses plus ou moins longues. Lorsque les mots produits par les apprenants sont reconnaissables en polonais même si la prononciation n’est pas native, nous les transcrivons en orthographe polonaise. Les noms produits par des apprenants avec un marquage casuel inapproprié en langue cible sont précédés par un astérisque. Nous indiquons en gras les désinences de la flexion nominale.
\end{sloppypar}

\ea%7
    \label{ex:watorek:7}

           Groupe FB

\ea
\gll trzeba   iść   {prosto} {++}  {na:+++ w}   ulic\textbf{ę} Grodzk\textbf{ą} \\
 {il faut} aller   {tout droit} {++}   {sur +++ dans} rue Grodzka.ACC\\
\glt ‘il faut aller tout droit à la rue Grodzka.’

\ex
\gll w prawo w prawo +++   na   ulic\textbf{ę} Poselsk\textbf{ą} \\
 à droite à droite +++   sur   rue Poselska.ACC\\
\glt     ‘tout droit tout droit à la rue Poselska.’

\ex
\gll i   na   ulic\textbf{ę} Kanonicz\textbf{ą} \++ \\
    et   sur   rue   Kanonicza.ACC\\
\glt     ‘et à la rue Kanonicza.’

\ex
\gll muzeum jest na +++    na + w ++  {na + prost} {hotel\textbf{u}} {Copernicus} \\
 musée est     sur +++ sur + dans ++ {sur + en face de} {hôtel.GEN} {Copernicus}\\
\glt ‘le musée est en face de l’hôtel Copernicus.’
\z
\z

\ea%8
    \label{ex:watorek:8}
           Groupe MB

  \ea
  \gll prosto   prosto  {na} {*ulic\textbf{y}} *Grodzk\textbf{a}\\
 droit   droit     {sur} {rue.LOC} Grodzka.NOM\\
\glt     ‘tout droit à la rue Grodzka.’

 \ex
\gll {w prawo} {na} {*ulic\textbf{y}} {+} {eh *Poselsk\textbf{a}}\\
 {à droite} {sur} {rue.LOC} {+} {eh Poselska.NOM}\\
\glt     ‘à droite à la rue Poselska.’

\ex
\gll {+ eh w lewo} {+ w lewo} ++\\
 {+ eh à gauche} {+ à gauche} ++\\
\glt     ‘à gauche à gauche.’

\ex
\gll prosto prosto na {*ulic\textbf{y}} {Kanoni ++} {*Kanonicz\textbf{a}}\\
 droit   droit   sur  {rue.LOC}   {Kanoni ++}       {Kanonicza.NOM}\\
\glt    ‘tout droit tout droit à la rue Kanonicza.’

\ex
\gll i w {lewo} \\
 et à gauche\\
\glt     ‘et à gauche.’

\ex
\gll {muzeum}   {jest}   {wprost}   *Copernicus *hotel \\
 {musée}   {est}   {en face}   Copernicus.NOM hôtel.NOM\\
\glt     ‘le musée est en face de l’hôtel Copernicus.’
\z
\z

Les exemples ci-dessus montrent que les discours produits par les apprenants des deux groupes sont comparables à la fois au niveau de la longueur (24 énoncés en moyenne) et de l’efficacité communicative. De même, on atteste relativement peu de formes idiosyncrasiques parmi les formes nominales produites. Dans les productions du groupe MB, lorsque la désinence ne correspond pas à la forme appropriée dans le contexte de la phrase, on observe d’autres désinences possibles en polonais mais inappropriées par rapport à la préposition utilisée, par exemple, le locatif, \textit{na ulicy} `sur la rue' en \ref{ex:watorek:8} au lieu de l’accusatif \textit{na ulic\textbf{ę}}. En effet, dans le contexte dynamique, la préposition \textit{na} implique l’accusatif (\textit{i}\textbf{\textit{ś}}\textit{ć na ulicę} `aller sur la rue' tandis que dans le contexte statique, l’emploi du locatif s’impose \textit{być na ulicy} `être sur la rue'. 

Par ailleurs, bien que les apprenants des deux groupes produisent plus d’énoncés sans verbe qu’avec un verbe, le groupe FB produit plus d’énoncés verbaux (49\%) que le groupe MB qui en produit seulement 33\%. Si l’on regarde de plus près la structure des énoncés avec un verbe, on constate une différence entre les deux groupes en ce qui concerne la complexité de l’énoncé, comme le montre la \figref{fig:watorek:6}.

\begin{figure}
\caption{Enoncés avec un verbe : V+ SP/SN ou V}
\label{fig:watorek:6}
\begin{tikzpicture}
  \begin{axis}[
      ybar,
      width=.66\textwidth,
      axis lines*=left,
      xtick=data,
      height=5cm,
      ymin=0,
      ymax=100,
      bar width=1ex,
      symbolic x coords = {FB,MB},
      cycle list name=langsci-greys-2,
      bar width=.75cm,
      nodes near coords,
      enlarge x limits=0.5,
      legend pos={outer north east},
      legend cell align=left
      ]
      \addplot coordinates {(FB,75) (MB,44)};
      \addlegendentry{V-SP/SN}
      \addplot coordinates {(FB,25) (MB,56)};
      \addlegendentry{V}
  \end{axis}
\end{tikzpicture}
\end{figure}

Le groupe FB produit nettement plus d’énoncés complexes de type \textit{Skrec (=skrecić) w prawo na ulicy Poselska} `tourner à droite sur la rue Poselska' tandis que le groupe MB produit davantage d’énoncés avec le verbe seul ou seulement accompagnés d'une expression adverbiale \textit{~iść w lewo} `aller à gauche'.

Dans les énoncés avec un verbe, le groupe FB produit également des formes nominales correctes tandis que le groupe MB n’en produit pas du tout (cf. en \REF{ex:watorek:10} et \figref{fig:watorek:7}).

\ea%9
    \label{ex:watorek:9}

           Groupe FB

\gll trzeba   iść   na   ulicę Grodzką\\
 {il faut}   aller   sur   rue Grodzka\\
\glt `il faut aller à la rue Grodzka.’
\z

\ea%10
    \label{ex:watorek:10}
           Groupe MB

\gll iść   na   *ulicy *Poselska\\
 aller   sur   rue Poselska\\
\glt ‘aller à la rue Poselska.’
\z

\begin{figure}
\caption{\label{fig:watorek:7}Marquage casuel sur le nom dans les énoncés avec le verbe}
\begin{tikzpicture}
  \begin{axis}[
      ybar,
      width=.66\textwidth,
      axis lines*=left,
      xtick=data,
      height=5cm,
      ymin=0,
      ymax=100,
      bar width=1ex,
      symbolic x coords = {FB,MB},
      cycle list name=langsci-greys-2,
      bar width=.75cm,
      nodes near coords,
      enlarge x limits=0.5,
      legend pos={outer north east},
      legend cell align=left
      ]
      \addplot coordinates {(FB,44) (MB,0)};
      \addlegendentry{correcte}
      \addplot coordinates {(FB,56) (MB,100)};
      \addlegendentry{incorrecte}
  \end{axis}
\end{tikzpicture}
\end{figure}

\begin{sloppypar}
En ce qui concerne les énoncés sans verbe se composant uniquement d’un SPrep (prép-N), le groupe FB produit majoritairement des formes correctes contrairement au groupe MB (cf. \figref{fig:watorek:8}).
\end{sloppypar}

\begin{figure}
\caption{Marquage casuel sur le nom dans les énoncés sans verbe}
\label{fig:watorek:8}
\begin{tikzpicture}
  \begin{axis}[
      ybar,
      width=.66\textwidth,
      axis lines*=left,
      xtick=data,
      height=5cm,
      ymin=0,
      ymax=100,
      bar width=1ex,
      symbolic x coords = {FB,MB},
      cycle list name=langsci-greys-2,
      bar width=.75cm,
      nodes near coords,
      enlarge x limits=0.5,
      legend pos={outer north east},
      legend cell align=left
      ]
      \addplot coordinates {(FB,90) (MB,6)};
      \addlegendentry{correcte}
      \addplot coordinates {(FB,1) (MB,94)};
      \addlegendentry{incorrecte}
  \end{axis}
\end{tikzpicture}
\end{figure}

\ea%11
    \label{ex:watorek:11}

           Groupe FB

\gll i   na   ulicę   Kanoniczą\\
 et   sur   rue   Kanonicza.ACC\\
\glt ‘et à la rue Kanonicza.’
\ex%12
    \label{ex:watorek:12}
          Groupe MB

\gll *Prosto  *hotel\\
 {tout droit}   *hôtel.NOM\\
\glt ‘tout droit l’hôtel.’
\z\largerpage

Ainsi, lorsque les apprenants des deux groupes produisent des énoncés plus complexes avec le verbe, ils font moins attention à la précision dans le marquage casuel. Ce résultat suggère qu’à cette étape d’acquisition très initiale, les apprenants des deux groupes rencontrent des difficultés à gérer à la fois la complexité syntaxique et le marquage morphologique. Cependant, le groupe FB réussit un peu mieux que le groupe MB à marquer le cas dans les énoncés complexes, mais ce marquage est minoritaire.

En revanche, la production des énoncés syntaxiquement simples conduit le groupe FB exposé à l’\textit{input} axé sur la forme, à augmenter nettement la production des formes nominales avec le marquage approprié. Le groupe MB, qui ne produit pas du tout de formes correctes dans les énoncés complexes, arrive à produire quelques formes correctes, dans les énoncés sans verbe. 

La complexité syntaxique des énoncés semble donc impacter la production de la morphologie flexionnelle appropriée.

\subsection{Jugements de grammaticalité vs production}\label{sec:watorek:5.3}

Si l’on compare les scores dans les deux tâches, on peut conclure que l’exposition à l’\textit{input} avec la focalisation sur la forme enseignée favorise clairement la production sollicitée par la tâche d’indications d’itinéraire en polonais tandis que le processus du traitement testé par la tâche de jugements de grammaticalité est affecté dans une moindre mesure, comme le montre la Figure ci-dessous.

\begin{figure}
\caption{\label{fig:watorek:9}Scores obtenus par les deux groupes dans les deux tâches}
\begin{tikzpicture}
  \begin{axis}[
      ybar,
      width=.66\textwidth,
      axis lines*=left,
      xtick=data,
      height=5cm,
      ymin=0,
      ymax=100,
      bar width=1ex,
      symbolic x coords = {Form based,Meaning based},
      cycle list name=langsci-greys-2,
      bar width=.75cm,
      nodes near coords,
      enlarge x limits=0.5,
      legend pos={outer north east},
      legend cell align=left
      ]
      \addplot coordinates {(Form based,64) (Meaning based,56.3)};
      \addlegendentry{rép. correctes JG}
      \addplot coordinates {(Form based,60) (Meaning based,5)};
      \addlegendentry{rép. correctes Production}
  \end{axis}
\end{tikzpicture}
\end{figure}

Etant donné que les items lexicaux cibles dans nos tâches correspondent aux deux noms \textit{ulica} `rue' et \textit{hotel} `hôtel', nous avons fait une analyse comparant les résultats concernant ces deux noms dans les deux tâches. 

La fréquence d’emploi de ces mots dans les deux types de cours est similaire. Le tableau ci-dessous fournit le nombre d’occurrences de ces items employés dans les deux cours par l’enseignante, ce qui a été possible grâce à la transcription de la parole de l’enseignante durant les séances analysées.


\begin{table}
\caption{\label{tab:watorek:5}Fréquence d’emploi des items lexicaux cibles dans le discours de l’enseignante dans la séance 9}
\begin{tabular}{lrr} 
\lsptoprule
            & \textit{Form based input} & \textit{Meaning based input}\\\midrule
Ulicy (Loc) & 25 & 26\\
Ulicę (Acc) & 36 & 32\\
Hotel (Nom) & 3 & 3\\
Hotelu (Gen) & 12 & 12\\
\lspbottomrule
\end{tabular}
\end{table}

En ce qui concerne l’emploi d’\textit{ulicę} `rue', comme le montre la \figref{fig:watorek:10}, le groupe FB produit un score correct aux deux tests avec un résultat très similaire et assez élevé. Le groupe MB arrive au même score que le groupe FB dans la tâche de jugements de grammaticalité mais ne produit pas cette forme de façon appropriée au contexte.


\begin{figure}
\caption{\label{fig:watorek:10}Marquage casuel du nom \textit{ulica} `rue' dans les deux tâches}
\begin{tikzpicture}
  \begin{axis}[
      ybar,
      width=.66\textwidth,
      axis lines*=left,
      xtick=data,
      height=5cm,
      ymin=0,
      ymax=100,
      bar width=1ex,
      symbolic x coords = {Form based,Meaning based},
      cycle list name=langsci-greys-2,
      bar width=.75cm,
      nodes near coords,
      enlarge x limits=0.5,
      legend pos={outer north east},
      legend cell align=left
      ]
      \addplot coordinates {(Form based,66) (Meaning based,62)};
      \addlegendentry{rép. correctes JG}
      \addplot coordinates {(Form based,62) (Meaning based,0)};
      \addlegendentry{rép. correctes Production}
  \end{axis}
\end{tikzpicture}
\end{figure}

Quant à l’item \textit{~hotel~}, les scores dans les deux tests sont plus élevés mais la même différence subsiste selon le mode d’exposition.

\begin{figure}
\caption{\label{fig:watorek:11}Marquage casuel du nom \textit{hotel} `hôtel' dans les deux tâches}
\begin{tikzpicture}
  \begin{axis}[
      ybar,
      width=.66\textwidth,
      axis lines*=left,
      xtick=data,
      height=5cm,
      ymin=0,
      ymax=100,
      bar width=1ex,
      symbolic x coords = {Form based,Meaning based},
      cycle list name=langsci-greys-2,
      bar width=.75cm,
      nodes near coords,
      enlarge x limits=0.5,
      legend pos={outer north east},
      legend cell align=left
      ]
      \addplot coordinates {(Form based,79) (Meaning based,57)};
      \addlegendentry{rép. correctes JG}
      \addplot coordinates {(Form based,80) (Meaning based,20)};
      \addlegendentry{rép. correctes Production}
  \end{axis}
\end{tikzpicture}
\end{figure}

La différence dans la production de ces deux items pourrait être liée à la transparence lexicale entre le français et le polonais. Contrairement à \textit{ulica}, \textit{hotel} est transparent par rapport au français. La transparence d’un item pourrait jouer un rôle plus ou moins important, notamment en production pour les apprenants du groupe MB. 


\subsection{Retour aux questions de recherche}\label{sec:watorek:5.4}

Rappelons nos questions de départ :

\begin{itemize}
\item Les apprenants ayant reçu l’\textit{input} axé sur la forme réussissent-ils mieux la tâche de jugements de grammaticalité ?
\item Les apprenants ayant reçu l’\textit{input} axé sur la forme produisent-ils plus de formes appropriées dans des structures morpho-syntaxiques complexes ?
\item Les apprenants ayant reçu l’\textit{input} axé sur la forme réussissent-ils mieux à généraliser des règles sur de nouveaux items lexicaux ?
\item Les apprenants ayant reçu l’\textit{input} axé sur le sens produisent-ils des discours plus longs, composés d’un plus grand nombre d’énoncés mais avec plus d’erreurs dans le marquage casuel ? 
\end{itemize}

Bien qu’il s’agisse d’une étude exploratoire où la variation du type d’\textit{input} (FB vs MB) n’a été réalisée que dans une seule séance de 45 minutes, il est possible de donner quelques éléments de réponses, à confirmer pas des études futures. Les deux premières questions reçoivent une réponse positive. Les apprenants ayant reçu l’\textit{input} axé sur la forme réussissent mieux la tâche de jugements de grammaticalité de même qu’ils produisent plus de formes appropriées dans des structures morpho-syntaxiques complexes. Quant à notre troisième question, les deux groupes ont obtenu des scores similaires dans le jugement de la catégorie 3 des phrases intégrant de nouveaux items lexicaux. Il semblerait donc que la focalisation sur la forme n’augmente pas la capacité à généraliser les règles. C’est un résultat qui demande tout de même des études plus précises à l’avenir. Enfin, la question 4 ne reçoit pas de réponse positive en ce qui concerne la longueur des énoncés, les apprenants des deux groupes produisant des discours de longueur similaire. Cependant, on peut répondre positivement à cette question en ce qui concerne le marquage casuel, qui est nettement plus faible dans les productions du groupe MB. La focalisation sur la forme a ainsi un effet non négligeable sur la production appropriée de la morphologie flexionnelle. Par ailleurs, l’analyse de la production des deux items \textit{~ulica~} `rue' et \textit{~hotel~} `hôtel' montre qu’il serait intéressant de prendre en compte le degré de transparence des mots de la langue cible par rapport à la langue source de l’apprenant (cf. \citealt{DimrothEtAl2013}).

La mise en relief des morphèmes flexionnels pour attirer l’attention des apprenants sur des formes semble aider les apprenants débutants dans les jugements de grammaticalité et dans la production de la morphologie nominale flexionnelle du polonais L2. On atteste très peu de formes idiosyncrasiques, absentes de l’\textit{input}, dans les productions des apprenants des deux groupes, contrairement aux résultats provenant de l’analyse des apprenants en milieu naturel (cf. projet ESF: \citealt{Perdue1993}, \citealt{WatorekEtAl2009}). Même si le marquage morphologique approprié dans les productions des apprenants du groupe MB est sporadique, leurs productions restent communicativement efficaces par rapport à la tâche. Ceci rejoint les résultats des travaux sur les apprenants en milieu naturel (cf. \citealt{Perdue1993}) selon lesquels les apprenants débutants acquièrent tardivement le marquage morphologique tout en maintenant l’efficacité communicative de leurs productions. 


\section{Discussion}\label{sec:watorek:6}

La présente étude, tout comme le projet VILLA, différencie les deux types d’\textit{input} essentiellement par la mise en relief des différentes informations grammaticales dans les supports didactiques. Il s’agit là plus précisément de la sémiotisation du référent\footnote{Nous entendons par «~sémiotisation~» le recours à des procédés sémiotiques visant à guider l’attention de l’apprenant sur des objets à apprendre (cf. \cite{Schneuwly2000})} qui consiste à représenter un objet par des procédés visuels. C’est une stratégie d'encodage d'un point de langue à faire comprendre et acquérir, pour faciliter la mémorisation. Ceci est rendu sur les supports de cours grâce à l’utilisation des caractères gras/italiques, des encadrements ou surlignements, couleurs, dessins, symbolisation, schématisation, fléchage, etc.

Dans une des publications récentes de Daniel Véronique (\citealt[173]{Véronique2021}) au sujet du projet VILLA qui fait suite à notre étude exploratoire, nous lisons, qu’à part la sémiotisation, 
\begin{quote}
    ~Il faudrait (…) caractériser les démarches \textit{form-based} et \textit{meaning-based} par rapport à des critères comme la place du recours à l’écrit, la place du recours à l’illustration, la nature des conduites de l’enseignant et des élèves (…). Il conviendrait également de quantifier les prises de parole de l’enseignante et des apprenants sous les deux conditions didactiques, et évidemment, de catégoriser les prises de parole de l’enseignante par rapport au script de la leçon~
\end{quote}

Comme cela a été précisé dans la section sur la méthodologie, le recours à l’écrit reste constant quel que soit le type d’\textit{input}. Pour ce qui est de la conduite de l’enseignante et des étudiants, comme nous l’avons précisé auparavant, les enregistrements de la séance 9 ne permettent pas d’analyser les prises de parole dans cette étude exploratoire. L’enseignante avait tout de même comme directive d’interagir différemment avec les élèves des deux groupes. Dans la séance du groupe FB, le \textit{feedback} correctif portait sur les formes et l’enseignante reprenait les apprenants lorsqu’ils faisaient des erreurs formelles. Dans l’autre groupe, des corrections éventuelles portaient sur le sens du message. L’enseignante interagissait avec l’apprenant en cas de malentendus de type communicatif.

Riche de l’expérience de cette étude exploratoire et avec des moyens plus importants, le projet VILLA dispose d’enregistrements plus précis, à la fois vidéos et audios. Notamment, les apprenants ont été enregistrés avec un dispositif multi-pistes, ce qui permettra dans des études futures d’examiner avec précision les échanges et les interactions entre apprenants, et avec l’enseignante.

Par ailleurs, \citet[174]{Véronique2021} émet une autre réticence par rapport à l’utilité des résultats du projet VILLA pour la didactique :

\begin{quote}
~Quand on réfléchit aux rapports entre les recherches acquisitionnelles et la DLE [didactique des langues étrangères ], on peut penser que les travaux en RAL [recherche en acquisition des langues (étrangères) ] peuvent aider à l’élaboration et à l’amélioration de curriculums et de syllabus linguistiques, à l’évaluation formative et sommative des apprenants, voire à améliorer la formation des maîtres et à l’amélioration des gestes pédagogiques, tout particulièrement des activités de correction en interaction. La nature du programme de VILLA ne semble pas lui permettre de contribuer dans aucun de ces trois secteurs.~
\end{quote}

Cependant, est-ce la seule façon de faire dialoguer ces deux champs disciplinaires ? La présente étude exploratoire ainsi que le projet VILLA permettent une analyse précise et systématique du traitement des formes et des items enseignés grâce au contrôle complet de l’\textit{input}. Ainsi, les résultats montrent l’impact des caractéristiques de l’\textit{input} (telles que la focalisation plus ou moins importante sur la forme) sur le traitement en fonction des capacités mobilisées par les tâches langagières auxquelles les apprenants sont confrontés, par exemple celles de l’étude présentée dans cette contribution, jugements de grammaticalité et production semi-dirigée. La conception d’une méthodologie rigoureuse permettant le contrôle de différentes variables contraint et appauvrit les aspects «~naturels~» du processus d’apprentissage\slash enseignement, d’où l’impression que les données disponibles ne pourraient pas apporter de résultats qui alimenteraient la réflexion sur l’amélioration des gestes pédagogiques.

Cependant, la manipulation de l’\textit{input} en termes de sémiotisation aboutit à des résultats qui peuvent être précieux pour l’enseignement. Notre étude exploratoire montre qu’en l’absence de focalisation sur la forme durant le cours, les apprenants portent leur attention sur le sens du message à transmettre dans une tâche communicative complexe. Par conséquent, le marquage casuel approprié est quasi absent dans les productions du groupe MB. Ce résultat a été confirmé par des recherches sur les données du projet VILLA. \citet{Latos2021} qui a analysé et comparé les productions issues de la tâche d’indications d’itinéraires réalisées par les deux groupes d’apprenants, MB et FB, aboutit à des résultats similaires. Elle atteste des différences au niveau morphosyntaxique même si l’efficacité communicative est comparable. Les apprenants exposés à l’\textit{input} FB arrivent à produire plus de formes différentes tandis que les apprenants du groupe MB produisent essentiellement des formes invariables du nominatif.

Cependant, le fait que la production soit pauvre en formes casuelles n’exclut pas la possibilité que les apprenants traitent ces formes. Les résultats de la présente étude exploratoire suggèrent que la différence entre les deux groupes n’est pas aussi nette lorsqu’il s’agit de la tâche de jugements de grammaticalité (cf. \figref{fig:watorek:9}). Ce résultat va dans le même sens que celui de \citet{WatorekEtAl2016} basé sur les données du projets VILLA où l’on a comparé la production des formes casuelles dans trois tâches de production, une tâche de répétition de phrases, une tâche de production ciblée (une question-une réponse) et une tâche communicative complexe d’indications d’itinéraire. Plus la tâche est ciblée sur la forme, plus les apprenants arrivent à produire des formes appropriées. Dans la tâche communicativement complexe, l’apprenant est centré sur le but communicatif et le sens du message. Son attention sur la forme diminue fortement.

Ces résultats montrent qu’une intervention didactique pour attirer l’attention de l’apprenant sur la forme enseignée peut accélérer l’apprentissage des formes et la mémorisation de segments plus longs. Le fait d’utiliser des supports qui permettent d’attirer l’attention sur une forme accroît la possibilité que les apprenants remarquent les caractéristiques grammaticales de la langue cible (selon l’hypothèse du «~noticing~» de \citealt{Schmidt1990}), ce qui va les aider à élaborer, mémoriser et utiliser ces formes dans leurs productions. Un tel résultat est sans doute d’une grande utilité pour la didactique des langues.


\section{Conclusions}\label{sec:watorek:7}

L’étude des premières étapes du traitement de l’\textit{input} en milieu guidé selon deux modalités différentes d’enseignement illustre bien l’intérêt à mettre en relation les résultats des études en acquisition menées dans un contexte de classe, notamment sur les premières étapes d’exposition à une nouvelle langue et les travaux en didactique. Ce type d’étude pourrait pallier  la difficulté de dialogue entre deux champs disciplinaires, recherche en acquisition des langues (RAL) et la didactique des langues étrangères (DLE), (cf. \citealt{Véronique2005Interrelations,Véronique2019,Véronique2021}).

La question de la focalisation sur la forme dans ce type de recherche s’écarte de la réflexion plus globale sur l’enseignement explicite/implicite et des recherches sur \textit{focus on form} / \textit{focus on meaning} que nous avons discutées dans la section \sectref{sec:watorek:1}. Il s’agit ici plutôt d’examiner une technique minimale d’explicitation qui attire l’attention de l’apprenant sur des morphèmes flexionnels. Dans ce sens, l’étude exploratoire présentée dans cet article a permis de formuler des questions de recherche précises sur l’impact de la manipulation des contenus grammaticaux d’un cours d’initiation à une langue. Ainsi, dans le projet VILLA la centration sur le code dans l’approche \textit{Form-based} a été opérationnalisée par la mise en relief de l’\textit{input} écrit grâce à des procédés graphiques (mise en gras, sur/soulignements, encadrés, etc) pour augmenter la saillance perceptuelle de la/des forme(s) visée(s). Cette démarche relève méthodologiquement de l’\textit{input enhancement} et conduit à évaluer la focalisation de l’attention (\textit{noticing}) chez les apprenants. A ce titre, des études comme celle présentée dans cette contribution, permettent de tester empiriquement et définir théoriquement des propositions pour l’enseignement du point de vue acquisitionnel. Une telle démarche conduit à ouvrir un champ de recherche en acquisition des langues secondes appliquée à la classe de langue et de définir un programme de recherche pour tester une proposition d’enseignement.



\sloppy\printbibliography[heading=subbibliography,notkeyword=this]

\begin{paperappendix}
%
% A. Quelques diapositives des cours : différences entre Meaning Based et Form Based input
%
% B. Tâche de Jugements de grammaticalités, 64 phrases
%
% C. Tâche d’indications d’itinéraire


\section{Quelques diapositives des cours : Différences entre Meaning Based et Form Based input}

\begin{figure}[H]
\begin{subfigure}{.5\textwidth}\centering
\fbox{\includegraphics[width=\linewidth]{figures/Watorek-Appendix-1.png}}
\caption{\textit{Meaning based input}}
\end{subfigure}\begin{subfigure}{.5\textwidth}\centering
\fbox{\includegraphics[width=\linewidth]{figures/Watorek-Appendix-2.png}}
\caption{\textit{Form based input}}
\end{subfigure}
\caption{Exemple 1}
\end{figure}\clearpage

\begin{figure}[H]
\begin{subfigure}{.45\textwidth}\centering
\fbox{\includegraphics[width=\linewidth]{figures/Watorek-Appendix-3.png}}
\caption{\textit{{Meaning based input}}}
\end{subfigure}\begin{subfigure}{.45\textwidth}\centering
\fbox{\includegraphics[width=\linewidth]{figures/Watorek-Appendix-4.png}}
\caption{\textit{Form based input}}
\end{subfigure}
\caption{Exemple 2}
\end{figure}\largerpage[2]
\begin{figure}[H]
\begin{subfigure}{.45\textwidth}\centering
\fbox{\includegraphics[width=\linewidth]{figures/Watorek-Appendix-5.png}}
\caption{\textit{Meaning based input}}
\end{subfigure}\begin{subfigure}{.45\textwidth}\centering
\fbox{\includegraphics[width=\linewidth]{figures/Watorek-Appendix-6.png}}
\caption{\textit{Form based input}}
\end{subfigure}
\caption{Exemple 3}
\end{figure}
\begin{figure}[H]
\begin{subfigure}{.45\textwidth}\centering
\fbox{\includegraphics[width=\linewidth]{figures/Watorek-Appendix-7.png}}
\caption{\textit{Meaning based input}}
\end{subfigure}\begin{subfigure}{.45\textwidth}\centering
\fbox{\includegraphics[width=\linewidth]{figures/Watorek-Appendix-8.png}}
\caption{\textit{Form based input}}
\end{subfigure}
\caption{Exemple 4}
\end{figure}

\section{Tâche de Jugements de grammaticalité, 64 phrases}

\subsection{Phrases correctes (39)}

\appendixsubsubsection{Phrases dont les constituants ont été présentés dans l’\textit{input} (17)}

\begin{exe}
\exi{9.}
\gll Trzeba   iść   prosto   do   rynku.\\
    faut     aller   droit   à   {place du marché (GEN)}\\
\glt (Il faut aller tout droit à la place du marché.)

\exi{15.}
\gll Trzeba   iść   prosto   do   szkoły.\\
      faut    aller  droit  à  {école (GEN)}\\
\glt (Il faut aller tout droit à l’école.)

\exi{18.}
\gll Trzeba   iść   prosto do   placu.\\
      faut    aller  droit  à  {place (GEN)}\\
\glt (Il faut aller tout droit à la place.)

\exi{23.}
\gll Trzeba   iść   prosto do   domu.\\
      faut    aller  droit  à  {maison (GEN)}\\
\glt (Il faut aller tout droit à la maison.)

\exi{25.}
\gll Trzeba   iść   prosto do   ulicy.\\
      faut    aller  droit  à  {rue (GEN)}\\
\glt (Il faut aller tout droit à la rue.)

\exi{45.}
\gll Trzeba   iść   ulicą     Szewską.\\
      faut    aller   {rue (INSTR)}  {Szewska (INSTR)}\\
\glt (Il faut aller sur la rue Szewska/Il faut prendre la rue Szewska.)

\exi{47.}
\gll Trzeba   iść   prosto     rynkiem.\\
      faut    aller  droit    {place du marché (INSTR)}\\
\glt (Il faut aller tout droit par la place du marché.)

\exi{55.}
\gll Trzeba   skręcić   w prawo   w   ulicę     Bracką.\\
      faut    tourner  à droite    dans  {rue (ACC)} {Bracka (ACC)}\\
\glt (Il faut tourner à droite dans la rue Bracka.)

\exi{61.}
\gll Trzeba   skręcić   w lewo     na   ulicę     Solskiego.\\
      faut    tourner  à gauche  sur  {rue (ACC)}   {Solskiego (GEN)}\\
\glt (Il faut tourner à gauche dans la rue de Solski.)

\exi{21.}
\gll Trzeba   skręcić   w prawo   na   rynek.\\
      faut    tourner  à droite    sur  {place du marché (ACC)}\\
\glt (Il faut tourner à droite sur la place du marché.)

\exi{2.}
\gll Trzeba   skręcić   w lewo     w   rynek.\\
    faut    tourner  à gauche  dans   {place du marché (ACC)}\\
\glt (Il faut tourner à gauche dans la place du marché.)

\exi{30.}
\gll Dom     Moniki     {znajduje się}   {na} {ulicy   Gołębiej.}\\
      {maison (NOM)} {Monika (GEN)} {se trouve} {sur} {rue (LOC) Gołębia (LOC).}\\
\glt (La maison de Monika se trouve dans la rue Gołębia.)

\exi{37.}
\gll Muzeum     jest   na   placu     {Wiosny Ludów.}\\
     {musée (NOM)}   est   sur   {place (LOC)} {Wiosny Ludów (GEN)}\\
\glt (Le musée est sur la place de Wiosna Ludów.)

\exi{24.}
\gll Kościół     Mariacki       jest   na   rynku.\\
     {église (NOM)} {Mariacki (Adj. NOM)} est  sur  {place du marché (LOC)}\\
\glt (L’église de Sainte Marie est sur la place du marché.)

\exi{40.}
\gll Muzeum     znajduje się   na wprost   hotelu.\\
     {musée (NOM)} se trouve   en face     {hôtel (GEN)}\\
\glt (Le musée se trouve en face de l’hôtel.)

\exi{13.}
\gll Szkoła     jest   na wprost   domu     Moniki.\\
     {école (NOM)} est   en face     {maison (GEN)}   {Monika (GEN)}\\
\glt (L’école se trouve en face de la maison de Monika.)

\exi{29.}
\gll Trzeba   skręcić   w prawo   na   rynku.\\
      faut     tourner   à droite     sur   {place du marché (ACC)}\\
\glt (Il faut tourner à droite dans la place du marché.)
\end{exe}

\appendixsubsubsection{Phrases où seuls les SPrép ont été présentés dans l’\textit{input} (13)}

\begin{exe}
\exi{8.}
\gll  Złodziej     jedzie     do     rynku.\\
     {voleur (NOM)}   va     jusqu’à     {place du marché (GEN)}\\
\glt (Le voleur va à la place du marché.)

\exi{10.}
\gll Uczniowie     biegną     do     szkoły.\\
      {élèves (NOM, pl)} courent   jusqu’à     {école (GEN)}\\
\glt (les élèves courent à l’école.)

\exi{17.}
\gll Ludzie     {spieszą się}     do     domu.\\
      {gens (NOM pl)} {se dépêchent}    jusqu’à    {maison (GEN)}\\
\glt (Les gens vont vite à la maison.)

\exi{19.}
\gll Nauczyciel   idzie   ulicą     Szewską.\\
      enseignant   va   {rue (INSTR)} {Szewska (INSTR)}\\
\glt (L’enseignant prend la rue Szewska.)

\exi{22.}
\gll Należy   wejść     w     ulicę     Bracką.\\
      convient   entrer     dans     {rue (ACC)}   {Bracka (ACC)}\\
\glt (Il convient d’entrer dans la rue Bracka.)

\exi{36.}
\gll Należy   wjechać   na   ulicę.\\
      convient   entrer    sur   {rue (ACC)}\\
\glt (Il convient d’entrer dans la rue.)

\exi{46.}
\gll Musisz     biec     na     rynek.\\
      {(tu)  dois} courir     sur     {place du marché (ACC)}\\
\glt (Tu dois courir sur la place du marché.)

\exi{51.}
\gll Musimy     wjechać     w   rynek.\\
      {(nous) devons} {entrer en voiture}   dans   {place du marché (ACC)}\\
\glt (Nous devons entrer dans la place du marché.)

\exi{54.}
\gll Zamek       widać     {na wprost}   hotelu.\\
      {château (NOM)} {(on) voit} {en face}     {hôtel (GEN)}\\
\glt (On voit le château en face de l’hôtel.)

\exi{56.}
\gll Budynek   {stoi}     {na wprost}   domu     Moniki. \\
      immeuble   {est débout} {en face}     {maison (GEN)} {Monika (GEN)}\\
\glt (L’immeuble est en face de la maison de Monika.)

\exi{62.}
\gll Wieża     stoi     na   {rynku.}\\
      {tour (NOM)} {est débout}   sur   {place du marché (LOC)}\\
\glt (La tour est sur la place du marché.)

\exi{34.}
\gll Ławka     stoi     na   placu.\\
      {banc (NOM)} {est débout} sur   {place (LOC)}\\
\glt (Le banc est sur la place.)

\exi{50.}
\gll Samochód     stoi     na   ulicy     Gołębiej.\\
     {voiture (NOM)}   {est débout}   sur   {rue (LOC)}   Gołębia\\
\glt (La voiture est dans la rue Gołębia.)
\end{exe}

\appendixsubsubsection{Phrases où seul le nom dans le Sprép est absent de l’\textit{input} (9)}

\begin{exe}
\exi{27.}
\gll Trzeba   iść   prosto   do   ogrodu.\\
      faut     aller   droit   jusqu’à   {jardin (GEN)}\\
\glt (Il faut aller tout droit jusqu’au jardin.)

\exi{64.}
\gll Trzeba   iść   prosto   do   {wieży.}\\
      faut     aller   droit   jusqu’à {tour (GEN)}\\
\glt (Il faut aller tout droit jusqu’à la tour.)

\exi{41.}
\gll Trzeba   skręcić   w prawo   w   aleję.\\
      faut     tourner   à droite     dans   {allée (ACC)}\\
\glt (Il faut tourner à droite dans l’allée.)

\exi{52.}
\gll Trzeba   skręcić   w lewo     w   zaułek.\\
      faut     tourner   à gauche   dans   {ruelle (ACC)}\\
\glt (Il faut tourner à gauche dans une ruelle.)

\exi{4.}
\gll Trzeba   skręcić   w prawo   na   targ.\\
    faut     tourner   à droite     sur   {marché (ACC)}\\
\glt (Il faut tourner à droite dans le marché.)

\exi{12.}
\gll Trzeba   skręcić   w lewo     na   łąkę.\\
      faut     tourner   à gauche   sur   {le pré (ACC)}\\
\glt (Il faut tourner à gauche dans le pré.)

\exi{58.}
\gll Muzeum   znajduje się   na wprost   sądu.\\
     musée   se trouve   en face     {tribunal (GEN)}\\
\glt (Le musée se trouve en face du tribunal.)

\exi{3.}
\gll Dom     Moniki     jest   na   skwerku.\\
    {maison (NOM)} {Monika (GEN)}  est   sur   {square (LOC)}\\
\glt (La maison de Monika est sur le square.)

\exi{63.}
\gll Szkoła     znajduje się   w   osiedlu.\\
      {école (NOM)} se trouve   dans   {cité (LOC)}\\
\glt (L’école se trouve dans la cité.)
\end{exe}

\subsection{Phrases incorrectes (15)}

Les violations concernent seulement la morphologie nominale. Les cas correspondant à la préposition ont été remplacés par le Nominatif sauf \textit{rynek} en 31 où on a utilisé la forme du génitif car la forme de l’accusatif masculin exigée ici correspond à celle du nominatif masculin.

\begin{exe}
\exi{59.}
\gll Trzeba   iść   prosto     do     rynek.\\
      faut     aller  droit     jusqu’à     {place du marché (NOM)}\\
\glt (Il faut aller tout droit jusqu’à la *place du marché.)

\exi{35.}
\gll Trzeba   iść   prosto     do     szkoła.\\
      faut     aller  droit     jusqu’à     {école (NOM)}\\
\glt (Il faut aller tout droit jusqu’à l’*école.)

\exi{7.}
\gll  Trzeba   iść   prosto     do     plac.\\
     faut     aller  droit     jusqu’à     {place (NOM)}\\
\glt (Il faut aller tout droit jusqu’à la *place.)

\exi{49.}
\gll Trzeba   iść   prosto     do     dom.\\
      faut     aller  droit     jusqu’à     {maison (NOM)}\\
\glt (Il faut aller tout droit jusqu’à la *maison.)

\exi{33.}
\gll Trzeba   iść   prosto     do     ulica.\\
      faut     aller  droit     jusqu’à     {rue (NOM)}\\
\glt (Il faut aller tout droit à la *rue.)

\exi{43.}
\gll Trzeba   iść   ulica     Szewską.\\
      faut     aller  {rue (NOM)}   {Szewska (NOM)}\\
\glt (Il faut prendre la *rue Szewska.)

\exi{16.}
\gll Trzeba   iść   prosto     na   rynek.\\
      faut     aller  droit      sur   {place du marché (NOM)}\\
\glt (Il faut aller tout droit sur la *place du marché.)

\exi{20.}
\gll Trzeba   skręcić   w prawo   w   ulica     Bracką.\\
      faut     tourner   à droite     dans   {rue (NOM)} {Bracka (NOM)}\\
\glt (Il faut tourner à droite dans la *rue Bracka.)

\exi{11.}
\gll Trzeba   skręcić   w lewo     na   ulica     Solskiego.\\
      faut     tourner   à gauche   sur   {rue (NOM})   {Solskiego (GEN)}\\
\glt (Il faut tourner à gauche dans la *rue de Solski.)

\exi{31.}
\gll Trzeba   skręcić   w lewo     w   rynku.\\
      faut     tourner   à gauche   dans   {place du marché (GEN)}\\
\glt(Il faut tourner à gauche à la *place du marché.)

\exi{44.}
\gll Dom     Moniki     znajduje się   na ulica     Gołębiej.\\
      {maison (NOM)}   {Monika (GEN)} se trouve   sur {rue (NOM)} {Gołębia (NOM)}\\
\glt (La maison de Monika se trouve sur la *rue Gołębia.)

\exi{1.}
\gll Muzeum     jest   na   plac     Wiosny Ludów.\\
    {musée (NOM)} est   sur   {plac (NOM)}   {Wiosna} Ludów (GEN)\\
\glt (Le musée est sur la *place de Wiosny Ludów.)

\exi{28.}
\gll Kościół     Mariacki     jest   na   rynek.\\
      {église (NOM)}   {Mariacki (NOM)}   est   sur   {place du marché (NOM)}\\
\glt (L’église Mariacki est sur la *place du marché.)

\exi{53.}
\gll Muzeum     znajduje się   na wprost   hotel.\\
      {musée (NOM)} se trouve   en face     {hôtel (NOM)}\\
\glt (Le musée se trouve en face de l’hôtel.)

\exi{42.}
\gll Szkoła     jest   na wprost   dom     Moniki.\\
      {école (NOM)} est   en face     {maison (NOM)}   {Monika (GEN)}\\
\glt (L’école est en face de la *maison de Monika)
\end{exe}

\subsection{Phrases distracteurs (10)}

\begin{exe}
\exi{48.}
\gll W   Krakowie     można     dobrze     zjeść.\\
      à   {Krakow (LOC)}     {(on) peut}   bien     manger.\\
\glt (A Cracovie on peut bien manger.)

\exi{26.}
\gll Pierogi     ruskie       są   najlepszym   daniem.\\
      {ravioles (NOM)}   {ruskie (NOM)}     sont   {le meilleur}   {plat (GEN)}\\
\glt (Les ravioles à la russe sont le meilleur plat.)

\exi{60.}
\gll Na     nastepne   wakacje    {} jadę     do   Niemiec.\\
      sur     suivantes   {vacances (ACC)} (je) vais en   {Allemagne (GEN)}\\
\glt (Pour les prochaines vacances je vais en Allemagne.)

\exi{39.}
\gll Studiuję   biologię     na   uniwersytecie     w   Lublinie.\\
      {(je) étudie}   {biologie (ACC)}     sur   {université (LOC)}   à   {Lublin (LOC)}\\
\glt (J’étudie la biologie à l’université à Lublin.)

\exi{6.}
\gll Dzieci     idą   do   restauracji   z   babcią.\\
    {enfants (NOM)} vont   à   restaurant   avec   {grand-mère (INSTR)}\\
\glt (Les enfants vont au restaurant avec la grand-mère.)

\exi{5.}
\gll Tato     Moniki     jest   lekarzem. \\
    {papa (NOM)} {Monika (GEN)}   est   {médecin (INSTR)}\\
\glt (Le papa de Monika est médecin.)

\exi{57.}
\gll Moja     koleżanka   jest   profesorem     francuskiego.\\
      {ma (NOM)}     {copine (NOM)}   est   {professeur (INSTR)}   {français (GEN)}\\
\glt (Ma copine est professeure de français.)

\exi{32.}
\gll Moj     brat     zna        polski     i   angielski.\\
      {mon (NOM)}   {frère (NOM)}   connait     {polonais (ACC)}   et   {anglais (ACC)}\\
\glt (Mon frère connaît le polonais et l’anglais.)

\exi{14.}
\gll Siostra     Piotra     lubi   Paryz.\\
      {soeur (NOM)}   {Piotr (GEN)}   aime   {Paris (ACC)}\\
\glt (La soeur de Piotr aime Paris.)

\exi{38.}
\gll Pies     ciągnie     pana.\\
     chien  tire     {monsieur (ACC)}\\
\glt (Le chien tire le monsieur.)
\end{exe}


\section{Tâche d’indication d’itinéraire}

\begin{figure}[H]
\includegraphics[width=10cm]{figures/Watorek-img019.jpg}
\caption{Plan}
\end{figure}

\ea Consigne:\\
\textit{En se servant d’un plan, décrire à quelqu’un comment se rendre dans un lieu indiqué.}
\glt `Comment il faut aller au musée de Stanislaw Wyspianski?'
\z

\end{paperappendix}
\end{otherlanguage}
\end{document}
