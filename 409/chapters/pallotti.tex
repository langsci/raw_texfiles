\documentclass[output=paper]{langscibook}
\ChapterDOI{10.5281/zenodo.10280602}

\author{Gabriele Pallotti\orcid{}\affiliation{University of Modena and Reggio Emilia}}

\title{Appropriate complexity}

\abstract{One of the most widespread assumptions in SLA research is that complexity grows over time, so that increasing proficiency in a second language would imply that one’s productions become more and more complex. However, this assumption needs to be qualified and further investigated. Firstly, complexity does not grow at the same rate in different linguistic sub-domains (e.g. lexicon, syntax, morphology). Secondly, linguistic complexity varies across tasks and modalities, so that it is not always the case that “the more, the better” -- there are adequate levels of complexity, and sometimes more can actually mean worse, at least for certain linguistic structures and sub-domains. This chapter presents results from a longitudinal four-year project involving adolescent learners, together with native speaker controls, performing a number of oral communicative tasks. The analysis looks at the complexity of telephone calls, demonstrating that the degree of syntactic complexity depends on the task’s interactional requirements. More specifically, higher levels of syntactic complexity compete with the need to rapidly exchange turns and to direct the interlocutor’s attention. As learners progress in the L2, their syntactic complexity in this task tends to decrease, while they develop more sophisticated pragmatic and interactional skills. Similar results are found using the conversation-analytic notion of Turn Constructional Unit (TCU) to compare syntactic complexity in telephone calls and in a narrative retelling task. The conclusion is that linguistic complexity must be interpreted, not just measured. Nowadays several tools are available to calculate dozens of complexity measures, and there is a risk of accumulating results with little reflection about how they should be understood in terms of language development and communicative adequacy. This has pedagogical implications, too, for language teachers must see complexity not as an objective in itself, but in a wider context of linguistic proficiency and functional appropriateness.
\keywords{language complexity, syntactic complexity, interaction, L2 pragmatics, functional adequacy, telephone calls}
}

\IfFileExists{../localcommands.tex}{
  \addbibresource{../localbibliography.bib}
  % add all extra packages you need to load to this file

\usepackage{tabularx,multicol}
\usepackage{url}
\urlstyle{same}

\usepackage{listings}
\lstset{basicstyle=\ttfamily,tabsize=2,breaklines=true}

\usepackage{langsci-basic}
\usepackage{langsci-optional}
\usepackage{langsci-lgr}
\usepackage{langsci-osl}
% \usepackage{./langsci/styles/langsci-lgr}
% \usepackage{./langsci/styles/langsci-osl}
% \usepackage{langsci-gb4e}

\usepackage{tikz}
\usetikzlibrary{patterns,calc}
\pgfdeclarepatternformonly{south east lines}{\pgfqpoint{-0pt}{-0pt}}{\pgfqpoint{3pt}{3pt}}{\pgfqpoint{3pt}{3pt}}{
    \pgfsetlinewidth{0.6pt}
    \pgfpathmoveto{\pgfqpoint{0pt}{3pt}}
    \pgfpathlineto{\pgfqpoint{3pt}{0pt}}
    \pgfpathmoveto{\pgfqpoint{.2pt}{-.2pt}}
    \pgfpathlineto{\pgfqpoint{-.2pt}{.2pt}}
    \pgfpathmoveto{\pgfqpoint{3.2pt}{2.8pt}}
    \pgfpathlineto{\pgfqpoint{2.8pt}{3.2pt}}
    \pgfusepath{stroke}}
    
\usepackage{stmaryrd}
\usepackage{wasysym}
\usepackage{multirow}
\usepackage{caption}
\usepackage{subcaption}
\usepackage{mathrsfs}
\usepackage{qtree}

\usepackage{linguex}


  %pminos do not split footnotes
% \interfootnotelinepenalty=10000 %Footnote in Laporte chapters has to be split SN


%\DeclareIndexNameFormat{default}{%
%\nameparts{#1}%
%\usebibmacro{index:name}%
%{\index[names]}%
%{\namepartfamily}%
%{\namepartgiveni}%
% {}% L1
% {}% L2
%{\namepartprefix}% generates spurious space L3
%{\namepartsuffix}% generates spurious space L4
%}

%  {\DeclareIndexNameFormat{default}{%
%     \usebibmacro{index:name}{\index[names]}{#1}{#3}{#5}{#7}}}

%\DeclareIndexNameFormat{default}{%
%  \usebibmacro{index:name}{\sindex[nom]}{#1}{#3}{#5}{#7}}

%\DeclareIndexNameFormat{default}{%
%  \usebibmacro{index:name}{\sindex[person]}{#1}{#3}{#5}{#7}}
%\DeclareIndexNameFormat{default}{%
%\nameparts{#1} \usebibmacro{index:name}{\sindex[person]]}{\namepartfamily}{‌​\namepartgiven}{\nam‌​epartprefix}{\namepa‌​rtsuffix}}

%\newcommand{\smiley}{:)}

%\renewbibmacro*{index:name}[5]{%
%\usebibmacro{index:entry}{#1}%
%{\iffieldundef{usera}{}{\thefield{usera}\actualoperator}\mkbibindexname{#2}{#3}{#4}{#5}}}

% \newcommand{\noop}[1]{}

%remove for final
%\overfullrule=1mm

\newcommand{\tobi}[2]}}
\renewcommand{\S}[1]{\tobi{#1}{\textsc{*}}}

% this volume references
% puts: [this volume]
% already defined: \citetv
%\newcommand{\citepv}[1]{(\citeauthor{#1} \citeyear*{#1} [this volume])}
\newcommand{\citealtv}[1]{\citeauthor{#1} \citeyear*{#1} [this volume]}

%parentheses around example number
\newcommand{\pref}[1]{(\ref{#1})}

% in-text examples

\newcommand{\lnex}[1]{\textit{#1}} %target lang word
\newcommand{\lnlit}[1]{(lit.: `#1')} %literal reading
\newcommand{\lnlat}[1]{(#1)} % latinization
\newcommand{\lntrans}[1]{`#1'} %translation
\newcommand{\lnexl}[2]%
{\lnex{#1}{} \lnlat{#2}} % ex with latinization
\newcommand{\lnexlat}[3]{\lnex{#1}{} \lnlat{#2}{} \lntrans{#3}} % ex with latinization and tranl.

%ch01
\newcommand{\co}[1]{\mbox{\textbf{#1}}}

%ch09

\newcommand{\cyrbulg}[1]{\begin{otherlanguage*}{bulgarian}#1\end{otherlanguage*}}


%ch10
\newcommand{\nlp}{{\small NLP}}
\newcommand{\mwe}{{\small MWE}}
\newcommand{\rae}{{\small RAE}}
\newcommand{\lvc}{{\small LVC}}
\newcommand{\pos}{{\small P}o{\small S}}
%\newcommand{\todo}[1]{ \textcolor{red}{#1} }

%\renewcommand{\labelenumi}{\theenumi}
%\ainamefmt{{vv}{ll}{, ff}{, jj}} % fullname

\newcommand{\biberror}[1]{{\color{red}#1}}

\newcommand{\osenovaitem}{--~} 
  %% hyphenation points for line breaks
%% Normally, automatic hyphenation in LaTeX is very good
%% If a word is mis-hyphenated, add it to this file
%%
%% add information to TeX file before \begin{document} with:
%% %% hyphenation points for line breaks
%% Normally, automatic hyphenation in LaTeX is very good
%% If a word is mis-hyphenated, add it to this file
%%
%% add information to TeX file before \begin{document} with:
%% %% hyphenation points for line breaks
%% Normally, automatic hyphenation in LaTeX is very good
%% If a word is mis-hyphenated, add it to this file
%%
%% add information to TeX file before \begin{document} with:
%% \include{localhyphenation}
\hyphenation{
    Beck-man
    Ngu-yen
    back-chan-nel
    back-chan-nels
    mo-not-o-nous
    ste-reo-typ-i-cal
}

\hyphenation{
    Beck-man
    Ngu-yen
    back-chan-nel
    back-chan-nels
    mo-not-o-nous
    ste-reo-typ-i-cal
}

\hyphenation{
    Beck-man
    Ngu-yen
    back-chan-nel
    back-chan-nels
    mo-not-o-nous
    ste-reo-typ-i-cal
}
 
  \togglepaper[1]%%chapternumber
}{}

\begin{document}
\AffiliationsWithoutIndexing{}
\maketitle 

\section{Introduction}\label{sec:pallotti:1}

Second language acquisition (SLA) is clearly a multidimensional phenomenon. The introduction of the triad Complexity, Accuracy, Fluency (CAF) a few decades ago (\citealt{Skehan1998, Wolfe-QuinteroEtAl1998}) displayed precisely this awareness, by identifying three main, conceptually independent dimensions to describe language development. This was certainly an improvement from a naive conception seeing the acquisition of a new language as merely a matter of accuracy, that is, equating learning with making fewer mistakes. However, it has also been noted that the three dimensions, important as they are, do not exhaust all the aspects inherent in language learning. In particular, they do not take into account the communicative adequacy of linguistic productions: these might be very complex, accurate and fluent, although completely ineffective in achieving a communicative purpose; conversely, a message could be perfectly functional even if it is minimally complex, accurate and fluent \citep{Pallotti2009}. Communicative language proficiency, i.e. the ability to communicate adequately and functionally, underpins all modern language teaching approaches and is one of the key constructs of current language tests, yet it has been neglected by most SLA research so far. 

In this chapter, I will specifically discuss the relationship between linguistic complexity and communicative adequacy. Nowadays, measuring complexity is not a problem (at least for English and a few other languages), with several online tools available for calculating a gamut of scores and indices. The real problem is the way in which these scores are interpreted. Once we learn that mean length of words or clauses, vocabulary range, number of dependents per head, etc. have higher values in text A than in text B, what do we do with it? Can we take this to mean that text A is more advanced, sophisticated, “better” than text B, that it indexes a higher level of linguistic and communicative competence? Are there cases in which more complexity is not necessarily better, which raise the question of what is the appropriate, optimal level of complexity? 

This chapter will try to address some of these questions, firstly from a conceptual point of view and then by presenting a study challenging the assumption that “the more complexity, the better”, on a relatively under-explored domain, that is, telephone calls in an additional language.

\section{Background}\label{sec:pallotti:2}

As mentioned, for several decades now, much SLA research has included linguistic complexity among the key variables to be measured, relating it in particular to general interlanguage development and to various characteristics of communicative tasks. There are now several syntheses of these studies, which have also highlighted some critical issues and unresolved problems. One of the recurring points concerns the theoretical and operational definition of the construct. Several authors have noted that the term “complexity” is polysemic, indicating both structural intricacy and cognitive difficulty (\citealt{Pallotti2009, Pallotti2015, BultéHousen2012}). In this chapter we will restrict use of the term to the first, formal-structural meaning, having to do with the number of elements in a linguistic structure and the intricacy of their relationships. Secondly, construct operationalization has been called into question, too (\citealt{NorrisOrtega2009, BultéHousen2012, Pallotti2015}). For example, many studies under-represented the language complexity construct by identifying it with a single measure, e.g. the subordination index, while in other cases redundant measures were used, such as the subordination index and the number of clauses per higher unit (for a review of these issues, see \citealt{NorrisOrtega2009, BultéHousen2012}). Furthermore, some measures reported to be about complexity are in fact related to difficulty, such as word rarity or the order of acquisition of grammatical rules.

As regards the results accumulated by this line of research, many studies report that lexical, morphological and syntactic complexity grow over time and with increasing linguistic proficiency, so that more advanced coincides in many cases with more complex (e.g. \citealt{CrossleyEtAl2011, BultéHousen2018, Vercellotti2018, BarrotAgdeppa2021}). Other studies also noticed that higher levels of complexity tend to be associated with higher scores in holistic ratings (e.g. \citealt{YangEtAl2015, KyleCrossley2018, LahuertaMartínez2018, BiJiang2020}). Thus it would seem that, in general, the more complexity, the better, or, in other words, that a more complex text indexes a more advanced (that is, typical of later stages of acquisition) and sophisticated (that is, typical of more skilled language users) linguistic system.

However, this general assumption has at times been called into question. \citet[648]{BiberEtAl2016} for instance note that “there are numerous grammatical devices associated with complexity, and so texts can be complex in very different ways in addition to being complex to differing extents”. To substantiate this point, in a series of corpus-based studies, \citet{BiberEtAl2011, BiberEtAl2016, BiberEtAl2020} showed that the subordination ratio, one of the most widely used indicators of linguistic complexity in SLA research, is not appropriate for assessing progress in academic writing in English. In this language, more competent writers (native users or very advanced learners) tend to produce texts that do not contain many subordinate clauses, but prefer another type of complexity, at the level of phrasal embedding. Biber and colleagues conclude that the assessment of complexity must always be related to a particular register and textual genre and that developmental trajectories may vary, e.g. between the production of oral conversations and the writing of academic texts. 

A similar hypothesis was also formulated by \citet{Ortega2003, Ortega2012} and \citet{NorrisOrtega2009}, who, based on a number of studies mostly dealing with English, hypothesized three stages in the development of syntactic complexity. Initially, it increases at the level of coordination, and is then followed by an increase in subordinate structures; the last level of complexification concerns phrases, which become longer and semantically more intricate, for instance with the use of nominalizations. \citet{KyleCrossley2018} later demonstrated that fine-grained phrasal complexity indices are better predictors of overall writing quality than general measures of syntactic complexity as well as fine-grained measures at the clausal level. Even the “subordination” construct should not be treated in a unified manner. For example, \citet{LambertNakamura2019} noted that, in a picture description task, novice learners of English tend to use more nominal clauses, such as \textit{I think that}..., while advanced learners employ a wider range and number of adverbial and relative clauses. As regards first language (L1) development at different ages, \citet{NippoldEtAl2005} found that subordination rate remains constant between ages 11 and 29, although in this age span nominal clauses tend to decrease while relative clauses increase. 

These and other studies show that syntactic complexity is in fact a multidimensional construct: not only is it conceptually incorrect to describe it with a single measure such as the subordination ratio, but its development over time must be described in more nuanced ways than a generic trend such as “complexity increases”, because different types of complexity and different genres and registers must be differentiated. Moreover, the few studies we have reviewed concern English. Even narrowing the field down to academic written communication, research on contrastive rhetoric shows that levels of syntactic complexity that are deemed optimal vary across languages and cultures. For instance, while academic English favors a style with little subordination and relatively simple and short multi-clausal units, academic prose in other languages, such as Italian, Spanish or German, may have different orientations (\citealt{Connor2002,Connor2018, Ortega2012}). 

Another area where the principle “the more the better” does not always apply is textual cohesion. At school, students are often taught, both in their mother tongue and in additional languages, to use a large number and variety of textual connectives, because this is supposed to characterize high-level academic prose. However, even this is not always true. For instance, \citet{CrossleyEtAl2011} found that younger L1 English writers, and those receiving lower proficiency ratings, tended to employ more cohesive devices, while more advanced writers produced more complex phrases. These findings on L1 development echo the following ironic remarks on second language (L2) teaching:  

\begin{quote}
Sometimes (though not always!) in language teaching, what you teach gets learned, and what you do not teach does not get learned. In the late 1970s, when much attention was given to teaching cohesion, one sometimes found the strangest examples of student writing as a result. An essay might contain very many (unnaturally many, in fact) sophisticated “link expressions”, such as \textit{however, moreover, nevertheless, on the other hand}. But between these elegant link expressions might be the most awful English, full of errors. This was the result of teaching the joining-together skills and not much else. Beautiful joints, holding nothing together.\hbox{}\hfill\hbox{(\citealt[294]{Johnson2018})}
\end{quote}

The same may apply for vocabulary, too. \citegen{DurrantBrenchley2019} results challenge the widely held assumption that, as children get older, they use an increasingly sophisticated vocabulary with rarer words. In their large-scale study of the writing of 6--16 year-olds in the UK, they noted that the proportion of rare words in texts did not increase with age. A closer analysis shows that this is due to the fact that younger L1 writers use more low-frequency nouns (e.g., \textit{caldron, fairy, hideout, wisp}), but tend to repeat them more often; older children use a greater variety of low-frequency adjectives, verbs and adverbs. This shows, once again, that global constructs such as “lexical complexity” are too coarse and that developmental tendencies occur at the level of sub-constructs such as different word classes (nouns vs. verbs and adjectives) or types of complexity (diversity vs. word frequency). In a follow-up study based on the same cross-sectional corpus of written essays, \citet{DurrantDurrant2022} explicitly raise the issue of the “appropriateness” of lexical choices, introducing a further level of distinction, that between different school subjects (English, Science, Humanities) and genres (academic, fiction, news). The evolutionary path does not simply consist of an increase in rare or “academic” words, but in a gradual convergence towards lexical choices appropriate to each writing genre and subject area, implying greater lexical diversification when required by the context and topic. This is also the conclusion reached by \citet[222]{Leńko-Szymańska2021} in her study comparing objective-analytic forms of vocabulary assessments, such as word diversity and frequency, with holistic ratings by human judges: “Sophisticated words used in a text should also be appropriate for the register selected by the writer and the content of an essay. If the whole text is written in conversational style, sophisticated words may not fit. As suggested by the judges, advanced words are also inappropriate to express banal ideas; then they can sound `forced and contrived'.” 

All these studies call into question the simplistic assumption that “the more complexity, the better”, showing that in some cases the optimal complexity level does not coincide with the highest. From a practical point of view, however, how can one determine whether and to what extent a certain level of complexity is “appropriate”, or more appropriate than another? 

First of all, an ample body of research shows that linguistic complexity systematically varies across communicative tasks and genres, for both first- and additional-language users (e.g. \citealt{FosterTavakoli2009, Ellis2011, Michel2011, MichelEtAl2019, Pallotti2019, LarssonKaatari2020}). Thus, one should alway speak of complexity that is appropriate for a specific task. Even restricting the scope to one specific task, the question remains how to establish what levels of complexity are appropriate for that task, and when one can talk of under- or over-complexification. A first solution would be asking expert judges to rate the quality of linguistic productions and then see what complexity levels are associated to the highest ratings. Several studies have adopted this approach, providing raters with a series of descriptor scales to assess various dimensions of text quality and then correlate objectively calculated complexity levels with these subjective ratings (e.g. \citealt{LahuertaMartínez2018, BiJiang2020}). A second possibility, which may be considered as a shortcut with respect to the first, is to look at language users who are \textit{a priori} assumed to produce high-quality texts. Traditionally, these are identified with native speakers, although this category has been repeatedly called into question (for a review, see \citealt{DewaeleEtAl2021}). \citet{Pallotti2019} proposes the more neutral term “top language performers”, although this again implies the first way of establishing appropriateness -- a top performer is one achieving high quality scores, regardless of their biographical profile. While this approach remains the most valid from a conceptual point of view, it requires the empirical identification, for each task and communicative situation, of a reference sample of participants scoring at the highest level. A more practical alternative rests on the fact that several studies have shown that native speakers, that is, people who started using a language from birth and kept using it regularly for all their life, invariably perform at top levels, at least in what \citet{Hulstijn2015} calls “core language proficiency”, involving linguistic structures common to all registers and varieties (\citealt{GranenaLong2013, Dąbrowska2019}). When it comes to more formal and academic language uses, native speakers exhibit more variation, so that the assumption that they all perform at top levels become more questionable (\citealt{Andringa2014, Hulstijn2015,Hulstijn2019, Dąbrowska2019}).

In this chapter, native language users will be taken as a reference for establishing appropriate complexity levels. In fact, the tasks we will be looking at all involve mundane language where native speakers are expected to form a rather homogeneous category. Furthermore, both native and non-native speakers belong to a larger population of high school students, so that their educational levels can be assumed to be similar. The distinction between native and non-native speakers is convenient and it follows a long tradition. However, this does not preclude that in future studies a more effortful but valid classification may be produced, based on text quality assessment without regard to biographical characteristics.

\section{The study}\label{sec:pallotti:3}

Previous research has mainly focused on how different levels of complexity may be appropriate for academic communicative contexts, such as essay writing or oral proficiency interviews. In this chapter the scope will be broadened by discussing syntactic complexity in a rather unexplored genre, telephone conversations, a highly interactive, unplanned everyday practice. The aim is to show that, in this domain, too, higher levels of syntactic complexity are not necessarily associated to higher skills or more effective performance.

Data come from the VIP corpus (\textit{Variabilità dell’Interlingua Parlata} [Variability of Spoken Interlanguage]; \citealt{PallottiEtAl2011}). Informants aged 15--20 at the beginning of data collection took part in the study, which tracked their linguistic development over a period of four years, with recording sessions held every year. 14 were intermediate-advanced L2 learners of Italian with a variety of L1s, while 10 were native Italian speakers, who were tested twice at about 2 years’ distance. Native speakers were slightly younger in order to match class level of the learners, most of whom started high school late or had to repeat one or more school years. Written consent was collected from all participants and, in the case of informants under 18 years of age, also from their families. All names were replaced by pseudonyms.

\begin{table}
\begin{tabular}{lllcc} 
\lsptoprule
& {Country of origin} & {L1} & {Years in Italy at T1} & {Age at T1}\\\midrule
Pandita & India & Punjabi & 4 & 17\\
Catherine & Ghana & Twi & 6 & 19\\
Eden & Eritrea & Tigrinya & 6 & 19\\
Shirley & Nigeria & English & 6 & 15\\
Elisa & Italy & Italian & -- & 15\\
Valentina & Italy & Italian & -- & 15\\
\lspbottomrule
\end{tabular}
\caption{Participants in the study}
\label{tab:pallotti:1}
\end{table}

Data transcription and coding is still under way, so that only a subset of data is currently available for analysis. The present investigation will consider four language learners and two native speakers (\tabref{tab:pallotti:1}). It is thus a small-scale study, whose main goal is not to make inferential generalizations but to suggest new avenues for expanding the debate on complexity and its appropriateness.

Participants performed a variety of oral communicative activities, so that their linguistic skills could be assessed in a range of contexts. The procedure consisted in two sessions on two different days. The first session involved a series of essentially monologic tasks and began with a semi-structured interview, followed by retelling a silent film and a picture story, then by a map task with the adult interviewer. The second session proposed more interactive tasks, with participants working in pairs. There was another map task, this time with the peer, and two information-seeking activities, one requiring the participants to plan a school trip, the other to select a present for a friend. Both of these tasks required making a number of phone calls to shops, travel agencies, restaurants and hotels, and to a list of “experts” (both youths and older adults), who were asked to provide advice and information. All these phone calls were real, in the sense that real businesses and homes were contacted, so that there was no staging and advanced planning of the interaction. However, all participants, including phone calls receivers, signed a consent form declaring that they were willing to be recorded in the coming weeks. Apart from the initial ice-breaking conversation, all the other tasks were presented in a counter-balanced order in different sessions. To minimize task repetition effects, the tasks performed every year had similar characteristics to ensure comparability, but varied in content. For example, every year a new video clip for the story retelling was presented, but their duration and narrative complexity were held as constant as possible. Likewise, maps in the map task varied every year, as well as the destinations for the school trip or the type of present to select.\largerpage

This study will focus on two tasks: the film retelling and telephone call openings. Retelling a silent movie is an elicitation procedure widely employed in SLA research, as it allows researchers to observe relatively long stretches of monologic speech, with a good balance between spontaneity and standardization. Telephone calls, on the other hand, have hardly ever been the object of systematic investigation in L2 studies (see however \citealt{NuzzoGauci2012, Thörle2016, FantLundell2019}), although the literature on them in L1 speakers is extensive, especially as regards their opening phases (\citealt{Schegloff1986, LukePavlidou2002}). These two tasks represent two poles of a continuum of interactionality. The film retelling is a highly monologic task, where the interlocutor normally just provides a few backchannels, while telephone call openings are characterized by a rapid exchange of very short turns, with both participants contributing to the establishment of a “ritual state of ratified mutual participation” \citep[100]{Goffman1963} before they move on to the main business of the call.

Transcription followed a modified version of the Chat-CA system (see Appendix A), with English glosses trying to capture the original Italian text, including non-standard forms. Transcribed data were prepared for quantitative analysis by first dividing them into clauses and AS-Units. The latter can be defined as “a single speaker’s utterance consisting of an independent clause or sub-clausal unit, together with any subordinate clause(s) associated with either” (\citealt[365]{FosterEtAl2000}).  A sub-clausal unit, in turn, is a “phrase which can be elaborated to a full clause by means of recovery of ellipted elements from the context [three, to the office]... or a minor utterance or “nonsentence” \citep{QuirkEtAl1985}”, like \textit{Oh poor woman, Thank you very much, Yes} (\citealt[366]{FosterEtAl2000}). A slightly modified version of AS-Unit, more suitable for Italian, was adopted, as discussed by \citet{Ferrari2020}.\largerpage

The following excerpts (from \citealt{FosterEtAl2000}) exemplify the units:

\begin{quote}
\transcript{{\textbar} when I was in the university :: er I have specialized in this er subject {\textbar}} (1 AS-Unit, 2 clauses; :: indicates clause boundary)

\transcript{{\textbar} three months for this one {\textbar}} (1 AS-Unit, 1 sub-clausal Unit) 
\end{quote}

This segmentation was carried out by students and research assistants on about 60\% of the data, and then checked by the principal investigator; after some initial training and discussions, inter-rater agreement was always over 85\%. The remainder of the data were coded by the principal investigator only. 

Telephone call openings were also segmented into Turn-Constructional Units (TCU, \citealt{SacksEtAl1974}), that is, “the smallest interactionally relevant complete linguistic units in their given context” \citep[512]{Selting2000}. The term is commonly used in conversation analysis to indicate units of interaction, which may coincide with syntactic units, such as phrases or clauses, although this is not always the case. Their fundamental characteristic is that they constitute “blocks” of language which may constitute a complete turn, and whose end may thus allow for the beginning of a new turn by another speaker. In other words, “each TCU is a coherent and self-contained utterance, recognizable in context as “possibly complete”. Each TCU’s completion establishes a Transition-Relevance Place (TRP) where a change of speakership becomes a salient possibility that may or may not be realized” \citep[151]{Clayman2012}.

Syntactic complexity was assessed using the following measures:

\begin{description}
\item[Words per AS-Unit (words/AS-U).] The length in words of each AS-Unit (including clauses, multi-clausal and sub-clausal units).

\item[Words per clause (words/clause).]  The length in words of each complete clause, that is, a syntactic structure minimally consisting of a verb and all its necessary arguments. This category includes clauses with zero subjects, which are grammatical in Italian, but not sub-clausal units such as \textit{in the evening} or \textit{really?} 

\item[Dependent clauses per AS-Unit.] All clauses depending on a matrix clause, divided by the total number of AS-Units. The criteria for establishing what is a dependent clause are those exposed in \citet{Ferrari2020}, which only partially match \citegen{FosterEtAl2000} approach.

\item[Dependent clauses per clause.] All clauses depending on a matrix clause, divided by the total number of clauses. This is the standard subordination ratio.
\end{description}

\begin{sloppypar}
The first two measures tap length of syntactic unit, while the last two concern the degree of clausal embedding. For both sets, the denominator is either the clause or the AS-unit. While this introduces some redundancy (clause-based and AS-unit-based measures are clearly correlated), there are some reasons for retaining both operationalizations. Clause-based measures, in fact, refer to a purely linguistic, syntactic unit such as the clause, while the AS-unit is an interactional unit, relating more to the notion of conversational move and including a number of syntactically incomplete, though pragmatically fully appropriate, sub-claus\-al units. The two types of units also allow one to compare results with previous research on written and oral productions, and are thus relevant to answer questions about appropriate levels of complexity in different modalities.
\end{sloppypar}

\section{Results}\label{sec:pallotti:4}\largerpage[2]

The next pages will present the results of the study, with a quantitative analysis and a qualitative discussion of some excerpts (a preliminary and complementary description of individual and group developmental trajectories may be found in \citealt{Ferrari2012}). The quantitative analysis will be based on the averages of the four learners relative to the four years of the study (T1--T4), in order to observe any developmental trends, while the native speakers’ column reports the average scores  over the two data collections for the two Italian participants. We will first discuss syntax, with classical measures of complexity such as unit length and various subordination indices. This will be followed by an analysis of interactional dynamics based on the notion of TCU. 

\subsection{\label{sec:pallotti:4.1} Syntactic analysis}

\begin{table}
\small
\begin{tabularx}{\textwidth}{lYYYYY} 
\lsptoprule
& {T1} & {T2} & {T3} & {T4} & {NS}\\
\midrule
words\slash AS-U &  5.73 

(4.56--7.28) &  5.77 

(4.89--6.41) &  6.22 

(5.56--6.83) &  5.66 

(5.04--6.10) &  6.97 

(4.86--8.36)\\
words\slash clause &  4.55 

(4.31--4.91) &  4.97 

(3.96--5.63) &  5.61 

(5.00--6.83) &  4.90 

(4.80--5.08) &  5.68 

(4.58--6.73)\\
dep cl\slash AS-U &  0.37 

(0.22--0.53) &  0.31 

(0.23--0.39) &  0.26 

(0.22--0.33) &  0.30 

(0.20--0.38) &  0.42 

(0.21--0.69)\\
dep cl\slash clause &  0.29 

(0.21--0.36) &  0.27 

(0.20--0.33) &  0.23 

(0.19--0.26) &  0.25 

(0.19--0.30) &  0.33 

(0.25--0.43)\\
\lspbottomrule
\end{tabularx}
\caption{Film retelling -- syntactic complexity (group average and min-max range)}
\label{tab:pallotti:2}
\end{table}

\begin{sloppypar}
In the narrative task, native speakers’ productions were syntactically more complex than the learners’ were. \tabref{tab:pallotti:2} shows that both their AS-units (6.97 words, vis-à-vis values between 5.66 and 6.22 for the learners) and their clauses (5.68 words vs 4.55--5.61) were longer. They also produced more dependent clauses per AS-Unit (0.42) and per total number of clauses (subordination ratio = 0.33). The learners did not seem to follow a linear developmental pattern, as their scores for both length of unit and subordination fluctuated over the four years, without displaying an increasing or decreasing overall trend, although they always remained below native speakers’ values. These were intermediate-advanced learners and previous research has shown that subordination and length of unit tend to increase mostly in the initial stages of L2 acquisition and then stabilize in a plateau \citep{VerspoorEtAl2012}, which might also be the case for this group of participants.
\end{sloppypar}\largerpage[2]


Things are rather different regarding telephone call openings (\tabref{tab:pallotti:3}). Here, too, native speakers tend to produce longer AS-units, with an average length of 4.97 words, higher than any of the learners’ values over the four years. Their average clause length (6.48) is slightly higher, too, although it lies within the learners’ range of values over the four years (between 5.65 and 6.71). The main difference concerns subordination. Native speakers produce very few dependent clauses per AS-Unit (0.08) and per clause (0.11), while these are more frequent in learners’ speech, with values between 0.09 and 0.19 and 0.13 and 0.21, respectively. This propensity to subordinate is particularly striking in the first year, when learners produce almost twice as many dependent clauses per AS-Unit and per clause than native speakers. This subordination ratio drops considerably in the second year and remains constant for the three subsequent years, which somehow resonates with \citeauthor{Ortega2003}’s (\citeyear{Ortega2003, Ortega2012}) remarks on written language development. 

\begin{table}[H]
\small
\begin{tabularx}{\textwidth}{>{\raggedright\arraybackslash}p{.12\textwidth}YYYYY} 
\lsptoprule
& {T1} & {T2} & {T3} & {T4} & {NS}\\

\midrule
words\slash AS\nobreakdash-U &  4.01

(3.42--4.91) &  4.05 

(3.40--5.16) &  3.99 

(3.14--5.25) &  3.99 

(2.84--5.41) &  4.97 

(4.31--6.11)\\
words\slash clause &  5.65 

(4.54--6.68) &  6.71 

(4.25--10.33) &  6.14 

(5.07--7.87) &  5.78 

(5.53--6.23) &  6.48 

(5.74--7.10)\\
{dep cl}/ AS-U &  0.19 

(0.13--0.33) &  0.09 

(0.00--0.15) &  0.11 

(0.04--0.20) &  0.10 

(0.03--0.18) &  0.08 

(0.00--0.22)\\
{dep cl}/ clause &  0.21 

(0.13--0.35) &  0.13 

(0.00--0.26) &  0.13 

(0.05--0.26) &  0.14 

(0.02--0.25) &  0.11 

(0.00--0.22)\\
\lspbottomrule
\end{tabularx}
\caption{Telephone calls -- syntactic complexity  (group average and min-max range)}
\label{tab:pallotti:3}
\end{table}

These quantitative trends may be better understood by looking at a few examples. The first two excerpts (Tables~\ref{tabex:pallotti:1} and \ref{tabex:pallotti:2}) show how participants with Italian as an L1 typically opened their phone calls. They uttered rather short turns, consisting of independent clauses (\textit{I wanted to ask some information; do you sell cell phones?}) or sub-clausal units (\textit{the sagem MYX 52?; to london;  a class of young people}). This differs from the learners’ openings, which, especially in the first period, contained many long and complex syntactic structures, such as \textit{I’d like to know how much simsang SGHA 800 costs} (\tabref{tabex:pallotti:3}) or \textit{we’re a group of four friends who would like to go on a trip to london} (\tabref{tabex:pallotti:4}) or \textit{I need the information to go to barcelona} (\tabref{tabex:pallotti:5}).


After a few years, the learners de-complexified their openings, which became more similar to native speakers’. For instance, they often prefaced their requests with a pre-request, such as \textit{can I ask you some information} (\tabref{tabex:pallotti:6}) or \textit{excuse me I've got to ask me some information} (\tabref{tabex:pallotti:7}).

In this last example, \tabref{tabex:pallotti:8}, Eden’s opening at Time 3 contains many short clauses and sub-clausal units, and the most complex syntactic structure comes at the end in the form of a very standard pre-request formula (\textit{I’d like to know if}) followed by a simple clause (\textit{you have rear window}, i.e. the famous film by Alfred Hitchcock). \sectref{sec:pallotti:4.3} will be specifically devoted to discussing the dynamics of these interactional moves which gradually lead participants to the main concern of the phone call. 

\begin{table}
\caption{Ex. 1: Elisa, NS}
\label{tabex:pallotti:1}

\begin{tabularx}{\textwidth}{QQ}

\lsptoprule
\transcript{SH4:  telecom buongiorno ?\#

ELI:  .hh e:: buongiorno.    volevo  chiederle un'informazione,

 \#0\_2

ELI:  e:  vendete cellulari?

SH4:  sì?

ELI:  .h e:: sì.

ELI:  volevo sapere alcune informazioni su due due cellulari.

\#0\_2

SH4:  se li abbiamo volentieri

ELI:  e:: okey.   il sagem emme ipsilon ics cinquantadue?} & \transcript{SH4:  telecom good morning ?\#

ELI:  hh e:: good morning.  I wanted  to ask some information,

\#0\_2

ELI:  e: do you sell cell phones?

SH4:  yes?

ELI:  h e:: yes.

ELI:  I wanted to know some information about two two cell phones.

\#0\_2

SH4:  certainly, if we have them

ELI:  e:: okay.   the sagem MYX 52?}\\
\lspbottomrule
\end{tabularx}
\end{table}

\begin{table}
\caption{Ex. 2: Valentina, NS}
\label{tabex:pallotti:2}

\begin{tabularx}{\textwidth}{QQ}

\lsptoprule
\transcript{AG2:  moito viaggi \#

VAL:  buonasera

AG2:  sì

VAL:  volevo  chiederle informazione   \#

AG2:  sì

VAL:  e: per londra   \#

AG2:  sì

VAL:  una classe di: ragazzi

AG2:  mh mh

VAL:  e: qualcosa di conveniente che:: \#  che c'è} & \transcript{AG2:  moito viaggi \#

VAL:  good evening

AG2:  yes

VAL:  I’d like  to ask you information   \#

AG2:  yes

VAL:  er: to london   \#

AG2:  yes

VAL:  a class of young people

AG2:  mh mh

VAL:  er: something inexpensive that:: \#  that’s available}\\
\lspbottomrule
\end{tabularx}
\end{table}

\begin{table}
\caption{Ex. 3: Pandita, NNS, T1}
\label{tabex:pallotti:3}


\begin{tabularx}{\textwidth}{QQ}

\lsptoprule
\transcript{SH4:  pronto .

\#0\_5

PAN:  pronto

PAN:  buonasera: .  \#

SH4:  buonasera: .

PAN:  vorrei  sapere  quanto costa simsang esse gi acca a ottocento .} & \transcript{SH4:  hallo .

\#0\_5

PAN:  hallo

PAN:  good evening  \#

SH4:  good evening.

PAN:  I’d like  to know  how much simsang SGHA 800 costs .}\\
\lspbottomrule
\end{tabularx}
\end{table}

\begin{table}
\caption{Ex. 4: Shirley, NNS, T1 }
\label{tabex:pallotti:4}


\begin{tabularx}{\textwidth}{QQ}

\lsptoprule
\transcript{AG3:  moito viaggi?

SHI:  .hhh \#0\_8 buonasera

\#0\_5

AG3:  buonasera

SHI:  allora noi siamo un \# gruppo di quattro \# amici  che \# vorremmo  far un viaggio \# a londra

\#0\_4

AG3:     [sì

SHI:    [perciò volevamo  chiedere il costo dell'aerio e gli \# orari  \#0\_5} & \transcript{AG3:  moito viaggi?

SHI:  .hhh \#0\_8 good evening

\#0\_5

AG3:  good evening

SHI:  ok we’re a \# group of four \# friends  who \# would like  to go on a trip \# to london

\#0\_4

~

AG3:  [yes

SHI:    [so we wanted  to ask the price of the plane and the \# schedule  \#0\_5}\\
\lspbottomrule
\end{tabularx}
\end{table}


\begin{table}
\caption{Ex. 5: Eden. NNS, T1\protect\footnote{This example comes from a learner, Aisha, who was not included in the sub-corpus used for quantitative analyses in this chapter.}}
\label{tabex:pallotti:5}


\begin{tabularx}{\textwidth}{QQ}

\lsptoprule
\transcript{AG1:  ci bi esse buongiorno sono daniela.

ST4:  buongiorno. ho bisogno dell’informazione per andare a barcellona} & \transcript{AG1:  ci bi esse good morning daniela speaking.

ST4:    good morning. I need the information to go to barcelona}\\
\lspbottomrule
\end{tabularx}
\end{table}


\begin{table}
\small
\caption{Ex. 6: Pandita, NNS, T3}
\label{tabex:pallotti:6}


\begin{tabularx}{\textwidth}{QQ}

\lsptoprule
\transcript{SH4:  mediaworld buongiorno \#

PAN:  buongiorno \# scusa posso chiedere un'informazione

SH4:  sì \#

PAN:  io vorrei sapere  se magari lei sa \# un titolo di un ci di   \#   mana  \#

SH4:  vi passo il reparto un attimo eh

PAN:  okey} & \transcript{SH4:  mediaworld good morning  \#

PAN:  good morning   \#  escuse me can I ask you some information

SH4:  yes \#

PAN:  I'd like to know if by any chance you know \# the title of a CD  by  \#   mana  \#

SH4:  I'll put you through the deparment just a minute eh

PAN:  okey}\\
\lspbottomrule
\end{tabularx}
\end{table}

\begin{table}\small
\caption{Ex. 7: Catherine, NNS, T3 }
\label{tabex:pallotti:7}


\begin{tabularx}{\textwidth}{QQ}

\lsptoprule
\transcript{SH3:  ricordi mediastore buongiorno sono marco

CAT:  pronto  buongiorno

\#0\_5 .

CAT:  hhh

SH3:  prego

CAT:   eh mi scusi  mi devo chiedere un un'informazione

\#0\_2

SH3:  prego

CAT:  l'ultimo cidi di norah jones

SH3:  sì

CAT:  che si intitola not too [late

SH3:                               [too late \# sì sì ce l'abbiamo

CAT:  quanto costerà secondo lei

\#0\_2

~

SH3:  e:::: uf dovrebbe essere venti euro e novanta \#0\_8

[...]} & \transcript{SH3:  ricordi mediastore good morning marco speaking

CAT:  hallo good morning

\#0\_5 .

CAT:  hhh

SH3:  please

CAT:  eh excuse me I've got to ask me some information

\#0\_2

SH3:  please

CAT:  the last CD by norah

jones

SH3:  yes

CAT:  entitled  not too [late

~

SH3:                           [too late \# yes yes we've got it

CAT:  how much will it be in your opinion

\#0\_2

SH3:  e:::: uf it should be twenty euros ninety \#0\_8

[...]}\\
\lspbottomrule
\end{tabularx}
\end{table}

\begin{table}
\caption{Ex. 8: Eden, NNS, T3}
\label{tabex:pallotti:8}


\begin{tabularx}{\textwidth}{QQ}

\lsptoprule
\transcript{SH3:  libreria (memoli) buonasera sono tatiana

\#0\_2

EDE:  eh::: buonasera

\#0\_5

EDE:  allora \# voi vendete divudi::?

\#0\_5

SH3:  si:::.

EDE:  si::

\#

EDE:  okey

\#

EDE:  volevo sapere  se avete:: \# la finestra sul cortile

[...]} & \transcript{SH3:  (memoli) bookstore good evening  tatiana speaking

\#0\_2

EDE:  eh::: good evening

\#0\_5

EDE:  now \# do you sell DVDs?

~

\#0\_5

SH3:  yes::.

EDE:  yes::

\#

EDE:  okey

\#

EDE:  I'd like to know if you have::\# rear window

[...]}\\
\lspbottomrule
\end{tabularx}
\end{table}

\subsection{\label{sec:pallotti:4.2} Learning syntactic variation}

\citet{Ferrari2012} looked at the development of complexity, accuracy and fluency in the same four participants observed in this study. Her analysis included two more tasks, the interview and picture-story retelling, and also discussed the individual learners’ trajectories over time. In her conclusions, she states that “Italian students exhibit a high degree of variation across tasks with respect to measures of syntactic complexity and fluency. On average, this variation is less pronounced in L2 learners and there seems to be little change for the group as a whole over the three years of the study” \citep[291]{Ferrari2012}\footnote{By ``three years'' Ferrari means the years following the first one, thus totaling four years as in this contribution.}. However, she did not provide an explicit quantitative account of these claims, which will be the object of this section.

\tabref{tab:pallotti:4} shows the differences between scores for film retellings and telephone calls, in the four longitudinal data collection points for the learners and in the two data collections for the native speakers. The column labeled “NNS 4yrs” reports the learners’ average score over the four years. Phone calls are syntactically less complex (negative values) than the narratives for all measures except words per clause. This may be explained by the fact that this score refers to complete clauses only, thus disregarding sub-clausal units, which are instead included in the AS-Unit count. Although phone calls contained many short sub-clausal units, which is reflected in their lower AS-Unit length, the few complete clauses they had tended to be slightly longer than those in the film retelling, of about one word (1.06 for learners and 0.80 for native speakers). This measure is outstanding in another regard, as it is the only one where the difference between film retelling and phone calls is larger for learners than for native speakers. In all other cases, the spread of values across the two tasks for native speakers is larger than that for the L2 users. The spread between the two tasks did not vary much for the learners over the four observation times, as suggested by \citet{Ferrari2012}, although her more fine-grained analysis shows that this group pattern may be further divided into two sub-patterns, at least as regards the subordination ratio. Pandita and Catherine, who were less advanced at the beginning of the study, increased the difference in subordination between monologic and dialogic tasks over time, while Eden and Shirley went in the opposite direction. Ferrari hypothesizes, with the necessary caveats due to the very small number of observations, that, after reaching a certain proficiency level, syntactic complexity choices become a matter of individual style and preferences, less constrained by task features and demands.\footnote{As many other authors, Ferrari does not give any exact indication as to this proficiency threshold where the change would take place. This remains a challenge for research in this area, which would greatly benefit from studies collecting fine-grained measures of both learners’ proficiency and linguistic features like complexity. Normally, proficiency is operationalized in terms of CEFR levels or scores in standardized tests, which provide only a coarse grained characterization.}

The degree of subordination is indeed a prime example of what \citet[120]{Pallotti2015} calls “stylistic complexity”, which “is always, at least to some extent, a matter of speaker’s or writer’s choice”. In another study based on the VIP corpus, looking at the 10 native speakers, \citet{Pallotti2019} found in fact that, in five different communicative tasks (film retelling, phone calls, interview, map task, discussion among peers) the inter-individual coefficient of variation was always larger for syntactic complexity measures, such as length of AS-Unit and subordination, than for other complexity dimensions such as lexical and morphological complexity. It thus seems that syntax – and subordination more specifically – is the linguistic domain leaving more room for individual stylistic variation, although, even here, different tasks tend to activate on average different levels of complexity.\footnote{Although \citegen{Pallotti2019} corpus was larger than in the present study, including the two native speakers considered here together with eight more participants of the same group, he looked at a smaller number of measures, so that the two studies are not entirely comparable. However, the two measures that can be compared, that is words/AS-Unit and dependent clauses/AS-Unit, confirm our results. Compared to the learners in this study, the ten participants in \citet{Pallotti2019} produced longer AS-Units with more dependent clauses in the film retelling, while in telephone calls  their AS-Units were shorter and with fewer dependent clauses than the learners’ at T1. It is important to note that these are group tendencies rather than categorical distinctions -- the values for some individual learners fall within the range of native speakers, which is also due to the high inter-individual variation in syntactic behaviors.}

\begin{table}
%\small
\begin{tabularx}{\textwidth}{lYYYYYY} 
\lsptoprule
& {T1} & {T2} & {T3} & {T4} & {NNS 4yrs} & {NS}\\\midrule
words/AS-U & {}$-$1.72 & {}$-$1.73 & {}$-$2.23 & {}$-$1.67 & {}$-$1.84 & {}$-$2.00\\
words/clause & 1.11 & 1.74 & 0.53 & 0.88 & 1.06 & 0.80\\
dep cl /AS-U & {}$-$0.18 & {}$-$0.22 & {}$-$0.15 & {}$-$0.19 & {}$-$0.18 & {}$-$0.35\\
dep cl/clause & {}$-$0.08 & {}$-$0.14 & {}$-$0.10 & {}$-$0.12 & {}$-$0.11 & {}$-$0.22\\
\lspbottomrule
\end{tabularx}
\caption{Variation between film retelling and telephone calls}
\label{tab:pallotti:4}
\end{table}

\subsection{\label{sec:pallotti:4.3}Interactional analysis}

Most SLA research on complexity deals with various forms of linguistic complexity, such as lexical, morphological, syntactic. However, dialogic oral communication may also be analyzed in different terms, looking at how interaction unfolds in a series of “moves” whereby participants co-construct courses of (inter)action. This is the perspective taken by conversation analysis and interactional sociolinguistics, and may complement the linguistic perspective we have discussed so far (for applications of this approach to SLA, see  \citealt{PekarekDoehlerLauzon2015, PekarekDoehlerPochon-Berger2015, Véronique2017Converser, SkogmyrMarianBalaman2018} and the Special issue of \textit{Modern Language Journal}, 2022, S1). Analysis in this section rests on the notion of Turn-Constructional Unit (TCU) introduced before. However, this notion does not necessarily coincide with syntactic units like the clause or phrase; although it depends on them in most cases, as syntactic completeness is one of the major indicators that a turn can be concluded and that a transition to next speaker is potentially relevant \citep{Selting2000}. Another important construct is that of “pre-”. Many conversational moves, such as requests, refusals, or proposals, are not produced directly, but get introduced by some linguistic and non-linguistic signals that prepare the ground for the main action to come. For example, a request for information is rarely produced directly, as in \textit{Do you have rear window?}, but is normally prefaced by a series of pre-request moves, as done by Eden in \tabref{tabex:pallotti:8}, such as \textit{Do you sell DVDs?} or \textit{okey}. This allows the listener to gradually orient to the request and process it, as it were, bit by bit. In an interplay between syntax and interaction (\citealt{Couper-KuhlenSelting2017, PekarekDoehlerEskildsen2022}), pre-requests by native speakers tend to occur as separate TCUs, allowing the receiver to step in even for a very brief acknowledgment token, like the \textit{yes} and \textit{mhmh} produced by the shop assistant while Valentina, a native speaker, builds her pre-request and request step-by-step (\tabref{tabex:pallotti:2}). In these cases, both parties converge towards a “ritual state of ratified mutual participation” \citep[100]{Goffman1963} in a coordinated fashion, which implies the ability to gain and yield the floor several times in a rapid turn exchange. L2 learners, especially in the first year of observation, followed this strategy less frequently. For instance, Pandita at T1 (\tabref{tabex:pallotti:3}) uttered a single complex TCU containing both the pre-request (\textit{I’d like to know}) and the request proper (\textit{how much simsang SGHA 800 costs}). Likewise, Aisha at T1 (\tabref{tabex:pallotti:5}), produced a long TCU directly containing the request (\textit{I need the information to go to barcelona}). In these cases, the listener must wait until the whole pre-request+request sequence is produced before a Transition-Relevant Place is reached.

Quantitative analysis confirms these insights (\tabref{tab:pallotti:5}). Native speakers produced very few pre-requests embedded within a larger TCU (14.29\%), while this proportion was much larger in non-native speakers, especially at T1 (71.43\%; at this time there were also several instances, 21.43\%, that could not be readily classified as belonging to either category, given the difficulty of segmenting them into syntactic units), and then gradually decreased over time at T2--T4. This shows a trade-off between syntactic complexity and interactional fluency. Initially, learners preferred more complex syntactic structures, packing into one single unit both the pre- and the request, which accounts for their high subordination ratio at T1. This sort of syntactic embedding requires a certain degree of grammatical competence, which was already attained by these intermediate-advanced learners, but lowers interactional demands, as learners don’t run the risk of losing the floor and having to regain it immediately afterwards. This, apparently, seems to require higher interactional skills that these participants, at least at the beginning of the observation period, did not seem to possess, or doubted possessing. As their interactional competence grew, they were more at ease with taking and yielding turns, even very short ones, and thus de-complexified their productions from a syntactic point of view, while at the same time making them interactionally more fluent.

\begin{table}
\begin{tabular}{lrrr} 
\lsptoprule
& {pre in TCU} & {pre as TCU} & {?}\\
\midrule
NNS - T1 & 71.43 &  7.14 & 21.43\\
NNS - T2 & 55.56 & 44.44 &  0.00\\
NNS - T3 & 22.22 & 77.78 &  0.00\\
NNS - T4 & 33.33 & 66.67 &  0.00\\
NS       & 14.29 & 85.71 &  0.00\\
\lspbottomrule
\end{tabular}
\caption{Percentage of pre-’s realized as separate TCUs or within a TCU}
\label{tab:pallotti:5}
\end{table}

\section{Conclusions}\label{sec:pallotti:5}

This study, albeit limited in scope and with no claim to statistical generalizability, contributes to the debate on appropriate complexity. In line with other contributions, the conclusion is that, for complexity, more is not always better.\footnote{Noticing that native speakers’ turns and utterances were interactionally more fluent does not imply that learners’ contribution were completely inadequate. Indeed, they often got their message across, although perhaps in a more effortful and convoluted way.} Complexity and appropriateness are related constructs in many ways, but they are not isomorphic and it is important to keep them conceptually apart and then describe how they interact. This needs to be borne in mind from a research and a pedagogical perspective, both of which have always been central in Daniel Véronique’s work.

From a research perspective, Véronique’s work has coherently followed a functionalist orientation (see e.g. \citealt{Véronique1995acquisition,Véronique2013dislocation, Véronique2017c}). This means starting from observing the communicative purposes of messages and then studying the forms through which they are realized. Such a functionalist orientation is lacking in much CAF research, and it is only in recent times that the relationships between the performance dimensions of complexity, accuracy and fluency, and indicators of communicative effectiveness – such as functional adequacy (\citealt{Pallotti2009, deJongEtAl2012, KuikenVedder2022}, Special issue of \textit{Task}, 2022) or quality judgments of linguistic productions (\citealt{KyleCrossley2018, LahuertaMartínez2018, BiJiang2020}) – have become the object of more systematic investigations. It also means seeing linguistic competence in the context of a larger discourse competence, which is a key theme in the French approach to Second Language Acquisition (\citealt{LenartLeclercq2021}) -- the analysis needs to go beyond the level of the individual clause or sentence, to investigate how these and other linguistic structures play a role in monologic and dialogic texts.

Daniel Véronique has also constantly been concerned with how SLA research may interact with language teaching (see e.g. \citealt{Véronique2005Interrelations, Véronique2017Vers, LaurensVéronique2017}), which leads us to discuss some pedagogical implications of the present study. The main conclusion seems to be that, rather than “teaching complexity”, we should teach appropriateness – to task, register, situation – and students should not learn complexity as an objective in itself, but in terms of when, how and why it is appropriate to be more or less complex; always allowing for a certain degree of individual stylistic variation. 

Learning appropriate task-related variation thus becomes one of the main objectives of a communicative approach to language teaching, and our results show that it takes many years to achieve an appropriate range of variation across tasks. A similar observation was made by \citet{Wiklund2002}, in her study on native and nonnative Swedish-speaking adolescents. Each youth produced a written academic composition and was interviewed orally; both texts were analyzed with the same set of “verbal sophistication” measures regarding word length and word variety. The results show that native speakers had a higher verbal sophistication index than non-native speakers in written compositions, while it was lower in interviews. In other words, the difference between one text type and another was greater in native speakers, and the author considered the magnitude of this difference to be indicative of the “capacity to adjust to different repertoires” (p. 82).

Findings like these are in line with those of the present study concerning variation in syntactic complexity between film retelling and telephone calls, and make clear that the educational goal should become promoting the capacity to \textit{adjust complexity}, that is, to move at ease in a (socio)linguistic space made up of varieties, genres, registers, each with its appropriate complexity levels. As we have seen, sometimes linguistic complexity may be intertwined with other levels of communicative competence, for example, with interactional fluency \citep{Peltonen2020}, as the present study shows with regard to telephone call openings. To attain appropriate levels of interactional fluency, syntactic complexity might have to be reduced, not increased. This resonates with what \citet[612]{LambertKormos2014} state about learning how to write: “It is frequently the case that expert speakers and writers express complex ideas more simply than novices. This is not due to the availability of linguistic resources but rather to practiced mastery in efficient and effective message formation.”

The point is not teaching the “optimal level of complexity”, with a normative orientation prescribing values like “between 5 and 6 words per clause” or “a subordination ratio around 0.45”. The point is that teachers should become aware of how complexity levels vary across genres and registers. This means they should not urge students to be more complex for the sake of it or provide negative feedback if their productions are not complex enough, when in fact they may be perfectly adequate, possibly because of their simplicity. As regards students, at least after a certain age and proficiency level, rather than being taught how to be “ideally complex”, they should be enabled to reflect on how different genres, registers, communicative situations may imply different complexity levels. This will turn them into self-regulated, autonomous learners, who appreciate sociolinguistic variation, choose where to stand in the language space and are aware of the communicative consequences of their linguistic actions. In order to do so, they may analyze the complexity of their texts, or others they may want to take as possible models, using approaches and instruments that currently belong mostly to the research domain. For example, they may consult some of the complexity analysis tools available online, thus becoming student-researchers who consciously reflect on their learning experience and meaningfully use language to discuss about language. 

\section*{Acknowledgements}
Daniel Véronique was one of the first SLA researchers I met in the early stages of my career. He was a regular attendee of Eurosla conferences and of the meetings of the international network \textit{The structure of learner varieties}. He was and still is one of the leading representatives of the functional-typological approach to SLA, which was one of the first to be proposed and is still among the most valuable avenues to studying language acquisition. Like many others, I have always been struck by his gentle manners, his positive attitude, and his ability to see what is good and promising in each one of us, even when I was a very young researcher. This is why I am very happy to dedicate this chapter to him and to thank him for his contribution to a better SLA community.

\section*{Transcription conventions}

A modified version of the CHAT-CA system was used for transcribing data (\url{ca.talkbank.org}). CAPITAL LETTERS are used to indicate high volume only, and not for proper names, sentence beginnings etc. as in standard orthography.

\begin{tabbing}
AG1, AG2, SH4, etc\hspace{2ex}\= travel agencies’ or shops’ codes\kill
SHI, PAN, VAL, etc \> participants’ names\\
AG1, AG2, SH4, etc \> travel agencies’ or shops’ codes\\
\#1\_2 , \#0\_5, etc \> pause length, in these case 1.2 and 0.5 seconds\\
\# \> micropauses, shorter than 0.2 seconds\\
\textit{.hhh} \> audible inbreath\\
\textit{hhh} \>  audible outbreath\\
\textit{wo:::rd} \> word lengthening\\
{[} \> overlapped speech\\
. \> falling intonation\\
? \> raising intonation\\
, \> suspended intonation\\
\end{tabbing}

\printbibliography[heading=subbibliography,notkeyword=this]
\end{document} 
