\documentclass[output=paper]{langscibook}
\ChapterDOI{10.5281/zenodo.10280594}
\author{Sibylle Kriegel\affiliation{Laboratoire Parole et Langage, CNRS, Aix-Marseille Université}}
\title{Les allocutions en créole d’Auguste Le Duc, Galéga en 1835}

\abstract{Cette contribution est consacrée à une présentation et à une première étude de textes créoles du 19\textsuperscript{e} siècle qui n’ont jamais fait l’objet d’une analyse linguistique. Ces textes, rédigés entre 1827 et 1835, sont extraits du journal d’Auguste Le Duc, planteur dans l’Ile de Galéga (aujourd’hui Agaléga) située à proximité de l’archipel des Seychelles. Il s’agit notamment de deux allocutions prononcées dans le créole local par l’auteur du journal, Auguste Le Duc. Elles commentent le \textit{Bill} d’affranchissement de 1835 qui abolit l’esclavage à l’Ile Maurice et dans ses dépendances. La présentation de l’édition est suivie d’une analyse des éléments du journal d’Auguste Le Duc qui révèlent des informations sur la biographie linguistique de l’auteur et de la situation sociolinguistique dans laquelle les allocutions ont été prononcées. La reproduction des textes et de leur traduction française (par leur auteur) est suivie d’une première analyse linguistique se focalisant sur le syntagme nominal. En conclusion, l’hypothèse d’un rapprochement au créole seychellois en vue de la situation géographique de Galéga ne s’avère pas être pertinente. En revanche, on constate que les tendances évolutives observées se retrouvent dans des textes en créole mauricien datant de la même époque.

\textbf{Mots-clés :} créole mauricien, créole seychellois, Galéga, textes anciens}

\IfFileExists{../localcommands.tex}{
  \addbibresource{../localbibliography.bib}
  \usepackage{langsci-optional}
\usepackage{langsci-gb4e}
\usepackage{langsci-lgr}

\usepackage{listings}
\lstset{basicstyle=\ttfamily,tabsize=2,breaklines=true}

%added by author
% \usepackage{tipa}
\usepackage{multirow}
\graphicspath{{figures/}}
\usepackage{langsci-branding}

  
\newcommand{\sent}{\enumsentence}
\newcommand{\sents}{\eenumsentence}
\let\citeasnoun\citet

\renewcommand{\lsCoverTitleFont}[1]{\sffamily\addfontfeatures{Scale=MatchUppercase}\fontsize{44pt}{16mm}\selectfont #1}
   
  %% hyphenation points for line breaks
%% Normally, automatic hyphenation in LaTeX is very good
%% If a word is mis-hyphenated, add it to this file
%%
%% add information to TeX file before \begin{document} with:
%% %% hyphenation points for line breaks
%% Normally, automatic hyphenation in LaTeX is very good
%% If a word is mis-hyphenated, add it to this file
%%
%% add information to TeX file before \begin{document} with:
%% %% hyphenation points for line breaks
%% Normally, automatic hyphenation in LaTeX is very good
%% If a word is mis-hyphenated, add it to this file
%%
%% add information to TeX file before \begin{document} with:
%% \include{localhyphenation}
\hyphenation{
affri-ca-te
affri-ca-tes
an-no-tated
com-ple-ments
com-po-si-tio-na-li-ty
non-com-po-si-tio-na-li-ty
Gon-zá-lez
out-side
Ri-chárd
se-man-tics
STREU-SLE
Tie-de-mann
}
\hyphenation{
affri-ca-te
affri-ca-tes
an-no-tated
com-ple-ments
com-po-si-tio-na-li-ty
non-com-po-si-tio-na-li-ty
Gon-zá-lez
out-side
Ri-chárd
se-man-tics
STREU-SLE
Tie-de-mann
}
\hyphenation{
affri-ca-te
affri-ca-tes
an-no-tated
com-ple-ments
com-po-si-tio-na-li-ty
non-com-po-si-tio-na-li-ty
Gon-zá-lez
out-side
Ri-chárd
se-man-tics
STREU-SLE
Tie-de-mann
} 
  \togglepaper[1]%chapternumber
}{}

\begin{document}
\begin{otherlanguage}{french}
\AffiliationsWithoutIndexing{}
\lsFrenchChapterSettings{}
\maketitle 

\section{Introduction}
Les relations historiques qui se sont nouées entre les îles des Mascareignes et des Seychelles et leurs conséquences linguistiques ont fait l’objet de controverses depuis les années 1970. Georges Daniel Véronique les résume dans plusieurs articles (\citealt{Véronique2014Origine, FonSingVeronique2020, }). Après une présentation des différentes versions de la thèse du bourbonnais comme proto-langue des créoles de l’océan Indien \citep{Chaudenson1974} ainsi que de la thèse adverse, dite de \textit{Isle de France} (\citealt{BakerCorne1982}), Véronique invite à repenser la question généalogique et typologique des créoles de l’océan Indien. Même si la présente contribution ne s’inscrit pas directement dans ce débat, elle fournit de nouveaux éléments linguistiques permettant «~de reprendre des comparaisons grammaticales entre les langues créoles de l’océan Indien~», démarche dont Véronique rappelle l’urgence en 2021 \citep{Véronique2021Pour} : Cette contribution est consacrée à une première présentation de deux textes en créole datant de 1835 dans lesquels le planteur Auguste Le Duc s’adresse à ses esclaves devenus apprentis suite à la promulgation du \textit{Bill} d’affranchissement\footnote{Le \textit{Bill} d’affranchissement ou la Proclamation d’Abolition de l’esclavage est le premier document officiel publié à l’Ile Maurice en créole le 17 janvier 1835. Il «~fut rédigé en trois versions, anglais et français d’une part (versions présentées «~en parallèle~»), créole d’autre part (au-dessous de la version française)~» (\citealt{Chaudenson1981} : 115).} qui annonce l’abolition de l’esclavage à l’Ile Maurice. Cependant, le théâtre des allocutions d’Auguste Le Duc n’est pas l’Ile Maurice même, mais une de ses dépendances, le petit archipel de Galéga (aujourd’hui Agaléga), situé à 1064 km au Nord de Maurice et à 657 km au Sud des Seychelles. La première allocution suit la lecture du \textit{Bill} d’affranchissement et en explique les conséquences locales. Dans une deuxième allocution, Auguste Le Duc explique les conséquences locales de l’instauration des 45 heures de travail hebdomadaires obligatoires stipulée par le \textit{Bill}. A ce jour, ces textes ne sont pas mentionnés dans la littérature linguistique sur les textes créoles anciens de l’océan Indien (voir p.ex. \citealt{Chaudenson1981, FurlongRamharai2006, BakerEtAl2007, Davidson2021}). Cependant, Robert Chaudenson a publié un bref compte-rendu de l’ouvrage qui les contient \citep{Chaudenson1995}.\footnote{Robert Chaudenson a attiré, des années plus tard, mon attention sur l’existence de ces textes. En 2014, j’en ai fait une présentation orale à Aix-en-Provence lors du 14\textsuperscript{e} colloque du «~Comité International des Etudes Créoles~».}

Après une présentation de l’édition (\sectref{sec:kriegel:2.1}), l’article situe l’auteur et son journal dans leur contexte (\sectref{sec:kriegel:2.2}) pour ensuite fournir des détails sur la population à Galéga et sur leur langue {\sectref{sec:kriegel:2.3}}. Dans \sectref{sec:kriegel:3}, les deux allocutions sont reproduites avec leur traduction littérale en français, en suivant la version qu’en donne l’ouvrage de \citet{Pourcelet1994} (voir ci-dessous). La 4\textsuperscript{e} partie est consacrée à une première analyse de la graphie et de la phonétique (\sectref{sec:kriegel:4.1}) ainsi qu’à une étude du syntagme nominal (marque du pluriel, déterminants, pronoms) (\sectref{sec:kriegel:4.2}). En conclusion (\sectref{sec:kriegel:5}), je montre que les textes reflètent des tendances évolutives déjà observées dans d’autres textes mauriciens du milieu du 19\textsuperscript{e} siècle et que leur rapprochement au créole seychellois ne semble pas être pertinent.

\section{Eléments de contexte}\label{sec:kriegel:2}
\subsection{Présentation de l’édition}\label{sec:kriegel:2.1}


François Pourcelet, historien des colonisations, présente dans un ouvrage publié en 1994 l’édition d’un manuscrit de 365 pages. Le texte du manuscrit contient des documents recueillis et mis en forme par Saint Elme Le Duc, frère d’Auguste Le Duc. L’ouvrage de Pourcelet propose le récit d’Auguste concernant ses années passées à Galéga entre 1827–1839, puis celui de ses années passées aux Iles Poivre, Desroches et Saint-Joseph entre 1842–1851.\footnote{Ces îles font partie de l’archipel des Seychelles.} Pourcelet y a inclus les additions de Saint Elme Le Duc qui sont typographiquement distinguées du récit d’Auguste (pour les détails concernant l’édition, voir \citealt{Chaudenson1995}). Le manuscrit relate, sous forme de journal, les douze années de séjour d’Auguste Le Duc, né à Paris en 1790, dans l’Ile de Galéga où il était planteur. Dans le commentaire de Saint Elme Le Duc on lit :

\begin{quote}
Galéga, ce grain de sable jeté au milieu de l’océan Indien, pouvant n’être pas connue de personnes même très instruites en géographie, nous leur disons qu’elle est située à environ 200 lieues au nord de l’Ile de France, à mi-chemin de cette colonie à l’équateur.\\\hbox{}\hfill\hbox{(\citealt{Pourcelet1994} : 19, commentaire de Saint Elme le Duc, non daté)}
\end{quote}


L’archipel de Galéga ou aujourd’hui Agaléga est formé de deux îles reliées par un isthme sablonneux. Bien que beaucoup plus proche de l’archipel des Seychelles, il a toujours appartenu à l’Ile Maurice. Depuis 2018, la construction d’une piste d’atterrissage de 3 km inquiète ses rares habitants bien que le gouvernement mauricien continue à démentir des informations selon lesquelles les îles seraient destinées à servir de base arrière de l’armée indienne.\footnote{France Info, 19 septembre 2022, \url{https://la1ere.francetvinfo.fr/reunion/maurice-agalega-ne-servira-pas-de-base-militaire-a-l-inde-selon-le-gouvernement-1322880.html}} 

\subsection{L’auteur et son journal : Auguste Le Duc}\label{sec:kriegel:2.2}

Auguste Le Duc quitte Paris à l’âge de 24 ans pour arriver à l’Ile Maurice en 1814. Il s’y occupe du transport et de la commercialisation des produits de Galéga au service de la famille Barbé qui a obtenu, en 1808, la concession de cette île déserte peuplée de tortues et couverte de cocotiers. La famille Barbé y crée un «~établissement~» mais les administrateurs successifs échouent dans leur entreprise. En 1827, la famille Barbé confie la direction de l’exploitation à Auguste Le Duc qui y produit de l’huile de coco pendant une douzaine d’années entre 1827 et 1839 – avec grand succès, non seulement à en croire son propre récit mais aussi de nombreux témoignages de ses contemporains :

\begin{quote}
 C’était un homme doué d’une grande intelligence, de beaucoup de tact, de savoir même et encore plus de savoir-faire. Il réunissait toutes les qualités voulues pour être le roi d’un petit état, comme Ulysse avait, jadis, toutes celles requises pour être roi d’Itaque.

(\citealt{Pourcelet1994} : 12, témoignage de Louis Boutou à la Société Royale des Arts et Sciences de Maurice, 1870)
\end{quote}

Cependant, après un bref séjour en France, Auguste Le Duc ne peut pas retourner à Galéga, partiellement vendue. Il entre en procès avec la famille Barbé et finit par acheter les îles Poivre, Desroches et Saint-Joseph en 1842. Il essaie de les exploiter pendant quelques années, contrairement à Galéga sans grand succès, et meurt à Mahé aux Seychelles en 1851, aigri et appauvri.

Le manuscrit édité par Pourcelet comprend avant tout le journal qu’a tenu Auguste Le Duc pendant les douze années prospères passées à Galéga mais également le journal de ses dernières années dominées par les embrouilles juridiques avec la famille Barbé. La période de 1827–1839 à Galéga est au cœur de la présente contribution. Elle comprend l’année 1835, année de l’abolition de l’esclavage par les Britanniques à l’Ile Maurice et aux Seychelles.

Le récit commence avec un mythe fondateur qui repose probablement sur des faits partiellement réels. Auguste Le Duc dit résumer des témoignages directs dont il aurait «~composé une narration~». Cette narration raconte le sort malheureux d’une belle fille créole de Maurice, Adélaïde, enlevée par un vieux corsaire amoureux d’elle. L’enlèvement finit mal et ils échouent sur Galéga où on retrouve leurs squelettes en 1808. Auguste Le Duc finit le récit en affirmant avoir trouvé en 1833 une bouteille contenant un message du corsaire amoureux meurtri de remords. Par la suite, le journal d’Auguste Le Duc prend des allures moins romanesques et fournit des détails sur la vie quotidienne dans les îles de l’océan Indien et à Galéga en particulier. \citet{Chaudenson1995} présente un résumé des spécificités lexicales du récit pour constater que de nombreux termes désignant des \textit{realia} se retrouvent dans le dictionnaire du créole seychellois \citet{D’OffayLionnet1982} alors qu’ils sont absents du dictionnaire du créole mauricien de \citet{BakerHookoomsing1987}, fait qui ne surprend pas si on tient compte de la situation géographique de Galéga.\footnote{Une vérification avec le \textit{Dictionnaire étymologique des créoles de l’Océan indien} s’impose mais dépasserait le cadre de cette contribution.} 

\subsection{La population de Galéga et sa langue}\label{sec:kriegel:2.3}

Dans son introduction, Pourcelet résume les informations sur le nombre et la composition de la population servile sur l’île de Galéga : 

\begin{quote}
Il [Auguste Le Duc] se trouve seul Blanc pour diriger une plantation de cocotiers principalement et une fabrique d’huile de coco, avec 175 esclaves dont le nombre sera porté à 230 avant l’affranchissement, tous Malgaches, Mozambiques, Créoles (nom donné par Auguste le Duc pour toute personne de couleur d’origine imprécise).\footnote{Pourcelet se trompe sur la signification «~créole~». Comme dans les autres textes de l’Ile Maurice de cette époque, «~créole~» désigne toute personne, peu importe sa couleur, née à Maurice (ou peut-être Galéga). Une lecture nuancée du journal d’Auguste Le Duc conforte cette interprétation. D’ailleurs, \citet{Chaudenson1995} attire l’attention sur un certain nombre d’inexactitudes dans l’édition du manuscrit.}\\
\hbox{}\hfill\hbox{(\citealt{Pourcelet1994} : 11, commentaire de l’éditeur)}
\end{quote}

Un rapport sur les établissements de Galéga de 1838 fait par M. Anderson, un juge spécial, contient une information cruciale pour notre étude linguistique :

\begin{quote}
Le désir d’acheter la liberté entière n’existe pas ici, et \textit{la plus grande portion de la population étant né sur les lieux} et n’ayant jamais vu d’autres endroits, je ne crois pas qu’un seul individu ait le désir de le quitter en 1841.
\smallskip\\(\citealt{Pourcelet1994} : 137, journal d’A. Le Duc qui reproduit le rapport du juge Anderson)
\end{quote}

Auguste Le Duc~parle à plusieurs reprises du taux élevé de naissances locales à Galéga. Dans une lettre à son frère en 1838, il se prononce également sur la provenance des esclaves qui ne sont pas nés sur l’île :

\begin{quote}
Depuis 12 ans, je n’ai vécu qu’avec des Malgaches, des Mozambiques et des Créoles de l’une ou de l’autre caste. Les Malgaches venus directement ont perdu de leurs habitudes et de leur caractère distinctif par leur longue cohabitation avec les esclaves venus de Maurice (...)\\\hbox{}\hfill\hbox{(\citealt{Pourcelet1994} : 114, lettre d’A. Le Duc à son frère du 4 novembre 1838)}
\end{quote}

Le texte ne fournit pas d’autres détails sur les esclaves directement venus de Madagascar. Ceci aurait été intéressant dans le contexte de la discussion lancée par \citet{Larson2009} qui insiste sur la pratique prolongée de la culture et de la langue malgaches au sein des populations serviles dans l’océan Indien pendant le 19\textsuperscript{e} siècle. A en croire la brève remarque d’Auguste Le Duc, les Malgaches auraient donc vite «~perdu leurs habitudes et leur caractère distinctif~», ce qui pour le cas spécifique de Galéga ne confirme pas l’hypothèse promulguée par \citet{Larson2009} pour l’Ile Maurice. Freycinet, un contemporain d’Auguste Le Duc parle au sujet de la langue créole pratiquée à Maurice de «~ce curieux et singulier idiome, dont il existe plusieurs variétés~». Il observe la situation suivante pour l’Ile Maurice : «~On distingue donc le créole mozambique de celui des noirs indiens, malais ou malgaches, et plus encore du créole usité, par goût ou par habitude, parmi les mulâtres et les personnes riches de l’île.~» (\citealt{Freycinet1827}, cité d’après \citealt{Chaudenson1981} : 99; voir aussi \citealt{BakerEtAl2007} : 9) \citet[161]{Chaudenson1981} émet des doutes sur l’existence de variétés différentes du créole mauricien au 19\textsuperscript{e} siècle tout en admettant l’existence d’interférences avec les langues maternelles des locuteurs. Quoiqu’il en soit, on peut conclure que la situation sociolinguistique à l’Ile Maurice a dû être relativement complexe. Cependant, à Galéga, il est fort probable que le créole local ait connu peu de variation en raison du taux de natalité élevé dans cette île et de l’isolement de la population sur Galéga. 

Le journal d’Auguste Le Duc contient plusieurs passages dans le créole local de Galéga. Les travaux de recherche sur la diachronie du créole mauricien ne font aucune allusion à l’existence de ces textes. Selon ses dires, Auguste Le Duc, né à Paris, les aurait prononcés lui-même dans des allocutions orales et les a transcrites ensuite dans son journal. Quelques remarques métalinguistiques d’Auguste Le Duc nous renseignent sur son rapport au parler local, qu’il appelle «~patois créole~». Au début de son journal dans une \textit{captatio benevolentiae}~où il s’excuse de ne pas être un «~homme de plume~» mais «~un homme d’action~» :

\begin{quote}
Le patois créole, qui, depuis vingt-cinq ans, est à peu près le seul jargon que j’entende parler, a dû me faire oublier, si je l’ai su jamais, l’allure gracieuse de ma langue maternelle. \\\hbox{}\hfill
\hbox{(\citealt{Pourcelet1994} : 23, début du journal d’A. Le Duc, non daté)}
\end{quote}

Et, en 1828, peu après son arrivée de l’île Maurice : 

\begin{quote}
Ma position est véritablement affligeante, je n’ai pas près de moi un être à qui je puisse confier mes idées et qui puisse me donner quelques conseils ; au contraire, je suis souvent et même toujours obligé de barbariser mon langage, afin de me faire comprendre.\\\hbox{}\hfill\hbox{(\citealt{Pourcelet1994} : 59, journal d’Auguste Le Duc, 25 janvier 1928)}
\end{quote}

Les auteurs de presque tous les textes créoles anciens étaient d’abord des francophones d’horizons divers (voir \citealt{Chaudenson1981} pour la Réunion et Maurice; \citealt{Bollée2007} : 1 pour la Réunion; \citealt{Hazaël-Massieux2008} : 17 pour la Caraïbe). Les textes présentés ici n’échappent pas à cette règle. Le lecteur d’aujourd’hui doit donc garder en tête qu’Auguste Le Duc a écrit ces textes à travers le filtre du français. Même s’il a vécu au contact intense avec des créolophones monolingues pendant de longues années, il n’en reste pas moins qu’il est né et a été élevé à Paris. Les passages en créole du journal d’Auguste Le Duc concernent :

\begin{itemize}
\item des répliques d’esclaves dans la première partie fictive du journal qui relate l’enlèvement de la jeune fille Adélaïde mais aussi dans le récit des faits quotidiens qui se sont produits dans l’ile ;
\item une allocution en créole faite par Auguste Le Duc après la lecture du \textit{Bill} d’affranchissement de l’esclavage en 1835, suivi de sa propre traduction «~littérale~» en français (reproduite dans \sectref{sec:kriegel:3.1}) ;
\item une deuxième allocution d’Auguste Le Duc au sujet des 45 heures de travail hebdomadaires dues par les apprentis, suivie elle aussi d’une traduction française par Auguste Le Duc (reproduite dans \sectref{sec:kriegel:3.2}).
\end{itemize}

\section{Les deux allocutions en créole et leur traduction littérale française par Auguste Le Duc}\label{sec:kriegel:3}

Dans le journal d’Auguste Le Duc, on trouve donc la transcription de deux allocutions qu’il dit avoir faites à ses anciens esclaves devenus apprentis à Galéga. La première, du 5 avril 1835 donnée suite à la lecture du \textit{Bill} d’affranchissement explique les conséquences de l’abolition de l’esclavage. Après le texte en créole, on trouve sa «~traduction littérale~»\footnote{Comme le mentionne déjà \citet{Chaudenson1995} un des problèmes de l’édition consiste dans le fait que le lecteur ne sait pas toujours clairement attribuer les commentaires à l’éditeur ou à l’auteur du journal lui-même. Ici, on peut penser que c’est Auguste Le Duc lui-même qui parle de traduction littérale.} française (voir \citealt{Pourcelet1994} : 92-96). La deuxième, datée du 16 août 1835, est consacrée au sujet des 45 heures de travail hebdomadaires et à la ration de vivres pour les apprentis. Dans cette contribution, il m’a semblé intéressant de présenter la version créole et la version française des deux textes, écrits par le même auteur, juxtaposées dans deux colonnes. Effectivement, le texte français, «~traduction littérale du discours en créole~» (\citealt{Pourcelet1994} : 94) reste très proche du texte créole et il paraît très probable qu’Auguste Le Duc n’ait effectivement pas traduit du français au créole mais du créole au français, en restant délibérément proche de la version créole. Ainsi, la traduction française contient des constructions qui imitent étroitement le créole, p.ex. l’évitement de la pronominalisation du complément d’objet indirect par un pronom personnel clitique en le remplaçant par la forme non clitique (p.ex. «~faire acquérir \textit{à vous autres} de l’esprit, pour montrer \textit{à vous autres} ce que vous avez à faire~»). Cette imitation du créole dans la traduction française n’empêche pas que le texte créole contienne de nombreux traits acrolectaux, proches du français. Ainsi, on constate un emploi très fréquent de la préposition \textit{à} disparue dans les créoles actuels et déjà très rare dans les autres textes créoles datant de la même époque que les textes analysés ici : le complément d’objet indirect se trouve souvent introduit par \textit{à} (p. ex. ‘\textit{vou donn’ bon l’ésempl’ \textbf{à} vou-pitits}’). On trouve également deux occurrences du verbe \textit{être} pour introduire un attribut du sujet (‘\textit{pour \textbf{êtr’} heureux’} ; ‘\textit{pour \textbf{êtr’} bon sizet’}) alors que la structure ‘\textit{pour}’\,+\,copule (sans SN ou pronom) a disparu ou que la copule a été supplée par un verbe plus iconique dans les créoles (p.ex. ‘\textit{vini}’ en créole mauricien).\footnote{Je tiens à remercier un relecteur anonyme pour cette précision.} 

\subsection{Allocution à l’occasion de l’abolition de l’esclavage (\citealt[92–96]{Pourcelet1994})}\label{sec:kriegel:3.1}

Journal. Le 5 avril 1835, allocution en \textit{créole} aux Noirs de l’établissement après la lecture du \textit{Bill} d’affranchissement.\footnote{Le journal d’Auguste Le Duc ne précise pas en quelle langue le \textit{Bill} d’affranchissement a été lu. Il est fort probable qu’il s’agit de sa version créole reproduite dans \citet[118]{Chaudenson1981}.} MM. Boyer, Delisse et d’Anglès\footnote{Il s’agit de trois spécialistes d’histoire naturelle.}, naturalistes, amis, présents :\footnote{Tous les éléments qui font l’objet d’une analyse dans la partie 4 sont mis en italique. La présente version respecte scrupuleusement les choix graphiques de \citet{Pourcelet1994}.}


% \TabPositions{6cm}
\begin{xltabular}{\textwidth}{@{}>{\hangindent=1em}Q>{\hangindent=1em}Q@{}}                                                                                                                                                                   
«~\textit{Mon} \textit{zenfant}, vou libr’ auzourdi. & «~Mes enfants, vous êtes libres aujourd’hui.\\
A vla \textit{ein}’ la loi & Voici une loi \\
\textit{que} le roi grand’terr’ fini envoyé & que le roi du continent vient d’envoyer\\
pour fair’ vou libr’. & pour vous rendre la liberté.\\
Li fini acété vou \textit{le corps}; & Il vient de racheter vos corps\\
mais comm’ li n’a pas été y en a assez la monnaie, & mais, comme il n’avait pas assez de monnaie,\\
li fini donn’ sel’ment la moitié, & il n’en a donné seulement que la moitié\\
et pour que \textit{vou-maîtr}’ n’a pas \textit{perdi} son l’argent, & et, pour que vos maîtres ne perdent pas leur argent,\\
\textit{li} laiss’ vou travail’ encor’ six \textit{bann’anné} & ils vous laissent travailler encore six années entières\\
pour li sans payé. & pour eux sans être payés.\\
«~Vou fini çanze nom; & «~Vous avez changé de nom ;\\
à s’t’hèr vou libr’ ; & à cette heure vous êtes des apprentis, vous n’êtes plus esclaves, vous êtes libres,\\
vou comm’à moi, comm’à tout \textit{blanc} & vous êtes comme moi, comme tous les Blancs,\\
la mêm’ soze ; la loi égal’ pour \textit{nous} tous. & la même chose ; la loi est égale pour nous tous.\\
Si moi fair’ vou tort, la loi va pini moi ; & Si je vous fais du tort, la loi va me punir ;\\
si vou fair’ moi tort, la loi va pini vou. & si vous me faites du tort, la loi vous punira.\\
«~\textit{Ca} six \textit{bann’anné} que \textit{vou} va zapprentis, & «~Ces six années entières que vous allez être apprentis, \\
ça pour donn’ \textit{zaut}’ ciprit, & c’est pour vous faire acquèrir à vous autres de l’esprit, \\
pour êtr’ bon sizet, pour bien travail’, & pour montrer à vous autres ce que vous avez à faire pour être de bons sujets, pour bien travailler,\\
pour n’a pas paress’, pour n’a pas volor. & pour n’être pas paresseux, pour n’être pas voleurs. \\
Ca la loi là n’a pas badiné ; & Cette loi-là ne plaisante pas; \\
n’a pas besoin croir’ & il ne faut pas croire\\
\textit{que} vou va fini demand’ grâce ; & que vous allez venir demander grâce;\\
li n’a pas pardonn’ vou, & elle ne vous pardonnera pas \\
comm’ moi été fair touzours. & comme j’ai toujours fait.\\
Si vou volor sel’ment ein’zépi maye, assez ; & Si vous volez seulement un épi de maïs, c’est assez ;\\
li renvoie vou dans \textit{ein’} l’autr’ péye ; & elle vous renvoie dans un autre pays ;\\
vou blizé laiss’ là \textit{vou-femm’}, vou-pitits, & vous oblige à laisser là votre femme, vos enfants, \\
tout ça qu’vou y en a. & tout ce que vous avez.\\
«~A s’t’hèr qu’vou libr’, & «~A cette heure que vous êtes libres,\\
vou capab’ allé côt’ vou voulé, & vous pourrez aller du côté que vous voudrez, \\
quand vou l’apprentissaze fini ; & quand votre apprentissage sera fini ;\\
personn n’a pas capab’ empèce vou. & personne ne pourra vous en empêcher. \\
mais dans quèqu’ pèye qu’vou allé, & Mais dans quelque pays que vous alliez, \\
i faut touzours \textit{manzé} pour vivr’, et pou ça & il faudra toujours manger pour vivre, et pour ça, \\
faut travaill’ et non pas volor ; & il faudra travailler et ne pas voler ; \\
comm’ moi été dir’ vou, & comme je vous l’ai dit,\\
ça qui volor li cerce \textit{son-la-misèr’}. & celui qui vole cherche à être malheureux.\\
«~Ca qui voulé resté travaill’ avé moi, & «~Ceux qui veulent rester à travailler avec moi,\\
moi va péyé li, à caus’ pourquoi moi pli content & je vais les payer parce que j’aime mieux\\
gardé quéqu’ein’ qui fini coutumé & garder quelqu’un avec qui je suis habitué\\
que fair’ vini Noirs trauzé. & que de faire venir des Noirs étrangers. \\
Moi va préfér’ zaut’ & Moi, je vous préférerai aux autres, \\
tant que vou bon sizets. & tant que vous serez bons sujets.\\
«~Aut’fois, quand vou été \textit{saclaves}, & «~Autrefois, quand vous étiez esclaves, \\
\textit{vou-maîtr}’ était blizé gardé ça qui bon, & votre maître était obligé de garder celui qui était bon,\\
ça qui \textit{mauvais} ; à s’t’hèr li va \textit{soziré}. & celui qui était mauvais ; à cette heure, il va choisir.\\
Ca qui n’a pas travaill’, & Celui qui ne veut pas travailler,\\
ça qui \textit{sicaner}, ça qui y en a trop la bousse, & qui est chicaneur, qui est trop sur la bouche,\\
ça qui fair tort cam’rad’, li va dir’ : & qui fait tort à son camarade, il va lui dire : \\
«~Mon garçon, va-t-en, & «~Mon garçon, va-t-en, moi, \\
moi n’a pli besoin toi~».~ & je n’ai plus besoin de toi.~» \\
«~Tout ça qu’moi dir’vou là, li dans vou-la-prièr’. & Tout ce que je vous dis là est dans votre prière,\\
Acouté bien : & écoutez bien :\\
«~Un seul Dieu tu adoreras & «~Un seul Dieu tu adoreras\\
et aimeras parfaitement. & et aimeras parfaitement.\\
Tes père et mère honoreras & Tes père et mère honoreras\\
afin de vivre longuement. & afin de vivre longuement.\\
Le bien d’autrui tu ne prendras & Le bien d’autrui tu ne prendras\\
ni retiendra à ton escient. & ni retiendra à ton escient.\\
«~Un seul Dieu tu adoreras : vous bien conné. & «~Un seul Dieu tu adoreras : vous savez bien ça.\\
«~Tes père et mère honoreras : & «~Tes père et mère honoreras :\\
 vou voir qui faut & vous voyez qu’il faut\\
vou donn’ bon l’ésempl’ à \textit{vou-pitits} ; & que vous donniez le bon exemple à vos enfants ;\\
\textit{vou-pitits} devoir touzours respect pour son papa, & vos enfants doivent toujours du respect à leur papa,\\
pour \textit{son} maman, pour vié di mound’ ; & à leur maman, aux personnes âgées ;\\
\textit{zenfants} qui manqu’ respect à vié di mound’, & les enfants qui manquent de respect aux vieillards,\\
Bou Dié va \textit{pini} li. & le Bon Dieu va les punir.\\
«~Le bien d’autrui tu ne prendras : & «~Le bien d’autrui tu ne prendras :\\
pour n’a pas volor, & pour ne pas voler,\\
accoutm’ zaut’à l’ouvraze, & il faut vous accoutumer à travailler,\\
donn’ la main pour fair’ manze animaux, & aider à faire manger les animaux,\\
manze coçons, travaill’ dans zardin. & manger les cochons, travailler dans le jardin.\\
Tou ça qu’ein di mound’ doit travaill’. & Tout le monde doit travailler.\\
Moi qui command’’zaut, & Moi qui commande à vous autres, \\
& moi-même je travaille, chacun son ouvrage ; moi, mon travail est de veiller sur vous autres,\\
pour maziné tout ça qui en a pour fair’ ; & d’imaginer tout ce qu’il y a à faire ;\\
vou travaill avé vou-la-main, & vous, vous travaillez avec la main,\\
moi, mon-la-têt’ qui travaill’ & moi, c’est ma tête qui travaille\\
pour nourri zaut’ pour soign’ zaut’. & pour nourrir vous autres, pour soigner vous autres.\\
Ca qui n’a pas voulé travaill’, & Celui qui ne veut pas travailler,\\
li voulé gagn’ touzours ça qu’son cam’rad’ li en a ; & qui veut gagner toujours ce qui est à son camarade,\\
pour ça li volor & celui-là est un voleur,\\
et li croir’ di mound’ n’a pas trouv’ li, & il croit que personne ne peut le trouver ;\\
& mais quand même personne ne pourrait le trouver,\\
Bon Dié, partout, Bon Dié trouv’ li ; & le Bon Dieu qui est partout, le Bon Dieu le trouve.\\
ici mêm côt’ moi parlé, & Ici même, à côté de moi qui vous parle,\\
li guett’ nou, comm’à li là ; & Il nous guette, Il est là ; \\
quand vou allé dans zardin cam’rad & quand vous allez dans le jardin d’un camarade \\
pour volor maye, & pour voler du maïs, \\
li trouv’ vou la même soze. & Il vous voit la même chose. \\
Quéqu’zour li va pini vou ; quand vou mort, & Quelque jour Il vous punira ; quand vous serez morts, \\
vou allé dans ein’ diabr’ di péye & vous irez dans un diable de pays\\
\textit{qui blancs} appell’ l’enfer. & que les Blancs nomment l’enfer.\\
Là touzours travaill’, & Là, on travaille toujours, \\
n’a pas y en a dimance, n’a pas y en a berloque, & il n’y a pas de congé, il n’y a pas de breloque, \\
n’a pas coup de sec, n’a rien ; & pas de coup de sec, rien ; \\
quand mêm’li zour, quand mêm’ la nuit, & qu’il soit jour ou même qu’il soit nuit,\\
touzours travaill’, touzours, touzours. & on travaille toujours, toujours.\\
Là toute sorte l’ouvrage ; & Là, il y a toutes sortes d’ouvrages ;\\
là, barriqu’ qui n’a pas y en a fond, i faut rempli, & là une barrique sans fond qu’il faut remplir,\\
n’a pas capab’; là, moulin di l’huil’ & on ne le peut pas ; là un moulin à huile\\
tourn’touzours, touzours, n’a pas rété, & qu’il faut tourner toujours, toujours, sans s’arrêter ;\\
n’a pas donn’li temps ; & vous n’avez pas de repos ;\\
là i en a ein’ grand’ bougr’ montagn’, & là, il y a une grande bougre de montagne,\\
vou blizé roul’ barriqu’ là-haut, là-haut ; & vous êtes obligés de rouler une barrique tout en haut,\\
quand vou arriv’ procé là, n’a pas capab’, & quand vous approchez, il n’y a pas moyen, \\
n’a pas trouv’ son boute, li tomb’ encor’, & on ne peut en trouver le bout, elle tombe encore,\\
touzours, touzours. & toujours, toujours.\\
«~Ma ça qui bon sizet, & «~Mais celui qui est bon sujet, \\
ça qui content travaill’, & celui qui travaille avec plaisir, \\
ça qui n’a pas paress’, & celui qui n’est pas paresseux, \\
ça qui n’a pas volor, & celui qui n’est pas voleur, \\
y en a l’autr’ bon péye & pour celui-là, il y a un autre bon pays \\
pour manze plein ventr’, & où l’on mange à plein ventre, \\
pour dormi’ quand \textit{zaut’} voulé, & où vous dormez tant que vous voulez, \\
pour pose, & où vous vous reposez, \\
pour fair’ tout ça qu’\textit{zaut}’contents. & où vous faites tout ce qu’il vous plaît.\\
«~Ainsi, \textit{mon} \textit{zenfants}, & «~Ainsi, mes enfants,\\
vou voir qu’ pour êtr’heureux, & vous voyez que pour être heureux,\\
\textit{ça} la liberté là n’a rien si vou n’a pas travaill’, & cette liberté-là n’est rien si vous ne travaillez pas,\\
si vous fair’ tort cam’rad’. & si vous faites du tort à vos camarades. \\
Ca qui fair’ tort cam’rad’, li gagn’ \textit{pèr}, & Celui qui fait tort à ses camarades, la peur le gagne, \\
sou-li-kèr batt’ batté ; & son cœur bat vite et fort ; \\
quand li pass’ proce son maîtr’ & quand il passe près de son maître, \\
marce en bas en bas, & il marche la tête baissée, \\
tandi que ça qui fair’ bien son devoir, & tandis que celui qui fait bien son devoir \\
marce en déboutant, li guett’ son maîtr’ & marche ferme et droit, il regarde son maître, \\
dir’ li : & il lève sa tête en l’air et son maître lui dit : \\
«~Bonzour mon garçon !~». & «~Bonjour, mon garçon !~»\\
«~Si vou cout’bien ça qu’moi dir’vou, & «~Si vous écoutez bien ce que je vous dis, \\
moi va bien content zaut’, & je vais être bien content de vous autres \\
et tout dimance \textit{nou} fair’ la prièr’ & et tous les dimanches nous ferons la prière\\
pour dimand’ Bon Dié & pour demander au Bon Dieu\\
donn’ vou bou ciprit~». & qu’Il vous donne beaucoup d’esprit.~»\\
% % % % « Mon zenfant, vou libr’ auzourdi.      &      Mes enfants, vous êtes libres aujourd’hui.\\
% % % % A vla ein’ la loi que le roi grand’terr’   &  Voici une loi que le roi du continent \\
% % % % fini envoyé pour fair’ vou libr’.       &       vient d’envoyer pour vous rendre la liberté.  \\
% % % % Li fini acété      &        Il vient de racheter\\
% % % % vou le corps;   mais   &      vos corps mais, \\
% % % % comm’ li n’a pas    &  comme il n’avait pas  \\
% % % %   y en a assez la monnaie,    &  assez de monnaie,\\
% % % %  li fini donn’ sel’ment la moitié,  &  il n’en a donné seulement que la moitié\\
% % % % et pour que vou-maîtr’       &      et, pour que vos maîtres  \\
% % % %   n’a pas perdi son l’argent,      &      ne perdent pas leur argent,\\
% % % % li laiss’ vou travail’      &      ils vous laissent travailler \\
% % % %  encor’ six bann’anné      &       encore six années entières  .\\
% % % %  pour li sans payé.      &    pour eux sans être payés.\\
% % % % « Vou fini çanze nom;     &      Vous avez changé de nom  ,\\
% % % %    {}                     &      ; à cette heure vous êtes des apprentis, \\
% % % %     {}                    &         vous n’êtes plus esclaves ,\\
% % % %  à s’t’hèr vou libr’ ;      &       vous êtes libres,\\
% % % % vou comm’à moi, comm’à tout blanc      &      vous êtes comme moi, comme tous les Blancs,\\
% % % % la mêm’ soze ;       &      la même chose ; \\
% % % %   la loi égal’ pour nous tous.      &       la loi est égale pour vous tous.\\
% % % % Si moi fair’ vou tort,      &      Si je vous fais du tort, \\
% % % %   la loi va pini moi ;      &       la loi va me punir ;\\
% % % % si vou fair’ moi tort,    &      si vous me faites du tort,  \\
% % % %  la loi va pini vou.      &     la loi vous punira.\\
% % % % « Ca six bann’anné      &      Ces six années entières  \\
% % % %  que vou va zapprentis,      &      que vous allez être apprentis,\\
% % % % ça pour donn’ zaut’,      &      c’est pour vous faire acquèrir à vous autres  \\
% % % %  ciprit,      &      de l’esprit,\\
% % % % pour êtr’ bon sizet,    &      pour montrer à vous autres ce que vous avez à faire pour être de bons sujets,  \\
% % % %   pour bien travail’,      &       pour bien travailler,\\
% % % % pour n’a pas paress’,     &      pour n’être pas paresseux, \\
% % % %   pour n’a pas volor.      &       pour n’être pas voleurs.\\
% % % % Ca la loi là n’a pas badiné ;  & Cette loi-là ne plaisante pas;\\
% % % % n’a pas besoin croir’       &       il ne faut pas croire  \\
% % % %   que vou va fini demand’ grâce ;      &    que vous allez venir demander grâce;\\
% % % % li n’a pas pardonn’ vou,      &      elle ne vous pardonnera pas\\
% % % % comm’ moi été fair touzours.      &      comme j’ai toujours fait.\\
% % % % Si vou volor sel’ment      &      Si vous volez seulement  \\
% % % %   ein’zépi maye, assez ;      &    un épi de maïs, c’est assez ;\\
% % % % li renvoie vou dans ein’ l’autr’ péye ;      &      elle vous renvoie dans un autre pays ;\\
% % % % vou blizé laiss’        &      vous oblige à laisser  ,\\
% % % %   là vou-femm’, vou-pitits,      &       là votre femme, vos enfants,\\
% % % % tout ça qu’vou y en a.      &      tout ce que vous avez.\\
% % % % « A s’t’hèr qu’vou libr’,    &      A cette heure que vous êtes libres, \\
% % % %   vou capab’ allé    &     vous pourrez aller  \\
% % % %  côt’ vou voulé,      &      du côté que vous voudrez,\\
% % % % quand vou l’apprentissaze fini ;      &      quand votre apprentissage sera fini ;\\
% % % % personn n’a pas capab’ empèce vou.      &      personne ne pourra vous en empêcher.\\
% % % % mais dans quèqu’ pèye qu’vou allé,      &      Mais dans quelque pays que vous alliez,\\
% % % % i faut touzours manzé pour vivr’,      &      il faudra toujours manger pour vivre, \\
% % % %   et pou ça      &      et pour ça,\\
% % % % faut travaill’ et non pas volor ;      &      il faudra travailler et ne pas voler ;\\
% % % % comm’ moi été dir’ vou,     &      comme je vous l’ai dit, \\
% % % %  ça qui volor li cerce      &       celui qui vole cherche\\
% % % % son-la-misèr’.      &      à être malheureux.\\
% % % % « Ca qui voulé resté       &      « Ceux qui veulent rester  \\
% % % %  travaill’ avé moi,      &       à travailler avec moi,\\
% % % %  moi va   péyé li,      &      je vais les payer\\
% % % %  à caus’ pourquoi moi pli content gardé quéqu’ein’ qui fini coutumé      &      parce que j’aime mieux garder  quelqu’un avec qui je suis habitué\\
% % % % que fair’ vini Noirs trauzé.      &      que de faire venir des Noirs étrangers.\\
% % % % Moi va préfér’ zaut’      &      Moi, je vous préférerai aux autres,\\
% % % % tant que vou bon sizets.      &      tant que vous serez bons sujets.\\
% % % % « Aut’fois, quand vou été saclaves,      &      « Autrefois, quand vous étiez esclaves,\\
% % % % vou-maîtr’ était blizé gardé ça qui bon,      &      votre maître était obligé de garder celui qui était bon,\\
% % % % ça qui mauvais ; à s’t’hèr li va soziré.      &      celui qui était mauvais ; à cette heure, il va choisir.\\
% % % % Ca qui n’a pas travaill’,      &      Celui qui ne veut pas travailler,\\
% % % % ça qui sicaner, ça qui y en a trop la bousse,      &      qui est chicaneur, qui est trop sur la bouche,\\
% % % % ça qui fair tort cam’rad’, li va dir’ :      &      qui fait tort à son camarade, il va lui dire :\\
% % % % « Mon garçon, va-t-en,      &      « Mon garçon, va-t-en, moi,\\
% % % % moi n’a pli besoin toi ».       &      je n’ai plus besoin de toi. »\\
% % % % « Tout ça qu’moi dir’vou là, li dans vou-la-prièr’. Acouté bien :      &      Tout ce que je vous dis là est dans votre prière, écoutez bien :\\
% % % % « Un seul Dieu tu adoreras      &      « Un seul Dieu tu adoreras\\
% % % % et aimeras parfaitement.      &      et aimeras parfaitement.\\
% % % % Tes père et mère honoreras      &      Tes père et mère honoreras\\
% % % % afin de vivre longuement.      &      afin de vivre longuement.\\
% % % % Le bien d’autrui tu ne prendras      &      Le bien d’autrui tu ne prendras\\
% % % % ni retiendra à ton escient.      &      ni retiendra à ton escient.\\
% % % % « Un seul Dieu tu adoreras : vous bien conné.      &      « Un seul Dieu tu adoreras : vous savez bien ça.\\
% % % % « Tes père et mère honoreras : vou voir qui faut      &      « Tes père et mère honoreras : vous voyez qu’il faut\\
% % % % vou donn’ bon l’ésempl’ à vou-pitits ;      &      que vous donniez le bon exemple à vos enfants ;\\
% % % % vou-pitits devoir touzours respect pour son papa,      &      vos enfants doivent toujours du respect à leur papa,\\
% % % % pour son maman, pour vié di mound’ ;      &      à leur maman, aux personnes âgées ;\\
% % % % zenfants qui manqu’ respect à vié di mound’,      &      les enfants qui manquent de respect aux vieillards,\\
% % % % Bou Dié va pini li.      &      le Bon Dieu va les punir.\\
% % % % « Le bien d’autrui tu ne prendras :      &      « Le bien d’autrui tu ne prendras :\\
% % % % pour n’a pas volor,      &      pour ne pas voler,\\
% % % % accoutm’ zaut’à l’ouvraze, donn’ la main pour fair’ manze animaux,      &      il faut vous accoutumer à travailler, aider à faire manger les animaux,\\
% % % % manze coçons, travaill’  dans zardin.      &      manger les cochons, travailler dans le jardin.\\
% % % % Tou ça qu’ein di mound’ doit travaill’.      &      Tout le monde doit travailler.\\
% % % % Moi qui command’’zaut,      &      Moi qui commande à vous autres,\\
% % % % moi-même je travaille, chacun son ouvrage ; moi, mon travail est de veiller sur vous autres,      &      {}\\
% % % % pour maziné tout ça qui en a pour fair’ ;      &      d’imaginer tout ce qu’il y a à faire ;\\
% % % % vou travaill avé vou-la-main,      &      vous, vous travaillez avec la main,\\
% % % % moi, mon-la-têt’ qui travaill’      &      moi, c’est ma tête qui travaille\\
% % % % pour nourri zaut’ pour soign’ zaut’.      &      pour nourrir vous autres, pour soigner vous autres.\\
% % % % Ca qui n’a pas voulé travaill’,      &      Celui qui ne veut pas travailler,\\
% % % % li voulé gagn’ touzours ça qu’son cam’rad’      &      qui veut gagner toujours ce qui est à son camarade,\\
% % % % li en a ; pour ça li volor      &      celui-là est un voleur,\\
% % % % et li croir’ di mound’ n’a pas trouv’li,      &      il croit que personne ne peut le trouver ;\\
% % % % Bon Dié, partout, Bon Dié trouv’li ;      &      le Bon Dieu qui est partout, le Bon Dieu le trouve.\\
% % % % ici mêm côt’ moi parlé,      &      Ici même, à côté de moi qui vous parle,\\
% % % % li guett’nou, comm’à li là ;      &      Il nous guette, Il est là ;\\
% % % % quand vou allé dans zardin cam’rad      &      quand vous allez dans le jardin d’un camarade\\
% % % % pour volor maye,      &      pour voler du maïs,\\
% % % % li trouv’ vou la même soze.      &      Il vous voit la même chose.\\
% % % % Quéqu’zour li va pini vou ; quand vou mort,      &      Quelque jour Il vous punira ; quand vous serez morts,\\
% % % % vou allé dans ein’ diabr’   di péye      &      vous irez dans un diable de pays\\
% % % %  qui blancs  appell’ l’enfer.      &      que les Blancs nomment l’enfer.\\
% % % %  Là touzours travaill’,      &      Là, on travaille toujours,\\
% % % % n’a pas y en a dimance, n’a pas y en a berloque,      &      il n’y a pas de congé, il n’y a pas de breloque,\\
% % % % n’a pas coup de sec, n’a rien ;      &      pas de coup de sec, rien ;\\
% % % % quand mêm’li zour, quand mêm’ la nuit,      &      qu’il soit jour ou même qu’il soit nuit,\\
% % % % touzours travaill’, touzours, touzours.      &      on travaille toujours, toujours.\\
% % % % Là toute sorte l’ouvrage ;      &      Là, il y a toutes sortes d’ouvrages ;\\
% % % % là, barriqu’ qui n’a pas y en a fond, i faut rempli,      &      là une barrique sans fond qu’il faut remplir,\\
% % % % n’a pas capab’; là, moulin di l’huil’      &      on ne le peut pas ; là un moulin à huile\\
% % % % tourn’touzours, touzours, n’a pas rété,      &      qu’il faut tourner toujours, toujours, sans s’arrêter ;\\
% % % % n’a pas donn’li temps ;      &      vous n’avez pas de repos ;\\
% % % % là i en a ein’ grand’ bougr’ montagn’,      &      là, il y a une grande bougre de montagne,\\
% % % % vou blizé roul’ barriqu’ là-haut, là-haut ;      &      vous êtes obligés de rouler une barrique tout en haut,\\
% % % % quand vou arriv’ procé là, n’a pas capab’,      &      quand vous approchez, il n’y a pas moyen,\\
% % % % n’a pas trouv’ son boute, li tomb’ encor’,      &      on ne peut en trouver le bout, elle tombe encore, toujours, toujours.\\
% % % % touzours, touzours.\\
% % % % « Ma ça qui bon sizet,      &      « Mais celui qui est bon sujet,\\
% % % % ça qui content travaill’,      &      celui qui travaille avec plaisir,\\
% % % % ça qui n’a pas paress’,      &      celui qui n’est pas paresseux,\\
% % % % ça qui n’a pas volor,      &      celui qui n’est pas voleur,\\
% % % % y en a l’autr’ bon péye      &      pour celui-là, il y a un autre bon pays\\
% % % % pour manze plein ventr’,      &      où l’on mange à plein ventre,\\
% % % % pour dormi’ quand zaut’ voulé,      &      où vous dormez tant que vous voulez,\\
% % % % pour pose,      &      où vous vous reposez,\\
% % % % pour fair’ tout ça qu’zaut’contents.      &      où vous faites tout ce qu’il vous plaît.\\
% % % % « Ainsi, mon zenfants,      &      « Ainsi, mes enfants,\\
% % % % vou voir qu’pour êtr’heureux,      &      vous voyez que pour être heureux,\\
% % % % ça la liberté là n’a rien si vou n’a pas travaill’,      &      cette liberté-là n’est rien si vous ne travaillez pas,\\
% % % % si vous fair’ tort cam’rad’.      &      si vous faites du tort à vos camarades.\\
% % % % Ca qui fair’ tort cam’rad’, li gagn’ pèr,      &      Celui qui fait tort à ses camarades, la peur le gagne,\\
% % % % sou-li-kèr batt’ batté ;      &      son cœur bat vite et fort ;\\
% % % % quand li pass’ proce son maîtr’      &      quand il passe près de son maître,\\
% % % % marce en bas en bas,      &      il marche la tête baissée,\\
% % % % tandi que ça qui fair’ bien son devoir,      &      tandis que celui qui fait bien son devoir\\
% % % % marce en déboutant, li guett’ son maîtr’      &      marche ferme et droit, il regarde son maître,\\
% % % % dir’ li :      &      il lève sa tête en l’air et son maître lui dit :\\
% % % % « Bonzour mon garçon ! ».      &      « Bonjour, mon garçon ! »\\
% % % % « Si vou cout’bien ça qu’moi dir’vou,      &      Si vous écoutez bien ce que je vous dis,\\
% % % % moi va bien content zaut’,      &      je vais être bien content de vous autres\\
% % % % et tout dimance nou fair’ la prièr’ pour      &      et tous les dimanches nous ferons la prière pour\\
% % % % dimand’ Bon Dié  donn’ vou bou ciprit ».      &       demander au Bon Dieu qu’Il vous donne beaucoup d’esprit. »\\
\end{xltabular}


\subsection{Allocution créole d’Auguste Le Duc au sujet des 45 heures le 16 août 1835 (\citealt{Pourcelet1994} : 138-140)}\label{sec:kriegel:3.2}

\begin{xltabular}{\textwidth}{@{}>{\hangindent=1em}Q>{\hangindent=1em}Q@{}}                                                                                                                                                                   
«~\textit{Mon zenfants}, & «~Mes enfants, \\
ça \textit{la loi} qu’moi fini lir’ vou longtemps, & cette loi que je vous ai lue il y a longtemps, \\
ça que \textit{blanc} grand’terr’appell’bill, & que les Blancs sur le continent appellent bill,\\
ça qui été fair’ vous zapprentis libr’, & qui vous a faits apprentis libres, \\
ça \textit{la loi} là marqué 45 hèr’par semaine de travail & cette loi là marque 45 heures par semaine de travail \\
blizé pour vou-maitr’ & que vous êtes obligés de faire pour votre maître, \\
comm’à dir’ 9 hèr’ par zour \textit{zusqu’à} vendredi soir & c’est-à-dire 9 heures par jour jusqu’à vendredi soir, \\
en donnant dimance sam’di & en vous donnant congé samedi \\
et dimance dimance mêm’. & et congé dimanche encore.\\
Comm’ça mêm’moi été fair’ & C’est comme ça que j’ai fait\\
pace que moi maziné & parce que je me suis imaginé\\
que vou va gagn’ bon morceau lé temps & que vous auriez gagné tout ce temps\\
pour travaill’ & pour travailler \\
pour vou dans \textit{vou-zardins}, \textit{vou-la-caze}, & dans vos jardins, votre case, \\
\textit{vou-poulaillers, vou-parcs}, tout ça là.  & vos poulaillers, vous parcs, partout. \\
Ma moi n’a pas été longtemps & Mais je n’ai pas été longtemps\\
sans voir que ça n’a pas été bien comm’ça, & sans voir que ça n’était pas bien comme ça,\\
à caus’ pourquoi tout di mound’ & parce que tout le monde \\
n’a pas mêm’ li kèr. & n’a pas le même cœur. \\
La moitié, au lièr de s’occupé, & La moitié, au lieu de s’occuper, \\
été allé fair' diabr’ & sont allés faire le diable \\
pendant \textit{ça dé} zours là à l’îl’ du nord. & pendant ces deux jours-là à l’île du nord. \\
Tout ça zène zens surtout été montés & Tous ces jeunes gens-là surtout ont monté \\
cocos verts, batt’zozos, & sur les cocos verts pour battre les oiseaux, \\
coupé palmist’, la sasse poul’, & couper les palmistes, chasser les poules, \\
viré torti, et n’a pas vini & retourner les tortues, et ils ne sont pas venus \\
dimance matin à la prièr’. & dimanche matin à la prière.\\
Pour empèce ça, \textit{moi} été blizé fair’ vou & Pour empêcher ça, je suis obligé \\
travaill’ sam’di. & de vous faire travailler le samedi.\\
Sel’ment, au lièr donn’ moi 9 hèr’, & Seulement, au lieu de me donner 9 heures \\
vou n’a pli donn’ \textit{moi} que 7 hèr’ ½, & vous ne me donnerez plus que 7 heures et demi, \\
c’est à dir’ que vou doit moi depuis 6 hèr’ & c’est-à-dire que vous me devez depuis 6 heures \\
Bon matin \textit{zusqu’à} 4 hèr’ après-midi, y compris & du matin jusqu’à 4 heures après-midi, y compris \\
2 hèr’1/2, l’hèr’ sibancouque et dizené. & 2 heures et demie, l’heure sibancouque et déjeuner. \\
1 caus’ ça mêm’ moi été diminué la tâce, & C’est à cause de ça que j’ai diminué la tâche, \\
comm’ vou été voir. & comme vous allez voir.\\
A s’t’hèr’acouté bien ça qu’moi dir’vou. & A cette heure, écoutez bien ce que je vais vous dire.\\
Comm’ vou libr’, & Comme vous êtes libres, \\
ça qui n’a pas voulé, arranz’ li, & que ceux qui ne veulent pas s’arrangent, \\
ça qui voulé va lèv’la main. & que ceux qui veulent bien lèvent la main. \\
Ca la tâce que moi parl’ vou là, & Cette tâche dont je vous parle, \\
vou que doit moi li ; & c’est ce que vous me devez ;\\
 ça que vous fair’en pliss, & ce que vous ferez en plus,\\
moi qui doit vou li, & moi, je vous le devrai,\\
comm’ ça mêm’zaut’fair’à Maurice & comme les autres font à Maurice\\
que ça zens là appell «~extra-service~». & et que ces gens-là appellent «~extra-service~». \\
Ca qui ramass’ mentor bon Dié va pini li. & Celui qui est menteur le Bon Dieu va le punir.\\
Zapprentis passé ein’l’engazement, ein’papié, & Les apprentis ont passé un engagement, un papier \\
avé \textit{zaut} maîtr’ devant ça \textit{blanc} & avec d’autres maîtres devant un Blanc \\
qui appell’ zize protectèr, & qu’on appelle juge-protecteur,\\
et ça qui voulé travaill’, & et ceux qui veulent travailler\\
gagn’ morceau l’arzent & gagnent de l’argent\\
pour aceté de quoi fair’bouillon & pour acheter de quoi faire du bouillon, \\
comm’ zaut’ y en a ici. & comme vous autres en avez ici.\\
Avé \textit{ça dé} trois «~cass~»’-là & Avec ces 2, 3 caches-là,\\
zaut’ acét’vivr’ & les autres achètent les vivres \\
pour marmaill’, la harde, morceau di sel, tabac, & pour la marmaille, les hardes, sel, tabac, \\
la viand’ salée, morceau brèd’, coup de sec, & viande salée, brèdes, coup de sec, \\
enfin tout quèqu’soze que zaut’ & enfin toutes les choses dont ces autres \\
y en a besoin à caus’ pourquoi à Maurice & ont besoin parce que, à Maurice, \\
\textit{la loi} n’a pas embrass’ & la loi n’entre pas\\
tout \textit{ça} zistoir’ bouillon là. & dans toutes ces histoires de bouillon-là.\\
Pourvu qu’ein’ noir gagné son bol 1/2 dou riz sec, & Pourvu qu’un Noir gagne son bol et demi de riz sec\\
assé ; li n’a pas n’a rien pour dir’ ; & c’est assez ; il n’a rien à dire ;\\
tandi qu’à Galéga y en a tout ça pour n’a rien. & tandis qu’à Galéga, vous avez tout ça pour rien.\\
Ici vou n’a pas misèr’ bouillon, cocos zermés, & Ici, vous avez abondamment bouillon, cocos germés,\\
cocos di l’eau, zozo, poisson, cypails, poul’, & cocos d’eau, oiseaux, poissons, cipails, poules, \\
di zefs gaulett’. & œufs de goëlette. \\
Moi donn’zaut’sel’ment ein’bol & Moi, je vous donne à vous autres seulement un bol\\
au lièr d’ein’bol ½ maye, & au lieu de un bol et demi de maïs ; \\
ma moi donn’ zaut’ coup de sec, & mais moi, je donne à vous autres coup de sec,\\
tabac, savon, pounac pour \textit{zaut’} coçons, & tabac, savon, pounac pour vos cochons,\\
la terr’ pour fair’ zardin & de la terre pour faire vos jardins \\
autant qu’zaut’voulé, & autant que vous voulez,\\
gardien pour \textit{dé} semain’ & un gardien pendant deux semaines\\
pour pouss’bébér’ et sicougoué (perdrix) & pour chasser les bêtes et les perdrix ;\\
quand maye endi lair, gardien & quand le maïs est en lait, un gardien \\
pour les aucangues (pintades), pour cipails, & pour les pintades, pour les cipails, \\
pour veill’cam’rad’. & pour veiller sur vos camarades. \\
Quand bou Dié soulaz’zaut’, & Quand le Bon Dieu vous favorise,\\
vou vend’maye, & vous vendez du maïs, \\
vou vend’la graiss’ ; avé ça & vous vendez de la graisse ; avec ça, \\
zaut’acét’ la toil’, mouçoirs, etc., etc. & vous achetez de la toile, des mouchoirs etc., etc.\\
A s’t’hèr acouté bien. & A cette heure, écoutez bien.\\
Vou voulé travaill’ depuis 6 hèr’ bon matin & Si vous voulez travailler depuis 6 heures du matin\\
\textit{zusqu’à} 4 hèr’ après-midi, compris  & jusqu’à 4 heures après-midi, compris\\
2 hèr’ ½ pour sibancouque et dizené ; & 2 heures et demie pour sibancouque et déjeuner ; \\
lors vou n’a pas gagn’ & alors vous n’aurez pas\\
gardien pour zardin, & de gardien pour vos jardins, \\
vou n’a pas gagn’ tabac, ni pounac, ni savon, & vous ne gagnerez ni tabac, ni pounac, ni savon,\\
ni coup de sec, & ni coup de sec,\\
mais vou va gagn’ ein’ bol ½ dou riz & mais vous gagnerez un bol et demi de riz\\
ou maye moulé, comm’à Maurice. & ou de maïs moulu comme à Maurice.\\
Ou si bien vou voulé continué & Ou bien, si vous voulez continuer \\
comm’ vou été \textit{touzours} fair’, & comme vous avez toujours fait, \\
vou va gagn’ tou ça qu’moi fini donné zaut’ & vous gagnerez tout ce que je vous ai donné \\
\textit{zusqu’à} présent.  & jusqu’à présent. \\
Vou va fair’ vou-l’ouvraze \textit{touzours} à la tâce & Vous ferez votre ouvrage toujours à la tâche \\
comm’ avant ; et, & comme auparavant et, \\
quand moi y en a besoin corvée ou & quand j’aurai besoin de vous pour corvée ou \\
«~extra-service~» vou fair’li. & «~extra-service~», vous le ferez.\\
Si li bien comm’ça, moi content~». & Si vous trouvez ça bien, je suis content.~»\\
\end{xltabular}

\section{Premiers éléments d’une analyse linguistique}\label{sec:kriegel:4}
\begin{sloppypar}
Dans les paragraphes qui suivent, je proposerai de premiers éléments d’analyse linguistique des textes (pour quelques généralités voir \sectref{sec:kriegel:3}) en me limitant à quelques remarques sur la graphie et la phonétique (\sectref{sec:kriegel:4.1}) et à une présentation du syntagme nominal (\sectref{sec:kriegel:4.2}). Je fournirai de brèves comparaisons avec des textes contemporains provenant de l’île Maurice ainsi qu’avec les traits correspondants en créoles mauricien et seychellois actuels. Comme mentionné ci-dessus, Galéga se trouve géographiquement proche de l’archipel des Seychelles où Auguste Le Duc a passé les dernières années de sa vie. Il est donc tentant de chercher dans les textes des similitudes avec cette variété de créole. Cependant, la prudence est de mise : le créole seychellois dont nous n’avons pas de témoignages jusqu’au début du 20\textsuperscript{e} siècle \citep{Young1983} doit être considéré comme une continuation de variétés stables du créole mauricien.
\end{sloppypar}

\subsection{Graphie et phonétique}\label{sec:kriegel:4.1}

Sans surprise, la graphie choisie par Auguste Le Duc est relativement proche de la graphie française du milieu du 19\textsuperscript{e} siècle mais la variation est importante et ses choix n’ont rien de systématique. Souvent les lettres muettes à la fin du mot sont maintenues~(p.ex. “corps”, “zenfant”, “blanc”, “mauvais” etc.),\footnote{Les exemples provenant des deux allocutions reproduites dans \sectref{sec:kriegel:3.1} et \sectref{sec:kriegel:3.2} sont donnés entre parenthèses. Les exemples qui proviennent d’échantillons non reproduit dans le présent article (voir liste en \sectref{sec:kriegel:2.3}) font l’objet d’une citation qui indique la page dans l’ouvrage de Pourcelet.} y compris, en règle générale, le -\textit{s} du pluriel nominal (p.ex. “saclaves”, “zardins”, “poulaillers”, “parcs”). Seulement dans de rares cas le -\textit{s} est omis bien qu’une lecture plurielle s’impose (p.ex. “ça que \textbf{blanc} grand’terr’ appell’ bill”).

Auguste Le Duc rend compte de quelques changements phonétiques majeurs comme par exemple 

\begin{itemize}\sloppy
\item le changement des voyelles palatales labiales en voyelles non labiales présent dans tous les créoles français, à savoir /y/>/i/ (p.ex. \textit{pini},~\textit{perdi} etc.) ; /∅/>/e/ (p.ex. \textit{dé} etc.) ainsi que /oe/>/ɛ/ (p.ex. \textit{pèr}) sauf dans quelques cas dans lesquels la graphie française en “u” est maintenue (p.ex. \textit{zusqu’à}).
\item il rend également compte de façon quasi-systématique de la perte des chuintantes /\href{https://en.wikipedia.org/wiki/Voiceless_postalveolar_fricative}{ʃ}/ et /\href{https://en.wikipedia.org/wiki/Voiced_postalveolar_fricative}{{ʒ}}/, produites comme sifflantes dans les créoles de l’océan Indien~(p.ex. \textit{touzours}, \textit{manzé}, \textit{soziré}, \textit{sicaner}, \textit{zusqu’à}).
\end{itemize}

Les nombreuses agglutinations d’articles sont soit séparées par un espace~(p.ex. \textit{ein’ \textbf{la loi}}, \textit{vou \textbf{le corps}}, \textit{ca \textbf{la loi}} etc.), soit notées avec un trait d’union comme le font déjà sporadiquement des auteurs comme \citet{Chrestien1822} ou \citet{Freycinet1827} (cités d’après \citealt{Chaudenson1981} : 87ss.). Chez Auguste Le Duc, la notation des agglutinations avec un trait d’union se fait surtout en présence d’un déterminant possessif (p.ex. \textit{vou-la}{}-\textit{caze}, \textit{son}{}-\textit{la}{}-\textit{misèr}’), le déterminant possessif étant relié au nom (sans agglutination) auquel il se réfère par un trait d’union aussi~(p.ex. \textit{vou}{}-maîtr’, \textit{vou}{}-\textit{femm}’, \textit{vou}{}-\textit{pitits}). Pour la 2\textsuperscript{e} personne, il pourrait s’agir d’un souci de distinction des déterminants possessifs des pronoms personnels correspondants (voir ci-dessous). Le trait d’union entre le déterminant possessif et le nom est déjà attesté chez \citet{Chrestien1818}, mais pas chez les auteurs ultérieurs du corpus étudié par \citealt[13]{BakerEtAl2007}. Une spécificité graphique des textes d’Auguste Le Duc consiste dans le remplacement quasi~– mais pas entièrement~– systématique de \textit{e} muet par un apostrophe, surtout en fin de mot.

\subsection{Syntagme nominal}\label{sec:kriegel:4.2}

\subsubsection{Marque du pluriel}

La \textsc{marque du pluriel} \textit{ban} présente dans toutes les variétés actuelles des créoles de l’océan Indien n’est pas attestée. Ce fait n’est guère surprenant si on considère que les premières attestations de \textit{ban} comme marque du pluriel ne datent que de la fin du 19\textsuperscript{e} siècle \citep{Bollée2000}. Nous trouvons deux attestations de “bann’anné” mais comme l’explique \citet[135ss]{Baker2003} cette forme, présente dans d’autres textes anciens, ne doit aucunement être interprétée comme une marque du pluriel mais comme «~la première syllabe de [banane], un seul morphème en mauricien, dérivé de \textit{bonne année} (…)~». Pour référer à des noms au pluriel, Auguste Le Duc emploie presque systématiquement la marque du pluriel française -\textit{s} inaudible à l’oral (voir \sectref{sec:kriegel:4.1}).

\subsubsection{Déterminants}

Dans les textes d’Auguste Le Duc, les \textsc{agglutinations d’articles} français sans fonction sont très fréquentes (voir \sectref{sec:kriegel:4.1}) comme en créole mauricien et seychellois actuels.

On y trouve de nombreuses attestations de l’\textsc{article indéfini} (p.ex. \textit{ein’ la loi}, \textit{ein l’autr’ péye}), attesté depuis 1818 (\citealt{BakerEtAl2007} : 9ss.). Concernant les syntagmes nominaux définis et spécifiques, le cas non marqué est \textsc{l’absence d’article} (sauf en cas d’agglutination). Par ailleurs, nous trouvons plusieurs attestations de \textit{sa} antéposé (p.ex. \textit{\textbf{Ca} six bann’anné que}…~, \textit{\textbf{ça} blanc qui appell’ zize}, \textit{Tout \textbf{ça} zène zens surtout été montés cocos verts}). Alors que le créole seychellois actuel retient \textit{sa} antéposé~au nom, le créole mauricien actuel retient \textit{la} postposé au nom. Cependant, la présence de plusieurs attestations d’un “ça” antéposé n’est pas un argument pour rapprocher les textes d’Auguste Le Duc du créole seychellois, qui, de toutes les façons, n’existait probablement pas encore comme variété indépendante en 1835. Comme \citet[66]{Guillemin2007} le constate, il existe d’autres attestations de \textit{sa} antéposé sans qu'elles ne soient suivies de \textit{la} dans des textes anciens du créole mauricien, datant respectivement de 1818 et de 1822. Elles se trouvent dans des contextes où elles sont suivies d’une relative.~Ceci est également le cas pour deux des exemples relevés chez Auguste Le Duc (voir ci-dessus). Quelques cas d’\textsc{articles démonstratifs} qui sont attestés dès 1749 \citep[66]{Guillemin2007} se trouvent également dans les textes examinés ici (p.ex.\textit{\textbf{Ca} la loi \textbf{là}}, \textit{\textbf{ça} la liberté \textbf{là}}, \textit{\textbf{ça} dé zours \textbf{là}}, \textit{\textbf{ça} dé trois «~cass~»’-\textbf{là}}, \textit{\textbf{ça} zistoir’ bouillon \textbf{là}}). Dans les allocutions, les \textsc{déterminants possessifs} (1SG\textit{ \textbf{mon} zenfant}, 3SG \textit{\textbf{son} maman}, 2PL \textit{\textbf{vou} le corps}, 3PL \textit{\textbf{zaut} maîtr}) ne sont pas représentés à toutes les personnes. Ce fait est probablement dû à l’absence de contextes appelant ces formes. Les formes attestées correspondent au paradigme que l’on connaît en créole mauricien et seychellois actuels, exception faite de la 2\textsuperscript{e} personne du pluriel où nous rencontrons majoritairement \textit{vou} (p.ex.\textit{\textbf{vou}\textit{-}zardins}, \textit{\textbf{vou}\textit{-}la-caze},\textit{\textbf{vou}\textit{-}poulaillers},\textit{\textbf{vou-}parcs}). La forme \textit{zot} qui est la forme du déterminant possessif à la deuxième personne du pluriel en créole mauricien (forme formelle) et seychellois actuels est également attestée mais se limite à un seul exemple (\textit{\textbf{zaut}\textit{’} coçons}). 

\subsubsection{Pronoms personnels}

Comme il s’agit d’une allocution d’un maître qui a pour objectif d’organiser la vie de travail d’une petite communauté de personnes qui dépendent de lui, les pronoms personnels de 1\textsuperscript{re} personne du singulier (le locuteur) et de 2\textsuperscript{e} personne du pluriel (les allocutaires) sont de loin les plus fréquents dans les textes. Les deux méritent un intérêt particulier parce que des changements se sont produits au cours du 19\textsuperscript{e} siècle : \citet[187]{Chaudenson1981} retrace le passage de la forme de première personne du singulier sujet \textit{moi}, qui correspond à la forme non clitique française, au profit de la forme \textit{mo}, forme sujet du pronom personnel 1\textsuperscript{re} personne en créole mauricien actuel (\textit{mon} en créole seychellois actuel). Il constate qu’en 1835 (Bill d’affranchissement) seule la forme “moi” est employée et qu’il est impossible de savoir s’il s’agit éventuellement d’une forme acrolectale. Dans les allocutions d’Auguste le Duc, nous trouvons exclusivement la forme “moi” (p.ex. “\textit{moi} été blizé fair’ vou travail”). En revanche, quelques occurrences de la forme “mo” sont attestées dans d’autres passages en créole. Ces exemples se trouvent exclusivement dans des passages où Auguste Le Duc fait parler des esclaves, comme par exemple dans l’épisode fictif de l’enlèvement d’Adélaïde. L’esclave Lundi explique son retard à sa maîtresse Adélaïde : 

\begin{quote}
\textit{Mo} té fair’ ein’ pitit’ tournée à Flacq pour compagné moussié Zul’ Dilett’ chez moussié Belzim par son zordre vou maman. \textit{Mo} té laiss’li avec son zamir, et pis vini, ah, vla tout… Pourtant, faut \textit{mo} dir’ vous, grand’ madam’ té voulé vini pour prendi vous, mais li n’a pas té capab’ à caus’li un peu malad’\hbox{}\hfill\hbox{(\citealt{Pourcelet1994} : 34)}
\end{quote}

Alors que Chaudenson se questionne sur un éventuel caractère acrolectal de \textit{moi} (voir ci-dessus), \citet{Baker1976} est plus explicite :

\begin{quote}
Examination of the texts favouring one or other form of the 1st person pronoun strongly suggests that it was slaves who first began to use the \textit{mo} form and that whites were reluctant to adopt this form for several decades.\\\hbox{}\hfill\hbox{(\citealt[46–48]{Baker1976}, cite d’après \citealt{BakerEtAl2007} : 7)}
\end{quote}

Les textes analysés ici permettent donc de confirmer les hypothèses énoncées dans la littérature existante.

Concernant les pronoms personnels de 1\textsuperscript{re} personne en fonction objet, l’emploi de la forme \textit{moi} (p.ex. “vou n’a pli donn’ \textit{moi} que 7 hèr”) est stable et traverse époques et registres.

Dans les textes d’Auguste Le Duc, il n’y a que de très rares attestations de la forme informelle de la 2\textsuperscript{e} personne singulier en position sujet. L’une d’elles se trouve toujours dans l’épisode fictif de l’enlèvement. La jeune créole Adélaïde s’adresse à son esclave Lundi :

\begin{quote}
Eh bien, Lundi, ah vla \textit{toi} donc? Longtemps nous séparé \textit{toi} vini.\\\hbox{}\hfill\hbox{(\citealt{Pourcelet1994} : 34)}
\end{quote}

L’autre se trouve dans une réplique que fait Auguste Le Duc à un de ses esclaves. 

Il semblerait que cette forme soit réservée aux contextes dans lesquels le maître s’adresse à un esclave. Cette forme informelle d’adresse a disparu en créole seychellois actuel au profit de la forme formelle \textit{ou}. En créole mauricien actuel, elle a été préservée et il existe une différenciation en forme objet \textit{twa} et en forme sujet \textit{to} dont nous ne trouvons pas de trace chez Auguste le Duc. La forme formelle ou de politesse \textit{vou} pour la deuxième personne du singulier est attestée dans le même passage : un esclave s’adresse à la demoiselle Adélaïde de la bourgeoisie créole. Cette forme apparaît en position sujet et en position objet.

\begin{quote}
Quoi \textit{vous} voulé, mamzell’?\\
Y en a grand misèr pour passé, moi dir’ \textit{vous}.\hbox{}\hfill\hbox{(\citealt{Pourcelet1994} : 34)}
\end{quote}

En ce qui concerne la 3\textsuperscript{e} personne du singulier, nous trouvons \textit{li} en position sujet comme en position objet (p.ex. “\textit{li} laiss’ vou travail’ encor’ six bann’ anné pour \textit{li} sans payé”) comme en créole mauricien actuel. La forme \textit{i}, qui a pris une évolution spécifique en créole réunionnais et en créole seychellois (\citealt{Michaelis2000}, \citealt{Watbled2016}) est attestée dans des constructions impersonnelles. Les textes d’Auguste Le Duc ne contiennent que très peu d’attestations du pronom personnel de 1\textsuperscript{re} personne du pluriel ‘nous’, en position sujet comme en position objet (p.ex. “\textit{nou} fair’ la prièr’ pour dimand’ Bon Dié donn’ vou bou ciprit’; ‘la loi égal’ pour \textit{nous} tous’), ce qui correspond à l’usage en créole mauricien et seychellois actuels mais également au modèle français. Quant à la 2e personne du pluriel, l’évolution qui consiste à remplacer \textit{vou} par \textit{zot} (réduction phonétique à partir de \textit{vous autres}) est en plein cours : nous trouvons ‘vou’ en position sujet de façon presque systématique alors qu’en position objet nous relevons une majorité d’occurrences de ‘zaut’ (p.ex. ‘Ca six bann’anné que \textit{vou} va zapprentis, ça pour donn’ \textit{zaut}’ ciprit’). Cette distribution des formes sujet et objet nous donne une indication sur le déroulement du changement de \textit{vous} en fonction sujet et en fonction objet du français aux créoles mauricien et seychellois actuels où on trouve \textit{zot} en position sujet et en position objet. Le passage du pronom personnel de 2\textsuperscript{e} personne du pluriel en position sujet ‘vou’ à la forme ‘zaut’, qui correspond à l’usage actuel des créoles mauricien et seychellois est attestée à deux reprises (‘quand \textit{zaut}’ voulé, pour pose, pour fair’ tout ça qu’\textit{zaut’}contents’). Pour la 3\textsuperscript{e} personne du pluriel, la forme ‘zaut’ est employée systématiquement pour la fonction sujet et pour la fonction objet, comme c’est toujours le cas en créole mauricien et seychellois actuels. 
En ce qui concerne la 3\textsuperscript{e} personne du singulier, nous trouvons \textit{li} en position sujet comme en position objet (p.ex. “\textit{li} laiss’ vou travail’ encor’ six bann’ anné pour \textit{li} sans payé”) comme en créole mauricien actuel. La forme \textit{i}, qui a pris une évolution spécifique en créole réunionnais et en créole seychellois (\citealt{Michaelis2000}, \citealt{Watbled2016}) est attestée dans des constructions impersonnelles. Les textes d’Auguste Le Duc ne contiennent que très peu d’attestations du pronom personnel de 1\textsuperscript{re} personne du pluriel ‘nous’, en position sujet comme en position objet (p.ex. “\textit{nou} fair’ la prièr’ pour dimand’ Bon Dié donn’ vou bou ciprit’; ‘la loi égal’ pour \textit{nous} tous’), ce qui correspond à l’usage en créole mauricien et seychellois actuels mais également au modèle français. Quant à la 2e personne du pluriel, l’évolution qui consiste à remplacer \textit{vou} par \textit{zot} (réduction phonétique à partir de \textit{vous autres}) est en plein cours : nous trouvons ‘vou’ en position sujet de façon presque systématique alors qu’en position objet nous relevons une majorité d’occurrences de ‘zaut’ (p.ex. ‘Ca six bann’anné que \textit{vou} va zapprentis, ça pour donn’ \textit{zaut}’ ciprit’). Cette distribution des formes sujet et objet nous donne une indication sur le déroulement du changement de \textit{vous} en fonction sujet et en fonction objet du français aux créoles mauricien et seychellois actuels où on trouve \textit{zot} en position sujet et en position objet. Le passage du pronom personnel de 2\textsuperscript{e} personne du pluriel en position sujet ‘vou’ à la forme ‘zaut’, qui correspond à l’usage actuel des créoles mauricien et seychellois est attestée à deux reprises (‘quand \textit{zaut}’ voulé, pour pose, pour fair’ tout ça qu’\textit{zaut’}contents’). Pour la 3\textsuperscript{e} personne du pluriel, la forme ‘zaut’ est employée systématiquement pour la fonction sujet et pour la fonction objet, comme c’est toujours le cas en créole mauricien et seychellois actuels. 
En ce qui concerne la 3\textsuperscript{e} personne du singulier, nous trouvons \textit{li} en position sujet comme en position objet (p.ex. “\textit{li} laiss’ vou travail’ encor’ six bann’ anné pour \textit{li} sans payé”) comme en créole mauricien actuel. La forme \textit{i}, qui a pris une évolution spécifique en créole réunionnais et en créole seychellois (\citealt{Michaelis2000}, \citealt{Watbled2016}) est attestée dans des constructions impersonnelles. Les textes d’Auguste Le Duc ne contiennent que très peu d’attestations du pronom personnel de 1\textsuperscript{re} personne du pluriel ‘nous’, en position sujet comme en position objet (p.ex. “\textit{nou} fair’ la prièr’ pour dimand’ Bon Dié donn’ vou bou ciprit”; “la loi égal’ pour \textit{nous} tous”), ce qui correspond à l’usage en créole mauricien et seychellois actuels mais également au modèle français. Quant à la 2e personne du pluriel, l’évolution qui consiste à remplacer \textit{vou} par \textit{zot} (réduction phonétique à partir de \textit{vous autres}) est en plein cours : nous trouvons \textit{vou} en position sujet de façon presque systématique alors qu’en position objet nous relevons une majorité d’occurrences de \textit{zaut} (p.ex. “Ca six bann’anné que \textit{vou} va zapprentis, ça pour donn’ \textit{zaut}’ ciprit”). Cette distribution des formes sujet et objet nous donne une indication sur le déroulement du changement de \textit{vous} en fonction sujet et en fonction objet du français aux créoles mauricien et seychellois actuels où on trouve \textit{zot} en position sujet et en position objet. Le passage du pronom personnel de 2\textsuperscript{e} personne du pluriel en position sujet \textit{vou} à la forme \textit{zaut}, qui correspond à l’usage actuel des créoles mauricien et seychellois est attestée à deux reprises (“quand \textit{zaut}’ voulé, pour pose, pour fair’ tout ça qu’\textit{zaut’}contents”). Pour la 3\textsuperscript{e} personne du pluriel, la forme \textit{zaut} est employée systématiquement pour la fonction sujet et pour la fonction objet, comme c’est toujours le cas en créole mauricien et seychellois actuels. 

\begin{quote}
\textit{Zaut’} campé dans Cure-pipe. Ca grand coquin Lindor command’ \textit{zaut.}\\\hbox{}\hfill\hbox{(\citealt{Pourcelet1994} : 34)}
\end{quote}

\subsubsection{Pronoms relatifs~et complémentiseur}

Dans ses allocutions, Auguste le Duc maintient majoritairement un usage proche du français en employant la forme \textit{que} aussi bien comme pronom relatif objet (p.ex. “A vla ein’ la loi \textit{que} le roi grand’terr’ fini envoyé”) que comme complémentiseur (p.ex. “n’a pas besoin croir’ \textit{que} vou va fini demand’ grâce”) et le pronom relatif \textit{qui} comme pronom relatif sujet (p.ex. “Moi \textit{qui} command’ zaut”), sauf un cas d’emploi de \textit{qui} en position de relatif objet (“vou allé dans ein’ diabr’ di péye \textit{qui} blancs appell’ l’enfer”) ainsi qu’un cas d’emploi de \textit{que} en position de relatif sujet (“vou \textit{que} doit moi li”). Dans les créoles mauriciens et seychellois actuels, la forme \textit{ki} couvre tous ces emplois. Dans des textes contemporains aux allocutions, p.ex. le Bill d’affranchissement de 1835, l’emploi de \textit{que} comme pronom relatif objet ou comme complémentiseur a déjà quasiment disparu au profit de \textit{qui}. 


\section{Conclusion}\label{sec:kriegel:5}
La «~découverte~» de ces textes dans une édition déjà existante mais inconnue des linguistes enrichit notre corpus de textes anciens. Il s’agit de genres textuels atypiques~dans la mesure où ce sont des textes argumentatifs qui portent sur des sujets quotidiens. Ils ne relèvent pas de l’instruction religieuse et ne sont pas non plus des adaptations ou des traductions littéraires, ni de formes rimées (pour les genres textuels dominants dans les textes anciens voir \citealt{Kriegel2015} : 648). Même s’il constate dans son compte-rendu de 1995 que «~le créole parlé à Agaléga est du mauricien~» (1995 : 114), Chaudenson (c.p.) avait suggéré de regarder si les textes créoles présentés dans l’édition de \citet{Pourcelet1994} ne se rapprochaient pas du créole seychellois, notamment en tenant compte du fait que le journal d’Auguste Le Duc contient des lexèmes uniquement attestés en créole seychellois. L’hypothèse d’un rapprochement au créole seychellois est certes tentante si on tient compte de la situation géographique de Galéga et de la biographie d’Auguste Le Duc, qui a passé la dernière partie de sa vie dans l’archipel des Seychelles. Pourtant, l’examen des textes et du contexte sociohistorique nous amène à donner une réponse négative~à cette question, et ceci pour plusieurs raisons, d’ordre sociohistorique et d’ordre linguistique. Une lecture détaillée du journal d’Auguste Le Duc révèle que ses esclaves à Galéga sont venus de Maurice, de Madagascar ou qu’ils étaient nés sur place. Par ailleurs, le créole seychellois en tant que variété sensiblement différente du mauricien a dû se former plus tard que 1835. Dans les adaptations des fables de La Fontaine par Rodolphine Young au début du 20\textsuperscript{e} siècle \citep{Young1983} nous trouvons encore des formes proches du mauricien. Dans les allocutions, le seul trait qui pourrait faire penser à du créole seychellois~est le déterminant démonstratif/défini \textit{sa} antéposé. Cependant, il est également attesté dans d’autres textes anciens en créole mauricien. L’analyse linguistique, qui doit être nuancée et complétée notamment par une étude du syntagme verbal, ne nous révèle pas de grandes surprises~mais permet plutôt de confirmer des tendances évolutives observées dans d’autres textes mauriciens du milieu du 19\textsuperscript{e} siècle.

\section*{Remerciements}
Cet article est dédié à Georges Daniel Véronique qui m’a beaucoup inspiré sur la question de la diachronie des créoles.

\printbibliography[heading=subbibliography,notkeyword=this]
\end{otherlanguage}
\end{document}
