\documentclass[output=paper]{langscibook}
\ChapterDOI{10.5281/zenodo.10280592}
\author{Cyrille Granget\orcid{}\affiliation{Laboratoire de NeuroPsychoLinguistique, UR 4156, Université de Toulouse Jean-Jaurès} and Guillaume Fon Sing\orcid{}\affiliation{Laboratoire de Linguistique Formelle, CNRS, Université Paris Cité} and Isabel Repiso\orcid{}\affiliation{Paris Lodron Universität Salzburg}}
\title{Introduction}
\abstract{\noabstract}
\IfFileExists{../localcommands.tex}{
  \addbibresource{../localbibliography.bib}
  % add all extra packages you need to load to this file

\usepackage{tabularx,multicol}
\usepackage{url}
\urlstyle{same}

\usepackage{listings}
\lstset{basicstyle=\ttfamily,tabsize=2,breaklines=true}

\usepackage{langsci-basic}
\usepackage{langsci-optional}
\usepackage{langsci-lgr}
\usepackage{langsci-osl}
% \usepackage{./langsci/styles/langsci-lgr}
% \usepackage{./langsci/styles/langsci-osl}
% \usepackage{langsci-gb4e}

\usepackage{tikz}
\usetikzlibrary{patterns,calc}
\pgfdeclarepatternformonly{south east lines}{\pgfqpoint{-0pt}{-0pt}}{\pgfqpoint{3pt}{3pt}}{\pgfqpoint{3pt}{3pt}}{
    \pgfsetlinewidth{0.6pt}
    \pgfpathmoveto{\pgfqpoint{0pt}{3pt}}
    \pgfpathlineto{\pgfqpoint{3pt}{0pt}}
    \pgfpathmoveto{\pgfqpoint{.2pt}{-.2pt}}
    \pgfpathlineto{\pgfqpoint{-.2pt}{.2pt}}
    \pgfpathmoveto{\pgfqpoint{3.2pt}{2.8pt}}
    \pgfpathlineto{\pgfqpoint{2.8pt}{3.2pt}}
    \pgfusepath{stroke}}
    
\usepackage{stmaryrd}
\usepackage{wasysym}
\usepackage{multirow}
\usepackage{caption}
\usepackage{subcaption}
\usepackage{mathrsfs}
\usepackage{qtree}

\usepackage{linguex}


  %pminos do not split footnotes
% \interfootnotelinepenalty=10000 %Footnote in Laporte chapters has to be split SN


%\DeclareIndexNameFormat{default}{%
%\nameparts{#1}%
%\usebibmacro{index:name}%
%{\index[names]}%
%{\namepartfamily}%
%{\namepartgiveni}%
% {}% L1
% {}% L2
%{\namepartprefix}% generates spurious space L3
%{\namepartsuffix}% generates spurious space L4
%}

%  {\DeclareIndexNameFormat{default}{%
%     \usebibmacro{index:name}{\index[names]}{#1}{#3}{#5}{#7}}}

%\DeclareIndexNameFormat{default}{%
%  \usebibmacro{index:name}{\sindex[nom]}{#1}{#3}{#5}{#7}}

%\DeclareIndexNameFormat{default}{%
%  \usebibmacro{index:name}{\sindex[person]}{#1}{#3}{#5}{#7}}
%\DeclareIndexNameFormat{default}{%
%\nameparts{#1} \usebibmacro{index:name}{\sindex[person]]}{\namepartfamily}{‌​\namepartgiven}{\nam‌​epartprefix}{\namepa‌​rtsuffix}}

%\newcommand{\smiley}{:)}

%\renewbibmacro*{index:name}[5]{%
%\usebibmacro{index:entry}{#1}%
%{\iffieldundef{usera}{}{\thefield{usera}\actualoperator}\mkbibindexname{#2}{#3}{#4}{#5}}}

% \newcommand{\noop}[1]{}

%remove for final
%\overfullrule=1mm

\newcommand{\tobi}[2]}}
\renewcommand{\S}[1]{\tobi{#1}{\textsc{*}}}

% this volume references
% puts: [this volume]
% already defined: \citetv
%\newcommand{\citepv}[1]{(\citeauthor{#1} \citeyear*{#1} [this volume])}
\newcommand{\citealtv}[1]{\citeauthor{#1} \citeyear*{#1} [this volume]}

%parentheses around example number
\newcommand{\pref}[1]{(\ref{#1})}

% in-text examples

\newcommand{\lnex}[1]{\textit{#1}} %target lang word
\newcommand{\lnlit}[1]{(lit.: `#1')} %literal reading
\newcommand{\lnlat}[1]{(#1)} % latinization
\newcommand{\lntrans}[1]{`#1'} %translation
\newcommand{\lnexl}[2]%
{\lnex{#1}{} \lnlat{#2}} % ex with latinization
\newcommand{\lnexlat}[3]{\lnex{#1}{} \lnlat{#2}{} \lntrans{#3}} % ex with latinization and tranl.

%ch01
\newcommand{\co}[1]{\mbox{\textbf{#1}}}

%ch09

\newcommand{\cyrbulg}[1]{\begin{otherlanguage*}{bulgarian}#1\end{otherlanguage*}}


%ch10
\newcommand{\nlp}{{\small NLP}}
\newcommand{\mwe}{{\small MWE}}
\newcommand{\rae}{{\small RAE}}
\newcommand{\lvc}{{\small LVC}}
\newcommand{\pos}{{\small P}o{\small S}}
%\newcommand{\todo}[1]{ \textcolor{red}{#1} }

%\renewcommand{\labelenumi}{\theenumi}
%\ainamefmt{{vv}{ll}{, ff}{, jj}} % fullname

\newcommand{\biberror}[1]{{\color{red}#1}}

\newcommand{\osenovaitem}{--~} 
  %% hyphenation points for line breaks
%% Normally, automatic hyphenation in LaTeX is very good
%% If a word is mis-hyphenated, add it to this file
%%
%% add information to TeX file before \begin{document} with:
%% %% hyphenation points for line breaks
%% Normally, automatic hyphenation in LaTeX is very good
%% If a word is mis-hyphenated, add it to this file
%%
%% add information to TeX file before \begin{document} with:
%% %% hyphenation points for line breaks
%% Normally, automatic hyphenation in LaTeX is very good
%% If a word is mis-hyphenated, add it to this file
%%
%% add information to TeX file before \begin{document} with:
%% \include{localhyphenation}
\hyphenation{
    Beck-man
    Ngu-yen
    back-chan-nel
    back-chan-nels
    mo-not-o-nous
    ste-reo-typ-i-cal
}

\hyphenation{
    Beck-man
    Ngu-yen
    back-chan-nel
    back-chan-nels
    mo-not-o-nous
    ste-reo-typ-i-cal
}

\hyphenation{
    Beck-man
    Ngu-yen
    back-chan-nel
    back-chan-nels
    mo-not-o-nous
    ste-reo-typ-i-cal
}
 
  \togglepaper[1]%%chapternumber
}{}

\begin{document}
\begin{otherlanguage}{french}
\maketitle

Ce n’est certainement pas un hasard si le champ scientifique de l’acquisition des langues secondes se structure au XXème siècle quelques années après que les pays européens ont massivement fait appel à de la main d’œuvre étrangère pour les besoins de leurs appareils de production. Ces nouvelles situations de contact de langue et de bi- voire multilinguisme interpellent les linguistes : ces parlers de migrants ne sont-ils pas un objet linguistique digne d’intérêt et si c’est le cas, comment les décrire? Dès les premières études linguistiques des parlers de travailleurs migrants installés en Allemagne, en France ou aux Etats-Unis, un parallèle est établi à la fin des années 1960 et au début des années 1970 entre d’une part les propriétés des discours produits dans ces contextes et celles des pidgins et créoles déjà décrits ainsi qu’entre le processus d’acquisition d’une langue seconde et celui de pidginisation, créolisation, voire décréolisation (\citealt{Clyne1968, Schumann1974, Véronique1979}).

A la fin des années 1960, \citet{Clyne1968} voit en effet dans l’allemand parlé par les travailleurs «~invités~» ({\textit{Gastarbeiter}}) une langue pidginisée, car simplifiée et réduite à un simple moyen de communication. Partant de 15 entretiens enregistrés sur cassette avec des hommes et des femmes multilingues de langue première espagnole, grecque, turque, slovène vivant dans la région de Bonn et utilisant l’allemand au travail mais aussi dans les commerces, administrations et transports, Clyne décrit, illustre et analyse les propriétés de leur allemand, à savoir des phrases souvent constituées d’un seul mot, une interprétation fortement dépendante du contexte, de nombreux adverbes ou locutions adverbiales, des phrases dépourvues de verbes, l’omission des pronoms, des prépositions, un usage partagé, même si dans des proportions variables, de verbes à la forme infinitive là où une forme conjuguée est attendue, absence de proposition subordonnée... autant de traits qui soulignent selon l’auteur à quel point les usages habituellement décrits dans les revues de linguistique sont redondants et montrent {\textit{a contrario}} comment quelques morphèmes et lexèmes suffisent pour se faire comprendre. Même si aucun créole à base allemande ne s’est développé dans les colonies allemandes, Clyne conclut que la langue qu’il décrit est bien une langue pidginisée au sens où l’entend \citet{Bloomfield1933}, à savoir le résultat d’une double simplification, lorsqu’elle est adressée au travailleur par le contre-maitre et lorsqu’elle est apprise par le travailleur. Cette étude qui contient en germe l’hypothèse de ce qui sera plus tard appelé la Variété de Base (\citealt{KleinPerdue1997}), va fortement inspirer la recherche longitudinale sur l’allemand pidginisé de Heidelberg ({\textit{Heidelberger Pidgindeutsch}}, HPD)~qui s’attachera non seulement à décrire mais aussi à observer le développement de l’allemand des travailleurs sur deux ans, et à le mettre en lien avec la variété d’allemand environnante parlée par leurs collègues de langue première à l’usine et dans la ville de Heidelberg ainsi qu’avec les conditions sociales d’apprentissage et plus globalement d’existence de ces immigrés venus d’Espagne, du Portugal ou d’Italie (\citealt{KleinDittmar1979} ; \citealt{AhrenholzRost-Roth2021}). \citet{Véronique1980} qui mènera à la suite de \citet{MorslyVasseur1976} de semblables observations auprès d’apprenants arabophones marocains à Marseille ouvre sa discussion de l’hypothèse de la pidginisation/décréolisation qu’il emprunte aux travaux de \citet{Schumann1974} en citant la communication de Dittmar sur l’allemand pidginisé de Heidelberg que celui-ci présente au colloque \textit{Acquisition d’une langue étrangère : perspectives de recherche} d’avril 1979 à Vincennes. La créolistique a ainsi fourni aux jeunes chercheurs européens attirés par l’émergence de variétés secondes un appareil descriptif et explicatif des divers phénomènes observés. {Partant de là,} {\citet{Véronique1986a} dresse un état des débats à propos des notions de pidginisation et de créolisation. Il établit  à cette occasion un inventaire des traits considérés comme typiques des pidgins pour montrer qu’ils sont tous vérifiés dans des interlangues d’apprenants arabophones de français. Il conclut à l’identité des fonctionnements linguistiques de certains états d’interlangue et de ce que l’on nomme des pidgins, et à l’identité des processus linguistiques et cognitifs qui assurent leurs réalités. Au vu des recherches en acquisition et des travaux sur les langues en contact, il serait judicieux selon lui de considérer que la pidginisation et l’éventuel pidgin qui en résulte, ne représente, dans une perspective dynamique, qu’une phase de l’acquisition d’une langue étrangère.}

Si le dialogue entre recherches acquisitionnelles et créolistiques est consubstantiel des débuts de la discipline naissante appelée plus tard en France Recherches sur l’Acquisition des Langues Secondes, siglée RAL (Véronique, 1992a), rares sont néanmoins les chercheurs qui poursuivent ces analyses comparatives entre l’émergence des créoles parlés dans les formations esclavagistes mises en place par les empires coloniaux européens dans l’océan indien ou les Antilles à partir du XVIème siècle et les parlers de travailleurs immigrés déplacés, embauchés et relégués dans les zones périurbaines des villes françaises ou allemandes depuis la deuxième moitié du XXème siècle. Daniel Véronique fait figure d’exception par le dialogue continu qu’il instaure entre ces deux domaines, pour comprendre globalement les phénomènes de glottogenèse ou émergence d’une nouvelle langue à partir d’une situation de contact linguistique, qu’il s’agisse d’un créole\footnote{{Parmi les débats concernant l’émergence des langues créoles, la question posée par D. \citet[27]{Véronique2005Interlangues} est la suivante : « Est-ce que l’émergence d’une langue nouvelle, le créole en l’espèce, provient d’un apprentissage imparfait de la langue coloniale dominante, ou est-ce le résultat de la créativité linguistique humaine élaborant un outil de communication interethnique ? ». Cette question maintes fois formulée relève selon lui d’un double malentendu, un malentendu sémantique et une incompréhension de la nature et du cours des activités socio-cognitives de production, d’interaction et d’appropriation langagières et linguistiques. Ainsi, selon lui, tous les courants de recherche en acquisition s’accordent sur le fait que la production et l’appropriation linguistiques constituent, dans le cas où un nouvel idiome est en cause, « les deux faces d’un même processus » (ibid.), ou que «~l’emploi linguistique contextualisé et l’acquisition ne sauraient être dissociés » (ibid). En d’autres termes, il serait vain de vouloir opposer la création d’un outil de communication linguistique inédit à l’apprentissage d’une langue nouvelle car les mêmes processus sociocognitifs y sont à l’œuvre.}} ou d’une variété seconde (\citealt{Véronique1980,Véronique1989,Véronique2005Interlangues,Véronique2009a,AslanovKriegelVéronique2022}).

La particularité de l’approche comparative des créoles et des parlers de migrants\footnote{Variétés, lectes, interlangues, dialectes, parler véhiculaire, langue seconde, la terminologie n’est pas neutre et donne lieu très tôt à de nombreuses discussions, voir notamment \citet{Véronique1980}. Néanmoins, nous faisons le choix de considérer que ces termes sont interchangeables et désignent le système complexe de l’apprenant à un moment donné, explicable par des facteurs à la fois environnementaux, sociaux et cognitifs.} envisagée par le chercheur est d’emblée et de façon constante de ne pas séparer la description linguistique des parlers des conditions sociales et historiques de leur émergence, ni le niveau individuel du niveau collectif, autrement dit tenir ensemble les locuteurs et locutrices qui incarnent les langues et les communautés qui font figure d’«~agence de socialisation et vecteur du bilinguisme~» (\citealt{Véronique1979} : 43). Daniel Véronique définit ainsi trois domaines d’étude conjoints, «~les conditions de l’apprentissage, la technologie de l’apprentissage et les productions en langue cible~» (ibid.). Cet intérêt précoce pour la socialisation langagière et l’étude des pratiques langagières complexes situées en divers lieux est fortement précurseur du «~tournant social~» (\citealt{Block2003,TheDouglasFirgroup2016}) 

Il faut néanmoins aussi souligner l’immense travail descriptif qui a permis de visibiliser les parlers d’apprenants dans le champ des sciences du langage, et en particulier ceux d’Abdelmalek, Zahra et Abdessamad, et de leur accorder, plus que quelques lettres de noblesse, un véritable statut d’objet linguistique (Giacomi, Stoffel et \citealt{Véronique2000,VéroniquePorquier1986c,Véronique1987,Véronique2009a,Véronique2010,Véronique2013d, Véronique2014b}). De même, on ne peut manquer de souligner les apports importants et prépondérants du chercheur dans la description grammaticale des créoles français tant du point de vue diachronique de leur émergence et de leur évolution (\citealt{Véronique1997b,Véronique1999,Véronique2000b,Véronique2003b,Véronique2007c,Véronique2007d,Véronique2012a,Véronique2013a,Véronique2013b,Véronique2014a,Véronique2017dynamique,Véronique2021d}) que du point de vue synchronique de leurs propriétés et de leur fonctionnement (\citealt{FonSingVéronique2007,Véronique1985,Véronique1986b,Véronique1992b,Véronique1993,Véronique1996,Véronique1997a,Véronique2000a,Véronique2001,Véronique2003a,Véronique2006,Véronique2007b,Véronique2009b,Véronique2012b,Véronique2013c,Véronique2013d,Véronique2020,Véronique2021c}).

Dans ce cadre d’étude conjointe des propriétés des parlers et des processus d’acquisition et de socialisation langagières, la classe de langue est une autre marge linguistique mais pas un lieu à part. Elle est, au contraire, un lieu parmi d’autres où sont mises en place des techniques d’apprentissage culturellement et historiquement déterminées et où se nouent ainsi des interactions hautement signifiantes et particulièrement dignes d’intérêt pour quiconque s’intéresse au processus situé d’appropriation. Véronique n’a pas cessé de rappeler à la RAL sa dimension, voire sa vocation, sociale : l’apprentissage ne peut pas seulement être évalué en termes de rendement mais aussi en termes d’échanges et de transformations des individus au contact de l’altérité. Grâce aux nombreux travaux de Véronique (1983, 2005, 2007, 2017a) et à ses efforts constants pour instaurer un dialogue entre les recherches didactiques et acquisitionnelles, on peut considérer qu’il existe aujourd’hui dans le domaine francophone une discipline intermédiaire prometteuse entre ces deux domaines de recherche, illustrée par quelques études de ce volume.

Ce que rappelle aussi la didactique des langues, c’est que les langues n’existent et ne peuvent être étudiées que parce qu’elles sont apprises et transmises. L’étude en temps réel ou dans le travail historique d’un de leurs contextes d’émergence, qu’il s’agisse d’un établissement insulaire, d’une grande ville française au bord de la Méditerranée ou d’un cours de français à l’université, est aussi l’étude d’un contexte de transmission et de transformation. Cet ouvrage permet, nous l’espérons, de rendre compte de la richesse de cette approche globale des transmissions langagières et des processus d’emmétamorphose qui les accompagnent.\footnote{L’emmétamorphose est le terme proposé par le traducteur de \citet{Rosa2022} pour le néologisme \textit{Anverwandlung} qui désigne une appropriation complexe, «~l'appropriation d'un fragment de monde, d'une matière, de telle sorte que l'on se transforme soi-même à son contact».} 

Les chapitres de cet ouvrage témoignent de contacts, compagnonnages et héritages intellectuels fructueux et résonnants au sens où pour beaucoup d’entre nous Daniel Véronique a fait «~crépiter nos consciences~» (Rosa, 2022). Sans étonnement, l’alpha et l’oméga des chapitres de cet ouvrage sont consacrés à deux études en créolistique. Le {\textbf{premier} \textbf{chapitre}} est une étude linguistique et sociolinguistique inédite d’un extrait du journal de bord d’un colon français qui tient une exploitation de noix de coco sur l’île de Galéga située entre l’île Maurice et les Seychelles. Dans son analyse de la reproduction de l’allocution que le maître adresse aux esclaves en créole suite à la promulgation de leur affranchissement et changement de statut par l’abolition de l’esclavage en 1835, Sibylle Kriegel apporte une contribution originale aux «~comparaisons grammaticales entre les langues créoles de l’océan indien~» que \citet{Véronique2021Pour} appelle de ses vœux pour arriver à une meilleure compréhension du lien entre les conditions socio-historiques de rassemblement d’esclaves dans les plantations, les situations de contact linguistique et de domination de langues et le développement de certains types de créole. Dans le {\textbf{second} \textbf{chapitre}}, Benazzo, Andorno et Dimroth revisitent la notion centrale de Variété de Base (VB) définie dans \citet{KleinPerdue1997} comme une protolangue ou variété linguistique permettant de communiquer {\textit{a minima}} dans le monde social, comme celles parlées par des travailleurs ou réfugiés étudiés dans la recherche européenne {\textit{The ecology of adult langage acquisition (EALA)}}, 1981-88. Elles se demandent si ce concept est pertinent pour rendre compte du développement langagier d’autres profils d’apprenants dans d’autres situations, en termes de contact linguistique et de bagage socio-culturel. Les autrices discutent ainsi d’études qui examinent des productions d’apprenants au prisme de la VB en isolant quelques paramètres : l’alphabétisation des locuteurs et leur conscience métalinguistique, leur exposition à un enseignement guidé et un guidage de l’attention sur les propriétés formelles des langues, et les propriétés typologiques des langues de l’environnement. Ce chapitre débouche sur un programme de recherche en lien avec la VB. Le {\textbf{chapitre} \textbf{suivant}} s’interroge aussi sur la spécificité des profils d’apprenants en comparant le positionnement social de deux populations d’apprenants et deux types d’ancrage dans la société d’accueil: des réfugiés syriens arabophones installés pour une longue durée en France dans les années 2011-2016 et des étudiants français effectuant un court séjour Erasmus en Irlande et au Royaume-Uni. Les autrices, Inès Saddour et Pascale Leclercq, étudient à la fois leur réseau de sociabilité et les marques discursives d’auto-positionnement par rapport aux ressortissants de leur pays, aux autres réfugiés ou étudiants ou au groupe dominant des ressortissants du pays d’installation. En dépit de profils socio-culturels, contextes migratoires et attentes différents par rapport à la société de résidence, elles montrent des tendances communes mais aussi des trajectoires de socialisation différentes. Dans le {\textbf{quatrième} \textbf{chapitre}}, Ågren et Granfeldt analysent les attitudes des élèves suédois envers le français dans le cadre transdisciplinaire proposé par le \citet{TheDouglasFirgroup2016}, autrement dit en embrassant la complexité de l’écosystème éducatif de l’élève suédois. Ils montrent ainsi comment sont imbriqués dans l’apprentissage d’une langue trois niveaux, c'est-à-dire individuel (micro), institutionnel (meso) et sociétal (macro) et comment les changements de politique linguistique éducative que connaît la Suède depuis son entrée dans l’Union Européenne ont réduit l’offre de français langue étrangère (FLE) et modifié durablement les attitudes des élèves. Partant aussi d’une définition du processus acquisitionnel comme multidimensionnel, le {\textbf{chapitre} \textbf{cinq}} rapporte aussi une étude, cette fois longitudinale, menée auprès d’adolescents apprenants de l’italien L2 et analyse la complexité syntaxique de leurs appels téléphoniques comparés à celle de leur récits, tout en comparant cette complexité avec celle d’un groupe contrôle. Pallotti montre qu’il ne suffit pas de mesurer la complexité, encore faut-il savoir l’interpréter en fonction de la situation d’énonciation. Cette réflexion s’avère d’autant plus importante dans l’enseignement, notamment dans les activités d’évaluation. L’importance du contexte énonciatif est également soulignée par Catherine Felce dans le {\textbf{chapitre}  \textbf{six} } qui présente son étude longitudinale de l’usage de l’allemand langue nouvelle par six étudiants grands débutants et sa proposition didactique de familiarisation avec des options de linéarisation dites complexes dans des modèles syntaxiques. Elle propose ainsi une alternative constructionniste à l’apprentissage par règles de la syntaxe de l’allemand et montre comment l’exposition à des constructions d’emblée complexes conduit à une appropriation de ces constructions et au développement d’un processus graduel de grammaticalisation imputable aux mécanismes cognitifs généraux d’analogie, catégorisation et schématisation mis en œuvre par les apprenants dans des situations pertinentes d’exposition et d’usage. Le {\textbf{chapitre} \textbf{sept}} est, comme le chapitre 5, une étude comparative de l’usage dans deux tâches distinctes du passé composé et de l’imparfait par deux groupes d’enfants bilingues, simultanés vs successifs. Kihlstedt montre, à travers deux types de tâches, les points communs et divergences entre les bilingues simultanés et successifs : elle relève ainsi un même usage discursif et une même aspectualisation morphologique des temps du passé, audible dans une tâche de production spontanée mais des différences de rythme et surtout de parcours dans une tâche plus contrainte où les enfants bilingues successifs, ayant appris le français après le suédois se comportent davantage comme des adultes en français langue seconde que comme des enfants en langue première. Dans le \textbf{huitième} \textbf{chapitre}, Watorek, Rast et Trévisiol présentent une étude exploratoire dans laquelle elles observent les effets de deux technologies distinctes de manipulation et présentation des données d’exposition au polonais en classe de langue, une mise en relief des morphèmes flexionnels des noms en polonais L2 opposée à son absence de surlignement et l’engagement des étudiants dans des tâches de communication signifiante. Les résultats sont discutés à la lumière de ceux de la recherche translinguistique VILLA qui a suivi ainsi que de l’avènement possible d’un champ de recherche consacré à l’élaboration et l’analyse des effets de dispositifs didactiques potentiellement acquisitionnels. Dans le {\textbf{chapitre} \textbf{suivant}}, Jacopo Saturno remet sur le métier la question du transfert en s’intéressant à l’absence non moins signifiante de transfert dans l’acquisition du polonais L2 par des apprenants ayant différentes langues premières slaves.  {L’auteur avance} comme facteurs explicatifs deux processus bien distincts~dans l’acquisition du cas accusatif : soit le transfert positif, soit une stratégie de surextension de la forme correspondant au cas nominatif pour les noms non-viriles au pluriel.
Pour ne pas conclure, un {\textbf{dernier} \textbf{et} \textbf{dixième} \textbf{chapitre}} de Alain Kihm, propose une étude du marquage du pluriel en trois langues dont deux créoles, le français parlé, le Fa d’Ambô, et le créole réunionnais, en faveur d’une autre lecture du morphème du pluriel Z. L’auteur explique les différences de marquage morphologique du pluriel entre d'un côté le français parlé contemporain et de l'autre le Fa d’Ambô et le créole réunionnais par des développements historiques distincts; il propose d'y voir l'illustration de deux types d'évolution linguistique, le changement linguistique ordinaire et la créolisation.

\printbibliography[heading=subbibliography,notkeyword=this]
\end{otherlanguage}
\end{document}
