\documentclass[output=paper,            colorlinks, citecolor=brown            		  ]{langscibook}
\ChapterDOI{10.5281/zenodo.10280612}
    
\author{Jacopo Saturno\orcid{}\affiliation{University of Bergamo}}

\title{Transfer (or lack thereof) and the accusative case in L2 Polish}

\abstract{This paper discusses the role of transfer in the L2 acquisition of the Polish accusative case by speakers of closely related languages (as L1 or L2). Although Slavic languages have been somewhat neglected so far in terms of empirical SLA research, the majority of L2 Polish learners in present-day Poland are indeed speakers of Slavic languages, which makes this language family a privileged research ground for matters related to transfer. Following a review of the theoretical approaches deemed most relevant for the analysis (mainly Processability Theory and the Learner Variety Approach), the paper presents three empirical studies conducted by the author. The discussion will compare transfer-based accounts of the interlanguage structure to explanations relying on universal acquisition principles.
\keywords{L2 Polish, East Slavic languages, transfer, intercomprehension, universal acquisition tendencies.}}

\IfFileExists{../localcommands.tex}{
  \addbibresource{../localbibliography.bib}
  \usepackage{langsci-optional}
\usepackage{langsci-gb4e}
\usepackage{langsci-lgr}

\usepackage{listings}
\lstset{basicstyle=\ttfamily,tabsize=2,breaklines=true}

%added by author
% \usepackage{tipa}
\usepackage{multirow}
\graphicspath{{figures/}}
\usepackage{langsci-branding}

  
\newcommand{\sent}{\enumsentence}
\newcommand{\sents}{\eenumsentence}
\let\citeasnoun\citet

\renewcommand{\lsCoverTitleFont}[1]{\sffamily\addfontfeatures{Scale=MatchUppercase}\fontsize{44pt}{16mm}\selectfont #1}
   
  %% hyphenation points for line breaks
%% Normally, automatic hyphenation in LaTeX is very good
%% If a word is mis-hyphenated, add it to this file
%%
%% add information to TeX file before \begin{document} with:
%% %% hyphenation points for line breaks
%% Normally, automatic hyphenation in LaTeX is very good
%% If a word is mis-hyphenated, add it to this file
%%
%% add information to TeX file before \begin{document} with:
%% %% hyphenation points for line breaks
%% Normally, automatic hyphenation in LaTeX is very good
%% If a word is mis-hyphenated, add it to this file
%%
%% add information to TeX file before \begin{document} with:
%% \include{localhyphenation}
\hyphenation{
affri-ca-te
affri-ca-tes
an-no-tated
com-ple-ments
com-po-si-tio-na-li-ty
non-com-po-si-tio-na-li-ty
Gon-zá-lez
out-side
Ri-chárd
se-man-tics
STREU-SLE
Tie-de-mann
}
\hyphenation{
affri-ca-te
affri-ca-tes
an-no-tated
com-ple-ments
com-po-si-tio-na-li-ty
non-com-po-si-tio-na-li-ty
Gon-zá-lez
out-side
Ri-chárd
se-man-tics
STREU-SLE
Tie-de-mann
}
\hyphenation{
affri-ca-te
affri-ca-tes
an-no-tated
com-ple-ments
com-po-si-tio-na-li-ty
non-com-po-si-tio-na-li-ty
Gon-zá-lez
out-side
Ri-chárd
se-man-tics
STREU-SLE
Tie-de-mann
} 
  \togglepaper[3]%%chapternumber
}{}

\begin{document}
\AffiliationsWithoutIndexing{}
\maketitle 



\section{Introduction: Genesis and rationale of the study}

My \textit{hommage} to Daniel Véronique is a discussion on transfer in the acquisition of L2 Polish by speakers of other Slavic languages, either as L1 or L2. With the partial exception of Russian, Slavic languages have been somewhat neglected by SLA research so far. Nevertheless, they are now likely to receive greater attention thanks to the notable role of L2 Polish in two recent, large-scale historical events. The first is the massive influx of job-seekers that entered Poland from the neighbouring Slavic-speaking countries following the country’s accession to the European Union in 2004, which in turn produced rapid economic growth over the subsequent two decades. The second event is the 2022 invasion of Ukraine. Along with other countries, Poland welcomed a great number of refugees, most of whom were speakers of East Slavic languages, i.e. Belarusian, Russian, and Ukrainian. Even before the invasion, East Slavic Speakers (ESSs from now on) represented the vast majority of foreigners in Poland (\citealt{GlownyUrzadStatystyczny2020a,GlownyUrzadStatystyczny2020b, GlownyUrzadStatystyczny2021}). Both job seekers and refugees are likely to remain in the country for at least some time, and will thus need to integrate into Polish society and acquire the local language. This situation offers SLA research the opportunity to study an acquisition process that concerns a great number of learners and is at the same time remarkably uniform and rather peculiar due to the proximity of the newcomers’ L1s to the target language. Conversely, it is to be hoped that the insights of SLA research may facilitate the refugees’ integration into the host country through more effective language teaching.

The paper is organised as follows. \sectref{sec:saturno:1} presents an overview of research on transfer. In \sectref{sec:saturno:2}, four Polish grammatical structures are described in detail and compared to their East Slavic counterparts in order to provide the necessary background information. The available literature on the acquisition of Slavic L2s is then reviewed. \sectref{sec:saturno:3} summarises three published studies conducted by the author on the acquisition of L2 Polish by learners with knowledge of Russian. After a description of the shared methodological points, the specificities of each study are presented along with the main results. The discussion in section 4 summarises, compares and comments on the findings of the three studies in light of general reflections on language transfer. 

\section{Theoretical foundations: From contrastive analysis to universal acquisition paths}
\label{sec:saturno:1}

In order to outline the theoretical background of the chapter, the present section will provide a general overview of the debate opposing transfer-based accounts of interlanguage structure to research on near-universal acquisition sequences. The focus is on Polish nominal morphology, with particular regard to the accusative case, a notoriously difficult structure to acquire across languages (\citealt{Artoni2020, BatenVerbeke2020, Magnani2020, Saturno2020a}). For reasons of space, it will not be possible to discuss general theoretical models.\footnote{Interested readers are referred to \citet{Bardel2019}.}

With no claim to exhaustiveness or chronological accuracy, the starting point of this discussion is the notion of \textit{contrastive analysis}, traditionally attributed to \citet{Lado1957}, although its origin arguably dates back to earlier work (see \citealt{Odlin2016}). In the behaviourist view \citep{Skinner1957}, language acquisition is seen as the result of the transfer of L1 “habits” into the target language. Differences and similarities between the L1 and the L2 are believed to respectively hamper and facilitate acquisition. Contrastive analysis aimed to identify potentially problematic points, based on the assumption that “those elements which are similar to [the learner’s] native language will be simple for him, and those elements that are different will be difficult” \citep[2]{Lado1957}. However, it soon became clear that predicted errors did not always occur in empirical data, while at the same time other types occurred that were not predicted. Most importantly, the same errors were made by learners irrespective of their L1. The  \textit{morpheme studies} of the 1970s (see \citealt{DulayEtAl1982} for a synthesis) focussed on this last point. Investigating the acquisition of inflectional morphemes in L2 English, \citet[256]{DulayBurt1973} concluded that there seemed to be “a common order of acquisition for certain structures” independently of the learner’s L1 (\figref{fig:saturno:1}).

  
\begin{figure}
%\includegraphics[width=\textwidth]{figures/SaturnoLANGSCI-img001.png}
\includegraphics[width=\textwidth]{figures/saturno-figure-1.pdf}

\caption{Acquisition sequences for Spanish- and Cantonese-speaking learners of L2 English (\citealt{DulayBurt1974}, cited in \citealt[206]{DulayEtAl1982})}
\label{fig:saturno:1}
\end{figure}

Over time, various attempts have been made to reconcile and integrate the positions insisting on the role of language transfer with those emphasizing the common features of the acquisition process. To exemplify, it has been shown that in general, learners initially process content words before anything else \citep{VanPatten2004Input}. Nonetheless, the L1 may lead them to rely more heavily on cues that are particularly important in their L1. In a set of studies on the acquisition of temporal reference, \citet{EllisSagarra2010b, EllisSagarra2010a, EllisSagarra2011} and \citet{SagarraEllis2013} found that speakers of morphologically poor languages such as Chinese and English tended to rely more heavily on lexical cues than did speakers of morphologically more complex languages, such as Spanish, Russian, or Romanian. 

Contrastive analysis is still a popular prism to analyse language errors, especially when the learners’ L1 is closely related to the target language. Most research on the acquisition of L2 Polish by speakers of Slavic languages indeed adopts this perspective (cf. \citealt{Saturno2022b} for a review), attributing most errors to the influence of the L1, whereas some may in fact have an alternative explanation involving universal acquisition tendencies.

A further debated topic in the approaches that admit a role for transfer is whether it can originate exclusively from the L1 or from other L2s as well. The latter position is supported by a growing body of research (see \citealt{Bardel2019} for a synthesis). In a study on the placement of negation in L3 Italian by learners with and without knowledge of other Romance languages, \citet{Bardel2006} demonstrated that the highest performance was achieved by learners with knowledge of L2 Spanish, whose negation system is closest to that of Italian.  \citet{RothmanCabrelliAmaro2010} found evidence of transfer from the L2 in the acquisition of L3 Italian and L3 French null subjects by L1 English learners with knowledge of L2 Spanish. Some go so far as to claim that even within the same language family, L2 influence dominates over L1 influence (\citealt{Bohnacker2006,BardelFalk2007}). 

The next section will narrow down this general discussion on transfer in L2 acquisition to the case of the Slavic language family.

\section{Acquisition of Slavic languages}\label{sec:saturno:2}

This section presents a selection of the available literature on the acquisition of Slavic languages by learners with and without knowledge of related languages. Before turning to the review, however, it seems worthwhile to first highlight the main similarities between the members of this language family that are directly relevant for the present discussion, in order to illustrate in what respects the acquisition process may be supposed to benefit from transfer.

\subsection{Proximity among Slavic languages}\label{sec:saturno:2.1}

Polish and East Slavic languages belong to the same Slavic branch of Indo-Eu\-ro\-pe\-an languages, although to different subgroups, Polish being a West Slavic language. A substantial proportion of words and structures can be regarded as mutually intelligible even by speakers without linguistic expertise. 

The proximity of Polish to East Slavic is illustrated in \REF{ex:saturno:2}, where the Russian and Ukrainian translation of the Polish sentence in \REF{ex:saturno:1} are presented. Since the texts in \REF{ex:saturno:2} are perfectly parallel, the morphological glosses and the translations are presented only once.

\ea\label{ex:saturno:1}
\gll W  lat-ach  pięćdziesiąt-ych   lata-li-śmy   nad   Europ-ą.\footnotemark\\
 in  year-\textsc{loc.pl}  fiftieth-\textsc{loc.pl.nonvir}  fly-\textsc{pst.vir-1pl}  over  Europe-\textsc{ins.sg}\\ 
\footnotetext{\url{https://www.lot.com/pl/pl/odkrywaj/o-lot/historia-lotu}}
\glt   ‘In the 1950s we were flying over Europe’.\footnotemark \jambox*{(L1 Polish)}
\footnotetext{\url{https://www.lot.com/pl/en/explore/about-lot/lot-history}}
\z

\ea\label{ex:saturno:2} 
\glll V  pjatidesjat-ych  god-ach    my  lete-li  nad  Evrop-oj.\footnotemark[4]\\
  U  pʺjatydesjat-ych  rok-ach    my  lita-ly  nad Jevrop-oju.\footnotemark[5]\\
  in  fiftieth-\textsc{loc.pl.inanim}  year-\textsc{loc.pl}    we:\textsc{nom}  fly-\textsc{pst.} \textsc{1pl}  over  Europe-\textsc{ins.sg}\\ 
\glt  ‘In the 1950s we were flying over Europe.’\footnotemark[6] \jambox*{(L1 Russian and L1 Ukrainian, in this order)}
\z
\footnotetext[4]{\url{https://www.lot.com/ru/ru/explore/about-lot/lot-history}}
\footnotetext[5]{\url{https://www.lot.com/ua/uk/explore/about-lot/lot-history}}
\footnotetext[6]{{\url{https://www.lot.com/pl/en/explore/about-lot/lot-history}}}
\stepcounter{footnote}\stepcounter{footnote}\stepcounter{footnote}

Some words are identical in the three languages in terms of meaning, phonology and use. To exemplify, \textit{nad} is pronounced /nad/, means ‘over’ and is followed by the instrumental case in all the languages considered. Further, intricate relations at times involve words that within a given text may not appear to be directly related, but may be intelligible in light of the whole L1 system. For speakers of Russian, the lexical morpheme in Ukranian \textit{rok-ach} ‘year\textsc{{}-loc.pl}’ may not be recognizable or correctly interpretable. In contrast – despite the fact that it is not related to its direct Russian counterpart \textit{god-ach} ‘year\textsc{{}-loc.pl}’ – the Polish form \textit{lat-ach} ‘year\textsc{{}-loc.pl}’ will probably cause no particular difficulty thanks to the existence in Russian of the form \textit{let} ‘year:\textsc{gen.pl}’, an allomorph of the lexeme \textit{god} ‘year’ comprised in the text. As demonstrated by \textit{{}-ach} ‘\textsc{loc.pl}’ in the preceding examples, an inflectional ending may well be recognizable even if the lexical morpheme is not. Finally, identifiable lexemes may be marked by opaque inflectional endings, like Pol. \textit{{}-ą} in \textit{Europą} ‘Europe-\textsc{ins.sg}’ for speakers of East Slavic. In fact, grammatical meaning is often deducible from other clearly recognisable elements, like the preposition \textit{nad} ‘over’. 

\subsection{Comparative descriptions of a few structures of interest}
\label{sec:saturno:2.2}

This section presents a comparative description of the accusative case in Polish (the target language of the studies presented in \sectref{sec:saturno:3}) and East Slavic languages (the learners’ L1), here illustrated using Russian.\footnote{This section only presents the information that is deemed relevant for the present discussion. For more detailed information on the languages in question, see \citet{Comrie2003}.} The target structure is of interest because of its slightly different realization in the languages considered. As will be shown in greater detail in \sectref{sec:saturno:2.2}, the Polish structure differs from its East Slavic counterpart either in the morph instantiating the accusative morpheme (\textit{{}-ę} vs. \textit{{}-u} in the \textsc{acc.sg}) or in the Differential Object Marking pattern triggered by the noun semantics (\textsc{acc.pl} \textsc{=} \textsc{nom.pl} vs. \textsc{acc.pl} \textsc{=} \textsc{gen.pl}). 

Before moving further, the choice of Russian as the representative of East Slavic languages requires clarification, since the most numerous group of L2 Polish learners in Poland is represented by citizens of Ukraine. First, Russian is a widely studied foreign language and functioned as the vehicular language of central Asia and Eastern Europe, where it is still widely represented in popular culture and the media. Since the studies presented in this paper were designed to address not only Ukrainian speakers, but also non-Slavic learners of L2 Russian, this language seemed the natural choice. Second, Ukrainians have a varied language repertoire. In the last official census, 67.5\% of respondents declared themselves speakers of Ukrainian, 29.6\% of Russian (\citealt{StateStatisticsCommitteeofUkraine2001}). In a more recent survey, Ukrainian and Russian were indicated as their mother tongue by respectively 63\% and 35\% of respondents \citep{Kantar2019}, with significant variation by communicative context, respondent age and area of residence, which may suggest a situation of bilingualism, or even diglossia. As a general note, survey data should be handled with care, since some responses may in fact result from national self-identification rather than actual language use. Finally, there exists a continuum of Russian-Ukrainian and Russian-Belarusian mixed varieties, respectively known as \textit{suržyk} (\citealt{Bilaniuk2004, Danylenko2016, DelGaudio2018}) and \textit{trasjanka} (\citealt{Woolhiser2014, Hentschel2017, DelGaudio2018}). 

\subsubsection{Morphological expression of subject and object}
\label{sec:saturno:2.2.1}
With the exception of Bulgarian and Macedonian, Slavic languages share a rich system of nominal inflectional morphology, expressing the categories of gender, number, and case. Although SVO is the predominant word order in Polish (\citealt{Siewierska1993, Dryer2013order}), the rich morphological system of this language in principle makes it possible to freely manipulate word order for pragmatic purposes. 

All the studies presented in section 3 are to some extent concerned with the opposition between the markers of the subject and object functions within the paradigm of feminine nouns ending in \textit{{}-a}, e.g. \textit{Anna} in \REF{ex:saturno:3}.\footnote{Proper nouns inflect like common nouns in the Slavic languages considered in the paper.} In Polish, the subject is expressed by the nominative case, instantiated by the ending \textit{{}-a} (/a/); the object function is expressed either by the accusative case in \textit{{}-ę} (/e/), if the verb is not negated, or by the genitive case in \textit{{}-y} or \textit{-i} (/ɨ/, /i/),\footnote{The choice of the allomorph depends on whether or not the final consonant of the stem is palatalised.} obligatory within the scope of negation \citep{Przepiórkowski2000}.

\ea\label{ex:saturno:3}  
    \ea\label{ex:saturno:3a}
    \gll Ann-a    lubi  ryb-ę.\\
    Anna-\textsc{nom}  likes  fish-\textsc{acc.sg}\\ 
    \glt ‘Anna likes fish.’ \jambox*{(L1 Polish)}

  \ex\label{ex:saturno:3b}  
  \gll Ann-a    nie  lubi  ryb-y.\\
    Anna-\textsc{nom}  not  likes  fish-\textsc{gen.sg}\\ 
   \glt  ‘Anna does not like fish.’ \jambox*{(L1 Polish)}
\z
\z

The morphological marking of the direct object in Russian differs in a few details. First, the accusative singular of feminine nouns in \textit{{}-a} ends in \textit{{}-u}, rather than \textit{{}-ę} \REF{ex:saturno:4a}; second, the accusative case may be used within the scope of negation \REF{ex:saturno:4b}, although the genitive is also possible \REF{ex:saturno:4c} depending on various factors, such as the semantics of the noun in question \citep{Harves2013}. The morpheme encoding the genitive singular is phonologically identical in Russian and Polish.

\ea\label{ex:saturno:4}
    \ea\label{ex:saturno:4a} 
    \gll Ann-a    ljubit  ryb-u.\\
    Anna-\textsc{nom}  likes  fish-\textsc{acc.sg}\\ 
   \glt  ‘Anna likes fish.’
   \jambox*{(L1 Russian)}

  \ex\label{ex:saturno:4b}  
  \gll Ann-a    nie  ljubit  ryb-u.\\
    Anna-\textsc{nom}  not  likes  fish-\textsc{acc.sg}\\ 
   \glt ‘Anna does not like fish.’
   \jambox*{(L1 Russian)}

    \ex\label{ex:saturno:4c} 
    \gll Ann-a    nie  ljubit  ryb-y.\\
    Anna-\textsc{nom}  not  likes  fish-\textsc{gen.sg}\\ 
   \glt ‘Anna does not like fish.’
   \jambox*{(L1 Russian)}
   \z
\z

\subsubsection{Grammatical gender and Differential Object Marking}
\label{sec:saturno:2.2.2}
Differential Object Marking\footnote{{Henceforth DOM.}} (\citealt{Bossong1998, Aissen2003}) patterns are found in both Polish and Russian, but in some cases may be triggered by a different semantic feature in the two systems. In the singular, the nouns of both languages are categorised into masculine, feminine and neuter. DOM only concerns masculine nouns, whose accusative form coincides with the nominative or the genitive depending on whether the referent is animate or inanimate.\footnote{{Animacy here cannot be considered an entirely semantic category due to the so-called “honorary animates”} {(\citealt[138]{SussexCubberley2006})}{, i.e. nouns that denote arguably inanimate referents, but inflect like animate nouns, e.g.} {\textit{trup}} {‘cadaver’,} {\textit{borowik ‘}}{mushroom of the genus Boletus}{\textit{’.}}} In the plural, both systems identify two genders; however, while Russian contrasts animate and inanimate entities, Polish envisages the so-called “virile” gender (Pol. \textit{męskoosobowy}, lit. ‘male personal’), comprising human, male, adult referents.\footnote{The term “virile” is commonly used in Polish linguistics {(e.g. \citealt{Brown1998, Janda1999})}{, though the alternative “masculine-personal” can also be found} {(\citealt{Rothstein2002, SussexCubberley2006})}{ For reasons of conciseness and unambiguousness, I will adopt the former term throughout the paper.}} This class contrasts with all other nouns, which are included in the “non-virile” gender (Pol. \textit{niemęskoosobowy} ‘not male personal’). In the animate and virile gender, the \textsc{acc.pl} coincides with the \textsc{gen.pl}, while in the non-virile and inanimate gender, the \textsc{acc.pl} is syncretic with the \textsc{nom.pl}. It follows that some entities, such as female human referents and some animals, are classified differently in the two languages, and therefore require different inflectional endings and agreement patterns \REF{ex:saturno:5}. The morphs instantiating the \textsc{nom.pl} and \textsc{gen.pl} are identical in the two languages (respectively \textit{ryb-y} ‘fish-\textsc{nom.pl}’ and \textit{ryb} ‘fish-\textsc{gen.pl}’). Shaded cells in \tabref{tab:saturno:1} highlight the paradigm forms requiring different endings in the two languages.  

\ea\label{ex:saturno:5}
    \ea\label{ex:saturno:5a}
    \gll Dzieci     karmią     złot-e        ryb-y.\\
    {child:\textsc{nom.pl}}  {feed:\textsc{pres.3pl}}  {golden-\textsc{acc.pl=nom.pl}}  {fish-\textsc{acc.pl=nom.pl}}\\ 
   \glt ‘The children feed the goldfish(\textsc{pl}.)’
   \jambox*{(L1 Polish)}

  \ex\label{ex:saturno:5b} 
  \gll 
   Deti     karmjat     zolot-ych      ryb.\\
   {child:\textsc{nom.pl}}  {feed:\textsc{pres.3pl}}  {golden-\textsc{acc.pl=gen.pl}}  {fish-\textsc{acc.pl=gen.pl}}\\ 
   \glt ‘The children feed the goldfish(\textsc{pl}.)’
   \jambox*{(L1 Russian)}
    \z
\z

\begin{table}
\small
\begin{tabularx}{\textwidth}{llQQQQQ}

\lsptoprule

\multicolumn{2}{p{1.7cm}}{male, adult (virile)} & \multicolumn{1}{c}{$+$} & \multicolumn{4}{c}{$-$}\\
\cmidrule(lr){3-3}\cmidrule(lr){4-7}
\multicolumn{2}{l}{human} & \multicolumn{2}{c}{+} & \multicolumn{3}{c}{$-$}\\
\cmidrule(lr){3-4}\cmidrule(lr){5-7}
\multicolumn{2}{l}{animate} & \multicolumn{3}{c}{+} & \multicolumn{2}{c}{$-$}\\
\cmidrule(lr){3-5}\cmidrule(lr){6-7}
{\scshape nom} & Rus & \coloredcell{\itshape eti rybak-i} & {\itshape eti mam-y} & {\itshape eti ryb-y} & {\itshape eti znak-i} & {\itshape eti okn-a}\\
& Pol & \coloredcell{\itshape ci rybac-y} & {\itshape te mam-y} & {\itshape te ryb-y} & {\itshape te znak-i} & {\itshape te okn-a}\\
\tablevspace
{\scshape gen} & Rus & {\itshape etich rybak-ov} & {\itshape etich mam-${\emptyset}$} & {\itshape etich ryb-${\emptyset}$} & {\itshape etich znak-ov} & {\itshape etich okon-${\emptyset}$}\\
& Pol & {\itshape tych rybak- ów} & {\itshape tych mam-${\emptyset}$} & {\itshape tych ryb-${\emptyset}$} & {\itshape tych znak-ów} & {\itshape tych okien-${\emptyset}$}\\
\tablevspace
{\scshape acc} & Rus & \multirow{2}{=}{= \textsc{gen}} & \coloredcell{= \textsc{gen}} & \coloredcell{= \textsc{gen}} & \multirow{2}{=}{= \textsc{nom}} & \multirow{2}{=}{= \textsc{nom}}\\
& Pol &  & \coloredcell{= \textsc{nom}} & \coloredcell{= \textsc{nom}} &  & \\
\midrule
& Eng & ‘fisherman’ & ‘mum’ & ‘fish’ & ‘sign’ & ‘window’\\
\lspbottomrule
\end{tabularx}
\caption{\label{tab:saturno:1}Gender system, plural. Agreement patterns exemplified with Rus. etot, Pol. ten ‘this’}
\end{table}

\subsection{Acquisition of Slavic L2s in the absence of positive transfer}
\label{sec:saturno:2.3}
This section summarises a selection of the existing research on the acquisition of Slavic languages by speakers \textit{without} knowledge of related languages, so as to establish a baseline to interpret the data presented in \sectref{sec:saturno:2.4}.

Research within the so-called \textit{learner variety approach} \citep{Perdue1993book} concluded that in the \textit{Basic Variety} -- the earliest stages of acquisition -- “lexical items typically occur in one invariant form” (\citealt[311]{KleinPerdue1997}), called the \textit{basic word form}. Grammatical and pragmatic meaning is expressed through word order \REF{ex:saturno:6}.\footnote{Throughout this chapter, the morphological glosses of L2 examples only indicate the target language form that most closely resembles the learner-produced form. They do {\textit{not}} imply that the interlanguage and native varieties share the same categories and organisational principles (e.g. case).}

\ea\label{ex:saturno:6}  
\gll {[Arˈtisk-a}     pozˈdravja   {tuˈmaʃk-a]}.\\
artist-\textsc{nom.sg}   cheers   interpreter-\textsc{nom.sg}\\ 
\glt ‘The artist says hello to the interpreter.’
\jambox*{(L2 Polish, \citealt[51]{SaturnoWatorek2020})}
\z

In addition to word order, semantic criteria such as animacy contrasts may also contribute to meaning disambiguation \REF{ex:saturno:7}. 

\ea\label{ex:saturno:7}  
\gll {[ˈDʒulj-a}  ˈlubi   herˈbat-e  ˈkol-a     i   {tʃokoˈlad-a]}.\\
Giulia-\textsc{nom}   likes   tea-\textsc{acc.sg}  coke-\textsc{nom.sg}  and   chocolate-\textsc{nom.sg}\\ 
\glt ‘Giulia likes tea, coke and chocolate.’
\jambox*{(L2 Polish, \citealt[131]{Saturno2019a})}
\z

Since the universality of the acquisition process derives from the general cognitive processes that shape it, within the learner variety approach the role of the L1 is typically seen as limited to details of otherwise largely shared tendencies, such as the headedness of compounds if the target language allows for both head-initial and head-final constructions \citep{BroederEtAl1993}.

From an entirely different point of view, Processability Theory (PT: \citealt{Pienemann1998, DiBiaseBettoni2015}) postulates that the “difficulty” of grammatical structures -- and thus, their place in an acquisition sequence -- is accounted for and predicted by their cognitive accessibility. Following a \textit{lemma access} stage in which words are produced in a single, invariant word form, typically (but not necessarily) modelled on the nominative singular, a sort of mini-paradigm \citep{BittnerEtAl2000} arises, opposing the basic word-form to a marked, “non-nominative” one \REF{ex:saturno:8}.

\ea\label{ex:saturno:8}  
    \ea\label{ex:saturno:8a}
    \gll Videla    volk-e.\\
    saw:\textsc{fem}  wolf-\textsc{non\_nom}\\ 
    \glt ‘(She) saw a wolf.’
    \jambox*{(L2 Russian, \citealt[188]{ArtoniMagnani2015})}

    \ex\label{ex:saturno:8b}
    \gll Su    videli    krevet-a.\\
    \textsc{aux:3pl}  seen:\textsc{pl}  bed-\textsc{non\_nom}\\ 
    \glt ‘(They) saw a bed.’
    \jambox*{(heritage Serbian, \citealt[209]{DiBiaseEtAl2015})}
    \z
\z

Object case marking first emerges in its canonical post-verbal position. In Ob\-ject-Subject structures, the accusative marking sometimes overextends to nouns performing the subject function, like \textit{balerina} in \REF{ex:saturno:9}. At this stage, pre-verbal nouns in the object function still appear in their basic word form, like \textit{vilka}.

\ea\label{ex:saturno:9}  
\gll Vilk-a     prinës     balerin-u.\\
fork-\textsc{nom.sg}  bring:\textsc{pst.sg.m}  dancer-\textsc{acc.sg}\\ 
\glt ‘It is the dancer that brought the fork.’
\jambox*{(L2 Russian, \citealt[190]{ArtoniMagnani2015})}
\z

Only at more advanced developmental stages do learners acquire the ability to case-mark the object in syntactically marked structures like OS, which in turn makes it possible to manipulate word order for pragmatic purposes \REF{ex:saturno:10}.

\ea\label{ex:saturno:10}  
\gll Vilk-u     prinesla     balerin-a.\\
fork-\textsc{acc.sg}  bring:\textsc{pst.sg.f}  dancer-\textsc{nom.sg}\\ 
\glt ‘It is the dancer that brought the fork.’
\jambox*{(L2 Russian, \citealt[189]{ArtoniMagnani2015})}
\z

Within PT, the L1 plays only a marginal role: even if a target language structure is already familiar to the learner from other known languages, transfer will only occur once the learner has reached the corresponding developmental stages (see \sectref{sec:saturno:2.4.2}).

\subsection{Acquisition of Slavic languages by speakers of related languages}\label{sec:saturno:2.4}

This section describes various aspects of the acquisition of Slavic L2s by speakers with knowledge of related languages, in order to highlight the differences that transfer may introduce compared to a situation in which the learner has no such knowledge (\sectref{sec:saturno:2.3}). Unless otherwise specified, the examples concern L2 Polish.

\subsubsection{Transfer in reception}\label{sec:saturno:2.4.1}

In the domain of comprehension, \citet{Saturno2019b} showed that L1 Italian university students of L2 Russian were often able to make appropriate grammatical inferences about Polish, a language they had no experience of. The following examples were elicited in a translation task. In \figref{fig:saturno:2}, the unfamiliar ending \textit{{}-u} in \textit{bagaż-u} ‘luggage-\textsc{gen.sg}’ is correctly interpreted based on the transparent \textit{{}-ego} ending in \textit{podręczn-ego} ‘hand(\textsc{adj})-\textsc{gen.sg}’, with which it agrees. 



  
\begin{figure}
%\includegraphics[width=.6\textwidth]{figures/SaturnoLANGSCI-img002.png}
 \includegraphics[width=\textwidth]{figures/saturno-figure-2.pdf}



\caption{Target-like analysis of a grammatical structure through intercomprehension}
\label{fig:saturno:2}
\end{figure}


While in Russian the ending \textit{{}-u} is not typically associated with the genitive case,\footnote{ {There is a handful of lexical items that may be marked by a “genitive” case in} {\textit{{}-u}}{, but that form is more accurately described as partitive, e.g.} {\textit{nemnogo sachar-u}} {‘a little sugar-}{\textsc{part}}{’.} } but rather with the dative or accusative \REF{ex:saturno:11}, \textit{{}-ego} is one of the allomorphs of the masculine and neuter genitive singular morpheme, e.g. Rus. \textit{choroš-ego} ‘good-\textsc{gen.sg}’.



\ea\label{ex:saturno:11}
\ea\label{ex:saturno:11a}
\textsc{dat.sg} of masculine and neuter nouns, e.g. \textit{transport-u} ‘transport(\textsc{m})-\textsc{dat.sg}’
\jambox*{(L1 Russian)}



  \ex\label{ex:saturno:11b}
  \textsc{acc.sg} of feminine nouns in \textit{{}-a}, e.g. \textit{transportirovk-u} ‘transportation(\textsc{f})-\textsc{acc.sg}’
  \jambox*{(L1 Russian)}
  \z
\z


In other cases, however, the assumption that the same morphosyntactic patterns apply in the unknown L2 and in the known languages may lead to erroneous inferences. In the expression in \REF{ex:saturno:12}, the reader is confronted with a grammatical structure that partially diverges between Polish and Russian (see \sectref{sec:saturno:2.2.1}). The masculine noun \textit{transportu} here appears in the \textsc{gen.sg} because it functions as direct object within the scope of negation, while in Russian it would probably occur in the \textsc{acc.sg} form (although the genitive is also possible, theoretically).



\ea\label{ex:saturno:12}  \gll Nie  możemy  zagwarantować   transport-u     bagaż-u.\\
  \textsc{neg}  can:\textsc{1pl}  guarantee:\textsc{inf}  transport-\textsc{gen.sg}  luggage-\textsc{gen.sg}\\ 
  \glt ‘We cannot guarantee that the luggage will be transported.’
  \jambox*{(L1 Polish)}
\z

\begin{figure}[b]
%\includegraphics[width=\textwidth]{figures/SaturnoLANGSCI-img003.png}
 \includegraphics[width=\textwidth]{figures/saturno-Figure-3.pdf}
\caption{Erroneous analysis of a grammatical structure through intercomprehension}
\label{fig:saturno:3}
\end{figure}

For a reader with knowledge of Russian, the form \textit{transportu} is likely to be unexpected for at least two reasons. First, an obviously transitive verb (\textit{zagwarantować} ‘guarantee’) is followed by a form that seems different from the expected accusative case, which in Russian is expressed by a zero morph (\textit{transport}, identical to \textsc{nom.sg}). Second, as shown in \REF{ex:saturno:11}, in Russian the ending \textit{{}-u} is associated with two meanings that both seem inappropriate in the present context. Thus, in addition to the correct explanation, the reader may choose among three possible accounts: a) \textit{zagwarantować} does not govern the accusative case, b) \textit{transportu} is the accusative form of a feminine noun, or c) \textit{transportu} is the accusative form of a masculine noun despite its odd-looking ending. In \figref{fig:saturno:3}, the student opted for either b) or c). All analyses are regrettably incorrect, although one may wonder to what extent such erroneous inferences may jeopardise comprehension. Moreover, the learner’s metalinguistic analysis of the fragment in question relies on grammatical categories that are indeed appropriate to describe the system of Polish.



Despite the possibility of erroneous grammatical inferences, these observations are reminiscent of the following argument: 

\begin{quote}
“Suppose someone to assert: \textit{The gostak distims the doshes}. You do not know what this means; nor do I. But if we assume that it is English, we know that the \textit{doshes} are \textit{distimmed} by the \textit{gostak}. We know too that one \textit{distimmer} of \textit{doshes} is a \textit{gostak}. If, moreover, the \textit{doshes} are galloons, we know that some galloons are \textit{distimmed} by the \textit{gostak}. And so we may go on, and so we often do go on”. \hbox{}\hfill\hbox{\citep[46]{OgdenRichards1923}}
\end{quote}

In the Slavic context, a similar point was made by the Soviet linguist Lev Ščerba through the aphorism \textit{leksika -- dura, grammatika -- molodec} ‘the lexicon is a fool, grammar is clever’ \citep[88]{Ščerba1974}. The maxim refers to the fact that a sentence like \textit{glokaja kuzdra šteko budlanula bokra i kudrjačit bokrenka}, though entirely composed of non-words, can still be described in terms of parts of speech and inflectional morphology based on knowledge of Russian grammar (\citealt[247--252]{Uspenskij1954}). 

From the evidence presented above, it can be concluded that thanks to the similarity of the grammatical system of Polish and East Slavic, even non-native speakers of a Slavic language exhibit rather good skills in the domain of comprehension and grammatical inference. The following section will consider the problem of language production in the same conditions. 

\subsubsection{Transfer in production}
\label{sec:saturno:2.4.2}
Interlanguage features attributable to East Slavic languages have been extensively documented and analysed in Polish scholarship, though almost exclusively from a non-quantitative perspective. In particular, the interlanguage realization of the Polish accusative case considered here seems to frequently betray L1 transfer. Starting with the accusative singular of feminine nouns, its realization as \textit{{}-u} (as opposed to \textit{-ę}) is described as a very common pattern \REF{ex:saturno:13}.

\ea\label{ex:saturno:13}  
\gll dobrze   znać   *histori-ju\\
  well    know:\textsc{inf} {history-\textsc{acc.sg} (East Slavic form)}\\
  \glt ‘to know history well’ 
   \jambox*{(L2 Polish, \citealt[363]{Górska2015})}
\z

The same is true of object marking within the scope of negation, which often appears in the accusative rather than in the expected genitive. Moreover, said accusative in turn may be expressed by its Polish form in \textit{-ę} \REF{ex:saturno:14a}, or its East Slavic counterpart in \textit{-u} \REF{ex:saturno:14b}.

\ea\label{ex:saturno:14}  
    \ea\label{ex:saturno:14a} 
    \gll Nie   przegap     *szans-ę.\\
  not  miss:\textsc{imp.2sg}    {opportunity-\textsc{acc.sg} (Polish form)}\\
  \glt ‘Don’t miss the opportunity.’ \jambox*{(L2 Polish, \citealt[325]{Izdebska-Długosz2021a})}

    \ex\label{ex:saturno:14b}
    \gll Nie   lubię     *herbat-u.\\
  not  like:1\textsc{sg}  {tea-\textsc{acc.sg} (East Slavic form)}\\ 
  \glt ‘I don’t like tea.’
  \jambox*{(L2 Polish, \citealt[224]{Izdebska-Długosz2021a})}
  \z
\z

Concerning DOM, entities belonging to different grammatical genders in Polish and East Slavic (‘women’ and ‘birds’ in the examples \ref{ex:saturno:15a} and \ref{ex:saturno:15b}) are often case-marked according to the East Slavic pattern in the interlanguage, i.e. with an accusative form syncretic with the \textsc{gen.pl}, rather than the \textsc{nom.pl} required by the target language \REF{ex:saturno:15}. The same is true of agreeing parts of speech, like the adjective \textit{wysportowan-ych} ‘sporty-\textsc{acc.pl.vir}’ (required \textit{wysportowan-e} ‘sporty-\textsc{acc.pl.nonvir}’).

\ea\label{ex:saturno:15}  
    \ea\label{ex:saturno:15a}
    \gll Wojtek   lubi     *wysportowan-ych   *kobiet-${\emptyset}$.\\
    Wojtek:\textsc{nom}  like:\textsc{3sg}  sporty-\textsc{acc.pl=gen.pl}  woman-\textsc{acc.pl=gen.pl}\\ 
    \glt ‘Wojtek likes sporty women.’
    \jambox*{(L2 Polish, \citealt[188]{Izdebska-Długosz2021a})}

    \ex\label{ex:saturno:15b}
    \gll Dzieci       karmią     *ptak-ów.\\
child\textsc{:nom.pl}    feed:\textsc{pres.3pl}  bird\textsc{{}-acc.pl=gen.pl} \\ 
   \glt ‘The children feed the birds.’
   \jambox*{(L2 Polish, \citealt[77]{DąbrowskaEtAl2010})}
   \z
\z

Transfer effects in the L2 acquisition of Slavic languages have also been investigated within PT (\sectref{sec:saturno:2.3}). Indeed, a component of the theory, the Developmentally Moderated Transfer Hypothesis (DMTH: \citealt{HåkanssonEtAl2002, PienemannEtAl2016}), predicts that the presence of a target language feature in another language already known by the learner cannot alter the universal acquisition path. At most, positive transfer can accelerate the systematic and accurate production of the structure in question, but only once it has emerged in the interlanguage. \citet{MagnaniArtoni2021} investigated the development of the accusative case marking among three groups of L2 Russian learners, characterised by typologically differentiated L1s, i.e. a) Italian, with no case-marking at all; b) non-Slavic languages marking case morphologically, but through different strategies than Russian; c) other Slavic languages, exploiting the same case-marking mechanisms as Russian. Their analysis shows that the target structure emerges more rapidly among Slavic speakers, although all learners follow the acquisition sequence hypothesised by PT. In particular, it is noteworthy that some Slavic speakers only case-mark the object when it occurs in its canonical postverbal position (cf. L2 Russian in \ref{ex:saturno:16a} vs. \ref{ex:saturno:16b}). Elsewhere, they produce the basic, uninflected word-form, consistently with the behaviour of non-Slavic learners (cf. example \ref{ex:saturno:9}).

\ea\label{ex:saturno:16}  
    \ea[*]{\label{ex:saturno:16a}
    \gll Gruš-a  prinël        medsestr-a.\\
  pear-\textsc{nom.sg}  {brought (creative form)}  nurse-\textsc{nom.sg}\\ 
\glt `The pear was brought by the nurse.’}
\jambox*{(L2 Russian, \citealt[75]{MagnaniArtoni2021})}

    \ex[]{\label{ex:saturno:16b}
    \gll Vrač-${\emptyset}$    prinës    trubu.\\
doctor-\textsc{nom.sg}  brought  trumpet\textsc{{}-acc.sg} \\ 
    \glt `The doctor brought the trumpet.'}
    \jambox*{(L2 Russian, \citealt[75]{MagnaniArtoni2021})}
    \z
\z

\section{Three studies on the acquisition of L2 Polish by learners with knowledge of Russian}
\label{sec:saturno:3}

The present section summarises three published studies on the acquisition of L2 Polish, here ordered by learners’ growing familiarity with Russian. The claims of the theoretical models discussed in \sectref{sec:saturno:1} and \sectref{sec:saturno:2} will be tested on L2 data regarding the accusative singular of feminine nouns in \textit{{}-a} and the accusative plural of animate, but non-virile nouns. Before turning to the specificities of each study, a few common methodological points will be described below. 

\subsection{Data collection} %3.1. /
\label{sec:saturno:3.1}

The studies made use of two language tasks with several points in common: first, in order to verify whether or not the use of a given morpheme is productive, all target forms appeared several times on different lexical items, as recommended by \citet{Pallotti2007}. Second, responses were recorded and transcribed by the author and double-checked by two native speakers of Russian and Polish with SLA expertise. 

\subsubsection{The “Animal Dinner” task}\label{sec:saturno:3.1.1}
Study 1 made use of an adaptation of the “Animal Dinner” task described in \citet{DiBiase2007}. Participants were asked to “read” from left to right a set of pictures in which the subject appeared on the left or on the right-hand side of a symbol (a heart) representing the verb \textit{like}. This produced SO and OS structures, respectively \REF{ex:saturno:17}. Target sentences were preceded by a question whose purpose was to topicalise the initial noun, i.e. the object in OS targets and the subject in SO targets. Referents appeared as drawings, in order not to suggest the target word-form in any way. The interrogative pronouns \textit{kto} ‘who:\textsc{nom}’, \textit{kogo} ‘who:\textsc{acc}’ and \textit{co} ‘what:\textsc{nom/acc}’ were represented by question marks. A target-like example is presented in \REF{ex:saturno:17}. Participants were required to “read” (i.e., produce) both turns.

\ea\label{ex:saturno:17}  
\gll Q.  Kto    lubi  rzek-ę?\\
    {} who:\textsc{nom}  likes  river-\textsc{acc.sg}?\\
\glt \hspace{2.5ex}‘Who likes the river?’

 \gll A.  Rzek-ę    lubi  lew-${\emptyset}$.\\
{} river-\textsc{acc.sg}  likes  lion-\textsc{nom}\\
\glt \hspace{2.5ex}‘It is the lion who likes the river’
\z

Target words belonged to the paradigm of feminine nouns in \textit{{}-a} and were easily recognisable Slavic cognates. Before the task, participants were given sufficient time to familiarise themselves with these words using a specially prepared visual glossary.

\subsubsection{The Elicited Imitation task}\label{sec:saturno:3.1.2}

In the Elicited Imitation task (EIT) used in study 2 and study 3, learners were asked to listen to a stimulus sentence, perform a distracting task to inhibit phonological memory, and then repeat the stimulus as accurately as possible. The rationale of this procedure is that the EIT does not require learners to merely repeat the stimulus sentence as a string of potentially meaningless sounds, but to de-code and re-encode its meaning based on the interlanguage grammar. The EIT has been shown to correlate well with semi-spontaneous speech (\citealt{IsbellSon2022, KostromitinaPlonsky2022, WuEtAl2022}) and is widely used to study L2 implicit competence without the risk that the structures of interest might not emerge in spontaneous learner output. 

Target items were embedded in carrier sentences of varying length \REF{ex:saturno:18}.

\ea\label{ex:saturno:18}  
\gll Najlepsz-e     przyjaciółk-i     pozna-ł-a-m     na   uniwersyteci-e.\\
  best-\textsc{acc.pl.nonvir}  girlfriend-\textsc{acc.pl}  meet-\textsc{pst-f-1sg}  at  university-\textsc{loc.sg}\\
\glt  ‘I met my best (female) friends at university.’
\z

Additionally, the stimuli employed in Study 2 form a simple story (\ref{ex:saturno:19}; target items are morphologically glossed).

\ea\label{ex:saturno:19}  
    \ea\label{ex:saturno:19a}
    \gll Jan  jest  studentem,   a  dzisiaj  akurat  ma   lekcję.\\
    Jan  is  student  and  today  just  has  classes-\textsc{acc.sg}\\
    \glt ‘Jan is a student, and today he just happens to have classes.’

    \ex\label{ex:saturno:19b} 
    \gll Ale  Jana  lekcja      nie  interesuje.  Obok  jest dziewczyna.\\
    But  Jan  lesson-\textsc{nom.sg}  not  interests.  Next  is    girl-\textsc{nom.sg}\\
    \glt ‘But Jan is not interested in the lesson. Next to him is a girl.’

    \ex\label{ex:saturno:19c}
    \gll Nie  lubisz  lekcji?      Dziewczynę  pyta  student.\\
    Not  like  lesson-\textsc{gen.sg}?  girl-\textsc{acc.sg}  asks  student\\
    \glt ‘Don’t you like the lesson? The student asks the girl.’
    \z
\z

The EIT was administered individually in a quiet room using a laptop computer equipped with headphones. Target sentences were embedded in a timed PowerPoint presentation.

\subsection{Study 1: Speakers of non-related languages with knowledge of a related L2 \citep{Saturno2020b}}
\label{sec:saturno:3.2}
The purpose of this study was to verify whether minimal Polish input coupled with appropriate metalinguistic explanations would be sufficient to allow L1 Italian university students of L2 Russian to case-mark L3 Polish nouns in the object function. In the absence of transfer, one would expect learners to go through a phase characterised by the overextension of a single, invariable word-form. Conversely, the rapid development of the target structure after minimal exposure should probably be attributed to the positive influx of a known related language.

\subsubsection{Methodology}\label{sec:saturno:3.2.1}

The Polish input was represented by a specially designed 20-hour course in Slavic linguistics, focusing on a comparative analysis of Russian and Polish grammar. In the data collection phase, participants performed the Animal Dinner task described in \sectref{sec:saturno:3.1.1}. The procedure was first carried out in L3 Polish and then repeated in L2 Russian, in order to establish a benchmark for the learners’ processing skills in the language from which transfer might originate. Participants were asked to perform the test as quickly and spontaneously as possible. 

\subsubsection{Participants}\label{sec:saturno:3.2.2}

Twenty-two L1 Italian university students of L2 Russian took part in the experiment. Thirteen of them were enrolled in the second or third year of their BA, while nine were at the MA level. None were familiar with any Slavic language other than Russian. 

\subsubsection{Results}\label{sec:saturno:3.2.3}

The data contain numerous instances of overextension of a basic, uninflected word form onto the expected accusative/genitive forms, like \textit{rzek-a} [ˈʒeka] ‘river-\textsc{nom.sg’} in \REF{ex:saturno:20}, cf. \textit{rzek-ę} [ˈʒeke] ‘river-\textsc{acc.sg’}. This pattern is consistent with the acquisition path of Slavic languages by speakers of non-related languages.

\ea\label{ex:saturno:20} 
\gll {[Lev}     ˈlubi   ˈlubi   {*ˈʒek-a]} \\
lion:\textsc{nom.sg}   likes   likes   river-\textsc{nom.sg} \\ 
\glt ‘The lion likes the river.’
\jambox*{(L2 Polish)}
\z

However, other errors seem to reflect the influence of L2 Russian. In \REF{ex:saturno:21}, the forms \textit{ryb-u} [ˈrɨbu] ‘fish-\textsc{acc.sg}’ and \textit{turyst-ek} [tuˈrystek] ‘female\_tourists:\textsc{acc.pl}’ are not instances of the basic word form, but are not target-like either. The accusative case is expressed by the morph that encodes it not in Polish, but in Russian, cf. expected \textit{ryb-ę} [ˈrɨbe], \textit{turystk-i} [tuˈrystki]. Though incorrect, these examples point to the learner’s ability to productively manipulate inflectional morphology. 

\ea\label{ex:saturno:21}  
    \ea\label{ex:saturno:21a} 
    \gll {[ˈnjedved-${\emptyset}$}   ˈlubi   {*ˈrɨb-u]}\\
    bear-\textsc{nom.sg}   likes   {fish-\textsc{acc.sg} (Russian morpheme)}\\ 
    \glt ‘The bear likes fish.’
    \jambox*{(L2 Polish)}

    \ex\label{ex:saturno:21b} 
    \gll {[*tuˈryst-ek}       ˈlubi   {ˈnjedved-${\emptyset}$]}\\
    tourist-\textsc{acc.pl}=\textsc{gen.pl}  likes  bear-\textsc{nom.sg}\\ 
    \glt ‘It is the bear that likes the tourists.’
    \jambox*{(L2 Polish)}
    \z
\z

\begin{sloppypar}
Finally, superficially target-like responses are often encountered. In \REF{ex:saturno:22}, \textit{tu\-rystk-i} [tuˈrystki] ‘tourist-\textsc{acc.pl}=\textsc{nom.pl’} conforms to the Polish pattern. If transfer from East Slavic applied, one would expect a form syncretic with the genitive, like \textit{turyst-ek} [tuˈrystek] ‘tourist-\textsc{acc.pl}=\textsc{gen.pl}’ in \REF{ex:saturno:21b}.
\end{sloppypar}

\ea\label{ex:saturno:22}  
\gll {[tuˈristk-i}         ˈlubi  {niʥˈvjeʥ-${\emptyset}$]}\\
  female\_tourist-\textsc{acc.pl=nom.pl}  likes  bear-\textsc{nom.sg}\\ 
\glt ‘It is the bear that likes the tourists.’\jambox*{(L2 Polish)}
\z

From a quantitative point of view, if both the target-like Polish forms and those modelled on Russian (like [ˈrɨbu] and [tuˈrystek] in example 21) are considered correct (“Pol+Rus” bars in \figref{fig:saturno:4}), accuracy scores can be very high. This is especially evident in the case of plural animate nouns (third and fourth pairs of bars, respectively), with respect to which the DOM patterns of Polish and Russian diverge the most.

Grouping together the two types of forms seems legitimate in light of the fact that both imply the production of an inflected word-form, as opposed to the nominative overextension typically observed in initial learner varieties.

\vfill
\begin{figure}[H]
\includegraphics[width=\textwidth]{figures/saturno-figure-4.pdf}
\caption{Accuracy by target structure (from \citealt[10]{Saturno2020b}). Abbreviations: AN = animate nouns; IN = inanimate nouns; OS = Object-Subject word order; PL = plural; Pol = target-like Polish morphology; Pol+Rus = target-like Polish morphology and Russian morphemes; SG = singular; SO = Subject-Object word order.}
\label{fig:saturno:4}
\end{figure}
\vfill\pagebreak


\subsection{Study 2: Speakers of a related L1 in an experimental setting \citep{Saturno2022a}}
\label{sec:saturno:3.3}
The VILLA NOVA\footnote{{Sincere thanks to Christine Dimroth, Roland Meyer and Dorota and Christian Hörrle for their invaluable contribution to the success of the experiment.}} project aimed to verify whether or not an exclusively communicative course could lead to the processing of inflectional morphology in the earliest stages of SLA.

\subsubsection{Methodology}\label{sec:saturno:3.3.1}

Participants were exposed to 4:30 hours of Polish communicative instruction (i.e. without any grammatical explanations) and took part in several communicative tasks aiming to elicit target structures in semi-spontaneous production. No written language or meta-language was used throughout the course: all referents and actions were illustrated by pictures. The EIT was administered at the end of the course. 

\subsubsection{Participants}\label{sec:saturno:3.3.2}

Since the VILLA NOVA approach was designed to be applicable in a university context, the course was open to all participants eligible to enrol in first-year Polish classes. The only requirement was a lack of knowledge of Polish. The analysis of the enrolment forms makes it clear that only three students had knowledge of Russian as an L2 (self-assessed B1), while the rest were native speakers of it. In order to focus on the status of the potential source of transfer (L1 or L2), therefore, it was decided to compare the results of the three L1 German learners of L2 Russian with an equal number of randomly selected L1 Russian speakers. Considering only six participants is an obvious limitation for the quantitative analysis, but does not preclude the opportunity to make relevant qualitative observations.

\subsubsection{Results}\label{sec:saturno:3.3.3}

Errors most typically consist in the overextension of the nominative case in contexts expressing a syntactic function other than the subject, as in \textit{Magd-a} [ˈmagd-a] in \REF{ex:saturno:23}.

\ea\label{ex:saturno:23}  
\gll {[Jan-${\emptyset}$}   zaˈpraʃa  {ˈmagd-a]}\\
  Jan-\textsc{nom.sg}  invites    Magda-\textsc{nom.sg}\\ 
 \glt ‘Jan invites Magda.’
 \jambox*{(L2 Polish)}
\z

The required Polish morphemes are sometimes substituted with their Russian counterpart, appropriately selected in accordance with the syntactic context. In \REF{ex:saturno:24}, for example, the required Polish accusative in \textit{{}-ę} (\textit{lekcj-ę} [lekʦje] ‘lesson-\textsc{acc.sg}’) is substituted with the Russian accusative form in \textit{{}-u} (cfr. Russian \textit{lekci-ju} [ˈl\textsuperscript{j}ekʦɪju]).

\ea\label{ex:saturno:24}  
\gll {[jan}   stuˈdent     i   {{}ˈlekʦj-u]}\\
  Jan  student:\textsc{nom.sg}  and  {classes-\textsc{acc.sg} (Russian morpheme)}\\ 
 \glt ‘Jan is a student and today he has classes.’
 \jambox*{(L2 Polish)}
\z

Overextensions of the basic word form occur in the output of all learners, independently of the status of Russian in their repertoire, while the overextension of \textit{{}-u} was only observed among Russian native speakers. 

From a quantitative point of view, errors do seem to be slightly more frequent in the case of the L2 Russian speakers due to the greater number of basic word-forms produced (\tabref{tab:saturno:2}). Nevertheless, the effect of the L1 did not prove statistically significant in a GLMM\footnote{{Generalised linear mixed model.}} with Poisson error structure, although one should consider the paucity of the data (192 relevant responses). 

\begin{table}
\begin{tabular}{llrrr}

\lsptoprule

L1 & participant & errors & correct responses & omissions\\
\midrule
Ger & P3 & 9 & 18 & 25\\
& P11 & 10 & 21 & 21\\
& P12 & 2 & 6 & 44\\
Rus & P1 & 1 & 3 & 48\\
& P10 & 6 & 16 & 30\\
& P13 & 3 & 25 & 24\\
\lspbottomrule
\end{tabular}
\caption{EIT output}
\label{tab:saturno:2}
\end{table}

\subsection{Study 3: Speakers of a related L1 in an experimental setting \citep{Saturno2023}\protect\footnote{The study was made possible thanks to the support of NAWA (\textit{Narodowa Agencja Wymiany Akademickiej}).}}
\label{sec:saturno:3.4}

The primary objective of the study was to identify an acquisition order for a set of L2 Polish morphosyntactic structures. A crucial variable in this respect was the participants’ L1 (Slavic vs. non-Slavic). The secondary objective was to correlate the observed linguistic results with sociolinguistic information such as length of stay in Poland, frequency of use of Polish in everyday life, motivation to learn the language, etc. 

\subsubsection{Methodology}\label{sec:saturno:3.4.1}

Participants were asked to fill in an online survey and then perform an EIT focussing on eleven morphosyntactic structures, identified as potentially problematic through a comparative analysis of Polish and Russian grammar. A group of L2 Polish teachers with experience in the teaching of East-Slavic Speakers was also consulted. Crucially, the set includes the two target structures considered so far. 

\subsubsection{Participants}\label{sec:saturno:3.4.2}

The study considers the output of 161 participants enrolled in the L2 Polish cour\-ses offered by a Polish university. 131 of them were ESSs, while the remaining 30 had a non-Slavic L1 background and no knowledge of Slavic languages other than L2 Polish.\footnote{{These learners’ L1s include the following: Amharic, Armenian, Azerbaijan, English, Estonian, Gujarati, Hindi, Italian, Kazakh, Kyrgyz, Latvian, Romanian, Shona, Spanish, Tajik, Turkish, Uzbek, Vietnamese. Some (e.g. Hindi, Romanian, Spanish) indeed exhibit DOM, but do so through patterns that cannot be compared to the Polish structure.}}

Unlike the previous two studies, the data were collected in uncontrolled input conditions. Moreover, all participants had been taking L2 Polish classes for a period ranging from less than three months to over five years. The survey data indicate that while most non-ESS were still enrolled in A1 courses even after a long stay in Poland, most ESS were enrolled in B1 courses, which they tended to attain after a relatively short time (\tabref{tab:saturno:3}). 

\begin{table}
\small
\begin{tabularx}{\textwidth}{llYYYYYrr}

\lsptoprule

 L1 & level & <3 months & <6 months & <1 
 
 year & <2 years & <5 years & other & total\\
 \midrule
 non-ESS & \coloredcell{A1} & 0.08 & \coloredcell{0.18} & \coloredcell{0.24} & 0.1 & \coloredcell{0.14} & 0.03 & \coloredcell{0.77}\\
& A2 & 0.03 & -- & 0.04 & 0.04 & -- & -- & 0.11\\
& B1 & -- & -- & 0.03 & 0.03 & 0.03 & 0.03 & 0.12\\
\tablevspace
 ESS & A1 & 0.01 & 0.01 & 0.02 & -- & 0.02 & 0.02 & 0.08\\
& A2 & 0.04 & 0.08 & 0.09 & 0.02 & -- & -- & 0.23\\
& \coloredcell{B1} & 0.03 & \coloredcell{0.13} & \coloredcell{0.24} & \coloredcell{0.17} & 0.06 & 0.06 & \coloredcell{0.69}\\
\lspbottomrule
\end{tabularx}
\caption{\label{tab:saturno:3}Proportion of learners by L1, length of stay in Poland and proficiency level}
\end{table}

\subsubsection{Results}\label{sec:saturno:3.4.3}

The quantitative analysis shows that global accuracy scores are decidedly higher for ESSs compared to non-ESSs (\figref{fig:saturno:5}). Here and in the following graphs, shaded areas represent credibility intervals, while the red segment indicates the mean score, also reported in figures. Data points represent each participant’s mean score. Note that because of the limited number of non-ESSs, the credibility interval for their mean is wider than in the case of the ESSs. 

  
\begin{figure}
%\includegraphics[width=\textwidth]{figures/SaturnoLANGSCI-img005.png}
\includegraphics[width=\textwidth]{figures/saturno-figure-5.pdf}
\caption{Mean scores by L1, all target structures}
\label{fig:saturno:5}
\end{figure}

Turning to the goal of identifying an acquisition sequence, \figref{fig:saturno:6} presents the mean scores obtained by ESS and non-ESS on the eleven target structures, ordered from top to bottom by decreasing accuracy within the East Slavic group.

  
\begin{figure}
%\includegraphics[width=\textwidth]{figures/SaturnoLANGSCI-img006.png}
\includegraphics[width=\textwidth, trim={0 2cm 0 2cm}, clip]{figures/saturno-figure-6.pdf}
\caption{Mean scores by target structure and L1. Abbreviations: cop = copula verb in the present tense; acc.sg(f) = accusative singular of feminine nouns in \textit{-a}; pres = present tense of a set of regular verbs; nom.pl(vir) = nominative plural of nouns belonging to the virile gender; ins.sg(m) = instrumental singular of masculine nouns; loc.sg = locative singular (in \textit{-e}) of nouns; acc.pl(nonvir) = accusative plural of nouns belonging to the non-virile gender; ins.sg(f) = instrumental singular of feminine nouns; pst = past tense of verbs; gdyby = conditional mood; dat.pl = plural dative of nouns.}
\label{fig:saturno:6}
\end{figure}

The acquisition order largely coincides in the two groups, with only minor exceptions. This observation lends empirical support to the view that the role of the L1 in L2 acquisition should not be overestimated. Nevertheless, ESS did achieve higher scores on all structures, a fact which does suggest a facilitating role for L1 proximity to the target language. 

Regarding the accusative marking of non-virile nouns – “acc.pl(nonvir)” in \figref{fig:saturno:6} – although the non-ESS’ mean score is lower than the ESS’, the difference (ESS = 0.79, non-ESS = 0.62) is not as great as in the case of many other structures. The accusative singular of feminine nouns – “acc.sg(f)” in \figref{fig:saturno:6} – is the second easiest structure for both ESSs and non-ESSs.

\section{Discussion}\label{sec:saturno:4}

The findings of the three studies presented in this chapter can be summarised as follows. First, one observes a tendency to overextend a single, invariable word-form to the required inflected forms. This tendency is common to all participants, regardless of their linguistic repertoire, but is somewhat less noticeable among Slavic learners. Given an adequate statistical sample, on average this group can be shown to perform better than their non-Slavic counterparts. 

Altogether, these conclusions suggest that knowledge of a Slavic language facilitates the acquisition of a related language in terms of ease and rapidity of acquisition, but cannot alter the fundamental acquisition path. In particular, it does not eliminate the errors that typically occur in initial learner varieties. In what follows, an analysis of potential arguments in favour of and against transfer will be proposed.

\subsection{Arguments in favour of transfer}\label{sec:saturno:4.1}
\subsubsection{Overextension from previously known languages}\label{sec:saturno:4.1.1}

While the overextension of a basic, uninflected word form can be expected independently of the learner’s previous linguistic knowledge, the overextension of a form that encodes the required morpheme in the L1 (e.g. \textit{lekʦj-u} in \ref{ex:saturno:24},\footnote{{Example replicated here from  \sectref{sec:saturno:3.3}.}} modelled on Russian) seems directly attributable to the knowledge of a closely related language.

\begin{exe}%24
    \exr{ex:saturno:24}
    \gll {[jan}   stuˈdent     i   {{}ˈlekʦj-u]}\\
  Jan  student:\textsc{nom.sg}  and  {classes-\textsc{acc.sg} (Russian morpheme)}\\ 
    \glt  ‘Jan is a student and today he has classes.’
    \jambox*{(L2 Polish)}
\end{exe}

Although from a normative perspective neither utterance is correct, \REF{ex:saturno:24} seems considerably closer to the organizational principles of the target language. Grammatical meaning is not conveyed by word order (not exclusively, at least), but by inflectional morphology: the learner knows and manipulates different forms of the same lexical item, selecting them in accordance with the syntactic context. Adopting the terminology of PT (\sectref{sec:saturno:2.3}), such behaviour indicates a more advanced developmental stage. The fact that the actual morph produced is not entirely target-like does not affect the organisational principles of the interlanguage. 

DOM patterns consistent with the system of East Slavic but not Polish may be easily interpreted in terms of (negative) transfer, too. To start with, the learner-produced forms are obviously very close to the East Slavic model (\textsc{acc=gen} instead of \textsc{acc=nom}). Second, it is improbable that the instances of \textsc{acc=gen} may result from the overextension of a basic word form modelled on the genitive case, since the latter does not seem a likely model for the basic word form (although such a possibility cannot be excluded). Lastly, compared to the nominative case, the genitive case occurs in a more limited set of contexts and is less frequent in language use. In the plural of feminine and neuter nouns, moreover, it is expressed by a zero morph, which is an anomalous pattern both within the Polish system and typologically. Following the criteria proposed by \citet[72--91]{Croft1990}, it can thus be argued that the East Slavic pattern is more marked than its Polish counterpart, and consequently, that it could hardly overextend in the absence of L1 transfer.

The fact that both Slavic and non-Slavic learners of L2 Polish at times inflect nouns following the East Slavic model (in terms of inflectional endings or DOM patterns) is consistent with the suggestion that transfer may occur from a language other than the L1 (see \sectref{sec:saturno:1}). If one accepts that the learners had reached the appropriate developmental stage at the time of data collection, the Developmentally Moderated Transfer Hypothesis (\sectref{sec:saturno:2.4.2}) is not in contrast with these findings, either.

These observations suggest that what matters is not the L1 or L2 status of the language from which transfer originates, but the learner’s familiarity with it. High competence is obviously implied if the language in question is a person’s L1, but cannot be excluded in the case of advanced L2 learners, either. A promising goal for further research is the identification of objective criteria to determine at what stage a learner becomes sufficiently familiar with a language to trigger positive transfer from it.

\subsubsection{Distribution by proficiency level}\label{sec:saturno:4.1.2}

The language proficiency distribution of the participants in study 3 could be interpreted as the result of a cross-sectional investigation, in which the output of learners potentially characterised by varying proficiency levels is analysed synchronically in order to deduce an acquisition sequence. While Slavic learners quickly progress through proficiency levels, often reaching the B2 threshold within a few months from their arrival in Poland, non-Slavic learners often remain stuck at lower levels (A1--A2) for several years. It cannot be excluded that Ukrainian learners took L2 Polish classes in Ukraine, where this language is more commonly taught than elsewhere; however, only a limited number of participants reported doing so. Thus, the rapidity of ESS’ acquisition may be interpreted as additional evidence pointing to a facilitative effect of their L1. To this it could be added that given their numbers, Slavic learners are often grouped together in exclusively Slavic classes, in which an adequately trained teacher can exploit language teaching approaches and materials designed to maximise the potential of the learners’ language repertoire. Indeed, L2 Polish language teaching resources specifically addressed at Slavic speakers have appeared over the last few years (\citealt{KołakEtAl2015, Izdebska-Dlugosz2017,Izdebska-Długosz2021b}), although until recently teachers were reported to adopt general-purpose handbooks (\citealt{Gębka-Wolak2018}).

\subsection{Arguments against transfer}\label{sec:saturno:4.2}
\subsubsection{Universal sequences}\label{sec:saturno:4.2.1}

Abstracting away from the obvious difference in mean scores – vastly superior in the case of ESSs – the acquisition sequences inferable from the distribution of accuracy scores in study 3 (\figref{fig:saturno:6}) are quite uniform regardless of the learners’ L1. This conclusion is consistent with the findings of previous studies (\sectref{sec:saturno:1}). A direct comparison with the developmental sequence identified for similar languages (e.g. \citealt{ArtoniMagnani2015} on L2 Russian) is unfortunately impossible because of the different methodological choices made.

A closer investigation of the two structures (out of eleven) in which the ESS’ and non-ESS’ sequences diverge can be helpful to better illuminate the dynamics of transfer. The first is DOM, in which non-ESS’ performance is inconsistently better than in the case of other structures. A possible explanation \textit{not} involving transfer will be outlined in  \sectref{sec:saturno:4.2.2}. The second structure concerns a set of consonant alternations occurring in the inflectional paradigm of numerous Polish nouns. The question is discussed in detail elsewhere (\citealt{Saturno2023}); for the time being it will be sufficient to explain that such consonant alternations occur in East Slavic languages to a more limited extent than they do in Polish, so that it is questionable whether transfer may apply directly.

Particularly noteworthy is also that all the target structures in Study 3 were selected among those that in light of comparative grammar could be hypothesised to cause difficulties to ESSs due to \textit{negative} transfer. Based on a strict evaluation of grammatical accuracy with respect to the target morphemes -- that is, leaving aside fluency, vocabulary, etc. -- one would thus expect ESSs to perform more poorly than their non-Slavic counterparts, who cannot suffer from negative transfer from other Slavic languages. Clearly, that was not the case. Two hypotheses may be formulated to account for this observation. First, transfer may simply play no role in SLA. Alternatively, despite the impeding effect of discrepancies between the L1 and the target language, extensive proximity between the two systems may be beneficial in terms of overall L2 processing ease. ESSs might be able to devote fewer cognitive resources than non-ESSs to the processing of those aspects of Polish that are similar to their East Slavic counterparts (vocabulary, morphosyntactic elements, etc.). If said resources are readdressed to the processing of diverging aspects, these would probably be produced more accurately. This view is reminiscent of \citegen[168]{Skehan1998} notion of \textit{processing competence} and \citegen[325--326]{VanMoere2012} \textit{processing efficiency}, defined as “the speed and accuracy with which a learner orally processes familiar language”, which in turn tends to “near effortless processing of language”, or automaticity \citep{DeKeyser2001}.

\subsubsection{Overextensions, DOM, markedness, and morphological complexity}
\label{sec:saturno:4.2.2}
In study 3 the DOM of plural nouns was one of the two structures in which the acquisition sequences of ESS and non-ESS diverged. Two explanations may be proposed. First, the gap may be attributed to (negative) transfer affecting ESS, but not non-Slavic respondents. Alternatively, the anomalously higher scores of non-ESS may reflect a structurally non-target-like form, which however happens to coincide with the expected target. To exemplify on L2 French, learner-produced verbal (or verb-like: \citealt[312]{KleinPerdue1997}) forms ending in [e], instantiating the basic word form of these lexical items, are homophonous with a variety of target language forms, some of which may be acceptable in a given context, like \textit{préparer} (infinitive), \textit{j’ai préparé} (passé composé \textsc{1sg}) and \textit{préparais} (imperfect indicative \textsc{1sg} and \textsc{2sg}) for [prepare] in \REF{ex:saturno:25}.

\ea\label{ex:saturno:25}  
\gll Je {[prepare]} le   por {[manʒe]}\\
I  prepare-?  \textsc{art}  {por (Spanish)}  eat-?  \\
\glt ‘I prepare/prepared something to eat.’
 \jambox*{(L2 French, \citealt[146]{BenazzoWatorek2021})}
\z

Turning back to the Polish data, in the early stages of acquisition the plural paradigm of a noun may oppose different values of number, but not case. The resulting “partially inflected” word form would probably be modelled on the nominative case due to its frequency and unmarkedness. Superficially target-like output like \textit{studentki} may then be the result of either target-like inflection, representing the morphological expression of number and case (iii. in \tabref{tab:saturno:4}), or nominative overextension, with only number being morphologically marked (ii. in \tabref{tab:saturno:4}).

\begin{table}
\begin{tabularx}{\textwidth}{QQl}

\lsptoprule

i. basic word form & ii. marked number & iii. marked number and case\\
\midrule
\multirow{4}{=}{\textit{studentk-a} (\textsc{sg)}} & \multirow{2}{=}{\textit{studentk-a} (\textsc{sg)}} & \textit{studentk-a} (\textsc{nom.sg)}\\
&  & \coloredcell{\textit{studentk-i} (\textsc{nom.pl)}}\\
& \coloredcell{} & \textit{studentk-ami} (\textsc{ins.pl)}\\
& \coloredcell{\multirow{-2}{=}{\textit{studentk-i} (\textsc{pl)}}} & \ldots\\
\lspbottomrule
\end{tabularx}
\caption{\label{tab:saturno:4}Alternative underlying interpretations of superficially correct \textsc{acc.pl} patterns in Polish}
\end{table}

Since in Polish the nominative and the accusative case of non-virile nouns are syncretic, the form [tuˈristki] in (\ref{ex:saturno:22}, reported here from \sectref{sec:saturno:3.2}) represents both the \textsc{nom.pl} and the \textsc{acc.pl} form. Based on the available data, it is thus impossible to state whether the learner intentionally produced the target accusative form, or simply relied on an under-inflected word-form that in this context happened to coincide with the expected target.

\begin{exe}%22
    \exr{ex:saturno:22}
    \gll {[tuˈristk-i}         ˈlubi  {niʥˈvjeʥ-${\emptyset}$]}\\
  female\_tourist-\textsc{acc.pl=nom.pl}    likes  bear-\textsc{nom.sg}\\
    \glt ‘It is the bear that likes the tourists.’\jambox*{(L2 Polish)}
\end{exe}

\subsection{Notes and limitations}\label{sec:saturno:4.3}

Since target items were selected among Slavic cognates, the Polish sentences in Study 1 closely resembled their Russian counterparts, from which they diverged only with respect to a few phonological and morphological details (e.g. position of the stress, vowel reduction, actual morphs instantiating inflectional morphemes, etc.). Moreover, these features had been the focus of a course in comparative grammar, a context that according to \citet{Paradis2009} could lead to the development of metalinguistic knowledge and affect the development of the L2 grammatical system (\citealt{FalkEtAl2015, Bardel2019}). This conclusion is apparently in contrast with the tenets of the Developmentally Moderated Transfer Hypothesis, which states that even identical traits cannot be transferred before the learner is developmentally ready to process them in the new language. However, authors working on the DMTH always make it very clear that their claims only concern spontaneous oral data, while study 1 made use of a structured task and studies 2 and 3 relied on an EIT. The former was indeed used in several PT-inspired works (e.g. \citealt{DiBiaseKawaguchi2002, BettoniDiBiase2011, ArtoniMagnani2015}). The EIT, though usually believed to represent a good approximation of spontaneous speech, is not considered a valid source of data by several authors working within the theory (\citealt{Pienemann2015, LantolfZhang2015}; discussion in \citealt{Saturno2019a}). Others, however, have successfully used the task to identify developmental stages (\citealt{BatenCornillie2019}). Altogether, it seems reasonable that because of the adopted methodology and the learners’ background, meta-linguistic knowledge may have played a role in study 1, in which the L2 Russian learners’ performance was quite remarkable in light of the limited input received and the great typological distance between the target language and their mother tongue (Italian). In contrast, the involvement of metalinguistic knowledge seems less conspicuous in studies 2 and 3, although in both cases participants were university students with at least some recent language learning experience.

Further, in all three studies, the statistical sample is not particularly large. This is especially true of studies 1 and 2, in which the output of a class-sized group was analysed. Despite the much larger number of respondents (161), study 3 is not ideal for the purposes of statistical analysis either because of the disproportion between Slavic and non-Slavic learners.\footnote{{This limitation can be attributed to the relative unpopularity of Polish as a foreign language among non-Slavic (in particular, non-Ukrainian) students. This situation in turn is reflected by the modest number of students of Polish outside Poland} {\citep{MiodunkaEtAl2018}}{, especially when compared to Russian, which is commonly taught at university and secondary school levels.}}

Finally, one could hypothesise a role for input-related correlates of the learners’ L1, such as opportunity to access native input, neighbourhood of residence, etc. \citep{Flege2009}. However, since all participants in study 3 were university students enrolled at the same institution, they are probably characterised by a relatively uniform profile and input basis. In any case, the available data do not make it possible to further pursue this question, which remains open for further research.

\section{Conclusion}\label{sec:saturno:5}

The studies discussed in this chapter suggest that SLA is rich in universal trends to a large extent independent of the learner’s L1. The same types of errors may be found in the output of Slavic and non-Slavic learners of L2 Polish. The most common pattern is the overextension of a basic, uninflected word-form to all accusative contexts, a tendency amply described in the literature on the acquisition of morphologically inflected languages. Further, Slavic and non-Slavic learners appear to share the same developmental path of Polish morphosyntax, again confirming a well-known acquisitional finding. 

At the same time, a beneficial effect of L1/L2 proximity is hard to deny, at least in terms of acquisition rate and communicative success following limited exposure to the input. Accuracy scores as measured by an EIT were decidedly higher in the case of Slavic learners. Further, some accusative forms were produced along the lines of another known language, from which transfer can reasonably be hypothesised. Indeed, probably based on such evidence, it is a common opinion among L2 Polish teachers that L1/L2 proximity may actually be detrimental to acquisition due to negative transfer and the risk of fossilization, the latter paradoxically resulting from the partial mutual intelligibility of Polish and East Slavic languages. Errors involving gender and DOM in L2 Polish are perceived as “glaring” and “highly disqualifying the idiolect”\footnote{ {Orig.} {\textit{błędy rażące i mocno dyskwalifikują idiolekt}}.} (\citealt[148--149]{Kravčuk2020}), but it seems that serious communicative disruptions can hardly occur because of them, if not perhaps in the domain of referent retrieval, given also the pro-drop character of Polish.\footnote{Following \citet{Martinet1967}, \citet[174--176]{FrajzyngierShay2003} argue that “efficiency of reference is indeed the main outcome of the presence of nominal classes” because of the “possibility to refer to a participant by means of an anaphora [including zero anaphora] instead of a full noun or a noun modified by one or more determiners”.} Adapting \citegen[144]{Mauranen2012} statement on \textit{English as a Lingua Franca}, one could describe the L2 produced by speakers of closely related languages as often “slightly wrong”, but also “approximately right”. 

This text aimed to make a contribution to the long-standing debate on language transfer by analysing L2 Polish data in light of the learners’ previous knowledge of other Slavic languages, a family that has been somewhat neglected by SLA research so far. It is the author’s hope that these reflections will prove useful to those concerned with the learning and teaching of Polish, a language that in the last few years has found itself at the core of large-scale, sometimes dramatic migration phenomena.

\section*{Acknowledgements} 

It is a great pleasure for me to contribute to this volume in honour of Daniel Véronique. I am especially grateful to him for a few fruitful exchanges we had early on in my career. Over time they provided me with much food for thought for my subsequent work, including the studies discussed in this chapter.

\section*{Abbreviations and definitions}

\begin{tabularx}{\textwidth}{@{}lQ@{}}
\textsc{adj} & adjective\\
\textsc{anim} & animate\\
Blr & Belorussian\\
\textsc{dat} & dative\\
DMTH & Developmentally Moderated Transfer Hypothesis\\
EIT & Elicited Imitation task\\
ESS & East Slavic speakers\\
\textsc{gen} & genitive\\
Ger & German\\
\textsc{imp} & imperative\\
\textsc{inanim} & inanimate\\
\textsc{ins} & instrumental\\
L1 & native language(s)\\
L2 & any non-native language in a wide sense, independently of the acquisition order (thus L2, L3 etc. in a narrow sense), unless explicitly stated otherwise\\
\textsc{loc} & locative\\
\textsc{nom} & nominative\\ 
\textsc{nonvir} & in Polish, non-virile gender (\sectref{sec:saturno:2.2.2})\\
\textsc{obj} & object\\
OS & object-subject\\
\textsc{part} & partitive\\
\textsc{pl} & plural\\
\textsc{pst} & past\\ 
Rus & Russian\\
\textsc{sg} & singular\\
SO & subject-object\\
\textsc{subj} & subject\\
TL & target language\\
V & verb\\
Ukr & Ukrainian\\
V2 & in Germanic languages, the placement of the finite verb in second position in declarative main clauses\\
\textsc{vir} & in Polish, virile gender (\sectref{sec:saturno:2.2.2})\\
\end{tabularx}

\sloppy\printbibliography[heading=subbibliography,notkeyword=this]
\end{document} 
