\documentclass[output=paper]{langscibook}
\ChapterDOI{10.5281/zenodo.10280596}


\author{Sandra Benazzo\orcid{}\affiliation{Université Paris 8} and     Christine Dimroth\orcid{}\affiliation{Universität Münster} and    Cecilia Andorno\orcid{}\affiliation{Università di Torino}}


\title[Back to the Basic Variety]{Back to the Basic Variety: Does it emerge only with specific learner profiles, environments and languages?}

\abstract{About thirty years ago, the thorough study of migrants' initial L2 acquisitional stages in the ESF project gave birth to the notion of the Basic Variety, a simple yet autonomous language system, efficient and well suited to many communicative purposes, which learners develop in the context of untutored acquisition in immersion. Our paper discusses this notion in the light of subsequent studies which adopted a similar view on learner varieties and applied it to different populations and learning situations. Our goal is to determine whether and to what extent the core features identified for the Basic Variety need to be adapted when different variables are considered, such as the instructional context, learners’ level of literacy, and the specificities of their source and target language.

\keywords{Basic Variety, L2 acquisition, literacy, instruction, source and target language typology}}


\IfFileExists{../localcommands.tex}{
  \addbibresource{../localbibliography.bib}
  % add all extra packages you need to load to this file

\usepackage{tabularx,multicol}
\usepackage{url}
\urlstyle{same}

\usepackage{listings}
\lstset{basicstyle=\ttfamily,tabsize=2,breaklines=true}

\usepackage{langsci-basic}
\usepackage{langsci-optional}
\usepackage{langsci-lgr}
\usepackage{langsci-osl}
% \usepackage{./langsci/styles/langsci-lgr}
% \usepackage{./langsci/styles/langsci-osl}
% \usepackage{langsci-gb4e}

\usepackage{tikz}
\usetikzlibrary{patterns,calc}
\pgfdeclarepatternformonly{south east lines}{\pgfqpoint{-0pt}{-0pt}}{\pgfqpoint{3pt}{3pt}}{\pgfqpoint{3pt}{3pt}}{
    \pgfsetlinewidth{0.6pt}
    \pgfpathmoveto{\pgfqpoint{0pt}{3pt}}
    \pgfpathlineto{\pgfqpoint{3pt}{0pt}}
    \pgfpathmoveto{\pgfqpoint{.2pt}{-.2pt}}
    \pgfpathlineto{\pgfqpoint{-.2pt}{.2pt}}
    \pgfpathmoveto{\pgfqpoint{3.2pt}{2.8pt}}
    \pgfpathlineto{\pgfqpoint{2.8pt}{3.2pt}}
    \pgfusepath{stroke}}
    
\usepackage{stmaryrd}
\usepackage{wasysym}
\usepackage{multirow}
\usepackage{caption}
\usepackage{subcaption}
\usepackage{mathrsfs}
\usepackage{qtree}

\usepackage{linguex}


  %pminos do not split footnotes
% \interfootnotelinepenalty=10000 %Footnote in Laporte chapters has to be split SN


%\DeclareIndexNameFormat{default}{%
%\nameparts{#1}%
%\usebibmacro{index:name}%
%{\index[names]}%
%{\namepartfamily}%
%{\namepartgiveni}%
% {}% L1
% {}% L2
%{\namepartprefix}% generates spurious space L3
%{\namepartsuffix}% generates spurious space L4
%}

%  {\DeclareIndexNameFormat{default}{%
%     \usebibmacro{index:name}{\index[names]}{#1}{#3}{#5}{#7}}}

%\DeclareIndexNameFormat{default}{%
%  \usebibmacro{index:name}{\sindex[nom]}{#1}{#3}{#5}{#7}}

%\DeclareIndexNameFormat{default}{%
%  \usebibmacro{index:name}{\sindex[person]}{#1}{#3}{#5}{#7}}
%\DeclareIndexNameFormat{default}{%
%\nameparts{#1} \usebibmacro{index:name}{\sindex[person]]}{\namepartfamily}{‌​\namepartgiven}{\nam‌​epartprefix}{\namepa‌​rtsuffix}}

%\newcommand{\smiley}{:)}

%\renewbibmacro*{index:name}[5]{%
%\usebibmacro{index:entry}{#1}%
%{\iffieldundef{usera}{}{\thefield{usera}\actualoperator}\mkbibindexname{#2}{#3}{#4}{#5}}}

% \newcommand{\noop}[1]{}

%remove for final
%\overfullrule=1mm

\newcommand{\tobi}[2]}}
\renewcommand{\S}[1]{\tobi{#1}{\textsc{*}}}

% this volume references
% puts: [this volume]
% already defined: \citetv
%\newcommand{\citepv}[1]{(\citeauthor{#1} \citeyear*{#1} [this volume])}
\newcommand{\citealtv}[1]{\citeauthor{#1} \citeyear*{#1} [this volume]}

%parentheses around example number
\newcommand{\pref}[1]{(\ref{#1})}

% in-text examples

\newcommand{\lnex}[1]{\textit{#1}} %target lang word
\newcommand{\lnlit}[1]{(lit.: `#1')} %literal reading
\newcommand{\lnlat}[1]{(#1)} % latinization
\newcommand{\lntrans}[1]{`#1'} %translation
\newcommand{\lnexl}[2]%
{\lnex{#1}{} \lnlat{#2}} % ex with latinization
\newcommand{\lnexlat}[3]{\lnex{#1}{} \lnlat{#2}{} \lntrans{#3}} % ex with latinization and tranl.

%ch01
\newcommand{\co}[1]{\mbox{\textbf{#1}}}

%ch09

\newcommand{\cyrbulg}[1]{\begin{otherlanguage*}{bulgarian}#1\end{otherlanguage*}}


%ch10
\newcommand{\nlp}{{\small NLP}}
\newcommand{\mwe}{{\small MWE}}
\newcommand{\rae}{{\small RAE}}
\newcommand{\lvc}{{\small LVC}}
\newcommand{\pos}{{\small P}o{\small S}}
%\newcommand{\todo}[1]{ \textcolor{red}{#1} }

%\renewcommand{\labelenumi}{\theenumi}
%\ainamefmt{{vv}{ll}{, ff}{, jj}} % fullname

\newcommand{\biberror}[1]{{\color{red}#1}}

\newcommand{\osenovaitem}{--~} 
  %% hyphenation points for line breaks
%% Normally, automatic hyphenation in LaTeX is very good
%% If a word is mis-hyphenated, add it to this file
%%
%% add information to TeX file before \begin{document} with:
%% %% hyphenation points for line breaks
%% Normally, automatic hyphenation in LaTeX is very good
%% If a word is mis-hyphenated, add it to this file
%%
%% add information to TeX file before \begin{document} with:
%% %% hyphenation points for line breaks
%% Normally, automatic hyphenation in LaTeX is very good
%% If a word is mis-hyphenated, add it to this file
%%
%% add information to TeX file before \begin{document} with:
%% \include{localhyphenation}
\hyphenation{
    Beck-man
    Ngu-yen
    back-chan-nel
    back-chan-nels
    mo-not-o-nous
    ste-reo-typ-i-cal
}

\hyphenation{
    Beck-man
    Ngu-yen
    back-chan-nel
    back-chan-nels
    mo-not-o-nous
    ste-reo-typ-i-cal
}

\hyphenation{
    Beck-man
    Ngu-yen
    back-chan-nel
    back-chan-nels
    mo-not-o-nous
    ste-reo-typ-i-cal
}
 
  \togglepaper[1]%%chapternumber
}{}

\begin{document}
\maketitle 

\section{Introduction}\label{sec:benazzo:1}

In the 1980s, the project \textit{Second Language Acquisition by Adult Immigrants}, also known as “ESF project” (with reference to the European Science Foundation as its funding institution) applied a crosslinguistic and longitudinal design to investigate the way in which foreign immigrant workers in industrialised European countries went about learning the language of their new social environment. Research done within this project led to a systematic description of the initial stages in untutored adult L2 acquisition. In particular, it showed that learners with different pairings of source (SL) and target languages (TL) initially develop a very similar linguistic system, called the “Basic Variety” (BV; \citealt{KleinPerdue1997}), in which utterances consisting of target language words are constructed on the basis of pragmatic and semantic principles such as \textit{Focus last\slash Agent first}, which are largely independent of the properties of source and target language. In spite of its formal limitations (e.g. no marking of case, number, gender, tense, aspect or agreement by morphology, absence of subordination), the BV was found to represent a simple and efficient means of communication, characterised by a transparent interplay between function and form. 

The ESF project set out to study initial L2 acquisition in a specific type of adult learners in immersion contexts, namely “[...] monolingual[s] with little or no initial knowledge of the TL, with little formal education in the SL and with no TL courses under way …” (\citealt[42]{Perdue1993vol1}). Even though not all participants of the ESF project fully matched this profile, this raises the question of whether the development of such a variety is related to a specific learner population and/or to particular acquisitional circumstances.\footnote{In addition to the ESF project, there were other research projects studying untutored L2 acquisition in a population of migrant workers with (typically) little education (cf. \citealt{Véronique2021Acquisition}). We include a list of projects and references below (without claiming completeness) for which one could, in principle, ask similar questions: Are the results limited to a specific learner profile or specific learning conditions?\label{fn:benazzo:1}

\begin{itemize}
\item Harvard project: English L2\slash Spanish L1 \citep{CazdenEtAl1975};
\item Heidelberger “Pidgin Deutsch”:  German L2\slash Italian or Spanish L1 (\citealt{KleinDittmar1979});
\item ZISA project: German L2\slash Italian, Spanish or Portuguese L1 (\citealt{MeiselPienemann1981, ClahsenEtAl1983});
\item P-Moll project: German L2\slash Polish or Italian L1 \citep{DittmarEtAl1990};
\item Pavia project: Italian L2\slash L1 Chinese, Tigrinya, German, English… (\citealt{Bernini1994, GiacaloneRamat2003}).
\end{itemize}}

In what follows, we report selected findings of some subsequent studies dealing with the initial stages of L2 development under a variety of conditions that use the notion of BV to interpret their results. The goal is to examine whether (or to what extent) extra-linguistic factors, such as learners’ general education, literacy and instruction, as well as linguistic factors, such as particular typological features of the target language or specific combinations of SL/TL, impede the emergence of a BV or have an impact on its characteristics. Some of these questions have been addressed by individual studies. We will try to discuss them in a more comprehensive perspective, without however any ambition of exhaustiveness. 

The paper is structured as follows: In \sectref{sec:benazzo:2}, we outline the approach taken by the ESF project, its goals, methods (with a focus on the learners’ profiles) and core findings, in particular concerning the main features of the BV. \sectref{sec:benazzo:3} presents our research questions and some methodological considerations. The results of our review are presented in \sectref{sec:benazzo:4}, where the impact of literacy, instruction, and source and target language properties are addressed in turn. The paper closes with a general discussion, some conclusions, and suggestions for further research in \sectref{sec:benazzo:5}.

\section{Background: The ESF project and the BV stage}\label{sec:benazzo:2}

\subsection{Aims and methodology}\label{sec:benazzo:2.1}

When the ESF project \citep{Perdue1993vol1} was conceived in the eighties, there was a strong interest in the acquisition of the local language by adults who had recently arrived in industrialised Western European countries for economic or political reasons (see footnote~\ref{fn:benazzo:1} for reference to a number of projects with similar goals). In these groups, second language acquisition was largely untutored: it took place through everyday interaction with the new linguistic environment rather than by tuition. Besides social issues concerning their integration, the study of adult migrants was therefore considered to provide a window on the natural development of an L2, driven by communication and not influenced by specific instructional approaches.

The ESF project stands out for its scale and scope. Thanks to the collaboration of six research teams, about 40 migrants, who had settled in five different European countries, were observed during a period of 30 months with the aim to: 

\begin{itemize}\sloppy
\item identify factors on which acquisition depends;
\item determine the overall structure of SLA processes (\textit{order of acquisition, speed and success});
\item understand the characteristics of the asymmetrical communication between native and non-native speakers (\textit{L2 language in use in} c\textit{ommunicative tasks}) \citep{Perdue1993vol1}.
\end{itemize}

The project adopted a longitudinal and crosslinguistic design with 10 different combinations of source and target languages (cf. \figref{fig:benazzo:1}), reflecting the sociolinguistic situation of migration in Europe at that time. The goal was (a) to study individual development over time, and (b) to disentangle common characteristics of L2 speech from the impact of specific (source or target) languages. Note, however, that all but one of the TLs belong to the Germanic language family. Moreover, according to typological studies, all TLs (including French) belong to the same \textit{Sprachbund} (identified as \textit{Standard Average European Bund}, cf. discussion in \citealt{Dahl1990}) and thus share many similarities at the morphosyntactic level. The project includes, however, combinations of both typologically distant and closely related languages (cf. Arabic vs. Spanish speaking learners of French L2). 

  
\begin{figure}
% \includegraphics[width=\textwidth]{figures/benazzo1.png}
\begin{tikzpicture}
  \node(punjabi)[inner sep=0mm,minimum width=2cm]{Punjabi\strut};
  \node(italian)[inner sep=0mm,minimum width=2cm,right =0cm of punjabi]{Italian\strut};
  \node(turkish)[inner sep=0mm,minimum width=2cm,right = 0cm of italian]{Turkish\strut};
  \node(arabic)[inner sep=0mm,minimum width=2cm,right = 0cm of turkish]{Arabic\strut};
  \node(spanish)[inner sep=0mm,minimum width=2cm,right = 0cm of arabic]{Spanish\strut};
  \node(finnish)[inner sep=0mm,minimum width=2cm,right = 0cm of spanish]{Finnish\strut};
%
  \node(english)[inner sep=0mm,minimum width=2cm,above of=punjabi, xshift=1cm]{English\strut};
  \node(german)[inner sep=0mm,minimum width=2cm,right = 0cm of english]{German\strut};
  \node(dutch)[inner sep=0mm,minimum width=2cm,right = 0cm of german]{Dutch\strut};
  \node(french)[inner sep=0mm,minimum width=2cm,right = 0cm of dutch]{French\strut};
  \node(swedish)[inner sep=0mm,minimum width=2cm,right = 0cm of french]{Swedish\strut};
  %
  \draw(punjabi.north)--(english.south);
  \draw(english.south)--(italian.north);
  \draw(italian.north)--(german.south);
  \draw(german.south)--(turkish.north);
  \draw(turkish.north)--(dutch.south);
  \draw(dutch.south)--(arabic.north);
  \draw(arabic.north)--(french.south);
  \draw(french.south)--(spanish.north);
  \draw(spanish.north)--(swedish.south);
  \draw(swedish.south)--(finnish.north);
\end{tikzpicture}


\caption{\label{fig:benazzo:1} Language pairs in the ESF project (TLs on top, SLs on bottom)}
\end{figure}

The following selection criteria had been formulated for the recruitment of participants: 

\begin{quote}
... she (or he) was to be monolingual with little or no initial knowledge of the TL, with little formal education in the SL and with no TL courses under way. S/he was to be aged between 18 and 30, not married with a TL speaker nor with children at school in the target country, whilst entering an environment where day to day contacts with the TL speakers were to be expected.\hbox{}\hfill\hbox{(\citealt[42]{Perdue1993vol1})}
\end{quote}

As it turned out to be difficult to find enough ideal participants, some compromises were necessary. As a consequence, the profile of the real participants was more heterogeneous. As can be seen in \tabref{tab:benazzo:1}, they present a variable length of stay in the country at the beginning of the project (from 1 month to about 1 year); none of them has a high formal education, but the number of schooling years varies from 0 for one participant, to 11 for another; and most of them followed some TL courses during the project. 

\begin{table}
\small
\begin{tabularx}{\textwidth}{llllrrlrrQ}
\lsptoprule
&  Sex &  SL &  TL & {Age} & {Stay} &  Fam & {SLScl} &  TLScl  & L3\\
&      &     &     & (yrs) & (mths) &    &  (yrs) & (hrs) &\\\midrule
{\itshape Madan} &  M &  Pun &  Eng &  25 &  19 & M &  6 &  0 & Hindi\\
{\itshape Ravinder} &  M &  Pun &  Eng &  21 &  12 & M &  7 &  150 & Hindi\\
{\itshape Andrea} &  M &  Ita &  Eng &  36 &  5 & M &  8 &  30 & ?\\
{\itshape Lavinia} &  F &  Ita &  Eng &  20 &  5 & M+1 &  8 &  600+ & ?\\
{\itshape Santo} &  M &  Ita &  Eng &  25 &  7 & S &  8 &  0 & ?\\
{\itshape Angelina} &  F &  Ita &  Ger &  21 &  12 & M+2 &  10 &  0 & ?\\
{\itshape Gina} &  F &  Ita &  Ger &  18 &  1 & S &  11 &  50 & ?\\
{\itshape Marcello} &  M &  Ita &  Ger &  23 &  9 & S &  11 &  0 & ?\\
{\itshape Tino} &  M &  Ita &  Ger &  20 &  9 & S &  8 &  0 & ?\\
{\itshape Ayshe} &  F &  Tur &  Ger &  17 &  4 & S &  6 &  500+ & rud. Eng\\
{\itshape Çevdet} &  M &  Tur &  Ger &  16 &  8 & S &  9 &  500+ & ~-\\
{\itshape Ilhami} &  M &  Tur &  Ger &  17 &  10 & S &  8 &  500+ & ~-\\
{\itshape Ergün} &  M &  Tur &  Dut &  18 &  11 & S &  5 &  60+ & ~-\\
{\itshape Mahmut} &  M &  Tur &  Dut &  20 &  9 & M &  5 &  0 & ~-\\
{\itshape Fatima} &  F &  MoA &  Dut &  25 &  12 & M &  2 &  70 & ~-\\
{\itshape Mohamed} &  M &  MoA &  Dut &  19 &  8 & S &  6 &  0 & rud. Fr\\
{\itshape Abdelmalek} &  M &  MoA &  Fre &  20 &  13 & S &  1 &  15 & rud. Sp\\
{\itshape Zahra} &  F &  MoA &  Fre &  34 &  13 & M+4 &  0 &  30 & ~-\\
{\itshape Alfonso}  &  M &  Spa &  Fre &  32 &  10 & M+2 &  6 &  180+ & ~-\\
{\itshape Berta}  &  F &  Spa &  Fre &  31 &  1 & M+3 &  8 &  180+ & ~-\\
{\itshape Paula}  &  F &  Spa &  Fre &  32 &  2 & M+2 &  6 &  180+ & ~-\\
{\itshape Fernando} &  M &  Spa &  Swe &  34 &  5 & M+2 &  9 &  400+ & rud. Eng\\
{\itshape Nora} &  F &  Spa &  Swe &  39 &  10 & M+3 &  6 &  600+ & ~-\\
{\itshape Leo} &  M &  Fin &  Swe &  18 &  4 & S &  11 &  400+ & Eng\\
{\itshape Mari} &  F &  Fin &  Swe &  22 &  10 & M &  6 &  600+ & rud. Eng\\
{\itshape Rauni} &  F &  Fin &  Swe &  29 &  7 & S &  8 &  300− & rud. Eng\\
\lspbottomrule
\end{tabularx}
\caption{\label{tab:benazzo:1}Sociobiographic profile of learners in the ESF project \citep[46]{Perdue1993vol1}. List of abbreviations: Fam (family status): S (single), M (married) + number of children; SLScl (source language schooling); TLscl (target language schooling), which is given in estimated hours, with +/− indicating ‘probably more/less than’; rud. indicates a rudimentary command of additional L2s. Stay is given in months.}
\end{table}


During the observation period, learners were regularly recorded while accomplishing a series of communicative tasks. Each session started with a free conversation (\textit{personal interview}), which was followed by different tasks (\textit{role plays, film retellings, picture descriptions}, \textit{route directions} …). The database\footnote{\url{http://www.mpi.nl/world/tg/lapp/esf/esf.html}} thus consists of a large sample of different types of spoken L2 discourse. 

\subsection{The developmental stages and the Basic Variety} \label{sec:benazzo:2.2}
\begin{sloppypar}
The theoretical position developed in the project is the “learner variety approach”, a functional approach to interlanguage according to which learner varieties are not considered as the imperfect imitation of the target language (as in the typical error analysis), “but systems in their own right, error-free by definition” (\citealt[308]{KleinPerdue1997}), just as other varieties of a language that sociolinguistics describe as relatively stable linguistic codes belonging to the repertoire of particular groups of speakers (dialects, for example). Learner varieties were studied and described as any other unknown variety of a language in order to capture their individual formal and functional properties. As learners develop a series of subsequent varieties, these are also referred to as acquisitional stages, the last ones corresponding to the fully-fledged varieties spoken by adult native speakers.
\end{sloppypar}

Researchers adopting a learner variety approach therefore avoid attributing the status and functions of TL categories to linguistic units produced by learners on the basis of their formal resemblance (\textit{closeness} \textit{fallacy}).  

The analyses applying this perspective led to the identification of three main stages of untutored second language acquisition, each one characterised by a specific linguistic repertoire, type of utterance organisation and communicative potential (\citealt{KleinPerdue1992, DietrichEtAl1995}).

At the first stage (\textit{Prebasic variety}), besides some formulaic chunks, the learners’ repertoire consists of lexemes roughly corresponding to TL-like bare nouns, adjectives, and adverbs with a sound-meaning correspondence. These items are combined with the help of a pragmatic principle based on information structure: constituents having background status precede those that have focus status (“focus last” principle), as in the following example taken from \citet{Véronique2013dislocation}: \textit{moi /li/ bar} ‘I li bar’ (roughly ‘I am\slash work at the bar’). Given the scarcity of clear verb forms at this stage, utterance organisation is defined as ‘nominal’. Learners’ oral production is heavily context-dependent and relies on the interlocutor’s scaffolding. 


The emergence of verb argument organization is the most important change leading to the next stage (\textit{Basic Variety}). Even though verb forms still lack functional morphology,\footnote{This does not exclude the use of inflected forms (for example, present tense forms), but either there is only one form occurring in different grammatical contexts, or several forms are in free variation.} the utterance is now structured by the verb and its arguments, which are ordered according to agentivity. The semantic principle “Agent first” is thus added to the previous pragmatic principle (cf. \textit{de mädch gucke de mann mit brot} ‘the girl look the man with bread’, \citealt[319]{KleinPerdue1997}). In spite of its formal simplicity, the BV system allows learners a certain discursive autonomy. Note that, for one third of all learners observed in the ESF project, development stopped at this stage: they enriched their lexical repertoire without altering the rules of the BV system.

\begin{sloppypar}
For the learners going beyond this stage (\textit{Postbasic varieties}), progress is marked by the development of functional verb morphology: the utterance is organised around a finite verb which progressively encodes the functional values of Tense, Mode, Aspect (TMA) and Agreement (a syntactic function of the subject). Discourse organisation also becomes more complex with the emergence of syntactic subordination. At this stage, learners start developing the specific features of the language to be learned and, at the same time, they show more SL influence. There are, however, also general tendencies shared by all learners. In particular, the emergence of finite verbs is gradual: the appearance of free morphemes, like auxiliaries and modals (\textit{initial Postbasic stage}) precedes the functional use of bound morphemes.
\end{sloppypar}

The stages as defined in the ESF project focus on the commonalities attested among the learner varieties of different TLs and therefore the variation attested in specific language pairs was not described in detail. Furthermore, inherent to the notion of variety, each stage was described through its core features (for example, a crucial feature is the presence\slash absence of functional verb morphology), with some variation at the periphery. The transition between different stages was not claimed to be clearcut, as organizing principles may overlap.\footnote{Subsequent studies have proposed quantitative criteria to operationalize these notions (cf. \citealt{BartningSchlyter2004, Pallotti2007} among others).} Learner varieties were rather distinguished on the basis of their characteristic properties. They should be seen as categories with central, prototypical instances and more peripheral cases (an analogy proposed by \citealt{Berthele2021} for other concepts in SLA).

The BV has been characterized as “a simple communicative system”, “largely (though not totally) independent of the specifics of S/TL organisation”, “with a transparent interplay of forms and functions” (\citealt[303--304]{KleinPerdue1997}), in which a lexicon, made of TL-like nouns, adjectives, adverbs and verbs is organised by pragmatic and semantic principles (\textit{focus last/agent first}).

It only includes a few function words, among which an item for negation and some focus particles, an elementary system of deictic pronouns to refer to the speaker and the hearer, and an anaphoric pronoun for third person reference, unspecified for gender or number. On the whole, the system lacks grammatical inflections, be they nominal or verbal. As \citet[311]{KleinPerdue1997} put it, there is “no marking of case, number, gender, tense, aspect, agreement by morphology”. Despite these restrictions, the BV turns out to be “simple, versatile and highly efficient for most communicative purposes” (\citeyear[303]{KleinPerdue1997}), which is probably one of the reasons explaining the fossilisation of some learners’ grammar at this stage. 

The following excerpts from ESF transcripts are meant to give an impression of the functioning of this variety in English and French as TLs. In \REF{ex:benazzo:1}, Andrea, an Italian learner of English L2, retells his daily routine. Verbs appear in their stem form. Nevertheless, the learner can easily localise situations in time with adverbial expressions (\textit{8 o’clock, 9 o’clock, after X}) and the principle of natural order; verbs like \textit{start} and \textit{finish} are used to mark the left and right boundary of the situations mentioned. 

\ea%1
    \label{ex:benazzo:1}
    \gllllll {\normalfont AN\hspace{2ex}}  \textit{I get up 8 o’clock}\\
            {} \textit{take coffee} \\
      {}  \textit{wash}\\
    {} \textit{after underground}\\
    {} \textit{9 o’clock in er + work start} \\
    {} \textit{half past two finish}\\
    \z

The next excerpt was produced by another Italian learner, Santo, a fluent speaker of the BV. This longer stretch of conversation illustrates his discursive autonomy in producing a rather complex explanation of his choice to go on holiday in December instead of September\footnote{ {Transcription conventions:  *...* enclose items in the SL and, in the French L2 examples, brackets [..] report the phonetic transcription of the relevant segments.}}.

\ea%2
    \label{ex:benazzo:2}
    \gllllllll {\normalfont SA}\hphantom{Speakerw}  \textit{me for holiday er no september} \\
        {} \textit{because er ++ er *se* (=if) I go in september for holiday} \\
    {} \textit{no possible christmas} \\
    {} \textit{you understand?} \\
    {} \textit{… i no like london christmas…} \\
    {} \textit{er last christmas in london} \\
    {} \textit{and next + in my country…..} \\
    {} \textit{when you when holiday you?}   \\
    
    \gll {\normalfont TL Speaker:}    \textit{I’ve just had a holiday} \\
  SA  \textit{and when you going another one?} \\
  \z

Besides the use of adverbial expressions for temporality, \REF{ex:benazzo:2} also shows some other typical features of the BV, like the preverbal position of the negator (\textit{I no like}) and the use of \textit{possible/ no possible} compensating the absence of modals. As in \REF{ex:benazzo:1}, most verbs appear in the stem form (\textit{like, understand, go}). The occasional presence of the suffix -\textit{ing} suggests it is not yet functional.

The following examples are taken from learners of French at the same level, having either Spanish \REF{ex:benazzo:3} or Arabic \REF{ex:benazzo:4} as their L1.

\ea%3
    \label{ex:benazzo:3}
    \ea\label{ex:benazzo:3a} (About Berta’s first skiing experience) \\
    \gllll {\normalfont BE}\hspace{2ex}  {\textit{[se]} \textit{très très très dur *por primera* fois}}\\
      {} {‘its very very hard for the first time’}\\
      {}  {\textit{[nEpa] possible} \textit{je} \textit{[mõte] *sobre* les [eski]}}\\
      {} {‘not possible I get on the skis’}\\

    \ex\label{ex:benazzo:3b} (About Berta’s working situation)\\
    \gll {\normalfont BE}\hspace{2ex}  \textit{avant je [travaj] maintenant non}\\
        {} {‘before I work, now no’}\\
    \ex\label{ex:benazzo:3c} (About Paula’s morning)  \\
    \gllll {\normalfont PA\hspace{2ex}} \textit{je [prepare] le … [mandZe]}\\
     {} {‘I prepare the … food’}\\
    {} {\textit{*y* je  [sorti] à 1h30}}\\
    {} {‘and I go out at 1.30’}\\
\z
\z
    
\ea%4
    \label{ex:benazzo:4}
    (Zarah tells the story of her arrival in France)\footnote{{Example taken from \citet[42--43]{Véronique2000}.}}\\
\gllllllll {\normalfont ZA}   \textit{avant euh /+ [e] mon mari [travaj] pas}\\
     {} {`before is/and my husband work not'}\\
    {} \textit{[e] mon mari [travaj]}\\
     {} {`is/and my husband work'}\\
    {} \textit{[e] [jãna] [igane] pas le sous beaucoup (…)}\\
    {} {`is/and there is he earn not money much'}\\
    {TL Speaker:} {(...) \textit{pendant ce temps-là toi tu étais où?}}\\
    {}  {`during that time where were you?'}\\
    
\gllllllll {\normalfont ZA}\hphantom{Speaker}  \textit{[res] le maroc avec les enfants}\\
    {} {`stay Morocco with the children'}\\
    {} \textit{et après mon mari [ekrije]} \\
    {} {`and then my husband write'}\\
    {} \textit{et [le domãd] avec le passeport à moi}\\
    {} {`and he ask with the passport of me'}\\
    {} \textit{moi je [revjen] à la france touriste}\\
    {} {`me I come France tourist'}\\
\z

In French L2, all verb forms were transcribed in phonetics. This choice was due to the opacity of French oral verb morphology, as the same phoneme can correspond to many different written verb forms.\footnote{For example, the [e] suffix in [paʁle] (‘speak’) can be interpreted as a mark of the infinitive (parler), but also of the past particle (parlé), or the 2nd person plural in present tense or imperative (parlez); the [i] suffix in [fini] can correspond to the present (1st, 2nd, or 3rd person), but also to the past participle of the verb \textit{finir} (‘finish’).}

As can be seen in \REF{ex:benazzo:3} and \REF{ex:benazzo:4}, the first verb forms in French L2 are marked by more morphological variation in comparison to English: Some verbs appear in a short form corresponding to the verb stem (V-0 like \textit{/travaj/, /revjen/}), others in long forms, either with [e] or [i] endings (V-e like /\textit{mõte/, /prepare/, /demande/}; V-i like /\textit{sorti/}); in addition, the Moroccan learners show some morphological variation in the area preceding the verb root (\textit{∅ / i / le}, cf. \citealt[48]{Véronique2000}). 

Learners’ production possibly reflects some sensitivity to morphological variation in the input, but in the BV these forms are not yet functionally opposed, i.e. form-function mappings are not yet operational. As shown for the English example in \REF{ex:benazzo:2}, the temporal anchoring of the situations is provided by adverbs and context (cf. ex. \ref{ex:benazzo:3}b--c, \ref{ex:benazzo:4} in French). Among the commonalities between the excerpts in both TLs, note also the use of (\textit{nepa}) \textit{possible} in \REF{ex:benazzo:3a}, instead of the modal \textit{pouvoir (‘be able to'),} as in \REF{ex:benazzo:2}.

Some learners just increase the lexical repertoire of the BV and become fluent speakers of this variety (e.g. Santo) without modifying its principles, whereas for others the BV is just a transitory stage in their development of the L2. Despite its overall communicative efficiency, the BV also presents clear limitations. There are discourse configurations that cannot be expressed without violating one of the pragmatic or semantic principles that characterize the BV (for. ex. Agent in focus); moreover, speaking the BV is socially stigmatizing. These might be strong incentives for learners to develop new formal means which are typical of the postbasic variety.

\section{Research questions and method}\label{sec:benazzo:3}

The ESF project aimed at investigating the emergence of common acquisitional paths in untutored L2 acquisition that are valid despite the specificities of the different SLs and TLs. To this end, the learners’ socio-biographical conditions were meant to be kept equal, while varying the SLs/TLs pairs. The longitudinal collection of comparable speech data from 40 adult migrants in five countries was undoubtedly a huge logistical challenge that the project accomplished with great success. The learners’ socio-biographical profiles were, however, slightly more varied than planned and the typological diversity of the five TLs included in the corpus was still relatively limited (an observation already present -- at least with respect to syntactic properties -- in \citealt{Schwartz1997}). 

The impact of learners’ socio-biographical characteristics was not investigated as such, with the exception of a study on lexical development (cf. \citealt{vanHoutStrömqvist1993}). Most studies were rather oriented towards the identification of developmental features common to the different combinations of SLs/TLs. The broad stages identified seem thus to be valid for adult migrants who have little formal education and learn the target language mainly through immersion. The question arises, however, whether the core features of the BV are also attested with learners whose profile, learning environment and TLs differ from the ones that were originally investigated. 

In this paper our aim is thus to:

\begin{itemize}
\item discuss the variation in the learners’ background that was present to a certain extent but not addressed as such;
\item explore the impact of TLs that are typologically more varied than those originally included. 
\end{itemize}

For doing so, we do not analyse new data, but rather review findings from studies conducted after the ESF project which (a) analyse the initial stages of L2 development under conditions that are partly different from those encountered by the ESF learners, and (b) use the developmental stages identified in the framework of the ESF project to interpret their results. Our comparison is mainly focused on the verbal domain, as the lack of finite verbal morphosyntax is crucial for the definition of the BV. More precisely, we discuss selected studies that examine the role of the following factors in the emergence of the BV:

\begin{itemize}
\item [a.]
the learners’ educational background, which may affect their metalinguistic awareness. RQ: does the level of \textit{literacy} influence the acquisitional path? To address this point, we examine some recent studies including immigrant learners with different degrees of literacy (low vs. non-literate).


\item [b.]   the learners’ exposure to TL \textit{instruction}, which may increase the possibility of noticing morphosyntactic structures. RQ: Is there a BV-stage in classroom SLA? Quite a few studies deal with the initial stages of L2 learning in the classroom. We will describe two of them asking to which extent BV features arise in university students exposed to tutored acquisition.

\item [c.]   the \textit{typological features} of the target language and specific L1\slash L2 combinations, which can provide learners different cues on the structure of the TL. RQ: To what extent do typological characteristics of the target language shape the form of the learner variety? Is there a BV stage in case the learners’ source language is very similar to their target language? For these questions, we will focus on studies concerning Italian as a TL. In comparison to the TLs considered in the ESF project, Italian shows a higher degree of morphological transparency and salience. 
\end{itemize}

Before turning to our discussion, two caveats must be added: on the one hand, for space reasons we had to exclude the consideration of other factors (in particular, the age factor\footnote{Although child L2 learners apparently do not systematically produce BV-like structures, the question whether in particular the acquisition of morphology is entirely different in nature, or just much faster, is a matter of debate. For example, \citegen{SchlyterThomas2012} study on Swedish L1 child acquisition of L2 French (initial exposure to the TL: between 3;5 and 6;7) highlights the presence of an initial stage with non-finite verb forms like the one attested for adult L2 learners, but such a stage seems to be comparatively shorter. Concerning syntax, \citet{SchimkeDimroth2018} show that child L2 learners of German pass through a BV-like stage in which utterance structure is determined by the semantic lightness rather than the finiteness of verbs.}) which are also highly relevant for our research question; on the other hand, the isolation of the three factors mentioned above is partly artificial, as the individual studies we present, even if focussed on one of the factors, very often do not allow the exclusion of others.

\section{Studies investigating initial L2 acquisition in other learner populations and acquisitional circumstances}\label{sec:benazzo:4}
\subsection{The role of Literacy}\label{sec:benazzo:4.1}

Although immigrants and refugees with limited formal education represent a high percentage of L2 learners worldwide, after the European projects of the 1980s, little attention has been paid to their acquisition of a new language: as summarised by \citet{Young-Scholten2013}, most studies in SLA have been (and are still) based on a population of highly educated learners, such as middle-class secondary school and university students. 

In the last decade, however, there has been a renewed interest in learners’ development of a second language in relation to their level of literacy (cf. the studies by the LESLLA\footnote{The LESLLA acronym initially stood for \textit{Low Educated Second Language and Literacy Acquisition for Adults}; in 2017 it was changed to \textit{Literacy Education and Second Language Learning for Adults} (cf. \url{http://www.leslla.org/}).} network since 2005). The impact of this factor, which has been largely ignored in previous research, is particularly relevant for the recent immigrant populations who, although often multilingual, frequently include adults with little or no schooling in their native language. Researchers belonging to the LESLLA network have highlighted the need to take into account the literacy variable both for pedagogical purposes and for estimating the reliability of previous L2 research results (cf. among others \citealt{vandeCraatsEtAl2006, Young-Scholten2013, Tarone2014}). 

A provocative paper by \citet{BigelowTarone2004} defends the idea that the study of low-/no literate learners might modify the actual picture of L2 acquisitional sequences, as this picture is mainly based on the observation of “literate adult L2 learners, child L2 learners, or learners with unverified L1 literacy skills” (\citeyear[695]{BigelowTarone2004}). The authors stress that even the old projects on migrants with little education (ESF, ZISA) neglected this variable, as “researchers apparently did not establish how literate their informants were” (\citeyear[695]{BigelowTarone2004}). 

This factor was indeed not controlled in the ESF project: the participants were working class migrants with a variable number of schooling years, but they did not undergo an independent test on literacy (reported schooling years do not necessarily coincide with literacy level). Even if some studies on highly educated learners confirm the developmental sequences attested in the ESF project (see, for example, \citealt{Schlyter2000, BartningSchlyter2004, Granget2017} for the acquisition of verb morphology in L2 French), it is legitimate to ask whether (or to what extent) literacy can make a difference. With respect to the ESF sequences, one could hypothesize, for example, a correlation between low literacy and the fossilization at the BV level, or the presence of different stages depending on the literacy variable, as suggested by \citet{BigelowTarone2004} and \citet{TaroneEtAl2013}.     

Literacy is a complex construct, which has an important impact on the development of linguistic skills both in L1 and in L2 (cf. \citealt{Hulstijn2011} among others). For our purposes, i.e. determining the role of literacy for the L2 acquisition of oral competences, we only refer to literacy in the narrow sense of using an alphabetic script for reading and writing. Previous research has shown that the acquisition of the ability to decode an alphabetic script (thus establishing grapheme/phoneme correspondences) affects the ability to process oral language in terms of identifying discrete segments (phonemes and words) in the speech stream. For example, illiterate adults (and even adults literate in a non-alphabetic script, for ex. Chinese) perform worse than literate ones in tasks demanding the manipulation of individual phonemes (such as the deletion/ replacement of initial or final phonemes) or the repetition of long non-words (cf. \citealt{TaroneEtAl2013} for a review of several studies). Inversely, non-literate or low-literate learners might display specific strengths at other aspects of L2 speech, like rhyme or prosody (cf. \citealt{Maffia2016}). 

Although L1 literacy seems to affect certain aspects of L2 oral processing, its possible role in the development of L2 grammatical structures is less straightforward. The few studies dealing with this topic present two different theoretical perspectives and interpretations.

On the one hand, Tarone and colleagues take a cognitive viewpoint in which literacy is supposed to play a crucial role. Starting from the idea that only alphabetic literacy provides strategies to process language irrespective of semantic content, they assume that learners without alphabetic literacy will have particular difficulties with the acquisition of function words \citep{TaroneEtAl2005}. To verify this hypothesis, \citet{TaroneEtAl2007} analyse the performance in English L2 of adult and adolescent Somali immigrants with low to moderate literacy, who are supposed to be at the same level in the TL according to their production of English questions. Three different tasks were used to respectively measure learners’ ability to notice and recall grammatical corrections (oral recasts on their erroneous question forms), to repeat long L2 sentences (elicited imitation), and to produce L2 grammatical forms (oral narratives). The authors found a correlation between the degree of literacy and learners’ performance in the first and second task, whereas the results of the third task are not so clearcut. As they put it:  

\begin{quote}
alphabetic print literacy affects oral L2 processing and use: it affects the recall of oral recasts of grammatical errors, and it affects accuracy in decontextualized elicited imitation tasks. Our data are less conclusive in suggesting that alphabetic literacy may even affect the grammatical forms used in oral narratives.\hbox{}\hfill\hbox{\citep[117]{TaroneEtAl2007}} 
\end{quote}

On the other hand, \citet{VainikkaEtAl2017} deal with this topic from a linguistic perspective, within a generativist framework, and are more cautious about the role of literacy \textit{per se.} They suggest that differences between literate and low or non-literate learners might be related to external factors, namely the latter’s poorer exposure to target language input (i.e. low or non-literates have to rely mainly or exclusively on aural input). 

Their study on Arabic-, Somali- and Urdu-speaking adults with varying levels of literacy in their L1 shows that they all follow the same acquisitional path in English L2. The illiterate learners tend, however, to overgeneralize multi-word sequences unrelated to the verb head to express morphosyntactic functions (for ex. \textit{in the} to mark progressive aspect, as in \textit{in the eat}). According to the authors, the presence of such sequences confirms that all learners are able to identify function words in the L2 input and to use them, but the sub-patterns attested in illiterates could be attributed to “a greater reliance on auditory as compared to visual memory”. 


Despite their different theoretical standpoints, researchers seem to agree that L2 competence is acquired more slowly by low- or no-literate than educated literate learners. At the same time, two difficulties become apparent: (a) how to disentangle the effect of limited literacy from other factors related to it\footnote{ {Both points emerged also in the ESF project in a study measuring the richness of learners’ vocabulary over time with respect to their variable socio-biographical characteristics (\citealt{vanHoutStrömqvist1993}). The results      showed the impact of} {\textit{Age and family status}} {on the one hand, and} {\textit{Education}} {on the other: the oldest learners, married, with children, who had the least number of schooling years of education scored comparably lower on the lexical measures. They were, however, also the ones with less contact to native speakers.}}, such as low exposure in contexts of social marginalisation or low access to written texts, and (b) how to compare the two studies, as they use different criteria to establish the learners’ developmental stage.


For our purposes, the study of \citet{Mocciaro2019, Mocciaro2020} is particularly relevant on both dimensions: she describes the development of L2 Italian morphosyntax by low or non-literate adult learners observed longitudinally, with reference to the prebasic, basic, and postbasic stages as identified in previous studies for Italian L2. The subjects are 20 young migrants, aged between 18 and 30, newly arrived in Southern Italy from West Africa and Bangladesh, with a variable level of literacy in their L1, but all sharing a similar type of (very limited) exposure to the target language: they had not been exposed to Italian before their arrival in Italy (10 to 21 months before the first interview), and since their arrival, their interactions in Italian were reduced to sporadic language courses. After a language and literacy test, they were divided into 3 groups (no literacy, poor literacy and literacy) and observed longitudinally over 13 months: during this period, they were tested 5 times in individual sessions (at a time interval of 3 to 6 months) with interviews and narrative tasks.  

At the beginning of the observation period, all learners were at a pre-basic or basic level; at the end, most of them had entered the initial post-basic variety, independently of the literacy variable. The unequal progression at the end of the observation period is attributed to TL exposure \citep{Mocciaro2019}, as learners going beyond the BV had benefitted from slightly higher-quality input (participation in courses and internships, or more frequent interactions with Italian peers). Input seems to affect the rate of development but not the acquisitional sequences, that are equivalent to those already identified for L2 Italian. Importantly, literacy does not seem to play a major role. As Mocciaro puts it:\largerpage 

\begin{quote}
literacy does not affect either the route or the rate of interlanguage development, as literate and low/non literate learners appear to follow the same path, both in terms of direction and results of the process […] However, learners’ degree of literacy may act in a subtler way, favouring the development of specific sub-patterns, namely non target analytical constructions, which appear to be more linked to the interlanguages of learners with limited literacy\hbox{}\hfill\hbox{\citep[171]{Mocciaro2020}}
\end{quote}

The idiosyncratic structures mentioned by Mocciaro appear in the transition beyond the BV, which is attested on independent criteria (e.g. productive use of the suffix -\textit{ato} to form past participles): they correspond to light verbs used as carriers of grammatical information -- \textit{fare} ‘do’ constructions, as in ex. \REF{ex:benazzo:5}; \textit{essere} ‘be’ constructions, as in ex. \REF{ex:benazzo:6} --, or prepositions (\textit{per/come}) used as precursors of subordination.

\ea%5
    \label{ex:benazzo:5}
    \gll {TL speaker:}   {\textit{la ragazza cos’ha fatto ?} (ex. taken from \citealt{Mocciaro2019})}\\
         {}   {The girl what has done ? (‘What did the girl do?’)}\\

    \gllll Ha     fatto     mangiare, anche\\
    have:3\textsc{sc}   do:\textsc{pst.ptcp}   eat:\textsc{inf} also\\
    Ha     fatto     prende libro\\
    have:3\textsc{sc}   do:\textsc{pst.ptcp}   take:3\textsc{sc} book\\
    \glt (Target form: ‘ha mangiato, ha anche preso un libro’)
\ex%6
    \label{ex:benazzo:6}
    \gll non è     continua   a lavorare\\
    not be:3\textsc{sc}  continue:\textsc{inf}   to work:\textsc{inf}\\
    \glt (Target form: ‘non ho continuato a lavorare’) 
\z

The presence of similar analytic constructions has already been attested in L2 Italian (cf. \sectref{sec:benazzo:4.3}, and in particular \citealt{Bernini2003}) as well as in other target languages (\citealt{BenazzoStarren2007, Starren2001}): they are typical of the initial post-basic stage, when learners have identified some functional elements in the input, but still use them in an idiosyncratic way. The selected forms are usually free morphemes which, in comparison to bound morphemes, can be more easily perceived in the input (phonologically more salient) and are more transparent at the functional-semantic level (auxiliaries, light verbs with a lexical counterpart, and prepositions which also express concrete spatial meanings). 

Note that in Mocciaro’s data all three groups of learners overgeneralize such functional forms. In other words, analytic constructions represent a natural path towards the acquisition of the Italian morphosyntax. However, they are more frequently selected by low-/non-literate learners, whereas literate learners use them in a more sporadic and transient way. 

In the discussion of her results, \citet[20]{Mocciaro2019} subscribes to \citegen{VainikkaEtAl2017} interpretation of similar interlanguage constructions in L2 English. Low-/non-literate learners’ stronger preference for lexical-syntactic strategies to build the L2 grammar might be attributed to their higher reliance on aural stimuli -- although this preference “does not alter the overall route they follow in developing grammar”. 

Coming back to the initial question, it seems that literacy \textit{per se} does not affect the initial steps of the acquisitional process: an oral BV system for communicative purposes is attested in both literate and low/non-literate adults. Progress beyond BV is possible for both populations: it is related to the quality and quantity of input rather than to literacy. Literacy provides, however, at least an additional type of input that may affect the transition to the subsequent stages: poor access to written input might favour a longer reliance on idiosyncratic lexical-syntactic strategies to express grammatical meanings. 

\subsection{The impact of instruction}\label{sec:benazzo:4.2}

The researchers of the ESF project characterised the BV as a type of linguistic organisation that untutored L2 learners construct on the basis of naturalistic input, i.e. some form of social interaction with TL speakers. As a consequence, foreign language classrooms were not considered a promising environment for the study of basic learner varieties. A handbook paper by \citet[567]{Klein2000}, for example, ends with a section entitled \textit{Weshalb findet sich die Basisvarietät nicht bei Kindern oder im Unterricht?} (‘Why is there no BV in children or in the classroom?’). In relation to classroom learning, the author finds that the question is not difficult to answer: Instructed acquisition is not guided by the learners’ natural language faculty, but by a particular syllabus presented in the classroom. A system like the BV, efficient as it might be for communication, is neither taught nor tolerated because it deviates from the norms of the target language, and it is therefore unlikely to be observable in instructed learning. 

What Klein had in mind is probably students in traditional foreign language classrooms rather than immigrants in situations of L2 immersion that are nowadays often accompanied by some form of supporting instruction. Classroom foreign language instruction can be focused on form or meaning and it typically includes written input, carefully pronounced and repetitive spoken input as well as some meta-information about the structure of the target language. In addition, classrooms present an environment in which (artificial) second language communication takes place for the mere purpose of learning. What matters is accuracy, complexity and fluency of second language use \citep{Housen2021}, rather than communicative success in social interaction. Even teachers relying on communicative approaches will encourage learners to incorporate TL properties exceeding the BV repertoire (e.g. some morphological distinctions). More importantly, they will make sure that there is no big gap between the learners’ L2 means and their communicative tasks (e.g. classroom exercises), so that a “correct” solution is in principle within the reach of the learners. 

However, there have been repeated reports about instructed learners following independent acquisitional sequences that seemed to be rather immune to didactic interventions (evidence of this sort is reported, amongst others, by \citealt{DiehlEtAl2000}, \citealt{Schlyter2000}, \citealt{Sun2003}, \citealt{BartningSchlyter2004}, \citealt{Granget2015, Granget2017}). Self-dynamic and partly autonomous development can apparently happen in foreign language classrooms and the production of BV-style utterances can be observed even after many years of instruction when the contexts of conversation are unfamiliar to the learners. The exclusion of instruction contexts from BV research might therefore be premature. In the following, we will address the question in a more nuanced way and ask whether and under which circumstances classroom learners without access to naturalistic input and social interaction can nevertheless resort to (important traits of) a BV system. We will review studies of instructed second language learning that have adopted a learner varieties framework and explicitly address the question of whether a BV can emerge in adult L2 learners under classroom conditions.  

The first study was conducted in a rather atypical classroom situation that was set up for experimental purposes in the framework of the VILLA project (\textit{Varieties of Initial Learners in Language Acquisition}). The second study investigated the acquisition of French as a foreign language by Japanese students following a rather traditional language course.\footnote{As a reviewer rightly points out, the constellation of languages covered by these studies also addresses the other factors discussed in the current paper (the background of the learners and the typological distance between the languages involved). We simply do not have enough studies yet that would allow to fully isolate all the variables.}

\subsubsection{Study 1: VILLA -- between mere exposure and instruction}\label{sec:benazzo:4.2.1}

For this study, groups of novice adult L2 learners with L1 Dutch, English, French, German, or Italian attended a series of 10 classroom sessions in Polish (14 hours in total) that were entirely recorded. The learners’ input was kept as constant as possible, it was monolingual and nearly exclusively oral and no meta-linguistic information about the target language was provided. Learners were not allowed to take notes or use written materials like grammar books or dictionaries. Comprehension and production experiments investigating the learners’ growing knowledge of Polish phonology, lexis, morphology, and syntax were repeatedly administered in a longitudinal design. Discourse production data were collected once after the last input session.\footnote{ {An outline of the VILLA methodology, including details about all tests, can be found in \citet{DimrothEtAl2013}, and a compilation of results from perception, comprehension, imitation, and judgement experiments is available in \citet{WatorekEtAl2017} and \citet{Saturno2016, Saturno2020}.}} Comparisons of the learner data with properties of the input that they had encountered made it possible to distinguish structures that the learners had autonomously developed from those that had frequently occurred in the input.

Two studies (\citealt{Saturno2020}\footnote{See also \citet{SaturnoWatorek2020}.} and \citealt{Dimroth2018}) explicitly addressed the question whether a BV would develop under such circumstances. \citet{Saturno2020} studied the relative weight that learners assigned to morphological case marking and word order in a sentence imitation task (controlled production), a picture-sentence matching task (comprehension), and a dialogue exercise (semi-spon\-ta\-ne\-ous production). The study found evidence for a strict word order scheme (SVO), but also for early knowledge of morphological oppositions (nominative and accusative case marking on nouns) that had occurred frequently in the input.  Importantly, though, most learners performed more accurately in the comprehension and production experiments than in the semi-spontaneous interactional task. In semi-spon\-ta\-ne\-ous production, they largely omitted case morphology. This did not hinder communication, however, since learners relied on semantic and syntactic means like animacy contrasts and word order (the BV principle “Agent first”) in order to achieve their communicative goals.

\citet{Dimroth2018} focused on learners’ discourse production and looked for structures reflecting an autonomous application of BV principles in film retellings elicited from groups with L1 Italian and German. The analyses revealed a mixed pattern with the syntactic and morphological properties shown in \tabref{ex:benazzo:2}. The three rightmost columns indicate whether the relevant structure was attested in both learner groups (L1 Italian and L1 German), whether it corresponds to a BV principle, and whether it was highly frequent in the learners’ classroom input and could thus be due to imitation.

The learners’ schematic agent-verb-patient order corresponds to the Basic Variety (“agent first”), albeit without fully exploiting its word order flexibility: Despite suitable contexts, learners barely used presentational utterances instantiating the “focus last” constraint. The learners uniformly produced preverbal negation although negation had a low input frequency and the L1 of the German group has a different surface word order (negation follows the finite verb). Operators (like the negation particle) regularly preceding their scope (the verb and its complements) are also characteristic for the BV of untutored learners with different SL-TL combinations (\citealt{PerdueEtAl2002, Andorno2005, Bernini2005, GiulianoVéronique2005, Dimroth2008}).\largerpage[-2]\pagebreak

\begin{sidewaystable}[H]
\begin{tabularx}{\textwidth}{lQQccc}
\lsptoprule
        &  &  & {In both}  &      & \\
        &  &  & L1         & {BV} & {High}\\
 Domain &  {Polish (survey’s TL)} & {Findings} & groups  & principle & frequency\\\midrule
\multicolumn{2}{l}{\itshape Syntax} \\
word order & default SVO & agent first & + & + & +\\
negation   & preverbal & neg [predicate] & + & + & −\\\addlinespace
\multicolumn{2}{l}{\itshape Morphology}\\
subject-verb agreement & distinct suffixes for 6 person/number contexts & inflected base forms in 3\textsuperscript{rd} person contexts & + & + & +\\
number                 & complex plural morphology depending on gender, case and animacy & some plural forms of nouns and pronouns (partly creative) & + & − & −\\
\lspbottomrule
\end{tabularx}
\caption{\label{tab:benazzo:2}Utterance structure in narratives elicited from adult learners (L1 Italian and German) in the VILLA project}
\end{sidewaystable}
\pagebreak

The morphological properties of the learners’ production presented a mixed pattern. The high proportion of inflected verb forms was not interpreted as a productive expression of finiteness, but rather as resulting from an imitation of highly frequent input forms. This was supported by the observation that learners sometimes combined two 3\textsuperscript{rd} person verb forms in one clause (“Man makes sleeps”). The situation was, however, slightly different with morphological number marking. Whereas some learners would stick to lexical markings (“two fireman“), others produced some target-like plural forms of nouns and personal pronouns, or reinterpreted case suffixes (“long forms”) as nominal plural markers. Very rarely, there were even attempts of number agreement on verbs (3\textsuperscript{rd} person plural). Given that the absence of functional inflectional morphology is one of the core features of the BV, these learners were thus working on properties indicative of a more elaborate post-basic variety, even though the structure of their utterances was in accordance with BV-principles in other domains. The most interesting property in this respect is probably negation. Despite comparably low input frequency, BV-like preverbal negation was observed even in the learners with L1 German (postverbal negation with finite verbs). 

Retelling a story after only 14 hours of exposure to a new target language is a challenging task. The learners relied on chunks and BV principles, but they also showed some attempts to express particular functions with the help of inflectional morphology (see above). The VILLA input differed from natural interaction in that it consisted of carefully pronounced speech that was constantly linked to meaning components (e.g. via pictures) and contained a great number of repetitions. The learners were university students with a high degree of literacy (compared to the ESF population). They were highly motivated to pay attention to the structure of the target language and to identify some of the regularities. Even under these conditions, however, they produced structures akin to the BV. Structures going beyond a BV repertoire were mainly observed in experimental tasks, but some traces were also visible in the production data investigated in the VILLA project. \citet[140]{Saturno2020} proposed the concept of an “Instructed BV” to capture the specific combination of basic principles and a capacity for tentative form-meaning associations of morphological features that are typical for more advanced developmental stages in untutored L2 acquisition.

\subsubsection{Study 2: French as a foreign language in Japan – traditional teaching}\label{sec:benazzo:4.2.2}

\citet{Kerrou2019} studied the expression of temporality in the oral production of university students of French Literature and Linguistics at the University of Tohoku, Japan. In contrast to the VILLA project, instruction was rather traditional and focused on written input and grammar lessons. The students thus had little experience with the production of connected discourse. \citet{Kerrou2019} elicited oral narratives (film retellings) in L2 French and metalinguistic reflections (interviews) in Japanese from students in the 2\textsuperscript{nd}, 3\textsuperscript{rd} and 4\textsuperscript{th} year. 

The macro-structure of the narratives produced by the 2\textsuperscript{nd} year students ($N=5$) followed the Principle of Natural Order, i.e. events were reported in the order in which they occurred in the film, and utterances were organised according to the Agent first principle. Utterance structure reflected the BV’s non-finite utterance organisation as illustrated for the untutored acquisition of French in \sectref{sec:benazzo:2} above.\footnote{Note that similar findings are also attested in classroom learners whose L1 is typologically more similar to their TL. In a study on adolescent classroom learners of French with L1 English, \citet[106]{Granget2017} found learner varieties with characteristic BV properties after as much as 320 hours of exposure.} Until the end of the 2\textsuperscript{nd} year, students did not produce any light verbs (no copula, no modal verbs, no auxiliaries). Their lexical verbs alternated between V-0 (bare stems), and verbs ending in /e/ or /i/ (\textit{V-é/V-i}) without a functional correlate. For purposes of illustration, \tabref{tab:benazzo:3} (adapted from Table~22 in \citealt{Kerrou2019}) lists the structures attested in the second of three film retellings collected during the learners’ 2\textsuperscript{nd} year of studying French.

\begin{table}
\begin{tabular}{lccccc}
\lsptoprule
Subject & {CHIA}\footnote{All learners except TOMO had 150 teaching hours.} & {CHIM} & {ERI}  & {MISA}  & {TOMO}\footnote{The learner TOMO followed 100 hrs of private courses at Alliance Française and had a total of 250 hrs.}\\\midrule
{\itshape être (copula)} & 0 & 0 & 0 & 0 & 0\\
{\itshape avoir (possession)} & 0 & 0 & 0 & 0 & 0\\
{\itshape existential il y a} & 0 & 0 & 1 & 0 & 0\\
{\itshape Lex verb: V-é/V-i} & 4 & 6 & 3 & 4 & 5\\
{\itshape Lex verb: V-0} & 2 & 1 & 3 & 2 & 4\\
\textit{Lex verb: infinitive}\footnote{{According to} Table~19 from \citet[166]{Kerrou2019}, this {category refers to infinitives of verbs of the 2}{\textsuperscript{nd}} {and 3}{\textsuperscript{rd}} {inflection class (}{\textit{tenir, boire}}{), whereas} {infinitives of -}{\textit{er}} {verbs (}{\textit{danser}}{) are counted under the} {\textit{V-é}} {category.}} & 0 & 1 & 0 & 0 & 1\\
{\itshape Modal + Lex verb} & 0 & 0 & 0 & 0 & 0\\
{\itshape Aux + Lex verb} & 0 & 0 & 0 & 0 & 10\\\addlinespace
{Total of verb forms}  & 6 & 8 & 7 & 6 & 20\\
\lspbottomrule
\end{tabular}
\caption{\label{tab:benazzo:3}Verb forms in elicited in narratives from university students of French in Japan}
\end{table}

In the metalinguistic interviews, the students reported that they had experienced a lack of (access to) lexical items in French as well as difficulties to produce the right forms of verbs despite declarative knowledge about a multitude of French verb forms (\textit{passé simple, passé composé, imparfait}, conditional, subjunctive). Some students recalled that they were silently repeating verbal paradigms they had learned in the classroom in order to retrieve the right form.

As a result of the traditional teaching method focusing on form and written input/tasks, the Japanese university students had difficulties proceduralizing their knowledge of French verb morphology during real time speech production in a complex oral task. In this situation, their production resembled the structure of utterances produced by the VILLA learners and by untutored BV speakers.\largerpage

With respect to the question whether instructed classroom learners can develop a BV, we can thus conclude from both studies that the nature of the task has a huge impact on the findings. Complex verbal tasks simultaneously require the conceptualization of the intended message, the retrieval of lexical items, the production of appropriate word forms, the composition of utterance units, and the organisation of a coherent macro-structure. Overburdened by these requirements, beginner and inexperienced classroom learners focus on the most urgent communicative needs and devote their attentional resources to the processing and production of meaning, rather than form (\citealt{SkehanFoster2001, VanPattenBenati2015, Saturno2020}). 

When\largerpage{} classroom learners are under communicative pressure because the production task clearly exceeds the available resources (VILLA) or because they have little experience in retrieving available knowledge in real time (French as a foreign language in Japan), certain features of the BV regularly emerge. However, the structures underlying the learners’ oral discourse production do probably not represent the type of stable knowledge state that was found with the untutored BV speakers from the ESF project. Whereas the latter were relatively fluent speakers of the BV and had no other, more advanced, variety of the target language at their disposal, the participants of the classroom studies summarized above did not have enough experience with spontaneous discourse production to become fluent users of their (basic) learner varieties.  In addition, the classroom learners’ high degree of conscious control and meta-linguistic awareness led to the impression of struggle and failure (as witnessed by the reports of the Japanese students of French), leading to an avoidance of overly demanding communicative tasks rather than an entrenchment of the simplified varieties. The untutored participants of the ESF project, on the other hand, could not avoid challenging communicative situations and some of them learned to make maximal use of their simplified varieties. Fluency and relative advanced comprehension skills contributed to their communicative success that might at the same time have fostered stabilisation for some of them.

\subsection{The role of the source and the target language}\label{sec:benazzo:4.3}
\largerpage
As we have seen in \sectref{sec:benazzo:2}, the research leading to the notion of a BV included several target and source languages: this research design aimed at identifying general tendencies in L2 acquisition. The BV is namely not meant as the simplified variety of a specific TL, but rather as a bundle of internally coherent pragmatic, semantic and linguistic properties, implemented with the lexicon of specific TLs. SL influence, although occasionally observed, does not significantly alter the basic properties of BV. Indeed, within the two possible interpretations of the ‘interlanguage continuum’ that have crossed the whole SLA research field since Selinker’ seminal work \citep{Corder1978}, the ESF results contributed to a view of L2 acquisition as a process of ‘reconstructing’ a specific language system from language-independent (pragmatic, semantic) principles, rather than a ‘restructuring’ process having the SL as its starting point and the TL as its driving force. 

However, as already observed in \sectref{sec:benazzo:2.2}, the original ESF design in the linguistic sample has not been exempt from criticism. Four out of five TLs included in the project (Dutch, English, German, Swedish) belong to the same language family (Indo-European Germanic), and all of them (including French) belong to the so-called Standard Average European \textit{Sprachbund} \citep{Dahl1990}. From a typological point of view, all TLs considered share important properties: they are all fusional – although to different degrees –, non pro-drop and have sentences organised around rather rigid syntactic rules (SVO or V2 in main sentences). The SLs are more varied: they belong to different language families (Indo-European Romance: Italian, Spanish; Indo-European Indo-Iranian: Punjabi; Afro-Asiatic Semitic: Arabic; Uralic Finno-Ugric: Finnish; Turkic: Turkish), have different word order organisation (SVO: Italian, Spanish, Moroccan Arabic, Finnish; SOV: Punjabi, Turkish) and morphological systems (fusional: Italian, Spanish, Arabic; or agglutinative: Finnish, Punjabi, Turkish). Still, one important linguistic type, namely isolating languages, has not been included. 

Although not undermining the perspective of a “reconstruction continuum”, some of the SL/TL properties might have affected the findings of the ESF project, i.e. the developmental patterns over different acquisitional stages as well as the shape of the specific stage named Basic Variety. Indeed, in the light of the relevance assigned by subsequent research to TL features in input processing, and to the possible role of the SL as a “filter” for attentional resources with respect to different linguistic cues (\citealt{VanPatten2015, EllisCollins2009}), the “universal” properties attributed to the BV should be evaluated against the specificities of further SL/TL pairs. Such an enterprise is far beyond the possibilities of this contribution and would deserve a whole new research project. In the current paragraph, we will consider the case of learner varieties of Italian, a language that has been investigated under the perspective of the BV in the so-called Pavia Project \citep{GiacaloneRamat2003} as well as in subsequent studies concerning similar learning conditions (mostly untutored acquisition; see footnote \textit{b} in \tabref{tab:benazzo:4} for further details) and similar types of data (oral communicative tasks and interviews). A schematic overview of these studies is provided in \tabref{tab:benazzo:4}.

\begin{sidewaystable}
\footnotesize\tabcolsep=.75\tabcolsep
\begin{tabularx}{\textwidth}{lcQQcc}
\lsptoprule
&  $N$ &  Source languages (only first languages) &  Learning conditions\footnote{{{Learners from the Pavia corpus, \citet{Mocciaro2020} and \citet{Vietti2005} are adult immigrants living in Italy (for Mocciaro see also \sectref{sec:benazzo:4.1}); \citet{WhittleLyster2016} consider Chinese children having lived in China until age 4--5, then migrated in Italy and attending Italian schools since then; \citet{LupicaSpagnoloForthcoming} observed immigrants who have temporarily lived in Italy during their migration journey through Europe, and eventually acquired Italian as a} }{{\textit{lingua franca} }}{{among immigrant communities; \citet{Schmid1994} observed Spanish immigrants having learnt Italian as a} }{{\textit{lingua franca}}}{ {among foreign workers in Swiss; \citegen{Caruana2003} learners are Maltese students having acquired Italian through naturalistic (not school-induced) exposure to the Italian TV in Malta.}}} & \multicolumn{1}{>{\raggedright\arraybackslash}p{\widthof{(years)}}}{Age (years)} &  \multicolumn{1}{>{\raggedright\arraybackslash}p{\widthof{(months)\textsuperscript{b}}}}{Exposure (months)\footnote{Length of exposure at the beginning of the data collection. In \citet{Caruana2003} exposure was not measured in length but rather in frequency (hours of TV watching per day).}}\\\midrule
 \citealt{GiacaloneRamat2003} &  20 &  Albanian, (Moroccan) Arabic, Cantonese, Mandarin and Wù Chinese, Chichewa, English, German, Malay, Morè, Tigrinya &  exposure to (non/)native speakers after migration &  12--45 &  1--48\\
 \citealt{Vietti2005} &  15 &  Peruvian Spanish &  exposure to (non/)native speakers after migration &  22--55 &  4--120\\
 \citealt{Mocciaro2020} &  20 &  Many, from Niger-Congo, Afroasiatic and Indo-Arian families &  exposure to (non/)native speakers after migration; short language courses &  18--30 &  10--21\\
 \citealt{WhittleLyster2016} &  14 &  Chinese (many dialects) &  exposure to (non/)native speakers after migration; public school attendance &  7--8 &  12--36\\
 \citeauthor{LupicaSpagnoloForthcoming} forthc. &  19 &  Mandinka &  exposure to (non/)native speakers after migration; currently not living in Italy &  20 &  0--36\\
 \citealt{Schmid1994} &  14 &  Castillan Spanish, Catalan &  exposure to (non/)native speakers after migration in German-speaking country (Switzerland) &  18--50 &  17--288\\
 \citealt{Caruana2003} &  26 &  Maltese &  exposure to TV input (no captions) &  25 &  --\\
\lspbottomrule
\end{tabularx}
\caption{\label{tab:benazzo:4}Socio-biographic features of learners in L2 Italian studies}
\end{sidewaystable}

\begin{sloppypar}
From a wider typological perspective, Italian -- a Romance language like French -- shares an important number of similarities with the TLs included in the ESF sample; still, unlike French, its inclusion in the core group of SAE languages is a matter of debate (see \citealt{Dahl1990, HeineKuteva2006}), for both syntactic and morphological reasons. More precisely, Italian can work as a first testing tool for the BV construct because some of its morpho-syntactic specificities allow us to examine relevant features attributed to the BV. One further reason for considering the case of L2 Italian lies in the fact that, among the SLs considered in the above mentioned studies, both distant SLs, from different typological types, including isolating languages (Chinese), and a very similar SL such as Spanish have been included.
\end{sloppypar} 

In the following, we will first discuss morphological and syntactic properties of Italian crucially differing from the ESF TLs sample, and their consequences for the development of a BV of Italian. We will then consider differences observed in learners from different SLs, and the specific case of the SL Spanish\slash TL Italian pair.\largerpage

Although all TLs considered in the ESF project are fusional as well, Italian shows a more transparent, salient and pervasive inflection: details of relevant features are given in \tabref{tab:benazzo:5} (examples mostly concern verb forms, whose development in leaner varieties will be considered later on).

\begin{table}
\small
\begin{tabularx}{\textwidth}{QQ}
\lsptoprule
{Target Languages in the ESF project} & {Italian as a Target Language}\\\midrule
\multicolumn{2}{c}{−\hfill Transparency\hfill +}\\
morphological affixes: suffixal and internal (apophony: GER, ENG; ENG. present/past verb forms (\textit{play/played} but \textit{speak/spoken}); ENG. singular/plural noun forms (\textit{hand/hands} but \textit{foot/feet})) & mostly suffixal (very limited apophony)\footnote{{Italian shows cases of verbal apophony in some verb paradigm, mostly limited to the} {\textit{passato remoto}} {forms} {(Present} {\textit{vedo –} }{Passato Remoto} {\textit{vidi}}{), a tense rather absent in the northern varieties of Italian, which constitute} {the native input}{ }{for the learners in the Pavia corpus}.}\\
less stable form-function relations  (syncretism: oral FR. [‘parl] (many persons of \textit{indicatif} and \textit{subjonctif} \textit{présent}) vs. [par’le] (2P \textit{présent}, many persons of \textit{imparfait}, 1P of \textit{passé simple}, \textit{participe passé}, \textit{infinitif}); ENG. \textit{{}-s} (plural of nouns, 3P of verbs, possessive); FR plural often perceived only on the determiner (\textit{l’arbre\slash les arbres})) & {more stable form-function relations}  (rare syncretism)\footnote{Italian shows some cases of syncretism between Present Indicative and Subjunctive, limited to Singular Persons; all other verb forms are distinct from each other. Number (and Gender) values are marked by different forms within articles, nouns and adjective inflection.}\\
\multicolumn{2}{c}{−\hfill Salience\hfill +}\\

consonants and central vowels as suffixes (verbal inflection: GER \textit{mache} [maxə] vs. \textit{machst} [maxst] \textit{macht} [maxt] vs. \textit{machte} [maxtə] vs. \textit{machen} [maxən]; ENG: \textit{play} vs. \textit{plays}) & {vowels and syllables as suffixes} (verbal inflection: Presente \textit{parlo, parli, parla, parliamo, parlate, parlano}; Past Participle: \textit{parlato}; Imperfetto: \textit{parlava, parlavi…})\\
many bare forms (nouns: FR, ENG: singular vs. plural; verbs: ENG: all persons but 3\textsuperscript{rd} sg.; oral FR: 4 persons out of 6) & {bare forms absent} {(no root forms; all forms carry at least a paradigmatic vowel)}\\
\multicolumn{2}{c}{−\hfill Pervasiveness\hfill +}\\

inflection limited to some word classes (number inflection: oral FR: in articles but not in adjectives, irregular in nouns; ENG: in nouns but not in adjectives and articles; person inflection: ENG: only in present tense) & inflection systematic for all verb forms and noun phrase components\\
\lspbottomrule
\end{tabularx}
\caption{\label{tab:benazzo:5}Morphological properties of TLs in the ESF project and of Italian}
\end{table}

These morphological differences can play a role in determining some of the properties observed for the BV. The BV in the ESF studies was described as deprived of inflectional morphology; possible variation in the form of lexical items does not carry any functional value: “lexical items typically occur in one invariant form. It corresponds to the stem, the infinitive or the nominative in the target language; but it can also be a form which would be an inflected form in the target language. Occasionally, a word shows up in more than one form, but this (rare) variation does not seem to have any functional value” (\citealt{KleinPerdue1997}: 311). Functional values such as tense-aspect-modality are rather expressed via lexical items; 1/3 of the learners did not develop any morphological repertoire even after several months. 

Instead, all learners of Italian in the Pavia project but one (Hagos, who was only observed in his first 7 months of stay, cf. \citealt{BanfiBernini2003}) developed a first nucleus of inflectional oppositions based on suffixes (with aspectual function: V-\textit{vowel} unmarked vs. V\textit{{}-to} for perfective marking); and 16 out of 20 develop their repertoire further, including functionally motivated forms of \textit{Infinitive}, \textit{Imperfect}, or even \textit{Conditional} and \textit{Future} (\citealt{BanfiBernini2003}). In the \citet{Mocciaro2020} corpus, 4/5 of the learners show the same aspectual opposition, and more than a half (11 out of 20) develop their repertoire further. The same happens among untutored learners only exposed to input from Italian television \citep{Caruana2003}: the first aspectual opposition showed up in all learners, and 22 out of 26 developed their repertoire further. This does not mean that a BV deprived of morphological distinctions is not attested at all in L2 Italian: apart from the above mentioned case of Hagos in the Pavia Corpus, 4 learners out of 20 in the Mocciaro corpus do not show any use of clear morphological oppositions even at the end of the observation period. The same is true for at least some of the learners observed in \citet{LupicaSpagnoloForthcoming}. It is worth noting, however, that \REF{ex:benazzo:1} all these cases are characterized by learning situations implying a very limited contact with native speakers; and that \REF{ex:benazzo:2} even in exposure-deprived situations – such as the one investigated by Lupica Spagnolo –, most learners developed at least some suffixal verbal distinction.

We can compare this situation with that attested in French L2 data studied in the ESF project by Daniel Véronique and Colette Noyau \citep{NoyauEtAl1995}: an opposition between bare V-forms vs. V-[e] forms, possibly exhibiting aspectual (but also modal) values, developed in learners within a period of exposure of 2--3 years, and no clear further development in the domain of verbal suffixes. As a whole, the authors conclude: “Systematic morphological distinctions are either hard to establish, or limited to the very last stages of development. […] Although there is a certain variability in the prefixing and suffixing of verbal lexemes […] there is no early, uniform tendency to forge functional distinctions with them. Moreover, the distinctions discerned at the end of the period of observation vary from learner to learner” (\citealt{NoyauEtAl1995}: 205). What is most interesting, in at least 3 of these learners the role of auxiliaries in expressing tempo-aspectual values precedes or competes with that of verbal suffixes: we will come back to this later.

At the syntactic level, other differences show up. The TLs included in the ESF project are all organised around formal syntactic rules: V2 (GER, SWE, DUT), or a rather rigid SVO (FR, ENG). In all of these languages finiteness plays a crucial role in many respects. Verb morphology is often marked by finite light verbs (auxiliaries, copula) signalling temporal and modal values. These forms can be separated from the non-finite verb forms and carry an individual accent (GER, DUT). Moreover, the position of negation, modal and tempo-aspectual adverbs depends on the finite verb, with the most salient (or unique) negator placed in the post-Vfin position (cf. \tabref{tab:benazzo:6}), and changes in the finite verb position characterise subordinated (GER, SWE, DUT), negative and interrogative sentences (FR, ENG, GER, DUT).

\begin{table}
\begin{tabular}{llllllll} 
\lsptoprule
    &  &  & Fin. verb & Post-fin. & \multicolumn{1}{l}{Non-fin.} & & \\
    &  &  &           & position  & Verb\\\midrule
GER & Ich &  & {\itshape habe}  & {\bfseries\scshape nicht} &  & Kuchen & {\itshape gebacken} \\
    &     &  & {\itshape will}  & {\bfseries noch}          &  & ein Bier & {\itshape trinken}\\
    &     &  & {\itshape trinke} &                          &  & Bier & \\\addlinespace
ENG & I &  & {\itshape have}    & {\bfseries\scshape not} & {\itshape seen} & the car\\
    &   &  &  \textit{do}       &                         & \textit{see}  & \\
    &   &  & {\itshape would} & {\bfseries still} & \textit{buy}  &  & \\\addlinespace
FRE & Je & (\textbf{ne}) & {\itshape suis} & {\bfseries\scshape pas} & {\itshape allé} & chez  \\
    &    &               & {\itshape ai} & {\bfseries rien} & \textit{vu} & lui & \\
    &    &               & {\itshape ai} & {\bfseries toujours} & {\itshape mangé} \\
    &    &               & {\itshape mange}   &  & \\
\lspbottomrule
\end{tabular}
\caption{\label{tab:benazzo:6}Negation, tempo-aspectual adverbs and finite/non-finite verb in German, French and English}
\end{table}

Italian is a pro-drop language and has a rather flexible SVO order (unlike DUT, GER, SWE, it has no fixed position for the verb; unlike English and French, VS inversion is possible); therefore, its sentence structure is more compatible with pragmatic and semantic needs. Italian has finite light verb forms as the ESF TLs, but unlike these languages the finite and non-finite verb form a rather strong syntactic and prosodic unit: auxiliaries are rarely independently marked by a pitch accent, even in contrastive contexts (\citealt{TurcoEtAl2013,TurcoEtAl2015, AndornoCrocco2018}); unlike the ESF TLs, the general negation \textit{non} does not occur in post-finite but in pre-verbal position, irrespective of the verb-type, and only very specific subclasses of adverbs (phasal adverbs \textit{ancora}, \textit{già, mai, più, sempre,} equivalent to Eng. \textit{again, already, never, not anymore, always}) can interrupt the finite + non-finite verb unit. As a consequence, the distinction between finite and non-finite verb forms does not play a similarly relevant role in the organisation of the sentence (\tabref{tab:benazzo:7}).

\begin{table}
\tabcolsep=.75\tabcolsep
\begin{tabular}{lllllll}
\lsptoprule
    &  &  & Fin. verb & Post-fin. position & \multicolumn{1}{l}{Non-fin. verb} & \\\midrule
ITA & (io) & {\bfseries\scshape non} & \textit{ho compro} & {\bfseries più} & \textit{visto} & l’auto niente\\
    &      &                         & \textit{voglio} & {\bfseries ancora} & {\itshape comprare} & una torta\\
\lspbottomrule
\end{tabular}
\caption{\label{tab:benazzo:7}Negation, tempo-aspectual adverbs and finite/non-finite verb in Italian}
\end{table}

In the description of the relevant properties of the BV, the absence of finiteness and finite verbs goes hand in hand with a sentence organisation based on pragmatic and semantic properties (topic first, focus last; agent before the verb; negation before negated constituent(s)). The development of finite forms, together with the appearance of the first morphological distinctions, cause a deep and long-lasting reorganisation of the sentence structure: departing from the principles mentioned above, learners develop formal syntactic ones (subject before finite verb; negation and aspectual adverbs after finite verb; GER, DUT, SWE: finite verb in second position, non-finite verb in final position) (cf. \sectref{sec:benazzo:2} and for a more recent overview \citealt{DimrothToappear}). Possibly because of all the steps needed for the development of a structurally coherent post-BVs sentence organization, not only untutored and low-literacy learners, but also tutored secondary school or university students (cf. \citealt{Kerrou2019} on French L2 by Japanese learners; \citealt{Granget2017} on French L2 by English learners; \citealt{Winkler2017} on German L2 by Italian students) take a long time to develop a fluent use of finite verb forms and to integrate them in the sentence structure. 

In L2 Italian, as we have seen, verb inflection first develops through suffixes on lexical verbs; this synthetic inflection co-occurs with the development of (possibly non-target) analytical verb forms, composed of a lexical verb together with a separate item carrying temporal, aspectual or personal values (as in: \textit{io ero parlare}, roughly ‘I was speak’, cf. \citealt{Bernini2003, Mocciaro2020}; cf. \sectref{sec:benazzo:4.1}). The development of functional verb forms (copula, auxiliaries) causes a partial restructuration of the sentence structure, but as only specific subsystems are concerned by this phenomenon (e.g. the position of phasal adverbs: BV: \textit{io sempre vado} → postBV \textit{io vado sempre} ‘I always go’, cf. \citealt{Andorno2005, Bernini2005}), it does not impact the organisation of the learner variety as a whole. 

Taken together, these findings suggest that finiteness and finite verbs could play different roles in the development of learner varieties, depending on the TLs involved. Finiteness and syntactic structure go hand in hand in the development after the BV stage observed in the ESF project. The BV in the relevant TLs has been characterised by a parallel development in the morphological and syntactic domains. It could, however, be the case that at least some of these features do not necessarily coalesce when TL systems are taken into account in which finiteness is less central for sentence organization. Although the BV model allowed to highlight peculiar aspect of the development of L2 Italian (e.g. the development of the copula and the phasal adverbs position), the development of verb morphology and of sentence structure in L2 Italian do not seem equally strictly intertwined.

The observations developed so far do not speak against the existence of a BV in L2 Italian. We rather see the need to stimulate a more careful discussion of the features that can be ascribed to the BV as a TL-independent construct. We will come back to this point in the conclusions.

When compared with this general picture, results from the specific subgroup of Chinese learners of Italian suggest that the SL can play a role as well. Within the learners included in the corpora mentioned above, Chinese learners show the slowest developmental rate throughout the morphological system of the TL. Base forms – not root forms, but rather inflected yet functionally unmarked forms – appear in Chinese learners even after years of exposure (\citealt{Banfi1990, BerrettaCrotta1991}); some adopt lexical items as the main way to express morphological values (\citealt{MassarielloMerzagora1990}); even young learners, who generally show more dynamic systems, show a morphological repertoire reduced to the distinction between a base form and the aspectually marked V-\textit{to} \citep{Valentini1992} and need form-focussed instruction to develop further (\citealt{WhittleLyster2016}).

Neither of these studies adopted an experimental protocol allowing to “measure” the relative speed of acquisitional paths with respect to all the variables involved; however, it is relevant to observe that 2 out of the 3 Chinese speakers in the Pavia corpus, and the Chinese learners in Whittle \& Lyster’s study are young learners, for which a fast development is expected; and yet, they show the least developed morphological repertoire. Chinese as an isolating language could in this case have an effect in “blinding” the learners’ attention for the inflectional features of the TL, and rather direct their attention towards alternative (lexical) cues to express the values conveyed by Italian verbal morphology. Such a behaviour has also been observed in learners from different SLs (\citealt{VanPatten2015}), but it is stronger in learners with an isolating SL (\citealt{EllisSagarra2010}) and can show up in Chinese learners of Italian despite the apparent facilitating role of the transparency, salience and pervasive nature of Italian verbal morphology.

The last dimension we want to consider about the way linguistic features could affect the development of a BV concerns the proximity/distance between SL and TL. The ESF project included closer and more distant SL-TL pairings in order to find communicative, language independent forces driving the acquisitional process. Across all language pairs, the BV was indeed a relevant stage in this reconstruction process; Spanish learners of French, for example, developed a BV as well as Arabic learners of French.

However, results coming from even more similar SL-TL pairs show that speakers can also resort to a different solution. The reconstruction of a new language can seem an ineffective strategy, when the SL offers a plausibly good-enough starting point for communicative as well as for acquisitional purposes. In this case, a \textit{restructuring} continuum could take place \citep{Corder1978}, as observed in other linguistic contact phenomena, such as the creole \textit{continua}, where the TL works as the roof variety. Studies on the acquisition of Italian by Spanish speakers, either in the context of a \textit{lingua franca} in Switzerland \citep{Schmid1994} or as the majority language in Italy \citep{Vietti2005}, demonstrate this possibility. In this case, SL and TL share a wide range of almost homophone lexical and grammatical morphemes, very similar or even identical grammatical categories and values in inflection classes and morphosyntactic rules. In a way, SL and TL can be understood as two varieties of the same parent language with a series of systematic phonological correspondences and occasionally differing lexical pairs. For instance, the same rule of desonorization\slash lenition of intervowel consonants can explain differences in lexical roots (SP \textit{cabo >} IT \textit{capo}; SP \textit{lado >} IT \textit{lato}…) and in functional morphemes (past participle SP -\textit{ado} > IT -\textit{ato}; imperfect SP \textit{{}-ab-} > IT \textit{{}-av{}-}); many monosyllabic morphemes only differ in the degree of opening anterior vowels (SP \textit{el, de, en, me, te, se…} > IT \textit{il, di, in, mi, ti, si…}). This proximity can orient learners toward the hypothesis that rules of convergence\slash correspondence with the SL could effectively lead to the TL. The following examples show the resulting productions, suggesting an ongoing restructuring process which can eventually stabilize in an ethnic variety of Italian \citep{Vietti2005}.\footnote{{Notation system in the examples: Italian;} {\textbf{Spanish}}{;} {\textit{both}}{; Italian with non-target phonology; Spanish with non-target phonology. According to \citet{Schmid1994}, similar results come from \citet{Mazzuri1990} on} {\textit{Portalienisch}}{, a variety of Italian spoken by Portuguese workers in Switzerland.}}\largerpage

\ea%10
    \citep{Schmid1994}\label{ex:benazzo:10}\\
    \gll            c’è        \textit{un}       \textbf{fall}{}-\textit{o}            \textit{no} perché   io \textit{quando} vad-\textit{o}     insieme \textit{con}   \textbf{los}        \textbf{español{}-}o-\textbf{s}  \textbf{y}     parl-\textit{o}         \textbf{español{}-∅}    sempre  dopo  vad-\textit{o}     \textit{a} parl-are      \textit{con}  \textit{un}        \textit{italian-o}      e     \textbf{ya} ++    \textit{le}   parl-\textit{o} +      più    \textbf{en} \textbf{español}{}-o     \textit{che}  \textbf{en}  \textit{italian}{}-\textit{o}      capiss{}-\textbf{es}\\
    there-is a.MS  mistake-MS no  because I   when     go-1SG together with the.MP Spanish-M-P   and speak-1SG Spanish-MS  always  after  go-1SG  to speak-INF with an.MS Italian-MS  and by-now {} her speak-1SG {} more in  Spanish-MS than in   Italian-MS  understand-2SG\\
\ex%11
    \citep[90]{Vietti2005}\label{ex:benazzo:11}\\
    \gll           \textit{la}         \textbf{situasion}{}-e       (in) mio \textbf{pais}      è  molto \textit{critic-a} + \textbf{no}  aßeß{}-\textit{a}                 molt-\textit{o}       sold-i          e     tutt{}-i         (\textbf{tiend}{}-\textit{a}) \textit{boutique}  \textbf{comersi{}-}\textit{o}         mercat{}-\textit{o}      tut{}-i                    \textit{un} po’ +++ \textbf{como} poso        dire        \textbf{de}  \textit{che} + facev-\textit{a}            \textbf{imßersion}{}-i\\
    the.FS  situation-MS     in  my   country is much  critical-FS {} not have.IMPF-3SG  much-MS money.MP and every-MP shop.FS  boutique  commerce-MS  market.MS   everything-MP  a bit {} how   can.1SG  say.INF  of  that  {} do.IMPF-3SG  investment-MP\\
\z

\section{Discussion and conclusions}\label{sec:benazzo:5}

In the previous sections we discussed the possible impact of some factors on the initial stages of L2 acquisition, in particular with respect to the emergence of a BV system. We are aware that the available studies are not sufficient to fully disentangle the three factors considered, since more than one factor is changed in many studies. Nevertheless, they suggest the following tentative generalizations which could be further explored in future research. 

\begin{description}
\item[Literacy:] even if this variable might affect some other aspects of SLA, it does not seem to play a major role for the initial stages of oral competence as defined in the ESF project. Some studies on highly educated learners (only cited in \sectref{sec:benazzo:4.1} for space reasons) roughly confirm the developmental sequence attested in the ESF project. More interestingly, the few studies controlling this variable (i.e. comparing learners with different degrees of literacy) in communicative production tasks show that there is no remarkable difference between literate\slash low-literate and non-literate learners, as they all seem to follow the same acquisitional path. In particular, development beyond the BV is possible independently of literacy. However, this variable affects the type of input that the learners have access to: it is plausible that a limited exposure to only aural input may foster a longer reliance either on a BV system or on lexical strategies when dealing with L2 morphosyntax.

\item[Instruction:] in spite of classroom input, BV-like systems appear when learners are faced with complex communicative tasks, especially when short periods of exposure and/or the teaching methods employed restricted their opportunities for acquiring or proceduralizing declarative knowledge. The same learners can, however, produce grammatical structures beyond the BV, for example in offline experimental tasks focusing on one particular grammatical property only. Their BV-like systems might therefore be described as a provisional solution rather than a ‘fully-fledged’ learner variety.

\item[Typological (morphosyntactic) features of the S/TL:] the impact of this factor has been discussed with respect to the L2 acquisition of Italian. The pervasive presence of salient and transparent morphology in this TL seems to motivate the appearance of some morphological oppositions together with the persistence of BV features at the syntactic level (copula and adverb positioning), and the fact that the development of verb finiteness does not cause a major reorganisation of the utterance structure. The observation of learners with typologically distant vs. close L1s also highlights the effect of the SL, which possibly leads to an acceleration or delay of specific subsystems (in case of Chinese: delay in the development of morphology) in the transition between stages. The case of Spanish learners of Italian suggests the possibility of skipping the BV stage in favour of an acquisitional path involving restructuring rather than reconstruction, when SL and TL are similar enough.
\end{description}

More generally, a closer look at L2 development in different combinations of S/TL raises some questions about both (a) the definition, and (b) the status of the BV. 

As for (a), the BV has been initially characterised both on the formal level (absence of inflectional morphology, be it nominal or verbal, and utterance organization based on semantic/pragmatic principles) and on the communicative level (expressive possibilities and limitations). However, we have seen that, according to the SL/TL combination (and\slash or learning setting), some functional albeit not necessarily target-like morphological markings (e.g. plural markings, or past participles), might appear quite early without altering the central architecture of the system and its communicative possibilities. In addition, its reorganisation with the emergence of a finite verb (postbasic Variety) is more manifest in some TLs than others. The absence of morphology in the BV and the crucial role of finiteness for further development could therefore be considered as language-specific manifestations rather than general properties of the relevant stages.

As for (b), the different circumstances under which the BV surfaces raise a question about its status: is it an acquisitional stage or rather a mode, i.e. a provisional solution to a communicative problem which can also become a fluent and stable variety?      

Other approaches would probably be necessary in order to answer these questions. It is remarkable in any case that speakers resort to the same resources and solutions in different contexts.\footnote{In addition to the L2 contexts analysed in this paper, there seems to be a convergence towards the same solutions also in other emerging communicative systems, e.g. the gestural systems of deaf speakers who are not exposed to sign languages (cf. \citealt{Benazzo2009} for a comparison between the development of Home signs and L2 varieties).} What orientates speakers in this direction is both the presence of a similar hierarchy of communicative needs and the experience of language practice leading towards similar formal solutions, which are economical and communicatively efficient.\footnote{On this point, Klein assumes that the principles at work at the BV “immediately reflect creative processes of the underlying human language faculty” (\citeyear[93]{Klein2001}); whereas \citet{Perdue2006} underlines the “neutral” linguistic status of the relevant principles, that are (to different degrees) also active in most fully fledged languages.}

Promising directions for further research would be to explore the features of a BV-like system in TLs with entirely different typological properties, for example with no morphology (Chinese), with different morphological (agglutinative, inflectional) or syntactical systems (SOV), or even in a different (signed, written) modality. The discussion of the BV thirty years after its first description is still exciting and continues to generate challenging questions.

\section*{Acknowledgements} 

Our contribution is a tribute to Daniel Véronique’s work on second language acquisition, in particular on the development of L2 grammar (cf. for ex. \citealt{Véronique1983,Véronique2000,Véronique2013dislocation,Véronique2021Acquisition, NoyauEtAl1995, VéroniqueEtAl2009}). We got to know Daniel and his work during our PhD studies, when researchers from the ESF project regularly met at the MPI in Nijmegen. As the head of the French ESF team at Université de Provence, where the data of the Moroccan learners of French were collected, Daniel Véronique substantially contributed to the success of the project.

\printbibliography[heading=subbibliography,notkeyword=this]
\end{document} 
