\author{Henrik Bergqvist\lastand Seppo Kittilä}
\title{Evidentiality, egophoricity, and engagement}

\renewcommand{\lsSeries}{sidl}
\renewcommand{\lsSeriesNumber}{30}

\BackBody{The expression of knowledge in language (i.e. epistemicity) consists of a number of distinct notions and proposed categories that are only partly related to a well explored forms like epistemic modals. The aim of the volume is therefore to contribute to the ongoing exploration of epistemic marking systems in lesser-documented languages from the Americas, Papua New Guinea, and Central Asia from the perspective of language description and cross-linguistic comparison. As the title of the volume suggests, part of this exploration consists of situating already established notions (such as evidentiality) with the diversity of systems found in individual languages. Epistemic forms that feature in the present volume include ones that singal how speakers claim knowledge based on perceptual-cognitive access (evidentials); the speaker’s involvement as a basis for claiming epistemic authority (egophorics); the distribution of knowledge between the speech-participants where the speaker signals assumptions about the addressee’s knowledge of an event as either shared, or non-shared with the speaker (engagement marking). }

%\dedication{Change dedication in localmetadata.tex}
\typesetter{Carla Bombi, Elena Moser, Sebastian Nordhoff}
\proofreader{Anca Gâţă,
Andreas Hölzl,
Annie Zaenen,
Benjamin Brosig,
Eva Schultze-Berndt,
Ezekiel Bolaji,
Gerald Delahunty,
Jeroen van de Weijer,
Lachlan Mackenzie,
Sean Stalley,
Tom Bossuyt,
Yvonne Treis
}

\renewcommand{\lsID}{261}
\BookDOI{10.5281/zenodo.3968344}
\renewcommand{\lsISBNdigital}{978-3-96110-269-3}
\renewcommand{\lsISBNhardcover}{978-3-96110-270-9}
