\documentclass[output=paper]{langsci/langscibook} 
\ChapterDOI{10.5281/zenodo.3975805}
\title{On the existence of egophoricity across clause types in Totoró Namtrik}
%\shorttitlerunninghead{}
% \renewcommand{\lsCollectionPaperFooterTitle}{On the existence of egophoricity across clause types in Totoró Namtrik}
\author{Geny Gonzales Castaño \affiliation{Laboratoire Dynamique du Langage, Université Lumière Lyon}} 

\abstract{The Barbacoan languages are known for having egophoricity systems (\citealt{Dickinson2000}, \citealt{Curnow2002a}, \citealt{CurnowLiddicoat1998}), which exhibit a verbal marking pattern in which “speaker subjects in statements are marked the same way as addressee subjects in questions” (\citealt[614]{Curnow2002a}). Nevertheless, the existence of such a pattern in Namtrik had not been recognized. A recent paper by \citet{Norcliffe2018} claims that in the Guambianovariety of Namtrik “the verb marking diverges from what might be considered canonical egophoricity marking, since it does not occur when the subject is second person in questions” (\citealt{Norcliffe2018}). The current paper presents novel data from the highly endangered variety of Totoró Namtrik and argues that this variety possesses a set of verbal suffixes exhibiting the cross-linguistically recurrent pattern of an egophoricity distribution. The goal of this paper is to show that although Namtrik’s egophoricity system is similar to the systems in other languages, it was not analyzed as a fully fleged egophoricity system in the past because the egophoricity suffixes are not always easily recognizable in interrogative clauses due to morphosyntactic and phonological processes. Additionally, this paper shows that Namtrik has an “undergoer” egophoric marker which exhibits a pattern of egophoricity distribution, shifting from speaker to addressee perspective in interrogatives.  
}

\begin{document}
\maketitle

\section{Introduction} 

Namtrik, also known as Guambiano, is a Barbacoan language (\citealt{CurnowLiddicoat1998}) spoken in the Colombian Andes. Traditionally, Namtrik speakers live in four so-called \emph{resguardos} (settlements recognized by the Colombian State, with territorial autonomy and ruled by traditional authorities named \emph{Cabildo}): Guambia, Ambaló, Quizgó and Totoró. These settlements are situated in two towns, Silvia and Totoró, located in the department of Cauca, in the Southwest of the country. Currently most researchers consider Totoró and Guambiano to be two dialects of the same language (\citealt{CurnowLiddicoat1998}), which the speakers refer to as \textit{Namtrik}, or \textit{Namuy Wam} ‘our sound’ (\citealt{GonzalesCastano2013}).


This paper will focus on a highly endangered variety of Namtrik, spoken in the community of Totoró. According to a census conducted by the authorities of the community in 2013, there are 76 native speakers (approximately 1\% of a total population of 7023 people), who are all over 50 years in age (\citealt[11]{GonzalesCastano2013}). In Totoró the use of Spanish has displaced Namtrik in all daily interactions and there are now two generations of monolingual Spanish speakers. There is no intergenerational transmission of the language. Namtrik in Totoró is clearly more endangered than the Guambiano dialect in Guambia, which numbers about 8000–9000 speakers (\citealt[66]{Adelaar1991}). 

The data presented in this paper, beside a few examples constructed in elicitation, consists mostly of naturally-occurring examples. This data comes from two sources. The first is a lexicon collected between 2006 and 2008 within the frame of workshops on morphosyntax of Namtrik for community teachers and Namtrik speakers, which were directed by Tulio Rojas Curieux and Beatriz Vasquez. One of the results of this work was a lexicon (382 words and 448 sentences) edited and published in 2009 as \emph{Léxico de la lengua Namtrik de Totoró} (\citealt{RojasCurieuxetal2009}). The other source of this data is a video and audio corpus, consisting of ten hours of natural speech data and additional elicitation data, collected between 2015 and 2017, in the framework of the project \emph{Documentation and Description of Namtrik, an endangered language of the Colombian Andes} founded by HRELP-SOAS.\footnote{The data presented in this paper is available at the Endangered Languages Archive at SOAS University of London: \url{https://elar.soas.ac.uk/Collection/MPI1012401}.} 

The Barbacoan languages are known for their egophoricity systems (\citealt{Dickinson2000}; \citealt{Curnow2002a}; \citealt{CurnowLiddicoat1998}; \citealt{Floyd2018}), which “employ special verb forms […] where the subject (and/or another privileged argument role) is first person in declaratives or second person in interrogatives” (\citealt{SanRoque2018}). Available grammatical sketches based on the Namtrik varieties of Guambiano and Totoró postulate the existence of a subject agreement marking system, which distinguishes between first person ‘locutor’ singular and plural and second and third person ‘non locutor’ (\citealt{Pabon1989}; \citealt{TrivinoGarzon1989}; \citealt{Vasquez1987}; \citealt{Vasquez1988}). 


According to \citeauthor{Vasquez1987} (\citealt{Vasquez1987}; \citealt{Vasquez1988}) and \citet{TrivinoGarzon1989} the Guambiano variety of Namtrik presents an agreement verbal marking system distinguishing between a speaker singular subject marker -\textit{(a)r}, a speaker plural subject marker -\textit{er} and a non-speaker marker, singular and plural -\textit{(a)n}. In the case of the Totoró variety, besides a different speaker singular subject marker -\textit{(o)r}, \citet{Pabon1989} proposes the same forms for the other egophoricity markers. In these previous descriptions of Namtrik (\citealt{Pabon1989}; \citealt{TrivinoGarzon1989}; \citealt{Vasquez1987}; \citealt{Vasquez1988}), there is no information available about the behavior of egophoricity markers in questions; nevertheless the data presented show clearly that declaratives in Namtrik follow the expected pattern of an egophoricity system.

The existence of an egophoricity system in Namtrik was not argued for until a recent paper by \citet{Norcliffe2018}, which is based on the Namtrik variety from Guambia. \citet{Norcliffe2018} notes, however, that while declaratives in Guambiano present the expected egophoricity pattern, the egophoricity markers do not appear in interrogatives. Nevertheless “layers of egophoric or egophoric-like marking are visible in Guambiano’s grammar” (\citealt{Norcliffe2018}), in Guambiano the verb marking pattern “does not exhibit the canonic egophoric pattern, since the egophoric markers do not appear in questions” (\citealt{Norcliffe2018}).

The first goal of this paper is to show that the Totoró variety of Namtrik (TTK) possesses an egophoricity verbal marking pattern across clause types which “distinguishes speakers from non-speakers in declaratives, and addressees from non-addressees in interrogatives" (\citealt[1]{Creissels2008}); however, some morphophonological processes make it occasionally difficult to recognize this pattern in interrogative clauses. A second goal is to demonstrate that Namtrik in Totoró has an “experiencer” marker, which exhibits an egophoricity distribution, shifting from speaker to addressee perspective in interrogatives. Although I will show some usage patterns of the egophoricity markers, the semantic analysis falls outside the scope of this paper.

This paper gives a first account of the egophoricity system in TTK, beginning with a discussion on the morphophonological processes which make it difficult to recognize egophoricity markers in interrogatives. It then focuses on the description of the general usage pattern of egophoricity markers across clause types. And it concludes with the description of the use of the experiencer egophoric marker \textit{-t} and the distribution of this marker in question predicate constructions.
  

\section{Morphophonological processes affecting egophoricity markers in Totoró Namtrik}\label{s:gg1}

Namtrik has a variety of morphophonological processes that occur at morpheme boundaries, which have not been fully argued for in previous descriptions. Understanding these processes is crucial for not only the description of the egophoricity system but also the description of Namtrik’s grammar in general. 

In TTK, there are two different morphophonological processes which make the presence of egophoricity markers in interrogative clauses unrecognizable: vowel deletion and consonant deletion.

Writing about the egophoricity system in Guambiano, \citet{Norcliffe2018} proposes the existence of a set of verbs that present an alternation correlated with the existence of a vestigial egophoricity contrast in Guambiano, which show palatalized consonant forms in ego environments. In this section it will be argued that this alternation is not directly correlated with Namtrik’s egophoricity system but with morphophonological processes. 


\subsection{Vowel deletion}\label{s:gg1-1}

The Totoró variety of Namtrik (TTK) has two egophoric suffixes: singular -\textit{or}\footnote{
	The egophoric marker -\textit{ar} which has been reported in Guambiano Namtrik is also attested in the Totoró variety in very few examples in the data. This egophoric marker has very restricted uses, occurring only in constructions with auxiliary verbs in declarative clauses, as is shown in example (\ref{ex:gg-FN}). The egophoric marker -\textit{ar} is not attested in interrogative clauses. For these reasons it falls outside the scope of this paper.
	
	\ea\label{ex:gg-FN}
    \langinfo{Totoró Namtrik}{}{LEX1/118}\\
	\gll  ye	 ma-ap wa-ar \\
      potato eat-\textsc{dur} sit.\textsc{sg-ego.sg}\\
	\glt ‘I’m eating potatoes.’
	\z} and plural -\textit{er}, and one non-egophoric suffix: -\textit{an}, as is shown in \tabref{tab:gg1}.
	
% table 1
\begin{table}
\begin{tabularx}{.5\textwidth}{Xlcr}
\lsptoprule
  & \textsc{sg}  & & \textsc{pl} \\
\midrule
\textsc{ego} & -\textit{or}/-\textit{ar} & & -\textit{er}\\
\textsc{non ego} & & -\textit{an} & \\
\lspbottomrule
\end{tabularx}
\caption{Egophoricity markers in Totoró Namtrik}
\label{tab:gg1}
\end{table}

The egophoricity markers are fully visible when they follow a verb stem ending in a consonant, as shown in the \tabref{tab:gg2}, which presents verb stems ending in a consonant, hosting the egophoric markers singular -\textit{or} and plural -\textit{er} and the non-egophoric marker -\textit{an}.

% table 2
\begin{table}[t]
\begin{tabularx}{\textwidth}{XXXl}
\lsptoprule
\textsc{verb} & \textsc{v-ego.sg} & \textsc{v-ego.pl} & \textsc{v-non.ego}\\
\midrule
\textit{par}- ‘cut’ & \textit{par-or} & \textit{par-er} & \textit{par-an}\\
\textit{trup}- ‘lose’ & \textit{trup-or} & \textit{trup-er} & \textit{trup-an}\\
\textit{nen}- ‘cook’ & \textit{nen-or} & \textit{nen-er} & \textit{nen-an}\\
\textit{kuakl}- ‘boil’ & \textit{kuakl-or} & \textit{kuakl-er} & \textit{kuakl-an}\\
\textit{kutr}- ‘wake up’ & \textit{kutr-or} & \textit{kutr-er} & \textit{kutr-an}\\
\lspbottomrule
\end{tabularx}
\caption{Egophoricity markers on verb stems ending in consonants}
\label{tab:gg2}
\end{table}
\begin{table}[b]
\begin{tabularx}{\textwidth}{XXXl}
\lsptoprule
\textsc{verb} & \textsc{v-ego.sg} & \textsc{v-ego.pl} & \textsc{v-non.ego}\\
\midrule
\textit{ña}- ‘spin’ & \textit{ñar} & \textit{ñer} & \textit{ñan}\\
\tablevspace
& \textit{ña-or} & \textit{ña-er} & \textit{ña-an}\\
& spin-\textsc{ego.sg} & spin-\textsc{ego.pl} & spin-\textsc{non.ego}\\
\tablevspace
\textit{kɨ}- ‘be’ & \textit{kor} & \textit{ker} & \textit{kɨn}\\
& \textit{kɨ-or} & \textit{kɨ-er} & \textit{kɨ-an}\\
& be-\textsc{ego.sg} & be-\textsc{ego.pl} & be-\textsc{ego.pl}\\
\tablevspace
\textit{tso}- ‘lie.\textsc{sg}’ &  \textit{tsor} & – & \textit{tson}\\
& \textit{tso-or} & & \textit{tso-an}\\
& lay-\textsc{ego.sg} & & lay-\textsc{non.ego}\\
\lspbottomrule
\end{tabularx}
\caption{Egophoricity markers on verb stems ending in vowels}
\label{tab:gg3}
\end{table}

In previous descriptions of Namtrik’s verbal morphology (\citealt{Pabon1989}; \citealt{TrivinoGarzon1989}; \citealt{Vasquez1987}, \citeyear{Vasquez1988}; \citealt{Norcliffe2018}), the existence of allomorphs of the egophoricity markers has been postulated. The egophoric singular marker has two realizations -\textit{ar} and -\textit{r} (-\textit{or} and -\textit{r} in the case of Totoró) and the non-egophoric marker has the realizations -\textit{an} and -\textit{n}. However, the phonological context determining these realizations has not been clearly identified, with the vowel of the suffix described as sometimes being absent “depending on the value of the final vowel of the verb stem” (\citealt{Norcliffe2018}).

\tabref{tab:gg3} shows examples of the TTK egophoric singular marker -\textit{or}, the egophoric plural marker -\textit{er} and the non-egophoric marker -\textit{an}, following verb stems ending in vowels. Examples in \tabref{tab:gg3} show that in all the cases either the final vowel of the verb stems or the initial vowel of the egophoricity markers is absent. The purpose of this section is to clarify the phonological conditioning of this process of deletion.

% table 3



Although in TTK it is possible to find sequences of two vowels within lexical stems or suffixes, the language puts constraints on vowel sequences at morphological boundaries. When a stem or a suffix ends in a vowel and it is followed by a suffix beginning in a vowel, one of the two vowels in this sequence is elided. The identity of the elided vowel does not depend upon the position of the vowels in the sequence, but on the particular combination of vowels. \tabref{tab:gg4} shows the combinations attested in the corpus and the results of vowel elision, as well as an example for each pattern.

% table 4
\begin{table}
\begin{tabularx}{\textwidth}{XXX}
\lsptoprule
attested combinations & outcome & example\\
\midrule
\textit{ɨo} & \textit{o} & \textit{kor}\\
& & \textit{kɨ-or}\\
& & \textsc{cop-ego.sg}\\
\hline
\textit{oɨ} & \textit{o} & \textit{tson}\\
& & \textit{tso-an}\\
& & lie.\textsc{sg}-\textsc{non.ego}\\
\hline
\textit{ao} & \textit{o} & \textit{ñor}\\
& & \textit{ña-or}\\
& & spin-\textsc{ego.sg}\\
\hline
\textit{oa} & \textit{o} & \textit{pon}\\
& & \textit{po-an}\\
& & arrive-\textsc{non.ego}\\
\hline
\textit{ua} & \textit{u} & \textit{atrun}\\
& & \textit{atru-an}\\
& & come.\textsc{sg}-\textsc{non.ego}\\
\hline
\textit{ui} & \textit{u} & \textit{llusruk}\\
& & \textit{llusru-ik}\\
& & spill-\textsc{nmz.sg}\\
\hline
\textit{ɨa} & \textit{ɨ} & \textit{kɨmɨn}\\
& & \textit{kɨ-mɨ-an}\\
& & \textsc{cop-neg-non.ego}\\
\hline
\textit{ia} & \textit{i} & \textit{kip}\\
& & \textit{ki-ap}\\
& & sleep-\textsc{dur}\\
\hline
\textit{ea} & \textit{e} & \textit{kosrep}\\
& & \textit{kosre-ap}\\
& & teach-\textsc{dur}\\
\hline
\textit{ɨi} & \textit{i} & \textit{yalik}\\
& & \textit{jalɨ-ik}\\
& & black-\textsc{nmz.sg}\\
\hline
\textit{ɨe} & \textit{e} & \textit{ker}\\
& & \textit{kɨ-er}\\
& & \textsc{cop-ego.pl}\\
\hline
\textit{ae} & \textit{a} & \textit{nosrkalɨ}\\
& & \textit{nosrka-elɨ}\\
& & brother-\textsc{nmz.pl}\\
\lspbottomrule
\end{tabularx}
\caption{Possible combinations of phonemic vowels}
\label{tab:gg4}
\end{table}

These patterns of vowel elision may be summarized in the rule illustrated in \figref{fig:gg1}, which also shows that there exists a vowel hierarchy in TTK with respect to this morphophonological process. In this hierarchy the vowels /i/, /e/, /o/ and/u/ are stronger than the high central vowel, which is stronger than the vowel /a/.



%TODO Figure 1
\begin{enumerate}
    \item Hierarchy of elision in vowels coalescence \label{fig:gg1}\\
    i \\
    e \\
    o  > ɨ  > a\\
    u
\end{enumerate}{}

% \begin{table}
% \begin{tabularx}{\textwidth}{XXX}
% \lsptoprule
% i & & \\
% e & & \\
% o & > ɨ & > a\\
% u & & \\
% \lspbottomrule
% \end{tabularx}
% \caption{Hierarchy of elision in vowel coalescence}
% \label{tab:ggfig1}
% \end{table}


The process of vowel elision was identified in TTK by \citet{Pabon1989}. \citeauthor{Pabon1989} proposes that TTK puts constraints on the combination of high vowels and high or middle high vowels (\citeyear[16]{Pabon1989}). Nevertheless, the constraints on the vowel sequences at morphological boundaries do not concern only these combinations of vowels, as shown in the \tabref{tab:gg4}. This is a regular and widespread phonological process in the language and does not solely affect the egophoricity markers.

As noted above, the process of vowel elision is found with other suffixes of the form -V(C) or -C(V) as well as the egophoric and non-egophoric markers. Every morpheme ending or beginning with a vowel can lose its last phoneme, as is the case of the vowel /ɨ/ in the morpheme -\textit{mɨ} ‘\textsc{neg}’ in coalition with the morpheme -\textit{elɨ} ‘\textsc{nmz.pl}’ in example (\ref{ex:gg1}), or its first phoneme, as example (\ref{ex:gg2}) shows in the case of the vowel /e/ in the morpheme -\textit{elɨ} ‘\textsc{nmz.pl}’ affixed to the noun \textit{nosrka}- ‘brother’.
 \newpage

\ea Totoró Namtrik (namtrik\_042/227)\label{ex:gg1}\\
    %\langinfo{Totoró}{}{}\\
	\glll mamelɨ\\
	 ma-mɨ-elɨ \\
     eat-\textsc{neg-nmz.pl}\\
	\glt ‘those that have not been eaten’ %TODO translation missing
	\z

\ea Totoró Namtrik (namtrik\_056/58)\label{ex:gg2}\\
    %\langinfo{Totoró}{}{}\\
	\glll nosrkalɨ\\
    nosrka-elɨ \\
    brother-\textsc{nmz.pl}\\
	\glt ‘brothers’ %TODO translation missing
	\z
	
This process makes some morphemes unrecognizable in certain phonological contexts. This is the case of the imperative second person singular marker -\textit{ɨ} which is visible when attached to a verb stem ending in consonant, as shown in examples (\ref{ex:gg3}) and (\ref{ex:gg4}), but fully deleted following a vowel ending verb stem as shown in (\ref{ex:gg5}).

\ea \label{ex:gg3}
    \langinfo{Totoró Namtrik}{}{namtrik\_122/13, elic}\\
	\gll ñi	kilka-wan isɨk-ɨ \\
      2	notebook-\textsc{dat}  store-\textsc{imp.sg}\\
	\glt ‘Store your notebook!’
	\z
	
\ea \label{ex:gg4}
    \langinfo{Totoró Namtrik}{}{namtrik\_036/9}\\
	\glll yusrɨ	amɨ\\
	yu-srɨ	amp-ɨ \\
      \textsc{loc-dir} go.\textsc{pl-imp.sg}\\
	\glt ‘You come here!’
	\z
	
\ea \label{ex:gg5}
    \langinfo{Totoró Namtrik}{}{elic}\\
	\glll kosro\\
	kosro-ɨ\\
      stand.up-\textsc{imp.sg}\\
	\glt ‘Stand up!’
	\z
	
The only sequence which has the vowel /a/ as outcome of the phonological process of vowel deletion is the sequence /ae/. The vowel /a/ is the ``weakest'' one in the hierarchy which determines this process in TTK. This is important to keep in mind for the description of egophoricity system in TTK; since the non-egophoric suffix -\textit{an} begins with /a/ and is often affected by the process of vowel elision, which makes it often difficult to recognize, especially in interrogative clauses. As we will see later, in Namtrik, the egophoricity markers lose their final consonant in interrogative clauses, so the non-egophoric suffix -\textit{an} suffers a total deletion in questions following a vowel-final verb stem. This is similar to the deletion of the imperative second person singular marker -\textit{ɨ} in this phonological context, as shown in example (\ref{ex:gg5}).



\subsection{Consonant deletion of the egophoricity markers in interrogative clauses}\label{s:gg1-2}

Questions in TTK are marked by a rising intonation, a high pitch on the last syllable of the clause and the deletion of the final consonant of the egophoric and non-egophoric suffixes or, less frequently, the replacement of the final consonant by a glottal stop /ʔ/. The following set of examples shows two sentences using -\textit{an} — one declarative and one interrogative – in similar kinds of predicate structure and with the same verb stem. 

In the second example of each set, it can be seen that the final consonant in the non-egophoric marker -\textit{an}, following verb stems ending in consonant — \textit{pen}- ‘fall’ in (\ref{ex:gg7}) and \textit{pasr}- ‘stand.\textsc{sg}’ in (\ref{ex:gg9}) — is retained in the surface form of the interrogative clauses.

\ea
    \ea \label{ex:gg6}
    \langinfo{Totoró Namtrik}{}{namtrik\_052/325}\\
	    \glll /unɨʧik	penan/\\
	    unɨ-chik pen-an\\
        child-\textsc{dim} fall-\textsc{non.ego}\\
	    \glt ‘a boy fell’
\ex \label{ex:gg7}
    \langinfo{Totoró Namtrik}{}{namtrik\_483/057}\\
	\glll chu	pen-a\\
	chu pen-an\\
    where fall-\textsc{non.ego}\\
	\glt ‘Where did it fall?’
	\z
\z

\ea
    \ea \label{ex:gg8}
    \langinfo{Totoró Namtrik}{}{elic}\\
	    \glll /ju	paʂan/\\
    	yu	pasr-an\\
        \textsc{loc} stand.\textsc{sg-non.ego}\\
	    \glt ‘It is here.’
\ex \label{ex:gg9}
    \langinfo{Totoró Namtrik}{}{namtrik\_058/112}\\
	\glll chine	marɨp	pasr-a\\
	chi-ne	ma-rɨp pasr-an\\
    \textsc{q}-3 do-\textsc{dur} stand.\textsc{sg-non.ego}\\
	\glt ‘What is she doing?’
	\z
\z

Although it is possible to recognize the non-egophoric marker -\textit{an} in interrogative clauses when it follows a verb ending in a consonant, when it follows a vowel-final verb it is completely omitted, as is illustrated in the following set of examples. The examples (\ref{ex:gg10}) and (\ref{ex:gg11}) (a declarative and an interrogative clause), show the non-egophoric marker -\textit{an} hosted by the copula \textit{kɨ}-. In example (\ref{ex:gg11}) the surface form of the verb seems to have a bare stem form, without any verbal marker. However, the absence of the non-egophoric marker is explained by the phonological processes of deletion of vowels in coalescence and the deletion of the final consonant of the egophoricity markers in interrogative clauses. 

\ea \label{ex:gg10}
    \langinfo{Totoró Namtrik}{}{namtrik\_034/64}\\
	\glll ɨnɨ	tɨwei	kɨn\\
	ɨ-nɨ tɨwei	kɨ-an\\
      \textsc{dist}-3 just \textsc{cop-non.ego}\\
	\glt ‘It is just that.’
	\z
	
	
\ea \label{ex:gg11}
    \langinfo{Totoró Namtrik}{}{namtrik\_058/124}\\
	\glll ɨnɨ	tɨwei	kɨ\\
	ɨ-nɨ tɨwei	kɨ-an\\
      \textsc{dist}-3 only \textsc{cop-non.ego}\\
	\glt ‘Is it only that?’
	\z


\subsection{Deletion of the non-egophoricity marker in interrogative clauses}\label{s:gg1-3}

As is claimed by \citet{SanRoque2018}, writing about interrogativity in evidential and egophoricity markers, a formal asymmetry with distinct interrogative morphemes or constructions is attested in several unrelated languages. Similarly in TTK there are two different question constructions, one for interrogatives involving third person subjects and another for interrogatives involving first and second person subjects.

Aside from the deletion of the final consonant of the non-egophoric marker and the rising intonation, the predicate question construction involving third person subjects has the same morphological and syntactic structure as the declarative predicate construction, as is shown in the following set of examples. Examples (\ref{ex:gg12}) and (\ref{ex:gg13}) show an interrogative and a declarative clause with similar kinds of predicate structure and the same verb stem.\footnote{In the examples below, square brackets denote a complex predicate construction.} 

\ea \label{ex:gg12} \ea
    \langinfo{Totoró Namtrik}{}{namtrik\_059/179}\\
    \glll kuaklap	pasra\\
    kuakl-ap pasr-an\\
         boil-\textsc{dur} stand.\textsc{sg}-\textsc{non.ego}\\
    \glt ‘Is it boiling?’
  \ex
  \langinfo{Totoró Namtrik}{}{namtrik\_083/91, elic}\\
    \gll pi=pe mani [kuakl-ap pasr-an]\\
         water=\textsc{pd}	now	boil-\textsc{dur} stand.\textsc{sg-non.ego}\\
    \glt ‘The water is boiling.’
\z \z

\ea \label{ex:gg13} \ea
    \langinfo{Totoró Namtrik}{}{namtrik\_059/43}\\
    \gll karopik	kucha	atru\\
         governor	too	come.\textsc{sg}-\textsc{non.ego}\\
    \glt ‘Is also the governor coming?’
  \ex
  \langinfo{Totoró Namtrik}{}{LEX1/104}\\
    \glll wasri	atrun\\
    wasri	atru-an\\
    sparrowhawk come.\textsc{sg-non.ego}\\
    \glt ‘The sparrowhawk comes.’
\z \z

The question predicate construction involving first and second person subjects is a complex predicate construction consisting of a main verb stem and an obligatory copula auxiliary \textit{kɨ}-, as is shown in example (\ref{ex:gg14})--(\ref{ex:gg16}) (see also \ref{ex:gg45} and \ref{ex:gg52}). First and second person questions are marked additionally by the deletion of the final consonant of the non-egophoric marker and rising intonation (cf. example \ref{ex:gg16}).

\ea \label{ex:gg14}
    \langinfo{Totoró Namtrik}{}{namtrik\_031/94}\\
	\glll wasrchikyupe chinɨ wai ko\\
	wasr-chik-yu=pe	chi-nɨ	wai kɨ-or\\
      bag-\textsc{dim-loc=pd}	\textsc{q}-3 put \textsc{cop-ego.sg}\\
	\glt ‘What did you drop in the bag?’
	\z
	
	
\ea \label{ex:gg15}
    \langinfo{Totoró Namtrik}{}{namtrik\_031/110}\\
	\glll chinetɨ	sro	amtru ke\\ 
	chi-ne-tɨ	sro	amtru kɨ-er\\
      \textsc{q}-3-\textsc{rest} carry come.\textsc{pl} \textsc{cop-ego.pl}\\
	\glt ‘What did they bring?’
	\z
	
	
\ea \label{ex:gg16}
    \langinfo{Totoró Namtrik}{}{namtrik\_045/140}\\
	\glll chu	pasramkɨ chi?\\
	chu	pasr-am=kɨ-an chi\\
      where	stand.\textsc{sg-irr=cop-non.ego}	\textsc{q}\\
	\glt ‘Where shall I put this?’
	\z

In questions involving second person subjects and first person plural subjects, the egophoric suffixes -\textit{or} and -\textit{er} are still recognizable in the surface form of questions, since the vowels /o/ and /e/ are not affected by the process of vowel deletion as illustrated in (\ref{ex:gg14}) and (\ref{ex:gg15}). However the non-egophoric marker -\textit{an} is not visible in the surface form of the interrogative clauses involving a first person singular subject. Nevertheless, as we already mentioned, the omission of the non-egophoric marker in interrogatives is not related directly to the grammar of the egophoricity system but to the morphophonological processes of vowel deletion and the deletion of the final consonant of the egophoricity markers in interrogative clauses.


\section{Verb alternation in verb stems ending in palatal, alveolar and dental consonants}\label{s:gg2}

Before describing the usage pattern of egophoricity markers across clause types in TTK, there is a final point to explore regarding morphophonological processes and their relationship with the egophoricity system. In the description of the morphophonology of the Guambiano variety of Namtrik, \citet[58]{Vasquez1987} claims the existence of a set of verbs which alternate between two verb stem forms, one ending in a palatal consonant and the other in a dental-alveolar consonant. Recently \citet{Norcliffe2018} proposed that this alternation is correlated with the existence of a vestigial egophoricity contrast in Guambiano. This set of verbs would show palatalized consonant forms in ego environments. 

\tabref{tab:gg5} shows examples of verb stems ending in palato-alveolar consonants hosting suffixes beginning in a vowel: the egophoric markers singular -\textit{or} and plural -\textit{er}, the non-egophoric marker -\textit{an}, the durative morpheme (-\textit{ip}/-\textit{ɨp}) and the verb nominalizer -\textit{ap}.\footnote{When the verb stem ends in the consonants, \textit{l}, \textit{ll}, \textit{n}, \textit{ñ}, \textit{ch}, and \textit{r} , the difference between the durative (-\textit{ip}/-\textit{ɨp}) and the verbal nominalizer -\textit{ap}  is neutralized, and the morpheme -\textit{ɨp} fulfills both functions.}

% table 5
\begin{table}
\begin{tabularx}{\textwidth}{lQlllll}
\lsptoprule
\textsc{verb} & & \textsc{v-ego.sg} & \textsc{v-ego.pl} & \textsc{v-non.ego} & \textsc{v-dur} & \textsc{v-nmz}\\
\midrule
\textit{pen}- & ‘fall’ & \textit{peñ-or} & \textit{peñ-er} & \textit{pen-an} & & \textit{peñ-ɨp}\\
\textit{kish}- & ‘cry’ & \textit{kish-or} & \textit{kish-er} & \textit{kis-an} & & \textit{kish-ɨp}\\
\textit{un}- & `walk.\textsc{sg}' & \textit{uñ-or} & — & \textit{un-an} & & \textit{uñ-ɨp}\\
\textit{amɨn}- & `walk.\textsc{pl}' & – & \textit{amɨn-er} & \textit{amɨn-an} & & \textit{amɨñ-ɨp-}\\
\textit{t}- & ‘say’ & \textit{ch-or} & \textit{ch-er} & \textit{t-an} & & \textit{ch-ip}\\
\textit{muts}- & ‘drink, suck, kiss’ & \textit{muts-or} & \textit{muts-er} & \textit{muts-an} & \textit{muts-ɨp} & \textit{much-ap}\\
\textit{kits}- & ‘grind’ & \textit{kits-or} & \textit{kits-an} & \textit{kits-an} & \textit{kits-ɨp} & \textit{kich-ap}\\
\textit{lall}- & ‘reap’ & \textit{lall-or} & \textit{lall-an} & \textit{lall-er} & & \textit{lall-ɨp}\\
\textit{lusr}- & ‘build’ & \textit{lusr-or} & \textit{lus-an} & \textit{lus-er} & & \textit{lusr-ɨp}\\
\lspbottomrule
\end{tabularx}
\caption{Verb alternant in verb stems ending in palatal and alveolar consonants}
\label{tab:gg5}
\end{table}



As is shown in the examples in \tabref{tab:gg5}, when the verbs end in /t/, /ɳ/ and /ʃ/, the final consonant of the verb stem is realized more alveolar: V (C palato-alveolar or alveolar\#) \_ a/ +alveolar. If the verb ends in a palato-alveolar affricate /ʧ/ or an alveolar affricate /ʦ/ the verb alternation is realized only when it is hosting the nominalizer -\textit{ap}.

Furthermore, when the verbs are followed by a suffix beginning with a consonant, the alternation is no longer observed, even in ego environments, since it is caused by the first vowel of the suffix following the verb stem, as is shown in the following sets of examples. 
Examples (\ref{ex:gg17})-(\ref{ex:gg19}) and (\ref{ex:gg20})-(\ref{ex:gg22}) show the verbs \textit{pen}- ‘fall’ and \textit{kish}- ‘cry’ followed by the negation marker -\textit{mɨ} and the egophoricity markers. The above analysis suggests that this alternation is not correlated with the egophoricity system but with a morphophonological process. 

\ea \textit{pen}- `fall' 
	\ea \label{ex:gg17}
    \glll penmor\\ 
    pen-mɨ-or\\
         fall-\textsc{neg-ego.sg}\\
    \glt ‘I don't fall.’
  \ex \label{ex:gg18}
    \glll penmer\\
    pen-mɨ-er\\
    fall-\textsc{neg-ego.pl}\\
    \glt ‘We don't fall.’
   \ex \label{ex:gg19}
   \glll penmɨn\\
   pen-mɨ-an\\
   fall-\textsc{neg-non.ego.pl}\\
   \glt 'S/he doesn't fall.'
\z \z


\ea \textit{kish}- `cry' 
	\ea \label{ex:gg20}
    \glll kishmor\\
    kish-mɨ-or\\
         cry-\textsc{neg-ego.sg}\\
    \glt ‘I don't cry.’
  \ex \label{ex:gg21}
    \glll kishmer\\
    kish-mɨ-er\\
         cry-\textsc{neg-ego.pl}\\
    \glt ‘We don't cry.’
   \ex \label{ex:gg22}
   \glll kishmɨn\\
   kish-mɨ-an\\
   cry-\textsc{neg-non.ego.pl}\\
   \glt 'S/he doesn't cry.'
\z \z



\section{The general usage pattern of egophoricity markers in Totoró Namtrik}\label{s:gg3}
% table 6
\begin{table}[b]
\begin{tabularx}{.8\textwidth}{XXl}
\lsptoprule
\textbf{Person} & \textbf{Declarative form} & \textbf{Interrogative form}\\
\midrule
1 & -\textit{s} & -\textit{y}\\
2 & -\textit{y} & -\textit{s}\\
3 & -\textit{y} & -\textit{y}\\
\lspbottomrule
\end{tabularx}
\caption{The general usage pattern of egophoricity markers in Awa Pit (\citealt{Curnow2002a})}
\label{tab:gg6}
\end{table}


Up to this point I have discussed the morphosyntactic and phonological processes which often make the presence of egophoricity markers in interrogative clauses unrecognizable. These processes are mainly of two types: elision of vowels through vowel coalescence and consonant deletion in egophoricity suffixes in interrogative clauses. This section is dedicated to the discussion of the usage of egophoricity systems across clause types in TTK. 

As noted above, egophoricity is canonically defined as a morphological category which marks speaker subjects in statements and addressee subjects in questions in the same way. Awa Pit, another Barbacoan language, shows a prototypical egophoricity system (\citealt[613]{Curnow2002a}), as is shown in \tabref{tab:gg6} and in the following sets of examples of declarative clauses (\ref{ex:gg23})-(\ref{ex:gg25}) and interrogative clauses (\ref{ex:gg26})-(\ref{ex:gg28}).%\footnote{}
Comparing the two sets of examples, it is observed that the declarative clause involving a first person subject (\ref{ex:gg23}) is marked in the same way, with the egophoric marker -\textit{s}, as the second person subject interrogative clause (\ref{ex:gg27}). These examples also show that in Awa Pit interrogatives involving first person subjects (\ref{ex:gg26}),
are marked in the same way, with the non-egophoric marker -\textit{y}, as the declarative clauses involving second person subjects (\ref{ex:gg24}). 



\ea 
\langinfo{Declaratives in Awa Pit}{}{\citealt[613]{Curnow2002a}}\\ 
	\ea \label{ex:gg23}
    \gll (na=na)	pala ku-mtu-s\\
         (1\textsc{s.(nom)=top}) plantain eat-\textsc{impf-ego}\\
    \glt ‘I am eating plantains.’
  \ex \label{ex:gg24}
    \gll (nu=na)	pala ku-mtu-y\\
   	(2\textsc{s.(nom)=top})	plantain	eat-\textsc{impf-n.ego}\\    
   	\glt ‘You are eating plantains.’
   \ex \label{ex:gg25}
   \gll (us=na)	atal	ayna-mtu-y\\
   (3\textsc{s.(nom)=top})	chicken	cook-\textsc{impf-n.ego}\\
   \glt He/she is cooking chicken.'
\z \z

\ea 
\langinfo{Interrogatives in Awa Pit}{}{\citealt[613, 614]{Curnow2002a}}\\ 
	\ea \label{ex:gg26}
    \gll min=ta=ma ashap-tu-y?\\
         who=\textsc{acc=int} annoy-\textsc{impf-n.ego}\\
    \glt ‘Whom am I annoying?’
  \ex \label{ex:gg27}
    \gll shi=ma 	ki-mtu-s?\\
   	what=\textsc{int} do-\textsc{impf}-\textsc{ego}\\    
   	\glt ‘What are you doing?’
   \ex \label{ex:gg28}
   \gll min=ta-s a-mtu-y?\\
   where=\textsc{loc-abl} 	come-\textsc{impf-n.ego}\\
   \glt 'Where is he coming from?'
\z \z

\tabref{tab:gg7} show the general usage pattern of egophoricity markers in TTK. Although it is similar to the Awa Pit system it also presents some differences. The TTK egophoric system has a contrast in the egophoric suffixes between singular -\textit{or} and plural -\textit{er} which does not exist in the non-egophoric suffix -\textit{an} and which is not observed in the other Barbacoan languages.

Furthermore, TTK’s system is not a fully symmetrical egophoricity system, since, as I already mentioned, there is a formal asymmetry with distinct interrogative constructions. TTK has two different predicate interrogative constructions, one for first and second person and another one for third person. In interrogatives involving first and second person subjects there is an obligatory copula auxiliary \textit{kɨ} which does not appear in interrogatives involving third person subjects.

% table 7
\begin{table}
\begin{tabularx}{\textwidth}{XXl}
\lsptoprule
\textbf{Person \& Number} & \textbf{Declarative} & \textbf{Interrogative}\\
\midrule
1\textsc{sg} & -\textit{or} & \textit{kɨ-a(n)}\\
2\textsc{sg} & -\textit{an} & \textit{kɨ-o(r)}\\
3\textsc{sg} & -\textit{an} & -\textit{a(n)}\\
\hline
1\textsc{pl} & -\textit{er} & \textit{kɨ-a(n)}\\
2\textsc{pl} & -\textit{an} & \textit{kɨ-e(r)}\\
3\textsc{pl} & -\textit{an} & -\textit{a(n)}\\
\lspbottomrule
\end{tabularx}
\caption{The general usage pattern of egophoricity markers in Totoró Namtrik}
\label{tab:gg7}
\end{table}


 


\subsection{Usage pattern of egophoricity markers in declaratives}\label{s:gg3-1}

TTK has two main kinds of predicate constructions: simple and complex predicate constructions. Simple predicate constructions, whose morphological structure is shown in \figref{fig:gg2}, are formed by a single verbal stem suffixed with the polarity marker, egophoricity markers and/or modal epistemic markers, as shown in examples (\ref{ex:gg29}) and (\ref{ex:gg30}).

%TODO figure 2
\begin{enumerate}[resume]
    \item Morphological structure of simple predicate constructions \label{fig:gg2}\\
\textsc{v}$_1$(-\textsc{neg})(-\textsc{ego.exp})-(\textsc{ego/non.ego})
\end{enumerate}



\ea \label{ex:gg29}
\langinfo{Totoró Namtrik}{}{namtrik\_012/32}\\
    \gll ichente na=pe nam-or\\
         so 1=\textsc{pd} get.angry-\textsc{ego.sg}\\
    \glt ‘Whom am I annoying?'
\z

\ea \label{ex:gg30}
\langinfo{Namtrik Totoró}{}{\citealt[29]{RojasCurieuxetal2009}}\\
    \glll nai	ishukpe	kumarmɨn\\
    na-wai	ishuk=pe kumar-mɨ-an\\
         1-\textsc{gen}	women=\textsc{pd}	sing-\textsc{neg-non.ego}\\
    \glt ‘My wife didn’t sing.’
\z

In TTK, simple predicate constructions do not carry aspectual morphology and the aspectual interpretation depends on the Aktionsart of the verb. While activities are interpreted as perfective past (\ref{ex:gg31}), states are interpreted as simple present (\ref{ex:gg32}).

\ea \label{ex:gg31}
\langinfo{Totoró Namtrik}{}{namtrik\_005/545}\\
    \glll nai	noshkai	yautomai	yan\\
    na-wai	noshkai	yauto-mai ya-an\\
         1-\textsc{gen} son Popayán-\textsc{dir} go.\textsc{sg-non.ego}\\
    \glt ‘My son went to Popayán.'
\z

\ea \label{ex:gg32}
\langinfo{Totoró Namtrik}{}{namtrik\_083/10, elic}\\
    \gll an=pe	mesa-yu	war-an\\
         money=\textsc{pd}	table-\textsc{loc} lain.down.\textsc{pl-non.ego}\\
    \glt ‘The money is on the table.’
\z

Namtrik also has a wide range of complex predicate constructions which vary widely in their morphosyntactic structure and the semantic relationships between the verbs comprising the complex predicates. \figref{fig:gg2} shows the minimal morphological structure of a complex predicate in TTK. Although V$_2$ and V$_1$ can be inflected with verbal morphology markers in complex predicates, there is a clear split between the kinds of markers that can be hosted by each one of the verbs in the complex predicate.

\largerpage[2]
As is shown in the examples (\ref{ex:gg33})--(\ref{ex:gg35}), the verb in V$_1$ position can host the polarity marker, egophoricity markers and/or modal epistemic markers, while the verb in position V$_2$ presents a non-finite, which can be inflected with the morpheme of prospective aspect (-\textit{tr}/-\textit{ch}/-\textit{ntr}) \textsc{pros}, the verbal nominalizers -\textit{ik} `\textsc{nmz.sg}' and -\textit{elɨ} `\textsc{nmz.pl}', and the morpheme of durative aspect (-\textit{ap}/-\textit{ip}/-\textit{ɨp}).

%TODO Figure 3
\begin{enumerate}[resume]
    \item Morphological structure of complex predicates V$_2$V$_1$ \label{fig:gg3}\\
    \textsc{v}$_2$(-\textsc{pros})(-\textsc{dur})(-\textsc{nmz.sg/-nmz.pl})	\textsc{v}$_1$(-\textsc{neg})(\textsc{ego exp})(-\textsc{ego.sg/-ego.pl})
\end{enumerate}


\ea \label{ex:gg33}
\langinfo{Totoró Namtrik}{}{namtrik\_036/39}\\
    \glll nai	kekpe	[kuallip	pasran]\\
    na-wai	kek=pe	kuall-ip pasr-an\\
         1-\textsc{gen}	husband=\textsc{pd}	work-\textsc{dur}	stand.\textsc{sg-non.ego}\\
    \glt ‘My husband went to work.'
\z

\ea \label{ex:gg34}
\langinfo{Totoró Namtrik}{}{namtrik\_034/73}\\
    \glll na	[isupik	war]\\
    na	isu-ap-ik wa-or\\
         1	think-\textsc{dur-nmz.sg}	sit.\textsc{sg}-\textsc{ego.sg}\\
    \glt ‘I think.'
\z

\ea \label{ex:gg35}
\langinfo{Totoró Namtrik}{}{namtrik\_059/154}\\
    \glll yu	misak	inchape	chineken	[pailantra	amɨn]\\
    yu	misak	incha=pe	chineken	paila-ntr-ap	amp-mɨ-an\\
         \textsc{loc} people	so nothing dance-\textsc{pros-dur} go.\textsc{pl-neg-non.ego}\\
    \glt ‘So, you guys aren't going to dance.'
\z

Although any verb of the language can fill the position in complex predicates V$_2$, the position V$_1$ can only be filled by a restricted set of verbs, which are shown in \tabref{tab:gg8}. In complex predicates the morphological structure of V$_2$ and the kind of auxiliary determine the modal and aspectual interpretation of the predicates. The complexity of the predicate constructions in TTK falls out of the scope of this paper, nevertheless it is important to keep in mind for the description of the egophoricity system that the verb in position V$_1$ hosts the egophoricity markers in complex predicates.

% table 8
\begin{table}
\begin{tabularx}{.8\textwidth}{XXXl}
\lsptoprule
 &  & \textsc{sg} & \textsc{pl}\\
\midrule
Posture auxiliaries & sit & \textit{wa}- & \textit{putr}-\\
& stand & \textit{pasr}- & \textit{pɨntr}-\\
& hang & \textit{mekua}- & \textit{mel}-\\
& lie & \textit{tso}- & \textit{war}-\\
\hline
Movement auxiliaries & go & \textit{ya}- & \textit{amp}-\\
& walk & \textit{uñ}-/\textit{un}- & \textit{amɨn}-/\textit{amɨñ}-\\
& come & \textit{atru}- & \textit{amtro}-\\
& arrive & \textit{po}- & \\
\hline
Others & copula & \textit{kɨ}- & \\
\lspbottomrule
\end{tabularx}
\caption{Auxiliary verbs in Totoró Namtrik}
\label{tab:gg8}
\end{table}


The canonical use of the egophoricity markers in declarative clauses is illustrated in examples (\ref{ex:gg36})-(\ref{ex:gg44}). The canonical use of the egophoric singular marker -or in declaratives is illustrated by examples (\ref{ex:gg36}) and (\ref{ex:gg37}). The canonical use of the egophoric plural marker -\textit{er} in declaratives involving first person plural subjects is illustrated by examples (\ref{ex:gg38}) and (\ref{ex:gg39}).

%1SG (EGO.SG)

\ea \label{ex:gg36}
\langinfo{Totoró Namtrik}{}{\citealt[12]{RojasCurieuxetal2009}}\\
    \gll mai-yu	peñ-or\\
         path-\textsc{loc} fall-\textsc{ego.sg}\\
    \glt ‘I fell at the path.'
\z

\ea \label{ex:gg37}
\langinfo{Totoró Namtrik}{}{namtrik\_012/54}\\
    \glll chente	mɨimpe	nape	unan	payor\\
    chente	mɨim=pe	na=pe	unɨ-wan	pay-or\\
         then	now=\textsc{pd}	1=\textsc{pd}	child-\textsc{dat} 	ask-\textsc{ego.sg}\\
    \glt ‘Then I asked the child.'
\z

%1PL (EGO.PL)

\ea \label{ex:gg38}
\langinfo{Totoró Namtrik}{}{namtrik\_031/107}\\
    \glll Mɨimpe	yu	wamiap	pɨntrer\\
    mɨim=pe	yu	wam-ia-ap	pɨntr-er\\
         now=\textsc{pd}	here	talk-\textsc{rec-nfv}	stand.\textsc{pl-ego.pl}\\
    \glt ‘Now we are talking here.'
\z

\ea \label{ex:gg39}
\langinfo{Totoró Namtrik}{}{namtrik\_037/224}\\
    \gll na-m=pe	pueblo-srɨ-mai	amp-er\\
         1-\textsc{pl=pd}	town-\textsc{loc.dist-dir}	go.\textsc{pl-ego.pl}\\
    \glt ‘We are going to town.'
\z

The canonical use of the non-egophoric -\textit{an} for second and third person, is illustrated in examples (\ref{ex:gg41})-(\ref{ex:gg44}). The non-egophoric suffix -\textit{an} is shown in a sentence with a second person singular subject in example (\ref{ex:gg41}), in (\ref{ex:gg42}) with a second person plural subject, in (\ref{ex:gg43}) with a third personal singular subject, and in (\ref{ex:gg44}) with a third plural person subject. 

The TTK egophoricity system has a contrast in the egophoric suffixes between singular -\textit{or} and plural -\textit{er} which does not exist in the non-egophoric suffix –\textit{an}. Nevertheless some verbal stems of auxiliary verbs, as is shown in \tabref{tab:gg8}, have a plural and a singular form, which allows us to identify the grammatical number of the subject. This is the case of the verbal stems in the examples (\ref{ex:gg43}) \textit{pasr-an} ‘stand.\textsc{sg-non.ego}’ and (\ref{ex:gg44}) \textit{putr-an} ‘sit-\textsc{pl-non.ego}’.

%2SG (NON EGO)

\ea \label{ex:gg41}
\langinfo{Totoró Namtrik}{}{namtrik\_037/149}\\
    \glll ñipe	nai	nimpasrwan	nilɨntrapɨk	kɨn\\
    ñi=pe	na-wai	nimpasr-wan	nilɨ-ntr-ap-ɨk	kɨ-an\\
         2=\textsc{pd}	1-\textsc{gen}	daughter-\textsc{dat }	steal-\textsc{pros-nfv-nmz.sg}	\textsc{cop-non.ego}\\
    \glt ‘You are about to steal my daughter.'
\z

%2PL (NON EGO)

\ea \label{ex:gg42}
\langinfo{Totoró Namtrik}{}{\citealt[20]{RojasCurieuxetal2009}}\\
    \glll  nimpe	kosrepik	kɨn\\
    ñi-m=pe	kosre-ap kɨn\\
         2=\textsc{pd}	teach-\textsc{dur} \textsc{cop-non.ego}\\
    \glt ‘You guys are teaching.'
\z

%3SG (non.ego)

\ea \label{ex:gg43}
\langinfo{Totoró Namtrik}{}{namtrik\_123/24, elic}\\
    \gll nɨ=pe	[peñ-ip	pasr-an]\\
         3=\textsc{pd}	fall-\textsc{dur}	stand.\textsc{sg-non.ego}\\
    \glt ‘He ‎‎is falling.'
\z
%TODO what does elic mean/refer to?

%3 PL (non.ego)

\ea \label{ex:gg44}
\langinfo{Totoró Namtrik}{}{\citealt[41]{RojasCurieuxetal2009}}\\
    \gll tul-io=pe	pɨn	wakra	putr-an\\
         field-\textsc{loc=pd}	three	cow	sit.\textsc{pl-non.ego}\\
    \glt ‘The cows are seat in the field.'
\z


\subsection{Usage pattern of egophoricity markers in interrogatives}\label{s:gg3-2}

\tabref{tab:gg9} shows the interrogatives predicate constructions in TTK. As we already mentioned TTK has two different interrogative constructions: one for first and second person and one for third person. The main difference between the two constructions is the obligatory use of a copula auxiliary \textit{kɨ}- in the interrogative clauses involving first and second person subjects which is not present in the interrogative clauses involving third person subjects.

% table 9
\begin{table}
\begin{tabularx}{.8\textwidth}{Xl}
\lsptoprule
\textsc{1sg} & (V$_2$) V$_1$-\textsc{irr-cop-non.ego}\\
\textsc{2sg} & (V$_2$) V$_1$-\textsc{cop-ego.sg}\\
\textsc{2pl} & (V$_2$) V$_1$-\textsc{cop-ego.pl}\\
\textsc{3} & (V$_2$) V$_1$-\textsc{non.ego}\\
\lspbottomrule
\end{tabularx}
\caption{Interrogative predicate constructions in Totoró Namtrik}
\label{tab:gg9}
\end{table}

The canonical use in interrogative clauses of the egophoricity markers is illustrated in examples (\ref{ex:gg45})-(\ref{ex:gg59}). Examples (\ref{ex:gg45}) and (\ref{ex:gg46}) show the use of the non-egophoric marker -\textit{an} in interrogatives involving a first person subject, singular in (\ref{ex:gg45}) and plural in (\ref{ex:gg46}). Since asking about one’s self is usually perceived as an odd situation for Namtrik speakers, data on first person subject interrogatives is very marginal in the corpus used. (\ref{ex:gg48}) is a natural speech example, and (\ref{ex:gg46}) was elicited based on the structure of natural speech examples of first person questions.

As \citet[13--14]{Floyd2018} notes regarding interrogatives in Chapalaa, first person interrogatives usually present a special marking which distinguishes them from second and third person interrogatives. As is shown in \tabref{tab:gg9} and in the examples (\ref{ex:gg45}) and (\ref{ex:gg46}), first person interrogatives additionally have an irrealis marker -\textit{am} which is not present in the interrogative clauses involving second and third person subjects.

Given the small amount of data concerning first person questions, it is not possible yet to argue if TTK presents a single construction for polar and content first person questions. However it is possible to recognize the use of the non-egophoric marker -\textit{an} in interrogative clauses involving first person subjects, keeping in mind the morphophonological processes affecting this marker in questions is already described in the first part of this paper.

%1SG (NON EGO)

\ea \label{ex:gg45}
\langinfo{Totoró Namtrik}{}{namtrik\_045/140}\\
    \glll chu pasramkɨ chi\\
    chu	 pasr-am=kɨ-an	chi\\
         where	stand.\textsc{sg-irr=cop-non.ego}	\textsc{q}\\
    \glt ‘Where shall I put this?'
\z


%1PL (NON EGO)

\ea \label{ex:gg46}
\langinfo{Totoró Namtrik}{}{namtrik\_078/86, elic}\\
    \glll ñipe	mai	kusrenanmɨtan shente nampe	makatɨ maramkɨ\\
    ñi=pe maik kusre-nan-mɨ-t-an shente na-m=pe makatɨ	mar-am-kɨ-an\\
         2=\textsc{pd} well learn-\textsc{caus-neg-exp-non.ego} then 1-\textsc{pl=pd}	how	do-\textsc{irr-cop-non.ego}\\
    \glt ‘If you don’t teach well, then what should we do?'
\z

The use of the singular egophoric marker -\textit{or} in interrogatives involving second singular person subjects is illustrated in examples (\ref{ex:gg47})-(\ref{ex:gg50}).
Examples (\ref{ex:gg47}) and (\ref{ex:gg48}) show the use of the egophoric marker -\textit{or} in content questions and examples (\ref{ex:gg49}) and (\ref{ex:gg50}) in polar questions. TTK has the same predicate construction for second person content and polar questions.


%2SG (EGO.SG)
%Content question

\ea \label{ex:gg47}
\langinfo{Totoró Namtrik}{}{namtrik\_037/80}\\
    \glll chine	marɨp	unko(.)	ñipe	ɨsrɨpe\\
    chi-ne	mar-ɨp	un-kɨ-or	ñi=pe	ɨ-srɨ=pe\\
         \textsc{q}-3	do-\textsc{dur}	walk.\textsc{sg}-\textsc{cop-ego.sg} 2=\textsc{pd}	\textsc{dist-dir=pd}\\
    \glt ‘What are you doing here?' (lit. What are you doing walking here?)
\z

\ea \label{ex:gg48}
\langinfo{Totoró Namtrik}{}{namtrik\_042/180}\\
    \glll ñipe	chine	isu	wako(.)	nai 	Geny\\
    ñi=pe	chi-ne	isu-ap	wa-kɨ-or na-wai	Geny\\
	2=\textsc{pd} \textsc{q-3} think-\textsc{dur} sit.\textsc{sg-cop-ego.sg} \textsc{1-gen} Geny\\
	\glt ‘What are you thinking my Geny?’ 
\z

%Polar question

\ea \label{ex:gg49}
\langinfo{Totoró Namtrik}{}{namtrik\_058/218}\\
    \glll ɨnɨpe	ñi	wako\\
    ɨ-nɨ=pe	ñi	wa-kɨ-or\\
	\textsc{dist}-3=\textsc{pd}	2	seat.\textsc{sg-cop-ego.sg}\\
	\glt ‘Are you seated in that one (bench)?’
\z

\ea \label{ex:gg50}
\langinfo{Totoró Namtrik}{}{namtrik\_058/64}\\
    \glll kishi	wako\\
    kish-ip	 wa-kɨ-or\\
	cry-\textsc{dur} seat.\textsc{sg-cop-ego.sg}\\
	\glt ‘Are you crying?’ 
\z

The use of the plural egophoric marker -\textit{er} in content questions involving plural second person subjects is illustrated in examples (\ref{ex:gg51}) and (\ref{ex:gg52}). The use of the plural egophoric marker in polar questions is shown in examples (\ref{ex:gg53}) and (\ref{ex:gg54})

%2PL (EGO.PL)

%Content question

\ea \label{ex:gg51}
\langinfo{Totoró Namtrik}{}{namtrik\_003/28}\\
	\glll chi pɨrɨ kishke\\
	chi	pɨrɨ kish-kɨ-er\\
	\textsc{q} why cry-\textsc{cop-ego.pl}\\
	\glt `Why are you (kids) crying?'
\z

\ea \label{ex:gg52}
\langinfo{Totoró Namtrik}{}{namtrik\_031/110}\\
	\glll chinetɨ warɨke\\
	chi-nɨ-tɨ war-ɨ-kɨ-er\\
	\textsc{q}-3-\textsc{rest}	lie.\textsc{pl-e-cop-ego.pl}\\
	\glt `What did you bring?'
\z

%Polar question

\ea \label{ex:gg53}
\langinfo{Totoró Namtrik}{}{namtrik\_031/154}\\
	\glll takiwan kucha pasrke\\
	taki-wan	kucha	pasr-kɨ-er\\
	chagla-\textsc{dat }	too	stand.\textsc{sg-cop-ego.pl}\\
	\glt `Did you leave the \textit{chagla} sticks there too?'
\z


\ea \label{ex:gg54}
\langinfo{Totoró Namtrik}{}{namtrik\_075/87}\\
	\glll uyu mɨrmɨke\\
	uyu mɨr-mɨ-kɨ-er\\
	here listen-\textsc{neg-cop-ego.pl}\\
	\glt `Didn’t you guys listen?’
\z



Regarding interrogatives in the variety of Namtrik spoken in Guambiano, \citet{Norcliffe2018} proposes that “in interrogatives involving second person subjects, the verb in its bare stem form is obligatorily followed by a particle which takes the form of either \textit{ku}/\textit{ke} or \textit{titru}. \textit{Ku} is used for singular second person addressees, while \textit{ke} is used for plural second person addressees'', as is shown in examples (\ref{ex:gg55}), (\ref{ex:gg56}) and (\ref{ex:gg57}). 

\ea \label{ex:gg55}
\langinfo{Guambiano Namtrik}{}{\citealt{Norcliffe2018}}\\
	\gll chi	mar-ku?\\
	\textsc{q} do-\textsc{ku}\\
	\glt ‘What are you (\textsc{sg}) doing?’
\z

\ea \label{ex:gg56}
\langinfo{Guambiano Namtrik}{}{\citealt{Norcliffe2018}}\\
	\gll chi	mar-ke?\\
	\textsc{q} do-\textsc{ke}\\
	\glt ‘What are you (\textsc{pl}) doing?’
\z

\ea \label{ex:gg57}
\langinfo{Guambiano Namtrik}{}{\citealt{Norcliffe2018}}\\
	\gll ñi	mana	lanchi-titru?\\
	2/3.\textsc{prox} when break-\textsc{titru}\\
	\glt ‘When did you break it?’
\z

In the case of TTK, the egophoricity system exhibit a verbal marking pattern, which is consistent with the description of egophoricity systems in other languages in which “speaker subjects in statements are marked the same way as addressee subjects in questions” (\citealt{Curnow2002a}). As is shown in examples (\ref{ex:gg47})-(\ref{ex:gg54}), the forms \textit{ko} and \textit{ke} observed in questions involving second person singular and plural in Guambiano Namtrik, correspond to the copula \textit{kɨ}- followed by the egophoric markers -\textit{or} and -\textit{er} in Totoró Namtrik. 

The canonical use of the non-egophoric marker -\textit{an} in questions involving a third person is illustrated in the examples (\ref{ex:gg58}) and (\ref{ex:gg59}). TTK also has the same construction for content and polar questions involving third person subjects.

%3 (non.ego)
%Content question

\ea \label{ex:gg58}
\langinfo{Totoró Namtrik}{}{namtrik\_057/126}\\
	\glll inɨpe tsik kɨ chine kalus kɨ chi\\
	i-nɨ=pe	tsik	kɨ-an	chi-ne	kalus	kɨ-an	chi\\
	\textsc{prox-3=pd}	wood	\textsc{cop-no.ego}	\textsc{q-3}	leather	\textsc{cop-no.ego}	\textsc{q}\\
	\glt ‘Is this wood or what, leather?’
\z


%Polar question

\ea \label{ex:gg59}
\langinfo{Totoró Namtrik}{}{namtrik\_049/14}\\
	\glll unɨ mɨkpe ciego kɨ chi?\\
	unɨ	mɨk=pe	ciego	kɨ-an	chi\\
	child	man=\textsc{pd}	blind	\textsc{cop-no.ego}	\textsc{q}\\
	\glt ‘Is the boy blind, or what?’
\z


\section{Egophoricity markers and the category of person}\label{s:gg4}

As San Roque and colleagues note regarding egophoricity marking and subject agreement marking, “under a person-marking account, egophoricity markers flag the coincidence of the epistemic authority and an argument role” (\citealt[33]{SanRoque2018}). Since subjects usually “have good grounds to declarative direct personal knowledge of an event, it naturally follows that declaratives with a first person subject usually take egophoric marking, as do interrogatives with a second person subject” (\citealt[33]{SanRoque2018}).

This approach may describe much of the patterns in TTK, but has trouble accounting for the data which show that under certain semantic and pragmatic factors, the non-egophoric suffix can be used with first person subjects in declaratives as well and that the egophoric plural marker -\textit{er} can be used in question clauses with a third person subject.

It was attested in egophoricity marking in different languages (\citealt{SanRoque2018}) that egophoric markers usually do not target non-volitional or non-control first person subjects. As is shown in examples (\ref{ex:gg60}), (\ref{ex:gg61}) and (\ref{ex:gg62}), in TTK egophoric marking is not used in declarative clauses which describe a non-volitional or non-controlled event performed by a first person subject.

Additionally, examples (\ref{ex:gg60}) and (\ref{ex:gg61}) also show the non-control marker -\textit{ra}, which has also been reported in Guambiano by \citet[98--99]{Vasquez2007} and serves to express the speaker’s lack of control or evidence with respect to the described event. This marker takes part in a specific complex predicate construction whose structure is shown in \figref{fig:gg4}, and which however is a construction restricted to first person subjects, and is never marked with the egophoric singular marker -\textit{or} but with the non-egophoric marker -\textit{an}.

%TODO figure 4
\begin{enumerate}[resume]
    \item Non-control event complex predicate construction \label{fig:gg4}\\
    \textsc{v}$_2$-\textsc{dur}-\textsc{non}.\textsc{control-nmz} \textsc{cop-non.ego}(-\textsc{dub})
\end{enumerate}

\ea \label{ex:gg60}
\langinfo{Totoró Namtrik}{}{namtrik\_070/35}\\
	\glll tsap kuaprein kɨntro pesanai\\
	tsa-ap	kua-ap-ra-in kɨ-an-tro	pesana-ik\\
	end-\textsc{dur} die-\textsc{dur-non.contr-nmz} \textsc{cop-non.ego-dub} faint-\textsc{nmz.sg}\\
	\glt ‘There being sick at night, I fainted.’  
\z

\ea \label{ex:gg61}
\langinfo{Totoró Namtrik}{}{namtrik\_078/75, elic}\\
	\glll kiprain	parin	kiprain	kɨn nape	mɨintɨ	kasrar\\
	ki-ap-ra-in	parin	ki-ap-ra-in	kɨ-an na=pe	mɨin-tɨ	kasra-or\\
	sleep-\textsc{dur-non.contr-nmz} a.lot sleep-\textsc{nfv-non.contr-nmz} \textsc{cop-non.ego} 1=\textsc{pd} now-\textsc{rest} wake.up-\textsc{ego.sg}\\
	\glt ‘‎‎I was sleeping a lot and I barely woke up.’ 
\z


\ea \label{ex:gg62}
\langinfo{Totoró Namtrik}{}{namtrik\_070/1}\\
	\glll nape	kuap	sreik	kɨpɨk	kɨn nape	srɨ	kasareik\\
	na=pe kua-ap sre-ik	kɨ-ap-ik kɨ-an na=pe srɨ kasar-ap-ik\\
	1=\textsc{pd}	 die-\textsc{nfv} escape-\textsc{nmz.sg} \textsc{cop-dur-nmz.sg} \textsc{cop-non.ego} 1=\textsc{pd} \textsc{loc.dist}	get.married-\textsc{nfv-nmz.sg}\\
	\glt ‘I almost died when I was just married.’ 
\z


Although the general usage pattern of the Namtrik egophoricity system seems to be very consistent, the system has some flexibility and there are examples where the markers are not used according to the expected pattern. In examples (\ref{ex:gg63})--(\ref{ex:gg64}) there is an egophoric plural marker -\textit{er} in question clauses with a third person subject. It is not clear what motivates the use of the egophoric plural marker -\textit{er} in these sentences. 


\ea \label{ex:gg63}
\langinfo{Totoró Namtrik}{}{namtrik\_035/15}\\
	\gll gorro=pe	muna	un-er\\
	cap=\textsc{pd}	where	walk.\textsc{sg}-\textsc{ego.pl}\\
	\glt‘Where is the hat?’ (lit. Where is the hat walking?)
\z 

\ea \label{ex:gg64}
\langinfo{Totoró Namtrik}{}{namtrik\_078/104}\\
	\glll nai Genype mai munik tsuwer\\
	na-wai	Geny=pe	maik mun-ik	tsu-wa-er\\
	1-\textsc{gen}	Geny=\textsc{pd}	well	wait-\textsc{nmz.sg}	lie.down-sit.\textsc{sg}-\textsc{ego.pl}\\
	\glt ‘Geny must be waiting for me? (I’m worried about it)’
\z 


\section{Undergoer and egophoricity}\label{s:gg5}

The experiencer egophoric marker -\textit{t} is involved in different kinds of constructions in TTK. In complex predicate constructions, this marker is always suffixed to the auxiliary verb. This marker has been reported also in previous descriptions of Guambiano by \citet[101]{Vasquez2007} and \citet[615]{TrivinoGarzon1994}. \citet{Vasquez2007} described the marker -\textit{t} as a first person applicative, which has epistemic-evidential functions, indicating that there is a first person which is perceiving some information described in the clause (\citealt[101]{Vasquez2007}). 
More recently \citet{Norcliffe2018}, writing about evidential and egophoricity markers in Guambiano, describes this marker as “a subtype of ego evidentiality ... used in contexts in which the speaker experiences an event as an undergoer, and thus has knowledge of the event as an affected participant” (\citeyear[25]{Norcliffe2018}).

In TTK the main function of this marker is similar to the function that has been described by \citet{Vasquez2007}, \citet{TrivinoGarzon1994} and \citet{Norcliffe2018}. The main function of the -\textit{t} morpheme is to mark a speaker which is in some way affected by the situation described in the clause, as illustrated in the examples below, which show the undergoer marker -\textit{t}  in intransitive declarative clauses. 

\newpage
\ea
\langinfo{Totoró Namtrik}{}{namtrik\_042/240, namtrik\_020/5, namtrik\_075/11}\\ 
	\ea \label{ex:gg65a}
    \glll nakish pesrik	kɨtan kashi\\
    nakish	pesrik	kɨ-t-an	kashi\\
         smoke	angry \textsc{cop-exp-non.ego} a.bit\\
    \glt ‘The smoke is bothering me a bit.’ (lit. The smoke is angry, which affects me a bit.)
  \ex \label{ex:gg65b}
    \glll kana sruk tɨwei wepitan\\
    kana	sruk	tɨwei	we-ap-i-t-an\\
   	one	stone only come.out-\textsc{dur-nmz.e-exp-non.ego}\\    
   	\glt ‘[My father] just found a stone, (which the duende had hidden to make my father him sick, my father said).’
   \ex \label{ex:gg65c}
   \glll ashampe si	kek	palɨtan\\
   ashan=pe	si	kek	pal-ɨ-t-an\\
   now=\textsc{pd} yes	husband	lack-\textsc{e-exp-non.ego}\\
   \glt `Now I need a husband.'
\z \z

The undergoer marker -\textit{t} may be suffixed to both intransitive verbs and transitive verbs. In declarative sentences with transitive verbs, the suffix -\textit{t} marks an ‘affected’ speaker, which in these cases is expressed as an object.

\ea
\langinfo{Totoró Namtrik}{}{\citealt[16, 25]{RojasCurieuxetal2009}}\\ 
	\ea \label{ex:gg66a}
    \glll nai kek kuakɨtan\\
    na-wai	kek	kuakɨ-t-an\\
         1-\textsc{gen}	husband	hit-\textsc{exp-non.ego}\\
    \glt ‘‎‎My husband hit me.’
  \ex \label{ex:gg66b}
    \glll nai notsak usmai larɨtan nan\\
    na-wai	notsak	usmai	lar-ɨ-t-an	na-wan\\
         1-\textsc{gen}	sister	head.down	see-\textsc{nmz.sg?-exp-non.ego}	1-\textsc{dat}\\    
   	\glt ‘My sister looked at me with her head down.’
\z \z

As is the case in other Barbacoan languages (\citealt[25]{Floyd2018}), TTK has specific predicate constructions for experience verbs, including internal states, endophatic states, emotions and desires. In these cases, the non-egophoric suffix -\textit{an} is used with first person subjects in combination with the undergoer egophoric marker -\textit{t}.

\newpage
\ea
\langinfo{Totoró Namtrik}{}{namtrik\_019/6, LEXIN2/62}\\ 
	\ea \label{ex:gg67a}
    \glll nante	kiri kitan chilliyu	puraintrap\\
    na-wan-te kiri kɨ-t-an chilli-yu pura-i-ntr-ap\\
         1-\textsc{dat-rest} get.scared \textsc{cop-exp-non.ego} mud-\textsc{loc} go.through-go.\textsc{sg-pros-nfv}\\
    \glt ‘I got scared going through the mud.’ (lit. There was fear in me going through the mud.)
  	\ex \label{ex:gg67b}
    \glll nante kitra kitan\\
    na-wan-te kitra kɨ-t-an\\
        1-\textsc{dat-rest}	be.cold	\textsc{cop-exp-non.ego}\\    
   	\glt ‘I am cold.’ (lit. Cold is to me.)
  	\ex \label{ex:gg67c}
  	\glll nante	intsa kɨtan\\
  	na-wan-te intsa kɨ-t-an\\
  	1-\textsc{dat-rest} laugh \textsc{cop-exp-non.ego}\\
	\glt `I laugh.' (lit. There is laughing for me.) 
\z \z

The data from Totoró Namtrik shows that the undergoer egophoric marker -\textit{t} seems to have an egophoricity distribution. That is to say, the suffix -\textit{t} refers to the speaker being affected in declarative clauses and to the hearer being affected in interrogative clauses. 

The predicate construction involving the undergoer egophoric marker, which is shown in \figref{fig:gg5}, consists of a main verb followed by the suffix -\textit{t} and the non-egophoric suffix -\textit{an}. In interrogative clauses, the consonant in the non-egophoric suffix -\textit{an} follows the same pattern of consonant deletion found in the predicate constructions without the suffix -\textit{t}.

%TODO figure 5
\begin{enumerate}[resume]
    \item Undergoer egophoric marker predicate construction \label{fig:gg5}\\
    \textsc{v-exp-non.ego}
\end{enumerate}

The following set of examples shows the pattern of usage of the undergoer egophoric marker in declarative clauses and interrogative clauses. 
\largerpage[1]

%Declarative
%First person

\ea
\langinfo{Totoró Namtrik}{}{namtrik\_057/7, LEXIN2\_101, namtrik\_052/307, namtrik\_059/86}\\ 
	\ea \label{ex:gg68a}
	\glll mɨimpe kana paso palɨtan\\
	mɨim=pe kana paso pal-ɨ-t-an\\
	now=\textsc{pd}	one	glass lack-\textsc{nmz.sg?-exp-non.ego}\\
	\glt ‘Now I need a glass ’ (lit. Now a glass lacks, which affects me.)
	
	\ex \label{ex:gg68b}
	\glll ɡuaranɡope	trumpik	itrɨtan\\
	ɡuaranɡo=pe	trumpik	itrɨ-t-an\\
	guarango=\textsc{pd} ugly smell-\textsc{exp-non.ego}\\
	\glt ‘The guarango wood stinks.’ (lit. Guarango wood smells ugly, which affects me.)
	
	\ex \label{ex:gg68c}
	\glll tapɨ	kɨtan\\
	tapɨ kɨ-t-an\\
	well \textsc{cop-exp-non.ego}\\
	\glt ‘It’s okay for me.’
	
	\ex \label{ex:gg68d}
	\glll nantɨ	pailakɨtan\\
	na-wan-tɨ paila-kɨ-t-an\\
	1-\textsc{dat-rest} dance-\textsc{cop-exp-non.ego}\\
	\glt ‘I want to dance.’ (lit. There is dancing for me.)
\z \z 

%takes addressee perspective in interrogatives
%Interrogative
%Second person

\ea
\langinfo{Totoró Namtrik}{}{namtrik\_059/126, namtrik\_058/256, namtrik\_057/438, namtrik\_037/144}\\ 
	\ea \label{ex:gg69a}
	\glll chillo	palɨta\\
	chillo	pal-ɨ-t-an\\
	knife	lack-\textsc{nmz.sg?-exp-non.ego}\\
	\glt ‘Do you need a knife? (lit. Does it lack a knife for you?)
	
	\ex \label{ex:gg69b}
	\glll maik itrɨta\\
	maik itrɨ-t-an\\
	delicious smell-\textsc{exp-non.ego}\\
	\glt ‘Does it smell delicious (according to you)?’ (lit. Does it smell good to you?)
	
	\ex \label{ex:gg69c}
	\glll tapɨ kɨta\\
	tapɨ kɨ-t-an\\
	well \textsc{cop-exp-non.ego}\\
	\glt ‘Is it okay for you (according to you)?’
	
	\ex \label{ex:gg69d}
	\glll mai mutsikɨta	ñipe\\
	maik muts-i-kɨ-t-an ñi=pe\\
	well drink-\textsc{nmz.sg?-cop-exp-non.ego} 2=\textsc{pd}\\
	\glt ‘Do you want to drink?’
\z \z


\section{Conclusions}\label{s:gg6}

Totoró Namtrik data shows a regular egophoricity pattern, exhibiting a very similar distribution as the one found in other Barbacoan languages. However, its first recognition is obscured by two morphophonological processes: vowel coalescence at a morpheme boundary and deletion of the final consonant of the egophoric and non-egophoric suffixes in questions. The fact that TTK possesses an egophoricity system is a good argument for claiming that egophoricity is a genetic feature in Barbacoan languages, as argued by \citet{CurnowLiddicoat1998}. 

The system that I illustrate in this paper is very similar to egophoricity systems in other languages. In declarative clauses the egophoric singular -\textit{or} and plural -\textit{er} are used for first person and the non-egophoric -\textit{an} for second and third person. In interrogative clauses, the egophoric singular suffix -\textit{or} is generally used for singular second person, the egophoric plural suffix -\textit{er} for plural second person and the non-egophoric -\textit{an} for first and third person. TTK shows a single egophoricity system both in declarative and interrogative clauses. In this respect the TTK data differ from the Guambiano data, as presented by \citet{Norcliffe2018}, which show two separate systems for declarative and interrogative clauses. 

Further the TTK data shows an undergoer egophoric suffix -\textit{t}, which seems to function as an egophoric undergoer suffix, signifying that the speaker is in some way affected in declarative sentences. The undergoer egophoric suffix -\textit{t} appears to be used in questions to refer to the hearer being affected. This would mean that this suffix shows an egophoric distribution as well. However, more data is needed to confirm this.

The Totoró Namtrik egophoricity system does not always behave according to the general usage pattern. These non-canonical uses of the suffixes in the egophoricity system will play an important role in the semantic analysis of this system.


  
\section*{Abbreviations}
\begin{tabularx}{.45\textwidth}{lQ}
1 & first person\\
2 & second person\\ 
3 & third person\\ 
\textsc{cop} & copula\\ 
\textsc{dat} & dative\\ 
\textsc{dist} & distal\\ 
\textsc{dir} & directive\\
\end{tabularx}
\begin{tabularx}{.45\textwidth}{lQ}
\textsc{dub} & dubitative\\
\textsc{dur} & durative\\
\textsc{e} & epenthesis\\
\textsc{ego} & egophoric\\
\textsc{exp} & experiencer\\
\textsc{gen} & genitive\\
\textsc{loc} & locative\\
\end{tabularx}

\begin{tabularx}{.45\textwidth}{lQ}
\textsc{imp} & imperative\\ 
\textsc{irr} & irrealis\\ 
\textsc{neg} & negation\\ 
\textsc{nmz} & nominalization\\ 
\textsc{non.contr} & non-control\\ 
\textsc{non.ego} & non-egophoric\\ 
%\textsc{om} & marked object\\ 
\textsc{pd} & discursive particle\\ 
\end{tabularx}
\begin{tabularx}{.45\textwidth}{lQ}
\textsc{pros} & prospective\\ 
\textsc{pl} & plural\\ 
\textsc{q} & question\\ 
\textsc{rec} & reciprocal\\ 
\textsc{rest} & restrictive\\ 
\textsc{sg} & singular\\
\\
\end{tabularx}

%\begin{tabularx}{.45\textwidth}{lQ}
%... & \\
%... & \\
%\end{tabularx}


\section*{Acknowledgements}
I would like to thank Martine Bruil for all her support, ideas and stimulating discussions which fed our presentation in the “Symposium on evidentiality, egophoricity, and engagement: descriptive and typological perspectives” (Stockholm University 17--18 March 2016) and this paper. I would also like to thank my advisers Tulio Rojas Curieux, Antoine Guillaume and Spike Gildea as well as Esteban Díaz for all their suggestions and ideas for my analysis and their support. I would also like to thank the Cabildo authorities, teachers and the Namtrik speakers from the community of Pueblo Totoroez: Erminia Sánchez, Gertrudis Benachí, Encarnación Sánchez, Gerardina Sánchez, Carmen Tulia Sánchez, Carolina Lúligo, Micaela Lúligo, Juanita Sánchez, Carlina Conejo, Ismenia Sánchez, Aristides Sánchez, Marco Antonio Ulcué, Transito Sánchez and  José María Sánchez. I would also like to express my deep gratitude to Lucy Elena Tunubalá Tombé for her invaluable work and help in the transcription and translation of the data presented in this paper. Finally, I would like to thank  the Ecole doctorale 3LA, Université Lumière Lyon 2, Laboratoire Dynamique Du Langage, the Grupo de Estudios Lingüísticos Pedagógicos y Socioculturales del Suroccidente Colombiano, Universidad del Cauca and the Hans Rausing Endangered Language Documentation Programme (HRELP-SOAS) for supporting my research.

\sloppy
\printbibliography[heading=subbibliography,notkeyword=this]
\end{document}
