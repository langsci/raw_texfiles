\documentclass[output=paper]{langsci/langscibook} 
\ChapterDOI{10.5281/zenodo.1048175} %will be filled in at production

\author{Tomáš Svoboda\affiliation{Charles University, Prague}\and 
Łucja Biel\affiliation{University of Warsaw}\lastand
Krzysztof Łoboda\affiliation{Jagiellonian University, Kraków}}
\title{Quality aspects in institutional translation: Introduction}
% \shorttitlerunninghead{If the normal title is too long in the page headers}
\abstract{\noabstract}
\maketitle

\begin{document}
  
\section{Introduction}

Quality has been on translation scholars’ minds since the emergence of Translation Studies (TS) as a discipline in the 1970s, with one of the seminal monographs by Juliane House being published in 1977. More recently, with TS shifting its focus to integrate non-literary texts more broadly (cf. \citealt{Rogers2015}), the quality aspect has been researched across various specialized fields and genres. One of these fields is Institutional Translation, where the quest for product and process quality underlies the \textit{raison d'être} of in-house translation teams. This field requires further in-depth research into quality aspects to combine and cross-fertilize theory and practice.

The purpose of this collective monograph is to explore key issues, approaches and challenges to quality in institutional translation by confronting academics’ and practitioners’ perspectives. What the reader will find in this book is an interplay of two approaches: academic contributions providing the conceptual and theoretical background for discussing quality on the one hand, and chapters exploring selected aspects of quality and case studies from both academics and practitioners, on the other hand. Our aim is to present these two approaches as a breeding ground for testing one vis-à-vis the other. 

This book studies institutional translation mostly\footnote{Except for Prieto Ramos’ and Vandepitte’s chapters which also survey supranational, intergovernmental and/or centralised national organizations.} through the lens of the European Union (EU) reality, and, more specifically, of EU institutions and bodies, due to the unprecedented scale of their multilingual operations and the legal and political importance of translation. Thus, it is concerned with the supranational (international) level, deliberately leaving national\footnote{See \citet{Svoboda2017} for literature review of quality aspects in national institutional translation settings.} and other contexts aside. Quality in supranational institutions is explored both in terms of translation processes and products – the translated texts. 

\section{Kraków and Prague TEW conferences as an initial stage for the book project}

This collective monograph is inspired, partially, by two conferences held as part of a joint Translating Europe Workshop event\footnote{The event was held under the \#TranslatingEurope project, which aims to bring together stakeholders in the translation profession across Europe. The project consists of the yearly forum organised in Brussels and the workshops, which are smaller events (conferences, seminars, round tables) targeted towards more regional level, at specialised audiences. The workshops are often organised in cooperation with EMT (European Master's in Translation) universities.} supported by the European Commission’s Directorate-General for Translation (DGT): a conference entitled \textit{Points of View on Translator's Competence and Translation Quality} held in Kraków in  November 2015 and the \textit{Quality Aspects in Institutional Translation} conference held in Prague in  November 2016. The former was organised with the aim of attracting a broad audience of both Translation Studies scholars and translation practitioners to tackle the concept of quality from as many angles as possible while the Prague follow-up built up on its findings and focused narrowly on quality in supranational institutions. Selected speakers were invited to contribute to this collective monograph with its overarching theme of quality. Subsequently, the invitation was extended to a few academics and practitioners working in this area.

\section{Institutional translation and quality: basic concepts}

This book addresses the institutional nature of translations, which has been acknowledged to be “a neglected factor” in Translation Studies (cf. \citealt[470]{Mason2003[2004]}). \textit{Institutional translation} can be defined in broad or narrow terms. We adopt Schäffner et al.’s definition, which seems to represent a balanced account:

\begin{quote}
In the widest sense, any translation that occurs in an institutional setting can be called institutional translation, and consequently the institution that manages translation is a translating institution. In Translation Studies, however, the label ``institutional translation'' is generally used to refer to translating in or for a specific organisation… Institutional translation is typically collective, anonymous and standardised. (\citeyear{SchäffnerEtAl2014}: 493–494)
\end{quote}

As Schäffner et al. argue, the fact that institutional translation is “typically collective, anonymous and standardised” (2014: 494) requires institutions to ensure the lexical, grammatical and stylistic consistency of translations. Such standardization is achieved through “style guides and CAT tools, revision procedures, and mentoring and training arrangements” (ibid.). Thus, standardization may be regarded as one of the defining features of institutional translation.

Given the divergent conceptualizations of the term ‘institutional translation’ and the narrow grounds against which the term was initially coined (i.e. supranational institutions, especially institutions/bodies of the EU), \citet{Koskinen2014} addresses the definition of institutional translation through the question of “what purpose(s) translation serves in institutions” (2014: 480) and studies the topic of governance in the context of translating institutions. Her approach is inspirational in two ways: it offers a way of approximating divergent research endeavours in the field and, beyond that, it offers a broad platform to interpret research results. 

The present book is an in-depth consideration of one of the many aspects of institutional translation – yet one of key importance – both as regards research and translation practice within institutions – namely quality. \textit{Quality} can be defined in many ways. In the industrial/commercial practice, with which institutional contexts tend to have increasingly more in common (cf. \citealt{Mossop2006}), quality – in connection with the ISO 9000 standards (cf. \citealt{ISO2015b}) – is often understood as a degree to which the inherent characteristics of a product or a process fulfil the clients’ expectations. 

An important distinction which is made in the literature and in the translation industry (cf. \citealt{Drugan2013}) holds between \textit{quality assessment}, that is attempts at objective measurements of quality of translated texts, versus \textit{quality assurance}, that is systematic attempts at controlling the quality of processes\footnote{See Drugan for an overview and differences between the academia and the industry (2013: 35--38), as well as for definitions of related terms: quality assurance, quality evaluation, quality control, quality assurance, quality planning and quality improvement (2013: 76--77).}. This book is concerned both with the quality of the translation process, including quality management policies, and – on a conceptual plane – with the outcome of the translation process, i.e. translation products (cf. mainly Vandepitte, in this volume). The process-oriented approach is linked with the notion of quality assurance (QA), which \citet{Mossop2001} defines as:

\begin{quote}
… the full set of procedures applied before, during and after the translation production process, by all members of a translating organisation, to ensure that quality objectives important to clients are being met. Quality assurance includes procedures to ensure [...] [q]uality of service [...,] [q]uality of the physical product [... and] [q]uality of the translation. [...] Where work is being done on contract, quality assurance includes selecting the best contractor. (2001: 92--93)
\end{quote}


Thus, quality assurance is understood in a holistic way to cover all stages of translation provision. This collective monograph adopts Mossop’s broad definition to explore how – in order to assure and control the quality of translations as products – institutions control processes, people and resources, including the hiring of quality managers (Prieto Ramos, this volume), terminology management (Stefaniak, this volume), standardization through style guides and translation manuals (Svoboda, this volume), as well as outsourcing evaluation (Strandvik, this volume), to name but a few of the aspects at hand. 

\section{Research on institutional translation through the lens of quality}

Quality aspects of translation at international/supranational level have been researched theoretically (cf. \citealt{PrietoRamos2015,PrietoRamos2015}) and practically, mainly in the context of the United Nations (UN) and the European Union (EU).

In respect of the UN,  \citet{DeSaintRobert2009} details UN’s approach to translation quality assessment, pointing out client orientedness as a major component of the UN communication strategy. \citet{Didaoui2009} locates the role of UN translators in the UN translation quality management (QM) system, thus putting the person of a translator in the foreground. The focus on human resources is maintained in a PhD dissertation by \citet{Lafeber2012}, who focuses on skills and knowledge required of translators and revisers in 24 inter-governmental organizations and correlates her findings with recruitment tests at such organizations, particularly with some insider knowledge from the UN translation service.

As for the EU, the topic of quality has been given more attention in recent years with a growing number of publications, both by academics and EU institutions. In respect of products, \citet[104--106]{Koskinen2008} approaches translation quality from the point of view of readability. A similar focus may be observed in empirical studies which analyze the textual fit of translations against national conventions for specific genres, e.g. multilingual legislation \citep{Biel2014}. Another textual-level aspect concerns terminological consistency of EU translations (\citealt{PachoAljanati2017}). Quite a few studies approach quality from the process perspective. A practical example of a guideline in translation quality in an institutional setting provides the European Commission \citet{DGT2009}. Another practice-oriented resource is the European Commission’s study (DGT 2012), which quantifies, among other things, potential losses in scenarios when less ambitious quality assurance measures would be applied within the Directorate-General for Translation (DGT). Similarly to \citet{Didaoui2009} and the way he discribes the UN translation department, \citet{Svoboda2008} follows the same aim of locating the individual within the quality management system within the DGT workflow. A review of the translation quality requirements with EU institutions’ outsourcing procedures is given in \citet{Sosoni2011}. Most recently, \citet{Strandvik2015,Strandvik2017} and \citet{Druganforthcoming} demonstrate the evolution of the approach to quality assurance in the European Commission, evidence the changing significance and definition of quality into fitness for purpose. Fitness for purpose emphasizes the scalability of quality (a concept which has its roots in the Skopos theory and its idea of degrees of translation adequacy, cf. \citealt{Nord2010}: 122) and allows institutions to prioritize certain aspects of quality, balancing political and legal risks with available resources. 

Yet, despite the growing number of publications on the topic of quality, there is still a dearth of empirical and narrowly-focused studies, discussing aspects of quality in a systematic way. This publication aspires to be a step forward towards filling in the niche.

\section{Structure of the book}

This book, which brings together eight contributions revolving around the central topic of (process/product) quality in institutional translation, is organized into three parts. The first part (Vandepitte, Biel) sets the conceptual and theoretical background for the study, identifying main components of quality in the institutional context. The next part studies selected aspects of quality assurance – quality managers (Prieto Ramos), style guides (Svoboda), terminology management (Stefaniak) and outsourcing evaluation (Strandvik). The last part contains empirical studies on two institutions – the Council and the Court of Justice (Hanzl and Beaven, Koźbiał). Contributions by practitioners (Stefaniak, Strandvik, Hanzl and Beaven) serve as a “reality check” for academic contributions, by describing quality procedures in major EU institutions (The European Commission, the Council and the Court). 

The authors and editors come from 7 institutions, of which there are five universities (Charles University, Prague, Ghent University, University of Geneva, Warsaw University, Jagiellonian University) and two EU institutions, i.e. the Translation Service of the General Secretariat of the Council of the EU and the European Commission’s Directorate-General for Translation.

\section{Overview of individual chapters}
\largerpage
The conceptual part opens with a chapter authored by Sonia \textbf{Vandepitte} from Ghent University (chapter title: “Translation product quality: A conceptual analysis”), who sets the ground for the ensuing discussions by reviewing the fundamental concepts related to quality. The chapter is adjusted to the actual (and broad) background of institutional translation, in which both the translation product and translation process have a role to play. Against this backdrop, Vandepitte deals extensively with the topic of translation quality assessment (TQA) with respect to the translation product. To this end, she employs the following parameters: the object, the purpose, and the criteria/quality levels of translation quality assessment (including their scaling and weighting), as well as the actors involved. She also raises the question (albeit as a one-off consideration) of cognitive processes implied in TQA – an aspect which most of TQA-related studies have neglected to consider so far. The chapter reflects in more detail on the actors involved in TQA, a vital feature to be tackled in the introductory chapter. She illustrates the use of parameters on the workings of a national institution (SCTA, the central translation service for German in Belgium). The chapter is both conceptual and empirical (SCTA survey) and, in its concluding part, is applicable in practice, too, thanks to a model where Vandepitte presents the above parameters as part of a coherent system. This makes the opening chapter a valuable asset in the bi-directional process of exchange between (Translation Studies) theory and (translation) practice.

In the second conceptual chapter, Łucja \textbf{Biel} from the University of Warsaw (chapter title: “Quality in institutional EU Translation: Parameters, policies and practices”) identifies key quality parameters of EU translation. Biel does so by analyzing and evaluating institutional policies as well as practices. Besides that, she compares and contrasts this view with the pertinent academic literature. The chapter deals with quality on two interrelated and overlapping planes: that of the textual level (where translations are viewed as products and are judged with the criteria of equivalence, consistency/continuity, on the one hand, and of textual fit and clarity on the other hand) and that of the process level (where translation is viewed as a service), which subsumes workflow management, human resources and tools. She observes that, recently, EU institutions have foregrounded quality aspects. This is particularly visible at the European Commission’s DGT, where the quality discourse has been reframed by linking translation quality (at the textual level) to genres and genre clusters, which has raised the visibility of the criterion of clarity. This shift is, most likely, effected by a managerial approach to assuring translation (product) quality and the concept of fit-for-purpose translations as part of what DGT refers to as Total Quality Management (TQM). 

The next part of the book opens with the contribution by Fernando \textbf{Prieto Ramos} from the University of Geneva, entitled “The evolving role of institutional translation service managers in quality assurance: Profiles and challenges”. Adopting a holistic approach to translation quality, this chapter foregrounds a neglected and under-researched component of quality assurance – namely, profiles of senior and mid-level translation service managers, that is translation service directors and heads of language units, who take fundamental decisions that affect the day-to-day management of translation units. Prieto Ramos surveys and contrasts the management structures at twelve intergovernmental and supranational organizations and studies their job descriptions in vacancy notices. The common ground in the scope of duties across the organizations is discussed around four groups: (1) strategic, administrative and financial matters; (2) staffing matters; (3) translation workflow coordination, and (4) contribution to translation, technical and quality control tasks. His study shows a reorientation from “one-fits-all quality control to a more modulated approach to quality variables”. The second part of the paper reports on structured interviews with service managers with a focus on quality assurance practices and challenges. The key interrelated challenges to quality are related to: (1) resource availability and productivity pressures due to cost-effective measures and budget limitations; (2) outsourcing procedures; and (3) workflow changes caused by technological developments, including new error patterns and new variables in the workflow. Prieto Ramos concludes with recommendations for an adequate balance between service managers’ translation expertise and managerial skills.

The next chapter by Tomáš \textbf{Svoboda} from Charles University, Prague, entitled “Translation manuals and style guides as quality assurance indicators: The case of the European Commission’s Directorate-General for Translation” asks to what extent the quality aspect of institutional translation is governed by rules, analyzing it through the prism of translation manuals and style guides. In his empirical quantitative study, Svoboda surveys 24 language pages of the DGT’s resource website, which is the largest resource of its kind, and contrasts the number and type of resources across language units, demonstrating shared areas and variation of language-specific resources up to 50\% of the links analyzed. The findings indicate that the structure and content of resources is largely standardized and harmonized, in particular as regards EU information – namely references to EU institutions, terminology resources and the Interinstitutional Style Guide. The highest variation was identified for language-specific resources, with significant differences between individual languages. The analysis of the content of link tags shows that resources are assigned a large variety of ‘labels’, ranging from names which strongly suggest the binding status of resources (e.g. decree, rules, instructions, requirements) to names which connote their less pressing nature (e.g. recommendations, tips, advice). In conclusions Svoboda comments on the complexity of institutional translation: “for their translations to be considered high quality, the translators (…) have to follow very many recommendations and instructions”.

Another related aspect of quality assurance – terminology management – is undertaken in the contribution by Karolina \textbf{Stefaniak}, a terminologist at the European Commission’s DGT (“Terminology work in the European Commission: Ensuring high-quality translation in a multilingual environment”). Stefaniak documents the daily work of a terminologist – a separate role assisting translators in terminological searches – on the example of the DGT’s Polish Language Department. The chapter explores the specificity of EU terminology, in particular, its supranational peculiarity, highly specialized or novel nature, occasional intended ambiguity, political sensitiveness and, last but not least, its systemic nature which requires terms to be internally consistent. Interestingly, Stefaniak reports that the majority (90\%) of translators’ queries deal with scientific terms rather than with legal terms which tend to be rare. The second part of the paper discusses criteria and techniques applied when solving terminological problems in the EU context. The author observes a strong preference for literal translation techniques, descriptive equivalents and neologisms. As for the quality criteria in the terminological decision-making process, they include accuracy, clarity and internal consistency of terminology, which often overrides other considerations. The standardization of terminology in translations is also achieved through terminological resources, including the IATE termbase, a major terminological achievement of EU institutions.

The next chapter by Ingemar \textbf{Strandvik}, a quality officer from the European Commission’s DGT (“Evaluation of outsourced translations. State of play in the European Commission's Directorate-General for Translation (DGT)”) addresses a novel and underresearched topic of evaluating outsourced translations, a trend gaining recently in importance in EU institutions due to budgetary constraints and limited human resources. Strandvik shares his insider knowledge of evaluation practices in the DGT, including assessment tools and the evaluation grid. The chapter draws attention to many outsourcing challenges, such as (1) the need to ensure the consistency of evaluation practice among 1600 in-house translators; (2) differences in the size of translation markets in various Member States; (3) time allocated for revision and evaluation; and (4) risks involved in mistranslation. These and other factors have contributed to the evolution of the reference model for translation quality management, and to a move from the fidelity to fitness-for-purpose approach to quality. Strandvik raises an important point of missing empirical evidence as to the correlation between sample sizes and assessment reliability. The chapter ends with a pertinent discussion on recent developments and further challenges related to translation evaluation and ensuring a translation quality policy at the interinstitutional level.

In their case study entitled “Quality assurance at the Council of the EU’s Translation Service”, Jan \textbf{Hanzl} and John \textbf{Beaven} from the Council’s General Secretariat offer an insider view on quality practices and policies within the Council, an institution which is far less outspoken about its quality policies compared to the European Commission. The authors discuss the specificity of translation work at the Council related in particular to the fact that texts are subject to numerous discussions and amendments until their content is supported by the Member States. Thus, translators rarely translate from scratch but work on interim and working texts at various stages of their amendment (“versions drawn up in a hurry by non-native English speakers, not final, well-edited and fine-tuned texts”), often against tight deadlines. Similarly to Stefaniak, the authors emphasize the importance of continuity and consistency at the terminological and phraseological level, both within a document and across related documents. Working in such a specific environment, the Council has adopted a pragmatic approach to quality based on the fit for purpose principle correlated with revision levels adjusted to the political visibility and legal/financial impact of text types in order to ensure “an optimal level of useful quality”. Quality requirements are divided into three sets of criteria: (1) linguistic aspects, including accuracy, clarity and fluency; (2) technical aspects, including the parallel pagination of language versions for practical reasons; and (3) timeliness to ensure the smooth operation of the Council. The authors conclude with the description of the Council’s ex-post quality monitoring system designed in 2009 to regularly and systematically evaluate translation samples.

Last but not least, the final case study addresses quality at the Court of Justice of the European Union (CJEU). Dariusz \textbf{Koźbiał} from the University of Warsaw focuses on key aspects underlying the Quality Assurance strategy in this multilingual and supranational judiciary institution. In the introductory section of the chapter “A two-tiered approach to quality assurance in legal translation at the Court of Justice of the European Union”, Koźbiał discusses the complexity of current language arrangements in the CJEU and the specificity of translations (their predominant legal nature) in this institution, drawing attention to the fact that “the goal of translation at the CJEU is to produce parallel texts that will allow uniform interpretation and application by national courts”. The main part of the chapter proposes a two-tiered approach to translation quality at the CJEU, which can be conceptualized at two interrelated levels, i.e. human resources and workflow. The former level comprises in-house lawyer-linguists, external contractors, revisers, auxiliary staff and project managers, whereas the latter consists of measures related to the translation process as well as intra- and interinstitutional co-operation.

\section{Conclusions}

This volume aims at contributing to the deeper understanding of institutional translation, mainly, but not exclusively in the domain of EU translation. By presenting a blend of conceptual and empirical studies, this collective monograph intends to offer an extension to research available so far, which is still far from being saturated. As Schäffner et al. put it, “[t]here is widespread agreement among researchers […] that institutional translation is still rather unexplored and that empirical studies are missing” (\citeyear{SchäffnerEtAl2014}: 494); similar remarks may also be observed in some chapters by the practitioners who contributed to this book. Likewise, the proposed reconciliation of both the academic and the professional views is suggested as a continuation of a dialogue, which has the potential of enriching and cross-fertilizing both areas. The discipline of Translation Studies is a witness to a bi-directional movement of academic reflection informing practical decisions of professionals on the one hand, and, on the other, observations from practice providing solid grounds and data for academic research.

 
\sloppy
\printbibliography[heading=subbibliography,notkeyword=this] 
\end{document}