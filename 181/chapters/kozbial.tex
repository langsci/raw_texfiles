\documentclass[output=paper]{langsci/langscibook} 
\ChapterDOI{10.5281/zenodo.1048198} %will be filled in at production

\author{Dariusz Koźbiał\affiliation{University of Warsaw}}
\title{Two-tiered approach to quality assurance in legal translation  at the Court of Justice of the European Union}
\shorttitlerunninghead{Two-tiered approach to quality assurance in legal translation at the CJEU}
\abstract{The objective of this chapter is to identify the key aspects of Quality Assurance (QA) affecting the quality of translations at the Court of Justice of the European Union (CJEU). The chapter starts with a brief clarification of the terms connected with QA, which are quite often used interchangeably and imprecisely. The next two sections set the background for the analysis by exploring the current language arrangements at the CJEU and associated challenges, and by discussing two standards that are relevant to the field of legal translation, namely EN 15038:2006 \nocite{EN2006} and \citeauthor{ISO2015}. The main part of the chapter proposes a two-tiered approach to translation quality at the CJEU. It is argued that it can be conceptualized at two interrelated levels, namely the human resources level and workflow level. While the human resources level comprises, inter alia, in-house lawyer-linguists, external contractors, revisers, auxiliary staff and project managers, the workflow level consists of measures aimed at achieving proper structurization of the translation process as well as intra- and interinstitutional co-operation.
}
\maketitle

\begin{document}

\section{Introduction}\label{sec:kozbial:1}

This chapter is aimed at identifying and evaluating the key quality aspects underlying the Quality Assurance strategy applied in the process of the translation of legal documents at the Court of Justice of the \isi{European Union} (\isi{CJEU}). However, owing to the fact that the \isi{terminology} used in reference to Quality Management (QM) is still unclear \citep[168]{Lušicky2017}, a crucial distinction has to be drawn between a number of mutually related terms, such as Quality Assurance (QA), Translation Quality Assessment (\isi{TQA}) and Quality Control (QC), which are most widely used in discussions on quality in translation, as they are the key elements in QM systems \citep[168]{Lušicky2017}.

The chapter does not consider aspects of \isi{TQA} (also referred to as quality \isi{evaluation}), which concerns itself with measuring and assessing the quality of an end product of the \isi{translation process} \citep[76]{Drugan2013}. In contrast, QA takes a broader look at the holistic process of translation and associated resources. As explained by \citet[129]{Mossop2014}:


\begin{quote}
Quality assurance is the full set of procedures applied not just after (as with \isi{quality assessment}) but also before and during the translation production process, by all members of a translating organization, to ensure that quality objectives important to clients are met.
\end{quote}


Bearing the above explanation in mind, \isi{TQA} could be regarded as strictly complementary to the general QA strategy adopted by a given institution or organization. Thus, QA refers to an all-encompassing system which aims at preventing quality-connected problems from occurring in the first place and is considered a \textit{global} approach to \isi{translation quality} at any stage of the \isi{translation process} (\citealt[76]{Drugan2013}). Another definition of QA is provided by \citet[342]{Popiołek2015}:

\begin{quote}
QA (Quality Assurance) is a model approach that ensures good results if the right combination of human and technical resources is used in a sequence of steps and tasks that constitute a process within a system.
\end{quote}


\citeauthor{Popiołek2015}’s definition of QA emphasizes the central role of \isi{human resources} as well as technical resources in the \isi{translation process} viewed as a whole, thus emphasizing the viewpoint that any consistent approach to QA cannot be piecemeal.

On the other hand, the goal of QC is to verify whether the \isi{translation product} or service meet stated quality specifications (\citealt[169]{Lušicky2017}, in \citealt{Lommel2015} (ed.)). Therefore, \isi{TQA} and QC enable verification of compliance with the planning and preventive measures set out in the general QA strategy \citep[169]{Lušicky2017}.

For the purposes of this chapter, the QA strategy applied by the \isi{translation service} (\isi{TS}) of the \isi{CJEU} is considered to rest on two key pillars, which enable the Court to communicate both internally and externally, namely \isi{human resources} and workflow processes. Such a division will enable a thorough and critical \isi{evaluation} of the Court’s approach to QA with regard to the process of translation of its documents as well as the relation of the \citeauthor{ISO2015}  standard, which is currently the most relevant standard for QA in \isi{legal translation}, to the said approach.

\section{Language arrangements within the institution}\label{sec:kozbial:2}

Although the Court of Justice of the \isi{European Union} (\isi{CJEU}) is an \isi{EU} institution which could be likened to a supreme or constitutional court of a Member State (\citealt[10]{Roper2011}; \citealt{Itzcovich2014}), there are several crucial differences between the \isi{CJEU} and Member States’ supreme and constitutional courts. These differences consist not only in the respective courts’ competences, as the \isi{CJEU}’s role is to settle issues connected primarily to \isi{EU} law,\footnote{As stated on the \isi{CJEU}’s website (\url{https://curia.europa.eu/jcms/jcms/Jo2_6999/en/} accessed 2017-05-02), the Court’s mission consists in ensuring that the Treaties are interpreted and applied according to \isi{EU} law. The \isi{CJEU}, inter alia, reviews the lawfulness of the acts of \isi{EU} institutions, ensures that the Member States comply with obligations resulting from the Treaties, and interprets \isi{EU} law at the request of the national courts and tribunals. Its competences include actions in areas such as competition, human rights, administrative, and constitutional law. The \isi{CJEU} does not have criminal jurisdiction.} but also in the adapted language systems, which are predominantly monolingual in the case of Member States and multilingual in the case of the \isi{EU} court. Despite this fact, the \isi{CJEU} has to operate in a way that allows for full adoption of the \isi{multilingualism} principle within its institutional setting and guarantees access to its \isi{case law}, which constitutes a source of law in the Member States via a binding precedent (\citealt[86]{Łachacz2013},  \citealt[626–628]{Arnull2006}, \citealt[221–232]{Sulikowski2005}).

The current language arrangements at the Court make it a truly multilingual institution which has no counterpart in any other court, mainly due to the fact that, in direct actions, each of the 24 \isi{official languages} of the \isi{European Union} can be the language of a case brought before the Court, i.e. the language in which the proceedings will be conducted.\footnote{Court of Justice – Presentation \url{https://curia.europa.eu/jcms/jcms/Jo2_7024/en/}  (accessed 2017-05-02.)} The \isi{CJEU}’s obligation to observe the principle of linguistic diversity arises from \textit{inter alia} Regulation No 1 of the Council,\footnote{Council Regulation No 1 of 15 April 1958 determining the languages to be used by the European Economic Community, Official Journal 017, 06/10/1958 P. 0385-0386} Article 3(3) of the Treaty on the \isi{EU},\footnote{Official Journal C 326, 26/10/2012 P. 0001–0390} dated 15 April 1958, under which the number of \isi{official languages} has gradually increased as new Member States have joined the Community, and Article 22 of the \isi{Charter of Fundamental Rights} of the \isi{EU},\footnote{\isi{Charter of Fundamental Rights} of the \isi{European Union} OJ C 326, 26.10.2012, p. 391–407} which calls upon European institutions to respect linguistic diversity. It is also reflected in the \isi{CJEU}’s Statute (Article 64),\footnote{Consolidated version of Protocol (No 3) on the Statute of the Court of Justice of the \isi{European Union} \url{https://curia.europa.eu/jcms/upload/docs/application/pdf/2016-08/tra-doc-en-div-c-0000-2016-201606984-05_00.pdf} (accessed 2017-05-02).} in the \isi{Rules of Procedure} of the Court of Justice\footnote{\isi{Rules of Procedure} of the Court of Justice of 25 September 2012 (OJ L 265, 29.9.2012), as amended on 18 June 2013 (OJ L 173, 26.6.2013, p. 65) and on 19 July 2016 (OJ L 217, 12.8.2016, p. 69) \url{https://curia.europa.eu/jcms/upload/docs/application/pdf/2012-10/rp_en.pdf} (accessed 2017-05-02.)} and the General Court\footnote{Rules of procedure of the General Court OJ L 105, 23.4.2015, p. 1–66 \url{http://eur-lex.europa.eu/legal-content/EN/TXT/?uri=uriserv:OJ.L_.2015.105.01.0001.01.ENG& toc=OJ:L:2015:105:TOC} (accessed 2017-05-02)} (\textit{language of the case}; Articles 36–42 and 44–49, respectively). The \isi{Rules of Procedure} of the \isi{CJEU} allow for the use of any one of all official \isi{EU} languages as the language of the case. What needs to be emphasized is that the language of the case automatically becomes \textit{the} \textit{authentic language} of the documents, unless another language has been designated. The \isi{Rules of Procedure} do not govern the use of languages within the administrative, internal activity of the Court. In order to guarantee equal access to justice for all citizens, it is essential for the parties to proceedings before the Court to be able to use their own language. Therefore, the \isi{CJEU} has to communicate with the parties in the language of the proceedings and with the wider public using the \isi{EU}’s \isi{official languages}, so that its \isi{case law} is easily available to all \isi{EU} citizens.

At the time of writing, the \isi{European Union} (still) has 28 Member States and 24 \isi{official languages}. Accordingly, upholding the principle of \isi{multilingualism} requires that \isi{EU} \isi{case law} be published in all 24 \isi{official languages}. However, as opposed to legislation, not all language versions of judgments are equally authoritative (cf. \citealt[69]{Kjær2007}). Despite this fact, it is undisputed that \isi{legal translation} plays a significant role in the functioning of \isi{EU} institutions, especially within the \isi{CJEU}’s setting. Due to the fact that for historical reasons the \isi{CJEU}’s internal working language has been French (\citealt[487]{McAuliffe2013a}, \citeyear[865]{McAuliffe2013b}), all procedural documents, pleadings and judgments need to be translated into this language. Since the creation of the \isi{CJEU} in 1952, the linguistic situation at the Court has become more complex with each successive accession and the addition of new \isi{official languages}, thus further increasing the total number of potential language combinations up to 552 (Annual Report, 2017).\footnote{For comparison, the maximum number of language combinations in 1952 amounted to 12 language combinations.} This, however, does not mean that all documents need to be translated into all of the 24 \isi{official languages} of the \isi{EU} (cf. \citealt{McAuliffe2012}, \citealt[250]{Künnecke2013}). A case before the Court may be examined in a single language, i.e. in the language of the case (applicant’s language), unless any Member State intervenes, thus creating the need for translations into that Member State’s official or designated language.

The legal \isi{translation service} of the Court has to deal with ever-growing volumes of work. According to the data provided in the annual report for the year 2016 issued by the \isi{CJEU} \citep{ar2017}, the institution’s Translation Service produced approximately 1,160,000 pages in the year under review. It needs to be pointed out that if it had not been for the introduction of internal economy measures aimed at reducing the amount of work, the total number of translated pages in 2016 would have reached 1,600,000.\footnote{Court of Justice of the \isi{European Union} – Annual Report 2016 \url{https://curia.europa.eu/jcms/upload/docs/application/pdf/2017-04/ragp-2016_final_en_web.pdf} (accessed 2017-05-02).} If one compares this number of translated pages with the output of other \isi{EU} institutions (e.g. in 2015, the \isi{European Commission}’s \isi{Directorate-General for Translation} had an output of almost 2 million pages;\footnote{2015 Annual Activity Report \isi{Directorate-General for Translation}. 2016. (https://ec.europa.eu/info/sites/info/files/activity-report-2015-dg-t\_april2016\_en.pdf, accessed 2017-05-02).} however, it can be assumed that a lesser portion of this amount constitutes \isi{legal translation} considering the types of documents translated at the \isi{CJEU} – cf. \citealt{McAuliffe2012}), one can easily notice that \isi{legal translation} is of paramount importance to the proper functioning of the Court. Uniform interpretation and application are perceived as critical determinants of quality when it comes to the translation of legal acts \citep[73]{Šarčević1997}; however, it can be assumed that they are equally essential when it comes to the translation of \isi{case law} and all types of procedural documents. Since the \isi{CJEU} essentially seeks to persuade its audience, i.e. national legal communities (judges, lawyers, academics, etc.), in favor of its understanding of \isi{EU} law, it needs to rely on translation so that its message can actually get across (\citealt[85]{Łachacz2013}, in \citealt[296]{Paunio2007}). As observed by \citet[9]{McAuliffe2014}, the goal of translation at the \isi{CJEU} is to produce parallel texts that will allow \isi{uniform interpretation} and application by national courts; or in other words, that they will have the same effect in all Member States. Of course, one could assume that quality does not always go hand in hand with quantity (especially with such a high output as provided above); however, in order to ensure the smooth functioning of the Court, its \isi{TS} not only needs to produce numerous translations within tight deadlines, but also needs to ensure that these translations are of high quality, which is made even more difficult by the complexity of the \isi{EU} linguistic and legal context \citep[69]{Kjær2007}, thus necessitating the proper selection of staff working at the Court’s \isi{TS}.

There are several main reasons for the complexity of translation work done at the \isi{CJEU}. \citet[70]{Kjær2007} observes that difficulties connected with \isi{EU} \isi{legal translation} stem from the interplay of the legal systems of individual Member States and the fact that \isi{EU} law (and thus also \isi{case law}) does not constitute an established system, which is in fact still in fluctuation. In order to effectively and correctly translate texts which are originally created in such a complex environment, it is crucial for the legal translator to possess in-depth knowledge of the participant legal systems, legal languages as well as be able to compare them \citep[71]{Kjær2007}. \citeauthor{Kjær2007} seems to have proposed a quite adequate term for the type of \isi{legal translation} produced in the \isi{EU} context, namely \textit{supranational translation} \citep[76]{Kjær2007}. This term conveys the fact that translation in the \isi{European Union} concerns both translation within and between legal systems, because the \isi{EU} legal system is still “under construction”. The proposed term refers not only to the translation of legislation, but also to the translation of, inter alia, judgments of the \isi{CJEU} and requests for preliminary rulings directed to the Court by national courts \citep[77]{Kjær2007}. On top of that, both legislation and \isi{case law} of the \isi{CJEU} carry legal effects, which could be extremely harmful as a result of translations of bad quality – construction of \isi{case law} (and therefore its translations) may affect the application of that law by national courts \citep[492]{McAuliffe2013a}. With this in mind, it is clear that only highly specialized translators can be responsible for the type of translation described above. As \citeauthor{Biel2011a} points out (\citeyear[25]{Biel2011a}), translations of legal texts need to be both “accurate and beautiful”. She stresses the fact that translators transferring information conveyed in legal documents have to bear in mind both the equivalence relation, i.e. the relation between the source text and target text, as well as the relation of \isi{textual fit}, i.e. the relation between the translated language and the naturally occurring non-translated language of a similar \isi{genre}. The former is of vital importance, as it involves \isi{accuracy} of the information transfer and use of correct \isi{terminology}, whereas the latter concerns the naturalness of translations \citep{Biel2011a}. Another problematic issue for legal translators concerns the standardization of legal \isi{terminology}, which is difficult to achieve in a multilingual environment, where legal \isi{terminology} expressed in 24 \isi{official languages} is rooted in 28 national legal systems \citep[75, 79]{Biel2011a}. Terminological problems are usually posed by incongruent levels of equivalence between legal concepts in the source and target legal systems \citep[16]{PrietoRamos2011}. The difference between the meaning of \isi{EU} and national \isi{terminology} has been stated by the Court of Justice in the CILFIT case:\footnote{Case C-283/81 – Judgment of the Court of 6 October 1982. - Srl CILFIT and Lanificio di Gavardo SpA v Ministry of Health [1982] ECR 3415, paragraph 19, \url{http://eur-lex.europa.eu/legal-content/EN/TXT/?uri=CELEX\%3A61981CJ0283} (accessed 2017-05-02).}

\begin{quote}
It must be borne in mind […] that Community law uses \isi{terminology} which is peculiar to it and that legal concepts do not necessarily have the same meaning in Community law and in the law of the various Member States.
\end{quote}


One type of solution to this problem are terminological databases, which are already being used at the Court. Terminological databases constitute a QA tool, because they allow translators to use appropriate \isi{terminology} in a consistent manner within and across texts \citep[166]{Lušicky2017}. Legal \isi{terminology}, as a special feature of legal discourses, is seen as a central component of \isi{legal translation} theory and practice \citep[15]{PrietoRamos2015}.

Another problematic issue concerns the sheer number of \isi{official languages} currently in use in the \isi{European Union} and its institutions. The accession of ten new states in the 2004 enlargement did not only entail the introduction of new legal traditions, thus necessitating a need for proper adjustment of the way of functioning of the Court, it also meant significant linguistic challenges \citep[812]{McAuliffe2008}. The accession of new states and the addition of new languages spurred the Court on to amend its \isi{Rules of Procedure} and reduce the number of pages published (and therefore translated) in the European Court Reports \citep{McAuliffe2008}. Another change which resulted from the 2004 enlargement pertained to how Advocates General drafted their opinions – before 2004 they used to draft them in their mother tongue, but after 2004 some of the Advocates General started to draft opinions in the Court’s pivot languages, which, in turn, influenced the style of the opinions, causing them to be more synthetic in nature (\citealt[816]{McAuliffe2008}; \citealt[254]{McAuliffe2010}, \citeyear[9]{McAuliffe2012}). The Court itself deliberates using the French language. For this reason, all procedural documents must be translated into French.

\section{EN 15038:2006 and ISO 17100:2015 standards}\label{sec:kozbial:3}

\largerpage[-1] % long distance
Since attention to \isi{translation quality} accelerated in the 1990s \citep[15]{PrietoRamos2015}, it translated into the willingness to establish standards encompassing the whole industry, which have become essential for assuring quality by means of systematic QM \citep[170]{Lušicky2017}. It is worth taking a look at what the European Committee for \isi{Standardization} (CEN) has worked out with regard to standardizing the approach to the quality of translations. In 2006, the CEN issued the EN 15038 standard entitled “Translation Services – Service Requirements” \citep[131]{Mossop2014}. It was the first pan-European standard regulating the quality of translation services \citep[16]{Biel2011a}. It is worth noting that EN 15038 used to define quality rather indirectly, in its statement about the task of the reviser: “the reviser shall examine the translation for its suitability for purpose”. The wording was unclear, for example, as to whether revision was required to include a comparative re-reading. However, the document did specify the requirement to revise every translation by a second translator \citep{Biel2011b}.

\largerpage[-2]
On May 1, 2015, the \isi{ISO} (International Organization for \isi{Standardization}) issued the \citeauthor{ISO2015} standard under the title “Translation services – Requirements for translation services”, which extended the scope of and superseded the EN-15038 standard. The structure of \citeauthor{ISO2015} has changed compared to EN 15038 and focuses more heavily on conventional translation processes. However, it still does not point out the exact qualities of a high-\isi{quality translation}, as was also the case with the previous standard (cf. \citealt{Biel2011a}: 18). The obligatory revision performed by a second person also remains key in the current standard. Performing a review remains optional. The \isi{translation service provider} has to ensure that a final verification of the translation project is performed before it is delivered to the client. Besides the actual standard, there are also attachments which explain certain aspects of the standard by means of examples or graphical cues to help visualize the processes. The \citeauthor{ISO2015} standard lists requirements for the core processes, resources and other aspects necessary for the provision of a quality \isi{translation service} that meets applicable specifications, and therefore (same as the previous standard) is also perceived as a compendium of what should be done in order to contribute to Quality Assurance in translation, assuming that if the QA measures are in place, the end product of translation will be of good quality (cf. \citealt[271]{Gouadec2010}). The use of raw output from \isi{machine translation} plus post-editing is outside the scope of \citeauthor{ISO2015}; it also does not apply to interpreting services. It does not define \textit{quality} per se; however, it does explain the meaning of the main concepts related to translation and translation services, \isi{translation workflow} and technology, language and content, \isi{human resources} involved in the provision of translation services, and control of the process of delivering a \isi{translation service} (\citeauthor{ISO2015}:2015). The \citeauthor{ISO2015}:2015 standard also lays out guidelines concerning human (translators, revisers, reviewers, proofreaders and project managers) and technical and technological resources, pre-production processes and activities, the production process and post-production processes. Currently, the \citeauthor{ISO2015}:2015 standard is the most relevant standard applicable to QA in \isi{legal translation},\footnote{It needs to be pointed out that there is currently another international standard being developed under the working name \isi{ISO} 20771 “Legal and other specialist translation services”. When completed and if adopted, it is supposed to provide the minimum requirements for the qualifications, competence, core processes, resources, training and other aspects necessary for the provision of \textit{legal or other specialist translation services} of quality that meet applicable specifications \citep{Popiołek2016}. It is expected to define the competences and qualifications of legal and other specialist translators, revisers and reviewers in the context of the process applied in legal and specialist translation and it will also address the specific professional and QA challenges in the area of \isi{legal translation} \citep{Popiołek2016}. Similarly to \isi{17100}:2015, \isi{ISO} 20771 will deal with concepts related to translation and translation services, \isi{translation workflow} and technology, language and content, the people (\isi{human resources}) involved in translation services, and the concepts related to the control of the \isi{translation service} process \citep{Popiołek2016}.} therefore, it will be referred to when discussing the key aspects underlying QA in the \isi{CJEU}’s \isi{TS}’s work.

As noted by \citet[1]{Drugan2013}, the establishment of objective criteria with regard to quality has always been a subject of general disagreement in Translation Studies, but it was indeed successful and led TSPs to apply standards in their work. It has to be noted, however, that some accidents may happen even in the most “quality-assured” environments \citep[271]{Gouadec2010}, as some methodological problems may continue to appear even in institutional contexts in which QA measures have already been implemented \citep[12]{PrietoRamos2015}. What is more, the introduction and proper application of QA measures may be quite costly \citep[272]{PrietoRamos2015}. Nevertheless, QA is presumed to contribute to the production of translations characterized by higher quality. This chapter aims at discussing various aspects of QA practices in the work of the \isi{CJEU}’s \isi{translation service}. It does not by any means attempt to be exhaustive and present a \isi{holistic approach} to QA in the \isi{Directorate-General for Translation}.

 
\section{Two-tiered approach to Quality Assurance}\label{sec:kozbial:4}

In this chapter, it is argued that the general Quality Assurance policy in the Court’s \isi{Directorate-General for Translation} is based on two key pillars. This section proposes to conceptualize the notion of \isi{translation quality} at the \isi{CJEU} through two key pillars, namely \isi{human resources} and well-structured workflow processes. While the \isi{human resources} level comprises, inter alia, in-house lawyer-linguists, \isi{external contractors}, auxiliary staff and project managers, the workflow level consists of measures aimed at achieving proper structurization of the \isi{translation process} as well as intra- and interinstitutional co-operation.


\subsection{Quality Assurance at the human resources level}\label{sec:kozbial:4.1}

Human resources constitute the first key pillar of QA in the Court’s \isi{TS} work, which includes translators, revisers, auxiliary staff, and project managers, etc.

Translators’ work done in the institutional setting of a European organization is of paramount importance for the institution itself, since without translations done in a timely manner it would be able to communicate neither internally, nor externally. Therefore, translators are in a way representatives of institutions for which they work. As \citet[24]{Koskinen2008} rightly pointed out with regard to her work done at the \isi{European Commission}’s Translation Service:

\begin{quote}
The translated text is not mine, nor does it have my name on it: it belongs to the institution, and it bears the name of the institution on it. It is not my trustworthiness but the trustworthiness of the translating institution that will be maintained, enhanced or harmed by my translation. In the Commission, my words are not mine; \textbf{I am a spokesperson for the institution} [emphasis added].
\end{quote}

The same applies to the work of translators working at the \isi{CJEU}, for whom translation is not an individual, personal act, but a part of a collective process thanks to which the institution is able to communicate both with the outside world as well as within itself as a consequence of its obligation to observe the principle of \isi{multilingualism}.

The work of the legal \isi{translation service} of the \isi{CJEU} is very complex and demanding due to the highly specialized nature of its tasks. The main reasons for this are the complicated language arrangements at the Court. Since this section focuses specifically on the \isi{CJEU}’s legal \isi{translation service}, it examines the work of the Court’s translators, that is lawyer-linguists, who are employed in this institution as opposed to lawyer-linguists employed in other institutions of the \isi{EU}.\footnote{Apart from the Court of Justice of the \isi{EU} lawyer-linguists work for the \isi{European Commission}, the Council of the \isi{European Union}, the European Parliament and the European Central Bank.} Lawyer-linguists’ work differs from the work of translators working for most of the European institutions, who possess, in most of the cases, a degree in translation but not a law degree as in the case of lawyer-linguists; lawyer-linguists’ work consists mainly of \isi{legal translation} exclusively into their mother tongue \citep[15]{McAuliffe2016} and the legal-linguistic revision of court documents, such as: applications or references for a preliminary ruling, written observations, reports for  hearings, Advocate Generals’ opinions and judgments of the Court.

Since the \isi{CJEU}’s decisions need to be properly understood by courts in the national context of Member States, it seems reasonable that the task of translation has been devolved to national lawyers who are in the best position to act as mediators in the Court’s communication with national courts \citep[85–86]{Łachacz2013}. This is thanks to their belonging to the same community as the target audience and their ability to interpret and translate the Court’s messages in a way that allows them to retain their intended persuasive value \citep{Łachacz2013}. For this reason alone, translations into all of the \isi{official languages} need to be of the highest possible quality.

Owing to the nature of documents translated at the \isi{CJEU} and the constant interplay between law and \isi{legal translation}, it is crucial that lawyer-linguists have legal thematic competence, which constitutes a distinctive feature of \isi{legal translation} competence \citep[11]{PrietoRamos2011}. Article 42 of the \isi{Rules of Procedure} of the Court of Justice specifies who is to form its \isi{translation service}:
\begin{quote}
The Court shall set up a language service staffed by experts with adequate legal training and a thorough knowledge of several \isi{official languages} of the \isi{European Union}.
\end{quote}

Those who want to work as either in-house lawyer-linguists or \isi{external contractors} need to meet strict requirements – essentially they need to have a law degree from their home country and have perfect command of at least two \isi{official languages} apart from their mother tongue \citep[10]{McAuliffe2016}. Lawyer-linguists should also be adept in the exercise of comparative law and be able to draft legal texts in their own languages. Since most of the documents translated at the Court are part of judicial proceedings, they carry specific legal effects. Any sub-par quality of translated documents might, for instance, mislead national courts and institutions or potentially cause delays in the proceedings \citep[9]{Izzo2014a}. Therefore, translations of appropriate quality need to not only convey the message of the original text, but also contain the right \isi{terminology} (which might be specific to either the \isi{EU} legal system or national legal systems), be free of grammatical and language errors as well as be written in the appropriate legal style. Lawyer-linguists should also be able to critically analyze translated documents both from a legal and linguistic perspective; they should always be on the lookout for any inconsistencies in the original texts in order to point out flaws to the authors of translated documents. Apart from the knowledge of \isi{EU} and national legal systems and languages, lawyer-linguists must also possess interpersonal skills, intercultural competence and high ethical standards.

The entity responsible for the \isi{recruitment} of lawyer-linguists\footnote{\isi{DGT} employs around 985 persons, of whom 613 are lawyer-linguists working in 23 language divisions. Thus, DGT staff represent 45 per cent of the whole staff of the Court (ca. 2,168). There is no separate Irish language division, as this forms a part of the English language division. (Information valid as of December 31, 2016. Source: \url{https://curia.europa.eu/jcms/jcms/P_80908/en/};
\url{https://curia.europa.eu/jcms/upload/docs/application/pdf/2017-04/ragp-2016_final_en_web.pdf},
accessed 2017-05-02.)} (as well as computer specialists, secretarial assistants, etc.) is the European Personnel Selection Office (EPSO). Therefore, in the light of the \citeauthor{ISO2015} standard, the \isi{CJEU}’s legal \isi{translation service} as the \isi{translation service provider} is not burdened with the task of selecting the people who are to perform translation tasks as lawyer-linguists. Most of the lawyer-linguists employed in all language units are Permanent Officials who belong to appropriate AD function groups based on their work experience.\footnote{\url{http://europa.eu/epso/doc/staff_cat_graph.pdf} (accessed 2017-05-02.)} After the successful conclusion of the EPSO procedure, selected candidates begin their probationary period in which they are trained in-house and work under the supervision of a senior lawyer-linguist. Tutorship allows new lawyer-linguists to learn the working methods of their respective language units and practical issues connected with the translation of court documents. After the probationary period ends with a positive result, the lawyer-linguist becomes a full-grown \isi{EU} official, whose duty is to continue to further hone their skills with regard to translation, \isi{terminology} and comparative law. As a means of ensuring quality with regard to lawyer-linguists’ output, their work is subject to yearly \isi{evaluation}.

In order to cope with the increasing workload, the Court uses temporary or contract staff to perform auxiliary tasks. Temporary or contract staff cannot be in active employment as officials or other servants of the \isi{European Union} when carrying out the specific work assignments described in framework contracts \citep[16]{FCPTS2017}. Contract staff working on outsourced translations also need to meet strict \isi{quality requirements}, since their translations of legal documents have to be of high quality allowing for immediate publication or any other application \citep[16]{FCPTS2017}. As stated in the framework contract for the provision of translation services available at the Court’s website, contract staff are required to ensure \citep[11]{FCPTS2017}:

\begin{enumerate}
\item 
compliance with specific instructions given by the Court;
\item 
correct, rigorous and precise use of the target language;
\item 
rigorous use of the appropriate legal language and \isi{terminology} of the target language;
\item 
strict use of the legal \isi{terminology} used in the reference documents (source and target languages);
\item 
rigorous citation of the relevant legislative and/or judicial texts;
\item 
use of the necessary legal databases (of the \isi{European Union} and national);
\item 
compliance with the Vade-Mecum of the Court (if appropriate);
\item 
delivery within the period agreed and specified in the order form. 
\end{enumerate}

Contractors’ work is subject to QC and failure to meet deadlines or the inadequate quality of completed assignments may lead to penalties described in the contract provisions \citep[11]{FCPTS2017}. If the contractor is unable to carry out the work assigned to them within the prescribed period, the contractor may be required to pay to the Court a penalty of 10 \% of the total amount invoiced per calendar day of delay by means of a deduction from the payment to be made to the contractor (Articles 5.5.1–5.5.2 of the framework contract). As a result of inadequate quality of translations, i.e. translations not compliant with the requirements stated in Article 5.6 of the framework contract, established by quality controllers, the contractor’s remuneration for the translation in question may be suspended and subject to further assessment. A definite confirmation of inadequacy in terms of the quality of a specific work assignment may lead to the Court’s refusal to pay in full or in part for that assignment.

Moreover, contractors are obliged to protect the confidentiality of all information communicated to them in the course of the performance of contracts \citep[16]{FCPTS2017}. External contractors receive appropriate support and assistance from the respective language units (access to glossaries, terminological databases, ability to participate in occasional workshops organized in Member States, where the external collaborators live, etc.). Strict requirements imposed on \isi{external contractors} seem to be in line with the guideline set out in the \citeauthor{ISO2015} standard, which concerns full responsibility of the \isi{TSP}, that is the \isi{CJEU}’s \isi{TS}, for sub-contracted work.


\subsection{Quality Assurance at the workflow level}\label{sec:kozbial:4.2}

Internal and external communication on the part of the Court is carried out by the \isi{Directorate-General for Translation}, which provides high-quality translations of different kinds of court documents (pleadings, opinions, judgments, orders, etc.) within tight deadlines. In 2015 alone, the total output of all language units equaled to 1,113,427 translated pages, whereas there was a total of 1,114,838 incoming pages (+1.4\% as compared to the previous year). Although it has been noted that the process of translation extends the time in which the Court needs to pass a decision and also creates a substantial burden of financial costs, access to the \isi{CJEU}’s \isi{case law} by \isi{EU} citizens and Member States’ authorities remains essential \citep[23]{Roper2011}. 

The \isi{CJEU}’s \isi{translation service} consists of 23 language units organized into two directorates, namely Directorate A (CS, ET, ES, FR, HU, LT, MT, NL, PL, RO, SK) and Directorate B (BG, DA, DE, EL, EN, FI, HR, IT, LV, PT, SL, SV), which are shared between the Court of Justice and the General Court. As a result of the 2004 enlargement, which added nine new \isi{official languages}, the Directorate General grew substantially in size. This required the introduction of a measure allowing the \isi{TS} to guarantee coverage of all \isi{official languages} – it turned out to be the pivot translation system \citep[184]{Šarčević2013}, the introduction of which was planned ahead of the 2004 enlargement.\footnote{The preparations began in 2011 when a new director took charge of the \isi{translation service}.} The system itself consisted in using several “bigger” languages, that is English, French, German, Italian and Spanish, to produce translations which were then used to translate into the “smaller” languages of new Member States \citep[810–816]{McAuliffe2008}.

All units are required to comply with rules and standards aimed at achieving harmonization. Each language unit at the \isi{Directorate-General for Translation} achieves this thanks to its own management personnel, composed of the Head of a Unit, a Quality Controller, a Resources Manager and a Head of Local Coordination. Proper management is especially important due to the existence of a large number of \isi{official languages} and tight deadlines for translating court documents. The post of a Quality Policy Coordinator working at the level of the \isi{Directorate-General for Translation} allows for ensuring the harmonization of quality policies among individual language units.

The Quality Policy Coordinator organizes sessions during which Quality Controllers from all language units review random samples submitted by individual language units. It is common for all units to submit the same range of pages from a given document, which allows for the harmonization of criteria across all 24 languages. Such sessions enable Quality Controllers to discuss problematic areas among themselves, issue corrigenda, \isi{terminology} notes to language units and to clarify \isi{quality guidelines}.

The process of translating court documents involves three main stages, i.e. translation, revision and proofreading \citep{Izzo2014b}. Before the actual translation work is begun, a special unit composed of assistants identifies already translated parts of the text or those parts which are similar to previous documents (frequently occurring phrases, quotes from legislation or \isi{case law}, etc.). After this preliminary work is done, the findings are then made available to lawyer-linguists who start translating the texts.

The actual translation consists of three main stages: analysis of the original text, which allows the lawyer-linguist to get to know the subject of the document and identify potential translation difficulties, terminological research and identification of sources, actual translation of the text, and using the research and sources identified at the earlier stage \citep{Izzo2014b}. It needs to be stressed that as a means of ensuring the high quality of translated documents, lawyer-linguists work exclusively into their own languages (\citealt[201]{Šarčević2013}, \citealt[15]{McAuliffe2016}). After the translation has been completed, it is time for the revision process, which aims to ensure legal and conceptual coherence with the original text and with other related documents of the Court. Its purpose is to verify whether the translation procedure has been followed according to the internal guidelines. Finally, the last stage involves proofreading, which is undertaken by a “pair of fresh eyes”. The goal of proofreading is to guarantee formal coherence of the translated text, which needs to correspond to the source text and linguistic correctness. The revision and proofreading stages make up the \isi{quality control} part of the \isi{translation process}.

There are other processes which indirectly form part of QC \citep{Izzo2014b}. One such example are processes aimed at maintaining terminological uniformity. This is ensured by using \isi{terminology} databases in all \isi{official languages}.\footnote{See, e.g., InterActive Terminology for Europe (IATE) – The \isi{European Union}’s multilingual term base \url{http://iate.europa.eu} (accessed 2017-05-02). The Court of Justice of the \isi{European Union} has its own internal comparative multilingual legal terminological database – CuriaTerm \citep[259]{Künnecke2013}.} The automatic translation of standard phrases is used, but this is not to be confused with rule-based or statistical \isi{machine translation} software being used on a large scale to produce translations of legal texts.\footnote{Machine translation carried out with the help of MT@EC allows \isi{EU} officials to receive quick, raw machine translations from and into any official \isi{EU} language (further information can be found for example at \url{https://ec.europa.eu/info/resources-partners/machine-translation-public-administrations-mtec_en} (accessed 2017-05-02).} Furthermore, seminars, meetings and conferences are regularly held, in which the lawyer-linguists or specialized guest speakers tackle particular legal topics for those involved in their translation.

\largerpage
Another key component of the QA process is perceived to be the co-operation between units and inter-institutional co-operation, which indirectly form a part of the workflow process. In order to maintain the highest level of coherence with the translated documents of other \isi{EU} institutions, the Court of Justice participates in a number of inter-institutional working groups, dealing with subjects such as training, \isi{human resources}, translation techniques, and \isi{terminology}. Inter-institutional co-operation allows smaller institutions to use the resources of the larger institutions, which allows for lawyer-linguists’ expertise to be taken advantage of by the translation services of other \isi{EU} institutions.

Such co-operation is, however, subject to certain restrictions \citep[1]{Annex2010}. The first major constraint refers to the type of documents translated by lawyer-linguists at the Court, which are “complex and structurally different” (i.e. \isi{case law} and procedural documents, \citealt{Annex2010}) from what is translated by translators at the three legislative institutions (i.e. the European Parliament, the Council of the \isi{European Union} and the \isi{European Commission}), and the question of confidentiality. Since, for example, the \isi{EU} institutions are often involved in the proceedings before the Court of Justice (CJ) or the General Court (GC), they are not allowed to translate their own pleadings \citep[2]{Annex2010}. The second major restriction concerns the workload of the \isi{CJEU}’s \isi{translation service}, which is shared between the CJ and the GC \citep[2]{Annex2010}. High workload translates into the lower availability of the Court’s \isi{TS} for other institutions requesting assistance. Due to the high workload of the \isi{TS}, it has been proposed that a part of translations generally outsourced to freelance translators with legal qualifications be entrusted to legally qualified translators in the TSs of some other \isi{EU} institutions.

The goal of inter-institutional co-operation is to avoid translating the same procedural documents translated by the French language units of different \isi{EU} institutions \citep[3]{Annex2010}. Such a form of an arrangement allows to cut back on double translations.

The most important forms of inter-institutional co-operation which contribute to QA in general are: sharing the products of terminological work and inter-institutional training activities \citep[3]{Annex2010}. The first measure contributes to the increased terminological \isi{consistency} of documents translated by not only the \isi{CJEU}, but also other institutions of the \isi{EU}. The second measure, i.e. inter-institutional training activities, entails conducting seminars on substantive matters of the law as well as \isi{terminology} by either \isi{CJEU} lawyer-linguists or outside expert legal scholars or judges. Such training is also open to translators from the TSs of other \isi{EU} institutions \citep[3]{Annex2010}.

\section{Conclusions}\label{sec:kozbial:5}

This chapter aimed at identifying the key aspects of Quality Assurance within the institutional setting of the \isi{CJEU}. It has been argued that quality aspects can be grouped around two key pillars of QA, namely \isi{human resources} and workflow processes. The \isi{human resources} level comprises, inter alia, in-house lawyer-linguists, \isi{external contractors}, auxiliary staff and project managers; the workflow level consists of measures aimed at achieving the proper structurization of the \isi{translation process} as well as intra- and interinstitutional co-operation. Both pillars enable the Court’s \isi{translation service} to provide high-quality translations in a timely manner, as evidenced by the way of functioning of the Court, which must rely on translations in order to work properly due to the complicated language arrangements that are in place.

Considering the fact that ensuring quality in translation by means of assessing only the final \isi{translation product} is not sufficient \citep[279]{Lušicky2017}, the influence of the two discussed aspects of QA is perceived to have the most significant impact on the \isi{translation process} and its effectiveness. I have described the current language arrangements at the Court of Justice of the \isi{European Union}, which due to their complexity might be viewed as an additional difficulty in the process of the translation of court documents. Moreover, I have pointed out the importance of industry-wide standards in \isi{legal translation} (cf. EN 15038:2006 \nocite{EN2006} and \citeauthor{ISO2015}:2015). Information on the profile of lawyer-linguists who are not “just” translators and \isi{external contractors} translating outsourced documents, the rigorousness of criteria for the selection of prospective candidates and assessment of \isi{translation quality} further show how complex and demanding the \isi{translation process} is. Although faced with many challenges, the Court of Justice of the \isi{European Union} and its legal \isi{translation service} are able to perform all their tasks without any major problems. This is evidenced by the data presented in, for instance, annual reports, which points to the fact that the Court is able to communicate both internally as well as externally with the wider public in the European context.
 
\section*{Acknowledgment}
This study was financed by research grant No. 2014/14/E/HS2/00782 from the National Science Centre, Poland.

 
% \section*{Abbreviations} 
\sloppy
\printbibliography[heading=subbibliography,notkeyword=this] 
\end{document}