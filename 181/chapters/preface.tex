\addchap{Notes on editors and contributors}
\lehead{Notes on editors and contributors}
\rohead{Notes on editors and contributors}
\noindent
\textbf{John L. Beaven}\\
After obtaining a PhD in Machine Translation from Edinburgh University, John Beaven worked as a computational linguistics researcher in academia (University of Cambridge) and industry (Sharp Laboratories of Europe, Oxford) in the fields of MT and linguistic databases. Since 1996, he has been working in the translation departments of different \isi{European Union} institutions and bodies, at first in the fields of MT and the deployment of CAT tools. He is currently responsible for the Quality Policy at the Translation Service of the General Secretariat of the Council of the \isi{European Union} (\isi{EU}).

\medskip\noindent
\textbf{Łucja Biel}\\
Łucja Biel is an Associate Professor at the Institute of Applied Linguistics, University of Warsaw, Poland, where she teaches and researches \isi{legal translation}. She is Secretary General of the European Society of Translation Studies (EST) and a deputy editor of the Journal of Specialised Translation. She was a Visiting Lecturer on the MA in Legal Translation at City University London from 2009 to 2014. She holds an MA in Translation Studies (Jagiellonian University of Kraków), PhD in Linguistics (University of Gdańsk), and Diploma in English and \isi{EU} Law (University of Cambridge) and a School of American Law diploma (Chicago-Kent School of Law and University of Gdańsk). She has participated in a number of internationally and nationally funded research projects and has published widely in the area of \isi{EU}/\isi{legal translation}, translator training and corpus linguistics, including a book\textit{ Lost in the Eurofog. The Textual Fit of Translated Law} (Peter Lang, 2014).

\medskip\noindent
\textbf{Jan Hanzl}\\
Jan Hanzl graduated from the Faculty of International Relations at the University of Economics in Prague and from the Faculty of Arts at the Charles University in Prague. Between 2004 and 2007 he worked as a reviser at the Office of the Government of the Czech Republic in a department responsible for the translation of \isi{EU} legislation adopted before the accession of the Czech Republic to the \isi{EU}, and as a freelance translator for \isi{EU} institutions. Since 2007 he has been working as a translator at the General Secretariat of the Council of the \isi{EU}.

\medskip\noindent
\textbf{Dariusz Koźbiał}\\
Dariusz Koźbiał is a PhD candidate at the Faculty of Applied Linguistics, University of Warsaw, Poland, writing a thesis entitled Translation of Judgments: A Corpus Study of the Textual Fit of \isi{EU} to Polish Judgments. In 2015, he graduated from the Institute of Applied Linguistics, University of Warsaw, with an MA in Applied Linguistics (English, German). He completed a three-month translation traineeship in the \isi{Directorate-General for Translation} at the European Parliament in Luxembourg. He is involved as an investigator in a research project ``The Eurolect: An \isi{EU} variant of Polish and its impact on administrative Polish'' funded by the Polish National Science Centre. His research interests include \isi{legal translation}, \isi{institutional translation} and corpus linguistics.

\medskip\noindent
\textbf{Krzysztof Łoboda }\\
Krzysztof Łoboda is a translator trainer and researcher at the Jagiellonian University in Kraków, where he earned an MA in Translation Studies to further continue PhD studies in linguistics. He has also completed postgraduate studies in Research Project Management and other courses such as Term Extraction and Management at Imperial College London. He is a member of the PKN Technical Committee 256 on Terminology, Other Language Resources and Content Management. Since 2004 Krzysztof has been an external translator/reviser of documents issued by \isi{EU} institutions (mostly EC, but also EP, CoR and EESC). His research interests include translation technology, specialized translation, and e-learning in translator training. He is currently involved in developing TRALICE consortium of translation professionals and researchers, a regional platform to facilitate cooperation between business and academia.

\largerpage
\medskip\noindent
\textbf{Fernando Prieto Ramos }\\
Fernando Prieto Ramos is Full Professor of Translation and Director of the Centre for Legal and Institutional Translation Studies (Transius) at the University of Geneva's Faculty of Translation and Interpreting. With a background in both Translation and Law, his work focuses on legal and \isi{institutional translation}, including interdisciplinary methodologies, international legal instruments and specialized \isi{terminology}. Former member of the Centre for Translation and Textual Studies at Dublin City University, he has published widely on \isi{legal translation}, and has received several research and teaching awards, including a European Label Award for Innovative Methods in Language Teaching from the \isi{European Commission}, an International Geneva Award from the Swiss Network for International Studies and a Consolidator Grant for his project on “Legal Translation in International Institutional Settings” (LETRINT). He has also translated for several organizations since 1997, including five years as an \isi{in-house translator} at the World Trade Organization (dispute settlement team).

\medskip\noindent
\textbf{Karolina Stefaniak }\\
Karolina Stefaniak is a linguist and translator. She obtained her PhD from University of Warsaw in 2008 with a dissertation in the field of critical discourse analysis on the communication between doctors and patients. Since 2008 she has been working in the Polish Language Department of the \isi{Directorate-General for Translation} (\isi{DGT}) of the \isi{European Commission} in Luxembourg, first as a translator, then as the main \isi{terminologist} and currently as a quality officer. She has published several articles on doctor-patient interaction, medical discourse and medicalization, and also on specialized and \isi{institutional translation}, including translation in \isi{EU} institutions. 

\medskip\noindent
\textbf{Ingemar Strandvik }\\
Ingemar Strandvik works as a \isi{quality manager} at the \isi{European Commission}’s \isi{Directorate-General for Translation}, where he was formerly a translator. He has a background as a state-authorized legal translator and court interpreter in Sweden, where he also taught translation at Stockholm University and participated in curriculum design for Translation Studies. For many years he was active as a lexicographer at the publishing house Norstedts. Apart from studies in Philology and degrees in Translation and Interpreting, he has a Master’s degree in \isi{EU} Law. He is currently involved in standardization work and regularly participates in conferences and publishes on \isi{translation quality}, multilingual law-making and \isi{terminology}.

\medskip\noindent
\textbf{Tomáš Svoboda}\\
Tomáš Svoboda is Head of German Department in the Institute of Translation Studies, Charles University, Prague, Czech Republic, from which he graduated in English and German translation. He earned his Ph.D. in Translation Studies in 2004. From 2004 to 2007 Tomáš worked as an \isi{in-house translator} and training coordinator for the Czech Language Department of the \isi{Directorate-General for Translation}, \isi{European Commission}, Luxembourg, and subsequently for four years as a contractor for the European Central Bank in Frankfurt, Germany. He lectures on technical and \isi{institutional translation}, translation tools and technologies, and translation history in the Institute of Translation Studies. He is an active translator and auditor under the \isi{ISO} \isi{17100} standard. His publications cover \isi{translation quality}, institutional and technical translation, translation technology, future of the translation profession as well as translation history. Tomáš is member of the Executive Board of the Czech Union of Translators and Interpreters, a coordinator of FIT Europe (Fédération Internationale des Traducteurs) Technology Group, and Board Member of the European Masters in Translation (EMT) network.

\medskip\noindent
\textbf{Sonia Vandepitte }\\
Sonia Vandepitte is a Full Professor at the Department of Translation, Interpreting and Communication at Ghent University and head of its English section. She teaches English, Translation Studies, and translation into and from Dutch. Publication topics include causal expressions in language and translation, methodology in Translation Studies, translation competences, anticipation in interpreting, international translation teaching projects and translation and post-editing processes. She is currently involved in eye-tracking research into reading and translation processes of translation problem-solving. She is also investigating peer feedback and other collaborative forms of learning in translation training and co-editing the Handbook of Research on Multilingual Writing and Pedagogical Cooperation in Virtual Learning Environments and Quality Assurance and Assessment Practices in Translation and Interpreting.