\chapter{Einleitung}
\label{chapter:einleitung}
Untersuchungsgegenstand dieser Arbeit sind (Kombinationen von) Modalpartikeln (MPn) im Deutschen.

(1) bis (4) zeigt Beispiele für das Einzelauftreten der Partikeln, die im Laufe der Arbeit behandelt werden.

\begin{exe}
	\ex\label{1} 
		England ist \textbf{ja} eine Stunde zurück.	
\end{exe}
\vspace{-0.65cm}
\begin{exe}
	\ex\label{2} 
		\begin{xlist}
		\ex\label{2a}A: Ich muss mir dann ein Zugticket in die Stadt kaufen. \\B: Du kannst \textbf{doch} Englisch.
		\ex\label{2b}A: Ich lande erst um 22:30 Uhr in Düsseldorf.\\B: Übernachte \textbf{doch} bei uns!	
	\end{xlist}
\end{exe}
\begin{exe}
	\ex\label{3} 
		\begin{xlist}
		\ex\label{3a}A: Warum werden denn die Ausweise kontrolliert? \\B: England ist \textbf{halt/eben} nicht Teil des Schengen-Raums.
		\ex\label{3b}A: Ich habe gar nicht mehr genügend Pfund.\\B: Dann zahl \textbf{halt/eben} mit Karte!
	\end{xlist}
\end{exe}
	
\begin{exe}
	\ex\label{4} 
		\begin{xlist}
		\ex\label{4a}A: Es wird schon um 16 Uhr dunkel. \\B: Wir haben \textbf{auch} Januar.
		\ex\label{4b}Tausch \textbf{auch} Pfund um! Sonst kannst du nichts kaufen.
	\end{xlist}
\end{exe}
Zu den Eigenschaften, die diesen Ausdrücken zugeschrieben werden (vgl. für einen Überblick \citealt{Diewald2007}, \citealt[628-630]{Thurmair2013}, \citealt[Kapitel 2]{Mueller2014b}), zählen u.a., dass sie kein Flexionsparadigma bilden, in der Regel unakzentuiert sind, nur im Mittelfeld auftreten, keinen wahrheitsfunktionalen Beitrag leisten, wenig bzw. nur abstrakte lexikalische Bedeutung aufweisen und eher kommunikative, spre\-cherbezogene, diskursstrukturelle Funktion übernehmen sowie dass sie \glq Dubletten\grq {} in anderen Wortarten aufweisen, von denen es sie abzugrenzen gilt.

Das Phänomen der MPn ist gut erforscht. Es gibt eine Vielzahl von Arbeiten zu den verschiedensten Aspekten und aus den verschiedensten Perspektiven. Einen Schwerpunkt bilden hier Arbeiten, die sich mit der Bedeutung und Funktion einzelner Partikeln beschäftigen (vgl. stellvertretend für zahlreiche Arbeiten z.B. \citealt{Franck1980}, \citealt{Hentschel1986}, \citealt{Thurmair1989} sowie die Sammelbände von \citealt{Weydt1977, Weydt1979, Weydt1983a, Weydt1986} in älterer Zeit und \citealt{Karagjosova2003, Karagjosova2004}, \citealt{Zimmermann2011}, \citealt{Egg2013}, \citealt{Mueller2014a, Mueller2016a}, \citealt{Gutzmann2015} für neuere Arbeiten). Auf systematischer Ebene hat man sich neben Fragen zur Bedeutung und Funktion der Elemente z.B. auch mit ihrer internen (vgl. z.B. \citealt[50-63]{Meibauer1994}, \citealt[37-41]{Ormelius-Sandblom1997}), \citealt[99-104]{Coniglio2011}, \citealt{Struckmeier2014}) und externen (vgl. z.B. \citealt[32-36, 43-45]{Ormelius-Sandblom1997}, \citealt[104-115]{Coniglio2011}, \citealt{Abraham1995, Abraham2012}, \citealt{Gutzmann2016}) Syntax, d.h. dem syntaktischen Status der MPn sowie ihren Stellungseigenschaften, beschäftigt. Letz\-terer Aspekt interagiert auch mit informationsstrukturellen Verhältnissen (vgl. hierzu z.B. \citealt[73-87]{Meibauer1994}, \citealt[101-115]{Ormelius-Sandblom1997}. Diachrone Gesichtspunkte werden z.B. in \citet[Kapitel 3]{Hentschel1986}, \citet[158-170]{Meibauer1994}, \citet[Kapitel 4.2]{Diewald1997}; \citeyearpar{ Diewald2008} untersucht. Hier stehen vor allem Fragen zur Entstehung der MPn im Zuge eines Grammatikalisierungsprozesses sowie semantische Verbindungen zwischen den Vormodalpartikellexemen und den aktuellen Partikelverwendungen im Zentrum des Interesses. In angewandteren Teildisziplinen sind auch kontrastive Arbeiten bzw. Untersuchungen zur Übersetzung/Übersetzbarkeit (vgl. z.B. \citealt{Schubiger1965}, \citealt{Burkhardt1995}, \citealt{Masi1996}, \citealt{Diewald2010}) sowie zum Zweitspracherwerb von MPn (vgl. z.B. \citealt{Rost-Roth1999}, \citealt{Moellering2004}) durchgeführt worden.

Eine weitere typische Eigenschaft von MPn ist ihre prinzipielle Kombinierbarkeit, d.h. MPn können in Reihe vorkommen, wie (\ref{5}) bis (\ref{7}) illustrieren. 

\begin{exe}
	\ex\label{5} 
		A: Jetzt kann ich sie nicht mehr anrufen. \\
		B: Wieso? England ist \textbf{ja doch} eine Stunde zurück.
\end{exe}

\begin{exe}
	\ex\label{6} 
		A: Warum werden denn die Ausweise kontrolliert? \\
		B: England ist \textbf{halt eben} nicht Teil des Schengen-Raums.
\end{exe}

\begin{exe}
	\ex\label{7} 
		A: Es wird schon um 16 Uhr dunkel. \\
		B: Wir haben \textbf{doch auch} Januar.
\end{exe}	
Obwohl es eine Vielzahl von Arbeiten zum Phänomen der MPn an sich gibt, beschäftigen sich die wenigsten Autoren mit MP-Sequenzen. Hier sind noch viele Fragen offen.

Man ist sich in der Literatur einig, dass man es mit zwei gestuften Beschränkungen zu tun hat: Zum einen können nicht alle MPn miteinander kombiniert werden (vgl. z.B. (\ref{8}) bis (\ref{10}) für inakzeptable Fälle).

\begin{exe}
	\ex\label{8} 
		*Fährt die Fähre \textbf{doch hübsch}/\textbf{hübsch doch} an Silvester?
\end{exe}
\vspace{-0.65cm}
\begin{exe}
	\ex\label{9} 
		*Haben Melanie und Philipp \textbf{eben etwa}/\textbf{etwa eben} im Sommer geheiratet?
\end{exe}
\vspace{-0.65cm}
\begin{exe}
	\ex\label{10} 
		*Familie Dicke fährt \textbf{schon denn}/\textbf{denn schon} in den Urlaub.
	\newline
	\hbox{}\hfill\hbox{\citet[84]{Mueller2014b}}
\end{exe}	
Zum anderen gibt es die Beobachtung, dass die Abfolgen innerhalb einer prinzi\-piell zulässigen Kombination restringiert sind. Die Standardannahme ist, dass die Abfolgen fest sind, d.h. nur eine der Abfolgen als akzeptabel einzustufen ist. Zwischen (\ref{5}) bis (\ref{7}) und (\ref{11}) bis (\ref{13}) stellt sich ohne Zweifel jeweils ein Akzep\-tabilitätsverlust ein.

\begin{exe}
	\ex\label{11} 
		A: Jetzt kann ich sie nicht mehr anrufen. \\
		B: Wieso? ??England ist \textbf{doch ja} eine Stunde zurück.
\end{exe}

\begin{exe}
	\ex\label{12} 
		A: Warum werden denn die Ausweise kontrolliert? \\
		B: England ist \textbf{eben halt} nicht Teil des Schengen-Raums.
\end{exe}

\begin{exe}
	\ex\label{13} 
		A: Es wird schon um 16 Uhr dunkel. \\
		B: ??Wir haben \textbf{auch doch }Januar.
\end{exe}
Diese Arbeit beschäftigt sich mit genau diesem Aspekt des kombinierten Partikel-Auftretens. Ich beabsichtige eine Klärung der Frage, warum die eine der beiden denkbaren Abfolgen der anderen vorgezogen wird. 

Das Rätsel um die (un)möglichen Kombinationen (mit Ausnahme einiger Spezi\-alfälle) scheint gelöst. Man nimmt an, dass die Partikeln satzmodal und funktional kompatibel sein müssen (vgl. Abschnitt~\ref{sec:distributionjd} zur detaillierten Ausführung und Illustration). Hinsichtlich der zweiten Beschränkung zu den Abfolgen herrscht hingegen größere Uneinigkeit. Es sind auch hier verschiedenste Vorschläge gemacht worden, die sich mit der Frage aus Perspektive von Syntax, Semantik, Pragmatik, Phonologie, Informationsstruktur und Diachro\-nie beschäftigen (vgl. Abschnitt~\ref{sec:forschung} zu einem Überblick). \\

\noindent
Mein Programm ist es, die Reihungsbeschränkung auf der Ebene der Interpretation anzusetzen. Konkret vertrete ich, dass die Form (die der Abfolge entspricht) die Funktion (repräsentiert durch den Diskursbeitrag der MPn) widerspiegelt. Ich argumentiere somit, dass ein ikonischer Zusammenhang besteht in dem Sinne, dass die Abfolge durch den Beitrag der Partikeln im Diskurs motiviert ist. Die MPn spiegeln dabei mit ihrer Ordnung der Applikation diskursstrukturelle Verhältnisse, die als unabhängig gültig anzusehen sind. Die MP-Kombinationen stellen somit quasi Diskursverläufe auf einer \glq Mikroebene\grq {} dar. \\

\noindent
Eine Betrachtung des Diskursbeitrags der Kombinationen setzt dabei natürlich eine Auseinandersetzung mit den Einzelpartikeln voraus. Dazu gehört, dass ein bestimmter MP-Zugang gewählt wird. 

Ich vertrete hier in Anlehnung an Arbeiten von \cite{Diewald1999a, Diewald2006, Diewald2007} und \cite{Diewald1998} in Verbindung mit der formalen Modellierung von Diskursverläufen im Rahmen von \cite{Farkas2010}, dass MPn Anforderungen an die (unmittelbar) vorausgehenden Kontextzustände stellen. Die vorliegenden Konstellationen zusammen mit dem regulären diskursiven Beitrag des Äuße\-rungstyps ergeben die unterschiedlichen Interpretationen verschiedener MP-Äu\-ßerungen.

Meine MP-Modellierung fällt unter sogenannte \textit{bedeutungsminimalistische} Zu\-gänge (im Gegensatz zu \textit{-maximalistischen} \is{Bedeutungsminimalismus/-maximalismus}). Ich gehe davon aus, dass die Zuschreibung einer abstrakten MP-Bedeutung möglich ist und nicht für jede Gebrauchs\-weise, wozu auch das Vorkommen derselben Partikel in verschiedenen Äuße\-rungstypen zählt, eine eigene Interpretation formuliert werden muss. \\

\noindent
Eine kontrovers diskutierte Frage in Verbindung mit der Interpretation der MP-Kombinationen betrifft die Entscheidung, von welcher Art von Skopusverhältnis auszugehen ist, wenn beide beteiligten Partikeln im Satz Skopus nehmen. Ich werde hier vertreten, dass beide Partikeln identischen Skopus nehmen, d.h. ihren Beitrag in Bezug auf dieselbe Proposition leisten. \\

\noindent
Im Zuge der Idee um einen ikonischen Zusammenhang nehme ich darüber hi\-naus an, dass es nicht nur \emph{eine} grammatische Ordnung gibt und die andere als völlig inakzeptabel und non-existent herausgefiltert werden muss. Vielmehr vertrete ich, dass man es mit einem Markiertheitsphänomen zu tun hat. Wie für markierte und unmarkierte Strukturen generell zu erwarten ist, wird die unmarkierte Struktur von Sprechern besser bewertet, ist häufiger in Korpora zu finden und hat eine weitere Verteilung als die markierte Form. 

Für die von mir untersuchten Kombinationen werde ich deshalb auch (entgegen aller mir bekannten Ansätze) argumentieren, dass die umgekehrten Abfolgen existieren. Soweit meine derzeitigen empirischen Ergebnisse dies zulassen, gehe ich davon aus, dass sie nur deutlich seltener gebraucht werden und (zwar, weil sie) auf spezielle Kontexte beschränkt sind. Bei der Beschäftigung mit den markierten Reihungen ist es daher eine Aufgabe, diese Umgebungen zu ermitteln und Erklärungen zu finden, warum sich gerade diese speziellen Kontexte für eine Umkehr der Partikelanordnungen anbieten. 

Ich werde hier argumentieren, dass den Partikeln in den markierten Anordnungen – bedingt durch den Satzkontext – ein anderes Gewicht zukommt als in den unmarkierten Abfolgen. Dies führt dazu, dass eine Abweichung von dem Diskursprinzip, das die unmarkierte Reihung steuert, möglich wird. \\

\noindent
Grammatiktheoretisches Ziel der Arbeit ist es, für die drei untersuchten MP-Kombinationen Abfolgebeschränkungen zu formulieren und damit – vor dem Hintergrund der generellen Idee um das Vorliegen diskursstruktureller Ikoni\-zität – jeweils eine Erklärung für den Markiertheitsunterschied zwischen den zwei Abfolgen zu entwickeln. \\

\noindent
Wenngleich ich oben konstruierte Beispiele angeführt habe, werde ich mich in meiner Argumentation vorwiegend auf authentische Belege beziehen. Besonders die Beschäftigung mit den umgekehrten Abfolgen wird zeigen, dass die Betrachtung sehr großer Datenmengen vonnöten ist. Da zudem konzeptionell an der Mündlichkeit orientierte Daten vorliegen müssen, damit überhaupt MPn auftreten, habe ich vor allem das Korpus \textit{DECOW}, das Webdaten beinhaltet, als Quelle verwendet. Wo sich Korpora nicht eignen, um empirische Evidenz anzufüh\-ren, habe ich Akzeptabilitätsstudien durchgeführt. 

Ich beabsichtige, auf diesem Weg eine solide Datenbasis zu schaffen. Zu oft fehlt theoretischen Arbeiten m.E. eine empirische Vorarbeit und empirischen Arbeiten andererseits ein theoretischer und erklärender Beitrag bzw. werden diese beiden Aspekte häufig noch voneinander entkoppelt und sehen Forscher sich als \glq empirisch\grq {} \underline{oder} \glq theoretisch\grq {} arbeitend. \\

\noindent
Ich versuche in dieser Arbeit, Empirie und Theorie gerecht zu werden. An einigen Stellen werden – trotz dieses Vorhabens – allerdings auch die Grenzen empirischer Studien (zu diesem Thema) deutlich werden, so dass ich in diesen Fällen nur vorsichtige Aussagen machen kann, weil bestimmte Verhältnisse (noch) nicht auszuschließen sind und in diesem Sinne empirische Unbekannte im Spiel sind.\\

\noindent
Ich möchte folglich empirische Ergebnisse für eine formal-pragmatische Modellierung nutzen. Ich beabsichtige, generalisierbare Aussagen zu formulieren, die in vielen Fällen nur auf der Basis entsprechend großer Datenmengen zu errei\-chen sind. Im Falle der Beschäftigung mit MPn führt dies dazu, dass die Belege alle gesichtet werden müssen, da anders nicht zu entscheiden ist, ob die vorliegende Form tatsächlich als MP gebraucht wird. Absolut ist diese Entscheidung natürlich nicht. Dieses Vorgehen führt aber sicherlich zu genaueren Ergebnissen, als wenn man es ausließe.\\

\noindent
Gehen bestehende Arbeiten zu MP-Kombinationen über eine rein deskriptive Erfassung der (un)zulässigen Ordnungen hinaus, beschäftigen sie sich mit einzelnen Partikel-Kombinationen. Dies gilt auch für meine Ausführungen. Ich werde mir Kombinationen aus \textit{ja} und \textit{doch}, \textit{halt} und \textit{eben} sowie \textit{doch} und \textit{auch} im Detail anschauen und vor dem Hintergrund eines übergeordneten Erklärungsmodells seine konkrete Wirkung jeweils ausführen. Es sollte natürlich der Anspruch an einen solchen Erklärungsrahmen sein, dass er prinzipiell alle MP-Kombinationen erfassen kann. Ich schließe dies nicht aus, bin aber der Meinung, dass Detailana\-lysen notwendig sind, um dies nachzuweisen. Da von 171 möglichen MP-Kombi\-nationen ausgegangen wird, von denen ca. 50 verwendet werden (vgl. \citealt[280]{Thurmair1989}), möchte ich diese weite Behauptung nicht unüberprüft machen. Meine Einzelanalysen werden zeigen, wie viele Details zu betrachten und entscheiden sind. Dennoch möchte ich das größere Bild nicht aus dem Blick verlieren und werde die Frage nach der Übertragbarkeit auf andere Kombinationen im Schluss\-teil thematisieren. \\

\noindent
Neben der Einleitung (Kapitel~\ref{chapter:einleitung}) und dem Schlussteil (Kapitel~\ref{chapter:schluss}) besteht die Arbeit aus vier größeren Kapiteln.
 
Kapitel~\ref{chapter:hintergrund} beschäftigt sich mit einigen allgemeinen Aspekten, vor deren Hintergrund ich meine eigenen Annahmen vertreten möchte. So werden in Abschnitt~\ref{sec:forschung} bestehende Arbeiten zu den MP-Kombinationen vorgestellt. Diese Darstellung erfolgt nicht chronologisch, sondern konzeptorientiert. Trotz unterschiedlicher konkreter Ausbuchstabierungen liegen den Ansätzen oftmals prinzi\-piell die gleichen Ideen zugrunde. Abschnitt~\ref{sec:transparenz} behandelt Fragen zur Interpretation der MP-Kombinationen. Hier geht es vor allem um die Möglichkeiten einer transparenten Bedeutungszuschreibung. In Abschnitt~\ref{sec:diskursmodell} und \ref{sec:zugang} werden das Diskursmodell von \cite{Farkas2010} sowie der MP-Zugang von \cite{Diewald1998} vorgestellt, auf die ich mich bei der Modellierung des Beitrags der MPn beziehe. 

Abschnitt~\ref{sec:ikonizität} gibt einen kurzen Überblick über das Konzept von Ikonizität an sich sowie verschiedene ihrer Spielarten. Der Zweck der Darstellung ist hier, aufzu\-zeigen, welche Art von Ikonizität ich bei meiner Beschränkung der MP-Reihungen als beteiligt ansehe.

Die Kapitel~\ref{chapter:jud}, \ref{chapter:hue} und \ref{chapter:dua} stellen jeweils Detailuntersuchungen zu einzelnen Kombinationen dar.

Gegenstand von Kapitel~\ref{chapter:jud} sind Sequenzen aus \textit{ja} und \textit{doch}. Abschnitt~\ref{sec:distributionjd} be\-stimmt die selbständigen und unselbständigen Sätze, in denen diese beiden Partikeln überhaupt kombiniert werden können. Abschnitt~\ref{sec:abfolgejd} skizziert Ansätze aus der Forschung, die für die einzig zulässige Reihung \textit{ja doch} argumentiert haben, bevor Abschnitt~\ref{sec:distributiondj} drei Kontexte eröffnet, in denen sich die umgekehrte Anordnung nachweisen lässt. In Abschnitt~\ref{sec:interpretationjd} stelle ich meine Modellierung des Beitrags der Einzelpartikeln dar. Es schließen sich in Abschnitt~\ref{sec:unmarkiert} und \ref{sec:markiert} meine Ableitungen der markierten und unmarkierten Ordnungen an. Abschnitt~\ref{sec:status} thematisiert einige empirische Fragen zur Sequenz \textit{doch ja}.

Mit Kombinationen aus \textit{halt} und \textit{eben} beschäftigt sich Kapitel~\ref{chapter:hue}. Abschnitt~\ref{sec:hueinliteratur} behandelt Vorstellungen zur Auftretensweise und Bedeutung der Einzelpartikeln in der Literatur. In Abschnitt~\ref{sec:modellierung} schlage ich eine diskursstrukturelle Modellierung für den Beitrag von \textit{halt} und \textit{eben} vor. Abschnitt~\ref{sec:empirie} diskutiert unter Bericht der Ergebnisse verschiedener eigener Studien empirische Fragen zur Reihung der beiden Partikeln, die sich aus Annahmen aus der Literatur ergeben. Abschnitt~\ref{sec:interpretationkombi} beschäftigt sich mit der Interpretation der Kombination. In Abschnitt~\ref{sec:erklärunghe} stelle ich meine Ableitung des unmarkierten Status von \textit{halt eben} und der Markiertheit von \textit{eben halt} vor. Abschnitt~\ref{sec:gebrauchheeh} untersucht am Beispiel des Auftretens der beiden Partikeln und ihrer Kombinationen in Relativsätzen, inwiefern sich vertreten lässt, dass die beiden Partikel-Folgen unterschiedlich verwendet werden.

Kapitel \ref{chapter:dua} befasst sich mit dem kombinierten Vorkommen von \textit{doch} und \textit{auch}. Abschnitt~\ref{sec:präferenz} legt die Untersuchung unter Bezug auf Annahmen aus bestehenden Ansätzen und Häufigkeitsverteilungen in Korpora auf die unmarkierte Sequenz \textit{doch auch} fest. Abschnitt~\ref{sec:distributionda} bestimmt die Äuße\-rungstypen, die durch die spätere Erklärung der unmarkierten Abfolge erfasst werden müssen. Abschnitt~\ref{sec:V2} führt über die Modellierung des isolierten Auftretens von \textit{doch} und \textit{auch} und ihrer Interpretation bei gemeinsamem Auftreten zu meiner Erklärung, warum das \textit{doch} dem \textit{auch} im unmarkierten Fall vorangeht. Während Abschnitt~\ref{sec:V2} sich aus\-schließlich mit (Standard-)V2-Deklarativsätzen befasst, untersucht Abschnitt~\ref{sec:Rand} die assertiven Randtypen der \textit{Wo}-VL- und V1-Deklarativsätze. Aufgrund von Annahmen aus der Literatur zur Rolle von \textit{doch} in diesen Satztypen steht hier die Frage im Raum, ob sich meine Erklärung der unmarkierten Sequenz auf diesen Kontext übertragen lässt. Abschnitt~\ref{sec:direktive} weitet den Blick auf das Vorkommen der beiden Partikeln und ihrer Kombination in Direktiven aus. Abschnitt~\ref{sec:distributionad} führt Belege an, die aufzeigen, dass auch die umgekehrte Abfolge \textit{auch doch} nicht gänz\-lich auszuschließen ist und stellt Überlegungen an, die das Vorfinden dieser Reihung in den ausgemach\-ten Kontexten motivieren können.

Kapitel~\ref{chapter:schluss} fasst die Ergebnisse zusammen und blickt auf einige allgemeinere Aspekte und Fragen, die sich aus der speziellen Untersuchung der drei MP-Kombi\-nationen ergeben.\\

\noindent
Bevor Kapitel~\ref{chapter:hintergrund} nun allgemeinere Aspekte zum Thema der MP-Kombinationen bzw. meines Zugangs beleuchtet, möchte ich einige Hinweise zur Organisation und Darstellungen (in) der Arbeit geben.

Die Korpusbelege stellen den Originalwortlaut dar, d.h. ich habe nahezu keine Veränderungen hinsichtlich Formatierung, Grammatik, Orthographie oder Interpunktion vorgenommen. Gerade bei den Webdaten sind hier mitunter Abwei\-chungen von der Norm zu beobachten.

Da im Laufe der Arbeit gleiche Konzepte auf verschiedene Art eine Rolle spielen, wiederhole ich bestimmte Aspekte an verschiedenen Stellen in der Darstellung kurz und verweise auf ihre erste ausführlichere Darstellung. Ebenfalls wiederholen sich manche Fragen bei der Untersuchung von jeder MP-Kombination. Ich führe diese Punkte bei ihrer ersten Erwähnung ausführlicher aus als bei ihrer erneuten Thematisierung. Dies führt dazu, dass manche Aspekte am Beispiel (der Kombinationen) von \textit{ja} und \textit{doch} in Kapitel~\ref{chapter:jud} detaillierter dargestellt werden als im Zuge der Untersuchung von \textit{halt} und \textit{eben} in Kapitel~\ref{chapter:hue} bzw. \textit{doch} und \textit{auch} in Kapitel~\ref{chapter:dua}. Die größere Ausführlichkeit in der Darstellung mancher Fragen in Kapitel~\ref{chapter:jud} ist nicht nur auf die Reihenfolge, in der ich die Partikel-Kombinationen behandle, zurückzuführen, sondern auch darauf, dass die meiste Forschungslite\-ratur zu diesen beiden MPn und auch ihrer Kombination vorliegt und deshalb idealerweise zu Anfang eingeführt wird.
	

