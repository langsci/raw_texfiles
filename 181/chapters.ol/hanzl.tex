\documentclass[output=paper]{langsci/langscibook} 
% \ChapterDOI{} %will be filled in at production

\author{Jan Hanzl\lastand
John Beaven\affiliation{Council of the European Union}}
\title{Quality assurance at the Council of the EU’s translation service}
% \shorttitlerunninghead{If the normal title is too long in the page headers}
\abstract{The aim of this chapter is to describe quality assurance mechanisms at the Translation Service of the General Secretariat of the Council of the EU (GSC). The first part will put the GSC's translation activity into a more general framework of the workings of the whole institution. Furthermore, the GSC's approach to translation quality will be explained, and tools and procedures that are used and help translators achieve the required quality of their products will be described. The next part will focus on the ex-post quality monitoring that was introduced a few years ago to systematically monitor both the quality of translations that leave the Translation Service and the individual performance of translators. The final part will be dedicated to a recently-adopted special procedure to ensure the best quality of the GSC Translation Service’s hallmark product – the European Council conclusions. The chapter is descriptively oriented and draws on everyday practice of a GSC's translator and the quality policy coordinator. Hopefully, it will raise awareness of the activities of the GSC's Translation Service, provide inspiration for other translation departments and practitioners and offer topics for further research for academia.

With approximately 600 translators and 300 other management and support staff, the General Secretariat of the Council of the EU’s Translation Service is a little smaller than the European Parliament’s Directorate-General for Translation and about half the size of the European Commission’s Directorate-General for Translation, but still large by most standards. Each year the Council’s service translates around 15,000 documents, which represent roughly 110,000 pages of source material and a yearly translation output, expressed as a sum of all target languages, in the range of 1.2 million pages (\citealt{Council2016a}, \citealt{Council2017}). However, quality rather than quantity has always been the primary focus of the Council’s Translation Service, and the aim of this chapter is to describe the quality assurance mechanisms that are used at the Council’s Translation Service to ensure the required quality of its products.
}
\maketitle

\begin{document}

\section{The Council(s), the General Secretariat of the Council and its Translation Service}\label{sec:hanzl:hanzl:1}

To understand the approach to translation quality at the General Secretariat of the Council of the EU's Translation Service, it is necessary to put its translation activity into a more general framework of the workings of the two institutions it serves, namely the Council of the European Union, formerly known as the Council of Ministers, where ministers meet to adopt legislation and coordinate policies, and the European Council, which brings together the heads of state or government in meetings also known as EU Summits.\footnote{The views expressed are our own and in no way reflect the views of the Council or the European Council.} Since the Lisbon Treaty, these two Councils are formally two EU institutions, but both are supported by one general secretariat — the General Secretariat of the Council (GSC) — of which the Translation Service is a part. The Translation Service itself is split into 24 units, one per official language. Each language unit usually consists of just over 20 translators, a Head of Unit, a Quality Controller and a number of assistants. Most of the time, translators work into their mother tongue, and the language of source documents is predominantly English\footnote{In 2016, nearly 90 \% of all translations at the GSC were done from English \citep{Council2017}.}.

Still, the Councils themselves are only the tip of an iceberg for the GSC Translation Service's work. Underneath, there are more than 150 specialized Working Parties and Committees which discuss and prepare the documents before their formal adoption. As the flow of documents through this structure of preparatory bodies up to the Councils themselves has major implications for the translation work, it deserves a brief description. Most EU legislation originates with a proposal from the Commission, presented in all 24 EU official languages. This proposal is discussed, often several times, by Member States' experts at relevant Working Parties. The experts usually make changes to the source-language version of the text, and, at certain point, the amended text is sent to the Translation Service for translation. Then the same process may be repeated at the same or higher level preparatory body, until there is sufficient support from all Member States and the text can be submitted to the Council for approval. In the case of legislative acts, the text is subsequently sent for finalization to lawyer-linguists from the GSC's Legal Service. In reality, the whole process is much more complicated, but the aim of this simplified description is to illustrate why translators' work at the GSC predominantly looks as shown in \figref{fig:hanzl:1}.

\begin{figure}
\includegraphics[height=.9\textheight]{figures/hanzl.png}
\caption{The ``Document compare'' working mode.}
\label{fig:hanzl:1}
\end{figure}

   
This document comparison is produced with a standard commercial tool (Workshare Compare) and shows what has changed since the last translated version: additions are shown in blue, deletions in red and moved text in green. The task of GSC's translators is to reflect these changes introduced by Member States' experts in the remaining language versions of the document.

The flow of documents described in \figref{fig:hanzl:1}\footnote{This is the standard flow of legal and political documents. Apart from these, the GSC's Translation Service also translates other documents like agendas, minutes, speeches, web articles, etc., which are naturally produced in a different way. Nevertheless, even these documents often rely on their previous versions or related legal or political texts, so even in their case the level of intertextuality as indicated by the \textit{document compare} remains quite high.} has several implications for GSC translators' work. One implication is that very little is translated from scratch. There is a large amount of repetition and interconnection among texts, both explicit and implicit intertextuality (see also \citealt{Koskinen2000}: 59, \citealt{Robertson2015}: 42). That is why consistency of terminology and phraseology both inside one document as well as across documents is paramount.

Another implication is that the documents translated are often working or interim versions drawn up in a hurry by non-native English speakers, not final, well-edited and fine-tuned texts\footnote{See \citet{Stefaniak2013} for a similar remark concerning texts translated by the European Commission.}. This has one big advantage and one big drawback. The drawback is that the quality of source documents may not be ideal; the advantage is that should an error in translation be spotted after the translated document has been delivered to the client, translators may get a chance to correct it in the following version, if there is one.

Last but not least, this system of work has also implications for the setting of deadlines. As the translated documents are needed for a specific meeting or serve as input for further precisely scheduled work, translation deadlines need to be adapted to the requirements of document users, and can sometimes be very short. Often, they are set for a specific hour of the very same day\footnote{In 2016, translations with a deadline shorter than one day accounted for around 45\% of all translated documents; in terms of net pages, their share was around 20\% \citep{Council2017}.}.

\section{The pragmatic approach to translation quality}\label{sec:hanzl:hanzl:2}

The GSC's Translation Service aims to deliver end products which transpose — into the target language and by the set deadline — the entire contents of the source document with clarity, fluency and precision, in terms of form and content, without any formal or material errors, and without any additions or omissions, while taking into account the nature and the relative importance of the original to be translated \citep[4]{Council2006}. This definition of translation quality is similar to the ISO Standard 17100 on Requirements for translation services (\citeyear{ISO2015}). However, the aim of the GSC's Translation Service is not to apply for ISO certification, but rather to ensure that its practice is broadly in line with international standards in the profession.

In essence, the GSC Translation Service's qualitative requirements, stemming from the above definition, can be roughly divided into three categories (see also the quality monitoring criteria in part 5 of this chapter). First, there are, naturally, \textbf{linguistic} aspects (“... [transpose] the entire contents of the source document with clarity, fluency and precision ...”). However, one of the findings from quality monitoring is that translators sometimes become fixated on the linguistic quality at the expense of everything else. This is why GSC's translators are kept aware that apart from linguistic aspects their translation work has at least two other qualitative dimensions. 

\textbf{Technical} quality aspects (\textit{‘… in terms of form …’}) include requirements that the layout of a translation correspond to that of the original and that technical and typographical conventions of the target language be respected. At the Council and its preparatory bodies specifically, parallel pagination of the different language versions of a text for instance is not just a formal requirement but a practical necessity. As there are usually at least 28 delegations discussing and expressing themselves on a particular document, it is absolutely crucial for all of them to be on the same page at least in the text, if not mentally. Otherwise their communication might collapse. Compliance with the technical requirement to use computer-aided translation tools also makes translations easier to recycle and helps to preserve the necessary continuity and terminological consistency within and among documents.

Finally, \textbf{timeliness} (\textit{‘... by the set deadline ...’}) is the third inherent part of quality requirements for the GSC translation. It follows from a legal provision stipulating that ministers or ambassadors can generally vote only on documents which are available in all official languages. A missing language version can therefore complicate or paralyze the whole decision-making process at the Council, and in such a case even a top-notch linguistic quality of a translation cannot compensate for its late submission.

Owing to the omnipresent pressure on public institutions to use resources efficiently, translation quality measures need to be directed towards achieving an optimal level of useful quality, fit for the intended purpose of the document. Hence the above definition of translation quality is implemented through the concept of fitness for purpose: a translation is considered fit for purpose when it is suitable for its intended communicative use, follows the linguistic and technical specifications and complies with the expressed and implied requirements of the client\footnote{The main clients of the GSC's Translation Service are the European Council and its President, the rotating presidencies of the Council of the EU, the Council and its preparatory bodies, the requesting departments in the Directorates-General of the GSC, Member States' delegations and national administrations, other EU institutions, the European External Action Service, stakeholders in the subject areas concerned, and the general public.} \citep[36]{Council2015}. 

What this principle of fitness for purpose means in practice is that the efforts devoted to spotting and correcting errors must be adjusted to the type of text in question. If there is a wrong date in a translation (e.g. 13 May instead of 31 May) in a footnote reference to published legislation (where everything else is correct), it is objectively a translation error, but not a serious one as the reader will still be able to find the relevant document with all the other information. A similar error in the summary minutes of a meeting which is intended for its participants could create some confusion, but once again it would not be considered very important as the participants know when the meeting took place. However, a wrong date in a translation of a notice of meeting would be very serious as it can result in delegates turning up on the wrong day. In this case, the same translation error, previously considered minor, would make the whole document unfit for its purpose.

\section{Quality-enhancing tools and procedures}\label{sec:hanzl:hanzl:3}

A number of tools and procedures are in place at the GSC's Translation Service to help translators achieve the required quality of their translations.

First, there are tools helping to ensure the necessary consistency of terminology, phraseology and style as well as to make the translation process more efficient. These include translation memory and other computer-aided translation tools, databases of translated documents (e.g. Eur-Lex), terminological databases (e.g. IATE) and other databases (e.g. lists of government members) as well as various style guides (\textit{Inter-institutional Style Guide} \citep{PublicationsOffice2011}, \textit{Manual of Precedents for Acts Established within the Council of the European Union} \citep{Council2010a}, web translation guidelines (e.g. \citealt{Council2016b}). These tools do not need to be discussed in detail here, as translation memories, databases and style guides are standard equipment of any major in-house translation department and are also discussed in the preceding chapters in this book (e.g. \citeauthor{Svoboda2013} on style guides).

Second, as it is common in translation practice, most translations are revised by a second pair of eyes. The GSC's Translation Service uses five levels of revision: (1) \textbf{thorough revision}, which is used only for the most important documents, such as the European Council conclusions and major political declarations or statements of its President, accession treaties, etc., and includes both bilingual revision and monolingual review; (2) \textbf{“standard” revision}, i.e. bilingual examination of the target language content against the source language content; (3) \textbf{light revision}, which combines monolingual review of the whole document and a bilingual revision of potentially problematic or most important parts; (4) \textbf{review}, i.e. monolingual examination of the target text; and (5) \textbf{optional revision}, where for reasons of efficiency no revision or review is carried out unless the translator asks for it\footnote{At the GSC's Translation Service, the terms \textit{revision} and \textit{review} are used as defined in the ISO Standard 17100 (\citeyear{ISO2015}). For a comparison of revision with the European Commission see \citet{Martin2007}.}.

Third, GSC’s translators may consult drafters if they are not sure about the correct meaning of a particular sentence in the original text. The translators’ questions are gathered centrally in order to avoid repetition, sent to relevant administrators whose replies are then shared between all the translators working on the relevant document by means of a Microsoft SharePoint platform on the corporate intranet. This practice helps to improve the quality of Council documents in general, because it makes it possible to spot and correct mistakes that may appear also in the originals. Furthermore, GSC’s translators may usually ask national experts for terminological advice and benefit from their expert knowledge already when working on a translation.

Specialization is another way that helps to ensure the necessary quality of GSC’s translations. Therefore, GSC’s translators have formed the so-called \textbf{functional groups}. Basically, there are four of them and they mirror the most important Council configurations — economy and finance, environment (which also includes agriculture and energy), foreign and security policy, justice and home affairs. Based on their education or areas of interest, translators are encouraged to join one of these groups, and subsequently to try to keep track of developments in their particular area. They also attend lectures to improve their knowledge in their field of specialization in general, as well as briefings on the most important specific legislative files that are going to be translated.

Similarly, the way documents are allocated for translation and revision can also contribute to the quality of their translation. As explained above, one given file can move backwards and forwards between the preparatory bodies of the Council and the Translation Service several times. Whenever possible, that particular file will be assigned to the same translators and revisers who know it from the previous rounds and can therefore deal with it better and faster. Obviously, this practice has its limits because at certain moments some translators would be overloaded while others left with nothing to do, but in terms of quality (and also job satisfaction) it has definitely proved its merits.

Finally, one more resource used at the GSC’s Translation Service deserves special attention. The Quality Controllers of the GSC’s Translation Service have drawn up and maintain a catalogue of Council documents which contains a taxonomy of documents translated in the GSC’s Translation Service as well as best practices recommended for the translation of each type of document \citep{Council2010b}. The translation recommendations follow from the assessment of political visibility and potential for legal and/or financial impact of each type of document, i.e. two variables which largely define the “fit for purpose” criterion.

\begin{figure}
\caption{Example of best practices for the European Council conclusions}
\label{fig:hanzl:2}

\fbox{\parbox{.9\textwidth}{
\textbf{{EUROPEAN COUNCIL CONCLUSIONS}}


\begin{itemize}
\item 
\textbf{{Political visibility}}


\begin{itemize}
\item 
{Very high}
\end{itemize}
\item 
\textbf{{Potential for legal / financial impact}}


\begin{itemize}
\item 
{Low}
\end{itemize}
\item 
\textbf{{Recommended level of revision}}


\begin{itemize}
\item 
{Thorough}
\end{itemize}
\item 
\textbf{{Minimum level of revision}}


\begin{itemize}
\item 
{Thorough}
\end{itemize}
\item 
\textbf{{Recommended best practices}}


\begin{itemize}
\item 
{(…) The members of the summit teams should, whenever possible, translate the guidelines for conclusions and preliminary drafts of the conclusions in the run-up to the summit; in any case all members of the summit team should read the draft conclusions before the summit and, where necessary, discuss the main translation issues (…).}
\end{itemize}
\end{itemize}
}~}
\end{figure}




\begin{figure}
\caption{Example of best practices for agendas}
\label{fig:key:3}

\fbox{\parbox{.9\textwidth}{

\textbf{{AGENDAS FOR THE COUNCIL / COREPER / CSA / PSC}} 


\begin{itemize}
\item 
\textbf{{Political visibility}}


\begin{itemize}
\item 
{Low}
\end{itemize}
\item 
\textbf{{Potential for legal / financial impact}}


\begin{itemize}
\item 
{Low}
\end{itemize}
\item 
\textbf{{Recommended level of revision}}


\begin{itemize}
\item 
{Optional}
\end{itemize}
\item 
\textbf{{Minimum level of revision}}


\begin{itemize}
\item 
{Optional}
\end{itemize}
\item 
\textbf{{Recommended best practices}}


\begin{itemize}
\item Date and place of the meeting should be double checked.\\
Where the reference document exists, the agenda item title should correspond to the title of the reference document, with no modifications or improvements. However, typos and serious grammar mistakes should be corrected.
\end{itemize}
\end{itemize}
}~}
\end{figure}

Altogether, the catalogue identifies over 25 types of documents\footnote{The exact number evolves over time. Apart from the two types mentioned in Figures 2 and 3, respectively, other document types include, for example, draft legislation at certain milestone stages, documents for adoption or discussion by the Council, Council minutes, declarations/statements by the High Representative or by the President of the European Council, appointments, manuals for use by national departments in Member States, speaking notes for the presidency, press releases, informative documents intended for the general public, such as brochures or various web content, etc.}, and, in addition to the best practices recommended generally, individual language units are allowed to add their own language-specific recommendations, if they consider it useful.

\section{Ex-post quality monitoring}\label{sec:hanzl:hanzl:5}

In 2006, a special report (9/2006) by the European Court of Auditors recommended that the GSC put in place both quantity and quality performance indicators for its translation work \citep{EuropeanCourt2006}. Consequently, results quality monitoring was introduced in 2009. It is a regular and systematic monitoring of representative samples of translations that leave the GSC’s Translation Service. Every week a random sample of 20 pages from at least 5 different documents is selected and the fitness for purpose of their translations into all languages is evaluated by Quality Controllers or delegated senior translators. All pages are equally likely to be chosen. The evaluated samples are discussed at weekly meetings of Quality Controllers, and this is a way of ensuring a certain degree of harmonization of criteria across different evaluators working in different languages. Dealing with problems detected varies from one case to another. It can involve sending a terminology note to a whole language unit, issuing joint requests for corrigenda, reviewing of best practices in place, and so on \citep[72]{Council2015}.

\largerpage
While the goal of the GSC's Translation Service is to have the proportion of pages considered “fit for purpose” as close as possible to 100\%, it is important to emphasize that the overall objective of results quality monitoring is to serve as a diagnostic tool providing warning of potential problems which can still be corrected, rather than to cause a fixation on a specific figure.

In addition to results quality monitoring, individual quality monitoring was introduced in 2013 to help assess the quality of work of individual translators. For each translator, at least 20 pages of translation and 15 pages of revision, coming from at least 5 different documents, are evaluated by their Quality Controller each year. Both results quality monitoring and individual quality monitoring are based on the same sets of criteria: linguistic (meaning, omission, terminology, grammar, style) and technical (styles, characters, typos, other). Reports from individual quality monitoring should document both strengths and weaknesses of the translation assessed across these categories, and the results of the evaluation are always discussed with the translator concerned \citep[72]{Council2015}. The main purpose of individual quality monitoring is to provide translators with systematic feedback.

\section{The special case of European Council conclusions}\label{sec:hanzl:hanzl:6}

The conclusions of the European Council are the most visible and politically sensitive document type that the GSC’s Translation Service produces. These conclusions are always immediately scrutinized by politicians, journalists and analysts and their implications are widely discussed in the media. Moreover, it is also a document which is translated completely under the responsibility of the GSC Translation Service — unlike, for example, legislative acts, where many other actors (translators from other institutions, lawyer-linguists, national experts) are involved and where the GSC translators are responsible only for a part of the bulk of translation work. For these reasons, the GSC's Translation Service has always handled the European Council conclusions with special care.

Such special care has been even enhanced since 2012, after one unfortunate incident. An omission of one word in the French translation of the Euro Area Summit Statement of 29 June 2012 reportedly caused a certain degree of confusion in communication between the German Chancellor and the French President \citep{Rousselin2012}. The disputed sentence in the English original reads as follows:

\begin{quote}
{When an effective single supervisory mechanism is established, involving the ECB, for banks in the euro 
area the ESM could, following a regular decision, have the possibility to recapitalize banks directly.}
\end{quote}

The same sentence in the French version of the statement originally read as follows:

\largerpage
\begin{quote}
{Lorsqu'un mécanisme de surveillance unique, auquel sera associée la BCE, aura été créé pour les banques de la zone euro, le MES pourrait, à la suite d'une décision ordinaire, avoir la possibilité de recapitaliser directement les banques.}

[Backtranslation into English: When a single supervisory mechanism is established, involving the ECB, for banks in the euro area the ESM could, following a regular decision, have the possibility to recapitalize banks directly.]
\end{quote}

The bone of contention was the issue of when the Eurozone's €500bn rescue fund, the European Stability Mechanism, would be able to pump cash directly into failing banks. The French version, where the information borne by the English word “effective” was missing, implied that the date in question would be 1 January 2013, when the structure of the single supervisory mechanism was due to be formally set up. However, the single supervisory mechanism became “effective”, i.e. actually started performing its tasks, only on 4 November 2014 \citep{ECB2014}; in the meantime, it was necessary, among other things, to carry out a comprehensive assessment of all the banks subjected to the single supervision so that the new supervisory mechanism could start with a clean slate and no skeletons in the banks’ cupboards. The amount at stake was reportedly worth €40bn.

As a result of this, a new procedure for “pre-reading” summit conclusions was established at the GSC's Translation Service at the end of 2012. It works like this:

The first draft of conclusions is translated at the GSC's Translation Service and sent to national capitals in the week before the summit. Before its translation, the terminology department of the GSC's Translation Service extracts important terms from the draft and provides useful terminological hints or recommendations, usually via updated entries in IATE. Moreover, the terminologists establish a list of documents to which the conclusions refer, so that translators can find them more quickly and easily. During their translation work, translators of the first draft send questions to quality coordinators whenever they encounter any ambiguity in the text or whenever they are not sure about the intended meaning of a particular sentence. After that, quality coordinators and terminologists meet to discuss the translation issues that emerged. Either they are able to solve them among themselves, or they send questions to the drafter, who, in turn, either provides the correct answer or helps to clarify the intended meaning, or in some cases redrafts the problematic parts in the following version of the conclusions.

Furthermore, one day ahead of the summit, before the last pre-summit draft is sent for translation, the translators who are going to work on the summit team meet with the relevant administrator who informs them of the expected course of the summit, explains which parts of the conclusions are the most contentious and why, and also answers additional questions that may have arisen in the meantime.

The final text agreed during the summit is, however, completely in the hands of the two or three translators (per language unit) working on the summit team. Here the deadline is extremely short, so there is no more time for consultations. Fortunately, the very last version is usually not much different from the penultimate version. At this final stage, focus is paramount, because here the task is to incorporate all the changes, no matter how small, in the correct place in the text as quickly as possible.

Generally, the pre-reading of the European Council conclusions has been helping to improve both the quality of the original text — where errors and unintended ambiguities can be spotted and corrected at an early stage — and the quality of its translations — where uniform interpretation and the use of correct terminology are enhanced. 

\section{Conclusion}\label{sec:hanzl:hanzl:7}

The aim of this chapter has been to illustrate how the GSC’s Translation Service manages the quality of the translations it produces. Its approach is a pragmatic one, which takes into account the importance of individual documents and the needs of their users. Being aware that anybody can make a mistake, the GSC’s Translation Service has set up tools and procedures to minimize their occurrence, or practical impact if they happen.

The quality-enhancing tools and procedures at the GSC’s Translation Service include the use of style guides, computer-aided translation tools and various databases to ensure the necessary consistency of terminology, phraseology and style both within and across documents. Translations are generally revised by a second pair of eyes, and for this purpose five levels of revision thoroughness have been defined and are applied depending on the type and importance of a particular document. The quality of translations is further enhanced by a possibility for translators to communicate with drafters of the originals, by specialization of translators as well as by allocation of documents for translation and/or revision based on their involvement in the work on previous versions of the same file. Last but not least, a taxonomy of documents translated in the GSC’s Translation Service has been compiled and provides, for each type of document, best practices recommended for its translation and revision. Special care, including a collective pre-reading of the original text, centralized terminological research and a meeting with the drafter, is dedicated to the GSC Translation Service’s hallmark product — the European Council conclusions.

The quality of translations produced at the GSC’s Translation Service is systematically monitored and evaluated. To this end, a tool to provide qualitative performance indicators has been introduced. We are not aware of any other large translation organization which would be monitoring the quality of its output by means of systematic random sampling.

 
% \section*{Abbreviations}
% \section*{Acknowledgements}

\sloppy
\printbibliography[heading=subbibliography,notkeyword=this] 
\end{document}