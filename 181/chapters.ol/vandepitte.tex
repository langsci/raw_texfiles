\documentclass[output=paper]{langsci/langscibook} 
% \ChapterDOI{} %will be filled in at production

\author{Sonia Vandepitte\affiliation{Ghent University}}
\title{Translation product quality: A conceptual analysis}
% \shorttitlerunninghead{If the normal title is too long in the page headers}
\abstract{Against a background in which both the translation product and translation process are briefly described as objects of quality assessment, this chapter presents an analysis of the concept of translation quality assessment focussing on the translation product. The following features will be presented as parameters along which product quality assessment practices in institutions can be described: the purpose of translation quality assessment, the criteria applied in the assessment, combined with their scaling and weighting, the translation quality levels aimed at, and the quality assessors involved. The characteristics will be illustrated by the translation quality assessment as applied in one Belgian institution. It is hoped that the analysis will lead to a fuller and deeper understanding of a translation's quality.}
\maketitle

\begin{document}


\section{The object of translation quality assessment: translation product and translation process}\label{sec:vandepitte:1}

In June 2015, a translation error delayed the renovation of Brussels opera hall De Munt / La Monnaie: planned renovations were postponed by four months. They should have started in June 2015, but could not because of an error in the translation that had legal consequences. In the public procurement, the Dutch phrase \textit{scenische werken} (`scene works') was translated as \textit{des travaux scénographiques} (`scenographic works'). In French, that description apparently also potentially has an artistic meaning and can be interpreted as meaning that the opera hall could employ a renovation company for artistic purposes. One consequence was that De Munt productions were to be seen at other locations in the city, but another consequence was also that a fair number of season ticket holders preferred to skip a season. Avoiding such – and any other – textual translation errors is most important to organizations, especially when it comes to sensitive documents or high risk communication and will be the main topic of this chapter. 


However, before I narrow down the scope to the translation product, another potential object of a quality assessment in the translation environment needs to be mentioned, i.e. the translation process. Let us first look at the following case, exemplified by a traffic sign in Swansea that revealed problems of a non-textual nature as reported by the BBC \citep{BBCNews2008}. While English-speaking motorists may well have understood the sign saying \textit{No entry for heavy goods vehicles. Residential site only}, monolingual Welsh lorry drivers must have been at a loss when they read \textit{Nid wyf yn y swyddfa ar hyn o bryd. Anfonwch unrhyw waith i'w gyfieithu}, which translates as `I am not in the office at the moment. Please send any work to be translated.' Here, the translation error did not result from a translator making an error during the micro-process of translating, but it resulted from poor translation management skills and a poor translation quality assurance process. The error was mentioned on \citeauthor{BBCNews2008}, Radio 4, on 31 October 2008: the sign was posted by Swansea Council, obviously commissioned by someone who did not know Welsh and therefore did not apply the required translation management skills. In other words, the macro-process of providing for the appropriate people to deal with the production of the translation failed, resulting in an inappropriate end product.

The difference between the two potential objects of a translation quality assessment, the translation product and the translation process, may also be illustrated by means of the following question: does the quality assessment concern the translation product only or does it also include aspects from the translation cognitive or textual process or the translation service? In other words, is it the translated text only that is assessed or does the assessment also include the way in which the translator(s) produced the text, or the ``core processes, resources, and other aspects necessary for the delivery of a quality translation service that meets applicable specifications'' \citep{ISO2015} or even the manner in which the translation is brought to the recipient? 

Such differences in quality assessment need to be identified clearly in the assessment as the first step in a translation quality assessment exercise: Is an institution interested in improving its procedures for producing higher quality translations or does it want to raise the quality of the final translations? Does an institution wish to identify the best translation produced by different people or does it wish to determine which are the best company processes that will guarantee high-quality translations?

The borderline between product and process issues is, however, not straightforward. Some characteristics, such as ``punctuality'', ``proactivity'', or ``initiative in upgrading the terminology'', can, indeed, only be related to one object, in case the processes that are part of the service provision. Other matters, however, like, for instance, ``compliance with a style guide''\footnote{On the topic of translation manuals and style guides cf. Svoboda in this volume.}, could both apply to a translator's acts, and be considered a process issue, and to the translation itself, and be regarded as a product issue. In addition, textual errors — whichever way they have been produced — can always be related to some process irregularity whether at the micro-level or at the macro-level and the translation product quality will invariably be relatable to the quality of the process. Hence, a study of the former may also reveal information about the latter, and the concept of \textit{translation quality} may often include references to issues of both products and processes, leaving it up to the reader or listener to disambiguate the phrase in the context in which it appears. In what follows, the scope of this chapter will be narrowed down to that of the quality of translated texts; for descriptions of quality assessment of translation processes, the reader is referred to, for instance, \citet{Mertin2006} or \citet{Drugan2013}. 

In order to avoid any further conceptual misunderstandings, it is also useful to clarify the difference between the two interpretations of the abbreviation \textit{TQA}, since both are equally important to both the industry, including institutions, and research. Although the occurrence of the abbreviation \textit{TQA} is fairly frequent, its meaning is not stable and may vary depending on the user. For some, it means translation quality \textit{assessment}, the topic of the present chapter. For others, however, it refers to translation quality \textit{assurance,} and is related to the translation process. The former interpretation establishes a link between TQA and the act of pronouncing a quality judgement about the translated text (like \textit{travaux scénographiques}), while the latter sees TQA as the provision in a company’s activities to take care of quality or its implementation, application and management of quality control. The industry group TAUS (Translation Automation User Society), which arguably may not reflect the view of translation scholars but aims at providing data services to both buyers and providers of language and translation services, defined translation quality assurance in their translation technology report as “a combination of technology and processes to prevent errors from creeping into translation projects” or the set of ``procedures in the whole translation process (from initial order to final delivery and file closure)'' \citep[22]{TAUS2013} in order to have the translation comply with standards that are recognized, such as the European norm \citet{EN2006} or the international standard \citet{ISO2015}. Hence, such a set of procedures consists of an ``ordered set of steps to guarantee quality'' \citep[22]{TAUS2013} both prior, during, and after the translation, and even after the delivery of the translation. Translation quality assurance includes, for instance, the ``decision process in translator assignment: which translator(s) are best match to the task, factoring in skill level, prior QA scores, availability and domain of expertise'' \citep[22]{TAUS2013}. Consequently, translation quality assurance both precedes and follows translation quality assessment. The assessment of translation quality, however, is in itself also ambiguous, since the term \textit{translation} may, as is well-known, refer to either the product of translation or the process, whether the latter is to be found at a micro-level – in the translator's brain – or at a macro-level – all the other processes whether they are part of translation service provision, initiate a translation to be produced or whether they are set in motion by a particular translation in a particular community. 

While the meaning of TQA may vary rather substantially, the concept of \textit{quality} itself seems to be more stable and commonly agreed upon. Princeton University's Wordnet, whose large English lexical database also interlinks the senses of the words, defines \textit{quality} as ``an essential and distinguishing attribute of something or someone'' (\citeauthor{WordNet2010} 3.1, 2017). Some other definitions are suggested throughout this book, yet they do not diverge significantly. However, the word \textit{quality} is mostly used in the sense of 'calibre', or, as the ``degree or grade of excellence or worth'' (WordNet 3.1 2017). Although this is a clear matter, there have been many debates about what exactly constitutes the 'calibre' of a text and many contributions to the topic have been produced by translation scholars, such as \citet{Lauscher2000,Maier2000,Lee-Jahnke2001,Colina2009,VanPoel2007} and \citet{Depraetere2011}. 

In spite of the terminological confusion, the above-mentioned scholars' findings and discussions were fruitful in that they have also brought a set of characteristics of translation quality assessment to the foreground and the present chapter will present the most prominent ones, discussing their conceptualization. This will be based on both the literature on translation quality and on translation quality practices in translation training, where teachers are experienced assessors of translations on a regular if not on a daily basis. Four main parameters have been distinguished which play a role in the assessment of the quality of a translation product. They are: purpose or functionality of the translation quality assessment, the translation quality level aimed at, the criteria including the weighting and scaling of the criteria, and actor performing the assessment. 

In order to illustrate the parameters, data will be presented that were gathered in a pilot case study about the translation quality assessment of one institution as reported on in 2008. Following \citet[493–494]{Schäffner2014} in taking the label ‘institutional translation’ to refer to translating in or for a specific organisation, the translation practices by the Service central de traduction allemande (\citeauthor{SCTAn.d.}) may well be considered as institutional translation. The \citeauthor{SCTA2015} is the central translation service provider for German in Belgium since 1976, situated in Malmedy and part of the Service publique fédérale Intérieur (abbreviated as SPF Intérieur, the Federal Public Service for Domestic Affairs), employing about 25 people for translation and coordination tasks. They translate federal laws adopted by the Belgian parliament into German as well as newsletters of the SPF Intérieur. They present their translations and terminological work in two databases, TRADUCTIONS and SEMAMDY respectively, both of which can be consulted on their website \citep{SCTAn.d.}.

\section{Purpose of the translation quality assessment}\label{sec:vandepitte:2}

The first parameter in a product translation quality assessment exercise is closely related to that of the object, i.e. it is the purpose of the translation product quality assessment. Although many translations may have been produced to an audience that is knowledgeable of the source language\footnote{An example is the publication of scholarly materials in the target language following language policies that protect languages with a lesser diffusion.}, the purpose of many other translations is to be read — and in the context of institutional translation also to be relied upon — by somebody who is assumed not to know the language of a source text. In the same way that a piece of clothing is produced to be worn by somebody and can be tested to fulfil different purposes — will the piece of clothing make people warm enough in freezing temperatures or protect people from a blazing sun, will it support a certain part of the human body so the wearer does not suffer, will it make someone sexually attractive? — the purpose of a translation quality assessment may vary considerably. Just like a piece of clothing can be tested, the quality of a translation can be tested with different intents: will the reader be attracted to buy a certain product, will the reader know what the important contents are of a product that has been bought, will the reader use the company's new machine or tool safely and efficiently, will the reader know what to do in certain life-threatening situations, will the reader understand the essential writing qualities of a foreign author, does a teacher find appropriate elements in students' translations so that their translation competence can be developed most smoothly, will a teacher allow a student to enter the market with a translation degree, or will a company hire an employee? There is even one instance in which the translation assessment itself becomes a central point of focus and is even made public. This is the case with literary translation criticism expressed in reviews in newspapers, magazines or journals which may help readers decide whether they find the translation worth reading.

Other purpose features can be related to elements from the translation situation itself. As a resource centre for language and translation industries worldwide TAUS, for instance, identifies further criteria. They identify the purpose of a translation as to whether it will be used as audio/video material, for marketing or online help purposes, as training material or user documentation, for a user interface or as website content; they also distinguish between translations to be used in a regulated industry versus those that operate within those industries that are not regulated, whether the translation is to be read by company staff only or not, and whether the communication channel fulfills a Business-to-Business, Business-to-Consumer or Consumer-to-Consumer purpose \citep{TAUS2013}.

When requested for more information on their quality assessment procedure, \citeauthor{SCTAn.d.} readily produced a workflow for their translation quality procedures with \textbf{two} revision and correction stages and their internal and external customer satisfaction surveys. Neither, however, revealed any explicit statements about the purpose of their translation product quality assessment. It is clear that they assume that any person interested will understand why the translation product needs to be assessed and revised twice. 

\section{Translation quality levels}\label{sec:vandepitte:3}

Setting a certain translation quality level has the aim to allow a person to judge a translation unacceptable if it turns out to have lower quality. While many professional translators proclaimed and still proclaim that they aim at the highest possible level of translation quality, the machine translation industry seems to have changed market expectations profoundly. At present, there is a wide variety among translation level distinctions both in the industry and education, going from just one level to classifications which yield five and even considerably more different levels of translation quality.

A broad distinction has, for instance, been made by \citet{Williams2004} between two main different quality levels in the industry. ``\textit{Revisable'' quality} (after linguistic quality inspection, LQI) is the quality achieved after \textit{proofreading}, i.e. after errors in translated texts have been identified and corrected in the areas of terminology, sentence level features such as spelling, punctuation, grammar (syntax, morphology), lexicon, textual level features like terminological consistency and contextual features like compliance with style guide. \textit{''Publishable'' quality,} in contrast, is produced after comparing the translation with source text, i.e. after identification and correction of mistranslations that are due to misinterpretation by unwarranted omissions, additions or changes. After such source text alignment, compliance with domain register and phraseology, stylistic consistency, and accuracy, usability and readability with regard to the specific target audience/end-user are assured. In Europe, this is often called revision / editing / review, although the terminology as used in the \citet{ISO2015} standard is slightly different and alignment does not need to precede review.

While referring to market practices in France, Gouadec distinguishes three levels: ``(1) rough-cut, (2) fit-for-delivery (but still requiring minor improvements or not yet fit for its broadcast medium), and (3) fit-for-broadcast translation (accurate, efficient, and ergonomic)'', recognizing the possibility of an intervening ``'fit-for-revision’ grade to describe translations that can be revised within a reasonable time at a reasonable cost'' \citep{Gouadec2010}. The distinction between the fit-for-delivery translation and the fit-for-revision translation seems vague, however, the former being formulated in terms of potential use and the latter in terms of time. The combination of those two different parameters allows for overlapping categories.


In education, students are ideally assessed like professionals. However, such expectations may well be unrealistic since students have not been able to build expertise out the years. In order to provide a fair system, different levels can be set up recognizing the pedagogical aims of the course and the items discussed in the course (see also \citealt{Vandepitteforthcoming}). Evaluation grids with various levels can be used to communicate criteria to students. A fair number of translation training programmes applies grids with different levels for academic purposes, but the EAGLES project at University of Geneva is an example in which four different quality levels in the industry have been recognized. Their \textit{raw translations} convey the central meaning of the original text, but there will be grammatical errors and misspellings. Scientific abstracts often take that form. Secondly, the quality level of a \textit{normal quality translation} is slightly higher since there are no grammatical errors. However, some passages may sound awkward. A typical example would be the translation of a technical manual. The next level of \textit{extra-quality translation} means that the translation is also idiomatic and culturally assimilated to the target culture. Translations of advertisement brochure or literature would belong to this category. Finally, an \textit{adaptation of an original text} does not need to correspond to the original and also omissions are acceptable \citep{KingEtAl1995}. With its reference to different text types, this scaling is not related to any pedagogical aims but it introduces students to the translation jobs that trainees may well be liable to translate in their future professional lives.



In the case of \citeauthor{SCTA2015}'s quality practice, neither their customer survey (\figref{fig:key:1} in section 4) nor description of their quality assessment process mentions any explicit differentiation of quality levels. Nevertheless, there is a symbolic colouring of the three bands into which the points on the scale have been grouped in \figref{fig:key:1} – with red for grades 1-4, orange for grades 5-7 and green for grades 8-10 – which may actually reveal the degree of acceptability of a grade. From this, it could also be assumed that their services will not aim at any levels of translation product quality lower than top quality. 


\section{Translation quality assessment criteria}\label{sec:vandepitte:4}

At international level, agreement has been reached by the ISO on the following items that are also relevant in the quality assessment of a translation product: codes and representations of languages and countries (\citeauthor{ISO2016a}, \citeauthor{ISO2013}, \citeauthor{ISO2016b}, \citeauthor{ISO2006}),
specialized vocabulary in the fields of micrographics, laboratory apparatus, heat treatments, shipbuilding, and so on, document formats \citeauthor{ISO2004}, 
information technology (\citeauthor{ISO/IEC10646}, \citeauthor{ISO1991}),
and computer applications in terminology (\citeauthor{ISO2003}). 

These standards are all related to either vocabulary and terminology or formatting and technology. However, those areas are not the sole criteria to be assessed, and, as it happens, the translation market is rife with varying views on the number and the selection of other criteria that need to be taken into account in a translation assessment exercise. Many organizations, whether private, such as Lionbridge or Sajan, or public, have set up their own criteria, including a subset from the following set of criteria: faithfulness to the source text, grammar, syntax, spelling, punctuation, vocabulary, style, register, coherence, cohesion, and fluency. 

Most of these are construed as \textit{error} categories. An approach that does not focus on errors is \citeauthor{Gouadec2010}'s description of four domains from which criteria can be taken to describe the translated text: the linguistic-stylistic-rhetorical-communicative domain, the factual-technical-semantic-cultural domain, the functional-ergonomic domain, and, finally, a 'domain', in which the translation is compared to the source text, taking into account any linguistic or cultural gaps and any intended changes in medium or audience ``even to the point that there remains very little parallelism between the original and the end product of the translating process'' \citep{Gouadec2010}. Noteworthy, the lack of similarity between source and target texts reveal how broad \citeauthor{Gouadec2010}'s approach is to be interpreted, since institutional translation would hardly ever find such differences acceptable, except perhaps for shorter, illustrative passages.

\subsection{Scaling of the criteria}\label{ref:4.1}

Like any other type of assessment, the quality of a translation in terms of a particular criterion can be decided on by assigning it a certain position on a scale going from low to high quality. Such scales would allow for comparison of different translations with each other. However, quality grades are not often made explicit as such in specifications. They mainly seem to appear in education, where translations need to be marked and marks will produce a ranking among students. Obviously, standards like the \citeauthor{ISO2015}, which are process-oriented, will not include scaling either. Nevertheless, most assessment criteria do allow for grading, and some companies' and/or institutions' assessments are operating with systems of scaled grades for quality criteria (see e.g. \citeauthor{Strandvik2015}, this volume), some of which are even fairly complex. 



A simple example can be seen in the satisfaction survey distributed by the \citet{SCTA2015}. \figref{fig:key:1} (\sectref{sec:vandepitte:6}) shows how the quality has been given 10 points on a scale which allows customers to assess according to a system that they have been used to in school. The combination of three criteria in one question, however, does not allow them to make distinctions and may result in average quality scores that will not reveal any problematic areas.


\subsection{Weighting of the criteria}\label{sec:vandepitte:4.2}

Finally, organizations also determine the value of each criterion vis-à-vis the other criteria. Depending on the settings of other parameters, and, in particular, that of the purpose of the assessment, certain aspects will carry more weight in the assessment than others. Such relative importance of the criteria components to each other can be visualized in, for example, a pie-chart.

\section{Actors involved in the translation quality assessment}\label{sec:vandepitte:5}

The final aspect to be discussed is the actors involved in the translation quality assessment. Although arguably actors may be seen as a major aspect in a \textit{process}, their impact on the assessment of the translation product is not to be underestimated and reans them a place in this survey of translation product criteria, too. Depending on the purpose of the quality assessment, certain actors will need to be involved in carrying out the translation quality assessment. As may already have become clear from the preceding paragraphs, various actors may be involved in TQA. 

On the one hand, there are people that set standards for translation quality, and, on the other hand, there are people that carry out the tests. In many cases, the two acts are carried out by the same person. There are \textit{standard-setters} at individual level (teachers in translation training, for instance) or at organizational level. In the translation context, the latter usually operate at international level. One such standard-setter for translation services was the European Committee for Standardization CEN in collaboration with the European Union of Associations of Translation Agencies, producing the European Standard \citet{EN2006}. This standard was later replaced by the \citet{ISO2015} by the International Organization for Standardization (\citeauthor{ISO2015}), the worldwide federation of national standards bodies. The latter organization carries out preparatory work in technical committees, in which each member body interested in a particular committee can be represented. Sometimes other international organizations, governmental and non-governmental, are also involved. Draft International Standards adopted by the technical committees are approved when three quarters of the member bodies agree. Some more private initiatives are also taken: an example is SAE International, a global association of engineers and technical experts in the aerospace, automotive and commercial-vehicle industries. They produced a \textit{translation quality metric} called SAE J2450\_201608, which is ``applicable to translations of automotive service information into any target language. The metric may be applied regardless of the source language or the method of translation (i.e., human translation, computer assisted translation or machine translation)'' (SAE, 2016\footnote{\url{http://standards.sae.org/j2450_201608}}), except for texts, the style of which is also important (e.g., owner's manuals or marketing literature). The metric, which acquired the status of a standard in the first decade of the twenty-first century, is assumed to provide for a more objective assessment of translations in the automotive industry. 

Testing the quality of a translation can be carried out by ``translators, executives, quality managers, heads of departments, project managers, clients, editors, revisers, terminologists, software engineers and sales and marketing staff'' alike \citep[3]{Drugan2013}. But other actors may also be involved: translation scholars have their own individual systems, sometimes moderated by some element of intersubjectivity when a few more testers are involved in the rating of translations or when audiences of subtitles are consulted \citep{Delia2014}. End-users as translation audiences are also sometimes consulted in the commercial world: buyers of products may bring to bear on the quality of a translation, by way of customer satisfaction surveys, usability testing (for instructive types of text, for instance) and \citeauthor{TAUS2015}'s Dynamic Quality Framework (DQF). 

Customer satisfaction surveys are actually also employed by the \citeauthor{SCTA2015}. \figref{fig:key:1} below shows their question about the translation product quality and culd be translated into English as Are the translations delivered faithful, of good readability and coherent?:

\begin{figure}
\caption{\label{fig:key:1} Extract from \citeauthor{SCTA2015} internal/external customer satisfaction survey \citep{SCTA2015}}
\includegraphics[width=\textwidth]{figures/a4VANDEPITTEfinal-img1.png}
\end{figure}

  The question put to \citeauthor{SCTA2015} customers immediately reveals its three main criteria: adequacy (\textit{fiables}), and two features of acceptability, i.e. readability (\textit{d'une bonne lisibilité}) and coherence (\textit{cohérentes}). 

The \citeauthor{TAUS2017dashboard} DQF was first established in 2012, undergoes regular updates and allows buyers of translations to decide on the type of quality test necessary to apply to the translation product which they have bought. The DQF requests its users to decide on the settings of parameters. The parameters available are content category, regulated industry, internal communication and channel, and on the basis of the user's settings, the DQF will automatically suggest one evaluation model among various evaluation models (error typology, adequacy/fluency and readability evaluation, for instance) and it will also perform an automated evaluation metric \citep{TAUS2017dashboard}. 

In order to avoid what the industry considers to be costly human labour in the translation quality assessment process, it is also looking for automated testing by means of software tools. A comparison of the performance of such tools can be found in \citet{Debove2011}, and the formal quality check performance of one such tool (QA Distiller) was tested by \citet{Depraetere2011} on Spanish into French student translations by comparing it with human measurements, investigating its degree of indicativeness of overall quality.

\section{Summary and conclusion}\label{sec:vandepitte:6}

Summing up the discussion above, the parameters in translation quality assessment can be presented as in \tabref{tab:vandepitte:1}. 

\begin{table}
\footnotesize
\begin{tabularx}{\textwidth}{Q}
\lsptoprule
{\textbf{Object} }\\
 process  \\
 product  \\
 service  \\
 
 \tablevspace 
{\textbf{Purpose} }\\
 sell the translation  \\
 sell another product  \\
 recommend (or not) a translation to the general public  \\
 hire an employee  \\
 grade a student  \\
 develop a student's translation competences, ...  \\
 
 
 \tablevspace 
{\textbf{Actor} }\\
 producer of translation quality: translator, whether a professional, an amateur or a student  \\
 producer of certain translation norms: ISO (EN and national norm-giving institutions)  \\
 quality testers: client / commissioner / employer in the professional field who expects a certain standard, trainer who expects a certain standard from a student, researcher into translation quality...  \\
 
 
 \tablevspace 
{\textbf{TQA-level aimed at} }\\
 {1-level: so-called 100\% quality}\\
 {2-level: e.g. revisable versus publishable quality}\\
 {3-level: e.g. green, orange and red bands}\\
 {more than 3 levels}\\
 
 
 \tablevspace 
{\textbf{Criteria involved in the product assessment}}\\
 {alignment with the source text}\\
 {style}\\
 {terminology}\\
%  terminology  \\
 grammar  \\
 syntax  \\
 spelling  \\
 punctuation  \\
 vocabulary  \\
 register  \\
 coherence  \\
 cohesion  \\
 fluency, ...  \\
 
 
 \tablevspace 
{\textbf{Scaling of criteria}}  \\
 measures used to identify the quality of the translation in terms of each criterion  \\
{\textbf{Weighting of criteria}}  \\
 relative importance of the criteria  \\
\lspbottomrule
\end{tabularx}
\caption{\label{tab:vandepitte:1}. Parameters in translation quality assessment}
\end{table}

This chapter has presented the following factors of translation quality assessment: its object, its purpose, the actors involved, the TQA-level aimed at, the criteria relevant to theassessment, their scaling and any weighting. 

While this summary may not include all potential variants of parameter settings, nor even all parameters themselves, the survey aims at presenting a clearer idea of the issue of translation product quality assessment and facilitating discussion of the topic among different stakeholders. 

Whether this survey is practically applicable to all types of translation products is something which future practice only will tell. The set of parameters as outlined above may certainly be helpful in cases where the assessment of a translation product turns out to be of the utmost importance, which is typical of institutional translation. 

The description of \citeauthor{SCTA2015}'s translation product assessment has shown, however, that its public statements do not contain much explicit information. In order to improve the visibility of the work involved in translation, however, more informative statements about the purpose of the assessment, the criteria assessed or the quality level aimed at would be welcome. 

% \section*{Abbreviations}
% \section*{Acknowledgements}
\clearpage 
\sloppy
\printbibliography[heading=subbibliography,notkeyword=this] 
\end{document}