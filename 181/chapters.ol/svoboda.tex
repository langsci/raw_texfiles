\documentclass[output=paper]{langsci/langscibook} 
% \ChapterDOI{} %will be filled in at production

\author{Tomáš Svoboda\affiliation{Charles University, Prague}}
\title{Translation manuals and style guides as quality assurance indicators: The case of the European Commission’s Directorate-General for Translation}
\shorttitlerunninghead{Translation manuals and style guides as quality assurance indicators}
\abstract{The aim of this chapter is to verify the assumption that institutional translation on the supranational level is, by definition, concerned primarily with terminology, style guides, it is standardized and its quality aspect is governed by rules (cf. \citealt{Koskinen2008}, \citealt{Schäffner2014}, it will concentrate on translation manuals and style guides, since extensive studies on this topic seem to have been missing from academic research. To fill this gap as regards inquiries into the workings of one particular (EU) institution, this chapter presents the results of research into translation manuals and style guides used by and within the European Commission’s Directorate-General for Translation (DGT). The DGT on-line collection of guidelines (referred to as the Resources website here), which primarily offers materials to DGT contractors, represents arguably the most extensive and most complex translation resource ever compiled. The present research is based on empirical data: as of the time of the study (the first half of 2017), a total of 793 links to individual translation manuals and style guides were included in a research corpus encompassing all the 24 official languages of the EU. The information was surveyed using a blend of quantitative and qualitative approaches. As for the results, the extensiveness of the DGT reference material could be shown together with its linkages to the translation quality aspect, whether these are explicit or implicit. As regards the structure of the resources, an overall top-down standardization approach could be proven, although, at the same time, they show a certain degree of variation. The following areas were identified as being the crucial requirements DGT has vis-à-vis its contractors: references to EU institutions and DGT departments, binding terminology resources and the Interinstitutional Style Guide. 
}
\maketitle

\begin{document}

\section{Introduction}\label{sec:svoboda:1} 

It has been shown for the European Commission’s Directorate-General for Translation (DGT) that there is enormous divergence among language departments both in the topics covered by translation manuals (TMs) and style guides (SGs) as well as in the level of detail of such resources.\footnote{Topics range from terminology to workflow rules and SG/TM volumes range from one-page materials to a manifold of that. For example, the German in-house style guide featured 690 pages in mid-2017; it was 1,009 pages long in 2013, cf. \citet{Svoboda2013}.} Research so far (cf. \sectref{sec:svoboda:2} below) has analyzed just a handful of individual guidelines and there has been no inquiry into the overall pool of the SGs and TMs used by and within DGT – in fact, Felipe \citet[56]{FelipeBoto2009} have made an explicit remark (“we must lament”) that little use is made of the material provided by the translation services of EU institutions. To gain a clearer understanding of the wealth of “house memory” in DGT (which, according to DGT’s own information and relevant literature, is the largest translation department in an institutional setting in the world) and of its relation to quality management, a study of the DGT Resource website entitled “Guidelines for translation contractors”\footnote{Available from: https://ec.europa.eu/info/resources-partners/translation-and-drafting-resources/guidelines-translation-contractors\_en (Accessed 2017-8-30).} has been performed. It has yielded a comprehensive account as to the structure (logical organization) of how the SGs/TMs are presented, the types and nature of the resources and their linkage to the overall quality assurance goal of the service. The current research findings can be of relevance both to the field of institutional translation within Translation Studies (as it contributes to the understanding of the specificities of the former within the discipline more generally) and to practice, since they offer insights into a set of materials that are aimed at translation services providers/translation contractors.

\section{State of the art in studying style guides and translation manuals}\label{sec:svoboda:2}

The notion of rules is constitutive to the concept of institutional translation (cf. e.g. \citealt[18]{Koskinen2008}, \citealt[343]{Halverson2008}, \citealt[142]{Kang2009} and in a less straight-forward way \citealt[159]{Becker-Mrotzek1990}). Various types of “rules” have been described or mentioned in the literature so far.\footnote{E.g.: established procedures, explicit principles, glossaries, guidance, (written and unwritten) guidelines, guides, guiding principles, institutional ‘group mind’, institutional doctrines, instructions, manuals, norms, official guidance on (translation) policy, organized procedures, style guides, terminology requirements, translator’s handbooks, algorithms (e.g. automatic TM analysis, pre-translation), codes of practice, (EU) culture, customs, etc.} In an interesting account, Ian Mason’s seminal text \citep{Mason2004[2003]} lists the reasons for issuing guidelines to govern the translation practice: they are in place either because they have been issued to the translators by the institution (e.g. glossaries, style guides, codes of practice) or as a result of a “development which grows over a period of years out of shared experience, the need to find common approaches to recurring problems or through advice and training offered to new employees” (\citeyear[470]{Mason2004[2003]}). 

One of very few explicit linkages between translation (product/process) quality and SGs/TMs has been suggested by \citeauthor{Sosoni2011} (in the context of translator training and the future roles of translation students as contractors for EU institutions): “[…] translation quality is inextricably related to the reliable implementation of the guidelines set out by the EU institutions” (\citeyear[100]{Sosoni2011}). Another overlap of the two concepts of SGs/TMs and translation quality can be found in a document issued by the European Commission’s \citet[17]{DGT2012}. This publication sees the ultimate role of style guides at the pre-translation stage: “Quality in translation in the stage before translation corresponds to prevention of poor quality and includes recruitment, training, terminology, style guides, etc.; during translation, quality is a matter of choosing the right translator for the job…”. This view obviously approaches quality from the point of view of quality assurance (QA). Consequently, by the statement “Quality in translation […] includes […] style guides” it means, most likely, that ensuring product quality involves preliminary investment in QA tools, such as SGs. Otherwise, a look at the SGs/TMs made available by the Commission’s own DGT reveals that they cover everything from how to deal best with searches, how to perform editing work, how to send translations and, eventually, how the billing procedure works, etc., which means that all the phases of the translation process are covered by SGs/TMs\footnote{For a more detailed discussion, see \citealt{Svoboda2008}.}, not just the preliminary stage.

An analysis of a particular style guide, specifically the EU \textit{Interinstitutional Style Guide}, may be found in \citet{Svoboda2013}. The paper also examines the work-flow at the DGT, linking it with the question of what style guides and translation guidelines are pertinent to what stages of the translation process. It also presents results of a brief analysis of a sample of guidelines for the DGT translation contractors. Apart from the study results, the paper also gives an extended bibliography overview on the topic of SGs and TMs.

Whereas it is generally assumed that guidelines are an organic part of a translating institution’s practice, this assumption calls for refinement when different types of institutions are compared. For example, it has been established in national contexts (in a comparison of EU countries’ central government bodies) that “best practice is hardly ever recorded in translation manuals and house style codes are rather an exception” \citep{Svoboda2017forthcoming}. Such findings contradict the common notion of quality assurance in institutional settings, since it was coined based on the top-level (supranational and/or international) translating institutions.

Typically, the aim of guidelines is to improve the product quality of the delivered texts and ultimately to save time and resources on the part of the translating institution, including avoiding extensive revision work. The guiding principle behind a manual is not always explicitly stated in the SGs/TMs themselves. As one of the exceptions, the EU \textit{Guide for external translators}~makes the following statement:

Its main aim is to provide the contractors with practical information to help them with the translation work assigned by the DGT and to facilitate the communication between the contractors and the Commission (DGT’s Language departments and External Translation Unit), by laying down certain rules for standardization (word-processing software, layout) and for the use of information technology. \citep[4]{DGT2008}.

Another study authored by the European Commission’s \citet[175–186]{DGT2013} gives results of an informative survey on the availability/unavailability of SGs in the drafting process in various international organizations. Although this material is not primarily aimed at translators, some important observations of a more widely applicable nature are made here as regards the extent to which guidelines are used: “Guidelines may not solve all possible problems […] Instructions tend to overlap […] [There is] concern […] about the streamlining of instructions issued. A constant reminder […] is that too many instructions can fail in their purpose, and simply be ignored by drafters” (\citeyear[175]{DGT2013}).

\largerpage
Kaisa \citet{Koskinen2011} refers to guidelines as she links them with standardization and other frequently occurring features of institutional translation, i.e. its collective and anonymous nature. Besides guidelines, she adds, contemporary institutions use databases, term banks and CAT tools (cf. \citealt[58]{Koskinen2011}). She concludes her article by stating that “customs and […] guidelines [related to institutions and institutional translation] are in no way uniform [among various types of institutions …]. Understanding institutional translation […] requires […] detailed case studies of different institutional contexts. This research has only just begun in Translation Studies.” \citealt[59]{Koskinen2011} The author of this chapter shares \citeauthor{Koskinen2011}’s view and presents his research as a contribution to fill these blind spots, for one particular institution (\citeauthor{DGT2008}).

\section{Methodology}\label{sec:svoboda:methodology}

In order to obtain necessary data for describing the DGT Resource website in a comprehensive manner as regards its surface representation\footnote{Individual link text was summarized, i.e. only text appearing as the (active) hyperlink, not accompanying (inactive) description texts. Owing to limits to this research, the linked resources were neither actually retrieved, nor analyzed in terms of their content.}, both quantitative and qualitative methods were applied, with a focus on the former.

Using a quantitative approach, 24 individual webpages, which constitute the DGT Resources website\footnote{Guidelines for translation contractors, referred to as the “Resources website” here. For the actual analysis of the webpage, see \sectref{sec:svoboda:4}.}, were surveyed in terms of the overlap and divergence of the material presented. To this end, the relevant material per language was copied into a specific table, with hyperlinks preserved.\footnote{All links were copied on 2017–6-30.}  Then the 24 individual tables were collated to form one comprehensive table representing the collection of the entire material available on the Resources website. This represents the actual research corpus, containing 793 links to individual SGs/TMs of all the 24 language departments. Afterwards, in a qualitatively informed matching exercise, clusters of similar information across the language specific webpages were identified and labelled ‘categories’ for the purposes of our study. Thus, the content of all the pages was cross-checked against the availability or absence of the same or similar categories on other webpages. Typically, any two identical pieces of information contained in any two webpages formed the basis for creating a specific category. The results of this part are presented in the \sectref{sec:svoboda:4.3}.

Using a qualitative method, then, the information carried by the actual hyperlinks was studied in detail with the aim of obtaining knowledge of the target sites/pages/files as referenced by the DGT resources hub. Domains, keywords, and expressions that are contained in the links as well as the surface text representation of the links were surveyed and analyzed with special attention paid to the translation quality aspect, in order to fulfil the initial aim of studying the linkages between DGT styleguides and translation quality. The results of this research phase are given below, in \sectref{sec:svoboda:4.4}.

Finally, given the fact that this area of institutional translation (i.e. TMs/SGs) was the subject of a pilot study using a limited research corpus in 2013\footnote{Cf. \citealt{Svoboda2013}.}, data will be presented to draw comparisons between the 2013 research and this chapter, which takes a more extensive approach.

Altogether, this chapter examines the below listed main assumptions. Their choice is governed by two main objectives: (i) Prior research has identified some features which are supposed to apply to institutional translation, however, due to limited research conducted so far, it is difficult to confirm, whether and to what extent they apply to specific institution types (cf. e.g. TMs and SGs in national contexts). In this respect, the present research is a contribution to testing the validity of previous claims on a particular institution (i.e. DGT). (ii) The below assumptions have been chosen due to the fact that they address the core area of interest here, i.e. the examination of SGs/TMs as quality assurance indicators:

\begin{itemize}
\item 
\textbf{Institutional translation} is a field characterized by \textbf{standardization}\footnote{This statement seems too obvious to yet come with the potential of yielding some new findings. However, it remains to be proven what kind of standardization actually applies to individual institutions, hence the DGT here.} \citep[58]{Koskinen2011}.
\item 
Despite the validity of hypothesis no. 1, the \textbf{amount and the nature of references} used as part of the translator resources will \textbf{differ among Resource pages}.\footnote{This hypothesis is based on the outcomes of a pilot study, cf. \citet{Svoboda2013}.}
\item 
As regards the content of the material researched, it is expected to relate to \textbf{reference materials}, \textbf{terminology}, \textbf{style/style guides} (cf. \citealt{Koskinen2008}, \citealt{Schäffner2014}) as these items have been identified as the most usual candidates for standardization/regulation.
\item 
The \textbf{quality} aspect of institutional translation is governed by \textbf{rules}.
\end{itemize}

\section{Analysis of the DGT’s Resources website}\label{sec:svoboda:4}

As the purpose of this chapter is to analyze the Resource website of the Directorate-General for Translation of the European Commission both in general and in terms of quality management, the following structure has been chosen for the analytical section. First, the location and structure of the Resource website is shown. Secondly, a quantitative overview of the resources is presented. This part is followed by an analysis of the content (the topics) of the individual resources. Finally, link texts of the hyperlinks that were included in the research corpus (i.e. a collection of the references – 793 links, representing the entire material available on the Resources website) are analyzed in detail. Finally, results are projected against the topic of quality management.

\subsection{Location, structure and overall description}\label{sec:svoboda:4.1}

 
The Resources website\footnote{A distinction is made between a website and a webpage. A website is a superordinate term referring to a bundle or a structure of webpages.} is accessible from the DGT homepage, under Related links (last reference in the side-bar, bottom left; cf. a current screenshot in \figref{fig:svoboda:1} below) and, subsequently, Translation and drafting resources (second item in the list), and Guidelines for translation contractors. The DGT website and its individual webpages are lean and touchscreen/mobile device optimized (in line with a recent overhaul of the Commission’s website Europa.eu). 

\begin{figure}
\caption{DGT homepage, available from: \url{https://ec.europa.eu/info/departments/translation_en} and the Related links reference, bottom left (Accessed 2017-10-10)}
\label{fig:svoboda:1}
\includegraphics[width=\textwidth]{figures/a7SVOBODAfinal-img1.png}
\end{figure}

 

\largerpage
Once the Resources website is displayed, it comes with a homogenous structure of 24 sections\footnote{It needs to be borne in mind that, presumably, a typical user hardly ever ventures to discover the information contained in more than two or three language specific sites. This means that for them, it may never have occurred, how vast the entire resource is.}, each representing “useful resources for translations on EU matters”, i.e. guidelines for contractors translating into one of the 24 EU official languages featured (see a screenshot in \figref{fig:svoboda:2} below).

\newpage 
\begin{figure}
\caption{DGT “Guidelines for translation contractors” site, upper part; available from: \url{https://ec.europa.eu/info/resources-partners/translation-and-drafting-resources/guidelines-translation-contractors_en} (Accessed 2017-10-10)}
\label{fig:svoboda:2}
\includegraphics[width=\textwidth]{figures/a7SVOBODAfinal-img2.png}
\end{figure}

 


In terms of their actual web address, the resources, after an overhaul of the EU’s official europa.eu website\footnote{Cf. the “Commission's new web presence” strategy, in which DGT is directly involved: “Work on the new web presence is led by the Commission’s communication, translation and IT departments” (https://ec.europa.eu/info/about-commissions-new-web-presence\_en (Accessed 2017–6-30)).}, are no longer featured directly as part of the DGT site, but under “Resources for partners”, which is directly subsumed to “European Commission” and its Info sub-site (cf. the address as depicted in the command line of \figref{fig:svoboda:2}). This hub page is available in English only, not even in the procedural languages.

The potential recipients of these resources are primarily DGT contractors, i.e. the materials pertain to the outsourcing aspect of DGT work. The outsourced production is part of institutional translation, because, in the end, the \citeauthor{DGT2008} eventually approves the translations and assumes responsibility for them. Yet even in-house translators do use these resources.\footnote{From personal experience (the author of this chapter worked as an in-house DGT translator in Luxembourg for three years after the Czech Republic joined the EU in 2004), I know that apart from the SGs/TMs targeting contractors, there are official SGs/TMs for internal use only (stored within the EC firewall), as well as individual/personal ones of an unofficial nature, created by members of staff for their personal use.} What has changed, however, since the recent (Europa website and, consequently) DGT website overhaul, is the fact that the resources section is no longer divided into two main areas. There used to be an intermediary page channeling users according to the translation direction: You chose either the “I translate into [language]” link, or that entitled “I translate out of [language]”. The latter section seems to have disappeared from the current Resources website, thus reducing the overall number of resource materials substantially.

In terms of origin, the materials can either be unique to a given language department or derived (translated and/or adapted) from a common source. They are topic and/or media specific and can assume various data formats. 

\subsection{Quantitative overview of the resources}\label{sec:svoboda:4.2}

It was possible to identify language departments with quantitatively the most and, eventually the least number of resources. The ultimate champion in terms of the highest number of resources is Lithuanian with some 71 items on the webpage, followed by Swedish with 70 such linked materials. Then, following behind by a distance of more than 20 items, there are just two pages showing in the 50–60 category: Romanian and Italian with 52, and 51 sources available, respectively. In the 40–50 tier, there is Bulgarian , whereas in the 30–40 band there are Estonian, Danish, Czech, Croatian, Spanish and German (listed in descending order). The most frequently represented category is the 20–30 tier (with the following 10 language pages represented in this band): Latvian, Portuguese, Maltese, Slovak, Slovenian, English, Irish, French, Greek and Finnish. Under the 20-hit threshold, there are Hungarian, Dutch and Polish.

\subsection{The content of individual pages}\label{sec:svoboda:4.3}

\subsubsection{The structure of the webpages}\label{sec:svoboda:4.3.1}

The structure of the individual, language-specific webpages is kept uniform, comprising three main sections: (I) “General EU information”, (II) “Contractor guides”, and (III) “Language-specific information”. The first section is further divided into two subsections (“EU institutions”, and “European legislation”), the second section is divided into three subsections (“Guidelines”, “EU terminology”, and “Style guides”) and so is the third section (“Terminology and glossaries”, “Models and templates”, and “Useful links (national legislation / authorities / expert bodies)”). The individual pages are all presented in English, which, on the one hand makes comparison much easier for the researcher, on the other, however, is a witness to the presumption that all users are sufficiently proficient in the language, which might be debatable.

Whereas, under Section I and II, the subsections are the same with all the 24 webpages, the picture differs in Section III as several pages present only two, one or none of the headings that are common to the majority of the other pages. For example, the subsection “Terminology and glossaries” occurs in 21 out of the 24 pages, whereas the last of the three subheadings (“Useful links […]”) appears in only 13 cases. Only one page (NL) does not feature any of the three subheadings in Section III, whereas 13 pages show all of them.

See \figref{fig:svoboda:3} for a screenshot of what a typical Resource page looks like.

\begin{figure}
\caption{DGT “Guidelines for contractors translating into Czech” site, upper part; available from: \url{https://ec.europa.eu/info/resources-partners/translation-and-drafting-resources/guidelines-translation-contractors/guidelines-contractors-translating-czech_en} (Accessed 2017-10-10)}
\label{fig:svoboda:3}
\includegraphics[width=\textwidth]{figures/a7SVOBODAfinal-img3.png}
\end{figure}

 


Tables \tabref{tab:svoboda:1}-\tabref{tab:svoboda:3} contain a graphic representation of the pages and their content. Dots refer to items that are available on the webpage. There are main sections (I. to III., marked as blue headings in the tables below, cf. Tables 1-3), subsections (orange headings) and categories (rows with their descriptions, clustered to form subsections). Categories summarize sources that occur twice or more times across the language versions of the Resource pages. The triangle (▴) reference means that there are other resources available that are unique, i.e. they are not repeated across the Resource pages (one triangle mark can represent several unique resources). Empty spaces indicate the absence of the specific item (i.e. Not Available). 


\begin{sidewaystable}
\caption{A comparison of DGT webpages entitled “Guidelines for contractors translating into [LANGUAGE]”, Section I. General EU information.
Available from: \url{https://ec.europa.eu/info/resources-partners/translation-and-drafting-resources/guidelines-translation-contractors_en} (Accessed 2017-6-30)}
\label{tab:svoboda:1}
\small
\begin{tabularx}{\textwidth}{>{\scriptsize}Qc@{\,}@{\,}c@{\,}c@{\,}c@{\,}c@{\,}c@{\,}c@{\,}c@{\,}c@{\,}c@{\,}c@{\,}c@{\,}c@{\,}c@{\,}c@{\,}c@{\,}c@{\,}c@{\,}c@{\,}c@{\,}c@{\,}c@{\,}c@{\,}r}
\lsptoprule
%&\multicolumn{24}{c}{\textbf{I. General EU information}}\\
& BG & HR & CS & DA & NL & EN & ET & FI & FR & DE & EL & HU & IT & GA & LV & LT & MT & PL & PT & RO & SK & SL & ES & SV\\
\midrule
& \multicolumn{24}{c}{\textbf{EU institutions}}\\
\midrule 
Names of institutions, bodies and agencies 
of the EU
% & ■ & ■ & ■ & ■ & ■ & ■ & ■ & ■ & ■ & ■ & ■ & ■ & ■ & ■ & ■ & ■ & ■ & ■ & ■ & ■ & ■ & ■ & ■ &  ■\\
&\BG&\HR&\CS&\DA&\NL&\EN&\ET&\FI&\FR&\DE&\EL&\HU&\IT&\GA&\LV&\LT&\MT&\PL&\PT&\RO&\SK&\SL&\ES&\SV\\
% &   &   &   &   &   &   &   &   &   &   &   &   & ▴ &   &   &   &   &   &   &   &   &   &   &   \\
\tablevspace
List of DGs and departments of the EC 
% & ■ & ■ & ■ & ■ & ■ & ■ & ■ & ■ & ■ & ■ & ■ & ■ & ■ & ■ & ■ & ■ & ■ & ■ & ■ & ■ & ■ & ■ & ■ &  ■\\
&\BG&\HR&\CS&\DA&\NL&\EN&\ET&\FI&\FR&\DE&\EL&\HU&\IT&\GA&\LV&\LT&\MT&\PL&\PT&\RO&\SK&\SL&\ES&\SV\\
% &   &   &   &   &   &   &   &   &   &   &   &   &   &   &   & ▴ &   &   &   &   &   &   &   &   \\
\tablevspace
List of EU Commissioners 
% & ■ & ■ &   & ■ & ■ & ■ & ■ &   & ■ & ■ & ■ &   & ■ & ■ & ■ & ■ &   &   & ■ & ■ &   &   & ■ &   \\
&\BG&\HR&   &\DA&\NL&\EN&\ET&   &\FR&\DE&\EL&   &\IT&\GA&\LV&\LT&   &   &\PT&\RO&   &   &\ES&   \\
% &   &   &   & ▴ &   &   &   &   &   &   &   &   & ▴ &   &   & ▴ &   &   &   &   &   &   &   &   \\
\tablevspace
Council configurations 
% & ■ & ■ &   & ■ & ■ &   & ■ &   & ■ & ■ & ■ &   & ■ &   &   &   &   &   & ■ & ■ &   &   & ■ &   \\
&\BG&\HR&   &\DA&\NL&   &\ET&   &\FR&\DE&\EL&   &\IT&   &   &   &   &   &\PT&\RO&   &   &\ES&   \\
% &   &   &   &   &   &   &   &   &   &   &   &   & ▴ &   &   &   &   &   &   &   &   &   &   &   \\
\tablevspace
EP committees 
% & ■ & ■ &   & ■ & ■ &   & ■ &   & ■ & ■ & ■ &   & ■ &   &   &   &   &   & ■ & ■ &   &   & ■ &   \\
&\BG&\HR&   &\DA&\NL&   &\ET&   &\FR&\DE&\EL&   &\IT&   &   &   &   &   &\PT&\RO&   &   &\ES&   \\
\tablevspace
CoR — commissions 
% & ■ & ■ &   & ■ & ■ &   &   &   &   & ■ & ■ &   & ■ &   &   &   &   &   & ■ & ■ &   &   &   &   \\
&\BG&\HR&   &\DA&\NL&   &   &   &   &\DE&\EL&   &\IT&   &   &   &   &   &\PT&\RO&   &   &   &   \\
\tablevspace
EESC — committees 
% & ■ &   &   & ■ &   &   &   &   &   &   &   &   &   &   &   &   &   &   &   &   &   &   &   &   \\
&\BG&   &   &\DA&   &   &   &   &   &   &   &   &   &   &   &   &   &   &   &   &   &   &   &   \\
% &   &   &   & ▴ &   &   &   &   &   &   &   &   &   &   &   &   &   &   &   &   &   &   &   &   \\
& \multicolumn{24}{c}{\textbf{European legislation}}\\
\midrule
% & ▴ &   &   &   &   &   & ▴ &   &   &   &   &   &   &   &   &   &   &   &   &   &   &   &   &   \\
EUR-Lex
% & ■ & ■ & ■ & ■ & ■ & ■ & ■ & ■ & ■ & ■ & ■ & ■ & ■ & ■ & ■ & ■ & ■ & ■ & ■ & ■ & ■ & ■ & ■ &  ■\\
&\BG&\HR&\CS&\DA&\NL&\EN&\ET&\FI&\FR&\DE&\EL&\HU&\IT&\GA&\LV&\LT&\MT&\PL&\PT&\RO&\SK&\SL&\ES&\SV\\
% &▴ & ▴ & ▴ &   &   &   &   &   &   &   &   &   &   &   &   &   &   &   &   &   &   &   &   &   \\
\tablevspace
PreLex / Legislative procedures
% & ■ & ■ & ■ & ■ & ■ &   & ■ &   &   &   &   &   &   &   &   &   &   &   &   &   &   &   & ■ &   \\
&\BG&\HR&\CS&\DA&\NL&   &\ET&   &   &   &   &   &   &   &   &   &   &   &   &   &   &   &\ES&   \\
Official documents from EU institutions
% & ■ &   &   &   &   &   & ■ &   &   &   &   &   & ■ &   &   &   &   &   &   &   &   &   &   &   \\
&\BG&   &   &   &   &   &\ET&   &   &   &   &   &\IT&   &   &   &   &   &   &   &   &   &   &   \\
\tablevspace
CCVista
% &   & ■ &   &   &   &   &   &   &   &   &   &   &   &   &   &   &   &   &   & ■ &   &   &   &   \\
&   &\HR&   &   &   &   &   &   &   &   &   &   &   &   &   &   &   &   &   &\RO&   &   &   &   \\
% &   &   &   & ▴ &   &   & ▴ & ▴ &   &   &   &   & ▴ &   &   & ▴ &   &   &   & ▴ & ▴ &   &   &   \\
\tablevspace
EEA Agreement
% &   &   &   &   &   &   &   &   &   &   & ■ &   &   &   &   &   &   &   &   &   & ■ &   &   &  ■\\
&   &   &   &   &   &   &   &   &   &   &\EL&   &   &   &   &   &   &   &   &   &\SK&   &   &\SV\\
\lspbottomrule
\end{tabularx} 
\end{sidewaystable}
 

\begin{sidewaystable} 
\caption{A comparison of DGT webpages entitled “Guidelines for contractors translating into [LANGUAGE]”, Section II. Contractor guides.
Available from: \url{https://ec.europa.eu/info/resources-partners/translation-and-drafting-resources/guidelines-translation-contractors_en} (Accessed 2017-6-30)}
\label{tab:svoboda:2} 
\small
\begin{tabularx}{\textwidth}{>{\scriptsize}Qc@{\,}@{\,}c@{\,}c@{\,}c@{\,}c@{\,}c@{\,}c@{\,}c@{\,}c@{\,}c@{\,}c@{\,}c@{\,}c@{\,}c@{\,}c@{\,}c@{\,}c@{\,}c@{\,}c@{\,}c@{\,}c@{\,}c@{\,}c@{\,}r}
\lsptoprule 
% & \multicolumn{24}{c}{\textbf{II. Contractor guides}}\\
% & BG & HR & CS & DA & NL & EN & ET & FI & FR & DE & EL & HU & IT & GA & LV & LT & MT & PL & PT & RO & SK & SL & ES & SV\\
& \multicolumn{24}{c}{\textbf{Guidelines}}\\
\midrule
% &   &   &   &   &   &   &   &   &   &   &   &   &   &   &   &   &   & ▴ &   &   &   &   &   &   \\
Guide for contractors translating for the European Commission [Instructions for eXtra portal]
  &\BG&\HR&\CS&\DA&\NL&\EN&\ET&\FI&\FR&\DE&\EL&\HU&\IT&\GA&\LV&\LT&\MT&\PL&\PT&\RO&\SK&\SL&\ES&\SV\\
% & ■ & ■ & ■ & ■ & ■ & ■ & ■ & ■ & ■ & ■ & ■ & ■ & ■ & ■ & ■ & ■ & ■ & ■ & ■ & ■ & ■ & ■ & ■ &  ■\\
\tablevspace
Translation checklist
% &   &   & ■ & ■ & ■ & ■ & ■ & ■ & ■ & ■ & ■ & ■ & ■ & ■ & ■ & ■ & ■ &   & ■ & ■ & ■ & ■ & ■ &  ■\\
&   &   &\CS&\DA&\NL&\EN&\ET&\FI&\FR&\DE&\EL&\HU&\IT&\GA&\LV&\LT&\MT&   &\PT&\RO&\SK&\SL&\ES&\SV\\
% &   &   &   &   &   &   &   &   &   &   &   &   &   &   &   &   &   &   &   & ▴ &   &   &   &   \\
\tablevspace
Translation Quality Info Sheets for Contractors 
% &   & ■ &   &   &   &   & ■ &   &   &   &   &   & ■ &   &   & ■ & ■ & ■ &   & ■ & ■ &   &   &   \\
&   &\HR&   &   &   &\EN& & & & & & &\IT& & &\LT&\MT&\PL& &\RO&\SK& & & \\
\tablevspace
Guidelines for translating into [LANGUAGE]
% &   &   &   &   & ■ &   &   &   &   & ■ &   &   &   &   &   &   & ■ &   & ■ & ■ & ■ &   &   &   \\
&   &   &   &   &\NL&   &   &   &   &\DE&   &   &   &   &   &   &\MT&   &\PT&\RO&\SK&   &   &   \\
% &   &   & ▴ & ▴ &   &   &   &   &   &   &   & ▴ &   &   &   &   &   &   &   &   &   &   & ▴ &  ▴\\
\tablevspace
Guidelines for translating and revising texts intended for the web
% &   & ■ &   &   &   &   &   &   &   &   &   &   &   &   &   &   &   &   &   &   &   &   &   &  ■\\
&   &\HR&   &   &   &   &   &   &   &   &   &   &   &   &   &   &   &   &   &   &   &   &   &\SV\\
% &   &   &   &   &   &   &   &   &   & ▴ &   &   &   &   &   & ▴ & ▴ &   &   & ▴ &   &   &   &  ▴\\
& \multicolumn{24}{c}{\textbf{EU terminology}}\\
\midrule
IATE — EU terminology database 
% & ■ & ■ & ■ & ■ & ■ & ■ & ■ & ■ & ■ & ■ & ■ & ■ & ■ & ■ & ■ & ■ & ■ & ■ & ■ & ■ & ■▴& ■ & ■ &  ■\\
&\BG&\HR&\CS&\DA&\NL&\EN&\ET&\FI&\FR&\DE&\EL&\HU&\IT&\GA&\LV&\LT&\MT&\PL&\PT&\RO&\SK&\SL&\ES&\SV\\
\tablevspace
TARIC codes — online customs tariff database
% & ■ & ■ & ■ & ■ & ■ & ■ & ■ & ■ & ■ & ■ & ■ & ■ & ■ & ■ & ■ & ■ & ■ & ■ & ■ & ■ & ■ & ■ & ■ &  ■\\
&\BG&\HR&\CS&\DA&\NL&\EN&\ET&\FI&\FR&\DE&\EL&\HU&\IT&\GA&\LV&\LT&\MT&\PL&\PT&\RO&\SK&\SL&\ES&\SV\\
\tablevspace
EU Budget online
% & ■ & ■ & ■ & ■ & ■ & ■ & ■ &   & ■ & ■ & ■ & ■ & ■ &   & ■ & ■ & ■ &   & ■ & ■ & ■ & ■ & ■ &   \\
&\BG&\HR&\CS&\DA&\NL&\EN&\ET&   &\FR&\DE&\EL&\HU&\IT&   &\LV&\LT&\MT&   &\PT&\RO&\SK&\SL&\ES&   \\
\tablevspace
Eurovoc — multilingual thesaurus covering EU policy fields 
% & ■ & ■ & ■ &   &   &   & ■ &   & ■ & ■ &   & ■ & ■ &   & ■ & ■ &   &   & ■ & ■ & ■ &   &   &   \\
&\BG&\HR&\CS&   &   &   &\ET&   &\FR&\DE&   &\HU&\IT&   &\LV&\LT&   &   &\PT&\RO&\SK&   &   &   \\
\tablevspace
RAMON — Eurostat’s statistical metadata website (including combined nomenclature) 
% &   & ■ & ■ & ■ & ■ & ■ & ■ &   & ■ & ■ & ■ & ■ & ■ &   & ■ & ■ & ■ &   & ■ & ■ & ■ &   & ■ &  ■\\
&   &\HR&\CS&\DA&\NL&\EN&\ET&   &\FR&\DE&\EL&\HU&\IT&   &\LV&\LT&\MT&   &\PT&\RO&\SK&   &\ES&\SV\\
% &   &   &   &   &   & ▴ &   &   &   &   &   &   &   &   &   & ▴ &   &   &   &   &   &   &   &   \\
\midrule
\end{tabularx}
\end{sidewaystable}

\begin{sidewaystable}
\begin{tabularx}{\textwidth}{>{\scriptsize}Qc@{\,}@{\,}c@{\,}c@{\,}c@{\,}c@{\,}c@{\,}c@{\,}c@{\,}c@{\,}c@{\,}c@{\,}c@{\,}c@{\,}c@{\,}c@{\,}c@{\,}c@{\,}c@{\,}c@{\,}c@{\,}c@{\,}c@{\,}c@{\,}r}
\midrule 
%&\multicolumn{24}{c}{\textbf{II. Contractor guides}}\\
% & BG & HR & CS & DA & NL & EN & ET & FI & FR & DE & EL & HU & IT & GA & LV & LT & MT & PL & PT & RO & SK & SL & ES & SV\\
% \midrule
& \multicolumn{24}{c}{\textbf{Style guides}}\\
\midrule
% &   &   &   & ▴ &   &   &   & ▴ &   & ▴ &   &   &   & ▴ &   &   &   & ▴ &   &   &   &   &   &   \\
{}[LANGUAGE] style guide — DG Translation in-house styleguide
% & ■ & ■ &   & ■ &   & ■ & ■ & ■ &   & ■ & ■ &   &   & ■ &   &   &   & ■ & ■ & ■ & ■ & ■ & ■ &  ■\\
&\BG&\HR&   &\DA& ~~  &\EN&\ET&\FI&   &\DE&\EL&   &   &\GA&   &   &   &\PL&\PT&\RO&\SK&\SL&\ES&\SV\\
% &   & ▴ &   &   &   & ▴ &   &   &   & ▴ &   &   &   & ▴ &   &   &   & ▴ & ▴ &   &   &   & ▴ &  ▴\\
\tablevspace
Interinstitutional Style Guide — for [LANGUAGE]
% & ■ & ■ & ■ & ■ & ■ & ■ & ■ & ■ & ■ & ■ & ■ & ■ & ■ & ■ & ■ & ■ & ■ & ■ & ■ & ■ & ■ & ■ & ■ &  ■\\
&\BG&\HR&\CS&\DA&\NL&\EN&\ET&\FI&\FR&\DE&\EL&\HU&\IT&\GA&\LV&\LT&\MT&\PL&\PT&\RO&\SK&\SL&\ES&\SV\\
% &   &   &   &   &   & ▴ &   &   &   & ▴ &   &   &   &   &   & ▴ &   &   & ▴ &   &   & ▴ & ▴ &   \\
\tablevspace
Joint practical guide for persons involved in the drafting of EU legislation — for [LANGUAGE]
% & ■ & ■ & ■ & ■ & ■ & ■ & ■ & ■ & ■ & ■ & ■ & ■ & ■ &   & ■ & ■ & ■ & ■ & ■ & ■ & ■ &   & ■ &  ■\\
&\BG&\HR&\CS&\DA&\NL&\EN&\ET&\FI&\FR&\DE&\EL&\HU&\IT&   &\LV&\LT&\MT&\PL&\PT&\RO&\SK&   &\ES&\SV\\
\tablevspace
Joint Handbook for the Presentation and Drafting of Acts subject to the Ordinary Legislative Procedure
% &   & ■ &   &   &   &   &   & ■ &   &   &   &   &   &   &   & ■ &   &   &   &   &   &   &   &   \\
&   &\HR&   &   &   &   &   &\FI&   &   &   &   &   &   &   &\LT&   &   &   &   &   &   &   &   \\
\tablevspace
How to write clearly
% &   &   &   &   &   & ■ &   &   &   & [■] &   &   & ■ &   &   &   & ■ &   &   &   &   &   &   &   \\
&   &   &   &   &   &\EN&   &   &   &[\DE]&   &   &\IT&   &   &   &\MT&   &   &   &   &   &   &   \\
% & ▴ &   &   &   &   & ▴ & ▴ & ▴ &   &       &   &   &   &   & ▴ &   &   &   &   &   &   &   &   &   \\
\lspbottomrule
\end{tabularx} 
\end{sidewaystable}

\begin{sidewaystable} 
\caption{A comparison of DGT webpages entitled “Guidelines for contractors translating into [LANGUAGE]”, Section III. Language-specific information.}
\label{tab:svoboda:3}
\small
\begin{tabularx}{\textwidth}{>{\scriptsize}Qc@{\,}@{\,}c@{\,}c@{\,}c@{\,}c@{\,}c@{\,}c@{\,}c@{\,}c@{\,}c@{\,}c@{\,}c@{\,}c@{\,}c@{\,}c@{\,}c@{\,}c@{\,}c@{\,}c@{\,}c@{\,}c@{\,}c@{\,}c@{\,}r}
\lsptoprule 
%&\multicolumn{24}{c}{\textbf{III. Language-specific information}}\\
% & BG & HR & CS & DA & NL & EN & ET & FI & FR & DE & EL & HU & IT & GA & LV & LT & MT & PL & PT & RO & SK & SL & ES & SV\\
\textbf{Terminology and glossaries} 
% \midrule
% & ■ & ■ & ■ & ■ &   & ■ & ■ & ■ & ■ &   & ■ & ■ & ■ & ■ & ■ & ■ & ■ &   & ■ & ■ & ■ & ■ & ■ & ■ \\
&\BG&\HR&\CS&\DA&   &\EN&\ET&\FI&\FR&   &\EL&\HU&\IT&\GA&\LV&\LT&\MT&   &\PT&\RO&\SK&\SL&\ES&\SV\\
EU glossaries — on EUROPA (the EU's website) 
% &   &   &   & ■ &   & ■ &   &   &   &   &   &   & ■ &   &   &   &   &   & ■ &   &   &   &   & ■ \\
&   &   &   &\DA&   &\EN&   &   &   &   &   &   &\IT&   &   &   &   &   &\PT&   &   &   &   &\SV\\
% & ▴ & ▴ & ▴ & ▴ &   & ▴ & ▴ & ▴ & ▴ &   & ▴ & ▴ & ▴ & ▴ & ▴ & ▴ & ▴ &   & ▴ & ▴ & ▴ & ▴ & ▴ &  ▴\\
\tablevspace
\textbf{Models and templates}
% & ■ & ■ & ■ & ■ &   & ■ & ■ & ■ & ■ & ■ & ■ & ■ & ■ &   & ■ & ■ & ■ & ■ & ■ & ■ &   & ■ & ■ & ■\\
&\BG&\HR&\CS&\DA&   &\EN&\ET&\FI&\FR&\DE&\EL&\HU&\IT&   &\LV&\LT&\MT&\PL&\PT&\RO&   &\SL&\ES&\SV\\
LegisWrite models
% &   & ■ & ■ &   &   &   & ■ &   &   &   & ■ & ■ &   &   & ■ & ■ &   & ■ & ■ & ■ &   &   &   &   \\
&   &\HR&\CS&   &   &   &\ET&   &   &   &\EL&\HU&   &   &\LV&\LT&   &\PL&\PT&\RO&   &   &   &   \\
% & ▴ & ▴ & ▴ & ▴ &   & ▴ & ▴ & ▴ & ▴ & ▴ & ▴ & ▴ & ▴ &   & ▴ & ▴ & ▴ & ▴ & ▴ & ▴ &   & ▴ & ▴ &  ▴\\
\tablevspace
\textbf{Useful links (national legislation / authorities / expert bodies)}
% & ■ &   & ■ & ■ &   & ■ & ■ &   & ■ &   &   & ■ & ■ &   & ■ &   &   &   & ■ & ■ &   & ■ & ■ &   \\
&\BG&   &\CS&\DA&   &\EN&\ET&   &\FR&   &   &\HU&\IT&   &\LV&   &   &   &\PT&\RO&   &\SL&\ES&   \\
\tablevspace
N-Lex — national legislation databases of EU countries
% &   &   &   & ■ &   & ■ &   &   & ■ &   &   &   &   &   &   &   &   &   & ■ &   &   &   &   &   \\
&   &   &   &\DA&   &\EN&   &   &\FR&   &   &   &   &   &   &   &   &   &\PT&   &   &   &   &   \\
% & ▴ &   & ▴ & ▴ &   &   & ▴ &   & ▴ &   &   & ▴ & ▴ &   & ▴ &   &   &   &   & ▴ &   & ▴ & ▴ &   \\
\tablevspace
Language magazine
% &   &   &   &   &   &   &   &   &   &   &   &   & ■ &   &   &   & ■ &   & ■ &   &   &   & ■ &   \\
&   &   &   &   &   &   &   &   &   &   &   &   &\IT&   &   &   &\MT&   &\PT&   &   &   &\ES&   \\
\tablevspace
Language network
% &   &   &   &   &   &   &   &   &   &   &   &   & ■ &   &   & ■ &   &   &   &   & ■ &   &   &   \\
&   &   &   &   &   &   &   &   &   &   &   &   &\IT&   &   &\LT&   &   &   &   &\SK&   &   &   \\
\lspbottomrule
\end{tabularx}  
\end{sidewaystable}

\subsubsection{Overview and analysis of the content of the Resource webpages}\label{sec:svoboda:4.3.2}

\paragraph*{Section I (“General EU information”)} 

The first subsection entitled “EU institutions” of Section I (cf. \tabref{tab:svoboda:1}) comprises a total of 7 groups of resources. The first two (“Names of institutions, bodies and agencies of the EU” and “List of DGs and departments of the EC”) are represented with each of the 24 webpages. The other five groups (“List of EU Commissioners”, “Council configurations”, “EP committees”, “CoR — commissions”, and “EESC — committees”) are represented in 16, 12, 12, 9, and 2 cases, respectively. Two webpages stand out here, i.e. BG and DA, since they show all of the 7 resources.\footnote{The majority of departments (8) features just 2 pieces of resources; however, another high number of departments (7) offer 6 resources in this section.}

The second major subsection within Section I is called “European legislation” and it offers 6 categories. The picture suggested by the numbers is a mixed one as was the one above, too: Whereas the first category (“EUR-Lex”) can be located in each of the 24 webpages, there are the remaining 4 categories (“PreLex / Legislative procedures”, “Official documents from EU institutions”, “CCVista”, and “EEA Agreement”) that are represented in the following way: 7, 3, 2 and 5, respectively. 

\paragraph*{Section II (“Contractor guides”)}

The first subsection (“Guidelines”) has one category that is represented 100\% across all the language-specific webpages, i.e. “Guide for contractors translating for the European Commission” (however, it needs to be considered in conjunction with “Instructions for eXtra portal”, which is just another description for one and the same category. With the second category “Translation checklist”, just three representations are missing, which shows that uniformity has not been achieved fully here. The remaining three (“Translation Quality Info Sheets for Contractors”, “Guidelines for translating into [LANGUAGE]”, “Guidelines for translating and revising texts intended for the web”) are represented in occurrences ranging from two to nine.

Under the second major subsection, “EU terminology”, there are five categories: “IATE — EU terminology database”; “TARIC codes — online customs tariff database”; “EU Budget online”; “Eurovoc — multilingual thesaurus covering EU policy fields” and “RAMON — Eurostat’s statistical metadata website (including combined nomenclature)”. Whereas the first two are represented 100\% across the language versions, the third (EU Budget online) sees 20 occurrences and the last of this lot (RAMON) — 19. The least represented category here is Eurovoc with roughly slightly more than half of Resource pages mentioning it (i.e. 13 cases).

When it comes to the last subsection (Style guides) in Section II, only one category is mentioned in all the 24 Resource pages: “Interinstitutional Style Guide — for [LANGUAGE]”. “Joint practical guide for persons involved in the drafting of EU legislation — for [LANGUAGE]” is heavily represented with only two languages missing (GA, SL). The other three categories (“[LANGUAGE] style guide — DG Translation in-house styleguide”; “Joint Handbook for the Presentation and Drafting of Acts subject to the Ordinary Legislative Procedure”; “How to write clearly”) are represented in numbers ranging from more than half of the cases to just a few mentions (16, 3 and 4 respectively).

\paragraph*{Section III (“Language-specific information”)}

This section shows the lowest degree of uniformity. Not even the three major subsections – “Terminology and glossaries”, “Models and templates”, and “Useful links (national legislation / authorities / expert bodies)” are represented in all the language specific webpages. Thus, we find that the NL page lacks Section III altogether (and, among the 24 pages, it is the only one to leave a whole section out). While the subsection “Terminology and glossaries” is missing from the NL, DE, and PL pages, the subsection of “Models and templates” is missing from the NL, GA, and SK pages. The “Useful links” subsection is represented with just above half of the pages (13). The categories typically shown in this section tend to be represented in rather small numbers: 5, 10, 4, 4, and 3, respectively.\footnote{I.e. “EU glossaries — on EUROPA (the EU's website)”; “LegisWrite models”; “N-Lex — national legislation databases of EU countries”; “Language magazine”; “Language network”.}

\subsubsection{Analysis of the content of individual pages}\label{sec:svoboda:4.3.3}
It should be borne in mind that any content analysis was carried out on the basis of the surface representation of links and/or link descriptions, not on an in-depth analysis of the individual sources. This limitation is intentional as an in-depth content analysis would not be practicable under the present research study design.

In terms of quantity, the three subsections with the highest number of categories \textbf{represented across all the 24 language Resource pages} (i.e. the most \textbf{harmonized subsections}) are the following: “EU institutions”, “European legislation”, and “Guidelines” (with 7, 6, and 6 categories, respectively). The categories that are represented 100\% are the following:

\begin{itemize}
\item 
Names of institutions, bodies and agencies of the EU
\item 
List of DGs and departments of the EC
\item 
EUR-Lex
\item 
IATE, EU terminology database
\item 
TARIC codes, online customs tariff database
\item 
Interinstitutional Style Guide for [LANGUAGE]
\item 
Guide for contractors translating for the European Commission [Instructions for eXtra portal]
\end{itemize}

The order of information as presented within subsections varies, yet not to a very significant degree. For example, while most of the Resource pages feature the category of “LegisWrite models” (i.e. templates for the layout of specific EU legislative documents) first in the list under the subsection “Models and templates” within Section III (this is true for 6 of the 10 pages that feature this source), we find it on 7\textsuperscript{th}, 4\textsuperscript{th}, 6\textsuperscript{th}, and 2\textsuperscript{nd} position among the sources listed on the Czech, Latvian, Lithuanian, and Polish webpages. Obviously, this \textbf{lack of complete cross-language alignment} is rather insignificant, especially as it still occurs within one and the same subsection across language versions. There are, however, cases of resources featured on locations varying in terms of whole sections, such as a resource on clear writing (“How to write clearly”). Whereas EN, IT and MT pages place it in Section II (in subsection “Style guides”\footnote{The German page features it in the same Section II, yet under subsection Guidelines.}), the Finnish page has it in Section III (subsection “Models and templates”).

Apart from the resources mentioned in the categories (i.e. those that occur twice or more times across the language versions), there is a considerable wealth of \textbf{unique information}, largely language and/or country specific. The proportion of this unique information is huge, amounting to almost exactly 50\% of the entire pool of the resources\footnote{In fact, 395 unique sources were counted, which is 49.8\% of the total of 793.}. Most of the unique links can be found in the Swedish and Lithuanian (50+ each) language pages, and Romanian, Italian, Bulgarian and Czech show more than twenty resources each. Less than 10 unique resources can be found on the Dutch, Hungarian, Greek, Polish, Portuguese, French, Finnish, English and Slovak pages (listed in ascending order in terms of the number of occurrences). The remaining 9 pages have between 10 and 19 unique sources. The circumstance that the French and English language departments do not deem it necessary to upload specific (unique) material in large volumes may be due to the fact that these are procedural languages and serve as source languages comparably more often than the other languages\footnote{For want of space, we abstain from discussing the role of the third procedural language (German).}.

\subsection{Analysis of links}\label{sec:svoboda:4.4}

The following is a detailed analysis of the links\footnote{To give a few examples of the link texts – and for the sake of interest – here are the two longest links of all those featured in the research corpus: “eur-lex.europa.eu/search.html?or0=DN\%3D32012r0966*,DN-old\%3D32012r0966*\& qid=1467623088973\& DTS\_DOM=EU\_LAW\& type=advanced\& lang=en\& SUBDOM\_INIT=LEGISLATION\& DTS\_SUBDOM=LEGISLATION”“ec.europa.eu/info/resources-partners/translation-and-drafting-resources/guidelines-translation-contractors/guidelines-contractors-translating-romanian\_de”On the contrary, the shortest links read “csic.es”, “uni.com”, “ritap.es”, “kotus.fi”, and “legex.ro”. A typical link would look like this: http://ec.europa.eu/info/files/czech-resources-combined-nomenclature-en-cs-tmx\_cs.} as featured on the individual 24 webpages of the DGT Resources page. The analysis of the wealth of data available was carried out on two levels. First, all the links were taken together and examined in terms of link address, and secondly, the link face text (which users normally see as the colored and underlined text) was analyzed, including some approximations on the actual content of the resources referred to. Whereas the findings are presented in this section, they are discussed in \sectref{sec:svoboda:5}.

\subsubsection{Analysis of link addresses and their provenance}\label{sec:svoboda:4.4.1}

Altogether, the DGT Resource website contains 24 webpages, including a total of 793 hyperlinks\footnote{On average, there are 33 resources per Resource webpage.}, i.e. interactive fields representing individual references. The number of resources contained on individual (language specific) webpages totals from 71 (Lithuanian) to 17 (Dutch and Polish).

As for first-level domains, the \textit{.eu} domain has 401 occurrences (i.e. 51\%), other top-level domains represented in the linked sites/files include the following\footnote{Interestingly, there are EU first-level domains that refer to an EU country where an EU official language is spoken, yet are not represented in the list above (such as .cz/, .mt/, .pl/, .pt/, .si/, .sk/), which means that the Czech, Maltese, Polish, Portuguese, Slovenian and Slovak language resource pages do not link to resources hosted on the respective national domain at all. This is explained by the fact that although there is language-specific content given on the Resource webpages, it may be placed on the europa.eu server, not linked directly to the country resource.}: 

\begin{itemize}
 \item \texttt{.it} (15 occurrences)
\item \texttt{.ie} (13 occurrences)
\item \texttt{.lt} (10 occurrences)
\item \texttt{.ro} (8 occurrences)
\item \texttt{.com}, \texttt{.es}, \texttt{.fi}, \texttt{.hr}, \texttt{.org} (6 occurrences each)
\item \texttt{.lv}, \texttt{.se} (4 occurrences each)
\item \texttt{.be} (3 occurrences)
\item \texttt{.bg}, \texttt{.ch}, \texttt{.dk}, \texttt{.ee}, \texttt{.hu}, \texttt{.nl}, \texttt{.si} (2 occurrences each)
\item \texttt{.de}, \texttt{.edu}, \texttt{.fr}, \texttt{.int}, \texttt{.lu}, \texttt{.ru}, \texttt{.uk} (1 occurrence each)

\end{itemize}

As regards second-level domains, the \textit{europa.eu} domain (EU official website domain) is represented by 396 occurrences, which means that only five of the 401 .eu hyperlinks link to websites other than the europa.eu website. Third-level domains include predominantly the EU Commission domain with the domain name “ec.europa.eu” occurring 136 times (17\% of all the links in the research corpus). Of this pool, there seem to be only five cases of a direct link to the DGT website (featuring the “ec.europa.eu/translation/” path). Other EC’s DGs and services referred to, according to the link address, include DG Agriculture and Rural Development, DG Budget, Eurostat, DG Health and Food Safety, Joint Research Centre, DG Justice and Consumers, and DG Taxation and Customs Union.

To make the above information complete, the proportion of linked sites vs. linked files in the resources pool was surveyed as well. The expressions \texttt{file/files/ .pdf/.zip} occur in 362 links (whether once or multiple times), which also indicates the number of files linked on the Resource website. 

\subsubsection{Approximation of the content based on the hyperlink surface text}\label{sec:svoboda:4.4.2}

When it comes to the most frequently occurring link tags, i.e. the hyperlink face text, which appears as the title of the hyperlink, “Interinstitutional Style Guide” (EU institutions’ main SG), “EUR-Lex” (the EU’s legislation repository), “IATE” (EU termbank) appear 25 times each. “List of directorates-general and departments of the European Commission” as well as “Names of institutions, bodies and agencies of the European Union” and “TARIC codes” have 24 occurrences each; “Joint practical guide for persons involved in the drafting of EU legislation” and “Translation checklist” occur 19 times. These link tags give a tentative indication of what topics are most frequently represented in the resources pool, thus pointing to the importance that the website owner has attributed to them.

\subsubsubsection{Explicit referrals to quality aspects}

In order to see, to what extent the goal of the SGs and translation manuals (which, ultimately, is maintaining/raising product quality) is mentioned explicitly in the caption texts, the following expressions were searched: qualit*, term*, glossar*, corp*, harmonis*, standard*, require* and others (see below). They stand for topics linked with ensuring standardization and harmonization (e.g. by managing terminology, glossaries, corpora), with introducing requirements and with providing further search and information sources.

\subsubsubsection{The expression of quality \textit{per se}}

The expression “qualit*”\footnote{The asterisk designates a truncated string of characters. This means that when searching for „qualit*”, expressions such as quality, qualities, qualitative, qualitatively, etc. can be found.} is encountered 9 times altogether in the research corpus, i.e. in three types of resources. The surface titles of the links are as follows: DG Translation Quality Guidelines; Quality criteria for translating into German; Translation Quality Info Sheet; and Translation quality info sheets for contractors (the latter occurs 6 times). 

\subsubsubsection{Guidelines, manuals, requirements}

The expression “guide*” is represented 109 times altogether. The type of resource most commonly referred to is a “guide” (89 occurrences), followed by “guidelines” (18 occurrences). According to the DGT resources pages, guides/guidelines can be “brief”, “specific”, “essential”. They differ according to their focus (Spelling guide, Editing guidelines for translators, Guide to eurojargon, Guide to writing clear administrative Italian, Guidelines on terminology) or scope (Interinstitutional Style Guide, Interinstitutional Style Guide — for Czech, Interinstitutional Style Guide — for French…; Joint practical guide for persons involved in the drafting of EU legislation — for Bulgarian, Joint practical guide for persons involved in the drafting of EU legislation — for German…). They may be text type specific: Guidelines for translating and revising Commission communications; or detailing the purpose of translations: Guidelines for translating and revising texts intended for the general public, Guidelines for translating and revising texts intended for the web.

Most of the language departments have their own SGs: English Style Guide, Finnish style guide, German translation style guide, Greek style guide, etc. Guides can differ according to the target group: “Guide for contractors translating for the European Commission”, “The Joint Handbook for the presentation and drafting of acts subject to the ordinary legislative procedure”, “Polish in-house style guide” (which is primarily, yet not exclusively intended for EC in-house staff), “DG Translation in-house handbook for Danish (Visdomsbogen)”. They can also be rather general: “DG Translation Quality Guidelines”, “Linguistic guidelines for translators”. There can even be “meta”-style guides: “Using style guides — in what order?” (on the German Resource page) or “Integrated system of Lithuanian language resources”.

The expression “manual” is used in a rather limited way: There are only five occurrences of this term: “Legislative drafting manual”, “Manual of precedents for acts established within the Council of the European Union” (under two language pages), “Manual of Precedents for International Agreements and Related Acts” and “Revision manual”. 

As regards other expressions used to denote the nature of the resource, there are some that suggest the binding nature of a document, such as: “decree for transliteration”, “rules”, “terms and linguistic norms”, “normative translation memory”, “translation conventions”. Besides, there is also a number of guide types, the binding nature of which is weaker: “Consolidated linguistic advice”, “Czech orthography recommendation”, “Tips for better language”. While “requirements” are featured just once (“Basic requirements for terms”), “instructions” tend to be rather frequent (“Instructions for eXtra portal”, “Instructions for translating”, “Instructions on the use of xliff files”, “Maltese Freelance Instructions”, “Specific instructions for different document types” etc.).

Finally, there are models (“Explanatory memorandum model”, “Legal Service models”, “Legislative financial statement model”, “LegisWrite models and translation memories”), templates (“Thematic templates and translation memories”), checklists (“Checklist for outgoing translations”, “Pre-delivery checklist”) and quite a lot of the guides are entitled using a \textit{How To} question: “How to refer to EFSA documents”, “How to search EU case law”, “How to use IATE?”, “How to write clearly (4x)”, etc. Altogether, there are well over 130 sources presented as the types of guides/guidelines/manuals as listed in this subsection.

\subsubsubsection{Harmonization/standardization}

The usage of the above expressions varies: “harmonis*” occurs just twice and seems to denote two different realities, one evoking the legislative process (“Harmonisation of Hungarian law with EU legislation”) and the other referring to consistency requirements (“Harmonised geographical names”). The expression “standard” can be found eight times in the following resource titles and, equally, pointing to a varied understanding of the expression, from official standards/norms to consistency issues: “Lithuanian standards board term base”; “National Organisation for Standardization”; “Register of standardised terms”; “Revised official Irish language standard”; “Standard clauses for explanatory memorandum”; “Standard ending of the notice (LT-EN-FR-DE)”; “Standard forms for public procurement”; “UN/ECE standard phrases”. The expression “common” denotes the following references: “Common phrases (EN-LT)”, “Common titles of German legislative acts” and others.

\subsubsubsection{Terminology rules}

The expression “term*” (standing for terminology, terms, etc.) appears 40 times.\footnote{Here are a few examples: \textit{Agreed terminology, solved linguistic issues and slogans; Application for recognition of a traditional term; Basic requirements for terms; Budget terminology; English-Lithuanian dictionary of polytechnics terms; European Parliament glossary database TermCoord; Finnish Centre for Technical Terminology; Fishery terms; Glossary of energy terms; Irish terminology handbook; Language and terminology newsletter; Lithuanian standards board term base; Non-Iate Termbase; Procurement terminology; Schengen and migration terminology; Slovak Terminology Network (STS); State aid terminology; Statistics terms; terminography; The National Terminology Database for Irish; Theory of terminology.}} There are 33 mentions of “glossar*” in the research corpus.\footnote{Here are selected examples: \textit{Anti-dumping glossary; Asylum and migration glossary; Civil and commercial law glossary; Council glossaries; Customs glossary; Environment glossary; EU budget glossary; Glossary in the field of concurrence; Glossary of administrative language; Glossary of languages and countries; Glossary of primary law; Glossary of security documents, security features~and other related technical terms; Nuclear energy glossary; Phytosanitary glossary; Railway glossary.}} Electronic corpora are referred to eight times in total.\footnote{These include: “Croatian Language Corpus”, “Croatian National Corpus” (HR), “Aligned corpus of English to Irish translations”, “Aligned corpus of English to Irish translations — legislative texts”, “The New Corpus for Ireland — corpus of Irish language texts” (GA), “Lithuanian language corpus”, “Corpus of Academic Lithuanian” and “Corpus of computer lexis and phrases” (LT).} Other relevant terms in this regard involve “vocabulary”\footnote{Cf. “Antidumping vocabulary”, “EU common procurement vocabulary”, “Gender equality vocabulary”.}, “expressions”\footnote{Cf. “Austrian expressions”.}, “dictionary”\footnote{Cf. “English-Bulgarian polytechnic dictionary”.} or “database”.\footnote{Cf. “Danish legal database”, “Finnish grammar database”, “Irish placenames database”.} Taken together, these terminology-related expressions appear 104 times.

\subsubsubsection{Information sites and further references}

Resources include not only manuals and templates to abide by when translating. Many other documents/references of an informative nature can be found in the featured resources as well: “Dumping explained”; “Misused English words and expressions in EU publications”; “The evaluation of Freelance Document”; “Translating from Slovak into English”; “Translating online content”; “Translation patterns”; “Translation problems and difficulties”; “Translation quality info sheets for contractors”; “Typical translation mistakes”. Designations like “Language and translation reference site” or “Legislative portal” are used to refer to other resources that are either a hub, a collection of further resources and/or a repository.

\section{Discussion of results}\label{sec:svoboda:5}

\subsection{Overall findings related to SGs/TMs}\label{sec:svoboda:5.1}

\subsubsection{Recipients}

Although called “Guidelines for translation contractors”, the resources are used by internal staff, too. Not only that, some of the materials are even labelled “in-house” (style guides), which would suggest that they had been created for the principal use of internal staff.

\subsubsection{Structure and statistics}

As was mentioned above, there are 793 hyperlinks/references to all the 24 DGT Resource webpages. Interpreting the quantitative results\footnote{Cf. section Quantitative Overview of the Resources}, we see a rather misbalanced picture: The webpage with the highest number of resources (Lithuanian) features almost five times more items than the pages of the least resources available (Dutch and Polish), thus accounting for almost 10\% of all the resources from the resource pool. The lack of linked materials with NL and PL can be explained either by the fact that the sites are still under construction due to the recent overhaul of the Europa website or, otherwise, by lack of resources, thus a low priority attributed to the issue of SGs/TMs. The cases of RO and BL ranking 3\textsuperscript{rd} and 5\textsuperscript{th} respectively might be due to the fact that the countries joined the EU quite recently and had a wealth of sources available at entry, which they later complemented with own materials. Another stimulus for an increased production and representation of SGs/TMs might also be poor experience with contractors. However, should the attempt be undertaken to explain the disparity between language departments by hinting to the year of accession to the EU alone and a degree of possible resource saturation resulting from it, they would be proven wrong easily: there can hardly be a better example but the 30–40 band with Estonian, Danish, Czech, Croatian, Spanish, and German: Although each country joined the EU in a different year (2004, 1973, 2004, 2013, 1986, 1957, respectively, while, obviously, Germany was among the founding member states), they still offer a very similar number of resources.

Other possible explanations come to mind\footnote{For the following hints I owe my appreciation to Łucja Biel’s comments. Other suggestions are included in \sectref{sec:svoboda:6}, Limitations of the study and outlook for further research.}, one of which is the idiosyncratic factor (i.e. outside the scope of the top-down coordination/harmonization effort), another is different approaches to collecting/presenting the material (e.g. the German in-house style guide is long and incorporates some information which, on other language departments’ webpages, is scattered across several documents). 

Ten out of the 24 Resource webpages come together in the 20–30 tier, and the share of this band within the entire resource pool accounts for a third of the resources (32\%). This suggests that should there be a further requirement for page layout uniformity and resource availability, a common structure of the pages could involve 11 common resources featured more or less across the board, complemented by some 10+ language department-specific resources. 

\subsubsection{Content}

As regards the content of individual pages, it was found that the distinction between the pages, which have \textbf{all three subheadings, and those that have less or none}, does not follow a pattern – neither according to the year of accession (e.g. 2004, when 10 new member states joined the EU), nor according to the status of the language represented (e.g. procedural vs. non-procedural languages). Thus, the fact that some subsections are missing in one page and are present in another may simply be linked with the (un)availability of specific resources for specific languages. This, in turn, is due to language policy and other demand generating aspects, such as cooperation with contractors, the perceived need on the part of members of the service, etc. Another and quite practical reason might be differences in human resources available to maintain the webpage.

The analysis of \textbf{the most harmonized sections} shows that Section II (Contractor guides) shares three subsections and three categories in 100\% within the resource pool. Section I (General EU information) shows two shared subsections and three categories represented across the board. This proves a harmonized approach in a number of areas, in a top-down attitude. The \textbf{seven categories that are represented 100\%} across all the language specific Resource pages signify the importance attributed to them by decision-makers; the categories include references to EU institutions and DGT departments, terminology resources and the Interinstitutional Style Guide (IISG). It comes as no surprise that it is Section III (Language-specific information) that shows the greatest variation – i.e. the least harmonization with only 5 categories, none of which is represented fully across the language-specific pages.

The study has also shown that in terms of \textbf{cross-language alignment} (i.e. similar sources featured at similar locations across language-specific pages), the structure of individual Resource pages is almost always kept homogenous. Although, admittedly, there are exceptions (sources shown at other locations within the page structure when comparing language versions), they do not pose a major challenge for the user, as hardly any user (apart from researchers) would be looking for resources across a great many of individual pages.

At first glance (taken together), the Resource website shows a strikingly even representation of \textbf{non-unique sources} (those featured at least twice across the board) and \textbf{unique sources} in a ratio of 50/50\%. However, a closer analysis of individual pages reveals that the majority show between 10 and 19 unique sources (9 pages out of the 24) while for example Swedish and Lithuanian top the list with 50+ such links each. This shows yet another aspect of \textbf{variation} among the Resource pages.

The \textbf{link text analysis} shows that the resources are largely Europe bound; in the majority of cases (50\%), they refer to the EU official site (europa.eu). As regards the percentage of linked files vs. site links, almost half (46\%) of the links refer to files, whereas the rest refers to other websites (where subsequent links are likely to be found). 

When the \textbf{content was extrapolated based on the hyperlink text,} it was observed that where there are frequent occurrences of a text sequence, it reflects the standard structure of the individual pages. As regards the description of individual sources and their (tentative) binding nature, they are listed under varying labels. The list ranges from tips, \textit{How-to} manuals, advice, recommendations, and checklists to templates/models, guidelines, requirements, instructions, rules, conventions, norms, and even a “decree”. The most frequently used expression to denote the nature of a resource is “guides”.

Terminology sources are listed under highly varying labels, too, that, arguably, are used as synonyms quite often. Quantitatively, the most frequently represented sources are guides/guidelines, followed by terminology resources and information sites and further references.

\subsection{Findings related to quality}\label{sec:svoboda:5.2}

Translation quality is explicitly mentioned only once on the DGT’s homepage under “Clear writing — translation quality”. This does not mean that quality was of little concern to the service:\footnote{In fact the opposite is true, cf. the publications section.} users only need to proceed further to tap into the information available.

It has been shown (see above) that major areas of interest when it comes to translation product quality (information/reference sources, harmonization of terminology and style guides) are covered substantially in the Resources website. Somewhat surprisingly, though, the analysis of explicit referrals to the quality aspects among the collection of resources shows that the expression “quality” occurs only 9 times overall, which accounts for one percent of all the links referred to. Moreover, these nine occurrences include only three types of linked resources as one of the three resource types is present across 7 language versions.

\subsection{Similarity and difference as to the survey of 2013}\label{sec:svoboda:5.3}

With the overhaul of the EU’s web, the concept of lean pages has been introduced, which is the rule of thumb today and it is useful when accessing the site from a mobile device. However, much detailed information had to be sacrificed to this leanness as the comparison to a previous study on SGs/TMs shows (cf. \citealt{Svoboda2013}). As regards organization and presentation of the material, the DGT resources still show a significant variation in terms of both the extent and the topics of material represented – this being an observation, not a judgement.

What is striking is the current lack of multilingual information presented on the pages. The Resources web is now fully Anglophone, even the pages concerning language-specific issues are presented in English only\footnote{There are just a handful of resources that are entitled in a language other than English, e.g. “Brocardi e latinismi”, “Gemeinsames Handbuch”, “Gwida Prattika Komuni”, “Moniteur belge”, “Rete per l’eccellenza dell’italiano istituzionale”. On the other hand, some resources are available in two or more languages indeed, yet the respective page is still in English only. What is also user unfriendly to a certain degree, is the fact that there is neither a link to DGT homepage, nor a possibility to go to the resources page, once the user has landed on a page offering the download of a specific resource document (e.g. when following a direct link and, hence the Go Back function is not available).}. Previously, there was an option to switch between at least two languages (typically English and the language of the Resource page concerned, e.g. English and Czech). The present state is a paradox given the EU’s proclaimed commitment to multilingualism and the DG’s merit (serving the multilingualism policy). Moreover, the pages are typically entitled “Guidelines for contractors translating into [language]” and serve users translating not only from English, thus there might be a considerable language barrier for users with little or no command of English. However, for the purposes of this research, the current extensive survey would be impossible, if the pages had remained localized in the source languages. The reason is that much information used to be incomparable on the old web as not all items used to be translated and the endeavor to have all the language specific references translated into one language of comparison first would prove impracticable under the present study design. What has improved substantially, though, is the user experience when navigating through the pages and a relatively easy-to-conceive structure of the pages. 

\subsection{Considerations as regards the initial assumptions}\label{sec:svoboda:5.4}

In \sectref{sec:svoboda:methodology}, Methodology, four assumptions have been presented given the fact that they concern the main area of interest here, i.e. SGs/TMs as quality assurance indicators. Here, they will be re-iterated and compared to the research findings. The \textbf{first assumption}, i.e. that institutional translation is a field characterized by standardization, is confirmed for DGT, when taken to mean a top-down approach, and when limiting it to the aspect of SGs/TMs. This is true for a number of reasons: (i) The existence of a resource site, especially in an institution, such as DGT, with multiple language departments, bears witness to a concerted (managerial) standardization effort. (ii) DGT shows a trend of standardizing even the guiding/standardization sources (compared to 2013, the number of sources has been reduced – an entire section has been discarded – and the overall structure of the Resource website has been harmonized – now, the webpages are featured in English only). The \textbf{second assumption}, i.e. that the amount and the nature of references used as part of the translator resources will vary among Resource pages, was verified given the fact that 50\% of resources are of a unique nature (cf. \sectref{sec:svoboda:5.1} above), there exists a certain (albeit limited) degree of variation as regards cross-language alignment of the sources and there is considerable variation when referring to a certain type of resource (cf. the numerous types of guides/guidelines and terminology resources). \textbf{Premise no. 3} (the resource material will relate to reference materials, terminology, style/style guides) has been verified for DGT in full (cf. \sectref{sec:svoboda:5.1} with the following areas that are 100\% harmonized across language versions: references to EU institutions and DGT departments, terminology resources and the IISG).\footnote{The analysis of harmonized sections suggests that they had been identified in a managerial approach, which, at the same time, reflects the fact that the sources were considered mission-critical for the institution.} \textbf{Premise no. 4} (The quality aspect of institutional translation is governed by rules) could be verified only indirectly for DGT, using the given research corpus. On the face of it, the Resource website features very many guides and instructions (i.e. rules) concerning a large array of topics, yet these rules are explicitly linked to quality in only 1\% of the resource titles (cf. \sectref{sec:svoboda:5.2}). This would suggest that there is no direct link between rules and the quality requirement. On the other hand, abandoning the explicit side of things and involving quality related features (such as harmonization of terminology and style, using recommended technology and information sources, etc.) into a broader aspect of quality assurance procedures, the link between the wealth of the DGT resource (i.e. rules which deal with exactly the above instances and aspects) and the quality aspect becomes obvious.

\section{Limitations of the study and implications for further research}\label{sec:svoboda:6}

The limitations include data collection and data processing accuracy, since, to a considerable extent, these processes took place manually and the resource pool was rather extensive. Any content analysis presented here was carried out on the basis of the surface representation of links and/or link descriptions, not based on any in-depth analysis of the individual sources (this means that, under the present research study design, hardly any of the almost 800 source pages of this world’s largest translation resource were actually accessed and consulted in terms of their actual content). Further investigation is needed into the responsibility structures for maintaining the resource pages; valuable information for the explanatory part would certainly be gained from interviewing the stakeholders. Likewise, it was outside of the scope of this study to actually trace the implementation of the SGs/TMs in translation products and their impact on actual translation product quality. Prospects for further research include corpus analysis based on the surface texts shown in individual webpages, in order to identify recurring keywords and further textual arrangements. In future studies, more detailed analyses of the individual materials will be necessary as well as conducting similar surveys of a general nature on translation departments of other institutions (EU and other) to show similarities, patterns, and differences and to point to specific and recurring phenomena, which will place the present data in a broader perspective.

\section{Conclusion}\label{sec:svoboda:7}

The intention of the present chapter was to present quantitative data on the number and extent of the resources available especially to DGT contractors, to categorize the material and to find significant representations in terms of the content and its implicit or explicit reference to the quality aspect. It has been observed that the Resource webpages are largely structured in a uniform way. They testify to the effort invested by the DGT service into standardization and harmonization of its translation process and products and, as a consequence, its goal to maintain and support process and product quality. On the other hand, despite this clearly documented goal, the DGT resources still show a significant variation in terms of both the extent and the topics of material represented. Nevertheless, the wealth and level of detail of the TMs and SGs represented in the DGT Resources website illustrate some of the challenges of the so-called institutional EU translation as a service: for their translations to be considered high quality, the translators (both contractors and in-house translators) have to follow very many recommendations and instructions.

Translation manuals have accompanied major translation projects in the history of translating (cf. \citealt[142]{Kang2009} and other research). First and foremost, they represent the prescriptive approach to regulating translator choices. As such, they are key to translation practice and, in effect, to Translation Studies researches, who study such practice. This is particularly true for the field of institutional translation. Translation Studies scholars should pay attention to TMs/SGs for a number of reasons. From the diachronic point of view, they are an invaluable account of a translation team’s deliberations and choices over time as their shared “institutional” memory. From the synchronic viewpoint, they offer a backdrop for evaluation of existing translation products within communication processes, where TMs/SGs are to be observed. If it is true that TMs/SGs are constitutive to the notion of translation quality and, in effect, to institutional translation in the supranational/international contexts (cf. \sectref{sec:svoboda:2} above), they need to be understood in more detail. Consequently, further studies into this area are needed to understand better and compare the practices at other institutions and in other settings. Such understanding will, in turn, contribute to singling out the specifics of the translation process/products in the field of institutional translation, and help distinguish this particular field within the discipline of Translation Studies at large.

 
% \section*{Abbreviations}
% \section*{Acknowledgements}

\sloppy
\printbibliography[heading=subbibliography,notkeyword=this] 
\end{document}