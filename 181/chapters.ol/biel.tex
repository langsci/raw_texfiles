\documentclass[output=paper]{langsci/langscibook} 
% \ChapterDOI{} %will be filled in at production

\author{Łucja Biel\affiliation{University of Warsaw}}
\title{Quality in institutional EU translation: parameters, policies and practices}
\shorttitlerunninghead{Quality in institutional EU translation}
\abstract{Over the last decade EU institutions have raised the profile of quality, as evidenced by an increased number of translation policies and guidelines. The objective of the chapter is to identify the quality parameters of EU translation by synthesizing and evaluating institutional policies and practices on the one hand and the academic literature on the other hand. Two interrelated dimensions will be distinguished: the quality of translation at the textual level and the quality of processes in translation service provision. The former covers two dimensions — equivalence and textual fit/clarity while the latter covers the management of people, processes and resources, as well as the availability of translations. One of the promising developments is the reframing of quality discourse by the explicit linking of translation quality at the textual level to genre clusters, with a shift of focus from equivalence to clarity. On the other hand, such classifications may be seen as being triggered by the need to prioritize documents, as part of the fit-for-purpose approach, in order to prudently manage resources and costs in line with the required level of quality. 

Translation has been at the core of the European integration, being its key enabler and facilitator from the very beginnings of the European Coal and Steel Community in the early 1950s. Because of the significance and scale of European Union (EU) translation, its quality is a fundamental concept which has gained increased attention in EU institutional policies, increasingly so in the last decade. Challenging some fundamental concepts of Translation Studies, such as the source text, target text, translation process, etc. (cf. \citealt{Biel2014}: 59ff), EU translation also challenges our understanding of translation quality. As emphasized by \citet{Dollerup2001}, the quality of EU translation tends to be evaluated “by criteria which do not really apply” (285) and “it is not translation alone that makes the product” (290), but a complex array of political, ideological and procedural factors. The objective of this chapter is to identify the quality parameters of EU translation by critically analyzing institutional policies and practices and by synthesizing the existing literature and viewpoints from within EU institutions and from the outside. The chapter will first define and categorize EU translation, map quality in the EU context and its components, making a fundamental distinction between the quality of translation as a product and quality of processes in translation service provision, and identifying key threats and challenges to quality.
}
\maketitle

\begin{document} 


\section{Mapping the field — what is EU translation?}\label{sec:biel:1}

EU translation is a multi-faceted, broad and fuzzy category which may be defined as translation rendered by and for European Union institutions. In the most prototypical sense, it is translation provided in-house by the translation services of EU institutions. However, EU translation also covers translations outsourced to external contractors and paid for and controlled, to some extent, by EU institutions.

As a consequence of having institutions as agents/commissioning entities, EU translation is naturally placed in a broader superordinate category of \textbf{institutional} translation. Institutional translation is described as self-translation because translation is institutions’ means of ‘speaking’” to the outside world and, hence, “the institution itself gets translated” \citep[22]{Koskinen2008}. Since institutions regulate and control behaviour, monitor compliance, and create a shared cognitive background (Scott, qtd. in \citealt{Koskinen2008}: 18), institutional translation is affected by institutional norms (\citealt[50]{Koskinen2000}; \citealt[65]{WagnerEtAl2002}; \citealt[101]{Felici2010}), institutional patterns of behaviour \citep[144]{Kang2011}, and institutional culture of translating (\citealt[470]{Mason2004[2003]}). 

By extension, and in view of the fact that the European Union is a supranational political union, EU translation may also be classified as \textbf{political} translation (cf. \citealt[147]{Trosborg1997}) or even narrower as diplomatic translation because many EU documents are a result of complex delicate negotiations and political compromise between the Member States\footnote{See \citet[44]{Šarčević2007} on EU law as \textit{droit diplomatique} for the same reason.}.

EU translation may also be defined through genres or text types which are translated. EU translation is often perceived stereotypically as legal translation. While legal translation, that is the translation of EU legislation and case law, is a special constituent \textit{category} of EU translation which is critically central to the functioning of the European Union, it comprises a salient (by no means silent) minority of documents translated by EU translation services. EU translation is diverse and covers a continuum from expert-to-expert to expert-to-lay communication (cf. \citealt[56]{Biel2014}). In addition to law, specialized (expert-to-expert) genres include official communications, institutional reports, minutes, international agreements, whereby institutions communicate with experts, such as national governments and MEPs. In expert-to-lay communication institutions communicate with the general public, e.g. citizens, through such genres as booklets, letters to citizens, press releases, as well as multimodal genres, such as institutional websites or tweets. 

\section{Translation quality as a scalar and dynamic concept}\label{sec:biel:2}

One of the well-known definitions of quality, borrowed from marketing, is the degree to which a product or service meets clients’ needs, expectations and specification \citep[146-148]{Kotler2006}. This view is also valid in the context of translation\footnote{For a comprehensive overview of theoretical and professional approaches to quality see e.g. \citet{Drugan2013} and \citet{House2015}.}, where the quality of translated texts is perceived as a gradable rather than a (good-bad) bipolar concept (cf. \citealt[15]{Biel2011a}). The perception of quality has recently evolved\footnote{See also  \citet{PrietoRamos2015} for a detailed discussion of quality models in legal translation.} in Translation Studies, especially with the advent of machine translation and crowdsourcing (cf. \citealt{Jiménez-Crespo2017}), shifting from the negative approach of error counting to the evaluation of translation against readers’ expectations, intended purpose and other communicative factors (\citealt[238]{Colina2009}; see also \citealt{Vandepitte2017}, an introductory chapter to this book, which discusses quality-related concepts in more depth). Quality has been reconceptualized as a dynamic negotiated concept which comprises varied degrees of quality, depending on a number of factors, including fitness for purpose, utility, time and price (cf. \citealt[478, 482]{Jiménez-Crespo2017}). This understanding of quality is also evident in industry standards, e.g. \citeauthor{ISO2015}: 2015 and \citeauthor{EN2006}: 2006, which see quality of translations as suitability for the agreed purpose (cf. \citealt{Biel2011b}). The dynamism and relativity of the concept of quality are also evident in its varied perceptions and understanding among stakeholders, e.g. translators, requesters, end users, due to their different expectations and needs (cf. \citealt[142-143]{Strandvik2015}). Key dimensions of translation quality will be discussed in the next section. 

\section{Quality of EU translation}\label{sec:biel:3} 

Thus, the concept of quality is multidimensional and it is proposed in the context of institutional EU translation to distinguish two interrelated and overlapping dimensions: the quality of translation at the textual level with \textbf{translation viewed as a product} and the quality of processes in translation service provision with \textbf{translation viewed as a service}. The former covers two dimensions — equivalence and textual fit/clarity while the latter covers the management of people, processes and resources. It should be noted that the quality of translation service provision strongly affects the quality of translation products.

\subsection{Quality of EU translation at the textual level: translation as a product}\label{sec:biel:3.1}

As aptly observed by \citeauthor{Strandvik2015}, quality is “the sum of a number of different quality characteristics which may need to be ranked in order of priority or may even be contradictory” \citeyear[142]{Strandvik2015}. Such quality characteristics are well visible, inter alia, in the European Commission’s specifications addressed to potential external translators: “[translations] can be used as they stand upon delivery, without any further formatting, revision, review and/or correction”\footnote{Omnibus,  Tender Specification, p.8, \url{https://infoeuropa.eurocid.pt/files/database/000064001-000065000/000064078_2.pdf} (Accessed 2017-07-01).}, which is phrased in terms of accuracy, consistency and clarity:

\begin{itemize}
\item 
“complete” (no omissions or additions)
\item 
“accurate and consistent rendering of the source text”
\item 
correct references to any already published documents
\item 
internal terminological consistency and consistency with reference materials
\item 
clarity, relevant register and observance of text-type conventions
\item 
no language errors and correct formatting
\item 
compliance with instructions\footnote{Omnibus, op. cit., p. 8-9.}.
\end{itemize}

It is worth noting that compared to 2008\footnote{Cf. “the target text is a faithful, accurate and consistent translation of the source text”, \citealt[6]{DGT2008}}, the word ‘faithful’ disappeared from the descriptors of quality targeted at external contractors. The above list may be regarded as a set of characteristics to be possessed by a high-quality translation in the EU context (see Hanzl and Beaven, this volume, for a similar set of descriptors at the Council). 

The quality of EU translation products is posited here, with some adaptations after \citet{Chesterman2004} and \citet{Biel2011b, Biel2014}, to comprise the following fundamental variables:

\begin{itemize}
\item 
Dimension 1: \textbf{Equivalence} of translation in relation to the source text (fidelity, accuracy of information transfer), in relation to other language versions (multilingual concordance) and in terms of \textbf{consistency/continuity} with preceding and/or higher-ranking texts,
\item 
Dimension 2: \textbf{Textual fit} (naturalness) of translation in relation to corresponding non-translated texts produced in the Member States, as well as the interrelated concept of \textbf{clarity} (readability) of translation.
\end{itemize}

I will illustrate below how these two dimensions are prioritized in the EU institutions’ translation policies and practices.

As a result of a more dire need to balance translation costs, demand and quality triggered after the last waves of accessions, recent years have observed an intensification of documents on translation quality coupled with a substantial renarration of translation quality discourse in EU institutions, in particular in the European Commission and the Council. The new narrative has downplayed faithfulness (that is the equivalence dimension) as the key characteristics of high-quality translation by reorientation towards more functional categories of ‘\textbf{fitness for purpose’}, with occasional explicit references to \citeauthor{ISO2015} (e.g. \citealt[3]{DGT2015j}), and consequently by linking quality requirements to genres grouped into a number of clusters (I will refer to it as the \textbf{genre-based approach}; however, it should be borne in mind that EU institutions do not frame it as such), which has put the concept of \textbf{clarity} to the fore (dimension 2). This reworking of the concept of quality should be evaluated as a proactive move which prioritizes relevant characteristics of quality, depending on a communicative purpose behind a genre, rather than relying on a stereotyped perception of EU translation as legal only. It is worth stressing that the new approach is in line with the prominence of the concept of genre in Translation Studies (cf. \citealt{Biel2017}) %\textit{forthcoming}).

EU institutions have attempted to classify documents to prioritize their translation and revision for more than a decade (cf. \citeauthor{EuropeanCourt2006}’ report 2006); however, recent years have brought more pronounced narratives linking quality levels with document types. This is especially visible in the European Commission’s \citeauthor{DGT2015j}, which evolved “towards a more conscious, structured and systematic approach to quality assurance” (\citealt{Strandvik2017}, forthcoming). Its quality advisers drafted a document called \textit{DGT Translation Quality Guidelines} in 2015, followed by a summary version for external contractors \textit{Translation Quality Info Sheets for Contractors} in 2017, where they introduce four categories of texts and link quality requirements and control to genre clusters and related risks (see also \citealt{Strandvik2017forthcoming}):


Category A: Legal acts;
\begin{itemize}
\item 
EU legal acts; 
\item 
documents used in administrative or legal proceedings and inquiries; 
\item 
documents for procurement or funding programmes, tenders, grants applications, contracts; 
\item 
recruitment notices, EPSO (European Personnel Selection Office) competition notices and test documents;
\end{itemize}

Category B: Policy and administrative documents

\begin{itemize}
\item 
accompanying documents not formally part of legal acts; 
\item 
white and green papers; 
\item 
other official administrative documents, e.g. budget, reports, guidelines.
\end{itemize}

Category C: Information for the public

\begin{itemize}
\item 
press releases, memos; 
\item 
articles to be published in the press, speeches, interviews;
\item 
leaflets, brochures, posters;
\item 
web texts. 
\end{itemize}

Category D: Input for EU legislation, policy formulation and administration – eight subcategories of incoming documents from Member States, other stakeholders, citizens and non-EU countries and external bodies.

The highest quality requirements are expected of Category A and within this genre cluster – multilingual legislation. Quality descriptors foreground the equivalence dimension (‘legal accuracy’, \citealt[6]{DGT2016b}) to ensure uniform interpretation and application (cf. \citealt[72]{Šarčević1997}) and the same result in all the official languages (cf. \citealt[1191]{Pozzo2012b}), which is also referred to as ‘multilingual concordance’ \citep[4]{DGT2016b} and the ‘horizontal dimension’ \citep[44]{Robertson2015}. In the context of multilingual law, equivalence relations are highly intertextual and complex — equivalence is presumed to exist between all language versions of a legal instrument, i.e. between the original and the target text, but also between translations into other languages. Equivalence is ‘proclaimed’ \citep[11]{Hermans2007} as ‘an a priori characteristic’ of EU translation \citep[49]{Koskinen2000} — it is ‘existential equivalence’ \citep[51]{Koskinen2000} and mandatory legal equivalence (\citealt[180–181]{Tosi2002}) due to the fact that all the language versions are presumed to have the same legal value regardless of whether or not they are equivalent as to their meaning (cf. \citealt[180]{Tosi2002}). This presumption is known as the principle of equal authenticity \citep[64]{Šarčević1997} and the principle of plurilinguistic equality \citep{vanEls2001}. The quality of multilingual legislation is also required, to a lesser degree, to take into account the textual fit and clarity dimension. Clear language\footnote{\textit{Legislative Drafting. A Commission Manual} goes as far as to include ‘reader-friendliness’ in the legislative checklist \citep[78]{EuropeanCommission1997}.} and language quality are explicitly mentioned as quality requirements with reference to legislation (cf. \citealt[6]{DGT2016b}). They are to ensure its accessibility, predictability and legal certainty \citep[146]{Strandvik2015}. As pointed out by \citeauthor{Strandvik2015}, there is a general consensus among the legal services of EU institutions that “all language versions of a piece of legislation should deviate as little as possible from the target cultures’ drafting conventions” (\citeyear[153]{Strandvik2015}, emphasis added). On the other hand, DGT Translation Quality Guidelines admit that there is “only limited leeway for ‘localising’ Category A” to target language (TL) conventions \citep[14]{DGT2016b}. Furthermore, it should be pointed out that the requirement of minimum deviations from TL conventions may be difficult to satisfy due to the high hybridity of EU language, resulting from a number of factors, such as the complex multilingual multi-stage drafting process intertwined with translation \citep[360]{Doczekalska2009paradoxes}, fusion of languages and the frequent involvement of non-native speakers \citep[76]{WagnerEtAl2002}, cultural neutralisation and hybridity of texts, unstable source texts \citep{Stefaniak2013}, quality of drafting (\citealt[184]{Tosi2002}, \citealt[22]{Šarčević2013}), preference for literal translation techniques \citep[54]{Koskinen2000}, as well as distortions typical of the translation process (cf. \citealt{Biel2014} for a detailed discussion). As a result, EU legislation has led to the emergence of distinct legal varieties of national legal languages (cf. \citealt{Biel2014} and the Eurolect Observatory project.\footnote{More information: \url{http://www.unint.eu/en/research/research-groups/39-higher-education/490-eurolect-observatory-interlingual-and-intralingual-analysis-of-legal-varieties-in-the-eu-setting.html} (Accessed 2017-07-01)}).

The highest clarity and textual fit requirements are set for Category C which comprises mainly informative texts addressed to the general public and EU citizens — expert-to-lay communication. Clarity is linked to the political objective of increasing the EU citizens’ confidence in the European Union, enhancing its positive image and creating more interest in EU matters (cf. \citealt[1, 12]{DGT2015j}). Quality is achieved through the high localization of translations to target language conventions, avoidance of jargon, naturalness and idiomaticity of translations: “a key quality \textit{desideratum} is to produce texts that read like originals in all languages” \citep[2, 13]{DGT2015j}. Translators have more freedom and are expected to provide translations which will function seamlessly in the target culture. 

Categories B and D are between these two extremes — they place more focus on accuracy than Category C but more focus on naturalness than Category A.

Quality indicators for translation products, which are connected with ex-post quality monitoring (cf. Hanzl and Beaven, this volume, with reference to the Council) and control, are explicitly linked to errors and the \textbf{correction rate} expressed through the number of corrigenda with the tolerance level set at < 0.5\%, e.g.:

\begin{itemize}
\item 
The correction rate is “the ratio between the number of translations formally corrected during one year and the number of translations of the same year and the preceding two years that can be subject to such corrections” (\citealt[4]{DGT2016a}, footnote 4).
\end{itemize}

The most critical type of error in the EU context is an error in EU legislation which leads to a different, unintended regulation of rights and duties of private and public entities in the Member States \citep[2]{Kapko2005}. Critical errors and corrigenda (cf. \citealt{Bobek2009}) are usually connected with inappropriate information transfer, that is — dimension 1 (equivalence, accuracy). It is also worth noting that, on the other hand, the descriptors of errors in the Commission have shifted from changes in information transfer (dimension 1) to impairment of product usability (dimension 2) (see \citealt{Strandvik2017} this volume).

To sum up, the fitness-for-purpose approach combined with the genre-based categorization of documents emphasize the cost-effective gradability of quality. On the one hand, they allow the institutions to manage their resources depending on the required level of quality — ‘very good’ or ‘good enough’: as observed by \citeauthor{Martin2007}, “the fit-for-purpose principle is an invaluable yardstick against which to balance risks and resources” \citeyear[60]{Martin2007}. On the other hand, they have contributed to foregrounding often overlooked characteristics in the EU context, such as naturalness and clarity.

\subsection{Quality of processes in EU translation service provision: translation as a service}\label{sec:biel:3.2}

The conceptualization of quality of translations through service provision rather than products is linked with market standards, such as \citeauthor{ISO2015}: 2015 and \citeauthor{EN2006}: 2006, and is part of quality assurance. The key characteristics of service provision are proposed in the EU context to cover:

\begin{itemize}
\item 
a prerequisite — availability of translations in EU citizens’ native languages,
\item 
workflow management,
\item 
people, 
\item 
translation resources (tools).
\end{itemize}

\subsubsection{Availability of translations as a \textit{sine qua non} condition: multilingualism and the selective translation policy}\label{sec:biel:3.2.1}

The \textit{sine qua non} condition for discussing the quality of translation as a service is the availability of translations in official languages, in particular the availability to EU citizens in their native language. This prerequisite stems from one of the EUs fundamental values protected in its primary legislation — respect for linguistic diversity\footnote{Article 3, Treaty on European Union (TEU), OJ C 326, 26.10.2012.}, and the resulting multilingualism policy which is intended to give citizens access to EU legislation and information in their native languages as long as they have the status of an official language. Above all, the multilingualism policy imposes an obligation to publish the Official Journal of the European Union in all the official languages, and in particular to ensure that regulations and “documents of general application” are available in all the official languages\footnote{Articles 5 and 6, Council \textstyleStrong{{Regulation No 1 determining the languages to be used by the European Economic Community,} }OJ 17, 6.10.1958.}. The multilingualism policy also enables Member States or citizens to write to EU institutions and receive a reply in one of the official languages and requires EU institutions to write to Member States and citizens in an official language of such a state/citizen\footnote{Articles 2 and 3, Council \textstyleStrong{{Regulation No 1, op. cit}}.}.

Currently, the EU multilingualism covers 24 official languages which are presumed to enjoy an equal status: Bulgarian, Croatian, Czech, Danish, Dutch, English, Estonian, Finnish, French, German, Greek, Hungarian, Irish, Italian, Latvian, Lithuanian, Maltese, Polish, Portuguese, Romanian, Slovak, Slovenian, Spanish and Swedish. Firstly, it should be noted that while the impressive scale of EU multilingualism includes speakers of 24 official languages, it does not include over 60 regional or minority languages and some co-official languages (e.g. Basque, Catalan, Welsh)\footnote{\url{https://europa.eu/european-union/topics/multilingualism_en} (Accessed 2017-07-01).}. Secondly, the presumption of the equal status of all the official languages is limited to the legal validity and authenticity of the EU-wide legislation translated into such languages. Furthermore, some languages formally enjoy a privileged status of procedural languages (as “members of an elite club” \citep[560]{Craith2006} — English, French and German, and/or a status of pivot languages for relay translation, e.g. in the Court of Justice — French, English, German, Spanish and Italian\footnote{It also extends to the preference of selected – mainly procedural - languages in some EU agencies. For example, the European Union Intellectual Property Office (EUIPO) has 5 ‘working languages’: English, French, German, Italian and Spanish.}. Starting with the accession of Scandinavian countries in 1995 and strengthened after the 2004 accession, English replaced French as the dominant procedural language in most EU institutions (a notable exception being the Court of Justice with French as the procedural language) and became the lingua franca of the European Union. As a result, most documents are drafted in English and are selectively translated into other official languages. The main reason for the selective policy is the allocation of insufficient funds by the EU Budgetary Authorities (i.e. the Council and the Parliament) to ensure translation into all official languages\footnote{Decision of the European Ombudsman on complaint 3191/2006/(SAB)MHZ against the European Commission, \url{https://www.ombudsman.europa.eu/en/cases/decision.faces/en/3248/html.bookmark} (Accessed 2017-07-01).}, as well as the increase in demand for translation in the last decade combined with the pressure on staff reductions in some institutions, which altogether evoke a strong need to prioritize categories of documents for translation (cf. \citeauthor[4]{DGT2016b}) and introduce structural demand-reducing measures (cf. \citealt{Strandvik2017forthcoming}).

This selective translation policy affects the availability of EU translations to citizens whose languages are underprivileged. Even though, as observed above, Council Regulation No. 1 of 1958 imposes an obligation to ensure that “documents of general application” are available in all the official languages, this seemingly broad term is in practice interpreted quite narrowly as other types of secondary legislation (in particular directives, some decisions), as well as case law and a few selected document types, as a compromise between demand, resources and costs, especially after the 2000s enlargements. The selective translation policy is referred to by the European Parliament itself as “controlled full multilingualism”\footnote{Cf. Article 1.2 of the European Parliament’s \textit{Code of Conduct on Multilingualism} of 16.06.2014: “The resources to be devoted to multilingualism shall be \textstylehighlight{controlled} by means of management on the basis of users’ real needs, measures to make users more aware of their responsibilities and more effective planning of requests for language facilities.” \url{http://www.europarl.europa.eu/pdf/multilinguisme/coc2014_en.pdf} (Accessed 2017-07-01).} 
or “a pragmatic approach”
\footnote{\url{http://www.europarl.europa.eu/sides/getDoc.do?pubRef=-//EP//TEXT IM-PRESS 20071017FCS11816 0 DOC XML V0//EN} (Accessed 2017-07-01).}. According to this policy full translation and interpreting applies only to the Parliament’s official documents and plenary sessions, while preparatory documents are translated only into languages which are actually needed. A similar policy is pursued by the European Commission, whereby legislation and key political documents are translated in all EU official languages, as well as general information on its EUROPA website, with the rationale being a legal requirement or ‘serious disadvantage’. Other documents are often translated into procedural language(s) only or those languages which are specifically needed — this applies in particular to correspondence with Member States or citizens, specialist information (technical information, campaigns, blogs, speeches, funding for research), news and urgent or ‘short-lived’ information\footnote{\url{http://ec.europa.eu/ipg/content/multilingualism/index_en.htm}. See also the 2006 \textit{Communication to the Commission} for an early categorisation of texts into groups which may be outsourced (\url{https://ec.europa.eu/transparency/regdoc/rep/2/2006/EN/2-2006-1489-EN-F1-1.Pdf}) (Accessed 2017-07-01).}. The (limited) choice of languages is framed as ‘evidence-based’ to be balanced with importance, cost-effectiveness, limited budget and human resources for translation\footnote{\url{http://ec.europa.eu/ipg/content/multilingualism/index_en.htm}, \url{https://europa.eu/european-union/abouteuropa/language-policy_en} (Accessed 2017-07-01).}. Obviously, this policy limits access to institutional information to speakers who do not know English and/or other procedural languages. The pragmatic approach is also adopted in the third largest EU institution — the Council. Its Language Service translates “almost all” legislation and “many major” policy documents into all official languages, admitting that “for efficiency’s sake” about 70\% of the Council’s total pages are not translated at all as “for practical purposes” the Working Parties tend to work on a text drafted in one language \citep[8]{GSC2012}.

This pragmatic language regime has been referred to critically by \citet{Krzyżanowski2011} as hegemonic multilingualism which may suppress national languages and disempower certain nations. \citeauthor{Mattila2013} goes even further and argues that the overuse of English as the main procedural language is indicative of unilingualism: “Despite the ideology underlining the multilingual character of the Union, one could speak of a development in the direction of unilingualism” (\citeyear[33]{Mattila2013}). Overall, the pragmatic approach, framed in the narrative of multilingualism and respect for linguistic diversity, reflects institutional policies connected with the realistic management of budgetary and human resources. 

\subsubsection{Quality of workflow management}\label{sec:biel:3.2.2}

The superordinate factor controlling the quality of translation as a service is workflow management — namely, how the provision of a translation service is managed against available resources. It ultimately contributes as a decisive factor to the quality of translation as a product. Management fundamentally affects the recruitment and allocation of human resources and the development of technical resources in light of budgetary constraints. It is also important for consistency of approach and for consistency of quality across and within institutions (cf. \citealt{DruganEtAlforthcoming}). 

At a more global level, quality can be affected by the organizational structure of translation service which prioritizes roles covering various aspects of quality assurance. It can be illustrated with the organization of the Directorate{}-General for Translation at the European Commission. As its organizational chart shows\footnote{As at 1.06.2017 \url{https://ec.europa.eu/info/sites/info/files/organisation_chart_translation_june_2017_en.pdf} (Accessed 2017-07-01).}, it is divided into six directorates, four of which (Directorates A to D) are in charge of Translation, including Directorate D which deals with procedural languages only, Directorate R in charge of Resources and Directorate S in charge of Customer relations. The Translation Directorates also subsume functions responsible for Quality Management, Language Applications and Interinstitutional cooperation. Directorate R/Resources covers new technologies, internal administrative matters, budget and finance, informatics and professional and organizational development while Directorate S manages customer relations, workflow systems, demand management, external translation, editing, evaluation and analysis and web rationalization task force. In particular, there is a need to balance demand management, budgetary resources, internal and external translation flows, as well as interinstitutional cooperation. The very existence of these functions points to their recognition as important (see also Prieto-Ramos, this volume, on translation service managers).

At a more local level, workflow management ensures quality control at the pre-translation, translation and post-translation stage. At the pre-translation stage, quality assurance mechanisms may involve planning, source file preparation (technically and linguistically through editing), terminology resources, translation resources, project management resources \citep[77-79]{Drugan2013}; in particular, the assignment of a job to a suitable translator \citep[23]{PrietoRamos2015}. At the translation stage in the institutional context, it may mean sufficient support for translators with terminology assistance, research by assistants, consultations with national experts, etc., as well as deadline management\footnote{See the \citeauthor{DGT2016b}’s recent commitment to ensure shorter deadlines for ‘political priority documents’ and to increase the deadline compliance rate (\% of pages produced within the deadline) from 95\% in 2009 to 99\% \citep[6, 9]{DGT2016b}.}. Workflow management also covers quality control and assessment, especially at the post-translation stage, including (bilingual) revision, (monolingual) review, random checks by quality officers, legal linguistic revision by lawyer linguists, editing of source texts by native speakers to improve their quality\footnote{For example, the European Commission plans to increase the editing of its major initiatives from 12\% in 2015 to 65\% in 2020 \citep{DGT2016b}.}, as well as strategic planning in relation to quality control, e.g. the introduction and monitoring of performance indicators, such as a customer satisfaction rate, deadline compliance rate or correction rate (cf. \citealt{DGT2016b}). 

\subsubsection{Quality of people: translators and support staff}\label{sec:biel:3.2.3 }

One of the key components of quality is the human resources involved in the provision of translation services\footnote{See \citet{Svoboda2008} on the human factor in the European Commission’s DGT.}. Such human resources cover translators, revisers, as well as supporting roles, including: linguistic assistants\footnote{See e.g. a notice of competition \url{http://eur-lex.europa.eu/legal-content/EN/TXT/HTML/?uri=OJ:C:2016:151A:FULL & from=EN} (Accessed 2017-07-01).}, terminologists (see Stefaniak, this volume), quality officers/controllers (see \citealt{DruganEtAlforthcoming}), and national experts. 

Translators can be divided into staff (in-house) translators and external translators (contractors). \textbf{In-house translators} are employed with the dual roles of translators and revisers. In-house translation services are available in the majority of EU institutions, e.g. in the European Commission, the Council, the European Parliament, the Court of Justice, the Economic and Social Committee, the Court of Auditors; while other bodies are services by the Luxembourg-based Translation Centre for the Bodies of the European Union. EU institutions have a long tradition of recruiting, training and managing translators. In-house translators were already employed in the High Authority of the European Coal and Steel Community established by the Treaty of Paris in 1952 and over the decades they substantially grew in numbers and raised their status, becoming permanent official in the late 1950s  \citep[18]{European2010}\footnote{\textstyleStrong{{Protocol (No 7) on the privileges and immunities of the European Union to the Treaty of Rome, 1957.}}}). In light of the highly specialized nature of texts, the role of translators’ specializations have been growing in importance since the 1990s \citep[12, 13]{European2010}. In-house translators are employed by most institutions after they pass the EPSO\footnote{https://epso.europa.eu/.} competition and meet the requirements specified in the competition notice. In general, candidates have to meet the following requirements: a bachelor’s degree and a perfect command of their mother tongue (C2 level) and two official EU languages (C1 level), of which at least one should be a procedural language\footnote{\url{https://ec.europa.eu/info/jobs-european-commission/translator-profile_en} (Accessed 2017-07-01).}. Interestingly, no professional experience is officially required\footnote{NB: notices of competitions explicitly state “No professional experience required”, e.g. \url{http://eur-lex.europa.eu/legal-content/EN/TXT/HTML/?uri=OJ:C:2016:205A:FULL & from=EN} (Accessed 2017-07-01).}; however, the translator’s profile at the European Commission’s website explicitly mentions thematic skills required to deal with political, economic, financial, legal, scientific and technical texts\footnote{\url{https://ec.europa.eu/info/jobs-european-commission/translator-profile_en} (Accessed 2017-07-01).}. The procedure comprises three stages: (1) computer-based multiple-choice question tests on verbal, numerical and abstract reasoning tests in language 1 (L1) and comprehension L2 and L3; (2) two translation tests into L1 — which are usually general but very idiomatic in nature and (3) three tests in L2 in the assessment centre (oral presentation, competency-based interview and group exercise) to test general competencies, such as analysis and problem solving, communicating, delivering quality and results, learning and development, prioritizing and organizing, resilience, working with others, and leadership\footnote{See e.g. Notice of open competitions: \url{http://eur-lex.europa.eu/legal-content/EN/TXT/HTML/?uri=OJ:C:2016:205A:FULL & from=EN} (Accessed 2017-07-01).}. At a first glance, the requirements may not seem excessively strict (e.g. no previous translation experience required and a general text to translate); however, due to financially attractive job prospects, the competition is tough and good candidates tend to be preselected. In-house translators are subject to continuous professional training to deepen their subject matter expertise and acquire new language skills, including  joint interinstitutional training events \citep[8, 15]{DGT2016b}. In line with the European Commission’s commitment, DGT planned to reduce its staffing levels by the end of 2016 by 10\% compared to 2012 \citep[14]{DGT2016a}.

The requirements are higher in the Court of Justice of the European Union which employs only \textbf{lawyer-linguists} as in-house and external translators (see Koźbiał, this volume). Lawyer linguists are required to have a law degree and a good command of three official languages (however, no formal education in languages is required) and like translators, they have to go through the EPSO competition. It is worth noting that while the Commission does not require its translators to have a legal background to translate EU legislation; however, it employs lawyer linguists to check translations, the Court does — it employs lawyer linguists to translate judgments and other court documents. Lawyer-linguists’ tasks differ across institutions and their role could be best defined in much broader terms as legal-linguistic revision. The term ‘lawyer-linguist’ is now used across all institutions; however, they used to have distinct names: legal revisers in the European Commission, jurist-linguists in the Council of the European Union and reviser lawyer-linguists in the European Parliament \citep[186, 188, 189]{Šarčević2013}. Legal-linguistic revision has a broader scope than the typical bilingual revision and may include a revision of the source text, linguistic and legal consistency check of a target text with other language versions as well as an occasional check of all language versions for consistency \citep[186]{Šarčević2013}\footnote{See also an example of a notice of open competitions for lawyer linguists \url{http://eur-lex.europa.eu/legal-content/EN/TXT/HTML/?uri=OJ:C:2016:457A:FULL & from=EN} (Accessed 2017-07-01)}. Lawyer linguists are also involved in the early interventions of drafts at the pre-translation stage to facilitate their translation into all the official languages \citep[187]{Šarčević2013}. 
First \textbf{support staff} included typists, stenographers and revisers in the 1950s while the role of the terminologist emerged in the Commission in the 1960s \citep[12, 21]{European2010}. Terminologists are involved in ad-hoc terminology work to support translators on the job, usually more difficult specialised terms, as well as they are involved in systematic terminology work, which consists of creating term records in term bases (see Stefaniak, this volume). Linguistic assistants assist translators and lawyer-linguists in translation and revision by pre-processing or post-processing texts in IT tools, databases and templates, acting as IT helpdesk/coordinator, managing linguistic and legal-linguistic information and documentation (reference documents, maintaining resources, updating translation memories, compiling information, and corresponding with national experts), incorporating changes in legislation\footnote{Cf. \url{http://eur-lex.europa.eu/legal-content/EN/TXT/HTML/?uri=OJ:C:2016:151A:FULL & from=EN} (Accessed 2017-07-01).}.

\textbf{External translators} are selected through open calls for tenders\footnote{It is worth noting that EU tender procedures and specifications in respect of translations are regarded as good practices due to the significant role of quality criteria in addition to price (cf. \citealt{Wołoszyk2017}).} and include both freelancers and translation agencies. The discussion below is illustrated with the Commission’s procedures. Award criteria are based on the “most economically advantageous tender”\footnote{Cf. Omnibus, op. cit., p. 16.}. Tenderers have to evidence that they and/or their translators, revisers and reviewers have: (1) the required level of tertiary education — usually, a Bachelor’s degree in any area; and (2) proven translation experience in the domain required in the specific language combination, e.g. 3000 pages over the period of 3 years\footnote{Cf. Omnibus, op. cit., p. 21.}. Contractors sign 1-year framework contracts which may be renewed for further three 1-year contracts (a total of 4 years). Contractors are offered orders in the order as they appear in the ranking which is re-ranked on a monthly basis according to the average quality of translations in the previous month, based on in-house evaluation (the so-called dynamic assessment system)\footnote{Cf. \url{https://cdt.europa.eu/en/dynamic-ranking} (Accessed 2017-07-01).}. External translations are usually reviewed, in most cases some parts of it only, e.g. the EC’s DGT revises only 10\% of the document, from 2 to 10 pages \citep[17]{DGT2012}, even though it used to revise entire texts until recently (\citealt{Strandvik2017}, forthcoming). The dynamic ranking of external contractors should be viewed positively as a step forward in controlling and assuring the quality of external translations. After the award of tender, some initial period of unstable and unpredictable quality may be expected which should level out after a few re{}-rankings, allowing institutions to identify underperforming contractors, who naturally fall down in the ranking. For example, after the award of the Omnibus tender in 2016, the Commission’s DGT experienced a fall of ‘very good’ and ‘good’ marks on external translations from the very high level of  94\%\footnote{It is worth noting that before 2016 the freelance quality was consistently growing from 91\% in 2011 to 94\% in 2015; cf. DGT’s \textit{Annual Activity Report 2015}. Ref. Ares(2016)1818629 – 18/04/2016, p. 8.} in 2015 to 87\% in 2016 as well as non-compliance with deadlines, the problems which were addressed through remedial measures, such as penalties and contract termination\footnote{DGT, European Commission. \textit{Annual Activity Report 2016}. Ref. Ares(2017)1826615 - 05/04/2017, pp. 4-5, 9}.

On the positive side, it should be highlighted that the criterion of quality has gained in importance over the years. First, by raising the quality level required of external translators through the replacement of the ‘Acceptable’ mark (6/10) with the ‘Below standard’ descriptor and the ‘Below standard’ (4/10) with ‘Insufficient’\footnote{European Commission, 06.2016, \textit{Instructions for Users,} pp. 41-42.}. Secondly, the mechanism has evolved to give more weight to quality over price — currently 70/30\footnote{Omnibus, op. cit., p. 22.} (e.g. a change from the first 50/50 and next 60/40). Yet the excessively high price competition on some markets, especially from newcomers, has reduced the prices to the level which has driven some more experienced contractors away. 

To illustrate this claim, we can analyze the price ranges in contracts for the translation of EU texts relating to the policies and administration of the European Union (OMNIBUS-15)\footnote{Contract notice 2015/S 037-062226, \url{http://ted.europa.eu/udl?uri=TED:NOTICE:62226-2015:TEXT:EN:HTML} (Accessed 2017-07-01).} awarded in 2016 by the European Commission. What is most striking is the high variation of prices between the new accession countries and most of the EU-15 countries which have more mature freelance markets. For example, the highest prices apply to Gaelic (due to the shortage of translators) and Northern European languages — Swedish, Danish, Finnish, Dutch, e.g. EN-GA (26.45-60.23 EUR per page\footnote{1,500 characters of the source text, excluding spaces (Omnibus, op. cit.)}), EN-SV ({\euro}33.5-58), EN-DA ({\euro}32.45-56.63), with top prices reaching 60 EUR per page, while the lowest prices are offered by contractors from Romania, Bulgaria, Latvia, Croatia, Poland, Hungary, Czech Republic, e.g. FR-RO ({\euro}6-14.99), EN-RO ({\euro}8-24), RO-EN ({\euro}9-11.99), BG-EN ({\euro}10-12,99), HR-EN ({\euro}13-16.5 EUR), FR-PL ({\euro}12-13); DE-PL ({\euro}12-15), with the lowest price reaching 6 EUR, that is 10 times less than the top price\footnote{\url{https://ec.europa.eu/info/sites/info/files/omnibus_15_2015.pdf} (Accessed 2017-07-01).}. What is also notable is the low variation of prices for some language pairs, e.g. DA-EN ({\euro}33.9-37.8), FI-EN ({\euro}31.5-33.9), DE-DA ({\euro}44.75-46.75) and high variation for some other, e.g. EN-EL ({\euro}12.5-40), EN-MT ({\euro}11.5-48.5). It should be noted that even the highest external prices (the range of {\euro}50-60) are much cheaper than in-house prices, estimated in the Court of Auditors’ Special Report 2006 at {\euro}119 per page at the Parliament, {\euro}194 at the Commission and {\euro}276 at the Council in 2005.

The current and future policy of EU institutions is to significantly increase the involvement of external translators through outsourcing in order to reduce costs or, in some cases, to meet peak demand. Outsourcing practices differ across institutions; yet there is a discernible upward trend within the EU combined with in-house staff reductions. The outsourcing practice is not new and has been put into place to meet an insufficient internal capacity to meet translation requests. The rate of outsourcing was 22\% for the Commission, 33\% for the Parliament and 0\% for the Council for all languages in 2003; the rate of outsourcing in 2005 fell to 20\% for the Commission, increased to 36\% for the Parliament and amounted to 2\% for the Council \citep{EuropeanCourt2006}. The outsourcing trends were much higher for new languages (2004 accessions) than for the EU-15 due to delays in the recruitment process. As early as in 2003-2004, the outsourcing objective was set by the Commission and the Parliament to reach 40\% and 30\%, respectively \citep{EuropeanCourt2006}, but was suspended in 2004 by the Commission due to a fall in demand. The outsourcing trend is planned to increase in the coming years. For example, according to the European Commission DGT’s Strategic Plan 2016-2020, the outsourcing rate of the DGT is targeted to increase progressively from 27\% in 2015 to 37\% of total pages translated by DGT in 2020 \citeyear[11]{DGT2016b}. 

The upward outsourcing trend is associated with quality risk. In-house translation is much more expensive but is characterized, as argued by the \citeauthor{EuropeanCourt2006} in 2006, by higher quality: “the quality of internal translation is recognized to be higher” \citeyear{EuropeanCourt2006}. In-house translators can ensure a better contextualization of translations and enjoy the benefit of dedicated trainings, internal resources and better integration of such resources, as well as having insider knowledge. Secondly, it is also a well-known fact among freelancers that some agencies win tenders with experienced (and expensive) translators’ CVs but outsource actual work to cheaper and less experienced translators\footnote{See for example discussions on Proz.com: \url{http://www.proz.com/forum/business_issues/291854-agencies_that_ask_for_too_much_info.html\#2472785} (Accessed 2017-07-01).}, even though some measures to curb this phenomenon have been put in place recently. Thirdly, while most institutions claim that they outsource “non-priority” texts, e.g. see the European Parliament “Documents of the highest priority, i.e. legislative documents and documents to be put to the vote in plenary are, as far as internal resources permit, translated in-house. Other types of documents, especially administrative texts, are frequently outsourced.”\footnote{\url{http://www.europarl.europa.eu/pdf/multilinguisme/EP_translators_en.pdf} (Accessed 2017-07-01).}, it is not always the case (see \citealt{Strandvik2017forthcoming}, on a move towards outsourcing policy documents and legislation) and the situation varies from one language unit to another, depending on internal capacity.

\subsubsection{Quality of resources (tools)}\label{sec:biel:3.2.4}

In order to ensure consistency and the standardization of translations, EU institutions invest considerable funds in the development of technological, terminological and linguistic resources which support translators during the translation process. Such resources enable EU institutions to regulate and control the language and format of STs and TTs. They ensure terminological consistency, uniform institutional style and textual patterns in translation with a view to keeping variation and idiosyncrasy to the minimum (e.g. \citealt[70]{Biel2014}). Tools differ to some extent between institutions and include among others:

\begin{itemize}
\item 
terminological resources: IATE\footnote{http://iate.europa.eu}, EuroVoc\footnote{http://eurovoc.europa.eu};
\item 
databases of documents: EUR-Lex\footnote{http://eur-lex.europa.eu/}, Curia\footnote{https://curia.europa.eu/jcms/jcms/j\_6/pl/};
\item 
style guides: joint for all the institutions, e.g. \textit{Interinstitutional Style Guide}\footnote{http://publications.europa.eu/code/en/en-000100.htm.}, institution-specific and language-specific style guides (e.g. \textit{Vademecum tłumacza}\footnote{http://ec.europa.eu/translation/polish/guidelines/documents/styleguide\_polish\_dgt\_pl.pdf} for Polish; see also \citealt{Svoboda2017} this volume);
\item 
CAT tools (SDL Trados Studio), translation memories\footnote{DGT publishes parts of its translation memories in 24 languages: \url{https://ec.europa.eu/jrc/en/language-technologies/dgt-translation-memory\#Statistics for the DGT Translation Memory} (Accessed 2017-07-01).} and translation memory management system (EURAMIS\footnote{Euramis, managed by the European Commission, is the system storing translation memories of most EU institutions; it searches and retrieves segments with matches; \url{http://ec.europa.eu/dpo-register/details.htm?id=41727} (Accessed 2017-07-01).});
\item 
machine translation system MT@EC
\item 
workflow and document management tools: \textit{Poetry, ManDesk, Trad}\textit{esk, DGT Vista} (European Commission 2016).
\end{itemize}

One of the most important components is the interface which integrates resources in one place to ensure a good speed of information retrieval. I will discuss selected resources below (for more detailed information see European Commission 2016).

As for the \textbf{CAT} tools, most EU institutions use SDL Trados Studio 2015, which was customized to the specific needs of EU translators \citep{Trousil2017}. A new server-based CAT environment is planned to be introduced in 2018-2019 (DGT 2016b: 7). The CAT tool is integrated with IATE through the term recognition window \citep{Trousil2017}. 

\textbf{IATE} — Inter-Active Terminology for Europe — is a major terminological achievement of EU institutions, which began to be built in 2000, used internally from 2004 and made public in 2007. It is a termbase of about 1.4 million multilingual entries, integrating the terminological resources of key EU institutions, including Eurodicautom, TIS, Euterpe, Euroterms, and CDCTERM\footnote{\url{http://iate.europa.eu/about_IATE.html} (Accessed 2017-07-01).}. It is a “one-stop consultation” resource for the institutions, with two interfaces — public and internal one \citep{Trousil2017}. One of its key functionalities is the evaluation of terminological information with reliability ratings and labels, such as ‘preferred’, ‘admitted’, ‘deprecated’, ‘obsolete’, as well as references with sources of information. The institutions are working on new improved IATE 2 to be released in 2018\footnote{\url{http://iate.europa.eu/IATE_2.html} (Accessed 2017-07-01).} but it should be stressed that the quality and functionality of IATE have improved significantly over the years. Other multilingual terminological resources include \textbf{EuroVoc}, a multilingual multi-disciplinary thesaurus on the activities of the European Union, with a first edition in 1984\footnote{\url{http://eurovoc.europa.eu/drupal/?q=abouteurovoc & cl=en} (Accessed 2017-07-01).}. It is also worth noting that some resources, e.g. electronic dictionaries and specialized databases, are developed by external contractors selected through tender procedures\footnote{Cf. \url{http://ted.europa.eu/udl?uri=TED:NOTICE:398481-2016:TEXT:EN:HTML} (Accessed 2017-07-01).}.

Another type of resource — \textbf{Tradesk} (Translator’s Desktop), which is a database with a document handling tool and a collection of translation comments entered by translators, facilitates communication between the coordinating translator and translators from the same or other institutions working on the same translation and its purpose is defined as “[i]mprove communication and exchange of best practices between translators of different institutions working on inter-institutional legislative proposals, in order to avoid double work and improve consistency and quality of EU legislation”\footnote{\url{http://ec.europa.eu/dpo-register/details.htm?id=35572} (Accessed 2017-07-01).}. The Tradesk interface provides in-house translators with access to reference documents, allows for comparisons between different versions and for the annotation of translation with information from experts and clarifications from the requester \citep{Trousil2017}. Access to documents is also available through such tools as: the document search engine DGT Vista, the text search tool DocFinder, the terminological metasearch tool Quest Metasearch \citep{Trousil2017}, as well as through more specialized publicly accessible databases, e.g. the legislation repository EUR-Lex. 

\textbf{MT@EC} is an online statistical machine translation system based on Moses and released in 2013 \citep{Mai2016}, which translates from and into EU official languages and is made available for free to public administration and universities of the European Masters’ in Translation network in EU countries\footnote{\url{https://ec.europa.eu/info/resources-partners/machine-translation-public-administrations-mtec_en} (Accessed 2017-07-01).}; however, interestingly, it is not available to external contractors. One of the DGT’s strategic objectives is to increase the use of its system by doubling direct requests for MT@EC by individual users and web services to 4 million pages in 2020 \citep[9]{DGT2016b}. The MT@EC system was trained on EU corpora (i.e. Euramis with 1 billion segments \citep{Mai2016}) and gives relatively good results on EU-related texts, except for highly inflected languages. The involvement of machine translation differs from one institution to another. For example, the German Language Department of the DGT uses it selectively in press releases, reports, and general communications but not for legislation and other legal texts \citep{Mai2016}. It is mainly applied as ‘lexical inspiration’ and a tool to speed up work by reducing typing and searches; however, its disadvantages include attention focusing on different (mainly linguistic) types of errors \citep{Mai2016}. Some translators also note that the use of machine translation output and a shift from translation to post-editing prevent the deep processing of and submergence in the source/target texts which are typical of human translation (cf. O'Brien et al. (eds) 2014). It is worth noting that MT@EC is a predecessor to eTranslation, part of the Connecting Europe Facility, which will incorporate neural machine translation solutions and will pool much larger resources.
% * <l.biel@uw.edu.pl> 2017-11-13T01:22:27.749Z:
% 
% > O'Brien et al. (eds) 2014
% Please add an item to the bibliography as:
% O'Brien, Sharon, Laura Winther Balling, Michael Carl, Michel Simard and Lucia Specia (eds). 2014. Post-editing of Machine Translation: Processes and Applications. Newcastle upon Tyne: Cambridge Scholars Publishing.
% 
% ^.

To sum up, the volume and quality of technological, linguistic and terminological resources are growing and they help translators ensure the consistency and standardization of translations and increase the efficiency of their work. Yet it should be stressed that some of the tools are not available at all or in full to external translators which may adversely affect the quality of outsourced translations.

\section{Concluding remarks: reframing of quality and threat to quality}\label{sec:biel:4}

Over the last decade EU institutions have boosted the profile of quality, which is evidenced in an increasing number of policies and guidelines addressing the quality of EU translation, as well as in attempts to quantify quality through performance indicators, such as correction rates and customer satisfaction rates. One of the promising developments is the reframing of quality discourse by the explicit linking of translation quality at the textual level to genres and genre clusters, with a resulting shift of focus from equivalence to clarity and textual fit. On the other hand, such classifications may be seen as triggered by the need to prioritize documents, as part of the fit-for-purpose approach, in order to prudently manage resources and costs in line with the required level of quality. Cost effective measures towards translation products are coupled with measures at the service provision level of translation quality, including selective translation policies and demand management, the growing burden on in-house staff, staffing reductions combined with the increasing rate of outsourcing, as well as the growing use of machine translation and its unknown impact on quality. This may pose a threat to quality in the long run.

 
\section*{Acknowledgement}
This work was supported by the National Science Centre (NCN) under Grant 2014/14/E/HS2/00782.

\sloppy
\printbibliography[heading=subbibliography,notkeyword=this] 
\end{document}