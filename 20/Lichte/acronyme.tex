\addchap{Abkürzungsverzeichnis}

\newcommand{\bflabel}[1]{\normalfont{\normalsize{#1}}\hfill}
\begin{acronym}[TT-MCTAG~~~~]
\setlength{\itemsep}{-\parsep}

\acro{ARG}{Argumenthaftigkeit}
\acro{ASSOZ}{Assoziiertheit}
\acro{BET}{Beteiligtheit}
\acro{CCG}{Combinatorial Categorial Grammar}
\acro{CETM}{Condition on Elementary Tree Minimality}
\acro{CFG}{Context-Free Grammar, kontextfreie Grammatik}
\acro{CG}{Categorial Grammar, Kategorialgrammatik}
\acro{DG}{Dependenzgrammatik}
\acro{DTG}{D-Tree Grammar}
\acro{EHC}{External Homogeneity Condition}
\acro{EXO}{Exozentrizität}
\acro{FOSP}{Formale Spezifität}
\acro{FTH}{Fundamental TAG Hypothesis}
\acro{GB}{Government-and-Binding Theory}
\acro{HPSG}{Head-Driven Phrase Structure Grammar}
\acro{ID/LP}{Immediate Dominance/Linear Precendence}
\acro{INSP}{Inhaltliche Spezifität}
\acro{LCFRS}{Linear Context-Free Rewriting Systems}
\acro{LF}{Logische Form}
\acro{LFG}{Lexical Functional Grammar}
\acro{LTAG}{Lexicalized Tree Adjoining Grammar}
\acro{MCS}{Mild Context-Sensitivity, schwache Kontextsensitivität}
\acro{MCTAG}{Multi-Component Tree Adjoining Grammar}
\acro{MG}{Minimalist Grammar}
\acro{NL-MCTAG}{Non-Local MCTAG}
\acro{NOT}{Notwendigkeit}
\acro{NPI}{Negative Polarity Item}
\acro{OFR}{Oberfeldregel}
\acro{PF}{Phonetische Form}
\acro{PPI}{Positive Polarity Item}
\acro{RNR}{Right-Node-Raising}
\acro{RRG}{Role-and-Reference Grammar}
\acro{SCC}{Strong Co-occurrence Constraint}
\acro{SCC}{String Continuation Condition}
\acro{Seg-TAG}{Segmented Tree Adjoining Grammar}
\acro{STAG}{Synchronous Tree Adjoining Grammar}
\acro{STUG}{Synchronous Tree Unification Grammar}
\acro{SUBKLASS}{Subklassenspezifik}
\acro{SL-MCTAG}{Set-Local MCTAG}
\acro{SN-MCTAG}{TL-MCTAG with Shared Nodes}
\acro{TAG}{Tree Adjoining Grammar, Baumadjunktionsgrammatik}
\acro{TL-MCTAG}{Tree-Local MCTAG}
\acro{TT-MCTAG}{MCTAG with Shared Nodes and Tree Tuples}
\acro{TUG}{Tree Unification Grammar}
\acro{UFR}{Unterfeldregel}
\acro{VPE}{VP-Ellipse}
\acro{V-TAG}{Vector-MCTAG}
\acro{WCC}{Weak Co-occurrence Constraint}

%\acro{VP}{Verbalphrase}
%\acro{NP}{Nominalphrase}
%\acro{PP}{Prepositionalphrase}

\end{acronym}
