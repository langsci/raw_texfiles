%!TEX root = main.tex
\chapter{Einführung und Überblick} \label{sec-einfuehrung}

\pagenumbering{arabic}

%\setlength{\epigraphwidth}{.37\textwidth}
\setlength{\epigraphwidth}{.5\textwidth}
\epigraph{\em Wo Sinn ist, mu\ss \ vollkommene Ordnung sein. -- Also mu\ss \ die vollkommene Ordnung auch im vagsten Satze stecken.\\[-5ex]}{\citet[\S 98]{Wittgenstein:84}}

\largerpage%
\noindent So gegensätzlich sie auch in vielerlei Hinsicht sein mögen -- die meisten, wenn nicht alle, prominenten Syntaxmodelle\footnote{Dazu zähle ich zumindest die Chomsky-Grammatiken, insbesondere die Government and Binding Theory (GB, \citealt{Chomsky:81}), au\ss erdem Dependenzgrammatik (DG, \citealt{Tesniere:59,Kunze:75,Heringer:96}), Lexical Functional Grammar (LFG, \citealt{Kaplan:Bresnan:82}), Combinatorial Categorial Grammar (CCG, \citealt{Steedman:00}), Head-Driven Phrase Structure Grammar (HPSG, \citealt{Pollard:Sag:94}), Tree Adjoining Grammar (TAG, \citealt{Joshi:Schabes:97}), Minimalist Grammar (MG, \citealt{Stabler:97}), Role and Reference Grammar (RRG, \citealt{Valin:05}). Tatsächlich ist mir kein Syntaxmodell bekannt, auf das diese Zuschreibung nicht zutrifft.} haben eines gemeinsam: Sie unterscheiden strukturell oder derivationell zwischen Valenzträgern (Köpfen, Regenten), Ergänzungen (Komplementen, Argumenten) und Angaben (Adjunkten, Modifizierern). Unabhängig davon, wie jeweils die syntaktische Struktur aussieht, die einem Satz wie \ref{ex-einf-1} zugewiesen wird, es wird darin ein Unterschied gemacht zwischen {\it repariert} als dem Valenzträger, den Nominalphrasen {\it Peter} und {\it den Kühlschrank} als dessen Ergänzungen und dem Temporaladverb {\it heute} als dessen Angabe.
  
\ex. \label{ex-einf-1} Peter repariert heute den Kühlschrank.

Dabei verläuft diese Unterscheidung nicht entlang morphologischer Klassen, was noch Tesni\`ere im Sinn hatte, denn nicht alle Nominalphrasen gelten als Ergänzungen und nicht alle Adverbien als Angaben. Vielmehr kommen lexikalische, d.\,h.\ idiosynkratische Eigenschaften der Valenzträger ins Spiel, die die Einteilung in Ergänzungen und Angaben lizenzieren, nämlich deren Valenz.   

Der Konsens in Bezug auf die Trichotomie der syntaktischen Einheiten in Abhängigkeit von Valenzeigenschaften scheint so breit und so wohlbegründet, dass die allermeisten formalgrammatischen Arbeiten ihre Modelle im Vertrauen auf deren weitgehende Unstrittigkeit und empirischen Gültigkeit errichten. Daran ist per se nichts Schlechtes, denn das wissenschaftliche Arbeiten, genauso wie das Denken im Allgemeinen, braucht unhinterfragte Axiome als Ausgangspunkt. Es stimmt aber auch, dass jedes Axiom den Blick in eine gewisse Richtung lenkt und Alternativen ausblendet. 

Das vorliegende Buch hat den Zweck, genau diesen gemeinhin ausgeblendeten Konsens aufzugreifen, nämlich das intime Verhältnis zwischen Syntaxmodell und Valenzeigenschaften hinsichtlich seiner Grundlagen und Auswirkungen zu untersuchen und schlie\ss lich auch in Frage zu stellen. Die Forschungsfragen lauten genauer: Welchen Stellenwert haben Valenzeigenschaften in solchen Syntaxmodellen, welche Ausweichstrategien provoziert dieser Stellenwert bei der Modellierung bestimmter Phänomenklassen (nämlich Kohärenz und Ellipse), wie können diese Ausweichstrategien vermieden werden und wie könnte ein Syntaxmodell ohne direkten Valenzbezug aussehen?

Bei der Behandlung dieser Forschungsfragen stehen Syntaxmodelle im Vordergrund, die sich einer Baumadjunktionsgrammatik (Tree Adjoining Grammar, TAG) oder einer ihrer Weiterentwicklungen bedienen. Sie eignen sich in zweierlei Hinsicht besonders gut: zum einen als Anschauungsobjekte für den relativ direkten Reflex von Valenzeigenschaften in den syntaktischen Elementarstrukturen; zum anderen aber auch, um zu zeigen, dass die daraus resultierenden Ausweichstrategien zumindest teilweise vermeidbar sind und von der Wahl des Formalismus abhängen. %\\  

Das Buch ist folgendermaßen aufgebaut: Den Anfang macht Kapitel~\ref{ch-mit-valenz} mit einer allgemeineren Darstellung des klassischen Valenzbegriffs. Die grundsätzlichen Schwierigkeiten hinsichtlich Definition, Ermittlung und Realisierung haben bereits andere Autoren ausgiebig diskutiert und können im Wesentlichen referiert werden. Hinzu kommt allerdings die Bestimmung von Realisierungsidealisierungen, die in den meisten Syntaxmodellen mit internalisiertem Valenzkonzept wirksam sind: die Idealisierung der Kontinuität und die Idealisierung der Vollständigkeit.

Im Anschluss werden zwei Datenklassen thematisiert, die im Widerspruch mit diesen Idealisierungen stehen. Zuerst betrachte ich in Kapitel~\ref{chap-kohaerenz} eine Klasse diskontinuierlicher Valenzrahmenrealisierungen, sogenannte kohärente Konstruktionen. Kohärente Konstruktionen zeichnen sich dadurch aus, dass Ergänzungen unterschiedlicher Verben, die untereinander durch Statusrektion verbunden sind, in einem bestimmten topologischen Bereich relativ frei angeordnet werden können. Beispiele dafür sind in \ref{ex-einf-2}:
\largerpage%     
 
  \ex. \label{ex-einf-2}
  \a. \label{ex-einf-2-a}Den Kühlschrank versucht Peter heute zu reparieren.
  \b. \label{ex-einf-2-b}dass Peter den Kühlschrank heute versucht zu reparieren
  \c. \label{ex-einf-2-c}dass den Kühlschrank Peter heute zu reparieren versucht

Vergleicht man die lineare Verteilung von Valenzträger, Ergänzung und Angabe in Satz \ref{ex-einf-2-a} und \ref{ex-einf-2-b} mit der in Satz \ref{ex-einf-1}, dann stellt man fest, dass nun ein übergeordneter Valenzträger, nämlich {\it versucht}, zwischen \textit{zu reparieren} und seiner Ergänzung \textit{den Kühlschrank} interveniert.\footnote{Das Nomen {\it Peter} wird aufgrund der Subjekt-Verb-Kongruenz oft als Ergänzung des Finitums behandelt, kann aber gleichzeitig als eine Ergänzung von {\it zu reparieren} betrachtet werden. In der vorliegenden Arbeit spielt diese Doppelfunktion eine untergeordnete Rolle. Siehe Abschnitt~\ref{sec-strukturfrage}.}  Diskontinuierlich ist die Valenzrealisierung von \textit{zu reparieren} also aufgrund einer anderen Valenzrealisierung, die \textit{zu reparieren} als Ergänzung enthält. Die Intervention der Angabe \textit{heute} ist dagegen nicht hinreichend, sofern es sich um eine Angabe von \textit{zu reparieren} handelt.
\largerpage%

Nach den kohärenten Konstruktionen stehen in Kapitel~\ref{chap-ellipse} unvollständige Valenzrahmenrealisierungen im Fokus, besser bekannt als Ellipsen. Wenn man wieder von Satz \ref{ex-einf-1}, nun wiederholt als \ref{ex-einf-3-a}, ausgeht, so öffnet sich abhängig vom Kontext (z.\,B.\ in Frage-Antwort-Folgen) ein weites Feld der Ellipsemöglichkeiten, die in \Next[b--g] angedeutet werden:       
  
  \ex. \label{ex-einf-3}
  \a. \label{ex-einf-3-a} Peter repariert heute den Kühlschrank. 
  \b. \label{ex-einf-3-b} Peter repariert heute \sout{den Kühlschrank}.
  \c. \sout{Peter} Repariert heute \sout{den Kühlschrank}.
  \d. Peter \sout{repariert} den Kühlschrank.
  \e. Peter \sout{repariert den Kühlschrank}.
  \f. \sout{Peter repariert} Den Kühlschrank.
  \f. \label{ex-einf-3-f} \sout{Peter repariert} Heute \sout{den Kühschrank}.
  %\f. \ldots

Von dieser Ellipsemöglichkeit können prinzipiell alle Valenzrahmenbestandteile betroffen sein, was die Idealisierung der Vollständigkeit konterkariert. 

\enlargethispage{1mm}
Die Wirksamkeit dieser beiden Idealisierungen erklärt wiederum die Mühe, die die Modellierung von Kohärenz und Ellipse gemeinhin bereiten. Um dies an einem konkreten Syntaxmodell zu demonstrieren, werde ich dann zum Framework der Baumadjunktionsgrammatiken greifen, das in Kapitel~\ref{sec-valenz-tag} ausführlich eingeführt wird. Die bisherigen Anstrengungen, kohärente Konstruktionen im Rahmen von TAG adäquat zu behandeln, werden anschlie\ss end in Kapitel~\ref{sec-kohaerenz-tag} dargestellt, bevor ich in Kapitel~\ref{sec-ttmctag} einen eigenen Vorschlag, der auf die TAG-Variante TT-MCTAG zurückgreift, detailliert ausarbeite. Tatsächlich hebt dieses TT"=MCTAG"=Modell die Idealisierung der Kontinuität weitgehend auf. 

Während kohärente Konstruktionen mittels TT-MCTAG adäquat modelliert werden können (trotz der im Vergleich zu anderen Frameworks geringeren Ausdrucksstärke), besteht noch immer das Desideratum der Ellipsemodellierung. Kapitel~\ref{sec-ellipsenanalyse} zeigt das Spektrum der Modellierungsstrategien, die dafür unter dem Diktat der Idealisierung der Vollständigkeit in früheren TAG-Arbeiten (aber nicht nur in TAG-Arbeiten) implementiert wurden. Dagegen fehlte bisher ein Ansatz, der sich von der Idealisierung der Vollständigkeit freimacht, vergleichbar mit der Aufgabe der Idealisierung der Kontinuität im TT-MCTAG-Modell. In Kapitel~\ref{ch-ohne-valenz} stelle ich erstmals ein solches Syntaxmodell vor, nämlich die sogenannte synchrone Baumunifikationsgrammatik (STUG). Das Wittgenstein-Zitat, das dieses Kapitel eröffnet hat, drückt dessen Credo aus: Kohärenz und Ellipse sind nicht aus vollkommenen Ordnungen abgeleitet, sie sind an sich schon vollkommen. %\\

Bevor es losgeht, möchte ich noch zwei Interessenschwerpunkte hervorheben, die die vorliegende Arbeit leiten. Zum einen besteht ein computerlinguistischer Interessenschwerpunkt: Ich interessiere mich für mathematisch-formal präzisierte Syntaxmodelle, die hinsichtlich der Analyse oder Generierung eines Satzes vollständig algorithmisierbar sind. Es muss also zumindest theoretisch möglich sein, auf Grundlage dessen eine sprachverarbeitende Software zu programmieren. Die Eigenschaften eines Syntaxmodells unter gewissen Implementierungsannahmen interessieren mich daher mindestens ebenso sehr wie die Plausibilität der Implementierungsannahmen selber.    

Zum anderen legt diese Arbeit ihren empirischen Betrachtungsschwerpunkt auf das Deutsche. Das hat zwei Gründe: Erstens hält das Deutsche ein reichhaltiges, produktives Reservoir an Kohärenzmustern bereit, die darüber hinaus in zahlreichen exzellenten Arbeiten beschrieben worden sind; zweitens handelt es sich beim Deutschen um meine Muttersprache. Letzteres kommt dann zum Tragen, wenn ich mich in Kapitel~\ref{chap-ellipse} an eine Systematisierung elliptischer Phänomene wage, für die auf keine entsprechende Vorarbeit zurückgegriffen werden kann.








