%!TEX root = main.tex
\chapter{Kontra Kontinuität: Kohärenz} \label{chap-kohaerenz}

Das vorangegangene Kapitel schloss mit der Formulierung dreier Idealisierungen\is{Idealisierung} des Verhältnisses zwischen Syntax und Valenz: (i) die Idealisierung der Vollständigkeit, (ii) die Idealisierung der Kontinuität und (iii) die Idealisierung der Funktionalität. In diesem Kapitel soll eine Form der Valenzrealisierung untersucht werden, die die Idealisierung der Kontinuität\is{Idealisierung!der Kontinuität} empirisch konterkariert: die kohärente Konstruktion.

Kohärenz\is{kohärente Konstruktion} ist ein Phänomen der Satzsyntax, bei dem die Ergänzungen und Angaben  unterschiedlicher, aber in einer Rektionsbeziehung befindlicher Verben so permutieren, dass einzelne Valenzrahmen diskontinuierlich\is{Diskontinuität} realisiert sind.\footnote{Eigentlich müsste hier von Dependenzfeldern (siehe Definition~\ref{def-dependenzfeld}, S.~\pageref{def-dependenzfeld}) statt von Valenzrahmenrealisierungen gesprochen werden, da natürlich Diskontinuität nicht allein dann vorliegen soll, wenn die Valenzrahmenbestandteile durch Angaben des Valenzträgers voneinander getrennt sind. Ich klammere im Folgenden Angaben jedoch weitgehend aus.} Identifizierbar sind kohärente Konstruktionen anhand der Wortstellung, der Valenzbeziehungen\is{Valenzbeziehung} (zu Ergänzungen) und der semantischen Dependenzbeziehungen (zu Angaben). Das ist beispielsweise in Satz \ref{ex-kohaerenz-einf-1} der Fall:

\ex. \label{ex-kohaerenz-einf-1} Das Fahrrad versuchte Peter zu reparieren.

Die Valenzrahmenrealisierung  {\it das Fahrrad zu reparieren} wird hier durch das Verb {\it versuchte} und seine Ergänzung {\it Peter} linear unterbrochen, und ist also diskontinuierlich. Wie im letzten Kapitel angekündigt, werde ich die Nominativ-Ergänzung jeweils dem maximal übergeordneten Valenzträger zuordnen, wenn ich sie nicht ganz ignoriere. Damit werde ich zwar dem oben formulierten Valenzbegriff nicht ganz gerecht (denn {\it Peter} muss wohl auch als ein Argument von {\it zu reparieren} angesehen werden und eo ipso als seine Ergänzung) und vernachlässige auch sträflich die Nicht-Funktionalität dieser Valenzrahmenrealisierungen, aber die Darstellung der \isi{Diskontinuität} von Valenzrahmenrealisierungen gewinnt dadurch erheblich an Klarheit -- insbesondere deswegen, weil ich enger an der Darstellung Gunnar Bechs bleiben kann. \cite{Bech:55} ist unbestreitbar der sichtbarste Wegbereiter einer stringenten, hinreichend expliziten Systematisierung des Kohärenzphänomens. Auch wenn nicht immer valenztheoretisch einwandfrei, lassen sich damit die unterschiedlichen Diskontinuitätstypen in einer einheitlichen, mittlerweile etablierten Schreibweise erfassen.

\section{Grundbegriffe und Grundstellung} \label{sec-kohaerenz-einf}

\subsection{Status, Kette, Rang und Verbalfeld}

Beginnen wir mit der Morphologie der nicht-finiten Verben\is{nicht-finites Verb}. Bech unterscheidet hier je drei \textsc{Status}\is{Status} der supinischen, d.\,h.\ nicht-flektierten, und der partizipialen, d.\,h.\ der adjektivischen, Verb\-for\-men. Tabelle~\ref{fig-status} fasst diese Taxonomie zusammen.   
\begin{table}[h] % mehr Platz um die Tabelle bei [h], dadurch Vermeidung von Umbruchproblem
\centering
\begin{tabular}{ccc}
\lsptoprule
& 1.Stufe	& 2.Stufe 	\\
& Supinum	& Partizipium	\\
\cmidrule{2-3}
1. Status	& \emph{lieben}	& \emph{liebend}(\emph{-er})\\
2. Status	& \emph{zu lieben} & \emph{zu liebend}(\emph{-er})\\
3. Status	& \emph{geliebt}& \emph{geliebt}(\emph{-er})\\
\lspbottomrule
\end{tabular}
\caption{Status der nicht-finiten Verben nach \cite{Bech:55} \label{fig-status}}
\end{table}

In dieser Arbeit sind auschlie\ss lich die supinischen Verb\-formen von Interesse. Deren Status entsprechen den Kasus der nominalen und adjektivischen Wortformen, indem sie den Bezugspunkt für Rektionsbeziehungen\is{Rektion} bilden. Man spricht hier, analog zur Kasusrektion, von der Statusrektion. 

In einem Satz können beliebig viele Verben durch eine Kette von Rektionsbeziehungen miteinander verknüpft sein, d.\,h.\ sie bilden eine sogenannte \textsc{subordinative} oder \textsc{hypotaktische Kette}\is{subordinative Kette}. Diese wird bei Bech durch eine Sequenz $V^{1} \ldots V^{n}$ mit $1, \ldots , n \in \mathbb{N}$ schematisiert, wobei gilt: Ein Verb $V^i$ regiert immer den Status eines Verbs $V^{i+1}$, so dass $V^i$ einen höheren \textsc{Rang}\is{Rang} einnimmt. Der Rangindex $i$ eines Verbs $V^i$ ist also in dieser Notation niedriger als der Rangindex des statusregierten Verbs $V^{i+1}$. Das ranghöchste Verb einer subordinativen Kette wird durch den Rangindex $1$ angezeigt. Wir erhalten also für einen Satz wie in \ref{ex-bech-1} das Rangschema $V^1 V^2 V^4 V^3$:

\exg. Ich habe nach drei Jahren gebeten, in der Einzelhaft bleiben {zu dürfen}.\\
{} $V^1$ {} {} {} $V^2$ {} {} {} $V^4$ $V^3$\label{ex-bech-1}\\
 
Das finite Verb {\it habe} regiert den 3.~Status von {\it gebeten}, {\it gebeten} wiederum den 2.~Status von {\it zu dürfen}, und das wiederum den 1. Status von {\it bleiben}. Jedes dieser Verben wird genau einem \textsc{Verbalfeld}\is{Verbalfeld} zugeordnet (und vice versa): "`Zum verbalfeld $F^n$ gehören au\ss er dem verbum $V^n$ alle bestandteile des satzes, die von $V^n$ abhängen, au\ss er $V^{n+1}$ und diejenigen glieder, die von $V^{n+1}$ abhängen."' \citep[\S 36]{Bech:55} Es ist für mich nicht zu erkennen, wie für Bech die Abhängigkeit zwischen Bestandteilen des Satzes genau definiert ist. Die Verbalfelder für Satz \ref{ex-bech-1} sind wohl aber $F^1 =$ ({\it ich, habe}), $F^2 =$ ({\it nach drei Jahren, gebeten}), $F^4 =$ ({\it in der Einzelhaft, bleiben}), $F^3 =$ ({\it zu dürfen}).\footnote{Siehe \citet[\S 22]{Bech:55}.} Ein \isi{Verbalfeld} ist also nicht zu verwechseln mit dem \isi{Valenzrahmen} des enthaltenen Verbs, denn (i) das statusregierte Verbalfeld ist nur implizit über die Rangindizierung der Verbalfelder angedeutet und (ii) das Verbalfeld enthält nicht nur Ergänzungen, sondern auch Angaben. Mit Blick auf das Kohärenzphänomen ist die Unterscheidung von Verbalfeldern jedoch sinnvoll, da nicht nur die Ergänzungen unterschiedlicher Verbalfelder permutieren können, sondern natürlich auch deren Angaben. Schlie\ss lich (iii) ordnet Bech die Subjekte\is{Subjekt} den Verbalfeldern der finiten Verben zu, was nicht unüblich ist, aber dem im letzten Kapitel explizierten Valenzbegriff widerspricht, da etwa das Hilfsverb\is{Hilfsverb} {\it habe} dem Subjekt {\it ich} keine \isi{semantische Rolle} zuweist. Die Existenz diskontinuierlicher Valenzrahmenrealisierungen ist von dieser Zuordnung jedoch nicht abhängig, also belasse ich es bei Bechs Darstellung.  

\subsection{Kohärenzfeld} 

Während Verbalfelder je eine dependenzielle Einheit bilden, ist das \textsc{Kohärenzfeld}\is{Kohärenzfeld} die ma\ss gebliche topologische Einheit. Jedes Kohärenzfeld wird aus einem oder mehreren per Statusrektion verbundenen Verbalfeldern\is{Verbalfeld} gebildet, so dass (im Normalfall) die Verben im \textsc{Schlussfeld}\is{Schlussfeld} angeordnet werden, dem das \textsc{Restfeld}\is{Restfeld} mit den übrigen Bestandteilen der Verbalfelder linear vorangeht. Das finite Verb kann, im Unterschied zu Infinita, auch Bestandteil des Restfelds sein. Bech notiert das Kohärenzfeldschema\is{Kohärenzfeldschema} ähnlich wie in \ref{ex-kf-schema} \citep[\S 55]{Bech:55}:\footnote{Ich greife auf diese Notation insbesondere bei der Darstellung von Abweichungen in Abschnitt~\ref{sec-permutation-kohaerenzfeld} zurück.}  

\ex. $K^i = R^i S^i$, mit $i \in \mathbb{N}^+$ \label{ex-kf-schema}

$R$ ist hier das Restfeld und $S$ das Schlussfeld. Den heutigen Gepflogenheiten entsprechend nenne ich das Schlussfeld auch \textsc{Verbalkomplex}\is{Verbalkomplex}.\footnote{Manche Autoren sagen statt Verbalkomplex auch Verbkomplex, z.\,B.\ \cite{Vogel:09}. Aus meiner Sicht ist das eine Geschmacksfrage.} Die Rangindizierung\is{Rang} der Kohärenzfelder\is{Kohärenzfeld} ergibt sich aus der Rangindizierung der enthaltenen Verbalfelder: Bei Kohärenzfeldern $K^i$, $K^{i+1}$ regiert das rangniedrigste Verbalfeld aus $K^i$ das ranghöchste Verbalfeld aus $K^{i+1}$. Ein Satz umfasst also nicht selten mehrere Kohärenzfelder, so auch Satz \ref{ex-bech-1}, der über zwei Kohärenzfelder $K^1 =$ ({\it ich, habe, nach drei Jahren})({\it gebeten}) und $K^2 =$ ({\it in der Einzelhaft})({\it bleiben zu dürfen}) verfügt. Da es sich dabei um linear zusammenhängende Einheiten handelt, kann man die Kohärenzfeldpartition auch mit Klammern visualisieren:\footnote{Die Klammerdarstellung stammt aus \citet[Kapitel~17]{Mueller:99}.}

\ex. $\overbrace{\underbrace{\text{Ich habe nach drei Jahren}}_{~R^1} \underbrace{\text{gebeten}}_{~S^1}}^{~K^1}$, $\overbrace{\underbrace{\text{\phantom{g}\!\!\!in der Einzelhaft}}_{~R^2} \underbrace{\text{\phantom{g}\!\!\!bleiben {zu dürfen}}}_{~S^2}}^{~K^2}$.

Es ist aber auch eine alternative Kohärenzfeldpartition\is{Kohärenzfeld} mit kleineren Kohärenzfeldern denkbar -- dazu mehr im nächsten Abschnitt. 

Was die Binnenstruktur der Restfelder\is{Restfeld} betrifft, macht \cite{Bech:55} keine genauen Angaben. Man kann wohl annehmen, dass er hier keine spezifischen Gesetzmä\ss igkeiten erkennen kann, die auf Kohärenzfelder mit mehreren Verbalfeldern beschränkt ist. Das Hauptaugenmerk seiner Arbeit lenkt er stattdessen auf Gesetzmä\ss igkeiten in der Abfolge der Verben im Schlussfeld\is{Schlussfeld}. Als Grundstellung kann man hier das Rangschema in \ref{ex-sf-schema1} anführen, d.\,h.\ die subordinative Verbalkette ist linksläufig, so dass ein Verb $V_{i}$ immer rechts neben einem Verb $V_{i+1}$ steht, dessen Status $V_i$ regiert:

\ex. $V_n \ldots V_1$, mit $n \geq 1$ \label{ex-sf-schema1}

Um das Rangschema eines Schlussfelds\is{Schlussfeld} unabhängig von der Einbettung des Kohärenzfelds darzustellen, wechselt hier die Notation: Als Subskript wird der \textsc{relative, schlussfeldbezogene Rangindex}\is{Rang} angegeben, der im Unterschied zum absoluten, satzbezogenen Rangindex nur die \isi{subordinative Kette} eines einzelnen Schlussfelds indiziert. Das vollständige Rangschema des Schlussfelds des Kohärenzfelds $K^2 =$  ({\it in der Einzelhaft})({\it bleiben zu dürfen}) ist also $V^4_2 V^3_1$. Befindet sich das finite Verb im Restfeld, wirkt es sich nicht auf die relative Rangindizierung aus. Das Rangschema des Schlussfelds von $K^1 =$ ({\it ich, habe, nach drei Jahren})({\it gebeten}) ist deshalb $V^2_1$.\footnote{Man beachte, dass \cite{Bech:55} als Subskripte, soweit ich sehen kann, keine relativen, schlussfeldbezogenen Rangindizes einsetzt (vgl.\ \S 17), sondern vielmehr absolute Rangindizes, wobei jedoch das finite Verb im Restfeld, d.\,h.\ bei "`hauptwortsatzstellungen"', den Rangindex 0 erhält. Das hat zur Folge, das immerhin das Schlussfeld von Kohärenzfeldern mit dem finiten Verb über so etwas wie eine relative Rangindizierung verfügt. Für tiefer eingebettete Kohärenzfelder ohne finitem Verb gilt dies jedoch nicht. Hinsichtlich der Daten, die Bech bei der Untersuchung des Schlussfelds zu Rate zieht, ergibt sich allerdings kein Unterschied zu meiner etwas generelleren Notation der relativen Rangindizes.}     




\subsection{Kohärente Konstruktion vs.\ inkohärente Konstruktion}
\is{kohärente Konstruktion|(}\is{inkohärente Konstruktion|(}

Die "`verbindungsart"' \citep[\S 71]{Bech:55} zwischen zwei Verbalfeldern\is{Verbalfeld} $F^i$ und $F^{i+1}$ ist kohärent, falls $F^i$ und $F^{i+1}$ zu einem \isi{Kohärenzfeld} $K$ gehören, so dass sich (i) $V^i$ im \isi{Restfeld} von $K$ und $V^{i+1}$ im \isi{Schlussfeld} von $K$ oder (ii) $V^i$ und $V^{i+1}$ im Schlussfeld von $K$ befinden. Man spricht in solchen Fällen von einer \textsc{kohärenten Konstruktion}\is{kohärente Konstruktion}. Andernfalls ist die "`verbindungsart"' zwischen $F^i$ und $F^{i+1}$ inkohärent und es liegt eine \textsc{inkohärente Konstruktion}\is{inkohärente Konstruktion} vor. Subordinative Ketten\is{subordinative Kette} mit mehr als einem Kohärenzfeld enthalten trivialerweise sowohl kohärente als auch inkohärente Konstruktionen, wie Satz \ref{ex-bech-1} beweist: Die Kohärenzfelder $K^1 =$ ({\it ich, habe, nach drei Jahren})({\it gebeten}) und $K^2 =$  ({\it in der Einzelhaft})({\it bleiben zu dürfen}) enthalten zwar kohärent-konstruierte Verbalfelder, aber die "`verbindungsart"' zwischen {\it gebeten} und {\it zu dürfen} ist inkohärent.

Die Kohärenzfeldpartition von \ref{ex-bech-1} durch $K^1$ und $K^2$ ist jedoch nicht die einzig mögliche. $K^1$ und $K^2$ lassen sich weiter aufteilen, so dass z.\,B.\ die Kohärenzfelder $K^{11} =$ ({\it ich, habe})(), $K^{12} =$ ()({\it gebeten}), $K^{21} =$ ()({\it zu dürfen}), $K^{22} =$ ({\it in der Einzelhaft})({\it bleiben}) entstehen. Bei dieser Kohärenzfeldpartition durch $K^{11}$, $K^{12}$, $K^{21}$, $K^{22}$ gibt es keine kohärenten Verbalfelder mehr. Dieses Beispiel zeigt also, dass das Vorliegen von Kohärenz und Inkohärenz abhängig von der Kohärenzfeldpartition ist. Das einzig eindeutige Indiz für die Zugehörigkeit zweier Verbalfelder  $F^i$ und $F^{i+1}$ zu einem Kohärenzfeld $K$ ist die Diskontinuität von  $F^{i+1}$.

Die Kohärenzfeldpartition wird also primär durch das Kohärenzfeldschema in \ref{ex-kf-schema} und das Rangschema in \ref{ex-sf-schema1}, zu dem in Abschnitt \ref{sec-verbalkomplex} weitere Rangschemata hinzukommen, bestimmt. Dies sind jedoch nicht alle Faktoren, die auf die Zulässigkeit einer Kohärenzfeldpartition einwirken. Weitere Faktoren kommen in einer Reihe von Testverfahren zum Ausdruck, die Bech zur Unterscheidung von kohärenten und inkohärenten Konstruktionen entwirft: die Rangprobe \citep[\S 71]{Bech:55}, die Statusprobe \citep[\S 73]{Bech:55} und die Skopusprobe bei Negation bzw.\ Negationskohäsion \citep[\S 80]{Bech:55}. Hier reicht es, nur auf die Rangprobe einzugehen: Die \textsc{Rangprobe}\is{Rangprobe} besteht darin, "`da\ss\ man sämtlichen verben der hypotaktischen kette [im mutma\ss lichen Schlussfeld, T.\,L.] einen niedrigeren unteren rang [d.\,h.\ einen höheren relativen Rangindex, T.\,L.] gibt, ohne da\ss\ man die reihenfolge der restfeldglieder -- oder die konstruktion in irgendwelcher anderen beziehung -- verändert"'. Kann danach noch immer eine regelkonforme Schlussfeldtopologie ermittelt werden, so liegt Kohärenz vor, andernfalls Inkohärenz. Bech veranschaulicht die Rangprobe anhand der folgenden Umstellungen, derzufolge die Verben {\it wagt} und {\it zu stören} in \ref{ex-bech-71-a} kohärent und in \ref{ex-bech-71-b} inkohärent sein sollen:

\ex. \label{ex-bech-71}
\a. \label{ex-bech-71-a}Sie wagt$_0$ ihn nicht zu stören$_1$. \\
$\leadsto$ \ldots dass sie ihn nicht (zu stören$_2$ wagt$_1$)
\b. \label{ex-bech-71-b}Sie wagt$_0$ nicht, ihn zu stören$_1$. \\
$\leadsto$ *\ldots dass sie nicht ihn (zu stören$_2$ wagt$_1$)
\z. \citep[\S 71]{Bech:55}

Von den zwei möglichen Kohärenzfeldpartitionen, die \ref{ex-bech-71-b} prima facie enthält, bleibt also nur die inkohärente Konstruktion, wenn man die Rangprobe anwendet. Der Faktor, der hier meiner Meinung nach zum Ausdruck kommt, ist die Markierung des rechten Rands des Restfelds durch den Negationsmarker {\it nicht}. Eine solche Markierung ergibt sich auch durch abgetrennte Verbzusätze. Bech geht an dieser Stelle nicht auf die Ursache für die Ungrammatikalität der Umstellung von \ref{ex-bech-71-b} ein.

In \ref{ex-bech-71} ist es die Veränderung des Satztyps, die einen höheren relativen Rangindex der Verben des ursprünglichen Schlussfelds ({\it zu stören}) bewirkt. Bei Verbzweit"=Sätzen ist das ein probates Mittel, um das finite Verb aus dem Restfeld in das potentielle Schlussfeld eines Kohärenzfelds zu stellen. Dagegen muss man bei Verbletzt-Sätzen anders vorgehen: Ein Hilfsverbs wird hier als ranghöchstes Verb im Schlussfeld eingesetzt, welches den 1. oder 3. Status regiert (und deshalb mit dem regierten Verb ein Kohärenzfeld bilden muss). In Satz \ref{ex-rangprobe-1} ist es das Hilfsverb {\it hat}:

\ex. \label{ex-rangprobe-1}\ldots dass sie nicht wagt$_1$ zu schlafen$_2$ \\
$\leadsto$ \ldots dass sie nicht gewagt$_2$ hat$_1$ zu schlafen$_3$

Dieses Beispiel ist in gewisser Weise schlecht, da im Ausgangssatz nur eine Kohärenzfeldpartition mit einer inkohärenten Konstruktion möglich ist, denn das Rangschema $V_1 V_2$ ist im Schlussfeld nicht zulässig. Ein in dieser Hinsicht besseres Beispiel werden wir in \ref{ex-bech-4} in Abschnitt \ref{sec-verbalkomplex} sehen, wenn weitere zulässige Rangschemata im \isi{Schlussfeld} thematisiert werden.

\is{kohärente Konstruktion|)}\is{inkohärente Konstruktion|)}


\subsection{Lexikalische Klassifizierung der Verben}

Eine wichtige lexikalische Klassifizierung der statusregierenden Verben gründet auf ihrer \textsc{Kohärenztauglichkeit}\is{Kohärenztauglichkeit} bzw.\ \textsc{Kohärenzuntauglichkeit}. Anhand der Kohärenzfeldpartitionierung ist der Unterschied so zu erklären, dass kohärenzuntaugliche Verben und die Verben, deren Status sie regieren, immer in getrennten Kohärenzfeldern platziert sind, wohingegen kohärenztaugliche Verben dieser Regel nicht unterworfen sind. Der etabliertere Terminus für kohärenzuntaugliche Verben ist der der \textsc{obligatorisch inkohärent konstruierenden Verben}\is{Kohärenz!obligatorische/fakultative}. Die kohärenztauglichen Verben werden dagegen gemeinhin in \textsc{fakultativ kohärent konstruierende Verben} und \textsc{obligatorisch kohärent konstruierende Verben} unterschieden. Die obligatorisch kohärent konstruierenden Verben sind nur dort zulässig, wo sie mit dem statusregierten Verb ein Kohärenzfeld teilen. Die fakultativ kohärent konstruierenden Verben harmonieren dagegen prinzipiell (d.\,h.\ unter bestimmten Umständen, siehe unten) mit beiden, kohärenten und inkohärenten Konstruktionen. Man kann hier also auch so wie z.\,B.\ \citet[38]{Meurers:99} von \textsc{optional inkohärent konstruierenden Verben} sprechen. In Satz \ref{ex-bech-1} lassen sich Instanzen beider Globalklassen beobachten: die Verben {\it haben} und {\it dürfen} konstruieren obligatorisch kohärent, das Verb {\it bitten} dagegen obligatorisch inkohärent. 

Bech hat in seiner "`Kohärenzregel"' bereits ein recht zuverlässiges Indiz für Kohärenz\-(un)\-taug\-lich\-keit festgehalten:\footnote{Kohärenzregel nach \cite[\S 65]{Bech:55}: "`Zwei verbalfelder, $F'$ und $F''$, die -- per definitionem -- dadurch verbunden sind, da\ss\ $V'$ den status von $V''$ regiert, sind kohärent, wenn $V''$ im 1. oder 3. status steht, können aber je nach den umständen kohärent oder inkohärent sein, wenn $V''$ im 2. status steht."'} Alle Verben, die den 1. oder 3. Status regieren, sind kohärenztauglich, ja konstruieren (meist) sogar obligatorisch kohärent. Weniger eindeutig verhält es sich bei Verben, die den 2. Status regieren. Sie können in allen drei Subklassen vorkommen. Auch eine Querklassifizierung dieser Verben anhand von Kontroll- und Anhebungseigenschaften\is{Kontrolle}\is{Anhebung} erbringt kein klares Verteilungsmuster, denn immer lassen sich Ausnahmefälle finden (siehe dazu \citealt[Abschnitt~2.3]{Meurers:99}).\footnote{Ich setze die Begriffe der Anhebung\is{Anhebung} und der Kontrolle\is{Kontrolle} hier als bekannt voraus. Ihr konzeptuelles Korrelat verbirgt sich bei Bech hinter den Begriffen "`Orientierung"' der statusregierten Verben bzw.\ "`Koeffizient"' der statusregierenden Verben \citep[Kapitel~3]{Bech:55}. Eine detaillierte, korpusgeleitete Untersuchung des Zusammenhangs zwischen Kontrolle und Kohärenz leistet \cite{Grosse:05}. }\footnote{\cite{Cook:01} schlägt daher eine informationsstrukturelle Fundierung der Kohärenzklassen vor, die \citet[Abschnitt~4.4]{Grosse:05} kritisch sieht.} Dies betrifft insbesondere die Klasse der obligatorisch inkohärent konstruierenden Verben\is{Kohärenz!obligatorische/fakultative}, die in Tabelle~\ref{fig-kohaerenzklassen} von der Klasse der Akkusativkontrollverben\is{Kontrolle} besetzt ist. \cite{Grosse:05} diskutiert in diesem Zusammenhang das Akkusativkontrollverb {\it bitten}, dem ein Potential zu kohärenter Konstruktionsweise zugesagt wird. Grundsätzlich scheint es so zu sein, dass obligatorisch inkohärent konstruierende Verben empfänglich für einen gewissen "`Trainingseffekt"' sind, unter dem sie in eine kohärente Konstruktion "`gezwungen"' werden können.\footnote{Diese Einschätzung teilt \cite{Grosse:05} mit \citet[19, Fußnote 12]{Meurers:99}, wo es hei\ss t: "`The class of obligatorily incoherent verbs appears to be rather fragile in that one often manages to force such verbs into a coherent pattern if one tries long enough. This `training effect' does not surface with obligatorily coherent verbs, which when forced into an incoherent pattern always cause ungrammaticality."'} Tabelle~\ref{fig-kohaerenzklassen} stellt daher nur eine grobe Verallgemeinerung der in der Literatur vorgefundenen Klassifizierung dar.\footnote{Vgl.\ z.\,B.\ \citet[Kapitel~1]{Grosse:05} und \citet[Abschnitt~3.5.2, Tabelle~5, 207]{Colomo:11}, wobei sich Colomo letztlich dagegen ausspricht, obligatorische Inkohärenz als lexikalische Eigenschaft zu behandeln.}

{\setfootnoterule{0pt}
\begin{table}[ht]
\centering
\begin{tabular}{p{2.3cm}p{9cm}}
\lsptoprule
obligatorisch koh\"arent & Verben, die den 1. oder 3. Status regieren (Hilfs- und Modalverben), bestimmte Modalverben, die den 2. Status regieren ({\it brauchen}), Anhebungsverben ({\it scheinen, \ldots}), bestimmte Kontrollverben ({\it wissen, suchen})\footnote{Siehe \cite{Reis:01}.}\\
\midrule
fakultativ \mbox{kohärent} & Subjektkontrollverben ({\it versuchen, wagen, glauben, lernen, \ldots}), Dativkontrollverben ({\it ermöglichen,} \ldots), bestimmte Anhebungsverben ({\it drohen, versprechen})\footnote{Siehe \cite{Reis:05}.} \\
\midrule
obligatorisch inkohärent & Akkusativkontrollverben ({\it auf"|fordern, bitten, anflehen, unterlassen, ersuchen,} \ldots)\footnote{Siehe \citet[55]{Grosse:05}. \citet[Abschnitt~2.1.7]{Mueller:02} zeigt dagegen, dass Akkusativkontrollverben kohärent konstruieren können.} \\
\lspbottomrule
\end{tabular}
\caption{Kohärenzklassen der Verben\label{fig-kohaerenzklassen}}\is{Kohärenz!obligatorische/fakultative}
\end{table}
}

\largerpage%
Aufgrund dieser klassifikatorischen Schwammigkeit und der Ergebnisse ihrer Korpusstudie plädiert Grosse für ein graduelles Verständnis der Kohärenzeigenschaften von Kontrollverben, die von "`kohärenzfordernden und kohärenzstörenden Faktoren"' semantischer, pragmatischer und struktureller Art beeinflusst werden können. Auch sieht sie hinsichtlich der fakultativ kohärent konstruierenden Verben Evidenz dafür gegeben, "`dass es sich bei diesen Verben um keine homogene Gruppe handelt, sondern dass es innerhalb dieser prinzipiell kohärenztauglichen Kontrollverbgruppe Abstufungen bezüglich der gezeigten Kohärenzstärke gibt"' \citep[11]{Grosse:05}. Dabei zeigt sich in ihren Korpusdaten, dass selbst bei "`prinzipieller Kohärenztauglichkeit"' eine Bevorzugung der  inkohärenten Konstruktionsweise vorliegt.\footnote{Grosse plädiert deswegen dafür, anstelle von "`fakultativ kohärent"' den Terminus "`schwach kohärent"' zu benutzen.} 

Neben der empirischen Unschärfe der Kohärenzklassen wurden auch Subklassifizierungen der fakultativ kohärenten Verben und der obligatorisch inkohärenten Verben diskutiert. Bei \cite{Reis:Sternefeld:04} etwa erhält die Klasse der fakultativ kohärenten Verben ("`K2"') zwei Subklassen: (i) Verben aus K2, die die Bildung der sogenannten dritten Konstruktion erlauben ("`K3"'), und (ii) Verben aus K2, die die Bildung des sogenannten Fernpassivs erlauben ("`K4"'). Auf beide Konstruktionstypen werden wir detaillierter in Abschnitt~\ref{sec-permutation-kohaerenzfeld} bzw.\ \ref{sec-abweichung-rektion} eingehen. Auslöser dieser Subklassifizierung ist eine Beobachtung in \citet[318]{Woellstein:01}, derzufolge {\it ablehnen} und {\it aufgeben} zwar gut Fernpassiv\is{Passiv!Fern-} bilden können, aber schlecht die 3.~Konstruktion\is{kohärente Konstruktion!3.~Konstruktion}. \cite{Wurmbrand:01} teilt dagegen die fakultativ kohärenten Verben in zwei komplementäre Subklassen auf: solche, die Fernpassiv erlauben ("`lexical restructuring"') und solche, die Fernpassiv nicht erlauben ("`lexical non restructuring"'). Schlie\ss lich ist auch vorgeschlagen worden, die Klasse der obligatorisch inkohärenten Verben dahingehend zu differenzieren, ob Intraposition\is{kohärente Konstruktion!Intraposition} möglich ist oder nicht, d.\,h.\ ob die Verben "`schwach inkohärent"' oder "`stark inkohärent"' konstruieren (siehe \citealt{Sternefeld:08}).\largerpage% 

In dieser Arbeit werde ich Grosses Schlussfolgerungen und die Subklassifizierungsvorschläge anderer Autoren  weitgehend ignorieren und die Klassifizierung\is{Kohärenz!obligatorische/fakultative} in Tabelle~\ref{fig-kohaerenzklassen} als empirischen Fixpunkt heranziehen. Der Grund dafür ist, dass in dieser Arbeit allein die syntaktische Modellierung der Konstruktionstypen im Vordergrund steht, und nicht die Erfassung lexikalischer Generalisierungen. 




\section{Wortstellung im Verbalkomplex} \label{sec-verbalkomplex}

Das Rangschema in \ref{ex-sf-schema1} wurde als Grundstellung\is{Verbalkomplex!Grundstellung} des Schlussfelds oder Verbalkomplexes, wie es von nun an hei\ss en soll, eingeführt. Tatsächlich existieren noch weitere Stellungmöglichkeiten, die als Permutierungen dieser Grundstellung betrachtet werden können. 

Die geläufigste Permutierung ist die \textsc{Oberfeldumstellung}\is{Verbalkomplex!Oberfeldumstellung}, bei der eine Gruppe von Verben mit höheren Rängen den restlichen, niederrangigen Verben vorangestellt wird. Der Verbalkomplex zerfällt also in zwei Teilfelder, das \textsc{Oberfeld}\is{Verbalkomplex!Oberfeld} und das \textsc{Unterfeld}\is{Verbalkomplex!Unterfeld}, und erhält das generalisierte Rangschema in \ref{ex-sf-schema2}: 

\ex.   $\overbrace{\underbrace{V_1 ... V_{n}}_{\text{Oberfeld}} \ \underbrace{V_{m} ... V_{n+1}}_{\text{Unterfeld}}}^{\text{Verbalkomplex}}$, mit $1 \leq n < n+1 \leq m$ \label{ex-sf-schema2}

Das Rangschema der Teilfelder ist gegenläufig, d.\,h.\ das Unterfeld ist wie in der Grundstellung linksläufig, während das Oberfeld rechtsläufig ist, also ein $V_i$ vor $V_{i+1}$ steht. Au\ss erdem regiert das maximal untergeordnete Verb des Oberfelds $V_n$ den Status des maximal übergeordneten Verbs des Unterfelds $V_{n+1}$. Neben einem solchen Rangschema muss der Verbalkomplex auch eine gewisse Mindestgrö\ss e aufweisen, d.\,h.\ das Unterfeld muss mindestens aus zwei Verben bestehen.\footnote{Es ist aber nicht vollkommen ausgeschlossen, dass das Unterfeld weniger als zwei Verben enthält, siehe \citet[83]{Meurers:99}.} Die damit lizenzierten Rangschemata des Verbalkomplexes sind in Tabelle~\ref{fig-oberfeld} zusammengefasst.
\begin{table}[t]
\centering
\begin{tabular}{cccccc}
\lsptoprule
& 1	& 2	& 3	& 4	& 5	\\
\cmidrule{2-6}
0	& V$_1$	& V$_2$V$_1$ & V$_3$V$_2$V$_1$ & V$_4$V$_3$V$_2$V$_1$ & V$_5$V$_4$V$_3$V$_2$V$_1$ \\
1	& 	& 	     & V$_1$V$_3$V$_2$ & V$_1$V$_4$V$_3$V$_2$ & V$_1$V$_5$V$_4$V$_3$V$_2$ \\
2	&	&	     &		       & V$_1$V$_2$V$_4$V$_3$ & V$_1$V$_2$V$_5$V$_4$V$_3$ \\
3	& 	& 	     & 		       & 		      & V$_1$V$_2$V$_3$V$_5$V$_4$ \\
\lspbottomrule
\end{tabular} 
\caption{Rangschemata inklusive Oberfeldumstellung nach \citet[\S 61]{Bech:55} für Verbalkomplexe mit bis zu fünf Verben. Die Spalten enthalten Rangschemata gleicher Gesamtgrö\ss e, und die Zeilen Rangschemata mit gleicher Oberfeldgrö\ss e.\label{fig-oberfeld}}\is{Verbalkomplex!Oberfeldumstellung}
\end{table}
\citet[\S 61]{Bech:55} beschränkt sich bei dieser Aufstellung auf Verbalkomplexe bzw.\ Schlussfelder bis zu einer Grö\ss e von fünf Verben, denn "`Schlu\ss felder mit mehr als vier verben sind [\ldots] äu\ss erst selten"'. Tatsächlich kann ich in \cite{Bech:55} keinen einzigen Korpus-Beleg für einen fünfgliedrigen Verbalkomplex finden. Bech konstruiert jedoch die Verbalkomplexpermutationen in \ref{ex-bech-2}, um das theoretisch vorhandene Potential vorzuführen:

\ex. dass man ihn hier \label{ex-bech-2}
\a. liegen$_5$ bleiben$_4$ lassen$_3$ können$_2$ wird$_1$
\b. wird$_1$ liegen$_5$ bleiben$_4$ lassen$_3$ können$_2$ 
\c. wird$_1$ können$_2$ liegen$_5$ bleiben$_4$ lassen$_3$  \label{ex-bech-2-c}
\d. wird$_1$ können$_2$ lassen$_3$ liegen$_5$ bleiben$_4$  \label{ex-bech-2-d}
\z. (vgl.\ \citealt[\S 61]{Bech:55})  

\largerpage% 
Abgesehen von der Instanziierung eines der lizenzierten Rangschemata aus Tabelle~\ref{fig-oberfeld} muss ein gegebener \isi{Verbalkomplex} noch weitere Bedingungen erfüllen. Zum einen besteht das Oberfeld nur aus Verben mit bestimmten Status: 

\ex. \label{ex-ofr} \textbf{Oberfeldregel (OFR)}\footnote{Dies ist eine Generalisierung der 2. Wortstellungsregel aus \citet[\S 11]{Bech:63}, die nur für eingliedrige Oberfelder gilt.} \\
Das Oberfeld enthält kein Verb im 2. Status.\is{Verbalkomplex!Oberfeldregel}

Sätze wie \ref{ex-bech-3} sind also ungrammatisch:\largerpage% 

\ex. *ich freue mich sie zu haben kommen gesehen \hfill \citep[\S 11]{Bech:63}\label{ex-bech-3}

Zum anderen nimmt u.\,a.\ \citet[75ff]{Meurers:99} die Einschränkung an, dass nur bestimmte Verben, die Hilfsverben\is{Hilfsverb} {\it werden} und {\it haben}, im Oberfeld\is{Verbalkomplex!Oberfeld} auftreten können. Die Oberfeldumstellungen in \ref{ex-bech-2-c} und \ref{ex-bech-2-d} seien demgegenüber veraltet und nicht mehr im selben Ma\ss \ gebräuchlich. Insbesondere das Auftreten der Verben {\it lassen} und {\it sein} im Oberfeld führt zu einer merklichen Verminderung der Akzeptanz:

\ex. 
\a. da\ss \ man ihn hier lässt liegen bleiben \hfill \citep[\S 61]{Bech:55} 
\b. *da\ss \ der Brief ist abgeschickt worden \hfill \citep[(125a)]{Meurers:99}

Was Modalverben\is{Modalverb} wie {\it können}, {\it müssen} und {\it sollen} betrifft, ist die Akzeptanzminderung wohl nicht so ausgeprägt. 

Für das Unterfeld\is{Verbalkomplex!Unterfeld} existieren anscheinend keine morphologischen Einschränkungen ähnlich der Oberfeldregel: Verben mit 2. oder 3. Status finden dort ohne Schwierigkeiten einen Platz (siehe \citealt[80ff]{Meurers:99}). Man muss bei der Betrachtung von Verben des 2. Status allerdings Vorsicht walten lassen, wenn sie das maximal übergeordnete Verb des Unterfelds sein sollen. Bech nimmt beispielsweise für den Verbalkomplex  in \ref{ex-bech-4} das Rangschema $V_1 V_3 V_2$ an, wobei $V_2$ durch das Status-2-Verb {\it zu können} instantiiert wird:

\ex. \label{ex-bech-4} da\ss\ sie eine Absicht glaubten$_1$ verbergen$_3$ zu können$_2$ \hfill \citep[\S 62]{Bech:55} 

Wendet man auf diesen Verbalkomplex die \isi{Rangprobe} an, dann kann er zu dem in \ref{ex-meurers-127} umgebildet werden:

\ex. da\ss\ sie eine Absicht geglaubt$_2$ hatten$_1$ verbergen$_4$ zu können$_3$ \\ \citep[(127)]{Meurers:99}\label{ex-meurers-127}

Die dadurch entstandene subordinative Kette entspricht jedoch nicht den Stellungsmöglichkeiten eines einzelnen Schlussfelds: Die Schlussfeldstellung $V_2 V_1$ $V_4 V_3$ gibt es nicht, denn die Abfolge $V_2 V_1$ ist nicht die eines regelkonformen Oberfeldes. Statt der Rangindizierung in \ref{ex-bech-4} muss man also vielmehr die folgende Rangindizierung einer inkohärenten Konstruktion\is{inkohärente Konstruktion} ansetzten:

\ex. \label{ex-bech-4-b}da\ss\ sie eine Absicht glaubten$_1$ | verbergen$_2$ zu können$_1$
 
Dem Ergebnis der Rangprobe zufolge enthalten diese Sätze also nicht einen Verbalkomplex, sondern zwei. Das Verb {\it glaubten} ist nicht im Oberfeld und die Kette {\it verbergen zu können} im dazugehörigen Unterfeld wie in \ref{ex-bech-4}, vielmehr sind sie Bestandteile getrennter Unterfelder wie in \ref{ex-bech-4-b}.

Doch damit nicht genug. Betrachtet man diese Konstruktionen unter dem Eindruck der Bech'schen Systematik genauer, so stellt man Erstaunliches fest: Zwar können zwei getrennte Schlussfelder identifiziert werden, was auf Inkohärenz schlie\ss en lässt, aber die dazugehörigen vermeintlichen Restfelder stehen nicht adjazent zum jeweiligen Schlussfeld, was auf Kohärenz hindeutet. Es entsteht also ein Widerspruch, d.\,h.\ solche Konstruktionen, die als \textsc{3.~ Konstruktion}\is{kohärente Konstruktion!3.~Konstruktion} \citep{denBesten:Rutten:89} bezeichnet werden, lassen sich in der Bech'schen Kohärenzfeldtypisierung nicht sauber einordnen. Darauf werde ich im nächsten Abschnitt nochmals ausführlicher eingehen. Als R\'esum\'e dieser Gedankengänge kann die Hypothese festgehalten werden, dass das maximal übergeordnete Verb des Unterfelds aus analytischen Gründen nicht im 2. Status stehen kann. %\\ 
 
Eine weitere, seltenere Permutationsmöglichkeit im Verbalkomplex ist die \linebreak \textsc{Zwischenstellung}\is{Verbalkomplex!Zwischenstellung}  (\citealt{Meurers:94}; \citealt[84ff]{Meurers:99}). Dabei interveniert das Oberfeld gleichsam im Unterfeld, so dass ein Schlussfeld wie in \ref{ex-zwischenstellung} entstehen kann: 

\ex. da\ss\ er das Examen bestehen$_3$ wird$_1$ können$_2$ \hfill \citep[(138)]{Meurers:99}\label{ex-zwischenstellung}

Die Meinungen darüber, ob dies zum Standarddeutschen gerechnet werden soll oder ob es sich dabei um ein auf süddeutsche Dialekte begrenztes Stellungsphänomen handelt, gehen auseinander (siehe \citealt[84ff]{Meurers:99}). \cite{Bech:55} unterlässt es jedenfalls, sie zu erwähnen -- er schlie\ss t sie allerdings auch nicht aus.

\largerpage% 
\largerpage[-1]%
Eine Intervention ganz anderer Art ist die sogenannte \textsc{Linksstellung}\is{Linksstellung}, bei der der Verbalkomplex zwischen Ober- und Unterfeld durch nicht-verbales Material aus dem Restfeld unterbrochen ist, oder mit anderen Worten, bei der das Oberfeld vom Unterfeld getrennt und nach links versetzt ist.\footnote{Der Terminus geht auf \citet[218ff]{Meurers:97} zurück. \cite{Kefer:Lejeune:74}, die die erste umfassendere Untersuchung dazu durchführen, sprechen von "`Einklammerung"', \citet{Haegeman:Riemsdijk:86} von "`verb projection raising"' und \citet[Abschnitt~14.3]{Mueller:99} von "`Unterbrechung des Verbalkomplexes"'. Übrigens streift auch \citet[\S63]{Bech:55} dieses Phänomen kurz.} Ein Beispiel dafür liefert Satz \ref{ex-linksstellung} mit dem linksversetzten Verb {\it hätte}:

\ex. \label{ex-linksstellung} ohne da\ss\ der Staatsanwalt hätte darum bitten müssen.\\
\citep[(145a)]{Meurers:99}  
%??Den alten Wagen$^3$ hat er zu fahren$^3$ noch nicht gelernt. \hfill \citep[(9a)]{Askedal:83}

Da hier strenggenommen keine Permutation innerhalb des Verbalkomplexes \linebreak stattfindet, werden wir uns diesem Phänomen (und dem verwandten Phänomen der Linksstellung maximal untergeordneter Verben) etwas ausführlicher in Ab\-schnitt~\ref{sec-linksstellung} zuwenden.   
\largerpage%

\section{Abweichungen bei der Rektion} \label{sec-abweichung-rektion}

Die Bestandteile eines Verbalkomplex können neben der Linearisierungsposition auch hinsichtlich der Flexionsform von einer bestimmten Normvorstellung abweichen. Davon sind nicht nur Verben, sondern im Fall der Fernpassiv-Konstruk\-tion auch deren nominalen Ergänzungen betroffen. Doch beginnen wir mit einem noch relativ regelhaften und häufigen Phänomen aus dem Bereich der Verbalmorphologie.  

\subsection{Ersatzinfinitiv}
\is{Ersatzinfinitiv|(}

Bestimmte Verben können im 1. Status realisiert werden, obwohl das sie regierende Verb den 3. Status fordert. Solch ein Infinitiv, der quasi den 3. Status ersetzt, wird üblicherweise als \textsc{Ersatzinfinitiv} oder als \textsc{infinitivus pro participio (IPP)} bezeichnet. Während die Generalisierung belastbar zu sein scheint, dass Ersatzinfinitive ausschlie\ss lich vom Perfekt-Hilfsverb\is{Hilfsverb} {\it haben} regiert werden können (siehe \citealt[62ff]{Meurers:99}), ist die Frage, welche Verben in welchen Konfigurationen im Ersatzinfinitiv realisiert werden können oder gar müssen, nicht ganz so einfach zu beantworten.\footnote{\citet[53]{Meurers:99} drückt es so aus: "`there is a fair amount of dialectal and inter-speaker variaton concerning the classification of verbs which can or have to occur as substitute infinitives [Ersatzinfinitiv, T.\,L.]."' }

Recht eindeutig ist die Situation bei Modalverben\is{Modalverb}, die den 1. Status regieren und dabei keine Partizip-II-Form haben, also \textit{dürfen}, \textit{können}, \textit{mögen}, \textit{müssen}, \textit{sollen}, \textit{wollen}. Falls sie vom perfektiven {\it haben} regiert werden, kann nur der Ersatzinfinitiv stehen:\footnote{Aber auch hier gibt es schon erste Irritationen. \cite{Bech:55} etwa erwähnt den Fund in \ref{ex-bech-gewollt} mit einem Modalverb in Partizip-II-Form:\\
\fnex{
\ex. \label{ex-bech-gewollt} Christian machte eine heftige Bewegung danach, obgleich sie es ihm ohnedies hatte rei-\linebreak chen gewollt.\hfill \citep[\S 62]{Bech:55} 

}
\citet[53]{Meurers:99} wertet solche Verwendungen jedoch als veraltet, weshalb er sie mit einem Stern versieht. Es ist damit aber nicht gesagt, dass {\it gewollt} nicht gebildet werden kann. Tatsächlich erlauben viele der Modalverben bei einer Verwendung als transitives Verb das Partizip-II:\\
\fnex{
\ex. Ich habe das nicht gewollt.

}
Diese Verwendungsmöglichkeit als Partizip-II-Form, wenn auch in einem anderen syntaktischen Kontext, wirkt wahrscheinlich einem klaren Urteil über die Verwendung als Modalverb entgegen.}
  
\ex. Er hat heute Schokolade essen dürfen / *gedurft. \hfill \citep[(82)]{Meurers:99} 

Regiert das perfektive {\it haben} einen Ersatzinfinitiv, gilt für seine Position eine spezielle Regel, die \cite{Bech:63}  \textsc{1. Wortstellungsregel (1.WR)} nennt, und die ich hier als \textsc{Unterfeldregel}\is{Ersatzinfinitiv!Unterfeldregel} paraphrasiere:\footnote{Im Original: "`Falls ein Satz drei Verben in finaler Position: $V_1$, $V_2$ und $V_3$ enthält, von denen $V_1$ das Verbum {\it haben} ist, während $V_2$ und $V_3$ im Inf. stehen, indem der Inf. bei $V_2$ das Part. Prät. ersetzt, so ist die Reihenfolge $V_1 V_3 V_2$ die einzig mögliche [\ldots]."' \citep[\S 10]{Bech:63}} 

\ex. {\bf Unterfeldregel (UFR)} \label{ex-ufr}\\
Das regierende Verb des Ersatzinfinitivs darf nicht im Unterfeld stehen.

Die UFR erfasst korrekterweise die relativ eindeutige Unverfügbarkeit der Verbalkomplexgrundstellung\is{Verbalkomplex!Grundstellung} in \ref{ex-ufr-modal-a}, erlaubt jedoch die Zwischenstellung\is{Verbalkomplex!Zwischenstellung} in \ref{ex-ufr-modal-b}, die Bechs 1.~WR aussortiert, und die Oberfeldumstellung\is{Verbalkomplex!Oberfeldumstellung} in \ref{ex-ufr-modal-c}:

\ex. 
\a. *dass er heute Schokolade essen dürfen hat \label{ex-ufr-modal-a}
\b. ?dass er heute Schokolade essen hat dürfen \label{ex-ufr-modal-b}
\c. dass er heute Schokolade hat essen dürfen \label{ex-ufr-modal-c}

Das Fragezeichen deutet es an: Was die Grammatikalität der Zwischenstellung bei Ersatzinfinitiven betrifft, möchte ich damit aber nicht das letzte Wort gesprochen haben.

Im Unterschied dazu erlauben AcI-Verben\is{AcI-Verb} wie {\it sehen, hören, lassen, fühlen}, das \isi{Modalverb} {\it brauchen} oder sonstige Verben wie {\it helfen} oft eine Alternation zwischen Partizip-II-Form und Ersatzinfinitiv, auch wenn in der Literatur bisweilen das Gegenteil behauptet wird (siehe dazu \citealt[55ff]{Meurers:99}). Das Beispiel in \ref{ex-meurers-89} zeigt dieses Potential anhand des AcI-Verbs {\it hören}: 

\ex. Ich hab's in meiner Schulter krachen hören / gehört.\\ (vgl.\ \citealt[(89a)]{Meurers:99})\label{ex-meurers-89}

%\largerpage%
Diese Alternation ist jedoch nicht immer ohne Weiteres verfügbar, wobei vielfältige Faktoren Einfluss nehmen können. Leider vermag ich hier nicht genügend Platz einzuräumen, um der teils diffusen Datenlagen gerecht zu werden. Nur einen Hinweis möchte ich hier noch abschlie\ss end anbringen. Die Vorhersagen der UFR\is{Ersatzinfinitiv!Unterfeldregel} bezüglich der fakultativen Ersatzinfinitive beim AcI-Verb {\it sehen} in \ref{ex-bech-5}  decken sich zwar mit den Grammatikalitätsurteilen von \cite{Bech:63}:%\largerpage%

\ex. \label{ex-bech-5}
\a. da\ss\ ich sie kommen *sehen / gesehen habe
\b. da\ss\ ich sie habe kommen sehen / gesehen
\z. (vgl.\ \citealt[\S 20]{Bech:63}) 

Es finden sich aber auch grammatische Sätze wie \ref{ex-zifonun}, in deren Verbalkomplex der Regent des Ersatzinfinitvs im Unterfeld\is{Verbalkomplex!Unterfeld} steht:

\ex. \label{ex-zifonun}weil Hans ihn will kommen sehen haben \hfill 
\citep[1286]{Zifonun:etal:97}

Während hier das den Ersatzfinitiv regierende Verb {\it haben} den 1. Status trägt, gibt es Indizien dafür, dass an dieser Stelle auch der 2. Status verwendet werden kann, dass also Sätze wie \ref{ex-2status-uf} grammatisch sind:

\ex. ?Ich freue mich sie kommen sehen zu haben. \label{ex-2status-uf}

Eine kurze Recherche meinerseits förderte folgenden Beleg zutage:

\ex. Sie bekundete, merkwürdige Geräusche aus der Wohnung der E bemerkt und einen Mann, der in Grö\ss e und Statur dem H ähnelte, aus der Wohnung kommen sehen zu haben.\footnote{Entnommen aus: Uwe Hellman (ed.). 2007. \textit{Fallsammlung zum Strafprozessrecht}. Springer. S.\,151.}\label{ex-2status-uf-corpus}

Die UFR\is{Ersatzinfinitiv!Unterfeldregel} (ebenso wie Bechs 1. WR) ist also zu strikt, um für das ganze Spektrum der Ersatzinfinitive gelten zu können. Immerhin darf weiterhin vermutet werden, dass die UFR die Stellungsmöglichkeiten der Regenten von obligatorischen Ersatzinfinitiven korrekt generalisiert.
\is{Ersatzinfinitiv|)}


\subsection{\emph{Zu}-Verschiebung oder Ersatz-\emph{zu}-Infinitiv}

\largerpage% 
\largerpage[-0.3]%  
\is{Ersatz-\emph{zu}-Infinitiv|(}%
Die Statusrektion innerhalb des Verbalkomplexes\is{Verbalkomplex} mit 132-Schema in Satz \ref{ex-kohaerenz-zuersatz-1} ist in zweierlei Hinsicht irregulär: Erstens ist das höchste Verb {\it haben} nicht im 2. Status, obwohl dieser vom regierenden Verb {\it glaube} zugewiesen ist; zweitens ist das von {\it haben} abhängende Verb {\it zu können} nicht im 3. Status, den {\it haben} \nopagebreak üblicherweise zuweist:\footnote{Anders als bei den Fällen mit \isi{Abkopplungseffekt} bzw.\ mit Skandalkonstruktion (siehe unten) ist hier kein Rangschema denkbar, unter dem die angesprochenen Irregularitäten in der Statusrektion verschwinden.}
\ex. 
\a. Ich glaube es haben$_1$ tun$_3$ zu können$_2$. \label{ex-kohaerenz-zuersatz-1}
\b. *Ich glaube es zu haben$_1$ tun$_3$ können$_2$. \label{ex-kohaerenz-zuersatz-1-b}
\c. *Ich glaube es tun$_3$ können$_2$ zu haben$_1$. \label{ex-kohaerenz-zuersatz-1-c}
\z. \citep[\S 14]{Bech:63}

\cite{Bech:63} zufolge sehen wir hier "`eine Art Kompromi\ss "' für den folgenden Regel-Wider\-spruch:\footnote{Auch andere Autoren gehen auf dieses Phänomen ein. Siehe \citet[70]{Meurers:99} für eine Zusammenstellung.} einerseits besteht gemä\ss\ der OFR\is{Verbalkomplex!Oberfeldregel} eine Unvereinbarkeit des \emph{zu}-Infinitivs mit dem Oberfeld (siehe \ref{ex-kohaerenz-zuersatz-1-b}) und andererseits verlangt die UFR\is{Ersatzinfinitiv!Unterfeldregel}, dass {\it zu haben} als Regent des Ersatzinfinitivs im Oberfeld stehen muss (siehe \ref{ex-kohaerenz-zuersatz-1-b}): {\it zu haben} wird dadurch quasi "`zerissen"'. 

Bech ist jedoch nicht dazu bereit, den 2. Status ad hoc in {\it zu}-Markierung und 1. Status aufzuteilen, denn das wäre eine Ausnahme von einer Regel "`ganz grundlegender Art"', nämlich der Unabtrennbarkeit der {\it zu}-Markierung.\footnote{Wie wir im Zusammenhang mit Abkopplungseffekten\is{Abkopplungseffekt} bzw.\ mit Skandalkonstruktionen sehen werden (siehe S.~\pageref{sec-skandal}), hat \cite{Vogel:09} diesbezüglich keine Skrupel.} Stattdessen schlägt Bech neben dem Ersatzinfinitiv eine weitere Ersetzungsregel vor, bei der in solchen Konfliktfällen ein Verb im 3. Status durch einen \emph{zu}-Infinitiv ersetzt wird. Man kann deshalb auch von einem Ersatz-\emph{zu}-Infinitiv sprechen, wie das etwa \citet[70ff]{Meurers:99} tut ("`substitute \emph{zu}-infinitive"'). Interessanterweise entwirft Bech die Ersetzungsregeln für Ersatz-Infinitive (1.~ER) und Ersatz-\emph{zu}-Infinitive (2.~ER) als konkurrierende Regeln, wobei der 2.~ER Vorrang eingeräumt wird. Es scheint  mir dagegen bedenkenswert, die 2.~ER als Folgeregel zur 1.~ER zu betrachten, so dass die Ersatzinfinitive\is{Ersatzinfinitiv} unter Umständen durch \emph{zu}-Infinitive ersetzt werden. Dieses Verständnis des Ersatz-\emph{zu}-Infinitivs wird durch die Tatsache unterstützt, dass nur solche Verben einen Ersatz-\emph{zu}-Infinitiv zugewiesen werden kann, die auch einen Ersatzinfinitiv erlauben. Zur Illustration zeigt \ref{ex-bech-6-a} einen Ersatz-\emph{zu}-Infinitiv des AcI-Verbs\is{AcI-Verb} {\it sehen}. Wie bei der Darstellung der Ersatzinfinitive angesprochen wurde, erlauben solche Verben anders als Modalverben eine Alternation mit der Partizip-II-Form. Dadurch ist in \ref{ex-bech-6-b}, anders als in \ref{ex-kohaerenz-zuersatz-1-c}, eine Vermeidung des Ersatz-\emph{zu}-Infinitivs möglich:\footnote{Statt des Partizip-II in \ref{ex-bech-6-b} kann auch der Ersatzinfinitiv stehen, wie wir in \ref{ex-2status-uf} und \ref{ex-2status-uf-corpus} gesehen haben.}

\ex. 
\a. Ich freue mich sie haben kommen zu sehen. \label{ex-bech-6-a}
\b. Ich freue mich sie kommen gesehen zu haben. \label{ex-bech-6-b}
\z. \citep[\S 20]{Bech:63}

Die Annahme von Ersatz-\emph{zu}-Infinitiven erklärt jedoch nur einen Teilaspekt des Phänomens: Zwar ist nun erklärt, wie {\it haben} ein Verb im 2. Status regieren kann, weiterhin unklar ist aber, warum in den obigen Beispielen {\it freuen} und {\it glauben} ein Verb im 1. Status regieren. Anders gesagt, bisher ist nur das Ziel der \emph{zu}-Verschiebung betrachtet worden -- was aber passiert am Ursprung, d.\,h.\ am Verb {\it haben}, das ja eigentlich den 2. Status tragen soll? Leider lässt sich bei \cite{Bech:63} nichts dazu finden. Im Geiste der Ersatz-Status lie\ss e sich ein weiterer Ersatzinfinitivtyp vorstellen, der ausschlie\ss lich den 2. Status des Perfekt-Hilfsverbs {\it haben} ersetzt, sofern dessen unmittelbar regiertes Verb einen Ersatz-\emph{zu}-Infinitiv aufweist.  

Einen anderen interessanten Vorschlag macht \citet[189ff]{Meurers:99}: {\it haben} in \ref{ex-kohaerenz-zuersatz-1} und \ref{ex-bech-6-a} sei kein regierender Kopf, sondern ein \isi{Markierer} und in dieser Funktion nicht Teil der subordinativen Kette\is{subordinative Kette}. {\it haben} regiert also weder den Ersatz-\emph{zu}-Infinitiv noch wird es regiert. Beim Ersatz-\emph{zu}-Infinitiv handelt es sich demzufolge um einen gewöhnlichen \emph{zu}-Infinitiv, der vom vermeintlichen {\it haben}-Regens direkt regiert wird. Meurers wendet diesen Markiereransatz auch bei anderen Oberfeldum- und Zwischenstellungen\is{Verbalkomplex!Oberfeldumstellung}\is{Verbalkomplex!Zwischenstellung} an und wir werden im Theorieteil in Abschnitt \ref{sec-ttmctag-rekt} kurz darauf zurückkommen. 
\is{Ersatz-\emph{zu}-Infinitiv|)}    


\subsection{Fernpassiv}\label{sec-fernpassiv}\is{Passiv!Fern-|(}

Auf eine Besonderheit der Kasusmarkierung\is{Kasus} in passivierten kohärenten Konstruktionen hat zuerst \citet[175ff]{Hoehle:78} hingewiesen. Bei der Passivierung\is{Passiv} von \ref{ex-meurers-303-a} in \ref{ex-meurers-303-b} beispielsweise kann das Objekt des eingebetteten Verbs {\it zu reparieren} im Nominativ stehen, obwohl eine Akkusativrektion durch {\it zu reparieren} vorliegt:

\ex.
\a. wenn Karl den Wagen zu reparieren versucht \label{ex-meurers-303-a}
\b. wenn der Wagen zu reparieren versucht wird \label{ex-meurers-303-b}
\z. \citep[176]{Hoehle:78}

Solche Passivkonstruktionen werden üblicherweise \textsc{Fernpassiv} genannt. Der Aspekt der Alternation von Nominativ und Akkusativ wird in diesem Zusammenhang auch oft als \textsc{Kasuskonversion} (\citealt[Abschnitt~9.3]{Haider:93}; \citealt[Abschnitt~4.1]{Woellstein:01}) bezeichnet. 

Man beachte, dass in \ref{ex-meurers-303-b} eine Kasusmarkierung mit dem Akkusativ noch immer möglich ist. Diese Ambiguität bezüglich der Kasusmarkierung, die \ref{ex-meurers-307-a} vor Augen führt, wird durch die Verfügbarkeit unterschiedlicher Kohärenzfeldanalysen\is{Kohärenzfeld} erklärt, d.\,h.\ für \ref{ex-meurers-307-a} kann sowohl eine kohärente als auch eine inkohärente Analyse angegeben werden. Dass tatsächlich nur kohärente Konstruktionen\is{kohärente Konstruktion} von der Kasuskonversion betroffen sind, zeigt der Kontrast zu \ref{ex-meurers-307-b}. Durch die Vorfeldstellung\is{Vorfeldstellung} der Nominalphrase ist die inkohärente Analyse, also die Bestimmung zweier Kohärenzfelder, unmöglich. Es muss deshalb im Nominativ stehen. Bei eindeutig inkohärenten Konstruktionen\is{inkohärente Konstruktion} wie \ref{ex-meurers-307-c} ist die Konversion des Akkusativs dagegen unzulässig und das Fernpassiv nicht bildbar (\citealt[136]{Kiss:95}; \citealt[304]{Meurers:99}):  

\ex. \label{ex-meurers-307}
\a. wenn der/den Wagen zu reparieren versucht wird \label{ex-meurers-307-a}
\b. Der/*Den Wagen wird zu reparieren versucht.\footnote{Die Unakzeptabilität des Akkusativs ist bei Meurers nicht im Beispiel annotiert, aber aus dem Text erschlie\ss bar.} \label{ex-meurers-307-b}
\c. Obwohl versucht wurde, *der/den Wagen zu reparieren \label{ex-meurers-307-c}
\z. \citep[(306), (307)]{Meurers:99}

Die Kasuskonversion macht auch nicht halt vor obligatorischen Ergänzungen\is{Ergänzung}, obwohl sie mir hier schwieriger erscheint als bei fakultativen Ergänzungen, vgl.\ \ref{ex-fernpassiv-obl}: 

\ex. \label{ex-fernpassiv-obl}
\a. Der prügelnde Ehemann wurde wiederholt zu verlassen versucht.
\b. weil der Konflikt zu beenden versucht wurde \hfill \citep[(1a)]{Sabel:01}
\c. Die Nordwand wird im Sommer wieder zu besteigen versucht.

Unter diesem Eindruck ist die Erklärung unplausibel, dass in Fällen der Kasuskonversion Verben ohne Akkusativobjekt verwendet werden. Darüber hinaus ist erstaunlich, dass der Akkusativ selbst dann nicht realisiert werden kann, wenn unpersönliches Passiv\is{Passiv!unpersönliches} zur Verfügung stehen sollte. Beispielsweise ist die Bildung eines unpersönlichen Passivs mit einem Dativ-Objekt eines eingebetteten Verbs in \ref{ex-fernpassiv-dativ} möglich, während die analoge kohärente Konstruktion mit Akkusativobjekt \ref{ex-fernpassiv-acc} deutlich schlechter ist:

\ex. 
\a. Ihm wird zu helfen versucht. \label{ex-fernpassiv-dativ}
\b. *Ihn wird zu reparieren versucht. \label{ex-fernpassiv-acc}

Diese formale Akkusativphobie des Passivs spricht dafür, dass das Akkusativobjekt in kohärenten Konstruktionen syntaktisch nicht primär in Beziehung zum regierenden Verb steht, sondern zum \isi{Verbalkomplex} als gebündelter Einheit (siehe \citealt[253]{Haider:93}).\footnote{Unter bestimmten Umständen, etwa bei Vorhandensein einer generischen Lesart, scheint der Akkusativ bei Passivkonstruktionen aber möglich zu sein. Siehe \citet[301ff]{Meurers:99}.} %\\

In der Literatur ist der grammatiktheoretische Status des Fernpassivs umstritten. Einerseits gibt es Stimmen, das Fernpassiv als ein gewöhnliches Kohärenzphänomen aufzufassen, das prinzipiell alle kohärent konstruierenden Verben involvieren kann (siehe z.\,B.\ \citealt{Wurmbrand:01}; \citealt{Sabel:01}); andererseits wird auch die Sichtweise vertreten, dass das Fernpassiv ein Ausdruck idiosynkratischer Eigenschaften weniger kohärent konstruierender Verben und im Allgemeinen ungrammatisch ist (siehe z.\,B.\ \citealt[177f]{Hoehle:78}; \citealt[Abschnitt~3.3.1.4]{Kiss:95}; \citealt{Reis:Sternefeld:04}; \citealt{Grosse:05}). Die Beweislast zugunsten oder zuungunsten einer dieser Positionen hängt ma\ss geblich von dem Anteil der kohärent konstruierenden Verben ab, die Fernpassiv ermöglichen. Tatsächlich ist die Menge der anerkannten Fernpassiv-Verben sehr klein, auch wenn hier unterschiedliche Einschätzungen konkurrieren. 

Seit \cite{Hoehle:78} stehen die Verben {\it versuchen} und {\it erlauben} im Fokus der Fernpassiv"=Untersuchungen, während andere Verben in dieser Verwendung keine breite Akzeptanz gefunden haben.\footnote{Eine Fernpassiv-Konstruktion mit \textit{erlauben} findet sich bereits bei \cite{Bech:55}:\\
\fnex{
\ex. Keine Zeitung wird ihr zu lesen erlaubt. \hfill \citep[\S 350]{Bech:55}

}} Dazu zählen die Verben, die \citet[16]{Wurmbrand:01} "`lexical restructuring predicates"' nennt: {\it beabsichtigen}, {\it beginnen}, {\it empfehlen}, {\it erlauben}, {\it gelingen}, {\it gestatten}, {\it misslingen}, {\it untersagen}, {\it verbieten}, {\it vergessen}, {\it vermeiden}, {\it versäumen}, {\it versuchen}, {\it wagen}. Bei der introspektiven Bewertung solcher Verben sind allerdings häufig Unsicherheiten im Spiel, denen mittels Fragezeichenurteilen Ausdruck verliehen wird.\footnote{Auch hier kann man wohl einen Trainingeffekt vermuten. \citet[31]{Grosse:05} jedenfalls beobachtet, dass {\it versuchen} und {\it erlauben} "`hochfrequente"' Kontrollverben\is{Kontrolle} sind, und sieht sich daher zur Formulierung folgender These berechtigt: "`Nur bei sehr frequenten Verben und bei Abwesenheit von \glq Störfaktoren\grq\ können solch sensible Konstruktionen wie Fernpassiv unmarkierte Ergebnisse erzielen"'.} Eine bessere Entscheidungsgrundlage versprechen da Korpusstudien zu sein. So hat \citet[30f]{Grosse:05} in COSMAS\footnote{\url{http://www.ids-mannheim.de/cosmas2/}} nur Belege für Fernpassiv-Konstruktionen mit {\it versuchen} und {\it erlauben} finden können (vgl.\ auch \citealt[Abschnitt~3.1.4.1]{Mueller:02}), wohingegen es \cite{Wurmbrand:03} mittels einer Google-Recherche gelingt, Fernpassiv-Konstruktionen mit {\it beginnen}, {\it vergessen} und {\it wagen} empirisch nachzuweisen. Hier gibt es sicherlich noch mehr zu entdecken.  
\is{Passiv!Fern-|)}
 
\subsection{Abkopplungseffekt oder Skandalkonstruktion} \label{sec-skandal}
\is{Abkopplungseffekt|(}
  
Zuletzt möchte ich auf eine interessante Beobachtung von Marga Reis \citep{Reis:79} zu sprechen kommen, die ebenfalls nicht recht in die Bech'sche Systematik zu passen scheint, sofern es die Interpretation der Rektionsbeziehung betrifft. Der Satz in \ref{ex-semantik-reis} ist oberflächlich betrachtet regelkonform:
{\linepenalty100
\ex. \label{ex-semantik-reis} Eine Pariserin namens Dimanche soll sich ein gewaltiges Stirnhorn operativ entfernt$_3$ haben$_2$ lassen$_1$. \hfill \citep[15]{Reis:79}

}
\noindent Die folgende \isi{subordinative Kette} ist erkennbar: Das finite Verb {\it soll} regiert den 1. Status von {\it lassen}, {\it lassen} regiert den 1. Status von {\it haben}, und schlie\ss lich {\it haben} den 3. Status von {\it entfernt}. Wenn dementsprechend die Rektionsbeziehungen\is{Rektion} auf eine Funktor-Argument-Struktur abgebildet werden, so dass jeweils der Regent der Funktor und das regierte Verb sein Argument ist, dann ergibt sich für \ref{ex-semantik-reis} im Wesentlichen die Funktor-Argument-Struktur in \ref{ex-semantik-reis-fa-a}:\footnote{Vgl.\ \citet[(159)]{Meurers:99}.}

\ex. 
\a. {\tt angeblich(lassen(D,perf(entfernen(S))))} \label{ex-semantik-reis-fa-a}
\b. {\tt angeblich(perf(lassen(D,entfernen(S))))} \label{ex-semantik-reis-fa-b}   

Tatsächlich soll jedoch die Funktor-Argument-Struktur in \ref{ex-semantik-reis-fa-b} zutreffend sein, in der die Funktor-Argument-Beziehung zwischen {\tt perf}, der Interpretation von {\it haben}, und {\tt lassen} im Vergleich zu \ref{ex-semantik-reis-fa-a} invertiert ist. Die Funktor-Argument-Beziehung ist hier also von der Rektionsbeziehung abgekoppelt. Dass dies kein Ergebnis semantischer Esoterik darstellt, zeigen die Paraphrasen in \ref{ex-semantik-reis-alt}, in denen {\it haben} und {\it lassen} alternierend aus dem \isi{Verbalkomplex} in die linke Satzklammer \glqq angehoben\grqq\ wurden: 
  
\ex. \label{ex-semantik-reis-alt}
\a. Sie hat es sich entfernen lassen. \label{ex-semantik-reis-alt-a}
\b. *Sie lie\ss \ es sich entfernt haben. \label{ex-semantik-reis-alt-b}
\z. (vgl.\ \citealt[(160)]{Meurers:99})

Überraschend ist hier zunächst, dass die Anhebung von {\it lassen} in \ref{ex-semantik-reis-alt-b} zu einem unakzeptablen Ergebnis führt, obwohl {\it lassen} in \ref{ex-semantik-reis} die Position des maximal übergeordneten Verbs des Verbalkomplexes einnimmt. Im Normalfall sollte eine solche Anhebung ohne Probleme durchführbar sein. Anscheinend ist es aber so, dass die Anhebung von {\it lassen} nur die Interpretation in \ref{ex-semantik-reis-fa-a} zulässt, welche aus irgendwelchen semantischen Gründen schlecht ist. Dafür spricht auch, dass die Anhebung von {\it haben} keinerlei Probleme erkennen lässt und eindeutig mit der Interpretation in \ref{ex-semantik-reis-fa-b} korreliert. Interessant ist auch, dass eine Invertierung des Rektionsverhältnisses von {\it haben} und {\it lassen} wie in \ref{ex-semantik-invert} im Vergleich zu \ref{ex-semantik-reis} keine gravierenden Bedeutungsunterschiede bewirkt:\largerpage%    

\ex. \label{ex-semantik-invert}
\a. Sie soll es sich operativ haben entfernen lassen. \label{ex-semantik-invert-a}
\b. ?Sie soll es sich operativ entfernen lassen haben.

Es erscheint mir also überlegenswert, für den Abkopplungseffekt in \ref{ex-semantik-reis} und die Unverfügbarkeit von \ref{ex-semantik-reis-alt-b} semantische Gesetzmä\ss igkeiten verantwortlich zu machen. Eine Ausarbeitung dieser Hypothese ist jedoch au\ss erhalb des eigentlichen Interesses dieser Arbeit.  

Einen anderen Erklärungsansatz für diese Syntax-Semantik-Diskrepanz wählt \cite{Vogel:09}. Er geht von einer strikten Isomorphie von Rektionsbeziehung und Funktor-Argument-Bezie\-hung aus und nimmt deshalb für Verbalkomplexe wie {\it entfernt haben lassen} in \ref{ex-semantik-reis}, die er als \textsc{Skandalkonstruktion} bezeichnet, kein 321-Schema an, sondern ein 312-Schema. Der Verbalkomplex zeigt also eine Zwischenstellung\is{Verbalkomplex!Zwischenstellung}. Vogel muss nun plausibel machen, warum die Rektionseigenschaft von {\it lassen} verletzt wird, indem es anstatt des 1. Status nun den 3. Status von {\it entfernt} regiert. Noch schlimmer kommt es bei Sätzen wie \ref{ex-skandal-1}:\largerpage% 

\ex. Er bedauert, es nicht verhindert haben zu können. \hfill \citep[(1)]{Vogel:09}\label{ex-skandal-1}

Unterstellt man auch hier dem Verbalkomplex ein 312-Schema, dann ist auch die Rektion des 2. Status von {\it zu können} erklärungsbedürftig, denn {\it haben} regiert ja gemeinhin den 3. Status. Au\ss erdem müsste {\it haben} als maximal übergeordnetes Verb in dieser Konstruktion den 2. Status tragen, der von {\it bedauert} regiert wird. Damit sind also alle Verben des Verbalkomplexes im falschen Status. Dies verwundert um so mehr, als es ja Alternativen zu \ref{ex-skandal-1} mit einer wohlgeformten Rektionskette gibt:  

\ex. Er bedauert, es nicht verhindern gekonnt zu haben. \hfill \citep[(9d)]{Vogel:09}\label{ex-skandal-2}

Wie motiviert also Vogel die Existenz dieser für die Bech'sche Systematik wahrlich skandalösen Konstruktionen? Seine Analysestrategie besteht im Wesentlichen aus der Formulierung verbalkomplexglobaler Positionsregeln für die Statusmorphologie, die in einem optimalitätstheoretischen Mechanismus\is{Optimalitätstheorie} mit den Rektionseigenschaften der Verben interagieren. Die Statusmorphologie dient demnach nicht nur dazu, lexikalische Selektionsverhältnisse abbzubilden, sondern auch dazu, hierarchische Verhältnisse in der subordinativen Verbalkette\is{subordinative Kette} zu signalisieren. Vogel stipuliert dafür eine "`Hierarchie der verbalen Statusformen"' und erklärt die Grammatikalität von \ref{ex-skandal-1} mit einer im Vergleich zu \ref{ex-skandal-2} stärkeren Hierarchiesignalisierung, die die Abweichung von der semantisch erwarteten Rektionskette ausgleichen soll.\footnote{\citet[(52)]{Vogel:09} nimmt die ideale Hierarchie "`finites Verb > Infinitiv > Partizip"' an. Das erfüllt die Skandalkonstruktion in \ref{ex-skandal-1}, aber nicht die Variante in \ref{ex-skandal-2}.}

Doch man könnte stattdessen auch die Isomorphieannahme zwischen Rektionskette und Funktor-Argument-Struktur abschwächen. Das Perfekthilfsverb\is{Hilfsverb} würde dann die Fähigkeit besitzen, bei der Interpretation aus der Rektionskette auszubrechen und weiten Skopus zu nehmen.\footnote{Dieser Ansatz passt auch gut zur Markiererhypothese\is{Markierer} von \citet[Abschnitt~8.2]{Meurers:99}, auf die in Abschnitt~\ref{sec-ttmctag-rekt} kurz eingegangen werden wird.} In jedem Fall ist an diesen Konstruktionen nicht die Rektionskette an sich, sondern die Semantik der Rektionskette entscheidend. Und da die in dieser Arbeit eigentlich nicht im Vordergrund steht, werde ich darauf im Analyseteil auch nicht eingehen.  
\is{Abkopplungseffekt|)}

\section{Diskontinuierliche Kohärenzfelder} \label{sec-permutation-kohaerenzfeld}

Gemä\ss\ \cite{Bech:55} besteht ein \isi{Kohärenzfeld} aus einem kontinuierlichen \isi{Restfeld} und einem kontinuierlichen Schlussfeld\is{Verbalkomplex}, wobei das Restfeld dem Schlussfeld unmittelbar vorangeht.  Das Kohärenzfeld ist also selbst eine kontinuierliche topologische Einheit. In \ref{ex-kf-schema} wurde dieser topologische Zusammenhang als $K^i = R^i S^i$ aufgeschrieben.  Wir haben jedoch schon bei dem Datum in \ref{ex-bech-4} eine Ahnung davon erhalten, dass diese Charakterisierung in manchen Fällen von Kohärenz nicht zutrifft. Bech scheinen die im Folgenden dargestellten Stellungsmöglichkeiten, die seinem Kohärenzfeldbegriff widersprechen, entgangen zu sein. 

\subsection{3.~Konstruktion}
\is{kohärente Konstruktion!3.~Konstruktion|(}

Wie oben bereits angesprochen, muss man für Satz \ref{ex-bech-4-b2} entsprechend der \isi{Rangprobe} zwei Schlussfelder und damit zwei getrennte Kohärenzfelder annehmen: 

\ex. da\ss\ sie eine Absicht glaubten$_1$ | verbergen$_2$ zu können$_1$ \label{ex-bech-4-b2}

%\largerpage%
Dieser Satz müsste also entsprechend der Bech'schen Kohärenzfeldtopologie die Struktur $R^1$ $S^1 S^2$ haben. Schaut man sich jedoch das, was $R^1$ sein sollte, genauer an, so stellt man eine unzulässige Unterbrechung fest: $R^1$ enthält auch das zu $S^2$ gehörige Objekt {\it eine Absicht}. Die Kohärenzfeldstruktur ist also tatsächlich $R^1 R^2 S^1 S^2$, womit ein Widerspruch zur Bech'schen Kohärenzfeldtopologie hergestellt ist, da die Restfelder nicht linksadjazent zum jeweiligen Schlussfeld stehen.\footnote{Auf dieses Problem haben zuerst \cite{Kvam:79} und \citet[331, Fußnote 4]{Hoehle:86} hingewiesen.} Satz \ref{ex-bech-4-b2} weist also gleichzeitig kohärente Eigenschaften (bezgl.\ des Restfelds) und inkohärente Eigenschaften (bezgl. des Schlussfelds) auf.  

Dieses Zwitterwesen wird in der Folge von \cite{denBesten:Rutten:89} als \textsc{3. Konstruktion} bezeichnet. Die Abgrenzung zu den anderen beiden Konstruktionstypen wird in \ref{ex-3konstr-1} nochmals exemplarisch verdeutlicht:

\ex. \label{ex-3konstr-1}
\a. dass er versucht, den Wagen ohne Werkzeug zu reparieren	\\
(\isi{Extraposition}, inkohärent) \label{ex-3konstr-1-a}	
\b. dass er den Wagen ohne Werkzeug zu reparieren versucht	\\
(Intraposition\is{kohärente Konstruktion!Intraposition}, kohärent oder inkohärent) \label{ex-3konstr-1-b}
\c. dass er den Wagen versucht, ohne Werkzeug zu reparieren \\
dass er ohne Werkzeug versucht, den Wagen zu reparieren \\
(3. Konstruktion, partiell kohärent) \label{ex-3konstr-1-c}

Der Konstruktionstyp in \ref{ex-3konstr-1-a} zeigt die Extraposition eines vollständigen Kohärenzfelds und ist somit eindeutig inkohärent. Dagegen kann die Intraposition eines statusregierten Kohärenzfelds in \ref{ex-3konstr-1-b} als kohärent analysiert werden. Ein entsprechendes Beispiel für die 3.~Konstruktion ist schlie\ss lich in \ref{ex-3konstr-1-c} wiedergegeben. Man erkennt die \textsc{partielle Extraposition} des Kohärenzfelds $K^2 =$ ({\it den Wagen, ohne Werkzeug}) ({\it zu reparieren}), wovon {\it den Wagen} herausgelöst im Mittelfeld platziert ist. Im Unterschied zu \ref{ex-bech-4-b2} sehen wir in \ref{ex-3konstr-1-c} au\ss erdem, dass der extraponierte Teil eines Kohärenzfelds auch aus nicht-verbalen Bestandteilen (hier {\it ohne Werkzeug}) bestehen kann, dass also das \isi{Restfeld} diskontinuierlich und Teile des Restfelds extraponiert sein können. Die Kohärenzfeldstruktur des Satzes ist demnach so etwas wie $R^1 R^2 S^1 R^2 S^2$. Das extraponierte Restfeld verhält sich aber in einem wichtigen Punkt anders als das intraponierte Restfeld: Während das intraponierte Restfeld in dem Restfeld $R^1$ des statusregierenden Verbs in kohärenter Weise aufgehen kann, ist das extraponierte Restfeld für Bestandteile von $R^1$ unzugänglich. Es ist also nicht möglich, das $R^1$-Nomen {\it er} zusammen mit dem extraponierten Teil von $K^2$ im \isi{Nachfeld} zu platzieren:

\ex. *dass den Wagen versucht, er ohne Werkzeug zu reparieren. 

Dass diese Einschränkung nicht nur für das \isi{Subjekt} gilt, sondern für alle Bestandteile der regierenden Verbalfelder\is{Verbalfeld}, demonstrieren die Beispiele in \ref{ex-3konstr-2}:
\largerpage%

\ex. \label{ex-3konstr-2}
\a. *Er hat gebeten, ihn zu kommen.
\b. Er hat versprochen, am Montag am Mittwoch zu kommen. \\
$\neq$ Er hat am Montag versprochen, am Mittwoch zu kommen.     

Einen deutlichen kasusmorphologischen Reflex dieses Nebeneinanders von kohärenten und inkohärenten Bereichen kann man bei der Bildung des Fernpassivs\is{Passiv!Fern-} beobachten. So regiert das finite Passivhilfsverb\is{Hilfsverb} in \ref{ex-woellstein-12-a} den Nominativkasus\is{Kasus} des Objekts von {\it zu füttern}. Dagegen kann die Nominativrektion nicht auf Bestandteile des extraponierten Restfelds in \ref{ex-woellstein-12-b} einwirken. Das zuvor im Nominativ realisierte Objekt trägt nun den Akkusativ, d.\,h.\ die Rektionseigenschaften von {\it zu füttern} kommen frei zur Entfaltung:      

\ex. 
\a. weil der Hund vergessen wurde zu füttern\\
\citep[(12c)]{Woellstein:01}\label{ex-woellstein-12-a} 
\b. weil vergessen wurde, *der/den Hund zu füttern\label{ex-woellstein-12-b}
  
Bemerkenswert an der 3.~Konstruktion ist auch, dass das partiell extraponierte Kohärenzfeld von einem beliebig tief eingebetteten Verb des übergeordneten Kohärenzfelds regiert werden kann:

\ex. dass er den Wagen hätte$^1$ versuchen$^3$ können$^2$, ohne Werkzeug zu repa\-rie\-ren$^4$.

Diesbezüglich herrscht kein Unterschied zur vollständigen Extraposition.

Tritt die 3.~Konstruktion mehrmals in einer subordinativen Kette mit fakultativ kohärent konstruierenden Verben\is{Kohärenz!obligatorische/fakultative} auf, so erwarten wir, dass ein Restfeld R$^i$ teilweise oder ganz in einem höhergeordneten Restfeld $R^j$ mit $i<j$ aufgehen kann. Diese Erwartung findet sich in einem Beispiel von \cite{Sabel:95} leider nur zum Teil bestätigt:\footnote{Siehe auch die Daten in \citet[18]{Rambow:94}.}

{\setlength{\Exlabelsep}{0.8em} \setlength{\SubExleftmargin}{1.7em}
\ex.
\a. da\ss\ keiner wagte, dem Fritz$^2$ zu erlauben$^2$, den Wagen$^3$ zu reparie\-ren$^3$
\b. *da\ss\ keiner wagte, den Wagen$^3$ dem Fritz$^2$ zu erlauben$^2$, zu reparie\-ren$^3$\label{ex-sabel-28-b}
\c. da\ss\ den Wagen$^3$ dem Fritz$^2$ keiner wagte, zu erlauben$^2$, zu reparie\-ren$^3$
\d. da\ss\ dem Fritz$^2$ keiner wagte, zu erlauben$^2$, den Wagen$^3$ zu reparie\-ren$^3$
\e. da\ss\ den Wagen$^3$ keiner wagte, dem Fritz$^2$ zu erlauben$^2$, zu reparie\-ren$^3$
\z. \citep[(28)]{Sabel:95}

}

\largerpage[-1] %long distance for page+2
\noindent Satz \ref{ex-sabel-28-b} sollte unserer Erwartung entsprechend ebenfalls akzeptabel sein, ist es aber nach Sabels Einschätzung nicht.\footnote{Sabel führt die Ungrammatikalität von \ref{ex-sabel-28-b} auf eine spezifische Bewegungsbeschränkung zurück, die eine Bewegung in Infinitiv-Gliedsätze ("`infinitivals"') verhindert (vgl.\ \citealt[418]{Sabel:95}). } Sein Grammatikalitätsurteil wird allerdings von \cite{Kulick:00} in Zweifel gezogen, der es in einer (nicht näher spezifizierten) Informantenbefragung nicht zu reproduzieren vermag und anstelle der strikten Ungrammatikalität für ein Fragezeichen-Bewertung plädiert. Kulick stellt darüber hinaus fest, dass \ref{ex-sabel-28-b} mit einer Intraposition anstelle der partiellen Extraposition die Tendenz des Grammatikalitätsurteils nicht signifikant beeinflusst:

\ex. ?/*\label{ex-kulick-37}da\ss\ keiner wagte, dem Fritz$^2$ den Wagen$^3$ zu reparieren$^3$ zu erlauben$^2$ \\ \citep[(37)]{Kulick:00} 

Möglicherweise haben wir es hier also mit einem \isi{Verarbeitungseffekt} zu tun. Zumindest in \ref{ex-kulick-37} erscheint mir das als wahrscheinlich, da hier das Grammatikalitätsempfinden schlagartig verbessert werden kann, wenn man das Verbalfeld {\it den Wagen zu reparieren} durch  eine NP (z.\,B.\ {\it dieses Unterfangen}) ersetzt. 

%\ex. ?da\ss\ dem Fritz$^2$ keiner wagte, den Wagen$^3$ zu erlauben$^2$ zu reparieren$^3$ \\ \cite[(36-f)]{Kulick:00} 

Kommen wir noch kurz zur Klärung der Frage, welche Verben die maximal übergeordneten Verben eines partiell extraponierten Kohärenzfelds regieren können. Die Vermutung liegt nahe, dass solche Verben zwei Eigenschaften haben müssen: (i) sie konstruieren kohärent und (ii) sie erlauben die Extraposition des statusregierten Verbs, d.\,h.\ sie können auch inkohärent konstruieren. Mit anderen Worten: Die Erwartung besteht, dass genau die fakultativ kohärent konstruierenden Verben\is{Kohärenz!obligatorische/fakultative} diesen Konstruktionstyp zulassen. Mit Blick auf Verben, die den 1. oder 3. Status regieren und obligatorisch kohärent konstruieren, wird diese Vermutung bestätigt. Die vollständige Extraposition in \ref{ex-3konstr-modal-b} und \ref{ex-3konstr-modal-d}, sowie die partielle Extraposition in \ref{ex-3konstr-modal-a} und \ref{ex-3konstr-modal-c} sind zweifelsohne unakzeptabel:\footnote{Man beachte, dass die Sätze in \ref{ex-3konstr-modal} wegen des fehlenden Oberfelds keine \isi{Linksstellung} aufweisen (siehe S.~\pageref{sec-linksstellung}). } 

\ex. \label{ex-3konstr-modal}
\a. *dass er den Wagen muss ohne Werkzeug reparieren \label{ex-3konstr-modal-a}
\b. *dass er muss den Wagen ohne Werkzeug reparieren \label{ex-3konstr-modal-b}
\c. *dass er den Wagen hat ohne Werkzeug repariert \label{ex-3konstr-modal-c}
\d. *dass er hat den Wagen ohne Werkzeug repariert \label{ex-3konstr-modal-d}

Auch obligatorisch kohärente Anhebungsverben\is{Anhebung} wie {\it scheinen}, die den 2. Status regieren, sind wohl aufgrund des inkohärenten Teilaspekts mit der 3. Konstruktion nicht vereinbar:

\ex. *dass er ohne Werkzeug scheint, den Wagen zu reparieren\label{ex-3konstr-scheinen}
 
Im Unterschied dazu sollten Anhebungsverben wie {\it drohen} und {\it anfangen}, denen fakultative Kohärenz nachgesagt wird (siehe Tabelle~\ref{fig-kohaerenzklassen}), die 3. Konstruktion zulassen. Das ist auch tatsächlich der Fall:\footnote{Siehe \citet[Abschnitt~3.2.1]{Meurers:99}, \citet[Abschnitt~2.1.4.3]{Mueller:02} für weitere Daten und Literaturangaben.} 

{\linepenalty100
\ex.
\a. Im Herbst schlie\ss lich stoppte Apple die Auslieferung einiger Power Books, weil sie drohten, sich zu überhitzen und in Flammen aufzugehen. \hfill (\citealt[(71a)]{Meurers:99}; \citealt[(111a)]{Mueller:02})\footnote{Der Korpusbeleg geht auf Stefan Müller zurück.} 
\b. als der Ballon drohte, in einen Strudel zu geraten \\ \citep[41a]{Grosse:05}\footnote{Grosse zitiert hier aus einer unveröffentlichten Arbeit von Marga Reis.}
\c. Als mir erneut anfing, schlecht zu werden, \ldots \hfill \citep[(70a)]{Meurers:99}

}
\noindent Bei anderen fakultativ kohärent konstruierenden Verben\is{Kohärenz!obligatorische/fakultative} stellt dagegen \cite{Woellstein:01}, wie oben schon angesprochen, eine unerwartete Asymmetrie zwischen der 3.~Konstruktion und dem Fernpassiv\is{Passiv!Fern-} fest:

\ex. 
\a. obwohl der Wagen zu reparieren abgelehnt wurde
\b. \% obwohl er den Wagen ablehnte zu reparieren
\z. \citep[318]{Woellstein:01} 

Nach Wöllstein-Leistens Einschätzung ist {\it ablehnen} (und auch {\it aufgeben}) in dieser Konstruktion deutlich markiert. Die oben formulierte Erwartung scheinen also einzelne fakultativ kohärente Verben zu enttäuschen. Erwartungsgemä\ss\ können dagegen Akkusativkontrollverben\is{Kontrolle} wie {\it auf"|fordern}, {\it anflehen}, {\it unterlassen}, {\it ersuchen}, die ausschlie\ss lich inkohärent konstruieren, \cite{Grosse:05} zufolge die 3.~Konstruktion nicht bilden:

\ex. 
\a. *weil der Richter den Zeugen das Geschehen$^2$ auf"|fordert vorzutragen$^2$
\b. *weil Oma ihren Enkel das Abenteuer$^2$ anfleht zu unterlassen$^2$
\c. *weil Hans den Nachbarn die Bäume$^2$ ersucht zu fällen$^2$ 
\z. \citep[(32)]{Grosse:05}

Diese Sätze als strikt ungrammatisch hinzustellen, halte ich jedoch für übertrieben.
\is{kohärente Konstruktion!3.~Konstruktion|)}



\subsection{Partielle Voranstellung} \label{sec-kohaerenz-voran}
\is{Voranstellung!partielle|(}
 
Eine gewisse Parallelität zur 3.~Konstruktionen beobachtet man bei der Voranstellung von Kohärenzfeldern\is{Kohärenzfeld}, d.\,h.\ bei der Platzierung im \isi{Vorfeld} von V2-Sätzen\is{Satz!V2-}. 
Auch hier lässt sich eine vollständige und eine partielle Form der Voranstellung unterscheiden:\footnote{Für eine Literaturübersicht und eine Einordnung in andere (partielle) Voranstellungsphänomene siehe z.\,B.\ \citet[Kapitel~9]{Meurers:99}.}

\ex. \label{ex-voran-1}
\a. Den Wagen$^2$ ohne Werkzeug$^2$ zu reparieren$^2$ versucht er. \\
(vollständige Voranstellung, inkohärent?)
\b. Ohne Werkzeug$^2$ zu reparieren$^2$ versucht er den Wagen$^2$. \\
(partielle Voranstellung, partiell kohärent)\label{ex-voran-1-b}

Bei der partiellen Voranstellung können, wie bei der partiellen Extraposition\is{kohärente Konstruktion!3.~Konstruktion}, kohärente und inkohärente Bereiche identifiziert werden. Bezogen auf \ref{ex-voran-1-b} können Bestandteile des Restfelds des Kohärenzfelds $K^2$ von {\it zu reparieren} zwar im Mittelfeld und Nachfeld platziert werden, also im Kohärenzfeld $K^1$ von {\it versucht}, für Bestandteile von $R^1$ ist das vorangestellte Restfeld $R^2$ aber unzugänglich. Das gilt in \ref{ex-voran-1} etwa für das Subjekt \textit{er} als Bestandteil von $R^1$:

\ex. *Er ohne Werkzeug zu reparieren versucht den Wagen. \label{ex-voran}    

Damit ist aber nicht gesagt, das die Voranstellung des Subjekts\is{Subjekt} mit einer Verbalphrasen generell ausgeschlossen ist. Dazu gleich mehr.

Auch die Kasusmarkierung\is{Kasus} in manchen Fällen von Fernpassiv\is{Passiv!Fern-} lässt darauf schlie\-ßen, dass die Voranstellung vergleichbar der Extraposition einen inkohärenten Bereich definiert, während das Mittelfeld weiterhin Kohärenz zulässt. Beispielsweise ist in \ref{ex-meurers-315} die Kasusüberschreibung mit dem Nominativ, der von {\it wurde} regiert wird, im vorangestellten Restfeld nicht möglich: 
  
\ex. *Der/Den Wagen zu reparieren wurde lange Zeit versucht.\\
\citep[(315b)]{Meurers:99} \label{ex-meurers-315} 

Steht das Zielnomen jedoch im Mittelfeld so wie in \ref{ex-meurers-305} und \ref{ex-reisstern-7}, kann (und muss) der von {\it zu reparieren} regierte Akkusativ ignoriert und stattdessen der Nominativ verwendet werden \citep{Pollard:94}:  

\ex. Zu reparieren versucht wurde der/*den Wagen. \hfill \citep[(305)]{Meurers:99} \label{ex-meurers-305} 

\ex. \label{ex-reisstern-7}
\a. Zu stehlen versucht wurde der/*den Apfel.
\b. Zu entziffern gelungen ist der/*den Brief.
\z. \citep[(7)]{Reis:Sternefeld:04}

Ein weiteres Indiz für die Inkohärenz des vorangestellten Kohärenzfelds (bezogen auf das Kohärenzfeld des regierenden Verbs) liefern obligatorisch inkohärent konstruierende Verben wie {\it glauben} und {\it empfehlen}, die erwartungsgemä\ss\ keine partielle Voranstellung ihrer regierten Verbalfelder erlauben, siehe \ref{ex-meurers-211}: 

\ex. \label{ex-meurers-211}
\a. *Zu verkaufen glaubt er das Pferd nicht.\footnote{Für Stefan Müller (persönliche Mitteilung, 2012) ist der Satz auch ohne partielle Voranstellung aus semantischen Gründen schlecht: {\it Er glaubt, das Pferd nicht zu verkaufen.}} \hfill \citep[(211)]{Meurers:99}
\b. *Zu verkaufen empfahl er ihr das Pferd. \hfill \citep[(205c)]{Meurers:99}

Allenfalls die vollständige Voranstellung ist hier möglich:

\ex. Das Pferd zu verkaufen wird er ihr noch heute empfehlen.\\
\citep[(212c)]{Meurers:99}

Irritationspotential hat die Eigenschaft obligatorisch kohärenter Verben\is{Kohärenz!obligatorische/fakultative}, mit beiden Voranstellungsvarianten kompatibel zu sein. Dies steht in auf"|fälligem Kontrast zu ihrer Unverträglichkeit mit der 3.~Konstruktion\is{kohärente Konstruktion!3.~Konstruktion} (siehe \ref{ex-3konstr-modal}): 

\ex. 
\a. Den Wagen ohne Werkzeug reparieren muss er nicht.
\b. Den Wagen ohne Werkzeug repariert hat er noch nicht.

\ex. Das Pferd verkaufen wird er noch heute wollen. \hfill \citep[(212a)]{Meurers:99}

Die entscheidende Einschränkung besteht jedoch hinsichtlich der Zugänglichkeit des vorangestellten Kohärenzfelds für das \isi{Subjekt} des übergeordneten finiten Verbs. Wie in \ref{ex-voran-er} verdeutlicht, ist das Vorfeld für das Subjekt {\it er} nicht zugänglich:

\ex. \label{ex-voran-er}
\a. \label{ex-voran-er-a} *Er reparieren muss den Wagen nicht.
\b. \label{ex-voran-er-b} *Er repariert hat den Wagen noch nicht.

Bis hierhin spricht alles dafür, dass das vorangestellte Kohärenzfeld einen für übergeordnete Verbalfelder unzugänglichen, ergo inkohärenten Bereich festlegt. Leider wird diese Generalisierung durch eine Reihe von Daten konterkariert, die Indizien dafür liefern, dass das vorangestellte Kohärenzfeld unter bestimmten Umständen kohärent, d.\,h.\ für Einflüsse höhergeordneter Verben zugänglich, sein kann. Beispielsweise erhält \cite{Meurers:99} nach Voranstellung des gesamten nicht-finiten Schlussfelds der Fernpassiv-Konstruktion in \ref{ex-meurers-315} das folgende Ergebnis:

\ex. Der/Den Wagen zu reparieren versucht wurde lange Zeit.\\
\citep[(316b)]{Meurers:99}\label{ex-meurers-316} 

\largerpage%
Seiner Einschätzung zufolge ist im Gegensatz zu \ref{ex-meurers-315} nun die Nominativrektion des Akkusativobjekts von {\it zu reparieren} zulässig. Wie ist das möglich? Der Satz in \ref{ex-meurers-316} reiht sich in eine Gruppe von Sätzen ein, bei denen das \isi{Subjekt} zusammen mit einer nicht-finiten Verbalphrase vorangestellt ist:\footnote{Ich unterschlage hier infinite Kopulakonstruktionen\is{Kopulakonstruktion} wie in \ref{ex-kopula-1}, die zusammen mit einer Nominativ-NP offensichtlich auch ohne Anhebungsverb vorangestellt werden können:\\
\fnex{
\ex. \label{ex-kopula-1} 
Ein Held zu sein macht Spa\ss . \hfill (Gert Webelhuth, zitiert nach \citealt[(322)]{Meurers:99})
 
}
Ausschlaggebend scheint hier die Eigenart von Kopulaverben zu sein, auch in anderen Infinitivkonstruktionen mit einem Prädikatkomplement im Nominativ auftreten zu können, vgl.\ die AcI-Konstruktion\is{AcI-Verb} in \ref{ex-kopula-2}:\\
\fnex{
\ex. \label{ex-kopula-2} Baby, lass mich dein Tanzpartner sein. \hfill \citep[(15.12d)]{Mueller:99} %\cite[273]{Mueller:99}

}
Siehe \citet[314ff]{Meurers:99} für eine kurze Diskussion und weitere Referenzen.}
{\linepenalty100
\ex. 
\a. Ein Fehler unterlaufen ist ihr noch nie. \hfill \citep[(10a)]{Haider:90}
\b. Ein Au\ss enseiter gewonnen hat hier noch nie.\hfill \citep[(10d)]{Haider:90}
\c. Ein Au\ss enseiter zu gewinnen scheint hier eigentlich nie.\\
\citep[(265)]{Meurers:99}\label{ex-meurers-265}
\d. Der Führerschein abgenommen wurde einem Autofahrer am Samstag abend bei Friedrichsdorf. \hfill \citep[(283)]{Meurers:99}

}
\noindent Der gemeinsame Nenner dieser Daten ist Meurers zufolge die Beteiligung eines finiten Anhebungsverbs\is{Anhebung}, das die vorangstellte Verbalphrase regiert und das \isi{Subjekt} anhebt \citep[289ff]{Meurers:99}. Dabei ist es nebensächlich, ob das Subjekt oder das Akkusativobjekt angehoben wird, wie das Beispiel in \ref{ex-meurers-275} mit dem AcI-Verb\is{AcI-Verb} {\it sehen} deutlich macht:\largerpage%    
  
\ex. Den Kanzler tanzen sah der Oskar. \hfill \citep[(275)]{Meurers:99}\label{ex-meurers-275}

Dagegen stellt sich erwartungsgemä\ss\ eine Verschlechterung ein, sobald das Anhebungsverb durch ein Kontrollverb ersetzt wird. \ref{ex-meurers-266} ist daher deutlich schlechter als \ref{ex-meurers-265}: 

\ex. *Ein Au\ss enseiter zu gewinnen versuchte hier noch nie.\\
\citep[(266)]{Meurers:99}\label{ex-meurers-266}

Wir sehen also eine auf Anhebungsverben und deren Anhebungsobjekte be-\linebreak schränkte Form der Kohärenz mit dem regierten, vorangestellten Kohärenzfeld. An diese Voranstellungskonstellationen knüpfen sich der in Literatur meist theoriespezifische Fragen der Kasusmarkierung\is{Kasus}.\footnote{Grundsätzlich muss folgendes (theoriespezifisch) beantwortet werden: Wie erhält das Subjekt seinen Kasus, wenn sich der Kasusregent in einem anderen Konhärenzfeld befindet, zu dem er sich eigentlich inkohärent verhält? Diese Frage stellt sich auch bei Extrapositionsdaten\is{Extraposition} wie \ref{ex-meurers-270}:\\
\fnex{
\ex. Obwohl damals anfing, der/*den Mond zu scheinen. \hfill \citep[(270)]{Meurers:99}\label{ex-meurers-270}

}
\citet[316ff]{Meurers:99} gibt eine Antwort hierauf im Rahmen der HPSG\is{Head-driven Phrase Structure Grammar (HPSG)}.} Ich möchte darauf an dieser Stelle jedoch nicht weiter eingehen, sondern verweise auf den entsprechenden Theorieteil in den Abschnitten~\ref{sec-ttmctag-rekt} und \ref{sec-stug-einfach}.

Doch nicht allein die Beschränkung auf Anhebungsverben\is{Anhebung} scheint für eine geglückte Subjektvoranstellung\is{Subjekt} entscheidend. Wenn man sich nochmals das Datum in \ref{ex-voran-er-b}, hier wiederholt als \ref{ex-voran-er-b2}, in Erinnerung ruft und beispielsweise mit \ref{ex-gerdes-fig6} flankiert, droht das Bild sogleich diffuser zu werden: 

\ex. 
\a. *Er repariert hat den Wagen noch nicht.\label{ex-voran-er-b2} 
\b. *Peter gelesen hat den Roman. \hfill \citep[Abbildung~6]{Gerdes:04}\label{ex-gerdes-fig6} 

Obwohl mit {\it hat} ein Anhebungsverb involviert ist, wirkt die Voranstellung des Subjekts {\it er}/""{\it Peter} in Verbindung mit dem nicht-finiten Verb {\it repariert}/{\it gelesen} unakzeptabel. Wie lässt sich das erklären? Zum einen hat \cite{Haider:90} darauf hingewiesen, dass die Akzeptabilität solcher Voranstellungen durch die Definitheit des Subjekts generell abnimmt ("`definiteness effect"'):   

\ex. 
\a. ??Dieser Fehler unterlaufen ist ihr noch nie.
\b. ??Der Au\ss enseiter gewonnen hat hier noch nie.
\z. \citep[(10)]{Haider:90}

Ebenfalls relevant sei, so Haider, das Vorhandensein und die Platzierung des Akkusativobjekts, was in folgendem Kontrast deutlich werde: 

\ex. \label{ex-haider-12} 
\a. *Ein Au\ss enseiter gewonnen hat da noch nie das Derby.\label{ex-haider-12-a}
\b. Ein Au\ss enseiter gewonnen hat das da noch nie.\label{ex-haider-12-b}
\z. \citep[(12b), (12c)]{Haider:90}

Wird das Akkusativ-Objekt realisiert, dann führt also eine Platzierung in der Wackernagel-Position (als Pronomen) wie in \ref{ex-haider-12-b} zu einem besseren Ergebnis. Ein dritter Faktor scheint mir hingegen in \cite{Haider:90} nicht berücksichtigt zu werden. Man beobachtet nämlich einen rapiden Akzeptanzverfall, wenn das Mittelfeld vollständig oder bis auf das Akkusativ-Objekt entleert ist:%\largerpage% 

\ex. 
\a. ??Ein Au\ss enseiter gewonnen hat.
\b. ??Ein Au\ss enseiter gewonnen hat das.

Beachtet man diese drei Faktoren, dann kann man tatsächlich auch für \ref{ex-voran-er-b2} und \ref{ex-gerdes-fig6} leicht eine weitaus akzeptablere Variante angeben:

\ex.
\a. ?Ein Lehrling repariert hat den Wagen noch nicht.  
\b. ?Ein Kritiker gelesen hat den Roman noch nicht.

Es scheint mir am wahrscheinlichsten, dass hier diskurs- und informationsstrukturelle Faktoren\is{Informationsstruktur} am Werk sind, dass also die hier betrachteten Subjektvoranstellungen\is{Subjekt} nicht primär aus syntaktischen Gründen unakzeptabel oder marginal sind. Vielmehr lässt sich für sie nur schwer ein passender Äu\ss erungskontext imaginieren.\footnote{Siehe die informationsstrukturelle Beschränkung der VP-Voranstellung bei \citet[Kapitel~4, Abschnitt~5.2.1]{Cook:01}: Die vorangestellte Kette müsse eine informationsstrukturelle Einheit bilden ("`the initial string must be a single informational unit"', S.\,198), wobei die Agentivität des Subjekts die Bildung einer solchen Einheit erschwere. Ähnlich äußert sich z.\,B.\ \citet[Abschnitt~5]{Gerdes:04}, rückt aber die Rolle der Agentivität stärker in den Vordergrund.} 

Schlie\ss lich sollte noch die folgende Gesetzmä\ss igkeit erwähnt werden: Die subordinative Kette in der rechten Satzklammer kann nur durch ein Verb in der linken Satzklammer regiert werden. Es ist also nicht zulässig, wie in \ref{ex-mueller-243} eine Lücke in die \isi{subordinative Kette} zwischen rechter und linker Satzklammer zu schlagen: 

\ex. *Müssen wird er ihr ein Märchen erzählen. \hfill \citep[(569a)]{Mueller:02}\label{ex-mueller-243}

Man beachte hier, dass ein analoges Beispiel mit einem vorangestellten fakultativ kohärenten Verb\is{Kohärenz!obligatorische/fakultative} wie {\it versuchen} zwar vorstellbar ist, dass dann aber eine inkohärente Analyse unvermeidbar erscheint: 

\ex. Versuchen wird er, ihr ein Märchen zu erzählen.\label{ex-voran-versuchen} 

\is{Voranstellung!partielle|)}


\subsection{Linksstellung des Oberfelds} \label{sec-linksstellung}
\is{Linksstellung!des Oberfelds|(}

Wie bereits im Zusammenhang mit Verbalkomplexpermutationen\is{Verbalkomplex} erwähnt, be\-steht die Möglichkeit, das Oberfeld\is{Verbalkomplex!Oberfeld} vom Unterfeld\is{Verbalkomplex!Unterfeld} durch nicht-verbales Material des Restfelds zu trennen, es also quasi nach links in das \isi{Mittelfeld} zu verschieben.\footnote{\cite{Haegeman:Riemsdijk:86} nennen das Phänomen im Hinblick auf eine bestimmte Modellierung, auf die wir in Abschnitt \ref{sec-ttmctag-modellierungsstrategien} kurz eingehen werden, "`verb projection raising"'. Man beachte, dass \citet[\S63]{Bech:55} auch in solchen Fällen ein kontinuierliches Restfeld diagnostiziert -- allerdings mit nicht-verbalem Einsprengsel: "`Das schlu\ss feld umfa\ss t im allgemeinen nur verben, und zwar supina und eventuell das verbum finitum, V1(0). Wenn aber ein oberfeld vorhanden ist, so kommt es bisweilen vor, da\ss\ ein nicht-verbales glied, das irgendwie eine nahe verbindung mit dem maximal untergeordneten verbum des schlu\ss feldes hat, unmittelbar vor diesem verbum steht."'} Einige Beispiele sind in \ref{ex-meurers-145} zu sehen:

\ex. \label{ex-meurers-145}
\a. ohne da\ss\ der Staatsanwalt hätte darum bitten müssen
\b. wenn ich nur ein einziges Mal habe glücklich sein dürfen
\c. Es war ein Wackelkontakt, den er mit ein paar Handgriffen hätte in Ordnung bringen können.
\d. da\ss\ er es habe genau erkennen lassen
\z. \citep[(145)]{Meurers:99}

Es ist unklar, ob hinsichtlich des intervenierenden Restfeldmaterials handfeste syntaktische Einschränkungen bestehen. Zwar nehmen etwa \cite{Kefer:Lejeune:74} aufgrund von Daten wie \ref{ex-meurers-146a} an, dass Subjekte\is{Subjekt} von dieser Position ausgeschlossen sind, doch weist \cite{Meurers:99} darauf hin, dass sich hierzu ohne Schwierigkeiten Gegenbeispiele wie \ref{ex-meurers-146b} finden lassen:   

\ex. 
\a. *Sie wu\ss te, da\ss\ vielleicht hätte Paul kommen sollen. \label{ex-meurers-146a}
\b. Da\ss\ ihn gestern hätte jemand besiegen können, ist unwahrscheinlich. \label{ex-meurers-146b}
\z. \citep[(146)]{Meurers:99}

Es bleibt die vage Vermutung, dass das eingeschlossene Restfeldmaterial dazu in der Lage sein muss, "`irgendwie eine nahe Verbindung"' \citep[\S 63]{Bech:55} zum Unterfeld einzugehen.

Vom syntaktischen Standpunkt leichter beantworten lässt sich die Frage, wie weit das Oberfeld nach links verschoben werden kann: Meurers (in Anlehnung an Marga Reis) nimmt an, dass ein Oberfeld nie direkt hinter dem \isi{Komplementierer} stehen kann \citep[89, Fußnote 21]{Meurers:99}.\footnote{Im Original: "`[\ldots] an upper-field verb can never immediately follow the complementizer [\ldots]"' \citep[89, Fußnote 21]{Meurers:99}} Der Satz \ref{ex-meurers-146b-var}, eine Umbildung von \ref{ex-meurers-146b}, sei aus diesem Grund ungrammatisch:

\ex. *Da\ss\ hätte ihn gestern jemand besiegen können, ist unwahrscheinlich. \label{ex-meurers-146b-var}

Unterfelder sind dagegen nicht (oder jedenfalls nicht immer) von einer solchen Einschränkung betroffen, wie \ref{ex-meurers-89fn21-2} beweist: 

\ex. \label{ex-meurers-89fn21-2} wenn ansteht, diese Dinge zu erledigen\hfill\citep[89, Fußnote 21]{Meurers:99}

%\largerpage[1]% 
Meurers Linksstellungsregel erfasst jedoch auch Oberfelder, die nicht linksversetzt sind und trotzdem direkt an den Komplementierer grenzen. Die Konstruktion mit unpersönlichem Passiv in \ref{ex-meurers-89fn21-1} liefert dafür ein Beispiel, das vielleicht marginal, aber keinesfalls ungrammatisch auf mich wirkt:\footnote{Eine kurze Internetrecherche bringt immerhin zwei historische Belege aus dem 19. Jahrhundert ans Licht:\\
\fnex{
\ex. Ich werde die Geduld des Reichstages nicht sehr lange auf die Probe stellen, in dem ich sehr wohl sehe, da\ss\ ein gro\ss er Theil der Ansicht ist, da\ss\ hätte geschlossen werden müssen. (\textit{Stenographische Berichte über die Verhandlungen des Reichstages des Norddeutschen Bundes im Jahre 1867. Erster Band}. S.~344. \url{http://www.reichstagsprotokolle.de/Blatt3_nb_bsb00000436_00372.html})

\ex. Ebendarum finde ich auch das adolescentem ganz an seiner Stelle und glaube nicht, dass hätte geschrieben werden müssen: Quum ego adolescenti gratulatus essem et tamen eum essem cohortatus. (Iohannes Caspar Orellius (ed.). 1829. \textit{M.~Tulii Ciceronis Opera}. Band 3, S. 21. \url{https://books.google.de/books?id=5QY9AAAAcAAJ})

}}

\ex. \label{ex-meurers-89fn21-1} ??dass hätte getanzt werden sollen \hfill \citep[89, Fußnote 21]{Meurers:99}

Auch Meurers sieht hier kein klares *-Urteil gerechtfertigt. Ich halte es daher für überlegenswert, Meurers Linksstellungsregel folgenderma\ss en zu modifizieren: Das  linksversetzte Oberfeld kann nicht direkt hinter dem Komplementierer stehen. Dadurch ist das Datum in \ref{ex-meurers-89fn21-1} nicht tangiert, denn hier ist nicht zwingend ein linksversetztes Oberfeld involviert. 
\is{Linksstellung!des Oberfelds|)}


\subsection{Linksstellung des maximal untergeordneten Verbs}
\is{Linksstellung!des maximal untergeordneten Verbs|(}

Nicht nur das Oberfeld kann in Form der obigen Linksstellung vom Unterfeld getrennt werden, auch das maximal untergeordnete Verb kann vom Unterfeld\is{Verbalkomplex!Unterfeld} getrennt ins \isi{Mittelfeld} versetzt sein. Man kann also auch von einer Linksstellung (von Teilen) des Unterfelds sprechen. Zwei Beispiele dafür gibt es  in \ref{ex-askedal-9}:\footnote{Für \citet[238]{Meurers:99} sind diese Sätze akzeptabel.}\largerpage[-1]% 

\ex. \label{ex-askedal-9}
\a. ??Den alten Wagen$^3$ hat er zu fahren$^3$ noch nicht gelernt.
\b. ??Nach dem Vornamen des Fremden$^3$ hat er sie zu fragen$^3$ niemals gewagt.
%\c. Zahlreiche Bedeutungen ist er auszudrücken nicht imstande.
\z. \citep[(9)]{Askedal:83}

Für solche linksversetzte Elemente des Unterfelds scheint die Beschränkung zu bestehen, dass sie den 2.~Status tragen müssen, siehe \ref{ex-links-kohaerent}: 

\ex. \label{ex-links-kohaerent}
\a. *Den alten Wagen hat er reparieren noch nicht müssen. 
\b. *Den alten Wagen sollte er repariert noch nicht haben. 

Es ist deshalb nicht unplausibel, wie \citet[238]{Meurers:99} anzunehmen, dass hier eine \isi{inkohärente Konstruktion} zugrundeliegt, aus der heraus topikalisiert wurde.\footnote{Daran anschlie\ss end könnte gefragt werden, wie sich Verben verhalten, die den 2.~Status regieren und obligatorisch kohärent konstruieren\is{Kohärenz!obligatorische/fakultative}. Wenn es stimmt, dass diese Form der Linksstellung aus inkohärenten Konstruktionen abgeleitet ist, müsste eigentlich folgen, dass ein obligatorisch kohärentes Verb wie {\it brauchen} keine Linksstellung erlaubt. Wie das Beispiel in \ref{ex-links-brauchen} jedoch zeigt, ist dieser Zusammenhang nicht immer eindeutig konstruierbar:\\ 
\fnex{
\ex. \label{ex-links-brauchen}
\a. ??Er hat den alten Wagen zu reparieren nicht mehr gebraucht. \label{ex-links-brauchen-a}
\b. ?Den alten Wagen hat er zu reparieren nicht mehr gebraucht. \label{ex-links-brauchen-b}

}
Ich finde die Linksstellung in \ref{ex-links-brauchen-b} deutlich akzeptabler als die intraponierte inkohärente Konstruktion in \ref{ex-links-brauchen-a}.} Man muss jedoch im Auge behalten, dass die Topikalisierung, d.\,h.\ die Vorfeldbesetzung, keine notwendige Bedingung darstellt. Wie die Beispiele in \ref{ex-links-verbletzt} zeigen, ist die Linksversetzung auch ohne Einbezug des Vorfelds möglich:\footnote{Vgl.\ auch das folgende Datum aus G.\,\cite{GMueller:98} mit linksversetztem Partizipium:\\ 
\fnex{
\ex. weil gelesen es keiner hat \hfill \citep[G.][159]{GMueller:98} 

}}     

\ex. \label{ex-links-verbletzt}
\a. ??dass ihn der Mechaniker zu reparieren nicht gewagt hat \label{ex-links-verbletzt-a}
\b. dass zu reparieren der Mechaniker ihn nicht gewagt hat \label{ex-links-verbletzt-b}

Hier befindet sich das disjunkte Verbalfeld $F^3 = (${\it ihn, zu reparieren}$)$ vollständig im Mittelfeld, wobei beide Linearisierungsmöglichkeiten grundsätzlich zur Verfügung stehen, d.\,h.\ {\it ihn} kann vor oder hinter dem regierenden Verb {\it zu reparieren}  stehen.
\is{Linksstellung!des maximal untergeordneten Verbs|)}


\section{Zusammenfassung}

Die Darstellung des Phänomenbereichs der kohärenten Strukturen konnte sich weitgehend an Vorarbeiten anlehnen, zuallererst an \cite{Bech:55}. Anhand dieser Darstellung lässt sich die kohärenzverschuldete \isi{Diskontinuität} realisierter Valenzrahmen in mehrere Schweregrade einteilen: Da wären zunächst die Diskontinuitäten im Rahmen prototypischer Kohärenzfelder\is{Kohärenzfeld} bestehend aus Restfeld und adjazentem Schlussfeld bzw.\ Verbalkomplex; dann die Diskontinuitäten im \isi{Verbalkomplex}, namentlich die Zwischenstellung; und schlie\ss lich die Diskontinuität ganzer Kohärenzfelder, bekannt als 3.~Konstruktion, partielle Voranstellung und Linksstellung.  

Diese empirischen Herausforderungen können durch bestimmte syntaxtheoretische Manöver gemeistert und damit die Idealisierung der Kontinuität gelockert werden. Abschnitt \ref{sec-ttmctag-modellierungsstrategien} wird darüber einen Überblick verschaffen. Das in Kapitel \ref{sec-ttmctag} vorgestellte TT-MCTAG-Modell verzichtet dagegen weitgehend auf die Idealisierung der Kontinuität\is{Idealisierung!der Kontinuität}. Dass die Rede von Schweregraden aber auch dort gerechtfertigt ist, zeigt sich darin, dass bei der 3.~Konstruktion und der partiellen Voranstellung besondere Ma\ss nahmen getroffen werden müssen, um TT-MCTAG mit der nötigen Ausdrucksstärke zu versehen (siehe Abschnitt~\ref{sec-ttmctag-grenzen} und \ref{sec-ttmctag-spinal}). 

Orthogonal dazu verhalten sich Rektionsabweichungen wie der Ersatzinfinitiv, die \emph{zu}-Ver\-schie\-bung oder das Fernpassiv. Auch dieser Aspekt wird in Abschnitt~\ref{sec-ttmctag-rekt} bei der Darstellung des TT-MCTAG-Modells berührt, obgleich generell der Diskontinuitätsaspekt im Vordergrund stehen soll.

Die Erfahrungen bei der Entwicklung des TT-MCTAG-Modells flie\ss en schlie\ss lich in das STUG-Modell  ein, das in Kapitel \ref{ch-ohne-valenz} eingeführt wird und sowohl die Idealisierung der Kontinuität als auch die Idealisierung der Vollständigkeit vermeidet. Zunächst werden wir uns aber im nächsten Kapitel einem weiteren Phänomenbereich zuwenden, nämlich dem Phänomenbereich der Ellipse, der im Widerspruch zur Idealisierung der Vollständigkeit steht.





 