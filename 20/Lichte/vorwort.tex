%!TEX root = main.tex

\addchap{Danksagung}

Dieses Buch ist eine leicht überarbeitete Fassung meiner Doktorarbeit, die ich im November 2013 an der Philosophischen Fakultät der Eberhard-Karls-Universität Tübingen verteidigen durfte. Besonderer Dank gilt daher zuerst meinen Betreuern Laura Kallmeyer und Detmar Meurers, die mich fürsorglich und vertrauensvoll durch den Promotionsprozess gelotst haben. 

Die Doktorarbeit entstand größtenteils in Laura Kallmeyers Emmy"=Noe\-ther-Nachwuchsgruppe in Tübingen, finanziert durch die Deutsche Forschungsgemeinschaft (DFG). Laura hat es verstanden, dort eine familiäre, anregende und experimentierfreudige Atmosphäre zu schaffen, wozu auch Johannes Dellert, Kilian Evang, Miriam Käshammer, Wolfgang Maier und Yannick Parmentier erheblich beigetragen haben. Ihnen allen möchte ich dafür sehr danken. Insbesondere mit Wolfgang über längere Zeit ein Büro zu teilen, war einfach ein ungeheurer Glücksfall -- fachlich, menschlich und musikalisch. Bedanken möchte ich mich auch bei Piklu Gupta, Wolfgang Maier, Daniela Marzo, Yannick Parmentier, Georg Rehm, Oliver Schonefeld, Holger Wunsch und Thomas Zastrow für ihre regelmäßige Teilnahme am Mittwochsstammtisch. Mein Dank gilt außerdem Beate Starke, die, vom Sekretariat der Sonderforschungsbereiche 441 und 833 aus, organisatorische Steine jeglicher Größe mit einem Lächeln aus dem Weg geräumt hat.

Abgeschlossen habe ich die Doktorarbeit am Institut für Sprache und Information und am Sonderforschungsbereich 991 der Universität Düsseldorf. Ich möchte dem Kollegenkreis dort für die freundliche Aufnahme danken, die dafür gesorgt hat, dass Düsseldorf schnell zu einer neuen Heimat geworden ist.    

Außerdem bin ich Andreas Konietzko und Stefan Müller dankbar für ausführliche fachliche Hilfestellungen, Anders S\o{}gaard für die entscheidende terminologische Schützenhilfe, und Lucas Champollion und Albert Ortmann für allerlei Hinweise. Im Zuge der Veröffentlichung bei Language Science Press hatte ich das Glück, dass sich Kim Gerdes, Stefan Müller und Owen Rambow den Text nochmals kritisch vorgenommen haben. Ihre Anmerkungen habe ich zu berücksichtigen versucht, aber selbstverständlich liegt die Verantwortung für die alten und neuen Unzulänglichkeiten ausschließlich bei mir.

%\largerpage

Es ist bei einer Doktorarbeit naturgemäß leicht zu sagen, wann alles ein Ende gefunden hat; die Anfänge sind dagegen vielfältig und verlieren sich irgendwo im Nebel der Erinnerungen. Was die Zeit vor der Emmy-Noether-Nachwuchsgruppe betrifft, möchte ich aber Frank Richter und Manfred Sailer hervorheben. Nicht nur wegen ihrer Begeisterung für formale Grammatiken und Grammatiktheorie, die mich angesteckt hat, sondern auch weil sie im entscheidenden Moment an mich gedacht haben. Ohne ihre Vermittlung wäre dieses Buch wohl nicht entstanden. 

Ganz sicher hätte ich dieses Buch nicht geschrieben, wenn Verena mir nicht über all die Jahre ihre Liebe, Zuversicht und dann noch Henning, Marlene und Clemens geschenkt hätte. Ohne Dich wäre alles nichts.
