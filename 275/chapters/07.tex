\documentclass[output=paper]{langsci/langscibook}
\ChapterDOI{10.5281/zenodo.3972838}

\author{Lisa Travis\affiliation{McGill University}}
\title{Little words -- big consequences}

% \chapterDOI{} %will be filled in at production

\abstract{This paper investigates the interaction of \isi{E-language} and I-language
    within the context of the macro- vs.\ micro-parameter debate.  It presents
    a case study of variation found in the focus construction\is{focus} in Western
    Malayo-Polynesian languages, \ili{Tagalog}, and three dialects of \ili{Malagasy} ---
    Merina, Bezanozano, and Betsimisaraka. The grammatical role of the
    functional element that appears directly after the focused element, which
    is only subtly indicated in the E-language, turns out to be crucial as its
    role can have significant repercussions in the I-language.  More
    specifically, depending on whether this element is a determiner, a
    relativizer, or a complementizer\is{complementizers}, the construction itself can vary between
    a pseudo-cleft construction and a cleft construction.  The hypothesis is
    made that the shift from the pseudo-cleft to the cleft construction opens
the door to a possible \isi{reanalysis} of these verb-initial languages as having SVO
word order.}

\maketitle

\begin{document}\glsresetall

\section{Introduction}

\begin{quotation}
    \dots{} study of the principles of syntax is not and cannot be a separate
    enterprise from study of the \isi{parameters}. \parencite[9]{Kayne:2005k}
\end{quotation}

\noindent It is hard to separate the study of syntax from the study of
parameters.  In the 80s and 90s, interest was in macro-parameters\is{parameters} such as
bounding \parencite{Rizzi:1982}, pro-drop (e.g.\ \citealt{Chomsky:1981}), and
word order.  More recently, interest has turned to \textcite{Kayne:2005k}. In a
system that recognizes I(nternal)-language and E(xternal) language, we find a
tension is created between macro-parameters\is{parameters} and micro-parameters\is{parameters}.
Macro-parameters are best suited to explain the speed of language acquisition\is{language acquisition}.
Acquiring one smaller language detail will entail that many other language
facts will follow because one parameter will account for a cluster of
language-specific phenomena.  If one were to design the perfect
\isi{I-language} system, a system of macro-parameters\is{parameters} would
appear to be the most efficient way to go. However, we know that language
changes gradually given that the \isi{E-language} between two generations on
the chain of \isi{language change} will have to be mutually intelligible.  So
as far as \isi{E-language} goes, a system of micro-pa\-ram\-e\-ters would appear to
be the right way to go.

In this paper I argue that a small surface difference in the \isi{E-language}
might well indicate a large difference in the \isi{I-language}.  This would allow
shifts in a macro-parameter that could well not interfere with mutual
intelligibility.  The particular change that I will be investigating is a
hypothesized change from VOS to SVO in Austronesian.  I will look at a focus
construction\is{focus} in three dialects of Malagasy\footnote{Malagasy is the
    name of a variety of dialects spoken in Madagascar by about 18 million
people.} -- Merina, Bezanozano, and Betsimisaraka -- and compare this to its
Austronesian cousin, \ili{Tagalog}.  The claim will be that while \ili{Tagalog}
and Bezanozano, the most conservative \ili{Malagasy} dialect of the three, can
be argued to use pseudo-clefting\is{pseudo-clefts|seealso{clefts}} for their
focus\is{focus} construction, both Merina and Betsimisaraka appear to have
moved to a cleft\is{clefts} construction, which I argue makes them closer to
becoming SVO languages.  The important part of this proposal is that this shift
all rests on the analysis of one functional category -- a very small surface
difference that points to a substantial underlying difference.

\section{Clefts and pseudo-clefts}

In this section I give some background data on the relevant construction and I
introduce the issue of distinguishing between \isi{pseudo-clefts} and \isi{clefts} in
predicate-initial\is{verb-initial constituent order} languages that lack copulas\is{copulas} and expletives.  I will argue
that it is lack of transparency in these constructions that leads to \isi{reanalysis}
and \isi{language change}.   All of the languages/dialects under investigation are
predicate-initial,\is{verb-initial constituent order} but all have a focus construction\is{focus} in which a designated DP,
which some analyses label the subject, appears sentence-initially.  My argument
will be that it is this construction that can eventually undergo \isi{reanalysis} as
a pure SVO structure.  Whether or not it is susceptible to \isi{reanalysis} will
depend on how salient the signs are in this construction that the language
remains predicate-initial\is{verb-initial constituent order}.  If the construction is clearly marked as a
pseudo-cleft, its predicate-initial\is{verb-initial constituent order} status will be clear.  If the construction
is a cleft construction, it will be subject to \isi{reanalysis}.  Why this is so will
be explained in this section.

\subsection{Background data}

\ili{Tagalog}, the most well-documented language spoken in the Philippines, is
clearly verb-initial with variable word order following the verb.  As I will be
comparing \ili{Tagalog} to the \ili{Malagasy} dialects, I give a brief overview of
its focus construction\is{focus} here.  In the \ili{Tagalog} clause, there is a
designated argument that I will call the Pivot, that is marked by the particle
\emph{ang}.\footnote{There are debates about the syntactic status of the
    \emph{ang} DP, whether it is the subject, the \isi{topic}, or the
    absolutive\is{absolutive case}
    marked argument.  In a parallel fashion, there is a debate about what the
    particle \emph{ang} is -- nominative\is{nominative case} case, default case, or absolutive
    case.\is{absolutive case}  What is important for the purpose of this paper is that it is a
    functional category that is part of the nominal extended projection.} In
    \eqref{ex:Tbasic} below, we see that the sentence begins with the verb
    \emph{bumili} `buy' and that the Agent, acting as the Pivot, appears with
    the particle \emph{ang}.

\ea\label{ex:Tbasic}\ili{Tagalog}\\
    \gll Bumili ng bigas \textbf{ang} \textbf{babae}\\
    \At{}.buy \Acc{} rice \Nom{} woman\\
    \glt `The woman bought rice.'
\z

In order to create the focus construction\is{focus}, the \emph{ang} DP is
fronted and that fronted DP is followed by another particle
\emph{ang}.\footnote{I will be using boxes to highlight the \enquote{little
words} referred to in the title of this chapter at relevant points.}

\ea \ili{Tagalog}\\
    \gll  \textbf{Ang} \textbf{babae} \custombox{ang} bumili ng bigas\\
    \Nom{} woman \Nom{} \At.\textsc{buy} \Acc{} rice\\
    \glt `It is the woman who bought rice.'
\z

The Merina dialect of Malagasy\footnote{Merina is the main dialect, very close
to what is called Official \ili{Malagasy}, and is spoken in the capital region.} also
has a Pivot DP, in this case indicated by its sentence-final position.

\ea Merina\\
	\gll Manasa ny lambanay \textbf{Rakoto}\\
    \Prs.\At{}.wash \Det{} clothes.\Fpl.\Excl{} Rakoto\\
    \glt `Rakoto is washing our clothes.'
\z

In a focus construction\is{focus}, this Pivot DP appears sentence-initially and is
followed by the particle \emph{no}.

\ea Merina\\
	\gll  \textbf{Rakoto} \custombox{no} manasa ny lambanay\\
    Rakoto \emph{no} \Prs.\At{}.wash \Det{} clothes.\Fpl.\Excl{}\\
    \glt `It is Rakoto who is washing our clothes.'
\z

The focus of this paper will be this construction and more specifically the
role of the particle that follows the focussed\is{focus} DP.  I will argue that this
particle can be a nominal functional category (as we will see for \ili{Tagalog}) or a
verbal functional category (as we will see for the Merina dialect of \ili{Malagasy})
and that the former indicates a pseudo-cleft construction while the latter
indicates a cleft construction.  We will see that in the pseudo-cleft
construction, the clause remains firmly predicate-initial\is{verb-initial constituent order}, while in the cleft
construction, the word order within the clause is less obvious and therefore
susceptible to \isi{reanalysis}.

\subsection{Discovering (pseudo)-clefts}

The first goal of the paper is to show that these constructions are \isi{clefts} of
some form. In order to do this, I follow arguments taken from the literature on
\ili{Malagasy} (e.g.\
\citealt{Keenan:1976,Paul:2001,Pearson:2009a,Potsdam:2006,Law:2007}).  The
first task is to show that the sentence-initial DP is preceded by a (silent)
verb.  Using examples from Merina, we can see below that both negation
\eqref{ex:neg} and the \isi{raising} predicate \emph{toa} `seems'  \eqref{ex:raising} can
precede the DP.  Since both negation and \isi{raising} predicates select verbal
projections and not DPs, the conclusion has been made that there is a covert
copula preceding the focussed\is{focus} DPs.

\ea Merina\\
	\gll  \textbf{Tsy} Rakoto no manasa ny lambanay  \label{ex:neg}\\
    \Neg{} Rakoto \emph{no} \Prs.\At{}.wash \Det{} clothes.\Fpl.\Excl{}\\
    \glt `It isn't Rakoto who is washing our clothes.'
\ex Merina\\
	\gll  \textbf{Toa} Rakoto no manasa ny lambanay\label{ex:raising}\\
    Seems Rakoto \emph{no} \Prs.\At{}.wash \Det{} clothes.\Fpl.\Excl{}\\
    \glt `It seems to be Rakoto who is washing our clothes.'
\z

While remaining silent on what the structure is that follows the DP as this is
the topic of the paper, we know that the first part of the construction
contains an unrealized verb.

\ea
    {}[ Neg/RaisingV  [$_V$ ⟨\Cop{⟩} ] DP \dots{} ]
\z

Now we take a brief excursion to discuss the distinction between \isi{clefts} and
pseudo-clefts in predicate-initial\is{verb-initial constituent order} languages, why the distinction is very
subtle, and why this distinction is important to the issue at hand.   We start
with an \ili{English} cleft where an object, \eqref{ex:clefta}, or a subject,
\eqref{ex:cleftb}, has been extracted.  Eventually we will look only at subject
extraction so I have put that example in bold.

\ea Cleft
    \ea It is a small dog that the child saw. \label{ex:clefta}
    \ex \textbf{It is a small dog that saw the child.\label{ex:cleftb}}
    \z
\z

Now we look at pseudo-cleft \eqref{ex:pclefta}.  In order to create a structure
that works well with subject extraction which is crucial in our discussion of
the change in word order from VOS to SVO, I change the construction slightly in
\eqref{ex:pcleftb} by substituting \emph{what} with \emph{the thing}.  I am
assuming that this change does not make any relevant difference in the
structure itself.  Finally we see this structure with subject extraction in
\eqref{ex:pcleftc} as this is what we will be comparing with the \ili{Malagasy}
structure.

\ea  Pseudo-cleft
    \ea \textbf{What} the child saw is a small dog.\label{ex:pclefta}
    \ex \textbf{The thing} that the child saw is a small dog. \label{ex:pcleftb}
    \ex \textbf{The thing that saw the child is a small dog.\label{ex:pcleftc}}
    \z
\z

In this exercise we will compare only the subject \isi{clefts} \eqref{ex:cleftb} and
pseudo-clefts \eqref{ex:pcleftc} since these are the two constructions resembling
most closely the \ili{Tagalog}/\ili{Malagasy} structures that we will encounter.  In these
languages, extraction is for the most part restricted to the Pivot DP.  In
order to simplify the discussion, we will start by focusing our attention only
on sentences where the Agent is the Pivot.

\textit{Step 1}:  Our first task in understanding what our expectations are for
clefts and \isi{pseudo-clefts} in \ili{Malagasy} and \ili{Tagalog}, both predicate-initial\is{verb-initial constituent order}
languages, is to determine what we expect the order of elements to be.  In
order to do that, we first separate predicate from subject in clefts
\eqref{ex:csp} and \isi{pseudo-clefts} \eqref{ex:pcsp} and then front the
predicates in the \ili{English} examples \eqref{ex:cps} and
\eqref{ex:pcps}.\footnote{In examples
(\ref{travis:10}--\ref{travis:14}), subjects are in bold-face, predicates are
in italics. In examples (\ref{travis:10}--\ref{travis:15}), unpronounced
material is set in angled brackets.}\largerpage[1.5]

\ea\label{travis:10} Cleft
    \ea \textbf{It}  \emph{is a small dog that saw the child}\label{ex:csp} \hfill \textbf{Subj} \emph{Pred}
     \ex \emph{is a small dog that saw the child } \textbf{it}\label{ex:cps} \hfill \emph{Pred} \textbf{Subj}
    \z
\ex  Pseudo-cleft
    \ea  \textbf{The thing that saw the child }  \emph{is a small dog}\label{ex:pcsp} \hfill \textbf{Subj} \emph{Pred}
     \ex \emph{Is a small dog} \textbf{the thing that saw the child} \label{ex:pcps} \hfill \emph{Pred} \textbf{Subj}
    \z
\z

\textit{Step 2}:  Because we know that these languages do not have overt
copulas, we can take these out of our expected structures.

\ea
    \ea \emph{⟨is⟩ a small dog that saw the child } \textbf{it} \hfill \textsc{cleft}
    \ex \emph{⟨is⟩ a small dog} \textbf{the thing that saw the child} \hfill \textsc{pseudo-cleft}
    \z
\z

\textit{Step 3}:  Because we know that these languages do not have expletive\is{expletives}
subjects, we can take these out of the relevant expected structures (i.e. the
cleft).

\ea
    \ea \emph{⟨is⟩ a small dog that saw the child } ⟨it⟩ \hfill \textsc{cleft}
    \ex \emph{⟨is⟩ a small dog} \textbf{the thing that saw the child} \hfill \textsc{pseudo-cleft}
    \z
\z

\textit{Step 4}:  Because we know that these languages have headless relatives,
we can the head of the relative out of the relevant structure (i.e. the
pseudo-cleft).

\ea\label{travis:14}
    \ea \emph{⟨is⟩ a small dog that saw the child } ⟨it⟩ \hfill \textsc{cleft}
    \ex \emph{⟨is⟩ a small dog} \textbf{the} ⟨\textbf{thing}⟩ \textbf{that saw the child}  \hfill \textsc{pseudo-cleft}
    \z
\z

When we put the remaining pieces of the cleft and the pseudo-cleft side by
side, we can now see (a) how minimally different these are on the surface yet
(b) how dissimilar they are in the underlying structure.  Both begin with a DP
followed by some functional material and it is within this functional material
that we get the only clues as to whether we are dealing with a cleft (C) or a
pseudo-cleft (PC) construction.  The only distinguishing elements are, in
English, the complementizer\is{complementizers} \emph{that} for the cleft  and the determiner
\emph{the} and the relativizer \emph{that} for the pseudo-cleft.  Yet
structurally these two constructions are very different with the cleft
construction having the predicate \emph{e the small dog that saw the child} and
no pronounced subject while the pseudo-cleft has the predicate \emph{e the
small dog} and the subject \emph{the e that saw the child}.

\ea\label{travis:15}
    \gll C: [ ⟨is⟩ \textbf{a small dog} ~ ~ \custombox{that} ~ ~ \textbf{saw} \textbf{the} \textbf{child} ][ ⟨it⟩ ] \label{ex:littewords}\\
    PC:  [ ⟨is⟩ \textbf{a small dog} ] [ \custombox{the} ⟨thing⟩ \custombox{that} \textbf{saw} \textbf{the} \textbf{child} ] \\
\z

Now the question is why this is so important.  I will argue that this
distinction is crucial in the shift from a VOS language to an SVO language.
Notice that only in the pseudo-cleft do we get information on where the subject
is, and this information confirms that the language is predicate-initial
(subject-final).  In the cleft structure, since the (expletive) subject is not
pronounced, we have no indication as to whether the structure is SVO or VOS.
Note also if, for some reason, the functional category is not realized, we are
left with the remaining elements \emph{the small dog saw the child}, in other
words a simple SVO sentence. The lack of information of the cleft construction
and the fragility of these functional categories will become important later in
the paper when I speculate on how languages move from a VOS word order to an
SVO word order.

 Having derived some word order expectations from this exercise,  we return to
 the issue of the languages/dialects under study.  Since the functional words
 that follow the sentence-initial DP are crucial in determining whether the
 focus constructions\is{focus} are \isi{clefts} or \isi{pseudo-clefts}, they will become the target
 of the investigation. To not prejudge the questions, I will for now just call
 these functional words particles.  The question will be whether these
 particles are part of the nominal extended projection or the verbal extended
 projection.  I will end up classifying them into three types deriving from the
 three functional elements we find in \eqref{ex:littewords} -- the nominal
 particles (such as \emph{the}), the relativizing particles (such as
 \emph{that}), and the complementizer\is{complementizers} particles (such as \emph{that}).  To make
 it even clearer how difficult this is, we can think of \ili{English} and the
 demonstrative\is{demonstratives} \emph{that}, the relativizer \emph{that}, and the complementizer\is{complementizers}
 \emph{that}.  Very slight differences in pronunciation (where the relativizer
 and the complementizer\is{complementizers} \emph{that} but not the demonstrative\is{demonstratives} \emph{that} may
 have a reduced vowel) and position can indicate quite different structures.

\section{Tagalog and the Malagasy dialects}

In this section I will be comparing the different particles that we find in the
focus constructions in \ili{Tagalog} and three \ili{Malagasy} dialects --
Merina (Official \ili{Malagasy}), Bezanozano, and Betsimisaraka.  By seeing how they
behave in other parts of the grammar, I hope to  determine whether they are
part of the nominal extended projection, a relativizer, or a complementizer\is{complementizers}
(a part of the verbal functional projection).

\subsection{Tagalog}

Tagalog immediately makes it fairly clear which particle we find following the
focussed DP.  We do not have to look very far to see that the particle
\emph{ang} is used as a nominal marker.\footnote{For more details on \ili{Tagalog}
see Aldridge \citeyearpar{Aldridge:2013}, Kroeger \citeyearpar{Kroeger:1993},
Richards \citeyearpar{Richards:1998a}, and for Polynesian languages, Potsdam
and Polinsky \citeyearpar{Potsdam:2011a}.} Below I have repeated our basic
Tagalog sentence from above, as well as the focus construction\is{focus}.  In the basic
clause  \eqref{ex:Tbasic2} we see \emph{ang} appearing as a nominal marker on the
Pivot DP.  In  \eqref{ex:cleftT}, \emph{ang} appears twice, once before the now
focussed and fronted Pivot DP, and once following this DP acting as the
focussing particle.

\ea \ili{Tagalog}
    \ea
    \gll Bumili ng bigas \textbf{ang} \textbf{babae} \label{ex:Tbasic2}\\
    \At{}.buy \Acc{} rice \Nom{} woman\\
    \glt `The woman bought rice.'\\
    \ex
    \gll  \textbf{Ang} \textbf{babae} \custombox{ang} bumili ng bigas\label{ex:cleftT}\\
    \Nom{} woman \Nom{} \At.\textsc{buy} \Acc{} rice\\
    \glt `It is the woman who bought rice.'
    \z
\z

There have been a variety of analyses of \emph{ang} which co-vary with the
analysis of the syntactic structure of \ili{Tagalog} clauses.  However, whether it is
nominative case marker, an \isi{absolutive case} marker, a Topic\is{topic} marker, or a
determiner, it is a functional head\is{functional items} along the extended projection of the noun.
As for its other uses in the grammar, we can see below that when it precedes a
predicate that is missing its Pivot DP, it creates a DP which refers to the
missing argument.  In \eqref{ex:angNOMat} below we see the predicate \emph{bumili
ng bigas} `buy rice' preceded by \emph{ang} and it means something like `the
one who bought rice' or `the rice-buyer'.


\ea \ili{Tagalog}\\
    \gll Pagod \textbf{ang} \emph{bumili} \emph{ng} \emph{bigas}\label{ex:angNOMat}\\
    tired \Nom{} \At{}.buy \Acc{} rice\\
    \glt `The one who bought rice is tired.'
\z

The verb can appear in a different form (the Theme Topic\is{topic} form) changing the
Pivot from the Agent to the Theme as in \eqref{ex:angNOMtt}.  When this form of
the predicate is preceded by \emph{ang}, it now means something like `the thing
that was bought by the woman' or `the woman's bought thing'.

\ea \ili{Tagalog}
    \ea
    \gll \emph{binili} \emph{ng} \emph{babae} ang bigas\label{ex:angNOMtt}\\
    \Tto{}.buy \Gen{} woman \Nom{} rice\\
    \glt `The rice was bought by the woman.'\\
    \ex
    \gll mahal \textbf{ang} \emph{binili} \emph{ng} \emph{babae}\\
    expensive \Nom{} \Tto{}.buy \Gen{} woman\\
    \glt `The thing bought by the woman is expensive.'
    \z
\z

While one of the translations given above is a headless relative, we know that
\emph{ang} is not the relativizer itself.   When we do have a relative clause\is{relative clauses},
the \emph{ang} appears before the head of the relative, and the relativizer has
a different form, either \emph{ng} or \emph{na}.  This form, sometimes called a
\textsc{linker}, is also used between a nominal head and an adjective (see
\ref{ex:ALNK1} and \ref{ex:ALNK2}).

\ea \ili{Tagalog} \label{ex:relcl}
    \ea
    \gll  Pagod ang babae\textbf{ng} bumili ng bigas\\
    tired \Nom{} woman-\Rel{} \At{}.buy \Acc{} rice\\
    \glt `The woman who bought rice is tired.' \\
    \ex
    \gll mahal \textbf{ang} bigas \textbf{na} \emph{binili} \emph{ng} \emph{babae}\\
    expensive \Nom{} bigas \Rel{} \Tto{}.buy \Gen{} woman\\
    \glt `The rice bought by the woman is expensive.' \\
    \ex
    \gll  mahirap ang babae\textbf{ng} pagod.\label{ex:ALNK1}\\
    poor \Nom{} woman-\Lnk{} tired\\
    \glt `The tired woman is poor.' \\
    \ex
    \gll malasa ang bigas \textbf{na} mahal\label{ex:ALNK2}\\
    tasty \Nom{} rice \Lnk{} expensive\\
    \glt `The expensive rice is tasty.'
    \z
\z

A plausible analysis for the focus construction\is{focus}, then, is one where the
material following the focus particle is some sort of nominal that I will
translate as `the x that ...' -- the translation that I have given to the
pseudo-cleft in \eqref{ex:pcleftc} above.  I repeat our \ili{Tagalog} focus
construction below and give it now a pseudo-cleft translation.

\ea \ili{Tagalog}\\
    \gll  \textbf{Ang} \textbf{babae} \custombox{ang} bumili ng bigas\\
    \Nom{} woman \Nom{} \At.\textsc{buy} \Acc{} rice\\
    \glt `The one who bought rice is the woman.'
\z

The predicate of the clause is an unpronounced copula\is{copulas} followed by the DP
\emph{ang babae} `the woman', and the subject of the clause is \emph{ang bumili
ng bigas} `the one who bought rice'.

A construction that will become important in our determination of the nature of
the focus particle is the focussed\is{focus} PP construction.  \ili{Tagalog} and all of the
three \ili{Malagasy} dialects that we are comparing allow PPs to be fronted and
focussed.  We see the \ili{Tagalog} PP Focus below.  Note that when the focussed\is{focus}
constituent is a PP, the focus particle \emph{ang} is disallowed.

\ea \ili{Tagalog}\\
    \gll Sa palengke (*ang) bumili ng bigas ang babae\\
    \Prep{} market (\Nom{}) \At.buy  \Acc{} rice \Nom{} woman\\
    \glt `It was at the market that the woman bought rice.'
\z

In fact, the inability to have a nominal functional category in this position
makes sense because it is not clear what this nominal phrase would refer to.
There is no missing Pivot in the material following the focussed\is{focus} element.  What
is missing is a PP but this is not nominal.  Looking at \ili{English} \isi{clefts} and
pseudo-clefts, we can see that with clefted PPs, it is sufficient to just have
the complementizer\is{complementizers} \emph{that}.  However, with \isi{pseudo-clefts}, we need to have
the relevant \textsc{wh}-word to give the PP meaning.  Notice that with a
relative clause in \ili{English}, we cannot drop the \textsc{wh}-word the same way
that we can with DP arguments.

\ea
    \ea It was at the market \textbf{that} I bought rice.
    \ex  \emph{Where} I bought rice was at the market.
    \ex  That is rice \emph{which}/\textbf{that} the woman bought.
    \ex  That is the woman \emph{who}/\textbf{that} bought rice.
    \ex That is the market \emph{where}/\textbf{*that} the woman bought rice.
    \z
\z

Likewise in \ili{Tagalog}, a DP relative clause\is{relative clauses} head that would originate within a PP
in the embedded clause needs to be followed by a complementizer\is{complementizers} and a
contentful \textsc{wh}-word (here \emph{kung saan} `if where').  It cannot
simply be followed by the linker as was the case in the relative clause\is{relative clauses}
constructions given in \eqref{ex:relcl}.

\ea \ili{Tagalog}\\
    \gll Malayo ang palengke-*ng / kung saan bumili ng bigas\\
    far \Nom{} market-\Lnk{} {} if where \At.buy  \Acc{} rice\\
    \glt `The market where the woman bought rice is far.'
\z

I would argue, then, that in \ili{Tagalog}, when the Pivot is focussed\is{focus}, we have
a pseudo-cleft construction signaled by the nominal functional category
\emph{ang}. When the PP is focussed\is{focus}, however, we have a cleft construction.
What is important for the purpose of this paper, however, is that there is no
mistaking a focus construction\is{focus} as having an SVO word order.  If a DP
Pivot appears sentence-initially, it is clearly followed by a DP signalled by
the presence of \emph{ang}.

\subsection{Merina (Official Malagasy)}

Now we turn to Merina, the most documented dialect of \ili{Malagasy}.  Since I will
be comparing it to other dialects of \ili{Malagasy}, I will identify it as Merina.
We see below that the focus particle is \emph{no}.  This particle is much more
difficult to categorize.

\ea Merina\\
	\gll  \textbf{Rakoto} \custombox{no} manasa ny lambanay\\
    Rakoto \emph{no} \Prs.\At{}.wash \Det{} clothes.\Fpl.\Excl{}\\
    \glt `It is Rakoto who is washing our clothes.'
\z

Unlike \emph{ang} in \ili{Tagalog}, the particle \emph{no} in Merina is not
used as a nominal functional category.  We can see below that while the
determiner, \emph{ny}, is very similar in form, \emph{no} cannot be used in its
place.

\ea Merina\\
    \gll mangatsika \textbf{ny/*no} tranoko\\
    cold \Det{} house.\Fsg.\Gen{}\\
    \glt `My house is cold.'
\z

Given this, it is not surprising that \emph{no} \emph{can} be used with
focussed PPs.

\ea Merina\\
	\gll Amin'ny penina \textbf{no} manorotra aho\\
    with.\Gen{}.\Det{} pen \emph{no} \Prs.\At{}.write \Fsg{}.\Nom{}\\
    \glt `It's with a pen that I am writing.'
\z

The fact that it can be used with a focussed\is{focus} PP correlates with what we have
seen in \ili{Tagalog}.  I argued that \emph{ang} couldn't appear with a focussed\is{focus} PP
precisely because it was a nominal functional category.  Since we have seen
that Merina \emph{no} is not nominal, we would expect no clash with the PP.

Having seen that \emph{no} is not nominal, we now can see that it is also not a
relativizer.  The relativizer in Merina is \emph{izay}.

\ea Merina\\
	\gll  vizaka ny lehilahy (\textbf{izay})/*no manasa ny lambanay\\
    tired  \Det{} man \hphantom{(}\Rel{} \Prs.\At{}.wash \Det{} clothes.\Fpl.\Excl{}\\
    \glt `The man who is washing our clothes is tired.'
\z

The question arises, however, where else the particle \emph{no} can appear.
Interestingly, it is used to link two clauses together with a variety of
effects (see~\citealt{Pearson:2009a} for details).  Below we have two clauses
that are temporally connected and it is the particle \emph{no} that creates the
link.

\ea Merina\\
	\gll Natory Rakoto \textbf{no} naneno ny telefaonina\label{ex:temporal1} \\
    \Pst.\At{}.sleep Rakoto \emph{no} \Pst.\At{}.ring \Det{} telephone\\
    \glt `Rakoto was sleeping when the phone rang.'
\z

While \emph{no} is not used as a complementizer\is{complementizers} (the most commonly used
complementizer is \emph{fa}), examples such as \eqref{ex:temporal1} above
suggest that it is a particle that is part of the verbal extended projection.
This makes it very different from \emph{ang} in \ili{Tagalog}, suggesting that
the focus construction\is{focus} has a distinct underlying analysis.  More specifically,
I will argue that while  DPs in \ili{Tagalog} are focussed\is{focus} through a
pseudo-cleft construction, they are focussed\is{focus} in Merina through a cleft
construction.

\subsection{Bezanozano}

Now we turn to Bezanozano, a more conservative dialect of \ili{Malagasy} (see
\citealt{Ralalaoherivony:2015} and \citealt{Ranaivoson:2015} for more on
Bezanozano).  Not surprisingly, perhaps, it patterns more like \ili{Tagalog},
which represents a more conservative form of Western Malayo-Polynesian sentence
structure and morphology. Bezanozano has an interesting twist, however, that
indicates a stage somewhere between \ili{Tagalog} and Merina.  We start with a basic
sentence in Bezanozano that is not very different from what we have seen for
Merina.  The main difference is that the determiner, rather than being
\emph{ny}, is \emph{i}.

\ea Bezanozano\\
    \gll Manasa \textbf{i} lambanay Rakoto \\
    \Prs.\At{}.wash \Det{} clothes.\Fpl.\Excl{} Rakoto \\
    \glt `Rakoto is washing our clothes.'
\z

Turning now to the focus construction\is{focus}, we see that instead of the particle
\emph{no}, we find \emph{i}.

\ea Bezanozano\\
	\gll Rakoto \textbf{\custombox{i}} manasa \textbf{i} lambanay\\
    Rakoto \Det{} \Prs.\At{}.wash \Det{} clothes.\Fpl.\Excl{}\\
    \glt `It is Rakoto who is washing our clothes.'
\z

The similarity with \ili{Tagalog} now is clear.  The focussing particle is the
same as the nominal functional category, most likely a determiner.  What
confirms this identity is the fact that the determiner \emph{i} and the
particle \emph{i} show the same allomorphic variation, sometimes appearing as
\emph{ni} and sometimes as \emph{i}.  Given the fact that both Bezanozano and
\ili{Tagalog} use nominal functional categories in the focus constructions\is{focus}, we
would expect distribution of these particles  to work the same way in both
languages.  This is where the twist comes.  In \ili{Tagalog}, we saw that
focussed PPs could not be followed by the nominal \emph{ang}.  We can see
below, however, that focussed\is{focus} PPs in Bezanozano can optionally be followed by
the nominal \emph{i}.

\ea Bezanozano\\
	\gll Amin'i penin-janako (\textbf{i}) manorotra aho\\
    with.\Gen{}.\Det{} pen-child.\Fsg.\Gen{} \Det{} \Prs.\At{}.write \Fsg{}.\Nom{}\\
    \glt `It's with my child's pen that I am writing.'
\z

Just as we were not surprised at the fact that in \ili{Tagalog} \emph{ang} could not
follow PPs, we should be surprised that \emph{i} can follow PPs in Bezanozano.
One small consolation is that the \emph{i} which follows the PP is not
identical with the \emph{i} that follows DPs in that the former is optional
while the latter is not.  Preliminary work on this dialect has not provided any
more information on the distribution of this optional \emph{i}, but given its
distribution, I tentatively propose that obligatory \emph{i} is a nominal
functional head and optional \emph{i} is a verbal functional head\is{functional items} (though I
have not yet found it in any other construction).

Important for the line of argumentation in this paper is that Bezanozano lies
somewhere between \ili{Tagalog} and Merina.  Focussed DP constructions are
pseudo-cleft constructions where the particle is actually a determiner
signalling that the construction is still subject-final.  But with the
appearance of a homophonous particle that is not nominal in nature following
the PP, there is a possibility of reanalyzing this particle as necessarily not
being nominal (since it can follow a PP) allowing for a \isi{reanalysis} of the DP-initial structures as \isi{clefts} rather than \isi{pseudo-clefts}. This would lead to a
status such as that of Merina.

\subsection{Betsimisaraka}

While Bezanozano is more conservative than Merina, I will argue that
Betsimisaraka is more innovative.  My work on this dialect is quite
preliminary, but I have elicited the following constructions.  Starting again
with the basic sentence, we can see that it is quite similar to the other two
dialects.\newpage

\ea Betsimisaraka\\
	\gll  manasa lamba Rakoto\\
     \Prs.\At{}.wash clothes Rakoto\\
    \glt `Rakoto is washing clothes.'
\z

Some differences start appearing, however, in the focus construction\is{focus},
precisely in the choice of the material that follows the focussed\is{focus} constituent.
Below we first have a Merina example for comparison.  This Merina construction
shows that the same focus construction\is{focus} is used to form
\textsc{wh-}questions.  This example is followed by two examples from
Betsimisaraka, one where a DP \textsc{wh}-word is in the focus position and one
where a PP \textsc{wh}-word is in the focus position.

\ea
    \ea Merina\\
    \gll Iza \textbf{no} manasa lamba\\
    who \emph{no} \Prs.\At{}.wash clothes\\
    \glt `Who is washing clothes?'\\
    \ex Betsimisaraka\\
    \gll Zovy (\textbf{my/sy}) manasa lamba\\
    who \emph{my/sy} \Prs.\At{}.wash clothes\\
    \glt `Who is washing clothes?'\\
    \ex Betsimisaraka\\
    \gll Akeza (\textbf{my/sy}) manasa lamba Rakoto\\
    where \emph{my/sy} \Prs.\At{}.wash clothes Rakoto\\
    \glt `Where is Rakoto washing clothes?'\\
    \z
\z

This preliminary work on Betsimisaraka shows that either nothing or one of two
different elements can be found in the position following the sentence-initial
constituent.  The two elements that may appear are very dissimilar from the
particles we find in Merina and Bezanozano.  Further, they don't have a nominal
function along the lines of the particle in Bezanozano, nor a clausal function
along the lines of the particle in Merina.  It turns out that they are \isi{adverbs}
that carry the parts of the meaning of a (pseudo)-cleft construction -- where
\isi{pseudo-clefts} have a meaning of focus and of exhaustivity.  The adverb
\emph{sy} in Betsimisaraka (\emph{mihitsy} in Merina) means something like
`indeed' and the adverb \emph{my} in Betsimisaraka (\emph{ihany} in Merina)
means `only'.   Technically, then, Betsimisaraka has no focus particle but when
pressed to place something in this position, the choice is to put \isi{adverbs} that
lend the same flavour as a cleft.  The position of these \isi{adverbs} is not
surprising as \isi{adverbs} are often found together with the particle \emph{no} in
Merina.

\ea Betsimisaraka
    \ea
    \gll tsy ny olona \textbf{mihitsy} \custombox{\textbf{no}} tokony hiaro an'Andriamanitra\\
    \Neg{} \Det{} people indeed \emph{no} should \Fut{}-\At{}.protect \Acc{}-God\\
    \glt `It isn't in fact the people who should protect God.' \\
    (from \url{https://www.facebook.com/notes/ravonihanitra-lydia/sainam-pirenena-malagasy/10152939742301218/})\\
    \ex
    \gll 15\%n'ny \ili{Malagasy} \textbf{ihany} \custombox{\textbf{no}} manana jiro\\
    15\%-\Gen{} \ili{Malagasy} only \emph{no} \Prs.\At.have electricity\\
    \glt `It is only 15\% of \ili{Malagasy} that have electricity.'\\
    (from
    \url{http://www.sobikamada.com/index.php/vaovao/item/9918-jirama-15-n\%E2\%80\%99ny-malagasy-ihany-no-manana-jiro.html})
    \z
\z

Now we have a dialect that has no particle following the focussed\is{focus} phrase,
basically resulting in SVO.  Work needs to be done to determine in what
situations this structure can be used, and with what restrictions.  In other
words, it remains to be determined what information a language learner will be
exposed to that would indicate that this is not the basic word order of
Betsimisaraka.  But it is clear that the indications that this is \textbf{not}
a basic word order become less and less accessible as we move from \ili{Tagalog} to
Bezanozano to Merina to Betsimisaraka and it all turns on the existence and
function of the focussing particle.

\section{Summary}

Moving then from \ili{Tagalog}, to Bezanozano, to Merina, to Betsimisaraka, we
see a slow chipping away at the information given to the language learner by
the focus particle.  I am assuming that in all of these languages/dialects, the
focussed XP is within a predicate headed by an unpronounced copula\is{copulas}. I gave the
tests for this for Merina in \eqref{ex:neg} and \eqref{ex:raising}. In these
examples, negation and a raising verb respectively precede the focussed\is{focus}
element, thereby indicating the presence of a verbal element.

Turning now to the particle that follows the focussed\is{focus} XP, we have seen that in
Tagalog, the particle \emph{ang} clearly marks the left edge of a nominal
indicating that the material following the focussed\is{focus} element is a DP and the
subject of the clause.\largerpage[-1]

\ea  \ili{Tagalog}: Focussed DP $=$ Pseudo-cleft (there is a  nominal marker \emph{ang})\\
    \ea
    \gll  \textbf{Ang} \textbf{babae} \custombox{ang} \emph{bumili} \emph{ng} \emph{bigas}\\
    \Nom{} woman \Nom{} \At.\textsc{buy} \Acc{} rice\\
    \glt `[\textsubscript{DP} \custombox{The} ⟨one who⟩ \emph{bought rice} ] [\textsubscript{VP} ⟨is⟩ \textbf{the woman} ]' \\
    \ex
    \gll {$[$\textsubscript{VP} ⟨\Cop{⟩} \textbf{DP} ]} {~~} {[\tss{DP} \custombox{\emph{ang}} \emph{V O} ]}\\
         \textsc{Predicate} {} \textsc{Subject}\\
    \z
\z

The \ili{Tagalog} focussed\is{focus} PP, in contrast, is found in a cleft construction.  There
is no \emph{ang} to indicate a nominal phrase, therefore the material following
the focussed\is{focus} constituent will not be interpreted as the subject of the clause.
The subject of the clause, then, is an unpronounced expletive\is{expletives}.

\ea  \ili{Tagalog}: Focussed PP $=$ Cleft (there is no \emph{ang})\\
    \ea
    \gll \textbf{Sa}       \textbf{palengke}                  bumili                           ng bigas ang babae\\
    \Prep{} market  \At.buy  \Acc{} rice \Nom{} woman\\
    \glt `[\textsubscript{VP} ⟨was⟩ \textbf{at the market} ⟨that⟩ the woman bought rice. ] [\tss{DP} ⟨It⟩ ]' \\
    \ex
    \gll {$[$\textsubscript{VP} ⟨\Cop{⟩} PP} {~~} {[\textsubscript{CP}  V O S ] ]} ⟨Expletive⟩\\
        {\textsc{Predicate}} {} {} {\textsc{Subject}}\\
    \z
\z

Bezanozano is similar to \ili{Tagalog} in that it uses a clear nominal functional
category for the DP focussed\is{focus} construction.  This nominal functional category
gives the language learner a clear indication that the language is VOS since
the predicate, which  contains the unpronounced copula\is{copulas} and the focussed\is{focus} DP, is
followed by the nominal phrase indicated by nominal functional category
\emph{i}.

\ea Bezanozano: Focussed DP $=$ Pseudo-cleft (there is a  nominal marker  \emph{i})
    \ea
    \gll \textbf{Rakoto} \textbf{\custombox{i}} \emph{manasa} \emph{\textbf{i}} \emph{lambanay}\\
    Rakoto \Det{} \Prs.\At{}.wash \Det{} clothes.\Fpl.\Excl{}\\
    \glt `[\tss{DP} \custombox{The} ⟨one who⟩ \emph{is washing our clothes} ] [\textsubscript{VP} ⟨is⟩ \textbf{Rakoto} ]' \\
    \ex
    \gll {$[$\textsubscript{VP} ⟨\Cop{⟩} \textbf{DP} ]} {~~} {[\tss{DP} \custombox{\emph{i}} \emph{V O} ]}\\
        {\textsc{Predicate}} {} {\textsc{Subject}}\\
    \z
\z

The way the Bezanozano differs from \ili{Tagalog}, however, is that there is a
particle that is used optionally within the PP focussed\is{focus} construction.  For now
I'm going to assume that the fact that it is optional while the one that is
used in the DP focussed\is{focus} construction indicates a structural difference of some
type that allows this construction to be a cleft rather than a pseudo-cleft.

\ea  Bezanozano: Focussed PP $=$ Cleft (there is an optional \emph{i})\\
    \ea
    \gll \textbf{Amin'i} \textbf{penin-janako} \custombox{(\textbf{i})} manorotra aho\\
    with.\Gen{}.\Det{} pen-child.\Fsg.\Gen{} \emph{i} \Prs.\At{}.write \Fsg{}.\Nom{}\\
    \glt `[\textsubscript{VP} ⟨was⟩ \textbf{with my child's pen} \custombox{(that)} I am writing. ] [\tss{DP} ⟨It⟩ ]' \\
    \ex
        \gll {$[$\textsubscript{VP} ⟨\Cop{⟩} PP} {~~} {[\textsubscript{CP} \custombox{\emph{(i)}} V} {S ] ] ⟨Expletive⟩}\\
            {\textsc{Predicate}} {} {} {\textsc{Subject}}\\
    \z
\z

What is interesting is that this is the same particle that is used for the DP
focussed construction.  When it is not used, then, it falls into the
\ili{Tagalog} pattern where there is a particle in the DP focussed\is{focus} construction
and no particle in the PP focussed\is{focus} construction.  When it is used, it falls
into the Merina pattern which uses the same particle for both the DP and the PP
focussed construction.  The thought is that these mixed messages allowed for
reanalysis that eventually leads to the Merina pattern.

Merina uses the same particle for both the DP and the PP focussed\is{focus} constructions
and this particle is used elsewhere to link clauses.  This suggests that the
particle is part of the verbal extended projection, and both types of the focus
constructions are \isi{clefts}.  Since the expletive\is{expletives} subject of a
cleft\is{clefts} is not pronounced, with these constructions, there are fewer
signals as to the VOS order.  In the DP focussed\is{focus} construction, since the
surface order is S \emph{no} VO, and since the \emph{no} is not a nominal
marker, it could be susceptible for \isi{reanalysis}.

\ea Merina: Focussed DP $=$ Cleft (there is a clausal marker \emph{no})\\
       \ea
    \gll  \textbf{Rakoto} \custombox{no} manasa ny lambanay\\
    Rakoto \emph{no} \Prs.\At{}.wash \Det{} clothes.\Fpl.\Excl{}\\
    \glt `It is \textbf{Rakoto} who is washing our clothes.'  \\
    \glt `[\textsubscript{VP} ⟨was⟩ \textbf{Rakoto} \custombox{that} is washing our clothes. ] [\tss{DP} ⟨It⟩ ]' \\
    \ex
    \gll {$[$\textsubscript{VP} ⟨\Cop{⟩} DP} {~~} {[\textsubscript{CP} \custombox{\emph{no}} V} {S ] ] ⟨Expletive⟩}\\
        {\textsc{Predicate}} {} {} \textsc{Subject}\\
    \z
\ex  Merina
    \ea
    \gll Amin'ny penina \custombox{no} manorotra aho\\
    with.\Gen{}.\Det{} pen \emph{no} \Prs.\At{}.write \Fsg{}.\Nom{}\\
    \glt `[\textsubscript{VP} ⟨was⟩ \textbf{with my child's pen} \custombox{that} I am writing. ] [\tss{DP} ⟨It⟩ ]' \\
    \ex
    \gll {$[$\textsubscript{VP} ⟨\Cop⟩ PP} {~~} {[\textsubscript{CP} \custombox{\emph{no}} V} {S ] ] ⟨Expletive⟩}\\
    \textsc{Predicate} {} {} \textsc{Subject}\\
    \z
\z

In the last stage, we see that the identifying focus particle is dropped
completely.  Adverbs can appear in this position, but these \isi{adverbs} can also
appear in the Merina and Bezanozano construction.  So now without any particle,
a simple SVO order surfaces.

\ea Betsimisaraka\\
	\gll  Rakoto manasa lamba.\\
    Rakoto \Prs.\At{}.wash clothes\\
    \glt `It is Rakoto who is washing clothes.'
\z

The task remains, however, to determine the status of this order in the
language.  We know that it can be given the cleft interpretation.  We also know
that it co-exists with the VOS word order.  Whether or not the transition to
SVO can be argued to be complete, it is at least imaginable how it can happen.
It is also clear that the change turns on the \isi{reanalysis} of small functional
words that play central structural roles.

\section{Conclusion}

The purpose of this paper was to show first that small surface differences in
closely related languages can point to large underlying differences.  It also
shows how  functional words are signposts to structure and that the multiple
roles that they play both within the extended projection of one category and
across different categorial projections can increase the flexibility of
structures as well as increase the possibilities of \isi{reanalysis}.

\printchapterglossary{}

\section*{Acknowledgements}

This paper benefitted from funding from SSHRC grant 435-2016-1331 (PI:
Lisa deMena Travis) and SSHRC grant 410-2011-0977 (PI: Ileana Paul), as well as crucial
input from Baholisoa Simone Ralalaoherivony and Jeannot Fils Ranaivoson.
Further, I thank Ian Roberts.  It is hard to separate the study of \isi{parameters}
from his work over the years.  Any serious study of \isi{parameters} includes at
least \textcite{Biberauer:2016}, \textcite{RobHol2005},
\textcite{Roberts:2014}, as well as his work in \textcite{Roberts1993}.  I also
thank him for contributing to, as well as challenging, my understanding of too
many areas of syntax to list -- head-movement, diachrony, macro- and
micro-parameters, clitics, V-movement, VP-movement, pro-drop, to mention a few.
He has always been a model of research breadth and depth as well as research
integrity.

{\sloppy
\printbibliography[heading=subbibliography,notkeyword=this]
}
\end{document}
