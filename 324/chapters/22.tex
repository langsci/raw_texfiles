\chapter{Word Classes}
\label{chap: word classes}

Choguita Rarámuri lexemes may be classified into two major word classes, namely nominals and verbs. Membership of lexemes to each class is distinguished on semantic, morphological, and syntactic grounds. In addition to these major word classes, Choguita Rarámuri possesses minor word classes, which include numerals, quantifiers, and demonstratives, among others, as well as a complex set of clitics and modality particles. This chapter provides an overview of word classes and the criteria to distinguish them.

\section{Nouns}
\label{sec:22:nominals}

%%Morphosyntactic, morphological and phonological properties. Tonally heterogeneous, as well as displaying the stressed/unstressed contrast (i.e., with fixed stress vs. alternating stress depending on the morphological context).

\subsection{Kinship terms}
\label{subsec:22:kinship terms}

\subsection{Body-part terms}
\label{subsec:22:body-part terms}

\section{Pronouns}
\label{sec:22:pronouns}

\subsection{Personal Pronouns}
\label{subsec:22:personal pronouns}

%%a)	First and second person subjects are encoded by subject pronouns. 

%%b)	First and second person objects are encoded by object pronouns. 

%%Table 2.1: pronominal forms
	%%Subject		Object	
	%%Full	Enclitic	Full	Enclitic
%%1st person singular	nehé	=ni	tamí	
%%2nd person singular	muhé	=mi	mi	=mi
%%1st person plural	tamuhé ~ tamó	=ti	tamí	
%%2nd person plural	émi	=timi	mi	=mi

\subsection{Interrogative Pronouns}
\label{subsec:22:interrogative pronouns}

%%Insert cross-reference to Chapter 9 (constructions)
%%Table 2.2: Choguita Rarámuri interrogative pronouns

\subsection{Possessive Pronouns}
\label{subsec:22:possessive pronouns}

\subsection{Demonstratives}
\label{subsec:22:demonstratives}

\section{Verbs}
\label{sec:22:verbs}

\section{Adverbs}
\label{sec:22:adverbs}

\subsection{Spatial Adverbs}
\label{subsec:22:spatial adverbs}

\subsection{Temporal Adverbs} 
\label{subsec:22:temporal adverbs}

\section{Numerals}
\label{sec:22:numerals}

\subsection{Epistemic Particles}
\label{subsec:22:epistemic particles}

\subsection{Final Clitics}
\label{subsec:22:final clitics}

\subsection{Pronominal Clitics}
\label{subsec:22:pronominal clitics}

\subsection{Empahtic Clitics}
\label{subsec:22:empahtic clitics}

\subsection{Negative Particles}
\label{subsec:22:negative particles}

\section{Interjections}
\label{sec:22:interjections}

\section{Summary}
\label{sec:22:summary}

%%The prosodic and grammatical status of formatives as pragmatic markers – 
%%The question of whether the clitics are Wackernagel (second position) clitics (reference to Chiro’s thesis)
