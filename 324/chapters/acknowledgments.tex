\chapter{Acknowledgments}

I would like to express my deepest gratitude to all the Choguita Rarámuri language experts who patiently and generously invested large amounts of time into teaching me about Rarámuri language and culture. I am especially grateful to elder †Luz Elena León Ramírez, Sebastián Fuentes Holguín, and Bertha Fuentes Loya for their deep involvement in this project and constant encouragement over the years. I am also grateful to María Del Rosario Cervantes Guerrero, Rosa Isela Chaparro Gardea, María Guadalupe Diaz, elder Mateo Espino, Santos Fuentes, †Rosa Fuentes, elder †José María “Chémale” Fuentes, Miguel Angel Fuentes Diaz, Virginia Fuentes Diaz, Angelina Fuentes Diaz, Reyes Fuentes Guerrero, elder †Cornelio Fuentes Hernández, elder †Morales Fuentes Hernández, Carlos Fuentes Moreno, Guillermina Fuentes Moreno, Valentina Fuentes Moreno, Vicente Fuentes Moreno, Giltro Fuentes Palma, Teresa Guerrero Herrera, elder María Dolores Holguín, Javier Holguín Fuentes, Jesusita Loya Guerra, elder †Federico León Pacheco, Francisco Moreno Fuentes and Sebastián Moreno Morales for their contributions as authors, consultants, teachers and creators of language documentation. Many community members also welcomed me into their homes, healing ceremonies, trips to neighboring communities, house parties, graduations, gatherings for races, and harvest celebrations. These relationships contributed immensely to both my understanding of Rarámuri language and culture and my enjoyment of life in Choguita.

All language experts I worked with made crucial contributions to the analyses presented in this grammar.  That this grammar is published under my name reflects the fact that these analyses are written in my words. However, this should not be taken to assert authority over the Choguita Rarámuri language itself. The language, the examples of its use quoted in this grammar, and the knowledge expressed therein remain the intellectual property of the Rarámuri people.

Michael Casaus first introduced me to the community of Choguita, and shared local knowledge and advice. I learned a great deal from his high standards in his work, and benefited from the strong ties he developed in Choguita.

Some portions of this grammar were written as part of my doctoral dissertation at UC Berkeley, where I was incredibly fortunate to have been supervised and guided by Andrew Garrett, Sharon Inkelas and Johanna Nichols. I am grateful to them for playing a crucial role in my understanding of the Choguita Rarámuri language and how it should be best described. Many further improvements were based on feedback and discussions with many colleagues in subsequent years. I am particularly indebted to (in alphabetic order) Andrés Aguilar, Karen Dakin, Marc Garellek, Austin German, Alice Harris, Jason Haugen, Vsevolod Kapatsinski, Enrique Palancar, Bert Remijsen, Sharon Rose, †Enrique Servín, Maziar Toosarvandani and Verónica Vázquez Soto. Johanna Nichols, in particular, gave me enormously valuable feedback in later states of the writing of this grammar. I am also grateful to Lucien Carroll, who contributed greatly to description and analyses of the language as a doctoral research assistant at UCSD, collaborator and co-author. He also developed the web-based user interface that allows access to corpus materials for both community and academic audiences. 

I benefited a great deal from Martin Haspelmath’s thorough reading of this manuscript, which produced many constructive comments and suggestions that eradicated many errors. I am extremely grateful to have had such a wise, meticulous and kind editor. I am also grateful to all anonymous reviewers for their valuable feedback and to Sebastian Nordhoff for his patience and efficiency in providing support with technical and typesetting matters at Language Science Press.

My research on Choguita Rarámuri was funded by Mexico’s Consejo Nacional de Ciencia y Tecnología (CONACYT), the University of California Institute for Mexico and the United States (UCMEXUS), the UC Berkeley Linguistics Department, UC Berkeley’s Survey of California and Other Indian Languages, the Endangered Languages Documentation Programme (ELDP) (under awards IGS0042 and IPF0138), and the National Science Foundation/Documenting Endangered Languages program (NSF-DEL) (under award 1160672).  

Finally, I would like to thank my family and friends for their unwavering support throughout this process. In particular, I am grateful to Cecilia Hernández Carrillo, †María del Carmen Carrillo Velázquez and †Aristeo Hernández León for their love, encouragement and support since the very beginning. Without them, this book would not be possible. And to Paloma, for being a constant source of joy and for sharing my love of Choguita and the Sierra Tarahumara.