\chapter{Other word-level supra-segmental processes}
\label{chap: other word-level suprasegmental phonology}

In addition to stress and tone, Choguita Rarámuri features other word-level pho\-nol\-o\-gical phenomena that involve supra-segmental processes and/or prosodic domains (for further discussion of word-prosodic phenomena in the language, see \chapref{chap: prosody}). These phenomena are addressed in this chapter and include:
(i) a disyllabic window for glottal stops (§\ref{subsec: glottal stop});
(ii) minimality effects in verbs (§\ref{subsec: minimal word size}); and
(iii) loanword prosodic adaptation patterns (§\ref{subsec: optional prosodic loanword adaptation patterns}).


\section{Glottal stop: an initial disyllabic window}
\label{subsec: glottal stop}

As discussed in \chapref{chap: syllables}, Choguita Rarámuri syllables are underlyingly CV in shape, with no codas, except for glottal stop. This glottal stop displays strict restrictions on its distribution: glottal stop can only occur between the first and second syllable, either intervocalically (\ref{ex: intervocalic glottal stop}) or pre-consonantally (\ref{ex: pre-consonantal glottal stop}).

\ea\label{ex: intervocalic glottal stop}
{Intervocalic glottal stop}

    \ea[]{
    \textit{koˈ\textbf{ʔ}á}  \\
    ‘eat’\\
    ‘comer’ \corpuslink{tx905[00_540-00_577].wav}{GFM tx905:0:54.0}\\
}
        \ex[]{
        \textit{raˈ\textbf{ʔ}ìtʃ͡a}\\
        ‘speak’\\
        ‘hablar’ \corpuslink{co1235[06_095-06_111].wav}{JLG co1235:6:09.5}\\
    }
            \ex[]{
            \textit{ba\textbf{ʔ}uˈrâ  }\\
            ‘brand livestock’\\
            ‘marcar con fierro los animales’ < SFH VDB/el >    \\
        }
                \ex[]{
                \textit{bo\textbf{ʔ}aˈlâ}\\
                 bo\textbf{ʔ}a-ˈlâ	 \\
                 feather-\textsc{poss}\\
                 ‘feathers, fur’\\
                 ‘plumas, pelaje’ < LEL 5:127-142/el >\\
            }
                    \ex[]{
                    \textit{moˈ\textbf{ʔ}è }  \\
                    ‘weave’\\
                    ‘tejer’  < SFH 05 1:143/el >   \\
                }
   \z
\z

\ea\label{ex: pre-consonantal glottal stop}
{Pre-consonantal glottal stop}

    \ea[]{
    \textit{ra\textbf{ʔ}ˈná }  \\
    ‘explode’\\
    ‘tronar’ < SFH 04 1:81/el >\\
}
        \ex[]{
        \textit{a\textbf{ʔ}ˈtá }  \\
        ‘arch’\\
        ‘arco’     < SFH NDB/el >\\
    }
            \ex[]{
            \textit{ba\textbf{ʔ}tʃ͡iˈwí }  \\
            ‘sound of water dripping’\\
            ‘sonido de goteo’     < BFL 07 2:31/el >\\
        }
                \ex[]{
                \textit{bi\textbf{ʔ}ˈrì }   \\
                ‘twist’\\
                ‘torcerse’     < SFH 04 1:114/el >\\
            }
                    \ex[]{
                    \textit{ba\textbf{ʔ}ˈwí}\\
                    `water'\\
                    `agua' \corpuslink{co1234[06_044-06_053].wav}{JLG co1234:6:04.4}\\
                }
                        \ex[]{
                        \textit{raʔˈtàtʃ͡i}\\
                        `hot/dry season'\\
                        `tiempo de calor' \corpuslink{tx904[01_429-01_470].wav}{GFM tx904:1:42.9}\\
                    }
    \z
\z

The examples in (\ref{ex: Glottal stop is not immediately pre-tonic}) show that glottal stop is associated with the word-initial syllable, as opposed to being associated to the immediately pretonic position: each of the following examples involves forms with a CV\textbf{ʔ}VCV syllable structure with stress in the third syllable.

\ea\label{ex: Glottal stop is not immediately pre-tonic}
{Glottal stop is not immediately pre-tonic}

    \ea[]{
    \textit{bo\textbf{ʔ}oˈbû}\\
    \textit{bo\textbf{ʔ}o-ˈbû}  \\
    unpluck-\textsc{tr.imp.sg}\\
    `Unpluck it!’ \\
    ‘¡Desplúmalo!’ < SFH 08 1:51/el >\\
}
        \ex[]{
        \textit{ra\textbf{ʔ}aˈmâsima}  \\
        \textit{ra\textbf{ʔ}aˈmâ-si-ma}\\
         give.advice-\textsc{mot-fut.sg}\\
         `S/he will go along giving advice.'\\
         ‘Va a ir aconsejándolos.’ \corpuslink{tx816[00_399-00_434].wav}{JMF tx816:0:39.9}\\
    }
            \ex[]{
           \textit{ra\textbf{ʔ}iˈtʃ͡âsa}  \\
           \textit{ra\textbf{ʔ}iˈtʃ͡â-sa}\\
            speak-\textsc{cond}\\
            ‘if she speaks’\\
            ‘si habla’ \corpuslink{tx221[05_048-05_111].wav}{LEL tx221:5:04.8}\\
        }
    \z
\z

% {On the other hand, there are no roots with both first syllable stress and glottal stop, nor monosyllabic roots with glottal stop (CV}{\textbf{ʔ}}{) attested in the corpus.}

Finally, when verb roots bearing a glottal stop, incorporate a body-part nominal root, the glottal stop underlying to the verbal root is deleted in the surface form, as exemplified in (\ref{ex: Glottal stop deletion in incorporation}) (in these forms, the compound form has stress in the first syllable of the verb, the head of the compound; for more details about the stress and tone properties of compounds, see §\ref{subsec: stress and tone properties of compounds}).

\ea\label{ex: Glottal stop deletion in incorporation}
{Glottal stop deletion in noun incorporation}

    \ea[]{
    [kutaˈbîri] \\
    /kuˈtâ+bi\textbf{ʔ}ˈrì/   \\
    neck+twist\\
    `to neck-twist'\\
    `torcer el cuello' < BFL 07 1:163/el >\\
}\label{ex: Glottal stop deletion in incorporationa}
        \ex[]{
        [tʃ͡omaˈbîwa] \\
        /tʃ͡oˈmá+bi\textbf{ʔ}ˈwá/\\
        mucus+clean \\
        `to clean one's nose'\\
        `limpiarse la nariz' < SFH 07 VDB/el >\\
    }\label{ex: Glottal stop deletion in incorporationb}
            \ex[]{
            [witaˈbîwa]\\
            /witá+biʔˈwá/  \\
            excrement+clean\\
            `to clean excrement'\\
            `limpiar excremento' < SFH 07 1:187/el >\\
        }\label{ex: Glottal stop deletion in incorporationc}
    \z
\z

As shown in these examples, the underlying glottal stop of the verbs \textit{bi\textbf{ʔ}rì} `twist' (\ref{ex: Glottal stop deletion in incorporationa}) and \textit{biʔwá} ‘clean’ (\ref{ex: Glottal stop deletion in incorporation}b--c) is lost in noun incorporation. This process suggests that glottal stop must not only emerge within an initial disyllabic window within the prosodic word (after the first vowel of this domain), but it must also emerge in the lexical root to which it is underlyingly associated.

Similar distributional restrictions of glottal stop in closely related \ili{Mountain Guarijío} are analyzed in \citet{caballero2006templatic} and \citet{haugen2014laryngeals} as resulting from surface glottal stops being underlyingly a floating constricted glottis ([+c.g.]) feature. Alternatively, these distributional facts can be analyzed as resulting from distributional restrictions of a glottal stop segment. Outside of the \ili{Uto-Aztecan} language family, similar restrictions on laryngeal features have been documented in varieties of \ili{Basque}, where the glottal fricative ([h]) and aspirated stops are restricted to occur within an initial disyllabic window \citep{hualde2018aspiration}.

Finally, the phonemic glottal stop should be distinguished from glottalization associated with HL tones in utterance final position, a tone-specific effect where a stressed vowel bearing a HL tone in the IP boundary may be rearticulated (/î/ → [íʔì]), as described above in §\ref{subsec: non-tonal encoding of intonation}.

\section{Minimality effects}
\label{subsec: minimal word size}

There is evidence in Choguita Rarámuri for a minimal word size for verbs. Verbal inflectional categories like recent past, present and the imperative singular may be realized by the bare verbal stem. An impressionistic assessment of unsuffixed, monosyllabic verbs inflected for these categories reveals they involve significantly longer vocalic nuclei than their suffixed counterparts. In (\ref{ex: Vowel lengthening of monosyllabic roots}), I present examples of these monosyllabic verb roots when inflected for imperative singular (realized as the lengthened, bare stem), which contrast with suffixed forms that do not involve lengthening (shown in the second column).

\ea\label{ex: Vowel lengthening of monosyllabic roots}
{Vowel lengthening of monosyllabic roots}

    \ea[]{
   [ˈpáa]   \\
   \glt     /ˈpá/\\
   throw.\textsc{imp.sg}\\
    ‘Throw!’  \\
    `¡Tíralo!' \corpuslink{co1234[03_198-03_208].wav}{JLG co1234:3:19.8}\\
}
        \ex[]{
       [ˈpáma]   \\
       \glt /ˈpá-ma/\\
        throw-\textsc{fut.sg}  \\
        `S/he will throw it.'\\
        `Lo va a tirar.' \corpuslink{co1234[16_356-16_372].wav}{JLG co1234:16:35.6}\\
}
        \ex[]{
        [ˈmàa]  \\
        \glt     /ˈmà/\\
        run.\textsc{sg.imp.sg}\\
        ‘Run!' \\
        `¡Corre!'\\
    }
            \ex[]{
            [ˈmàma]    \\
            \glt    /ˈmà-ma/\\
            run.\textsc{sg-fut.sg}  \\
            `S/he will run.'\\
            `Va a correr' < BFL 06 el14/el >\\
        }
                \ex[]{
                [ˈnóo]    \\
                \glt    /ˈnó/\\
                look.\textsc{imp.sg}\\
                ‘Look!’  \\
                `¡Mira!'\\
            }
                    \ex[]{
                    [ˈnóma]    \\
                     \glt   /ˈnó-ma/\\
                     look-\textsc{fut.sg}  \\
                     `S/he will look.'\\
                    `Va a mirar.' < BFL 06 el38/el >\\
                }
                        \ex[]{
                        [ˈàa]   \\
                        \glt    /ˈa/\\
                        find.\textsc{imp.sg}\\
                        `Find it!'  \\
                        `¡Encuéntralo!' < BFL 05 1:112/el >\\
                    }
                            \ex[]{
                             [ˈàma] \\
                             \glt   /ˈà-m/a\\
                             find-\textsc{fut.sg}\\
                             `S/he will find it.'\\
                            `Lo va a encontrar.' < BFL 05 1:112/el >\\
                        }
    \z
\z

Precise measurements of vowel length of a small sample of monosyllabic roots in unsuffixed and suffixed contexts confirms there is a difference between the vowel durations of unsuffixed, monosyllabic (imperative or present inflected) words and their suffixed (disyllabic) counterparts. As exemplified in (\ref{ex: Vowel duration of monosyllabic verbal roots}), the vowel duration of unsuffixed monosyllabic verbs ranges between 140--200ms, while the vowel duration of the suffixed counterpart ranges between 60--120ms.

%convert to table
\pagebreak

\ea\label{ex: Vowel duration of monosyllabic verbal roots}
{Vowel duration of monosyllabic verbal roots }

\begin{tabular}{lllll}
       & \textit{Form} & \textit{V duration} & \textit{Gloss}  & \textit{Translation} \\
     a.& {ˈmêe}  & 170ms & win.\textsc{imp.sg}	&  ‘Win!'\\
     b.& {ˈmê-ma} & 100ms &  win-\textsc{fut.sg} & `S/he will win.'\\
     c.& {ˈpáa} & 140ms & throw.\textsc{imp.sg} & ‘Throw it!' \\
     d.& {ˈpá-ma} & 60ms & throw-\textsc{fut.sg} & `S/he will throw it.' \\
     e.& {ˈmàa} 	& 200ms &  run.\textsc{imp.sg} & ‘Run!’  \\
     f.& {ˈmà-sa} 	& 110ms &  run-\textsc{imp.sg} & ‘Run!’ \\
     g.& {ˈàa} 	& 200ms & find.\textsc{imp.sg} & `Find it!’   \\
     h.& {ˈà-mi} 	& 120ms & find-\textsc{mot.imp} & `S/he will go find it.'  \\
     i.& {ˈâa} 	& 170ms &  give.\textsc{imp.sg} & ‘Give it!’   \\
     j.& {ˈà-ki} 	& 110ms & give-\textsc{pst.ego} & `I gave it.'’   \\
\end{tabular}
    \z

Minimality constraints are asymmetrical in Choguita Rarámuri, as not all morphological classes of words show a minimality condition. As commonly found cross-linguistically, function words are canonically monosyllabic and do not undergo lengthening or undergo any prosodic augmentation process. Furthermore, other major word classes in Choguita Rarámuri are also exempt from minimality requirements. This is exemplified with monosyllabic nouns, which do not exhibit lengthening in unsuffixed contexts like verbal monosyllables do. Measurements of vocalic durations of monosyllabic nouns in unsuffixed and suffixed contexts shows that unsuffixed nouns have comparable durations to the vowels of monosyllabic verb roots when suffixed (\ref{ex: vowel duration of monosyllabic of nominal roots}).

\ea\label{ex: vowel duration of monosyllabic of nominal roots}
{Vowel duration of monosyllabic nominal roots}

\begin{tabular}{lllll}
       & \textit{Form} & \textit{V duration} & \textit{Gloss}  & \textit{Translation} \\
     a.& ˈkú  & 90ms & wood	&  ‘wood'\\
     b.& ˈkú-riri & 60ms &  wood-\textsc{loc} &  `in the wood' \\
     c.& ˈlá & 90ms & blood & ‘blood' \\
     d.& ˈlá-riri & 70ms & blood-\textsc{loc} &  `in the blood' \\
     e.& ˈmè	& 100ms &  agave & ‘agave’  \\
     f.& ˈmè-riri 	& 60ms &  mezcal-\textsc{loc}  & ‘in the agave’ \\
     g.& ˈwá 	& 110ms & arrows & ‘arrows’   \\
     h.& ˈwá-ti 	& 100ms & arrows-\textsc{inst} &  `with the arrows' \\
\end{tabular}
    \z

Thus, there is a minimal word constraint which targets verbs. This constraint is defined in (\ref{ex: Choguita Rarámuri minimal word constraint}):

\ea\label{ex: Choguita Rarámuri minimal word constraint}
{Choguita Rarámuri minimal word constraint for verbs}

Minimal word (X\textsubscript{0}) = [[$\mu$$\mu$]]\textsubscript{${\sum}$}

\z

According to this constraint, all verbs in Choguita Rarámuri are at least two moras, where consonants are non-moraic. While vowel length is not contrastive in Choguita Rarámuri, long vowels are derived to satisfy a minimal word constraint, as proposed in this section. Long vowel sequences can also be found after semi-vowel deletion (§\ref{subsec: semi-vowel deletion}) and \textit{h} deletion (§\ref{subsec: vowel sequences}), and in morphologically restricted contexts (addressed in \chapref{chap: verbal morphology}, such as compensatory lengthening (§\ref{subsubsec: compensatory lengthening}), and lengthening triggered by a specific suffix (§\ref{subsubsec: past passive-conditioned lengthening})). The role of vowel length and the minimality restriction of Choguita Rarámuri inflected verb stems is discussed further in \chapref{chap: prosody}.

\section{Loanword prosody}
\label{sec: loanword prosody}

This section presents an overview of prosodic adaptation of \ili{Spanish} loanwords in Choguita Rarámuri (see also \citealt{caballero2013procesos}). As discussed in this section, Choguita Rarámuri retains the original prominence from \ili{Spanish}, which features a stress-accent system, but deploys several repair strategies to make loanwords conform to  prosodic restrictions on morphological structure. Specifically, repair strategies are sensitive to strong prosodic restrictions on morphological structure in Choguita Rarámuri, predominance of certain word/morpheme types and acoustic properties of \ili{Spanish} prominence.

Speakers of Rarámuri varieties have been in contact with \ili{Spanish} speakers since the seventeenth century (according to \citet{alegre1956historia}[1767], the first contact with \ili{Spanish} speakers in the Sierra Tarahumara was in 1607).\footnote{While Rarámuri speakers have been in contact  with speakers of other \ili{Uto-Aztecan} languages (including speakers of \ili{Nahuatl} (\ili{Aztecan})) and other North American Indian languages, before and after colonial times, the focus here is in Rarámuri-\ili{Spanish} language contact phenomena.} As discussed in §\ref{subsec: history of contact with Europeans}, while the Rarámuri have historically avoided contact with the \textit{mestizo} (non-indigenous Mexican) population through the retreat to isolated, more mountainous areas (\citealt{merrill2014ralamuli}, \citealt{cortina2012hijos}), Rarámuri speakers in Choguita have had intense contact with \ili{Spanish} speakers for almost the entirety of the 20th century (e.g., the first road in Choguita was built  circa 1920 \parencite{casaus2008quantitative}. This intense language contact across the Sierra Tarahumara has brought about a high level of bilingualism and language attrition (see also discussion in §\ref{sec: social and historical context}).

Prosodic loanword adaptation processes in Choguita Rarámuri have a bearing on the question of what may be possible patterns of loanword prosody adaptation in contact situations with languages with divergent word prosodic systems (stress, tone, “pitch accent”) \parencite{kubozono2006does, kuang2013tonal}. In contrast to Choguita Rarámuri, which features a `hybrid' word prosodic system (with both stress and tone), \ili{Spanish} has lexically contrastive stress, but no lexical tone. The main features of the \ili{Spanish} stress systems are the following:

\ea\label{ex: Spanish stress-accent system}
{\ili{Spanish} stress-accent system}
    \ea[]{
    Primary stress falls within a final three-syllable window \parencite{hualde2012stress}\\
}
        \ex[]{
        Stress assignment involves trochaic feet \citep[][560]{lipski1997spanish}\\
    }
            \ex[]{
            There is iterative stress assignment aligned to the right edge of the prosodic word, with a predominance of penultimate stress in the language\\
        }
                \ex[]{
                There is no phonological vowel length contrast \parencite{chavez2007loanword, hualde2012stress}\\
            }
    \z
\z

The \ili{Spanish} stress-accent system finds a direct correlate in the stress-accent system of Choguita Rarámuri, while the tonal properties of loanwords can be interpreted as arising from either a default grammatical tone process (as described in §\ref{subsubsec: stress-based tonal neutralization}) or as a reinterpretation of \ili{Spanish} post-lexical tonal phenomena.

The focus here is on loanwords that exhibit some degree of phonological adaptation, either segmentally or prosodically. No assumptions are made regarding the degree to which hypothesized loanwords which exhibit no adaptation constitute instead code-switching of bilingual speakers. Given the lack of sociolinguistic data, there are also no assumptions made here regarding the degree of adaptation as a diagnosis for sociolinguistic situation/level of bilingualism (see discussion in \citealt{sicoli1999loanwords}) nor the relative chronology of incorporation of loanwords. These questions are left for further research.

The \ili{Spanish} loanword data examined are mostly penultimate stressed \ili{Spanish} nouns. Some of the loanwords were likely introduced via \ili{Nahuatl}, e.g., \textit{basa}ˈ\textit{lówa} ‘stroll’ (from \ili{Spanish} `pasear’ and the Náhuatl \textit{{}-oa} impersonal suffix) and \textit{ko}ˈ\textit{máare} ‘comadre’ and \textit{kom}ˈ\textit{páare} ‘compadre’ \parencite{nordell1984spanish}.

\subsection{Exceptionless prosodic loanword adaptation patterns}
\label{subsec: exceptionless prosodic loanword adaptation patterns}

There are several prosodic loanword adaptation patterns that are exceptionless in Choguita Rarámuri. These are listed in (\ref{ex: exceptionless prosodic loanword  adaptation processes}).

\pagebreak

\ea\label{ex: exceptionless prosodic loanword  adaptation processes}
{Exceptionless prosodic loanword adaptation patterns}\mbox{}
    \ea[]{
    Stressed syllables in \ili{Spanish} are borrowed as the prominent syllables in Choguita Rarámuri via stress and a HL tone.\\
}
        \ex[]{
        Stressed vowels in \ili{Spanish} retain their original quality when incorporated into Choguita Rarámuri.\\
    }
            \ex[]{
            Consonant final source words from \ili{Spanish} correspond to vowel final loanwords in Choguita Rarámuri.\\
        }
                \ex[]{
                Pre-tonic syllable truncation of sourcewords preserves prominence in loanwords within the Choguita Rarámuri initial three syllable stress window.\\
            }
                    \ex[]{
                    Compounds and phrases are borrowed with prominence in the second element.\\
                }
    \z
\z

As shown in (\ref{ex: tone in loanwords from Spanish}) in §\ref{subsubsec: stress-based tonal neutralization} above, stressed syllables of \ili{Spanish} source words are retained as the stressed syllables of loanwords in Choguita Rarámuri. Without exception, the stressed syllables of loanwords bear a HL tone. This is exemplified in (\ref{ex: stress and tone in loanwords}) (stressed syllables in \ili{Spanish} source words are highlighted in boldface).\footnote{\textit{Limeta} in (\ref{ex: stress and tone in loanwords}a) is an archaic word no longer in use in contemporary Northern Mexican \ili{Spanish}. As for the loanword \textit{leˈhîdotʃ͡i} in (\ref{ex: stress and tone in loanwords}i): as discussed in §\ref{sec: geographic location and physical environment} above, \textit{ejido} is a rural land plot for collective use by community members (ejidatarios) administered by the federal government.}

\ea\label{ex: stress and tone in loanwords}
{Stress and tone properties of \ili{Spanish} loanwords}

\begin{tabular}{lllll}
    & \textit{Loanword} & \textit{Translation} & \textit{Source word} & \textit{Source} \\
     a.& liˈmêta-ʧ͡i & `bottle' & li\textbf{me}ta \\
     b.& maˈsâna & `apple' & man\textbf{za}na \\
     c.& ˈsâbaru &	`Saturday' &	\textbf{sá}bado & \corpuslink{el1318[15_224-15_237].wav}{MFH el1318:15:22.4}\\
     d.& boˈtêja-ʧ͡i &`bottle' & bo\textbf{te}lla& \corpuslink{tx191[00_299-00_319].wav}{BFL tx191:0:29.9}\\
     e. & esˈkwêla & `school' & es\textbf{cue}la & \corpuslink{tx12[08_067-08_089].wav}{SFH tx12:8:06.7}\\
     f. & ˈtʃ͡îba &  `goat' & \textbf{chi}va & \corpuslink{in484[05_026-05_058].wav}{ME in484:5:02.6}\\
     g. & sibiˈrîko & `Federico' & Fede\textbf{ri}co & \corpuslink{el417[00_109-00_127].wav}{BFL el417:0:10.9}\\
     h. & {basaˈlôa}& `to stroll'& {pa\textbf{sear}}&\corpuslink{tx84[00_538-00_554].wav}{LEL tx84:0:53.8}\\
     i. & leˈhîdotʃ͡i& `ejido' & e\textbf{ji}do & \corpuslink{tx817[01_030-01_066].wav}{JMF tx817:1:03.0}\\
     j. & rupuˈrâni & `plane' & aero\textbf{pla}no & \corpuslink{tx12[02_147-02_180].wav}{SFH tx12:2:14.7}\\
\end{tabular}
    \z


As shown in these examples, while unstressed vowels of source words may undergo adaptation in Choguita Rarámuri (e.g. (\ref{ex: stress and tone in loanwords}c)), stressed vowels in loanwords retain the original \ili{Spanish} source word quality when incorporated into Choguita Rarámuri.

Another exceptionless prosodic pattern attested in loanwords concerns syllable structure: consonant final source words correspond to vowel-final loanwords in Choguita Rarámuri, given the requirement of all Prosodic Words in Choguita Rarámuri to be vowel-final (see §\ref{sec: defining the prosodic word and other prosodic domains in CR} for more details about this). This is shown in (\ref{ex: adaptation of C-final loanwords}).

\ea\label{ex: adaptation of C-final loanwords}
{Adaptation of consonant-final source words}

\begin{tabular}{llll}
    & \textit{Loanword} & \textit{Translation} & \textit{Source word} \\
     a.& raˈniêli & `Daniel' & Daniel\\
     b.& mehoˈrâra & `acetaminophen' & mejoral\\
     c.& aˈsûkara &	`sugar' &	azúcar \\
     d.& ˈôso \textasciitilde ˈôsu &	`sickle' &	hoz \\
\end{tabular}
    \z

As shown in these examples, vowel epenthesis word finally repairs loanwords to satisfy the optimal prosodic structure of Choguita Rarámuri. The quality of the epenthesized vowel is either identical to the quality of the stressed vowel if the vowel is low (e.g., \ref{ex: adaptation of C-final loanwords}b--c), or may undergo optional reduction if it is a mid, back rounded vowel (e.g., \ref{ex: adaptation of C-final loanwords}d). If the final vowel is the front, mid vowel [e], the epenthesized vowel is [i] (\ref{ex: adaptation of C-final loanwords}a).

\ili{Spanish} loanwords with final or penultimate stress where stress would fall outside the Choguita Rarámuri initial three syllable stress window undergo pre-tonic syllable truncation in order to confirm to this strict prosodic requirement of the language. Relevant examples are shown in (\ref{ex: pre-tonic syllable truncation in Spanish lopanwords}), where truncated syllables from the \ili{Spanish} source words are highlighted in boldface.

\ea\label{ex: pre-tonic syllable truncation in Spanish lopanwords}
{Pre-tonic syllable truncation in \ili{Spanish} loanwords}

    \ea[]{
    \glt    \doublebox{\textit{nauguˈrârpo}}{  Sp. \textit{\textbf{i}naugurar}}\\
    \glt    \textit{nauguˈrâr-po}\\
    \glt    inaugurate-\textsc{fut.pl}\\
    \glt    `We will inaugurate.'\\
    \glt    `Vamos a inaugurar.'\\
}
        \ex[]{
        \glt    \doublebox{\textit{seraˈdêroʧi}}{  Sp. \textit{\textbf{a}serradero}}\\
        \glt    \textit{seraˈdêro-ʧi}\\
        \glt    sawmill-\textsc{loc}\\
        \glt    `sawmill'\\
    }
            \ex[]{
           \glt  \doublebox{\textit{kiriˈsâante}}{  Sp. \textit{\textbf{fer}tilizante}}\\
            \glt    `fertilizer’\\
        }
    \z
\z

\ili{Spanish} compounds and phrases, including very commonly compound proper names, are borrowed into Choguita Rarámuri with the prosodic structure of compounds in Choguita Rarámuri (see §\ref{subsec: stress and tone properties of compounds}), which without exception involves prominence in the second element. As seen in (\ref{ex: loanword compounds}), compounds and phrases are incorporated with the original prominence of the second member of the construction.

\ea\label{ex: loanword compounds}
{Loanword prosodic adaptation of compounds and phrases}

\begin{tabular}{llll}
    & \textit{Loanword} & \textit{Source word} & \textit{Source}\\
     a.& mariˈsûsi & María Jesusa & \\
     b.& mariˈnâsia & María Ignacia & \corpuslink{in61[04_512-04_523].wav}{FLP in61:4:51.2}\\
     c.& susmaˈrîa & Jesús María & \corpuslink{in484[06_582-06_594].wav}{ME in484:6:58.2}\\
     d.& ʃimaˈrîa \textasciitilde semaˈrîa & José María \\
     e.& ripiˈrâar-ʧ͡i & Día (del) Pilar-\textsc{loc}&\corpuslink{tx12[06_108-06_147].wav}{SFH tx12:6:10.8} \\
\end{tabular}
    \z


These examples show that pre-tonic syllable truncation of the source word (e.g., \textit{sus} from \textit{Jesús} in (\ref{ex: loanword compounds}c)) keeps stress in the original \ili{Spanish} stress pattern without violating the strict requirement that stress is placed within the initial three-syllable stress window margin.

\subsection{Optional prosodic loanword adaptation patterns}
\label{subsec: optional prosodic loanword adaptation patterns}

In addition to exceptionless loanword prosodic adaptation patterns, there are also patterns where adaptation does not involve repair of illicit prosodic sequences and patterns where an adaptation process applies only optionally. These processes, however, bring the prosodic form of loanwords in conformity with the canonical prosodic structures of Choguita Rarámuri (for discussion of canonical prosodic structures in the language, see §\ref{subsec: canonical prosodic shapes of roots and suffixes} below).

One first case involves loanwords where pre-tonic truncation applies to the source word, despite that there is no violation of a prosodic constraint in Choguita Rarámuri. Relevant examples are given in (\ref{ex: pre-tonic truncation with no violation}), with the truncated syllable highlighted in boldface in the \ili{Spanish} sourceword.

\ea\label{ex: pre-tonic truncation with no violation}
{Pre-tonic truncation with no violation of Choguita Rarámuri prosody}

\begin{tabular}{llll}
    & \textit{Loanword} & \textit{Translation} & \textit{Source word} \\
     a.& neˈrâali &  `generals' & \textbf{ge}nerales\\
     b.& maˈnâke &  `calendar' & \textbf{al}manaque\\
     c.& terˈnâdo-ʧ͡i & `boarding house' & \textbf{in}ternado \\
     d.& prosˈtâante &  `protestant' & {pro\textbf{tes}tante} \\
     e.& niˈsêta &  `Aniseta' & \textbf{A}niseta \\
\end{tabular}
    \z

Without syllable truncation, these forms would conform to the three-syllable stress window. However, syllable truncation in these cases renders these loanwords with the most frequent prosodic forms attested in the native vocabulary, namely stems with second syllable stress (see §\ref{subsec: canonical prosodic shapes of roots and suffixes}), which are represented both by forms with lexically specified stress as well as forms that receive stress by default in the second syllable of the root (see §\ref{sec: stress properties of roots, stems and suffixes}).

There is also optional consonant cluster repair, with pre-tonic consonant clusters being variably repaired by epenthesis (\ref{ex: optional consonant cluster repair}a--b) or consonant deletion (\ref{ex: optional consonant cluster repair}c--e). Post-tonic consonant clusters, on the other hand, are generally not repaired (\ref{ex: optional consonant cluster repair}f--j) (\ili{Spanish} consonant clusters that are adapted or fail to undergo adaptation are highlighted in boldface).

\ea\label{ex: optional consonant cluster repair}
{Optional consonant cluster repair in loanwords}

\begin{tabular}{llll}
    & \textit{Loanword} & \textit{Translation} & \textit{Source word} \\
     a.& kaˈrâsia &  `thank you' & \textbf{gr}acias\\
     b.& piriˈhîna  & `Virginia' & Vi\textbf{rg}inia\\
     c.& apaˈnêra    & `wife' & co\textbf{mp}añera \\
     d.& katiˈrâria-ʧ͡i & `day of the Candelaria' & {Ca\textbf{nd}elaria} \\
     e.& maˈsâna &  `apple' & ma\textbf{nz}ana \\
     f.& βiˈrînko  & `American' & gri\textbf{ng}o\\
     g.& raˈsîska \textasciitilde ranˈsîska & `Francisca' & Fransi\textbf{sc}a\\
     h.& kapoˈsâanto-ʧ͡i &`graveyard' & camposa\textbf{nt}o \\
     i.& piriˈnânto & `Fernando' & Ferna\textbf{nd}o \\
     j.& loˈβêrto &  `Roberto' & Robe\textbf{rt}o \\
\end{tabular}
    \z

While the underlying syllable structure of Choguita Rarámuri is CV (with optional glottal stop codas in the first syllable of the Prosodic word), stress-based vowel deletion yields consonant clusters in the surface form (§\ref{subsubsec: stress-based vowel reduction and deletion}). However, the distribution of CC clusters in surface form is not homogeneous: being a suffixing language with predominance of disyllabic roots, a highly frequent pattern of morphologically complex words in Choguita Rarámuri is the existence of post-tonic consonant clusters at stem-suffix boundaries. Thus, the lower frequency of repaired consonant clusters in post-tonic position in \ili{Spanish} loanwords is related to this asymmetric distribution of consonant clusters in native Choguita Rarámuri words.

Finally, some of the loanword examples above exhibit long vowel sequences (e.g., (\ref{ex: pre-tonic syllable truncation in Spanish lopanwords}c), (\ref{ex: loanword compounds}e) and (\ref{ex: optional consonant cluster repair}d)). As discussed in \chapref{chap: phonology}, \chapref{chap: prosody} and §\ref{ex:the role of quantity in CR} above, there is no contrastive vowel length in Choguita Rarámuri, but there are phonological phenomena that are quantity-sensitive and surface long vowel sequences that are derived through morpheme-specific effects (including suffix-triggered vowel lengthening, §\ref{subsubsec: past passive-conditioned lengthening} and compensatory lengthening, §\ref{subsubsec: compensatory lengthening}). Further examples of loanwords with long vowels are provided in (\ref{ex: long vowel sequences in loanwords}).\footnote{Some of these loanwords also exhibit additional segmental adaptation, including final V epenthesis in (\ref{ex: long vowel sequences in loanwords}c).}\textsuperscript{,}\footnote{A ritual dancer.}

\ea\label{ex: long vowel sequences in loanwords}
{Long vowel sequences in loanwords}

\begin{tabular}{lllll}
    & \textit{Loanword} & \textit{Translation} & \textit{Source word} & \textit{Source}\\
     a.& neˈrâali &   `generals' & generales \\
     b.& awaˈsîiri & `sheriffs' & alguaciles\\
     c.& koˈrâara    &  `corrals' & corral \\
     d.& aˈrîin-ʧ͡i & `flour' &{harina} \\
     e.& moˈrîin-ʧ͡i & `mill' & molino & \corpuslink{tx60[01_212-01_250].wav}{BFL tx60:1:21.2} \\
     f.& moˈrâal-ʧ͡i &`woven bag' & morral & \corpuslink{tx60[01_482-01_535].wav}{BFL tx60:1:48.2}\\
     g.& moˈnâarko & `monarch' & monarca\\
     h.& siˈrâantro & `cilantro' & cilantro \\
\end{tabular}
    \z

Long vowels in Choguita Rarámuri loanwords are variably attested. This particular pattern may reflect durational properties of \ili{Spanish} stressed vowels, as discussed in \citet{chavez2007loanword}. Specifically, long vowels in Choguita Rarámuri loanwords match “semi-long” \ili{Spanish} stressed vowels, attested in words with penultimate stress before a voiced consonant and in words with final stress before [l]. The phonetic durational properties of \ili{Spanish} stressed vowels and summarized in \tabref{fig: phonetic duration of stressed vowels in Spanish}.

%\pagebreak

\begin{table}
%\includegraphics[width=\textwidth]{figures/OtherSuprasegmental_img1.png}
\begin{tabularx}{\textwidth}{lQl}
\lsptoprule
& \textbf{Contexts} & \textbf{Examples}\\
\midrule
Long & Final stress (except in words with [n] or [l] coda) & \makecell[tl]{\textit{papá} /pa.\textquotesingle pa/\\
\textit{matar} /ma.\textquotesingle tar/}\\
\tablevspace
Semi-ling & Final stress with final [n] or [l]; penultimate stress in open syllables & \makecell[tl]{\textit{corazón} /ko.ra.\textquotesingle son/\\
\textit{pasa} /\textquotesingle pa.sa/}\\
\tablevspace
& Penultimate stress in open syllable with following voiced C & \textit{bala} /\textquotesingle ba.la/\\
\tablevspace
Short & Penultimate stress in closed syllable and antepenultimate stress & \makecell[tl]{\textit{alto} /\textquotesingle al.to/\\
católico /ka.\textquotesingle to.li.ko/}\\
\tablevspace
& Penultimate stress in open syllable with following voiceless C & \textit{bote} /\textquotesingle bo.te/\\
\lspbottomrule
\end{tabularx}
\caption{
\label{fig: phonetic duration of stressed vowels in Spanish}
Phonetic duration properties of {Spanish} stressed syllables (adapted from \citealt{chavez2007loanword})}
\end{table}

Most loanwords in Choguita Rarámuri with long vowel sequences also have a post-tonic consonant cluster, derived through post-tonic vowel deletion. As in the native vocabulary, deletion deletion of the post-tonic vowel triggers lengthening of a preceding syllable’s stressed vowel (for details about this process, see §\ref{subsubsec: compensatory lengthening}). Examples of this pattern in loanwords are shown in (\ref{ex: long vowel sequences in loanwords}d--h). Examples of this pattern in native words is shown in (\ref{ex: long vowels in loanwords compensatory lengthening}) below.

\ea\label{ex: long vowels in loanwords compensatory lengthening}
{Long vowel sequences in native words and post-tonic consonant clusters}

    \ea[]{
    [raˈpáampo]\\
    /raˈpá-na-po/\\
    tear.\textsc{appl-tr-fut.pl}\\
    `We will tear it.'\\
    `Vamos a romperlo.'\\
}
		\ex[]{
		[roˈmíinpo]\\
        /romí-na-po/\\
        fold-\textsc{tr.fut.pl}\\
        `We will fold it.'\\
        `Vamos a doblarlo.'\\
	}
	        \ex[]{
	        [ramuˈwéeltʃ͡ane]\\
	        /ramuwéli-tʃ͡ane/ \\
	        play.with.in.laws-\textsc{ev}\\
	        `It sounds like they are playing with their in laws.'\\
	        `Se oye que están vacilando con sus cuñados.'\\
	   }
    \z
\z

In sum, the following properties characterize loanword prosodic adaptation in Choguita Rarámuri:

\ea\label{ex: summary of loanword patterns}
    \ea[]{
    The \ili{Spanish} source word stressed syllable is incorporated as left aligned prominence with stress-accent and a single tonal melody, HL.\\
}
        \ex[]{
        There are exceptionless adaptation patterns that repair loanwords to comply with high ranked prosodic requirements (e.g., stress must be within a three-syllable stress window, prosodic words must be vowel final).\\
    }
            \ex[]{
            There are variable adaptation processes related to frequent/canonical prosodic structures of Choguita Rarámuri (e.g., consonant cluster repairs depending on stress location).\\
        }
                \ex[]{
                There are variable adaptation processes that reflect the acoustic properties of \ili{Spanish} source words (e.g., stressed semi-long \ili{Spanish} vowels that are interpreted as phonemically long).\\
            }
    \z
\z
