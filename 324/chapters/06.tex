\chapter{Tone and intonation}
\label{chap: tone and intonation}

This chapter addresses the Choguita Rarámuri lexical tone and intonation systems. All documented \ili{Uto-Aztecan} languages are described as having stress,\footnote{\ili{Proto-Uto-Aztecan} is reconstructed as having a weight-sensitive, rhythmic stress with left alignment (\citealt{munro1977towards}); the synchronic stress systems of daughter languages exhibit are widely diverse, though they exhibit a predominance of left alignment (\citealt{caballero2011behind}; \chapref{chap: word prosody}).} with tonogenesis being independently innovated in several branches, predominantly in the US Southwest/Northern Mexico area (\citealt{caballero2020oxford}).\footnote{The diachronic development of tone has been linked to the loss of laryngeal features in \ili{Hopi} (\citealt{manaster1986genesis}), \ili{Northern Tepehuan} (\citealt{shaul2000comparative}), and varieties of \ili{Balsas Nahuatl} (\citealt{guion2010word}).} In addition to Choguita Rarámuri, other \ili{Uto-Aztecan} languages that have been described as having both stress and tone in their word prosody include \ili{Yaqui} (\ili{Cahitan}; \citealt{demers1999prominence}), \ili{Mayo} (\ili{Cahitan}; \citealt{hagberg1989floating}, \citealt{hagberg1993autosegmental}), \ili{Northern Tepehuan} (\ili{Tepiman}; \citealt{bascom1959tonomechanics}, \citealt{woo1970tone}), \ili{Cora} (\ili{Corachol}; \citealt{McMahon-1967}), \ili{Huichol} (\ili{Corachol}; \citealt{grimes1959huichol}), \ili{Hopi} (\citealt{manaster1986genesis}) and \ili{Balsas Nahuatl} (Aztecan; \citealt{guion2010word}).\footnote{Recent research argues that some varieties previously analyzed as being tonal exhibit only stress: \citet{reyes2014fonologia} explicitly rejects a tonal analysis of \ili{Santa María de Ocotán Tepehuan} (\ili{Southern Tepehuan}; \ili{Tepiman}). Work on tone and prosodic documentation more generally in the language family has not yet addressed dialect variation across the different branches.}

All tonal \ili{Uto-Aztecan} languages have a restricted distribution of tonal contrasts to stressed syllables. Restricted tone systems as documented in these languages are similar in some regards to other languages described as `pitch accent' in the literature (\citealt{hyman2006word}, \citealt{van2011pitch}). In this grammar, I assume Choguita Rarámuri is better characterized as having both stress and tone as independent systems and that pitch is part of the lexical realization of at least some morphemes, as defined for tonal languages cross-linguistically (\citealt{yip2002tone}; \citealt{hyman2001tone}; \citealt[][229]{hyman2006word}. Choguita Rarámuri is the only \ili{Uto-Aztecan} language documented to date to feature a three-way tone contrast as opposed to a binary one, suggesting its tone system involves a recent innovation.

Choguita Rarámuri deploys f0 in its intonation system, resulting in different accommodation strategies when the lexical phonology, the morphology and the intonational phonology assign conflicting tones to the same target tone bearing units. Intonation has been understudied for \ili{Uto-Aztecan} languages: prior work on intonation in the language family includes studies on \ili{Nahuatl} (\ili{Aztecan}; \citealt{guion2010word}, \citealt{patino2014intonation}, \citealt{aguilar2020phonology}) and \ili{Northern Tepehuan} (\ili{Tepiman}; \citealt{burgoinmarcacion}). This chapter provides a basic description of intonation patterns in declarative and interrogative sentences of this language. Declaratives feature optional H\% boundary tones and the presence of optional pitch targets (`rhythmic lead tones') preceding each of the lexical tones: a low target before high and falling tones, and a high target before low tones (\citealt{garellek2015lexical}, \citealt{kubuzono2020raramuri}). Interrogatives are intonationally characterized by H\% boundary tones and raised register across the utterance.

While tone also encodes morphosyntactic information, this chapter addresses only the phonologically predictable aspects of tone in this language, saving the description and analysis of its morphological contribution to the chapters devoted to the verbal morphology (\chapref{chap: verbal morphology}) and nominal morphology (\chapref{chap: nominal morphology}). Interactions between lexical, morphological and phrasal tones are addressed in Chapter ~\ref{chap: prosody}.

\section{Tone}
\label{sec: tone}

\subsection{Tonal inventory}
\label{subsec: tonal inventory}

The Choguita Rarámuri tone system involves a three-way lexical contrast between HL, L and H tones.\footnote{H tones are described as M in \citet{caballero2015tone}.}  The HL lexical tone is a phonological unit, a contour tone that is not decomposable into two tonal primitives.

Tone is not conditioned by voicing or laryngeal setting of the preceding consonant, nor any other phonological features. Lexical tonal contrasts in this language are associated exclusively with surface stressed syllables, i.e., there is only one lexical tone per prosodic word and stressless syllables lack lexical tone (see §\ref{sec: defining the prosodic word and other prosodic domains in CR} for discussion of criteria that define prosodic words in the language). Stressed syllables may bear any of the three lexical tones. Tone is also obligatory, i.e., there are no toneless words.

The tone-bearing unit is the mora: falling tones have their high target on the stressed syllable, with the fall starting in the tonic and continuing through a post-tonic syllable, if there is one. H tones may spread their high f0 to the post-tonic syllable (\citealt{caballero2015tone}, \citealt{garellek2015lexical}, \citealt{kubuzono2020raramuri}). Figure \ref{fig: lexical tone f0 trajectories} shows the f0 trajectories of the three lexical tones in stressed syllables.

\begin{figure}
\includegraphics[width=\textwidth]{figures/ToneIntonation-img1.png}
\caption{
\label{fig: lexical tone f0 trajectories}
Mean f0 tracks (in semitones with base 100 Hz for male speakers, with base 200 Hz for female speakers) over utterance-medial tonic vowels, averaged over vowel thirds (\citealt{kubuzono2020raramuri}). }
\end{figure}

As illustrated in Figure \ref{fig: lexical tone f0 trajectories}, lexical tonal contrasts are acoustically encoded both through pitch height and slope. There are relatively small differences in terms of f0 between the three lexical tones in tonic syllables: \citet{caballero2015tone} report an average difference of 2 semitones between L and H tones, while \citet{kubuzono2020raramuri} report a 3 semitone difference between L and H tones (illustrated here). There are also some speaker differences, with some speakers showing even narrower f0 distinctions between certain tones.\footnote{It should be noted, however, that the data examined for both studies involved differences in terms of the size of the corpus, number of speakers and types of examples recorded.} These pitch differences are among the smallest compared to pitch differences for lexical tonal contrasts in other languages with similar tone inventories (\citealt{alexander2010theory}, \citealt{silva2006acoustic}).\footnote{ Other languages that have been reported with similar narrow pitch differences include \ili{Edo}, a \ili{Volta-Niger} language spoken in Nigeria with a binary tone system and \ili{Hausa} (\ili{Chadic}), a language with a three-way tone contrast with H, M and HL tones (\citealt{maddieson1979tone}). In the Americas, \citet{guion2010word} report narrow pitch differences in the realization of the two-tone system recently developed in \ili{Balsas Nahuatl}.}

Despite the narrow f0 differences, lexical tonal categories are reliably distinguished by f0 alone and consistent across speakers in the realization of pitch of stressed syllables. Furthermore, as reported in \citet{garellek2015lexical}, \citet{kubuzono2020raramuri} and discussed below, lexical tonal contrasts are further differentiated by f0 changes from the pre-tonic to the tonic and post-tonic syllables (§\ref{subsec: optional rhythmic lead tones}) and by additional acoustic parameters in different phrase positions (§\ref{subsec: non-tonal encoding of intonation}).

\subsection{Tonal (near-)minimal pairs}
\label{subsec: minimal pairs}

The (near) minimal pairs in (\ref{ex: tone near-minimal pairs}) show that tone is contrastive in Choguita Rarámuri in both nouns and verbs,\footnote{The audio examples illustrating verb forms in the set in (\ref{ex: tone near-minimal pairs}) may be inflected and/or provided within a phrase.} in monosyllabic roots (\ref{ex: tone near-minimal pairs}a--f), disyllabic roots with second-syllable stress (\ref{ex: tone near-minimal pairs}g--l), and disyllabic roots with first syllable stress (\ref{ex: tone near-minimal pairs}m--n) (a circumflex accent indicates a falling HL tone <ô>, a grave accent indicates a L tone <ò>, and an acute accent indicates a H tone <ó>). Only lexical tones are represented in the phonemic transcription.\footnote{As discussed in §\ref{sec: intonation} and \chapref{chap: prosody}, there is evidence of H\% boundary tones and post-lexical tonal targets associated with lexical tones in declarative intonation, as well as non-modal phonation and lengthening of specific lexical tones in phrasal boundaries.}

\ea\label{ex: tone near-minimal pairs}
{Tone (near-)minimal pairs}

\begin{tabularx}{.9\textwidth}{lQlll}
       & \textit{Lexical tone} & \textit{Form} & \textit{Gloss} & \textit{Source}\\
    a. & HL & \textit{tô}&‘to bury' & \corpuslink{el1240[03_295-03_300].wav}{MAF el1240:3:29.5}\\
    b. & L & \textit{tò} & ‘to take' &\corpuslink{el1274[01_261-01_269].wav}{JLG el1274:1:26.1}\\
    c. & HL & \textit{mê} & `to win' & \corpuslink{el1242[01_563-01_570].wav}{MAF el1242:1:56.3}\\
    d. & H & \textit{mé} & `to bring' & \corpuslink{el83[01_179-01_195].wav}{SFH el83:1:17.9}\\
    e. & L & \textit{mè} & ‘mezcal’ & \corpuslink{co1140[02_066-02_087].wav}{MDH co1140:2:06.6}\\
    f. & H & {\textit{pá}} & {‘to throw’}& \corpuslink{el1240[02_468-02_477].wav}{MAF el1240:2:48.8}\\
    g. & L & {\textit{pà}}&{‘to bring’}& \corpuslink{el728[02_479-02_487].wav}{BFL el728:2:44.9}\\
    h. & HL & {\textit{kolî}}&{‘chile pepper’}& \corpuslink{el728[05_588-05_599].wav}{BFL el728:5:11.6}\\
    i. & L & {\textit{kolì}}&{`spatial root’}& \corpuslink{el549[03_527-03_544].wav}{SFH el549:3:52.7}\\
    j. & HL & {\textit{isî}}&{‘to urinate’} &  \corpuslink{el728[04_018-04_027].wav}{BFL el728:03:24.8}\\
    k. & L & {\textit{isì}}&{‘to do’} & \corpuslink{el728[03_497-03_506].wav}{BFL el728:3:53.8}\\
    l. & H & {\textit{niʔwí}}& {‘to be lightning’}& \corpuslink{co1237[10_527-10_538].wav}{JLG co1237:10:52.7}\\
    m. & L & {\textit{niwì}}&{‘to marry’} & \corpuslink{el1318[28_549-28_565].wav}{MFH el1318:28:54.9}\\
    n. & HL & {\textit{nôtʃ͡a}}&{‘pretentious'} & \corpuslink{el728[08_035-08_044].wav}{BFL el728:8:03.5}\\
    o. & L & {\textit{nòtʃ͡a}}&{‘hard working'} & \corpuslink{el728[07_417-07_425].wav}{BFL el728:7:41.7}\\
\end{tabularx}

\z

The examples in (\ref{ex: tone near-minimal pairs}) show examples of tone in monosyllabic and disyllabic roots with different stress properties. Tonal contrasts are also realized on the stressed syllable of trisyllabic roots, as shown in (\ref{ex: tonal contrasts in trisyllabic roots}).

%\break

\ea\label{ex: tonal contrasts in trisyllabic roots}
{Tonal contrasts in trisyllabic roots}

\begin{tabularx}{.9\textwidth}{lQlll}
       & \textit{Lexical tone} & \textit{Form} & \textit{Gloss} & \textit{Source}\\
    a. & HL & {\textit{napaˈbû}}&{‘to get together’}&\corpuslink{in243[04_168-04_180].wav}{FLP in243:4:16.8}\\
    b. & L & \textit{sikiˈrè}&{‘to cut’}&\corpuslink{el1080[13_377-13_399].wav}{SFH el1080:13:37.7}\\
    c. & L & \textit{naʔˈsòwa}&{‘to stir, mix’}&\corpuslink{co1234[08_419-08_430].wav}{JLG co1234:8:41.9}\\
    d. & HL & \textit{ˈhûmisi}&{‘to take off, \textsc{pl}’} &\corpuslink{tx19[01_403-01_442].wav}{LEL tx19:1:40.3}\\
    e. & H & \textit{wipiˈsó}& `to hit with stick'&\corpuslink{el1080[03_558-03_577].wav}{SFH el1080:3:55.8}\\
    f. & H & \textit{aˈnátʃ͡a}&`to endure'&\corpuslink{el1318[19_581-19_598].wav}{MFH el1318:19:58.1}\\
\end{tabularx}

\z

\subsection{Tone patterns by root type and stress position}
\label{subsec: tone melodies by root type and stress position}

As described in §\ref{sec: stress properties of roots, stems and suffixes}, Choguita Rarámuri roots can be characterized as lexically stressed or unstressed based on their stress properties in morphologically complex words. Both lexically stressed and lexically unstressed roots may have lexical tone realized in the surface tonic syllable. Unstressed roots may bear either H or L tone when bare or in neutral contexts, which can be attributed to a lexical tonal specification. Stressed roots, on the other hand, may bear HL, H or L tone, i.e., there are no unstressed HL-toned roots in the language.

From the logically possible lexical tone melodies by root type (monosyllabic, disyllabic and trisyllabic) and stress position (first, second and third syllable stress), two patterns are not yet attested, namely H and L-toned trisyllabic roots with first syllable stress. This gap in the tonal melody inventory is likely due to the combined effect of the relative infrequency of both trisyllabic roots and first-syllable stress (from a corpus of 1040 roots, only 15 (1.5\%) are trisyllabic roots with first-syllable stress). It is likely that the missing tonal contrasts will be found upon further investigation.

As discussed in \chapref{chap: prosody}, tonal melodies show systematic distributions in terms of phonological characteristics (whether a stem contains a lexically stressed or unstressed root), as well as morphological factors (the type of morphological constructions involved in inflected words).

\subsection{Stress-based tonal neutralization}
\label{subsubsec: stress-based tonal neutralization}

As discussed in §\ref{sec: stress properties of roots, stems and suffixes}, stress distribution is governed by morphological factors. Given that tone distribution is dependent on stress, stress shifts result in tonal alternations in morphologically complex words. Specifically, morphologically-conditioned stress shifts result in tone neutralization patterns, originally described in \citet{caballero2015tone}. There are two possibilities in terms of surface tone in contexts where stress shifts. If a stress-shifting suffix is stressed after a stress shift, the stressed suffix syllable will bear the lexical tone of that suffix (this is the case when a monosyllabic or disyllabic unstressed root attaches a stress-shifting suffix). This pattern is shown in the examples provided in \tabref{tab:suffix-tones}.

\begin{table}
\caption{Suffix lexical tone after stress shift}
\label{tab:suffix-tones}

\begin{tabularx}{\textwidth}{XXll}
\lsptoprule
\textbf{Tone} & \textbf{Stem}  & \textbf{Gloss}&\textbf{Source}\\
\midrule
L  &     ˈtò-a     &   take-\textsc{prog} & \corpuslink{co1136[08_457-08_484].wav}{MDH co1136:8:45.7}\\
HL &    to-ˈkâ &  take\textsc{-imp.sg}    & \corpuslink{tx152[10_368-10_422].wav}{SFH tx152:10:36.8}\\
H &   raˈhá-li  & light.fire-\textsc{pst}   & 	< LEL  el1907 > \\
HL   &  raha-ˈkâ   &    light.fire-\textsc{imp.sg} & < LEL el1907 >\\
L & ˈtò-li   &  take-\textsc{pst} & \corpuslink{tx84[07_152-07_161].wav}{LEL tx84:7:15.2}\\
L    & to-ˈsì   &  take-\textsc{imp.pl} & \corpuslink{el505[03_483-03_507].wav}{SFH el505:3:48.3}\\
H    & kiˈmá-li    & put.on.blanket-\textsc{pst}  &  < BFL el1909 >\\
L    & kimi-ˈsì    &  put.on.blanket-\textsc{imp.pl}  & < BFL el1909 >\\
H    & uˈkú-li    &  to.rain-\textsc{pst} & < LEL el1918 >\\
H    & uku-ˈnále    &  to.rain-\textsc{desid}  &  < LEL el1918 >\\
H    & kiˈmá-li     &  put.on.blanket-\textsc{pst} & < BFL el1909 >\\
H    & kimi-ˈnále    &  put.on.blanket-\textsc{desid} & < BFL el1909 >\\
\lspbottomrule
\end{tabularx}
\end{table}

What these examples show is that suffixes bear their underlying tone when stressed, e.g., HL in the imperative singular \textit{-ka}̂ (\textit{to-ˈkâ} `take it!' and \textit{raha-ˈkâ} `light it up!' in \tabref{tab:suffix-tones}), L in the imperative plural \textit{-sì} (\textit{to-ˈsì} `you all take it!' and \textit{kimi-ˈsì} `cover yourselves in blankets!' in \tabref{tab:suffix-tones}), and H in the desiderative \textit{-nále} (\textit{uku-ˈnále} `it is about to rain' and \textit{kimi-ˈnále} `s/he wants to cover themselves in a blanket' in \tabref{tab:suffix-tones}). The lexical tone of the root, which surfaces when the root attaches neutral morphological contexts, is deleted after the stress shift.

\hspace*{-6pt}If a trisyllabic unstressed root attaches a stress-shifting suffix, the newly stressed syllable will be a stem syllable. A newly stressed stem syllable after a stress shift will bear a HL tone in these contexts, regardless of what the lexical tone of the root is (H tone in \textit{roʔˈsówa} ‘cough’ or L tone in \textit{naʔˈsòwa} ‘stir’ in \tabref{tab:stem-tone-stress-shift}).

\begin{table}
\caption{Stem tone after stress shift}
\label{tab:stem-tone-stress-shift}

\begin{tabularx}{\textwidth}{lXXl}
\lsptoprule
\textbf{Tone} & \textbf{Stem}  & \textbf{Gloss} & \textbf{Source}\\
\midrule
H  &     roʔˈsówa-a      &   cough-\textsc{prog} & 	\corpuslink{el261[11_241-11_254].wav}{SFH,MGD el261:11:24.1}\\
HL &    roʔsoˈwâ-ma & cough-\textsc{fut.sg}   &  \corpuslink{el262[00_070-00_086].wav}{SFH,MGD el262:0:16.8}\\
H &  roʔˈsówa-li   & cough-\textsc{pst} & 	< LEL el2060 >    \\
HL   &  roʔsoˈwâ-si   &  cough-\textsc{imp.pl} & 		< LEL el2060 >\\
L & naʔˈsòwa-li  &  stir-\textsc{pst} &  \corpuslink{co1239[06_125-06_136].wav}{JLG co1239:6:12.5}\\
HL    & naʔsoˈwâ-ma   & stir-\textsc{fut.sg} & \corpuslink{co1234[08_454-08_469].wav}{JLG co1234:8:45.4}\\
L   & naʔˈsòwa-i    & stir-\textsc{impf} &  < BFL e1957 >  \\
HL    & naʔsoˈwâ-bo    & stir-\textsc{fut.pl} & \corpuslink{el658[07_205-07_225].wav}{BFL el658:7:20.5} \\
\lspbottomrule
\end{tabularx}
\end{table}

%edit the following

One important aspect of this second pattern is that the surface tonal pattern of these morphologically complex words that have undergone a stress shift is not predictable based on the lexical tonal properties of root morphemes nor the lexical tones of suffixes. An example discussed in \citet{caballero2021grammatical} is that of unstressed roots attaching the imperative plural \textit{-sì} suffix. As shown in \tabref{tab:suffix-tones} above, this suffix bears its lexical L tone when stressed (e.g., \textit{to-ˈsì} `you all take it!'), but it will not surface if the stressed syllable after a stress shift is a stem syllable, i.e., the hypothetical forms \textit{*rosoˈwà-si} (vs. attested \textit{rosoˈwâ-si} `you all cough!') and \textit{*naʔsoˈwà-si} (vs. attested \textit{naʔsoˈwâ-si} `you all stir it') in \tabref{tab:stem-tone-stress-shift}, with L tone when attaching imperative plural \textit{-sì} are unattested.

This tonal pattern is analyzed in \citet{caballero2021grammatical} as resulting from a process of HL tone insertion after a stress shift has deleted the root's lexical tone. This analysis is based on the following assumptions (listed in (\ref{ex: assumptions for analysis of grammatical tone})):

\ea\label{ex: assumptions for analysis of grammatical tone}
{HL tone insertion in stress-based tone neutralization}

\begin{itemize}
\item Each morpheme has one and only one tone, which is lexically associated with one and only one Tone-Bearing-Unit (a mora within the tonic syllable)
\item In words containing lexically unstressed roots and neutral morphological constructions, stress is assigned within the root; the stressed syllable bears the underlying lexical tone of these roots
\item Stress shifts in shifting environments cause lexical root tones to delete
\item If the newly stressed syllable after a stress shift is a stem syllable, it is toneless and acquires an HL tonal melody
\end{itemize}
\z

It could be argued that this HL tone insertion process is also at play in \ili{Spanish} loanwords (\citealt{caballero2013procesos}; see also §\ref{subsec: exceptionless prosodic loanword adaptation patterns} and \chapref{chap: prosody}). Loanwords from \ili{Spanish} are incorporated into Choguita Rarámuri with faithful prominence to the stress location of the source words.\footnote{As
    discussed in §\ref{sec: Spanish loan nouns}, nouns borrowed from \ili{Spanish} are sometimes incorporated into the Choguita Rarámuri lexicon with the \textit{-tʃ͡í} suffix, which has locative semantics in other contexts. This morphological strategy for loanword nativization is shown in (\ref{ex: tone in loanwords from Spanish}g--h).
} The stressed syllable in loanwords has a HL tone (no exceptions have yet been documented to this pattern). Relevant examples are given in (\ref{ex: tone in loanwords from Spanish}) (the \ili{Spanish} source words are provided in their orthographic form, where boldface indicates the stressed syllable):\footnote{As shown in (\ref{ex: tone in loanwords from Spanish}g--h), some loanwords in Choguita Rarámuri have the same meaning but different source words: in the case of `bottle', one loanword has as its source word \textit{limeta}, an archaic word no longer in use in Northern Mexican \ili{Spanish}, which coexists with the loanword originating from `bottle', \textit{botella}, a contemporary \ili{Spanish} term.}

\ea\label{ex: tone in loanwords from Spanish}
{Tone in loanwords from Spanish}

\begin{tabular}{lllll}
       & \textit{Loanword} & \textit{Source word} & \textit{Gloss} &\\
    a. & {[toˈmâʃi]}&{To\textbf{más}}&‘Thomas’ \\
    b. & {[maˈsâna]}&{man\textbf{za}na}&‘apple’ \\
    c. & [ˈsâbaɾu] &{\textbf{sá}bado }&‘Saturday’&\corpuslink{el1318[15_224-15_237].wav}{MFH el1318:15:22.4}\\
    d. & {[ˈhuân]}&{\textbf{Juan}}&‘John’&\corpuslink{in61[04_449-04_473].wav}{FLP in61:4:44.9}\\
    e. & {[saˈhuâni]}&{San \textbf{Juan}}&‘John’&\corpuslink{tx109[00_531-00_557].wav}{LEL tx109:0:53.1}\\
    f. & {[raˈniêli]}&{Da\textbf{niel}}&‘Daniel’\\
    g. & {[liˈmêta-ʧ͡i]}&{li\textbf{me}ta}&‘bottle’\\
    h. & [boˈtêja-ʧ͡i]&bo\textbf{te}lla&`bottle'&\corpuslink{tx191[00_299-00_319].wav}{BFL tx191:0:29.9}\\
    i. & {[basaˈlôa]}&{pa\textbf{sear}}& `to stroll'&\corpuslink{tx84[00_538-00_554].wav}{LEL tx84:0:53.8}\\
    j. & [ˈkûɾsi]&\textbf{cruz}&`cross'&\corpuslink{co1234[13_132-13_147].wav}{JLG co1234:13:13.2}\\
    k. & [toˈɾôka]&\textbf{tro}ca&`truck'&\corpuslink{co1136[08_153-08_181].wav}{MDH co1136:8:15.3}\\
    l. & [ˈsôpa]&\textbf{so}pa&`soup'& \corpuslink{co1136[00_255-00_282].wav}{MDH co1136:0:25.5} \\
    m. & [ˈjêɾbas] & \textbf{hier}bas & `herbs'& \corpuslink{tx785[02_220-02_257].wav}{GFM tx785:2:22.0}\\
    n. & [koˈrêaka] & co\textbf{rrea}& `strap' & \corpuslink{tx785[01_125-01_167].wav}{GFM tx785:1:12.5}\\
\end{tabular}
    \z

The analysis involving a HL tone insertion process for loanwords would involve the assumption that \ili{Spanish} loanwords are lexically stressed but toneless in Choguita Rarámuri. Alternatively, the tonal properties of loanwords may be analyzed as involving a reinterpretation of the acoustic properties of Mexican \ili{Spanish} prominence, which has been argued to include a H* pitch accent in the stressed syllable in certain intonational contexts (focus-marked words in declarative sentences) (\citealt{prieto1995tonal}).

In sum, morphological factors condition stress shifts. Given the dependency tone has on stress for its distribution, the tonal alternations resulting in these contexts are largely predictable based on the lexical tonal properties of the morphemes that make up a morphologically complex word. Further discussion on the mechanism of tonal patterns in morphologically complex words is provided in \chapref{chap: prosody}.

\section{Intonation}
\label{sec: intonation}

This section presents a basic description of the intonation patterns of declarative and interrogative sentences in Choguita Rarámuri and the realization of lexical tones in different phrasal contexts. The description of declarative intonation presented here builds on results discussed in \citet{caballero2014tone}, \citet{garellek2015lexical}, and \citet{kubuzono2020raramuri}, based on qualitative and quantitative analysis of instrumental data recorded with four native Choguita Rarámuri speakers (two male, two female) using controlled elicitation.\footnote{The role of f0, duration and voice quality in intonation was assessed in these studies through examination of a corpus of data recorded with Choguita Rarámuri speakers. In this corpus, lengths of words and phrases, location of lexical stress and syntactic structure of phrases were manipulated in order to assess the timing and sequencing of tonal (lexical and post-lexical) patterns. Targets were balanced for lexical tone and stress, as well as utterance context (medial vs. final) and elicited in carrier phrases from a written prompt.} Interactions between lexical tone, grammatical tone and intonation in this language are further addressed in \chapref{chap: prosody}. The intonational characteristics of interrogative constructions are addressed in §\ref{sec: interrogative intonation} and \chapref{chap: sentence types}.

%All recorded sentences were declaratives with intended broad focus.

%The measures assessed were f0 and duration (using VoiceSauce \parencite{shue2013voicesauce}, as well as Contact Quotient (CQ), an EGG measure of glottal contact (measured using EggWorks).

This description assumes intonational primitives as laid out in the autosegmen\-tal-metrical framework (\citealt{pierrehumbert1980phonology}, \citealt{beckman1986intonational}, \citealt{ladd1986intonational}). These assumptions include the following:

\ea\label{ex: assumptions for analysis of CR intonation}
{Assumptions from Austosegmental-metrical framework}

\begin{itemize}
\item Sequences of tonal targets (T) on an autosegmental tier yield intonational contours
\item There are two kinds of tonal targets: (i) tones associating with stressed syllables usually marking focal information or \textit{pitch accents} (T*) and (ii) tones associating with the edges of phrasal constituents or \textit{boundary tones} (T\%)
\item Tonal sequences are licensed by different phonological domains, arranged in a prosodic hierarchy (\citealt{selkirk1986derived}, \citealt{nespor1986prosodic})
\end{itemize}

\z

\hspace*{-3pt}In Choguita Rarámuri declarative intonation there is evidence of (i) a H\% boundary tones, (ii) `lead' (rhythmic) tones associated with lexical tones, and (iii) general and tone-specific non-tonal devices that encode intonation, including vowel rearticulation and lengthening at phrasal boundaries, all of which exhibit both inter- and intra-speaker variation. As discussed in \chapref{chap: sentence types}, interrogative constructions also involve register manipulation.

The declarative sentences examined all involved canonical word order with intended broad focus. As described in \chapref{chap: basic clause types}, Choguita Rarámuri is a head-final language with SOV word order. Examples of this structure are provided in (\ref{ex: canonical SOV word order}), with a clause headed by a ditransitive predicate with both pronominal and NP arguments (\ref{ex: canonical SOV word ordera}) and by a transitive predicate with pronominal arguments (\ref{ex: canonical SOV word orderb}).

%replace example (b) here with a sentence with full NPs

\ea\label{ex: canonical SOV word order}
{Canonical SOV word order in Choguita Rarámuri}

    \ea[]{
    \textit{ˈmò ˈjêla taˈmí haˈré  gaˈjêta ˈàko}\\
    \gll    [ˈmò ˈjê-la]\textsc{\textsubscript{subj}}  [taˈmí]\textsc{\textsubscript{p. obj.}}\textsubscript{} [haˈré  gaˈjêta]\textsc{\textsubscript{s. obj.}} ˈà{}-ki-o\\
            2\textsc{sg.nom} mother-\textsc{poss} \textsc{1sg.acc} some  cookie     give-\textsc{pst.ego-ep}\\
    \glt     ‘Your mom gave me some cookies.’\\
    \glt    `Tu mamá me dio galletas.' < BFL 09 1:89/el >\\
}\label{ex: canonical SOV word ordera}
        \ex[]{
        \textit{baʔaˈrîni ˈmí ˈàma}\\
        \gll    baʔaˈrî=[ne]\textsc{\textsubscript{subj.}} [mí]\textsc{\textsubscript{obj.}} ˈà-ma \\
                tomorrow=\textsc{1sg.nom} \textsc{2sg.acc} look.for-\textsc{fut.sg}\\
        \glt    ‘I’ll look for you tomorrow.’ \\
        \glt    `Mañana te voy a buscar.' < LEL 09 1:70/el >\\
    }\label{ex: canonical SOV word orderb}
    \z
\z

Other Rarámuri varieties are also described as having SOV as their canonical word order, including \ili{Western Tarahumara} \citep{Burgess-1984}, Rarómuri (\ili{Urique Tarahumara}) \citep{valdez2014predication}, \ili{Pahuírachic Rarámuri} \citep{estrada2013complementos} and \ili{Rochéachi Rarámuri} \citep{moralesmoreno2016rochecahi}. It should be noted, however, that SVO order is also frequently attested in Choguita Rarámuri declarative sentences with broad focus.\footnote{Whether this alternative word order results from contact with \ili{Spanish} or other factors is a question I leave for future research.}

\subsection{H\% boundary tones in declarative sentences}
\label{subsec: H boundary tones}

The highest domain in the prosodic hierarchy assumed here is the Intonation Phrase (IP), which corresponds to a clause. The prosodic hierarchy is schematized in (\ref{ex:6:prosodic hierarchy}) (\citealt{selkirk1980prosodic, selkirk1996prosodic, nespor1986prosodic, hayes1989prosodic}).

\pagebreak

\ea\label{ex:6:prosodic hierarchy}
{The prosodic hierarchy}\mbox{}

   Intonational Phrase

     \hspace*{1cm}⎜

   Phonological Phrase

     \hspace*{1cm}⎜

   Clitic Group

     \hspace*{1cm}⎜

   Prosodic Word
   \z

Counter to a strong cross-linguistic trend, where many languages exhibit a downward trend of pitch over the course of an utterance (\citealt{cohen1982declination}, \citealt{ladd1984declination}), Choguita Rarámuri declaratives generally end with a final rise in f0, suggesting a H\% boundary tone at the right edge of the Intonational Phrase (IP). This is shown in \figref{fig: H boundary tone in declaratives} with a sentence composed of words with lexical L tones, \textit{reˈhòi suˈnù oˈhòli} `The man dekerneled corn'. As shown in this figure, H\% boundary tones accommodate lexical tones: both the lexical L pitch target and the post-lexical H pitch target are clearly differentiated.

\begin{figure}
\includegraphics[width=\textwidth]{figures/Phonology-img1.png}
\caption{
\label{fig: H boundary tone in declaratives}
High boundary tone in declaratives \parencite{garellek2015lexical}. Lexical pitch targets are represented with `*' in the first tier; stressed syllables are represented with `S' in the bottom tier.}
\end{figure}

A final rise in pitch at the IP boundary is not attested if the IP contains a lexical HL tone at the right edge: the rise expected with the presence of a H\% is replaced by a pitch fall, i.e., H\% boundary tones are overridden by lexical falling tones. This is shown in \figref{fig: no H boundary tone in declaratives with lexical HL tones} with a sentence composed of words with lexical HL tones, \textit{Maˈnuêli oˈkwâ koˈlî iʔˈkîli} `Manuel bit two chili peppers'. As discussed in \citet{garellek2015lexical}, this effect may enhance the distinction between lexical L and HL tones in phrase final position. All three lexical tones are clearly differentiated for all speakers across different intonational contexts (see further discussion in \chapref{chap: prosody}).


\begin{figure}
\includegraphics[width=\textwidth]{figures/Phonology-img2.png}
\caption{
\label{fig: no H boundary tone in declaratives with lexical HL tones}
No high boundary tone in declaratives with lexical HL tones \parencite{garellek2015lexical}.  Lexical pitch targets are represented with `*' in the first tier; stressed syllables are represented with `S' in the bottom tier.}
\end{figure}

Finally, there is also evidence that sometimes a L\% tone is found at the end of declarative sentences. This is exemplified in \figref{fig: L boundary tone in declaratives with lexical L tones}. It is yet an open question what factors may govern the variability in the boundary tone, including inter-speaker differences or other variables that have not yet been investigated, though the presence of H\% vs. L\% boundary tones does not seem to be conditioned by list intonation.

\begin{figure}
\includegraphics[width=\textwidth]{figures/ToneIntonation-img2.png}
\caption{
\label{fig: L boundary tone in declaratives with lexical L tones}
L\% boundary tone in a declarative sentence with lexical L tones \parencite{kubuzono2020raramuri}. Lexical tones are marked as L* on the stressed syllables.}
\end{figure}

\subsection{Optional rhythmic `lead tones}
\label{subsec: optional rhythmic lead tones}

In addition to the presence of boundary tones and pitch raising at prosodic boundaries, \citet{garellek2015lexical} and \citet{kubuzono2020raramuri} report on optional pitch targets that precede the lexical tonal targets, referred to as `lead' tones in these studies. Lead tones are always an opposite pitch target than their lexical counterpart: if the lexical tone is H or HL, the lead tone involves a low pitch target; if the lexical tone is L, on the other hand, the lead tone involves a high pitch target. Low lead tones of lexical H and HL tones are exemplified in \figref{fig: Lead L tone preceding a lexical H tone} and \figref{fig: Lead L tone preceding a lexical HL tone}, respectively. A high lead tone preceding a lexical L tone is exemplified in \figref{fig: Lead H tone preceding a lexical L tone}; this figure also exemplifies the optional nature of lead tones: a high target is attested prior to the lexical L tone of \textit{suˈnù} `corn', but not prior to \textit{oˈhòli} `threshed'.

\begin{figure}
\includegraphics[width=0.6\textwidth]{figures/Phonology-img3.png}
\caption{
\label{fig: Lead L tone preceding a lexical H tone}
Lead low tone preceding a lexical H tone \parencite{garellek2015lexical}. Lexical pitch targets are represented with `*'' in the first tier; stressed syllables are represented with `S' in the bottom tier.}
\end{figure}

\begin{figure}
\includegraphics[width=0.4\textwidth]{figures/Phonology-img4.png}
\caption{
\label{fig: Lead L tone preceding a lexical HL tone}
Lead low tone preceding a lexical HL tone \parencite{garellek2015lexical}. Lexical pitch targets are represented with `*' in the first tier; stressed syllables are represented with `S' in the bottom tier.}
\end{figure}

\begin{figure}
\includegraphics[width=\textwidth]{figures/Phonology-img5.png}
\caption{
\label{fig: Lead H tone preceding a lexical L tone}
Lead high tone preceding a lexical L tone \parencite{garellek2015lexical}. Lexical pitch targets are represented with `*' in the first tier; stressed syllables are represented with `S' in the bottom tier.}
\end{figure}

Another property of lead tones is that they are variable in their alignment: while they tend to occur in the pre-tonic syllable, they may be aligned with a syllable prior to the pretonic or the beginning of the tonic syllable \parencite{garellek2015lexical}. In contrast, lexical tones are always consistently aligned with the tonic syllable and the fall of HL tones is always consistently realized with a post-tonic syllable, if there is one (see also \citealt{caballero2015tone}).

The fact that optional lead tones are dependent on the pitch height of lexical tones which they precede and that H\% boundary tones are not found when a lexical HL tone is at the end of the phrase in order not to override the lexical tone's f0 target suggest that preserving and enhancing the lexical tones of the language are important feature of the tonal grammar of Choguita Rarámuri. As will be discussed in \chapref{chap: prosody}, lexical tones may be replaced by grammatical tones in Choguita Rarámuri, but they appear to resist neutralization in their interaction with post-lexical tones.


\subsection{Intonation patterns of declarative sentences}
\label{subsec: F0 in intonation}

The following subsections report on work published in \citet{kubuzono2020raramuri} and illustrate typical intonation patterns of declarative sentences with sequences of identical tones in contexts intending broad focus.

\subsubsection{Declarative sentences with lexical L tones}
\label{subsubsec: intonation declaratives L}

As shown in \figref{fig: declaratives with lexical L tones}, a sentence with a sequence of lexical L tones will typically exhibit alternating low pitch targets for lexical tones and higher pitch targets for the preceding lead tones. This Figure also exemplifies the final high pitch target associated with the H\% boundary tone which may be realized in the same stressed syllable as the final lexical L tone, creating a rising pitch contour. A lexical L tone target may be realized with a higher pitch lead tone in the same syllable if followed by another lexical L tone as attested in the word \textit{raˈrài} ‘sandals’, where the high pitch target is associated with \textit{tʃ͡aˈbè} `before', the final word in the IP. This example also shows a gradual lowering in f0 of both high-pitched targets of the lead tones as well as the lexical L tones. Thus, sentences with lexical L tones may exhibit declination.\footnote{As discussed in \citet{kubuzono2020raramuri} and §\ref{subsec: H boundary tones} above, there is also evidence that lead tones are optional and that L\% boundary tones may also be attested instead of the typical H\% boundary tone of declaratives (as exemplified in \figref{fig: L boundary tone in declaratives with lexical L tones}).}

\begin{figure}
\includegraphics[width=\textwidth]{figures/ToneIntonation-img3.png}
\caption{
\label{fig: declaratives with lexical L tones}
Intonation of a declarative sentence containing only L lexical tones \parencite{kubuzono2020raramuri}. Lexical tones are marked as L* on the stressed syllables. Non-lexical pitch targets, including the H\% boundary tone and high “lead” tones, are marked on the second tier.}
\end{figure}

As exemplified in \figref{fig: Lead H tone preceding a lexical L tone} and \figref{fig: L boundary tone in declaratives with lexical L tones} above, higher pitch targets associated with lead tones are optional. Thus, as discussed in more detail in \citet{kubuzono2020raramuri}, lexical L tones are only associated with a low pitch target in stressed syllables.

\subsubsection{Declarative sentences with lexical H tones}
\label{subsubsec: intonation declaratives H}

Declarative sentences with sequences of lexical H tones also show an alternating pattern between the high pitch targets associated with lexical tones and preceding lower pitch targets of lead tones. This is exemplified in \figref{fig: declaratives with lexical H tones}. This example also shows the highest pitch peak at the end of the IP associated to the H\% boundary tone.

Furthermore, the example in \figref{fig: declaratives with lexical H tones} shows that H lexical tones in a sequence exhibit a progressive rise in the f0 of the lexical H targets. The low-pitched lead pitch targets may also be upstepped (represented in the figures as “\^  L”), given that they are low with respect to the following lexical H tone, but not the preceding one.

\begin{figure}
\includegraphics[width=\textwidth]{figures/ToneIntonation-img4.png}
\caption{
\label{fig: declaratives with lexical H tones}
Intonation of a declarative sentence containing only H lexical tones \parencite{kubuzono2020raramuri}. Lexical tones are marked as H* on the stressed syllables. Non-lexical pitch targets, including the H\% boundary tone and low “lead” tones, are marked on the second tier.}
\end{figure}

This example also shows the spreading of H tones onto following post-tonic syllables, as in \textit{aˈwíame} ‘dancers', where the H tone spreads from the tonic to the post-tonic syllable. On the other hand, no spreading is attested after \textit{baˈʰt͡ʃí} `zucchini'. Instead, a low lead tone is attested immediately preceding the following H tone in \textit{koˈʔáli} `ate'.

Finally, there is also evidence of the optionality of low lead tones preceding H lexical tones, as exemplified in \figref{fig: optional L lead tones in declaratives with lexical H tones}: there is no low pitch target between the two lexical H tones in \textit{riˈʰtê ˈpáli} `(they) threw stones'.

\begin{figure}
\includegraphics[width=\textwidth]{figures/ToneIntonation-img6.png}
\caption{
\label{fig: optional L lead tones in declaratives with lexical H tones}
Intonation of a declarative sentence containing only H lexical tones with optional low lead tones \parencite{kubuzono2020raramuri}. Lexical tones are marked as H* on the stressed syllables. Non-lexical pitch targets, including the H\% boundary tone and low “lead” tones, are marked on the second tier.}
\end{figure}


\subsubsection{Declarative sentences with lexical HL tones}
\label{subsubsec: intonation declaratives HL}

As discussed in §\ref{subsec: H boundary tones} above, in sentences with sequences of lexical HL tones (represented as H*L, since the low pitch target of this lexical tone may be realized in a post-tonic syllable), there is no evidence of H\% boundary tones. As seen in \figref{fig: declaratives with lexical HL tones}, HL lexical tones may have optional low lead tones; the high pitch target of the last lexical HL tone is higher than the preceding two high pitch targets, and the low pitch targets of the pre-final HL tones exhibit declination. This suggests that HL tones in the final position of the IP exhibit pitch range expansion.

\begin{figure}
\includegraphics[width=\textwidth]{figures/ToneIntonation-img5.png}
\caption{
\label{fig: declaratives with lexical HL tones}
Intonation of a declarative sentence containing only HL lexical tones \parencite{kubuzono2020raramuri}. Lexical tones are marked as H*L on the stressed syllables. Non-lexical pitch targets (low “lead” tones) are marked on the second tier.}
\end{figure}

As discussed in the next section (§\ref{subsec: non-tonal encoding of intonation}), utterance fnal HL tones may be optionally rearticulated.

\subsection{Non-tonal encoding of intonation}
\label{subsec: non-tonal encoding of intonation}

Encoding of intonational contrasts in Choguita Rarámuri also involves non-tonal effects. Non-tonal effects associated with Choguita Rarámuri intonational encoding at prosodic boundaries include: (i) vowel rearticulation and (ii) lengthening of L tones. This section summarizes the findings reported on \citet{caballero2014tone} and \citet{aguilar2015multi}. Controlled elicited data was recorded with four native Choguita Rarámuri speakers (two male and two female), and assessment of the role of lengthening and non-modal phonation assessed through quantitive analysis of acoustic and electroglottographic (EGG) data.\footnote{Lengthening was analyzed using VoiceSauce \parencite{shue2011voicesauce} and phonation was assessed through analysis of CQ (Contact Quotient) calculated in EggWorks.}

Vowel rearticulation is exclusively attested with HL tones at the right edge of the IP boundary. This is an effect that is optional for some speakers but very robust for others. Vowel rearticulation may involve glottal closure or glottalization, with frequent devoicing of the final portion of the vowel (VʔV̥ or VV̰V). An example is shown in \figref{fig: Rearticulation of HL tones}.

\begin{figure}
\includegraphics[width=\textwidth]{figures/Phonology-img6.png}
\caption{
\label{fig: Rearticulation of HL tones}
Rearticulation of HL tones in utterance-final position \parencite{caballero2014tone}}
\end{figure}

% %%please move the includegraphics inside the {figure} environment
% \includegraphics[width=\textwidth]{figures/GrammardraftJuly182017-img6.png}

The following examples show the contrast between rearticulated HL tones and non-rearticulated H and L tones of (near-)minimal pairs elicited in a frame sentence (`\textit{Maˈnuêli riˈwàli oˈkwâ/waʔˈlû} \_' ‘Manuel saw two/a big \_’).


\ea\label{ex: rearticulated vs. non-rearticulated tones}
{Rearticulated vs. non-rearticulated tones in phrase-final position}

    \ea[]{
    \glt    \doublebox{[naˈʰpóʔò]}{{HL}}\\
    \glt    /naˈʰpô/\\
    \glt    `prickly pear’\\
    \glt    `tuna' < BFL 11-nahpo > \\
}
        \ex[]{
        \glt    \doublebox{[oˈʰkó]}{{H}}\\
        \glt    /oˈʰkó/\\
        \glt    `pine tree’\\
        \glt    `pino' < BFL 12-ohko >\\
    }
\pagebreak
            \ex[]{
            \glt    \doublebox{[naˈʔíʔì]}{{HL}}\\
            \glt    /naˈʔî/\\
            \glt    `here’\\
            \glt    `aqui'  < BFL 11-naqi >\\
    }
                \ex[]{
                \glt    \doublebox{[naˈʔì]}{{L}}\\
                \glt    /naˈʔì/\\
                \glt    `fire’\\
                \glt    `fuego' < BFL 12-naqi >\\
            }
    \z
\z

Rearticulation of HL tones is an effect robustly attested for participant female speakers and only optionally attested or marginal for participant male speakers. A systematic study controlling for demographic factors may reveal whether gender plays a role in inter-speaker variation. A follow up study of the role of phonation in tonal encoding in Choguita Rarámuri reported in \citet{kubuzono2020raramuri} reveals that voice quality improves discrimination of lexical tones, though voice quality measures exhibit patterns that are speaker-dependent. Readers are referred to this reference for more details.

There is also evidence for significant intra-speaker variation in the non-tonal encoding of lexical tones in certain prosodic positions. As shown in the following examples, speakers may deploy additional strategies in utterance final position, including voiceless aspiration. In the example in \figref{fig: phasing of creaak aand aspiration}, a rearticulated HL tone exhibits modal phonation, followed by creaky phonation, followed by breathy phonation and then aspiration.

%\break

\begin{figure}
\includegraphics[width=\textwidth]{figures/ToneIntonation-img7.png}
\caption{
\label{fig: phasing of creaak aand aspiration}
Voiceless aspiration utterance-final and phasing of the rearticulated vowel for BFL (female speaker) \parencite{caballero2014tone}.}
\end{figure}

Finally, there is evidence to suggest that an additional non-tonal device in the encoding of Choguita Rarámuri intonation is lengthening: \citet{caballero2014tone} and \citet{aguilar2015multi} report that there is significant lengthening utterance-finally of L tones, where L tones have greater duration than HL and H tones at the right edge of the IP boundary for some speakers. It should be noted, however, that a follow up study did not find a significant role for duration in the encoding of lexical tones in the language using linear discriminant analysis \citep{kubuzono2020raramuri}.

In sum, preliminary research shows Choguita Rarámuri exhibits the following phonation-lexical tone interactions: (i) rearticulation exclusive to HL tones, and (ii) increased lengthening of L tones. Fundamental frequency (f0) and voice quality interact in many languages in the realization of tonal contrasts (\citealt{kingston2005phonetics}, \citealt{kuang2013tonal}) (e.g., glottalization in \ili{Dagbani} (\ili{Gur}; \citealt{hyman1993structure}), and Roman and Tuscan \ili{Italian} (\citealt{di2015glottalization}); breathiness/creakiness in \ili{Chickasaw} (\ili{Muskogean}; \citealt{gordon2005autosegmental}); lengthening in \ili{Bantu} languages (\citealt{downing2008focus,downing2010accent}; \citealt{hyman2011tonal}, among other effects). More detailed examination of the tone-phonation interaction in Choguita Rarámuri will inform a growing body of literature that seeks to understand the interaction of multiple phonetic dimensions in the implementation of prosodic contrasts in a range of typologically diverse languages.

\subsection{Interrogative intonation}
\label{sec: interrogative intonation}

Choguita Rarámuri interrogative constructions are encoded through morphosyntactic and/or intonational devices. \chapref{chap: sentence types} provides a comprehensive description of the morphosyntactic and prosodic encoding of polar interrogatives (§\ref{subsec: polar questions}) and content questions (§\ref{subsec: content questions}). This section summarizes the intonational properties of both types of constructions. As discussed here and in §\ref{sec: interaction between lexical tone and intonation}, the specific intonation patterns attested in interrogatives are shaped by the interaction between lexical tones and intonemes in utterance-final position.

%adapt what follows

\subsubsection{Polar question intonation}
\label{subsubsec: polar question intonation}

Polar questions may be classified morphosyntactically into three types: (i) morphosyntactically unmarked polar questions; (ii) polar questions with interrogative particles; and (iii) polar questions with interrogative tags.

In the first type, the interrogative clause is equivalent morphosyntactically to its declarative counterpart, differing only in intonation. A minimal pair between a declarative sentence and its morphosyntactically unmarked polar question counterpart in Choguita Rarámuri is provided in (\ref{ex: intonation marked polar questions 2}). \figref{fig: declarative intonation 2} and \figref{fig: morphosyntactically unmarked polar intonation 2} show the intonational difference between the declarative sentence in (\ref{ex: intonation marked polar questions 2a}) and the polar interrogative in (\ref{ex: intonation marked polar questions 2b}), respectively.

\ea\label{ex: intonation marked polar questions 2}
{Declarative vs. morphosyntactically unmarked polar question}

    \ea[]{
    {\textit{ˈmá ˈtôlo}}\\
    \gll    ˈmá ˈtô-li\\
            already bury-\textsc{pst}\\
    \glt    ‘S/he buried him/her.’\footnote{In this particular example, the suffix vowel surfaces as [o] given an optional round harmony process, where non-round vowels of certain suffixes may become round when preceded by a stressed back stem vowel. This optional process is only attested in the speech of younger speakers. For more details about this process, see §\ref{subsubsec: round harmony}.}\\
    \glt    `Lo enterró.' {< BFL el1170 >}\\
}\label{ex: intonation marked polar questions 2a}
        \ex[]{
        {\textit{ˈmá ˈtôli?}}\\
        \gll    ˈmá ˈtô-li\\
              	already bury-\textsc{pst}\\
	    \glt    ‘Did s/he bury him/her?’\\
	    \glt    `¿Lo enterró?' {< BFL el1307 >}\\
    }\label{ex: intonation marked polar questions 2b}
    \z
\z

\begin{figure}
\includegraphics[width=\textwidth]{figures/SentenceTypes-img2.png}
\caption{
\label{fig: declarative intonation 2}
Declarative utterance: \textit{ˈmá ˈtôlo} `S/he buried him/her' (< BFL el1170 >)}
\end{figure}

\begin{figure}
\includegraphics[width=\textwidth]{figures/SentenceTypes-img3.png}
\caption{
\label{fig: morphosyntactically unmarked polar intonation 2}
Morphosyntactically unmarked polar interrogative: \textit{ˈmá ˈtôli?} `Did s/he bury him/her?' (< BFL el1307 >)}
\end{figure}

In the declarative sentence exemplified in \figref{fig: declarative intonation 2}, there is a falling f0 contour of the lexical HL tone of the verb (\textit{ˈtô} `to bury'). As consistently attested in declarative entences with HL lexical tones in utterance-final position, there is no evidence of a H\% boundary tone, as the falling lexical tone in the stressed syllable overrides the boundary tone (see §\ref{subsec: H boundary tones} above and §\ref{sec: interaction between lexical tone and intonation} in \chapref{chap: prosody}). When compared to the declarative sentence in \figref{fig: declarative intonation 2}, the interrogative sentence in \figref{fig: morphosyntactically unmarked polar intonation 2}  exhibits raised f0 in the stressed syllable (396 Hz in this example vs. 289 Hz in the declarative in \figref{fig: declarative intonation 2} for the same female speaker (BFL)). As discussed in §\ref{subsubsec: morphosyntactically unmarked polar questions}, this effect may be attributed to the interrogative H\% intoneme aligning with the peak of the lexical HL tone in this stressed syllable.

A different intonation pattern is documented in interrogative sentences with a final stressed syllable specified for lexical L tone. This is shown in \figref{fig: L tone plus H boundary tone 2}.


\begin{figure}
\includegraphics[width=\textwidth]{figures/SentenceTypes-img5.png}
\caption{
\label{fig: L tone plus H boundary tone 2}
Accommodation of L tone and H\% boundary tone in \textit{ˈmá ˈnèli?} `Did s/he see him/her?' (< BFL el1307 >)}
\end{figure}

As shown in the example in \figref{fig: L tone plus H boundary tone 2}, the lexical tone of the verb root is associated with the stressed syllable and the H\% boundary tone docks on a following, unstressed syllable. Lexical L tones are preserved in their interaction with intonemes, as attested in declarative sentences (see §\ref{sec: interaction between lexical tone and intonation} for further discussion).

A second type of polar interrogative sentence is encoded through a polar interrogative particle (\textit{ˈátʃ͡e} or its reduced form \textit{a}), which occurs in clause initial position. The following minimal pair illustrates the intonational difference between a declarative with a lexical L tone in utterance-final position ((\ref{ex: declarative vs. polar interrogative ache 2a}) in \figref{fig: declarative with L tone 2}) and its polar interrogative counterpart with an interrogative particle ((\ref{ex: declarative vs. polar interrogative ache 2b}) in \figref{fig: polar interrogative with ache lexical L  tone 2}).

\ea\label{ex: declarative vs. polar interrogative ache 2}
{Declarative vs. polar interrogative with \textit{ˈátʃ͡e}}

    \ea[]{
    {\textit{ˈmá naˈwàli}}\\
    \gll    ˈmá naˈwà-li\\
            already arrive-\textsc{pst}\\
    \glt    `S/he already arrived.'\\
    \glt    `Ya llegó.' {< SFH-nawa-arrive-L-minimal-sets >}\\
}\label{ex: declarative vs. polar interrogative ache 2a}
        \ex[]{
        {\textit{\textbf{ˈátʃe} ˈmá naˈwàli}}\\
        \gll    \textbf{ˈátʃe} ˈmá naˈwà-li\\
                Q already arrive-\textsc{pst}\\
        \glt    `Did s/he already arrive?'\\
        \glt    `¿Ya llegó?' < SFH-nawa-arrive-L-minimal-sets >\\
    }\label{ex: declarative vs. polar interrogative ache 2b}
    \z
\z

\begin{figure}
\includegraphics[width=\textwidth]{figures/SentenceTypes-img6.png}
\caption{
\label{fig: declarative with L tone 2}
Declarative with utterance-final lexical L tone in \textit{ˈmá naˈwàli} `S/he already arrived.' }
\end{figure}

\begin{figure}
\includegraphics[width=\textwidth]{figures/SentenceTypes-img7.png}
\caption{
\label{fig: polar interrogative with ache lexical L  tone 2}
Polar interrogative with \textit{ˈátʃ͡e} and utterance-final lexical L tone in \textit{ˈátʃ͡e ˈmá naˈwàli} `Did s/he already arrive?' }
\end{figure}

As shown in these examples, the intonational encoding of polar questions with interrogative particles involves (i) presence of an obligatory H\% boundary tone (recall from §\ref{subsec: H boundary tones} above that declaratives generally end with a H\% tone, but that sometimes declaratives end with a L\% tone instead), and (ii) raised register across the utterance. As attested in other interrogative and declarative sentences, lexical L tones are preserved: as seen in \figref{fig: polar interrogative with ache lexical L  tone 2}, the sharp rise and peak of the H\% tone is aligned with the final, post-tonic syllable. Thus, while H\% tones are optional in declaratives, these appear to be required in interrogatives. Furthermore, interrogatives exhibit raised register. as shown next, content questions are also characterized intonationally by raised ragister across the utterance.


\subsubsection{Content question intonation}
\label{subsubsec: content question intonation}

Content questions in Choguita Rarámuri have a question marker (which may be complex) that appears in clause initial position, as exemplified in (\ref{ex: content question examples 2}).

\ea\label{ex: content question examples 2}
{Content questions}

    \ea[]{
    {\textit{\textbf{ˈpîri } iʔˈkîli koˈtʃ͡î}? }  \\
    \gll    \textbf{ˈpîri}  iʔˈkî-li koˈtʃ͡î? \\
            what  bite-\textsc{pst}    dog\\
    \glt    ‘What did the dog bite?’ \\
    \glt    `¿Qué mordió el perro?' < BFL 09 el725/el >\\
}
        \ex[]{
        {\textit{\textbf{ˈhêpi ˈkwâ}mi ˈʔâbo    maˈjêi    winoˈmî?}}\\
        \gll   \textbf{ˈhêpi} \textbf{ˈkwâ}=mi ˈʔâ-bo    maˈjê-i    winoˈmî?\\
                who who=\textsc{2sg.nom} give\textsc{-fut.pl} think\textsc{-impf} money\\
        \glt    ‘Who did you think they were going to give the money to?’ \\
        \glt    `¿A quién creías que le iban a dar el dinero?' < BFL 09 1:12/el >\\
    }
    \z
\z

Content (or information) questions are not only morphosyntactically distinct from their declarative counterparts, but are also distinctive in terms of their intonational pattern: as shown in the contrast between a declarative sentence (shown in (\ref{ex: declarative vs. content question 2a}), \figref{fig: declarative with L tone-2 2}) and a content question (shown in (\ref{ex: declarative vs. content question 2b}), \figref{fig: content question lexical L  tone 2}).

\ea\label{ex: declarative vs. content question 2}
{Declarative vs. content question}

 \ea[]{
    {\textit{ˈmá naˈwàli}}\\
    \gll    ˈmá naˈwà-li\\
            already arrive-\textsc{pst}\\
    \glt    `S/he already arrived.'\\
    \glt    `Ya llegó.' {< SFH-nawa-arrive-L-minimal-sets >}\\
}\label{ex: declarative vs. content question 2a}
        \ex[]{
        {\textit{\textbf{ˈhêpi ˈkwâ} naˈwàli}}\\
        \gll    \textbf{ˈhêpi} \textbf{ˈkwâ} naˈwà-li\\
                who who arrive-\textsc{pst}\\
        \glt    `Who arrived?'\\
        \glt    `¿Quién llegó?' {< SFH-nawa-arrive-L-minimal-sets >}\\
    }\label{ex: declarative vs. content question 2b}
    \z
\z


\begin{figure}
\includegraphics[width=\textwidth]{figures/SentenceTypes-img13.png}
\caption{
\label{fig: declarative with L tone-2 2}
Declarative with utterance-final lexical L tone}
\end{figure}

\begin{figure}
\includegraphics[width=\textwidth]{figures/SentenceTypes-img8.png}
\caption{
\label{fig: content question lexical L  tone 2}
Content question with utterance-final lexical L tone}
\end{figure}

The f0 contour of the content question in \figref{fig: content question lexical L  tone 2} shows register raising across the utterance, as well as a H-toned target in utterance-final position, which can be attributed to a boundary H\% tone in the last unstressed syllable of the utterance.

A H\% boundary tone is also attested in content questions where a lexical H tone is associated with the final stressed syllable of the utterance. This is shown in (\ref{ex: declarative vs. content question with H tone 2}), with a contrast between a declarative in (\ref{ex: declarative vs. content question with H tone 2a}) (illustrated in \figref{fig: declarative with H tone 2}) and a content question in (\ref{ex: declarative vs. content question with H tone 2b}) (illustrated in \figref{fig: content question lexical H tone 2}).

\ea\label{ex: declarative vs. content question with H tone 2}
{Declarative vs. content question: utterance-final, lexical H tone}

 \ea[]{
    {\textit{ˈmá muˈrúli}}\\
    \gll    ˈmá muˈrú-li\\
            already carry.in.arms-\textsc{pst}\\
    \glt    `S/he already carried it in their arms'\\
    \glt    `Ya lo cargó en brazos' {< BFL-muru-carry-H-minimal-sets >}\\
}\label{ex: declarative vs. content question with H tone 2a}
        \ex[]{
        {\textit{\textbf{ˈhêpi ˈkwâ} muˈrúli}}\\
        \gll    \textbf{ˈhêpi} \textbf{ˈkwâ} muˈrú-li\\
                who who carry.in.arms-\textsc{pst}\\
        \glt    `Who carried it in their arms?'\\
        \glt    `¿Quién lo cargó en brazos?' {< BFL-muru-carry-H-minimal-sets >}\\
    }\label{ex: declarative vs. content question with H tone 2b}
    \z
\z

\begin{figure}
\includegraphics[width=\textwidth]{figures/SentenceTypes-img11.png}
\caption{
\label{fig: declarative with H tone 2}
Declarative with utterance-final lexical H tone}
\end{figure}

\begin{figure}
\includegraphics[width=\textwidth]{figures/SentenceTypes-img12.png}
\caption{
\label{fig: content question lexical H tone 2}
Content question with utterance-final lexical H tone}
\end{figure}

In the declarative sentence in \figref{fig: declarative with H tone 2} there is no evidence of a H\% boundary tone, which, as discussed in §\ref{subsec: H boundary tones}, is optional in declaratives. In the content question represented in \figref{fig: content question lexical H tone 2}, on the other hand, there is a clear high pitch target in the final, unstressed syllable of the utterance. This high pitch target is higher than the one associated with the lexical H tone of the stressed syllable (a difference of almost 30Hz in this particular example).

Finally, in content interrogative sentences containing a lexical HL tone in utterance final position, there is no evidence of a H\% boundary tone nor any register manipulation when the content question contains a lexical HL tone. This is shown in the contrast between the declarative sentence in (\ref{ex: declarative vs. content question with HL tone 2a}) (\figref{fig: declarative with HL tone 2}) and its content question counterpart in (\ref{ex: declarative vs. content question with HL tone 2b}) (\figref{fig: content question with lexical HL tone 2}).

\ea\label{ex: declarative vs. content question with HL tone 2}
{Declarative vs. content question: utterance-final, lexical HL tone}

 \ea[]{
    {\textit{ˈmá iˈsîli}}\\
    \gll    ˈmá iˈsî-li\\
            already pee-\textsc{pst}\\
    \glt    `S/he already peed.'\\
    \glt    `Ya orinó.' {< BFL-isi-pee-HL-minimal-set >}\\
}\label{ex: declarative vs. content question with HL tone 2a}
        \ex[]{
        {\textit{\textbf{ˈhêpi ˈkwâ} iˈsîli}}\\
        \gll    \textbf{ˈhêpi} \textbf{ˈkwâ} iˈsî-li\\
                who who pee-\textsc{pst}\\
        \glt    `Who peed?'\\
        \glt    `¿Quién orinó?' {< BFL-isi-pee-HL-minimal-set >}\\
    }\label{ex: declarative vs. content question with HL tone 2b}
    \z
\z

\begin{figure}
\includegraphics[width=\textwidth]{figures/SentenceTypes-img9.png}
\caption{
\label{fig: declarative with HL tone 2}
Declarative with utterance-final lexical HL tone}
\end{figure}

\begin{figure}
\includegraphics[width=\textwidth]{figures/SentenceTypes-img10.png}
\caption{
\label{fig: content question with lexical HL tone 2}
Content question with utterance-final lexical HL tone}
\end{figure}

In these examples (produced by the same female speaker, BFL), the highest pitch peak associated with the lexical HL tone is comparable in both sentences (270Hz in the declarative in \figref{fig: declarative with HL tone 2} and 260Hz in the content question in \figref{fig: content question with lexical HL tone 2}), and their pitch countours largely equivalent. Thus, the comparison between the intonational contours of these two sentences suggests that there is no distinctive intonational encoding of a content question where HL tones override the H\% tone associated with interrogative constructions elsewhere. This stands in contrast to polar questions, where, as shown in \figref{fig: morphosyntactically unmarked polar intonation 2} above, there is significantly raised f0 in the final stressed syllable of the interrogative utterance.

\subsubsection{Summary}

In Choguita Rarámuri, interrogative sentences are characterized by the following intonational characteristics:

%\break

\ea\label{ex: prosodic properties of interrogatives 2}
{Intonational properties of interrogative constructions}

\begin{itemize}
    \item A boundary H\% tone targets the last stressed syllable of the utterance.\\
    \item There is raised register across the utterance.\\
    %\item A H* pitch accent that associates with the question word of the construction (if any).\\
\end{itemize}
\z

As shown in this section, interrogative constructions that are morphosyntactically equivalent to declaratives are encoded exclusively through intonation, but prosody also plays a role in those interrogative constructions where there is a morphosyntactic device encoding the interrogative meaning. Cross-linguistically, the existence of morphosyntactic encoding of different utterance types may preclude the use of distinctive intonational structures for the same purposes in some languages (e.g., \ili{Navajo} (\ili{Athabaskan}; \citealt{mcdonough2002prosody})).

\citet{miller1996guarijio} reports that closely-related \ili{Mountain Guarijío} (\ili{Taracahitan}; \ili{Uto-Aztecan}) also has polar questions that are morphosyntactically equivalent to their declarative counterparts, with the interrogative meaning encoded through ascending intonation (``\textit{[g]eneralmente tienen entonación ascendente}") (\citeyear[112]{miller1996guarijio}).\footnote{To the best of my knowledge, no study has addressed the intonation of \ili{Mountain Guarijío} through assessment of instrumental data.}

While polar questions show some degree of variation in the realization of the boundary tone and the degree to which register is raised (discussed below and in \chapref{chap: prosody}), they consistently exhibit a high pitch target utterance-finally. As shown in this section, the magnitude of the peak is correlated with the presence/absence of an overt morphological device to encode a question, with the highest pitch excursions attested utterance-finally with morphosyntactically unmarked polar questions.
