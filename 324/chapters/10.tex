\chapter{Minor word classes}
\label{chap: particles, adverbs and other word classes}

This chapter addresses the properties of Choguita Rarámuri minor word classes. Minor word classes can be classified into two groups, depending on whether they may head noun phrases or combine with head nouns in noun phrases and those that cannot. The first group includes pronouns (§\ref{sec:10:pronouns}), demonstratives (§\ref{sec:10:demonstratives}), adjectives (§\ref{sec: adjectives}), numerals (§\ref{sec: numerals}), quantifiers (§\ref{sec: quantifiers}), and definite articles (§\ref{sec: articles}). The second class of minor word classes includes adverbs (§\ref{sec: adverbs}) and a set of particles and enclitics that bear a range of discourse pragmatic functions (§\ref{sec: particles and clitics}).

From these word classes, adjectives and adverbs are the ones that have the largest sets of forms, though they both have significantly fewer lexical items than major word classes. As documented for other \ili{Uto-Aztecan} languages (e.g., \ili{Cupeño} (\ili{Takic}; \citealt{hill2005grammar}), underived adjectives in Choguita Rarámuri comprise a limited set, with most property concepts encoded by words derived by regular derivational mechanisms from verbs (addressed in \chapref{chap: verbal morphology}). In contrast, there is an elaborate system of adverbial free and bound morphemes, and a particularly developed system encoding direction/location, as documented in closely related \ili{Mountain Guarijío} (\ili{Taracahitan}; \citealt{miller1996guarijio}). This chapter provides an overview of the morphological properties of these small word classes, while their syntactic behavior is addressed in other chapters.

\section{Pronouns}
\label{sec:10:pronouns}

In contrast to nouns, free personal pronouns and pronominal enclitics are case-marked. Four different sets of pronouns are described: free personal pronouns in §\ref{subsec: personal pronouns}, pronominal enclitics in §\ref{subsec: pronominal enclitics}, emphatic pronouns in §\ref{subsec: emphatic pronouns} and interrogative pronouns in §\ref{subsec: interrogative pronouns}.

\subsection{Personal pronouns}
\label{subsec: personal pronouns}

%Will need to check whether these forms are used for subjects of both main and subordinate clauses
Choguita Rarámuri personal pronouns distinguish two person values (first and second) and two numbers (singular and plural). In addition to these distinctions, pronouns may encode a binary nominative-accusative case distinction. Nominative pronouns encode subjects of both matrix clauses and subordinate clauses, as well as possessors in noun phrases. Accusative pronominal forms are used to encode single objects of transitive predicates and both primary and secondary objects of ditransitive predicates.\footnote{Nominative/accusative case distinctions in pronominal forms are documented for other Rarámuri varieties, including \ili{Norogachi Rarámuri} (\citealt{brambila1953gramatica}), and \ili{Rochéachi Rarámuri} (\citealt{moralesmoreno2016rochecahi}). For closely related \ili{Mountain Guarijío}, \citet{miller1996guarijio} reports a binary distinction between subject pronominal forms and oblique pronominal forms, with the latter employed to encode objects of transitive and ditransitive clauses, subjects of subordinate clauses and nominal possessors (1996:230).} A subset of pronominal forms, namely first person forms and the second singular form, are morphologically complex and composed of person forms reconstructed from \ili{Proto-Uto-Aztecan} (\citealt{langacker1977uto}) and a \textit{ˈhê} formative that appears to function as a demonstrative in a limited set of contexts. This \textit{ˈhê} formative may be lacking in reduced forms of these pronominal forms.\footnote{Contexts where \textit{ˈhê} appears to function as a demonstrative are described in §\ref{sec:10:demonstratives} below. \citet{villalpando2019grammatical} documents a similar pattern in \ili{Norogachi Rarámuri} free pronouns (\citeyear[37]{villalpando2019grammatical}).} The second singular and first plural subject pronominal forms also exhibit a change in the final vowel of the reduced form (from [u] to [o]), a change also attested in the first plural object pronominal form. The paradigm of Choguita Rarámuri free personal pronouns is shown in \tabref{tab:free-pronouns}.



\begin{table}
\caption{Free personal pronouns}
\label{tab:free-pronouns}

\begin{tabularx}{.5\textwidth}{lll}
\lsptoprule
& \textbf{Subject} & \textbf{Object}\\
\midrule
 \textsc{1sg} & neˈhê, ˈnè & taˈmí\\
 \textsc{2sg} & muˈhê, ˈmò & ˈmí\\
 \textsc{1pl} & tamuˈhê, taˈmò & taˈmò\\
 \textsc{2pl} & ˈémi & ˈmí\\
\lspbottomrule
\end{tabularx}
\end{table}
\hspace{1cm}

Reduced pronominal forms (\textit{ˈnè}, \textit{ˈmò} and \textit{taˈmò}) are prosodically independent: they are stressed and may host enclitics. Reduced pronouns for first and second singular pronouns are monosyllabic and do not undergo lengthening as attested with monosyllabic inflected verbs, i.e., these pronominal forms are exempt from the minimality requirement described in §\ref{sec: vowel length, stress and minimality effects} that applies to verbs stems. Examples of free personal pronouns in main clauses, both unreduced and reduced, are provided in (\ref{ex: free personal pronouns}).


\newpage

\ea\label{ex: free personal pronouns}
{Free personal pronouns}

    \ea[]{
    \textit{neˈhê 	  	ˈpé   	oˈkwâ 	raʔiˈtʃâma 	koriˈmá 	ˈhîtara}\\
    \gll   \textbf{neˈhê} 	  	ˈpé   	oˈkwâ 	raʔiˈtʃâ-ma 	koriˈmá 	ˈhîtara\\
	        {1\textsc{sg.nom}} 	just 	couple 	speak-\textsc{fut.sg} 	fire.bird	about\\
	\glt    `I’ll speak a little about the \textit{korima} (the fire bird). ’\\
	\glt    `Yo voy a hablar poquito del pájaro \textit{korimá} (el pájaro de fuego).’  \corpuslink{tx5[00_229-00_264].wav}{LEL tx5:00:22.9}\\
}
        \ex[]{
        \textit{``ˈnè ʃiˈmêo ˈlá naʔˈpôʃia",  ˈhê aˈní}\\
        \gll   \textbf{ˈnè} siˈ-mêa oˈlá naʔˈpô-si-a  ˈhê aˈní\\
               {1\textsc{sg.nom}} go-\textsc{fut.sg} \textsc{cer} weed-\textsc{mot-prog} \textsc{dem} say.\textsc{prs}\\
        \glt    ```I'm going to weed", it is said like that.'\\
        \glt    ```Voy a escardar", así se dice.'    \corpuslink{el1274[17_175-17_195].wav}{JLG el1274:17:17.5}\\
    }
	        \ex[]{
		        \textit{noˈrînima ˈétʃ͡i biˈléara tʃ͡oˈkêami taˈmí ruˈwèʃia}\\
                \gll    noˈrîni-ma ˈétʃ͡i biˈléara tʃ͡oˈkêami \textbf{taˈmí} ruˈwè-si-a\\
                        arrive-\textsc{fut.sg} \textsc{dem} another bet.settler {1\textsc{sg.acc}} tell-\textsc{mot-prog}\\
                \glt    `The other bet settler (of \textit{ariweta} race) arrives to tell me.'\\
                \glt    `Viene la otra chokéami (apuntadora de carrera de \textit{ariweta}) a decirme.'  \corpuslink{tx19[01_179-01_236].wav}{LEL tx19:01:17.9}\\
	   }
           \ex[]{
                    \textit{aʔˈlì tamuˈhê ˈmá … ˈmá aʔˈlì raˈwè ˈmá ti napaˈwí ... hiˈrâmia}\\
                    \gll    aʔˈlì \textbf{tamuˈhê} ˈmá aʔˈlì raˈwè ˈmá=ti napaˈwí hiˈrâ-mi-a\\
                            and {\textsc{1pl.nom} }then then later day already=\textsc{1pl.nom} gather bet-\textsc{mov-prog}\\
                    \glt    `And then we gather that day to bet.’\\
                    \glt    `Entonces nosotros ese día ya nos juntamos para apostar.’ \corpuslink{tx19[01_323-01_373].wav}{LEL tx19:01:32.3}\\
            }
                    \ex[]{
                    \textit{ˈkíti ku taˈmò raˈlàmuli ˈpé kuˈrì oˈtʃêrikam ˈhú taˈmò ko}\\
                    \gll    ˈkíti ku \textbf{taˈmò} raˈlàmuli ˈpé kuˈrì oˈtʃêra-kame ˈhú taˈmò=ko\\
                                because \textsc{emph} {1\textsc{pl.nom}} Rarámuri just recently grow-\textsc{pst.ptcp} \textsc{cop} 1\textsc{pl.nom}=\textsc{emph}\\
                    \glt    `Because us the Rarámuri have grown up just recently, our people.'\\
                    \glt    `Porque hace poco que crecimos nosotros los rarámuri, la gente como nosotros.' \corpuslink{in243[18_578-19_049].wav}{FLP in243:18:57.8}\\
                }
\newpage
                        \ex[]{
                        \textit{aʔˈlì ˈmá ke iˈtêo ˈlá ˈnà ˈháp taˈmò tʃ͡iriˈká ruˈjèma ko}\\
                        \gll    aʔˈlì ˈmá ke iˈtê oˈlá ˈnà ˈhápi \textbf{taˈmò} ˈétʃ͡i riˈká ruˈ-è-ma=ko\\
                                and	already	\textsc{neg} not.exist \textsc{cer} \textsc{prox}	\textsc{sub} {\textsc{1pl.acc}} \textsc{dem} like say\textsc{-appl-fut.sg}=\textsc{emph}\\
	                    \glt    `And then there is nothing for us to be told.’\\
                        \glt    `Y luego ya no hay para que nos digan a nosotras.’ \corpuslink{tx905[02_041-02_085].wav}{GFM tx905:02:04.1}\\
                    }
    \z
\z

\hspace*{-1.8pt}Second person pronouns do not encode a number distinction in their accusative form (this is also reported in closely related \ili{Mountain Guarijío} (\citealt{miller1996guarijio})). Examples of second person pronominal forms are provided in (\ref{ex: 2nd person free personal pronouns}).

\ea\label{ex: 2nd person free personal pronouns}
{Second person free personal pronouns}

    \ea[]{
        \textit{muˈhê ˈmá ke ˈtʃ͡i ˈbíri rimuˈrú ˈhípi ko ba?}\\
        \gll    \textbf{muˈhê}ˈmá ke ˈtʃ͡i ˈbíri rimu-ˈrú ˈhípi=ko ba?\\
               {2\textsc{sg.nom}} anymore \textsc{neg} which kinds dream-\textsc{prs} today=\textsc{emph} \textsc{cl}\\
        \glt    `And you don't dream many things anymore?'\\
        \glt    `¿Y ahora ya no sueñas muchas cosas?' \corpuslink{co1136[16_384-16_409].wav}{MDH co1136:16:38.4}\\
}
            \ex[]{
            \textit{ˈmò ˈwé biˈnè ˈétʃ͡i}\\
            \gll   \textbf{ˈmò} ˈwé biˈnè ˈétʃ͡i\\
                    {2\textsc{sg.nom}} \textsc{int} know.\textsc{prs} \textsc{dem}\\
            \glt    ```You know a lot about that...'''\\
            \glt    ```Tu sabes mucho de eso...''' \corpuslink{tx5[05_189-05_230].wav}{LEL tx5:05:18.9}\\
        }
                \ex[]{
                    \textit{baʔaˈrîni		ˈmí		ˈàma}\\
                    \gll    baʔaˈrî=ni		\textbf{ˈmí	}	ˈà-ma\\
                            tomorrow=1\textsc{sg.nom}	{2\textsc{sg.acc}}	look.for-\textsc{fut.sg}\\
	                \glt    `I’ll look for you tomorrow.’\\
    	            \glt    `Mañana te busco.’  <LEL 09 1:70/el>\\
            }
                        \ex[]{
                        \textit{aʔˈlì ˈétʃ͡i ˈmín aˈnèma aʔˈlì}\\
                        \gll    aʔˈlì ˈétʃ͡i \textbf{ˈmí}=ni aˈn-è-ma aʔˈlì\\
                                and \textsc{dem} \textsc{{2sg.acc}=1sg.nom} say-\textsc{appl-fut.sg} later\\
                        \glt    ```And then I’ll tell you'''\\
                        \glt    ```Y entonces te digo''' \corpuslink{tx19[01_135-01_179].wav}{LEL tx19:01:13.5}\\
                    }
                \newpage
                        \ex[]{
                        \textit{kuˈrí oˈtʃêrirami ˈémi ko ba}\\
                        \gll    kuˈrí oˈtʃêri-r-ame \textbf{ˈémi}=ko ba\\
                                recently grow-\textsc{pst.pass-ptcp} \textsc{2pl.nom}=\textsc{emph} \textsc{cl}\\
                        \glt    `You all who have recently grew up.'\\
                        \glt    `Ustedes los crecidos hace poco’ \corpuslink{in61[00_383-00_421].wav}{FLP in61:00:38.3}\\
                    }
                            \ex[]{
                            \textit{ˈémi ko ˈnè aʔˈlá ˈnâtami ˈníbo ˈlá ba}\\
                            \gll    \textbf{ˈémi}=ko ˈnè aʔˈlá ˈnât-ame ˈní-bo oˈlá ba\\
                                    {\textsc{2pl.nom}}=\textsc{emph} \textsc{int} well think-\textsc{ptcp} \textsc{cop-fut.pl} \textsc{cer} \textsc{cl}\\
                            \glt    `You all must think well.’\\
                            \glt    `Ustedes piensen bien.’ \corpuslink{tx12[11_404-11_417].wav}{SFH tx12:11:40.4}\\
                        }
                                \ex[]{
                                \textit{ˈpé biˈlá tʃ͡utʃ͡uˈrú ˈnà ˈmí ruˈwè ˈémi ˈkîni ˈkûtʃ͡uwa ba ˈne}\\
                                \gll    ˈpé biˈlá tʃ͡utʃ͡uˈrú ˈnà \textbf{ˈmí} ruˈ-è ˈémi ˈkîni ˈkûtʃ͡uwa ba ˈne\\
                                        just really that.much \textsc{dem} {\textsc{2pl.acc}} say-\textsc{appl} \textsc{2pl.nom} \textsc{1poss} children \textsc{cl} \textsc{cl}\\
                                \glt    `Just that much I say to you all, you, my children.’\\
                                \glt    `Nomás de ese tanto les digo, ustedes, mis hijos.’ \corpuslink{tx12[12_461-12_492].wav}{SFH tx12:12:46.1}\footnote{As discussed in \chapref{chap: noun phrases} (§\ref{subsubsec: pronominal possessors}), Choguita Rarámuri has a dedicated pronominal form, namely \textit{ˈkîni}, which encodes first person possessors (singular or plural) of kinship terms.} \\
                            }
    \z
\z

Third person arguments may be left unmarked (\ref{ex: third person arguments encodinga}), or they may be encoded through demonstratives (e.g., the demonstrative \textit{ˈétʃ͡i} in (\ref{ex: third person arguments encodingb})) or through an emphatic pronoun (e.g., \textit{biˈnôi} `himself' in (\ref{ex: third person arguments encodingc})).

\ea\label{ex: third person arguments encoding}

    \ea[]{
    \textit{aʔˈlì ke muˈríwia ruˈwá}\\
    \gll    aʔˈlì ke muˈríwi-a ru-ˈwá\\
            and \textsc{neg} get.close-\textsc{prog} say-\textsc{mpass}\\
    \glt    `And they say they didn't get close.'\\
    \glt    `Y dicen que ellos no se arrimaban.’  \corpuslink{tx109[02_142-02_164].wav}{LEL tx109:02:14.2}\\
}\label{ex: third person arguments encodinga}
        \ex[]{
        \textit{aʔˈlì ˈétʃ͡i taˈmí ``kuˈmûtʃ͡i" aˈnèma ba?}\\
        \gll    aʔˈlì \textbf{ˈétʃi} taˈmí kuˈmûtʃ͡i aˈn-è-ma ba\\
                and {\textsc{dem}} \textsc{1sg.acc} younger.maternal.uncle say-\textsc{appl-fut.sg} \textsc{cl}\\
        \glt    `And will they call me ``kumuchi" (younger maternal uncle)?'\\
        \glt    `¿Y ellos me van a decir ``kumuchi" (tío materno menor que la mamá)?'   \corpuslink{in484[13_594-14_013].wav}{SFH in484:13:59.4}\\
    }\label{ex: third person arguments encodingb}
%\break
            \ex[]{
            \textit{ˈápi aʔˈlì biˈnôi wikaˈrâ ko ˈhê aˈní ˈrú}\\
            \gll    ˈápi aʔˈlì \textbf{biˈnôi} wikaˈrâ=ko ˈhê aˈní ˈrú\\
                    \textsc{sub} then {himself} sing.\textsc{prs=emph} \textsc{dem} say.\textsc{prs} say.\textsc{prs}\\
            \glt    ‘when he sings he says this’\\
            \glt    ‘cuando canta él así dice’  \corpuslink{tx71[03_157-03_183].wav}{LEL tx71:03:15.7}\\
        }\label{ex: third person arguments encodingc}
    \z
\z

The use of demonstratives to refer to third person arguments is discussed in more detail in §\ref{sec:10:demonstratives} below.

\subsection{Pronominal enclitics}
\label{subsec: pronominal enclitics}

Free subject pronouns have corresponding enclitic forms, phonologically bound formatives that are prosodically dependent on their host and are unrestricted regarding the syntactic category of the words they attach to, two properties that may serve as diagnostics of clitics cross-linguistically \parencite{bickel2007inflectional}. Choguita Rarámuri pronominal enclitics are unstressed, monosyllabic forms with high front vowels, a trait that may be attributed to general processes of post-tonic vowel reduction (as described in §\ref{subsubsec: stress-based vowel reduction and deletion}). \tabref{tab:key:22} illustrates the clitic pronominal forms (free subject and object pronouns are given in parenthesis).

\begin{table}
\caption{Pronominal enclitic forms}
\label{tab:key:22}

\begin{tabularx}{.5\textwidth}{lll}
\lsptoprule
& \textbf{Subject} & \textbf{Object}\\
\midrule
 \textsc{1sg} & =ni (neˈhê, ˈnè) & (taˈmí)\\
 \textsc{2sg} & =mi (muˈhê, ˈmò) & (ˈmí)\\
 \textsc{1pl} & =ti (tamuˈhê, taˈmò) & (taˈmò)\\
 \textsc{2pl} & =timi (ˈémi) & (ˈmí)\\
\lspbottomrule
\end{tabularx}
\end{table}
%\hspace{3cm}

%This should be edited, given the chapter on verbs (or description from there can be removed)

%This chapter should also include discussion about the status of these forms given Morales Moreno thesis -

Choguita Rarámuri person enclitics can attach to verbs and hosts of virtually any category, and, like many other \ili{Uto-Aztecan} languages (\citealt{steele1976law}) and other Rarámuri varieties (\citealt{moralesmoreno2016rochecahi}), are generally in what is traditionally called the Wackernagel position, immediately after the first accented phrase or sub-constituent of a phrase (\citealt{bickel2007inflectional}). The following examples illustrate the distribution of person enclitics hosted by a wide range of word classes in a variety of syntactic contexts: (i) subordinating morphemes in subordinate clauses (\ref{ex: person enclitic hostsa}); (ii) demonstratives (including definite articles) within Noun Phrases (\ref{ex: person enclitic hostsb}); (iii) preposed particles (\ref{ex: person enclitic hostsc}); (iv) negative adverbs (\ref{ex: person enclitic hostsd}); (v) epistemic particles (\ref{ex: person enclitic hostse}); (vi) nouns (\ref{ex: person enclitic hostsf}); and (vii) free person pronouns (\ref{ex: person enclitic hostsg}). As shown in these examples, pronominal enclitic forms may undergo vowel deletion. Clitic hosts are underlined.

%check preposed particles label

\ea\label{ex: person enclitic hosts}
{Person enclitic hosts}

    \ea[]{
    {Subordinator}\\
    \textit{riˈmùini ˈnáptim noˈkáo}\\
    \gll    riˈmù-i=ni [ˈnápi\textbf{=timi} noˈká-o]\\
            dream-\textsc{impf=1sg.nom} \textsc{sub{=2pl.acc}} move-\textsc{ep}\\
    \glt    `I used to dream that you all were moving.’\\
    \glt    `Yo soñaba que ustedes se movían.’   < BFL 05 1:114/el >\\
}\label{ex: person enclitic hostsa}
        \ex[]{
        {Demonstratives (within a noun phrase)}\\
        \textit{ˈtin toˈrí siʔˈrítimo ˈlá}\\
        \gll    ˈti\textbf{=ni} toˈrí siʔˈrí-ti-ma oˈlá\\
                \textsc{def.bad={1sg.nom}} chicken drown.\textsc{intr-caus-fut.sg} \textsc{cer}\\
        \glt    `I will drown the chicken.’\\
        \glt    `Voy a ahogar al pollo.’  < BFL 05 2:49/el >\\
    }\label{ex: person enclitic hostsb}
            \ex[]{
            {Preposed particles}\\
            \textit{aʔˈlì ˈkun noˈrînima}\\
            \gll    aʔˈlì ˈku={ni} noˈrîni-ma\\
                    later \textsc{rev}={1\textsc{sg.nom}} return-\textsc{fut.sg}\\
            \glt    ‘I will come back later.’\\
            \glt    ‘Al rato vuelvo.’   < BFL 05 2:49/el >\\
        }\label{ex: person enclitic hostsc}
                \ex[]{
                {Negative adverbs}\\
                \textit{ˈkeni ˈtâʃi ˈtʃ͡ó maˈnâ baˈʰtâri}\\
                \gll    ˈke=\textbf{ni} ˈtâsi ˈtʃ͡ó maˈnâ baˈʰtâri\\
                        \textsc{neg={1sg.nom}} \textsc{neg} yet  make.beverage corn.beer\\
                \glt    ‘I haven’t made corn beer yet.’  \\
                \glt    ‘No he hecho tesgüino todavía.’  < BFL 05 2:56/el >\\
            }\label{ex: person enclitic hostsd}
                    \ex[]{
                    {Epistemic particles}\\
                    \textit{noˈkèli ˈlén ˈmáo}\\
                    \gll    noˈk-è-li aˈlé\textbf{=ni} ˈmá-o\\
                            move-\textsc{pst} \textsc{dub={1sg.nom}} maybe-\textsc{ep}\\
                    \glt    ‘Maybe I moved him.’\\
                    \glt    ‘A lo mejor lo moví.’  < BFL 05 1:114/el >\\
                }\label{ex: person enclitic hostse}
%                \break
\newpage
                        \ex[]{
                        {Nouns}\\
                        \textit{napaʔˈlì noˈkáli ronoˈtʃ͡íni oʔˈkô}\\
                        \gll    ˈnápi aʔˈlì noˈká-li [ronoˈtʃ͡í\textbf{=ni} oʔˈkô]\\
                                \textsc{sub} later move-\textsc{pst} legs=\textsc{{1sg.nom}}  hurt\\
                        \glt    ‘When I moved, my legs hurt.’  \\
                        \glt    ‘Cuando me moví me dolieron las piernas.’   < BFL 05 1:114/el >\\
                    }\label{ex: person enclitic hostsf}
                            \ex[]{
                            {Full pronouns}\\
                            \textit{ˈpé taˈmòm naˈhâta iˈʃì}\\
                            \gll    ˈpé taˈmò=\textbf{mi} naˈhâta iˈsì\\
                                    just 1\textsc{pl.acc={dem}} follow do.\textsc{prs}\\
                            \glt    ‘It went like that, following us around.’ \\
                            \glt    ‘Así anduvo siguiéndonos.’  < BFL 05 text 2/tx>\\
                        }\label{ex: person enclitic hostsg}
    \z
\z

Although the list of possible hosts in (\ref{ex: person enclitic hosts}) is not exhaustive, it illustrates clearly the unrestrictedness of possible hosts for the person enclitics in Choguita Rarámuri.

\subsection{Emphatic pronouns}
\label{subsec: emphatic pronouns}

There are two emphatic pronominal forms in Choguita Rarámuri, listed in (\ref{ex: empahtic pronouns}).

\ea\label{ex: empahtic pronouns}
{Emphatic pronouns}

    \ea[]{
    \textit{biˈnôi} - singular\\
}
        \ex[]{
        \textit{aˈbôi} - plural\\
    }
    \z
\z

These pronominal forms focus attention on the participants encoded as subjects in contexts where other potential arguments could be subjects. This is exemplified in (\ref{ex: emphatic pronouns}).

\ea\label{ex: emphatic pronouns}

    \ea[]{
    \textit{aʔˈlì ˈétʃ͡i ˈnápu roˈwéma ˈlé ko \textbf{biˈnôi} biˈlá aˈní “ˈjénan ˈá saˈjèrima” ˈá aˈní}\\
    \gll    aʔˈlì ˈétʃ͡i ˈnápu roˈwé-ma aˈlé=ko \textbf{biˈnôi} biˈlá aˈní “ˈjéna a=ni ˈá saˈjèri-ma” ˈá aˈní\\
    and \textsc{dem} \textsc{sub} women.race-\textsc{fut.sg} \textsc{dub}=\textsc{emph} \textsc{emph.sg} really say.\textsc{prs} yes indeed=\textsc{1sg.nom} indeed take.on-\textsc{fut.sg} indeed say.\textsc{prs}\\
    \glt    ‘And then the one who will run, herself, says: “yes, I will take on the challenge”.’\\
    \glt    ‘Y entonces la que va a correr ella misma dice “sí le voy a entrar”.’  \corpuslink{tx19[00_398-00_451].wav}{LEL tx19:00:39.8}\\
}\label{ex: emphatic pronounsa}
        \ex[]{
        \textit{ˈnápu riˈká ˈne aʔˈlá beˈnèrpo ˈtʃ͡ó ˈkîni ˈkûtʃ͡uwa ˈtʃ͡ó ˈkûtʃ͡uwa ˈtʃ͡ó ˈémi \textbf{aˈbôi} ba niˈbí}\\
        \gll    ˈnápu riˈká ˈne aʔˈlá beˈnè-ri-po tʃ͡o ˈkîni ˈkûtʃ͡uwa ˈtʃ͡ó ˈkûtʃ͡uwa tˈʃo ˈémi aˈbôi ba niˈbí\\
                \textsc{sub} like \textsc{int} well learn-\textsc{caus-fut.pl} also my children also children also 2\textsc{pl.nom} \textsc{emph.pl} \textsc{cl} nibí\\
        \glt    ‘So that we can teach well our children, children, you all.’\\
        \glt    ‘Para enseñarles bien a nuestros hijos, los hijos, ustedes mismos.’ \corpuslink{tx12[05_300-05_365].wav}{SFH tx12:05:30.0}\\
}\label{ex: emphatic pronounsb}
    \z
\z

In example (\ref{ex: emphatic pronounsa}), the singular emphatic pronoun \textit{biˈnôi} makes clear that the runner, and not other potential actors, is the source of the quoted speech. In (\ref{ex: emphatic pronounsb}), the  plural emphatic pronoun \textit{aˈbôi} is used to emphasize the addressees, the children of the speaker who he is giving advice to.

\subsection{Interrogative pronouns and phrases}
\label{subsec: interrogative pronouns}

%%Insert cross-reference to Chapter 9 (constructions)
%%Table 2.2: Choguita Rarámuri interrogative pronouns

Choguita Rarámuri has a set of interrogative pronouns and phrases, most of which are morphologically complex.  Of this set, only four forms are morphologically simplex. This inventory also includes the interrogative pronoun \textit{(he)ˈkwâ}, where the \textit{ˈhê} formative present in pronominal forms (see \tabref{tab:key:22}) is optional in the interrogative pronoun. These are provided in \tabref{tab:interrogative-pronouns}.

\begin{table}
\caption{Choguita Rarámuri interrogative pronouns: basic forms}
\label{tab:interrogative-pronouns}

\begin{tabularx}{.5\textwidth}{ll}
\lsptoprule
\textbf{Forms}  & \textbf{Gloss} \\
\midrule
\textit{t͡ʃú?} & How? (\textit{¿Cómo?})\\
\textit{ˈpîri?}  & What? (\textit{¿Qué?})\\
\textit{(he)ˈkuâ} & Who? (\textit{¿Quién?})\\
\textit{ˈkámi?/ˈkúmi?}  & Where? (\textit{¿Dónde?})\\
\textit{kaˈbú?} & When? (\textit{¿Cuándo?})\\
\lspbottomrule
\end{tabularx}
\end{table}
%\hspace{3cm}

\hspace*{-4pt}A recurrent pattern in morphologically complex question words and phrases involves the interrogative word \textit{tʃ͡ú} followed by another morpheme, whether bound (e.g., \textit{-ˈrupi} in \textit{tʃ͡ú ˈrupi} `How much?') or free (e.g.,  a demonstrative (e.g., \textit{(ˈnà-ti} `that-with' in \textit{tʃ͡ú ˈnà-ti} `with what?)). The set of morphologically complex question words is shown in \tabref{tab:interrogative-pronouns-2}.

%\break

\begin{table}
\caption{Choguita Rarámuri interrogative pronouns}
\label{tab:interrogative-pronouns-2}

\begin{tabularx}{.9\textwidth}{ll}
\lsptoprule
\textbf{Forms}  & \textbf{Gloss} \\
\midrule
\textit{t͡ʃú (t͡ʃe) riˈká?} & How? (\textit{¿Cómo?})\\
\textit{t͡ʃí ˈjíri?} & Which kind? (\textit{¿Qué tipo?})\\
\textit{(t͡ʃú) ˈkípi?} & How many? (\textit{¿Cuántos?})\\
\textit{tʃ͡ú ˈrúpi?} & How much? (\textit{¿Qué tanto?})\\
\textit{tʃ͡ú ˈjêni?} & How much? (\textit{¿Qué tanto?})\\
\textit{tʃ͡ú t͡ʃurú?} & How much? (\textit{¿Qué tanto?})\\
\textit{tʃ͡ú iˈkíana?} & How many places? (\textit{¿Cuántos lugares?})\\
\textit{tʃ͡ú kiˈnápi?} & At how many places? (\textit{¿Qué tantos lugares?})\\
\textit{tʃ͡ú riˈkó?}  & When? (\textit{¿Cuándo?})\\
\textit{tʃ͡ú (t͡ʃe) oˈlá?} & Why? (\textit{¿Por qué?})\\
\textit{tʃ͡ú ˈjêni?} & At what time? (\textit{¿A qué hora?})\\
\textit{tʃ͡ú kiˈrípi?} & How long? (\textit{¿Cuánto tiempo?})\\
\textit{tʃ͡ú ˈnà-ti?} & With what? (\textit{¿Con qué?}\\
\textit{ˈpîri ˈnà-ti?} & With what? (\textit{¿Con qué?})\\
\textit{(he pi) ˈkwâ ˈjûa?} & With whom? (\textit{¿Con quién?})\\
\lspbottomrule
\end{tabularx}
\end{table}
%\hspace{3cm}

In contrast to free pronominal forms, interrogative pronouns are not case marked, except for the forms \textit{ˈpîri ˈnà-ti?} and \textit{ˈtʃ͡ú ˈnà-ti?} `with what', where the question words (\textit{ˈpîri} and \textit{ˈtʃ͡ú} are followed by the proximal demonstrative \textit{ˈnà} bearing the \textit{-ti} instrumental case marker.

In contrast to closely related \ili{Mountain Guarijío} (\citealt{miller1996guarijio}), Choguita Rarámuri does not deploy interrogative pronouns as indefinite pronouns.

%describe here these forms - provide examples
%describe whether these forms may be broken up by other elements


\section{Demonstratives}
\label{sec:10:demonstratives}

Demonstratives are defined here as a set of deictic elements that may refer or restrict reference to referents in a situational (exophoric) usage, identifying entities in a surrounding physical situation, but also may be used endophorically  in discourse and recognitional deixis (\citealt{himmelmann1996demonstratives}, \citealt{enfield2003demonstratives}). Demonstratives in Choguita Rarámuri may function anaphorically as pronouns (§\ref{subsec: demonstrative pronouns}) or modifying a nominal element (§\ref{subsec: adnominal demonstratives}). Both sets involve the same set of forms, and are distinguished in the following sections in terms of their function. Other potential uses of demonstratives in discourse are yet to be examined and are left out of the scope of this grammar.

\subsection{Demonstrative pronouns}
\label{subsec: demonstrative pronouns}

As described in §\ref{subsec: personal pronouns}, reference to third person arguments may be achieved through demonstrative pronouns. Two degrees of distance and orientation with respect to the speaker/addressee is encoded by these forms, listed in (\ref{ex: demonstratives}):

\ea\label{ex: demonstratives}
{Demonstrative pronouns}

    \ea[]{
    \textit{ˈnà} `this one' - proximal, close to the speaker\\
}
        \ex[]{
        \textit{ˈétʃ͡i} `that one' - proximal, close to the addressee or the speaker\\
    }
    \z
\z

The use of both proximal demonstrative pronouns is frequently attested in the Choguita Rarámuri corpus. Examples of these demonstrative pronouns is provided in (\ref{ex: demonstrative pronouns}).

\ea\label{ex: demonstrative pronouns}
{Demonstrative pronouns}

    \ea[]{
    \textit{ˈpé ˈnàbi tʃ͡e ˈt͡ʃéti koˈlì biˈtíam ku}\\
	\gll    ˈpé \textbf{ˈnà}=bi tʃ͡e ˈt͡ʃéti koˈlì biˈtí-ame ku\\
	       just {{this.one}}=just	also \textsc{det.pl} over.there	lie.down.\textsc{pl-ptcp} \textsc{rev}\\
	\glt    `Just these ones (the dead people) who are lying down over there also.’\\
	\glt    `Nada más estos que están (acostados) ahí por de aquel lado (los muertos).’      <FLP 07 in243(511)/in>\\
}
        \ex[]{
        \textit{``ˈnà ko ˈpâalichi paˈkótami" ˈhê biˈlá aˈní ba aˈní}\\
        \gll    \textbf{ˈnà}=ko ˈpâalichi paˈkó-t-ame" ˈhê biˈlá aˈní ba aˈní\\
                {{this.one}}=\textsc{emph} priest wash-\textsc{pacient-ptcp} \textsc{dem} really say.\textsc{prs} \textsc{cl} say.\textsc{prs}\\
        \glt    ```This one is baptized by a priest" that's what they say.'\\
        \glt    ```Este es bautizado por padre" así dicen.'  \corpuslink{tx475[08_039-08_066].wav}{SFH tx475:08:03.9}\\
    }
            \ex[]{
            \textit{ˈétʃ͡i ko tʃ͡oˈmí ˈtʃ͡ó muˈtʃûwi}\\
            \gll   \textbf{ˈétʃi}=ko tʃ͡oˈmí ˈtʃ͡ó muˈtʃûwi\\
	                {{those.ones}}=\textsc{emph} there also	sit.\textsc{pl.prs}\\
	        \glt    `They were (sitting) also over there.’\\
		    \glt    `Ellos también estaban allá.’  <FL 06 in61(302)/in>\\
        }
                \ex[]{
                \textit{ˈétʃ͡i ko ˈwé aˈnátʃ͡a}\\
                \gll    \textbf{ˈétʃi}=ko ˈwé aˈnátʃ͡a\\
                        {{that.one}}=\textsc{emph} \textsc{int} endure.\textsc{prs}\\
                \glt    `That one really endures (to run).'\\
                \glt    `Ese aguanta mucho (correr).'    \corpuslink{el1278[03_596-04_026].wav}{JLG el1278:03:59.6}\\
            }
    \z
\z

In addition to these demonstrative pronouns, there is a third, more restricted demonstrative pronoun \textit{ˈhê}. This demonstrative pronoun is exclusively found as the pronominal complement of utterance predicates (`say', `tell'), as well as other predicates that express a positive attitude regarding the truth of the proposition expressed as their complement (`think', `believe'). Crucially, this formative appears in contexts where there is quoted speech, and its function appears to be to index the direct speech. Relevant examples are provided in (\ref{ex: he demonstrative}).

\ea\label{ex: he demonstrative}
{Demonstrative \textit{ˈhê} in quoted speech contexts}

    \ea[]{
    \textit{aʔˈlì ˈhê aˈníli: “ˈnè ko a maˈtʃ͡í ˈkúmi ˈjéna biˈtê ˈétʃ͡i oˈhí”}\\
    \gll    aʔˈlì \textbf{ˈhê} aˈní-li ˈnè=ko a maˈtʃ͡í ˈkúmi ˈjéna biˈtê ˈétʃ͡i oˈhí”\\
            and {\textsc{dem}} say-\textsc{pst} \textsc{1sg.nom}=\textsc{emph} indeed know.\textsc{prs} where indeed live.\textsc{prs} that bear\\
    \glt    `And they said: “I do know where the bear lives”.’\\
    \glt    `Y entonces dijeron: “yo sí se dónde vive el oso”.’    \corpuslink{tx32[02_110-02_154].wav}{LEL tx32:02:11.0}\\
}
        \ex[]{
        \textit{aʔˈlì ˈhê aˈní ˈtîo: “a riˈwèʃi, ˈkíti biˈlé tʃ͡apiˈʃì ˈétʃ͡i”}\\
        \gll    aʔˈlì \textbf{ˈhê} aˈní ˈtîo: “a riˈwè-si, ˈkíti biˈlé tʃ͡api-ˈsì ˈétʃ͡i”\\
                and {\textsc{dem}} say.\textsc{prs} uncle indeed leave-\textsc{imp.pl} \textsc{neg.imp} one grab-\textsc{imp.pl} that\\
        \glt    `And my uncle said: ``leave it, don't touch it''.'\\
        \glt    `Y dijo mi tío: “déjenlo no lo tienten”.’    \corpuslink{tx84[05_436-05_465].wav}{LEL tx84:05:43.6}\\
    }
    \z
\z

The cognate of the \textit{ˈhê} demonstrative in \ili{Norogachi Rarámuri} is described as a proximal demonstrative in (\citealt{villalpando2019grammatical}). Examples provided of this formative in \citet{brambila1953gramatica} and \citet{villalpando2019grammatical} appear in the same environments as in Choguita Rarámuri, namely in the context of utterance and other predicates that introduce quoted speech.

%this needs to be confirmed
%Demonstratives can be marked with case:

%échi  	ná-ti   	ró ...
%dem 	dem-inst	q
%‘Por aqui dicen que hay a veces ... así...de eso’
%[FLP 06 in61(26)/in]


\subsection{Adnominal demonstratives}
\label{subsec: adnominal demonstratives}

The two demonstrative forms that may be used pronominally may also be used as adnominal demonstratives: \textit{ˈnà} `this, speaker-proximate' and \textit{ˈétʃ͡i} `that, speaker or addressee proximate'. As shown in the examples in (\ref{ex: adnominal demonstrative examples}), adnominal demonstratives precede the head noun (further details relating the syntactic order of elements within noun phrases is addressed in \chapref{chap: noun phrases}).

\ea\label{ex: adnominal demonstrative examples}

    \ea[]{
    \textit{``ˈpîri ˈtʃêtimi oˈlá taˈmí ke biˈlé ˈpé ta ukuˈwèami u pa?" ˈhê riˈká biˈlá iˈjòani ˈétʃ͡i Patricio, ˈá biˈlá ko}\\
    \gll    ˈpîri ˈtʃê=timi oˈlá taˈmí ke biˈlé ˈpé ta ukuˈwè-ame ˈhu pa? ˈhê riˈká biˈlá iˈjòani [\textbf{ˈétʃi} Patricio] ˈá biˈlá=ko\\
            why why=\textsc{2pl.nom} why 1\textsc{sg.acc} \textsc{neg} one just little ukuwea-\textsc{ptcp} \textsc{cop} \textsc{cl} \textsc{dem} like really angry.\textsc{prs} {that} Patricio indeed really=\textsc{emph}\\
    \glt    ```Why didn't you do that ukuweruwa thing?" says sometimes Patricio.'\\
    \glt    ```¿Porqué ustedes no me hicieron eso del ukuwéruwa?" a veces nos dice Patricio enojados.'    \corpuslink{tx475[09_168-09_212].wav}{SFH tx475:09:16.8}\\
}\label{ex: adnominal demonstrative examplesa}
        \ex[]{
        \textit{aʔˈlì ko ˈmá wiˈlí siˈkôtʃ͡i ˈúmiri ˈétʃ͡i koriˈmá}\\
        \gll    aʔˈlì=ko ˈmá wiˈlí siˈkô-tʃ͡i ˈúmiri [\textbf{ˈétʃi} koriˈmá]\\
                and=\textsc{emph} already stand.\textsc{sg.prs} corner-\textsc{loc} long.time {this} fire.bird\\
        \glt    `And then that \textit{korima} (fire bird) had been standing in the corner already for a long time.’\\
        \glt    `Y entonces ya andaba mucho rato ahí por el rincón el \textit{korimá} (pájaro de fuego).’   \corpuslink{tx5[01_284-01_333].wav}{LEL tx5:01:28.4}\\
    }\label{ex: adnominal demonstrative examplesb}
            \ex[]{
            \textit{ˈnà leˈhîdotʃ͡i ˈtʃ͡ó ˈmáti waˈná ˈnà roˈhàna ˈtʃ͡ó}\\
            \gll    [\textbf{ˈnà} leˈhîdotʃ͡i] ˈtʃ͡ó ˈmá=ti waˈná ˈnà roˈhà-na ˈtʃ͡ó\\
                    {this} ejido also already=\textsc{1pl.nom} aside this separate\textsc{-tr.prs} also\\
            \glt    `This \textit{ejido} also, we were separated.'\\
            \glt    `Este ejido también, ya nos apartaron.’  \corpuslink{tx817[01_030-01_066].wav}{JMF tx817:01:03.0}\\
        }\label{ex: adnominal demonstrative examplesc}
                \ex[]{
                \textit{tiˈwé ˈnà ˈtʃîkle koˈʔáli}\\
                \gll    tiˈwé [\textbf{ˈnà} ˈtʃîkle] koˈʔá-li\\
                        girl this gum eat-\textsc{pst}\\
                \glt    `The girl chewed this gum.'\\
                \glt    `La niña se comió este chicle.'  \corpuslink{el1028[04_158-04_176].wav}{SFH el1028:04:15.8}\\
            }\label{ex: adnominal demonstrative examplesd}
    \z
\z

As shown in these examples, the ``addressee/speaker-proximate'' demonstrative \textit{ˈétʃ͡i} `that' indexes an object or person accessible to the addressee and/or the speaker: in the case of (\ref{ex: adnominal demonstrative examplesa}), the speaker's son, known also to me, the addressee; in the case of (\ref{ex: adnominal demonstrative examplesb}), the topic of the narrative, the \textit{korima} (fire bird) that has been previously introduced in discourse. The `speaker-proximate' demonstrative \textit{ˈnà} `this' is used to index an entity accessible to the speaker, e.g., the \textit{ejido} of Choguita in (\ref{ex: adnominal demonstrative examplesc}).

Cross-linguistically, demonstratives encode more than aspects of the spatial configuration, and their use is determined by interactional, cognitive and discourse pragmatic factors (\citealt{enfield2003demonstratives}, \citealt{hanks2005explorations}). Future research will reveal in which ways these factors may determine the use and function of demonstratives in Choguita Rarámuri.

\section{Adjectives}
\label{sec: adjectives}

Adjectival forms in Choguita Rarámuri are encoded through a small, close set of words, primary adjectives (described in §\ref{subsec: primary adectives}) and a set of morphologically derived forms from verbs through participial suffixes (described in §\ref{subsec: property concepts derived from verbs}) (\citet{islas2010caracterizacion} presents a description and analysis of adjectives and morphologically derived forms that encode property concepts in Choguita Rarámuri).

\subsection{Primary adjectives}
\label{subsec: primary adectives}

Primary adjectives in Choguita Rarámuri are monomorphemic (underived) lexical items that denote property concepts and may appear as modifiers of head nouns in noun phrases (see \chapref{chap: noun phrases}). Primary adjectives are not derived from other morphemes. A comprehensive, though non-exhaustive list of primary adjectives is provided in (\ref{ex: primary adjectives}). As discussed in §\ref{sec: adverbs}, \textit{waʔˈlû} `big, sg.' may be use adverbially to modify predicates.

%tones missing
\ea\label{ex: primary adjectives}
{Choguita Rarámuri primary adjectives}

    \ea[]{
    \doublebox{\textit{ˈkútʃ͡i}}{`little, small, pl.'}\\
}
        \ex[]{
        \doublebox{\textit{ˈtàa}}{`little, small, sg.'}\\
    }
            \ex[]{
            \doublebox{\textit{waʔˈlû}}{`big, sg.'}\\
        }
                \ex[]{
                \doublebox{\textit{(i/o)ʔˈwéri}}{`big, pl.'}\\
            }
                    \ex[]{
                    \doublebox{\textit{ˈtʃáti}}{`ugly'}\\
                }
                        \ex[]{
                        \doublebox{ \textit{waˈrìna}}{`fast, light'}\\
                    }
                            \ex[]{
                            \doublebox{\textit{naˈsína}}{`lazy'}\\
                        }
                                \ex[]{
                                \doublebox{ \textit{seˈmáti}}{`beautiful'}\\
                            }
                                    \ex[]{
                                    \doublebox{\textit{wiˈrì}}{`long'}\\
                                }
    \z
\z

%should check if number suppletion in verbs is addressed in verbal morphology chapter: pairs like die and kill, following the work by Jason Haugen - there are also instances of suppletion with nouns, child, children - should be addressed in nominal morphology chapter
As shown in these examples, some primary adjectives have suppletive forms for singular and plural (e.g., \textit{ˈkútʃ͡i} `little, small, pl.' (\ref{ex: primary adjectives}a) vs. \textit{ˈtàa} `little, small, sg.' (\ref{ex: primary adjectives}b)), but not all primary adjectives encode number distinctions (e.g., \textit{naˈsina} `lazy' (\ref{ex: primary adjectives}g)). Among a number of morphological mechanisms to encode singular-plural contrasts, number-sensitive suppletion is attested across \ili{Uto-Aztecan} for both nouns (\citealt{hill2000marked}) and verbs (\citealt{haugen2015kill}). As discussed in \chapref{chap: verbal morphology} and \chapref{chap: nominal morphology}, number suppletion is attested in Choguita Rarámuri verbs and nouns in limited instances. \footnote{Closely related \ili{Mountain Guarijío} has a small class (about a dozen lexical items) of primary adjectives that include cognate forms of the Choguita Rarámuri primary adjectives, e.g., \textit{werumá} `long', \textit{weré} `wide' (\citealt[238]{miller1996guarijio}). In contrast to Choguita Rarámuri, primary adjectives in \ili{Mountain Guarijío} do not exhibit number suppletion, and instead encode number contrasts through a productive process of prefixing reduplication (\citeyear[238]{miller1996guarijio}).}

%udefine degree morphology and gradable predicate
Primary adjectives are morphologically characterized for their ability to take on degree morphology: as shown in (\ref{ex: degree -be suffix with adjectives}), primary adjectives may attach the suffix \textit{-bê}, a stress-shifting, HL-toned suffix that encodes a high degree of the property encoded by the adjective.

%fix tone and stress

\ea\label{ex: degree -be suffix with adjectives}
{Degree morphology with primary adjectives}

    \ea[]{
    \textit{ka tʃ͡è wikaˈbê 	riˈhòoru 	ba}\\
    \gll    ka tʃ͡è 	\textbf{wiʰka-ˈbê} 	riˈhò-li 	ba\\
	        because.\textsc{neg} because.\textsc{neg} {many-more}	inhabit.\textsc{pl-pst}	\textsc{cl}	\\
    \glt    ‘Because there were not that many more people living here.’\\
    \glt    ‘Porque no había mucha más gente.'    \corpuslink{tx817[00_369-00_429].wav}{JMF tx817:00:36.9}\\
}
        \ex[]{
        \textit{ˈníma be ˈlào ˈpé a wiliˈbê iˈnáma ˈlé ˈmò ko ba}\\
        \gll    ˈní-ma be ˈlà-o ˈpé a \textbf{wili-ˈbê} iˈná-ma aˈlé ˈmò=ko 	ba \\
                \textsc{cop-fut.sg}	be	think-\textsc{ep}	just \textsc{aff}	{long-more}	go.along.\textsc{sg-fut.sg}	\textsc{dub}	2\textsc{sg.nom=emph}	\textsc{cl}\\
	    \glt    ‘I think so, you will be around (live) for a longer time.’\\
        \glt    ‘Yo creo que si, tu si vas a andar mucho tiempo.’ <FL 06 in61(704)/in>\\
    }
    \z
\z

Primary adjectives with and without degree morphology may undergo further affixation, as shown in (\ref{ex: adjectives plus addtional morphology}), where adjectives marked with \textit{-bê} (\ref{ex: adjectives plus addtional morphology}a--b) or without degree morphology (\ref{ex: adjectives plus addtional morphologyc}) may additionally attach the progressive \textit{-a} suffix followed by a nominalizer suffix (stress-shifting \textit{-ra}) in superlative forms.

%\pagebreak

\ea\label{ex: adjectives plus addtional morphology}
{Superlative forms}

    \ea[]{
    \textit{aʔˈlì ˈétʃ͡i taˈbêara biniˈlâ ko ˈá riˈpîli}\\
	\gll    aʔˈlì ˈétʃ͡i \textbf{ta-ˈbê-a-ra} bini-ˈlâ=ko ˈá riˈpî-li\\
	        and \textsc{dem} {small-more-\textsc{prog-nmlz}} small.sister-\textsc{poss}=\textsc{emph} \textsc{aff} stay-\textsc{pst}\\
	\glt    ‘And then the youngest sister stayed.’\\
	\glt    ‘Y entonces la hermana menor se quedó.’   \corpuslink{tx32[04_050-04_084].wav}{LEL tx32:04:05.0}\\
}\label{ex: adjectives plus addtional morphologya}
        \ex[]{
        \textit{sereˈbêrio 	be 	ko 	riˈwèi 			taˈbêara ko}\\
        \gll    sereˈbêrio 	be=ko 	riˈwè-i 			\textbf{ta-ˈbê-a-ra}=ko\\
	            Silverio	be=\textsc{emph}	be.named-\textsc{impf}	{small-more-\textsc{prog}}-\textsc{nmlz=emph}\\
	    \glt    ‘Silverio was the name of the youngest one.’\\
        \glt    ‘Se llamaba Silverio el más chico.’ <FL 06 in61(286)/in>\\
    }\label{ex: adjectives plus addtional morphologyb}
            \ex[]{
	        \textit{baʔaˈrîna riˈkó ˈmá waʔˈlûara kotʃ͡iˈlâ ko ʃiˈmíli ˈtʃîba ˈmèra}\\
	        \gll    baʔaˈrî-na riˈkó ˈmá \textbf{waʔˈlû-a-ra} kotʃ͡i-ˈlâ=ko siˈmí-li ˈtʃîba ˈmèra\\
	                tomorrow-\textsc{distr} \textsc{emph} already {big-\textsc{prog-nmlz}} daughter-\textsc{poss}=\textsc{emph} go.\textsc{sg-pst} goat herd.\textsc{prs}\\
	        \glt    `The next dat the oldest daughter went to take care of the goats.'\\
	        \glt    ‘Al otro día la hija más grande se fue a cuidar a las chivas.’    \corpuslink{tx32[03_576-04_050].wav}{LEL tx32:03:57.6}\\
	   }\label{ex: adjectives plus addtional morphologyc}
	\z
\z

I posit that primary adjectives in superlative forms undergo a zero conversion process, which allows them to take tense morphology before being nominalized. Alternatively, and given that very few forms present this morphological pattern, it could be hypothesized that these forms are lexicalized and not a productive sequence of morphemes available to all primary adjectives. These questions remains open for further research.

\subsection{Property concepts derived from verbs}
\label{subsec: property concepts derived from verbs}

Most property concepts in Choguita Rarámuri are derived from verbs through participial suffixes (\textit{-ame} and \textit{-kame}), which are also attested in the derivation of nouns from verbal bases (see §\ref{subsec: agentive, patientive and experiencer nominalizations}). Examples of derived adjectives are provided in (\ref{ex: derived adjectives with ame}).

%\pagebreak

%fix stress and tone marking here
\ea\label{ex: derived adjectives with ame}
{Derived adjectives}

    \ea[]{
    \doublebox{\textit{roˈsâ-kame}}{`white, sg.'}\\
}\label{ex: derived adjectives with amea}
        \ex[]{
        \doublebox{\textit{toˈsâ-kame}}{`white, pl.'}\\
    }\label{ex: derived adjectives with ameb}
            \ex[]{
            \doublebox{\textit{siˈtá-kame}}{`red, sg.'}\\
        }\label{ex: derived adjectives with amec}
                \ex[]{
                \doublebox{\textit{i-siˈrá-kame}}{‘red, pl.’}\\
            }\label{ex: derived adjectives with amed}
                    \ex[]{
                    \doublebox{\textit{naˈjú-ami}}{`sick'}\\
                }\label{ex: derived adjectives with amee}
                        \ex[]{
                        \doublebox{\textit{iˈwê-ami}}{`strong'}\\
                    }\label{ex: derived adjectives with amef}
    \z
\z

\hspace*{-1.3pt}The cognate forms of these participial suffixes (\textit{-me} and \textit{-kame}) are documented in \ili{Mountain Guarijío} (\citealt{miller1996guarijio}), and are analyzed by Miller as lacking tense marking in the case of \textit{-me} and as encoding past tense in the case of \textit{-kame} (\citeyear[180]{miller1996guarijio}). As discussed in §\ref{subsec: agentive, patientive and experiencer nominalizations}, there are clear instances of Choguita Rarámuri participial forms derived through \textit{-kame} that encode a past tense meaning and forms derived with \textit{-ame} that lack a similar tense specification. However, in the case of property concepts, such TAM contrasts are not identifiable, which suggests that \textit{-ame} and \textit{-kame} function as suppletive allomorphs in the derivation of these forms. The role of participial suffixes in deriving relative clauses in Choguita Rarámuri is discussed further in §\ref{sec: relative clauses}.

%here insert reference to number contrasts (pluractionality, cross-reference to verb chapter)

As described in §\ref{sec: pluractional marking} in \chapref{chap: nominal morphology}, Choguita Rarámuri verbs encode event plurality or pluractionality though a number of morphological mechanisms, including prefixation, consonant mutation or both prefixation and consonant mutation in a pattern of multiple exponence. This is exemplified in the derived adjectives above, where the contrast between (\ref{ex: derived adjectives with amea}) and (\ref{ex: derived adjectives with ameb}) shows a number distinction marked through mutation of the first stem consonant (/r/-/t/), while the contrast between (\ref{ex: derived adjectives with amec}) and (\ref{ex: derived adjectives with amed}) shows the number distinction marked through both prefixation and mutation of the second root consonant.

\section{Numerals}
\label{sec: numerals}

Choguita Rarámuri numerals constitute a separate word-class defined morphologically by the ability to take multiplicative/frequentative, inclusive and collective morphology. Syntactically, numerals may modify head nouns in noun phrases or head noun phrases as described in \chapref{chap: noun phrases}. The numeral system can be characterized as an arithmetic base-10 system. The basic numerals are shown in \tabref{tab:numerals}. From the lower numbers in this set (one to ten), two forms are morphologically complex, namely `eight' (\textit{oˈsá naˈó} `two times four')\footnote{The word for number `four' exhibits inter-speaker variation in its pronunciation with an optional palatal glide as an onset of the stressed syllable (\textit{naˈó} {\textasciitilde} \textit{naˈ\textbf{j}ó})} and `nine' (`not nine'), while the rest are monomorphemic. Higher numbers are all morphologically complex and involve derivation of the lower numerals in the expression of the tens-series. Speakers exhibit a strong tendency to use \ili{Spanish} loanwords for numbers above ten, including adoption of the \ili{Spanish} words \textit{ciento} and \textit{mil} for `one hundred' and`'one thousand', respectively.

\begin{table}
\caption{Choguita Rarámuri numerals}
\label{tab:numerals}

\begin{tabularx}{.6\textwidth}{ll}
\lsptoprule
\textbf{Forms}  & \textbf{Gloss} \\
\midrule
\textit{biˈlé} & `One'\\
\textit{oˈkwâ} & `Two'\\
\textit{biˈkiá} & `Three'\\
\textit{naˈó \textasciitilde naˈjó} & `Four'\\
\textit{maˈrí} & `Five'\\
\textit{uˈsàni} & `Six'\\
\textit{kiˈtʃào} & `Seven'\\
\textit{oˈsá naˈó \textasciitilde oˈsá naˈjó}  & `Eight' (`two times four') \\
\textit{ki maˈkò} & `Nine' (`not ten')\\
\textit{maˈkò} & `Ten'\\
\textit{maˈkò biˈlé} & `Eleven'\\
\textit{o-ˈsá maˈkò} & `Twenty'\\
\textit{be-ˈsá maˈkò} & `Thirty'\\
\textit{biˈlé ˈsiênto} & `One hundred'\\
\textit{biˈlé ˈmîli} & `One thousand'\\
\lspbottomrule
\end{tabularx}
\end{table}
%\hspace{3cm}

%Numerals have a distributive form, derived with suffix \textit{-na}, and a form that means 'X number of times', encoded with the suffix \textit{-sa}. The derived forms of numerals is shown in Table X.

\hspace*{-2.8pt}Numeral bases may be derived with a multiplicative/frequentative suffix (stress-shifting \textit{-sá}), which encodes the meaning `x-times'. This suffixing construction is exemplified in (\ref{ex: multiplicative numerals}).

\largerpage
\ea\label{ex: multiplicative numerals}
{Multiplicative/frequentative numerals}

    \ea[]{
    \textit{beˈsá naˈhùura ˈruá ˈá ˈrú}\\
    \gll    be-ˈsá naˈhù-ra ru-wá ˈá ˈrú\\
            three-\textsc{mltp} fall-\textsc{rep} say-\textsc{mpass} \textsc{aff} say.\textsc{prs}\\
    \glt    `They say (the earth) fell down three times.'\\
    \glt    `Dicen que se cayó (la tierra) tres veces.'  \corpuslink{in243[18_517-18_540].wav}{FLP in243:18:51.7}\\
}
%\break
        \ex[]{
        \textit{ˈmá naˈósa ku aˈwíli ˈnè}\\
        \gll    ˈmá naˈó-sa ku aˈwí-li ˈnè\\
                already four-\textsc{mltp} again dance-\textsc{pst} 1\textsc{sg.nom}\\
        \glt    `I already danced four times.'\\
        \glt    `Ya bailé cuatro veces yo.'  \corpuslink{co1136[12_406-12_423].wav}{MDH co1136:12:40.6}\\
    }
    \z
\z

Numerals may also attach an `inclusive' suffix (H-toned, stress-shifting suffix \textit{-ná}), which encodes the meaning `all of N', where N equals the total number of entities referred to by the head noun. Examples of this construction are shown in (\ref{ex: inclusive numerals}).

%marks definiteness and specificity in a numeral base, i.e., that reference is made to a specific two, three, etc. items of reference.

%fix stress and tone

\ea\label{ex: inclusive numerals}
{Inclusive numerals}

    \ea[]{
    \textit{okoˈná	siˈkâla}\\
    \gll    oko-ˈná	seˈkâ-la\\
            two-\textsc{incl}	hand-\textsc{poss}\\
    \glt    ‘in both hands’\\
    \glt    ‘en las dos manos’ < BFL 09 1:34/el >\\
}
        \ex[]{
        \textit{naʔˈsòoka koʔˈpôo okoˈná koʔˈpô biˈlá ba}\\
        \gll    naʔˈsò-wi-ka koʔ-ˈpô oko-ˈná koʔ-ˈpô biˈlá ba\\
                mix-\textsc{mpass-ger} eat-\textsc{fut.pl} two-\textsc{incl} eat-\textsc{fut.pl} indeed \textsc{cl}\\
        \glt    `One must eat mixing it, one must eat the two foods.'\\
        \glt    `Hay que comer revolviendo, hay que comer de las dos comidas.'   \corpuslink{co1136[00_292-00_319].wav}{MDH co1136:00:29.2}\\
    }
            \ex[]{
            \textit{ˈnè bikiˈná	maˈkúsarani	tʃ͡iˈwí	ko}\\
            \gll    ˈnè biki-ˈná	maˈkúsa-ra=ni tʃ͡iˈwí ko\\
		            1\textsc{sg.nom} three-\textsc{incl} fingers-\textsc{poss=1sg.nom}	hit	\textsc{emph}\\
		    \glt    ‘I hit myself in the three fingers.’\\
		    \glt    ‘Me pegué en los tres dedos.’ < BFL 09 1:34/el >\\
        }
	            \ex[]{
	            \textit{uˈsànna ˈá	maˈní baʔˈwí}\\
		        \gll    uˈsàni-na ˈá maˈní	baʔˈwí\\
		                 six-\textsc{incl} \textsc{aff}	be.liquid water\\
		        \glt    ‘There is water at six places.’\\
		        \glt    ‘En seis partes hay agua.’   < BFL 09 1:35/el >\
	       }
    \z
\z

Numeral bases may also attach the stress-neutral collective suffix (a toneless \textit{-ka} suffix) to mean that an activity is performed jointly by a given number of participants. This construction is exemplified in (\ref{ex: collective numerals}).

%fix stress and tone
%\break

\ea\label{ex: collective numerals}
{Collective numerals}

    \ea[]{
        \textit{oˈkwâka	ˈtúpisi baʔˈwí}\\
        \gll    oˈkwâ-ka ˈtú-pi-si baʔˈwí\\
		        two-\textsc{coll} bring.\textsc{imp}-\textsc{mot.imp .pl} water\\
		\glt    ‘Bring water between the two of you!’\\
		\glt    ‘¡Traigan agua entre los dos!’  < BFL 09 1:34/el >\\
}
        \ex[]{
        \textit{ˈmá biˈláti aˈbôi oˈkwâka ka aˈní tamuˈhê tʃ͡oˈkêami ˈpîri raˈwé roˈwéma umuˈkî}\\
        \gll    ˈmá biˈlá=ti aˈbôi oˈkwâ-ka  ka aˈní tamuˈhê tʃ͡oˈkêami ˈpîri raˈwé roˈwé-ma umuˈkî\\
                already indeed=1\textsc{pl.nom} \textsc{emph.pl} two-\textsc{coll} \textsc{emph} say.\textsc{prs} 1\textsc{pl.nom} bet.settlers which day race.women-\textsc{fut.sg} women\\
        \glt    `Then us two, the bet settlers, we say which day the women will race.'\\
        \glt    `Ya entre las dos decimos, nosotras las chokéami que día van a correr las mujeres.’   \corpuslink{tx19[01_236-01_323].wav}{LEL tx19:01:23.6}\\
    }
    \z
\z

\tabref{tab:numerals-derived} provides the multiplicative, inclusive and collective derivation of basic numerals (data on inclusive derivations is from < BFL 09 1:34/el >). Some derived forms are unattested in the corpus, and these are marked with ``-''.

\begin{table}
\caption{Numerals and derived forms}
\label{tab:numerals-derived}

\begin{tabularx}{.9\textwidth}{lllll}
\lsptoprule
\textbf{Forms}  & \textbf{Gloss} & \textbf{Multiplicative} & \textbf{Inclusive} & \textbf{Collective}\\
\midrule
\textit{oˈkwâ} & `Two' & \textit{o-ˈsá} & \textit{oko-ˈná} & \textit{oˈkwâ-ka}\\
\textit{biˈkiá} & `Three' & \textit{be-ˈsá} & \textit{biki-ˈná} & \textit{biˈkiá-ka}\\
\textit{naˈ(j)ó} & `Four' & \textit{naˈ(j)ó-sa} & - & \textit{naˈ(j)ó-ka}\\
\textit{maˈrí} & `Five' & \textit{maˈrí-sa} & - & \textit{maˈrí-ka}\\
\textit{uˈsàni} & `Six' & \textit{uˈsàn-sa} & \textit{uˈsàn-na} & \textit{uˈsàn-ka}\\
\textit{kiˈtʃào} & `Seven' & - & \textit{kiˈtʃào-na} & \textit{kiˈtʃáo-ka}\\
\textit{oˈsa naˈ(j)ó}  & `Eight' & \textit{oˈsa naˈ(j)ó-sa} & - & \textit{oˈsa naˈ(j)ó-ka}\\
\textit{ki maˈkò} & `Nine' & \textit{ki maˈkò-sa} & \textit{ki maˈkò-na} & \textit{ki maˈkò-ka}\\
\textit{maˈkò} & `Ten' & \textit{maˈkò-sa} & \textit{maˈkò-na} & \textit{maˈkò-ka}\\
\lspbottomrule
\end{tabularx}
\end{table}
%\hspace{3cm}

Finally, numeral bases may undergo multiple suffixation and attach both the inclusive and collective suffixes. The inclusive \textit{-ná} suffix attaches directly to the numeral root and the collective \textit{-ka} suffix attaches to the inclusive-marked numeral base. Relevant examples are shown in (\ref{ex: multiple suffixation with numeral bases}).

\ea\label{ex: multiple suffixation with numeral bases}
{Multiple suffixation with numeral bases}

    \ea[]{
    \textit{ˈápi iˈsêlikami ˈká ˈlé biˈkiánika suˈwâba maˈjôra ma}\\
    \gll    ˈápi i-ˈsêrikami ˈká aˈlé biˈkiá-ni-ka suˈwâba maˈjôra ma\\
            \textsc{sub} \textsc{pl}-governor.\textsc{pl} \textsc{cop.irr} \textsc{dub} three-\textsc{incl-coll} all mayor also\\
    \glt    `Like the governors, all three of them, all of the mayors, too.'\\
    \glt    `Así como los gobernadores, los tres, todos los mayores también.' \corpuslink{tx816[00_367-00_399].wav}{JMF tx816:00:36.7}\\
}\label{ex: multiple suffixation with numeral basesa}
        \ex[]{
        \textit{ˈétʃ͡o ˈnà biˈtʃ͡èti ʃiˈmíbiki naˈjónka ka}\\
        \gll    ˈétʃ͡i tʃ͡oˈnà biˈtʃ͡è=ti siˈmíbi-ki naˈjó-na-ka ka\\
                \textsc{dem} there there=\textsc{1pl.nom} go.\textsc{pl.pst-pst.ego} four-\textsc{incl-coll} \textsc{emph}\\
        \glt    `Over there all four of us went.'\\
        \glt    `Allí fuimos los cuatro.’ \corpuslink{tx84[04_479-04_505].wav}{LEL tx84:04:47.9}\\
    }\label{ex: multiple suffixation with numeral basesb}
    \z
\z

In the case of (\ref{ex: multiple suffixation with numeral basesa}), the numeral marked with the inclusive and the collective refers to the three governors of the local, traditional government performing an event collectively. In (\ref{ex: multiple suffixation with numeral basesb}), the numeral refers to a group of four people that includes the speaker, a referent established previously in discourse.

%The following examples show that numeral bases may add both the definiteness marker -na and the collective suffix -ka:

%[Figure out: I wrote down in my notes [09 1:35] that the numeral base has a high tone, but the derived word with the definiteness marker and the collective marker has a low tone;

%a. 	marí-ni-ka
%	b.	usá



\section{Quantifiers}
\label{sec: quantifiers}

There is a small set of quantifiers in Choguita Rarámuri. They are listed in (\ref{ex: quantifiers}). These quantifiers do not take any morphological marking, though they may be morphologically complex themselves. From this list, two quantifiers are derived from numerals: \textit{oˈkwâ} `few' is derived from the numeral `two', while \textit{iˈbiri} `some, a few' is derived from the pluractional form of the numeral one (\textit{biˈlé}) (see §\ref{subsubsec: pluractionality} for details on pluractional marking). In contrast to their numeral sources, numeral quantifiers do not take any inclusive, collective or other morphology that numerals may take (as described in §\ref{sec: numerals} above).

%fix tone marking
\ea\label{ex: quantifiers}
{Choguita Rarámuri quantifiers}

    \ea[]{
    \doublebox{\textit{wiʰˈkâ}}{`many'}\\
}
        \ex[]{
        \doublebox{\textit{oˈkwâ}}{`few'}\\
    }
            \ex[]{
            \doublebox{\textit{iˈbíli}}{`some, a few'}\\
        }
                \ex[]{
                \doublebox{\textit{suˈwâba}}{`all'}\\
            }
                    \ex[]{
                    \doublebox{\textit{siˈnêami}}{`everyone'}\\
                }
                        \ex[]{
                        \doublebox{\textit{suʔuˈma}}{`everywhere'}\\
                    }
    \z
\z

As shown in (\ref{ex: examples of quantifiers}), these quantifiers can function as modifiers of head nouns in noun phrases, and are compatible with demonstrative pronouns used as definite articles, as in (\ref{ex: examples of quantifiersb}) (the syntactic behavior of these quantifiers is further addressed in \chapref{chap: noun phrases}).

\ea\label{ex: examples of quantifiers}

    \ea[]{
    \textit{aʔˈlì ˈhípi ko biˈlá ˈmá ˈbéti wiʰˈkâ riˈhòi ˈhípi ko}\\
    \gll    aʔˈlì ˈhípi=ko biˈlá ˈmá ˈbé=ti \textbf{wiʰˈkâ} riˈhòi ˈhípi=ko\\
            and today=\textsc{emph} really already \textsc{emph}=1\textsc{pl.nom} {many} people today=\textsc{emph}\\
    \glt    ‘And today there is a lot of us people.’\\
    \glt    ‘Ahora ya habemos mucha gente.’ \corpuslink{tx12[01_005-01_059].wav}{SFH tx12:01:00.5}\\
}\label{ex: examples of quantifiersa}
        \ex[]{
        \textit{aʔˈlì ˈétʃ͡i ʃiˈnêami raˈlàmuli ko  ˈhê aˈníli}\\
        \gll    aʔˈlì ˈétʃ͡i \textbf{siˈnêami} raˈlàmuli=ko  ˈhê aˈní-li\\
                and \textsc{dem} {everyone} men=\textsc{emph} \textsc{dem} say-\textsc{pst}\\
        \glt    `And then all of the men said.'\\
        \glt    ‘Entonces todos los hombres dijeron.'  \corpuslink{tx32[06_325-06_355].wav}{LEL tx32:06:32.5}\\
    }\label{ex: examples of quantifiersb}
            \ex[]{
            \textit{suˈwâba naˈmûti raʔlaˈká ˈká ruˈwá biˈlá tʃ͡oˈnà?}\\
            \gll    \textbf{suˈwâba} naˈmûti raʔla-ˈká ˈká ru-ˈwá biˈlá tʃ͡oˈnà\\
                    {all} things buy-\textsc{ger} \textsc{cop.irr} say-\textsc{mpass} really there\\
            \glt    `Everything is bought there?'\\
            \glt    `¿Todo es comprado allá?'  \corpuslink{co1137[12_208-12_238].wav}{MDH co1137:12:20.8}\\
        }\label{ex: examples of quantifiersc}
    \z
\z

Quantifiers may also be used pronominally and fill an argument slot subcategorized for by the verb. This is shown in (\ref{ex: pronominal nature of quantifiers}), where quantifiers may fill the agent-subject role in (\ref{ex: pronominal nature of quantifiers}a--b) or the object-theme in (\ref{ex: pronominal nature of quantifiersc}).

%update tone mark
\ea\label{ex: pronominal nature of quantifiers}

    \ea[]{
    \textit{ʃiˈnêami ˈénili tʃ͡oˈnà ˈnènia}\\
    \gll    \textbf{siˈnêami} ˈéni-li tʃ͡oˈnà ˈnèni-a\\
            {everyone} go.around.\textsc{pl-pst} \textsc{dem} watch.\textsc{pl-prog}\\
    \glt    `and then they were all going around seeing'\\
    \glt    `y entonces andaban todos ahí viendo'  \corpuslink{tx32[07_115-07_149].wav}{LEL tx32:07:11.5}\\
}\label{ex: pronominal nature of quantifiersa}
        \ex[]{
        \textit{oh! ˈnápi aʔˈlì wiʰˈkâ siˈsâa biˈlá beˈlá ku ˈmèta ba}\\
        \gll    oh! ˈnápi aʔˈlì \textbf{wiʰˈkâ} si-ˈsâ beˈlá beˈlá ku ˈmèt-a ba\\
                yes \textsc{sub} then {many} arrive.\textsc{pl-cond} really really \textsc{rev} drive-\textsc{progr} \textsc{cl}\\
        \glt    `Yes, when many arrive they go back driving.'\\
        \glt    `Si, cuando llegan muchos se van todos otra vez manejando.' \corpuslink{co1136[08_084-08_115].wav}{MDH co1136:08:08.4}\\
    }\label{ex: pronominal nature of quantifiersb}
    \newpage
            \ex[]{
            \textit{ˈmáti suˈwâba ruˈjè ˈétʃ͡i ˈtîo ba}\\
            \gll    ˈmá=ti \textbf{suˈwâba} ru-ˈè ˈétʃ͡i ˈtîo ba\\
                    already=\textsc{1pl.nom} {all} say-\textsc{appl} \textsc{dem} uncle \textsc{cl}\\
            \glt    `And we told everything to the uncle.'\\
            \glt    `Y ya le dijimos todo al tío.’  \corpuslink{tx84[04_296-04_318].wav}{LEL tx84:04:29.6}\\
        }\label{ex: pronominal nature of quantifiersc}
    \z
\z

\section{Definite articles}
\label{sec: articles}

Choguita Rarámuri possesses a set of definite articles that encode a definite referent in a noun phrase. These definite articles encode number (singular vs. plural) and affective stance (positive/neutral vs. negative). The Choguita Rarámuri definite articles are listed in (\ref{ex: articles}).

\ea\label{ex: articles}
{Choguita Rarámuri definite articles}

    \ea[]{
    \textit{ˈtá} (lit. `small, \textsc{sg}') -- Singular article, positive or neutral evaluation\\
}
        \ex[]{
        \textit{ˈtí} -- Singular article, negative evaluation\\
    }
            \ex[]{
            \textit{ˈkútʃ͡i} (lit. `small, \textsc{pl}') -- Plural article, positive or neutral evaluation\\
        }
                \ex[]{
                \textit{ˈtʃéti} -- Plural, negative evaluation\\
            }
    \z
\z

As stated above, definite articles in this language may encode affective stance. ``Affect'' is defined here as: ``... a broader term than emotion, to include feelings, moods, dispositions and attitudes" \citep[7]{ochs1989language}; see also \citet{neely2019linguistic}. In the case of Choguita Rarámuri definite articles, choice of the definite article involves a positive/neutral or negative evaluation of the referent of the head noun of the noun phrase. The positive/neutral article forms are derived from the adjective `small', which also encodes number distinctions and is also used to mean literal small size. The contrast between positive and negative affective stance in definite articles is exemplified in (\ref{ex: positive negative definite articles}).

%\break

\ea\label{ex: positive negative definite articles}
{Positive vs. negative affective stance in definite articles}

    \ea[]{
    {\textsc{context:} the speaker announces the arrival of an acquaintance with whom they have a good/neutral relationship.}\\
    \textit{\textbf{ˈtá} muˈkî naˈwàli}\\
    \gll    \textbf{ˈtá} muˈkî naˈwà-li\\
            {\textsc{def.good}} woman arrive\textsc{-pst}\\
    \glt    `The woman arrived.'\\
    \glt    `La mujer llegó.'\\
}\label{ex: positive negative definite articlesa}
        \ex[]{
        {\textsc{context:} the speaker announces the arrival of an acquaintance who they dislike.}\\
         \textit{\textbf{ˈtí} muˈkî naˈwàli}\\
        \gll    \textbf{ˈtí} muˈkî naˈwà-li\\
                {\textsc{def.bad}} woman arrive\textsc{-pst}\\
        \glt    `The woman arrived.'\\
        \glt    `La mujer llegó.'\\
    }\label{ex: positive negative definite articlesb}
    \z
\z

In this pair of examples, the choice of definite article encodes affective stance toward the referent, either a positive or neutral one with the definite article \textit{ta} (\ref{ex: positive negative definite articlesa}) or a negative one with the definite article \textit{ti} (\ref{ex: positive negative definite articlesb}).

Positive/neutral evaluation articles are further exemplified in (\ref{ex: positive articles}).

\ea\label{ex: positive articles}
{Positive/neutral evaluation definite articles}

    \ea[]{
    \textit{aʔˈlì biˈlá ko ˈmá nataˈkêli ˈlé \textbf{ˈtá} koˈtʃ͡î}\\
    \gll    aʔˈlì biˈlá=ko ˈmá nataˈkê-li aˈlé \textbf{ˈtá} koˈtʃ͡î\\
            and really=\textsc{emph} already faint-\textsc{pst} \textsc{dub} \textsc{def.neutral} dog\\
    \glt    `And then the dog fainted.'\\
    \glt    `Y luego se desmayó el perro.’ \corpuslink{tx152[02_490-02_513].wav}{SFH tx152:02:49.0}\\
}\label{ex: positive articlesa}
        \ex[]{
        \textit{ˈwé ˈá ritiˈwî oˈlá \textbf{ˈkútʃi} uˈmûri ko}\\
        \gll    ˈwé ˈá ritiˈwî oˈlá \textbf{ˈkútʃi} uˈmûri=ko\\
                \textsc{int} \textsc{aff} see.\textsc{mpass} \textsc{cer} {\textsc{def.good}} great.grand.parents=\textsc{emph}\\
        \glt    `One indeed gets to see (meet) the great grandparents.'\\
        \glt    `Uno si ve (conoce) a los bisabuelos.' \corpuslink{in485[05_170-05_201].wav}{ME in485:05:17.0}\\
    }\label{ex: positive articlesb}
    \z
\z

In (\ref{ex: positive articlesa}), from the Frog Story narrative, the singular definite article \textit{ta} (from the noun phrase \textit{ta koˈtʃ͡î} `the dog') refers to an argument that was already introduced in discourse where the speaker expresses a neutral stance. In (\ref{ex: positive articlesb}), the plural definite article \textit{ˈkútʃ͡i} is used with a positive stance, encoding respect and affection towards elders (\textit{uˈmuri}, the great grand parents).

Negative evaluation articles are further exemplified in (\ref{ex: negative articles}).

\ea\label{ex: negative articles}
{Negative evaluation definite articles}

    \ea[]{
   	\textit{ke biˈlé wiˈtʃ͡íami ˈhú es \textbf{ˈtí} ritʃ͡aˈní riˈpári moˈéna ba}\\
   	\gll    ke biˈlé wiˈtʃ͡í-ame ˈhú etʃ͡i \textbf{ˈtí} ritʃ͡aˈní riˈpá-ri moˈén-a ba\\
   	        \textsc{neg} indeed fall-\textsc{ptcp} \textsc{cop.prs} \textsc{dem} {\textsc{def.bad}} giant above-\textsc{all} go.up-\textsc{prs} \textsc{cl}\\
    \glt    `He wouldn't fall the giant (richaní) when he would climb up.'\\
    \glt    `No se caía el gigante (richaní) cuando se iba para arriba.'  \corpuslink{tx43[02_527-02_599].wav}{SFH tx43:02:52.7}\\
}\label{ex: negative articlesa}
        \ex[]{
        \textit{ˈnàa akiˈná noˈrînima ˈlé \textbf{ˈtʃéti} naˈkôam pa}\\
        \gll    ˈnà akiˈná noˈrîni-ma aˈlé \textbf{ˈtʃéti} naˈkô-ame pa\\
                then over.here arrive-\textsc{fut.sg} \textsc{dub} {\textsc{def.bad}} fist.fight-\textsc{ptcp} \textsc{cl}\\
        \glt    `Then the fierce ones will arrive.'\\
        \glt    `Entonces van a llegar los que pelean.'   \corpuslink{tx221[03_178-03_238].wav}{LEL tx221:03:17.8}\\
    }\label{ex: negative articlesb}
    \z
\z

In (\ref{ex: negative articlesa}), the singular article \textit{ti} is used derogatorily in reference to \textit{ritʃ͡aˈni}, a giant in a myth of creation that kidnapped a woman and would eat children. In (\ref{ex: negative articlesb}), the plural definite article \textit{ˈtʃ͡éti} encodes a negative evaluation of a group of people who are fierce and combative.

There are also instances of \textit{ti} and \textit{ˈtʃ͡éti} attested in the corpus where these forms do not convey a negative stance, but rather seem to be deployed in a stance-neutral way. Examples of this are provided in (\ref{ex: ti and cheti with no negative stance}).

\ea\label{ex: ti and cheti with no negative stance}

    \textit{aʔˈlì biˈlá ˈhípi biˈlá ˈmá noˈkí maˈkò maˈrí bamˈpáma ˈlé \textbf{ti} tiˈwé niˈhê ˈkútʃ͡ara baˈtʃ͡áwara ba}\\
    \gll    aʔˈlì biˈlá ˈhípi biˈlá ˈmá noˈkí maˈkò maˈrí bam-ˈpá-ma aˈlé [\textbf{ti} tiˈwé] niˈhê ˈkútʃ͡a-la baˈtʃ͡á-wa-la ba\\
            and indeed now indeed already almost ten five have.birthday-\textsc{inch-fut.sg} \textsc{dub} {\textsc{def.sg}} girl \textsc{1sg.nom} child-\textsc{poss} first-\textsc{vblz-poss} \textsc{cl}\\
    \glt    `And now the girl, my daughter, will turn fifteen, my oldest child.'\\
    \glt    `Pues ahora ya mi hija ya va a cumplir quince años mi hija la más grande.'   \corpuslink{tx43[04_056-04_120].wav}{SFH tx43:04:05.6}\\

\z



\section{Adverbs}
\label{sec: adverbs}

Choguita Rarámuri has an elaborate system of adverbial elements, which are formally and semantically heterogeneous. This word class is identified primarily by the function of modifying predicates. Formally, they are characterized by a relatively unrestricted distribution and, in a number of cases, the boundary between adverbs and particles is tenuous (see, e.g., discussion of evidential and epistemic particles, addressed in §\ref{sec: particles and clitics}). Given the formal and semantic diversity that adverbs exhibit, this section addresses the morphological properties of individual adverbial subclasses in terms of their meaning. These subclasses include spatial adverbs (§\ref{subsec: spatial adverbs}), temporal adverbs (§\ref{subsec: temporal adverbs}), and manner adverbs (§\ref{subsec: manner adverbs}).

\subsection{Spatial adverbs}
\label{subsec: spatial adverbs}

The spatial adverb subclass is the most elaborate in Choguita Rarámuri, both in terms of the number of semantic contrasts encoded and the different morphological status of individual forms. This subclass also exhibits formal and semantic similarities with demonstratives (described in §\ref{sec:10:demonstratives} above), though the diachronic developments that led to these similarities is left out of the scope of this grammar. Spatial adverbs can be further classified into deictic (§\ref{subsubsec: deictic adverbs}) and directional (§\ref{subsubsec: directional adverbs}).

\subsubsection{Deictic adverbs}
\label{subsubsec: deictic adverbs}

The subclass of deictic adverbs is defined as the set of elements that situate an event with respect to the deictic center. The set of Choguita Rarámuri deictic adverbs is provided in (\ref{ex: deictic adverbs}).

%check tone marking
\ea\label{ex: deictic adverbs}
{Deictic adverbs}

    \ea[]{
    \doublebox{\textit{naˈʔî}}{`here'}\\
}\label{ex: deictic adverbsa}
        \ex[]{
        \doublebox{\textit{ˈmí}}{`there'}\\
    }\label{ex: deictic adverbsb}
            \ex[]{
            \doublebox{\textit{miˈká}}{`far'}\\
        }\label{ex: deictic adverbsc}
                \ex[]{
                \doublebox{\textit{aʔˈmí}}{`over there, far away'}\\
            }\label{ex: deictic adverbsd}
                    \ex[]{
                    \doublebox{\textit{waʔˈmí}}{`over there, far away'}\\
                }\label{ex: deictic adverbse}
                    \ex[]{
                    \doublebox{\textit{kaˈʔé}}{`very far away'}\\
                }\label{ex: deictic adverbsf}
                        \ex[]{
                        \doublebox{\textit{muˈrípi}}{`close'}\\
                    }\label{ex: deictic adverbsg}
                        \ex[]{
                        \doublebox{\textit{muruˈbê}}{`close'}\\
                    }\label{ex: deictic adverbsh}
    \z
\z

The adverbs \textit{miˈká} `far' (\ref{ex: deictic adverbsc}) and \textit{aʔˈmi, waʔˈmi} `over there, far away' (\ref{ex: deictic adverbsd}) appear to be morphologically derived from \textit{mi} `there' (\ref{ex: deictic adverbsb}). None of the deictic adverbs take any morphology. Examples of these deictic elements are provided in (\ref{ex: examples of deictic adverbs}).

\ea\label{ex: examples of deictic adverbs}

    \ea[]{
    \textit{ˈétʃ͡i baˈtʃ͡á oˈtʃêrami ˈam tʃ͡aˈnía atʃ͡aˈní ˈmí kaˈʔé kaˈnôtʃ͡i}\\
    \gll    ˈétʃ͡i baˈtʃ͡á oˈtʃê-r-ame ˈa=m tʃ=aˈní-a atʃ͡aˈní \textbf{mi} \textbf{kaˈʔé} kaˈnôtʃ͡i\\
            \textsc{dem} first grow.up-r-\textsc{ptcp} \textsc{aff=dem} that-say-\textsc{prog} say.\textsc{prs} {there} {very.far.away} Kanochi\\
    \glt    `those that lived (grew up) first there, very far away, in Kanochi, they say'\\
    \glt    `esos los que vivieron (crecieron) mucho primero allá lejos en kanochi, dicen'    \corpuslink{tx43[10_451-10_486].wav}{SFH tx43:10:45.1}\\
}\label{ex: examples of deictic adverbsa}
        \ex[]{
        \textit{ˈnápu riˈká haˈré ˈtʃ͡ó aʔˈmí raˈbô piˈrêami}\\
        \gll    ˈnápu riˈká haˈré ˈtʃ͡ó \textbf{aʔˈmí} raˈbô piˈrê-ami\\
                \textsc{sub} like some also {over.there} Rayebó live.\textsc{pl-ptcp}\\
        \glt    `Like others who live there in Rayebó.'\\
        \glt    `Así como otros los de allá de Rayebó.' \corpuslink{tx43[00_376-00_410].wav}{SFH tx43:00:37.6}\\
    }\label{ex: examples of deictic adverbsb}
            \ex[]{
            \textit{aʔˈmi biˈlá, aʔˈmí ˈwé miˈká siˈmíbira, aʔˈmí biˈléana, biˈléana}\\
            \gll    \textbf{aʔˈmí} biˈlá, \textbf{aʔˈmí} ˈwé \textbf{miˈká} siˈmíbira, \textbf{aʔˈmí} biˈléana, biˈléana\\
                    {over.there} really {over.there} very {far} go.\textsc{pl.pst} {over.there} another.place another.place\\
            \glt    `Over there very far they went, over there to another place, another place.'\\
            \glt    `Allá muy lejos se fueron, allá a otra parte, a otra parte.'  \corpuslink{tx88[01_301-01_338].wav}{LEL tx88:01:30.1}\\
        }\label{ex: examples of deictic adverbsc}
    \z
\z

As shown in these examples, deictic adverbs exhibit a very unrestricted distribution, and some may be combined with one another (e.g., \textit{mi kaˈʔé} in (\ref{ex: examples of deictic adverbsa})) and/or with other adverbs (e.g., \textit{we miˈká} in (\ref{ex: examples of deictic adverbsc})), but possible combinations and their collocations are restricted, i.e., while \textit{mi} and \textit{kaˈʔé} are frequently compounded, there are no attestations in the corpus of these adverbs with an alternative order.

\subsubsection{Directional adverbs}
\label{subsubsec: directional adverbs}

%General: this section needs to incorporate reference to Wick Miller’s analysis of spatial language in Guarijío}

Choguita Rarámuri possesses a geomorphic spatial reference system, a type of system where the anchor is defined by a feature or gradient of the environment, e.g., rivers, mountains or other inclinations (\citealt[842]{o2011spatial}).\footnote{Geomorphic spatial reference systems are subsumed under the `intrinsic' category of other spatial reference systems classifications (see \citealt{levinson1996language}).} As noted below, frames of reference where the anchor is the speaker and/or addressee (egocentric intrinsic) are also available to Choguita Rarámuri speakers. As in other geomorphic systems, the orientation of the anchor is relevant, e.g., whether an object is located upstream or downstream from a river depends on which direction the river is flowing. In the case of Choguita Rarámuri, one central anchor is a creek that cross-cuts the valley where Choguita is located. In the following examples (in (\ref{ex: upstream downstream examples})), the speaker instructs a young child to pull a door lock in opposite directions.

\ea\label{ex: upstream downstream examples}

    \ea[]{
    \textit{kaˈʔó		rakiˈbú!}\\
    \gll    kaˈʔó		rakiˈbú!\\
            upstream	pull.\textsc{imp.sg}\\
    \glt    `Pull it upstream!'\\
    \glt    ‘¡Jálalo río arriba!’   <BFL 09 2:93/el >\\
}
        \ex[]{
        \textit{ˈtû-na	 	rakiˈbú!}\\
        \gll    ˈtû-na	 	rakiˈbú!\\
                downstream-toward	pull.\textsc{imp.sg}\\
        \glt    `Pull it downstream!'\\
        \glt    ‘¡Jálalo río abajo!’    <BFL 2011 1:3/el >\\
    }
    \z
\z

Speakers anchor the events which they describe in narratives and conversations through a set of directional and locative roots and a set of directional bound markers that encode movement through a path or from a source. The Choguita Rarámuri directional system bears some resemblances with that of \ili{Mountain Guarijío}, described in \citet{miller1996guarijio}, a language spoken in a similar ecological context (the rugged, mountainous areas of the Sierra Madre Occidental in Northern Mexico).

%Notable similarities and differences between the two systems are thus noted in this section.
%Missing in this section - reference to Mountain \ili{Guarijío} - include

A non-exhaustive list of directional and locative roots is provided in (\ref{ex: directional roots}). This list includes terms that involve an egocentric frame of reference, e.g., \textit{boˈʔó} `opposite bank' (from the speaker and/or addressee perspective) in (\ref{ex: directional rootse}).

\ea\label{ex: directional roots}
{Directional and locative roots}

    \ea[]{
    \doublebox{ \textit{kaˈʔó}}{`upstream', `río arriba'}\\
}\label{ex: directional rootsa}
        \ex[]{
        \doublebox{\textit{ˈtû}}{`downstream', `río abajo'}\\
    }\label{ex: directional rootsb}
            \ex[]{
            \doublebox{ \textit{riˈpá}}{`up', `arriba'}\\
        }\label{ex: directional rootsc}
                \ex[]{
                \doublebox{\textit{reʔˈré}}{`down', `abajo'}\\
            }\label{ex: directional rootsd}
                    \ex[]{
                    \doublebox{\textit{boˈʔó}}{`opposite bank',\footnote{From the speaker and/or addressee perspective.} `por la otra banda'}\\
                }\label{ex: directional rootse}
                        \ex[]{
                        \doublebox{\textit{paˈtʃ͡á}}{`inside', `adentro'}\\
                    }\label{ex: directional rootsf}
                            \ex[]{
                            \doublebox{\textit{maˈtʃ͡í}}{`outside', `afuera'}\\
                        }\label{ex: directional rootsg}
                                \ex[]{
                                \doublebox{\textit{naˈsîpa}}{`middle', `en medio'}\\
                            }\label{ex: directional rootsh}
                                    \ex[]{
                                    \doublebox{\textit{suˈwè}}{`edge', `orilla'}\\
                                }\label{ex: directional rootsi}
                                        \ex[]{
                                        \doublebox{\textit{siˈkôtʃ͡i}}{`corner', `esquina'}\\
                                    }\label{ex: directional rootsj}
    \z
\z

The directional vocabulary and grammar of Choguita Rarámuri includes landscape terms that reflect the local ecological context of its speakers. A non-exhaus\-tive list of these terms is provided in (\ref{ex: landscape vocabulary}), with both English and {Spanish} approximate translations.

\ea\label{ex: landscape vocabulary}
{Landscape vocabulary}

    \ea[]{
    \textit{paˈní} \\
    `slight elevation'\\
    `pequeña subida'\\
}
        \ex[]{
        \textit{reʔˈré} \\
        `slight depresssion'\\
        `pequeña bajada'\\
    }
            \ex[]{
            \textit{koˈlì}\\
            `around the corner' (diagonally placed in reference to an object and out of sight)\\
            `a la vuelta (diagonalmente ubicado en referencia a un objecto y fuera de vista)'\\
        }
                \ex[]{
                \textit{naˈpûtʃ͡i} \\
                `slight land depression (descent followed by a rise)'\\
                `pequeña depresión en el terreno'\\
            }
                    \ex[]{
                    \textit{boˈʔóri/boˈʔóra}\\
                    `slight slope (rise followed by a descent)'\\
                    `pequeña subida en el terreno'\\
                }
                        \ex[]{
                        \textit{riˈʰtʃ͡í} \\
                        `ridge line'\\
                        `reliz'\\
                    }
                            \ex[]{
                            \textit{iˈpô}\\
                            `valley'\\
                            `valle'\\
                        }
                                \ex[]{
                                \textit{rokoˈáta, roˈkáata} \\
                                `canyon'\\
                                `barranca'\\
                            }
                                    \ex[]{
                                    \textit{risoˈtʃ͡í} \\
                                    `cave'\\
                                    `cueva'\\
                                }
                                        \ex[]{
                                        \textit{kaˈwì} \\
                                        `mountain, sierras, world'\\
                                        `monte, sierra, mundo'\\
                                    }
    \z
\z


Directional and landscape roots may combine with a set of directional bound morphemes that encode motion toward a goal or away from a source. An example is provided in (\ref{ex: example of -ki locative suffix}), where the \textit{-ki} suffix `on top of, on the surface of' attaches to the directional root \textit{riˈpá} `up'.

\ea\label{ex: example of -ki locative suffix}

    {\textit{“riˈpáki   risoˈt͡ʃí     biˈtê”     aˈníli}}\\
    \gll    riˈpá-ki   risoˈt͡ʃí     biˈtê”     aˈní-li\\
            up-\textsc{supe}  cave  inhabit.\textsc{prs}    say-\textsc{pst}\\
    \glt    ```He lives up on top of a cave”, he said.’\\
    \glt    ```Vive por arriba en una cueva", dijo.' \corpuslink{tx32[02_392-02_417].wav}{LEL tx32:02:39.2}\\

\z

Directional suffixes encode allative (motion towards), superessive (motion or location on the surface of), and ablative (motion away from) meanings, as well as overall orientation of referents (here labelled as adessive). \tabref{tab:directional-suffixes} lists the directional suffixes identified in the Choguita Rarámuri corpus with relevant examples.

%\pagebreak

\begin{table}
\caption{Directional suffixes}
\label{tab:directional-suffixes}
\small
\begin{tabularx}{\textwidth}{lQQ}
\lsptoprule

\textbf{Suffix} & \textbf{Meaning} & \textbf{Example}\\
\midrule
\textit{-ka} &
orientation (adessive) \linebreak
(specialized marker for directional roots) &
{
\gll \textit{reʔˈré}\textbf{\textit{-ka}} \textit{tʃ͡uˈkú}\\
down-\textsc{ad} to.be.curved.\textsc{prs}\\
\glt ‘it is down’ \\
\glt < BFL 2011 1:3>\\
}\\
\textit{{}-ki} &
superessive &
{
\gll \textit{riˈpá-}\textbf{\textit{ki}} \textit{ˈhúm-a} \textit{ikaˈní-a} \\
    up-\textsc{supe}  go\textsc{.sg-prog}  fly-\textsc{prog} \\
\glt ‘it is flying on top’ \\
\glt < BFL 2011 1:2> \\
}\\
\textit{-mi} &
allative &
{
\gll \textit{reʔˈré-}\textbf{\textit{mi}} \textit{ˈjéa-tʃ͡}\textit{i} \\
down-\textsc{all}  door-\textsc{loc} \\
\glt ‘toward under the door’ \\
\glt < BFL 09 3:66>\\
}\\
\textit{-ri} &
allative, ablative &
{
\gll \textit{riˈpá-\textbf{ri}} \textit{iʔkaˈní-si-a} \textit{aˈkíbo} \\
    up-\textsc{all}  fly-\textsc{mot-prog} go.\textsc{pst} \\
\glt ‘It flew upward.’ \\
\glt < BFL 09 1:49/el > \\
}
{
\gll \textit{reʔˈré-}\textbf{\textit{ri}} \textit{iˈnâ-ro} \\
down-\textsc{abl} go.\textsc{sg}-\textsc{mov}\\
\glt ‘It comes from  downward.’ \\
\glt < FMF 09 2:65>\\
}\\
\textit{-na} &
allative, ablative &
{
\gll \textit{riˈpá-}\textbf{\textit{na}} \\
    up-\textsc{abl}\\
\glt ‘from upward’\\
\glt < BFL 2011 1:1>\\
}
{
\gll \textit{ˈmí} \textit{reʔˈré-\textbf{na}} \textit{iˈnâ-ro} \\
    \textsc{dist}    down-\textsc{all}  go.\textsc{sg-mov} \\
\glt ‘It goes downward (downhill).’ \\
\glt < FMF 09 2:65/el >
}\\

\lspbottomrule
\end{tabularx}
\end{table}

The following examples (in (\ref{ex: allative case examples})) illustrate the allative \textit{-mi,} \textit{-ri} and \textit{-na} suffixes.

%Are the markers suppletive? Lexically conditined or...?

\ea\label{ex: allative case examples}
{Allative suffixes}

    \ea[]{
    {\textit{ˈmá     paˈt͡ʃámi  iʔˈníʃilo}}\\
    \gll    ˈmá    paˈt͡ʃá-mi  iʔˈnísi-li-o\\
            already  inside-\textsc{all} fly-\textsc{pst-ep}\\
    \glt    ‘It already flew inside.’\\
    \glt    ‘Ya voló hacia adentro.’    < BFL 09 1:49/el >\\
}
        \ex[]{
        {\textit{riʔˈlémi ˈjéat͡ʃi  ʃiˈmírilo  biˈlé  t͡ʃuruˈkí}} \\
        \gll    reʔˈlé-mi    ˈjéa-t͡ʃi  siˈmíri-li-o  biˈlé  t͡ʃuruˈkí\\
                down-\textsc{all}  door-\textsc{loc}  pass-\textsc{pst-ep}  one  bird\\
        \glt    ‘A bird passed through under the door.’\\
        \glt     ‘Un pájaro pasó por debajo de la puerta.'    < BFL 09 3:66/el >\\
    }
            \ex[]{
            {\textit{riˈpári  iʔkaˈnîsia    aˈkíbo}}\\
            \gll    riˈpá-ri  iʔkaˈnî-si-a    aˈkíbo\\
                    up-\textsc{all}  fly-\textsc{mot-prog}  left\\
            \glt    ‘It left by flying upward.’\\
            \glt    ‘Voló hacia arriba’ (lit. ‘Se fue volando hacia arriba')   \mbox{< BFL 09 1:49/el >}\\
        }
        \newpage
                \ex[]{
                {\textit{ˈmí    riʔˈléna  iˈnâlo  biˈlé  reˈhòi}}\\
                \gll    ˈmí    reʔˈlé-na  iˈnâ-li-o    biˈlé  reˈhòi\\
                         \textsc{dist}  down-\textsc{all}  go-\textsc{pst-ep}  one  man\\
                \glt     ‘A man is going down there (downhill).’\\
                \glt    ‘Un hombre va para abajo (cerro abajo).’   < FMF 09 2:65/el >\\
            }
    \z
\z

%The \textit{-na} suffix is also used with numeral bases to indicate distributed number, and is probably derived from the demonstrative \textit{na} (further discussion in \chapref{chap: morphology of small class lexical items}).

The example in (\ref{ex: ablative case example}) shows that the \textit{-ri} suffix can also be used to encode motion away from a source, an ablative meaning.

\ea\label{ex: ablative case example}

    {\textit{reʔˈréri    iˈnâlo}}\\
    \gll    reʔˈré-ri    iˈnâ-li-o\\
            down-\textsc{abl} go.\textsc{sg}-\textsc{pst-ep}\\
    \glt    `He comes from below.’\\
    \glt    ‘Viene de abajo.’    < FMF 09 2:65 >\\

\z

The following pair illustrates the difference between the description of a static location (in this instance morphologically unmarked) (\ref{ex: allative case example 2a}) and description of motion toward a goal, marked with the allative suffix (\ref{ex: allative case example 2b}).

\ea\label{ex: allative case example 2}
{Static location vs. motion toward a goal}

    \ea[]{
    \textit{maˈt͡ʃí   wiˈlí    toˈwí}\\
    \gll    maˈt͡ʃí   wiˈlí    toˈwí\\
            outside  stand.\textsc{prs}    boy\\
    \glt    ‘The boy stands outside.’\\
    \glt    ‘El niño está parado afuera.’  < FMF 09 3:26/el >\\
}\label{ex: allative case example 2a}
        \ex[]{
        \textit{maˈt͡ʃími}\\
        \gll    maˈt͡ʃí-mi\\
                outside-\textsc{all}\\
        \glt    ‘toward outside’\\
        \glt    ‘hacia afuera’    < FMF 09 3:26/el >\\
}\label{ex: allative case example 2b}
    \z
\z

%The next examples show that an allative case marked nominal requires a motion predicate (\ref{ex: allative case with motion predicate}a). As (\ref{ex: allative case with motion predicate}b) shows, a static postural predicate like \textit{wiˈlí} ‘stand’ is incompatible with a nominal marked with the allative suffix.

%\ea\label{ex: allative case with motion predicate}

 %   \ea[]{
  %  \textit{riˈpámi    ˈjéat͡ʃi ʃiˈmírilo  biˈlé  t͡ʃuruˈkí}\\
   % \gll    \ riˈpá-mi    ˈjéa-t͡ʃi  siˈmíri-li-o  biˈlé  t͡ʃuruˈkí/\\
    %        up-\textsc{all}  door-\textsc{loc}  pass-\textsc{pst-ep}  one  bird\\
    %\glt    ‘A bird passed through the top of the door’\\
    %\glt    ‘Un pájaro pasó por arriba de la puerta’\\
    %\glt    < BFL 09 3:66/el >\\
%}
 %       \ex[]{
  %      \textit{*riˈpámi    ˈjéa-t͡ʃi  wiˈlíalo    biˈlé  t͡ʃuruˈkí}\\
 %       \gll    \ ripá-mi    ˈjéa-t͡ʃi  wiˈlí-a-li-o    biˈlé  t͡ʃuruˈkí/\\
  %              up-\textsc{all}    door-\textsc{loc}  stand-a-\textsc{pst-ep}  one  bird\\
   %     \glt    ‘A bird stood at the top of the door’\\
    %    \glt    ‘Un pájaro se paró arriba de la puerta’\\
     %   \glt    < BFL 09 3:66/el >\\
%}
%     \z
% \z

The superessive \textit{-ki} suffix and the allative/ablative \textit{-na} suffix may co-occur in some forms where motion involves motion on top of a surface and towards a goal or away from a source. This is exemplified in (\ref{ex: -ki-na affix sequence example}).


\ea\label{ex: -ki-na affix sequence example}

    \ea[]{
    {\textit{ˈtûkuna}}\\
    \gll    ˈtû-ki-na\\
            downriver-\textsc{supe-all}\\
    \glt    ‘Going downward (downriver)’\\
    \glt    ‘De bajada (río abajo)’   < FMF 09 2:65/el >, \corpuslink{in485[07_314-07_350].wav}{ME in485:07:31.4}\\
}\label{ex: -ki-na affix sequence examplea}
        \ex[]{
        \textit{riˈpákina ˈku tʃ͡uˈkúl ti tʃ͡imoˈlí ko iʔˈnèka tʃ͡ú oˈlása}\\
        \gll    riˈpá-ki-na ˈku tʃ͡uˈkú-li ti tʃ͡imoˈlí=ko iʔˈnè-ka t͡ʃú oˈlá-sa\\
                up-\textsc{supe-all} \textsc{rev} to.be.curved-\textsc{pst} \textsc{def.bad} squirrel=\textsc{emph} watch-\textsc{ptcp} what do-\textsc{cond}\\
        \glt    `From up above the squirrel was watching, what will it do.'\\
        \glt    `Desde por arriba estaba la ardilla viendo, a ver qué estaba haciendo.'  \corpuslink{tx191[03_376-03_412].wav}{BFL tx191:03:37.6}\\
    }\label{ex: -ki-na affix sequence exampleb}
    \z
\z

As shown in (\ref{ex: -ki-na affix sequence examplea}), the superessive \textit{-ki} suffix may undergo round vowel harmony in specific phonological contexts (details of the process of round vowel harmony is discussed in §\ref{subsubsec: round harmony}).

\subsection{Temporal adverbs}
\label{subsec: temporal adverbs}

Temporal adverbs can be classified as deictic or non-deictic. Deictic temporal adverbs are defined as those that situate the event in time with respect to the speech event or another reference point. A list of deictic temporal adverbs is provided in (\ref{ex: deictic temporal adverbs}).

\ea\label{ex: deictic temporal adverbs}
{Deictic temporal adverbs}

\begin{tabular}{llll}
    a. & {\textit{ˈhípi}}&{`today, now, nowadays'}& {`hoy, ahora, estos tiempos'}\\
    b. & {\textit{baʔaˈrî}}&{`tomorrow'}& {`mañana'}\\
    c. & {\textit{raˈpâko}}&{`yesterday'}& {`ayer'}\\
    d. &{\textit{tʃ͡aˈbè}}&{`long ago'}& {`antes, hace mucho tiempo'}\\
    e. & {\textit{kuˈrí}}&{`recently'}& {`apenas, hace poco'}\\
\end{tabular}
    \z

In contrast to deictic temporal adverbs, non-deictic temporal adverbs situate events within periods of time, whether concerning the diurnal cycle (\ref{ex: non-deictic temporal adverbs}) or the seasonal cycle (\ref{ex: seasonal temporal adverbs}).\footnote{The form \textit{biˈʔà roˈkò} `early morning' (lit. `early night') in (\ref{ex: non-deictic temporal adverbs}d) refers to the time of the day before sunrise.)}

\ea\label{ex: non-deictic temporal adverbs}
{Non-deictic temporal adverbs - diurnal cycle}
\setlength{\tabcolsep}{4pt}
\begin{tabular}{llll}
    a.&  {\textit{roˈkò}}&`night, nightime'&`noche'\\
    b. & {\textit{raˈwé}}&`day, daytime'& `día'\\
    c. & {\textit{biˈʔà}}&{`early, morning time'}& {`temprano, de mañana}\\
    d. & {\textit{biˈʔà roˈkò}}&{`early morning'}& {`madrugada'}\\
    e. & \textit{aʔˈlì}&{`late in the day'}&{`tarde en el día'}\\
    \end{tabular}
\begin{tabular}{llll}
    f. & \textit{(pe) ˈtèri}&{`a (little) while'}&{`un rato corto'}\\
    g. & \textit{naˈsîpa raˈwé}&{`noon' (lit. `half day')}& {`mediodía'}\\
    h. & \textit{naˈsîpa roˈkò}&{`midnight' (lit. `half night')}& {`medianoche'}\\
\end{tabular}
    \z

\ea\label{ex: seasonal temporal adverbs}
{Non-deictic temporal adverbs - seasonal cycle}
\setlength{\tabcolsep}{2pt}
\begin{tabular}{lllll}
     a.& \textit{baˈrà} &  `rainy season' & `tiempo de lluvias' & \corpuslink{tx904[02_024-02_035].wav}{GFM tx904:2:02.4}\\
     b.& \textit{roˈmò}&{`cold season'}& {`tiempo de frío'}&\corpuslink{tx904[01_388-01_407].wav}{GFM tx904:1:38.8}\\
     c. & \textit{kuˈwè}&{`hot/dry season'}& {`tiempo de calor/seco'}& \corpuslink{tx904[01_470-01_494].wav}{GFM tx904:1:47.0}\\
\end{tabular}
    \z

Some temporal adverbs are morphologically related to other word classes: this is the case with \textit{baˈrá} `rainy season' (\ref{ex: seasonal temporal adverbs}a), which is related to the verb \textit{baˈrâmi} `be thirsty'. None of the temporal adverbs takes any morphological marking.


\subsection{Manner adverbs}
\label{subsec: manner adverbs}

Choguita Rarámuri has a small set of manner adverbs, some of which are exemplified in (\ref{ex: manner adverbs}). As described in §\ref{sec: adjectives}, the root \textit{waʔˈrû} may function both as an adverb modifying predicates (`greatly, a lot') or as an adjective modifying nouns (`big, sg.').\footnote{Some of the forms in this list are homophonous with forms in other word classes, such as \textit{ˈtʃ͡áti}, which is homophonous with adjective \textit{ˈtʃ͡áti} `ugly'.}

\ea\label{ex: manner adverbs}
{Manner adverbs}

\begin{tabular}{llll}
    a. & {\textit{(k)aʔˈlá}}&{`well'}& {`bien'}\\
    b. & {\textit{saˈpù}}&{`fast'}& {`rápido'}\\
    c. & {\textit{kiˈrì}}&{`slowly, peacefully'} &{`despacio, tranquilamente'}\\
    d. & {\textit{biˈlá}}&{`truly'} &{`verdaderamente'}\\
    e. & {\textit{(ˈétʃ͡i) riˈká}}&{`like this, thus'}&{`así'}\\
    f. & {\textit{raˈsíra}}&{`strongly'}&{`fuerte, recio}\\
    g. & {\textit{tʃ͡áti}}&{`badly'}&{`mal'}\\
    h. & {\textit{waʔˈrû}}&{`greatly, a lot'}&`{mucho'}\\
    i. & {\textit{ˈwé}}&{`very'}& {`muy'}\\
\end{tabular}
    \z

Manner adverbs do not take any inflectional or derivational morphology, nor do they appear to be morphologically derived from other word classes. As discussed in \chapref{chap: noun phrases} (§\ref{subsec: adjectives in noun phrases}), adverbs may also modify adjectives in noun phrases.

\section{Discourse particles and enclitics}
\label{sec: particles and clitics}

From all the minor word classes, discourse particles and enclitics constitute a set of heterogeneous word classes that are closed and morphologically simple, bearing no inflection or derivation. Each class of particles is composed of fewer than a dozen members per class. Particles have a wide range of functions and meanings. Some, but not all, particles are also characterized by being phonologically reduced, and not being subject to the minimality requirement that holds for verbs (see §\ref{sec: vowel length, stress and minimality effects} for discussion of minimality effects in Choguita Rarámuri).

\subsection{Interjections}
\label{subsec: interjections}

Interjections are characterized by their ability to be uttered in isolation as a complete utterance, and as not belonging to any other word class. The inventory of interjections attested in the Choguita Rarámuri corpus is provided in (\ref{ex: interjections}) with approximate English and \ili{Spanish} translations (some forms from this set are loanwords from \ili{Spanish}, e.g. (\ref{ex: interjections}f)).

\ea\label{ex: interjections}
{Interjections}
\setlength{\tabcolsep}{3pt}
\begin{tabular}{llll}
    a. & {\textit{uˈrí}}&{`yes'}&`sí'\\
    b. & {\textit{ˈkuíra (ba)}}&{`hello'}&`hola'\\
    c. & {\textit{niˈbí}}&{`look!' (entails surprise)}&`¡mira! (con sorpresa)'\\
    d. & {\textit{ˈhémi}}&{`get away (to dog)'}& `¡fuera! (a un perro)'\\
    e. & {\textit{(ho) ˈúʔnuko (le)}}&{`all right'}& {`órale'}\\
    f. & {\textit{aˈriôsi ba}}&{`goodbye'}& {`adiós'}\\
    g. & {\textit{ˈné}}&{`you don't say!'}& {`a poco'}\\
    h. & {\textit{maˈtêtala ba}}&{`thank you'}&{`gracias'}\\
\end{tabular}
    \z

Interjections tend to appear as independent clauses and appear separated from the rest of a larger utterance through a pause (indicated as <...> pr <,>). This is exemplified in (\ref{ex: interjection examples 1}) (which represents a section of a dialogue) and (\ref{ex: interjection examples 2}).

%fix this example
\ea\label{ex: interjection examples 1}

    \ea[]{
    [ME]: \textit{boniˈlâ...}\\
    \gll    boni-ˈlâ\\
            be.younger.brother-\textsc{poss}\\
    \glt    `younger brother'\\
    \glt    `hermano menor'\\
}\label{ex: interjection examples 1a}
        \ex[]{
        [SFH]: \textit{boˈnèsa ba?}\\
        \gll    boˈnè-sa ba?\\
                be.younger.brother-\textsc{cond} \textsc{cl}\\
        \glt    `As if he were a younger brother?'\\
        \glt    `¿Como si fuera hermano menor?'\\
    }\label{ex: interjection examples 1b}
            \ex[]{
            [ME]: \textit{\textbf{uˈrí}, boniˈlâ ˈnísa ˈlá ba ˈni}\\
            \gll    uˈrí, boni-ˈlâ ˈní-sa oˈlá ba ˈni\\
                    yes be.younger.brother-\textsc{poss} \textsc{cop-cond} \textsc{cer} \textsc{cl} \textsc{emph}\\
            \glt    `Yes, as if he were a younger brother indeed.'\\
            \glt    `Si, si fuera hermano menor, asi es.' \corpuslink{in485[02_229-02_252].wav}{ME in485:02:22.9}\\
        }\label{ex: interjection examples 1c}
    \z
\z

\ea\label{ex: interjection examples 2}

    \textit{ˈmá ˈpé tʃ͡iˈkí ˈhú pa ... \textbf{maˈtêtala ba}}\\
    \gll    ˈmá ˈpé tʃ͡iˈkí ˈhú pa maˈtêtala ba\\
            already just that.much \textsc{cop.prs} \textsc{cl} thank.you \textsc{cl}\\
    \glt    `That is all (lit. that is the little bit there is) thank you.'\\
    \glt    `Ya es todo ... gracias.'  \corpuslink{tx88[03_474-03_507].wav}{LEL tx88:03:47.4}\\

\z

In both of these examples, the interjections are separated from the following (\ref{ex: interjection examples 1c}) or preceding (\ref{ex: interjection examples 2}) clause by a pause and an intonational break.

\subsection{Connective particles}
\label{subsec: connective particles}

Connective particles are defined as particles that serve the function of connecting sentences within discourse. A list of monomorphemic connective particles in Choguita Rarámuri is provided in (\ref{ex: connective particles}).

\ea\label{ex: connective particles}
{Monomorphemic connective particles}

\begin{tabular}{llll}
    a. & {\textit{aʔˈlì}}&{`and, then, so'}& {`y, entonces'}\\
    b. & {\textit{ˈnà}}&{`then'}& {`entonces'}\\
    c. & {\textit{ˈkíti}}&{`that is why'}& {`because', `por eso, porque'}\\
\end{tabular}
    \z

These connective particles do not take any morphology and are analyzed here as basic, but other connectives may be morphologically complex. Among the basic connectives, \textit{aʔˈlì} `and, then, so' (\ref{ex: connective particles}a) (derived from the temporal adverb \textit{aʔˈlì} `late in the day', `tarde en el día') may combine with other morphemes to form complex connectives. These and other morphologically complex connectives are exemplified in (\ref{ex: complex connectives}).

\ea\label{ex: complex connectives}
{Complex connectives}

\begin{tabular}{llll}
    a. & {\textit{aʔˈlì biˈlá}}&{`then indeed'}&{`entonces verdaderamente'}\\
    b. & {\textit{aʔˈlì tʃ͡iˈhônsa}}&{`and then, so'}&{`entonces'}\\
    c. & {\textit{ka ˈtʃ͡è}}&{`because'}& {`porque'}\\
    d. & {\textit{naˈlìna}}&{`but'}&{`pero'}\\
\end{tabular}
    \z

Connectives, whether monomorphemic or complex, occur in between the two sentences they connect. This is illustrated in the examples in (\ref{ex: connective examples 1}), (\ref{ex: connective examples 2}) and (\ref{ex: connective examples 3}).

\ea\label{ex: connective examples 1}

    \ea[]{
    \textit{ˈnà napaˈwía noˈkáli aˈlé ˈet͡ʃi ˈnà biˈlé reˈhòi aʔˈlì biˈlé biˈlé t͡ʃaˈbôt͡ʃi}\\
    \gll    ˈnà napaˈwí-a noˈk-á-li aˈlé ˈet͡ʃi ˈnà biˈlé reˈhòi aʔˈlì biˈlé biˈlé t͡ʃaˈbôt͡ʃi sî\\
            \textsc{dem} get.together-\textsc{prog} move-\textsc{tr-pst} \textsc{sub} \textsc{dem} \textsc{dem} one man and one one mestizo also\\
    \glt    ‘A (Rarámuri) man and a mestizo man got together.’\\
    \glt    ‘Se juntaron un hombre (rarámuri) y un mestizo.’ < SFH tx choma (2) >
}
        \ex[]{
        \textit{aʔˈlì beˈlá ko beˈlá ko maˈníìla ruˈámi ka}\\
        \gll    aʔˈlì beˈlá=ko beˈlá=ko maˈní-ì-la ru-ˈá=mi ka\\
                and really=\textsc{emph} really=\textsc{emph} be.liquid-\textsc{appl-rep} say-\textsc{mpass=dem} \textsc{cop.irr}\\
        \glt    ‘And then it is said they were given something (a liquid).’\\
        \glt    ‘Y entonces dicen que les pusieron algo (un líquido).’   < SFH tx choma (3) >
    }

    \z
\z

\ea\label{ex: connective examples 2}

    \ea[]{
    \textit{ˈkíti beˈlá we}\\
    \gll    ˈkíti beˈlá we\\
            that.is.why really \textsc{int}\\
    \glt    ‘that is why’\\
    \glt    ‘por eso’  < SFH tx choma (24) >
}
        \ex[]{
        \textit{aʔˈlì tʃ͡iˈhônsa ˈét͡ʃi reˈhòi baˈhîsa ˈká ˈét͡ʃi  t͡ʃoʔˈmá}\\
        \gll    aʔˈlì tʃ͡iˈhônsa ˈét͡ʃi reˈhòi baˈhî-sa ˈká ˈét͡ʃi t͡ʃoʔˈmá\\
                and thene \textsc{dem} man drink-\textsc{cond} \textsc{irr} \textsc{dem} snot\\
        \glt    ‘and if the man had drunk the snot...’\\
        \glt    ‘y si el hombre hubiera tomado el moco...’  < SFH tx choma (25) >\\
    }
    \z
\z

\ea\label{ex: connective examples 3}

    \ea[]{
    \textit{biˈlé naˈmûti tʃ͡iˈká noˈrîno, waʔˈlû roˈsâkame}\\
    \gll    biˈlé naˈmûti tʃ͡iˈká noˈrîno, waʔˈlû roˈsâ-kame\\
            one thing over.here arrive big.\textsc{sg} be.white-\textsc{pst.ptcp}\\
    \glt    ‘One thing came, a white, big thing.’\\
    \glt    ‘Una cosa vino, una cosa blanca, grande.’ \corpuslink{tx_rosakame[00_137-00_156].wav}{ROF tx\_rosakame:00:13.7}\\
}
        \ex[]{
        \textit{aʔˈlì tamuˈhê ˈpé maˈhâli ko}\\
        \gll    aʔˈlì tamuˈhê ˈpé maˈhâ-li=ko\\
                then 1\textsc{pl.nom} little get.scared-\textsc{pst}=\textsc{emph}\\
        \glt    ‘And then we got scared.’\\
        \glt    ‘Y entonces nosotros nos asustamos.’    \corpuslink{tx_rosakame[00_171-00_201].wav}{ROF tx\_rosakame:00:17.1}\\
    }
    \z
\z

The interpretation and function of connectives in complex clauses is discussed in \chapref{chap: clause combining in complex sentences}.

%One particular morphologically complex connective, \textit{aʔˈlìna} 'but', is disjunctive in interpretation, but contains the basic connective \textit{aʔˈlì} which functions as a conjunction. The disjunctive interpretation of \textit{aʔˈlìna} is exemplified in (\ref{ex: disjunctive connective}).

%\ea\label{ex: disjunctive connective}

%\textit{aʔˈlìna muˈki ko ke na naˈkiwiri}
%but woman EMPH NEG that let-PST
%‘But the woman didn’t let him (do it)’
%‘Pero la mujer no lo dejó’
%\corpuslink{tx5[01_183-01_213].wav}{LEL tx5:01:18.3}

%\z

%different functions in different types of complex clause constructions
% S. Murray 2017 pn connectives in \ili{Cheyenne} - some basic forms may function as conjunctions, but when combined with other morphemes they can encode disjunction (e.g., alína)

\subsection{Negative particles}
\label{subsec: negative particles}

Choguita Rarámuri possesses two negative particles, \textit{ke}, a clausal negator (which can be used as interjection), and \textit{ˈkíti}, a prohibitive (negative imperative) particle. Negative particles may combine with other morphemes, yielding morphologically complex negative markers: in Choguita Rarámuri, the clausal negator \textit{ke} combines with other morphemes. The set of negative particles and complex negative markers available in the language are provided in \tabref{tab:negative-constructions}, with their gloss, function and approximate translation.

%\pagebreak

\begin{table}
\caption{Negative markers}
\label{tab:negative-constructions}

\begin{tabularx}{\textwidth}{llQQ}
\lsptoprule
\textbf{Form} & \textbf{Gloss} & \textbf{Function} & \textbf{Translation}\\
\midrule
\textit{ke} & \textsc{neg} & interjection, clausal negation & ‘No’\\
\textit{ˈkíti} & \textsc{proh} & prohibitive (negative imperative) & `Don't!'\\
\textit{ke ˈtâsi} & \textsc{neg neg} & interjection, clausal negation & `No’\\
\textit{ˈpé ke biˈlé} & just \textsc{neg} one & emphatic interjection & `Not at all!'\\
\textit{ka ˈt͡ʃè} & \textsc{neg.irr} again & clausal negation & `Not again/anymore’\\
\textit{ke/ˈtâsi t͡ʃo} & \textsc{neg} yet & clausal negation & ‘Neither’\\
\textit{ke/ˈtâsi biˈlé} & \textsc{neg} one & clausal negation, constituent neg. & ‘Nothing at all’,

‘No single’\\
\textit{ni biˈlé} & nor one & constituent neg. & ‘Nor any’\\
\lspbottomrule
\end{tabularx}
\end{table}
\hspace{3cm}

The syntactic properties and functions of each of these negative particles and negative constructions are discussed in further detail in §\ref{sec: negative constructions}.

\subsection{Epistemic particles and enclitics}
\label{subsec: epistemic particles}

%Following Ochs (1996), we understand epistemic stance as a central meaning component of social acts and social identities that refers to knowledge or belief vis-à-vissome focus of concern including degrees of certainly of knowledge, degrees of commitment to truth of propositions, and sources of knowledge among other epistemic qualities.

%from verbal morphology chapter

As discussed in §\ref{sec: the verbal complex clitics and modal particles}, Choguita Rarámuri possesses modality particles that encode epistemic stance. Epistemic modality is defined here as the expression of the degree of certainty speakers have towards the actuality of an event or its potential of occurring at a time later than the speech event. A list of particles that encode epistemic stance is provided in (\ref{ex: epistemic modality particles list}).

\ea\label{ex: epistemic modality particles list}
{Epistemic modality particles}

    \ea[]{
    \doublebox{\textit{(a)ˈlé}}{`dubitative'}\\
}
        \ex[]{
        \doublebox{\textit{(o)ˈlá}}{`certainty'}\\
    }
            \ex[]{
            \doublebox{\textit{ˈá}}{`indeed, truly'}\\
        }
                \ex[]{
                \doublebox{\textit{(a)ˈjéna}}{`indeed, truly'}\\
            }
                    \ex[]{
                    \doublebox{\textit{biˈlá}}{`indeed, truly'}\\
            }
    \z
\z

The dubitative \textit{aˈlé} (with reduced form \textit{lé}) and certainty \textit{oˈlá} (with reduced form \textit{lá}) markers are frequently attested post-verbally. Their different epistemic functions can be appreciated in the contrast in meaning they create when associated to predicates inflected for future tense: in (\ref{ex: epistemic modality markersa}), the dubitative particle encodes lack of knowledge on the speaker's part, while in (\ref{ex: epistemic modality markersb}), the certainty marker conveys a high degree of commitment from the speaker about the likelihood the event encoded by the verbal predicate will take place in a future time.\footnote{As described in §\ref{sec: the verbal complex clitics and modal particles}, epistemic particles are vowel initial and may coalesce with the final vowel of the future suffix.}

\ea\label{ex: epistemic modality markers}
{Epistemic modality markers }

    \ea[]{
    \textit{ˈnârma ˈlé}\\
    \gll    ˈnâri-ma aˈlé\\
            ask-\textsc{fut.sg} \textsc{dub}\\
    \glt    ‘(He) will probably ask.’\\
    \glt    ‘Probablemente va a preguntar.’ < BFL 05 1:152/el >\\
}\label{ex: epistemic modality markersa}
        \ex[]{
        \textit{ˈnârmo ˈlá}\\
        \gll    ˈnâri-ma oˈlá\\
                ask-\textsc{fut.sg} \textsc{cer} \\
        \glt    ‘S/he will definetly ask.’\\
        \glt    ‘Seguramente que va a preguntar.’  < BFL 05 1:152/el >\\
    }\label{ex: epistemic modality markersb}
    \z
\z

%move here example of ungrammatciality of using ola with weather predicates

While frequently attested with verbs inflected with future tense, epistemic markers are not obligatory in these contexts. Lack of an epistemic marker involves a neutral interpretation with respect to the speaker’s commitment to the likelihood the event will take place in these contexts. Examples of this are shown in (\ref{ex: no epistemic marker}) (relevant predicates lacking epistemic markers are highlighted in boldface).

\ea\label{ex: no epistemic marker}

    \ea[]{
        \textit{“ˈmán ku \textbf{ʃiˈmêa} ˈhípi ko, ˈmá ˈwé miˈká iˈnârtʃ͡ani”  ˈhê aˈníli ˈétʃ͡i tiˈwé ko}\\
        \gll    “ˈmá=ni ku si-ˈmêa ˈhípi=ko, ˈmá ˈwé miˈká iˈnâr-tʃ͡ani” ˈhê aˈní-li ˈétʃ͡i tiˈwé=ko\\
                already=\textsc{1sg.nom} \textsc{rev} go.\textsc{sg-fut.sg} now\textsc{=emph} already \textsc{int} far go.around-\textsc{ev} \textsc{dem} say-\textsc{pst} \textsc{dem} young.woman=\textsc{emph}\\
        \glt    ```I'm already going now, it already sounds like they're far away" said the young woman.'\\
        \glt    ```Ya me voy a ir, ya se oye muy lejos” dijo la muchacha.’ \corpuslink{tx32[10_345-10_394].wav}{LEL tx32:10:34.5}\\
    }\label{ex: no epistemic markera}
%    \pagebreak
            \ex[]{
            \textit{aʔˈlì ˈnè ko ˈmá bitiˈtʃ͡í ˈá buˈjèa aˈtí, \textbf{noˈrînima} ˈétʃ͡i biˈléara tʃ͡oˈkêami taˈmí ruˈwèʃia}\\
            \gll    aʔˈlì ˈnè=ko ˈmá bitiˈtʃ͡í ˈá buˈjè-a aˈtí, noˈrîna-ma ˈétʃ͡i biˈléara tʃ͡oˈkê-ame taˈmí ru-ˈwè-si-a\\
                    and 1\textsc{sg.nom}=\textsc{emph} already house \textsc{aff} wait-\textsc{prog} \textsc{cop.sg} arrive-\textsc{fut.sg} \textsc{dem} another settle.bet-\textsc{ptcp} \textsc{1sg.acc} tell-\textsc{appl-mot-prog}\\
            \glt    ‘And then I wait for her in the house, the other bet settler arrives to tell me.’\\
            \glt    ‘Y entonces yo ya la espero en la casa, viene la otra chokéami a decirme.’  \corpuslink{tx19[01_179-01_236].wav}{LEL tx19:01:17.9}\\
        }\label{ex: no epistemic markerb}
                \ex[]{
                \textit{aʔˈlì tʃ͡iˈhônsa ˈmám \textbf{aˈtʃèma} aˈsûkar aʔˈlì aˈrîna}\\
                \gll    aʔˈlì tʃ͡iˈhônsa ˈmá=mi aˈtʃ͡-è-ma aˈsûkar aʔˈlì aˈrîna\\
                        and then already=\textsc{2sg.nom} add-\textsc{appl-fut.sg} sugar and flour\\
                \glt    `And then you add to it sugar and flour.'\\
                \glt    `Y entonces ya le echas la azúcar y la harina.' \corpuslink{tx60[02_081-02_115].wav}{BFL tx60:02:08.1}\\
            }\label{ex: no epistemic markerc}
    \z
\z

In (\ref{ex: no epistemic markera}), the main predicate inflected for future tense that lacks an epistemic marker is reported speech, while in (\ref{ex: no epistemic markerb}) and (\ref{ex: no epistemic markerc}), the relevant future predicates with no epistemic markers are part of procedural texts, a description of the procedure for settling a bet for a women's race and a recipe for making corn beer, respectively.

% example c shows coordination with no conjunctive particle

The dubitative marker is also attested with verbs inflected for other tense\slash aspect\slash mood specifications, and may have specific modal interpretations when combined with certain inflection values. When combined with past or present progressive inflected verbs, the dubitative particle encodes lack of certainty from the speaker about a past event (\ref{ex: dubitative with past}a--b) or a current event (\ref{ex: dubitative with pastc}).

\ea\label{ex: dubitative with past}

    \ea[]{
    \textit{ah! ˈmí \textbf{ruˈwèli aˈlé} ko}\\
    \gll    ah ˈmí ru-ˈè-li aˈlé=ko\\
            oh \textsc{2sg.acc} say-\textsc{appl-pst} \textsc{dub=emph}\\
    \glt    `Oh! They told me, I think.'\\
    \glt    `Ah! Me dijeron, creo.'  \corpuslink{in243[08_079-08_094].wav}{FLP in243:08:07.9}\\
}\label{ex: dubitative with pasta}
        \ex[]{
        \textit{raˈlàmuli \textbf{umuˈrútili aˈlé} ˈku}\\
        \gll    raˈlàmuli u-muˈrúti-li aˈlé ˈku\\
                men \textsc{pl}-carry.\textsc{pl-pst} \textsc{dub} \textsc{emph}\\
        \glt    `He took men (in the truck), I think.'\\
        \glt    `Llevó señores yo creo (en la troca).' \corpuslink{co1136[08_367-08_394].wav}{MDH co1136:08:36.7}\\
    }\label{ex: dubitative with pastb}
            \ex[]{
            \textit{ke piˈlá biˈlé \textbf{aˈwìa aˈlé}, ˈkíti ke biˈlé uˈkú ba}\\
            \gll    ke biˈlá biˈlé aˈwì-a aˈlé, ˈkíti ke biˈlé uˈkú ba\\
                    \textsc{neg} really one dance-\textsc{prog} \textsc{dub} that.is.why \textsc{neg} one rain \textsc{cl}\\
            \glt    `They don't dance at all, I think, that is why it won't rain at all.'\\
            \glt    `Es que no bailan, por eso no llueve, yo creo.'    \corpuslink{co1136[06_204-06_227].wav}{MDH co1136:06:20.4}\\
        }\label{ex: dubitative with pastc}
    \z
\z

When combined with a verb inflected for conditional mood, the dubitative particle confers a deontic reading: in (\ref{ex: deontic dubitativea}), the speaker states that an event (giving water to the horse) is an obligation, while in (\ref{ex: deontic dubitativeb}), the speaker encourages his children to not forget rituals and carry them to future generations.

\ea\label{ex: deontic dubitative}

    \ea[]{
    \textit{baʔˈwí maˈnèsa aˈlé ra ba}\\
    \gll    baʔˈwí maˈn-è-sa aˈlé ra ba\\
            water be.located.water-\textsc{appl-cond} \textsc{dub} ra \textsc{cl}\\
    \glt    `We must give water to the horse.'\\
    \glt    `Hay que darle agua al caballo.'   \corpuslink{co1236[06_213-06_227].wav}{JLG co1236:06:21.3}\\
}\label{ex: deontic dubitativea}
        \ex[]{
        \textit{ke biˈlá ˈtâsi wikaˈwâsa ˈlé naˈlìna ˈhípi tamuˈhê...}\\
        \gll    ke biˈlá ˈtâsi wikaˈwâ-sa aˈlé naˈlìna ˈhípi tamuˈhê\\
                \textsc{neg} really \textsc{neg} forget-\textsc{cond} \textsc{dub} but today \textsc{1pl.nom}\\
        \glt    `(we) should not forget, but nowadays we...'\\
        \glt    `no debemos olvidar nosotros, nomás que ahora nosotros ...' \corpuslink{tx475[08_330-08_368].wav}{SFH tx475:08:33.0}\\
    }\label{ex: deontic dubitativeb}

    \z
\z


% this may belong in a syntax chapter

%The dubitative \textit{(a)ˈle} and certainty \textit{(o)ˈla} markers are frequently attested in the immediately post-verbal position, as shown in the examples so far, but these are also attested in other positions within the verb phrase, as exemplified in (\ref{ex: non post-verbal collocation of ale}).

%\ea\label{ex: non post-verbal collocation of ale}

 %   \ea[]{
  %  \textit{amiˈnábi ˈá ˈsirpo ˈtʃ͡ó \textbf{aˈle} taˈmò ko oˈtʃ͡êrʃia ˈtʃ͡ó ba ˈne}\\
 %   \gll    amiˈnábi ˈá ˈsir-po ˈtʃ͡ó aˈle taˈmò=ko oˈtʃ͡êr-ʃi-a ˈtʃ͡ó ba ˈne\\
  %          little.by.little aff become.pl-fut.pl also dub 1pl.nom=emph grow.old-mot-prog also cl emph\\
   % \glt    'Little by little we are growing older, I think'\\
   % \glt    'Poco a poquito nos vamos haciendo viejos, yo creo'\\
    %\glt    \corpuslink{tx43[13_255-13_291].wav}{SFH tx43:13:25.5}\\
%}

%     \z
% \z

Like the dubitative marker, the certainty particle \textit{oˈlá} is attested with predicates inflected with a range of tense/aspect/mood values. This is exemplified in (\ref{ex: ola examples with TAM}).

\newpage
\ea\label{ex: ola examples with TAM}

    \ea[]{
    \textit{“ˈnè ko ˈjéna ˈá ˈwé kaˈlé, ˈnè ko ˈá riˈká ˈwé tiˈbúmo ˈlá, ˈá oˈtʃ͡êrima ˈlé”}\\
    \gll    ˈnè=ko ˈjéna ˈá ˈwé kaˈlé ˈnè=ko ˈá riˈká ˈwé tiˈbú-ma oˈlá ˈá oˈtʃ͡êri-ma aˈlé\\
            1\textsc{sg.nom=emph} \textsc{aff} \textsc{aff} \textsc{int} love.\textsc{prs} 1\textsc{sg.nom=emph} \textsc{aff} like \textsc{int} take.care-\textsc{fut.sg} \textsc{cer} \textsc{aff} grow-\textsc{fut.sg} \textsc{dub}\\
    \glt    ```I do love him very much, I will always take care of him, he will grow."\\
    \glt    ``Yo si lo quiero mucho, yo siempre lo voy a cuidar mucho, sí va a crecer." \corpuslink{tx32[14_111-14_170].wav}{LEL tx32:14:11.1}\\
}
%\pagebreak
        \ex[]{
        \textit{benˈtûra riˈwè oˈlá ˈro, kaˈlîstro onoˈlâ ro}\\
        \gll    benˈtûra riˈwè oˈlá ˈru, kaˈlîstro ono-ˈlâ ro\\
                Ventura be.named.\textsc{prs} \textsc{cer} say Calixto father-\textsc{poss} say.\textsc{prs}\\
        \glt    `They say Ventura was his name, Calixto's father.'\\
        \glt    `Dicen que se llamaba Ventura el papá de Calixto.'  \corpuslink{in484[02_191-02_210].wav}{ME in484:02:19.1}    \\
    }
            \ex[]{
            \textit{ˈí biˈláni ˈnà ruˈmê oˈlá ˈnápu riˈkám ˈnâriani ˈnâria ˈnà tʃ͡ú riˈká bukuˈwêruwa}\\
            \gll    ˈí biˈlá=ni ˈnà ru-ˈmêa oˈlá ˈnápu riˈká=mi ˈnâri-a=ni ˈnâri-a ˈnà tʃ͡ú riˈká bukuˈwêruwa\\
                    here indeed=\textsc{1sg.nom} this say-\textsc{fut.sg} \textsc{cer} \textsc{sub} like=\textsc{2sg.nom} ask-\textsc{prog=1sg.acc} ask-\textsc{prog=1sg.nom} this how how make.bukuweruwa\\
            \glt    `Here I will tell because I was asked about how the bukuwéruwa ritual works.'\\
            \glt    `Aqui voy a contar porque me preguntaste sobre el bukuwéruwa.'   \corpuslink{tx475[00_276-00_335].wav}{SFH tx475:00:27.6}\\
        }
    \z
\z

The certainty marker \textit{oˈlá} may also convey deontic mood when combined with future tense inflected predicates. This is shown in (\ref{ex: deontic ola}).

\ea\label{ex: deontic ola}

    \textit{naˈlìna aʔˈlá ˈnâtika nokiˈbôo oˈlá ˈnà}\\
    \gll    naˈlìna aʔˈlá ˈnâti-ka noki-ˈbô oˈlá ˈnà\\
            but well think-\textsc{ger} do-\textsc{fut.pl} \textsc{cer} then\\
    \glt    ‘But you all should think well.’\\
    \glt    ‘Nomás que tienen que pensar bien.’ \corpuslink{tx12[11_314-11_344].wav}{SFH tx12:11:31.4}\\

\z

The particles \textit{a} and  \textit{biˈlá} have a different distribution than the dubitative and certainty markers, and generally appear preceding the predicate or other particles in the beginning of sentences or clauses. These markers appear in contexts where speakers claim epistemic authority about the propositional content of the utterance. This is exemplified in the following examples: in (\ref{ex: examples of a and bilaa}), the speaker, an expert seamstress in the community, begins a narrative in which she describes a procedure of making a skirt; in (\ref{ex: examples of a and bilab}), the speaker recounts an experience from her childhood.

%\pagebreak

\ea\label{ex: examples of a and bila}

    \ea[]{
    \textit{ˈnè ko \textbf{biˈlá} aniˈmêa \textbf{oˈlá} tʃ͡ú reˈká niwaˈrîa biˈlé ˈpúra ba}\\
    \gll    ˈnè=ko \textbf{biˈlá} ani-ˈmêa oˈlá tʃ͡ú reˈká niwaˈrîa biˈlé ˈpúra ba\\
            1\textsc{sg.nom=emph} really say\textsc{-fut.sg} \textsc{cer} Q how make-\textsc{mpass-prog} one belt \textsc{cl}\\
    \glt    ‘I am going to say how a faja (belt) is made.’\\
    \glt    ‘Yo voy a decir como se hace una faja.’ \corpuslink{tx1[00_207-00_249].wav}{BFL tx1:00:20.7}\\
}\label{ex: examples of a and bilaa}
        \ex[]{
        \textit{aʔˈlì ko ˈtʃ͡ó ˈnà ˈhônsa ko ˈmá biˈlán \textbf{ˈá} riˈwá ˈtʃ͡ó ʃi ˈtʃ͡ó ba}\\
        \gll    aʔˈlì=ko ˈtʃ͡ó ˈnà ˈhônsa=ko ˈmá biˈlá=ni ˈá riˈwá ˈtʃ͡ó ʃi ˈtʃ͡ó ba\\
                and=\textsc{emph} \textsc{dem} \textsc{prox} since=\textsc{emph} already indeed=\textsc{1sg.nom} \textsc{aff} see also also also \textsc{cl}\\
        \glt    ‘And then from there I also saw him.’\\
        \glt    ‘Y entonces de ahí yo también ya vi.’ \corpuslink{tx71[04_363-04_397].wav}{LEL tx71:04:36.3}\\
    }\label{ex: examples of a and bilab}

    \z
\z

Other markers encoding epistemic stance in Choguita Rarámuri include evidential markers. This includes the reportative markers {}-\textit{la} (reportative, different subject) and \textit{-lo} (reportative, same subject),\footnote{\citet{miller1996guarijio} describes a reportative enclitic \textit{=ra} in \ili{Mountain Guarijío} (1996:312), though no alternations are discussed that would reflect a contrast between same vs. different subject as in Choguita Rarámuri.} as well as the auditory evidential \textit{ˈtʃáne} suffix.

\subsection{Pragmatic enclitic}
\label{subsec: empahtic clitics}

Choguita Rarámuri also possesses a pragmatic marker that confers prominence to a word or phrase within discourse, the emphatic enclitic \textit{=ko}.\footnote{The term `emphatic' is used in the description of cognate forms of other Rarámuri and \ili{Guarijío} varieties, including \ili{Western Tarahumara} (\citealt{Burgess-1984}), \ili{Urique Rarómari} (\citealt{valdez2014predication}), \ili{Rochéachi Rarámuri} (\citealt{moralesmoreno2016rochecahi}), and \ili{Mountain Guarijío} (\citealt{miller1996guarijio}), among others.} This morpheme is fairly unrestricted in terms of the word classes and syntactic constituents it can attach to, which is taken as evidence of its status as a clitic. In terms of its function, this emphatic enclitic is described in other varieties as encoding pragmatic statuses compatible with focus (\citealt{valdez2014predication}) or both focus and topic, as proposed by \citet{moralesmoreno2016rochecahi} for \ili{Rochéachi Rarámuri}, and as suggested by \citet{miller1996guarijio} for Mountain \ili{Guarijío} (\citeyear[313--315]{miller1996guarijio}).

The following examples illustrate the distribution of this enclitic, which may attach to nouns (\ref{ex: examples of ko}a--c), pronouns (\ref{ex: examples of ko}b, d--e), adverbs (\ref{ex: examples of kof}) and conectives (\ref{ex: examples of ko}g--h). As shown in (\ref{ex: examples of koh}), the emphatic enclitic may also appear in final clause position.

\ea\label{ex: examples of ko}

    \ea[]{
    \textit{aʔˈlì tʃ͡aˈbôtʃ͡i ko ˈwé biˈlá raˈʔìla baˈhîla ruˈá ˈétʃ͡i tʃ͡oʔˈmá ba}\\
    \gll    aʔˈlì tʃ͡aˈbôtʃ͡i=\textbf{ko} ˈwé biˈlá raˈʔì-la baˈhî-la ru-ˈwá ˈétʃ͡i tʃ͡oʔˈmá ba\\
            and mestizo\textsc{={emph}} \textsc{int} truly like-\textsc{rep} drink-\textsc{rep} say-\textsc{mpass} \textsc{dem} snot \textsc{cl}\\
    \glt    `And the Mexican man they say he really enjoyed drinking it.'\\
    \glt    `Y el mestizo dicen que se lo tomó muy a gusto.’ \corpuslink{tx128[01_478-01_525].wav}{SFH tx128:01:47.8}\\
}\label{ex: examples of koa}
        \ex[]{
        \textit{``hierbas" ko ba, ke beti ... ke beni maˈtʃí ˈnè ko ba}\\
        \gll    hierbas=\textbf{ko} ba, ke be=ti ke be=ni machí ˈnè=\textbf{ko} ba\\
                grasses=\textsc{{emph}} \textsc{cl} \textsc{neg} really=\textsc{1pl.nom} \textsc{neg} really=\textsc{1sg.nom} know.\textsc{prs} \textsc{1sg.nom={emph}} \textsc{cl}\\
        \glt    `(They call them) ``grasses", we don't ... I don't know, me.'\\
        \glt    `(Les dicen) ``hierbas", nosotros no ... yo no se, yo.' \corpuslink{tx785[02_220-02_257].wav}{GFM tx785:02:22.0}\\
}\label{ex: examples of kob}
            \ex[]{
            \textit{amiˈná \textbf{haˈré} ko ˈá ˈmá biˈlá}\\
            \gll    amiˈná haˈré=\textbf{ko} ˈá ˈmá biˈlá\\
                    later some\textsc{{=emph}} \textsc{aff} already indeed\\
            \glt    `later the other ones already did'\\
            \glt    `después los otros ya' \corpuslink{tx475[09_529-09_551].wav}{SFH tx475:09:52.9}\\
        }\label{ex: examples of koc}
                \ex[]{
                \textit{ˈtâsi biˈlé maˈtʃí ˈnè ko ba}\\
                \gll    ˈtâsi biˈlé maˈtʃí ˈnè=\textbf{ko} ba\\
                        \textsc{neg} one know.\textsc{prs} \textsc{1sg.nom={emph}} \textsc{cl}\\
                \glt    `I don't know.'\\
                \glt    `Yo no se.'     \corpuslink{in485[08_056-08_074].wav}{ME in485:08:05.6}\\
            }\label{ex: examples of kod}
            \newpage
                    \ex[]{
                    \textit{aʔˈlì biˈlá ˈá maˈtʃía ruˈwá biˈnôi ko ˈnà ...}\\
                    \gll    aʔˈlì biˈlá ˈá maˈtʃí-a ru-ˈwá biˈnôi=\textbf{ko} na\\
                            and indeed \textsc{aff} know-\textsc{prog} say-\textsc{mpass} \textsc{emph.sg={emp}h} then\\
                    \glt    ‘and they say that he himself knows when’\\
                    \glt    ‘y dicen el mismo sabe cuando’ \corpuslink{tx71[02_502-02_529].wav}{LEL tx71:02:50.2}\\
            }\label{ex: examples of koe}
                        \ex[]{
                        \textit{ʔwîbo ˈmá biˈlá ˈhípi ko ba}\\
                        \gll    ʔwî-bo ˈmá biˈlá ˈhípi=\textbf{ko} ba\\
                                harvest-\textsc{fut.pl} already truly now=\textsc{{emph}} \textsc{cl}\\
                        \glt    `Now we will harvest (it is time to harvest).'\\
                        \glt    `Ahora vamos a pizcar (es tiempo de pizca).' \corpuslink{co1136[02_299-02_322].wav}{MDH co1136:02:29.9}\\
                    }\label{ex: examples of kof}
                           \ex[]{
                            \textit{aʔˈlì ko biˈláti ˈmá ˈpé a iˈwêami ˈnòtʃ͡ika koʔˈpô aˈlé ˈmá aʔˈlá}\\
                            \gll    aʔˈlì=\textbf{ko} biˈlá=ti ˈmá ˈpé a iˈwê-ame ˈnòtʃ͡i-ka koʔ-ˈpô aˈlé ˈmá aʔˈlá\\
                                    and=\textsc{{emph}} really=\textsc{1pl.nom} already just \textsc{aff} strong-\textsc{ptcp} work-\textsc{ger} eat-\textsc{fut.pl} \textsc{dub} already well\\
                            \glt    ‘In that time we will work hard to eat.’\\
                            \glt    ‘Ya en ese tiempo ya vamos a trabajar duro para comer.’  \corpuslink{tx12[08_477-08_538].wav}{SFH tx12:08:47.7}\\
                    }\label{ex: examples of kog}
                                \ex[]{
                                \textit{aʔˈlì ko tamuˈhê ˈpé maˈhâli ko}\\
                                \glt    \gll     aʔˈlì=\textbf{ko} tamuˈhê ˈpé maˈhâ-li=\textbf{ko}\\
                                        then\textsc{={emph}} \textsc{1pl.nom} little get.scared\textsc{-pst={emph}}\\
                                \glt    `Then we got scared.'\\
                                \glt    ‘Entonces nosotros nos asustamos.’   \corpuslink{tx_rosakame[00_171-00_201].wav}{ROF tx\_rosakame:00:17.1}\\
                            }\label{ex: examples of koh}

    \z
\z

In (\ref{ex: examples of ko}a--c), the emphatic enclitic attaches to nouns whose referents have been introduced previously in discourse: a protagonist of a narrative, a mestizo man in (\ref{ex: examples of koa}); the topic of a conversation, medicinal plants in (\ref{ex: examples of kob}); and a group of people in a historical narrative in (\ref{ex: examples of koc}). The examples in (\ref{ex: ko focus}) show that the emphatic enclitic \textit{=ko} may also attach to interrogative pronouns, which may be taken as evidence of the status of the emphatic enclitic as a marker of pragmatic focus (see \citet{moralesmoreno2016rochecahi} on discussion of the distribution of the emphatic enclitic in \ili{Rochéachi Rarámuri}).


\ea\label{ex: ko focus}

   \ea[]{
    \textit{ma ˈpîri ko naˈʔèbo muˈnî baʔaˈlî pa?}\\
    \gll    ˈmá ˈpîri=\textbf{ko} naˈʔ-è-bo muˈnî  baʔaˈlî pa?\\
            so what=\textsc{{emph}} fire-\textsc{make-fut.pl} beans tomorrow \textsc{cl}\\
    \glt    `So what are we going to make the fire with (to cook) the beans tomorrow?'\\
    \glt    `¿Entonces qué le vamos a echar de lumbre a los frijoles (para cocinar) mañana?' \corpuslink{co1136[01_128-01_150].wav}{MDH co1136:01:12.8}\\
}

        \ex[]{
        \textit{ˈétʃ͡i oˈtʃêrami ke ˈlé pa, ˈpîri ko ˈhú aˈlé? ke piˈlá=m matʃ͡i-ˈsâa ˈníla ba}\\
        \gll    ˈétʃ͡i oˈtʃêrami ke aˈlé pa ˈpîri=\textbf{ko} ˈhú aˈlé ke piˈlá=mi matʃ͡i-ˈsâ ˈní-la ba\\
                \textsc{dem} old.people \textsc{neg} \textsc{dub} \textsc{cl} \textsc{what={emph}} \textsc{cop.prs} \textsc{dub} \textsc{neg} really=\textsc{dem} know-\textsc{cond} \textsc{cop-pot} \textsc{cl}\\
        \glt    ``Well, I think its the old people, or what might it be? Because I don't know what it might be.'\\
        \glt    ``Pues pienso que son los viejitos, o qué pueda ser? Yo no se que sea.   \corpuslink{tx223[04_092-04_144].wav}{LEL tx223:04:09.2}\\
    }

    \z
\z

In the following example, the emphatic enclitic confers pragmatic focus to a second person pronominal form in a question (\ref{ex: ko focus 2b}), as part of a conversational exchange.

\ea\label{ex: ko focus 2}

    \ea[]{
    [MDH]\textit{ˈwé riˈmùli roˈkò}\\
    \gll    ˈwé riˈmù-li roˈkò\\
            \textsc{int} dream-\textsc{pst} night\\
    \glt    `He did dream it a lot last night.'\\
    \glt    `Si lo soñó mucho en la noche.'   \corpuslink{co1136[18_430-18_445].wav}{MDH co1136:18:43.0}\\
}\label{ex: ko focus 2a}
        \ex[]{
        [MDH]: \textit{ˈmò \textbf{ko }ba?}\\
        \gll    ˈmò=ko ba\\
                \textsc{2sg.nom=emph} \textsc{cl}\\
        \glt    `And you?'\\
        \glt    `¿Y tu?' \corpuslink{co1136[18_491-18_501].wav}{MDH co1136:18:49.1}\\
    }\label{ex: ko focus 2b}

    \z
\z

The emphatic enclitic may undergo a post-lexical process of lenition, with the voiceless plosive onset surfacing as voiced (pos-lexical lenition of voiceless plosives is addressed in §\ref{subsec: lenition of voiceless plosives}). Examples of the lenited production of \textit{=ko} are shown in (\ref{ex: lenis forms of ko}).

\ea\label{ex: lenis forms of ko}

    \ea[]{
    \textit{taˈmò \textbf{ko} ˈhê 	riˈgá 	ˈnòtʃ͡ami 	hu 	ˈnà 	iˈsêligam \textbf{go} ...}\\
    \gll    taˈmò=\textbf{ko} ˈhê  riˈká 	ˈnòtʃ͡-ame	hu 	ˈnà 	iˈsêli-kame=\textbf{ko}\\
            \textsc{1pl.nom={emph}} \textsc{dem} like	work-\textsc{ptcp} \textsc{cop.prs}	\textsc{dem} be.governor.\textsc{pl-pst.ptcp={emph}}\\
	\glt    ‘We, that is how we work, the ones who have been governors...’\\
	\glt    ‘Nosotros los que hemos sido gobernadores así trabajamos...’    \corpuslink{tx816[00_000-00_070].wav}{JMF tx816:00:00.0}\\
}
        \ex[]{
        \textit{aʔˈlám riˈkátʃ͡imi aʔˈlá iˈwéami raʔaˈmâbi ˈlé f\textbf{ɣo} a mí raʔaˈmâmi ˈlé paˈgótami  baʔaˈlî}\\
        \gll    aʔˈlá=mi riˈkátʃ͡i=mi aʔˈlá iˈwé-ame raʔaˈmâ-bi aˈlé=\textbf{ko} a mí raʔaˈmâ-mi aˈlé paˈgótami  baʔaˈlî\\
	            well=\textsc{2pl.acc} like.that=2\textsc{pl.acc} well	strong-\textsc{ptcp} give.advice-\textsc{irr.pl}	\textsc{dub=emph} \textsc{aff}	there give.advice-\textsc{irr.sg} \textsc{dub}	people tomorrow\\
	    \glt    ‘Perhaps tomorrow people will give you all advice.’\\
	    \glt    ‘A lo mejor de aqui a mañana llegan gentes a darles consejos.’ \corpuslink{tx1132[00_303-00_351].wav}{MFH tx1132:00:30.3}  \\
    }

    \z
\z

The cognate forms of this enclitic in \ili{Guarijío} (\textit{=ka} in \ili{River Guarijío} and \textit{=ga, =go} in \ili{Mountain Guarijío}) are described with similar functions as the ones attested in Choguita Rarámuri. \citet{miller1996guarijio} proposes this enclitic is diachronically derived from a copular predicate, \textit{=ga}, plus a subordination suffix \textit{-o} (see §\ref{subsec: types of copulas} for description of copular predicates in Choguita Rarámuri).

\subsection{Final particles}
\label{subsec: final clitics}

The particle \textit{pa} (the onset of which may also undergo post-lexical lenition, as described in §\ref{subsec: lenition of voiceless plosives}), may be categorized as a discourse particle (glossed here as \textsc{cl)} that has the main function of marking syntactic and/or discourse boundaries. It is frequently attested in natural discourse (narratives and conversations) in clause and sentence final position, generally followed by a pause and/or an intonation break (see also \citealt{moralesmoreno2016rochecahi} for a description of the function and distribution of this particle in \ili{Rochéachi Rarámuri}). Examples of the final particle \textit{pa} are provided in (\ref{ex: final particle ba}). In these examples, clauses are marked with square brackets.

\ea\label{ex: final particle ba}

    \ea[]{
    \textit{aʔˈlì ˈétʃ͡i biˈláti beˈnèli tamuˈhê \textbf{ba} tʃ͡ú riˈká tiˈbúsa ˈlé \textbf{pa} ˈnà kaˈwì \textbf{ba}}\\
    \gll    aʔˈlì ˈétʃ͡i biˈlá=ti beˈnè-li tamuˈhê \textbf{ba}] tʃ͡ú riˈká tiˈbú-sa aˈlé \textbf{pa}] ˈnà kaˈwì \textbf{ba}\\
            and \textsc{dem} really\textsc{=1pl.nom} learn\textsc{=pst} \textsc{1pl.nom} \textsc{{cl}} how how care-\textsc{cond} \textsc{dub} \textsc{{cl}} \textsc{dem} land \textsc{{cl}}\\
    \glt    `So this is how we learnt, how to tend for it, the land.'\\
    \glt    `Entonces así aprendimos nosotros, cómo cuidarla, la tierra.'   \corpuslink{tx977[00_600-01_062].wav}{SFH tx977:00:60.0}\\
}\label{ex: final particle baa}
\newpage
        \ex[]{
        \textit{ˈnápu koˈlìki biˈtí \textbf{ba}}\\
        \gll    ˈnápu koˈlì-ki biˈtí \textbf{ba}]\\
 	            \textsc{sub} around.the.side-\textsc{loc} lie.\textsc{pl.prs} \textsc{{cl}}\\
	    \glt    ‘Like the ones who lie in that other side (by the graveyard).’\\
	    \glt    ‘Como los que están (acostados) de aquel lado (del panteón).’  <GFM co1136[17\_430-17\_445]>\\
    }\label{ex: final particle bab}
            \ex[]{
            \textit{aʔˈlì biˈlá ko waˈbé biˈlá 	kiˈʔà ˈníla ra \textbf{pa} kuˈrí ke ˈtʃ͡ó me biwaˈtʃ͡êatʃ͡i ˈnà kaˈwì \textbf{β̩a}}\\
            \gll    aʔˈlì biˈlá=ko waˈbé biˈlá 	kiˈʔà ˈní-la ra \textbf{pa}] kuˈrí ke ˈtʃ͡ó me biwaˈtʃ͡ê-a-tʃ͡i ˈnà 	kaˈwì \textbf{ba}]\\
	                and	indeed=\textsc{emph} \textsc{int} indeed long.ago \textsc{cop-rep} \textsc{rep}	\textsc{{cl}}	recently \textsc{neg} yet almost solidify-\textsc{prog-temp} this earth \textsc{{cl}}\\
	        \glt    `And so it was a long time before this earth was solid.’\\
	        \glt    `Y entonces fue mucho cuando todavía no amacizara este mundo.’  <tx43[11\_112-11\_182]>\\
        }\label{ex: final particle bac}
                \ex[]{
                \textit{baˈhîm baʔˈwí \textbf{pa}?}\\
                \gll    baˈhî=mi baʔˈwí \textbf{pa}?\\
                        drink=\textsc{2sg.nom} water \textsc{{cl}}\\
                \glt    `Did you drink water?'\\
                \glt    `¿Tomaste agua?'  \corpuslink{co1140[12_108-12_121].wav}{MDH co1140:12:10.8}\\
            }\label{ex: final particle bad}
    \z
\z

As shown in these examples, the final particle \textit{=pa} is not only attested at the end of declaratives, but may also close interrogative clauses (\ref{ex: final particle bad}).

% describe here these examples

%%The prosodic and grammatical status of formatives as pragmatic markers -
%%The question of whether the clitics are Wackernagel (second position) clitics (reference to Chiro’s thesis)
