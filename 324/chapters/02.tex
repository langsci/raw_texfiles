\chapter{Grammatical overview}
\label{chap: grammatical overview}

\section{Phonology}
\label{sec: phonology}

\subsection{Segmental inventory and processes}
\label{subsec: segmental inventory and processes}

The Choguita Rarámuri phonemic consonant and vocalic monophthong inventories are provided in Table \ref{tab:consonants} and Table \ref{tab:vowels}, respectively.\footnote{The labio-velar approximant is doubly assigned in the bilabial place of articulation column and in the velar place of articulation column in Table \ref{tab:consonants}.}

\begin{table}
\caption{{Phonemic inventory of Choguita Raramuri consonants}}
\label{tab:consonants}
\fittable{
\begin{tabular}{lccccccc}
\lsptoprule
& {Bilabial} & {Alveolar} & {Alveo-palatal} & {Retroflex} & {Palatal} & {Velar} & {Glottal}\\
\midrule
{{Plosive}} & {p} \textup{ʰ}\textup{p} \textup{b} & {t} \textup{ʰ}\textup{t} &  &  &  & {k} \textup{ʰ}\textup{k}  & {ʔ}\\
{{Affricate}} &  &  & {tʃ͡} \textup{ʰ}\textup{tʃ͡} &  &  &  & \\
{{Nasal}} & {m} & {n} &  &  &  &  & \\
{{Flap}} &  & {[\textup{ɾ]} <r>}   &  & {[ɽ] <l> }  &  &  & \\
{{Fricative}} &  & {s} &  &  &  &  & {h}\\
{{Approx.}} & {w} &  &  &  & {j} & {w} & \\
\lspbottomrule
\end{tabular}
}
\end{table}
\hspace{3cm}

\begin{table}
\caption{{Choguita Rarámuri monophthong vowel system}}
\label{tab:vowels}

\begin{tabularx}{\textwidth}{XXXX}
\lsptoprule
& Front & Central & Back\\
\midrule
 High & i &  & u\\
 Mid & e [ɛ] &  & o [ɔ] \\
 Low &  & a & \\
\lspbottomrule
\end{tabularx}
\end{table}
\hspace{3cm}

Choguita Rarámuri displays complex patterns of allophonic variation, most of which involve lenition processes. Voiced bilabial plosives in inter-vocalic position may be optionally realized as voiced bilabial fricatives ([β]) (\ref{ex: spirantization of ba}) or voiced approximants ([β̞]) (\ref{ex: spirantization of bb}). Lenition of /b/ in unstressed word-initial syllables (exemplified in (\ref{ex: spirantization of bc})) leads to neutralization of the contrast between /w/ and /b/.

\ea\label{ex: spirantization of b}
{Spirantization of /b/}

    \ea[]{
    [ziˈ\textbf{β}óò]\\
    /si-ˈ\textbf{b}ô/\\
    go.\textsc{pl}-\textsc{fut.pl}\\
    `we will go' \\
    `vamos a ir' \corpuslink{tx1133[01_172-01_240].wav}{MFH tx1133:01:17.2}  \\
}\label{ex: spirantization of ba}
        \ex[]{
        [ˈè\textbf{β̞}əma]\\
        /ˈè\textbf{b}i-ma/\\
        bring\textsc{-fut.sg}\\
        `s/he will take it.'\\
        `va a llevarlo.' < BFL 06 6:73/el >\\
    }\label{ex: spirantization of bb}
                \ex[]{
                [waʔˈwí]\\
                /\textbf{b}aʔˈwí/\\
                `water'\\
                `agua' < SFH 04 1:17/el >\\
            }\label{ex: spirantization of bc}
    \z
\z

For some speakers, the voiced bilabial stop is realized as a labio-velar semi-vowel ([w]) pre-consonantally after post-tonic vowel deletion (\ref{ex: gliding of b}).

\ea\label{ex: gliding of b}

            [ˈè\textbf{w}tiki] \\
            /ˈè\textbf{b}i-ti-ki/ \\
            bring-\textsc{caus-pst.ego}\\
            `I made him/her bring it.'\\
            `Lo/a hice traerlo.' < BFL 06 6:73/el >\\
\z

Alveopalatal affricates may be produced as alveolar affricates, optionally depalatalizing before low, central vowels. Alveolar affricates may also further reduce, becoming deaffricated in fast speech, a process that frequently targets function words like demonstrative \textit{ˈétʃ͡i}. These processes are exemplified in (\ref{ex: optional depalatalization and deaffrication of alveopalatal affricate}a--b) and (\ref{ex: optional depalatalization and deaffrication of alveopalatal affricate}c--d), respectively.

\break

\ea\label{ex: optional depalatalization and deaffrication of alveopalatal affricate}
{Optional depalatalization and deaffrication of alveopalatal affricate}

    \ea[]{
    [aˈkâ\textbf{tʃ}a-la]   {\textasciitilde} [aˈkâ\textbf{ts}a-la] \\
    /aˈkâ\textbf{tʃ}a-la/\\
    paternal.grandmother-\textsc{poss}\\
    `her/his paternal grandmother'\\
    `su abuela paterna' \corpuslink{in243[08_094-08_121].wav}{SFH in243:08:09.4}  \\
}
        \ex[]{
        [aˈ\textbf{tʃ}â-sa] {\textasciitilde} [aˈ\textbf{ts}â-sa]   \\
        /aˈ\textbf{tʃ}â-sa/\\
        to.sit.\textsc{sg.tr-cond}\\
        `if s/he sits him/her down'\\
        `si lo sienta’ < SFH 04 1:38/el >\\
    }
                \ex[]{
                [e\textbf{s}ˈtá]\\
                \glt    /ˈétʃ͡i ˈtá/\\
                \glt        \textsc{dem} \textsc{det}\\
                \glt    `that one'\\
                \glt    `ese' < LEL 06 6:141-162/tx >\\
            }
                    \ex[]{
                    [ˈé\textbf{s}  ˈkútʃ͡i]\\
                    \glt   /ˈé\textbf{tʃ}i ˈkútʃ͡i/\\
                    \glt        \textsc{dem} small\\
                    \glt    ‘those small ones’\\
                    \glt    ‘esos pequeños’ < BFL 07 frog story\_2/tx >\\
        }
    \z
\z

\largerpage
%fix approximant vs. frivative for Beta
A striking pattern of segmental reduction involves lenition of voiceless plosives, which are gradiently realized within a continuum ranging from voiceless aspirated stops to complete deletion: [pʰ > p > b > β > β̞ > w > Ø]. I analyze this reduction process, exemplified in (\ref{ex: lenition examples}), as a phonetic process that is dependent on rate of speech and position of segments within the Intonational Phrase: productions on the lenis end of the continuum (fricatives and approximants) tend to be produced in utterance-final position, while productions on the fortis end (voiceless and voiced stops) tend to be produced utterance-medially, a factor that may suggest that these alternations are sensitive to phrasal phonological effects.

\ea\label{ex: lenition examples}
{Phonetic reduction of voiceless plosives}

 \ea[]{
    \textit{tʃ͡ú riˈká tiˈbúsa ˈlé pa ˈnà kaˈwì βa}\\
    \gll    tʃ͡ú riˈká tiˈbú-sa aˈlé \textbf{pa}] ˈnà kaˈwì \textbf{βa}] \\
            how that take.care-\textsc{cond}   \textsc{dub} {\textsc{cl}} \textsc{prox}  land  {\textsc{cl}} \\
    \glt    ‘(we learnt) how to take care of it, this earth’\\
    \glt    ‘(aprendimos) cómo cuidarla, la tierra’ \corpuslink{tx977[00_600-01_062].wav}{SFH tx977:00:60.0}\\
}
        \ex[]{
        \textit{ˈnà kaˈwì \textbf{β̞a}}\\
        \gll    ˈnà kaˈwì \textbf{β̞a} \\
                \textsc{dem} earth  {\textsc{cl}}\\
        \glt    ‘this earth’\\
        \glt    ‘este mundo’ \corpuslink{tx43[11_112-11_182].wav}{SFH tx43:11:11.2}\\
        }
    \z
\z

In addition to these processes, Choguita Rarámuri exhibits cross-linguistically common phonological processes, including  palatalization of alveolar fricatives before high vowels. This process may be rendered opaque by post-tonic vowel deletion. This is exemplified in (\ref{ex: surface opaque fricative palatalization}).

\ea\label{ex: surface opaque fricative palatalization}
{Surface opaque fricative palatalization}

        [aˈtí\textbf{ʃ}-li]  \\
        /aˈtí\textbf{si}-li/    \\
        to.sneeze-\textsc{pst}  \\
        `S/he sneezed.'\\
        `Estornudó.' < BFL 05 1:111/el >  \\

\z

In addition to consonantal allophonic patterns, vocalic segments undergo various stress-based reduction processes. Specifically, there are three distinct, optional patterns or degrees of vowel reduction: (i) /e/ raises to [i] both pre-tonically and posttonically (\ref{ex: stress-based vowel reduction}a--b) (a robust process that exhibits optionality only for some speakers); (ii) non-final post-tonic /a/ and /o/ raise to [i] (\ref{ex: stress-based vowel reduction}c--d); and (iii) high vowels reduce to schwa post-tonically (\ref{ex: stress-based vowel reduction}e--f).

\ea\label{ex: stress-based vowel reduction}
{Stress-based vowel reduction}

    \ea[]{
    [n\textbf{i}ˈhê]  {\textasciitilde} [n\textbf{e}ˈhê] \\
    /n\textbf{e}ˈhê/    \\
    ‘I’\\
    ‘yo’    < SFH 07 2:63/el >\\
}
        \ex[]{
        [b\textbf{i}ˈnè]  {\textasciitilde} [b\textbf{e}ˈnè]  \\
        /b\textbf{e}ˈnè/    \\
        ‘learn’\\
        ‘aprender’ < ROF 04 1:9/el >\\
    }
            \ex[]{
            [ˈtʃ͡ôt\textbf{i}{}-li]   {\textasciitilde}  [ˈtʃ͡ôt\textbf{a}{}-li]  \\
            /ˈtʃ͡ôt\textbf{a}{}-li/ \\
            begin\textsc{-pst}  \\
            `It began.'\\
            `Comenzó.'  < SFH 07 in243/in >\\
        }
                \ex[]{
                 [rono+ˈbâk\textbf{i}{}-ma]   {\textasciitilde}  [rono+ˈbâk\textbf{o}-ma] \\
                 /roˈno+paˈk\textbf{o}{}-ma/  \\
                 feet+wash-\textsc{fut.sg}\\
                 `S/he will wash their feet.'\\
                 `Lavará sus pies.'\\
            }
                    \ex[]{
                    [ˈpòl\textbf{ə}{}-ki] {\textasciitilde}   [ˈpòl\textbf{i}-ki] \\
                    /ˈpòl\textbf{i}{}-ki/ \\
                    cover-\textsc{pst.ego}  \\
                    `I covered myself.'\\
                    `Me tapé.' < AHF 05 1:125/el >\\
            }
                        \ex[]{
                        [naˈwí-n\textbf{ə}la] {\textasciitilde}  [naˈwí-n\textbf{i}la] \\
                        /naˈwí-n\textbf{u}la/  \\
                        sing-\textsc{order} \\
                        `They ordered them to sing.'\\
                        `Los mandaron a que cantaran.' < BFL 07 VDB/el >\\
                }
    \z
\z

The underlying syllable structure in Choguita Rarámuri is (C)V(ʔ), with optional onsets and optional glottal stop codas. Stress-based vowel deletion yields consonant clusters in surface forms, which include geminates for some speakers, most frequently oral and nasal bilabial ones (e.g., \textit{moˈtép\textbf{i}-po} →\textit{moˈté\textbf{p-p}o} `we will braid her hair').

There is no contrastive vowel length, but long vowel sequences arise at morpheme junctures (e.g., \textit{biʔˈw-\textbf{à-a}} `they are cleaning') or as a morphologically-conditioned effect (e.g., the past passive \textit{-ru} suffix conditions lengthening of the stressed stem vowel). There are also diphthongs with falling sonority, which occur morpheme internally and at morpheme boundaries (e.g., \textit{k\textbf{ai}ˈnâ-ma} `they will harvest',\textit{ seˈm\textbf{è-i}} `he used to play the violin').

\subsection{Stress, tone and prosodic structure}
\label{subsec: stress and tone}

Stress in Choguita Rarámuri is lexically contrastive and assigned to the first, second or third syllable within an initial three-syllable window. Stress in inflected words is determined by the stress properties of both roots and suffixes: roots are stressed or unstressed (with fixed or shifting stress across paradigms, respectively), while suffixes are stress-shifting or stress-neutral. Stress distribution by root and suffix type according to their stress properties is exemplified in \tabref{tab:stress2}:

\break

\begin{table}
\caption{Stress patterns of morphologically complex verbs}
\label{tab:stress2}

\begin{tabularx}{\textwidth}{XXXX}
\lsptoprule
\textbf{Stem} & \textbf{Stress-neutral} & \textbf{Stress-shifting} & \\
 & \textbf{Past \textit{-li}} & \textbf{Conditional \textit{-sa}} & \\
 \midrule
beˈnè `learn’ & be\textbf{ˈnè}-li & be\textbf{ˈnè}-sa & \textbf{Stressed roots} \\
baˈhî `drink' & ba\textbf{ˈhî}-li & ba\textbf{ˈhî}-sa & \\
tʃ͡aˈpí `grab' & tʃ͡a\textbf{ˈpí}-li & tʃ͡api-\textbf{ˈsâ} & \textbf{Unstressed roots} \\
saˈkí `toast corn' & sa\textbf{ˈkí}-li & saki-\textbf{ˈsâ} & \\
\lspbottomrule
\end{tabularx}
\end{table}
%\hspace{3cm}

A three-way lexical tonal contrast between HL (â), L (à) and H (á) tones is exclusively realized on surface stressed syllables: there is only one lexical tone per Prosodic Word and stressless syllables lack lexical tone. The tone-bearing unit is the mora: falling tones have their high target on the stressed syllable, with the fall starting in the tonic and continuing through a post-tonic syllable, if there is one; H tones may spread their high f0 to the post-tonic syllable.

Morphologically governed stress shifts result in lexical tonal alternations: if a stress-shifting suffix is stressed after a stress shift, the stressed suffix syllable will bear the lexical tone of that suffix, either a L tone (\ref{ex: suffix tonesb}), a HL tone (\ref{ex: suffix tonesd}) or a H tone (\ref{ex: suffix tonese}).

\largerpage
\ea\label{ex: suffix tones}
{Root and suffix tones}

    \ea[]{
    \glt    \doublebox{[kiˈmáli]}{H}\\
    \glt    /kimá-li/\\
    \glt    put.on.blanket-\textsc{pst}\\
    \glt    `S/he covered with a blanket.'\\
    \glt    `Se encobijó.' < BFL el1909 > \\
}\label{ex: suffix tonesa}
        \ex[]{
        \doublebox{[kimiˈsì]}{L}\\
        \glt    /kimá-sì/\\
        \glt    put.on.blanket-\textsc{imp.pl}\\
        \glt    `You all cover yourselves with a blanket!'\\
        \glt    `¡Encobíjense!' < BFL el1909 >\\
    }\label{ex: suffix tonesb}
            \ex[]{
            \doublebox{[ˈtòli]}{L}\\
           \glt     /tò-li/\\
           \glt  take-\textsc{pst}\\
           	\glt    `S/he took it.'\\
           	\glt    `Se lo llevó.' < RIC el921 > \\
        }\label{ex: suffix tonesc}
                \ex[]{
                \doublebox{[toˈkâ]}{HL}\\
                \glt    /tò-kâ/\\
                \glt    {take\textsc{-imp.sg}}\\
                \glt    `Take it!'\\
                \glt    `¡Llévalo!'  < BFL el1882 >\\
            }\label{ex: suffix tonesd}
                    \ex[]{
                    \doublebox{[toˈnále]}{H}\\
                    \glt    /tò-nále/\\
                    \glt    take\textsc{-desid}\\
                    \glt    `S/he wants to take it.'\\
                    \glt    `Quiere llevarlo.' < BFL el1882 >\\
                }\label{ex: suffix tonese}
    \z
\z

Trisyllabic unstressed roots have second syllable stress and the root's lexical tone is realized on the stressed syllable, e.g., H tone in (\ref{ex: tone neutralizationa}) or L tone in (\ref{ex: tone neutralizationd}). When attaching a stress-shifting suffix, these stems have third syllable stress in a root syllable which bears a HL tone regardless of what the lexical tones of the root and suffix(es) are, as shown in (\ref{ex: tone neutralization}b--c, e--f).



\TabPositions{4cm,8cm}

\ea\label{ex: tone neutralization}
{Tonal neutralization}

    \ea[]{
    \glt    \doublebox{[roʔˈsówali]}{H}   \\
    \glt    /roʔsówa-li/ \\
    \glt    cough-\textsc{pst}	\\
    \glt    `S/he coughed.'\\
    \glt    `Tosió.' < LEL el2060 >\\
}\label{ex: tone neutralizationa}
        \ex[]{
        \glt    \doublebox{[roʔsoˈwâma]}{HL}\\
        \glt    /roʔˈsówa-mâ/\\
        \glt    cough-\textsc{fut.sg} \\
        \glt    `S/he will cough.'\\
        \glt    `Va a toser.' < LEL el2060 >\\
    }\label{ex: tone neutralizationb}
            \ex[]{
            \glt    \doublebox{[roʔsoˈwâsi]}{HL}\\
            \glt    /roʔsówa-sì/	\\
            \glt    cough-\textsc{imp.pl}\\
            \glt    `You all cough!'\\
            \glt    `¡Tosan!' < LEL el2060 >\\
        }\label{ex: tone neutralizationc}
                \newpage
                \ex[]{
                \glt    \doublebox{[naʔˈsòwali]}{L}\\
                \glt    /naʔsòwa-li/ \\
                \glt    stir-\textsc{pst} \\
                \glt    `S/he stirred it.'\\
                \glt    `Lo revolvió.' < BFL el1957 >	\\
            }\label{ex: tone neutralizationd}
                    \ex[]{
                    \glt    \doublebox{[naʔsoˈwâma]}{HL}\\
                    \glt    /naʔsòwa-mâ/ \\
                    \glt    stir-\textsc{fut.sg}\\
                    \glt    `S/he will stir.'\\
                    \glt    `Va a revolver.' < BFL el1957 >\\
                }\label{ex: tone neutralizatione}
                        \ex[]{
                        \glt   \doublebox{[naʔsoˈwâsi]}{HL}\\
                        \glt    /naʔsòwa-sì/\\
                        \glt    stir-\textsc{imp.pl} \\
                        \glt    `You all stir!'\\
                        \glt    `¡Revuelvan!' < BFL e1957 > \\
                    }\label{ex: tone neutralizationf}
    \z
\z



\TabPositions{2cm,4cm,6cm,8cm}

Tone alone may be the sole exponent of inflectional categories (e.g., L tone is an allomorph of the imperative singular construction, e.g., HL \textit{niˈkâ} `it barks' vs. L \textit{niˈkà} `bark!'). In other grammatical tone patterns, stem tones are conditioned by suffixes in certain verb classes.

Phonological and morphological processes and phenomena make reference to the Prosodic Word, which in Choguita Rarámuri can be identified by the following criteria:

\ea\label{ex: prosodic word criteria}
{The prosodic word in Choguita Rarámuri}

	\ea[]{
     Each prosodic word is assigned a single, main stress within the first three syllables, which bears a lexical tone (§\ref{subsec: stress and stress-dependent phenomena}).\\
}
    	\ex[]{
    	 Inflected verbs are minimally bimoraic (§\ref{subsec: minimal word size}).\\
    }
    		\ex[]{
    		 Prosodic words are vowel final (§\ref{subsubsec*: stress-conditioned vowel deletion}).\\
    	}
    			\ex[]{
    			 Body part incorporation combines two morphological roots into a single prosodic word, where the new lexical item is assigned a single, main stress in the first syllable of the head of the compound (§\ref{subsec: body-part incorporation}).\\
    		}
    				\ex[]{
    				 The glottal stop only emerges within the first two syllables of the prosodic word (§\ref{subsec: glottal stop}).\\
    			}
    				\z
   				\z

\section{Pronouns and demonstratives}
\label{sec: pronouns and demonstratives}

Choguita Rarámuri personal pronouns distinguish two person values (first and second), two numbers (singular and plural) and encode a binary nominative-accusative case distinction (second person object pronouns do not encode a number distinction). Subject pronouns have full and reduced forms. \tabref{tab:pronouns} summarizes the subject/object distinctions of free pronominal forms.

%\hspace{3cm}

\begin{table}
\caption{Free personal pronouns}
\label{tab:pronouns}

\begin{tabularx}{.5\textwidth}{lll}
\lsptoprule
& \textbf{Subject} & \textbf{Object}\\
\midrule
 \textsc{1sg} & neˈhê, ˈnè & taˈmí\\
 \textsc{2sg} & muˈhê, ˈmò & ˈmí\\
 \textsc{1pl} & tamuˈhê, taˈmò & taˈmò\\
 \textsc{2pl} & ˈémi & ˈmí\\
\lspbottomrule
\end{tabularx}
\end{table}
%\hspace{3cm}


Subject pronominal forms may also be encoded though second (Wackernagel) position enclitics, attaching immediately after the first accented phrase or sub-constituent of a phrase. \tabref{tab:enclitics} presents the paradigm of enclitic pronominal forms.

\begin{table}
\caption{Pronominal enclitic forms}
\label{tab:enclitics}

\begin{tabularx}{.5\textwidth}{lll}
\lsptoprule
& \textbf{Subject} & \textbf{Object}\\
\midrule
 \textsc{1sg} & =ni (neˈhê, ˈnè) & (taˈmí)\\
 \textsc{2sg} & =mi (muˈhê, ˈmò) & (ˈmí)\\
 \textsc{1pl} & =ti (tamuˈhê, taˈmò) & (taˈmò)\\
 \textsc{2pl} & =timi (ˈémi) & (ˈmí)\\
\lspbottomrule
\end{tabularx}
\end{table}
%\hspace{3cm}

Some representative examples of pronominal forms are provided in (\ref{ex: pronoun examples}).

\ea\label{ex: pronoun examples}

    \ea[]{
    \textit{niˈhê ˈmí saʔˈpá aˈwênima}\\
    \gll    \textbf{neˈhê} \textbf{ˈmí} saʔˈpá aˈwê-ni-ma\\
            {1\textsc{sg.nom}} {2\textsc{sg.acc}} meat grill-\textsc{appl-fut.sg}\\
    \glt    `I will grill meat for you.'\\
    \glt    `Te voy a asar una carne.' \corpuslink{el685[02_205-02_222].wav}{SFH el685:02:20.5}\\
}
        \ex[]{
        \textit{taˈmò ma ˈmêli}\\
        \gll   \textbf{taˈmò} ma ˈmê-li\\
               { 1\textsc{pl.nom}} already win-\textsc{pst}\\
        \glt    `We won.'\\
        \glt    `Ganamos.' \corpuslink{el1278[00_144-00_160].wav}{JLG el1278:00:14.4}\\
    }
            \ex[]{
            \textit{ka ˈtʃ͡èmi siˈrîruami hu}\\
            \gll    ka ˈtʃ͡è\textbf{=mi} siˈrî-ru-ame hu\\
                    because not={\textsc{2sg.nom}} cut-\textsc{pst.pass-ptcp} \textsc{cop}\\
            \glt    `because you had not been cut'\\
            \glt    `porque no habías sido cortado' \corpuslink{tx475[07_079-07_109].wav}{SFH tx475:07:07.9}\\
        }
                \ex[]{
                \textit{ke taˈmí ˈʔàki}\\
                \gll    ke \textbf{taˈmí} ˈà-ki\\
                        \textsc{neg} {\textsc{1sg.acc}} give-\textsc{pst}\\
                \glt    `S/he didn't give me any.'\\
                \glt    `No me dió.' \corpuslink{co1237[05_481-05_491].wav}{JLG co1237:05:48.1}\\
            }
    \z
\z

Third person arguments may be left unmarked (\ref{ex: third person arguments encoding 2a}), or they may be encoded through demonstratives (e.g., \textit{ˈétʃ͡i} in (\ref{ex: third person arguments encoding 2b})) or through an emphatic pronoun (e.g., \textit{biˈnôi} `himself' in (\ref{ex: third person arguments encoding 2c})).

\ea\label{ex: third person arguments encoding 2}

    \ea[]{
    \textit{aʔˈlì ke muˈríwia ruˈwá}\\
    \gll    aʔˈlì [ke muˈríwi-a] ru-ˈwá\\
            and \textsc{neg} get.close-\textsc{prog} say-\textsc{mpass}\\
    \glt    `and they say they didn't get close'\\
    \glt    `y dicen que ellos no se arrimaban’ \corpuslink{tx109[02_142-02_164].wav}{LEL tx109:02:14.2}\\
}\label{ex: third person arguments encoding 2a}
        \ex[]{
        \textit{aʔˈlì ˈétʃ͡i taˈmí ``kuˈmûtʃ͡i" aˈnèma ba?}\\
        \gll    aʔˈlì \textbf{ˈétʃi} taˈmí kuˈmûtʃ͡i aˈn-è-ma ba?\\
                and {\textsc{dem}} \textsc{1sg.acc} kumuchi say-\textsc{appl-fut.sg} \textsc{cl}\\
        \glt    `and will they call me ``kumuchi"?'\\
        \glt    `¿y ellos me van a decir ``kumuchi"?' \corpuslink{in484[13_594-14_013].wav}{SFH in484:13:59.4}\\
    }\label{ex: third person arguments encoding 2b}
            \ex[]{
            \textit{ˈápi aʔˈlì biˈnôi wikaˈrâ ko ˈhê aˈní ˈrú}\\
            \gll    [ˈápi aʔˈlì \textbf{biˈnôi} wikaˈrâ=ko] ˈhê aˈní ˈrú\\
                    \textsc{sub} then {himself} sing.\textsc{prs}=\textsc{emph} \textsc{dem} say.\textsc{prs} say.\textsc{prs}\\
            \glt    ‘when he sings he says this’\\
            \glt    ‘cuando canta él así dice’ \corpuslink{tx71[03_157-03_183].wav}{LEL tx71:03:15.7}\\
        }\label{ex: third person arguments encoding 2c}
    \z
\z

\hspace*{-2.5pt}Demonstratives in Choguita Rarámuri may function anaphorically as pronouns or as nominal modifiers and encode two degrees of distance and orientation with respect to the speaker/addressee: (i) \textit{ˈnà} (‘this one’) is proximal and encodes closeness to the speaker; and (ii) \textit{ˈétʃ͡i} ‘that one’ is proximal and encodes closeness to either the addressee or the speaker. A third demonstrative (\textit{ˈhê}) is exclusively found in complement clauses of utterance predicates and other complement-taking predicates introducing quoted speech.

\section{Discourse particles}
\label{sec: discourse particles and enclitics}

Choguita Rarámuri possesses a large amount of discourse particles that constitute a set of heterogeneous word classes. These classes are closed and morphologically simple, bearing no inflection or derivation. Each class of particles is composed of fewer than a dozen members per class. A subset of these forms are presented in \tabref{tab:particles}, with their form, meaning, gloss and cross-reference to the section of the grammar where they are discussed in detail (if available).

\begin{table}
\caption{Discourse particles}
\label{tab:particles}

\begin{tabularx}{.7\textwidth}{lQll}
\lsptoprule
\textbf{Form} & \textbf{Meaning}  & \textbf{Gloss} & \\
\midrule
\textit{(o)ˈlá}  &     `certainty'      &   \textsc{cer} & §\ref{subsec: epistemic particles}\\
\textit{(a)ˈlé} &    `dubitative' &  \textsc{dub}    & §\ref{subsec: epistemic particles} \\
\textit{ˈa} &   `affirmative', epistemic &  \textsc{aff}    & §\ref{subsec: epistemic particles} \\
\textit{biˈlá} &   epistemic &  indeed    & §\ref{subsec: epistemic particles} \\
\textit{ko} &   `emphatic'  &  \textsc{emph} & §\ref{subsec: empahtic clitics} \\
\textit{ba}  &    final particle   &   \textsc{cl} & §\ref{subsec: final clitics}\\
\textit{we} & `intensifier'    & \textsc{int}  & \\
\lspbottomrule
\end{tabularx}
\end{table}


Discourse particles in Choguita Rarámuri have a wide range of functions and meanings. These include modality particles that encode epistemic stance, such as the particles \textit{(o)ˈla} and \textit{(a)ˈlé}, which encode certainty or doubt, respectively, that speakers may have about the actuality or likelihood that an event takes place. Epistemic particles are highly frequently, found post-verbally with future-marked verbs. Examples of the contrast between these epistemic markers are provided in (\ref{ex: epistemic modality overview}).

\largerpage

%\pagebreak

\ea\label{ex: epistemic modality overview}

    \ea[]{
    \textit{ˈnârma ˈlé}\\
    \gll    ˈnâri-ma aˈlé  \\
            ask-\textsc{fut.sg} \textsc{dub}\\
    \glt    ‘(He) will probably ask.’\\
    \glt    ‘Probablemente va a preguntar.’ < BFL 05 1:152/el >\\
}
        \ex[]{
        \textit{ˈnârmo ˈlá}\\
        \gll    ˈnâri-ma oˈlá \\
                ask-\textsc{fut.sg} \textsc{cer} \\
        \glt    ‘S/he will definetly ask.’\\
        \glt    ‘Seguramente que va a preguntar.’  < BFL 05 1:152/el >\\
    }
    \z
\z

Another highly frequent discourse marker is \textit{=ko}, a pragmatic enclitic that confers prominence to a word or phrase within discourse. In some contexts it functions as a topic marker, attaching to nouns whose referents have been introduced previously in discourse (\ref{ex: ko overviewa}). In other contexts, \textit{=ko} functions as a pragmatic focus marker (\ref{ex: ko overviewb}).

\ea\label{ex: ko overview}

    \ea[]{
    \textit{aʔˈlì tʃ͡aˈbôtʃ͡i ko ˈwé biˈlá raˈʔìla baˈhîla ˈrá ˈétʃ͡i tʃ͡oʔˈmá ba}\\
    \gll    aʔˈlì tʃ͡aˈbôtʃ͡i=\textbf{ko} ˈwé biˈlá raˈʔì-la baˈhî-la ru-ˈwá ˈétʃ͡i tʃ͡oʔˈmá ba\\
            and mestizo\textsc{={emph}} \textsc{int} truly like-\textsc{rep} drink-\textsc{rep} say-\textsc{mpass} \textsc{dem} snot \textsc{cl}\\
    \glt    `And the Mexican man they say he really enjoyed drinking it.'\\
    \glt    `Y el mestizo dicen que se lo tomó muy a gusto.’ \corpuslink{tx128[01_478-01_525].wav}{SFH tx128:01:47.8}\\
}\label{ex: ko overviewa}
        \ex[]{
        \textit{ˈétʃ͡i oˈtʃ͡êrami ke ˈlé pa, ˈpîri ko ˈhú aˈlé?}\\
        \gll    ˈétʃ͡i oˈtʃ͡êrami ke aˈlé pa ˈpîri=\textbf{ko} ˈhú aˈlé\\
                \textsc{dem} old.people \textsc{neg} \textsc{dub} \textsc{cl} \textsc{what={emph}} \textsc{cop.prs} \textsc{dub}\\
        \glt    ``Well, I think its the old people, or what might it be? Because I don't know what it might be'\\
        \glt    ``Pues pienso que son los viejitos, o qué pueda ser? Yo no se que sea' \corpuslink{tx223[04_092-04_144].wav}{LEL tx223:04:09.2}\\
    }\label{ex: ko overviewb}
    \z
\z

Finally, the particle \textit{pa} marks syntactic and/or discourse boundaries and is also frequently found in natural discourse (see also (\ref{ex: lenition examples}) above and (\ref{ex: ko overviewb})).

In addition to these discourse markers, Choguita Rarámuri possesses a large set of negative markers. Two of these forms are basic negative particles, \textit{ke}, a clausal negator (which can be used as interjection), and \textit{ˈkíti}, a prohibitive (negative imperative) particle. Negative particles may combine with other morphemes, yielding morphologically complex negative markers. The set of negative particles and complex negative markers available in the language are provided in \tabref{tab:key:3}, with their gloss, function and approximate translation.

%\break

\begin{table}
\caption{Negative markers}
\label{tab:key:3}

\begin{tabularx}{\textwidth}{llQQ}
\lsptoprule
\textbf{Form} & \textbf{Gloss} & \textbf{Function} & \textbf{Translation}\\
\midrule
\textit{ke} & \textsc{neg} & interjection, clausal negation & ‘no’\\
\textit{ˈkíti} & \textsc{proh} & prohibitive (negative imperative) & `don't!'\\
\textit{ke ˈtâsi} & \textsc{neg neg} & interjection, clausal negation & ‘no’\\
\textit{ˈpé ke biˈlé} & just \textsc{neg} one & emphatic interjection & `not at all!'\\
\textit{ka t͡ʃè} & \textsc{neg.irr} again & clausal negation &`not again/anymore’\\
\textit{ke/ˈtâsi t͡ʃo} & \textsc{neg} yet & clausal negation & `neither’\\
\textit{ke/ˈtâsi biˈlé} & \textsc{neg} one & clausal negation, constituent neg. & ‘nothing at all’, ‘no single’\\
\textit{ni biˈlé} & nor one & constituent neg. & ‘nor any’\\
\lspbottomrule
\end{tabularx}
\end{table}
%\hspace{3cm}


\section{Nouns and noun phrases}
\label{sec: nouns}

Nouns in Choguita Rarámuri may be inflected for instrumental and locative case and may exhibit complex patterns of possessive marking. A subset of nouns (those referring to animate referents) may also be marked as plural.

Instrumental and locative case markers are productive. Locative case is encoded with two suffixes: \textit{-tʃ͡í} `on, at', with an adessive reading, and \textit{-rare} ‘at, among, between’, with an inessive reading, e.g., \textit{saʔpaˈtʃ͡í} `in the (live, human)  flesh' vs.\textit{ saʔˈpáriri} `in the (severed, animal's) meat', reflecting an alienable/inalien\-able distinction. The choice between locative suffixes for some nouns is however a matter of lexical choice, and some nouns may be optionally marked with either locative suffix (e.g., \textit{maˈtá-riri} \textasciitilde \textit{mata-ˈtʃ͡í} `in the metate').

Possessive constructions in Choguita Rarámuri encodes possessors and mero\-nym\-ic (part-whole) relationships. Inalienable (kinship and body-part terms) and alienable nouns may register a possessor with the stress-shifting possessive \textit{-lâ} suffix. Alienable nouns are characterized by being able to appear in non-pos\-ses\-sive constructions, while inalienable nouns are exclusively found with possessive marking. The possessive \textit{-lâ} suffix attaches to the head noun in a possessive nominal phrase, following the head-marking profile of the language. Examples of the possessive constructions are shown in (\ref{ex: possessive examples overview}).

\newpage

\ea\label{ex: possessive examples overview}
{Possessive marking}

    \ea[]{
    \textit{aliˈwâla}\\
    \textit{aliˈwâ-la}\\
    soul-\textsc{poss}\\
    `his soul'\\
    `su alma'{\corpuslink{tx5[00_307-00_350].wav}{LEL tx5:00:30.7}}\\
}
        \ex[]{
        \textit{maiˈrála}\\
        \textit{maiˈrá-la}\\
        father.female.ego-\textsc{poss}\\
        `her father'\\
        `su papá (de ella)'{< FLP in61(260)/in >}\\
    }
            \ex[]{
            \textit{aʔkaˈlâ}\\
            \textit{aʔka-ˈlâ}\\
            sandals-\textsc{poss}\\
            `X's sandals'\\
            `sus huaraches' {< BFL 09 1:60/el >}\\
        }
                 \ex[]{
                 \textit{boʔeˈlâ}\\
                 \textit{boʔe-ˈlâ}\\
                 road-\textsc{poss}\\
                 `X's road'\\
                 `su camino' {< BFL 06 5:128/el >}\\
            }
    \z
\z

In addition to this construction, possessives may be marked in a `double possessive' marking pattern with a \textit{-wa-la} suffix sequence, e.g., \textit{waˈsá-\textbf{wa-la}} ‘X’s sowing field’ (`su campo de cultivo'). For some speakers, the single vs. double marking of possession encodes a singular/plural distinction, e.g., \textit{ne waˈrî-\textbf{la}} `my palm basket’ (`mi canasta de palma') vs. \textit{ne waˈrî-\textbf{wa-la}} `my palm baskets’ (`mis canastas de palma').

Nouns may be modified by demonstratives, adjectives, numerals, definite articles and quantifiers in noun phrases. One common modifier found in noun phrases are definite articles, which encode number and affective stance, either positive or neutral, on the one hand, or negative. The paradigm of definite articles is shown in (\ref{ex: articles overview}).

\ea\label{ex: articles overview}
{Choguita Rarámuri definite articles}
  \ea  \textit{ˈtá}\hphantom{siʃ͡i}\hspace{1ex}  Singular article, positive or neutral evaluation (lit. `small, \textsc{sg}')
  \ex \textit{ˈtí}\hphantom{stʃ͡i}\hspace{1ex}  Singular article, negative evaluation
  \ex  \textit{ˈkútʃ͡i}\hspace{1ex} Plural article, positive or neutral evaluation (lit. `small, \textsc{pl}')
  \ex \textit{ˈtʃ͡éti}\hphantom{i}\hspace{1ex}  Plural, negative evaluation
    \z
\z

\section{Verbs}
\label{sec: verbs}

Verbal morphology in Choguita Rarámuri is largely concatenative, though exponence is also achieved through stress shifts, grammatical tone and other non-concatenative processes. Furthermore, a large number of morphologically-con\-di\-tioned phonological process affect the surface form of inflected verbs.

Verbal roots can be divided in three classes depending on their underlying stress and vowel specifications: (i) \textbf{Class 1} verbs are stressed (with fixed stress across paradigms) and do not exhibit any vocalic alternations (e.g., \textit{beˈnè} `to learn'); (ii) \textbf{Class 2} verbs are unstressed (exhibiting stress shifts when attaching stress-shifting suffixes) and do not exhibit vocalic alternations (e.g., \textit{suˈkú} `to scratch'); and (iii) \textbf{Class 3} verbs are unstressed and undergo final root vowel raising in addition to stress shifts when attaching shifting suffixes (e.g., \textit{raʔˈlá} `buy'). These verbal classes are illustrated in \tabref{tab:verb-classses}.

\begin{table}
\caption{Choguita Rarámuri verbal root classes}
\label{tab:verb-classses}

\begin{tabularx}{\textwidth}{lXXXX}
\lsptoprule
& \textbf{Class 1} & \textbf{Class 2}  & \textbf{Class 3} & \\
& \textit{Stressed} & \textit{Unstressed} & \textit{Unstressed} &  \\
& &  & \textit{V raising} & \\
\midrule
\textsc{pst} &  beˈnè-li  &     suˈkú-li      &   raʔˈlà-li & \textbf{Neutral}\\
\textsc{prog} &  beˈnè-a &    suˈkú-a &  raʔˈlà-a    & \textbf{Constructions}\\
\textsc{impf} &   beˈnè-i &   suˈkú-i  &  raʔˈlà-i &              \\
\tablevspace
\textsc{fut.sg} & beˈnè-ma   &          suku-ˈmêa   &     rari-ˈmêa & \textbf{Shifting}\\
\textsc{cond} &  beˈnè-sa & suku-ˈsâ    &  rari-ˈsâ &   \textbf{Constructions}\\
\textsc{desid} &  beˈnè-nale    & suku-ˈnále    &  rari-ˈnále & \\
\lspbottomrule
\end{tabularx}
\end{table}


Class 2 and Class 3 roots may undergo valence related alternations through the affixation of transitive and applicative vocalic suffixes that replace the final vowel of the stem. \tabref{tab:valence-allomorphy} schematizes the three-way contrast of these stems (syntactic/semantic gaps are symbolized by dashes).

%\break

\begin{table}
\caption{Valence stem allomorphy}
\label{tab:valence-allomorphy}

\begin{tabularx}{.9\textwidth}{llllQ}
\lsptoprule
& \textbf{Intransitive}  &   \textbf{Transitive} &  \textbf{Applicative } &     \textbf{Gloss}\\
\midrule
a. & suˈwí    & suˈwá     &     suˈw-è &                  ‘run out/finish up’\\
b. & saˈwí          &   - &                 saˈw-è &                  ‘cure, heal’\\
c. &  - &               raʔˈlá      &       raʔˈl-è &                   ‘buy'\\
d. &  noko      &      - &                noˈk-è      &            ‘move’\\
e. &   - &              iˈtʃ͡á      &        iˈtʃ͡-ì  &                   ‘plant'\\
f. &  uku  &             - &                uˈk-è     &             ‘rain'\\
g. & wili-  &             wiˈlá  &         wiˈl-è      &            ‘stand’\\
h. & tʃ͡oʔi  &             tʃ͡oʔˈá &           tʃ͡oˈʔ-ì &                   ‘extinguish’\\
i. &  -     &               oˈsá &           oˈs-è     &               ‘write’\\
j. &  -      &              kiˈmá  &         kiˈm-è     &            ‘cover with blanket’\\
\lspbottomrule
\end{tabularx}
\end{table}

Finite Choguita Rarámuri verbs are marked for tense and/or aspect, mood distinctions (including imperative and reportative) and voice, as well as number and, in the case of the past egophoric suffix, person, conflated in portmanteaux suffixes. \tabref{tab:inflection-paradigms} illustrates a subset of inflectional exponents in verbs of different prosodic characteristics.\footnote{Note this table does not provide an exhaustive list of possible TAM inflection in Choguita Rarámuri nor the verbs represented constitute a complete sample of prosodic types of verbs.}

\begin{table}
\caption{Choguita Rarámuri verbal root classes}
\label{tab:inflection-paradigms}

\begin{tabularx}{\textwidth}{XXXX}
\lsptoprule
\textbf{TAM} & \textbf{\textit{ˈtò} `take'} & \textbf{\textit{iˈsî} `urinate'}  & \textbf{\textit{raˈʔìtʃa} `speak'} \\
\midrule
\textsc{pst} &  ˈtò-li  &     iˈsî-li      &  raˈʔìtʃ͡a-li \\
\textsc{pst.ego} &  ˈtò-ki  &     iˈsî-ki      &   raˈʔìtʃ͡a-ki \\
\textsc{prog} &  ˈtò-a &    iˈsî-a &  raˈʔìtʃ͡-a    \\
\textsc{impf} &   ˈtò-i &   iˈsî-i  &  raˈʔìtʃ͡a-i  \\
\textsc{pst.pass} &  ˈtòo-ru    & iˈsîi-ru    &  raʔiˈtʃ͡âa-ru \\
\textsc{fut.sg} & to-ˈmêa   &          iˈsî-ma   &    raʔiˈtʃ͡â-ma \\
\textsc{fut.pl} & to-ˈbô   &          iˈsî-bo   &     raʔiˈtʃ͡â-bo \\
\textsc{cond} &  to-ˈsâ & iˈsî-sa    &  raʔiˈtʃ͡â-sa \\
\textsc{imp.sg} &  to-ˈkâ    & iˈsì    &  raʔiˈtʃ͡â \\
\textsc{imp.pl} &  to-ˈsì    & iˈsî-si    &  raʔiˈtʃ͡â-si \\
\lspbottomrule
\end{tabularx}
\end{table}

While most inflectional categories involve non-flexive formatives, some constructions exhibit lexically conditioned allomorphy. This is illustrated by the imperative singular construction, which may be encoded by suffixes (e.g., \textit{toˈkâ!} `take it!'), a L tone (\textit{iˈsì!} `urinate!'), or a stress shift (e.g., \textit{raʔiˈtʃ͡â!} `speak!'). Other TAM exponents exhibit phonologically-conditioned allomorphy: the future singular has an allomorph \textit{-ˈmêa} when attaching to unstressed roots (e.g., \textit{to-ˈmêa} `s/he will take it'), and an allomorph \textit{-ma} when attaching to stressed roots (e.g., \textit{iˈsî-ma}).

This table also illustrates the morphologically-conditioned shifts undergone by unstressed roots (e.g., \textit{ˈtò} `take' and \textit{raˈʔìtʃ͡a} `speak'), tonal alternations in verb stems after stress shifts (e.g., \textit{raˈʔìtʃ͡a-li} `s/he spoke' vs. \textit{raʔiˈtʃ͡â-bo} `they will speak'), and morphologically-conditioned phonological effects, such as final stem vowel lengthening triggered by the past passive suffix \textit{-ru} (e.g., \textit{ˈtòo-ru} `it was taken').

Clauses with verbs inflected for past egophoric, encoding an event carried out in the past by a first-person subject in statements and by a second-person subjects in question, have optional nominative-marked pronominal marking. This is exemplfieid in (\ref{ex: past egophoric}).

\ea\label{ex: past egophoric}

[ˈpòləki]\\
\glt    /ˈpòli-ki/\\
\glt        cover-\textsc{pst.ego}\\
\glt    `I covered it.'\\
\glt    `Lo tapé.' < AHF 05 1:125/el >\\

\z

\largerpage
In addition to stress changes, grammatical tone patterns, and other morphol\-o\-gic\-ally-conditioned phonological effects, Choguita Rarámuri inflected verbs exhibit templatic effects in some morphological constructions. Specifically, syllable truncation is attested in denominal verb forms, body part incorporation and V-V compounding to satisfy requirements on output surface forms. An example is provided below, where disyllabic mood/aspect markers undergo syllable truncation in V-V compounding when attaching an outer inflectional suffix (\ref{ex: templatic truncation in aspect/mood}a--b), but remain disyllabic otherwise (\ref{ex: templatic truncation in aspect/moodc}).

\ea\label{ex: templatic truncation in aspect/mood}
{Templatic truncation in V-V compounding}

    \ea[]{
    [aˈtʃ͡ènisa]\\
    \glt    /aˈtʃ͡è-nale-sa/ \\
    \glt        pour-\textsc{desid-cond}\\
    \glt    `if s/he wants to pour it'\\
    \glt    `si lo quiere echar, verter' < SFH 07 romara/tx >\\
}\label{ex: templatic truncation in aspect/mooda}
        \ex[]{
		[tiˈtʃíiksima]\\
		\glt    /tiˈtʃíi-ki-simi-ma/\\
		 \glt       comb-\textsc{appl-mot-fut.sg}\\
	    \glt    `s/he will go along making them comb her/him'\\
		\glt    `va a ir haciéndola que la peine' < SFH 07 2:67/el >\\
	}\label{ex: templatic truncation in aspect/moodb}
	        \ex[]{
            [ˈnârisimi]\\
            \glt    /ˈnâre-simi/\\
             \glt       ask-\textsc{mot} \\
            \glt    `s/he is going along asking'\\
            \glt    `va a ir preguntando' < SFH 08 1:148/el >\\
        }\label{ex: templatic truncation in aspect/moodc}
    \z
\z

\section{Word order}
\label{sec: word order}

Choguita Rarámuri is a head-final language with canonical SOV word order. In ditransitive clauses with non-pronominal noun phrases, the order
of arguments is S-T(heme)-R(ecipient), with variable placement of object noun phrases with respect to the verbal predicate. Word order in ditransitive clauses is exemplified in (\ref{ex: ditransitive clauses overview}) (other ordering possibilities are exemplified and discussed in §\ref{subsec: ditransitive clauses}).

\ea\label{ex: ditransitive clauses overview}
{Ditransitive clauses}\\

    \ea[]{
    \textbf{S-T-V-R}\\
    \textit{mi     muˈkî} \textit{ˈdûlse    ˈàli    ˈkûruwi}\\
    \gll    [mi    muˈkî]\textsubscript{S}    [ˈdûlse]\textsubscript{T}    ˈà-li    [ˈkûruwi]\textsubscript{R}\\
             \textsc{dist}    woman  candy  give-\textsc{pst}  children\\
    \glt    `That woman gave the children candy.'\\
    \glt    `Esa mujer les dió dulces a los niños.' < SFH 09 3:51/el >\\
 }
        \ex[]{
        \textbf{S-T-V-R}\\
        \textit{kumuˈtê ˈlâmina ˈèbili raˈlàmuli}\\
        \gll    /[kumuˈtê]\textsubscript{S} [ˈlâmina]\textsubscript{T}  ˈèbi-li      [raˈlàmuli]\textsubscript{R}/\\
                sherif  tin.roof  bring.\textsc{appl-pst}  people\\
        \glt    `The sherif brought tin roof for the people.`\\
        \glt    `El comisariado les trajo lámina a la gente.’ < SFH 09 3:51/el >\\
    }
    \z
\z

Nominal modifiers precede head nouns in noun phrases, as exemplified in (\ref{ex: noun phrase order overview}) (noun phrases are marked in square brackets).

\ea\label{ex: noun phrase order overview}

    \ea[]{
    \textit{ˈtʃ͡êram ˈsûs ba, ti ˈtʃ͡êram bauˈtîʃ ma ba}\\
    \gll    [ˈtʃ͡êrame ˈsûs] ba, [ti ˈtʃ͡êrame bauˈtîsi] ma ba\\
            elder Jesús \textsc{cl} \textsc{def.sg} elder Bautista also \textsc{cl}\\
    \glt    `elder Jesús, also elder Bautista'\\
    \glt    `Don Jesús, también Don Bautista' \corpuslink{in484[07_185-07_223].wav}{ME in484:07:18.5}\\
}
        \ex[]{
        \textit{biˈlé ariˈmúli, oˈkwâ ariˈmúli ma, biˈkiá ariˈmúli ma}\\
        \gll    [biˈlé ariˈmúli] [oˈkwâ ariˈmúli] ma [biˈkiá ariˈmúli] ma\\
                one decaliter two decaliter or three decaliter or\\
        \glt    `one decaliter or two decaliters or three decaliters'\\
        \glt    ‘un decalitro o dos decalitros o tres decalitros’ \corpuslink{tx68[00_258-00_292].wav}{LEL tx68:00:25.8}\\
    }
            \ex[]{
            \textit{waʔˈlû kapaˈnî  aˈnítʃ͡ini ba}\\
            \gll    [waʔˈlû kapaˈnî]  aˈní-tʃ͡ini ba\\
                    big bell make.sound-\textsc{ev} \textsc{cl}\\
            \glt    `the big bell rang (was heard)'\\
            \glt    `se oyó sonar la campana grande' \corpuslink{tx223[03_291-03_318].wav}{LEL tx223:03:29.1}\\
    }
    \z
\z

\section{Appositive possessive constructions and relative clauses}
\label{sec: appositive possessive constructions and relative clauses}

As discussed in §\ref{sec: nouns} above, Choguita Rarámuri has possessive constructions that are head-marking. In addition, there is an alternative way of encoding possession in the language through an appositional construction with \textit{ˈníwa}, a grammatically specialized possessive noun used in possessive constructions derived from a verb (`to have'). The root \textit{ˈníwa} as a possessive noun is marked with the possessive \textit{-lâ} suffix and is part of a possessive phrase, exemplified in (\ref{ex: appositive possession overview}).

\ea\label{ex: appositive possession overview}

{\textit{ˈnà    ko   ˈnè     ˈníala   ˈlîbro   ko}}\\
\gll    ˈnà=ko  ˈnè    ˈníwa-lâ  ˈlîbro=ko\\
        \textsc{prox}=\textsc{emph}  \textsc{1sg.nom} own-\textsc{poss} book=\textsc{emph}\\
\glt    ‘This here is my book.’\\
\glt    ‘Este es mi libro.’ < BFL 06 4:187-189/el >\\

\z

Another type of complex noun phrase involves relative clause formation. Choguita Rarámuri exhibits two types of headed relative clauses: (i) those formed via nominalization (\ref{ex: relative clauses overviewa}) and (ii) those that involve finite predicates and subordinators such as \textit{ˈ(n)api} (\ref{ex: relative clauses overview}c--d).

\ea\label{ex: relative clauses overview}

    \ea[]{
    \textit{ˈétʃ͡i ˈtʃ͡îba \textbf{muˈkúami} ko baˈsûa koˈʔáli}\\
    \gll    ˈétʃ͡i ˈtʃ͡îba muˈkú-\textbf{ami}=ko ma baˈsû-a koˈʔá-li\\
            \textsc{dem} goat die.\textsc{sg-ptcp=emph} already cook-\textsc{prog} eat-\textsc{pst}\\
    \glt    ‘that dead goat (goat that is dead) was already eaten cooked by the dwellers of (the ones that inhabit) that house’\\
    \glt    ‘esa chiva (que está) muerta ya se la comieron cociéndola los (que habitan) de esa casa’ \corpuslink{tx_mawiya[02_439-02_492].wav}{LEL tx\_mawiya:02:43.9}\\
}\label{ex: relative clauses overviewa}
        \ex[]{
        \textit{aʔˈlì ˈétʃ͡i ˈnápu roˈwéma ˈlé ko biˈnôi biˈlá  aˈní}\\
        \gll    aʔˈlì ˈétʃ͡i [\textbf{ˈnápi} roˈwé-ma aˈlé]=ko biˈnôi biˈlá  aˈní\\
                and \textsc{dem} \textsc{sub} run.womens.race-\textsc{fut.sg} \textsc{dub=emph} herself indeed say.\textsc{prs}\\
        \glt    ‘and then the one who will run, herself, says’\\
        \glt    ‘y entonces la que va a correr, ella misma, dice’ \corpuslink{tx19[00_398-00_451].wav}{LEL tx19:00:39.8}\\
    }\label{ex: relative clauses overviewd}
            \ex[]{
            \textit{ripuˈrá...ˈnápu riˈká miˈtʃ͡ípu kuˈʃì ba?}\\
            \gll    ripuˈrá [\textbf{ˈnápi} riˈká miˈtʃ͡ípu kuˈsì] ba\\
                    ax \textsc{sub} that carve sticks \textsc{cl}\\
            \glt    `an ax with which to carve the sticks'\\
            \glt    ‘hacha con que labrar los palos’ \corpuslink{in61[03_306-03_330].wav}{SFH in61:03:30.6}\\
        }\label{ex: relative clauses overviewc}
    \z
\z

\section{Complement clauses and clause chaining}
\label{sec: complement clauses and clause chaining}

Choguita Rarámuri has four major types of complement clauses: (i) finite complement clauses with complementizer (§\ref{subsec: finite complement clauses with complementizer}); (ii) interrogative complement clauses (§\ref{subsec: interrogative final clause}); (iii) asyndetic finite verb complement clauses (§\ref{subsec: asyndetic finite verb}); and (iv) reduced complement clauses (§\ref{subsec: reduced complement clauses}). The first type is exemplified in (\ref{ex: finite complement clause overview}), where the complement clause is introduced by the subordinator \textit{ˈ(n)api} (also used to introduce other subordinate clauses, such as relative clauses and adverbial clauses).

\ea\label{ex: finite complement clause overview}

    \textit{ka ni maˈtʃ͡íki ˈnápu ˈtòoru ba}\\
    \gll    ka=ni maˈtʃ͡í-ki [\textbf{ˈnápi}  ˈtò-ru ba]\\
            \textsc{neg=1sg.nom}  know-\textsc{pst.ego} \textsc{sub} take-\textsc{pst.pass} \textsc{cl}\\
    \glt    `I didn’t know he had been taken’\\
    \glt    `no sabía que se lo habían llevado’ < BFL 09 1:39/el >\\

\z

Choguita Rarámuri also has specialized constructions that involve complementation, including a periphrastic construction encoding indirect causation. In this construction, exemplified in (\ref{ex: indirect causative overview}), a main jussive predicate takes the caused event as a complement and is characterized by the following properties: (i) the lower verb is additionally marked with the jussive verbal affix \textit{nula} ‘order, command’ deriving a co-lexicalized structure within the complement; and (ii) although there are two causative verbs, the causer is expressed only once.

\ea\label{ex: indirect causative overview}

    \ea[]{
     \textit{boˈrêko ma niˈsènula nuluˈrîa}\\
    \gll    [boˈrêko ma niˈsè-\textbf{nula}] nulu-ˈrîa\\
            sheep also shepherd-\textsc{{order}} order-\textsc{hab.pass}\\
    \glt    `they are sent to shepherd sheep, too'\\
    \glt    `los mandan a cuidar borregos también' \corpuslink{tx48[00_548-01_007].wav}{BFL tx48:00:54.8}\\
}\label{ex: indirect causative overviewa}
        \ex[]{
        \textit{ˈémi    taˈmí aˈnèki niˈhê ˈtònula}  \\
        \gll    ˈémi taˈmí aˈn-è-ki [niˈhê \textbf{ˈtò-nula}]\\
                \textsc{2pl.nom} \textsc{1sg.acc} tell-\textsc{appl-pst.ego} \textsc{1sg.nom} take-\textsc{order}\\
         \glt    `You all told me to take it.’\\
         \glt    `Ustedes me hicieron que me lo llevara.’ < BFL 06 4:94/el >\\
}\label{ex: indirect causative overviewb}
            \ex[]{
            \textit{ˈʔwínula tʃ͡e ˈtâsa riˈké la ˈró}\\
            \gll    [ˈʔwí-\textbf{nula} tʃ͡e] ˈtâ-sa riˈké la ˈró\\
                    harvest-\textsc{{order}}  again  ask.for-\textsc{cond} \textsc{dub} perhaps  \textsc{dub}\\
            \glt    `Maybe we can ask (him/her) to harvest again.’\\
            \glt    `A lo mejor le pedimos que vuelva a pizcar.’ < BFL 06 4:94/el >    \\
        }\label{ex: indirect causative overviewc}
    \z
\z

As shown in these examples, only \textit{nula} may appear co-lexicalized with the lower predicate regardless of the specific jussive predicate in the matrix clause (\textit{nula} `order' (\ref{ex: indirect causative overviewa}), \textit{aˈnè} `tell' (\ref{ex: indirect causative overviewb}) or \textit{ˈtâ} `ask' (\ref{ex: indirect causative overviewc})).

A second type of specialized complementation strategy involves a reportative clause construction that features switch reference marking. These constructions involve a matrix clause with a speech predicate and a complement clause, the content of the reported event. When the notional subjects are coreferential, the dependent verb is marked for tense/aspect and with the epistemic \textit{-o} suffix (\ref{ex: switch reference reportative overviewa}). When the notional subjects are not coreferential, the dependent verb suffixes the different referent reportative \textit{-la} suffix (\ref{ex: switch reference reportative overviewb}).

\ea\label{ex: switch reference reportative overview}

    \ea[]{
    \textit{maˈrîa  ko  ke  ʃiˈmíko ˈrú}\\
    \gll    maˈrîa=ko  [ke  ʃiˈmí-\textbf{ki-o}]    ˈrú\\
            Maria=\textsc{emph} \textsc{neg} go.\textsc{sg-}{\textsc{pst.ego-ep}}  say.\textsc{prs}\\
    \glt    `Maria says she didn’t go.’  \\
    \glt    `Dice María que no fue.’ < BFL 09 3:115/el >\\
}\label{ex: switch reference reportative overviewa}
        \ex[]{
        \textit{maˈrîa  ko  ˈhê   aˈní    hoˈsê  ke  ʃiˈmíla ˈruá}\\
        \gll    maˈrîa=ko  ˈhê   aˈní    [hoˈsê  ke  siˈmí\textbf{-la}]    ru-ˈwá\\
                Maria=\textsc{emph}   it  say.\textsc{prs} José  \textsc{neg}  go.\textsc{sg}{\textsc{{}-rep.dr}}  say-\textsc{mpass}\\
        \glt    `Maria says that José didn’t go.’\\
        \glt    `Dice María que José no fue.’ < BFL 09 3:115/el >\\
    }\label{ex: switch reference reportative overviewb}
    \z
\z

In Choguita Rarámuri clause chaining structures, one of the clauses may be marked with canonical inflection, while other clauses in the chaining structure can only be marked with special inflection (the gerundive suffix \textit{-ká}) and overall involve more restricted structures. This inflection mainly conveys a temporal relation of chronological overlap or chronological sequence (temporal notions that may have extended semantic meanings in some contexts): the marked clause may conveys that two events (drinking and resting) take place simultaneously (\ref{ex: clause chaining overviewa}) or that the events conveyed occur in a temporal sequence (\ref{ex: clause chaining overviewb}).

\ea\label{ex: clause chaining overview}

    \ea[]{
      \textit{ˈwé pi ko ne ku iˈsâbika baˈhîba ˈlé}\\
    \gll    ˈwé pi=ko ne ku [iˈsâbi-\textbf{ka}] baˈhî-ba aˈlé\\
            \textsc{int} just=\textsc{emph} \textsc{int} \textsc{rev} rest-{\textsc{ger}} drink-\textsc{irr.pl} \textsc{dub}\\
    \glt    `they need to drink while they rest'\\
    \glt    `necesitan tomar descansando’ \corpuslink{in243[17_222-17_273].wav}{FLP in243:17:22.2}\\
}\label{ex: clause chaining overviewa}
        \ex[]{
        \textit{ku aˈwílitʃ͡i ʃimiˈká, wiˈrómpo ˈkútʃ͡i paˈtʃ͡î ba}\\
        \gll    [ku aˈwílitʃ͡i ʃimi-\textbf{ˈká}], wiˈróm-po ˈkútʃ͡i paˈtʃ͡î ba\\
               \textsc{rev} ritual.patio go.\textsc{sg-{ger}} make.blessing-\textsc{fut.pl} \textsc{def} corn \textsc{cl}\\
         \glt   `having gone back to the ritual patio, we make the blessing (lit. ``throw the water'') with corn'\\
        \glt    `yendo al patio ritual, hacemos la bendición (``echamos el agua") con el maíz' \corpuslink{in485[07_114-07_155].wav}{ME in485:07:11.4}\\
    }\label{ex: clause chaining overviewb}
    \z
\z

\section{Complex predicates}
\label{sec: complex predicates}

There are four types of constructions in Choguita Rarámuri that may be broadly characterized as involving complex predicates: light verb constructions (§\ref{subsec: light verb and auxiliary constructions}), auxiliary verb constructions (§\ref{subsec: auxiliary verb constructions}), serial verb constructions (§\ref{subsec: serial verb constructions}) and multi-predicate verb constructions involving V-V incorporation (exemplified in (\ref{ex: templatic truncation in aspect/mood}) above and addressed in §\ref{subsec: V-V incorporation constructions}).

An example of a light verb construction is provided in (\ref{ex: light verb overviewb}). The verb \textit{noˈká} may be used as a main verb form as a change of posture predicate (glossed as ‘move’) as in (\ref{ex: light verb overviewa}) or it may be a semantically bleached verb bearing inflection in multi-predicate constructions with activity verbs bearing descriptive content, as in (\ref{ex: light verb overviewb})

\ea\label{ex: light verb overview}

    \ea[]{
        \textit{ma noˈkáli}\\
        \gll    \textit{ma} \textbf{\textit{noˈká-li}}\\
                already {move-\textsc{pst}}\\
        \glt    `S/he already moved.’\\
        \glt    `Ya se movió.’ <BFL 05 1:114/el>\\
    }\label{ex: light verb overviewa}
            \ex[]{
            \textit{napaˈwía noˈkáli lé ˈétʃ͡i ˈnà biˈlé riˈhò aʔˈlì biˈlé tʃ͡aˈbôtʃ͡i ˈʃîʔi}\\
            \gll    [\textbf{napaˈwí-a}  \textbf{noˈká-li} aˈlé]   ˈétʃ͡i   ˈnà   biˈlé   riˈhò aʔˈlì   biˈlé   tʃ͡aˈbôtʃ͡i  ˈʃî\\
                    {get.together-\textsc{prog}} {do-\textsc{pst}} \textsc{dub} \textsc{dem} \textsc{dem} one man and one mestizo also\\
            \glt    `A (Rarámuri) man and a \textit{mestizo} (mixed mexican) man got together.’\\
            \glt    `Se juntaron un hombre (Rarámuri) y un mestizo.’ <SFH 06 choma(2)/tx>\\
        }\label{ex: light verb overviewb}
    \z
\z

In contrast, in Choguita Rarámuri auxiliary verb constructions, stative and inchoative posture predicates are deployed in auxiliary verb constructions encoding progressive aspect. The auxiliary verb bears tense marking (present, past or future), while the main lexical verb is inflected for present tense regardless of the tense marking on the auxiliary. This is exemplified in (\ref{ex: auxiliary overview}).

\ea\label{ex: auxiliary overview}

    \ea[]{
    \textit{ma nataˈkêa buˈʔíli ˈnà biˈʔà roˈkò}\\
    \gll    ma [\textbf{nataˈkê-a} \textbf{buˈʔí-li}] ˈnà biˈʔà roˈkò\\
            already {faint-\textsc{prog}} {lie.down.\textsc{sg-pst}} then early night\\
    \glt    `He had already fainted before dawn.'\\
    \glt    `Ya estaba desmayado en la madrugada.' \corpuslink{tx5[04_037-04_070].wav}{LEL tx5:04:03.7}\\
}\label{ex: auxiliary overviewa}
        \ex[]{
        \textit{“tʃ͡in oˈlá ko ˈétʃ͡i”, ˈhê biˈlá ko ˈlàa aˈsáli ˈlé ruˈtûkuri ko ba}\\
        \gll    tʃ͡i=ni oˈlá=ko ˈétʃ͡i ˈhê biˈlá=ko [\textbf{ˈlà-a} \textbf{aˈsá-li}] aˈlé ruˈtûkuri=ko ba\\
                how=\textsc{1sg.nom} do.\textsc{prs=emph} \textsc{dem} that indeed=\textsc{emph} {think-\textsc{prog}} {sit.\textsc{sg-pst}} \textsc{dub} owl=\textsc{emph} \textsc{cl}\\
        \glt    ```That's how I did it to them" that's what he was thinking, the owl.''\\
        \glt    ```Así les hice a esos” eso estaba pensando, el tecolote’ \corpuslink{tx152[07_019-07_050].wav}{SFH tx152:07:01.9}\\
    }\label{ex: auxiliary overviewb}
    \z
\z

As shown in these examples, auxiliary verbs impose no selectional restrictions on the verbs they combine with (e.g., the descriptive verb may be a telic (bounded) predicate like \textit{nataˈkê} ‘faint’ in (\ref{ex: auxiliary overviewa}) or a stative predicate like \textit{ˈlà} ‘think’ in (\ref{ex: auxiliary overviewb})).
