\chapter{Pronouns and demonstratives}
\label{chap: pronouns and demonstratives}

%This chapter should be modelled after the description of River \ili{Guarijío} in Miller 1996

This chapter addresses the morphological properties of pronouns and demonstratives in Choguita Rarámuri, before addressing the internal structure of nominal phrases (addressed in \chapref{chap: noun phrases}).

\section{Pronouns}
\label{sec:21:pronouns}

In contrast to nouns, free personal pronouns and pronominal enclitics are case-marked. Three different sets of pronouns are described: personal pronouns in §\ref{subsec:21:personal pronouns}, interrogative pronouns in §\ref{subsec:21:interrogative pronouns} and emphatic pronouns in §\ref{subsec:21:emphatic pronouns}.

\subsection{Personal Pronouns}
\label{subsec:21:personal pronouns}

%Will need to check whether these forms are used for subjects of both main and subordinate clauses
Choguita Rarámuri personal pronouns distinguish two person values (first and second) and two numbers (singular and plural). In addition to these distinctions, pronouns may encode a binary nominative-accusative case distinction. Nominative pronouns encode subjects of both matrix clauses and subordinate clauses, as well as possessors in noun phrases. Accusative pronominal forms are used to encode single objects of transitive predicates and both primary and secondary objects of ditransitive predicates.\footnote{Nominative/accusative case distinctions in pronominal forms are documented for other Rarámuri varieties, including \ili{Norogachi Rarámuri} (\citealt{brambila1953gramatica}), and \ili{Rochéachi Rarámuri} (\citealt{moralesmoreno2016rochecahi}). For closely related River \ili{Guarijío}, \citet{miller1996guarijio} reports a binary distinction between subject pronominal forms and oblique pronominal forms, with the latter employed to encode objects of transitive and ditransitive clauses, subjects of subordinate clauses and nominal possessors (1996:230).} A subset of pronominal nominative, namely first person forms and the second singular form, are morphologically complex and composed of person forms reconstructed from \ili{Proto-Uto-Aztecan} (\citealt{langacker1977uto}) and a \textit{ˈhê} formative that appears to function as a demonstrative in a limited set of contexts. This \textit{ˈhê} formative may be lacking in reduced forms of these pronominal forms.\footnote{Contexts where \textit{ˈhê} appears to function as a demonstrative are described in §\ref{sec:10:demonstratives} below. \citet{villalpando2019grammatical} documents a similar pattern in \ili{Norogachi Rarámuri} free pronouns (2019:37).} The second singular and first plural nominative pronominal forms also exhibit a change in the final vowel of the reduced form (from [u] to [o]), a change also attested in the first plural accusative pronominal form. The paradigm of Choguita Rarámuri free personal pronouns is shown in \tabref{tab:21:21}.

%\hspace{3cm}
\begin{table}
\begin{tabularx}{.5\textwidth}{lll}
\lsptoprule
& \textbf{Nominative} & \textbf{Accusative}\\
\midrule
 \textsc{1sg} & neˈhê, ˈnè & taˈmí\\
 \textsc{2sg} & muˈhê, ˈmò & ˈmí\\
 \textsc{1pl} & tamuˈhê, taˈmò & taˈmò\\
 \textsc{2pl} & ˈémi & ˈmí\\
\lspbottomrule
\end{tabularx}
\caption{Free personal pronouns}
\label{tab:21:21}
\end{table}
%\hspace{3cm}

Reduced pronominal forms (\textit{ˈnè}, \textit{ˈmò} and \textit{taˈmò}) are prosodically independent: they are stressed, they exhibit associated lexical tonal contrasts and may host enclitics. Reduced pronouns for first and second singular pronouns are monosyllabic and do not undergo lengthening as attested with monosyllabic inflected nouns and verbs, i.e., these pronominal forms are exempt from the minimality requirement described in §\ref{sec: vowel length, stress and minimality effects} that applies to nouns and verbs. Examples of free personal pronouns in main clauses, both unreduced and reduced, are provided in (\ref{ex:21:free personal pronouns}).

\ea\label{ex:21:free personal pronouns}
{Free personal pronouns}

\textit{1\textsc{sg} Nominative}

    \ea[]{
    \textit{\textbf{neˈhê }	  	pe   	oˈkuá 	raʔiˈchâma 	koriˈmá 	ˈhîtara}\\
    \gll    neˈhê 	  	pe   	oˈkuá 	raʔiˈchâ-ma 	koriˈmá 	ˈhîtara\\
	        1\textsc{sg.nom} 	just 	couple 	speak-\textsc{fut.sg} 	fire.bird	about\\
	\glt    ‘I’ll speak a little about the \textit{korima} (the fire bird)’\\
	\glt    ‘Yo voy a hablar poquito del pájaro \textit{korimá} (el pájaro de fuego)’\\
	\glt    < LEL tx5:0:22.9 >\\
}
        \ex[]{
        \textit{``\textbf{ˈnè} ʃiˈmêo ˈrá naʔˈpôʃia", he aˈni}\\
        \gll    ``ˈnè ʃiˈ-mê-o ˈra naʔˈpô-ʃi-a", he aˈni\\
                1\textsc{sg.nom} go-\textsc{fut.sg-ep} \textsc{cer} weed-\textsc{mot-prs} \textsc{dem} say.\textsc{prs}\\
        \glt    ```I'm going to weed", it is said like that'\\
        \glt   ```Voy a escardar", así se dice'\\
        \glt    < JLG el1274:17:17.5 >\\
    }

\textit{1\textsc{sg} Accusative}

	        \ex[]{
		        \textit{noˈrínima ˈétʃi biˈréara tʃoˈkêami \textbf{taˈm}í ruˈwèʃia}\\
                \gll    noˈríni-ma ˈétʃi biˈréara tʃoˈkêami taˈmí ruˈwè-ʃi-a\\
                        arrive-\textsc{fut.sg} \textsc{dem} another bet.settler 1\textsc{sg.acc} tell-\textsc{mot-prs}\\
                \glt    `The other bet settler (of \textit{ariweta} race) arrives to tell me'\\
                \glt    `Viene la otra chokéami (apuntadora de carrera de \textit{ariweta}) a decirme'\\
                \glt    < LEL tx19:1:17.9 >\\
	   }

\textit{1\textsc{pl} Nominative}

                \ex[]{
                    \textit{aʔˈrì \textbf{tamuˈhê} ma … ma aʔˈrì raˈwè ma ti napaˈwí ... hiˈrâmia}\\
                    \gll    aʔˈrì \textbf{tamuˈhê} ma … ma aʔˈrì raˈwè ma ti napaˈwí ... hiˈrâ-mi-a\\
                            and \textsc{1pl.nom} then ... then later day already \textsc{1pl.nom} gather bet-\textsc{mov-prs}\\
                    \glt    ‘And then we gather that day to bet’\\
                    \glt    ‘Entonces nosotros ese día ya nos juntamos para apostar’\\
                    \glt    < LEL tx19:1:32.3 >\\
            }
                    \ex[]{
                    \textit{...ˈkíti ku \textbf{taˈmò} raˈràmuli pe kuˈrì oˈtʃêrikam hu taˈmò ko}\\
                    \gll    ...ˈkíti ku taˈmò raˈràmuli pe kuˈrì oˈtʃêri-k-am hu taˈmò ko\\
                                because \textsc{emph} 1\textsc{pl.nom} Rarámuri just recently grow-k-\textsc{ptcp} \textsc{cop} 1\textsc{pl.nom} \textsc{emph}\\
                    \glt    '... because us the Rarámuri have grown up just recently, our people'\\
                    \glt    '...porque hace poco que crecimos nosotɾos los rarámuri, la gente como nosotros'\\
                    \glt    < FLP in243:18:57.8 >\\
                }

\textit{1\textsc{pl} Accusative}

                        \ex[]{
                        \textit{aʔˈrì ma ke iˈtêo ˈrá na ˈháp \textbf{taˈmò} tʃiriˈká ruˈjèma ko}\\
                        \gll    aʔˈrì ma ke iˈtê-o ˈrá na ˈhápi \textbf{taˈmò} ˈétʃi riˈká ruˈ-jè-ma ko\\
                            and	already	\textsc{neg} not.exist-\textsc{ep} \textsc{cer} \textsc{prox}	\textsc{comp} \textsc{1pl.acc} \textsc{dem} like say\textsc{-appl-fut.sg}	\textsc{emph}\\
	                    \glt    ‘And then there is nothing for us to be told’\\
                        \glt    ‘Y luego ya no hay para que nos digan a nosotras’\\
	                    \glt    < GFM tx905:2:04.1 >\\
                    }
    \z
\z

Second person pronouns do not encode a number distinction in their accusative form (this is also reported in closely related River \ili{Guarijío} (\citealt{miller1996guarijio})). Examples of second person pronominal forms are provided in (\ref{ex:21:2nd person free personal pronouns}).

\ea\label{ex:21:2nd person free personal pronouns}
{Second person free personal pronouns}

\textit{2\textsc{sg} Nominative}

    \ea[]{
        \textit{\textbf{muˈhê} ma ke ˈtʃi ˈbiri rimuˈrú ˈhipi ko ba?}\\
        \gll    muˈhê ma ke ˈtʃi ˈbiri rimu-ˈrú ˈhipi ko ba?\\
                2\textsc{sg.nom} anymore \textsc{neg} which kinds dream-\textsc{prs} today \textsc{emph} \textsc{cl}\\
        \glt    'And you don't dream many things anymore?'\\
        \glt    '¿Y ahora ya no sueñas muchas cosas?'\\
        \glt    < MDH, GCH co1136:16:38.4 >\\
}
            \ex[]{
            \textit{\textbf{ˈmò} we biˈnè ˈétʃi}\\
            \gll    ˈmò we biˈnè ˈétʃi\\
                    2\textsc{sg.nom} \textsc{int} know.\textsc{prs} \textsc{dem}\\
            \glt    '“You know a lot about that...”'\\
            \glt    '"Tu sabes mucho de eso..."'\\
            \glt    < LEL tx5:5:18.9 >\\
        }

\textit{2\textsc{sg} Accusative}

                \ex[]{
                    \textit{baʔaˈrîni		\textbf{ˈmí}		ˈàma}\\
                    \gll    baʔaˈrî=ni		ˈmí		ˈà-ma\\
                            tomorrow=1\textsc{sg.nom}	2\textsc{sg.acc}	look.for-\textsc{fut.sg}\\
	                \glt    ‘I’ll look for you tomorrow’\\
    	            \glt    ‘Mañana te busco’\\
		            \glt    <LEL 09 1:70/el>\\
            }
                        \ex[]{
                        \textit{... aʔˈrì ˈétʃi \textbf{ˈmí}n aˈnèma aʔˈrì}\\
                        \gll    ... aʔˈrì ˈétʃi ˈmí=n aˈn-è-ma aʔˈrì\\
                                ... and \textsc{dem} \textsc{2sg.acc=1sg.nom} say-\textsc{appl-fut.sg} later\\
                        \glt    '"...and then I’ll tell you”'\\
                        \glt    '"...y entonces te digo"'\\
                        \glt     < LEL tx19:1:13.5 >\\
                    }

\textit{2\textsc{pl} Nominative}

                        \ex[]{
                        \textit{kuˈrí oˈtʃêrirami \textbf{ˈémi} ko ba}\\
                        \gll    kuˈrí oˈtʃêri-r-ami ˈémi ko ba\\
                                recently grow-r-\textsc{ptcp} \textsc{2pl.nom} \textsc{emph} \textsc{cl}\\
                        \glt    'You all who have recently grew up'\\
                        \glt    ‘Ustedes los crecidos hace poco’\\
                        \glt    < FLP in61:0:38.3 >\\
                    }
                            \ex[]{
                            \textit{\textbf{ˈémi} ko ne aʔˈrà ˈnâtami ˈníbo ˈra ba}\\
                            \gll    \textbf{ˈémi} ko ne aʔˈrà ˈnât-ami ˈníbo ˈra ba\\
                                    \textsc{2pl.nom} \textsc{emph} \textsc{int} well think-\textsc{ptcp} \textsc{cop-fut.pl} ra \textsc{cl}\\
                            \glt    ‘You all must think well’\\
                            \glt    ‘Ustedes piensen bien’\\
                            \glt    < SFH tx12:11:40.4 >\\
                        }

\textit{2\textsc{pl} Accusative}
                                \ex[]{
                                \textit{pe biˈlá tʃutʃuˈrú na \textbf{ˈmí} ruˈwè ˈémi ˈkíni ˈkûtʃuwa ba ˈne}\\
                                \gll    pe biˈlá tʃutʃuˈrú na ˈmí ruˈ-wè ˈémi ˈkíni ˈkûtʃuwa ba ˈne\\
                                        just really that.much \textsc{dem} \textsc{2pl.acc} say-\textsc{appl} \textsc{2pl.nom} \textsc{1pl.poss} children \textsc{cl} \textsc{cl}\\
                                \glt    ‘Just that much I say to you all, you, my children’\\
                                \glt    ‘Nomás de ese tanto les digo, ustedes, mis hijos’\\
                                \glt < SFH tx12:12:46.1 >\\
                            }
    \z
\z

Third person arguments may be left unmarked (\ref{ex:21:third person arguments encoding}a), or they may be encoded through demonstratives (e.g., the demonstrative \textit{ˈétʃi} in (\ref{ex:21:third person arguments encoding}b)) or through an emphatic pronoun (e.g., \textit{biˈnôi} 'himself' in (\ref{ex:21:third person arguments encoding}c)).

\ea\label{ex:21:third person arguments encoding}

    \ea[]{
    \textit{aʔˈrì ke muˈríwia ruˈwá}\\
    \gll    aʔˈrì ke muˈríwi-a ru-ˈwá\\
            and \textsc{neg} get.close-\textsc{prs} say-\textsc{mid}\\
    \glt    'And they say they didn't get close'\\
    \glt    ‘Y dicen que ellos no se arrimaban’\\
    \glt    < LEL tx109:2:14.2 >\\
}
        \ex[]{
        \textit{aʔˈrì \textbf{ˈétʃi} taˈmí "kuˈmûtʃi" aˈnèma ba?}\\
        \gll    aʔˈrì ˈétʃi taˈmí "kuˈmûtʃi" aˈn-è-ma ba?\\
                and \textsc{dem} \textsc{1sg.acc} kumuchi say-\textsc{appl-fut.sg} \textsc{cl}\\
        \glt    'And will they call me "kumuchi"?'\\
        \glt    '¿Y ellos me van a decir "kumuchi"?'\\
        \glt    < SFH in484:13:59.4 >\\
    }
            \ex[]{
            \textit{ˈapi aʔˈrì \textbf{biˈnôi} wikaˈrá ko ˈhe aˈní ˈru}\\
            \gll    ˈapi aʔˈrì \textbf{biˈnôi} wikaˈrá ko ˈhe aˈní ˈru\\
                    \textsc{comp} then himself sing.\textsc{prs} \textsc{emph} \textsc{dem} say say\\
            \glt    ‘When he sings he says this’\\
            \glt    ‘Cuando canta el asi dice’\\
            \glt    < LEL tx71:3:15.7 >\\
        }
    \z
\z

The use of demonstratives to refer to third person arguments is discussed in more detail in §\ref{sec:21:demonstratives} below.

\subsection{Pronominal enclitics}
\label{subsec:21:pronominal enclitics}

Free nominative pronouns have corresponding enclitic forms, phonologically bound formatives that are prosodically dependent on their host and are unrestricted regarding the syntactic category of the words they attach to, two properties that may serve as diagnostics of clitics cross-linguistically \parencite{bickel2007inflectional}. Choguita Rarámuri pronominal enclitics are unstressed, monosyllabic forms with high front vowels, a trait that may be attributed to general processes of post-tonic vowel reduction (as described in §\ref{subsubsec: stress-based vowel reduction and deletion}). \tabref{tab:21:21} illustrates the clitic pronominal forms (free nominative and accusative pronouns are given in parenthesis).

%\hspace{3cm}

\begin{table}
\begin{tabularx}{.55\textwidth}{lll}
\lsptoprule
& \textbf{Nominative} & \textbf{Accusative}\\
\midrule
 \textsc{1sg} & =ni (neˈhê, ˈnè) & (taˈmí)\\
 \textsc{2sg} & =mi (muˈhê, ˈmò) & (ˈmí)\\
 \textsc{1pl} & =ti (tamuˈhê, taˈmò) & (taˈmò)\\
 \textsc{2pl} & =timi (ˈémi) & (ˈmí)\\
\lspbottomrule
\end{tabularx}
\caption{Pronominal enclitic forms}
\label{tab:21:22}
\end{table}
%\hspace{3cm}

%This should be edited, given the chapter on verbs (or description from there can be removed)

%This chapter should also include discussion about the status of these forms given Morales Moreno thesis -

Choguita Rarámuri person enclitics can attach to verbs and hosts of virtually any category, and, like many other \ili{Uto-Aztecan} languages (\citealt{steele1976law}) and other Rarámuri varieties (\citealt{moralesmoreno2016rochecahi}), are generally in what is traditionally called the Wackernagel position, right after the first accented phrase or sub-constituent of a phrase (\citealt{bickel2007inflectional}). The following examples (in (\ref{ex:21:person enclitic hosts})) illustrate the distribution of person enclitics hosted by a wide range of word classes in a variety of syntactic contexts: (i) complementizers in subordinate clauses (\ref{ex:21:person enclitic hosts}a); (ii) demonstratives within Noun Phrases (\ref{ex:21:person enclitic hosts}b); (iii) preposed particles (\ref{ex:21:person enclitic hosts}c); (iv) negative adverbs (\ref{ex:21:person enclitic hosts}d); (v) epistemic particles (\ref{ex:21:person enclitic hosts}e); (vi) nouns (\ref{ex:21:person enclitic hosts}f); and (vii) free person pronouns (\ref{ex:21:person enclitic hosts}g). As shown in these examples, pronominal enclitic forms may undergo vowel deletion.

%check preposed particles label

\ea\label{ex:21:person enclitic hosts}
{Person enclitic hosts}

\textit{Complementizer}

    \ea[]{
    \textit{riˈmùini [ˈnáp\textbf{tim} noˈkáo}\\
    \gll    riˈmù-i=ni [ˈnáp\textbf{=tim} noˈká-o]\\
            dream-\textsc{impf=1sg.subj} \textsc{compl=2pl.obj} move-\textsc{ep}\\
    \glt    ‘I used to dream that you all were moving’\\
    \glt    ‘Yo soñaba que ustedes se movían’ \\
    \glt    [BL 05 1:114/el]\\
}

\textit{Demonstratives (within a noun phrase)}

        \ex[]{
        \textit{ˈti\textbf{n} toˈrí siʔˈritimo ˈlá}\\
        \gll    ˈti\textbf{=n} toˈrí siʔˈri-ti-mo ˈlá\\
                \textsc{dem=1sg.subj} chicken drown.\textsc{intr-caus-fut.sg} \textsc{cer}\\
        \glt    ‘I will drown the chicken’\\
        \glt    ‘Voy a ahogar al pollo’\\
        \glt    [BFL 05 2:49/el]\\
    }

\textit{Preposed Particles}

            \ex[]{
            \textit{aʔˈrì ˈku\textbf{n} noˈrînima}\\
            \gll    aʔˈrì ˈku=n noˈrîni-ma\\
                    later \textsc{rev}=1\textsc{sg.subj} return-\textsc{fut.sg}\\
            \glt    ‘I will come back later’\\
            \glt    ‘Al rato vuelvo’ \\
            \glt   [BFL 05 2:49/el]\\
        }

\textit{Negative adverbs}

                \ex[]{
                \textit{ˈke\textbf{ni} ˈtáʃi tʃ͡o maˈnâ baˈ\textsuperscript{h}târi}\\
                \gll    ˈke=ni ˈtáʃi tʃ͡o maˈnâ baˈ\textsuperscript{h}târi\\
                        \textsc{neg=1sg.subj} \textsc{neg} yet  make.beverage corn.beer\\
                \glt    ‘I haven’t made corn beer yet’  \\
                \glt    ‘No he hecho tesgüino todavía’ \\
                \glt    [BFL 05 2:56/el]\\
            }

\textit{Epistemic particles}

                    \ex[]{
                    \textit{noˈkèli ˈlé\textbf{n} ˈmáo}\\
                    \gll    noˈk-è-li ˈlé=n ˈmá-o\\
                            move-\textsc{pst} \textsc{dub=1sg.subj} maybe-\textsc{ep}\\
                    \glt    ‘Maybe I moved him’\\
                    \glt    ‘A lo mejor lo moví’\\
                    \glt    [BFL 05 1:114/el]\\
                }

\textit{Nouns}

                        \ex[]{
                        \textit{napaˈrí noˈkáli ronoˈtʃ͡í\textbf{ni} oˈkó}\\
                        \gll    napaˈrí noˈká-li [ronoˈtʃ͡í=ni oˈkó]\\
                                when move-\textsc{pst} legs=\textsc{1sg.subj}  hurt\\
                        \glt    ‘When I moved, my legs hurt’  \\
                        \glt    ‘Cuando me moví me dolieron las piernas’  \\
                        \glt    [BFL 05 1:114/el]\\
                    }

\textit{Full pronouns}

                            \ex[]{
                            \textit{pe taˈmò\textbf{m} naˈhâta iˈʃi}\\
                            \gll    pe taˈmò=m naˈhâta iˈʃì\\
                                    just 1\textsc{pl.obj=dem} follow doing\\
                            \glt    ‘It went like that, following us around’ \\
                            \glt    ‘Así anduvo siguiéndonos’  \\
                            \glt    [BFL 05 text 2/tx]\\
                        }
    \z
\z

Although the list of possible hosts in (\ref{ex:21:person enclitic hosts}) is not exhaustive, it illustrates clearly the unrestrictedness of possible hosts for the person enclitics in Choguita Rarámuri.

\subsection{Emphatic pronouns}
\label{subsec:21:emphatic pronouns}

There are two emphatic pronominal forms in Choguita Rarámuri, listed in (\ref{ex:21:empahtic pronouns}).

\ea\label{ex:21:empahtic pronouns}
{Emphatic pronouns}

    \ea[]{
    \textit{biˈnôi} - singular\\
}
        \ex[]{
        \textit{aˈbôi} - plural\\
    }
    \z
\z

These pronominal forms focus attention on the participants encoded as subjects in contexts where other potential arguments could be subjects. This is exemplified in (\ref{ex:21:emphatic pronouns}).

\ea\label{ex:21:emphatic pronouns}

    \ea[]{
    \textit{aʔˈrì ˈétʃi ˈnápu roˈwéma ˈle ko \textbf{biˈnôi} biˈlá aˈní “ˈjénan ˈa saˈjèrima” ˈa aˈní}\\
    \gll    aʔˈrì ˈétʃi ˈnápu roˈwé-ma ˈle ko \textbf{biˈnôi} biˈlá aˈní “ˈjén a=n ˈa saˈjèri-ma” ˈa aˈní\\
    and \textsc{dem} \textsc{comp} women.race-\textsc{fut.sg} \textsc{dub} \textsc{emph} \textsc{refl.sg} really say yes indeed=\textsc{1sg.nom} indeed take.on-\textsc{fut.sg} indeed say.\textsc{prs}\\
    \glt    ‘And then the one who will run, herself, says: “yes, I will take on the challenge”’\\
    \glt    ‘Y entonces la que va a correr ella misma dice “sí le voy a entrar”’\\
    \glt    < LEL tx19:0:39.8 >\\
}
        \ex[]{
        \textit{ˈnápu riˈká ˈne aˈʔrà beˈnèrpo ˈtʃo ˈkíni ˈkûtʃuwa tʃo ˈkûtʃuwa tˈʃo ˈémi \textbf{aˈbôi} ba niˈbí}\\
        \gll    ˈnápu riˈká ˈne aˈʔrà beˈnè-r-po ˈtʃo ˈkíni ˈkûtʃuwa tʃo ˈkûtʃuwa tˈʃo ˈémi aˈbôi ba niˈbí\\
                \textsc{comp} like \textsc{int} well learn-\textsc{caus-fut.pl} also my children also children also 2\textsc{pl.nom} yourselves \textsc{cl} nibí\\
        \glt    ‘So that we can teach well our children, children, you all’\\
        \glt    ‘Para enseñarles bien a nuestros hijos, los hijos, ustedes mismos’\\
        \glt    < SFH tx12:5:30.0 >\\
}
    \z
\z

In example (\ref{ex:21:emphatic pronouns}a), the singular emphatic pronoun \textit{biˈnôi} makes clear that the runner, and not other potential actors, is the source of the quoted speech. In (\ref{ex:21:emphatic pronouns}b), the  plural emphatic pronoun \textit{aˈbôi} is used to emphasize the addressees, the children of the speaker who he is giving advice to.

\subsection{Interrogative Pronouns and Phrases}
\label{subsec:21:interrogative pronouns}

%%Insert cross-reference to Chapter 9 (constructions)
%%Table 2.2: Choguita Rarámuri interrogative pronouns

Choguita Rarámuri has a set of interrogative pronouns and phrases, most of which are morphologically complex.  Of this set, only four forms are morphologically simplex. These are provided in \tabref{tab:21:1}.

%\hspace{3cm}

\begin{table}
\begin{tabularx}{.5\textwidth}{ll}
\lsptoprule
\textbf{Forms}  & \textbf{Gloss} \\
\midrule
\textit{t͡ʃu?} & How? (\textit{¿Cómo?})\\
\textit{ˈpiri?}  & What? (\textit{¿Qué?})\\
\textit{ˈkami?/ˈkumi?)}  & Where? (\textit{¿Dónde?})\\
\textit{kaˈbu?} & When? (\textit{¿Cuándo?})\\
\lspbottomrule
\end{tabularx}
\caption{Choguita Rarámuri interrogative pronouns: basic forms}
\label{tab:21:1}
\end{table}
%\hspace{3cm}

A recurrent pattern in morphologically complex question words and phrases involve the interrogative word \textit{t͡ʃu} followed by another word, a bound morpheme (e.g., \textit{-ˈrupi} in \textit{t͡ʃu ˈrupi} `How much?') or a demonstrative (e.g., \textit{(ˈnà-ti} `that-with' in \textit{t͡ʃu ˈnà-ti} Wwith what?'). The set of morphologically complex question words is shown in \tabref{tab:21:2}.

%\hspace{3cm}

\begin{table}
\begin{tabularx}{\textwidth}{lQ}
\lsptoprule
\textbf{Forms}  & \textbf{Gloss} \\
\midrule
\textit{t͡ʃu (t͡ʃe) riˈka?} & How?((\textit{¿Cómo?})\\
\textit{t͡ʃi ˈjiri?} & Which kind? (\textit{¿Qué tipo?})\\
\textit{(t͡ʃu) ˈkipi?} & How many? (\textit{¿Cuántos?})\\
\textit{t͡ʃu ˈrupi?} & How much? (\textit{¿Qué tanto?})\\
\textit{t͡ʃu ˈjeni?} & How much? (\textit{¿Qué tanto?})\\
\textit{t͡ʃu iˈkiana?} & How many places? (\textit{¿Cuántos lugares?})\\
\textit{t͡ʃu kiˈnapi?} & At how many places? (\textit{¿Qué tantos lugares?})\\
\textit{t͡ʃu riˈko?}  & When? (\textit{¿Cuándo?})\\
\textit{t͡ʃu (t͡ʃe) oˈra?} & Why? (\textit{¿Por qué?})\\
\textit{t͡ʃu ˈjeni?} & At what time? (\textit{¿A qué hora?})\\
\textit{t͡ʃu kiˈripi?} & How long? (\textit{¿Cuánto tiempo?})\\
\textit{ˈpiri ˈnà-ti?} & With what? (\textit{¿Con qué?})\\
\textit{t͡ʃu ˈnà-ti?} & With what? (\textit{¿Con qué?}\\
\textit{(he pi) ˈkua ˈjua?} & With whom? (\textit{¿Con quién?})\\
\lspbottomrule
\end{tabularx}
\caption{Choguita Rarámuri interrogative pronouns}
\label{tab:21:2}
\end{table}
%\hspace{3cm}

In contrast to free pronominal forms, interrogative pronouns are not case marked, except for the forms \textit{ˈpiri ˈna-ti?} and \textit{ˈt͡ʃu ˈna-ti?} `with what', where the question word (\textit{ˈpiri} or \textit{ˈt͡ʃu} is followed by the proximal demonstrative \textit{ˈnà} bearing the \textit{-ti} instrumental case marker.

In contrast to closely related Mountain \ili{Guarijío} (\citealt{miller1996guarijio}), Choguita Rarámuri does not deploy interrogative pronouns as indefinite pronouns.

%describe here these forms - provide examples
%describe whether these forms may be broken up by other elements

\section{Demonstratives}
\label{sec:21:demonstratives}

Demonstratives are defined here as a set of deictic elements that may refer or restrict reference to referents in a situational (exophoric) usage, identifying entities in a surrounding physical situation, but also may be used endophorically  in discourse and recognitional deixis (\citealt{himmelmann1996demonstratives}, \citealt{enfield2003demonstratives}). Demonstratives in Choguita Rarámuri may function anaphorically as pronouns (§\ref{subsec:21:demonstrative pronouns}) or modifying a nominal element (§\ref{subsec:21:adnominal demonstratives}). Both sets involve the same set of forms, and are distinguished in the following sections in terms of their function. Other potential uses of demonstratives in discourse are yet to be examined and are left out of the scope of this grammar.

\subsection{Demonstrative pronouns}
\label{subsec:21:demonstrative pronouns}

As described in §\ref{subsec:21:personal pronouns}, reference to third person arguments may be achieved through demonstrative pronouns. Two degrees of distance and orientation with respect to the speaker/addressee is encoded by these forms, listed in (\ref{ex:21:demonstratives}):

\ea\label{ex:21:demonstratives}
{Demonstrative pronouns}

    \ea[]{
    \textit{ˈnà} `this one' -- proximal, close to the speaker\\
}
        \ex[]{
        \textit{ˈétʃi} `that one' -- proximal, close to the addressee or the speaker\\
    }
    \z
\z

The use of both proximal demonstrative pronouns is extended in the Choguita Rarámuri corpus. Examples of these demonstrative pronouns is provided in (\ref{ex:21:demonstrative pronouns}).

\ea\label{ex:21:demonstrative pronouns}
{Demonstrative pronouns}

    \ea[]{
    \textit{pe \textbf{ˈnà}bi tʃe ˈtʃéti ko'ríi bi'tíam ku}\\
	\gll    pe ˈnà=bi tʃe ˈtʃéti ko'ríi bi'tí-am ku\\
	       just \textsc{that.one}=just	also \textsc{det.pl} over.there	lie.down.\textsc{pl-ptcp} \textsc{rev}\\
	\glt    ‘Just these ones (the dead people) who are lying down over there also...’\\
	\glt    ‘Nada más estos que están (acostados) ahí por de aquel lado... (los muertos)’  \\
	\glt    <FLP 07 in243(511)/in>\\
}
        \ex[]{
        \textit{"\textbf{ˈnà} ko ˈpâalichi paˈkótami" ˈhê biˈlá aˈní ba aˈní}\\
        \gll    ``\textbf{ˈnà} ko ˈpâalichi paˈkó-t-ami" ˈhê biˈlá aˈní ba aˈní\\
                this.one \textsc{emph} priest wash-\textsc{pacient-ptcp} that really say.\textsc{prs} \textsc{cl} say.\textsc{prs}\\
        \glt    ```This one is baptized by a priest" that's what they say'\\
        \glt    ```Este es bautizado por padre" así dicen'\\
        \glt    < SFH tx475:8:03.9 >\\
    }
            \ex[]{
            \textit{\textbf{ˈétʃi} ko tʃoˈmí ˈtʃo muˈtʃûwi}\\
            \gll    ˈétʃi ko tʃoˈmí ˈtʃo muˈtʃûwi\\
	                those.ones \textsc{emph} there also	sit.\textsc{pl.prs}\\
	        \glt    ‘They were (sitting) also over there’\\
		    \glt    ‘Ellos también estaban allá’\\
	        \glt    <FL 06 in61(302)/in>\\
        }
                \ex[]{
                \textit{\textbf{ˈétʃi} ko we aˈnátʃa}\\
                \gll    ˈétʃi ko we aˈnátʃa\\
                        that.one \textsc{emph} \textsc{int} endure.\textsc{prs}\\
                \glt    `That one really endures (to run)'\\
                \glt    `Ese aguanta mucho (correr)'\\
                \glt    < JLG el1278:3:59.6 >\\
            }
    \z
\z

In addition to these demonstrative pronouns, a formative \textit{ˈhê} appears followed by utterance predicates (`say', `tell'), as well as other predicates that express a positive attitude regarding the truth of the proposition expressed as their complement (`think', `believe'). Crucially, this formative appears in contexts where there is quoted speech, and its function appears to be to index the direct speech. Relevant examples are provided in (\ref{ex:21:he demonstrative}).

\ea\label{ex:21:he demonstrative}
{Demonstrative \textit{ˈhê} in quoted speech contexts}

    \ea[]{
    \textit{aʔˈrì \textbf{ˈhê} aˈníli: “ˈnè ko a maˈtʃí ˈkúmi ˈjéna biˈtê ˈétʃi oˈhí”}\\
    \gll    aʔˈrì ˈhê aˈní-li: “ˈnè ko a maˈtʃí ˈkúmi ˈjéna biˈtê ˈétʃi oˈhí”\\
            and that say-\textsc{pst} \textsc{1sg.nom} \textsc{emph} indeed know.\textsc{prs} where indeed live.\textsc{prs} that bear\\
    \glt    ‘And they said: “I do know where the bear lives”’\\
    \glt    ‘Y entonces dijeron: “yo sí se dónde vive el oso”’\\
    \glt    < LEL tx32:2:11.0 >\\
}
        \ex[]{
        \textit{aʔˈrì \textbf{ˈhê} aˈní ˈtîo: “a riˈwèʃi, ˈkíti biˈlé tʃapiˈʃì ˈétʃi”}\\
        \gll    aʔˈrì \textbf{ˈhê} aˈní ˈtîo: “a riˈwè-ʃi, ˈkíti biˈlé tʃapi-ˈʃì ˈétʃi”\\
                and that say.\textsc{prs} uncle indeed leave-\textsc{imp.pl} \textsc{neg.imp} one grab-\textsc{imp.pl} that\\
        \glt    `And my uncle said: "leave it, don't touch it"'
        \glt    ‘Y dijo mi tío: “déjenlo no lo tienten”’\\
        \glt    < LEL tx84:5:43.6 >\\
    }
    \z
\z

%concluding remarks

\subsection{Adnominal demonstratives}
\label{subsec:21:adnominal demonstratives}

Choguita Rarámuri makes use of the same demonstrative forms that may be used pronominally as adnominal demonstratives: \textit{ˈnà} `this, speaker-proximate' and \textit{ˈétʃi} `that, speaker or addressee proximate'. As shown in the examples in (\ref{ex:21:adnominal demonstrative examples}), adnominal demonstratives precede the head noun (further details relating the syntactic order of elements within noun phrases is addressed in \chapref{chap: noun phrases}).

\ea\label{ex:21:adnominal demonstrative examples}

    \ea[]{
    \textit{"ˈpîri ˈtʃêtimi oˈlá taˈmí ke biˈle pe ta ukuˈwèami u pa?" ˈhê riˈká biˈlá iˈjòani \textbf{ˈétʃi Patricio}, ˈa biˈlá ko}\\
    \gll    ``ˈpîri ˈtʃê=timi oˈlá taˈmí ke biˈle pe ta ukuˈwè-ami u pa?" ˈhê riˈká biˈlá iˈjòani ˈétʃi Patricio, ˈa biˈlá ko\\
            why why=\textsc{2pl.nom} why 1\textsc{sg.acc} \textsc{neg} one just little ukuwea-\textsc{ptcp} \textsc{cop} \textsc{cl} that like really angry.\textsc{prs} that Patricio indeed really \textsc{emph}\\
    \glt    ```Why didn't you do that ukuweruwa thing?" says sometimes Patricio'\\
    \glt    ```¿Porqué ustedes no me hicieron eso del ukuwéruwa?" a veces nos dice Patricio enojados'\\
    \glt    < SFH tx475:9:16.8 >\\
}
        \ex[]{
        \textit{aʔˈrì ko ma wiˈrí siˈkôtʃi ˈúmiri \textbf{ˈétʃi koriˈmá}}\\
        \gll    aʔˈrì ko ma wiˈrí siˈkô-tʃi ˈúmiri ˈétʃi koriˈmá\\
                and \textsc{top} already stand.\textsc{sg.prs} corner-\textsc{loc} long.time this fire.bird\\
        \glt    `And then that korima had been standing in the corner already for a long time’\\
        \glt    ‘Y entonces ya andaba mucho rato ahí por el rincón el korimá’\\
        \glt    < LEL tx5:1:28.4 >\\
    }
            \ex[]{
            \textit{\textbf{ˈnà} leˈhîdotʃi ˈtʃo ˈmati waˈna ˈnà roˈhana ˈtʃo}\\
            \gll    ˈnà leˈhîdotʃi ˈtʃo ˈma=ti waˈna ˈnà roˈhána ˈtʃo\\
                    this ejido also already=\textsc{1pl.nom} aside this separate.\textsc{prs} also\\
            \glt    `This \textit{ejido} also, we were separated'\\
            \glt    ‘Este ejido también, ya nos apartaron’\\
            \glt    < JMF tx817:1:03.0 >\\
        }
                \ex[]{
                \textit{tiˈwé \textbf{ˈnà} ˈtʃîkle koˈʔáli}\\
                \gll    tiˈwé ˈnà ˈtʃîkle koˈʔá-li\\
                        girl this gum eat-\textsc{pst}\\
                \glt    `The girl chewed this gum'\\
                \glt    `La niña se comió este chicle'\\
                \glt    < SFH el1028:4:15.8 >\\
            }
    \z
\z

As shown in these examples, the `addressee/speaker-proximate' demonstrative \textit{ˈétʃi} `that' indexes an object or person accessible to the addressee and/or the speaker: in the case of (\ref{ex:21:adnominal demonstrative examples}a), the speaker's son, known also to me, the addressee; in the case of (\ref{ex:21:adnominal demonstrative examples}b), the topic of the narrative, the \textit{korima} (fire bird) that has been previously introduced in discourse. The `speaker-proximate' demonstrative \textit{ˈnà} 'this' is used to index an entity accessible to the speaker, the \textit{ejido} of Choguita in (\ref{ex:21:adnominal demonstrative examples}c) or a piece of gum (\ref{ex:21:adnominal demonstrative examples}d).

These examples illustrate that demonstratives in Choguita Rarámuri encode more than aspects of the spatial configuration, and their use likely determined by interactional, cognitive and discourse pragmatic factors. This topic is left out of the scope of this grammar.
