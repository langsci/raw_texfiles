\chapter{Word Order}
\label{chap: word order}

This chapter addresses basic aspects of word order, including topicalization and focus structures. 

\section{Comparatives}
\label{sec:20:comparative constructions}

Choguita Rarámuri comparative clauses may involve an adjectival predicate (and contain a copula verb) or they may contain a verbal predicate. The gradable predicate is expressed through an adjective predicate and the degree is optionally expressed through an adverb or a suffix in the predicate. The object of comparison is a nominal phrase that is marked as subject if pronominal and optionally followed by an emphatic particle. Finally, the standard of comparison is introduced through a post-positional phrase headed by \textit{yúa} ‘with’ or \textit{kítara} ‘about’. (25) shows a schema of the order of elements in comparative constructions:

\begin{itemize}
\item Order of elements in comparative constructions
\end{itemize}

  Object of     (Adverb)  Gradable  (Copula)  Standard of   Postposition  

  comparison NP       Predicate      comparison NP

  The examples in (26) illustrate the structure of comparative constructions: 

 \textbf{obj. of comp.              adv.    grad. predicate} 

\begin{itemize}
\item a.   [\textit{shiká  watoná}] \textit{t͡ʃukú-am     ka  we}  \textit{rasíram} [\textit{wá-ami}]   
\end{itemize}

/seká  watoná  t͡ʃukú-ame    ka  we  rasírame  wá-ame/  

arm  right    be.bent-\textsc{ptcp  emph  int} more strong-\textsc{ptcp} 

 \textbf{copula  standard of comp.  postposition}  

  hu  [\textit{kohiná}]   \textit{yúa}

  /hu   kohiná      yúa/

  \textsc{cop.prs}  left        with

  ‘The right arm is stronger than the left one’

  ‘El brazo derecho es más fuerte que el izquierdo’

  [BFL 09 1:30/el]

 \textbf{obj. of comp.    grad. predicate  copula  standard of comp.  postposition}

 b.   [\textit{mo}] \textit{ko} [\textit{wi’rí}] \textit{hu} [\textit{taˈmi}] \textit{yúa}

\textsc{2sg.subj  emph} tall      \textsc{cop.prs  1sg.obj} with

‘You are taller than me’

‘Tu estás más alto que yo’

[LEL 09 2:6/el]

  \textbf{obj. of comp.    grad. predicate  copula  standard of comp.  postposition}

  c.  [\textit{mo}] \textit{ko} [\textit{téeri}] \textit{hu} [\textit{taˈmi}] \textit{yúa}

 /mo    ko    téri      hu    taˈmi        yúa/

\textsc{2sg.subj  emph} tall      \textsc{cop.prs  1sg.obj} with

‘You are shorter than me’

‘Tu estás más chaparro que yo’

[GFM 09 3:101/el]

  The example sentences in (27) exemplify negative comparisons (‘less than’). Negative comparisons are expressed through negative particles that have under their scope the gradable predicate (describe more). 

\begin{itemize}
\item a.   \textit{ét͡ʃi    tiwé  ko  ke  me    waríni  loréna  yúa}
\end{itemize}

/ét͡ʃi  tewé  ko  ke  me    warína  lorena  yúa/

\textsc{dem} girl  \textsc{emph  neg} almost  fast    Lorena  with

‘That girl is less light (fast) than Lorena’

‘Esa muchacha es menos ligera que Lorena’

[LEL 09 2:6/el]

 b.  \textit{he    na  aséite    ko  ke  me    natikí    he  na  kítara}

/he  na   aséite    ko  ke  me    natikí    he  na  kítara/

it  \textsc{prox} oil \textsc{emph  neg} almost expensive  it  \textsc{prox} about 

‘This oil i  s less expensive yhan this other one’

‘Este aceite es menos caro que este otro’

[LEL 09 2:6 Steffel/el]

  c.  \textit{wasat͡ʃí  ko  ke  me    aparú-ame  hu    wasat͡ʃí  wa’rú-a-ra  úa}

   coyote  emph  neg  almost  angry-ptcp  cop.prs coyote  big-a-nmlz  with

    ‘The coyote is less dangerous than the wolf’

    ‘El coyote es menos bravo que el lobo’

    [SFH 09 3:85/el]

Comparison by conjunction (parataxis) with verbal ellipsis:

  c.  \textit{na’í  ko    ke  me    ruruwá,  nihé    bitérit͡ʃi  ko  we}

 here  emph    neg  almost  cold    1sg.subj  house  emph  int

    ‘It is less cold here than were I live’

    (lit. ‘It is not very cold here. It is very (cold) where I live’)

    ‘Hace menos frio aqui que en donde vivo’

  [SFH 09 3:85/el]

Comparative constructions may also include a gradable predicate which is morphologically marked as comparative through the suffix \textit{{}-bé} ‘more/surpass’. The examples sentences in (28) illustrate this morphological marker:

\begin{itemize}
\item a.   \textit{ka   t͡ʃe}   \textbf{\textit{wika-bé}} \textit{rihóo-ri   ba   oˈkua   rihóo-r-am     ba} 
\end{itemize}

/ka  t͡ʃe  \textbf{wika-bé}  rihói-li    ba  oˈkua  rihói-li-ame    ba/

  \textsc{neg  neg} \textbf{far-more}  inhabit.\textsc{pl-pst  cl} few  inhabit\textsc{.pl-pst-ptcp  cl}

\textit{pe   be-sá     makói   rihóo-ri   ré} 

/pe  be-sá    makói  rihói-li    aré/

just   three-times  ten  inhabit.\textsc{pl-pst  dub}

‘Because there were almost no people living here, more far away, just like thirty lived here’

‘Porque casi no había gente aquí, había muy poquita como treinta yo creo’

[JMF 09 tx817(6)/tx]

 b.  \textit{ní-ma     be  rá-o     pe   a} \textbf{\textit{wiri-bé}} \textit{iná-ma} 

\textsc{cop-fut.sg  be} think-\textsc{ep} just  \textsc{aff} \textbf{long-more}  go.along\textsc{.sg-fut.sg} 

\textit{ré  mo     ko   ba} 

\textsc{dub  2sg.subj  emph  cl}

  ‘I think so, you will be around (live) for a loger time’

‘Yo creo que si, tu si vas a andar mucho tiempo’

  [FLP 06 in61(704)/in]

c.  \textit{nápi   shibiríko   días   atí   kát͡ʃi} \textbf{\textit{wiri-bé} } \textit{wasa-rú}\textit{{}-i     ba}

  /nápi  sibiríko  días  atí  kát͡ʃe  wiri-bé    wasa-rú-i    ba/

\textsc{comp}  Federico  Díaz  sit.\textsc{sg}  \textsc{neg} \textbf{wide\textsc{{}-}}\textbf{more} barbechar-\textsc{nmlz-impf}  \textsc{cl}

  ‘Where Federico Díaz lives there was not much plowed land (the land was ot very wide)’

‘Donde está Federico Díaz casi no estaba barbechado (no estaba muy ancha la tierra)’

  [FLPP 06 in61(94)/in]

d.   \textit{ah!   abé=mi} \textbf{\textit{ripa-bé}} \textit{shi-mé     orá-ri}

  /ah  abé=mi    \textbf{ripa-bé}  si-méa    orá-ri/

Ah  more=2\textsc{sg.subj}  up-more  go.\textsc{sg}{}-\textsc{fut.sg}  make-ri

  ‘Oh, you were going to go up higher (in studies, smarter)’

‘Ah, ibas a ir muy arriba (inteligente)’  

  [SFH 06 in61(277)/in]

This morphological construction also marks superlative constructions, as shown in (29) below.

\begin{itemize}
\item a.   \textit{a’rí   ét͡ʃi   ta-}\textbf{\textit{bé}}\textit{{}-a-ra       bini-rá     ko   á} 
\end{itemize}

  and  \textsc{dem} small-\textbf{more}{}-\textsc{prs-nmlz} small.sister-\textsc{poss  emph  aff} 

 \textit{ripí-ri     ét͡ʃi   ko}

 remain-\textsc{pst   dem  emph}

  ‘And then the youngest sister stayed’

  ‘Y entonces la hermana menor se quedó’

  [LEL tx32(33)/tx]

 b.   \textit{serebério   be   ko   riwé-i       ta-}\textbf{\textit{bé}}\textit{{}-a-ra     ko}

  Silverio  be  \textsc{emph}  be.named-\textsc{impf}  small-more-\textsc{prs-nmlz}  \textsc{emph}

  ‘Silverio was the name of the youngest one’

‘Se llamaba Silverio el más chico’

  [FLP 06 in61(286)/in]

Comparatives may also involve a verbal predicate, as the following examples (36) show. (36a) contains no gradable predicate.

\begin{itemize}
\item a.  \textit{ne     ko  we  mat͡ʃí  shimé-a  biolín  mo    kítara}
\end{itemize}

/ne    ko  we  mat͡ʃí  semé-a    biolín  mo    kítara/

1sg.subj  emph  int  know  play-prs  violin  2sg.subj  about

‘I know how to play the violin better than you’

‘Yo se tocar más el violín que tu’

[GFM 09 3:101/el]

  b.  \textit{ne    ko  we  rasíra  nót͡ʃiri    orá  bené-a    muhé}

 /ne    ko  we  rasíra  nót͡ʃi-li  olá  bené-a    muhé  

1sg.subj  emph  int  more  struggle-pst  cer  learn-prs  2sg.subj

yúa

/yúa/

with

‘I struggle more to leasrn than you’

‘Batallo más para aprender que tu’

[GFM 09 3:101/el]

Choguita Rarámuri comparative construction are of a type that has been characterized as ‘particle comparative’ \citealt{stassen1984comparative}, and is a type of construction identified in other \ili{Uto-Aztecan} languages (refs). This construction is characterized across typologically diverse languages as including [X]
