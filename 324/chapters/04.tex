\chapter{Syllables}
\label{chap: syllables}

This chapter describes the syllabification patterns of Choguita Rarámuri. Native speakers segment words into syllables in careful speech pronunciation, and several suprasegmental processes make crucial reference to the syllabic structure of words, such as stress assignment (described in §\ref{subsec: stress and lexical tone}) and glottal stop prosody (described in §\ref{subsec: glottal stop}).

The structure of this chapter is as follows. §\ref{subsec: underlying syllable structure} provides an overview of the underlying syllabic structure of Choguita Rarámuri, followed by a description of attested derived consonant and vowel sequences in §\ref{subsec: consonant sequences} and §\ref{subsec: vowel sequences}, respectively. Finally, §\ref{sec: semi-vowels} covers syllable structures involving semi-vowels.
%note that this contains stuff that belongs in processes (or should be cross-referenced)

\section{Underlying syllable structure}
\label{subsec: underlying syllable structure}

The underlying syllable in Choguita Rarámuri is an open syllable ((C)V). Onsets are optional, and never elaborate to a cluster. As in the closely related \ili{Taracahitic} language, \ili{Guarijío} \parencite{miller1996guarijio}, the only possible (underlying) coda is glottal stop. CV, V, CVʔ, and Vʔ syllables in word-initial position are illustrated in (\ref{ex:4:word-initial CV syllables}--\ref{ex:4:word-initial V syllables}). These syllable types are exemplified in monomorphemic words.

\ea\label{ex:4:word-initial CV syllables}
{Word-initial CV syllables}

    \ea[]{
    \textit{ta.ˈkí}   \\
    ‘instrumental violin piece’   \\
    `canción instrumental en violín' < SFH 06 6:73/el >\\
}
        \ex[]{
        \textit{pa.ˈkó}    \\
        ‘to wash (dishes)’\\
        `lavar trastes' < JHF 04 1:3/el >\\
    }
            \ex[]{
            \textit{ra.ˈpá}   \\
            ‘to split’ \\
            `partir' < AHF 05 1:131/el >\\
        }
                \ex[]{
                \textit{ba.ˈkí}  \\
                ‘to go.in, \textsc{sg}’   \\
                `entrar, \textsc{sg}' \corpuslink{tx191[00_299-00_319].wav}{BFL tx191:0:29.9}\\
            }
    \z
\z

\ea\label{ex:4:word-initial Onsetless V syllables}
{Word-initial (onsetless) V syllables}

    \ea[]{
    \textit{o.se.ˈrí}   \\
    ‘paper'\\
    `letter’  \corpuslink{co1235[10_215-10_229].wav}{JLG co1235:10:21.5}\\
}
        \ex[]{
        \textit{i.ˈwí}   \\
        ‘breathe’ \\
        `respirar' < JHF 04 1:1/el >\\
    }
            \ex[]{
            \textit{u.ˈtʃ͡ú }   \\
            ‘sting (a bug)' \\
            `picar (un bicho)' < LEL 07 1:6/el >\\
        }
                \ex[]{
                \textit{a.ˈtʃ͡í}    \\
                ‘to laugh'\\
                `reirse' < SFH 05 1:98/el >\\
            }
    \z
\z

\ea\label{ex:4:word-initial CVʔ syllables}
{Word-initial CVʔ syllables}
    \ea[]{
    \textit{saʔ.ˈpá }  \\
    ‘meat’ \\
    `carne' \corpuslink{tx43[06_078-06_151].wav}{SFH tx43:06:07.8}\\
}
        \ex[]{
        \textit{biʔ.ˈwà}   \\
        \textit{biʔ.ˈw-à}   \\
        ‘to clean, \textsc{tr}’\\
        `limpiar, \textsc{tr}' < BFL 04 1:112/el >\\
    }
            \ex[]{
            \textit{waʔ.ˈkó}  \\
            ‘grunt’  \\
            `gruñir' < AHF 05 1:147/el >\\
        }
                \ex[]{
                \textit{raʔ.ˈlá }  \\
                ‘buy’  \\
                `comprar' < SFH 04 1:74/el >\\
            }
    \z
\z

\ea\label{ex:4:word-initial V syllables}
{Word-initial Vʔ syllables}

    \ea[]{
    \textit{uʔ.ˈká}  \\
    ‘choose’   \\
    `escoger' < SFH 07 DB/el >\\
}
        \ex[]{
        \textit{aʔ.ˈtá}  \\
        ‘arch’   \\
        `arco' < SFH 07 DB/el >\\
    }
            \ex[]{
            \textit{iʔ.ˈpè}   \\
            ‘to gather small things’  \\
            `recoger cosas pequeñas' < BFL 06 4:189/el >\\
        }
                \ex[]{
                \textit{oʔ.ˈté.li }   \\
                ‘to burp’\\
                `eruptar' < SFH 05 1:177/el >\\
            }
    \z
\z

The only types of syllables found word-medially in monomorphemic words are CV syllables. These are exemplified in (\ref{ex:4:word-medial CV syllables}). There are no word-medial CVʔ syllables (i.e., with glottal stop coda), since glottal stop is restricted to appear within a disyllabic window (as the coda of a word-initial syllable or the onset of the second syllable of the prosodic word) (for further details about this restriction, see §\ref{subsec: glottal stop}).

\ea\label{ex:4:word-medial CV syllables}
{Word-medial CV syllables}
    \ea[]{
    \textit{ko.ri.ˈmè.ni  }\\
    ‘bees, honey’\\
    `abejas, miel' \corpuslink{tx191[01_276-01_288].wav}{BFL tx191:01:27.6}\\
    }
        \ex[]{
        \textit{na.ha.ˈrâ.pa }  \\
        ‘to wrestle’  \\
        `luchar' < BFL 07 VDB/el >\\
    }
            \ex[]{
            \textit{su.tu.ˈbé.tʃ͡i}  \\
            ‘to trip' \\
            `tropezarse' < LEL 06 5:35/el > \\
        }
                \ex[]{
                \textit{ba.ʔa.ˈlî  }\\
                `tomorrow'\\
                `mañana' \corpuslink{co1234[08_454-08_469].wav}{JLG co1234:08:45.4}\\
            }
                    \ex[]{
                    \textit{ku.ˈʔî.ri } \\
                    `to help'\\
                    `ayudar' \corpuslink{tx5[04_348-04_395].wav}{LEL tx5:04:34.8}\\
                }
    \z
\z

Word-medial onsetless (V) syllables are not attested in monomorphemic verbal roots, but can be found in polysyllabic nouns. Some of these nominal roots (e.g. (\ref{ex:4:word-medial V syllablesa}) and (\ref{ex:4:word-medial V syllablesc})) are lexicalized compounds.

\ea\label{ex:4:word-medial V syllables}
{Word-medial V syllables}
    \ea[]{
    \textit{a.wa.ˈkó.\textbf{a.}ni}  \\
    ‘scorpion’  \\
    `alacrán' < SFH NDB/el >\\
}\label{ex:4:word-medial V syllablesa}
        \ex[]{
        \textit{ki\textbf{.o.}ˈrîo  }  \\
        ‘esquiate (corn drink)’   \corpuslink{tx904[00_045-00_075].wav}{GFM tx904:00:04.5}\\
    }\label{ex:4:word-medial V syllablesb}
            \ex[]{
            \textit{tʃ͡oˈ.ké\textbf{.a.}ri} \\
            ‘toasted beans'  \\
            `frijol tostado' < SFH NDB/el >\\
        }\label{ex:4:word-medial V syllablesc}
                \ex[]{
                \textit{ko\textbf{.a.}ˈlâ } \\
                \textit{ko\textbf{a}{}-ˈlâ } \\
                forehead-\textsc{poss}\\
                `their forehead'\\
                `su frente' < MGD 06 1:107/el >\\
            }\label{ex:4:word-medial V syllablesd}
    \z
\z

Word-medial onsetless (V) syllables are found after morpheme boundaries in morphologically complex words. Examples are given in (\ref{ex:4:word-medial V syllables in morpheme boundaries}).

\ea\label{ex:4:word-medial V syllables in morpheme boundaries}
{Word-medial V syllables at morpheme boundaries}

    \ea[]{
    \textit{wi.pi.ˈsó.a}\\
   \textit{wipiˈsó\textbf{{}-a}}  \\
    hit-\textsc{prog}\\
    `S/he is hitting.'\\
    `Está pegando.' \corpuslink{tx223[03_035-03_113].wav}{LEL tx223:03:03.5}\\
}
        \ex[]{
        \textit{bu.su.ˈrê.a }\\
         \textit{busuˈré\textbf{{}-a} }\\
         wake.up-\textsc{prog}\\
         `S/he is waking up.'\\
         `Se está despertando.' \corpuslink{tx816[00_166-00_191].wav}{JMF tx816:00:16.6}\\
    }
            \ex[]{
            \textit{su.ˈrí.a}\\
            \textit{suˈrí-\textbf{a}}\\
            fight.over.something-\textsc{prog}\\
            `S/he is fighting about something'\\
            `Se está peleando.' < SFH 05 1:100/el >\\
        }
                    \ex[]{
                    \textit{na.ˈkí.o}\\
                    \textit{naˈki\textbf{{}-o}}\\
                    want-\textsc{ep}\\
                    `S/he wants'\\
                    `Quiere.' < BFL 04 1:91/el >\\
                }
                        \ex[]{
                        \textit{no.ˈká.o}\\
                        \textit{noˈk-á\textbf{{}-o}}\\ 
                        move.\textsc{tr-ep}\\
                        `S/he moves it.'\\
                        `Lo mueve.' < BFL 05 1:114/el >\\
                    }
                            \ex[]{
                           \textit{pi.ˈrê.o}\\ 
                           \textit{piˈrê\textbf{{}-o}}\\ 
                           dwell.\textsc{pl-ep}\\
                            `They dwell.'\\
                            `Viven.' \corpuslink{tx32[02_511-03_012].wav}{LEL tx32:02:51.1}\\
                        }
    \z
\z

All phonemic consonants can be onsets. There are no complex onsets.

\tabref{tab:key:6} summarizes the types of syllables in Choguita Rarámuri and examples of each syllable type in word-initial and word-medial position.

% %%please move \begin{table} just above \begin{tabular .
\begin{table}
\caption{Syllable types and their distribution in Choguita Rarámuri}
\label{tab:key:6}

\begin{tabularx}{.75\textwidth}{lll}
 \lsptoprule
 \textbf{Type} & \textbf{Word-initial}

 \textbf{position} & \textbf{Word-medial}

 \textbf{position}\\
 \midrule
 V & \textbf{i.}ˈwí

 ‘breathe’ & a.wa.ˈkó\textbf{.a}.ni

 ‘scorpion’\\
 CV & \textbf{pa.}ˈkó

 ‘wash dishes’ & o.\textbf{se.}ˈrí

 ‘paper/letter’\\
 CVʔ & \textbf{saʔ.}ˈpá

 ‘meat’ & \\
 Vʔ & \textbf{oʔ.}ˈkô

 ‘pain’ & \\
\lspbottomrule
\end{tabularx}
\end{table}
%\hspace{3cm}

Taking onset complexity (singleton vs. CC onsets) and presence and elaboration of codas as indices of syllable structure complexity (following \citealt{maddieson2005issues}), we can conclude that Choguita Rarámuri has a simple syllable structure in underlying representations: there are no elaborate onsets beyond a single consonant and no codas, except for glottal stop.

In surface forms, however, posttonic deletion gives rise to heterosyllabic consonant clusters, yielding a moderate level of syllable complexity in surface realizations.\footnote{As discussed in more detail in §\ref{subsubsec: stress-based vowel reduction and deletion} below, Choguita Rarámuri exhibits complex patterns of stress-based vowel reduction and deletion. I assume deletion is a categorical phonological process, with vowel deletion yielding a gestural reorganization that affects the syllabic structure of the resulting word. One pending question, however, is whether these cases involve extreme phonetic vowel reduction instead of a phonological process of deletion. As suggested by an anonymous reviewer, in the latter scenario we may expect the syllable pattern of the target word to remain the same, i.e., with the consonant sequence behaving as if the vowel would have remained in place, as proposed for vowel deletion involving some consonant clusters in \ili{Tokyo Japanese} \citep{kawahara2022voweldeletion}. I leave this question for future research.} This is discussed next.

\section{Consonant sequences}
\label{subsec: consonant sequences}

There are very few restrictions as to the types of derived clusters that are possible in Choguita Rarámuri. One of these restrictions involves voiced oral stops. As described in more detail above (§\ref{subsec: underlying syllable structure}), there is a productive rule of oral stop devoicing in post-consonantal position, fed by stress-governed posttonic vowel deletion. There are, thus, no surface sequences involving a voiced stop following a voiceless consonant in a consonant cluster. This is schematized in (\ref{ex:4:stop devoicing}).

\ea\label{ex:4:stop devoicing}
{Stop devoicing}

*[-voice] C - [+voice] stop

\z

Bilabial oral stops in general form heterosyllabic clusters with other voiceless stops and with nasal stops. Nasal stops can form consonant clusters with all other consonants except for other sonorants (although some CC sequences involving identical sonorants are discussed below).

Alveopalatal affricates, on the other hand, may form a heterosyllabic cluster with a following nasal stop (\ref{ex:4:chN derived clusters}a--b) or a velar stop (\ref{ex:4:chN derived clustersc}).

\ea\label{ex:4:chN derived clusters}
{tʃ͡N derived clusters}

    \ea[]{
    [saʔˈmè\textbf{tʃm}a]\\
    /saʔˈmètʃ͡a-ma/ \\
    soak-\textsc{fut.sg}\\
    `It will soak.'\\
    `Se va a mojar' < BFL 05 1:135/el >\\
}\label{ex:4:chN derived clustersa}
        \ex[]{
        [rataˈbá\textbf{tʃn}i]\\
        /rata-bá-tʃ͡a-ni/ \\
        heat-\textsc{inch-tr.pl-appl}\\
        `S/he will heat it up for them.'\\
        `Se lo va a calentar.' < SFH 05:123/el >    \\
    }\label{ex:4:chN derived clustersb}
            \ex[]{
           [rataˈbá\textbf{tʃk}i]\\
            /rata-bá-tʃ͡a-ki/ \\
            heat-\textsc{inch-tr.pl-appl}\\
            `S/he will heat it up for them.'\\
            `Se lo va a calentar.' < SFH 05:123/el >    \\
        }\label{ex:4:chN derived clustersc}
    \z
\z

Another restriction on derived consonant clusters in Choguita Rarámuri involves alveopalatal affricates: a sequence of an alveopalatal affricate followed by an alveolar stop forms an illicit consonant cluster. Repairs include metathesis (\ref{ex:4:repair of cht cluser}a--b), and progressive, assimilatory deaffrication of the alveopalatal stop (\ref{ex:4:repair of cht cluserc}):

\break

\ea\label{ex:4:repair of cht cluser}
{Repair of \textit{tʃ͡t} cluster}

    \ea[]{
    [sutuˈbé\textbf{t}i\textbf{tʃ}ili] \\
    /sutuˈbétʃ͡i-ti-li/ \\
    trip-\textsc{caus-pst}\\
    `S/he made them trip.'\\
    `Lo hizo que se tropezara.' < SFH 09-05-07/el >\\
}\label{ex:4:repair of cht clusera}
        \ex[]{
        [sutuˈbé\textbf{ttʃ}ima]  \\
        /sutuˈbétʃ͡i-ti-ma/\\
        trip-\textsc{caus-fut.sg} \\
        `S/he will make thhem trip.'\\
        `Lo va a hacer que se tropiece.' < SFH 07 1:183/el >\\
    }\label{ex:4:repair of cht cluserb}
            \ex[]{
            [sutuˈbé\textbf{t}tili]\\
            /sutuˈbétʃ͡i-ti-li/    \\
            trip-\textsc{caus-pst}\\
            `S/he made them trip.'\\
            `Lo hizo que se tropezara.' < SFH 07 1:183/el >\\
        }\label{ex:4:repair of cht cluserc}
    \z
\z

%\tabref{tab:key:7} exemplifies some of the derived consonant clusters attested in the CR corpus.

% %%please move \begin{table} just above \begin{tabular .
%\begin{table}
%\caption{Derived CC clusters}
%\label{tab:key:7}
%\end{table}

%\begin{tabularx}{\textwidth}{XXX}
%\lsptoprule
%\textbf{C1} & \textbf{C2} & \textbf{Example words}

%\textbf{\textit{Form                  Gloss}}\\
%p & k & ˈtô-\textbf{p-k}i                ‘bury-\textsc{rev.pst.ego}’\\
%p & m & tʃ͡aˈbo-\textbf{p-m}a          ‘beard\textsc{-rev.fut.sg}’\\
%p & t & raraˈhi\textbf{p-t}i-tʃ͡͡ane    ‘run.race\textsc{-caus-ev}’   \\
%p & tʃ͡ & rono-ˈree\textbf{p-tʃ͡͡}in-o  ‘leg+cut\textsc{-ev-ep)’\\
%b & t & tʃ͡uˈkú-\textbf{b-t}i-si-a     ‘be.bent-\textsc{inch-caus-mot-prog}’/ \\
%t & tʃ͡ & naˈlà-\textbf{t-tʃ͡͡}ane         ‘cry-\textsc{caus-ev}’\\
%t & k & uˈba-\textbf{t-k}i               ‘bathe-\textsc{caus-pst.ego}’\\
%k & t & tiˈtʃi-\textbf{k-t}i-ma          ‘comb-\textsc{appl-caus-fut.sg}’\\
%k & tʃ͡ & tiˈtʃi\textbf{k}{}-\textbf{tʃ͡}ane            ‘comb\textsc{ev}’\\
%k & m & miˈtʃ͡i-\textbf{k-m}a           ‘carve\textsc{-appl-fut.sg}’\\
%m & t & baˈtʃ͡i\textbf{m-t}i-po        ‘sprinkle\textsc{-caus-fut.pl}’\\
%m & p & baˈhi\textbf{m-p}o           ‘sprinkle-\textsc{fut.pl}’\\
%n & k & ˈlaa\textbf{n-k}i                ‘bleed-\textsc{pst.ego}’\\
%n & m & baˈhi\textbf{n-m}a            ‘swell-\textsc{fut.sg}’\\
%n & tʃ͡ & ˈsû-\textbf{n-tʃ͡͡}ane            ‘sew\textsc{appl-ev}’\\
%n & s & iˈkî-\textbf{n-s}i-o             ‘bite-\textsc{desid-mot-ep}’\\
%r & s & ko’-ˈna\textbf{r-s}a           ‘eat-\textsc{desid-cond}’\\
%r & m & beˈne-\textbf{r-m}a           ‘learn-\textsc{caus-fut.sg}’\\
%r & n & na-ˈku\textbf{r-n}ir-o       ‘Pl-help-desid-ep’\\
% r & p & wa’ˈru-\textbf{r-p}o         ‘be.big-caus-fut.pl’\\
% r & t & ˈtaa\textbf{r-t}i-ma           ‘count-caus-fut.sg’\\
% r & tʃ͡ & koˈtʃi-\textbf{r-tʃ͡͡}ane        ‘sleep-caus-ev’\\
% l & tʃ͡ & ru-ˈnaa\textbf{l-tʃ͡͡}in-o      ‘say-desid-ev-ep’\\
% s & p & iˈki\textbf{s-p}o               ‘happen-mot-fut.pl’\\
% s & tʃ͡ & oˈpe\textbf{s-tʃ͡͡}ani            ‘vomit-ev’\\
% s & t & oˈpe\textbf{s-t}i-nil-mo    ‘vomit-caus-desid-fut.sg’/\\
% s & n & iˈki-\textbf{s-n}iri             ‘happen-mot-desid’\\
% w & tʃ͡ & raˈri-\textbf{w-tʃ͡͡}ane         ‘buy-appl-ev’\\
% w & n & aˈsa-\textbf{w-nale}          ‘sit-appl-desid’\\
% \lspbottomrule
%\end{tabularx}

There are thus almost no restrictions as to possible derived consonant clusters in Choguita Rarámuri, except for sequences involving alveopalatal affricates and voiceless stops.

Posttonic vowel deletion also yields sequences of identical stops in morphologically derived environments. The most common type of derived geminates in Choguita Rarámuri involves both oral (\ref{ex:4:bilabila oral stop geminates}) and nasal (\ref{ex:4:bilabial nasal stop geminates}) stops at the bilabial place of articulation.

\ea\label{ex:4:bilabila oral stop geminates}
{Bilabial oral stop geminates}

    \ea[]{
    [naˈté\textbf{pp}o] \\
    /naˈtépi-po/   \\
    meet.up-\textsc{fut.pl}\\
    `They will gather.'\\
    `Se van a reunir.' < BFL 07 el339/el >\\
}
        \ex[]{
        [tʃ͡omaˈî\textbf{p}o]\\
        /tʃ͡omaˈîpi-po/ \\
        cover.face\textsc{fut.pl}\\
        `They will cover their face.'\\
        `Se van a cubrir la cara.' < BFL 07 1:181/el >\\
    }
\pagebreak
            \ex[]{
            [ˈtô\textbf{pp}o] \\
            /ˈtô-pi-po/  \\
            bury-\textsc{rev-fut.pl} \\
            `They will get unearthed.'\\
            `Se van a desenterrar.' < BFL 05 1:113/el >\\
        }
                \ex[]{
                [moˈté\textbf{pp}o] \\
                /moˈtépi-po/  \\
                make.braids-\textsc{fut.pl}  \\
                `They will make braids.'\\
                `Van a trenzar(se).' < BFL 05 1:113/el >\\
            }
                    \ex[]{
                    [tʃ͡aˈbó\textbf{pp}o] \\
                    /tʃ͡aˈbó-pi-po/ \\
                    beard-\textsc{rev-fut.pl}\\
                    `They will shave.'\\
                    `Se van a rasurar.' < BFL 05 1:113/el >\\
                }
    \z
\z

\ea\label{ex:4:bilabial nasal stop geminates}
{Bilabial nasal stop geminates}

    \ea[]{
    [kuˈnà\textbf{mm}a]  \\
    /kuˈnà-mi-ma/  \\
    husband-die-\textsc{fut.sg} \\
    `She will become widowed.'\\
    `Va a enviudar.' < BFL 04 1:37/el >\\
}
        \ex[]{
        [baˈrâ\textbf{mm}a]\\
        /baˈrâmi-ma/  \\
        be.thirsty-\textsc{fut.sg} \\
        `S/he will be thirsty.'\\
        `Tendrá sed.' < BFL 05 1:132/el >\\
    }
            \ex[]{
            [baˈtʃ͡í\textbf{mm}a]  \\
            /baˈtʃ͡ími-ma/ \\
            sprinkle-\textsc{fut.sg} \\
            `It will sprinkle'\\
            `Va a rociar.' < BFL 05 1:135/el >\\
        }
                \ex[]{
                [ˈú\textbf{mm}a]\\
                /ˈhúmi-ma/ \\
                run.\textsc{pl-fut.sg} \\
                `They will run.'\\
                `Van a correr.' < JHF 04 1:19/el >  \\
            }
    \z
\z

Marginally attested are stop geminates that are shown in (\ref{ex:4:marginal types of geminates}). These geminates involve alveolar oral stops (\ref{ex:4:marginal types of geminatesa}), velar oral stops (\ref{ex:4:marginal types of geminatesb}), and alveolar nasal stops (\ref{ex:4:marginal types of geminatesc}).

\ea\label{ex:4:marginal types of geminates}
{Marginal types of geminates}

    \ea[]{
    [naˈhî\textbf{tt}ipo]\\
    /naˈhîti-ti-po/  \\
    become-\textsc{caus-fut.pl} \\
    They will make them turn into something.'\\
    `Los van a hacer que se conviertan en algo.' < BFL 07 el339/el >\\
}\label{ex:4:marginal types of geminatesa}
        \ex[]{
        [ˈjó\textbf{kk}i] \\
        /ˈjóki-ki/   \\
        paint-\textsc{pst.ego}\\
        `I painted.'\\
        `Pinté.' < BFL 07 el339/el >\\
    }\label{ex:4:marginal types of geminatesb}
            \ex[]{
             [piˈtʃ͡í\textbf{nn}ilma] \\
             /piˈtʃ͡í-ni-nale-/ \\
             sweep-\textsc{appl-desid} \\
             `S/he will want to sweep for them.'\\
             `Le va a querer barrer.' < BFL 06 4:145/el >\\
        }\label{ex:4:marginal types of geminatesc}
    \z
\z

Fricative geminates ([ss]) and alveopalatal affricate geminates ([tʃ͡tʃ͡]) are not attested. In contexts where these sequences would arise, there are other mechanisms (such as syllable deletion) that block the application of vowel deletion (e.g., /aˈsísi-sa/ → [aˈsíi-sa] `If she were to wake up', see §\ref{subsubsec: stem-suffix haplology} for more details).

Geminates derived through post-tonic vowel deletion are variably attested across speakers, as there are sequences of syllables with identical onsets that undergo haplology (see §\ref{subsubsec: stem-suffix haplology}). The choice between derived geminates and haplology seems to be correlated with idiolects, although both phenomena are documented with all speakers in the Choguita Rarámuri corpus.

\section{Vowel sequences}
\label{subsec: vowel sequences}

While vowel length is not contrastive in Choguita Rarámuri, there are different morphophological and phonological sources for derived long vowels. Two of these sources, compensatory lengthening and passive-induced lengthening, are discussed in \chapref{chap: verbal morphology}. A third source for derived long vowels is found at morpheme boundaries: in (\ref{ex:4:derived long, low, central vowels}), the vowel initial progressive \textit{-a} suffix creates a long vowel sequence with roots with final, stressed \textit{a}.

\break

\ea\label{ex:4:derived long, low, central vowels}
{Derived long, low, central vowels}

    \ea[]{
    [huˈrâa]\\
    /huˈr\textbf{â-a}/   \\
    send-\textsc{prog}\\
    `S/he is sending it.'\\
    `Lo está mandando.' < BFL 05 1:151/el >\\
}
        \ex[]{
        [biʔˈwàa]\\
        /biʔˈw\textbf{-à-a}/\\
        clean-\textsc{prog} \\
        `S/he is cleaning it.'\\
        `Lo está limpiando.' < BFL 05 1:112/el >\\
    }
            \ex[]{
            [oˈsàa]\\
            /oˈs\textbf{à-a}/ \\
            read.write-\textsc{prog}\\
            `S/he is writing.'\\
            `Está escribendo.' \corpuslink{co1238[02_407-02_420].wav}{JLG co1238:02:40.7}\\
        }
                \ex[]{
                [tʃ͡iˈwáa]\\
                /tʃ͡iˈw\textbf-{á-a}/\\
                rip-\textsc{prog} \\
                `S/he is ripping it.'\\
                `Lo está trozando.' < ROF 04 1:104/el >\\
            }
    \z
\z

The imperfective suffix \textit{-e} undergoes vowel reduction: it is realized in the surface as a high front vowel [i] in post-tonic position, and yields a long vowel sequence with roots with final, stressed \textit{i}, as shown in (\ref{ex:4:derived long, high, front vowels}).

\ea\label{ex:4:derived long, high, front vowels}
{Derived long high, front vowels}

    \ea[]{
    [suˈwíi]\\
    /suˈw\textbf{í-i}/    \\
    eat.up-\textsc{impf}\\
    `S/he used to eat it up.'\\
    `Se lo acababa' < SFH 04 1:119/el >\\
}
        \ex[]{
        [iˈsîi]\\
        /iˈs\textbf{î-i}/ \\
        urinate-\textsc{impf}\\
        `S/he used to urinate.'\\
        `Orinaba.' < SFH 05 1:80/el >\\
    }
            \ex[]{
            [aˈwìi]\\
            /aˈw\textbf{ì-i}/ \\
            dance-\textsc{impf}\\
            `S/he used to dance.'\\
            `Bailaba.' \corpuslink{co1136[12_339-12_374].wav}{MDH co1136:12:33.9}\\
        }
                \ex[]{
                [tʃ͡aˈpíi]\\
                /tʃ͡apˈ\textbf{í-i}/\\
                grab-\textsc{impf}\\
                `S/he used to grab it.'\\
                `Agarraba.' < SFH 05 1:100/el >\\
            }
    \z
\z

Finally, there are also minimal pairs developed through \textit{h} deletion (\ref{ex:4:h deletion}) and labio-velar semi-vowel deletion (\ref{ex:4:labio-velar semi-vowel deletion}) in word-medial position. Both of these processes (exemplified in (\ref{ex:4:h deletionb}) and (\ref{ex:4:labio-velar semi-vowel deletionb}) below) yield a long vowel sequence. As pointed out by an anonymous reviewer, the fact that there is stress retraction in (\ref{ex:4:h deletionb}) makes it clear that this form is not just a case of vowel hiatus (where /VhV/ → [V.V]), but rather involves a true long vowel.

\ea\label{ex:4:h deletion}
{\textit{h} deletion}

    \ea[]{
    [ˈnâta]  \\
    /nâta/ \\
    ‘think’\\
    `pensar' \corpuslink{tx12[01_572-01_588].wav}{SFH tx12:01:57.2}\\
}\label{ex:4:h deletiona}
        \ex[]{
        [naˈhâta] {\textasciitilde} [ˈn\textbf{áà}ta] \\
        /na\textbf{ˈh}âta/  \\
        ‘follow’  \\
        `seguir' \corpuslink{el658[04_145-04_157].wav}{BFL el658:04:14.5}, \corpuslink{in243[15_359-15_390].wav}{FLP in243:15:35.9}\\
    }\label{ex:4:h deletionb}
    \z
\z

\ea\label{ex:4:labio-velar semi-vowel deletion}
{Labio-velar semi-vowel deletion}

    \ea[]{
    [ˈnârma]    \\
    /ˈnâri-ma/  \\
    ask-\textsc{fut.sg}\\
    `S/he will ask.'\\
    `Va a preguntar.'\\
}\label{ex:4:labio-velar semi-vowel deletiona}
        \ex[]{
        [naˈwáruma] {\textasciitilde} [ˈn\textbf{áà}rma]\\
        /na\textbf{ˈw}áru-ma/ \\
        send-\textsc{fut.sg}\\
        `S/he will command.'\\
        `Va a mandar.'\\
    }\label{ex:4:labio-velar semi-vowel deletionb}
    \z
\z

Forms with derived long vowels in (\ref{ex:4:h deletionb}) and (\ref{ex:4:labio-velar semi-vowel deletionb}) coexist with forms with no \textit{h}{}-deletion and labio-velar semi-vowel deletion. These forms are subject to a great deal of speaker variation, and are likely to represent a change in progress.

Other vowel sequences involve diphthongs, tautosyllabic V͡V sequences attested in both stressed and unstressed syllables. This analysis is based on how native speakers segment words into syllables in careful speech pronunciation. Crucially, there are no phonological processes in Choguita Rarámuri that allow probing the phonological behavior of these sequences as similar or different than single vowels. Attested diphthongs in Choguita Rarámuri include falling diphthongs (Vi). These are exemplified in (\ref{ex:4:falling diphthongs}).

\ea\label{ex:4:falling diphthongs}
{Falling diphthongs}

    \ea[]{
    \textit{ei}    \\
    \glt    [seˈmèi]\\
    /seˈm\textbf{è-i}/\\
    play.violin-\textsc{impf}\\
    `He used to play the violin.'\\
    `Tocaba el violín' < SFH 05 1:85/el >\\
}\label{ex:4:falling diphthongsa}
        \ex[]{
        \textit{oi} \\
        \glt    [reˈhòi]\\
        /reˈh\textbf{òi}/     \\
        ‘man’  \\
        `hombre' \corpuslink{co1234[16_063-16_085].wav}{JLG co1234:16:06.3}\\
    }\label{ex:4:falling diphthongsb}
            \ex[]{
            \textit{ui} \\
            \glt    [siˈkúi]\\
            /siˈk\textbf{úi}/       \\
            ‘ant’ \\
            `hormiga' < SFH 05 \textit{láchimi}/tx >\\
        }\label{ex:4:falling diphthongsc}
                \ex[]{
                \textit{ai}   \\
                \glt    [kaiˈnâma]\\
                /k\textbf{ai}ˈnâ-ma/  \\
                yield.harvest-\textsc{fut.sg}\\
                `It will yield harvest.'\\
                `Se va a dar la cosecha.' \corpuslink{tx977[09_367-09_397].wav}{SFH tx977:09:36.7} \\
            }\label{ex:4:falling diphthongsd}
    \z
\z

These falling diphthongs occur lexically in word-final syllables (\ref{ex:4:falling diphthongsa}-\ref{ex:4:falling diphthongsb}), mor\-pheme-internally after stress shifts (\ref{ex:4:falling diphthongsc}) and across morpheme boundaries in word-final syllables (\ref{ex:4:falling diphthongsa}). As the examples in (\ref{ex:4:falling diphthongs}) show, in these V\textsubscript{1}V\textsubscript{2} sequences V\textsubscript{1} is often but not necessarily stressed.

High front vowels are also attested in rising diphthongs with low central vowel off-glides (\ref{ex:4:rising diphthongs}a--b) and mid back vowel off-glides (\ref{ex:4:rising diphthongs}c-\ref{ex:4:rising diphthongs}d). Crucially, these vowel sequences are parsed as tautosyllabic by native speakers in unstressed syllables: as seen in these examples, these cases involve rising diphthongs in post-tonic position in inflected verbs where no further suffixes are added. These diphthongs are therefore all attested word-finally.

\ea\label{ex:4:rising diphthongs}
{Rising diphthongs}

    \ea[]{
    [tʃ͡i.ˈwà.nia]\\
    /tʃ͡iˈwà-n\textbf{i-a}/\\
    rip\textsc{-tr-prog}\\
    `S/he is ripping it.'\\
    `Lo está trozando.' < ROF 04 1:104/el >\\
}
        \ex[]{
        [a.ˈtí.sia]\\
        /aˈtís\textbf{i-a}/ \\
        sneeze-\textsc{prog}\\
        `S/he is sneezing.'\\
        `Está estornudando.' < BFL 05 1:111/el >\\
    }
            \ex[]{
            [wiˈtʃ͡ôsia]\\
            /wiˈtʃ͡ô-s\textbf{i-a}/ \\
            wash-\textsc{mot-prog}\\
            `S/he is washing clothes.'\\
            `Está lavando ropa.' \corpuslink{tx32[04_084-04_119].wav}{LEL tx32:4:08.4}  \\
        }
                \ex[]{
                [tʃ͡o.ˈʔì.sio]\\            
                /tʃ͡oˈʔì-s\textbf{i-o}/ \\
                extinguish\textsc{-mot-ep}\\
                `It's going along getting extinguished (the fire).'\\
                `Se va apagando' < BFL 05 1:112/el >\\
            }
                    \ex[]{
                    [si.ˈrûn.s io]\\
                    /siˈrûn-s\textbf{i-o}/\\
                    hunt-\textsc{appl-mot-ep}\\
                    `S/he goes along hunting.'\\
                    `Va cazando.' < BFL 05 1:112/el >\\
                }
    \z
\z

Finally, labio-velar onset semi-vowels turn into labio-velar offglides after posttonic vowel deletion targets the nucleus of the labio-velar semi-vowel onset. This is shown in (\ref{ex:4:diphthongization of labio-velar semi-vowels1}).

%\pagebreak

\ea\label{ex:4:diphthongization of labio-velar semi-vowels1}
{Diphthongization of labio-velar semi-vowels}

    \ea[]{
    [ˈì.\textbf{w}i.li  → ˈì\textbf{w}.li]  \\
    /ˈì\textbf{w}i-li/\\
    bring.for-\textsc{pst}   \\
    `S/he brought it for them.'\\
    `Se lo trajo.' < BFL 06 5:75/el >\\
}
        \ex[]{
        [ku.ˈtʃ͡í\textbf{w}i.ma  → ku.ˈtʃ͡í\textbf{w}.ma] \\
        /kuˈtʃ͡í\textbf{wi}-ma/\\
        have.kids-\textsc{fut.sg}\\
        `S/he will have children.'\\
        `Va a tener hijos.' < BFL 06 6:74/el >\\
    }
            \ex[]{
            [wi.noˈmî.\textbf{w}i.pi   → wi.no.ˈmî\textbf{w}{}.pi]\\
            /winoˈmî\textbf{w}i-pi/\\
            have.money-\textsc{irr.pl} \\
            `Maybe they will have money.'\\
            `A lo mejor van a tener dinero.'\\
            < BFL 06 6:74/el >\\
        }
    \z
\z

As discussed in §\ref{subsec: semi-vowel monophthongization}, these labio-velar offglides can be weakened and monophthongized with its nucleus.

Finally, Choguita Rarámuri also has a series of heterosyllabic vowel sequences. Attested hiatus sequences across morpheme boundaries involve a stressed vowel followed by a low central vowel (\ref{ex:4:hiatus sequences with low, central vowels}) or by a mid back vowel, (\ref{ex:4:hiatus sequences with mid, back vowels}). These forms contrast with the forms shown in (\ref{ex:4:rising diphthongs}, where parallel vowel sequences in unstressed syllables are parsed as tautosyllabic.

\ea\label{ex:4:hiatus sequences with low, central vowels}
{Hiatus sequences with low central vowels }

    \ea[]{
    [ˈmé.a]\\
    /ˈm\textbf{é-a}/    \\
    bring-\textsc{prog}\\
    `S/he is bringing it.'\\
    `Lo está trayendo.' < SFH 04 1:73/el >\\
}
        \ex[]{
        [re.ˈʔè.a]\\
        /reˈʔ\textbf{è-a}] \\
        play-\textsc{prog}\\
        `S/he is playing.'\\
        `Está jugando.' < SFH 04 1:76-78/el >\\
    }
            \ex[]{
            [wi.pi.ˈsó.a]\\
            /wipiˈs\textbf{ó-a}/  \\
            hit.with.stick-\textsc{prog}\\
            `S/he is hitting with stick.'\\
            `Está apaleando.' < BFL 04 1:112/el >\\
        }
                \ex[]{
                [biʔ.ˈtò.a]\\
                /biʔˈt\textbf{ò-a}/ \\
                twist-\textsc{prog}\\
                `It is twisting.'\\
                `Se está torciendo.' < SFH 04 1:109/el >\\
            }
                    \ex[]{
                    [tʃ͡a.ˈpí.a]\\
                    /tʃ͡a\textbf{ˈpí-a}/ \\
                    grab\textsc{-prog}\\
                    `S/he is grabbing it.'\\
                    `Lo está agarrando.' < BFL 05 1:133/el >\\
                }
                        \ex[]{
                        [ti.ˈtʃ͡í.a]\\
                        /tiˈtʃ͡\textbf{í-a}/ \\
                        comb-\textsc{prog}\\
                        `S/he is combing.'\\
                        `Está peinando.' < ROF 04 1:116/el >\\
                    }
                            \ex[]{
                            [ri.ˈmù.a]\\
                            /riˈm\textbf{ù-a}/ \\
                            dream-\textsc{prog}\\
                            `S/he is dreaming.'\\
                            `Está soñando.' < ROF 04 1:107/el > \\
                        }
                                \ex[]{
                                [ˈʃû.a]\\
                                /ˈʃ\textbf{û-a}/ \\
                                sew-\textsc{prog}\\
                                `S/he is sewing.'\\
                                `Está cosiendo.' < ROF 04 1:81-82/el >\\
                            }
    \z
\z

\ea\label{ex:4:hiatus sequences with mid, back vowels}
{Hiatus sequences with mid, back vowels}

    \ea[]{
    [ni.ˈkâ.o]\\
    /niˈk\textbf{â-o}/ \\
    bark-\textsc{ep}\\
    `It barks.'\\
    `Ladra.' < BFL 05 1:114/el >\\
}
        \ex[]{
        [ˈpá.o]\\
        /ˈp\textbf{á-o}/  \\
        throw-\textsc{ep}\\
        `S/he throws it.'\\
        `Lo tira.' < BFL 04 VDB/el >\\
    }
            \ex[]{
            [ni.ˈkè.o]\\
            /niˈk\textbf{-è-o}/\\
            bark-\textsc{appl-ep}\\
            `It barks to them.'\\
            `Les ladra.' < BFL 07 VDB(53)/el >\\
        }
                \ex[]{
                [pi.ˈrê.o]\\
                /piˈr\textbf{ê-o}/   \\
                dwell.\textsc{pl-ep}\\
                `They dwell.'\\
                `Viven.' < BFL 05 1:161/el >\\
            }
                    \ex[]{
                    [na.ˈkí.o]\\
                    /naˈk\textbf{í-o}/ \\
                    want-\textsc{ep}\\
                    `S/he wants (it).'\\
                    `Quiere.' < BFL 04 1:91/el >\\
                }
                        \ex[]{
                        [bo.ti.ˈwí.o]\\
                        /botiˈw\textbf{í-o}/  \\
                        sink-\textsc{ep}\\
                        `It sinks.'\\
                        `Se hunde.' < SFH 05 1:120/el >\\
                    }
                            \ex[]{
                            [ʃi.ˈrû.o]\\    
                            /siˈr\textbf{û-o}/  \\
                            hunt-\textsc{ep}\\
                            `It hunts.'\\
                            `Caza.' < SFH 05 1:136/el >\\
                        }
                                \ex[]{
                                [na.ˈrú.o]\\
                                /naˈr\textbf{ú-o}/  \\
                                exist-\textsc{ep}\\
                                `It exists.'\\
                                `Existe.' < BFL 04 1:93/el >\\
                            }
    \z
\z

So far, the examples presented show verbal roots with final stress followed by the progressive suffix \textit{-a} (\ref{ex:4:hiatus sequences with low, central vowels}) or by the epistemic suffix \textit{-o} (\ref{ex:4:hiatus sequences with mid, back vowels}). There are no underlying monomorphemic hiatus sequences with verbal roots (although semi-vowel deletion yields vowel hiatus sequences, as discussed below in §\ref{subsec: semi-vowel deletion}). Root-internal vowel hiatus, however, is attested with nominal roots (\ref{ex:4:nominal root-internal hiatus}). As mentioned above, these root-internal hiatus sequences are only attested with stressed vowels followed by mid, central vowels. To date, no monomorphemic hiatus sequences with final mid back vowels have been recorded in the Choguita Rarámuri corpus.

\ea\label{ex:4:nominal root-internal hiatus}
{Nominal root-internal hiatus}

    \ea[]{
    [tʃ͡oˈ.k\textbf{é}.ri]  \\
    /tʃ͡oˈk\textbf{é}ri/\\
    `mountain dove’\\
    `paloma de monte’ < SFH NDB/ el >\\
}
        \ex[]{
        [ko.ˈtʃ͡\textbf{í.a}la]\\
        /koˈtʃ͡\textbf{ía}-la/ \\
        eyebrow-\textsc{poss}\\
        `their eyebrow'\\
        `su ceja' < MGD 06 1:107/el >\\
    }
            \ex[]{
            [ˈw\textbf{î.a}]   \\
            /ˈw\textbf{îa}/\\
            `rope’\\
            ‘mecate’ < JHF 04 1:17/ el >\\
        }
                \ex[]{
                [a.wa.ˈk\textbf{ó.a}.ni]\\
                /awaˈk\textbf{óa}ni/\\
                `scorpion’\\
                `alacrán’ < SFH NDB/el >\\
            }
    \z
\z

Hiatus sequences, thus, show an interesting asymmetry in the phonological behavior of words of different lexical categories: hiatus sequences are licensed with nominal roots, but are only attested across morpheme boundaries with verbal roots.\footnote{In Chapter \ref{chap: verbal morphology}, I discuss how vocalic (onsetless) suffixes may induce deletion of the preceding morpheme’s vowel in morphologically defined contexts.}

\section{Semi-vowels}
\label{sec: semi-vowels}
\subsection{Semi-vowel deletion}
\label{subsec: semi-vowel deletion}

Choguita Rarámuri has word-medial palatal and labio-velar semi-vowels. The example words in (\ref{ex:4:word-medial palatal semi-vowels}--\ref{ex:4:labio-velar semi-vowels}) show that semi-vowels occur independent of any particular vowel quality.

\largerpage
\ea\label{ex:4:word-medial palatal semi-vowels}
{Word-medial palatal semi-vowels}

    \ea[]{
    [kiˈ\textbf{j}ótʃ͡i]  \\
    /kiˈ\textbf{j}ótʃ͡i/\\
    ‘fox’\\
    ‘zorra’    < SFH 08 1:103/el >\\
}
        \ex[]{
        [koˈ\textbf{j}á]\\
        /koˈ\textbf{j}á/\\
        `squat’\\
        ‘estar en cuclillas’  < SFH 08 1:103/el >\\
    }
            \ex[]{
            [naˈ\textbf{j}ú]  \\
            /naˈ\textbf{j}ú/\\
            `to be sick’\\
            `estar enfermo’ \corpuslink{el1318[27_239-27_251].wav}{MFH el1318:27:23.9}\\
        }
                \ex[]{
                [hiˈ\textbf{j}é]   \\
                /hiˈ\textbf{j}é/\\
                `to follow a trace’\\
                `seguir la huella’, ``huellar" < SFH 05 1:98/el >\\
            }
    \z
\z

\ea\label{ex:4:labio-velar semi-vowels}
{Labio-velar semi-vowels}

    \ea[]{
    [kuˈ\textbf{w}é]   \\
    /kuˈ\textbf{w}é/\\
    `dry season’\\
    `tiempo de secas’  < NDB/el >\\
}
        \ex[]{
        [aˈrí\textbf{w}i]  \\
        /aˈrí\textbf{w}i/\\
        `to sun-set (dusk)' \\
        `atardecer’   < NDB/el >\\
    }
            \ex[]{
            [maˈ\textbf{w}étʃ͡i] \\
            /maˈ\textbf{w}étʃ͡i/\\
            `bean cultivating field’\\
            `campo de cultivo de frijol’\footnote{This refers to the burnt field where beans are sown.} < NDB/el >\\
        }
                \ex[]{
                [kaˈ\textbf{w}ì]  \\
                /kaˈ\textbf{w}ì/\\
                `hill, world’\\
                `cerro, mundo’ \corpuslink{tx43[11_112-11_182].wav}{SFH tx43:11:11.2}\\
            }
    \z
\z

Word-medial semi-vowels are optionally deleted.\footnote{As observed by an anonymous reviewer, a similar analysis is proposed in \citep{davidson2014hiatus} arguing against the claim that glides are inserted in hiatus environments in English.} The palatal semi-vowel, for instance, is optional when preceded by a low central vowel and followed by a stressed, front mid vowel. This is shown in (\ref{ex:4:optional deletion of palatal semi-vowels}).

%\break

\ea\label{ex:4:optional deletion of palatal semi-vowels}
Optional deletion of palatal semi-vowels

    \ea[]{
    [raˈ\textbf{j}èniri]   {\textasciitilde} [raˈèniri]  \\
    /raˈ\textbf{j}èniri/\\
    `sun’\\
    `sol’  \corpuslink{in485[07_048-07_084].wav}{SFH in485:7:04.8}, \corpuslink{in484[10_544-10_551].wav}{SFH in484:10:54.4}\\
}
        \ex[]{
        [maˈ\textbf{j}ê]   {\textasciitilde} [maˈê]   \\
        /maˈ\textbf{j}ê/\\
        `think’\\
        `pensar’ < BFL 07 el326/el >\\
    }
            \ex[]{
            [kaˈ\textbf{j}ènili]  {\textasciitilde} [kaˈènili]   \\
            /kaˈ\textbf{j}èni-li/\\
            harvest-\textsc{pst}\\
            `S/he harvested.'\\
            `Cosechó.’  < SFH 07 el327/el >\\
        }
                \ex[]{
                [paˈ\textbf{j}éri]   {\textasciitilde} [paˈéri]  \\
                /paˈ\textbf{j}éri/\\
                `to dance sutubúri’\\
                `bailar sutubúri’ < SFH 07 2:34/el >  \\
            }
    \z
\z

There are also examples of optional labio-velar semi-vowel deletion in the Choguita Rarámuri data. The examples in (\ref{ex:4:optional labio-velar semi-vowels}) involve a stressed mid, back vowel and a high, front vowel flanking the labio-velar semi-vowel. Optionally, these words are produced with a falling diphthong.

\ea\label{ex:4:optional labio-velar semi-vowels}
{Optional labio-velar semi-vowel deletion}

    \ea[]{
    [siˈnó\textbf{w}i]   {\textasciitilde} [siˈn\textbf{ói}] \\
    /siˈnó\textbf{w}i/\\
    `snake’ \\
    `víbora’ < SFH 04 1:17/el >\\
}
        \ex[]{
        [reˈrò\textbf{w}i]  {\textasciitilde} [reˈr\textbf{òi}]  \\
        /reˈrò\textbf{w}i/\\
        `potato’\\
        `papa’  < SFH 07 NDB200/el >\\
    }
    \z
\z

There are also cases where there is no palatal (\ref{ex:4:no semi-vowel deletion}a--c) or labio-velar (\ref{ex:4:no semi-vowel deletion}d--f) semi-vowel deletion word-medially. These cases all involve words where the semi-vowel is the onset of a stressed syllable. The unattested forms in the second column in (\ref{ex:4:no semi-vowel deletion}) show hypothetical forms with no word-medial semi-vowel.

\ea\label{ex:4:no semi-vowel deletion}
{No semi-vowel deletion}

    \ea[]{
    \glt    \doublebox{[iˈ\textbf{j}óni]}{\textit{*iˈóni}}\\
    \glt    /iˈ\textbf{j}óni/\\
    \glt    `to nag’\\
    \glt    `regañar’ < SFH 05 1:83/el >\\
}
        \ex[]{
         \glt   \doublebox{[niˈ\textbf{j}ú]}{\textit{*niˈú}}\\
         \glt   /niˈ\textbf{j}ú/\\
         \glt   `to escape’\\
         \glt   `escaparse’ < ROF 04 1:118/el >\\
    }
            \ex[]{
             \glt   \doublebox{[koˈ\textbf{j}êra]}{\textit{*koˈêra}}\\
             \glt   /koˈ\textbf{j}êra/\\
             \glt   `headband’\\
             \glt   `koyéra’ < LEL 06 5:127/el >\\
        }
                \ex[]{
                \glt   \doublebox{[neˈ\textbf{w}à]}{\textit{*neˈà}}\\
                \glt    /neˈ\textbf{w}à/\\
                \glt   `to make’\\
                \glt   `hacer’ < SFH 04 1:67/el >\\
            }
                    \ex[]{
                     \glt   \doublebox{ruruˈ\textbf{w}á}{\textit{*ruruˈá}}\\
                     \glt   /ruruˈ\textbf{w}á/\\
                     \glt   `to be cold’\\
                     \glt   `tener frío’ \corpuslink{tx130[03_436-03_468].wav}{LEL tx130:3:43.6}\\
                }
                        \ex[]{
                         \glt   \doublebox{[naˈ\textbf{w}à]}{\textit{*naˈà}}\\
                         \glt   /naˈ\textbf{w}à//\\
                         \glt   `to arrive’\\
                         \glt   `llegar’ < SFH 05 1:73/el >\\
                    }
    \z
\z

While semi-vowels delete productively, there is no evidence of productive semi-vowel epenthesis in Choguita Rarámuri. The only forms with apparent epenthetic semi-vowels are shown in (\ref{ex:4:lexicalized cases of semi-vowel epenthesis}). In these examples, a monosyllabic root adds the applicative suffix \textit{-è}. As shown here, either a palatal or a labio-velar glide may be epenthesized between the stem and the applicative suffix in a pattern of intra-speaker variation (speaker LEL produced both (\ref{ex:4:lexicalized cases of semi-vowel epenthesisa}) and (\ref{ex:4:lexicalized cases of semi-vowel epenthesisb}), and speaker ROF produced both (\ref{ex:4:lexicalized cases of semi-vowel epenthesisc}) and (\ref{ex:4:lexicalized cases of semi-vowel epenthesisd}).

\ea\label{ex:4:lexicalized cases of semi-vowel epenthesis}
{Lexicalized cases of semi-vowel epenthesis}

    \ea[]{
     [ruˈ\textbf{j}è]  \\
     /ru-ˈè/\\
     tell-\textsc{appl}\\
     `to tell someone' \\
     `contarle' \corpuslink{tx32[08_429-08_482].wav}{LEL tx32:8:42.9}\\
}\label{ex:4:lexicalized cases of semi-vowel epenthesisa}
        \ex[]{
        [ruˈ\textbf{w}è]   \\
        /ru-ˈè/    \\
         tell-\textsc{appl}\\
        `to tell someone' \\
        `contarle' \corpuslink{tx32[05_139-05_164].wav}{LEL tx32:5:13.9}\\
    }\label{ex:4:lexicalized cases of semi-vowel epenthesisb}
            \ex[]{
            [buˈ\textbf{j}è] \\
            /bu-ˈè/\\
            wait-\textsc{appl}\\
            `to wait for someone' \\
            `esperarlo' < ROF 04 1:67/el >\\
        }\label{ex:4:lexicalized cases of semi-vowel epenthesisc}
                \ex[]{
                [buˈ\textbf{w}è]  \\
                /bu-ˈè/ \\
                 wait-\textsc{appl}\\
                `to wait for someone' \\
                `esperarlo' < ROF 04 1:67/el >\\
            }\label{ex:4:lexicalized cases of semi-vowel epenthesisd}
    \z
\z

Other instances of  applicative \textit{-è} suffixation do not involve optional palatal and labio-velar semi-vowel epenthesis. This is exemplified in (\ref{ex:4:no glide epenthesis}).

\ea\label{ex:4:no glide epenthesis}
{No semi-vowel epenthesis}

    [ruˈèa]\\
    /ru-ˈè-a/\\
    tell-\textsc{appl-prog}\\
    `They are telling them.'\\
    `Les están diciendo.' \corpuslink{tx43[04_430-04_471].wav}{SFH tx43:04:43.0}\\

\z

There is thus no evidence for a productive, general semi-vowel epenthesis process in the language.

The next section lays out details of several general phonological processes targeting both consonantal and vocalic segments.

\subsection{Semi-vowel monophthongization}
\label{subsec: semi-vowel monophthongization}

Labio-velar onset semi-vowels turn into labio-velar offglides after posttonic vowel deletion. Some examples are provided in (\ref{ex:4:diphthongization of labio-velar semi-vowels2}).

\ea\label{ex:4:diphthongization of labio-velar semi-vowels2}
{Diphthongization of labio-velar semi-vowels}

    \ea[]{
    [ˈì.\textbf{w}i.li]  →  [ˈì\textbf{w}.li]  \\
    /ˈì\textbf{w}i-li/\\
    bring.\textsc{appl-pst}\\
    `S/he brought it for them.' \\
    `Se lo trajo.'   < BFL 06 5:75/el >\\
}
        \ex[]{
        [ku.ˈtʃ͡î.\textbf{w}i.ma] → [ku.ˈtʃ͡î\textbf{w}.ma] \\
        /kuˈtʃ͡î\textbf{w}i-ma/\\
        have.children-\textsc{fut.sg}\\
        `S/he will have children'\\
        `Va a tener hijos' < BFL 06 6:74/el >\\
    }
            \ex[]{
            [wi.noˈmî.\textbf{w}i.pi]  → [wi.no.ˈmî\textbf{w}.pi]\\
            /winoˈmî-\textbf{w}i-pi/\\
            money-\textsc{have-irr.pl}\\
            `Maybe they will have money' \\
            `A lo mejor van a tener dinero'  < BFL 06 6:74/el >\\
        }
    \z
\z

There is a gradient semi-vowel weakening process: after posttonic vowel deletion, labio-velar semi-vowels can range from a fully diphthongal velar rhyme to a completely monphthongized variant. This gradient process is schematized in \REF{exfig: semi-vowel weakening and monophthongization}.


\newpage
\ea\label{exfig: semi-vowel weakening and monophthongization}
Semi-vowel weakening and monophthongization\\
    \ea \textsubscript{σ}[w  $\to$ Vw]\textsubscript{σ}
    \ex Vw]\textsubscript{σ} $\to$ V\textsuperscript{w}]\textsubscript{σ}
    \ex V\textsuperscript{w}]\textsubscript{σ} $\to$ V\textsubscript{1}V\textsubscript{1}
    \z
\z

% \begin{figure}
%
% \includegraphics[width=5cm]{figures/Syllables-img1.png}
% \caption{
% \label{fig: semi-vowel weakening and monophthongization}
% Semi-vowel weakening and monophthongization}
% \end{figure}

What this scheme illustrates is a process whereby a labio-velar semi-vowel in onset position is re-syllabified as a coda after post-tonic deletion (a); this coda labio-velar semi-vowel may optionally be weakened to a short off-glide of the nucleus vowel (b); the short off-glide may be further weakened by undergoing monophthongization with the nucleus vowel host, yielding a long vowel sequence (c).

This weakening process is exemplified in (\ref{ex:4:gradient weakening and monophthongization of /w/}). The segments in question are in bold face. The labio-velar semi-vowel in (\ref{ex:4:gradient weakening and monophthongization of /w/b}) is an underlying voiced, bilabial stop (realized as a voiced bilabial approximant in (\ref{ex:4:gradient weakening and monophthongization of /w/a})). (\ref{ex:4:gradient weakening and monophthongization of /w/d}) and (\ref{ex:4:gradient weakening and monophthongization of /w/f}) have underlying labio-velar semi-vowels.

\ea\label{ex:4:gradient weakening and monophthongization of /w/}
{Gradient weakening and monophthongization of /w/}

    \ea[]{
    [ruruˈtʃ͡í\textbf{β̞}ia] \\
    /ruruˈtʃ͡í\textbf{w}i-a/\\
    sprinkle\textsc{-prs} \\
    `It's sprinkling.' \\
    `Está rociando.’ < SFH 07 el170/el >\\
}\label{ex:4:gradient weakening and monophthongization of /w/a}
        \ex[]{
        [ruruˈtʃ͡í\textbf{w}ma]   \\
        /ruruˈtʃ͡í\textbf{w}i-ma/\\
        sprinkle-\textsc{fut.sg}  \\
        `S/he will sprinkle.' \\
        `Va a rociar.’  < SFH 07 el170/el >\\
    }\label{ex:4:gradient weakening and monophthongization of /w/b}
            \ex[]{
            [ruruˈtʃ͡\textbf{íi}ma] \\
            /ruruˈtʃ͡í\textbf{wi}-ma/\\
            sprinkle-\textsc{fut.sg}\\
            `S/he will sprinkle.'\\
            `Va a rociar.’ < SFH 07 el170/el >\\
        }\label{ex:4:gradient weakening and monophthongization of /w/c}
                \ex[]{
                [tiˈlú\textbf{w}i]  \\
                /tiˈlú\textbf{w}i/\\
                `Gargle!’ \\
                `¡Haz gárgaras!’   < BFL 07 1:43/el >\\
            }\label{ex:4:gradient weakening and monophthongization of /w/d}
                    \ex[]{
                    [tiˈl\textbf{úu}{}ma]  \\
                    /tiˈlú\textbf{w}i-ma/\\
                    gargle-\textsc{fut.sg} \\
                    `S/he will gargle.' \\
                    `Va a hacer gárgaras.' < BFL 07 1:43/el >\\
                }\label{ex:4:gradient weakening and monophthongization of /w/e}
                        \ex[]{
                        [biniˈhî-\textbf{w}ma]  \\
                        /biniˈhî-\textbf{w}i-ma/\\
                        accuse-\textsc{appl-fut.sg} \\
                        `S/he will accuse them.' \\
                        `Lo va a acusar.' < BFL 07 2:48/el >\\
                    }\label{ex:4:gradient weakening and monophthongization of /w/f}
                            \ex[]{
                            [biniˈh\textbf{îi}ma]  \\
                            /biniˈhî-\textbf{w}i-ma/\\
                            accuse-\textsc{appl-fut.sg} \\
                            `S/he will accuse them.' \\
                            `Lo va a acusar.' < BFL 07 2:48/el >\\
                        }\label{ex:4:gradient weakening and monophthongization of /w/g}
                                \ex[]{
                                [naˈpí\textbf{w}iri]  \\
                                /naˈpí\textbf{w}iri/\\
                                ‘nixtamal’    < 08 1:34/Conv, el >\\
                            }\label{ex:4:gradient weakening and monophthongization of /w/h}
                                    \ex[]{
                                    [naˈpí\textbf{w}ipo]  \\
                                    /naˈpí\textbf{w}i-po/\\
                                    make.nixtamal-\textsc{fut.pl}\\
                                    `They will make nixtamal.' \\
                                    'Van a hacer nixtamal.’   < 08 1:34/Conv, el >\\
                                }\label{ex:4:gradient weakening and monophthongization of /w/i}
                                        \ex[]{
                                        [naˈp\textbf{íi}po]  \\
                                        /naˈpí\textbf{w}i-po/\\
                                        make.nixtamal-\textsc{fut.pl} \\
                                        `They will make nixtamal.' \\
                                        `Van a hacer nixtamal.'     < LEL 08 1:34/el >\\
                                    }\label{ex:4:gradient weakening and monophthongization of /w/j}
    \z
\z

This process is not subject to speaker variation, and the choice between the labio-velar offglide and monophthongization seems to be correlated with rate of speech and care of pronunciation.

% \textbf{Note: different reduction patterns for word-initial labio-velar glides/bilabial stops (which may also be found alternating with zero) – this is related to the question of voicing alternations (status of /b/ underlyingly and how voicing is an innovation in Rarámuri varieties within the Tara-\ili{Guarijío} Taxon).}
