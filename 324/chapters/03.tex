\chapter{Segmental phonology}
\label{chap: phonology}

This chapter is devoted to the segmental phonological inventory and phonological processes of Choguita Rarámuri. While many phonological alternations in the language are morphologically conditioned (addressed in \chapref{chap: nominal morphology}, \chapref{chap: verbal morphology} and \chapref{chap: prosody}), this chapter addresses those segmental processes that are analyzed as being fully phonologically general. This chapter also addresses patterns of phonetic variability of consonant phonemes.

The chapter is laid out as follows. §\ref{sec: overview sound} provides an overview of the Choguita Rarámuri sound system assumed in this and following chapters. §\ref{sec: phonological inventory} introduces the phonemic segmental inventory of the language. §\ref{sec: minimal pairs} illustrates the phonemic status of segments with minimal pairs. §\ref{sec: general phonological processes} describes the general phonological processes that yield the allophonic variation displayed by obstruents, nasals, rhotics, stops and vowels. The chapter concludes in §\ref{sec: phonetic reduction} with a description of patterns of phonetic reduction of consonants.

\section{Overview of the Choguita Rarámuri phonological system}
\label{sec: overview sound}

The phonological system of Choguita Rarámuri is characterized by a relatively small phoneme inventory, with eighteen consonants, five contrastive vowels and no contrastive vowel length, and a three-way laryngeal contrast for stops between pre-laryngealized stops (canonically realized as pre-aspirated voiceless stops), plain voiceless stops and voiced stops. This language is also characterized by a simple syllable structure in underlying representations, with no elaborate onsets and only glottal stop as a possible coda. While displaying low elaboration of the consonant inventory and a low level of complexity of syllabic structure (a correlation expected according to \citealt{maddieson2005issues}), Choguita Rarámuri displays a high degree of allophonic variation, as well as a complex word-prosodic system. Choguita Rarámuri syllable structure and syllabic processes are addressed in \chapref{chap: syllables}, while various aspects of its word prosody are addressed in \chapref{chap: word prosody} (stress), \chapref{chap: tone and intonation} (tone and intonation), and \chapref{chap: other word-level suprasegmental phonology} (other word-level suprasegmental phonological processes). Interactions between lexical, morphological and phrase-level phonological phenomena are addressed in Chapter ~\ref{chap: prosody}, dedicated to prosodic interactions.

General phonological processes target almost all segments and involve different reduction processes. Alternations between plain (non-laryngealized) stops and voiced stops involve historically related pairs of segments (p {\textasciitilde} b, k {\textasciitilde} g, t {\textasciitilde} r), some of which have a phonemic status synchronically, but which may also display phonologically, morphologically or lexically conditioned allophonic variation. Many patterns of variation in surface forms are phonetic in nature and display different degrees of inter- and intra-speaker variation, are affected by speech rate and style, are always surface-transparanet and do not interact with phonological rules. Some widespread patterns of variation and optionality are addressed in the analyses presented in this and subsequent chapters.

In terms of its word prosody, Choguita Rarámuri has both stress and a three-way lexical tonal contrast. The complexity of the stress system in this language stems from several factors. First, stress conditions optional vowel reduction and deletion processes. While these processes are complex and, to some extent, gradient, there are at least three different identifiable patterns of vowel reduction targeting different vowel qualities, with more reduction occurring post-tonically than pre-tonically. Syncope yields derived heterosyllabic consonant clusters in coda position word-medially. Thus, surface forms display a moderate level of complexity of syllabic structure. Second, stress is restricted to an initial three-syllable window, a highly marked typological pattern (\citealt{kager2012stress}). Finally, the stress system of this language also features morphological factors that govern stress placement involving stress shift patterns depending on the lexical stress properties of roots and suffixes. The phonological and morphological properties of stress and associated morpho-phonological properties are addressed in \chapref{chap: word prosody}.

Tone is exclusively realized in stressed syllables. While tone distribution is dependent on stress, both stress and tone are independent phonologically and encoded through different acoustic means \parencite{caballero2015tone}. \ili{Proto-Uto-Aztecan} is reconstructed as having a stress-accent system \parencite{munro1977towards}, with different languages belonging to different branches developing tonal contrasts in addition to stress (e.g., \ili{Hopi} \parencite{manaster1986genesis}, \ili{Northern Tepehuan} (\ili{Tepiman}; \citealt{woo1970tone}), \ili{Balsas Nahuatl} (\ili{Aztecan}; \citealt{guion2010word}), \ili{Cora} (\ili{Corachol}; \citealt{McMahon-1967}), \ili{Huichol} (\ili{Corachol}; \citealt{grimes1959huichol}), \ili{Yaqui} (\ili{Cahitan}; \citealt{demers1999prominence}) and \ili{Mayo} (\ili{Cahitan}; \citealt{hagberg1989floating})). Since the first reports of lexical tone in Choguita Rarámuri (in \cite{caballero2008choguita} and \cite{caballero2015tone}), \citet{moralesmoreno2016rochecahi} also reports the existence of lexical tone in \ili{Rochéachi Rarámuri}, another Central Rarámuri variety, and it is possible that other Rarámuri varieties are also tonal, though no other variety of Rarámuri has been described as being tonal. Choguita Rarámuri also deploys f0 intonationally, resulting in different accommodation strategies when the lexical phonology, the morphology and the intonational phonology assign conflicting tones to the same target tone bearing units. In addition to f0, duration and non-modal phonation encode intonational meaning in the language. Tone and intonation are described in \chapref{chap: tone and intonation}.

In addition to featuring a three-syllable stress window, the word-level prosody of Choguita Rarámuri features a disyllabic window for the distribution of the  glottal stop. Distributional restrictions on the glottal stop and a three-syllable stress window have also been documented in the closely related \ili{Tara-Guarijío} language \ili{Guarijío} (\citealt{miller1996guarijio}, \citealt{haugen2014laryngeals}, \citealt{haugen2014laryngeals}), and window-like restrictions on stress placement are described across Rarámuri varieties (including \citeauthor{brambila1953gramatica}'s (1953) description of \ili{Norogachi Rarámuri} and \citeauthor{moralesmoreno2016rochecahi}'s (2016) description of \ili{Rochéachi Rarámuri}). In addition to the prosodically restricted distribution of glottal stop, other word-level suprasegmental phonological phenomena in this language include minimal word size restrictions and prosodic loanword adaptation processes. All of these are discussed in \chapref{chap: other word-level suprasegmental phonology}.


%% Notes on cross-references to chapters on verbal/nominal morphology: (Feel free to delete when no longer needed)
%% - I created new files for these chapters (04.tex and 05.tex)
%% - The labels are defined in those files (e.g. "chap: verbal morphology")
%% - The ~ is interpreted by latex just like a space between words, but it prevents line breaks between "Chapter" and the number
%% - Note that the numbers that latex inserts depend on which chapters are currently "enabled," i.e. not commented out in main.tex
%% - When you comment out the files in main.tex that include the label you're referencing, latex inserts "??" instead of the number; nothing's broken, it just means that latex can't find the label you're referring to in any of the active files.
%% - Unfortunately, overleaf does not support autocomplete to labels outside of the current file. You'll have to remember/look up the label of an element you want to cross-reference if it's in a different file (such as 04.tex/05.tex in this case)...


\section{Phonological inventory}
\label{sec: phonological inventory}
\largerpage
\subsection{Consonants}
\label{subsec: consonants}

Choguita Rarámuri has a relatively small consonant inventory and a high degree of allophonic variation. The phonemic consonant inventory, presented in \tabref{tab:consonants-phonology}, is significantly similar to proposed reconstructions for the \ili{Proto-Uto-Aztecan} (PUA) consonant system (\citealt{voegelin17voegelin}, \citealt{miller1967uto}, \citealt{langacker1977uto}). In contrast to the reconstructed PUA consonant system, Choguita Rarámuri features a three-way laryngeal contrast in stops, with pre-laryngealized, unaspirated and voiced stops. Voicing is only contrastive at the bilabial and alveolar place of articulation (where, as discussed below, the voiceless alveolar plosive alternates with a voiced alveolar flap). \citet{miller1996guarijio} also documents a contrast between pre-laryngealized and plain voiceless stops in closely related \ili{Mountain Guarijío}, though \ili{Mountain Guarijío} differs from Choguita Rarámuri in lacking contrastive voicing of stops. In this chart, the voiced labiovelar approximant is represented in the bilabial column; the alveolar and retroflex flaps are represented orthographically as <r> and <l>, respectively. Pre-laryngealized obstruents are represented with pre-aspiration in this chart.


\begin{table}
\caption{{Phonemic inventory of Choguita Raramuri consonants}}
\label{tab:consonants-phonology}
\fittable{
\begin{tabular}{lccccccc}
\lsptoprule
& {Bilabial} & {Alveolar} & {Alveo-palatal} & {Retroflex} & {Palatal} & {Velar} & {Glottal}\\
\midrule
{{Plosive}} & {p} \textup{ʰ}\textup{p} \textup{b} & {t} \textup{ʰ}\textup{t} &  &  &  & {k} \textup{ʰ}\textup{k}  & {ʔ}\\
{{Affricate}} &  &  & {tʃ͡} \textup{ʰ}\textup{tʃ͡} &  &  &  & \\
{{Nasal}} & {m} & {n} &  &  &  &  & \\
{{Flap}} &  & {<r>} [\textup{ɾ]}  &  & {<l> [ɽ]}  &  &  & \\
{{Fricative}} &  & {s} &  &  &  &  & {h}\\
{{Approx.}} & {w} &  &  &  & {j} & {w} & \\
\lspbottomrule
\end{tabular}
}
\end{table}
%\hspace{3cm}

A broad phonemic transcription is used, and the symbols used in this table will be used throughout this grammar to represent these segments. The labiovelar approximant is doubly assigned in the bilabial place of articulation column and in the velar place of articulation column. This notation indicates that the approximant is both velar and bilabial.

\largerpage
The consonant inventory presented in \tabref{tab:consonants-phonology} includes a retroflex flap that can be characterized as a `lateral flap', as defined by \citep[][234]{ladefoged1996sounds}. This segment is articulated by making a ballistic contact with the tongue tip in the post-alveolar region, but the sides of the tongue allow air to flow laterally, resulting in a sound that auditorily resembles both a lateral approximant and an alveolar (or, in the case of Choguita Rarámuri, slightly retroflexed) flap. The environments that favor the production/perception of the lateral variant are discussed in §\ref{subsec: processes targeting rhotics}. This has led to orthographic representations of this sound as either <r> or <l>, including the name of the language, which is alternatively spelled `Rarámuri', with word-medial <r>, or `Ralámuli', with word-medial <l>. The contrast between the two rhotic sounds is ilustrated in (\ref{ex: r vs. l}) in §\ref{sec: minimal pairs} below.

As mentioned above, Choguita Rarámuri features a three-way laryngeal contrast of consonants, with pre-laryngealized, unaspirated and voiced consonants. Pre-laryngealized consonants are only attested as onsets of stressed syllables. ‘Plain’ obstruents are realized with no pre-laryngealization, are not restricted in terms of their distribution and may be realized with post-aspiration as onsets of stressed syllables as part of a process of onset augmentation (see discussion of acoustic correlates of stress in §\ref{subsec: stress and stress-dependent phenomena} and \citealt{caballero2015tone}). Pre-laryngealized consonants (canonically realized as pre-aspirated stops [ʰp, ʰt, ʰk, ʰtʃ͡]) are analyzed here as specified with a [+spread glottis] feature and `plain' obstruents as laryngeally unspecified ([p, t, k, tʃ͡]).\footnote{This analysis is similar to the one provided in \citet{dicanio2012phonetics} for the lenis-fortis contrasts in San Martín \ili{Ituñoso Trique}, where `fortis' obstruents are realized with an early adduction (spreading) glottalic gesture, which results in a short period of pre-aspiration before the consonant (\citeyear[257]{dicanio2012phonetics}). See also \citet{arellanes2009sistema} for a similar analysis of the lenis-fortis contrast in \ili{San Pablo Güilá Zapotec}.} Given that for some speakers pre-aspirated stops are variably realized as pre-aspirated or pre-glottalized, the term pre-laryngealized is used throughout this grammar when referring to these segments.

For some speakers pre-laryngealized consonants can be realized without any laryngeal features in stressed position, neutralizing in the surface with ‘plain’ consonants, an instance of inter-speaker variation in the realization of these segments. Some examples are shown in (\ref{ex: laryngeal neutralization}).\footnote{In (\ref{ex: laryngeal neutralizationc}), there is post-aspiration as onset augmentation when pre-laryngealization is neutralized in a stressed syllable.}

\ea\label{ex: laryngeal neutralization}
{Variable neutralization between pre-laryngealized and `plain' stops}

    \ea[]{
    [reˈ\textsuperscript{\textbf{h}}\textbf{t}ê] {\textasciitilde}  [reˈ\textbf{t}ê] \\
    /reˈʰtê/\\
    ‘stone’\\
    `piedra' < GFM 14 1:111 >\\
}\label{ex: laryngeal neutralizationa}
        \ex[]{
        [ri\textbf{ˈʰtʃ}í] {\textasciitilde} [ri\textbf{ˈtʃ}í] \\
        /riˈʰt͡ʃí/\\
        `landslide'\\
        ‘reliz' < BFL CR2014-LSC:13 >, < GFM 14 1:111 >, < SFH CR2014-LSC:86 >\\
        }\label{ex: laryngeal neutralizationb}
                \ex[]{
                [aˈʰkâ] {\textasciitilde} [aˈkâ] \\
                /aˈʰkâ/\\
                `to be sweet or salty’\\
                `ser dulce o salado' < SFH CR2014-LSC:85 >\\
            }\label{ex: laryngeal neutralizationc}
        \z
\z

These patterns of variation in the articulation of pre-laryngealized stops may suggest that the pre-laryngealized contrast may be marginal for some speakers, though most speakers realize the contrast consistently.\footnote{A caveat is that even if pre-aspiration seems to disappear in the speech of some speakers it may leave residual breathy voice on preceding vowels (Marc Garellek p.c.). Whether this is the case in Choguita Rarámuri is a question that remains for future research.}

There are no voiced coronal or velar stops in the phonological inventory of Choguita Rarámuri. Allophonic alternations show that the voiced counterpart of /t/ is a coronal flap (/ɾ/) (there is no voiced coronal stop /d/ in the Choguita Rarámuri phonological inventory).

\subsection{Vowels}
\label{sec: vowels}
Choguita Rarámuri makes a phonemic distinction among five cardinal vowels in stressed position. There is no contrastive vowel length. Mid vowels are phonetically open-mid and the only back vowels are rounded. The vocalic inventory is given in \tabref{tab:key:4}.

% %%please move \begin{table} just above \begin{tabular .
\begin{table}
\caption{{Choguita Rarámuri Monophthong Vowel System}}
\label{tab:key:4}

\begin{tabularx}{.8\textwidth}{XlCr}
\lsptoprule
& Front & Central & Back\\
\midrule
 High & i &  & u\\
 Mid & e [ɛ] &  & o [ɔ] \\
 Low &  & a & \\
\lspbottomrule
\end{tabularx}
\end{table}
%\hspace{3cm}

Most other Rarámuri varieties are reported as having five cardinal vowels, the qualities of which mostly match the inventory given for Choguita Rarámuri (\citealt{brambila1953gramatica}, \citealt{lionnet1972elementos}, \citealt{moralesmoreno2016rochecahi}, inter alia), except for \ili{Ojachichi Rarámuri}, which is described as having a high, back, \textit{unrounded} [ɯ] vowel \parencite{servin2002ralamuli}.


\section{Minimal pairs}
\label{sec: minimal pairs}

\subsection{Consonant minimal pairs}
\label{subsec: minimal pairs consonants}

This section presents minimal pairs that demonstrate the phonemic status of Choguita Rarámuri consonants. Allophonic variation and patterns of neutralization are addressed in §\ref{sec: general phonological processes}.

The examples in (\ref{ex: pre-aspirated obstruent vs. plain onstruent contrast}) include (near-)minimal pairs that illustrate the contrast between pre-laryngealized and plain (unaspirated) voiceless stops.

\ea\label{ex: pre-aspirated obstruent vs. plain onstruent contrast}
{Choguita Rarámuri pre-laryngealized obstruent vs. plain obstruent contrast}

    \ea[]{
    [wi\textbf{ˈʰk}â]  \\
    /wi\textbf{ˈʰk}â/\\
    ‘many’\\
    `muchos'     \corpuslink{tx191[04_215-04_247].wav}{BFL tx191:04:21.5} \\
}\label{ex: pre-aspirated obstruent vs. plain onstruent contrasta}
        \ex[]{
        [wiˈ\textbf{k}á]\\
        /wiˈ\textbf{k}á/\\
        ‘constellation, plow’\\
        `constelación, barbecho' \corpuslink{in484[12_175-12_210].wav}{ME in484:12:17.5} \\
    }\label{ex: pre-aspirated obstruent vs. plain onstruent contrastb}
            \ex[]{
            [ri\textbf{ˈʰt}ûli]\\
            /ri\textbf{ˈʰt}û-li/ \\
            freeze-\textsc{pst} \\
            `It froze.'\\
            `Se congeló.'     \corpuslink{el1028[00_438-00_454].wav}{SFH el1028:00:43.8}\\
        }\label{ex: pre-aspirated obstruent vs. plain onstruent contrastc}
                \ex[]{
                [riˈ\textbf{t}ù] \\
                /riˈ\textbf{t}ù/\\
                ‘encourage’\\
                `animar'      \corpuslink{tx19[01_442-01_491].wav}{LEL tx19:01:44.2} \\
            }\label{ex: pre-aspirated obstruent vs. plain onstruent contrastd}
                    \ex[]{
                   [na\textbf{ˈʰp}ô] \\
                   /na\textbf{ˈʰp}ô/\\
                    ‘prickly pear’\\
                    `tuna'       \corpuslink{el728[07_094-07_103].wav}{BFL el728:07:09.4}\\
                }\label{ex: pre-aspirated obstruent vs. plain onstruent contraste}
                        \ex[]{
                        [naˈ\textbf{p}ò] \\
                        /naˈ\textbf{p}ò/\\
                        ‘to break, intr.’\\
                        `romperse'     \corpuslink{el1242[10_137-10_148].wav}{MAF el1242:10:13.7}\\
                    }\label{ex: pre-aspirated obstruent vs. plain onstruent contrastf}
                            \ex[]{
                            [ri\textbf{ˈʰtʃ}í] \\
                            /ri\textbf{ˈʰtʃ}í/\\
                            `landslide'\\
                            `reliz’   < LEL 14 1:9 >  \\
                        }\label{ex: pre-aspirated obstruent vs. plain onstruent contrastg}
                                \ex[]{
                               [riʔˈ\textbf{tʃ}ì] \\
                                /riʔˈ\textbf{tʃ}ì/\\
                                ‘paternal uncle younger than father’\\
                                `tío paterno menor que el padre'   < LEL 14 1:9 >\\
                            }\label{ex: pre-aspirated obstruent vs. plain onstruent contrasth}
                                    \ex[]{
                                    [o\textbf{ˈʰk}ó] \\
                                    /o\textbf{ˈʰk}ó/\\
                                    ‘pine tree’\\
                                    `pino'   \corpuslink{tx84[05_002-05_053].wav}{LEL tx84:05:00.2} \\
                                }\label{ex: pre-aspirated obstruent vs. plain onstruent contrasti}
                                        \ex[]{
                                       [ro\textbf{ˈk}ò]\\
                                       /ro\textbf{ˈk}ò/\\
                                        ‘night’\\
                                        `noche'  \corpuslink{tx68[00_313-00_380].wav}{LEL tx68:00:31.3}\\
                                    }\label{ex: pre-aspirated obstruent vs. plain onstruent contrastj}
                                    \newpage
                                            \ex[]{
                                            [raˈʰtári]\\
                                            /raˈ\textbf{ʰt}á-ri/\\
                                            be.hot-\textsc{nmlz}\\
                                            ‘heat, fever’\footnote{This root has a L tone for some speakers. The extent to which inter-speaker tonal variation is attested in the Choguita Rarámuri corpus and factors that may govern this variability are questions left for future research.}\\
                                            `calor, calentura'      \corpuslink{tx785[02_564-02_582].wav}{GFM tx785:02:56.4} \\
                                        }\label{ex: pre-aspirated obstruent vs. plain onstruent contrastk}
                                                \ex[]{
                                                [raʔˈ\textbf{t}á] \\
                        /raʔˈ\textbf{t}á/\\
                        ‘to pop, blow’\\
                                                `explotar'                      \corpuslink{co1137[05_345-05_365].wav}{MDH co1137:05:34.5}\\
                                            }\label{ex: pre-aspirated obstruent vs. plain onstruent contrastl}
                                                    \ex[]{
                                                     [ba\textbf{ˈʰtʃ}í]\\
                                                     /ba\textbf{ˈʰtʃ}í/\\
                                                     ‘zucchini’\\
                                                     `calabacita' < tx130[00\_525-01\_011] >\\
                                                }\label{ex: pre-aspirated obstruent vs. plain onstruent contrastm}
                                                        \ex[]{
                                                        [baʔtʃ͡iˈlâ]\\
                                                        /baʔ\textbf{tʃ}i-ˈlâ/\\
                                                        older.brother-\textsc{poss}\\
                                                        ‘his older brother’\\
                                                        `su hermano mayor'    \corpuslink{co1136[18_361-18_386].wav}{MDH co1136:18:36.1}\\
                                                    }\label{ex: pre-aspirated obstruent vs. plain onstruent contrastn}
        \z
\z

The minimal pairs in (\ref{ex: pre-aspirated obstruent vs. plain onstruent contrast}a--f) show that pre-aspirated obstruents contrast with plain stops. This contrast is illustrated in Figure~\ref{fig: preaspirated stop illustration} (with pre-aspirated [ʰk]) and Figure~\ref{fig: plain stop illustration} (with the plain (unaspirated) voiceless counterpart [k]).

\begin{figure}
\includegraphics[width=\textwidth]{figures/Segmental_Phonology_img2.png}
\caption{
\label{fig: preaspirated stop illustration}
Spectrogram showing realization of pre-aspirated [ʰk] in the word [o\textbf{ˈʰk}ól] /oˈʰkó-li/ ‘(in the) pine tree’ \corpuslink{tx84[05_002-05_053].wav}{LEL tx84:05:00.2}.}
\end{figure}

\begin{figure}
\includegraphics[width=\textwidth]{figures/Segmental_Phonology_img3.png}
\caption{
\label{fig: plain stop illustration}
Spectrogram showing realization of plain [k] in the word [ro\textbf{ˈkò}] ‘night’
\corpuslink{el1028[00_438-00_454].wav}{SFH el1028:00:43.8}.}
\end{figure}

%reconsider
\largerpage[2]
As shown in (\ref{ex: pre-aspirated obstruent vs. plain onstruent contrast}h, l, n), plain stops may be preceded by a glottal stop. These glottal stop-consonant sequences are analyzed here as (heterosyllabic) consonant clusters ([ʔC]). An alternative (suggested by an anonymous reviewer) is to treat these sequences as pre-glottalized consonants ([\textsuperscript{ʔ}C]) that are only licensed in stressed onsets, which would parallel the distribution of pre-aspirated stops in the language (a restriction widely documented in languages with pre-laryngelized consonants \citep{clayton2010natural}). An argument against this analysis is that there is evidence that glottal gestures before consonants are not restricted to a stressed onset, e.g. \textit{ba\textbf{ʔ}tʃ͡i-ˈlâ}  ‘his older brother’ (in (\ref{ex: pre-aspirated obstruent vs. plain onstruent contrastn})) and \textit{a\textbf{ʔ}wiˈjó} `germinated' \corpuslink{tx68[00_512-00_558].wav}{LEL tx68:00:51.2}. While the glottal stop may be reduced in unstressed position (e.g., in the context of stress shift, where the glottal may be produced as glottalization of an adjacent vowel or sonorant), the glottal gesture is consistently realized regardless of stress conditioning, which contrasts with the realization of pre-aspirated consonants. Thus, [ʔC] consonant sequences largely pattern like other (heterosyllabic) consonant clusters in the language (for discussion of (derived) consonant clusters in the language, see §\ref{subsec: consonant sequences}). The only mismatch between [ʔC] consonant sequences and other consonant clusters result from the restriction that applies to the glottal stop, regardless of whether the glottal stop occurs inter-vocalically or pre-consonantally (see §\ref{subsec: glottal stop}).


\largerpage[2]
Finally, while these examples show that pre-aspirated stops contrast with plain stops that may be preceded by a glottal stop, for some speakers pre-aspirated stops are variably realized as pre-aspirated or pre-glottalized, e.g., \textit{aˈ}\textbf{\textit{ʰ}}\textbf{\textit{k}}\textit{à {\textasciitilde} a}\textbf{\textit{\textsuperscript{ʔ}}}\textbf{\textit{ˈk}}\textit{à} ‘sandal’ in a pattern that resembles a pattern of optional pre-aspiration or pre-laryngealization of fortis consonants in \ili{Southern Nevada Northern Paiute} (SNNP; Western \ili{Numic}) \parencite{kataoka2010phonetic}.\footnote{Different \ili{Numic} varieties exhibit different reflexes of the reconstructed gradation system of Proto-\ili{Numic}, with the lenis series involving voicing, spirantization, and rhotacization, with no lenition of the fortis series. In \ili{Mono Lake Northern Paiute} (MLNP; Western \ili{Numic}), a system with lenis, voiced fortis and fortis oral stops and affricates, the fortis-lenis contrast is argued to involve closure duration, aspiration duration and voicing during the closure interval \citep[][234]{babel2013descent}. In the closely related \ili{Southern Nevada Northern Paiute} (SNNP; Western \ili{Numic}) the main correlate of the fortis-lenis contrast involves the discontinuity of the acoustic signal for the fortis series, achieved by longer consonant duration and optional co-occurrence of preaspiration or preglottalization  \parencite{kataoka2010phonetic}.}

The contrast between voiced and voiceless stops at the bilabial and alveolar place of articulation is shown in (\ref{ex: voicing contrast}). There are no voiced coronal or velar stops in the phonological inventory of Choguita Rarámuri. The voiced counterpart of /t/ is a coronal flap (allophonic alternations involving these two segments in the phonology of this language are discussed below in §\ref{subsec: processes targeting rhotics}).

\ea\label{ex: voicing contrast}
{Voicing contrast for bilabial and alveolar oral stops}

    \ea[]{
    [\textbf{p}aˈtʃ͡í]   \\
    /\textbf{p}aˈtʃ͡í/\\
    ‘corn cob’\\
    `elote'     \corpuslink{co1136[03_362-03_380].wav}{MDH co1136:03:36.2}  \\
}
        \ex[]{
        [\textbf{b}aˈ\textbf{ʰ}tʃ͡í] \\
        /\textbf{b}aˈ\textbf{ʰ}tʃ͡í/\\
        ‘zucchini’\\
        `calabacita'     \corpuslink{tx130[00_525-01_011].wav}{LEL tx130:00:52.5}\\ %[\textbf{w}aˈ\textbf{ʰ}tʃi]?\\
    }
            \ex[]{
            [\textbf{p}aˈtʃ͡á]  \\
            /\textbf{p}aˈtʃ͡á/\\
            ‘inside’\\
            `adentro'         \corpuslink{tx71[04_107-04_148].wav}{LEL tx71:04:10.7}\\
        }
                \ex[]{
                [\textbf{b}aˈtʃ͡á]  \\
                /\textbf{b}aˈtʃ͡á/\\
                ‘first’\\
                `primero'    \corpuslink{tx1[01_105-01_160].wav}{BFL tx1:01:10.5} \\
                %Q: [\textbf{w}aˈtʃa]?\\
            }
                    \ex[]{
                    [\textbf{t}aˈrâri]\\
                    /\textbf{t}aˈrâri/\\
                    ‘week’\\
                    `semana'       \corpuslink{tx84[01_457-01_479].wav}{LEL tx84:01:45.7}\\
                }
%        \pagebreak
                        \ex[]{
                        [\textbf{r}aʔlaˈká]\\
                        /\textbf{ɾ}aʔla-ˈká/ \\
                        buy-\textsc{ger}\\
                        ‘buying’\\
                        `comprando'\footnote{As attested in this example, the coronal flap is allophonically realized as a trill in word-initial position.} \corpuslink{tx_falda[00_269-00_339].wav}{BFL tx\_falda:00:26.9}\\
                    }
    \z
\z

Given that the contrast between pre-laryngealized and plain voiceless stops is neutralized in unstressed word-initial position (since pre-laryngealized stops are only attested as onsets of stressed syllables), these minimal pairs reveal a contrast between voiced stops, on the one hand, and either plain voiceless stops or pre-laryngealized voiceless stops, on the other.\footnote{As pointed out by Marc Garellek (p.c.), voiced stops in these examples could potentially be also analyzed as underlyingly plain voiceless, given a process of gradient lenition of these segments which yields voiced segments in fast speech (addressed below in §\ref{subsec: lenition of voiceless plosives}). In these cases, however, we would expect voiced stops to undergo different lenition processes. For instance, word-initial voiced bilabial stops may reduce to labiovelar glides ([\textbf{w}aˈkotʃ͡i] /\textbf{b}aˈkôtʃ͡͡i/ ‘river’< SFH 04 1:17/el >), as described in more detail in §\ref{subsec: spirantization of voiced bilabial stops}. In other words, lenition of stops in Choguita Rarámuri is generally non-neutralizing, a generalization that appears to hold cross-linguistically (\citealt{gurevich2013lenition}).}

The voiced bilabial stop contrasts with the labiovelar semivowel. This is shown in the examples in (\ref{ex: w vs. b})(for a description of processes targeting labio-velar glides in different positions within the syllable, see §\ref{subsec: semi-vowel monophthongization} in \chapref{chap: syllables}.)

\newpage
\ea\label{ex: w vs. b}
{/w/ vs. /b/}

    \ea[]{
    [wasaˈtʃ͡í]\\
    /\textbf{w}asa-ˈtʃ͡í/  \\
    field- \textsc{loc}\\
    `in the field'\\
    `en el campo de cultivo' < BFL 07 el328/el >\\
}
        \ex[]{
        [basaˈtʃ͡í]\\
        /\textbf{b}asaˈtʃ͡í/ \\
        ‘coyote’   < BFL 07 el328/el > \\
    }
            \ex[]{
            [biʔriˈbáma]\\
            /\textbf{b}iʔri-ˈbá-ma/\\
            clean.\textsc{intr-inch-fut.sg}\\
            `It will get cleaned.'\\
            `Se va a limpiar.' < BFL 08 1:21/el >\\
        }
%\pagebreak
                \ex[]{
                [wiliˈbáma]\\
                /\textbf{w}ili-ˈbá-ma/\\
                stand.\textsc{intr-inch-fut.sg}\\
                `It will stand up.'\\
                `Se va a parar.' < BFL 08 1:21/el >\\
            }
    \z
\z

The contrast between the voiced bilabial stop and the labiovelar semivowel in word medial position is shown in (\ref{ex: b vs. w}a--b).

\ea\label{ex: b vs. w}
{/b/ vs. /w/}

    \ea[]{
   [ka\textbf{ˈb}í] \\
   /ka\textbf{ˈb}í/\\
    ‘to roll something’\\
    `enrollar'  < BFL 07 el321/el >\\
}
        \ex[]{
        [ka\textbf{ˈw}í] \\
        /ka\textbf{ˈw}í/\\
        ‘sunrise’\\
        `amanecer'   < BFL 07 el321/el >\\
    }
    \z
\z

The contrast between presence and absence of glottal stop before a consonant is shown in (\ref{ex: w vs. b}c--d) and also in (\ref{ex: ? vs. 0}).\footnote{The near-minimal pair \textit{niˈwì} ‘marry’ and \textit{niʔˈwí} ‘to be lightning’ differ in terms of absence or presence, respectively, of /ʔ/, in addition to lexical tone differences.}

\ea\label{ex: ? vs. 0}
{/ʔ/ vs. Ø}

    \ea[]{
   [niˈwì] \\
   /niˈwì/\\
    `to marry in church'\\
    `casarse en la iglesia'  \corpuslink{el1318[28_549-28_565].wav}{MFH el1318:28:54.9}\\
}
        \ex[]{
       [ni\textbf{ʔ}ˈwí] \\
       /ni\textbf{ʔ}ˈwí/\\
        ‘to be lightning’\\
        `relampaguear'    \corpuslink{el1274[07_114-07_127].wav}{JLG el1274:07:11.4}\\
    }
            \ex[]{
            [raˈnê]\\
            /raˈn-ê/  \\
            offspring-\textsc{have}\\
            ‘have children’\\
            `tener hijos'      \corpuslink{tx32[08_595-09_033].wav}{LEL tx32:08:59.5}\\
        }
                \ex[]{
                [ra\textbf{ʔ}nè]   \\
                /ra\textbf{ʔ}nè/\\
                `shoot at'\\
                `disparar'  \corpuslink{tx221[02_227-02_279].wav}{LEL tx221:02:22.7}\\
            }
                    \ex[]{
                    [ka\textbf{ˈw}í]\\
                    /ka\textbf{ˈw}í/\\
                    ‘sunrise’\\
                    `amanecer'   < BFL 07 el321/el >\\
                }
                        \ex[]{
                        [ka\textbf{ʔ}ˈwí]\\
                        /ka\textbf{ʔ}ˈwí/\\
                        ‘bring wood’\\
                        `traer leña'   < BFL 07 el321/el >\\
                    }
    \z
\z

The contrast between the alveolar flap (<r> [ɾ]) and the lateral flap (<l> [ɽ]) is shown in (\ref{ex: r vs. l}).

%check tones and add references
\ea\label{ex: r vs. l}
{/r/ vs. /l/}

    \ea[]{
    [\textbf{r}oˈwí]\\
    /\textbf{r}oˈwí/\\
    `rabbit'\\
    `conejo'\\
}
        \ex[]{
        [\textbf{l}oˈwí]\\
        /\textbf{l}oˈwí/\\
        `silly, distracted'\\
        `tonto, distraído' \corpuslink{co1136[09_421-09_441].wav}{MDH co1136:09:42.1}\\
    }
            \ex[]{
            [loˈwè]\\
            /\textbf{l}oˈw-è/\\
            stir-\textsc{appl}\\
            `stir for someone'\\
            `menearlo'  \corpuslink{tx60[02_342-02_378].wav}{BFL tx60:02:34.2}\\
        }
                \ex[]{
                [\textbf{r}oˈwé]\\
                /\textbf{r}oˈwé/\\
                `to run (women's) \textit{ariweta} race'\\
                `correr carrera de \textit{ariweta}  \corpuslink{el497[02_560-02_580].wav}{SFH el497:02:56.0}\\
            }
    \z
\z

The contrast between the two rhotic sounds is illustrated in \figref{fig: alveolar flap illustration} (with the alveolar flap phoneme <r> [\textup{ɾ}] realized as a trill word-initially) and \figref{fig: retroflex flap illustration} (with the retroflex flap phoneme <l> [ɽ] word-initially).

\begin{figure}
\includegraphics[width=\textwidth]{figures/Segmental_Phonology_img4.png}
\caption{
\label{fig: alveolar flap illustration}
Spectrogram showing realization of the alveolar flap phoneme <r> [\textup{ɾ}] as a trill word-initially in [iˈwé \textbf{r}oˈwétia] /iˈwé \textbf{ɾ}oˈwé-ti-a/ `The girls will run (a women's race)' \corpuslink{tx19[00_265-00_312].wav}{LEL tx19:00:26.5}.}
\end{figure}

\begin{figure}
\includegraphics[width=\textwidth]{figures/Segmental_Phonology_img5.png}
\caption{
\label{fig: retroflex flap illustration}
Spectrogram showing realization of the retroflex flap phoneme <l> [ɽ] word-initially in [we \textbf{l}oˈwèma] /we \textbf{l}oˈw-è-ma/ `you stir it well' \corpuslink{tx60[02_342-02_378].wav}{BFL tx60:02:34.2}.}
\end{figure}

The contrast between the bilabial nasal phoneme and the alveolar nasal phoneme is evidenced in the contrasts depicted in the verbal stems in (\ref{ex: m vs. n}).

\ea\label{ex: m vs. n}
{/m/ vs. /n/}

    \ea[]{
    [\textbf{m}iˈhí]  \\
    /\textbf{m}iˈhí/\\
    ‘cook mezcal’   \\
    `cocinar mezcal' < SFH 07 2:12/el >\\
}
        \ex[]{
        [\textbf{n}iˈhî]  \\
        /\textbf{n}iˈhî/\\
        ‘give away, gift’ \\
        `dar, regalar' < ROF 04 1:67/el >\\
    }
            \ex[]{
            [maˈlála]\\
            /\textbf{m}aˈlá-la/\\
            daughter.male\_ego-\textsc{poss}\\
            `his daughter' \\
            `su hija de él' < BFL 05 1:155/el >\\
        }
                \ex[]{
                [naˈlàla]\\
                /\textbf{n}aˈlà-la/ \\
                cry\textsc{.pot} \\
                `s/he can cry'   \\
                `puede llorar' < SFH 05 1:69/el >\\
            }
    \z
\z

The next examples show that the alveolar fricative (\ref{ex: s vs. sh}a, c) contrasts with the alveo-palatal affricate (\ref{ex: s vs. sh}b, d).

%\newpage

\ea\label{ex: s vs. sh}
{/s/ vs. /tʃ͡/}

    \ea[]{
    [\textbf{s}iˈmí]   \\
    /\textbf{s}iˈmí/\\
    ‘go, \textsc{sg}’\\
    `ir, \textsc{sg}'     \corpuslink{tx177[02_253-02_355].wav}{LEL tx177:02:25.3}\\
}
        \ex[]{
        [\textbf{tʃ}iˈmí] \\
        /\textbf{tʃ}iˈmí/\\
        `over there'\\
        `allá’        < BFL 07 el325/el >\\
    }
            \ex[]{
            [i\textbf{ˈs}î]\\
            /i\textbf{ˈs}î/\\
            ‘urinate’\\
            `orinar'      \corpuslink{el1318[04_133-04_141].wav}{MFH el1318:04:13.3}>\\
        }
                \ex[]{
                [i\textbf{ˈtʃ}ì] \\
                /i\textbf{ˈtʃ}ì-/\\
                `plant’\\
                `sembrar'       \corpuslink{tx130[03_078-03_133].wav}{LEL tx130:03:07.8}\\
            }
    \z
\z

%to check: the analysis of pre-laryngealized as both pre-glottalized and pre-aspirated

The next set examples show the contrast between pre-aspirated voiceless velar stops and ʔ-voiceless velar consonant sequences (\ref{ex: hk  vs. ?}).

\ea\label{ex: hk  vs. ?}
{/ʰk/ vs. /ʔk/}

    \ea[]{
    [aˈ\textbf{ʰ}kâ]  \\
    /aˈ\textbf{ʰ}kâ/\\
    ‘sweet/salty’  \\
    ‘dulce/salado' < BFL 07 el324/el >\\
}
        \ex[]{
        [a\textbf{ʔ}ˈká]\\
        /a\textbf{ʔ}ˈká/\\
        ‘sandal’ \\
        `huarache' < BFL 07 el324/el >\\
    }
            \ex[]{
            [oˈ\textbf{ʰ}kó] \\
            /oˈ\textbf{ʰ}kó/\\
            ‘pine tree’      \\
            `pino' < BFL 07 el323/el >\\
        }
                \ex[]{
                [o\textbf{ʔ}ˈkô]  \\
                /o\textbf{ʔ}ˈkô/\\
                ‘pain, to hurt’   \\
                `dolor/doler' < BFL 07 el323/el >\\
            }
    \z
\z

Choguita Rarámuri possesses only two semivowels: a voiced labiovelar /w/ and a palatal approximant /j/. The phonemic status of the labiovelar semivowel has already been shown in (\ref{ex: b vs. w}) and (\ref{ex: w vs. b}) above. Some of the examples in (\ref{ex: b vs. w}) and (\ref{ex: w vs. b}) are repeated below in (\ref{ex: b vs. w 2}), where the voiced bilabial stop (\ref{ex: b vs. w 2}a, c) contrasts with the labiovelar semivowel (\ref{ex: b vs. w 2}b, d).

\ea\label{ex: b vs. w 2}
{/b/ vs. /w/}

    \ea[]{
    [ka\textbf{ˈb}í] \\
    /ka\textbf{ˈb}í/\\
    ‘to roll something’  \\
    `enrollar' < BFL 07 el321/el >\\
}
        \ex[]{
        [ka\textbf{ˈw}í] \\
        /ka\textbf{ˈw}í/\\
        ‘sunrise’     \\
        `amanecer' < BFL 07 el321/el >\\
    }
            \ex[]{
            [\textbf{b}iʔriˈbáma]\\
            /\textbf{b}iʔri-ˈbá-ma/ \\
            clean.\textsc{intr-inch-fut.sg}\\
            `It will become clean.'\\
            `Se va a limpiar.' < BFL 08 1:21/el >\\
        }
                \ex[]{
                [\textbf{w}iliˈbáma]\\
                /\textbf{w}ili-ˈbá-ma/\\
                stand.\textsc{intr-inch-fut.sg}\\
                `It will stand up.'\\
                `Se va a parar.' < BFL 08 1:21/el >\\
            }
    \z
\z

The phonemic status of the palatal semivowel, on the other hand, is evidenced in the minimal pairs in (\ref{ex: ch vs. j}), where /j/ (\ref{ex: ch vs. j}a, c) contrasts word-medially with a glottal stop (\ref{ex: ch vs. j}b, d).

\ea\label{ex: ch vs. j}
{/j/ vs. /ʔ/}

    \ea[]{
    [tʃ͡oˈ\textbf{j}á]  \\
    /tʃ͡oˈ\textbf{j}á/\\
    ‘shrink'  \\
    `encogerse' < BFL 06 5:44/el >\\
}
        \ex[]{
        [tʃ͡oˈ\textbf{ʔ}á] \\
        /tʃ͡oˈ\textbf{ʔ}á/\\
        ‘extinguish (fire)'  \\
        `extinguirse' < SFH 04 1:71/el >\\
    }
            \ex[]{
            [koˈ\textbf{j}á]\\
            /koˈ\textbf{j}á/\\
            ‘squat’       \\
            `ponerse en cuclillas' < BFL 05 1:186/el > \\
        }
                \ex[]{
                [koˈ\textbf{ʔ}á] \\
                /koˈ\textbf{ʔ}á/\\
                ‘eat’    \\
                `comer' < SFH 04 1:69/el > \\
            }
    \z
\z

More about the phonemic status of labiovelar and palatal semivowels, as well as processes related to these segments, are discussed in §\ref{sec: general phonological processes}.

\subsection{Vocalic minimal pairs}
\label{subsec: minimal pairs vowels}

The examples in (\ref{ex: i vs. u}) show minimal pairs involving high vowels (the contrasting vowels are indicated with boldface).

\ea\label{ex: i vs. u}
{/i/ vs. /u/}

    \ea[]{
    [t͡ʃ\textbf{i}ˈkûri]\\
    /t͡ʃ\textbf{i}ˈkûri/\\
    ‘mouse' \\
    `ratón' \corpuslink{co1136[13_082-13_107].wav}{MDH co1136:13:08.2}\\
}
        \ex[]{
        [t͡ʃuˈkúli]\\
        /t͡ʃ\textbf{u}ˈkú-li/ \\
        be.bent\textsc{-pst}\\
        `It was (bent).'  \\
        `Estaba (curvado).' \corpuslink{tx177[08_128-08_194].wav}{LEL tx177:08:12.8}\\
    }
            \ex[]{
            [h\textbf{i}ˈrâ] \\
            /h\textbf{i}ˈrâ/\\
            ‘to bet’   \\
            `apostar' \corpuslink{tx19[01_323-01_373].wav}{LEL tx19:01:32.3}\\
        }
                \ex[]{
                [h\textbf{u}ˈrá]   \\
                /h\textbf{u}ˈrá/\\
                ‘to send’    \\
                `mandar' \corpuslink{el261[00_205-00_243].wav}{SFH, MGD el261:00:20.5}\\
            }
    \z
\z

Minimal pairs involving back vowels are shown in (\ref{ex: o vs. u}).

\ea\label{ex: o vs. u}
{/o/ vs. /u/}

    \ea[]{
    [ˈt\textbf{ô}]  \\
    /ˈt\textbf{ô}/\\
    ‘bury’   \\
    `enterrar' \corpuslink{el1240[03_295-03_300].wav}{MAF el1240:3:29.5}\\
}
        \ex[]{
        [ˈt\textbf{û}] \\
        /ˈt\textbf{û}/\\
        ‘down’       \\
        `abajo' \corpuslink{tx5[01_523-01_561].wav}{LEL tx5:01:52.3}\\
    }
            \ex[]{
            [ˈk\textbf{o}]\\
            /ˈk\textbf{o}/\\
            `emphatic clitic (\textsc{emph})'   \\
            `clítico enfático' \corpuslink{tx43[06_587-07_007].wav}{SFH tx43:06:58.7}\\
        }
                \ex[]{
                [ˈk\textbf{ù}]\\
                /ˈk\textbf{ù}/\\
                ‘wood’   \\
                `leña' \corpuslink{in61[05_519-05_548].wav}{FLP in61:05:51.9}\\
            }
    \z
\z

Front vowels also create phonemic contrasts, as shown in (\ref{ex: i vs. e}).

\ea\label{ex: i vs. e}
{/i/ and /e/}

    \ea[]{
    [ˈw\textbf{î}]\\
    /ˈw\textbf{î}/\\
    ‘harvest’   \\
    `cosechar' \corpuslink{el549[04_420-04_431].wav}{SFH el549:04:42.0}\\
}
        \ex[]{
        [ˈw\textbf{e}] \\
        /ˈw\textbf{e}/\\
        ‘very’     \\
        `muy, mucho' \corpuslink{tx84[07_283-07_290].wav}{LEL tx84:07:28.3}\\
    }
            \ex[]{
            [ˈt\textbf{i}]\\
            /ˈt\textbf{i}/\\
            `singular definite article, negative evaluation (\textsc{def.bad})'   \\
            `artículo definido singular, evaluación negativa' \corpuslink{tx5[04_517-04_538].wav}{LEL tx5:04:51.7}\\
        }
                \ex[]{
                [ˈt\textbf{é}]\\
                /ˈt\textbf{é}/\\
                ‘lice’   \\
                `piojos' < SFH 04 1:17/el >\\
            }
    \z
\z

Below, the contrast between (\ref{ex: i vs. e in pre-tonic positiona}) (with the verb \textit{biˈtí,} ‘stand up up’) and (\ref{ex: i vs. e in pre-tonic positionb}) (with the verb \textit{beˈte} ‘stay overnight’) shows that the contrast between front vowels is not restricted to stressed position (although there is a widespread vowel reduction process that targets unstressed vowels (discussed in detail in §\ref{subsubsec: stress-based vowel reduction and deletion}).

\ea\label{ex: i vs. e in pre-tonic position}
{/i/ vs. /e/ in pre-tonic position}

    \ea[]{
    [bitiˈbáma]\\
    /b\textbf{i}t\textbf{i}-ˈbá-ma/ \\
    stand.up-\textsc{inch-fut.sg}\\
    `It will stand up.'\\
    `Se va a parar.' \corpuslink{tx1[00_416-00_451].wav}{BFL tx1:00:41.6}\\
}\label{ex: i vs. e in pre-tonic positiona}
        \ex[]{
        [beteˈbása]\\
        /b\textbf{e}t\textbf{e}-ˈbá-sa/\\
        stay.overnight-\textsc{inch-imp.sg}\\
        `If it stays overnight.'\\
        `Si se queda.' \corpuslink{in243[01_295-01_347].wav}{FLP in243:01:29.5}\\
    }\label{ex: i vs. e in pre-tonic positionb}
    \z
\z

Finally, the examples in (\ref{ex: e vs. a}--\ref{ex: o vs. e vs. a}) show contrasts between [$-$high] vowels.

\ea\label{ex: e vs. a}
{/e/ vs. /a/}

    \ea[]{
    [ˈ\textbf{è}]  \\
    /ˈ\textbf{è}/\\
    ‘take away’  \\
    `llevar' < BFL 07 el336/el >\\
}
        \ex[]{
        [ˈ\textbf{ʔà]} \\
        /ˈ\textbf{ʔà}/\\
        ‘give'\\
        `dar' < BFL 07 el336/el >\\
    }
            \ex[]{
            [iˈréli]\\
            /iˈr\textbf{é-}li/ \\
            lock-\textsc{pst} \\
            `It locked it.'\\
            `Lo encerró.' < BFL 07 VDB/el >\\
        }
                \ex[]{
                [iˈr\textbf{á}ri]\\
                /iˈr\textbf{á}ri/\\
                ‘godparent, godchild’  \\
                `padrino, madrina, ahijado, ahijada'\\
            }
    \z
\z

\ea\label{ex: o vs. e vs. a}
{/o/ vs. /e/ vs. /a/}

    \ea[]{
    [ˈm\textbf{ó}]    \\
    /ˈm\textbf{ó}/\\
    ‘go up, sg.’ \\
    `subir' \corpuslink{in243[20_588-21_021].wav}{FLP in243:20:58.8}\\
}
        \ex[]{
        [ˈm\textbf{ê}]\\
        /ˈm\textbf{ê}/\\
        ‘win’ \\
        `ganar' \corpuslink{el1242[01_563-01_570].wav}{MAF el1242:01:56.3}\\
    }
            \ex[]{
            [ˈm\textbf{à}]\\
            /ˈm\textbf{à}/\\
            ‘run’   \\
            `correr' \corpuslink{tx177[05_208-05_246].wav}{LEL tx177:05:20.8}\\
        }
                \ex[]{
                [ˈt\textbf{ò}] \\
                /ˈt\textbf{ò}/\\
                ‘Give it to me!’ \\
                `¡Dámelo!'\\
            }
                    \ex[]{
                    [ˈt\textbf{é}]\\
                    /ˈt\textbf{é}/\\
                    ‘lice’     \\
                    `piojo'     < SFH 04 1:17/el >\\
                }
                        \ex[]{
                        [ˈt\textbf{á}] \\
                        /ˈt\textbf{á}/\\
                        ‘small’ \\
                        `pequeño' \corpuslink{co1238[01_532-01_549].wav}{JLG co1238:01:53.2}\\
                    }
    \z
\z

The next section lays out details of several general phonological processes targeting both consonantal and vocalic segments.

\section{Processes}
\label{sec: general phonological processes}

The following is a comprehensive description of phonological processes targeting consonantal segments in Choguita Rarámuri, including palatalization of alveolar fricatives, nasal place assimilation, processes targeting rhotics, postconsonantal devoicing and spirantization of voiced bilabial stops.

%semi-vowel processes?

\subsection{Palatalization of alveolar fricatives}
\label{subsec: palatalization of alveolar fricatives}

Alveolar fricatives palatalize before high vowels. The fricative palatalization rule is schematized in (\ref{ex: fricative palatalization rule}).

\ea\label{ex: fricative palatalization rule}
Fricative palatalization rule

s  	→  ʃ / {\longrule} [+ high] V
\z

The degree of palatalization is subject to speaker variation: for many speakers, the allophone is realized as a slightly retroflexed sibilant. Some speakers, however, produce a full-fledged alveopalatal fricative. The following examples show palatalized fricatives in word-medial (\ref{ex: fricative paltalization before high vowels}a--c) and word-initial position (\ref{ex: fricative paltalization before high vowels}d--f).

\ea\label{ex: fricative paltalization before high vowels}
{Fricative palatalization before high vowels}

    \ea[]{
    [kaˈ\textbf{ʃ}ì]    \\
    /kasì/   \\
    `to shatter’\\
    `quebrar’      < ROF 04 1:59/el >\\
}
        \ex[]{
        [oˈ\textbf{ʃ}ì]    \\
        /os-ì/  \\
        write.read-\textsc{appl}
        ‘to read or write for someone’\\
        ‘escribirle o leerle a alguien’    < JHF 04 1:5/el >\\
    }
            \ex[]{
            [bu\textbf{ʃ}uˈrê]  \\
            /busurê/ \\
            `to wake up’\\
            `despertarse’ < BFL 05 1:133/el >\\
        }
                \ex[]{
                [\textbf{ʃ}utuˈbétʃ͡i] \\
                /sutubétʃ͡i/\\
                `to trip’\\
                `tropezarse’ < BFL 05 1:187/el >\\
            }
                    \ex[]{
                    [\textbf{ʃ}uˈní]  \\
                    /suní/  \\
                    `to finish’\\
                    `terminar’   < BFL 06 tx1(18)/tx >\\
                }
                        \ex[]{
                        [\textbf{ʃ}iˈmêa] \\
                        /si-ˈmêa/ \\
                        go.\textsc{sg-fut.sg}\\
                        `S/he will go.'\\
                        `Irá.' < BFL 06 tx48(24)/tx >\\
                    }
    \z
\z

The rule of fricative palatalization is counterbled by post-tonic vowel deletion. In (\ref{ex: opaque fricative palatalization}), palatalization overapplies since the trigger high vowel from the underlying representation has been deleted.

\ea\label{ex: opaque fricative palatalization}
{Opaque fricative palatalization}

        \ea[]{
        [aˈtí\textbf{ʃ}li] \\
        /aˈtísi-li/ \\
        to.sneeze-\textsc{pst}  \\
        `S/he sneezed.'\\
        `Estornudó.' < BFL 05 1:111/el >  \\
    }
            \ex[]{
            [miʔˈà\textbf{ʃ}nula] \\
            /miʔˈà-si-nula/ \\
            to.kill-\textsc{mot-order} \\
            `S/he made them kill it.'\\
            `Hizo que lo matara.' < BFL 06 4:145/el > \\
        }
                \ex[]{
                [nuˈlè\textbf{ʃ}ti-] \\
                /nuˈl-è-si-ti-/  \\
                to.command-\textsc{mot-caus-} \\
                `S/he makes them go along commanding.'\\
                `Los hace ir mandando.' < BFL 06 4:145/el >  \\
            }
                    \ex[]{
                    [wiliˈbá\textbf{ʃ}nila]\\
                    /wili-ˈbá-si-nula/ \\
                    stand-\textsc{inch-mot-order} \\
                    `S/he makes them go along standing.'\\
                    `Lo hace que se vaya parando.' < BFL 06 1:10 > \\
                }
    \z
\z

The overapplication is schematized in (\ref{ex: counterbleeding}):

\ea\label{ex: counterbleeding}
{Counterbleeding}

{Underlying Representation  /atísi-li/}

    \ea[]{
    s  	→   ʃ   / {\longrule} [+high] V  [aˈti\textbf{ʃ}i-li]\\}
        \ex[]{
        V  	→  Ø  / stressed syllable \_ [aˈti\textbf{ʃ}{}-li]\\
    }
{Surface Representation  [aˈti\textbf{ʃ}-li]}\\
    \z
\z

There are thus many palatalized fricatives with no surface-apparent conditioning environment.

\subsection{Optional nasal place assimilation}
\label{subsec: optional nasal place assimilation}

Nasal phonemes in Choguita Rarámuri display very little allophonic variation. Alveolar nasals display optional nasal place assimilation when they precede bilabial stops. This process, as shown in (\ref{ex: optional nasal place assimilation}), is fed by post-tonic vowel deletion (the stressed vowels undergo compensatory lengthening, cf. §\ref{subsubsec: compensatory lengthening}).

\ea\label{ex: optional nasal place assimilation}
{Optional nasal place assimilation}

    \ea[]{
    [ˈée\textbf{m}pi]\\
    /ˈéna-pi/  \\
    go.\textsc{pl-irr.fut.pl} \\
    `Maybe they will go.'\\
    `A lo mejor van a ir.' < BFL 07 el35/el > \\
}
        \ex[]{
        [ˈsûu\textbf{m}po]\\
        /ˈsû-ni-po/  \\
        sew-\textsc{appl-fut.pl} \\
        `They will sew for them.'\\
        `Les van a coser (a ellas).' < SFH 05 1:80/el >\\
    }
                \ex[]{
                [raˈpáa\textbf{m}po]\\
                /raˈpá-na-po/  \\
                split.\textsc{appl-tr-fut.pl} \\
                `They will split (the wood) for them.'\\
                `Les van a partir (la leña).' < AHF 05 1:131/el >\\
            }
    \z
\z

The optionality of nasal place assimilation is evident from examples like (\ref{ex: no nasal place assimilation}), where an underlying alveolar nasal does not assimilate to a following bilabial stop after post-tonic vowel deletion (the forms with asterisk indicate unattested forms with vowel deletion and nasal place assimilation).

\ea\label{ex: no nasal place assimilation}
{No nasal place assimilation before bilabial stops}

    \ea[]{
     \glt   [ˈtʃóoo\textbf{n}po]\\
     \glt   /ˈtʃ͡ó\textbf{n}i-po/  \\
     \glt   step\textsc{-fut.pl}\\
     \glt `They will step.'\\
     \glt `Van a pisar.' < BFL 07 VDB/el >\\
     \glt   {\textit{*ˈtʃ͡óo\textbf{m}po}}\\
}
        \ex[]{
        [ˈjóo\textbf{n}po]  \\
        \glt /ˈjó\textbf{n}i-po/ \\
        \glt nag-\textsc{fut.pl}\\
        \glt `They will nag.'\\
        \glt `Van a regañar.' < BFL 07 VDB/el >\\
        {\textit{*ˈjóo\textbf{m}po}}\\
    }
            \ex[]{
            [ˈlàa\textbf{n}po] \\
            \glt /ˈlà\textbf{n}i-po/  \\
            \glt bleed-\textsc{fut.pl}\\
            \glt `They will bleed.'\\
            \glt `Van a sangrar.' < ROF 04 1:82/el >   \\
            {\textit{*ˈlàa\textbf{m}po}} \\
        }
                \ex[]{
                [roˈmíi\textbf{n}po]  \\
                \glt /roˈmí-\textbf{n}a-po/  \\
                \glt fold-\textsc{tr-fut.pl}\\
                \glt `They will fold it.'\\
                \glt `Van a doblarlo.' < AHF 05 1:126/el >     \\
                {\textit{*roˈmíi\textbf{m}po}}\\
            }
    \z
\z

There is no evidence that alveolar and bilabial nasals assimilate in place of articulation to other (non-bilabial) following segments. Some relevant examples are shown in (\ref{ex: no nasal place assimilation 2}).

\ea\label{ex: no nasal place assimilation 2}
{No nasal place assimilation}

    \ea[]{
    [ˈtâa\textbf{n}ko]\\
    /ˈtâ-ni-ki-o/  \\
    ask-\textsc{appl-appl-ep}\\
    `S/he asks them'.\\
    `Le pregunta.' < BFL 07 1:31/el >\\
}
        \ex[]{
        [tʃ͡aʔiˈbûu\textbf{n}ki] \\
        /tʃ͡aʔi-ˈbû\textbf{{}-}ni-ki/\\
        stuck-\textsc{tr-appl-pst.ego}\\
        `I stuck it for them.'\\
        `Se lo atoré.' < BFL 07 1:32/el >\\
    }
            \ex[]{
            [muruˈbêe\textbf{n}ki]\\
            /muruˈbê-ni-ki/  \\
            get.close-\textsc{appl-pst.ego}\\
            `I got it close for them.'\\
            `Se lo acerqué.' < BFL 06 6:146/el >\\
        }
                \ex[]{
                [ˈnáa\textbf{m}tu]\\
                /ˈnámi-ru/  \\
                hear-\textsc{nmlz}\\
                `a noise'\\
                `sonido' < SFH 04 1:112/el >\\
            }
                    \ex[]{
                    [baˈrâa\textbf{m}ki] \\
                    /baˈrâmi-ki/  \\
                    be.thirsty-\textsc{pst.ego}\\
                    `I was thirsty.'\\
                    `Tuve sed.' < BFL 05 1:132/el >\\
                }
                        \ex[]{
                        [baˈtʃ͡íi\textbf{m}tipo]\\
                        /baˈtʃ͡ími-ti-po/ \\
                        sprinkle-\textsc{caus-fut.pl}\\
                        `They will cause them to sprinkle.'\\
                        `Lo van a hacer que rocíe agua.' < BFL 05 1:135/el >\\
                    }
    \z
\z

In (\ref{ex: no nasal place assimilation 2}a--c) post-tonic deletion yields a heterosyllabic cluster with a nasal followed by a voiceless velar; the alveolar nasals, however, are not velarized. Bilabial nasals do not assimilate in place of articulation to following alveolar or velar stops either (as shown in (\ref{ex: no nasal place assimilation 2}d--f)).

Finally, there is no place assimilation when an alveolar nasal precedes a bilabial nasal, as exemplified in (\ref{ex: no nasal place assimilation of alveolar nasal}).

\ea\label{ex: no nasal place assimilation of alveolar nasal}
{No place assimilation of alveolar nasal}

    [raˈjè\textbf{n}ma] \\
    \glt    /raˈjèni-ma/  \\
            be.sunny-\textsc{fut.sg}\\
    \glt    `It will be sunny.'\\
    \glt    `Va a estar soleado.' < SFH 06 1:128/el >\\

\z

There are no cases in the corpus where an alveolar nasal assimilates in place of articulation to a following bilabial nasal.

\subsection{Processes targeting rhotics}
\label{subsec: processes targeting rhotics}
\largerpage
Recall from §\ref{subsec: consonants} above that there are two rhotics in Choguita Rarámuri, an alveolar flap (/ɾ/ <r>) and a lateral flap (/ɽ/ <l>). The alveolar flap is realized as an alveolar trill word-initially. The rule in (\ref{ex: word-initial allophone of r}) represents this exceptionless generalization.\footnote{An anonymous reviewer asks whether this process may be due to contact with \ili{Spanish}, where a similar pattern is attested. According to \citet{merrill2014ralamuli}, word-initial [r] in Rarámuri varieties developed from a *t > r sound change in the early nineteenth century (\citeyear[242]{merrill2014ralamuli}), at a time period where colonial expansion had taken place. This change could thus have involved contact-induced transfer of a phonological pattern, though it may also be an internal development.}

\ea\label{ex: word-initial allophone of r}
{Word-initial allophone of /ɾ/}

/ɾ/ → [r] / \#{\longrule}

\z

The examples in (\ref{ex: word-initial trill}) include an IPA notation (where [r] represents the alveolar trill and /ɾ/ stands for the alveolar flap in the underlying representation). Since alveolar trills and flaps are in complementary distribution, these sounds are represented with <r> in the rest of the grammar.

%\pagebreak

\ea\label{ex: word-initial trill}
{Word-initial trill}

    \ea[]{
    [\textbf{r}aʔˈlá]   \\
    /ɾaʔˈlá/    \\
    ‘to buy’\\
    ‘comprar’  < AHF 07 1:74/el >\\
}
        \ex[]{
        [\textbf{r}oˈnò]  \\
        /ɾoˈnò/ \\
        ‘to boil’\\
        ‘hervir’ < SFH 04 1:81/el >\\
    }
            \ex[]{
            [\textbf{r}eˈká]   \\
            /ɾeˈká/ \\
            ‘to lay down’\\
            ‘acostarse’   < SFH 04 1:67/el >\\
        }
                \ex[]{
                 [\textbf{r}itiˈwá] \\
                 /ɾitiˈwá/  \\
                 ‘to watch’\\
                 ‘ver’  < JHF 04 1:2/el >\\
            }
                    \ex[]{
                    [\textbf{r}uˈhì]  \\
                    /ɾuˈhì/ \\
                    ‘to hail’\\
                    ‘granizar’ < SFH 04 1:123/el >\\
                }
    \z
\z

Alveolar trills are not uncommon in word-medial position in loanwords from \ili{Spanish}: [ba\textbf{ˈr}îkatʃ͡i], ‘bucket’, from \textit{barrica}, and [mo\textbf{ˈr}âltʃ͡i], ‘bag’, from \textit{morral}, are just a couple of examples. For some speakers, there are also trills in word-medial position in native words: [ha\textbf{}ˈré], ‘some’ < LEL 06 5:100/el, BF 06 rec48/tx >, and [wasa\textbf{ˈr}á-ma] ‘s/he will plow’ < BFL 06 tx48 nururía/tx >.

The lateral flap, as discussed in §\ref{sec: phonological inventory}, is a sound that is auditorily both like a lateral approximant and like an alveolar (or retroflex) flap. Choguita Rarámuri back vowels following the flap condition its perception/production as a lateral.\footnote{The same conditioning environment is reported to favor the lateral production of the lateral flap in \ili{Naasioi} (East \ili{Papuan}), \ili{Barasana} (\ili{Tucanoan}) and \ili{Tucano} (\ili{Tucanoan}) \citep[][243]{ladefoged1996sounds}. \citet{lionnet1972elementos} identified the same conditioning enviroment for the lateral perception of the flap in \ili{Norogachi Rarámuri}.} While the perceptual salience of the lateral quality of the segment is stronger in word-initial position (\ref{ex: lateral allophone of the flap}a--e), there are many examples of the lateral flap being unambiguously produced as a lateral word-medially (\ref{ex: lateral allophone of the flap}f--i).

\ea\label{ex: lateral allophone of the flap}
{Lateral allophone of  the lateral flap}

    \ea[]{
     [ˈ\textbf{l}àna]   \\
     /ˈlàna/ \\
     ‘bleed’\\
     ‘sangrar’  < ROF 04 1:82/el >\\
}
        \ex[]{
        [\textbf{l}oˈká]   \\
        /loˈká/   \\
        ‘drink pinole’\\
        ‘tomar pinole’ \corpuslink{co1136[10_259-10_272].wav}{MDH co1136:10:25.9} \\
    }
            \ex[]{
            [\textbf{l}oˈwâ]    \\
            /loˈwâ/   \\
            ‘stir’\\
            ‘revolver’  < SFH 04 1:71/el >\\
        }
                \ex[]{
                [\textbf{l}aˈké]  \\
                /laˈke/   \\
                ‘carve (wood)’\\
                ‘labrar (madera)’ < BFL 04 1:111/el >\\
            }
                    \ex[]{
                    [\textbf{l}otʃ͡é] \\
                    /loˈtʃ͡e/   \\
                    ‘be hungry’\\
                    ‘tener hambre’ \corpuslink{co1235[02_242-02_269].wav}{JLG co1235:2:24.2}\\
                }
                        \ex[]{
                        [riˈʰtû\textbf{l}o]\\
                        /riˈʰtû-ri-li-o/ \\
                        freeze-\textsc{caus-pst-ep} \\
                        `They froze them.'\\
                        `Los congelaron.' < BFL 05 1:112/el >\\
                    }
                            \ex[]{
                            [waˈ\textbf{l}ó]   \\
                            /waˈló/  \\
                            ‘dry’\\
                            ‘secar’    < SFH 04 1:110/el >\\
                        }
                                \ex[]{
                                [tʃ͡iˈ\textbf{l}ó]  \\
                                /tʃ͡iˈlo/ \\
                                ‘to sizzle’ \\
                                `chisporrotear' < BFL 05 1:154/el >\\
                            }
%                        \pagebreak
                                    \ex[]{
                                    [basaˈ\textbf{l}ôwi] \\
                                    /basaˈlowa-i/\footnote{A loan from \ili{Spanish} \textit{pasear}.}\\  stroll-\textsc{impf}\\
                                    `They used to stroll.'\\
                                    `Paseaban.' < BFL 05 1:162/el >\\
                                }
    \z
\z

The flap is also optionally realized as a lateral in cluster with another consonant (after post-tonic syncope), including other rhotics (\ref{ex: pre- and post-consonantal allophone of /l/}a--b), voiceless stops (\ref{ex: pre- and post-consonantal allophone of /l/}c--d), fricatives (\ref{ex: pre- and post-consonantal allophone of /l/}e) and nasals (\ref{ex: pre- and post-consonantal allophone of /l/}f).
%\todo[inline]{There are no (30g) and (30h)}
%fixed

\ea\label{ex: pre- and post-consonantal allophone of /l/}
{Pre- and postconsonantal allophone of /l/}

%
    \ea[]{
    [ˈnâr\textbf{l}i]   \\
    /ˈnâri-li/  \\
    ask-\textsc{pst}
    `They asked.'\\
    `Preguntaron.' \corpuslink{co1136[03_133-03_150].wav}{MDH co1136:3:13.3}\\
}\label{ex: pre- and post-consonantal allophone of /l/a}
        \ex[]{
        [ˈkô\textbf{l}ri]   \\
        /ˈkôli{}-li/ \\
        visit-\textsc{pst}\\
        `They visited.'\\
        `Visitaron.' < BFL 04 1:111/el >\\
    }\label{ex: pre- and post-consonantal allophone of /l/b}
            \ex[]{
            [tʃ͡iˈkó\textbf{l}tia]\\
            /tʃ͡iˈkóli-ti-a/ \\
            tickle-\textsc{caus-prog}\\
            `They are making them tickle.'\\
            `Los están haciendo haacer cosquillas.' < AHF 05 1:146/el >\\
        }\label{ex: pre- and post-consonantal allophone of /l/c}
                \ex[]{
                [kaˈpô\textbf{l}tia] \\
                /kaˈpôli-ti-a/ \\
                spherical-\textsc{caus-prog}\\
                `They are making them spherical.'\\
                `Los están haciendo esféricos.' < AHF 05 1:149/el >\\
            }\label{ex: pre- and post-consonantal allophone of /l/d}
                    \ex[]{
                    [ˈkó\textbf{l}sia] \\
                    /ˈkóli-si-a/ \\
                    be.spicy-\textsc{mot-prog}\\
                    `It's becoming spicy.'\\
                    `Se va haciendo enchiloso.' < ROF 04 1:82/el >\\
                }\label{ex: pre- and post-consonantal allophone of /l/e}
                        \ex[]{
                        [kaˈwé\textbf{l}nale]  \\
                        /kaˈwéli-nale-/ \\
                        reject-\textsc{desid-}\\
                        `It wants to reject it.'\\
                        `Lo quiere rechazar.' < BFL 07 1:155/el >\\
                    }\label{ex: pre- and post-consonantal allophone of /l/f}
    \z
\z

In all of the examples above, the lateral flap forms a heterosyllabic consonant cluster with a preceding or following rhotic or oral stop after posttonic vowel deletion.

The following examples show the environments which favor the flap (non-lateral) variant of this phoneme.  These environments, as shown in (\ref{ex: flap allophone of /l/}), overwhelmingly involve front vowels.

\ea\label{ex: flap allophone of /l/}
{Flap allophone of /l/}

    \ea[]{
    [naˈwà\textbf{ɾ}i]  \\
    /naˈwà-li/ \\
    arrive-\textsc{pst}\\
    `They arrived.'\\
    `Llegaron.' < JHF 04 1:1/el >\\
}\label{ex: flap allophone of /l/a}
        \ex[]{
        [koˈ\textbf{ɾ}î] \\
        /koˈlî/ \\
        ‘chile pepper’\\
        `chile' \corpuslink{co1137[00_501-00_534].wav}{MDH co1137:0:50.1}\\
    }\label{ex: flap allophone of /l/b}
            \ex[]{
            [tʃ͡iˈkó\textbf{ɾ}ia] \\
            /tʃ͡iˈkóli-a/ \\
            have.itch\textsc{-prog}\\
            `They are itchy.'\\
            `Tienen comezón.' < AHF 05 1:146/el >\\
        }\label{ex: flap allophone of /l/c}
                \ex[]{
                 [kaˈpô\textbf{ɾ}itia]  \\
                 /kaˈpôli-ti-a/  \\
                 be.round-\textsc{caus-prog}\\
                 `They are making it round.'\\
                 `Lo está haciendo redondo.' < AHF 05 1:149/el >\\
            }\label{ex: flap allophone of /l/d}
                    \ex[]{
                    [ˈkô\textbf{ɾ}i]   \\
                    /ˈkôli/ \\
                    `to visit’\\
                    ‘visitar’ < BFL 05 1:111/el >\\
                }\label{ex: flap allophone of /l/e}
                        \ex[]{
                        [nakaˈwé\textbf{ɾ}i] \\
                        /nakaˈwéli/ \\
                        ‘to reject’\\
                        ‘rechazar’  < BFL 07 1:155/el >\\
                    }\label{ex: flap allophone of /l/f}
                            \ex[]{
                            [ˈmêni\textbf{ɾ}i]  \\
                            /ˈmê-nále/\\
                            win-\textsc{desid}\\
                            `They want to win.'\\
                            `Quieren ganar.'< ROF 04 1:81/el >\\
                        }\label{ex: flap allophone of /l/g}
                                \ex[]{
                                [simiˈná\textbf{ɾ}e] \\
                                /simí-ˈnále/ \\
                                go.\textsc{sg-desid}\\
                                `They want to go.'\\
                                `Quieren ir.' < SFH 05 1:86/el >\\
                            }\label{ex: flap allophone of /l/h}

    \z
\z

The flap allophones are produced slightly retroflexed for some speakers. For other speakers, however, the flap allophone is indistinguishable from the alveolar flap phoneme. For each example in (\ref{ex: flap allophone of /l/}), however, there is evidence that the flap has a related form with the lateral allophone. For instance, the past suffix /{}-li/ is realized with a flap allophone in (\ref{ex: flap allophone of /l/a}), but with a lateral allophone in (\ref{ex: pre- and post-consonantal allophone of /l/a}) above after posttonic syncope.

\subsection{Post-consonantal devoicing}
\label{subsec: post-consonantal devoicing}

Finally, oral stops are also subject to a general phonological rule: without exception, stops devoice post-consonantally. In (\ref{ex: post-consonantal voiceless oral stops}), post-tonic vowel deletion yields an environment in which the onset of the future plural is necessarily voiceless. For instance,  as shown in (\ref{ex: post-consonantal voiceless oral stopsa}), \textit{ˈnâar-}\textbf{\textit{p}}\textit{o}, but not \textit{*ˈnâar-}\textbf{\textit{b}}\textit{o}, is unattested after posttonic deletion. There are no examples in the Choguita Rarámuri corpus with a voiced/lenis allophone appearing post-consonantally.

\ea\label{ex: post-consonantal voiceless oral stops}
{Post-consonantal voiceless oral stops}

    \ea[]{
    [ˈnâarpo]\\
    /ˈnâri-\textbf{p}o/   \\
    ask-\textsc{fut.pl}\\
    `They will ask.'\\
    `Van a preguntar.'     < SFH 07 in243/in >\\
    \textit{*ˈnaar-\textbf{b}o} \\
}\label{ex: post-consonantal voiceless oral stopsa}
%\pagebreak
        \ex[]{
        [desfiˈlârpa]\\
        /desfiˈlâr-\textbf{p}a/  \\
        parade-\textsc{fut.pl} \\
        `They will paarade.'\\
        `Van a desfilar.'         < LEL 06 Nov5/el >\\
        \textit{*desfiˈlâr-\textbf{b}a}  \\
    }\label{ex: post-consonantal voiceless oral stopsb}
            \ex[]{
            [bamˈpása]\\
            /bam-ˈ\textbf{p}á-sa/ \\
            year-\textsc{inch-cond} \\
            `if they become older'\\
            `si cumple años' < SFH 06 tx12/tx >\\
            \textit{*bam-ˈ\textbf{b}a-sa} \\
        }\label{ex: post-consonantal voiceless oral stopsc}
                \ex[]{
                [samˈpá]\\
                /sam-ˈ\textbf{p}á/   \\
                be.wet-\textsc{inch}\\
                `to become wet'\\
                `mojarse'     < SFH 04 1:113/el >\\
                \textit{*sam-\textbf{b}á}   \\
            }\label{ex: post-consonantal voiceless oral stopsd}
    \z
\z

There is evidence that the onsets of the causative, future plural, and inchoative in (\ref{ex: post-consonantal voiceless oral stops}) have voiced onsets with the same roots in other morphological contexts (this is addressed in detail in \chapref{chap: verbal morphology}). The examples in (\ref{ex: phonologically conditioned devoicing}) show that the voiceless allomorphs in (\ref{ex: phonologically conditioned devoicingb}) and (\ref{ex: phonologically conditioned devoicingd}) are not lexically determined, but phonologically-conditioned. The root \textit{bami} has an inchoative suffix allomorph with a voiced onset in (\ref{ex: phonologically conditioned devoicinga}) and a voiceless allomorph in (\ref{ex: phonologically conditioned devoicingb}) after pre-tonic vowel deletion; the transitive stem \textit{rapa-na,} ‘to split, \textsc{tr}’ (\textit{‘partir}, \textsc{tr}’), has a future plural suffix allomorph with a voiced onset in (\ref{ex: phonologically conditioned devoicingc}) and a voiceless suffix allomorph in (\ref{ex: phonologically conditioned devoicingd}) after post-tonic vowel deletion. \footnote{Some lexical items have variable pronunciations with bilabial stop and bilabial nasal alternants. The positional verbal predicate for liquids /maˈna/, for instance, has alternative pronunciations with a bilabial nasal stop (\textbf{\textit{m}}\textit{aˈna}) and with a voiced bilabial oral stop (\textbf{\textit{b}}\textit{aˈna}).}

\ea\label{ex: phonologically conditioned devoicing}
{Phonologically-conditioned devoicing}

    \ea[]{
    [baˈmíbali]\\
    /baˈmí-\textbf{b}a-li/  \\
    year-\textsc{inch-pst}\\
    `They turned one year older.'\\
    `Cumplieron años.' < SFH 06 tx12/tx >\\
}\label{ex: phonologically conditioned devoicinga}
%\pagebreak
        \ex[]{
        [bamˈpása]\\
        /bam-ˈ\textbf{p}á-sa/ \\
        year-\textsc{inch-cond}\\
        `If they turn one year older.'\\
        `Si cumplen años.' < SFH 06 tx12/tx >\\
        \textit{*bam-ˈ\textbf{b}a-sa}    \\
    }\label{ex: phonologically conditioned devoicingb}
                \ex[]{
                [rapaˈnâbo]\\
                /rapa-ˈnâ-\textbf{b}o/ \\
                split-\textsc{tr-fut.pl}\\
                `They will split it (the wood).'\\
                `La van a partir (la leña).' < AHF 05 1:131/el >\\
            }\label{ex: phonologically conditioned devoicingc}
                    \ex[]{
                    [raˈpámpo]\\
                    /raˈpá-m-\textbf{p}o/ \\
                    split.\textsc{appl-tr-fut.pl}\\
                    `They will split it for her/him.'\\
                    `Se la van a partir (la leña).' < AHF 05 1:131/el >\\
                    \textit{*raˈpá-m-\textbf{b}o}\\
                }\label{ex: phonologically conditioned devoicingd}
    \z
\z

These alternations result from application of the rule in (\ref{ex: post-consonantal stop devoicing rule}), which is fed by pre- or posttonic vowel deletion.

\ea\label{ex: post-consonantal stop devoicing rule}
{Post-consonantal stop devoicing}

[+ voice] stop  →    [-voice] / C{\longrule}

\z
%\todo{use \textbackslash phonrule}


%Another allophonic process targets voiced oral bilabial stops in contexts where stress-related syncope yields surface consonant clusters. In these contexts, no voiced stops can occur in post-consonantal position, a constraint that can be formalized as *CC[+voice]. This constraint triggers a general phonological process of devoicing that applies after post-tonic vowel deletion has taken place (Caballero 2008). This rule is schematized in (8).

%Finally, and as discussed in further detail in §2.6 below, the future plural suffix exemplified in (9e-f) has suppletive allomorphs with underlying voiced and voiceless onsets whose distribution is dependent upon the lexical identity of the preceding morpheme. The forms in (9e-f) show, however, that in this case a general phonological restriction is at play in conditioning the voicing value of the suffix onset: the future plural suffix appears in the same morphological environment, but the stress shift and post-tonic vowel deletion yield the environment in which devoicing applies.

%In addition to the laryngeal contrast for stops, Choguita Rarámuri has a set of plain voiced stops ([b, ɾ]). This last series feature two properties: first, the coronal series involves a concomitant change in manner of articulation; and second, there is no voiced counterpart for the velar voiceless stop. Finally, Choguita Rarámuri voiced stops are subject to several allophonic processes, including devoicing triggered by a constraint on post-consonantal voiced stops (*CC[+voice]), and two lenition processes, namely (i) inter-vocalic spirantization and (ii) gliding in pre-consonantal position. These lenition processes are the output of the lexical phonology and co-exist with voicing alternations that are conditioned by the morphology of the language. These alternations are addressed next.

\subsection{Spirantization of voiced bilabial stops}
\label{subsec: spirantization of voiced bilabial stops}

Word-medially, voiced bilabial stops are lenited and realized as voiced bilabial fricatives or approximants intervocalically, and, for some speakers, as labiovelar semivowels pre-consonantally after further weakening. The spirantization rule is schematized in (\ref{ex: voiced bilabial stop lenitiona}) and the pre-consonantal gliding of bilabial stops is schematized in (\ref{ex: voiced bilabial stop lenitionb}). Pre-consonantal gliding is fed by post-tonic vowel deletion.\footnote{As suggested by an anonymous reviewer, the interaction between vowel deletion and pre-consonantal gliding may be argued to show that either (i) vowel deletion is phonological, despite being a variable process, or (ii) that both vowel deletion and pre-consonantal gliding are both phonetic processes. Vowel deletion interacts with other variable phonological processes (e.g., nasal place assimilation, described in §\ref{subsec: optional nasal place assimilation}).}

%\pagebreak

\ea\label{ex: voiced bilabial stop lenition}
{Voiced bilabial stop lenition}

    \ea[]{
    /b/ 	→	 [β] {\textasciitilde} [β̞] / V {\longrule} V \\
}\label{ex: voiced bilabial stop lenitiona}
        \ex[]{
         /b/	→  [w] / \_C \\
    }\label{ex: voiced bilabial stop lenitionb}
    \z
\z


The following examples show the lenition of /b/ to a voiced fricative ([β]) (\ref{ex: word-medial lenition of b}a--b) and a voiced approximant ([β̞]) (\ref{ex: word-medial lenition of b}c--d) intervocalically.

\ea\label{ex: word-medial lenition of b}
{Intervocalic lenition of /b/}

    \ea[]{
    [ziˈ\textbf{β}óò]  \\
    /si-ˈ\textbf{b}ô/  \\
    go.\textsc{pl}-\textsc{fut.pl}\\
    `Let's go!'\\
    `¡Vamos!' \corpuslink{tx1133[01_172-01_240].wav}{MFH tx1133:01:17.2}  \\
}
        \ex[]{
        [apaˈ\textbf{β}éza] \\
        /apaˈ\textbf{b}é-sa/  \\
        go.around.\textsc{pl-cond}\\
        `if they go around'\\
        `si andan' \corpuslink{tx1133[01_172-01_240].wav}{MFH tx1133:01:17.2}\\
    }
            \ex[]{
             [ˈè\textbf{β̞}əma] \\
            /ˈè\textbf{b}i-ma/  \\
            bring\textsc{-fut.sg}\\
            `S/he will bring it.'\\
            `Lo va a traer' < BFL 06 6:73/el >\\
        }
                \ex[]{
                [muˈtʃíβ̞ali] \\
                /muˈtʃ͡í-\textbf{b}a-li/ \\
                sitting.\textsc{pl-inch-pst}\\
                `They sat down.'\\
                `Se sentaron.' < BFL 06 6:73/el >\\
            }
    \z
\z

The forms in (\ref{ex: lenition of b to w}) exemplify the lenition of /b/ to a labiovelar glide ([w]) pre-consonantally.

\ea\label{ex: lenition of b to w}
{Pre-consonantal lenition of /b/}

    \ea[]{
    [ˈè\textbf{w}tiki] \\
    /ˈè\textbf{b}i-ti-ki/ \\
    bring-\textsc{caus-pst.ego}\\
    `I made him/her bring it.'\\
    `Lo hice traerlo.' < BFL 06 6:73/el >\\
}
        \ex[]{
        [muˈtʃ͡í\textbf{w}po]  \\
        /muˈtʃ͡í-\textbf{b}a-po/ \\
        be.siting.\textsc{pl-inch-fut.pl}\\
        `They will sit down.'\\
        `Se van a sentar.' < BFL 06 6:73/el > \\
    }
    \z
\z

More examples of the word-medial labiovelar semivowel allophone of /b/ are provided in (\ref{ex: word-medial allophones of /b/}).

\ea\label{ex: word-medial allophones of /b/}
{Pre-consonantal [w] allophone of /b/}

    \ea[]{
    [ˈì\textbf{w}ma] \\
    /ˈì\textbf{b}i-ma/  \\
    bring-\textsc{fut.sg}\\
    `S/he will bring it.'\\
    `Lo va a traer.' < BFL 06 6:75/el >\\
}
        \ex[]{
        [ˈì\textbf{w}ki]   \\
        /ˈì\textbf{b}i-ki/   \\
        bring-\textsc{pst.ego}\\
        `I brought it.'\\
        `Lo traje.' < BFL 06 6:75/el >\\
    }
            \ex[]{
            [aˈtʃ͡í\textbf{w}ma]  \\
            /aˈtʃ͡i-\textbf{b}a-ma/ \\
            sit.\textsc{sg.tr-inch}-\textsc{fut.sg}\\
            `S/he will sit her down.'\\
            `La va a sentar.' < BFL 06 6:146-148 >\\
        }
    \z
\z

In word-initial position, underlying /b/ may surface as voiced bilabial stops (e.g., (\ref{ex: word-initial allophones of ba})) or undergo lenition and surface spirantized (e.g., (\ref{ex: word-initial allophones of bb})). This latter pattern is the most frequently attested in the Choguita Rarámuri corpus.

\ea\label{ex: word-initial allophones of b}
{Word-initial allophones of /b/}

    \ea[]{
    [\textbf{b}iˈnèʃia]\\
    /\textbf{b}eˈnè-simi-a/\\
    learn-\textsc{mot-prs}\\
    `(they) go along learning'\\
    `van aprendiendo' \corpuslink{tx73[02_039-02_070].wav}{LEL tx73:02:03.9}\\
}\label{ex: word-initial allophones of ba}
        \ex[]{
        [\textbf{β}enˈtâantʃ͡i]\\
        /\textbf{b}enˈtânitʃ͡i/\\
        `window'\\
        `ventana'  \corpuslink{tx152[02_144-02_200].wav}{SFH tx152:02:14.4}\\
    }\label{ex: word-initial allophones of bb}
    \z
\z

For some speakers, word-initial underlying voiced bilabial stops undergo gliding (surfacing as [w]), thus neutralizing the phonemic contrast between /w/ and /b/ in word-initial position. Some examples are given in (\ref{ex: word-initial neutralization of w and b}).

\ea\label{ex: word-initial neutralization of w and b}
{Word-initial neutralization of /w/ and /b/}

    \ea[]{
    [\textbf{w}aˈkôtʃ͡i]   \\
    /\textbf{b}akôtʃ͡͡i/ \\
    `river’\\
    `río’ < SFH 04 1:17/el >\\
}\label{ex: word-initial neutralization of w and ba}
        \ex[]{
        [\textbf{w}aʔˈwí]   \\
        /\textbf{b}aʔwí/   \\
        `water’\\
        `agua’ < SFH 04 1:17/el >\\
    }\label{ex: word-initial neutralization of w and bb}
            \ex[]{
            [\textbf{w}aˈrâmi-sa]   \\
            /\textbf{b}arâmi-sa/  \\
            be.thirsty-\textsc{cond}\\
            `if s/he gets thirsty'\\
            `si tiene sed' < LEL 06 6:121/tx >\\
        }\label{ex: word-initial neutralization of w and bc}
                \ex[]{
                [\textbf{w}aʔˈwéra]  \\
                /\textbf{b}aʔwéra/ \\
                `water pot’\\
                `olla para agua’ < SFH 07 6:163-175/el >\\
            }\label{ex: word-initial neutralization of w and bd}
                    \ex[]{
                    [\textbf{w}isaˈrô] \\
                    /\textbf{b}isarô/ \\
                    ‘plant’\\
                    `planta’  < SFH 07 6:163-175/el >\\
                }\label{ex: word-initial neutralization of w and be}
                        \ex[]{
                        [\textbf{w}aˈrásiri]  \\
                        /\textbf{b}arásiri/ \\
                        ‘strong rain’\\
                        `lluvia fuerte’ < SFH 07 1:163-175/el >\\
                    }\label{ex: word-initial neutralization of w and bf}
                            \ex[]{
                            [\textbf{w}aʔˈwíwa] \\
                            /\textbf{b}aʔwíwa/ \\
                            ‘icy rain’\\
                            ‘agua-nieve’ < SFH 06 6:74/el >\\
                        }\label{ex: word-initial neutralization of w and bg}
    \z
\z

As these examples show, this neutralization is prevalent before low, central vowels, and marginally attested before high, front vowels (cf. (\ref{ex: word-initial neutralization of w and be})). There are no examples of this neutralization in word-initial position before mid, front vowels or round, back vowels. For the speakers that display this neutralization, the word-initial underlying stop is produced as either a full-fledged labiovelar semivowel or a bilabial approximant; when asked to give a careful pronunciation of these words, these speakers produce a bilabial stop.

Underlying labiovelar semivowels, on the other hand, may be neutralized and have surface realizations as voiced bilabial approximants intervocalically. As the examples in (\ref{ex: optional word-medial neutralization of b and w}) show, the neutralization is favored before /a/ (\ref{ex: optional word-medial neutralization of b and w}a--b) and /i/ (\ref{ex: optional word-medial neutralization of b and w}c--d). These are the same environments that favor velarization of /b/ in word-initial position (as shown in (\ref{ex: word-initial neutralization of w and b})).

\ea\label{ex: optional word-medial neutralization of b and w}
{Optional word-medial neutralization of /b/ and /w/}

    \ea[]{
    [roˈtʃ͡í\textbf{β̞}ari] \\
    /rotʃ͡í\textbf{w}ari/ \\
    ‘quelite’ < SFH 07 6:163-175/el >\\
}
        \ex[]{
        [sa\textbf{β̞}aˈróame] \\
        /sa\textbf{w}aróame/ \\
        ‘yellow’\\
        `amarillo’ < SFH 07 6:163-175/el >\\
    }
            \ex[]{
             [wiˈkâ\textbf{β̞}i]  \\
             /wikâ\textbf{w}i/ \\
             ‘to forget, forgive’\\
             `olvidar, perdonar’ < SFH 07 1: 183/el >\\
        }
                \ex[]{
                [reˈrò\textbf{β̞}i]\\
                /rerò\textbf{w}i/ \\
                `potato’\\
                `papa’  < FLP 06 in61/in >\\
            }
    \z
\z

This neutralization is optional and is not attested with the onsets of stressed syllables. This is shown in (\ref{ex: no occlusivization of w in stressed syllables}).

\ea\label{ex: no occlusivization of w in stressed syllables}
{No glide hardening in stressed syllables}

    \ea[]{
    [baʔˈ\textbf{w}í\textbf{β}a]  \\
    /baʔˈ\textbf{w}íwa/ \\
    ‘icy rain’\\
    ‘agua-nieve’ < SFH 06 6:74/el >\\
}
        \ex[]{
        [loˈ\textbf{w}â]   \\
        /loˈ\textbf{w}â/   \\
        ‘stir’\\
        ‘revolver’ < SFH 04 1:71/el >\\
    }
            \ex[]{
            [iˈ\textbf{w}épi]  \\
            /iˈ\textbf{w}épi/   \\
            ‘wrestle’\\
            ‘luchar’   < SFH 04 1:96/el >\\
        }
                \ex[]{
                [riˈ\textbf{w}è]   \\
                /reˈ\textbf{w}è/ \\
                ‘leave’\\
                ‘dejar’      < AFH 05 1:181/el >\\
            }
    \z
\z


% \z
\section{Phonetic reduction processes}
\label{sec: phonetic reduction}

This section addresses patterns of reduction/lenition of consonant segments that appear to be phonetic in nature, being affected by speech rate and style, not exhibiting any morphological conditioning and not exhibiting interactions with clearly phonological rules. This phonetic reduction, which is perceptually highly salient to Choguita Rarámuri speakers as a social/regional marker, does not involve contrast neutralization, a crucial difference between this process and the lenition processes operating in the lexical phonology. Several of these patterns exhibit a high degree of inter- and intra-speaker variation.

\subsection{Lenition of voiceless plosives}
\label{subsec: lenition of voiceless plosives}

%fix approximant vs. frivative for Beta
Plain (non-laryngealized) voiceless stops exhibit a large variety of surface realizations. Specifically, voiceless oral stops are gradiently realized within a continuum that ranges from a voiceless aspirated stop (if stressed) to a labiovelar glide or deletion altogether (schematically: [pʰ > p > b > β > β̞ > w > Ø]). Voicing effects in these contexts are so strong that they are clearly perceptible without any instrumental analysis. This continuum for the bilabial place of articulation is exemplified in (\ref{ex: gradient production of pa}) with instances of the realization of \textit{pa} (in boldface), a monosyllabic function word that marks clause and/or utterance boundaries (square brackets in the examples below indicate clausal boundaries in multiclausal sentences).\footnote{While these processes mostly target weak functional items like clitics and particles, voiceless stops in roots and affixes may also undergo this type of reduction, e.g. (\ref{ex: gradient voicing of velar stops}), (\ref{ex: spirantization of voiceless coronal stops in fast speech}).}

%\pagebreak

\ea\label{ex: gradient production of pa}
{Gradient production of \textit{pa}, clause final marker (\textsc{cl)}}

    \ea[]{
    \textit{aʔˈlì ˈétʃ͡i biˈláti beˈnèli tamuˈhê \textbf{ba} tʃ͡ú riˈká tiˈbúsa ˈlé \textbf{pa} ˈnà kaˈwì \textbf{βa}}\\
    \gll    aʔˈlì ˈétʃ͡i biˈlá=ti beˈnè-li tamuˈhê \textbf{ba}] tʃ͡ú riˈká tiˈbú-sa aˈlé \textbf{pa}] ˈnà kaˈwì \textbf{βa}] \\
            then \textsc{dem} indeed=1\textsc{pl.nom} learn-\textsc{pst} 1\textsc{pl.nom} {\textsc{cl}} how that take.care-\textsc{cond}   \textsc{irr} {\textsc{cl}} \textsc{prox}    land   {\textsc{cl}} \\
    \glt    ‘Then that is how we learned, how to take care of it, this earth.’\\
    \glt    ‘Entonces así aprendimos nosotros, cómo cuidarla, la tierra.’ \corpuslink{tx977[00_600-01_062].wav}{SFH tx977:00:60.0}\\
}\label{ex: gradient production of paa}
        \ex[]{
       \textit{ˈnápu koˈlìki biˈtí \textbf{ba}}\\
        \gll    ˈnápu koˈlì-ki biˈtí \textbf{ba}\\
                \textsc{sub}  around.the.side-\textsc{loc}  lie\textsc{.pl.prs} {\textsc{cl}}\\
        \glt     ‘Like the ones who lie in that other side (by the graveyard).’\\
        \glt    ‘Como los que están (acostados) de aquel lado (del panteón).’  \corpuslink{co1136[17_430-17_445].wav}{MDH co1136:17:43.0}\\
    }\label{ex: gradient production of pab}
            \ex[]{
            \textit{ˈkíti tʃ͡iˈhùnuɾam tʃ͡oʔˈmá \textbf{βa}}\\
            \gll    ˈkíti tʃ͡iˈhùnu-ɾ-ame tʃ͡oʔˈmá \textbf{βa} \\
                    because  be.disgusted-\textsc{ag}{}-\textsc{ptcp} snot {\textsc{cl}}\\
            \glt    ‘Because he was disgusted by the snot.’\\
            \glt    ‘Porque le tuvo asco al moco.’ \corpuslink{tx128[02_282-02_309].wav}{SFH tx128:02:28.2}\\
        }\label{ex: gradient production of pac}
                \ex[]{
                \textit{aʔˈlì biˈlá ko waˈbé biˈlá kiˈʔà ˈníla ra \textbf{pa} kuˈrí ke ˈtʃ͡ó me biwaˈtʃ͡êatʃ͡i ˈnà kaˈwì \textbf{β̞a}}\\
                \gll    aʔˈlì biˈlá=ko waˈbé biˈlá kiˈʔà ˈní-la ra \textbf{pa}] kuˈrí ke ˈtʃ͡ó me biwaˈtʃ͡ê-a-tʃ͡i ˈnà kaˈwì \textbf{β̞a}] \\
                        and indeed=\textsc{emph} \textsc{int} indeed long.ago \textsc{cop-rep} say.\textsc{prs} {\textsc{cl}} recently \textsc{neg} yet  almost solidify-\textsc{prs-temp} \textsc{dem} earth  {\textsc{cl}}\\
                \glt    ‘And so it was a long time before this earth was solid.’\\
                \glt    ‘Y entonces fue mucho cuando todavía no amacizaba este mundo.’ \corpuslink{tx43[11_112-11_182].wav}{SFH tx43:11:11.2}\\
            }\label{ex: gradient production of pad}
    \z
\z

As shown in these examples, the onset of this function word is realized as a voiceless stop (\ref{ex: gradient production of pa}a,d), a voiced stop (\ref{ex: gradient production of pa}a,b), a voiced fricative (\ref{ex: gradient production of pac}) or a voiced approximant (\ref{ex: gradient production of pad}), often displaying different realizations in different positions within the same utterance.

Voiceless velar stops also undergo gradient production, with variable voicing and spirantization. The forms in (\ref{ex: gradient production of ko}) exemplify this process with the emphatic enclitic \textit{=ko} (in boldface).

\ea\label{ex: gradient production of ko}
{Gradient production of \textit{ko}, topic marker (\textsc{emph})}

    \ea[]{
   \textit{aʔˈlì biˈlá \textbf{ko} miˈtíira ˈlé riˈhò \textbf{go}, tʃ͡iˈhùna tʃ͡uˈkúlam ba ˈétʃ͡i tʃ͡oˈmá ba}\\
    \gll    aʔˈlì biˈlá=\textbf{ko} miˈtí-ra aˈlé riˈhò=\textbf{go}, tʃ͡iˈhùna tʃ͡uˈkúl-ame ba ˈétʃ͡i tʃ͡oˈmá ba \\
            then indeed={\textsc{emph}} win.\textsc{pst.pass-rep} \textsc{dub} man={\textsc{emph}} be.disgusted be.curved-\textsc{ptcp} \textsc{cl} \textsc{dem} snot \textsc{cl}\\
    \glt    ‘And that is why the Rarámuri person was beaten, because he was disgusted by the snot.’\\
    \glt    ‘Y por eso le ganaron al tarahumar, porque le estuvo teniendo asco al moco.’ \corpuslink{tx128[01_262-01_312].wav}{SFH tx128:01:26.2}\\
}\label{ex: gradient production of koa}
        \ex[]{
        \textit{niˈhê ˈjêla \textbf{go} ˈpâma ˈlé koˈbísi eɾmoˈsîjo ˈka}\\
        \gll    niˈhê ˈjê-la=\textbf{go} ˈpâ-ma aˈlé koˈbísi eɾmoˈsîjo ˈka \\
                1\textsc{sg.nom} mom-\textsc{gen}={\textsc{emph}} bring-\textsc{fut.sg} \textsc{dub} pinole Hermosillo \textsc{irr}\\
        \glt    ‘My mom will bring pinole to Hermosillo.’\\
        \glt    ‘Mi mamá va a traer pinole a Hermosillo.’  \corpuslink{el444[00_578-01_017].wav}{SFH el444:00:57.8}\\
    }\label{ex: gradient production of kob}
            \ex[]{
            \textit{taˈmò \textbf{ko} ˈhê riˈgá ˈnòtʃ͡ami ˈhú ˈnà ˈwé iˈsêligam \textbf{go} ...}\\
            \gll    taˈmò=\textbf{ko} ˈhê riˈgá ˈnòtʃ͡-ame ˈhú ˈnà ˈwé iˈsêli-g-am=\textbf{go} \\
                    \textsc{1pl.nom}={\textsc{emph}} like  that  work-\textsc{ptcp} \textsc{cop} \textsc{prox} \textsc{int} be.governor.\textsc{pl-}g-\textsc{ptcp}={\textsc{emph}}\\
            \glt    ‘We, that is how we work, the governors.’\\
            \glt    ‘Nosotros los gobernadores así trabajamos.’ \corpuslink{tx816[00_000-00_070].wav}{JMF tx816:00:00.0}\\
        }\label{ex: gradient production of koc}
                \ex[]{
                \textit{maniˈké ˈlàr ba ni ˈmátimi aʔˈlá boˈsáli aˈlé   baˈʰtâli \textbf{go} ba ne}\\
                \gll    mani=ˈké ˈlàr ba ni ˈmá=timi aʔˈlá boˈsá-li aˈlé   baˈʰtáli=\textbf{go} ba ne\\
                        to.be.liquid=\textsc{cop.imp} think \textsc{cl} \textsc{emph} already=2\textsc{pl.nom} well full-\textsc{pst} \textsc{dub} corn.beer={\textsc{emph}} \textsc{cl} \textsc{emph}\\
                \glt    ‘You all got full with the corn beer, I think.’\\
                \glt   ‘Se llenaron ustedes con el tesgüino, yo creo.’    \corpuslink{tx1133[00_173-00_222].wav}{MFH tx1133:00:17.3}\\
            }\label{ex: gradient production of kod}
                    \ex[]{
                    \textit{tʃ͡iliˈká iˈnásma tʃ͡oˈná ˈhônsa \textbf{ko} ˈtʃ͡ó pa ne tʃ͡iriˈká biˈlá pe}\\
                    \gll    tʃ͡iliˈká iˈná-s-ma tʃ͡oˈná ˈhônsa=\textbf{ko} ˈtʃ͡ó pa ne tʃ͡iriˈká biˈlá pe\\
                            like.that go.\textsc{sg-mot-fut.sg} that  from={\textsc{emph}} again  \textsc{cl} \textsc{emph} like.that indeed just\\
                    \glt     ‘So that you won’t be thinking any of that.’\\
                    \glt    ‘Para que de eso no vayas pensando nada.’ \corpuslink{tx1133[01_060-01_094].wav}{MFH tx1133:01:06.0}\\
                }\label{ex: gradient production of koe}
                        \ex[]{
                       \textit{aʔˈlám riˈkátʃ͡imi aʔˈlá iˈwéami raʔaˈmâbi ˈlé \textbf{ɣo} a mi raʔaˈmâmi ˈlé paˈgótami baʔaˈlî}\\
                        \gll    aʔˈlám riˈkátʃ͡i=mi aʔˈlá iˈwé-ame raʔaˈmâ-bi aˈlé=\textbf{ɣo} a mi raʔaˈmâ-mi aˈlé paˈgótami baʔaˈlî\\
                                well like.that=\textsc{2.acc} well  strong-\textsc{ptcp}  give.advice\textsc{-irr.pl} \textsc{dub}={\textsc{emph}} \textsc{aff} there  give.advice\textsc{-irr.sg} \textsc{dub} people tomorrow\\
                        \glt    ‘Perhaps tomorrow people will give you all advice.’\\
                        \glt    ‘A lo mejor de aqui a mañana llegan gentes a darles consejos.’   \corpuslink{tx1132[00_303-00_351].wav}{MFH tx1132:00:30.3}\\
                    }\label{ex: gradient production of kof}
    \z
\z

As with bilabial stops, velar stops display a range or surface realizations, ranging from voiceless stops (\ref{ex: gradient production of ko}a,c,e), voiced stops (\ref{ex: gradient production of ko}a,b,c,d), and voiced fricatives (\ref{ex: gradient production of kof}). These effects are also attested in intervocalic position within morphologically complex words, as shown in (\ref{ex: intervocalic voicing of k}):


\ea\label{ex: intervocalic voicing of k}
{Intervocalic voicing of /k/ in morphologically complex words}

    \ea[]{
    \textit{aʔˈlì mámi baʔˈwí roˈʔèma oˈhòsa aˈnáuka biˈlé baˈrîka}\\
    \gll    aʔˈlì má=mi baʔˈwí roˈʔ-è-ma oˈhò-sa aˈnáu-\textbf{ka} biˈlé baˈrîka\\
            and already=\textsc{2sg.nom} water pour.\textsc{appl-fut.sg} dekernel-\textsc{cond}  measure-{\textsc{ger}} one cask\\
    \glt    `When you dekernel it you pour water, having measured a cask.’\\
    \glt    `Cuando lo desgranas y ya le echas agua, ya que mides una barrica.’ \corpuslink{tx60[00_272-00_305].wav}{BFL tx60:00:27.2}\\
}\label{ex: intervocalic voicing of ka}
        \ex[]{
        \textit{riˈpákina ˈku tʃ͡uˈkúli ti tʃ͡imoˈrí ko iʔˈnèga tʃ͡ú oˈlása}\\
        \gll    riˈpáki-na ˈku tʃ͡uˈkú-li ti tʃ͡imoˈrí=ko iʔˈnè-\textbf{ga} tʃ͡ú oˈlá-sa\\
                above-\textsc{abl} \textsc{rev} stand-\textsc{pst}  \textsc{dem} squirrel=\textsc{emph} watch-{\textsc{ger}} that  do\textsc{-cond}\\
        \glt    `And the squirrel stood watching when he did that to them.’\\
        \glt    `Y la ardilla lo estuvo viendo de arriba cuando les hizo eso.’ \corpuslink{tx191[03_376-03_412].wav}{BFL tx191:03:37.6}\\
    }\label{ex: intervocalic voicing of kb}
    \z
\z

These examples show how the voiceless velar stop (in the simultaneous action \textit{-ka} suffix) emerges as either voiceless (\ref{ex: intervocalic voicing of ka}) or voiced (\ref{ex: intervocalic voicing of kb}) in intervocalic position. This effect is part of a phenomenon of variable voicing of velar stops that is also attested within lexical items, as shown in (\ref{ex: gradient voicing of velar stops}):

\ea\label{ex: gradient voicing of velar stops}
{Gradient voicing of velar stops}

    \ea[]{
    [tʃ͡iˈ\textbf{k}ô-l-ame] {\textasciitilde} [tʃ͡iˈ\textbf{g}ô-l-ame]\\
    /tʃ͡iˈ\textbf{k}ô-l-ame/\\
    steal-l-\textsc{ptcp}\\
    `thief’\\
    `ladrón’  \\
}
        \ex[]{
        [paˈ\textbf{k}ó-t-ame] {\textasciitilde} [paˈ\textbf{g}ó-t-ame]\\
        /paˈ\textbf{k}ó-t-ame/\\
        wash-t-\textsc{ptcp}\\
        `people’\\
        `gente’ \corpuslink{tx5[05_054-05_095].wav}{LEL tx5:05:05.4} \corpuslink{tx_korimaka[00_104-00_127].wav}{CFH tx\_korimaka:00:10.4} \\
    }
            \ex[]{
            [\textbf{k}aˈlí]   {\textasciitilde} [\textbf{g}aˈlí]     \\
            /\textbf{k}aˈlí/\\
            `house’\\
            `casa’\\
        }
    \z
\z

Recall from §\ref{subsec: consonants} above that there is no voicing contrast at the velar place of articulation in Choguita Rarámuri. While variable voicing of velar stops is attested across Choguita Rarámuri speakers regardless of their exposure to other varieties of Rarámuri, voiced velar stops are identified by Choguita Rarámuri speakers as characteristic pronunciations of other dialects (e.g., Norogachi Rarámuri), and seem to function as a highly salient social/regional marker, though no detailed sociolinguistic study has been carried out in this area to date. Crucially, surface voiced velar stops have a more restricted distribution than surface voiced bilabial stops and coronal flaps in monomorphemic words and do not emerge in any morphological alternations.\footnote{Voiced velar stops are only marginally attested in underlying forms in some place names, as in the form \textit{basiˈ}\textbf{\textit{g}}\textit{otʃ͡i} ‘Basigóchi’, a toponym containing the root \textit{basiˈko} and the locative suffix \textit{{}-tʃ͡í} that means ‘place where \textit{basiˈko} grows’. The root \textit{basiˈ}\textbf{\textit{k}}\textit{o} is synchronically used to refer to a plant species, and it can be productively derived with the locative suffix, \textit{basiˈ}\textbf{\textit{k}}\textit{o-tʃ͡i}. This last form with the voiceless velar stop in the stem plus the locative suffix has the meaning ‘on top of the plant \textit{basikó}’.}

Finally, voiceless coronal stops also undergo lenition in fast speech, and this process is attested in intervocalic position. In contrast to bilabial and velar stops, coronal stops do not voice when undergoing lenition; instead, they undergo spirantization and a change in place of articulation. That is, in contrast to the pattern found in other morphological and phonological environments where the voiceless coronal stop alternates with a voiced coronal flap, the lenis counterpart of [t] in these fast speech contexts is a voiceless inter-dental fricative. This effect is exemplified in (\ref{ex: spirantization of voiceless coronal stops in fast speech}).

%\pagebreak

\ea\label{ex: spirantization of voiceless coronal stops in fast speech}
{Spirantization of voiceless coronal stops in fast speech}

    \ea[]{
    [paˈɣó\textbf{θ}am]\\
    /paˈgó\textbf{t}-ame/\\
    baptize-\textsc{ptcp}\\
    `people'\\
    `gente' \corpuslink{tx1132[00_303-00_351].wav}{MFH tx1132:00:30.3}\\
}
        \ex[]{
        [maˈ\textbf{θ}ê\textbf{θ}ɾ̥a]\\
        /maˈ\textbf{t}ê\textbf{t}ara/\\
        `thank you'\\
        `gracias'  \corpuslink{tx1133[02_388-02_406].wav}{MFH tx1133:02:38.8}\\
    }
    \z
\z

Preliminary examination of phonetic data reveals these gradient realizations in fast speech reflect a lenition process in which productions on the lenis end of the continuum (fricatives and approximants) tend to be produced in utterance-final position, while productions on the fortis end (voiceless and voiced stops) tend to be produced utterance-medially, a factor that may suggest that these alternations are sensitive to phrasal phonological effects.\footnote{Lenition results in the neutralization of the voicing contrast at the bilabial place of articulation, with the voiced bilabial stop also displaying a gradient realization ranging from a stop proper, to a fricative, an approximant, a labiovelar glide or deletion ([b{\textasciitilde}β{\textasciitilde}β̞ ̞{\textasciitilde}w{\textasciitilde}Ø]).}{}

While the precise role of post-lexical phonology in conditioning lenition in these environments is still under investigation, there is a clear effect of a constraint that precludes [+continuant] consonants ([β{\textasciitilde}$β̞̞${\textasciitilde}w{\textasciitilde}Ø]) in post-consonantal position, a constraint that applies across word boundaries, e.g., \textit{oˈwaam} \textbf{\textit{pa}}, \textit{tʃ͡uˈkú\-lam} \textbf{\textit{b}}\textit{a}, \textit{iˈsêligam} \textbf{\textit{g}}\textit{o}. In the corpus data examined so far, there are no [-continuant] consonants in these environments, suggesting that a constraint like *CC[+con\-tin\-u\-ant] is at play. This reduction process does involve contrast neutralization, a crucial difference between this process and the lenition processes operating in the lexical phonology.

\subsection{Depalatalization and deaffrication of alveopalatal affricates}
\label{subsec: depalatalization and deaffrication of alveopalatal affricates}

Alveopalatal affricates can optionally depalatalize before low, central vowels and can be produced as alveolar affricates, as schematized in the rule in (\ref{ex: depalatalization of alveopalatal affricates}). This allophonic variation is exemplified in (\ref{ex: optional depalatalization of alveopalatal affricate}), where alveopalatal affricates are optionally depalatalized root-internally (\ref{ex: optional depalatalization of alveopalatal affricate}a--c) or as onsets of suffixes (\ref{ex: optional depalatalization of alveopalatal affricate}d--f). Alternative pronunciations with the alveo-palatal affricate are also shown in (\ref{ex: optional depalatalization of alveopalatal affricate}).

\ea\label{ex: depalatalization of alveopalatal affricates}
{Depalatalization of alveopalatal affricates}

tʃ͡  	→	  ts / {\longrule} a

\z

\ea\label{ex: optional depalatalization of alveopalatal affricate}
{Optional depalatalization of alveopalatal affricate}

    \ea[]{
    [aˈkâ\textbf{ts}ala]  {\textasciitilde} [aˈkâ\textbf{tʃ͡}ala] \\
    /aˈkâ\textbf{tʃ͡}a-la/\\
    paternal.grandmother-\textsc{poss}\\
    `their paternal grandmother'\\
    `su abuela paterna'   < FLP 07 in243/tx >  \\
}
        \ex[]{
        [aˈ\textbf{ts}âsa]   {\textasciitilde} [aˈ\textbf{tʃ͡}âsa]  \\
        /aˈ\textbf{tʃ͡}â-sa/\\
        to.sit.\textsc{sg.tr-cond}\\
        `if they sit it down'\\
        `si lo sienta’     < SFH 04 1:38/el >\\
    }
            \ex[]{
            [ˈnò\textbf{ts}ari]  {\textasciitilde} [ˈnò\textbf{tʃ͡}ari]  \\
            /ˈnò\textbf{tʃ͡}ari/\\
            `work’\\
            `trabajo’   < AADB(292)/el >\\
        }
                \ex[]{
                [kuˈsú-\textbf{ts}ani]  {\textasciitilde} [kuˈsú-\textbf{tʃ͡}ani]  \\
                /kuˈsú-\textbf{tʃ͡}ani/\\
                make.animal.noise-\textsc{ev} \\
                `It sounds like an animal is making noise.'\\
                `Suena que canta (un animal).'   < SFH 05 1:81/el >\\
            }
                    \ex[]{
                    [oˈsì\textbf{ts}ana]  {\textasciitilde} [oˈsì\textbf{tʃ͡}ana]  \\
                    /oˈsì-\textbf{tʃ͡}an-a/\\
                    write.read-\textsc{ev-prog}\\
                    `It sounds like they're writing/reading.'\\
                    `Suena que están escribiendo/leyendo.'  < SFH 05 1:88/el >\\
                }
                        \ex[]{
                        [tʃ͡iˈwá\textbf{ts}a]  {\textasciitilde} [tʃ͡iˈwá\textbf{tʃ͡}a]  \\
                        to.tear-\textsc{tr.pl}\\
                        `to tear something many times'\\
                        `rasgar muchas veces'     < ROF 04 1:59/el >\\
                    }
    \z
\z

In fast speech, alveopalatal affricates may depalatalize and deaffricate in a high frequency word combination: the distal demonstrative \textit{ˈétʃ͡i} and a following adjective, are pronounced as a single word within nominal phrases, with stress in the second syllable. This involves the deletion of an underlying high vowel, as shown in the examples from text provided in (\ref{ex: depalatalization and deaffrication of ch}).

\ea\label{ex: depalatalization and deaffrication of ch}
{Depalatalization and deaffrication of \textit{tʃ͡}}

    \ea[]{
    e\textbf{s}ˈtá   ˈtʃ͡êlami   ko   ba\\
    \gll    ˈétʃ͡i ˈtá oˈtʃ͡êlam=ko ba\\
            \textsc{dist} \textsc{det}   old.man=\textsc{emph} \textsc{cl}\\
    \glt    `that old man'\\
    \glt    `ese viejo'  < LEL 06 6:141-162/tx >\\
}
        \ex[]{
        ˈnè ko aniˈmê oˈlá ˈhê ˈnà e\textbf{s} ˈtá raʔˈìtʃ͡iri\\
        \gll    ˈnè=ko ani-ˈmêa oˈlá ˈhê ˈnà ˈe\textbf{tʃ}i  ˈtá raʔˈìtʃ͡i-ri\\
                1\textsc{sg.nom=emph} say-\textsc{fut.sg} \textsc{cer} \textsc{dem} \textsc{prox} \textsc{dist} \textsc{det} speak-\textsc{nmlz}\\
        \glt    ‘I am going to say these words.’\\
        \glt    ‘Yo voy a decir esta plática.’   < BFL 07 frog story\_2/tx >\\
    }
            \ex[]{
            we   kaˈní-lam   ˈtʃ͡ó  ˈníra    ˈlé  é\textbf{s}  ˈkútʃ͡i oˈkwâka\\
            \gll    we kaˈní-l-ame ˈtʃ͡ó ˈní-ra aˈlé ˈé\textbf{tʃ}i ˈkútʃ͡i oˈkwâ-ka\\
                    \textsc{int}  love-r-\textsc{ptcp}  also  \textsc{cop-pot} \textsc{dub} \textsc{dem} small  two-\textsc{coll}\\
            \glt    ‘They loved each other, those two, I think.’\\
            \glt    ‘Se querían mucho también, yo creo, ellos dos.’  < BFL 07 frog story\_2/tx >\\
        }
    \z
\z

The depalatalization process exemplified in (\ref{ex: depalatalization and deaffrication of ch}a--b) can be analyzed as a reinterpretation of a phonetically ambiguous form, a consonant cluster with an alveo\-pal\-atal affricate followed by an alveolar voiceless stop. This consonant sequence would arise across word boundaries in these forms in an intermediate representation after posttonic deletion (\{ˈetʃ͡ ˈtá\}). The same analysis can be extended to other contexts of depalatalization of affricates: in (\ref{ex: depalatalization and deaffrication of ch 2}) below, the alveolpalatal affricate undergoes non-local anticipatory assimilation with the following alveolar stop. The posttonic front, high vowel that intervenes between the alveopalatal affricate and the alveolar stop is extra short and can be deleted altogether in fast speech.

\ea\label{ex: depalatalization and deaffrication of ch 2}
{Depalatalization and deaffrication of \textit{tʃ͡}}

    ma     ˈkárua     ˈkú\textbf{t}i   toˈlí\\
    \gll    ma    ˈkáru-a    ˈkú\textbf{tʃ}i  toˈlí\\
            already cackle-\textsc{prog}  small  chicken  \\
    \glt    ‘The chickens already cackled.’\\
    \glt    ‘Ya cacaraquearon las gallinas.’   < SFH 07 el170 (12:25)/el >
\z

% Note - the summary is for more than laid out in the chapter - more appropriate for an introduction section that gives an overview of the sound system

%\section{Summary}
%\label{sec: summary chapter}

%The phonology of Choguita Rarámuri is characterized by a small phonological inventory and a simple syllable structure in underlying forms (with no onset ellaboration and only one possible coda, glottal stop), but a great amount of variation and reduction processes of segments. Stop alternations are perhaps the hallmark example of this intra-linguistic variation, by displaying phonologically, morphologically and lexically conditioned variation (as well as free, speech-rate dependent gradient voicing), alternations which are addressed in more detail in \chapref{chap: voicing alternations}. Another source of complexity in the phonological system is found in the stress system and the stress-dependent phenomena that yields gradient, optional surface patterns (such as three distinct patterns of unstressed vowel reduction); stress-conditioned vowel deletion, on the other hand, produces derived consonant clusters and geminates, giving Choguita Rarámuri a moderate syllable structure on the surface level, which contrasts with the simple syllable structure at the underlying representation level.

%In addition, the prosodic level of Choguita Rarámuri presents several analytical and typological challenges. First, stress, whether lexically or morphologically conditioned, is restricted to emerge in the first three syllables of the word, instantiating a system that has only been documented in few other languages of the world (including closely related Mountain \ili{Guarijío} within the \ili{Taracahitan} branch of \ili{Uto-Aztecan}). Second, glottal stops are restricted to emerge in an initial disyllabic window. Third, while tone has been described as part of the lexical representation of morphemes in several Southern UA languages, no variety of Rarámuri had been described as having a tonal contrast. As shown here, and building on previous publications on the language (\citealt{caballero2008choguita}, \citealt{caballero2015tone}), there is robust evidence for positing a three-way tonal contrast that is acoustically distinct from the stress-accent system. The prosodic complexity of Choguita Rarámuri extends to its intonation, with both tonal and non-tonal devices that exhibit complex tone-intonation interactions, as well as a great deal of inter-speaker variation. Complex interactions within the prosodic domain of the language are addressed in more detail in \chapref{chap: prosody}.
