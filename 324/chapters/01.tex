\chapter{Introduction}
\label{chap: introduction}

\section{Linguistic profile of Choguita Rarámuri}

Despite the relevance of the {Uto-Aztecan} language family in terms of its geographical extension, number of languages, number of speakers and its descriptive tradition (spanning over four centuries), there are still many important gaps in our knowledge of this language family. This book provides the first comprehensive grammatical description of Choguita Rarámuri, a {Uto-Aztecan} language spoken in the Sierra Tarahumara, a mountainous range in the northern Mexican state of Chihuahua belonging to the Sierra Madre Occidental. The number of speakers of all Rarámuri varieties in the Sierra Tarahumara and diaspora communities in Northern Mexico is estimated to be between 85,000 \citep{catalogo2021inali} to 100,000 \citep{embriz2012mexico,merrill2014ralamuli}. Choguita Rarámuri is the native language of approximately 1,000 people, most of whom use the language in their daily interactions, though continued political violence in the area has led to increased displacement of the Rarámuri people and consequent accelerated language attrition. As documented since the first contact with European settlers, the Rarámuri people continue to resist the encroachment of outsiders on their land and violations of their autonomy as a sovereign nation.

%adapt according to last edits
In the remainder of this introductory Chapter, I provide a typological profile of Choguita Rarámuri (§\ref{subsec: Choguita Rarámuri in typological context}), I discuss the genetic relationship between Choguita Rarámuri and other language varieties (§\ref{subsec: varieties and genetic relationships}), and I survey previous work on the language (§\ref{subsec: previous work}). The following sections focus on the geographical (§\ref{sec: geographic location and physical environment}), historical and socio-political (§\ref{sec: social and historical context}) contexts Choguita Rarámuri is spoken within. The history of contact with Europeans is addressed in §\ref{subsec: history of contact with Europeans}, §\ref{subsec: linguistic ecology and socioolitical context} addresses the linguistic ecology and the current socio-political context of the area, while §\ref{subsec: literacy} considers the development of literacy and bilingual education projects sponsored by the Mexican government.

This chapter concludes in §\ref{sec: this grammar} with a meta-description of this grammar, which includes a description of how this project developed (§\ref{subsec: project development}), theoretical assumptions made in this work (§\ref{subsec: theoretical assumptions}), the data it draws upon and methodologies used to gather it (§\ref{subsec: data, corpora and methodology}), the Choguita Rarámuri language experts who contributed to the data and analysis (§\ref{subsec: language experts and collaborators}), and conventions employed in the presentation of examples (§\ref{subsec: representation of examples}). Finally, this chapter provides an overview of the topics covered in this grammar and the content covered in each chapter (§\ref{sec: overview of the grammar}).

\subsection{Choguita Rarámuri in typological context}
\label{subsec: Choguita Rarámuri in typological context}

% include discussion of grammatical tone, tone-intonation interactions, morphologically conditioned truncation; also: complex predicate constructions
%%develop] –  see J. Hill analogous section on \ili{Cupeño}
%%Head marking, highly synthetic, agglutinative

Choguita Rarámuri is a highly synthetic, agglutinating language with a complex morphological system. It displays many of the recurrent structural features documented across \ili{Uto-Aztecan}, including a predominance of suffixation, head-marking (as defined in \citet{nichols1986head}), and patterns of noun-incorporation and compounding \citep{sapir1921introduction,whorf1935comparative,haugen2008morphology}. Other recurrent Uto-Az\-tec\-an features present in Choguita Rarámuri include a complex word prosodic system, a wide range of morphologically-conditioned phonological processes, and exuberant derivational morphological marking, all of which are grammatical phenomena that are of great typological and theoretical interest. Choguita Rarámuri is also of great comparative/historical importance: while several analytical works of \ili{Uto-Aztecan} languages of Northern Mexico have been produced in the last years (\citealt{valenzuela2006structure}, \citealt{garcia2014clause}, \citealt{reyes2014fonologia}, \citealt{moralesmoreno2016rochecahi}, \citealt{villalpando2019grammatical}, \textit{inter alia}), many varieties still lack comprehensive linguistic description and documentation.

In traditional morphological typology, agglutinative morphologies are placed within a scale of agglutination-flexion, with isolating languages on one end of the spectrum and introflexive (or non-linear) languages at the other (isolating > agglutinative > flexive > non-concatenative (or introflexive)). This single scalar hierarchy results from conflating the parameters of phonological fusion, the degree to which individual exponents are phonologically fused to their host (isolating > concatenative > non-concatenative), and flexivity, the degree to which individual exponents exhibit lexically-conditioned variance (suppletive allomorphy) \parencite{bickel2007inflectional}. Agglutinative languages thus involve concatenative, non-flexive morphological patterns. \ili{Uto-Aztecan} languages have been described as prototypically agglutinative, with complex verbal morphological systems, a high degree of synthesis, a low degree of phonological cohesion between exponents, and a low degree of cumulation in morphological exponence \citep[see e.g.,][158]{langacker1977uto}. Choguita Rarámuri morphology, which is also highly synthetic and almost exclusively suffixing, displays the following agglutinative-like properties (\ref{ex: agglutinative-like properties of the Choguita Rarámuri verb}):

%may want to add in a footnote what other criteria are assumed to pertain to agglutinating languages
\ea\label{ex: agglutinative-like properties of the Choguita Rarámuri verb}
{Agglutinative-like properties of Choguita Rarámuri}

\begin{itemize}
    \item Mostly concatenative, separative exponence \\
    \item Limited flexive exponence \\
    \item   Zero exponence \\
    \item   Moderate syncretism\\
    \item   Large derivational paradigms \\
    \item    Widespread multiple exponence (through multiple affixation)\\
    \item   Widespread optional marking \\
\end{itemize}

\z

%expand here with reference to Caballero & Kapatsinski to appear on morphological typology and linguistic cognition
While Choguita Rarámuri shares several morphological and morpho-phonol\-o-gic\-al properties and phenomena with other morphologically complex languages that have been characterized as agglutinative, it also crucially departs from the canonical ``agglutinative" type in that it has less transparent morpheme boundaries, due to a fair amount of phonological cohesion between exponents closer to the stem, a pattern more frequently attested in flexive morphological patterns \citep{caballero2008choguita,kapatsinski2021agglutinative}. Departures from prototypical agglutinative characteristics are documented elsewhere in {Uto-Aztecan} (e.g., \ili{Cupeño}, a \ili{Takic} language; \citealt{hill2005grammar}).

%expand and insert references to other \ili{Uto-Aztecan} languages.
Description of particular word prosodic systems in {Uto-Aztecan} languages has traditionally received a great deal of attention in the arealist literature, as well as in the typological and theoretical literature, given their complexity. \ili{Uto-Aztecan} languages have been documented to have a wide variety of stress-accent systems. Some recurrent features of these systems include presence of lexical stress, iterative stress assignment, left-edge alignment, and window restrictions \citep{munro1977towards}. One of the most typologically unusual features of the word-prosodic system of Choguita Rarámuri is its initial three-syllable stress window. Stress is assigned within the first three syllables of the word, and there are alternations that maintain this ternarity, such as compounding and multiple affixation \citep{caballero2008choguita,caballero2011morphologically}. This kind of stress system is extremely uncommon cross-linguistically, described in the literature in only four other languages of the world \citep{kager2012stress}, and predicted to be unattested by some factorial typologies \citep{elenbaas1999ternary}.

% Detailed knowledge of this system is not only relevant for developing theories of stress that shape their models in terms of attested and unattested patterns cross-linguistically, but also contributes important comparative evidence to test diachronic hypotheses (e.g., Munro 1977, who reconstructed second syllable stress for \ili{Proto-Uto-Aztecan}).

%Choguita Rarámuri features a highly complex stress system with elaborate morphological conditioning governing its distribution and an initial three-syllable stress window, a pattern that has been documented in only four other languages of the world outside this language family \citep{caballero2011morphologically}. In addition to stress, Choguita Rarámuri possesses a tone system that contrasts HL, H and L tones in stressed syllables. As part of an ongoing project that investigates the structural and phonetic properties of prosodic phenomena in this language, I have examined its phonological distribution, its morphological role, and its interaction with the stress system of the language.

The stress system of Choguita Rarámuri is also a topic of particular interest given its morphological conditioning. Although stress is lexically governed in some morphemes, there is also emergent default stress assignment. As described for other {Uto-Aztecan} languages (e.g. \ili{Cupeño}, \citet{hill1968stress,alderete2001dominance}), there is a contrast between underlyingly stressed and unstressed roots in Choguita Rarámuri. Affixes, on the other hand, are either stress-shifting or stress-neutral, meaning that they can perturb the root’s stress or be neutral regarding stress assignment, respectively. The interaction between roots and affixes of different prosodic types yield complex interactions in verbal paradigms: in words containing no underlyingly stressed roots or stress-shifting suffixes, stress falls by default in the second syllable of the stem; in words containing an underlyingly unstressed root and a stress-shifting affix, stress falls by default in the third syllable of the word (the immediately adjacent stress-shifting suffix with disyllabic roots). This is exemplified below in \tabref{tab:stress} (stressed syllables are in boldface).

\begin{table}
\caption{Stress patterns of morphologically complex verbs}
\label{tab:stress}

\begin{tabularx}{\textwidth}{QQQl}
\lsptoprule
\textbf{Stem} & \textbf{Stress-neutral} & \textbf{Stress-shifting} & \\
 & \textbf{Past \textit{-li}} & \textbf{Conditional \textit{-sa}} & \\
 \midrule
beˈnè `learn’ & be\textbf{ˈnè}-li & be\textbf{ˈnè}-sa & \textbf{Stressed roots} \\
baˈhî `drink' & ba\textbf{ˈhî}-li & ba\textbf{ˈhî}-sa & \\
tʃ͡aˈpí `grab' & tʃ͡a\textbf{ˈpí}-li & tʃ͡api-\textbf{ˈsâ} & \textbf{Unstressed roots} \\
saˈkí `roast corn' & sa\textbf{ˈkí}-li & saki-\textbf{ˈsâ} & \\
\lspbottomrule
\end{tabularx}
\end{table}
\hspace{3cm}

These morphological restrictions on stress assignment interact in complex ways with the trisyllabic stress window restriction in compound and other morphological constructions (yielding stress shifts and truncation of nominal roots), and with two subsystems of valency encoded through root allomorphy marked through vocalic alternations and fixed stress for applicative derivation \citep{caballero2008choguita}.

In addition to stress, the Choguita Rarámuri word prosodic system involves tone. While the development of tonal contrasts has been documented for a number of {Uto-Aztecan} languages (including \ili{Northern Tepehuan} (\ili{Tepiman}; \citet{woo1970tone}), \ili{Hopi} \citep{manaster1986genesis}, \ili{Huichol} (\ili{Corachol}; \citet{grimes1959huichol}), \ili{Cahitan} (\ili{Yaqui} \citep{demers1999prominence} and \ili{Mayo} \citep{hagberg1989floating}); and \ili{Balsas Nahuatl} (\ili{Aztecan}; \citet{guion2010word}), no variety of Rarámuri, to the best of my knowledge, had been previously described as featuring a tonal contrast. Except for \ili{Balsas Nahuatl}, tonal {Uto-Aztecan} languages are located in the Southwest, a linguistic area where tone has also developed in other language families \citep{caballero2020oxford}.

The tone system of Choguita Rarámuri has a restricted distribution, as lexical tone contrasts are exclusively realized on surface stressed syllables (i.e., there is a single lexical tone per prosodic word). This restricted distribution of tone in Choguita Rarámuri is a characteristic shared by all other tonal {Uto-Aztecan} languages, all of which have ``hybrid" word prosodic systems (featuring both stress and tone). The Choguita Rarámuri lexical tone inventory features a three-way contrast (/HL/, /L/ and /H/) in stressed syllables \citep{caballero2015tone}, which contrasts with the binary tone systems documented in neighboring \ili{Cahitan} languages (\ili{Mayo} and \ili{Yaqui}). Stress and tone in Choguita Rarámuri are not only phonologically independent systems, but they are also acoustically distinct: duration and intensity are correlates of stress, whereas fundamental frequency (f0) is the primary correlate of tone (\citealt{caballero2015tone}).

Tone also plays a role in the morphological system of Choguita Rarámuri: tone alone may be the exponent of morphological information, it may be associated with specific affixes as a morphologically-conditioned effect, or it may have a morphomic distribution in verbal paradigms \citep{caballero2021grammatical}. This type of grammatical tone phenomena has not yet been documented in other tonal {Uto-Aztecan} languages.

In addition to encoding lexical and morphological contrasts, f0 is also deployed in the Choguita Rarámuri intonation system. Declarative sentences are characterized by a high boundary tone, which stands in contrast to the general cross-linguistic tendency for low boundary tones in these types of sentences \citep{jun2014prosodic}. Analysis of acoustic data reveals that lexical tones may be enhanced in utterance-final position and that both lexical and grammatical tones take precedence over phrase-level tones if these tones conflict (\cite{caballero2014tone}, \cite{aguilar2015multi}, \cite{garellek2015lexical}). In addition to f0, Choguita Rarámuri implements intonational contrasts through duration and non-modal phonation \citep{aguilar2015multi,kubuzono2020raramuri}. Both tone and intonation are under-studied for {Uto-Aztecan} languages, except for recent work that addresses \ili{Nahuatl} language varieties \citep{guion2010word,patino2014intonation,aguilar2020phonology}.

%In terms of its morphology, Choguita Rarámuri presents affix order and exponence patterns that are highly theoretically relevant. Affix ordering and exponence are topics that lie at the core of morphological theory and form ideal testing grounds for determining the nature of the interface between different components of grammar. Affix order has been claimed to be driven by a variety of factors, but the precise nature and limits of their interaction remain a topic for further research. Furthermore, growing documentation of lesser known languages reveals patterns that challenge previous assumptions of possible affix order systems (e.g., “free” affix order in \ili{Kiranti} (\ili{Sino-Tibetan}; Bickel et al. 2007) and \ili{Totonacan} languages (Beck 2007)). My dissertation (Caballero 2008) and recently published paper (Caballero 2010) document a new case of free affix order, where alternative orders are determined by scope, templatic constraints, phonological subcategorization and phonological conditions on stem shape.

%Choguita Rarámuri also displays a pervasive tendency towards Multiple Exponence (ME) of morphological categories. While much attention has been paid in the morphological literature to the nature and patterns of morphological blocking (Anderson 1982, Andrews 1990), patterns of morphological redundancy have been generally neglected. Despite its critical theoretical ramifications and increasing number of documented cases, there is still no clear sense as to what is the possible range of variation in ME patterns cross-linguistically. The Rarámuri case offers an important opportunity to explore the properties of this morphological phenomenon. So far, I have been able to establish that ME in Choguita Rarámuri: i) involves derivational information (contra suggestions that ME is exclusively displayed by markers of inflectional categories (Matthews 1972, Stump 2001)); and ii) that it involves categories in specific areas of the layered structure of the verb with characteristic morpho-prosodic properties which make them difficult to parse and prone to be reanalyzed as part of the stem (Caballero 2008, to appear a, to appear b). My interest in this topic has led me to investigate the typological and theoretical implications of this phenomenon (Caballero and Harris to appear; Caballero and Inkelas to appear).

In terms of its morphological system, Choguita Rarámuri exhibits a case of free affix order where alternative orders are determined by semantic scope, templatic constraints, phonological subcategorization and phonological conditions on stem shape \parencite{caballero2010scope}. Choguita Rarámuri also exhibits complex patterns of multiple (extended) exponence (ME), a one-to-many mapping between morphological categories and their formal expression. The Choguita Rarámuri case offers an important opportunity to explore the properties of this morphological phenomenon, given that ME in Choguita Rarámuri: i) involves derivational information (\textit{contra} suggestions that ME is exclusively displayed by markers of inflectional categories (\citealt{matthews1972inflectional}, \citealt{stump2001inflectional})); and ii) that it involves categories in specific areas of the layered structure of the verb with characteristic morpho-prosodic properties which make them difficult to parse and prone to be reanalyzed as part of the stem (\citealt{caballero2008choguita}, \citealt{caballero2011morphologically}). Choguita Rarámuri also provides a relevant testing ground for investigating the potential functional role complex morpho-phonological patterns in morphologically complex languages may have \citep{caballero2015perceptual,kapatsinski2021agglutinative}.

%I have investigated the perceptual functionality of ME in Choguita Rarámuri through a perception experiment with Choguita Rarámuri speakers (\citealt{caballero2015perceptual}). Our results show a significant effect of adding a redundant marker: the redundant (ME) pattern helped with recognition of the cued meaning when this meaning is unexpected from context, but it was detrimental when the meaning was expected from context. We interpreted these results as evidence of a mechanism of pragmatic inference at play in morphological processing, whereby listeners expect the speaker to produce as little as possible while successfully transmitting the intended information.

Finally, Choguita Rarámuri is also typologically relevant given a rich set of valency-changing morphology, and a complex system of case marking and lexical distinctions to refer to spatial and topographic properties of the landscape, also documented for closely-related \ili{Guarijío} \parencite{miller1996guarijio}. These and other aspects of the structure of Choguita Rarámuri are addressed in this reference grammar.

\subsection{Rarámuri language varieties and genetic/genealogical relationships}
\label{subsec: varieties and genetic relationships}

\subsubsection{Dialect variation}
\label{subsubsec: dialect variation}

%Rarámuri language varieties are spoken in a dialect continuum.
Rarámuri dialect diversity has not yet been systematically investigated, but a local-government sponsored dialect survey carried out two decades ago yielded a classification with five dialect areas: North (\textit{Norte}), South (\textit{Sur}), Central (\textit{Centro}), West (\textit{Oeste}) and Highland (\textit{Cumbres}) (\cite{valinas2001lengua}; see also \citet{catalogo2021inali}).\footnote{\citet{merrill2014ralamuli} translate the term \textit{Cumbres} as ``Interior", noting this dialect area does not necessarily correspond with areas that are at a higher altitude, which is implied by the term \textit{Cumbre}.} This classification was established on the basis of the assessment of lexical, phonological and syntactic variation.\footnote{Some of the phonological parameters include: use of word-initial [g], [k] or zero; initial syllable truncation; word final vowel deletion; height neutralization of /e/; and pre-aspiration of voiceless stops \citep[122]{valinas2001lengua}.} These five dialect areas roughly correspond to the dialect areas proposed by the SIL International published in \textit{Ethnologue}, labeled Central, Western, Northern, Southeastern and Southwestern. There is a high degree of overlap between these classifications, but there is no consensus about the precise boundaries of each dialect: the Ethnologue’s Central Tarahumara dialect, for instance, corresponds to an area occupied by two dialects in the local government survey, Central Tarahumara and Northern Tarahumara \citep{valinas2001lengua}. The Mexican National Institute for Indigenous Languages (INALI)\footnote{\textit{Instituto Nacional de Lenguas Indígenas}.} adopts the classification proposed in \citet{valinas2001lengua}. \tabref{tab:dialect} presents the two classifications side by side. The classification adopted by INALI also provides endonyms for each variety \citep{catalogo2021inali}.

%\break

\begin{table}
\caption{Rarámuri dialect areas}
\label{tab:dialect}

\begin{tabularx}{\textwidth}{Xl}
\lsptoprule
\textbf{Ethnologue} & \textbf{INALI}\\
\midrule
{Western Tarahumara} [tac] & Rarómari Raicha (\textit{Oeste})\\
Central Tarahumara [tar] & Ralámuli Raicha (\textit{Centro})\\
{Southeastern Tarahumara} [tcu] & Ralámuli Raicha (\textit{Cumbres})\\
Northern Tarahumara [thh] & Ralámuli Raicha (\textit{Norte})\\
{Southwestern Tarahumara} [twr] & Rarámari Raicha (\textit{Sur})\\
\lspbottomrule
\end{tabularx}
\end{table}
%\hspace{3cm}

An uncontroversial main distinction exists between two major sets of dialects, Tarahumara de la Alta (Rarámuri) and Tarahumara de la Baja (Rarómari), which may include mutually unintelligible varieties. Choguita Rarámuri is part of the Alta Tarahumara dialect continuum within a ``Central" dialect area within the Ethnologue’s classification ([tar]; Eberhard et al. 2021) and is located in a transitional area between the Central Tarahumara and Northern Tarahumara dialects in \citep{valinas2001lengua}.

Speakers are aware of dialect differences, but view all Rarámuri varieties as a single language, different from the neighboring languages. The location of several modern Rarámuri communities and closely related \ili{Guarijío} (Warihío) is shown in \figref{fig: neighboring lgs}. Choguita is located 35km northwest of Norogachi. More comprehensive assessments are necessary to determine the number and location of all varieties of Rarámuri, degrees of intelligibility amongst these, and number of speakers per variety.

\begin{figure}
% \includegraphics[width=\textwidth]{figures/Introduction-neighboring-lgs.png}
\includegraphics[width=\textwidth]{figures/modernregion.pdf}
\caption{
\label{fig: neighboring lgs}
Modern Rarámuri communities and neighboring  {Guarijío} ({Warihio}) territory (map adapted from \citealt[][232]{merrill2014ralamuli}).}
% Alexis Rojas, CC BY-SA 3.0 <https://creativecommons.org/licenses/by-sa/3.0>, via Wikimedia Commons
\end{figure}\il{Guarijío}\il{Warihio}

\subsubsection{Alternative names}
\label{subsubsec: alternative names}

%%needs to be expanded
\textit{Rarámuri} [raˈɽámuɽi] is the name applied by the Rarámuri people to their own language, as well as the Rarámuri people, land and culture. The phrase [raˈɽámuɽi raˈʔìtʃ͡a] (`Rarámuri language') is used in discourse when disambiguating reference to the language vs. other possible meanings. When referring to people, Rarámuri is the term often used to refer to members of the Rarámuri nation, as opposed to \textit{mestizo} (non-indigenous Mexican) and other non-Indigenous people, but this term has different meanings that are dependent on context. Specifically, \citet{merrill2001identidad} identifies four levels of self denomination of the term ``Rarámuri'': (i) all human beings; (ii) indigenous people (vs. \textit{mestizo} and people of European descent); (iii) Rarámuri people (vs. other indigenous people); and (iv) Rarámuri men (vs. Rarámuri women) (\citeyear[88]{merrill2001identidad}).

Rarámuri has been mostly known as \textit{Tarahumara} in mass media and previous descriptions and depictions of the language (e.g., the ISO code of the language is [tar]). The term \textit{Tarahumara} was first used in the seventeenth century in the correspondence of Catholic missionaries and the first published works about the language, Tomás de Guadalajara’s 1683 grammar and Matthäus Steffel’s 1791 dictionary. The term \textit{Rarámuri} was not used in published materials until Miguel de Tellechea’s 1826 \textit{Compendio gramatical para la inteligencia del idioma tarahumar} (where the spelling used for the language was \textit{rarámari}) \citep[][77]{merrill2001identidad}.

The term \textit{Rarámuri} is one of several spellings found in written media, including school textbooks and other texts written in Rarámuri, as well as publications by the Mexican government's National Institute of Indigenous Languages, INALI. In INALI's publications, Rarámuri varieties are referred to by their endonyms, listed in \tabref{tab:key:1} above. The variation in the spelling of the language is rooted in the phonological inventory of the language: as discussed in \chapref{chap: phonology} below, Choguita Rarámuri (like other Rarámuri varieties), has two contrastive liquid sounds, an alveolar flap that resembles the \ili{Spanish} coronal flap, and a lateral flap, which auditorily resembles both a flap and a lateral (with certain phonological environments favoring one or the other impresionistically) (see §\ref{sec: phonological inventory}). The name of the language (/ɾaˈɽamuɽi/ [raˈɽamuɽi]) in Choguita Rarámuri features an alveolar flap word-initially (allophonically a trill) and two lateral flaps word-medially, leading to orthographic representations of the lateral flaps as either ‘r’ or ‘l’ (\textit{Ra}\textbf{\textit{r}}\textit{ámu}\textbf{\textit{r}}\textit{i} or \textit{Ra}\textbf{\textit{l}}\textit{ámu}\textbf{\textit{l}}\textit{i}). It is an open question whether other Rarámuri varieties feature this contrast. In addition to spelling variations concerning the liquid consonants of the language, there is also variation in vocalic segments that reflect dialect differences in terms of phonemic vocalic inventories and processes targeting vowels in each variety.

While the name of the language is represented orthographically in this grammar with a single symbol <r>, lateral flaps are uniformily represented with [l] in the data examples presented in this grammar.

\subsubsection{Genealogical affiliation}
\label{subsubsec: genertic affiliation}

Rarámuri belongs to the {Uto-Aztecan} (UA) language family, which spans several cultural areas of the North American continent and has a time-depth of between 4000 and 5000 years \citep{campbell2000american,fowler1983some,hill2010dating,silver1998american}. The \ili{Uto-Aztecan} language family is one of the largest language families in the Americas in terms of geographical extension, ranging from the Great Basin (where \ili{Numic} varieties are spoken) to El Salvador and Nicaragua (where varieties of Aztecan are spoken) (\citealt{miller1984classification}; \citealt{campbell2000american}; \citealt{mithun2001languages}). The location of each of the individual branches of the {Uto-Aztecan} language family is illustrated in the map in \figref{fig: UA languages}.

\begin{figure}
% \includegraphics[width=\textwidth]{figures/Introduction-UA-Languages.png}
\includegraphics[width=\textwidth]{figures/Uto-Aztecan_mod.pdf}
\caption{
\label{fig: UA languages}
{Uto-Aztecan} language subgroups}% \citep{merrill2013genetic}}
\licencebox{adapted from \url{https://upload.wikimedia.org/wikipedia/commons/e/e4/Uto-Aztecan_map.svg}
CC-BY-SA \url{https://commons.wikimedia.org/wiki/User:Noahedits}}
\end{figure}

Subgrouping within the {Uto-Aztecan} language family has been the subject of a long debate in the literature \citep{hill2011subgrouping}. Rarámuri and closely related \ili{Guarijío} are dialect continua that form two branches of a larger, uncontested \ili{Tarahumara-Guarijío} or \ili{Tarahumaran} branch.\footnote{Referred to as ``Tarawarihian" in \citet{merrill2014ralamuli}} The \ili{Tarahumaran} branch is generally classified within a \ili{Taracahitic} subgroup (\citealt{langacker1977uto}, \citealt{campbell2000american}, \citealt{mithun2001languages}, \citealt{miller1984classification}), and as part of a larger \ili{Sonoran} branch within Southern \ili{Uto-Aztecan} (\citealt{miller1984classification}, \citealt{hill2001proto}). The place of Rarámuri within a traditionally assumed subgrouping of \ili{Uto-Aztecan} is shown in \figref{fig: branches of UA} (\citealt{campbell2000american}, \citealt{mithun2001languages}, \citealt{hill2011subgrouping}). There is currently no consensus as to the genealogical status of \ili{Southern Uto-Aztecan} and \ili{Taracahitan} (for recent discussion see \citealt{hill2011subgrouping} and \citealt{merrill2013genetic}), and this is indicated with parentheses in these branches. \figref{fig: Taracahitan branch} illustrates the languages within the traditionally assumed \ili{Taracahitan} branch.

\begin{figure}
\small
% \includegraphics[width=\textwidth]{figures/Introduction_Taracahitan_in_UA.png}
\fittable{
\begin{forest}
  [\ili{Uto-Aztecan}
    [\ili{Northern Uto-Aztecan}
        [\ili{Numic}]
        [\ili{Tübatulabal}]
        [\ili{Hopi}]
        [\ili{Takic}]
    ]
    [\ili{Southern Uto-Aztecan}
        [\ili{Tepiman}]
        [\textbf{Taracahitan}]
        [\ili{Corachol}]
        [\ili{Aztecan}]
    ]
  ]
\end{forest}
}

\caption{
\label{fig: branches of UA}
{Uto-Aztecan} language family (adapted from \citealt{langacker1977uto}, \citealt{campbell2000american} and \citealt{mithun2001languages})}
\end{figure}

\begin{figure}
% \includegraphics[width=\textwidth]{figures/Introduction_\ili{Taracahitan}.png}
\begin{forest}
  [\ili{Taracahitic}
    [\ili{Cahitan}
        [\ili{Yaqui}]
        [\ili{Mayo}]
    ]
    [\ili{Tarahumaran}
        [\textbf{Rarámuri}]
        [\ili{Guarijío}]
    ]
    [\ili{Opatan}
        [\ili{Opata}$\dag$]
        [\ili{Eudeve}$\dag$]
    ]
    [\ili{Tubar}$\dag$]
  ]
\end{forest}

\caption{
\label{fig: Taracahitan branch}
{Taracahitan} branch (adapted from \citealt{campbell2000american})}
\end{figure}\il{Taracahitan}

\subsection{Previous work}
\label{subsec: previous work}

%%needs to be updated
Rarámuri varieties have been described since the seventeenth century in the form of grammars, dictionaries, vocabularies and texts. The first known documentation of Rarámuri is by Tomás de Guadalajara, a Jesuit missionary who worked in the missions in the Sierra Tarahumara in 1675, and published a brief grammatical description in 1683, \textit{Compendio del arte de la lengua de los tarahvmares y guazapares}. This small grammatical description was followed more than one hundred years later by Matthäus Steffel’s publication in 1791 of a German-Tarahumara dictionary based on German orthography.

After these early grammatical descriptions, most existing documentation of Rarámuri has been produced during the twentieth century. This includes several grammatical descriptions, dictionaries, vocabularies and texts for several varieties, mainly of \ili{Norogachi Rarámuri}, a Northern variety (\citealt{brambila1953gramatica}, \citealt{brambila1976diccionario}, \citealt{brambila1983dicctionario}, \citealt{lionnet1968intensivos}, \citealt{lionnet1972elementos}, \citealt{lionnet1985relaciones}). The most comprehensive of these works is David Brambila’s (1953) grammar, which is written in the style of colonial grammars, but also provides many examples from texts.\footnote{A list of published references on the Rarámuri language is provided in Appendix 1 of this grammar.}

In addition to these works on \ili{Norogachi Rarámuri}, several linguistic articles, grammars and short manuscripts about diverse aspects of different dialects of the language were published in the second half of the twentieth century, including descriptions of \ili{Samachique Rarámuri} (a Central variety) \citep{hilton1993diccionario}, and of \ili{Western Rarámuri} (\citealt{burgess1970tarahumara}, \citealt{Burgess-1984}, \citealt{burgess2001como}, \textit{inter alia}), which differs significantly from Northern and Central varieties. More recent linguistic work includes a BA thesis on the morphosyntax of property concepts in Choguita Rarámuri \citep{islas2010caracterizacion}, a PhD thesis on Urique Tarahumara syntax (\citealt{jara2013predication}), a master’s thesis analyzing basic clause structure and other syntactic aspects of \ili{Rochéachi Rarámuri} (\citealt{moralesmoreno2016rochecahi}), and a PhD thesis focusing on the analysis of grammatical aspect of \ili{Norogachi Rarámuri} \citep{villalpando2019grammatical}. A pedagogical grammar of Rarámuri was published by the state government of Chihuahua by Enrique Servín Herrera in collaboration with Rarámuri poet and activist Dolores Batista, a native speaker of \ili{Ojachichi Rarámuri} \citep{servin2002ralamuli}.

Other records of Rarámuri are found in ethnographic studies that have documented ethnobotanical and historical knowledge \citep{bennett1935tarahumara,bye1975ethnobotany,bye1976ethnoecology,merrill1988raramuri,pintado2012hijos}, as well as audio recordings of language and traditional music. Audio recordings are housed at the government-owned regional radio station, Radio XETAR, where they have been broadcasted since 1982 in Rarámuri, \ili{Northern Tepehuan} and \ili{Pima}.\footnote{The Radio XETAR is part of a government office program of indigenous radio stations. The governemnt office is the National Comission for the Development of Indigenous Peoples (\textit{Comisión Nacional para el Desarrollo de los Pueblos Indígenas}, or CDI).} To the best of my knowledge, many of the audio recordings that document speech are scripted and do not constitute a representative sample of patterns of spontaneous language use.

%%More on Burguess work
%%More on recent work, including Estrada Fernández and \citealt{Tona2013}, Estrada2013, Valdez Jara2009? and Villalpando2010}}
%%Give special emphasis to Morales Moreno2016}}

In addition to the publications that have a linguistic focus, much work has been carried out in the state of Chihuahua documenting traditional narratives, poetry and other forms of verbal art of the Rarámuri nation by Rarámuri poets and language activists Dolores Batista and Martín Makawi, and by Enrique Servín Herrera, linguist, activist and poet in charge of a Program for Indigenous and Minority Languages hosted by the state government’s Cultural Development Office.\footnote{\textit{Programa Institucional de Atención a las Lenguas Indígenas y Minoritarias.}} Their efforts have produced several publications in different Rarámuri varieties and both bilingual and monolingual books and resources for Rarámuri speakers and second language learners (\citealt{batista1994amanece}, \citealt{servin2002ralamuli}, \citealt{makawi2012eka}, inter alia).

\section{Geographic location and physical environment}
\label{sec: geographic location and physical environment}


Rarámuri is spoken in the southwestern part of the Mexican State of Chihuahua, a rugged area in the Sierra Tarahumara that includes the Copper Canyon. The Sierra Tarahumara is part of the Sierra Madre Occidental, a mountain range that extends from the Southwest United States to Central Mexico, with an area of approximately 50,000 square kilometers (\citealt{cortina2012hijos}). This grammar describes variety spoken in Choguita (also known as ``Choguita de Guachochi'' or ``Choguita de Norogachi''), in the municipality of Guachochi. Geographically adjacent \ili{Uto-Aztecan} languages include \ili{Guarijío} (or Warihó), \ili{Yaqui} (or \ili{Hiaki}), \ili{Mayo}, \ili{Northern Tepehuan} and \ili{O’ob Nook Pima}. The location of Choguita Rarámuri and neighboring \ili{Uto-Aztecan} languages in the Mexican Northwest is shown in the map in \figref{fig: CR and neighboring languages}.

\begin{figure}[t]
% \includegraphics[width=\textwidth]{figures/Introduction-CR-neighboring-UA.png}
\includegraphics[width=.8\textwidth]{figures/neighboring2.pdf}
\caption{
\label{fig: CR and neighboring languages}
Location of Choguita Rarámuri and neighboring {Uto-Aztecan} languages}

\licencebox[.8]{
\url{https://commons.wikimedia.org/wiki/File:Mexico_topographic_map-blank_2.svg}

\url{https://commons.wikimedia.org/wiki/User:GrandEscogriffe}

CC BY-SA 3.0}
\end{figure}

The community of Choguita is part of the \textit{ejido} system, a Mexican land usage system, where rural land plots are devoted for collective use by community members (\textit{ejidatarios}). Choguita is the head community of the \textit{ejido} of Choguita, one of the largest \textit{ejidos} in the Sierra with a total surface area of 285.6 km\textsuperscript{2} (28,560 hectares, 110.3 miles\textsuperscript{2}) \citep{casaus2008quantitative}. Figure~\ref{fig: boundary of the ejido of choguita} shows the boundaries of the ejido of Choguita, which includes the communities of Bokimoba, Huichachi, Capochi, Basigochi, Coechi, Rayabó, Cochirachi, Ireachi, Rorichi, Rochibo, Upachi, Sehuarachi and Cocohuichi. According to the Mexican government 2000 national census, the community of Choguita has 234 inhabitants, and the entire population of the \textit{ejido} is approximately 1050 (\citealt{casaus2008quantitative}). With the exception of secondary school teachers and Protestant missionaries, the community is native Rarámuri.




%location

%local topography

%climate


\begin{figure}
\includegraphics[width=\textwidth]{figures/GrammardraftJuly182017-img2.jpg}
\caption{
\label{fig: boundary of the ejido of choguita}
Boundary of the ejido of Choguita (Topographic map, \citealt{casaus2008quantitative})}
\end{figure}

\section{Choguita Rarámuri in social and historical context}
\label{sec: social and historical context}

This section discusses the historical, social and political contexts in which Choguita Rarámuri is used. First, §\ref{subsec: history of contact with Europeans} addresses the history of contact between the Rarámuri people and Europeans (and after the establishment of the Mexican state, Mexicans of European descent, referred to as \textit{chabóchi} (a Rarámuri word) in the Sierra Tarahumara). This is followed by discussion of Rarámuri language use in §\ref{subsec: linguistic ecology and socioolitical context}, including bilingualism and the political and social factors affecting speech communities since the twentieth century. This section concludes in §\ref{subsec: literacy}, which focuses on programs sponsored by the Mexican government to promote so-called ``bilingual /bicultural" education in the sierras and standardization of the language, as well as emerging use of the language in social media by native speakers.

\subsection{History of contact with Europeans}
\label{subsec: history of contact with Europeans}

The earliest contact between the Rarámuri and Europeans was in the late sixteenth century \citep{merrill2014ralamuli}. The first incursion of Jesuit missionaries took place in 1607, with the arrival of Joan Fonte to the Valle de San Pablo, an area in the border between the land of the Rarámuri and the Odami ({Tepehuan}) nations in what is now known as the district of Balleza \citep{alegre1842historia,pintado2012hijos}.\footnote{Historical accounts report that this area was also inhabited by Guazápares, Chínipas, Témoris, Guarijíos, Jovas, Pimas, Conchos, Janos, Julimes, Chinarras, Tobosos, Acoclames, Chizos, Tubares, Tzoes and Cocoymes, and members of the N’dee/N’nee/Ndé (Apache) nation (\citealt[][53]{neumann1991historia}, \citealt[368]{gonzalez1987cronicas}; cited in \citealt{pintado2012hijos}).}

Ethnographic studies report that at the time of contact the Rarámuri lived in small settlements (villages of 5 to 20 households) relatively spread out across an area of approximately 45,000 square kilometers in the southwestern and central area of what is now the state of Chihuahua \citep{pennington1963tarahumar,merrill2014ralamuli}. Figure \ref{fig: Raramuri land 17th century} shows the location and extension of the Rarámuri territory in the seventeenth century.

\begin{figure}
% \includegraphics[width=.99\textwidth]{figures/Introduction-seventeenth.png}
\includegraphics[width=.99\textwidth]{figures/seventeenth2.pdf}
\licencebox{© OpenStreetMap contributors, Open Database License}
\caption{
\label{fig: Raramuri land 17th century}
Territory of the Rarámuri nation in the seventeenth century adapted from \citet[230]{merrill2014ralamuli}
}
\end{figure}

European impact has been felt severely in the area for the past 400 years. The campaign of religious conversion and political control that began in the seventeenth century was met with strong resistance from the Rarámuri, who protested and fought back the invasion of their land. Four major rebellion movements are documented between 1648 and 1697, in which many missions were destroyed \citep{neumann1991historia,gonzalez1982tarahumara,levi1999hidden,rodriguez1991testimonios}. For much of the seventeenth century, the conflict pushed many Rarámuri out of their communities, who took refuge in other areas, including the southwestern region of the Sierra. The Rarámuri population was significantly decimated by the violence inflicted by settlers and also epidemics that afflicted the population across the whole area \citep{pintado2012hijos}.

The Rarámuri continued to resist occupation of their land during the following centuries. Population movements to isolated areas to avoid encroaching settler populations became the main form of resistance \citep{merrill1983tarahumara,levi1999hidden}. Some Rarámuri communities adopted some of the social and religious norms of life in the missions. This process began prior to the expulsion of Jesuits from the Americas in 1767, after which there was another period of relative isolation of Rarámuri communities from colonial institutions \citep{pintado2012hijos}. The Rarámuri from these communities self-describe as \textit{Pagótame} `baptized ones' (literally `washed ones'). It should be noted that adoption of foreign religious norms was only a partial process, as \textit{Pagótame} practice a form of syncretic Christianity \citep{merrill1983tarahumara,merrill1988raramuri}. Other Rarámuri communities, the \textit{Simalóni} (\textit{cimarrón})\footnote{The term \textit{Cimarrón} was used during that time period to refer to people from African descent who escaped slavery and founded free Afro-Mexican towns; the missionaries used this term to refer to Rarámuri people who escaped forced labor and religious conversion imposed in the missions \citep[53]{pintado2012hijos}.} or `gentile' (unbaptized), resisted adopting any form of Christianity and retreated to the most remote areas of the Sierra.

\largerpage
A new stage in the interaction between the Rarámuri and Europeans began in the nineteenth century with the founding of the Mexican nation state in 1810 and the state of Chihuahua in 1823. Shortly after the founding of the state of Chihuahua, a new Law of Colonization (in 1825) allowed Mexican \textit{mestizo} settlers to buy land that had belonged to the missions, causing a new wave of displacement of the Rarámuri, who sought to keep their political autonomy \citep{pintado2012hijos}. Mining became a dominant economic activity in the late nineteenth century for several decades, and an intense and unregulated exploitation of the forests began in the middle of the twentieth century. In addition to this, the Mexican revolution brought about changes to government institutions in the early twentieth century, including reforms in the land ownership system and the creation of \textit{ejidos}, as discussed in §\ref{sec: geographic location and physical environment}. Together, these factors led to major shifts in the economic and political life of the Sierra in the last century. The effects of these shifts in terms of the attempts to assimilate the Rarámuri through government-sponsored education programs is discussed in §\ref{subsec: literacy} below.

\subsection{Linguistic ecology and sociopolitical context}
\label{subsec: linguistic ecology and socioolitical context}

%this belongs to the socio-political section
At the beginning of the twenty-first century, Choguita Rarámuri was spoken by approximately one thousand speakers \citep{casaus2008quantitative}, a number that includes the inhabitants of all the villages within the \textit{ejido}, of which the community of Choguita is the head village. There are several factors that suggest that the domains of usage of Choguita Rarámuri are contracting. In recent years, an increasing number of speakers are relocating to larger towns within the Sierra Tarahumara and in the capital city of Chihuahua.

As discussed in §\ref{subsec: history of contact with Europeans}, the Rarámuri have faced great pressures to assimilate to settler society since the seventeenth century, and their land was reduced to half its original size \citep{paciotto1996tarahumara}. Some of the main factors currently threatening the cultural and political autonomy of the Rarámuri nation are: (i) doubling of the \textit{mestizo} population in the Sierra over the last century \citep{merrill2013genetic}, (ii) increasing forest exploitation, (iii) depletion of water resources, (iv) expansion of road construction, and (v) recurrent violations of indigenous land property, to name only a few. The historic retreat of the Rarámuri to mountainous, isolated areas in order to avoid conflict with the \textit{mestizo} population has had negative repercussions for their economic and political autonomy. Specifically, this displacement into areas highly adverse for maize agriculture is one of the main factors behind the severe marginalization of the Rarámuri \citep[][77]{merrill1988raramuri}. Most recently, Choguita has had many young people migrating to urban centers, including \textit{mestizo} towns in the sierras, the state’s capital, Chihuahua \citep{munoz2019reproduccion}, and border city Ciudad Juárez, as well as to agricultural fields in the region.

\hspace*{-.35pt}Language decline has been documented in varying degrees in the Sierras. Some communities display interrupted intergenerational transmission of the language, while some others remain completely monolingual. Most communities present an intermediate situation with varying levels of bilingualism \parencite{paciotto1996tarahumara}. In Choguita, Rarámuri is being learnt by children, who remain monolingual until they attend primary school. Some primary school teachers are native Rarámuri speakers (most of whom speak non-local varieties of the language), but none of the secondary school teachers even know Rarámuri as a second language. In school, Rarámuri is marginally used between first and fourth grade in order to gradually introduce children to \ili{Spanish}, but children are exposed exclusively to {Spanish} in the classroom after fifth grade (Severiano González (Choguita primary school director), p.c.).

Choguita has also undergone increasing contact with the \textit{mestizo} population for the past few decades due to the improvement of roads that connect Choguita with \textit{mestizo} enclaves in the Sierra. In their interactions with health promoters, government officials, traders, and religious missionaries, native Rarámuri speakers must switch to {Spanish}. Rarámuri is used in local administration, traditional ritual contexts, and spoken communication in joint community agricultural activities and drinking parties. Recently, however, Rarámuri speakers have started to switch to {Spanish} to communicate with each other in these spaces as well, as speakers themselves note and as I have been able to assess during the time I have spent in the community. The advancement of {Spanish}, thus, can be felt in every sphere of Rarámuri life, and older members of the community express their concern about the proficiency of younger speakers in Rarámuri.

%cite Vinicio Muñoz about language transmission in diapora commmunities in Chihuahua

%In sum, like many other minority languages of Mexico, the Rarámuri spoken in Choguita is increasingly vulnerable to escalating pressures imposed by the \ili{Spanish} speaking population. The marginalization factors mentioned above have already triggered contraction of domains of usage of the language, and it is possible that language shift will occur within a generation or two.

%%expand – note the use of social media by speakers

\subsection{Mexican government sponsored ``bilingual/bicultural
" education and literacy}
\label{subsec: literacy}

% modify and address attempts to standardize the language, and new work created by Rarámuri scholars, work by E. Servín
%edit
In 1989, a Chihuahua local-state office attempted to create a standarized orthographical system for Rarámuri, but the project was never completed (\citealt{pintado2004tarahumaras}), and the existing published materials display a great amount of variation. In Choguita, written materials in Rarámuri play a very limited role. The only written materials in Rarámuri are some sections of the official textbooks used in the local elementary school. The official schooling process is mainly devoted to promoting literacy in \ili{Spanish}, as has been observed to occur in other indigenous communities in Mexico \parencite{lastra2001otomi}. The official “bilingual/bicultural program”, designed by the Mexican Government and argued to reflect a concern about  reflecting local cultural characteristics, in actual practice has served only to increase {Spanish} proficiency among the indigenous population and as a tool of linguistic assimilation in Rarámuri communities. The schooling process, alien to community interests and reality, reinforces stigmatization of native languages.

While there is currently no component of formal education that promotes literacy in Rarámuri in Choguita, speakers nevertheless use a {Spanish}-based orthography to communicate in Rarámuri in social media and via text messaging. Thus,  access to new resources, such as smartphones, allows Choguita Rarámuri speakers to use their language in new contexts. Making the language documentation collection accessible and usable to all interested users therefore involves standing challenges, which also includes anticipating changing needs and possible future agendas, such as language reclamation efforts in diaspora communities.

\section{This grammar}
\label{sec: this grammar}

This grammar provides a comprehensive linguistic description of the phonology, morphology, and syntax of Choguita Rarámuri. This section discusses historical information of the project that gave rise to this grammar (§\ref{subsec: project development}). This is followed by discussion of theoretical assumptions made in this work (§\ref{subsec: theoretical assumptions}) and description of the data sources used and the methodologies employed to elicit this data (§\ref{subsec: data, corpora and methodology}). This section also addresses contributions made by individual language experts to this grammar (§\ref{subsec: language experts and collaborators}), followed by an overview of how data examples are presented (§\ref{subsec: representation of examples}).

\subsection{Project development}
\label{subsec: project development}

This grammar is the product of work carried out together with Choguita Rarámuri language experts, as well as students and other collaborators who became part of the Choguita Rarámuri Language Project. I first became involved with the study of Choguita Rarámuri as a graduate student in 2002, through contacts with researchers and community members who were interested in the language and history of the community. Since my first visit to Choguita, I addressed community members in local assemblies presided over by Choguita's authorities to ask for permission to spend time in the community and learn the language. In these meetings I expressed my interest in writing about the language for my studies and my desire to engage in a long-term relationship with community members interested in language, culture and history documentation. I received approval to study in Choguita, and several people expressed enthusiasm about the idea of having the language documented and interest in working with me.

In every visit to Choguita, I addressed community members to inform them of the activities carried out with individual collaborators and overall progress with the project and to request their continued support. Initiatives by community members to set their own documentation agenda arose only after several visits. After learning of developments of this project, several community members expressed interest in creating a record of the speech of elders, as well as the community’s historical past and receding ritual, cultural, and artistic practices. In particular, the initiative was that a team of local experts (Sebastián Fuentes Holguín, Giltro Fuentes Palma and Francisco Moreno Fuentes) would be in charge of planning and undertaking documentation activities, and requested that video documentation training be made available to community members. A community-developed repository of materials would be created and mobilized with the goal of having younger generations have access to these materials, incentivize their continued transmission and have documentation outcomes available for future generations. In response to this request, several video documentation workshops with adults and high school students were held with the help of a videographer (Jorge Esteban Moreno Romero). Project participants designed and carried out their own video documentation projects, which included interviews with elders, recording of pedagogical materials for children, and recording of different community events. Ritual and traditional events documented include healing ceremonies, races, rain and harvest ceremonies, ritual appointment of local authorities, and Easter celebrations.

The documentary materials produced through this initiative were digitized and copies given to individual creators of materials, as well as local authorities, the \textit{siríame} (governors). Copies of these video recordings, along with recording equipment and a projector, were deposited at the local school, where Mr. Giltro Fuentes Palma would be in charge of using these materials in the school curricula. Mr. Fuentes Palma was appointed by local authorities to have custody of these materials and lead video documentation projects and mobilization of materials. Some of the participants who received the initial training were no longer able to continue carrying out documentation, as some left the community and others acquired time-consuming obligations, including Mr. Fuentes Palma. A standing challenge of this project has thus been to enable a sustainable infrastructure within Choguita for continued community-based language and culture documentation and safe-keeping and use of materials by main stakeholders in this project.

I completed a dissertation on the phonology and morphology of Choguita Rarámuri at the University of California, Berkeley in 2008 \citep{caballero2008choguita}. Since then, I have continued to revise chapters of the dissertation given continued analysis of the phonology and morphology of the language. This work was also supplemented with data obtained in subsequent field trips which expanded description of morphosyntax and syntactic structures to develop the present grammar. I moved to UC San Diego in 2010, where then linguistics graduate student Lucien Carroll joined the project. Carroll primarily carried out annotation and analysis of field data, conducting his own in-situ fieldwork in 2014 and 2018. Other UCSD researchers, both students and faculty members, have collaborated in the analysis of morphological and phonological phenomena. Many Choguita Rarámuri language experts have collaborated as authors, consultants and creators of documentation and analysis. Choguita Rarámuri language experts who have have had a deeper involvement with this project in multiple roles include Rosa Isela Chaparro Gardea, elder †José María “Chémale” Fuentes, elder †Morales Fuentes Hernández, Sebastián Fuentes Holguín, Bertha Fuentes Loya, Guillermina Fuentes Moreno, Giltro Fuentes Palma, elder †Luz Elena León Ramírez, elder †Federico León Pacheco and Francisco Moreno Fuentes (details of their contributions are provided in §\ref{subsec: language experts and collaborators}).

%detail here what's still missing: sustainable access to Choguita community members, resources on the web for users in diaspora communities, translation of reference grammar into \ili{Spanish} and deposit materials in archives where Rarámuri users will have access (i.e., in Chihuahua city) - also a user-interface with legacy materials, where annotation is possible by language experts

\subsection{Theoretical assumptions}
\label{subsec: theoretical assumptions}

This grammar aims to provide a comprehensive and careful description and analysis of Choguita Rarámuri without employing formalisms that are likely to date the grammar and make its content uninterpretable or inaccessible in the future \citep{ameka2006catching}. Following a long tradition of grammatical description that seeks to characterize patterns and phenomena in individual languages on each language's own terms, each section in this work outlines the language-internal criteria and evidence that motivate postulating the grammatical categories identified in Choguita Rarámuri (see \citealt{cristofaro2008organization} for discussion).

Nevertheless, this grammar also seeks to identify the ways in which Choguita Rarámuri resembles other languages (related {Uto-Aztecan} languages, or languages of Northern Mexico and beyond), and thus links description of patterns and phenomena to relevant typological, theoretical and descriptive literature where pertinent. This includes the use of standard conventions of glossing and terminology that would facilitate use by those interested in the typological properties of Choguita Rarámuri and languages of the area. More broadly, this grammar draws from typological and theoretical assumptions about morphological organization, prosodic structure, and the relationship between morphology and phonological processes, areas which display particular complexity in this language.

With respect to the interaction between phonology and morphology, the analysis presented in this grammar is compatible with construction-based approaches to morphology and phonology, and follows the assumption that morpho-phon\-ol\-o\-gical processes are intimately related to the word’s layered structure, with phonology being able to apply to nested subconstituents in a word. This is a key assumption made in this work in light of the evidence in Choguita Rarámuri for an organization of the morphological structure of this language in domains, a structure which is exploited to understand the constraints on stress assignment, the domains of application of several phonological processes, and the limited appearance of multiple exponence and variable suffix order in morphologically complex words. I also make the assumption that languages may contain several phonological sub-grammars pertaining to lexical class, morphological categories, or particular morphological constructions. Both layering and construction-specific phonological processes are compatible with frameworks that posit `cophonologies', phonological sub-grammars associated with individual morphological constructions (\citealt{orgun1996sign,anttila2002morphologically,inkelas2005reduplication}; see overview and discussion in \citealt{inkelas2014interplay}).

With respect to prosodic structure,  I adopt the proposal in \citet{selkirk1980prosodic}, \citet{selkirk1996prosodic}, \citet{nespor1986prosodic}, and \citet{hayes1989prosodic} for the prosodic hierarchy, where the Prosodic Word is the smaller unit within the hierarchy (further discussed in \chapref{chap: prosody}). I also draw assumptions from property-driven approaches to word-prosodic typology laid out in \citet{hyman2006word} and \citet{hyman2009not}, and describe the prosodic system of Choguita Rarámuri in terms of the canonical properties of stress and tone systems, dispensing with the notion of ``pitch-accent" found elsewhere in the literature. Finally, in the analysis of stress, tone and intonation, the description and analysis presented here makes reference to principles and assumptions from the Autosegmental-Metrical (AM) framework (\citealt{pierrehumbert1980phonology}, \citealt{beckman1986intonational}, \citealt{ladd1986intonational}) and metrical stress theory \citep{hayes1995metrical}. These frameworks provide useful tools, including phonological representations, that allow presenting key phenomena in this language. I sought a balance, whereby the notions from any specific theoretical, typological or descriptive approach employed here would not preclude the possibility of alternative analyses that may be more insightful than those presented here.

Given that the data analyzed here comes from a documentary corpus that is heterogeneous in terms of contributing language experts and genres of speech represented, this grammar attempts to describe a single language variety while at the same time addressing and exemplifying some patterns of variation present in the speech community. Reference grammars continue to provide the empirical backbone of developing linguistic theories, research in linguistic typology, and the creation of pedagogical materials for language maintenance, revitalization and reclamation. However, as noted by \citet{evans2006introduction}, using the metaphor of grammar writing as ``catching” language, reference grammars (such as this one) can only aspire to capture static pictures, a very small fraction of a complex linguistic system. Though still limited, documentary corpora provide a more representative window into language as a dynamic system with significant variation and change in progress.

In an effort to articulate the link between this grammar and the documentary corpus in which it is based, I provide details of the content and structure of the Choguita Rarámuri corpus in §\ref{subsec: data, corpora and methodology} below. This grammar is linked to the corpus from which it stems in two other ways. First, each example is provided with source information, which allows users to retrieve the documents from which examples come from in the documentary collections of the language (see more details in §\ref{subsec: representation of examples}). Second, a subset of examples of this grammar are provided with links to sound files; these are intended not only to strengthen the link between grammar and corpus, but also to increase transparency of the analyses presented here, while providing interested users with a deeper level of access to the data on which generalizations are based upon (see \citet{remijsen2018descriptive} for an example of grammatical description that includes sound examples). This more direct representation of sound in the grammar responds to calls for a new paradigm in grammatical description where users are able to form their own conclusions without their being mediated by the analytical interpretation of grammar writers \citep{rice2014sounds}, enhancing the ``reproducibility" of grammatical description. ``Reproducible" research is understood here as ``research [that] aims to provide scientific accountability by facilitating access for other researchers to the data upon which research conclusions are based” in cases where true replicability (the ability to produce new data by recreating research conditions faithfully) is not possible \citep[4]{berez2018reproducible}. Most importantly, sound in grammar is intended to enable and improve access to original language data by community members who are native speakers or language learners who are interested in language preservation, revitalization and reclamation (see also \citet{rice2014sounds} for discussion).

While the present grammar aims to provide a glimpse of variation within the speech community, it should be noted that a full account of how lectal variation and bilingualism operate in the community of Choguita and how it may have an impact on Choguita Rarámuri grammar is a topic for future research.

\subsection{Data sources and methodology}
\label{subsec: data, corpora and methodology}

A documentary corpus of Choguita Rarámuri informs the analysis and is the source of the examples presented in this grammar. This corpus comprises data collected during several visits I made to Choguita between 2003 and 2018. The contexts in which data were elicited are outlined in the following subsections. The corpus consists of over 200 hours of audio and/or video recordings of elicited data, personal, historical, and procedural narratives, conversations, interviews, prayers, and oratory, as well as several hours of sessions where language experts chose the topics to be covered in language teaching sessions.

Recordings and associated annotation materials are available in two archival collections. The first collection \citep{caballero2009corpus} is housed at the Endangered Languages Archive,\footnote{This collection is available on-line at \url{http://elar.soas.ac.uk/deposit/0056}.} and contains approximately 130 hours of digital audio recordings, 10 hours of video recordings, and a substantial amount of digital transcription and annotations. These materials were obtained from 2003 to 2009. An overview of the background and contents of this collection is provided in \citet{caballero2017choguita}. The second documentary collection \citep{chaparro2019corpus} is housed at the Survey of California and Other Indian Languages at UC Berkeley.\footnote{This collection can be retrieved on-line at \url{http://dx.doi.org/doi:10.7297/X2HH6H70}.} This second collection contains over 1,300 digital files, including 115 hours of sound recordings that were primarily obtained between 2011 and 2018.

All of the data examples cited in this work are part of materials where individual contributors consented to be included in this work and other linguistic publications. A subset of the materials of the documentary collections are, however, not cited in this work, as there is a community-wide consensus that certain materials are not to be accessed by outsiders. Specifically, and as decided in local government assemblies, special sensitivities are placed on video recordings of ritual celebrations and activities, as outsiders in the past have recorded them without consulting local authorities or the community at large.

The Choguita Rarámuri documentary corpus comprises a wide sample of different speech genres with different degrees of planning (including conversations, monologues, narratives, myths, ceremonial speeches (\textit{nawésari}), interviews of elders by native speakers, and ritualistic chants and prayers (healing ceremonies). The corpus also includes recordings of elicitation sessions where speakers undertake the role of language teachers, which allows for a great deal of contextualization of the data elicited. Other kinds of elicitation conducted included contextualized- and text-based elicitation, translation, metalinguistic judgements, and elicitation prompted by culturally relevant visual props, as well as and participant observation. The following sections describe the different classes of data collected between the duration of the project, as well as the methodologies employed to this end.

\subsubsection{Narratives}
\label{subsubsec: narratives}

Many examples in this grammar come from narratives, a series of documents that mainly involve monologic speech, of a kind closer to the planned speech event end of a spontaneity continuum. These texts include genres such as general descriptions of events, historically contextualized descriptions, myth narratives, procedural texts, and narratives from visual prompts, recorded (in both audio and video) with single speakers.
Audio recordings of texts were typically made in indoor, private spaces, while video-recorded texts generally involve procedural texts and historical descriptions made in situ. These narratives are crucial to linguistic analysis in this reference grammar and are part of the developing record of the history, culture and language of the community as envisaged by community members.

While contributing language experts were sometimes prompted very generally (with questions about how Choguita was during their childhood, questions about pictures of the community, or asking for a traditional story), the topics covered in the narratives were generally selected by each contributing language expert. In some cases, speakers had family members as an audience when recording their text, when this was contextually appropriate. In some cases, a narrative would evolve from a conversation about particular topics or through elicitation of lexical items, blurring the lines between narrative proper, conversation and elicitation. Example sentences that come from narratives are indicated as `tx' in the source code (see §\ref{subsec: representation of examples} for more details).

\subsubsection{Conversations}
\label{subsubsec: conversation}
\largerpage
A subset of recordings in the corpus involve audio and/or video-recorded conversations. In these sessions, speaker participants engaged in conversations, either with other native speakers or with me (as a second language user of Choguita Rarámuri). Conversations recorded can be classified into three types: (i) unstructured conversations that arose in the context of elicitation; (ii) elicited conversations between native speakers; and (iii) semi-structured conversations between monolingual language experts and me for lexical or grammatical elicitation. In conversations arising in the context of elicitation, these mainly involve informal, unstructured exchanges between a language expert and me and focused on lexical items (such as kinship terms), contexts of use of particular structures or linguistically and/or culturally relevant aspects of topics covered in elicitation.\footnote{In a few recordings, the exchanges were exclusively carried out between native speaker participants discussing the meaning of lexical items and constructions elicited using culturally relevant still pictures.} In these types of conversation, both Choguita Rarámuri and \ili{Spanish} are used. The second type of conversations, less frequent, involved eliciting conversations between native speakers in arranged recording sessions, generally with prompted topics of conversation. These types of conversations generally ended up having the structure and tone of a monologic narrative by one speaker and back-channeling by other speakers. Finally, in the third type of conversation, semi-structured conversations, the sessions were designed to elicit particular lexical items or grammatical constructions with monolingual speakers (for example, to elicit tonal minimal pairs).

Since all these interactions took place in the context of elicitation sessions or typically involved topics and constructions that were recorded for the purpose of informing linguistic analysis, they are part of the set of materials from the deposited collections that have an exclusively linguistic focus. None of these conversations can be taken to be representative of Choguita Rarámuri conversational norms, either because the exchanges involved my limited fluency or because the presence of the recorder influenced the interaction (all of the conversational data in the corpus was obtained on-record, with explicit acknowledgement and permission from contributing speakers).

\subsubsection{Interviews}
\label{subsubsec: interviews}

Some of the examples cited in this grammar are drawn from interviews with elders, which were obtained by community member collaborators as part of the initiative to document the history of the community. These interviews had a special focus on endangered domains of knowledge still possessed by elders, but in attrition or non-existent for younger speakers, such as the complex kinship term system that at the time of recording was not known with its full complexity by speakers in their thirties and younger. Other topics included past agricultural practices, food preparation, ritual practices, as well as personal biographical information of interviewees. The interviews include accounts by interviewees of their experience growing up in Choguita and the differences brought about by recent changes in the community, such as the introduction of the government-run local school or the introduction of electricity. Interviews were video recorded with high quality audio, with additional, separate audio recordings made at the time of the recording with the solid-state recorder. These interactions were almost exclusively dyadic and carried out in Choguita Rarámuri, i.e., no code-switching was attested in these recordings.

\subsubsection{Elicited data}
\label{subsubsec: elicited data}

Much of the data used in the grammar comes from lexical or grammatical elicitation designed to gather specific types of evidence for phonological, morphological, or syntactic analysis. While it is desirable to exemplify patterns and structures drawn from speech events with a lower degree of planning (such as narratives or conversations), some types of constructions are rarely found in other speech genres due to the highly specific semantic contexts they involve. Elicitation was thus critical in assessing the nature and properties of the grammatical constructions discussed in this grammar.

The distinction between conversation, elicitation and narrative was blurred in certain contexts of language work, but data examples are identified as elicited in this work if it meets one of the following criteria: (i) it was obtained by asking speakers to translate a word or phrase from {Spanish} into Choguita Rarámuri or viceversa (translation elicitation); or (ii) it was uttered by speakers when provided a context and being asked what they may say in such a context; or (iii) it was an answer provided by speakers when offered a Choguita Rarámuri word or clause (either overheard or from annotation of a text) and asked to elaborate about the context in which they would utter it (contextualized and text-based elicitation).\footnote{This type of elicitation proved to be useful for elucidating properties of grammatical constructions or lexical items: particular constructions encountered during an off-record conversation or in the process of annotating individual texts would serve as starting points for exploring grammatical or lexical aspects of the constructions in question.}

In addition to these strategies, some sessions also involved eliciting metalinguistic judgments or eliciting responses using visual props, such as pictures of rituals or agricultural practices in other Rarámuri communities. Other sessions were video recorded and focused on in-situ descriptions of agricultural terms and the language of space, including deictic terms, topographic terms and landscape-based standardized place names. Finally, one other type of elicitation involved requesting grammaticality judgments of morphologically complex words or clauses in Choguita Rarámuri; the offered forms were either constructed forms with logically possible affix orderings or stress patterns or forms produced by other speakers. I would ask speakers to assess the grammaticality of the offered forms and, if judged grammatical, to discuss their meanings in detail. I have avoided exemplifying any given pattern with this kind of evidence, except for cases where negative evidence (i.e., the ungrammaticality of a particular suffix sequence) is relevant in the discussion. Any data examples that arose through this methodology are indicated as `[pr.]' (for prompted) in the source code.

\subsubsection{Language teaching sessions}
\label{subsubsec: language teaching}

A final type of data examples come from language teaching sessions, where native speaker participants explained lexical items, constructions and expressions, and their cultural contextualization to me. In some cases, I prompted the topics of discussion, which included overheard expressions and terms recorded during elicitation sessions. In most of these sessions, however, speakers were just asked to teach me anything they decided would be appropriate for me to learn as a second language learner and for the purpose of my description and analysis of the language. In these sessions it was frequent that language experts would correct my mistakes when using Choguita Rarámuri, clarifying the form and/or meaning of the expression in question, as well as the appropriate context for its utterance. Other topics covered in these sessions include politeness formulas, vocabulary associated with rituals and traditional agricultural practices, culturally relevant contextualization for the use of some terms and colloquial expressions, and discussions about dialect differences between Choguita Rarámuri and neighboring varieties.

\subsection{Language experts and collaborators}
\label{subsec: language experts and collaborators}

To date, thirty-four community members have participated in the Choguita Rarámuri language project in different roles, including as authors, consultants, and creators of audio and video recordings, annotation and analysis. All native
% speaker
contributors wanted their utterances attributed to them and to be acknowledged for their expertise and role in this project. A full list of language experts who have contributed to this project is provided below in \tabref{tab:key:1}, with contributors' initials, names and roles in the project. Speakers who contributed their expertise in elicitation sessions are listed as consultants. Primary speakers in monologic texts and interviewees in interviews are listed as authors, while interviewers are listed as interviewers. The initials that precede each contributor's name corresponds to the language expert reference in the source file names listed immediately below example sentences (details of how data examples are represented are provided in §\ref{subsec: representation of examples}).

\begin{table}[t]
\caption{Names, initials and roles of contributing language experts}
\label{tab:key:1}
\fittable{
\begin{tabular}{lll}
\lsptoprule
RCG & María Del Rosario Cervantes Guerrero & consultant \\
RIC & Rosa Isela Chaparro Gardea & consultant, author\\
MGD & María Guadalupe Diaz & consultant, author\\
ME & Mateo Espino & author\\
SF & Santos Fuentes & consultant\\
ROF & †Rosa Fuentes & author\\
JMF & †José María ``Chémale" Fuentes & author\\
MAF & Miguel Angel Fuentes Diaz & consultant\\
VFD & Virginia Fuentes Diaz & participant\\
AFD & Angelina Fuentes Diaz & participant\\
RFG & Reyes Fuentes Guerrero & consultant, interviewer\\
CFH & †Cornelio Fuentes Hernández & author\\
MFH & †Morales Fuentes Hernández & consultant, author\\
SFH & Sebastián Fuentes Holguín & consultant, author, interviewer\\
BFL & Bertha Fuentes Loya & consultant, author\\
YFL & Yeni Fuentes Loya & consultant\\
CFM & Carlos Fuentes Moreno & consultant\\
GFM & Guillermina Fuentes Moreno & consultant, author\\
VFM & Valentina Fuentes Moreno & consultant\\
ViFM & Vicente Fuentes Moreno & consultant\\
GFP & Giltro Fuentes Palma & consultant, author, creator, interviewer\\
RGF & Rocío Guerrero Fuentes & consultant\\
TGH & Teresa Guerrero Herrera & consultant\\
MDH & María Dolores Holguín & author\\
JHF & Javier Holguín Fuentes & author\\
AHF & †Alicia Holguín Fuentes & consultant \\
JLG & Jesusita Loya Guerra & consultant, author\\
RLH & Roberto León Holguín & consultant\\
FLP & †Federico León Pacheco & author\\
LEL & †Luz Elena León Ramírez & consultant, author\\
FMF & Francisco Moreno Fuentes & consultant\\
SMM & Sebastián Moreno Morales & consultant\\
ChR & Cherame Rosesio & author\\
MIV & Ma. Ignacia Valencia Nevárez & consultant\\
\lspbottomrule
\end{tabular}
}
\end{table}

From the project participants listed in \tabref{tab:key:1}, the main contributors to this grammar are the language experts listed below (in alphabetic order). Below, I briefly describe their role in the project.

\medskip

\noindent \textbf{Rosa Isela Chaparro Gardea (RIC)} joined the project in 2011 and quickly became one of the main contributors to the project, authoring several narratives, helping me annotate and translate narratives authored by her and other language experts and participating in elicitation sessions, as well as offering insights about variation between Choguita Rarámuri and closely related \ili{Norogachi Rarámuri}, where she was born and raised.

\medskip

\noindent Elder \textbf{†José María ``Chémale" Fuentes [JMF]} was a longstanding leader and authority in Choguita (having served as the head governor several times) who had a wealth of knowledge about the history of the community, was an accomplished violin player and an expert of the \textit{nawésali} register, the oratory speech style that only a few (those recognized as the wisest in the community) master. He graciously contributed historical and mythical narratives, as well as descriptions of agricultural practices past and present.

\medskip

\noindent Elder \textbf{†Morales Fuentes Hernández [MFH]} was a ritual singing shaman (\textit{sikaláme}), violin player who had a great interest in recording the music and rituals of the community. His recordings were always done by his enthusiastic initiative, and with a sense of urgency: ritual music and shamanic singing are some of the domains of local culture that is mainly mastered by elders, with few young people learning this form of art. His own personal style of performance is reflected in the recordings made. He also participated in grammatical elicitation sessions structured as conversations, as he was one of the elders in the community who had very limited knowledge of \ili{Spanish}.

\medskip

\noindent \textbf{Sebastián Fuentes Holguín [SFH]} was one of my key teachers and collaborators since the beginning of the project, and has contributed extensively to every aspect of the project, from authoring several monologic narratives and ceremonial speeches, to being a patient collaborator and teacher in elicitation sessions. During elicitation sessions I benefited from his insights and expertise, and he was always enthusiastic to discuss possible contexts of utterance for constructions elicited. He is a passionate advocate for the language and one of the main leaders of the community-based initiative to document the cultural, historical and linguistic heritage of Choguita.

\medskip

\noindent \textbf{Bertha Fuentes Loya [BFL]} was also one of my main teachers and collaborators. She authored narratives and contributed as a consultant for many elicitation questions and was also an excellent language instructor. She also was a key contributor of annotation of texts recorded with other language experts and also shared her expertise in variation between Choguita Rarámuri and neighboring Rarámuri varieties. She contributed to in-situ elicitation focused in agricultural practices and landscape terms. She is also an expert seamstress and authored several narratives and procedural texts about her art.

\medskip

\noindent \textbf{Guillermina Fuentes Moreno [GFM]} is a community leader who contributed her knowledge of the history of Choguita in several historical narratives and also shared her vision of the community into the future. She was also a consultant in elicitation sessions and unstructured conversations.

\medskip

\noindent \textbf{Giltro Fuentes Palma [GFP]} contributed to the project as an author of narratives, as a consultant in elicitation sessions, with annotation and analysis as well as creation of video documentation. As part of the team that lead the documentation and mobilization of the history, culture and language of Choguita, he focused on interviews with elders as well as the creation of materials to be used in the local school curriculum. Giltro was appointed by local authorities as the custodian of the documentation materials deposited in Choguita and to head the mobilization of
documentation outcomes in the community.

\medskip

\noindent Elder \textbf{†Luz Elena León Ramírez [LEL]} was a master storyteller and an invaluable source of linguistic, cultural and historical knowledge. She graciously contributed many narratives to the corpus, including procedural texts, historical narratives (including contact with members of the N’dee/N’nee/Ndé (Apache) nation and sightings of the armed forces of Pancho Villa in the early twentieth century), descriptions of culturally relevant events in the community (such as an epidemic suffered by the community), and personal history from her childhood. She also collaborated in lexical and grammatical elicitation sessions, and helped translate and annotate her own and others' speakers texts. She was a trained midwife and medical assistant and beloved in Choguita and beyond for her generosity and loving service to her community.

\medskip

\noindent Elder \textbf{†Federico León Pacheco [FLP]} contributed a wealth of historical and cultural knowledge about Choguita and the Rarámuri nation. One of the few remaining \textit{owirúame} (shaman) in the community, he was one of the bearers of specialized knowledge, including the complex kinship system, which he shared in interviews led by community members. He also contributed his knowledge in interviews focusing on the early history of Choguita, his escape from a boarding school in nearby Norogachi in his childhood, and his path of becoming a healing shaman.

\medskip

\noindent \textbf{Francisco Moreno Fuentes [FMF]} was involved in video documentation training and led community-based teams that created records of traditional celebrations and rituals. He also contributed to the project as a consultant in elicitation sessions.

\medskip

Many other people contributed to this grammar indirectly through informal interactions and other forms of support during my time in Choguita. Though
they are not acknowledged by name here, their contribution is deeply appreciated.

\subsection{Representation of examples}
\label{subsec: representation of examples}

Each glossed example minimally provides, from top to bottom: (i) a broad phonetic transcription in IPA; (ii) a phonemicized transcription (also in IPA) with morpheme breaks; (iii) glosses; (iv) an English translation; (v) a {Spanish} translation; and (vi) a source code. Examples may also include underlying phonological representations where relevant. Examples are glossed following the conventions established in the Leipzig Glossing Rules \citep{comrie2008leipzig}. Any departures from this standard are justified where introduced.

A number of examples in the grammar have accompanying audio recordings; these examples are provided with a hyperlink in the source code that directs readers to an open access repository where individual audio files are available for download (\citealt{caballero_gabriela_2022_7268366}). The goal is to allow readers to access the larger contexts from which the grammatical description is based upon, which enables wider dissemination of the results and the ability of interested community members and academics interested in the language to carefully examine the analyses and description set forth in this grammatical description.

All examples include a {Spanish} translation (reflecting the local Northern Mexican variety spoken in the Sierra), the language used for translation elicitation and annotation of texts. {Spanish} translations were given by bilingual language experts of their own utterances or of the words of a different speaker when contributing annotation of texts. When the {Spanish} translation offered by language experts departs from standard Northern Mexican {Spanish}, a more standard {Spanish} translation is supplied and is accompanied by the language expert's verbatim translation, enclosed in double quotation marks. Information included in the translation that is inferred from the context and noted as such by contributing language experts is provided in parentheses. Example sentences are translated by me or by native English speaker student research assistants.

The source code is provided in angled brackets in the last line and provides the contributor’s initials and a unique identifier that links the particular example to the documentary corpus. If the source example is taken from handwritten field notes, this unique identifier will reference the year the example was recorded, book and page number (e.g., `05 1:125'). Examples from written field notes also identify the type of document where the data comes from as follows: elicitation (`el'), text (`tx'), interview (`in'), or conversations (`co'). If the source example has been annotated in a digital format, the unique identifier will reference the label of the source document.\footnote{This is an automatically generated identifier generated in \textit{Kwaras}, an ELAN corpus management tool created by Russell Horton (Linguistics MA 2012, UCSD) and further developed by Lucien Carroll (Linguistics PhD 2015, UCSD) \citep{caballero2019accessing}. This unique identifier is derived from the source file name (e.g., ‘co12-37’) plus the time stamp of the annotation referenced (e.g., ‘0:49.6’).} When available, a hyperlink in the source code enables retrieval of an audio file from an open access repository (\citealt{caballero_gabriela_2022_7268366}). For examples where a corresponding audio file is available, the source code is provided in curly brackets and gray font. An example of how data is cited is illustrated in (\ref{ex: sample citation form}):

\ea\label{ex: sample citation form}

    \ea[]{
    \textit{muˈhê   taˈmí   saˈpâto  raˈrèma}\\
    \gll    muˈhê   taˈmí   saˈpâto  raˈr-è-ma\\
            2\textsc{sg.nom} \textsc{1sg.acc} shoes buy-\textsc{appl-fut.sg}\\
    \glt    ‘You’ll buy shoes from me.’ \\
    \glt    ‘Me vas a comprar zapatos (que vendo yo).’ < SFH 05 1:74/el > \\
}\label{ex: sample citation forma}
        \ex[]{
        \textit{neˈhê	  	ˈpé   	oˈkwâ 	raʔiˈtʃ͡âma 	koriˈmá 	ˈhîtara}\\
        \gll    neˈhê 	  	ˈpé   	oˈkwâ 	raʔiˈtʃ͡â-ma 	koriˈmá 	ˈhîtara\\
	           1\textsc{sg.nom} 	just 	couple 	speak-\textsc{fut.sg} 	fire.bird	about\\
    	\glt    ‘I’ll speak a little about the \textit{korima} (the fire bird).’\\
    	\glt    ‘Yo voy a hablar poquito del pájaro \textit{korimá} (el pájaro de fuego).’   \corpuslink{tx5[00_229-00_264].wav}{LEL tx5:00:22.9}\\
        }\label{ex: sample citation formb}
    \z
\z

In (\ref{ex: sample citation forma}), the example sentence was uttered by SFH in 2005 in an elicitation context (`el'), and the handwritten notes of that session are located in book 1, page 74. In (\ref{ex: sample citation formb}), the example sentence was uttered by LEL; the code next to the contributing speaker's initials is the unique identifier of that example, derived from the source file name (a text labelled `tx5') followed by the time stamp of the annotation referenced (`0:22.9’).\footnote{In many cases, authors themselves contributed to annotation and transcription of their own speech. In cases where a different language expert collaborated with me in transcribing and translating the texts, only the author of the text referenced is credited in the source code.} In this example, the source code is represented in curly brackets and grey font since it contains a hyperlink that directs users to an audio file corresponding to this data example available for download.


Where an example presents an excerpt from a dialogue, each speaker is identified by their initials at the start of the transcription line, as shown in (\ref{ex: example of dialogue}).

\largerpage

\ea\label{ex: example of dialogue}

    \ea[]{
    [ME] \textit{boniˈlâ ...}\\
    \gll    boni-ˈlâ\\
            be.younger.brother-\textsc{poss}\\
    \glt    `Younger brother...'\\
    \glt    `Hermano menor...'\\
}
    %\newpage
        \ex[]{
        [SFH] \textit{boˈnêsa ba?}\\
        \gll    boˈn-ê-sa ba?\\
                younger.brother-\textsc{-have-cond} \textsc{cl}\\
        \glt    `As if he were a younger brother?'\\
        \glt    `¿Como si fuera hermano menor?'\\
    }
            \ex[]{
            [ME] \textit{\textbf{uˈrí}, boniˈlâ ˈnísa ˈlá ba ˈni}\\
            \gll    uˈrí, boni-ˈlâ ˈní-sa oˈlá ba ˈni\\
                    yes be.younger.brother-\textsc{poss} \textsc{cop-cond} \textsc{cer} \textsc{cl} \textsc{emph}\\
            \glt    `Yes, as if he were a younger brother indeed.'\\
            \glt    `Si, si fuera hermano menor, asi es.' \corpuslink{in485[02_229-02_252].wav}{ME in485:02:22.9}\\
        }
    \z
\z

Highly common phrases or expressions are given without a source. These are phrases or expressions that are too common to be ascribed a single source.

All examples provided in this grammar reflect idiolectal variation, which is particularly abundant in terms of pronunciation and various phonological reduction processes. Thus, there is no normalization of transcriptions to reflect any single pronunciation as ``standard". Where a particular form illustrates a particular process only attested in the speech of one or few language experts, the divergence is usually illustrated with an underlying phonological representation and a footnote that includes a cross-reference to the chapters/sections addressing those specific processes.

Finally, the reader should not assume that the example sentences in this work accurately reflect the culture, interests, priorities or personalities of the speakers that uttered them: a large amount of examples used to illustrate patterns and phenomena here have been obtained in contexts where the goal was to illustrate structural aspects of the language, and are not necessarily sociologically or ecologically representative.

\section{Overview of the grammar}
\label{sec: overview of the grammar}


Description in this grammar has been organized in a way that aims to facilitate discovery of particular topics in the grammar of Choguita Rarámuri by typologists and linguists interested in Choguita Rarámuri or languages of this region. This book has features of the predominant ascending macrostructure organization of grammars (phonology > morphology > syntax), where description begins with chapters devoted to the sound system to then move to increasingly complex units of analysis (words, phrases, clauses and sentences).

This structure or any organization where domains of linguistic structure are compartmentalized into discrete chapters, however, presents challenges in the organization of the description of linguistics systems, especially with respect to aspects of structure that cross-cut all levels of grammar, such as prosody \citep{mosel2006grammaticography}. As shown in this grammar, the prosodic system of Choguita Rarámuri involves complex interactions between lexical, post-lexical and grammatical information that require discussion in its own right. Confronted with this challenge, this grammar includes a chapter where interactions of prosodic patterns and processes across grammatical domains are addressed in detail with the goal of providing a clear picture of how these complex interactions yield surface forms in this language. This supplements the description that pertains to each individual domain of description, such as description to the lexical role of tone in the chapter devoted to tone and intonation and description of the grammatical role of tone in chapters devoted to morphology. Thus, there is some degree of redundancy built into the description, which seeks to enable different contextualization of the data and the analyses provided.
%Prosodically annotated paradigms are provided in Appendix 3, which illustrate the surface illustration of different verb classes with different lexical and grammatical prosodic patterns.

Like most reference grammars, this volume also adopts a semasiological, form-to-function organization, which means that related semantic or conceptual domains may be addressed in different chapters depending on how they are morphosyntactically encoded. There are abundant cross-references provided in the grammar that are intended to help readers identify topics that may be semantically or conceptually related.

This chapter concludes with a summary of each of the following chapters of this grammar.

\medskip
\noindent \textbf{\chapref{chap: grammatical overview}} lays out the core structures of the language, spanning phonological patterns and processes, morphological contrasts encoded by pronouns and demonstratives and their function, morphosyntactic and morpho-phonological properties of nouns and verbs, word order patterns, appositive possessive constructions, relative clauses, complement clauses, clause chaining, and complex predicates.

\medskip

\noindent \textbf{\chapref{chap: phonology}} addresses the segmental phonology of Choguita Rarámuri. First, it provides the phonological inventory and complex patterns of allophonic variation. This is followed by a description of the various phonological processes, including processes that target labio-velar semi-vowel and voiced bilabial stops, alveopalatal affricates, alveolar fricatives, nasals and rhotics. This chapter includes description of post-consonantal devoicing and voiceless plosive lenition processes.

\medskip

\noindent \textbf{\chapref{chap: syllables}} describes the syllabification patterns of Choguita Rarámuri, as several suprasegmental processes make crucial reference to the syllabic structure of words. This chapter includes discussion of underlying syllable structure patterns, as well as surface consonant and vowel sequences.

\medskip

\noindent \textbf{\textbf{\chapref{chap: word prosody}}} addresses the word-level stress system of Choguita Rarámuri, including acoustic and distributional properties of stress and stress-dependent phenomena. There is also detailed consideration of the lexical stress properties of both roots and suffixes and description of the typologically unusual initial three syllable stress window that restricts the location of stress in the language.

\medskip

\noindent \textbf{\chapref{chap: tone and intonation}} is devoted to tone and intonation. This includes a detailed description of the tonal inventory and its acoustic encoding, tone melodies by root types and interaction with word-level stress, as well as stress-based tonal neutralization. This is followed by description of the intonational characteristics of declarative sentences, tone-specific intonation patterns and non-tonal encoding of intonation.

\medskip

\noindent \textbf{\chapref{chap: other word-level suprasegmental phonology}} outlines other word-level suprasegmental processes of Choguita Rarámuri that contribute to its prosodic complexity. These include restrictions on the distribution of glottal stops in an initial disyllabic window, minimal word size restrictions and loanword prosodic adaptation patterns.

\medskip

\noindent \textbf{\chapref{chap: nominal morphology}} describes the morphology of nouns. First, it addresses the morphotactic generalizations of morphologically complex nouns. This is followed by description of each of the nominal morphological categories, including plural\slash pluractionals, case, possessive marking, and derivational morphological proces\-ses, including agentive, patientive and experiencer nominalizations and abstract noun nominalizations. This chapter concludes with discussion of adaptation of {Spanish} loan nouns and tone patterns in morphologically complex nouns.

\medskip

\noindent \textbf{\chapref{chap: verbal morphology}} provides a detailed overview of the verbal morphology and morphol\-o\-gic\-ally-conditioned phonological processes of Choguita Rarámuri. First, this chapter details the distinctions between verbal root classes in terms of their stress properties in different morphological environments, as well as their transitivity properties. This is followed by a comprehensive description of grammatical tone patterns in morphologically complex words. This chapter also addresses evidence for positing morphological domains in a layered, hierarchical structure of the verb; the evidence reviewed includes phonological processes that apply in specific verbal domains and morphotactic evidence for suffix ordering generalizations and patterns of variable affix order. This chapter concludes with discussion of the verbal complex, which includes clitics and modal particles.

\medskip

\noindent \textbf{\chapref{chap: particles, adverbs and other word classes}} describes the morphological properties of minor word classes, including pronouns, demonstratives, adjectives, numerals, quantifiers, definite articles, adverbs and discourse particles and enclitics. The discussion is organized on a classification of these word classes into two groups, depending on whether they may head noun phrases or combine with head nouns in noun phrases and those that cannot.

\medskip

\noindent \textbf{\chapref{chap: prosody}} is devoted to prosodic structures and processes that cross-cut the grammar of Choguita Rarámuri, including lexical phonological processes, morphological processes and post-lexical phonological phenomena. First, this chapter outlines the criteria for determining the Prosodic Word in Choguita Rarámuri, and discusses the domains for phonological and morpho-phonological processes below the level of the Prosodic Word. This is followed by discussion of phonological phenomena that are quantity-sensitive in the language vis-à-vis the lack of contrastive vowel length in the language. This chapter also addresses complex prosodic interactions: (i) between stress and tone and between lexical tone and grammatical tone in morphologically complex words, and (ii) between lexical tone and intonation, which includes non-tonal encoding. This chapter concludes with a discussion of prosodic constraints associated with different morphological constructions.

\medskip

\noindent \textbf{\chapref{chap: noun phrases}} examines how nouns and various forms addressed in \chapref{chap: particles, adverbs and other word classes} combine in forming noun phrases in Choguita Rarámuri. This chapter organizes the description in terms of each of the possible modifiers of nouns, detailing restrictions on co-occurrence, agreement patterns and word order properties. This chapter includes discussion of simple noun phrases, which contain a nominal head and a single modifier, and noun phrases that involve possessive constructions.

\medskip

\noindent \textbf{\chapref{chap: basic clause types}} characterizes basic clause types in terms of their transitivity properties (intransitive, transitive and ditransitive), as well as locative clauses, copular clauses and existential clauses. This chapter includes detailed discussion of the characteristics of clauses headed by postural verbs, and considers the semantics of postural verb themselves and the postural constructions they head.

\medskip

\noindent \textbf{\chapref{chap: sentence types}} moves on to consider the morphosyntactic properties of different types of sentence types, namely interrogative, negative, imperative and comparative constructions. The prosodic properties of interrogatives are assessed in relation to declarative sentences.

\medskip

\noindent \textbf{\chapref{chap: clause combining in complex sentences}} details the complex clause structures found in Choguita Rarámuri. This includes discussion of complement clauses, adverbial clauses, relative clauses as well as clausal conjunction, disjunction and adversative coordination. This chapter also provides a description of complex predicate constructions, which includes light verb, auxiliary, serial and V-V incorporation constructions. The properties of constructions with depictive and resultative semantics are also addressed.
