\chapter{Complex clauses and complex predication}
\label{chap: clause combining in complex sentences}

%Note that the system in CR looks more like a canonical logophoric system, not a switch reference one (though note the discussion in van Gijk about non-canonical SR being similat to switch reference

This chapter is concerned with the ways in which basic clauses may be extended through subordination, coordination and other devices in Choguita Rarámuri. The constructions addressed in this chapter are assumed to involve complex relationships between distinct events. These complex relationships are generally encoded cross-linguistically through subordination and coordination structures.

Three main types of subordinate clause constructions are discussed in the first section of this chapter: complement clauses in §\ref{sec: complement clauses}, adverbial clauses in §\ref{sec: adverbial clauses}, and relative clauses in §\ref{sec: relative clauses}). This classification depends on the main functions that subordinate clauses may have: to serve as the complement of a verbal predicate (complementation) or to modify a verb phrase or a whole clause in the periphery of a matrix clause (adverbial subordination). Choguita Rarámuri also exhibits other specialized complex clause constructions, including indirect causative constructions (§\ref{subxsec: indirect causatives}) and complements of speech verbs (§\ref{subsec: complementation with speech verbs}).  A class of verbs, namely reportative verbs, exhibit switch reference marking in the complement clauses they head; these types of clauses are addressed in §\ref{subsec: reportative clauses}.

This chapter also addresses clausal coordination constructions, the type of syntactic constructions where two or more constituents that have equivalent status are combined into a larger syntactic unit and still have the same semantic relations with other surrounding elements (\citealt{haspelmath2007coordination}). Choguita Rarámuri clause coordination constructions include conjunction (§\ref{subsec: conjunction}), disjunction (§\ref{subsec: disjunction}) and adversative conjunction (§\ref{subsec: adversative conjunction}).

Another type of complex construction attested in the language addressed in this chapter involves verbal chaining structures (§\ref{sec: clause chaining}), where two or more predicates may convey that some events take place simultaneously or in a temporal sequence. Additionally, these constructions may imply associated semantic meanings related to chronological overlap and succession, such as cause and effect, result, and manner.

Finally, this chapter addresses constructions that may be analyzed as involving complex predicates, including light verb constructions (§\ref{subsec: light verb and auxiliary constructions}), auxiliary verb constructions (§\ref{subsec: auxiliary verb constructions}), serial verb constructions (§\ref{subsec: serial verb constructions}) and V-V incorporation constructions(§\ref{subsec: V-V incorporation constructions}).

%Quoted speech constructions include indirect quotes (§\ref{subsubsec: indirect citation}) and direct quotes (§\ref{subsubsec: direct citation}).

\section{Complement clauses}
\label{sec: complement clauses}

% define complementation
% state structural types of complementation

Complementation involves syntactic configurations where a notional sentence or predication is an argument of a matrix predicate (\citealt{noonan2007complementation}). Complement clauses in Choguita Rarámuri are identified as such if the complement clause fulfills a syntactic argument of the matrix clause. In most instances, the complement clause is a syntactic argument of the matrix predicate; in the case of Choguita Rarámuri, complement clauses involve a matrix transitive predicate whose object is a complement clause. Four major types of complement clauses can be identified on the basis of their morphosyntactic properties:

\ea\label{ex: types of complement clauses}
{Choguita Rarámuri complement clause types}

    \ea[]{
    Finite complement clauses with complementizer (§\ref{subsec: finite complement clauses with complementizer}).\\
}
        \ex[]{
        Interrogative complement clauses (§\ref{subsec: interrogative final clause}).
    }
            \ex[]{
            Asyndetic finite verb complement clauses (§\ref{subsec: asyndetic finite verb}).
        }
                \ex[]{
                Reduced complement clauses (§\ref{subsec: reduced complement clauses}).
            }
    \z
\z

In addition to these four types of complement clauses, Choguita Rarámuri has specialized constructions that involve complementation, including an indirect causative construction (§\ref{subxsec: indirect causatives}), a reportative clause construction that features switch reference marking (§\ref{subsec: reportative clauses}), and complement clauses of speech verbs (§\ref{subsec: complementation with speech verbs}).
%Choguita Rarámuri complement clauses have a number of defining formal properties, including the optional appearance of a complementizer, the order S [complement clause] main V, and the possibility of having discontinuous elements of complement clauses.

\subsection{Finite complement clauses with complementizer}
\label{subsec: finite complement clauses with complementizer}

This type of complement clause has almost all of the properties of an independent clause and is introduced by a complementizer, the subordinating particle \textit{ˈnápi} or its reduced form \textit{ˈ(n)ápi}, which also is used in other types of subordinate clause constructions (§\ref{sec: adverbial clauses}) and relative clauses (§\ref{sec: relative clauses}). This type of complement clause is exemplified in (\ref{ex: finite complement clause with complementizer}). The matrix clause always precedes the complement clause (represented in square brackets in the examples below).

\newpage
\ea\label{ex: finite complement clause with complementizer}

    \ea[]{
    \textit{ˈnè ko ˈá biˈlá ˈwé ˈtʃ͡ó ˈá maˈjêa ˈníli ˈápi ˈhípi roˈkò ˈá noˈrînima}\\
    \gll    ˈnè=ko ˈá biˈlá ˈwé ˈtʃ͡ó ˈá maˈjê-a ˈní-li [\textbf{ˈápi} ˈhípi roˈkò ˈá noˈrîni-ma]\\
            \textsc{1sg.nom=emph} \textsc{aff} indeed \textsc{int} also  \textsc{aff} think-\textsc{prog} \textsc{cop-pst} {\textsc{sub}} today night \textsc{aff} return-\textsc{fut.sg}\\
    \glt    `I also think that he will return tonight.'\\
    \glt    `Yo también pienso que va a venir hoy en la noche.'   \corpuslink{tx32[11_367-11_434].wav}{LEL tx32:11:36.7}\\
}\label{ex: finite complement clause with complementizera}
        \ex[]{
        \textit{ˈká ni maˈtʃ͡íki ˈnápu ˈtòoru ba}\\
        \gll    ˈká=ni maˈtʃ͡í-ki [\textbf{ˈnápi}  ˈtò-ru ba]\\
                \textsc{neg=1sg.nom}  know-\textsc{pst.ego} {\textsc{sub}} take-\textsc{pst.pass} \textsc{cl}\\
        \glt    `I didn’t know he had been taken.’\\
        \glt    `No sabía que se lo habían llevado.’   < BFL 09 1:39/el >\\
        }\label{ex: finite complement clause with complementizerb}
    \z
\z

The finite verb of the complement clause appears in clause final position and is unrestricted in terms of TAM markers available in matrix clauses. In these particular examples, the complement verb is marked as future (\ref{ex: finite complement clause with complementizera}) or may be passivized (\ref{ex: finite complement clause with complementizerb}). The matrix clause and complement clause can also be negated independently from each other: in (\ref{ex: finite complement clause with complementizerb}), only the main predicate is negated, but the complement is in the positive polarity, which points to the relative low degree of integration between the complement and the matrix clause in this type of complement clause structure.

\subsection{Interrogative complement clauses}
\label{subsec: interrogative final clause}

Another type of complement clause involves a subordinate finite clause introduced by an interrogative expression (see §\ref{subsec: interrogative pronouns}). Several verbs of speech and cognition can take this type of complement, as shown in (\ref{ex: interrogative complement clauses}). The interrogative word occurs in clause initial position as in matrix interrogative clauses (§\ref{sec: interrogative constructions}).

\ea\label{ex: interrogative complement clauses}

    \ea[]{
    \textit{ˈétʃ͡i riˈká biˈlá ko matʃ͡iˈnália ˈí ˈnàʔ ko ˈnáp ... tʃ͡ú riˈkám ˈká riˈwèeli tʃ͡aˈbèe}\\
    \gll    ˈétʃ͡i riˈká biˈlá=ko matʃ͡i-ˈnáli-a ˈí ˈnà=ko [\textbf{tʃú} \textbf{riˈká}=mi ˈká riˈwè-li tʃ͡aˈbè]\\
            \textsc{dem} like indeed=\textsc{emph} know-\textsc{desid-prog} here here=\textsc{emph} {\textsc{q} how}=\textsc{dem} \textsc{cop.irr} was.like-\textsc{pst} before\\
    \glt    `She wanted to know how it (life) was here before.'\\
    \glt    `Ella quería saber cómo era (la vida) antes aquí.'   \corpuslink{tx475[00_544-00_586].wav}{SFH tx475:00:54.4}\\
}
        \ex[]{
        \textit{taˈmò ˈkútʃ͡uala ˈtʃ͡ó beˈnèma ˈtʃ͡ó ˈnà ˈtʃ͡útimi riˈká ˈnàwa noˈká ˈnà oˈtʃ͡êla noˈká tʃ͡aˈbè kiˈʔà ba}\\
        \gll    taˈmò ˈkútʃ͡ua-la ˈtʃ͡ó beˈnè-ma ˈtʃ͡ó ˈnà [\textbf{ˈtʃú}=timi riˈká=m ˈnàwa no-ˈká ˈnà oˈtʃ͡ê-la no-ˈká tʃ͡aˈbè kiˈʔà ba]\\
                \textsc{1pl.nom} children-\textsc{poss} also learn-\textsc{fut.sg} also then {how}=\textsc{2pl.nom} indeed=\textsc{dem} arrive.\textsc{prs} do-\textsc{ger} then grow.old-\textsc{rep} do-\textsc{ger} before long.ago \textsc{cl}\\
        \glt    `Our children are also going to learn how you all lived, how you grew up.'\\
        \glt    `Nuestros hijos también van a aprender cómo crecieron antes ustedes, como vivían.’    \corpuslink{in61[00_467-00_489].wav}{SFH in61:00:46.7}, \corpuslink{in61[00_492-00_531].wav}{SFH in61:00:49.2}\\
    }
            \ex[]{
            \textit{ˈmá ˈmí baʔaˈrîna ˈmí riˈkáatʃ͡i ˈmá aʔˈlá matʃ͡iˈsâa ˈká ˈtʃ͡i ˈníria ˈétʃ͡i ˈtáa ba}\\
            \gll    ˈmá ˈmí baʔaˈrîna ˈmí riˈkátʃ͡i ˈmá aʔˈlá matʃ͡i-ˈsâ ˈká [\textbf{ˈtʃú} ˈnír-a ˈétʃ͡i ˈtá ba]\\
                    already \textsc{dem} next.day \textsc{dem} perhaps already well know-\textsc{cond} \textsc{cop.irr} {how} feel-\textsc{prog} \textsc{def} small.one \textsc{cl}\\
            \glt    `And then already the next morning I think he knows how the child is feeling.'\\
            \glt    `Ya a la mañana siguiente yo creo que va a saber qué siente el niño.'    \corpuslink{tx475[02_447-02_487].wav}{SFH tx475:02:44.7}\\
        }
    \z
\z

As described above for declarative complement clauses with complementizers (§\ref{subsec: finite complement clauses with complementizer}), the matrix clause always precedes the complement clause. All properties identified for declarative finite complement clauses with complementizers hold for embedded questions, including the lack of restrictions in terms of TAM marking in the complement clause. As shown in the following example, negation also applies independently in the matrix clause and the complement clause: in (\ref{ex: negation in embedded question}), the negative particle in sentence initial position does not have scope over the complement clause.

\ea\label{ex: negation in embedded question}

    \textit{ke piˈláti biˈlé matʃ͡iˈká moˈtʃ͡íli taˈmò ˈká tʃ͡è wiˈlé ˈtʃ͡ó matʃ͡iˈwá aʔˈlì ko ba ˈpîri ˈnà ta naˈjúka suwiˈjá ba}\\
    \gll    ke biˈlá=ti biˈlé matʃ͡i-ˈká moˈtʃ͡í-li taˈmò ˈká ˈtʃ͡è biˈlé ˈtʃ͡ó matʃ͡i-ˈwá aʔˈlì=ko ba [\textbf{ˈpîri} ˈnà ta naˈjú-ka suwi-ˈjá ba]\\
            \textsc{neg} indeed=\textsc{1pl.nom} one know-\textsc{ger} sit.\textsc{pl-pst} \textsc{1pl.nom} because \textsc{neg} then also know-\textsc{mpass} and=\textsc{emph} \textsc{cl} {what} \textsc{dem} \textsc{1pl.nom} be.sick-\textsc{ger} finish.off-\textsc{prog} \textsc{cl}\\
    \glt    `We were without knowing, us, because then it wasn't known yet what it was (that was) making us sick and dying (lit. `being finished off').'\\
    \glt    `Pues nosotros estábamos sin saber, porque en ese tiempo todavía no se sabía nada de qué nos enfermábamos y nos moríamos.'  \corpuslink{tx372[00_414-00_459].wav}{LEL tx372:00:41.4}, \corpuslink{tx372[00_474-00_507].wav}{LEL tx372:00:47.4}\\

\z

\subsection{Asyndetic finite verb complement constructions}
\label{subsec: asyndetic finite verb}
\largerpage
Another type of complement clause in Choguita Rarámuri involves asyndetic subordination, where the matrix and subordinate clause are juxtaposed without any overt complementizer. These constructions are exemplified in (\ref{ex: asyndetic finite verb construction}).
%Complement clauses without overt complementizers may be finite or non-finite complements.

\ea\label{ex: asyndetic finite verb construction}

    \ea[]{
    \textit{ˈnàri ˈkútʃ͡i iˈwé ˈmá ... ˈwé ˈá maˈtʃí ˈtʃ͡ó peˈlâto paˈkóa}\\
    \gll    ˈnàri ˈkútʃ͡i iˈwé ˈmá ˈwé ˈá maˈtʃí ˈtʃ͡ó [peˈlâto paˈkó-a]\\
            then young.\textsc{pl} girls already \textsc{int} \textsc{aff} know.\textsc{prs} also plates wash-\textsc{prog}\\
    \glt    `And then the girls know very well how to wash the dishes.'\\
    \glt    `Y las niñas saben muy bien lavar los platos.’    \corpuslink{tx73[02_134-02_185].wav}{LEL tx73:02:13.4}\\
}
        \ex[]{
        \textit{ˈnè     ko   ˈpé   ke   maˈjê   reˈmê   naˈhíbo naˈʔî} \textit{oˈnátʃ͡i}\\
        \gll    ˈnè=ko  ˈpé  ke  maˈjê [reˈmê naˈhí-bo naˈʔî oˈnátʃ͡i]\\
                \textsc{1sg.nom=emph} \textsc{neg} \textsc{neg} think.\textsc{prs} tortillas give.away-\textsc{fut.pl} here  here\\
        \glt    `I don’t think they will give away tortillas here.’\\
        \glt    `No creo que vayan a dar tortillas aquí.’    < BFL 09 1:39/el >\\
        }
            \ex[]{
            \textit{riˈwáki   ˈtʃ͡éni ke ˈmónulo ˈtrôkatʃ͡i}\\
            \gll    riˈwá-ki   ˈtʃ͡é=ni [ke ˈmó-nul-o ˈtrôkatʃ͡i]\\
                    see-\textsc{pst.ego} also=\textsc{1sg.nom} \textsc{neg} go.up-\textsc{desid-ep} truck\\
            \glt    `I saw that he didn’t want to get on the truck.’\\
            \glt    `Vi que no se quiso subir a la troca.'   < BFL 09 1:39/el >\\
            }
                    \ex[]{
                    \textit{aʔˈlì ˈmá iʔˈnèli ˈpé ˈmá mukuˈká buˈʔíli}\\
                    \gll    aʔˈlì ˈmá iʔˈnè-li [ˈpé ˈmá muku-ˈká buˈʔí-li]\\
                            and already see-\textsc{pst} little already die-\textsc{ger} lie.down-\textsc{pst}\\
                    \glt    `Then they saw that he was already lying down dead.’\\
                    \glt    `Entonces ya vieron que ya estaba muerto.’   \corpuslink{tx5[02_428-02_458].wav}{LEL tx5:02:42.8}\\
                }
    \z
\z

In asyndetic finite verb complementation, the complement clause may have a different subject from the matrix clause. This is exemplified in (\ref{ex: asyndetic finite verb construction}b--d).

Utterance predicates may also encode complements through the asyndetic finite verb construction, as exemplified in (\ref{ex: utterance verbs complements}).

\ea\label{ex: utterance verbs complements}

    \ea[]{
        \textit{waˈbê biˈlá kiˈʔà ˈníla ... ˈhê biˈlá riˈkám aˈnía  aˈní biˈsá ˈtʃ͡é tʃ͡oˈkêami naˈhùla ˈrá ˈnà kaˈwì}\\
        \gll    waˈbê biˈlá kiˈʔà ˈní-la ˈhê biˈlá riˈká=mi aˈní-a  aˈní [bi-ˈsá ˈtʃ͡é tʃ͡oˈkê-ame naˈhù-la ru-ˈwá ˈnà kaˈwì]\\
                \textsc{int} indeed long.ago \textsc{cop-rep} it indeed like=\textsc{dem} say-\textsc{prog} say.\textsc{prs} three-\textsc{mltp} again begin-\textsc{ptcp} fall.down-\textsc{rep} say\textsc{mpass} this world\\
        \glt    `They say that long, long ago the world began to fall apart three times.'\footnote{In this sentence, the adverb \textit{kiˈʔà} `long ago' modifies the complement clause.} \\
        \glt    `Dicen que mucho antes tres veces se empezó a caer este mundo.'  \corpuslink{tx43[11_182-11_245].wav}{SFH tx43:11:18.2}\\
    }\label{ex: utterance verbs complementsa}
            \ex[]{
            \textit{aʔˈlì ˈhê aˈníli: ``ˈnè ko ˈá raˈè ˈwé biˈlá ˈnà ... ˈwé biˈlá wiʰˈkâ ˈnà ...}\\
            \gll    aʔˈlì ˈhê aˈní-li: [ˈnè=ko ˈá raˈè ˈwé biˈlá ˈnà ˈwé biˈlá wiʰˈkâ ˈnà]\\
                    and it say-\textsc{pst} \textsc{1sg.nom=emph} \textsc{aff} know \textsc{int} indeed \textsc{dem} \textsc{int} indeed many \textsc{dem}\\
            \glt    `and then he said: ``I do know (where he lives), (with) many ...”’\\
            \glt    `y entonces dijo: ``Yo sí conozco (donde vive), (con) muchos...”    \corpuslink{tx32[02_290-02_336].wav}{LEL tx32:02:29.0}\\
        }\label{ex: utterance verbs complementsb}
                \ex[]{
                \textit{aʔˈlì ˈhe aˈnèli: ``ˈwé saˈpù aˈsísika! ˈpîli tʃ͡uˈkú naˈʔî?"}\\
                \gll    aʔˈlì ˈhe aˈn-è-li [ˈwé saˈpù aˈsísi-ka] [ˈpîli tʃ͡uˈkú naˈʔî]\\
                        and it say-\textsc{appl-pst} \textsc{int} hurry get.up-\textsc{imp.sg} what to.be.bent.\textsc{prs} here\\
                \glt    `And then she told him: ``Hurry, get up! What is that sitting here?’\\
                \glt    `Y luego le dijo: ``¡Levántate pronto! ¿Qué hay (está) aquí?”’   \corpuslink{tx5[00_527-00_572].wav}{LEL tx5:00:52.7}\\
            }\label{ex: utterance verbs complementsc}
    \z
\z

As shown in these examples, both indirect (\ref{ex: utterance verbs complementsa}) and direct (\ref{ex: utterance verbs complements}b--c) speech is introduced through this type of construction. More details about the properties of complements of speech verbs are provided in §\ref{subsec: complementation with speech verbs}.

%\subsection{Non-finite complements}
%\label{subsec: non-finite complements}

%are there any?

\subsection{Reduced complement clauses}
\label{subsec: reduced complement clauses}

A fourth type of complement clause involves a reduced complement: there is argument sharing, and complement-taking verbs with this complement clause type take only same subject complements. These verbs are modal and phase verbs and other verbs with similar semantic characteristics. There appears to be no constraint on the transitivity or valence of the complement verb, and any type of subject is eligible for argument sharing. The complement clause is also restricted in terms of the TAM specification of the complement clause, and marked only as progressive or medio-passive. This type of complement clause is exemplified with complements of the phase verbs \textit{ˈtʃ͡ôta} `begin' (\ref{ex: complement verb chota}) and \textit{suˈní} `finish' (\ref{ex: complement verb suni}).

\ea\label{ex: complement verb chota}

    \ea[]{
    \textit{ˈmá ˈtʃ͡ôtali aˈwìa}\\
    \gll    ˈmá ˈtʃ͡ôta-li [aˈwì-a]\\
            already begin-\textsc{pst} dance-\textsc{prog}\\
    \glt    `They started dancing.'\\
    \glt    `Ya empezaron a bailar.'  \corpuslink{co1237[08_121-08_137].wav}{JLG co1237:08:12.1}\\
}\label{ex: complement verb chotaa}
        \ex[]{
        \textit{ˈmán raˈpâko ˈtʃ͡ôtiki ˈʃûa}\\
        \gll    ˈmá=ne raˈpâko ˈtʃ͡ôti-ki [ˈsû-wa]\\
                already=\textsc{int} yesterday start-\textsc{pst.ego} sew-\textsc{mpass}\\
        \glt    `They already started sowing yesterday.'\\
        \glt    `Ya comenzaron a coser ayer.'  \corpuslink{el259[11_549-11_573].wav}{BFL el259:11:54.9}\\
    }\label{ex: complement verb chotab}
    \z
\z

\ea\label{ex: complement verb suni}

    \ea[]{
    \textit{ˈmá ʃuˈníli aˈwìa}\\
    \gll    ˈmá suˈní-li [aˈwì-a]\\
            already finish-\textsc{pst} dance-\textsc{prog}\\
    \glt    `S/he already finished dancing.'\\
    \glt    `Ya terminó de bailar.'    \corpuslink{el1275[00_274-00_290].wav}{JLG el1275:00:27.4}\\
}
        \ex[]{
        \textit{suˈnísa naʔˈpôa paˈtʃî ... muˈnî ko ˈro}\\
        \gll    suˈní-sa [naʔˈpô-a paˈtʃî] muˈnî=ko ˈru\\
                finish-\textsc{cond} weed-\textsc{prog} corn beans=\textsc{emph} say.\textsc{prs}\\
        \glt    `When they finish weed for (planting) corn, then the beans.'\\
        \glt    `Cuando terminan de escardar el maíz, después el frijol.' \corpuslink{co1136[03_332-03_357].wav}{MDH co1136:03:33.2}\\
    }
            \ex[]{
            \textit{suˈnísa ˈʔwîa ˈmá ˈtʃ͡étʃ͡o ku wiˈlíram ke wasaˈráa}\\
            \gll    suˈní-sa [ˈʔwî-a] ˈmá ˈtʃ͡é ˈtʃ͡ó ku wiˈlí-r-ame ke wasaˈrá\\
                    finish-\textsc{cond} harvest-\textsc{prog} already again again \textsc{rev} stand-\textsc{pst.passs-ptcp} \textsc{cop.impf} plow.\textsc{prs}\\
            \glt    `When they finish harvesting, then he'd start plowing again.'\\
            \glt    `Cuando terminaban de pizcar luego se ponía a barbechar otra vez.’    \corpuslink{tx130[04_167-04_196].wav}{LEL tx130:04:16.7}\\
    }
    \z
\z

Both  \textit{ˈtʃ͡ôta} `begin' and \textit{suˈní} `finish' precede the complement clause, as in the other complement types surveyed so far. The verb in the complement clause is marked present progressive (\ref{ex: complement verb chotaa}, \ref{ex: complement verb suni}) or medio-passive (\ref{ex: complement verb chotab}).

Constructions with the modal verb \textit{oˈmèra} `can' also involve a reduced complement clause with the same properties observed for phase predicates like `begin' and `finish' exemplified above. As shown in (\ref{ex: complement verb omera}), the ordering main verb -- complement clause is also attested in complements of this modal verb, and the subject argument is shared by the matrix and complement verbal predicates.

\ea\label{ex: complement verb omera}

    \ea[]{
    \textit{ˈmá ke oˈmèro ripiˈjáti sikiˈréa}\\
    \gll    ˈmá ke oˈmèr-o [ripiˈjá-ti sikiˈré-a]\\
            already \textsc{neg} can-\textsc{ep} knife-\textsc{inst} cut-\textsc{prog}\\
    \glt    `S/he cannot cut with the knife anymore.'\\
    \glt    `Ya no puede cortar con el cuchillo.'    \corpuslink{co1234[00_095-00_117].wav}{JLG co1234:00:09.5}\\
}\label{ex: complement verb omeraa}
        \ex[]{
        \textit{ˈmá ke oˈmèra raˈʔìtʃ͡a, ˈwé naˈjúami ˈú}\\
        \gll    ˈmá ke oˈmèr-a [raˈʔìtʃ͡a] ˈwé naˈjú-ame ˈhú\\
                anymore \textsc{neg} can-\textsc{prog} speak.\textsc{prs} \textsc{int} be.sick-\textsc{ptcp} \textsc{cop.prs}\\
        \glt    `He can't speak anymore, he's very sick.'\\
        \glt    `Ya no puede hablar, está muy enfermo.'    \corpuslink{co1235[05_029-05_055].wav}{JLG co1235:05:02.9}\\
    }\label{ex: complement verb omerab}
            \ex[]{
            \textit{ke oˈmèra ˈsèbia ba ku ˈétʃ͡i ˈápi baˈtʃ͡ámi maˈwá    }\\
            \gll    ke oˈmèra [ˈsèbi-a ba ku ˈétʃ͡i ˈápi baˈtʃ͡ámi ma-ˈwá]\\
                    \textsc{neg} can.\textsc{prs} reach-\textsc{prog} \textsc{cl} \textsc{rev} \textsc{dem} \textsc{sub} first run-\textsc{mpass}\\
            \glt    `She can’t reach whoever runs first.’\\
            \glt    `No puede alcanzar a quien vaya delante.’    \corpuslink{tx19[03_269-03_297].wav}{LEL tx19:03:26.9}\\
        }\label{ex: complement verb omerac}
                \ex[]{
                \textit{ˈmá ke oˈmèrili ˈtʃ͡ó ˈmêa ˈtʃ͡ó ˈnápi niˈhê wiˈláli ko ba}\\
                \gll    ˈmá ke oˈmèri-li ˈtʃ͡ó [ˈmê-a ˈtʃ͡ó] ˈnápi niˈhê wiˈlá-li=ko ba\\
                        already \textsc{neg} can-\textsc{pst} either win-\textsc{prog} either \textsc{sub} \textsc{1sg.nom} stand.\textsc{tr-pst=emph} \textsc{cl}\\
                \glt    `She couldn’t win, the one I chose (lit. the one I stood, appointed).’\\
                \glt    `No pudo ganar la que puse yo (lit. la que yo paré).'   \corpuslink{tx19[04_280-04_356].wav}{LEL tx19:04:28.0}\\
            }\label{ex: complement verb omerad}
    \z
\z

The example in (\ref{ex: complement verb omerad}) shows that the main verb and the subordinate clause need not be adjacent in these complement clauses, as the main verb may be followed by an adverb modifying it (\textit{ˈtʃ͡ó} `also'). The examples below show reduced complement clauses where the shared subject argument (\ref{ex: overt subject omeraa}), other adverbs (\ref{ex: overt subject omerab}), and object noun phrases (\ref{ex: overt subject omerac}) may be realized inside the complement clause.

\ea\label{ex: overt subject omera}

    \ea[]{
    \textit{aʔˈlì biˈlá ˈnà ˈá riˈká ke oˈmèaki ti roˈplân riˈkîina}\\
    \gll    aʔˈlì biˈlá ˈnà ˈá riˈká ke oˈmèa-ki [ti roˈplân riˈkî-na]\\
            and really then \textsc{aff} like \textsc{neg} be.able-\textsc{ego} \textsc{plane} plane down-\textsc{vblz.prs}\\
    \glt    `Ya no pudo bajar el avión.’\\
    \glt    `The plane could not go down.”    \corpuslink{tx12[02_107-02_147].wav}{SFH tx12:02:10.7}\\
}\label{ex: overt subject omeraa}
        \ex[]{
        \textit{aʔˈlì ke me oˈmèra paˈtʃ͡ána siˈmíra}\\
        \gll    aʔˈlì ke me oˈmèra [paˈtʃ͡á-na siˈmíra]\\
                and \textsc{neg} almost can.\textsc{prs} inside-\textsc{all} pass.\textsc{prs}\\
        \glt    `But he couldn't get all the way inside (lit. `pass inside')'\\
        \glt    `Pero no pudo meterse todo adentro (lit. `pasar adentro')’   \corpuslink{tx177[00_452-00_489].wav}{LEL tx177:00:45.2}\\
    }\label{ex: overt subject omerab}
            \ex[]{
            \textit{aʔˈlì ke oˈmèrali ˈnà ku matʃ͡iˈbûa moˈʔôla}\\
            \gll    aʔˈlì ke oˈmèra-li [ˈnà ku matʃ͡i-ˈbû-a moˈʔô-la]\\
                    and \textsc{neg} can-\textsc{pst} \textsc{prox} \textsc{rev} outside-\textsc{vblz-prog} head-\textsc{poss}\\
            \glt    `And he couldn't take his head out.'\\
            \glt    `Y no pudo sacar la cabeza.’   \corpuslink{tx177[01_488-01_585].wav}{LEL tx177:01:48.8}\\
        }\label{ex: overt subject omerac}
    \z
\z

There are also instances where the shared subject argument of the main and complement clause may be postposed, as shown in (\ref{ex: post-posed subject}).\footnote{\citet{moralesmoreno2016rochecahi} analyzes similar constructions in \ili{Rochéachi Rarámuri} as instances of right-hand dislocation to encode a topical argument. Given (\ref{ex: post-posed subject}) involves a third person argument, an expected co-referential pronominal form in the clause is not attested.}

\ea\label{ex: post-posed subject}

    \textit{ke ˈtâsi ˈtʃ͡ó oˈmèri biˈhí raˈʔìtʃ͡a ˈétʃ͡i ko}\\
    \gll    ke ˈtâsi ˈtʃ͡ó oˈmèra [biˈhí raˈʔìtʃ͡a] ˈétʃ͡i=ko\\
            \textsc{neg} \textsc{neg} also can.\textsc{prs} yet speak.\textsc{prs} \textsc{dem=emph}\\
    \glt    `She can't talk yet that one.'\\
    \glt    `Todavía no puede hablar esa.'   \corpuslink{co1235[06_031-06_055].wav}{JLG co1235:06:03.1}\\

\z

Finally, these phase and modal complement-taking predicates are also attested without an overt complement, i.e., as intransitive predicates. This is shown in (\ref{ex: modal and phase verbs with no complement}). As shown in these examples, the lexical meaning of these predicates is the same, whether they host a complement clause or not.

\ea\label{ex: modal and phase verbs with no complement}

    \ea[]{
   \textit{baʔaˈrîn ˈtʃ͡ôtima ˈri}\\
   \gll     baʔaˈrî=ni \textbf{ˈtʃôti-ma} ˈri\\
            tomorrow=\textsc{1sg.nom} begin-\textsc{fut.sg} \textsc{emph}\\
    \glt    `Tomorrow I will start.'\\
    \glt    `Mañana voy a comenzar.'   \corpuslink{el259[12_425-12_443].wav}{BFL el259:12:42.5}\\
}
        \ex[]{
        \textit{ˈmá biˈlán ʃuˈníma ˈa riˈké pa}\\
        \gll    ˈmá biˈlá=ni \textbf{suˈní-ma} ˈa riˈké pa\\
                already indeed=\textsc{1sg.nom} finish-\textsc{fut.sg} \textsc{aff} would \textsc{cl}\\
        \glt    `I was about to finish.'\\
        \glt    `Ya iba a acabar yo.'  \corpuslink{co1137[09_430-09_450].wav}{MDH co1137:09:43.0}\\
    }
            \ex[]{
            \textit{ˈátʃ͡i ˈlé oˈmèli ˈmá ba? ˈjén ˈá aˈlé oˈmèra ˈlé ˈa, ˈtòa ˈa}\\
            \gll    ˈátʃ͡i aˈlé \textbf{oˈmèli} ˈmá ba ˈjén ˈá aˈlé \textbf{oˈmèra} aˈlé ˈá ˈtò-a ˈa\\
                    Q \textsc{dub} can.\textsc{prs} also \textsc{cl} \textsc{aff} \textsc{aff} \textsc{dub} can.\textsc{prs} \textsc{dub} \textsc{aff} take-\textsc{prog} \textsc{aff}\\
            \glt    `Who knows if he'd be able to (take him on)? I think he can (take him on), take him.'\\
            \glt    `¿Quién sabe si lo podrá? Yo creo que si lo puede, a lo mejor si puede llevárselo.'   \corpuslink{co1136[08_572-09_019].wav}{MDH co1136:08:57.2}\\
        }
    \z
\z

\citet{villalpando2019grammatical} describes several auxiliary verbs in \ili{Norogachi Rarámuri}, which he proposes have been recently grammaticalized from matrix predicates and encode aspectual meanings in auxiliary verb constructions. Two of such verbs have cognates in Choguita Rarámuri that are identified here as involving complementation instead: (i) constructions with the verb \textit{ˈtʃ͡ôta} `begin'; and (ii) constructions with the verb \textit{suˈní} `finish'. It is argued here that the phase and modal verbs in Choguita Rarámuri described in this section still exhibit the properties of full lexical predicates that take reduced complement clauses and have not yet undergone any process of auxiliation. Choguita Rarámuri auxiliary verb constructions are described in §\ref{subsec: auxiliary verb constructions}.

\subsection{Indirect causative construction}
\label{subxsec: indirect causatives}

Choguita Rarámuri possesses a distinct class construction encoding indirect causation. The contrast between direct and indirect causation in Choguita Rarámuri is encoded morphosyntactically: a co-lexicalized structure entails physical contact between the causer and the causee (\ref{ex: direct vs. indirect causativea}), while a periphrastic construction (exemplified in (\ref{ex: direct vs. indirect causativeb})) entails that the causee may act on its own, with varying degrees of volition.

\ea\label{ex: direct vs. indirect causative}

    \ea[]{
    \textit{niˈhê toˈwí koˈʔátili}     \\
    \gll    niˈhê toˈwí koˈʔá-ti-li\\
            1\textsc{sg.nom} boy eat-\textsc{caus-pst} \\
    \glt    `I fed the boy’ (forcing the spoon into his mouth).’ \\
    \glt    `Hice comer al niño (dándole con una cuchara en la boca).’ \\
}\label{ex: direct vs. indirect causativea}
        \ex[]{
        \textit{niˈhê toˈwí koˈʔánula nuˈlèki}\\
        \gll    niˈhê toˈwí koˈʔá-nula nuˈl-è-ki\\
                1\textsc{sg.nom} boy eat-\textsc{order} order-\textsc{appl-pst.ego}\\
        \glt    `I ordered the boy to eat.’\\
        \glt    `Hice comer al niño (le ordené que comiera).’ \\
    }\label{ex: direct vs. indirect causativeb}
    \z
\z

Indirect causative constructions in Choguita Rarámuri involve a periphrastic construction in which a main jussive predicate takes the caused event as a complement. Some properties that characterize this structure are: (i) the lower verb is additionally marked with the jussive verbal affix \textit{nula} ‘order, command’ deriving a co-lexicalized structure within the complement; (ii) although there are two causative verbs, the causer is expressed only once, as a core argument of the matrix predicate, i.e. the two jussive verbs share the actor. This is exemplified in (\ref{ex: indirect causative construction examples}).\footnote{In (\ref{ex: indirect causative construction examplesd}), there are three clauses: the matrix clause headed by \textit{maˈjê} `believe', a complement clause headed by the jussive predicate \textit{nuˈlè} `to order', and the lower clause headed by the predicate \textit{miˈʔà} `to kill'.}

\ea\label{ex: indirect causative construction examples}

    \ea[]{
    \textit{ˈémi    taˈmí aˈnèki niˈhê ˈtònula}  \\
    \gll    ˈémi taˈmí aˈn-è-ki [niˈhê \textbf{ˈtò-nula}]\\
            \textsc{2pl.nom} \textsc{1sg.acc} tell-\textsc{appl-pst.ego} \textsc{1sg.nom} take-\textsc{order}\\
    \glt    `You all told me to take it.’\\
    \glt    `Ustedes me hicieron que me lo llevara’  \\
}\label{ex: indirect causative construction examplesa}
%\pagebreak
        \ex[]{
        \textit{ˈhuâni  ko  taˈmí toˈlí ˈpónula nuˈlè} \\
        \gll    ˈhuâni=ko  taˈmí [toˈlí \textbf{ˈpó-nula}] nuˈl-è\\
                Juan=\textsc{emph} \textsc{1sg.acc} chicken pluck-\textsc{order} order\textsc{-appl.prs}\\
        \glt    `Juan makes me pluck the chicken.’\\
        \glt    `Juan me hace que desplume pollos.’   \\
    }\label{ex: indirect causative construction examplesb}
            \ex[]{
            \textit{ˈpêgro   ko nuˈlèli miˈʔàsinula toˈlí ˈhuân}\\
            \gll    ˈpêgro=ko nuˈl-è-li [\textbf{miˈʔà-si-nula} toˈlí ˈhuân]\\
                    Pedro=\textsc{emph} order-\textsc{appl-pst} kill.\textsc{sg-mot-order} chicken Juan\\
            \glt    `Pedro ordered Juan to go along killing chickens.’\\
            \glt    `Pedro hizo que Juan fuera matando pollos ’  \\
        }\label{ex: indirect causative construction examplesc}
                \ex[]{
                \textit{ˈpêgro   ko   nuˈlèlo   maˈjêli   miˈʔànula   toˈlí   ˈhuân}\\
                \gll    ˈpêgro=ko nuˈl-è-l-o maˈjê-li [\textbf{miˈʔà-nula} toˈlí ˈhuân]\\
                        Pedro=\textsc{emph}  order-\textsc{appl-pst-ep}  believe-\textsc{pst}  kill-\textsc{order}  chicken Juan\\
                \glt    `Pedro thought that he ordered Juan to kill the chicken.'\\
                \glt    `Pedro pensó que le había ordenado a Juan matar el pollo.’    < BFL 06 4:145-146, 148 >\\
            }\label{ex: indirect causative construction examplesd}
    \z
\z

In these constructions, the causee may be expressed once, e.g. as an accusative argument in the main clause in (\ref{ex: indirect causative construction examplesb}), or twice, as an accusative in the main clause and nominative within the lower clause in (\ref{ex: indirect causative construction examplesa}). The lower verb cannot bear any inflection marker (though derivational morphology is possible as in (\ref{ex: indirect causative construction examplesc})) nor can it be modified by temporal adverbs or negation. The position of the dependent unit also varies, i.e. it may appear extraposed to the right (\ref{ex: indirect causative construction examplesa}), but the sentence can display the canonical word order of main clauses, when the unit is embedded and followed by the matrix predicate or causee in clause-final position, as in (\ref{ex: indirect causative construction examples}b--c).
%two complement clauses: the complement of \textit{mayé-li} ‘think’, and the complement of the jussive predicate with the indirect causative: the

While any jussive predicate may be the matrix predicate in indirect causatives, giving different shades to the manipulative force, only the predicate \textit{nula} may appear co-lexicalized with the lower predicate. This is exemplified in (\ref{ex: different jussive predicates}).

\ea\label{ex: different jussive predicates}

    \ea[]{
    \textit{ˈʔwînula ˈtʃ͡é ˈtâsa riˈké la ˈró}\\
    \gll    [ˈʔwî-\textbf{nula} ˈtʃ͡é] ˈtâ-sa riˈké la ˈró\\
                harvest-\textsc{{order}}  again  ask-\textsc{cond} \textsc{dub} perhaps   \textsc{dub}\\
    \glt    `Maybe we can ask (him/her) to harvest again.’\\
    \glt    `A lo mejor le pedimos que vuelva a pizcar.’    < 06 4:94/el >    \\
}
        \ex[]{
        \textit{ˈmán ˈhùraki raʔˈlìnula ˈtiêndatʃ͡i\\}
        \gll    ˈmá=ni ˈhùra-ki [raʔˈlì-\textbf{nula} ˈtiêndatʃ͡i]\\
                now=\textsc{1sg.nom}   send.to-\textsc{pst.ego} buy-\textsc{{order}}  store\\
        \glt    `I sent (him/her) to go buy to the store.’ \\
        \glt    `Ya lo mandé a comprar a la tienda’ < 06 2:48/el >\\
        }
            \ex[]{
            \textit{aʔˈlì  tʃ͡iˈhônsa  ko      ˈmá         ˈpé    oˈt͡ʃérisa      ko nuluˈrîa baˈʔwí tuˈwúnula}\\
            \gll    aʔˈlì  tʃ͡iˈhônsa=ko      ˈmá         ˈpé    oˈt͡ʃéri-sa=ko nulu-ˈrîa [baˈʔwí tuˈwú-\textbf{nula}]\\
                    and  then=\textsc{emph}  already    little   grow-\textsc{cond=emph}    order-\textsc{hab.pass} water bring-\textsc{{order}}\\
            \glt    `And then when [they, the children] already grow a little, they were sent to bring water.’  \\
            \glt    `Y ya cuando crecen un poquito los mandan a traer agua.’    \corpuslink{tx48[00_414-00_457].wav}{BFL tx48:00:41.4}, \corpuslink{tx48[00_457-00_496].wav}{BFL tx48:00:45.7} \\
        }
    \z
\z

The following example (in (\ref{ex: multiple subordinate nula})) shows that a clause headed by a jussive predicate may have more than one subordinate clause, each of which will have \textit{ˈnula} colexicalized with each lower predicate.

\ea\label{ex: multiple subordinate nula}

    \textit{ˈwàas ˈmá niˈsènula ˈtʃ͡îba ˈmá boˈrêko ˈmá niˈsènula nuluˈrîa}\\
    \gll    [ˈwàsi ˈmá niˈsè-\textbf{nula}] [tʃ͡îba ˈmá boˈrêko ˈmá niˈsè-\textbf{nula}] nulu-ˈrîa\\
            cows also shepherd-\textsc{{order}} goats also sheep also shepherd-\textsc{{order}} order-\textsc{hab.pass}\\
    \glt    `They are sent to shepherd cows, too, to take care of goats and sheep.'\\
    \glt    `Los mandan a cuidar vacas también, a cuidar chivas, borregos también.'    \corpuslink{tx48[00_548-01_007].wav}{BFL tx48:00:54.8}\\

\z

The periphrastic structure is often used as an imperative, where the speaker verbally commands the addressee to instruct a third participant (present or absent) to perform an action, as below. Speakers use this formula often in every-day interactions, and they report it as a formula of polite request. This is exemplified in (\ref{ex: causative nula as imperative}).

\ea\label{ex: causative nula as imperative}

\textit{gaˈbriêla nuˈlè biˈlé tʃ͡imoˈrí  ʃiˈrûnula}\\
\gll    gaˈbriêla nuˈl-è [biˈlé tʃ͡imoˈrí  ʃiˈrû-\textbf{nula}]\\
        Gabriela order\textsc{-appl.imp} one  squirrel  hunt-\textsc{{order}}\\
\glt    `Tell Gabriela to hunt a squirre.l’\\
\glt    `Dile a Gabriela que cace una ardilla’.\\

\z

\ili{Mountain Guarijío} is also documented to have indirect causative structures, including a periphrastic structure with double encoding of the causative \citep[]{miller1996guarijio}, as shown in (\ref{ex: indirect causative in mountain guarijioc}).


\newpage
\ea\label{ex: indirect causative in mountain guarijio}
{Indirect Causative construction in \ili{Mountain Guarijío} \citep[219]{miller1996guarijio}}

    \ea[]{
    \textit{noʔó koʔkónulani teurusio}\\
    \gll    noʔó koʔkó-\textbf{nula}-ni teurusio\\
            1\textsc{sg.acc} eat-\textsc{{order}-pres} Tiburcio\\
    \glt    `Tiburcio commands me to eat.’ \\
    \glt    `Tiburcio manda que yo coma.'\\
}\label{ex: indirect causative in mountain guarijioa}
        \ex[]{
        \textit{nulani teurusio noʔó koʔkómi    ruhka}\\
        \gll   \textbf{nula}-ni teurusio noʔó koʔkó-mi    ruhka\\
                {order}-\textsc{pres}   Tiburcio  1\textsc{sg.acc}   eat-\textsc{adv}    \textsc{adv}\\
        \glt    `Tiburcio commands that I eat.’\\
        \glt    `Tiburcio manda que yo coma.'\\
    }\label{ex: indirect causative in mountain guarijiob}
            \ex[]{
            \textit{nulani    teurusio noʔó koʔkónurega}\\
            \gll    \textbf{nula-ni}    teurusio noʔó koʔkó-\textbf{nur}-e-ga\\
                    {command-\textsc{pres}}   Tiburcio   1\textsc{sg.acc} eat-\textsc{{order}-appl-ptc.prs}\\
            \glt    `Tiburcio commands that I eat (lit. that/while I am eating).’\\
            \glt    `Tiburcio manda que yo coma.'\\
        }\label{ex: indirect causative in mountain guarijioc}
    \z
\z

As shown in these examples, the predicate \textit{nula/nuˈle}, cognate of Choguita Rarámuri \textit{nuˈla}, may appear: (i) co-lexicalized with a matrix predicate (\ref{ex: indirect causative in mountain guarijioa}); (ii) as the matrix predicate in a complementation construction (\ref{ex: indirect causative in mountain guarijiob}); or (iii) both as the matrix predicate and co-lexicalized with the lower verb deriving a co-lexicalized structure within the complement (\ref{ex: indirect causative in mountain guarijioc}). Only the latter strategy is attested in Choguita Rarámuri.

\subsection{Switch reference in reportative clauses}
\label{subsec: reportative clauses}

%\subsubsection{Switch-reference}
%\label{subsubsec: switch reference}
%re-do definition of switch refernce
Switch-reference is a phenomenon where a set of morphemes associated with the juncture of two clauses encodes whether a prominent argument in each clause is co-referent. Typically, co-referent arguments are subject arguments, although in some switch-reference systems the tracked co-referent arguments are not the subjects of their clauses \parencite{mckenzie2015survey}. A phenomenon with an areal diffusion, switch reference is attested in the West and Southwest of North America \parencite{jacobsen1967switch, jacobsen1983typological, mckenzie2015survey}. Within the \ili{Uto-Aztecan} language family, switch reference has been documented in \ili{Southern Paiute} (\ili{Numic}; \citealt{sapir1930southern}), \ili{Comanche} (\ili{Numic}; \citealt{charney1993grammar}), \ili{Hopi} \citep[]{jeanne2019argument}, \ili{Tohono O'odham} (\ili{Tepiman}; \citealt{hale1983papago}; \citealt{hale1992subject}), \ili{Cupeño} (\ili{Takic}; \citealt[]{hill2005grammar}), and closely related \ili{Mountain Guarijío} (\ili{Tara-Guarijío}; \citealt[]{miller1996guarijio}).

In Choguita Rarámuri, switch reference is attested in the reportative construction, an evidential construction that indicates that the speaker’s source of information is not direct, but hearsay. Reportative constructions involve a matrix clause with a speech predicate and a complement clause, the content of the reported event. It involves a stress-neutral productive marker, and is added to the verb of the complement clause. When the notional subjects are correferential, the dependent verb is marked for tense/aspect and the epistemic \textit{-o} suffix (\ref{ex: switch reference in reportativea}) (this suffix is a replacive suffix that triggers vowel deletion on the base onto which it attaches). When the notional subjects are not correferential, the dependent verb suffixes the reportative \textit{-la} suffix, which indicates disjoint reference (\ref{ex: switch reference in reportativeb}).

\ea\label{ex: switch reference in reportative}

    \ea[]{
    \textit{maˈrîa  ko  ke  ʃiˈmíko ˈrú}\\
    \gll    maˈrîa=ko  ke  siˈmí-\textbf{ki-o}    ˈrú\\
            Maria=\textsc{emph} \textsc{neg} go.\textsc{sg-}{\textsc{pst.ego-ep}}  say.\textsc{prs}\\
    \glt    `Maria\textsubscript{i} says (she\textsubscript{i}) didn’t go.’  \\
    \glt    `Dice María\textsubscript{i} que (ella\textsubscript{i}) no fue.’   < BFL 09 3:115/el >\\
}\label{ex: switch reference in reportativea}
        \ex[]{
        \textit{maˈrîa  ko  ˈhê  aˈní    hoˈsê  ke  ʃiˈmíla ˈruá}\\
        \gll    maˈrîa=ko  ˈhê  aˈní    hoˈsê  ke  siˈmí\textbf{-la}    ru-ˈwá\\
                Maria=\textsc{emph}  \textsc{dem}  say.\textsc{prs} José  \textsc{neg}  go.\textsc{sg}{\textsc{{}-rep.ds}}  say-\textsc{mpass}\\
        \glt    `Maria says that José didn’t go.’\\
        \glt    `Dice María que José no fue.’   < BFL 09 3:115/el >\\
    }\label{ex: switch reference in reportativeb}
    \z
\z

%describing the syntactic distribution of the SR morphemes, especially with respect to tense, aspect, or mood
% The structure of switch-reference marked clauses
%Canonical switch-reference is not linked to any particular type of clause, while non-canonical switch-reference is restricted to coordination and clause-chains \parencite{mckenzie2015survey}.

%Structural aspects of SR marked clauses:
%- Coordination
%- Subordination: adverbial clauses, complement clauses, relative clauses, conditional clauses

%Regarding the morphological properties of switch-reference, some questions to answer: (i) what is the form of the SR morpheme? (ii) where is the SR morpheme situated in the pivot clause? (iii) What effects does the presence of SR have on other verbal morphemes?

More examples of reportative constructions with same and different referents are provided in (\ref{ex: switch reference same referent}) and (\ref{ex: switch reference dif referent}), respectively.

\ea\label{ex: switch reference same referent}
{Switch reference: same reference}

    \ea[]{
    \textit{ˈá   biˈlá   ko         aˈní      ˈmâgre  neˈhê    amaˈtʃíkoro ˈruá}\\
    \gll    ˈá  biˈlá=ko   aˈní  ˈmâgre  neˈhê    amaˈtʃí-ki-li-o    ru-ˈwá\\
            \textsc{aff}  really=\textsc{emph} say.\textsc{prs}  nuns  1\textsc{sg.nom}    pray-\textsc{appl-pst-ep} say-\textsc{mpass}\\
    \glt    `The nuns\textsubscript{i} say that (they\textsubscript{i}) prayed for me.’\\
    \glt    `Las monjas\textsubscript{i} dicen que (ellas\textsubscript{i}) me rezaron.’   \\
}
        \ex[]{
        \textit{maˈnuêli    ko     ˈwé   biˈlá    riˈkúlo ˈrú}\\
        \gll    maˈnuêli=ko  ˈwé  beˈla  riˈkú-li-o ˈrú\\
                Manuel=\textsc{emph}    \textsc{int}   really   get.drunk.\textsc{sg-pst-ep}  say.\textsc{prs}\\
        \glt    `Manuel\textsubscript{i} says (he\textsubscript{i}) got drunk.’\\
        \glt    `Manuel\textsubscript{i} dice que (él\textsubscript{i}) se emborrachó.’   \\
    }
    \z
\z

\ea\label{ex: switch reference dif referent}
{Switch reference: different reference}

    \ea[]{
    \textit{á   biˈlá   oˈkám   tʃ͡aˈnía     ne   ka  ˈhém isiˈmâtara ruˈa     tʃ͡aˈbè} \\
    \gll    á  beˈla  oˈká=mi  tʃ͡aˈní-a    ne  ka  ˈhémi i-siˈmâta-\textbf{ra}    ru-ˈwá   tʃ͡aˈbè\\
            \textsc{aff}  really  many=\textsc{dem}  sound-\textsc{prs} \textsc{int}  ka  here \textsc{pl-}pass.\textsc{pl-}{\textsc{rep.ds}} say-\textsc{mpass}  before\\
    \glt    `Many people\textsubscript{i} say that they\textsubscript{j} used to pass through here long time ago.’\\
    \glt    `Muchas personas\textsubscript{i} dicen que por aquí pasaban\textsubscript{j} mucho antes.’    \corpuslink{tx223[04_176-04_224].wav}{LEL tx223:04:17.6}\\
}
    \ex[]{
    \textit{tʃ͡iˈnà  ba   ˈétʃ͡i    biˈlá    tòola ˈruá aliˈwâ-la  ba}\\
    \gll    ˈétʃ͡i ˈnà ba   ˈétʃ͡i    biˈlá    tò-\textbf{la} ˈru-wá aliˈwâ-la  ba\\
            \textsc{dem} there   \textsc{cl} \textsc{dem}  indeed  take.\textsc{pst.pass-}{\textsc{rep.ds}} say-\textsc{mpass} soul-\textsc{poss} \textsc{cl}\\
    \glt    `They\textsubscript{i} say that that one\textsubscript{j} got his soul stolen there.’\\
    \glt    `Cuentan\textsubscript{i} que a ese\textsubscript{j} ahí le llevó el alma.’      \corpuslink{tx_muerto[00_358-00_391].wav}{BFL tx\_muerto:0:35.8}\\
}
    \z
\z

This switch-reference system is restricted to reportative constructions, as it is not generalized to all constructions involving dependent clauses in Choguita Rarámuri. For \ili{Western Tarahumara}, \citet{Burgess-1984} argues that switch reference is attested in adverbial clauses with the suffixes \textit{-sa} and \textit{-so}, glossed as `when/having' (\citeyear[137]{Burgess-1984}) and cognate with the Choguita Rarámuri conditional \textit{-sa} suffix. There is no evidence of switch reference in adverbial clauses of this type in Choguita Rarámuri (see §\ref{subsec: conditional clauses} below).

%Summarize how this fits with the typology of switch reference

\subsection{Direct speech complements}
\label{subsec: complementation with speech verbs}

Direct speech complementation involves a matrix clause with a verb of speaking (e.g., `say') and a complement clause with reported speech as a direct quote, where the verb is finite and the clause is introduced without a complementizer. The following examples illustrate the properties of direct speech complements, where the complement may follow (\ref{ex: complements of speech verbsa}) or precede (\ref{ex: complements of speech verbs}b--c) the matrix clause. Prosodically, the reported clause constitutes its own Intonational Phrase (IP), and has the canonical intonation of the sentence type of the complement clause, e.g., declarative (\ref{ex: complements of speech verbs}a--b) or interrogative (\ref{ex: complements of speech verbsc}) (for details of the prosodic properties od dedclaratives and interrogatives, see §\ref{sec: intonation}, as well as §\ref{sec: declarative sentences} and §\ref{sec: interrogative constructions}).

%\break

\ea\label{ex: complements of speech verbs}

    \ea[]{
    \textit{ˈhê  beˈla   aˈnía ruˈwá tʃ͡aˈbôtʃ͡i: “ah, si aʔˈlá ka ˈmí raˈʔìa ka ˈmí pa ˈnàri  baˈhîwa   ba”}  \\
    \gll    ˈhê beˈla aˈní-a ru-ˈwá tʃ͡aˈbôtʃ͡i [a si ne aʔˈlá ka=mi raˈʔì-a ka=mi pa ˈnàri baˈhî-wa   ba]\\
            \textsc{dem}  really  say-\textsc{prog}  say-\textsc{mpass}  mestizo  ah  \textsc{int} \textsc{int} good  \textsc{emph=dem} like-\textsc{prog} \textsc{emph=dem} \textsc{cl} that  drink-\textsc{mpass} \textsc{cl}\\
    \glt    `And the \textit{mestizo} was saying: “ah, it is so good and tasty to drink this”.’\\
    \glt    `Así dijo el mestizo: ``¡Ay, cómo está rico tomar esto!”.’  < SFH 06 choma(10) >\\
}\label{ex: complements of speech verbsa}
        \ex[]{
        \textit{``á  si ˈném raˈʔìa ˈkam  pa ˈnàri  tʃ͡oʔˈmá ko  ba” aˈnía  ˈrá}  \\
        \gll    [a  si ˈné=mi  raˈʔìa  ˈka=mi    pa  ˈnàri  tʃ͡oʔˈmá=ko  ba]  aˈní-a    ru-ˈwá\\
                ah \textsc{int} \textsc{int=dem}  tasty \textsc{emph=dem} \textsc{cl} \textsc{prox} snot=\textsc{emph} \textsc{cl} say-\textsc{prog} say-\textsc{mpass}\\
        \glt    ```This snot is really tasty”, he would say.’\\
        \glt    ```Está muy rico este moco”, así decía.’    < SFH 06 choma(15) >\\
    }\label{ex: complements of speech verbsb}
        \ex[]{
        \textit{``okoˈká ropaˈtʃ͡í ba?" ˈhê biˈlá ko aˈnèla ˈrá ˈétʃ͡i ko ˈkôasa aˈlé ba}\\
        \gll    [oko-ˈká ropaˈtʃ͡í ba] ˈhê biˈlá=ko aˈn-è-la ru-ˈwá ˈétʃ͡i=ko ˈkôa-sa aˈlé ba\\
                hurt-\textsc{ger} belly \textsc{cl} it indeed=\textsc{emph} say-\textsc{appl-rep} say-\textsc{mpass} \textsc{dem=emph} feed-\textsc{cond} \textsc{dub} \textsc{cl}\\
        \glt    ```Does your belly hurt?" asked those who were feeding him.'\\
        \glt    ```Te duele el estómago?" le preguntaban los que le dieron de comer.'   \corpuslink{tx43[10_015-10_055].wav}{SFH tx43:10:01.5}\\
    }\label{ex: complements of speech verbsc}
    \z
\z

Indirect speech complements are encoded through the reportative clause construction described in §\ref{subsec: reportative clauses} above.

\section{Adverbial clauses}
\label{sec: adverbial clauses}

This section addresses adverbial clauses, which
constitute a major type of non-matrix clause,
together with complement clauses and relative clauses. While a complement clause functions as an argument of a higher predicate (\citealt{cristofaro2005subordination}, \citealt{noonan2007complementation}), adverbial clauses provide contextual information about the event described in the main clause, syntactically acting as a modifier of verb phrases or entire clauses (\citealt{thompson1985adverbial}). The following subsections describe the different sub-types of adverbial clauses in Choguita Rarámuri in terms of their function and morphosyntactic properties. Six main types of adverbial clause types are identified in this language: (i) conditional clauses (§\ref{subsec: conditional clauses}); (ii) purpose clauses (§\ref{subsec: purpose clauses}); (iii) reason clauses (§\ref{subsec: reason clauses}); (iv) locative clauses (§\ref{subsec: locative clauses 2}); (v) temporal clauses (§\ref{subsec: temporal clauses}); and (vi) manner clauses (§\ref{subsec: manner clauses}).

\subsection{Conditional clauses}
\label{subsec: conditional clauses}
\largerpage[2]
Conditional clauses express a conditional relationship between two events and share the following structural properties in Choguita Rarámuri: (i) the protasis (condition clause) generally precedes the apodosis (description of the actual\slash potential outcome); (ii) the verb of the protasis clause is obligatorily marked with the stress-shifting conditional suffix \textit{-sâ};\footnote{\citet{felix2006grammar} reports thar closely-related \ili{River Guarijío} has a conditional suffix \textit{-so}, while \citet{miller1996guarijio} describes that \ili{Mountain Guarijío} has adverbial suffixes \textit{-sa}, \textit{-sao} that may be translated as `when'.} and (iii) the protasis is optionally marked by \textit{ka} in clause final position, a morpheme that is used as an irrealis copula in other contexts, but in conditional clauses marks the conditional clause as not yet realized. This is exemplified in (\ref{ex: conditional clauses}) (protasis clauses are marked with squared brackets).

\ea\label{ex: conditional clauses}

    \ea[]{
    \textit{aʔˈlì   ˈmá     aʔˈlì     ko   ˈmá   aʔˈlá   waʃiˈsâa   ka   ku sipiˈbáti ka ˈnà}\\
    \gll [aʔˈlì   ˈmá     aʔˈlì=ko   ˈmá   aʔˈlá   wasi-ˈsâ   ka]   ku sipi-ˈbá-ti-ka ˈnà\\
        and  already  afterwards=\textsc{emph}  already  well  cook-\textsc{cond} \textsc{irr} \textsc{rev} be.cold-\textsc{inch-caus-ger} \textsc{dem}\\
    \glt    `And when it is cooked, you cool it off again.’\\
    \glt    `Y ya cuando se cose se enfría otra vez.’  \corpuslink{tx68[01_306-01_358].wav}{LEL tx68:01:30.6}\\
}\label{ex: conditional clausesa}
        \ex[]{
        \textit{aʔˈlì   ˈmá     biˈkiá   roˈkò   ˈnà   tʃ͡ereˈbása     ka ˈmá kosˈtâaltʃ͡i muˈtʃ͡ûwuka}    \\
        \gll    [aʔˈlì   ˈmá     biˈkiá   roˈkò   ˈnà   tʃ͡ereˈbá-sa   ka] ˈmá kosˈtâaltʃ͡i muˈtʃ͡ûwu-ka\\
                and  already  three  night  \textsc{dem} spend.night\textsc{-cond} \textsc{irr} already sac put.in.\textsc{pl-ger}\\
        \glt    `And when three nights have passed, you put it in a sac.’\\
        \glt    `Y ya cuando pasan tres noches, se pone en un costal.’   \corpuslink{tx68[00_380-00_404].wav}{LEL tx68:00:38.0}, \corpuslink{tx68[00_404-00_462].wav}{LEL tx68:00:40.4}\\
      }\label{ex: conditional clausesb}
    %\pagebreak
            \ex[]{
            \textit{niʔˈwísa ka ˈmá ukuˈmêa aˈlé}\\
            \gll    [niʔˈwí-sa ka] ˈmá uku-ˈmêa aˈlé\\
                    to.be.lightning-\textsc{cond} \textsc{irr} already rain-\textsc{fut.sg} \textsc{dub}\\
            \glt    `When there is lightning, then it will rain.'\\
            \glt    `Cuando relampaguea, ya empieza a llover.' \corpuslink{el1318[05_564-05_579].wav}{MFH el1318:05:56.4}\\
        }\label{ex: conditional clausesc}
        \clearpage
                \ex[]{
                \textit{aʔˈlì ˈétʃ͡i kaiˈnâsa ˈmá tʃ͡oˈnà moˈʔíram ˈkúutʃ͡i ba ˈne}\\
                \gll    [aʔˈlì ˈétʃ͡i kaiˈnâ-sa] ˈmá ˈétʃ͡i ˈnà moˈʔí-r-ame ˈkútʃ͡i ba ˈne\\
                        and \textsc{dem} finish-\textsc{cond} already \textsc{dem} there go.in.\textsc{pl-pst.pass-ptcp} children \textsc{cl} ne\\
                \glt    `And when that one was finished, the children went there.’\\
                \glt    `Y ya cuando se terminó esa, ya ahí entraron los niños.’   \corpuslink{tx12[02_406-02_449].wav}{SFH tx12:02:40.6}\\
            }\label{ex: conditional clausesd}
    \z
\z

While the verb in the protasis clause is always marked with the conditional mood suffix \textit{-sâ}, the verb of the apodosis clause may be marked with a limited set of TAM and other markers: in (\ref{ex: conditional clauses}a--b), as part of a procedural text, the suffix \textit{-ká} in the apodosis has a gerundive meaning (see §\ref{sec: clause chaining} for discussion of constructions where the suffix \textit{-ká} is used in chaining structures); alternatively, the apodosis may be marked with future tense (\ref{ex: conditional clausesc}) or it may be nominalized with a particpial marker (\ref{ex: conditional clausesd}).

The examples in (\ref{ex: apodosis first examples}) show that, while the protasis generally precedes the apodosis, the opposite order is also attested.

\ea\label{ex: apodosis first examples}

    \ea[]{
    \textit{suˈnù niˈhîbo ˈlá, waˈrî niˈhîsa ba}\\
    \gll    suˈnù niˈhî-bo oˈlá [waˈrî niˈhî-sa ba]\\
            corn give-\textsc{fut.pl} \textsc{cer} palm.basket give-\textsc{cond} \textsc{cl}\\
    \glt    `We give corn when they give palm baskets.'\\
    \glt    `Damos maíz cuando dan waris (canastas de palma).' \corpuslink{co1140[00_206-00_227].wav}{MDH co1140:00:20.6}\\
}
        \ex[]{
        \textit{ˈpé ˈnabi ˈlá awiˈmêa ˈlé, ˈhípi ˈmá baˈtùsa ka ba}\\
        \gll    ˈpé ˈna=bi oˈlá awi-ˈmêa aˈlé [ˈhípi ˈmá baˈtù-sa ka ba]\\
                just \textsc{prox}=just \textsc{cer} dance-\textsc{fut.sg} \textsc{dub} now already grind-\textsc{cond} \textsc{irr} \textsc{cl}\\
        \glt    `This one will dance, I think, now that we grind the corn (for the corn beer).'\\
        \glt    `Este va a bailar yo creo ahora que ya muélamos (el tesgüino).'   \corpuslink{co1140[15_268-15_296].wav}{MDH co1140:15:26.8}\\
        }
            \ex[]{
            \textit{uˈmûa ruˈwá naˈmûti ˈhêmi ˈsírsa, ˈwàsi, kaˈbâjo}\\
            \gll    uˈmûa ru-ˈwá naˈmûti [ˈhêmi ˈsíri-sa] ˈwàsi kaˈbâjo\\
                    all.kinds throw-\textsc{mpass} things over.there pass.\textsc{pl-cond} cows horses\\
            \glt    `All kinds of things would be thrown when they would pass by over there, to the cows, the horses.'\\
            \glt    `De ahí les tiraban a los animales cuando pasaban por alˈla, a las vacas, los caballos.’  \corpuslink{tx109[01_438-01_478].wav}{LEL tx109:01:43.8}\\
        }
    \z
\z

While the order of the protasis and apodosis is relatively flexible, the apodosis surfaces with a subset of TAM suffixes available in the language. As shown above, the apodosis may also be marked as resultative or it may be nominalized, pointing to the morphosyntactic integration of clauses in the conditional construction. In addition to this, the two clauses are prosodically and semantically integrated: protasis and apodosis have a single prosodic contour, and a clause bearing conditional marking and having the conditional function may not stand as a complete utterance.

\subsection{Purpose clauses}
\label{subsec: purpose clauses}

Adverbial clauses that express purpose are coded by a clause in which the verb is the minimally required element. The main clause encodes an event that is performed in order to bring about the event depicted in the purpose clause. The subordinate clause may be used with a purposive function even in the absence of dedicated morphology to indicate this function. Purpose clauses may appear at the right periphery of the main clause and are exemplified in (\ref{ex: purpose clauses 1}).

\ea\label{ex: purpose clauses 1}

    \ea[]{
    \textit{aʔˈlì   tamuˈhê   ˈmá …  ˈmá   aʔˈlì   raˈwé   ˈmáti napaˈwí hiˈrâmia} \\
    \gll    aʔˈlì   tamuˈhê   ˈmá  aʔˈlì   raˈwé   má=ti napaˈwí [hiˈrâ-mi-a]\\
            and   1\textsc{pl.nom} already  later   day   already=1\textsc{pl.nom} gather   bet-\textsc{mot-prog}\\
    \glt    `And then we gather that day to bet.’  \\
    \glt    `Entonces nosotros ese día ya nos juntamos para apostar.’   \corpuslink{tx19[01_323-01_373].wav}{LEL tx19:01:32.3}\\
}\label{ex: purpose clauses 1a}
%\pagebreak
        \ex[]{
        \textit{kiˈsâra koʔˈpôo ba!}\\
        \gll    kiˈsâra [koʔ-ˈpô ba]\\
                cook.\textsc{imp.sg} eat-\textsc{fut.pl} \textsc{cl}\\
        \glt    `Cook so you all eat!'\\
        \glt    `¡Guisa para que coman!' \corpuslink{co1136[00_134-00_155].wav}{MDH co1136:00:13.4}\\
    }\label{ex: purpose clauses 1b}
            \ex[]{
            \textit{ku napaˈbûa ˈmá ku kaˈtêupo aˈlé pa}\\
            \gll    ku napaˈbû-a [ˈmá ku kaˈtêwi-po aˈlé pa]\\
                    \textsc{rev} gather-\textsc{prog} already \textsc{rev} put.away-\textsc{fut.pl} \textsc{dub} \textsc{cl}\\
            \glt    `They are going to gather them to put them away.'\\
            \glt    `Ya lo van a juntar para guardarlos.' \corpuslink{co1234[16_221-16_239].wav}{JLG co1234:16:22.1}\\
        }\label{ex: purpose clauses 1c}
                \ex[]{
                \textit{aʔˈlì ˈnà ˈwé biˈláni aˈnè ˈnà riˈká aˈsâ ˈtʃ͡ó koˈbísi ˈkôbia ˈtʃ͡ó roˈwéam ba ˈwé iˈwêri ˈnà ˈmàmam pa}\\
                \gll    aʔˈlì ˈnà ˈwé biˈlá=ni aˈn-è ˈnà riˈká aˈsâ ˈtʃ͡ó koˈbísi ˈkôbi-a ˈtʃ͡ó roˈwé-ame ba [we iˈwêri ˈnà ˈmà-m-ame pa]\\
                        and \textsc{dem} \textsc{int} really=\textsc{1sg.nom} say-\textsc{appl} \textsc{dem} like.that sit.\textsc{sg.tr} also pinole give.pinole-\textsc{prog} also women.race-\textsc{ptcp} \textsc{cl} \textsc{int} strong \textsc{dem} run.\textsc{sg-m-ptcp} \textsc{cl}\\
                \glt    `And then I tell them that I will give pinole to the runner, so that she will run faster.’\\
                \glt    `Y luego les digo que le estoy dando pinole a la corredora para que corra más recio.’  \corpuslink{tx19[04_027-04_084].wav}{LEL tx19:04:02.7}\\
            }\label{ex: purpose clauses 1d}
    \z
\z

Purpose clauses may be marked with a subset of TAM specifications, including present (\ref{ex: purpose clauses 1a}) or future (\ref{ex: purpose clauses 1}b--c), or they may be nominalized (\ref{ex: purpose clauses 1d}).

While purpose clauses are generally placed at the right edge of the main clause as in (\ref{ex: purpose clauses 1}), there are also purpose clauses that precede the main clause, as shown in (\ref{ex: order of clauses}).

\ea\label{ex: order of clauses}

    \textit{onoˈlâ ku ˈáʃia aˈkíbo}\\
    \gll    [ono-ˈlâ ku ˈá-si-a] aˈkíbo\\
            father.male.ego\textsc{-poss} \textsc{rev} look.for-\textsc{mot-prog} left\\
    \glt    `He left to look for her dad.’\\
    \glt    `Se fue a buscar a su papá.’ < GFM 09 2:25/el >\\

\z

Purpose clauses may also be introduced by a subordinator, \textit{ˈnápu} or \textit{ˈkíti}, as exemplified in (\ref{ex: purpose clauses with complementizer}). These subordinating morphemes, as discussed in this chapter, may introduce different types of subordinate clauses.

%\pagebreak

\ea\label{ex: purpose clauses with complementizer}

    \ea[]{
    \textit{aʔˈlì biˈláti ˈʔá ˈjéntʃ͡o ˈá miʔˈlípi ˈlé pa, biˈlé toˈlí ba aʔˈlì tʃ͡iˈhônsa reˈmênipi ˈléti ˈûa ba ˈnápu riˈkáti ˈâbo ˈétʃ͡i}\\
    \gll    aʔˈlì biˈlá=ti ˈá ˈjén ˈtʃ͡ó ˈá miʔˈlí-pi aˈlé pa biˈlé toˈlí ba aʔˈlì tʃ͡iˈhônsa reˈmêni-pi aˈlé=ti ˈûa ba [\textbf{ˈnápu} riˈká=ti ˈâ-bo ˈétʃ͡i]\\
            and indeed=\textsc{1pl.nom} \textsc{aff} \textsc{aff} also \textsc{aff} kill.\textsc{pl-irr.pl} \textsc{dub} \textsc{cl} one chicken \textsc{cl} and then make.tortillas-\textsc{irr.pl} \textsc{dub=1pl.nom} with \textsc{cl} {\textsc{sub}} like.that=\textsc{1pl.nom} give-\textsc{fut.pl} \textsc{dem}\\
    \glt    `And we'll also kill a chicken and then make some tortillas so we can give it to them.'\\
    \glt    `Y también vamos a matar una gallina y luego hacerle unas tortillas para que le demos.' \corpuslink{tx475[04_209-04_268].wav}{SFH tx475:04:20.9}\\
}
        \ex[]{
        \textit{aʔˈlìmi noˈkèema aʔˈlá ˈnàri ˈkíti ko ke muˈjâma}\\
        \gll    aʔˈlì=mi noˈk-è-ma aʔˈlá ˈnàri [\textbf{ˈkíti}=ko ke muˈjâ-ma]\\
                then=\textsc{2sg.nom} move-\textsc{appl-fut.sg} well then \textsc{{sub}}=\textsc{emph} \textsc{neg} become.rotten-\textsc{fut.sg}\\
        \glt    `And then you will stir it well (often) so that it won't rot.'\\
        \glt    `Y lo vas a mover bien (cada rato) para que no se pudra.'  \corpuslink{tx60[00_431-00_474].wav}{BFL tx60:00:43.1}\\
    }
    \z
\z

As with other subordinate clause constructions, a single intonation contour applies to the complex as a whole.

\subsection{Reason clauses}
\label{subsec: reason clauses}

While purpose clauses encode a motivating event that is unrealized at the time of occurrence of the main event, reason clauses encode a motivating event that is realized at the time of occurrence of the event encoded by the main clause (\citealt{thompson1985adverbial}). In Choguita Rarámuri, reason clauses are marked as past tense (\ref{ex: reason clausesa}), mediopassive (\ref{ex: reason clausesb}) or other TAM distinctions that encode that the proposition encoded by the verbal predicate is realized. As shown in the examples in (\ref{ex: reason clauses}), reason clauses also appear at the right periphery of the main clause and are optionally introduced by the subordinator \textit{ˈkíti} (\ref{ex: reason clausesc}).

%\pagebreak

\ea\label{ex: reason clauses}

    \ea[]{
    \textit{ˈá ˈjén ˈá, aʔˈlá kaˈjèna tʃ͡aˈbè, ˈwé ˈpé  ˈwé aʔˈlá uˈkúli ˈlìna tʃ͡aˈbè ko ba}\\
    \gll    ˈá ˈjén ˈá aʔˈlá kaˈjèna tʃ͡aˈbè [ˈwé ˈpé ˈwé aʔˈlá uˈkú-li aʔˈlìna tʃ͡aˈbè=ko ba]\\
            \textsc{aff} \textsc{aff} \textsc{aff} well yield.harvest.\textsc{prs} before \textsc{int} just \textsc{int} well rain-\textsc{pst} because before=\textsc{emph} \textsc{cl}\\
    \glt    `Yes, there was indeed a good yield (harvest) because it did rain a lot before.'\\
    \glt    `Sí, sí se daba muy bien (la cosecha) porque sí llovía mucho antes.' \corpuslink{in484[00_385-00_427].wav}{ME in484:00:38.5}\\
}\label{ex: reason clausesa}
        \ex[]{
        \textit{ˈmá biˈlá ko oʔˈmôna tiˈbîli ˈlé ˈétʃ͡i ˈwênala ko ˈmá toˈwáa ko biˈlé ba}\\
        \gll    ˈmá biˈlá=ko oʔˈmôna tiˈbî-li aˈlé ˈétʃ͡i ˈwêna-la=ko [ˈmá to-ˈwá=ko biˈlé ba]\\
                already indeed=\textsc{emph} be.sad.\textsc{prs} remain.\textsc{pl-pst} \textsc{dub} \textsc{dem} parents-\textsc{poss=emph} already take-\textsc{mpass} one \textsc{cl}\\
        \glt    `The parents remained very sad because one of them (their children) was taken away.'\\
        \glt    `Ya los papás se quedaron muy tristes porque se llevaron a uno de ellos (de sus hijos).’ \corpuslink{tx152[11_115-11_148].wav}{SFH tx152:11:11.5}\\
    }\label{ex: reason clausesb}
            \ex[]{
            \textit{ˈpé aˈwêni be ko riˈhòwili tʃ͡aˈbè ˈkíti ˈpé koˈlì tiˈjôpatʃ͡i ˈníla rá siˈnéwi ko ba}\\
            \gll    ˈpé aˈwêni be=ko riˈhòwi-li tʃ͡aˈbè [\textbf{ˈkíti} ˈpé koˈlì tiˈjôpatʃ͡i ˈní-la ru-ˈwá siˈnéwi=ko ba]\\
                    just alone.\textsc{pl} just=\textsc{emph} live.people-\textsc{pst} before \textsc{sub} just side church \textsc{cop-rep} say-\textsc{mpass} first.time=\textsc{emph} \textsc{cl}\\
            \glt    `There were just a few people living (here) before, that is why it is said it was over there by the church the first time.’\\
            \glt    `Vivían poquitos antes, por eso dicen que fue allá por aquel lado de la iglesia la primera vez.' \corpuslink{tx12[00_520-00_565].wav}{SFH tx12:00:52.0}\\
        }\label{ex: reason clausesc}
    \z
\z

As shown above, reason clauses may involve no subordinating morpheme (\ref{ex: reason clauses}a--b) or they may involve the subordinating morpheme \textit{ˈkíti} (\ref{ex: reason clausesc}). Reason clauses may also be introduced by the subordinating marker \textit{ˈká ˈtʃ͡è} (a complex marker often surfacing in a reduced form as \textit{ˈkátʃ͡i}). For some speakers, this marker is exclusively attested in negative polarity clauses (e.g., (\ref{ex: reason clauses negative polaritya})), where there is no additional negative polarity morpheme). For other speakers, however, this morpheme may introduce reason clauses that have positive polarity, as shown in (\ref{ex: reason clauses negative polarityb}).

\ea\label{ex: reason clauses negative polarity}

    \ea[]{
    \textit{ˈmáti ... tʃ͡iˈlàa muˈtʃ͡í ba weti ˈmá a riˈgá nosoˈwípi ˈlé ˈkátʃ͡i biˈlé oˈwâami niˈlú kiˈʔà ko ˈmá ba}\\
    \gll    ˈmá=ti ˈétʃ͡i ˈlà-a muˈtʃ͡í ba we=ti ˈmá a riˈká nosoˈwí-pi aˈlé [\textbf{ˈká} \textbf{ˈtʃè} biˈlé oˈwâ-ame         niˈlú kiˈʔà=ko ˈmá ba]\\
            already=\textsc{1pl.nom} \textsc{dem} think-\textsc{prog} think \textsc{cl} \textsc{int=1pl.nom} already \textsc{aff} like.that be.finished-\textsc{irr.pl} \textsc{dub} because \textsc{neg} one cure-\textsc{ptcp} \textsc{exist} before=\textsc{emph} also \textsc{cl}\\
    \glt    `And then we thought perhaps we will die (be finished) because there was no medicine before.'\\
    \glt    `Y entonces pensábamos a lo mejor nos vamos a morir porque antes no había medicina.'   \corpuslink{tx372[02_000-02_055].wav}{LEL tx372:02:00.0}\\
}\label{ex: reason clauses negative polaritya}
        \ex[]{
        \textit{aʔˈlì ˈnà ˈwé biˈláti ˈá mahaˈwá taˈmò aʔˈlì ˈnà ba ˈká ˈtʃè ˈwé miˈká riˈhòi ˈmá ba haˈré ˈtʃ͡ó ko ba}\\
        \gll    aʔˈlì ˈnà ˈwé biˈlá=ti ˈá mahaˈwá taˈmò aʔˈlì ˈnà ba [\textbf{ˈká} \textbf{ˈtʃè} ˈwé miˈká riˈhòi         ˈmá ba] haˈré ˈtʃ͡ó=ko ba\\
                and then \textsc{int} indeed=\textsc{1pl.nom} \textsc{aff} be.affraid.\textsc{prs} \textsc{1pl.nom} and then \textsc{cl} because because \textsc{int} far.away inhabit also \textsc{cl} some also=\textsc{emph} \textsc{cl}\\
        \glt    `And then we were very scared because other people lived very far away.'\\
        \glt    `Y ya teníamos mucho miedo nosotras porque otros vivían muy lejos de allí.' \corpuslink{tx84[07_202-07_259].wav}{LEL tx84:07:20.2}\\
    }\label{ex: reason clauses negative polarityb}
    \z
\z

\subsection{Locative adverbial clauses}
\label{subsec: locative clauses 2}

Locative adverbial clauses in Choguita Rarámuri are introduced with the subordinating morpheme \textit{ˈnápi} plus the proximal demonstrative \textit{ˈna}. Locative clauses are exemplified in (\ref{ex: locative clauses}).

\ea\label{ex: locative clauses}

\textit{ˈwé  mikaˈbê   niˈlá ko   ˈnápo  ˈnà   tʃ͡aˈbôtʃ͡i   ˈmí riˈhòoa   riˈwèli   ko    sekunˈdâria   ba}    \\
\gll    ˈwé   mikaˈbê   niˈlá=ko   [\textbf{ˈnápi  ˈnà}   tʃ͡aˈbôtʃ͡i   ˈmí riˈhòwi-a riˈwè-li=ko]         sekunˈdâria   ba\\
        \textsc{int} far-more  \textsc{cop-pot=emph} {\textsc{sub} \textsc{prox}}  mestizos   \textsc{dist}   live\textsc{.sg-prog}   remain-\textsc{pst=emph}  secondary.school   \textsc{cl}  \\
\glt    `It must have been very far away, where the mestizos lived,  the secondary school.’\\
\glt    `Sería muy lejos, donde vivían los mestizos, la secundaria.’  \corpuslink{tx12[01_294-01_334].wav}{SFH tx12:1:29.4}\\

\z

The main clause may contain a demonstrative (in boldface) that is coreferential with the locative adverbial clause. This is shown in (\ref{ex: demonstrative coreferential}).

\ea\label{ex: demonstrative coreferential}

    \ea[]{
    \textit{ˈnápo   ˈnà   ti   roˈplâno   buˈjèi   ˈétʃ͡o  ˈnà biˈláni má     tʃ͡ukuˈká   neˈhê}\\
    \gll    [ˈnápi ˈnà ti roˈplâno buˈj-è-i] \textbf{ˈétʃi  ˈnà} biˈlá=ni ˈmá tʃ͡uku-ˈká neˈhê\\
            \textsc{sub} \textsc{prox}   \textsc{def.bad} plane     road-\textsc{have-impf} {\textsc{dem} \textsc{prox}}  really=1\textsc{sg.nom} already   crawl-\textsc{ger} 1\textsc{sg.nom}      \\
    \glt    `There, where there was a road for the plane, I was going there.’\\
    \glt    `Ahí donde era el camino del avión, ahí iba yo.’   \corpuslink{tx12[02_147-02_180].wav}{SFH tx12:02:14.7}\\
}\label{ex: demonstrative coreferentiala}
        \ex[]{
        \textit{ˈpé   biˈlé   riˈʰtê   a  riˈpî tʃ͡oˈnà ko ˈnápo   ˈnàti muˈtʃ͡ûwili}   \\
        \gll    ˈpé   biˈlé   riˈʰtê    a   riˈpî   \textbf{tʃoˈnà}=ko   [ˈnápi ˈnà=ti muˈtʃ͡ûwi-li]\\
                just   one   stone   \textsc{aff} remain {there}=\textsc{emph} \textsc{sub}  \textsc{prox}=1\textsc{pl.nom} sit.tr\textsc{.pl}-\textsc{pst}\\
        \glt    `Just one stone remains there where we put them.’\\
        \glt    `Nomás queda una piedra ahí donde las pusimos.’ \corpuslink{tx19[02_350-02_389].wav}{LEL tx19:02:35.0}\\
    }\label{ex: demonstrative coreferentialb}
    \z
\z

Locative adverbial clauses may precede (\ref{ex: demonstrative coreferentiala}) or follow (\ref{ex: demonstrative coreferentialb}) the main clause.\footnote{Clauses headed by locative predicates are addressed in §\ref{subsec: locative clauses} above.}
%\textbf{Any cases of discontinuous locative clauses?}

\subsection{Temporal clauses}
\label{subsec: temporal clauses}

There are two strategies in Choguita Rarámuri for encoding temporal clauses: (i) through a subordinating suffix \textit{-tʃ͡i} that attaches to the dependent verb conjugated for present tense (\ref{ex: temporal clausesa}); and (ii) through the morphologically complex subordinator \textit{(n)apaˈlì} (a contraction from the subordinating particle \textit{ˈnápi} and the time adverb \textit{aʔˈlì} (\ref{ex: temporal clausesb}).

\ea\label{ex: temporal clauses}

    \ea[]{
    \textit{aʔˈlì   aˈjá   saˈjèli   naˈwàatʃ͡i ˈétʃ͡i koriˈmá paˈtʃ͡á bitiˈtʃ͡í baˈkíli} \\
    \gll    aʔˈlì   aˈjá   saˈjè-li   [naˈwà-a-\textbf{tʃi} ˈétʃ͡i koriˈmá paˈtʃ͡á] bitiˈtʃ͡í baˈkí-li\\
            and  soon  feel-\textsc{pst}   arrive-\textsc{prog-loc}   \textsc{dem} \textit{korima} inside   house go.in.\textsc{sg}{}-\textsc{pst}\\
    \glt    `And then they felt when the \textit{korima} arrived and went inside the house.’\\
    \glt    `Y luego sintieron cuando llegó el \textit{korima} y entró a la casa.’   \corpuslink{tx5[00_428-00_483].wav}{LEL tx5:00:42.8}\\
}\label{ex: temporal clausesa}
        \ex[]{
        \textit{aʔˈlì  ˈmá ku apaˈlì ˈmá ke ˈtékili moˈtʃ͡íʃili ˈétʃ͡i   haˈréara   ko} \\
        \gll    aʔˈlì  ˈmá ku [\textbf{apaˈlì} ˈmá ke ˈtéki-li] moˈtʃ͡í-si-li ˈétʃ͡i   haˈréara=ko\\
                and   already \textsc{rev} when   already    \textsc{neg} be.drunk.\textsc{pl}-\textsc{pst} get.up.\textsc{pl}-\textsc{mot-pst} \textsc{def} some=\textsc{emph}\\
        \glt    `And then when they were no longer drunk, the others (another group of people) got up.’\\
        \glt    `Y entonces ya cuando ya no estaban borrachos, los demás (otros) se levantaron.’  \corpuslink{tx5[02_357-02_399].wav}{LEL tx5:02:35.7}\\
        }\label{ex: temporal clausesb}
    \z
\z

These two examples encode different semantic relationships between the temporal clauses and the main clause in terms of the temporal sequence between the two events they encode. Thus, in (\ref{ex: temporal clausesa}), the subordinate clause marked with the subordinating suffix \textit{-tʃ͡i} encodes a proposition that is simultaneous to the proposition encoded by the main clause. In contrast, the subordinate clause introduced by the morpheme \textit{napaʔˈlì} in (\ref{ex: temporal clausesb}) encodes an event that occurs before the event encoded by the main clause.

%elaborate on this section

\subsection{Manner clauses}
\label{subsec: manner clauses}

Another type of adverbial clause attested in the Choguita Rarámuri corpus are manner subordinate clauses, which encode notions that can be replaced by a single word expressing manner relationships (\citealt{thompson1985adverbial}). As shown in (\ref{ex: manner clauses}), manner subordinate clauses in Choguita Rarámuri may involve finite verbal predicates (\ref{ex: manner clauses}a, c) or verbs with the gerundive \textit{-ká} suffix (\ref{ex: manner clausesb}) (constructions with clauses marked with this suffix in clause chaining are discussed in §\ref{sec: clause chaining}).

\ea\label{ex: manner clauses}

    \ea[]{
    \textit{aʔˈlì ˈnà ˈétʃ͡i ˈnà moʔoˈtʃ͡íki tʃ͡uˈkúrili tʃ͡apiˈnála ˈétʃ͡i reˈhòi aliˈwâla toˈmêa oˈláli}\\
    \gll    aʔˈlì ˈnà ˈétʃ͡i ˈnà moʔoˈtʃ͡íki tʃ͡uˈkúri-li tʃ͡api-ˈnále-a ˈétʃ͡i reˈhòi aliˈwâ-la to-ˈmêa oˈlá-li\\
            and then \textsc{dem} there headboard go.around-\textsc{pst} grab-\textsc{desid-prog} \textsc{dem} man soul-\textsc{poss} take-\textsc{fut.sg} make-\textsc{pst}\\
    \glt    `And then there it was going around near the headboard wanting to grab it, to take the man’s soul.’\\
    \glt    `Y entonces ahí andaba por la cabecera queriéndolo agarrar, llevar el alma del señor.'  \corpuslink{tx5[01_113-01_183].wav}{LEL tx5:01:11.3}\\
}\label{ex: manner clausesa}
%\pagebreak
        \ex[]{
        \textit{ˈpé riˈsóorka aˈjénti tʃ͡aˈbèi ko maˈtʃ͡í raˈrèka be ti ko ˈám beˈnèa noˈká}\\
        \gll    [ˈpé riˈsó-ro-ka aˈjéna=ti] tʃ͡aˈbèi=ko [maˈtʃ͡í raˈrè-ka] be=ti=ko ˈá=mi beˈnè-a no-ˈká\\
                just suffer-\textsc{mov-ger} go.around.\textsc{pl}=\textsc{1pl.nom} before=\textsc{emph} go.barefoot go.barefoot-\textsc{ger} be=\textsc{1pl.nom}=\textsc{emph} \textsc{aff=dem} learn-\textsc{prog} do.\textsc{prs} just\\
        \glt    `We were going around suffering long ago, barefoot, that’s how we would learn (go to school).’\\
        \glt    `Nosotros andábamos así batallando, descalzos, así aprendíamos.’ \corpuslink{tx12[03_051-03_094].wav}{SFH tx12:03:05.1}\\
    }\label{ex: manner clausesb}
            \ex[]{
            \textit{ˈwé bi ˈko riˈsóa riˈká ˈmá koʔˈwá ba tʃ͡aˈbèi ko}\\
            \gll    ˈwé bi=ˈko [riˈsó-a riˈká] ˈmá koʔ-ˈwá ba tʃ͡aˈbèi=ko\\
                    \textsc{int} bi=\textsc{emph} suffer-\textsc{prog} like already eat-\textsc{mpass} \textsc{cl} before=\textsc{emph}\\
            \glt    `Before people would eat struggling.’\\
            \glt    `Antes comían batallando.’  \corpuslink{tx12[04_082-04_099].wav}{SFH tx12:04:08.2}\\
        }\label{ex: manner clausesc}
    \z
\z

As shown in these examples, manner adverbial clauses involve parataxis.

\section{Relative clauses}
\label{sec: relative clauses}

This subsection addresses the structural properties of relative clauses in Choguita Rarámuri. Relative clauses are defined in the typological literature as subordinate clauses that restrict or delimit the reference of a noun phrase (\citealt{andrews2007relative}). Strictly speaking, this description applies to \textit{headed relative clauses}. In contrast to headed relative clauses, \textit{headless relative clauses} do not combine syntactically with a head noun but rather act like referential or quantificational nominals themselves (\citealt{caponigro2020introducing}).

Choguita Rarámuri exhibits two types of headed relative clauses, namely those that involve nominalization (§\ref{subsec: RC nominalization}) and those that involve finite predicates and subordinators (§\ref{subsec: RC finite clauses}). The question of whether there are instances of headless relative clauses in Choguita Rarámuri is left out of the scope of this grammar. Other subordination strategies in the language are addressed in \chapref{chap: clause combining in complex sentences}.

\subsection{Relative clauses via nominalization}
\label{subsec: RC nominalization}

Choguita Rarámuri employs a nominalization strategy to form relative clauses, as documented in closely related \ili{Mountain Guarijío} (\citealt[179]{miller1996guarijio}) and \ili{Yaqui} (\citealt{gonzalez2012relative}, \citealt{guerrero2012relative}) and other \ili{Uto-Aztecan} languages (for a summary, see \citealt{garciasalidoheadless}). The nominalization strategy involves attaching participial suffixes (\textit{-ame} or \textit{-kame}) to the verb of the relative clause. This is exemplified in (\ref{ex: RC via nominalization examples}), with both relativized subjects and objects. As shown in these examples, the relative clause may precede the noun it modifies (\ref{ex: RC via nominalization examplesa}), but most frequently the relative clause follows the noun it modifies (\ref{ex: RC via nominalization examples}b--e).

\ea\label{ex: RC via nominalization examples}

    \ea[]{
    \textit{aʔˈlì ko ˈétʃ͡i ko ˈhê biˈlá aniˈwá tʃ͡aˈbè kiˈʔà ˈétʃ͡i ... ˈétʃ͡i \textbf{ˈtʃêlame} reˈhòi}\\
    \gll    aʔˈlì=ko ˈétʃ͡i=ko ˈhê biˈlá ani-ˈwá tʃ͡aˈbè kiˈʔà ˈétʃ͡i oˈtʃ͡êl-\textbf{ame} reˈhòi\\
            and=\textsc{emph} \textsc{dem=emph} it really say-\textsc{mpass} before long.ago \textsc{dem} grow.old-\textsc{ptcp} man\\
    \glt    ‘and then they say that long ago that old man (the man who is old)’\\
    \glt    ‘y allí dicen que antes ese señor viejito (el señor que es viejito)'   \corpuslink{tx71[01_439-01_481].wav}{LEL tx71:01:43.9}\\
}\label{ex: RC via nominalization examplesa}
        \ex[]{
        \textit{aʔˈlì ˈétʃ͡i ˈtʃîba \textbf{muˈkúami} ko ma baˈsûa koˈʔáli ˈétʃ͡o ˈnà \textbf{peˈrêami}}\\
        \gll    aʔˈlì ˈétʃ͡i ˈtʃîba muˈkú-\textbf{ame}=ko ma baˈsû-a koˈʔá-li ˈetʃ͡i ˈnà peˈrê-\textbf{ame}\\
                and \textsc{dem} goat die.\textsc{sg-ptcp=emph} already cook-\textsc{prog} eat-\textsc{pst} \textsc{dem} \textsc{there} inhabit.\textsc{pl-ptcp}\\
        \glt    ‘That dead goat (goat that is dead) was eaten cooked by the dwellers of (the ones that inhabit) that house.’\\
        \glt    ‘Esa chiva (que está) muerta se la comieron los (que habitan) de esa casa.’    \corpuslink{tx_mawiya[02_439-02_492].wav}{LEL tx\_mawiya:02:43.9}\\
    }\label{ex: RC via nominalization examplesb}
            \ex[]{
            \textit{taˈmò ˈhípi \textbf{oˈtʃêlami} ke me aʔˈlá matʃ͡iˈká uˈtʃ͡úwi}\\
            \gll    taˈmò ˈhípi oˈtʃ͡êl\textbf{-ame} ke me aʔˈlá matʃ͡i-ˈká uˈtʃ͡úwi\\
                    \textsc{1pl.nom} today grow.old-\textsc{ptcp} \textsc{neg} almost well know-\textsc{ger} \textsc{lie.down.pl.prs}\\
            \glt    `Those of us growing older now almost don't know things well.'\\
            \glt    ‘Los que crecemos ahora, casi no estamos sabiendo bien.’ \corpuslink{in61[00_362-00_389].wav}{SFH in61:00:36.2}\\
    }\label{ex: RC via nominalization examplesc}
                \ex[]{
                \textit{iʔˈwîra ˈtʃ͡ó ˈkîni ˈnà … kompaˈɲêro ˈtʃ͡ó ˈí tʃ͡oˈgîta \textbf{piˈrêami} ˈtʃ͡ó ba}\\
                \gll    iʔˈwîra ˈtʃ͡ó ˈkîni ˈnà kompaˈɲêro ˈtʃ͡ó ˈí tʃ͡oˈgîta piˈrê-\textbf{ame} ˈtʃ͡ó ba\\
                        help.\textsc{pl} also \textsc{1poss} \textsc{dem} friends also here Choguita dwell.\textsc{pl-ptcp} also \textsc{cl}\\
                \glt    ‘helping out out friends, the ones that live here in Choguita’\\
                \glt    ‘ayudándoles a nuestros compañeros, los que viven aqui en Choguita’    \corpuslink{tx12[12_101-12_155].wav}{SFH tx12:12:10.1}\\
        }\label{ex: RC via nominalization examplesd}
                    \ex[]{
                        \textit{ˈétʃ͡i koliˈmî \textbf{aniˈrîame} biˈlá ˈwîʃula ruˈwá}\\
                        \gll    ˈétʃ͡i koliˈmî ani-ˈrîwa-\textbf{ame} biˈlá ˈwî-si-la ru-ˈwá\\
                                \textsc{dem} rainbow say-\textsc{mpass-ptcp} indeed take-\textsc{mot-rep} say-\textsc{mpass}\\
                        \glt    `It is said that the one that is called rainbow (\textit{korimí}) goes along taking it.'\\
                        \glt    ‘Se dice que ese que le dicen arcoiris (\textit{korimí}) lo va llevando.’    \corpuslink{tx_muerto[01_223-01_254].wav}{BFL tx\_muerto:01:22.3}\\
                    }\label{ex: RC via nominalization examplese}
    \z
\z

Relative clauses formed via nominalization may lack an antecedent, as shown in (\ref{ex: RC via nominalization with no antecedents}), where the relative clause encodes the nominal predicate in a copular clause.

\ea\label{ex: RC via nominalization with no antecedents}

        \ea[]{
        \textit{ˈwé \textbf{sukuˈrûame} ˈú ku}\\
        \gll    ˈwé sukuˈrû-\textbf{ame} ˈhú=ko\\
                \textsc{int} do.evil.witchcraft-\textsc{ptcp} \textsc{cop.prs=emph}\\
        \glt    ‘He is very evil.’\\
        \glt    ‘Es muy hechicero.’    \corpuslink{tx_korimaka[00_223-00_241].wav}{CFH tx\_korimaka:00:22.3}\\
    }
            \ex[]{
            \textit{ˈápi\textbf{ iˈsêlikami} ka ˈlé, biˈkiánika, suˈwâba maˈjôra ma}\\
            \gll    ˈnápi i-ˈsêli-\textbf{kame} ka aˈlé biˈkiá-ni-ka suˈwâba maˈjôra ma\\
                    \textsc{sub} \textsc{pl}-governor.\textsc{pl-ptcp} \textsc{cop.irr} \textsc{dub} three-\textsc{incl-coll} all manager also\\
            \glt    `those who are governors, the three of them, all of the managers, too'\\
            \glt    `los que son gobernadores, los tres (gobernadores), todos los mayores también'    \corpuslink{tx816[00_367-00_399].wav}{JMF tx816:00:36.7}\\
        }
    \z
\z

As described for \ili{Mountain Guarijío} (\citealt[180]{miller1996guarijio}), it is possible to analyze derived adjectives as relative clauses with an overt antecedent, and deverbal nouns like \textit{sukuˈrûame} in (\ref{ex: RC via nominalization with no antecedents}) as relative clause without an overt antecedent.

\subsection{Relative clauses via finite clauses}
\label{subsec: RC finite clauses}

A second productive strategy to encode relative clauses in Choguita Rarámuri involves a finite predicate and a subordinating particle \textit{ˈnápi} (or its reduced form \textit{ˈápi}) and no co-referencing morphology within the relative clause (a case of ``gapping'' as defined in the typological literature \citealt{lehmann1986typology}). As discussed in §\ref{sec: complement clauses} and §\ref{sec: adverbial clauses}, this subordinating morpheme introduces complement clauses and certain classes of adverbial clauses. This strategy is exemplified in (\ref{ex: RC finite clauses}). Relative clauses are indicated with square brackets.

\ea\label{ex: RC finite clauses}

    \ea[]{
    \textit{siˈrínala ˈbél ˈá aniˈrîa ˈétʃ͡i ˈnápu tʃ͡oˈpém ʃiˈmèla noˈká, ˈétʃ͡i ˈnápu ukuˈwéa noˈká}\\
    \gll    siˈrína-la beˈlá ˈá ani-ˈrî-a ˈétʃ͡i [\textbf{ˈnápi} tʃ͡oˈpé=mi siˈmè-la noˈká] ˈétʃ͡i [\textbf{ˈnápi} ukuˈwéa noˈká]\\
            god.father-\textsc{poss} indeed \textsc{aff} say-\textsc{mpass-prog} \textsc{dem} \textsc{sub} Montezuma.pine=\textsc{dem} pass-\textsc{rep} do.\textsc{prs} \textsc{dem} \textsc{sub} ukuwéa.ceremony do.\textsc{prs}\\
\newpage
    \glt    `They call him sirínala (godfather) the one that passes the Montezuma pine (over someone's head), the one that blesses with Ukuwéa.'\\
    \glt    `Le dicen sirínala (padrino) a ese que hace con el ocote pasándolo, al que hace el ukuwéa.'    \corpuslink{tx475[07_333-07_377].wav}{SFH tx475:07:33.3}\\
}
        \ex[]{
        \textit{aʔˈlì ˈétʃ͡i ˈápu roˈwéma ˈlé ko biˈnôi biˈlá aˈní: “ˈjénan ˈá saˈjèrima” ˈá aˈní}\\
        \gll    aʔˈlì ˈétʃ͡i [\textbf{ˈnápi} roˈwé-ma aˈlé]=ko biˈnôi biˈlá  aˈní “ˈjéna=ni ˈá saˈjèri-ma” ˈá aˈní\\
                and \textsc{dem} \textsc{sub} run.womens.race-\textsc{fut.sg} \textsc{dub=emph} herself indeed say.\textsc{prs} \textsc{aff=1sg.nom} \textsc{aff} take.on-\textsc{fut.sg} \textsc{aff} say.\textsc{prs}\\
        \glt    ‘And then the one who will run, herself, says: “yes, I will take on the challenge”.’\\
        \glt    ‘Y entonces la que va a correr ella misma dice “sí le voy a entrar”.’    \corpuslink{tx19[00_398-00_451].wav}{LEL tx19:00:39.8}\\
    }
    \z
\z

The examples in (\ref{ex: RC finite clauses}) involve relativized subject arguments. Relative clauses encoded via finite predicates and subordinators may also encode other core arguments, such as objects (\ref{ex: relativized arguments RC}).

\ea\label{ex: relativized arguments RC}

    \ea[]{
    \textit{aʔˈlì tʃ͡oˈnà ˈhônsa ko ti ˈnà ˈmáti ˈnâri ˈápu roˈwéma ˈlé ˈátʃ͡i kaˈnílsa ˈníli roˈwéa}\\
    \gll    aʔˈlì tʃ͡oˈnà ˈhônsa=ko=ti ˈnà ˈmá=ti ˈnâri [\textbf{ˈnápi} roˈwé-ma aˈlé ˈátʃ͡i kaˈníli-sa ˈní-li roˈwé-a\\
            and then from\textsc{=emph}=\textsc{1pl.nom} \textsc{dem.prox} already=\textsc{1pl.nom} ask.\textsc{prs} \textsc{sub} run.womens.race-\textsc{fut.sg} \textsc{dub} if be.happy\textsc{-cond} \textsc{cop-pst} run.womens.race-\textsc{prog}\\
    \glt    ‘And then we ask the one (woman) who will run if she would like to race.’\\
    \glt    ‘Y entonces de ahí ya le preguntamos a la que va a correr a ver si se siente a gusto.' \corpuslink{tx19[00_332-00_398].wav}{LEL tx19:00:33.2}\\
}
    \z
\z

Relative clauses encoded with finite predicates and the subordinating particle \textit{ˈnápi} can also encode adjunct arguments, including a time (\ref{ex: RC as adjunctsa}), an instrument (\ref{ex: RC as adjunctsb}), a location (\ref{ex: RC as adjunctsc}) and a comitative argument (\ref{ex: RC as adjunctsd}).

\ea\label{ex: RC as adjuncts}

    \ea[]{
    \textit{ˈpé ˈá ˈmá buˈàli ˈtʃ͡ó ˈétʃ͡i ˈnà rupuˈlá tʃ͡aˈbèi ˈnáp aʔˈlì muˈhê ˈnâtatʃ͡i ko ba?}\\
    \gll    ˈpé ˈá ˈmá buˈà-li ˈtʃ͡ó ˈétʃ͡i ˈnà rupuˈlá tʃ͡aˈbèi [\textbf{ˈnápi} aʔˈlì muˈhê ˈnâta-tʃ͡i]=ko ba\\
            just \textsc{aff} recently come.out-\textsc{pst} also \textsc{dem} that ax before \textsc{\textbf{sub}} then \textsc{2sg.nom} think-\textsc{temp=emph} \textsc{cl}\\
    \glt    `They had recently started using (appeared, come out) axes before, since the time you remember?'\\
    \glt    `¿Ya apenas habían salido antes las hachas antes, cuando tu te acuerdas?   \corpuslink{in484[03_238-03_278].wav}{SFH in484:03:23.8}\\
    }\label{ex: RC as adjunctsa}
        \ex[]{
        \textit{ripuˈlá...ˈnápu riˈká miˈtʃ͡ípu kuˈʃì ba?}\\
        \gll    ripuˈlá [\textbf{ˈnápi} riˈká miˈtʃ͡ípu kuˈsì] ba\\
                ax \textsc{sub} that carve sticks \textsc{cl}\\
        \glt    `an ax with which to carve the sticks'\\
        \glt    ‘hacha con que labrar los palos’  \corpuslink{in61[03_306-03_330].wav}{SFH in61:03:30.6}\\
        }\label{ex: RC as adjunctsb}
            \ex[]{
            \textit{ˈnápi ˈnà ˈá piˈrê aˈbôni siˈné ˈkátʃ͡i ˈmáti ˈá ˈhâwamti ˈá ˈníbo ko}\\
            \gll     [\textbf{ˈnápi} ˈnà ˈá piˈrê] aˈbôni siˈné ˈkátʃ͡i ˈmá=ti ˈá ˈhâwa-ame=ti ˈá ˈní-bo=ko\\
                    \textsc{sub} \textsc{dem.prox} \textsc{aff} inhabit.\textsc{pl} \textsc{emph.pl} some times already=\textsc{1pl.nom} \textsc{aff} stand.\textsc{pl-ptcp=1pl.nom} \textsc{aff} \textsc{cop-fut.pl=emph}\\
            \glt    `Where we live, perhaps we will some times be elected as authorities.’\\
            \glt    ‘Donde vivimos a lo mejor en veces vamos a ser autoridades.’    \corpuslink{tx12[11_542-11_592].wav}{SFH tx12:11:54.2}\\
        }\label{ex: RC as adjunctsc}
    %\pagebreak
                \ex[]{
                \textit{ ˈá biˈlá ˈtʃ͡ó biˈhí iˈjêna ˈtʃ͡ó ˈétʃ͡i apaˈnêrala ko ˈhípi ˈrú ˈnápi ˈûa iˈjênili ˈrú ba}\\
                \gll    ˈá biˈlá ˈtʃ͡ó biˈhí iˈjêna ˈtʃ͡ó ˈétʃ͡i apaˈnêra-la=ko ˈhípi ˈrú [\textbf{ˈnápi} ˈjûa iˈjêni-li ˈrú] ba\\
                        \textsc{aff} indeed still still go.around.\textsc{sg.prs} still \textsc{dem} partner-\textsc{poss=emph} today say.\textsc{prs} \textsc{sub} with go.around.\textsc{-pst} say.\textsc{prs} \textsc{cl}\\
                \glt    `He is still together with his wife now, with the one he used to live with before.'\\
                \glt    `Todavía anda con su mujer ahora, con la que vivía (andaba) antes.'    \corpuslink{tx43[04_340-04_387].wav}{SFH tx43:04:34.0}\\
            }\label{ex: RC as adjunctsd}
    \z
\z

In the case of the relativized time clause in (\ref{ex: RC as adjunctsa}), the relativized clause makes reference to an adverb, \textit{tʃ͡aˈbèi} `before'. In the relativized locative clause in (\ref{ex: RC as adjunctsc}), there is no antecedent, nominal or otherwise (notionally, the relative clause refers to a place, Choguita). The relative clauses in (\ref{ex: RC as adjunctsb}) and (\ref{ex: RC as adjunctsd}), on the other hand, have nominal antecedents, \textit{ripuˈlá} `ax' and \textit{apaˈnêrala} `partner, wife', respectively. The head of the relative clause precedes the relative clause in every example, whether the relativized clause encodes a core argument or an adjunct.

The structure of Choguita Rarámuri relative clauses encoded through finite predicates has the same properties as the equivalent constructions in \ili{Rochéachi Rarámuri} (\citealt[38]{moralesmoreno2016rochecahi}). The subordinating particle in \ili{Rochéachi Rarámuri} is \textit{ˈmapu} (cognate of Choguita Rarámuri \textit{ˈnápi}), and also found in the speech of elderly Choguita Rarámuri speakers to introduce other subordinate clauses (addressed in \chapref{chap: clause combining in complex sentences}).

\section{Coordination}
\label{sec: coordination}

This section describes clausal coordination in Choguita Rarámuri, where two or more clauses are combined to form a larger syntactic unit. Three main types of coordination strategies are identified, namely conjunction (§\ref{subsec: conjunction}), disjunction (§\ref{subsec: disjunction}) and adversative conjunction (§\ref{subsec: adversative conjunction}). Within each section, the description is organized in terms of the morphosyntactic properties of the different constructions of each coordination type.

\subsection{Conjunction}
\label{subsec: conjunction}

\subsubsection{Conjunction marked with \textit{aʔˈlì}}
\label{subsubsec: conjunction marked with ali}

Phrasal and clausal conjunction in Choguita Rarámuri may involve syndetic coordination with the connective \textit{aʔˈlì} ‘and (then)', that is placed before the second conjunct. Syndetic coordination marked with \textit{aʔˈlì} is exemplified in (\ref{ex: syndetic coordination}).

\ea\label{ex: syndetic coordination}

    \ea[]{
    \textit{baˈtʃ͡á biˈlá riˈkása aʔˈlì moʔiˈrîa ba}\\
    \gll   baˈtʃ͡á biˈlá  reˈká-sa \textbf{aʔˈlì} moʔiˈrîa ba\\
            first indeed lay.down-\textsc{cond} {and} weave-\textsc{mpass-prog} \textsc{cl}\\
    \glt    `First you lay it down and then you weave it.’\\
    \glt    `Primero se pone y después se teje.’ \corpuslink{tx1[01_105-01_160].wav}{BFL tx1:01:10.5}\\
}\label{ex: syndetic coordinationa}
    \ex[]{
    \textit{ˈnè ˈtʃ͡é biˈlé ˈjàsa ˈrú bi aʔˈlì ˈétʃ͡i ˈmín aˈnèma ˈlì}\\
    \gll    ˈnè ˈtʃ͡é biˈlé ˈà-sa ˈrú bi \textbf{aʔˈlì} [ˈétʃ͡i ˈmí=ni aˈn-è-ma aʔˈlì \\
            1\textsc{sg.nom} again one look.for-\textsc{cond} say.\textsc{prs} just and \textsc{dem} 2\textsc{sg.acc=1sg.nom} say-\textsc{appl-fut.sg} \textsc{then}\\
    \glt    ```Let me look for another one and then I’ll tell you”.'\\
    \glt    ```Deja busco otra entonces te digo”.’ \corpuslink{tx19[01_135-01_179].wav}{LEL tx19:01:13.5}\\
}\label{ex: syndetic coordinationb}
        \ex[]{
        \textit{ˈétʃ͡i baˈtʃ͡á hiˈrâsa aʔˈlì ˈmá ˈhúmiʃi}\\
        \gll    ˈétʃ͡i baˈtʃ͡á hiˈrâ-sa \textbf{aʔˈlì} ˈmá ˈhúmasi\\
                \textsc{dem} first bet-\textsc{cond} and already take.off.\textsc{pl.prs}\\
        \glt    `First they bet and then they take off.'\\
        \glt    `Primero apuestan y ya arrancan.’ \corpuslink{tx19[01_403-01_442].wav}{LEL tx19:01:40.3}\\
    }\label{ex: syndetic coordinationc}
            \ex[]{
            \textit{ˈwé aˈnè ˈhùria aʔˈlì ˈwé ˈá ʃiˈmí ˈétʃ͡i roˈwéami ko}\\
            \gll    ˈwé aˈn-è ˈhùri-a \textbf{aʔˈlì} ˈwé ˈá siˈmí ˈétʃ͡i roˈwé-ami=ko\\
                    \textsc{int} say-\textsc{appl} send.off-\textsc{prog} and \textsc{int} \textsc{aff} go.\textsc{sg} \textsc{def} women.race-\textsc{ptcp=emph}\\
            \glt    `They tell her to run, and the runner then goes faster.’\\
            \glt    `Le dicen que corra y la corredora va corriendo más recio.'   \corpuslink{tx19[03_409-03_449].wav}{LEL tx19:03:40.9}\\
        }\label{ex: syndetic coordinationd}
                \ex[]{
                \textit{ˈmá aʔˈlá nataˈkêa aˈkíbi aʔˈlì ˈmá ke aˈnítʃ͡ani ˈtʃ͡ó}\\
                \gll    ˈmá aʔˈlá nataˈkê-a aˈkíbi \textbf{aʔˈlì} ˈmá ke aˈní-tʃ͡ani ˈtʃ͡ó\\
                        already well faint-\textsc{prog} went and already \textsc{neg} make.noise-\textsc{ev} also\\
                \glt    `She fainted and then it wasn't making noise anymore.'\\
                \glt    `Ya se desmayó y ya no se oía.’    \corpuslink{tx71[04_436-04_463].wav}{LEL tx71:04:43.6}\\
            }\label{ex: syndetic coordinatione}
    \z
\z

As shown in these examples, coordination marked with \textit{aʔˈlì} involves a temporal relationship between the conjuncts where the event described in the second clause temporally follows the event described in the first clause. As shown in these examples, coordination involves full clauses with independent predicates and arguments, even if arguments may be co-referential (e.g., (\ref{ex: syndetic coordinationc})).

Clausal conjunction may involve ellipsis of the predicate of the second clause. This is exemplified in (\ref{ex: ellipsis in conjunction}), where angled brackets <> indicate the ellipsis site in the second clause.

\ea\label{ex: ellipsis in conjunction}

    \ea[]{
    \textit{aʔˈlì reˈhòi ko ˈmá buˈʔíli aʔˈlì muˈkî ko ke ˈtʃ͡ó}\\
    \gll    aʔˈlì reˈhòi=ko ˈmá buˈʔí-li \textbf{aʔˈlì} muˈkî=ko ke <> ˈtʃ͡ó\\
            and man=\textsc{emph} already lay.down.\textsc{sg-pst} and woman=\textsc{emph} \textsc{neg} <> yet\\
    \glt    `Y el señor ya estaba acostado y la mujer todavía no.’\\
    \glt    `And the man was already asleep but not the woman.’  \corpuslink{tx5[00_483-00_527].wav}{LEL tx5:00:48.3}\\
}
    \z
\z

As shown in (\ref{ex: ellipsis in conjunction}), the connective may appear immediately preceding the first of two connected clauses or at the beginning of a clause or a sequence of clauses, as exemplified in (\ref{ex: connective at the beginning}).

\ea\label{ex: connective at the beginning}

    \ea[]{
    \textit{aʔˈlì biˈlá ko ˈnà ˈétʃ͡i muˈkî ko ˈèbiʃuwa ˈétʃ͡i saʔˈpá oʔˈwí ˈpé biˈlá ko ˈá riˈká koˈʔáa ruˈá ˈrám pa ˈnápu riˈká ˈnàa}\\
    \gll    aʔˈlì biˈlá=ko ˈnà ˈétʃ͡i muˈkî=ko ˈèbi-suwa ˈétʃ͡i saʔˈpá oʔˈwí ˈpé biˈlá=ko ˈá riˈká koˈʔá-a ru-ˈwá ru-ˈwá=mi pa ˈnápi riˈká ˈnà\\
            and indeed=\textsc{emph} then \textsc{dem} woman=\textsc{emph} bring-\textsc{cond.pass} \textsc{dem} meat raw just indeed=\textsc{emph} \textsc{aff} like.that eat-\textsc{prog} say-\textsc{mpass} say\textsc{-mpass=dem} \textsc{cl} that like \textsc{dem}\\
    \glt    `And then the woman had been brought raw meat and she would eat it like that, they say.'\\
    \glt    `Y entonces a la mujer le traía carne cruda y tenía que comer así.'   \corpuslink{tx43[06_173-06_249].wav}{SFH tx43:06:17.3}\\
}
    \z
\z

Details of the function of this and other connectives in Choguita Rarámuri discourse are left out of the scope of this grammar.

\subsubsection{Asyndetic conjunction}
\label{subsubsec: asyndetic conjunction}

Clausal conjunction in Choguita Rarámuri may also be encoded through asyndetic conjunction, i.e., through juxtaposition of finite clauses with no overt conjunction marker. This is exemplified in (\ref{ex: asyndetic conjunction}). Conjoined clauses exhibit prosodic unification.

\ea\label{ex: asyndetic conjunction}

    \ea[]{
    \textit{ˈpé niˈhê aˈníami  ˈú: ``ku rohoˈnâsa, roˈhàʃi ba"}\\
    \gll    ˈpé niˈhê aˈní-ame ˈhú ku roha-ˈnâ-sa roˈhà-si ba\\
            just 1\textsc{sg.nom} say-\textsc{ptcp} \textsc{cop} \textsc{rev} separate\textsc{-tr-imp.sg} separate\textsc{-impl.pl} \textsc{cl}\\
     \glt   `I just tell them (the women betting with me): ``Separate it (the bet), and separate yourselves".'\\
    \glt    `Yo nomás digo: ``Aparten (la apuesta) y apártense (las que apuestan conmigo)".’    \corpuslink{tx19[04_249-04_280].wav}{LEL tx19:04:24.9}\\
}
            \ex[]{
            \textit{banaˈká ˈpáli ripiˈjá, ˈnè ku ˈtèaki aʔˈlì ˈlé ba}\\
            \gll    banaˈká ˈpá-li ripiˈjá, ˈnè ku ˈtèa-ki aʔˈlì aˈlé ba\\
                    over.there throw-\textsc{pst} knife \textsc{1sg.nom} \textsc{rev} find-\textsc{pst.ego} later \textsc{dub} \textsc{cl}\\
            \glt    `He threw the knife there and I found it later.'\\
            \glt    `Tiró el cuchillo allá y yo lo encontré más tarde.'  \corpuslink{co1234[03_294-03_323].wav}{JLG co1234:03:29.4}\\
        }
    \z
\z

As in conjunction constructions that involve a connective, the ordering of clauses is iconic, with the first clause describing an event that precedes temporally the event described by the second clause, as shown in (\ref{ex: asyndetic conjunction}). In other cases, however, asyndetic coordination does not involve temporal succession. This is illustrated in (\ref{ex: asyndetic conjunction with no tenporal succession}).

\ea\label{ex: asyndetic conjunction with no tenporal succession}

        \textit{ˈkárka naˈpònili, ˈkárka kaˈsìnili}\\
        \gll    ˈká riˈká naˈpò-na-li ˈká riˈká kaˈsì-na-li\\
                everything like.that break-\textsc{tr-pst} everything like.that tear.apart-\textsc{tr-pst}\\
        \glt    `He broke it all and tore it all.'\\
        \glt    `Lo quebró todo y lo desmoronó.'  \corpuslink{co1137[00_362-00_387].wav}{MDH co1137:00:36.2}\\
\z

\subsection{Disjunction}
\label{subsec: disjunction}

Disjunctive clausal coordination is achieved through a variety of morphosyntactic means, including: (i) a construction with the borrowed \ili{Spanish} disjunctive conjunction \textit{o} `or'; (ii) a disjunctive prepositive conjunction \textit{wera} `or'; (iii) a postpositive enclitic \textit{=ma} `or' that attaches to each disjunctive clause; or (iv) through parataxis. Each of these strategies is described next.

\subsubsection{Disjunction marked with \textit{\textit{o}} `or'}
\label{subsubsec: disjunction o}

A frequently attested strategy for encoding disjunction in the Choguita Rarámuri corpus involves the borrowed \ili{Spanish} conjunction \textit{o} `or'. Its use in clausal disjunctive coordination is exemplified in (\ref{ex: disjunction with o}), which shows a fragment of a narrative.

\ea\label{ex: disjunction with o}

    \ea[]{
    \textit{biˈlé sukuˈrûame ˈhùria ruˈwá}\\
    \gll    biˈlé sukuˈrûame ˈhùri-a ru-ˈwá\\
            one evil.shaman send-\textsc{prog} say-\textsc{mpass}\\
    \glt    `They say that an evil shaman sends it (the fire bird).’\\
    \glt    `Dicen que lo manda un curandero brujo (malo) (al ˈpájaro de fuego).’  \corpuslink{tx5[05_095-05_118].wav}{LEL tx5:05:09.5}\\
}
        \ex[]{
        \textit{ˈétʃ͡i ˈhápi ke ˈmí kaˈléa ˈníli, o biˈlé ˈtʃó͡ ˈmá ˈhê aˈníame ka ruˈwá: ``ˈnà ˈétʃ͡i ˈhùrimi ˈnà}"\\
        \gll    ˈétʃ͡i ˈnápi ke ˈmí kaˈlé-a ˈní-li \textbf{o} biˈlé ˈtʃ͡ó ˈmá ˈhê aˈní-ame ka ru-ˈwá ˈnà ˈétʃ͡i ˈhùri-mi ˈnà\\
                \textsc{dem} \textsc{sub} \textsc{neg} \textsc{dem} like-\textsc{prog} \textsc{cop-pst} or one also already it say-\textsc{ptcp} \textsc{cop.irr} say-\textsc{mpass} \textsc{prox} \textsc{dem} send-\textsc{mot.imp.sg} \textsc{prox}\\
        \glt    `(He sends it) to someone he doesn't like or someone else tells him:``go and send it to that one”.'\\
        \glt    `(Se lo manda) a uno que no le cae bien u otro le dice: ``mándaselo a ese".'   \corpuslink{tx5[05_118-05_155].wav}{LEL tx5:05:11.8}, \corpuslink{tx5[05_155-05_189].wav}{LEL tx5:05:15.5}\\
    }
        \z
\z

The use of the borrowed \ili{Spanish} disjunctive conjunction \textit{o} is also attested in coordination of interrogative clauses, as shown in (\ref{ex: disjunction interrogative clauses}).

\ea\label{ex: disjunction interrogative clauses}

    \textit{naʔˈpôma? o ˈí ʃiˈkâ-la kaˈpòma?}\\
    \gll    naʔˈpô-ma \textbf{o} ˈí siˈkâ-la kaˈpò-ma?\\
            weed-\textsc{fut.sg} or here hand-\textsc{poss} break-\textsc{fut.sg}\\
    \glt    `(You mean) to weed? Or to break one's hand here?'\footnote{The speaker points at her own hand.}\\
    \glt    `¿(Quieres decir) escardar? ¿O quebrarse aqui en la mano?'    \corpuslink{co1236[00_064-00_093].wav}{JLG co1236:00:06.4}\\

\z

\subsubsection{Disjunction marked with \textit{wèra} `or'}
\label{subsubsec: disjunction with we}

A second construction that employs syndetic coordination to encode disjunction is one that employs the marker \textit{ˈwèra}, which has the same distribution as the borrowed disjunctive conjunction \textit{o}. This construction is exemplified in (\ref{ex: disjunction with we}).

%\pagebreak

\ea\label{ex: disjunction with we}

    \ea[]{
    \textit{ˈkúmi ˈátʃ͡i ˈtôola ˈlé, ˈwèra ˈpé risoˈtʃ͡í roˈʔáli ˈlé}\\
    \gll    ˈkúmi ˈátʃ͡i ˈtô-la aˈlé \textbf{ˈwèra} ˈpé risoˈtʃ͡í roˈʔ-á-li aˈlé\\
            where Q take.\textsc{pst.pass-rep} \textsc{dub} or just cave lay.down.\textsc{sg-tr-pst} \textsc{dub}\\
    \glt    `They don't know if they would bury him or they would lay him down in a cave.'\\
    \glt    `No saben si lo enterraban o lo llevaban a una cueva.’  \corpuslink{tx109[03_027-03_057].wav}{LEL tx109:03:02.7}\\
}
    \z
\z

\subsubsection{Disjunction marked with \textit{=ma} `or'}
\label{subsubsec: disjunction with ma}

Clausal disjunction may also be achieved through cliticization of the disjunctive connective \textit{=ma}, a postpositive marker. This is exemplified in (\ref{ex: disjunction with =ma}).

\ea\label{ex: disjunction with =ma}

       \ea[]{
        \textit{koˈbísi loˈkása ba, baʔˈwí ma, baˈhîrʃia ˈma ba}\\
        \gll    koˈbísi loˈká-sa ba baʔˈwí=ma baˈhî-ri-si-a=ma ba\\
                pinole take.pinole-\textsc{cond} \textsc{cl} water=or drink-\textsc{caus-mot-prog}=or \textsc{cl}\\
        \glt    `When she takes pinole, or water, or giving her to drink some water.’\\
        \glt    `Cuando toma pinole o agua o dándole de tomar agua.’    \corpuslink{tx19[04_084-04_127].wav}{LEL tx19:04:08.4}\\
    }
    \z
\z

Disjunction of phrases is also attested with the enclitic \textit{=ma}. This is exemplified below, where the disjunction morpheme appears after each phrase in a monosyndetic construction (\ref{ex: disjunction of nominal phrasesa}), or the two last phrases in a construction with multiple coordinands (\ref{ex: disjunction of nominal phrasesb}).

\ea\label{ex: disjunction of nominal phrases}

    \ea[]{
    \textit{ˈpé biˈlé baˈrîka ma ˈpé oˈkwâ ma}\\
    \gll    ˈpé biˈlé baˈrîka=ma ˈpé oˈkwâ=ma\\
            just one tank=or just two=or\\
    \glt    `One cask or two.'\\
    \glt    `Una barrica o dos.'   \corpuslink{tx60[00_248-00_269].wav}{BFL tx60:00:24.8}\\
}\label{ex: disjunction of nominal phrasesa}
        \ex[]{
        \textit{biˈlé ariˈmûli, oˈkwâ ariˈmûli ma biˈkiá ariˈmûli ma}\\
        \gll    biˈlé ariˈmûli, oˈkwâ ariˈmûli=ma biˈkiá ariˈmûli=ma\\
                one decaliter two decaliter=or three decaliter=or\\
        \glt    `One decaliter, two deacaliters or three decaliters.'\\
        \glt    `Un decalitro o dos decalitros o tres decalitros.’ \corpuslink{tx68[00_258-00_292].wav}{LEL tx68:00:25.8}\\
    }\label{ex: disjunction of nominal phrasesb}
    \z
\z

\subsubsection{Disjunction through parataxis}
\label{subsubsec: disjunction through parataxis}

Disjunction may also be achieved through parataxis. This is exemplified in the following text fragment, where (\ref{ex: disjunction parataxisb}) is interpreted involving disjunction within the larger context.

\ea\label{ex: disjunction parataxis}

    \ea[]{
    \textit{ˈkúm ko ˈtòli ba?}\\
    \gll    ˈkúmi=ko ˈtò-li ba\\
            where=\textsc{emph} take-\textsc{pst} \textsc{cl}\\
    \glt    `Where did he take it?'\\
    \glt    `¿Dónde lo llevaría?' \corpuslink{tx84[07_152-07_161].wav}{LEL tx84:07:15.2}\\
}\label{ex: disjunction parataxisa}
        \ex[]{
        \textit{ˈmá biˈlá biˈnôi ʃuˈwára aˈlé, ˈnà riˈpáki raˈbô ˈtòli ˈlé pa}\\
        \gll    ˈmá biˈlá biˈnôi suˈwára aˈlé ˈnà riˈpá-ki raˈbô ˈtò-li aˈlé pa\\
                already indeed himself finish.\textsc{prs} \textsc{dub} \textsc{dem} up-\textsc{supe} hill take-\textsc{pst} \textsc{dub} \textsc{cl}\\
        \glt    `He finished it up or took him up the hill.'\\
        \glt    `Se lo acabó o se lo llevó para el cerro.'  \corpuslink{tx84[07_161-07_186].wav}{LEL tx84:07:16.1}\\
    }\label{ex: disjunction parataxisb}
            \ex[]{
            \textit{ke piˈláti maˈtʃ͡í ˈkúm ˈtòli ba}\\
            \gll    ke biˈlá=ti maˈtʃ͡í ˈkúmi ˈtò-li ba\\
                    \textsc{neg} really=\textsc{1pl.nom} know.\textsc{prs} where=\textsc{dem} take-\textsc{pst} \textsc{cl}\\
            \glt    `We don't know where he took him.'\\
            \glt    `No sabemos dónde se lo llevó.' \corpuslink{tx84[07_186-07_202].wav}{LEL tx84:07:18.6}\\
        }\label{ex: disjunction parataxisc}
    \z
\z

\subsection{Adversative conjunction}
\label{subsec: adversative conjunction}

Adversative conjunction may be encoded through asyndesis or through syndetic coordination with a dedicated adversative conjunction. Each of these constructions is addressed next.

\subsubsection{Asyndetic adversative conjunction}
\label{subsubsec: asyndetic adversative conjunction}

Choguita Rarámuri possesses an adversative conjunction construction, which may be encoded through parataxis, like other forms of clausal coordination. This is shown in (\ref{ex: adversative conjunction parataxis}).

%\pagebreak

\ea\label{ex: adversative conjunction parataxis}

    \ea[]{
    \textit{wiˈʰkâ napaˈwíka raˈʔìtʃ͡ili ˈmí riʔˈlé-ki esˈk\textsuperscript{w}êlitʃ͡i ko, ˈnè ke ʃiˈné iˈjêna}\\
    \gll    wiˈʰkâ napaˈwí-ka raˈʔìtʃ͡i-li ˈmí reʔˈlé-ki esˈk\textsuperscript{w}êla-tʃ͡i=ko, ˈnè ke siˈné iˈjêna\\
            many gather-\textsc{ger} speak-\textsc{pst} \textsc{dist} below-\textsc{supe} school-\textsc{loc=emph} \textsc{1sg.nom} \textsc{neg} once go.\textsc{sg.pst}\\
    \glt    `Many (people) got together to speak over there at the school, but I've never gone.'\\
    \glt    `Se juntaron muchos allá en la escuela a platicar, pero yo nunca he ido.'   \corpuslink{co1137[07_348-07_394].wav}{MDH co1137:07:34.8}\\
}
        \ex[]{
        \textit{ˈmá   biˈlá   rikiˈnâmo     oˈlá   riˈké  ti   roˈplâno   ke   biˈlá   oˈmêaki   riˈkîna}\\
        \gll    ˈmá   biˈlá   rikiˈnâ-mo    oˈlá   riˈké  ti   roˈplâno   ke   biˈlá   oˈmêaki   riˈkîna\\
                already  really   go.down-\textsc{fut.sg} \textsc{cer}   perhaps  \textsc{def.bad} plane     \textsc{neg} really   be.able-\textsc{pst} go.down  \\
        \glt    `And the plane was about to go down, but it couldn’t go down.’\\
        \glt    `Ya se iba a bajar el avión, nomás que no pudo bajar.'   \corpuslink{tx12[01_588-02_035].wav}{SFH tx12:01:58.8}\\
    }
            \ex[]{
            \textit{ke biˈlé muˈrúli ˈhípi ko, ˈétʃ͡i biˈlá ˈá muˈrúma aˈlé ba}\\
            \gll    ke biˈlé muˈrú-li ˈhípi=ko ˈétʃ͡i biˈlá ˈá muˈrú-ma aˈlé ba\\
                    \textsc{neg} one carry.with.arms-\textsc{pst} today=\textsc{emph} \textsc{dem} indeed \textsc{aff} carry.with.arms-\textsc{fut.sg} \textsc{dub} \textsc{cl}\\
            \glt    `He didn't carry any (wood), but he will bring it.'\\
            \glt    `Ahora no trajo (leña), pero él la va a acarrear.'  \corpuslink{co1236[03_211-03_237].wav}{JLG co1236:03:21.1}\\
        }
    \z
\z

\subsubsection{Adversative conjunction marked with \textit{naˈlîna} `but'}
\label{subsubsec: adversative conjunction with nalina}

Adversative conjunction may also be marked through the connective \textit{naˈlîna} ‘but’ (or its reduced form \textit{aˈlîna} or \textit{ˈlîna})). As the following examples show, in these constructions the adversative conjunction appears between the two conjoined clauses (e.g., (\ref{ex: adversative conjunction with lina})).

\ea\label{ex: adversative conjunction with lina}

    \ea[]{
    \textit{aʔˈlì ˈnà kotʃ͡iˈká buˈʔílo maˈjêli, ˈlîna ke ˈtâsi kotʃ͡iˈká buˈʔíli ˈétʃ͡i reˈhòi ko}\\
    \gll    aʔˈlì ˈnà kotʃ͡i-ˈká buˈʔí-l-o maˈjê-li \textbf{ˈlîna} ke ˈtâsi kotʃ͡i-ˈká buˈʔí-li ˈétʃ͡i reˈhòi=ko\\
            and   then   sleep-\textsc{ger} lay.down.\textsc{sg-pst-ep} think-\textsc{pst} {but} \textsc{neg} \textsc{neg} sleep-\textsc{ger} lay.down-\textsc{pst} \textsc{dem} man=\textsc{emph}\\
    \glt    `And then he thought he was asleep, but the man was not asleep.’  \\
    \glt    `Nomás que pensó que estaba dormido (lit. acostado durmiendo), nomás que no estaba dormido el señor.’  \corpuslink{tx5[00_350-00_383].wav}{LEL tx5:00:35.0},  \corpuslink{tx5[00_383-00_409].wav}{LEL tx5:00:38.3}\\
}
        \ex[]{
        \textit{moˈʔêa tʃ͡uˈkúrili ˈtʃ͡ó ˈáa ba, paˈtʃ͡á ˈsàbia ke ˈlé ba, naˈlîna be ke ˈtâsi riˈwáli ˈtʃ͡ó ˈá riˈká ba}\\
        \gll    moˈʔ-ê-a tʃ͡uˈkúri-li ˈtʃ͡ó ˈá-a ba paˈtʃ͡á ˈsàwi-a ke aˈlé ba, \textbf{naˈlîna} be ke ˈtâsi riˈwá-li ˈtʃ͡ó ˈá riˈká ba\\
                head-\textsc{vblz-prog} become.bent-\textsc{pst} also look.for-\textsc{prog} \textsc{cl} inside smell-\textsc{prog} perhaps \textsc{dub} \textsc{cl} {but} be \textsc{neg} \textsc{neg} see-\textsc{pst} also \textsc{aff} indeed \textsc{cl}\\
        \glt    `He got his head in it (the bottle) looking for it, smelling inside, but he didn't see him.'\\
        \glt    `Metió la cabeza buscando, oliendo para buscarlo, pero de todas formas no lo vió.’    \corpuslink{tx152[01_326-01_361].wav}{SFH tx152:01:32.6}, \corpuslink{tx152[01_361-01_400].wav}{SFH tx152:01:36.1}\\
    }
            \ex[]{
            \textit{ˈpé ˈtáa biˈlá ko riˈhòi ˈníla ˈrá aʔˈlì ˈétʃ͡i ko naˈlîna ˈwé naˈlîna ˈwé hiˈwêlami ˈtʃ͡ó ˈníla ˈlá ˈnápu riˈká}\\
            \gll    ˈpé ˈtá biˈlá=ko riˈhòi ˈní-la ru-ˈwá aʔˈlì ˈétʃ͡i=ko \textbf{naˈlîna} ˈwé \textbf{naˈlîna} ˈwé hiˈwê-l-ame ˈtʃ͡ó ˈní-la ru-ˈwá\\
                    just small.\textsc{sg} indeed=\textsc{emph} man \textsc{cop-rep} say-\textsc{mpass} and \textsc{dem=emph} but \textsc{int} {but} \textsc{int} be.strong-\textsc{pst-ptcp} also \textsc{cop-rep} say-\textsc{mpass}\\
            \glt    `He was very small, but he was a very strong man.'\\
            \glt    `Era muy chiquito, pero nomás que era un hombre muy fuerte.'  \corpuslink{tx43[02_389-02_442].wav}{SFH tx43:02:38.9}\\
        }
    \z
\z

%Adversative conjunction marked with \textit{we}

%ˈa bilani pee riˈkom ˈnata ˈnili we ni ho ke me aʔˈlám ˈnata ˈnili\\
%\corpuslink{tx43[12_597-13_032].wav}{SFH tx43:12:59.7}\\

\section{Verbal chaining structures}
\label{sec: clause chaining}

Choguita Rarámuri has a construction that involves a sequence of clauses, where one of the clauses may be marked with canonical inflection (and show no restrictions in terms of TAM marking), while the remaining clauses can only be marked with special inflection (the gerundive suffix \textit{-ká}) and show overall more restricted structures. This inflection mainly conveys a temporal relation of chronological overlap or chronological sequence, though, as discussed below, these temporal notions may have extended semantic meanings in some cases. In the following example, the first clause, marked with the stress-shifting suffix \textit{-ká}, glossed as gerundive (\textsc{ger}), conveys that two events (drinking and resting) take place simultaneously (\ref{ex: simultaneous reading -ka}).

%\pagebreak

\ea\label{ex: simultaneous reading -ka}

    \textit{ˈwé pi ko ne ku iˈsâbika baˈhîba ˈlé, ˈmá ˈòwisa ˈnà paˈtʃ͡í pa}\\
    \gll    ˈwé pi=ko ne ku iˈsâbi-\textbf{ka} baˈhî-ba aˈlé ˈmá ˈòwi-sa ˈnà paˈtʃ͡í pa\\
            \textsc{int} just=\textsc{emph} \textsc{int} \textsc{rev} rest-{\textsc{ger}} drink-\textsc{irr.pl} \textsc{dub} already fertilize-\textsc{cond} \textsc{dem} corn \textsc{cl}\\
    \glt    `They need to drink while they rest when they (are done) fertilizing this corn.'\\
    \glt     `Necesitan tomar descansando hasta que (terminen de) fertilizar este maíz.’ \corpuslink{in243[17_222-17_273].wav}{FLP in243:17:22.2}\\

\z

In contrast, in (\ref{ex: temporal sequence reading -ka}) the events conveyed occur in a temporal sequence, such that the event described in the finite clause (performing a ritual blessing) occurs after a series of events, described by the \textit{ka}-marked clauses, have taken place.

\ea\label{ex: temporal sequence reading -ka}

    \textit{baʔˈwí rataˈbátʃ͡ika, tʃ͡oˈpé ˈjûa ʃiˈmèrika, ku aˈwílitʃ͡i ʃimiˈká, wiˈrónipo ˈkútʃ͡i paˈtʃî ba}\\
    \gll    baʔˈwí rata-ˈbá-tʃ͡i\textbf{-ka} tʃ͡oˈpé ˈjûa siˈmèri-\textbf{ka} ku aˈwílitʃ͡i simi-\textbf{ˈká} wiˈró-ni-po ˈkútʃ͡i paˈtʃî ba\\
            water be.hot-\textsc{inch-tr-{ger}} Montezuma.pine with pass.on.top-\textsc{{ger}} \textsc{rev} ritual.patio go.\textsc{sg-{ger}} make.blessing-\textsc{appl-fut.pl} \textsc{def} corn \textsc{cl}\\
    \glt    `Having heated up the water, and having passed on top with a (lit) Montezuma pine, and having gone back to the ritual patio, we make the blessing with corn.'\\
    \glt    `Calentando agua, pasando por arriba con un ocote (prendido), yendo al patio ritual, hacemos la bendición (``echamos el agua") con el maíz.'    \corpuslink{in485[07_084-07_107].wav}{ME in485:07:08.4}, \corpuslink{in485[07_114-07_155].wav}{ME in485:07:11.4}\\

\z

This construction is extremely frequent and attested extensively across speech genres in the Choguita Rarámuri corpus. The following examples are from procedural texts, where the temporal relations involve sequences of events (\ref{ex: -ka clauses in procedural texts}) (the finite clauses in these examples are marked as medio-passive).

\ea\label{ex: -ka clauses in procedural texts}

    \ea[]{
    \textit{aˈnáwika beˈlá rupuˈnàwa ba}\\
    \gll    aˈnáwi-\textbf{ka} beˈlá rupu-ˈnà-wa ba\\
            measure-\textsc{{ger}} really tear-\textsc{tr-mpass} \textsc{cl}\\
    \glt    `Having measured it, it is torn up.’\\
    \glt    `Una vez medido se troza.’   \corpuslink{tx1[01_214-01_228].wav}{BFL tx1:01:21.4}\\
}
%\pagebreak
        \ex[]{
        \textit{suˈnù ku aˈnáaga aˈrîo oˈlá waˈrîtʃ͡i ko}\\
        \gll    suˈnù ku aˈnáwi-\textbf{ka} a-ˈrîwo oˈlá waˈrî-tʃ͡i=ko\\
                corn \textsc{rev} measure-\textsc{{ger}} give-\textsc{mpass} \textsc{cer} palm.basket-\textsc{loc}=\textsc{emph}\\
        \glt    `Having measured the corn, it is given to them, in the \textit{wari} (palm basket).'\\
        \glt    `Se mide el maíz y se les da, en el wari (canasta de palma).'  \corpuslink{co1136[05_448-05_468].wav}{MDH co1136:05:44.8}\\
    }
            \ex[]{
            \textit{aʔˈlì ˈmá ˈnà ... ˈmá oˈtʃ͡é ku ˈnà baʔˈwêtʃ͡i ˈnà ... ˈnà muˈtʃ͡ûuka ku ˈnà paˈkóka ˈnà batuˈʃíwa mataˈtʃ͡í}\\
            \gll    aʔˈlì ˈmá oˈtʃ͡é ku ˈnà baʔˈwê-tʃ͡i ˈnà muˈtʃ͡ûwi-\textbf{ka} ku ˈnà               paˈkó-\textbf{ka} ˈnà batuˈsí-wa mata-ˈtʃ͡í\\
                    and already once.again \textsc{rev} \textsc{dem} water-\textsc{loc} \textsc{dem} put-\textsc{{ger}} \textsc{rev} \textsc{dem} wasah-\textsc{{ger}} \textsc{dem} grind-\textsc{mpass} grinding.stone-\textsc{loc}\\
            \glt    `And then having soaked (placed) it again in the water, and having washed (rinsed) it, then it is ground in the grinding stone.'\\
            \glt    `Y luego se remoja otra vez en agua y se enjuaga ya para molerlo en el metate.’  \corpuslink{tx68[00_588-01_091].wav}{LEL tx68:00:58.8}\\
        }
    \z
\z

Clauses marked with \textit{-ká} are non-finite, while the final clause in each of these examples (the `unmarked' clause) may be inflected like an independent sentence. Furthermore, marked clauses are not introduced by any overt subordinator. These properties are characteristic of clause chaining structures (see \citealt{longacre2007sentences} for an overview), though the formal and semantic properties of the Choguita Rarámuri construction with \textit{ka}-marked clauses blur the lines between clause chaining and other types of related constructions in this language.\footnote{\citet{moralesmoreno2016rochecahi} analyzes cognate constructions with the suffix \textit{-ká} as involving secondary predication in \ili{Rochéachi Rarámuri}. This suffix is described as a resultative state suffix in \ili{Norogachi Rarámuri} \citep{villalpando2019grammatical}. The cognate \textit{-ka/ga} suffix is described as a past participle \citep{miller1996guarijio} in closely-related \ili{Mountain Guarijío}, as a participial in \ili{River Guarijío} \citep{felix2006grammar} and as a participial also in \ili{Yaqui} (\citealt{dedrick1999sonora}, \citealt{guerrero2004yaqui}, \citealt{guerrero2019fenomeno}).}

As exemplified in (\ref{ex: simultaneous reading -ka}) above, the \textit{ka} construction in Choguita Rarámuri often involves two verbs which are covalent, i.e. they share one or more arguments, a property also present in serial verb constructions (see §\ref{subsec: serial verb constructions} below). As shown in (\ref{ex: temporal sequence reading -ka}) and below in (\ref{ex: overt object in ka marked clause}), the \textit{ka}-marked clauses may contain an overt object argument not shared with the unmarked clause.

%\pagebreak

\ea\label{ex: overt object in ka marked clause}

   \ea[]{
   \textit{ke biˈlé ˈpé ta ˈsôda baˈhîka oˈtʃ͡êrila tʃ͡aˈbè go}\\
   \gll     ke biˈlé ˈpé ta ˈsôda baˈhî-\textbf{ka} oˈtʃ͡êri-la tʃ͡aˈbè=ko\\
            \textsc{neg} one just \textsc{def} soda drink-\textsc{{ger}} grow-\textsc{rep.ds} before=\textsc{emph} \textsc{cl}\\
    \glt    `They\textsubscript{j} say they\textsubscript{k} grew up without drinking soda before.’\\
    \glt    `Dicen\textsubscript{j} que crecieron\textsubscript{k} sin tomar soda antes.’  \corpuslink{tx12[04_187-04_211].wav}{SFH tx12:04:18.7}\\
}\label{ex: overt object in ka marked clausea}
        \ex[]{
        \textit{ˈtáa tʃ͡iˈkîto muˈtûka ku naˈwàli ˈtʃ͡ó}\\
        \gll    ˈtá tʃ͡iˈkîto muˈtû-\textbf{ka} ku naˈwà-li ˈtʃ͡ó\\
                small small hold.on.arms-\textsc{{ger}} \textsc{rev} arrive-\textsc{pst} also\\
        \glt    `She arrived holding a small one in her arms too.'\\
        \glt    `Llegó con un chiquito en los brazos también.’   \corpuslink{tx32[09_033-09_051].wav}{LEL tx32:09:03.3}\\
    }\label{ex: overt object in ka marked clauseb}
            \ex[]{
            \textit{tʃ͡aˈbèe go, siˈnêam ˈká biˈlá ko ˈwé awiˈwái tʃ͡aˈbèe go ba, napaˈbûka ba suˈnù}\\
            \gll    tʃ͡aˈbè=ko siˈnê-ame ˈká biˈlá=ko ˈwé awi-ˈwá-i tʃ͡aˈbè=ko ba, napa-ˈbû-\textbf{ka} ba suˈnù\\
                    before=\textsc{emph} all-\textsc{ptcp} \textsc{cop.irr} indeed=\textsc{emph} \textsc{int} dance-\textsc{mpass-impf} before=\textsc{emph} \textsc{cl} get.together-\textsc{tr-{ger}} \textsc{cl} corn\\
            \glt    `Long time ago, everybody used to dance a lot long ago, while gathering corn.'\\
            \glt    `Hace mucho, todos bailaban mucho antes, juntando maíz.'   \corpuslink{co1140[15_463-15_510].wav}{MDH co1140:15:46.3}\\
        }\label{ex: overt object in ka marked clausec}
    \z
\z

In each of the examples in (\ref{ex: overt object in ka marked clause}), the finite clause is headed by an intransitive verbal predicate. In (\ref{ex: non-coreferential objects}) below, the finite clause is headed by a transitive verb, and the \textit{ka}-marked clause contains an overt object (\textit{biˈlé baˈrîka} `one cask') that is non-coreferential with the object argument of the finite clause (\textit{baʔˈwí} `water'). Both clauses share an external argument, the second singular subject argument marked in the main, finite clause.

\ea\label{ex: non-coreferential objects}

    \textit{aʔˈlì ˈmámi baʔˈwí roˈʔèma oˈhòsa aˈnâuka biˈlé baˈrîka}\\
    \gll    aʔˈlì ˈmá=mi baʔˈwí roˈʔ-è-ma oˈhò-sa aˈnâwi-\textbf{ka} biˈlé baˈrîka\\
            and already=\textsc{2sg.nom} water pour.\textsc{appl-fut.sg} dekernel-\textsc{cond}  measure-\textsc{{ger}} one cask\\
    \glt    `When you dekernel it you pour water, having measured one cask.’\\
    \glt    `Cuando lo desgranas y ya le echas agua, ya que mides una barrica.’  \corpuslink{tx60[00_272-00_305].wav}{BFL tx60:00:27.2}\\

\z

In this example (as in (\ref{ex: overt object in ka marked clausec})), the clause marked with \textit{-ká} appears after the finite clause. Clause chaining structures are often described as `medial-final' clause chaining in the typological literature \citep{longacre2007sentences}, but in Choguita Rarámuri the linear order of the marked and unmarked clauses is variable (as also documented in \ili{Northern Paiute} (\ili{Numic}) \citep{toosarvandani2016temporal}). Another example of this is shown in (\ref{ex: verb with -ka after}).

\ea\label{ex: verb with -ka after}

    \textit{noˈkí naˈwàko ˈlá poˈtʃ͡íka}\\
    \gll    noˈkí naˈwà-ki oˈlá poˈtʃ͡í-\textbf{ka}\\
            almost arrive-\textsc{pst.ego} \textsc{cer} jump-\textsc{{ger}}\\
    \glt    `He almost arrived jumping (``with a single jump").'\\
    \glt    `Casi llegó a la puerta brincando (``de un brinco").’    \corpuslink{tx84[03_311-03_330].wav}{LEL tx84:03:31.1}\\

\z

In (\ref{ex: verb with -ka after}), the \textit{ka}-marked clause depicts an event that temporally precedes the event described in the finite clause, and can be interpreted as encoding manner (the event, arriving to a location, was accomplished through jumping). Other examples where the \textit{ka}-marked clauses may be interpreted in a manner sense are shown in (\ref{ex: cause and effect constructions}).

\ea\label{ex: cause and effect constructions}

    \ea[]{
    \textit{kuˈsìtini     wipiˈsóka    miˈʔàki}\\
    \gll    kuˈsì-ti=ni     wipiˈsó-\textbf{ka}    miˈʔà-ki\\
            stick-\textsc{inst=1sg.nom} hit.w.stick-{\textsc{ger}} kill\textsc{.sg}-\textsc{pst.ego}\\
    \glt    `I hit it to death with a stick.’\\
    \glt    `La maté a palos (a la rata).’  < BFL 09 1:88/el >\\
}
        \ex[]{
        \textit{baˈtʃ͡ókati  niˈwâbo  kaˈlí}\\
        \gll    baˈtʃ͡ó-\textbf{ka}=ti      niˈwâ-bo  kaˈlí\\
                fix.with.mud-\textsc{{ger}=1pl.nom} make-\textsc{fut.pl} house\\
        \glt    `We made the house by fixing the walls with mud.’\\
        \glt    `Enzoquetando hicimos la casa.'    < BFL 09 1:88/el >\\
    }
            \ex[]{
            \textit{weˈpáka  miˈʔàlo  basaˈtʃ͡î}\\
            \gll   weˈpá-\textbf{ka}  miˈʔà-li basaˈtʃ͡î\\
                    hit-\textsc{{ger}} kill.\textsc{sg-pst} coyote\\
            \glt    `They hit the coyote to death.’\\
            \glt   `Mataron a golpes al coyote.’    < BFL 09 1:88/el >\\
        }
                \ex[]{
                \textit{naˈhâtika  miˈʔàa    ruˈwáo  tʃ͡imoˈrí    tʃ͡aˈbèi}\\
                \gll   naˈhâti-\textbf{ka}  miˈʔà-a    ru-ˈwá-o  tʃ͡imoˈrí    tʃ͡aˈbèi\\
                        chase-\textsc{{ger}} kill\textsc{.sg-prog} say-\textsc{mpass-ep} deer    before\\
                \glt    `They say they would chase deer to death before.’\\
                \glt    `Dicen que persiguiéndolo lo mataban al venado antes.’  < BFL 09 1:88/el >\\
            }
                    \ex[]{
                    \textit{baniˈʃúkani moʔoˈbûko    ˈkâha}\\
                    \gll   baniˈsú-\textbf{ka}=ni    moʔo-ˈbû-ki   ˈkâha\\
                           pull-\textsc{{ger}=1sg.nom} go.up\textsc{-tr-pst.ego} box\\
                    \glt    `I got the box up by pulling it.’\\
                    \glt    `Jalando subí la caja.’  < BFL 09 1:88/el >\\
                }
                        \ex[]{
                        \textit{``niˈhê ˈmá ku ˈmètika naˈwàki" aˈní}\\
                        \gll    niˈhê ˈmá ku ˈmèti-ka naˈwà-ki aˈní\\
                                \textsc{1sg.nom} already \textsc{rev} drive-\textsc{{ger}} arrive-\textsc{pst.ego} say.\textsc{prs}\\
                        \glt    ```I arrived driving" it is said.'\\
                        \glt    ```Llegué manejando" se dice.'   \corpuslink{el1278[03_475-03_502].wav}{JLG el1278:03:47.5}\\
                    }
    \z
\z

In these examples, the marked clauses precede temporally the event described in the finite clause. In other cases, as shown in (\ref{ex: simultaneous -ka with manner}), the sense of manner is conveyed also in cases where the temporal relationship would be one of chronological simultaneity.

\ea\label{ex: simultaneous -ka with manner}

    \ea[]{
        \textit{ˈétʃ͡i riˈká biˈlám wiˈróka riˈká biˈlá ku baˈjèa ˈrú ba ˈni}\\
        \gll    ˈétʃ͡i riˈká biˈlá=m wiˈró-\textbf{ka} riˈká biˈlá ku baˈjè-a ˈrú ba ˈni\\
                \textsc{dem} like indeed=\textsc{dem} throw.up-\textsc{{ger}} like indeed \textsc{rev} call-\textsc{prog} say.\textsc{prs} \textsc{cl} \textsc{emph}\\
        \glt    `Like that, throwing water up (blessing), that is how they would call it (the rain).'\\
        \glt    `Así tirando para arriba (bendiciendo) así era como le llamaban (a la lluvia).'   \corpuslink{in484[05_298-05_329].wav}{ME in484:05:29.8}\\
    }
            \ex[]{
            \textit{ˈpé itʃ͡aˈká ko koˈʔáa-li}\\
            \gll    ˈpé itʃ͡a-\textbf{ˈká}=ko koˈʔá-li\\
                    just sow-\textsc{{ger}=emph} eat-\textsc{pst}\\
            \glt    `They ate from what they planted.’\\
            \glt    `Sembrando comían (``comen sembrado").' \corpuslink{tx12[04_099-04_131].wav}{SFH tx12:04:09.9}\\
          }
                \ex[]{
                \textit{ˈétʃ͡i riˈgá naʔˈsòoka ˈbel koʔˈpôo ba}\\
                \gll    ˈétʃ͡i riˈká naʔˈsòwi-\textbf{ka} ˈbel koʔ-ˈpô ba\\
                        \textsc{dem} like.that mix-\textsc{{ger}} indeed eat-\textsc{fut.pl} \textsc{cl}\\
                \glt    `One has to eat like that, by mixing (the foods).'\\
                \glt    `Hay que comer así revuelto.'  \corpuslink{co1136[00_528-00_548].wav}{MDH co1136:00:52.8}\\
            }
    %\pagebreak
                        \ex[]{
                        \textit{ˈmèti ˈbél aˈlé ba, ˈmèka ˈbél isiˈmípili aˈlé ku ba}\\
                        \gll    ˈmèti beˈlá aˈlé ba ˈmè-\textbf{ka} beˈlá i-siˈmípi-li aˈlé ku ba\\
                                drive.\textsc{prs} indeed \textsc{dub} \textsc{cl} drive-\textsc{{ger}} indeed \textsc{pl-}go.\textsc{pl-pst} \textsc{dub} \textsc{rev} \textsc{cl}\\
                        \glt    `They drive, they go back (in their trucks) by driving.'\\
                        \glt    `Manejan, se van (en las camionetas) otra vez manejando.'   \corpuslink{co1136[07_572-08_006].wav}{MDH co1136:07:57.2}\\
                    }
                            \ex[]{
                            \textit{ˈnâli ˈpé beˈláni ko ˈnè iˈjêla iʔˈnèka biˈnè ˈnè ko tʃ͡ú reˈká oˈlá ˈétʃ͡i ba}\\
                            \gll    ˈnâli ˈpé beˈlá=ni=ko ˈnè iˈjê-la iʔˈnè-ka biˈnè ˈnè=ko tʃ͡ú reˈká oˈlá ˈétʃ͡i ba\\
                                    then just really=\textsc{1sg.nom=emph} \textsc{1sg.nom} mom-\textsc{poss} watch-\textsc{{ger}} learn.\textsc{pst} \textsc{1sg.nom=emph} how like do.\textsc{prs} \textsc{dem} \textsc{cl}\\
                            \glt    `Then I learned by watching my mom, how would she make it.’\\
                            \glt    `Entonces yo aprendí viendo a mi mamá a ver cómo le hacía.’  \corpuslink{tx1[00_555-01_020].wav}{BFL tx1:00:55.5}\\
                        }
    \z
\z

In the following example (\ref{ex: cause and effect -ka}), the temporal sequence between the clauses has an interpretation of cause and effect.

\ea\label{ex: cause and effect -ka}

    \textit{baniˈwîtika, witʃ͡iˈmêa ˈlémi!}\\
    \gll    baniˈwîti-\textbf{ka} witʃ͡i-ˈmêa aˈlé=mi\\
            entangle-\textsc{{ger}} fall-\textsc{fut.sg} \textsc{dub=2sg.nom}\\
    \glt    `You will trip and you will fall!'\\
    \glt    `¡Te vas a enredar y te vas a caer!'  \corpuslink{co1136[12_186-12_204].wav}{MDH co1136:12:18.6}\\

\z

According to \citet[400]{longacre2007sentences}, the temporal relations encoded in clause chaining may give rise to associated semantic readings, such as result, cause and effect, among others. The sentences in (\ref{ex: cause and effect constructions}) and (\ref{ex: simultaneous -ka with manner}) may thus be analyzed as clause chaining.\footnote{Alternatively, the gerundive \textit{-ka} suffix may be deployed in a different type of construction that may be better analyzed as involving adverbial modification, with \textit{ka}-marked clauses better analyzed as sentence margins in a co-ranking structure.}

As exemplified so far, none of the \textit{ka}-marled clauses are introduced by an overt subordinator. There are, however, cases, where an overt coordinator may appear in a clause chain. This is shown in (\ref{ex: overt coordinator ka clauses}), where the coordinator occurs sentence-initially, as well as in the clause chain.

%\pagebreak

\ea\label{ex: overt coordinator ka clauses}

\textit{aʔˈlì baʔˈwí ˈhûa naˈsòo-ka aʔˈlì ˈhê riˈká taˈmò uˈtʃ͡-è-i}\\
\gll    \textbf{aʔˈlì} baʔˈwí ˈhûa naˈsòwi-ka \textbf{aʔˈlì} ˈhê riˈká taˈmò uˈtʃ-è-i\\
       {and} water with mix-\textsc{ger} {and} it like.that \textsc{1pl.acc} put-\textsc{appl-impf}\\
\glt    `And so mixing it with water, and then he would put it on us like that.'\\
\glt    `Y entonces mezclándola con agua y entonces así nos lo ponía.'    \corpuslink{tx785[02_496-02_531].wav}{GFM tx785:02:49.6}\\

\z

As discussed in \citet{toosarvandani2021chaining}, other \ili{Uto-Aztecan} languages that are documented to have clause chaining allow overt coordinators in chaining structures.

Finally, there are cases where negation only has scope over the \textit{ka}-marked clause. This is exemplified in (\ref{ex: scope of negation}).

\ea\label{ex: scope of negation}

\textit{ke piˈláti biˈlé matʃ͡iˈká moˈtʃ͡íli taˈmò ˈká ˈtʃ͡è wiˈle ˈtʃ͡ó matʃ͡iˈwá aʔˈlì ko ba}\\
\gll    ke biˈlá=ti biˈlé [matʃ͡i-\textbf{ˈká} moˈtʃ͡í-li] taˈmò, ˈká ˈtʃ͡è biˈlé ˈtʃ͡ó matʃ͡i-ˈwá aʔˈlì=ko ba\\
        \textsc{neg} indeed=\textsc{1pl.nom} one know-\textsc{{ger}} be.sitting.\textsc{pl-pst} \textsc{1pl.nom} because \textsc{neg} one yet know-\textsc{mpass} then\textsc{=emph} \textsc{cl}\\
\glt    `Well, we were without knowing, because then nothing was known yet.'\\
\glt    `Pues nosotros estábamos sin saber porque en ese tiempo todavía no se sabía nada.' \corpuslink{tx372[00_414-00_459].wav}{LEL tx372:00:41.4}\\

\z

The Choguita Rarámuri \textit{-ká} suffix is cognate with a \textit{-ka} marker in \ili{Hiaki} (\ili{Taracahitan}) that is analyzed as a medial verb marker \citep{fabian2006yaqui}, which may convey both a simultaneous and a sequential interpretation.\footnote{It should be noted, however, that other studies of \ili{Yaqui} syntax (\citealt{dedrick1999sonora}, \citealt{guerrero2004yaqui}, \citealt{guerrero2019fenomeno}) do not analyze this marker as a medial verb marker.} Elsewhere in the \ili{Taracahitan} branch, \ili{Mountain Guarijío} is also described as possessing a system of medial markers, five morphemes, which may encode same vs. different subject distinctions, in addition to sequential and simultaneous distinctions \citep[200]{miller1996guarijio}. \citet{miller1996guarijio} remarks that this morphological system of medial verb markers is a recent innovation in the language.

\section{Complex predicates}
\label{sec: other complex constructions}

Choguita Rarámuri has complex constructions that can be broadly defined as involving complex predicates. While variously characterized in the literature depending on the particular morphosyntactic properties of individual languages under consideration, complex predicate constructions are nevertheless defined as involving a complex argument structure and a grammatical functional structure of a single predicate \citep[108]{butt1995structure}. Complex predicates thus involve a single `inflectional domain’ containing two or more distinct predicates, each of which selects for at least one argument in its argument structure \citep[247]{baker1997thematic} and each of which contributes to the predicate information that in other constructions is associated with a head (\citealt{butt2001semi}).

%``The analysis and typology of complex predicates presents a long term challenge to linguists across theoretical frameworks (see e.g., Kuroda 2003, Bowern 2008 for two general overviews of the phenomenon). One main challenge has been to identify the nature and properties of complex verb groupings, of which there are several types: serial verb constructions, typical of African and Austronesian languages; light verb constructions involving some kind of complement verb, and verbal complexes involving auxiliaries (see Butt 1993, Svenonius 2008)."

Complex predicate constructions may be further classified into sub-types, including light verb constructions and serial verb constructions, based on their syntactic and semantic properties (\citealt{butt1995structure}, \citealt{svenonius2008complex}). The status of auxiliary verb constructions as a type of complex predicate construction varies within the literature: \citet{butt1995structure}, for instance, excludes auxiliary verb constructions from the class of complex predicates in her typology, but the boundaries between this type of construction and others uncontroversially defined as involving complex predicates may be blurry \citep{svenonius2008complex}. There are four types of constructions in Choguita Rarámuri that may be broadly characterized as involving complex predicates: light verb constructions (§\ref{subsec: light verb and auxiliary constructions}), auxiliary verb constructions (§\ref{subsec: auxiliary verb constructions}), serial verb constructions (§\ref{subsec: serial verb constructions}) and multi-predicate verb constructions involving V-V incorporation that have been analyzed as serial verb constructions or light verb constructions in other \ili{Uto-Aztecan} languages (§\ref{subsec: V-V incorporation constructions}).

\subsection{Light verb constructions}
\label{subsec: light verb and auxiliary constructions}

While the term ``light verb'' is typically employed in the literature to characterize a wide range of structures cross-linguistically, it is typically used to refer to a class of verbs that are semantically bleached and which appear in a syntactically formed complex predicate (\citealt{brugman2001light}, \citealt{brugman2001light}, \citealt{butt2001semi}, \citealt{bowern2004bardi}). According to \citet{butt1995structure}, one key defining characteristic of light verbs is that they have a main verb use, which contrasts with its light counterpart in bearing descriptive content (see also \citealt{svenonius2008complex}; cf. \citealt{tubino2014affixal}). In addition to these characteristics, light verbs may impose selectional restrictions on the types of complements and arguments they combine with (e.g., light verbs may require their complement be of a certain transitivity value), in contrast to auxiliary verbs and serial verbs, which do not impose selectional restrictions. For discussion of light verb constructions in the closely related \ili{Hiaki} language (\ili{Taracahitan}), see \citet{tubino2014affixal}.

A set of constructions in Choguita Rarámuri fit the definition of light verb constructions. These constructions consist of two predicates, the first of which (the heavy predicate) contributes most of the lexical meaning, while the second (the light verb) contributes finite verbal categories, stance or valence. Choguita Rarámuri light verb constructions include the \textit{noˈká} ‘do’ construction (§\ref{subsubsec: the noka 'do' construction}), the \textit{ˈní-} ‘do’ construction (§\ref{subsubsec: the ni- 'do' construction}), and the \textit{iˈsì} ‘do’ construction (§\ref{subsubsec: the iʃi 'do' construction}).

\subsubsection{The \textit{noˈká} ‘do’ construction}
\label{subsubsec: the noka 'do' construction}

Like light verb constructions documented cross-linguistically, the verb \textit{noˈká} has a main verb use and a light verb use. As a main verb form, \textit{noˈká} means ‘move’, a predicate that describes a change in posture and which may undergo valence related alternations through the affixation of transitive and applicative suffixes that replace the final vowel of the stem (see §\ref{subsec: valence alternations}). Examples of this predicate as a main, free standing verbal predicate are given in (\ref{ex: main verb use of noka}).

\ea\label{ex: main verb use of noka}

    \ea[]{
    \textit{ˈmá noˈkáli}\\
    \gll    \textit{ˈmá} \textbf{\textit{noˈká-li}}\\
            already {move\textsc{-pst}}\\
    \glt    `S/he already moved.’\\
    \glt    `Ya se movió.’    <BFL 05 1:114/el>\\
}
        \ex[]{
        \textit{niˈhê ˈmá nokoˈmêa}\\
        \gll    niˈhê ˈmá \textbf{noko-ˈmêa}\\
                \textsc{1sg.nom} already {move\textsc{-fut.sg}} \\
        \glt    `I will move.’\\
        \glt    `Ya me voy a mover.’  <SFH 05 1:80/el>\\
    }
            \ex[]{
            \textit{niˈhê ˈmí ˈtrôka noˈkèli}\\
            \gll    niˈhê  ˈmí ˈtrôka \textbf{noˈk-è-li}  \\
                    1\textsc{sg.nom} \textsc{2sg.acc} truck {move-\textsc{appl-pst}}\\
            \glt    `I will move the truck for you.’ \\
            \glt    `Te voy a mover la troca.’   <SFH 05 1:80/el>\\
        }
    \z
\z

This predicate has a semantically bleached version found in multi-predicate constructions, acting as a light verb bearing inflection and following a verbal predicate that bears descriptive content (the descriptive (heavy) verb always precedes the light verb). This is exemplified in (\ref{ex: light verb use of noka}), where the heavy-light verb sequence is shown in square brackets.

%\pagebreak

\ea\label{ex: light verb use of noka}

    \ea[]{
    \textit{napaˈwía noˈkáli ˈlé ˈétʃ͡i ˈnà biˈlé riˈhò aʔˈlì biˈlé tʃ͡aˈbôtʃ͡i ˈʃîʔi}\\
    \gll    [\textbf{napaˈwí-a}  \textbf{noˈká-li} aˈlé]   ˈétʃ͡i   ˈnà   biˈlé   riˈhò aʔˈlì   biˈlé   tʃ͡aˈbôtʃ͡i ˈsî\\
            {get.together-\textsc{prog}} {do-\textsc{pst}} \textsc{dub} \textsc{dem} \textsc{dem} one man and one mestizo also\\
    \glt    `A (Rarámuri) man and a \textit{mestizo} (mixed mexican) man got together.’\\
    \glt    `Se juntaron un hombre (Rarámuri) y un mestizo.’  <SFH 06 choma(2)/tx>\\
}\label{ex: light verb use of nokaa}
        \ex[]{
        \textit{ˈnáp aʔˈlì ˈí biˈlá ... napaˈbûa noˈkíla ˈétʃ͡i tʃ͡aˈbôtʃ͡i aʔˈlì ˈétʃ͡i riˈhò ˈûa ba}\\
        \gll    ˈnápi aʔˈlì ˈí biˈlá [\textbf{napaˈbû-a} \textbf{noˈkí-la}] ˈétʃ͡i tʃ͡aˈbôtʃ͡i aʔˈlì ˈétʃ͡i riˈhò ˈjûa ba\\
                \textsc{rel} then here really {gather-\textsc{prog}} {do-\textsc{mpass}} \textsc{dem} mestizo and \textsc{dem} man with  \textsc{cl}\\
        \glt    `When the man and the mestizo were gathered.’\\
        \glt    `Cuando los juntaron al hombre y al mestizo.’    <SFH 06 choma(36)/tx>\\
    }\label{ex: light verb use of nokab}
                \ex[]{
                \textit{taˈmò ˈkútʃ͡uala ˈtʃ͡ó beˈnèma ˈtʃ͡ó ˈnà ˈtʃ͡útimi riˈkám ˈnàwa noˈká ˈnà oˈtʃ͡êra noˈká tʃ͡aˈbè kiˈʔà ba}\\
                \gll    taˈmò ˈkútʃ͡ua-la ˈtʃ͡ó beˈnè-ma ˈtʃ͡ó ˈnà ˈtʃ͡ú=timi riˈká=mi [\textbf{ˈnàwa} \textbf{noˈká}] ˈnà [\textbf{oˈtʃêr-a} \textbf{noˈká}] tʃ͡aˈbè kiˈʔà ba\\
                        1\textsc{pl.nom}  children-\textsc{poss}  also  learn-\textsc{fut.sg}  also  then   Q=2\textsc{pl.nom}  like=\textsc{dem} {arrive.\textsc{prs}} {do\textsc{.prs}} then {grow-\textsc{prog}} {do\textsc{.prs}} before before  \textsc{cl}\\
                \glt    `Our children will learn how you all grew up like before, how you used to live.’\\
                \glt    `Nuestros hijos también van a aprender cómo crecieron ustedes antes, como vivían.’ \corpuslink{in61[00_467-00_489].wav}{SFH in61:00:46.7},   \corpuslink{in61[00_492-00_531].wav}{SFH in61:00:49.2}\\
            }\label{ex: light verb use of nokac}
                    \ex[]{
                    \textit{nabiˈsûra   noˈkítʃ͡ino}\\
                    \gll    [\textbf{nabiˈsûr-a}   \textbf{noˈkí-tʃino}]\\
                            {form.line-\textsc{prog}} {move-\textsc{ev}} \\
                    \glt    `It sounds like they are forming a line.’    \\
                    \glt    ‘Se oye que se andan acomodando en fila.’  < BFL 07 el 30 04]\\
                }\label{ex: light verb use of nokad}
            %\pagebreak
                        \ex[]{
                        \textit{haˈré ko=ti ke  tʃ͡iˈkó=timi risoˈtʃ͡í má=m kaˈtêw-a noˈká-la ˈrú?}\\
                        \gll    haˈré=ko=ti ke  tʃ͡iˈkó=timi risoˈtʃ͡í má=mi [\textbf{kaˈtêwi-a} \textbf{noˈká-la}] ˈrú\\
                                some=\textsc{emph=1pl.nom} \textsc{neg} tʃ͡iˈkó=2\textsc{pl.nom} caves already=\textsc{dem} {keep-\textsc{prog}} {do-\textsc{rep}} say.\textsc{prs}\\
                        \glt    `Don’t some say that you all used to keep (corn) like that in caves?’\\
                        \glt    `¿Que no dicen que unos guardaban así en cuevas?’  \corpuslink{in61[03_434-03_466].wav}{SFH in61:03:43.4}\\
                    }\label{ex: light verb use of nokae}
    \z
\z

As shown in these examples, the light verb construction with \textit{noˈká} is attested with heavy predicates that encode activities, e.g., \textit{napaˈwía} `get together' (\ref{ex: light verb use of nokaa}), \textit{ˈnàwa} `arrive' (\ref{ex: light verb use of nokac}), \textit{nabiˈsûra} `get in a line' (\ref{ex: light verb use of nokad}) and \textit{kaˈtêwa} `store' (\ref{ex: light verb use of nokae}). In contrast, other light verbs may select for other types of predicates (e.g., the light verb \textit{ni-} `do' selects for stative and mental attitude predicates, as described in §\ref{subsubsec: the ni- 'do' construction} below).

The verb \textit{noˈká} has a morphologically related form \textit{onoˈká}, a plural/pluractional form derived through a prefix which is aligned in color with the vowel quality of the first vowel of the root (see §\ref{sec: pluractional marking} for more details about pluractional marking in this language). While the semantically full version of this predicate may be used in contexts involving multiple participants, it is not clear that the light verb \textit{onoˈká} marks plurality. Examples of this predicate in light verb constructions is shown in (\ref{ex: onoka examples}).
%It is also necessary to determine if there are any possible semantic differences between \textit{noˈka} and \textit{onoˈka} in complex predicate constructions.

\ea\label{ex: onoka examples}

    \ea[]{
        \textit{suˈnù   ʃuˈwâbuka   ˈtʃ͡útimi riˈkám iˈtʃ͡à onoˈká tʃ͡aˈbè}\\
        \gll    suˈnù   suˈwâbuka   ˈtʃ͡ú=timi riˈká=mi [\textbf{iˈtʃà} \textbf{onoˈká}] tʃ͡aˈbè\\
                corn  everything  Q=2\textsc{pl.nom}  like=\textsc{dem}  {sow.\textsc{prs}} {do.\textsc{prs}}  before\\
        \glt    `Corn, everything, how did you use to plant the crops before.’\\
        \glt    `El maíz, todo, cómo sembraban antes.’ \corpuslink{in61[01_027-01_058].wav}{SFH in61:01:02.7}\\
    }
            \ex[]{
            \textit{ˈtʃ͡ú ˈjêni  ˈtʃ͡émi kaˈjèni onoˈká ˈétʃ͡i suˈnù itʃ͡iˈsûa ˈétʃ͡i biˈlé baˈmíbiri  tʃ͡aˈbè ...?}\\
            \gll    ˈtʃ͡ú ˈjêni  ˈtʃ͡é=mi [\textbf{kaˈjèni} \textbf{onoˈká}] ˈétʃ͡i suˈnù itʃ͡i-ˈsûa ˈétʃ͡i biˈlé baˈmíbiri tʃ͡aˈbè\\
                    how much also=\textsc{dem} {yield.harvest.\textsc{prs}} {do\textsc{.prs}}  \textsc{dem} corn  sow-\textsc{cond.pass}  \textsc{dem}  one  year    before\\
            \glt    `And how much corn would the harvest yield in a year when you used to sow, before?’\\
            \glt    `¿Y cuánto se les daba de maíz cuando sembraban en un año, antes?’    \corpuslink{in61[02_425-02_478].wav}{SFH in61:02:42.5}\\
        }
                \ex[]{
                \textit{ˈpé riˈpòpa  ˈbítimi  ˈá kawiˈká naˈʔá onoˈká ba?} \\
                \gll    ˈpé riˈpòpa ˈbi=timi ˈá kawi-ˈká [\textbf{naˈʔá} \textbf{onoˈká}] ba \\
                        just back just=2\textsc{pl.nom} \textsc{aff} bring.wood-\textsc{ger} {make.fire} {do\textsc{.prs}} \textsc{cl}\\
                \glt    `Just with your backs did you used to carry wood, that’s how you used to make fire?’\\
                \glt    `Nomás en el lomo ustedes traían leña, ¿así hacían lumbre?’    < SFH 06 in61(233)/in >\\
            }
    \z
\z

There seem to be no restrictions as to the inflectional marking the light verb might bear. In most cases, the light predicate is inflected for present tense (as in, e.g., (\ref{ex: onoka examples})), but in the following examples, the light verb is marked as imperfective (\ref{ex: noka construction with different TAMa}) or past tense (\ref{ex: noka construction with different TAMb}).

\ea\label{ex: noka construction with different TAM}

    \ea[]{
    \textit{aʔˈlìtimi  ˈtʃ͡útimi  iˈjêni kotimi? ˈpîri  ˈûatimi iˈtʃ͡à onoˈká-i tʃ͡aˈbè ko?}\\
    \gll    aʔˈlì=timi  ˈtʃ͡ú=timi  iˈjêni=ko=timi ˈpîri  ˈjûa=timi [\textbf{iˈtʃà} \textbf{onoˈká-i}] tʃ͡aˈbè=ko? \\
            and=2\textsc{pl.nom}  Q=2\textsc{pl.nom}  go\textsc{.pl}=\textsc{emph}=2\textsc{pl.nom}  Q  with=2\textsc{pl.nom} {sow} {do-\textsc{impf} } before=\textsc{emph}  \\
    \glt    `And how did you go? With what did you use to sow before?’\\
    \glt    `¿Y ustedes cómo iban? ¿Con qué sembraban antes?’ \corpuslink{in61[02_155-02_205].wav}{SFH in61:02:15.5}\\
}\label{ex: noka construction with different TAMa}
\newpage
        \ex[]{
        \textit{ˈtʃ͡útimi  riˈkám oˈmáwa onoˈkáli ba ... ˈtʃ͡útimi riˈká=m kaˈjèna ba  ...}\\
        \gll    ˈtʃ͡ú=timi  riˈká=mi [\textbf{oˈmáwa} \textbf{onoˈká-li}] ba ˈtʃ͡ú=timi riˈká=mi kaˈjèna ba\\
            Q=2\textsc{pl.nom}  like=\textsc{dem} {make.party.\textsc{prs}} {do-\textsc{pst}} \textsc{cl} Q=2\textsc{pl.nom}  like=\textsc{dem} yield.harvest\textsc{.prs} \textsc{cl}\\
        \glt    `How did you make the parties? How good were the harvests?’\\
        \glt    `¿Cómo hacían ustedes las fiestas? ¿Cómo se les daban las cosechas?’ \corpuslink{in61[00_534-00_559].wav}{SFH in61:00:53.4}, \corpuslink{in61[00_560-00_573].wav}{SFH in61:00:56.0}\\
    }\label{ex: noka construction with different TAMb}
    \z
\z

Finally, the light verb \textit{noˈká} is also frequently found in constructions where the more descriptive predicate is an adjective. This is shown in (\ref{ex: noka plus cha}), with the adjectival predicate \textit{ˈtʃ͡á}, `ugly’.

%\pagebreak

\ea\label{ex: noka plus cha}

    \ea[]{
    \textit{ˈétʃ͡i   ˈmí   ʃiˈpá-am-ti       ˈwé ˈtʃ͡á   noko-ˈká siˈpá-ame  ru-ˈwá     riˈké   tʃ͡aˈbèi   m=aˈní-a  biˈlá   aˈní}\\
    \gll    ˈétʃ͡i   ˈmí  siˈpá-ame-ti       ˈwé [\textbf{ˈtʃá}   \textbf{noko-ˈká}] siˈpá-ame  ru-ˈwá     riˈké   tʃ͡aˈbèi   mi=aˈní-a  biˈlá   aˈní \\
            \textsc{dem} \textsc{dem} use.peyote-\textsc{ptcp-nmlz} \textsc{int} {ugly}  {do-\textsc{ger}} use.peyote-\textsc{ptcp} say-\textsc{mpass} like.that before \textsc{dem=}say-\textsc{prog} really  say.\textsc{prs}\\
    \glt    `That the peyote shamans would do bad things before, they would say.’\\
    \glt    `Que los raspadores hacían cosas malas raspaban antes, decían.’ < SFH 06 in61(580)/in >\\
}
        \ex[]{
        \textit{ˈwé ˈtʃ͡á nokoˈká ʃiˈpáami riˈhòli tʃ͡aˈbèe ko maˈnía biˈlá  aˈní }\\
        \gll    ˈwé [\textbf{ˈtʃá} \textbf{noko-ˈká}] siˈpá-ame    riˈhò-li     tʃ͡aˈbè=ko m=aˈní-a   biˈlá   aˈní \\
                \textsc{int} {ugly}  {do\textsc{-ger}}  use.peyote-\textsc{ptcp} man-\textsc{vblz}  before=\textsc{emph}  \textsc{dem}=say-\textsc{prog}  really  say.\textsc{prs}\\
        \glt    `They would do many bad things the peyote shamans before.’\\
        \glt    `Hacían muchas cosas malas antes para raspar.’    < SFH 06 in61(601)/in >\\
    }
    \z
\z

\subsubsection{The \textit{ˈní-} `do’ construction}
\label{subsubsec: the ni- 'do' construction}

Another light verb construction in Choguita Rarámuri involves the bound verbal root \textit{ˈní-}. As a main predicate, this verb is a copular predicate, as exemplified in (\ref{ex: copular verb examples light verb}) (see §\ref{subsec: types of copulas} for more details of copular predicates).

\ea\label{ex: copular verb examples light verb}

    \ea[]{
    \textit{ˈnè     uˈmûala ˈníli ˈétʃ͡i   ba?}  \\
    \gll    ˈnè    uˈmûa-la \textbf{ˈní-li} ˈétʃ͡i   ba  \\
            1\textsc{sg.nom}  great.grandfather-\textsc{poss}  {\textsc{cop-pst}}  \textsc{dem}  \textsc{cl}\\
    \glt    `Was he my great grandfather?’\\
    \glt    `¿Era mi bisabuelo él?’ < SFH 06 in61(117)/in >\\
}
        \ex[]{
        \textit{“tamuˈhê   biˈlá   ˈlîna ˈníma riˈké pa ˈétʃ͡i ˈlîna ˈtòli ba” ˈhê  biˈlá  aˈníli ba}\\
        \gll    tamuˈhê biˈlá ˈlîna \textbf{ˈní-ma} riˈké pa ˈétʃ͡i ˈlîna ˈtò-li ba” ˈhê biˈlá aˈní-li ba\\
                1\textsc{pl.acc} really but {\textsc{cop-fut}} perhaps \textsc{cl} \textsc{dem} but take-\textsc{pst} \textsc{cl} \textsc{dem} really   say-\textsc{pst} \textsc{cl}\\
        \glt    ```It was going to be us, but it took him”, that’s what she said.’\\
        \glt    ```Ibamos a ser nosotros y fue él” así dijo.'    \corpuslink{tx5[04_263-04_303].wav}{LEL tx5:04:26.3}\\
    }
    \z
\z

The following examples show the verb \textit{ˈní-} used as a light verb (\ref{ex: ni- as light verb}). Like in the constructions with \textit{noˈká}, these light verb constructions involve an order where the descriptive predicate precedes the auxiliary verb. In these constructions, the light verb may only appear with past tense marking, as exemplified below.\footnote{In contrast, the copular verb \textit{ˈni-} may bear other TAM markers (see §\ref{subsec: types of copulas}).} The descriptive (heavy) predicate in the construction is marked present or progressive.

\ea\label{ex: ni- as light verb}

    \ea[]{
    \textit{ˈkíti   ni   ke   oˈsáa bahuˈréma   ˈpâli,   ˈnè ˈlàa ˈníli}\\
    \gll    ˈkíti=ni   ke   o-ˈsá     bahuˈré-ma  ˈpâli   ˈnè [\textbf{ˈlà-a} \textbf{ˈní-li}]\\
            so.that=1\textsc{sg.nom}  \textsc{neg}  two-\textsc{mltp} invite-\textsc{fut.sg}  priest  1\textsc{sg.nom} {think-\textsc{prog}} {do\textsc{-pst}}\\
    \glt    `So that I won’t have to invite the priest twice, I was thinking.’\\
    \glt    `Para no invitar dos veces al padre, pienso.’    < SFH 06 in61(693)/in >\\
}\label{ex: ni- as light verba}
        \ex[]{
        \textit{ˈétʃ͡i ˈhápi ke ˈmí kaˈléa ˈníli}\\
        \gll    ˈétʃ͡i ˈhápi ke ˈmí [\textbf{kaˈlé-a} \textbf{ˈní-li}]\\
                \textsc{dem} \textsc{subl} \textsc{neg} \textsc{dem} {like-\textsc{prog}} {do\textsc{-pst}}\\
        \glt    `That, the one he doesn’t like.’\\
        \glt    `Ese, el que no le cae bien.'    \corpuslink{tx5[05_118-05_155].wav}{LEL tx5:05:11.8}\\
    }\label{ex: ni- as light verbb}
            \ex[]{
            \textit{ˈnà   kaˈènila   ˈrá     ˈmí   ˈtû   ˈnè     ˈpé   ˈám ˈnâta     ˈníli}\\
            \gll    ˈnà   kaˈèni-la   ru-ˈwá     ˈmí   ˈtû   ˈnè     ˈpé   ˈá=mi [\textbf{ˈnâta} \textbf{ˈní-li}]\\
                    then  finish-\textsc{rep} say-\textsc{mpass}   \textsc{dist} down   1\textsc{sg.nom} just   \textsc{aff=dem}  {remember.\textsc{prs}} {do\textsc{-pst}}\\
            \glt    `They said it was finished down there, I can barely remember.’\\
            \glt    `Dicen que se hizo allá abajo, yo apenas me acuerdo.’   \corpuslink{tx12[01_464-01_488].wav}{SFH tx12:01:46.4}\\
        }\label{ex: ni- as light verbc}
                \ex[]{
                \textit{ˈpé   ˈbí   ke   me riˈʃí    ˈníli ˈtʃ͡ó   aˈlé ...}    \\
                \gll    ˈpé   ˈbí   ke   me [\textbf{riˈsí}     \textbf{ˈní-li}] ˈtʃ͡ó   aˈlé     \\
                        just  just  \textsc{neg}  almost  {be.tired.\textsc{prs}} {do-\textsc{pst}}  also  \textsc{dub}    \\
                \glt    `But one would not get tired.’\\
                \glt    `Nomás que casi no se cansaba uno.’ < SFH 06 in61(235)/in >\\
            }\label{ex: ni- as light verbd}
            %\pagebreak
                    \ex[]{
                    \textit{aʔˈlì ˈnà ˈpéni maˈjêa ˈníli ˈwé aʔˈlá saʔˈmîami ˈníla ˈlé tʃ͡oˈnà ˈétʃ͡i kaˈwì}\\
                    \gll    aʔˈlì ˈnà ˈpé=ni [\textbf{maˈjê-a} \textbf{ˈní-li}] ˈwé aʔˈlá saʔˈmî-ame ˈní-la aˈlé tʃ͡oˈnà ˈétʃ͡i kaˈwì\\
                            and then just=\textsc{1sg.nom} {believe-\textsc{prog}} {do-\textsc{pst}} \textsc{int} well humid-\textsc{ptcp} \textsc{cop-rep} \textsc{dub} there \textsc{dem} land\\
                    \glt    `I believe that land was very humid.'\\
                    \glt    `Yo creo que esa tierra estaba muy bien mojado.’    \corpuslink{tx130[02_210-02_292].wav}{LEL tx130:02:21.0}\\
                }\label{ex: ni- as light verbe}
    \z
\z

The light verb construction with \textit{ˈní} `do' selects for stative or mental attitude heavy predicates: \textit{ˈla} `think’ (\ref{ex: ni- as light verba}), \textit{kaˈlé} `like’ (\ref{ex: ni- as light verbb}), \textit{ˈnâta} `remember' (\ref{ex: ni- as light verbc}), \textit{riˈsí} `be tired' (\ref{ex: ni- as light verbd}) and \textit{maˈjê} `believe' (\ref{ex: ni- as light verbe}). This stands in contrast with the constructions with \textit{noˈká} where the light verb selects for activity predicates.

\largerpage
\subsubsection{The \textit{iˈsì} ‘do’ construction}
\label{subsubsec: the iʃi 'do' construction}

A third light verb construction involves the verb \textit{iˈsì} ‘do’. As the main predicate in a clause, this verb has the meaning `do', as exemplified in (\ref{ex: ishi as main verb}).

\ea\label{ex: ishi as main verb}

    \ea[]{
    \textit{ˈhê ˈkwâ iˈsìli?}\\
    \gll    ˈhê ˈkwâ \textbf{iˈsì-li}\\
            who who {do-\textsc{pst}}\\
    \glt    `Who did it?'\\
    \glt    `¿Quién lo hizo?' \corpuslink{el1240[02_293-02_304].wav}{MAF el1240:02:29.3}\\
}
        \ex[]{
        \textit{ˈá ˈjén ˈá loˈmí tʃ͡í iˈsìa ba}\\
        \gll    ˈá ˈjén ˈá loˈmí tʃ͡í \textbf{iˈsì-a} ba\\
                \textsc{aff}  \textsc{aff}  \textsc{aff} know how {do-\textsc{prog}} \textsc{cl}\\
        \glt    `I do know how to do that.'\\
        \glt    `Sí se hacer eso.' \corpuslink{co1236[03_430-03_446].wav}{JLG co1236:03:43.0}\\
    }
    \z
\z

As a light verb accompanying a heavy verb, the verb \textit{iˈsì} bears tense/aspect inflection, while the descriptive predicate can only be marked present or progressive. As in the other light verb constructions described so far, the order is descriptive verb -- light verb. This is shown in (\ref{ex: ishi as light verb}).

\ea\label{ex: ishi as light verb}

    \ea[]{
    \textit{ˈpé       biˈlá      ˈpé    kiˈlìm ˈtʃ͡óna   iˈsìi ba}\\
    \gll    ˈpé  biˈlá  ˈpé  kiˈlì=mi  [\textbf{ˈtʃón-a}    \textbf{iˈsì-i}]    ba  \\
    little   really  little  slowly=\textsc{dem} {smash-\textsc{prog}} {do-\textsc{impf}} \textsc{cl}\\
    \glt    `I would smash it slowly.’  \\
    \glt    `Le aplastaba despacito.’    \corpuslink{tx1[01_042-01_063].wav}{BFL tx1:01:04.2}\\
}\label{ex: ishi as light verba}
        \ex[]{
        \textit{ˈpé   aʔˈlì   biˈlá  ˈpé   kuˈrím kaˈjèna   iˈsìa ruˈwá esˈk\textsuperscript{w}êlatʃ͡i   aʔˈlì   ba}\\
        \gll    ˈpé   aʔˈlì   biˈlá  ˈpé   kuˈrí=mi [\textbf{kaˈjèni-a}   \textbf{iˈsì-a}] ru-ˈwá esˈk\textsuperscript{w}êlatʃ͡i   aʔˈlì   ba\\
                just  later  really  just  recently=\textsc{dem} {finish-\textsc{prog}} {do-\textsc{prog}} say\textsc{-mpass} school then  \textsc{cl}\\
        \glt    `That time it was said that they were just recently finishing the school.’\\
        \glt    `Esa vez dicen que apenas estaba terminando la escuela.’  \corpuslink{tx12[02_358-02_406].wav}{SFH tx12:02:35.8}\\
    }\label{ex: ishi as light verbb}
        \ex[]{
        \textit{aʔˈlì biˈlá ko ˈá riˈká ˈnà boˈtêja   mokoˈʔôka biˈlá   ko   riˈpámi tʃ͡óm itʃ͡uˈkûba  iˈsìli aˈlé   ˈnà   benˈtâantʃ͡i   ba}   \\
        \gll    aʔˈlì   biˈlá=ko ˈá riˈká ˈnà boˈtêja mokoˈʔô-ka biˈlá=ko riˈpá-mi tʃ͡ó=mi [\textbf{itʃuˈkûba} \textbf{iˈsì-li}] aˈlé ˈnà benˈtân-tʃ͡i ba   \\
                and  really=\textsc{emph} \textsc{aff} like  \textsc{dem} bottle  in.head.crown-\textsc{ger}  really=\textsc{emph} up-\textsc{supe} also=\textsc{dem} {peek.\textsc{prs}} {do-\textsc{pst}} \textsc{dub} \textsc{dem} window-\textsc{loc} \textsc{cl}\\
        \glt    `Like that, he was peeking through the window with the bottle in his head (like a crown).’\\
        \glt    `Así como ya se asomaba por la ventana con la botella puesta.’  \corpuslink{tx152[02_144-02_200].wav}{SFH tx152:02:14.4}\\
    }\label{ex: ishi as light verbc}
    \z
\z

In these examples, the light verb is marked imperfective (\ref{ex: ishi as light verba}), progressive (\ref{ex: ishi as light verbb}) and past (\ref{ex: ishi as light verbc}). In all instances, the heavy predicate is marked present tense or present progressive.

In the following example (in (\ref{ex: ugly plus light verb})), \textit{iˈsì} is nominalized through an agentive participial marker and preceded by the adjectival predicate \textit{ˈtʃ͡á} ‘ugly’, which contributes the main descriptive content of the construction. The light verb construction is in turn the complement of a higher predicate (the motion verb \textit{aˈjéna} `go').

\ea\label{ex: ugly plus light verb}

\textit{ˈtʃ͡á    iˈsìkam ˈjénam koriˈmá ba}\\
\gll    [\textbf{ˈtʃá}  \textbf{iˈsì-kame}]  aˈjéna=mi  koriˈmá    ba\\
        {ugly}   {do-\textsc{pst.ptcp}} go.\textsc{sg=dem} korima   \textsc{cl}\\
\glt    ```The \textit{korima} (fire bird) was bothering (lit. doing bad/ugly)”.'\\
\glt    ```Anduvo molestando (lit. haciendo feo/mal) el \textit{korimá} (pájaro de fuego)”.' \corpuslink{tx5[03_430-03_448].wav}{LEL tx5:03:43.0}\\

\z

Whether the descriptive predicate is an adjective or a verb, the light verb in this construction appears to indicate that the event encoded by the construction is an activity.

\subsection{Auxiliary verb constructions}
\label{subsec: auxiliary verb constructions}

Auxiliary verbs are defined in this grammar as functional verbal elements that are combined with a semantically descriptive verbal predicate in a mono-clausal structure; the descriptive verbal predicate contributes lexical content to the construction, while auxiliary verbs may encode aspectual, temporal, modal or evidential information \citep{anderson2006auxiliary,svenonius2008complex}. While auxiliary verb constructions and light verb constructions (and other complex predicate types) are proposed to be discrete classes of constructions \citep{butt2001semi} (cf. \citealt{anderson2006auxiliary}), the boundary between them may not be sharp \citep{svenonius2008complex}. Two main criteria are assumed here in distinguishing auxiliary verb constructions from light verb constructions in Choguita Rarámuri: (i) while light verbs may impose selectional restrictions on their complements, auxiliary verbs  do not impose such selectional restrictions nor do they contribute lexical-semantic (argument structural) content (\citealt{butt1995structure}); and (ii) while auxiliary verbs and light verbs both involve semantic bleaching, auxiliary verbs have less semantic content than light verbs or may lack descriptive content altogether (\citealt{butt1995structure}; see also \citealt{tubino2014affixal}).

Choguita Rarámuri features multi-verb constructions that may be analyzed as auxiliary verb constructions. These involve the paradigm of positional/posture verbs, which exhibit suppletion and affixation encoding number (singular vs. plural) and transitivity (stative, inchoative and causative) contrasts when functioning as main predicates in locative clauses. The morphological and syntactic properties of posture predicates are described in §\ref{subsec: locative clauses}. The posture predicate paradigm is presented in \tabref{tab:15:positional-predicates}.

%\break

\begin{table}
\caption{Posture/positional predicates in Choguita Rarámuri}
\label{tab:15:positional-predicates}

\begin{tabularx}{\textwidth}{Qllllll}
\lsptoprule
& \multicolumn{2}{X}{

 \textbf{Stative}} & \multicolumn{2}{X}{

 \textbf{Inchoative}} & \multicolumn{2}{X}{

 \textbf{Causative}}\\
 \cmidrule(lr){2-3} \cmidrule(lr){4-5} \cmidrule(lr){6-7}
& \textbf{Sg} & \textbf{Pl} & \textbf{Sg} & \textbf{Pl} & \textbf{Sg} & \textbf{Pl}\\
\midrule
‘sit’ & \textit{aˈtí} & \textit{muˈtʃúwi} & \textit{asi/aˈsá} & \textit{moˈtʃíwi} & \textit{aˈtʃ͡â} & \makecell[tl]{\textit{muˈtʃ͡ûwi/}\\\textit{mutʃ͡uˈwâ}}\\
\tablevspace
‘sit (container)’ & \makecell[tl]{\textit{maˈní}\\\textit{{\textasciitilde}baˈní}} & \textit{a-ˈmáni} & \makecell[tl]{\textit{bani-ˈbá}\\\textit{{\textasciitilde}mani-ˈbá}} & \makecell[tl]{\textit{bani-ˈbá/}\\\textit{baˈní-ba}\\\textit{{\textasciitilde}mani-ˈbá}} & \makecell[tl]{\textit{baˈná}\\\textit{{\textasciitilde}maˈná}} & \textit{a-maˈná}\\
\tablevspace
‘stand’ & \textit{wiˈlí} & \textit{ˈhâwi} & \textit{wiˈlísi} & \textit{ˈhási} & \textit{wiˈlá} & \textit{haˈwá}\\
\tablevspace
‘lie down’ & \textit{buˈʔí} & \textit{biˈtí} & \textit{buˈʔu-} & \textit{biˈtí} & \textit{riki/riˈká} & \textit{roˈʔá}\\
\tablevspace
‘bent, curved, on four legs’ & \textit{tʃ͡uˈkú} & \textit{uˈtʃúwi} & \makecell[tl]{\textit{tʃ͡uˈkú-ba/}\\\textit{tʃ͡uku-ˈbá}} & \makecell[tl]{\textit{i-ˈtʃúpi/}\\\textit{i-tʃ͡uˈpá}\\\textit{{\textasciitilde}u-ˈtʃúpi}} & \textit{uˈtʃá} & \makecell[tl]{\textit{i-ˈtʃútʃ͡i}\\\textit{{\textasciitilde}u-ˈtʃútʃ͡i}}\\
\lspbottomrule
\end{tabularx}
\end{table}

A subset of these predicates, namely stative and inchoative posture predicates, are deployed in auxiliary verb constructions encoding progressive aspect, indicating that an event takes place continuously over a given time frame. These constructions are analytic, with auxiliary verbs being separate prosodic words (the criteria for defining the prosodic word in Choguita Rarámuri are discussed in §\ref{sec: defining the prosodic word and other prosodic domains in CR}). The auxiliary verb bears tense marking (present, past or future), while the main lexical verb is inflected for present tense regardless of the tense marking on the auxiliary. Auxiliary verb constructions are exemplified in (\ref{ex: posture auxiliary verb construction examples}).

\ea\label{ex: posture auxiliary verb construction examples}

    \ea[]{
    \textit{aʔˈlì tʃ͡i ko ˈétʃ͡i ˈnà ˈhônsa ko aʔˈlì biˈléna ˈtʃ͡ó bitiˈtʃ͡í siˈmíli ˈápu ko ˈnà baˈhîa moˈtʃ͡íli}\\
    \gll    aʔˈlì tʃ͡i=ko ˈétʃ͡i ˈnà ˈhônsa=ko aʔˈlì biˈlé-na ˈtʃ͡ó bitiˈtʃ͡í siˈmí-li ˈnápi=ko ˈnà [\textbf{baˈhî-a} \textbf{moˈtʃí-li}]\\
            and tʃ͡i=\textsc{emph} \textsc{dem} \textsc{prox} since=\textsc{emph} and one-\textsc{incl} also house go\textsc{sg-pst} \textsc{sub=emph} \textsc{dem}  {drink-\textsc{prog}} {sit.down.\textsc{pl-pst}}\\
    \glt    `And then from there he went to another house where people were drinking.’  \\
    \glt    `Y entonces de ahí se fue en otra casa donde estaban tomando.’ \corpuslink{tx5[01_434-01_485].wav}{LEL tx5:01:43.4}\\
}\label{ex: posture auxiliary verb construction examplesa}
        \ex[]{
        \textit{ˈmá nataˈkêa buˈʔíli ˈnà biˈʔà roˈkò}\\
        \gll    ˈmá [\textbf{nataˈkê-a} \textbf{buˈʔí-li}] ˈnà biˈʔà roˈkò\\
                already {faint-\textsc{prog}} {lie.down.\textsc{sg-pst}} then early night\\
        \glt    `He had already fainted before dawn.'\\
        \glt    `Ya estaba desmayado en la madrugada.'    \corpuslink{tx5[04_037-04_070].wav}{LEL tx5:04:03.7}\\
    }\label{ex: posture auxiliary verb construction examplesb}
            \ex[]{
            \textit{aʔˈlì ˈnè ko ˈmá bitiˈtʃ͡í ˈá buˈjèa aˈtí}\\
            \gll    aʔˈlì ˈnè=ko ˈmá bitiˈtʃ͡í ˈá [\textbf{buˈjè-a} \textbf{aˈtí}]\\
                    and  1\textsc{sg.nom=emph} then house \textsc{aff} {wait-\textsc{prog}} {sit.\textsc{sg.prs}}\\
            \glt    `And then I am waiting for her in the house.’\\
            \glt    `Y entonces yo ya la estoy esperando en la casa.’  \corpuslink{tx19[01_179-01_236].wav}{LEL tx19:01:17.9}\\
        }\label{ex: posture auxiliary verb construction examplesc}
                    \ex[]{
                    \textit{“tʃ͡ín oˈlá ko ˈétʃ͡i”, ˈhê biˈlá ko ˈlàa aˈsáli ˈlé ruˈtûkuri ko ba}\\
                    \gll    tʃ͡í=ni oˈlá=ko ˈétʃ͡i ˈhê biˈlá=ko [\textbf{ˈlà-a} \textbf{aˈsá-li}] aˈlé ruˈtûkuri=ko ba\\
                            how=\textsc{1sg.nom} do.\textsc{prs=emph} \textsc{dem} that indeed=\textsc{emph} {think-\textsc{prog}} {sit.\textsc{sg-pst}} \textsc{dub} owl=\textsc{emph} \textsc{cl}\\
                    \glt    ```That's how I did it to them" that's what he was thinking, the owl".'\\
                    \glt    ```Así les hice a esos” eso estaba pensando, el tecolote".' \corpuslink{tx152[07_019-07_050].wav}{SFH tx152:07:01.9}\\
                }\label{ex: posture auxiliary verb construction examplesd}
    \z
\z

In contrast to light verbs, auxiliary verbs do not impose any selectional restrictions on the verb encoding the main lexical content of the clause, e.g., the descriptive verb of the construction may be an activity predicate (\textit{baˈhî} `drink' in (\ref{ex: posture auxiliary verb construction examplesa})), a telic (bounded) predicate (\textit{nataˈkê} `faint' in (\ref{ex: posture auxiliary verb construction examplesb})) or a stative predicate (\textit{ˈlà} `think' in (\ref{ex: posture auxiliary verb construction examplesd})). The progressive auxiliary verb construction encodes that an event is ongoing at some specified time frame internal to ongoing discourse: in (\ref{ex: posture auxiliary verb construction examplesa}), an activity (`drinking') is carried out over a period of time (in this context, a drinking party). This construction may also denote that a state of affairs is transitory, rather than permanent: when combining with a stative predicate, this construction denotes a type of temporary activity, as in (\ref{ex: posture auxiliary verb construction examplesd}), where a participant is in a state of thinking about an action carried out.

The verbs bearing inflection in these constructions are not completely devoid of the semantic contrasts they encode as main predicates in locative clauses. Specifically, number contrasts and some lexical distinctions are retained. In terms of their lexical meaning, auxiliaries may be selected in terms of the posture of referents, e.g., a man that has fainted lies down (\ref{ex: posture auxiliary verb construction examplesb}), and the default collocation for human beings are the auxiliaries derived from the posture predicate `sit' ((\ref{ex: posture auxiliary verb construction examplesa}) and (\ref{ex: posture auxiliary verb construction examplesc})) (for more discussion about default collocations and categorization involving posture predicates, see §\ref{subsec: locative clauses}).

Number distinctions are also retained in auxiliary verbs. This is shown in (\ref{ex: posture auxiliary verb construction examplesa}): the auxiliary employed \textit{moˈtʃ͡í} encodes plural number as a positional verb and is used as an auxiliary in a context where the event described is a drinking party involving multiple participants; in contrast, in (\ref{ex: posture auxiliary verb construction examplesb}) the auxiliary employed (\textit{buˈʔí}) encodes singular number and is used in a context where the event involves a single participant (a man who is in the state of having fainted). Likewise, in (\ref{ex: posture auxiliary verb construction examples}c--d), the auxiliary verbs (\textit{aˈtí} and \textit{aˈsá}, respectively) encode singular number with singular subjects. Further examples of number contrasts in auxiliaries is exemplified in (\ref{ex: sg pl contrast in auxiliary verbs}), where the auxiliaries in each case derive from the singular (\textit{wiˈli}) and plural (\textit{ˈhâwi}) forms of the positional predicate `stand'.

\ea\label{ex: sg pl contrast in auxiliary verbs}

    \ea[]{
	\textit{ˈnè ˈlèa wiˈlí tʃ͡oˈnà rapiˈtʃ͡i ˈnápu ko ˈnà meˈʔàli ba}\\
	\gll    ˈnè \textbf{ˈl-è-a} \textbf{wiˈlí} tʃ͡oˈnà rapi-ˈtʃ͡í ˈnápi=ko ˈnà meˈʔàli ba\\
	        \textsc{int} {blood-\textsc{vblz-prog}} {stand.\textsc{prs}} there sandstone-\textsc{loc} \textsc{sub=emph} \textsc{dem} go.around-\textsc{pst} \textsc{cl}\\
    \glt    `It was bloody over there by the sandstone where he had been.'\\
    \glt    `Estaba sangrado allí en las lajas donde anduvo.'    \corpuslink{tx84[05_584-06_031].wav}{LEL tx84:05:58.4}\\
}\label{ex: sg pl contrast in auxiliary verbsa}
        \ex[]{
        \textit{aʔˈlì ˈnà ˈwé ja tʃ͡aˈkêna raˈwéa ˈhâwipo ruˈwá}\\
        \gll    aʔˈlì ˈnà ˈwé ja tʃ͡aˈkêna [\textbf{raˈwé-a} \textbf{ˈhâwi-po}] ru-ˈwá\\
                and then \textsc{int} fast another.side  {turn-\textsc{prog}} {stand\textsc{.pl-fut.pl}} say-\textsc{mpass}  \\
        \glt    `And then one has to stand turning to another side, it is said.’\\
        \glt    `Y entonces dicen que pronto hay que voltearse de otro lado.’     \corpuslink{tx5[04_538-04_572].wav}{LEL tx5:04:53.8}\\
    }\label{ex: sg pl contrast in auxiliary verbsb}
    \z
\z

In (\ref{ex: sg pl contrast in auxiliary verbsa}), the auxiliary verb \textit{wiˈli} encodes singular number in a construction denoting a transitory state of a location (the sandstone that is bloody), while in (\ref{ex: sg pl contrast in auxiliary verbsb}) the auxiliary verb \textit{ˈhâwi} encodes plural number in a context where an ongoing process (turning to face another side) is something that must be done by hearers of the advice provided in the narrative.

While number and other lexical contrasts are retained in auxiliation, not all forms in the posture/positional predicate paradigm are attested in auxiliary verb constructions. Specifically, there are no instances in the corpus where a causative positional predicate appears in an auxiliary verb construction. Furthermore, it appears the choice between a stative positional verb (\textit{buˈʔí} `lie down, \textsc{sg}', \textit{aˈtí} `sit, \textsc{sg}' or \textit{ˈhâwi} `stand, \textsc{pl}') and an inchoative positional verb (\textit{ moˈtʃ͡íwi} `sit, \textsc{pl}' or \textit{aˈsá} `sit, \textsc{sg}') is arbitrary. That semantic bleaching in this construction is only partial is not surprising considering that posture/positional predicates are common lexical sources for auxiliary verb constructions encoding progressive aspect and that the multi-predicate constructions do not significantly differ semantically from their lexical sources in locative clauses \citep{bybee1989creation}. But while the number contrasts are retained in auxiliary verb constructions, transitivity contrasts encoded by the posture predicates as main predicates are not retained in their auxiliary function.
%describe

%However, %these constructions can also be used when the agent’s posture is not an issue, or when the posture denoted by the auxiliary does not correspond to the posture of the main verb, for example as illustrated in (9).

\subsection{Serial verb constructions}
\label{subsec: serial verb constructions}

% García Salido: \ili{Southeastern Tepehuan} has some limited complex verb phrases formed by the combination of two verbs with no overt subordination or coordination markers. In addition, these constructions exhibit the following properties of SVCs: i. they share the same subject and markers of tense and aspect; ii. they contain two verbs without overt markers of coordination or subordination; iii. the verbs form a single predicate; iv. serialized verbs also occur alone in non-serial constructions; and v. the negation particle cham has scope over the two verbs. All this characteristics suggest that these expressions exhibit properties of a single event.

Serialization forms complex predicates with specialized functions. Cross-linguis\-tically, serial verb constructions can be characterized by the following properties (\citealt{foley1985clausehood}, \citealt{sebba1987syntax}, \citealt{aikhenvald2006serial}):

\ea\label{ex: serial verb constructions defined}
{Main properties of Serial Verb Constructions}

    \ea[]{
    Each serial verb construction contains at least two verbs without any overt marker of subordination or coordination.\\
}
        \ex[]{
        Either verb of the construction could function as the predicate of its own clause.\\
    }
            \ex[]{
            The construction depicts what could be conceived as a single event.\\
        }
                \ex[]{
                The argument structure of the construction corresponds to that of a single clause, with a single internal and external argument.\\
            }
                    \ex[]{
                    The construction has a single set of tense, aspect, mood and polarity values.\\
                }
                        \ex[]{
                        Intonationally, serial verb constructions may behave like a single clause.\\
                    }
    \z
\z

A relatively infrequent construction in Choguita Rarámuri involving motion verbs meets the criteria for serial verb constructions: (i) the verbal complex lacks any overt marker of subordination or coordination; (ii) each verb of the construction may also function in non-serial constructions as a main predicate; (iii) the two verbs form depict a single event; (iv) the overall argument structure corresponds to that of a single clause; (v) the construction features a single value for tense, mood, aspect and polarity; and (vi) the construction may have a prosodic contour that corresponds to a single Intonational Phrase (IP). These properties are illustrated in (\ref{ex: serialized directionals}).

\ea\label{ex: serialized directionals}

    \ea[]{
    \textit{ˈétʃ͡i   ko   ˈwé naˈlàʃia   naˈwàli} \\
    \gll    ˈétʃ͡i=ko   ˈwé [\textbf{naˈlà-si-a}   \textbf{naˈwà-li}] \\
            \textsc{dem=emph} \textsc{int} {cry-\textsc{mot-prog}} {arrive-\textsc{pst}}\\
    \glt    `She arrived crying.’\\
    \glt    `Llegó llorando.’ \corpuslink{tx5[03_096-03_114].wav}{LEL tx5:03:09.6}\\
}\label{ex: serialized directionalsa}
        \ex[]{
        \textit{ˈhê  aˈnè aˈníʃia naˈwàli ˈétʃ͡i naˈmú niˈrà ʃuˈwá ba ˈá ruˈwè-li} \\
        \gll    ˈhê  aˈn-è [\textbf{aˈní-si-a} \textbf{naˈwà-li}] ˈétʃ͡i naˈmú niˈrà suˈwá ba ˈá ruˈw-è-li\\
                \textsc{dem} say-\textsc{appl} {say-\textsc{mot-prog}} {arrive-\textsc{pst}} \textsc{dem} something relatives everybody \textsc{cl} \textsc{aff} say-\textsc{pst}\\
        \glt    `A relative arrived saying, telling everybody.’\\
        \glt    `Llegó diciendo un familiar, diciéndoles a todos.’    \corpuslink{tx5[04_070-04_105].wav}{LEL tx5:04:07.0}\\
    }\label{ex: serialized directionalsb}
            \ex[]{
            \textit{aʔˈlì  ˈétʃ͡i ˈápu roˈwéa ˈúmi ko ˈá biˈlá ritiˈwá ˈtʃ͡ú tʃ͡uˈrú aˈtí}\\
            \gll    aʔˈlì  ˈétʃ͡i ˈnápi [\textbf{roˈwé-a} \textbf{ˈhúmi}]=ko ˈá biˈlá riti-ˈwá ˈtʃ͡ú tʃ͡uˈrú aˈtí\\
                    and   \textsc{dem} \textsc{sub} {women.race-\textsc{prog}} {{run\textsc{.pl.prs}=emph}} \textsc{aff} indeed see.\textsc{-mpass} how much sit\textsc{.sg.prs}\\
            \glt    `And then the ones running the ariweta race see how many things there are (lit. sit) (things people bet).’ \\
            \glt    `Y entonces las que andan corriendo carrera de ariweta ven que tanto va.'    \corpuslink{tx19[03_104-03_149].wav}{LEL tx19:03:10.4}\\
        }\label{ex: serialized directionalsc}
                \ex[]{
                \textit{apaˈlì ˈmá biˈlé ˈtòoru ne paˈtʃ͡ûsia iˈnâliro riˈpáti raˈbô}\\
                \gll    ˈnápi aʔˈlì ˈmá biˈlé ˈtò-ru ne [\textbf{paˈtʃû-si-a} \textbf{iˈnâli-li-ro}] riˈpá-ti raˈbô\\
                        when then already one   take-\textsc{pst.pass} \textsc{int} {drip-\textsc{mot-prog}} {go-\textsc{pst-mov}} up-\textsc{all}  hill  \\
                \glt    `When one was already taken it goes dripping something by the top of the hill.’\\
                \glt    `Cuando ya lleva uno va goteando algo por arriba del cerro.’   \corpuslink{tx5[06_171-06_216].wav}{LEL tx5:06:17.1}\\
            }\label{ex: serialized directionalsd}
    \z
\z

\largerpage
The meaning of this construction is that of motion with associated manner. In Choguita Rarámuri serial verb constructions, the open class verb is marked with progressive aspect, while the motion verb, the defining member of the construction, may be inflected with a variety of tense values (e.g., past tense (\ref{ex: serialized directionals}a--b, d) or present tense (\ref{ex: serialized directionalsc})).\footnote{In example (\ref{ex: serialized directionalsd}) the final stem syllable and past tense suffix syllable have identical onsets, and undergo stem-suffix haplology (see §\ref{subsubsec: stem-suffix haplology}).} Thus, the open class predicate bears argument marking for the entire clause, while the motion predicate bears tense marking for the entire clause. The open class verb may additionally bear associated motion marking, as in (\ref{ex: serialized directionals}a--b,d). The motion predicate may also be marked with an unproductive motion suffix, \textit{-ro}, as in (\ref{ex: serialized directionalsd}).\footnote{The cognate suffix in \ili{Mountain Guarijío} (\textit{-tó/ro}) is described with a distributive sense, e.g., `to go do X in more than one place' or `do X more than once' \citep[165]{miller1996guarijio}. It is not possible to determine if the Choguita Rarámuri \textit{-ro} suffix has the same meaning given how infrequently it is attested in the corpus.} Each verb is a separate prosodic word (in contrast to V-V incorporation constructions, where the two verbs are combined in a stem, equivalent to a single prosodic word, as described in §\ref{subsec: V-V incorporation constructions} below).

A second construction In Choguita Rarámuri has properties of a serial verb construction and involves predicates of speaking. This construction is exemplified in (\ref{ex: serial verb with verb of speech}).

\ea\label{ex: serial verb with verb of speech}

    \ea[]{
    \textit{aʔˈlì  biˈlá tʃ͡oˈná tʃ͡uˈkúli ˈtʃ͡ó ˈlé bo baˈjèa aˈnía esˈta toˈwí tʃ͡oˈná paˈtʃ͡á atiˈkó maˈjêli ... ˈétʃ͡i riˈmò ba}\\
    \gll aʔˈlì  biˈlá tʃ͡oˈná tʃ͡uˈkú-li ˈtʃ͡ó aˈlé bo [\textbf{baˈjè-a} \textbf{aˈní-a}] ˈétʃ͡i  ˈtá toˈwí tʃ͡oˈná paˈtʃ͡á ati=ˈkó maˈjê-li ˈétʃ͡i riˈmò ba\\
        and indeed there be.bent-\textsc{pst} also \textsc{dub} \textsc{emph} {call-\textsc{prog}} {say-\textsc{prog}} \textsc{dem} \textsc{def} boy there inside sit.\textsc{sg=emph} believe-\textsc{pst} \textsc{dem} frog \textsc{cl}\\
    \glt    `And then the boy was there calling it (lit. calling saying), he thought that the frog was in there'\\
    \glt    `Y entonces allí estuvo llamándole (lit. llamando diciendo) el niño, pensó que allí adentro estaba el sapo.’ \corpuslink{tx152[05_252-05_312].wav}{SFH tx152:05:25.2}\\
}\label{ex: serial verb with verb of speecha}

        \ex[]{
        \textit{aʔˈlì biˈlá ko ti ˈétʃ͡i ruˈká biˈlámi kom ˈnâra atʃ͡aˈní ba}\\
        \gll    aʔˈlì biˈlá=ko=ti ˈétʃ͡i ru-ˈká biˈlá=mi=ko=ni [\textbf{ˈnâri-a} \textbf{atʃaˈní}] ba\\
                and indeed=\textsc{emph=1pl.nom} \textsc{dem} say-\textsc{ger} indeed=\textsc{dem=emph=1sg.nom} {ask-\textsc{prog}} {make.noise.\textsc{prs}} \textsc{cl}\\
        \glt    `Then saying I'd ask (lit. ask make noise) her.'\\
        \glt    `Entonces diciendo yo le preguntaba (lit. pregunto hago ruido).'  \corpuslink{tx43[01_453-01_497].wav}{SFH tx43:01:45.3}\\
    }\label{ex: serial verb with verb of speechb}
    \z
\z

In (\ref{ex: serial verb with verb of speecha}), the second verb of the complex verb construction is the verb that defines the construction (\textit{aˈní} `say'), while the first verb describes the manner of speaking (\textit{baˈjè} `call'). Both verbs are marked with progressive aspect and describe a single event with a single internal and external argument. In (\ref{ex: serial verb with verb of speechb}), the construction encodes that the act of speech (encoded with the verb \textit{atʃ͡aˈní} `make noise') involves asking.

Serial verb constructions are not generally documented in \ili{Uto-Aztecan} languages, but some complex verb constructions are analyzed as involving serialization in \ili{Southeastern Tepehuan} (\ili{Tepiman}) \citep{garcia2007serial} and \ili{Northern Paiute} (\ili{Numic}) \citep{thornes2011dimensions}, where an open class verb combines with a verb of a restricted class (primarily motion predicates in \ili{Southeastern Tepehuan}, and motion and posture predicates in \ili{Northern Paiute}). The constructions analyzed as serial verb constructions in \ili{Northern Paiute} involve a single phonological word, which can be characterized as verb-verb compounding or incorporation. A different set of constructions in Choguita Rarámuri have this formal property and are addressed next.

\subsection{V-V incorporation (secondary verb constructions)}
\label{subsec: V-V incorporation constructions}

Choguita Rarámuri features V-V incorporation construction. As discussed in \sectref{sec: verbal structure and verbal domains}, the Choguita Rarámuri verbal morphological structure includes a domain, namely the ``aspectual stem", where suffixes encoding desiderative, associated motion and auditory evidential meanings are transparently related to independent verbs in the language. \tabref{tab:15:aspectual-suffixes} lists these suffixes, their grammaticalized meanings and their independent lexical verb sources.

\begin{table}
\caption{Choguita Rarámuri aspectual suffixes and their lexical counterparts}
\label{tab:15:aspectual-suffixes}

\begin{tabularx}{\textwidth}{Xl}
\lsptoprule

\textbf{Aspectual suffixes} & \textbf{Independent lexical verb}\\
\midrule
\textit{-ˈnále}	`desiderative (\textsc{desid})' & \textit{ˈnále} `want'\\
\textit{-simi} `associated motion (\textsc{mot})' & \textit{siˈmí} `go.\textsc{sg}'\\
\textit{-tʃ͡ane} `auditory evidential (\textsc{ev})' & \textit{(a)ˈtʃ͡áne} `say, make noise'\\
\lspbottomrule
\end{tabularx}
\end{table}

Verbal roots attaching these suffixes have properties that are characteristic of complex predicates, with the derivational suffixes in \tabref{tab:15:aspectual-suffixes} attaching to a verbal root of an open class and forming a verb cluster. In the resulting complex, the suffix may introduce its own arguments to the construction and take the embedded phrase as its semantic complement. The verb complex has a single value for tense, aspect, mood and polarity.

These structures resemble a multi-verb construction documented across \ili{Uto-Aztecan} languages and known in the Uto-Aztecanist literature as the ``secondary verb" construction \citep{crapo1970origins,thornes2011dimensions}. These constructions involve a verb which may be used as a main verb in a clause or may be appear phonologically bound to another verb. In their suffixed form, secondary verbs encode aspectual or adverbial meanings. In Choguita Rarámuri, V-V incorporation is productive and involves two verbal predicates in sequence, the first of which is an open class verb followed by the defining member of the construction. V-V incorporation is exemplified in (\ref{ex: multi-verb construction example}). Similar multi-verb constructions in other \ili{Uto-Aztecan} languages have been alternatively analyzed as involving serialization (e.g., in \ili{Northern Paiute} (\ili{Numic}) \citep{thornes2011dimensions}) or light verb constructions (e.g., in \ili{Hiaki} (\ili{Taracahitan}) \citep{tubino2014affixal}).

\ea\label{ex: multi-verb construction example}

    \ea[]{
    \textit{ˈwé aˈnátʃ͡o, ka biˈlé iˈʃînili ba}\\
    \gll    ˈwé aˈnátʃ͡a-o ka biˈlé iˈsî-\textbf{nale} ba\\
            \textsc{int} endure-\textsc{ep} \textsc{neg} one urinate-{\textsc{desid}} \textsc{cl}\\
    \glt    `She endures a lot, she doesn't want to pee.'\\
    \glt    `Aguantan mucho, no quiere orinar.'   \corpuslink{co1140[11_304-11_324].wav}{MDH co1140:11:30.4}\\
}
        \ex[]{
        \textit{ˈmá busuˈrêsimi}\\
        \gll    ˈmá busuˈrê-\textbf{simi}\\
                already wake.up-\textsc{{mot}}\\
        \glt    `She's going along waking up.'\\
        \glt    `Va despertando.'   \corpuslink{el1007[01_547-01_568].wav}{SFH el1007:01:54.7}\\
    }
            \ex[]{
            \textit{aʔˈlì biˈlá meˈtêsima ˈrá ba}\\
            \gll    aʔˈlì biˈlá meˈtê-\textbf{simi}-ma ru-ˈwá ba\\
                    and indeed cut.with.ax-\textsc{{mot}-fut.sg} say-\textsc{mpass} \textsc{cl}\\
            \glt    `And then they say they will go along cutting with the ax.'\\
            \glt    ‘Y luego dicen que van a ir cortando con el hacha.’  \corpuslink{in61[05_589-06_010].wav}{FLP in61:05:58.9}\\
        }
                \ex[]{
                \textit{``ˈmá ˈwé aˈwítʃ͡ani" aˈnítʃ͡ani aˈbé ˈlôla riʔˈlína ˈrú}\\
                \gll    ˈmá ˈwé aˈwí-\textbf{tʃane} aˈní-\textbf{tʃane} aˈbe ˈlôla riʔˈlí-na ˈrú\\
                        already \textsc{int} dance-\textsc{{ev}} say-\textsc{{ev}} earlier Lola down-\textsc{all} say.\textsc{prs}\\
                \glt    ``It sounds like they're already dancing a lot" (it sounded like) Lola said earlier down there.'\\
                \glt    ``Ya se oyen bailar mucho" así (se oyó que) dijo Lola hace rato allá abajo.'  \corpuslink{co1237[08_151-08_183].wav}{JLG co1237:08:15.1}\\
        }
    \z
\z

In addition to the desiderative, associated motion and auditory evidential, indirect causative constructions also involve V-V incorporation: as detailed in §\ref{subxsec: indirect causatives} and exemplified in (\ref{ex: V-V incorporation causative}), the complement of the jussive predicate \textit{nuˈlé} `order, command' is a clause where the lower verb is additional marked with the jussive verbal affix \textit{nula} `order, command’ deriving a co-lexicalized structure within the complement. This is another instance of V-V incorporation.

\ea\label{ex: V-V incorporation causative}

    \ea[]{
    \textit{wiˈtʃ͡ônula nuluˈrîa ˈpé ˈkútʃ͡i ˈká ˈhônsa}\\
    \gll    wiˈtʃ͡ô-\textbf{nula} nulu-ˈrîa ˈpé ˈkútʃ͡i ˈká ˈhônsa\\
            wash.clothes-\textsc{{order}} order-\textsc{mpass} just small \textsc{cop.irr} since\\
    \glt    `They make them wash clothes since they are little.'\\
    \glt    `Las mandan a lavar la ropa desde que son chiquitas.'  \corpuslink{tx48[02_015-02_041].wav}{BFL tx48:02:01.5}\\
}
        \ex[]{
        \textit{ˈá haˈré ko ku biˈlá muˈrúnula nuluˈrîwi ˈru ba, ˈkútʃ͡i ˈkûruwi ko}\\
        \gll    ˈá haˈré=ko ku biˈlá muˈrú-{nula} nulu-ˈrîwi ˈru ba ˈkútʃ͡i ˈkûruwi=ko\\
                \textsc{aff} some=\textsc{emph} wood indeed gather.with.hands-\textsc{order} order-\textsc{mpass} say.\textsc{prs} \textsc{cl} small children=\textsc{emph}\\
        \glt    `Some send them to bring wood, the children.'\\
        \glt    `Unos los mandan a traer leña, a los niños.’   \corpuslink{tx73[00_312-00_365].wav}{LEL tx73:00:31.2}\\
    }
    \z
\z
