\chapter{Sentence types}
\label{chap: sentence types}
\largerpage
This chapter is concerned with non-basic, non-complex clause types that encode different illocutionary acts and other pragmatically-marked structures, including interrogatives, negatives and imperatives. These constructions are achieved by a number of syntactic and morphosyntactic strategies. This chapter also addresses the properties of comparative constructions. This description is organized in terms of morphosyntactic and syntactic properties of each construction, as well as the prosodic properties of a subset of these constructions. Each construction is described in terms of a comparison to the morphosyntactic (and, where relevant, prosodic) characteristics of declarative sentences.

This chapter is organized as follows. The basic properties of declarative sentences are summarized in §\ref{sec: declarative sentences}, to serve as a baseline for the description of non-basic sentence types. Interrogatives are described in §\ref{sec: interrogative constructions}, including the distinction between polar questions (§\ref{subsec: polar questions}) and content questions (§\ref{subsec: content questions}). Negative clauses are addressed in §\ref{sec: negative constructions}. Imperatives are addressed in §\ref{sec: imperative constructions}. Finally, comparative constructions are reviewed in §\ref{sec:14:comparative constructions}.

%Missing here: description of intonational properties or other encoding of intonational structure here

\section{Declarative sentences}
\label{sec: declarative sentences}

Choguita Rarámuri is a highly agglutinative, head-final language, where word order is canonically SOV, a pattern documented across \ili{Uto-Aztecan} languages (\citealt{langacker1977uto}). The basic word order is exemplified below, with a clause headed by a ditransitive predicate with both pronominal and NP arguments (\ref{ex:14:canonical SOV order examplesa}) and by a transitive predicate with pronominal arguments (\ref{ex:14:canonical SOV order examplesb}).

\ea\label{ex:14:canonical SOV order examples}
{SOV word order in Choguita Rarámuri}

    \ea[]{
    {\textit{ˈmò ˈjêla	taˈmí haˈré	gaˈjêta ˈʔàko}}\\
    \gll    [ˈmò 		ˈjê-la]\textsubscript{agent} [taˈmí]\textsubscript{recipient}	[haˈré	gaˈjêta]\textsubscript{theme}	ˈʔà-k-o\\
            2\textsc{sg.nom}	mother-\textsc{poss}	1\textsc{sg.acc}	  some	cookie		give-\textsc{pst.ego-ep}\\
    \glt    ‘Your mom gave me some cookies’\\
    \glt    `Tu mamá me dió unas galletas' < BFL 09 1:89/el >\\
}\label{ex:14:canonical SOV order examplesa}
        \ex[]{
        {\textit{baʔaˈrîni ˈmí	ˈàma}}\\
        \gll    baʔaˈrî=[ni]\textsubscript{agent}	[ˈmí]\textsubscript{theme}	ˈà-ma\\
                tomorrow=1\textsc{sg.nom}	2\textsc{sg.acc}	look.for-\textsc{fut.sg}\\
		\glt    ‘I’ll look for you tomorrow’\\
		\glt    `Te voy a buscar mañana' < LEL 09 1:70/el >  \\
    }\label{ex:14:canonical SOV order examplesb}
    \z
\z

Declarative sentences can be described morphosyntactically and syntactically in terms of the basic clause types as described in \chapref{chap: basic clause types} (§\ref{subsec: basic clause types and transitivity properties}). The main morphosyntactic properties of basic clauses are summarized in (\ref{ex:14:Basic clauses: morphosyntactic properties}):

\ea\label{ex:14:Basic clauses: morphosyntactic properties}
{Basic clauses: morphosyntactic properties}

\begin{itemize}
    \item   The basic clause consists of a predicate and the arguments it subcategorizes for.\\
    \item   Pronominal forms encode core grammatical relations (subject and object), but there is no case marking in noun phrases.\\
    \item   Given that noun phrases can be elided, a minimal clause in Choguita Rarámuri consists of an inflected verbal predicate. \\
\end{itemize}
\z

Declaratives can also be characterized by the following prosodic properties (see §\ref{sec: intonation}; \citealt{caballero2014tone}, \citealt{garellek2015lexical}):

% check intonation section to match these two descriptions

\ea\label{ex:14:intonation of declaratives}
{Prosodic properties of declarative sentences}

\begin{itemize}
    \item There is evidence of H\% boundary tones\\
    \item Boundary H\% tones may be overridden by lexical or grammatical HL tones\\
    \item There are ``lead" (rhythmic) tones associated with lexical tones: lexical H and HL tones are optionally preceded by a L lead tone, while lexical L tones are optionally preceded by a H lead tone \parencite{garellek2015lexical}\\ %check actual subsection
    \item There are general and tone-specific non-tonal devices that encode intonation, including vowel re-articulation and lengthening at phrasal boundaries \parencite{aguilar2015multi, caballero2014tone}\\
    %\item Tonal and non-tonal intonation devices exhibit inter- and intra-speaker variation.\\
\end{itemize}
\z

\figref{fig: H boundary tone in declaratives} and \figref{fig: no H boundary tone in declaratives with lexical HL tones} from \chapref{chap: phonology} (repeated here as \figref{fig: H boundary tone in declaratives 3} and \figref{fig: no H boundary tone in declaratives with lexical HL tones 3} for  convenience) show the presence of H\% boundary tones and their absence in sentences with a final lexical or grammatical HL tone, respectively.

%\break

\begin{figure}
\includegraphics[width=\textwidth]{figures/Phonology-img1.png}
\caption{
\label{fig: H boundary tone in declaratives 3}
High boundary tone in declaratives \parencite{garellek2015lexical}}
\end{figure}

\begin{figure}
\includegraphics[width=\textwidth]{figures/Phonology-img2.png}
\caption{
\label{fig: no H boundary tone in declaratives with lexical HL tones 3}
No high boundary tone in declaratives with lexical HL tones \parencite{garellek2015lexical}}
\end{figure}

%Having addressed the morphosyntactic and prosodic properties of declarative utterances, we turn to interrogative constructions next.
%find better transition
The morphosyntactic and prosodic properties of declaratives provide the baseline for comparison when describing interrogative and imperative constructions. These are addressed next.

\section{Interrogative constructions}
\label{sec: interrogative constructions}

Different languages exploit a variety of morphosyntactic means to encode different speech acts, including assertions, commands and requests. While it is not uncommon that assertions or declarative sentences are unmarked, requests for information, or interrogative clauses, are cross-linguistically encoded through several grammatical devices. This section presents an overview of the conventionalized grammatical means Choguita Rarámuri exploits to express such requests for information.

\hspace*{-3pt}Interrogative constructions in Choguita Rarámuri include a distinction between polar questions (discussed in §\ref{subsec: polar questions}) and content questions (discussed in §\ref{subsec: content questions}), and each type in turn can be classified into different subtypes depending on their morphosyntactic and prosodic properties.

As discussed below, interrogative constructions that are morphosyntactically equivalent to declaratives are encoded exclusively through prosodic means, but prosody also plays a role in those interrogative constructions where there is a morphosyntactic device encoding the interrogative meaning. The existence of morphosyntactic mechanisms encoding different utterance types may preclude the use of distinctive intonational structures for the same purposes in some  languages (e.g., \ili{Navajo} (\ili{Athabaskan}; \citealt{mcdonough2002prosody})). In Choguita Rarámuri, both polar and content interrogative constructions are characterized by two main prosodic properties. These are listed in (\ref{ex:14:prosodic properties of interrogatives}):

\ea\label{ex:14:prosodic properties of interrogatives}
{Prosodic properties of interrogative constructions}

\begin{itemize}
    \item A boundary H\% tone targets the last stressed syllable of the utterance.\\
    \item There is raised register across the utterance.\\
    %\item A H* pitch accent that associates with the question word of the construction (if any).\\
\end{itemize}
\z

While polar questions show some degree of variation in the realization of the boundary tone and the degree to which register is raised (discussed below and in \chapref{chap: prosody}), they consistently exhibit a high pitch target utterance-finally. As shown below, the magnitude of the peak is correlated with the presence/absence of an overt morphological device to encode a question, with the highest pitch excursions attested utterance-finally with morphosyntactically unmarked polar questions.

\newpage
\subsection{Polar questions}
\label{subsec: polar questions}

Polar questions may be classified morphosyntactically into three types:

\ea\label{ex:14:morphosyntactic types of polar questions}
{Polar questions: morphosyntactic types}

    \begin{itemize}
        \item Morphosyntactically unmarked polar questions.\\
        \item Polar questions with interrogative particles.\\
        \item Polar questions with interrogative tags.\\
    \end{itemize}
\z

Each type is addressed in the following subsections.

\subsubsection{Morphosyntactically unmarked polar questions}
\label{subsubsec: morphosyntactically unmarked polar questions}

Polar interrogative clauses may be equivalent morphosyntactically to their declarative counterparts, differing only in their prosodic make-up. Some examples of unmarked polar questions are provided in (\ref{ex:14:examples of unmarked polar questions}) from conversational data.

\ea\label{ex:14:examples of unmarked polar questions}
{Morphosyntactically unmarked polar questions}

\ea[]{
  \textit{pe uˈsànabi?}\\
    \gll    pe uˈsàni-na=bi?\\
            just six-\textsc{distr}=just\\
    \glt    ‘Just in six places?  \\
    \glt    ‘Nomás en seis partes?’ \corpuslink{in61[04_378-04_391].wav}{SFH in61:04:37.8}\\
}
        \ex[]{
        \textit{ke   biˈlé   pe   ˈtâʃi   iˈtêeli     ˈònam     t͡ʃaˈbè   ko?}\\
        \gll    ke  biˈlé   pe   ˈtâsi  iˈtê-li    ˈòna-ame    t͡ʃaˈbè=ko\\
                \textsc{neg}  one  \textsc{neg}  \textsc{neg}  exist.\textsc{neg-pst}  cure-\textsc{ptcp}  before=\textsc{emph}\\
        \glt    ‘There was no medicine before?’\\
        \glt    ‘No había medicina antes?’ \corpuslink{in61[06_396-06_420].wav}{SFH in61:06:39.6}\\
    }
    \z
\z

As discussed in \chapref{chap: tone and intonation}, closely-related \ili{Mountain Guarijío} (\ili{Taracahitan}; \ili{Uto-Aztecan}) is reported to also have polar questions that are morphosyntactically equivalent to their declarative counterparts, with the interrogative meaning encoded through ascending intonation (``\textit{[g]eneralmente tienen entonación ascendente}") \citep[112]{miller1996guarijio}. A minimal pair between a declarative sentence and its morphosyntactically unmarked polar question counterpart in Choguita Rarámuri is provided in (\ref{ex:14:intonation marked polar questions}). \figref{fig: declarative intonation} and \figref{fig: morphosyntactically unmarked polar intonation} show the intonational difference between the declarative sentence in (\ref{ex:14:intonation marked polar questionsa}) and the polar interrogative in (\ref{ex:14:intonation marked polar questionsb}), respectively.

\pagebreak

\ea\label{ex:14:intonation marked polar questions}
{Declarative vs. morphosyntactically unmarked polar question}

    \ea[]{
    {\textit{ˈmá ˈtôlo}}\\
    \gll    ˈmá ˈtô-li\\
            already bury-\textsc{pst}\\
    \glt    ‘S/he buried him/her.’\footnote{In this particular example, the suffix vowel surfaces as [o] given an optional round harmony process, where non-round vowels of certain suffixes may become round when preceded by a stressed back stem vowel. for more details about this process, see §\ref{subsubsec: round harmony}.}\\
    \glt    `Lo enterró.' {< BFL el1170 >}\\
}\label{ex:14:intonation marked polar questionsa}
        \ex[]{
        {\textit{ˈmá ˈtôli?}}\\
        \gll    ˈmá ˈtô-li?\\
              	already bury-\textsc{pst}\\
	    \glt    ‘Did s/he bury him/her?’\\
	    \glt    `¿Lo enterró?' {< BFL el1307 >}\\
    }\label{ex:14:intonation marked polar questionsb}
    \z
\z

\begin{figure}
\includegraphics[width=\textwidth]{figures/SentenceTypes-img2.png}
\caption{
\label{fig: declarative intonation}
Declarative utterance: \textit{ˈmá ˈtôlo} `S/he buried him/her' (< BFL el1170 >)}
\end{figure}

\begin{figure}
\includegraphics[width=\textwidth]{figures/SentenceTypes-img3.png}
\caption{
\label{fig: morphosyntactically unmarked polar intonation}
Morphosyntactically unmarked polar interrogative: \textit{ˈmá ˈtôli?} `Did s/he bury him/her?' (< BFL el1307 >)}
\end{figure}

The declarative sentence in \figref{fig: declarative intonation} features the falling pitch contour of the lexical HL tone of the verb (\textit{ˈtô} `to bury') (note there is no evidence of a H\% boundary tone found in declaratives in this utterance since the falling lexical tone in the stressed syllable overrides the boundary tone, as discussed in §\ref{sec: interaction between lexical tone and intonation}). The comparison between the declarative utterance in \figref{fig: declarative intonation} and the interrogative one in \figref{fig: morphosyntactically unmarked polar intonation} shows raised register and significantly raised f0 in the stressed syllable in the interrogative (from 289 Hz to 396 Hz in this particular example for the same female speaker (BFL)). The high pitch target in the stressed syllable may be attributed to the interrogative H\% intoneme aligning with the peak of the lexical HL tone in this stressed syllable.

A different intonation pattern is documented in interrogative sentences with a final stressed syllable specified for lexical L tone. This is shown in \figref{fig: L tone plus H boundary tone}.

%\break

\begin{figure}
\includegraphics[width=\textwidth]{figures/SentenceTypes-img5.png}
\caption{
\label{fig: L tone plus H boundary tone}
Accommodation of L tone and H\% boundary tone in \textit{ˈmá ˈnèli?} `Did s/he see him/her?' (< BFL el1307 >)}
\end{figure}

As shown in this Figure, the lexical tone of the verb root is associated to the stressed syllable, while the H\% boundary tone docks on a following, unstressed syllable. Thus, and as documented in declarative contexts, lexical tones are preserved over intonational ones (see §\ref{sec: interaction between lexical tone and intonation} for further discussion).

\subsubsection{Polar questions with interrogative particles}
\label{subsubsec: polar questions with interrogative particles}

Polar questions may also be encoded through polar interrogative particle \textit{ˈátʃ͡e} or its reduced form \textit{ˈá}, which occur in clause initial position. Some examples of polar questions are provided in (\ref{ex:14:interrogative particle polar questions}) (in these examples polar interrogative words are highlighted in \textbf{boldface}).

\ea\label{ex:14:interrogative particle polar questions}
{Polar questions with interrogative particles}

    \ea[]{
    {\textit{\textbf{ˈátʃe} wiˈtʃ͡iara  hu?}}\\
    \gll    \textbf{ˈátʃe } wiˈtʃ͡iara  hu?\\
            Q  truth \textsc{cop.prs}\\
    \glt    ‘Is it true?’ \\
    \glt    `¿Es verdad? < SFH 06 in61(580)/in >\\
}
        \ex[]{
        \textbf{\textit{ˈátʃi}} \textit{ˈmí   riˈwèki?}\\
        \gll    \textbf{ˈátʃe}  ˈmí    riˈwè-ki\\
                {Q} \textsc{2sg.nom}  leave-\textsc{pst.ego}\\
        \glt    ‘Did you leave it?’\\
        \glt    ‘¿Lo dejaste?’ {< GFP 09 3:14/el >}\\
    }
            \ex[]{
            {\textit{\textbf{ˈá}m ku ˈpàa saˈmîra?}}\\
            \gll    \textbf{ˈá}=mi ku ˈpàa saˈmîra?\\
                    {Q}=\textsc{2sg.nom} \textsc{rev} bring.\textsc{prs} Samira\\
            \glt    `Did you already bring Samira?'\\
            \glt    `¿Ya trajiste a Samira?' {< JLG co1237[0\_496-0\_506 >}\\
        }
    \z
\z

The following minimal pair illustrates the contrast between declarative sentences (e.g., (\ref{ex:14:declarative vs. polar interrogative with achea})) and their polar interrogative counterparts (e.g., (\ref{ex:14:declarative vs. polar interrogative with acheb})).

\ea\label{ex:14:declarative vs. polar interrogative with ache}
{Declarative vs. polar interrogative with \textit{ˈátʃ͡e}}

    \ea[]{
    {\textit{ˈmá       naʔˈpôli   naˈʔî}}\\
    \gll    ˈmá naʔˈpô-li   naˈʔî\\
            already weed-\textsc{pst} here\\
    \glt    ‘S/he already weeded here.’\\
    \glt    `Ya escardó aqui.' < SFH el1586 >\\
}\label{ex:14:declarative vs. polar interrogative with achea}
        \ex[]{
        {\textit{\textbf{ˈátʃe} ˈmá naʔˈpôli naˈʔî?}}\\
        \gll    \textbf{ˈátʃe}  ˈmá naʔˈpo-li   naˈʔî?\\
                Q     yet weed-\textsc{pst}  here\\
        \glt    ‘Did s/he already weed here?’\\
        \glt    `¿Ya escardó aqui?' < SFH el1586 >\\
    }\label{ex:14:declarative vs. polar interrogative with acheb}
    \z
\z

As seen in this minimal pair, there is no change in the order of constituents in the interrogative construction.

The following minimal pair illustrates the prosodic difference between a declarative with a lexical L tone in utterance-final position ((\ref{ex:14:declarative vs. polar interrogative with achea}) in \figref{fig: declarative with L tone}) and its polar interrogative counterpart with an interrogative particle ((\ref{ex:14:declarative vs. polar interrogative with acheb}) in \figref{fig: polar interrogative with ache lexical L  tone}).

\ea\label{ex:14:declarative vs. polar interrogative ache}
{Declarative vs. polar interrogative with \textit{ˈátʃ͡e}}

    \ea[]{
    {\textit{ˈmá naˈwàli}}\\
    \gll    ˈmá naˈwà-li\\
            already arrive-\textsc{pst}\\
    \glt    `S/he already arrived.'\\
    \glt    `Ya llegó.' {< SFH-nawa-arrive-L-minimal-sets >}\\
}
%\break

        \ex[]{
        {\textit{\textbf{ˈátʃe} ˈmá naˈwàli?}}\\
        \gll    \textbf{ˈátʃe} ˈmá naˈwà-li\\
                Q already arrive-\textsc{pst}\\
        \glt    `Did s/he already arrive?'\\
        \glt    `¿Ya llegó?' < SFH-nawa-arrive-L-minimal-sets >\\
    }
    \z
\z

\begin{figure}
\includegraphics[width=\textwidth]{figures/SentenceTypes-img6.png}
\caption{
\label{fig: declarative with L tone}
Declarative with utterance-final lexical L tone in \textit{ˈmá naˈwàli} `S/he already arrived.' }
\end{figure}

\begin{figure}
\includegraphics[width=\textwidth]{figures/SentenceTypes-img7.png}
\caption{
\label{fig: polar interrogative with ache lexical L  tone}
Polar interrogative with \textit{ˈátʃ͡e} and utterance-final lexical L tone in \textit{ˈátʃ͡e ˈmá naˈwàli?} `Did s/he already arrive?' }
\end{figure}

As attested in polar interrogatives that are morphosyntactically unmarked, the intonational encoding of polar questions with interrogative particles also involves the presence of a H\% boundary tone, raised register and the preservation of lexical L tone: as seen in \figref{fig: polar interrogative with ache lexical L  tone}, the sharp rise and peak of the H\% tone is aligned with the final, post-tonic syllable.

\subsubsection{Polar questions with interrogative tags}
\label{subsubsec: Polar questions with interrogative tags}

A third type of polar question involves the use of a negative interrogative tag (formally a phrase) at the end of the sentence. The negative interrogative tag characterizes sentences as questions, but also contributes an expectation that the answer will be positive. An example of this construction is provided in (\ref{ex:14:interrogative tags for polar questions}):

\ea\label{ex:14:interrogative tags for polar questions}
{Polar questions with interrogative tags}

    \ea[]{
        {\textit{ˈmá    ku  siˈmíli    maˈnûel,  \textbf{we ra ˈkeo}?}}\\
        \gll    ˈmá    ku  siˈmí-li    maˈnûel,  \textbf{we  ra ˈke-o}?\\
                already    \textsc{rev} go.\textsc{sg-pst} Manuel  {or  or \textsc{neg-ep}}\\
        \glt    ‘Manuel already left, or not?’ \\
        \glt    `Manuel ya se fue, ¿o no?' < BFL 09 1:45/el >\\
}
    \z
\z

The tag question involves a disjunctive conjunction and a negative particle. In this particular example, the speaker expects a positive answer.

\subsection{Content questions}
\label{subsec: content questions}

In the canonical structure of content questions in Choguita Rarámuri, the question word appears in clause initial position. This is exemplified in (\ref{ex:14:content question examples}).

\ea\label{ex:14:content question examples}
{Content questions}

    \ea[]{
    {\textit{\textbf{ˈpîri } iʔˈkîli koˈtʃ͡î}? }  \\
    \gll    \textbf{ˈpîri}  iʔˈkî-li koˈtʃ͡î? \\
            what  bite-\textsc{pst}    dog\\
    \glt    ‘What did the dog bite?’ \\
    \glt    `¿Qué mordió el perro?' < BFL 09 el725/el >\\
}
        \ex[]{
        {\textit{\textbf{ˈhêpi ˈkwâ}mi ˈʔâbo    maˈjêi    winoˈmî?}}\\
        \gll   \textbf{ˈhêpi ˈkwâ}=mi ˈʔâ-bo    maˈjê-i    winoˈmî?\\
                who=\textsc{2sg.nom} give\textsc{-fut.pl} think\textsc{-impf} money\\
        \glt    ‘Who did you think they were going to give the money to?’ \\
        \glt    `¿A quién creías que le iban a dar el dinero?' < BFL 09 1:12/el >\\
    }
    \z
\z

As mentioned above, content (or information) questions are also characterized by a distinctive intonational contour: as shown in the contrast between a declarative sentence (shown in (\ref{ex:14:declarative vs. content questiona}), \figref{fig: declarative with L tone-2}) and a content question (shown in (\ref{ex:14:declarative vs. content questionb}), \figref{fig: content question lexical L  tone}).

\ea\label{ex:14:declarative vs. content question}
{Declarative vs. content question}

 \ea[]{
    {\textit{ˈmá naˈwàli}}\\
    \gll    ˈmá naˈwà-li\\
            already arrive-\textsc{pst}\\
    \glt    `S/he already arrived.'\\
    \glt    `Ya llegó.' {< SFH-nawa-arrive-L-minimal-sets >}\\
}\label{ex:14:declarative vs. content questiona}
        \ex[]{
        {\textit{\textbf{ˈhêpi ˈkwâ} naˈwàli?}}\\
        \gll    \textbf{ˈhêpi} \textbf{ˈkwâ} naˈwà-li\\
                who who arrive-\textsc{pst}\\
        \glt    `Who arrived?'\\
        \glt    `¿Quién llegó?' {< SFH-nawa-arrive-L-minimal-sets >}\\
    }\label{ex:14:declarative vs. content questionb}
    \z
\z


\begin{figure}
\includegraphics[width=\textwidth]{figures/SentenceTypes-img13.png}
\caption{
\label{fig: declarative with L tone-2}
Declarative with utterance-final lexical L tone}
\end{figure}

\begin{figure}
\includegraphics[width=\textwidth]{figures/SentenceTypes-img8.png}
\caption{
\label{fig: content question lexical L  tone}
Content question with utterance-final lexical L tone}
\end{figure}

The pitch track of the content question in \figref{fig: content question lexical L  tone} shows a boundary H\% tone in the last unstressed syllable of the utterance and register raising across the utterance.

\largerpage
A H\% boundary tone is also present in content questions where a lexical H tone is associated to the final stressed syllable of the utterance. This is shown in (\ref{ex:14:declarative vs. content question with H tone}), with a contrast between a declarative in (\ref{ex:14:declarative vs. content question with H tonea}) (illustrated in \figref{fig: declarative with H tone}) and a content question in (\ref{ex:14:declarative vs. content question with H toneb}) (illustrated in \figref{fig: content question lexical H tone}).

\ea\label{ex:14:declarative vs. content question with H tone}
{Declarative vs. content question: utterance-final, lexical H tone}

 \ea[]{
    {\textit{ˈmá muˈrúli}}\\
    \gll    ˈmá muˈrú-li\\
            already carry.in.arms-\textsc{pst}\\
    \glt    `S/he already carried it in their arms'\\
    \glt    `Ya lo cargó en brazos' {< BFL-muru-carry-H-minimal-sets >}\\
}\label{ex:14:declarative vs. content question with H tonea}
        \ex[]{
        {\textit{\textbf{ˈhêpi ˈkwâ} muˈrúli?}}\\
        \gll    \textbf{ˈhêpi} \textbf{ˈkwâ} muˈrú-li\\
                who who carry.in.arms-\textsc{pst}\\
        \glt    `Who carried it in their arms?'\\
        \glt    `¿Quién lo cargó en brazos?' {< BFL-muru-carry-H-minimal-sets >}\\
    }\label{ex:14:declarative vs. content question with H toneb}
    \z
\z

\begin{figure}
\includegraphics[width=\textwidth]{figures/SentenceTypes-img11.png}
\caption{
\label{fig: declarative with H tone}
Declarative with utterance-final lexical H tone}
\end{figure}

\begin{figure}
\includegraphics[width=\textwidth]{figures/SentenceTypes-img12.png}
\caption{
\label{fig: content question lexical H tone}
Content question with utterance-final lexical H tone}
\end{figure}

In the declarative sentence represented in \figref{fig: declarative with H tone} there is no evidence of a H\% boundary tone, which is optional in declaratives. In the content question represented in \figref{fig: content question lexical H tone}, on the other hand, there is a clear pitch target in the final, unstressed syllable of the utterance, which is higher than the one associated with the lexical H tone of the stressed syllable (a difference of almost 30Hz in this particular example).

There is, however, no evidence of a H\% boundary tone nor any register manipulation when the content question contains a lexical HL tone in utterance final position. This is shown in the contrast between the declarative sentence in (\ref{ex:14:declarative vs. content question with HL tonea}) (\figref{fig: declarative with HL tone}) and its content question counterpart in (\ref{ex:14:declarative vs. content question with HL toneb}) (\figref{fig: content question with lexical HL tone}).

%\break

\ea\label{ex:14:declarative vs. content question with HL tone}
{Declarative vs. content question: utterance-final, lexical HL tone}

 \ea[]{
    {\textit{ˈmá iˈsîli}}\\
    \gll    ˈmá iˈsî-li\\
            already pee-\textsc{pst}\\
    \glt    `S/he already peed.'\\
    \glt    `Ya orinó.' {< BFL-isi-pee-HL-minimal-set >}\\
}\label{ex:14:declarative vs. content question with HL tonea}
        \ex[]{
        {\textit{\textbf{ˈhêpi ˈkwâ} iˈsîli?}}\\
        \gll    \textbf{ˈhêpi} \textbf{ˈkwâ} iˈsî-li\\
                who who pee-\textsc{pst}\\
        \glt    `Who peed?'\\
        \glt    `¿Quién orinó?' {< BFL-isi-pee-HL-minimal-set >}\\
    }\label{ex:14:declarative vs. content question with HL toneb}
    \z
\z

\begin{figure}
\includegraphics[width=\textwidth]{figures/SentenceTypes-img9.png}
\caption{
\label{fig: declarative with HL tone}
Declarative with utterance-final lexical HL tone}
\end{figure}

\begin{figure}
\includegraphics[width=\textwidth]{figures/SentenceTypes-img10.png}
\caption{
\label{fig: content question with lexical HL tone}
Content question with utterance-final lexical HL tone}
\end{figure}

In this pair of examples (produced by the female speaker BFL), the highest pitch peak associated with the lexical HL tone is comparable (270Hz in the declarative in \figref{fig: declarative with HL tone} and 260Hz in the content question in \figref{fig: content question with lexical HL tone}), and their pitch countours largely equivalent. Thus, the comparison between the intonational contours of these two sentences shows that there is no evidence of any distinctive intonational encoding of a content question where HL tones override the H\% tone associated with interrogative constructions elsewhere. This stands in contrast to polar questions, where, as shown in \figref{fig: morphosyntactically unmarked polar intonation} above, there is significantly raised f0 in the final stressed syllable of the interrogative utterance.

%As exemplified in \figref{fig:key:39}, a boundary tone will associate with a toneless (/Ø/) stressed syllable in utterance final position.

%\textbf{\figref{fig:key:39}:} toneless /Ø/ root (\textit{aˈwi}\textit{{}-ki} ‘dance- \textsc{pst.1’}) in phrase-final position, content question

%%please move the includegraphics inside the {figure} environment
%%\includegraphics[width=\textwidth]{GrammardraftJuly182017-img11.wmf}

%\textit{ˈkúmi=mi        aˈwi}  \textit{{}-ki?}
%where=\textsc{2sg.nom}    dance- \textsc{pst.1}
%‘Where did you dance?’
%< BFL el1324]

Content or information questions use one of a set of interrogative pronouns and phrases (see also §\ref{subsec: interrogative pronouns}), which mark the clause as an interrogative one and fulfill the function of indicating which information is being requested. \tabref{tab:interrogative-words} displays the question words and phrases encountered in the Choguita Rarámuri corpus.


\begin{table}[b]
\caption{Choguita Rarámuri interrogative words and phrases}
\label{tab:interrogative-words}

\begin{tabularx}{\textwidth}{lQ}
\lsptoprule
\textbf{Forms}  & \textbf{Gloss} \\
\midrule
\textit{ˈpîri?}  & What? (\textit{¿Qué?})\\
\textit{he (pi) ˈkwâ? (uʔuˈka)} & Who? (\textit{¿Quién?}) \\
\textit{ˈkámi? ˈkúmi?}  & Where? (\textit{¿Dónde?})\\
\textit{tʃ͡ú (t͡ʃe) riˈká?} & How?(\textit{¿Cómo?})\\
\textit{tʃ͡ú?} & How? (\textit{¿Cómo?})\\
\textit{t͡ʃi ˈjíri?} & Which kind? (\textit{¿Qué tipo?})\\
\textit{(tʃ͡ú) ˈkípi?} & How many? (\textit{¿Cuántos?})\\
\textit{tʃ͡ú ˈrúpi?} & How much? (\textit{¿Qué tanto?})\\
\textit{tʃ͡ú ˈjêni?} & How much? (\textit{¿Qué tanto?})\\
\textit{tʃ͡ú iˈkíana?} & How many places? (\textit{¿Cuántos lugares?})\\
\textit{tʃ͡ú kiˈnápi?} & At how many places? (\textit{¿Qué tantos lugares?})\\
\textit{tʃ͡ú riˈkó?}  & When? (\textit{¿Cuándo?})\\
\textit{kaˈbú?} & When? (\textit{¿Cuándo?})\\
\textit{tʃ͡ú (t͡ʃe) oˈlá?} & Why? (\textit{¿Por qué?})\\
\textit{tʃ͡ú ˈjêni?} & At what time? (\textit{¿A qué hora?})\\
\textit{tʃ͡ú kiˈrípi?} & How long? (\textit{¿Cuánto tiempo?})\\
\textit{ˈpîri ˈnà-ti?} & With what? (\textit{¿Con qué?})\\
\textit{tʃ͡ú ˈnà-ti?} & With what? (\textit{¿Con qué?}\\
\textit{(he pi) ˈkwâ ˈjûa?} & With whom? (\textit{¿Con quién?})\\
\lspbottomrule
\end{tabularx}
\end{table}

While the canonical position of interrogative words in content questions is in clause initial position (as in (\ref{ex:14:in situ interrogative wordsa})), they may also appear \textit{in situ} (as in (\ref{ex:14:in situ interrogative wordsb})).

\ea\label{ex:14:in situ interrogative words}
{Position of interrogative words in content questions}

    \ea[]{
    {\textit{Fronted}}\\
    \textit{\textbf{\textit{ˈpîri}} \textit{iʔˈkîli koˈt͡ʃî?}}\\
    \gll    \textbf{ˈpîri} iʔˈkî-li koˈt͡ʃî?\\
            what  bite-\textsc{pst}    dog\\
    \glt    ‘What did the dog bite?’\\
    \glt    ‘¿Qué mordió el perro?’ {< BFL 09 el725/el >}\\
}\label{ex:14:in situ interrogative wordsa}
        \ex[]{
        {{\textit{In situ}}}\\
        \textit{koˈt͡ʃî \textbf{ˈpîri} iʔˈkî-li?} \\
        \gll    koˈt͡ʃî \textbf{ˈpîri} iʔˈkî-li \\
                dog  what  \textsc{pst}\\
        \glt    ‘What did the dog bite?’\\
        \glt    ‘¿Qué mordió el perro?’ {< BFL 09 el725/el >}\\
    }\label{ex:14:in situ interrogative wordsb}
    \z
\z

\largerpage
As shown in (\ref{ex:14:in situ interrogative wordsb}), the question word for the object appears pre-verbally and after the Subject NP, as in the unmarked SOV word order in Choguita Rarámuri (as described in §\ref{sec: declarative sentences}).

%more examples?
The examples so far have shown constituent interrogatives in simple clauses. In complex clauses, it is possible to ask questions where the question word stands for an argument of a complement clause. In these cases, the interrogative word is fronted in the matrix clause. This is shown in the contrast between a declarative sentence with a subordinate clause (in (\ref{ex:14:question of argument of embedded clausea})) and the content question counterpart of this declarative (in (\ref{ex:14:question of argument of embedded clauseb})). The question word is highlighted in boldface.

\ea\label{ex:14:question of argument of embedded clause}
{Constituent interrogative: argument of subordinate clause}\\

    \ea[]{
    \textit{ˈnè     ko   raˈmôn     ˈâbo     maˈjêki   winoˈmî}\\
    \gll    ˈnè=ko raˈmôn ˈâ-bo maˈjê-ki winoˈmî\\
            1\textsc{sg.nom=emph} Ramón give-\textsc{fut.pl} think-\textsc{pst.ego} money\\
    \glt    ‘I thought they were going to give the money to Ramón.’\\
    \glt    ‘Yo pensé que le iban a dar el dinero a Ramón.’ {< BFL 09 1:12/el >}\\
}\label{ex:14:question of argument of embedded clausea}
        \ex[]{
        \textit{\textbf{ˈhêpi ˈkwâ}mi ˈʔâbo maˈjêi winoˈmî}\\
        \gll    \textbf{ˈhêpi} \textbf{ˈkwâ}=miˈ ʔâ-bo maˈjê-i wenoˈmî\\
                who who=\textsc{2sg.nom}  give\textsc{-fut.pl} think\textsc{-impf} money\\
        \glt    ‘Who did you think they were going to give the money to?’\\
        \glt    ‘¿A quién pensaste que le iban a dar el dinero?’ {< BFL 09 1:12/el >}\\
    }\label{ex:14:question of argument of embedded clauseb}
    \z
\z

As can be seen in these examples, constituent order is not altered beyond placement of the question word in sentence initial position: both sentences share the property of having the theme argument of the complement clause in sentence final position (\textit{winoˈmî} `money'), and the dependent verb (\textit{ˈʔà} `give') preceding the matrix verb (\textit{maˈjê} `think').

Further examples of constituent interrogatives of arguments of complement clauses are provided in (\ref{ex:14:question of argument of embeddeed clause 2}).

%\break

\ea\label{ex:14:question of argument of embeddeed clause 2}
{Constituent interrogative: argument of subordinate clause}\\
    \ea[]{
    \textit{\textbf{ˈpîri}mi taˈmí ˈàlo riˈmù?}\\
    \gll    \textbf{ˈpîri}=mi    taˈmí    ˈà-li-o    riˈmù\\
            what=\textsc{2sg.nom}  \textsc{2sg.acc} give-\textsc{pst-ep} dream\\
    \glt    ‘What did you dream he gave me?’\\
    \glt    ‘¿Qué soñaste que me dió? {< BFL 09 1:12/el >}\\
    }\label{ex:14:question of argument of embeddeed clause 2a}
        \ex[]{
        \textit{\textbf{ˈhêpu ˈkwâ}mi riˈmù  [ˈnápuni} \textit{ˈàlo biˈlé riˈmê]?}\\
        \gll    ˈhêpi ˈkwâ=mi riˈmù ˈnápi=ni ˈà-li-o biˈlé reˈmê\\
                who who=\textsc{2sg.nom} dream \textsc{sub=1sg.nom} give-\textsc{pst-ep} one tortilla\\
        \glt    ‘Who did you dream I gave a tortilla to?’\\
        \glt    ‘¿A quién soñaste que le di una tortilla?’ {< BFL 09 1:12/el >}\\
    }\label{ex:14:question of argument of embeddeed clause 2b}
    \z
\z

These examples show a content question where the question word stands for an argument of a complement clause, an object argument in (\ref{ex:14:question of argument of embeddeed clause 2a}) and a recipient argument (\ref{ex:14:question of argument of embeddeed clause 2b}) of the ditransitive predicate \textit{ˈʔà} `to give' (the complement clauses are indicated with square brackets). Further details on the syntax of complement clauses are provided in \sectref{sec: complement clauses}.

\section{Negation}
\label{sec: negative constructions}

Negation may be expressed in Choguita Rarámuri through a number of mechanisms. First, there are several negative free forms, addressed in §\ref{subsec: negative interjections} below, which are antonyms of positive free forms (\textit{\textit{aˈjena}} and \textit{ˈtio} `yes'). Some of these forms are also deployed in clausal negation, as discussed in §\ref{subsec: clausal negation}. Negative forms that function as constituent negators are described in §\ref{subsec: constituent negation}. Finally, negative existential and negative locative clauses are addressed in §\ref{subsec: negative existence and legative locative clauses}.

%this also in small word classes
The set of negative particles and complex negative markers available in Choguita Rarámuri are provided in \tabref{tab:negative-markers}, with their gloss, function and approximate translation (for more details about the morphological properties of negative particles, see §\ref{subsec: negative particles}).\footnote{As discussed in \chapref{chap: particles, adverbs and other word classes} (§\ref{sec: particles and clitics}), `particles' are defined as a set of heterogeneous word classes that are characterized by being closed and morphologically simple, bearing no inflection or derivation and, in some cases, being phonologically reduced. Each class of particles is composed of fewer than a dozen members per class and may have a wide range of functions and meanings.}

%\break

\begin{table}
\caption{Negative markers}
\label{tab:negative-markers}

\begin{tabularx}{\textwidth}{llQl}
\lsptoprule
\textbf{Form} & \textbf{Gloss} & \textbf{Function} & \textbf{Translation}\\
\midrule
\textit{ke} & \textsc{neg} & interjection, clausal negation & ‘No’\\
\textit{ˈkíti} & \textsc{proh} & prohibitive (negative imperative) & `Don't!'\\
\textit{ke ˈtâsi} & \textsc{neg neg} & interjection, clausal negation & ‘No’\\
\textit{pe ke biˈlé} & just \textsc{neg} one & emphatic interjection & `Not at all!'\\
\textit{ka ˈt͡ʃè} & \textsc{neg.irr} again & clausal negation &`Not again/anymore’\\
\textit{ke/ˈtâsi t͡ʃó} & \textsc{neg} yet & clausal negation & ‘Neither’\\
\textit{ke/ˈtâsi biˈlé} & \textsc{neg} one & clausal negation, constituent neg. & \makecell[tl]{‘Nothing at all’,\\‘No single’}\\
\textit{ni biˈlé} & nor one & constituent neg. & ‘Nor any’\\
\lspbottomrule
\end{tabularx}
\end{table}
\hspace{3cm}

In addition to these negative particles and complex negative markers, there are morphologically complex negative quantifiers in the language which involve negative particles, including \textit{ke/ˈtâsi (biˈlé) naˈmûti} `none, nothing' and \textit{ke/ˈtâsi ˈwêsi} `nobody'.

%\textbf{Scope of negation of particles?}
%\textbf{Constituent negation?}
%\textbf{Order of negative particle in clausal negation and constituent?}

%Examples of broad and narrow scope of negation

%I’m \textit{ne       ˈkíti     ˈbe=ni} \textbf{\textit{ke   ˈtaʃi}} \textit{waʔˈru   bené-ka   t͡ʃo}
 %I’m  1\textsc{sg.nom}   because  just=1\textsc{sg.nom}  \textbf{\textsc{neg   neg}} a.lot   learn-ger also
  %\textit{ne     ko   ba}
   %1\textsc{sg.nom   emph   cl}
% ‘That’s why I didn’t learn that much’
%‘Por eso no aprendí mucho yo’
 %< SFH 06 tx12(19)/tx >

\subsection{Negative free forms}
\label{subsec: negative interjections}

A range of negative forms, including \textit{ke}, and \textit{ke ˈtâsi}, are frequently found in answers to questions, whether the questions are negative (as in (\ref{ex:14:negative interjections: answer question})) or positive (as in (\ref{ex:14:negative interjections: answer question 2}) and (\ref{ex:14:negative interjections: anwser questions 3})) indifferently (in contrast to languages that have specialized markers for answers contradicting negative questions, e.g. \ili{French} \textit{si}, \ili{German} \textit{doch}, inter alia).

\newpage
\ea\label{ex:14:negative interjections: answer question}
{Negative forms: answering negative questions}

    \ea[]{
    [GCH]: \textit{ke me apaˈrûame ˈétʃ͡i}\\
    \gll    ke me apaˈrûame ˈétʃ͡i?\\
            \textsc{neg} almost fierce \textsc{dem}\\
    \glt    `Is he not that fierce that one?\\
    \glt    `¿No es tan bravo ese?' {< GCH co1136 >}\\
}
%\pagebreak
        \ex[]{
        [MDH]: \textit{B: \textbf{ke}, ke me oˈbáta ˈlé}\\
        \gll   \textbf{ ke}, ke me oˈbáta aˈlé\\
                \textsc{neg} \textsc{neg} almost fierce.\textsc{pl} \textsc{dub}\\
        \glt    `No, they are not that fierce'\\
        \glt    `No, casi no son bravos (no se enojan)'  {\corpuslink{co1136[14_414-14_430].wav}{MDH co1136:14:41.4}}\\
    }
    \z
\z

\ea\label{ex:14:negative interjections: answer question 2}
{Negative forms: answering positive questions}

    \ea[]{
    {[SFH]: \textit{naˈlìna ke pe ˈá ˈtʃ͡étimi itʃ͡aˈkámi koˈʔáa onoˈkáli aˈlé tʃ͡aˈbèewami ko ˈnápu riˈká ˈnàa ... ˈnápu riˈká reˈpôjo ˈmí itʃ͡iˈwáwamti?}}\\
    \gll    naˈlìna ke pe ˈá ˈtʃ͡e=timi itʃ͡aˈ-kámi koˈʔá-a onoˈká-li aˈlé tʃ͡aˈbè-w-ami=ko ˈnápu riˈká ˈnà ˈnápu riˈká reˈpôjo ˈmí itʃ͡iˈ-wá-w-am-ti?\\
            so \textsc{neg} just \textsc{aff} also=\textsc{2pl.nom} plant-\textsc{ptcp} eat-\textsc{prog} do-\textsc{-pst} \textsc{dub} before-\textsc{nmlz-ptcp}=\textsc{emph} \textsc{sub} like \textsc{dem} \textsc{sub} like cabbage there plant-\textsc{mpass-w-ptcp-com}\\
    \glt    `And did you all indeed also plant (for eating) like ... like the planting of cabbage?\\
    \glt    `¿Y sí sembrarían antes (cosas) como ... así como la siembra de repollo?' {\corpuslink{in242[03_420-03_481].wav}{SFH in242:3:42.0}, \corpuslink{in242[03_528-03_557].wav}{in242:3:52.8}}\\
}
        \ex[]{
        {[FLP]: \textit{\textbf{ke ˈtâsi}}}\\
        \gll    \textbf{ke} \textbf{ˈtâsi}\\
                {\textsc{neg}} {\textsc{neg}}\\
        \glt    `No.' {\corpuslink{in242[03_557-03_578].wav}{FLP in242:03:55.7}}\\
    }
    \z
\z

\ea\label{ex:14:negative interjections: anwser questions 3}
{Negative forms: answering positive questions}

    \ea[]{
    {[SFH]: \textit{tʃ͡ú riˈká? ˈnè waˈjéla ˈjûa aʃiˈsâ?}}\\
    \gll    tʃ͡ú riˈká ˈnè waˈjé-la ˈjûa asi-ˈsâ?\\
            Q how 1\textsc{sg.nom} younger.sister.male.ego-\textsc{poss} with sit-\textsc{cond}\\
    \glt    `How? As if he were (together) with my younger sister?'\\
    \glt    `¿Cómo? ¿Cómo si estuviera (emparejado) con mi hermana menor? {\corpuslink{in485[02_320-02_334].wav}{SFH in485:02:32.0}}\\
}
        \ex[]{
        {[ME]: \textit{\textbf{ke!}}}\\
        \gll   \textbf{ ke!}\\
                {\textsc{neg}}\\
        \glt    `No!' {\corpuslink{in485[02_338-02_347].wav}{ME in485:02:33.8}}\\
    }
    \z
\z

These negative forms are also used to contradict statements assumed by the speaker to be incorrect, as in (\ref{ex:14:negative interjections: contradict statements}).

\ea\label{ex:14:negative interjections: contradict statements}
{Negative forms: contradicting statements}

    \ea[]{
    {[GCH]: \textit{ˈmá naˈwàli miˈkêli}}\\
    \gll    ˈmá naˈwà-li miˈkêli\\
            already arrive-\textsc{pst} Michael\\
    \glt    `Michael already arrived'\\
    \glt    `Ya llegó Miguel'\\
}
        \ex[]{
        {[MDH]: \textit{ke ˈtâsi, pe ke ˈtʃ͡ó naˈwàli}}\\
        \gll    \textbf{ke} \textbf{ˈtâsi} pe ke ˈtʃ͡ó naˈwà-li\\
                {\textsc{neg}} {\textsc{neg}} just \textsc{neg} yet arrive-\textsc{pst}\\
        \glt    `No, he hasn't arrived yet'\\
        \glt    `No, todavía no llega'\\
    }
    \z
\z


The form \textit{pe ke biˈlé} is an emphatic interjection, adding emphasis to a question or statement to express a strong disagreement. Examples of this emphatic interjection are provided in (\ref{ex:14:emphatic negative interjection}) and (\ref{ex:14:emphatic negative interjection 2}).

\ea\label{ex:14:emphatic negative interjection}
{Emphatic negative interjection}

    \ea[]{
    {[FLP]: \textit{ˈnápi sibiˈrîko ˈdîas aˈtí ˈká ˈtʃ͡è wiliˈbê wasaˈrúi ba}}\\
    \gll    ˈnápi sibiˈrîko ˈdîas aˈtí ˈká ˈtʃ͡è wiliˈbê wasa-ˈrú-i ba\\
            \textsc{sub} Federico Diaz be.sitting.\textsc{prs} \textsc{neg} \textsc{neg} be.lying.\textsc{prs} plow-\textsc{pst.pass-impf} \textsc{cl}\\
    \glt    `Because where Federico Díaz is (lives) it was barely plowed (the land).'\\
    \glt    `Porque donde está Federico Díaz casi no estaba barbechado (no estaba muy ancha la tierra).' {\corpuslink{in61[04_218-04_257].wav}{FLP in61:04:21.8}}\\
}
        \ex[]{
        {[SFH]: \textit{\textbf{pe ke biˈlé}!}}\\
        \gll    \textbf{pe} \textbf{ke} \textbf{biˈlé}!\\
                {just} {\textsc{neg}} {one}\\
        \glt    `Not at all!'\\
        \glt    `No, nada!'  {\corpuslink{in61[04_254-04_264].wav}{SFH in61:04:25.4}}\\
    }
    \z
\z

\ea\label{ex:14:emphatic negative interjection 2}
{Emphatic negative interjection}

    \ea[]{
    {[SFH]: \textit{ke biˈlé kiliˈsântemi ˈòuka riˈká}}\\
    \gll    ke biˈlé kiliˈsân=timi ˈòwi-ka riˈká\\
            \textsc{neg} one fertilizer=\textsc{2pl.nom} cure.fertilize-\textsc{ger} like\\
    \glt    `Didn't you all cure (fertilize) with fertilizer?'\\
    \glt    `¿No curaban (fertlizaban) con fertilizante?'  {\corpuslink{in484[01_500-01_515].wav}{SFH in484:01:50.0}}\\
}
        \ex[]{
        {[ME]: \textit{\textbf{ˈpé ke biˈlé! ˈpé ke biˈlé!} ˈpé kuˈríbi ko ˈhônsa niˈlú aˈlé}}\\
        \gll    \textbf{ˈpé} \textbf{ke} \textbf{biˈlé}! \textbf{ˈpé} \textbf{ke} \textbf{biˈlé}! ˈpé kuˈrí=bi=ko ˈhônsa niˈlú aˈlé\\
                {just} {\textsc{neg}} {one} {just} {\textsc{neg}} {one} just recently=just=\textsc{emph} since exist.\textsc{prs} \textsc{dub}\\
        \glt    `Not at all! Not at all! It just recently started (fertilizing crops)'\\
        \glt    `No, nada! No, nada! Hace poco que empezó (lo de fertilizar)'  {\corpuslink{in484[01_506-01_532].wav}{ME in484:01:50.6}}\\
    }
    \z
\z

\largerpage
The emphatic negative interjection \textit{ˈpé ke biˈlé} is not found with other constituents in clauses. The negative interjections \textit{ke} and \textit{ke ˈtâsi}, on the other hand, may also occur in phrases and clauses and are involved in clausal negation, as described next.

\subsection{Clausal negation}
\label{subsec: clausal negation}

Clausal negation in Choguita Rarámuri is achieved using the forms \textit{ke}, \textit{ke ˈtâsi}, \textit{ka tʃ͡è} and \textit{ke/ˈtâsi ˈtʃ͡ó}. Negative forms typically occur before the predicate, following a cross-linguistic trend noted in \citep[][105]{dryer2007clause}.\footnote{In clausal negation in closely related \ili{Mountain Guarijío}, \citet{miller1996guarijio} reports that the negative form \textit{kaʔí} appears in clause-initial position (\citeyear[119]{miller1996guarijio}).} Negative forms may also occur in clauses where the predicate has been elided. This is shown in (\ref{ex:14:clausal negationa}) and (\ref{ex:14:clausal negation}b--c), respectively.

\ea\label{ex:14:clausal negation}
{Clausal negation}

    \ea[]{
    \textbf{\textit{ke}} \textit{niˈlú?}\\
    \gll    \textbf{ke} niˈlú?\\
            {\textsc{neg}} \textsc{exist}\\
    \glt    ‘There wasn't any?’\\
    \glt    ‘¿No había?’ \corpuslink{in61[02_283-02_293].wav}{SFH in61:2:28.3}\\
}\label{ex:14:clausal negationa}
        \ex[]{
        {\textit{ˈkôt͡ʃini  buˈkê, toˈlí    ko} \textbf{\textit{ke}}}\\
        \gll    ˈkôt͡ʃi=ni buˈkê toˈlí=ko \textbf{ke}\\
                pigs=1\textsc{sg.nom} have.domesticated.animals.\textsc{prs} chicken=\textsc{emph} {\textsc{neg}}\\
        \glt    ‘I have pigs, chickens no (I don't).’ \\
        \glt    ‘Tengo cochis, pollos no.’ {< BFL 09 3:113/el >}\\
    }\label{ex:14:clausal negationb}
    %\break
            \ex[]{
            \textit{t͡ʃaˈbè  ko=ti \textbf{ke} \textbf{biˈlé} ˈníwi ˈlûsi ru, aʔˈlì \textbf{ke} \textbf{biˈlé} baʔˈwí}\\
            \gll    t͡ʃaˈbè  ko=ti \textbf{k}e \textbf{biˈlé} ˈníwi ˈlûsi ru aʔˈlì \textbf{ke} \textbf{biˈlé} baʔˈwí\\
                    before  \textsc{emph=1pl.nom} {\textsc{neg}} {one} have electricity say.\textsc{prs} and {\textsc{neg}} {one} water\\
            \glt    `Long ago we didn't have electricity and no water.'\\
            \glt    `Antes no teníamos luz ni agua.’ {< SFH 09 4:2/el >}\\
        }\label{ex:14:clausal negationc}
    \z
\z

Further examples of clausal negation are provided in (\ref{ex:14:clausal negation 2}).

\ea\label{ex:14:clausal negation 2}
{Clausal negation}

    \ea[]{
    {\textit{\textbf{ke ˈtâsi} ˈʃíli}}\\
    \gll    \textbf{ke} \textbf{ˈtâsi} ˈsí-li\\
            {\textsc{neg}} {\textsc{neg}} come.\textsc{pl}-\textsc{pst}\\
    \glt    `They didn't come.'\\
    \glt    `No vinieron.' {\corpuslink{co1136[03_120-03_133].wav}{MDH co1136:03:12.0}}\\
}\label{ex:14:clausal negation 2a}
        \ex[]{
        {\textit{\textbf{ˈká ˈtʃè} kaiˈnâma ˈlé ke naʔˈpôʃuwa ˈká ba}}\\
        \gll    \textbf{ˈká} \textbf{ˈtʃè} kaiˈnâ-ma aˈlé ke naʔˈpô-suwa ˈká ba\\
                {\textsc{neg}} {anymore} yield.harvest-\textsc{fut.sg} \textsc{dub} \textsc{neg} weed-\textsc{cond.pass} \textsc{cop.irr} \textsc{cl}\\
        \glt    `There won't be any (harvest) yield if there is no weeding.'\\
        \glt    `No se da (la cosecha) si no se escarda.'  {\corpuslink{co1136[04_395-04_418].wav}{MDH co1136:04:39.5}}\\
    }\label{ex:14:clausal negation 2b}
            \ex[]{
            {\textit{ˈkíti \textbf{ke tʃo} riˈhòoli ˈmá naˈʔî ba}}\\
            \gll    ˈkíti \textbf{ke} \textbf{tʃo} riˈhòo-li ˈmá naˈʔî ba \\
                    because {\textsc{neg}} {yet} inhabit.\textsc{pl}-\textsc{pst} already here \textsc{cl}\\
            \glt    `Because there was almost no people (living) here yet.' \\
            \glt    `Porque casi no había gente aqui todavía.' \corpuslink{tx817[00_369-00_429].wav}{JMF tx817:00:36.9}\\
        }\label{ex:14:clausal negation 2c}
                \ex[]{
                \textit{``\textbf{ka biˈlé tʃo} aˈwí ba" aˈnè}\\
                \gll    \textbf{ka} \textbf{biˈlé} \textbf{tʃo} aˈwí ba aˈn-è\\
                        {\textsc{neg}} {one} {yet} dance \textsc{cl} say.\textsc{imp.sg-appl}\\
                \glt    ` ``They haven't danced yet" you (should) tell them.'\\
                \glt    ` ``Todavía no bailan", diles.' {< JLG co1237[9\_156-9\_167] >}\\
            }\label{ex:14:clausal negation 2d}
    \z
\z

Clausal negation with \textit{ka tʃ͡è}, as in (\ref{ex:14:clausal negation 2b}), is attested in conditional sentences where \textit{ka tʃ͡è} appears in the main apodosis (consequence) clause, preceding the protasis clause. Other kinds of uses of \textit{ka tʃ͡è} in clausal negation are attested in the context of a sequence of clauses where the negative form involves a negative consequence (roughly translated here as `because not'). These uses are exemplified in (\ref{ex:14:ka che with because not readings}) and (\ref{ex:14:ka che with because not readings 2}).

\ea\label{ex:14:ka che with because not readings}
{Clausal negation in clause chaining contexts}

    \ea[]{
    {\textit{ˈpé ˈmá ˈhêm naˈwâsa taˈmí, aʔˈlì ˈnà ... aʔˈlì ˈpé ˈmá riˈká waʔˈlû buʔuˈrâaro ˈlá ˈnà}}\\
    \gll    ˈpé ˈmá ˈhêm naˈwâ-sa taˈmí, aʔˈlì ˈnà aʔˈlì ˈpé ˈmá riˈká waʔˈlû buʔu-ˈrâ-ro oˈlá ˈnà \\
            just already here arrive-\textsc{cond} 1\textsc{sg.acc} and  \textsc{dem} and just already like big road-\textsc{vblz-pst.pass} \textsc{cer} \textsc{dem}\\
    \glt    `Once I already arrived here then the big road was made.'\\
    \glt    `Ya cuando llegué yo aqui entonces hicieron el camino grande.' {\corpuslink{in485[06_456-06_496].wav}{ME in485:06:45.6}}\\
}
\newpage
        \ex[]{
        {\textit{\textbf{ˈká ˈtʃè} ko uluˈbê buʔuˈrúi tʃ͡aˈbèi ko}}\\
        \gll    \textbf{ˈká} \textbf{ˈtʃè}=ko waʔlu-ˈbê buʔu-ˈrú-i tʃ͡aˈbèi=ko\\
                {\textsc{neg}} {because}=\textsc{emph} big-\textsc{more} road-\textsc{vblz-impf} before=\textsc{emph}\\
        \glt    `Because there was no road before.'\\
        \glt    `Porque antes no había camino grande.' {\corpuslink{in485[06_506-06_534].wav}{ME in485:06:50.6}}\\
    }
    \z
\z

\ea\label{ex:14:ka che with because not readings 2}
{Clausal negation in clause chaining contexts}

    \ea[]{
    {\textit{ˈkíti ke ˈtʃ͡ó riˈhòoli ˈmá naˈʔî ba}}\\
    \gll    ˈkíti ke ˈtʃ͡ó riˈhò-li ˈmá naˈʔî ba \\
            because \textsc{neg} yet inhabit.\textsc{pl-pst} already here \textsc{cl}\\
    \glt    `Because there were almost no people here yet.'\\
    \glt    `Porque casi no había gente aquí.'  {\corpuslink{tx817[00_369-00_429].wav}{JMF tx817:00:36.9}}\\
}
        \ex[]{
        {\textit{\textbf{ˈká ˈtʃè} wiʰkaˈbê riˈhòoli ba oˈkwâ riˈhòoram ba ˈpé beˈsá maˈkòi riˈhòoli ˈlé}}\\
        \gll    \textbf{ˈká ˈtʃè} wiʰka-ˈbê riˈhò-li ba oˈkwâ riˈhò-r-ame ba ˈpé be-ˈsá maˈkòi riˈhò-li aˈlé\\
                {\textsc{neg} because} many-\textsc{more} inhabit.\textsc{pl-pst} \textsc{cl} few inhabit.\textsc{pl-pst-ptcp} \textsc{cl} just three-times ten inhabit.\textsc{pl-pst} \textsc{dub}\\
        \glt    `Because there were not that many people, (just) a few people, just about thirty people lived here, I think.'\\
        \glt    ‘Porque casi no había gente, había muy poquita como treinta yo creo.’ {\corpuslink{tx817[00_369-00_429].wav}{JMF tx817:00:36.9}}\\
    }
    \z
\z

As shown in the following example (\ref{ex:14:discontiguous negative forms}), the elements of morphologically complex negative markers need not be contiguous (negative forms are highlighted in boldface).

\ea\label{ex:14:discontiguous negative forms}
{Discontiguous negative forms}

    \textit{aʔˈlì ˈmá \textbf{ke} ˈnà ˈtòli \textbf{ˈtʃó} ˈét͡ʃi  riˈhò aliˈwâla ko}\\
    \gll    aʔˈlì ˈmá \textbf{ke} ˈnà ˈtò-li \textbf{ˈtʃó} ˈét͡ʃi  riˈhò aliˈwâ-la=ko\\
            and   already  {\textsc{neg}} that take-\textsc{pst} anymore \textsc{dem} man   soul-\textsc{poss}=\textsc{emph}\\
    \glt    `And then he didn't take the man's soul anymore.'\\
    \glt    `Y entonces ya no se llevó el alma del señor.' \corpuslink{tx5[01_387-01_434].wav}{LEL tx5:1:38.7}\\

\z

\subsection{Constituent negation}
\label{subsec: constituent negation}

Constituent negation in Choguita Rarámuri is encoded through a series of negative constructions, involving \textit{ke} `no', \textit{ke ˈtâsi biˈlé} `not one', \textit{ni biˈlé} `nor anything',\footnote{The form \textit{ni} is a borrowing from \ili{Spanish}, the negative conjunction.} \textit{ke ˈtâsi naˈmûti} `not a thing', and \textit{ke ˈtâsi ˈwêsi} `nobody'. These negative forms appear immediately preceding the modified constituent. Examples of constituent negation are provided in (\ref{ex:14:constituent negation examples}).

\ea\label{ex:14:constituent negation examples}
{Constituent negation}\\

        {\textit{``\textbf{ˈká ˈtʃè biˈlé} kaˈsè maˈtʃ͡íni ba ˈni"}}\\
        \gll    \textbf{ˈká} \textbf{ˈtʃè} \textbf{biˈlé} kaˈsè maˈtʃ͡íni ba ˈni\\
                {\textsc{neg}} {anymore} {one} place come.out.\textsc{sg} \textsc{cl} ni\\
        \glt    ```It's not coming out in any place."'\\
        \glt    ```No sale en ninguna parte."' \corpuslink{tx152[06_375-06_426].wav}{SFH tx152:6:37.5}\\

\z

%\textbf{Examples of negative clauses} \textbf{< BFL 09 1:45]}
As described above, Choguita Rarámuri also expresses negation through two negative quantifiers: \textit{(ke) ˈtâsi (biˈlé) naˈmûti} `none, nothing' (where the noun \textit{naˈmûti} means `thing') and \textit{(ke) ˈtâsi ˈwêsi} `nobody' (where \textit{ˈwêsi} means `nobody, no one', as it is not attested outside of negative contexts). These negative quantifiers are exemplified in (\ref{ex:14:negative quanitifers}).

\ea\label{ex:14:negative quanitifers}
{Negative quantifiers}\\

    \ea[]{
        \textit{aʔˈlì \textbf{ke biˈlé naˈmûti} reˈwáli hiˈjéa}\\
        \gll    aʔˈlì \textbf{ke} \textbf{biˈlé} \textbf{naˈmûti} reˈwá-li hiˈjé-a\\
                and {\textsc{neg}} {one} {thing} see-\textsc{pst} find.trace-\textsc{prog}\\
        \glt    `And they found no trace.'\\
        \glt    `Y no hallaron ninguna huella.' \corpuslink{tx_mawiya[02_099-02_132].wav}{LEL tx\_mawiya:2:09.9}\\
}
            \ex[]{
            \textit{\textbf{ke biˈlé naˈmûti }ˈníuka moˈtʃ͡íli ˈnà riˈpá riʔˈlé}\\
            \gll    \textbf{ke} \textbf{biˈlé} \textbf{naˈmûti} ˈníu-ka moˈtʃ͡í-li ˈnà riˈpá reʔˈlé\\
                   {\textsc{neg}} {one} {thing} have-\textsc{ger} inhabit.\textsc{pl-pst} \textsc{dem} above down\\
            \glt    `They didn't have anything down there where they lived.'\\
            \glt    `Que no tenían nada allá abajo en donde vivían.' {\corpuslink{tx109[02_481-02_503].wav}{LEL tx109:2:48.1}}\\
        }
                \ex[]{
                \textit{aʔˈlì ˈmò ro? \textbf{ke ˈwêsi} beˈnèri ko? }\\
                \gll    aʔˈlì ˈmò ro \textbf{ke} \textbf{ˈwêsi} beˈnè-ri=ko\\
                        and 2\textsc{sg.nom} and {\textsc{neg}} {nobody} learn-\textsc{caus}=\textsc{emph}\\
                \glt    `And you? Don't you teach anybody?'\\
                \glt    `¿Y tu? ¿No le enseñas a nadie?' \corpuslink{in243[11_277-11_308].wav}{SFH in243:11:27.7}\\
            }
    \z
\z

In cases where there is both clausal negation and constituent negation using negative quantifiers, the clausal negation form appears in clause initial position, preceding the negative quantifiers. This is shown in (\ref{ex:14:clausal and constituent negation 2}). The negative particles and quantifiers are highlighted in boldface.

\ea\label{ex:14:clausal and constituent negation 2}

\textit{\textbf{ˈtâsi}=ni    \textbf{biˈlé  ˈwêsi}    ˈnè-nali}\\
\gll    \textbf{ˈtâsi}=ni    \textbf{biˈlé}  \textbf{ˈwêsi}    iʔˈnè-nale\\
        \textsc{{neg}=1sg.nom} {one}  {nobody}  see-\textsc{desid}\\
\glt    `I don’t want to see anybody.'\\
\glt    `No quiero ver a nadie.'  \\

\z

In this particular example, the 1st person singular subject is encoded through the enclitic \textit{=ni}, which attaches after the first constituent, following the pattern for Wackernagel position clitics.

Finally, there are also cases where complex negative expressions are discontinuous in clauses with constituent negation. This is shown in (\ref{ex:14:discontinuous negatives in constituent negation}).

\ea\label{ex:14:discontinuous negatives in constituent negation}
{Discontinuous negative forms in constituent negation}

        \textit{ˈpé ˈwé kaʔˈlá kaˈjèni kiˈʔàa ko ... ˈwé aʔˈlá \textbf{ke} bi ko \textbf{biˈlé} witaˈt͡ʃí naˈkí ba}\\
        \gll    ˈpé ˈwé kaʔˈlá kaˈjèni kiˈʔà=ko ˈwé aʔˈlá \textbf{ke}=bi=ko \textbf{biˈlé} witaˈt͡ʃí naˈkí  ba\\
                just \textsc{int} well harvest.\textsc{prs} before=\textsc{emph} \textsc{int}  well {\textsc{neg}}=just=\textsc{emph} {one} fertilizer need \textsc{cl}  \\
        \glt    `Crops would turn out well, there wasn’t any need for fertilizer.'\\
        \glt    `Se daba muy bien antes la cosecha, no se necesitaba abono.'  \corpuslink{in61[02_468-02_488].wav}{FLP in61:02:46.8 0:02.0}, \corpuslink{in61[02_498-02_522].wav}{FLP in61:2:49.8}\\

\z

While discontiguous, the complex negative marker still precedes the modified constituent (the noun \textit{witaˈt͡ʃí} `fertilizer').

\subsection{Negative existential and locative clauses}
\label{subsec: negative existence and legative locative clauses}

Existential negation is encoded through a dedicated negative predicate of existence used in conjunction with the negative particle \textit{ke}, which contrasts with the positive polarity postural predicates deployed in locative clauses (see §\ref{subsec: locative clauses} and §\ref{subsec: predicates of possession}). Examples of negative existential predication are provided in (\ref{ex:14:negative existential predication}).

\ea\label{ex:14:negative existential predication}
{Negative existential predication}

    \ea[]{
        {\textit{ˈmá \textbf{ke iˈtê} ku, ˈmá oˈkwâ biˈtí ko}}\\
        \gll    ˈmá \textbf{ke} \textbf{iˈtê} ku ˈmá oˈkwâ biˈtí=ko\\
                anymore {\textsc{neg}} {be.\textsc{neg}} wood already two be.lying.\textsc{prs}=\textsc{emph}\\
        \glt    `There is no more wood, there is only a little left.'\\
        \glt    `Ya no hay leña, ya nomás quedan unos pocos.' {\corpuslink{co1136[01_099-01_123].wav}{MDH co1136:01:09.9}}\\
    }
            \ex[]{
            \textit{ke biˈlé ˈpé \textbf{ˈtâʃi iˈtêeli} ˈònam tʃ͡aˈbè ko?}\\
            \gll    ke biˈlé ˈpé \textbf{ˈtâsi} \textbf{iˈtê-li} ˈòn-ame tʃ͡aˈbè=ko\\
                    \textsc{neg} one just {\textsc{neg}} {be.\textsc{neg-pst}} cure-\textsc{ptcp} before=\textsc{emph}\\
            \glt    `There was no doctor before?'\\
            \glt    `¿No había doctor antes?' {\corpuslink{in61[06_396-06_420].wav}{SFH in61:06:39.6}}\\
        }
                \ex[]{
                \textit{ˈmá \textbf{ˈkátʃi iˈtêli} baʔaˈrîna ma nataˈkêa buˈʔíli ˈétʃ͡i reˈhòi}\\
                \gll    ˈmá \textbf{ˈká ˈtʃè} \textbf{iˈtê-li} baʔaˈrî-na ma nataˈkê-a buˈʔí-li ˈétʃ͡i reˈhòi\\
                        already {\textsc{neg} \textsc{neg}} {be.\textsc{neg-pst}} tomorrow-at already faint-\textsc{prog} lie.down.\textsc{sg-pst} \textsc{dem} man\\
                \glt    `He wasn't there anymore the next day, he already lay fainted that man.’\\
                \glt    `Ya no estaba al otro día, ya estaba desmayado ese señor.' {\corpuslink{tx5[02_331-02_357].wav}{LEL tx5:02:33.1}}\\
            }
    \z
\z

Negative locative predication employs the same negative morphemes. This is shown in (\ref{ex:14:postural verbs in negative clauses}).

\ea\label{ex:14:postural verbs in negative clauses}
{Negative existence markers \textit{ke iˈtê} in negative locative clauses}\\

    \textit{naˈʔî  \textbf{ke iˈtê} ˈbôte}\\
    \gll    naˈʔî  \textbf{ke} \textbf{iˈtê} ˈbôte\\
            here  {\textsc{neg}} {be.\textsc{neg}} can\\
    \glt    ‘The can is not here.’\\
    \glt    ‘Aqui no está el bote.’ < LEL 09 1:74/el >\\

\z

\section{Imperatives}
\label{sec: imperative constructions}

Imperatives are defined as constructions that encode directive speech acts (including orders, requests, warnings, invitations, etc.), usually directed at addres\-sees (second persons) \citet{konig2007speech}; in some cases the term ``imperative'' is extended to constructions where commands, requests, etc. are addressed to the first or third person. This section addresses the multiple constructions that encode directive speech acts in Choguita Rarámuri, including positive imperatives, prohibitives, hortatives and motion imperatives.

%[will need to edit here the tonal effects of the imperative and the interaction with lexical tone, unless this is addressed in another chapter - Prosody]

\subsection{Positive imperative}
\label{subsec: positive imperatives}

Imperatives in Choguita Rarámuri are headed by an imperative-marked verb (for a description of imperative morphology, see \chapref{chap: verbal morphology}). Imperative suffixes encode subject number distinctions, with a set of concatenative and non-concatena\-tive allomorphs encoding a single addressee and a suffix encoding that the directive speech act is addressed to a group.

As discussed in §\ref{subsec: tone as realizational morphology}, there are four allomorphs of the imperative singular: the stress-shifting \textit{-kâ} suffix (\ref{ex:14:singular imperative allomorphsa}), the stress-shifting \textit{-sâ} suffix (\ref{ex:14:singular imperative allomorphs}b--c), a rightward stress shift (\ref{ex:14:singular imperative allomorphsd}) and a L tonal exponent (\ref{ex:14:singular imperative allomorphsf}).

\ea\label{ex:14:singular imperative allomorphs}
{Singular addressee imperatives}

    \ea[]{
    \textit{ˈwé saˈpù aˈsíska! ˈpîri t͡ʃuˈkú naˈʔî?}\\
    \gll    ˈwé  saˈpù aˈsí-si-ka ˈpîri t͡ʃuˈkú naʔî\\
            \textsc{int} hurry sit.up-\textsc{mot-imp.sg} what be.bent here\\
    \glt    `Hurry, get up! What's here? (lit. What sits in four legs here?)’\\
    \glt    `¡Levántate pronto! ¿Qué hay (lit. está sentado en cuatro patas) aquí?’ \corpuslink{tx5[00_527-00_572].wav}{LEL tx5:0:52.7}\\
}\label{ex:14:singular imperative allomorphsa}
        \ex[]{
        \textit{koˈsâ!} \\
        \gll    ko-ˈsâ!    \\
                eat\textsc{\textsc{.imp.sg}}\\
        \glt    `Eat!'   \\
        \glt    `¡Come!'\\
    }\label{ex:14:singular imperative allomorphsb}
            \ex[]{
            \textit{ˈmàsa}\\
            \gll    ˈmà-sa!\\
                    run-\textsc{imp.sg}\\
            \glt    `Run!'\\
            \glt    `¡Corre!' {< BFL 04/11/06/el >}\\
        }\label{ex:14:singular imperative allomorphsc}
                \ex[]{
                {\textit{naʔsoˈwâ!}}\\
                \gll    naʔsoˈwâ\\
                        stir.\textsc{imp.sg}\\
                \glt    `Stir it!'\\
                \glt    `¡Revuélvelo! < BFL el1957 >\\
            }\label{ex:14:singular imperative allomorphsd}
                \ex[]{
                \textit{cf. naʔˈsòwa}\\
                `S/he stirs it.'\\
                `Lo revuelve.'\\
            }\label{ex:14:singular imperative allomorphse}
                    \ex[]{
                    {\textit{hiˈràa!}}\\
                    \gll    hiˈrà\\
                            bet.\textsc{imp.sg}\\
                    \glt    `Bet!'\\
                    \glt    `¡Apuesta!' {< SFH el 1925 >}\\
                }\label{ex:14:singular imperative allomorphsf}
                        \ex[]{
                        \textit{cf. hiˈrâ}\\
                        {`S/he bet.'}\\
                        `Apuesta.'\\
                    }\label{ex:14:singular imperative allomorphsg}
    \z
\z

The stress-shift imperative allomorph is phonologically-conditioned, and is selected by unstressed trisyllabic roots (e.g., \textit{naʔ'sòwa} `to stir' in (\ref{ex:14:singular imperative allomorphsd})). The L tone allomorph of the imperative singular, on the other hand, is realized on the stressed syllable of HL-toned stems (e.g., \textit{hiˈrâ} `to bet' in (\ref{ex:14:singular imperative allomorphsf})) (see §\ref{subsec: grammatical tone} in \chapref{chap: prosody}). The distribution of the allomorphs is otherwise lexically determined, though there are stems where the same lexical root may be attested with more than one allomorph in an apparent case of free variation (i.e., with no apparent semantic differences) and without any multiple exponence (i.e., one or another allomorph will surface, but not both at the same time), as shown in (\ref{ex:14:marginal cases of free variation}) and (\ref{ex:14:marginal cases of free variation 2}). This availability of several imperative morphemes may be related to strategies to attenuate directive speech acts. Such distinctions are not apparent in the available corpus of the language, though a larger corpus with greater contextualization of social contexts may reveal differences in these terms for the different imperative devices available.\footnote{This seconds a suggestion made in \citep[][107]{miller1996guarijio} about potential distinctions between imperative constructions in closely-related \ili{Mountain Guarijío}, where several imperative suffixes appear to be in free variation.}

%\pagebreak
\largerpage

\ea\label{ex:14:marginal cases of free variation}
{Free variation in imperative allomorph selection}

    \ea[]{
    {\textit{niʔˈkîka!}\\
    \gll    niʔˈkî-ka\\
            bite.\textsc{sg}-\textsc{imp.sg}\\
    \glt    `Bite it!'\\
    \glt    `¡Muérdelo!'{< BFL 2014:65 >}\\
    }
}
        \ex[]{
        {\textit{niʔˈkìí!}}\\
        \gll    niʔˈkì\\
                bite.\textsc{sg.imp.sg}\\
        \glt    `Bite it!'\\
        \glt    `¡Muérdelo!'{< BFL 2014:65 >}\\
    }
    \z
\z

\ea\label{ex:14:marginal cases of free variation 2}
{Free variation in imperative allomorph selection}

    \ea[]{
    \textit{tòˈkâ}\\
    \gll    tòˈ-kâ\\
            take-\textsc{imp.sg}\\
    \glt    `Take it!'\\
    \glt     `¡Llévatelo!' {< BFL el1882 >}\\
}
        \ex[]{
        \textit{tòˈsâ}\\
        \gll    tò-ˈsâ\\
                take-\textsc{imp.sg}\\
        \glt    `Take it!'\\
        \glt    `¡Llévatelo! {< BFL el1882 >}\\
    }
            \ex[]{
            \textit{ˈtòo}\\
            \gll    ˈtòo\\
                    take.\textsc{imp.sg}\\
            \glt    `Take it!'\\
            \glt    `¡Llévatelo!' {< BFL el1882 >}\\
        }
    \z
\z

As shown in these examples, variation in allomorph selection may involve a choice between two allomorphs (a suffix allomorph and a tonal allomrph in (\ref{ex:14:marginal cases of free variation})), or it may involve three morphological marking options (two suffixal allomorphs and the tonal allomorph in (\ref{ex:14:marginal cases of free variation 2})). In the latter example, the L tone allomorph of the imperative singular vacuously applies in the case of a L-toned stem like \textit{tò} `to take'.

In contrast to the imperative singular, the imperative plural involves no allomorphy, but a single productive suffix (stress-shifting \textit{-sì}), exemplified in (\ref{ex:14:plural imperative}).

\ea\label{ex:14:plural imperative}
{Plural addressee imperative}

    \ea[]{
    {\textit{aʔˈlì ˈmá ˈhê aniˈmêa Lola ``ku baˈhîsi ne ko!" aniˈmêa}}\\
    \gll    aʔˈlì ˈmá ˈhê ani-ˈmêa Lola ``ku baˈhî\textbf{-si} ne=ko!" ani-ˈmêa\\
            and already it say-\textsc{fut.sg} Lola \textsc{rev} drink-{\textsc{imp.pl}} \textsc{exh=emph} say-\textsc{fut.sg}\\
    \glt    `And Lola will promptly say: ``Go on, drink up!", she will say."\\
    \glt    `Y ya va a decir Lola: ``¡Ya tómenle pues!", va a decir.'' \corpuslink{co1234[18_282-18_308].wav}{JLG co1234:18:28.2}\\
}\label{ex:14:plural imperativea}
        \ex[]{
        {\textit{aʔˈlì ˈmá wiˈrôsa ko ˈhê aniˈmêa ``ja baˈhîsi ba!" ˈhê aniˈmêa}}\\
        \gll    aʔˈlì ˈmá wiˈrô-sa=ko ˈhê ani-ˈmêa ``ja baˈhî\textbf{-si} ba!" ˈhê ani-ˈmêa\\
                and already throw.up.in.air-\textsc{cond}=\textsc{emph} it say-\textsc{fut.sg} already drink-{\textsc{imp.pl}} \textsc{cl} it say-\textsc{fut.sg}\\
        \glt    `And when they throw it up in the air (the corn beer), they say ``You all drink already!", that's how it is said.'\\
        \glt    `Y ya cuando lo tiran (el tesgüino) para arriba, se dice ``¡Ya tomen!", así se dice.' \corpuslink{co1234[12_522-12_550].wav}{JLG co1234:12:52.2}\\
    }\label{ex:14:plural imperativeb}
            \ex[]{
            {\textit{bueno ko ... kaʔˈlá ˈnè ko iˈwêra ku aʔˈpésa siˈmási ne koʔˈwámi aʔˈlá biˈlátimi baˈnèrlila ro ˈpé ˈkútʃ͡i riˈkáatʃ͡i ko ba ne, ˈmá biˈlátimi ˈsèbili}}\\
            \gll    bueno=ko kaʔˈlá ˈnè=ko iˈwê-ra ku aʔˈpé-sa siˈmá-\textbf{si} ne koʔ-ˈwá-ame aʔˈlá biˈlá=timi maˈn-è-r-li-la ˈru ˈpé ˈkútʃ͡i riˈká=tʃ͡i=ko ba ne ˈmá biˈlá=timi ˈsèbi-li\\
                    well=\textsc{emph} good \textsc{exh=emph} be.strong-\textsc{nom} \textsc{rev} take-\textsc{cond} go.\textsc{pl-imp.pl} \textsc{int} eat-\textsc{mpass-partc} good well=\textsc{2pl.nom} be.located.container-\textsc{appl-caus-caus-rep} say.\textsc{prs} just little like.that=\textsc{dem=emph} \textsc{cl} \textsc{int} already like=\textsc{2pl.nom} be.enough-\textsc{pst}\\
            \glt    `Well ... go on in peace, take the food, good thing is that you were given a little, it was so so enough for you all ...'\\
            \glt    `Bueno ... váyanse agusto, llévense la comida, lo bueno es que lo pusieron de a poquito, más o menos les alcanza a ustedes ...' \corpuslink{tx1131[00_096-00_172].wav}{MFH tx1131:00:09.6}\\
        }\label{ex:14:plural imperativec}
    %\pagebreak
                \ex[]{
                \textit{ku iˈnârsi kiˈrì aˈwênili a wikaˈbêbi ˈtʃ͡étimi ku ˈsíli ˈtʃ͡ó aˈlé}\\
                \gll    ku iˈnâri-\textbf{si} kiˈrì aˈwênili a wika-ˈbê=bi ˈtʃ͡é=timi ku ˈsíli ˈtʃ͡ó aˈlé\\
                        \textsc{rev} go.\textsc{imp.pl} peacefully alone.\textsc{pl} \textsc{aff} far-more=just also=\textsc{1pl.nom} \textsc{rev} go.\textsc{pl.pl} also \textsc{dub}\\
                \glt    `Go back in peace on your own, I think you may go with others.'\\
                \glt    `Regrésense tranquilos solos, yo creo que a lo mejor van entre muchos.’ \corpuslink{tx1133[02_305-02_341].wav}{MFH tx1133:02:30.5}\\
                }\label{ex:14:plural imperatived}
    \z
\z

As described above, the plural imperative is used when the speaker is addressing a group with a directive command. Crucially, the group excludes the speaker, as directive speech forms including the speaker deploy a different morphological construction, namely the hortative construction, described below in §\ref{subsec: hortative}.\footnote{\citet[][110]{miller1996guarijio} describes that closely related \ili{Mountain Guarijío} does not make a distinction between single and multiple addresses in imperative morphology, but the future plural suffix (\textit{-po/bo}, cognate with the Choguita Rarámuri future plural \textit{-pô} suffix) is used in conjunction with imperative suffixes to attenuate the force of the  directive command. There are no equivalent forms in the Choguita Rarámuri corpus.}

A special form is used in positive imperatives, where the speaker conveys encouragement, possibly as a politeness strategy, in the directive speech. This form, involving the imperative-inflected verb plus an exhortative particle \textit{ne} and the emphatic enclitic \textit{ko}, is attested in a variety of contexts involving a polite encouragement. This is exemplified in the commonplace expression in (\ref{ex:14:plural imperativea}) and (\ref{ex:14:plural imperativeb}) above and (\ref{ex:14:ne ko construction}).

\ea\label{ex:14:ne ko construction}

    \textit{ja koˈʔá ne ko!}\\
    \gll    ja koˈʔá ne=ko!\\
            already eat.\textsc{imp.sg} \textsc{exh=emph}\\
    \glt    `Go on, eat!'\\
    \glt    `¡Come, pues!'\\

\z

This construction is also deployed in affirmative answers to polite directive speech acts, as exemplified in another commonplace type of exchange (\ref{ex:14:ne ko in answers to directive speech actsb}).

\ea\label{ex:14:ne ko in answers to directive speech acts}

    \ea[]{
    A: \textit{ja baˈhîsi kaˈhêke!}\\
    \gll    ja baˈhî-si kaˈhêke!\\
            already drink-\textsc{imp.pl} coffee\\
    \glt    `You all drink coffee!'\\
    \glt    `¡Ya tomen café!'\\
}\label{ex:14:ne ko in answers to directive speech actsa}
        \ex[]{
        B: \textit{hoʔu ne ko!}\\
        \gll    hoʔu ne=ko!\\
                yes \textsc{exh=emph}\\
        \glt    `All right!'\\
        \glt    `¡Órale pues!'\\
    }\label{ex:14:ne ko in answers to directive speech actsb}
    \z
\z

\subsection{Prohibitive}
\label{subsec: prohibitives}

Negative imperatives or prohibitives in Choguita Rarámuri are encoded through a construction involving a dedicated prohibitive morpheme and an imperative-marked verb. The imperative morphology in the inflected verb is the same as in positive imperative clauses (see §\ref{subsec: positive imperatives}). Prohibitive clauses contain the prohibitive marker \textit{ˈkíti} in clause initial position. This is shown in (\ref{ex:14:kiti clauses}).

\ea\label{ex:14:kiti clauses}
{Prohibitive clauses}

    \ea[]{
    \textit{ˈkíti naˈlàka!}\\
    \gll    \textbf{\textit{ˈkíti}} \textit{naˈlà-ka}\\
            {\textsc{proh}}   cry-\textsc{imp.sg}\\
    \glt    `Don’t cry!’ \\
    \glt    `¡No llores!’  <BFL 05 2:89/el > \\
}
        \ex[]{
        \textbf{\textit{ˈkíti} } \textit{riʔˈná    bitiˈʃí!}\\
        \gll    \textbf{ˈkíti}  riʔˈná    biti-ˈsí\\
                {\textsc{proh}}   on.back   lie.down.\textsc{pl}{}-\textsc{imp.pl}\\
        \glt    ```Don’t sleep on your back!”’\\
        \glt    ```No se acuesten boca arriba!” \corpuslink{tx5[06_001-06_037].wav}{LEL tx5:06:00.1} \\
    }
            \ex[]{
            \textit{``ˈkíti ˈpáka loˈkási!"}\\
            \gll    ˈkíti ˈpá-ka loˈká-si!"\\
                    {\textsc{proh}} throw-\textsc{ger} drink.pinole-\textsc{imp.pl}\\
            \glt    ``Don't drink pinole spilling it!'\\
            \glt    ``¡No tomen pinole tirándolo!" \corpuslink{co1136[10_259-10_272].wav}{MDH co1136:10:25.9}\\
        }
    \z
\z

The marker \textit{ˈkíti} is also used in subordinate clauses introducing reason clauses, as exemplified in (\ref{ex:14:kiti in subordination}) (see \chapref{chap: clause combining in complex sentences}, §\ref{subsec: reason clauses}).

\ea\label{ex:14:kiti in subordination}

    \ea[]{
    \textit{ke biˈlé aˈwìo ˈlá, \textbf{ˈkíti} ke uˈkú aˈlé ko}\\
    \gll    ke biˈlé aˈwì oˈlá, \textbf{ˈkíti} ke uˈkú aˈlé=ko\\
            \textsc{neg} one dance \textsc{cer} \textsc{sub} \textsc{neg} rain.\textsc{prs} \textsc{dub=emph}\\
    \glt    `They don't dance (ritually), that is why it doesn't rain'\\
    \glt    `Pues no bailan, por eso no llueve' \corpuslink{co1136[06_231-06_258].wav}{MDH co1136:06:23.1}\\
}\label{ex:14:kiti in subordinationa}
        \ex[]{
        \textit{aʔˈlìmi noˈkèema aʔˈlá ˈnàri \textbf{ˈkíti} ko ke muˈjâma}\\
        \gll    aʔˈlì=mi noˈk-è-ma aʔˈlá ˈnàri \textbf{ˈkíti}=ko ke muˈjâ-ma\\
                and=2\textsc{sg.nom} move-\textsc{appl-fut.sg} well like.that \textsc{{sub}=emph} \textsc{neg} rot-\textsc{fut.sg}\\
        \glt    `And you will move it well (often) so that it won't rot.'\\
        \glt    `Y lo vas a mover bien (cada rato) para que no se pudra.' \corpuslink{tx60[00_431-00_474].wav}{BFL tx60:00:43.1}\\
    }\label{ex:14:kiti in subordinationb}
            \ex[]{
            \textit{ˈpé aˈwêni be ko riˈhòwili tʃ͡aˈbè \textbf{ˈkíti} ˈpé koˈlì tiˈjôpatʃ͡i ˈníla ˈra siˈnêwi ko ba}\\
            \gll    ˈpé aˈwêni be=ko riˈhòwi-li tʃ͡aˈbè [\textbf{ˈkíti} ˈpé koˈlì tiˈjôpatʃ͡i ˈní-la ˈra siˈnêwi=ko ba]\\
                    just alone.\textsc{pl} just=\textsc{emph} inhabit.\textsc{pl}-\textsc{pst} before {\textsc{sub}} just side church \textsc{cop-rep} say.\textsc{mpass} first.time=\textsc{emph} \textsc{cl}\\
            \glt    `There were just a few people living (here) before, that is why it is said it was over there by the church the first time’\\
            \glt    `Vivían poquitos antes, por eso dicen que fue allá por aquel lado de la iglesia la primera vez' \corpuslink{tx12[00_520-00_565].wav}{SFH tx12:00:52.0}\\
        }\label{ex:14:kiti in subordinationc}
    \z
\z

As shown in these examples, the clause introduced by \textit{ˈkíti} may be either positive (\ref{ex:14:kiti in subordination}a--b) or negative (\ref{ex:14:kiti in subordinationc}).

\subsection{Exhortative}
\label{subsec: hortative}

Imperative constructions in Choguita Rarámuri with an exhortative meaning are marked with the clause initial particle \textit{to} and may optionally have a closing exhortative particle \textit{bo}. This construction is exemplified in (\ref{ex:14:exhortative examples}).

\ea\label{ex:14:exhortative examples}

    \ea[]{
    \textit{ˈtoo} \textit{ˈjêni   ˈdûlse   ˈìwkitili jaˈdîra}\\
    \gll    \textbf{ˈto}  ˈjêni  ˈdûlse  ˈìwi-ki-ti-li  jaˈdîra\\
            {\textsc{exh}} Yéni  candy  bring.\textsc{appl-appl-caus-imp.sg} Yadira\\
    \glt    ‘Go on, make Yeni bring candy for Yadira!’\\
    \glt    ‘¡Ve, haz que Yeni le traiga dulces a Yadira!’   < BFL 07 1:62/el >\\
}
        \ex[]{
        \textit{ˈtoo} \textit{miˈtʃíktili} \textbf{\textit{bo}}\\
        \gll    \textbf{ˈto}  miˈtʃí-ki-ti-li     \textbf{bo}\\
                {\textsc{exh}}  carve-\textsc{appl-caus-imp.sg} {\textsc{exh}}\\
        \glt    `Go on, carve it for him!’\\
        \glt    `¡Anda, lábraselo!’ < BFL 08 1:107/el >\\
    }
            \ex[]{
            \textit{niˈbi!} \textit{ˈtoo} \textit{iˈʔné} \textit{bo!}\\
            \gll    niˈbi \textbf{ˈto} iˈʔné \textbf{bo}\\
                    look  {\textsc{exh}} see.\textsc{imp.sg} {\textsc{exh}}\\
            \glt    `Look! Take a look at it!’\\
            \glt    `¡Mira! ¡Míralo!’ < BFL 06 Nov04/el >\\
        }
    \z
\z

In these examples, the speaker encourages the addressee to carry out an activity in a polite fashion. These types of polite requests are frequent in every day conversation.

%\textbf{Falta: descripción sintáctica de las construcciones imperativas}
%\textbf{Atenuación de las construcciones imperativas? Cortesía?}

\subsection{Motion Imperatives}
\label{subsec: motion imperatives}

One type of imperative construction in Choguita Rarámuri involves suffixes that, alone or in combination with imperative mood suffixes, encode a motion associated with the main verb of the event plus a command (`Go and do X!'), directed at a single or plural addressee. The stress-shifting suffix \textit{-me} is used for single addressees. When there are multiple adressees, the Motion Imperative involves the stress-shifting suffix \textit{-pi} (with stress-shifting allomorph \textit{-bô}, followed by the imperative plural suffix. Examples of the motion imperative construction are provided in (\ref{ex:14:motion imperative sentences}).

\newpage
\ea\label{ex:14:motion imperative sentences}

    \ea[]{
    \textit{ˈétʃ͡i   ˈhùrimi na!}\\
    \gll    ˈétʃ͡i   ˈhùri-\textbf{mi} na\\
            \textsc{dem} send-{\textsc{mot.imp}} there\\
    \glt    `Go and send (the fire bird) to him!’\\
    \glt    `¡Ve y mándale (el pájaro de fuego) a ese!’ < LEL 06 tx5(80)/tx >\\
}
        \ex[]{
        \textit{taˈmí   ku ˈákipisi}\\
        \gll    taˈmí    ku ˈá-ki-\textbf{pi-si}\\
                1\textsc{sg.acc} \textsc{rev} look.for-\textsc{appl-{mot.imp.pl-imp.pl}}\\
        \glt    `You all go and look for it for me!’\\
        \glt    `Vayan a buscármelo!’   < BFL 08 1:164/el > \\
    }
    \z
\z

As described in \chapref{chap: verbal morphology}, there is an associated motion marker used in declarative sentences (for discussion about associated motion in cross-linguistic perspective, see \citealt{guillaume2016associated}).

% add more examples
% describe examples
%expand on the propteries of this construction

\section{Comparatives}
\label{sec:14:comparative constructions}

The final type of specialized sentence type addressed in this chapter is comparative constructions. Across languages, comparative constructions involve a predicate that encodes a predicative scale, and two noun phrases, one which encodes the object of comparison (the Comparee NP), and another which encodes the standard of comparison (the Standard NP) (for an overview of the typological parameters of variation of comparative constructions, see \citealt{stassen1985comparison} and \citealt{haspelmath2017equative}). The Choguita Rarámuri comparative construction involves a gradable predicate expressed through an adjective, a deverbal nominalized form or a verbal predicate. If the gradable predicate is encoded through an adjective or a nominalized form, the comparative also features an associated copular verb. Degree is optionally expressed through an adverb or a comparative suffix in the predicate. The object of comparison is a nominal phrase optionally followed by an emphatic particle. Finally, the standard of comparison is introduced through the postpositions \textit{ˈ(j)ua} ‘with’ or \textit{ˈkítara} ‘about’. The schema in (\ref{ex:14:order of elements in comparative constructions}) shows the structure of Choguita Rarámuri comparative constructions.

\ea\label{ex:14:order of elements in comparative constructions}
{Structure of comparative constructions}

[Comparee NP] [(Adverb)] [Gradable predicate] [(Copula)] [Standard NP] [Postposition] \\

\z

The examples in (\ref{ex:14:comparative examples}) illustrate the structure of comparative constructions.

\ea\label{ex:14:comparative examples}

    \ea[]{
    %\textbf{obj. of comp.              adv.    grad. predicate} \textbf{copula  standard of comp.  postposition}\\
    \textit{siˈkâ  watoˈná tʃ͡uˈkúam     ˈká  ˈwé  raˈsíra ˈwáami  ˈhú kohiˈná ˈjûa}\\
    \gll    siˈkâ  watoˈná tʃ͡uˈkú-ame     ˈká  ˈwé raˈsíra ˈwá-ame ˈhú kohiˈná ˈjûa\\
            arm.hand  right    be.bent-\textsc{ptcp}  \textsc{cop.irr} \textsc{int} more be.strong-\textsc{ptcp} \textsc{cop.prs}  left \textsc{than}\\
    \glt    `The right arm/hand if flexed is stronger than the left one.’\\
    \glt    `El brazo derecho si está flexionado es más fuerte que el izquierdo.’ < BFL 09 1:30/el > \\
}\label{ex:14:comparative examplesa}
        \ex[]{
         %\textbf{obj. of comp.    grad. predicate  copula  standard of comp.  postposition}\\
        \textit{ˈmò ko wiʔlí ˈhú taˈmí ˈjûa}\\
        \gll    ˈmò=ko wiʔˈlí ˈhú taˈmí ˈjûa\\
                \textsc{2sg.nom=emph} tall \textsc{cop.prs} \textsc{1sg.acc} \textsc{than}\\
        \glt    `You are taller than me.’\\
        \glt    `Tu estás más alto que yo.’ < LEL 09 2:6/el >\\
    }\label{ex:14:comparative examplesb}
            \ex[]{
            %\textbf{obj. of comp.    grad. predicate  copula  standard of comp.  postposition}\\
            \textit{ˈmò ko ˈtéeri ˈhú taˈmí ˈjûa}\\
            \gll    ˈmò=ko    ˈtéri      ˈhú    taˈmí        ˈjûa\\
                    \textsc{2sg.nom=emph} tall  \textsc{cop.prs} \textsc{1sg.acc} \textsc{than}\\
            \glt    `You are shorter than me.’\\
            \glt    `Tu estás más chaparro que yo.’ < GFM 09 3:101/el >\\
        }\label{ex:14:comparative examplesc}
    \z
\z

All three examples in (\ref{ex:14:comparative examples}) involve an adjective or nominalized gradable predicate and the associated present tense copula \textit{hu} `is'. In (\ref{ex:14:comparative examplesa}), the object of comparison noun phrase includes a relative clause, while the gradable predicate, the nominalized form \textit{ˈwáami} `strong', is modified by two degree adverbs (\textit{we} `very' and \textit{raˈsíram} `more'). In all of the examples in (\ref{ex:14:comparative examples}), the object of comparison appears in the initial position in the construction, followed by the gradable predicate, the copula verb, the standard of comparison and the postposition.

Negative comparisons (`less than') are encoded through negative particles that have under their scope the gradable predicate. Negative comparison constructions are exemplifed in (\ref{ex:14:negative comparison}).

\ea\label{ex:14:negative comparison}

    \ea[]{
    \textit{ˈét͡ʃi    tiˈwé  ko  ke  me    waˈríni  loˈrêna  ˈjûa}\\
    \gll    ˈét͡ʃi  teˈwé=ko  ke  me    waˈrína  loˈrêna  ˈjûa\\
            \textsc{dem} girl=\textsc{emph}  \textsc{neg} almost light    Lorena  \textsc{than}\\
    \glt    `That girl is less light (fast) than Lorena.’\\
    \glt    `Esa muchacha es menos ligera que Lorena.' < LEL 09 2:6/el >\\
}
\largerpage[2]
        \ex[]{
        \textit{ˈhê   ˈnà  aˈsêite    ko  ke  me    natiˈkí    ˈhê ˈnà  ˈkítara}\\
        \gll    ˈhê ˈnà   aˈsêite=ko  ke  me    natiˈkí    ˈhê ˈnà  ˈkítara\\
                \textsc{dem}  \textsc{prox} oil=\textsc{emph} \textsc{neg} almost expensive \textsc{dem}  \textsc{prox} \textsc{than}\\
        \glt    `This oil is less expensive than this other one.’\\
        \glt    `Este aceite es menos caro que este otro.' < LEL 09 2:6 Steffel/el >\\
    }
            \ex[]{
            \textit{wasaˈtʃ͡î ko  ke  me     apaˈrûame ˈhú    wasaˈtʃ͡î  waʔˈlûara  ˈjûa}\\
            \gll    basaˈtʃ͡î=ko  ke  me     apaˈrû-ame  ˈhú    basaˈtʃ͡î  waʔˈlû-a-ra ˈjûa\\
                    coyote=\textsc{emph}  \textsc{neg}  almost  angry-\textsc{ptcp} \textsc{cop.prs} coyote  big-a-\textsc{nmlz}  \textsc{than}\\
            \glt    `The coyote is less dangerous than the wolf.'\\
            \glt    `El coyote es menos bravo que el lobo.’ < SFH 09 3:85/el >\\
        }
    \z
\z

Comparative constructions may also be encoded via asyndetic conjunction or parataxis, the juxtaposition of clauses with no overt conjunction, with verbal ellipsis in the second clause. This is exemplified in (\ref{ex:14:comparative by parataxis}).

\ea\label{ex:14:comparative by parataxis}

    \textit{naˈʔî  ko    ke  me    ruruˈwá,  niˈhê    biˈtêritʃ͡i  ko  we}\\
    \gll    [naˈʔî=ko    ke  me    ruruˈwá] [niˈhê    biˈtêritʃ͡i=ko  we]\\
            here=\textsc{emph}  \textsc{neg}  almost  cold  \textsc{1sg.nom} house=\textsc{emph} \textsc{int}\\
    \glt    `It is less cold here than were I live’ (lit. `It is not very cold here. It is very (cold) where I live.’)\\
    \glt    `Hace menos frío aquí que en donde vivo’ (lit. `Aquí casi no hace frío. Donde yo vivo, mucho.') < SFH 09 3:85/el >\\

\z

Comparative constructions may also include a gradable predicate which is morphologically marked as comparative through the suffix \textit{{}-bê} ‘more/surpass’. The examples in (\ref{ex:14:comparatives with -be}) illustrate this morphological marker:

\ea\label{ex:14:comparatives with -be}

    \ea[]{
    \textit{ˈká ˈt͡ʃè}   \textbf{\textit{wikaˈbê}} \textit{riˈhòo-li   ba   oˈkwâ   riˈhòorame     ba \textit{ˈpé   be-ˈsá     maˈkòi   riˈhòo-li   ˈlé}\\}
    \gll    ˈká  ˈt͡ʃè  \textbf{wikaˈ-bê}  rihói-li    ba  oˈkwâ  rihói-li-ame    ba ˈpé           be-ˈsá    maˈkòi  riˈhòi-li    aˈlé\\
            \textsc{neg} \textsc{neg} {far-more}  inhabit.\textsc{pl-pst} \textsc{cl} few  inhabit\textsc{.pl-pst-ptcp} \textsc{cl} just   three-times  ten  inhabit.\textsc{pl-pst} \textsc{dub}\\
    \glt    `Because there were almost no people living here, more far away, just like thirty lived here.’\\
    \glt    `Porque casi no había gente aquí, había muy poquita como treinta yo creo.’ < JMF 09 tx817(6)/tx >\\
}
        \ex[]{
        \textit{ˈní-ma     be  ˈlà-o     ˈpé   a wili-ˈbê iˈná-ma ˈlé  ˈmò     ko   ba}\\
        \gll    ˈní-ma be ˈlà-o ˈpé   a \textbf{wili-ˈbê} iˈná-ma aˈlé  ˈmò=ko   ba\\
                \textsc{cop-fut.sg} be think-\textsc{ep} just  \textsc{aff} {long-more}  go.along\textsc{.sg-fut.sg} \textsc{dub}  \textsc{2sg.nom=emph} \textsc{cl}\\
        \glt    `I think so, you will be around (live) for a longer time.’\\
        \glt    `Yo creo que si, tu si vas a andar mucho tiempo.’ < FLP 06 in61(704)/in >\\
    }
            \ex[]{
            \textit{ˈnápi ʃibiˈrîko ˈdîas   aˈtí   ˈkát͡ʃi wiliˈbê wasaˈrúi ba}\\
            \gll    ˈnápi  ʃibiˈrîko ˈdîas  aˈtí  ˈkát͡ʃi \textbf{wili-ˈbê} wasa-ˈrú-i    ba\\
                    \textsc{sub}  Federico  Díaz  sit.\textsc{sg.prs}  \textsc{neg} {wide\textsc{-more}} plow-\textsc{nmlz-impf}  \textsc{cl}\\
            \glt    `Where Federico Díaz lives there was not much plowed land (the land was not very wide).’\\
            \glt    `Donde está Federico Díaz casi no estaba barbechado (no estaba muy ancha la tierra).’ < FLP 06 in61(94)/in >        \\
        }
                \ex[]{
                \textit{ah!   aˈbé=mi ripa-ˈbê ʃi-ˈmê   oˈlá-li}\\
                \gll    ah!   aˈbé=mi ripa-ˈbê si-ˈmêa     oˈlá-li\\
                        Ah! more=2\textsc{sg.nom} {up-more} go.\textsc{sg}{}-\textsc{fut.sg}  make-\textsc{pst}\\
                \glt    `Oh, you were going to go up higher (studying more).’\\
                \glt    `Ah, ibas a ir muy arriba (estudiando más).’  < SFH 06 in61(277)/in >\\
            }
  \z
\z

This morphological construction also marks superlative constructions, as shown in (\ref{ex:14:superlative examples}) below.

\ea\label{ex:14:superlative examples}

    \ea[]{
    \textit{aʔˈlì   ˈétʃ͡i   taˈbêara biniˈlá     ko    ˈá riˈpîli     ˈétʃ͡i   ko}\\
    \gll    aʔˈlì ˈétʃ͡i  ta-\textbf{ˈbê}-a-ra biniˈlá=ko   ˈá riˈpî-li  ˈétʃ͡i=ko\\
            and \textsc{dem} small-{more}-\textsc{prog-nmlz} small.sister-\textsc{poss=emph} \textsc{aff} remain-\textsc{pst} \textsc{dem=emph}\\
    \glt    `And then the youngest sister stayed.’\\
    \glt    `Y entonces la hermana menor se quedó.’ < LEL tx32(33)/tx >\\
}
        \ex[]{
        \textit{sereˈbêrio   be   ko   riˈwèi       taˈbêara     ko}\\
        \gll    sereˈbêrio   be=ko   riˈwè-i       ta-\textbf{ˈbê}-a-ra=ko\\
                Silverio  be=\textsc{emph}  be.named-\textsc{impf}  small-more-\textsc{prog-nmlz=emph}\\
        \glt    `Silverio was the name of the youngest one.’\\
        \glt    `Se llamaba Silverio el más chico.’ < FLP 06 in61(286)/in >\\
    }
    \z
\z

Comparatives may also involve a non-copular verbal predicate, as the following examples in (\ref{ex:14:comparatives with verbal predicate}) show. The example in (\ref{ex:14:comparatives with verbal predicatea}) contains no gradable predicate.

\ea\label{ex:14:comparatives with verbal predicate}

    \ea[]{
    \textit{ˈˈnè    ko  ˈwé  maˈt͡ʃí ʃimè-a  bioˈlîn  ˈmò   ˈkítara}\\
    \gll    ˈnè=ko  ˈwé  maˈt͡ʃí  seˈmè-a    bioˈlîn  ˈmò   ˈkítara\\
            1\textsc{sg.nom=emph}  \textsc{int}  know  play-\textsc{prog}  violin  \textsc{2sg.nom}  \textsc{than}\\
    \glt    `I know how to play the violin better than you.’\\
    \glt    `Yo se tocar más el violín que tu.’ < GFM 09 3:101/el >\\
}\label{ex:14:comparatives with verbal predicatea}
        \ex[]{
        \textit{nè    ko  ˈwé  raˈsíra  ˈnót͡ʃiri    oˈlá  beˈnè-a    muˈhê ˈjûa\\}
        \gll    ˈnè=ko  ˈwé  raˈsíra  ˈnót͡ʃi-li  oˈlá  beˈnè-a    muˈhê ˈjûa\\
                1\textsc{sg.nom=emph}  \textsc{int}  more  struggle-\textsc{pst}  \textsc{cer}  learn-\textsc{prog}  2\textsc{sg.nom} \textsc{than}\\
        \glt    `I struggle more to learn than you.’\\
        \glt    `Batallo más para aprender que tu.’ < GFM 09 3:101/el >\\
    }\label{ex:14:comparatives with verbal predicateb}
    \z
\z

The Choguita Rarámuri comparative construction is of a type that may be characterized as a `locational/adverbial comparative' \citet{stassen1984comparative}, as the markers of the standard of comparison are postpositions in the language. The construction consists of a single clause and the comparee NP can fulfill any grammatical function. In contrast, other \ili{Uto-Aztecan} languages have been analyzed as having a `particle comparative', including \ili{Comanche} (\ili{Numic}; \citealt{smalley1953phonemic}), \ili{Tohono O'odham} (\ili{Tepiman}; \citealt{zepeda1983tohono}), and \ili{Tümpisa Shoshone} (\ili{Numic}; \citealt{dayley1989tumpisa}) (see discussion in \citealt{wals-121}).
