\chapter{Noun phrases}
\label{chap: noun phrases}

This chapter addresses the internal structure of noun phrases in Choguita Rarámuri. While the development of syntactic arguments for an in-depth characterization of these constructions lies beyond the scope of this grammar, there is evidence for the existence of a noun phrase constituent broadly defined as a syntactic constituent that may function as an argument of verbal predicates (\citealt{dryer2007noun}). As described in \chapref{chap: particles, adverbs and other word classes}, a subset of minor word classes can be characterized by their ability to head noun phrases (pronouns) or combine with head nouns in noun phrases (demonstratives, adjectives, numerals, definite articles and quantifiers). Noun phrases which contain a head noun and one of these modifiers are described in §\ref{sec: simple noun phrases}. Noun phrases that involve genitive constructions or other complex modifiers are addressed in §\ref{sec: complex noun phrases}.

\section{Simple noun phrases}
\label{sec: simple noun phrases}

Word order in other Rarámuri varieties (\ili{Rochéachi Rarámuri} (\citealt[64]{moralesmoreno2016rochecahi})) and closely related varieties (\ili{Mountain Guarijío} (\citealt[73]{miller1996guarijio})) is described as being verb final, with order of constituents exhibiting both syntactic and pragmatic conditioning. While the ordering of constituents in the Choguita Rarámuri clause likewise appears to exhibit syntactic restrictions and pragmatically-determined order variation, word order within the noun phrase can be described in terms of a template, which is schematized in (\ref{ex: word order in noun phrases}).

\ea\label{ex: word order in noun phrases}
{Word order in noun phrases}

Demonstrative -- Definite article -- [Numeral/Quantifier] -- Adjective -- Noun

\z

Noun phrases, replicating the basic word order of clauses, are head final, i.e., the order of constituents in noun phrases involves modifiers preceding the head noun. In contrast to nouns, pronouns are case marked (see §\ref{sec:10:pronouns}) and occur as the single constituent in noun phrases, though there are examples attested in the Choguita Rarámuri corpus where pronouns occur adpositionally with other nouns, as shown in (\ref{ex: noun plus pronoun adposition}) (see also \citealt{miller1996guarijio}:229) for description of the same phenomenon in \ili{Mountain Guarijío}).

\ea\label{ex: noun plus pronoun adposition}

\textit{ˈkíti ˈbéli ti ˈwé riˈsóati ka ruˈwá \textbf{tamuˈhê raˈlàmuli} ko ba}\\
\gll    ˈkíti ˈbéli=ti ˈwé riˈsó-a=ti ka ru-ˈwá tamuˈhê raˈlàmuli=ko ba\\
        that.is.why indeed=\textsc{1pl.nom} \textsc{int} struggle\textsc{-prog=1pl.nom} \textsc{cop.irr} say-\textsc{mpass} \textsc{1pl.nom} people\textsc{=emph} \textsc{cl}\\
\glt    `That is why it is said that we are very poor (struggle a lot), us the Rarámuri people.'\\
\glt    `Por eso dicen que estamos muy pobres (batallamos mucho) nosotros los tarahumaras.’ \corpuslink{tx128[01_588-02_025].wav}{SFH tx128:01:58.8}\\

\z

With the exception of numerals and quantifiers, which do not co-occur, all other modifiers may potentially co-occur with each other and head nouns. The discussion below addresses the behavior of each of the different elements that may modify a head noun in noun phrases and the relative ordering of modifiers when they co-occur in noun phrases.

\subsection{Demonstratives}
\label{subsec: demonstratives in noun phrases}

As discussed in §\ref{subsec: adnominal demonstratives}, demonstratives may be use adnominally. Demonstratives are attested preceding the head noun in noun phrases.\footnote{For \ili{Mountain Guarijío}, \citet{miller1996guarijio} reports that demonstratives may follow nouns in noun phrases, though he notes this is rarely attested (\citeyear[235]{miller1996guarijio}). No similar cases have been documented so far in Choguita Rarámuri.} This is exemplified in (\ref{ex: adnominal demonstratives NP}). Noun phrases are indicated with square brackets.

\ea\label{ex: adnominal demonstratives NP}

    \ea[]{
    \textit{aʔˈlì \textbf{ˈétʃi muˈkî} ko ˈwé saˈpù ˈwé siˈnàka na ``ˈwé saˈpù aˈsísi!" aˈnèli}\\
    \gll    aʔˈlì [ˈétʃ͡i muˈkî]=ko ˈwé saˈpù ˈwé siˈnà-ka na ˈwé saˈpù aˈsísi aˈn-è-li\\
            and \textsc{dem} woman=\textsc{emph} \textsc{int} fast \textsc{int} scream-\textsc{ger} then \textsc{int} fast get.up.\textsc{imp.sg} say-\textsc{appl-pst}\\
    \glt    ‘And then the woman with a hurry shouted: “hurry up, get up”, she said to him.’\\
    \glt    ‘Y entonces la mujer apurada gritando le dijo: “pronto levántate" le dijo.' \corpuslink{tx5[01_213-01_263].wav}{LEL tx5:01:21.3}\\
}
%\break
        \ex[]{
        \textit{aʔˈlì biˈlá ko waˈbé biˈlá kiˈʔà ˈníla ra pa, kuˈrí ke ˈtʃ͡ó  me biwaˈtʃ͡êatʃ͡i \textbf{ˈnà kaˈwì} ba}\\
        \gll    aʔˈlì biˈlá=ko waˈbé biˈlá kiˈʔà ˈní-la ra pa kuˈrí ke ˈtʃ͡ó  me biwa-ˈtʃ͡ê-a-tʃ͡i [na kaˈwì] ba\\
                and really=\textsc{emph} long.ago really before \textsc{cop-rep.sg} say.\textsc{prs} \textsc{cl} recently \textsc{neg} yet almost hard-\textsc{vrbl-prog-loc} this world \textsc{cl}\\
        \glt    `And then they say it was a long time ago when this world hadn't consolidated (become solid) yet.'\\
        \glt    `Y entonces dicen que fue mucho antes, cuando todavía no amacizaba bien este mundo.' \corpuslink{tx43[11_112-11_182].wav}{SFH tx43:11:11.2}\\
    }

    \z
\z

The following examples (in (\ref{ex: demonstrative plus modifiers})) show that demonstratives appear in initial position when co-occurring with other modifiers within the noun phrase.\footnote{As described in §\ref{subsec: depalatalization and deaffrication of alveopalatal affricates}, alveopalatal affricates may depalatalize and deaffricate in fast speech in high frequency word combinations that include the demonstrative \textit{ˈétʃ͡i}, as exemplified in (\ref{ex: demonstrative plus modifiersb}).}

\ea\label{ex: demonstrative plus modifiers}

    \ea[]{
    \textit{ˈápu riˈká raˈpâko \textbf{ˈétʃi} biˈlé muˈkî ke naˈwàlo la ˈro}\\
    \gll    ˈnápi riˈká raˈpâko [ˈétʃ͡i biˈlé muˈkî] ke naˈwà-lo-la ˈru\\
            \textsc{sub} like.that yesterday \textsc{dem} one woman \textsc{neg} arrive-\textsc{pst-rep.ds} say.\textsc{prs}\\
    \glt    `like yesterday that one woman didn't arrive, they say'\\
    \glt    `así como ayer esa mujer no llegó, dicen'    \corpuslink{co1237[03_316-03_342].wav}{JLG co1237:03:31.6}\\
}\label{ex: demonstrative plus modifiersa}
        \ex[]{
        \textit{ˈwé maˈtʃína iˈnáli \textbf{ˈés} ti koriˈmá}\\
        \gll    ˈwé maˈtʃína iˈnáli [ˈétʃ͡i ti koriˈmá]\\
                \textsc{int} see.\textsc{prs} go.\textsc{sg-ep} \textsc{dem} \textsc{def.bad} fire.bird\\
        \glt    ‘It is very noticeable when the \textit{korima} (fire bird) is passing by.’\\
        \glt    ‘Se ve muy bien cuando va (pasa) el \textit{korimá} (pájaro de fuego).’   \corpuslink{tx5[04_517-04_538].wav}{LEL tx5:04:51.7}\\
    }\label{ex: demonstrative plus modifiersb}

    \z
\z

In (\ref{ex: demonstrative plus modifiersa}), the demonstrative occurs in phrase initial position, followed by a numeral (\textit{biˈlé} `one') and the head noun (\textit{muˈkî} `woman'). In (\ref{ex: demonstrative plus modifiersb}), the demonstrative precedes a definite article (\textit{ti} `the, negative stance') and the head noun (\textit{koriˈmá} `fire bird') in the noun phrase.

\subsection{Definite articles}
\label{subsec: definite articles in noun phrases}

Definite articles, which encode number and affective stance of a definite referent (§\ref{sec: articles}), precede head nouns in noun phrases, as exemplified in (\ref{ex: definite articles NP}).\footnote{As described in §\ref{sec: articles}, while the definite articles encoding negative stance are generally attested when the speaker conveys a negative attitude towards the referent, they can also be attested with a neutral connotation, as in (\ref{ex: definite articles NPc}).}

\ea\label{ex: definite articles NP}

    \ea[]{
    \textit{ke ˈpé ˈtâʃi ˈhú pa, ˈwé beti ˈwé wiˈtʃ͡íwami \textbf{ˈtʃéti erˈmâno} ba}\\
    \gll    ke ˈpé ˈtâsi ˈhú pa ˈwé be=ti ˈwé wiˈtʃ͡í-w-ame [ˈtʃ͡éti erˈmâno] ba\\
            \textsc{neg} little \textsc{neg} \textsc{cop.prs} \textsc{cl} \textsc{int} \textsc{emph=1pl.nom} \textsc{int} believe-\textsc{mpass-ptcp} \textsc{def.pl.bad} evangelical.brothers \textsc{cl}\\
    \glt    `Well no, we believe the evangelical brothers very much.'\\
    \glt    `Pues no, les hacemos mucho caso a los hermanos evangélicos.' \corpuslink{in243[02_084-02_124].wav}{FLP in243:02:08.4}\\
}\label{ex: definite articles NPa}
        \ex[]{
        \textit{ka ˈtʃ͡è ko niˈrú ... ˈnàri ... \textbf{ˈtʃéti ˈâba} tʃ͡aˈbèi ko ba}\\
        \gll    ka ˈtʃ͡è=ko niˈrú ˈnàri [ˈtʃ͡éti ˈâba] tʃ͡aˈbèi=ko ba\\
                because because\textsc{=emph} \textsc{cop.impf} then \textsc{def.pl.bad} broad.bean before\textsc{=emph} \textsc{cl}\\
        \glt    `because there were no broad beans before.'\\
        \glt    `porque no había habas antes.'    \corpuslink{in61[02_244-02_283].wav}{FLP in61:02:24.4}\\
    }\label{ex: definite articles NPb}
            \ex[]{
            \textit{aʔˈlì biˈlá ˈhípi biˈlá ˈmá noˈkí maˈkò maˈrí bamˈpáma ˈlé \textbf{ˈtí tiˈwé} niˈhê ˈkútʃ͡ara baˈtʃ͡áwara ba}\\
            \gll    aʔˈlì biˈlá ˈhípi biˈlá ˈmá noˈkí maˈkò maˈrí bam-ˈpá-ma aˈlé [ˈtí tiˈwé] niˈhê ˈkútʃ͡a-la baˈtʃ͡á-wa-la ba\\
                    and indeed now indeed already almost ten five have.birthday-\textsc{inch-fut.sg} \textsc{dub} \textsc{def.sg} girl \textsc{1sg.nom} child-\textsc{poss} first-\textsc{vblz-poss} \textsc{cl}\\
            \glt    `And now the girl, my daughter, will turn fifteen, my oldest child.'\\
            \glt    `Pues ahora ya la muchacha ya va a cumplir quince años, mi hija la más grande.'  \corpuslink{tx43[04_056-04_120].wav}{SFH tx43:04:05.6}\\
        }\label{ex: definite articles NPc}

    \z
\z

Definite articles may co-occur with other modifiers in nominal phrases, including demonstratives, as shown above in (\ref{ex: demonstrative plus modifiersb}), and adjectives, as shown in (\ref{ex: definite article plus adjective}).

\ea\label{ex: definite article plus adjective}

\textit{ˈtʃêram ˈsûs ba,\textbf{ ti ˈtʃêram bauˈtîʃ} ma ba}\\
\gll    [ˈtʃêram ˈsûs] ba, [ti ˈtʃêram bauˈtîʃ] ma ba\\
        elder Jesús \textsc{cl} \textsc{def.sg} elder Bautista also \textsc{cl}\\
\glt    `elder Jesús, also elder Bautista'\\
\glt    `don Jesús, también don Bautista'  \corpuslink{in484[07_185-07_223].wav}{ME in484:07:18.5}\\

\z

\subsection{Numerals}
\label{subsec: numerals in noun phrases}

Numerals, like other modifiers, are attested preceding the head noun in nominal phrases. This is exemplified in (\ref{ex: numerals in NP}).

\ea\label{ex: numerals in NP}

    \ea[]{
    \textit{aʔˈlì ˈétʃ͡i ˈkútʃ͡uala ko ˈpé ˈkúutʃ͡i ˈníli \textbf{oˈkwâ ˈkútʃi} ... \textbf{oˈkwâ iˈwé}, \textbf{oˈkwâ ˈkûruwi}}\\
    \gll    aʔˈlì ˈétʃ͡i ˈkútʃ͡ua-la=ko ˈpé ˈkútʃ͡i ˈní-li [oˈkwâ ˈkútʃ͡i] [oˈkwâ iˈwé] [oˈkwâ ˈkûruwi]\\
            and \textsc{dem} child-\textsc{poss=emph} little small.\textsc{pl} \textsc{cop-pst} two girls two girls two boys\\
    \glt    `And their children were small, two girls...two girls and two boys.’\\
    \glt    `Y los hijos de ellos eran chiquitos, dos niñas... dos niñas y dos niños.’   \corpuslink{tx32[02_047-02_110].wav}{LEL tx32:02:04.7}\\
}
        \ex[]{
        \textit{\textbf{biˈlé ariˈmûli}, \textbf{oˈkwâ ariˈmûli} ma, \textbf{biˈkiá ariˈmûli} ma}\\
        \gll    [biˈlé ariˈmûli] [oˈkwâ ariˈmûli] ma [biˈkiá ariˈmûli] ma\\
                one decaliter two decaliter or three decaliter or\\
        \glt    `one decaliter or two decaliters or three decaliters'\\
        \glt    ‘un decalitro o dos decalitros o tres decalitros’   \corpuslink{tx68[00_258-00_292].wav}{LEL tx68:00:25.8}\\
    }

    \z
\z

The following examples show the relative ordering of numerals and other modifiers, including demonstratives.

\ea\label{ex: numeral adjective ordering NP}

    \ea[]{
    \textit{ˈmí \textbf{naˈó} raˈwé ko ˈmá ku ˈʔnèni ˈpîri ˈjâaro}\\
    \gll    [ˈmí naˈó raˈwé]=ko ˈmá ku iʔˈnèni ˈpîri ˈâ-ru\\
            \textsc{dem} four day=\textsc{emph} already \textsc{rev} see.\textsc{prs} what give.\textsc{pst.pas-pst.pass}\\
    \glt    ‘After four days they then see what they were given.’\\
    \glt    ‘Ya en los cuatro días ya ven lo que les dieron.’   \corpuslink{tx19[05_370-05_404].wav}{LEL tx19:05:37.0}\\
}\label{ex: numeral adjective ordering NPa}
        \ex[]{
        \textit{\textbf{biˈlé} waʔˈlû fiêsta oriˈwábatʃ͡i biˈléna bitiˈtʃ͡í}\\
        \gll    [biˈlé waʔˈlû fiêsta] oriˈwába-tʃ͡i biˈlé-na bitiˈtʃ͡í\\
                one big party make.\textsc{mpass}-\textsc{temp} one-\textsc{incl} house\\
        \glt    `when one big party is made at one house'\\
        \glt    `cuando se hace una fiesta grande en una casa'   \corpuslink{tx1009[00_449-00_540].wav}{GFP tx1009:00:44.9}\\
    }\label{ex: numeral adjective ordering NPb}

    \z
\z

As shown in these examples, numerals follow demonstratives (\ref{ex: numeral adjective ordering NPa}) but precede adjectives (\ref{ex: numeral adjective ordering NPb}).

The behavior and distribution of numerals in noun phrases in Choguita Rarámuri contrasts with that of closely related \ili{Mountain Guarijío}, where the most frequent order in noun phrases is that of head nouns followed by numerals (\citealt[244]{miller1996guarijio}).

\subsection{Quantifiers}
\label{subsec: quantifiers in noun phrases}

Quantifiers in Choguita Rarámuri must also precede the head noun in noun phrases, as exemplified in (\ref{ex: quantifiers in NP}).

\ea\label{ex: quantifiers in NP}

    \ea[]{
    \textit{aʔˈlì ko ˈtʃ͡ó  ˈnà ˈhônsa ko ˈá biˈláni ˈwé ... ˈwé ˈtʃ͡ó  ˈam tʃ͡i riˈká wiˈtʃ͡íwi ˈtʃ͡ó  ˈnè ˈá ˈmá \textbf{wiˈʰkâ naˈmûti} ˈa}\\
    \gll    aʔˈlì=ko ˈtʃ͡ó  ˈnà ˈhônsa=ko ˈá biˈlá=ni ˈwé we ˈtʃ͡ó  ˈam tʃ͡i riˈká wiˈtʃ͡íwi ˈtʃ͡ó  ˈnè ˈá ˈmá [wiˈʰkâ naˈmûti] ˈa\\
            and\textsc{=emph} \textsc{dem} \textsc{dem} since\textsc{=emph} \textsc{aff} truly=\textsc{1sg.nom} \textsc{int} \textsc{int} also \textsc{aff=dem} \textsc{dem} like.that believe.\textsc{prs} also \textsc{1sg.nom} \textsc{aff} already many things indeed\\
    \glt    `and then from there I knew how to believe in all of these things’\\
    \glt    `entonces de ahí ya supe cómo creer yo muchas cosas’  \corpuslink{tx71[04_486-04_552].wav}{LEL tx71:04:48.6}\\
}
        \ex[]{
        \textit{ˈwé biˈlá ... riˈsóa iˈkí \textbf{wiˈʰkâ paˈkótam} pa}\\
        \gll    ˈwé biˈlá riˈsó-a iˈkí [wiˈʰkâ paˈkót-ame] pa\\
                \textsc{int} indeed suffer-\textsc{prog} happened many wash-\textsc{ptcp} \textsc{cl}\\
        \glt    `Indeed many people died (they endured suffering).'\\
        \glt    `En verdad murieron muchas personas (les pasó sufrimiento).' \corpuslink{tx372[04_515-04_557].wav}{LEL tx372:04:51.5}\\
}
            \ex[]{
            \textit{ˈápi iˈsêlikami ˈká ˈlé, biˈkiánika, \textbf{suˈwâba maˈjôra} ma}\\
            \gll    ˈápi [i-ˈsêri-kami] ˈká aˈlé biˈkiá-ni-ka [suˈwâba maˈjôra] ma\\
                    \textsc{sub} \textsc{pl}-governor.\textsc{pl-ptcp} \textsc{cop.irr} \textsc{dub} three-\textsc{incl-coll} all manager also\\
            \glt    `those who are governors, the three of them, all of the managers, too'\\
            \glt    `los que son gobernadores, los tres (gobernadores), todos los mayores también'    \corpuslink{tx816[00_367-00_399].wav}{JMF tx816:00:36.7}\\
        }
    \z
\z

As described in §\ref{sec: quantifiers}, some quantifiers in Choguita Rarámuri are derived from numerals (e.g., \textit{oˈkwâ} 'few’ is derived from the numeral `two’). There are no recorded instances of numerals and quantifiers co-occurring in noun phrases. Quantifiers may nevertheless co-occur with other modifiers, including demonstratives, as shown in (\ref{ex: demonstrative quantifier ordering}).

\ea\label{ex: demonstrative quantifier ordering}

    \textit{aʔˈlì \textbf{ˈétʃi ʃiˈnêami} raˈlàmuli ko ˈhê aˈníli}\\
    \gll    aʔˈlì [ˈétʃ͡i siˈnêami raˈlàmuli]=ko ˈhê aˈní-li\\
            then \textsc{dem} everyone men=\textsc{emph} it say-\textsc{pst}\\
    \glt    `then all the men said'\\
    \glt    `entonces todos los hombres dijeron”  \corpuslink{tx32[06_325-06_355].wav}{LEL tx32:06:32.5}\\

\z

\subsection{Adjectives}
\label{subsec: adjectives in noun phrases}

Adjectives in Choguita Rarámuri may be basic (a small class of underived adjective roots) or may be derived through morphological means (§\ref{sec: adjectives}). Adjectives, both basic and derived, may also be used predicatively in copular clauses (§\ref{subsec: locative clauses}). A comprehensive description of adjectives in Choguita Rarámuri is provided in \citet{islas2010caracterizacion}. This section addresses the distribution of basic adjectives in noun phrases. As shown in (\ref{ex: adjectives NP}), adjectives precede head nouns.

\ea\label{ex: adjectives NP}

    \ea[]{
    \textit{aʔˈlì ne maˈhâali es \textbf{ˈkútʃi koˈtʃî}] ko}\\
    \gll    aʔˈlì ne maˈhâ-li es [ˈkútʃ͡i koˈtʃ͡î]=ko\\
            and \textsc{int} fear\textsc{-pst} \textsc{dem} little dog=\textsc{emph}\\
    \glt    `And the dog got very scared.'\\
    \glt    `Y se asustó mucho el perrito.'  \corpuslink{tx191[04_247-04_269].wav}{BFL tx191:04:24.7}\\
}

        \ex[]{
        \textit{\textbf{waʔˈlû kapaˈnî} aˈnítʃ͡ini ba}\\
        \gll    [waʔˈlû kapaˈnî] aˈní-tʃ͡ini ba\\
                big bell make.sound-\textsc{ev.sound} \textsc{cl}\\
        \glt    `The big bell rang (was heard).'\\
        \glt    `Se oyó sonar la campana grande.'    \corpuslink{tx223[03_291-03_318].wav}{LEL tx223:03:29.1}\\
    }

    \z
\z

The following example (shown above in (\ref{ex: numeral adjective ordering NPb}) and repeated here in (\ref{ex: adjectives and modifiers NP})) shows the relative ordering of adjectives with respect to other modifiers within noun phrases, namely following numerals and preceding the head noun.

\ea\label{ex: adjectives and modifiers NP}

    \textit{\textbf{biˈlé waʔˈlû} fiêsta oriˈwábatʃ͡i biˈléna bitiˈtʃ͡í}\\
    \gll    [biˈlé waʔˈlû fiêsta] oriˈwába-tʃ͡i biˈlé-na bitiˈtʃ͡í\\
            one big party make.\textsc{mpass}-\textsc{temp} one-\textsc{incl} house\\
    \glt    `when one big party is made at one house'\\
    \glt    `cuando se hace una fiesta grande en una casa' \corpuslink{tx1009[00_449-00_540].wav}{GFP tx1009:00:44.9}\\

\z

Adjectives may be modified by adverbs, as shown in (\ref{ex: adverb adjective combination}), where a degree adverb, \textit{we} `very', modifies the adjective \textit{ˈtʃ͡áti}, which in turn modifies the head noun (the deverbal noun \textit{nawiˈri} `disease').

\ea\label{ex: adverb adjective combination}

\textit{apʔaˈlì nawiˈrí naˈwà ˈsíli \textbf{we ˈtʃáti} nawiˈrí ˈhêmi naˈʔî puêblo}\\
\gll    ˈnápi aʔˈlì nawi-ˈrí naˈwà ˈsí-li [we ˈtʃ͡áti nawi-ˈrí] ˈhêmi naˈʔî puêblo\\
        \textsc{sub} and get.sick-\textsc{nmlz} arrive go.\textsc{sg-pst} \textsc{int} ugly get.sick-\textsc{nmlz} this here town\\
\glt    `when a very dangerous (ugly) disease arrived in this town'\\
\glt    `cuando llegó una enfermedad muy peligrosa en este pueblo'    \corpuslink{tx372[00_240-00_293].wav}{LEL tx372:00:24.0}\\

\z

The distribution of adjectives in noun phrases in Choguita Rarámuri differs from that of adjectives in \ili{Mountain Guarijío} noun phrases, where adjectives can be attested post-nominally or may appear in discontinuous constructions (\citealt{miller1996guarijio}). With adjectives and other types of modifiers, Choguita Rarámuri exhibits stricter ordering restrictions within the noun phrase than those attested in \ili{Mountain Guarijío}.

\section{Complex noun phrases: Possessive constructions}
\label{sec: complex noun phrases}

%MILLER 1996:249-50 describes the types of semantic relationships between nouns and noun phrases in genitive constructions (absolutive and possessive)

%semantic relationships encoded in complex noun phrases include existence, part-whole relations (partitive) alienable and inalienable possession (see Miller 2006:249-50

% head-marking pattern - see discussion in Druer 2007:22

% The constructions with niwa are interesting because they recall some of the minimal appositive possessive constructions, e.g. in \ili{Ainu}, which uses its verb 'have' in just this way.  Two questions about examples like nehé nía-ra sa'pá  'It is mine the meat (to eat)':  Can this also be an NP meaning 'my meat'? or does it have to be predicative?   And, is the nehé niá part a relative clause or similar to a relative clause?  (Or maybe the whole thing is an internally headed relative?)
%• Actually, three questions.  Do you know if it's possible to use any other verbs in this construction?  E.g. 'eat' instead of 'have', for nouns like 'meat'?

%In connection with seeing these as relative clauses, in several languages I've seen a pattern where what looks like an NP (including relative clauses) is translated as an NP where the head noun is possessible but as a clause where it's non-possessible.  As I recall Tümpisa Shoshone or Eastern Shoshone has examples like that, and just recently I found something similar in Kamaiurá (\ili{Tupi}-Guarani).  Maybe it's just the predictable consequence of possession not being natural for non-possessibles, but I thought it was curious.

Choguita Rarámuri possessive constructions involve a noun modified by a noun phrase encoding a range of meanings, including part-whole relationships and possession or ownership. The term `possessive' is used here to refer to adnominal constructions, even if the semantic relations encoded by the construction do not involve possession. Examples of possessive constructions with different meanings are exemplified in (\ref{ex: possessive constructions semantics}).

\ea\label{ex: possessive constructions semantics}
{Choguita Rarámuri possessive constructions}

    \ea[]{
    \textit{Meronymic (part-whole)}\\
    {\textit{ˈmêsa   roˈnôla}}\\
    \gll    ˈmêsa  roˈnô-la\\
            table  leg-\textsc{poss}\\
    \glt    ‘the table’s leg’\\
    \glt    ‘la pata de la mesa’  < BFL 09 1:33/el >  \\
}
        \ex[]{
        \textit{Possession}\\
        \textit{raˈniêli uˈpîla}\\
        \gll    raˈniêli uˈpî-la\\
                Daniel wife-\textsc{poss}\\
        \glt    `Daniel's wife'\\
        \glt    `la esposa de Daniel'  \corpuslink{tx5[03_038-03_096].wav}{LEL tx5:03:03.8}\\
    }
    \z
\z

As discussed in §\ref{subsec: possessive}, Choguita Rarámuri exhibits possessive classification, where there is a contrast between two types of possession, alienable and inalienable, determined by the lexical properties of possessed nouns. Inalienable nouns (kinship terms, body parts and other nouns) are obligatorily possessed. Both alienable and inalienable nouns may be used in a possessive construction. The schema in (\ref{ex: possessive construction template}) shows the possessive construction template.\footnote{\ili{Mountain Guarijío} (\citealt{miller1996guarijio}) and \ili{Norogachi Rarámuri} (\citealt{brambila1953gramatica, brambila1976diccionario}) are documented to possess a classifier for domesticated animals (\textit{puhkú} in \ili{Mountain Guarijío}, \textit{bukú} in \ili{Norogachi Rarámuri}), marked with the possessive suffix and followed by the specific animal name. This construction has not been documented in Choguita Rarámuri, which instead encodes possession of domesticated or mythical animals through a specialized denominal verb of possession, as described in §\ref{subsubsec: the verbalizer -e suffix} (a strategy also documented in \ili{Mountain Guarijío} (\citealt[149]{miller1996guarijio})).}

\ea\label{ex: possessive construction template}

[[possessor: noun/pronoun]\textsubscript{N1} [possessum: noun-\textsc{poss}]\textsubscript{N2}]\textsubscript{NP}

\z

The possessum takes the possessive suffix (\textsc{poss}) \textit{-la} (optionally preceded by a suffix \textit{-wa} with some nouns) that encodes a possessor, with no person marking on the possessed noun.

\subsection{Nominal possessors}
\label{subsubsec: nominal possessors}

%As described in §\ref{sec: case marking}, Choguita Rarámuri may be marked for instrumental or locative case.
Examples of possessive constructions with nominal possessors are provided in (\ref{ex: nominal possessor}), which follow the template schematized in (\ref{ex: possessive construction template}): a possessor noun precedes the possessum marked with the possessive suffix.

\ea\label{ex: nominal possessor}

    \ea[]{
    \textit{tʃ͡omaˈlî aʔˈwâla}\\
    \gll    tʃ͡omaˈlî aʔˈwâ-la\\
            deer horn-\textsc{poss}\\
    \glt    `deer's horn'\\
    \glt    ‘cuerno de venado’  \corpuslink{tx152[07_404-07_454].wav}{SFH tx152:07:40.4}\\
}
        \ex[]{
        \textit{teˈwé maiˈlâ}\\
        \gll    teˈwé mai-ˈlâ\\
                girl father.\textsc{female.ego-poss}\\
        \glt    `girl's father'\\
        \glt    `papá de la muchacha'  \corpuslink{tx32[13_567-14_022].wav}{LEL tx32:13:56.7}\\
    }
            \ex[]{
            \textit{reˈhòi ariˈwâla}\\
            \gll    reˈhòi ariˈwâ-la\\
                    man soul\textsc{-poss}\\
            \glt    `man's soul'\\
            \glt    `alma del hombre' \corpuslink{tx5[01_113-01_183].wav}{LEL tx5:01:11.3}\\
        }
    \z
\z

There are no attested instances of possessive constructions with nominal possessors with alternative order of possessor and possessum, nor instances of intervening elements between possessor and possessum in the Choguita Rarámuri corpus (a separate appositional construction that encodes a possession relationship through a specialized possessive noun is described in §\ref{subsubsec: appositive possessive constructions} below).

\subsection{Pronominal possessors}
\label{subsubsec: pronominal possessors}

The Choguita Rarámuri possessive construction with pronominal possessors exhibits the same structure as the one with nominal possessors: the possessor precedes the possessum, which is marked with the suffix \textit{-la} to encode a possessor. Examples of pronominal possessors are provided in (\ref{ex: pronominal possessors}). As discussed in §\ref{sec:10:pronouns}, personal pronouns encode case in Choguita Rarámuri. Pronominal possessors are encoded with the nominative-marked pronoun series.

%In these examples, the pronominal forms exhibit variation between the full and reduced forms with no apparent grammatical conditioning, e.g., neˈhe iˈje-ra \corpuslink{tx475[06_173-06_214].wav}{SFH tx475:06:17.3} vs. ne iˈje-ra \corpuslink{tx1[00_555-01_020].wav}{BFL tx1:00:55.5} 'my mother'. This kind of system is, to the best of my knowledge, not reported in other Rarámuri varieties. (this is now footnote 13) In fact, revisiting the other references and my own corpus, I see no mention of the reduced forms, while in the CR corpus, the long vs. short forms seem interchangeable, with a higher frequency of the short forms (though I haven't done detailed counts nor controlled for larger contexts). But I don't detect differences in grammatical context.

\ea\label{ex: pronominal possessors}

    \ea[]{
    {\textit{ˈnè    waˈsála}}\\
    \gll    ˈnè    waˈsá-lâ\\
            \textsc{1sg.nom}  cultivation.land-\textsc{poss} \\
    \glt    ‘my cultivation land'\\
    \glt    ‘mi tierra de cultivo’ < BFL 09 1:61/el > \\
}
%\break
        \ex[]{
        \textit{ˈmò uˈmûala}\\
        \gll    ˈmò uˈmûa-la\\
                \textsc{2sg.nom} grandfather-\textsc{poss}\\
        \glt    `your great grandfather'\\
        \glt    `tu bisabuelo'  \corpuslink{in61[04_552-04_587].wav}{FLP in61:04:55.2}\\
    }
            \ex[]{
            \textit{ˈémi oˈtʃ͡îpala}\\
            \gll    ˈémi oˈtʃ͡îpa-la\\
                    \textsc{2pl.nom} paternal.grandfather-\textsc{poss}\\
            \glt    `your (pl.) grandfather'\\
            \glt    `abuelo de ustedes'    \corpuslink{tx12[09_138-09_183].wav}{SFH tx12:09:13.8}\\
        }
        \newpage
                \ex[]{
                \textit{ˈnè waˈrîla}\\
                \gll    ˈnè waˈrî-la\\
                        \textsc{1sg.nom} basket-\textsc{poss}\\
                \glt    `my basket'\\
                \glt    `mi canasta'  < BFL 11/07/09/el >\\
            }
    \z
\z

Pronominal possessors exhibit variation between the full and reduced forms in possessive constructions with no apparent grammatical conditioning, as shown by the examples in (\ref{ex: full and reduced pronominal possessors}), which are equivalent in meaning.\footnote{This kind of system is also reported in \ili{Rochéachi Rarámuri}, though \citet{moralesmoreno2016rochecahi} describes restrictions in the distribution of reduced pronominal forms in \ili{Rochéachi Rarámuri} (\citeyear[72]{moralesmoreno2016rochecahi}). In the Choguita Rarámuri corpus the long vs. short forms appear to be interchangeable, with a higher frequency of the short forms, though no detailed counts have been carried out controlling for discourse contexts.}

\ea\label{ex: full and reduced pronominal possessors}

    \ea[]{
    \textit{neˈhê iˈjêla}\\
    \gll    neˈhê iˈjê-la\\
            \textsc{1sg.nom} mother-\textsc{poss}\\
    \glt    `my mother'\\
    \glt    `mi madre'   \corpuslink{tx475[06_173-06_214].wav}{SFH tx475:06:17.3}\\
}
        \ex[]{
        \textit{ˈnè iˈjêla}\\
        \gll    ˈnè iˈjê-la\\
                \textsc{1sg.nom} mother-\textsc{poss}\\
        \glt    `my mother'\\
        \glt    `mi madre'   \corpuslink{tx1[00_555-01_020].wav}{BFL tx1:00:55.5}\\
    }
    \z
\z

There are cases attested where the pronominal possessor is omitted, if the possessor is a third person argument. This is exemplified in (\ref{ex: omitted pronominal possessor}), where the possessor is established previously in the discourse.

\ea\label{ex: omitted pronominal possessor}

\textit{aʔˈlì ko \textbf{ˈkútʃala} biˈlá onoˈkáli ba ˈkîni ˈtîo ba}\\
\gll    aʔˈlì=ko ˈkútʃ͡a-la biˈlá onoˈká-li ba ˈkîni ˈtîo ba\\
        and\textsc{=emph} children-\textsc{poss} indeed do-\textsc{pst} \textsc{cl} \textsc{1.poss} uncles \textsc{cl}\\
\glt    `And then later the children do it (plow), my uncles.'\\
\glt    `Y luego ya después lo hacen (barbechan) los hijos, mis tíos.’    \corpuslink{tx130[05_133-05_170].wav}{LEL tx130:05:13.3}\\

\z

Choguita Rarámuri also has a dedicated pronominal form encoding first person possessors (singular or plural) of kinship terms. This is exemplified above in (\ref{ex: omitted pronominal possessor}) (\textit{ˈkini ˈtîo} `my uncles'), and further exemplified in (\ref{ex: pronominal possession kinship}).


\ea\label{ex: pronominal possession kinship}

    \ea[]{
    \textit{``ˈnà ko ʃiʔˈrînuwala paˈkóala ˈkám pa" ˈhê biˈlá  aˈní ˈétʃ͡i \textbf{ˈkîni} ˈwênuwa tʃ͡aˈbèe}\\
    \gll    ˈna=ko siʔˈrînuwala paˈkó-a-la ˈká=mi pa ˈhê biˈlá  aˈní ˈétʃ͡i ˈkîni ˈwênuwa tʃ͡aˈbè\\
            \textsc{this.one=emph} shirínuwala.ritual baptize-\textsc{prog-purp} \textsc{cop.irr=dem} \textsc{cl} it indeed say.\textsc{prs} \textsc{dem} \textsc{1.poss} parents before\\
    \glt    ```This is the one that baptizes with shirínuwala", that's what our parents used to say long ago.'\\
    \glt    ```Este es el que bautiza de shirínuwala", eso decían nuestros padres antes.'    \corpuslink{tx475[07_579-08_024].wav}{SFH tx475:07:57.9}\\
}\label{ex: pronominal possession kinshipa}
        \ex[]{
        \textit{ˈnápu ˈriˈká ke naˈtâsa \textbf{ˈkini} ˈwênuwala ˈro ba ˈni}\\
        \gll    ˈnápu ˈriˈká ke naˈtâ-sa ˈkini ˈwênuwa-la ˈru ba ˈni\\
                like that \textsc{neg} think-\textsc{cond} \textsc{1.poss} parents-\textsc{poss} say.\textsc{prs} \textsc{cl} indeed\\
        \glt    `Like when they say our parents don't think.'\\
        \glt    `Como cuando dicen que no piensan nuestros padres.'    \corpuslink{tx475[08_177-08_207].wav}{SFH tx475:08:17.7}\\
    }\label{ex: pronominal possession kinshipb}
    \z
\z

The use of the possessive suffix appears to be optional when the pronominal possessor is \textit{ˈkini}, as exemplified above, where the possessive suffix is omitted in (\ref{ex: pronominal possession kinshipa}) but attested in the same lexical item in (\ref{ex: pronominal possession kinshipb}). As shown in (\ref{ex: kinship possession options}), both a nominative-marked pronoun and a possessive pronoun can be employed in pronominal possessive constructions involving kinship terms.

\ea\label{ex: kinship possession options}

    \ea[]{
    \textit{ˈkîni apaˈlótʃ͡i}\\
    \gll   ˈkîni apaˈlótʃ͡i\\
            \textsc{1poss} maternal.grandfather\\
    \glt    `my maternal grandfather'\\
    \glt    `mi abuelo materno'   \corpuslink{tx109[00_284-00_321].wav}{LEL tx109:00:28.4}\\
}\label{ex: kinship possession optionsa}
        \ex[]{
        \textit{niˈhê apaˈlótʃ͡ala}\\
        \gll    niˈhê apaˈlótʃ͡a-la\\
                \textsc{1sg.nom} maternal.grandfather-\textsc{poss}\\
        \glt    `my maternal grandfather'\\
        \glt    `mi abuelo materno'    \corpuslink{tx109[00_468-00_531].wav}{LEL tx109:00:46.8}\\
    }\label{ex: kinship possession optionsb}
    \z
\z

While the possessive suffix is omitted with the possessive pronominal in (\ref{ex: kinship possession optionsa}), it is attested, and in fact, required, with the nominative-case marked pronominal possessor in (\ref{ex: kinship possession optionsb}).

In other Rarámuri varieties, a full set of dedicated possessive pronouns for kinship terms are documented: in \ili{Norogachi Rarámuri}, \citet{brambila1976diccionario} reports possessive pronouns that contrast both number and person for speech act participants (\textit{kene} `1\textsc{sg.poss}', \textit{kemu} `2\textsc{sg.poss}', \textit{keti} `1\textsc{pl.poss}', and \textit{ketumu} `2\textsc{pl.poss}'), with a single form encoding third person possessors (\textit{kepu} `3\textsc{poss}') (\citeyear[246--248]{brambila1976diccionario}). In contrast, only the first person possessor is encoded with a dedicated pronominal form in Choguita Rarámuri.

\subsection{Appositive possessive constructions}
\label{subsubsec: appositive possessive constructions}

In addition to the head-marking strategy for marking possession, Choguita Rarámuri has an alternative way of expressing a possession relationship through an appositional construction with \textit{ˈníwa}, which may be used predicatively as a verb meaning `to have', but which behaves as a grammatically specialized possessive noun in possessive constructions \textit{ˈníwa} (glossed below as `have'). The root \textit{ˈníwa} as a possessive noun is marked with the possessive \textit{-lâ} suffix and is part of a possessive phrase. These constructions are exemplified in (\ref{ex: periphrastic construction niwa}):

\ea\label{ex: periphrastic construction niwa}
{Appositive possessive noun construction}\\

    \ea[]{
    {\textit{niˈhê ˈníala  ˈét͡ʃi  koˈbísi}}\\
    \gll    neˈhê  níwa-lâ  ˈét͡ʃi  koˈbísi\\
            1\textsc{sg.nom}  have-\textsc{poss}  \textsc{dem}  pinole\\
    \glt    ‘It is mine that pinole.’\\
    \glt    ‘Es mío ese pinole.’  < FMF 09 3:32/el >\\
}
        \ex[]{
        {\textit{ˈkúmi   buˈʔí   ˈnè   ˈníala   wiˈt͡ʃá}}\\
        \gll    ˈkúmi  bu'ʔi  ˈnè  ˈníwa-lâ  wit͡ʃá\\
                where  lie.\textsc{sg} 1\textsc{sg.nom} have-\textsc{poss}  needle\\
        \glt    `Where is my needle?'\\
        \glt    ‘¿Dónde está mi aguja?’    < BFL 11/07/09/el >\\
    }
            \ex[]{
            {\textit{nà     ko   ˈnè     ˈníala   ˈlîbro   ko}}\\
            \gll    na=ko  ˈnè  ˈníwa-lâ  ˈlîbro  ko\\
                    \textsc{prox}=\textsc{emph}  \textsc{1sg.nom} have-\textsc{poss} book  \textsc{emph}\\
            \glt    ‘This here is my book.’\\
            \glt    ‘Este es mi libro.’    < BFL 06 4:187-189/el >\\
        }
    \z
\z

The nouns in these constructions with the possessive noun \textit{ˈníwa} have not been documented with the head-marking posessive strategy, which suggests this constitutes a class of non-possessible nouns in Choguita Rarámuri. Further evidence of the difference between the head-marking possessive vs. possessive noun strategy is shown in the following example, where the possessed noun \textit{saʔˈpá} ‘flesh’ appears with head-marking when inalienably possessed (\ref{ex: alienable and inalienable possessiona}), but appears with the possessive noun construction in the alienable possession reading (\ref{ex: alienable and inalienable possessionb}) (where `meat' refers to meat severed from a cow).

\ea\label{ex: alienable and inalienable possession}
{Inalienable vs. alienable possession}\\

    \ea[]{
    {\textit{niˈhê    saʔˈpála}}\\
    \gll    neˈhê  saʔˈpá-lâ\\
            1\textsc{sg.nom}  meat-\textsc{poss}\\
    \glt    ‘my flesh’ \\
    \glt    ‘mi carne (de mi cuerpo)’ < FMF 09 3:32/el >\\
}\label{ex: alienable and inalienable possessiona}
        \ex[]{
        {\textit{niˈhê   ˈníala   saʔˈpá}}\\
        \gll    neˈhê  ˈníwa-lâ  saʔˈpá\\
                1\textsc{sg.nom}   have-\textsc{poss}  meat\\
        \glt    ‘my meat (to eat)’\\
        \glt    ‘mi carne (para comer)’ < FMF 09 3:32/el >\\
    }\label{ex: alienable and inalienable possessionb}
    \z
\z

These constructions resemble minimal appositive possessive constructions as documented in \ili{Ainu} (\citealt{bugaevaappositive}; Johanna Nichols, p.c.).

% datos del capítulo de morfología nominal
