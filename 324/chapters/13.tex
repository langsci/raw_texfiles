\chapter{Basic clause types}
\label{chap: basic clause types}

This chapter describes several specialized basic clause types in Choguita Rarámuri in terms of their internal structure and the type of predicate heading each clause type. A first distinction is made between clauses headed by verbal predicates and those headed by nonverbal predicates. Clauses headed by verbal predicates are described in terms of their argument structure, including the distinction between intransitive and transitive clauses. Non-verbal clauses are described in terms of whether they are headed by nominal or locative predicates. The basic Choguita Rarámuri clause consists of a predicate and the arguments it subcategorizes for. Core arguments take the form of noun phrases, whether nouns (with or without modifiers), or personal pronouns  (see \chapref{chap: noun phrases}). From these, only pronominal forms encode core grammatical relations (i.e., there is no case marking in nominals). Given that noun phrases can be elided, a minimal clause in Choguita Rarámuri consists of an inflected verbal predicate. The classification of clauses into intransitive, transitive and ditransitive is thus based on their potential syntactic configuration, even if core arguments are not overtly expressed in the clause.

This chapter is organized according to both structural and functional characteristics of clause types. Clauses headed by verbal predicates are described in §\ref{sec: verbal predicates}. The morphosyntactic properties of core grammatical relations in basic verbal clauses is discussed in §\ref{subsec: basic clause types and transitivity properties}. The basic properties of intransitive clauses are described in §\ref{subsec: intransitive clauses}, those of monotransitive clauses are addressed in §\ref{subsec: transitive clauses}, while ditransitive clauses are described in §\ref{subsec: ditransitive clauses}. Clauses headed by non-verbal predicates are addressed in §\ref{sec: nonverbal predicates}. A brief description of different types of copulas is given in §\ref{subsec: types of copulas}; copular clauses involving nominal predicates are described in §\ref{subsec: copular clauses headed by nominal predicates}; clauses headed by locative predicates are described in §\ref{subsec: locative clauses}. The chapter concludes in §\ref{subsec: predicates of possession} with a description of existential clauses headed by predicates of possession.

\section{Verbal clauses}
\label{sec: verbal predicates}

This section describes the properties of intransitive, monotransitive and ditransitive clauses. Before turning to each type of predicate, a description of the morphosyntactic criteria that allow identifying core grammatical relations in Choguita Rarámuri is provided first.

\subsection{Basic clause types and transitivity properties}
\label{subsec: basic clause types and transitivity properties}

Clauses are structurally distinguished by predicate type and valency. Determining a particular predicate's valency is, however, not trivial, since there is no case marking of core grammatical functions in nouns, and only pronominal forms exhibit a distinction for subject and object marking. Some morphosyntactic criteria that help identify core grammatical relations are summarized in (\ref{ex:13:morphosyntactic properties of core grammatical relations}):

\ea\label{ex:13:morphosyntactic properties of core grammatical relations}
{The morphosyntactic properties of core grammatical relations in Choguita Rarámuri}\\

    \ea[]{
    First and second person subjects are encoded by subject-marked pronouns.\\
}\label{ex:13:morphosyntactic properties of core grammatical relationsa}
        \ex[]{
        First and second person objects are encoded by object-marked pronouns.\\
    }\label{ex:13:morphosyntactic properties of core grammatical relationsb}
            \ex[]{
            The subject of a basic intransitive clause corresponds to the direct object of a derived transitive clause when the verb undergoes a causative (valence increasing) operation.\\
        }\label{ex:13:morphosyntactic properties of core grammatical relationsc}
                \ex[]{
                The new participant introduced through a causative (valence increasing) operation is encoded as the subject.\\
            }\label{ex:13:morphosyntactic properties of core grammatical relationsd}
                    \ex[]{
                    A valence decreasing operation (passive) will demote the subject, direct object or recipient argument (primary object in ditransitive clauses) to a peripheral status (the adjunct argument appears in a postpositional phrase or is completely omitted). \\
                }\label{ex:13:morphosyntactic properties of core grammatical relationse}
                        \ex[]{
                        Benefactive participants in ditransitive clauses (basic or derived through applicativization) may be encoded by object pronouns.\\
                    }\label{ex:13:morphosyntactic properties of core grammatical relationsf}
    \z
\z

\tabref{tab:pronouns-2} and \tabref{tab:enclitics-2} below summarize the subject/object distinctions of free pronominal forms and their enclitic counterparts (free subject and object pronouns are given in parenthesis). For the description of the morphological properties of pronominal forms, see \chapref{chap: particles, adverbs and other word classes}.

%\break

\begin{table}
\caption{Free personal pronouns}
\label{tab:pronouns-2}

\begin{tabularx}{.5\textwidth}{lll}
\lsptoprule
& \textbf{Subject} & \textbf{Object}\\
\midrule
 \textsc{1sg} & neˈhê, ˈnè & taˈmí\\
 \textsc{2sg} & muˈhê, ˈmò & ˈmí\\
 \textsc{1pl} & tamuˈhê, taˈmò & taˈmò\\
 \textsc{2pl} & ˈémi & ˈmí\\
\lspbottomrule
\end{tabularx}
%\hspace{3cm}
\end{table}

%\hspace{3cm}

\begin{table}
\caption{Pronominal enclitic forms}
\label{tab:enclitics-2}

\begin{tabularx}{.5\textwidth}{lll}
\lsptoprule
& \textbf{Subject} & \textbf{Object}\\
\midrule
 \textsc{1sg} & =ni (neˈhê, ˈnè) & (taˈmí)\\
 \textsc{2sg} & =mi (muˈhê, ˈmò) & (ˈmí)\\
 \textsc{1pl} & =ti (tamuˈhê, taˈmò) & (taˈmò)\\
 \textsc{2pl} & =timi (ˈémi) & (ˈmí)\\
\lspbottomrule
\end{tabularx}
\end{table}
%\hspace{1cm}

The forms in (\ref{ex:13:example of morphosyntactic properties}) illustrate properties (\ref{ex:13:morphosyntactic properties of core grammatical relations}a--d): in the causative construction in (\ref{ex:13:example of morphosyntactic propertiesc}), the object corresponds to the subject of its basic, non-causative counterpart (\ref{ex:13:example of morphosyntactic propertiesa}) and its applicative, non-causative counterpart (\ref{ex:13:example of morphosyntactic propertiesb}) (property (\ref{ex:13:morphosyntactic properties of core grammatical relationsc})); the causer argument introduced through the valence increasing derivation is case marked as subject (property (\ref{ex:13:morphosyntactic properties of core grammatical relationsd})).


\ea\label{ex:13:example of morphosyntactic properties}
{Morphosyntactic marking in a basic clause and related causative construction}

\largerpage
    \ea[]{
    \textit{Basic construction}\\
    \textit{aʔˈlì ˈkútʃ͡i iˈwé ko ˈmá tʃ͡í biˈlá ˈtʃ͡ó ˈnàri koˈbîria ˈtʃ͡ó, riˈmêa, moˈlêra, wiˈtʃ͡ôa ma}\\
    \gll    aʔˈlì ˈkútʃ͡i iˈwé=ko ˈmá tʃ͡í biˈlá ˈtʃ͡ó ˈnàri koˈbî-ri-a ˈtʃ͡ó \textbf{riˈmê-a} moˈlêr-a wiˈtʃ͡ô-a ma\\
            and young girls=\textsc{emph} like all really also \textsc{dem} pinole-\textsc{vblz-prog} also {make.tortillas-\textsc{prog}} grind-\textsc{prog} wash.clothes-\textsc{prog} also\\
    \glt    `And the girls also (learn how to do) many things, making pinole, making tortillas, grinding, washing, too.'\\
    \glt    ‘Y las niñas también (aprenden a hacer) muchas cosas, a hacer pinole, hacer tortillas, moler, lavar también.’\\ \corpuslink{tx73[00_460-00_551].wav}{LEL tx73:0:46.0}\\
}\label{ex:13:example of morphosyntactic propertiesa}
        \ex[]{
        \textit{Applicative (non-causative construction)}\\
        {\textit{ˈnè}} {\textit{mi}} \textit{riˈmênira}\\
        \gll     \textbf{ˈnè=mi}    {reˈmê-ni-ra} \\
                {\textsc{1sg.nom=2sg.acc}} {make.tortillas\textsc{{}-appl-pot}} \\
        \glt    `I can make tortillas for you.’\\
        \glt    `Yo te hago tortillas.’  < BFL 08 1:161/el >\\
    }\label{ex:13:example of morphosyntactic propertiesb}
        \ex[]{
        \textit{Causative construction}    \\
        {\textit{ˈmín}} \textit{ˈnè   onoˈlâ riˈmêen}\textit{ti}\textit{ma}\\
        \gll    \textbf{ˈmí=ni}  ˈnè  ono-ˈlâ  {riˈmê-ni-\textbf{ti}{}-ma}  \\
                {\textsc{2sg.acc=1sg.nom}} \textsc{1sg.nom} father\textsc{{}-poss} {make.tortillas\textsc{{}-appl-}{\textsc{caus}}\textsc{{}-fut.sg}} \\
        \glt    `I will make you make tortillas for my dad.'\\
        \glt    `Te voy a hacer que le hagas tortillas a mi papá.'  < BFL 08 1:161/el >\\
    }\label{ex:13:example of morphosyntactic propertiesc}
    \z
\z

Example (\ref{ex:13:basic-passive example}) illustrates properties (\ref{ex:13:morphosyntactic properties of core grammatical relations}e--f): in the basic construction in (\ref{ex:13:basic-passive examplea}), the benefactive argument of an active ditransitive verb is encoded by an object pronoun (property (\ref{ex:13:morphosyntactic properties of core grammatical relationsf})); in the passive correlate of this basic construction, the agent argument of the ditransitive verb is demoted and omitted (property (\ref{ex:13:morphosyntactic properties of core grammatical relationse})).


\largerpage
\ea\label{ex:13:basic-passive example}

    \ea[]{
    \textit{Basic construction}\\
    \textit{ˈmò     ˈjéla} {\textit{taˈmí} } \textit{haˈré  gaˈjêta  ˈà}\textit{ko}\\
    \gll    ˈmò    ˈjé-la    \textbf{taˈmí}    haˈré  gaˈjêta  ˈà-ki-o\\
            2\textsc{sg.nom} mother-\textsc{poss} {\textsc{1sg.acc}} some  cookie  give-\textsc{pst.ego-ep}\\
    \glt    `Your mom gave me some cookies.'\\
    \glt    `Tu mamá me dió unas galletas.'   < BFL 09 1:89/el >\\
}\label{ex:13:basic-passive examplea}
        \ex[]{
        \textit{Passive construction}\\
        \textit{aˈjéna,   ˈpé   ˈtáa   ka}{\textit{n}} \textit{ˈhônsa   ˈtʃ͡íri    ˈà}\textit{ari}\\
        \gll    aˈjéna  ˈpé  ˈtá  ka=\textbf{ni}      ˈhônsa  ˈétʃ͡i  ˈjíri  ˈà-ru\\
                yes  little  small  \textsc{cop.irr}={1\textsc{sg.nom}}  since  \textsc{dem} kind  give-\textsc{pst.pass}\\
        \glt    ‘Yes, since I was very small I was given that.’\\
        \glt    ‘Sí, desde chiquito me dieron eso.’  < FLP 06 in61(441)/in >\\
    }\label{ex:13:basic-passive exampleb}
    \z
\z

The next subsections review the basic properties of intransitive (§\ref{subsec: intransitive clauses}), transitive (§\ref{subsec: transitive clauses}) and ditransitive (§\ref{subsec: ditransitive clauses}) clauses.

\subsection{Intransitive clauses}
\label{subsec: intransitive clauses}

Choguita Rarámuri intransitive verbs are monovalent verbs that subcategorize for a single, nominative case argument. As mentioned above, morphological case marking of core grammatical relations is only evident in pronominal forms, with dedicated subject and object forms. The following examples illustrate intransitive clauses with pronominal arguments (\ref{ex:13:intransitive clauses}a--c) and noun phrase subject arguments (\ref{ex:13:intransitive clauses}d--e). Subject arguments are indicated in square brackets.

\newpage

\ea\label{ex:13:intransitive clauses}
{Intransitive clauses}

\ea[]{
    \textit{neˈhê} \textit{ˈpé     oˈkwâ   raʔiˈtʃ͡áma   koriˈmá   ˈhítara}\\
    \gll    [\textbf{neˈhê}]\textsubscript{\textsc{subj}} ˈpé     oˈkwâ   raʔiˈtʃ͡á-ma   koriˈmá   ˈhítara\\
            {1\textsc{sg.nom}} just   couple   speak-\textsc{fut.sg}   fire.bird about\\
    \glt    `I’ll speak a little (lit. just a couple) about the \textit{korima}, the fire bird.'\\
    \glt    `Yo voy a hablar poquito del pájaro \textit{korimá}.'\corpuslink{tx5[00_229-00_264].wav}{LEL tx5:0:22.9}\\
}
        \ex[]{
        \textit{naˈlìna ho ˈhíp ko ˈmá βiˈlá  ke ˈmé wiˈtʃ͡íwa  aˈlé pa}\\
        \gll    ˈnà aˈlì ˈnà=ko ˈhípi=ko ˈmá biˈlá  ke ˈmé wiˈtʃ͡íwa  aˈlé pa\\
                \textsc{dem} and \textsc{dem=emph} today=\textsc{emph} anymore indeed \textsc{neg} almost believe.\textsc{prs} \textsc{dub} \textsc{cl}\\
        \glt    `And now today we don’t believe anymore.'\\
        \glt    `Y ahora ya no creemos.'  \corpuslink{tx905[01_471-01_510].wav}{GFM tx905:1:47.1}\\
    }
            \ex[]{
            \textit{aʔˈlì   ˈhípi   ko   biˈlá}\textit{timi} \textit{ke   biˈlé   ˈpé   ˈtâʃi  tʃ͡iriˈká}  \textit{ˈéni} {\textit{ˈémi} } \textit{ko   ba    aˈní   bí}\\
            \gll        aʔˈlì  ˈhípi=ko  beˈlá=[\textbf{timi}]\textsubscript{\textsc{subj}} ke  biˈlé  ˈpé  ˈtâsi  ˈétʃ͡i  riˈká  ˈéna ˈémi=ko  ba  aˈní  bi\\
                        and  now=\textsc{emph}  really={\textsc{2pl.nom}} \textsc{neg}  one just  \textsc{neg}  \textsc{dem} like go.\textsc{pl} \textsc{2pl.nom=emph} \textsc{cl} say.\textsc{prs} \textsc{emph}\\
            \glt    `And now you all do not live (go around) like that, you all.' \\
            \glt    `Y ahora ustedes no andan así, ustedes.'   \corpuslink{tx12[03_427-03_460].wav}{SFH tx12:3:42.7}\\
        }
                \ex[]{
                \textit{tʃ͡aˈbè     kiˈʔà         ná} \textit{biˈlé   koriˈmá} \textit{naˈwà}\textit{{}-li        biˈlé-na  bitiˈtʃ͡í} \\
                \gll    tʃ͡aˈbè  kiˈʔà  ná \textbf{[biˈlé} \textbf{ koriˈmá}]\textsubscript{\textsc{subj}} naˈwà{}-li biˈlé-na  bitiˈtʃ͡í\\
                        before long.ago  there {one} {fire.bird} arrive-\textsc{pst} one-\textsc{incl} house \\
                \glt    ‘Long time ago, one fire bird (\textit{korima}) arrived there at one house.’\\
                \glt    ‘Hace mucho tiempo llegó un pájaro de fuego (\textit{korimá}) en una casa.’ \corpuslink{tx5[00_264-00_307].wav}{LEL tx5:0:26.4}   \\
            }
        %\break
                    \ex[]{
                    \textit{we   tʃ͡aˈbèe kiˈʔà   ko   ne   biˈlá   raˈsíra   naˈtêami   ˈníli} \textit{ˈétʃ͡i ˈnàri   raˈʔìtʃ͡iri} \textit{aˈlé}   \\
                    \gll    we  tʃ͡aˈbè kiˈʔà=ko  ne  beˈlá  raˈsíra  naˈtê-ame  ˈní-li [\textbf{étʃi} \textbf{ˈnàri} \textbf{raˈʔìtʃa-li}]\textsubscript{\textsc{subj}}\\
                            \textsc{int} before  long=\textsc{emph} \textsc{int} indeed  more  have.value-\textsc{ptcp} \textsc{cop-pst} {\textsc{dem}} {this}  {speak-\textsc{nmlz}}\\
                    \glt    ‘Long time ago, this talk was very important.’\\
                    \glt    ‘Antes era más valiosa la plática.’  \corpuslink{tx817[00_000-00_086].wav}{JMF tx817:0:00.0}\\
                }
    \z
\z

Intransitive clauses may be headed by verbs that have been derived through a valence-decreasing operation, such as the passive construction exemplified in (\ref{ex:13:passive example}).

\ea\label{ex:13:passive example}

\textit{ˈtòoru    grabaˈdôra}\\
\gll    ˈtò-ru    [grabaˈdôra]\textsubscript{\textsc{subj}}\\
        take-\textsc{pst.pass}  recorder\\
\glt    `The recorder was taken.'\\
\glt    `Se llevaron la grabadora.'  < SFH 08 1:45/el >\\

\z


Intransitive clauses tend to display a verb final order, but, as shown in (\ref{ex:13:passive example}), there are no strict restrictions about order of verb and noun phrase in these clauses.

\subsection{Transitive clauses}
\label{subsec: transitive clauses}

Predicates heading transitive clauses subcategorize for a subject noun phrase (marked as subject if pronominal) and a single object noun phrase (marked as object if pronominal). The forms in (\ref{ex:13:transitive clauses}) illustrate transitive clauses where the single object noun phrase is indicated in square brackets.

\ea\label{ex:13:transitive clauses}
{Transitive clauses}

    \ea[]{
    \textit{baʔaˈrîni} \textit{ˈmí} \textit{ˈáma}\\
    \gll    baʔaˈrî=ni    [\textbf{ˈmí}]\textsubscript{\textsc{obj}}    ˈá-ma\\
            tomorrow=\textsc{1sg.nom} {\textsc{2sg.acc}} look.for-\textsc{fut.sg}\\
    \glt    `I’ll look for you tomorrow.'\\
    \glt    `Mañana te busco.' < LEL 09 1:70/el >\\
}
        \ex[]{
        \textit{aʔˈlì   tʃ͡iˈhônsa   ˈétʃ͡i   riˈhò   baˈhîsa  ˈká} \textit{ˈétʃ͡i   tʃ͡oʔˈmá}\\
        \gll    aʔˈlì  tʃ͡iˈhônsa  ˈétʃ͡i  riˈhòi  baˈhî-sa    ˈká [\textbf{ˈétʃi}   \textbf{tʃoʔˈmá}]\textsubscript{\textsc{obj}}  \\
                and  then    \textsc{dem}  man  drink-\textsc{cond} \textsc{irr} {\textsc{dem}} {snot}\\
        \glt    `and if the man would have drank the snot'\\
        \glt    `y si el hombre hubiera tomado el moco' < SFH 06 choma(21)/tx >\\
    }
            \ex[]{
            \textit{matʃ͡iˈbô   biˈláti ˈlá  tamuˈhê   na} \textit{ˈétʃ͡i   muˈhê} \textit{ˈnâtara   ˈnápumi   riˈká   ˈhônsa   oˈtʃérili   muˈhê}\\
            \gll    matʃ͡i-ˈbô  beˈlá=ti oˈlá  tamuˈhê  na  [\textbf{ˈétʃi}  \textbf{muˈhê} \textbf{ˈnâta-ra}  \textbf{ˈnápi=mi} \textbf{reˈká} \textbf{ˈhônsa} \textbf{oˈtʃéri-li} \textbf{muˈhê}]\textsubscript{\textsc{obj}}\\
                    know-\textsc{fut.pl}  really=1\textsc{pl.nom}  \textsc{cer} 1\textsc{pl.nom} \textsc{dem} {\textsc{dem}} {2\textsc{sg.nom}} {think-\textsc{poss}} {\textsc{sub}=2\textsc{sg.nom}} {like}  {since}  {grow-\textsc{pst}} {2\textsc{sg.nom}}\\
            \glt    `We will know your thoughts of the time when you were growing up.’\\
            \glt    `Nosotros vamos a saber tus pensamientos desde que tu creciste.'  \corpuslink{in61[00_221-00_279].wav}{SFH in61:0:22.1} \\
                }
                        \ex[]{
                        \textit{ˈhê  riˈgá   biˈláni ˈmí   ruˈwèma ˈétʃ͡i   suˈwâbuka paˈgótami}\\
                        \gll    ˈhê reˈká  beˈlá=ni    ˈmí  ru-ˈè-ma   [\textbf{étʃi}  \textbf{suˈwâbika} \textbf{paˈgótame}]\textsubscript{\textsc{obj}}\\
                                \textsc{dem}  like  indeed=\textsc{1pl.nom} \textsc{dist} say-\textsc{appl-fut.sg} {\textsc{dem}} {all}  {people}\\
                        \glt    `That is how it is over there, that is what I will tell all the people.'\\
                        \glt    `Así es por allí, así les voy a decir a toda la gente.'   \corpuslink{tx817[01_100-01_129].wav}{JMF tx817:1:10.0}  \\
                    }
                            \ex[]{
                            \textit{apaʔˈlì    itʃ͡iˈrûa  wiliˈsâ ˈká  ˈnàri} \textit{paˈt͡ʃí   \textit{muˈrúʃia          ruˈwá    ma}} \\
                            \gll    ˈnápi aʔˈlì itʃ͡i-ˈrû-a    wili-ˈsâ     ˈká   ˈnàri   [\textbf{paˈtʃí}]\textsubscript{\textsc{obj}} muˈrú-si-a ru-ˈwá ma  \\
                                    \textsc{sub}   and   plant-\textsc{pst.pass-prog}   stand-\textsc{cond}  \textsc{irr}   then  {corn} carry.in.arms-\textsc{mot-prog}   say-\textsc{mpass}  also \\
                            \glt    `And when it was sown (standing plant corn) they say the corn was taken.'\\
                            \glt    `Y cuando está sembrado el maíz era llevado, dicen.' \corpuslink{tx32[01_347-01_412].wav}{LEL tx32:1:34.7} \\
                        }
    \z
\z

Transitive clauses may be derived through valence increasing operations, such as causative, transitive and applicative, applying to intransitive predicates, as exemplified in (\ref{ex:13:example of morphosyntactic properties}) above. Transitive predicates may also take clausal complements (for a discussion of the syntax of complementation cross-linguistically, see \citealt{cristofaro2005subordination} and \citealt{noonan2007complementation}). Examples of complement-taking transitive verbs are given in (\ref{ex:13:complement-taking transitive verbs}) (in these examples, complement-taking verbs are in boldface and complement clauses in square brackets).

%\break

\ea\label{ex:13:complement-taking transitive verbs}
{Complement-taking transitive verbs}

    \ea[]{
    \textit{aʔˈlì a ˈjá   saˈjèli} \textit{naˈwàatʃ͡i ˈétʃ͡i   koriˈmá  paˈtʃ͡á  bitiˈtʃ͡í  baˈkíli}\\
    \gll    aʔˈlì  a ˈjá  \textbf{saˈjè-li} [naˈwà-a-tʃ͡i  ˈétʃ͡i  koriˈmá  paˈtʃ͡á  bitiˈtʃ͡í  baˈkí-li]\\
            and \textsc{aff} soon feel-\textsc{pst}   arrive-\textsc{prog-loc}   \textsc{dem} fire.bird inside house go.in.\textsc{sg}{}-\textsc{pst}\\
    \glt    `And then they felt when the fire bird arrived and when he went inside the house.’\\
    \glt    `Y luego sintieron cuando llegó el \textit{pájaro de fuego} y cuando entró adentro a la casa.’  \corpuslink{tx5[00_428-00_483].wav}{LEL tx5:00:42.8}  \\
}
        \ex[]{
        \textit{ˈnè   ko} \textit{ˈtʃ͡é  ˈémi    raˈpâko  ku  ʃiˈmêo} \textit{maˈjê}  \\
        \gll    ˈnè=ko  [ˈtʃ͡é  ˈémi    raˈpâko  ku  si-ˈmêa]   \textbf{ maˈjê}\\
                \textsc{1sg.nom}=\textsc{emph} again  \textsc{2pl.acc} yesterday \textsc{rev} go.\textsc{sg-fut.sg} believe\\
        \glt    `I thought you guys would come back yesterday.'\\
        \glt    `Yo creía que iban a llegar ayer.'   < BFL 09 1:57/el >  \\
    }
    \z
\z

Complement clauses in Choguita Rarámuri are described in §\ref{sec: complement clauses}.

\subsection{Ditransitive clauses}
\label{subsec: ditransitive clauses}

% Note that core grammatical roles need to be described in different terms (A, P, O)

%Specify here whether only one argument can be marked as object - is it possible to enode two objects with pronominal forms in a ditransitive clause

Ditransitive clauses in Choguita Rarámuri subcategorize for three core arguments, one subject noun phrase (case-marked as nominative if pronominal), and two objects, a recipient-like (``R'') argument and a theme-like (``T'') argument (with Theme defined as ``something which undergoes a change in location or to which a location is attributed" \parencite{dryer2007clause}). As the following examples show, recipient (R) arguments are marked as objects, patterning together with single object arguments of monotransitive verbs: in (\ref{ex:13:ditransitive clauses}a--b), the recipient argument is encoded through the object pronoun (\textit{taˈmí} `\textsc{1sg.acc}'). In contrast, the theme (T) argument, if pronominal, is encoded with the subject pronoun (\textit{ˈémi} `\textsc{2pl.nom}').\footnote{This contrasts with the case marking pattern found in closely-related \ili{Yaqui} (\ili{Taracahitan}), where some ditransitive verbs license accusative marking on both a theme and beneficiary arguments \citep{guerrero2004yaqui}.}

\ea\label{ex:13:ditransitive clauses}
{Ditransitive clauses}\\

    \ea[]{
    \textit{riwiˈrítʃ͡i biˈtêami tʃ͡e taˈmí ˈémi ˈàki}\\
    \gll    [riwiˈrítʃ͡i  biˈte-ame]\textsubscript{\textsc{subj}} tʃ͡e [\textbf{taˈmí}]\textsubscript{\textsc{R}} [ˈémi]\textsubscript{\textsc{T}} ˈà-ki\\
            Tewirichi inhabit\textsc{.sg-ptcp} tʃ͡e  {\textsc{1sg.acc}} \textsc{2pl.nom} give-\textsc{pst.ego}\\
    \glt    `People from Tewirichi gave you guys to me (for adoption).'\\
    \glt    `Unos de Tewirichi me los dieron a ustedes (en adopción).' < BFL 09 1:89/el >\\
}
        \ex[]{
        {\textit{ˈmò ˈjêla} \textit{taˈmí}  \textit{haˈré  gaˈjêta} \textit{ˈà}\textit{ko}}\\
        \gll    [ˈmò ˈjê-la]\textsubscript{\textsc{subj}} [\textbf{taˈmí}]\textsubscript{\textsc{R}} [haˈré  gaˈjêta]\textsubscript{\textsc{T}}    ˈà-k-o\\
                2\textsc{sg.nom} mother-\textsc{poss} {\textsc{1sg.acc}} some  cookie    give-\textsc{pst.ego-ep}\\
        \glt    `Your mom gave me some cookies.'\\
        \glt    `Tu mamá me dió unas galletas.'    < BFL 09 1:89/el > \\
    }
    \z
\z

% But make reference to how the choice of which noun phrase gets the object marking may depend on animacy, topicality, etc.?

This pattern, where recipient arguments of ditransitive predicates and objects of monotransitive predicates pattern together as being object-marked, suggests that Choguita Rarámuri is a primary object language, as argued for other \ili{Uto-Aztecan} languages, e.g., \ili{Huichol} (\ili{Corachol}; \citealt{comrie1982grammatical}) (see also \citealt{dryer1986primary} and \citealt{dryer2007clause}). Other \ili{Uto-Aztecan} languages that have been described as primary object languages include \ili{Cora} (\ili{Corachol}; \citealt{soto2002some}), \ili{Southeastern Tepehuan} (\ili{Tepiman}; \citealt{salido2007voz}), and \ili{Yaqui} (\ili{Taracahitan}; \citealt{felix2000relaciones}, \citealt{guerrero2004yaqui}).

Ditransitive clauses with three non-pronominal core argument noun phrases are very infrequently attested in the Choguita Rarámuri corpus. Documented ditransitive clauses with non-pronominal noun phrases suggest that the order of arguments is S-T(heme)-R(ecipient), with a high degree of flexibility regarding the placement of object noun phrases with respect to the verbal predicate, as exemplified in (\ref{ex:13:order of NPs in ditransitives}).


\ea\label{ex:13:order of NPs in ditransitives}
{Order of nominal noun phrases in ditransitive clauses}\\

    \ea[]{
    {S-T-V-R}\\
    \textit{ˈmí     muˈkî} \textit{ˈdûlse    ˈàli    ˈkûruwi}\\
    \gll    [ˈmí    muˈkî]\textsubscript{subj}    [ˈdûlse]\textsubscript{T}    ˈà-li    [ˈkûruwi]\textsubscript{R}\\
             \textsc{dist}    woman  candy  give-\textsc{pst}  children\\
    \glt    `That woman gave the children candy.'\\
    \glt    `Esa mujer les dió dulces a los niños.' < SFH 09 3:51/el >\\
 }
% \hspace{1cm}
        \ex[]{
        {S-T-V-R}\\
        \textit{kumuˈtê ˈlâmina ˈèbili raˈlàmuli}\\
        \gll    [kumuˈtê]\textsubscript{subj} [ˈlâmina]\textsubscript{T}  ˈèbi-li      [raˈlàmuli]\textsubscript{R}\\
                sheriff  tin.roof  bring.\textsc{appl-pst}  people\\
        \glt    `The sheriff brought tin roof for the people.`\\
        \glt    `El comisariado les trajo lámina a la gente.’  < SFH 09 3:51/el >\\
    }
%\break
            \ex[]{
            {S-V-T-R}\\
            \textit{ˈétʃ͡i muˈkî ko ka tʃ͡e ˈàli liˈmôsna} \textit{ta roˈwéami} \textit{ba}\\
            \gll    [ˈétʃ͡i  muˈkî]\textsubscript{\textsc{subj}}=ko ka tʃ͡e ˈà-li [liˈmôsna]\textsubscript{T} [ta roˈwé-ame]\textsubscript{R} ba \\
                    \textsc{dem} woman=\textsc{emph}  \textsc{neg} \textsc{neg} give-\textsc{pst} present  \textsc{det.sg} run.womens.race-\textsc{ptcp}  \textsc{cl}\\
            \glt    `Those women did not give a present to the (winning) runner.’\\
            \glt    `Esas mujeres no le dieron limosna a la corredora.’ < BFL 09 1:89/el >\\
        }
%    \hspace{1cm}
                \ex[]{
                {{V-T-R}}\\
                \textit{ˈpé aʔˈlá   amiˈnábi   ˈàa     busuˈrêro     ˈtʃ͡ó   ˈlé}  [\textit{ˈkîni   ˈkútʃ͡uwa}] \textit{ˈlîna   ˈnòtʃ͡a     koˈʔá     ˈtʃ͡ó} \\
                \gll    ˈpé aʔˈlá amiˈnábi ˈà-a [busuˈrê-ro]\textsubscript{T} ˈtʃ͡ó aˈlé [ˈkîni  ˈkútʃ͡uwa]\textsubscript{R} ˈlîna ˈnòtʃ͡a koˈʔá ˈtʃ͡ó\\
                        just  well  more  give-\textsc{prog} wake.up-\textsc{nmlz} also \textsc{dub} our children  so.that  work.\textsc{prs} eat.\textsc{prs} also\\
                \glt    `But we must give our children more advice so that they will eat (sustain themselves) from their work.'\\
                \glt    `Pero hay que darle más consejo a nuestros hijos, para que trabajen y coman (de su trabajo).' < SFH 06 in61(713)/in >\\
            }
%    \hspace{1cm}
                    \ex[]{
                    {S-V-T-R}\\
                    \textit{ˈmí     muˈkî} \textit{ˈwé  naˈtêami    ˈàli    aˈsûkar  ˈkûruwi}\\
                    \gll    [ˈmí    muˈkî]\textsubscript{subj}    ˈwé  naˈtê-ame    ˈà-li    [aˈsûkar]\textsubscript{T}  [ˈkûruwi]\textsubscript{R}/\\
                            \textsc{dist}    woman  \textsc{int}  cost-\textsc{ptcp}    give-\textsc{pst}  sugar    children\\
                    \glt    `That woman gave (sold) the sugar to the children for very expensive.’  \\
                    \glt    `Esa mujer les dió (vendió) azúcar muy cara a los niños.'    {< SFH 09 3:52/el >}\\
                }
    \z
\z

As shown in these examples, the theme argument may appear pre-verbally (\ref{ex:13:order of NPs in ditransitives}a--b), or both theme and recipient may appear post-verbally (\ref{ex:13:order of NPs in ditransitives}c--e). On the other hand, in clauses with both pronominal and nominal noun phrases, pronominal arguments tend to be expressed pre-verbally, while non-pronominal arguments tend to be expressed post-verbally: in (\ref{ex:13:order of NPs in ditransitives}a--b, e) above, both theme and recipient noun phrases are non-pronominal and post-verbal, while in (\ref{ex:13:ditransitive clauses with pronominal NPs}) below, pronominal objects are placed pre-verbally (the recipient argument in (\ref{ex:13:ditransitive clauses with pronominal NPs}a--b), and the theme argument in (\ref{ex:13:ditransitive clauses with pronominal NPsc}).

\largerpage
\ea\label{ex:13:ditransitive clauses with pronominal NPs}
{Order of nominal and pronominal elements in ditransitive clauses}

    \ea[]{
    {{S-R-V-T} }\\
    {\textit{ˈétʃ͡i   muˈkî  ko} \textit{ka  tʃ͡e  taˈmí    à}\textit{a    liˈmôsna  ba}}\\
    \gll    [ˈétʃ͡i  muˈkî]\textsubscript{subj}=ko  ka  tʃ͡e  [taˈmí]\textsubscript{R}    à-a    [liˈmôsna]\textsubscript{T}  ba\\
            \textsc{dem} woman=\textsc{emph} \textsc{neg} \textsc{neg} \textsc{1sg.acc} give-\textsc{prog} present \textsc{cl}\\
    \glt    `Those women didn't give me a present.’\\
            `Esas mujeres no me dieron limosna.'  < BFL 09 1:89/el >\\
}\label{ex:13:ditransitive clauses with pronominal NPsa}
%\hspace{1cm}
            \ex[]{
            { {S-R-V-T}}\\
            {\textit{ˈmí  muˈkî    ˈwé  naˈtêami  taˈmí    raʔliˈkí  aˈsûkar}}\\
            \gll    [ˈmí  muˈkî]\textsubscript{S}    ˈwé  naˈtê-ame  [taˈmí]\textsubscript{R}    raʔliˈkí  [aˈsûkar]\textsubscript{T}\\
                    \textsc{dist}  woman  \textsc{int}  cost-\textsc{ptcp} \textsc{1sg.acc}   sell.\textsc{pst}  sugar\\
            \glt    `That woman sold me the sugar at a very high price.'\\
            \glt    `Esa mujer me vendió muy cara el azúcar.'  < SFH 09 3:51/el >\\
    }\label{ex:13:ditransitive clauses with pronominal NPsb}
%\hspace{1cm}
                \ex[]{
                {{S-T-V-R}}\\
                { \textit{niˈhê    ˈmí    ˈâma    miˈgêl}}\\
                \gll    [neˈhê]\textsubscript{S}    [ˈmí]\textsubscript{T}    ˈâ-ma    [miˈgêl]\textsubscript{R}\\
                        1\textsc{sg.nom}  \textsc{2sg.acc} give-\textsc{fut.sg}  Miguel\\
                \glt    ‘I will give you to Miguel (so he can be your godfather).’\\
                \glt    ‘Te voy a dar a Miguel (para que sea tu padrino).’ < SFH 09 3:51/el >\\
            }\label{ex:13:ditransitive clauses with pronominal NPsc}
    \z
\z

\citet{guerrero2004yaqui} propose that, while predominantly a primary{\slash}secondary object language, \ili{Yaqui} also has ditransitive predicates with a direct{\slash}indirect object pattern. Whether Choguita Rarámuri also has ditransitive clauses with a direct/indirect object pattern is a question left for further research.\footnote{As discussed in \citet{guerrero2004yaqui}, \ili{Yaqui} exhibits a direct\slash indirect object pattern given the existence in the language of two different case markers that allows this marking. An anonymous reviewer suggests that given the fact that Choguita Rarámuri lacks case marking in nouns, a similar pattern may not be possible.}


\section{Locative, copula and existential clauses}
\label{sec: nonverbal predicates}
\largerpage

Cross-linguistically, clauses with nonverbal predicates whose properties exhibit variation across languages include those involving nominal predicates, adjectival predicates and locative predicates \parencite{dryer2007clause}. Given that the class of basic, underived adjectives in Choguita Rarámuri is very small (see \chapref{chap: particles, adverbs and other word classes}, §\ref{sec: adjectives}), this section focuses on clauses headed by nominal and locative predicates. This section also addresses existential clauses headed by predicates of possession. Before turning to these clauses, a brief description of types of copulas in Choguita Rarámuri is provided next.

\subsection{Types of copulas}
\label{subsec: types of copulas}
Copular clauses are a clause type in which the contentful predicate is not a verb, but some other category like an adjective, a noun or a preposition. Copular clauses in Choguita Rarámuri have an internal structure that includes a copula verb, a subject nominal phrase and an unmarked complement. There are four suppletive stems of copula verbs corresponding to different tense/aspect distinctions. These are: (i) the copula \textit{hu} for present tense; (ii) the copula \textit{ke} for past imperfective; (iii) the copula \textit{ka} for irrealis; and (iv) the copula \textit{ni-} for all other tense/aspect distinctions. Specifically, the present, past imperfective and irrealis copula verbs do not attach any tense/aspect or mood suffixes, but the copula \textit{ni-} is a bound root that requires suffixation. \tabref{tab:key:25} summarizes the set of Choguita Rarámuri suppletive copular verbs, their gloss and approximate translation.



\clearpage

\begin{table}
\caption{Choguita Rarámuri copula verbs}
\label{tab:key:25}
\begin{tabularx}{\textwidth}{XXXX}
\lsptoprule
\textbf{Tense/aspect} & \textbf{Form}  & \textbf{Gloss} & \textbf{Translation}\\
\midrule
Present & \textit{hu} & \textsc{cop.prs} & `is’\\
Past imperfective & \textit{ke} & \textsc{cop.impf} & `used to be’\\
Irrealis & \textit{ka} & \textsc{cop.irr} & `would/might be'\\
Other TAM  & \textit{ní-} & \textsc{cop-} & \\
\lspbottomrule
\end{tabularx}
\end{table}

Examples of the bound root \textit{ni-} copula with different TAM suffixes is provided in (\ref{ex:13:inflected copula ni-}).

\ea\label{ex:13:inflected copula ni-}
{Examples of inflected copula verb \textit{ni-}}

    \ea[]{
    \textit{ˈníli}\\
    {\textit{ˈní-li}} \\
    {\textsc{cop-pst}}\\
    {‘it was’}\\
}
        \ex[]{
        \textit{ˈníra}\\
        {\textit{ˈní-ra}} \\
        {\textsc{cop-rep}} \\
        {`they say it was’}\\
    }
            \ex[]{
            {\textit{ˈníma}}\\
            {\textit{ˈní-ma}} \\
            {\textsc{cop-fut.sg}} \\
            {`(it/she/he) will be’}\\
        }
                \ex[]{
                \textit{ˈníbo}\\
                {\textit{ˈní-bo}}\\
                {\textsc{cop-fut.pl}}\\
                {‘(they/we) will be’}\\
            }
                    \ex[]{
                    \textit{ˈnísa}\\
                    {\textit{ˈní-sa}} \\
                    { \textsc{cop-cond}} \\
                    {`if it were'}\\
                }
    \z
\z

The list of inflected forms of the copula verb \textit{ni-} in (\ref{ex:13:inflected copula ni-}) is not exhaustive, since there are no restrictions as to which TAM markers can attach to this verb. Examples of these copula verbs in context are provided in the sections below.

\subsection{Clauses headed by nominal predicates}
\label{subsec: copular clauses headed by nominal predicates}

In copular clauses headed by nominal predicates, the characteristic of the subject is a noun phrase. The examples in (\ref{ex:13:copular clauses headed by nominal predicates}) illustrate copular clauses headed by nominal predicates.


\ea\label{ex:13:copular clauses headed by nominal predicates}
{Copular clauses headed by nominal predicates}

    \ea[]{
    \textit{enferˈmêra} \textit{ˈkéʔe}?\\
    \gll    enferˈmêra  ˈke\\
            nurse    \textsc{cop.impf}\\
    \glt    ‘Did she use to be a nurse?’\\
    \glt    ‘¿Era enfermera?’ < BFL 09 1:85/el >\\
}
        \ex[]{
        \textit{ˈnè     uˈmûala} \textit{ˈníli} \textit{ˈétʃ͡i   ba?}  \\
        \gll    ˈnè    uˈmûa-la    ˈní-li   ˈétʃ͡i  ba\\
                1\textsc{sg.nom}  bisabuelo-\textsc{poss}  {\textsc{cop-pst}}  \textsc{dem}  \textsc{cl}\\
        \glt    ‘Was he my great grandfather?’\\
        \glt    ‘¿Era mi bisabuelo él?’  < SFH 06 in61(117)/in >\\
    }
            \ex[]{
            \textit{ˈhêʔ  ˈnà  sakaˈrá  ko  waˈsía} \textit{hu}\\
            \gll    ˈhê ˈnà  sakaˈrá=ko  wasˈía hu\\
                    \textsc{dem} \textsc{prox}  plant=\textsc{emph} chuchupate  {\textsc{cop.prs}}\\
            \glt    ‘This plant is chuchupate.’\\
            \glt    ‘Esta planta es chuchupate.’ < GMF 09 3:63/el >    \\
        }
                \ex[]{
                \textit{ˈgîltro  ko  komiˈsârio} \textit{ˈhú}\\
                \gll    ˈgîltro=ko  komiˈsârio ˈhú\\
                        Giltro=\textsc{emph}  sheriff  \textsc{cop.prs}\\
                \glt    ‘Giltro is sheriff.’\\
                \glt    ‘Giltro es comisario.’  < GMF 09 3:63/el >\\
            }
    \z
\z

% This was removed because it is likely not attested in naturalistic speech - but should check

Copula verbs may precede the complement in copular clauses, as exemplified in (\ref{ex:13:copula before complement}), but examples of this kind are marginally attested in the corpus (the only attested examples showing this pattern were obtained through translation elicitation using \ili{Spanish}, but are so far unattested in natural discourse).

\ea\label{ex:13:copula before complement}

    \ea[]{
    {\textit{ˈétʃ͡i  ˈgîltro} \textit{ˈhú komiˈsârio}}\\
    \gll    ˈétʃ͡i  ˈgîltro \textbf{\textit{ˈhú}} komiˈsârio\\
            \textsc{dem}  giltro  {\textsc{cop.prs}} sheriff\\
    \glt    ‘Giltro is sheriff.’\\
    \glt    ‘Giltro es comisario.’ < FMF 09 3:57/el >\\
}
        \ex[]{
        {\textit{ˈétʃ͡i  tiˈwé} \textit{ˈníma} \textit{ˈlé  sukuˈrûame}}\\
        \gll    ˈétʃ͡i  tiˈwé \textbf{\textit{ˈní-ma}} aˈlé  sukuˈrû-ame\\
                \textsc{dem}  girl  {\textsc{cop-fut.sg}}  \textsc{dub}  shaman-\textsc{ptcp}\\
        \glt    ‘That girl will be a shaman.’\\
        \glt    `Esa niña va a ser curandera.’  < FMF 09 3:57/el >\\
    }
    \z
\z

As described in \chapref{chap: particles, adverbs and other word classes} (§\ref{sec: adjectives}) and \chapref{chap: clause combining in complex sentences} (§\ref{sec: relative clauses}), property concepts in Choguita Rarámuri are encoded through verbal predicates, which may be nominalized through a participial morpheme. Copular clauses headed by these nominalized forms, exemplified in (\ref{ex:13:copular clauses with nominalized predicates}), exhibit the same internal structure as other copular clauses headed by nominal predicates.

\largerpage[2]
\ea\label{ex:13:copular clauses with nominalized predicates}
{Copular clauses with nominalized predicates}

    \ea[]{
    {\textit{miˈná   biˈlátimi   ˈmá   {oˈtʃérami}} \textit{ˈníbo} \textit{ˈlé}}\\
    \gll    amiˈná  beˈlá=timi    ˈmá    \textbf{oˈtʃéra-ame} ˈní-bo  aˈlé\\
            later  really=\textsc{2pl.nom} already    {grow-\textsc{ptcp}} \textsc{cop-fut.pl} \textsc{dub}\\
    \glt    ‘Later on when you all will be adults.’\\
    \glt    ‘Después ya van a ser crecidos ustedes.’    \corpuslink{tx12[12_367-12_388].wav}{SFH tx12:12:36.7}\\
}
        \ex[]{
         {\textit{aʔˈlì  tamuˈhê    biˈlá  ko  naˈlìna  ˈwé  winoˈmîwam} \textit{ˈníbira} \textit{ˈrái   ba}}\\
        \gll    aʔˈlì   tamuˈhê   beˈlá=ko   naˈlìna   ˈwé   \textbf{winoˈmî-w-ame}    ˈní-bi-la     ru-ˈwá    ba\\
                then  \textsc{1pl.nom}  really=\textsc{emph}  then  \textsc{int}  {money-have-\textsc{ptcp}} \textsc{cop-irr.pl-rep} say-\textsc{mpass}  \textsc{cl}\\
        \glt    ‘Then we would have been the wealthy ones.’\\
        \glt    ‘Entonces nosotros hubiéramos sido los del dinero.’  < SFH 06 choma(25)/tx >\\
    }
            \ex[]{
            {\textit{á   biˈlá   ˈwé   ˈkáira   ro   ˈwé   baˈhîbam} \textit{ˈká}}\\
            \gll    á  beˈlá  ˈwé  ˈkáira  ro  ˈwé  \textbf{baˈhî-wa-ame}    \textbf{ˈká}\\
                    \textsc{aff}  really  \textsc{int}  happy  ro  \textsc{int}  {drink-\textsc{mpass}{}-\textsc{ptcp} } \textsc{cop.irr}\\
            \glt    ‘It feels really good when one is drinking.’\\
            \glt    ‘Se siente uno muy agusto cuando anda tomando.’  < SFH 06 in61(407)/tx >\\
        }
                \ex[]{
                {\textit{we   tʃ͡aˈbèe kiˈʔà   ko   ne   biˈlá   raˈsíra  naˈtêami} \textit{ˈníli} \textit{étʃ͡i  ˈnàri   raˈʔìtʃ͡ili   aˈlé }}\\
                \gll    we  tʃ͡aˈbè kiˈʔà=ko  ne  beˈlá  raˈsíra \textbf{ naˈtê-ame} ˈní-li ˈétʃ͡i  ˈnàri  raˈʔìtʃ͡a-li aˈlé\\
                        \textsc{int} before  before=\textsc{emph} \textsc{int} really  more  {cost-\textsc{ptcp}} \textsc{cop-pst} \textsc{dem} this  speak-\textsc{pst}  \textsc{dub} \\
                \glt    ‘Long ago these words were more valuable.’\\
                 \glt    ‘Antes era más valiosa la plática.’  \corpuslink{tx817[00_000-00_086].wav}{JMF tx817:0:00.0}\\
            }
    \z
\z\clearpage

While most property concepts are encoded through verbal predicates, there is a small set of true adjectives. Copular clauses headed by these adjectival predicates have the same internal structure than copular clauses headed by nominal predicates. This is shown in (\ref{ex:13:copular clauses headed by adjectival predicates}).

\ea\label{ex:13:copular clauses headed by adjectival predicates}
{Copular clauses headed by adjectival predicates}

    \ea[]{
    {\textit{ˈétʃ͡i tiˈwé ˈwé seˈmáti ˈhú }}\\
    \gll    ˈétʃ͡i tiˈwé ˈwé seˈmáti ˈhú \\
            \textsc{dem} girl \textsc{int} pretty \textsc{cop.prs}\\
    \glt    `This girl is very pretty.'\\
    \glt    `Esta niña está muy bonita'.\\
}
        \ex[]{
        {\textit{ˈkíti    ˈwé  ˈrîko    ˈhú     haˈré  ba}}\\
        \gll    ˈkíti     ˈwé   ˈrîko    ˈhú      haˈré   ba\\
                because  \textsc{int}  wealthy  \textsc{cop.prs}  some  \textsc{cl}\\
        \glt    ‘And that is why some are really wealthy.’\\
        \glt    ‘Y por eso unos son muy ricos.’ < SFH 06 choma(30)/tx >\\
        }
    \z
\z

% is this worth bringing up? They are just nominalizations

%\textbf{Position of copula with respect to subject and complement: consistent?}
%\textbf{Marking on complement and subject}

%\subsubsection{Sentential predicates}
%\label{subsubsec: clauses with sentential predicates}

%A third set of copular clauses is headed by sentential predicates. This clause type is exemplified in (\ref{ex:13:copular clauses with sentential predicates}):

%\ea\label{ex:13:copular clauses with sentential predicates}
%{Copular clauses with sentential predicates}

   % \ea[]{
    %\textit{ ˈhê  aní-li     shiné-ami    ˈhê  aní-li:   “ˈjena   a   witʃíari} %\textbf{\textit{hu}} \textit{étʃ͡i   korimá    ko”}
%/ ˈhê aní-li    siné-ame   ˈhê aní-li  aˈjena  á  witʃíari  \textbf{hu} étʃ͡i  korimá   ko/
%it   say-\textsc{pst} all-\textsc{ptcp} it   say-\textsc{pst aff   aff} true     \textbf{\textsc{cop}} \textsc{dem} \textit{korima} \textsc{emph}
 % ‘Y entonces todos dijeron “sí es cierto lo del \textit{korimá}”
  %‘And everybody said: “it is in fact true that about the \textit{korima}”
  %< LEL 06 tx5(63)/tx >
%}
       % \ex[]{
        %\textit{á  witʃía-ro    ˈhú     nápu    napawí-pa    ru-á-o}\\
         %\textit{á  witʃía-ro    ˈhú     nápu    napawí-pa    ru-wá-o}
%        aff  believe-nmlz  cop.prs  comp    gather-fut.pass  say-mpass-ep
 %       ‘It’s true there will be a meeting’
  %      ‘Sí es cierto que va a haber reunión’
   %     < BFL 09 3:68/el >
%    }
 %   \z
%\z

Copular clauses with nominal predicates may express a partitive semantic relation (‘made of’), a relation that is sometimes cross-linguistically encoded with genitive case. This is shown in (\ref{ex:13:copular clauses expressing 'made of'}).

%\break

\ea\label{ex:13:copular clauses expressing 'made of'}
{Copular clauses with partitive meaning}

    \ea[]{
    {\textit{ˈnà   ˈmêsa  ko  kuˈʃì} \textbf{\textit{ˈhú}}}\\
    \gll    ˈnà  ˈmêsa=ko  kuˈsì    \textbf{ˈhú}\\
            \textsc{prox}  table=\textsc{emph} log  {\textsc{cop.prs}}\\
    \glt    `This table is made out of wood.’\\
    \glt    `Esta mesa es de madera.’  < GFM 09 2:98/el >\\
}
        \ex[]{
        {\textit{ˈsîja  ˈûle} \textbf{\textit{ˈhú}}}\\
        \gll    ˈsîja   ˈûle    \textbf{ˈhú}\\
                chair  plastic  {\textsc{cop.prs}}\\
        \glt    `a chair made out of plastic'\\
        \glt    `silla de plástico' < GFM 09 2:98/el >\\
    }
    \z
\z

%What other semantic relations are encoded through copular clauses?

The next section addresses a second type of nonverbal clauses, namely those headed by locative predicates.

\subsection{Clauses headed by locative predicates}
\label{subsec: locative clauses}

Clauses headed by locative predicates in Choguita Rarámuri deploy one of a set of positional/postural verbs. Positional/postural verbs in this language denote positions of entities (both animate and inanimate) dependent of the physical properties of the referent in terms of their inherent or transitional shape, though the use of postural predicates also reveals a grammatical system of noun classification (see \citealt{ameka2007introduction} for a discussion of the semantic and grammatical properties of postural verbs cross-linguistically).

In their basic (morphologically unmarked form), posturals in Choguita Rarámuri are stative, intransitive verbs that denote a particular posture or position of their single argument (which is encoded as the subject). These verbs can be derived to function as change of state predicates, which may be either intransitive or transitive. When intransitive, the change of state into a particular posture is brought about by the entity undergoing the change itself (encoded as the single argument subcategorized for by the verb), and when transitive, the change of posture of the object argument is brought about by a second participant, an agent-like argument encoded by the subject. Derivation from a stative postural verb to a change-of-state predicate generally involves an intransitive/transitive morphological alternation: a stem final \textit{{}-i} suffix for intransitives and a stem final \textit{-a} suffix for transitives, as exemplified in (\ref{ex:13:intransitive-transitive postural verbs}) (see §\ref{subsec: valence alternations} for more details on valence changing processes in Choguita Rarámuri).\footnote{Cognate suffixes to the Choguita Rarámuri intransitive \textit{-i} and transitive \textit{-a} suffixes are widely attested across the \ili{Uto-Aztecan} language family, including in the \ili{Aztecan} branch, in \ili{Cahitan}, \ili{Southern Paiute}, as well as in \ili{Cupan} language varieties \citep[32]{heath1977uto}.}

\ea\label{ex:13:intransitive-transitive postural verbs}

    \ea[]{
    \textit{wiˈlí}\\
    \textit{wiˈl-í}\\
    stand-\textsc{intr}\\
    `S/he stands.'\\
    `Se para.'\\
}
        \ex[]{
        \textit{wiˈlá} \\
        \textit{wiˈl-á} \\
        stand-\textsc{tr}\\
        `S/he stands someone up.'\\
        `Lo para.'\\
    }
    \z
\z

The stative postural predicates and their corresponding intransitive and transitive change-of-state counterparts may also be encoded through suppletive forms. These are exemplified in (\ref{ex:13:suppletive postural verbs}).

\ea\label{ex:13:suppletive postural verbs}
{Positional/postural verbs with different transitivity properties}

    \ea[]{
    \textit{aˈtí} \\
    to.be.sitting.\textsc{sg}\\
    `S/he is sitting.'\\
    `Está sentada.'\\
}\label{ex:13:suppletive postural verbsa}
        \ex[]{
        \textit{aˈsí}\\
        \textit{aˈsí} \\
        to.sit.\textsc{sg.intr}\\
        `S/he sits.'\\
        `Se sienta.'\\
    }\label{ex:13:suppletive postural verbsb}
            \ex[]{
            \textit{aˈtʃ͡â}\\
            \textit{aˈtʃ͡â}\\
            to.sit.\textsc{sg.tr}\\
            `S/he sits someone down.'\\
            `Lo sienta (e.g., al niño).'\\
        }\label{ex:13:suppletive postural verbsc}
    \z
\z

The contrast encoded through suppletion involves a state (\ref{ex:13:suppletive postural verbsa}), an intransitive, change of state (\ref{ex:13:suppletive postural verbsb}) and a transitive change of state (\ref{ex:13:suppletive postural verbsc}). The consonantal changes in the stem in this set is not predictable through any synchronically productive phonological processes, and are thus analyzed here as involving a lexical contrast.

The following examples illustrate the stative (\ref{ex:13:stative postural verbs}), intransitive change of state (\ref{ex:13:intransitive change of state postural verb}) and transitive change of state (\ref{ex:13:transitive change of state postural verb}) uses of postural verbs (a full inventory of postural predicates in Choguita Rarámuri is provided in \tabref{tab:key:26}):

%\break

\ea\label{ex:13:stative postural verbs}
{Stative postural verbs}\\

    \ea[]{
    {\textit{ˈétʃ͡i   ko   tʃ͡oˈmí   ˈtʃ͡ó} \textbf{\textit{muˈtʃûwi} } \textit{aʔˈlì}}\\
    \gll    ˈétʃ͡i=ko  tʃ͡oˈmí  ˈtʃ͡ó  \textbf{muˈtʃûwi}  aʔˈlì\\
            \textsc{dem}=\textsc{emph}  there  also  {sit\textsc{.pl}}    and\\
    \glt    ‘They were (sitting) also over there, and...’\\
    \glt    ‘Ellos también estaban allá y...’    < FLP 06 in61(302)/in >\\
}
        \ex[]{
        {\textit{aʔˈlì   ˈwé   biˈlá   kiˈrì   ke   biˈlé   saˈjèka} \textbf{\textit{tʃuˈkúli}} \textit{wiˈtʃ͡ôa     ˈétʃ͡i ko}}\\
        \gll    aʔˈlì  ˈwé  beˈlá  kiˈrì  ke  biˈlé  saˈjè-ka \textbf{tʃuˈkú-li} wiˈtʃ͡ô-a    étʃ͡i=ko\\
                and  \textsc{int} really  still  \textsc{neg} one  realize-\textsc{ger} {bent.\textsc{sg-pst}} wash.clothes-\textsc{prog} \textsc{dem=emph}\\
        \glt    `And then she was still without realizing anything, washing clothes, her.’\\
        \glt    `Y entonces estaba sin pensar nada, sin darse cuenta de nada, lavando, ella.’ \corpuslink{tx32[04_119-04_174].wav}{LEL tx32:4:11.9}  \\
    }
            \ex[]{
            {\textit{ˈpé   ˈnábi   tʃ͡e   ˈtʃéti   koˈlìi} \textbf{\textit{biˈtíam}} \textit{ku} }\\
            \gll    ˈpé  ˈna=bi    tʃé  ˈtʃéti  koˈlì    \textbf{biˈtí-ame}       ku\\
                    just  \textsc{prox=}just  also \textsc{def.pl.bad} around.corner  {be.lying.down.\textsc{pl}}{\textsc{ptcp}} \textsc{rev}\\
            \glt    `They say that those that are lying down over there around the corner...’\\
            \glt    `Dicen que estos que están ahí por de aquel lado a la vuelta...’   \corpuslink{in243[20_078-20_117].wav}{FLP in243:20:07.8}\\
        }
    \z
\z

%\pagebreak

\ea\label{ex:13:intransitive change of state postural verb}
{Intransitive change of state postural verbs}\\

    \ea[]{
    {\textit{aʔˈlì    ˈhê  aˈnèli:  “ˈwé  saˈpù} \textbf{\textit{aˈʃíska,}} \textit{ˈpîri     tʃ͡uˈkú     naˈʔî?”}}\\
    \gll    aʔˈlì   ˈhê aˈnè-li  ˈwé  saˈpù  \textbf{aˈsí-si-ka}    ˈpîri  tʃ͡uˈkú    naˈʔî\\
            and   \textsc{dem}   say-\textsc{pst} \textsc{int} hurry   {sit.up-\textsc{mot-ger}}   what   four.legs  here\\
    \glt    `And then  she told him: “Hurry up getting up! What is that sitting here?’\\
    \glt    `Y luego le dijo: ``¡Levántate pronto! ¿Qué está aquí?”’ \corpuslink{tx5[00_527-00_572].wav}{LEL tx5:0:52.7}\\
}
        \ex[]{
        {\textit{ˈnápu   riˈká   ˈnà   muˈrípi   roˈkáata} \textbf{\textit{tʃuˈkúbuka}} \textit{biˈlá  ko} \textit{aʔˈwâli   ba}}\\
        \gll    ˈnápi  riˈká  ˈnà  muˈrípi  roˈkáta    \textbf{tʃuˈkú-ba-ka}    beˈlá=ko aʔˈwâ-li  ba \\
                \textsc{sub} like  then  close  cliff    {four.legs-\textsc{inch-ger}} indeed=\textsc{emph} throw.\textsc{pl-pst}  \textsc{cl}\\
        \glt    `Because he (the deer) stood (on four legs) close to the cliff and then threw them.’\\
        \glt    `Así como se paró (el venado, en cuatro patas) cerca de la orilla del barranco y los tiró.’ \corpuslink{tx152[08_491-08_548].wav}{SFH tx152:8:49.1}\\
    }
            \ex[]{
            \textbf{\textit{wiˈlíbura}} \textit{ruˈwá           ˈétʃ͡i   ˈnà   maˈtʃí     riˈtériri}\\
            \gll    \textbf{wiˈlí-ba-la}  ru-ˈwá    ˈétʃ͡i  ˈnà  maˈtʃí    reˈté-riri\\
                    {stand-\textsc{inch-rep}} say\textsc{-mpass}  \textsc{dem} \textsc{dem}    outside   rock-\textsc{loc}\\
            \glt    `Se paró allá afuera en una piedra.’\\
            \glt    `He stood there outside in a rock.’  \corpuslink{tx5[02_069-02_106].wav}{LEL tx5:2:06.9}\\
}
    \z
\z

\newpage
\ea\label{ex:13:transitive change of state postural verb}
{Transitive change of state postural verbs}\\

    \ea[]{
    {\textit{ˈpé   biˈlé   riˈʰtê    ˈá   riʔˈpí   tʃ͡oˈnà     ko   ˈnápo   ˈnàti  \textbf{muˈtʃûwili}}}\\
    \gll    ˈpé   biˈlé  riʰˈtê   ˈá  riʔˈpí  ˈétʃ͡i  ˈná=ko  ˈnápi  ˈnà=ti \textbf{muˈtʃûwi-li}\\
            just   one   stone   \textsc{aff} remain \textsc{dem}   \textsc{prox=emph} where   \textsc{dem}=1\textsc{pl.nom} {sit.\textsc{tr.pl}}{{}-}{\textsc{pst}}\\
    \glt    ‘Just one stone remains there where we put them.’  \\
    \glt    ‘Queda una piedra ahí donde las pusimos.’  \corpuslink{tx19[02_350-02_389].wav}{LEL tx19:2:35.0}\\
}
%\pagebreak
        \ex[]{
        {\textit{siˈné   ˈkátʃ͡i  ˈmáti}   \textit{ˈá} \textbf{\textit{ˈhâwamti}}\footnote{In this case, the verb ‘make stand’ means ‘appoint’ (in \ili{Spanish}, literally ‘parar autoridades’).} \textit{á}  \textit{ˈníbo}}\\
        \gll    siˈné  ˈkátʃ͡i ˈmá=ti      ˈá   \textbf{ˈhâwa-ma=ti}       ˈá ˈní-bo\\
                some  times  already=1\textsc{pl.nom} \textsc{aff} {stand\textsc{.tr.pl-fut.sg=1pl.nom}} \textsc{aff} \textsc{cop-fut.pl}\\
        \glt    ‘Where we live, perhaps we will some times be elected as authorities.’\\
        \glt    ‘Donde vivimos a lo mejor en veces vamos a ser autoridades.’  \corpuslink{tx12[11_542-11_592].wav}{SFH tx12:11:54.2}\\
    }
            \ex[]{
            \textit{ˈmá     ˈbéni          aʔˈlá   maˈtʃ͡í   ˈnè     ko  ˈmá   ke   oˈmèrili} \textit{ˈtʃ͡ó   ˈmêa        ˈtʃ͡ó  ˈnápi   niˈhê} \textbf{\textit{wiˈláli}} \textit{ko   ba} \\
            \gll   ˈmá    ˈbé=ni      aʔˈlá  maˈtʃ͡í  ˈnè=ko ˈmá  ke  oˈmèra-li ˈtʃ͡ó  ˈmê-a    ˈtʃ͡ó  ˈnápi  neˈhê    \textbf{wiˈlá-li}=ko  ba\\
                    then   because=1\textsc{sg.nom} well   know   1\textsc{sg.nom=emph} then   \textsc{neg} be.able-\textsc{pst} either  win-\textsc{prog} either  \textsc{sub}   \textsc{1sg.nom} {stand.\textsc{tr}}\textbf{{}-}{\textsc{pst}}=\textsc{emph} \textsc{cl} \\
            \glt    ‘Because I know very well that she couldn’t win, the one I chose (lit. the one I stood).’\\
            \glt    ‘Porque ya se que no pudo ganar la que puse yo (lit. ‘la que yo paré).’  \corpuslink{tx19[04_280-04_356].wav}{LEL tx19:4:28.0}\\
        }
    \z
\z

Postural predicates may also have suppletive forms encoding singular-plural distinctions or these contrasts may be derived through the plural/pluractional construction, which may also involve stem consonant mutation plus prefixation (cf. §\ref{sec: pluractional marking}). This is further exemplified in (\ref{ex:13:pluractional form of postural verb}).

\newpage
\ea\label{ex:13:pluractional form of postural verb}
{Plural/pluractional form of positional/postural verb}

    \ea[]{
    \doublebox{\textit{tʃ͡uˈkú}}{‘bent, curved, on four legs, \textsc{sg.}'}\\
}
        \ex[]{
        \doublebox{\textit{u-ˈtʃ͡úwi}}{‘bent, curved, on four legs, \textsc{pl}'}\\
    }
    \z
\z

\tabref{tab:key:26} depicts the set of Choguita Rarámuri postural predicates, with transitivity and singular-plural/pluractional contrasts.

%\break

\begin{table}
\caption{Postural/positional predicates in Choguita Rarámuri}
\label{tab:key:26}
\small
\begin{tabularx}{\textwidth}{Qllllll}
\lsptoprule
& \multicolumn{2}{X}{

 \textbf{Stative}} & \multicolumn{2}{X}{

 \textbf{Inchoative}} & \multicolumn{2}{X}{

 \textbf{Causative}}\\
 \cmidrule(lr){2-3} \cmidrule(lr){4-5} \cmidrule(lr){6-7}
& \textbf{Sg} & \textbf{Pl} & \textbf{Sg} & \textbf{Pl} & \textbf{Sg} & \textbf{Pl}\\
\midrule
‘sit’ & \textit{aˈtí} & \textit{muˈtʃûwi} & \textit{asi/aˈsá} & \textit{moˈtʃíwi} & \textit{aˈtʃá} & \makecell[tl]{\textit{muˈtʃûwi/}\\\textit{mutʃ͡uˈwâ}}\\
\tablevspace
‘sit (container)’ & \makecell[tl]{\textit{maˈní}\\\textit{{\textasciitilde}baˈní}} & \makecell[tl]{\textit{a-ˈmáni}\\\textit{{\textasciitilde}a-ˈbáni}} & \makecell[tl]{\textit{bani-ˈbá}\\\textit{{\textasciitilde}mani-ˈbá}} & \makecell[tl]{\textit{bani-ˈbá/}\\\textit{baˈní-ba}\\\textit{{\textasciitilde}mani-ˈbá}} & \textit{baˈná {\textasciitilde}maˈná} & \makecell[tl]{\textit{a-maˈná}\\\textit{{\textasciitilde}a-baˈná}}\\
\tablevspace
‘stand’ & \textit{wiˈlí} & \textit{ˈhâwi} & \textit{wiˈlísi} & \textit{ˈhâsi} & \textit{wiˈlá} & \textit{haˈwá}\\
\tablevspace
‘lie down’ & \textit{buˈʔí} & \textit{biˈtí} & \textit{buˈʔu-} & \textit{biˈtí} & \textit{riki/riˈká} & \textit{roˈʔá}\\
\tablevspace
‘bent, curved, on four legs’ & \textit{tʃ͡uˈkú} & \textit{uˈtʃúwi} & \makecell[tl]{\textit{tʃ͡uˈkú-ba/}\\\textit{tʃ͡uku-ˈbá}} & \makecell[tl]{\textit{i-ˈtʃúpi/}\\\textit{i-tʃ͡uˈpá}\\\textit{{\textasciitilde}u-ˈtʃúpi}} & \textit{uˈtʃá} & \makecell[tl]{\textit{i-ˈtʃútʃ͡i}\\\textit{{\textasciitilde}u-ˈtʃútʃ͡i}}\\
\lspbottomrule
\end{tabularx}
\end{table}



%\textbf{another cell in the table: motion}


% if variation is mentioned here it should be exemplified

Clauses with stative postural predicates may also have the function of encoding the existence of the noun referent. This is shown in (\ref{ex:13:locative predicate in existential clause}), where the postural predicate in this context can be interpreted as non-specific with respect to the specific posture of the referent (sitting, standing, etc.).

\ea\label{ex:13:locative predicate in existential clause}
{Postural/positional predicates in existential clauses}

    {\textit{noˈrátʃ͡i  ˈben     ko} \textbf{\textit{aˈtí} }\textit{tʃ͡ó   ba}}\\
    \gll    noˈrátʃ͡i ˈbe=ni=ko  \textbf{aˈtí}  ˈtʃ͡ó  ba\\
            Norogachi  be=1\textsc{sg.nom}=\textsc{emph}  {sit\textsc{.sg}}  \textsc{emph}  \textsc{cl} \\
    \glt    ‘I was in Norogachi.’\\
    \glt    `Yo estaba en Norogachi.’  < FLP 06 in61(264)/in >\\

\z

%more examples?

The predicate in these cases specifies both the existence of the referent the position in which entities are configured, making reference to their semantic properties, or provides an indication of their grammatical classification. In terms of semantic categorization, the postural verb set in Choguita Rarámuri involves a partitioning of the ``etic'' space, defined as the ``structure of the natural world \ldots\, [an] objective description of the domain which makes maximal discriminations", which may be contrasted with an `emic' space where languages group in abstract way these discriminations that is based on the semantic properties of the referents \citep[][8]{levinson2006grammars}. Nouns have a canonical posture depending on the physical properties (e.g., volume, shape, dimensions, etc.) of their referents, and these characteristics may determine the choice of a postural predicate: four-legged (e.g., a dog, a truck) vs. two-legged (e.g., a person, a ladder) vs. a recipient with a base (e.g., a pot) vs. a cylindric and thin object (e.g., a stick, a pencil), etc. Choice of postural predicates may also reflect circumstantial or transitional physical characteristics of the referent at the time of utterance (e.g., a person washing clothes or grinding corn in a \textit{metate} is \textit{tʃ͡uˈkú} ‘bent/curved’.) The semantic categorization properties of postural verb is exemplified in (\ref{ex:13:examples of postural predicates}) (\tabref{tab:noun-classes} below provides examples of the classification system dependent on the postural verb nouns combine with in locative constructions.).

%% Beginning of problematic part

\ea\label{ex:13:examples of postural predicates}
{Positional/postural predicates: semantic categorization}

    \ea{
    {\textit{aʔˈlì     ˈhê aˈnèli:     “ˈwé  saˈpù  aˈʃíska!     ˈpîri} \textbf{ \textit{tʃuˈkú}} \textit{naˈʔî?”}}\\
    \gll    aʔˈlì    ˈhê  aˈn-è-li      ˈwé   saˈpù   aˈsí-si-ka      ˈpîri   \textbf{tʃuˈkú}   naˈʔî \\
            and   \textsc{dem}    say\textsc{-appl-pst}  \textsc{int}   hurry  sit.up-\textsc{mot-ger}  what  {be.curved}  here \\
    \glt    ‘And then she told him: “hurry up getting up, what is that standing (in four legs) here?’\\
    \glt    `Y luego le dijo: ``¡Levántate pronto! ¿Qué está aquí?”’ \corpuslink{tx5[00_527-00_572].wav}{LEL tx5:0:52.7}\\
}\label{ex:13:examples of postural predicatesa}
        \ex[]{
        {\textit{ˈí  oˈnátʃ͡i} \textbf{\textit{tʃukuˈbáma}} \textit{ˈlé  toˈwí}}\\
        \gll    ˈí  oˈnátʃ͡i  \textbf{tʃuku-ˈbá-ma}    aˈlé  toˈwí\\
                here  here    {be.curved-\textsc{inch-fut.sg}} \textsc{dub}  boy\\
        \glt    ‘The boy will bend here.’\\
        \glt    ‘Aquí se va a agachar el niño.’ < BFL 11/03/09/el >\\
        }\label{ex:13:examples of postural predicatesb}
            \ex[]{
            {\textit{ˈétʃ͡i   tʃ͡oʔˈmá    biˈlá} \textbf{\textit{maˈníira} } \textit{ˈráa    iˈbíli    beˈtôli}}\\
            \gll    ˈétʃ͡i    tʃ͡oʔˈmá   beˈlá   \textbf{maˈní-ri-la}     ru-ˈwá     i-ˈbíli    beˈtôli\\
                    \textsc{dem}  snot    really  {sit.container-\textsc{caus-rep}} say-\textsc{mpass}  \textsc{pl}-one\textsc{.pl}  plates\\
            \glt    ‘But it is said that they were given snot to drink in some plates.’\\
            \glt    ‘Pero dicen que les dieron moco en unos platos.’  < SFH 06 choma(6)/tx >\\
    }\label{ex:13:examples of postural predicatesc}
            \ex[]{
            {\textit{ˈí \textbf{aˈmáni} ˈbôte}}\\
            \gll    ˈí \textbf{aˈmáni} ˈbôte\\
                    here  {\textsc{pl}-sit.container}  cans\\
            \glt    ‘Here are the cans.’\\
            \glt    ‘Aquí están los botes.'  < BFL 11/03/09/el >\\
            }\label{ex:13:examples of postural predicatesd}
%\pagebreak
                \ex[]{
                {\textit{ˈpé   ˈnàbi   tʃ͡é   t͡ʃéti     koˈlì} \textbf{\textit{biˈtíam}} \textit{ku}}\\
                \gll    ˈpé  ˈnà=bi    tʃé  tʃ͡éti  ko'lì    \textbf{biˈtí-ame}      ku\\
            just  \textsc{prox}=just  also  \textsc{det.pl} around.there  {lie.down.\textsc{pl-ptcp}} \textsc{rev}\\
            \glt    ‘These ones (the dead people) who are lying down over there, also.’\\
            \glt    ‘Estos que están (acostados) ahí por de aquel lado (los muertos).’     < FLP 07 in243(511)/in >\\
        }\label{ex:13:examples of postural predicatese}

    \z %% second-level (lettered) examples?
\z %% top-level (numbered) example

%% End of problematic part

In (\ref{ex:13:examples of postural predicatesa}), the noun referent about which the location is being predicated is in sight and stands in four legs, and in (\ref{ex:13:examples of postural predicatesb}) the referent is a child about to crawl, requiring the use of the predicate \textit{tʃ͡uˈkú} `to be curved'. The referents in (\ref{ex:13:examples of postural predicates}c--d) involve containers with bases, for which the predicate \textit{mani} `to sit (a container)' is required. In (\ref{ex:13:examples of postural predicatesc}), the referent requires the use of the predicate \textit{biti} `to lie down, plural',\footnote{This is a euphemism used to refer to dead people in this narrative.}

While the categorization of postural predicates closely reflects fine-grained distinctions about postures of referents, there are constructions where the assignment of a particular entity to a particular postural verb seems arbitrary, revealing their classificatory nature. Consider, for instance, example (\ref{ex:13:posturals as classificatory device}), where the existence of a ring in the hand or blood on the floor is described with the verb \textit{tʃ͡uˈkú}, which typically describes the position of entities that are on four legs, bent or curved.

\ea\label{ex:13:posturals as classificatory device}
{Postural predicates as grammatical classificatory devices}

    \ea[]{
    {\textit{tʃ͡oˈnà} \textbf{\textit{tʃuˈkú}} \textit{aˈnîjo  siˈkâratʃ͡i}}\\
    \gll    ˈétʃ͡i  ˈnà  \textbf{tʃuˈkú}  aˈnîjo  seˈkâ-ra-tʃ͡i\\
            \textsc{dem}  {prox}  {be.bent}  ring  hand-\textsc{poss-loc}\\
    \glt    ‘The ring is in the hand.’\\
    \glt    ‘El anillo está en la mano.’  < LEL 09 1:74/el >\\
}
        \ex[]{
        {\textit{“ˈétʃ͡o ˈnà ˈá} \textbf{\textit{tʃuˈkú}} \textit{ˈlá    maˈtʃ͡í   ˈnápu  ko    ˈnà ˈhônsa  ˈtòli}  \textit{ˈétʃ͡i     aliˈwâla}”}\\
        \gll    ˈétʃ͡i   ˈnà  ˈá   \textbf{tʃuˈkú}   ˈlá   maˈtʃí   ˈnápi=ko   ˈnà  ˈhônsa  ˈtò-li ˈétʃ͡i   aliˈwâ-la \\
                \textsc{dem}   \textsc{prox}  \textsc{aff} {be.bent}  blood    outside  \textsc{sub}=\textsc{emph}  \textsc{prox}  from    take-\textsc{pst}  \textsc{dem}  soul-\textsc{poss} \\
        \glt    “There is the blood outside where he took the soul from.”\\
        \glt    “Ahí se ve la sangre afuera de donde se llevó el alma.”  \corpuslink{tx5[03_592-04_037].wav}{LEL tx5:3:59.2}\\
    }
   \z
\z

Thus, locative constructions not only reflect semantic categorization but also a classification system that is grammatical, as it may be applied to abstract nouns or concrete nouns where their physical properties are not considered for their classification (e.g., space objects like sun, stars, moon, etc. are also assigned a postural predicate in existential clauses). An overview of postural predicates and the semantic categorization of nouns in other \ili{Uto-Aztecan} languages (including \ili{Yaqui} (\ili{Taracahitan}), \ili{Mayo} (\ili{Taracahitan}), \ili{Guarijío} (\ili{Taracahitan}) and \ili{Lower Pima} (\ili{Tepiman})) is provided in \citet{o2015typological}. Nouns in the language are grouped into different classes depending on the postural verb they combine with in locative constructions. \tabref{tab:noun-classes} gives a sample of how nouns are classified depending on the postural verb they combine with in locative constructions.

%though there are some speaker differences as to membership of nouns to each class - if saying this, show examples

\begin{table}[b]
\caption{Noun classes and locative constructions}
\label{tab:noun-classes}

\begin{tabularx}{\textwidth}{llQ}
\lsptoprule

\multicolumn{2}{l}{\textbf{Postural verb} } & \textbf{Noun referents} \\
\cmidrule(lr){1-2}
\textbf{Sg / Pl (stative)} & \textbf{Gloss} & \\
\midrule
\textit{aˈtí / muˈtʃûwi} & ‘sit’ & \textit{tʃ͡iˈní} `cloth', \textit{reˈmê} ‘tortillas’, \textit{muˈnî} ‘uncooked beans’, \textit{maˈlêta} ‘suitcase/bag’, \textit{koˈlî} ‘round chiles (chiltepin)’, \textit{aˈsûkar} ‘sugar (in a bag)’, \textit{naˈpátʃ͡a} ‘shirts’, \textit{wiˈrá} ‘earrings on table’\\
\tablevspace
\textit{maˈní / a-maˈní} & \makecell[tl]{‘sit a container’} & \textit{sikoˈrí} `pots', \textit{ˈbôtetʃ͡i} `cans', \textit{liˈmêtatʃ͡i} `bottles', \textit{muˈnî} ‘cooked beans’, \textit{baʔˈwí} ‘water’, \textit{aˈsûkar} ‘sugar (in a container)’, \textit{laˈbá} ‘weja’ \\
\tablevspace
\textit{wiˈlí / ˈhâwi} & ‘stand’ & \textit{ˈpôste} `pole', \textit{oˈʰkó} ‘pine’, \textit{kaˈlí} ‘house’, \textit{reˈhòi} ‘man’, \textit{ˈkâmara} ‘camera’, \textit{paˈpêl} ‘standing paper towel roll’, \textit{riˈkóa} ‘troje’, \textit{kalenˈtôntʃ͡i} ‘heater’\\
\tablevspace
\textit{buˈʔí / biˈtí} & ‘lie down’ & \textit{kuˈsí} `stick', \textit{lapiˈsêro} `pen/pencil', \textit{ˈpâla} ‘shovel’, \textit{suˈnù} ‘corn’, \textit{koˈlî} ‘green chile pepper’, \textit{ˈkâble} ‘cable’, \textit{aʔˈkà} ‘sandals’ \\
\tablevspace
\textit{tʃ͡uˈkú / uˈtʃúwi} & \makecell[tl]{‘bent, curved,\\on four legs’} & \textit{ˈwàsi} `cow', \textit{ˈtrôka} ‘truck’, \textit{naˈrútʃ͡iri} ‘spider’, \textit{ˈmúsa} ‘tʃ͡ukú’, \textit{esˈpêho} ‘mirror (mounted on wall)’, \textit{wiˈrá} ‘hanging earrings’\\
\lspbottomrule
\end{tabularx}
\end{table}
%\break

%Ojo: fotos montadas en la pared: \textit{foto tʃ͡ukú}; fotos recargadas sobre un objeto: \textit{foto háwi}.

\hspace*{-1.3pt}In addition to exhibiting valency-related morphological changes, postural predicates may also attach the movement \textit{-ro} suffix to encode that the event depicted by the predicate is a process that is incomple and may involve motion. This is exemplified in (\ref{ex:13:posture verbs with motion suffix}).

\ea\label{ex:13:posture verbs with motion suffix}
{Use of motion \textit{-ro} suffix with postural/positional verbs}

    \ea[]{
    \textit{ˈpeta naˈʔî ˈá ˈhârpo ˈrú}\\
    \gll    ˈpe=ta naˈʔî ˈá ˈhâwi-ro-po ˈrú\\
            just=\textsc{1pl.nom} here \textsc{aff} stand.\textsc{pl.intr-mov-fut.pl} say.\textsc{prs}\\
    \glt    `We will just be hanging out here standing.'\\
    \glt    ‘Aquí vamos a andar parados.’\\
}
        \ex[]{
        \textit{ˈápta ˈne waraˈká ˈhârpo ˈapu riˈkáti so raʔaˈmâbo ˈnà raˈlàmuli ba ʃuˈwâba ˈmá ˈká riˈká muˈkî ˈúa ba}\\
        \gll    ˈnápi=ta ˈne wara-ˈká ˈhâwi-ro-po ˈapu riˈká=ti=so raʔaˈmâ-bo ˈnà raˈlàmuli ba suˈwâba ˈmá ˈká riˈká muˈkî ˈjúa ba\\
                \textsc{sub=1pl.nom} remember remember-\textsc{ger} stand.\textsc{pl.intr-mov-fut.pl} \textsc{sub} like=\textsc{1pl.nom}=\textsc{emph} give.advice-\textsc{fut.pl} \textsc{prox} men \textsc{cl} all \textsc{already} all all women with \textsc{cl}\\
                \largerpage
        \glt    `everything we stand  knowing (remembering), to give advice to the men and everyone, too, and all the women'\footnote{The emphatic particle \textit{so} is only attested in the speech of older speakers and often in the context of \textit{nawesari}, ceremonial speeches. Younger speakers use the emphatic particle \textit{ko} in the same contexts where \textit{\textit{so} is attested.}}\\
            ‘todo lo que estamos (parados) sabiendo (recordando), dar consejos a los hombres y a todos también y a todas las mujeres’  \corpuslink{tx816[00_103-00_166].wav}{JMF tx816:00:10.3}
    }
    \z
\z

The cognate form of this motion suffix in \ili{Mountain Guarijío} is the stress-shifting suffix \textit{-to/ro}, described in  \citep[][165]{miller1996guarijio} as a verbal suffix that encodes motion with a ``distributive" sense (```ir a ... en más de un lugar' [o] `en más de un tiempo'" [```to go to more than one place' or `in more than one occasion'"].\footnote{ This suffix contrasts with another ``motion'' suffix in \ili{Mountain Guarijío}, \textit{-si} (cognate of the Choguita Rarámuri associated motion \textit{-si(mi)} suffix), which is described as bearing a more aspectual meaning.} Further examples of this construction with Choguita Rarámuri postural predicates are exemplified in (\ref{ex:13:posturals in motion constructions}).

%\break

\ea\label{ex:13:posturals in motion constructions}
{Postural/positional predicates in motion constructions}

    \ea[]{
    {\textit{ˈtá  toˈwí  tʃ͡uˈkúro}}\\
    \gll    ˈtá  toˈwí  tʃ͡uˈkú-ro\\
            small boy be.bent-\textsc{mov}\\
    \glt    `The boy is crawling.'\\
    \glt    `El niño anda gateando.’    < BFL 11/03/09/el >\\
}
        \ex[]{
        {\textit{ta  toˈwí  tʃ͡uˈkúrma    ˈlé}}\\
        \gll    ta  toˈwí  tʃ͡uˈkú-ro-ma   aˈlé\\
                small boy be.bent-\textsc{mov.fut.sg} \textsc{dub}\\
        \glt    `The small boy will be crawling here.'\\
        \glt    `Aquí va a andar gateando el niño.’   < BFL 11/03/09/el >\\
    }
            \ex[]{
            {\textit{ˈíni  wiˈlírmo  ˈlá}}\\\
            \gll    ˈí=ni  wiˈlí-ro-ma  oˈlá\\
                    here=1\textsc{sg.nom} be.standing.\textsc{sg-mov-fut.sg} \textsc{cer}\\
            \glt    `I will be hanging out here (standing).'\\
            \glt    `Aquí voy a andar parado.’ < BFL 11/03/09/el >\\
        }
    \z
\z

%revise this given the complex predicate section!
Postural verbs often occur in complex predicate constructions in which a descriptive predicate qualifies the way in which an entity came to be in a particular position (complex predicate constructions are addressed in \chapref{chap: clause combining in complex sentences}). In the examples below, the descriptive predicate (which precedes the postural/positional predicate) qualifies that the position (in this instance lying down) came about by fainting (\ref{ex:13:postural verbs in complex predicate constructionsa}), dying (\ref{ex:13:postural verbs in complex predicate constructionsb}) or sleeping (\ref{ex:13:postural verbs in complex predicate constructionsc}).

\largerpage[2]
\ea\label{ex:13:postural verbs in complex predicate constructions}
{Postural verbs in complex predicate constructions}\\

    \ea[]{
    \textit{ˈmá    nataˈkêa} \textbf{\textit{buˈʔíli} } \textit{biˈʔà   roˈkò}\\
    \gll    ˈmá   nataˈkê-a  \textbf{buˈʔí-li}      biˈʔà   roˈkò  \\
            already    faint-\textsc{prog}  {lie.down.\textsc{sg}}{{}-}{\textsc{pst}} early night\\
    \glt    ‘He was already lying down unconscious early in the morning.’\\
    \glt    ‘Ya estaba desmayado en la madrugada.’ \corpuslink{tx5[04_037-04_070].wav}{LEL tx5:04:03.7}\\
}\label{ex:13:postural verbs in complex predicate constructionsa}
        \ex[]{
        \textit{aʔˈlì   ˈmá  ˈʔnèli   ˈpé ˈmá  mukuˈká} \textbf{\textit{buʔíli}} \\
        \gll    aʔˈlì  ˈmá iʔˈnè-li  ˈpé  ˈmá  muku-ˈká  \textbf{buˈʔí-li}\\
                and   already  see-\textsc{pst} little   already   die-\textsc{ger}   {lie.down.\textsc{sg}}{{}-}{\textsc{pst}}\\
        \glt    ‘Then they saw that he was already lying down dead’\\
        \glt    ‘Entonces ya vieron y ya estaba muerto.’   \corpuslink{tx5[02_428-02_458].wav}{LEL tx5:02:42.8}  \\
    }\label{ex:13:postural verbs in complex predicate constructionsb}
            \ex[]{
             \textit{aʔˈlì   ˈnà …   aʔˈlì   ˈnà      kotʃ͡iˈká} \textbf{\textit{buˈʔílo}} \textit{maˈjêli}\\
            \gll    aʔˈlì  ˈnà  aʔˈlì  ˈnà  kotʃ͡i-ˈká  \textbf{buˈʔí-li-o}    maˈjê-li\\
                    and   then   and   then   sleep-\textsc{sim} {lie.down.\textsc{sg}}{{}-}{\textsc{pst-ep}} think-\textsc{pst}\\
            \glt    ‘And then he thought he was asleep (laid down sleeping).’\\
            \glt    ‘Nomás que pensó que estaba dormido.’   \corpuslink{tx5[00_350-00_383].wav}{LEL tx5:00:35.0}\\
        }\label{ex:13:postural verbs in complex predicate constructionsc}
    \z
\z

In negative locative clauses, a predicate exclusively found in negative polarity copular clauses (\textit{iˈtê}), is found. The negative locative clause in (\ref{ex:13:postural verbs in negative clauses 2firsta}) is contrasted with its positive polarity locative counterpart in (\ref{ex:13:postural verbs in negative clauses 2firstb}).

\ea\label{ex:13:postural verbs in negative clauses 2first}
{Negative polarity predicate \textit{ité} in negative locative clauses}\\

    \ea[]{
    \textit{naˈʔî  ke} \textbf{\textit{iˈtê}} \textit{ˈbôte}\\
    \gll    naˈʔî  ke \textbf{\textit{iˈtê}} ˈbôte\\
            here  \textsc{neg} {be.\textsc{neg}} can\\
    \glt    ‘The can is not here.’\\
    \glt    ‘Aquí no está el bote.’ < LEL 09 1:74/el >\\
}\label{ex:13:postural verbs in negative clauses 2firsta}
        \ex[]{
        \textit{naˈʔî} \textbf{\textit{maˈní} } \textit{ˈbôte}\\
        \gll    \textit{naˈʔî} \textbf{\textit{maˈní} } \textit{ˈbôte}\\
                here  {sit.container}  can\\
        \glt    ‘Here is (sits) the can.’\\
        \glt    ‘Aquí está el bote.’      < LEL 09 1:74/el >    \\
    }\label{ex:13:postural verbs in negative clauses 2firstb}
    \z
\z

Further examples of negative locative clauses with the predicate \textit{iˈtê} in negative locative clauses are shown in (\ref{ex:13:postural verbs in negative clauses 2second}).

\ea\label{ex:13:postural verbs in negative clauses 2second}
{Negative polarity predicate \textit{iˈtê}: further examples}\\

    \ea[]{
    \textit{amiˈná     ke} \textbf{\textit{iˈtêrimi}} \textit{ˈlé   wiˈkarâami   amiˈná} \textit{ko   ba}\\
    \gll    amiˈnábi  ke  \textbf{iˈtê-ri-mi}    aˈlé  wikaˈrâ-ame  amiˈnábi=ko  ba\\
            afterwards  \textsc{neg}  {be.\textsc{neg-ri-irr.sg}} \textsc{dub} sing-\textsc{ptcp} afterwards=\textsc{emph} \textsc{cl}\\
    \glt    `Afterwards there won’t be any ritual singers.’\\
    \glt    `Después ya no va a haber cantador.’    \corpuslink{in243[12_453-12_501].wav}{FLP in243:12:45.3}      \\
}
%\break
        \ex[]{
        \textit{ˈmá ˈkátʃ͡i \textbf{iˈtêli} biˈlé  muˈkî     aˈpâtʃ͡e   ba amiˈnábi ko ˈá wiˈká riˈhòrimi tʃ͡ó aˈlé ke ˈnà }\\
        \gll    ˈmá ˈká tʃ͡è \textbf{iˈtê-li} biˈlé  muˈkî     aˈpâtʃ͡e   ba amiˈnábi=ko ˈá wiˈká riˈhò-ri=mi tʃ͡ó aˈlé ke ˈnà \\
                already \textsc{neg} \textsc{neg} {be.\textsc{neg-pst}} one woman apache \textsc{cl} afterwards=\textsc{emph} \textsc{aff} many man-\textsc{vblz=dem} anymore \textsc{dub} \textsc{neg} then\\
        \glt    `because there weren’t any single Apache women, (otherwise) afterwards there would have been more (Apache) people’\\
        \glt    `porque ya no había ninguna mujer apache, (si no) hubiera habido mucha gente (apache)’  \corpuslink{tx110[02_418-02_446].wav}{LEL tx110:2:41.8}\\
    }
    \z
\z

Postural verbs may also be used in negative polarity constructions with other negative particles. Examples of positional predicates in negative constructions are provided in (\ref{ex:13:postural verbs in negative polarity constructions}).

\ea\label{ex:13:postural verbs in negative polarity constructions}
{Postural verbs in negative locative clauses}

    \ea[]{
    \textit{ˈká    ˈtʃ͡è  ˈwêsi} \textbf{\textit{moˈtʃí}} \textit{bitiˈt͡ʃí  ko  ba}\\
    \gll    ˈká    ˈtʃ͡è  ˈwêsi    \textbf{moˈtʃí}  bitiˈt͡ʃí=ko  ba\\
            \textsc{neg} \textsc{neg} nobody  {sit.\textsc{pl}} house=\textsc{emph} \textsc{cl}\\
    \glt    `Nobody stays (sits) at home.’\\
    \glt    `No se está nadie en la casa.’   < BFL 09 1:65/el >\\
}\label{ex:13:postural verbs in negative polarity constructionsa}
        \ex[]{
        \textit{ˈí    oˈnátʃ͡i  ˈká  ˈtʃ͡è} \textbf{\textit{wiˈlí}} \textit{esˈk\textsuperscript{w}êla  ba}\\
        \gll    ˈi    oˈnátʃ͡i  ˈká  ˈtʃ͡è  \textbf{wiˈlí}    esˈk\textsuperscript{w}êla  ba\\
                here  here    \textsc{neg} \textsc{neg} {stand.\textsc{sg}}  school  \textsc{cl}\\
        \glt    `The school is not here.’\\
        \glt    `Aquí no está la escuela.’   < BFL 09 1:65/el >\\
    }\label{ex:13:postural verbs in negative polarity constructionsb}
    \z
\z

Postural verbs in negative constructions may be used when the particular posture is in the scope of negation: in (\ref{ex:13:postural predication in scope of negation}), for instance, it is predicated that nobody is sitting (i.e., but there might be people standing).

\ea\label{ex:13:postural predication in scope of negation}
{Positional/postural predicates in the scope of negation}

    \textit{tʃ͡iˈnà    ˈká  ˈtʃ͡è    ˈwêsi} \textbf{\textit{muˈtʃûi}} \textit{ba}\\
    \gll    ˈétʃ͡i  ˈnà  ˈká  ˈtʃ͡è    ˈwêsi    \textbf{muˈtʃûwi}  ba\\
            \textsc{dem}  \textsc{dem}  \textsc{neg}  \textsc{neg} nobody  {sit.\textsc{pl} } \textsc{cl}\\
    \glt    ‘There is nobody sitting there.’\\
    \glt    ‘No hay nadie sentado ahí.’ < BFL 09 1:65/el >\\
\z

A relevant question to ask in any language with positional verbs is which verb may be used as a ``default'', as languages with these systems are documented to exhibit a default collocation, which may be subject to pragmatically-induced variation \parencite{ameka2007introduction}. In the case of Choguita Rarámuri, a good candidate for a default postural copular verb in locative constructions for human beings and other referents  is the verb \textit{atí} ‘sit, \textsc{sg}’ (and plural suppletive verb \textit{mutʃúwi}). In the following narrative, a woman is scared by a coyote sitting outside (\ref{ex:13:default postural verbsa}); in the next clause, the verb \textit{mutʃúwi} ‘sit, \textsc{pl}’ is used to describe how the scared woman, together with her friends, locks herself inside the house (\ref{ex:13:default postural verbsb}); the postural verb in this case describes that the women are located inside of the locked house. The referents are not in sight and there is nothing inherent in their configuration that would preclude a description with, for instance, the verb \textit{háwi} ‘stand, \textsc{pl}’.

\ea\label{ex:13:default postural verbs}
{Default postural predicates}

    \ea[]{
    \textit{aʔˈlì       ˈwé     maˈhâli           ˈétʃ͡i     muˈkî   basaˈt͡ʃî  riwiˈsâ}\\
    \gll    aʔˈlì  ˈwé  maˈhâ-li  ˈétʃ͡i  muˈkî  basaˈt͡ʃî    riwi-ˈsâ\\
            and   \textsc{int}   get.scared-\textsc{pst}  \textsc{dem} woman  coyote    see-\textsc{cond}\\
    \glt    ‘So she got very scared when she saw the coyote.’\\
    \glt    ‘Entonces se asustó mucho la mujer cuando vió al coyote.’ \corpuslink{tx_mawiya[01_026-01_075].wav}{LEL tx\_mawiya:1:02.6}\\
}\label{ex:13:default postural verbsa}
        \ex[]{
        \textit{we   ˈérika \textbf{muˈtʃîli} patʃ͡á}\\
        \gll    we  ˈéri-ka    \textbf{muˈtʃûwi-li}  paˈtʃá\\
                \textsc{int} close-\textsc{ger} {sit.\textsc{pl-pst}}   inside\\
        \glt    ‘She closed the door shut and they were inside.’\\
        \glt    ‘Cerró fuerte la puerta y estuvieron adentro.’ < LEL tx\_mawiya:1:07.5 >\\
    }\label{ex:13:default postural verbsb}
    \z
\z

Thus, the default collocation when referring to human beings is `sit'. This is also shown in example (\ref{ex:13:postural verbs in negative polarity constructions}) above, where the use of positional predicates in negative existential and negative locative clauses provides a test for default collocations. In contrast to human beings `sitting' by default, the canonical position for buildings is to `stand', as shown in (\ref{ex:13:postural verbs in negative polarity constructionsb}) above. This means that the default collocation involves presupposing, not asserting, the properties of the figure \citep[][859]{ameka2007introduction}.
%cite page number here

% Other examples? - Ameka & Levinson
%More research is needed in order to determine if there is indeed a postural verb that is the default in certain locative constructions.

Finally, there are recorded examples where postural verbs may also have grammaticalized functions as aspectual markers, a common development cross-linguis\-tically \parencite{ameka2007introduction}.\footnote{“It is cross-linguistically common for postural verbs such as `sit', `stand', and `lie' to have grammaticalised functions (e.g. as copula verbs, aspect markers, etc.).  The regularity with which these postural verbs operate as a linker in locative constructions, in particular, led \citet{ameka2007introduction} to coin the label ``postural-type language'' \citep[][461]{gaby2006grammar}.} The following examples show a contrast between a construction where the postural predicate \textit{tʃ͡uˈkú}, ‘be on four legs{\slash}bent{\slash}curved’, has a habitual aspect meaning (\ref{ex:13:postural predicates as aspect markersa}) and a construction where the postural predicate \textit{aˈtí}, ‘sit’, has a non-habitual reading (\ref{ex:13:postural predicates as aspect markersb}) (There is no evidence in the Choguita Rarámuri corpus that other postural predicates have grammaticized as aspectual markers.).

\ea\label{ex:13:postural predicates as aspect markers}
{Positional predicates with aspectual marking function}

    \ea[]{
    \textit{ˈnâtili    ˈàa    tʃ͡uˈkú}\\
    \gll    ˈnâta-li    ˈà-a    tʃ͡uˈkú\\
            think-\textsc{nmlz}  give-\textsc{prog} bent.\textsc{prs}\\
    \glt    ‘He is giving advice (always).’\\
    \glt    ‘(Siempre) dando consejos.’  \corpuslink{tx12[12_402-12_461].wav}{SFH tx12:12:40.2}   \\
}\label{ex:13:postural predicates as aspect markersa}
        \ex[]{
        \textit{ˈétʃ͡i  ˈwé  ˈnâtili    ˈàa    aˈtí}  \\
        \gll    ˈétʃ͡i  ˈwé  ˈnâta-li    ˈà-a    aˈtí\\
                \textsc{dem} \textsc{int} think-\textsc{nmlz} give-\textsc{prog} sit.\textsc{prs}\\
        \glt    ‘He is giving advice (for only a while).’\\
        \glt    ‘Dando consejos (sólo un rato).’ < FMF 09 3:57/el >  \\
    }\label{ex:13:postural predicates as aspect markersb}
    \z
\z


In sum, the system of positional predicates in Choguita Rarámuri may be characterized as featuring properties characteristic of languages with small sets (3--5) of contrastive locative verbs, which contrast with languages with large-set positional languages (which may feature between 9 to up to several hundred predicates). The properties of small-set positional languages are described in \citep[][858--9]{ameka2007introduction}, some of which are summarized in (\ref{ex:13:properties of small set positional predicates}).

\ea\label{ex:13:properties of small set positional predicates}
{Properties of small-set positional systems}

    \begin{itemize}
        \item There is a central role of human posture verbs (e.g., `sit', `stand', `lie down', etc.).\\
        \item There is a semantic categorization component, with reference to the spatial configuration of the referent (abstract axial and/or geometric).\\
        \item The default collocation use of positionals involves presupposing (vs. asserting) the properties of the figure.\\
        \item There is a grammatical classification of nominal concepts which may be independent.\\
    \end{itemize}
\z

As has been shown in this section, Choguita Rarámuri exhibits these properties in its positional predicate system.

\subsection{Existential clauses expressing predicate possession}
\label{subsec: predicates of possession}

In contrast to existential clauses containing locative predicates and a positional verb, where the existence of a given referent is given in function to a particular spatial location, ‘true’ existential clauses state the existence of a referent without providing any further information about the position of the referent or their grammatical classification. The latter kind of existential clauses in Choguita Rarámuri involve the verb \textit{niˈlú} (with a suppletive stem \textit{naˈlú} appearing in free variation), a lexical verb meaning `exist' or `to be in existence'. The examples in (\ref{ex:13:existential constructions with niru and naru}) illustrate this construction:

\ea\label{ex:13:existential constructions with niru and naru}
{Existential constructions with predicate of existence}

    \ea[]{
    \textit{ˈpé   kuˈríbi   ni   biˈlé   buˈrîto} \textbf{\textit{niˈlú} } \textit{pa}\\
    \gll    ˈpé   kuˈrí=bi    ni  biˈlé  buˈrîto    \textbf{niˈlú}  pa\\
            just  recently=just  nor  one  donkey    {exist}  \textsc{cl}\\
    \glt    ‘Just barely, there were no donkeys either.’\\
    \glt    ‘Nomás apenas, ni burros había tampoco.’   < FLP 06 in61(230)/in >\\
}
        \ex[]{
        \textit{tʃ͡i   ˈbíri   ˈnápi   ˈkim} \textbf{\textit{naˈlúa} ˈlé   ...  kuˈtʃâra}\\
        \gll    ˈétʃ͡i  ˈbíri  ˈnápu  ˈki=mi    \textbf{naˈlú-a}    aˈlé   kuˈtʃâra\\
                \textsc{dem} kind  \textsc{sub} \textsc{sub=dem} {exist\textsc{{}-prog}} \textsc{dub} spoons\\
        \glt    ‘There are a lot of things of that kind... spoons.’\\
        \glt    \mbox{‘Muchas cosas que hay de ese tipo...cucharas.’  < SFH 06 in61(725)/in >}
    }
            \ex[]{
            \textit{aˈpítʃ͡iri     aʔˈlì   ˈpîri   ˈʃî   ˈtʃé   ˈá} \textbf{\textit{niˈlú}}? \\
            \gll    aˈpítʃ͡iri    aʔˈlì  ˈpîri  ˈsî  ˈtʃé  ˈá  \textbf{niˈlú}\\
                    great.grandchildren  and what  else  also  \textsc{aff} {exist\textsc{.prs}}\\
            \glt    ‘Greatgrandchildren and then, what else is there?’\\
            \glt    ‘Tataranietos y luego, ¿qué más hay?’  < SFH 06 in63(143)/in >  \\
        }
                \ex[]{
                \textit{napaʔˈlì     ke   ˈtʃ͡ó} \textbf{\textit{niˈlúi} } \textit{ko   sekunˈdâria   ba}\\
                \gll    ˈnápi  aʔˈlì  ke  ˈtʃ͡ó  \textbf{niˈlú-i}=ko  sekunˈdâria  ba \\
                        \textsc{sub}   later   \textsc{neg} yet   {exist-\textsc{impf}}=\textsc{emph} secondary   \textsc{cl}\\
                \glt    `when there wasn’t any secondary school yet’\\
                \glt    `cuando todavía no había secundaria’   \corpuslink{tx12[01_222-01_249].wav}{SFH tx12:1:22.2}\\
            }
    \z
\z

Existential verbs may bear the full range of inflectional marking attested in other verbs. Some examples of the predicate \textit{niˈlú} with several tense/aspect distinctions are provided in (\ref{ex:13:niru plus inflection}).

\ea\label{ex:13:niru plus inflection}
{Predicate \textit{niˈlú} in different inflectional contexts}

    \ea[]{
    \textit{we} \textbf{\textit{niˈlúmi}} \textit{ˈlé   ˈlóali       ba}\\
    \gll    we \textbf{niˈlú-mi} aˈlé  ˈlówa-li      ba \\
            \textsc{int} {exist-\textsc{irr.sg}} \textsc{dub} be.hungry-\textsc{nmlz}  \textsc{cl} \\
    \glt    ‘Perhaps there will be a lot of hunger.’\\
    \glt    ‘A lo mejor va a haber mucha hambre.’      \corpuslink{in243[20_238-20_254].wav}{SFH in243:20:23.8}\\
}\label{ex:13:niru plus inflectiona}
        \ex[]{
        \textit{ˈpé   kuˈríi     ˈhônsa   niˈlúuli   ˈnápi   riˈká   ˈmá} \textbf{\textit{niˈlúsa}} \textit{ˈétʃ͡i  rupuˈrá   ba?} \\
        \gll    ˈpé  kuˈrí    ˈhônsa  niˈlú-li  ˈnápu  riˈká  ˈmá   \textbf{niˈlú-sa/} ˈétʃ͡i  ripuˈrá    ba  \\
                just  recently  since  exist\textsc{.sg-pst} \textsc{sub} like  already    {exist.\textsc{sg-cond}} \textsc{dem} ax    \textsc{cl} \\
        \glt    ‘Just recently, when there were axes already?’\\
        \glt    ‘¿Hace poquito cuando ya hubo hacha?’    \corpuslink{in242[00_332-00_401].wav}{SFH in242:0:33.2}\\
    }\label{ex:13:niru plus inflectionb}
            \ex[]{
            \textit{ˈkátʃ͡i   ka} \textbf{\textit{niˈlúi}} \textit{tiˈjôpa   ˈí  kiˈʔà     ko} \textit{noˈráatʃ͡i   be   ko   naˈlìna     oˈmáwa   nokiˈwái     pa}\\
            \gll    ˈká ˈtʃ͡è   ka \textbf{niˈlú-i} tiˈjôpa   ˈí  kiˈʔà=ko noˈráatʃ͡i   be=ko   naˈlìna     oˈmáwa   noki-ˈwá-i     pa\\
                  because \textsc{neg}  ka  {exist-\textsc{impf}}  church  here  before=\textsc{emph} Norogachi  but=\textsc{emph}  because  party  make-\textsc{mpass}{}-\textsc{impf}  \textsc{cl}\\
            \glt    ‘Because there was no church here before, they would make the celebrations in Norogachi.’\\
            \glt    ‘Porque antes aquí no había aquí iglesia, en Norogachi era donde hacían las fiestas.’  < FLP 06 in61(424)/in >\\
        }\label{ex:13:niru plus inflectionc}
                \ex[]{
                \textit{we   aʔˈlá   rutuˈbúri   oraˈkáli   ba   baˈhîtʃ͡i   ˈbéli} \textbf{\textit{niˈlúami}} \textit{baʔˈwí   pa}  \\
                \gll    we  aʔˈlá  rutuˈbúri  oraˈká-li  ba  baˈhîtʃ͡i    beˈlá \textbf{niˈlú-ame} baʔˈwí  pa \\
                        \textsc{int} well  rutubúri  make\textsc{-pst} \textsc{cl} well indeed {exist-\textsc{ptcp}} water \textsc{cl}  \\
                \glt    ‘Yes, a well done \textit{rutubúri}, with water from the well.’\\
                \glt    ‘Sí, bien hecho el \textit{rutubúri}, con agua del aguaje.’   \corpuslink{in243[00_321-00_369].wav}{FLP in243:0:32.1}\\
            }\label{ex:13:niru plus inflectiond}
    \z
\z

As seen in this examples, the predicate \textit{niˈlú} may be inflected for irrealis singular (\ref{ex:13:niru plus inflectiona}), conditional (\ref{ex:13:niru plus inflectionb}), imperfective (\ref{ex:13:niru plus inflectionc}) and participial (\ref{ex:13:niru plus inflectiond}), among others.

Negative existential constructions may also be expressed through the predicate \textit{niˈlú} and negative adverbs, as shown in (\ref{ex:13:negation with niru}).

\ea\label{ex:13:negation with niru}

    \textit{ˈkátʃ͡i} \textbf{\textit{niˈlúuli}} \textit{ba   ripuˈrá   pa}\\
    \gll    ˈká  ˈtʃé  \textbf{niˈlú-li}   ba  ripuˈrá  pa\\
            because \textsc{neg}  {exist-\textsc{pst}} \textsc{cl}  axes  \textsc{cl}\\
    \glt    ‘There were no axes.’   \\
    \glt    ‘No había hachas.’   \corpuslink{in243[19_469-19_486].wav}{FLP in243:19:46.9}\\
\z

Alternatively (and more frequently), negative existential clauses may be expressed through the negative predicate \textit{iˈtê}, as shown for negative existential clauses involving positional/postural predicates, and as shown in (\ref{ex:13:negative predicate ite in existential clauses}).

%\break

\ea\label{ex:13:negative predicate ite in existential clauses}
{The negative predicate \textit{iˈte} in existential clauses}

    \ea[]{
    \textit{ke   biˈlé   ˈpé   ˈtâʃi} \textbf{\textit{iˈtêeli} } \textit{ˈónam     tʃ͡aˈbè   ko?}\\
    \gll    ke  biˈlé  ˈpé  ˈtâsi  \textbf{iˈtê-li}    ˈóna-ame  tʃ͡aˈbè=ko\\
            \textsc{neg}  one  just  \textsc{neg}  {exist.\textsc{neg}}\textbf{{}-}{\textsc{pst}}  cure-\textsc{ptcp}  before=\textsc{emph}\\
    \glt    `There wasn't any doctor before?'\\
    \glt    ‘¿No había doctor antes?’   \corpuslink{in61[06_396-06_420].wav}{SFH in61:6:39.6}\\
}
        \ex[]{
        \textit{aʔˈlì  ˈmá  ke} \textbf{\textit{iˈtêo}} \textit{ˈlá ˈnà   ˈháp   taˈmò     tʃ͡iriˈká \textit{ruˈjèma     ko} }\\
        \gll    aʔˈlì  ˈmá  ke  \textbf{iˈtê-o}   oˈlá  ˈnà  ˈhápi  taˈmò    ˈétʃ͡i  riˈká ru-ˈè-ma=ko\\
                and  already  \textsc{neg} {exist.\textsc{neg}-\textsc{ep}} \textsc{cer} \textsc{prox} \textsc{sub} \textsc{1pl.acc} \textsc{dem} like say-\textsc{appl-fut.sg=emph}\\
        \glt    ‘And then there is nothing for us to be told.’\\
        \glt    ‘Y luego ya no hay para que nos digan a nosotras.’  \corpuslink{tx905[02_041-02_085].wav}{GFM tx905:2:04.1}   \\
    }
            \ex[]{
            \textit{aˈmí   noˈráatʃ͡i   niˈrúam   ˈníli   ˈnà   leˈhîdotʃ͡i   ˈápu   ke ˈtʃ͡ó   \textit{aˈnáur-a-tʃ͡i   ko   naˈʔî   ke   ˈtʃ͡ó} \textbf{\textit{iˈtê-a-tʃi}} \textit{ko  leˈhîdotʃ͡i   ba} ˈtʃ͡ó} \\
            \gll    aˈmí  noˈráatʃ͡i  niˈrú-ame  ˈní-li  ˈnà  leˈhîdotʃ͡i  ˈápu  ke  ˈtʃ͡ó aˈnáuri-a-tʃ͡i=ko  naˈʔî  ke  ˈtʃ͡ó  \textbf{iˈtê-a-tʃi}=ko  leˈhîdotʃ͡i ba\\
                    \textsc{dist} Norogachi  exist-\textsc{ptcp} \textsc{cop.prs} this  ejido  \textsc{sub} \textsc{neg} yet  measure-\textsc{prog-temp=emph} here  \textsc{neg} yet  {exist.\textsc{neg}-\textsc{prog-temp}}=\textsc{emph} ejido  \textsc{cl}\\
            \glt    ‘This \textit{ejido} was Norogachi’s when this ejido wasn’t measured yet, when it wasn’t yet an ejido.’\\
            \glt    ‘Este ejido era de Norogachi cuando todavía no se medía este ejido, cuando todavía no era ejido aquí.’ \corpuslink{tx817[00_086-00_147].wav}{JMF tx817:0:08.6}\\
        }
    \z
\z

% add some form of conclusion here
