\chapter{Voicing alternations}
\label{chap: voicing alternations}

Voiced plosives in CR involve a single surface form, but they exhibit patterns in their distributions which are governed by distinct factors. Specifically, voiced plosives are part of the phonological inventory of the language and are manipulated by lexical phonological processes, but they may also (i) result from morphologically conditioned phonological processes, (ii) exhibit a suppletive allomorphic distribution, or (iii) result from a low level phonetic lenition process. We have addressed different aspects of this phenomenon and its distribution in ~\ref{chap: phonology}, ~\ref{chap: verbal morphology} and ~\ref{chap: prosody}. The purpose of this chapter is to address this synchronic picture which is motivated by recent diachronic prosodic changes undergone by CR.

As noted in ~\ref{chap: phonology}, there is a three-way laryngeal contrast for stops, with voiceless pre-laryngealized (pre-apirated) stops (specified with the feature [+ spread glottis]), voiceless ‘plain’ stops (with no laryngealization feature involved) and voiced stops. Plain voiceless stops and voiced stops, both laryngeally unspecified, surface in complex patterns of consonant quality alternations involving voicing. These alternations may result from a variety of phonological and morphological factors, including: (i) consonant mutation in suppletive allomorphy, (ii) general phonological restrictions on voicing in consonant clusters, (iii) morphologically conditioned voicing alternations in body part incorporation, (iv) encoding of morphological contrasts, and (v) the outcome of optional, gradient effects in rapid speech. This chapter describes the alternations that arise in consonat mutation, the general restrictions on voicing in derived consonant clusters, and the gradient, post-lexical lenition effects. Voicing alternations that arise in morphological pluractional contrasts and body part incorporation are addressed in X and Y, respectively).

\section{Underlying laryngeal contrasts of oral stops}
\label{sec: underlying laryngeal contrasts}

As discussed in \chapref{chap: phonology}, Choguita Rarámuri has a relatively small phonemic inventory, featuring some degree of laryngeal complexity, with two glottal segments (a glottal stop and a glottal fricative), as well as a laryngeal contrast for its oral stops and alveopalatal affricate. This laryngeal contrast is between ‘pre-aspirated’ and ‘plain’ oral obstruents (stops and palato-alveolar affricate). Pre-aspirated stops ( [hp, ht, hk, htʃ]) are analyzed here as specified with a [+spread glottis] feature and ‘plain’ obstruents as laryngeally unspecified ([p, t, k, tʃ]). This analysis is provided in Di Canio (2012) for the lenis-fortis contrasts in San Martín \ili{Ituñoso Trique}, where ‘fortis’ obstruents are realized with an early adduction (spreading) glottalic gesture, which results in a short period of pre-aspiration before the consonant (2012: 257) (see also Arellanes (2009) for a similar analysis of the lenis-fortis contrast in San Pablo Güilá Zapotec). Pre-aspirated obstruents in Choguita Rarámuri are only attested as onsets of stressed syllables. ‘Plain’ obstruents are realized with no pre-laryngealization, are not restricted in terms of their distribution and may be realized with post-aspiration as onsets of stressed syllables (Caballero \& Carroll 2015). The examples in (5) include (near-)minimal pairs that illustrate this contrast.

(5)	Choguita Rarámuri pre-aspirated obstruent vs. plain obstruent contrast

	Preaspirated					Plain
a.	wiˈʰká   	‘many’ 		[BFL 14 1:17]		wiˈká 	‘constellation/plow’ [BFL 14 1:17]
b.	riˈʰtú-li  	‘freeze-PST’	[SFH el1028:0.43.8]	riˈturi 	‘encourage’ 	       [MDH co1136:19:47.3]
c.	naˈʰpó 	‘prickly pear’	[BFL el728:7:09.4]		naˈpò	‘to break, intr.’	       [BFL 14 1:18]
d.	riˈʰtʃí	‘reliz’		[LEL 14 1:9]		riʔˈtʃì	‘fish’		       [LEL 14 1:9]
e.	oˈʰkó	‘pine tree’	[LEL 14 1:3]		oʔˈkó 	‘pain’		        [LEL 14 1:9]
f.	raˈʰtá	‘hot’		[LEL 14 1:2]		raʔˈtá	‘to pop, blow’          [LEL 14 1:2]
g.	baˈʰt͡ʃi	‘squash’	[BFL 14 1:17]		baʔˈt͡ʃi-	‘older brother’	       [MDH co1136:18.36.1]

As shown in (5d-g), pre-aspirated obstruents contrast with plain stops that may be preceded by a glottal stop.

Some speakers have variable realization of these series as pre-aspirated or pre-laryngealized, e.g., ahˈka ~ aʔˈka ‘sandal’ in a pattern that resembles the optional pre-aspiration or pre-laryngealization of fortis consonants in SNNP noted in 1.2 (Kataoka 2010). Furthermore, for some speakers pre-aspirated stops can be realized without any laryngeal features in stressed position, neutralizing in the surface with ‘plain’ stops, an instance of inter-speaker variation in the realization of these segments. Some examples include:; riˈhtʃí [BFL CR2014-LSC:13] ~ riˈtʃí ‘reliz’ [GFM 14 1:111] [SFH CR2014-LSC:86]; ahˈka [ref] ~ aˈkha ‘to be sweet or salty’ [SFH CR2014-LSC:85]. These patterns of variation in the articulation of pre-aspirated stops may suggest that the pre-aspirated contrast may be marginal for some speakers, though most speakers realize the contrast consistently. A caveat, suggested by Marc Garellek (pc), may however be added: even if pre-aspiration seems to disappear in the speech of some speakers, it may leave residual breathy voice on preceding vowels in these forms.

There are no voiced coronal or velar stops in the phonological inventory of Choguita Rarámuri. Allophonic alternations show that the voiced counterpart of /t/ is a coronal flap (/ɾ/) (there is no voiced coronal stop /d/ in the Choguita Rarámuri phonological inventory). The contrast between voiced and voiceless stops is shown in (6).\footnote{The coronal flap is allophonically realized as a trill in word-initial position (e.g., (5f)). CR has a second liquid, a lateral flap that is articulated by making a ballistic contact with the tongue tip in the post-alveolar region, with the sides of the tongue allowing air to flow laterally. As described in Ladefoged \& Maddieson (1996:243), the lateral flap auditorily resembles both a flap and a lateral. This has led to orthographic representations of this sound as either ‘r’ or ‘l’, including the name of the language, which is alternatively spelled Rarámuri, with word-medial ‘r’, or Ralámuli, with word-medial ‘l’. In this paper, lateral flaps are uniformily represented with [l] in the data of all Rarámuri varieties discussed.}

(6)	Voicing contrast for bilabial and coronal oral stops

a.	paˈtʃi 		‘corn cob’ 							[SFH in448:8:47.8]
b.	baˈʰtʃi 		‘zucchini’							[LEL tx130:0.52.5]
c.	paˈtʃa	 	‘adentro’ 							[LEL tx71:4:10.7]
d.	baˈtʃa  		‘primero’							[BFL tx1:1:10.5]
e.	taˈɾa		‘count’								[JHF 04 1:1/el]
f.	raˈɾa		‘buy’								[BFL tx\_falda:0:26]

Given that the contrast between pre-aspirated and plain voiceless stops is neutralized in word initial position (since pre-aspirated stops are only attested as onsets of stressed syllables), these minimal pairs reveal a contrast between voiced stops, on the one hand, and either plain voiceless stops or pre-aspirated voiceless stops.\footnote{Marc Garellek points out (pc) that voiced stops in these examples could be analyzed as underlyingly plain voiceless, given a process of gradient lenition of these segments which yields voiced segments in fast speech (addressed below in 2.5). It would thus be possible to posit that perhaps the contrast is between plain and pre-aspirated contrasts, both in lenited versions. However, since these stops do not display gradient voicing lenition effects (e.g., there is no *paˈʰtʃi alternating with baˈʰtʃi for ‘zucchini’), we assume here they are indeed underlyingly voiced stops that undergo different allophonic processes than plain voiceless stops in the language.}

In inter-vocalic position, voiced bilabial stops are realized primarily as voiced fricatives (7a-b) and marginally as approximants (7c-d). For some speakers, voiced bilabial stops are also reduced to [w] in pre-consonantal position (7e).

(7)	Surface realization of voiced bilabial stops

a.	[zi-ˈβóo]		/si-ˈboo/	‘go.pl-FUT.PL’					[MFH tx1133:1:17.2]
b.	[apaˈβeza]		/apaˈbe-sa/	‘X-COND’					[MFH tx1133:1:17.2]
c.	[ˈeβ̩ə-ma]		/ˈebi-ma/	‘bring-FUT.SG’	   				[BFL 06 6:73/el]
d.	[muˈtʃi-β̩a-ri]	/muˈtʃí-ba-li/	‘be.sitting.PL-INCH-PST’   			[BFL 06 6:73/el]
e.	[ˈew-ti-ki] 		/ˈebi-ti-ki/ 	‘bring-caus-PST.1’ 				[BFL 06 6:73/el]



\section{A phonological process of consonant devoicing}
\label{sec: a phonological process of consonant devoicing}

Another allophonic process targets voiced oral bilabial stops in contexts where stress-related syncope yields surface consonant clusters. In these contexts, no voiced stops can occur in post-consonantal position, a constraint that can be formalized as *CC[+voice]. This constraint triggers a general phonological process of devoicing that applies after post-tonic vowel deletion has taken place (Caballero 2008). This rule is schematized in (8).

(8)	Post-consonantal stop devoicing

	[+ voice], [-cont], [-son]      [-voice], [-cont], [-son]   / C\_\_

The pairs of morphologically related words in (9) show the effect of this process on underlying voiced bilabial stops in pairs of morphologically related words.

(9)	Devoicing of bilabial stops

a.	bam-ˈpá-sa				*bam-ˈbá-sa
	/bami-ba-sa/
	‘have.year-INCH-COND’
	‘If/when (s/)he turns a year older’
	[SF 06 tx12/Text]

b.	baˈmi-ba-li
 	/bami-ba-li/
	‘have.year-INCH-PST’
	‘(S/)he turned one year older’
	[LEL tx372:4:40.5]

c.	sam-ˈpá-ma				*sam-ˈbá-ma
	/sami-ba-ma/
	‘be.wet-INCH-FUT.SG’
	‘(It) will become wet’
	[SF 04 1:113/el]

d.	sami-ˈba-ma
	/sami-ba-ma/
	‘be.wet-INCH-FUT.SG’
	‘(It) will become wet’
	[SF 04 1:113/el]

e.	raˈpa-m-po				*raˈpa-m-bo
	/rapa-na-bo/
	‘split.APPL-TR-FUT.PL’
	‘We will split it for them’
	[AH 05 1:131/el]

f.	rapa-ˈná-bo
	/rapa-na-bo/
	‘split-TR-FUT.PL’
	‘We will split it’
	[AH 05 1:131/el]

The lexical root baˈmi (a denominal form of the root ‘year’) has an inchoative suffix allomorph with a voiced onset in (9b), but a voiceless onset in (9a) after a stress shift suffix conditions pre-tonic vowel deletion; the example in (9c), a form with pre-tonic deletion and a voiceless onset, contrasts with example (9d), in which the same morphologically complex word and no pre-tonic deletion features the same affix with a voiced onset. Finally, and as discussed in further detail in §2.6 below, the future plural suffix exemplified in (9e-f) has suppletive allomorphs with underlying voiced and voiceless onsets whose distribution is dependent upon the lexical identity of the preceding morpheme. The forms in (9e-f) show, however, that in this case a general phonological restriction is at play in conditioning the voicing value of the suffix onset: the future plural suffix appears in the same morphological environment, but the stress shift and post-tonic vowel deletion yield the environment in which devoicing applies.

As described above, the laryngeal contrast in CR stops can be analyzed as involving a difference in terms of phonological specification, with pre-aspirated obstruents being specified with a [+spread glottis] feature (surface pre-aspirated) ([hp, ht, hk, htʃ]), and ‘plain’ obstruents being laryngeally unspecified [p, t, k, tʃ]. In addition to the laryngeal contrast for stops, CR has a set of plain voiced stops ([b, ɾ]). This last series feature two properties: first, the coronal series involves a concomitant change in manner of articulation; and second, there is no voiced counterpart for the velar voiceless stop. Finally, CR voiced stops are subject to several allophonic processes, including devoicing triggered by a constraint on post-consonantal voiced stops (*CC[+voice]), and two lenition processes, namely (i) inter-vocalic spirantization and (ii) gliding in pre-consonantal position. These lenition processes are the output of the lexical phonology and co-exist with voicing alternations that are conditioned by the morphology of the language. These alternations are addressed next.


\section{Voicing alternations in suppletive allomorphy}
\label{sec: voicing alternations in suppletive allomorphy}

Voicing alternations in suffixation involves a system that resembles the consonant gradation system attested in the \ili{Numic} branch of UA. In the CR mutation system, which can be characterized as involving two grades, some suffixes have suppletive allomorphs with a voiced oral stop or voiced tap onset, the VOICED grade, or a voiceless stop onset, the VOICELESS grade. The shape and distribution of these allomorphs is not governed by phonological conditions, and are thus suppletive. These allomorphs are lexically determined by the preceding morpheme to which the allomorphs attach. This is exemplified below with the distribution of voiceless and voiced allomorphs of the future plural suffix (10a-f), the causative suffix (10g-h), the potential suffix (10i-k), the irrealis plural suffix (10k-m), and the future passive suffix (10n-o).

(10)	Voicing mutation in suppletive allomorphy

a.	naˈra-po		‘cry-FUT.PL’						[BFL 04 1:74/el]
b.	neˈwa-bo		‘make-FUT.PL’						[SFH 04 1:67/el]
c.	ˈwí-bo 		‘harvest- FUT.PL’					[SFH 04 1:69/el]
d.	teˈtʃí-po		‘comb-FUT.PL’						[SFH 04 1:69/el]
e.	paˈko-po		‘wash.dishes-FUT.PL’’					[SF 04 1:69/el]
f.	wiˈtʃo-bo		‘wash.clothes-FUT.PL’					[AH 04 1:69/el]
g.	naˈpa-ti-ma  	‘hug-CAUS- FUT.SG’					[BFL VDB/el]
h.	peˈwa-ri-ma	‘smoke-CAUS-FUT.SG’   					[RF 04 1:122/el]
i.	ˈtu-ta		‘bring-POT’						[BFL 07 2:21/El]
j.	maˈha-ra		‘scare-POT’						[05 1:154/El]
k.	niwa-ni-pi 		‘make-DESID-IRR.PL’					[tx1131]
l.	aˈni-bi		‘say-IRR.PL’						[SFH in61:380 14:22]
m.	tó-bi 		‘bury-IRR.PL’ 						[SF 08 1:3/El]
n.	siˈru-nu-pa  	‘hunt-APPL-FUT.PASS’					[ref]
o.	ruʔuˈtʃi-wi-ba 	‘sprinkle-APPL-FUT.PASS’ 				[ref]

These examples show that voiced and voiceless allomorphs occur in the same segmental and prosodic contexts, e.g. in (10a-b) and (10g-h) both grades occur after a stressed low vowel, in (10c-d) after a stressed high, front vowel, etc. There are some differences in terms of the prosodic size of some of the bases to which the allomorphs attach to, with some allomorphs attaching to monosyllabic stems ((10d), (10j), (10l)), but these differences are not correlated with any particular grade, precluding a phonological account of allomorph distribution. The allomorphy is suppletive and allomorph distribution is lexically conditioned by the preceding morpheme.

An abstract autosegmental analysis is available for this pattern: the relevant suffixes would be assumed to be underlyingly voiceless, with voicing resulting from mutation-triggering stems that underlyingly possess a [+voice] feature which coalesces with the following suffix-initial oral stop. This kind of analysis is posited for Tümpisa (\ili{Panamint}) Shoshoni (\ili{Numic}), where consonant gradation results from abstract morpheme-final consonants that link to adjacent, morpheme-initial consonants (Dayley 1989). Alternatively, the voicing alternations in these contexts results directly from the application of phonological rules or constraints associated with different sets of lexical items that trigger either voicing or devoicing in a following suffix. Here I limit myself to the observation that in either analysis it is necessary to assume that specific morphemes are lexically determined in order to correctly derive the distributions of the voiced and voiceless grades in the CR allomorphy patterns.


\section{Morphologically conditioned voicing alternations: noun incorporation}
\label{sec: morphologically conditioned voicing alternations}

While most suffixes with coronal or bilabial stop onsets display mutation in allomorphy, there are some CR morphological constructions with potential targets of mutation that appear consistently in a particular grade. One such case is attested in body-part incorporation, where the second member, the head of the construction, consistently exhibits a voiced onset. This is shown in (11).

(11)	Consonant quality alternations in body-part incorporation

a.	ronoˈbaki	 	/roˈno + paˈko-/	   	feet+wash		‘to feet-wash’		[ref]
b.	rameˈbaki		/raˈme + paˈko/	   	teeth+wash		‘to teeth-wash’		[ref]
c.	kupaˈbaki		/kuˈpa + paˈko-/   	hair+wash		‘to hair-wash’		[ref]
d.	tʃ͡omaˈbiwa 	/tʃ͡oʔˈma + biʔˈwa/	snot+clean		‘to snot-claen’		[ref]
e.	witaˈbiwa		/wiˈta +	 biʔˈwa/	excrement+clean	‘to excrement-clean’	[ref]
f.	moʔoˈrepi		/moˈʔo- + riˈpu/	head+cut		‘to head-cut’		[ref]
g.	ronoˈrepi		/roˈno- + riˈpu/		foot+cut		‘to foot-cut’ 		[ref]
h.	ropaˈkasi		/roˈpa- + kaˈsi/		stomach+break		‘to have a miscarriage’	[ref]
i.	busuˈkasi		/buˈsi- + kaˈsi/		eye+break		‘to become blind’	[ref]

As described in §2.2, these constructions have stress in the first syllable of the head of the construction, the verbal root. There is a consonant quality alternation if this verbal root has a word-initial voiceless oral stop in isolation, e.g., paˈko ‘wash (dishes)’ in (11a-c). As shown in (11d-g), there is no alternation of the verb root onset if it is underlyingly voiced (e.g., biʔˈwa ‘clean’ (11d-e), riˈpu ‘cut’ (11f-g)), and, as shown in (11h-i), there is no alternation if the second member underlyingly has a velar stop.  These examples also show there is no consonant quality alternation of other oral stops in incorporated nominal stems (e.g., kupaˈbaki ‘to hair-wash’(11c), but * kubaˈbaki, witaˈbiwa ‘to excrement-clean’ (11e), but *wiraˈbiwa).

Consonant quality alternations in incorporation could be characterized as resulting from a process of inter-vocalic voicing of (plain) oral stops that exclusively targets the onset of the incorporated verbal stems, a case of morphologically conditioned phonology. Alternatively, the voicing alternation can be analyzed as involving mutation in allomorphy, where verbal stems with plain voiceless oral stop onsets have a stem allomorph with a voiced grade onset in incorporation constructions. Under this alternative, which I adopt here, the voicing alternation in this context is akin to the one found in cases of suffixes that have suppletive allomorphs with voiced and voiceless grades.


\section{voicing alternations as process morphology}
\label{sec: voicing alternations as process morphology}

Consonant quality alternations in CR are also involved in the exponence of morphological information. The relevant construction marks plural number with nouns, plural subject with verbs, or that a discrete event takes place or is being performed by the same agent several times, or by several agents several times. These meanings are related in that they refer to event plurality or ‘pluractionality’, a phenomenon that has been defined to encompass meanings that range from iterative and frequentative to distributive and extensive action (Newman 1980, Wood 2007, Henderson 2012). Exponence of pluractionality in CR involves three patterns: (i) prefixation of V-, a vocalic prefix that acquires its quality through assimilation to the quality of first vowel of the root (12a-b);  consonant mutation (12c-j); and (iii) both prefixation and consonant mutation (12k-o). Crucially, consonant mutation in this morphological context targets bilabial and coronal stops. Consonant mutation is highlighted in boldface.

(12)	Pluractional exponence in CR

	SG			PL
a.	ˈtʃoni 		o-ˈtʃoni 	‘to become black’			[AH 05 2:24/El]
b.	siˈli-ame		i-ˈseli-kame	‘governor’				[BFL 05 1:156/El]

c.	kaˈpo-	 	kaˈbo-		‘to be round’				[BFL 05 1:155/El]
d.	rikuˈri		ˈtekiri		‘drunk person’				[BFL 05 1:156/El]
e.	kaˈpi-		kaˈbi		‘to be cylindrical’			[BFL 05 1:156/El]
f.	buˈke		puˈke		‘to own domesticated animals’		[RF 04 1:66/el]
g.	remaˈri		ˈtemuri		‘young person’				[BFL 05 1:155/El]
h.	saˈpe		saˈbe		‘to be/become fat’			[BFL 05 1:156/El]
i.	roˈsa		toˈsa		‘to be white’				[BFL 05 1:157/El]
j.	baˈtʃa		paˈtʃa		‘to be the first one’			[SFH tx12:11:34.4]

k.	kiˈpa 		i-kiˈba		‘to snow’				[SFH 05 2:8/El]
l.	siˈta-kame 		i-siˈra-kame 	‘to be red’				[BFL 05 1:157/El]
m.	baˈhi 		a-paˈhi		‘to drink’				[SFH 08 1:46/El]
n.	muˈki		o-muˈgi		‘woman’				[BFL 05 1:156/El]
o.	raˈna-la		a-taˈna-la	‘offspring’				[BFL 05 1:156/El]

Given that both prefixation and consonant mutation are attested as independent exponents of pluractionality, the forms in (12k-o) may be analyzed as an instance of multiple (extended) exponence, where two independent markers of the construction are redundantly deployed in single word forms.

In terms of the distribution of the different exponents of pluractionality, two generalizations obtain. First, the cases where pluractionality is only marked through prefixation (12a-b) involve no oral bilabial or coronal stops in the stem; mutation may thus be analyzed as a regular exponent of pluractionality and realized when there is a possible target in the stem, whether it appears as the single exponent of pluractionality or comes along with prefixation. Second, with the exception of one lexicalized form (o-muˈgi ‘women’, muˈki ‘woman’ in (12n)), there are no attested pluractional stems with velar stop mutation in the CR data, e.g. (12c-f) and (12k-l).

CR consonant mutation as the single exponent of pluractionality brings about a typologically relevant case of exponence, since this pattern instantiates a case of polarity, a phenomenon where a segment surfaces with a sub-segmental binary feature value opposite to its own input value as the exponent of a morphological category ([] [] and [] [] in a given context) (Baerman 2007, Trömmer 2008, De Lacy 2012). In the case of CR, this means that no single voicing feature value is consistently associated with either singular or pluractional forms; instead, pluractional marking involves opposite voicing features of oral stops between base stems and their pluractional counterparts. This is illustrated in (13).

(13)	Polarity in CR pluractional marking

	[-voice] → [+voice]
		SG		PL
a.	kaˈpo 	kaˈbo 		‘to be round (pl. obj.)’  				[ref]
b.	saˈpe-	saˈbe-		‘to be fat (pl. obj.)’				[ref]
c.	kaˈpi-	kaˈbi-l-ame	‘cylindrical (pl. obj)’				[ref]
d.	siˈta- 	i-siˈra- 		‘to be red’ (pl. obj)’				[BF 05 1:157/El]

		[+voice] → [-voice]
e.	buˈke	puˈke		‘own domestic animals’	 			[BFL tx48:20]
f.	biˈte	i-piˈre 		‘dwell’						[BF 05 1:186/El]
g.	roˈsa-	toˈsa-		‘to be white (pl. obj.)’				[ref]
h.	remaˈli	ˈtemuli		‘young men’					[ref]

As shown in (13), the pattern of distribution of mutation in singular-pluractional pairs is consistent: whenever there is a voiceless stop in the singular, this stop will be voiced in the pluractional (13a-d), and whenever there is a voiced stop in the singular, it will be a voiceless stop in the pluractional form (13e-h).

Patterns of polarity in the theoretical literature have been analyzed as involving a generative phonological component (polarity as an ‘exchange rule’ or ‘morphophonological’ polarity) or, conversely, as involving exclusively the morphological component (‘morphological polarity’).  In the former type, an exchange of phonological features encodes a morphological category in the same phonological environment, as in Dholuo (aka Luo; West \ili{Nilotic}) plural marking, where the rightmost root oral stop in the singular has the opposite [voice] value in the plural (Okoth-Okombo 1982, Tucker 1994, Anderson 1992; cf. Trömmer 2008, 2012; DeLacy 2012). In contrast, morphological polarity can be expressed as involving distinct morphemes that have constant phonological forms, but that appear to toggle or reverse the value of a morphological feature in the base, e.g., gender marking in \ili{Hebrew}, where masculine and feminine are marked by suffixes that take the opposite functional value depending on the morphological context (adjective or numeral bases) (Baerman 2007). The question is thus whether the reversal involves a phonological or a morphological process.

Consonant mutation in pluractional marking in the closely related \ili{Norogachi Rarámuri} variety is given an analysis in Lionnet (1968) that is compatible with morphophonological polarity. \ili{Norogachi Rarámuri} is a Northern variety (part of the Alta Tarahumara dialect continuum) with the same morphological and prosodic characteristics as CR. According to Lionnet, most stems in this variety have second or third syllable stress, fortis (voiceless) oral stops in even-numbered syllables and lenis (voiced) oral stops in odd-numbered syllables (e.g., biˈtoli ‘bowl’, goˈtʃi ‘to sleep’). When these stems are marked for the pluractional, fortis stops mutate to lenis and lenis to fortis in pluractional forms, whether or not prefixation is also involved in pluractional marking (e.g., piˈroli ‘bowls’, o-koˈtʃi ‘to sleep, PL’) (1968:137-9).  Thus, under this analysis, the distribution of fortis and voiced stops in base forms is characterized in terms of syllable count, but their corresponding pluractional forms have distributions of fortis and lenis stops that are not predictable from syllable count or another morphophonological property in the derived form, given that pluractional forms may or may not have concomitant prefixation. Thus, the pluractional consonant quality alternation is analyzed as involving the exchange of voicing features in the same phonological environment.

This pattern may, however, be reanalyzed as involving no morphophonological polarity but two distinct mutation processes that are sensitive to MORPHOLOGICAL STRUCTURE in pluractional forms, where pluractional marking requires inter-vocalic mutation of oral stops to the voiced grade, except in stem-initial position, where the voiceless grade is required. In this analysis, the STEM is a domain that includes the root plus any suffixes, but excludes prefixes (all unproductive in this variety). This is exemplified in (14), where stem boundaries are marked with square brackets.

(14)	Distribution of consonant mutation in \ili{Norogachi Rarámuri} pluractional marking

	SG			PL
a.	uˈpe		[uˈbe]			‘to marry’ (casarse)				[L137]
b.	tʃiˈtu		[tʃiˈru]			‘to be round’ (redondo)				[L138]
c.	biˈtoli		[piˈroli] 		‘bowl’ (cajete)					[L137]
d.	roˈpa		[toˈba]			‘to surpass’ (sobrepasar)			[L138]
e.	riˈku		[ˈtegu]			‘to be drunk’ (emborracharse)			[L143]
f.	goˈtʃi		o-[koˈtʃi] 		‘to sleep’ (dormir)				[L139]
g.	buˈkula		u- [puˈgula]		‘cattle’ (res)					[L139]
h.	raˈta		a-[taˈra]		‘to be hot’ (estar caliente)			[L140]
i.	bahoˈni		a-[paˈhoni]		‘to cross a river/stream’ (vadearse [sic])		[L140]

As shown in these examples, mutation in pluractional marking involves the full oral stop series, including velars (e.g., (14e, g)), in contrast to CR where velar stops are not targets of mutation.
	This analysis can be extended to pluractional marking in CR. This is exemplified in (15). Fortis stops in pluractional forms are highlighted in boldface and stem boundaries are marked with square brackets.

(15)	Distribution of mutation in CR pluractional marking

	SG			PL
a.	kaˈpo	 	[kaˈbo]			‘to be round’				[BF 05 1:155/El]
b.	kaˈpi-		[kaˈbi]			‘to be cylindrical’			[BF 05 1:156/El]
c.	buˈke		[puˈke]			‘to own domesticated animals’		[RF 04 1:66/el]
d.	saˈpe		[saˈbe]			‘to be/become fat’			[BF 05 1:156/El]
e.	roˈsa		[toˈsa]			‘to be white’				[BF 05 1:157/El
f.	kiˈpa 		i-[kiˈba]		‘to snow’				[SF 05 2:8/El]
g.	kupuˈwe		u-[kuˈbe]		‘to grill pepper(s)’			[SF 08 1:46/El]
h.	siˈtakame 		i-[siˈrak-ame] 		‘to be red’				[BF 05 1:157/El]
i.	baˈhi 		a-[paˈhi]		‘to drink’				[SF 08 1:46/El]
j.	muˈki		o-[muˈgi]		‘woman’				[BF 05 1:156/El]
k.	raˈna-la		a-[taˈna-la]		‘offspring’				[BF 05 1:156/El]

As in \ili{Norogachi Rarámuri}, CR requires its pluractional forms to mutate its oral stops to the voiced grade in intervocalic position, except for the ones in stem-initial position, which require the voiceless grade. In the absence of any prefixation in other morphological environments (as in the constructions described in §2.2), the left edge of the stem and the prosodic word are isomorphic. Thus, in both in \ili{Norogachi Rarámuri} and CR, pluractional marking involves two mutation patterns that appear in complementary morphological contexts


\section{Gradient, speech rate dependent phonetic lenition}
\label{sec: gradient, speech rate dependent phonetic lenition}

In addition to the phenomena outlined above, CR undergoes a process of phonetic, post-lexical phenomenon that targets plain (non-laryngealized) voiceless stops, which exhibit a large variety of surface realizations: voiceless oral stops are gradiently realized within a continuum that ranges from a voiceless aspirated stop (if stressed) to a labio-velar glide or deletion altogether (schematically: [ph > p > b > β > β̩ > w > Ø]). These effects in these contexts are so strong that they are clearly perceptible without any instrumental analysis. This continuum is exemplified in (16) for the bilabial and velar places of articulation with instances of the realization of particle pa, a monosyllabic function word that marks clause boundaries (marked with square brackets) (16a-b), and with the topic marker ko particle (16c-d), respectively.

(16)	Gradient production of pa – clause final marker (CL) - and ko ~ go – topic marker (TOP)

a.	napu 	koˈɾi-ki 		biˈti 		ba]
 	COMP	around.the.side-LOC	lie.PL.PRS	CL
	‘Like the ones who lie in that other side (by the graveyard)’
	‘Como los que están (acostados) de aquel lado (del panteón)’
	[GFM co1136[17\_430-17\_445]]

b.	aʔˈɾi 	biˈla 	ko 	waˈbe 	biˈɾa 	kiˈʔa 		ˈni-ɾa 		ra 	pa] 	kuˈɾi
	and		indeed	TOP	INT	then	long.ago	COP-REP	REP	CL	recently

	ke 		tʃo 	me 	biwaˈtʃe-a-tʃi 		na 	kaˈwi 	β̩a]
	NEG	yet	almost	solidify-PRS-TEMP	PROX	earth	CL
	‘And so it was a long time before this earth was solid’
	‘Y entonces fue mucho cuando todavía no amacizara este mundo’
	[tx43[11\_112-11\_182]]

c.    taˈmo 	ko 	he 	riˈga 	ˈnotʃ-ami 	hu 	na 	iˈseɾi-g-am 		go ...
we		TOP	like	that	work-PTCP	COP	PROX	be.governor.PL-g-PTCP	TOP
	‘We, that is how we work, the governors...’
	‘Nosotros los gobernadores así trabajamos...’
	[JMF tx816:0:00.0]

d.   aʔlám 	rikátʃi=mi 	aʔlá 	iwé-ami 	raʔamá-bi 		re 	ɣo 	a mí
	well	like.that=2.OBJ	well	strong-PTC	give.advice-IRR.PL	DUB	TOP	there

	raʔamá-mi 		ré 	pagótami 	ba’alí
	give.advice-IRR.SG	DUB	people		tomorrow
	‘Perhaps tomorrow people will give you all advice’
	‘A lo mejor de aqui a mañana llegan gentes a darles consejos’
	 [MFH tx1132:0:30.3]

These forms exemplify voiced (16a) and voiceless (16b) stop realizations of /p/, as well as voiced approximant productions (16b). As with bilabial stops, velar stops display a range or surface realizations, ranging from voiceless stops (16c) and voiced stops (16c) to voiced fricatives (16d).

Voiceless coronal stops also undergo lenition in fast speech, and this process is attested in inter-vocalic position. In contrast to bilabial and velar stops, however, coronal stops do not voice when undergoing lenition; instead, they undergo spirantization and a change in place of articulation. That is, in contrast to the pattern found in other morphological and phonological environments where the voiceless coronal stop alternates with a voiced coronal flap, the lenis counterpart of [t] in these fast speech contexts is a voiceless inter-dental fricative. This effect is exemplified in (17).

(17)	Spirantization of voiceless coronal stops in fast speech

a.	ˈnámi 	ɾiˈkaàtʃi=mi 	aʔˈla 	iˈwe-ami 	raʔaˈmá-bi 		re 	ɣo 	ˈàmi
	DEM	perhaps=PROX	well	strong-PARTC	give.advice-IRR.PL	DUB	TOP	PROX

	raʔaˈmámi 	rè 	paˈgóθami 	baʔaˈlí
	give.advice 	DUB	people		tomorrow
	‘A lo mejor de aquí a mañana llegan gentes a darles consejos’
	<MFH tx1132:0:30.3>

b.	ˈnáθa	kú	seˈbá-bo	βa
	so.that	REV	reach-FUT.PL	CL
	‘So that (you all) arrive well’
	‘Para que lleguen bien’
	[MFH tx1133:1:24.0]

In contrast to the voicing alternations and allophonic processes that operate in the lexical phonology, this post-lexical lenition process does involve contrast neutralization. While pre-aspirated stops may lose aspiration in unstressed position, they resist the kind of lenition displayed by voiceless unaspirated stops. Voiced stops, on the other hand, undergo the same lenition processes as voiceless unaspirated stops. Thus, post-lexical lenition may yield contrast neutralization in fast speech between voiceless unaspirated stops and pre-aspirated stops in unstressed position, and neutralization of contrast between voiceless unaspirated stops and voiced stops in both stressed and unstressed position.

Finally, while the precise role of post-lexical phonology in conditioning lenition in these environments is still under investigation, there is a clear effect of a constraint that precludes [+continuant] consonants ([β~β̞~w~Ø]) in post-consonantal position, a constraint that applies across word boundaries, e.g., oˈwaam pa, tʃuˈkulam ba, iˈserigam go. In the corpus data examined so far, there are no [+continuant] consonants in these environments, though they are systematically attested in word-initial position following vowel-final words, suggesting that a constraint like *CC[+continuant] is at play.

%stuff below is from previous version

Plain (non-laryngealized) voiceless stops exhibit a large variety of surface realizations as a post-lexical phenomenon. Specifically, voiceless oral stops are gradiently realized within a continuum that ranges from a voiceless aspirated stop (if stressed) to a labio-velar glide or deletion altogether (schematically: [p\textsuperscript{h} > p > b > $\beta $\textbf{ > }$\beta ̩$ > w > Ø]). Voicing effects in these contexts are so strong that they are clearly perceptible without any instrumental analysis. This continuum for the bilabial place of articulation is exemplified in (14) with instances of the realization of particle \textit{pa} (in boldface), a monosyllabic function word that marks clause and/or utterance boundaries (marked with square brackets).

% Gradient production of \textit{pa} – clause final marker (\textsc{cl)}

% \begin{itemize}
% \item
% aʔˈɾi        ˈetʃi     biˈla=ti       beˈne-li     tamuˈhe    \textbf{ba}]
% \end{itemize}

%   then   \textsc{dem}    indeed=1\textsc{pl.nom} learn-\textsc{pst} 1\textsc{pl.nom}   \textbf{\textsc{cl} }

%   tʃu   riˈka    tiˈbu-sa      ˈle     \textbf{pa}]    na      kaˈwi    \textbf{ba}]

%  \textsc{q} that take.care-\textsc{cond}   \textsc{irr} \textbf{\textsc{cl}} \textsc{prox}    land   \textbf{\textsc{cl}}

%     ‘Then that is how we learned, how to take care of it, this earth’

%     ‘Entonces así aprendimos nosotros, cómo cuidarla, la tierra’

%     [SFH tx977:0:60.0]

% \begin{itemize}
% \item
% napu   koˈɾi-ki     biˈti     \textbf{ba}]
% \end{itemize}

%    \textsc{comp}  around.the.side-\textsc{loc}  lie\textsc{.pl.prs} \textbf{\textsc{cl}}



%   ‘Like the ones who lie in that other side (by the graveyard)’



%   ‘Como los que están (acostados) de aquel lado (del panteón)’



%   [GFM co1136[17\_430-17\_445]]


% \begin{itemize}
% \item
% ˈkiti   tʃiˈhunu-ɾ{}-am     tʃoʔˈma   \textbf{$\beta $a}]
% \end{itemize}

%   Because  be.disgusted-\textsc{ag}{}-\textsc{ptcp} snot \textbf{\textsc{cl}}

%   ‘Because he was disgusted by the snot’

%   ‘Porque le tuvo asco al moco’

%   [SFH tx128[2\_282-2\_309]]

% \begin{itemize}
% \item
% aʔˈɾi   biˈla   ko   waˈbe   biˈɾa   kiˈʔa     ˈni-ɾa     ra
% \end{itemize}

%   and    indeed  \textsc{top}  \textsc{int} then  long.ago  \textsc{cop-rep  rep}

%  \textbf{pa}]   kuˈɾi

%  \textbf{\textsc{cl}} recently

%   ke     tʃo   me   biwaˈtʃe-a-tʃi     na   kaˈwi   \textbf{$\beta ̩$a]}

%   \textsc{neg} yet  almost  solidify-\textsc{prs-temp  prox} earth  \textbf{\textsc{cl}}

%   ‘And so it was a long time before this earth was solid’

%   ‘Y entonces fue mucho cuando todavía no amacizara este mundo’

%   [tx43[11\_112-11\_182]]

As shown in these examples, the onset of this function word is realized as a voiceless stop (14a,d), a voiced stop (14a,b), a voiced fricative (14c) or a voiced approximant (14d), often displaying different realizations in different positions within the same utterances.

Voiceless velar stops also undergo gradient production, with variable voicing and spirantization. The forms in (15) exemplify this process with the topic marker \textit{ko} particle (in boldface).

% \textit{ko {\textasciitilde} go} – topic marker (\textsc{top})

% \begin{itemize}
% \item
% aʔˈɾi   biˈla    \textbf{ko}   miˈtii-ra            ˈle      riˈho   \textbf{go},
% \end{itemize}

%   then  indeed   \textbf{\textsc{top}}  win.\textsc{pst.pass-rep  dub}  man  \textbf{\textsc{top}}

%   tʃiˈhuna     tʃuˈkul-am      ba   ˈetʃi    tʃoˈma   ba

%   be.disgusted   be.curved-\textsc{ptcp    cl     dem}   snot    \textsc{cl}

% ‘And that is why the Rarámuri person was beaten, because he was disgusted by the snot’

% ‘Y por eso le ganaron al tarahumar, porque le estuvo teniendo asco al moco’


% \begin{itemize}
% \item
% niˈhe   ˈje-ɾa     \textbf{go}   ˈpa-ma   ˈre   koˈbise
% \end{itemize}

%   1\textsc{sg.nom} mom-\textsc{gen} \textbf{\textsc{top}} bring-\textsc{fut.sg  dub} pinole

%   eɾmoˈsijo     ˈka

%   Hermosillo    \textsc{irr}

%   ‘My mom will bring pinole to Hermosillo’

%   ‘Mi mamá va a traer pinole a Hermosillo’

%   [SFH el444:0:57.8]

% \begin{itemize}
% \item
% taˈmo   \textbf{ko}   he   riˈga   ˈnotʃ-ami   hu   na   we
% \end{itemize}

% \textbf{ }\textsc{1pl.subj} \textbf{\textsc{top}} like  that  work-\textsc{ptcp  cop  prox  int}

% iˈseɾi-g-am     \textbf{go} ...

% be.governor.\textsc{pl-}g-\textsc{ptcp} \textbf{\textsc{top}}

%   ‘We, that is how we work, the governors...’

%   ‘Nosotros los gobernadores así trabajamos...’

%   [JMF tx816:0:00.0]

% \begin{itemize}
% \item
% mani=ˈké        ˈlár   ba   ni   ma=timi     aʔˈla
% \end{itemize}

%   to.be.liquid=\textsc{cop.imp} lár   \textsc{top}   ni  already=2\textsc{pl.subj} well

%   boˈsa-li   aˈle   baˈ\textsuperscript{h}tali   \textbf{go}   ba   ne

%   full-\textsc{pst   dub} corn.beer  \textbf{\textsc{top}} \textsc{cl} ne

%   ‘You all got full with the corn beer, I think’

%   ‘Se llenaron ustedes con el tesgüino, yo creo’

%   [MFH tx1133:0:17.3]

% \begin{itemize}
% \item
% tʃiliká   iná-s-ma     tʃoná   hónsa   \textbf{ko}   tʃo   pa   ne
% \end{itemize}

% like.that  go.\textsc{sg-mot-fut.sg} that  from  \textbf{\textsc{top}} again  \textsc{cl} ne

% (ka)  tʃiriká     bilá   pe

% ka    like.that   indeed  pe

% ‘So that you won’t be thinking any of that’

% ‘Para que de eso no vayas pensando nada’

% [MFH tx1133:1:06.0]

% \begin{itemize}
% \item
% aʔlám   rikátʃi=mi   aʔlá   iwé-ami   raʔamá-bi     re
% \end{itemize}

%   well  like.that=\textsc{2.obj}  well  strong-\textsc{ptc}  give.advice\textsc{{}-irr.pl  dub}

%   \textbf{ɣo}     a mí   raʔamá-mi     ré   pagótami   ba’alí

%   \textbf{\textsc{top}}  there   give.advice\textsc{{}-irr.sg  dub} people    tomorrow

%   ‘Perhaps tomorrow people will give you all advice’

%   ‘A lo mejor de aqui a mañana llegan gentes a darles consejos’

%    [MFH tx1132:0:30.3]

As with bilabial stops, velar stops display a range or surface realizations, ranging from voiceless stops (15a,c,e), voiced stops (15a,b,c,d), and voiced fricatives (15f). These effects are also attested in inter-vocalic position within morphologically complex words, as shown in (16):

% Inter-vocalic voicing of /k/ in morphologically complex words

% \begin{itemize}
% \item
% aʔˈɾi   ma=m       baʔˈwi   roˈʔe-ma
% \end{itemize}

%   and    already=\textsc{2sg.subj}  water  pour.\textsc{appl-fut.sg}

% oˈho-sa     aˈnau-\textbf{ka}     biˈɾe   baˈrika

%   dekernel-\textsc{cond}  measure-sim    one  bucket

% ‘You dekernel it and then you pour water, measuring in a 200 lt. container’

% ‘Lo desgranas y ya le echas agua, primero lo mides en una barrica’

% [BFL tx60:0:27.2]

% \begin{itemize}
% \item
% riˈpaki-na   ˈku   tʃuˈku-li   ti   tʃimoˈɾi   ko   iʔˈne-\textbf{ga}
% \end{itemize}

%   above-\textsc{abl  rev} stand-\textsc{pst  dem} deer    \textsc{top} watch-sim

%   tʃu   oˈɾa-sa

%  that  do\textsc{{}-cond}

%   ‘And the deer stood twatching when he did that to them’

%   ‘Y el venado lo estuvo viendo de arriba cuando les hizo eso’

%   [BFL tx191:3:37.6]

These examples show how the voiceless velar stop (in the simultaneous action \textit{–ka} suffix) emerges as either voiceless (16a) or voiced (16b) in inter-vocalic position, despite the fact there is no voicing contrast at the velar place of articulation. This effect is part of a phenomenon of variable voicing of velar stops that is also attested within lexical items, as shown in (17):

% Gradient voicing of velar stops

% \begin{itemize}
% \item
% tʃiˈ\textbf{k}o-l-ame     {\textasciitilde} tʃiˈ\textbf{g}o-l-ame         ‘thief’/’ladrón’
% \end{itemize}

% [ref ]      [ref]

% \begin{itemize}
% \item
% paˈ\textbf{k}o-t-ame     {\textasciitilde} paˈ\textbf{g}o-t-ame       ‘people’/’gente’
% \end{itemize}

% [LEL tx5:5:05.4]    [CFH tx\_korimaka:0:10.4]

% \begin{itemize}
% \item
% \textbf{k}aˈri       {\textasciitilde} \textbf{g}aˈri        ‘house’’casa’
% \end{itemize}

%   [ref]       [ref]

While variable voicing of velar stops is attested across Choguita Rarámuri speakers regardless of their exposure to other varieties of Rarámuri, voiced velar stops are identified by Choguita Rarámuri speakers as characteristic pronunciations of other dialects, and seem to function as a highly salient social/regional marker, though no detailed sociolinguistic study has been carried out in this area to date. Crucially, surface voiced velar stops have a more restricted distribution than surface voiced bilabial stops and coronal flaps in monomorphemic words and, as discussed in §2.4, do not emerge in any morphological alternations.\footnote{Voiced velar stops are only marginally attested in underlying forms in some place names, as in the form \textit{basiˈ}\textbf{\textit{g}}\textit{otʃi} ‘Basigóchi’, a toponym containing the root \textit{basiˈko} and the locative suffix \textit{{}-tʃi} that means ‘place where \textit{basiˈko} grows’. The root \textit{basiˈ}\textbf{\textit{k}}\textit{o} is synchronically used to refer to a plant species, and it can be productively derived with the locative suffix, \textit{basiˈ}\textbf{\textit{k}}\textit{o-tʃi}. This last form with the voiceless velar stop in the stem plus the locative suffix has the meaning ‘on top of the plant \textit{basikó}’.}

Finally, voiceless coronal stops also undergo lenition in fast speech, and this process is attested in inter-vocalic position. In contrast to bilabial and velar stops, coronal stops do not voice when undergoing lenition; instead, they undergo spirantization and a change in place of articulation. That is, in contrast to the pattern found in other morphological and phonological environments where the voiceless coronal stop alternates with a voiced coronal flap, the lenis counterpart of [t] in these fast speech contexts is a voiceless inter-dental fricative. This effect is exemplified in (18).

% Spirantization of voiceless coronal stops in fast speech

% \begin{itemize}
% \item
% ˈnámi   ɾiˈkaàtʃi=mi   aʔˈla   iˈwe-ami   raʔaˈmá-bi     re
% \end{itemize}

%   \textsc{dem}  perhaps=\textsc{prox}  well  strong-\textsc{partc}  give.advice-\textsc{irr.pl}  \textsc{dub}

%   ɣo     ˈàmi   raʔaˈmámi   rè   paˈgó\textbf{$\theta $}ami   baʔaˈlí

%   \textsc{top    prox}    give.advice   \textsc{dub}  people    tomorrow

%   ‘A lo mejor de aquí a mañana llegan gentes a darles consejos’

%   <MFH tx1132:0:30.3>

% \begin{itemize}
% \item
% ˈná\textbf{$\theta $}a  kú  seˈbá-bo  $\beta $a
% \end{itemize}

%   so.that  \textsc{rev}  reach-\textsc{fut.pl  cl}

%   ‘So that (you all) arrive well’

%   ‘Para que lleguen bien’

%   [MFH tx1133:1:24.0]

Preliminary examination of phonetic data reveals these gradient realizations in fast speech reflect a lenition process in which productions on the lenis end of the continuum (fricatives and approximants) tend to be produced in utterance-final position, while productions on the fortis end (voiceless and voiced stops) tend to be produced utterance-medially, a factor that may suggest that these alternations are sensitive to phrasal phonological effects.\footnote{Lenition results in the neutralization of the voicing contrast at the bilabial place of articulation, with the voiced bilabial stop also displaying a gradient realization ranging from a stop proper, to a fricative, an approximant, a labio-velar glide or deletion ([b{\textasciitilde}$\beta ${\textasciitilde}$\beta ̞${\textasciitilde}w{\textasciitilde}Ø]).}{}

While the precise role of post-lexical phonology in conditioning lenition in these environments is still under investigation, there is a clear effect of a constraint that precludes [+continuant] consonants ([$\beta ${\textasciitilde}$\beta ̞${\textasciitilde}w{\textasciitilde}Ø]) in post-consonantal position, a constraint that applies across word boundaries, e.g., \textit{oˈwaam} \textbf{\textit{pa}}, \textit{tʃuˈkulam} \textbf{\textit{b}}\textit{a} (15a), \textit{iˈserigam} \textbf{\textit{g}}\textit{o} (15c). In the corpus data examined so far, there are no [-continuant] consonants in these environments, suggesting that a constraint like *CC[+continuant] is at play. This post-lexical lenition process does involve contrast neutralization, a crucial difference between this process and the lenition processes operating in the lexical phonology.

%everythign below this line needs to be re-assessed

The structural relationship between pairs of ‘plain’ voiceless stops and their corresponding voiced counterparts is evidenced in alternations that show up synchronically as suppletive allomorphy:\footnote{It has been proposed that this unpredictable, idiosyncratic variation is a remnant of historical alternations involving voiceless/fortis stops and voiced/lenis stops, a phenomenon present in the \ili{Numic} branch of \ili{Uto-Aztecan} (cf. \ili{Southern Paiute} \citet{sapir1930southern}, \ili{Kawaiisu} (\citealt{zigmond1991kawaiisu})). For \ili{Numic}, the voicing alternations have been attributed to once productive phonological processes of gemination, spirantization, and pre-nasalization \parencite{sapir1930southern}. For \ili{Taracahitic}, on the other hand, the voicing alternations have been argued to be conditioned by stress position, a process that has been documented as still productive in languages like River \ili{Guarijío} \citep[][52]{miller1996guarijio}. In his analysis, Miller proposes that voiceless bilabial and velar stops voice in intervocalic position followed by an atonic vowel, and voiceless alveolar stops rhotacize intervocalically in posttonic position.} roots and suffixes with these segments have a phonetic shape that cannot be predicted from an underlying form, but are rather idiosyncratic. The \textit{p {\textasciitilde} b} alternation is exemplified with the productive future plural suffix. This suffix can have either a voiced bilabial stop onset (42a-c) or a voiceless bilabial onset (42d-f) post-vocalically.

%   p {\textasciitilde} b in future plural suffix allomorphs

%   a.    witʃ͡ó-\textbf{b}o  ‘wash(clothes)-fut.pl’      [AHF 04 1:69/el]

% b.   newá-\textbf{b}o  ‘make-fut.pl’/‘hacer-fut.pl’    [SFH 04 1:67/el]

% c.   wí-\textbf{b}o     ‘harvest-fut.pl  ’/‘pizcar-fut.pl’   [SFH 04 1:69/el]

% d.   pakó-\textbf{p}o  ‘wash(dishes)-fut.pl’       [SFH 04 1:69/el]

% e.  nará-\textbf{p}o  ‘cry-fut.pl’/‘llorar-fut.pl’    [BFL 04 1:74/el]

%   f.  tetʃ͡í{}-\textbf{p}o    ‘comb-fut.pl’/‘peinar-fut.pl’    [SFH 04 1:69/el]

These examples show how the voicing of the future plural suffix onset is not rule governed, but is instead lexically suppletive: voicing of the bilabial stop onsets in the future plural suffixes in (42) cannot be predicted by their intervocalic position, the quality of the preceding vowel, nor stress position; the allomorphy must thus be assumed to be lexically determined by each root. There is also speaker variation as to the choice of the allomorph.\footnote{For example, one consultant strongly rejected \textit{*pá-}\textbf{\textit{b}}\textit{o,}‘throw-fut.pl’ and corrected it to \textit{pá-}\textbf{\textit{p}}\textit{o}, with a voiceless onset for the future plural suffix. Another speaker gave precisely the form \textit{pá-}\textbf{\textit{b}}\textit{o}, with a voiced onset for the future plural, spontaneously during elicitation of the same morphologically complex word. For some speakers, the lexically suppletive allophonic alternations might be the subject of a change in progress, where the allomorphy is reinterpreted as dependent on the quality of the preceding vowel. For these speakers, the distribution of voiced and voiceless plosives seems to be dependent on vowel height, and prefer voiced/lenis stops after [–high] vowels.}

An example of the \textit{t {\textasciitilde} r} alternation is found in the distribution of potential and causative suffix allomorphs: both an allomorph with an alveolar flap onset (43a-b) and an alveolar stop onset (43c-d) can be found in intervocalic environments.

%   t {\textasciitilde} r in potential and causative suffix allomorphs

% a.   mahá-\textbf{r}a  ‘scare-pot’/‘asustar-pot’     [05 1:154/el]

% b.   ko’á-\textbf{r}i-a  ‘eat-caus-prog’/‘comer-caus-prog’  [RF 04 1:109/el]

%   c.   tú-\textbf{t}a    ‘bring-pot’/‘traer-pot’     [BFL 07 2:21/el]

%   d.  napá-\textbf{t}i-ma  ‘hug-caus-fut.sg’       [BFL VDB/el]

There are no suffixes with velar stop onsets that display the alternations described above. Velar stops, however, are optionally voiceless or voiced in fast speech as the onsets of functional words, as a result of a post-lexical lenition process, described in more detail below (\ref{sec: phonetic reduction}).

Finally, oral stops are also subject to a general phonological rule: without exception, stops devoice post-consonantally. In (47) posttonic vowel deletion yields an environment in which the onset of the future plural is necessarily voiceless. For instance,  as shown in (47a), \textit{náar-}\textbf{\textit{p}}\textit{o,} but not \textit{*náar-}\textbf{\textit{b}}\textit{o,} is unattested after posttonic deletion. There are no examples in my data with a voiced/lenis allophone appearing post-consonantally.

%   Post-consonantal voiceless oral stops

% \textit{Attested  Gloss      Unattested}

% a.   náar-\textbf{p}o   ‘ask-fut.pl’/    *náar-\textbf{b}o   [SFH 07 in243/in]

% b.  desfilár-\textbf{p}a  ‘parade-fut.pl’/  *desfilár-\textbf{b}a  [LEL 06 Nov5/el]

% c.  bam-\textbf{p}á-sa  ‘year-inch-Cond’/  *bam-\textbf{b}á-sa  [SFH 06 tx12/tx]

% d.  sam-\textbf{p}á    ‘be.wet-inch’/    *sam-\textbf{b}á   [SFH 04 1:113/el]

There is evidence that the onsets of the causative, future plural, and inchoative in (47) have voiced onsets with the same roots in other morphological contexts. The examples in (48) show that the voiceless allomorphs in (48b) and (48e) are not lexically determined, but phonologically conditioned. The root \textit{bami} has an inchoative suffix allomorph with a voiced onset in (48a) and a voiceless allomorph in (48b) after pre-tonic vowel deletion; the transitive stem \textit{rapa-na,} ‘split-tr’ (\textit{‘partir}{}-tr’), has a future plural suffix allomorph with a voiced onset in (48d) and a voiceless allomorph in (48e) after posttonic vowel deletion.{} \footnote{Some lexical items have variable pronounciations with bilabial stop and bilabial nasal alternants. The positional verbal predicate for liquids /maná/, for instance, has alternative pronunciations with a bilabial nasal stop (\textbf{\textit{m}}\textit{aná}) and with a voiced bilabial oral stop (\textbf{\textit{b}}\textit{aná}).}

%   Phonologically conditioned devoicing

% \textit{Form    Gloss}

% a.  bamí-\textbf{b}a-ri  ‘year-inch-pst’/‘año-inch-pst’

% b.   bam-\textbf{p}á-sa  ‘year-inch-Cond’/‘año-inch-Cond’  [SFH 06 tx12/tx]

% c.    *bam-\textbf{b}á-sa

% d.  rapa-ná-\textbf{b}o  ‘split-tr-fut.pl’/‘partir-tr-fut.pl’  [AHF 05 1:131/el]

% e.   rapá-m-\textbf{p}o  ‘split.appl-tr-fut.pl’       [AHF 05 1:131/el]

% f.  *rapá-m-\textbf{b}o

% These alternations result from application of the rule in (49), which is fed by pre- or posttonic vowel deletion.

%   Post-consonantal stop devoicing

% [+ voice] stop  [F0D5?]    [-voice] / C\_\_

%% Need to assess where this information goes - chapter on voicing alternations?

%the stuff below appears to be repeated from above
\subsubsection{Laryngeal contrasts for stops}
\label{subsubsec: laryngeal constrasts for stops}

As noted previously, there is a three-way laryngeal contrast for stops, with voiceless pre-laryngealized (pre-apirated) stops (specified with the feature [+ spread glottis]), voiceless ‘plain’ stops (with no laryngealization feature involved) and voiced stops. Plain voiceless stops and voiced stops, both laryngeally unspecified, surface in complex patterns of consonant quality alternations involving voicing. These alternations may result from a variety of phonological and morphological factors, including: (i) consonant mutation in suppletive allomorphy, (ii) general phonological restrictions on voicing in consonant clusters, (iii) morphologically conditioned voicing alternations in body part incorporation, (iv) encoding of morphological contrasts, and (v) the outcome of optional, gradient effects in rapid speech. This section describes the alternations that arise in consonat mutation, the general restrictions on voicing in derived consonant clusters, and the gradient, post-lexical lenition effects. Voicing alternations that arise in morphological pluractional contrasts and body part incorporation are addressed below in Chapter {chap: verbal morphology} (§4.3.1.2 and §4.3.3, respectively).

The structural relationship between pairs of ‘plain’ voiceless stops and their corresponding voiced counterparts is evidenced in alternations that show up synchronically as suppletive allomorphy:\footnote{It has been proposed that this unpredictable, idiosyncratic variation is a remnant of historical alternations involving voiceless/fortis stops and voiced/lenis stops, a phenomenon present in the \ili{Numic} branch of \ili{Uto-Aztecan} (cf. \ili{Southern Paiute} \citep{Sapir1931}, \ili{Kawaiisu} (\citealt{zigmond1991kawaiisu})). For \ili{Numic}, the voicing alternations have been attributed to once productive phonological processes of gemination, spirantization, and pre-nasalization \parencite{sapir1930southern}. For \ili{Taracahitic}, on the other hand, the voicing alternations have been argued to be conditioned by stress position, a process that has been documented as still productive in languages like River \ili{Guarijío} \citep[][52]{Miller1996}. In his analysis, Miller proposes that voiceless bilabial and velar stops voice in intervocalic position followed by an atonic vowel, and voiceless alveolar stops rhotacize intervocalically in posttonic position.} roots and suffixes with these segments have a phonetic shape that cannot be predicted from an underlying form, but are rather idiosyncratic. The \textit{p {\textasciitilde} b} alternation is exemplified with the productive future plural suffix. This suffix can have either a voiced bilabial stop onset (42a-c) or a voiceless bilabial onset (42d-f) post-vocalically.

%   p {\textasciitilde} b in future plural suffix allomorphs

%   a.    witʃ͡ó-\textbf{b}o  ‘wash(clothes)-fut.pl’      [AHF 04 1:69/el]

% b.   newá-\textbf{b}o  ‘make-fut.pl’/‘hacer-fut.pl’    [SFH 04 1:67/el]

% c.   wí-\textbf{b}o     ‘harvest-fut.pl  ’/‘pizcar-fut.pl’   [SFH 04 1:69/el]

% d.   pakó-\textbf{p}o  ‘wash(dishes)-fut.pl’       [SFH 04 1:69/el]

% e.  nará-\textbf{p}o  ‘cry-fut.pl’/‘llorar-fut.pl’    [BFL 04 1:74/el]

%   f.  tetʃ͡í{}-\textbf{p}o    ‘comb-fut.pl’/‘peinar-fut.pl’    [SFH 04 1:69/el]

These examples show how the voicing of the future plural suffix onset is not rule governed, but is instead lexically suppletive: voicing of the bilabial stop onsets in the future plural suffixes in (42) cannot be predicted by their intervocalic position, the quality of the preceding vowel, nor stress position; the allomorphy must thus be assumed to be lexically determined by each root. There is also speaker variation as to the choice of the allomorph.\footnote{For example, one consultant strongly rejected \textit{*pá-}\textbf{\textit{b}}\textit{o,}‘throw-fut.pl’ and corrected it to \textit{pá-}\textbf{\textit{p}}\textit{o}, with a voiceless onset for the future plural suffix. Another speaker gave precisely the form \textit{pá-}\textbf{\textit{b}}\textit{o}, with a voiced onset for the future plural, spontaneously during elicitation of the same morphologically complex word. For some speakers, the lexically suppletive allophonic alternations might be the subject of a change in progress, where the allomorphy is reinterpreted as dependent on the quality of the preceding vowel. For these speakers, the distribution of voiced and voiceless plosives seems to be dependent on vowel height, and prefer voiced/lenis stops after [–high] vowels.}

An example of the \textit{t {\textasciitilde} r} alternation is found in the distribution of potential and causative suffix allomorphs: both an allomorph with an alveolar flap onset (43a-b) and an alveolar stop onset (43c-d) can be found in intervocalic environments.

%   t {\textasciitilde} r in potential and causative suffix allomorphs

% a.   mahá-\textbf{r}a  ‘scare-pot’/‘asustar-pot’     [05 1:154/el]

% b.   ko’á-\textbf{r}i-a  ‘eat-caus-prog’/‘comer-caus-prog’  [RF 04 1:109/el]

%   c.   tú-\textbf{t}a    ‘bring-pot’/‘traer-pot’     [BFL 07 2:21/el]

%   d.  napá-\textbf{t}i-ma  ‘hug-caus-fut.sg’       [BFL VDB/el]

There are no suffixes with velar stop onsets that display the alternations described above. Velar stops, however, are optionally voiceless or voiced in fast speech as the onsets of functional words, as a result of a post-lexical lenition process, described in more detail below ().

\section{Summary}
\label{sec: summary}

CR exhibits a three-way contrast between voiceless oral stops specified for a [+spread glottis] feature (surface pre-aspirated), laryngeally unspecified ‘plain’ voiceless stops, and voiced stops. Consonant quality alternations, whether phonological or morphological, exclusively target plain voiceless stops and involve manipulation of voicing. Alternations at the velar place of articulation are only attested in the post-lexical domain, a pattern that is attributable to the fact that there is no contrastive voiced velar in CR. \tabref{tab:23:3} summarizes these patterns.

\begin{table}
\begin{tabularx}{\textwidth}{Qlll}
\lsptoprule
Pattern	& p - b &	t - r	& k – g\\
\midrule
General phonological constraints (*CC[+voice]) &	\ding{51}&	\ding{51}&	\ding{55}\\
Consonant mutation in suppletive allomorphy	& \ding{51}&	\ding{51}&	\ding{55}\\
Consonant mutation in incorporation	&\ding{51}	&\ding{51}	&\ding{55}\\
Consonant mutation as a morphological exponent (pluractional)	&\ding{51}	&\ding{51}&	\ding{55}\\
Gradient, phonetic lenition in fast speech (*CC[+continuant])&	\ding{51}&	\ding{51}&	\ding{51}\\
\lspbottomrule
\end{tabularx}
\caption{Consonant quality alternations of plain (unaspirated) CR stops}
\label{tab:23:3}
\end{table}

While voicing alternations in suppletive allomorphy, incorporation and pluractional marking are amenable to abstract phonological analyses, I argue that they require morphological conditioning nonetheless and that their phonological similarities stem from their historical development.
