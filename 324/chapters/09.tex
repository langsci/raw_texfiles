\chapter{Verbs and the verbal complex}
\label{chap: verbal morphology}

%%Discussion of inflection classes: morphomic classes; tonal classes and interaction with tonal morphology

%%Situate the larger picture of stem alternations across UA – how they are used as a tool in comparative/diachronic analyses, and how individual languages receive largely morpho-phonological analyses, though recent literature reflects the alternations as tied to syntactic structure [here make reference to H. Harley’s work on \ili{Yaqui} – the morpho-phonology and morphosyntax of \ili{Yaqui} bound stems – talk given at UCSD, references therein]

This chapter provides an overview of the verbal morphology, the verbal complex and morphologically-conditioned phonological processes of Choguita Rarámuri and provides evidence for a hierarchical structure of the verb. The chapter is structured along two important topics in the language: (i) phonological and morphological properties of verb roots and morphological constructions (affixes and non-concatenative processes), and (ii) the morphological structure of verbs. Choguita Rarámuri verbal morphology is highly synthetic and agglutinative (where agglutination is understood to involve mostly separative-non-flexive morphology \parencite{bickel2007inflectional}). The morphological structure of the language does not instantiate a position-class system (as defined in \citealt{simpson1986pronominal} and \citealt{inkelas1993nimboran}),\footnote{Position-class or templatic analyses are posited when affix order in a given language is not governed by semantic, syntactic or phonological principles, and every
morpheme in the system is assumed to be lexically indexed for a particular fixed position in a total linear arrangement of position classes. In this kind of system, morphemes are rigidly ordered, there are formal dependencies between discontinuous suffixes, inflectional and derivational exponents are interspersed within the verbal structure, and semantically compatible suffixes might be in complementary distribution due to their membership to the same position class (\citealt{inkelas1993nimboran}; see also \citealt{rice2011principles}). None of these properties characterize the Choguita Rarámuri morphological system.} but is rather arranged in verbal zones, in concentric layers that are evidenced by the degrees of morpho-phonological fusion displayed by verbal suffixes.

This chapter provides evidence for twelve suffix positions that are grouped into six verbal domains, schematized in \tabref{tab:verb-stem-levels intro}: an Inner Stem, the input to suffixation, a Derived Stem, a Syntactic Stem, an Aspectual Stem, a Finite Verb domain and a Subordinate Verb domain. Each of these domains is semantically, morphotactically, and morpho-phonologically motivated, with morphological exponents closer to the root displaying a higher degree of phonological fusion.

\begin{table}
\caption{Choguita Rarámuri verbal stem domains}
\label{tab:verb-stem-levels intro}
\fittable{
\begin{tabular}{lll}
\lsptoprule
\textbf{Position}  & \textbf{Marker}  & \textbf{Stem domain} \\
\midrule
 & Pluractionality, verbalization, noun incorporation & Inner Stem\\
S1 & Inchoative & Derived Stem \\
S2 & Transitives & \\
S3 & Applicatives & Syntactic Stem\\
S4 & Causative & \\
S5 & Applicative & \\
S6 & Desiderative & Aspectual Stem\\
S7 & Associated Motion & \\
S8 & Auditory Evidential & \\
S9 & Voice/Aspect/Tense & Finite Verb\\
S10 & Mood & \\
S11 & TAM & \\
%\hhline{~~-}
S12 & Deverbal morphology & Subordinate Verb\\
\lspbottomrule
\end{tabular}
}
\end{table}

%an Inner Stem (the input to suffixation), a Derived Stem, a Syntactic Stem, an Aspectual Stem, a Finite Verb level, and finally a Subordinate Verb level. - include cross-references to sections where each level is addressed in this chapter and in the appendix

The Inner Stem, described in §\ref{sec: the inner stem}, is the domain of body part incorporation, non-concatenative processes and unproductive concatenative markers. The morphological processes taking place at this stage are more tightly fused phonologically to the root than any later morphological process as they may undergo haplology (\ref{ex: inner stem examplesa}), compensatory lengthening (\ref{ex: inner stem examplesa}),\footnote{In (\ref{ex: inner stem examplesa}), there is syllable deletion in avoidance of adjacent identical syllable onsets (/tʃ͡aˈbó\textbf{-pi-po}/, as well as compensatory lengthening of the stressed syllable as a result of this deletion process ([tʃ͡a\textbf{ˈbóo}po]) More details about haplology and compensatory lengthening can be found in §\ref{subsubsec: stem-suffix haplology} and §\ref{subsubsec: compensatory lengthening}, respectively.} vowel lengthening triggered by the past passive suffix (\ref{ex: inner stem examplesb}), and round harmony, among other processes (a comprehensive description of the morphologically conditioned phonological processes attested in Choguita Rarámuri verbs is provided in §\ref{subsec: phonological transparency and morpheme boundary strenght}).

\ea\label{ex: inner stem examples}
{Inner Stem examples}

    \ea[]{
    \textit{{tʃ͡a\textbf{ˈbóo}po}}\\
    {/[tʃ͡aˈbó-pi]\textsubscript{InnerStem}-po/}\\
    {beard-\textsc{rev-fut.pl}}\\
    `They will remove their beards.'\\
    `Se van a quitar la barba.' {< SFH 08 1:5/el >}\\
}\label{ex: inner stem examplesa}
        \ex[]{
        \textit{o\textbf{ˈsìi}ru}\\
        \glt    /[oˈs-ì]\textsubscript{InnerStem}-ru/\\
        \glt    write-\textsc{appl-pst.pass}\\
        \glt    ‘It was written.’\\
        \glt    ‘Fue escrito.’   < SFH 08 1:45/el >\\
    }\label{ex: inner stem examplesb}
    \z
\z

The first verbal domain involving suffixation, the Derived Stem, includes inchoative and transitive suffixes (\ref{ex: derived stem examplesa}), derivational suffixes that are only attested with change-of-state predicates (further discussion of these markers can be found in Appendix~\ref{sec: the derived stem}). This stem domain is characterized morpho-phonologically by: (i) a non-concatenative morphological process of stress shift to the final syllable of the domain to encode imperative singular; (ii) vowel lengthening induced by the past passive suffix (\ref{ex: derived stem examplesb}); and (iii) morphologically conditioned stress shifts.

\ea\label{ex: derived stem examples}
{Derived Stem examples}

    \ea[]{
    \textit{tʃ͡okoˈbánali}\\
    \glt    /[[tʃ͡oko]\textsubscript{InnerStem}-ˈbá-na]\textsubscript{DerivedStem}-li/\\
    \glt    be.sour{-\textsc{inch-tr}}-\textsc{pst}\\
    \glt    ‘S/he made them go sour.’\\
    \glt    ‘Hizo que se agriaran.’ < SFH 04 1:113/el > \\
}\label{ex: derived stem examplesa}
        \ex[]{
        \textit{rapa\textbf{ˈnâa}ru}\\
        \glt    /[[rapa]\textsubscript{InnerStem}-nâ]\textsubscript{DerivedStem}-ru/\\
        \glt    split-\textsc{tr-pst.pass}\\
        \glt    ‘She was already operated (lit. cut).’\\
        \glt    ‘Ya la operaron (cortaron).’   < SFH 08 1:84/el > \\
    }\label{ex: derived stem examplesb}
    \z
\z

\hspace*{-.3pt}The next verbal domain, the Syntactic Stem, is comprised of valence-increasing suffixes that display variable ordering, as well as multiple exponence (\ref{ex: syntactic stem examplesa}).\footnote{Two causative markers are attested in this example, instantiating multiple (extended) exponence. See \citet{caballero2011multiple} for discussion of this phenomenon in Choguita Rarámuri.} In terms of their morpho-phonological properties, these suffixes form a coherent domain since they are stress-neutral and part of the domain of a phonological process of round harmony (\ref{ex: syntactic stem examplesb}). Further discussion and examples of these markers can be found in Appendix~\ref{sec: the syntactic stem}.

\ea\label{ex: syntactic stem examples}
{Syntactic Stem examples}

    \ea[]{
    \textit{noˈkèrtikiri!}\\
    \glt  /[[noˈka-è]\textsubscript{InnerStem}-ri-ti-ki]\textsubscript{SyntacticStem}-ri/\\
    \glt        move.\textsc{appl-caus-caus-appl-imp.sg}\\
    \glt    ‘Move it for someone!’\\
    \glt    ‘¡Muéveselo!’ < BFL 08 1:28/el >\\
}\label{ex: syntactic stem examplesa}
    \ex[]{
    \textit{ʃuˈkú\textbf{ku}po}\\
    \glt    {/[[suˈkú]\textsubscript{InnerStem}-ki]\textsubscript{SyntacticStem}-po/}\\
    \glt    {scratch-\textsc{appl-fut.pl}}\\
    \glt    `They will scratch her.'\\
    \glt    `La van a arañar.' {< BFL 05 1:116/el >}\\
}\label{ex: syntactic stem examplesb}
    \z
\z

The Aspectual Stem domain is composed of disyllabic modal and aspectual suffixes that are transparently related to independent verbs in the language. As discussed in more detail in §\ref{subsec: V-V incorporation constructions}, these markers involve a process of V-V incorporation. Aspectual Stem markers may undergo round harmony (\ref{ex: aspectual stem examplesa}) and exhibit variable ordering (\ref{ex: aspectual stem examples}b--c). More details about the markers belonging to this stem domain are provided in Appendix~\ref{sec: the aspectual stem}.

\ea\label{ex: aspectual stem examples}
{Aspectual Stem examples}

    \ea[]{
    {\textit{{baniˈsú\textbf{tusu}ma}}}\\
    \glt    {/[[[baniˈsú]\textsubscript{InnerStem}-ti]\textsubscript{SyntacticStem}-simi]\textsubscript{AspectualStem}-ma/}\\
    \glt    {pull-\textsc{caus-mot-fut.sg}}\\
    \glt    `S/he will go along making them pull it.'\\
    \glt    `Los va a ir haciendo que lo jalen.' {< SFH 07 2:67 rec487 /el >}\\
}\label{ex: aspectual stem examplesa}
        \ex[]{
        \textit{sutuˈbéetʃ͡-nale}\\
        {/[[sutuˈbétʃ͡i]\textsubscript{InnerStem}-tʃ͡ane-nale]\textsubscript{AspectualStem}/}\\
        {trip-\textsc{ev-desid}}\\
        \glt    `It sounds like they want to trip.'\\
        \glt    `Se oye que se quieren tropezar.' < BFL \textsc{07} rec300/el >\\
    }\label{ex: aspectual stem examplesb}
            \ex[]{
            \glt \textit{ˈsûuntʃ͡una}\\
            \glt    /[[ˈsû]\textsubscript{InnerStem}-nale-tʃ͡ane]\textsubscript{AspectualStem}-a/\\
            \glt    sow-\textsc{desid-ev-prog}\\
            \glt    ‘It sounds like she wants to sow.’ \\
            \glt    ‘Se oye como que quiere coser.’        < SFH 07 1:9/el >\\
        }\label{ex: aspectual stem examplesc}
    \z
\z

The next two verb domains involve inflectional markers. The Finite Verb domain comprises mood, voice, tense, and aspect suffixes that close verbs used in non-subordinate clauses (\ref{ex: finite and subordinate verb examplesa}). Verbs may also add suffixes from a final, Subordinate Verb domain when used in subordinate clauses (\ref{ex: finite and subordinate verb examplesb}). Suffixes in these last two domains are largely separable and do not undergo vowel lengthening, rounding harmony or other morpho-phonological effects attested in inner verbal domains.

%\pagebreak

\ea\label{ex: finite and subordinate verb examples}
{Finite and Subordinate Verb examples}

    \ea[]{
    \textit{poˈt͡ʃítisima}\\
    \glt    /[[[[poˈt͡ʃí]\textsubscript{InnerStem}-ti]\textsubscript{SyntacticStem}-simi]\textsubscript{AspectualStem}-ma]\textsubscript{FiniteVerb}/\\
    \glt    jump-\textsc{caus-mot-fut.sg}\\
    \glt    ‘I will go along making the dog jump.’ \\
    \glt    ‘Voy a ir haciendo que brinque el perro.’    < SFH 08 1:72/el >\\
}\label{ex: finite and subordinate verb examplesa}
        \ex[]{
        \textit{omoˈwáruatʃ͡i}\\
        \glt    /[[[omoˈwá]\textsubscript{InnerStem}-riwa-a]\textsubscript{FiniteVerb}-tʃ͡i]\textsubscript{SubordinateVerb}/\\
        \glt    make.party-\textsc{mpass-prog-temp}\\
        \glt    ‘like with parties, when parties are made’\\
        \glt    ‘así como las fiestas, cuando hacen fiesta’ \corpuslink{tx12[05_405-05_444].wav}{SFH tx12:5:40.5}\\
    }\label{ex: finite and subordinate verb examplesb}
    \z
\z

Finally, inflected verbal forms are part of a larger unit, referred to here as the Verbal Complex (discussed in detail in §\ref{sec: the verbal complex clitics and modal particles}), which includes person enclitics (\ref{ex: verbal complex examplesa}) and epistemic modality markers (\ref{ex: verbal complex examplesb}). Person enclitics (which encode person, number and a nominative-accusative case distinction) are unrestricted in terms of the syntactic category of the words they may attach to. Epistemic markers (which encode the degree of certainty speakers have towards the actuality of an event) are prosodically independent markers that may only follow inflected verbs. Epistemic markers show some degree of fusion with their bases, as they may condition phonological changes in the inflected verb.\footnote{In (\ref{ex: verbal complex examplesb}), the initial vowel of the certainty marker \textit{oˈlá} replaces the final vowel of the future singular suffix \textit{-ma}.}

\ea\label{ex: verbal complex examples}
{Verbal Complex: person clitics and epistemic markers}

    \ea[]{
    \glt    \textit{kaˈtʃ͡îkini}\\
    \glt    /[[[kaˈtʃ͡î]\textsubscript{InnerStem}-ki]\textsubscript{FiniteVerb}]=ni]\textsubscript{VerbalComplex}/\\
    \glt    spit-\textsubscript{pst.ego=1sg.nom}\\
    \glt    `I spitted.'\\
    \glt    `Escupí.' \corpuslink{el656[01_276-01_296].wav}{BFL el656:1:27.6}\\
}\label{ex: verbal complex examplesa}
        \ex[]{
        \glt    \textit{wipiˈsóm\textbf{o} ˈlá}\\
        \glt    /[[[wipiˈsó]\textsubscript{InnerStem}-ma]\textsubscript{FiniteVerb} oˈlá]\textsubscript{VerbalComplex}/\\
        \glt    hit.w.stick-\textsc{fut.sg} \textsc{cer}\\
        \glt    `(I) will definitely hit it with a stick.'\\
        \glt    `(Le) voy a pegar con palo.' \corpuslink{el1080[03_290-03_309].wav}{SFH el1080:3:29.0}\\
    }\label{ex: verbal complex examplesb}
    \z
\z

As argued in this chapter, this organization of the morphology into these verbal domains is critical in understanding the patterns of morphologically-condi\-tioned phonology and variable affix ordering patterns attested in this language.

This chapter also addresses the morphological role of tone, in terms of tonal properties of verb classes, tone alternations as phonological effects imposed by certain morphological constructions, and tone as the exponent of morphological categories. The strictly phonological aspects of the interaction between stress and tone in the word prosodic system of the language are addressed in \chapref{chap: word prosody}, while the interaction between tone and intonation is addressed in \chapref{chap: tone and intonation}. Complex prosodic interactions involving stress, tone (lexical and grammatical) and intonation are described in \chapref{chap: prosody}.

This chapter is organized in two parts, which address the verb from the inside-out. The first part is concerned with verbal root classes and the prosodic properties of roots and suffixes (in §\ref{sec: verbal root classes in shifting and neutral constructions}), grammatical tonal patterns in verbal paradigms (in §\ref{sec: the role of tone in verbal morphology}), and other non-concatenative morphological processes taking place at the Inner Stem (in §\ref{sec: the inner stem}). The second part of the chapter is concerned with the suffixation domain and the clitics and modal particles that occur in the verbal complex (in §\ref{sec: verbal structure and verbal domains} and §\ref{sec: the verbal complex clitics and modal particles}). Topics addressed in this second half include the morphotactic evidence for positing different positions in the verbal template (§\ref{subsec: morphotactic evidence for affix order generalizations}), and the morpho-phonological evidence for positing verbal domains (§\ref{subsec: phonological transparency and morpheme boundary strenght}), including the distribution of stress-shifting and stress-neutral suffixes in the verbal stem. A description of the basic phonological and morphosyntactic properties of verbal suffixes in each verbal domain can be found in Appendix~\ref{sec: the derived stem}--\ref{sec: the subordinate verb}.

\section{Verbal root classes}
\label{sec: verbal root classes in shifting and neutral constructions}

\subsection{The contrast between stressed and unstressed roots}
\label{subsec: contrast between stressed and unstressed roots}

Choguita Rarámuri verbal roots can be divided in three classes depending on their underlying stress and vowel specifications. Membership in these classes is manifested through presence or absence of stress shifts and vowel quality differences, triggered by specific affixes and non-concatenative morphological processes.

Roots can be first characterized as either underlyingly stressed or unstressed. Stressed roots do not display any stress shifts or vocalic alternations in any morphological context (as exemplified in (\ref{ex: stressed roots})). All unstressed roots, on the other hand, shift stress one syllable to the right in certain morphological contexts (as exemplified in (\ref{ex: unstressed roots})). Stress syllables are in bold face.

%\doublebox{a}{b}\\
%\triplebox{c}{d}{e}\\
%\quadruplebox{f}{g}{h}{i}\\
%\gcbox{j}{k}{l}{m}{n}{o}\\

\TabPositions{2cm,3.5cm,6cm}
\ea\label{ex: stressed roots}
{Stressed roots}

    \ea[]{
    \triplebox{ri\textbf{ˈmù}-li}{‘dream-\textsc{pst}’}{2\textsuperscript{nd} syllable stress} \\
}
        \ex[]{
        \triplebox{ri\textbf{ˈmù}-la}{‘dream-\textsc{rep.ds}’}{2\textsuperscript{nd} syllable stress}\\
    }
            \ex[]{
            \triplebox{ri\textbf{ˈmù}-bo}{‘dream-\textsc{fut.pl}’}{2\textsuperscript{nd} syllable stress}\\
        }
                \ex[]{
                \triplebox{ri\textbf{ˈmù}-sa}{‘dream-\textsc{cond}’}{2\textsuperscript{nd} syllable stress}\\
            }
    \z
\z


\ea\label{ex: unstressed roots}
{Unstressed roots}

    \ea[]{
    \triplebox{su\textbf{ˈkú}{}-li}{‘scratch-\textsc{pst}’}{2\textsuperscript{nd} syllable stress}\\
}
        \ex[]{
        \triplebox{su\textbf{ˈkú}{}-la}{‘scratch-\textsc{rep.ds}’}{2\textsuperscript{nd} syllable stress}\\
    }
            \ex[]{
            \triplebox{suku-\textbf{ˈbô}}{‘scratch-\textsc{fut.pl}'}{3\textsuperscript{rd} syllable stress}\\
        }
                \ex[]{
                \triplebox{suku-\textbf{ˈsâ}}{‘scratch-\textsc{cond}’}{3\textsuperscript{rd} syllable stress}\\
            }
    \z
\z

A partial list of stress-shifting suffixes is provided in \tabref{tab:key:8}.

\begin{table}
\caption{Stress-shifting suffixes}
\label{tab:key:8}

\begin{tabularx}{\textwidth}{Xl}
\lsptoprule
 \textbf{Suffix} & \textbf{Root \textit{sukú} }\textbf{‘scratch’ (‘rasguñar’)}\\
 \midrule
 Conditional \textit{-sâ} & suku-\textbf{ˈsâ}\\
 Future sg.  \textit{-ˈmêa} /-ma & suku-\textbf{ˈmê}a\\
 Motion imperative \textit{-mê} & suku-\textbf{ˈmê}\\
 Desiderative \textit{-nále} & suku\textbf{{}-ˈná}le\\
 Future pl. \textit{-bȏ} & suku-\textbf{ˈbȏ}\\

\lspbottomrule
\end{tabularx}
\end{table}
%\hspace{3cm}

These suffixes contrast with another class of suffixes, stress-neutral suffixes, that do not trigger any stress shifts on any class of roots, exemplified in \tabref{tab:key:9}.

%change stress and tone!
%%please move \begin{table} just above \begin{tabular .
\begin{table}
\caption{Stress-neutral suffixes}
\label{tab:key:9}

\begin{tabularx}{\textwidth}{Xl}
\lsptoprule
 \textbf{Suffix} & \textbf{Root \textit{sukú} }\textbf{‘scratch’ (‘rasguñar’)}\\
 \midrule
 Past, 1\textsuperscript{st} person \textit{-ki} & su\textbf{ˈkú}{}-ki\\
 Evidential \textit{-tʃ͡ane} & su\textbf{ˈkú}{}-tʃ͡ane\\
 Imperfective \textit{-e} & su\textbf{ˈkú-}i\\
 Associated Motion \textit{-simi} & su\textbf{ˈkú-}simi\\
 Irrealis pl. \textit{-pi} & su\textbf{ˈkú-}pi\\
\lspbottomrule
\end{tabularx}
\end{table}
%\hspace{3cm}

Suffixes and non-concatenative processes that condition stress and other mor\-pho-phonological changes on unstressed roots are referred to here as ``shifting'' morphological constructions. Stress-neutral suffixes and non-concatetative processes, on the other hand, do not trigger any morpho-phonological alternations, and form ``neutral'' morphological constructions. Here I employ the term ``morphological construction'' to refer to any morphological process or pattern that combines two sisters into a single constitutent to form a complex word \citep[][12]{inkelas2005reduplication}. Each individual affix or non-concatenative process thus involves a unique morphological construction. By grouping morphological constructions into ``shifting'' and ``neutral'' classes, I suggest that morphological constructions in the language belong to two classes based on their phonological properties. The details of the phonological properties of particular morphological constructions are addressed in §\ref{sec: stress properties of roots, stems and suffixes}.


Unstressed roots can be further subdivided in two classes: (i) unstressed roots with fully specified vowels (exemplified in (\ref{ex: Unstressed roots with specified vowels (Class 2)})); and (ii) unstressed roots with root final unspecified vowels (exemplified in (\ref{ex: Unstressed roots with final unspecified vowels (Class 3)})). While the former class of roots (Class 2)  has vowels that are fully specified in their underlying representation, I assume that the latter class of roots (Class 3) have a final V slot, whose features are dependent on the morphological construction in which the root takes place. Class 2 roots display no vocalic alternations in both shifting and neutral constructions, and Class 3 roots have final root vowel raising concomitant to the stress shift in shifting constructions.

\ea\label{ex: Unstressed roots with specified vowels (Class 2)}
{Unstressed roots with specified vowels (Class 2)}

\begin{tabular}{lllll}
     & \textit{Stem} & \textit{Gloss} & \textit{Unstressed} & \textit{Stressed}\\
     & & & \textit{stems} & \textit{stems}\\
     a. & {tʃ͡oˈʔí} &‘extinguish, intr.'& {tʃ͡oʔ\textbf{i}-}&{tʃ͡oʔˈ\textbf{í}}\\
     b. & {roˈnò} & {‘boil'} & {ron\textbf{o}-} & {roˈn\textbf{ò}} \\
     c. & {kuˈrú} &`become thick'&{kur\textbf{u}-}&{kuˈr\textbf{ú}}\\
     d. & {raˈná} &{‘give birth’} & {ran\textbf{a}-} & {raˈn\textbf{á}}\\
\end{tabular}
    \z

\ea\label{ex: Unstressed roots with final unspecified vowels (Class 3)}
{Unstressed roots with final unspecified vowels (Class 3)}

\begin{tabular}{lllll}
     & \textit{Stem} & \textit{Gloss} & \textit{Unstressed} & \textit{Stressed}\\
     & & & \textit{stems} & \textit{stems}\\
     a. & {tʃ͡oʔnV} & {‘hit with fist'} & {tʃ͡oʔn\textbf{i}} & {tʃ͡oʔˈn\textbf{á}}\\
     b. & {raʔlV} & {`buy'} & {raʔl\textbf{i}-} & {raʔˈl\textbf{á}} \\
     c. & {rewV} & {‘find’} & {rew\textbf{i}-} & {reˈw\textbf{á}}\\
     d. & {itʃ͡V} & {`sow'} & {itʃ\textbf{i}-} & {iˈtʃ\textbf{á}}\\
\end{tabular}

      \z

Stressed roots, with no morpho-phonological changes, are the third type of roots in Choguita Rarámuri (Class 1). The difference between Choguita Rarámuri verbal root classes is summarized in \tabref{tab:verb-classes}.

\begin{table}
\caption{Choguita Rarámuri verbal root classes}
\label{tab:verb-classes}

\begin{tabularx}{\textwidth}{lllQl}
\lsptoprule
& \textbf{Class 1} & \textbf{Class 2}  & \textbf{Class 3} & \\
& \textit{Stressed} & \makecell[tl]{\textit{Unstressed}\\\textit{specified final V}}& \textit{Unstressed} \textit{unspecified final V}&  \\
\midrule
\textsc{pst} &  beˈnè-li  &     suˈkú-li      &   raʔˈlà-li & \textbf{Neutral}\\
\textsc{prog} &  beˈnè-a &    suˈkú-a &  raʔˈlà-a    & \textbf{Constructions}\\
\textsc{impf} &   beˈnè-i &   suˈkú-i  &  raʔˈlà-i &              \\
\tablevspace
\textsc{fut.sg} & beˈnè-ma   &          suku-ˈmêa   &     raʔˈli-ˈmêa & \textbf{Shifting}\\
\textsc{cond} &  beˈnè-sa & suku-ˈsâ    &  raʔˈli-ˈsâ &   \textbf{Constructions}\\
\textsc{desid} &  beˈnè-nale    & suku-ˈnále    &  raʔˈli-ˈnále & \\
\lspbottomrule
\end{tabularx}
\end{table}

Class 3 roots have a final stressed low vowel in neutral constructions. There are Class 2 roots that have a final low vowel, but these roots do not undergo final stem vowel raising in shifting constructions. As shown in (\ref{ex: Class 2 roots’ final specified vowels}), there are Class 2 roots with final specified \textit{o} (\ref{ex: Class 2 roots’ final specified vowels}a--b), \textit{i} (\ref{ex: Class 2 roots’ final specified vowels}c--d), \textit{u} (\ref{ex: Class 2 roots’ final specified vowels}e--f) and \textit{a} (\ref{ex: Class 2 roots’ final specified vowels}g--h).

\ea\label{ex: Class 2 roots’ final specified vowels}
{Class 2 roots’ final specified vowels}

\begin{tabular}{lllll}
     & \textit{Shifting} & \textit{Neutral}\\
     & \textit{\textsc{fut.sg}} & \textit{\textit{\textsc{pst}}} & \textit{Gloss} \\
     a. & {ron\textbf{o}-ˈmêa} & {roˈn\textbf{ò}-li} & {‘boil’}\\
     b. & {mor\textbf{o}-ˈmêa} & {moˈr\textbf{ò}-li} & {‘to smoke’} & {cf. moˈri ‘smoke’}\\
     c. & {wiʔr\textbf{i}-ˈmêa} & {wiʔˈr\textbf{í}-li} & {‘to stand, sg’} & {}\\
     d. & {tʃ͡aʔ\textbf{i}-ˈmêa} & {tʃ͡aʔˈ\textbf{í}-li} & {`grab'} & {}\\
     e. & {uk\textbf{u}-ˈmêa} & {uˈk\textbf{ú}-li} & {`to rain'} & {cf. uˈki `rain'}\\
     f. & {muk\textbf{u}-ˈmêa} & {muˈk\textbf{ú}{}-li} & {`die'} & {}\\
     g. & {ran\textbf{a}{}-ˈmêa} & {raˈn\textbf{á}{}-li} & {`give birth'} & {cf. raˈná `offspring'}\\
     h. & {ik\textbf{a}{}-ˈmêa} & {iˈk\textbf{á}{}-li} & {`be windy'} & {cf. iˈká `wind'}\\
\end{tabular}
    \z

Most Class 3 roots, on the other hand, end in high, front vowels in shifting constructions (but can also end in back, mid vowels (e.g., \textit{noko-ˈmêa} in (\ref{ex: Class 3 roots’ final specified vowels}d)). Without exception, these roots end in \textit{a} in neutral constructions. Some examples of these roots with alternating final vowels are provided in (\ref{ex: Class 3 roots’ final specified vowels}):

\ea\label{ex: Class 3 roots’ final specified vowels}
{Class 3 roots’ final specified vowels}

\begin{tabular}{lllll}
 & \textit{Shifting} & \textit{Neutral}\\
     & \textit{\textsc{fut.sg}} & \textit{\textit{\textsc{pst}}} & \textit{Gloss} \\
     a. & {os\textbf{i}{}-ˈmêa} & {oˈs\textbf{à}{}-li} & {`write, read'} & {< AHF 05 1:127/el >}\\
     b. & {itʃ\textbf{i}{}-ˈmêa} & {iˈtʃ\textbf{á}{}-li} & {`sow'} & {< SFH 05 1:78/el >}\\
     c. & {rah\textbf{i}-ˈmêa} & {raˈh\textbf{á}{}-li} & {`light up (fire)'} & {< ROF 04 1:62/el >}\\
     d. & {nok\textbf{o}-ˈmêa} & {noˈk\textbf{á}{}-li} & {`to move'} & {< BFL 05 1:114/el >}\\
\end{tabular}

    \z

The generalization is that roots with final \textit{a} are split between Class 2 roots (with no vowel alternation in shifting constructions as in (\ref{ex: Class 2 roots’ final specified vowels}g--h)), and Class 3 roots (with vowel alternations in neutral constructions (as in (\ref{ex: Class 3 roots’ final specified vowels})).

\subsection{Stress-shifting and stress-neutral constructions across Uto-Aztecan}
\label{subsec: semantic accounts of shifting and neutral constructions in Uto-Aztecan}

\subsubsection{Morphosyntactic-based accounts}
\label{subsubsec: morphosyntactic and semantic accounts}

Some descriptive and historical-comparative works on \ili{Uto-Aztecan} languages treat the stress shifts and vocalic alternations of roots just described, present in other \ili{Uto-Aztecan} languages, as stem suppletive allomorphy, where roots with no stress shifts or vocalic alternations encode non-future meanings (like past, perfective, and imperfective), and roots with stress shifts and vocalic alternations encode a future or ``unrealized'' stem meaning (like irrealis, counterfactual, imperative, and potential) \citep[][133]{langacker1977uto}. Thus, the morpho-phonological alternations attested in verb roots could be analyzed as being morphosyntactically motivated. There are, however, shortcomings to a classification of shifting and neutral constructions along a future/irrealis scale. Specifically, many of the categories in either shifting or neutral constructions cannot be characterized as having either a non-future or an ``unrealized'' meaning (such as valency-increasing categories). In addition, the past passive suffix, which is stress-shifting, patterns along with the ‘unrealized’ or future forms, contrary to the morphosyntactically-based stem allomorphy proposal.

In Choguita Rarámuri, then, the constructions that trigger a stress shift with certain roots do not themselves encode any constant morphosyntactic features or properties, but rather are a heterogenous class of morphological constructions. This fits in the definition given for \textit{morphomic} stems by \citet{aronoff1994morphology}, where ``the mapping from morphosyntax to phonological realization is not direct, but rather passes through an intermediate level" (\citeyear[25]{aronoff1994morphology}), which is purely morphological (see also \citealt{blevins2003stems}). The two sets of morphological constructions identified (shifting and neutral constructions) cannot be characterized morphosyntactically or semantically, but only phonologically (see also §\ref{subsubsec: grammatical tone distributed by morphological class} and \citealt{caballero2021grammatical}).


\subsubsection{Conjugation class analysis alternative}
\label{subsubsec: conjugation class analysis alternative}

There is an alternative analysis for Choguita Rarámuri verbal roots, and that is to treat the alternations as indicative of conjugation classes (as \citealt{brambila1953gramatica} and \citealt{lionnet1972elementos} do for other Rarámuri varieties). Under such an analysis, the stress and vocalic alternation properties of roots would reflect an arbitrary division of the lexicon; that is, the lexically conditioned stem allomorphy would be expressed through item-based alternations (stress shifts and vocalic alternations) when inflected for certain morphological categories.

Under the conjugation class analysis, there would also be three root classes in Choguita Rarámuri.\footnote{\citet{brambila1953gramatica}  and \citet{lionnet1972elementos} propose three and four conjugation classes, respectively. Lionnet includes irregular forms as additional verbal classes in his classification.} Class 1 roots would not display any stress shift or vocalic alternation when inflected for any morphological category. Class 2 roots would have a ``primary stem'' and a ``secondary stem''. For two-syllable stems, the primary stem will have second-syllable stress, while the secondary stem will have third-syllable stress, on a suffix adjacent to the root. Finally, Class 3 roots also have a primary stem and a secondary stem marked by a stress shift and a concomitant vowel alternation: the primary stem has a stem final low vowel (usually \textit{a}, but also \textit{o}), while the secondary stem has a stem final high, front high vowel (\textit{i}). \tabref{tab:inflection-classes} exemplifies these conjugational classes.

\begin{table}
\caption{Choguita Rarámuri inflection classes}
\label{tab:inflection-classes}

\begin{tabularx}{\textwidth}{XXXXX}
\lsptoprule
& \textbf{Class 1} & \textbf{Class 2}  & \textbf{Class 3} & \\
\midrule
\textsc{pst} &  beˈnè-li  &     suˈkú-li      &   raʔˈlà-li & \textbf{Primary}\\
\textsc{prog} &  beˈnè-a &    suˈkú-a &  raʔˈlà-a    & \textbf{stem}\\
\textsc{impf} &   beˈnè-i &   suˈkú-i  &  raʔˈlà-i &              \\
\tablevspace
\textsc{fut.sg} & beˈnè-ma   &          suku-ˈmêa   &     raʔli-ˈmêa & \textbf{Secondary}\\
\textsc{cond} &  beˈnè-sa & suku-ˈsâ    &  raʔli-ˈsâ &   \textbf{stem}\\
\textsc{desid} &  beˈnè-nale    & suku-ˈnále    &  raʔˈli-ˈnále & \\
\lspbottomrule
\end{tabularx}
\end{table}

Conjugational classes, as flexive formatives in general, are characterized by displaying item-based variation; this variation, however, is lexically, not morpho-phonologically, conditioned \parencite{bickel2007inflectional}. A diagnosis in favor of the conjugational class analysis would be the existence of other phenomena that would correlate with the stem types of Choguita Rarámuri besides the stress shift and vocalic alternations identified above. That is, in order to argue that a conjugational class analysis is the correct one for the stem alternations in Choguita Rarámuri, there should be other phenomena correlating with the arbitrary division of the lexicon that could not be explained as arising from morpho-phonological alternations. There are two phenomena in Choguita Rarámuri that could potentially fit this definition. First, Class 2 and Class 3 stems (both of which are unstressed) pattern together in participating in valence-related alternations (described in §\ref{subsec: valence alternations}). However, this valence marking system does not distinguish between any subgroup of the unstressed stems, so there would not be evidence for the distinction between Class 2 and Class 3 stems. Second, there is an apparently lexically-conditioned allomorph of a habitual passive suffix: \textit{{}-rîwa} for Class 1 stems and \textit{-wá} for the rest.

The evidence for conjugation classes in Choguita Rarámuri is therefore weak. Stress shifts and vocalic alternations of Choguita Rarámuri stems are analyzed here as arising from regular morpho-phonological processes and the existence of two phonological kinds of stems and suffixes: roots with underlying, lexically pre-specified stress, and roots without it; and stress-perturbing suffixes and stress-neutral suffixes. These two kinds of suffixes would be associated with two kinds of morphological constructions. Stress shifts result from regular morpho-phonological principles that apply to the interactions of the different roots and affixes, rather than from an arbitrary classification of the lexicon into conjugation classes.

While this morpho-phonological analysis of Choguita Rarámuri roots has implications for the analysis of the morphologically-conditioned stress, the conjugational class analysis might be used as a simplified notational device for  lexicographic and pedagogical works. Specifically, it is crucial to distinguish three classes of verbal roots in this language in order to predict the roots’ behavior in morphological constructions.

\subsection{The interaction of shifting and neutral morphological constructions: stress and vocalic alternations}
\label{subsec: interaction of shifting and neutral morphological constructions}

The agglutinating nature of the Choguita Rarámuri verbal template allows for the interaction of different kinds of suffixes in terms of their stress properties (as stress-neutral or stress-shifting) in the same word. Given the phonological effects of each type of suffix with unstressed roots, it is necessary to look at unstressed roots that undergo multiple affixation to determine what suffix imposes its phonological properties on the whole word, i.e., which phonological effects percolate up to the word level.

As we have seen in the previous section, Class 3 roots have a stem with second syllable stress and final \textit{a}) when attaching a stress neutral suffix and a stem with third syllable stress and final unstressed \textit{i} when attaching a stress shifting suffix. In cases of multiple affixation, the stress properties of the word are invariably defined by the first suffix added: in each form exemplified in (\ref{ex: class 3 roots with multiple suffixes}--\ref{ex: class 3 roots with multiple suffixes 4}), words have third syllable stress when the first suffix is stress shifting (\ref{ex: class 3 roots with multiple suffixes}, \ref{ex: class 3 roots with multiple suffixes 2}), and second syllable stress when the first suffix is stress neutral (\ref{ex: class 3 roots with multiple suffixes 3}--\ref{ex: class 3 roots with multiple suffixes 4}), regardless of the stress type of outer suffixes. Multiply suffixed forms where the root is immediately followed by a stress neutral suffix, are, however, split with respect to their vocalic qualities: these roots can either have a final stressed \textit{i} vowel (e.g., (\ref{ex: class 3 roots with multiple suffixes 3}a--b) and (\ref{ex: class 3 roots with multiple suffixes 4}a--b)) or a final stressed \textit{a} vowel (e.g., (\ref{ex: class 3 roots with multiple suffixes 3}c--d) and (\ref{ex: class 3 roots with multiple suffixes 4}c--d)).

\ea\label{ex: class 3 roots with multiple suffixes}
{Class 3 verb root + stress shifting suffix + stress shifting suffix}

    \ea[]{
    \textit{osiˈnálsa}\\
    \textit{osi-ˈnále-sa}\\
    {write.read-\textsc{desid-cond}}\\
    `If s/he wants to read/write.'\\
    `Si quiere leer/escribir.' {< BFL 08 1:18/el>}\\
}
        \ex[]{
        \textit{riwiˈbôsi}\\
        \textit{riwi-ˈbô-si}\\
        {find.see-\textsc{mot.imp.pl-imp.pl}}\\
        `You all go see/find it!'\\
        `¡Vayan a ver/encontrar!' {< BFL 08 1:16/el >}\\
    }
    \z
\z

\ea\label{ex: class 3 roots with multiple suffixes 2}
{Class 3 verb root + stress shifting suffix + stress neutral suffix}

    \ea[]{
            \textit{osiˈnáliki}\\
            \textit{osi-ˈnále-ki}\\
            {write.read-\textsc{desid-pst.ego}}\\
            `I wanted to read/write.'\\
            `Quería leer/escribir.' {< BFL 08 1:18/el >}\\
        }
                \ex[]{
                \textit{riwiˈwái}\\
                \textit{riwi-ˈwá-i}\\
                {find.see-\textsc{mpass-impf}}\\
                `It used to be seen/found.'\\
                `Era visto/encontrado.' {< BFL 08 1:16/el >}\\
            }
    \z
\z

\pagebreak

\ea\label{ex: class 3 roots with multiple suffixes 3}
{Class 3 verb root + stress neutral suffix + stress shifting suffix}

            \ea[]{
                    \textit{oˈsìsima}\\
                    \textit{oˈs\textbf{ì}-si-ma}\\
                  {write.read-\textsc{mot-fut.sg}}\\
                  `S/he will go along reading/writing.'\\
                    `Va a ir leyendo/escribiendo.' {< SFH 05 1:88/ el >}\\
                }
                        \ex[]{
                        \textit{riˈwíwkima}\\
                        \textit{riˈw\textbf{í}{}-wi-ki-ma}\\
                        {find.see-\textsc{appl-appl-fut.sg}}\\
                        `S/he will find/see it for them.'\\
                        `Va a encontrárselo/vérselo.' {< BFL 08 1:16/el >}\\
                    }
                            \ex[]{
                            \textit{oˈsàrma}\\
                            \textit{oˈs\textbf{à}{}-ri-ma}\\
                            {write.read-\textsc{caus-fut.sg}}\\
                            `S/he will make them read/write.'\\
                            `Va a hacer que lea/esscriba.' {< BFL 08 1:10/el >}\\
                        }
                                \ex[]{
                                \textit{raʔˈlártima}\\
                                \textit{raʔˈl\textbf{á}{}-ri-ti-ma}\\
                                {buy-\textsc{caus-caus-fut.sg}}\\
                                `S/he will make them buy it.'\\
                               `Va a hacer que lo compre.' {< BFL 08 1:10/el >}\\
                            }
    \z
\z

\ea\label{ex: class 3 roots with multiple suffixes 4}
{Class 3 verb root + stress neutral suffix + stress neutral suffix}

                \ea[]{
                                    \textit{oˈsìrili}\\
                                    \textit{oˈs\textbf{ì}{}-ri-li}\\
                                    {write.read-\textsc{caus-pst}}\\
                            `S/he made them read/write.'\\
                              `Lo hizo leer/escribir.' {< SFH 08 1:41/ el >}\\
                                }
                                        \ex[]{
                                        \textit{riˈwísio}\\
                                        \textit{riˈw\textbf{í}{}-si-o}\\
                                        {find.see-\textsc{mot-ep}}\\
                            `S/he goes along seeing/finding.'
                                        {< BFL 08 1:16/el >}\\
                                    }
                                            \ex[]{
                                            \textit{oˈsàrki}\\
                                            \textit{oˈs\textbf{à}{}-ri-ki}\\
                                            {write.read-\textsc{caus-pst.ego}}\\
                                `I made them read/write.'\\
                                `Los hice leer/escribir.'    {< BFL 08 1:10/el >}\\
                                        }
                                                \ex[]{
                                                \textit{raʔˈlártiki}\\
                                                \textit{raˈl\textbf{á}{}-ri-ti-ki}\\
                                                {buy-\textsc{caus-caus-pst.ego}}\\
                            `I made them buy it.'\\
                            `Los hice comprarlo.'     {< BFL 08 1:10/el >}\\
                                            }
    \z
\z

Irrespective of the stress type of the suffixes added, then, words built from Class 3 roots will have a stress make-up dependent on the first suffix added, but either a final \textit{i} stem vowel or a final \textit{a} stem vowel.\footnote{The fact that there is variation of stem shape in constructions with multiple suffixes could suggest a ``look-ahead'' effect of morphology, where variation in vocalic quality is dependent on having multiple suffixation (i.e., the outer suffixes `see' inside the preceding morphological structure, a violation of bracket erasure). An alternative is to assume that variation in stem selection is exclusively found when Class 3 roots add suffixes of inner layers (e.g., the Syntactic Stem and Aspectual Stem verb domains defined below in §\ref{sec: verbal structure and verbal domains}). Given the hierarchical structure of the verb proposed in this grammar, suffixes belong to different verbal domains. The phonological features of certain domains will thus percolate up to the word level and impose its phonological properties to the word form.}

\subsection{Lexical tone in lexically stressed and unstressed verbs}
\label{subsec: tone in verbal classes}

In addition to the contrast in terms of lexically specified stress (stressed and unstressed), verbal roots bear lexical tonal contrasts with a distribution that is partly dependent on the lexical stress properties of roots. First, all tonal contrasts (HL, L and H) are realized with lexically stressed roots (those with fixed stress across paradigms). This is exemplified in (\ref{ex: tonal properties of lexically stressed roots}), where stress is fixed with both neutral (e.g., past \textit{-li}) and shifting (e.g., future sg. \textit{-ma}) suffixes:\footnote{As described in §\ref{subsec: alternating tone stems} below, a subset of lexically stressed roots have ``alternating" tone patterns, or morphologically determined tone in inflection: L tone in neutral morphological contexts and HL tone in shifting morphological contexts. This class differs from other verbal roots since it displays tonal changes in the absence of stress shifts.}

\ea\label{ex: tonal properties of lexically stressed roots}
{Tonal properties of lexically stressed roots}

\begin{tabular}{lllll}
& & \textit{Neutral} & \textit{Shifting}\\
      & & \textit{\textsc{pst}} & \textit{\textsc{fut.sg}}\\
     a. { niˈhî}&{`give away'} & {niˈhî-li } & {niˈhî-ma } & {< BFL el1904 >}\\
     b. { wiˈt͡ʃô}&{`wash (clothes)'} & {wiˈt͡ʃô-li} & {wiˈt͡ʃô-ma } & {< SFH el101 >}\\
     c. { oˈhò}& {`thresh'} & {oˈhò-li } & {oˈhò-ma} & {< BFL el1906 >}\\
     d. { t͡ʃiˈhà}& {`spread, intr.'} & {t͡ʃiˈhà-li}&{t͡ʃiˈhà-ma } & {< BFL 2014:64 >}\\
     e. { poˈt͡ʃí}& {`jump'} & {poˈt͡ʃí-li } & {poˈt͡ʃí-ma} & {< SFH el1900 >}\\
     f. { pa'kó}& {`wash (dishes)'} & {paˈkó-li} & {paˈkó-ma} & {< SFH el1901 >}\\
\end{tabular}

    \z

On the other hand, lexically unstressed roots are predominantly H-toned (e.g., (\ref{ex: tonal properties of lexically unstressed roots}a--d), though there are also unstressed L-toned roots documented (e.g. (\ref{ex: tonal properties of lexically unstressed roots}e--f). Crucially, there are no documented HL-toned unstressed roots. The examples in (\ref{ex: tonal properties of lexically unstressed roots}) show how these roots, like all unstressed roots, shift stress in shifting contexts (e.g., when inflected for future singular \textit{-mêa} {\textasciitilde} \textit{-ma}), but keep stress in the stem in neutral contexts (e.g., when inflected for past \textit{-li}). The lexical tone of the root emerges in bare stems and when inflected for neutral constructions.

%These examples should have the tone marked in a row format
\ea\label{ex: tonal properties of lexically unstressed roots}
{Tonal properties of lexically unstressed roots}

\begin{tabular}{lllll}
        & & \textit{Neutral} & \textit{Shifting}\\
       & &\textit{\textsc{pst}} & \textit{\textsc{fut.sg}}\\
     a. {aˈwí} & {`dance'} & aˈwí-li  & {awi-ˈmêa } & {< BFL el1883>}  \\
     b. {aˈsá} &{`sit down'} & {aˈsá-li } & {asi-ˈmêa } & {< RIC el1892>}\\
     c. {maˈt͡ʃí}& {`know'} & {maˈt͡ʃí-li } & {mat͡ʃi-ˈmêa} & {< BFL el1909>}\\
     d. {roˈnò} &{`boil'} & {roˈnò-li } & {rono-ˈmêa } & {< BFL el1903>}\\
     e. {moˈlò}& {`to be smoky'} & {moˈlò-li } & {molo-ˈmêa } & {< BFL el1914>}\\
     f. {t͡ʃuʔˈmì}& {`to hiccup'} & {t͡ʃuʔˈmì-li } & {t͡ʃuʔmì-ˈmêa} & {< BFL 2014:147>}\\
\end{tabular}

    \z

Given the surface tonal distributions in morphologically complex verbs in Choguita Rarámuri, the analysis assumed here is that all lexical tones are specified, in both lexically stressed and lexically unstressed verb roots. The stress and tonal properties of morphologically complex verbs are discussed further in §\ref{sec: the role of tone in verbal morphology} below and in \chapref{chap: prosody}.

%Analysis:

%\begin{itemize}
%\item Roots may be stressed or unstressed\\
%\item There are three lexical tones in stressed syllables\\
%\item Unstressed syllables are toneless\\
%\item There are tonal properties of morphologically complex words that cannot be derived from underlying stress and tone lexical properties of roots and suffixes, but result instead from membership to inflection classes - these properties are orthogonal to lexical tone information
%\item One class of verbs receives tonal assignment from the type of morphological construction attaching to a root: L tone in neutral constructions and HL tone with shifting constructions; these verbs can be either stressed or unstressed
%\item As noted above, unstressed verbs are overwhelmingly H-toned. These verbs are the ones that show the morphologically conditioned tonal effects of imperfective and present progressive
%\end{itemize}


%%The conjugational class analysis requires greater contextualization in terms of abstractive morphological frameworks. How do the tonal properties of roots interact with stress properties in yielding defined verbal classes?

%%relevant notes: notebook 2014 and notes on Caballero \& \citealt{Carroll2015}

\subsection{Valence alternations}
\label{subsec: valence alternations}

In the previous section, we have seen that there are three identifiable verbal root classes in Choguita Rarámuri, which have characteristic stress and vowel alternation properties in morphologically defined contexts. Unstressed roots (Class 2 and Class 3 roots) can, in addition, undergo valence related alternations through the affixation of transitive and applicative vocalic suffixes that replace the final vowel of the stem.

In this valence alternation system, there are three kinds of stems: intransitive, transitive and applicative stems. Intransitive stems end in an unstressed vowel (\ref{ex: valence stem allomorphy}). Transitive stems replace the final stem vowel with a stressed \textit{{}-a} suffix (\ref{ex: valence stem allomorphy 2}).\footnote{\citet{heath1978uto} provides evidence that an intransitive-transitive contrast marked by \textit{i} for intransitive and \textit{a} for transitive goes back to \ili{Proto-Uto-Aztecan}.} Applicative stems replace the final stem vowel with a stressed, low toned front vowel suffix (\textit{{}-è} or \textit{-ì}) (\ref{ex: valence stem allomorphy 3}).\footnote{There is only one example that shows a semantic difference between an applicative stem formed with a final stressed high front vowel and an applicative stem with a final stressed mid front vowel: the verb \textit{raʔˈlá/raʔli-} ‘to buy’, has two applicative bases: \textit{raʔˈl-è} ‘buy from)’ (\ref{ex: buy for vs. by froma}) and \textit{raʔˈlì} ‘buy for’ (\ref{ex: buy for vs. by from}):

\ea\label{ex: buy for vs. by from}
    \ea[]{
   \textit{muˈhê   taˈmí   saˈpâto  raʔˈlèma}\\
    \gll    muˈhê   taˈmí   saˈpâto  raʔˈl-è-ma\\
             2\textsc{sg.nom} \textsc{1sg.acc} shoes buy.from-\textsc{fut.sg}\\
    \glt    ‘You’ll buy shoes from me (that I sell).’ \\
    \glt    ‘Me vas a comprar zapatos (a mi, que yo vendo).’   < SFH 05 1:74/ el > \\
}\label{ex: buy for vs. by froma}
        \ex[]{
        \textit{muˈhê   taˈmí   saˈpâto   raʔˈlìma}\\
        \gll    muˈhê   taˈmí   saˈpâto   raʔˈl-ì-ma\\
                2\textsc{sg.nom}    1\textsc{sg.acc}   shoes    buy-\textsc{appl-fut.sg}\\
        \glt    ‘You will buy me shoes (for me to use).'\\
        \glt    ‘Me vas a comprar zapatos (para mi, para que yo use).’  < SFH 05 1:74/el >\\
    }\label{ex: buy for vs. by fromb}
    \z
\z

There are no other examples where two applicative stems have been lexicalized with different meanings.}

\ea\label{ex: valence stem allomorphy}
{Valence stem allomorphy: Intransitive}

    \ea[]{
    \textit{niˈhê  ˈmá   nok\textbf{o}ˈmêa}\\
    \gll    niˈhê   ˈmá nok\textbf{o}{}-ˈmêa \\
            1\textsc{sg.nom}   already   move.\textsc{intr-fut.sg}\\
    \glt    ‘I will move.’ \\
    \glt    ‘Ya me voy a mover.’   < SFH 05 1:80/el >\\
}
        \ex[]{
        \textit{uk\textbf{u}ˈmêa}\\
        \gll    uk\textbf{u}{}-ˈmêa\\
                rain.\textsc{intr-fut.sg}\\
        \glt    ‘It will rain.’  \\
        \glt    ‘Va a llover.’     < SFH 05 1:82/el >\\
    }
    \z
\z

\ea\label{ex: valence stem allomorphy 2}
{Valence stem allomorphy: Transitive}

            \textit{ˈmá raʔˈl\textbf{á}{}ki}\\
            \gll    ˈmá raʔˈl-\textbf{á}{}-ki\\
                    already buy-\textsc{tr-pst.ego} \\
            \glt   ‘(He) already bought it.’  \\
            \glt    ‘Ya lo compró.’    < AHF 05 1:130/el > \\
\z

\ea\label{ex: valence stem allomorphy 3}
{Valence stem allomorphy: Applicative}

        \ea[]{
        \textit{niˈhê ˈmí ˈtrôka noˈk\textbf{è}li}\\
        \gll    niˈhê ˈmí ˈtrôka noˈk-\textbf{è}-li \\
                1\textsc{sg.nom} 2\textsc{sg.acc}  truck move-\textsc{appl-pst}  \\
        \glt    ‘I will move the truck for you.’ \\
        \glt    ‘Te voy a mover la troca.’ < SFH 05 1:80/el >  \\
            }\label{ex: valence stem allomorphy 3a}

                    \ex[]{
                    \textit{ˈá=mi taˈmí raʔˈl\textbf{è}ma }\\
                    \gll    ˈá=mi taˈmí raʔˈl-\textbf{è}{}-ma \\
                            \textsc{aff=2sg.nom}   1\textsc{sg.acc}    buy-\textsc{appl-fut.sg}  \\
                    \glt    ‘Will you buy (it) from me?’  \\
                    \glt    ‘¿Me lo compras (yo lo vendo)?’     < SFH 05 1:75/el >  \\
                }\label{ex: valence stem allomorphy 3b}

                        \ex[]{
                        \textit{mi uˈk\textbf{è}li}\\
                        \gll    mi uˈk-\textbf{è}{}-li \\
                                \textsc{dem} rain-\textsc{appl-pst}  \\
                        \glt    ‘It rained for (him).’\\
                        \glt    ‘Le llovió.’       < SFH 05 1:84/el > \\
                    }\label{ex: valence stem allomorphy 3c}
    \z
\z

\tabref{tab:valence-stem-allomorphy} schematizes the three-way contrast between intransitive, transitive and applicative stems (syntactic/semantic gaps are symbolized by dashes).

\begin{table}
\caption{Valence stem allomorphy}
\label{tab:valence-stem-allomorphy}

\begin{tabularx}{\textwidth}{Xllll}
\lsptoprule
& \textbf{Intransitive}  &   \textbf{Transitive} &  \textbf{Applicative } &     \textbf{Gloss}\\
\midrule
a.& suˈwí    & suˈwá     &     suˈw-è &                  ‘run out/finish up’\\
b.& saˈwí          &   - &                 saˈw-è &                  ‘cure, heal’\\
c.&   - &               raʔˈlá      &       raʔˈl-è &                   ‘buy'\\
d.&   noko      &      - &                noˈk-è      &            ‘move’\\
e.&    - &              iˈtʃ͡á      &        iˈtʃ͡-ì  &                   ‘plant'\\
f.&   uku  &             - &                uˈk-è     &             ‘rain'\\
g.&  wili-  &             wiˈlá  &         wiˈl-è      &            ‘stand’\\
h.&  tʃ͡oʔi  &             tʃ͡oʔˈá &           tʃ͡oˈʔ-ì &                   ‘extinguish’\\
i.&   -     &               oˈsà &           oˈs-è     &               ‘read, write’\\
j.&   -      &              kiˈmá  &         kiˈm-è     &            ‘cover with blanket’\\
\lspbottomrule
\end{tabularx}
\end{table}

Verbs that exhibit this valence alternation may exhibit multiple exponence: the examples in (\ref{ex: valence stem allomorphy and multiple exponence}--\ref{ex: valence stem allomorphy and multiple exponence 3}) show how the applicative is redundantly marked, as there are two applicative markers in (\ref{ex: valence stem allomorphy and multiple exponence}b), (\ref{ex: valence stem allomorphy and multiple exponence 2}b) and (\ref{ex: valence stem allomorphy and multiple exponence 3}b), but only one benefactive or malefactive argument introduced. These forms are semantically equivalent to their counterparts with only one applicative exponent (in (\ref{ex: valence stem allomorphy and multiple exponence}a), (\ref{ex: valence stem allomorphy and multiple exponence 2}a), and (\ref{ex: valence stem allomorphy and multiple exponence 3}a), respectively) and speakers use the two types of forms interchangeably.
%\todo[inline]{There are no (20a) and (20b).} - FIXED
%consider not referring to the applicative forms as stem allomorphs - or discuss why the terminology might be retained in the context of stem-forming suffixes

\ea\label{ex: valence stem allomorphy and multiple exponence}

    \ea[]{
    \textit{ˈmáni ˈmí suˈwèli reˈmê}\\
    \gll    ˈmá=ni ˈmí suˈw-è-li reˈmê\\
            already=1\textsc{sg.nom} \textsc{2sg.acc}  finish-\textsc{appl-pst}   tortillas\\
    \glt    ‘I already finished (ate) up your tortillas.’ \\
    \glt    ‘Ya me acabé tus tortillas.’    < SFH 05 1:119/el >\\
}\label{ex: valence stem allomorphy and multiple exponencea}
        \ex[]{
        \textit{ˈmáni ˈmí suˈwèkili      reˈmê}\\
        \gll    ˈmá=ni ˈmí suˈw-è-ki-li      reˈmê\\
                already=1\textsc{sg.nom} \textsc{2sg.acc}  finish-\textsc{appl-appl-pst}   tortillas\\
        \glt    ‘I already finished (ate) up your tortillas.’  \\
        \glt    ‘Ya me acabé tus tortillas.’    <LEL 06 5:123/el >\\
    }\label{ex: valence stem allomorphy and multiple exponenceb}
    \z
\z

\ea\label{ex: valence stem allomorphy and multiple exponence 2}

    \ea[]{
            \textit{oˈʃìma}\\
            \gll    oˈs-ì-ma \\
                    write-\textsc{appl-fut.sg}\\
            \glt    ‘She will write him (a letter).’\\
            \glt    ‘Le va a escribir (una carta).’    < BFL 06 2:98/el >
        }
                \ex[]{
                \textit{oˈʃìkima}\\
                \gll    oˈs-ì-ki-ma\\
                        write-\textsc{appl-appl-fut.sg}\\
                \glt    ‘She will write him (a letter).’\\
                \glt    ‘Le va a escribir (una carta).’     < BFL 06 2:98/el >\\
            }
    \z
\z

\ea\label{ex: valence stem allomorphy and multiple exponence 3}

    \ea[]{
        \textit{roˈnèma}\\
        \gll    roˈn-è-ma \\
                    boil-\textsc{appl-fut.sg}\\
        \glt    ‘He will boil it for her.’\\
        \glt    ‘Se la va a hervir.’      < BFL 06 2:101/el >  \\
    }
            \ex[]{
            \textit{roˈnèkima}\\
            \gll    roˈn-è-ki-ma\\
                    boil-\textsc{appl-appl-fut.sg}\\
            \glt    ‘He will boil it for her.’\\
            \glt    ‘Se la va a hervir.’    < BFL 06 2:101/el >  \\
        }
    \z
\z


Finally, transitive and applicative stems block any stress shift or vocalic alternation conditioned by shifting and neutral morphological constructions. That is, transitive and applicative stems always have stress on their final stem vowel, and block any stress shift or vocalic alternation imposed by shifting morphological constructions.

\subsection{Change-of-state predicates}
\label{subsec: change of state predicates}

In addition to the valence stem allomorphy of unstressed roots, Choguita Rarámuri has a second valency-increasing stem allomorphy system, marked through thematic suffixes and stress alternations in a semantically defined class of verbs, change-of-state predicates. In Choguita Rarámuri, the class of change-of-state predicates is composed of unstressed roots that are marked as intransitive, transitive or applicative through the presence or abscence of transitive suffixes and specific stress patterns. All change-of-state verb roots are disyllabic.

The \ili{Guarijío} cognates of this verb class are defined as predicates that denote an irreversible change-of-state or condition, generally produced by some kind of contact, with prototypical intransitive-causative pairs ‘break’, ‘twist’, ‘torn apart’, ‘shatter’, ‘spill’ \citep[][153]{miller1996guarijio}. This class of verbs shares morphological traits accross the \ili{Uto-Aztecan} language family: \citet{heath1978uto} reconstructs a morphological class of verbs for \ili{Proto-Uto-Aztecan} (PUA) and the intermediate protolanguages, \ili{Proto Northern Uto-Aztecan} (PNUA) and \ili{Proto Southern Uto-Aztecan} (PSUA), and gives descriptions of this class of verbs in \ili{Southern Paiute}, \ili{Mono}, \ili{Luiseño}, \ili{Cupeño}, \ili{Cahuilla}, \ili{Serrano}, \ili{Hopi}, \ili{Tepiman}, Tarahumara and \ili{Aztecan} language varieties. This class is composed of disyllabic roots with thematic variation, and it includes verbs denoting events of physical change-of-state, even though the actual verbs are not cognate from one language to another.

Intransitive stems involve no theme suffixes and have fixed stress: stress is on the second syllable, whether a stress-shifting suffix (future singular in (\ref{ex: intransitive change of state predicatesa}) and conditional in (\ref{ex: intransitive change of state predicatesc})) or a stress-neutral suffix (past in (\ref{ex: intransitive change of state predicatesb}) and (\ref{ex: intransitive change of state predicatesd})) is attached.

\ea\label{ex: intransitive change of state predicates}
{Intransitive change-of-state predicates}

    \ea[]{
    \textit{tʃ͡iˈháma ˈlé  ˈtʃ͡îba}\\
    \gll    tʃ͡iˈhá-ma aˈlé  ˈtʃ͡îba\\
            scatter-\textsc{fut.sg}  \textsc{dub}  goats\\
    \glt    ‘The goats will scatter.’\\
    \glt    ‘Se van a desparramar las chivas.’    < SFH 07 1:17/el >\\
}\label{ex: intransitive change of state predicatesa}
\pagebreak
        \ex[]{
        \textit{ˈmá     tʃ͡iˈháli    ˈtʃ͡îba}\\
        \gll    ˈmá    tʃ͡iˈhá-li    ˈtʃ͡îba\\
                already    scatter-\textsc{pst}  goats\\
        \glt    ‘The goats already scattered.’\\
        \glt    ‘Ya se desparramaron las chivas.’    < SFH 07 1:17/el >\\
    }\label{ex: intransitive change of state predicatesb}
            \ex[]{
            \textit{riˈpúsa    kuˈsì}\\
            \gll    riˈpú-sa    kuˈsì\\
                    cut-\textsc{cond}  wood\\
            \glt    ‘If the wood is cut.’\\
            \glt    ‘Si se corta la leña.’    < BFL 08 1:25/el >\\
        }\label{ex: intransitive change of state predicatesc}
                \ex[]{
                \textit{ˈmá   riˈpúli    kuˈsì}\\
                \gll    ˈmá    riˈpú-li    kuˈsì\\
                        already    cut-\textsc{pst}    wood\\
                \glt    ‘The wood was already cut.’ \\
                \glt    ‘Ya está cortada la leña.’       < BFL 08 1:25/el >\\
            }\label{ex: intransitive change of state predicatesd}
    \z
\z

Transitive stems, on the other hand, are marked by suffixes (the transitive \textit{-ˈnâ} suffix exemplified in (\ref{ex: transitive change of state predicates}) and the transitive pluractional \textit{-tʃâ} suffix exemplified in (\ref{ex: vocalic alternations in forms with transitive suffixesc})), and are also sensitive to shifting and neutral constructions: stress is shifted to the transitive suffix, the third syllable, when attaching stress-shifting morphemes ((\ref{ex: transitive change of state predicatesa}) and (\ref{ex: transitive change of state predicatesc})), but have final root (second syllable) stress when attaching stress-neutral suffixes ((\ref{ex: transitive change of state predicatesb}) and (\ref{ex: transitive change of state predicatesd})).

\ea\label{ex: transitive change of state predicates}
{Transitive change-of-state predicates}

    \ea[]{
    \textit{tʃ͡ihaˈnâsa    naˈpàtʃ͡i}\\
    \gll    tʃ͡iha-ˈnâ-sa    naˈpàtʃ͡i\\
            scatter-\textsc{tr-cond}  blouse\\
    \glt    ‘if she scatters the blouses...’\\
    \glt    ‘si desparrama las blusas’ < SFH 07 1:17/el >\\
}\label{ex: transitive change of state predicatesa}
        \ex[]{
        \textit{ˈpîrim  oˈlá  tʃ͡iˈhánili    naˈmûti}\\
        \gll    ˈpîri=mi  oˈlá  tʃ͡iˈhá-na-li    naˈmûti\\
                what=2\textsc{sg.nom} why scatter.\textsc{tr-tr-pst}  things\\
        \glt    ‘Why did you scatter the things?’\\
        \glt    ‘¿Por qué desparramas las cosas?’ < SFH 07 1:17/el >\\
    }\label{ex: transitive change of state predicatesb}
%\pagebreak
            \ex[]{
            \textit{ma   ripuˈnâma  ˈlé  kuˈsì}\\
            \gll    ma   ripu-ˈnâ-ma  aˈlé  kuˈsì\\
                    now  cut-\textsc{tr-fut.sg}  \textsc{dub}  wood\\
            \glt    ‘S/he’ll cut the wood now.’\\
            \glt    ‘Ya va a cortar la leña.’  <RF 04 verbs/el >\\
        }\label{ex: transitive change of state predicatesc}
                \ex[]{
                \textit{seˈrûtʃ͡o    riˈpúnili  kuˈsì}\\
                \gll    seˈrûtʃ͡o    riˈpú-na-li  kuˈsì\\
                        saw    cut-\textsc{tr-pst}  wood\\
                \glt    ‘The saw cut the wood.’\\
                \glt    ‘El serrucho cortó la leña.’  < BFL 08 1:25/el >\\
            }\label{ex: transitive change of state predicatesd}
    \z
\z


The transitive \textit{-ˈnâ} suffix, thus, has a stressed and an unstressed allomorph. The same is attested for the transitive pluractional \textit{-tʃ͡a} suffix.\footnote{When bearing stress, the transitive suffix often causes the final root vowel to align in color with the first root vowel, as in (\ref{ex: vocalic alternations in forms with transitive suffixesa}).}

\ea\label{ex: vocalic alternations in forms with transitive suffixes}

    \ea[]{
    \textit{niˈhê   kuʔr\textbf{u}ˈnâma}\\
    \gll    niˈhê   kuʔr\textbf{u}{}-ˈnâ-ma  \\
            1\textsc{sg.nom}   turn-\textsc{tr-fut.sg}\\
    \glt    ‘I will turn it (on its own axis).’ \\
    \glt    ‘Le voy a dar vuelta (en su propio eje).’  < BFL 05 1:187/el >\\
}\label{ex: vocalic alternations in forms with transitive suffixesa}
        \ex[]{
        \textit{niˈhê  kuʔr\textbf{î}ma}\\
        \gll    niˈhê  kuʔr\textbf{î}{}-ma\\
                1\textsc{sg.nom}  turn-\textsc{fut.sg} \\
        \glt    ‘I will turn (on my own axis).’ \\
        \glt    ‘Voy a dar vuelta (en mi propio eje).’  < SFH 05 1:140/el >\\
    }\label{ex: vocalic alternations in forms with transitive suffixesb}
            \ex[]{
            \textit{ˈmáni    kuʔˈr\textbf{î}tʃ͡ima}\\
            \gll    ˈmá=ni    kuʔˈr\textbf{î}-tʃ͡a-ma\\
                    already=1\textsc{sg.nom}  turn-\textsc{tr.pl-fut.sg}\\
            \glt    ‘I will now turn it several times (on its own axis)’. \\
            \glt    ‘Ya le voy a dar muchas vueltas (en su propio eje)’. < BFL 05 1:187/el > \\
        }\label{ex: vocalic alternations in forms with transitive suffixesc}
    \z
\z

Finally, applicative stems are marked through the addition of a transitive suffix and fixed final root (second syllable) stress. As shown in the next examples, the applicative stems of change-of-state predicates have fixed second syllable stress, whether they  are in a shifting construction (future singular (\ref{ex: applicative  change of state predicatesb})) or a neutral construction (past (\ref{ex: applicative  change of state predicatesd})). Applicative stems are shown with a corresponding non-transitive version of these predicates.

\newpage
\ea\label{ex: applicative  change of state predicates}
{Applicative change-of-state predicates}
%check the tones of this forms
    \ea[]{
    \textit{ˈpé uˈtʃ͡íin tʃ͡iˈwáli boʔˈnà muˈnî ˈnè}\\
    \gll    ˈpé uˈtʃ͡í=ni tʃ͡iˈwá-li boʔˈnà muˈnî ˈnè\\
            just small.space=\textsc{1sg.nom} tear-\textsc{pst} harvest.beans beans \textsc{1sg.nom}\\
    \glt    `Just a little bit (in a small piece of land) I tore beans for havest, I.'\\
    \glt    `Nomás muy poquito (en un pedazo de tierra chiquito) alcancé a arrancar frijol, yo.' \corpuslink{co1137[09_361-09_386].wav}{MDH co1137:9:36.1}\\
}\label{ex: applicative  change of state predicatesa}
        \ex[]{
        \textit{niˈhê ˈmí  tʃ͡iˈwánima}\\
        \gll    niˈhê ˈmí  tʃ͡iˈwá-na-ma\\
                1\textsc{sg.nom}  2\textsc{sg.acc}  tear.\textsc{appl-tr-fut.sg}\\
        \glt    ‘I’m going to tear it for you.’\\
        \glt    ‘Yo te lo voy a trozar.’  < SFH 07 1:21/el >
    }\label{ex: applicative  change of state predicatesb}
            \ex[]{
            \textit{ˈtʃ͡îba tʃ͡iˈháli}\\
            \gll    ˈtʃ͡îba tʃ͡iˈhá-li\\
                    goats scatter-\textsc{pst}\\
            \glt    `The goats got scattered.'\\
            \glt    `Las chivas se desparramaron.'\\
        }\label{ex: applicative  change of state predicatesc}
                \ex[]{
                \textit{niˈhê   ˈmámi    tʃ͡iˈhánili  naˈpàtʃ͡i}\\
                \gll    niˈhê   ˈmá=mi    tʃ͡iˈhá-na-li  naˈpàtʃ͡i\\
                        1\textsc{sg.nom}  already=2\textsc{sg.acc}  scatter-\textsc{tr-pst}  blouses\\
                \glt    ‘I scattered your blouses.’ \\
                \glt    ‘Ya te desparramé las blusas.’  < SFH 07 1:21/el >
        }\label{ex: applicative  change of state predicatesd}
    \z
\z

The differences between the different kinds of stems are schematized in \tabref{tab:key:13}

%%please move \begin{table} just above \begin{tabular .
\begin{table}
\caption{Change-of-state predicates and thematic alternations}
\label{tab:key:13}

\begin{tabularx}{\textwidth}{lllQ}
\lsptoprule
& \textbf{Neutral} & \textbf{Shifting} & \textbf{Marker}\\
\midrule
 \textbf{Intransitive} & kaˈsì-li & kaˈsì-ma & { fixed 2\textsuperscript{nd} syllable stress, no transitive suffix}\\
 \textbf{Transitive} & kaˈsì-\textbf{ni}{}-li & kasi-\textbf{ˈnâ}{}-ma & { transitive -\textit{nâ}}\\
 \textbf{Applicative} & kaˈs\textbf{ì}{}-\textbf{ni}{}-li & kˈas\textbf{ì-ni}{}-ma & { fixed 2\textsuperscript{nd} syllable stress}

 (plus transitive \textit{-nâ})\\
\lspbottomrule
\end{tabularx}
\end{table}

The effects of neutral and shifting morphological contexts only produce stress shifts with transitive stems of change-of-state predicates.

\subsection{Summary}
\label{subsec:9:summary:verbalroot}

All open class words in Choguita Rarámuri have surface stress, but verbal (and other) roots can be characterized as either lexically stressed or unstressed, a distinction of their underlying phonological representation. This phonological distinction involves significant differences in the prosodic makeup of words, depending on the interaction between roots and different types of suffixes and morphological constructions. Unstressed roots can be further subdivided into two classes if the roots undergo vowel alternations concomitant to the stress shifts. I have presented a morpho-phonological analysis of these three types of roots, and have given arguments in favor of this analysis over a conjugational class analysis.


Unstressed verbs can participate in a valence stem allomorphy system and undergo internal, non-concatenative changes when specified for intransitive, transitive or applicative meanings. Another subset of the unstressed verbs, semantically defined as a class of verbs of change-of-state, participates in a second valence stem allomorphy system. In these systems, final stem vowel changes and fixed stress patterns are the markers of morphological categories (valence-increasing categories), while any concomitant stress shifts are epiphenomenal of certain morphological constructions, a byproduct of the phonological make up of roots and affixes and their compositional interaction.


In addition to stress shifts and vocalic alternations, morphologically complex verbs are characterized by tonal patterns governed by both phonological and morphological conditioning factors. All lexical tones are analyzed here to be underlyingly specified. Lexically stressed verbs may exhibit any of the three lexical tones of the language (HL, L or H), while lexically unstressed verbs are only specified for H and L tones (i.e., there are no lexically unstressed verbs with lexical HL tones). I turn to the role of tone in morphologically complex verbs in the next section.
%%How to introduce a space here?

\section{The role of tone in verbal morphology}
\label{sec: the role of tone in verbal morphology}

%% This section needs to be redone in light of the latest manuscript

As described in \chapref{chap: tone and intonation}, Choguita Rarámuri features a three-way lexical tonal contrast. In addition to lexical tone patterns and processes, tone in Choguita Rarámuri has purely morphological functions and properties that require description and analysis independently of its phonological characteristics.

This section is devoted to describing the role of \textsc{grammatical tone} in the verbal morphology of Choguita Rarámuri. Grammatical tone may be defined as a tonal pattern or process that is not general across the phonological grammar of a language, but is instead associated to a specific morpheme or construction, or a natural class of morphemes or constructions (see \citealt{rolle2018grammatical}).\footnote{Given that verbal morphology in Choguita Rarámuri is more complex than nominal morphology and the morphology of other word classes, this chapter addresses morphological tone within the verbal domain. The grammatical tonal properties of Choguita Rarámuri nouns are addressed in \chapref{chap: nominal morphology} (§\ref{sec: tone in morphologically complex nouns}).}

As discussed in this grammar, stress carries a high functional load in the verbal morphology of the language. While tone appears to carry a smaller functional load than stress, and while lexical tone distribution is itself constrained by stress placement (with morphological distributions of tone being a by-product of the morphological factors that govern stress distribution), tone nonetheless plays an autonomously morphological role in this language. Before addressing each one of these grammatical tone properties, the next section provides an overview of the lexical tonal properties of suffixes in Choguita Rarámuri.

\subsection{Lexical tones of suffixes}
\label{subsec: lexical tone in suffixes}

Evidence for three contrastive lexical tones in Choguita Rarámuri is given in \chapref{chap: tone and intonation} and their distribution in verbal roots was addressed in §\ref{subsec: tone melodies by root type and stress position}. Suffixes also have underlying lexical tones, which are evidenced when they are stressed in morphologically complex words. Only those suffixes associated with shifting constructions (stress-shifting) attaching to unstressed roots will reveal their underlying tonal properties in the surface form (as there is no evidence that suffixes belonging to the neutral class bear lexical tones since they are never stressed). The forms in (\ref{ex: tones in suffixes with root suku}) examplify suffixes and their tonal properties when attaching to the lexically unstressed root \textit{sukú} `to scratch' (\ili{Spanish} \textit{rascarse}).

\ea\label{ex: tones in suffixes with root suku}
{Lexical tone of suffixes in partial verbal paradigm}

\begin{tabular}{llll}
     a. & \textsc{fut.sg} \textit{-ˈmêa} &  {\textit{suku-ˈmêa}} & {‘(S)he will scratch.’} \\
     b. & \textsc{desid} \textit{-ˈnále} & {\textit{suku-ˈnále}} & {‘(S)he wants to scratch.’}\\
     c. & \textsc{cond} \textit{-ˈsâ} & {\textit{suku-ˈsâ}} & {‘if (s)he scratches’}\\
     d. & \textsc{imp.pl} \textit{-ˈsì} & {\textit{suku-ˈsì}} & {`Scratch yourselves!'}\\
        & & & <LEL el2059>\\
\end{tabular}
    \z


As shown in these examples, suffixes may bear any of the three lexical tones of the language. (\ref{ex: tonal properties of suffixes}) provides a comprehensive list of shifting suffixes and their underlying tonal properties.

\newpage
\TabPositions{2cm, 3.5cm, 6cm}
%\begin{tabular}{l{2cm}l{3.5cm}l{6cm}
\ea\label{ex: tonal properties of suffixes}
{Lexical tone properties of shifting suffixes}

\begin{tabular}{llll}
     a. & \textsc{tr} & \textit{-bû} & {HL}  \\
     b. & \textsc{desid} & \textit{-nále} & {H}\\
     c. &\textsc{mpass} & \textit{-rîwa} & {HL}\\
     d. & \textsc{cond.pass} &  \textit{-sûwa} & {HL}\\
     e. & \textsc{fut.sg} & \textit{-mêa} & {HL}{}\\
     f. & \textsc{fut.pl} & \textit{-bô} & {HL}{}\\
     g. & \textsc{mot.imp} & \textit{-mê} & {HL}{}\\
     h. & \textsc{cond} & \textit{-sâ}&{HL}{}\\
     i. & \textsc{irr.sg}& \textit{-mê}&{HL}{}\\
     j. & \textsc{imp.sg} &  \textit{-sâ} & {HL}{}\\
     k. & \textsc{imp.sg} & \textit{-kâ} & {HL}{}\\
     l. & \textsc{imp.pl} & \textit{-sì} &{L}\\
     m. & \textsc{ger} & \textit{-ká} & {H}\\
\end{tabular}
\z

If one of these suffixes is stressed within a complex word, the surface tone pattern of the word may be characterized as involving lexical tone. Lexical tone contrasts with grammatical tone, as the latter involves tonal patterns that are governed directly by morphological factors. I address these next (in §\ref{subsec: tone as realizational morphology} and in §\ref{subsec: grammatical tone}).

\subsection{Tone as realizational morphology}
\label{subsec: tone as realizational morphology}

Choguita Rarámuri deploys tone as a morphological exponent in the marking of imperative mood. The imperative singular has four allomorphs, two concatenative exponents (\textit{-kâ} and \textit{-sâ}) and two non-concatenative ones (a floating L tone and a final stress shift).

The low tonal allomorph replaces stem lexical HL tones, neutralizing the contrast between lexical HL and L tones in the imperative mood. The data in (\ref{ex: four allomorphs of the imperative singular}) exemplify the four allomorphs of the imperative singular construction, contrasting the bare stem (inflected for present tense) and the corresponding imperative singular form.

%\pagebreak

\ea\label{ex: four allomorphs of the imperative singular}
{The imperative singular construction: four allomorphs}
\setlength{\tabcolsep}{4pt}
\begin{tabular}{llllll}
    & \textit{\textsc{prs}} & & \textit{\textsc{imp.sg}} &  &\textit{Gloss}\\
     a.& niˈkâ & HL &  niˈkà &L &	‘Bark!' < BFL el1910 >\\
     b.& tiˈsô &HL& tiˈsò &L &	‘Walk with cane!' < SFH el2042 >\\
     c.& niˈwâ &	HL & niˈwâ-sa  &HL&	‘Make it!' < BFL 2014:61 >\\
     d.& muˈrú & H &	muˈrú-ka &HL&	‘Carry in arms!' < BFL el1883 >\\
     e.& naʔˈsòwa 	& L& naʔsoˈwâ  & HL & ‘Stir it!' < BFL el1957 >\\
\end{tabular}
    \z


As shown in these examples, the L tonal allomorph of the imperative singular replaces the HL tone of the stem (\ref{ex: four allomorphs of the imperative singular}a--b), but tonal replacement does not take place if a suffixal allomorph of the imperative singular is used instead (\ref{ex: four allomorphs of the imperative singular}c--d). Finally, (\ref{ex: four allomorphs of the imperative singular}e) shows that a stress shift within the stem may encode the imperative singular inflectional value.

The examples in (\ref{ex: partial paradigm of mato}) show a HL-toned verb, \textit{ma}ˈ\textit{tô} ‘to carry something in shoulders’, inflected for different tense/aspect and person/number values, including the imperative singular (\ref{ex: partial paradigm of matoe}).

\ea\label{ex: partial paradigm of mato}
{Partial paradigm of the verb \textit{maˈtó} ‘to carry in shoulders’}

    \ea[]{
    \textit{maˈtôa}\\
    \textit{maˈtô-a}\\
    {carry.shoulders-\textsc{prs}}\\
    {‘(S)/he carries it on the shoulders.’}  \\
    `La carga en los hombros.'\\
}\label{ex: partial paradigm of matoa}
        \ex[]{
        \textit{maˈtôki}\\
        \textit{maˈtô-ki}\\
        {carry.shoulders-\textsc{pst.ego}}\\
        {‘I carried it on the shoulders.’}\\
        `La cargué en los hombros.'\\
    }\label{ex: partial paradigm of matob}
            \ex[]{
            \textit{maˈtôma}\\
            \textit{maˈtô-ma}\\
            {carry.shoulders-\textsc{fut.sg}}\\
            {‘I will carry it on the shoulders.’}\\
            `La voy a cargar en los hombros.'\\
        }\label{ex: partial paradigm of matoc}
                \ex[]{
                \textit{maˈtôi}\\
                \textit{maˈtô-i}\\
                {carry.shoulders-\textsc{impf}}\\
                {‘I was carrying it on the shoulders.’} \\
                `La cargaba en los hombros.'\\
            }\label{ex: partial paradigm of matod}
                    \ex[]{
                    \textit{maˈtòo}\\
                    \textit{maˈt\textbf{ò}}\\
                    {carry.shoulders\textsc{imp.sg}}\\
                    {`Carry it on the shoulders!’}\\
                    `¡Cárgala en los hombros!'\\
                }\label{ex: partial paradigm of matoe}
{< BFL 2014 1:152>}
    \z
\z

As shown in the contrast between these inflected forms, a HL-toned verb will keep its lexical tonal melody across derivations without any stress shifts (\ref{ex: partial paradigm of mato}), except when inflected for the imperative singular tonal allomorph: in this case, the morphological L tone replaces the HL lexical tone of the stem (\ref{ex: partial paradigm of matoe}). In this case, the stem vowel undergoes lengthening in order to accommodate the imperative L-tone allomorph of the imperative singular on the first mora and the H\% boundary tone on the second mora of the stem vowel (for more discussion of boundary tones, see §\ref{subsec: H boundary tones}).

%This is represented in autosegmental terms - develop this further in the description of the intonation
		%% add representation here
		%Explain why a H% boundary tone here - is this a property of all imperative forms? H% only confirmed so far in declarative utterances in the indicative mood

The following examples show how the imperative singular L tone is a grammatical tone that replaces stem tones, which includes morphologically complex stems composed of roots and stem-forming suffixes, such as the transitive \textit{-na} suffix:

\ea\label{ex: imperative L tone replaces stem tone}

    \ea[]{
    \textit{supaˈnàá}\\
    \gll   \textit{supa-ˈnà}\\
            unravel-\textsc{tr.imp.sg}\\
    \glt    ‘Unravel it!’\\
    \glt    ‘¡Deshílalo!’  < SFH 2014 1:142>\\
}\label{ex: imperative L tone replaces stem tonea}
        \ex[]{
        \textit{supaˈnâma}\\
        \gll    \textit{supaˈ-nâ-ma} \\
                 unravel-\textsc{tr-fut.sg}\\
        \glt    ‘(S)/he will unravel it.’\\
        \glt     ‘Lo va a deshilar.’ < SFH 2014 1:142>\\
    }\label{ex: imperative L tone replaces stem toneb}
    \z
\z

As shown in the contrast between (\ref{ex: imperative L tone replaces stem tonea}) and (\ref{ex: imperative L tone replaces stem toneb}), a HL tone in a stem containing a stressed transitive \textit{-na} suffix is replaced when inflected for the imperative singular, since the imperative L-tone allomorph associates with the stressed syllable, replacing the stem HL tone.\footnote{\chapref{chap: prosody} provides more details about patterns of tonal replacement in morphologically complex words.}

\hspace*{-3.5pt}In contrast, L-toned stems and H-toned stems do not undergo any tonal changes when inflected for the imperative singular. Suffixing allomorphs and stress shifts are available in words containing these stems in order to encode this inflectional construction. This is exemplified in (\ref{ex: no tonal replacement in imp sg of L and H toned verbs}).

\ea\label{ex: no tonal replacement in imp sg of L and H toned verbs}
{No tonal replacement }

\begin{tabular}{llllll}
     & \textit{Stem} & &\textit{\textsc{imp.sg}} & & \textit{Gloss} \\
     a. & {oˈhò} &	L &oˈhò&L&	‘Thresh!' < BFL el1906>\\
     b. & {seˈmè} & L & seˈmè & L & ‘Play the violin!’ < BFL el1920>\\
     c. & {biʔˈtò} &	L&	biʔˈtò& L& ‘Twist your ankleǃ’ <RIC el2024>\\
     \end{tabular}

\begin{tabular}{llllll}
     d. & {saˈkú} & H & saˈkú &  H & ‘Dry it in the sun!’ < BFL el1923>\\
     e. & {kiˈmá}	& H & kiˈmá &	H &	`Put on blanket!' < BFL el1909>\\
     f. & {sutuˈbú}	& H & sutuˈbú&H&‘Tie its legsǃ < BFL el1911>\\
\end{tabular}
    \z

Replacement of HL tones with grammatical L imperative singular tone thus leads to neutralization of HL and L lexical tones in this construction. Finally, and as shown in (\ref{ex: four allomorphs of the imperative singular}c--d) above, the L-tone imperative allomorph is incompatible with other imperative singular marking. Thus, there is no multiple exponence involved for this inflectional category.


%{}- The verb ‘to hug’ in 2014 1:152 is annotated as having a L tone in the imperative with a lengthened stem vowel and risingtone – a boundary H\% tone?


%\textit{{}- Here also show how there is a H\% boundary tone with L-toned stems, showing this effect is independent of the lexical tone. Any examples of H-toned stems with a H\% tone?}


\subsection{Morphologically-conditioned tone}
\label{subsec: morphologically conditioned tone}

The lexical tonal properties of suffixing constructions is addressed in §\ref{subsec: lexical tone in suffixes}. Recall that these suffixing constructions belong to the shifting class, triggering stress shifts with lexically unstressed roots. There are, however, suffixes that condition tonal alternations onto the stems they attach with without triggering any stress shifts. This is the case of the imperfective \textit{-i} suffix and the present progressive \textit{-a} suffix, two suffixes that impose a L tone on a stressed stem syllable of a complex word containing an underlyingly unstressed root without conditioning any stress shifts: both the imperfective and the present progressive are neutral morphological suffixing constructions that do not trigger stress shifts and are unable to bear stress. The data in (\ref{ex: morphologically conditioned L imperfective and present progressive}) exemplify this morphologically-conditioned tone pattern.

\ea\label{ex: morphologically conditioned L imperfective and present progressive}
{Morphologically-conditioned L tone}

%% show data here of verbs with the same root where there is a L tone with imperfective and present progressive
%% to these examples add the present progressive
\begin{tabular}{llllllll}
        & \textsc{pst} \textit{-li} & & {\textsc{impf} \textit{-i}} & & {\textsc{prog} \textit{-a}} & &\\
     a. &  {aˈwí-li} &  H & aˈwì-i	& L & {aˈwì-a} & L & `dance'\\
     b. & {riˈwá-li} & H & {riˈwà-i} & L & {riˈwà-a} & L & {`find'}\\
     c. & {raʔˈlá-li} & H & {raʔˈlà-i} & L & {raʔˈlà-a} & L & {`buy'}\\
\end{tabular}
    \z


As the examples in (\ref{ex: morphologically conditioned L imperfective and present progressive}) show, the imperfective and the present progressive are associated with a L tone in the stem without triggering stress shifts, in contrast to other neutral constructions such as the past \textit{-li} suffix, which do not trigger any tonal alternations in the stems to which they attach (i.e., the H tone in the stem when attaching the past \textit{-li} suffix is the underlying tone of the verbal root).

As mentioned above, this morphologically-conditioned effect is only attested when the imperfective and progressive suffixes attach to lexically unstressed roots. When the roots exemplified in (\ref{ex: morphologically conditioned L imperfective and present progressive}) above attach a shifting construction, the resulting morphologically complex word exhibits a stress shift (in these cases, stress falls on the shifting suffixes). This is shown in (\ref{ex: only unstressed roots have morphologically conditioned L tone}).

\ea\label{ex: only unstressed roots have morphologically conditioned L tone}
{Morphologically-conditioned tone only targets unstressed roots}

\begin{tabular}{llllllll}
      & \textsc{pst} \textit{-li} & & {\textsc{impf} \textit{-i}} & & {\textsc{prog} \textit{-a}} & &\\
     a. & aˈwí-li & H & {aˈwì-i} & L & {awi-ˈsâ} & HL & {‘dance’}\\
     b. & {riˈwá-li} & H & {riˈwà-i} &	{L} & {riwi-ˈsâ} & {HL} & {‘find’}    \\
     c. & raʔˈlá-li & H & {raʔˈlà-i} &	L & {raʔla-ˈsâ} &  HL & {`buy'}\\
     d. & {hiˈrâ-li} & HL & {hiˈrâ-i} & HL & {hiˈrâ-sa} & HL & {`bet'}\\
     e. & {muˈrú-li} & H & {muˈrú-i} & H & {muˈrú-sa} & H & {`carry in arms'}\\
     f. & {iʔˈtʃ͡à-li} & L & {iʔˈtʃ͡à-i} & L & {iʔˈtʃ͡à-sa} & L & {`plant'}\\
\end{tabular}
    \z

As in the examples in (\ref{ex: only unstressed roots have morphologically conditioned L tone}a--c) show, stress shifts with lexically unstressed roots; if stress is on the suffix, the lexical tone associated with that suffix emerges (in the case of the conditional \textit{-sâ} suffix, a falling HL tone). When stress falls on the stem containing these lexically unstressed roots, the surface tone is either the lexical root tone (when attaching the past tense \textit{-li} suffix) or the morphologically-conditioned L tone (when attaching the imperfective \textit{-i} suffix or the present progressive \textit{-a} suffix). The forms in (\ref{ex: only unstressed roots have morphologically conditioned L tone}d--f) show that no tonal alternations are attested when the imperfective and progressive suffixes attach to lexically stressed roots (which do not exhibit or any tone changes across morphological paradigms).

Finally, it should be noted that the tonal effect that the imperfective and present progressive suffixes have in the stems to which they attach can be traced diachronically to recent changes in the segmental and prosodic make up of these suffixes: as evidenced in data from \ili{Norogachi Rarámuri} \parencite{brambila1953gramatica}, and the closely related \ili{Mountain Guarijío} \parencite{miller1996guarijio}, both of these suffixes were stressable suffixes that recently lost a palatal glide onset. For more discussion of this recent development, see \chapref{chap: prosody}.

\subsection{Alternating tone stems}
\label{subsec: alternating tone stems}

A subset of stressed verbal stems in Choguita Rarámuri (referred to as ``alternating'' in \citealt{caballero2015tone}) exhibit alternating tone patterns that are predictable in terms of the inflectional morphology with which they combine. As exemplified in (\ref{ex: paradigmatic tone examples}), these stems exhibit a HL tone in shifting morphological contexts (those that condition stress shifts and other morpho-phonological alternations) and a L tone in neutral contexts. There are no stress shifts in either context, and only the tonal alternation is indicative of either the shifting or neutral morphological context. These generalizations hold when considering two-level constructions (stem and only one morphological construction).
%Paraphrase the next section, taken verbatim from the Prosody chapter

\newpage
\ea\label{ex: paradigmatic tone examples}
{Alternating verbs: tonal alternations}\mbox{}

\begin{tabular}{llllll}
     & \textit{Neutral} & & \textit{Shifting} & & \\
     & \textsc{pst/pst.ego}	& & \textsc{fut.sg} & & \\
     a. & ˈpà-li & 	L &  ˈpâ-ma &	HL  &  	‘to bring’ 	<RIC el1921>\\
     b. & naˈwà-li   &	L  &	naˈwâ-ma &	HL &	‘to arrive’	<RIC el1921>\\
     c. & biʔˈw-à-li  &	L &   	biʔˈw-â-ma  &   HL &  	‘to clean (tr.)’ < BFL el1903>\\
     d. & aʔ ̍wà-li &   L & 	aʔ ̍wâ-ma &    HL  & 	`to swallow'	<LEL 2063>\\
     e. & ne ̍wà-ki &	L &	ni'wâ-ma &	HL &	‘to make, do	<RIC 1892>\\
\end{tabular}
    \z

The examples in (\ref{ex: paradigmatic tone of biwa - neutral}) and (\ref{ex: paradigmatic tone of biwa - shifting}) show the tone alternations of a single verb \textit{biʔw-a} ‘to clean’ (a stem composed of the root \textit{biʔw-} ‘to clean’ and the transitivizer \textit{-a} suffix), where the surface tonal properties of the word are determined by the type of morphological construction: L if neutral (\ref{ex: paradigmatic tone of biwa - neutral}) and HL if shifting (\ref{ex: paradigmatic tone of biwa - shifting}).

\ea\label{ex: paradigmatic tone of biwa - neutral}
{Neutral constructions}
    \ea[]{
    \triplebox{{biʔˈw-à-ki}}{L}{Past egophoric}\\
}
        \ex[]{
        \triplebox{{biʔˈw-à-li}}{L}{Past}\\
    }
	        \ex[]{
	        \triplebox{{biʔˈw-à-i}}{L}{Imperfective}\\
	    }
	            \ex[]{
	            \triplebox{{biʔˈw-à-a}}{L}{Present progressive}\\
	        }
    \z
\z

\ea\label{ex: paradigmatic tone of biwa - shifting}
{Shifting constructions}
	                \ea[]{
	                \triplebox{{biʔˈw-â-ma}}{HL}{Future singular}\\
	            }
	                    \ex[]{
	                    \triplebox{{biʔˈw-â-ʃi}}{HL}{Imperative plural}\\
	                }
                            \ex[]{
                            \triplebox{{biʔˈw-â}}{HL}{Imperative singular}\\
                        }
                                \ex[]{
                                \triplebox{{biʔˈw-âa-ru}}{HL}{Past passive}\\
                                < BFL el1903, LEL 18:164>\\
                            }
	\z
\z

In these cases, it is not possible to identify any underlying lexical tone properties of the verb roots and tonal alternations cannot be predicted from the lexical tonal properties of suffixes. These tonal alternations are also independent of stress alternations and other phonological properties. Stress and tone, though closely related, are orthogonal dimensions in Choguita Rarámuri.

\subsection{Summary}
\label{subsec:9:summary:tone}

The following is a summary of the different grammatical tone patterns documented Choguita Rarámuri (for further discussion, see \chapref{chap: prosody}, as well as \citealt{caballero2021grammatical}):

\ea\label{ex: morphological dimensions of CR tone}
{Choguita Rarámuri grammatical tone patterns}

\begin{itemize}
\item
Tone may be a morphological exponent of inflection (tone as realizational morphology) (§\ref{subsec: tone as realizational morphology}) \\
\item
Suffixes have tonal properties and may induce changes on the stems to which they attach (tone as morphologically-conditioned phonology) (§\ref{subsec: morphologically conditioned tone})\\
\item
Tone is systematically distributed across verb classes in terms of neutral vs. shifting inflectional morphology constructions (§\ref{subsec: alternating tone stems}) \\
\end{itemize}

\z

%Might make more sense to put this whole section after discussion of the hierarchical structure of the verb
Having addressed the properties of root classes in Choguita Rarámuri, including the tone properties of roots as well as grammatical tone patterns, we can now turn to the non-productive and non-concatenative processes occurring at the next level of the Choguita Rarámuri verb, the Inner Stem.


\section[The Inner Stem]{The Inner Stem: noun incorporation, non-concatenative morphology and unproductive processes}
\label{sec: the inner stem}


The Inner Stem, the input to suffixation, is composed of a denominalized noun root, an incorporated verb root, or a verbal root that has optionally undergone a non-concatenative process or a process that is no longer fully productive, including subject number prefixation and pluractional marking.


This first stem domain can also be characterized morpho-phonologically: this is the domain of application of compensatory lengthening (described in §\ref{subsubsec: compensatory lengthening}), and a morphological incorporation stress rule that assigns stress to the first syllable of the head of an incorporated verb (described in more detail in §\ref{subsec: body-part incorporation}). In general, the morphological processes taking place at this stage are more tightly fused phonologically to the root than any later morphological process.

This sub-section is organized as follows: §\ref{subsec: non-concatenative processes} describes non-concatenative processes; §\ref{subsec: instrumental prefixes} gives an overview of instrumental prefixes; §\ref{subsec: body-part incorporation} describes the process of body-part incorporation; §\ref{subsec: number marking: suppletion and plural prefixes} describes number suppletion and plural prefixation; and §\ref{subsec: denominal verbs} gives an overview of the unproductive verbalizer suffixes and their phonological effects on their bases.

\subsection{Non-concatenative processes}
\label{subsec: non-concatenative processes}

Non-concatenative processes in Choguita Rarámuri include conversion (§\ref{subsubsec: conversion}), pluractional consonant mutation (§\ref{subsubsec: pluractionality}), stress and tone changes to encode imperative singular (§\ref{subsubsec: imperative final stem stress}), and stress shifts encoding a derivational verbalization process (§\ref{subsubsec: stress shift as verbalization}).


\subsubsection{Conversion}
\label{subsubsec: conversion}


Some nominal stems (including some nouns refering to weather) can take verbal morphology with no overt denominal marking. The examples in (\ref{ex: conversion examples}) are listed with the future singular suffix. With some exceptions (like (\ref{ex: conversion examples}a)), most of these zero derived verbs belong to the unstressed stem class in terms of their stress behavior.

\ea\label{ex: conversion examples}
{Noun-to-verb conversion}

\begin{tabular}{lllll}
 & \textit{Verb stem} & & \textit{Noun stem}\\
     a. & reˈmê-ma & {make.tortillas-\textsc{fut.sg}} & {reˈmê} & {‘tortillas’}\\
     b. & {ika-ˈmêa} & {be.windy-\textsc{fut.sg}} & {iˈká} & {`wind'}\\
     c. & {moro-ˈmêa} & {be.foggy-\textsc{fut.sg}} & {moˈrí} & {`smoke'}\\
     d. & {uku-ˈmêa} & {rain-\textsc{fut.sg}} & {uˈkí} & {‘rain’}\\
     e. & {nori-ˈmêa} & {be.cloudy-\textsc{fut.sg}} & {noˈrí} & {`cloud'}\\
     f. & {saki-ˈmêa}&{make.esquite-\textsc{fut.sg}}&{saˈkí}&{‘esquite’}\\
\end{tabular}
\z

Examples (\ref{ex: conversion examples}c--d) show also that the unstressed vowel of the derived verbal form can harmonize with the stem’s first vowel (\textit{moˈri} ‘smoke’ becomes \textit{morò-} ‘to be smoky/foggy’ and \textit{uˈki} ‘rain’ becomes \textit{uku-} ‘to rain’). This harmonizing process is absent in cognate forms of zero derivation in \ili{Mountain Guarijío} (cf. \textit{yuʔkí} ‘rain’, \textit{yuʔki-má} ‘to rain’) \citep[][148]{miller1996guarijio}.


\subsubsection{Pluractionality: prefixation and consonant mutation}
\label{subsubsec: pluractionality}

There is a process that marks plural subject with verbs, or that an action occurs or is being performed by the same agent several times, or by several agents several times. When used with nouns it marks plural number. These meanings are related in that they refer to event plurality or ``pluractionality''. Pluractionals have been defined as encompassing meanings that range from iterative and frequentative to distributive and extensive action (\citealt{newman1990nominal}, \citealt{newman2012pluractional}, \citealt{wood2007semantic}).

Choguita Rarámuri pluractional forms are marked through a prefixed element analyzed in other descriptive works of Rarámuri and \ili{Guarijío} as reduplication (\citealt{lionnet1968intensivos}, \citealt{lionnet1985lionnet}).\footnote{\citet{lionnet2001elementos} labels these ``intensive''.} In \ili{Mountain Guarijío}, the cognate process (labelled ``plural subject, iterative or durative''), is more clearly analyzed as reduplication, since the prefixed element is (C)V- (e.g. \textit{saé}, \textit{sa-saé} ‘smell’, \textit{isí, i-isí} ‘walk’) \citep[][62]{miller1996guarijio}.

Pluractionals in Choguita Rarámuri are marked in three ways: (i) through a prefixed vowel (\ref{ex: pluractionals}a--b) (where the vowel quality of the prefix can be harmonized to the root’s first syllable vowel); (ii) through consonant mutation (\ref{ex: pluractionals}c--h); or (iii) through both consonant mutation and a prefix element (\ref{ex: pluractionals}i--o).\footnote{It has been suggested that the prefix-like element was originally a prefix \textit{i-} that has been leveled in color with the first stem vowel in contemporary Rarámuri varieties (\citealt{lionnet2001elementos}).}

\ea\label{ex: pluractionals}
{Pluractionals}
\setlength{\tabcolsep}{4pt}
\begin{tabular}{lllll}
        & \textit{Base} & \textit{Pluractional} & \textit{Gloss} & \\
     a. & ˈtʃ͡óni & {o-ˈtʃ͡óni} & {`become black'} & < AHF 05 2:24/el > \\
     b. & {siˈrî-ame} & {i-ˈsêri-k-ame} & {`governor'} & {< BFL 05 1:156/el >}\\
     c. & {kaˈ\textbf{p}ô-r-ame} & {kaˈ\textbf{b}ô-r-ame} & {`round one'} & {< BFL 05 1:155/el >}\\
     d. & {\textbf{r}emaˈrí} &{ˈ\textbf{t}êmuri}&{‘young person’} & {< BFL 05 1:155/el >}\\
     e. & {\textbf{r}ikuˈrí}&{ˈ\textbf{t}êkiri}&{`drunk person'} & {< BFL 05 1:156/el >}\\
     f. & {kaˈ\textbf{p}í-r-ame}&{kaˈ\textbf{b}í-r-ame}&{`cylindrical one'} &{< BFL 05 1:156/el >}\\
     g. & {saˈ\textbf{p}ê-ami}&{saˈ\textbf{b}ê-ami}&{`fat one'} & {< BFL 05 1:156/el >}\\
     h. & {\textbf{r}oˈsâ-k-ami}&{\textbf{t}oˈsâ-k-ami}&{`white one'}& {< BFL 05 1:157/el >}\\
     i. & {kiˈ\textbf{p}á}&{i-kiˈ\textbf{b}á}&{`to snow'} & {< SFH 05 2:8/el >}\\
     j. & {ku\textbf{p}uˈwé}&{u-kuˈ\textbf{b}é}&{`grill peppers'}\\
     k. & {siˈ\textbf{t}á-k-ame}&{i-siˈ\textbf{r}á-k-ame}&{`red one'}& {< BFL 05 1:157/el >}\\
     l. & \textbf{b}aˈhî & {a-\textbf{p}aˈhî}&{`to drink'} & {< SFH 08 1:46/el >}\\
     m. & tʃ͡aˈ\textbf{b}ȏtʃ͡i  &{i-ˈtʃ͡â\textbf{p}otʃ͡i}&{`mestizo'}& {< BFL 05 1:155/el >}\\
     n. & muˈ\textbf{k}î&{o-muˈ\textbf{g}î}&{`woman'}& {< BFL 05 1:156/el >}\\
     o. & \textbf{r}aˈnâra&{a-\textbf{t}aˈnâra}&{‘offspring’}& {< BFL 05 1:156/el >}\\
     p. & {siˈ\textbf{t}â-k-ame}&{i-siˈ\textbf{r}â-k-ame}&{`red one'}& {< BFL 05 1:156/el >}\\
\end{tabular}
\z

Consonant mutation involves a voicing toggle, since it produces voicing or lenition of a voiceless stop (\ref{ex: pluractionals}b--f), and devoicing or hardening of a voiced plosive (\ref{ex: pluractionals}f).

As shown in (\ref{ex: g in pluractional marking}), consonant mutation also targets voiceless velar stops to encode pluractionality.

\ea\label{ex: g in pluractional marking}
{k \textasciitilde g alternation in pluractional marking}

\begin{tabular}{llll}
    & \textit{Forms} & \textit{Gloss} & \\
     a. &  {paˈ\textbf{k}ó-t-ami}&{‘good person (baptized)’}&{< SFH 06 in61/in >}\\
     b. & {paˈ\textbf{g}ó-t-ami}&{‘good people’, ‘people’}&{< FLP 06 in61/in >}\\
\end{tabular}

\z

The example above shows that the plural of participle \textit{paˈ}\textbf{\textit{k}}\textit{ó-t-ami} (\ref{ex: pluractionals}a), with a word-medial voiceless velar stop, is \textit{paˈ}\textbf{\textit{g}}\textit{ó-t-ami}, with a voiced word-medial velar stop (\ref{ex: pluractionals}b). The data in (\ref{ex: stop alternations in pluractional constructions}) shows more examples of stop alternations in pluractional constructions.

\ea\label{ex: stop alternations in pluractional constructions}
{Stop alternations in pluractional constructions}

\begin{tabular}{lllll}
 & \textit{Base} & \textit{Pluractional} & \textit{Gloss} &\\
     a. &  \textbf{b}iˈ\textbf{t}êli & {i-\textbf{p}iˈ\textbf{r}ê} & {`dwell'}& {< BFL 05 1:186/el >}\\
     b. & {\textbf{b}aˈhî}&{a-\textbf{p}aˈhî}&{`drink'}&   {< BFL 05 2:23/el >}\\
     c. & {kaˈ\textbf{p}ô-r-ame}&{kaˈ\textbf{b}ô-r-ame}&{`round thing'}&      {< BFL 05 1:155/el >}\\
     d. & {kiˈ\textbf{p}á}&{i-kiˈ\textbf{b}á}&{`to snow'}&   {< SFH 05 2:8/el >}\\
     e. & {siˈ\textbf{t}â-k-ame}&{i-siˈ\textbf{r}â-k-ame}&{`red thing'}&  { < BFL 05 1:157/el >}\\
     f. & {\textbf{p}iˈwâ}&{i-ˈ\textbf{b}éwa}&{`to smoke'}&  {< BFL 05 2:24/el >}\\
\end{tabular}
    \z


\subsubsection{Imperative L tone and final stem stress}
\label{subsubsec: imperative final stem stress}

%Need to reexamine given the tonal allomorph of the imperative

\largerpage
As described in §\ref{subsec: tone as realizational morphology}, the imperative singular construction has non-concatena\-tive allomorphs, including stress shifts and grammatical tone. Relevant examples are shown in (\ref{ex: imperative stress shift}).

\ea\label{ex: imperative stress shift}
{Imperative stress shift}

    \ea[]{
    \gll    raʔaˈm\textbf{à}\\
            give.advice.\textsc{imp.sg}\\
    \glt    `Give advice!'\\
    \glt    `¡Aconseja!'\\
}\label{ex: imperative stress shifta}
        \ex[]{
        \textit{raʔaˈmâbo}\\
        \gll    raʔaˈm\textbf{â}-bo\\
                give.advice-\textsc{fut.pl}\\
        \glt    `They will give advice.'  \\
        \glt    `Van a aconsejar.'\\
    }\label{ex: imperative stress shiftb}
            \ex[]{
            \textit{raˈʔàmili}\\
            \gll    raˈʔ\textbf{à}ma-li\\
                    give.advice-\textsc{pst} \\
            \glt    `S/he gave advice.'\\
            \glt    `Dió consejo.' < ROF 04 1:64/el >\\
        }\label{ex: imperative stress shiftc}
%        \pagebreak
                \ex[]{
                \textit{raʔiˈtʃ͡à}\\
                \gll    raʔiˈtʃ͡\textbf{à}\\
                        speak.\textsc{imp.sg}\\
                \glt    `Speak!'\\
                \glt    `¡Habla!' < SFH 2014:60>\\
            }\label{ex: imperative stress shiftd}
            \newpage
                    \ex[]{
                    \textit{raʔiˈtʃ͡âma}\\
                    \gll    raʔiˈtʃ͡\textbf{â}-ma\\
                            speak-\textsc{fut.sg}\\
                    \glt    `S/he will speak.'\\
                    \glt    `Va a hablar.'  < SFH 2014:60>\\
                }\label{ex: imperative stress shifte}
                        \ex[]{
                        \textit{raˈʔìtʃ͡iki}\\
                        \gll    raˈʔ\textbf{ì}tʃ͡a-ki\\
                                speak-\textsc{pst.ego}\\
                        \glt    `I spoke.'\\
                        \glt    `Hablé.'   < SFH 2014:60>\\
                    }\label{ex: imperative stress shiftf}
    \z
\z

Final stem imperative stress of unstressed roots contrasts with the stress pattern of these roots in neutral constructions: in (\ref{ex: imperative stress shift}), the trisyllabic unstressed roots \textit{raʔáma} ‘give advice’, and \textit{raʔìtʃ͡a} ‘speak’, have third syllable stress in the imperative (\ref{ex: imperative stress shifta}) and (\ref{ex: imperative stress shiftd})) as well as in shifting constructions ((\ref{ex: imperative stress shiftb}) and (\ref{ex: imperative stress shifte})), but second syllable stress in neutral constructions ((\ref{ex: imperative stress shiftc}) and (\ref{ex: imperative stress shiftf})).\footnote{As discussed in more detail in \chapref{chap: prosody}, there are systematic tonal alternations associated with stress shifts on morphologically complex words containing unstressed roots: HL falling tones are associated with stress shifts (e.g., when inflected for the future plural (e.g., (\ref{ex: imperative stress shiftb})) or future singular, as in (\ref{ex: imperative stress shifte})); in the case of the imperative singular encoded through a stress shift, speakers may exhibit variation in the tonal makeup of these forms; the examples above show forms where the imperative singular forms have a L tone, which can be analyzed as the grammatical L tone associated with this construction, an instance of multiple exponence.}

As discussed in further detail in §\ref{sec: verbal structure and verbal domains}, the stress shift to mark imperative is restricted to occur within a defined verbal zone, the Derived Stem.

In addition to stress, Choguita Rarámuri deploys tone in the exponence of the imperative singular: in addition to two concatenative suppletive allomorphs, \textit{{}-sâ} and \textit{{}-kâ}, a stress shift allomorph, the imperative singular may also be encoded through a L tonal allomorph that replaces stem lexical HL tones. The examples in (\ref{ex: partial paradigm of mato}) above, repeated here in (\ref{ex: paradigm of mato}), show a HL-toned verb, \textit{ma}ˈ\textit{tó} ‘to carry something in shoulder’, inflected for different tenses/aspect and person/number combinations.

%\pagebreak

\ea\label{ex: paradigm of mato}
{Partial paradigm of the verb \textit{maˈtó} ‘to carry in shoulders’}

    \ea[]{
    \textit{maˈtóa}\\
    \textit{maˈtó-a}\\
    {carry.shoulders-\textsc{prs}}\\
    {‘(S)/he carries it on the shoulders.’}  \\  `Lo carga en sus hombros.'\\
}
        \ex[]{
        \textit{maˈtóki}\\
        \textit{maˈtó-ki}\\
        {carry.shoulders-\textsc{pst.ego}}\\
        {‘I carried it on the shoulders.’}\\
        `Lo cargué en los hombros.'\\
    }
            \ex[]{
            \textit{maˈtóma}\\
            \textit{maˈtó-ma}\\
            {carry.shoulders-\textsc{fut.sg}}\\
            {‘I will carry it on the shoulders’}\\
            `Lo voy a cargar en los hombros.'\\
        }
                \ex[]{
                \textit{maˈtói}\\
                \textit{maˈtó-i}\\
                {carry.shoulders-\textsc{impf}}\\
                {‘I was carrying it on the shoulders’} \\
                `Lo cargaba en los hombros.'\\
            }
                    \ex[]{
                    \textit{maˈtòo}\\
                    \textit{maˈt\textbf{ò}}\\
                    {carry.shoulders\textsc{.imp.sg}}\\
                    {‘Carry it on the shoulders!’}\\
                    `Cárgalo en los hombros!'\\
                }
{< BFL 2014 1:152 >}
    \z
\z

As shown in the contrast between these inflected forms, a HL-toned verb will keep its lexical tonal melody across derivations without any stress shifts, except when inflected for the imperative singular tonal allomorph: in this case, the morphological L tone replaces the HL lexical tone of the stem. This construction contributes to the list of non-concatenative processes of the language.

\subsubsection{Stress shift as verbalization}
\label{subsubsec: stress shift as verbalization}

Some nouns can be denominalized through a stress shift one syllable to the right. The stress shift in these cases cannot be said to be triggered by a shifting construction, since (\ref{ex: verbalizing stress shifta}) is inflected with a stress-shifting suffix and (\ref{ex: verbalizing stress shiftc}) is inflected with a stress-neutral suffix. The stress shift is, thus, only attributable to the verbalizing process.

\ea\label{ex: verbalizing stress shift}
{Verbalizing stress shift}

    \ea[]{
    \textit{sipuˈtʃ͡âma}\\
    \gll    sipuˈtʃ͡â-ma\\
            wear.skirt\textsc{-fut.sg}\\
    \glt    `She will wear a skirt.'\\
    \glt    `Se va a poner falda'\\
}\label{ex: verbalizing stress shifta}
        \ex[]{
        \gll    siˈpútʃ͡a\\
                `skirt'\\
        \glt     `falda' < BFL 07 Sept 6/el >\\
    }\label{ex: verbalizing stress shiftb}
            \ex[]{
            \textit{napaˈtʃ͡âli}\\
            \gll    napaˈtʃ͡â-li\\
                    wear.blouse-\textsc{pst}\\
            \glt    `She wore the blouse.'\\
            \glt    `Se puso la blusa.' < BFL 07 Sept 6/el >\\
        }\label{ex: verbalizing stress shiftc}
                \ex[]{
                \gll    naˈpátʃ͡a\\
                        ‘blouse’\\
                \glt    `blusa' < BFL 07 Sept 6/el >\\
            }\label{ex: verbalizing stress shiftd}
    \z
\z

Some of the nouns that undergo this stress shift may alternatively undergo other processes to mark verbalization. For instance, the denominal verb in (\ref{ex: verbalizing stress shifta}) can be alternatively realized as \textit{sipu-ˈtâ-ma}, `she will wear a skirt', with denominalizing suffix \textit{-tâ}, which involves a concomitant noun truncation process (see also discussion of this process in §\ref{subsec: denominal verbs}). I have not detected any semantic difference between these alternative forms.

%About this section below: how many details belong here vs. the section on morphological tone below? for the time being, this short section is repeated below

\subsection{Instrumental prefixes}
\label{subsec: instrumental prefixes}

A well-known phenomenon of \ili{Uto-Aztecan} languages is the presence of a set of instrumental prefixes that indicate the instrument with which a transitive activity is performed, or the manner in which the activity is carried out. These prefixes (approximately 20) are reconstructed for \ili{Proto-Uto-Aztecan}. A list of reconstructions is given in (\ref{ex: UA instrumental prefixes}) (from \citealt[][92]{dayley1989tumpisa}):


\ea\label{ex: UA instrumental prefixes}
{\ili{Uto-Aztecan} instrumental prefixes}

\begin{tabular}{llll}
     a. & {‘with heat or fire’}&{*kuh}&{`fire'}\\
     b. & {‘with the teeth or mouth’}&{*kü’i}&{`bite'}\\
     c. & {‘with the hand’}&{*maa}&{`hand'}\\
     d. & {‘with the nose’}&{*mu-pi}&{`nose'}\\
     e. & {‘(with or pertaining to) water’}&{*paa}&{`water'}\\
     f. & {‘with the butt or behind’}&{*pih}&{`back'}\\
     g. & {‘with or from cold’}&{*süp}&{`cold'}\\
\end{tabular}

\begin{tabular}{llll}
     h. & ‘with the mind, by feeling or sensation’&{*suuna}&{`heart'}\\
     i. & {‘with the foot’}&{*tannah}&{`foot'}\\
\end{tabular}

\z

According to \citet{langacker1977uto}, these instrumental prefixes are morphologically active only in the \ili{Numic} and \ili{Tepiman} branches of \ili{Uto-Aztecan}, and have only lexicalized remnants in the rest of the \ili{Uto-Aztecan} family. This is true in the case of Choguita Rarámuri. Instrumental incorporation in Choguita Rarámuri is only found in a few lexicalized items, some of which are exemplified in (\ref{ex: CR instrumental prefixes}).

\ea\label{ex: CR instrumental prefixes}
{Choguita Rarámuri instrumental prefixes}

\begin{tabular}{lllll}
     a. &  {ma+ˈtʃ͡ó}&{hand +hit}&{‘hit with hand’}&{< ROF 04 1:124/el >}\\
     b. & {ma+ˈhó}&{hand+dig}&{`fondle'}&{< BFL 04 Com/el >}\\
     c. & {rara+ˈhó}&{foot+dig}&{‘dig with foot’}&{< SFH 08 1:48/el >}\\
     d. & {sika+ˈtʃ͡ó}&{hand+hit}&{`hit with hand'}&{< SFH 04 1:123/el >}\\
\end{tabular}

\z

In some cases, the incorporated noun does not have a corresponding independent nominal form, as \textit{ma-} (\ref{ex: CR instrumental prefixes}a). \citep[87]{lionnet2001sufijos} identifies several instrumental prefixes in the Norogachi dialect (\textit{ba(ʔ)-} ‘water’, \textit{ku-} ‘wood’, \textit{ma-} ‘hand’, \textit{mo(ʔ)-} ‘head’, and \textit{na-} ‘fire’), but does not provide any examples. The Choguita Rarámuri verbs \textit{moˈtot͡ʃi} ‘hit (oneself) in the forehead’, \textit{moˈtépia} ‘braid hair' are likely lexicalizations of verbal roots plus the instrumental prefix \textit{mo-} ‘head’, but no alternations are documented to support a synchronic analysis of these forms as morphologically complex.


\subsection{Body-part incorporation}
\label{subsec: body-part incorporation}

Another Inner Stem building process in Choguita Rarámuri is noun incorporation. The language has N-V constructions that are restricted to nouns referring to body parts and bodily fluids, with properties that are prototypical of ``body part incorporation''. This is a restricted kind of noun incorporation, which is common in languages of the Americas, including the \ili{Uto-Aztecan} language family \parencite{baker1988incorporation}. Relevant examples are shown in (\ref{ex: bodypart  incorporation}), where the body-part nouns (\textit{moˈʔô} ‘head’, \textit{buˈsi} ‘eye’, \textit{roˈpâ} ‘belly’, etc.) precede a verbal root, the head of the incorporated construction.

\ea\label{ex: bodypart  incorporation}
{Body part incorporation constructions}
%change the underlying representation?
    \ea[]{
    \textit{moʔoˈrêpi}\\
    \textit{moʔo+ˈrêpu}\\
    {head+cut}\\
    {‘to behead’}\\
    `cortar la cabeza' {< SFH 07 1:185/ el >}\\
}
        \ex[]{
        \textit{busiˈkâsi}\\
        \textit{busi+ˈkâsi}\\
        {eye+break}\\
        {`to be blind'}
        `ser ciego' {< SFH 06 1:112/el >}\\
    }
            \ex[]{
            \textit{ropaˈkâsi}\\
             \textit{ropa+ˈkâsi}\\
             {belly+break}\\
             {‘to have a miscarriage’} \\
             `tener un aborto' {< BFL 07 VDB/el >}\\
        }
                \ex[]{
                \textit{{siwaˈbôti}}\\
               \textit{siwa+ˈbôti}\\
               {guts+loosen}\\
               {‘to become disemboweled’}\\
               `destriparse' {< SFH 07 1:186/el >}\\
            }
                    \ex[]{
                    \textit{kutaˈbîri}\\
                   \textit{kuta+ˈbîri}\\
                   {neck+twist}\\
                   {‘to twist one's neck’}\\
                   `torcerse el cuello' {< SFH 07 VDB/el >}\\
                }
                        \ex[]{
                        \textit{tʃ͡omaˈbîwa}\\
                     \textit{tʃ͡oma+ˈbîwa}\\
                     {mucus+clean}\\
                     {‘to clean mucus’}\\
                    `limpiarse los mocos' {< SFH 08 1:51/el >}\\
                    }
    \z
\z

Some of these constructions (e.g. (\ref{ex: bodypart  incorporation}d--f)) are fully compositional and have transparent semantics. The incorporated noun generally fills the syntactic role of object, decreasing the verb’s valence. Incorporated constructions can also be externally modified, as example (\ref{ex: external modification of incorporated verb}) shows.

\ea\label{ex: external modification of incorporated verb}
{External modification of incorporated verb}

{\textit{ˈmá  koˈsúrti    ronoˈrêpili  ˈmôno}}\\
\gll    ˈmá   koˈsúriti  roˈnô+reˈpù-li  ˈmôno\\
        already    left    leg+cut-\textsc{pst}  doll\\
\glt    ‘The doll’s left leg already fell’\\
\glt    ‘Ya se le cayó la pata izquierda al mono’        < SFH 07 1:186/el >\\
\z

Incorporated body-part nouns, however, do not form an open class. Only a few body-part terms are found in these constructions, and not the full range of lexical items in this semantic class. Body-part terms such as \textit{raˈníka-la} ‘heel’, \textit{koˈtʃ͡ía-la} ‘eyebrow, lashes’, and \textit{loˈmíritʃ͡i} ‘wrist’, among others, have not yet been documented in incorporated constructions.

As for the prosodic make-up of incorporated forms, stress is assigned in the first syllable of the head of the construction, the verbal root, regardless of the underlying lexical stress information the roots might carry (see also (\ref{ex: incorporation stress rule}) on p. \pageref{ex: incorporation stress rule}). In forms where the first element, the noun, is disyllabic and where the second element, the head verb, has second-syllable stress in non-incorporated structures, stress surfaces on the first syllable of the second member (\ref{ex: stress shift in incorporated forms}).

\ea\label{ex: stress shift in incorporated forms}
{Stress shift in incorporated forms}

    \ea[]{
    {[busiˈk\textbf{â}si]}\\
    \glt    /buˈsí+kaˈs\textbf{ì}/\\
    \glt        eye+break\\
    \glt    `to become blind (lit. eye-break)'\\
    \glt    `quedarse ciego (lit. quebrar-ojo)' < BFL 07 1:163/el >\\
}
        \ex[]{
        {[kutaˈb\textbf{î}ri]}\\
        \glt    /kuˈtâ+biʔr\textbf{ì}/\\
        \glt         neck+twist\\
        \glt    `to twist one's neck'\\
        \glt    `torcerse el cuello' < BFL 07 1:163/el >\\
    }
            \ex[]{
            {[ronoˈr\textbf{ê}pi]}\\
            \glt    /roˈnô+reˈp\textbf{ù}/\\
            \glt        food+cut\\
            \glt    `to cut one's foot'\\
            \glt    `cortarse el pie'  < BFL 07 1:163/el >\\
        }
                \ex[]{
                {[sikaˈr\textbf{ê}pi]}\\
                \glt    /siˈkâ+reˈp\textbf{ù}/\\
                        hand+cut\\
                \glt    `to cut one's hand'\\
                \glt    `cortarse la mano' < BFL 07 1:163/el >  \\
            }
    \z
\z

As shown in these examples, the tonal properties of incorporated verbs are homogeneous: the stressed syllable (the first syllable of the head of the compound) bears a HL tone, regardless of what the underlying tonal properties of the members of the compound are (for more discussion, see \sectref{subsec: stress and tone properties of compounds}).

Incorporated body-part nouns can also be truncated to fit the prosodic template of incorporated forms. In (\ref{ex: truncation of terasyllabic nouns in incorporation}), the trisyllabic nouns \textit{tʃ͡aˈmêka}, ‘tongue’, and \textit{tʃ͡ereˈwá}, ‘sweat’, truncate their final syllable in the incorporated form (the prosod\-ic properties of incorporated forms are addressed in more detail in \sectref{subsec: prosodic templates in CR}).
)

\ea\label{ex: truncation of terasyllabic nouns in incorporation}
{Truncation of tetrasyllabic nouns in incorporation}

%include tones of clean and cur
    \ea[]{
    {[tʃ͡ameˈrêpu]}\\
    \glt  /tʃ͡aˈmê\textbf{ka}+reˈpù/\\
    \glt        tongue+cut\\
    \glt    `to cut one's tongue'\\
    \glt    `cortarse la lengua' < SFH 07 1:184/el >\\
}
        \ex[]{
        {[tʃ͡ereˈbîwa]}\\
        \glt    /tʃ͡ere\textbf{ˈwá}+biʔwá/\\
        \glt        sweat+clean\\
        \glt    `to clean one's sweat'\\
        \glt    `limpiarse el sudor'   < SFH 07 1:187/el > \\
    }
    \z
\z

Finally, body part incorporation constructions exhibit voicing alternations. In (\ref{ex: stop alternations in incorporated verbs}), the verb \textit{paˈkó} ‘to wash’ has a voiced bilabial stop onset on its first syllable in incorporation.

\ea\label{ex: stop alternations in incorporated verbs}
{Stop alternations in incorporated verbs}

%include tones of teeth
    \ea[]{
    {[ronoˈ\textbf{b}âki-]}\\
    \glt    /ronô+\textbf{p}akó-/\\
    \glt         feet+wash\\
    \glt    `to wash one's feet'\\
    \glt    `lavarse los pies' < SFH 04 CS/el >\\
}
        \ex[]{
        {[rameˈ\textbf{b}âki-]}\\
        \glt    /rame+\textbf{p}akó-/ \\
        \glt        teeth+wash\\
        \glt    `to wash one's teeth'\\
        \glt    `lavarse los dientes' < SFH 04 CS/el >\\
    }
            \ex[]{
            {[kupa+ˈ\textbf{b}âki-]}\\
            \glt    /kupá+\textbf{p}akó-/\\
            \glt        hair+wash\\
            \glt    `to wash one's hair'\\
            \glt    `lavarse el pelo' < SFH 04 CS/el >\\
        }
    \z
\z

This voicing alternation in incorporated constructions is analyzed as a morpho\-log\-i\-cally-conditioned phonological effect.

%State here what the analysis is of these voicing alternations: a productive process? A morphologically conditioned effect?

\subsection{Suppletion and prefixation in pluractional marking}
\label{subsec: number marking: suppletion and plural prefixes}

Choguita Rarámuri lacks inflectional agreement marking, but displays a few lexically restricted resources to mark plurality and pluractionality on both nouns and verbs. This section describes plural and pluractional marking in verbs.

Like most \ili{Uto-Aztecan} languages, Choguita Rarámuri has number suppletion for certain roots, mostly positional verbs. Singular and plural agreement is expressed with intransitive patient-like subjects (\ref{ex: number verb suppletion}a--d) and with transitive objects (\ref{ex: number verb suppletion}e), in an ergative pattern.

\ea\label{ex: number verb suppletion}
{Number verb suppletion}

\begin{tabular}{lllll}
     & \textit{Singular} & \textit{Plural} & \\
     a. & {aˈsá-}&{moˈtʃ͡í-}&{‘to sit down'}&{< BFL 05 2:45-46/el >}\\
     b. & {wiˈlí-}&{haˈwá-}&{‘to stand up’}&{< BFL 05 2:46/el >}\\
     c. & {boˈʔí-}&{biʔˈtí-}&{‘to lie down’}&{< BFL 05 2:47/el >}\\
     d. & {baˈkí-}&{moˈʔí-}&{‘to go in’}&{< BFL 05 1:133/el >}\\
     e. & {miʔˈrí-}&{koˈʔí-}&{‘to kill’}&{< ROF 04 1:103/el >}\\
\end{tabular}
\z


In addition, there are two non-suppletive, semi-productive processes in Choguita Rarámuri to mark plurality. The first one is used exclusively with verbal roots and it marks subject number in an accusative pattern (S and A). When the subject is singular, a prefix \textit{ni-} is added or the root remains unprefixed; when the subject is plural, the prefix \textit{na-} is used (\ref{ex: number prefixation}).


\ea\label{ex: number prefixation}
{Number prefixation}

\begin{tabular}{lllll}
     & \textit{Singular} & \textit{Plural} & \textit{Gloss}&\\
     a. & {suˈrí/ni-suˈrí} & {na-suˈrí} &{‘to fight’} & {< BFL 05 2:107/el >}\\
    b. & {iʔˈkî/ ni-iʔˈkî} & {na-ˈkî} & {‘to bite’} &    {< BFL 05 2:107/el >}\\
    c. & {oˈtʃ͡ó/ni-ˈtʃ͡ô}& {naˈ-tʃ͡ó} & {`to punch'} & {< BFL 05 2:107/el >}\\
    d. & {kuˈʔîri/ ni-ˈkûri}&{na-ˈkûri} & {‘to help’}& {< ROF 04 VDB/el >}\\
    e. & {ni-hiˈrâpi} & {na-hiˈrâpi} & {‘to wrestle’}  & {< AHF 05 2:100/el >}\\
    f. & {ni-hiˈsípi} & {na-hiˈsípi} & {‘to reach’} & {< BFL 05 2:106/el >}\\
    g. & {ni-ˈtówi}&{na-ˈtówi}&{‘to surpass’}& < BFL 05 2:106/el >\\
    h. & {ni-hiˈsá}&{na-hiˈsá}&{‘to challenge’} & < BFL 05 2:106/el >\\
    i. & {ni-ˈtè} & {na-ˈtè} & {‘to step on’} & < BFL 05 2:107/el >\\
\end{tabular}
\z

The examples in (\ref{ex: number prefixation}a--d) have an independent root form with no prefix for the singular, but the forms in (\ref{ex: number prefixation}e--f) are not attested alternatively without a subject number prefix.

\subsection{Denominal verbs}
\label{subsec: denominal verbs}

Choguita Rarámuri verbs may be derived through a variety of morphological mechanisms applying to nominal bases. The mechanisms to derive verbs from nominal bases include several productive verbalizing suffixes such as the \textit{-ta} `make' suffix, the reversive \textit{-bu} suffix, the \textit{-tu} ‘gather’ suffix and the \textit{-ê} `have, wear' suffix, among others. Denominal verb constructions of this kind are documented across the \ili{Uto-Aztecan} language family, with meanings that include notions like `make', `put on', `have', `remove', `use', `get' and `become', among others (\citealt{haugen2008denominal, haugen2017derived}). These constructions are productive in Choguita Rarámuri. This subsection describes each of them in detail.

%Say here what the synchronic analysis in Haugen 2008

\subsubsection{The suffix \textit{-tâ} `make/become'}
\label{subsubsec: the make/become -ta suffix}

A noun derived with the verbalizer \textit{-tâ (-râ)} suffix, a stress-shifting suffix, becomes a verb meaning ‘to make/become N’ (\ref{ex: verbalizer -ta make/becomea}) or, when used with nouns referring to a piece of clothing, ‘to wear N’ (\ref{ex: verbalizer -ta make/become}b, c). It follows the nominal root forming a base to which inflectional morphology is added.

\ea\label{ex: verbalizer -ta make/become}

    \ea[]{
    \textit{noriˈrâma     ˈlé}\\
    \gll    nori-ˈrâ-ma     aˈlé\\
            cloud-\textsc{vblz-fut.sg}   \textsc{dub}\\
    \glt    ‘It will get cloudy, it seems.’\\
    \glt    ‘Se va a nublar, parece.’   < BFL 04 1:92/el >\\
}\label{ex: verbalizer -ta make/becomea}
        \ex[]{
        \textit{niˈhê   aʰkaˈrâsa     saˈpâto}\\
        \gll    niˈhê   aʰka-ˈrâ-sa     saˈpâto\\
                1\textsc{sg.nom}   sandal-\textsc{vblz-cond}    shoes\\
        \glt    ‘I will wear shoes.’\\
        \glt    ‘Voy a ponerme zapatos.’   < SFH 08 1:47/el >\\
    }\label{ex: verbalizer -ta make/becomeb}
            \ex[]{
            \textit{sipuˈtâa    tʃ͡uˈkú}\\
            \gll    sipu-ˈtâ-a    tʃ͡uˈkú\\
                    skirt-\textsc{vblz-prog}  bend \\
            \glt    ‘(She is) putting a skirt.’ \\
            \glt    ‘Se está poniendo la falda.’   < BFL 07 Sept 6/el >\\
        }\label{ex: verbalizer -ta make/becomec}
    \z
\z

Further examples of this denominal constructions are shown in (\ref{ex: more examples of verbalizer -ta}).

\ea\label{ex: more examples of verbalizer -ta}
{Denominal verbs}

\begin{tabular}{lllll}
     a. & {boʔi-ˈrâ}&{‘to make a road’}&{boˈʔí} {‘road’}& {< BFL 06 5:128/el >}\\
     b. &{wari-ˈrâ}&{‘to make a basket’}&{waˈrî ‘basket’}&  {< BFL 06 5:129/el >}\\
     c. & {sapa-ˈrâ}&{‘to fatten’}&{saʔˈpá ‘flesh’}& {< BFL 06 5:129/el >}\\
     d. & {kali-ˈrâ}&{`to make house'}&{kaˈlí ‘house’}&{< BFL 06 5:127/el >}\\
\end{tabular}

\z

When this suffix is added to trisyllabic nouns, the last syllable of the noun is truncated in order to meet the requirement of this construction to have stress on the denominalizing suffix (\ref{ex: noun truncation in verbalizing constructions}) (the truncated syllable of the noun is highlighted in boldface in the phonemic representation with morpheme breaks).

\ea\label{ex: noun truncation in verbalizing constructions}
{Noun truncation in verbalizing constructions with the \textit{-ta} suffix}

    {\textit{ˈhê    ˈnáni     siˈpútʃ͡a   sipuˈtâmo    ˈlá}}\\
    \gll    ˈhê   na=ni   siˈpútʃ͡a  sipu\textbf{tʃa}-ˈtâ-ma  oˈlá\\
            \textsc{dem}  \textsc{prox}=1\textsc{sg.nom}  skirt  skirt-\textsc{vblz-fut.sg}  \textsc{cer}\\
    \glt    ‘I will wear this skirt.’\\
    \glt    ‘Me voy a poner esta falda.’  < BFL 07 Sept 6/el >\\

\z

This truncation process resembles the truncation that trisyllabic nominal roots undergo in incorporation constructions (see §\ref{subsec: body-part incorporation} above). Discussion of morpho\-logically-conditioned truncation and templatic effects in Choguita Rarámuri is further discussed in §\ref{sec: prosodic constraints on morphological shapes}.

\subsubsection{The reversive suffix \textit{-bû}}
\label{subsubsec: the reversive -bu suffix}

The suffix \textit{-bû}, on the other hand, has a ``reversive'' meaning and it derives verbs from nouns is a non-productive affix that has the meaning ‘remove’ or ‘undo’.\footnote{A cognate suffix can be found in \ili{Guarijío}, and Miller identifies the verb \textit{puha} ‘to take away’, as its source (\citeyear[151]{miller1996guarijio}). The Rarámuri cognate of this verb is \textit{bu(ʔ)è} ‘to take away’.} Some examples of this derivational suffix are provided in (\ref{ex: reversive -bu suffix}).

\ea\label{ex: reversive -bu suffix}

    \ea[]{
    \textit{niˈhê  toˈlí    boʔoˈbûma}\\
    \gll    niˈhê  toˈlí    boʔo-ˈbû-ma\\
            1\textsc{sg.nom}  chicken  feather-\textsc{rev-fut.sg}\\
    \glt    ‘I will pluck the chicken.’\\
    \glt     ‘Voy a desplumar la gallina.’ < SFH 08 1:51/el >  \\
}
        \ex[]{
        \textit{paˈtʃ͡á   aˈti   muˈnî   riʔiˈbûa}\\
        \gll    paˈtʃ͡á   aˈti   muˈnî   riʔi-ˈbû-a\\
                inside  sitting  beans  stone-\textsc{rev-prog}\\
        \glt    ‘He is sitting inside taking out stones from beans.’\\
        \glt    ‘Está (sentado) adentro limpiando frijol (quitándole las piedras).’  < SFH 08 1:51/el >\\
    }
            \ex[]{
            \textit{ˈmá   tʃ͡omoˈbûka}\\
            \gll    ˈmá    tʃ͡omo-ˈbû-ka!\\
                    already    mucus-\textsc{rev-imp.sg}\\
            \glt    ‘Clean your nose already!’  \\
            \glt     ‘Ya límpiate la nariz!’ < SFH 08 1:51/el >\\
        }
  \z
\z

\subsubsection{The `gather' suffix \textit{-tú}}
\label{subsubsec: the verbalizer -tu suffix}

The suffix \textit{-tú (-rú)}, which means ‘to gather’ or ‘to bring’, is another derivational suffix that attaches to nominal roots to form verbs. It is an unproductive suffix restricted to attach to only a few nouns. It is always stressed, it occurs next to the nominal root. As part of the Inner Stem, it is sensitive to the morpho-phonological effects that characterize this stem domain, including compensatory lengthening. Some examples are given in (\ref{ex: verbalizer -tu suffix}).

\ea\label{ex: verbalizer -tu suffix}

    \ea[]{
    \textit{niˈhê   baʔwiˈtúma}\\
    \gll    niˈhê   baʔwi-ˈtú-ma\\
            1\textsc{sg.nom}   water-gather-\textsc{fut.sg}\\
    \glt    ‘I will bring water.’\\
    \glt    ‘Voy a traer agua.’\\
}
        \ex[]{
        \textit{mati     ilaˈrúpo}\\
        \gll    ma=ti     ila-ˈrú-po\\
                now=1\textsc{pl.nom} cactus-gather-\textsc{fut.pl}\\
        \glt    ‘Let’s gather cactus now.’\\
        \glt    ‘Vamos juntando nopales.’   < SFH 08 1:52/el >\\
    }
            \ex[]{
            \textit{mi  riʔˈri  naʔiˈrúmisa}\\
            \gll    mi  riʔˈri  naʔi-ˈrú-mi-sa\\
                    \textsc{dem}  there  fire-gather-\textsc{mot.imp-imp.sg}\\
            \glt    ‘Go get fire over there!’\\
            \glt    ‘¡Ve a traer lumbre allá!’  < SFH 08 1:52/el >\\
        }
                \ex[]{
                \textit{ˈhípi  oˈmêatʃ͡i    rakiˈrúpa ˈlá}\\
                \gll    ˈhípi  oˈmêatʃ͡i    rakiˈ-rú-pa oˈlá\\
                        today  Sunday    palm-gather-\textsc{fut.pass}  \textsc{cer}\\
                \glt    ‘Palms will be gathered today, Sunday.’\\
                \glt    ‘Hoy domingo van a recibir la palma.’ (lit. ‘las palmas serán juntadas’)  < SFH 08 1:52/el >\\
            }
    \z
\z

\subsubsection{The `have' \textit{-ê} suffix}
\label{subsubsec: the verbalizer -e suffix}

The \textit{-ê} suffix is a derivational suffix that may be characterized as a ``denominal verb of possession''. I follow \citet{haugen2017derived} in defining ``denominal verb of possession'' as ``a construction where an identifiable noun root (or stem) N can be converted into a verb by the [affixation] ... of some affix X such that the N+X complex is treated as a regular verb, able to be inflected with markers of tense-aspect-mood and other categories, and having the meaning of ‘to have N’'' (\citeyear[164]{haugen2017derived}).

The \textit{-ê} suffix, which has the general meaning of ‘to have’ or ‘to wear’, is a replacive suffix. It targets the nominal stem final vowel, and it is always stressed.\footnote{The cognate of this suffix (\textit{-e}) in \ili{Mountain Guarijío} is described in \citet{miller1996guarijio} as exhibiting different prosodic properties, yielding variable stress patterns in the morphologically complex words where it is attested (with stress on the stem or the suffix) and not replacing the final stem vowel (e.g., \textit{puhkú-e} \textasciitilde \textit{puhku-é} `to have an animal' (\citeyear[149]{miller1996guarijio})).} The examples in (\ref{ex: verbalizer suffix -e}) show this derivation.\footnote{For discussion of verbless possessive constructions in \ili{Yaqui} (\ili{Taracahitan}), see \citet{jelinek1988verbless}.}


\ea\label{ex: verbalizer suffix -e}
{Verbalizer suffix \textit{-ê}}
\setlength{\tabcolsep}{4pt}
\begin{tabular}{lllll}
    & \textit{Verb stem} & & \textit{Noun stem} & \\
     a. &  {kuˈn-ê-} & {‘to marry a man’}& {kuˈnà-} & {`husband'}\\
    & & & & {< BFL 05 1:116/el >}\\
     b. & {uˈp-ê-}&{‘to marry a woman’}& {uˈpî-} & {‘wife’}\\
     & & &   & {< BFL 05 1:116/el >}\\
     c. & aˈʰk-ê & `to wear sandals' & {aˈʰkà}& {‘sandal'}\\
         & & &  &   {< BFL 05 1:116/el >}\\
     d. & {wiˈr-ê} & {‘to wear earrings’}&{wiˈrá} &  {‘earring’}\\
            & & & &     {< BFL 07 sept 6/el >}\\
     e. & {motoˈs-ê}&{‘to have white hair’}&{motoˈsá} & { ‘white hair’}\\
          & & & & {< BFL 07 sept 6/el >}\\
\end{tabular}
\z

As shown in these examples, while the suffix is always stressed, there are no homogeneous tonal properties that the morphologically complex verbs containing this verb share (i.e., each of the three lexical tones is attested in these cases, in what appears to be a lexically conditioned pattern).

Further examples of this construction are shown in (\ref{ex: further examples of verbalizer -e}).

\ea\label{ex: further examples of verbalizer -e}
{Verbalizer suffix \textit{-ê}: further examples}
\setlength{\tabcolsep}{3pt}
\begin{tabular}{lllll}
      & \textit{Verb stem} & \textit{Gloss} & \textit{Base Noun} & \textit{Gloss} \\
    a.& kaˈl-ê & {‘to have a house’}&{kaˈlí }&{‘house’}\\
       & & & & {< BFL 06 5:127/el >}\\
    b. & kaˈw-ê & {‘to have eggs’}&{kaˈwá}&{‘egg’}\\
    & & & &    {< BFL 06 5:127/el >}\\
    c. & {waˈs-ê}&{‘to have mother-in-law’}&{waˈsí}&{‘mother-in-law’}\\
    & & & &   {< BFL 06 5:127/el >}\\
    d.&{roˈn-ê}&{‘to have feet'}&{roˈnô}&{ ‘feet’}\\
    & & & &   {< BFL 06 5:127/el >}\\
    e.& {kaˈs-ê} &{‘to have legs’}&{kaˈsî}& {‘legs’}\\
    & & & &     {< BFL 06 5:127/el >}\\
    f. & {siˈk-ê}&{‘to have hands, arms’}&{siˈkâ}& {‘hands, arms’}\\
    & & & &    {< BFL 06 5:128/el >}\\
    g. & {rihiˈm-ê}&{‘to have relatives’}&{rihiˈmá}& {‘relatives’}\\
    & & & &   {< BFL 06 5:128/el >}\\
    h. & {aˈʰk-ê}&{‘to have sandals'}&{aˈʰkà}& {‘sandals'}\\
    & & & &{< BFL 06 5:128/el >}\\
    i. & {koroˈk-ê}&{‘to have a necklace’}&{koroˈká}& {‘necklace’}\\
    & & & & {< BFL 06 5:129/el >}\\
\end{tabular}
    \z

This construction is highly productive. There are no apparent restrictions in terms of the semantic properties of the nouns that can undergo this derivation.

\subsubsection{The verbalizer \textit{-wi} suffix}

In addition to the verbalizer \textit{-ê} suffix, some nominal stems attach the neutral \textit{-wi} suffix to derive a verb with the meaning `to own/have N'. Examples of this construction are shown in (\ref{ex: Verbalizer suffix -wi}).

\ea\label{ex: Verbalizer suffix -wi}
{Verbalizer suffix \textit{-wi}}
\setlength{\tabcolsep}{3pt}
\begin{tabular}{lllll}
    & \textit{Verb stem} & \textit{Gloss} & \textit{Noun stem} & \textit{Gloss} \\
    a. & {winoˈmí-wi}&{‘to have money’}&{winoˈmí}& {‘money’}\\
    & & & & {< BFL 06 5:127/el >}\\
    b. & {waˈsá-wi}&{‘to have land’}&{waˈsá}& {‘land for cultivation’}\\
        & & & & {< BFL 06 5:127/el >}\\
    c.&{ˈkútʃ͡u-wi}&{‘to have children’}&{ˈkútʃ͡i}& {‘little ones’}\\
    & & & & {< BFL 06 5:127/el >}\\
    d.&{suˈnù-wi}&{‘to have corn’}&{suˈnù}&{`corn'}\\
     & & & & {< BFLL 06 5:127/el >}\\
    e.&{muˈnî-wi}&{‘to have beans’}&{muˈnî}&{‘beans’}\\
     & & & & {< BFL 06 5:127/el >}\\
    f.&{rihiˈmá-wi}&{‘to have relatives’}&{rihiˈmá}& {‘relatives’}\\
    & & & & {< BFL 06 5:128/el >}\\
    g.&{koˈbîsu-wi}&{‘to have pinole’}&{koˈbîsi}& {‘pinole’}\\
    & & & & {< BFL 06 5:128/el >}\\
    h.& {aˈʰkà-wi}&{‘to have sandals’}&{aˈʰkà}& {‘sandals’}\\
     & & & &{< BFL 06 5:128/el >}\\
    h.&{ˈpúru-wi}&{‘to have a knitted }&{ˈpúra}& {‘knitted belt‘}\\
     & & belt’ & &{< BFL 06 5:128/el >}\\
    i.&{koˈmâru-wi}&{‘to have a comadre’}&{koˈmâare}& {‘comadre’}\\
     & & & &{< BFL 06 5:128/el >}\\
\end{tabular}
    \z

This construction is also highly productive in the language.

A potentially related morphological strategy to derive denoninal verbs with the meaning `have' or `own' is through the suffixation of \textit{-i}, which replaces the final vowel of the stem, which may be a reduced form of the verbalizer \textit{-wi} suffix. Relevant examples are shown in (\ref{ex: denominal verbs with -i}).


\ea\label{ex: denominal verbs with -i}
{Verbalizer suffix \textit{-i}}
\setlength{\tabcolsep}{3pt}
\begin{tabular}{lllll}
       & \textit{Verb stem} & \textit{Gloss} & \textit{Noun stem} & \textit{Gloss} \\
     a.& siˈpútʃ͡-i&{‘to have a skirt'}&{siˈpútʃ͡a}&{‘skirt' < BFL 06 5:128/el >}\\
     b.& naˈpátʃ͡-i&{‘to have a blouse'}&{naˈpátʃ͡a}& {‘blouse'}{< BFL 06 5:128/el >}\\
\end{tabular}
    \z

The verbalizer \textit{-wi} suffix and other verbalizing morphological markers may be available with the same nominal stems. In some cases, the choice of marker involves a semantic difference. This is illustrated with the minimal pair in (\ref{ex: multiple verbalizers minimal pairs}) where choice of marker entails a different semantic interpretation.

\newpage
\ea\label{ex: multiple verbalizers minimal pairs}
{Semantic differences of denominal verbs}

    \ea[]{
    \textit{saʔˈpê}\\
    \textit{saʔˈp-ê}\\
    meat-\textsc{vblz}\\
    ‘to have flesh (one’s own)’\\
    `tener carne (del cuerpo)' {< BFL 06 5:129/el >}\\
}\label{ex: multiple verbalizers minimal pairsa}
        \ex[]{
        \textit{saʔˈpáwi}\\
        \textit{saʔˈpá-wi}\\
        meat-\textsc{vblz}\\
        ‘to have (e.g. cow’s) meat’\\
        `tener carne (e.g., de res)' {< BFL 06 5:129/el >}\\
    }\label{ex: multiple verbalizers minimal pairsb}
            \ex[]{
            {cf. \textit{saʔˈpá}}\\
            {‘flesh’}\\
            {< BFL 06 5:129/el >}\\
        }\label{ex: multiple verbalizers minimal pairsc}
    \z
\z

As these examples show, the choice of verbalizing suffix (\textit{-ê} in (\ref{ex: multiple verbalizers minimal pairsa}) and \textit{-wi} in (\ref{ex: multiple verbalizers minimal pairsb})) may involve a semantic distinction. Specifically, the choice of verbalizer in this case entails a distinction in terms of an inalienable reading vs. an alienable one, respectively (for more discussion of alienable/inalienable distinctions and their morphological expression in Choguita Rarámuri, see §\ref{subsec: possessive}). However, semantic distinctions of this kind are not widespread, which suggests that these derived forms have undergone some degree of lexicalization.

Finally, there are a few constructions that may also encode the meaning `to own/have N' through a periphrastic construction, in addition to a suffixing construction. This periphrastic construction is exemplified in (\ref{ex: auxiliary construction for 'own'}):

\ea\label{ex: auxiliary construction for 'own'}

    \ea[]{
    \gll    waˈsá ˈníwi\\
            cultivation.land have \\
    \glt    `to have land'  \\
    \glt    `tener tierra de cultivo'\\
}
        \ex[]{
        \gll    ˈsôda ˈníwi\\
                soda have\\
        \glt    `to have soda'\\
        \glt    `tener soda'\\
    }
    \z
\z

As shown in these examples, the construction involves a noun followed by the verb \textit{ˈníwi} ‘to have’. This auxiliary verb is the likely source of the \textit{-wi} verbalizer suffix (the language also has a related appositive possessive construction, described in §\ref{sec: appositive possessive constructions and relative clauses} below).

\subsubsection{The verbalizer \textit{-pi} suffix}
\label{subsubsec: the verbalizer -pi suffix}

Another verbalizer suffixing construction is a derivational suffix, \textit{-pi}, which means `to remove'. This suffix is exemplified in (\ref{ex: remove derivation}).

\ea\label{ex: remove derivation}
{Verbalizer \textit{-pi} suffix}
\setlength{\tabcolsep}{3pt}
\begin{tabular}{lllll}
      & \textit{Form} & \textit{Gloss} & \textit{Noun stem} & \textit{Gloss} \\
    a.& sipuˈtʃ͡â-pi&{‘to remove skirt’}&{siˈpútʃ͡a}& {‘skirt'}\\
    & & & & {< BFL 06 5:128/el >}\\
    b.&{napaˈtʃ͡â-pi}&{‘to remove blouse'}&{naˈpátʃ͡a}& {‘blouse'}\\
     & & & & {< BFL 06 5:128/el >}\\
    c.&{puˈrâ-pi}&{‘to remove knitted belt’}&{ˈpúra}& {‘knitted belt’}\\
      & & & &{< BFL 06 5:128/el >}\\
    d.&{tʃ͡aˈbó-pi}&{‘to remove beard'}&{tʃ͡aˈbó}& {‘beard’}\\
\end{tabular}
    \z

This suffix is unproductive in Choguita Rarámuri.

\subsubsection{Non-concatenative verbalizing constructions}
\label{subsubsec: non-concatenative verbalizing constructions}

There are a few cases where denominal verbs meaning ‘make/wear N’ are derived through a stress shift. Specifically, the derived verb will have stress one syllable to the right with respect to the stress location in the related nominal base. This is shown in the example in (\ref{ex: denominal stress shift}):

\ea\label{ex: denominal stress shift}
{Derivational stress shift}

    \ea[]{
    \doublebox{napaˈtʃ͡â}{‘to wear a shirt’}\\
}
        \ex[]{
        \doublebox{naˈpátʃ͡a}{‘shirt'}\\
    }
    {< BFL 06 5:128/el >} \\
    \z
\z

Denominalization may also involve truncation of the last, unstressed syllable of the base noun even with no overt attachment of any suffix. This is shown in the examples in (\ref{ex: denominal truncation}):

\ea\label{ex: denominal truncation}
{Stem truncation in denominalization}

\begin{tabular}{lllll}
      & \textit{Verb} & \textit{Gloss} & \textit{Noun} & \textit{Gloss} \\
    a.& {siˈpú}&{‘to wear skirt’}& {siˈpútʃ͡a}& {‘skirt'}\\
    & & & & {< BFL 06 5:128/el >}\\
    b.& {pú}&{‘to wear a knitted belt’}& {ˈpúra} &‘{knitted belt’} \\
    & & & & {< BFL 06 5:128/el >}\\
\end{tabular}
\z

These non-concatenative processes to derive verbs from nouns are unproductive in the language.

\subsection{Summary}
\label{subsec:9:summary}

Verbal roots in Choguita Rarámuri may undergo semi-productive and unproductive processes, both concatenative and non-concatenative, before adding any further suffixes. These processes include conversion, pluractional consonant mutation, stress shifts and grammatical tone to mark imperative or verbalization or body-part incorporation. Having described root classes and the processes taking place at the innermost level of the verbal stem, I now turn to the suffixation domain.

\section{Verbal structure and verbal domains}
\label{sec: verbal structure and verbal domains}

\subsection{Overview}
\label{subsec: overview}

The suffix positions and categories expressed in the Choguita Rarámuri verbal structure are schematized in \tabref{tab:suffix-positions}.

\begin{table}
\caption{Suffix positions and categories of the Choguita Rarámuri verb}
\label{tab:suffix-positions}

\begin{tabularx}{.65\textwidth}{lll}
\lsptoprule
\textbf{Position} & \textbf{Type} & \textbf{Categories}\\
\midrule
S1 & Derivation & Inchoative\\
S2 & Derivation & Transitive\\
S3 & Derivation & Applicative\\
S4 & Derivation & Causative\\
S5 & Derivation & Applicative\\
S6 & Modality & Desiderative\\
S7 & Derivation & Associated Motion\\
S8 & Modality & Auditory Evidential\\
S9 & Inflection & Voice/Aspect/Tense\\
S10 & Inflection & Mood\\
S11 & Inflection & TAM\\
S12 & Subordination & Deverbal morphology\\
\lspbottomrule
\end{tabularx}
\end{table}

The suffixes in each position do not generally co-occur in the same word, due to their semantic incompatibility (though there are systematic exceptions; these will be discussed in this chapter and the rest of this grammar).
A summary of the verbal suffixes and their position in the verbal template is given in \tabref{tab:key:15}. A basic description and examples of the suffixes can be found in Appendix \ref{chap: verbal suffixes}. The ``Reference'' column in \tabref{tab:key:15} refers to the section where each individual suffix is described in this Appendix.

\begin{table}
\caption{Choguita Rarámuri verbal suffixes}
\label{tab:key:15}
\small
\begin{tabularx}{\textwidth}{lp{.2\textwidth}Ql}
\lsptoprule
& \textbf{Category} & \textbf{Suffix} & \textbf{Reference}\\
\midrule
 \textbf{S1} & Inchoative & Inchoative  \textit{-bá} (\textsc{inch}) & §\ref{subsec: inchoative}\\
 \textbf{S2} & Transitives & Transitive  \textit{-nâ} (\textsc{tr}) & §\ref{subsec: transitive -na}\\
 &  & Pluractional transitive  \textit{-tʃ͡a} (\textsc{tr.pl}) & §\ref{subsec: pluractional transitive}\\
 &  & Transitive  \textit{-bû} (\textsc{tr}) & §\ref{subsec: transitive -bû}\\
 \textbf{S3} & Applicatives & Applicative  \textit{-ni} (\textsc{appl}) & §\ref{subsubsec: applicative ni}\\
&  & Applicative  \textit{-si} (\textsc{appl}) & §\ref{subsubsec: applicative si}\\
&  & Applicative  \textit{-wi} (\textsc{appl}) & §\ref{subsubsec: applicative wi}\\
 \textbf{S4} & Causative & Causative  \textit{-ti} (\textsc{caus}) & §\ref{subsec: causative}\\
 \textbf{S5} & Applicative & Applicative  \textit{-ki} (\textsc{appl}) & §\ref{subsec: applicative ki}\\
 \textbf{S6} & Desiderative & Desiderative  \textit{-nále} (\textsc{desid}) & §\ref{subsec: desiderative}\\
 \textbf{S7} & A. Motion & Associated Motion  \textit{-simi} (\textsc{mot}) & §\ref{subsec: associated motion}\\
 \textbf{S8} & A. Evidential & Auditory Evidential  \textit{-tʃ͡ane} (\textsc{ev}) & §\ref{subsec: auditory evidential}\\
 \textbf{S9} & Tense, Aspect, & Past Passive  \textit{{}-ru} (\textsc{pst.pass}) & §\ref{subsubsec: past passive}\\
& Mood, Voice & Future Passive  \textit{{}-pa} (\textsc{fut.pass}) & §\ref{subsubsec: future passive}\\
&  & Medio-Passive  \textit{-rîwa, -wá} (\textsc{mpass}) & §\ref{subsubsec: medio-passive}\\
&  & Conditional Passive \textit{-sûwa} (\textsc{cond.pass}) & §\ref{subsubsec: conditional passive}\\
&  & Future Sg.  \textit{-ˈmêa, -ma} (\textsc{fut.sg}) & §\ref{subsubsec: future singular}\\
&  & Future Pl.  \textit{-pô} (\textsc{fut.pl}) & §\ref{subsubsec: future plural}\\
&  & Motion Imperative  \textit{-mê} (\textsc{mot.imp}) & §\ref{subsec: motion imperative}\\
&  & Conditional  \textit{-sâ} (\textsc{cond}) & §\ref{subsec: conditional}\\
&  & Irrealis sg.  \textit{-mê} (\textsc{irr.sg}) & §\ref{subsubsec: irrealis singular}\\
&  & Irrealis pl.  \textit{-pi} (\textsc{irr.pl}) & §\ref{subsubsec: irrealis plural}\\
 \textbf{S10} & Mood & Potential  \textit{-râ} (\textsc{pot}) & §\ref{subsec: potential}\\
&  & Imperative sg.  \textit{-kâ} (\textsc{imp.sg})  & §\ref{subsubsec: imperative singular ka}\\
&  & Imperative sg.  \textit{-sâ} (\textsc{imp.sg})  & §\ref{subsubsec: imperative singular sa}\\
&  & Imperative pl.  \textit{-sì} (\textsc{imp.pl}) & §\ref{subsubsec: imperative plural}\\
 \textbf{S11} & Tense, Aspect, & Reportative different subj.  -\textit{la} (\textsc{rep.ds}) & §\ref{subsubsec: reportative different subject}\\
& Mood & { Reportative same subj. \textit{-lo} (\textsc{rep.ss})} & §\ref{subsubsec: reportative same subject}\\
&  & Past \textit{-li} (\textsc{pst}) &  §\ref{subsec: past}\\
&  & Past perfective egophoric \textit{-ki} (\textsc{pst.ego}) & §\ref{subsec: past egophoric}\\
&  & Imperfective  \textit{-e} (\textsc{impf}) & §\ref{subsec: imperfective}\\
&  & Progressive  \textit{-a} (\textsc{prog}) & §\ref{subsec: progressive}\\
&  & Indirect causative  \textit{-nula}  & §\ref{subsec: indirect causative}\\
 \textbf{S12} & Subord. & Temporal  \textit{{}-tʃ͡i} (\textsc{temp}) & §\ref{subsec: temporal}\\
&  & Epistemic  \textit{-o} (\textsc{ep}) & §\ref{subsec: epistemic}\\
&  & Gerund  \textit{-ká} (\textsc{ger}) & §\ref{subsec: gerund}\\
&  & Purposive  \textit{-ra} (\textsc{pur}) & §\ref{subsec: purposive}\\
&  & Participial  \textit{-ame} (\textsc{ptcp}) & §\ref{subsec: participial}\\
\lspbottomrule
\end{tabularx}
\end{table}
%\textbf{Bickel \& Nichols:}

%% Section below from Bickel and Nichols - for a description of how the conjunct person works in CR

%While the distinction between first and second person as indexes to the speaker and addressee, respectively, is the most common type worldwide, recent research has established that this is not the only one possible. A few languages in Asia and South America have grammaticalized a completely different categorization, at least in verb agreement. One person, usually labeled ‘CONJUNCT’,\textsuperscript{} refers to the speaker in statements and to the addressee in questions (excluding rhetorical questions, which are really statements in function). Thus, the conjunct person form \textit{won}—å in Newar, the Tibeto- Burman language of the \ili{Nepalese} capital Kathmandu, can mean ‘I went’ or ‘did you go?’. This is in opposition to what is called a DISJUNCT form, \textit{wona}, which is used for all other situations, i.e. meaning ‘you went’ or ‘s/he went’ or ‘did s/he go?’ or, where this makes sense in context, ‘did I go?’. What is at the functional core of the conjunct person category is the indexing of the EPISTEMIC AUTHORITY, i.e. the person who the speaker supposes or claims to have direct and personal knowledge of the situation (\citealt{Hargreaves1990}, 1991). In statements the epistemic authority is the speaker if he or she is a participant of the situation; in questions it is the addressee if he or she plays a role in the situation. If the epistemic authority plays no role in the situation, the form is coded as disjunct.

%\textsuperscript{22} The term [conjunct] is from A. Hale’s (1980) pioneering description of the phenomenon in Newar. The less then ideally transparent terminology derives from the use of conjunct forms in reported speech where the form marks coreference (referential ‘conjunction’) of the subject with the speaker referent reported in the matrix clause (i.e. it has the same effect as a logophoric marker, on which see \sectref{sec:key:9.1.4}). Alternative terms found in the literature are ‘locutor’, ‘egophoric’, ‘subjective’, and ‘congruent’; cf. \citet{Curnow2000}.

%add references n evidentiality, including paper by:

%San Roque, Lila, Simeon Floyd \& Elisabeth Norcliffe. Evidentiality and interrogativity. Lingua 186-187, January-\citealt{Februart2017}, pp. 12-143.

%Article that focuses on understanding the behavior of evidentials in interrogative contexts, and role of representing addressee perspective (vs. a speaker-centric view of evidentials)

This verbal structure does not imply a slot-and-filler, template-like structure, i.e. the labels S1\ldots S12 are not intended to imply a flat structure as in a slot matrix. In the verbal morphology of this language, morphotactic, and morphophonological processes define a hierarchical structure of the verb, with suffixes closer to the Inner Stem displaying less salient morpheme junctures (given by phonological transparency and productivity). In the structure represented in \tabref{tab:verb-stem-levels}, I identify five verbal zones after the Inner Stem domain: a Derived Stem, a Syntactic Stem, an Aspectual Stem, a Finite Verb domain, and finally a Subordinate Verb domain.

\begin{table}
\caption{Choguita Rarámuri verbal stem domains}
\label{tab:verb-stem-levels}

\begin{tabularx}{\textwidth}{lQl}
\lsptoprule
\textbf{Position}  & \textbf{Marker}  & \textbf{Stem domain} \\
\midrule
{} & Pluractionality, number, Verbalization, etc. & Inner Stem\\
S1 & Inchoative & Derived Stem \\
S2 & Transitives & \\
S3 & Applicatives & Syntactic Stem\\
S4 & Causative & \\
S5 & Applicative & \\
S6 & Desiderative & Aspectual Stem\\
S7 & Associated Motion & \\
S8 & Auditory Evidential & \\
S9 & Voice/Aspect/Tense & Finite Verb\\
S10 & Mood & \\
S11 & TAM & \\
%\hhline{~~-}
S12 & Deverbal morphology & Subordinate Verb\\
\lspbottomrule
\end{tabularx}
\end{table}
%\hspace{3cm}

The first identifiable layer in the suffixation domain is the Derived Stem. This layer of the verbal stem includes two kind of semantically restricted, unproductive derivational suffixes (an inchoative suffix and three transitive suffixes). These suffixes are restricted to a semantically defined class of verbs, change-of-state verbs. There are three morphological and morpho-phonological criteria that allow identifying this stem domain as a defined sub-constituent of the Choguita Rarámuri verb: first, the non-concatenative imperative singular (described in §\ref{subsubsec: imperative final stem stress}) is marked as final stress of this stem domain; second, this level is also the domain of the passive-induced lengthening (discussed below); finally, the Derived Stem undergoes the stress shift that characterizes unstressed stems when combined with shifting suffixes (addressed in §\ref{sec: stress properties of roots, stems and suffixes}).

The Derived Stem is the input to the next stage of the construction of the Choguita Rarámuri verb, the Syntactic Stem. This next stem domain includes suffixes in S3--S5, suffixes that mark valence-increasing operations. Within this stem domain, suffixes are attested in variable order and display multiple (or extended) exponence. The suffixes in this level also form a coherent domain within the Choguita Rarámuri verb in morpho-phonological terms: these suffixes are stress-neutral, forming a small pocket of unstressable suffixes within a larger, stressable domain. Finally, this stem domain is part of the domain of round harmony, as defined in §\ref{subsec: phonological transparency and morpheme boundary strenght}.

Another layer of the verbal stem is the Aspectual Stem, composed of suffixes in positions S6 to S9, marking desiderative, associated motion, and auditory evidential. These suffixes are formally and semantically related to independent verb forms in the language. \tabref{tab:aspectual-suffixes} lists these suffixes, their grammaticalized meanings and their independent lexical verb sources.

\begin{table}
\caption{Choguita Rarámuri aspectual suffixes and their lexical counterparts}
\label{tab:aspectual-suffixes}

\begin{tabularx}{\textwidth}{XXXXX}
\lsptoprule

\textbf{Aspectual suffixes} & \textbf{Independent lexical verb}\\
\midrule
\textit{-nále}	`desiderative (\textsc{desid})' & \textit{nále} `want'\\
\textit{-simi} `associated motion (\textsc{mot})' & \textit{simi} `go.\textsc{sg}'\\
\textit{-tʃ͡ane} `auditory evidential (\textsc{ev})' & \textit{(a)tʃ͡ane} `say, make noise'\\
\lspbottomrule
\end{tabularx}
\end{table}
%\hspace{3cm}

\hspace*{-2.4pt}These aspectual markers exhibit the same properties as constructions described for other \ili{Uto-Aztecan} languages as ``secondary verb'' constructions, where grammaticalized formatives derived from independent verbs that encode aspectual-like or adverbial-like meanings (\citealt{crapo1970origins}), a type of V-V incorporation construction analyzed as involving light verbs in \ili{Hiaki} (\ili{Taracahitan}; \citealt{tubino2014affixal}) and serialization in \ili{Northern Paiute} (\ili{Numic}; \citealt{thornes2011dimensions}) (the syntactic properties of these constructions are discussed in §\ref{subsec: V-V incorporation constructions}). These aspectual suffixes in Choguita Rarámuri are disyllabic and have monosyllabic allomorphs and they are integrated prosodically with the stem in a single phonological word. The phonological factors determining the distribution of dysillabic and monosyllabic allomorphs of these suffixes are discussed in \chapref{chap: prosody} (§\ref{sec: prosodic constraints on morphological shapes}). These suffixes are also part of the domain for round harmony and constitute the last layer where this process applies (that is, suffixes to the right of this domain are not targets for spreading of the harmony).

The final stage in constructing a Choguita Rarámuri inflected verb consists in adding the suffixes in positions S9--S11 in the finite verb, the Finite Verb level suffixes. The grammatical categories marked at this level are mood distinctions (including imperative and reportative), voice, tense, and aspect (and number and person marginally), conflated in portmanteaux suffixes.  In this domain there are inflectional affixes that produce an idiosyncratic meaning when combined.

Finally, a finite verb can be the input for another, optional layer of morphology, in order to be used in subordinate clause constructions. These suffixes, in position S12, are the last stage of affixation of the Choguita Rarámuri verb, and are stress neutral. They produce nominalizations, and are at the word boundary.

\tabref{tab:key:17} summarizes the linear order facts, and the semantic, morphophonological and prosodic properties of affixes that have motivated the verbal zones proposed for the Choguita Rarámuri verb.

%%please move \begin{table} just above \begin{tabular .
\begin{table}
\caption{Characteristics of the Choguita Rarámuri verb}
\label{tab:key:17}
\small
\begin{tabularx}{\textwidth}{>{\raggedright\arraybackslash}p{.1\textwidth}QlQl}
\lsptoprule

\textbf{Stem domain} & \textbf{Categories expressed} & \textbf{Morpho}\textbf{tactics} & \textbf{Phonology} & \makecell[tl]{\textbf{Stress}\\\textbf{properties}}\\
\midrule
Inner Stem & Body part incorporation,

pluractional,

number prefixes,

verbalization & Fixed order & Haplology,

compensatory lengthening,

passive length,

imperative stress,

round harmony & Shifting\\
\tablevspace
Derived Stem & Inchoative

Transitive & Fixed order & Passive length,

imperative stress,

round harmony & Shifting\\
\tablevspace
Syntactic Stem & applicative

Causative & \makecell[tl]{Variable order,\\Multiple\\exponence} & Round harmony & Neutral\\
\tablevspace
Aspectual Stem & Desiderative

Ass. Motion

Evidential & Variable order & Round harmony,

short allomorphs & \makecell[tl]{Shifting /\\Neutral}\\
\tablevspace
Finite Verb & Voice

TAM

Indirect Causative & Fixed order &  & \makecell[tl]{Neutral\\(except S10)}\\
\tablevspace
Subord. Verb & Nominalizat.,

subordination & Fixed order &  & Neutral\\
\lspbottomrule
\end{tabularx}
\end{table}

The evidence for the proposed structure will be laid out as follows. In §\ref{subsec: morphotactic evidence for affix order generalizations}, I present the morphotactic evidence for positing the positions in the verbal template, and in §\ref{subsec: phonological transparency and morpheme boundary strenght}, I give the morpho-phonological evidence for positing different verbal domains.

\subsection{Morphotactic evidence for affix ordering generalizations}
\label{subsec: morphotactic evidence for affix order generalizations}
\largerpage
This section provides the morphotactic evidence for positing the suffix positions or slots in the verbal template. This evidence comes from the linear ordering properties of the suffixes, as well as their exponence and permutation possibilities. The evidence is presented progressively describing the positions from the Inner Stem towards the outer layers of affixation.

The positions closest to the Inner Stem in the verbal template are occupied by suffixes that are only used with change-of-state predicates: the inchoative suffix (in S1) and a set of transitive suffixes in (S2). Their ordering is illustrated in (\ref{ex: S1 - S2  affix order}).

\ea\label{ex: S1 - S2  affix order}
{\textsc{inch} (S1) -- \textsc{tr.pl} (S2)}

    \ea[]{
    \textit{ˈmán rataˈbátʃ͡iki         koʔˈwáami}\\
    \gll    ˈmá=ni rata\textbf{-ˈbá-}\textbf{tʃa}-ki         koʔˈwá-ame\\
            already=1\textsc{sg.nom} heat{\textsc{-inch-tr.pl}}-\textsc{pst.ego}  eat-\textsc{ptcp}\\
    \glt    ‘I already heated up the food.’  \\
    \glt    ‘Ya calenté la comida.’   < BFL 08 1:20/el >\\
}
        \ex[]{
        \textit{muˈhê  muˈnî  ˈmá    tʃ͡okoˈbánali}\\
        \gll    muˈhê  muˈnî  ˈmá   tʃ͡oko{{}-ˈbá-na}{}-li\\
                 2\textsc{sg.nom}  beans  already    be.sour{{}-\textsc{inch-tr}}{}-\textsc{pst}\\
        \glt    ‘You already made the beans go sour.’\\
        \glt    ‘Ya hiciste que se agriaran los frijoles.’ < SFH 04 1:113/el > \\
    }
    \newpage
            \ex[]{
            \textit{ˈnè  ˈmá  aʰkaˈbátʃ͡ili    kaˈhê}\\
            \gll    ˈnè  ˈmá   aʰka-\textbf{ˈbá-tʃa}-li    kaˈhê  \\
                    1\textsc{sg.nom} already    sweet-\textsc{inch{{}-tr.pl}{}-pst}  coffee\\
            \glt    ‘I already sweetened the coffee.’\\
            \glt    ‘Ya endulcé el café.’   < BFL 08 1:20/el >\\
        }
                \ex[]{
                \textit{ˈnè   ˈmí  baʔˈwí  rataˈbátʃ͡kira}\\
                \gll    ˈnè   ˈmí  baʔˈwí  rata-\textbf{ˈbá-tʃa}{}-ki-ra\\
                        1\textsc{sg.nom} 2\textsc{sg.acc}  water  heat-\textsc{inch{{}-tr}{}-appl-pot}\\
                \glt    ‘Shall I heat the water for you?’\\
                \glt    ‘¿Te caliento el agua?’  < BFL 08 1:21/el >\\
            }
    \z
\z

Position S3 is occupied by a set of applicative suffixes. These applicative suffixes might have encoded semantic differences in a previous diachronic stage,\footnote{Some semantic differences are still retained in cognate suffixes in the closely related \ili{Mountain Guarijío} \parencite{miller1996guarijio}.} but synchronically they are lexically selected by the roots to which they attach. For instance, the applicative suffix \textit{-ni} is the only applicative suffix that can be attached to bases derived with transitive suffix \textit{-bû} (e.g., (\ref{ex: S2 - S3 affix order}a--b)). This applicative suffix is also attested with stems derived through other transitive suffixes (\ref{ex: S2 - S3 affix orderc}).

\ea\label{ex: S2 - S3 affix order}
{\textsc{tr} {(S2) -- \textsc{appl} (S3)}}

    \ea[]{
    \textit{ˈwé ˈnè moʔoˈbûnima toˈwí ˈétʃ͡i muˈkî}\\
    \gll    ˈwé ˈnè moʔo-\textbf{ˈbû-ni}-ma toˈwí ˈétʃ͡i muˈkî\\
            \textsc{int} \textsc{1sg.nom}   go.up-\textsc{{tr-appl}-fut.sg}   boy \textsc{dist }  woman \\
    \glt    ‘I will lift the boy for that woman.’ \\
    \glt    ‘Voy a levantarle el niño a esa mujer.’   < BFL 05 1:39/el >\\
}\label{ex: S2 - S3 affix ordera}
        \ex[]{
        \textit{ˈwé ta rakiˈbûnibo}\\
        \gll    ˈwé ta raki\textbf{-ˈbû-ni}-bo\\
                \textsc{int}  1\textsc{pl.nom }  push{-\textsc{tr-appl}}-\textsc{fut.pl}\\
        \glt    ‘Let’s push it for him.’ \\
        \glt    ‘Vamos empujándoselo.’ < SFH 05 1:61/el >\\
    }\label{ex: S2 - S3 affix orderb}
            \ex[]{
            \textit{riʔoˈnânima}\\
            \gll    riʔo\textbf{-ˈnâ-ni}-ma\\
                    sandpaper{-\textsc{tr-appl}-\textsc{fut.sg}}\\
            \glt    ‘They will sandpaper (the wood).’\\
            \glt    ‘Se lo va a lijar (la madera).’  < SFH 05 1:175/el >\\
        }\label{ex: S2 - S3 affix orderc}
    \z
\z

The applicative suffixes in slot S3 are in turn followed by a productive causative suffix in slot S4. In (\ref{ex: S3 - S4 affix order}), applicative suffixes \textit{-ni, -si} and \textit{{}-wi} preceed the causative suffix \textit{-ti}.

\ea\label{ex: S3 - S4 affix order}
{\textsc{appl} (S3) -- \textsc{caus} (S4)}

    \ea[]{
    \textit{gaˈbriêlo suˈwíwtima ˈlé ba}\\
    \gll    gaˈbriêlo suˈwí-\textbf{wi-ti}-ma aˈlé ba\\
            Gabriela finish.up.\textsc{appl-{appl-caus}-fut.sg} \textsc{dub} \textsc{cl}\\
    \glt    ‘Gabriela will make her finish up (his tortillas).’\\
    \glt    ‘Gabriela va a hacer que ella se las acabe (sus tortillas).’  < BFL 08 1:27/el >\\
}
        \ex[]{
        \textit{ˈnè a ˈmí ˈʃûntikisa ˈró}\\
        \gll    ˈnè a ˈmí ˈsû-\textbf{ni-ti}-ki-sa ˈró   \\
                1\textsc{sg.nom} \textsc{aff} 2\textsc{sg.acc} sew-\textsc{{appl-caus}-appl-cond}  \textsc{q}\\
        \glt    ‘What if I made you sew her a skirt?’\\
        \glt    ‘¿Qué tal si te hago coserle una falda?'  < BFL 08 1:28/el >\\
    }
            \ex[]{
            \textit{to  jaˈdîra ˈpáʃtiri  bo!}\\
            \gll    to  jaˈdîra ˈpá\textbf{si-ti}-ri  bo!\\
                    \textsc{exh}  Yadira  throw-\textsc{{appl-caus}-imp.sg}  \textsc{exh}\\
            \glt    ‘Let’s see, throw it to Yadira!’\\
            \glt    ‘¡A ver, tíraselo a Yadira!’       < BFL 08 1:28/el >\\
        }
                \ex[]{
                \textit{oˈʃìwtimo ˈlá ˈnè ˈjéla}\\
                \gll    oˈsì-\textbf{wi-ti}-ma oˈlá ˈnè ˈjé-la\\
                        write-\textsc{{appl-caus}-fut.sg}  \textsc{cer}  1\textsc{sg.nom}  mom-\textsc{poss}\\
                \glt    ‘She’ll make him write my mom (a letter).’\\
                \glt    ‘Va a hacer que le escriba a mi mamá.’   < BFL 08 1:28/el >\\
            }
    \z
\z

The productive causative suffix \textit{-ti} appears ordered before the productive applicative suffix \textit{-ki} (as shown in (\ref{ex: S5-S6 affix order})), motivating a slot S5 for the applicative suffix.


\ea\label{ex: S5-S6 affix order}
{\textsc{caus} (S5) -- \textsc{appl} (S6)}

    \ea[]{
        \textit{taˈmí=komi oʔˈpéstikima ˈlé ba}\\
        \gll    taˈmí=ko=mi oʔˈpési-\textbf{ti-ki}-ma aˈlé ba\\
                1\textsc{sg.acc}=\textsc{emph=2sg.nom}  vomit-\textsc{caus{-appl}-fut.sg} \textsc{dub}  \textsc{cl}\\
        \glt    ‘You’ll make him throw up on me.’\\
        \glt    ‘Vas a hacer que me vomite encima.’    < BFL 08 1:27/el >\\
    }
            \ex[]{
            \textit{aʰkaˈbátikini}\\
            \gll    aʰka-ˈbá-\textbf{ti-ki}=ni\\
                    sweet-\textsc{inch-caus-pst.ego=1sg.nom}\\
            \glt    ‘I sweetened it.’\\
            \glt    ‘Lo endulcé.’    < BFL 08 1:18/el >\\
        }
                \ex[]{
                \textit{taˈmí  noˈkèrtikiri!}\\
                \gll    taˈmí  noˈka-è-ri-\textbf{ti-ki}{}-ri\\
                        1\textsc{sg.acc}  move.\textsc{appl-caus-caus{-appl}-imp.sg}\\
                \glt    ‘Move it for me!’\\
                \glt    ‘¡Muévemelo!’ < BFL 08 1:28/el >\\
            }
    \z
\z

While the Causative-applicative (\textit{{}-ti-ki}) order is the most commonly attested, these suffixes can also permutate their order. In the examples in (\ref{ex: appl-caus order}) the applicative suffix \textit{{}-ki} is ordered before the causative suffix \textit{-ti}.

\ea\label{ex: appl-caus order}
{\textsc{appl (S6) -caus (S5)} \textit{-ki-ti} order}

    \ea[]{
    \textit{tòo, ˈjêni ˈdûlse ˈìwkitiri jadîra}\\
    \gll    tòo, ˈjêni ˈdûlse ˈìwi\textbf{-ki-ti}-ri jadîra\\
            go!  Yeni candy  bring.\textsc{appl-appl-caus-imp.sg} Yadira\\
    \glt    ‘Make Yeni bring candy for Yadira!’\\
    \glt    ‘¡Ve, haz que Yeni le traiga dulces a Yadira!’    < BFL 07 1:62/el >\\
}
        \ex[]{
        \textit{to miˈtʃ͡íktiri bo}\\
        \gll    to miˈtʃ͡í-\textbf{ki-ti}-ri bo!\\
                \textsc{exh} carve-\textsc{{appl-caus}-imp.sg} \textsc{exh}\\
        \glt    ‘Carve it for him’\\
        \glt    ‘¡Lábraselo!’  < BFL 08 1:107/el >\\
    }
            \ex[]{
            \textit{ˈnè  tʃ͡oʔˈmá  biʔˈwìktimo   ˈlá  tiˈwé}\\
            \gll    ˈnè  tʃ͡oʔˈmá  biʔˈwì-\textbf{ki-ti}-ma  oˈlá  tiˈwé\\
                    1\textsc{sg.nom}  mucus  clean-\textsc{{appl-caus}-fut.sg} \textsc{cer} girl\\
            \glt    ‘I’ll make her clean the girl’s nose.’\\
            \glt    ‘La voy a hacer que le limpie los mocos a la niña.’    < BFL 08 1:55/el >\\
        }
    \z
\z

The applicative-causative suffix order, as exemplified in (\ref{ex: appl-caus order}), is marginally attested in the Choguita Rarámuri corpus (the factors that may condition this affix permutation pattern are discussed in \citealt{caballero2010scope}). The overwhelming preference for the causative-applicative order motivates positing a separate slot, S5, for the productive applicative suffix \textit{-ki}, separate from the rest of the applicative suffixes in S3. An additional argument in favor of keeping the two applicative positions distinct comes from the difference in productivity between the inner applicatives and the later, productive suffix \textit{-ki}.

The applicative suffix is in turn followed by the desiderative suffix \textit{-nále} (in slot S6). This suffix is exemplified in (\ref{ex: S5 - S6 order}).

\ea\label{ex: S5 - S6 order}
{\textsc{appl} (S5) -- \textsc{desid} (S6)}

    \ea[]{
    \textit{ˈnè mi  biˈlé ˈwàsi miʔˈríkinili muˈhê   oˈmáwaratʃ͡i}\\
    \gll    ˈnè=mi  biˈlé ˈwàsi miʔˈrí\textbf{-ki-nale} muˈhê   oˈmáwaratʃ͡i\\
            1\textsc{sg.nom}=2\textsc{sg.acc} one cow kill\textsc{{-appl-desid}} 2\textsc{sg.nom} party\\
    \glt    ‘I want to kill one cow for you, for your party.’\\
    \glt    ‘Quiero matar una vaca para ti (para tu fiesta).'   < SFH 07 2:65-66/el >\\
}
        \ex[]{
        \textit{ja taˈmí  piˈtʃ͡í̀kinili niˈhê biˈtêritʃ͡͡i}\\
        \gll    ja taˈmí  piˈtʃ͡í̀\textbf{-ki-nale} niˈhê biˈtêritʃ͡i\\
                already 1\textsc{sg.acc} sweep\textsc{{-appl-desid}} 1\textsc{sg.nom}  house\\
        \glt    ‘They already want to sweep my house for me.’\\
        \glt    ‘Ya me quieren barrer la casa.’      < SFH 07 2:65-66/el >\\
    }
            \ex[]{
            \textit{ˈmí siˈmèkinira ˈrú}\\
            \gll    ˈmí siˈmè-\textbf{ki-nale}-ra ˈrú\\
                    2\textsc{sg.acc}  play-\textsc{{appl-desid}-rep}  say.\textsc{prs}\\
            \glt    ‘He says he wants to play a song for you.’\\
            \glt    ‘Dice que te quiere tocar una canción.’    < BFL 08 1:60/el >\\
        }
                \ex[]{
                \textit{ˈém   t͡ʃiˈmí   siˈmíra   baniˈsúkinima}\\
                \gll    ˈémi   t͡ʃiˈmí   siˈmí-ra   baniˈsú-\textbf{ki-nale}-ma\\
                        2\textsc{pl.nom}  there  go-\textsc{pot}    pull-\textsc{{appl-desid}-fut.sg}\\
                \glt    ‘They will want to go and pull it for them.’\\
                \glt    ‘Van a querer ir jalándoselo.’    < SFH 08 1:75/el >\\
            }
    \z
\z

The desiderative suffix \textit{-ˈnále} is then followed by the associated motion \textit{-simi}, as exemplified in (\ref{ex: S6-S7 affix order}). Further discussion and analysis of the nature of the short-long alternation for this and other suffixes is provided in \chapref{chap: prosody} (§\ref{subsec: prosodic templates in CR}).


\ea\label{ex: S6-S7 affix order}
{\textsc{desid} (S6) – \textsc{mot} (S7)}

    \ea[]{
    \textit{ˈnè ˈtʃ͡á kotʃ͡iˈnálsi iˈnâli}\\
    \gll    ˈnè ˈtʃ͡á kotʃ͡i\textbf{-ˈnále-simi}-i iˈnâli\\
            1\textsc{sg.nom} \textsc{int} sleep\textsc{{-desid-mot-impf}} go\\
    \glt    ‘I really wanted to go along sleeping (e.g., riding in a bus).’\\
    \glt    ‘Realmente quise irme durmiendo.’   < SFH 07 2:72-73/el >\\
}
        \ex[]{
        \textit{koʔˈnálsia iˈnâli}\\
        \gll    koʔ\textbf{{}-ˈnále-simi}-a iˈnâli\\
                eat\textsc{{-desid-mot}-prog} go\\
        \glt    ‘He went wanting to go along eating.’\\
        \glt    ‘Se fue queriendo comer.’      < SFH 07 2:72-73/el >\\
    }
            \ex[]{
            \textit{ˈnè  iˈsîinsia  iˈnâli}\\
            \gll    ˈnè  iˈsî-\textbf{nale-simi}-a  iˈnâli \\
                    1\textsc{sg.nom}  urinate-\textsc{desid{-mot}-prog}  go\\
            \glt    ‘I’m going along wanting to urinate.’\\
            \glt    ‘Voy queriendo orinar.’   < BFL 08 1:61/el >\\
        }
                \ex[]{
                \textit{ˈmán   ˈnè  t͡ʃaˈkéna ˈwánsia      iˈnâli  riˈʰtê}\\
                \gll    ˈmá=n   ˈnè  t͡ʃaˈkéna ˈwá-\textbf{nale-simi}-a      iˈnâli  riʰˈtê\\
                        already  1\textsc{sg.nom} aside throw-\textsc{desid{-mot}-prog} go  stone\\
                \glt    ‘I go along wanting to throw away the stones.’\\
                \glt    ‘Ya me dan ganas de ir quitando las piedras.’   < BFL 08 1:88/el >\\
            }
    \z
\z

The desiderative and associated motion suffixes are also attested in the inverse order, as shown in (\ref{ex: mot-desid order}).

\ea\label{ex: mot-desid order}
{\textsc{mot-desid} affix order}

    \ea[]{
    \textit{riʔiˈbúsnili riʰˈtê  buʔuˈtʃ͡ími}\\
    \gll    riʔi-ˈbú\textbf{-simi-nale} riʰˈtê   buʔuˈtʃ͡ími\\
            stone-\textsc{rev{-mot-desid}}  stone  road\\
    \glt    ‘(He) wants to go along the road removing stones.’\\
    \glt    ‘Quiere irse por el camino quitando las piedras.’    < SFH 07 2:72-73/el >\\
}
        \ex[]{
        \textit{aˈwísinili}\\
        \gll    aˈwí\textbf{-simi-nale}-i\\
                dance\textsc{{-mot-desid}-impf}\\
        \glt    ‘She wanted to go along dancing.’\\
        \glt    ‘Quería irse bailando.’       < SFH 07 2:72-73/el >\\
    }
            \ex[]{
            \textit{ˈnèn naˈlàsnila iˈnâli}\\
            \gll    ˈnè=n naˈlà\textbf{-simi-nale}-a iˈnâli\\
                    \textsc{int=1sg.nom}  cry-\textsc{mot{-desid}-prog}  go\\
            \glt    ‘I’m going along feeling like crying.’\\
            \glt    ‘Voy queriendo llorar.’ < BFL 08 1:89/el >\\
        }
                \ex[]{
                \textit{ˈá  biˈlá  taˈmí  ˈjòrsinira      ˈruá}\\
                \gll    ˈá   biˈlá  taˈmí  ˈjò-ri-\textbf{simi-nale}-ra     ru-ˈwá\\
                        \textsc{aff}  really  1\textsc{sg.acc}  mad-\textsc{caus-mot{{}-desid}-rep}  say-\textsc{mpass}\\
                \glt    ‘He says he wants to go along making me mad.’\\
                \glt    ‘Dice que me quiere ir haciendo enojar.’    < SFH 08 1:72/el >
            }
    \z
\z

The next position is occupied by the auditory evidential suffix \textit{-tʃ͡ane}, another disyllabic suffix that is transparently related to an independent verb in the language, \textit{(a)tʃ͡ane}, ‘say, sound like’. The following examples show the evidential suffix ordered after the associated motion suffix (\ref{ex: s7-s8 affix order}) and the desiderative suffix (\ref{ex: s6-s8 affix order}).

\ea\label{ex: s7-s8 affix order}
{\textsc{mot} (S7) -- \textsc{ev} (S8)}

    \ea[]{
    \textit{wikuˈwâstʃ͡ina}\\
    \gll    wikuˈwâ\textbf{-simi-tʃane}-a\\
            whistle\textsc{{-mot-ev}-prog}\\
    \glt    ‘It sounds like they are going around whistling.’\\
    \glt    ‘Se oye que van chiflando.’  < SFH 07 2:74, el492/el >\\
}
        \ex[]{
        \textit{ˈwîstʃ͡ane}\\
        \gll    ˈwî-\textbf{simi-tʃane}\\
                harvest-\textsc{mot{-ev}}\\
        \glt    ‘It sounds like they are going along harvesting.’ \\
        \glt    ‘Se oye que van pizcando.’  < SFH 08 1:132/el >\\
    }
            \ex[]{
            \textit{ˈá biˈlá uˈbârstʃ͡ani}\\
            \gll    ˈá biˈlá uˈbâ-ri-\textbf{simi-tʃane}\\
                    \textsc{aff} really bathe-\textsc{caus-mot{-ev}}\\
            \glt    ‘It sounds like they are going along bathing them.’\\
            \glt    ‘Se oye que van bañándolos.’  < SFH 08 1:150/el >\\
        }
                \ex[]{
                \textit{ˈá biˈlá ˈwé ˈàastʃ͡ani wikoˈki    uʔˈpá}\\
                \gll    ˈá biˈlá ˈwé ˈà-\textbf{simi-tʃane} wikoˈki    uʔˈpá\\
                        \textsc{aff} really \textsc{int} look.for-\textsc{mot{-ev}} mushrooms  back\\
                \glt    ‘It sounds like they are going along looking for mushrooms back there.’\\
                \glt    ‘Se oye que van buscando hongos atrás.’   < SFH 08 1:145/el >\\
            }
    \z
\z

\ea\label{ex: s6-s8 affix order}
{\textsc{desid} (S6) -- \textsc{ev} (S8)}

    \ea[]{
    \textit{wikaˈrântʃ͡ane}\\
    \gll    wikaˈrâ\textbf{-nale-tʃane}\\
           sing-\textsc{desid{-ev}}\\
    \glt    ‘It sounds like they want to sing.’\\
    \glt    ‘Se oye como que quieren cantar.’  < SFH 07 1:9/el >\\
}
        \ex[]{
        \textit{ˈmá koʔ-ˈnáltʃ͡ani}\\
        \gll    ˈmá koʔ-ˈ\textbf{nále-tʃane}\\
                already eat-\textsc{desid{{}-ev}}\\
        \glt    ‘It seems they already want to eat.’\\
        \glt    ‘Como que ya quieren comer.’  < SFH 08 1:124/el >\\
    }
     %       \ex[]{
       %     \textit{roˈkò̀ á reʔˈètʃ͡ani}\\
        %    \gll    roˈkò̀ á reʔˈè-tʃ͡an-i\\
         %           last.night \textsc{aff} play-\textsc{desid\textbf{-ev}-impf}\\
          %  \glt    ‘Last night it sounded like they wanted to play.’\\
           % \glt    ‘Anoche se oía que querían jugar.’  < SFH 08 1:124/el >\\
      %  }
      \newpage
                \ex[]{
                \textit{roˈkò mi boˈʔóri ˈnà haˈré moʔiˈnáaltʃ͡ini}\\
                \gll    roˈkò mi boˈʔóri ˈnà haˈré moʔi-\textbf{ˈnále-tʃane}-i\\
                        night  there  up.there  there  some  enter.\textsc{pl-desid{-ev}-impf}\\
                \glt    ‘Last night it sounded like theey wanted to go inside up there.’ \\
                \glt    ‘Anoche se oía que querían entrar allá arriba.’    < SFH 08 1:124/el >\\
            }
    \z
\z

The desiderative and evidential suffixes, too, can appear in the inverse order, as shown in (\ref{ex: evidential-desidertiave order}), due to factors discussed in more detail in \citet{caballero2010scope}.\footnote{In (\ref{ex: evidential-desidertiave order}a--b), the last vowel of the desiderative suffix is replaced by the epistemic suffix \textit{-o}, that encodes an event is the result of a psychological or mental state, in this case, `want'. Epistemic marking in desiderative-marked verbs is attested in the speech of some speakers (e.g., BFL in (\ref{ex: evidential-desidertiave order}a--b), but not others (e.g., SFH in (\ref{ex: evidential-desidertiave order}c--d)). More details about this suffix can be found in Appendix~\ref{subsec: epistemic}.}

\ea\label{ex: evidential-desidertiave order}
{Evidential-desiderative affix order}

    \ea[]{
    \textit{oʔˈpéstʃ͡analo}\\
    \gll    oʔˈpés\textbf{-tʃane-nale}-o\\
            vomit-\textsc{{ev-desid}-ep}\\
    \glt    ‘It sounds like they want to vomit.’ \\
    \glt    ‘Se oye como que quieren vomitar.’ < BFL 07 rec300/el >\\
}
        \ex[]{
        \textit{paraˈértʃ͡analo}\\
        \gll    paraˈér\textbf{-tʃane-nale}-o\\
                dance.paraeri-\textsc{{ev-desid}-ep}\\
        \glt    ‘It sounds like they want to dance paraéri.’ \\
        \glt    ‘Se oye como que quieren bailar paraéri.’  < BFL 07 1:182/el >\\
    }
            \ex[]{
            \textit{ˈá  biˈlá  ˈt͡ʃîkle ˈkétʃ͡itʃ͡anili ˈkûruwi}\\
            \gll    ˈá  biˈlá  ˈt͡ʃîkle ˈkétʃ͡i\textbf{-tʃane-nale} ˈkûruwi\\
                    \textsc{aff} really  gum  chew-\textsc{{ev-desid}}  kids\\
            \glt    ‘It seems like the kids want to chew gum.’\\
            \glt    ‘Se oye que los niños quieren mascar chicle.’  < SFH 08 1:146/el >\\
        }
                \ex[]{
                \textit{ˈnápi  ˈlé  aˈtístʃ͡anala}\\
                \gll     ˈnápi  aˈlé  aˈtísi-\textbf{tʃane-nale}-a\\
                        \textsc{rel} \textsc{dub} sneeze-\textsc{{ev-desid}-prog}\\
                \glt    ‘It is like somebody wants to sneeze (it sounds like it).’\\
                \glt     ‘Como que se oye que quieren estornudar.’   < SFH 08 1:122/el >\\
            }
    \z
\z

The desiderative, associated motion and evidential suffixes precede a set of stress-shifting suffixes, posited to occupy slot (S9). All of the suffixes in this position are suffixes encoding both voice and tense/aspect. A list is provided in (\ref{ex: S9 suffixes}) (more details about each individual suffix can be found in the Appendix in the cross-referenced sections).\footnote{The motion imperative suffix and irrealis sigular suffix are homophonous, but distinct suffixes in the language.}

\ea\label{ex: S9 suffixes}
{Inflectional suffixes: slot (S9) }

\begin{tabular}{lll}
\textit{-ru} & Past Passive (\textsc{pst.pass}) & §\ref{subsubsec: past passive}\\
\textit{-pa} & Future Passive (\textsc{fut.pass}) & §\ref{subsubsec: future passive}\\
\textit{-rîwa, -wá} & Medio-Passive (\textsc{mpass}) & §\ref{subsubsec: medio-passive}\\
\textit{-sûwa} & Conditional Passive (\textsc{cond.pass}) & §\ref{subsubsec: conditional passive}\\
\textit{-ˈmêa, -ma} & Future Singular  (\textsc{fut.sg}) & §\ref{subsubsec: future singular}\\
\textit{-pô} & Future Plural (\textsc{fut.pl}) & §\ref{subsubsec: future plural}\\
\textit{-mê} & Motion Imperative (\textsc{mot.imp}) & §\ref{subsec: motion imperative}\\
\textit{-sâ} & Conditional (\textsc{cond}) & §\ref{subsec: conditional}\\
\textit{-mê} & Irrealis Singular (\textsc{irr.sg}) & §\ref{subsubsec: irrealis singular}\\
\textit{-pi} & Irrealis Plural (\textsc{irr.pl}) & §\ref{subsubsec: irrealis plural}\\

\end{tabular}

\z

\largerpage
As shown in the next examples, the desiderative suffix (\ref{ex: S6- S9 affix order}), the associated motion suffix (\ref{ex: S7 - S9 affix order}) and the evidential suffix (\ref{ex: S8-S9 affix order}) all precede these inflectional suffixes, which are more peripheral in the verbal template.


\ea\label{ex: S6- S9 affix order}
{\textsc{desid} (S6) -- \textsc{fut.sg} (S9)}

    \ea[]{
    \textit{piˈtʃ͡írnimo ˈlá}\\
    \gll    piˈtʃ͡í-ri\textbf{-nale-ma} oˈlá  \\
            sweep-\textsc{caus{-desid-fut.sg}} \textsc{cer}\\
    \glt    ‘He will want to make him sweep.’\\
    \glt    ‘Va a querer hacerlo barrer.’    < BFL 07 EDCW(81)/el >\\
}
        \ex[]{
        \textit{niˈhê ko ˈá kaˈhê ˈpàksinima}\\
        \gll    niˈhê ko ˈá kaˈhê ˈpàki-si-\textbf{nale-ma}\\
                1\textsc{sg.nom} \textsc{emph} \textsc{aff} coffee brew-\textsc{mot-{desid-fut.sg}}\\
        \glt    ‘I will go along wanting to brew some coffee.’\\
        \glt    ‘Voy a querer ir colando café.’   < SFH 08 1:147/el >\\
    }
            \ex[]{
            \textit{poˈtʃ͡ítnimon oˈlá jaˈdîra}\\
            \gll    poˈtʃ͡í-ti-\textbf{nale-ma}=ni oˈlá jaˈdîra\\
                    jump-\textsc{caus-{desid-fut.sg}=1sg.nom} \textsc{cer} Yadira\\
            \glt    ‘I will want to make Yadira jump.’\\
            \glt    ‘Voy a querer hacer brincar a Yadira.’ < BFL 08 1:62/el >\\
        }
    \z
\z

\ea\label{ex: S7 - S9 affix order}
{\textsc{mot} (S7) -- \textsc{fut.sg} (S9)}

    \ea[]{
    \textit{niˈhê ˈmí tiˈtʃ͡íksima}\\
    \gll    niˈhê ˈmí tiˈtʃ͡í-ki\textbf{-simi-ma}\\
            1\textsc{sg.nom} 2\textsc{sg.acc} comb-\textsc{appl{-mot-fut.sg}}\\
    \glt    ‘I will go along the way combing your hair.’\\
    \glt    ‘Voy a ir peinándote.’   < SFH 07 2:67/el >\\
}
        \ex[]{
        \textit{ˈnè koˈtʃí poˈt͡ʃítisima}\\
        \gll    ˈnè koˈtʃí poˈt͡ʃí-ti-\textbf{simi-ma}\\
                1\textsc{sg.nom} dog jump-\textsc{caus-{mot-fut.sg}}\\
        \glt    ‘I will go along making the dog jump.’ \\
        \glt    ‘Voy a ir haciendo que brinque el perro.’    < SFH 08 1:72/el >\\
    }
            \ex[]{
            \textit{ˈmín piˈwârsimo ˈlá}\\
            \gll    ˈmí=ni piˈwâ-ri-\textbf{simi-ma} oˈlá\\
                    2\textsc{sg.acc=1sg.nom} smoke-\textsc{caus-{mot-fut.sg}} \textsc{cer}\\
            \glt    ‘I’ll make you go along smoking.’\\
            \glt    ‘Voy a hacer que vayas fumando.’   < BFL 08 1:91/el >\\
        }
    \z
\z


\ea\label{ex: S8-S9 affix order}
{\textsc{ev} (S8) -- \textsc{mpass} (S9)}

    \textit{ˈnè  iˈtʃ͡ìirtʃ͡unua}\\
    \gll    /ˈnè  itʃ͡-ì-ri-\textbf{tʃane-wa}\\
            1\textsc{sg.nom}  plant.\textsc{appl-appl-{ev-mpass}}\\
    \glt    ‘It sounds like (corn) is being planted for me.’   \\
    \glt    ‘Se oye como que me están sembrando maíz.’   < SFH 07 1:10/el >\\

\z

The evidential suffix can also appear ordered after these suffixes, under circumstances described in \citet{caballero2010scope}. The examples below show the evidential suffix preceeded by the stressed allomorph of the future singular suffix (\ref{ex: fut/hab pass -  evidential order}a--b) and by the habitual passive suffix (\ref{ex: fut/hab pass -  evidential orderc}).

\ea\label{ex: fut/hab pass -  evidential order}
{Future/Habitual Passive -- Evidential affix order}

    \ea[]{
    \textit{ˈnápi   ˈlé  ˈmá  awiˈmêtʃ͡ani}\\
    \gll    ˈnápi   aˈlé  ˈmá   awi\textbf{-ˈmê-tʃane}\\
            \textsc{rel}  \textsc{dub}  already    dance\textsc{{-fut.sg-ev}}\\
    \glt    ‘It sounds like they are going to dance.’\\
    \glt    ‘Se oye como que van a bailar.’   < SFH 07 1:140/el >\\
}\label{ex: fut/hab pass -  evidential ordera}
        \ex[]{
        \textit{ˈnápi ˈlé naˈkómtʃ͡ana ˈwàsi}\\
        \gll    ˈnápi aˈlé naˈkó\textbf{-ma-tʃane}-a ˈwàsi\\
                \textsc{rel} \textsc{dub} fight\textsc{{-fut.sg-ev}-prog} cows\\
        \glt    ‘It sounds like the cows are going to fight.’\\
        \glt    ‘Se oye como que las vacas se van a pelear.’   < SFH 07 1:140/el >\\
    }\label{ex: fut/hab pass -  evidential orderb}
            \ex[]{
            \textit{ˈnápi riˈménuwatʃ͡ana}\\
            \gll    ˈnápi riˈmé-nu\textbf{-wa-tʃane}-a\\
                    \textsc{rel} make.tortillas-\textsc{appl{-mpass-ev}-prog}\\
            \glt    ‘It sounds like they are making him tortillas.’\\
            \glt    ‘Como que se oye que le están haciendo tortillas.'  < SFH 07 2:69/el >\\
        }\label{ex: fut/hab pass -  evidential orderc}
    \z
\z

Finally, there is another slot of stress-shifting affixes that mark mood (potential (\textsc{pot}) \textit{-ra}, imperative singular (\textsc{imp.sg}) suffixes \textit{{}-kâ} and \textit{-sâ}, and imperative plural (\textsc{imp.pl}) \textit{-sì}). In (\ref{ex: S9-S10 affix order}), the potential suffix and the imperative singular suffix \textit{-sa} are ordered after the motion imperative suffix (in S8):\footnote{There is no example that demonstrates the relative ordering between the evidential suffix (posited in S9) and the mood suffixes in (S10). The evidential is semantically incompatible with at least the imperative mood.}


\ea\label{ex: S9-S10 affix order}
{\textsc{mot.imp} (S9) -- Mood (S10)}

    \ea[]{
    \textit{ˈjurka osiˈmêra ˈlé}\\
    \gll    ˈjuri-ka osi\textbf{-ˈmê-ra} aˈlé  \\
            take-\textsc{imp} write\textsc{{-mot.imp-pot}} \textsc{dub}\\
    \glt    ‘Go, take him to see if he writes.’\\
    \glt    ‘Ve y llévalo a ver si escribe.’ < BFL 08 1:94/el >\\
}
        \ex[]{
        \textit{ˈâamsa}\\
        \gll    ˈâ-\textbf{me-sa}\\
                give-{\textsc{mot.imp-imp.sg}}\\
        \glt    ‘Go give it to her!’\\
        \glt    ‘¡Ve y dáselo!’    < ROF 04 1:112/el >\\
    }
    \z
\z

As shown so far, there is a fair amount of morphotactic evidence for positing the positions of a complex verbal template in Choguita Rarámuri. The evidence laid out in this subsection involves attested linear ordering of suffixes. There are, however, two other important morphotactic phenomena in the Choguita Rarámuri verb, namely variable order of suffixes and multiple (or extended) exponence.

Several of the examples above show that several suffixes do not have a fixed order with respect to other suffixes, in interactions that are specific to defined pairs of suffixes belonging to the Syntactic Stem and Aspectual Stem domains of the verb. Suffixes that might switch their order include: causative and applicative; desiderative and associated motion; and desiderative and evidential.

In addition, the suffixes in the Syntactic Stem can display multiple exponence, i.e., they can be multiply marked without an equivalent semantic recursivity. This is exemplified in (\ref{ex: causative and applicative multiple exponence}).

\ea\label{ex: causative and applicative multiple exponence}
{Causative and Applicative Multiple Exponence}

    \ea[]{
    \textit{niˈhê biˈlá iʔnèrtimo ˈlá}\\
    \gll    niˈhê biˈlá iʔnè\textbf{{}-ri-ti-}ma oˈlá\\
            1\textsc{sg.nom} really  look-\textsc{caus{-caus}-fut.sg}  \textsc{cer} \\
    \glt    ‘I’ll make him look at it.’\\
    \glt     ‘Lo voy a hacer que lo vea.’   < SFH 06 3:181/el >\\
}
        \ex[]{
        \textit{niˈhê  iʔnèrili}\\
        \gll    niˈhê  iʔnè\textbf{-ri}-li\\
                1\textsc{sg.nom} look-\textsc{caus-pst}\\
        \glt    ‘I made him look at it.’  \\
        \glt    ‘Lo hice que lo viera.’     < SFH 06 3:181/el >\\
    }
            \ex[]{
            \textit{botoˈbûunkirini  ˈbôte}\\
            \gll    boto-ˈbû\textbf{-ni-ki}-ri=ni  ˈbôte \\
                    sink-\textsc{tr-{appl-appl}-pst.pass=1sg.nom}  can \\
            \glt      ‘They sank my can (in the river).’\\
            \glt      ‘Me hundieron el bote.’     < SFH 07 2:32/el >\\
        }
                \ex[]{
                \textit{botoˈbûunirini}\\
                \gll    boto-ˈbû\textbf{-ni}-ri=ni\\
                        sink-\textsc{tr-{appl}-pst.pass=1sg.nom}  \\
                \glt    ‘They sank my can (in the river).’\\
                \glt      ‘Me hundieron el bote.’   < SFH 07 2:32/el >\\
            }
    \z
\z

Variable suffix ordering and multiple exponence might render the verbal structure proposed in \tabref{tab:suffix-positions} a highly abstract representation. This structure, however, will be retained as a descriptive device since variable orders of suffixes are restricted to specific pairs of suffixes. The applicative, for instance, while variably ordered with respect to the causative, has a fixed position preceding the desiderative suffix and the associated motion suffix, among others. Moreover, variable suffix ordering and multiple exponence are only found with suffixes that belong to particular layers or domains in the verb.
%\tabref{tab:key:18} schematizes the proposed verbal domains in the Choguita Rarámuri verb and the order and exponence generalizations of the suffixes in each domain.


%\begin{table}
%\caption{Order and exponence properties of suffixes by verbal domain}
%\label{tab:key:18}

%\begin{tabularx}{\textwidth}{lQQ}
%\lsptoprule
%\textbf{Position} & \textbf{Stem level} & \textbf{Order and exponence} \\
%\midrule
%S1 & Derived Stem & Fixed suffix order\\
%S2 &  & \\
%S3 & Syntactic Stem & Variable suffix order, multiple exponence \\
%S4 &  & \\
%S5 &  & \\
%S6 & Aspectual Stem & Variable suffix order\\
%S7 &  & \\
%S8 &  & \\
%S9 & Finite Verb & Fixed suffix order\\
%S10 &  & \\
%S11 &  & \\
%S12 & Subordinate Verb & Fixed suffix order\\
%\lspbottomrule
%\end{tabularx}
%\end{table}
%\hspace{2cm}

There is other evidence showing that the concatenation of the Choguita Rarámuri verb is not unidimensional, and that there is an internal organization or hierarchy of processes. The next section will address the morphologically-conditioned phonology that make suffixes in the inner layers of the verb more tightly fused with the root than outer, inflectional suffixes.

\subsection{Phonological transparency and morpheme boundary strength}
\label{subsec: phonological transparency and morpheme boundary strenght}

The Choguita Rarámuri verb displays a nested structure that can be characterized in terms of the semantics and overall function of clusters of suffixes (valence-increasing, aspectual, etc.), and the morphologically-conditioned phonology that yields different degrees of morpho-phonological fusion between suffixes. This sub-section is concerned with the latter phenomena.

%Figure-\ref{fig: morph conditioned phonology} shows the distribution of the morphologically-conditioned phonology in the different domains of the Choguita Rarámuri verb.

%%please move \begin{table} just above \begin{tabular .
%\begin{table}
%\caption{Morphologically conditioned phonology by verbal domain}
%\label{tab:key:19}

%\begin{figure}
%\includegraphics[width=\textwidth]{figures/VerbalMorphology-Fig1.png}
%\caption{
%\label{fig: morph conditioned phonology}
%Morphologically-conditioned phonology by verbal domain}
%\end{figure}

%\begin{tabularx}{\textwidth}{lQlQQQQ}
%\lsptoprule
%& \textbf{Inner Stem} & CL & Haplology & \multirow[t]{3}{=}{Passive-triggered lenghthening} & \multirow[t]{3}{=}{Imperative stress-shift} & \multirow[t]{9}{=}{Round Harmony}\\
%S1 & \multirow[t]{2}{=}{\textbf{Derived Stem}} &&&&\\
%S2 &&&&&\\
%S3 & \multirow[t]{3}{=}{\textbf{Syntactic Stem}}&&&&\\
%S4 &&&&&\\
%S5 &&&&&\\
%S6 & \multirow[t]{3}{=}{\textbf{Aspectual Stem}}&&&&\\
%S7 &&&&&\\
%S8 &&&&&\\
%S9 & \multirow[t]{3}{=}{\textbf{Finite Verb}}&&&&\\
%S10 &&&&&\\
%S11 &&&&&\\
%S12 & \textbf{Sub Verb}&&&&\\
%\lspbottomrule
%\end{tabularx}

%transposed
%\begin{tabularx}{\textwidth}{XXXXXXXXXXXXX}
%\lsptoprule
%& S1 & S2 & S3 & S4 & S5 & S6 & S7 & S8 & S9 & S10 & S11 & S12\\
%\midrule
%\textbf{Inner Stem} & \multicolumn{2}{X}{\textbf{Derived Stem}} & \multicolumn{3}{X}{\textbf{Syntactic Stem}} & \multicolumn{3}{X}{\textbf{Aspectual Stem}} & \multicolumn{3}{X}{\textbf{Finite Verb}} & \textbf{Sub Verb}\\
%CL & \multicolumn{2}{X}{} & \multicolumn{3}{X}{} & \multicolumn{3}{X}{} & \multicolumn{3}{X}{} & \\
%Haplology & \multicolumn{2}{X}{} & \multicolumn{3}{X}{} & \multicolumn{3}{X}{} & \multicolumn{3}{X}{} & \\
%\multicolumn{3}{X}{Passive-triggered lenghthening} & \multicolumn{3}{X}{} & \multicolumn{3}{X}{} & \multicolumn{3}{X}{} & \\
%\multicolumn{3}{X}{Imperative stress-shift

%} & \multicolumn{3}{X}{} & \multicolumn{3}{X}{} & \multicolumn{3}{X}{} & \\
%\multicolumn{9}{X}{ Round Harmony
%} & \multicolumn{3}{X}{} & \\
%\lspbottomrule
%\end{tabularx}
%\end{table}
%\hspace{2cm}

The phonological phenomena discussed in this section are root-suffix haplology (§\ref{subsubsec: stem-suffix haplology}), compensatory lengthening (§\ref{subsubsec: compensatory lengthening}), past-passive induced lenthening (§\ref{subsubsec: past passive-conditioned lengthening}), imperative stress shift (§\ref{subsubsec: imperative singular stem formation}), round harmony (§\ref{subsubsec: round harmony}), and the distribution of stress-shifting and stress-neutral suffixes in the verb (§\ref{subsubsec: Stress and the verb: stress-shifting and stress-neutral suffixes}).


\subsubsection{Stem-suffix haplology}
\label{subsubsec: stem-suffix haplology}


We have seen that stress-conditioned vowel deletion results in derived consonant clusters and geminates in Choguita Rarámuri (see §\ref{subsubsec*: stress-conditioned vowel deletion} and §\ref{subsec: consonant sequences}). As highlighted above, derived geminates are subject to inter-speaker (and to a lesser extent, intra-speaker) variation. The alternative to having a derived geminate is to have syllable deletion in avoidance of adjacent identical syllable onsets. In this language, haplology between a final syllable in the Inner Stem and a following suffix syllable with identical onsets takes place in morphologically complex constructions. In (\ref{ex: root-suffix haplology}a,c,e), the root’s underlying final, unstressed syllable and the immediately adjacent suffix syllable have identical onsets, leading to deletion of one syllable. The examples in (\ref{ex: root-suffix haplology}b,d) show how the root’s final syllable is not deleted in other morphological constructions. The forms in (\ref{ex: root-suffix haplology}a,c,e) show unattested, hypothetical forms with adjacent root and suffix syllables with identical onsets.
%\todo[inline]{There are no (89f, g, h, i).} - FIXED

\ea\label{ex: root-suffix haplology}
{Root-suffix haplology}

    \ea[]{
    \glt    [aˈsíisa]\\
    \glt    {/aˈsísi-sa/}\\
    \glt    {wake.up-\textsc{cond}}\\
    \glt    `if s/he wakes up'\\
    \glt    `si se despierta' {< BFL 08 1:1/el >}\\
    \glt    \textit{*aˈsísi-sa}\\
}\label{ex: root-suffix haplologya}
        \ex[]{
        \glt    [aˈsísma]\\
        \glt    {/aˈsísi-ma/}\\
        \glt    {wake.up-\textsc{fut.sg}}\\
        \glt    `s/he will wake up'\\
        \glt    `se va a despertar' {< BFL 08 1:1/el >}\\
    }\label{ex: root-suffix haplologyb}
                \ex[]{
                \glt    [sutuˈbétʃ͡ini]\\
                \glt    {/sutuˈbétʃ͡i-tʃ͡ane/}\\
                \glt    {trip-\textsc{ev}}\\
                \glt    `it sounds like they are tripping'\\
                \glt    `suena que se tropiezan' {< SFH 07 1:143/el >}\\
                \glt    {\textit{*sutuˈbetʃ͡itʃ͡ini}}\\
            }\label{ex: root-suffix haplologyc}
                    \ex[]{
                    \glt    [sutuˈbétʃ͡inili]\\
                    \glt    {/sutuˈbétʃ͡i-nale/}\\
                    \glt    {trip-\textsc{desid}}\\
                    \glt    `s/he is about (wants) to trip'\\
                    \glt    `se quiere tropezar' {< BFL 07 1:138/el >}\\
                }\label{ex: root-suffix haplologyd}
                            \ex[]{
                            \glt    [sikoˈráanili]\\
                            \glt    {/sikoˈrá\textbf{na}-nale/}\\
                            \glt    {eye.secrete-\textsc{desid}}\\
                            \glt    `s/he is about (lit. wants) to have an eye secretion'\\
                            \glt    `quiere lagañear' {< BFL 08 1:1/el >}\footnote{The \ili{Spanish} translation of this verb form is ‘le quieren salir lagañas, quiere lagañear’, which is translated to English as ‘it is imminent that s/he will have eye secretion’.}\\
                            \glt    \textit{*sikoˈrananale}\\
                        }\label{ex: root-suffix haplologye}
    \z
\z

Example (\ref{ex: inner stem suffix-suffix haplologya}) shows how haplology also targets suffixes belonging to the Inner Stem domain:

\ea\label{ex: inner stem suffix-suffix haplology}
Inner Stem suffix-suffix haplology

    \ea[]{
    [tʃ͡aˈbóopo]\\
    {/tʃ͡aˈbó-\textbf{pi}-po/}\\
    {beard-\textsc{rev-fut.pl}}\\
    `They will remove their beards.'\\
    `Se van a quitar la barba.' {< SFH 08 1:5/el >}\\
}\label{ex: inner stem suffix-suffix haplologya}
        \ex[]{
        [tʃ͡aˈbópili]\\
        {/tʃ͡aˈbó-\textbf{pi}-po/}\\
        {beard-\textsc{rev-pst}}\\
        `They removed their beards.'\\
        `Se quitaron la barba.' {<SFH 08 1:5/el >}\\
    }\label{ex: inner stem suffix-suffix haplologyb}
    \z
\z

No haplology takes place between identical syllable sequences within roots (e.g. the form \textit{*aˈsí-ma} /aˈsísi-ma/, with the expected reading `S/he will wake up' (‘wake.up-\textsc{fut.sg}’) in (\ref{ex: root-suffix haplologyb}), is unattested). Syllables with identical onsets belonging to two suffixes can optionally undergo deletion, as shown in (\ref{ex: optional suffix haplology})).

\ea\label{ex: optional suffix haplology}
{Optional suffix-suffix haplology}

    \ea[]{
    [raʔaˈmânkiki]\\
    {/raʔaˈmâ-na-ki-ki/}\\
    {advise-\textsc{desid-appl-pst.ego}}\\
    `I wanted to advise them.'\\
    `Quise aconsejarlos.' {< BFL 06 5:132/el >}\\
}
        \ex[]{
        [miˈtʃ͡íiki]\\
        {/miˈtʃ͡í-\textbf{ki}-ki/}\\
        {carve-\textsc{appl-pst.ego}}\\
        `I carved it for them.'\\
        `Se lo labré.' {< BFL 08 1/el >}\\
    }
    \z
\z

Similar cases in the literature are treated as instances of a ``Repeated Morpheme Constraint'' \parencite{menn1984repeated}, a phenomenon where sequences of morphemes that are homophonous are prohibited. The Choguita Rarámuri case could be analyzed to instantiate this phenomenon, even though it involves only phonologically similar sequences of morphemes (for an overview of this phenomenon and theoretical implications, see \citet{inkelas2014interplay}).

\subsubsection{Compensatory lengthening}
\label{subsubsec: compensatory lengthening}

As discussed in \chapref{chap: phonology} and further addressed in \chapref{chap: prosody} below, there is no evidence of contrastive vowel length in Choguita Rarámuri. Surface long vowel sequences, however, are not uncommon and are salient acoustically in this language. There are several processes that yield these vowel sequences in surface representations. One of such processes is a word minimal size constraint affecting verbs (§\ref{sec: defining the prosodic word and other prosodic domains in CR}). Another source for surface vowel length is compensatory lengthening (CL), the phenomenon whereby the deletion of one element triggers a corresponding lengthening of another element.\footnote{CL has been treated as the transfer or preservation of a phonological unit, i.e. a mora, within a prosodic unit in the phonological literature (\citealt{hyman2003theory}, \citealt{mccarthy1996prosodic}, inter alia), or as a phonetically-based process that results from isochrony, the preservation of phonetic duration (\citealt{timberlake1983compensatory}, \citealt{barnes2000compensatory}). I assume that while this process was based phonetically, it is now part of the lexical phonology of the language.}  I address this process in this section.

The more widespread CL pattern in Choguita Rarámuri involves deletion of a vowel that triggers lengthening of a preceding syllable’s stressed vowel. This is exemplified in (\ref{ex: compensatory lengthening examples}).

\ea\label{ex: compensatory lengthening examples}
{ˈCVCV  >  ˈCV:C}

    \ea[]{
    \glt    \doublebox{\textit{ˈlàni >}}{\textit{ˈlàa}}\\
    \glt    {`bleed'}\\
    \glt    `sangrar'\\
}
        \ex[]{
        \glt    \doublebox{\textit{ˈmáli- >}}{\textit{ˈmáal-}} \\
        \glt    {`swim'}\\
        \glt    `nadar'\\
    }
            \ex[]{
            \glt    \doublebox{\textit{ˈnâri- >}}{\textit{ˈnâar-}}\\
            \glt    {`ask'}\\
            \glt    `preguntar'\\
        }
                \ex[]{
               \glt     \doublebox{\textit{muruˈbê-ni- >}}{\textit{muruˈbêe-n-}}\\
               \glt     {get.close-\textsc{appl}}\\
               \glt     `to get something close to something else'\\
               \glt     `acercarlo'\\
            }
                    \ex[]{
                    \glt    \doublebox{\textit{ramuˈwéli- >}}{\textit{ramuˈwéel-}}\\
                    \glt    {joke with in-laws}\footnote{This verb is translated into \ili{Spanish} by Choguita Rarámuri speakers as `vacilar con los cuñados'. This verb more accurately refers to a very specific kind of social interaction that involves joking playing around with a sister- or brother-in-law, a register documented also in \ili{Mountain Guarijío} \citet{miller1996guarijio}.}\\
                    \glt    `vacilar con cuñados'\\
                }
    \z
\z

Vowel deletion in these contexts occurs due to posttonic syncope in derived environments (described in more detail in \sectref{subsubsec*: stress-conditioned vowel deletion}). Some of the examples in (\ref{ex: compensatory lengthening examples}) are given in context in (\ref{ex: CL with intervening sonorant}) below. In these cases, CL takes place when the intervening consonant is a sonorant. Lengthened vowels are underlined in the surface form and deleted vowels are in bold face in the underlying representation.

\ea\label{ex: CL with intervening sonorant}
{CL with intervening sonorant}

    \ea[]{
    [ˈnâarta]\\
    {/ˈnâr\textbf{i}-ra/}\\
    {ask-\textsc{pot}}{}\\
    `S/he can ask.'\\
    `Puede preguntar.' {< SFH 08 1:82/el >}\\
}
        \ex[]{
        [muruˈbêenti]\\
        {/muruˈbê-n\textbf{i}-ti/}\\
        {get.close-\textsc{appl-caus}}{}\\
        `S/he makes them get it closer for them.'\\
        `Hace que se lo acerque.'  {< BFL 07 6:07/el >}\\
    }
            \ex[]{
            [ramuˈwéeltʃ͡ane]\\
            /ramuˈwél\textbf{i}-tʃ͡ane/\\
            {joke.with.in.laws-\textsc{ev}}{}\\
            `It sounds like they're joking with the in-laws.'\\
            `Se oye que están vacilando con los cuñados.' {< BFL 07 1:181/el >}\\
        }
                \ex[]{
                [ˈlàanki]\\
                {/làni-ki/} \\
                {bleed-\textsc{pst.ego}}\\
                `I bled.'\\
                `Sangré.' {< BFL 08 1:94/el >}\\
            }
    \z
\z

%Though vowel CL through vowel loss has been documented for other languages, this process is more uncommon than CL through consonant loss (cf. \citealt{Kavitskaya2001}:3, \citealt{Kavitskaya2002}).

Cases of CL triggered by deletion of a whole syllable have not been reported or even mentioned, to the best of my knowledge, as a logical type of CL. A second pattern of apparent vowel CL in Choguita Rarámuri, however, involves precisely the deletion of a syllable. In (\ref{ex: syllable deletion triggered CLa}), the tetrasyllabic root \textit{nabiˈsûri} truncates the final syllable when attaching the disyllabic desiderative suffix \textit{-nále}. The result is a stem with a long stressed vowel. There are no other potential sources for lengthening in this case (such as passive-conditioned lengthening or vowel loss), so the lengthening must be attributed to syllable deletion. CL takes place with an intervening voiceless affricate (\ref{ex: syllable deletion triggered CLc}), and an intervening voiceless fricative (\ref{ex: syllable deletion triggered CLe}). Below each example of syllable-triggered CL includes a related form with no deletion.

\ea\label{ex: syllable deletion triggered CL}
{Syllable deletion triggered CL}

    \ea[]{
    [nabiˈsûunili]\\
    {/nabiˈsû\textbf{ri}-nale/}\\
    {form.line-\textsc{desid}}\\
    \glt    `They want to form a line.'\\
    \glt    `Se quieren formar.' < BFL 07, SF 08 1:83 /el >\\
}\label{ex: syllable deletion triggered CLa}
        \ex[]{
        [nabiˈsûrima]\\
        {/nabiˈsûri-ma/}\\
        {form.line-\textsc{fut.sg}}\\
        \glt    `S/he will form a line.'\\
        \glt    `Se va a formar.' < BFL 07 VDB/el >\\
    }\label{ex: syllable deletion triggered CLb}
            \ex[]{
            [sutuˈbéetʃ͡-nale]\\
            {/sutuˈbé\textbf{tʃ͡i}-tʃ͡a-nale/}\\
            {trip-\textsc{ev-desid}}\\
            \glt    `It sounds like they want to trip.'\\
            \glt    `Se oye que se quieren tropezar.' < BFL \textsc{07} rec300/el >\\
        }\label{ex: syllable deletion triggered CLc}
                \ex[]{
                [sutuˈbétʃ͡ima]\\
                {/sutuˈbétʃ͡i-ma/}\\
                {  trip-\textsc{fut.sg}}\\
                \glt    `S/he will trip.'\\
                \glt    `Se va a tropezar.' <LEL 06 5:35/el >\\
            }\label{ex: syllable deletion triggered CLd}
                    \ex[]{
                    [aˈsíisa]\\
                    {/aˈsí\textbf{si}-sa/}\\
                    {wake.up-\textsc{cond}}\\
                    \glt    `if s/he wakes up'\\
                    \glt    `si se despierta' < SFH 08 1:82/el >\\
                }\label{ex: syllable deletion triggered CLe}
                        \ex[]{
                        [aˈsísima]\\
                        {/aˈsísi-ma/}\\
                        {wake.up-\textsc{fut.sg}}\\
                        \glt    `S/he will wake up.'\\
                        \glt    `Se va a despertar.' < SFH 08 1:82/el >\\
                    }\label{ex: syllable deletion triggered CLf}
    \z
\z

We could alternatively analyze CL triggered by syllable deletion as CL triggered by consonant deletion after cyclically applied posttonic syncope. That is, deletion would not target the syllable as a unit. Instead, the consonant, after being syllabified as coda of the preceeding syllable, would be the target of a phonetic weakening process to a semi-vowel and subsequent monophthongization (as proposed for other cases of CL by \citealt{de1979compensatory}). This can be represented schematically as in (\ref{ex: CL derived through syncope, gliding and monophthongization}):

\ea\label{ex: CL derived through syncope, gliding and monophthongization}
{CL derived through syncope, gliding and monophthongization}

\quadruplebox{UR}{Syncope}{C Gliding}{Monophthongization}

\quadruplebox{/CVCVC\textbf{V}/}{CVCVC}{CVCV\textbf{G}}{CVCV\textbf{V}}

\z

There is no evidence, however, that all derived coda consonants can glide, except for /b/ (cf. §\ref{subsec: semi-vowel monophthongization}). Furthermore, not all labio-velar semi-vowels undergo monophthongization, as shown in the examples in (\ref{ex: non-monophthongized labio-velar semi vowels}). Their existence makes it hard to posit a special set of semi-vowels that would not weaken and monopthongize with the syllable nucleus.

\ea\label{ex: non-monophthongized labio-velar semi vowels}
{Non-monophthongized labio-velar semi-vowels}

    \ea[]{
    \textit{ˈnè ko ˈmí raʔˈlìwtima paˈtrîsio}\\
    \gll    ˈnè=ko ˈmí raʔˈl-ì-wi-ti-ma paˈtrîsio\\
            1\textsc{sg.nom}=\textsc{emph}  2\textsc{sg.acc}  buy-\textsc{appl-caus-fut.sg}  Patricio\\
    \glt    ‘I will make you buy a soda for Patricio.’\\
    \glt    ‘Voy a hacer que le compres soda a Patricio.’< BFL 07 2:39/el >\\
}
        \ex[]{
        \textit{basaˈrôwmi ˈlé ma baʔaˈrîo}\\
        \gll    basaˈrôwa-mi aˈlé ma baʔaˈrî-o\\
                stroll.around-\textsc{irr.sg} \textsc{dub}  perhaps  tomorrow-\textsc{ep}\\
        \glt    ‘Perhaps she will take a stroll tomorrow.’\\
        \glt    ‘A lo mejor va a pasear mañana.’     < BFL 07 1:150/el >\\
    }
    \z
\z

Whether we analyze this last set of cases as instances of CL or not, the cases of vowel lengthening shown above are uncontroversially a case of CL triggered by V loss. CL is seemingly restricted to targeting stressed vowels of roots or derivational suffixes in the Inner Stem, delimitating this stem domain.

\subsubsection{Past passive-conditioned lengthening}
\label{subsubsec: past passive-conditioned lengthening}

Another morphologically-conditioned phonological effect involves vowel lengthening triggered by the past passive construction. The past passive suffix \textit{-ru} is a stress-shifting affix with a stressed and an unstressed allomorph. The unstressed allomorph has the property of triggering lengthening of the final stem stressed vowel. This is exemplified in (\ref{ex: vowel lengthening induced by past passive suffix}).

\ea\label{ex: vowel lengthening induced by past passive suffix}
{Vowel lengthening induced by past passive suffix }

    \ea[]{
    \textit{naˈʔî oˈsìiru}\\
    \gll    naˈʔî oˈs-ì-ru\\
            here  write-\textsc{appl-pst.pass}\\
    \glt    ‘Something was written here’\\
    \glt    ‘Aqui escribieron’   < SFH 08 1:45/el >\\
}
        \ex[]{
        \textit{ˈkani bahuˈréero ba}\\
        \gll    ka=ni bahuˈré-ru  ba\\
                \textsc{ˈneg=1sg.nom} invite.to.drink-\textsc{pst.pass} \textsc{cl}\\
        \glt    ‘I wasn’t invited to drink’\\
        \glt    ‘No me invitaron al tesgüino’ < BFL 07 2:33/el >\\
    }
            \ex[]{
            \textit{ˈtòoru grabaˈdôra}\\
            \gll    ˈtò-ru grabaˈdôra\\
                    take-\textsc{pst.pass} recorder\\
            \glt    ‘The recorder was taken.’\\
            \glt    ‘Se llevaron la grabadora.’ < SFH 08 1:45/el >\\
        }
                \ex[]{
                \textit{naˈʔî iˈtʃ͡íiru}\\
                \gll    naˈʔî itʃ͡i-ru\\
                        here  sew-\textsc{pst.pass}\\
                \glt    ‘It was sewn here.’\\
                \glt    ‘Aquí sembraron.’ < SFH 08 1:45/el >\\
            }
    \z
\z

This effect, which cannot be predicted from the prosodic or phonological properties of the affix, can be considered as an instance of dominance. `Dominant' affixes (as opposed to `recessive' affixes) have been defined as affixes which delete or neutralize contrasts in the base to which they attach (\citealt{kiparsky1982cyclic}, \citealt{inkelas1998theoretical}). Although dominant affixes are typically described as involving the deletion of accentual or tonal information from the base, there are also cases of dominant affixes that neutralize vowel length in the base (such as \ili{Mam Maya} \parencite{willardrainbow}). I argue that the past passive suffix is a dominant suffix which imposes lengthening in a preceding stressed syllable.

There are instances where the vowel quality of the past passive suffix (a high, back round vowel) is neutralized in height in posttonic position. This yields a suffix form that is homophonous with the active voice past suffix (\textit{-li}). In (\ref{ex: neutralized vowel quality of past passive suffix}--\ref{ex: past passive induced lengthening of inner stem suffix}), the passive constructions would thus be homophonous with past active constructions, except that the lengthening in the stressed root vowel is a clear index of the passive construction. It is possible that there is a change in progress where the lengthening is being reanalyzed as the marker of past passive.

\ea\label{ex: neutralized vowel quality of past passive suffix}
{Neutralized vowel quality of past passive suffix}

    \ea[]{
    \textit{naˈʔî ko ˈwé ˈtʃ͡\textbf{óo}rtiri}\\
    \gll    naˈʔî=ko ˈwé ˈtʃ͡óri-ti-ru\\
            here=\textsc{emph} \textsc{int} have.cramps-\textsc{caus-pst.pass}\\
    \glt    ‘People felt cramps here.’ \\
    \glt    ‘Aquí se acalambraba la gente.’  < BFL 05 2:41/el >\\
}\label{ex: neutralized vowel quality of past passive suffixa}
        \ex[]{
        \textit{ˈnè ko biˈlá ruˈw\textbf{èe}ri ˈwé kaˈníla ˈrá}\\
        \gll    ˈnè ko biˈlá ruˈw-è-ru ˈwé kaˈní-la ru-ˈwá\\
                1\textsc{sg.nom} \textsc{emph} really tell-\textsc{pst.pass} \textsc{int} happy-\textsc{rep} say-\textsc{mpass}\\
        \glt    ‘I was told he got really happy.’\\
        \glt    ‘Me contaron que se puso bien contento.’   < SFH 08 1:84/el >\\
    }\label{ex: neutralized vowel quality of past passive suffixb}
    \z
\z

While the lengthening in (\ref{ex: neutralized vowel quality of past passive suffixa}) could be alternatively analyzed as compensatory lengthening triggered by syncope, the lengthening in (\ref{ex: neutralized vowel quality of past passive suffixb}) cannot be attributed to compensatory lengthening, since there is no posttonic syncope in this form.

Passive-induced lengthening targets the root or a derivational suffix in the Inner Stem. (\ref{ex: past passive induced lengthening of inner stem suffixa}) shows the derivational suffix \textit{-rú} undergoing legthening. The same base does not undergo vowel lengthening with the active past suffix \textit{-li}.


\ea\label{ex: past passive induced lengthening of inner stem suffix}
Past passive induced lengthening of Inner Stem suffix

    \ea[]{
    \textit{naˈʔî rakiˈrúuru}\\
    \gll    naˈʔî raki-ˈrú-ru\\
            here palm-gather-\textsc{pst.pass}\\
    \glt    ‘Palms were gathered here.’\\
    \glt    ‘Aquí juntaron palmas.’   < SFH 08 1:97/el >\\
}\label{ex: past passive induced lengthening of inner stem suffixa}
        \ex[]{
        \textit{haˈsînto rakiˈrúli}\\
        \gll    haˈsînto raki-ˈrú-li\\
                Jacinto    palm-gather-\textsc{pst}\\
        \glt    ‘Jacinto gathered palms.’\\
        \glt    ‘Jacinto juntó palmas.’    < SFH 08 1:97/el >\\
    }\label{ex: past passive induced lengthening of inner stem suffixb}
    \z
\z

The target of passive-induced lengthening includes the transitive suffixes in position S2, in the Derived Stem domain, as shown in (\ref{ex: past passive induced lengthening of derived stem suffixes}).

\ea\label{ex: past passive induced lengthening of derived stem suffixes}
{Past passive induced lengthening of Derived Stem suffixes}

    \ea[]{
    \textit{miˈgêl tʃ͡aʔiˈbúuru siˈkâla}\\
    \gll    miˈgêl tʃ͡aʔi-ˈbú-ru siˈkâ-la\\
            Miguel get.stuck-\textsc{tr-pst.pass} hand-\textsc{poss}\\
    \glt    ‘Miguel’s hand got stuck (by somebody else).’\\
    \glt    ‘Le atoraron la mano a Miguel.’     < SFH 08 1:97/el >\\
}
        \ex[]{
        \textit{ˈmá  tʃ͡ihanâari napátʃ͡i}\\
        \gll    ˈmá tʃ͡iha-nâ-ru naˈpátʃ͡a\\
                already scatter-\textsc{tr-pst.pass} blouses\\
        \glt    ‘The blouses were thrown around.’\\
        \glt    ‘Ya desparramaron las blusas.’     < SFH 07 1:17-21/el >\\
    }
            \ex[]{
            \textit{ˈmá rapaˈnâaru}\\
            \gll    ˈmá rapa-nâ-ru\\
                    already split-\textsc{tr-pst.pass}\\
            \glt    ‘She was already operated (lit. cut).’\\
            \glt    ‘Ya la operaron (cortaron).’   < SFH 08 1:84/el > \\
        }
    \z
\z

On the other hand, suffixes on the Syntactic Stem (and any later morphological stem domains) block lenghtening of the stressed syllable in past passive constructions. In each example in (\ref{ex: no past passive induced  lengthening of syntactic stem suffixes}), the past passive suffix does not trigger lengthening of an immediately preceding applicative suffix.

\ea\label{ex: no past passive induced  lengthening of syntactic stem suffixes}
{No past passive induced lengthening of Syntactic Stem suffixes}

    \ea[]{
    \textit{ˈnè amaˈtʃ͡îkiru}\\
    \gll    ˈnè amaˈtʃ͡î-ki-ru\\
            1\textsc{sg.nom} pray-\textsc{appl-pst.pass}\\
    \glt    ‘They were praying for me’ (lit. ‘I was being prayed for’).'\\
    \glt    ‘Me rezaron.’     < SFH 05 2:105/el >  \\
}
        \ex[]{
        \textit{ˈmáni baʔiˈrúkuru baʔˈwí}\\
        \gll    ˈmá=ni baʔi-ˈrú-ki-ru baʔˈwí\\
                already=\textsc{1sg.nom} water-gather-\textsc{appl-pst.pass} water\\
        \glt    ‘They already brought me water.’\\
        \glt    ‘Ya me trajeron agua.’   < SFH 08 1:84/el >\\
    }
    \newpage
            \ex[]{
            \textit{niˈhê koˈbísi ˈpásiru}\\
            \gll    niˈhê koˈbísi ˈpá-si-ru\\
                    1\textsc{sg.nom} pinole throw-\textsc{appl-pst.pass}\\
            \glt    ‘They threw my pinole.’\\
            \glt    ‘Me tiraron el pinole.’    < SFH 08 1:85/el > \\
        }
    \z
\z

There are no constructions in the corpus where the past passive suffix imposes lengthening on a base including the suffixes in positions S3--S4 either, which implies that the domain of lengthening is restricted to the suffixes up to position S2, the Derived Stem domain. As shown in (\ref{ex: LEL passive data}), for at least some speakers, the past passive suffix is \textit{-liru} when attaching to V-final stems, with the liquid flap syllabifying as coda of the preceding syllable (due to general post-tonic vowel deletion and reduction process described in §\ref{subsubsec: stress-based vowel reduction and deletion}, the suffix surfaces as [-lri] in the examples below).

\largerpage
\ea\label{ex: LEL passive data}

    \ea[]{
    [sukuˈtʃ͡ûulri]\\
            /sukuˈtʃ͡û-liru/\\
            scratch-\textsc{pst.pass}\\
    \glt    `They were scratched.'\\
    \glt    `Fueron rasguñados.'  {<LEL 14 1:12/el >}\\
}
        \ex[]{
        [baˈkiâalri]\\
                /bakiˈjâ-liru/\\
                offer.corn.beer-\textsc{pst.pass}\\
        \glt    `They were offered corn beer.'\\
        \glt    `Les ofrecieron tesgüino.'\footnote{The \ili{Spanish} translation given for this verb is `franquear', which likely has a limited use in Northern Mexico among native \ili{Spanish} speakers; the drinking referred to is ritual drinking in Rarámuri communities associated with community-based work, where the host of the community work offers invitees corn beer to share.}   {<LEL 14 1:13/el >}\\
    }
            \ex[]{
            [rakiˈbûulri]\\
                    /rakiˈbû-liru/\\
                    push-\textsc{pst.pass}\\
            \glt    `They were pushed.'\\
            \glt    `Fueron empujados.   {<LEL 14 1:13/el >}\\
        }
                \ex[]{
                [rakiˈrûulri]\\
                        /raki-ˈrû-liru/\\
                        palm-gather-\textsc{pst.pass}\\
                \glt    `Palm was gathered.'\\
                \glt    `Juntaron palma.'  {<LEL 14 1:13/el >}\\
            }
    \z
\z

In these examples, the past passive suffix with the extra consonant not found with other speakers also triggers lengthening in the stressed syllable of the stem. This form of the suffix appears to reflect a more conservatve form of this suffix and reveals the source of morphologically-conditioned lengthening in compensatory lengthening. The cognate suffixes in \ili{Mountain Guarijío} and \ili{Norogachi Rarámuri} are disyllabic \textit{{}-reru} {\textasciitilde} \textit{-riru} and \textit{liri}, respectively (\citealt[][143]{miller1996guarijio}; \citealt[]{brambila1953gramatica}). What this suggests is that the source of vowel lengthening associated with the Choguita Rarámuri past passive arose through compensatory lengthening upon vowel (and eventually syllable) deletion of the suffix.

\subsubsection{Imperative singular stem formation: final stem stress shift and tonal alternations}
\label{subsubsec: imperative singular stem formation}

% Austin's notes on variability among speakers in the tonal realization in contexts of imperative sg. induced stress shifts - competing generalizations

% See latest version of grammatical tone manuscript

As described in §\ref{subsec: non-concatenative processes}, the imperative may be marked as final stem stress. This marking is restricted to be realized in a domain that includes the Derived Stem domain. Transitive stems of change-of-state predicates (described in §\ref{subsec: change of state predicates}) have an imperative with stress on the transitive suffix. This is shown in (\ref{ex: imperative stress shift 2a}).

\ea\label{ex: imperative stress shift 2}
{Imperative stress shift}

    \ea[]{
    \textit{Imperative, transitive}\\
    \glt {\textit{kasi\textbf{ˈnâ}}}\\
    \glt {\textit{kasi-\textbf{ˈnâ}}}  \\
    \glt break-\textsc{tr.imp.sg}\\
    \glt ‘Break it!'\\
    \glt `¡Rómpelo!'\\
}\label{ex: imperative stress shift 2a}
        \ex[]{
        \textit{Transitive stem + shifting suffix}\\
        \glt \textit{kasi\textbf{ˈnâ}ma}\\
        \glt    \textit{kasi\textbf{-ˈnâ}-ma}\\
        \glt break-\textsc{tr-fut.sg}\\
        \glt    `S/he will break it.'\\
        \glt    `Lo va a romper.'\\
    }\label{ex: imperative stress shift 2b}
            \ex[]{
            \glt     \textit{Transitive stem + neutral suffix}\\
            \glt     {\textit{ka\textbf{ˈsì}nali}}\\
            \glt    \textit{ka\textbf{ˈsì-}na-li}\\
                    break-\textsc{tr-pst}\\
            \glt    `S/he broke it.'\\
            \glt    `Lo rompió.'\\
        }\label{ex: imperative stress shift 2c}
    \z
\z

Stress on the transitive suffix is characteristic of transitive stems in shifting constructions (\ref{ex: imperative stress shift 2b}), and contrasts with second syllable stress of the same transitive stems in neutral constructions (\ref{ex: imperative stress shift 2c}). The imperative of intransitive change-of-state predicates will, on the other hand, involve fixed second syllable stress plus an imperative singular suffix (as in (\ref{ex: imperative change of state predicates})).

\ea\label{ex: imperative change of state predicates}
{Imperative of change-of-state predicates}

    \ea[]{
    \textit{kaˈsíka}\\
    \textit{kaˈsí-ka}\\
    break-\textsc{imp.sg}\\
    ‘Break yourself!’\\
    `¡Rómpete! < SFH 08 1:98/el > \\
}
        \ex[]{
        \textit{waˈtʃ͡íka}\\
        \textit{waˈtʃ͡í-ka}\\
        be.straight\textsc{-imp.sg}\\
         ‘Straighten up!'\\
         `¡Enderézate!' < SFH 08 1:98/el >\\
    }
    \z
\z

Since the transitive suffixes of change-of-state predicates are part of the Derived Stem, we can identify final stem stress to mark imperative as a process restricted to this verbal zone.

\largerpage[2]
\subsubsection{Round harmony}
\label{subsubsec: round harmony}

Choguita Rarámuri has a round harmony process,\footnote{I refer to this process as round harmony, although this process is gradient rather than categorical.} where non-round vowels of certain suffixes may become round when preceded by a stressed back stem vowel. The following examples show the role of stem stressed vowels as triggers of the rounding of the following suffix vowels: in (\ref{ex: round  harmony triggersa}) and (\ref{ex: round  harmony triggersc}), a stem final high, back vowel triggers rounding in the vowels of the causative, applicative and associated motion suffixes; in (\ref{ex: round  harmony triggersb}) and (\ref{ex: round  harmony triggersd}), on the other hand, there is no rounding of applicative suffix vowels with a stem final high, front vowel.

\ea\label{ex: round  harmony triggers}
{Round harmony triggers}

    \ea[]{
    \textit{Round harmony}\\
    {\textit{{baniˈsú\textbf{tusu}ma}}}\\
    \gll    {baniˈsú-ti-si-ma}\\
           {pull-\textsc{caus-mot-fut.sg}}\\
    \glt `S/he will go along making them pull it.'\\
    \glt `Los va a ir haciendo que lo jalen.' {< SFH 07 2:67 rec487 /el >}\\
}\label{ex: round  harmony triggersa}
        \ex[]{
        \textit{No harmony}\\
        \glt {\textit{tiˈtʃ͡í\textbf{ksi}ma}}\\
        \glt {tiˈtʃ͡í-ki-si-ma}\\
            comb-\textsc{appl-mot-fut.sg}\\
        \glt `S/he will go along combing her.'\\
        \glt `Lo va a ir peinando.' {< SFH 07 2:67 rec487/el >}\\
}\label{ex: round  harmony triggersb}
            \ex[]{
            \textit{Round harmony}\\
            \glt \textit{ʃuˈkú\textbf{ku}po}\\
            \gll {suˈkú-ki-po}\\
                {scratch-\textsc{appl-fut.pl}}\\
            \glt `They will scratch her.'\\
            \glt `La van a arañar.' {< BFL 05 1:116/el >}\\
        }\label{ex: round  harmony triggersc}
                \ex[]{
                \textit{No harmony}\\
                \glt {\textit{noˈkè\textbf{ki}lo}}\\
                \gll {noˈk-è-ki-li-o}\\
                     {move-\textsc{appl-appl-pst-ep}}\\
                \glt `S/he moved it for him.'\\
                \glt `Se lo movió (a él).' {< BFL 05 1:116/el >}\\
            }\label{ex: round  harmony triggersd}
    \z
\z

While round harmony in Choguita Rarámuri resembles other vowel harmony systems in its perseveratory, root-controlled nature, there is also evidence that harmony can be blocked or favored by the vocalic quality of an inflectional suffix following the target vowels (these suffixes are themselves outside of the domain of harmony). In (\ref{ex: anticpatory nature of round harmonya}) and (\ref{ex: anticpatory nature of round harmonyc}), the applicative suffix \textit{-ki} and the causative suffix \textit{-ti} are realized with a round vowel after a stem with a final back vowel \textit{if} the following inflectional suffix has a back vowel as well. The role of the final inflectional suffix in the harmony can be appreciated in (\ref{ex: anticpatory nature of round harmonyb}) and (\ref{ex: anticpatory nature of round harmonyd}), where the applicative and causative suffixes do not undergo round harmony when followed by an inflectonal suffix with a high, front vowel.

\ea\label{ex: anticpatory nature of round harmony}
{Anticipatory nature of round harmony}

    \ea[]{
    \textit{kupuˈrók\textbf{u}ma}\\
    \textit{kupuˈró-ki-ma}\\
    {blink-\textsc{appl-fut.sg}}\\
    `S/he will blink to her.'\\
    `Va a parpadearle.' {< BFL 05 2:22/el >}\\
}\label{ex: anticpatory nature of round harmonya}
        \ex[]{
        \textit{kupuˈrók\textbf{i}ki}\\
        \textit{kupuˈro-ki-ki}\\
        {blink-\textsc{appl-pst.ego}}\\
        `I blinked to her.'\\
        `Le parpadeó.' {< BFL 05 2:22/el >}\\
    }\label{ex: anticpatory nature of round harmonyb}
            \ex[]{
            \textit{kupuˈrót\textbf{u}ma}\\
            \textit{kupuˈró-ti-ma}\\
            {blink-\textsc{caus-fut.sg}}\\
            `S/he will make her blink.'\\
            `Va a hacer que parpadée.' {< BFL 05 2:22/el >}\\
        }\label{ex: anticpatory nature of round harmonyc}
                \ex[]{
                \textit{kupuˈrótiki}\\
                \textit{kupuˈró-ti-ki}\\
                {blink-\textsc{caus-pst.ego}}\\
                `I made her blink.'\\
                `La hice parpadear.' {< BFL 05 2:22/el >}\\
            }\label{ex: anticpatory nature of round harmonyd}
    \z
\z

This shows, then, than while clearly root-controlled, this process is also partially anticipatory. It also shows that back harmony is restricted to a subconstituent of the hierarchical structure of the verb. We have seen examples of round harmony targeting the vowels of causative \textit{-ti} (S4), applicative \textit{-ki} (S5), and associated motion \textit{-simi} (S7). The examples in (\ref{ex: round harmony in derived stem and syntactic stem}) show an array of suffixes undergoing rounding harmony. These suffixes belong in the Derived Stem, Syntactic Stem and Aspectual Stem domains.

\ea\label{ex: round harmony in derived stem and syntactic stem}
{Round harmony in the Derived Stem and the Syntactic Stem}

    \ea[]{
    {Applicative \textit{-si} (S3): Round harmony}\\
    \glt {\textit{ˈpá\textbf{ʃu}ru}}\\
    \glt    ˈpá-si-ru \\
            throw-\textsc{appl-pst.pass}\\
    \glt    ‘It was thrown in my direction (for me).’\\
    \glt    ‘Me lo tiraron (hacia mi).’  <RF 04 1:82/el >  \\
}\label{ex: round harmony in derived stem and syntactic stema}
        \ex[]{
        {Applicative \textit{-si} (S3): No harmony}\\
        \glt {\textit{ˈpásiki}}\\
        \glt    ˈpá-si-ki \\
                throw-\textsc{appl-pst.ego}\\
        \glt    ‘I threw it for him.’\\
        \glt    ‘Se lo tiré.’  < ROF 04 1:82/el >\\
    }\label{ex: round harmony in derived stem and syntactic stemb}
            \ex[]{
            {Applicative \textit{-ni} (S4): Round harmony}\\
            \textit{ˈnè ko biˈlé tʃ͡omaˈlî siˈrûnupa ˈlé}\\
            \gll    ˈnè=ko biˈlé tʃ͡omaˈlî   siˈrû-ni-pa aˈlé\\
                    1\textsc{sg.nom}=\textsc{emph} one deer hunt-\textsc{appl-fut.pass} \textsc{dub}\\
            \glt    ‘I will have a deer hunted.’\\
            \glt    ‘Me van a cazar un venado.’ < SFH 05 1:136/el >\\
        }\label{ex: round harmony in derived stem and syntactic stemc}
                \ex[]{
                 {Evidential \textit{-tʃane} (S9: Round harmony)}\\
                 \glt {{\textit{ˈsûuntʃ͡una}} \footnote{There is evidence that a following inflectional suffix with a front high vowel blocks the rounding harmony process: \textit{ˈsû-n-tʃ͡}\textit{an-i} /ˈsû-ni-tʃ͡ane-i/ ‘sow\textsc{-appl-ev-imp}’ ‘It used to sound like they were sowing stuff for her’ ‘Se oía como que le cosían’ (< SFH 07 1:9/el >)}}\\
                \glt    ˈsû-nale-tʃ͡ane-a\\
                \glt sow-\textsc{desid-ev-prog}\\
                \glt    ‘It sounds like she wants to sow.’ \\
                \glt    ‘Se oye como que quiere coser.’        < SFH 07 1:9/el >\\
            }\label{ex: round harmony in derived stem and syntactic stemd}
                    \ex[]{
                     \glt   {Evidential \textit{-tʃane} (S9: Round harmony)}\\
                     \glt    {\textit{miˈsútʃ͡una}}\\
                    \gll    miˈsú-tʃ͡ane-a\\
                           catch-\textsc{ev-prog}\\
                    \glt    ‘It sounds like they are catching (mice).’  \\
                    \glt    ‘Se oye como que andan atrapando ratones.’ < SFH 07 1:10/el >\\
                }\label{ex: round harmony in derived stem and syntactic steme}
    \z
\z

There are cases where harmony appears to be blocked: in (\ref{ex: blocked harmony}) there is no round harmony with the applicative and evidential suffixes, despite the presence of the trigger (a back, stressed vowel in the stem) and the following inflectional suffix with a back vowel. Instead, the back vowel of the evidential suffix has undergone height neutralization.

\ea\label{ex: blocked harmony}
{Blocked harmony}

    \textit{miˈsúkitʃ͡ina}\\
    \gll   miˈsú-ki-tʃ͡ane-a\\
             catch-\textsc{appl-ev-prog}\\
    \glt    ‘It sounds like they are catching (some mice) for somebody.’\\
    \glt    ‘Se oye como que le están atrapando ratones.' < SFH 07 1:10/el >\\
\z

We have seen that the [+round] feature can spread over more than one vowel (e.g. (\ref{ex: round harmony in derived stem and syntactic stema})), so it cannot be argued that harmony is limited in its rightward (or leftward) spreading. Instead, it is possible that this pattern has been rendered opaque by posttonic vowel height neutralization.

Finally, back round vowels also favor reduction of posttonic unstressed vowels to schwa. Posttonic vowel reduction to schwa occurs frequently when preceded by a back, stressed vowel. The examples below, however, show that reduction to schwa takes place after central (\ref{ex: vowel reduction to schwaa}) and front, mid vowels (\ref{ex: vowel reduction to schwab}) as well.

\ea\label{ex: vowel reduction to schwa}
{/i/ > ə / á, é \longrule}\\

    \ea[]{
    \textit{ˈnâr\textbf{ə}ma}\\
    \textit{ˈnâr\textbf{i}-ma}\\
    {ask-\textsc{fut.sg}}\\
    `S/he will ask.'\\
    `Va a preguntar.' {< SFH 05 1:86/el >}\\
}\label{ex: vowel reduction to schwaa}
        \ex[]{
        \textit{naˈtêp\textbf{ə}ma}\\
        \textit{naˈtêp\textbf{i}-ma}\\
        {greet-\textsc{fut.sg}}\\
        `S/he will greet her.'\\
        `Va a saludarla.' {< BFL 05 1:111/el >}\\
    }\label{ex: vowel reduction to schwab}
    \z
\z

The examples in (\ref{ex: vowel reduction to schwa}) thus shows that the gradient process of unstressed vowel reduction to schwa and round harmony do not appear in the same  vocalic environments.

In sum, rounding harmony in Choguita Rarámuri is stem-controlled but is simultaneously sensitive to outer inflectional suffixes, which are in turn out of the harmony domain. The targets of rounding harmony include the root and Inner Stem processes, as well as suffixes up to position S9. There is no evidence that other potential targets occurring in outer positions of the stem undergo rounding harmony. Thus, rounding harmony constitutes another phenomenon that contributes to creating less salient junctures between suffixes of certain inner domain of the verbal stem.

\subsubsection{Stress and the morphologically complex verb: stress-shifting and stress-neutral suffixes}
\label{subsubsec: Stress and the verb: stress-shifting and stress-neutral suffixes}

It has been shown so far that the agglutinating structure of the Choguita Rarámuri verb is not uniform with respect to its morpho-phonological properties. There is yet another important property of the verbal stem that suggests an internal, layered organization: the characterization of suffixes as stress-shifting and stress-neutral. This section discusses how suffixes are grouped into layers in the verb according to their stress properties. Stress-shifting suffixes combine with the Inner Stem and comform the stress domain, while non-shifting suffixes are outside the stressable domain. The stress properties of Choguita Rarámuri suffixes align with other properties that define the stem levels or domains, since stress-shifting and stress-neutral suffixes are grouped in interleaved layers in the morphological structure of the stem.

It has been proposed that accent systems where the interaction between prespecified information and word formation processes yields competing lexical accents, prosody is determined by morphology: a ``headmost'' accent wins, and the phonological properties of this morphological head percolate to the word level (\citealt[3--4]{revithiadou1998lexical}). Under this account, ``heads'' are characterized as derivational morphemes (not inflectional ones). In Choguita Rarámuri there is no correlation between the suffixes’ prosodic properties and their status as derivational or inflectional morphology.


\section{The verbal complex: clitics and modal particles}
\label{sec: the verbal complex clitics and modal particles}

As described in \chapref{chap: particles, adverbs and other word classes}, pronominal forms have corresponding enclitic forms, which are prosodically dependent  on their host that do not carry any restrictions about the syntactic category of the words they attach to \parencite{bickel2007inflectional}. \tabref{tab:key:21} lists the clitic pronominal forms (free pronouns are given in parenthesis). Third person is marked with  a demonstrative \textit{mi}, both as a free form and as an enclitic.

\begin{table}
\caption{Pronominal enclitic forms}
\label{tab:key:21}

\begin{tabularx}{.5\textwidth}{lll}
\lsptoprule
& \textbf{Subject} & \textbf{Object}\\
\midrule
 \textsc{1sg} & =ni (neˈhê) & (taˈmí)\\
 \textsc{2sg} & =mi (muˈhê) & (ˈmí)\\
 \textsc{1pl} & =ti (tamuˈhê/taˈmò) & (taˈmí)\\
 \textsc{2pl} & =timi (ˈémi) & (ˈmí)\\
\lspbottomrule
\end{tabularx}
\end{table}
%\hspace{3cm}

%cross-reference here to morphology of small word classes
Choguita Rarámuri also has epistemic modality markers (the morphological characteristics of these markers are addressed in \chapref{chap: particles, adverbs and other word classes}). Epistemic modality, or the expression of the degree of certainty speakers have towards the actuality of an event, is marked in Choguita Rarámuri through two modal particles that follow inflected verbs: \textit{aˈlé}, which expresses doubt and uncertainty (\ref{ex: epistemic modality markers 2a}), and \textit{oˈlá}, which marks certainty, and often volition (\ref{ex: epistemic modality markers 2b}). Forms lacking such particles have a neutral interpretation with respect to the speaker’s commitment to the truth value of the proposition.

\ea\label{ex: epistemic modality markers 2}
{Epistemic modality markers }

    \ea[]{
    \textit{ˈnârma ˈlé}\\
    \gll    ˈnâri-ma aˈlé \\
            ask-\textsc{fut.sg} \textsc{dub}\\
    \glt    ‘(He) will probably ask.’\\
    \glt    ‘Probablemente va a preguntar.’    <BL 05 1:152/el >\\
}\label{ex: epistemic modality markers 2a}
        \ex[]{
        \textit{ˈnârmo ˈlá}\\
        \gll    ˈnâri-ma oˈlá \\
                ask-\textsc{fut.sg} \textsc{cer} \\
        \glt    ‘S/he will definetly ask.’\\
        \glt    ‘Seguramente que va a preguntar.’     <BL 05 1:152/el >\\
    }\label{ex: epistemic modality markers 2b}
    \z
\z

As the examples in (\ref{ex: epistemic modality markers 2}) show, these particles have the phonological effect of inducing vowel deletion of the final vowel of the singular future suffix. This phenomenon has led some to describe these epistemic elements as ``suffixes with independent stress'' in other Rarámuri dialects (\citealt{Burgess-1984}). These particles, however, show their independent-word status through their prosodic independence, and their ability to appear after person clitics. In a strong hypothesis of syntax-phonology interactions, cliticization follows syntax, which predicts that clitics are able to attach to other clitics, but affixes cannot attach to clitics \parencite{zwicky1983cliticization}.

Finally, deletion between the final future suffix and the epistemic particles takes place with an intermediate enclitic that has lost its vowel. This is exemplified in (\ref{ex: vowel deletion induced by epistemic markers}).

\ea\label{ex: vowel deletion induced by epistemic markers}
{Vowel deletion induced by epistemic markers}

    \ea[]{
    \textit{tʃ͡aʔiˈmê\textbf{o}n ˈlá}\\
    \gll    tʃ͡aʔi-ˈmêa=ni oˈlá\\
            stuck-\textsc{fut.sg=1sg.nom} \textsc{cer}\\
    \glt    ‘I will get stuck.’\\
    \glt    ‘Me voy a atorar.’ \  < BFL 05 1:133/el >  \\
}
\newpage
        \ex[]{
        \textit{baˈhîm\textbf{o}n oˈlá}\\
        \gll    baˈhî-ma=ni oˈlá\\
                drink-\textsc{fut.sg=1sg.nom} \textsc{cer} \\
        \glt    ‘I will drink.’  \\
        \glt    ‘Voy a tomar.’  < AHF 05 2:101/el >\\
    }
            \ex[]{
            \textit{riˈkùm\textbf{a}n ˈlé}\\
            \gll    riˈkù-ma=ni aˈlé\\
                    be.drunk-\textsc{fut.sg=1sg.nom} \textsc{dub}\\
            \glt    ‘I will probably get drunk.’\\
            \glt    ‘Probablemente me voy a emborrachar.’   < BFL 05 2:120/el >\\
        }
                \ex[]{
                \textit{riˈkùm\textbf{o}n oˈlá}\\
                \gll    riˈkù-ma=ni  oˈlá\\
                        be.drunk-\textsc{fut.sg=1sg.nom} \textsc{cer}\\
                \glt    ‘I will get drunk.’\\
                \glt    ‘De seguro me voy a emborrachar.’  < BFL 05 2:120/el >\\
            }
    \z
\z

Vowel deletion takes place post-lexically after the intermediate person clitic loses its vowel.

\section{Summary}
\label{sec: chapter summary}

The verbal structure of Choguita Rarámuri displays morphotactic, prosodic and morpho-phonological properties that define a concentric organization of suffixes, with more fused suffixes closer to the root and more separable suffixes in the outer layer of the verb. The verbal structure scheme proposed in this chapter is repeated in \tabref{tab:stem-levels}.

\begin{table}[t]
\caption{Choguita Rarámuri verbal stem domains}
\label{tab:stem-levels}

\begin{tabularx}{.85\textwidth}{lQl}
\lsptoprule
\textbf{Positions} & \textbf{Categories}  & \textbf{Stem domain}\\
\midrule
 & {Root + unproductive}

 {and semiproductive processes} & {Inner Stem}\\
S1 & Inchoative & {Derived Stem}\\
S2 & Transitive & \\
S3 & Applicative & {Syntactic Stem} \\
S4 & Causative & \\
S5 & Applicative & \\
S6 & Desiderative & {Aspectual Stem}  \\
S7 & Associated Motion & \\
S8 & Auditory Evidential & \\
S9 & Voice/Aspect/Tense & {Finite Verb}\\
S10 & Mood & \\
S11 & TAM & \\
S12 & Deverbal morphology & {Subordinate Verb}\\
\lspbottomrule
\end{tabularx}
\end{table}

Despite having mostly a fixed position, the ordering of suffixes is not arbitrary and conforms to general principles. There are no discontinuous dependencies across suffix positions, as is frequent in position class morphologies. The structure proposed, instead, fits Bybee’s lexical-derivational-inflectional continnum, and generally conforms to the universal principles of relevance, derivation within inflection and scope. This is a property attributed to layered morphologies vs. templatic or position class morphologies, where general semantic and syntactic principles do not determine the whole range of affix ordering facts (\citealt{bickel2007inflectional}, \citealt{stump1993position}). In the analysis proposed here, the layered structure of the Choguita Rarámuri verb accounts for a zone of variable order, multiple exponence in the Syntactic Stem, and the fixed order in the rest of the zones.

\newpage

~

