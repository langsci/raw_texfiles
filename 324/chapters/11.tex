\chapter{Prosody: domains and interactions}
\label{chap: prosody}

This chapter is devoted to prosodic structures and processes that cross-cut the grammar of Choguita Rarámuri, including lexical phonological processes, morphological processes and post-lexical phonological phenomena. Choguita Rarámuri is a prosodically complex language, with both phonological and morphological factors affecting the realization of prosodic phenomena. Many aspects of the description addressed in this chapter are already provided in other parts of the grammar, including \chapref{chap: tone and intonation} and \chapref{chap: sentence types}. This chapter is devoted to examining in detail the interactions of prosodic patterns and processes across grammatical domains in this \ili{Uto-Aztecan} language with the goal of addressing how these complex interactions yield surface forms in the language.

The structure of this chapter is as follows. §\ref{sec: defining the prosodic word and other prosodic domains in CR} presents the criteria for determining the Prosodic Word in Choguita Rarámuri, and discusses the domains for phonological and morpho-phonological processes below the level of the Prosodic Word. §\ref{sec: vowel length, stress and minimality effects} discusses the relationship between different phenomena in Choguita Rarámuri that are quantity-sensitive and the lack of contrastive vowel length in the language. §\ref{sec: prosodic properties of morphologically complex verbs} addresses the interaction between stress and tone and between lexical tone and grammatical tone in morphologically complex words. §\ref{sec: interaction between lexical tone and intonation} summarizes the tonal and non-tonal encoding of intonation in Choguita Rarámuri and the interaction between lexical tone and intonation in this language. Finally, §\ref{sec: prosodic constraints on morphological shapes} addresses prosodic constraints associated to different morphological constructions and their relationship to the general prosodic organization of the language.

\section[Defining the Prosodic Word and other prosodic domains]{Defining the Prosodic Word and other prosodic domains in Choguita Rarámuri}
\label{sec: defining the prosodic word and other prosodic domains in CR}

In this grammar, I adopt the proposal in \citet{selkirk1980prosodic, selkirk1996prosodic, nespor1986prosodic} and \citet{hayes1989prosodic} for the prosodic hierarchy schematized in (\ref{ex:11:prosodic hierarchy}), where the Prosodic Word is the smaller unit within the hierarchy and the Intonational Phrase its largest constituent.

%\break
\ea\label{ex:11:prosodic hierarchy}
{The prosodic hierarchy}\mbox{}

   Intonational Phrase

     \hspace*{1cm}⎜

   Phonological Phrase

     \hspace*{1cm}⎜

   Clitic Group

     \hspace*{1cm}⎜

   Prosodic Word
   \z

The Prosodic Word in Choguita Rarámuri can be identified through phonotactic constraints and as the domain of application of several phonologically-general processes. These diagnostic criteria include:

\ea\label{ex: prosodic word}
 {The prosodic word in Choguita Rarámuri}
	\ea[]{
     Each prosodic word is assigned a single, main stress within the first three syllables, which bears a lexical tone (§\ref{subsec: stress and stress-dependent phenomena}).\\
}
    	\ex[]{
    	 Inflected verbs are minimally bimoraic (§\ref{subsec: minimal word size}).\\
    }
    		\ex[]{
    		 Prosodic words are vowel final (§\ref{subsubsec: stress-based vowel reduction and deletion}).\\
    	}
    			\ex[]{
    			 Body part incorporation combines two morphological roots into a single prosodic word, where the new lexical item is assigned a single, main stress in the first syllable of the head of the compound (§\ref{subsec: body-part incorporation}).\\
    		}
    				\ex[]{
    				 The glottal stop only emerges within the first two syllables of the prosodic word (§\ref{subsec: glottal stop}).\\
    			}
    				\z
   				\z

Taken together, these diagnostic criteria motivate the identification of prosodic words in Choguita Rarámuri. While the domain of stress assignment and of the distribution of the glottal stop are coextensive with the left edge of the prosodic word, the right edge boundary of the prosodic word is characterized by avoidance of vowel reduction and deletion.\footnote{This restriction of the right edge may have a functional motivation, since TAM inflectional values are often encoded by single vowel exponents that may replace the final vowel of the stem to which they attach (see Chapter \ref{chap: verbal morphology}).} Thus, identifying the Prosodic Word in Choguita Rarámuri involves a convergence of evidence that characterizes different edges of this prosodic unit. The minimality condition, which requires all Prosodic Words in the language to be minimally bimoraic, and the requirement of having a single stress per Prosodic Word, are the only criteria that hold for entire Prosodic Words in this language.

There is also evidence for phonological domains below the word level. As discussed in detail in \chapref{chap: verbal morphology}, Choguita Rarámuri exhibits a large number of phonological processes that apply to restricted subconstituents of complex verbs. The notion of morphological domains for phonological processes is thus an important one in this language, as well as the notion of mismatches between phonological and morphological domains, as analyzed in Bantu languages (\citealt{hyman1987prosodic,hyman1998positional,hyman2008directional}; \citealt{odden1996phonology}; \citealt{myers1998surface}; \citealt{downing2001generalizable}; \citealt{bickmore2007cilungu}), and Athabaskan languages (\citealt{mcdonough2000bipartite}), among other language families (see also \citealt{inkelas2014interplay} for an overview). These domains are motivated in Chapter \ref{chap: verbal morphology}, and can be summarized as follows in terms of the phonological processes attested in the language.

\ea\label{ex: verb domains for phonological processes}
 {Choguita Rarámuri verbal domains for phonological processes}
	\ea[]{
     \textbf{Haplology}: in a sequence of two unstressed, syllables with identical onset consonants at the Inner Stem plus suffix boundary, one syllable gets deleted.\\
    }
		\ex[]{
         \textbf{Compensatory lengthening}: deletion of a vowel or syllable in a suffix triggers compensatory lengthening of an adjacent Inner Stem stressed vowel.\\
        }
				\ex[]{
                 \textbf{Passive-conditioned lengthening}: the past passive \textit{-ru} suffix triggers lengthening of an adjacent Inner Stem stressed vowel.\\
                }
                	\ex[]{
                     \textbf{Imperative singular stress shift}: the imperative singular may be encoded morphologically via a rightward stress shift within the Derived Stem.\\
                    }
                    	\ex[]{
                         \textbf{Round vowel harmony}: non-round vowels of certain suffixes may become round when preceded by a stressed round stem vowel within the Aspectual Stem.\\
                        }
                        \z
                	\z

An additional domain below the word level is a constituent composed by the Stem (root plus any stem-forming suffixes), plus the first morphological layer that attaches to the Stem (see \chapref{chap: phonology} and \chapref{chap: verbal morphology}), whatever this morphological layer may be. Furthermore, in terms of stress assignment, the classification of constructions in shifting vs. neutral distinguishes suffixing constructions that are either \textit{cohering}, forming part of the stress domain, or \textit{non-cohering}, being outside of the stress domain, respectively (\chapref{chap: phonology}). The cohering vs. non-cohering distinction represents a structural mismatch between phonological domains and morphological subconstituents: cohering suffixes form a single phonological domain with the base they attach to, while non-cohering form a separate domain.

Finally, this chapter addresses whether some phonological and morphophonological processes in the language are better analyzed as being constrained by metrical structure (see §\ref{subsec: prosodic templates in CR}).

\section{Vowel length, stress and minimality effects}
\label{sec: vowel length, stress and minimality effects}

Recall from \chapref{chap: phonology} that there is no evidence for contrastive vowel length in Choguita Rarámuri, yet there is evidence that quantity plays a role in the phonological grammar of the language:

\ea\label{ex:the role of quantity in CR}
 {The role of quantity in Choguita Rarámuri}
	\ea[]{
     A minimal prosodic word is bimoraic (§\ref{sec: vowel length, stress and minimality effects}).\\
    }
		\ex[]{
		Vowel deletion may trigger compensatory lengthening (§\ref{subsubsec: compensatory lengthening}).\\
    }
        	\ex[]{
            The past passive suffix conditions vowel lengthening in the stressed stem syllable (§\ref{subsubsec: past passive-conditioned lengthening}).\\
        }
      	\z
   	\z

While quantity distinctions are referenced by different phonological and morphophonological processes, vowel length is not relevant for stress assignment in Choguita Rarámuri. As discussed in \citet{caballero2008choguita} and \citet{caballero2011morphologically} (and in more detail in §\ref{subsec: stress patterns and metrical feet} below), a metrical analysis of stress assignment in Choguita Rarámuri involves an unbounded, weight-insensitive iambic system. While typologically rare and claimed to be universally dispreferred \parencite{hayes1995metrical}, other similar cases have been documented in the literature (see, e.g., \citealt{graf2003emergent}). As discussed in \chapref{chap: phonology}, \ili{Proto-Uto-Aztecan} has been reconstructed as having a weight-sensitive, stress-accent system \parencite{munro1977towards}. Choguita Rarámuri stress might then have become lexicalized when vowel length distinctions from the proto-language were lost.

That vowel length in Choguita Rarámuri is relevant for some, but not all, phonological phenomena in the language resembles a case of what \citet{fitzgerald2012prosodic} terms \textit{prosodic inconsistency} in \ili{Tohono O'odham}, a related \ili{Uto-Aztecan} language from the \ili{Tepiman} branch. In contrast to Choguita Rarámuri, \ili{Tohono O'odham} does have contrastive vowel length. Quantity, however, exhibits an asymmetry in terms of its role in the phonological system of the language, with a quantity-insensitive stress system and a quantity-sensitive prosodic morphological patterns.

\section{Prosodic properties of morphologically complex verbs}
\label{sec: prosodic properties of morphologically complex verbs}

Stress systems are defined as those in which there is an increased prominence associated with one or more syllables in a word \parencite{gordonvanderhulst2018stress}. Choguita Rarámuri displays both phonological and phonetic properties that are characteristic of stress systems, including culminativity (each lexical word has at most one syllable which carries the highest degree of prominence) and obligatoriness (each lexical word has at least one syllable that carries the highest degree of prominence) (\citealt{hyman1977nature,hyman1978tone,hyman2006word}; \citealt{beckman1986stress}; \citealt{hayes1995metrical}), as well as unstressed vowel deletion and reduction (fewer vocalic contrasts are realized in unstressed syllables) (\citealt{caballero2008choguita,caballero2011morphologically}). Choguita Rarámuri is also a tonal language, fitting the definition of a “\ldots\, language \ldots\, in which an indication of pitch enters into the lexical realization of at least some morphemes", i.e., tone is the output of lexical phonology \citep[][229]{hyman2006word}. Stress and tone in Choguita Rarámuri are not only phonologically distinct systems, but they are also encoded through independent acoustic means \parencite{caballero2015tone}.

The surface form of morphologically complex verbs in this language results from the stress and tone properties of root morphemes and morphological constructions and complex interactions between phonologically general processes and morphologically governed phonological processes. Stress and tone assignment apply within a morphological domain, namely the Stem. This section synthesizes the different components necessary in deriving the surface prosodic forms of inflected verbs.

\subsection{Stress patterns and metrical feet}
\label{subsec: stress patterns and metrical feet}

The Choguita Rarámuri stress-accent system can be analyzed in metrical terms (\citealt{caballero2008choguita,caballero2011morphologically}). The generalizations about the distributional properties of word-level stress in Choguita Rarámuri are summarized in (\ref{ex:CR stress-accent properties}) (see also Chapter ~\ref{chap: word prosody}):

\ea\label{ex:CR stress-accent properties}
{Choguita Rarámuri word-level stress properties}
	\ea[]{
     Each prosodic word has a single main stress.\\
    }
    	\ex[]{
         There is no secondary stress (i.e., stress is non-iterative).\\
        }
        	\ex[]{
             Roots can be lexically stressed (and have fixed stress across paradigms) or lexically unstressed (with shifting stress in certain morphological contexts).\\
            }
            	\ex[]{
                 Morphological constructions are either stress-shifting, triggering stress shifts with unstressed roots, or stress-neutral, triggering no stress changes.\\
                }
                	\ex[]{
                     Stress-shifting suffixes are part of the stressable domain, while stress-neutral suffixes are outside the stressable domain and never stressed.\\
                    }
                    	\ex[]{
                         In words containing unstressed roots and a stress-neutral construction, stress is assigned by default on the second syllable of the root (the only syllable of monosyllabic roots) (see §\ref{sec: stress properties of roots, stems and suffixes}).\\
                        }
                        	\ex[]{
 							 In words containing unstressed roots and a stress-shifting construction, stress is assigned on the third syllable of the prosodic word.\\                                 }
								\ex[]{
                                 Noun incorporation constructions have a construction-specific stress rule (stress the first syllable of the head of the construction, the second element).\\
                                }
                                	\ex[]{
                                     Stress is restricted to appear within an initial three-syllable window.\\
                                    }
                                    \z
                              	\z

While there is no secondary stress assignment in the language, a metrical analysis can nevertheless be posited. Default stress assignment, second syllable stress within the root, can be modeled using a single iambic foot aligned to the left edge of the prosodic word, as exemplified in (\ref{ex: iambic foot parsing in default stress assignment}), with words containing the stress-neutral past \textit{-li} suffix:

\ea\label{ex: iambic foot parsing in default stress assignment}
{Single iambic foot parsing in default stress assignment}
 	\ea[]{
      \quadruplebox{(tʃ͡aˈpí)li}{tʃ͡aˈpí-li}{grab-\textsc{pst}}{`S/he grabbed it.'}\\
     }
     	\ex[]{
     	\quadruplebox{(raˈʔì)tʃ͡ali}{raˈʔìtʃ͡a-li}{speak-\textsc{pst}}{`S/he spoke.'}\\
         }
    \z
\z

\hspace*{-2.3pt}The metrical analysis of morphologically complex words containing unstressed roots and stress-shifting constructions (with third syllable stress), as well as noun incorporation constructions (stress in the first syllable of the construction's head, the third syllable), may be analyzed through a ternary metrical constituent, an iambic foot with a single left-adjoined syllable (\citealt{zoll2004ternarity}; see also \citealt{selkirk1980prosodic, dresher1991germanic, rice1993binarity, ito1992weak}, and \citealt{blevins1999trimoraic}). This ternary constituent is schematized in \figref{fig: ternary foot}, where F = Foot and Ad-Foot = Adjoined Foot).

\begin{figure}

\begin{forest}
  [AD-FOOT
    [(<σ>,tier=bottom]
    [\textbf{F}
        [<σ>,tier=bottom]
        [<σ>),tier=bottom]
    ]
  ]
\end{forest}
% \includegraphics[width=0.3\textwidth]{figures/Prosody-img1.png}
\caption{
\label{fig: ternary foot}
Ternary metrical constituent with single left-adjoined syllable \parencite{zoll2004ternarity}}
\end{figure}


\hspace*{-2.3pt}The metrical analysis of morphologically complex words containing unstressed roots and stress-shifting suffixes (exemplified with the stress-shifting conditional \textit{-sâ} suffix) is provided in (\ref{ex: single extended iambic foot parsing}). The metrical analysis of noun incorporation constructions is provided in (\ref{ex: single extended iambic foot parsing in noun incorporation}).

\ea\label{ex: single extended iambic foot parsing}
{Single, extended iambic foot parsing in suffixing constructions}
    \ea[]{
    (<tʃ͡a>piˈsâ) \\
    /tʃ͡api-ˈsâ/\\
    grab-\textsc{cond}  \\
    `if s/he were to grab it'\\
    `si lo agarra'\\
}
        \ex[]{
        (<ra>ʔiˈtʃ͡â)sa\\
        /raʔiˈtʃ͡â-sa/\\
        speak-\textsc{cond}	\\
        `if s/he were to speak'\\
        `si hablara' \corpuslink{el1274[13_521-13_550].wav}{JLG el1274:13:52.1}\\
    }
    \z
\z

\ea\label{ex: single extended iambic foot parsing in noun incorporation}
{Single, extended iambic foot parsing in noun incorporation constructions}
    \ea[]{
    (<si>waˈbô)ta	\\
    /siwa-ˈbôta/	\\
    tripe-come.out	\\
    `Its tripes come out.'\\
    `Se destripa.'\\
}
        \ex[]{
        (<bu>siˈkâ)si	\\
        /busi-ˈkâsi/	\\
        eye-break	\\
        `It becomes blind.'\\
        `Se vuelve ciego.'\\
    }
    \z
\z

In sum, a metrical analysis of Choguita Rarámuri involves positing a single metrical constituent, a binary iambic foot or an extended ternary iambic foot, at the left edge of the prosodic word. Given that there is no role for contrastive vowel length in stress assignment, this is a weight-insensitive iambic system. The distribution of feet is morphologically governed: binary feet in neutral morphological environments and ternary metrical feet in shifting morphological environments. Further discussion of the use of multiple types of metrical feet in different morphological environments can be found in \citet{bennett2012foot} and \citet{bennett2013uniqueness}. Further discussion of the use of ternary feet include \citet{martinez2013exploration} and \citet{martinez2015binary}.

\subsection{Lexical tone patterns}
\label{subsec: lexical tone patterns}

%edit this section
As discussed in \chapref{chap: tone and intonation}, Choguita Rarámuri has three lexical tones: falling (/HL/, <ô>), low (/L/, <ò>), and high (/H/, <ó>). Lexical tones are exclusively realized on surface stressed syllables, i.e., there is only one lexical tone per prosodic word and stressless syllables lack lexical tone.  The examples in (\ref{ex:tone minimal pairs}) exemplify the three-way tonal contrast \citep [][465]{caballero2015tone} in both nouns and verbs.

\ea\label{ex:tone minimal pairs}
{Tonal (near-)minimal pairs}
\setlength{\tabcolsep}{4pt}
\begin{tabular}{lllll}
       & \textit{Lexical tone} & \textit{Form} & \textit{Gloss} & \textit{Source}\\
    a. & HL & \textit{tô}&‘to bury' & \corpuslink{el1240[03_295-03_300].wav}{MAF el1240:3:29.5}\\
    b. & L & \textit{tò} & ‘to take' &\corpuslink{el1274[01_261-01_269].wav}{JLG el1274:1:26.1}\\
    c. & HL & \textit{mê} & `to win' & \corpuslink{el1242[01_563-01_570].wav}{MAF el1242:1:56.3}\\
    d. & H & \textit{mé} & `to bring' & \corpuslink{el83[01_179-01_195].wav}{SFH el83:1:17.9}\\
    e. & L & \textit{mè} & ‘mezcal’ & \corpuslink{co1140[02_066-02_087].wav}{MDH co1140:2:06.6}\\
    f. & H & {\textit{pá}} & {‘to throw’}& \corpuslink{el1240[02_468-02_477].wav}{MAF el1240:2:48.8}\\
    g. & L & {\textit{pà}}&{‘to bring’}& \corpuslink{el728[02_479-02_487].wav}{BFL el728:2:44.9}\\
    h. & HL & {\textit{kolî}}&{‘chile pepper’}& \corpuslink{el728[05_588-05_599].wav}{BFL el728:5:11.6}\\
    i. & L & {\textit{kolì}}&{`spatial root’}& \corpuslink{el549[03_527-03_544].wav}{SFH el549:3:52.7}\\
    j. & HL & {\textit{isî}}&{‘to urinate’} & \corpuslink{el728[04_018-04_027].wav}{BFL el728:03:24.8}\\
    k. & L & {\textit{isì}}&{‘to do’} & \corpuslink{el728[03_497-03_506].wav}{BFL el728:3:53.8}\\
    l. & H & {\textit{niʔwí}}& {‘to be lightning’}& \corpuslink{co1237[10_527-10_538].wav}{JLG co1237:10:52.7}\\
    m. & L & {\textit{niwì}}&{‘to marry’} & \corpuslink{el1318[28_549-28_565].wav}{MFH el1318:28:54.9}\\
    n. & HL & {\textit{nôtʃ͡a}}&{‘pretentious'} & \corpuslink{el728[08_035-08_044].wav}{BFL el728:8:03.5}\\
    o. & L & {\textit{nòtʃ͡a}}&{‘hard working'} & \corpuslink{el728[07_417-07_425].wav}{BFL el728:7:41.7}\\
\end{tabular}

\z

\hspace*{-2.7pt}Tone is not dependent on voicing of preceding consonants nor any other phonological features, and there is no evidence of tone spreading. The Tone-Bearing-Unit (TBU) is the mora: falling tones have their high target on the stressed syllable, with the fall continuing through a post-tonic syllable, if there is one (H*L) \citep{caballero2015tone}. As described above, stressless syllables lack lexical tones: the surface tonal properties of these syllables vary depending on intonation \citep{garellek2015lexical}.

\subsection{Canonical prosodic shapes of roots and suffixes}
\label{subsec: canonical prosodic shapes of roots and suffixes}

%% update numbers of nominal and verbal roots
Most roots in Choguita Rarámuri are disyllabic or trisyllabic: from a corpus of 1,004 nominal and verbal roots, 47\% percent are disyllabic and 40\% trisyllabic. There are tetrasyllabic roots (10\%), but most of these are, with different degrees of transparency, internally complex, so a maximally tetrasyllabic prosodic size cannot be established. Finally, there are also monosyllabic roots. While a mere 3\% of the corpus, the existence of these roots make it impossible to establish a disyllabic minimality restriction for open class lexical morphemes.\footnote{When considering the proportions of roots separately for nouns and verbs, there are some differences: most monosyllables are verbs (5\% of verbs are monosyllabic and only 2\% of nouns are monosyllabic), and most tetrasyllables are nouns (19\% of nouns are tetrasyllabic, while only 5\% of verbs are tetrasyllabic).}

As for suffixes, most are monosyllabic, but there are a few disyllabic suffixes, including the Desiderative \textit{-ˈnále} suffix, the Associated Motion \textit{-simi} suffix, the Habitual Passive \textit{-ˈrîwa} suffix, the Future Singular \textit{-ˈmêa} suffix,\footnote{\citet{miller1996guarijio} has proposed that the disyllabic future singular suffix \textit{-mêa}, also present in \ili{Guarijío}, derives from the reconstructed verb \textit{*mi(l)a} ‘run, go’ \citep[133]{miller1996guarijio} from “\ili{Proto-Sonoran}”. Miller defines “\ili{Sonoran}” as a sub-branch of \ili{Uto-Aztecan} located in Mexico’s northwest which would include \ili{Tepiman} and \ili{Taracahitic} languages.} the Auditory Evidential \textit{-tʃ͡ane} suffix, and the Indirect Causative \textit{-nula} suffix. Almost all of these disyllabic suffixes are related to synchronically active free lexical morphemes with varying degrees of transparency (Desiderative \textit{-nále} and the verb \textit{naki}, ‘want’; Associated Motion \textit{-simi} and the verb \textit{simi}, ‘go singular’; Auditory Evidential \textit{-tʃ͡ane} and the verb \textit{tʃ͡ane}, ‘make noise, say’). Suffixes, like roots, are thus not restricted to a canonical prosodic shape. \footnote{Althhough, as discussed in §\ref{subsec: truncation in aspect/mood marking constructions} below, a set of disyllabic suffixes undergo truncation in certain contexts.}

While no generalizations in terms of maximal size applies to roots and no generalization about minimal or maximal size applies for suffixes, there are certain formal characteristics that usually apply to roots vs. suffixes. Given the stress properties of roots (with mainly second or third syllable stress) and the post-tonic vocalic reduction and deletion processes operating in the language (described above in \chapref{chap: word prosody}) the boundaries between roots and suffixes are often the target of posttonic syncope. This, in turn, yields derived stems with shared formal properties, such as final consonants. Roots derived with the productive causative suffix \textit{-ti} are a good example of this, since posttonic syncope often targets the vowel of the causative suffix, generating a class of causative stems ending in a lateral flap.  These causative stems are then recursively suffixed with the allomorph \textit{-ti} of the causative suffix. This is shown in the examples provided in (\ref{ex:causative stem example}):

%\pagebreak

\ea\label{ex:causative stem example}
{Causative stems and recursive affixation}
	\ea[]{
	\textit{ˈnè uˈbârtipo}\\
  	\gll 	ˈnè uˈbâ\textbf{-ri}-ti-po\\
			1\textsc{sg.nom} bathe-\textsc{\textbf{caus}}-\textsc{caus-fut.pl}\\
	\glt 	‘I’ll be forced to bathe.’ \\
	\glt 	‘Van a hacer que me bañe.’ 	< LEL 05 ECME(40)/el > \\
    }
		\ex[]{
		\textit{ˈnè ˈmí ˈmêrtima oˈlá}\\
        \gll 	ˈnè ˈmí ˈmê-\textbf{ri}-ti-ma oˈlá\\
        		1\textsc{sg.nom} 2\textsc{sg.acc} win-\textsc{\textbf{caus}-caus-fut.sg} \textsc{cer}\\
        \glt 	‘I will make you win.’\\
        \glt 	‘Te voy a hacer ganar.’	< LEL 05 ECME(3)/el >\\
        }
			\ex[]{
			\textit{koˈʔártinale}\\
            \gll 	koˈʔá-\textbf{ri}-ti-nale\\
            		eat-\textsc{\textbf{caus}-caus-desid}\\
            \glt 	‘She wants to make him eat.’\\
            \glt 	‘Quiere hacerlo comer.’	< BFL 06 ECDW(55)/el >\\
            }
    	\z
	\z

Syncope, thus, participates in the creation of a new paradigm of causative stems with particular word-internal (stem-delimiting) codas.

\subsection{Prosodic properties of roots and morphological constructions}
\label{subsec: prosodic properties of roots and morphological constructions}

As discussed in \chapref{chap: word prosody}, root morphemes belong to two classes depending on their stress behavior: (lexically) stressed roots retain stress in a fixed syllable in different morphological environments, while in (lexically) unstressed roots the location of stress is morphologically governed. The analysis proposed in this grammar is that this asymmetric behavior in terms of stress results from an underlying phonological difference between the two root classes, with stressed roots being lexically prespecified with a diacritic that is phonetically realized as stress in surface forms, and unstressed roots lacking this diacritic and receiving stress by default \parencite{caballero2011multiple}. All prosodic words in Choguita Rarámuri, whether they contain a lexically stressed root or an unstressed lexical root, have surface stress, a syntagmatic prominence cued acoustically via intensity and duration \parencite{caballero2015tone}.

\hspace*{-4.3pt}Morphological constructions (affixes and non-concatenative morphological processes) also exhibit an asymmetric phonological behavior. They are classified as belonging to one of two classes in terms of their stress properties: they may be stress-shifting, triggering stress shifts when attaching to unstressed roots, or stress-neutral, not conditioning any stress changes in the bases with which they combine.

\tabref{fig: stress and tone in verbal paradigms} illustrates the contrast between stressed and unstressed roots in neutral and shifting morphological environments and the tonal patterns associated to stressed syllables in different morphological environments \parencite{caballero2008choguita, caballero2011morphologically}. Shading highlights cells with morphologically complex words undergoing stress shifts.

\begin{table}
%\includegraphics[width=\textwidth]{figures/Prosody-img2.png}
\begin{tabularx}{\textwidth}{Qllll}
\lsptoprule
& \multicolumn{2}{l}{Bare stem (PRS)} & Neutral (PST \textit{-li}) & Shifting (COND \textit{-sa})\\
\midrule
Stressed & \textit{\textquotesingle tô} & `bury' & \textit{\textquotesingle tô-li} & \textit{\textquotesingle tô-sa}\\
roots & \textit{tʃ͡i\textquotesingle hà} & `spread' & \textit{tʃ͡i\textquotesingle hà-li} & \textit{tʃ͡i\textquotesingle hà-sa}\\
& \textit{mu\textquotesingle rú} & `carry' & \textit{mu\textquotesingle rú-li} & \textit{mu\textquotesingle rú-sa}\\
& \textit{bini\textquotesingle hí} & `accuse' & \textit{bini\textquotesingle hí-li} & \textit{bini\textquotesingle hí-sa}\\
\tablevspace
Unstressed  & \textit{\textquotesingle tò} & `take' & \textit{\textquotesingle tò-li} & \cellcolor{lsLightGray}\textit{to-\textquotesingle \textbf{sâ}}\\
roots & \textit{u\textquotesingle kú} & `rain' & \textit{u\textquotesingle kú-li} & \cellcolor{lsLightGray}\textit{uku-\textquotesingle \textbf{sâ}}\\
& \textit{ra\textquotesingle ʔìtʃ͡a} & `speak' & \textit{ra\textquotesingle ʔìt͡ʃa-li} & \cellcolor{lsLightGray}\textit{raʔi\textquotesingle\textbf{tʃâ}-sa}\\
\lspbottomrule
\end{tabularx}
\caption{
\label{fig: stress and tone in verbal paradigms}
Morphologically-conditioned stress shifts and tonal alternations in verbs \parencite{caballero2008choguita, caballero2015tone}}
\end{table}

This table exemplifies how stressed roots retain stress in a fixed location across morphological contexts (in the first, second or third syllable), while unstressed roots exhibit a rightward stress shift in shifting environments (in this case, exemplified with the conditional \textit{-sâ} suffix).

Unstressed roots receive stress by default on the second syllable of the root in neutral environments (i.e., monosyllabic roots bear stress on the root). The default stress assignment algorithm is summarized in (\ref{ex: default stress assignment rule}).

\ea\label{ex: default stress assignment rule}
\textbf{Default stress assignment in Choguita Rarámuri}: words containing unstressed disyllabic or trisyllabic roots and neutral morphological constructions have second syllable
stress \parencite{caballero2011morphologically}.
\z

The two classes of morphological constructions are not only distinguished by their stress effects, but also in terms of other morpho-phonological effects they may trigger: shifting constructions may induce vocalic alternations or other morpho-phonological changes, a phenomenon documented across the Uto-Az\-tec\-an language family \parencite{heath1977uto, heath1978uto}. In Choguita Rarámuri there are vocalic alternations that unstressed roots may undergo, which motivates a characterization of Choguita Rarámuri verbs into three classes, if both stress and vocalic alternations are considered (for full discussion, see Chapter \ref{chap: verbal morphology}). Lexically stressed roots constitute the first class, Class 1, and undergo no morpho-phonological changes. Lexically unstressed roots can be divided into two sub-classes: (i) unstressed roots with fully specified vowels (Class 2 verbs) and (ii) unstressed roots with root final unspecified vowels (Class 3 verbs). Class 3 roots have a final V slot, whose features are dependent on the morphological construction that combines with the root, with final root vowel raising in shifting constructions. The three verbal root classes are summarized in \tabref{tab:verb-classes2}.

\begin{table}
\caption{Choguita Rarámuri verbal root classes}
\label{tab:verb-classes2}

\begin{tabularx}{\textwidth}{lllQl}
\lsptoprule
& \textbf{Class 1} & \textbf{Class 2}  & \textbf{Class 3} & \\
& \textit{Stressed} & \makecell[tl]{\textit{Unstressed}\\\textit{specified final V}} & \textit{Unstressed} \textit{unspecified final V} &  \\
\midrule
\textsc{pst} &  beˈnè-li  &     suˈkú-li      &   raʔˈlà-li & \textbf{Neutral}\\
\textsc{prog} &  beˈnè-a &    suˈkú-a &  raʔˈlà-a    & \textbf{Constructions}\\
\textsc{impf} &   beˈnè-i &   suˈkú-i  &  raʔˈlà-i &              \\
\hspace{3cm}\\
\textsc{fut.sg} & beˈnè-ma   &          suku-ˈmêa   &     raʔˈli-ˈmêa & \textbf{Shifting}\\
\textsc{cond} &  beˈnè-sa & suku-ˈsâ    &  raʔˈli-ˈsâ &   \textbf{Constructions}\\
\textsc{desid} &  beˈnè-nale    & suku-ˈnále    &  raʔˈli-ˈnále & \\
\lspbottomrule
\end{tabularx}
\end{table}

As shown in \tabref{tab:verb-classes2}, Class 2 roots have a final stressed low vowel in neutral constructions. Most Class 3 roots, on the other hand, end in high, front vowels in shifting constructions (such as the future singular \textit{-ˈmêa}), although some verbs of this class can also end in back, mid vowels (e.g., \textit{noko-ˈmêa} in (\ref{ex: class 3 roots final specified vowels}d). Without exception, these roots end in \textit{a} when attaching a stress neutral suffix (such as the past \textit{-li}). Some examples of these roots with alternating final vowels are provided in (\ref{ex: class 3 roots final specified vowels}):

%\break

\ea\label{ex: class 3 roots final specified vowels}
{Class 3 roots: vowel alternations}

\begin{tabular}{llllll}
     & \textit{Stem} & \textit{Gloss} & \textit{Shifting} & \textit{Neutral}\\
     & & & \textit{\textsc{fut.sg}} & \textit{\textsc{pst}}\\
     a. & {osa} & {‘write'} & {os\textbf{i}-ˈmêa} & {oˈs\textbf{á}-li} & < AHF 05 1:127/el >\\
     b. & {itʃ͡a} & {`sow'} &{itʃ\textbf{i}-ˈmêa}  & {iˈtʃ\textbf{á}-li}&  < SFH 05 1:78/el > \\
     c. & {raha} & {‘light up’} & {rah\textbf{i}-ˈmêa} & {raˈh\textbf{á}-li} & < ROF 04 1:62/el >\\
     d. & {noka} & {`move'} & {nok\textbf{o}-ˈmêa} & {noˈk\textbf{á}-li} & < BFL 05 1:114/el >\\
\end{tabular}

      \z

Finally, the analysis of stress and tone presented here assumes that all lexical tones are underlyingly specified (see also \citealt{caballero2021grammatical}). Unstressed roots may bear either H or L tone when bare or in neutral contexts, which can be attributed to a lexical tonal specification. Stressed roots, on the other hand, may bear HL, H or L tones. There is thus an asymmetry between stressed and unstressed roots in terms of lexical tone, as there are no unstressed HL-toned roots in the language.

When stressed, shifting suffixes may bear any of the three lexical tones. Given that tonal contrasts are only realized in stressed syllables, there is no evidence that stress neutral suffixes have underlying lexical tone, since they are never stressed. The tonal properties of stress shifting suffixes are shown in a fragment of the paradigm for the root \textit{suˈkú } `to scratch' in (\ref{ex: tone of shifting suffixes}).

\ea\label{ex: tone of shifting suffixes}
{Neutral and shifting suffixing constructions}\\

\begin{tabular}{lllll}
     a. & {Past \textit{-li}} & {\textit{suˈkú-li}} & {H} & < RIC el2020 >\\
     b. & {Participial \textit{-ame}} & {\textit{suˈkú-ame}} &{H}  & < RIC el2020 >\\
     c. & {Past egophoric \textit{-ki}}&{\textit{suˈkú-ki}}&{H}&{< RIC el2020 >}\\
     d. & {Evidential \textit{-tʃ͡ane}}&{\textit{suˈkú-tʃ͡ane}}&{H}&{< LEL el2059 >}\\
    e. & {Future singular \textit{-mêa}}&{\textit{suku-ˈmêa}}&{HL}&{< RIC el2020 >}\\
    f. & Desiderative \textit{-nále}&{\textit{suku-ˈnále}}&{H}&{< RIC el2019 >}\\
    g.& {Conditional \textit{-sâ}}&{\textit{suku-ˈsâ}}&{HL}&{< LEL el2059}\\
    h. & {Imperative plural \textit{-sì}}&{\textit{suku-ˈsì}}&{L}&{< RIC el2020 >}
    \end{tabular}
      \z

Words containing an unstressed root like \textit{suˈkú} and a stress neutral suffix (e.g., the past \textit{-li} suffix (\ref{ex: tone of shifting suffixes}a)), bear stress on the second syllable, given the default stress assignment rule given in (\ref{ex: default stress assignment rule}). Words containing an unstressed root and a stress shifting suffix (e.g., the future singular \textit{-mêa} suffix (\ref{ex: tone of shifting suffixes}e)), bear stress on the suffix, the third syllable of the prosodic word. More details about the tonal properties of roots and morphological constructions (affixes and non-concatenative morphological exponents) is provided in §\ref{sec: prosodic properties of morphologically complex verbs}.

\subsection{Stress and lexical tone}
\label{subsec: stress and lexical tone}

As reported in \chapref{chap: word prosody} and \chapref{chap: verbal morphology}, all prosodic words have a surface stressed syllable where lexical tonal contrasts are realized. This generalization may be stated as follows:

\ea\label{ex: tone is a property of surface stressed syllables}
Lexical tone is a property of morphemes associated to the surface stressed syllable.
\z

All three lexical tones (H, L, and HL) may be associated with the stressed position, whether the first, second or third syllable. As stated in (\ref{ex: default stress assignment rule}) above, unstressed roots receive stress by default on the second syllable of the root in stress neutral environments (i.e., monosyllabic roots bear stress on the root).

\hspace*{-1.8pt}Morphologically-conditioned stress shifts result in tone neutralization patterns. As shown in (\ref{ex: tone of shifting suffixes}) above, if a stress-shifting suffix is stressed after a stress shift, the stressed suffix syllable will bear the lexical tone of that suffix (this is the case when a monosyllabic or disyllabic unstressed root attaches a stress-shifting suffix). This pattern is further exemplified in \tabref{tab:suffix-tone}.

\begin{table}
\caption{Suffix lexical tone after stress shift}
\label{tab:suffix-tone}

\begin{tabularx}{\textwidth}{lQlQl}
\lsptoprule
&\textbf{Neutral} & \textbf{Tone}  & \textbf{Shifting} & \textbf{Tone}\\
& \textbf{constructions} & & \textbf{constructions} & \\
\midrule
a.& tò-li & L  & to-ˈkâ &  HL  \\
&‘take-\textsc{pst}’ &   & ‘take-\textsc{imp.sg}' &	\\
b.& raˈhá-li & H & raha-ˈkâ & HL\\
&‘light.fire-\textsc{pst}’ &   & ‘light.fire-\textsc{imp.sg}’ &	 \\
c.& ˈtò-li & L & to-ˈsì  & L \\
&‘take-\textsc{pst}’ &  & ‘take-\textsc{imp.pl}’ &  \\
d.& kiˈmá-li & H & kimi-ˈsì  & L\\
&`put.blanket-\textsc{pst}' &  & ‘put.blanket\textsc{-imp.pl}’ &  \\
e.& uˈkú-li & H &  uku-ˈnále  & H \\
&‘to.rain-\textsc{pst}' &  & ‘to.rain-\textsc{desid}' & \\
f.& kiˈmá-li & H & 	kimi-ˈnále & H\\
&‘put.blanket.\textsc{pst}' &  & ‘put.blanket-\textsc{desid}' & \\
\lspbottomrule
\end{tabularx}
\end{table}

As shown in (\ref{ex: tone of shifting suffixes}) and \tabref{tab:suffix-tone}, shifting suffixes bear their underlying tone when stressed (e.g., HL in the imperative singular \textit{-kâ} in forms (a--b) in \tabref{tab:suffix-tone}, L in the imperative plural \textit{-sì} in forms (c--d) in \tabref{tab:suffix-tone}. The lexical tone of the root, which surfaces in neutral morphological contexts, is deleted after the stress shift. On the other hand, if a trisyllabic unstressed root attaches a stress-shifting suffix, the newly stressed syllable will be a stem syllable. As shown in \tabref{tab:stem-tone}, a newly stressed stem syllable will bear a HL tone in these contexts, regardless of what the lexical tone of the root is (H tone in \textit{roʔˈsówa} ‘cough’ (examples (a--b) in \tabref{tab:stem-tone}), L tone in \textit{naʔˈsòwa} ‘stir’ (examples (c--d) in \tabref{tab:stem-tone}).

\begin{table}
\caption{Stem tone after stress shift}
\label{tab:stem-tone}

\begin{tabularx}{\textwidth}{lQlQl}
\lsptoprule
&\textbf{Neutral} & \textbf{Tone}  & \textbf{Shifting} & \textbf{Tone}\\
&\textbf{constructions} & & \textbf{constructions} & \\
\midrule
a.& roʔˈsówa-li  &	H      &	roʔsoˈwâ-ma     &	HL \\
&‘cough-\textsc{pst}’ &  &	‘cough-\textsc{fut.sg}’ &  \\
b.& roʔˈsówa-li 	& H  & roʔsoˈwâ-si &   	HL  \\
&‘cough-\textsc{pst}’ &  &	‘cough-\textsc{imp.pl}’ &  \\
c.& naʔˈsòwa-li   & L    &	naʔsoˈwâ-ma &  	HL \\
&‘stir-PST’ & 		& ‘stir-\textsc{fut.sg}’ &  \\
d.&naʔˈsòwa-i	& L  &  naʔsoˈwâ-si &   HL\\
&‘stir-\textsc{impf}’ & < & ‘stir-\textsc{imp.pl}’ &  \\
\lspbottomrule
\end{tabularx}
\end{table}


%edit the following
One important aspect of this second pattern is that the surface tonal pattern of these morphologically complex words that have undergone a stress shift is not predictable based on the lexical tonal properties of root morphemes nor the lexical tones of suffixes. An example discussed in \citet{caballero2021grammatical} is that of unstressed roots attaching the imperative plural \textit{-sì} suffix. As shown in examples (c--d) in \tabref{tab:suffix-tone} above, this suffix bears its lexical L tone when stressed, but it will not surface if the stressed syllable after a stress shift is a stem syllable, i.e., the hypothetical forms \textit{*rosoˈwà-si} (for example (b) in \tabref{tab:stem-tone} and \textit{*naʔsoˈwà-si} (for example (d) in \tabref{tab:stem-tone}) with L tone when attaching imperative plural \textit{-sì}, are unattested.

This tonal pattern is analyzed in \citet{caballero2021grammatical} as resulting from a process of default HL tone insertion after a stress shift has deleted the root's lexical tone. This analysis is based on the following assumptions:

\ea\label{ex: default HL tone insertion}
    \ea[]{
    Each morpheme has one and only one tone, which is lexically associated with one and only one mora (within the tonic syllable)
}
        \ex[]{
        In words containing lexically unstressed roots and neutral morphological constructions, the stressed syllable bears the underlying lexical tone of these roots
    }
            \ex[]{
            Stress shifts in shifting environments cause lexical root tones to delete
        }
                \ex[]{
                If the newly stressed syllable after a stress shift is a stem syllable, it is toneless and acquires a default HL tonal melody
            }
    \z
\z

It could be argued that this default HL tone insertion process is also at play in \ili{Spanish} loanwords \parencite{caballero2013procesos} (see also §\ref{sec: loanword prosody}). Loanwords from \ili{Spanish} are incorporated into Choguita Rarámuri with faithful prominence to the stress location of the source words. The stressed syllable in loanwords has a HL tone (no exceptions have yet been documented to this pattern). Relevant examples are given in \tabref{tab:loanwords} (the \ili{Spanish} sourcewords are provided in their orthographic form, where boldface indicates the stressed syllable).

\begin{table}
\caption{Tone patterns in Spanish loanwords}
\label{tab:loanwords}

\begin{tabularx}{\textwidth}{lXXXX}
\lsptoprule
&\textbf{Spanish} & \textbf{Loanword}  & \textbf{Tone} & \textbf{Gloss}\\
&\textbf{sourceword} & &  & \\
\midrule
a.& To\textbf{más} &	 [toˈmâʃi] 	&	HL 	&	‘Thomas’\\
b.& man\textbf{za}na & [maˈsâna] &		HL &		‘apple’ \\
c.& \textbf{sá}bado &	[ˈsâbuto] 	&	HL 	&	‘Saturday’\\
d.& \textbf{Juan} 	& [ˈhuâni]  &		HL	&	‘Juan’\\
e.& Da\textbf{niel} &	[raˈniêri] &		HL &		‘Daniel’\\
f.& li\textbf{me}ta &	[liˈmêta-ʧi] &	HL 	&	‘bottle’\\
g.& pa\textbf{sear}	& [basaˈlôwa]	&	HL	&	`to take a stroll'\\
\lspbottomrule
\end{tabularx}
\end{table}

The analysis involving a default HL tone insertion process for loanwords would involve the assumption that \ili{Spanish} loanwords are lexically stressed but toneless in Choguita Rarámuri. Alternatively, the tonal properties of Choguita Rarámuri loanwords may be analyzed as involving a reinterpretation of the acoustic properties of Mexican \ili{Spanish} prominence, which has been argued to include a H* pitch accent in the stressed syllable in certain intonational contexts (focus-marked words in declarative sentences) (\citealt{prieto1995tonal}).

In sum, morphological factors condition stress shifts. Given the dependency tone has on stress-accent for its distribution, the tonal alternations resulting in these contexts are largely predictable based on the lexical tonal properties of the morphemes that make up a morphologically complex word. Further discussion on the mechanism of default tonal assignment is provided in §\ref{subsubsec: grammatical tone distributed by morphological class}.

\subsection{Stress and tone properties of compounds}
\label{subsec: stress and tone properties of compounds}

Body part incorporation constructions involve a single Prosodic Word, with a construction specific rule assigning a single stress to the first syllable of the head of the compound. As addressed in \chapref{chap: word prosody} and \chapref{chap: verbal morphology}, in these N-V constructions the noun root is fully integrated with the verb morphologically, and both the noun root and the verb root are attested as independent roots in the language. As described in §\ref{subsec: body-part incorporation}, stress in these constructions is actively constrained by the grammar: if the head, the verb, has second syllable stress in isolation and if the first member, the incorporated body-part noun, is two syllables long, stress retracts to the verb’s first syllable, the construction’s third syllable. The stressed syllable in body-part incorporation forms bears a HL tone. Relevant examples are given in (\ref{ex: stress and tone properties of body-part incorporation forms}).

\ea\label{ex: stress and tone properties of body-part incorporation forms}
{Stress and tone properties of body-part incorporation forms}

    \ea[]{
    \textit{busiˈkâsi}\\
    \textit{busi+ˈkâsi}	\\
    eye+break\\
    `to become blind'\\
}
        \ex[]{
        \textit{ropaˈkâsi}\\
        \textit{ropa+ˈkâsi}	\\
        stomach+break \\
        `to have a miscarriage'\\
    }
            \ex[]{
            \textit{busiˈbôta}\\
            \textit{busi+ˈbôta }\\
            eye+come.out  \\
            `for eyes to come out'\\
        }
                \ex[]{
                \textit{kutaˈbîri}\\
                \textit{kuta+ˈbîri} \\
                neck+twist  \\
                `to neck-twist'\\
            }
                    \ex[]{
                    \textit{tʃ͡omaˈbîwa}\\
                    \textit{tʃ͡oma+ˈbîwa} \\
                    mucus+clean  \\
                    `to mucus-clean'\\
                }
    \z
\z

As discussed in \chapref{chap: verbal morphology}, the stress properties of incorporated forms can be analyzed as resulting from a mophological stress rule specific to these constructions that requires stress to be assigned in the first syllable of the head of the incorporated construction:

\ea\label{ex: incorporation stress rule}
{Incorporated verb stress rule}\mbox{}

{The head of the incorporation construction, the verbal root, must bear stress in the first syllable}\\
\z

This morphological stress rule is a process that is analyzed here as involving both stress deletion and stress re-assignment. Thus, HL tone assignment in verbal compounds can be analyzed as resulting from the process of default HL tone insertion that is attested in other morphological contexts where a stress shift has deleted a lexical tone (e.g., \tabref{tab:stem-tone}).

\subsection{Grammatical tone}
\label{subsec: grammatical tone}

Tone patterns that play a morphological role in any linguistic system have been defined in a variety of ways in the literature (for an overview, see \citealt{hyman2016morphological}). This grammar adopts the term \textit{grammatical tone} and assumes the following definition (\citealt{rolle2018grammatical}; see also \citealt{caballero2021grammatical}):

\ea\label{ex: grammatical tone definition}
{\textbf{Grammatical tone}: A tonal pattern or process that is not general across the phonological grammar of a language, but is instead associated to a specific morpheme or construction, or a natural class of morphemes or constructions.}
\z

While several \ili{Uto-Aztecan} languages have developed lexical tone, there is little information in the literature as to whether these languages feature grammatical tone patterns. Exceptions include \ili{Milpa Alta Nahuatl}, where tone is reported to distinguish present from imperfect verb forms among other contrasts \parencite{whorf1993pitch} and \ili{Wixárika} (\ili{Huichol}; \ili{Corachol})), where tone encodes tense distinctions (present vs. past) \parencite{banerji2014huichol}.

Both derivational and inflectional morphological constructions may exhibit an underlying lexical tone and/or trigger a tonal alternation in the stems to which they attach. This section provides an overview of grammatical tone patterns as documented in inflectional constructions and as attested in morphologically complex words composed of a stem plus one inflectional morphological construction, as described in \citet{caballero2021grammatical}. In terms of the presence/absence of affixation concomitant to grammatical tone and the type of tonal alternation, there are at least three patterns of grammatical tone: (i) tone as an exponent of morphological information; (ii) tone as a morphologically-conditioned tonal effect and (iii) tone that is determined by the type of morphological construction (stress-shifting or stress-neutral) in a class of verbs. Each of these patterns is described next. Grammatical tone patterns reveal that stress and tone, though intricately related in their distribution, are separate dimensions in the classification of verbal root morphemes into different classes.

\subsubsection{Tone as a morphological exponent}
\label{subsubsec: tone as realizational morphology}

A first pattern of grammatical tone attested in Choguita Rarámuri involves tone as an exponent of inflection (also referred to as a ``tonal morpheme'' in \citealt{welmers1959tonemics} and \citealt{hyman2016morphological}), a type of non-concatenative morphology that may also be referred to as \textit{realizational morphology}, where a morphological category is encoded by a phonological process other than concatenation of segmental morphemes (see \citealt{inkelas2014interplay} for discussion). Tonal exponence in Choguita Rarámuri is attested in the imperative singular construction, which features four lexically conditioned suppletive allomorphs, two of which are concatenative ((i) the \textit{-ka}̂ suffix and (ii) the \textit{-sa}̂ suffix), and two of which are non-concatenative ((iii) a stress shift and (iv) a L tone realized in the stressed syllable of HL-toned stems). The L imperative singular morphological construction is exemplified in \tabref{tab:imperative-sg-allomorphs}.

\begin{table}
\caption{Imperative singular allomorphs}
\label{tab:imperative-sg-allomorphs}

\begin{tabularx}{.9\textwidth}{l lX lX l}
\lsptoprule
&\textbf{Present} &   & \textbf{Imperative sg.} &  & \textbf{Gloss}\\
\midrule
a.& niˈkâ &		HL & 	niˈkà	&	L &	‘to bark'\\
& & & & & <BFL el1910>\\
b.& tiˈsô &  	HL & 	tiˈsò     &	L & ‘to walk with cane’\\
& & & & & <SFH el2042>\\
c.& niˈwâ &		HL &	niˈwâ-sa &	HL &	`to make’\\
& & & & & <BFL 2014:61>\\
d.& muˈrú &		H &	muˈrú-ka &	HL &	`to carry in arms'\\
& & & & & <BFL el1883>\\
e.& naʔˈsòwa &	L	& naʔsoˈwâ &	HL	& ‘to stir’\\
& & & & & <BFL el1957>\\
\lspbottomrule
\end{tabularx}
\end{table}

The L tone allomorph of the imperative singular overwrites the lexical HL tone of the root (examples (a--b) in \tabref{tab:imperative-sg-allomorphs}): HL \textit{ni'kâ} `it barks' vs. L \textit{ni'kà} `bark!'. The L tone allomorph does not combine with a verb stem with a suffixal allomorph of the imperative singular (examples (c--d) in \tabref{tab:imperative-sg-allomorphs}, i.e., there is no multiple exponence of the imperative singular. The second non-concatenative allomorph of the imperative singular is exemplified in example (e) in \tabref{tab:imperative-sg-allomorphs}, where a rightward stress shift within the stem encodes the inflectional value: the verb has second syllable stress (and lexical L tone) when inflected for present (\textit{naʔ\textbf{'sò}wa} `s/he stirs it'), but third syllable stress (and HL tone assigned by default) when inflected for imperative singular (\textit{naʔso\textbf{'wâ}} `stir it!').

As stated above and exemplified in \tabref{tab:imperative-sg-allomorphs}, only HL stem tones are replaced by the imperative singular L tone. In the case of L-toned stems, the L imperative singular allomorph vacuously applies (examples (a--c) in \tabref{tab:tone-imperative-sg}), but it is blocked from applying to H-toned stems (examples (d--f) in \tabref{tab:tone-imperative-sg}).

%\break

\begin{table}
\caption{Tone in imperative singular forms}
\label{tab:tone-imperative-sg}

\begin{tabularx}{.9\textwidth}{l lX lX l}
\lsptoprule
&\textbf{Present} &   & \textbf{Imperative sg.} &  & \textbf{Gloss}\\
\midrule
a.&  oˈhò	&	L &	oˈhò & 		L &‘to thresh' \\
& & & & & <BFL el1906>\\
b.& seˈmè   &   	L	& seˈmè  &          	L	& ‘to play violin’ \\
& & & & &  <BFL el1920>\\
c.& biʔˈtò	&	L	& biʔˈtò &		L & ‘to twist ankle’\\
& & & & & < RIC el2024 >\\
d.& saˈkú    &   	H	& saˈkú  &       H	& ‘to dry in sun’ \\
& & & & & < BFL el1923 >\\
e.& kiˈmá	&	H &	kiˈmá &		H & ‘to put on blanket' \\
& & & & & <BFL el1909>\\
f.& ʃutuˈbú &	H	& ʃutuˈbú &	H	& ‘to tie legs’\\
& & & & & < BFL el1911>\\
\lspbottomrule
\end{tabularx}
\end{table}

As exemplified here, tonal overwritting of HL tones by the grammatical L tone neutralizes the contrast between lexical HL and L tones in the imperative singular.

A subset of inflected verbs displays inter-speaker variation in terms of grammatical tone: unstressed roots (those that exhibit stress shifts when attaching a stress-shifting suffix or non-concatenative process) may surface with one of two tonal patterns: (i) a grammatical L tone that replaces stem HL tone or (ii) a HL stem tone associated with the shifting morphological class. This variation in surface forms is illustrated in \tabref{tab:variation-tone}.

%The examples shown so far involve stressed verbs. Some data suggests that speakers may have more than one generalization available with imperative singular inflected words containing unstressed verbs: recall from §\ref{subsec: stress and lexical tone} the process of stress-based tonal neutralization, where a stress shift deletes a lexical tone and a default HL tone is assigned to the stressed stem syllable. The HL tone of the imperative form in (\ref{ex: imperative sg allomorphs}e), \textit{naʔso'wâ} ‘Stir it!’, can be analyzed as resulting from this process (where the lexical L tone of the verb is associated to the second stressed syllable in neutral contexts, e.g., \textit{naʔ'sòwa} ‘S/he stirs it’, but deleted after the stress shift). The stress shift in this case indicates that the imperative singular construction patterns together with shifting constructions.  The affix-less imperative singular form is thus associated with two competing tone melodies: (i) a grammatical L tone that replaces stem HL tone and (ii) a HL stem tone associated with the shifting morphological class. Speakers in fact exhibit variation in terms of which generalization they chose for the tonal make up of imperative forms of unstressed verbs, as exemplified in (\ref{ex: variation in tone}), with the stem \textit{raʔ'sà-n}a ‘smash-\textsc{tr}’:

%\break

\begin{table}
\caption{Tonal variation in imperative singular forms}
\label{tab:variation-tone}

\begin{tabularx}{.7\textwidth}{lllll}
\lsptoprule
&\textbf{Present} &   & \textbf{Imperative sg.} &  \\
\midrule
a.&  raʔˈsà-na &	L	&	raʔsaˈ'nà &	L\\
& < RIC 2014:102 >\\
b.& raʔˈsà-na &	L	&	raʔsaˈnâ &	HL \\
&< BFL el1911]\\
c.& raʔˈsà-na &	L	&	raʔsa-ˈnâ &	HL \\
&< LEL el2019 >\\
\lspbottomrule
\end{tabularx}
\end{table}

\newpage
As shown in these examples, all speakers produced a form with a rightward stress shift in the imperative singular, but they show different tonal patterns in the derived imperative singular: one speaker (RIC) produces a L tone in the stressed syllable (example (a) in \tabref{tab:variation-tone}), the expected L grammatical tone of the imperative singular; in contrast, two other speakers (BFL and LEL) produce a HL tone in the stressed syllable (examples (b--c) in \tabref{tab:variation-tone}), the expected pattern if lexical tone has been neutralized and a default HL tone assigned. This latter pattern results from a process of stress-based tonal neutralization, where a stress shift deletes a lexical tone and a default HL tone is assigned to the stressed stem syllable (§\ref{subsec: stress and lexical tone}). This kind of inter-speaker variation suggests that speakers may have more than one generalization available regarding the tonal properties of these inflected forms. This kind of variation is not yet documented in other morphological contexts.

\subsubsection{Morphologically-conditioned tone}
\label{subsubsec: morphologically conditioned tone}

In addition to tonal exponence, Choguita Rarámuri possesses grammatical tone patterns that are associated with specific suffixes when attaching to a prosodically-defined class of verbs. Specifically, unstressed, H-toned roots surface with a L tone when attaching the imperfective \textit{-i} suffix and the progressive \textit{-a} suffix. The imperfective and progressive suffixes are stress-neutral, which means the tonal changes they condition with unstressed, H-toned roots take place without any concomitant stress shifts nor any other (morpho-)phonological effects. The examples in \tabref{tab:morphologically-conditioned-tone} show the contrast between the imperfective and progressive constructions, on the one hand, and the past tense \textit{-li} suffix, a stress-neutral suffix that does not impose a L tone on the stems to which it attaches. In the latter case, the lexical tone of the root (H or L) is associated with the stressed syllable.

\begin{table}
\caption{Morphologically-conditioned L tone}
\label{tab:morphologically-conditioned-tone}

\begin{tabularx}{\textwidth}{lQllllll}
\lsptoprule
&\textbf{Verb} & \textbf{\textsc{impf}}  & & \textbf{\textsc{prog}} &  & 	\textbf{\textsc{pst}} & \\
\midrule
a.& ‘dance’ & aˈwì-i	& L &   	aˈwì-a	& L  & 	aˈwí-li  &  H \\
&<RIC el1921>\\
b.& `find' & riˈwà-i & L &   riˈwà-a  & L &  	riˈwá-li &  	H	\\
&<LEL el2062>\\
c.& `buy' & raʔˈlà-i	&  L  &  	raʔˈlà-a	& L  &	raʔˈlá-lo & H	\\
&<BFL el1905>\\
d.& `grind' & raʔˈsà-na-i &	L	& raʔˈsà-na-a 	& L	& raʔˈsà-na-li 	& L\\
&<BFL el1911>\\
e.& `smoke' & moˈlò-i &	L	& moˈlò-a &	L	& moˈlò -li &  \\
&<BFL el1914-5>\\
\lspbottomrule
\end{tabularx}
\end{table}

As shown in examples (d--e) in \tabref{tab:morphologically-conditioned-tone}, L-toned unstressed roots do not exhibit any tonal changes when inflected for imperfective or progressive, showing the process vacuously applies with these verbs.

The contrast between unstressed roots (examples (a--c) in \tabref{tab:morphologically-conditioned-L-tone}) and stressed roots (examples (d--f) in \tabref{tab:morphologically-conditioned-L-tone}) demonstrates that only unstressed roots exhibit a L tone when attaching these suffixes.

%\break

\begin{table}
\caption{Morphologically-conditioned L tone: stressed vs. unstressed verbs}
\label{tab:morphologically-conditioned-L-tone}

\begin{tabularx}{\textwidth}{llllllll}
\lsptoprule
& & \textbf{\textsc{impf}}  & & \textbf{\textsc{prog}} &  & 	\textbf{\textsc{cond}} & \\
 \midrule
a.& ‘dance’ & aˈwì-i	& L &   	aˈwì-a	& L  & 	awi-ˈsâ   &  HL \\
&<RIC el1921>\\
b.& `find' & riˈwà-i & L &   riˈwà-a  & L &  	riwi-ˈsâ &  	HL	\\
&<LEL el2062>\\
c.& ‘grind’ &  raʔˈsà-na-i & L 	& raʔˈsà-na-a 	& L  & raʔsa-ˈnâ-sa  & HL\\
&<BFL el1911>\\
d.& `bet' &  hiˈrâ-i	& HL    &	hiˈrâ-a	 &  	HL 	& hiˈrâ-sa & HL\\
&<SFH el1925>\\
e.& `carry in arms' & muˈrú-i 	& H  &    	muˈrú-a     &	H  & 	muˈrú-sa  & H\\
&<LEL el921>\\
f.& `cry' &  naˈlà-i  & 	L  & naˈlà-a    &	L  & 	naˈlà-sa    & L \\
&<LEL el2062>\\
\lspbottomrule
\end{tabularx}
\end{table}

As discussed previously and seen in these examples, the rightward stress shift in shifting environments is only attested with unstressed verbs (\tabref{tab:morphologically-conditioned-tone}a--c). Stressed roots, on the other hand, do not undergo stress shifts (\tabref{tab:morphologically-conditioned-tone}d--f). Stressed roots do not replace their lexical tones when attaching the imperfective and progressive suffixes.

These morpheme-specific phonological effects originated through recent diachronic changes. Comparative evidence from closely related \ili{Norogachi Rarámuri} \parencite{brambila1953gramatica}  and \ili{Mountain Guarijío} \parencite{miller1996guarijio} shows that the progressive \textit{-a} suffix and the imperfective \textit{-i} suffix have recently developed from suffixes that were shifting.\footnote{In \ili{Mountain Guarijío}, Miller describes a progressive \textit{-a} suffix that may bear stress (e.g., \textit{yuʔku-á=ga} ‘it is windy’ (\citeyear[140]{miller1996guarijio})), the cognate form of the Choguita Rarámuri progressive \textit{-a} suffix.}  The shifting allomorphs of these suffixes are attested in the speech of Choguita Rarámuri native speakers with command of the closely related \ili{Norogachi Rarámuri} (NR). For these speakers, the Choguita Rarámuri forms co-exist with the NR forms that exhibit the conservative form of these suffixes, which are stressed, include a palatal glide onset and have a H tone. These forms are exemplified in \tabref{tab:norogachi-raramuri}.

\begin{table}
\caption{Norogachi Rarámuri cognate suffixes: imperfective and progressive}
\label{tab:norogachi-raramuri}

\begin{tabularx}{\textwidth}{llll}
\lsptoprule
&\textbf{Norogachi} & \textbf{Choguita} & \textbf{Gloss} \\
\midrule
a.& awi-ˈjé <BFL el1921> & aˈwì-i <BFL el1921> &dance-\textsc{impf}\\
b.& ika-ˈjá  <el1915> & iˈkà-a  <LEL el1912>&	be.windy-\textsc{prog}\\
c.& ika-ˈjé  <el1915> & iˈkà-i  <LEL el1912> & be.windy-\textsc{prog}\\
d.& to-ˈjá	<LEL tx110> & ˈtò-a <MDH co1136> & take-\textsc{prog}\\
\lspbottomrule
\end{tabularx}
\end{table}

The morphologically-conditioned tonal effects exhibit by the progressive and imperfective suffixes are thus the result of a recent innovation of morphologically-conditioned tone.

\subsubsection{Grammatical tone distributed by morphological class}
\label{subsubsec: grammatical tone distributed by morphological class}

A third type of grammatical tone pattern in Choguita Rarámuri is attested in a class of lexically stressed verbs referred to as ``alternating'' in \citet{caballero2015tone}. In these verbs, surface tonal patterns are conditioned by the type of inflection the verb exhibits: a HL tone in shifting morphological environments and a L tone in neutral morphological environments. Stress-shifting constructions do not trigger stress shifts when combining with lexically stressed roots, but cause a rightward stress shift with lexically unstressed roots (§\ref{sec: stress properties of roots, stems and suffixes}). In the alternating verb class, stress-shifting suffixes condition specific tone patterns on the stem without triggering any stress shifts. These tonal alternations attested in alternating verbs are exemplified in \tabref{tab:paradigmatic-tone}.

%\break

\begin{table}
\caption{Alternating verbs: tonal alternations}
\label{tab:paradigmatic-tone}

\begin{tabularx}{.8\textwidth}{llllll}
\lsptoprule
 & \textbf{\textsc{impf}}  & & \textbf{\textsc{prog}} &  &\\
 \midrule
a.& ‘bring’ &ˈpà-li	& L &   ˈpâ-ma   	& HL   \\
&<RIC el1921>\\
b.& `arrive' & naˈwà-li & L &   naˈwâ-ma & HL	\\
&<RIC el1921>\\
c.& ‘clean, \textsc{tr}' &  biʔˈw-à-li  & L 	& 	biʔˈw-â-ma  	& HL\\
&<BFL el1903>\\
d.& `swallow' &  aʔˈwà-li	& L    &	aʔˈwâ-ma	 &  HL\\
&<LEL 2063/el>\\
e.& `make, do' & neˈwà-ki 	& L  &    	neˈwâ-    &	HL\\
&<RIC 1892>\\
\lspbottomrule
\end{tabularx}

\end{table}

The examples below show the tone alternations of a single verb \textit{biʔˈw-a} ‘to clean’ (a stem composed of the root \textit{biʔw-} ‘to clean’ and the transitivizer \textit{-a} suffix), where the surface tonal properties of the word are determined by the type of inflection: L if the construction is stress-neutral (\ref{ex: paradigmatic tone of biwa}) and HL if the construction is stress-shifting (\ref{ex: paradigmatic tone of biwa 2}).

\ea\label{ex: paradigmatic tone of biwa}
{Stress-neutral constructions}

    \ea[]{
    \triplebox{\textit{biʔˈw-à-ki}}{L}{`Past egophoric'}\\
}
        \ex[]{
        \triplebox{\textit{biʔˈw-à-li}}{L}{`Past'}\\
    }
	        \ex[]{
	        \triplebox{\textit{biʔˈw-à-i}}{L}{`Imperfective'}\\
	    }
	            \ex[]{
	            \triplebox{\textit{biʔˈw-à-a}}{L}{`Progressive'}\\
	             < BFL el1903, LEL 18:164 >\\
	        }
	\z
\z

\ea\label{ex: paradigmatic tone of biwa 2}
{Stress-shifting constructions}

    \ea[]{
    \triplebox{\textit{biʔˈw-â-ma}}{HL}{`Future singular'}\\
}
	       \ex[]{
	       \triplebox{\textit{biʔˈw-â-ʃi}}{HL}{`Imperative pl.'}\\
	    }
                    \ex[]{
                    \triplebox{\textit{biʔˈw-â}}{HL}{`Imperative singular'}\\
                }
                        \ex[]{
                        \triplebox{\textit{biʔˈw-âa-ru}}{HL}{`Past passive'}\\
                        < BFL el1903, LEL 18:164 >\\
                            }
	\z
\z

Alternating verbs contrast with other lexically stressed verbs: as shown below, these verbs (like the verb \textit{raʔˈnè} ‘to shoot' exemplified below) have fixed stress across paradigms and there are no tonal alternations conditioned by the type of morphological construction: there is fixed stress and constant tone values, regardless of whether the verb combines with stress-neutral (\ref{ex: paradigmatic tone of ra'ne 1}) or stress-shifting (\ref{ex: paradigmatic tone of ra'ne 2}) markers.

\ea\label{ex: paradigmatic tone of ra'ne 1}
{Stress neutral constructions}

    \ea[]{
    \triplebox{\textit{raʔˈnè-ki}}{L}{`Past egophoric'}\\
}
        \ex[]{
        \triplebox{\textit{raʔˈnè-li}}{L}{`Past'}\\
    }
            \ex[]{
            \triplebox{\textit{raʔˈnè-i}}{L}{`Imperfective'}\\
        }
                \ex[]{
                \triplebox{\textit{raʔˈnè-a}}{L}{`Progressive'}\\
            }
    \z
\z

\ea\label{ex: paradigmatic tone of ra'ne 2}
{Stress-shifting constructions}

    \ea[]{
    \triplebox{\textit{raʔˈnè-ma}}{L}{`Future singular'}\\
}
        \ex[]{
        \triplebox{\textit{raʔˈnè-ʃi}}{L}{	`Imperative plural'}\\
    }
            \ex[]{
            \triplebox{\textit{raʔˈnè}}{L}{`Imperative singular'}\\
        }
                \ex[]{
                \triplebox{\textit{raʔˈnèe-ri}}{L}{`Past passive'}\\
                < BFL el1912, el1957 >\\
            }
    \z
\z

For the alternating verb class there is thus no evidence of underlying lexical tone nor are tonal alternations predictable based on the lexical tonal properties of suffixes. These tonal alternations are also independent of stress alternations and other phonological properties of roots and suffixes and non-concatenative processes. Stress and tone, though closely related, are thus orthogonal dimensions in Choguita Rarámuri.

Finally, the tonal patterns of alternating verbs just discussed are relevant to an alternative analysis where stress-shifting and stress-neutral constructions constitute a morphosyntacatically motivated class distinction: as discussed in §\ref{subsec: semantic accounts of shifting and neutral constructions in Uto-Aztecan}, it is assumed in the  Uto-Aztecanist literature that neutral constructions are posited to realize ``non-future'' morphosyntactic categories, including realis inflectional categories such as past, perfective and imperfective, while shifting constructions are posited to realize ``future'' or ``unrealized'' categories, which map onto irrealis categories, such as future, imperative and potential \citep[][133]{langacker1977uto}. The consequence of such an analysis for grammatical tone patterns would involve positing two tonal morphemes, a L tone for ‘non-future’(realis) categories and a HL tone for ``future'' (irrealis) categories.

The arguments against a morphosyntactic account of the shifting vs. neutral distinction are provided in §\ref{subsec: semantic accounts of shifting and neutral constructions in Uto-Aztecan} above. As noted, a realis/irrealis distinction is not amenable for all shifting and neutral constructions. One relevant example involves the past passive construction (exemplified above in (\ref{ex: paradigmatic tone of biwa 2}d), \textit{biʔ'wâar}u ‘it was cleaned’), which cumulatively encodes passive voice and past tense. The past passive may be categorized as a ``realis'' (or ``non-future'') construction in a morphosyntactically-based account, but it nevertheless patterns morphophonologically with the stress-shifting class, contrary to expectation if the distinction were morphosyntactically motivated.

The shifting/neutral distinction in Choguita Rarámuri tonal alternations conforms to the predictions made on a \textit{morphomic} account: the inflected forms of lexemes as shifting or neutral are not determined by morphosyntactic inflectional features, nor any semantic or phonological principles, but rather by purely (or autonomously) morphological properties. The stress-shifting vs. stress-neutral distinction thus involves morphosyntactically heterogeneous classes of morphological constructions.

\subsection{Stress and tonal properties of inflected verbs}
\label{subsec: stress and tonal properties of inflected verbs}

The previous subsections (§\ref{subsec: stress patterns and metrical feet}--§\ref{subsec: grammatical tone}) have addressed each of the factors that shape the surface prosodic properties of inflected verbs in Choguita Rarámuri, including the phonological size of roots, the lexical stress and tone properties of roots and morphological constructions, as well as the different kinds of interactions between lexical and grammatical tone which may lead to lexical tone replacement and grammatical tone assignment. In terms of lexical-grammatical tone interaction, Choguita Rarámuri grammatical tone patterns can be classified as belonging to one of the following classes:

\ea\label{ex: grammatical tone patterns in CR}
{Grammatical tone patterns in Choguita Rarámuri}\mbox{}

    \ea[]{
    \textbf{Tone as a morphological exponent}: the imperative singular L allomorph is a tone-replacing construction that involves no affixation and replaces stem HL tone with a L tone.\\
}\label{ex: grammatical tone patterns in CRa}
        \ex[]{
       \textbf{Morphologically-conditioned tone}: the imperfective \textit{-i} and progressive \textit{-a} suffixes replace H tone of stems containing an unstressed root with a L tone.\\
    }\label{ex: grammatical tone patterns in CRb}
            \ex[]{
            \textbf{Tone determined by morphological class ('alternating verbs')}:  tone is determined by the type of inflection construction the root combines with (HL if shifting, L if neutral) in morphosyntactically heterogeneous classes; there is no evidence of lexical tone.\\
        }\label{ex: grammatical tone patterns in CRc}
    \z
\z

Patterns (\ref{ex: grammatical tone patterns in CRa}) and (\ref{ex: grammatical tone patterns in CRb}) involve tonal overwriting (i.e., a grammatical tone pattern replaces lexical tone), while the pattern in (\ref{ex: grammatical tone patterns in CRc}) involves the assignment of a grammatical tone to a class of underlyingly toneless verbs.

The surface tonal patterns of two-level constructions (roots plus one layer of inflection) is summarized in \tabref{fig: surface tonal melodies}, based on a classification in terms of: (i) the stress properties of roots (unstressed or unstressed); (ii) the tonal properties of constructions (HL, L or H); and the ability of morphological constructions to trigger morpho-phonological changes or not onto stems (shifting or neutral, respectively). Shading represents cells in the paradigm that exhibit a grammatical tone pattern.

\begin{table}
%\includegraphics[width=\textwidth]{figures/Prosody-img3.png}

\small
\begin{tabularx}{\textwidth}{QQllllllll}
\lsptoprule
& & \multicolumn{4}{l}{\textbf{Lexically stressed roots}} & \multicolumn{4}{l}{\textbf{Lexically unstressed roots}}\\
\cmidrule(lr){3-6}
& & & & & \makecell[tl]{\textbf{Alter-}\\\textbf{nat-}\\\textbf{ing}} & & & &\\
\midrule
\textbf{Neutral construc-} & Present

\textit{Bare stem} & HL & L & H & \cellcolor{lsLightGray}L & \multicolumn{2}{l}{L} & \multicolumn{2}{l}{H}\\
\textbf{tions}& Past Ego.

\textit{-ki} & HL & L & H & \cellcolor{lsLightGray}L & \multicolumn{2}{l}{L} & \multicolumn{2}{l}{H}\\
& Past

\textit{-li} & HL & L & H & \cellcolor{lsLightGray}L & \multicolumn{2}{l}{L} & \multicolumn{2}{l}{H}\\
& Present Progr.

\textit{-a} & HL & L & H & \cellcolor{lsLightGray}L & \multicolumn{2}{l}{L} & \cellcolor{lsLightGray}L &\cellcolor{lsLightGray}\\
& Imperfective

\textit{-e} & HL & L & H & \cellcolor{lsLightGray}L & \multicolumn{2}{l}{L} & \cellcolor{lsLightGray}L &\cellcolor{lsLightGray}\\
\tablevspace
& & & & & & Stem & Affix & Stem & Affix\\
& & & & & & stress & stress & stress & stress\\
\midrule
\textbf{Shifting constructions} & Future sg.

\textit{-ma, -\textquotesingle mêa} & HL & L & H & \cellcolor{lsLightGray}HL & HL & HL & HL & HL\\
& Future pl.

\textit{-\textquotesingle bô} & HL & L & H & \cellcolor{lsLightGray}HL & HL & H & HL & H\\
& Conditional

\textit{-\textquotesingle sâ} & HL & L & H & \cellcolor{lsLightGray}HL & HL & HL & HL & HL\\
& Imperative pl.

\textit{-\textquotesingle sì} & HL & L & H & \cellcolor{lsLightGray}HL & HL & L & HL & L\\
& Imperative sg.

\textit{-\textquotesingle kâ, -\textquotesingle sâ} & HL & L & H & \cellcolor{lsLightGray}HL & HL & HL & HL & HL\\
& Imperative sg.

\textit{zero affix} & \cellcolor{lsLightGray}L & L & H & \cellcolor{lsLightGray}HL & HL & -- & HL & --\\
\lspbottomrule
\end{tabularx}
\caption{
\label{fig: surface tonal melodies}
Surface tonal melodies in inflected verbs (root + inflectional construction) \parencite{caballero2021grammatical}}
\end{table}

An analysis of this system based on tonal underspecification is proposed in \citet{caballero2021grammatical}. This analysis is based on the following assumptions:

\ea\label{ex: underspecification analysis of CR tone}
{Tonal underspecification analysis of Choguita Rarámuri}

    \ea[]{
    All tones (HL, L and H) are lexically specified\\
}
        \ex[]{
        Stress and tone are orthogonal dimensions in the phonological grammar of Choguita Rarámuri\\
    }
            \ex[]{
            Verb roots are either specified or unspecified lexically for tone\\
        }
    \z
\z

In this analysis, morphologically complex words which contain tonally specified verbs may exhibit interactions between lexical and grammatical tones, given that specific constructions may be associated with grammatical tone melodies that trigger the replacement of base stem tone. In contrast, in morphologically complex words containing tonally unspecified verbs (alternating roots and unstressed roots where lexical tone has been deleted after a stress shift), tonal patterns realize inflectional morphological information.\footnote{\citet{caballero2021grammatical} refer to the latter kind as \textit{paradigmatic tone}. \citet{rolle2018grammatical} defines paradigmatic grammatical tone as follows: ``[i]n a grammatical paradigm consisting of grammatical categories, tonal values to the root/stem which (i) show extensive inconsistency within grammatical categories (no paradigmatic consistency across rows or columns), and (ii) show extensive inconsistency across roots/stems in parallel paradigms (no ‘transparadigmatic’ consistency across rows or columns), and (iii) there being little positive evidence for determining the underlying tone of the root/stem'' (\citeyear[109]{rolle2018grammatical}). While surface tonal patterns of Choguita Rarámuri alternating verbs exhibit consistent association within grammatical categories (shifting vs. neutral), there is little positive evidence of the underlying tone of the stems. The term \textit{paradigmatic tone} here refers to the fact that, in these cases, cells in the paradigms are associated with tonal values that are not predictable from lexical tone properties of individual morphemes.}

%assess whether the same or not as C&G 2021
Based on their lexical specification for stress and tone, Choguita Rarámuri verbs may thus be characterized as follows:
%examples of each class necessary here

\newpage
\ea\label{ex: CR verb classes in terms of stress and tone}
{Choguita Rarámuri verb classes in terms of stress and tone}\mbox{}

    \ea[]{
    Lexically stressed with lexically specified tone: stress is fixed across paradigms; underlying lexical tone is realized on the stressed stem syllable; grammatical tone may replace lexical tone (in the imperative singular).\\
}
        \ex[]{
        Lexically unstressed with lexically specified tone: stress shifts in shifting environments; the lexical tone of the stem is associated to the stressed stem syllable in neutral contexts; the lexical tone of the suffix is associated to the stressed suffix syllable in shifting contexts; grammatical tone may replace lexical tone (in the imperfective and progressive).\\
    }
            \ex[]{
            Lexically stressed with no lexical tone (alternating roots): stress is fixed across paradigms; surface tone is dependent on the type of inflection, L in neutral environments and HL in shifting environments.\\
        }
                \ex[]{
                Lexically unstressed with no lexical tone (tonal neutralization): lexically unstressed verbs have a lexically specified tone that is deleted after a stress shift. A HL tone is associated to the newly stressed, toneless stem syllable.\\
            }
    \z
\z

\section{The interaction between lexical tone and intonation}
\label{sec: interaction between lexical tone and intonation}

This section provides details of how lexical tone and intonation interact in the tonal grammar of Choguita Rarámuri. As discussed in \chapref{chap: tone and intonation}, lexical tonal contrasts are implemented through a variety of acoustic means, some of which are speaker-dependent and some of which are dependent on the intonational context. As discussed in this section, there is also robust evidence that both lexical and grammatical tones are preserved in tone-intonation interactions (\citealt{caballero2014tone, aguilar2015multi, garellek2015lexical}).

\subsection{Tone-intonation interactions in declaratives}
\label{subsec: tone-intonation interactions in declarative sentences}

As discussed in §\ref{sec: intonation}, Choguita Rarámuri does not only deploy f0 to encode lexical tonal contrasts, but also to encode intonation: declarative sentences exhibit a High boundary tone (H\%) \citep{caballero2014tone, garellek2015lexical}. This is shown in \figref{fig: H boundary tone in declaratives 2} (\figref{fig: H boundary tone in declaratives} in \chapref{chap: tone and intonation}) with a sentence composed of words with lexical L tones, \textit{reˈhòi suˈnù oˈhòli} `The man dekerneled corn' (lexical pitch targets are represented with ``*'' in the first tier; stressed syllables are represented with ``S'' in the bottom tier).

\begin{figure}

\includegraphics[width=\textwidth]{figures/Phonology-img1.png}
\caption{
\label{fig: H boundary tone in declaratives 2}
High boundary tone in declaratives \parencite{garellek2015lexical}}
\end{figure}

As shown here, in a sentence with lexical L tones, H\% boundary tones accommodate lexical tones: both the lexical L pitch target and the post-lexical H pitch target are clearly differentiated.

On the other hand, the rise expected with the presence of a H\% is replaced by a pitch fall if the Intonational Phrase (IP) contains a lexical HL tone at the right edge. This is shown in \figref{fig: no H boundary tone in declaratives with lexical HL tones 2} (\figref{fig: no H boundary tone in declaratives with lexical HL tones} in \chapref{chap: tone and intonation}) with a sentence composed of words with lexical HL tones, \textit{Maˈnuêli oˈkwâ koˈlî iʔˈkîli} `Manuel bit two chili peppers'.

%\break

\begin{figure}
\includegraphics[width=\textwidth]{figures/Phonology-img2.png}
\caption{
\label{fig: no H boundary tone in declaratives with lexical HL tones 2}
No high boundary tone in declaratives with lexical HL tones \parencite{garellek2015lexical}}
\end{figure}

Thus, H\% boundary tones are overridden by lexical falling tones, an effect which may enhance lexical/morpho-lexical tones in phrase final position: lexical tones are clearly differentiated for all speakers across different intonational contexts (see further discussion in \citealt{garellek2015lexical}).

\subsection{Tone-intonation interactions in interrogatives}
\label{subsec: tone-intonation interactions in interrogative constructions}

The morphosyntactic and prosodic properties of different types of interrogative constructions is addressed in §\ref{sec: interrogative constructions}. This section addresses the patterns that emerge in contexts where a lexical tone and an intonational tone conflict in terms of their association in different types of interrogative constructions. Lexical tonal contrasts are prioritized in these constructions.

Different types of interrogative constructions share the following intonational characteristics:

\ea\label{ex: prosodic properties of interrogatives}
{Intonational properties of Choguita Rarámuri interrogative constructions}

\begin{itemize}
    \item A boundary H\% tone targets the last stressed syllable of the utterance.\\
    \item There is raised register across the utterance.\\
    %\item A H* pitch accent that associates with the question word of the construction (if any).\\
\end{itemize}
\z

%re-do the section below to not replicate exactly the stuff in chapter 11

Consider first the behavior of lexical tone in phrase-final position in unmarked polar interrogative constructions. In these cases, the boundary H\% tone aligns with a lexical HL tone in the stressed syllable. \figref{fig: prosody declarative intonation} and \figref{fig: prosody morphosyntactically unmarked polar intonation} (repeated from §\ref{subsubsec: morphosyntactically unmarked polar questions}) show a minimal intonational pair between a declarative utterance and its morphosynactically equivalent, polar interrogative counterpart, respectively. A lexical HL tone is associated with the penultimate syllable of the utterance. As shown in the contrast between the two Figures, f0 is significantly raised in the final stressed syllable of the polar question in \figref{fig: prosody morphosyntactically unmarked polar intonation} due to the association of an interrogative H\% intoneme aligning with the pitch peak of the lexical HL tone in the stressed syllable.


\begin{figure}
\includegraphics[width=\textwidth]{figures/SentenceTypes-img2.png}
\caption{
\label{fig: prosody declarative intonation}
Declarative utterance: \textit{ma ˈtôlo} `S/he buried him/her' (< BFL el1170])}
\end{figure}

%\break

\begin{figure}
\includegraphics[width=\textwidth]{figures/SentenceTypes-img3.png}
\caption{
\label{fig: prosody morphosyntactically unmarked polar intonation}
Morphosyntactically unmarked polar interrogative: \textit{ma ˈtôli?} `Did s/he bury him/her?' < BFL el1307 >}
\end{figure}

In contrast, when an interrogative utterance has a lexical L tone in the final stressed syllable, the lexical tone takes precedence and is preserved in the stressed syllable, and the H\% boundary tone docks on the following, post-tonic syllable. \figref{fig: prosody L tone plus H boundary tone} illustrates this pattern.

%\break

\begin{figure}
\includegraphics[width=\textwidth]{figures/SentenceTypes-img5.png}
\caption{
\label{fig: prosody L tone plus H boundary tone}
Accommodation of L tone and H\% boundary tone in \textit{ma ˈnèli?} `Did s/he bury him/her?' < BFL el1307 >)}
\end{figure}

Finally, in interrogative constructions where the last stressed syllable of the utterance bears a lexical H tone, an interrogative H\% tone associates in the final, unstressed syllable, with a pitch target that is higher than the one associated with the lexical H tone in the stressed syllable. This is illustrated in \figref{fig: prosody content question lexical H tone}, with an example of a content question (see also §\ref{subsec: content questions}).

\begin{figure}
\includegraphics[width=\textwidth]{figures/SentenceTypes-img12.png}
\caption{
\label{fig: prosody content question lexical H tone}
Content question with utterance-final lexical H tone in \textit{ˈhêpi ˈkwâ muˈrú-li} `Who did s/he carry?'}
\end{figure}

\largerpage
In sum, as has been documented in declarative utterances, the three-way lexical tonal contrast of Choguita Rarámuri is preserved in interrogative constructions.

\subsection{Summary}
\label{subsec: summary tone-intonation}

The cross-linguistic study of intonational encoding of tonal languages document several possibilities in terms of the phonological relationship between lexical tonal systems and intonational phonology. These include:

\ea\label{ex: cross-linguistic tone-intonation}
{Cross-linguistic tone-intonation interactions}
\begin{itemize}
\item Manipulation of register (\ili{Hausa} (\ili{Chadic}; Nigeria) \parencite{inkelas1987phonology})
\item Prosodic rephrasing (\ili{Chitumbuka} (\ili{Bantu}; Malawi) \parencite{downing2007focus})
\item Different tonal accommodation strategies (e.g., intonational tones override lexical tones in \ili{Coreguaje} (\ili{Tukanoan}; Colombia) \parencite{gralow1985coreguaje})
\item No use of f0 (\ili{Navajo} (\ili{Athabaskan}) \parencite{mcdonough2001intonation}).
\item Non-tonal encoding of intonation, including lengthening (\ili{Shekgalagari} (\ili{Bantu}; Botswana) (\citealt{hyman2011tonal})).
\end{itemize}
\z

Preservation of lexical and morphological tones in tone-intonation interactions as documented in Choguita Rarámuri have been  reported for other languages,  which include \ili{Serbo-Croatian} \citep{godjevac2006transcribing}, \ili{Stockholm Swedish} \parencite{riad2006scandinavian} and \ili{Curacao Papiamentu} (\citealt{remijsen2005stress}).

In addition to boundary tones interacting with lexical tones, there are intonational pitch targets, ``lead" tones, that are dependent on the lexical tones of Choguita Rarámuri (as discussed in §\ref{subsec: optional rhythmic lead tones}, there are optional post-lexical tones that are opposite to the lexical tones they precede, namely a L target before H and HL lexical tones, and a H tone before a lexical L tone). Lead tones are thus a kind of tone-dependent intonational phenomenon, that may also contribute to preserving lexical tone contrasts and/or their enhancement in different intonational contexts \parencite{garellek2015lexical}.

In addition to tone-intonation interactions concerning f0 effects, Choguita Rarámuri also exhibits the following tone-specific and general non-tonal effects at prosodic boundaries (see §\ref{subsec: non-tonal encoding of intonation}; \citealt{caballero2014tone, aguilar2015multi}):

\ea\label{ex: non-tonal effects of tones at prosodic boundaries}
{Non-tonal effects of Choguita Rarámuri lexical tones at prosodic boundaries}\mbox{}
    \ea[]{
    Rearticulation exclusive to HL tones\\
}
        \ex[]{
        Increased lengthening of L tones\\
    }
    \z
\z

%show an example here?

One important question yet to be addressed in depth in Choguita Rarámuri is the role that multiple dimensions of phonetic realization of tones may be playing in the enhancement of the lexical tone contrast of the language: as reported in \citet{caballero2015tone}, the pitch differences between H, HL and L tones in Choguita Rarámuri are reliable, but very narrow. These results are comparable to those found for other \ili{Uto-Aztecan} tonal languages (e.g., \ili{Balsas Nahuatl} \parencite{guion2010word}, though there are still many gaps in our knowledge in the phonetic implementation of prosodic contrasts in \ili{Uto-Aztecan} languages. While Choguita Rarámuri is not a ``high-density'' tonal language, the functional load of tone may require additional cues to increase the dispersion of the tonal inventory.

%summaruze here the patterns in interrogatives: manipulation of register

\section{Prosodic constraints on morphological shapes}
\label{sec: prosodic constraints on morphological shapes}

This section addresses phenomena in Choguita Rarámuri that may be classified under the rubric of morphologically-determined prosodic templatic effects, where specific morphological constructions impose output prosodic shape constraints. As described in \chapref{chap: verbal morphology}, some verbal morphological constructions induce truncation of the bases to which they attach.  These constructions may be analyzed as imposing prosodic restrictions where surface forms must conform to an output templatic form. As shown below, the prosodic constituents referenced in these templatic constraints are attested and deployed in different areas in the Choguita Rarámuri grammar and are thus independently motivated.

The Choguita Rarámuri constructions associated with truncation are body part incorporation (described in \sectref{subsec: truncation in body-part incorporation}), the denominal suffix construction `make, wear' (described in \sectref{subsec: truncation in denominal verb constructions}), both of which induce truncation of the final syllable in trisyllabic bases, as well as aspect/mood suffixes with alternating long (disyllabic) and short (monosyllabic) prosodic forms. In the analysis proposed here, allomorphy of aspect/mood suffixes that results in the long/short alternation in morphologically complex words results from a general morpho-phonological syllable truncation process, the same mechanism that is deployed in body-part incorporation and denominal constructions to enforce a prosodic templatic shape in output forms. We address each of these constructions next.

\subsection{Truncation in body-part incorporation}
\label{subsec: truncation in body-part incorporation}

As described in \chapref{chap: verbal morphology}, a body-part term may be incorporated to a verb stem, the head of the construction. The result is a single prosodic word, with a single stress in the first syllable of the head of the compound. Disyllabic body part terms do not undergo any prosodic changes in incorporation, but trisyllabic body part terms undergo syllable truncation when combined with verbs in this construction. The contrast between the two types of bases is shown in the examples in (\ref{ex: truncation in body part incorporation}):

\ea\label{ex: truncation in body part incorporation}
{Truncation in body-part incorporation}

    \ea[]{
     {/buˈsi+kaˈsì/		→ \textit{busi-ˈkâsi}}\\
     eye+break	\\
     `to become blind (lit. to eye-break)’\\
     `quedarse ciego (lit. romperse-ojo)' < SFH 06 1:112/el >\\
}
        \ex[]{
        {/tʃ͡aˈmèka+reˈpu/	→	\textit{tʃ͡ameˈrêpu}}\\
        tongue+cut\\
        `to cut one's tongue’	\\
        `cortarse la lengua'   < SFH 07 1:187/el >\\
    }
            \ex[]{
            {/tʃ͡eˈrewa+biʔˈwa/	→	\textit{tʃ͡ere-ˈbîwa}}\\
            sweat+clean’\\
            `to clean sweat’\\
            `limpiarse el sudor'  < SFH 07 1:187/el >\\
        }
    \z
\z

Truncation of nominal bases in noun incorporation ambiguously results from a noun incorporation stress rule, repeated in (\ref{ex: body-part incorporation rule}) below, or the initial three-syllable stress window, which requires stress to be assigned to the first three syllables of the prosodic word (see \chapref{chap: phonology} and \chapref{chap: verbal morphology}).\footnote{On the other hand, there is evidence that syllable truncation may also be a device to satisfy the three-syllable stress window: as described in §\ref{sec: loanword prosody}, \ili{Spanish} loanwords may exhibit truncation of the source word in order to keep the original prominence within the initial three syllable stress window (e.g., \textit{nauguˈraripo} 'let us inaugurate it', from Sp. \textit{\textbf{i}nauguˈrar}, with a truncated first syllable).}

\ea\label{ex: body-part incorporation rule}
\textbf{Body part incorporation stress rule}: the head of the construction must have the stress in the first syllable.\footnote{This stress rule is formalized in \citet{caballero2011morphologically} as ACC-TO-HEAD(1), a language-specific constraint.}\\

\z

\subsection{Truncation in denominal verb constructions in \textit{-ta}}
\label{subsec: truncation in denominal verb constructions}

A second construction that involves truncation of nominal bases is the deverbal \textit{-tâ} `make, wear' suffix, addressed in §\ref{subsubsec: the make/become -ta suffix}. As with body-part terms in noun incorporation, disyllabic nouns attaching this denominal suffix undergo no changes (\ref{ex: truncation in denominal verb forms}a, c), but trisyllabic nouns truncate their final syllable when combining with this construction (\ref{ex: truncation in denominal verb formse}).

%add tones in these examples
\ea\label{ex: truncation in denominal verb forms}
{Truncation in denominal verb forms in \textit{-tâ}}

    \ea[]{
    \glt    {\textit{aʰkaˈrâsa}}\\
    \gll    aʰka-ˈrâ-sa \\
            sandal-\textsc{vblz}-\textsc{cond}\\
    \glt    ‘if they wear sandals’\\
    \glt    `si se enhuaracha'  < SFH 08 1:47/el >\\
}\label{ex: truncation in denominal verb formsa}
        \ex[]{
        cf. \textit{aˈʰkà} \\
        ‘sandal’\\
        `huarache'\\
    }\label{ex: truncation in denominal verb formsb}
        \ex[]{
        \glt    {\textit{noriˈrâma}}\\
        \gll    nori-ˈrâ-ma\\
                cloud-\textsc{vblz}-\textsc{fut.sg}\\
        \glt    `It will become cloudy.'\\
        \glt    ‘Se va a nublar.’  < BFL 04 1:92/el >\\
    }\label{ex: truncation in denominal verb formsc}
            \ex[]{
            cf. \textit{noˈrí} \\
            ‘cloud’\\
            `nube'\\
        }\label{ex: truncation in denominal verb formsd}
            \ex[]{
            \glt    {\textit{sipuˈtâma}}\\
            \gll    sipu-ˈtâ-ma\\
                    skirt-\textsc{vblz}-\textsc{fut.sg}\\
            \glt    `S/he will wear a skirt.'\\
            \glt    `Se va a poner falda.’   < LEL 06 4:185/el >\\
        }\label{ex: truncation in denominal verb formse}
                \ex[]{
                cf. \textit{siˈpútʃ͡a}\\
                ‘skirt'\\
                `falda'\\
            }\label{ex: truncation in denominal verb formsf}
    \z
\z


Like body-part incorporation, this construction imposes an output shape constraint in the derived forms: the derived forms must have third syllable stress, with stress assigned in the derivational suffix. Syllable truncation in this construction thus fulfills the same prosodic function documented in body part incorporation. The prosodic template involved here is analyzed as an extended iambic foot aligned to the left edge of the prosodic word in \citet{caballero2011morphologically}.

\subsection{Truncation in aspect/mood marking constructions}
\label{subsec: truncation in aspect/mood marking constructions}

%% Use the term formative from Bickel & Nichols 2007
Another phenomenon in Choguita Rarámuri where a complex interaction between phonological and morphological factors takes place is one involving prosodic constraints on morphological shapes in aspectual/mood marking. A general description of these suffixes within the context of the verbal morphological structure is provided in \chapref{chap: verbal morphology}, where they are described as aspectual/mood suffixes that are clustered in the Aspectual Stem, a domain within the verbal structure of Choguita Rarámuri comprising suffix positions S6 to S9. As discussed in §\ref{subsec: V-V incorporation constructions}, these constructions result from V-V incorporation.

%Show verbal structure again

The relevant aspectual/mood markers, which encode desiderative, associated motion, and auditory evidential meanings, exhibit an allomorphy pattern contrasting ``long'' (disyllabic) forms and ``short'' (monosyllabic) ones. The correspondences between allomorphs is provided in (\ref{ex: long and short allomorphs of aspect/mood suffixes}):


\ea\label{ex: long and short allomorphs of aspect/mood suffixes}
{Long and short allomorphs of aspect/mood suffixes}

\begin{tabular}{llll}
        &  &\textit{Long alllomorph} & \textit{Short allomorph}\\
     a.& Desiderative &  \textit{-nále} & {\textit{-na}}\\
     b.& {Associated motion}& {\textit{-simi}}&{\textit{-si}}\\
     c.&{Auditory evidential} & {\textit{-tʃ͡ane} }&{\textit{-t͡ʃa}}\\
\end{tabular}
    \z

As shown here, the short (monosyllabic) allormphs correspond transparently to the first syllable of the long (disyllabic) allomorphs. These suffixes are in turnrelated to independent verbal predicates, as shown in (\ref{ex: Aspect/mood suffixes and corresponding independent verbs}):

%add tones
\ea\label{ex: Aspect/mood suffixes and corresponding independent verbs}
{Aspect/mood suffixes and corresponding independent verbs}

\begin{tabular}{lllll}
       &  \textit{Suffix} & &\textit{Verb}\\
     a.& Desiderative &  \textit{-nále} & \textit{naˈkí}&	‘want’\\
     b.& {Associated motion}& {\textit{-simi}}&\textit{siˈmí}&	‘go (singular)’\\
     c.&{Auditory evidential} & \textit{-tʃ͡ane} &	\textit{(a)ˈtʃ͡ane}	&  ‘say, make noise'\\
\end{tabular}
    \z

As can be appreciated from this comparison, the relationship between the suffixes and the corresponding independent verbs is also highly transparent: there is a perfect match in terms of segmental phonological material in the case of the associated motion and the auditory evidential, and a high degree of phonological similarity with the desiderative, where the first syllable of the suffix corresponds to the first syllable of the corresponding, independent verb. In contrast to the independent verb forms, which are prosodically independent and bear stress, the suffixes are prosodically dependent on a verb host.

A single, morphologically complex verb, may exhibit a recursion of attachment of these constructions. Their relative order with respect to each other and other constructions in the complex verb is governed by several factors, including semantic, phonological and morphological factors. An overview of the principles governing affix order is provided in \citet{caballero2010scope}. A separate mechanism governs the form and distribution of allomorphs of aspect/mood markers in Choguita Rarámuri. This is addressed next.

\subsubsection{Allomorph distribution}
\label{subsubsec: allomorph distribution}


The distribution of long and short allomorphs is predictable on the basis of
% whether or not there are
the existence of
outer suffixes attached to a base containing aspect/mood markers: short allomorphs are always followed by other suffixes, while long allomorphs are aligned to the right edge of the prosodic word. These two distributions are shown in (\ref{ex: Short allomorph distribution}) and (\ref{ex: long allomorph distribution}).

\ea\label{ex: Short allomorph distribution}
{Short allomorph distribution}

    \ea[]{
    {[aˈtʃ͡ènisa]}\\
    \glt   /aˈtʃ͡è-nale-sa/ \\
    \glt        pour-\textsc{desid-cond}\\
    \glt    `if s/he wants to pour it'\\
    \glt    `si lo quiere echar, verter'    < SFH 07 romara/tx >\\
}
        \ex[]{{[nahaˈrâpnima]}\\
        \glt    /nahaˈrâpi-nale-ma/\\
        \glt        wrestle-\textsc{desid-fut.sg} \\
        \glt    `s/he will want to wrestle'\\
        \glt    `va a querer luchar'   < BFL 07 1:152/el >\\
    }
            \ex[]{
            {[riʔiˈbûrsili]}\\
            \glt    /reʔe-ˈbû-ri-simi-li/\\
            \glt stone-gather-\textsc{caus-mot-pst} \\
            \glt    `s/he went along making them gather stones'\\
            \glt    `va a querer ir juntando piedras'  < SFH 07 2:63/el >\\
        }
		        \ex[]{
		        {[tiˈtʃ͡iksima]}\\
		        \glt    /tiˈtʃ͡i-ki-simi-ma/\\
		        \glt     comb-\textsc{appl-mot-fut.sg}\\
		        \glt    `S/he will go along making them comb her/him.'\\
		        \glt    `Va a ir haciéndola que la peine.' < SFH 07 2:67/el >\\
		    }
		    \newpage
                    \ex[]{
                    {[aˈtístʃ͡anala]}\\
                    \glt    /aˈtísi-tʃ͡ane-nale-a/\\
                    \glt     sneeze-\textsc{ev-desid-prog}\\
                    \glt    `It sounds like they want to sneeze.'\\
                    \glt    `Suena a que quieren estornudar.'    < SFH 08 1:122/el >\\
                }
    \z
\z

\ea\label{ex: long allomorph distribution}
{Long allomorph distribution}

    \ea[]{
    {[ˈsûrnili]}\\
    \glt  /ˈsû-ri-nale/\\
    \glt     sew-\textsc{cause-desid}\\
    \glt    `S/he wants to make them sew.'\\
    \glt    `Quiere hacerlos coser.' < BFL EDCW(52/el >\\
}
        \ex[]{
        {[poˈlâptinili]}\\
        \glt   /poˈlâ-pi-ti-nale/\\
        \glt       cover-\textsc{rev-refl-desid}\\
        \glt    `S/he wants to uncover herself.'\\
        \glt    `Se quiere destapar.'   < BFL 08 1:56/el >\\
    }
            \ex[]{
            {[ˈnârisimi]}\\
            \glt    /ˈnâri-simi/\\
            \glt        ask-\textsc{mot} \\
            \glt    `S/he is going along asking.'\\
            \glt    `Va a ir preguntando.'   < SFH 08 1:148/el >\\
        }
                \ex[]{
                {[toˈrétʃ͡ani]}\\
                \glt   /toˈré-tʃ͡ane\\
                \glt        cackle-\textsc{ev}\\
                \glt    `It sounds like cackling.'\\
                \glt    `Suena a que están cacareando.' < SFH 07 1:7/el >\\
            }
                    \ex[]{
                    {[uˈbástʃ͡ani]}\\
                    \glt  	/uˈbá-simi-tʃ͡ane/\\
                    \glt    bathe-\textsc{mot-ev}\\
                    \glt    `It sounds like they are going along bathing.'\\
                    \glt    `Suena que van bañándose.'  < SFH 08 1:150/el >\\
                }
    \z
\z

One exception to this distribution is found in morphologically complex words containing an unstressed root that immediately attaches a desiderative suffix: in these cases, the long allomorph of the suffix is selected, regardless of whether there are outer suffixes or not. This pattern results from the stress properties of the suffix and the bases with which it may combine: the desiderative suffix is the only stress shifting suffix among the aspect/mood markers described here (the associated motion and the evidential are stress neutral). The long, stressed allomorph may undergo post-tonic syncope of the final suffix vowel when combining with unstressed roots, but is always attested in its long form. The relevant examples are shown in (\ref{ex: stressed allomorph of desiderative}).

\ea\label{ex: stressed allomorph of desiderative}
{Stressed allomorph of the desiderative}

    \ea[]{
    {[ronoˈnáli]}\\
    \glt    /ronò-ˈnále/\\
    \glt        boil-\textsc{desid}\\
    \glt    `It's about to boil (lit. it wants to boil).'\\
    \glt    `Está a punto de hervir (lit. quiere hervir).'   < SFH 08 1:125/el >\\
}\label{ex: stressed allomorph of desiderativea}
        \ex[]{
        {[kotʃ͡iˈnálsiani]}\\
        \glt   /kotʃ͡i-ˈnále-simi-a=ni/\\
        \glt     sleep-\textsc{desid-mot-prog=1sg.nom} \\
        \glt    `I am going along wanting to sleep.'\\
        \glt    `Voy a ir queriendo dormir.'   < BFL 08 1:60/el >\\
    }\label{ex: stressed allomorph of desiderativeb}
            \ex[]{
            {[awiˈnálsili]}\\
            \glt   /awi-ˈnále-simi-li/\\
            \glt       dance-\textsc{desid-mot-pst}’ 	\\
            \glt    `S/he went along wanting to dance.'\\
            \glt    `Se fue queriendo bailar.'    < SFH 08 1:75/el >\\
        }\label{ex: stressed allomorph of desiderativec}
                \ex[]{
                {[koʔˈnálimi]}\\
                \glt   /koʔ-ˈnále-mi/\\
                \glt    eat-\textsc{desid-irr.sg}\\
                \glt    `S/he might want to eat.'\\
                \glt    `A lo mejor va a querer comer.' < SFH 08 1:122/el >\\
            }\label{ex: stressed allomorph of desideratived}
    \z
\z

As (\ref{ex: stressed allomorph of desiderativeb}) and (\ref{ex: stressed allomorph of desiderativec}) show, the last vowel of the long allomorph may undergo deletion (\ref{ex: stressed allomorph of desiderativeb}), but there is no evidence for syllable truncation, regardless of the presence or absence of outer suffixes.

Outside of this pattern, long and short allomorph distribution is not dependent on the stress properties nor the stress position within the stem. This is shown in (\ref{ex: allomorph distribution and stress position}) and (\ref{ex: allomorph distribution and stress position 2}).

%\pagebreak

\ea\label{ex: allomorph distribution and stress position}
{Allomorph distribution and stress position: immediately post-tonic position}\\

    \ea[]{
    {[noriˈwísimi]}\\
    \glt   /noriˈwí-simi/\\
    \glt        vanish-\textsc{mot}\\
    \glt    ‘It goes along vanishing.’\\
    \glt    `Se va desapareciendo.'  [FLP in61(482)/in >\\
}
        \ex[]{
        {[wikaˈrâsika]}\\
        \glt    /wikaˈrâ-si-ka/\\
        \glt        sing-\textsc{mot-ger}\\
        \glt    ‘It was going along singing.’\\
        \glt    `Iba de pasada cantando.'   < BFL 06 EJP(10)/el >\\
    }
    \z
\z

\ea\label{ex: allomorph distribution and stress position 2}
{Allomorph distribution and stress position: non-immediately post-tonic position}\\

        \ea[]{
            {[ˈkétʃ͡isimi]}\\
            \glt   /ˈkétʃ͡i-simi/\\
            \glt        chew-\textsc{mot}\\
            \glt    ‘S/he goes along chewing.'\\
            \glt    `Va queriendo mascar.' < SFH 08 1:145/el >\\
        }
                \ex[]{
                {[poˈlâptisio]}\\
                \glt  /poˈlâ-p-ti-si-o/	\\
                \glt     cover-\textsc{rev-refl-mot-ep}\\
                \glt    ‘It goes along uncovering itself.’\\
                \glt    `Va destapándose.'  < BFL 08 1:56/el >\\
            }
    \z
\z

Long and short allomorphs are not lexically conditioned by their bases, either. This is shown in (\ref{ex:  no lexical conditioning}).

\ea\label{ex: no lexical conditioning}

    \ea[]{
    {[oˈpés\textbf{tʃane}]}\\
    \glt  /oˈpési-\textbf{tʃane}/\\
    \glt        vomit-\textsc{{ev}}\\
    \glt    ‘It sounds like somebody is throwing up.’\\
    \glt    ‘Se oye que vomitan.’    < BFL 07 rec300/el >\\
}
%\pagebreak
        \ex[]{
        {[oˈpés\textbf{tʃi}nilo]}\\
        \glt    /oˈpési-\textbf{tʃane}-nale-o/\\
        \glt        vomit-\textsc{{ev}-desid-ep}\\
        \glt    ‘It sounds like somebody wants to throw up.’\\
        \glt    ‘Se oye que quieren vomitar.’  < BFL 07 rec300/el >\\
}
            \ex[]{
            {[ˈnâri\textbf{simi}]}\\
            \glt    /ˈnâri-\textbf{simi}/\\
            \glt        ask-\textsc{{mot}}\\
            \glt    ‘He goes along asking.’\\
            \glt    ‘Va preguntando.’  < SF 08 1:148/el >\\
        }
                \ex[]{
                {[ˈnâr\textbf{si}ma]}\\
                 \glt  /ˈnâri-\textbf{simi}-ma/\\
                \glt    ask-\textsc{{mot}-fut.sg}\\
                \glt    `S/he will go along asking.’\\
                \glt    ‘Va a ir preguntando.’   < SFH 08 1:148/el >\\
            }
    \z
\z

Long allomorphs of aspect/mood markers may be followed by vocalic TAM suffixes. These vocalic suffixes replace the final vowel of the allomorph, as shown in (\ref{ex: Long allomorph vowel replacement}).

\ea\label{ex: Long allomorph vowel replacement}
{Replacement of final vowels of long allomorphs}

    \ea[]{
    {[kaˈtʃ͡ísnili]}\\
    \glt    /katʃ͡í-simi-nale-i/\\
    \glt        spit-\textsc{mot-desid-impf}\\
    \glt    `S/he wants to go along spitting.'\\
    \glt    `Quiere ir escupiendo.' < SFH 08 1:75/el >\\
}
        \ex[]{
        {[toˈnáltʃ͡ino]}\\
        \glt    /tò-nále-tʃ͡ane-o/\\
        \glt        take-\textsc{desid-ev-ep}\\
        \glt    `It sounds like they want to take it.'\\
        \glt    `Se oye que se lo quieren llevar.' < BFL 06 5:148/el >\\
    }
    \z
\z

As shown in (\ref{ex: no replacement with short allomorphs}), short allomorphs do not have their vowel replaced by these vocalic suffixes.

\pagebreak

\ea\label{ex: no replacement with short allomorphs}
{No vowel replacement with short allomorphs}

    \ea[]{
    {[kotʃ͡iˈnálsia]}\\
    \glt  	/kotʃ͡i-ˈnále-simi-a/\\
    \glt       sleep-\textsc{desid-mot-progr}\\
    \glt    `S/he is going along wanting to sleep.' \\
    \glt    `Va queriéndose dormir.'   < BFL 08 1:60/el >\\
}
    \z
\z

The hypothetical and unattested form *\textit{kotʃ͡iˈ-nal-s-a}, where the progressive \textit{-a} suffix would replace the vowel of the short allomorph of the associated motion (\textit{-si}), is ungrammatical with the intended reading of example (\ref{ex: no replacement with short allomorphs}).

A single morphologically complex word may contain more than one aspect{\slash}mood marker, given that these constructions are semantically compatible and do not exhibit any blocking effects. In these cases, the main generalizations about their distribution is the same as the one stated above: if a base containing two aspect/mood markers attaches more suffixes, then both aspect/mood markers will be short. This is shown in (\ref{ex: short-short sequences}).

\ea\label{ex: short-short sequences}

    \ea[]{
    {[ˈpák\textbf{sini}mi]}\\
    \glt    /ˈpáki-\textbf{simi-nale}-mi/\\
    \glt        brew-\textsc{{mot-desid}-irr.sg}\\
    \glt    `Perhaps s/he may want to go along brewing.'\\
    \glt    `A lo mejor va a querer ir colando.'  < SFH 08 1:147/el >\\
}
        \ex[]{
        {[iˈtʃí\textbf{nsi}ma]}\\
        \glt    /iˈtʃí-\textbf{nale-simi}-ma/\\
        \glt        sow-\textsc{{desid-mot}-fut.sg}\\
        \glt    `S/he will go along wanting to sow.'\\
        \glt    `Va a querer ir sembrando.'  < LEL 06 EDCW123/el >\\
    }
    \z
\z

\subsubsection{A prosodically motivated morpho-phonological alternation}
\label{subsubsec: a prosodically motivated morphophonological alternation}

The differences in prosodic shape between long and short allomorphs may be analyzed as resulting from a phonological process of syllable truncation, albeit one that is morphologically-conditioned, as the syllable truncation processes attested in body part incorporation and denominal verb constructions with the \textit{-ta} suffix. The claim here is that syllable truncation, whether deleting a stem syllable or an affix syllable, is deployed in specific morphological constructions in order to satisfy the prosodic shape output forms must have.

The prosodic shape of aspect/mood markers in Choguita Rarámuri was treated in \citet{caballero2008choguita} as instantiating a suppletive allomorphy pattern. The distinction between a suppletive allomorphic pattern, a morpho-lexical distinction, vs. a morpho-phonological alternation, where allomorphs are derived via phonological processes from a single underlying form, is not a trivial one (\citealt{kiparsky1996allomorphy,paster2006phonological}).\footnote{Two criteria discussed in \citet{kiparsky1996allomorphy} are (i) whether the allomorphy pattern is specific to lexical items (suppletion) or general in the language (morpho-phonology), and (ii) whether the allomorphy pattern involves more than one segment (suppletion) or a single segment (morpho-phonology). The argument being that it is expected that morpho-phonological alternations are predictable based on general phonological processes and that general phonological processes generally target a single segment.} The proposal here is that short allomorphs are derived from long allomorphs via a morphologically specific process of syllable truncation.

Choguita Rarámuri aspect/mood markers form a single morphological construction, which can be analyzed as involving V-V incorporation. The `long' allomorphs are disyllabic, matching the canonical shape of stems in the language. These constructions are analyzed here as associated to a prosodic template, a bimoraic foot (σμσμ,  σμμ) which matches the minimal prosodic word in Choguita Rarámuri (more details are provided in \sectref{sec: defining the prosodic word and other prosodic domains in CR} below) aligned to the \textit{right} edge of the prosodic word.  Relevant examples showing the proposed metrical structure of output forms is provided in (\ref{ex: a right-aligned prosodic template}):

\ea\label{ex: a right-aligned prosodic template}
{A bimoraic foot aligned to the right edge of the prosodic word}

    \ea[]{
    \glt    /koˈʔá-nale/		→		[(koˈʔá)\textsubscript{Ft}-(nale)\textsubscript{Ft}]\textsubscript{PrWd}\\
    \glt        eat-\textsc{desid}\\
    \glt    `S/he wants to eat.'\\
    \glt    ‘Quiere comer.’\\
}
        \ex[]{
        \glt    /noriˈwi-simi/	→	[(<no>riˈwi)\textsubscript{Ft}-(simi)\textsubscript{Ft}]\textsubscript{PrWd}\\
        \glt        vanish-\textsc{mot}\\
        \glt    `It goes along vanishing.'\\
        \glt    ‘Iba desapareciendo.’\\
    }
            \ex[]{
            \glt   /toˈre-tʃ͡ane/	→	[(toˈre)\textsubscript{Ft}-(tʃ͡ane)\textsubscript{Ft}]\textsubscript{PrWd}\\
            \glt        cackle-\textsc{ev}\\
            \glt    ‘It sounds like they are cackling.’ \\
            \glt    `Suena que cacarean.'\\
        }
    \z
\z

TAM vocalic suffixes that replace the stem final vowel do not alter the templatic requirement, allowing these vocalic suffixes to attach to disyllabic suffixes without inducing truncation. The parsing of these suffixes into the metrical structure is as follows:

\ea\label{ex: Parsing of TAM vocalic suffixes}
{Parsing of vocalic TAM suffixes with long allomorphs}

    \ea[]{
    \glt    /katʃ͡í-simi-nale-i/	→	[(kaˈtʃ͡í-s)\textsubscript{Ft}-(nil-i)\textsubscript{Ft}]\textsubscript{PrWd}\\
    \glt        spit-\textsc{mot-desid-impf} \\
    \glt    `S/he used to want to go along spitting.'\\
    \glt    ‘Quería ir escupiendo.’\\
}
        \ex[]{
        \glt    /tò-nále-tʃ͡ane-o/	→	 [(to-nál)\textsubscript{Ft}-(tʃ͡in-o)\textsubscript{Ft}]\textsubscript{PrWd}\\
        \glt        take-\textsc{desid-ev-ep}\\
        \glt    `It sounds like s/he wants to take it.'\\
        \glt    ‘Se oye que se lo quiere llevar.’ \\
    }
    \z
\z

TAM suffixes with the CV shape do induce truncation of aspect/mood suffixes in order to satisfy the prosodic template requirement associated with these constructions. This is shown in (\ref{ex: Parsing of CV suffixes}), where deleted syllables are highlighted in boldface in the underlying representation.

\ea\label{ex: Parsing of CV suffixes}
{Truncation with TAM suffixes with CV shape}

    \ea[]{
    \glt    /aˈtʃ͡è-na\textbf{le}-sa/	→	(aˈtʃ͡è)\textsubscript{Ft}-(ni-sa)\textsubscript{Ft}]\textsubscript{PrWd}\\
     \glt       pour-\textsc{desid-cond}\\
    \glt    `If s/he wanted to pour it (salt).'\\
    \glt    ‘Si quisiera echarle (sal).’ \\
}
        \ex[]{
        \glt    /reʔe-ˈbû-ri-si\textbf{mi}-li/	→ (<ri>ʔi-ˈbû-r)\textsubscript{Ft}-(si-ri)\textsubscript{Ft}]\textsubscript{PrWd} \\
        \glt    stone-gather-\textsc{caus-mot-pst}\\
        \glt    `S/he made him/her go along gathering stones.'\\
        \glt    ‘Fue haciéndolo juntar piedras.’ \\
    }
    \z
\z

Other aspects of the phonological shape of long and short allomorphs of aspect/mood suffixes is predictable based on fully general phonological processes of the language, namely the vowel quality of individual forms and the possibility of vowel deletion dependent of stress placement and position within the prosodic word. Specifically, long and short allomorphs undergo post-tonic vowel reduction or deletion, which are widespread in the language (and as described in §\ref{subsubsec: stress-based vowel reduction and deletion}): non-final post-tonic vowels may be neutralized in terms of their height (e.g., /-tʃ͡\textbf{a}ne/ [-tʃ͡ine] `\textsc{ev}’, /-n\textbf{a}le/ [-ni] `\textsc{desid}') (e.g., (\ref{ex: vowel reduction and deletion of long and short allomorphs}a--b) or they may be deleted (e.g., /-nal\textbf{e}/ [-nal] ‘\textsc{desid}’, /-s\textbf{i}/ [-s] ‘\textsc{mot}’) (e.g., (\ref{ex: vowel reduction and deletion of long and short allomorphs}c--d)).

%\pagebreak

\ea\label{ex: vowel reduction and deletion of long and short allomorphs}
{Vowel reduction and deletion of long and short allomorphs}

    \ea[]{
    {[rosoˈwâtʃ\textbf{i}no]}\\
    \glt   /rosowâ-tʃ\textbf{a}ne-o/\\
    \glt        cough-\textsc{ev-ep} \\
    \glt    ‘It sounds like coughing.’ \\
    \glt    ‘Se oye que tosen.’  < BFL 07 rec301/el > \\
}
        \ex[]{
        {[naˈlàn\textbf{i}ma]}\\
        \glt   /naˈlà-n\textbf{a}le-ma/\\
        \glt        cry-\textsc{desid-fut.sg}\\
        \glt    ‘S/he will want to cry.’\\
        \glt    ‘Va a querer llorar.’ < SFH 08 1:125/el >\\
    }
            \ex[]{
            {[koʔˈnált\textbf{i}ma]}\\
            \glt   /koʔa-ˈnál\textbf{e}-ti-ma/\\
            \glt    eat-\textsc{desid-caus-fut.sg}\\
            \glt    ‘She will make him want to eat.’ \\
            \glt    ‘Lo va a querer hacer comer.’   < SFH 07 EDCW(30)/el >\\
        }
                \ex[]{
                {[wikuˈbastʃ͡ani]}\\
                \glt   /wikuˈba-s\textbf{i}mi-tʃ͡ane/ \\
                \glt        whistle-\textsc{mot-ev}\\
                \glt   ‘It sounds like someone is going along whistling.’\\
                \glt    ‘Se oye como que van chiflando.’    < SFH 08 1:158/el >\\
            }
    \z
\z

%A final note about the analysis of these formatives as verbal compounds - keeping the analysis of suffixes in the Aspectual stem?

There are alternative analyses for the form and distribution of long and short allomorphs of aspect/mood suffixes in Choguita Rarámuri. One alternative would involve positing syllable truncation as a strategy to reduce the number of unparsed syllables in Prosodic Words. This hypothesis would be similar to proposals made for \ili{Hopi} (Southwest US) (\citealt{gouskova2003deriving}) and \ili{Southeastern Tepehuan} (\ili{Tepiman}; Mexico) (\citealt{kager1997rhythmic}).

While Choguita Rarámuri lacks secondary stress, a possible alternative analysis of syllable truncation would involve optimization of metrical structure if truncation is assumed to take place when affixation contributes unparsed syllables post-tonically. Such an analysis requires the following assumptions: (i) there is iterative syllable parsing; (ii) feet are binary; and (iii) the conditioning environment for syllable truncation is rendered opaque after truncation. This hypothetical analysis can be exemplified with a comparison between the posited metrical structure of actual forms and corresponding hypothetical abstract forms with no syllable truncation. It would thus be possible to posit that syllable truncation in these contexts is due to a mechanism that reduces unparsed syllables within Prosodic Words in Choguita Rarámuri (a mechanism enforced by a constraint such as Parse-σ or another similar rule or constraint).

This hypothesis would explain syllable truncation patterns in morphologically complex words with more than one aspect/mood disyllabic suffix (recursive compounding), where truncation would result in a form with hypothetical exhaustive parsing. This is exemplified with the analyzed form in (\ref{ex: truncation in recursive affixation}).

\ea\label{ex: truncation in recursive affixation}

    \ea[]{
    \glt /uba-simi-tʃ͡ane/ 		→	(uˈba-s)\textsubscript{Ft}-(tʃ͡ani)\textsubscript{Ft}]\textsubscript{PrWd}\\
            bathe-\textsc{mot-ev}\\
    \glt    `It sounds like they are going along bathing.'\\
    \glt    ‘Suena que se van bañando.’\\
}
    \ex[]{
    \glt    cf. /simi/ → [-s]\\
    }
    \z
\z

This hypothesis would also predict that truncation would apply to every eligible disyllabic suffix if the output form would be exhaustively parsed into metrical feet. However, and as shown in the examples in (\ref{ex: no truncation with multiple affixation}), truncation does not take place every time there is recursive affixation of aspect/mood markers, even if this would optimize the surface phonological form (these examples are shown with a foot aligned to the right edge of the Prosodic word, but an alternative parsing does not affect the main argument):

\ea\label{ex: no truncation with multiple affixation}

    \ea[]{
    \glt    /raraˈhîpa-ti-tʃ͡ane/	→	(raraˈhîp)\textsubscript{Ft}ti(tʃ͡ane)\textsubscript{Ft}\\
            run.rarajipa-\textsc{caus-ev}\\
    \glt    `It sounds like they are making them run rarajipa.'\\
    \glt    ‘Suena que los están haciendo correr rarajipa.’\\
}
    \ex[]{
    \glt   *(raraˈhîp)\textsubscript{Ft}(ti-tʃ͡i)\textsubscript{Ft}\\
}
        \ex[]{
        \glt   /aˈwí-ri-si-nale/	→	(aˈwí-r)\textsubscript{Ft}-si-(nili)\textsubscript{Ft}\\
                dance-\textsc{caus-mot-desid}\\
        \glt    `S/he wants to make her/him go long dancing.'\\
        \glt    ‘Quiere ir haciéndola bailar’\\
    }
    \z
\z

As shown in these examples, the attested forms have an unparsed syllable, while the hypothetical but unattested words with recursive syllable truncation would yield Prosodic Words with syllables exhaustively parsed into metrical feet. Thus, a single mechanism of syllable truncation to reduce unparsed syllables does not predict the attested surface forms in these contexts in Choguita Rarámuri.

In sum, the alternation that yields monosyllabic allomorphs from disyllabic ones for aspect/mood suffixes in Choguita Rarámuri does not result in global phonological optimization in the language.
%Will have to contextualize what "global optimization" means in this case

\subsection{Prosodic templates in Choguita Rarámuri}
\label{subsec: prosodic templates in CR}

The proposal set forth here is that aspect/mood markers are a single type of construction associated to a syllable truncation process, a phenomenon also associated to body part incorporation constructions and deverbal constructions with the \textit{-ta} `wear/make' suffix (addressed in §\ref{subsec: truncation in body-part incorporation} and §\ref{subsec: truncation in denominal verb constructions}, above).

Syllable truncation satisfies different prosodic templates: as discussed in (X), a morphological stress rule is proposed for body part incorporation, requiring the head of the construction, the second member of the compound, to have stress in its first syllable. The first member of the compound may undergo truncation to satisfy this requirement. The same prosodic output shape is involved in the deverbal construction with the \textit{ta} `make/wear' suffix. In both cases, the resulting surface form has an extended iambic foot aligned to the left edge of the prosodic word. This prosodic shape is schematized in (\ref{ex: prosodic template in body part incorporation and deverbal construction}):

\ea\label{ex: prosodic template in body part incorporation and deverbal construction}
{Prosodic template in body part incorporation and deverbal constructions}

(<σ>σˈσ)\textsubscript{PrWd}

\z

In the case of aspect/mood suffixes, the proposal above is that the prosodic template associated with these constructions involves a bimoraic foot (σ\textmu\textmu, σ\textmuσ\textmu). This prosodic template is aligned to the right of the Prosodic Word. This prosodic shape is schematized in (\ref{ex: prosodic template with aspect/mood markers}):

\ea\label{ex: prosodic template with aspect/mood markers}
{Prosodic template in aspect/mood markers}

(σ\textmuσ\textmu)\textsubscript{PrWd}\\
(σ\textmu\textmu)\textsubscript{PrWd}\\


\z

%Aspect/mood suffixes as verbal compounds
%Evidence for compounding vs. phrase: a single set of inflection; a single syntagmatic prominence (stress-accent); no internal modifiers
% A single Prosodic word
% Prosodic template: segments and prosodic form of the complex word are mapped onto a particular prosodic shape
% references to tenplatic word formation
% Reference to \ili{Guarijío} abbreviated reduplication (templatic back-copying)
% The bimoraic foot is a relevant unit in CR word formation

Given their transparent relationship with independent verbs, these aspect{\slash}mood markers may be analyzed as involving \textit{verbal compounding} as part of a process of V-V incorporation. Their association with a minimal Prosodic Word would thus be consistent with the observation that stems are canonically disyllabic, while affixes are canonically monosyllabic. In \ili{Mountain Guarijío}, \citet{miller1996guarijio} analyzes some verbal suffixes as straddling the boundary between verbal compounds and suffixes, including the associated motion form \textit{-si}, the cognate form of Choguita Rarámuri associated motion \textit{-simi} suffix, which he analyzes as an intermediate form between a compounded verbal stem and a suffix.

% morphosyntactic criteria for considering them compounds?

The allomorphy pattern documented with aspect/mood suffixes provides some clues about the prosodic organization of Choguita Rarámuri. There are two morphoprosodic restrictions in morphologically complex words in Choguita Rarámuri:

\begin{itemize}
    \item   Left edge of the prosodic word: stress is assigned within a three-syllable stress window. This is an exception-less generalization. A left aligned, extended iambic foot is associated with some morphological constructions.
    \item  Right edge of the prosodic word: a minimal prosodic word template is associated with some morphological constructions.
\end{itemize}
