\chapter{Complex predicate constructions}
\label{chap: complex predicate constructions}

Complex predicates have been variously characterized in the literature depending on particular properties of the morphosyntactic type of the language under consideration. However, there seems to be agreement in the literature about two essential properties of these constructions, namely the presence of a complex argument structure and a grammatical functional structure of a single predicate \citep{butt1995structure}, p. 108. That is, there will be a single ‘inflectional domain’ (e.g., with a single subject or object argument) containing two or more distinct predicates, each of which selects for at least one argument in its argument structure \citep[247]{baker1997thematic}.

There are several constructions in Choguita Rarámuri that may fit this definition. This chapter is devoted to these constructions and presents a description and examples of each type, as well as discussion of the potential analytical problems in discerning the difference between multipredicate sequences that are monoclausal (light verb constructions, auxiliaries and serial verb constructions) and multipredicate sequences that involve more than one inflectional domain (complementation structures and adverbial clauses that do not involve overt complementizers). More details of the properties of each type of construction might further clarify these distinctions, including:

\begin{itemize}
\item Scope of negation
\item Scope and relative position of adverbs (modifying the whole verb phrase or just a single predicate?)
\item Differences between these structures and coordination
\item Scrambling possibilities
\end{itemize}

%For future fieldwork:It would be interesting to determine the structure (if any) of the verbal complex. Are there any restrictions in the ordering of main verbs, light verbs, auxiliaries and clitics? Are light verbs and auxiliaries restricted to follow the main verb? Can the elements of a complex predicate construction be discontinuous?} 


This chapter is organized as follows. In §\ref{sec: light verb and auxiliary constructions}, I consider light verb and auxiliary constructions. In §\ref{sec: serial verb constructions}, I present a set of constructions that could be characterized as serial verb constructions. Finally, in §\ref{sec: positionals in complex predicate constructions} I discuss multipredicate forms that involve positional predicates. 

\section{Light verb and auxiliary constructions}
\label{sec: light verb and auxiliary constructions}

The term ‘light verb’ may be employed in the literature to characterize a wide range of structures cross-linguistically (\citealt{brugman2001light}, \citealt{brugman2001light}, \citealt{butt2001semi}, \citealt{bowern2004bardi}), but it is systematically employed to refer to a class of verbs that is semantically bleached or lexically weak and that appears in a multipredicate construction. In contrast, multipredicate constructions involving auxiliary verbs, although also making use of a semantically weak or bleached verb, involve a system of verbs that may be used contrastively as part of a paradigmatic set \citep{gaby2006grammar}[428].

There are a few constructions in Choguita Rarámuri that potentially fit the definition of auxiliary or light verb constructions. These are: the \textit{noká} ‘do’ construction (§2.1), the \textit{ní-} ‘do’ construction (§2.2), the \textit{ishí} ‘do’ construction (§2.3), and the \textit{orá} ‘make’ construction (§2.4).

\subsection{The \textit{noká} }\textbf{‘do’ construction}
\label{subsec: the noka 'do' construction}

As a free form, the verb \textit{noká} means ‘move’, a predicate that describes a change in posture. Examples of this free standing verb are given in (1-3). Valence of the predicate is marked through stem ablaut (\textit{noko-/noká} ‘move, intransitive’ and \textit{noké} ‘move, applicative’).

\begin{itemize}
\item \textit{ma} \textbf{\textit{noká-ri}}
\end{itemize}

/ma  \textbf{noká-li}/

  already  \textbf{move.\textsc{intr-pst}}

  ‘I already moved’

  ‘Ya me moví’

  [BFL 05 1:114/el] 

\begin{itemize}
\item \textit{nihé      ma} \textbf{\textit{noko-méa}} 
\end{itemize}

/nehé    ma    \textbf{noko-méa}/

  \textsc{1sg.nom}    already   \textbf{move\textsc{.intr-fut.sg}} 

    ‘I will move’  

  ‘Ya me voy a mover’    

   [SFH 05 1:80/el] 

\begin{itemize}
\item \textit{nihé      mi      troka} \textbf{\textit{noké-ri}}  
\end{itemize}

/nehé    mi    tróka  \textbf{noké-li}/

  1\textsc{sg.nom    2sg.acc}   truck  \textbf{move.\textsc{appl-pst}}   

  ‘I will move the truck for you’  

  ‘Te voy a mover la troca’    

  [SFH 05 1:80/el]  

This predicate has a semantically bleached version found in multipredicate constructions, acting as a light verb bearing inflection (e.g. (4-9)). 

\begin{itemize}
\item    \textbf{\textit{napawí-a     noká-ri}} \textit{ré   étʃ͡i   na   biré   rihó}  
\end{itemize}

    /\textbf{napawí-a    noká-li} aré ét͡ʃi  na  bilé  rihói/   

    \textbf{get.together-\textsc{prs}} \textbf{do-\textsc{pst}} \textsc{dub  dem  dem}  one  man

    \textit{a’rí   biré   tʃ͡abótʃ͡i     ʃí'i}

    /a’rí  bilé  t͡ʃabót͡ʃi  sí/

    and  one  mestizo  also

    ‘A (Rarámuri) man and a \textit{mestizo} (mixed mexican) man got together’

    ‘Se juntaron un hombre (rarámuri) y un mestizo’

 [SFH 06 choma(2)/tx]

\begin{itemize}
\item   \textit{pe   birá   [}\textbf{\textit{risó     noko-ká]} }\textit{éni-ri       tʃ͡ó}   
\end{itemize}

    just  really  \textbf{struggle.\textsc{prs}} \textbf{do-sim} go.around.\textsc{pl-pst}  also  

   \textit{étʃ͡i   rihó   á   batʃ͡á   bahí-sa   ka   étʃ͡i   tʃ͡o’má   ba}

    \textsc{dem}  man  \textsc{aff}  first  drink-\textsc{cond  emph  dem}  snot  \textsc{cl}

    ‘They would go around struggling too if the man would have drank the snot first’

    ‘Anduvieran también batallando si el hombre hubiera tomado primero el moco’

    [SFH 06 choma(32)/tx]

\begin{itemize}
\item    \textit{náp   arí  í   birá ...} \textbf{\textit{napabú-a   nokí-ra}} \textit{étʃ͡i   tʃ͡abótʃ͡i}   
\end{itemize}

    \textsc{rel}  then  here  really  \textbf{gather-\textsc{prs} }\textbf{do-\textsc{mpass}}  \textsc{dem}  mestizo

    \textit{a'rí   étʃ͡i   rihó   úa   ba}

    and   \textsc{dem}  man  with  \textsc{cl}

    ‘When the man and the mestizo were gathered’

    ‘Cuando los juntaron al hombre y al mestizo’    

    [SFH 06 choma(36)/tx]

\begin{itemize}
\item \textit{nápu   riká=ti   tamó     kút͡ʃua-ra   t͡ʃo   bené-ma  t͡ʃo   na}
\end{itemize}

\textsc{rel}  like=1\textsc{pl.nom}  1\textsc{pl.nom}  children-\textsc{poss}  also  learn-\textsc{fut.sg}  also  then  

  \textit{t͡ʃú=timi   riká=m} \textbf{\textit{náwa     noká}} \textit{na} \textbf{\textit{ot͡ʃér-a}} \textbf{\textit{noká}} 

Q=2\textsc{pl.nom}  like=\textsc{dem}  \textbf{arrive.\textsc{prs}} \textbf{do\textsc{.prs}}  then  \textbf{grow-\textsc{prs} }\textbf{do\textsc{.prs} }

\textit{t͡ʃabé    ki’á   ba}

  before     before  \textsc{cl}

‘Like the way we (did it), our children will learn how you all grew up like before, how you used to live’

‘Así como nosotros, nuestros hijos también van a aprender cómo crecieron antes, como vivían’

 [SFH 06 in61(11)/in]

\begin{itemize}
\item \textbf{\textit{nabisúr-a   nokí-t͡ʃino}}
\end{itemize}

\textbf{form.line-\textsc{prs}} \textbf{move-\textsc{ev}} 

‘It sounds like they are forming a lie’    

‘Se oye que se andan acomodando’

[07 elación 30 de abril]

\begin{itemize}
\item \textit{a’rí ...   ét͡ʃi ...  ét͡ʃi   na   bat͡ʃókiri ...  ke=m ...  harré   ko=ti ...}
\end{itemize}

  and  \textsc{dem}  \textsc{dem}  \textsc{dem}  mud    \textsc{neg}=\textsc{dem} some  \textsc{emph}=1\textsc{pl.nom}

\textit{ké   t͡ʃikó=timi     risot͡ʃí má=m} \textbf{\textit{katéw-a   noká-ra}} \textit{ru?} 

\textsc{neg}  t͡ʃikó=2\textsc{pl.nom} caves  already=\textsc{dem} \textbf{keep-\textsc{prs}} \textbf{do-\textsc{rep}}    say.\textsc{prs}

‘And about this mud, don’t some say that you all used to keep (corn) like that in caves?’

‘Y este zoquete, que no dicen que unos guardaban así en cuevas?’

  [SFH 06 in61(75)/in]

 The verb \textit{noká} has a related form \textit{onoká} (e.g. (10-13)). I analyze this form as deriving from plural/intensive marking, through a prefix which is aligned in color with the vowel quality of the first vowel of the root (cross-reference here to the verbal morphology chapter). While the semantically full version of this predicate may be used in contexts involving multiple participants or intensive aspect, it is not clear that the light verb \textit{onoká} marks plurality/intensive. It is also necessary to determine if there are any possible semantic differences between \textit{noká} and \textit{onoká} in complex predicate constructions.

\begin{itemize}
\item \textit{t͡ʃú=timi  riká=m} \textbf{\textit{omáwa     onoká-ri} } \textit{ba ...} 
\end{itemize}

   Q=2\textsc{pl.nom}  like=\textsc{dem}  \textbf{make.party.\textsc{prs}} \textbf{do.\textsc{pl}}\textbf{{}-}\textbf{\textsc{pst} }\textsc{Cl}  

  \textit{t͡ʃú=timi   riká=m   kayéna       ba   ...}

  Q=2\textsc{pl.nom}  like=\textsc{Dem}   yield.harvest\textsc{.prs}  \textsc{Cl}

‘How did you make the parties? How good were the harvests?’

  ‘Cómo hacían ustedes las fiestas? Cómo se les daban las cosechas?’

 [SFH 06 in61(13)/in]

\begin{itemize}
\item \textit{sunú   shuwábuka   t͡ʃú=timi   riká=m} \textbf{\textit{it͡ʃá     onoká} }\textit{t͡ʃabé}
\end{itemize}

  corn  everything  Q=2\textsc{pl.nom}  like=\textsc{dem}  \textbf{sow.\textsc{prs}} \textbf{do\textsc{.pl}}  before

  ‘Corn, everything, how did you use to plant the crops before’

‘El maiz, todo, como sembraban antes’

  [SFH 06 in61(15)/in]

\begin{itemize}
\item \textit{a’rí   ét͡ʃi   t͡ʃú   yéni   t͡ʃémi} \textbf{\textit{kayéni       onoká} }\textit{ét͡ʃi   sunú}    
\end{itemize}

  and  \textsc{dem}  Q  much  t͡ʃémi  \textbf{yield.harvest.\textsc{prs}} \textbf{do\textsc{.pl}}  \textsc{dem}  corn  

\textit{it͡ʃi-súa     ét͡ʃi   biré   bamíbiri   t͡ʃabé  (inaud.) ...}

  sow-\textsc{cond.pass}  \textsc{dem}  one  year    before

  ‘And how much corn would the harvest yield in a year when you used to sow, before...?’

‘Y cuánto se les daba de maiz cuando sembraban en un año, antes ...?’

  [SFH 06 in61(55)/in]

\begin{itemize}
\item \textit{pe   ripópa   bí=timi   á} \textbf{\textit{kawi-ká     [na’á     onoká]}}   
\end{itemize}

  just  back  just=2\textsc{pl.nom  aff}  \textbf{bring.wood-sim  make.fire  do.\textsc{pl.prs}}  

  \textit{ba}

  \textsc{cl}

  ‘Just with your backs did you used to carry wood, that’s how you used to make fire?’

‘Nomás en el lomo ustedes traían leña, así hacían lumbre?’

  [SFH 06 in61(233)/in]

There seems to be no restrictions as to the inflectional marking the light verb might host. In most cases the head predicate is inflected for present tense (4-13), but in the following examples, the head predicate is marked with an epistemic modality marker (14) and a future singular tense marker (15).

\begin{itemize}
\item \textit{a’rí=timi  t͡ʃú=timi   iyéni  ko=timi?     píri   úa=timi}   
\end{itemize}

and=2\textsc{pl.nom}  Q=2\textsc{pl.nom}  go\textsc{.pl}  \textsc{emph}=2\textsc{pl.nom}  Q  with=2\textsc{pl.nom}    

\textbf{\textit{it͡ʃá-o     onoká-i}} \textit{t͡ʃabé   ko?} 

\textbf{sow-\textsc{ep}} \textbf{do-\textsc{impf}}  before  \textsc{emph}  

  ‘And how did you go? With what did you use to sow before?’

‘Y ustedes cómo iban? Con qué sembraban antes?’

  [SFH 06 in61(42)/in]

\begin{itemize}
\item \textit{pe=m} \textbf{\textit{naku-méa     onok-ám}}\footnote{The form \textit{onokám} may be analyzed as containing a nominalizing participial marker \textit{–am(e)} or it can be analyzed as containing a demonstrative clitic \textit{mi}.} \textit{ba   t͡ʃiriká}   
\end{itemize}

/pe=mi    naku-méa    onoká-ame    ba  ét͡ʃi  riká/

just=\textsc{dem}  \textbf{bother-\textsc{fut.sg}} \textbf{do\textsc{.pl-ptcp}}    \textsc{cl}  \textsc{dem} like  

  \textit{not͡ʃá-bo  ra   káat͡ʃi    ba}

/not͡ʃá-bo  orá  káat͡ʃe    ba/    

work-\textsc{fut.pl}  \textsc{cer}  sometimes  \textsc{cl}

  ‘They bother you when we are going to work like that’

‘Molestan cuando vamos a trabajar asi’

  [SFH 06 in61(540)/in]

In the examples considered so far the head predicate and the light verb are adjacent, with the light verb ordered last. In the following example (16), the verb \textit{noká} appears with a participial marker, and the head predicate precedes \textit{noká} but is not immediately adjacent to it. 

\begin{itemize}
\item \textit{pe} \textbf{\textit{tá-ka}} \textit{bél=ti} \textbf{\textit{noká-ami}} \textit{á   mí   mikám}
\end{itemize}

/pe  tá-ka    béla=ti      noká-ame  á  mí  mikámi  

just  \textbf{request-sim}  really=1\textsc{pl.nom}  \textbf{do-\textsc{ptcp}}  \textsc{aff}  \textsc{dist}  far  

ba

/ba/

\textsc{cl}  

  ‘We requested it over there, far away’

‘Pedimos de por allá lejos’

  [FLP 06 in61(187)/in]

  Another example of a light verb with a nominalizing marker (participial) is shown in (17). In this case, the participial attaches to a medio-passive stem.

\begin{itemize}
\item \textit{pe   owá-ami   narína} \textbf{\textit{at͡ʃé-a   noki-wá-ami} } \textit{hípi   ko}    
\end{itemize}

/pe  owá-ame  narína    at͡ʃé-a    noki-wá-ame    hípi  ko/

just  cure-\textsc{ptcp}  then    \textbf{put-\textsc{prs}} \textbf{do-\textsc{mpass}}\textbf{{}-}\textbf{\textsc{ptcp}}  hoy  \textsc{emph}  

  \textit{ba}

 /ba/

\textsc{cl}

‘Nowadays they just put in the medicine (baking soda) (to the corn beer)’

‘Ahora le echan medicina (levadura) (al tesgüino)’

  [SFH 06 in61(384)/in]

Finally, the semantically bleached \textit{noká} can also be found in constructions where there is no other verbal predicate. In (18-19) the semantically full predicate is non-verbal (\textit{t͡ʃá} ‘ugly’). In (20), \textit{noká} is the only predicate in the clause, marked with a participial suffix.

\begin{itemize}
\item \textit{ét͡ʃi   mi   shipá-am-ti       we} \textbf{\textit{t͡ʃa   noko-ká}}
\end{itemize}

\textsc{/}ét͡ʃi  mi  sipá-ame-ti      we  t͡ʃa  noko-ka/  

\textsc{dem  dem} use.peyote-\textsc{ptcp-nmlz  int} \textbf{ugly  do-sim}

\textit{sipá-ame    ru-wá     riké   t͡ʃabéi   m=aní-a  birá   aní} 

/sipa-ame    ru-wá    riké  t͡ʃabé  mi=aní-a  belá  aní/

use.peyote-\textsc{ptcp} say-\textsc{mpass} riké before  \textsc{dem=}say-\textsc{prs} really  say.\textsc{prs}

‘That the peyote shamans would do bad things before, they would say’

‘Que los raspadores hacían cosas malas raspaban antes, decían’

  [SFH 06 in61(580)/in]

\begin{itemize}
\item \textit{we} \textbf{\textit{t͡ʃa   noko-ká}} \textit{shipá-ami    rihó-ri     t͡ʃabée ko}  
\end{itemize}

/we  t͡ʃa  noko-ká  sipá-ame    rihó-ri    t͡ʃabé  ko/

\textsc{int} \textbf{ugly  do\textsc{{}-sim}}  use.peyote-\textsc{ptcp} man-\textsc{vblz}  before  \textsc{emph} 

\textit{m=aní-a   birá   aní} 

\textsc{/}mi=aní-a  belá  aní/

\textsc{dem}=say-\textsc{prs}  really  say

‘They would do many bad things the peyote shamans before’

‘Hacían muchas cosas malas antes para raspar’

  [SFH 06 in61(601)/in]

\begin{itemize}
\item \textit{ét͡ʃi} \textbf{\textit{noká}} \textit{...  ét͡ʃi} \textbf{\textit{noká-ami}} \textit{rihói   harré}  
\end{itemize}

/ét͡ʃi  noká    ét͡ʃi  noká-ame  rihói  haré/

\textsc{dem} \textbf{do}    \textsc{dem} \textbf{do-\textsc{ptcp}}  man  some

‘That’s how some people are, they do that’

‘Así son unos, hacen eso’

  [FLP 06 in61(583)/in]

\subsection{The \textit{ní-} }\textbf{‘do’ construction}
\label{subsec: the ni- 'do' construction}

Another multipredicate construction in Choguita Rarámuri involves the bound verbal root \textit{ní-}, which is found as the single copular predicate in some constructions (e.g. (21-22), cross-reference to chapter on Basic Clause Types):

\begin{itemize}
\item \textit{ne     umúa-ra} \textbf{\textit{ní-ri}} \textit{ét͡ʃi   ba?}  
\end{itemize}

  1\textsc{sg.nom}  great.grandfather-\textsc{poss}  \textsc{cop-pst}  \textsc{dem}  \textsc{cl}

  ‘Was he my great grandfather?’

‘Era mi bisabuelo él?’

  [SFH 06 in61(117)/in]

\begin{itemize}
\item \textit{“tamuhé   bilá   lína} \textbf{\textit{ní-ma} } \textit{riké     pa   ét͡ʃi   lína} 
\end{itemize}

/tamuhé  belá  alína  ní-ma    riké    pa  ét͡ʃi  alína/

1\textsc{pl.acc}   really   lina   \textbf{\textsc{cop-fut}}   perhaps   \textsc{cl   dem} lina   

\textit{tó-ri     ba”   he   birá   aní-ri     ba}

/tó-ri    ba  he  belá  aní-li    ba/

take-\textsc{pst   cl} it   really   say-\textsc{pst   cl}

  “It was going to be us, but it took him”, that’s what she said’

  ‘“Ibamos a ser nosotros y fue él” así dijo’

  [LEL 06 tx5(67)/tx]

  The following examples show the verb \textit{ní-} in multipredicate constructions (23-26). Like in the constructions with \textit{noká}, these constructions involve a main predicate followed by \textit{ní-}. These constructions are not as frequently attested as the ones with \textit{noká}, and there seem to be more restrictions as to the inflectional marking possible in the light verb. Specifically, in all of the examples shown below the verb \textit{ní-} is marked with past tense. The head predicate is marked for present tense (23-26).

\begin{itemize}
\item \textit{kíti   ni   ke   osáa     bahuré-ma   pári,   ni} \textbf{\textit{lá-a}}   
\end{itemize}

/kíti  ne  ke  osá    bahuré-ma  pári  ne    lá-a/

so.that  1\textsc{sg.nom}  \textsc{neg}  twice    invite-\textsc{fut.sg}  priest  1\textsc{sg.nom}  \textbf{think-\textsc{prs}}

\textbf{\textit{ní-ri}}

/ní-li/

\textbf{do\textsc{{}-pst}}

  ‘So that I won’t have to invite the priest twice, I was thinking’

‘Para no invitar dos veces al padre, pienso’

  [SFH 06 in61(693)/in]

\begin{itemize}
\item \textit{ét͡ʃi ...  et͡ʃi   hápi   ke   mi} \textbf{\textit{karé-a     ní-ri}}
\end{itemize}

  \textsc{dem   dem   rel}\MakeUppercase{} \textsc{neg   dem} \textbf{like-\textsc{prs} }\textbf{do\textsc{{}-pst}}\MakeUppercase{} 

  ‘That, the one he doesn’t like’

‘ese (el) que no le cae bien’

  [LEL 06 tx5(79)/tx]

\begin{itemize}
\item \textit{na   kaéni-ra   ru-ái     mi   tú   ne     pe   ám} \textbf{ }
\end{itemize}

then  finish-\textsc{rep} say-\textsc{mpass}   \textsc{dist} down   1\textsc{sg.nom} just   \textsc{aff}  

  \textbf{\textit{nát-a     ní-ri}}

\textbf{think-\textsc{prs}} \textbf{do\textsc{{}-pst}}

  ‘It was finished down there, I can barely remember’

‘Que se hizo allá abajo, yo apenas me acuerdo’

  [SFH 06 tx12(22)/tx]

\begin{itemize}
\item \textit{pe   bi   ke   me} \textbf{\textit{rishí     ní-ri}} \textit{t͡ʃo   aré ...}    
\end{itemize}

  just  just  \textsc{neg}  almost  \textbf{be.tired.\textsc{prs}} \textbf{do-\textsc{pst}}  also  \textsc{dub}    

‘But one would not get tired’

‘Nomás que casi no se cansaba uno’

  [SFH 06 in61(235)/in]


The main predicates in these constructions involve stative or mental attitude predicates (‘think’, ‘like’, ‘endure’ (23-27)), and they stand in contrast with the constructions with \textit{noká} where the head predicates are often activity predicates.

Finally, consider the following examples with \textit{ní-}, where the head predicates are not marked for present tense, but instead bear markers found in subordinate constructions, namely the temporal suffix \textit{–t͡ʃi} (27) (added to a medio-passive stem) and the conditional suffix \textit{–sa} (28-29).\footnote{Examples (28) and (29), in addition, involve declarative sentences where a Wh word is used as an ‘ignorative’ pronoun (\citealt{evans2008word}). The properties of this sentence type requires further investigation.} 


\begin{itemize}
\item \textit{a’rí   ét͡ʃi   eskuéla   berá   nám} \textbf{\textit{niwa-ríwa-t͡ʃi   ní-ri}}
\end{itemize}

  and   \textsc{dem} school     really   \textsc{dem}   \textbf{make-\textsc{mpass-temp} }\textbf{do\textsc{{}-pst}}   

  ‘And that was the time when the school was being made’

  ‘Y era ese tiempo en que estaban haciendo la escuela’

  [SFH 06 tx12(26)/tx]

\begin{itemize}
\item \textit{t͡ʃu   riká} \textbf{\textit{anáti-sa   ní-ri}} \textit{abói     lína   ú-ma}
\end{itemize}

  Q  like  \textbf{endure-\textsc{cond} }\textbf{do \textsc{-pst}} \textsc{refl.pl} lina run.\textsc{pl}{}-\textsc{fut.sg} 

  ‘The way they themselves think they can endure’

  ‘Como se quieran aguantar ellas mismas’

  [LEL 06 tx19(24)/tx]

\begin{itemize}
\item \textit{píri   ko} \textbf{\textit{t͡ʃiliwé-sa      ní-ra}} \textit{ápu   ná ...   ápu   mé-ri}   
\end{itemize}

  what  \textsc{emph} \textbf{give.gift-\textsc{cond}} \textbf{do\textsc{{}-pot}}  \textsc{rel  dem  rel}  win-\textsc{pst}

 \textit{ko   ba   ét͡ʃi   riká   birá} 

  \textsc{emph  cl  dem} that  really

  ‘Whatever they, the ones that won, want to give her’

  ‘Lo que quisieran regalarle las que ganaron’ 

  [LEL 06 tx19(63)/tx]

\subsection{The \textit{ishí} }\textbf{‘do’ construction}
\label{subsec: the ishi 'do' construction}

A third multipredicate construction involves the verb \textit{ishí} ‘do’. This verb may appear as the single predicate in some clauses, but in the following examples it is found next to another verb, which is semantically the head of the construction. The verb \textit{ishí} bears more inflectional contrasts than the head predicate.

\begin{itemize}
\item \textit{pe      birá      pe   kirí=m} \textbf{\textit{t͡ʃón-a   ishí-i} } \textit{ba}
\end{itemize}

/pe  belá  pe  kirí=mi  t͡ʃóni-a    isí-i    ba/  

little   really  little  slowly=\textsc{dem} \textbf{smash-\textsc{prs}} \textbf{do-\textsc{impf}} \textsc{cl}

  ‘I would smash it slowly’  

  ‘Le aplastaba despacito’

  [BFL 06 tx1(11)/tx]

\begin{itemize}
\item \textit{pe   arí}   birá  \textit{pe   kurím} \textbf{\textit{kayén-a   ishí-a}} \textit{ru-wá-i}    
\end{itemize}

  just  later  really  just  recently  \textbf{finish-\textsc{prs}} \textbf{do-\textsc{prs}} say\textsc{{}-mpass-impf}

\textit{eskuélat͡ʃi   arí   ba}

  school    then  \textsc{cl}

‘That time it was said that they were just recently finishing the school’

‘Esa vez dicen que apenas estaba terminando la escuela’

  [SFH 06 tx12(37)/tx]

\begin{itemize}
\item \textit{a’rí   birá   kó  á  riká   na   botéya   moko’óka   birá   ko   ripámi}   
\end{itemize}

/a’rí  berá  ko  á  riká  na  botéya  moko’óka  berá  ko  ripámi/  

and  really  \textsc{emph  aff} like  \textsc{dem} bottle  in.head.crown  really  \textsc{emph} up

\textit{t͡ʃóm} \textbf{\textit{i  t͡ʃukú  ba   ishí-ri}} \textit{aré   na   bentáan-t͡ʃi   ba}

/t͡ʃómi    /  

  also \textbf{peek.\textsc{prs}} \textbf{do-\textsc{pst}} \textsc{dub  dem} window-\textsc{loc  cl}

‘Like that, he was peeking through the window with the bottle in his head (like a crown)’

‘Así como ya se asomaba por la ventana con la botella puesta’

 [SFH 07 tx152(23)/tx]

In the following example, \textit{ishí} and the causative verb /\textit{mahá-ri}/ are not adjacent and both marked for potential mood. It is possible that this is a biclausal structure.

\begin{itemize}
\item \textit{a’rí   birá  ne     birá   na} \textbf{\textit{maháa-ra}} \textit{t͡ʃo} \textbf{\textit{ishí-ra}}   
\end{itemize}

/a’rí  belá  ne    belá  na  mahá-ra  t͡ʃo  isí-ra/

and  really  1\textsc{sg.nom}  really  \textsc{dem} \textbf{fear.\textsc{caus}}\textbf{{}-}\textbf{\textsc{pot}}also  \textbf{do-\textsc{pot}} 

  \textit{ne    pe   ipó   má   t͡ʃukú-r-ami   t͡ʃó   ko}

  /ne    pe  ipó  má  t͡ʃukú-r-ame  t͡ʃó  ko/

1\textsc{sg.nom} just  valley  run  be.bent-r-\textsc{ptcp} also  \textsc{emph}

  ‘And also I made him afraid because I was running in the valley’

‘Y es que yo también lo asusté porque iba en el llano corriendo’

  [SFH 06 tx12(36)/tx]

Finally, in the following example, \textit{ishí} is preceded by the non-verbal predicate \textit{t͡ʃá} ‘ugly’ and it is nominalized through an agentive participial marker.

\begin{itemize}
\item \textbf{\textit{t͡ʃá    ishí-k-am}} \textit{yéna=m   korimá   ba}
\end{itemize}

/t͡ʃá  isí-k-ame  ayéna=mi  korimá    ba/

\textbf{ugly   do-k-\textsc{ptcp} }go.\textsc{sg=dem} korima   \textsc{cl}

  “The \textit{korima} was bothering (us)”

“Anduvo molestando el \textit{korimáka}”

  [LEL 06 tx5(53)/tx]

\subsection{The \textit{orá} }\textbf{‘make’ construction}
\label{subsec: the ora 'make' construction}

So far I have been analyzing the form \textit{olá} as a modal particle often occurring with verb forms inflected for future tense which indicates more certainty from the speaker as to the actual realization of the event. This form contrasts with \textit{aré}, a form I have analyzed as a dubitative marker, where the speaker indicates doubt or lack of certainty about the realization of an event in the future. The contrast between future constructions with \textit{olá} and future constructions with \textit{aré} is appreciated in the following examples:

\begin{itemize}
\item \textit{nár-ma   ré}
\end{itemize}

/nári-ma        aré/        

ask-\textsc{fut.sg   dub}

  ‘(He) will probably ask’

  ‘Probablemente va a preguntar’      

[BFL 05 1:152/el]

\begin{itemize}
\item \textit{nár-mo   l}\textit{á}
\end{itemize}

/nári-ma  olá/         

ask-\textsc{fut.sg  cer}      

‘(he) will definetly ask’

‘Seguramente que va a preguntar’      

[BFL 05 1:152/el]

The form \textit{olá}, however, is also found as the single verbal predicate in a clause with the meaning ‘do’ or ‘make’ bearing verbal inflection, as shown in (37-39):\footnote{The dubitative form \textit{aré} does not appear bearing verbal inflection and as the single predicate in any clause.}

\begin{itemize}
\item \textit{“ne   gará=m   okará=mi     iwéri-ri=m} \textbf{\textit{orá-ma}} 
\end{itemize}

/ne  kará=m  okará=mi    iwéri-ri=mi      olá-ma/ 

  \textsc{int}  well    okará=2\textsc{sg.nom}  make.effort-\textsc{nmlz}=\textsc{dem}  \textbf{make-\textsc{fut.sg} }

\textit{aparím}   \textit{ra’ít͡ʃi-ri   a-ría=n    ani-ría} 

  /nápi  a’rí  ra’ít͡ʃa-ri  a-ría=ni    ani-ría/  

\textsc{comp} when  speak-\textsc{nmlz}   give-\textsc{mpass}=1\textsc{sg.nom}  say-\textsc{Mpass}

  ‘”Make an effort when they they talk to you or give you advice” I would be told’

‘“Hazle la lucha cuando te hablen o te den plática”, me decían’

  [FLP 06 in61(480)/in]

\begin{itemize}
\item \textit{“t͡ʃu   birá} \textbf{\textit{orá-bo}} \textit{ré   pa,   ká=ti     riwá     ba   ní}”
\end{itemize}

  Q  really  make-\textsc{fut.pl}  \textsc{dub  cl}  \textsc{neg-1pl.nom} find.\textsc{prs   cl} ni

  ‘“What shall we do? We can’t find them”’

‘“Que le vamos a hacer? no los encontramos”’

  [LEL 06 tx32(77)/tx]

\begin{itemize}
\item \textit{we   risó} \textbf{\textit{oríi-ra}}\footnote{In text-based elation it was discussed that the future form in the active voice of \textit{risó} \textbf{\textit{oríi-ra}} would be \textit{risó} \textbf{\textit{orá-bo}} \textit{ré}.} \textit{ru-á,     ma   we   okó-am} 
\end{itemize}

/we  risó  orí-ra      ru-wá    ma  we  okó-ame/  

  \textsc{int}       harm   \textbf{make.\textsc{mpass-rep}}   say-\textsc{mpass}  already  \textsc{int} pain-\textsc{ptcp}

  \textit{ku   má-ra     ré}

 /ku  má-ra    aré/ 

  \textsc{rev} run-\textsc{pot  dub}

  ‘Que le hicieron mucho daño, ya se fue mucho con mucho dolor, yo creo’

  [LEL 06 tx32(145)/tx]

  In the following example, \textit{orá} appears both inflected (for past tense) and together with a verb inflected for future tense. 

\begin{itemize}
\item \textit{a’lí   na      ét͡ʃo   ná       ma …     ét͡ʃi   ná     mo’ot͡ʃíki    t͡ʃukúri-ri}  
\end{itemize}

and   then   \textsc{dem} there   already   \textsc{dem} there   cabecera    go.around-\textsc{pst}

 \textit{t͡ʃapi-nára   ét͡ʃi  rihói   ariwá-ra} \textbf{\textit{to-mé     orá-ri}}

  grab-\textsc{desid}  \textsc{dem} man   soul-\textsc{poss} \textbf{take-\textsc{fut.sg} }\textbf{make-\textsc{pst}}

‘and then there it was going around near the head (\textit{cabecera}) wanting to grab it, to take the man’s soul’

  ‘y entonces ahí…andaba por la cabecera queriéndolo agarrar, llevar el alma del señor’

  [LEL 06 tx5(14)/tx]

  This form then suggests that an alternative analysis exists for future constructions with the form \textit{orá} as a light verb or auxiliary construction that has specific pragmatic functions. In (41), \textit{orá} is glossed as the bare stem of the predicate ‘make’ marked for present tense.

\begin{itemize}
\item \textit{a’rí   birá=ni     á   t͡ʃi  rá   ní-ri   hípi   nihé    á}     
\end{itemize}

and  really=1\textsc{sg.nom  aff} that  think  do-\textsc{pst} now  1\textsc{sg.nom  aff} 

\textit{birá=ni     mat͡ʃí} \textbf{\textit{pá-ma     orá}} 

really=1\textsc{sg.nom} outside    \textbf{throw-\textsc{fut.sg} }\textbf{make.\textsc{prs}}

‘And now that’s what I think, I will fulfill my commitment (lit. take it out)’

‘Y ahora pienso eso yo, sí lo voy a sacar’

[SFH 06 in61(in)/in]

\section{Serial verb constructions}
\label{sec: serial verb constructions}

Serial verb constructions can be characterized by the following properties (\citealt{foley1985clausehood}, \citealt{sebba1987syntax}, \citealt{aikhenvald2006serial}):

\begin{itemize}
\item There is only one syntactic subject, as these constructions are monoclausal.
\item Neither verb will be gramatically dependent on the other.
\item The construction contains at least two verbs without any overt marker of subordination or coordination.
\item The events depicted by the verbs are simultaneous or consecutive.
\item Negation has scope over the whole string.
\item Serial verb constructions and multi-clausal coordinated structures contrast in meaning. In the first type, the sequence of verbs are associated, and the second verb depicts an event that is a result or goal of the event depicted by the first verb. 
\item Intonationally, serial verb constructions may behave like a single clause. 
\item Adverbial operators must modify both verbs simultaneously.  
\item If clefting is possible, the fronted argument will move to the front of the entire verbal string; clefting will not be possible if the construction involves two separate clauses.
\item Some common verbs appearing in serial verb constructions include verbs of motion, posture, process and statives.  
\end{itemize}

In the following example (42), there are two verbs that do not show any grammatical dependency to each other, they share the same subject argument and share inflectional marking.

\begin{itemize}
\item \textit{a’rí   birá   ko   étʃ͡i   tʃ͡abótʃ͡i     naríni  ne} \textbf{\textit{ra'í-ra     bahí-ra}}   
\end{itemize}

/a’rí  belá  ko  ét͡ʃi  t͡ʃabót͡ʃi  narína  ne  \textbf{ra’í-ra    bahí-ra/}

    and  really  \textsc{emph  det}  mestizo  but  \textsc{int}  \textbf{like-\textsc{rep}} \textbf{drink-\textsc{rep}} 

    \textit{r-á=m       pa   étʃ͡i}

 /ru-wá=mi    pa  ét͡ʃi/

 say-\textsc{mpass=dem  cl  dem}

    ‘But the \textit{mestizo} did drink it happily’

    ‘Pero el mestizo sí se lo tomó muy a gusto’

    [SFH 06 choma (8)/tx]

Some potential candidates for serial verb constructions include the following examples. In these cases the two verbs do not share the same tense or aspectual markers, but do seem to share the same syntactic arguments. Every example involves a motion verb.

\begin{itemize}
\item \textit{a’lí   na      ét͡ʃo   ná       ma …     ét͡ʃi   ná     mo’ot͡ʃíki}      
\end{itemize}

and   then   \textsc{dem} there   already   \textsc{dem} there   \textit{cabecera}   

 \textbf{\textit{t͡ʃukúri-ri     t͡ʃapi-nára}} \textit{ét͡ʃi  rihói   ariwá-ra   to-mé} 

\textbf{go.around-\textsc{pst}} \textbf{grab-\textsc{desid}}  \textsc{dem} man   soul-\textsc{poss} take-\textsc{fut.sg} 

\textit{orá-ri}

make-\textsc{pst}

‘and then there it was going around near the head (\textit{cabecera}) wanting to grab it, to take the man’s soul’

  ‘y entonces ahí…andaba por la cabecera queriéndolo agarrar, llevar el alma del señor’

  [LEL 06 tx5(14)/tx]

\begin{itemize}
\item \textit{ét͡ʃi   ko   we} \textbf{\textit{nará-shi-a   nawá-ri}} 
\end{itemize}

  \textsc{dem   emph   int}   \textbf{cry-\textsc{mot-prs} }\textbf{arrive-\textsc{pst}}

  ‘She arrived crying’

  ‘Llegó llorando’

  [LEL 06 tx5(45)/tx]

\begin{itemize}
\item \textit{he   ané} \textbf{\textit{aní-shi-a         nawá-ri}} \textit{ét͡ʃi    namú nirá} \textit{shuwá         ba}  
\end{itemize}

it    say    \textbf{say-\textsc{mot-prs} }\textbf{arrive-\textsc{pst}} \textsc{dem} relatives        everybody   \textsc{cl} 

\textit{á       ruwé-ri} 

\textsc{aff} say-\textsc{pst}    

  ‘A relative arrived saying, telling everybody’

  ‘Llegó diciendo un familiar, diciéndoles a todos’ 

[LEL 06 tx5(61)/tx]

\begin{itemize}
\item \textit{naparí   ma     biré   tó-ru     ne} \textbf{\textit{pat͡ʃú-shi-a}} \textbf{\textit{inár-to}} 
\end{itemize}

  when   already   one   take-\textsc{pst.pass  int} \textbf{drip-\textsc{mot-prs}}    \textbf{go-\textsc{mov}}

  \textit{ripá-ti     rabó}

up-through   hill  

  ‘cuando ya lleva uno va goteando algo por arriba del cerro’ 

  ‘when one was already taken it goes driping something by the top of the hill’

[LEL 06 tx5(99)/tx]

\begin{itemize}
\item \textit{siné   kát͡ʃi   ke   me} \textbf{\textit{sébi-ri     anát͡ʃi}} \textit{nayó}
\end{itemize}

  some   times   \textsc{neg} almost  \textbf{reach-\textsc{pst} }\textbf{endure.\textsc{prs}}   four

  ‘Some times they don’t endure four (lapses)’

  ‘En veces no aguantan las cuatro (vueltas)’

  [LEL 06 tx19(30)/tx]

\begin{itemize}
\item a’rí   ét͡ʃi   nápu   \textbf{rowé-a     úum}     ko   á   birá   ritiwá   
\end{itemize}

  and   \textsc{dem   rel}   \textbf{women.race-\textsc{prs} }\textbf{run\textsc{.pl.prs}} \textsc{emph   aff} really   see.\textsc{prs}

 t͡ʃú t͡ʃurú   atí

how.much   sit\textsc{.sg}

  ‘And then the ones running see how many things there are (lit. sit) (things people bet)’ 

‘Y entonces las que andan corriendo ven que tanto va’

  [LEL 06 tx19(36)/tx]

\begin{itemize}
\item pe   birá   [\textbf{risó     noko-ká]   éni-ri}       tʃ͡ó   
\end{itemize}

    just  really  \textbf{struggle  do-sim  go.around.\textsc{pl-pst}}  also  

    étʃ͡i   rihó   á   batʃ͡á   bahí-sa   ka   étʃ͡i   tʃ͡o’má   ba

    \textsc{dem}  man  \textsc{aff}  first  drink-\textsc{cond  emph  dem}  snot  \textsc{cl}

    ‘They would go around struggling too if the man would have drank the snot first’

    ‘Anduvieran también batallando si el hombre hubiera tomado primero el moco’

    [SFH 06 choma(32)/tx]

Possible examples:

a’rí   ét͡ʃi   t͡ʃabót͡ʃi   ko   birá     we   birá     \textbf{ra’í-ra}   

  y  \textsc{dem} mestizo  \textsc{emph} de.veras  \textsc{int} de.veras  \textbf{gustando-\textsc{rep}}

\textbf{bahí-ra}   ru-wá     ét͡ʃi   t͡ʃomá   biré   bitóri  ba    

\textbf{beber-\textsc{rep}} decir-\textsc{mpass  dem} moco  uno  plato  \textsc{cl}

‘Y ese mestizo dicen que se tomó muy a gusto el moco del cajete’

[SFH 07 tx128(14)/tx]

%Some of the properties I plan to investigate with respect to these constructions through guided elation are:
%What are possible combinations of predicates in these kind of constructions in terms of transitivity properties?
%Are the verbs always contiguous?
%Does negation have scope over the whole verbal string?

\section{Positionals in complex predicate constructions}
\label{sec: positionals in complex predicate constructions}

Positional predicates are often involved in multipredicate constructions. These constructions may constitute a sub-type of serial verb constructions.

\begin{itemize}
\item \textit{a’rí   na …   a’rí   na      [}\textbf{\textit{kot͡ʃi-ká    bu’í-r-o]}} \textit{mayé-ri}
\end{itemize}

  and   then   and   then   \textbf{sleep-sim }\textbf{lay.down.\textsc{sg}}\textbf{{}-}\textbf{\textsc{pst-ep}} think-\textsc{pst}

  ‘And then he thought he was asleep (laid down sleeping)’

  ‘Nomás que pensó que estaba dormido’

  [LEL 06 tx5(4)/tx]

\begin{itemize}
\item a’rí   t͡ʃi   ko        t͡ʃiná  hónsa  ko       a’rí    birén      t͡ʃo   bitit͡ʃí   
\end{itemize}

  and   t͡ʃi   \textsc{emph} there    from  \textsc{emph} and    another    also   house   

shimí-ri  napu   ko   ná   \textbf{bahí-a     mot͡ʃí-ri} 

go-\textsc{pst    rel}   \textsc{emph} there   \textbf{drink-\textsc{prs} }\textbf{sit.down.\textsc{pl}}\textbf{{}-}\textbf{\textsc{pst}}

  ‘And then from there he went to another house where there was drinking going on’  

  ‘Y entonces de ahí se fue en otra casa donde estaban tomando’

  [LEL 06 tx5(22)/tx]

\begin{itemize}
\item wa’rú   na   \textbf{lé-a       wiri-ká} ru-wá     ba’arínila   bi’á
\end{itemize}

  big        there   \textbf{have.blood-\textsc{prs}} \textbf{stand.\textsc{sg}}\textbf{{}-sim} say-\textsc{mpass}  next.day  early

‘It was very \textit{sangrado} (bled) the next day early’

‘Estaba muy sangrado al otro día en la mañana’  

[LEL 06 tx5(29)/tx]

\begin{itemize}
\item a’rí   ma ...  a’rí   ma   \textbf{muku-ká     ashíshi-ri}   ba’arína  bi’á
\end{itemize}

  and   then  later   then   \textbf{die-sim }\textbf{get.up-\textsc{pst}} next.day   early

  ‘And then he was dead (lit. got up dead) early the next day’

  ‘Y entonces ya se levantó (amaneció) muerto al otro día’

  [LEL 06 tx5(35)/tx]

\begin{itemize}
\item mábi     \textbf{nataké-a   bu’í-ri}   ét͡ʃi   rihói
\end{itemize}

  already    \textbf{faint-\textsc{prs}} \textbf{lay.down.\textsc{sg}}\textbf{{}-}\textbf{\textsc{pst}} \textsc{dem} man

  ‘He was already lying down unconscious that man’

  ‘Ya estaba desmayado ese señor’

  [LEL 06 tx5(37)/tx]

\begin{itemize}
\item a’ri   ma    né-li     pe   ma     \textbf{muku-ká   bu’í-ri} 
\end{itemize}

  and   already    see-\textsc{pst} little   already   \textbf{die-sim   lie.down-\textsc{pst}}

  ‘Then they saw that he was already lying down dead’

  ‘Entonces ya vieron y ya estaba muerto’

  [LEL 06 tx5(40)/tx]  

\begin{itemize}
\item a’rí   lá   ko   mat͡ʃí     ritéi-ri     t͡ʃuku-ká      ru-wá   
\end{itemize}

and   blood   \textsc{emph} outside    stone-\textsc{loc} sit\textsc{.sg}{}-sim say-\textsc{mpass} 

 ne    \textbf{shitá-na  wiri-ká}   ru-wá     ét͡ʃo   na

\textsc{int} \textbf{be.red-\textsc{tr}} \textbf{stand\textsc{.sg}}\textbf{{}-sim}   say-\textsc{mpass   dem} there

  ‘And then the blood was outside on a stone, they say it was red there’

  ‘Y entonces la sangre estaba afuera en una piedra, estaba colorada’

  [LEL 06 tx5(41)/tx]

\begin{itemize}
\item a’rí   ma     shiné-ami   we   \textbf{mahá-ka   mot͡ʃí-ri}   ét͡ʃo   ná 
\end{itemize}

  and   already   all-\textsc{ptcp   int} \textbf{fear-sim }\textbf{sit.\textsc{pl}}\textbf{{}-}\textbf{\textsc{pst}} \textsc{dem} there

   nápu   ru’u-ríwi   ko

\textsc{rel}   say-\textsc{mpass   emph}

‘And then averybody was scared, everybody who had been told’  

‘Y entonces ya todos estaban asustados a todos los que les dijeron’

 [LEL 06 tx5(62)/tx]  

\begin{itemize}
\item a’rí   na …  we   ya   t͡ʃakéna     \textbf{rawé-a    háu-po}        ru-wá
\end{itemize}

  and   then   \textsc{int} fast   another.side  \textbf{turn-\textsc{prs} }\textbf{stand\textsc{.pl-fut.pl}} say-\textsc{mpass}  

‘And then we have to stand turning to anther side, it is said’

‘Y entonces pronto hay que voltearse de otro lado’  

[LEL 06 tx5(74)/tx]  

\begin{itemize}
\item a’rí   birá … ne     birá   na   maháa-ra   t͡ʃo   ishí-ra   ne     
\end{itemize}

  and  really  1\textsc{sg.nom}  really  \textsc{dem} fear.\textsc{caus}{}-\textsc{pot} also  do-\textsc{pot}  1\textsc{sg.nom}

  pe   ipó   \textbf{má   t͡ʃukú-r-ami}     t͡ʃó   ko

  just  valley  \textbf{run  be.bent-r-\textsc{ptcp}} also  \textsc{emph}

  ‘And it’s also true that I frigthened him because I was running in the valley’

‘Y es que yo también lo asusté porque iba en el llano corriendo’

 [SFH 06 tx12(36)/tx]

\begin{itemize}
\item a’rí   ne       ko   ma   bitit͡ʃí   á   \textbf{buyé-a   atí}
\end{itemize}

  and   1\textsc{sg.nom   emph} then   house    \textsc{aff?} \textbf{wait-\textsc{prs} }\textbf{sit.\textsc{sg}}

  ‘And then I wait for her in the house’

  ‘Y entonces yo ya la espero en la casa’

  [LEL 06 tx19(13)/tx]

  In the examples above the positional verb follows the other verb, but in the next couple of examples the positional verb precedes the other verb:

\begin{itemize}
\item a’rí   pe   t͡ʃuwé     \textbf{bu’u-ká     nataké-a}    ru-wá     ba
\end{itemize}

  and  just  like.that      \textbf{lie.down\textsc{.sg-sim} }\textbf{faint-\textsc{prs}} say-\textsc{mpass   cl}  

  ‘And anyhow it is said that he (they?) lying down faints (dies)’

‘Y entonces dicen que acostados se desmayan (mueren)’

[LEL 06 tx5(89)/tx]  

\begin{itemize}
\item naparí   mat͡ʃí     \textbf{biti-súa       kot͡ʃí-a}     
\end{itemize}

when  outside    \textbf{lie.down\textsc{.pl}}\textbf{{}-}\textbf{\textsc{cond.pass} }\textbf{sleep-\textsc{prs}} 

he   aní-ami   hu:    “kíti   ri’ná     biti-shí”

that   say-\textsc{ptcp   cop.prs} don’t   on.back   lie.\textsc{pl}{}-\textsc{imp.pl}    

  ‘When somebody woould sleep outside it was said: “don’t sleep on your back”’

‘Cuando dormimos afuera nos dicen: “no se acuesten boca arriba”

  [LEL 06 tx5(95)/tx]
