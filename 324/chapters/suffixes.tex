\chapter{Verbal suffixes}
\label{chap: verbal suffixes}

\begin{tabbing}
\textit{-ˈmêa, -ma}\hspace{1ex} \= Reportative different subject (\textsc{rep.ds})\hspace{1ex} \= §\ref{subsubsec: applicative ni}\kill
\textit{-bá}  \> Inchoative (\textsc{inch}) \> §\ref{subsec: inchoative}\\
\textit{-na}  \> Transitive (\textsc{tr}) \> §\ref{subsec: transitive -na} \\
\textit{-tʃ͡a}  \> Pluractional transitive (\textsc{tr.plc}) \> §\ref{subsec: pluractional transitive}\\
\textit{-bû} \> Transitive (\textsc{tr}) \> §\ref{subsec: transitive -bû}\\
\textit{-ni}  \> Applicative (\textsc{appl}) \> §\ref{subsubsec: applicative ni}\\
\textit{-si}  \> Applicative (\textsc{appl}) \> §\ref{subsubsec: applicative si}\\
\textit{-wi} \> Applicative (\textsc{appl}) \> §\ref{subsubsec: applicative wi}\\
\textit{-ti}  \> Causative \textsc{caus} \> §\ref{subsec: causative}\\
\textit{-ki}   \> Applicative (\textsc{appl}) \> §\ref{subsec: applicative ki}\\
\textit{-nále} \> Desiderative (\textsc{desid}) \> §\ref{subsec: desiderative}\\
\textit{-simi} \> Associated Motion (\textsc{mot}) \> §\ref{subsec: associated motion}\\
\textit{-tʃ͡ane} \> Auditory Evidential (\textsc{ev}) \> §\ref{subsec: auditory evidential}\\
\textit{-ru} \> Past Passive (\textsc{pst.pass}) \> §\ref{subsubsec: past passive}\\
\textit{-pa} \> Future Passive (\textsc{fut.pass}) \> §\ref{subsubsec: future passive}\\
\textit{-rîwa, -wá} \> Medio-Passive (\textsc{mpass}) \> §\ref{subsubsec: medio-passive}\\
\textit{-sûwa} \> Conditional Passive (\textsc{cond.pass}) \> §\ref{subsubsec: conditional passive}\\
\textit{-ˈmêa, -ma} \> Future Singular  (\textsc{fut.sg}) \> §\ref{subsubsec: future singular}\\
\textit{-pô} \> Future Plural (\textsc{fut.pl}) \> §\ref{subsubsec: future plural}\\
\textit{-mê} \> Motion Imperative (\textsc{mot.imp}) \> §\ref{subsec: motion imperative}\\
\textit{-sâ} \> Conditional (\textsc{cond}) \> §\ref{subsec: conditional}\\
\textit{-mê} \> Irrealis singular (\textsc{irr.sg}) \> §\ref{subsubsec: irrealis singular}\\
\textit{-pi} \> Irrealis plural (\textsc{irr.pl}) \> §\ref{subsubsec: irrealis plural}\\
\textit{-ra} \> Potential (\textsc{pot}) \> §\ref{subsec: potential}\\
\textit{-kâ} \> Imperative singular (\textsc{imp.sg})  \> §\ref{subsubsec: imperative singular ka}\\
\textit{-sâ} \> Imperative singular (\textsc{imp.sg})  \> §\ref{subsubsec: imperative singular sa}\\
\textit{-sì} \> Imperative plural (\textsc{imp.pl}) \> §\ref{subsubsec: imperative plural}\\
\textit{-la} \> Reportative different subject (\textsc{rep.ds}) \> §\ref{subsubsec: reportative different subject}\\
\textit{-lo} \> {Reportative same subject (\textsc{rep.ss})} \> §\ref{subsubsec: reportative same subject}\\
\textit{-li} \> Past (\textsc{pst}) \>  §\ref{subsec: past}\\
\textit{-ki} \> Past perfective egophoric (\textsc{pst.ego}) \> §\ref{subsec: past egophoric}\\
\textit{-e} \> Imperfective (\textsc{impf}) \> §\ref{subsec: imperfective}\\
\textit{-a}  \> Progressive (\textsc{prog}) \> §\ref{subsec: progressive}\\
\textit{-nula} \> Indirect causative  \> §\ref{subsec: indirect causative}\\
\textit{-tʃ͡i} \> Temporal (\textsc{temp}) \> §\ref{subsec: temporal}\\
\textit{-o} \> Epistemic (\textsc{ep}) \> §\ref{subsec: epistemic}\\
\textit{-ká} \> Gerund (\textsc{ger}) \> §\ref{subsec: gerund}\\
\textit{-ra} \> Purposive (\textsc{pur}) \> §\ref{subsec: purposive}\\
\textit{-ame} \> Participial (\textsc{ptcp}) \> §\ref{subsec: participial}\\
\end{tabbing}

\section{The Derived Stem: inchoative and transitivity markers}
\label{sec: the derived stem}

The first identifiable layer in the suffixation domain of morphologically complex verbs is the Derived Stem. This verbal domain includes semantically restricted, unproductive derivational suffixes (an inchoative suffix and three transitive suffixes). These suffixes are restricted to attach to a semantically defined class of verbs, change-of-state verbs.

\subsection{Inchoative \textit{-bá}}
\label{subsec: inchoative}

The inchoative suffix is productively used with positional or stative predicates to indicate a dynamic change-of-state; a state is turned into a process, meaning ‘to become X’. This stress-shifting suffix is exemplified in (\ref{ex: inchoative example}b, d).

\ea\label{ex: inchoative example}

    \ea[]{
    \textit{ˈwé aˈʰkâame ˈhú}\\
    \gll    ˈwé aˈʰkâ-ame ˈhú\\
            \textsc{int} be.sweet-\textsc{ptcp} \textsc{cop.prs}\\
    \glt    `It is very sweet.'\\
    \glt    `Está muy dulce.'\\
}

        \ex[]{
        \textit{ˈmá aʰkaˈbátʃ͡anali}\\
        \gll    ˈmá aʰka-\textbf{ˈbá}-tʃ͡a-na-li\\
            already be.sweet-\textsc{\textbf{inch}-tr.plc-tr-pst}\\
        \glt    ‘S/he has already sweetened it.’\\
        \glt    ‘Ya lo endulzó.’ < BFL 05 2:56/el > \\
}
        \ex[]{
         \textit{ˈwé raˈtâame ˈhú}\\
        \gll    ˈwé raˈtâ-ame ˈhú\\
                \textsc{int} be.hot-\textsc{ptcp} \textsc{cop.prs}\\
        \glt    `It is very hot.'\\
        \glt    `Está muy caliente.'\\
    }
        \ex[]{
        \textit{rataˈbámam pa}\\
        \gll    rata-ˈbá-ma=mi pa\\
                be.hot-\textsc{\textbf{inch}-fut.sg=dem} \textsc{cl}\\
        \glt    `(so) it will become hot'\\
        \glt    `(para que) se caliente' \corpuslink{tx68[00_462-00_512].wav}{LEL tx68:0:46.2}\\
}
    \z
\z

%\subsection{Transitives}
%\label{subsec: transitives}

%The transitive suffixes in S2 are semantically and lexically restricted suffixes of limited productivity. Transitive suffixes \textit{-na} (§2.1) and \textit{-tʃá} (§2.2) increase the valence of intransitive change-of-state predicates (described in Chapter 3, §3.3.3).

\subsection{Transitive \textit{-nâ}}
\label{subsec: transitive -na}

The transitive \textit{-nâ} suffix is a stress-shifting suffix that attaches to change-of-state predicates, increasing the valency of the verb stem. Stems derived with this transitive suffix are ``alternating stems'', where surface tonal patterns are conditioned by the stress properties of inflection markers: a HL tone if the inflection morpheme is stress-shifting and a L tone if the inflection morpheme is stress-neutral (see §\ref{subsubsec: grammatical tone distributed by morphological class} for more details).

The following examples show transitive derivations with suffix \textit{-nâ}: the intransitive (inchoative) forms is not suffixed (e.g., (\ref{ex: transitive -naa})), and the corresponding transitive counterpart is marked with the transitive \textit{-nâ} suffix (e.g., (\ref{ex: transitive -nab})).

\ea\label{ex: transitive -na}

    \ea[]{
    \textit{ˈmá tʃ͡͡iˈwáli siˈpútʃ͡a}\\
    \gll    ˈmá tʃ͡iˈwá-li siˈpútʃ͡a\\
            already tear-\textsc{pst} skirt\\
    \glt    ‘The skirt already tore.’\\
    \glt    ‘Ya se rompió la falda.’ < SFH 07 1:17-21/el >\\
}\label{ex: transitive -naa}
        \ex[]{
        \textit{ˈá riˈwè! tʃ͡iwaˈnâra!}\\
        \gll    ˈá riˈwè tʃ͡iwa-\textbf{ˈnâ}-ra\\
                \textsc{aff} leave.\textsc{imp.sg} tear-\textsc{tr-pot}\\
        \glt    ‘Leave it, you are going to tear it!’\\
        \glt    ‘¡Déjalo! ¡Lo vas a trozar!’ < SFH 07 1:17-21/el >\\
    }\label{ex: transitive -nab}
    \z
\z

\subsection{Pluractional transitive \textit{-tʃa }}
\label{subsec: pluractional transitive}

The pluractional transitive \textit{-tʃ͡a} suffix is a stress-neutral transitive suffix that can attach to the same change-of-state predicates that may be marked with the transitive \textit{-nâ} suffix described in Appendix~\ref{subsec: transitive -na}. The  suffix, historically reconstructed to \ili{Proto-Uto-Aztecan} (\citealt{heath1978uto}), may encode (i) an event that is performed repeatedly (the succession or discernible, discrete events) or (ii) that more than one entity is affected by an event. Of limited productivity in Choguita Rarámuri, this stress-shifting suffix is found with a pluractional sense. Compare the transitive form with suffix \textit{-na} in (\ref{ex: pluractional transitive exampleb}) and the pluractional transitive form with suffix \textit{-tʃ͡á} in (\ref{ex: pluractional transitive examplec}) of the same base predicate \textit{kuʔˈrí} `to turn’.

\ea\label{ex: pluractional transitive example}

    \ea[]{
    \textit{niˈhê kuʔˈríma}\\
    \gll    niˈhê kuʔˈrí-ma\\
            1\textsc{sg.nom} turn-\textsc{fut.sg}\\
    \glt    ‘I will turn (on my own axis).’\\
    \glt    ‘Voy a dar vuelta (en mi propio eje).’ < SFH 05 1:140/el >\\
}\label{ex: pluractional transitive examplea}
%\pagebreak
        \ex[]{
        \textit{niˈhê kuʔˈru-nâ-ma}\\
        \gll    niˈhê kuʔˈru-nâ-ma\\
                1\textsc{sg.nom} turn-\textsc{tr-fut.sg}\\
        \glt    ‘I will turn it (on its own axis).’\\
        \glt    ‘Le voy a dar vuelta (en su propio eje).’ < BFL 05 1:187/el >\\
    }\label{ex: pluractional transitive exampleb}
            \ex[]{
            \textit{ˈmáni kuʔˈrítʃ͡ima}\\
            \gll    ˈmá=ni kuʔˈrí-tʃ͡a-ma\\
                    already=\textsc{1sg.nom} turn-\textsc{tr.plc-fut.sg}\\
            \glt    ‘I will now turn it several times (on its own axis).’\\
            \glt    ‘Ya le voy a dar muchas vueltas (en su propio eje).’ < BFL 05 1:187/el >
        }\label{ex: pluractional transitive examplec}
    \z
\z

Non-pluractional uses of transitive \textit{-tʃ͡a} can also be found. An example of this is provided in (\ref{ex: non-pluractional use of chab}):

\ea\label{ex: non-pluractional use of cha}

    \ea[]{
    \textit{rataˈbámam pa}\\
        \gll    rata-ˈbá-ma=mi pa\\
                be.hot-\textsc{{inch}-fut.sg=dem} \textsc{cl}\\
        \glt    `(so) it will become hot'\\
        \glt    `(para que) se caliente' \corpuslink{tx68[00_462-00_512].wav}{LEL tx68:0:46.2}\\
}\label{ex: non-pluractional use of chaa}
        \ex[]{
        \textit{niˈhê rataˈbátʃ͡ama koˈʔámi}\\
        \gll    niˈhê rata-ˈbá-tʃ͡a-ma koˈʔ-ámi\\
                1\textsc{sg.nom} heat-\textsc{inch-tr.pl-fut.sg} eat-\textsc{ptcp}\\
        \glt    ‘I’m going to heat up the food.’\\
        \glt    ‘Voy a calentar la comida.’ < LEL 06 4:151/el >\\
    }\label{ex: non-pluractional use of chab}
    \z
\z

\subsection{Transitive \textit{-bû}}
\label{subsec: transitive -bû}

There is a third transitivizing suffix, \textit{-bû}, which is also unproductive and lexically restricted. This stress-shifting suffix is exemplified in (\ref{ex: transitive -bu exampleb}):\footnote{Lengthening of the transitive suffix vowel in example (\ref{ex: transitive -bu exampleb}) is triggered by the past passive suffix. This effect is discussed below (§\ref{subsubsec: past passive-conditioned lengthening}).}

\ea\label{ex: transitive -bu example}

    \ea[]{
    \textit{toˈwí ˈmá ˈmóli}\\
    \gll    toˈwí ˈmá ˈmó-li\\
            boy already go.up-\textsc{pst}\\
    \glt    ‘The boy already went up.’\\
    \glt    ‘Ya se subió el niño.’ < BFL 06 4:189/el >\\
}\label{ex: transitive -bu examplea}
        \ex[]{
        \textit{ˈmáni moˈbûuro}\\
        \gll    ˈmá=ni mo-ˈbû-ru\\
                already=\textsc{1sg.nom} go.up-\textsc{tr-pst.pass}\\
        \glt    ‘I was already taken up.’\\
        \glt    ‘Ya me subieron.’ < BFL 06 4:189/el >\\
    }\label{ex: transitive -bu exampleb}
    \z
\z

\section{The Syntactic Stem: causative and applicative markers}
\label{sec: the syntactic stem}

After the derived stem domain, the next verbal zone in the Choguita Rarámuri verbal complex is the ``syntactic stem'' which includes suffixes in suffix positions S3 to S5 that encode valence-increasing operations, namely applicative and causative operations.

\subsection{Applicatives}
\label{subsec: applicatives}

\subsubsection{Applicative \textit{-ni}}
\label{subsubsec: applicative ni}

The Applicative \textit{-ni} suffix increases the valency of the verb, adding a benefactive argument (‘to do X for Y’). This suffix is unproductive and lexically conditioned by the roots to which it attaches. The contrast between a basic, two-place base predicate and its applicative derivation is exemplified in (\ref{ex: applicative -ni exampleb}).

\ea\label{ex: applicative -ni example}

    \ea[]{
    \textit{ˈnè ˈmá ˈʔwîma suˈnù}\\
    \gll    ˈnè ˈmá ˈʔwî-ma suˈnù\\
            \textsc{1sg.nom} now harvest-\textsc{fut.sg} corn\\
    \glt    ‘I’ll harvest corn now.’\\
    \glt    ‘Ya voy a pizcar maíz.’ < LEL 06 4:151/el > \\
}\label{ex: applicative -ni examplea}
        \ex[]{
        \textit{ˈʔwînimon oˈlá ˈnè ˈjéra suˈnù}\\
        \gll    ˈʔwî-ni-ma=ni oˈlá ˈnè ˈjé-ra suˈnù\\
                harvest-\textsc{appl-fut.sg=1sg.nom} \textsc{cer} 1\textsc{sg.nom} mom-\textsc{poss} corn\\
        \glt    ‘I will harvest the corn for my mom.’\\
        \glt    ‘Le voy a pizcar el maiz a mi mamá.’ < BFL 06 5:146/el >\\
    }\label{ex: applicative -ni exampleb}
    \z
\z

\subsubsection{Applicative \textit{-si}}
\label{subsubsec: applicative si}

The suffix \textit{-si} is another unproductive, lexically conditioned stress-neutral applicative marker that increases the valency of the verb by adding a benefactive argument. This applicativesuffix is exemplified in (\ref{ex: applicative -si exampleb}):

\ea\label{ex: applicative -si example}

    \ea[]{
    \textit{ˈpáka}\\
    \gll    ˈpá-ka\\
            throw-\textsc{imp.sg}\\
    \glt    ‘Throw it!’\\
    \glt    ‘¡Tira!’ < BFL 06 5:147/el >\\
}\label{ex: applicative -si examplea}
        \ex[]{
        \textit{taˈmí ku ˈpáʃiri peˈlôta}\\
        \gll    taˈmí ku ˈpá-si-ri peˈlôta\\
                1\textsc{sg.acc} \textsc{rev} thow-\textsc{appl-imp.sg} ball\\
        \glt    ‘Throw the ball back at me!’\\
        \glt    ‘¡Tírame la pelota de vuelta!’ < BFL 06 5:147/el >\\
    }\label{ex: applicative -si exampleb}
    \z
\z

\largerpage
\subsubsection{Applicative \textit{-wi}}
\label{subsubsec: applicative wi}

A third Applicative suffix in position S3 is \textit{-wi}, another stress-neutral, unproductive suffix that adds a benefactive argument to a transitive predicate. This suffix is exemplified in (\ref{ex: applicative -wi exampleb}):

\ea\label{ex: applicative -wi example}

    \ea[]{
    \textit{waʔˈlû ˈnà aˈtʃ͡â biˈlé aʔˈpériti aˈnè}\\
    \gll    waʔˈlû ˈnà aˈtʃ͡-â biˈlé aʔˈpéri=ti aˈn-è\\
            big \textsc{dem} sit-\textsc{tr} one lump=1\textsc{pl.nom} say-\textsc{appl}\\
    \glt    ‘They put (lit. sit) a lot of what we call an \textit{apéri} (a lump).’\\
    \glt    ‘Ponen mucho de lo que le decimos una \textit{apéri} (una “moruca”, una bola con todo).’ \corpuslink{tx19[03_001-03_043].wav}{LEL tx19:3:00.1}\\
}\label{ex: applicative -wi examplea}
        \ex[]{
        \textit{ˈmín naˈpítʃ͡i aˈtʃ͡íwmo ˈlá toˈwí}\\
        \gll    ˈmí=ni naˈpítʃ͡i aˈtʃ͡wi-wi-ma oˈlá toˈwí\\
                2\textsc{sg.acc=1sg.nom} fire sit.\textsc{appl-appl-fut.sg} \textsc{cer} boy\\
        \glt    ‘I will sit down your boy next to the fire.’\\
        \glt    ‘Te voy a sentar al niño cerca de la lumbre.’ < BFL 06 6:146/el >\\
    }\label{ex: applicative -wi exampleb}
    \z
\z

\subsection{Causative \textit{-ti}}
\label{subsec: causative}

The Causative \textit{-ti} suffix is a stress neutral suffix that introduces an agent (causer) argument to the argument structure of a predicate. Causativization applies to both intransitive and transitive verbs. In the causative construction exemplified in (\ref{ex: causastive -ti exampleb}), the object corresponds to the subject argument of its basic, non-causative counterpart. The introduced agent argument causes the undergoer to perform the activity described by the verbal root.

\ea\label{ex: causastive -ti example}

    \ea[]{
    \textit{ˈnèmi riˈmênira}\\
    \gll    ˈnè=mi riˈmê-ni-ra\\
            1\textsc{sg.nom}=2\textsc{sg.acc} make.tortillas-\textsc{appl-pot}\\
    \glt    ‘I can make you tortillas.’\\
    \glt    ‘Yo te hago tortillas.’ < BFL 08 1:161/el >\\
}\label{ex: causastive -ti examplea}
        \ex[]{
        \textit{ˈmín ˈnè onoˈlá riˈmêentima}\\
        \gll    ˈmí=ni ˈnè ono-ˈlá riˈmê-ni-ti-ma\\
                2\textsc{sg.acc=1sg.nom} 1\textsc{sg.nom} father-\textsc{poss} make.tortillas-\textsc{appl-caus-fut.sg}\\
        \glt    ‘I will make you make tortillas for my dad.’\\
        \glt    ‘Te voy a hacer que le hagas tortillas a mi papá.’ < BFL 08 1:161/el >\\
    }\label{ex: causastive -ti exampleb}
    \z
\z

The Causative suffix has two lexically determined allomorphs, \textit{-ti} and \textit{-ri}. The allomorphy is also partially phonologically determined, since there is a phonological process that devoices voiced/lenis consonants after another consonant (a derived environment stemming from stress-conditioned syncope) (cf. §\ref{subsec: post-consonantal devoicing}). Examples of the distribution of allomorph \textit{-ti} are provided in (\ref{ex: phonological distribution of -ti}).

\ea\label{ex: phonological distribution of -ti}

    \ea[]{
    \textit{ˈlàantiki}\\
    \glt    \textit{ˈlàna-ti-ki}\\
    \glt    bleed-\textsc{caus-pst.ego}\\
    \glt    `I made them bleed.'\\
    \glt    `Lo hice sangrar.' < SFH 05 1:102/el >\\
    \glt    \textit{*làan-ri-ki}\\
}
        \ex[]{
        \textit{sikiˈréptiki}\\
        \glt    \textit{sikiˈrép-ti-ki}\\
         \glt   cut-\textsc{caus-pst.ego} \\
         \glt   `I made them cut.'\\
         \glt   `Lo hice cortar.' < BFL 05 1:113/el >\\
         \glt   \textit{*sikiˈrép-ri-ki}\\
    }
            \ex[]{
            \textit{oʔˈpéʃtia}\\
            \glt    \textit{oʔˈpés-ti-a}\\
             \glt   vomit-\textsc{caus-prog} \\
             \glt   `S/he is making them vomit.'\\
             \glt   `Los está haciendo vomitar.' < BFL 05 1:136/el >\\
             \glt   \textit{*oʔˈpés-ri-a}\\
        }
    \z
\z

The Causative \textit{-ti} suffix is highly productive, displaying no restrictions as to the bases to which it can attach.

\subsection{Applicative \textit{-ki}}
\label{subsec: applicative ki}

The Applicative suffix \textit{-ki} (S5) is another productive, stress-neutral suffix. This suffix introduces an additional argument to one-place or two-place predicates. The argument introduced is a benefactive or malefactive argument,\footnote{The cognates of Choguita Rarámuri Applicative suffixes in the closely related \ili{Mountain Guarijío} (\citealt{miller1996guarijio}) introduce other semantic roles in addition to benefactive/malefactive (e.g., instrumental). There is however no evidence that the suffix -ki or any of the other Applicatives in Choguita Rarámuri introduce semantic roles other than the benefactive/malefactive.} i.e., the object can be favorably or adversely affected. This suffix is exemplified in (\ref{ex: applicative -ki example}).

\ea\label{ex: applicative -ki example}

    \ea[]{
    \textit{ˈmá=n rataˈbátʃ͡iki koʔˈwáami}\\
    \gll    ˈmá=ni rata-ˈbá-tʃ͡a-ki koʔˈwá-ame\\
            already=1\textsc{sg.nom} be.hot-\textsc{inch-tr.pl-pst.ego} eat-\textsc{ptcp}\\
    \glt    ‘I already heated up the food.’\\
    \glt    ‘Ya calenté la comida.’ < BFL 08 1:20/el >\\
}\label{ex: applicative -ki examplea}
        \ex[]{
        \textit{ˈnèmi baʔˈwí rataˈbátʃ͡kira?}\\
        \gll    ˈnè=mi baʔˈwí rata-ˈbá-tʃ͡a-ki-ra\\
            1\textsc{sg.nom}=2\textsc{sg.acc} water be.hot-\textsc{inch-tr-appl-pot}\\
        \glt    ‘Shall I heat the water for you?’
        \glt    ‘¿Te caliento el agua?’ < BFL 08 1:21/el >\\
    }\label{ex: applicative -ki exampleb}
    \z
\z

In (\ref{ex: applicative -ki exampleb}), the Applicative introduces a benefactive argument as an unmarked object (\textit{ˈmí} ‘2\textsc{sg.acc}’) to a basic transitive predicate (an argument which would be expressed through a postpositional phrase in a non-applicative construction). The distribution of this suffix does not exhibit restrictions on its distribution, unlike the Applicative suffixes in suffix possition S3 (described in Appendix~\ref{subsec: applicatives}s above).

\section[The Aspectual Stem]{The Aspectual Stem: desiderative, associated motion and evidential markers}
\label{sec: the aspectual stem}

Another layer of the verbal stem is the Aspectual Stem, composed of suffixes in positions S6 to S9, marking desiderative, associated motion, and auditory evidential. The prosodic and morphosyntactic properties of these suffixes are discussed in §\ref{subsec: truncation in aspect/mood marking constructions} and §\ref{subsec: V-V incorporation constructions}, respectively. As discussed in §\ref{subsec: V-V incorporation constructions}, these suffixes may be analyzed as involving V-V incorporation.
%In this section I describe the meaning and morphophonological properties of these suffixes.

\subsection{Desiderative \textit{-nále}}
\label{subsec: desiderative}

The disyllabic Desiderative suffix \textit{-ˈnále} is a stress-shifting suffix of agent-oriented modality. Derived from the verb \textit{naˈkí} ‘want’, it has the meaning ‘X wants to/feels like doing X’, where the ‘wanter’ and the subject of the desideratum predication are correferent (when these two arguments are not correferent, a periphrastic construction must be used). Examples from context are shown in (\ref{ex: desiderative nale example}).

\ea\label{ex: desiderative nale example}

    \ea[]{
    \textit{ˈnè biˈlá niˈjúrka seˈbánili ba}\\
    \gll    ˈnè biˈlá niˈjúri-ka seˈbá-nale ba\\
            \textsc{int} indeed fight-\textsc{ger} reach-\textsc{desid} \textsc{cl}\\
    \glt    ‘He really wanted to reach it (against all odds).’\\
    \glt    ‘Lo quería alcanzar realmente (a fuerzas).’ \corpuslink{tx191[01_316-01_427].wav}{BFL tx191:1:31.6}\\
}
        \ex[]{
        \textit{aʔˈlì ˈnà moʔoˈtʃ͡íki tʃ͡uˈkúrili tʃ͡apiˈnála}\\
        \gll     aʔˈlì ˈnà moʔoˈtʃíki tʃ͡uˈkúri-li tʃ͡api-ˈnále-a\\
                and \textsc{dem} headboard go.around-\textsc{pst} grab-\textsc{desid-prog}\\
        \glt    ‘And then he was going around near the headboard wanting to get him.’\\
        \glt    ‘Y entonces andaba por la cabecera queriéndolo agarrar.’ \corpuslink{tx5[01_113-01_183].wav}{LEL tx5:1:11.3}\\
    }
    \z
\z

As discussed and exemplified in §\ref{subsec: truncation in aspect/mood marking constructions}, the desiderative and other affixes belonging to the aspectual stem undergo truncation when attaching CV suffixes encoding TAM categories. In the case of the desiderative suffix, it's truncated form is \textit{-na}.

\subsection{Associated motion \textit{-simi}}
\label{subsec: associated motion}

The Associated Motion suffix \textit{-simi} is a stress-neutral suffix derived from the free-standing motion verb \textit{simi} ‘go (sg.)’ (in it's truncated form, this suffix surfaces as \textit{-si}). Verbs marked with this suffix encode an event that is carried out while in motion (e.g., ‘X goes along doing V’). This is exemplified in (\ref{ex: associated motion suffix example}).

\ea\label{ex: associated motion suffix example}

    \ea[]{
    \textit{ˈwé koˈʔásimi}\\
    \gll    ˈwé koˈʔá-simi\\
            \textsc{int} eat-\textsc{mot}\\
    \glt    ‘They’re going along eating a lot.’\\
    \glt    ‘Van comiendo mucho.’ < SFH 08 1:71/el >\\
}
        \ex[]{
        \textit{toˈwí ˈwé ˈnârisimi buʔuˈtʃ͡ími}\\
        \gll    toˈwí ˈwé ˈnâri-simi buʔuˈtʃ͡ími\\
                boy \textsc{int} ask-\textsc{mot} road\\
        \glt    ‘The boy is going along the road asking a lot.’\\
        \glt    ‘El niño va preguntando muchas cosas por el camino.’ < SFH 08 1:148/el >\\
    }
    \z
\z

\subsection{Auditory evidential \textit{-tʃane}}
\label{subsec: auditory evidential}

The auditory evidential \textit{-tʃ͡ane} suffix is a productive epistemic modality marker that indicates that the evidence of the proposition encoded by the predicate has an auditory (i.e. non-visual) source (‘it sounds like X is taking place’). This stress-neutral suffix is exemplified in (\ref{ex: evidential suffix example}).

\ea\label{ex: evidential suffix example}

    \ea[]{
    \textit{ˈtʃ͡éti toˈrí ˈmá toˈrétʃ͡ane}\\
    \gll    ˈtʃ͡éti toˈrí ˈmá toˈré-tʃ͡ane\\
            \textsc{def.bad} chicken already cackle-\textsc{ev}\\
    \glt    ‘It sounds like the chicken are already cackling’\\
    \glt    ‘Ya se oyen cacarear las gallinas’ < SFH 08 1:160/el >\\
}\label{ex: evidential suffix examplea}
        \ex[]{
        \textit{tʃ͡oˈnítʃ͡ane maˈtʃ͡í}\\
        \gll    tʃ͡oˈní-tʃ͡ane maˈtʃ͡í\\
                fight-\textsc{ev} outside\\
        \glt    ‘It sounds/it seems like fighting is happening outside.’\\
        \glt    ‘Se oye/parece que pelean afuera.’ < BFL 08 1:17/el >\\
    }\label{ex: evidential suffix exampleb}
    \z
\z

In these constructions, the source for the evidence is the noise generated by the event itself that the predicate describes (e.g., ‘cackling’ or ‘fighting’ in (\ref{ex: evidential suffix examplea}) and (\ref{ex: evidential suffix exampleb}), respectively). The evidence can also come indirectly from another event, as in example
    (\ref{ex: evidential as inference example}), where the speaker infers that dancing will take place because of other non-visual cues (i.e., sound of the rattles used in the dance, people talking about starting to dance, etc.).

\ea\label{ex: evidential as inference example}

    \textit{ˈnápi ˈlé ˈmá awiˈmêt͡ʃani}\\
    \gll    ˈnápi aˈlé ˈmá awi-ˈmê-t͡ʃani\\
            \textsc{sub} \textsc{dub} already dance-\textsc{fut.sg-ev}\\
    \glt    ‘It sounds like they are about to dance.’\\
    \glt ‘Se oye como que van a bailar.’ < SFH 07 1:140/el >\\

\z

As the desiderative and associated motion suffixes, the auditory evidential suffix undergoes truncation when attaching outer CV suffixes encoding TAM distinctions. The truncated form of the auditory evidential suffix is \textit{t͡ʃa}.

\section{The Finite Verb: voice, tense, aspect and mood markers}
\label{sec: the finite verb}

The final layer of morphology in inflected verbs in Choguita Rarámuri involves suffixes in positions S9--S11 , the Finite Verb level suffixes. The inflectional categories encoded by these suffixes involve mood distinctions (including imperative and reportative), voice, tense, and aspect, with number and person marginally conflated with tense/aspect in portmanteaux suffixes. This final layer of morphology is required in all inflected verbs.

\subsection{Passive}
\label{subsec: passive}
\largerpage
\subsubsection{Past passive \textit{-ru}}
\label{subsubsec: past passive}

This suffix is a productive marker that encodes the object argument of the active transitive base has been promoted to subjecthood, while also encoding past tense. The subject of the active construction is not overtly expressed in the corresponding passive construction. As discussed in §\ref{subsubsec: past passive-conditioned lengthening}, the past passive suffix triggers lengthening of a preceding stressed syllable. Example (\ref{ex: past passive exampleb}) illustrates the past passive sense and concomitant lengthening in the stressed vowel of the base.

\ea\label{ex: past passive example}

    \ea[]{
    \textit{tòsnaleni}\\
    \gll    tò-si-nale=ni\\
            take-\textsc{mot-desid=1sg.nom}\\
    \glt    ‘I want to go along taking them.’\\
    \glt    ‘Quiero írmelas llevando.’ < BFL 06 5:149/el >\\
}\label{ex: past passive examplea}
        \ex[]{
        \textit{ˈtòoru grabaˈdôra}\\
        \gll    ˈtò-ru grabaˈdôra\\
                take-\textsc{pst.pass} recorder\\
        \glt    ‘The recorder was taken.’\\
        \glt    ‘Se llevaron la grabadora.’ < SFH 08 1:45/el >\\
    }\label{ex: past passive exampleb}
    \z
\z

\subsubsection{Future passive \textit{-pa}}
\label{subsubsec: future passive}

This is a productive, stress-shifting suffix that concomitantly marks a passive derivation and future tense. The following example illustrates the contrast between a basic active construction (\ref{ex: future passive examplea}) and a future passive construction (\ref{ex: future passive exampleb}).

\ea\label{ex: future passive example}

    \ea[]{
    \textit{ˈpîrim oˈlá tʃ͡iˈhánili naˈmûti}\\
    \gll    ˈpîri=mi oˈlá tʃ͡iˈhá-na-li naˈmûti\\
            why=\textsc{2sg.nom} make scatter-\textsc{tr-pst} things\\
    \glt    ‘Why did you scatter everything?’\\
    \glt    ‘¿Por qué desparramaste las cosas?’ < SFH 07 1:17-21/el >\\
}\label{ex: future passive examplea}
        \ex[]{
        \textit{naˈpátʃ͡i tʃ͡ihaˈnába ˈlé}\\
        \gll    naˈpátʃ͡a tʃ͡iha-ˈná-ba aˈlé\\
                shirts scatter-\textsc{tr-fut.pass} \textsc{dub}\\
        \glt    ‘The shirts will be scattered.’\\
        \glt    ‘Van a desparramar las blusas.’ < SFH 07 1:17-21/el >\\
    }\label{ex: future passive exampleb}
    \z
\z

The onset of this suffix undergoes lenition (for more details on this process, see §\ref{subsec: lenition of voiceless plosives} above).

\subsubsection{Medio-passive \textit{-rîwa, -wá}}
\label{subsubsec: medio-passive}

This suffix is used in constructions where the actor participant is backgorunded or left unspecified, and the undergoer participant is emphasized. The medio-passive suffix has two allomorphs, \textit{-riwa} and \textit{-wa}. Both allomorphs are stress-shifting. Examples (\ref{ex: mediopassive example}) illustrate the medio-passive morphological construction:

\ea\label{ex: mediopassive example}

    \ea[]{
        \textit{aˈnáwka biˈlá rupuˈnáruwa ba}\\
        \gll    aˈnáwi-ka biˈlá ripu-ˈná-riwa ba\\
                measure-\textsc{ger} really tear-\textsc{tr-mpass} \textsc{cl}\\
        \glt    ‘While measuring it, it is tore.’\\
        \glt    ‘Midiendo se troza.’ \corpuslink{tx1[01_214-01_228].wav}{BFL tx1:1:21.4}\\
}
        \ex[]{
        \textit{ˈnápi ˈlé riˈmênuwa ˈtʃ͡ém fedeˈrîko reˈmê}\\
        \gll    ˈnápi aˈlé riˈmê-ni-wa ˈtʃ͡ém fedeˈrîko reˈmê\\
                like \textsc{dub} tortillas-\textsc{appl-mpass} Mr. Federico tortillas\\
        \glt    ‘It seems like tortillas are being made for Mr. Federico.’\\
        \glt    ‘Como que le hacen tortillas a Don Federico.’ < SFH 07 2:69-72/el >\\
    }
    \z
\z

\subsubsection{Conditional passive \textit{-sûwa}}
\label{subsubsec: conditional passive}

The conditional passive suffix \textit{-sûwa} is a stress-shifting suffix that is productively used in complex clauses cumulatively expressing a conditional relationship and passive voice. The predicate marked with the conditional passive is the predicate of the protasis clause (describing the condition), not the apodosis (describing the potential result). An active-conditional construction is contrasted with a passive conditional construction in (\ref{ex: conditional passive example}).

\ea\label{ex: conditional passive example}

    \ea[]{
    \textit{ˈnè ˈámi ˈʃûntikisa ˈró}\\
    \gll    ˈnè ˈá=mi ˈsû-n-ti-ki-sa ˈró\\
            1\textsc{sg.nom} \textsc{aff=2sg.acc} sew-\textsc{appl-caus-appl-cond} Q\\
    \glt    ‘What if I made you sew her a skirt?’\\
    \glt    ‘¿Qué tal si te hago coserle una falda? < BFL 08 1:28/el >\\
}
        \ex[]{
        \textit{ˈkátʃ͡i aʔˈlá ˈʃûbo ˈlé riʔˈréki baˈtʃ͡á  ˈʃûʃuwa ko ba}\\
        \gll    ˈkátʃ͡i aʔˈlá ˈsû-bo aˈlé riʔˈré-ki baˈtʃ͡á ˈsû-suwa=ko ba\\
                because.\textsc{neg} well sew-\textsc{fut.pl} \textsc{dub} down-\textsc{supe} first sew-\textsc{cond.pass=emph} \textsc{cl}\\
        \glt    ‘Because we won’t sew it well if it is sewed on the bottom first.’\\
        \glt    ‘Porque no vamos a coserle bien si se cose abajo primero.’ \corpuslink{tx_falda[01_492-01_519].wav}{BFL tx\_falda:1:49.2} \\
}
    \z
\z

\subsection{Future}
\label{subsec: future}

\subsubsection{Future singular \textit{-ˈmêa, -ma}}
\label{subsubsec: future singular}

There are two future tense suffixes in Choguita Rarámuri: \textit{-mêa} \textasciitilde \textit{-ma}, for future, singular subject, and \textit{-pô} for future, plural subject. Historically, these suffixes developed from \ili{Proto-Sonoran} \textit{*mi(l)a} ‘go, run, \textsc{sg}’, and \textit{*po} ‘go, run, \textsc{pl}’ \citep[133]{miller1996guarijio}. The future
singular suffix has an unstressed allomorph (\textit{-ma}) and a stressed allomorph (\textit{-mêa}). Both the unstressed and stressed allomorphs of the future singular suffix are exemplified in (\ref{ex: future singular example}).

\ea\label{ex: future singular example}

    \ea[]{
    \textit{ˈhê ˈná=ni siˈpútʃ͡a sipuˈ-tá-mo ˈlá}\\
    \gll    ˈhê ˈná=ni siˈpútʃ͡a sipuˈ-tá-ma oˈlá\\
            it \textsc{prox=1sg.nom} skirt skirt-\textsc{vblz-fut.sg} \textsc{cer}\\
    \glt    ‘I will wear this skirt.’\\
    \glt    ‘Me voy a poner esta falda.’ < BFL 07 Sept 6/el >\\
}\label{ex: future singular examplea}
        \ex[]{
        \textit{ˈmá mukuˈmêa raˈjénali}\\
        \gll    ˈmá muku-ˈmêa raˈjénali\\
                already die-\textsc{fut.sg} sun\\
        \glt    ‘There will be an eclipse.’ (lit. ‘The sun will die.’)\\
        \glt    ‘Va a haber un eclipse.’ (lit. ‘se va a morir el sol.’) < SFH 05 2:63/el >\\
    }\label{ex: future singular exampleb}
    \z
\z

As described in §\ref{subsec: epistemic particles}, Choguita Rarámuri has epistemic modality markers that indicate the degree of certainty speakers have towards the actuality of an event. These modal particles are frequently found in future tense constructions, as exemplified in (\ref{ex: future singular examplea}). This example also illustrates the phonological effect that these particles have on the inflected verb’s final vowel, namely vowel deletion. Forms lacking such particles have a neutral interpretation with respect to the speaker’s commitment to the expectation that the event encoded by the predicate will take place or not in the future.

\subsubsection{Future plural \textit{-pô}}
\label{subsubsec: future plural}

The future plural suffix \textit{-pô} is a stress-shifting suffix used when the subject is either first or second person plural. Clauses with a third person plural subject may optionally be marked with the future plural suffix or the future singular suffix. The future plural suffix is exemplified in (\ref{ex: future plural example}).

%This suffix has two allomorphs, \textit{-pô} and \textit{-bô}.

\ea\label{ex: future plural example}

    \ea[]{
    \textit{ke naˈkíu-po ruˈwá taˈmí ˈwísia ruˈwá ˈétʃ͡i arˈiwála}\\
    \gll    ke naˈkíwi-po ru-ˈwá taˈmí ˈwí-si-a ru-ˈwá ˈétʃ͡i arˈiwá-la\\
            \textsc{neg} allow-\textsc{fut.pl} say-\textsc{mpass} 1\textsc{pl.acc} take-\textsc{mot-prog} say-\textsc{mpass} \textsc{dem} soul-\textsc{poss}\\
    \glt    ‘It’s said that we can’t let (the \textit{korimáka}) get to us, because they say it goes along taking our souls.’\\
    \glt    ‘Dicen que no hay que dejarnos de él (del \textit{korimáka}) porque dicen que nos roba el alma.’ \corpuslink{tx5[05_006-05_054].wav}{LEL tx5:5:00.6} \\
}
\newpage
        \ex[]{
        \textit{ruˈbô}\\
        \gll    ru-ˈbô\\
                say-\textsc{fut.pl}\\
        \glt    ‘They will say something.’\\
        \glt    ‘Van a decir.’ < SFH 04 1:27/el >\\
    }
    \z
\z

When used with the first person plural, the construction has a hortative reading (‘let us do X’). The hortative use of the future plural suffix is illustrated in (\ref{ex: hortative example}).

\ea\label{ex: hortative example}

    \ea[]{
    \textit{ˈmáti ilaˈrúpo}\\
    \gll    ˈmá=ti ila-ˈrú-po\\
            now=\textsc{1pl.nom} cactus-gather-\textsc{fut.pl}\\
    \glt    ‘Let’s gather cactus now!’\\
    \glt    ‘¡Vamos juntando nopales!’ < SFH 08 1:52/el >\\
}
        \ex[]{
        \textit{ˈmáti poˈtʃ͡ítisia iˈnârtipo?\\}
        \gll    ˈmá=ti poˈtʃ͡í-ti-si-a iˈnâ-ri-ti-po\\
                already=\textsc{1pl.nom} jump-\textsc{caus-mot-prog} go\textsc{.sg-caus-caus-fut.pl}\\
        \glt    ‘Shall we go along making them jump?’\\
        \glt    ‘¿Vamos haciéndolo que brinque?’ < BFL 07 2:32/el >\\
    }
    \z
\z

\subsection{Motion imperative \textit{-mê}}
\label{subsec: motion imperative}

The Motion Imperative suffix \textit{-mê} is a stress shifting suffix. It is a productive suffix that often occurs in conjunction with the imperative suffix \textit{-sa} (in position S8). Motion Imperative constructions with the suffix \textit{-mê} have the meaning ‘go and do X!’, used for a single addressee. When unstressed, the suffix vowel reduces to \textit{i} or undergoes complete deletion, following the general unstressed vowel reduction and deletion processes operating in the language (§\ref{subsubsec: stress-based vowel reduction and deletion}). This suffix is exemplified in (\ref{ex: motion imperative example}).

\ea\label{ex: motion imperative example}

    \ea[]{
    \textit{ˈàamsa}\\
    \gll    ˈà-me-sa\\
            give\textsc{-mot.imp-imp.sg}\\
    \glt    ‘Go give it to her!’\\
    \glt    ‘¡Ve y dáselo!’ < SHF 04 1:112/el >\\
}
\newpage
        \ex[]{
        \textit{iˈʃîmi}\\
        \gll    iˈsî-mi\\
                urinate-\textsc{mot.imp}\\
        \glt    ‘Go and urinate!’\\
        \glt    ‘¡Ve a orinar!’ < BFL 08 1:13/el >\\
    }
            \ex[]{
            \textit{ˈjúrka osiˈmêra ˈlé}\\
            \gll    ˈjúri-ka osi-ˈmê-ra aˈlé\\
                    take-\textsc{imp.sg} write-\textsc{mot.imp-pot} \textsc{dub}\\
            \glt    ‘Go take him to see if he writes.’\\
            \glt ‘Ve y llévalo a ver si escribe.’ < BFL 08 1:94/el >\\
        }
    \z
\z

When there are multiple adressees, the motion imperative construction involves the stress-shifting suffix \textit{-pi} (with stress-shifting allomorph \textit{-bô}), followed by the imperative plural suffix\textit{ -si}. This is exemplified in (\ref{ex: motion imp multiple}).

\ea\label{ex: motion imp multiple}

    \ea[]{
    \textit{osiˈbôsi}\\
    \gll    osi-ˈbô-si\\
            write-\textsc{mot.imp.pl-imp.pl}\\
    \glt    ‘You all go and write’\\
    \glt    ‘¡Vayan a escribir!’ < BFL 05 2:94/el >\\
}
        \ex[]{
        \textit{taˈmí ku ˈàkipisi}\\
        \gll    taˈmí ku ˈà-ki-po-si\\
            1\textsc{sg.acc} \textsc{rev} look.for-\textsc{appl-mot.imp.pl-imp.pl}\\
       \glt    ‘You all go and look for it for me!’\\
        \glt    ‘¡Vayan a buscármelo!’ < BFL 08 1:164/el >\\
    }
    \z
\z

\subsection{Conditional \textit{-sâ}}
\label{subsec: conditional}

This is a productive, stress-shifting suffix used in constructions that express a conditional relationship in the active voice (contrast with the conditional passive suffix described in Appendix~\ref{subsubsec: conditional passive} above). The verbal predicate marked with the conditional suffix is the predicate of the protasis clause. \citet{steele1975protoUA} reconstructs the cognate form of this suffix for \ili{Proto-Uto-Aztecan} as meaning “must/speaker wish” (\citeyear[216]{steele1975protoUA}). This stress-shifting suffix is exemplified in (\ref{ex: conditional suffix example}).

\ea\label{ex: conditional suffix example}

    \ea[]{
    \textit{ˈwé waˈrínami ˈnísako, ˈá mahaˈwá}\\
    \gll    ˈwé waˈrín-ame ˈní-sa=ko ˈá mahaˈwá\\
            \textsc{int} light-\textsc{ptcp} \textsc{cop-cond=emph} \textsc{aff} be.affraid\\
    \glt    ‘If she (the other runner) is really fast, she gets affraid’\\
    \glt    ‘Si es muy ligera (la otra corredora), sí le tiene miedo’ \corpuslink{tx19[00_451-00_484].wav}{LEL tx19:0:45.1}\\
}
        \ex[]{
        \textit{riˈhé ukuˈsâ ro, tʃ͡ú t͡ʃé=timi riˈkám méra?}\\
        \gll    riˈhé uku-ˈsâ ro, tʃ͡ú tʃ͡é=timi riˈká=mi mé-ra?\\
                hail rain-\textsc{cond} Q how how=\textsc{1pl.nom} like=\textsc{dem} scare.away-\textsc{pot}\\
        \glt    ‘And when it would hail? How did you guys scare it away?’\\
        \glt    ‘¿Y cuando llovía granizo? ¿Cómo lo espantaban?’ < SFH 07 in 243/in >\\
    }
    \z
\z

\subsection{Irrealis}
\label{subsec: irrealis}

\subsubsection{Irrealis singular \textit{-mê}}
\label{subsubsec: irrealis singular}

The irrealis singular suffix is used in constructions where the speaker has no certainty that a particular event will take place in the future, or if a particular event holds true in a hypothetical or contingent world. This stress-shifting suffix is highly productive (I have not documented any restrictions on its occurrence), and is used when the subject argument is singular. Examples of its use are presented in (\ref{ex: irrealis singular example}).

\ea\label{ex: irrealis singular example}

    \ea[]{
    \textit{koʔˈnálimi}\\
    \gll    koʔ-ˈnále-me\\
            eat-\textsc{desid-irr.sg}\\
    \glt    ‘She might want to eat.’\\
    \glt    ‘A lo mejor va a querer comer.’ < SFH 08 1:122/el >\\
}
        \ex[]{
        \textit{basaˈrówmi ˈlé ˈmá baʔaˈlîo}\\
        \gll    basaˈrówa-me aˈlé ˈmá baʔaˈlî-o\\
            stroll.around-\textsc{irr.sg} \textsc{dub} perhaps tomorrow-\textsc{ep}\\
        \glt    ‘Perhaps she will take a stroll tomorrow.’\\
        \glt    ‘A lo mejor va a pasear mañana.’ < BFL 07 1:150/el >\\
    }
            \ex[]{
            \textit{sukuˈmê ˈlé ˈmáo}\\
            \gll    suku-ˈmê aˈlé ˈmáo\\
                    scratch-\textsc{irr.sg} \textsc{dub} perhaps\\
            \glt    ‘Maybe he’ll sratch himself.’\\
            \glt    ‘A lo mejor se va a rascar.’ < SFH 08 1:45/el >\\
        }
    \z
\z

\subsubsection{Irrealis plural \textit{-pi}}
\label{subsubsec: irrealis plural}

Irrealis constructions with a plural subject argument are marked with the suffix -pi. This suffix is stress-neutral and, like the irrealis singular suffix described above, is highly productive. This suffix has two allomorphs, with a voiced and a voiceless stop onset (\textit{-pi} and \textit{-bi}). Examples are shown in (\ref{ex: irrealis pliural example}).

\ea\label{ex: irrealis pliural example}

    \ea[]{
    \textit{ˈmá ˈtôbi ˈlé ma}\\
    \gll    ˈmá ˈtô-bi aˈlé ma\\
            already bury-\textsc{irr.pl} \textsc{dub} perhaps\\
    \glt    ‘Maybe they will bury it already.’\\
    \glt    ‘A la mejor ya lo van a enterrar.’ < SFH 08 1:3/el >\\
}
        \ex[]{
        \textit{koʔˈnálpi ˈléti ˈmáo}\\
        \gll    koʔ-ˈnále-pi aˈlé=ti ˈmáo\\
                eat-\textsc{desid-irr.pl} \textsc{dub=1pl.nom} perhaps\\
        \glt    ‘Perhaps we might want to eat.’\\
        \glt    ‘A lo mejor vamos a querer comer.’ < BFL 06 5:140/el >\\
    }
    \z
\z

\subsection{Potential \textit{-râ}}
\label{subsec: potential}

This suffix is used in constructions expressing the possibility of occurrence of an event, ability or wishes (with an optative reading). This is a stress-shifting suffix and is exemplified in (\ref{ex: potential example}).

%\textit{-ta} and \textit{-ra}. Allomorph distribution is lexically and phonologically conditioned, governed by the conditions mentioned above (e.g., §4) and in Chapter 2 (§2.2.4).

\ea\label{ex: potential example}

    \ea[]{
    \textit{ˈétʃ͡i ˈa ˈmáalta ˈlé}\\
    \gll    ˈétʃ͡i  ˈa ˈmáli-ta aˈlé\\
            \textsc{dist} \textsc{aff} swim-\textsc{pot} \textsc{dub}\\
    \glt    ‘Let that one swim!’\\
    \glt    ‘¡Déjenlo nadar!’ < BFL 05 1:154/el >\\
}
        \ex[]{
        \textit{nuruˈrîa biˈlá baˈtʃ͡á aʔˈlá ˈnátika ˈénira}\\
        \gll    nuru-ˈrîwa biˈlá baˈtʃ͡á aʔˈlá ˈnáti-ka ˈéni-ra\\
                oblige-\textsc{mpass} really first well think-\textsc{ger} go.around-\textsc{pot}\\
        \glt    ‘They are sent first to go around carefully (lit. thinking well).’\\
        \glt    ‘Primero los mandan a que se cuiden bien.’ \corpuslink{tx48[00_321-00_375].wav}{BFL tx48:0:32.1}\\
    }
            \ex[]{
            \textit{witʃ͡iˈrâ}\\
            \gll    witʃ͡i-ˈrâ\\
                    fall-\textsc{pot}\\
            \glt    ‘You might fall!’\\
            \glt    ‘¡Te caes!’ (lit. ‘the puedes caer’)\\
        }
    \z
\z

\subsection{Imperative}
\label{subsec: imperative}

\subsubsection{Imperative singular \textit{-kâ}}
\label{subsubsec: imperative singular ka}

Imperatives may be marked through the bare stem, but there are also affixal exponents of imperative mood. One of such markers is suffix \textit{-kâ}, a productive, stress-shifting suffix. This suffix is exemplified in (\ref{ex: imperative ka example}).

%\pagebreak

\ea\label{ex: imperative ka example}

    \ea[]{
    \textit{ˈkíti naˈlàka!}\\
    \gll    ˈkíti naˈlà-ka\\
            \textsc{neg} cry-\textsc{imp.sg}\\
    \glt    ‘Don’t cry!’\\
    \glt    ‘¡No llores!’ < BFL 05 2:89/el >\\
}
        \ex[]{
        \textit{ˈwé simiˈkâ!}\\
        \gll    ˈwé simi-ˈkâ\\
                int go\textsc{.sg-imp.sg}\\
        \glt    ‘Go!’\\
        \glt    ‘¡Ve!’\\
    }
    \z
\z

\subsubsection{Imperative singular \textit{-sâ}}
\label{subsubsec: imperative singular sa}

Another imperative suffix used in constructions where the addressee is singular is \textit{-sâ}. This stress-shifting suffix is exemplified in (\ref{ex: imperative sa example}).

\ea\label{ex: imperative sa example}

    \ea[]{
    \textit{koʔˈsâ!}\\
    \gll    koʔ-ˈsâ\\
            eat-\textsc{imp.sg}\\
    \glt    ‘Eat!’\\
    \glt    ‘¡Come!’\\
}
        \ex[]{
        \textit{ˈmà-sa}\\
        \gll    ˈmà-sa\\
                run-\textsc{imp.sg}\\
        \glt    ‘Run!’\\
        \glt    ‘¡Corre!’ < BFL 04/11/06/el >\\
    }
    \z
\z

\subsubsection{Imperative plural \textit{-sì}}
\label{subsubsec: imperative plural}

Imperative constructions where the are multiple addressees are distinguished from imperatives with a single addressee with a productive, stress-shifting suffix, \textit{-sì}. Examples of this suffix are provided in (\ref{ex: imperative plural example}).

\ea\label{ex: imperative plural example}

    \ea[]{
    \textit{ko-ˈsì reˈmêke!}\\
    \gll    ko-ˈsì reˈmêke\\
            eat-\textsc{imp.pl} tortillas\\
    \glt    ‘You all eat tortillas!’\\
    \glt    ‘¡Coman tortillas!’\\
}
        \ex[]{
        \textit{taˈmí ku riˈwíisi}\\
        \gll    taˈmí ku riˈwí-i-si\\
                1\textsc{sg.acc} \textsc{rev} find-\textsc{appl-imp.pl}\\
        \glt    ‘You all find it for me!’\\
        \glt    ‘¡Vayan a encontrármelo!’ < BFL 08 1:16/el >\\
    }
    \z
\z

\subsection{Reportative}
\label{subusec: reportative}
\largerpage
The reportative suffix is an evidential suffix that indicates that the speaker’s source of information is hearsay. This productive marker, also used in direct quotation constructions, is a stress-neutral suffix which is added to the dependent verb of the complex sentence.  This switch-reference system is restricted, as it is not generalized to all constructions involving dependent clauses in Choguita Rarámuri.

\subsubsection{Reportative different subject \textit{-la}}
\label{subsubsec: reportative different subject}

When the notional subjects are not correferential, the dependent verb suffixes the different referent reportative \textit{-la} suffix (\ref{ex: reportative different subject example}).

\ea\label{ex: reportative different subject example}

    \ea[]{
    \textit{ˈá biˈlá oˈkám tʃ͡aˈnía ne ka ˈhémi isiˈmâtala ruˈá tʃ͡aˈbè}\\
    \gll   ˈá biˈlá oˈká=mi tʃ͡aˈní-a ne ka ˈhémi isiˈmâta-la ru-ˈwá tʃ͡aˈbè\\
            \textsc{aff} really many=\textsc{dem} sound-\textsc{prog} \textsc{int} ka here pass.\textsc{pl-rep} say-\textsc{mpass} before\\
    \glt    ‘Many people\textsubscript{i} say that they\textsubscript{j} used to pass through here long time ago.’\\
    \glt    ‘Muchas personas\textsubscript{i} dicen que por aquí pasaban\textsubscript{j} mucho antes.’ \corpuslink{tx223[04_176-04_224].wav}{LEL tx223:4:17.6}\\
}
        \ex[]{
        \textit{tʃ͡iˈnà ba ˈétʃ͡i biˈlá ˈtòola ruˈá aliˈwâla ba}\\
        \gll    ˈétʃ͡i ˈnà ba ˈétʃ͡i biˈlá ˈtò-la ru-ˈwá aliˈwâ-la ba\\
                \textsc{dem} \textsc{dem} \textsc{cl} \textsc{dem} really take.\textsc{pst.pass-rep} say-\textsc{mpass} soul-\textsc{poss} \textsc{cl}\\
        \glt    ‘They\textsubscript{i} say that that one\textsubscript{j} got his soul stolen there’\\
        \glt    ‘Cuentan\textsubscript{i} que a ese\textsubscript{j} ahí le llevó el alma’ < BFL rihói mukúri 6/tx >\\
    }
%\pagebreak
            \ex[]{
            \textit{ˈnápu riˈká ˈlátimi ˈwé uˈkúla ruˈá ˈníam tʃ͡aˈbèe=ko ba ní}\\
            \gll    ˈnápi riˈká oˈlá=timi ˈwé uˈkú-la ru-ˈwá ˈní-ame tʃ͡aˈbè=ko ba ní\\
                    \textsc{sub} like \textsc{cer=2pl.nom} \textsc{int} rain-\textsc{rep} say-\textsc{mpass} \textsc{cop-ptcp} before=\textsc{emph} \textsc{cl} \textsc{int}\\
            \glt    ‘So you all say that it used to rain a lot ling time ago.’\\
            \glt    ‘Pues así como dicen ustedes que llovía mucho antes.’ \corpuslink{in243[00_210-00_248].wav}{SFH in243:0:21.0}\\
        }
    \z
\z

\subsubsection{Reportative same subject \textit{-lo}}
\label{subsubsec: reportative same subject}

 When the notional subjects are correferential, the dependent verb is marked with the same referent reportative \textit{-lo} suffix (\ref{ex: reportative same subject example}).

\ea\label{ex: reportative same subject example}

    \ea[]{
    \textit{ˈá biˈláko aˈní ˈmâgre neˈhê amaˈtʃ͡îkolo ruˈá}\\
    \gll    ˈá biˈlá=ko aˈní ˈmâgre neˈhê amaˈtʃ͡î-ko-lo ru-ˈwá\\
            \textsc{aff} really=\textsc{emph} say.\textsc{prs} nuns 1\textsc{sg.nom} pray-\textsc{appl-rep} say-\textsc{mpass}\\
    \glt    ‘The nuns\textsubscript{i} say that they\textsubscript{i} prayed for me.’\\
    \glt    ‘Las monjas\textsubscript{i} dicen que (ellas\textsubscript{i}) me rezaron.’\\
}
        \ex[]{
        \textit{maˈnuêliko ˈwé biˈlá riˈkúlo ˈrú}\\
        \gll    maˈnuêli=ko ˈwé biˈlá riˈkú-lo ˈrú\\
                Manuel=\textsc{emph} \textsc{int} really get.drunk.\textsc{sg-rep.ss} say.\textsc{prs}\\
        \glt    ‘Manuel\textsubscript{i} says he\textsubscript{i} got drunk.’\\
        \glt    ‘Manuel\textsubscript{i} dice que (él\textsubscript{i}) se emborrachó.’\\
    }
    \z
\z

\subsection{Past \textit{-li}}
\label{subsec: past}

Past tense is marked by the suffix \textit{-li}, a stress-neutral suffix. The past perfective both situates the event in a point prior to the time of the speech act and indicates that the event has been completed. Examples of this construction are given in (\ref{ex: past suffix example}).

\ea\label{ex: past suffix example}

    \ea[]{
    \textit{aʔˈlìko ˈmá biˈlá ʃiˈnêami wiʰˈkâ ˈsíli tʃ͡oˈnà ˈétʃ͡i ˈná ˈétʃ͡i riˈhòi biˈtêritʃ͡i}\\
    \gll    aʔˈlì=ko ˈmá biˈlá siˈnê-ame wiʰˈkâ ˈsí-li tʃ͡oˈnà ˈétʃ͡i ˈná ˈétʃ͡i riˈhòi biˈtêritʃ͡i\\
            and=\textsc{emph} already really every-\textsc{ptcp} many arrive-\textsc{pst} there \textsc{dem} \textsc{prox} \textsc{dem} man house\\
    \glt    ‘And then everybody arrived there, to that man’s house.’\\
    \glt    ‘Y ya llegaron todos ahí a la casa de ese señor.’ \corpuslink{tx32[11_434-11_503].wav}{LEL tx32:11:43.4}\\
}
        \ex[]{
        \textit{ˈhê aˈnè aˈníʃia naˈwàli ˈétʃ͡i naˈmú niˈrá ʃuˈwá ba ˈá ruˈwèli}\\
        \gll    ˈhê aˈn-è aˈní-si-a naˈwà-li ˈétʃ͡i naˈmú niˈrá suˈwá ba ˈá ruˈw-è-li\\
                it say-\textsc{appl} say-\textsc{mot-prog} arrive-\textsc{pst} \textsc{dem} thing relative everybody \textsc{cl} \textsc{aff} say-\textsc{appl-pst}\\
        \glt    ‘He arrived saying that, that relative, he told everybody.’\\
        \glt    ‘Llegó diciendo un familiar, les dijo a todos.’ \corpuslink{tx5[04_070-04_105].wav}{LEL tx5:4:07.0}\\
    }
    \z
\z

\subsection{Past egophoric \textit{-ki}}
\label{subsec: past egophoric}

Verbs marked with the past egophoric \textit{-ki} suffix encode an event carried out in the past by a first-person subject in statements and by a second-person subjects in questions, have optional nominative-marked pronominal marking. This stress-neutral suffix is exemplified in (\ref{ex: past egophoric example}).
%In past egophoric constructions when the subject (34a) or object (34b) is first person, either singular or plural, the suffix used is \textit{-ki}.

\ea\label{ex: past egophoric example}

    \ea[]{
    \textit{pen soˈmáki bi}\\
    \gll    pe=ni soˈmá-ki=bi\\
            just=\textsc{1sg.nom} wash.head-\textsc{pst.ego}=just\\
    \glt    `I just washed my head.'\\
    \glt    `Nomás me lavé la cabeza.' \corpuslink{co1235[06_458-06_472].wav}{JLG co1235:6:45.8}\\
}
        \ex[]{
        \textit{ˈmí biˈtʃ͡éni kaˈlí piˈt͡ʃínula nuˈlèki ˈró}\\
        \gll    ˈmí biˈtʃ͡é=ni kaˈlí piˈt͡ʃí-nula nuˈl-è-ki ˈrú\\
                2\textsc{sg.acc} turn=1\textsc{sg.nom} house sweep-\textsc{order} order-\textsc{appl-pst.ego} say.\textsc{prs}\\
        \glt    ‘I told you to sweep the house!’\\
        \glt    ‘¡Te dije que barrieras la casa!’ < BFL 06 4:145/el >\\
    }
    \z
\z

For some speakers, this suffix is used in the past tense when the object argument is first person. This is shown in (\ref{ex: egophoric 1st oebject example}).

\ea\label{ex: egophoric 1st oebject example}

    \ea[]{
    \textit{ˈétʃ͡i taˈmí ˈúrki riʔˈréti}\\
    \gll    ˈétʃ͡i taˈmí ˈúri-ki reʔˈré-ti\\
            \textsc{dem} 1\textsc{sg.acc} take-\textsc{pst.ego} down-\textsc{all}\\
    \glt    ‘He took me down (the river).’\\
    \glt    ‘Él me llevó para abajo.’ < FLP 06 in61/in >\\
}
\newpage
        \ex[]{
        \textit{``ˈpé ke taˈmí ˈàki" ˈhê aniˈká}\\
        \gll    ˈpé ke taˈmí ˈà-ki ˈhê ani-ˈká\\
                just \textsc{neg} 1\textsc{sg.acc} give-\textsc{pst.ego} like.that say-\textsc{ger}\\
        \glt    ```They didn't give me" that's how you tell them.'\\
        \glt    ```A mi no me dieron" así se dice.' \corpuslink{co1237[05_455-05_473].wav}{JLG co1237:5:45.5}\\
    }
    \z
\z

As shown in (\ref{ex: past egophoric questions}a--b), the past egophoric suffix is also used with verbs encoding past tense and a second person subject in interrogative constructions. In contrast, interrogative clauses in the past tense with third person argument subjects are marked with the past tense \textit{-li} suffix, as in (\ref{ex: past egophoric questionsc}).

\ea\label{ex: past egophoric questions}

    \ea[]{
    \textit{kaˈbó ˈmí raʔˈláki saˈpâto?}\\
    \gll    kaˈbó ˈmí raʔˈl-á-ki saˈpâto\\
            when 2\textsc{sg.nom} buy-\textsc{tr-pst.ego} shoes\\
    \glt    ‘When did you buy the shoes?’\\
    \glt    ‘¿Cuándo compraste los zapatos? < SFH 05 1:74/el >\\
}\label{ex: past egophoric questionsa}
        \ex[]{
        \textit{``kúmi ˈpáki?" ˈhê aˈnè}\\
        \gll    ˈkúmi=mi ˈpá-ki ˈhê aˈn-è\\
                where=\textsc{2sg.nom} throw-\textsc{pst.ego} like.that say-\textsc{appl}\\
        \glt    ```Where did you throw it?" tell them like that."\\
        \glt    ```¿Dónde lo tiraste?" así di.' \corpuslink{co1235[10_317-10_330].wav}{JLG co1235:10:31.7}\\
    }\label{ex: past egophoric questionsb}
            \ex[]{
            \textit{ˈkúmi aˈsáli?}\\
            \gll    ˈkúmi aˈsá-li\\
                    where sit.\textsc{sg-pst}\\
            \glt    `Where was he?'\\
            \glt    `¿Dónde  estaba?' < MDH co1137:0:22.6 >\\
        }\label{ex: past egophoric questionsc}
    \z
\z

%There is speaker variation with respect of this use, but \textit{-ki} is mainly used when either the subject or object is first person. Some speakers use this suffix in constructions encoding a conjunct person (first person in declarative clauses (as in (34) above), and the addressee in questions (as in (35) below).

\subsection{Imperfective \textit{-e}}
\label{subsec: imperfective}

In contrast to the past tense suffix, the imperfective emphasizes the internal duration of the event depicted by the predicate. The Choguita Rarámuri imperfective encodes an incomplete or habitual event that takes place over a period of time. The imperfective is marked with the stress-neutral suffix \textit{-e}, a marker which does not display any allomorphy or occurence restrictions. Due to the general process of post-tonic vowel reduction that operates in the language (see §\ref{subsubsec: stress-based vowel reduction and deletion}), this suffix is realized with the high, front vowel \textit{-i} for most speakers. Examples are provided in (\ref{ex: imperfective suffix example}).

\ea\label{ex: imperfective suffix example}

    \ea[]{
    \textit{napaˈlì ke ˈtʃ͡ó niˈrúi ko sekunˈdâria ba}\\
    \gll    ˈnápi aʔˈlì ke ˈtʃ͡ó niˈrú-i=ko sekunˈdâria ba\\
            \textsc{sub} then \textsc{neg} yet exist-\textsc{impf=emph} secondary \textsc{cl}\\
    \glt    ‘When it didn’t use to be any secondary school yet’\\
    \glt    ‘Cuando todavía no había secundaria’ \corpuslink{tx12[01_222-01_249].wav}{SFH tx12:1:22.2}\\
}
        \ex[]{
        \textit{aˈwísinili}\\
        \gll    aˈwí-si-nale-i\\
                dance-\textsc{mot-desid-impf}\\
        \glt    ‘She wanted to go along dancing’\\
        \glt    ‘Quería irse bailando’ < SFH 07 2:72-73/el >\\
    }
            \ex[]{
            \textit{ˈhúmtʃ͡ani}\\
            \gll    ˈhúmi-tʃ͡ane-i\\
                    take.off.\textsc{pl-ev-impf}\\
            \glt    ‘It sounded like they were taking off’\\
            \glt    ‘Se oía como que se arrancaban’ < SFH 07 1:7/el >\\
        }
    \z
\z

\subsection{Progressive \textit{-a}}
\label{subsec: progressive}

The progressive is encoded by the stress-neutral suffix \textit{-a}, and it indicates that the event described by the predicate is an ongoing process which is independent of time reference. Uses of this marker are exemplified in (\ref{ex: progressive suffix example}).

\ea\label{ex: progressive suffix example}

    \ea[]{
    \textit{ˈnári witʃ͡ónula ˈmá nuluˈría wiˈtʃ͡ôa}\\
    \gll    ˈnári witʃ͡ó-nula ˈmá nulu-ˈría wiˈtʃ͡ô-a\\
            then wash-\textsc{order} also oblige-\textsc{mpass} wash-\textsc{prog}\\
    \glt    ‘And then they are also sent to wash clothes.’\\
    \glt    ‘Y también las mandan a lavar ropa.’ \corpuslink{tx48[01_554-02_015].wav}{BFL tx48:1:55.4}\\
}
        \ex[]{
        \textit{ˈwé biˈláti ˈwé kaˈnírami ˈhú tamuˈhêko ˈnà umuˈkî roˈwétia, iˈwé roˈwétia, ˈkúutʃ͡i ˈkûruwi ˈmá raraˈhîptia}\\
        \gll    ˈwé biˈlá=ti ˈwé kaˈnír-ame ˈhú tamuˈhê=ko ˈnà umuˈkî roˈwé-ti-a, iˈwé roˈwé-ti-a, ˈkúutʃ͡i ˈkûruwi ˈmá raraˈhîp-ti-a\\
                \textsc{int} really=\textsc{1pl.nom} \textsc{int} happy-\textsc{ptcp} \textsc{cop.prs} 1\textsc{pl.nom=emph} \textsc{dem} run.womens.race-\textsc{caus-prog} girls run.womens.race-\textsc{caus-prog} small children also race.for.men\textsc{caus-prog}\\
                \newpage
        \glt    ‘We like it a lot indeed, to make women, girls and also young children run a race.’\\
        \glt    ‘Nos gusta mucho hacer correr a las mujeres, a las niñas y a los niños chiquitos.’ \corpuslink{tx19[00_236-00_265].wav}{LEL tx19:00:23.6}, \corpuslink{tx19[00_265-00_312].wav}{LEL tx19:0:26.5}\\
    }
    \z
\z

\subsection{Indirect causative \textit{nula}}
\label{subsec: indirect causative}

In indirect causative constructions, a causer exerts indirect manipulation on a causee which retains certain degree of autonomy.
Indirect causative constructions in Choguita Rarámuri involve a periphrastic
construction in which a main jussive predicate takes the caused event as a complement. The lower predicate is marked with the suffix \textit{-nula}. This stress-neutral suffix is derived from the independent verb \textit{nula} ‘to order, to command’. The main predicate can be inflected with any tense or aspect, but the lower predicate marked with \textit{-nula} is closed to further suffixation. For more etails about the indirect causative construction, see Appendix~\ref{subsec: indirect causative}. Examples of the indirect causative are provided in (\ref{ex: indirect causative example}).

\ea\label{ex: indirect causative example}

    \ea[]{
    \textit{aʔˈlì tʃ͡iˈhônsako ˈmá ˈpé oˈtʃ͡êrisako nuruˈría baʔˈwí ˈtúnula}\\
    \gll    aʔˈlì tʃ͡iˈhônsa=ko ˈmá ˈpé oˈtʃ͡êri-sa=ko nuru-ˈría baʔˈwí ˈtú-nula\\
            and then=\textsc{emph} already little grow-\textsc{cond=emph} oblige-\textsc{mpass} water bring-\textsc{order}\\
    \glt    ‘And then when they grow a little they are ordered to bring water.’\\
    \glt    ‘Y entonces ya cuando crecen más los mandan a traer agua.’ \corpuslink{tx48[00_414-00_457].wav}{BFL tx48:00:41.4}, \corpuslink{tx48[00_457-00_496].wav}{BFL tx48:00:45.7 0:03.9}\\
}
        \ex[]{
        \textit{ˈmán ˈhúaki raʔˈlìnula ˈtiêndatʃ͡i}\\
        \gll    ˈmá=ni ˈhúa-ki raʔˈlì-nula ˈtiêndatʃ͡i\\
                already=1\textsc{sg.nom} send-\textsc{pst.ego} buy-\textsc{order} store\\
        \glt    ‘I already sent him to the store to buy’\\
        \glt    ‘Ya lo mandé comprar a la tienda.’ < BFL 06 2:48/el >\\
    }
    \z
\z

%The suffix \textit{-nula} has a monosyllabic allomorph -na. The details of allomorph distribution and conditions of selection are addressed in X.

\section{The Subordinate Verb: deverbal morphology}
\label{sec: the subordinate verb}

\subsection{Temporal \textit{-tʃi}}
\label{subsec: temporal}

The temporal \textit{-tʃ͡i} suffix is a stress-neutral morpheme added to predicates of adverbial clauses which encode a temporal relation between two events (translated into English as ‘when’ clauses). The base for affixation of this suffix is a verb inflected for progressive aspect. The following examples (in (\ref{ex: temporal suffix example})) illustrate the use of this suffix.

%\pagebreak

\ea\label{ex: temporal suffix example}

    \ea[]{
    \textit{ˈnápu riˈká oˈmáwiri ˈnám omoˈwáruatʃ͡i}\\
    \gll    ˈnápi riˈká oˈmáwa-ri ˈná=mi omoˈwá-riwa-a-tʃ͡i\\
            \textsc{sub} like make.party-\textsc{nmlz} then=\textsc{dem} make.party-\textsc{mpass-prog-temp}\\
    \glt    ‘when parties are made’\\
    \glt    ‘cuando hacen fiesta’ \corpuslink{tx12[05_405-05_444].wav}{SFH tx12:5:40.5}\\
}
        \ex[]{
        \textit{ˈnè ˈkrîlitʃ͡i ʃiˈmêa ˈmá ʃuˈwíbatʃ͡i ˈhê ná taˈrári}\\
        \gll    ˈnè ˈkrîlitʃ͡i si-ˈmêa ˈmá suˈwíb-a-tʃ͡i ˈhê ná taˈrári\\
                1\textsc{sg.nom} Creel go.\textsc{sg-fut.sg} already finish-\textsc{prog-temp} it this week\\
        \glt    `I'm going to Creel when this week is finished.'\\
        \glt    ‘Voy a ir a Creel cuando acabe esta semanana.’\\
    }
    \z
\z

\subsection{Epistemic \textit{-o}}
\label{subsec: epistemic}

The epistemic modality suffix marks lower predicates of complement clauses of main predicates that express a psychological or mental state, like ‘think’, ‘dream’, ‘sing’ or ‘say’. The use of this suffix is exemplified in the examples in (\ref{ex: espistemic suffix example}).

\ea\label{ex: espistemic suffix example}

    \ea[]{
    \textit{riˈmùini ˈnáptim noˈkáo}\\
    \gll    riˈmù-i=ni ˈnápi  =timi noˈká-o\\
            dream-\textsc{impf=1sg.nom} \textsc{sub=2pl.acc} move-\textsc{ep}\\
    \glt    ‘I used to dream that you all were moving.’\\
    \glt    ‘Yo soñaba que ustedes se movían.’\\
}
        \ex[]{
        \textit{aʔˈlì ˈnà kotʃ͡iˈká buˈʔílo maˈjêli}\\
        \gll    aʔˈlì ˈnà kotʃ͡i-ˈká buˈʔí-li-o maˈjê-li\\
        and then sleep-\textsc{ger} lay.down.\textsc{sg-pst-ep} think-\textsc{pst}\\
        \glt    ‘And then he thought he was asleep (laid down sleeping).’\\
        \glt    ‘Nomás que pensó que estaba dormido’ \corpuslink{tx5[00_350-00_383].wav}{LEL tx5:0:35.0}\\
    }
    \z
\z

For some speakers, the epistemic suffix \textit{-o} is also attested in V-V incorporation constructions, as shown in (\ref{ex: epistemic in V-V incorporation}).

\ea\label{ex: epistemic in V-V incorporation}

    \ea[]{
    \textit{oʔˈpéstʃ͡analo}\\
    \gll    oʔˈpés\textbf{-tʃa-nale}-o\\
            vomit-\textsc{ev{-desid}-ep}\\
    \glt    ‘It sounds like they want to vomit.’ \\
    \glt    ‘Se oye como que quieren vomitar.’ < BFL 07 rec300/el >\\
}
        \ex[]{
        \textit{paraˈértʃ͡analo}\\
        \gll    paraˈér\textbf{-tʃa-nale}-o\\
                dance.paraeri-\textsc{ev{-desid}-ep}\\
        \glt    ‘It sounds like they want to dance paraéri.’ \\
        \glt    ‘Se oye como que quieren bailar paraéri.’  < BFL 07 1:182/el >\\
    }
    \z
\z


\subsection{Gerund \textit{-ká}}
\label{subsec: gerund}

The gerund suffix \textit{-ká} occurs in subordinate clauses and marks a non-finite verbal construction denoting an ongoing event which occurs simultaneous to another event. This stress- neutral suffix is exemplified in (\ref{ex: gerund suffix example}):

\ea\label{ex: gerund suffix example}

    \ea[]{
    \textit{púlako biˈlá niwaˈrîa ˈnàri biʔˈrìnka baˈtʃ͡á ba biˈlé ˈtá kuˈʃìti ba}\\
    \gll    ˈpúla=ko biˈlá niwa-ˈrîwa ˈnàri biʔˈrì-na-ka baˈtʃ͡á ba biˈlé ˈtá kuˈsì-ti ba\\
            belt=\textsc{emph} really make-\textsc{mpass} then roll.up-\textsc{tr-ger} first \textsc{cl} one \textsc{det} stick-\textsc{instr} \textsc{cl}\\
    \glt    ‘The belt is made by rolling it up first with a stick.’\\
    \glt    ‘La faja se hace enrollándolo primero con un palo.’ \corpuslink{tx1[00_249-00_312].wav}{BFL tx1:0:24.9}\\
}
        \ex[]{
        \textit{aʔˈlì ˈnà kotʃ͡iˈká buˈʔílo maˈjêli}\\
        \gll    aʔˈlì ˈnà kotʃ͡i-ˈká buˈʔí-l-o maˈjê-li\\
                and then sleep-\textsc{ger} lay.down.\textsc{sg-pst-ep} think-\textsc{pst}\\
        \glt    ‘And then he thought he was asleep (laid down sleeping).’\\
        \glt    ‘Nomás que pensó que estaba dormido.’ \corpuslink{tx5[00_350-00_383].wav}{LEL tx5:0:35.0}\\
    }
    \z
\z

As discussed in §\ref{subsec: conditional clauses}, verbal predicates in apodosis clauses of conditional constructions are often marked with the gerundive \textit{-ká} suffix. Very frequently, verbs marked with the \textit{-ká} suffix appear in verb chaining structures conveying a temporal relation of chronological overlap or chronological sequence (with some extended semantic meanings in some cases). For more details about this, see §\ref{sec: clause chaining}.

\subsection{Purposive \textit{-ra}}
\label{subsec: purposive}

The purposive suffix \textit{-ra} is a stress-neutral suffix which derives a noun from a finite verb inflected for progressive aspect. The purposive indicates that the derived noun is an instrument or means involved in carrying out the event described by the predicate. This suffix is not limited to a few lexical items, and may be productively added to any finite verb inflected for progressive aspect. The forms in (\ref{ex: purposive example}) exemplify this nominalization process.

\ea\label{ex: purposive example}

    \ea[]{
    \textit{ˈpòara}\\
    \gll    ˈpò-a-ra\\
            cover-\textsc{prs-purp}\\
    \glt    ‘lid’ (lit. ‘for covering’)\\
    \glt    ‘tapadera’ (lit. ‘para tapar’) < SFH 07 in242/in >\\
}
        \ex[]{
        \textit{oˈsìara}\\
        \gll    oˈsì-a-ra\\
                write-\textsc{prog-purp}\\
        \glt    ‘pen’ (lit. ‘for writing’)\\
        \glt    ‘pluma’ (lit. ‘para escribir’)\\
    }
    \z
\z

\subsection{Participial \textit{-ame}}
\label{subsec: participial}

One final layer of morphologically complex verbs involves one optional layer of morphology (in suffix position S12), where a finite verb may attach morphemes that encode subordination. As discussed in §\ref{subsec: agentive, patientive and experiencer nominalizations}, affixation of the participial suffix \textit{-ame} to transitive bases or intransitive bases with an unergative argument, derive nominalizations with an agentive meaning (‘the one who performs V’) (\ref{ex: participial suffixa}). Patientive nominalizations, on the other hand, are formed through attachment of the participial suffix \textit{-ame} to passivized verbal bases (\ref{ex: participial suffixb}) or to intransitive verbs with a theme as subject argument (\ref{ex: participial suffixc}). Finally, theme nominalizations are derived though attachment of the participial \textit{-ame} suffix to a medio-passive base (\ref{ex: participial suffixd}).

\ea\label{ex: participial suffix}

    \ea[]{
   \textit{waʔˈlûm ˈwe ˈnà raˈhâame buˈsêame tʃuˈkú naˈʔî}\\
    \gll    waʔˈlû=mi ˈwe ˈnà raˈhâ-ame buˈs-ê-ame tʃuˈkú naˈʔî\\
            big\textsc{=dem} \textsc{int} then lit.up-\textsc{ptcp} eye-\textsc{-have-ptcp} sit.\textsc{sg.prs} here\\
    \glt    `A big one with lighten up eyes is (lit. sits) here.'\\
    \glt    `Muy grande, con unos ojotes está (se sienta) aquí.' \corpuslink{tx5[00_592-01_036].wav}{LEL tx5:0:59.2}\\
}\label{ex: participial suffixa}
        \ex[]{
        \textit{ˈétʃi ˈmi ˈnà ˈtòruame tʃiˈhônsa ko ba ˈnà ˈmá wiʰˈkâ riˈhòrame ba}\\
        \gll    ˈétʃi ˈmi ˈnà ˈtò-ru-ame tʃiˈhônsa=ko ba ˈnà ˈmá wiʰˈkâ riˈhò-r-ame ba\\
                \textsc{dem} \textsc{dist} \textsc{dem} take-\textsc{pst.pass-ptcp} then=\textsc{emph} \textsc{cl} \textsc{dem} already many live.people-r-\textsc{ptcp} \textsc{cl}\\
        \glt    ‘It was then taken there because there were a lot of people already.’\\
        \glt     ‘Para allá fue llevada entonces porque ya había mucha gente.’\\ \corpuslink{tx12[00_482-00_520].wav}{SFH tx12:0:48.2}\\
    }\label{ex: participial suffixb}
            \ex[]{
            \textit{ˈwé aʔˈláa ˈnísa ka baʰˈtâri ba ˈwé aʔˈlá waˈsiame ˈnísa ka ba}\\
            \gll    ˈwé aʔˈlá ˈní-sa ka baʰˈtâri ba ˈwé aʔˈlá waˈsiame ˈnísa ka ba \\
                    \textsc{int} well \textsc{cop-cond} \textsc{irr} corn.beer \textsc{cl} \textsc{int} well cook-\textsc{ptcp} \textsc{cop-cond} \textsc{irr} \textsc{cl}\\
            \glt    `It would be good the corn beer if it is well cooked.'\\
            \glt    ‘Sí será bueno el tesgüino si está bien cosido.’\\ \corpuslink{tx68[02_300-02_325].wav}{LEL tx68:2:30.0}\\
        }\label{ex: participial suffixc}
                \ex[]{
                \textit{ke biˈlé tʃaˈpí ˈétʃi liˈmôsna aniˈrîami ko, muˈkuî ko}\\
                \gll    ke biˈlé tʃaˈpí ˈétʃi liˈmôsna ani-ˈrî-ame=ko muˈkuî=ko\\
                        \textsc{neg} uno grab \textsc{dem} charity say-\textsc{mpass-ptcp=emph} women=\textsc{emph}\\
                \glt    ‘She doesn't take what it is called charity, the women. ’\\
                \glt    ‘No agarra lo que se dice ‘limosna’, las señoras.’ \corpuslink{tx19[05_207-05_243].wav}{LEL tx19:5:20.7}\\
            }\label{ex: participial suffixd}
    \z
\z
