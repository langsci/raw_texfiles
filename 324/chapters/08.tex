\chapter{Nouns}
\label{chap: nominal morphology}

Nouns in Choguita Rarámuri can be broadly defined morphosyntactically as the class of words that may be case-marked. Case marking in this language includes instrumental, locative and oblique case markers (nominative and accusative case is encoded exclusively on pronominal forms, as discussed in \chapref{chap: particles, adverbs and other word classes}). Some classes of nouns do not take all case markers available: kinship and body part terms, two closed classes of inalienable nouns, may only occur in possessive constructions and are not found with locative or instrumental case, but are instead obligatorily head-marked with a suffix encoding a possessive relation and indexing properties of the possessor. Possessive suffixes are also deployed with alienably possessed nouns, though this encoding is not obligatory.

This chapter is concerned with the morphological properties of Choguita Rarámuri nouns, their internal morphological structure, and the derivational and inflectional morphological constructions with which nouns may combine. This chapter begins with an overview of the morphotactic generalizations that hold across morphologically complex nouns §\ref{sec: morphotactic generalizations}; the following sections describe several nominal morphological constructions, including plural/pluractional marking (§\ref{sec: pluractional marking}), case marking (§\ref{sec: case marking}), posessive marking (§\ref{sec: possessive marking}), and deverbal noun marking (§\ref{sec: deverbal nouns}). This chapter also addresses how \ili{Spanish} noun loanwords are incorporated in the language (§\ref{sec: Spanish loan nouns}). The chapter concludes with a description of tone patterns in morphologically complex nouns (§\ref{sec: tone in morphologically complex nouns}). The syntactic properties of noun phrases are addressed in \chapref{chap: noun phrases}.

%%Morphosyntactic, morphological and phonological properties. Tonally heterogeneous, as well as displaying the stressed/unstressed contrast (i.e., with fixed stress vs. alternating stress depending on the morphological context).

%\subsection{Kinship terms}
%\label{subsec: kinship terms}

%\subsection{Body-part terms}
%\label{subsec: body-part terms}

\section{Morphotactic generalizations}
\label{sec: morphotactic generalizations}

Productive nominal inflectional morphology in Choguita Rarámuri is mostly restricted to case marking, though core grammatical roles such as subject and objects are not marked in nominal phrases. Pluractional/plural formation is a category that is analyzed as derivational for verbs (a matter of lexical choice) (see \chapref{chap: verbal morphology}, §\ref{subsubsec: pluractionality}), but may be analyzed as inflectional when used with nouns. Nominal stems may be morphologically derived (most frequently, through deverbal processes) or monomorphemic. The bulk of nominal morphology in the language is devoted to derivational processes, including several morphological constructions to derive nouns from verbal roots; such derivational processes are analyzed here as taking place within a nominal stem level which can then undergo inflectional affixation. Nominal morphology is, however, significantly less complex than verbal morphology, both in terms of agglutination possibilities as well as the sheer number of morphological processes available.

The order of inflectional affixes in nouns is illustrated in (\ref{ex: order of morphological processes in nominal stems}):

\ea\label{ex: order of morphological processes in nominal stems}
{Order of inflectional affixes in nouns}

\begin{tabular}{lllll}
     -1 & 0 & +1 & +2 & +3 \\
     \textsc{pl} & \textsc{stem} & \textsc{poss} \textit{-wa} & \textsc{poss} \textit{-lâ} & \textsc{loc, instr}\\
\end{tabular}
    \z

%\ea\label{ex: order of morphological processes in nominal stems}
%Order of inflectional affixes in nouns

%\quadruplebox{-1}{0}{+1}{+2}{+3}\\

%\quadruplebox{Plural}{\textsc{stem}}{Poss. \textit{-wa}}{Poss. \textit{-lâ}}{Loc., Instr.}\\

%\z

As described below (in §\ref{sec: possessive marking}), this morphological structure includes two suffix positions for possessive suffixes, which are ordered sequentially in certain semantic configurations.

Like verbs, the morphological structure of nouns conforms to the general affix ordering principle of derivation ordered within inflection \parencite{bybee1985morphology}, with derivational processes applying within the stem level and inflectional affixes occupying peripheral slots of the morphological structure. There are no documented permutations of affixes in morphologically complex nouns.

%show here the data that shows the ordering generalizations

The examples in (\ref{ex: genitive-possessive affix order}) show that the possessive \textit{-lâ} suffix is ordered before the locative case suffix (\ref{ex: genitive-possessive affix order}a--b) and that the possessive \textit{-wa} suffix is ordered before the possessive \textit{-lâ} suffix (\ref{ex: genitive-possessive affix order}c). In (\ref{ex: genitive-possessive affix order}a), the quality of the vowel in the possessive suffix is reduced post-tonically (see §\ref{subsec: stress and stress-dependent phenomena} in \chapref{chap: word prosody}).

\ea\label{ex: genitive-possessive affix order}
{Relative ordering of possessive and locative suffixes}

    \ea[]{
    \textit{niˈhê    moˈʔôlitʃ͡i ˈtʃ͡ó  ba}\\
    \gll    neˈhê moˈʔô\textbf{-lâ-tʃi} ˈtʃ͡ó ba\\
            \textsc{1sg.nom} head-{\textsc{poss-loc}} also  \textsc{cl}\\
    \glt    ‘in my head (memory)’\\
    \glt    ‘en mi cabeza (memoria)’  < SFH in61(750)/in >\\
}
        \ex[]{
        \textit{taˈmí     biˈlé   sioˈrí  iˈkîli roˈnôlatʃ͡i}\\
        \gll     taˈmí biˈlé sioˈrí iˈkî-li roˈnô\textbf{-lâ-tʃi}\\
                1\textsc{sg.acc}  one  fly  bite-\textsc{pst} foot-{\textsc{poss-loc}}\\
        \glt    ‘A fly bit me in the foot.’\\
        \glt    ‘Me picó un mosco en el pie.’   < FMF 09 3:61/el >\\
    }
\pagebreak
            \ex[]{
            \textit{ˈnè		sikoˈríwala}\\
            \gll    ˈnè			            sikoˈrí-\textbf{wa-lâ}\\
	                1sg.\textsc{subj}	pot-\textsc{{poss-poss}}\\
            \glt    `my pot'\\
            \glt    `mi olla'    < BFL 06 4:187 >\\
        }
    \z
\z


Example (\ref{ex: nominalizer-locative suffix ordera}) shows the relative order between the nominalizer suffix \textit{-li} (with\-in the nominal stem) and the locative \textit{-tʃ͡i} suffix; example (\ref{ex: nominalizer-locative suffix orderb}) shows the relative order between other derivational suffixes in the nominal stem and the instrumental \textit{-ti} suffix:

\ea\label{ex: nominalizer-locative suffix order}
{Nominalizer-locative and instrumental suffix order}

    \ea[]{
    {\textit{ˈpé    roʔaˈká     koˈʔá  ˈpé    baˈtʃ͡ôkilitʃ͡i ba}}\\
    \gll    ˈpé  roʔa-ˈká  koˈʔá  ˈpé  baˈt͡ʃôka\textbf{-li-tʃi}     ba \\
            just   put.in-\textsc{ger}  eat   just   build.with.mud-{\textsc{nmlz-loc}} \textsc{cl} \\
    \glt    `We eat putting on the bachókilichi.'\\
    \glt    `Comemos poniéndolos en los bachókilichi.’   \corpuslink{in61[03_168-03_208].wav}{FLP in61:3:16.8}\\
}\label{ex: nominalizer-locative suffix ordera}
        \ex[]{
        \textit{roˈnêamti}\\
        \gll    ronô\textbf{-ê-ame-ti}\\
                foot-{have-\textsc{ptcp-inst}}\\
        \glt    ‘with an instrument with legs’\\
        \glt    ‘con un instrumento que tiene piernas’  \corpuslink{tx1[00_381-00_416].wav}{BFL tx1:0:38.1}\\
    }\label{ex: nominalizer-locative suffix orderb}
    \z
\z


Nominal morphological constructions may be either stress-shifting or stress-neu\-tral, i.e., triggering stress shifts or causing no stress alternations. As in the rest of the word classes, all lexical tonal contrasts are realized in the stressed syllable of the word. The morphologically-conditioned phonological properties of morphologically complex words are addressed at length in \chapref{chap: word prosody}, \chapref{chap: verbal morphology} and \chapref{chap: prosody}. There are no distinct morpho-phonological properties of nouns in derived environments, so the characterizations below will thus only specify the phonological properties of both roots and suffixes. Morpho-pho\-nological alternations and semantic properties of nominal morphological constructions are addressed in detail in this chapter. Other morphosyntactic and semantic properties of nominal forms are addressed in \chapref{chap: noun phrases}.


\section{Plural/pluractional marking}
\label{sec: pluractional marking}

Nouns that refer to animate referents in Choguita Rarámuri are marked plural through a morphological construction that also combines with verbs.\footnote{This construction deploys the same formal mechanisms of exponence with both nouns and verbs. Plural/pluractional marking in both nouns and verbs is thus treated as at least diachronically related in this grammar. Whether pluractionality as a verbal category and plural number with nouns are indeed part of a single morphosyntactic construction or should be treated as separate morphosyntactic devices is a question left for further research.} With verbs, this construction marks either a plural subject or that an action occurs or is being performed by the same agent several times, or by several agents several times (see §\ref{subsec: number marking: suppletion and plural prefixes}). With nouns, the construction exclusively marks plural number. These meanings are related in that they refer to event plurality or ‘pluractionality’. Pluractionals have been defined as encompassing meanings that range from iterative and frequentative to distributive and extensive action (\citealt{newman1990nominal}, \citealt{newman2012pluractional}, \citealt{wood2007semantic}) .

Choguita Rarámuri pluractional/plural forms are marked through a prefixed element also documented in closely related Rarámuri varieties and \ili{Guarijío} (\citealt{lionnet1968intensivos}, \citealt{lionnet1985lionnet}, \citealt{miller1996guarijio}).\footnote{\citet{lionnet2001elementos} labels the cognate forms of these constructions as “intensive” in \ili{Norogachi Rarámuri}.} In \ili{Mountain Guarijío}, the cognate process (labelled “plural subject, iterative or durative”) is more clearly analyzed as reduplication, since the prefixed element is (C)V- (e.g. \textit{saˈé}, \textit{sa-saˈe} ‘smell’, \textit{iˈsi, i-iˈsi} ‘walk’ \citep[][62]{miller1996guarijio}).

Plurals/pluractionals in Choguita Rarámuri are marked in one of three ways: (i) through a prefixed vowel (\ref{ex: pluractionals -nouns}a-e) (where the vowel quality of the prefix may be harmonized to the root’s first syllable vowel, as in (\ref{ex: pluractionals -nouns}d) (see also \citealt{lionnet2001elementos});\footnote{And where plural/pluractional marking involves further phonological changes (as in \textit{iˈwé} ‘girls’ in (\ref{ex: pluractionals -nouns}b)).} (ii) through consonant mutation (\ref{ex: pluractionals -nouns}f); or (iii) through both consonant mutation and vowel prefixation (\ref{ex: pluractionals -nouns}g--j).\footnote{There is an example that shows an alternation between morphologically related nouns where the relation is not one of singular/plural, but rather of plural/collective: the deverbal noun \textit{ˈhâ}\textbf{\textit{w}}\textit{-ami}, ‘authorities’, has a pluractional counterpart with prefixation and consonant mutation, \textbf{\textit{i}}\textit{-ˈhâ}\textbf{\textit{p}}\textit{iʃi}, which refers the totality of the individual named authorities. This pair is unusual in that it displays a correspondence between the bilabial voiceless stop \textit{p} and the labio-velar glide \textit{w}, a correspondence not found in any of the other plural/pluractional alternations. In addition, the form \textit{ˈhâw-ami} is a nominalization through the participial suffix \textit{-ame}, while the pluractional form has an ending that does not correspond to any productive suffix in the language. Thus, we might conclude that this pair has been lexicalized.}

%explain here that the analysis in the cases of a vowel prefix is that a mid front vowel would predictably reduce to [i] in pre-tonic position given the rules of stress-based vowel reduction/deletion given in Ch. 2

\pagebreak

\ea\label{ex: pluractionals -nouns}
{Plural/pluractional marking with nouns}\footnote{The form \textit{ˈtôro} in (\ref{ex: pluractionals -nouns}l) is a loanword from \ili{Spanish} and differs from the rest ofthe pluractional-marked nouns in that the correspondence is between an alveolar voiceless stop in the base stem and an alveolar trill, not an alveolar flap, in the pluractional form. This form also has a long vowel. See §\ref{subsec: optional prosodic loanword adaptation patterns} for discussion and examples of \ili{Spanish} loanwords with long vowels.}

\begin{tabular}{llll}
    & \textit{Singular} & \textit{Plural/pluractional} & \textit{Gloss} \\
     a.& siˈríame & \textbf{i}-ˈsêrikame & `governor(s)' \\
      & \corpuslink{el1318[14_357-14_398].wav}{MFH el1318:14:35.7}& \corpuslink{tx816[00_367-00_399].wav}{JMF tx816:0:36.7} & \\
     b.& teˈwé & \textbf{i}-ˈwé & `girl(s)' \\
      & \corpuslink{el1027[00_329-00_352].wav}{SFH el1027:0:32.9} & \corpuslink{tx372[01_092-01_132].wav}{LEL tx372:1:09.2} &\\
     c.& biˈlé    & \textbf{i}-ˈbíli & `one/some'  \\
     & \corpuslink{tx60[01_319-01_338].wav}{BFL tx60:1:31.9} & \corpuslink{in61[02_371-02_381].wav}{FLP in61:2:37.1} &\\
     d.& koˈt͡ʃî &\textbf{o}-koˈtʃ͡î  & `dog(s)' \\
     & \corpuslink{tx84[02_523-02_542].wav}{LEL tx84:2:52.3}& < SFH NDB/el > &\\
     e. & muˈ\textbf{k}î & \textbf{o}-muˈ\textbf{k}î & `woman/ \\
      & \corpuslink{tx5[01_183-01_213].wav}{LEL tx5:1:18.3}&  \corpuslink{tx1009[01_033-01_055].wav}{GFP tx1009:1:03.3} & women'\\
     f.& \textbf{r}emaˈrí &  \textbf{ˈt}émuri & `young man/ \\
     & \corpuslink{tx475[06_499-06_539].wav}{SFH tx475:6:49.9} &   < BFL 05 1:155/el > & men'\\
     g. &  \textbf{r}emaˈrí &  \textbf{t}eˈmári & `young man/ \\
     &  \corpuslink{tx475[06_499-06_539].wav}{SFH tx475:6:49.9} & \corpuslink{tx109[01_054-01_109].wav}{LEL tx109:1:05.4} & men'\\
     h.& \textbf{r}ikuˈrí &  \textbf{ˈt}êkiri  & `drunk(ards)' \\
     & < SFH NDB/el > &  \corpuslink{in243[15_460-15_471].wav}{FLP in243:15:46.0} &\\
     i.& uˈ\textbf{p}êami & huˈ\textbf{b}êamti & `married man/ \\
     & < BFL 09 4:74/el > &  \corpuslink{tx816[00_434-00_491].wav}{JMF tx816:0:43.4} & men'\\
     j.& tʃ͡aˈ\textbf{b}ôtʃ͡i & \textbf{i}-ˈtʃ͡á\textbf{p}ot͡ʃi & `mestizo(s)' \\
     & \corpuslink{tx128[00_522-00_567].wav}{SFH tx128:0:52.2}& < BFL 05 1:155/el > & \\
     k.& \textbf{r}aˈnâra & \textbf{a}\textbf{t}aˈnâra & `offspring'\\
     & \corpuslink{co1140[15_089-15_108].wav}{MDH co1140:15:08.9}&  \corpuslink{tx152[10_446-10_461].wav}{SFH tx152:10:44.6} & \\
     l.& \textbf{ˈt}ôro & \textbf{i}ˈ\textbf{r}ôoro & `bull(s)' \\
     & < BFL 05 3:44/el > & < BFL 05 3:44/el > & \\
     m.&  \textbf{b}aˈtʃ͡âwala & \textbf{a}-\textbf{p}aˈt͡ʃâwala & `first one(s)' \\
     & \corpuslink{tx43[04_056-04_120].wav}{SFH tx43:4:05.6}& \corpuslink{in242[02_182-02_244].wav}{FLP in242:2:18.2}&  \\
\end{tabular}
    \z

%check UA cognates
%update numbering here
As shown in these examples, consonant mutation involves a voicing toggle, since it produces voicing or lenition of a voiceless stop (\ref{ex: pluractionals -nouns}i, l), and devoicing or hardening of a voiced plosive and rhotic (\ref{ex: pluractionals -nouns}f--h, j--k, m) in paradigmatically related nominal forms. Plural forms, in addition, may also involve a change in stress placement with respect to the base (see, e.g., (\ref{ex: pluractionals -nouns}a, c, f--h)), although there is no phonological predictability as to the location of stress in the derived form. This is also attested in cases where the same lexical item will have more than one pluractional form: this is the case with the pluractional of \textit{remaˈrí} `young man', which is \textit{ˈtémuri} for some speakers (\ref{ex: pluractionals -nouns}f), with first syllable stress, but \textit{\textit{teˈmári}} for others (\ref{ex: pluractionals -nouns}g), with second syllable stress. More details on the prosodic and other general phonological properties of plural/pluractional forms in Choguita Rarámuri are provided in §\ref{subsec: number marking: suppletion and plural prefixes}.

%examples where there is prefixation and no change in voicing

%j.& muˈ\textbf{k}î & \textbf{o}-muˈ\textbf{k}î & `woman/women' \\
 %    & \corpuslink{tx5[01_183-01_213].wav}{LEL tx5:1:18.3}&  \href{}{< GFP tx1009:1:03.3 >
%} &\\

\section{Case marking}
\label{sec: case marking}

Case is defined here as a property of noun phrases. Choguita Rarámuri nominals may bear instrumental (§\ref{subsec: instrumental case}) or locative (§\ref{subsec: locative case}) case. As mentioned above, core grammatical relations (subject and object) are not encoded morphologically with nouns, though the pronominal system evidences a nominative/accusative system, and accusative markers are attested in other \ili{Taracahitan} languages, e.g., \ili{Yaqui} (\citealt{lindenfeld1973yaqui}, \citealt{escalante1990setting}, \citealt{dedrick1999sonora}).

\subsection{Instrumental case}
\label{subsec: instrumental case}

Instrumental case is marked with the stress-neutral suffix \textit{{}-ti},\footnote{As discussed in §\ref{sec: stress properties of roots, stems and suffixes}, stress neutral suffixes are never stressed and do not trigger any stress changes to the stems they attach to.} and is found exclusively marking nouns that bear the semantic role of instruments, i.e., there are no recorded examples where this particular case marking is extended to other semantic roles. An example of instrumental marking on an instrumental NP is provided in (\ref{ex: instrumental NP}).

\ea\label{ex: instrumental NP}

    \ea[]{
        \textit{roˈnôti riˈkêka}\\
        \gll    roˈnô-ti riˈkê-ka\\
                foot-\textsc{instr} step-\textsc{ger}\\
        \glt    `You step on it with the foot.'\\
        \glt    `Con el pie se pisa.' \corpuslink{co1136[04_002-04_017].wav}{MDH co1136:04:00.2}\\
}
            \ex[]{
            \textit{{roˈnô}}\\
            `foot'\\
            `pie' {< BFL 06 5:127/el >}\\
        }
    \z
\z


Further examples of instrumental case marking are given in (\ref{ex: instrumental case examples}).

\pagebreak

\ea\label{ex: instrumental case examples}
{Instrumental case}

    \ea[]{
        \textit{{siˈkâti}}\\
        \textit{siˈkâ-ti}\\
        hand-\textsc{instr}\\
        ‘with the hand’\\
        `con la mano' {< BFL 06 5:128/el >}\\
    }
            \ex[]{
            \textit{{t͡ʃuˈmíti}}\\
            {\textit{t͡ʃuˈmí-ti}}\\
            snout-\textsc{instr}\\
            ‘with the snout'\\
            `con el pico' {< SFH 06 6:84/el >}\\
        }
                    \ex[]{
                    \textit{{koˈbísiti}} \\
                    \textit{koˈbísi-ti}\\
                    pinole-\textsc{instr}\\
                    {‘with the pinole’}\\
                    `con el pinole' {< BFL 06 5:128/el >}\\
                }
                            \ex[]{
                            \textit{{kiˈmáti}}\\
                            \textit{kiˈmá-ti}\\
                            {kiˈmá-\textsc{instr}}\\
                            `with the blanket'\\
                            `con la cobija' {< BFL 06 5:128/el >}\\
                        }
                                \ex[]{
                                \textit{{kuˈʃìti}} \\
                                \textit{kuˈsì-ti}\\
                                stick-\textsc{instr}\\
                                {‘with the stick'} \\
                                `con el palo' \corpuslink{tx1[00_249-00_312].wav}{BFL tx1:0:24.9}\\
                            }
                                    \ex[]{
                                    \textit{{t͡ʃiˈníti}}\\
                                    \textit{t͡ʃiˈní-ti}\\
                                    cloth-\textsc{instr}\\
                                    ‘with the cloth'\\
                                    `con la tela' {< SFH 06 6:84/el >}\\
                                }
                                        \ex[]{
                                        \textit{ˈwàsti}\\
                                        \textit{ˈwàsi-ti}\\
                                        cows-\textsc{instr}\\
                                        {`with the cows'}\\
                                        `con las vacas'        \corpuslink{in484[01_148-01_171].wav}{ME in484:1:14.8}\\
                                    }
                                    \pagebreak
                                            \ex[]{
                                            \textit{{reˈʰtêti}} \\
                                            \textit{reˈʰtê-ti}\\
                                            stone-\textsc{instr}\\
                                            {`with stone'} \\
                                            `con piedra' \corpuslink{in61[05_519-05_548].wav}{FLP in61:5:51.9}\\
                                        }
                                                \ex[]{
                                               \textit{{baˈrîkati}}\\
                                               \textit{baˈrîka-ti}\\
                                                bucket-\textsc{instr}\\
                                                {`with a bucket’}\\
                                                `con barrica' \corpuslink{tx60[01_123-01_147].wav}{BFL tx60:1:12.3}\\
                                            }
                                                    \ex[]{
                                                    \textit{{waˈrîti}}\\
                                                    \textit{waˈrî-ti}\\
                                                    basket-\textsc{instr}\\
                                                    {`with a basket'}\\
                                                    `con canasta' \corpuslink{tx60[01_482-01_535].wav}{BFL tx60:1:48.2}\\
                                                }
                                                        \ex[]{
                                                        \textit{ˈtînat͡ʃiti}\\
                                                        \textit{ˈtînat͡ʃi-ti}\\
                                                        bucket-\textsc{instr}\\
                                                        {`with a bucket'} \\
                                                        `con tina'\footnote{This example involves a loanword from \ili{Spanish}. As described in §\ref{sec: Spanish loan nouns} below, relatively recent loan nouns from \ili{Spanish} are adapted with the -tʃ͡i suffix, which is homophonous with the locative suffix.} { < SFH 06 6:84-ff/el >}\\
                                                    }
                                                            \ex[]{
                                                            \textit{{ripuˈráti}}\\
                                                            \textit{ripuˈrá-ti}\\
                                                            axe-\textsc{instr}\\
                                                            {`with an axe’} \\
                                                            `con hacha' {< SFH 06 6:84/el >}\\
                                                        }
                                                                \ex[]{
                                                                \textit{{busuˈt͡ʃíti}}\\
                                                                \textit{busuˈt͡ʃí-ti}\\
                                                                eyes-\textsc{instr}\\
                                                                {`with the eyes'} \\
                                                                `con los ojos' \corpuslink{tx_muerto[01_391-01_422].wav}{BFL tx\_muerto:1:39.1}\\
                                                            }
    \z
\z

As the examples in (\ref{ex: instrumental case examples}) show, this construction does not involve any allomorphy or trigger any phonological effects in the bases with which it combines. Instrumental case marking is very productive,\footnote{Many \ili{Uto-Aztecan} languages have a set of instrumental prefixes that, attached to verbs, indicate the instrument with which an activity is carried out. While still synchronically active in the \ili{Tepiman} and \ili{Numic} branches of \ili{Uto-Aztecan} \parencite{dayley1989tumpisa}, instrumental prefixes only have lexicalized remnants in Choguita Rarámuri (for a discussion of instrumental prefixes in Choguita Rarámuri, see §\ref{subsec: instrumental prefixes}).} and there seem to be no semantic restrictions of the bases that can combine with this construction (i.e., instrumental case marking is not exclusive to a closed class of nouns).

With inalienably possessed nouns, the base for affixation of the instrumental is the non-possessed, bare root (e.g., (\ref{ex: instrumental case examples})). The instrumental suffix may also be added to morphologically complex bases: in (\ref{ex: instrumental added to participial form}), for instance, an instrumental suffix is added to a nominal base with the participial \textit{-ame} suffix (the participal suffix is analyzed here as being part of the nominal stem level).

\ea\label{ex: instrumental added to participial form}

\textit{{roˈnêamti}}\\
\gll    ronô-ê-ame-ti\\
        foot-have-\textsc{ptcp-inst}\\
\glt    ‘with an instrument with legs’\\
\glt    ‘con un instrumento con piernas’   \corpuslink{tx1[00_381-00_416].wav}{BFL tx1:0:38.1}\\

\z

Though instrumental markers commonly display syncretism with comitative markers cross-linguistically (\citealt{Stassen-2000}, \citealt{Stolz-2001a}), Choguita Rarámuri comitatives are generally expressed through a different construction, a postpositional phrase, exemplified in (\ref{ex: comitative example with postpositional phrase}):

 \ea\label{ex: comitative example with postpositional phrase}

{\textit{ˈnè      ko     ˈhê   riˈká    wiˈlí        ˈkûrisi   \textbf{ˈjûa}}}\\
\gll    ˈnè=ko    ˈhê  riˈká  wiˈlí      ˈkûrisi    \textbf{ˈjûa}\\
        \textsc{1sg.nom}=\textsc{emph} \textsc{dem} like.that stand.\textsc{sg.prs} cross     {with} \\
\glt    ‘I was standing like that with the cross.’\\
\glt    ‘Yo estaba parado aquí por este lado con la cruz’ < FLP 06 in61(470)/in >\\

\z
%\todo[inline]{Check gloss} - FIXED [GC]

The suffix \textit{-ti} is thus a dedicated instrumental marker in the language.

%the section below should be moved to ch. 9, section on adverbs
\subsection{Locative case}
\label{subsec: locative case}

There are two productive locative case markers in Choguita Rarámuri: the stress-shifting suffix \textit{{}-tʃ͡í} and the stress-neutral suffix \textit{-rare}.\footnote{Post-tonic vowel reduction yields the surface form \textit{-riri}, as described in §\ref{subsubsec: stress-based vowel reduction and deletion}.} The locative suffix \textit{-tʃ͡í} means `on, at’ (and may be classified as adessive), while the suffix \textit{-rare} means ‘at’, and ‘among, between’ (with an inessive reading). The locative \textit{-tʃ͡i} suffix is exemplified in (\ref{ex: locative -chi examples}) and the locative \textit{-rare} suffix in (\ref{ex: locative -rare examples}) (the latter suffix has the surface form \textit{-riri} due to regular rules of post-tonic vowel reduction (see \chapref{chap: phonology}, §\ref{subsubsec: stress-based vowel reduction and deletion}).

\ea\label{ex: locative -chi examples}
{Locative (adessive) \textit{-tʃ͡í}}

    \ea[]{
    \textit{wasaˈtʃ͡í}\\
    \textit{wasa-ˈtʃ͡í}\\
    land-\textsc{loc}\\
    { `on the land'} \\
    `en la tierra' \corpuslink{tx43[07_443-07_505].wav}{SFH tx43:7:44.3}\\
}
        \ex[]{
        \textit{kawiˈtʃ͡í}\\
        \textit{kawi-ˈtʃ͡í}\\
        hill-\textsc{loc}\\
        { `on the hill'} \\
        `en el cerro' \corpuslink{tx977[01_266-01_288].wav}{SFH tx977:1:26.6}\\
    }
            \ex[]{
            \textit{koˈbísitʃ͡i}\\
            \textit{koˈbísi-tʃ͡i}\\
            corn.meal-\textsc{loc}\\
            {`on the corn meal'} \\
            `en el pinole' {< BFL 06 5:128/el >}\\
        }
                \ex[]{
                \textit{{sipuˈtʃ͡âtʃ͡i}}\\
                \textit{sipuˈtʃ͡â-tʃ͡i}\\
                skirt-\textsc{loc}\\
                {`on the skirt'}\\
                `en la falda' {< BFL 06 5:128/el >}\\
            }
                    \ex[]{
                    \textit{{ˈpúratʃ͡i}}\\
                    \textit{ˈpúra-tʃ͡i}\\
                    knitted.belt-\textsc{loc}\\
                    {`on the knitted belt'}\\
                    `en la faja' {< BFL 06 5:128/el >}\\
                }
                        \ex[]{
                       \textit{{iˈwákatʃ͡i}} \\
                       \textit{iˈwáka-tʃ͡i}\\
                       hole-\textsc{loc}\\
                        {`in the hole'}\\
                        `en el hoyo' \corpuslink{tx177[04_263-04_345].wav}{LEL tx177:4:26.3}\\
                    }
            \pagebreak
                            \ex[]{
                            \textit{{tiˈjôpitʃ͡i}} \\
                            \textit{tiˈjôpa-tʃ͡i}\\
                            church-\textsc{loc}\\
                            {`at the church'}\\
                            `en la iglesia' \corpuslink{el1275[00_118-00_143].wav}{JLG el1275:0:11.8}\\
                        }
                                \ex[]{
                                \textit{baˈtʃ͡ôkilitʃ͡i} \\
                                \textit{baˈtʃ͡ôki-li-tʃ͡i}\\
                                fix.with.mud-\textsc{nmlz-loc}\\
                                {`on the mud'} \\
                                `en el barro'  \corpuslink{in61[03_168-03_208].wav}{FLP in61:3:16.8}\\
                            }
                                    \ex[]{
                                    \textit{{sikoˈrítʃ͡i}}\\
                                    \textit{sekoˈrí-tʃ͡i}\\
                                   olla-\textsc{loc}\\
                                    `in the pot'\\
                                    `en la olla'\corpuslink{tx130[05_265-05_310].wav}{LEL tx130:5:26.5}\\
                                }
                                        \ex[]{
                                        \textit{{saʔpaˈtʃ͡í}} \\
                                       \textit{saʔˈpa-tʃ͡í}\\
                                        meat-\textsc{loc}\\
                                        {`on the meat'}\\
                                        `en la carne' {< SFH in61(490)/in >}\\
                                    }
                                            \ex[]{
                                            \textit{{osiˈrítʃ͡i}} \\
                                            \textit{osiˈrí-tʃ͡i}\\
                                            paper-\textsc{loc}\\
                                            {`on the paper'} \\
                                            `en el papel' {< FLP in61(739)/in >}\\
                                        }
                                                \ex[]{
                                                \textit{{beˈtôlitʃ͡i}} \\
                                                \textit{beˈtôli-tʃ͡i}\\
                                                plate-\textsc{loc}\\
                                                {`on the plate'}\\
                                                `en el cajete' \corpuslink{tx130[05_412-05_444].wav}{LEL tx130:5:41.2}\\
                                            }
                                                    \ex[]{
                                                    \textit{{baˈrîkatʃ͡i}} \\
                                                    \textit{baˈrîka-tʃ͡i}\\
                                                    bucket-\textsc{loc}\\
                                                    {`on the bucket'}\\
                                                    `en la barrica' \corpuslink{co1234[06_233-06_258].wav}{JLG co1234:6:23.3}\\
                                                }
                                    \pagebreak
                                                        \ex[]{
                                                       \textit{kosˈtâaltʃ͡i}\footnote{In this form, the long stressed vowel is due to compensatory lengthening triggered by post-tonic vowel deletion. More details about compensatory lengthening are provided in §\ref{subsubsec: compensatory lengthening}.} \\
                                                       \textit{kosˈtâal-tʃ͡i}\\
                                                       sack-\textsc{loc}\\
                                                        {`on the sack'}\\
                                                        `en el costal' \corpuslink{tx60[01_026-01_045].wav}{BFL tx60:1:02.6}\\
                                                    }
                                                            \ex[]{
                                                           \textit{{saˈrâpitʃ͡i}} \\
                                                           \textit{saˈrâpi-tʃ͡i}\\
                                                           blanket-\textsc{loc}\\
                                                            {`on the blanket'}\\
                                                            `en la cobija' \corpuslink{tx60[01_065-01_092].wav}{BFL tx60:1:06.5}\\
                                                        }
                                                                \ex[]{
                                                                \textit{moˈrîintʃ͡i}\footnote{This form also undergoes compensatory lengthening of the stressed vowel triggered by post-tonic vowel deletion.} \\
                                                                \textit{moˈrîin-tʃ͡i}\\
                                                                grinder-\textsc{loc}\\
                                                                {`on the grinder'}\\
                                                                `en el molino' \corpuslink{tx60[01_212-01_250].wav}{BFL tx60:1:21.2}\\
                                                            }
                                                                    \ex[]{
                                                                   \textit{{ronoˈtʃ͡í}} \\
                                                                   \textit{rono-ˈtʃ͡í}\\
                                                                   foot-\textsc{loc}\\
                                                                    {`on the foot'}\\
                                                                    `en el pie' \corpuslink{co1140[17_527-17_548].wav}{MDH co1140:17:52.7}\\
                                                                }
    \z
\z


\ea\label{ex: locative -rare examples}
{Locative (inessive) \textit{-rare}} ([-riri])

    \ea[]{
   \textit{{winoˈmîriri}}\\
   \textit{winoˈmî-riri}\\
   money-\textsc{loc}\\
    {`in the money'}\\
    `en el dinero' {< BFL 06 5:127/el >}\\
}
            \ex[]{
            \textit{{muˈnîriri}}\\
            \textit{muˈnî-riri}\\
            beans-\textsc{loc}\\
            {`in the beans'}\\
            `en los frijoles' {< BFL 06 5:127/el >}\\
        }
    \pagebreak
                \ex[]{
               \textit{{kiˈmáriri}}\\
               \textit{kiˈmá-riri}\\
               blanket-\textsc{loc}\\
                {`in the blanket'}\\
                `en la cobija' {< LEL 06 5:127-9/el >}\\
            }
                    \ex[]{
                   \textit{{wiˈtʃ͡îriri}}\\
                   \textit{wiˈtʃ͡î-riri}\\
                   skin-\textsc{loc}\\
                    {`in the skin'}\\
                    `en la piel' {< BFL 06 5:128/el >}\\
                }
                        \ex[]{
                      \textit{{kuˈsìriri}}\\
                      \textit{kuˈsì-riri}\\
                      sticks-\textsc{loc}\\
                        {`in the sticks'}\\
                        `en los palos' \corpuslink{tx1[00_312-00_348].wav}{BFL tx1:0:31.2}\\
                    }
                            \ex[]{
                            \textit{{reˈʰtêriri}}\\
                            \textit{reˈʰtê-riri}\\
                            stone-\textsc{loc}\\
                            {`in the stones'}\\
                            `en la piedra' \corpuslink{tx5[02_069-02_106].wav}{LEL tx5:2:06.9}\\
                        }
                                \ex[]{
                               \textit{{kaˈwíriri}}\\
                               \textit{kaˈwí-riri}\\
                               hill-\textsc{loc}\\
                               { `in the hill'}\\
                                `en el cerro' \corpuslink{tx109[01_174-01_197].wav}{LEL tx109:1:17.4}\\
                            }
                                    \ex[]{
                                    \textit{{roˈhàriri}}\\
                                    \textit{roˈhà-riri}\\
                                    oak-\textsc{loc}\\
                                    {`in the oak tree'} \\
                                    `en el encino' \corpuslink{tx152[05_572-06_001].wav}{SFH tx152:5:57.2}\\
                                }
                                        \ex[]{
                                        \textit{{oˈʰkóriri}}\\
                                        \textit{oˈʰkó-riri}\\
                                        pino-\textsc{loc}\\
                                        {`in the pine tree'}\\
                                        `en el pino' \corpuslink{in61[04_074-04_093].wav}{FLP in61:4:07.4}\\
                                    }
                        \pagebreak
                                            \ex[]{
                                           \textit{{kaˈlíriri}}\\
                                           \textit{kaˈlí-riri}\\
                                           house-\textsc{loc}\\
                                            {`in the house'}\\
                                            `en la casa' \corpuslink{tx_muerto[00_234-00_270].wav}{BFL tx\_muerto:0:23.4}\\
                                        }
                                                \ex[]{
                                                \textit{{kiʔoˈrîriri}}\\
                                                \textit{kiʔoˈrî-riri}\\
                                                toasted.corn-\textsc{loc}\\
                                                {‘in the toasted corn (esquiate)’}\\
                                                `in the toasted corn' {< LEL 06 5:127-9/el >}\\
                                            }
                                                    \ex[]{
                                                    \textit{{saʔˈpáriri}}\\
                                                    \textit{saʔˈpá-rare}\\
                                                    meat-\textsc{loc}\\
                                                    { `in the meat'}\\
                                                    `en la carne' {< LEL 06 5:127-9/el >}\\
                                                }
    \z
\z

\citet{miller1996guarijio} describes the cognates of these suffixes in \ili{Mountain Guarijío} as possessing a similar semantic difference: the \ili{Mountain Guarijío} locative \textit{-tʃ͡i/-átʃ͡i,} cognate of the Choguita Rarámuri \textit{-t͡ʃi} suffix, is glossed as ‘where’ and ‘at’ (\citeyear[192--193]{miller1996guarijio}), while the \ili{Mountain Guarijío} \textit{-tére}/\textit{-rere} suffix, with the short allomorph \textit{-te/-re}, the cognate of the Choguita Rarámuri \textit{-rare} suffix, is glossed as ‘below/among’ (\ili{Spanish} \textit{abajo} and \textit{entre}) (\citeyear[287]{miller1996guarijio}). In Choguita Rarámuri, however, the choice of suffix in many cases seems to be lexically conditioned. I have recorded alternative forms of the same bases with the two locative suffixes, and speakers have produced them both spontaneously and when prompted during elicitation. Some examples are provided in (\ref{ex: nouns with two locatives}):

%\textbf{Note: it is possible that the semantic distinction between \textit{-t͡ʃi} }\textbf{and \textit{-rare} }\textbf{has been undergoing bleaching.}

\ea\label{ex: nouns with two locatives}
{Nouns with two locative forms}

\begin{tabular}{llll}
    & \textit{Locative 1} & \textit{Locative 2} & \textit{Gloss} \\
     a.& maˈtá-riri  & mata-ˈtʃ͡í & `in the metate' \\
      & \corpuslink{tx904[00_075-00_118].wav}{GFM tx904:0:07.5} &  \corpuslink{tx68[00_588-01_091].wav}{LEL tx68:0:58.8} & \\
     b.& kuˈpá-riri  & kupa-ˈtʃ͡í & `in the hair' \\
      & < LEL 06 5:127-9/el > &  < LEL 06 5:127-9/el  >& \\
     c.& kiˈmá-riri & kima-ˈtʃ͡í & `in the blanket' \\
      & < LEL 06 5:127-9/el > &  < LEL 06 5:127-9/el > & \\
     d.& kiʔoˈrî-riri (pr.)   & kiʔoˈrî-tʃ͡i  & `in the toasted  \\
      & < LEL 06 5:127/el > &  < LEL 06 5:127-9/el > &  corn (esquiate)'\\
     e.& waˈrî-riri (pr.)  &  waˈrî-tʃ͡i &     `in the basket' \\
      & < LEL 06 5:127/el > &  < LEL 06 5:127-9/el > & \\
\end{tabular}
    \z

In the following pair, however, the choice of locative suffix reflects a semantic distinction between the two suffixes an alienable/inalienable distinction: the noun \textit{saʔˈpá} ‘meat’ means live flesh in (\ref{ex: adessive vs. inessive contrasta}) with the locative \textit{-tʃ͡i} suffix and meat severed from the body in (\ref{ex: adessive vs. inessive contrastb}) with the locative \textit{-rare} suffix; that is, the locative \textit{-t͡ʃi} suffix is used when the figure is located on the surface of the ground, while \textit{-rare} is used when the figure is located inside, or within a space defined by the ground, consistent with the description of these suffixes as adessive and inessive, respectively:

\ea\label{ex: adessive vs. inessive contrast}
{Contrast between adessive and inessive locative forms}

    \ea[]{
    {\textit{taˈmí  naˈmûti  iˈkîli    saʔpaˈtʃ͡í}}\\
    \gll    taˈmí  naˈmûti  iˈkî-li    saʔpá-tʃ͡í\\
            1\textsc{sg.acc}  something  bite-\textsc{pst}  flesh-\textsc{loc}\\
    \glt    ‘Something bit me in the flesh.’ (where \textit{sapa-ˈt͡ʃí} = in the live, human flesh)\\
    \glt    ‘Algo me picó en la carne.’ (donde \textit{sapa-ˈt͡ʃí = en la carne del cuerpo})  < FMF 09 3:60/el >\\
}\label{ex: adessive vs. inessive contrasta}
        \ex[]{
        {\textit{biˈlé    sioˈrí  t͡ʃuˈkú saʔˈpáriri}}\\
        \gll    biˈlé  sioˈrí  t͡ʃuˈkú      saʔˈpá-rare\\
                one fly  stand.four.legs.\textsc{prs}  meat-\textsc{loc} \\
        \glt    ‘There is a fly in the meat.’ (where \textit{saʔpá-riri} = in the severed cow’s meat)\\
        \glt    ‘Hay una mosca en la carne.’ (donde \textit{saʔpá-riri} = en el pedazo de carne de vaca)  < FMF 09 3:60/el >\\
    }\label{ex: adessive vs. inessive contrastb}
    \z
\z

In this case, the different locative suffixes seem to have been lexicalized with specific meanings. There are no other documented examples where other bases display a comparable semantic contrast.\footnote{A similar difference where the same root has an inalienable meaning with one morphological construction and an alienable meaning with another morphological construction has been reported in other \ili{Uto-Aztecan} languages: in \ili{Cora} (\ili{Corachol}), there is a contrast between \textit{n\textsuperscript{y}}\textit{i-we’} ‘my flesh’, and \textit{n\textsuperscript{y}}\textit{i-weʔ-raʔ}, with possessive suffix \textit{-raʔ,} ‘my meat (that I bought)’ \citep[][302]{dakin1991nahuatl}. See §\ref{sec: possessive marking} for details of the behavior of alienable and inalienable nouns in possessive constructions in Choguita Rarámuri.}

In sum, there are some semantic differences between locative \textit{{}-tʃ͡i} and locative \textit{-rare,} and each construction might be selected depending on the semantic characteristics of each base in some cases (‘to be among the trees’ vs. ‘to be on a rock’), but affix distribution seems to be largely a matter of lexical choice in most cases (i.e., there is no clear semantic property of the referent that requires one locative construction versus the other).

In addition to the basic locative meaning exemplified above, the suffix \textit{-rare} appears on a base where the locative construction seems to involve a collective meaning: in (\ref{ex: collective meaning of -rare}), the nominal base \textit{oˈʰkó} ‘pine tree’ plus the suffix \textit{-rare} encodes a group of pine trees:\footnote{This construction is an example of a copular clause headed by a nominal predicate. More information about this type of  clause is provided in §\ref{subsec: copular clauses headed by nominal predicates}.}

\ea\label{ex: collective meaning of -rare}

    {\textbf{\textit{oˈʰkóriri}} \textit{ke    mi  koˈrína  oʔˈná}}\\
    \gll    \textbf{oˈʰkó-rare}    ke    mi  koˈrí-na  oʔˈná\\
            pine.tree-\textsc{loc} \textsc{cop.impf} \textsc{dist} around-\textsc{all}  there  \\
    \glt    ‘There were many pine trees over there.’\\
    \glt    ‘Había muchos pinos allá.’ \corpuslink{in61[04_074-04_093].wav}{FLP in61:4:07.4} \\

\z

Consider, in contrast, two examples where the same root \textit{oˈʰkó} ‘pine tree’ plus the locative \textit{-rare} refers to a single tree; the difference between this form and the ‘collective’ one in (\ref{ex: collective meaning of -rare}) is that \textit{oˈʰkó} ‘pine tree’ is the deictic centre in (\ref{ex:  two examples of -rare}) below, but the figure in (\ref{ex: collective meaning of -rare}) above:

\ea\label{ex:  two examples of -rare}
{Collective reading of \textit{-rare}}
%check tones
    \ea[]{
    \textit{aʔˈlì   es   ta   koˈt͡ʃî   ko   ˈnà   riˈwáli   t͡ʃo  biˈlé   naˈmûti t͡ʃukuˈká  oˈná   riˈpá \textbf{  oˈʰkóriri }wiˈhá}\\
    \gll    aʔˈlì ˈét͡ʃi tá  koˈt͡ʃî=ko  ˈnà  riˈwá-li  t͡ʃo  biˈlé  naˈmûti t͡ʃuku-ˈká oˈná  reˈpá \textbf{oˈʰkó-rare}  wihá-a\\
            and    \textsc{dem}  \textsc{det} dog=\textsc{emph}  \textsc{prox} see-\textsc{pst} also  one  thing be.bent-\textsc{ger} \textsc{dist} up pine-\textsc{loc} hang-\textsc{prog}\\
    \glt    ‘Then the dog saw something up there in the pine tree hanging.’\\
    \glt    ‘Entonces el perro vió algo arriba en el pino colgando.’  \corpuslink{tx177[03_235-03_284].wav}{LEL tx177:3:23.5}\\
}
        \ex[]{
        {\textit{we   soʔˈéka     uˈt͡ʃáli   t͡ʃiˈná   paˈt͡ʃána ˈnà \textbf{oˈʰkóriri}}}\\
        \gll    we   soʔˈé-ka     uˈt͡ʃá-li   t͡ʃiˈná   paˈt͡ʃá-na  ˈnà \textbf{oˈʰkó-rare}\\
                \textsc{int} get.stuck-\textsc{ger} put-\textsc{pst} \textsc{prox} inside-at  \textsc{prox} pine-\textsc{loc}\\
        \glt    ‘He left it stuck over there, inside of the pine tree.’ \\
        \glt    ‘Lo dejó muy bien atorado adentro en el pino.’  \corpuslink{tx84[05_053-05_086].wav}{LEL tx84:05:05.3}\\
    }
    \z
\z

In example (\ref{ex: -rare example 3}) below, \textit{oˈʰkó-riri} ‘pine-\textsc{loc}’ might be interpreted as either meaning that the pine trees are the figure in a deictic center and that the speaker is located at a different place (‘behind the trees’), or that the speaker is in fact the figure located  ‘among, between’ the trees, the deictic centre.

\ea\label{ex: -rare example 3}

    \textit{aʔˈlì   ko   ˈnè ˈmá   biˈlá    ˈnápa ˈnìi oˈʰkóriri ˈnè     wiˈlí    ˈnà   iˈt͡ʃîpika   ba   neˈká   ba}\\
    \gll    aʔˈlì=ko   ˈnè ˈmá     biˈlá    ˈnápa ˈnì oˈʰkóriri ˈnè     wiˈlí    ˈnà   iˈt͡ʃîpika   ba   neˈká   ba\\
            and=\textsc{emph}  \textsc{1sg.nom} already  really    \textsc{sub} \textsc{emph}  pine-\textsc{loc} \textsc{1sg.nom}   stand.\textsc{sg}  \textsc{prox}  hide-\textsc{ger} \textsc{cl} see-\textsc{ger} \textsc{cl}\\
    \glt    ‘So then I stood behind/between the trees, hiding, watching.’\\
    \glt    ‘Entonces ya me paré entre/atrás de los pinos, escondida, viendo.’ \corpuslink{tx223[01_293-01_373].wav}{LEL tx223:1:29.3}\\

\z

%cross-reference here to spatial adverb section
Locative case also encodes spatial relations relative to specific referents. In these cases, exemplified in (\ref{ex: locative case marked in head noun of NP}), locative case is marked on the head noun of the nominal phrase:

\ea\label{ex: locative case marked in head noun of NP}

    \ea[]{
    \textit{naˈsîpa  iˈjétʃ͡i}\\
    \gll    naˈsîpa  iˈjé-tʃ͡i\\
            middle  door-\textsc{loc}\\
    \glt    ‘in the middle of the door’\\
    \glt    ‘en medio de la puerta’ < FMF 09 3:27/el >\\
}
        \ex[]{
        \textit{suˈwè t͡ʃiˈnírari}\\
        \gll    suˈwé  t͡ʃiˈní-rari\\
                edge  cloth-\textsc{loc}\\
        \glt    ‘on the edge of the cloth’\\
        \glt    ‘por la orilla de la tela’ < FMF 09 3:27/el >\\
    }
    \z
\z

%this may be included in topnyms?
\hspace*{-2pt}Finally, locatives are found in many Choguita Rarámuri toponyms. Place names are derived from combining locative suffixes with nominal roots. Some examples of toponyms derived with these suffixes are provided in (\ref{ex: locatives in toponyms}) (all examples are also documented in \citet{brambila1976diccionario}, and the source information from this reference is indicated under each example with the initial ``B'', year (1976) and page where the data was found).

\pagebreak

\ea\label{ex: locatives in toponyms}
{Locatives in toponyms}\\

\begin{tabular}{lllll}
    & \textit{Toponym} & \textit{Gloss} & \textit{Base form} & \textit{Gloss} \\
     a.& wiˈt͡ʃatʃ͡i & `town of Guichachi' &  wiˈtʃ͡a &  ‘thorny bush' \\
      & & & &  [B 1976:59]\\
     b.& waˈt͡ʃot͡i & `town of Guachochi' & waˈt͡ʃo & `heron' \\
     c.& kuˈsarare & `town of Cusárare' & kuˈsa & `eagle' \\
     d.& guˈmisat͡ʃi & `town of Gumísachi'  & guˈmisari & `crest' \\
      & & & & [B 1976:191]\\
     e.& aboˈreat͡ʃi & `town of Aboréachi' & aboˈri & `juniper tree' \\
     & & & &   (Sp. \textit{táscate})\\
      & & & &   [B 1976:2]\\
     f.& ˈtonat͡ʃi & `town of Tónachi'  & ˈtona & `pillar' \\
      & & & &  [B 1976:557]\\
     g.& okoˈt͡ʃît͡ʃi& `town of Okochichi' & o-koˈt͡ʃî & `\textsc{pl-}dog' \\
      & & & & [B 1976:39]\\
     h.& okoˈwit͡ʃi & `town of Okowíchi' & okoˈwi & `kind of owl' \\
      & & & & [B 1976:401]\\
     i.& naˈrarat͡ʃi &  `town of Narárachi' & naˈlà-ra & cry-\textsc{nmlz} \\
     j.& soˈgit͡ʃi &  `town of Sisoguichi' & soˈgi &  `a type of plant' \\
      & & & &  [B 1976:531] \\
     k.& uˈreat͡ʃi &  `town of Uréachi' & uˈre & `ash tree'  \\
     & & & &  (Sp. \textit{fresno})\\
      & & & &  [B 1976:573]\\
     l.&  kaˈnot͡ʃi &  `town of Ganochi'  & kaˈno & `giant' \\
     m.& wisaˈrot͡ʃi &  `town of Guisaróchi' & wisaˈro & `poplar' \\
\end{tabular}
    \z

\section{Possessive marking}
\label{sec: possessive marking}

Possessive marking in Choguita Rarámuri nouns encodes possessors and meronymic (or part-whole) relationships. Other semantically related relations, such as the relation ‘made of’, are encoded through analytic constructions in the language. The function and syntactic properties of possessive constructions are addressed in §\ref{sec: complex noun phrases}.

\subsection{Alienable and inalienable possession}
\label{subsec: possessive}

Both inalienable (kinship and body-part terms) and alienable nouns may register a possessor with the stress-shifting possessive \textit{-lâ} suffix.\footnote{A suffix \textit{**-ra}, reconstructed for \ili{Proto-Sonoran} and possibly for \ili{Proto-Uto-Aztecan} \citep[][299]{dakin1991nahuatl}, has cognate forms in \ili{Tubar} (\citealt{Lionnet-1978}), Lowland \ili{(River) Guarijío} (\citealt{barreras1988posesion}), Highland \ili{(Mountain) Guarijío} \parencite{miller1996guarijio}, \ili{Cora} \parencite{dakin1991nahuatl}, and other Tarahumara varieties (\ili{Norogachi Rarámuri} (\citealt{brambila1953gramatica}, \citealt{lionnet1972elementos})). \citet{lionnet1972elementos} describes the suffix \textit{-ra} as a ‘determiner’ (\textit{“determinado”}), though no definitions of this suffix is provided (1972:18).}

Following the head-marking pattern of the language, the suffix \textit{-lâ} attaches to the head noun in a possessive nominal phrase, as shown in (\ref{ex: posessive suffix 1}):

\ea\label{ex: posessive suffix 1}
{Possessive marking}

    {\textit{ˈnè    waˈsála}}\\
    \gll    ˈnè   wasá-lâ\\
            \textsc{1sg.nom}  cultivation.land-\textsc{poss} \\
    \glt    ‘my cultivation land'\\
    \glt    ‘mi tierra de cultivo’  < BFL 09 1:61/el > \\

\z

Further examples of possessive constructions are provided in (\ref{ex: possessive examples 2}) (as other stress-shifting suffixes, the lexical tone of the suffix surfaces when stressed, (\ref{ex: possessive examples 2f})):

\ea\label{ex: possessive examples 2}
{Possessive marking}\\

    \ea[]{
    \textit{waˈsîla}\\
    \textit{waˈsî-la}\\
    mother.in.law-\textsc{poss}\\
    ‘their mother in law’\\
    `su suegra' {< BFL 06 5:127/el >}\\
}\label{ex: possessive examples 2a}
        \ex[]{
        \textit{ˈkútʃ͡ala}\\
        \textit{ˈkútʃ͡a-la}\\
        children-\textsc{poss}\\
        ‘their little ones, children'\\
        `sus hijos' \corpuslink{tx43[12_398-12_432].wav}{SFH tx43:12:39.8}\\
    }\label{ex: possessive examples 2b}
            \ex[]{
            \textit{{iˈjêla}}\\
            \textit{iˈjê-la}\\
            mother-\textsc{poss}\\
            `their mother'\\
            `su mamá' \corpuslink{tx475[07_114-07_128].wav}{SFH tx475:7:11.4}\\
        }\label{ex: possessive examples 2c}
                \ex[]{
                \textit{{aliˈwâla}}\\
                \textit{aliˈwâ-la}\\
                soul-\textsc{poss}\\
                `their soul'\\
                `su alma' \corpuslink{tx5[00_307-00_350].wav}{LEL tx5:0:30.7}\\
            }\label{ex: possessive examples 2d}
                    \ex[]{
                    \textit{{apaˈnêrala}}\\
                    \textit{apaˈnêra-la}\\
                    wife-\textsc{poss}\\
                    `their wife'\\
                    `su esposa' \corpuslink{tx5[00_409-00_428].wav}{LEL tx5:0:40.9}\\
                }\label{ex: possessive examples 2e}
                        \ex[]{
                        \textit{{onoˈlâ}}\\
                        \textit{ono-ˈlâ}\\
                        father.male.ego-\textsc{poss}\\
                        `their father (male ego)'\\
                        `su papá de él' \corpuslink{tx12[02_180-02_197].wav}{SFH tx12:2:18.0}\\
                    }\label{ex: possessive examples 2f}
                                \ex[]{
                                \textit{{kuˈrít͡ʃala}}\\
                                \textit{kuˈrít͡ʃa-la}\\
                                uncle-\textsc{poss}\\
                                `their uncle (husband of father's sister)'\\
                                `su tío (esposo de la hermana del papá)' {< FLP in61(510)/in >}\\
                            }\label{ex: possessive examples 2g}
                                    \ex[]{
                                    \textit{{kuˈmût͡ʃala}}\\
                                    \textit{kuˈmût͡ʃa-la}\\
                                    paternal.older.uncle-\textsc{poss}\\
                                     ‘their paternal older uncle'\\
                                    `su tío paterno mayor (que el papá)' \corpuslink{in485[00_002-00_039].wav}{SFH in485:0:00.2}\\
                                }\label{ex: possessive examples 2h}
                                        \ex[]{
                                         \textit{moʔoˈlîla}\\
                                         \textit{moʔoˈlî-la}\\
                                         daughter.in.law-\textsc{poss}\\
                                        `their daughter in law'\\
                                        `su nuera' \corpuslink{tx19[05_298-05_343].wav}{LEL tx19:5:29.8}\\
                                    }\label{ex: possessive examples 2i}
                                            \ex[]{
                                            \textit{{aʔkaˈlâ}}\\
                                            \textit{aʔka-ˈlâ}\\
                                            sandal-\textsc{poss}\\
                                            `their sandals'\\
                                            `sus huaraches' {< BFL 09 1:60/el >}\\
                                            \largerpage[2]
                                        }\label{ex: possessive examples 2j}
                                                \ex[]{
                                                \textit{boʔeˈlâ}\\
                                                \textit{boʔe-ˈlâ}\\
                                                road-\textsc{poss}\\
                                                `their road'\\
                                                `su camino' {< BFL 06 5:128/el >}\\
                                            }\label{ex: possessive examples 2k}
%                                        \pagebreak
                                                    \ex[]{
                                                    \textit{{siˈpútʃ͡ala}}\\
                                                    \textit{siˈpútʃ͡a-la}\\
                                                skirt-\textsc{poss}\\
                                                `their skirt'\\
                                                    `su falda' {< BFL 09 1:60/el >}\\
                                                }\label{ex: possessive examples 2l}
                                                        \ex[]{
                                                        \textit{{naˈpátʃ͡ala}}\\
                                                        \textit{naˈpátʃ͡a-la}\\
                                                        shirt-\textsc{poss}\\
                                                        `their shirt'\\
                                                        `su blusa' {< FMF 09 3:32/el >}\\
                                                    }\label{ex: possessive examples 2m}
                                                            \ex[]{
                                                            \textit{{baʔˈwíla}}\\
                                                            \textit{baʔˈwí-la}\\
                                                            water-\textsc{poss}\\
                                                            `their water'\\
                                                            `su agua' \corpuslink{tx68[00_404-00_462].wav}{LEL tx68:0:40.4}\\
                                                        }\label{ex: possessive examples 2n}
                                                                \ex[]{
                                                                \textit{{kaˈlíla}}\\
                                                                \textit{kaˈlí-la}\\
                                                                house-\textsc{poss}\\
                                                                `their house'\\
                                                                `su casa' {< SFH in61(695)/in >}\\
                                                            }\label{ex: possessive examples 2o}
                                                            \ex[]{
                                                            \textit{{ˈtéla}}\\
                                                            \textit{ˈté-la}\\
                                                            lice-\textsc{poss}\\
                                                            `their lice'\\
                                                                   `sus piojos' {< FMF 09 3:32/el >}\\
                                                                }\label{ex: possessive examples 2p}
    \z
\z

The examples above show that the suffix \textit{-lâ} attaches to inalienable nouns (\ref{ex: possessive examples 2}a--i) as well as alienable nouns (\ref{ex: possessive examples 2}j--p). Alienable nouns are characterized by being able to appear in non-possessive constructions, while inalienable nouns are exclusively found with possessive marking.

\largerpage
Choguita Rarámuri has a second possessive construction in which nominal roots are marked by a stress-neutral suffix \textit{-wa} followed by \textit{-lâ} (schematically, V\textit{-wa-lâ}). \citet{haugen2017derived} reconstructs \textit{*-wa} for \ili{Proto-Uto-Aztecan} as a nominal suffix of attributive possession (``possessed thing") (\citeyear[172]{haugen2017derived}) (see also \citealt{langacker1977uto} for an alternative proposal).\footnote{\citet{haugen2017derived}
    makes a distinction between possession in the nominal domain and possession in the verbal domain, a distinction largely equivalent to one of attributive possession and predicative posession, respectively. Both types of posession constructions are documented across the \ili{Uto-Aztecan} language family. \citet{langacker1977uto} reconstructs \textit{*-wa} as a denominal verb suffix.
} Reflexes of this reconstructed suffix are found in the \ili{Numic}, \ili{Takic}, \ili{Tepiman}, \ili{Taracahitan}, and \ili{Aztecan} branches of \ili{Uto-Aztecan} \citep[][186]{haugen2017derived}.

Examples of these ``double possessive'' constructions are given in (\ref{ex: double possessive construction}):\footnote{The meaning of the possessed forms is compositional, except for the form \textit{waˈsá-wa-la} in (\ref{ex: double possessive construction}b) which can be interpreted as `someone's hill', but is more generally interpreted as `someone's homeland'. In the case of \textit{baˈt͡ʃâ-wa-la} in (\ref{ex: double possessive construction}h), the meaning is `someone's ancestors' (lit. `their first ones').}

\ea\label{ex: double possessive construction}
{Double possessive constructions}
\setlength{\tabcolsep}{4pt}
\begin{tabular}{lllll}
    & \textit{Double posessive} & \textit{Stem} & \textit{Gloss} & \textit{Source}\\
     a.& waˈsá-wa-la  & waˈsá & `sowing field’ & < BFL 06 5:127/el >\\
     b.& kaˈwì-wa-la  &  kaˈwì &  `hill' & < BFL 06 5:127/el >\\
     c.& suˈnù-wa-la & suˈnù &  `corn' & < BFL 06 5:127/el > \\
     d.& ripuˈt͡ʃí-wa-la & ripuˈt͡ʃí  & `flea'& < FMF 09 3:32/el >\\
     e.& koroˈká-wa-la  &  koroˈká &     `necklace' & < BFL 09 1:61/el > \\
     f.& riˈmê-wa-la  &  riˈmê & `tortilla' & < SFH 06 6:73-75/el > \\
      g.& paˈpâ-wa-la  &  paˈpâ &     `father' & < LEL tx84(7)/tx >\\
       h.& baˈt͡ʃâ-wa-la & baˈt͡ʃâ & `first'& \corpuslink{tx43[04_056-04_120].wav}{SFH tx43:4:05.6} \\
      i.& ripuˈrá-wa-la &  ripuˈrá &     `axe'   & < LEL 2006.11.17/el > \\
       j.& kiriˈbá-wa-la  &  kiriˈbá &     `quelites'  & < SFH 09 3:86/el > \\
        k.& saˈté-wa-la &  saˈté &    `field'  & < BFL 09 3:70/el >\\
\end{tabular}
    \z

\citet{miller1996guarijio} describes the cognate forms of both the \textit{-wa} and \textit{-lâ} suffixes in \ili{Mountain Guarijío} as `possessive' and ‘absolutive’ (non-possessed), respectively (\citeyear[250--251]{miller1996guarijio}). The \ili{Mountain Guarijío} possessive \textit{-wa} suffix is described as being stress-shifting (causing a change of stress in the base), and the \ili{Mountain Guarijío} absolutive \textit{-la} suffix as stress-neutral, in contrast to the prosodic properties of the cognate forms in Choguita Rarámuri. While the suffix \textit{-wa} is used exclusively in possessive constructions in \ili{Mountain Guarijío}, the suffix \textit{-la} is also used in this language in partitive and what Miller labels ‘determinative’ constructions. While in \ili{Mountain Guarijío} \textit{-wa} may be used as the sole marker of possessors, in Choguita Rarámuri this formative is always followed by \textit{-la} as in the examples shown above.

For some speakers of Choguita Rarámuri the double possessive construction alternates with the \textit{{}-lâ} possessive construction to mark a plural-singular distinction, respectively, with count nouns. This is exemplified in (\ref{ex: single possessive vs. double possessive}) (all data from < BFL 11/07/09/el >):\footnote{In these examples, the pronominal forms exhibit variation between the full and reduced forms with no apparent grammatical conditioning, e.g., \textit{\textbf{neˈhê} iˈjê-la} \corpuslink{tx475[06_173-06_214].wav}{SFH tx475:06:17.3} vs. \textit{\textbf{ˈnè} iˈjê-la} \corpuslink{tx1[00_555-01_020].wav}{BFL tx1:00:55.5} `my mother'. This kind of variation is, to the best of my knowledge, not reported in other Rarámuri varieties.}

%\pagebreak

\ea\label{ex: single possessive vs. double possessive}
{A singular-plural distinction of possessed nouns}

\begin{tabular}{lll}
    & \textit{Form} & \textit{Gloss} \\
     a.& ˈnè waˈrî-la  & `my basket'\\
     b.& ˈnè waˈrî-wa-la  &  `my baskets'\\
     c.& ˈnè sikoˈrí-la &  `my pot' \\
     d.& ˈnè sikoˈrí-wa-la & `my pots'\\
     e.& ˈnè siˈpút͡ʃa-la & `my skirt'\\
     f.& ˈnè siˈpút͡ʃa-wa-la & `my skirts'\\
     g.& ˈnè naˈpát͡ʃa-la & `my shirt'\\
     h.& ˈnè naˈpát͡ʃa-wa-la & `my shirts'\\
\end{tabular}
    \z

It is not clear, however, that this singular-plural distinction is found with all count nouns. In (\ref{ex: optionality between single possesisve and double poasessive}), the possessive construction is ambiguous between a singular and a plural possessum. The base in (\ref{ex: optionality between single possesisve and double poasessivea}) lacks a possessive form with possessive \textit{-la}. In (\ref{ex: optionality between single possesisve and double poasessive}b--c), on the other hand, the \textit{-wa-lâ,} ‘double possessive’, construction in Choguita Rarámuri stands in variation with single-marked possessive nouns in certain forms. In this case, there are no semantic differences between the two alternative constructions.

\ea\label{ex: optionality between single possesisve and double poasessive}
{Optionality between single and double possessive constructions}

    \ea[]{
    {\textit{ˈnè     riˈmêwala}}\\
    \gll    ˈnè    reˈmê-wa-lâ\\
            1\textsc{sg.nom} tortillas-\textsc{poss-poss}\\
    \glt    `my tortillas'\\
    \glt    `mis tortillas' < BFL 11/07/09/el >\\
}\label{ex: optionality between single possesisve and double poasessivea}
        \ex[]{
        {\textit{ˈnè  waˈsála}}\\
        \gll    ˈnè  waˈsá-lâ\\
                \textsc{1sg.nom} sowing.field-\textsc{poss}\\
        \glt    ‘my sowing fields’\\
        \glt    `mis tierras de cultivo'  < LEL 06 5:127-9/el, BFL 09 1:60/el >\\
    }\label{ex: optionality between single possesisve and double poasessiveb}
            \ex[]{
            {\textit{ˈnè  waˈsáwala}}\\
            \gll    ˈnè waˈsá-wa-lâ\\
                    \textsc{1sg.nom} sowing.field-\textsc{poss-poss}\\
            \glt    ‘my sowing fields’\\
            \glt    `mis tierras de cultivo'    < LEL 06 5:127-9/el, BFL 09 1:60/el > \\
        }\label{ex: optionality between single possesisve and double poasessivec}
    \z
\z

For some nouns the only available possessive construction is the double possessive \textit{-wa-lâ}. This is exemplified in (\ref{ex: only double possessives available}): these nouns, spontaneously given with the \textit{-wa-lâ} ‘double possessive’ sequence, were rejected when prompted with the \textit{-lâ} possessive construction (< BFL 09 1:61/el >).

\ea\label{ex: only double possessives available}
{Nouns with obligatory double possessive constructions}\\

\begin{tabular}{llll}
        & \textit{Form} &  \textit{Ungrammatical} & \textit{Gloss}\\
     a.& ˈsaˈté-wa-la & *saˈté-la & `their land'\\
     b.& kaˈwì-wa-la  & *kaˈwì-la &   `their hill, homeland'\\
     c.& bahˈtâlu-wa-la &  *bahˈtâli-la & `their corn beer' \\
     d.& suˈnù-wa-la & *suˈnù-la  & `their corn'\\
\end{tabular}
    \z

For other nouns, the only available possessive construction is the one involving a single possessive marker (\textit{-lâ}), with the `double possessive' construction \textit{-wa-lâ} being unavailable. All documented cases of nouns lacking a `double possessive' construction are body-part terms, a closed class of inalienable nouns. Relevant examples are shown in (\ref{ex: only single possessive available}),\footnote{I represent body part roots as bound, since they require possessive marking as inalienable nouns.} with their spontaneously produced possessive form, as well as the ungrammatical double possessive construction.

\ea\label{ex: only single possessive available}
{Possessive forms of body-part nouns}\\
\setlength{\tabcolsep}{3pt}
\begin{tabular}{lllll}
        & \textit{Form} & \textit{Ungrammatical} & \textit{Gloss} & \textit{Source}\\
     a.& roˈnô-la & *roˈnô-wa-la & `their foot' & < BFL 09 1:60/el >\\
     b.& kaˈsî-la  & *kaˈsî-wa-la &   `their leg'& < BFL 09 1:60/el >\\
     c.& siˈkâ-la &  *siˈkâ-wa-la & `their arm\slash hand'  & < BFL 09 1:60/el >\\
     d.& kuˈpá-la & *kuˈpá-wa-la  & `their hair' & < BFL 11/07/09/el >\\
     e.& buˈsí-la & *buˈsí-wa-la  & `their eye' & < BFL 11/07/09/el >\\
     f.& t͡ʃoˈkówa-la & *t͡ʃoˈkówa-wa-la &  `their knee' & < BFL 11/09/el >\\
     g.& waˈt͡ʃíka-la & *waˈt͡ʃíka-wa-la  & `their rib' & < BFL 11/09/el >\\
\end{tabular}
    \z

The exception to this pattern is found with body part nouns that may have alternative readings as either a collection of entities with similar properties (e.g., teeth) or as discrete entities (e.g., a single tooth). For these nouns, the \textit{-lâ} construction refers to the possession of a single, discrete entity (e.g., a single finger, tooth, finger nail, etc.), and the double possessive construction is used when making reference to the possession of multiple fingers, teeth, etc.\footnote{Body part terms referring to body parts that come in pairs, such as \textit{buˈsí-} ‘eye’, \textit{kaˈsî-} ‘leg’, \textit{roˈnô-} ‘foot’, \textit{tʃ͡oˈkówa-} ‘knee’, or \textit{siˈkâ-} ‘hand’, do not have a singular-plural (dual) possessive form alternation as found in (\ref{ex: sg - pl distinctions with possessives}). Similarly, the term for ‘ribs’, \textit{waˈt͡ʃíka-}, while composed of multiple, discrete parts, does not have a singular-plural alternation in the possessive (though this might be due to the fact that ribs are all linked or that they are internal (K. Dakin, p.c.)).}  These forms are exemplified in (\ref{ex: sg - pl distinctions with possessives}) (all data from < BFL 11/07/09/el >):

\ea\label{ex: sg - pl distinctions with possessives}
{Singular vs. plural distinction in possessed body-part terms}\\

\begin{tabular}{lll}
        & \textit{Form} &  \textit{Gloss}\\
     a.& ˈnè maˈkúsa-la &  `my finger'\\
     b.& ˈnè maˈkúʃu-a-la  & `my fingers'\\
     c.& ˈnè maˈtáka-la &  `my molar' \\
     d.& ˈnè maˈták-wa-la & `my molars'\\
     e.& ˈnè ʃuˈtû-la & `my finger nail'\\
     f.& ˈnè ʃuˈtû-wa-la & `my finger nails'\\
\end{tabular}
    \z

Some kinship terms are also found with the single vs. double possessive construction encoding a singular-plural distinction. These are exemplified in (\ref{ex: sg - pl distinctions with kinship possessives}):

\ea\label{ex: sg - pl distinctions with kinship possessives}
{Singular vs. plural distinction in posessed kinship terms}

    \ea[]{
    \textit{{ˈnè kuˈrít͡ʃala}}\\
    \gll    ˈnè kuˈrít͡ʃa-la\\
            1\textsc{sg.nom} uncle-\textsc{poss}\\
    \glt     `my uncle (husband of father's sister)'\\
    \glt    `mi tío (esposo de la hermana de mi papá)'\\
}
        \ex[]{
        \textit{{jaˈdîra kuˈrít͡ʃuwala}}\\
        \gll    jaˈdîra kuˈrít͡ʃa-wa-la\\
                Yadira uncle-\textsc{poss-poss}\\
        \glt    {`Yadira's uncles (`husband of father's sisters)'}\\
        \glt    `los tíos de Yadira (esposo de la hermana del papá)'\\
}
        \ex[]{
        {\textit{ˈnè kuˈmût͡ʃala}}\\
        \gll    ˈnè kuˈmût͡ʃa-la\\
                1\textsc{sg.nom} paternal.uncle-\textsc{poss}\\
        \glt    {`my paternal uncle'}\\
        \glt    `mi tío paterno'\\
    }
            \ex[]{
            {\textit{neˈhê kuˈmût͡ʃu-wa-la}}\\
            \gll    neˈhê kuˈmût͡ʃi-wa-la\\
                    1\textsc{sg.nom} paternal.uncle-\textsc{poss-poss}\\
            \glt    {`my paternal uncles'}\\
            \glt    `mis tíos paternos'\\
    }
    \z
\z

Other kinship terms, however, do not display this alternation, e.g., \textit{uˈmû-a-la} ambiguously means ‘his/her great grandfather' or `his/her great grandfathers’ < FLP in61(116)/in, BFL 11/07/09/el >.\footnote{Another kinship term, \textit{paˈpâ-wa-la} ‘parents’, in (\ref{ex: double possessive construction}i), is a borrowed term (from \ili{Spanish} \textit{papá} `father') that refers to the two parents.} Thus, the distinction between singular vs. plural possessum associated to a single possessive marker vs. two, respectively, seems to be a matter of lexical choice.\footnote{There is at least one example where a singular/plural distinction in terms of possessors and/or possessed referents is encoded through two possessive constructions involving a concatenation of two distinct formatives: the noun \textit{muˈkî}, `woman', can be marked as possessed with a \textit{-a-lâ} suffix sequence (\textit{muˈkî-a-la}) or with a \textit{-wa-lâ} suffix sequence (\textit{muˈkî-wa-la}). The two forms were given with different interpretations, with either a singular possessor and possessum in the former case (\textit{muˈkî-a-la} `his wife') and a plural possessor and possessum in the latter (\textit{muˈkî-wa-la} `their wives') (<BFL 09 4:69/el>). This case also suggests a lexicalization process, given that these possessed forms are exclusively used when the interpretation of the noun \textit{muˈkî} is `wife'.}

Similar to the pattern attested in Choguita Rarámuri, in Lowland \ili{(River) Guarijío}, the cognate of the \textit{-la} suffix (also \textit{-la}) is found in possessive and non-possessive constructions, and the cognate of the \textit{-wa} suffix is used in addition to \textit{-la} to mark optional possession (``mediated possession'') with nouns that are not body part terms, kinship terms or abstract terms (\citealt{barreras1988posesion}, cited in \citealt[][301]{dakin1991nahuatl}. For \ili{Mountain Guarijío}, \citet{miller1996guarijio} describes that possessed nouns marked with the \textit{-wa-la} suffix sequence are generally interpreted as plural, though he notes singular readings are also available with this marking pattern (but no examples are provided).

There are inalienable nouns in possessive constructions that are marked with a \textit{-tʃ͡í} suffix, homophonous to the locative  \textit{-tʃ͡í} suffix described in §\ref{subsec: locative case}, which are added in non-locative contexts. The body-part terms exemplified below are either directly marked with suffix \textit{-t͡ʃí} (\ref{ex: posessives with -chia}) or attach this suffix to a base with suffix \textit{-lâ} (\ref{ex: posessives with -chi}b--c):

\ea\label{ex: posessives with -chi}
{Posessive constructions with suffix \textit{-tʃ͡i}}\\

    \ea[]{
    {\textit{ˈnè    t͡ʃa    t͡ʃoˈkóbitʃ͡i oʔˈkô}}\\
    \gll    ˈnè   t͡ʃa  t͡ʃoˈkówa-tʃ͡i oʔˈkô \\
            \textsc{1sg.nom} ugly  knees-\textsc{loc} hurt\\
    \glt    ‘My knees hurt.’\\
    \glt    ‘Me duelen mis rodillas.’ < BFL 09 1:60/el >\\
}\label{ex: posessives with -chia}
        \ex[]{
        {\textit{niˈhê  kaˈwìwalatʃ͡i}}\\
        \gll ˈnèˈhê  kaˈwì-wa-la-tʃ͡i\\
            \textsc{1sg.nom} hill-\textsc{poss-poss-loc}\\
        \glt    ‘my hill’\\
        \glt    ‘mi tierra’ < BFL 09 1:60/el >\\
    }\label{ex: posessives with -chib}
            \ex[]{
            {\textit{ˈnè    t͡ʃa  t͡ʃaˈmékalatʃ͡i}\footnote{The prompted form \textit{*t͡ʃaˈmeka-wa-la} was rejected, but \textit{t͡ʃaˈmék-wa-la,} with root-final vowel deletion, was accepted as grammatical (< BFL 09 1:61/el >).} \textit{oʔˈkô}}\\
            \gll    ˈnè   t͡ʃa  t͡ʃaˈméka-la-tʃ͡i oʔˈkô\\
                    \textsc{1sg.nom} ugly  tongue-\textsc{poss-loc} hurt\\
            \glt    ‘My tongue hurts.’\\
            \glt    ‘Me duele mi lengua.’ < BFL 09 1:60/el >\\
        }\label{ex: posessives with -chic}
    \z
\z

There are only a few examples with the \textit{-tʃ͡i} suffix with the possessive meaning, suggesting this construction is unproductive. Some of the nouns modified with this suffix in these contexts refer to space or body-parts as possible locations (e.g., for pain), suggesting that these constructions involve a frozen and reanalyzed locative (as in \ili{French} noun phrases with \textit{chez} and \textit{à}), as suggested by Johanna Nichols (p.c.).

The possessive construction induces truncation of the base noun in at least one construction. In (\ref{ex: truncation in genitive-possessive}), the root \textit{koroˈká}, ‘necklace’, is shortened to \textit{koˈro} when marked with the possessive:\footnote{This resembles a noun truncation process in body part noun incorporation constructions, described in detail in §\ref{subsec: body-part incorporation}.}

\ea\label{ex: truncation in genitive-possessive}
{Truncation in possessive construction}\\

    {\textit{ˈnè  koˈróala}}\\
    \gll    ˈnè koroˈká-wa-la\\
            \textsc{1sg.nom} necklace\textsc{-poss-poss} \\
    \glt    ‘my necklace’\\
    \glt    ‘mi collar’ < BFL 09 1:61/el > \\

\z

Finally, in addition to the head-marking strategy for marking possession, Choguita Rarámuri has an alternative way of expressing a possession relationship through an appositional construction with \textit{ˈníwa}, which may be used predicatively as a verb meaning `to have', but which behaves as a grammatically specialized possessive noun in possessive constructions (glossed below as `have'). More details of this syntactic construction are provided in §\ref{subsubsec: appositive possessive constructions}.

\subsection{Meronymic (part-whole) relationships}
\label{subsec: meronymic}

Meronymic or part-whole relationships are also expressed morphologically with the possessive \textit{-la} suffix marked on the possessum. The forms in (\ref{ex: genitive-meronymic examples}) exemplify the meronymic construction, where the possessor precedes the possessed noun:

\ea\label{ex: genitive-meronymic examples}
{Meronymic relations}\\

    \ea[]{
    {\textit{t͡ʃomaˈt͡ʃí   boʔˈwâla}}\\
    \gll    t͡ʃoma-ˈt͡ʃí  boʔˈwâ-la\\
            nose-\textsc{loc}    fur/hair-\textsc{poss} \\
    \glt    ‘the hairs of the nose’\\
    \glt    ‘los pelos de la nariz’ < BFL 09 1:33/el >  \\
}
        \ex[]{
        {\textit{kuaˈdêrno   ˈôhala}}\\
        \gll    kuaˈdêrno    ˈôha-la\\
                notebook    sheet-\textsc{poss}   \\
        \glt    ‘the sheets of the notebook’\\
        \glt    ‘las hojas del cuaderno’ < BFL 09 1:33/el >  \\
    }
    \z
\z

Further examples of the meronymic construction are given in (\ref{ex: genitive-meronymic examples 2}).

%\pagebreak

\ea\label{ex: genitive-meronymic examples 2}
{Meronymic possessive marking}\\

\begin{tabular}{llll}
        & \textit{Form} &  \textit{Gloss} & \textit{Source}\\
     a.& roˈnô-la &  `X's feet' & < BFL 06 5:127/el >\\
     b.& ˈkaˈsî-la  & `X's legs' & < BFL 06 5:127/el >\\
     c.& siˈkâ-la &  `X's hand/arms' & < BFL 06 5:128/el >\\
     d.& aliˈwâ-la & `X's soul' & < BFL muerto(6)/tx >\\
     e.& saʔˈpá-la & ‘X’s flesh’ & < MGD 06 6:73/el >\\
     f.& kuˈpá-la & ‘X’s hair’ & < LEL 06 5:127-7/el >\\
     g.& boʔˈâ-la & ‘X’s feathers/fur’ & < LEL 06 5:127-56/el >\\
     h.& sono-ˈlâ & ‘X’s lungs’ &  <MGD 06 1:107/el >\\
     i.& suˈrá-la & ‘X’s heart/chest’ & <MGD 06 1:107/el >\\
     j.& t͡ʃoˈkóba-la & ‘X’s knees’ & < MGD 06 1:107/el >\\
     k.& sitoˈká-la & ‘X’s elbow/arm’ & < MGD 06 1:107/el >\\
     l.& naˈkâ-la & ‘X’s ears’ & < MGD 06 1:107/el >\\
     m.& raˈníka-la & ‘X’s heel’ & < MGD 06 1:107/el >\\
     n.& noˈká-la & ‘X’s temples’ & < MGD 06 1:107/el >\\
     o.& koa-ˈlâ & ‘X’s forehead’ & < MGD 06 1:107/el >\\
     p.& riniˈbá-la & ‘X’s jaw' & < MGD 06 1:107/el >\\
\end{tabular}
    \z

As examples (\ref{ex: genitive-meronymic examples 2}b) above and (\ref{ex: genitive with inanimate noun}) below show, the meronymic construction is found with body-part terms used to refer to constituent parts of inanimate nouns:

%\pagebreak

\ea\label{ex: genitive with inanimate noun}
{Meronymic construction with inanimate noun}\\

    {\textit{ˈmêsa   roˈnôla  ko  kaˈpòlo}}\\
    \gll    ˈmêsa  roˈnô-la=ko  kaˈpò-li\\
            table  leg-\textsc{poss}=\textsc{emph} break-\textsc{pst}  \\
    \glt    ‘The table’s leg broke.’\\
    \glt    ‘La pata de la mesa se rompió.’ < BFL 09 1:33/el >  \\

\z

In (\ref{ex: recursive meronymic construction}a--b), both nouns are marked with the possessive suffix, since the whole is a body part term, in turn the part of another whole.

\ea\label{ex: recursive meronymic construction}
{Recursive possessive marking}\\

    \ea[]{
    {\textit{t͡ʃuʔaˈlâ  saʔˈpála}}\\
    \gll    t͡ʃuʔa-ˈlâ  saʔˈpá-la\\
            face-\textsc{poss} flesh-\textsc{poss}\\
    \glt    ‘the flesh from the face’\\
    \glt    ‘la carne de la cara’  < BFL 09 1:33/el >  \\
}
        \ex[]{
        {\textit{anaˈlâ    buˈʔâla}}\\
        \gll    ana-ˈlâ    buˈʔâ-la\\
                wing-\textsc{poss} hair/fuzz/feather-\textsc{poss}\\
        \glt    ‘the feathers of the wings’\\
        \glt    ‘las plumas de las alas’  < BFL 09 1:33/el >  \\
    }
            \ex[]{
            {\textit{ˈnè  sikaˈtʃ͡í  buˈʔâla      wiˈkótuko}}\\
            \gll    ˈnè   sika-ˈtʃ͡í  buˈʔâ-la      wiˈkóti-ki\\
                    \textsc{1sg.nom} arm-\textsc{loc}  hair/fuzz/feather-\textsc{poss} burn-\textsc{pst.ego}\\
            \glt    ‘I burned the hairs of my arm.’\\
            \glt    ‘Me quemé los pelos del brazo.’  < BFL 09 1:33/el > \\
        }

    \z
\z

There are no documented examples where this construction would encode other semantic roles (such as beneficiaries). Also, there are no documented meronymic constructions with the \textit{-wa-lâ} sequence found in forms marking a possessor.
%Thus, this construction seems to be exclusive to possessive constructions. -?

\section{Deverbal nouns}
\label{sec: deverbal nouns}

Choguita Rarámuri possesses several morphological means for deriving nominal stems from verbal and other roots. This section is devoted to describing in detail each of these devices.

\subsection{Agentive, patientive and experiencer nominalizations}
\label{subsec: agentive, patientive and experiencer nominalizations}


Agentive nominalizations, achieved through the affixation of the participial suffix \textit{-ame} to a transitive base or an intransitive base with an unergative argument, are nominalizations that derive a noun with a meaning ‘the one who performs V’. The examples in (\ref{ex: agentive derivations}) illustrate agentive derivations, both from bare roots (\ref{ex: agentive derivations}a--b) and morphologically complex bases (\ref{ex: agentive derivationsc}).

\ea\label{ex: agentive derivations}
{Agentive nominalizations}\\

    \ea[]{
    {\textit{miˈʔàami}}\\
    \gll    miˈʔà-ame\\
            kill.\textsc{sg-ptcp}\\
    \glt    ‘one who kills’\\
    \glt    ‘el que mata’ < BFL 09 1:43/el > \\
}\label{ex: agentive derivationsa}
        \ex[]{
        {\textit{aˈwíami}}\\
        \gll    aˈwí-ame\\
                dance-\textsc{ptcp}\\
        \glt    ‘one who dances’\\
        \glt    ‘el que baila’ < BFL 09 1:43/el >\\
    }\label{ex: agentive derivationsb}
            \ex[]{
            {\textit{biˈnèrami}\footnote{The example in (\ref{ex: agentive derivationsc}) also shows how the participial suffix may induce deletion of the previous suffix vowel. There are no recorded examples where the participial suffix deletes vowels of immediately preceding roots (see \chapref{chap: verbal morphology} for other inflectional suffixes that induce vowel deletion in their bases).}}\\
            \gll    biˈnè-ri-ame\\
                    learn-\textsc{caus-ptcp}\\
            \glt    ‘one who teaches’\\
            \glt    ‘el que enseña’ < BFL 09 1:43/el > \\
        }\label{ex: agentive derivationsc}

    \z
\z

Verbs that have been nominalized with the participial \textit{-ame} suffix do not take any TAM markers before the participial suffix, a fact that has also been observed for \ili{Mountain Guarijío} \citep[][180]{miller1996guarijio}. There are, however, agentive nominalization constructions with a meaning that involves past tense, i.e. ‘the one who has performed V (but no longer does)’. These past agentive nominalizations attach the suffix \textit{-kame}, the cognate of which \citet{miller1996guarijio} analyzes as composed diachronically of a past tense \textit{-ka/-ga} suffix and the participial \textit{-ame} suffix (note that the \ili{Mountain Guarijío} \textit{-ka/-ga} past tense suffix has no attested cognate in Choguita Rarámuri as a synchronically active past tense marker). Relevant examples are shown in (\ref{ex: past agentive nominalizations}):

\ea\label{ex: past agentive nominalizations}
{Past agentive nominalizations}\\

    \ea[]{
    {\textit{biˈnèkami}}\\
    \gll    beˈnè-kame\\
            learn-\textsc{pst.ptcp}\\
    \glt    ‘one who has studied’\\
    \glt    ‘el estudiado’  < BFL 09 1:43/el >\\
}
        \ex[]{
        {\textit{ˈnè   ˈpé   ˈsêstobi   kú   maˈt͡ʃínikami   ka   ˈt͡ʃó   ba}}\\
        \gll    ˈnè  ˈpé  ˈsêsto=bi  ku  maˈt͡ʃína-kame  ka ˈt͡ʃó  ba  \\
                \textsc{1sg.nom} just  sixth=just  \textsc{rev} go.out-\textsc{pst.ptcp}  \textsc{cop.irr} also  \textsc{cl}  \\
        \glt    ‘I also just finished sixth grade’\\
        \glt    ‘yo nomás soy salido de sexto también’ \corpuslink{tx12[11_172-11_195].wav}{SFH tx12:11:17.2}\\
    }
            \ex[]{
            {\textit{miˈʔàkami}}\\
            \gll    miˈʔà-kame\\
                    kill-\textsc{pst.ptcp}\\
            \glt    ‘someone who used to murder, former killer’\\
            \glt    ‘alguien que mataba, antiguo asesino’ < BFL 06 el41/el >\\
        }

    \z
\z

Participial suffixes are also involved in deriving adjectives and forming relative clauses, and this is discussed in detail in §\ref{sec: adjectives} and §\ref{sec: relative clauses}, respectively. As discussed in §\ref{sec: adjectives}, the participial suffixes \textit{-ame} and \textit{-kame} do not exhibit tense distinctions and are treated there as suppletive allomorphs.

Nominalizations involving other tense/aspect contrasts (e.g., future, as in ‘the one who will become a ritual singer’) are expressed through copular predicate constructions. In examples (\ref{ex: nominalizations in copular clauses}a--b), tense distinctions are marked in the copula. In the past tense nominalization, however, there is no copula available; the past tense of the nominalization is encoded by the  \textit{-kame} suffix (\ref{ex: nominalizations in copular clausesc}). The nominalization in (\ref{ex: nominalizations in copular clausesa}) encodes imperfective aspect.

\largerpage
\ea\label{ex: nominalizations in copular clauses}
{Tense/aspect distinctions with participial forms}

\ea[]{
\textit{wikaˈrâame   ˈnílo}\\
\gll wikaˈrâ-ame  ˈní-li-o\\
    sing-\textsc{ptcp}  \textsc{cop-pst-ep}\\
\glt ‘He used to be a ritual singer.’\\
\glt ‘Era cantador.’ < BFL 06 4:168-171/el >\\
 }\label{ex: nominalizations in copular clausesa}
%\pagebreak
\ex[]{
\textit{wikaˈrâame   ˈníma     riˈkó}\\
\gll wikaˈrâ-ame   ˈní-ma     riˈkó\\
    sing-\textsc{ptcp} \textsc{cop-fut.sg} \textsc{dub}\\
\glt ‘He will be a ritual singer ’\\
\glt ‘Va a ser cantador.’ < BFL 06 4:168-171/el >
}\label{ex: nominalizations in copular clausesb}
\newpage

\ex[]{
\textit{ˈét͡ʃi   ko   ˈmá     miˈʔàkami   ko}\\
\gll ˈét͡ʃi=ko  ˈmá   miˈʔà-kame=ko\\
    \textsc{dem=emph} already kill-\textsc{pst.ptcp=emph}\\
\glt ‘That one has already killed.’\\
\glt ‘Ese ya ha matado.’ < BFL 06 4:168-171/el >\\
}\label{ex: nominalizations in copular clausesc}

\ex[]{
\textit{suˈwíkame}\\
\gll suˈwí-kame\\
  finish.off.\textsc{intr}\textsc{-pst.ptcp}\\
\glt ‘dead ones’\\
\glt ‘muertos’ < SFH 06 11.06/el >\\
}\label{ex: nominalizations in copular clausesd}
\z
\z

%% TP: if you want to avoid indentation for 1-line paragraphs like this,
%% you can use \noindent
\noindent
Further examples of agentive nominalizations are given in (\ref{ex: agentive nominalizations 2}):

% TP: work in progress
\ea\label{ex: agentive nominalizations 2}
\ea[]{
% \setlength\tabcolsep{2pt} % default value: 6pt
% \begin{tabular}{llll}
%     \textit{siˈpa-ame} &
%     ‘raspar.peyote{}-\textsc{ptcp’} &
%     ‘peyote shaman’ &
%     < SFH 06 11.06/el > \\
% \end{tabular}
% \setlength\tabcolsep{6pt} % default value: 6pt
{\textit{siˈpâame}}\\
\textit{siˈpâ-ame}\\
{raspar.peyote{}-\textsc{ptcp}}\\
‘peyote shaman’\\
`raspador' < SFH 06 11.06/el >\\
}
\ex[]{
% \textit{aˈwí-ami} ‘dance-\textsc{ptcp’} ‘dancer’    < SFH 06 11.06/el >
% \setlength\tabcolsep{2pt} % default value: 6pt
% \begin{tabular}{llll}
%     \textit{aˈwí-ami} &
%     ‘dance-\textsc{ptcp’} &
%     ‘dancer’ &
%     < SFH 06 11.06/el > \\
% \end{tabular}
% \setlength\tabcolsep{6pt} % default value: 6pt
{\textit{aˈwíami}}\\
\textit{aˈwí-ame}\\
{dance-\textsc{ptcp}}\\
‘dancer’\\
`bailador' < SFH 06 11.06/el >\\
}
\ex[]{
\textit{kaˈréami}\\
\textit{kaˈré-ame}\\
lie-\textsc{ptcp}\\
‘liar’\\
`mentiroso'      < SFH 06 11.06/el >\\
}
\ex[]{
\textit{ˈwáami} \\
\textit{ˈwá-ame}\\
be.strong-\textsc{ptcp}\\
‘strong’\\
`fuerte'    < SFH 06 11.06/el >\\
}
\ex[]{
\textit{ˈhóami} \\
\textit{ˈhó-ame}\\
make.hole-\textsc{ptcp} \\
‘the one who makes holes'\\
`el que hace hoyos'   < SFH 06 11.06/el >\\
}
\ex[]{
\textit{raraˈhîpami} \\
\textit{raraˈhîp-ame}\\
ball.race-\textsc{ptcp} \\
‘ball race runner’\\
`corredor de bola (rarajipa)'  < SFH 06 11.06/el >  \\
}
\z
\z

%   g.  \textit{ˈmà-ami} ‘run-\textsc{ptcp}’    ‘runner’    < SFH 06 11.06/el >

%   h.  \textit{winiˈhi-ami} ‘denounce-\textsc{ptcp’} ‘one who denounces’  < SFH 06 11.06/el >

%   i.  \textit{biˈnè-ami} ‘learn-\textsc{ptcp}’    ‘student’  < BFL 06 4:168-171/el >

%   j.  \textit{raʔˈlá-ami} ‘buy-\textsc{ptcp’} ‘buyer’    < BFL 06 4:168-171/el >

%   k.  \textit{ˈsû-ami} ‘sew-\textsc{ptcp}’    ‘seamstress’  < BFL 06 4:168-171/el >

%   l.  \textit{oˈsá-ami} ‘write-\textsc{ptcp}’    ‘writer’  < BFL 06 4:168-171/el >

%   m.  \textit{wiˈt͡ʃô-ami} ‘wash-\textsc{ptcp}’    ‘one who washes’ < BFL 06 4:168-171/el >

%   o.  \textit{kaˈwi-ami} ‘gather.wood-\textsc{ptcp}’  ‘one who gathers wood' < SFH 06 4:168-171/el >


%   o.  \textit{koˈʔá-ami} ‘eat-\textsc{ptcp’} ‘one who eats’  < GFM 09 tx905(3)/tx >

%   p.  \textit{oˈla-ami} ‘do-\textsc{ptcp}’    ‘one who does’ < GFM 09 tx905(22)/tx >

%   q.  \textit{aˈnaur-ami} ‘measure-\textsc{ptcp}’  ‘one who measures’[JMF 09 tx817(13)/tx >

%   r.  \textit{siˈrux-ami} ‘hunt-\textsc{ptcp}’    ‘hunters’    < LEL tx110(31)/tx >

% \z

Patientive nominalizations, on the other hand, are formed in Choguita Rarámuri through the participial suffix \textit{-ame} which attaches to a verbal passive base (\ref{ex: patientive nominalizations 1}a--e) or to an intransitive verb with a theme as subject argument (\ref{ex: patientive nominalizations 1f}):\footnote{In these cases, the participial suffix undergoes optional post-tonic vowel reduction in word final position (with raising of [e] to [i]).}

\largerpage
\ea\label{ex: patientive nominalizations 1}
{Patientive nominalizations}
\ea[]{
\textit{miˈʔàaruami}\\
\gll miˈʔà-ru-ame\\
 kill.\textsc{sg-pst.pass-ptcp}\\
\glt ‘one who gets killed’\\
\glt ‘al que matan’ < BFL 09 1:43/el >\\
}\label{ex: patientive nominalizations 1a}
\ex[]{
\textit{aˈwèrami}\\
\gll aˈwí-è-ru-ame\\
     dance-\textsc{appl-pst.pass-ptcp}\\
\glt ‘one who gets danced for’\\
\glt ‘al que le bailan’ < BFL 09 1:43/el >\\
}\label{ex: patientive nominalizations 1b}
\ex[]{
\textit{saˈwèrami}\\
\gll  saˈwí-è-ru-ame\\
     cure\textsc{-appl-pst.pass-ptcp}\\
\glt ‘the healed one’\\
\glt ‘el curado’ < BFL 06 el41(15)/el >\\
}\label{ex: patientive nominalizations 1c}
\ex[]{
\textit{miˈʔàrami}\\
\gll    miˈʔà-ru-ame\\
     kill.\textsc{sg-pst.pass-ptcp}\\
\glt ‘murdered one’\\
\glt ‘el matado, asesinado’ < BFL 06 el41/el >\\
}\label{ex: patientive nominalizations 1d}
\ex[]{
{\textit{baˈt͡ʃàrami}}\\
    \gll    baˈt͡ʃà-ru-ame\\
            put.inside\textsc{.sg-pst.pass-ptcp}\\
    \glt    ‘imprisoned one’\\
    \glt    ‘encarcelado’ < SFH 06 11.06/el >\\
}\label{ex: patientive nominalizations 1e}
\ex[]{
\textit{baˈkíami}  \\
\gll  baˈkí-ame\\
     go.in.\textsc{sg}-\textsc{ptcp}\\
\glt ‘thing that has been put inside somewhere’\\
\glt ‘metido’ < SFH 06 11.06/el >  \\
}\label{ex: patientive nominalizations 1f}

\z
\z

%Patientive nominalizations can also be formed from non-verbal bases, as in (\ref{ex: patientive nominalizations from non-verbal bases}), where the participial suffix attaches to the root \textit{baˈt͡ʃá} ‘first'; the resulting nominalization is, however, not semantically transparent.

%\ea\label{ex: patientive nominalizations from non-verbal bases}
%{Patientive nominalization of a non-verbal base}\\


%\z

Further examples of patientive nominalizations are provided in (\ref{ex: patientive nominalizations 2}):

\ea\label{ex: patientive nominalizations 2}

    \ea[]{
    \textit{maˈhârami} \\
    \textit{maˈhâ-r-ame}\\
    be.affraid-\textsc{pst.pass-ptcp}\\
    ‘fearful’\\
    `temeroso'\\
}
        \ex[]{
        \textit{oˈhòrami}\\
        \textit{oˈhò-r-ame}\\
        dekernel-\textsc{pst.pass-ptcp} \\
        ‘dekerneled' \\
        `desgranado’ < SFH 06 11.06/el >\\
    }
            \ex[]{
            \textit{ˈèrami} \\
            \textit{ˈè-r-ame}\\
            take-\textsc{pst.pass-ptcp} \\
            ‘the thing taken away'\\
            `lo llevado' {< SFH 06 11.06/el >}\\
        }
                \ex[]{
                \textit{paˈkórami} \\
                \textit{paˈkó-r-ame}\\
                wash-\textsc{pst.pass-ptcp} \\
                ‘baptized one’ \\
                `bautizado' {< SFH 06 11.06/el >}\\
            }
                    \ex[]{
                    \textit{riˈwètami} \\
                    \textit{riˈwè-t-ame}\\
                    leave-\textsc{pst.pass-ptcp}\\
                    ‘abandoned one’\\
                    `abandonado' {< SFH 06 11.06/el >}\\
                }
                        \ex[]{
                        \textit{biniˈhîrami} \\
                        \textit{biniˈhî-r-ame}\\
                        denounce-\textsc{pst.pass-ptcp} \\
                        ‘denounced one’ \\
                        `denunciado' {< SFH 06 11.06/el >}\\
                    }
                            \ex[]{
                            \textit{saˈwèrami} \\
                            \textit{saˈw-è-r-ame}\\
                            cure-\textsc{appl-pst.pass-ptcp} \\
                            ‘cured one’  \\
                            `curado' {< BFL 06 4:168/el > } \\
                        }
                                \ex[]{
                                \textit{t͡ʃoʔˈnárami} \\
                                \textit{t͡ʃoʔˈná-r-ame}\\
                                hit.with.fist-\textsc{pst.pass-ptcp} \\
                                ‘beaten up one’ \\
                                `golpeado' {< BFL 06 4:168/el >}  \\
                            }
                                    \ex[]{
                                    \textit{miˈʔàrami} \\
                                    \textit{miˈʔà-r-ame}\\
                                    kill.\textsc{sg}{}-\textsc{pst.pass-ptcp}\\
                                    ‘killed one’ \\
                                    `asesinado' {< BFL 06 4:168/el >} \\
                                }
                                        \ex[]{
                                        \textit{koˈʔírame} \\
                                        \textit{koˈʔí-r-ame}\\
                                        kill.\textsc{pl}{}-\textsc{pst.pass-ptcp}  \\
                                        ‘killed ones’ \\
                                        `asesinados' {< BFL 06 4:168/el >}\\
                                    }
                                            \ex[]{
                                            \textit{iˈt͡ʃìrame} \\
                                            \textit{iˈt͡ʃ-ì-r-ame}\\
                                            sow-\textsc{appl-pst.pass-ptcp}  \\
                                            ‘thing sown’\\
                                            `lo plantado' {< BFL 06 4:168/el >}\\
                                        }
                                                \ex[]{
                                                 \textit{ˈʔwîrami} \\
                                                 \textit{ˈʔwî-r-ame}\\
                                                 harvest-\textsc{pst.pass-ptcp} \\
                                                 ‘thing harvested' \\
                                                 `lo cosechado' {< BFL 06 4:168/el >}\\
                                            }
%                                    \pagebreak
                                                    \ex[]{
                                                    \textit{wiˈt͡ʃôrami} \\
                                                    \textit{wiˈt͡ʃô-r-ame}\\
                                                    wash-\textsc{pst.pass-ptcp}  \\
                                                    ‘washed clothes’  \\
                                                    `lo lavado' {< SFH 06 4:168/el >}\\
                                                }
                                                \pagebreak
                                                            \ex[]{
                                                            \textit{ˈtòorami} \\
                                                            \textit{ˈtò-r-ame}\\
                                                            take-\textsc{pst.pass-ptcp}  \\
                                                            ‘one who was  taken away’ \\
                                                            `el llevado' \corpuslink{tx12[00_482-00_520].wav}{SFH tx12:0:48.2}\\
                                                        }
                                                                \ex[]{
                                                                \textit{rohoˈnârami} \\
                                                                \textit{roho-ˈnâ-r-ame}\\
                                                                separate-\textsc{tr-pst.pass-ptcp}\\
                                                                ‘ones who have  been separated’ \\
                                                                `los separados' \corpuslink{tx817[01_001-01_030].wav}{JMF tx817:1:00.1}\\
                                                            }
    \z
\z

There is variation among speakers in the prosodic make up of agentive and patientive nominalizations. Variation is mostly linked to stress placement in the base, but is also apparent when a given base includes a construction that has more than one possible morphological exponent for a given category. This variation is not exclusively found in nominalization constructions (see further discussion in \chapref{chap: verbal morphology}). Examples of prosodic variation are provided in (\ref{ex: prosodic variation in nominalized forms}), where both vocalic and stress placement differences are attested.

\ea\label{ex: prosodic variation in nominalized forms}
{Prosodic variation of nominalized forms}

    \ea[]{
    \textit{t͡ʃoʔˈnárami}\\
    \gll    t͡ʃoʔˈná-r-ame\\
    hit.with.fist-\textsc{pst.pass-ptcp} \\
    \glt    `the ones that were hit with the first'\\
    \glt    `los golpeados con el puño' < BFL 06 4:168-71/el > \\
}
        \ex[]{
        \textit{t͡ʃoʔˈníirami} \\
        \gll    t͡ʃoʔˈníi-r-ame\\
                hit.with.fist-\textsc{pst.pass-ptcp}\\
        \glt    `the ones that were hit with the first'\\
        \glt    `los golpeados con el puño' < BFL 06 4:168-71/el > \\
    }
            \ex[]{
            \textit{t͡ʃoʔniˈrúami} \\
            \gll    t͡ʃoʔni-ˈrú-ame\\
                    hit.with.fist-\textsc{pst.pass-ptcp}\\
            \glt `the ones that were hit with the first'\\
            \glt    `los golpeados con el puño'    < SFH 06 4:168-71/el >\\
        }
        \newpage
                \ex[]{
                \textit{miʔriˈrúami}  \\
                \gll    miʔri-ˈrú-ame\\
                        kill.\textsc{sg-pst.pass-ptcp} \\
                \glt    `murdered one’\\
                \glt    `asesinado' < SFH 06 4:168-71/el > \\
            }
                    \ex[]{
                    \textit{miʔˈríirami}    \\
                   \gll     miʔˈríi-r-ame\\
                    kill.\textsc{sg-pst.pass-ptcp} \\
                    \glt    `murdered one’\\
                    \glt    `asesinado' < SFH 06 4:168-71/el >\\
                }
                        \ex[]{
                        \textit{iˈt͡ʃiirami} \\
                        \gll    iˈt͡ʃii-r-ame\\
                                plant-\textsc{pst.pass-ptcp}  \\
                        \glt    ‘what has been planted'\\
                        \glt    `lo plantado’ < BFL 06 4:168-71/el > \\
                    }
                            \ex[]{
                            \textit{it͡ʃiˈrúami}\\
                            \gll    it͡ʃi-ˈrú-ame\\
                                    plant-\textsc{pst.pass-ptcp}\\
                            \glt    ‘what has been planted'\\
                            \glt    `lo plantado’  < SFH 06 4:168-71/el >\\
                        }
                                \ex[]{
                                \textit{niˈʔokirami} \\
                                \gll    niˈʔo-ki-r-ame\\
                                        speak.badly-\textsc{appl-pst.pass-ptcp} \\
                                \glt    ‘what has been written'\\
                                \glt    `lo escrito’ < BFL 06 4:168-71/el >\\
                            }
                                    \ex[]{
                                    \textit{oˈsírami} \\
                                    \gll    oˈsí-r-ame\\
                                            write-\textsc{pst.pass-ptcp} \\
                                    \glt    ‘what has been written'\\
                                    \glt    `lo escrito’  < BFL 06 4:168-71/el > \\
                                }
                                        \ex[]{
                                        \textit{osiˈrúami}    \\
                                        \gll    osi-ˈrú-ame\\
                                                write-\textsc{pst.pass-ptcp} \\
                                        \glt    ‘what has been written'\\
                                        \glt    `lo escrito’ < SFH 06 4:168-71/el >  \\
                                    }
    \z
\z

There is another class of nominalizations, what I will call ``theme nominalizations'', where the participial marker attaches to a medio-passive base, and the nominalization refers to an inanimate referent that has undergone a change of state. Theme nominalizations are exemplified in (\ref{ex: theme nominalizations}):

\ea\label{ex: theme nominalizations}
{Theme nominalizations}\\

    \textit{aˈt͡ʃèwami}\\
    \gll    aˈt͡ʃè-wa-ame\\
            throw.in-\textsc{mpass-ptcp}\\
    \glt    ‘the thing that is being thrown in’\\
    \glt    ‘lo que le echan’ < BFL 09 1:37/el >  \\
\z

%Note: the habitual passive construction without the nominal derivation has a suppletive allomorph: \textit{ache-ríwa},

The  example in (\ref{ex: agentive vs. patientive nominalization}) shows a comparison between an agentive and a patientive nominalization; in this case, there is a gap in the theme nominalization, since a clause with the complementizer \textit{ˈnápi} was given instead of a morphologically derived word in (\ref{ex: agentive vs. patientive nominalizationc}) (i.e. the theme nominalization does not contain the participial marker).

\ea\label{ex: agentive vs. patientive nominalization}
{Agentive vs. patientive nominalization}

    \ea[]{
    \textit{naˈmûti  niˈhîami}\\
    \gll    naˈmûti  niˈhî-ame\\
            thing  gift-\textsc{ptcp}  \\
    \glt    ‘the one who gifts things’\\
    \glt    ‘el que regala’   < BFL 09 1:43/el >\\
}\label{ex: agentive vs. patientive nominalizationa}
        \ex[]{
        \textit{naˈmûti  aˈrîwami}\\
        \gll    naˈmûti  a-rîwa-ame\\
                thing  give-\textsc{mpass-ptcp}  \\
        \glt    ‘the one who is given things’\\
        \glt    ‘al que le dan cosas’    < BFL 09 1:43/el >    \\
    }\label{ex: agentive vs. patientive nominalizationb}
            \ex[]{
            \textit{ˈnápu  niˈhíruwa}\\
            \gll    nápi  nihí-riwa\\
                    \textsc{sub} give-\textsc{mpass}\\
            \glt    ‘what is being given away'\\
            \glt    ‘lo que se regala’  < BFL 09 1:43/el >      \\
        }\label{ex: agentive vs. patientive nominalizationc}
    \z
\z

\newpage
In this case the patientive nominalization is formed from a habitual passive base, instead of a past passive base.\footnote{One possibility is that this particular base may derive a patientive nominalization from the habitual passive since the verb in question is \textit{à} ‘give’, a ditransitive verb with a theme argument encoded as a secondary object and a recipient argument encoded as a primary object (\chapref{chap: basic clause types} provides an overview of different clause types, including ditransitive clauses (§\ref{subsec: ditransitive clauses})).}

%Most important: different roots for agentive and theme nominalization (\textit{nihí} ‘give away’/ \textit{‘regalar’}, and for the patientive nominalization (\textit{à} ‘give’/’dar’). Also, the theme nominalization in this case is derived with the habitual passive, as in the other examples that include a participial. Test: \textit{nihí-ru-ami?}

The participial construction also appears in nominalizations that are neither agentive nor patientive. In (\ref{ex: experiencer nominalizations}) below, the resulting derivation yields ‘experiencer’ nominalizations:

\ea\label{ex: experiencer nominalizations}
{Experiencer nominalizations}

    \ea[]{
    \textit{ˈtû    piˈrêami}\\
    \gll    ˈtû    peˈrê-ame\\
            down.river  inhabit.\textsc{pl-ptcp}\\
    \glt    ‘the inhabitants down river’\\
    \glt    ‘los que viven río abajo’  < SFH 06 11.06/el >\\
}
        \ex[]{
        \textit{kuˈnêami}\\
        \gll    kuˈnà-ê-ame\\
                husband-\textsc{have-ptcp}\\
        \glt    ‘the ones who have husbands’\\
        \glt    ‘las que tienen marido’    < SFH 06 11.06/el >\\
    }
    \z
\z

\subsection{Deverbal nouns with \textit{-ri}}
\label{subsec: deverbal nouns with -ri}

% MOre details about this construction? Mountain \ili{Guarijío}?

There are nouns derived from verbal bases through the nominalizing suffix \textit{-ri}. Examples of this suffixes are shown in (\ref{ex: derverbal nouns in -ra}).\footnote{In these examples, the final vowel of the base undergoes post-tonic reduction (raising to [i]). More details about post-tonic vowel reduction are provided in §\ref{subsubsec: stress-based vowel reduction and deletion}.}

\ea\label{ex: derverbal nouns in -ra}
{Deverbal nouns in \textit{-ri}}\\

\begin{tabular}{lllll}
        &  {Deverbal} &   {Gloss} &  {Stem} &  {Gloss}\\
        &  {noun} \\
     a.& raˈʔìt͡ʃi-ri & `words, speech' & raˈʔìt͡a & `to speak' \\
      & &  < GFM 09 tx905(27)/tx >\\
     b.& ˈnâti-ri  & `thoughts, memories' & ˈnâta & `to think'\\
     & &  < SFH 06 in61(8)/in >\\
     c.& iˈwéri-ri &  `strength' &  iˈwéri & `to be strong' \\
      &   & < SFH 06 tx12(71)/tx > \\
     \end{tabular}

\begin{tabular}{lllll}
     d.& ˈnòt͡ʃi-ri & `job' & ˈnòt͡a & `to work' \\
      & & < SFH 06 yx12(177)/tx >\\
     e.& oˈmáwi-ri & ‘ceremony’ & oˈmáwa & `to make party' \\
       & & < SFH 06 tx12(87)/tx > \\
     f.& hiˈrâ-ri & ‘things for betting’ & hiˈrâ & ‘to bet’ \\
       & &  < LEL 06 tx19(49)/tx >\\
     g.& baˈt͡ʃôki-ri & `adobe’ & baˈt͡ʃôka &  ‘to fix walls \\
       & &  < SFH in61(70)/in > &  & with mud'\\
     h.& aʔˈpé-ri & ‘lump of things’ & aʔˈpe &  ‘to carry in  \\
       & &  < LEL 06 tx 19(33)/tx > & & back’\\
\end{tabular}
    \z

The form in (\ref{ex: non-transparent nominalization}) is an example of a phonologically non-transparent nominalization (where the vowel of the nominalizing suffix undergoes anticipatory asimilation with the following particle).

\ea\label{ex: non-transparent nominalization}

    \textit{ˈpé   aʔˈlá   amiˈnábi   ˈàa} \textbf{\textit{busuˈrêro}} \textit{ˈt͡ʃo  ˈlé} \\
    \gll    ˈpé   aʔˈlá   amiˈná=bi  ˈà    \textbf{busuˈrê-ri}  ˈt͡ʃo  aˈlé\\
            just  well  more=just    give.\textsc{prs}  {wake.up-\textsc{nmlz}}   also  \textsc{dub} \\
    \glt    ‘but we must also give more advice (to our children)’\\
    \glt    ‘pero hay que darle más consejo (a nuestros hijos)’ < SFH 06 in61(713)/in >\\

\z

%Reference to Mountain \ili{Guarijío}

\section{Spanish noun loanwords}
\label{sec: Spanish loan nouns}

The prosodic properties of loanwords from \ili{Spanish} are addressed in \chapref{chap: phonology} and \chapref{chap: other word-level suprasegmental phonology}. This subsection addresses the morphological strategies with which borrowed nouns from \ili{Spanish} are incorporated into the Choguita Rarámuri lexicon. Specifically, relatively recent loan nouns from \ili{Spanish} appear in constructions with the \textit{-tʃ͡i} suffix. This marker is homophonous with the locative suffix, but in none of the examples below is there any evidence for locative semantics. The formative’s sole function in these cases is to nativize the loanwords. This is exemplified in (\ref{ex: Spanish loanwords with -chi}).

\ea\label{ex: Spanish loanwords with -chi}
{Nativization of \ili{Spanish} noun loanwords}\\

\begin{tabular}{llll}
        & \textit{Loanword}  & \textit{Gloss} & \textit{\ili{Spanish} source}\\
     a.& esˈpêhotʃ͡i &  `mirror' & espejo \\
     b.& ˈtrêntʃ͡i  & train' & tren \\
     c.& kaˈmiôntʃ͡i &  bus' & camión\\
     d.& liˈmêtatʃ͡i & `bottle' & limeta \\
     e.& alaˈkûuntʃ͡i & `lake, lagoon’ & laguna \\
\end{tabular}
    \z

As described in §\ref{sec: loanword prosody}, there are a few loan nouns from \ili{Spanish} that do not bear any morphological marking, but display varying degrees of phonological adaptation. Some examples are given in (\ref{ex: varying degrees of adaptation in Spanish loanwords}):

\ea\label{ex: varying degrees of adaptation in Spanish loanwords}
{Varying degrees of adaptation in \ili{Spanish} loanwords}\\

\begin{tabular}{llll}
        & \textit{Loanword}  & \textit{Gloss} & \textit{\ilit{Spanish} source}\\
     a.& ˈûli &  `rubber' & \textit{hule} \\
     b.& koˈmâare  & `co-mother' & \textit{comadre} \\
     c.& komˈpâare &  `co-father' & \textit{compadre}\\
     d.& ˈtîna & `bucket' & \textit{tina} \\
     e.&  ˈsôru & `soda' & \textit{soda} \\
     f. & ˈwàsi & `cow' &\textit{vaca}\\
\end{tabular}
    \z

Finally, there are a few nouns where newer loans coexist with older loanwords, as exemplified in (\ref{ex: newer and older loanwords}).  In these pairs, older loans display more phonological adaptation (e.g., \textit{káwi}) than the newer loanword or are based in an archaic form of \ili{Spanish} (\textit{e.g., limeta}, a noun no longer in use in contemporary Northern Mexican \ili{Spanish}).

\ea\label{ex: newer and older loanwords}
{Coexistence of recent and archaic loanwords}

\begin{tabular}{lllll}
        & \textit{Archaic loanword}  & \textit{Recent loanword} & \textit{Gloss} & \textit{\ili{Spanish} source}\\
     a.& ˈkâwi &  kaˈbâjo & `horse' & \textit{caballo}\\
     b.& liˈmêtatʃ͡i  & boˈtêja  & `bottle' &\textit{botella, limeta} \\
\end{tabular}
    \z

There are no other documented morphological strategies to nativize \ili{Spanish} nouns in the corpus of Choguita Rarámuri.

% Predictable on any grounds for when the suffix will be used as a nativization strategy or not?

%Elaborate on the morphological properties of noun loanwords


\section{Tone in morphologically complex nouns}
\label{sec: tone in morphologically complex nouns}
\largerpage[2]
This section addresses the tonal properties of morphologically complex nouns. While nominal morphology is more limited than verbal morphology, and the interactions between tonal and stress properties of both roots and morphological constructions is largely a matter of general phonological processes, some tonal patterns in morphologically complex nouns exhibit lexical and morphological conditioning.

Nouns exhibit the following word-prosodic properties, which are shared with other major word classes (see also \chapref{chap: prosody}, §\ref{subsec: stress patterns and metrical feet} and §\ref{sec: prosodic properties of morphologically complex verbs}):

\ea\label{ex: word-prosodic properties of nouns}
{Word-prosodic properties of nouns}\\

\begin{itemize}
    \item   Each prosodic word has a single main stress.\\
    \item   There is no secondary stress (i.e., stress is non-iterative).\\
    \item   Stress is restricted to appear within an initial three-syllable window.\\
    \item   There is a three-way tonal contrast (HL, L, and H).\\
    \item   Tonal contrasts are exclusively realized in stressed syllables.\\
    \item   Morphological constructions are either stress-shifting, triggering stress shifts, or stress-neutral, triggering no stress changes in the stems to which they attach.\\
    \item    Stress-shifting suffixes are part of the stressable domain, while stress-neutral suffixes are outside the stressable domain and never stressed.\\
    \item   Stressed stem syllables bear the lexical tone of the root, while stressed suffixed syllables bear the lexical tone of the suffix.\\
\end{itemize}

\z

From the set of nominal morphological constructions, there are only two suffixes that trigger a stress shift in the bases to which they attach. These are listed in (\ref{ex: stress-shifting nominal suffixes}). Stress falls on the suffix syllable, where the underlying lexical tone of the suffix is realized.

\ea\label{ex: stress-shifting nominal suffixes}
{Nominal stress-shifting suffixes}\\

\begin{tabular}{lllll}
     a.& Locative-adessive  &  \textsc{loc.ad} & \textit{-tʃ͡í} & H \\
     b.& Possessive  & \textsc{poss}  & \textit{-lâ}  & HL \\
\end{tabular}
    \z

The examples in (\ref{ex: examples of nominal stress-shifting suffixes}) and (\ref{ex: nominal stress-shifting suffixes possessive}) illustrate the stress and lexical tonal properties of these suffixes.

%\pagebreak

\ea\label{ex: examples of nominal stress-shifting suffixes}
{Stress-shifting locative-adessive \textit{-tʃ͡í}}

\begin{tabular}{lllll}
       & \textit{Locative-} & \textit{Gloss} & \textit{Stem} & \textit{Gloss} \\
       & \textit{addesive}\\
     a.& wasa-ˈtʃ͡í & `on the land' &  waˈsá  & `land' \corpuslink{tx43[07_443-07_505].wav}{SFH tx43:7:44.3} \\
     b.& kawi-ˈtʃ͡í & `on the hill'  & kaˈwì  & `hill' \corpuslink{tx32[00_307-00_369].wav}{LEL tx32:0:30.7}  \\
     c. & rono-ˈtʃ͡í & `on the foot' & roˈnô &  `foot' {\corpuslink{co1140[17_527-17_548].wav}{MDH co1140:17:52.7}}\\
\end{tabular}
    \z

\ea\label{ex: nominal stress-shifting suffixes possessive}
{Stress-shifting possessive \textit{-lâ}}
\setlength{\tabcolsep}{4pt}
\begin{tabular}{lllll}
       & \textit{Possessive form} & \textit{Stem} & \textit{Gloss} & \textit{Source} \\
     a.& ono-ˈlâ & oˈnó & `father (male ego)' &{< SFH tx12(9)/tx >}\\
     b.& aʔka-ˈlâ & aʔˈkà & `sandals'& < BFL 09 1:60/el >  \\
     c. & boʔi-ˈlâ & boˈʔí & `road' &< BFL 06 5:128/el > \\
     d. & sono-ˈlâ & soˈnó &  `lungs' &{< MGD 06 1:107/el >}\\
\end{tabular}
    \z

As shown in these examples, stress shifts one syllable rightward to the suffix syllable, the third syllable of the word. In contrast, other nominal suffixes never condition stress shifts with the nouns to which they attach (e.g., the locative-inessive \textit{-rare} suffix never triggers stress shifts; compare, for instance, \textit{saʔˈpá-rare} `in the meat', with no stress shift and \textit{saʔpa-ˈtʃ͡í} `on the meat', with stress shift, in (\ref{ex: examples of nominal stress-shifting suffixes}c) above). There are no other documented cases in the corpus of Choguita Rarámuri where other nominal suffixes trigger stress shifts in the morphologically complex words in which they appear.

While nominal morphological constructions may be classified as stress-shif\-ting or stress-neutral, triggering stress shifts or not, respectively, nominal roots do not exhibit a stressed/unstressed distinction like verbs do. Recall that verbs may be classified as lexically stressed or unstressed, with lexically stressed verbs retaining stress in a fixed position across paradigmatically related words and lexically unstressed verbs consistently shifting stress when attaching a stress-shifting construction (see §\ref{subsec: stress and tonal properties of inflected verbs}). In words containing unstressed verbal roots and a stress-neutral construction, stress is assigned by default to the second syllable of the root (or the only syllable of monosyllabic roots) (see §\ref{sec: stress properties of roots, stems and suffixes}). In contrast to this pattern, stress shifts in nouns are not homogenously observed with all suffixes that may be stressed (analyzed here as `stress-shifting'). As shown in (\ref{ex: contrast between locative and possessive with respect to stress}) below, some nominal paradigms exhibit stress shifts with the stress-shifting locative-adessive \textit{-tʃ͡í} suffix, but not with the stress-shifting possessive \textit{-lâ} suffix (forms with stress shift are highlighted in boldface).

\ea\label{ex: contrast between locative and possessive with respect to stress}
{Stress behavior of morphologically complex nouns}

\begin{tabular}{llllll}
       & \textit{Stem} & \textit{Gloss} & \textit{Locative-adessive} & \textit{Possessive} & \textit{Locative-inessive}\\
     a.& saʔˈpá & `meat' &  sapa-ˈtʃ͡í  & saʔˈpá-la & saʔˈpá-rare \\
     b.&  siˈkâ & `hand' & sika-ˈtʃ͡í & siˈkâ-la & siˈkâ-ti \\
\end{tabular}
    \z

Suffix-triggered stress shifts in nominal forms are thus not homogeneously attested, but rather are a matter of lexical conditioning. However, the distinction between stress-shifting and stress-neutral is kept here for nominal morphological constructions, reflecting the ability to trigger a stress shift and bear stress or not, respectively.

%transition here needed

Examples of the three lexical tones in noun roots are provided in (\ref{ex: lexical tone in nouns}).

\ea\label{ex: lexical tone in nouns}
{Lexical tones in nouns}

\begin{tabular}{llll}
       & \textit{Tone} & \textit{Form} & \textit{Gloss} \\
    a. & [L] & {{mè}}& {`agave, mezcal'}\\
    b. & {[H]} & {{té}} & {`lice'}\\
    c. & {[L]} & {{koˈlì}} & {`spatial root'}\\
    d. & {[HL]} & {{koˈlî}} & {‘chile pepper’}\\
    e. & {[HL]} & {{koˈnâ}} & {`salt'}\\
    \end{tabular}
\begin{tabular}{llll}
    f. & {[L]} & {{koʔˈnà}} & `heart of corn cob' (\textit{olote)}\\
    g. & {[HL]} & {{ˈnôtʃ͡a}} &  ‘pretentious'\\
    h. &  {[L]} & {{ˈnòtʃ͡}a}&  `hard working'\\
    i. &  {[H]} & {{naˈpátʃ͡a}}& `blouse'\\
    j. &  {[HL]} & \textit{{tʃ͡aˈbôtʃ͡i}}&  `mestizo person'\\
\end{tabular}
    \z

As attested in morphologically complex verbs, the underlying lexical tone of nominal suffixes is revealed when stressed, as shown in (\ref{ex: lexical tone of nominal suffixes after stress shift}).

\ea\label{ex: lexical tone of nominal suffixes after stress shift}

\begin{tabular}[t]{llllll}
       &  & \textit{Locative-adessive} &  & \textit{Stem} & \textit{Gloss} \\
    a. &  [H]  & kawi-ˈtʃ͡í &  [L] &  kaˈwì & `hill, earth'\\
    b. & [H]  & saʔpa-ˈtʃ͡í & [H]  & saʔˈpá & `meat'\\
    c. &  [H] & rono-ˈtʃ͡í & [HL] & roˈnô & `foot'\\
\end{tabular}

    \z

As shown in these examples, the locative \textit{-tʃ͡i} suffix emerges with a H tone when stressed, regardless of the lexical tone of the nominal root (either L (\ref{ex: lexical tone of nominal suffixes after stress shift}a), H (\ref{ex: lexical tone of nominal suffixes after stress shift}b), or HL (\ref{ex: lexical tone of nominal suffixes after stress shift}c)).

These tonal patterns are thus lexical tone patterns, which are contingent on the morphologically-conditioned stress assignment that operates across the language. In addition to these lexical tone patterns, there is also evidence for grammatical tone in nouns. Specifically, the locative-innessive \textit{-rare} [-\textit{riri}] suffix, a stress-neutral suffix that is never stressed, replaces stem L tones with a grammatical H tonal melody in the stressed syllable of the stem.  This is shown in (\ref{ex: tonal replacement in nouns}). As shown in these examples, tonal replacement is only attested when the root bears a lexical L tone (\ref{ex: tonal replacement in nouns}a--b), but no changes are attested with nouns with other lexical tone patterns (\ref{ex: tonal replacement in nouns}c--d).

\ea\label{ex: tonal replacement in nouns}
{Tonal replacement in morphologically complex nouns}
\setlength{\tabcolsep}{2pt}
\begin{tabular}{llllll}
       &  & \textit{Locative-innessive} &  &  \textit{Stem} & \textit{Gloss} \\
       a. &  [H]  & kaˈwí-riri &  [L] &  kaˈwì & `hill, earth'\\
        & & {< LEL tx109:1:17.4 >} & & {< LEL tx88:1:42.0 >} & \\
    b. & [H]  & roˈhá-riri & [L]  & roˈhà & `oak tree'\\
        & & \corpuslink{tx71[03_596-04_033].wav}{LEL tx71:3:59.6}&  &  {< LEL el1917/el >} &\\
    c. &  [HL] & muˈnî-riri & [HL] & muˈnî & `beans'\\
     & & < BFL 06 5:127/el > & & < MDH co1136:0:10.0 > &\\
    d. & [H] & kiˈmá-riri & [H] & kiˈmá & `blanket'\\
    & & < LEL 06 5:127-9/el > & & < LEL tx372:4:06.5 > &\\
\end{tabular}
    \z

This pattern involves a morphologically-conditioned tonal effect, similar to the one documented with verbs inflected for imperfective or present progressive (see §\ref{subsec: morphologically conditioned tone}), where morphological constructions condition a tonal change in the stem without bringing about a stress change.

Morphologically-conditioned tonal replacement in nouns, however, exhibits both inter- and intra-speaker variation. Some examples are shown in (\ref{ex: no tonal replacement with -rare}):

\ea\label{ex: no tonal replacement with -rare}
\setlength{\tabcolsep}{4pt}
\begin{tabular}[t]{llllll}
       &  & \textit{Locative-innessive} &  &  \textit{Stem} & \textit{Gloss} \\
       a. &  [L]  & roˈhà-riri &  [L] & roˈhà & `oak tree'\\
        & & \corpuslink{tx152[05_572-06_001].wav}{SFH tx152:05:57.2} & &
        \corpuslink{tx152[05_236-05_252].wav}{SFH tx152:5:23.6} \\
    b. & [L]  & kaˈwì-riri & [L]  & kaˈwì & `hill, earth'\\
        & & \corpuslink{tx109[00_321-00_351].wav}{LEL tx109:0:32.1} & &
        \corpuslink{tx130[04_447-04_484].wav}{LEL tx130:4:44.7} \\
\end{tabular}
    \z

For speaker SFH, there is no replacement of the stem lexical L tone when attaching the locative-inessive suffix (\ref{ex: no tonal replacement with -rare}a) (in contrast to (\ref{ex: tonal replacement in nouns}b) where the same noun undergoes tonal replacement for speaker LEL). Speaker LEL exhibits variable realization of tonal patterns with nouns inflected with this suffix, as in (\ref{ex: no tonal replacement with -rare}b), which contrasts with example (\ref{ex: tonal replacement in nouns}a), where the same noun undergoes tonal replacement for the same speaker.

There is thus very marginal evidence for grammatical tone patterns in morphologically complex nouns, in contrast to tonal patterns in verbs (§\ref{subsec: grammatical tone}). This can be understood as instantiating a common cross-linguistic trend whereby nouns show greater `phonological privilege' than verbs, where phonological privilege is defined as the ability to support a greater array of phonological contrasts (\citealt{smith2011category}). Phonological privilege may involve larger number of underlying phonological distinctions, more variety in surface patterns and/or greater resistance to undergoing phonological processes. Lack of grammatical tone in morphologically complex nouns can be also understood as contributing to the phonological privilege of nouns in Choguita Rarámuri.
