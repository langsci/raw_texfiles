\appendix
\chapter{Choguita Rarámuri Morphological Constructions}

%Missing: introduction; list of nominal affixes, affixes of small word classes; unproductive verbal prefixes
%incorporate tone of shifting suffixes

\begin{tabularx}{\textwidth}{XXXX}
\lsptoprule
 \textbf{Postion} & \textbf{Category} & { \textbf{Suffix}} & \textbf{Reference}\\
 \textbf{S1} & Inchoative & Inchoative \textit{{}-ba} (\textsc{inch}) & §1\\
 \textbf{S2} & Transitives & Pluractional transitive \textit{-tʃ͡}\textit{a} (\textsc{tr.pl}) & §2.1\\
\hhline{~~--} &  & Transitive \textit{{}-na} (\textsc{tr}) & §2.2\\
\hhline{~~--} &  & Transitive \textit{{}-bu} (\textsc{tr}) & §2.3\\
 \textbf{S3} & Applicatives & Applicative \textit{{}-ni} (\textsc{appl}) & §3.1\\
&  & Applicative \textit{{}-si} (\textsc{appl}) & §3.2\\
&  & Applicative \textit{{}-wi} (\textsc{appl}) & §3.3\\
 \textbf{S4} & Causative & Causative \textit{{}-ti} (\textsc{caus}) & §4\\
 \textbf{S5} & Applicative & Applicative \textit{{}-ki} (\textsc{appl}) & §5\\
 \textbf{S6} & Desiderative & Desiderative \textit{{}-nale} (\textsc{desid}) & §6\\
 \textbf{S7} & A. Motion & Associated Motion \textit{{}-simi} (\textsc{mot}) & §7\\
 \textbf{S8} & A. Evidential & Auditory Evidential \textit{{}-tʃ͡ane} (\textsc{ev}) & §8\\
 \textbf{S9} & Tense/Aspect/Mood, Voice & Past Passive \textit{{}-ru} (\textsc{pst.pass}) & §9.1\\
&  & Future Passive \textit{-pa} (\textsc{fut.pass}) & §9.2\\
&  & Medio-Passive \textit{{}-riwa} (\textsc{mpass}) & §9.3\\
&  & Conditional Passive \textit{-suwa} (\textsc{cond.pass}) & §9.4\\
&  & Future Singular \textit{-ˈmea, -ma} (\textsc{fut.sg}) & §9.5\\
&  & Future Plural \textit{{}-po} (\textsc{fut.pl}) & §9.6\\
&  & Motion Imperative \textit{-me} (\textsc{mot.imp}) & §9.7\\
&  & (Active) Conditional \textit{-sa} (\textsc{cond}) & §9.8\\
&  & Irrealis singular \textit{{}-me} (\textsc{irr.sg}) & §9.9\\
&  & Irrealis plural \textit{{}-pi} (\textsc{irr.pl}) & §9.10\\
 \textbf{S10} & Mood & Potential \textit{-ra} (\textsc{pot}) & §10.1\\
&  & Imperative sg. \textit{–ka, -sa} (\textsc{imp.sg})  & §10.2\\
&  & Imperative pl. \textit{{}-si} (\textsc{imp.pl}) & §10.3\\
 \textbf{S11} & Tense/Aspect/Mood & Reportative -\textit{ra} (\textsc{rep}) & §11.1\\
&  & Past perfective \textit{{}-l}\textit{i} (\textsc{pst}) & §11.2\\
&  & Past perfective egophoric \textit{-ki} (\textsc{pst.1}) & §11.3\\
&  & Past imperfective \textit{-e} (\textsc{impf}) & §11.4\\
&  & Progressive \textit{-a} (\textsc{prog}) & §11.5\\
&  & Indirect causative \textit{–nula} (\textsc{caus.i)}  & §11.6\\
 \textbf{S12} & Subordination & Temporal \textit{{}-}\textit{tʃ͡i} (\textsc{temp}) & §12.1\\
&  & Epistemic \textit{{}-o} (\textsc{ep}) & §12.2\\
&  & Simultaneous action \textit{-ka} (sim) & §12.3\\
&  & Purposive \textit{-ra} (\textsc{pur}) & §12.4\\
&  & Participial \textit{-ame} (\textsc{ptcp}) & §12.5\\
\lspbottomrule
\end{tabularx}
\textbf{1. S1: Inchoative \textit{–ba} }\textbf{suffix.}


The inchoative suffix is productively used with positional or stative predicates to indicate a dynamic change of state; a state is turned into a process, meaning ‘to become X’. This stress-shifting suffix is exemplified in (1).


   Inchoative \textit{–ba} suffix example

a.   ma    aka-\textbf{bá}{}-tʃ͡a-na-ri 

already    be.sweet-\textbf{Inch}{}-Tr:Pl-Tr-Pst

‘It has been sweetened’

     ‘Ya lo endulzaron’          

    [BF 05 2:56/el]

b.   tamí   rata-\textbf{bá}{}-tʃ͡a-ri    ko’wáami

1sg.subj  be.hot-\textbf{Inch}{}-Tr-Imp  food

‘Heat up the food for me’

‘Caliéntame la comida’        

[LEL 06 4:151/el]


\begin{itemize}
\item \textbf{S2: Transitive suffixes.}
\end{itemize}

The transitive suffixes in S2 are semantically and lexically restricted suffixes of limited productivity. Transitive suffixes \textit{–na} (§2.1) and \textit{{}-tʃ͡a} (§2.2) increase the valence of intransitive change of state predicates (described in Chapter 3, §3.3.3). 

\begin{itemize}
\item \begin{itemize}
\item \textbf{Transitive\textit{–na} }\textbf{suffix.}
\end{itemize}
\end{itemize}

This suffix is stress-shifting. The following examples show transitive derivations with suffix \textit{–na} from intransitive change of state predicates. The intransitive (inchoative) forms are not suffixed (e.g., (2a)), and the corresponding transitive versions require the transitive suffix (e.g., (2b)).


   Transitive \textit{–na} suffix example

    \textit{Basic (intransitive) construction}

a.  ma    tʃ͡iwá-ri    sipútʃ͡a

already    tear-Pst  skirt

‘The skirt already tore’

‘Ya se rompió la falda’        

[SF 07 1:17-21/el]

    \textit{Transitive construction}

b.   á   riwé!    tʃ͡iwa-\textbf{ná}{}-ra!

Aff  leave:Imp:sg  tear-\textbf{Tr}  {}-Pot

    ‘Leave it, you are going to tear it!’

    ‘Déjalo! Lo vas a trozar!’        

    [SF 07 1:17-21/el]

\begin{itemize}
\item \begin{itemize}
\item \textbf{(Pluractional) transitive \textit{–}}\textbf{\textit{tʃ͡}}\textbf{\textit{a} }\textbf{suffix.}
\end{itemize}
\end{itemize}

This stress-shifting suffix is a transitive suffix that can be found with the same change of state predicates that add transitive suffix \textit{–na} (described above). This suffix, historically reconstructible to most \ili{Uto-Aztecan} languages \citep{heath1978uto}, encompasses two senses: 1) an action is performed repeatedly (the succession or discernible, discrete events) or 2) more than one entity is affected by an event. Of limited productivity in Choguita Rarámuri, this suffix can still be found with a pluractional sense in this variety. Compare the transitive form with suffix \textit{–na} in (3b) and the pluractional transitive form with suffix \textit{{}-tʃ͡a} in (3c) of the same base predicate \textit{ku’rí} ‘to turn’. 

   Pluractional transitive \textit{–tʃ͡a} suffix example

  \textit{Basic (intransitive) construction}

a.   nihé  ku’rí-ma

  1sg.subj  turn-Fut:sg  

‘I will turn (on my own axis)’    

‘Voy a dar vuelta (en mi propio eje)’       

[SF 05 1:140/el]

    \textit{Transitive construction}

b.  nihé   ku’ru-\textbf{ná}{}-ma  

1sg.subj   turn-\textbf{Tr}{}-Fut:sg      

‘I will turn it (on its own axis)’  

‘Le voy a dar vuelta (en su propio eje)’    

[BF 05 1:187/el]

    \textit{Pluractional transitive construction}

c.  ma=ni      ku’rí-\textbf{tʃ͡i}{}-ma

already=1sg.subj    turn-\textbf{Tr:Pl}{}-Fut:sg

‘I will now turn it several times (on its own axis)’  

‘Ya le voy a dar muchas vueltas (en su propio eje)’  

[BF 05 1:187/el]

  Non-pluractional uses of transitive \textit{–tʃ͡a} can also be found. An example of this is provided in (4b):

    Non-pluractional form with transitive \textit{–tʃ͡a} suffix  

  \textit{Basic (intransitive) construction}

a.   pe   napítʃ͡i   riké   rata-bá-ma=m     pa

    just  fire  Dub  heat-Inch-Fut:sg=Dem  Cl

    ‘Leaving it like that in the fire it will heat up’

    ‘Así nomás en el fuego se va a calentar’    

    [LEL tx68:7/Text]

  \textit{Transitive construction}

b.   nihé   rata-bá-\textbf{tʃ͡a}{}-ma       ko’-ámi

1sg.subj  heat-Inch-\textbf{Tr.pl}{}-Fut:sg  eat-Ptcp

‘I’m going to heat up the food’

‘Voy a calentar la comida’        

[LEL 06 4:151/el]

\begin{itemize}
\item \begin{itemize}
\item \textbf{Transitive \textit{{}-bu} }\textbf{suffix.}
\end{itemize}
\end{itemize}

There is a third transitivizing suffix, \textit{{}-bu}, which is also unproductive and lexically restricted. This stress-shifting suffix is exemplified in (5):\footnote{Lengthening of the transitive suffix vowel in example (5b) is triggered by the past passive suffix. This effect is discussed below (§9.1) and in Chapter 3 (§3.5.2.3).}

  Transitive \textit{–bu} suffix example

  \textit{Basic (intransitive) construction}

a.   towí  ma     mó-ri

boy  already    go.up-Pst

‘The boy already went up’

‘Ya se subió el niño’          

[BF 06 4:189/el]

    \textit{Transitive construction}

b.   ma=ni    mo-\textbf{búu}{}-ro

  already=1sg.subj  go.up-\textbf{Tr}{}-Pst.pass

  ‘I was already taken up’

    ‘Ya me subieron’          

    [BF 06 4:189/el]

\begin{itemize}
\item \textbf{S3: Applicatives.}

\begin{itemize}
\item \textbf{Applicative \textit{–ni} }\textbf{suffix.}
\end{itemize}
\end{itemize}

The Applicative \textit{–ni} suffix increases the valency of the verb, adding a benefactive argument (‘to do X for Y’). This suffix is unproductive and lexically conditioned by the roots to which it attaches. The contrast between a basic, two-place base predicate and its applicative derivation is exemplified in (6).

   Applicative suffix \textit{–ni} example

    \textit{Basic constrruction (two-place predicate)}

a.   ne   ma  wí-ma      sunú

    1sg.subj  now  harvest-Fut:sg    corn

    ‘I’ll harvest corn now’

    ‘Ya voy a pizcar maíz’        

    [LEL 06 4:151/el]

    \textit{Applicative construction (three-place predicate)}

b.   wí-\textbf{ni}{}-mo=n       orá   ne   jé-ra     sunú

  harvest-\textbf{Appl}{}-Fut:sg=1sg.subj  Cer  1sg.subj  mom-Poss  corn

  ‘I will harvest the corn for my mom’

‘Le voy a pizcar el maiz a mi mamá’      

[BF 06 5:146/el]

\begin{itemize}
\item \begin{itemize}
\item \textbf{Applicative \textit{–si} }\textbf{suffix.}
\end{itemize}
\end{itemize}

The suffix \textit{–si} is another unproductive, lexically conditioned applicative marker that increases the valency of the verb by adding a benefactive argument. A stress-neutral suffix, Applicative \textit{{}-si} is exemplified in (7):

   Applicative suffix \textit{–si} example

    \textit{Basic construction (two-place predicate)}

a.   pá-ka

  throw-Imp:sg

  ‘Throw it!’

‘Tira!’              

[BF 06 5:147/el]

  \textit{Applicative construction (three-place predicate)}

b.   tamí   ku   pá-\textbf{ʃi}{}-ri     pelóta

1sgA  Rev  thow-\textbf{Appl}{}-Imp:sg  ball

‘Throw the ball back at me!’

‘Tírame la pelota de vuelta!’        

[BF 06 5:147/el]

\begin{itemize}
\item \begin{itemize}
\item 
Applicative \textit{–wi} suffix
\end{itemize}
\end{itemize}

{A third Applicative suffix in position S3 is} {\textit{{}-wi}}{, another stress-neutral, unproductive suffix that adds a benefactive argument to a transitive predicate. This suffix is exemplified in (8):}


   Applicative suffix \textit{–wi} example

  \textit{Basic construction (two-place predicate)}

a.   wa’rú   na   atʃ͡á   biré   a’péri=ti   ané    

    big   Prox   sit:Tr   one  lump=1plN  say

    ‘They put (lit. sit) a lot of what we call an \textit{a’péri}  (a lump)’

‘Ponen mucho de lo que le decimos una “moruca” (una bola con todo)’

  [LEL tx19:33/Text]

  \textit{Applicative construction (three-place predicate)}


\begin{itemize}
\item    mi=n     napítʃ͡i   atʃ͡í-\textbf{w}{}-mo     rá   towí
\end{itemize}

/mi=ni    napítʃ͡i   atʃ͡í-\textbf{wi}{}-ma    orá  towí/

2sgA=1sg.subj  fire  sit:Appl-\textbf{Appl}{}-Fut:sg  Cer  boy

‘I will sit down your boy next to the fire’

‘Te voy a sentar al niño cerca de la lumbre’    

[BF 06 6:146/el]


\begin{itemize}
\item \textbf{S4: Causative \textit{–ti} }\textbf{suffix.}
\end{itemize}

The Causative \textit{{}-ti} suffix is a stress neutral suffix that introduces an agent (causer) argument to the argument structure of a predicate. Causativization applies to both intransitive and transitive verbs. In the causative construction exemplified in (9b), the object corresponds to the subject argument of its basic, non-causative counterpart. The introduced agent argument causes the undergoer to perform the activity described by the verbal root.

   Causative suffix example

  \textit{Basic construction}

a.   ne  mi  rimé-ni-ra

 1sg.subj  2sgA  make.tortillas-Appl-Pot 

  ‘I can make you tortillas’

  ‘Yo te hago tortillas’          

  [BF 08 1:161/el]

  \textit{Causative construction}

b.  mi=n    ne   ono-rá    rimée-n-\textbf{ti}{}-ma

2sgA=1sg.subj  1sg.subj  father-Poss  make.tortillas-Appl-\textbf{Caus}{}-Fut:sg 

‘I will make you make tortillas for my dad’

‘Te voy a hacer que le hagas tortillas a mi papá’  

[BF 08 1:161/el]

The Causative suffix has two lexically determined allomorphs, \textit{{}-ti} and \textit{–ri}. The allomorphy is also partially phonologically determined, since there is a phonological process that devoices voiced/lenis consonants after another consonant (a derived environment stemming from stress-conditioned syncope) (cf. Chapter 2, §2.2.4). Examples of the distribution of allomorph \textit{–ti} are provided in (10).

   Phonological distribution of Causative allomorph \textit{{}-ti}  

\textit{Form    Gloss      Unattested}

a.   láan\textbf{{}-ti}{}-ki  ‘bleed\textbf{{}-Caus}{}-Pst:1’  *lán-\textbf{r}i-ki  [SF 05 1:102/el]

  b.  sikirép-\textbf{ti}{}-ki  ‘cut-\textbf{Caus}{}-Pst:1’  *sikirép-\textbf{r}i-ki  [BF 05 1:113/el]

  c.  o’péʃ-\textbf{ti}{}-a  ‘vomit-\textbf{Caus}{}-Prog’  *o’péʃ-\textbf{r}i-a  [BF 05 1:136/el]

  The Causative \textit{–ti} suffix is extremely productive, displaying no restrictions as to the bases to which it can attach.

\begin{itemize}
\item \textbf{S5: Applicative \textit{–ki} }\textbf{suffix.}
\end{itemize}

The Applicative suffix \textit{–ki} (S5) is another productive, stress-neutral suffix. This suffix introduces an additional argument to one-place or two-place predicates. The  argument introduced is a benefactive or malefactive argument,\footnote{The cognates of Choguita Rarámuri Applicative suffixes in the closely related River \ili{Guarijío} (\citealt{miller1996guarijio}) introduce other semantic roles in addition to benefactive/malefactive (e.g., instrumental)). There is however no evidence that the suffix \textit{–ki} or any of the other Applicatives in Choguita Rarámuri introduce semantic roles other than the benefactive/malefactive.}  i.e., the object can be favorably or adversely affected. 

  Applicative suffix \textit{–ki} example

  \textit{Basic construction (two-place predicate)}

a.  ma=n      rata-bá-tʃ͡i-ki    ko’wá-ami     

already=1sg.subj  heat-Inch-Tr:Pl-Pt:1  eat-Ptcp 

‘I already heated up the food’        

    ‘Ya calenté la comida’        

    [BF 08 1:20/el]

 \textit{Applicative construction (three-place predicate)}

b.   ne   mi  ba’wí  rata-bá-tʃ͡-\textbf{ki}{}-ra?

    1sg.subj  2sgA  water  heat-Inch-Tr-\textbf{Appl}{}-Pot

    ‘Shall I heat the water for you?’

    ‘Te caliento el agua?’          

    [BF 08 1:21/el]

  In (11b), the Applicative introduces a benefactive argument as an unmarked object (\textit{mi} ‘2sgA’) to a basic transitive predicate (an argument which would be expressed through a postpositional phrase in a non-applicative construction). I have not found any restrictions in the distribution of this suffix like the ones affecting the Applicative suffixes in S3 (cf. §3 above).

\begin{itemize}
\item \textbf{S6: Desiderative \textit{–nale} }\textbf{suffix.}
\end{itemize}

The disyllabic Desiderative suffix \textit{–nale} is a stress-shifting suffix of agent-oriented modality. Derived from the verb \textit{nakí} ‘want’, it has the meaning ‘X wants to/feels like doing X’, where the ‘wanter’ and the subject of the desideratum predication are correferent (when these two arguments are not correferent, a periphrastic construction must be used). Examples from context are shown in (12).

   Desiderative suffix examples

a.   ne   biré   nijúrka   sebá-nale   ba    

    Int  one  stubbornly  reach-\textbf{Desid}  Cl  

    ‘He really wanted to reach it’

 ‘Lo quería alcanzar a fuerzas’      

[BF 07 tr191/Text]

  b.   a’rí    na   mo’otʃ͡íki  tʃ͡ukúri-ri     tʃ͡api-\textbf{nál}{}-a 

    and   then   \textit{cabecera}  go.around-Pst  grab-\textbf{Desid}{}-Prog

    ‘...and then he was going around near the head wanting to get him’

‘Y entonces andaba por la cabecera queriéndolo agarrar’ 

[LEL 06 tr5/Text]

  Like other disyllabic suffixes in the language, the Desiderative suffix has a ‘short’ monosyllabic allomorph (/\textit{{}-na}/) (cf. Chapter 6). 

\begin{itemize}
\item \textbf{S7: Associated motion \textit{–simi} }\textbf{suffix.}
\end{itemize}

The Associated Motion suffix \textit{–simi} (with short allomorph /\textit{–si}/) is a stress-neutral suffix derived from the free-standing motion verb \textit{simi} ‘go (sg.)’ (cf. Chapter 6, §6.2). This suffix is used when the event encoded by the verb is carried out while in motion (e.g., ‘X goes along doing V’). 

   Associated Motion \textit{–simi} suffix

  a.   we   ko’á-\textbf{simi}

    Int  eat-\textbf{Mot}

    ‘They’re going along eating’

    ‘Van comiendo’          

    [SF 08 1:71/el]

b.   towí  we   nári-\textbf{simi}  bu’utʃ͡ími

    boy  Int  ask-\textbf{Mot}  road

    ‘The boy is going along the road asking a lot of things’

    ‘El niño va preguntando muchas cosas por el camino’ 

    [SF 08 1:148/el]

\begin{itemize}
\item \textbf{S8: Auditory Evidential} \textbf{\textit{–tʃ͡ane}} \textbf{suffix.}
\end{itemize}

The Auditory Evidential \textit{–tʃ͡ane} suffix (with short allomorph /\textit{–tʃ͡a}/) is a productive epistemic modality marker that indicates that the evidence of  the proposition encoded by the predicate has an auditory (i.e. non-visual) source (‘it sounds like X is taking place’). This stress-neutral suffix is exemplified in (14).

  Auditory Evidential suffix example

a.   tʃ͡éti     torí     ma     toré-\textbf{tʃ͡ane}

Dem.pl    chicken  already    cackle-\textbf{Ev}   

‘It sounds like the chiken are already cackling’

‘Ya se oyen cacarear las gallinas’      

[SF 08 1:160/el]

  b.  tʃ͡oní-\textbf{tʃ͡ane}  matʃ͡í

    fight-\textbf{Ev}  outside

    ‘It sounds/it seems like fighting is happening outside’

    ‘Se oye/parece que pelean afuera’      

    [BF 08 1:17/el]

  In these constructions, the source for the evidence is the noise generated by the event itself that the predicate describes (e.g., ‘cackling’ or ‘fighting’ in (14a) and (14b), respectively). The evidence can also come indirectly from another event, as in example (15), where the speaker infers that dancing will take place because of other non-visual cues (i.e., sound of the rattles used in the dance, people talking about starting to dance, etc.). 

    Auditory Evidential, indirect evidence

  nápi   ré  ma    awi-mé-\textbf{tʃ͡ani}

Rel  Dub  already    dance-Fut:sg-\textbf{Ev}

‘It sounds like they are about to dance’

‘Se oye como que van a bailar’      

[SF 07 1:140/el]

\begin{itemize}
\item 
S9: Tense, Aspect, Mood and Voice.


\begin{itemize}
\item \textbf{Past Passive \textit{–ru} }\textbf{suffix.}
\end{itemize}
\end{itemize}

This suffix is a productive, stress-neutral marker that fulfills a passive function, promoting the object argument of the active transitive base to subjecthood, while also encoding past tense. The subject of the active construction is not overtly expressed in the corresponding passive construction. As discussed in Chapter 3 (§3.5.2.3), the past passive suffix trigges lengthening of a preceding stressed syllable. Example (16b) illustrates the past passive sense and concomitant lengthening in the stressed vowel of the base.

   Past passive suffix example

    \textit{Active construction}

a.  tò-s-nale=ni          

take-Mot-Desid=1sg.subj

‘I want to go along taking them (the flowers)’

‘Quiero írmelas llevando (las flores)’      

[BF 06 5:149/el]

    \textit{Passive construction}

  b.  tòo-\textbf{ru}    grabadóra

    take-\textbf{Pst.pass}  recorder

    ‘The recorder was taken’

    ‘Se llevaron la grabadora’        

    [SF 08 1:45/el]

\begin{itemize}
\item \begin{itemize}
\item \textbf{Future passive \textit{–pa} }\textbf{suffix.}
\end{itemize}
\end{itemize}

This is a productive, stress-shifting suffix that concomitantly marks a passive derivation and future tense. The following example illustrates the contrast between a basic active construction (17a) and a future passive construction (17b).


   Future passive suffix example

  \textit{Active construction}

a.   píri=m    orá  tʃ͡ihá-ni-ri  namúti

why=2sgN  make  scatter-Tr-Pst  things

‘Why did you scatter everything?’

‘Por qué desparramaste las cosas?’      

[SF 07 1:17-21/el]

  \textit{Passive construction}

b.   napátʃ͡i  tʃ͡iha-ná-\textbf{ba}    ré

shirts  scatter-Tr-\textbf{Fut.pass}  Dub

‘The shirts will be scattered’  

‘Van a desparramar las blusas’      

[SF 07 1:17-21/el]  


This productive suffix has two allomorphs, with a voiceless and voiced onset (\textit{–pa} or –\textit{ba}). Selection of past passive allomorphs is lexically conditioned but also follows general phonological rules (cf. Chapter 2, §2.2.4, §4 above). 


\begin{itemize}
\item \begin{itemize}
\item \textbf{Medio-passive \textit{–riwa} }\textbf{suffix.}
\end{itemize}
\end{itemize}

This suffix is used in constructions where the actor participant is backgorunded or left unspecified, and the undergoer participant is emphasized. The Medio-passive suffix has two allomorphs, \textit{{}-riwa} and \textit{–wa}. Both allomorphs are stress-shifting. Examples (18b-c) illustrate the medio-passive sense: 

   Medio-passive suffix examples

  \textit{Active construction}

a.  nihé   nurée-ri     margarita   rimé-ni-nula

1sg.subj  oblige-Pst.pass  Margarita  tortillas-Appl-Caus:I

‘They sent me to make tortillas for Margarita’

‘Me mandaron que le hiciera tortillas a Margarita’  

[SF 08 1:134/el]

  \textit{Medio-passive construction}  

b.   anáw-ka   birá   rupu-ná-\textbf{ruwa}          ba

measure-sim  really   tear-Tr-\textbf{Mpass}   Cl

‘While measuring it, it is tore’

‘Se mide y se troza’          

[BF 06 tx1/Text]

c.   nápi  ré  rimé-nu-\textbf{wa}    tʃ͡ém  federíko   remé

Rel  Dub  tortillas-Appl-\textbf{MPass}  Mr.  Federico  tortillas

‘It seems like tortillas are being made for Mr. Federico’

  ‘Como que le hacen tortillas a Don Federico’  

  [SF 07 2:69-72/el]

\begin{itemize}
\item \begin{itemize}
\item \textbf{Conditional passive \textit{–suwa} }\textbf{suffix.}
\end{itemize}
\end{itemize}

The conditional passive suffix \textit{–suwa} is a stress-shifting suffix that is productively used in complex clauses cumulatively expressing a conditional relationship and passive voice. The predicate marked with the conditional passive is the predicate of the protasis clause (describing the condition), not the apodosis (describing the potential result). An active-conditional construction is contrasted with a passive conditional construction in (19).


   Conditional passive suffix example

    \textit{Active construction}

  a.  ne  a  mi  ʃú-n-ti-ki-sa      ró        1sg.subj  Aff  2sgA  sew-Appl-Caus-Appl-Cond  Q

  ‘What if I made you sew her a skirt?’

‘Qué tal si te hago coserle una falda?      

[BF 08 1:28/el]

    \textit{Passive construction}

b.    kátʃ͡i     a’rá   ʃú-bo    ré   riréki     

because:Neg  well  sew-Fut.pl  Dub  bottom

batʃ͡á   ʃú-\textbf{ʃuwa}     ko   ba

first  sew-\textbf{Cond.pass}  Emph  Cl

‘Because we won’t sew it well if it is sewed from the bottom first’

‘Porque no vamos a coserle bien si se empieza a coser de abajo 

 primero’            

[BF 06 sipúcha/Text]

\begin{itemize}
\item \begin{itemize}
\item \textbf{Future singular \textit{–méa, -ma} }\textbf{suffix.}
\end{itemize}
\end{itemize}

There are two future tense suffixes in Choguita Rarámuri: \textit{{}-méa}, for future, singular subject, and -\textit{bo} for future, plural subject. Historically, these suffixes developed from \ili{Proto-Sonoran} \textit{*mi(l)a} ‘go, run, sg’, and \textit{*po} ‘go, run, pl’ \citep[133]{miller1996guarijio}. The future singular suffix has an unstressed allomorph \textit{–ma} and a stressed allomorph \textit{{}-méa}.  Both the unstressed and stressed allomorphs of the Future singular suffix are exemplified in (20).

   Future singular suffix examples   

a.    he   ná=ni     sipútʃ͡a   sipu-tá-\textbf{mo}     rá

it  Prox=1sg.subj  skirt  skirt-Vblz-\textbf{Fut:sg}  Cer

‘I will wear this skirt’

‘Me voy a poner esta falda’        

[BF 07 Sept 6/el]

b.   ma     muku-\textbf{méa}   rajénali

already   die-\textbf{Fut:sg}   sun

‘There will be an eclipse’ (lit. ‘the sun will die’)

‘Va a haber un eclipse’ (lit. ‘se va a morir el sol’)  

[SF 05 2:63/el]


As described in Chapter 3 (§3.6), Choguita Rarámuri has epistemic modality markers that indicate the degree of certainty speakers have towards the actuality of an event. These modal particles are frequently found in future tense constructions, as exemplified in (20a). This example also illustrates the phonological effect that these particles have on the inflected verb’s final vowel, namely vowel deletion. Forms lacking such particles have a neutral interpretation with respect to the speaker’s commitment to the expectation that the event encoded by the predicate will take place or not in the future.


\begin{itemize}
\item \begin{itemize}
\item \textbf{Future plural \textit{–po} }\textbf{suffix.}
\end{itemize}
\end{itemize}

The Future Plural suffix \textit{–po} is a stress-shifting suffix used when the subject is either first or second person plural. Clauses with a third person plural subject may optionally be marked with the Future Plural suffix or the Future Singular suffix. The Future Plural suffix is exemplified in (21). This suffix has two allomorphs, \textit{{}-po} and \textit{{}-bo}.


   Future plural suffix examples

a.   ke   nakíw-\textbf{po}   ru-wá     tamí   wí-ʃi-a   ru-wá

    Neg   allow-\textbf{fut.pl}   say-Mpass   1plA   take-Mot-Prog say-MPass

étʃ͡i   ariwá-ra

Dem   soul-Poss

    ‘It’s said that we can’t let (the korimáka) get to us, because they say it 

goes along taking our souls’

‘Dicen que no hay que dejarnos de él (del korimáka) porque dicen que nos roba el alma’            

[LEL 06 tx5/el]


\begin{itemize}
\item ru-\textbf{bó}    
\end{itemize}

say-\textbf{fut.pl}    

‘They will say something’

‘Van a decir’            

[JH 04 1:27/el]

When used with the first person plural, the construction has a hortative reading (‘let us do X’). The hortative use of the Future Plural suffix is illustrated in (22).

  Examples of the hortative reading of the future plural suffix

a.   ma=ti     ila-rú-\textbf{po}

now=1plN  cactus-gather-\textbf{fut.pl}

‘Let’s gather cactus now’

    ‘Vamos juntando nopales’        

    [SF 08 1:52/el]

b.    ma=ti    potʃ͡í-ti-si-a     iná-r-ti-\textbf{po}?  

already=1plN  jump-Caus-Mot-Prog  go-Caus-Caus-\textbf{fut.pl}

‘Shall we go along making them jump?’

‘Vamos haciéndolo que brinque?’      

[BF 07 2:32/el]


\begin{itemize}
\item \begin{itemize}
\item \textbf{The Motion Imperative \textit{–me} }\textbf{suffix.}
\end{itemize}
\end{itemize}

The Motion Imperative suffix \textit{{}-me} is a stress shifting suffix. It is a productive suffix that often occurs in conjunction with the imperative suffix \textit{–sa} (in position S8). Motion Imperative constructions with the suffix \textit{–me} have the meaning ‘go and do X!’, used for a single addressee. When unstressed, the suffix vowel reduces to \textit{i} or undergoes complete deletion, following the general unstressed vowel reduction and deletion processes operating in the language (cf. Chapter 2, §2.3.1.2).

   Motion imperative construction examples

a.   áa-\textbf{m}{}-sa

  give-\textbf{Mot:Imp}{}-Imp:sg

‘Go give it to her!’

‘Ve y dáselo!’            

[RF 04 1:112/el]

b.   iʃí-\textbf{mi}

  urinate-\textbf{Mot:Imp}

  ‘Go and urinate!’

  ‘Ve a orinar!’            

  [BF 08 1:13/el]

c.   júr-ka     osi\textbf{{}-mé}{}-ra ré    [pr.]

take-IMP   write\textbf{{}-Mot:Imp}{}-Pot   Dub

‘Go take him to see if he writes’

‘Ve y llévalo a ver si escribe’        

[BF 08 1:94/el]

  When there are multiple adressees, the Motion Imperative Construction involves the stress-shifting suffix \textit{–pi} (with stress-shifting allomorph \textit{–bo,} homophonous to the Future Plural allomorph \textit{–bo}), followed by the imperative plural suffix \textit{{}-si}. 

  Motion imperative plural costruction

a.   osi-\textbf{bó}{}-si

write-\textbf{Mot:Imp.pl}{}-Imp.pl      

‘You all go and write’  

‘Vayan a escribir!’          

[BF 05 2:94/el]

b.   tamí  ku  á-ki-\textbf{pi}{}-si

    1sgA  Rev  look.for-Appl-\textbf{Mot:Imp.pl}{}-Imp.pl

    ‘You all go and look for it for me!’

‘Vayan a buscármelo!’        

[BF 08 1:164/el]

\begin{itemize}
\item \begin{itemize}
\item \textbf{Conditional suffix \textit{{}-sa}}.
\end{itemize}
\end{itemize}

This is a productive, stress-shifting suffix used in constructions that express a conditional relationship in the active voice (cf. Conditional Passive suffix description in §9.4 above). The verbal predicate marked with the conditional suffix is the predicate of the protasis clause. \citet[216]{steele1975protoUA} reconstructs the cognate form for \ili{Proto-Uto-Aztecan} as meaning “must/speaker wish”. This stress-shifting suffix is exemplified in (25).

  Conditional suffix examples

a.   we   warín-ami   ní-\textbf{sa}     ko,   á   mahawá

Int   light-Ptcp   Cop-\textbf{Cond}   Emph   Aff   be.affraid

‘If she (the other runner) is really fast, she gets affraid’

‘Si es muy ligera (la otra corredora), sí le tiene miedo’  

                [LEL 06 tx19/Text]

b.  rihé   uku-\textbf{sáa}   ro,   tʃ͡ú   tʃ͡é=timi   rikám    

hail  rain-\textbf{Cond} Q  how  how=1plN  like  

mé-ra?     

scare.away-Pot  

‘And when it would hail? How did you guys scare it away?’

‘Y cuando llovía granizo? Cómo lo espantaban?’  

[SF 07 in 243/Interv]  

\begin{itemize}
\item \begin{itemize}
\item \textbf{Irrealis singular \textit{–me} }\textbf{suffix.}
\end{itemize}
\end{itemize}

The irrealis singular suffix is used in constructions where the speaker has no certainty that a particular event will take place in the future, or if a particular event holds true in a hypothetical or contingent world. This stress-shifting suffix is highly productive (I have not documented any restrictions on its occurrence), and is used when the subject argument is singular. Examples of its use are presented in (26).

   Irrealis singular suffix examples

a.  ko’-nári\textbf{{}-mi}     

   eat-Desid\textbf{{}-Irr:sg}    

  ‘She might want to eat’

‘A lo mejor va a querer comer’      

[SF 08 1:122/el]

b.   basarów\textbf{{}-mi}    ré  ma    ba’arí-o

stroll.around\textbf{{}-Irr:sg}  Dub  perhaps  tomorrow-Ep

‘Perhaps she will take a stroll tomorrow’

‘A lo mejor va a pasear mañana’      

[BF 07 1:150/el]

c.   suku-\textbf{mé}  ré  máo

  scratch-\textbf{Irr:sg}  Dub  perhaps

  ‘Maybe he’ll sratch himself’

  ‘A lo mejor se va a rascar’        

  [SF 08 1:45/el]  

\begin{itemize}
\item \begin{itemize}
\item \textbf{Irrealis plural \textit{–pi} }\textbf{suffix.}
\end{itemize}
\end{itemize}

Irrealis constructions with a plural subject argument are marked with the suffix \textit{–pi}. This suffix is stress-neutral and, like the irrealis singular suffix described above, is highly productive. This suffix has two allomorphs, with a voiced and a voiceless stop onset (\textit{{}-pi} and \textit{–bi}). Examples are shown in (27).

  Irrealis plural suffix examples

  a.  ma     tó-\textbf{bi}    ré  má

    already    bury-\textbf{Irr.pl}  Dub  perhaps

    ‘Maybe they will bury it already’

    ‘A la mejor ya lo van a enterrar’      

    [SF 08 1:3/el]

  b.  ko’-nár-\textbf{pi}     ré=ti     máo

eat-Desid-\textbf{Irr.pl} Dub=1plN  perhaps

‘Perhaps we might want to eat’

‘A lo mejor vamos a querer comer’      

[BF 06 5:140/el]

\begin{itemize}
\item \textbf{S10: Mood.}

\begin{itemize}
\item \textbf{Potential \textit{–ta} }\textbf{suffix.}
\end{itemize}
\end{itemize}

This suffix is used in constructions expressing the possibility of occurrence of an event, ability or wishes (with an optative reading). This is a stress-shifting suffix with two allomorphs, \textit{{}-ta} and \textit{–ra}. Allomorph distribution is lexically and phonologically conditioned, governed by the conditions mentioned above (e.g., §4) and in Chapter 2 (§2.2.4). 

   Potential suffix examples

a.   étʃ͡i   a   máal-\textbf{ta}   ré

Dist   Aff   swim-\textbf{Pot}   Dub

‘Let that one swim!’

‘Déjenlo nadar!’          

[BF 05 1:154/el]

b.   nuru-ría        birá    batʃ͡á   ará    náti-ka   énni-\textbf{ra} 

oblige-MPass   really  first   well     think-sim  go.around-\textbf{Pot}

   ‘They are sent first to go around carefully (lit. thinking well)’

  ‘Primero los mandan a que se cuiden bien’    

  [BF 06 tx48/Text]


\begin{itemize}
\item witʃ͡i-\textbf{rá}!
\end{itemize}

fall-\textbf{Pot}


    ‘You might fall!’



    ‘Te caes!’ (lit. ‘the puedes caer’)      



\begin{itemize}
\item \begin{itemize}
\item \textbf{Imperative singular \textit{–sa} }\textbf{suffix.}
\end{itemize}
\end{itemize}

Imperatives may be marked through the bare stem, but there are also affixal exponents of imperative mood. One of such markers is suffix \textit{–sa}, a productive, stress-shifting suffix. This suffix is exemplified in (29).

   Imperative singular \textit{–sa} suffix example

  a.  ko’-\textbf{sá}!

    eat-\textbf{Imp:sg}

    ‘Eat!’

    ‘Come!’            

  b.  má\textbf{{}-sa}

    run-Imp:sg

    ‘Run!’

    ‘Corre!’            

    [BF 04/11/06/el]

\begin{itemize}
\item \begin{itemize}
\item \textbf{Imperative sg \textit{–ka} }\textbf{suffix.}
\end{itemize}
\end{itemize}

Another imperative suffix used in constructions where the adressee is singular is \textit{–ka}. This stress-shifting suffix is exemplified in (30).

   Imperative singular \textit{–ka} suffix

a.    kíti   nará\textbf{{}-ka}

Neg   cry-\textbf{Imp:sg}      

‘Don’t cry!’  

‘No llores!’            

[BF 05 2:89/el]

  b.  we  simi-\textbf{ká}

 Int  go:sg-\textbf{Imp:sg}

 ‘Go!’

    ‘Ve!’              

\begin{itemize}
\item \begin{itemize}
\item \textbf{Imperative plural \textit{{}-si}}.
\end{itemize}
\end{itemize}

Imperative constructions where the are multiple addressees are distinguished from imperatives with a single addressee with a productive, stress-shifting suffix, \textit{{}-si}. Examples of this suffix are provided in (31).

   Imperative plural suffix examples

  a.  ko-\textbf{sí}    reméke!

    eat-\textbf{Imp.pl}  tortillas

    ‘You all eat tortillas!’

    ‘Coman tortillas!’          


\begin{itemize}
\item 
tamí  ku  riwí-i-\textbf{si}
\end{itemize}

1sgA  Rev  find-Appl-\textbf{Imp.pl}



  ‘You all find it for me!’



  ‘Vayan a encontrármelo!’        



  [BF 08 1:16/el]



\begin{itemize}
\item \textbf{S11: Tense, aspect, mood.}

\begin{itemize}
\item \textbf{Reportative suffix \textit{{}-ra}}.
\end{itemize}
\end{itemize}

The reportative suffix is an evidential suffix that indicates that the speaker’s source of information is hearsay. This productive marker, also used in direct quotation constructions, is a stress-neutral suffix which is added to the dependent verb of the complex sentence. When the notional subjects are correferential, the dependent verb is marked with the same referent reportative \textit{{}-ro} suffix (32a-b). When the notional subjects are not correferential, the dependent verb suffixes the different referent reportative \textit{–ra} suffix (32c-d).

  Reportative constructions with same and different referents

  \textit{Same referent}

a.        á    birá   ko        aní      magre  nehé  amatʃ͡í-ko-\textbf{ro}    ruá

Aff   really   Emph  say  nuns  1sg.subj  pray-Appl-\textbf{Rep} say

‘The nuns\textsubscript{i} say that they\textsubscript{i} prayed for me’

‘Las monjas\textsubscript{i} dicen que (ellas\textsubscript{i}) me rezaron’

b.   manueli   ko       we   birá    rikú-\textbf{ro}                       ru

Manuel   Emph    Int   really   get.drunk:sg-\textbf{Rep}   say

‘Manuel\textsubscript{i} says he\textsubscript{i} got drunk’

‘Manuel\textsubscript{i} dice que (él\textsubscript{i}) se emborrachó’ 

  \textit{Different referent}

c.   á   birá   oká=m   tʃ͡aní-a     ne   ka   hémi   

    Aff  really  many=Dem  sound-Prog  Int  ka  here

    isimáta-\textbf{ra}   ruá   tʃ͡abé

    pass:Pl-\textbf{Rep}  say  before

    ‘Many people\textsubscript{i} say that they\textsubscript{j} used to pass through here long time ago’

‘Muchas personas\textsubscript{i} dicen que por aquí pasaban\textsubscript{j} mucho antes’

  [LEL 07 tx223/Text]

d.   tʃ͡iná  ba   étʃ͡i    birá    tòo-\textbf{ra}                ruá  

there   Cl   Dist   really  take:Pst.pass-\textbf{Rep}     say  

ariwá-ra  ba

soul-Poss   Cl

‘They\textsubscript{i} say that that one\textsubscript{j} got his soul stolen there’

‘Cuentan\textsubscript{i} que a ese\textsubscript{j} ahí le llevó el alma’     

[BF rihói mukúri 6/Text]

e.  nápu   riká   rá=timi   we   ukú\textbf{{}-ra}   ruá   ní-am 

  Rel  like  Cer=2plN  Int  rain\textbf{{}-Rep} say  Cop-Ptcp

tʃ͡abée   ko   ba   ní 

before  Emph  Cl  Cl

‘So you all say that it used to rain a lot ling time ago’

‘Pues así como dicen ustedes que llovía mucho antes’

  [SF 07 in243/Interv]            

  This switch-reference system is restricted, as it is not generalized to all constructions involving dependent clauses in Choguita Rarámuri.

\begin{itemize}
\item \begin{itemize}
\item \textbf{Past perfective suffix \textit{{}-}}\textbf{\textit{li}}
\end{itemize}
\end{itemize}

The past perfective is marked by the suffix \textit{–li,} a stress-neutral suffix. The past perfective both situates the event in a point prior to the time of the speech act and indicates that the event has been completed. Examples of this construction are given in (33).

   Past perfective suffix examples

a.   a’rí   ko   ma     birá   ʃiné-ami   wiká   sí-\textbf{ri}   

and  Emph  already     really  every-Ptcp  many  arrive\textbf{{}-Pst}

tʃ͡oná   étʃ͡i   ná   étʃ͡i   rihói   bitéritʃ͡i

there  Dem  Prox  Dem  man  house

‘And then everybody arrived there, to that man’s house’  

 ‘Y ya llegaron todos ahí a la casa de ese señor’  

[LEL 06 tx32/Text]

b.   he   ané     aní-ʃi-a   nawá-\textbf{ri}     étʃ͡i    namú nirá 

    it  say:Appl  say-Mot-Prog  arrive-Pst  Dem  relative

ʃuwá         ba  á       ruwé-\textbf{ri}    

everybody  Cl  Aff  say:Appl-\textbf{Pst}

    ‘He arrived saying that, that relative, he told everybody’

    ‘Llegó diciendo un familiar, les dijo a todos’    

    [LEL 06 tx5/Text]

\begin{itemize}
\item \begin{itemize}
\item \textbf{Past perfective 1st person \textit{–ki} }\textbf{suffix.}
\end{itemize}
\end{itemize}

In past perfective constructions when the subject (34a) or object (34b) is first person, either singular or plural, the suffix used is \textit{{}-ki}. 

   Past perfective 1\textsuperscript{st} person suffix examples

a.    mi   bitʃ͡é=ni   karí   pitʃ͡í-nula   nuré-\textbf{ki}   ró

  2sgA  turn=1sg.subj  house  sweep-Caus:I  oblige\textbf{{}-Pst:1}  say

  ‘I told you to sweep the house!’  

‘Te dije que barrieras la casa!’      

[BF 06 4:145/el]

b.   hitó ...  étʃ͡i   tamí   úr\textbf{{}-ki}     ri’réti

yes  Dem  1sgA  take-\textbf{Pst:1}  down

‘Yes, isn’t it true?....he took me down (the river)’

‘Verdad?...Él me llevó para abajo’      

[FL 06 in61/Interv]

There is speaker variation with respect of this use, but \textit{–ki} is mainly used when either the subect or object is first person. Some speakers use this suffix in constructions encoding a conjunct person (first person in declarative clauses (as in (34) above), and the addressee in questions (as in (35) below).

   Conjunct use of suffix \textit{{}-ki}

a.   kabó   mi   rará\textbf{{}-ki}   sapáto

when   2sgN   buy\textbf{{}-Pst:1}   shoes

‘When did you buy the shoes?’

‘Cuándo compraste los zapatos?      

[SF 05 1:74/el]

\begin{itemize}
\item \begin{itemize}
\item \textbf{Imperfective \textit{{}-e}}.
\end{itemize}
\end{itemize}

In contrast to the past perfective (discussed above), the imperfective emphasizes the internal duration of the event depicted by the predicate. Choguita Rarámuri imperfective constructions encode an incomplete or habitual event that takes place over a period of time. The imperfective is marked  with the stress-neutral suffix \textit{–e}, a marker which does not display any allomorphy or occurence restrictions, nor does it trigger any phonological effects on the base to which it attaches. Due to the general process of post-tonic vowel reduction, this suffix is realized as \textit{{}-i}. Examples are provided in (36).


   Imperfective suffix examples

a   naparí   ke   tʃ͡o   narú\textbf{{}-i}     ko   sekundaria   ba

Rel   Neg   yet   exist\textbf{{}-Impf}   Emph   secondary   Cl

‘When it didn’t use to be any secondary school yet’

‘Cuando todavía no había secundaria’    

[SF 06 tx12/Text]


\begin{itemize}
\item    awí-si-nir\textbf{{}-i}
\end{itemize}

dance-Mot-Desid-\textbf{Impf}

    ‘She wanted to go along dancing’

    ‘Quería irse bailando’          

    [SF 07 2:72-73/el]

c.  húm-tʃ͡an-\textbf{i}    

            take.off.pl-Ev\textbf{{}-Impf}

    ‘It sounded like they were taking off’

    ‘Se oía como que se arrancaban’      

    [SF 07 1:7/el]


\begin{itemize}
\item \begin{itemize}
\item \textbf{Progressive \textit{–a} }\textbf{suffix.}
\end{itemize}
\end{itemize}

The progressive is encoded by the stress-neutral suffix \textit{–a}, and it indicates that the event described by the predicate is an ongoing process which is independent of time reference. Uses of this marker are exemplified in (37).


   Progressive suffix examples

  a.   a’rí    tʃ͡ihónsa  nári  witʃ͡ó-nula   ma  

and   then     then    wash-Caus.I   also   

nuru-ría   witʃ͡ó\textbf{{}-a}

oblige-Mpass  wash\textbf{{}-Prog} 

‘And then they are also sent to wash clothes’

‘Y también las mandan a lava ropa’      

[BF 06 tx48/Text]

b.   we   birá=ti     we   kanír-ami   hu   tamuhé ko   na

    Int   really=1plN   Int   happy-Ptcp   Cop   1plN    Emph   Dem

umukí     rowé-ti\textbf{{}-a},     iwé   rowé-ti\textbf{{}-a},     kúutʃ͡i 

    women    race-Caus\textbf{{}-Prog}  girls  race-Caus\textbf{{}-Prog}  small

kúruwi     ma   rarahíp-ti\textbf{{}-a}  

    children  also  race-Caus\textbf{{}-Prog}

‘We like it a lot indeed, to make women, girls and also young children     run a race’

‘Nos gusta mucho hacer correr a las mujeres, a las niñas y a los niños chiquitos’            

[LEL 06 tx19/Text]

\begin{itemize}
\item \begin{itemize}
\item \textbf{Indirect Causative suffix \textit{–nula}}.
\end{itemize}
\end{itemize}

In indirect causative constructions, a causer exerts indirect manipulation on a causee which retains certain degree of autonomy. Indirect causation is expressed through complement clause constructions, where the lower predicate is marked with the suffix      \textit{{}-nula}. This stress-neutral suffix is derived from the independent verb \textit{nure} ‘to oblige, to force’, which is often the main predicate (e.g., (38a)). The main predicate can be inflected with any tense or aspect, but the lower predicate marked with \textit{–nula} is closed to further suffixation. Examples of the indirect causative are provided in (38).

   Indirect Causative suffix examples

\begin{itemize}
\item    a’rí   tʃ͡ihónsa   ko        ma    pe   otʃ͡éri-sa  ko   and  then    Emph  already    little  grow-Cond  Emph
\end{itemize}

nuru-ría  ba’wí   tú\textbf{{}-nula}      

oblige-MPass  water  bring-\textbf{Caus.I}

 ‘And then when they grow a little they are ordered to bring water’

‘Y entonces ya cuando crecen más los mandan a traer agua’

[BF 06 tx48/Text]

b.   ma=n        húa-ki               rarí-\textbf{nula} tiéndatʃ͡i

already=1sg.subj   send-Pst.1  buy-\textbf{Caus.I}   store

‘I already sent him to the store to buy’ 

‘Ya lo mandé comprar a la tienda’      

[BF 06 2:48/el]


  The suffix \textit{–nula} has a monosyllabic allomorph \textit{{}-na}. The details of allomorph distribution and conditions of selection are addressed in Chapter 6.



\begin{itemize}
\item \textbf{S12: Subordination.}

\begin{itemize}
\item \textbf{Temporal suffix \textit{{}-tʃ͡i}}.
\end{itemize}
\end{itemize}

This morpheme is a stress-neutral suffix added to predicates of adverbial clauses which encode a temporal relation between two events (translated into English as ‘when’ clauses). The base for affixation of this suffix is a verb inflected for progressive aspect. The following examples illustrate the use of this suffix.

   Temporal suffix examples

a.   nápu   riká   omáwiri   ná=m     omowá-ru-a-\textbf{tʃ͡i}

Rel  like  parties    then=Dem  party-MPass-Prog-\textbf{Temp}

 ‘Like with parties, when parties are made’

‘Así como las fiestas, cuando hacen fiesta’    

[SF 06 tx12/Text]

b.    ne     kreelitʃ͡i  ʃi-méa     ma   ʃuwíb-a-\textbf{tʃ͡i}       1sg.subj  Creel    go.sg-Fut.sg  already  finish-Prog-\textbf{Temp}

hé   ná   tarári

it  this  week

‘Voy a ir a Creel cuando acabe esta semanana’

\begin{itemize}
\item \begin{itemize}
\item \begin{itemize}
\item \textbf{Epistemic \textit{–o} }\textbf{suffix.}
\end{itemize}
\end{itemize}
\end{itemize}

The Epistemic modality suffix marks lower predicates of complement clauses of main predicates that express a psychological or mental state, like ‘think’, ‘dream’, ‘sing’ or ‘say’. The use of this suffix is exemplified in the examples in (40).

   Epistemic suffix examples


a.    rimú-i=ni     náp=tim   noká-\textbf{o}



dream-Impf=1sg.subj   Rel=2plA   move-\textbf{Ep}



‘I used to dream that you all were moving’



‘Yo soñaba que ustedes se movían’      


  b.   a’rí   na      kochi-ká      bu’í-r-\textbf{o}       mayé-ri

and   then   sleep-sim   lay.down.sg-Pst-\textbf{Ep}   think-Pst

‘And then he thought he was asleep (laid down sleeping)’

‘Nomás que pensó que estaba dormido’    

[LEL 06 tx5/Text]

\begin{itemize}
\item \begin{itemize}
\item \begin{itemize}
\item \textbf{Simultaneous actiona \textit{–ka} }\textbf{suffix.} 
\end{itemize}
\end{itemize}
\end{itemize}

The suffix –\textit{ka} occurs in subordinate clauses and marks a non-finite verbal construction denoting an ongoing event which occurs simultaneous to another event. This stress-shifting suffix is exemplified in (41):

   Simultunaeous action suffix examples

a.    púra   ko       birá     niwa-ría  nári   birí-n-\textbf{ka}        

    belt    Emph   really  make-MPass   then   roll.up-Tr-sim 

    batʃ͡á  ba   biré    ta   kuʃí-ti     ba

first   Cl   one    Det   stick-Instr   Cl

    ‘The belt is made by rolling it up first with a stick’

    ‘La faja se hace enrollándolo primero con un palo’  

    [BF 06 tx1/Text]

b.   a’rí   na      kochi-\textbf{ká}      bu’í-l-\textbf{o}       mayé-li

and   then   sleep-sim   lay.down.sg-Pst-\textbf{Ep}   think-Pst

‘And then he thought he was asleep (laid down sleeping)’

‘Nomás que pensó que estaba dormido’    

[LEL 06 tx5/Text]

\begin{itemize}
\item \begin{itemize}
\item \begin{itemize}
\item \textbf{Purposive \textit{–ra} }\textbf{suffix.}
\end{itemize}
\end{itemize}
\end{itemize}

The purposive suffix \-\textit{–ra} is a stress-neutral suffix which derives a noun from a finite verb inflected for progressive aspect. The purposive indicates that the derived noun is an instrument or means involved in carrying out the event described by the predicate. This suffix is not limited to a few lexical items, and may be productively added to any finite verb inflected for progressive aspect.The forms in (42) exemplify this nominalization process.

   Purposive suffix examples


a.   pó-a\textbf{{}-ra}    



cover-Prs-\textbf{Purp}  



‘Lid’ (lit. ‘for covering’)



‘Tapadera’ (lit. ‘para tapar’)        



[SF 07 in242/Text]


  b.  osí-a\textbf{{}-ra}

\textbf{ }write-Prog-\textbf{Purp}

    ‘Pen’ (lit. ‘for writing’)

    ‘Pluma’ (lit. ‘para escribir’)


\begin{verbatim}%%move bib entries to  localbibliography.bib
\end{verbatim}