\chapter{Stress}
\label{chap: word prosody}

The word prosodic system of Choguita Rarámuri is complex, involving both stress and tone. While the distribution of tone is dependent on the location of stress, these are phonologically independent systems that are encoded through independent acoustic means (\citealt{caballero2015tone}). These systems, referred to as ``hybrid'' in the literature, have until relatively recently been understudied (cf. \citealt{remijsen2001dialectal}, \citealt{remijsen2002lexically}, \citealt{remijsen2005stress}) and are well attested in other \ili{Southern Uto-Aztecan} languages in addition to Choguita Rarámuri (this aspect of the Choguita Rarámuri word-prosodic system is discussed in more detail in \chapref{chap: tone and intonation} and \chapref{chap: prosody}). This chapter addresses the main phonological properties of the Choguita Rarámuri stress system, including the stress properties of morphemes and the morphological factors that govern its distribution. The system exhibits a complex interplay between phonological and morphological factors, which are addressed in \chapref{chap: nominal morphology}, \chapref{chap: verbal morphology} and \chapref{chap: prosody}, in addition to this chapter.

This chapter is structured as follows. §\ref{subsec: stress and stress-dependent phenomena} provides a basic description of the acoustic correlates and distributional properties of stress. §\ref{subsubsec: stress-based vowel reduction and deletion} addresses stress-dependent phonological phenomena of vowel reduction and deletion. §\ref{sec: stress properties of roots, stems and suffixes} provides a detailed description of the stress properties of roots and suffixes. Finally, §\ref{subsubsec: initial three-syllable stress window} addresses an overarching phonological restriction on stress distribution to an initial three-syllable stress window and how this restriction, together with morphological stress rules, yields stress alternations in the language.

\section{Acoustic correlates and distributional properties}
\label{subsec: stress and stress-dependent phenomena}

Stress is defined cross-linguistically as increased prominence associated with one or more syllables in a word (\citealt{gordonvanderhulst2018stress}). Choguita Rarámuri exhibits phonetic and phonological properties of languages with word level stress (referred to as ``stress-accent'' languages in \citealt{hyman2006word}). These properties include \textit{culminativity} (each lexical word has \textit{at most} one syllable which carries the highest degree of prominence) and \textit{obligatoriness} (each lexical word has \textit{at least} one syllable that carries the highest degree of prominence) (\citealt{hyman1977nature}, \citealt{hyman1978tone}, \citealt{beckman1986stress}, \citealt{hayes1995metrical}, \citealt{hyman2006word}). While these two criteria are also used to define languages characterized as ``pitch-accent'' in the literature, Choguita Rarámuri displays further phonetic and phonological properties that are unique to stress systems, including increased phonetic duration of segments in stressed syllables and reduction of unstressed vowels (fewer vocalic contrasts are realized in unstressed syllables) (the details of which will be discussed in §\ref{subsubsec: stress-based vowel reduction and deletion}). For general discussion about the distinction between ``stress(-accent)'' systems and ``pitch-accent'' systems, see \citet{poser1984phonetics}, \citet{hyman1991autosegmental}, \citet{hyman1977nature}, \citet{hyman2001tone}, and \citet{inkelas1988serbo}. In this description, I follow a property-driven approach to prosodic typology (as advocated in \citealt{hyman2006word} and \citealt{hyman2009not}), which identifies canonical properties of languages and tone languages, but dispenses with the notion of ``pitch-accent'' as a prosodic type.

In terms of its distribution, stress is underlyingly present for at least some morphemes in this language, and can create lexical contrasts. The examples in (\ref{ex: stress minimal pairs}) show stress minimal pairs.

%check tones
\ea\label{ex: stress minimal pairs}
{Choguita Rarámuri stress minimal pairs}

    \ea[]{
    \doublebox{\textit{ˈmúri}}{‘basket’, ‘canasta’}\\
}
        \ex[]{
        \doublebox{\textit{muˈrí}}{‘turtle’, ‘tortuga’}\\
    }
    \medskip
            \ex[]{
            \doublebox{\textit{ˈéka}}{‘close it!’, ‘cierra!’}\\
        }
                \ex[]{
                \doublebox{\textit{eˈká}}{‘wind’, ‘viento'}\\
            }
                \medskip
                    \ex[]{
                    \doublebox{\textit{ˈmútʃ͡i}}{‘baby’, ‘bebé’}\\
                }
                        \ex[]{
                        \doublebox{\textit{muˈtʃ͡í}}{‘vagina’}\\
                    }
                        \medskip
                            \ex[]{
                            \glt    \doublebox{\textit{ˈkôt͡ʃi}} {‘pig’, ‘cerdo’\footnote{This is a loanword from Northern Mexican \ili{Spanish} \textit{cochi} ‘pig’ (which in turn derives from standard Mexican \ili{Spanish} \textit{cochino} ‘pig’).}}\\
                            \glt    \corpuslink{co1137[04_058-04_073].wav}{MDH co1137:4:05.8}\\
                        }
                                \ex[]{
                                \glt    \doublebox{\textit{koˈt͡ʃî}}{ ‘dog’, ‘perro'}\\
                                \glt    \corpuslink{tx152[06_001-06_036].wav}{SFH tx152:6:00.1}\\
                            }
    \z
\z

As discussed in \chapref{chap: verbal morphology} and \chapref{chap: prosody}, stress in Choguita Rarámuri may also be the sole exponent of a morphological category (e.g., stress shifts may encode imperative mood, denominalization and applicative derivation). As shown in this chapter, the stress system of Choguita Rarámuri is in part phonologically determined and partly morphologically-conditioned, as attested in other lexical stress systems.

There is no evidence for secondary stress in Choguita Rarámuri. That is, there is no impressionistic evidence of rhythmic prominence, though, as discussed in §\ref{sec: prosodic constraints on morphological shapes}, there may be phonological evidence of phenomena sensitive to metrical structure in the language.\footnote{A metrical analysis is provided in \chapref{chap: prosody}.} Stress is left-aligned and assigned in the first (\ref{ex: stress on 1st, 2nd or 3rda}), second (\ref{ex: stress on 1st, 2nd or 3rd}b--c) or third (\ref{ex: stress on 1st, 2nd or 3rd}d--e) syllable of the word.

%Check tones
\ea\label{ex: stress on 1st, 2nd or 3rd}
{Left alignment of stress}

    \ea[]{
    \textit{ˈket͡ʃísinale}\\
    \gll    \textit{ˈ}ket͡ʃí-si-nale  \\
            chew-\textsc{mot-desid}\\
    \glt   `S/he wants to go along chewing.’\\
    \glt    `Quiere ir masticando.'  < SFH 08 1:146/el >\\
}\label{ex: stress on 1st, 2nd or 3rda}
        \ex[]{
        \textit{poˈt͡ʃípo}\\
        \gll    poˈt͡ʃí-po\\
                jump-\textsc{fut.pl}\\
        \glt    ‘They will jump.’\\
        \glt    `Van a brincar.'   < SFH 05 1:69/el >\\
    }\label{ex: stress on 1st, 2nd or 3rdb}
            \ex[]{
            \textit{poˈt͡ʃítisima}\\
            \gll    poˈt͡ʃí-ti-si-ma \\
                    jump-\textsc{caus-mot-fut.sg}\\
            \glt    ‘S/he will go along making them jump.’\\
            \glt    `Va a ir haciéndolos que brinquen.'\\
        }\label{ex: stress on 1st, 2nd or 3rdc}
               \ex[]{
               \textit{amaˈt͡ʃía}\\
                \gll    amaˈt͡ʃí-a\\
                        pray-\textsc{prs}\\
                \glt    `S/he is praying.’\\
                \glt    `Está rezando.' \corpuslink{el655[02_178-02_193].wav}{BFL el655:2:17.8}\\
            }\label{ex: stress on 1st, 2nd or 3rdd}
                    \ex[]{
                    \textit{amaˈt͡ʃísmo ˈlá}\\
                    \gll    amaˈt͡ʃí-si-ma oˈlá\\
                            pray-\textsc{mot-fut.sg} \textsc{cer}\\
                    \glt   `S/he will go along praying.'\\
                    \glt    `Va a ir rezando.' \corpuslink{el655[02_306-02_325].wav}{BFL el655:2:30.6}\\
                }\label{ex: stress on 1st, 2nd or 3rde}
    \z
\z

Left alignment of stress in stress systems is a pattern that is relatively uncommon cross-linguistically, but attested in all {Uto-Aztecan} languages with the exception of \ili{Nahuatl} language varieties \parencite{munro1977towards}.\footnote{\ili{Nahuatl} varieties have developed penultimate fixed stress as a Mesoamerican areal feature \citep{munro1977towards}.} Left aligned stress is also an areal trait of Native North American languages (\citealt{rice2010accent}, \citealt{caballero2020oxford}). Based on the recurrence of both second-mora and second-syllable accentual systems, \citet{munro1977towards} reconstructs a second mora stress system for \ili{Proto-Uto-Aztecan}.

Given that Choguita Rarámuri has both stress and tone, a relevant question to ask is how these systems may be encoded acoustically (\citealt{remijsen2002lexically}, \citealt{remijsen2005stress}). Stress may be encoded through a variety of language-specific acoustic cues (\citealt{gordonvanderhulst2018stress}), including increased duration, higher fundamental frequency (f0), greater intensity, and changes in spectral tilt (\citealt{roettger2017methodological}). Languages with both stress and tone tend to preclude deployment of f0 as an exponent of stress, restricting use of fundamental frequency to encoding of tonal contrasts (\citealt{remijsen2005stress}, \citealt{guion2010word}). Analysis of acoustic measures in Choguita Rarámuri reveals that stress has the following phonetic exponents in this language:

\ea\label{ex: Phonetic exponents of Choguita Rarámuri stress}
{Phonetic exponents of Choguita Rarámuri stress}

    \ea[]{
    Increased phonetic duration of stressed vowels.\\
}
        \ex[]{
        Augmentation of onsets in stressed syllables.\\
    }
            \ex[]{
            Greater intensity of stressed vowels.\\
        }
    \z
\z

The acoustic correlates of stress and tone are examined in \citet{caballero2015tone} based on the analysis of data obtained through structured elicitation with seven native speakers. The rest of this section summarizes the main results of that study regarding the acoustic encoding of stress in this language (interested readers are referred to this reference for details of data, measures and quantitative results). For the analysis of stress, acoustic correlates were evaluated based on a set of controlled data selected to contrast pretonic (unstressed) vowels with stressed vowels of identical quality in matching segmental contexts. These forms comprise a tonally heterogeneous set and are exemplified in (\ref{ex: stressed vs. unstressed vowel pairs in matching phonological contexts}).

\ea\label{ex: stressed vs. unstressed vowel pairs in matching phonological contexts}
{Stressed vs. unstressed (pre-tonic) vowel pairs in matching phonological contexts}

    \ea[]{
    \glt    \doublebox{\textit{ˈtô-ma}}{bury-\textsc{fut.sg}}\\
    \glt    \doublebox{\textit{to-ˈmêa~}}{take-\textsc{fut.sg}}\\
    }
        \ex[]{
        \glt    \doublebox{\textit{aˈná}\textit{tʃ͡a-li}}endure-\textsc{pst}{}\\
        \glt    \doublebox{\textit{anaˈt͡ʃâ-ma}}{endure-\textsc{fut.sg}}\\
    }

            \ex[]{
            \glt    \doublebox{\textit{ˈt͡ʃùta-li}}{sharpen-\textsc{pst}}\\
            \glt    \doublebox{\textit{t͡ʃuˈtâ-ma}}{sharpen-\textsc{fut.sg}}\\
        }

                \ex[]{
                \glt    \doublebox{\textit{ˈsóma-li}}{wash.head-\textsc{pst}}\\
                \glt    \doublebox{\textit{soˈmâ-ma}}{wash.head-\textsc{fut.sg}}\\
            }

                    \ex[]{
                    \glt    \doublebox{\textit{niˈwì-ma}}{have.wedding-\textsc{fut.sg}}\\
                    \glt    \doublebox{\textit{awiˈmêa}}{dance-\textsc{fut.sg}}\\
                }
    \z
\z

Segment duration of both vowels and consonants is longer in stressed syllables than in unstressed ones: the central duration value for unstressed vowels is 57 ms, with stressed vowels being about 55\% longer (at around 86 ms). The durations of onset consonants are similarly longer in stressed syllables than in unstressed syllables (the central duration value for unstressed onsets is 106 ms, and the stressed onsets are 28\% longer, around 136 ms). As discussed in \citet{caballero2015tone}, increased segmental duration associated to the encoding of stress in Choguita Rarámuri is related to patterns of vowel reduction and deletion in this language (and described in §\ref{subsubsec: stress-based vowel reduction and deletion} below).

In terms of intensity, stressed vowels have higher average intensity than unstressed vowels: the central intensity value of the stressed vowels is 1.9 dB greater than that of the unstressed vowels.\footnote{The parameters considered to assess the role of intensity were two measures of spectral tilt, low-band H1-H2 and mid-band H1-A2. For both of these measures, a lower value indicates greater vocal effort (indicated through a less shallow decline in spectral intensity), an effect expected for stressed vowels. This expectation was borne out for the Choguita Rarámuri data, though the results were not statistically significant for the mid-band H1-A2 measure, which also exhibited a significant amount of inter-speaker variation. Specifically, the results suggest differences between female and male speakers, but comprehensive investigation of the role of sociolinguistic factors in the acoustic realization of prosody would be necessary in order to determine whether there is a significant pattern in the Choguita Rarámuri data.}

Higher f0 is one phonetic cue to stress in many languages, but it is not directly associated with stress in Choguita Rarámuri, instead realizing tonal contrats in this language. Thus, and as observed for other languages with both stress and tone, stress in Choguita Rarámuri involves dedicated phonetic exponents for each of these systems, contributing to maximizing the contrast between the two.

Finally, while there was no significant association of f0 with stressed positions, there was also a high degree of variation among speakers in terms of the effect of stress on f0: while some speakers did exhibit a trend toward higher f0 in stressed syllables, no correlation could be detected for other speakers. Speakers also showed differences on how consistently intensity increased as an effect of stress. Speaker-variation in the realization of prosodic contrasts is a topic that needs to be investigated in more depth in this language variety.

\section{Stress-based vowel reduction and deletion}
\label{subsubsec: stress-based vowel reduction and deletion}

An important diagnostic of stress in Choguita Rarámuri is patterns of vocalic reduction and deletion. As we have seen in §\ref{sec: vowels}, Choguita Rarámuri contrasts five cardinal vowel qualities in stressed syllables ([a, e, i, o, u]). In unstressed syllables, however, these vowel quality contrasts are often collapsed, reflecting a reduction of the phonetic space. Specifically, vowel height contrasts are neutralized, where mid-front and low-central vowels raise to [i]. Neutralization of vowel height contrasts in unstressed syllables as attested in Choguita Rarámuri is the most common cross-linguistic pattern of unstressed vowel reduction (see \citealt{barnes2002positional,barnes2004vowel}).

There are three distinct patterns or degrees of vowel reduction. In the first pattern, /e/ optionally raises to [i] both pre-tonically and posttonically. In the second pattern, non-final posttonic /a/ and /o/ optionally raise to [i].  In the third pattern, high vowels optionally reduce to schwa posttonically. These patterns are schematized in (\ref{ex: unstressed vowel reduction patterns}).

\ea\label{ex: unstressed vowel reduction patterns}
{Unstressed vowel reduction patterns}

   \begin{itemize}
       \item    /e/     →   [i] in pre-tonic and posttonic syllables \\
       \item    /a/, /o/   →   [i] in non-final, posttonic syllables \\
       \item    /i/, /u/    →   [ə] in non-final, posttonic syllables \\
   \end{itemize}
\z

\tabref{tab:unstressed-vowels} lays out the surface realization of underlying vowels in pre-tonic, posttonic non-final and posttonic final position. Since all vowel qualities are licensed in stressed position, this chart only considers unstressed vowels.

%\pagebreak

\begin{table}
\caption{Surface realization of unstressed vowels}
\label{tab:unstressed-vowels}

\begin{tabularx}{\textwidth}{XXll}
\lsptoprule
& \textbf{Pre-tonic}  & \textbf{Posttonic non-final}  & \textbf{Posttonic final}\\
  \midrule
  /i/ & i & ə / i & i\\
  /e/ & i / e & i / e & i / e\\
  /a/ & a & i / a & a\\
  /o/ & o & i / o & o\\
  /u/ & u & ə / u & u\\
 \lspbottomrule
\end{tabularx}
\end{table}
%\hspace{3cm}

These patterns of unstressed vowel reduction are addressed below in §\ref{subsubsec*: unstressed mid front vowel reduction}, §\ref{subsubsec*: unstressed high vowel reduction to schwa}, and §\ref{subsubsec*: stress-conditioned vowel deletion}.

\subsection{Stress-conditioned vowel reduction patterns}

\subsubsection{Unstressed mid front vowel reduction to [i]}
\label{subsubsec*: unstressed mid front vowel reduction}

The first pattern of unstressed vowel reduction in Choguita Rarámuri involves mid front vowels reducing both pre-tonically and posttonically. Pre-tonic vowel reduction of mid front vowels is robust. Forms with surface pre-tonic \textit{e} are attested, but these are infrequent. Some examples are presented in (\ref{ex: optional pretonic mid, front vowel height neutralization}), where alternative forms with pretonic \textit{i} and pretonic \textit{e} are given. Relevant vowels are in bold face.

\ea\label{ex: optional pretonic mid, front vowel height neutralization}
{Optional pretonic mid, front vowel height neutralization}

    \ea[]{
    [n\textbf{i}ˈhê]  {\textasciitilde} [n\textbf{e}ˈhê] \\
    /n\textbf{e}ˈhê/  \\
    ‘I’\\
    ‘yo’ \corpuslink{el1278[03_200-03_213].wav}{JLG el1278:3:20.0},\corpuslink{el862[01_426-01_444].wav}{FMF el862:1:42.6}\\
}
        \ex[]{
        [r\textbf{i}ˈpòpa]  {\textasciitilde} [r\textbf{e}ˈpòpa] \\
        /r\textbf{e}ˈpòpa/ \\
        ‘back’\\
        ‘espalda’ < SFH 07 2:65/el >\\
    }
            \ex[]{
           [b\textbf{i}ˈnè]  {\textasciitilde} [b\textbf{e}ˈnè] \\
            /b\textbf{e}ˈnè/   \\
            ‘learn’\\
            ‘aprender’ \corpuslink{tx1[00_451-00_484].wav}{BFL tx1:0:45.1}, \corpuslink{co1136[09_047-09_074].wav}{MDH co1136:9:04.7}\\
        }
                \ex[]{
                [tʃ\textbf{i}ˈwá]  {\textasciitilde} [tʃ\textbf{e}ˈwá]\\
                /tʃ\textbf{e}ˈwá/  \\
                ‘hit’\\
                ‘pegar’ < ROF 1:67/el >  \\
            }
                    \ex[]{
                    [m\textbf{i}ˈhí]  {\textasciitilde} [m\textbf{e}ˈhí] \\
                    /m\textbf{e}ˈhí/  \\
                    ‘cook mezcal’\\
                    ‘cocer mezcal’ < SFH 07 2:12/el >\\
                }
                \newpage
                        \ex[]{
                        [m\textbf{i}ˈtá]  {\textasciitilde} [m\textbf{e}ˈtá] \\
                        /m\textbf{e}ˈtá/  \\
                        ‘crumble’\\
                        ‘desmoronarse’ < ROF 04 1:60/el >\\
                    }
                            \ex[]{
                            [m\textbf{i}ˈʔà] {\textasciitilde} [m\textbf{e}ˈʔà]\\
                            /m\textbf{e}ˈʔà/   \\
                            ‘kill’\\
                            ‘matar’  < JHF 04 1:1/el >\\
                        }
    \z
\z

Other vowel contrasts are preserved pretonically. The examples below show the contrasts licensed: central, front vowels (\ref{ex: pre-tonic vowel contrasts}a--c); back, high vowels (\ref{ex: pre-tonic vowel contrasts}d--e); and back, mid vowels (\ref{ex: pre-tonic vowel contrasts}f--g). The unattested forms listed in (\ref{ex: pre-tonic vowel contrasts}) show hypothetical forms with  pre-tonic, non-initial neutralized vowels. These hypothetical forms would be expected if pretonic vowel reduction would target all vowel qualities.

\ea\label{ex: pre-tonic vowel contrasts}
{Pre-tonic vowel contrasts}

    \ea[]{
    \glt    \doublebox{[akaˈbó]}{\textit{*ak\textbf{i}ˈbó}}\\
    \glt    /akaˈbó/\\
    \glt    ‘nose’\\
    \glt    ‘nariz’ \corpuslink{tx152[04_251-04_322].wav}{SFH tx152:4:25.1}\\
}
        \ex[]{
        \glt    \doublebox{[aʰkaˈrâ]}{\textit{*aʰk\textbf{i}ˈrâ}}\\
        \glt    aʰka-ˈrâ\\
        \glt    sandal-\textsc{vblz}\\
        \glt    `to put on sandals'\\
        \glt    `ponerse huaraches' < SFH 05 1:103/el >\\
    }
            \ex[]{
            \glt    \doublebox{[amaˈtʃ͡îa]}{\textit{*am\textbf{i}ˈtʃ͡îa}}\\
            \glt    /amaˈtʃ͡î-a/\\
                    pray-\textsc{prog}\\
            \glt    ‘praying’\\
            \glt    ‘rezando’  \corpuslink{el655[02_178-02_193].wav}{BFL el655:2:17.8}\\
        }
                \ex[]{
                \glt    \doublebox{[bahuˈrérua]}{\textit{*bah\textbf{i}ˈrérua}}\\
                \glt    /bahuˈré-rua/\\
                        invite-\textsc{mpass}\\
                \glt    ‘be invited’\\
                \glt    ‘ser invitado’  \corpuslink{tx68[02_253-02_275].wav}{LEL tx68:2:25.3}\\
            }
                    \ex[]{
                    \glt    \doublebox{[buruˈrútʃ͡i]}{\textit{*bur\textbf{i}ˈrútʃ͡i}}\\
                    \glt    /buruˈrútʃ͡i/\\
                    \glt    ‘tamales’ \\
                    \glt    < SFH 07 DB/el >\\
                }
                        \ex[]{
                        \glt    \doublebox{[bohoˈnîsa]}{\textit{*boh\textbf{i}ˈnîsa}}\\
                        \glt    /bohoˈnî-sa/\\
                                cross.river-\textsc{cond}\\
                        \glt    ‘when it crosses the river’  \\
                        \glt    `cuando cruce el río' \corpuslink{tx177[09_251-09_324].wav}{LEL tx177:9:25.1}\\
                    }
                            \ex[]{
                            \glt    \doublebox{[bokoˈwíma]}{\textit{*bok\textbf{i}ˈwíma}}\\
                            \glt    bokoˈwí-ma\\
                                    to.become.dark-\textsc{fut.sg}\\
                             \glt   ‘it will get dark (from the sun setting)’\\
                             \glt   ‘va a atardecer’  \corpuslink{tx84[01_598-02_060].wav}{LEL tx84:1:59.8}\\
                        }
    \z
\z

As discussed in \chapref{chap: verbal morphology}, there are particular pretonic vowel alternations that are specific of a group of verbal stems. These stems have a root final stressed \textit{a} and a final unstressed, pretonic \textit{i} in specific morphological constructions (e.g. \textit{raʔˈl}\textbf{\textit{à}}\textit{{}-li}, ‘buy-\textsc{pst}’ vs. \textit{raʔl}\textbf{\textit{i}}\textit{{}-ˈmêa}, ‘buy-\textsc{fut.sg}’). These vocalic alternations are characteristic of a group of stems where the alternations are morphologically-conditioned (§\ref{subsubsec: initial three-syllable stress window}), and are not related to the vowel reduction patterns described in this section.

There is inter-speaker variation in terms of reduction patterns, including vocalic reduction patterns. One apparent trend is that older speakers seem to use less mid-vowel neutralization than younger speakers. This is exemplified in the following examples which are part of a recorded conversation. In this interaction, SFH, a 35-year old speaker, uses the verbal stem /bete-/ with pretonic vowel raising (in (\ref{ex: inter-speaker variation in V reductiona})); FLP, an 80-year old speaker, on the other hand, responds using the same verbal form with no pretonic mid vowel neutralization (in (\ref{ex: inter-speaker variation in V reductionb})). The contrast between the younger speaker’s neutralization and lack of neutralization in the older speaker’s speech remains constant along the conversation.

\largerpage

\ea\label{ex: inter-speaker variation in V reduction}
{Inter-speaker variation in vowel reduction}

    \ea[]{
    \glt    {[SFH]: \textit{siˈné   roˈkò   \textbf{biti}{}ˈbása?}}\\
    \gll    siˈné  roˈkò  bete-ˈbá-sa\\
            one  night  stay.overnight-\textsc{inch-cond}\\
    \glt    ‘He would stay up all night?’ \\
    \glt    ‘Quedándose una noche?’  \corpuslink{in243[01_295-01_347].wav}{SFH in243:1:29.5}\\
}\label{ex: inter-speaker variation in V reductiona}
%\pagebreak
        \ex[]{
        \glt    {[FLP]:  \textit{siˈné roˈkò \textbf{bete}{}ˈbása ra ba}   }\\
        \gll    siˈné  roˈkò  bete-ˈbá-sa    ra  ba\\
                one  night  stay.overnight-\textsc{inch-cond} \textsc{rep}  \textsc{cl} \\
        \glt    ‘One whole night he would stay up’\\
        \glt    ‘Toda la noche hasta que cumpliera (se quedó sin dormir)’   \corpuslink{in243[01_295-01_347].wav}{FLP in243:1:29.5}\\
    }\label{ex: inter-speaker variation in V reductionb}
    \z
\z

% {}- Other examples from the corpus? Check JMF, MFH, MDH, FL

Mid front vowels may also be raised to \textit{i} post-tonically. Some examples are shown in (\ref{ex: post-tonic front vowel height neutralization}).

\ea\label{ex: post-tonic front vowel height neutralization}
{Post-tonic front vowel height neutralization}

    \ea[]{
    [beˈnèriam\textbf{i}]\\
    /beˈnè-ri-am\textbf{e}/  \\
    learn-\textsc{caus-ptcp}  \\
    `teacher' (lit. `the one who causes to learn')\\
    `maestro' (lit. `el que causa que se aprenda') < BFL 06 4:168/el >\\
}
        \ex[]{
         [raˈsíam\textbf{i}]\\
         /raˈsí-am\textbf{e}/  \\
         misbehave-\textsc{ptcp}  \\
         `mischievous person, prankster'\\
         `malcriado, bromista' < BFL 04 1:90/el >\\
    }
            \ex[]{
            [koʔˈá\textbf{i}]\\
            /koʔˈá-\textbf{e}/  \\
            eat-\textsc{impf}     \\
            `S/he used to eat.'\\
            `Comía.' < ROF 04 1:109/el >\\
        }
                \ex[]{
                [oˈsà\textbf{i}]   \\
                /oˈsà-\textbf{e}/   \\
                write-\textsc{impf}   \\
                `S/he used to write.'\\
                `Escribía'. < AHF 05 1:127/el >\\
            }
    \z
\z

Pre-tonic and posttonic raising of unstressed \textit{e} is subject to some degree of intra-speaker variation, but this reduction pattern is the one displaying the least amount of variation.

\subsubsection{[$-$high] Unstressed posttonic vowel reduction}

A second unstressed vowel reduction pattern in Choguita Rarámuri involves /a/ and /o/ rising to \textit{i} post-tonically. This reduction process does not take place in word-final position. Some examples are provided in (\ref{ex: post-tonic reductuion of low vowels}). Each vowel reduction example is followed by a related form with no vowel reduction.

\ea\label{ex: post-tonic reductuion of low vowels}
{Post-tonic reduction of low vowels}

    \ea[]{
    [ˈtʃ͡ôt\textbf{i}ki]     \\
    /ˈtʃ͡ôt\textbf{a}-ki/ \\
    begin-\textsc{pst.ego}\\
    `I began.'\\
    `Empecé.'   \corpuslink{el259[11_549-11_573].wav}{BFL el259:11:54.9}\\
}\label{ex: post-tonic reductuion of low vowelsa}
        \ex[]{
        [tʃ͡oˈt\textbf{â}nsa]\\
        /tʃ͡oˈt\textbf{â}-nale-sa/ \\
        begin-\textsc{desid-cond}\\
        `if they want to begin'\\
        `si quieren empezar' \corpuslink{in243[06_323-06_358].wav}{FLP in243:6:32.3}\\
    }\label{ex: post-tonic reductuion of low vowelsb}
            \ex[]{
            [tʃ͡iˈhán\textbf{i}{}li]\\
            /tʃ͡iˈhá-n\textbf{a}-li/ \\
            scatter-\textsc{tr-pst}\\
            `S/he scattered it.'\\
            `Lo desparramó.'  < SFH 07 1:17/el > \\
        }\label{ex: post-tonic reductuion of low vowelsc}
                \ex[]{
                [tʃ͡ihaˈn\textbf{â}sa] \\
                /tʃ͡iha-ˈn\textbf{a}-sa\footnote{As described in more detail in §\ref{sec: stress properties of roots, stems and suffixes}, there are morphologically giverneed stress shifts in complex words.}/ \\
                scatter-\textsc{tr-cond}\\
                `S/he will scatter it.'\\
                `Lo va a desparramar' < SFH 07 1:17/el >\\
            }\label{ex: post-tonic reductuion of low vowelsd}
                    \ex[]{
                    [sutuˈbétʃ͡in\textbf{i}li] \\
                    /sutuˈbétʃ͡i-n\textbf{a}le/\\
                    trip-\textsc{-desid}\\
                    `S/he wants to trip (is about to trip).' \\
                    `Se quiere tropezar' < BFL 07 1:138/el >\\
                }\label{ex: post-tonic reductuion of low vowelse}
                        \ex[]{
                        [ʃimiˈn\textbf{á}l]\\
                        /simi-ˈn\textbf{á}le-/  \\
                        go.\textsc{sg-desid}  \\
                        `S/he wants to go.'\\
                        `Quiere ir.' < BFL 06 EDCW/el >\\
                    }\label{ex: post-tonic reductuion of low vowelsf}
                    \newpage
                            \ex[]{
                            [tiˈjôp\textbf{i}tʃ͡i] \\
                            /tiˈjôp\textbf{a}-tʃ͡i/ \\
                            church-\textsc{loc}\\
                            `at the church'\\
                            `en la iglesia' \corpuslink{tx904[03_549-03_576].wav}{GFM tx904:3:54.9}\\
                        }\label{ex: post-tonic reductuion of low vowelsg}
                                \ex[]{
                                [tiˈjôp\textbf{a}]\\
                                /tiˈjôp\textbf{a}/  \\
                                `church’   \\
                                `iglesia' \corpuslink{tx223[03_035-03_113].wav}{LEL tx223:3:03.5}\\
                            }\label{ex: post-tonic reductuion of low vowelsh}
                                    \ex[]{
                                    [ˈtʃ͡ón\textbf{i}li] \\
                                    /ˈtʃ͡ón\textbf{a}-li/\\
                                    get.dirty-\textsc{pst}\\
                                    `It got dirty.'\\
                                    `Se ensució.' < LEL 06 6:78/el >\\
                                }\label{ex: post-tonic reductuion of low vowelsi}
                                        \ex[]{
                                        [ˈtʃ͡ón\textbf{a}] \\
                                        /ˈtʃ͡ón\textbf{a}/\\
                                        ‘get dirty’ \\
                                        `ensuciarse' < ROF 04 1:65/el >\\
                                    }\label{ex: post-tonic reductuion of low vowelsj}
    \z
\z

In (\ref{ex: post-tonic reductuion of low vowels}a--b), (\ref{ex: post-tonic reductuion of low vowels}c--d)\footnote{The forms in (\ref{ex: post-tonic reductuion of low vowels}a,b) involve a stress shift that is morphologically-conditioned, but the post-tonic reduction is subject to speaker variation; this contrasts with the morphologically-conditioned vocalic alternations to be discussed in \chapref{chap: verbal morphology} (§\ref{sec: verbal root classes in shifting and neutral constructions}), since these do not display speaker variation.}  and (\ref{ex: post-tonic reductuion of low vowels}e--f), the reduced forms can be contrasted with their stressed, non-reduced counterparts;  in (\ref{ex: post-tonic reductuion of low vowelsh}) and (\ref{ex: post-tonic reductuion of low vowelsj}), on the other hand, reduction does not take place because the vowels in question are in word-final position.

Unstressed vowel reduction is attested in incorporated forms as well. In incorporated forms stress is assigned in the first syllable of the morphological head of the incorporated verb (this morphological stress rule is discussed below in §\ref{subsec: body-part incorporation}). In the examples in (\ref{ex: post-tonic V reduction in incorporated verbs}), the stress shifts one syllable to the left, yielding reduction of underlying /o/ (\ref{ex: post-tonic V reduction in incorporated verbsa}) and /a/ (\ref{ex: post-tonic V reduction in incorporated verbsc}).

\ea\label{ex: post-tonic V reduction in incorporated verbs}
{Post-tonic vowel reduction in incorporated verbs}

    \ea[]{
    [ronoˈbâk\textbf{i}{}ma] \\
    /roˈnô+paˈk\textbf{ó}{}-ma/ \\
    feet+wash-\textsc{fut.sg} \\
    `S/he will wash her feet'\\
    `Se va a lavar los pies'\\
}\label{ex: post-tonic V reduction in incorporated verbsa}
        \ex[]{
        [paˈkóma]  \\
        /paˈkó-ma/   \\
        wash-\textsc{fut.sg}\\
        `S/he will wash'\\
        `Va a lavar'\\
    }\label{ex: post-tonic V reduction in incorporated verbsb}
            \ex[]{
            [siwaˈbôt\textbf{i}{}ma] \\
            /siˈwá+boˈt\textbf{á}{}-ma/  \\
            guts+loosen-\textsc{fut.sg}\\
            `It will get disemboweled'\\
            `Se va a destripar'\\
        }\label{ex: post-tonic V reduction in incorporated verbsc}
                \ex[]{
                [boˈtáma]    \\
                /boˈtá-ma/  \\
                loosen-\textsc{fut.sg}\\
                `It will become loose'\\
                `Se va a aflojar'\\
            }\label{ex: post-tonic V reduction in incorporated verbsd}
    \z
\z

Not all pre-final, post-tonic underlying /a/ raise to [i]. The potential targets for reduction in (\ref{ex: blocked vowel reduction}) share the characteristic of being the first vowel of a vowel-initial suffix. These suffix vowels do not undergo reduction, and hypothetical forms with reduced post-tonic vowels (exemplified in the second column in (\ref{ex: blocked vowel reduction})) are unattested.

\ea\label{ex: blocked vowel reduction}
{Blocked vowel reduction}

    \ea[]{
    [ˈʃûami]\\
    \glt    /ˈsû-\textbf{a}me/    \\
    \glt    sew-\textsc{ptcp}   \\
    \glt    `the one who sews'\\
    \glt    `el que cose' < BFL 06 4:168/el >\\
    \glt    \textit{*ˈʃû-\textbf{i}mi}\\
}
        \ex[]{
        [saˈwèrami]\\
        \glt    /saˈw-è-r-\textbf{a}me/\\
        \glt    cure-\textsc{appl-pst.pass-ptcp} \\
        \glt    `the one who cures'\\
        \glt    `el que cura' < BFL 06 4:168/el >\\
        \glt    \textit{*saˈw-è-r-\textbf{i}mi}\\
    }
\newpage
            \ex[]{
            [baˈjèatʃ͡i]\\
            \glt    /baˈj-è-\textbf{a}-tʃ͡i/\\
            \glt    call.out-\textsc{appl-prog-loc}\\
            \glt    `at the place where they invite, call out'\\
            \glt    `donde llaman, invitan' < BFL 05 2:56/el >\\
            \glt    \textit{*baˈ-j-è-\textbf{i}-tʃi}\\
        }
                \ex[]{
                [laˈmúami]\\
                \glt    /laˈmú-\textbf{a}me/\\
                \glt    purple-\textsc{ptcp} \\
                \glt    `purple'\\
                \glt    `morado' < LEL 06 6:79/el >\\
                \glt    \textit{*laˈmú-\textbf{i}mi}\\
            }
                    \ex[]{
                    [ˈpòara]\\
                    \glt    /ˈpò-\textbf{a}-ra/\\
                    \glt    cover-\textsc{prog-purp}\\
                    \glt    `lid'\\
                    \glt    `tapadera' < SFH 07 in242/in >\\
                    \glt    \textit{*ˈpò-\textbf{i}-ra}\\
                }
    \z
\z

Reduction, thus, can be blocked due to morphological restrictions.

\subsubsection{Unstressed high vowel reduction to schwa}
\label{subsubsec*: unstressed high vowel reduction to schwa}

Choguita Rarámuri has a third process of unstressed vowel reduction, where high vowels may reduce to schwa posttonically. The examples below show the target vowels, /i/ (\ref{ex: optional post-tonic reduction to schwa}a--d) and /u/ (\ref{ex: optional post-tonic reduction to schwa}e--f).\footnote{There is only one example where the target of reduction is an underlying low, central vowel: iˈna-n\textbf{ə}r-o (/iˈna-n\textbf{a}le-o/),‘walk-\textsc{order-ep}’ < BFL 06 DECW/ el >.}

\ea\label{ex: optional post-tonic reduction to schwa}
{Optional post-tonic reduction to schwa}

    \ea[]{
    [tʃ͡iˈpór\textbf{ə}ma]\\
    /tʃ͡iˈpó-r\textbf{i}{}-ma/ \\
    bounce-\textsc{caus-fut.sg} \\
    `S/he will make it bounce.'\\
    `Lo va a hacer rebotar.' < LEL 07Caus\_ME >\\
}
        \ex[]{
        [ˈʃûn\textbf{ə}po]\\
        /ˈsû-n\textbf{i}-po/ \\
        sew-\textsc{appl-fut.pl}\\
        `They will sew.'\\
        `Van a coser.' < BFL applicatives/el >\\
    }
            \ex[]{
            [ˈtôp\textbf{ə}ma]\\
            /ˈtô-p\textbf{i}{}-ma/ \\
            bury-\textsc{rev-fut.sg} \\
            `They will unearth it.'\\
            `Lo va a desenterrar.' < BFL 05 1:113/el >\\
        }
                \ex[]{
                [ˈpòl\textbf{ə}{}ki] \\
                /ˈpòl\textbf{i}{}-ki/  \\
                cover-\textsc{pst.ego}  \\
                `I covered it.'\\
                `Lo tapé.' < AHF 05 1:125/el >\\
            }
                    \ex[]{
                    [naˈwín\textbf{ə}la] \\
                    /naˈwí-n\textbf{u}la/  \\
                    sing-\textsc{order}  \\
                    `They will oblige them to sing.'\\
                    `Les van a ordenar cantar.' < BFL 07 VDB/el >\\
                }
                        \ex[]{
                        [ˈtòn\textbf{ə}la]  \\
                        /ˈtò-n\textbf{u}la/  \\
                        take-\textsc{order} \\
                        `They will oblige them to take them.'\\
                        `Les van a obligar a llevarlo.' < LEL 07 Ind\_Caus/el >\\
                    }
    \z
\z

High vowel reduction to schwa is gradient, and favored when preceding or following a back, round vowel or \textit{a}. I have not documented any cases of unstressed high vowel reduction to schwa in pre-tonic position with open class lexical items or in word-final position.\footnote{But it is common to find function words with pre-tonic vowels reducing to schwa in fast speech, e.g. the adverb \textit{biˈlá} ‘indeed’, is often realized as [b\textbf{ə}ˈlá].}

\subsection{Stressed-conditioned vowel deletion}
\label{subsubsec*: stress-conditioned vowel deletion}

Unstressed vowels may also undergo syncope in immediately post-tonic syllables. In (\ref{ex: post-tonic vowel deletion}), the deleted vowel is in bold face in the underlying representation.

%\break

\newpage
\ea\label{ex: post-tonic vowel deletion}
{Post-tonic vowel deletion}

    \ea[]{
    [muruˈbênti]  \\
    /muruˈbê-n\textbf{i}-ti-/\\
    get.close-\textsc{appl-caus} \\
    `to cause something to be closer for someone'\\
    `acercarle algo a alguien' < BFL 07 appl/el >\\
}\label{ex: post-tonic vowel deletiona}
                \ex[]{
                [waˈtónki]\\
                /waˈtó-n\textbf{a}-ki-/  \\
                stretch-\textsc{tr-appl} \\
                `to stretch something for somebody'\\
                `estirar algo para alguien' < SFH 07 Caus,\_ME/el >\\
            }\label{ex: post-tonic vowel deletionb}
                    \ex[]{
                    [waˈtónili]\\
                    /waˈtó-na-li/  \\
                    stretch-\textsc{tr-pst} \\
                    `S/he stretched it.'\\
                    `Lo estiró.' < SFH 06 6:73-77/el >\\
                }\label{ex: post-tonic vowel deletionc}
                        \ex[]{
                        [toˈnáltʃ͡ino] \\
                        /tò-nál\textbf{e}-tʃ͡ane-o/ \\
                        take-\textsc{desid-ev-ep}\\
                        `It sounds like they want to take it.'\\
                        `Suena que se lo quiere llevar.' < BFL 06 5:148-150/el >\\
                    }\label{ex: post-tonic vowel deletiond}
                                \ex[]{
                                [koʔˈnálti] \\
                                /koʔá-nál\textbf{e}{}-ti-/  \\
                                eat-\textsc{desid-caus}\\
                                `S/he will make them want to eat it.'\\
                                `Va a hacer que se lo quiera comer.' < SFH 06 6:75-76/el >\\
                            }\label{ex: post-tonic vowel deletione}
    \z
\z

As the examples above show, post-tonic vowel deletion targets underlying high (\ref{ex: post-tonic vowel deletiona}), low (\ref{ex: post-tonic vowel deletion}b--c), and mid (\ref{ex: post-tonic vowel deletion}d--e) vowels.

Deletion does not target word-final unstressed vowels. This is shown in (\ref{ex: no deletion of final vowels}).

\ea\label{ex: no deletion of final vowels}
{No post-tonic vowel deletion of word-final vowels}

    \ea[]{
    [naˈkûri]   \\
    /na-ˈkûri/  \\
    \textsc{pl}-help\\
    `They help.'\\
    `Ayudan.' < SFH 05 1:102/el >\\
}
        \ex[]{
        [toˈnáli]   \\
        /tò-nále/ \\
        take-\textsc{desid}\\
        `They want to take it.'\\
        `Quieren llevarlo.' < BFL 06 5:148-150/el > \\
    }
    \z
\z

Deletion does not target pretonic vowels as a general phonological process, but instances of pretonic deletion can be found in morphologically specific contexts. For instance, the forms in (\ref{ex: morphologically conditioned pretonic deletion}) involve pretonic deletion with root-inchoative sequences. Examples (\ref{ex: morphologically conditioned pretonic deletion}b, d, f) show a corresponding form with the same root with no deletion. Other pretonic effects which are stress-conditioned (such as syllable truncation in morphologically specific contexts) are discussed in \chapref{chap: verbal morphology} and \chapref{chap: prosody}.

\ea\label{ex: morphologically conditioned pretonic deletion}
{Morphologically-conditioned pretonic deletion}

    \ea[]{
    [wamˈpása]\\
    /bam\textbf{i}{}-bá-sa/  \\
    turn.year-\textsc{inch-cond}\\
    `if they become a year older'\\
    `si cumple años' \corpuslink{tx12[08_400-08_477].wav}{SFH tx12:8:40.0}\\
}
        \ex[]{
        [baˈm\textbf{í}biri]  \\
        /bamí-bá-ri/  \\
        turn.year-\textsc{inch-nmlz}\\
        `year'\\
        `año'  \corpuslink{in61[03_099-03_128].wav}{FLP in61:3:09.9}\\
    }
            \ex[]{
            [samˈpá{}ma]\\
            /saʔm\textbf{í}{}-bá-ma/ \\
            get.wet-\textsc{inch-fut.sg}\\
            `It will get wet'\\
            `Se va a mojar' < SFH 04 1:113/el >\\
        }
                \ex[]{
                [saʔˈm\textbf{í}li]  \\
                /saʔmí-li/  \\
                get.wet-\textsc{pst} \\
                `It got wet.'\\
                `Se mojó.'\\
            }
                    \ex[]{
                    [omˈpása]    \\
                    /omèra-bá-sa/ \\
                    be.able-\textsc{inch-cond}\\
                    `if s/he is (not) able to'\\
                    `si (no) puede'  \corpuslink{co1236[01_496-01_518].wav}{JLG co1236:1:49.6}\\
                }
                        \ex[]{
                        [oˈm\textbf{è}ri] \\
                        /omèra/  \\
                        be.able.\textsc{prs}\\
                        `S/he is (not) able.'\\
                        `(No) puede.'  \corpuslink{tx130[05_084-05_133].wav}{LEL tx130:5:08.4}\\
                    }
    \z
\z

In sum, while pre-tonic vowel deletion is uncommon, posttonic vowels generally reduce or delete. In the latter case, the choice between reduction and deletion results from a gradient process that is affected by rate of speech. This process may also exhibit intra-speaker variation, though this has not yet been systematically examined in the Choguita Rarámuri data.

\section{Stress properties of roots and suffixes}
\label{sec: stress properties of roots, stems and suffixes}

As described above, stress is part of the underlying representation of at least some morphemes in Choguita Rarámuri. Diachronic loss of contrastive vowel length, reconstructed for \ili{Proto-Uto-Aztecan} \parencite{campbell1978proto}, is argued to be the likely source of lexically contrastive stress in contemporary \ili{Uto-Aztecan} varieties \parencite{munro1977towards}.\footnote{Some \ili{Uto-Aztecan} prosodic systems exhibit weight-based stress systems (e.g., \ili{Numic} languages, like \ili{Southern Paiute} (\citealt{sapir1930southern}) and \ili{Tümpisa Shoshone} (\citealt{dayley1989tumpisa}) are documented to have rhythmic stress systems sensitive to mora count).} Choguita Rarámuri stress might have become unpredictable when the \ili{Proto-Uto-Aztecan} vowel length distinction was lost.

As proposed for other \ili{Uto-Aztecan} languages (e.g. \ili{Cupeño} (\ili{Cupan}; \citealt{hill1968stress}, \citealt{alderete2001dominance})), Choguita Rarámuri roots fall into two classes: \textit{stressed} and \textit{unstressed}. Stressed roots have fixed stress across morphologically related forms.\footnote{Stressed roots are analyzed in \citet{caballero2008choguita} and \citet{caballero2011morphologically} as lexically specified with a diacritic mark which is phonetically realized as stress in output forms.} Unstressed roots, on the other hand, lack underlying lexical specification for stress; stress in these cases is not fixed and its position determined by the phonological and morphological context. Specifically, unstressed roots may receive stress by default when attaching a stress-neutral suffix or shift stress rightward when attaching a stress-shifting suffix (the stress properties of suffixes are described below in §\ref{subsubsec*: stress properties of suffixes}). All prosodic words in Choguita Rarámuri, whether they contain a lexically stressed root or an unstressed lexical root, have surface stress, a syntagmatic prominence cued acoustically primarily via intensity and duration (as described in §\ref{subsec: stress and stress-dependent phenomena} above).

The contrast between stressed and unstressed roots is exemplified below with the different stress patterns of nominal roots that add the (stress-shifting) locative suffix \textit{{}-}\textit{tʃ͡í}, as exemplified in \tabref{tab:stressed-unstressed-contrasts}.

\begin{table}
\caption{Stressed and unstressed roots in Choguita Rarámuri}
\label{tab:stressed-unstressed-contrasts}

\begin{tabularx}{\textwidth}{lllXl}
\lsptoprule
&\textbf{Underlying} & \textbf{Gloss} & \textbf{Bare stem}  & \textbf{\textsc{loc} suffix} \\
\midrule
\multicolumn{2}{l}{\textbf{Stressed roots}}  \\
\midrule
a.& /ˈm{ú}li/ &   ‘basket’ `canasta' & ˈmúli &  ˈmúli-tʃ͡i \\
b.& /ˈp{ú}ra/  &  ‘belt’ `cinto'  &    ˈpúra & ˈpúra-tʃ͡i \\
c.& /ˈsôru/  &  ‘soda’ &     ˈs\textbf{ô}ru & ˈs\textbf{ô}ru-tʃ͡i  \\
d.& /waˈrî/  &  ‘basket’ `canasta' &   waˈr\textbf{î} & waˈr\textbf{î}-tʃ͡i \\
\tablevspace
\multicolumn{2}{l}{\textbf{Untressed roots}}  \\
\midrule
e.& /sekâ/ &   ‘hand’ `mano'  & seˈkâ  & seka-ˈtʃ͡í   \\
f.& /rapé/  &  ‘rock’ `roca'  &    raˈpé & rape-ˈtʃ͡í  \\
g.& /wasá/   & ‘field’ `tierra' &   waˈsá & wasa-ˈtʃ͡í\\
h.& /ronô/    & ‘foot’ `pie'  &  roˈnô & rono-ˈtʃ͡í \\
i.& /kupá/    & ‘hair’ `pelo'  &  kuˈpá & kupa-ˈtʃ͡í \\
\lspbottomrule
\end{tabularx}
\end{table}

Stressed roots have fixed stress in the first (a--c) or second (d) syllable, whether bare or suffixed with locative \textit{-tʃ͡í}. In unstressed roots, stress falls on the locative suffix, one syllable to the right with respect to their bare counterparts (e--i).

Lexical stress in roots blocks morphologically-conditioned stress shifts; stress in these cases will be fixed on the first, second or third syllable, regardless of what suffixes are attached. On the other hand, words containing unstressed roots have predictable stress assignment. In words containing unstressed roots and stress-neutral suffix, stress will be assigned by default on the second syllable of the root or the only syllable of monosyllabic roots. In words containing unstressed roots and stress-shifting suffixes, stress shifts rightward. The default stress assignment algorithm is summarized in (\ref{ex: default stress assignment in CR}):

\ea\label{ex: default stress assignment in CR}

\textbf{Default stress assignment in Choguita Rarámuri}: words containing unstressed disyllabic or trisyllabic roots and neutral morphological constructions (affixation and non-concatenative processes) have second syllable stress \parencite{caballero2011morphologically}.

\z

More details about the morphological dimension of stress in nominal and verbal paradigms is provided in  \chapref{chap: nominal morphology} and \chapref{chap: verbal morphology}, respectively. The stress properties of monosyllabic, disyllabic and trisyllabic roots in Choguita Rarámuri are addressed next.\footnote{As discussed in §\ref{subsec: body-part incorporation} and §\ref{subsec: stress and tone properties of compounds}, roots longer than three syllables are infrequent. The stress properties of noun incorporation is addressed in §\ref{subsec: body-part incorporation}}

\subsection{Stress properties of monosyllabic roots}
\label{subsec: stress properties of monosyllabic roots}

Most roots in Choguita Rarámuri are disyllabic or trisyllabic, but monosyllabic roots are attested. Monosyllabic roots are almost all stressed, but there are attested cases of unstressed monosyllabic roots. This contrast is exemplified in \tabref{tab:monosyllabic-stressed-unstressed}.\footnote{As discussed in §\ref{sec: vowel length, stress and minimality effects}, monosyllabic roots undergo lengthening when bare (e.g., present tense) in order to satisfy a prosodic word bimoraic minimality requirement.}\footnote{The tone properties of morphologically complex words is addressed in \chapref{chap: nominal morphology}, \chapref{chap: verbal morphology} and \chapref{chap: prosody}.}

\begin{table}
\caption{Stressed and unstressed monosyllabic roots}
\label{tab:monosyllabic-stressed-unstressed}

\begin{tabularx}{\textwidth}{lXXXl}
\lsptoprule
&\textbf{Underlying} & \textbf{Gloss} & \textbf{\textsc{pst} neutral}  & \textbf{\textsc{fut} shifting} \\
\midrule
a.& /ˈpá/ &   `throw’ `tirar’ & ˈpá-li  & ˈpá-ma \\
b.& /ˈsû/  & `coser' `sew'  & ˈsû-li &    ˈsû-ma  \\
c.& /tò/ &    `take’ `llevar &     ˈtò-li  &   to-ˈmêa   \\
d.& /rú/  &  `say' `decir'    &   ˈrú-li    &    ruˈmêa  \\
\lspbottomrule
\end{tabularx}
\end{table}

Stressed monosyllabic roots have fixed stress (\tabref{tab:monosyllabic-stressed-unstressed}a--b), regardless of the type of suffix it may combine with (stress neutral like the past \textit{-li} suffix or stress-shifting like the future \textit{-ma/ˈmêa} suffix).\footnote{The future singular suffix displays an interesting allomorphy: \textit{-ma}, used with stressed roots, and \textit{-ˈmêa}, used with unstressed roots. Consistently, the former is unstressed while the latter is stressed, and root stress seems to be the only parameter that plays a role in allomorph selection. The future singular suffix is the only suffix that displays this stress-conditioned suppletive allomorphy. \ili{Guarijio}, another \ili{Taracahitan} language, does not have this allomorphy for the cognate future suffix (in both stressed and unstressed contexts, the future singular suffix is \textit{-ma} \citealt{miller1996guarijio}).} Unstressed monosyllabic roots, on the other hand, may shift stress rightward onto the suffix if the suffix is stress-shifting (\tabref{tab:monosyllabic-stressed-unstressed}c--d). Unstressed monosyllabic roots have stress in the root (the first syllable of the prosodic word) when attaching stress-neutral suffixes, exemplifying that default stress is assigned in the root (stress neutral suffixes are never stressed).

\subsection{Stress properties of disyllabic roots}
\label{subsubsec*: stress properties of disyllabic roots}

Disyllabic roots may be underlyingly stressed or unstressed. Disyllabic stressed roots are exemplified in \tabref{tab:stressed-disyllabic}.

\begin{table}
\caption{Stressed disyllabic roots}
\label{tab:stressed-disyllabic}

\begin{tabularx}{\textwidth}{lXXXXl}
\lsptoprule
&\textbf{Underlying} & \textbf{Gloss} & \textbf{Stem} & \textbf{\textsc{fut} shifting} & \textbf{\textsc{pst} neutral}  \\
\midrule
a.& /ˈtâni/  &  ‘ask for’ & ˈtâni & ˈtâni-ma & ˈtâni-li \\
b.& /ˈpùtʃ͡i/ &  ‘blow’ & ˈp\textbf{ù}tʃ͡i &   ˈp\textbf{ù}tʃ͡i-ma &  ˈp\textbf{ù}tʃ͡i-li\\
c.& /ˈmèti/   & ‘drive’  & ˈm\textbf{è}ti &  ˈm\textbf{è}ti-ma & ˈm\textbf{è=}ti-li  \\
d.& /ˈéri/ & ‘close’  & ˈ\textbf{é}ri & ˈ\textbf{é}ri-ma   & ˈ\textbf{é}ri-li\\
e.& /ˈnâri/ & ‘ask’  &  ˈn\textbf{â}ri & ˈn\textbf{â}ri-ma & ˈn\textbf{â}ri-li\\
\tablevspace
f.& /kaˈtʃ͡í/&   ‘spit’ &   kaˈtʃ͡\textbf{í} &  kaˈtʃ͡\textbf{í}-ma &  kaˈtʃ͡\textbf{í}{}-li\\
g.& /aˈwê/   & ‘grill’   &  aˈw\textbf{ê} & aˈw\textbf{ê}{}-ma  & aˈw\textbf{ê}{}-li   \\
h.& /riˈwè/  & ‘leave’  &   riˈw\textbf{è} &  riˈw\textbf{è}{}-ma & riˈw\textbf{è}{}-li    \\
i.& /naˈpà/  & ‘hug’   & naˈp\textbf{à}  & naˈp\textbf{à}{}-ma  & naˈp\textbf{à}{}-li\\
j.& /seˈmè/ & ‘play violin’  &    seˈm\textbf{e}  & seˈm\textbf{e}{}-ma  & seˈm\textbf{e}{}-li\\
\lspbottomrule
\end{tabularx}
\end{table}

Most disyllabic roots with first syllable stress are lexically stressed. The great majority of these roots have first syllable stress that remains constant across morphological paradigms, as exemplified in \tabref{tab:stressed-disyllabic}a--e. As shown in these examples, whether these verbal roots combine with neutral constructions (like the past \textit{-li} suffix) or shifting constructions (like the future singular \textit{-ma {\textasciitilde} -mêa} suffix), stress remains fixed on the first syllable.

Unstressed disyllabic roots are exemplified in \tabref{tab:unstressed-disyllabic}.

\begin{table}
\caption{Unstressed disyllabic roots}
\label{tab:unstressed-disyllabic}

\begin{tabularx}{\textwidth}{lXXXXl}
\lsptoprule
&\textbf{Underlying} & \textbf{Gloss} & \textbf{Stem} & \textbf{\textsc{fut} shifting} & \textbf{\textsc{pst} neutral} \\
\midrule
a.& /ukú/  &  ‘rain’   &  uˈk\textbf{ú}  & uku-ˈm\textbf{ê}a    &     uˈk\textbf{ú}-li  \\
b.& /tʃ͡apí/ &  ‘grab’ &  tʃ͡aˈp\textbf{í} &  tʃ͡api-ˈm\textbf{ê}a &   tʃ͡aˈp\textbf{í}{}-li\\
c.& /tʃ͡aʔí/& 	‘get stuck' &	tʃ͡aˈʔí &	tʃ͡aʔi-ˈmêa	& tʃ͡aˈʔí-li\\
d.& /sawí/  & ‘cure \textsc{intr}’ & saˈw\textbf{í}   & sawi-ˈm\textbf{ê}a & saˈw\textbf{í}-li  \\
e.& /ronò/  & ‘boil’ &  roˈn\textbf{ò}  & rono-ˈm\textbf{ê}a        & roˈn\textbf{ò}{}-li   \\
\lspbottomrule
\end{tabularx}
\end{table}

These roots, with third syllable stress in shifting morphological contexts, increase the proportion of words with third syllable stress in the Choguita Rarámuri corpus.
%\footnote{\textbf{cross-reference here to details of distribution of roots by prosodic properties.}}

%Update information of verbs in database with different prosodic properties
A set of exceptional disyllabic roots have \textit{first} syllable stress with stress-neutral suffixes, and \textit{second} syllable stress with stress-shifting suffixes. From the Choguita Rarámuri corpus, only eight roots exhibit this behavior. An exhaustive list is given in \tabref{tab:exceptional-disyllabic}.

\begin{table}
\caption{Disyllabic roots with first and second syllable stress}
\label{tab:exceptional-disyllabic}

\begin{tabularx}{\textwidth}{lXXXl}
\lsptoprule
&\textbf{Underlying} & \textbf{Gloss} & \textbf{\textsc{fut} shifting} & \textbf{\textsc{pst} neutral} \\
\midrule
a.& /úba/  & ‘bathe’   &  uˈb\textbf{â}{}-ma  & \textbf{ˈú}bi-li \\
b.& /nòtʃ͡a/&  ‘work' &    noˈtʃ͡\textbf{â}{}-ma &  ˈn\textbf{ò}tʃ͡i-li   \\
c.& /sèba/  & ‘reach’ & seˈb\textbf{â}-ma & ˈs\textbf{è}bi-li    \\
d.& /tʃ͡úta/&  ‘sharpen’ &  tʃ͡uˈt\textbf{â}{}-ma &  ˈtʃ͡\textbf{ú}ti-li \\
e.& /péwa/  & ‘smoke’  & peˈw\textbf{â}{}-ma  &ˈp\textbf{é}wi-li    \\
f.& /tʃ͡ôta/&  ‘begin' &  tʃ͡oˈt\textbf{â}{}-ma & ˈtʃ͡\textbf{ô}ti-li    \\
g.& /sóma/  & `wash head’   & soˈm\textbf{â}-ma  & ˈs\textbf{ó}mi-li   \\
h.& /nâta/  & ‘think’  &  naˈt\textbf{â}{}-ma & ˈn\textbf{â}ti-li   \\
\lspbottomrule
\end{tabularx}
\end{table}

Comparison of these roots with their cognates in closely related \ili{Guarijío} (\ili{Tara-Guarijío}; \citealt{miller1996guarijio}), reveals that this set of roots is truly exceptional. Specifically, the \ili{Guarijío} cognates all have three syllables, suggesting that Choguita Rarámuri has innovated initial syllable truncation with these forms. The cognate forms are shown in \tabref{tab:guarijio-cognates}.

\begin{table}
\caption{Cognate forms: Choguita Rarámuri and Guarijío roots}
\label{tab:guarijio-cognates}

\begin{tabularx}{\textwidth}{lXXl}
\lsptoprule
&\textbf{C. Rarámuri} & \textbf{Guarijío} & \textbf{Gloss} \\
\midrule
a.& úba   & uʔuˈpa   & `bathe’ \\
&  < SFH 05 1:86 > & [M402] \\
b.& nòtʃ͡a  &  iˈnotʃ͡a   & `work'\\
&  < ROF 04 1:129 > &   [M340] \\
c.& sèba   & ahˈseba     & `reach’\\
&  < ROF 04 1:109 > &   [M323]\\
d.& tʃ͡ôta  &  ihtʃ͡oˈta  &  `begin' \\
&  < LEL 06 5:36 > &   [M337]\\
e.& sóma   & mo’so-ˈma    & `wash head or hair’ \\
  &< LEL 06 FN > & [M360] & \\
f.& nâta   & uʔnaˈta    &  `think’ \\
  &< JHF 04 1:2 > & [M401] \\
  \lspbottomrule
\end{tabularx}
\end{table}

It is not the case that all of the Choguita Rarámuri roots are one syllable shorter than the corresponding \ili{Guarijío} cognates (e.g. Choguita Rarámuri \textit{raʔìtʃ͡a} -- \ili{Guarijío} \textit{taʔˈitʃ͡a} (M391), ‘speak’; Choguita Rarámuri \textit{roʔˈsówa} -- \ili{Guarijío} \textit{tohˈsoá} (M396), ‘cough’). The forms in \tabref{tab:guarijio-cognates} are derived from trisyllabic roots through a recent diachronic development. This comparative evidence suggests these exceptional forms do not constitute a counterexample to the stress patterns found with unstressed roots elsewhere in the language (where stress shifts are attested), but instead follow the pattern of unstressed trisyllabic roots. I turn to these next.

\subsection{Stress properties of trisyllabic roots}
\label{subsubsec*: stress properties of trisyllabic roots}

Most trisyllabic roots are stressed, with fixed second syllable stress or fixed third syllable stress, aas shown in \tabref{tab:second-syllable-trisyllabic} and \tabref{tab:third-syllable-trisyllabic}, respectively.

\begin{table}
\caption{Second-syllable, stressed trisyllabic roots}
\label{tab:second-syllable-trisyllabic}

\begin{tabularx}{\textwidth}{lXXXXl}
\lsptoprule
&\textbf{Underlying} & \textbf{Gloss} & \textbf{Stem} & \textbf{\textsc{fut} shifting} & \textbf{\textsc{pst} neutral} \\
\midrule
a.& /naˈtêti/  &   ‘pay’   & naˈt\textbf{ê}ti  &  naˈt\textbf{ê}ti-ma  & naˈt\textbf{ê}ti-li\\
b.&  /naˈhâta/  & ‘follow’  & naˈh\textbf{â}ta &   naˈh\textbf{â}ti-ma & naˈh\textbf{â}ti-li\\
c.& /naˈhâta/ & ‘follow’ & naˈh\textbf{â}ta    & naˈh\textbf{â}ti-ma  & naˈh\textbf{â}ti-li\\
d.&  /seˈbâri/ &   ‘complete’  & seˈb\textbf{â}ri  &  seˈb\textbf{a}ri-ma  & seˈb\textbf{â}ri-li\\
e.& /oˈtʃ͡ópi/  &  ‘stick' &  oˈtʃ\textbf{ó}pi &    oˈtʃ\textbf{ó}pi-ma & oˈtʃ\textbf{ó}pi-li \\
\lspbottomrule
\end{tabularx}
\end{table}

\begin{table}
\caption{Third-syllable, stressed trisyllabic roots}
\label{tab:third-syllable-trisyllabic}

\begin{tabularx}{\textwidth}{lXXXXl}
\lsptoprule
&\textbf{Underlying} & \textbf{Gloss} & \textbf{Stem} & \textbf{\textsc{fut} shifting} & \textbf{\textsc{pst} neutral} \\
\midrule
a.& /biniˈhî/  &  ‘acuse’  & biniˈh\textbf{î}    & biniˈh\textbf{î}{}-ma  & biniˈh\textbf{î}{}-li\\
b.&  /bahuˈré/  &  ‘invite’   & bahuˈr\textbf{é}  &  bahuˈr\textbf{é}{}-ma  & bahuˈr\textbf{é}{}-li\\
c.&  /sukuˈtʃ͡ú/ &   ‘scratch' &  sukuˈtʃ\textbf{ú} &  sukuˈtʃ\textbf{ú}{}-ma & sukuˈtʃ\textbf{ú}{}-li\\
d.&  /wikaˈrâ/    & ‘sing’   & wikaˈr\textbf{â}  & wikaˈr\textbf{â}{}-ma  & wikaˈr\textbf{â}{}-li\\
\lspbottomrule
\end{tabularx}
\end{table}

There are also unstressed trisyllabic roots. As shown in (\ref{ex: unstressed trisyllabic roots}), these roots have shifting stress in morphologically complex constructions.

\ea\label{ex: unstressed trisyllabic roots}
{Unstressed trisyllabic roots}

    \ea[]{
    \textit{aˈn\textbf{á}tʃ͡ili}  \\
    \textit{aˈn\textbf{á}tʃ͡a-li}  \\
    endure\textsc{-pst}\\
    `S/he (didn't) endured it.'\\
    `(No) aguantó.' \corpuslink{co1237[03_471-03_481].wav}{JLG co1237:3:47.1}\\
}
        \ex[]{
        \textit{aˈn\textbf{á}tʃ͡iko}\\
       \textit{aˈn\textbf{á}tʃ͡a-ki}\\
        endure\textsc{-pst.ego}\\
        `I endured it.'\\
        `Aguanté.' \corpuslink{el658[05_323-05_347].wav}{BFL el658:5:32.3}\\
    }
                \ex[]{
                \textit{aˈn\textbf{á}tʃ͡ati} \\
                \textit{aˈn\textbf{á}tʃ͡a-ti} \\
                endure-\textsc{caus}     \\
                `to make someone endure it'\\
                `hacer aguantar' < ROF 04 1:123/el >\\
            }
                    \ex[]{
                     \textit{aˈn\textbf{á}tʃ͡ai}\\
                    \textit{aˈn\textbf{á}tʃ͡a-i} \\
                    endure-\textsc{impf}\\
                    `S/he used to endure it.'\\
                    `Aguantaba.' < ROF 04 1:123/el >\\
                }
                        \ex[]{
                        \textit{anaˈtʃ͡\textbf{â}ma}\\
                        \textit{anaˈtʃ͡\textbf{â}{}-ma}\\
                        endure\textsc{-fut.sg}\\
                        `She will endure it.'\\
                        `Va a aguantar.' < MDH co1140:10:56.0 >\\
                    }
                            \ex[]{
                            \textit{anaˈtʃ͡\textbf{â}bo}\\
                            \textit{anaˈtʃ͡\textbf{â}{}-bo}\\
                            endure-\textsc{fut.pl}\\
                            `They will endure it.'\\
                            `Van a aguantar.' \corpuslink{co1237[02_412-02_429].wav}{JLG co1237:2:41.2}\\
                        }
                                \ex[]{
                                \textit{anaˈtʃ͡\textbf{â}sa}\\
                                \textit{anaˈtʃ͡\textbf{â}{}-sa}\\
                                endure-\textsc{cond}\\
                                `if s/he endures it'\\
                                `si aguanta' \corpuslink{el1318[20_000-20_030].wav}{MFH el1318:20:00.0}\\
                            }
                                    \ex[]{
                                    \textit{anaˈtʃ͡\textbf{â}nale} \\
                                    \textit{anaˈtʃ͡\textbf{â}{}-nale} \\
                                    endure-\textsc{desid}\\
                                    `S/he wants to endure it.'\\
                                    `Quiere aguantar.' < ROF 04 1:123/el >\\
                                }
    \z
\z

The stress shifts that the verbal root \textit{aˈnátʃ͡a,} ‘endure’ (\textit{‘aguntar’}), undergoes in (\ref{ex: unstressed trisyllabic roots}) parallels those of unstressed disyllabic roots in different morphological constructions: the verb has second syllable stress when inflected with stress-neutral suffixes in (\ref{ex: unstressed trisyllabic roots}a--e), but third syllable stress when inflected with stress-shifting suffixes (\ref{ex: unstressed trisyllabic roots}f--i). Further examples of unstressed trisyllabic roots are given in \tabref{tab:unstressed-trisyllabic-2}.

\begin{table}
\caption{Unstressed trisyllabic roots}
\label{tab:unstressed-trisyllabic-2}

\begin{tabularx}{\textwidth}{llXll}
\lsptoprule
&\textbf{Stress-neutral} &  & \textbf{Stress-shifting} &  \\
\midrule
a.& naʔˈs\textbf{ò}wa-li  &   ‘mix-\textsc{pst}'  &  naʔsoˈw\textbf{â}{}-ma  & ‘mix-\textsc{fut.sg}’\\
b.& naʔˈs\textbf{ò}wa-ki  & ‘mix-\textsc{pst.ego}’   & naʔsoˈw\textbf{â}{}-sa   & ‘mix-\textsc{cond}’\\
c.& naʔˈs\textbf{ò}wa-a  &   ‘mix-\textsc{prog}’   & naʔsoˈw\textbf{â}{}-bo  & ‘mix-\textsc{fut.pl}’\\
d.& raˈʔ\textbf{à}ma-li  & ‘advice-\textsc{pst}’   & raʔaˈm\textbf{â}{}-ma   & ‘advice-\textsc{fut.sg}’\\
e.& raˈʔ\textbf{à}ma-a  & ‘advice-\textsc{prog}’ & raʔaˈm\textbf{â}{}-sa & ‘advice-\textsc{cond}’\\
f.& raˈʔ\textbf{à}ma-ki  &    ‘advice-\textsc{pst.ego}’ & raʔaˈm\textbf{â}{}-bo  & ‘advice-\textsc{fut.pl}’\\
g.& raˈʔ\textbf{ì}tʃ͡a-li & ‘speak\textsc{-pst}’ &  raʔiˈtʃ͡\textbf{â}-ma & ‘speak\textsc{-fut.sg}'\\
h.& raˈʔ\textbf{ì}tʃ͡a-a &‘speak\textsc{-prog}' &   raʔiˈtʃ͡\textbf{â}{}bo’ &  ‘speak-\textsc{fut.pl}'\\
i.& raˈʔ\textbf{ì}tʃ͡a-ki & ‘speak-\textsc{pst.ego}'&  raʔiˈtʃ͡\textbf{â}{}-sa' &  ‘speak-\textsc{cond}'\\
\lspbottomrule
\end{tabularx}
\end{table}

\subsection{Stress properties of suffixes}
\label{subsubsec*: stress properties of suffixes}

The stress patterns of unstressed trisyllabic roots in morphologically complex words (shown in (\ref{ex: unstressed trisyllabic roots}) and \tabref{tab:unstressed-trisyllabic-2} above) show that stress-neutral suffixes are not pre-stressing, as could have been assumed from the stress pattern of unstressed disyllabic roots. If stress-neutral suffixes were pre-stressing, we would expect third-syllable stress with trisyllabic unstressed roots, immediately preceding the suffixes, instead of the attested second syllable stress. These hypothetical, unattested forms are illustrated in the second column in (\ref{ex: attested and hypothetical forms with stress-neutral suffixes}).

\ea\label{ex: attested and hypothetical forms with stress-neutral suffixes}
{Attested and hypothetical forms with stress-neutral suffixes}

    \ea[]{
    \textit{naʔˈsòwili}\\
    \glt   {\textit{naʔˈs\textbf{ò}wa-li}}\\
     \glt   mix-\textsc{pst}\\
     \glt   `S/he mixed it.'\\
     \glt   `Mezcló.' \corpuslink{co1239[06_125-06_136].wav}{JLG co1239:6:12.5} \\
     \glt   \textit{*naʔsoˈw\textbf{â}{}-li }\\
}
        \ex[]{
        \textit{naʔˈsòwaki}\\
        \glt   {\textit{naʔˈs\textbf{ò}wa-ki}}\\
        \glt    mix-\textsc{pst.ego}\\
        \glt    `I mixed it.'\\
        \glt    `Mezclé'         < ROF 04 1:123/el >\\
        \glt    \textit{*naʔsoˈw\textbf{â}{}-ki}\\
    }
            \ex[]{
            \textit{naʔˈsòwaa}\\
            \glt    \textit{naʔˈs\textbf{ò}wa-a}\\
             \glt   mix-\textsc{prog}\\
             \glt   `S/he is mixing.'\\
             \glt   `Está mezclando.' \corpuslink{co1239[05_588-05_600].wav}{JLG co1239:5:58.8}\\
             \glt   \textit{*naʔsoˈw\textbf{â}{}-a}\\
        }
                \ex[]{
                {\textit{raˈʔàmali}}\\
                \glt    \textit{raˈʔ\textbf{à}ma-li}\\
                 \glt   give.advice-\textsc{pst}\\
                \glt    `S/he gave advice.'\\
                \glt    `Aconsejó.' < ROF 04 1:64/el >\\
                \glt    \textit{*raʔaˈm\textbf{â}{}-li}\\
            }
                    \ex[]{
                    \textit{raˈʔàmaa}\\
                    \glt    {\textit{raˈʔ\textbf{à}ma-a}}\\
                    \glt    give.advice-\textsc{prog}\\
                    \glt    `S/he is giving advice'\\
                    \glt    `Está aconsejando' < ROF 04 1:64/el >\\
                    \glt    \textit{*raʔaˈm\textbf{â}{}-a}\\
                }
%                \pagebreak
                        \ex[]{
                        \textit{raˈʔàmaki}\\
                        \glt    {\textit{raˈʔ\textbf{à}ma-ki}}\\
                        \glt    give.advice-\textsc{pst.ego}\\
                        \glt    `I gave advice.'\\
                        \glt    `Aconsejé.'  < ROF 04 1:64/el >\\
                        \glt    \textit{*raʔaˈm\textbf{â}{}-ki}\\
                    }
                            \ex[]{
                            \glt    \textit{raˈʔìtʃ͡ili}\\
                            \glt    {\textit{raˈʔ\textbf{ì}tʃ͡a-li}}\\
                            \glt    speak-\textsc{pst}\\
                            \glt    `S/he spoke.'\\
                            \glt    `Habló.' \corpuslink{el1318[14_245-14_261].wav}{MFH el1318:14:24.5}\\
                            \glt    \textit{*raʔiˈtʃ͡\textbf{a}-li}\\
                        }
                                \ex[]{
                                \glt    \textit{raˈʔìtʃ͡i}\\
                                \glt   {\textit{raˈʔ\textbf{ì}tʃ͡a-i}}\\
                                \glt    speak-\textsc{impf}\\
                                \glt    `S/he used to speak.'\\
                                \glt    `Hablaba' \corpuslink{co1239[02_152-02_172].wav}{JLG co1239:2:15.2}\\
                                \glt    \textit{*raʔiˈtʃ͡\textbf{a}{}-a}\\
                            }
                                    \ex[]{
                                    \textit{raˈʔìtʃ͡aki}\\
                                    \glt    {\textit{raˈʔ\textbf{ì}tʃ͡a-ki}}\\
                                    \glt    speak-\textsc{pst.ego}\\
                                    \glt    `I spoke.'\\
                                    \glt    `Hablé' < SFH 05 1:98/el >\\
                                    \glt    \textit{*raʔiˈtʃ͡\textbf{a}{}-ki}\\
                                }
    \z
\z

Finally, stress-neutral suffixes are never stressed, an observation also made in §\ref{subsec: stress properties of monosyllabic roots} above. Following the pattern of unstressed trisyllabic roots (in (\ref{ex: unstressed trisyllabic roots}) and (\ref{ex: attested and hypothetical forms with stress-neutral suffixes})), we would expect unstressed disyllabic roots adding a stress-neutral suffix (like causative \textit{{}-ti}) and a stress-shifting suffix (like future singular \textit{-ma}) to have third syllable stress. These verbs, however, have second-syllable stress. This is illustrated in (\ref{ex: the unstressability of stress-neutral suffixes}).

%\pagebreak

\ea\label{ex: the unstressability of stress-neutral suffixes}
{The unstressability of stress-neutral suffixes}

    \ea[]{
    {\textit{aˈwítisa}}\\
    \glt    {\textit{aˈw\textbf{í}{}-ti-sa}}\\
    \glt    dance-\textsc{caus-cond}\\
    \glt    `if s/he makes them dance'\\
    \glt    `si lo hace bailar' < SFH 08 1:112/el >\\
    \glt    \textit{*awi-ˈt\textbf{î}{}-sa}\\
}
        \ex[]{
        \glt    \textit{raʔˈlisima}\\
        \glt    {\textit{raʔˈl\textbf{ì}{}-si-ma}}\\
          \glt  buy-\textsc{mot-fut.sg}\\
          \glt  `S/he will go around buying.'\\
          \glt  `Va a ir comprando.' < AHF 05 1:130/el >\\
          \glt  \textit{*raʔli-ˈs\textbf{i}{}-ma}\\
    }
            \ex[]{
            \glt    \textit{oˈsìsima}\\
            \glt    \textit{oˈs\textbf{ì}{}-si-ma}\\
             \glt   read.write-\textsc{mot-fut.sg}\\
             \glt   `S/he will go around reading/writing.'\\
             \glt   `Va a ir leyendo/escribiendo.'   < SFH 05 1:78/el >\\
             \glt   \textit{*osi-ˈs\textbf{ì}{}-ma}\\
        }
                \ex[]{
                \glt    \textit{tʃ͡oˈníkima}\\
                \glt    \textit{tʃ͡oˈn\textbf{í}-ki-ma}\\
                  \glt  fist.fight-\textsc{appl-fut.sg}\\
                  \glt  `They will fist-fight for someone.'\\
                  \glt  `Van a pelear a chingazos por alguien.' < SFH 05 1:67/el > \\
                  \glt  \textit{*tʃ͡oni-ˈk\textbf{î}-ma}\\
            }
    \z
\z

The unstressability of stress-neutral suffixes is further evidenced by unstressed monosyllabic roots. Only two of twenty-seven monosyllabic verbal roots are unstressed (\textit{ru} ‘say’ (‘\textit{decir}’) and \textit{tó} ‘bring’ (‘\textit{traer}’) (exemplified above in §\ref{subsec: stress properties of monosyllabic roots} ). These roots shift stress to stress-shifting suffixes, as shown in \tabref{tab:unstressed-monosyllabic}.

\begin{table}
\caption{Unstressed monosyllabic roots}
\label{tab:unstressed-monosyllabic}

\begin{tabularx}{\textwidth}{lXXXl}
\lsptoprule
&\textbf{Form} & \textbf{Gloss} & \textbf{Unattested} &  \\
\midrule
a.& ˈr\textbf{ú}{}-ki   & ‘say-\textsc{pst.ego}’   &  *ru-ˈk\textbf{i}   &  < JHF 04 1:27/el >\\
b.& ˈr\textbf{ú}{}-li   &  ‘say-\textsc{pst}’ &   *ru-ˈr\textbf{i} & < JHF 04 1:27/el >\\
c.& ˈr\textbf{ú}{}-simi  &  ‘say-\textsc{mot}’  & *ru-ˈs\textbf{i}mi  & < ROF 04 1:102/el >\\
d.& ˈr\textbf{ú}{}-ra     & ‘say-\textsc{rep}’ &  *ru-ˈr\textbf{a} & < ROF 04 1:102/el >\\
e.& ru-ˈm\textbf{ê}a   & ‘say-\textsc{fut.sg}’  &    & < JHF 04 1:27/el >\\
f.& ru-ˈs\textbf{â}   & ‘say-\textsc{cond}’  & &< ROF 04 1:102/el >\\
g.& ru-ˈb\textbf{ô}   & ‘say-\textsc{fut.pl}’    &  & < JHF 04 1:27/el >\\
h.& ru-ˈn\textbf{á}le   & ‘say-\textsc{desid}’  &    & < ROF 04 1:102/el >\\
\lspbottomrule
\end{tabularx}
\end{table}

We might have expected the forms in \tabref{tab:unstressed-monosyllabic}a--d to have second syllable stress when adding stress-neutral suffixes, following the pattern of disyllabic and trisyllabic unstressed roots. Instead, stress in these words is in the root, the first syllable. These cases suggest that the stress rule associated with stress-neutral suffixes must also meet the condition of being assigned \textit{within} the root. Stress-neutral suffixes are not part of the stress domain and are thus non-cohering. Cohering suffixes are suffixes that form one prosodic word with the preceding stem (evidenced by their phonological behavior as identical to morphologically simple words), and non-cohering suffixes form prosodic words of their own \parencite{booij2002prosodic}. (For general discussion about cohering and non-cohering affixes, see \citealt{dixon1977some, booij1977dutch, booij1999phonology, booij2002prosodic}).

The morphological conditions for stress and the stress patterns of morphologically complex words are addressed in \chapref{chap: verbal morphology} and \chapref{chap: prosody}.

\section{Initial three-syllable stress window}
\label{subsubsec: initial three-syllable stress window}

As described for closely related \ili{Taracahitan} languages (\ili{Norogachi Rarámuri} (ISO code: tar) \parencite{brambila1953gramatica}, \ili{Mountain Guarijío} (ISO code: var) \parencite{miller1996guarijio}, \ili{Yaqui} \parencite{demers1999prominence}, and \ili{Mayo} \parencite{hagberg1989floating}), stress in Choguita Rarámuri is restricted to an initial stress window. Stress window systems are characterized by alternations that keep stress within a two- or three-syllable margin of the edge of the domain, whether stress is unpredictable or not within this margin (\citealt{kager2012stress}, \citealt{everett1988metrical}, \citealt{green1995lapse}).\footnote{In \ili{Pirahã} (spoken in the Amazon), for example, stress is assigned to the heaviest syllable within the last three syllables of the word (\citealt{everett1988metrical}, \citealt{green1995lapse}).} Window effects are evidenced by alternations in compound, reduplicated and multiple-affixation constructions that maintain stress within a disyllabic margin in the \ili{Cahitan} languages \ili{Yaqui} and \ili{Mayo}, and a trisyllabic margin in \ili{Guarijío}. Initial three syllable stress windows are uncommon cross-linguistically:  outside of the \ili{Taracahitan} branch, only a handful of languages (most of them in the Americas) have been documented to possess an initial three-syllable window, including the \ili{Uto-Aztecan} (\ili{Numic}) language \ili{Comanche} (ISO code: com) \citep[][299]{smalley1953phonemic} (cited in \citealt{kager2012stress}).

Earlier descriptions of other Rarámuri varieties and \ili{Guarijío} have documented that stress is left-aligned in these languages and never placed beyond the third syllable, with alternations in reduplication and compounding maintaining this three-syllable restriction (see \citealt[][245]{brambila1953gramatica} for \ili{Norogachi Rarámuri} and \citealt[][49--50]{miller1996guarijio} for \ili{Mountain Guarijío}). Choguita Rarámuri has lost productive prefixation processes, but the three-syllable window is evident in other morphological constructions. Closely related \ili{Mountain Guarijío}, where prefixing reduplication is still productive, does exhibit systematic alternations to keep stress within the window \parencite{miller1996guarijio}.

The forms below show that Choguita Rarámuri stress is
% indisputably
clearly
left-aligned. Since the longest monomorphemic roots in this language are tetrasyllabic at most, the evidence in (\ref{ex: left alignment of stress}) involves morphologically complex forms.

\largerpage
\ea\label{ex: left alignment of stress}
{Left alignment of stress}

    \ea[]{
    \textit{poˈtʃ͡ípo}\\
    \textit{poˈtʃ͡í-po} \\
    jump\textsc{-fut.pl}\\
    `They will jump.'\\
    `Van a brincar.' < SFH 05 1:69/el >\\
}
        \ex[]{
        \textit{poˈtʃ͡ítisima}\\
        \textit{poˈtʃ͡í-ti-si-ma}\\
        jump\textsc{-caus-mot-fut.sg}\\
        `S/he will go along making them jump.'\\
        `Los va a ir haciendo brincar.'    < SFH 08 1:72/el >\\
    }
            \ex[]{
            \textit{amaˈtʃ͡îma}\\
            \textit{amaˈtʃ͡î-ma} \\
            pray\textsc{-fut.sg} \\
            `S/he will pray.'\\
            `Va a rezar.' < SFH 04 1:133/el >\\
        }
                \ex[]{
                \textit{amaˈtʃ͡îtima}\\
               \textit{amaˈtʃ͡î-ti-ma}  \\
                pray\textsc{-caus-fut.sg}\\
                `S/he will make them pray.'\\
                `Los va a hacer rezar.' < BFL 08 1:108/el >\\
            }
                    \ex[]{
                    \textit{aˈtísima}\\
                    \textit{aˈtísi-ma}\\
                    sneeze-\textsc{fut.sg}\\
                    `S/he will sneeze.'\\
                    `Va a estornudar.' < BFL 05 1:111/el >\\
                }
                        \ex[]{
                        \textit{aˈtístʃ͡anale}\\
                        \textit{aˈtís-tʃ͡a-nale} \\
                        sneeze\textsc{-ev-desid}\\
                        `It sounds like they want to sneeze.'\\
                        `Suena que quieren estornudar.' < SFH 07 1:73/el >  \\
                    }
                            \ex[]{
                            \textit{basaˈrôwiki}\\
                            \textit{basaˈrôwa-ki}   \\
                            stroll-\textsc{pst.ego}\\
                            `I strolled.'\\
                            `Pasée.' < BFL 05 1:162/el >\\
                        }
                                \ex[]{
                                \textit{basaˈrôwinima}\\
                                \textit{basaˈrôwa-ni-ma}\\
                                stroll-\textsc{desid-fut.sg}\\
                                `S/he will want to stroll.'\\
                                `Va a querer pasear.' < SFH 07 1:150/el >\\
                            }
    \z
\z

While the forms in (\ref{ex: left alignment of stress}a--e, g) could be ambiguous between second and third syllable stress and penultimate and antepenultimate stress, embedding these forms in further morphology reveals that the correct generalization about stress assignment can only be made with respect to the left edge of the prosodic word. Each pair of morphologically related words ((\ref{ex: left alignment of stress}a--b), (\ref{ex: left alignment of stress}c--d), (\ref{ex: left alignment of stress}e--f) and (\ref{ex: left alignment of stress}g--h)) shows that stress is constantly on the second or third syllable.

Choguita Rarámuri, like \ili{Norogachi Rarámuri} and closely related \ili{Guarijío}, also has constructions that display stress alternations that maintain stress within a left-aligned window margin. There are N-V constructions that are restricted to nouns referring to body parts and bodily fluids. In these constructions, the noun root is fully integrated with the verb morphologically, and both the noun root and the verb root can be used independently. As discussed in \chapref{chap: verbal morphology}, these properties are prototypical of ``body part incorporation'', a restricted kind of noun incorporation, which is common in languages of the Americas \parencite{baker1996polysynthesis}. Stress in these constructions is actively constrained by the grammar. If the head, the incorporated verb, has second syllable stress in isolation and if the first member, the body-part noun, is two syllables long, stress retracts to the verb’s first syllable, the construction’s third syllable.\footnote{In (\ref{ex: stress retraction in incorporated constructions}e--g), the glottal stop associated with the verbal roots does not emerge in the surface incorporated form due to the glottal prosody place restriction described in §\ref{subsec: glottal stop}} This is exemplified in the forms provided in (\ref{ex: stress retraction in incorporated constructions}).

\pagebreak
\ea\label{ex: stress retraction in incorporated constructions}
{Stress retraction in incorporated constructions}

        \ea[]{
        /buˈsí+kaˈsì/  → \textit{busi+ˈk\textbf{â}si }\\
        eye+break    \\
        `to become blind'\\
        `volverse ciego'\\
    }\label{ex: stress retraction in incorporated constructionsa}
            \ex[]{
            /roˈpâ+kaˈsì/  → \textit{ropa+ˈk\textbf{â}si}\\
            stomach+break     \\
            `to have a miscarriage' \\
            `abortar'\\
        }\label{ex: stress retraction in incorporated constructionsb}
                \ex[]{
                /buˈsí+boˈtá/ → \textit{busi+ˈb\textbf{ô}ta }\\
                eye+come.out\\
                `for an eye to come out' \\
                `salirse el ojo'\\
            }\label{ex: stress retraction in incorporated constructionsc}
                    \ex[]{
                    /kaˈwá+boˈtá/  → \textit{kawa+ˈb\textbf{ô}ta} \\
                    egg+come out \\
                    `for an egg to come out' \\
                    `salirse el huevo'\\
                }\label{ex: stress retraction in incorporated constructionsd}
                        \ex[]{
                        /kuˈtâ+biʔˈrì/    → \textit{kuta+ˈb\textbf{î}ri}\\
                        neck+twist   \\
                        `to neck-twist' \\
                        `torcerse el cuello'\\
                    }\label{ex: stress retraction in incorporated constructionse}
                            \ex[]{
                            /tʃ͡oʔˈmá+biˈʔwa/  →  \textit{tʃ͡oma+ˈb\textbf{î}}wa \\
                            mucus+clean\\
                            `to mucus-clean'   \\
                            `limpiar los mocos'\\
                        }\label{ex: stress retraction in incorporated constructionsf}
                                \ex[]{
                                /tʃ͡eˈréwa+biˈʔwa/    → \textit{tʃ͡ere+ˈb\textbf{î}wa} \\
                                sweat+clean\\
                                `to sweat-clean'  \\
                                `limpiar el sudor'\\
                            }\label{ex: stress retraction in incorporated constructionsg}
    \z
\z

All possible interactions of underlyingly stressed and unstressed roots are attested in these forms: unstressed noun plus unstressed verb (\ref{ex: stress retraction in incorporated constructionsa}), stressed noun plus unstressed verb (\ref{ex: stress retraction in incorporated constructionsb}), unstressed noun plus stressed verb (\ref{ex: stress retraction in incorporated constructionsc}), and stressed noun plus stressed verb (\ref{ex: stress retraction in incorporated constructionsd}). Regardless of the underlying stress make-up of the roots of the construction, stress is assigned in the first syllable of the head of the construction, the verbal root. The stress retraction phenomenon involves actual deletion of lexical inherent root stress from the head of the construction. The verbal root \textit{biʔˈw-a} ‘to clean’, for instance, is a stressed root (with fixed stress when adding stress-shifting suffixes) (e.g. (\ref{ex: stress properties of bi'wa})):

\ea\label{ex: stress properties of bi'wa}
{Stress properties of verb root \textit{biʔw-a} ‘clean, \textsc{tr}’ (\textit{‘limpiar’})}

% [Warning: Draw object ignored]

    \ea[]{
    \textit{biʔˈw\textbf{â}ma}\\
    \textit{biʔˈw\textbf{-â}{}-ma}   \\
    clean-\textsc{fut.sg}\\
    `S/he will clean.'\\
    `Va a limpiar.' < SFH 05 1:72/el >\\
}\label{ex: stress properties of bi'waa}
        \ex[]{
        \textit{biʔˈw\textbf{â}sa }\\
        \textit{biʔˈw\textbf{-â}{}-sa }  \\
        clean-\textsc{cond}\\
        `if s/he cleans'\\
        `si limpia' < SFH 05 1:72/el >\\
    }\label{ex: stress properties of bi'wab}
            \ex[]{
            \textit{biʔˈw\textbf{â}bo}\\
            \textit{biʔˈw\textbf{-â}{}-bo}  \\
            clean-\textsc{fut.pl}\\
            `They will clean.'\\
            `Van a limpiar.' < SFH 05 1:72/el >\\
        }\label{ex: stress properties of bi'wac}
                \ex[]{
                \textit{biʔˈw\textbf{â}{}nale}\\
                \textit{biʔˈw\textbf{-â}{}-nale}\\
                clean-\textsc{desid}\\
                `S/he wants to clean.'\\
                `Quiere limpiar.' < SFH 05 1:72/el >\\
            }\label{ex: stress properties of bi'wad}
                    \ex[]{
                    \textit{biʔˈw\textbf{â}si}\\
                    \textit{biʔˈw\textbf{â}{}-si}   \\
                    clean-\textsc{imp.pl}\\
                    `You all clean!'\\
                    `¡Limpien!' < SFH 05 1:72/el >\\
                }\label{ex: stress properties of bi'wae}
    \z
\z

In incorporation, however, this verbal root undergoes a stress shift one syllable to the left (e.g., \textit{tʃ͡oma+ˈbiwa} in (\ref{ex: stress retraction in incorporated constructions}f) above).
%\todo[inline]{There is no (23f)}. - FIXED

Stress in incorporated verbs, therefore, involves both stress deletion and stress-reassignment. \citet{brambila1953gramatica} and \citet{miller1996guarijio} interpret similar stress deletion and re-assignment facts in the \ili{Taracahitan} languages as evidence for a three-syllable stress window. Fourth syllable stress, which would result in the incorporated forms in (\ref{ex: stress retraction in incorporated constructions}a--f) if there were no stress reassignment, would fall outside this window, and is therefore retracted one syllable to the left.

The stress alternations in the incorporated forms of Choguita Rarámuri can alternatively be attributed to a morphological stress rule specific to incorporated construction rather than a stress window. This morphological stress rule would require stress to be assigned in the first syllable of the head of the incorporated construction. This morphological stress rule is defined in (\ref{ex: incoporated verb stress rule}).

\ea\label{ex: incoporated verb stress rule}

\textbf{Incorporation stress rule:} The head of the incorporation construction (the verbal root) must bear stress in the first syllable

\z

There are, however, further testing grounds for the window hypothesis. The behavior of trisyllabic nouns in incorporation is crucial in this regard. Choguita Rarámuri, like other {Uto-Aztecan} languages (e.g. \ili{Southern Paiute} (\citealt{sapir1930southern}) and \ili{Kawaiisu} (\citealt{zigmond1991kawaiisu})), tends to shorten its trisyllabic nouns to a disyllabic form when incorporated. These truncated forms in incorporation are shown in \tabref{tab:truncation-incorporation}.

\begin{table}
\caption{Noun truncation in incorporation}
\label{tab:truncation-incorporation}

\begin{tabularx}{\textwidth}{QXXQ}
\lsptoprule
\textbf{Underlying} \textbf{representation}& \textbf{Gloss} & \textbf{Bare} \textbf{noun}& \textbf{Incorporated} \textbf{verb}\\
\midrule
/tʃ͡eˈréwa+biʔˈwa/ & ‘sweat+clean’ &  tʃ͡eˈréwa &  tʃ͡ere+ˈbîwa\\
/tʃ͡aˈmèka+reˈpu/ &  ‘tongue+cut’&  tʃ͡aˈmèka &  tʃ͡ame+ˈrêpu\\
\lspbottomrule
\end{tabularx}
\end{table}

Truncation of tetrasyllabic nouns in the incorporated forms in \tabref{tab:truncation-incorporation} is ambiguously triggered by either an initial three-syllable stress window or a morphological incorporation stress rule defined in (\ref{ex: incoporated verb stress rule}). However, while most speakers completely reject non-truncated versions of the forms in \tabref{tab:truncation-incorporation}, for some speakers such forms are in fact interpretable even if never produced spontaneously. These non-truncated forms are shown in (\ref{ex: interpretable non-truncatewd incorporated verbs}), where angled brackets indicate that these forms are abstract and not spontaneously produced.

%\pagebreak

\ea\label{ex: interpretable non-truncatewd incorporated verbs}
{Interpretable, non-truncated incorporated verbs}

    \ea[]{
    \textit{tʃ͡ameˈkârepu}\\
    <tʃ͡ameˈkâ+repu>\\
    tongue+cut\\
    `to cut the tongue'\\
    `cortar la lengua'\\
}
        \ex[]{
        \textit{kutaˈtʃ͡írepu}\\
        <kutaˈtʃ͡í+repu>  \\
        neck+cut\\
        `to cut the neck'\\
        `cortar el cuello' \\
    }
    \z
\z

These forms, with stress in the third syllable, are never spontaneously produced, but their intended meanings can be retrieved. Equivalent non-truncated forms with stress in the fourth syllable, on the other hand, were completely rejected and their intended meaning could not be recovered. These forms are shown in (\ref{ex: uninterpretable, non-truncated incorporated verbs}).

\ea\label{ex: uninterpretable, non-truncated incorporated verbs}
{Uninterpretable, non-truncated incorporated verbs}

    \ea[]{
    \doublebox{\textit{*tʃ͡ameka+ˈrêpu}}{tongue+cut}\\
}
        \ex[]{
        \doublebox{\textit{*kutatʃ͡i+ˈrêpu}}{neck+cut}\\
    }
    \z
\z

The incorporation stress rule is violated in the interpretable cases in (\ref{ex: interpretable non-truncatewd incorporated verbs}), but the initial three-syllable stress window is violated in the completely rejected forms in (\ref{ex: uninterpretable, non-truncated incorporated verbs}), suggesting that there is indeed an overarching, exceptionless rule that restricts stress to the first three syllables of the word in Choguita Rarámuri. There is indeed not a single form in the Choguita Rarámuri corpus that has stress outside this three-syllable range.

While the initial three-syllable window is mainly manifested synchronically as a static, exceptionless generalization in the native vocabulary of Choguita Rarámuri, loanwords from \ili{Spanish} contribute further evidence of this phenomenon, exhibiting alternations related to this restriction. Specifically, Choguita Rarámuri loanwords retain original prominence from the \ili{Spanish} source (\ref{ex: Spanish loanwords in CR}a--b), in addition to a default HL tone (see also \citealt{caballero2013procesos} and \chapref{chap: prosody} below about prosodic loanword adaptation). Given that \ili{Spanish} has a final three syllable window, there are cases of loanwords where the prominence would emerge beyond the initial three syllable margin in Choguita Rarámuri. As shown in (\ref{ex: Spanish loanwords in CR}c--e), the repair is to truncate a syllable of the \ili{Spanish} source to match the prosodic requirement of Choguita Rarámuri (in the examples below, stressed syllables are highlighted with boldface and truncated syllables with underlining in the \ili{Spanish} source forms).

\ea\label{ex: Spanish loanwords in CR}
{\ili{Spanish} loanwords in Choguita Rarámuri}

    \ea[]{
    \glt    {\textit{maˈ\textbf{sâ}na}}\\
    \glt    ‘apple’  \\
    \glt    Sp. \textit{man\textbf{za}na}\\
}
        \ex[]{
        \glt   {\textit{mehoˈ\textbf{râ}ra}}\\
        \glt   ‘acetaminophen’ \\
        \glt    Sp. \textit{mejo\textbf{ral}}\\
    }
            \ex[]{
            \glt    {\textit{naugu\textbf{ˈrâr}}}\\
            \glt      ‘inaugurate’ \\
            \glt    Sp. \textit{inaugu\textbf{rar}}\\
        }
                \ex[]{
                \glt    \textit{sera\textbf{ˈdê}roʧ͡i}\\
                \glt    {\textit{sera\textbf{ˈdê}ro-ʧ͡i}}\\
                \glt    log.house-\textsc{loc}\footnote{Loanword nouns from \ili{Spanish} are often incorporated in the lexicon with a locative suffix (for more details about loanword adaptation, see §\ref{sec: loanword prosody}).}
                \\
                \glt    {Sp. \textit{ aserra\textbf{de}ro}}\\
            }
                    \ex[]{
                    \glt    {\textit{kiri\textbf{ˈsâan}te}}\\
                    \glt    ‘fertilizer’    \\
                    \glt    Sp. \textit{fertili\textbf{san}te}\\
                }
    \z
\z

Truncation of the source words targets the initial syllable in each of the examples in (\ref{ex: Spanish loanwords in CR}).
%this paragraph needs to be rephrased or relocated

While languages with a final ternary stress window -- permitting only final, penultimate, or antepenultimate stress -- are not uncommon (e.g.  \ili{Imbabura Quechua}, \ili{Macedonian}, \ili{Greek}, \ili{Hebrew}, \ili{Spanish}, \ili{Polish}, \ili{Zoque}, \ili{Italian}, to name just a few), initial three-syllable stress windows are a typologically highly marked pattern (\citealt{kager2012stress}; see also \citealt{caballero2011morphologically}; \citealt{martinez2013exploration}). Initial three syllable stress windows have been described in languages outside the Uto-Az\-tec\-an language family, including \ili{Icua Tupi} (\ili{Tupi}), \ili{Terena} (\ili{Arawakan}), \ili{Wishram Chinook} (\ili{Chinookan}) (\cite{kager2012stress}), and \ili{Azkoitia Basque} (\cite{hualde1998gap}).
%update information on language with initial three-syllable windows


% %%please move the includegraphics inside the {figure} environment
% \includegraphics[width=\textwidth]{figures/GrammardraftJuly182017-img5.png}
