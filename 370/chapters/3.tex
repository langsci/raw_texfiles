\chapter{\label{bkm:Ref488657128}Corpus examples}\label{sec:3}

Most of the examples provided in this chapter are taken from the \textit{International Corpus of English} (\is{International Corpus of English}ICE), which is the corpus exclusively used in the quantitative analyses in Chapters~\ref{sec:7}–\ref{sec:11} (see \sectref{sec:6.1}). Some of these stem from varieties not otherwise considered in this study (but see \citealt{Schützler2018c}), namely US-\ili{American English} and \ili{New Zealand English}. Additional examples are taken from the \textit{Corpus of Historical American English} (\is{corpora!COHA}COHA, \citealt{Davies2010}), and from the eight corpora I collectively refer to as the \textit{extended Brown family of corpora} (or \is{corpora!xBrown}xBrown, for short; see \citealt{Baker2009}): the \is{corpora!BBrown}BBrown, \is{corpora!Brown}Brown, \is{corpora!Frown}Frown and \is{corpora!AmE06}AmE06 corpora of written \ili{American English} (comprising data from the early 1930s, the early 1960s, the early 1990s and the year 2006, respectively) and the corresponding \is{corpora!BLOB}BLOB, \is{corpora!LOB}LOB, \is{corpora!FLOB}FLOB and \is{corpora!BE06}BE06 corpora of written \ili{British English}.\footnote{\is{corpora!BBrown}BBrown
  (sometimes called “Lancaster 1931 corpus”) was compiled by Marianne \citet{Hundt2004} at the University of Zurich, and \is{corpora!AmE06}AmE06 was compiled by Paul \citet{Baker2010} at Lancaster University; standard references for the other six corpora are
  \citeauthor{FrancisKučera1979} (1979; \is{corpora!Brown}Brown),
  \citeauthor{HundtEtAl1999} (1999; \is{corpora!Frown}Frown),
  \citeauthor{LeechSmith2005} (2005; \is{corpora!BLOB}BLOB),
  \citeauthor{JohanssonEtAl1978} (1978; \is{corpora!LOB}LOB),
  \citeauthor{HundtEtAl1999} (1999; \is{corpora!FLOB}FLOB) and
  \citeauthor{Baker2009} (2009: 312–316; \is{corpora!BE06}BE06).
  }
Examples were selected
(i)~to illustrate the semantico-pragmatic properties of the three semantic types on the basis of more examples than was possible in \sectref{sec:2.2};
(ii)~to show different combinations of semantic types and conjunctions;
(iii)~to discuss semantically \is{ambiguity}ambiguous cases that defy a clear classification; and
(iv)~to show interesting syntactic realisations that do not follow the main patterns outlined in \sectref{sec:2.3}. The first two aspects will be discussed in \sectref{sec:3.1}–\ref{sec:3.3}, while
(iii) and
(iv) will be discussed in \sectref{sec:3.4} and \sectref{sec:3.5}, respectively.

\begin{sloppypar}
Most examples in \sectref{sec:3.1}–\ref{sec:3.3} follow the majority pattern, i.e. \is{finite clauses}finite clauses complementing subordinating conjunctions. Examples found in the corpora will sometimes be re-constructed, for example by altering the position of a conjunction and thereby changing the status of clauses (subordinate vs matrix).\footnote{The term \textit{re-constructed} is spelled with a hyphen to highlight that it is used in the sense of ‘constructed again, in an altered way’ and does not refer to inferred historical forms.} This kind of permutation can help to show the relatedness of semantic types, particularly regarding \is{concessives (types of)!anticausal}anticausal and \is{concessives (types of)!epistemic}epistemic CCs, and can thus contribute to a better understanding of how they were classified. Where this applies, original corpus examples will be indexed as “a”, while derived/re-constructed examples will rank lower in the index, i.e. as “b”, “c”, etc. Furthermore, the corpus source of original examples will be stated in brackets, but no such statement will be provided for derived (or re-constructed) examples. The connective in each example will be given in bold print, while its complement will be italicised~– a convention I already followed in Chapter \ref{sec:2}.
\end{sloppypar}

\section{\label{bkm:Ref487180766}Anticausal concessives}\label{sec:3.1}

The following (interrelated) characteristics are considered central to a definition of \is{concessives (types of)!anticausal}anticausal concessives, as introduced in \sectref{sec:2.2.1}:
(i)~Propositions are connected by a \isi{topos}, i.e. a \is{presupposition}presupposed relation of \isi{cause and effect};
(ii)~the \isi{topos} is based on real-world \is{adverbials!reason}causality and thus goes beyond the mere assumption of a likely concomitance of circumstances;
(iii)~the \isi{topos}~– and thus the relation between propositions~– is not reversible (one proposition is assumed to result in the other, but not vice versa); and
(iv)~the cause may be directly or indirectly connected to the effect. These four aspects will be discussed in connection with the examples presented in the following paragraphs.

\begin{sloppypar}
Examples (\ref{ex:40}a), \REF{ex:41} and \REF{ex:42} are typical instances of \is{concessives (types of)!anticausal}anticausal concessives constructed on the basis of the two conjunctions \textit{although} and \textit{even though}. In (\ref{ex:40}a), the \isi{topos} is that growing older is likely to result in greying hair, or, more generally formulated: \textsc{ageing~}→~\textsc{changed} \textsc{appearance}. The \is{cause and effect}cause-and-effect relation\-ship is perceived, even though the precise causes (e.g. lower concentrations of pigment as a concomitant of ageing) may not be fully known or understood. In this case, age is an indirect cause, but it is quite firmly linked to the effect (greying hair) and thus the intermediate chain of direct causes is redundant and does not need to be stated.
\end{sloppypar}

\ea\label{ex:40}
    \ea\label{bkm:Ref487265706}Patience was […] already greyer-haired than Miriam, \textbf{although} \textit{she was eleven years her junior}. (ICE-GB:W2F-007)\\
    \ex\label{ex:40b} \textbf{Although} \textit{Patience was already greyer-haired than Miriam}, she was eleven years her junior. (Re-constructed into epistemic concessive)\\
\z
\z

\begin{sloppypar}
The re-constructed variant illustrates what happens when we invert \is{concessives (types of)!anticausal}anticausal concessives: Example (\ref{ex:40}b) is of course a meaningful concessive construction, but it cannot be classified as anticausal since grey hair cannot be viewed as a direct or indirect real-world cause of advanced age (*\textsc{changed} \textsc{appearance~}→~\textsc{ageing}). Instead, grey hair may trigger certain conclusions concerning its possible underlying causes, which is why the re-constructed example is best read as an epistemic concessive. The juxtaposition of examples like these highlights crucial aspects of the difference between \is{concessives (types of)!anticausal}anticausal and \is{concessives (types of)!epistemic}epistemic concessives, as will be further discussed in \sectref{sec:3.2}.
\end{sloppypar}

In \REF{ex:41}, the long continuation of an environmental disaster~– in this particular case the 1979 Ixtoc I oil spill in the Gulf of Mexico~– will normally lead to more severe damages, which is the general \isi{topos} underlying this example. The extent to which an event of this kind is harmful depends (among other things) on its duration, but duration certainly does not depend on environmental consequences~– the \isi{topos} is not invertible, and the construction cannot be inverted either without changing its semantico-pragmatic status. Topoi like this one are quite complex in that they involve two conditions: \textsc{if} the effects of a situation are negative (a kind of prerequisite) \textsc{and} \textsc{if} the situation persists for a long time, \textsc{then} there will be particularly dire consequences. The processing of sentences like \REF{ex:41} poses few problems, which shows that even complex \is{topos}topoi are accessed quite routinely by language users.

\ea\label{ex:41}   \label{bkm:Ref487277467}And the Ixtoc blow-out in the Gulf of Mexico~– \textbf{even} \textbf{though} \textit{it gushed for months}~– did less harm than it might have […]. (ICE-GB:W2B-029)
\z

In \REF{ex:42}, the fact that someone has left a long time ago will normally be expected to result in the fading and loss of the memories associated with them. More generally, the passage of time has certain, normally expected effects on memory: \textsc{passage} \textsc{of} \textsc{time} → \textsc{forgetting}. Contrary to this topos, SP/W in the example states that they still have some remembrance of a person’s face, associated with pleasant sensations, even though that person has left long ago.

\ea\label{ex:42} \label{bkm:Ref488143559}  \textbf{Although} \textit{she has left me for a long time}, the rough sketch of her face still floats on my mind like a beautiful picture. (ICE-HK:W2F-008)
\z

Example \REF{ex:43} is based on a topos whereby achieving one’s purpose (in this case completing one’s studies) increases the likelihood of departing from a certain location. A (somewhat informal) \isi{topos} could be \textsc{mission} \textsc{accomplished} → \textsc{departure}, which is a chain of \isi{cause and effect} perhaps typical of university students, who are often viewed as highly mobile. The syntactic structure of the subordinate clause in this example is also quite interesting as it does not conform to \is{varieties of English!L1}L1 norms.\footnote{In the data, rare syntactic realisations like this were not assigned to a separate category, which is why some of them are discussed qualitatively in this section.}

\ea\label{ex:43}   \label{bkm:Ref487277508}\textbf{Though} \textit{I have finished my studies} I will stay few [sic] more years here. (ICE-IND:W1B-011)
\z

The following, syntactically rather complex example is best understood in an anticausal reading. The subordinate clause states that the proposal under discussion (a projected PhD program) has been commented upon favourably from various sides and that there are no apparent reasons for a delay in its implementation. The matrix clause states that another program, the Ed.D. (“Doctor of Education”), was planned later and is considerably more expensive, but was nevertheless launched earlier than the PhD.

\eanoraggedright\label{ex:44} \label{bkm:Ref497554696}  Meanwhile, \textbf{even} \textbf{though} \textit{our proposal has received both external and internal praise}, \textit{and neither Chancellor Price nor Provost Sellers has raised substantive questions or justified the delays}, the Ed.D. program~– planned after ours and costing far more~– is up and running. (ICE-USA, business letters)
\z

The concessive reading is strong and straightforward, although the fact that the PhD is not yet up and running is not overtly stated but merely implied, and although two additional arguments~– lesser cost and earlier planning of the PhD compared to the Ed.D.~– are provided in the matrix clause, i.e. not where they would conceptually belong. The interpretation of such examples poses no problems, which seems to suggest that meaning-making does indeed happen at the constructional level, i.e. on the basis of all the evidence that is provided, and not via a simple one-to-one comparison of propositions in subordinate and matrix clause. Four \is{topos}topoi could be argued to be effective here, two appearing as coordinated parts of the subordinate clause and another two “outsourced” to the matrix-clause parenthesis:
(i)~\textsc{positive} \textsc{evaluation} → \textsc{swift} \textsc{implementation},
(ii)~\textsc{no} \textsc{objections} \textsc{or} \textsc{questions} → \textsc{swift} \textsc{implementation},
(iii)~\textsc{early} \textsc{planning} → \textsc{swift} \textsc{implementation}, and
(iv)~\textsc{lower} \textsc{costs} → \textsc{swift} \textsc{implementation}. All four could of course be subsumed under a more general \isi{topos} linking positive attributes (like cost-efficiency and good organisation) to success (i.e. swift implementation), and they can also be argued to constitute an interacting causality chain, with early planning and low costs leading to positive evaluation and fewer questions being asked. The structure in the subordinate clause is an example of \citegen{Aarts1988} category of “coordinated concessives” (see \sectref{sec:2.3.2}) and finds its correlate in the coordinated structure of the matrix-clause parenthesis (\textit{planned after ours and costing far more}).

  Examples \REF{ex:45} and \REF{ex:46} provide further illustrations of \is{topos}topoi operative in \is{concessives (types of)!anticausal}anticausal concessives. The first one is straightforward: Nurses are expected to have been exposed to and be aware of all kinds of issues to do with the human body, including sexual ones, which is why the subject’s ignorance of condoms comes as a surprise. The \isi{topos} \textsc{medical} \textsc{training} → \textsc{knowledge} \textsc{of} \textsc{bodily} \textsc{issues} would then include not only nurses, but also doctors, for example.

\ea\label{ex:45}   \label{bkm:Ref487277585}[\textbf{A}]\textbf{lthough} \textit{a nurse}, she didn’t know what a condom was. (Frown, press reviews)\\
\z

In \REF{ex:46}, contrary to expectation, the removal of a law requiring goods from the American colonies to be shipped to Ireland indirectly (via English ports) does not lead to direct trade between America and Ireland. The \isi{topos} is not entirely universal but requires some understanding of (Western-hemisphere) trade mechanisms and of the relevant historical context, both of which are provided by the context that is not shown here.

\ea\label{ex:46}   \label{bkm:Ref487449431}\textbf{Though} \textit{this restriction was eliminated in 1731}, Irish trade continued throughout the eighteenth century to be primarily with England […]. (AmE06, learned and scientific)\\
\z

The examples of anticausal CCs discussed in this section illustrate some of the \is{topos}topoi that exist, and highlight on what basis occurrences were classified as \is{concessives (types of)!anticausal}anticausal. However, they can only represent a fraction of possible \is{cause and effect}cause-and-effect relationships that are stored as part of language users’ \isi{world knowledge} and can be drawn upon when constructing or decoding concessives.

\section{\label{bkm:Ref487443918}Epistemic concessives}\label{sec:3.2}

Three instances of epistemic CCs and their anticausal re-constructions are shown in (\ref{ex:47}–\ref{ex:49}). This semantic type is much rarer than anticausal and dialogic concessives (see results in \chapref{sec:8}). While the notion of the \textit{topos} is typically discussed in connection with \is{concessives (types of)!anticausal}anticausal concessives, it plays an integral role with regard to \is{concessives (types of)!epistemic}epistemic concessives, too. However, there is what could be called an inverted direction of \isi{inference}: Instead of an expected result or effect, an expected or likely cause or underlying factor is inferred to motivate the observed outcome.

In (\ref{ex:47}a), if someone is optimistic about certain developments, one possible conclusion might be that this is due to facts or information of some kind (here: “confirmation from Baghdad”). That is, given the observed outcome or “symptom”, one makes \is{inference}inferences concerning the possible underlying causes. The mechanism in the \is{concessives (types of)!epistemic}epistemic concessive of the example is based on the dissonance between inferred cause and actual fact.

\ea\label{ex:47}
    \ea\label{ex:47a}\label{bkm:Ref487278203}[\textbf{A}]\textbf{lthough} \textit{he was optimistic about the release}, he had received no confirmation from Baghdad. (ICE-GB:S2B-006)\\
    \ex\label{ex:47b}He was optimistic about the release, \textbf{although} \textit{he had received no confirmation from Baghdad}. (Re-constructed into anticausal concessive)\\
\z
\z

In (\ref{ex:47}b), the sentence has been re-constructed into an anticausal concessive based on the same intra-constructional mechanisms as the examples in the previous section. One could argue that there is a single \isi{topos} underlying both variant constructions, namely \textsc{positive} \textsc{signals} → \textsc{optimism}.

Example (\ref{ex:48}a) is about Goh Chok Tong, the second Prime Minister of Singapore, and how he grew up at the time of Singapore’s struggle for independence from the United Kingdom during the 1950s. Observing someone like him following his relatives to pro-independence rallies would naturally lead to the conclusion that he is generally involved in pro-independence politics, a conclusion that turns out to be false in this case. Along very similar lines as in \REF{ex:47} above, re-construction into the anticausal concessive in (\ref{ex:48}b) is relatively easy.

\ea\label{ex:48}
    \ea\label{ex:48a}\label{bkm:Ref487463711}[H]e was not really caught up in the struggle for independence like his uncle and aunt, \textbf{though} \textit{he followed them to rallies}. (ICE-SING:W2B-001)\\
    \ex\label{ex:48b}\textbf{Though} \textit{he was not really caught up in the struggle for independence like his uncle and aunt}, he followed them to rallies. (Re-constructed into anticausal concessive)\\
\z
\z

In an epistemic reading, the sentence in (\ref{ex:49}a) could be rephrased as follows: ‘Mount Abu~– a hill station in Rajasthan, India~– has a lot to offer to tourists, although one might conclude otherwise, seeing that it is not as well-known as other Indian hill stations’. Example (\ref{ex:49}b) once again demonstrates how closely related prototypical \is{concessives (types of)!epistemic}epistemic and \is{concessives (types of)!anticausal}anticausal CCs are, and how easily one can be transformed into the other.

\ea\label{ex:49}
    \ea\label{ex:49a}\label{bkm:Ref487278426}Mount Abu, \textbf{though} \textit{a lesser known hill station of the country}, has much to offer to tourists. (ICE-IND:S2B-002; commas added)\\
    \ex\label{ex:49b}Mount Abu, \textbf{though} \textit{it has much to offer to tourists}, is a lesser known hill station of the country. (Re-constructed into anticausal concessive)\\
\z
\z

Examples (\ref{ex:50}--\ref{ex:52}) are further typical examples of epistemic concessives. In \REF{ex:50}, speaking of flux encourages the conclusion that some kind of flow has been observed, but it cannot \textit{result} in there being flux, and thus an anticausal reading is not possible. The construction shown in \REF{ex:51} is an equally clear-cut case of epistemic concession. If one encounters a rug that is 5′7″ by 7′ in size (ca. 3.6 m\textsuperscript{2}), one might draw certain conclusions as to its functions, but that of a prayer rug is unlikely to be among them, as such rugs will normally~– or prototypically, in Western perception~– be smaller. Thus, in the example the proposition marked by \textit{although} triggers certain conclusions and inferences which do not agree with the facts. As I have pointed out in \citet[203, footnote 4]{Schützler2018b}, this example is not meaningful in contexts where large prayer rugs of this type are in fact used, and of course in societies or communities that know nothing about prayer rugs at all. Finally, in \REF{ex:52}, upward-staring, open but unseeing eyes are likely to lead to the conclusion that a person is dead, while the man in the example has merely fainted. Once again, conclusions drawn on the basis of observed evidence turn out to be in disagreement with reality, which is why the entire construction is categorised as an epistemic concessive.

\ea\label{ex:50}   \label{bkm:Ref487279094}[\textbf{A}]\textbf{lthough} \textit{we speak of flux}, there is nothing which actually flows. (LOB, learned)\\
    \ex\label{ex:51} \label{bkm:Ref487279077}[\textbf{A}]\textbf{lthough} \textit{five feet seven inches by seven feet in size}, it is a prayer rug […]. (COHA, 1904, magazines)\\
    \ex\label{ex:52} \label{bkm:Ref487279082}The man had only fainted, \textbf{even} \textbf{though} \textit{his eyes stared upward, open and unseeing}. (COHA, 1961, fiction)\\
\z

The examples of \is{concessives (types of)!epistemic}epistemic concessives in this section illustrate the different mechanisms involved in the construction and decoding of this semantic type. They have in common that, based on some observation expressed in the subordinate clause, certain \is{inference}inferences are made. Those \is{inference}inferences concern states of affairs (including mental states and personality traits) or events that can be interpreted as having caused or at least contributed to the “symptoms” stated in the subordinate component. It was demonstrated that \is{concessives (types of)!epistemic}epistemic concessives can in many cases be conceptualised as inverted \is{concessives (types of)!anticausal}anticausal concessives and can therefore easily be re-constructed into the latter type. \is{concessives (types of)!anticausal}Anticausal and \is{concessives (types of)!epistemic}epistemic CCs seem closely related: Both are explicable in terms of a single inferential mechanism, but they differ in the direction of the \isi{inference} (\textsc{cause} → \textsc{effect} vs \textsc{effect} → \textsc{cause}). Any proposition will trigger \is{inference}inferences about expected consequences and expected causes, but \is{concessives (types of)!anticausal}anticausal and \is{concessives (types of)!epistemic}epistemic concessives explicitly capitalise on this, emphasising what could be called \is{inference!progressive}\textit{progressive} (forward) or \is{inference!regressive}\textit{regressive} (backward) \textit{inference}.

\section{\label{bkm:Ref487180775}Dialogic concessives}\label{sec:3.3}
\begin{sloppypar}
The class of \is{concessives (types of)!dialogic}dialogic concessives is perhaps the most heterogeneous one of the three categories employed in this study. In contrast to anticausal and epistemic concessives, propositions that are juxtaposed in \is{concessives (types of)!dialogic}dialogic concessives are not linked inferentially. That is, \is{inference}inferences triggered by the proposition in the subordinate clause can of course not be switched off entirely, but they do not relate directly to the matrix-clause proposition.\footnote{At the end of \sectref{sec:3.2} I suggested that a proposition invariably triggers \is{inference}inferences in one or the other direction (\textit{progressively} or \textit{regressively}, as I put it), but concessives may or may not capitalise on this tendency in the way that propositions are fused into a single construction.} This definition of \is{concessives (types of)!dialogic}dialogic concessives ex negativo will be made clearer by the corpus examples in this section, which demonstrate some of the concrete mechanisms at work in this functional type. To repeat the essentials of what was explained in \sectref{sec:2.2.3}, the two propositions in \is{concessives (types of)!dialogic}dialogic concessives provide pragmatically different comments on the same situation in the sense that
(i)~they both suggest different conclusions or courses of action,
(ii)~one qualifies or corrects the other (e.g. curtailing its credibility or the authority on which it is made), or
(iii)~one provides an alternative perspective on the situation described by the other.
\end{sloppypar}

The matrix-clause proposition in \REF{ex:53} describes some cricket ground as “surrounded by slag heaps”. There is no obvious inferential link between this and the proposition in the subordinate clause, and the association between the two seems quite loose. What the subordinate clause (“I’ve not visited myself”) does, however, is comment on the credibility of the matrix-clause proposition: SP/W are explicit about not having been to the cricket ground themselves; by making the second-hand nature of the information transparent, the message is qualified, and a more reserved interpretation is encouraged.

\ea\label{ex:53}   \label{bkm:Ref487559655}And \textbf{although} \textit{I’ve not visited myself}, the cricket ground is surrounded by slag heaps […]. (ICE-GB:S2A-044; comma added)\\
\z

The concessive is dialogic in the sense that, metaphorically speaking, AD/R needs to negotiate a conflict that exists between propositions, resulting in a compromise solution for the overall pragmatic outcome. \citet[166]{Hilpert2013a} describes \is{concessives (types of)!dialogic}dialogic concessives (which he calls “\is{concessives (types of)!speech-act}speech-act concessives”, following \citealt{Sweetser1990}) as “mixed messages”, which agrees quite well with the example above. In dialogic CCs of this type, with the subordinate component undermining the \isi{authority} of \is{speaker/writer}SP/W, an inversion (via the reattachment of the subordinator to the other clause) is often not feasible.

In \REF{ex:54}, the age of a sacral building (here: Glasgow Cathedral) is given as seven hundred years, which is followed by a comment to the effect that religious activity in the same location goes back even further than that. This changes the pragmatics of the entire construction by further strengthening the sense of antiquity and tradition that is created. The addition of such informational nuances makes more complex and multi-faceted interpretations possible.

\ea\label{ex:54}   \label{bkm:Ref487280197}The best parts of this building are seven hundred years old, \textbf{though} \textit{there has been worship here for a great deal longer}. (ICE-GB:S2A-020)\\
\z

Example (\ref{ex:55}a) follows a relatively common semantic pattern, which could be labelled \textsc{unity in diversity}.\footnote{\label{fn38}Setting up a typology of such meaning patterns frequently found in \is{concessives (types of)!dialogic}dialogic concessives would be worth an independent research effort but goes beyond the scope of the present study. It may also turn out to be a bottomless pit for the researcher, due to the unknown and potentially vast number of such patterns, their culture-specificity, as well as their open-class character, i.e. the tendency for new ones to emerge.} The matrix-clause proposition focuses on differences between Confucianism and Christianity concerning “the ultimate”, while the \is{initial position}initial subordinate clause highlights the fact that the goal of finding or experiencing this ultimate is something both have in common. The two propositions provide two pieces of evidence on whose basis Christianity and Confucianism can be compared. Inferential trajectories between the two propositions hardly play a role in this kind of construction; rather, the focus is on their dialogic relationship, characterised by reciprocal \is{modification (or qualification)}qualification. In (\ref{ex:55}b), inverting the status of clauses by attaching the subordinator to the original matrix clause shifts the focus of the statement, but it hardly affects the interpretation of the whole.

\eanoraggedright\label{ex:55}
    \ea\label{ex:55a}\label{bkm:Ref487280198}[\textbf{A}]\textbf{lthough} \textit{both traditions direct human being towards the ultimate},  Confucianism discovers the ultimate immanent in human being whereas Christianity finds meaning in the ultimate only by transcending human being […]. (ICE-HK:W2A-005)\\
    \ex\label{ex:55b}Both traditions direct human being towards the ultimate, \textbf{although} \textit{Confucianism discovers the ultimate immanent in human being whereas Christianity finds meaning in the ultimate only by transcending human being}. (Re-constructed)\\
  \z
\z

Similar to \REF{ex:55} above, \REF{ex:56} is based on a relatively common pattern, which might be labelled \textsc{quantity vs quality} (cf. Footnote~\ref{fn38}). Discussing a particular film genre, the proposition in the matrix clause states that during a certain period in the past, many films of this type were produced in Hong Kong. The subordinate clause elaborates that the films referred to in the matrix clause were not on a particularly grand scale~– apparently compared to prototypical exemplars of the genre, or to a specific, present-day example.\footnote{The message may also be that the number of films was still relatively low compared to other countries, and we would need to turn to the context to work this out more precisely. In this case the label presented here (\textsc{quantity} \textsc{vs} \textsc{quality}) would not hold.}

\ea\label{ex:56}   \label{bkm:Ref487559663}Thirty years ago Hong Kong made many such films, \textbf{even though} \textit{not a} [sic] \textit{such grand scale}. (ICE-HK:S2B-033; comma added)\\
\z

In the example, it does not seem possible to argue for an anticausal or epistemic inferential trajectory between the two propositions; the construction as a whole simply presents two pragmatic stances, one pointing in a more positive direction (\textsc{quan\-ti\-ty}~= “many”), the other serving as a hedge (\textsc{quality}~= “not on such a grand scale”). Constructions of this type also illustrate the lack of a clear boundary between concessive and \is{adverbials!contrast}adversative meaning.

In \REF{ex:57}, the proposition in the initial matrix clause assures AD/R that their article will be published soon, only to undermine the meaning of \textit{soon} in the following sub\-or\-\-dinate clause and thus to imply that it might in fact still take a while for the article to appear.\footnote{Also note the interesting use of V-\textit{ing} in the matrix clause of this example from \il{Indian English}IndE.}

\ea\label{ex:57}\label{bkm:Ref487559681}So, now you can rest assured that the article is appearing soon, \textbf{though} \textit{one doesn’t know how to define ‘soon’}. (ICE-IND:W1B-008)\\
\z

While the dialogic element in \REF{ex:53} above lies in questioning the \isi{authority} of \is{speaker/writer}SP/W (whose evidence was qualified as being second-hand, not based on personal observation), \REF{ex:57} is dialogic in questioning the \isi{authority} (or precision) of language itself.

As in \REF{ex:57}, the qualifying proposition in \REF{ex:58} follows the matrix clause. The construction as a whole is mainly concerned with the chances of success of a proposed piece of legislation (the “local option proposal”).

\ea\label{ex:58}\label{bkm:Ref487629172}A House committee which heard his local option proposal is expected to give it a favorable report, \textbf{although} \textit{the resolution faces hard sledding later}. (Brown, press reportage)\\
\z

The matrix clause opens on an optimistic note, stating that a positive evaluation is expected in the initial stage of the procedure, while the subordinate clause dampens expectations by adding that “hard sledding”, i.e. a more critical assessment and perhaps resistance, is to be expected at a later stage. It is quite typical for the tension between the two different pragmatic stances in \is{concessives (types of)!dialogic}dialogic concessives not to be resolved; in fact, it is perhaps one of the main purposes of this functional type to involve \is{addressee/reader}AD/R in the meaning-making process (cf. \sectref{sec:2.2.3}).

\begin{sloppypar}
The situation in (\ref{ex:59}a) concerns the poet (and novelist) Thomas Hardy who revised his poems many times; this process of potentially far-reaching aesthetic consequences is qualified by saying that it did not result in dramatic stylistic changes. Re-constructing the sentence by moving the subordinator and thus changing the status of clauses (subordinate vs matrix) once again hardly changes the overall pragmatics, as is demonstrated by the variant example (\ref{ex:59}b). It is also interesting to note that the core elements of one proposition (\textit{he\textsubscript{} }\textit{revised}) resurface as the subject of the other (\textit{the revisions}). This kind of resumed topic~– regularly realised as a \textsc{pro}-form (typically \textit{this})~– makes explicit that both propositions are in fact concerned with a single situation.
\end{sloppypar}

\ea\label{ex:59}
  \ea\label{bkm:Ref487279631}  And \textbf{though} \textit{in his later years he revised his poems many times}, the revisions did not alter the essential nature of the style which he had established before he was thirty […]. (Brown, learned and scientific)
  \ex\label{ex:59b} In his later years he revised his poems many times, \textbf{though} \textit{the revisions did not alter the essential nature of the style which he had established before he was thirty} […]. (Re-constructed)
  \z
\z

Example (\ref{ex:60}a) hinges upon the juxtaposition of two states of affairs at different points in time: A situation (or state of mind) is altered, perhaps by changing circumstances. In this case, a political or ideological position initially held is modified by social events. Even if one does not fully understand what the sentence is about (namely the food riots in Milan, Italy, on 6–10 May 1898), it is immediately clear that the two propositions do not hold together via an anticausal or epistemic relation, but simply contrast an earlier stage with a later one. The dialogic element consists in the demonstration of changeability: By showing that it changed at a later time, the proposition in the subordinate clause of the original example is made less absolute. A convenient label for this particular type of dialogic concessive could be \textsc{sequential qualification} or, more simply, \textsc{mutability}. Once again, it matters little for the functioning of the concessive which of the two propositions is encoded in the subordinate clause, as shown by the re-constructed variant in \REF{ex:60b}. On the other hand, changing the ordering of clauses~– irrespective of the attachment of the conjunction~– could make the decoding of the message more difficult, since the actual temporal sequence of events would no longer correspond to the ordering of propositions.\footnote{See comments on \isi{iconicity} in \sectref{sec:2.3.1}.}

\ea\label{ex:60}
    \ea\label{bkm:Ref487634220}\textbf{Though} \textit{initially he felt his role was to resist the rising tide of mediocrity unleashed by modern mass society} […], the wide-spread food riots of 1898 left a deep impression on him. (FLOB, learned and scientific)\\
    \ex\label{ex:60b}Initially he felt his role was to resist the rising tide of mediocrity unleashed by modern mass society […], \textbf{though} \textit{the wide-spread food riots of 1898 left a deep impression on him}. (Re-constructed)\\
\z
\z


Examples like the following one are quite frequently found in the learned (scientific) texts of xBrown. Their somewhat more abstract structure can be paraphrased as \textsc{effect, but not statistically significant}. The presence of an effect (here: ‘higher resource use in the control group’) does not necessarily say anything about \textit{p}-values, so the two propositions are not linked at the anticausal or epistemic levels. What the construction does is present an interesting effect, which is then toned down by adding that it is not significant in statistical terms.

\ea\label{ex:61}
    \ea\label{ex:61a}[R]esource use among intervention patients tended to be lower than that among the control group, \textbf{although} \textit{none of these differences was statistically significant}. (BE06, learned and scientific)\\
    \ex\label{ex:61b}\textbf{Although} \textit{resource use among intervention patients tended to be lower than that among the control group}, none of these differences was statistically significant. (Re-constructed)\\
  \z
\z


As should be clear from the examples cited in this section, there is a vast number of general principles or patterns that may create coherence between the two propositions in a \is{concessives (types of)!dialogic}dialogic CC (e.g. \textsc{unity in diversity} or \textsc{sequential qualification}; see above). The identification, discussion and cataloguing both of dialogic subtypes of meaning (as discussed in this section) and \is{concessives (types of)!anticausal}anticausal \is{topos}topoi (as discussed in \sectref{sec:3.1}) can point to general cognitive mechanisms and ways in which humans structure their \isi{world knowledge}. There is a basic relationship between principles in the \is{concessives (types of)!anticausal}anticausal/\is{concessives (types of)!epistemic}epistemic and the \is{concessives (types of)!dialogic}dialogic domains, but I would still suggest that we need different terms to label them. While the concept of the \is{topos}\textit{topos} is well-established in connection with conditional and causal relations (as operative in \is{concessives (types of)!anticausal}anticausal and \is{concessives (types of)!epistemic}epistemic concessives), I propose to refer to typical, generalised configurations of \is{concessives (types of)!dialogic}dialogic propositions as \is{themes (in dialogic CCs)}\textit{themes}.

\section{\label{bkm:Ref487180782}Semantic ambiguity}\label{sec:3.4}

When coding the data for semantic types, categorical decisions were made: Unless they were truly opaque and had to be excluded, examples were classified as one of the three functional types, anticausal, epistemic or dialogic. There were of course a number of functionally \is{ambiguity}ambiguous cases, which tended to lean towards one of the semantico-pragmatic categories but could also have been plausibly interpreted as a different type (cf. \citealt{Mondorf2004}: 121–122). Some such examples are reproduced in this section. In many cases, the \isi{ambiguity} is between two functional categories (e.g. anticausal or episte\-mic), but there are also instances that display three-way \isi{ambiguity}, i.e. a potential wavering between all three functional types. CCs of this kind are functional shape-shifters that pose certain problems for the quantitative analysis: The forced classification as one of the three types results in a loss of information, since certain concessives may be characterised by precisely this intermediate position between different functional types and the consequent openness to different interpretations. On the other hand, the inclusion of different degrees of \isi{ambiguity} (and thus more categories) in the analysis would give rise to considerable complications for the quantitative analysis and the interpretation of results.

  Example \REF{ex:62} can be read in three different ways, depending on whether we regard the proposition that is negated in the subordinate clause (namely complete agreement among members) as
(i)~a prerequisite of the matrix-clause proposition (“official position of the Society of Friends”), as
(ii)~evidence pointing to the matrix-clause proposition as an underlying cause or motivation, or as
(iii)~a \is{modification (or qualification)}modification or \is{modification (or qualification)}qualification of the matrix-clause proposition. In the analysis, the \is{concessives (types of)!dialogic}dialogic reading was given precedence in cases like this.

\ea\label{ex:62}   \label{bkm:Ref488064830}\textbf{Although} \textit{not shared by all of its individual members}, this has been the official position of the Society of Friends from its inception in the seventeenth century down to the present time. (BBrown, belles lettres)
\z

The three-way \isi{ambiguity} is perhaps best understood if the respective thinking is paraphrased in a slightly more abstract way. An anticausal reading would result from the assumption that an official position must be shared by all members of the group, and that it is official \textit{because} it is generally shared. An epistemic reading is essentially an inversion of the first scenario and relies on the reasoning that if there is a lack in agreement, this may be (partly) due to the fact that there has not been any official position or policy concerning this point~– this is a possible, but perhaps less plausible interpretation. Finally, if in a dialogic reading general agreement is regarded as less compulsory, a possible paraphrase would be that ‘this has been the official (and therefore quite widely shared) position, but it is not shared by all’. The problem in \is{ambiguity}ambiguous cases like this may be that, while there seems to be some relation of cause/condition and effect, it is not easy to assign those functions to the respective propositions. I would argue that the underlying \isi{topos} is not clearly enough defined, possibly variable, and affected by subjective experience to a greater extent than in other cases; thus, in this case, classification as dialogic is the most conservative path for the analyst.

An example most likely classified as anticausal but also interpretable as epistemic is shown in \REF{ex:63}. Again, the direction of the causal (or conditional) trajectory is not quite clear: Someone may become untrue to themself and their readers by mixing with the wrong people (e.g. royalty and celebrities); conversely, becoming untrue to oneself and one’s readers may be viewed as a change in attitude prior to (and ultimately resulting in) mixing with the wrong people. Both an anticausal and an epistemic reading seem possible, and the difference essentially depends on whether one’s personal belief is that a change in mental state will result in a change of behaviour, or vice versa.

\ea\label{ex:63}\label{bkm:Ref488064831}\textbf{Though} \textit{she mixed with royalty and celebrities}, she always remained utterly true to herself and to her readers. (FLOB, press editorials)
\z

In the following example, categorised as anticausal, one might expect someone majoring in English to have a good command of the language to start with~– a relatively high proficiency in English would therefore be a prerequisite for taking a major in the subject, and the construction as a whole would be read as epistemic. On the other hand, one might think that taking a major in English will have the effect of improving a student’s command of the language. In this case, there would be a conditional or causal relation between the two propositions in the example, and the construction as a whole would be interpreted as anticausal.

\ea\label{ex:64}[…] I don’t speak good English either, \textbf{even} \textbf{though} \textit{I’m taking a major in English}. (ICE-HK:S1A-077; comma added)
\z

There are also cases that are difficult to classify altogether. The construction shown in (\ref{ex:65}a)~– again, most likely classified as dialogic~– could be argued to be purely \is{adverbials!contrast}adversative in meaning: There is no obvious causal or conditional connection between the two propositions and they have only a weak qualifying effect on each other. The two propositions (describing the legibility of frequent and infrequent words in an experiment) are merely in a relationship of contrast. What we could say, however, is that presenting both parts of the construction makes the message complete, as it would perhaps not be satisfactory to be told about frequent words only. In this sense, there is a weakly dialogic element in the construction. As shown in the re-constructed variant (\ref{ex:65}b), one could quite easily substitute \textit{but} or \textit{while} for \textit{although}, which would arguably make the sentence somewhat easier to interpret.

\ea\label{ex:65}
    \ea\label{ex:65a}   \label{bkm:Ref488064836}In word legibility tasks, frequent words were found to be as legible as single letters, \textbf{although} \textit{infrequent words are less legible than either}. (Frown, learned)\\
    \ex\label{ex:65b}In word legibility tasks, frequent words were found to be as legible as single letters, \textbf{but}/\textbf{while} \textit{infrequent words were less legible than either}. (Re-constructed and modified slightly)\is{if@\textit{but}}\is{if@\textit{while}}\\
\z
\z

Example \REF{ex:66} can be interpreted as anticausal or dialogic; the latter would once again be considered the most conservative option. In the anticausal reading one could argue that expectations will not be formed in the first place if one is aware that they are based on simplistic views. In the dialogic reading, the proposition in the subordinate clause qualifies what we know about the subject, Koesler: The matrix-clause proposition makes him look somewhat naïve, while the subordinate clause adds a more positive nuance to this kind of personal evaluation.

\ea\label{ex:66}\label{bkm:Ref488064838}Somehow, \textbf{though} \textit{he knew it was far too facile}, Koesler expected all Italians~– as well as Poles, Irish, and Hispanics~– to be Catholic. (Frown, mystery and detective fiction)
\z

This section has exemplified constructions whose internal semantic structure allows for alternative readings. Making a categorical decision, i.e. opting for what is felt to be the most plausible reading in a given context, inevitably results in some loss of information~– after all, it is possible that the frequency of \is{ambiguity}ambiguous constructions is meaningful in itself. However, the complexity of the quantitative component of this study would have increased considerably had such \is{ambiguity}ambiguous cases been included as a separate category, or even separate categories. A question that might need to be addressed independently is whether or not \is{ambiguity}ambiguous constructions can be shown to have a special function in discourse. In other words: Is the juxtapo\-sition of propositions that allow for multiple (and potentially competing) interpretations accidental or intentional and motivated from the context? Questions of this kind, however, are very complex and cannot be answered in the present study. They may well elude quantitative approaches and are perhaps better addressed in qualitative (e.g. \is{discourse analysis}discourse-analytical) studies.

\section{\label{bkm:Ref496533952}Further notes on syntax}\label{sec:3.5}

As anticipated at the end of \sectref{sec:2.3.2}, there are three syntactic phenomena that deserve a brief discussion, even if they are not treated as distinct categories in the quantitative analyses:
(i)~\is{correlative markers}correlative conjuncts,
(ii)~“overlapping” (or “double”) concessives, and
(iii)~\SchuetzlerIndexExpression{even although} as a \is{subordinators!marginal}marginal concessive subordinator.

The first point concerns correlative marking that consists of a subordinator proper and an optional \is{correlative markers}correlative conjunct, each placed in one of the two clauses that make up the CC. This phenomenon is shown in the constructed example (\ref{ex:67}a), in which the concessive relation is doubly marked by \textit{although} and \SchuetzlerIndexExpression{nevertheless}. Given certain syntactic modifications (i.e. the creation of two main clauses), it is possible to dispense with the subordinating conjunctions, as shown in the variant example (\ref{ex:67}b).

\ea\label{ex:67}
    \ea\label{ex:67a}\textbf{Although} \textit{he was only seventeen years old}, he was \textbf{nevertheless} one of the best chess players of the age.\is{if@\textit{nevertheless}}\\
    \ex\label{ex:67b}\textit{He was only seventeen years old}. \textbf{Nevertheless}, he was one of the best chess players of the age.\is{if@\textit{nevertheless}}\\
\z
\z


As \citet[1001]{QuirkEtAl1985} argue, the additional \is{conjuncts}conjunct in the matrix clause has an \is{emphasis}emphatic function, making the adverbial relation stronger or clearer. The use of a \is{correlative markers}correlative conjunct in the matrix clause may also be motivated by a heavy (i.e. long or complex) preceding subordinate clause, providing a particularly strong \is{cohesion}cohesive tie between sentence parts and supporting intra-sentential coherence. The following example shows the subordinator \textit{although} in combination with \SchuetzlerIndexExpression{yet} as a \is{correlative markers}correlative conjunct. It seems very likely that the selection of the correlative marker is motivated by the \isi{weight} of the subordinate clause in medial position.

\ea\label{ex:68}This luxurious cabin, \textbf{although} \textit{entirely novel to her in conception, design, and furnishing}, \textbf{yet} had about it something familiar and personal. (BLOB, adventure and western)
\z

Particularly in certain \is{varieties of English!L2}L2 varieties, \SchuetzlerIndexExpression{but} is sometimes encountered in addition to a subordinator, as in the following two examples from \il{Indian English}IndE and \il{Hong Kong English}HKE, respectively.

\ea\label{ex:69}\label{bkm:Ref488839457}\textbf{Though} \textit{he was found criminal in the eyes of the law} \textbf{but} he couldn’t convince himself that he is a criminal. (ICE-IND:S1B-017)\\
    \ex\label{ex:70}\label{bkm:Ref488839458}[\textbf{A}]\textbf{lthough} \textit{it seems that I use a lot of time on studying} \textbf{but} the result is not […] as satisfactory as others think. (ICE-HK:S1A-038)\\
\z

L1-oriented language users will most likely try to read \REF{ex:69} and \REF{ex:70} either as coordinate clauses (with an additional subordinator attached to the first clause) or as complex sentences with \SchuetzlerIndexExpression{but} used as a \is{correlative markers}correlative conjunct in the main clause. The \isi{parsing} strategy of an L1-oriented AD/R is given a jolt when the word \SchuetzlerIndexExpression{but} is encountered. For speakers of the respective \is{varieties of English!L2}L2 varieties of English, this may of course be quite different.\footnote{Another phenomenon that highlights the problematic and variable status of seemingly straightforward connectives is sentence-final \textit{but} (cf. \citealt{MulderThompson2008,MulderEtAl2009,Hancil2014,IzutsuIzutsu2014}), which is also listed as feature no. 211 in eWAVE (\citealt{KortmannEtAl2020}).}

The next example illustrates what could be described as two overlapping concessive relations. Two subordinate clauses relate to the same matrix clause, one preceding and the other following it. The example is from published written material, so this particular \is{concessives (types of)!double}double concessive construction must have been consciously planned.

\ea\label{ex:71}   \textbf{Although} \textit{the wing structure was only partly supported}, it is believed that the wing as a whole was capable of a flapping motion, \textbf{although} \textit{soaring and gliding was probably the main mode of flight}. (ICE-NZ:W2B-023)\\
\z

The first one of the two overlapping CCs is best treated as an anticausal concessive, since weak structural support of a wing would normally not result in the kind of belief stated in the matrix clause. The second part suggests a dialogic reading: “a flapping motion” is indicated as a possibility, but the opinion is expressed that this was not what the wing was mainly used for. No causal or conditional trajectory~– and thus no topos~– operates between the two propositions nearer the end of the sentence. The construction as a whole quite efficiently first sets the scene for the matrix-clause proposition, which is then qualified by another subordinate clause. Examples like this show that concessive construc\-tions may go some way beyond the simple juxtaposition of two linked propositions.\footnote{See also \REF{ex:44} on p. \pageref{bkm:Ref497554696} for an interesting, complex case.} Shared world knowledge and topoi hold the CC together and make it interpretable, even if there is great flexibility regarding its syntactic formation.

An interesting if very rare complex marker of concession is the conjunction \SchuetzlerIndexExpression{even although}. In a footnote, \citet[41]{Aarts1988} discusses a single occurrence he found not in his data but in a letter written “by a Scottish friend”; accordingly, he is not sure whether this is a feature of Scottish English or simply an idiosyncrasy. A single instance of \textit{even although} was also found in ICE-GB, reproduced as \REF{ex:72}, and it was possible to establish the identity of the speaker as Martin O’Neill, a Scottish Member of Parliament. I also came across \textit{even although} in a novel by Scottish writer Peter May, reproduced as \REF{ex:73}.\footnote{There is at least one more occurrence of \textit{even although} in the same novel \citep[369]{May2012}; this marker has also been independently spotted by another reader in Peter May’s \textit{The Chess Men} from the year 2013 (cf. \url{http://languagehat.com/even-although/}; last accessed 3 {October 2023}; however, this blog also suggests that the complex conjunction is generally more widespread).}

\ea\label{ex:72}\label{bkm:Ref489448030}\textbf{Even} \textbf{although} \textit{for a moment or two he perhaps enthused the street crowds with the idea that he stood up to the infidels and stood up to the West and to the Americans}, the fact is that he personally has probably killed more Muslims than any other person this century […]. (ICE-GB:S1B-036; comma added)\is{if@\textit{even although}}\\
\ex\label{ex:73}\label{bkm:Ref489448047}I knelt at his head, and kissed him, and prayed for his soul, \textbf{even} \textbf{although} \textit{I was no longer certain that there was a God out there}. (\citealt{May2012}: 383–384)\is{if@\textit{even although}}\\
\z

Since Aarts speculates about a possible Scottish association of this complex connective, and since the only instances that I have come across are from Scottish works of fiction, there may be reason to suspect that this particular form is indeed a Scotticism worth targeting in future research on \ili{Scottish Standard English} and \ili{Scots} (cf. \citealt{SchützlerEtAl2017}).

\section{Summary}\label{sec:3.6}

This chapter presented a selection of corpus examples of concessive constructions. Apart from offering a resource of authentic usage events for future work, it contributes to a better understanding of the concrete semantic mechanisms at work in CCs, which would remain entirely abstract if only the quantitative aspects of the present study were considered.  Concerning relations between propositions that are juxtaposed in CCs, the chapter also pointed to certain recurrent semantic patterns, which I conventionally call \is{topos}\textit{topoi} if there is an identifiable causal or conditional link between propositions (in \is{concessives (types of)!anticausal}anticausal or \is{concessives (types of)!epistemic}epistemic CCs) and \is{themes (in dialogic CCs)}\textit{themes} if propositions find themselves in a less narrowly defined, qualifying or corrective relation (in dialogic CCs). While the former term is well-established in the literature, the latter was newly proposed in this chapter. Although it would certainly be worthwhile to work towards a more comprehensive inventory (or typology) of such inter-propositional relations (\is{topos}\textit{topoi} and \is{themes (in dialogic CCs)}\textit{themes}), this is clearly beyond the scope of the present study. Furthermore, complications involved in the semantic classification of cases were highlighted; some of these may be of value as starting points in the development of future (more fine-grained) classification schemes. Finally, non-prototypical syntactic realisations of CCs were discussed. While most of these seem to be of very low frequency, they can play a role particularly in \is{varieties of English!L2}L2 contexts, and they may inform more exclusively syntax-oriented approaches. In sum, \chapref{sec:3} makes explicit what might tend to be lost in the quantitative analyses: CCs are intriguingly complex and in some cases not at all straightforward to categorise, semantically and syntactically. At the same time, language users routinely and effortlessly interpret them, presumably because they rest solidly on shared \isi{world knowledge} and pragmatic conventions.
