\chapter{Clause structure}\label{bkm:Ref1727507}\label{ch:11}\label{sec:11}\largerpage

Analysing the internal structure of subordinate clauses in CCs is the final step when progressing through the stages of the \is{constructional choice model}choice model formulated in \sectref{sec:4.1.3}. The realisation of a subordinate clause as \is{finite clauses}finite or \is{nonfinite clauses}nonfinite depends on all other factors~– semantic structure, clause position and the connective itself. In analogy to Chapters~\ref{sec:7}–\ref{sec:10}, this chapter first presents the statistical model that was used (\sectref{sec:11.1}), shows the results of the analysis (\sectref{sec:11.2}) and discusses them against the expectations that were formulated (\sectref{sec:11.3}).

\section{\label{bkm:Ref75782456}Statistical model}\label{sec:11.1}

The model employed for the analysis of clause-internal syntax (“Model~E”) is a binary \is{regression!logistic}logistic regression model since the outcome variable \textsc{nonfin} takes only two values, “\is{nonfinite clauses}nonfinite”  and “\is{finite clauses}finite”. The latter is the reference category and also happens to be the unmarked, much more frequent variant overall (see \tabref{tab:6.3} in \sectref{sec:6.3.6} for an overview of variables). The model, shown in (\ref{eq:11.1}), is essentially constructed in the same way as Models C \& D (see \sectref{sec:9.1} and \sectref{sec:10.1}): \is{mode of production}Mode of production (represented by the variable \textsc{spoken.ct}) interacts with each of the other fixed-part predictors, but these do not interact with each other. Additionally, the full set of fixed-part predictors, with the exception of \textsc{spoken.ct}, are assumed to vary randomly across the \isi{cluster variables} \textsc{genre} and \textsc{text}. As in the previous analyses, separate models with identical specifications were run for each of the nine varieties.

\ea
\label{bkm:Ref75782842}\label{eq:11.1}  
Model E: Syntax
\begin{lstlisting}
nonfin ~ spoken.ct * (anti.ct + final.ct + marker)
         + (anti.ct + final.ct + marker | genre)
         + (anti.ct + final.ct + marker | text)
\end{lstlisting}
\z


Although it contains the largest number of predictors, Model E is in fact intermediate in complexity between the less complex Model C and the more complex Model D, since the latter is a \is{regression!multinomial}multinomial model that generates a considerably larger number of parameters. More information concerning token numbers, the number of levels of both random factors (\textsc{genre} and \textsc{text}), the \isi{priors} (held constant across all nine models) and the number of posterior samples can be found in Appendix~\ref{appendix:B.5}. Data, scripts and model summaries (i.e. tables with regression coefficients) can be retrieved from the online repositories (cf. \sectref{sec:1.4}).

\section{\label{bkm:Ref75782487}Results}\label{sec:11.2}


This part of the chapter takes five perspectives on the results to highlight different factors and their impact on the outcome. Firstly, the effects of semantics, clause position and the concessive conjunction are controlled for in \sectref{sec:11.2.1}. This results in average values for individual varieties, and the only variety-internal differentiation happens along the spoken-written dimension. Next, \sectref{sec:11.2.2} isolates the effect of intra-constructional semantics, again considering the moderating effect of \isi{mode of production}. Thirdly, in \sectref{sec:11.2.3} the focus will lie on the relationship between clause position and the \is{finite clauses}finite/\is{nonfinite clauses}nonfinite status of subordinate clauses. In the fourth section (\sectref{sec:11.2.4}), likely combinations of conjunctions and \is{nonfinite clauses}nonfinite clauses are explored. Finally, \sectref{sec:11.2.5} shows a complete ranking of specific conditions, in analogy to the approach taken in \sectref{sec:10.2.4}.

\subsection{\label{bkm:Ref75783798}Average percentages}\label{sec:11.2.1}

As indicated above, the first step in analysing relative frequencies of \is{finite clauses}finite and \is{nonfinite clauses}nonfinite subordinate clauses was to establish average values for the nine varieties. Again, the assumption of neutral values for \isi{mode of production}, semantics and clause position is a hypothetical one, but it helps to arrive at a general impression. \figref{fig:11.1} shows percentages of \is{nonfinite clauses}nonfinite clauses in CCs for the nine varieties under investigation, arranged in ascending order. Once again, \is{varieties of English!L1}L1 varieties are shown in black and \is{varieties of English!L2}L2 varieties in grey.\largerpage

\begin{figure}[H]
\includegraphics{figures/CCs.Fig.11.1.pdf}
\caption{\label{bkm:Ref75785270}\label{fig:11.1}Average percentages of \is{nonfinite clauses}nonfinite subordinate clauses}
\end{figure}

A relatively small number of subordinate clauses are realised as \is{nonfinite clauses}nonfinite, the range of variety-specific median values extending from 3.8\% [1.5; 7.5] in \il{Nigerian English}NigE to 9.0\% [4.5; 14.5] in \il{Irish English}IrE. There is only a small difference between \is{varieties of English!L1}L1 and \is{varieties of English!L2}L2 varieties: The mean percentage of \is{nonfinite clauses}nonfinite clauses in the former is 7.0\% (as indicated by the dotted black line), while in the latter it is 6.4\% (as indicated by the grey line). Given the great variability between varieties in combination with the relatively high degree of intra-varietal uncertainty, this is not a substantial difference, and it will accordingly not be discussed any further.

\figref{fig:11.2} again orders varieties according to their average percentages of \is{nonfinite clauses}nonfinite subordinate clauses. This time, however, results in the lower panel are subdivided into values for the spoken and written mode (in grey and black, respectively). Additionally, the estimated difference between \is{mode of production}modes of production (in absolute percentage points) is shown in the upper panel. Mean differences between speech and writing are indicated separately for \is{varieties of English!L1}L1 and \is{varieties of English!L2}L2 varieties, using dotted lines with direct labels.

\begin{figure}
\includegraphics{figures/CCs.Fig.11.2.pdf}
\caption{\label{bkm:Ref75786254}\label{fig:11.2}Average percentages of \is{nonfinite clauses}nonfinite subordinate clauses by mode of production}
\end{figure}

The share of \is{nonfinite clauses}nonfinite subordinate clauses tends to be higher in \is{written language}writing. This general pattern is found in eight out of nine varieties, with \il{Singapore English}SingE as the only exception. Five varieties have a somewhat more substantial positive percentage-point difference (\il{Australian English}AusE: +3.7; \il{British English}BrE: +3.7; \il{Canadian English}CanE: +4.8; \il{Jamaican English}JamE: +7.2; \il{Indian English}IndE: +7.2), but only \il{Indian English}IndE has a value that seems robustly different from zero, with a 90\% uncertainty interval of [2.7; 13.1]. If the mean difference between \is{written language}writing and \is{spoken language}speech is inspected separately for \is{varieties of English!L1}L1 and \is{varieties of English!L2}L2 varieties, it turns out to be only minimally larger among the former (+3.4) compared to the latter (+2.9). \tabref{tab:11.1} explores potential differences between \is{varieties of English!L1}L1 and \is{varieties of English!L2}L2 varieties in some more detail by providing separate mean values for the spoken and written mode. In this table, differences between \is{written language}writing and \is{spoken language}speech in \is{varieties of English!L1}L1 and \is{varieties of English!L2}L2 varieties are not exactly the same as in the upper panel of \figref{fig:11.2}: A slight discrepancy arises between the per-group means of estimated, variety-specific differences on the one hand (as in \figref{fig:11.2}) and the difference between group-specific mean estimates for percentages on the other (as in \tabref{tab:11.1}; cf. comment in Footnote~\ref{fn87} on p. \pageref{fn87}), and rounding errors may also differ. To avoid confusion, further comparisons of this kind will therefore only be based on values shown in tables.

\begin{table}
\caption{\label{bkm:Ref75787839}\label{tab:11.1}Nonfinite realisations of subordinate clauses by \is{mode of production}mode and variety type (mean \%)}
\begin{tabular}{l *3{S[table-format=-1.1]}}
\lsptoprule
 & {spoken} & {written} & {written\,$-$\,spoken}\\\midrule
L2      & 4.6  & 7.7 & 3.1\\
L1      & 4.9  & 8.6 & 3.7\\
L2\,$-$\,L1   & -0.3 & -0.9 & \\
\lspbottomrule
\end{tabular}
\end{table}

The small general difference between \is{varieties of English!L2}L2 and \is{varieties of English!L1}L1 varieties, with the former characterised by slightly fewer \is{nonfinite clauses}nonfinite subordinate clauses, is remarkably similar in both \is{mode of production}modes of production. From the bird’s-eye perspective shown in \figref{fig:11.1} above (i.e. controlling for mode of production), this difference was 0.5 absolute percentage points. In \is{spoken language}speech and \is{written language}writing, it is 0.3 and 0.9 percentage points, respectively. There is neither a substantial difference between the two broad groups of varieties concerning finiteness/nonfiniteness, nor do the two groups differ markedly in their response to a change in \isi{mode of production}.

\subsection{\label{bkm:Ref75784106}Semantics}\label{sec:11.2.2}

The lower part of \figref{fig:11.3} isolates the effect of the intra-constructional semantics of CCs on the finiteness of subordinate clauses, with \is{concessives (types of)!dialogic}dialogic and \is{concessives (types of)!anticausal}anticausal types shown in grey and black, respectively. In the upper panel, the differences between the two conditions are shown, this time without an indication of mean values for \is{varieties of English!L1}L1 and \is{varieties of English!L2}L2 varieties. The plot applies the same horizontal ranking as the previous two figures, based on the average percentages of \is{nonfinite clauses}nonfinite clauses in the individual varieties.

\begin{figure}
\includegraphics{figures/CCs.Fig.11.3.pdf}
\caption{\label{bkm:Ref75862620}\label{fig:11.3}Average percentages of \is{nonfinite clauses}nonfinite subordinate clauses by variety and semantic type}
\end{figure}

The relationship between intra-constructional semantics and the finiteness status of the subordinate clause seems highly unsystematic~– in fact, the general pattern of differences in the upper panel of \figref{fig:11.3} is remarkably similar to the respective panel in \figref{fig:11.2} above. Four varieties (\il{Canadian English}CanE, \il{Indian English}IndE, \il{Singapore English}SingE and \il{Irish English}IrE) hardly make a difference between the two conditions, three varieties (\il{Nigerian English}NigE, \il{British English}BrE and \il{Hong Kong English}HKE) tend to use fewer \is{nonfinite clauses}nonfinite clauses in \is{concessives (types of)!dialogic}dialogic CCs, and the remaining two varieties (\il{Australian English}AusE and \il{Jamaican English}JamE) have a higher percentage of \is{nonfinite clauses}nonfinite clauses in \is{concessives (types of)!dialogic}dialogic CCs. In all cases, the difference comes with high degrees of uncertainty. \tabref{tab:11.2} provides a few basic summary statistics that underscore the absence of a pattern along this dimension of variation: Only individual varieties stand out from a rather nondescript general distribution, and there is little that could be said about differences between \is{varieties of English!L1}L1 and \is{varieties of English!L2}L2 varieties.\largerpage[-1]

\begin{table}
\caption{\label{bkm:Ref75863153}\label{tab:11.2}Nonfinite realisations of subordinate clauses by semantics and variety type (mean \%)}
\begin{tabular}{l *3{S[table-format=-1.1]}}
\lsptoprule
 & {anticausal} & {dialogic} & {anticausal\,$-$\,dialogic}\\\midrule
L2 & 6.0 & 6.5 & -0.5\\
L1 & 6.8 & 6.9 & -0.1\\
L2\,$-$\,L1 & -0.8 & -0.4 & \\
\lspbottomrule
\end{tabular}
\end{table}

\begin{sloppypar}
Unfolding the global patterns outlined above into the spoken and written mode, as shown in \figref{fig:11.4}, adds relatively little to our understanding of the relationship between semantics and (non)finiteness. As in the analogous plots in previous chapters, there is one component plot per variety. Estimated percentages of \is{nonfinite clauses}nonfinite clauses by mode and semantics are shown in the lower panels, while absolute percentage-point differences between \is{concessives (types of)!anticausal}anticausal and \is{concessives (types of)!dialogic}dialogic CCs~– still distinguishing the two modes~– are displayed in the upper panels. In contrast to the plots above, varieties are no longer ordered on quantitative grounds but according to the original arrangement introduced in \sectref{sec:4.3} and \sectref{sec:6.1} (cf. \tabref{tab:4.1} and \figref{fig:6.1}).
\end{sloppypar}

\begin{figure}
\includegraphics{figures/CCs.Fig.11.4.pdf}
\caption{\label{bkm:Ref75863438}\label{fig:11.4}Average percentages of \is{nonfinite clauses}nonfinite subordinate clauses by variety, mode and semantic type; a~= anticausal, d~= dialogic}
\end{figure}

There are only few patterns that suggest a moderating effect of the \isi{mode of production} on the selection of a \is{nonfinite clauses}nonfinite or \is{finite clauses}finite subordinate clause. In most cases, the effect of semantics on finiteness is either extremely (\il{Canadian English}CanE, \il{Nigerian English}NigE and \il{Indian English}IndE) or very (\il{British English}BrE, \il{Australian English}AusE and \il{Hong Kong English}HKE) similar between modes; in the remaining three varieties (\il{Irish English}IrE, \il{Jamaican English}JamE and \il{Singapore English}SingE), there is a more substantial difference between modes, but it lacks coherence in that the moderating effect points in different directions.

These findings do not suggest that there is a systematic, readily interpretable connection between the internal structure (finiteness/nonfiniteness) of a subordinate clause and the semantic relation that holds between clauses within a CC. It is quite probable that the few sporadic, variety-specific patterns that we \textit{can} see are due to the fact that each variety was addressed with its own separate model. Had the approach been to fit a single model, with \textsc{variety} as a grouping variable and the semantic predictor \textsc{anti.ct} varying randomly across it, the \isi{pooling} effect would quite possibly have reduced the effect even further (cf. \sectref{sec:6.3.1}), particularly seeing the low and evenly distributed absolute numbers of \is{nonfinite clauses}nonfinite cases as documented in Appendix~\ref{appendix:A.3}.

\subsection{\label{bkm:Ref75784167}Clause position}\label{sec:11.2.3}\largerpage

In analogy to the approach in the previous section, the lower panel of \figref{fig:11.5} directly contrasts in a simplified form the effect that clause positions have on the realisation of a subordinate clause as \is{finite clauses}finite or \is{nonfinite clauses}nonfinite. Values representing subclauses in nonfinal position are shown in grey, while values for clauses in final position appear in black.\footnote{It is awkward that the terms \textit{final}/\textit{nonfinal} and \textit{finite}/\textit{nonfinite} are so similar, phonologically. To increase processability, only \textit{finite} and \textit{nonfinite} will be used as direct attributes of \textit{clause}; that is, I will speak of \textit{\is{finite clauses}finite}/\textit{\is{nonfinite clauses}nonfinite clauses} and \textit{\is{final position}final}/\textit{\is{nonfinal position}nonfinal positions}, but not of “\is{final position}final/\is{nonfinal position}nonfinal clauses”.} The upper panel shows the differences that result when subtracting percentages (of \is{nonfinite clauses}nonfinite constructions) in \is{nonfinal position}nonfinal positions from those in \is{final position}final positions. The same arrangement of varieties as in the similar plots in \sectref{sec:11.2.1} and \sectref{sec:11.2.2} is retained.

\begin{figure}
\includegraphics{figures/CCs.Fig.11.5.pdf}
\caption{\label{bkm:Ref75876706}\label{fig:11.5}Average percentages of \is{nonfinite clauses}nonfinite subordinate clauses by variety and clause position}
\end{figure}

Compared to the rather noisy semantic pattern in the previous section, there is a much more systematic relationship between clause position and the internal structure of subordinate clauses: The proportion of \is{nonfinite clauses}nonfinite realisations is always lower if the subordinate clause is in \isi{final position}. Two of these differences are very close to zero (\il{Nigerian English}NigE: $-$1.0 [$-$5.4; 3.4]; \il{Hong Kong English}HKE: $-$0.7 [$-$5.6; 5.6]), but the remaining seven are not, with \il{British English}BrE ($-$4.6 [$-$9.7; $-$0.1]) and \il{Irish English}IrE ($-$11.7 [$-$20.4; $-$4.6]) forming the extreme points of the group. A few basic summary statistics are produced in \tabref{tab:11.3}, comparing \is{varieties of English!L1}L1 and \is{varieties of English!L2}L2 varieties. Given the discussion above, the basic pattern~– with higher percentages in \is{nonfinal position}nonfinal positions~– must of course obtain in both subsets, but the contrast between conditions is somewhat more striking in \is{varieties of English!L1}L1 varieties.\largerpage[2]

\vfill
\begin{table}[H]
\caption{\label{bkm:Ref75876810}\label{tab:11.3}Nonfinite realisations of subordinate clauses by clause position and variety type (mean \%)}
\begin{tabular}{l *3{S[table-format=-1.1]}}
\lsptoprule
      & {\is{final position}final} & {\is{nonfinal position}nonfinal} & {final\,$-$\,nonfinal}\\\midrule
L2    & 3.6 & 8.8 & -5.2\\
L1    & 2.9 & 10.8 & -7.9\\
L2\,$-$\,L1 & 0.7 & -2.0 & \\
\lspbottomrule
\end{tabular}
\end{table}
\vfill\pagebreak

The interaction of mode of production and clause position in conditioning the selection of (non)\is{finite clauses}finite clause realisations is shown in \figref{fig:11.6}. Three varieties~– \il{Irish English}IrE, \il{Australian English}AusE and \il{Singapore English}SingE~– show a relatively level pattern in the upper panel, which means that the difference (in absolute percentage points) between \is{final position}final and nonfinal clause positions is fairly stable across \is{mode of production}modes of production. In this group, the intuitively most plausible pattern can be seen in \il{Australian English}AusE, where both values (for \is{final position}final and \is{nonfinal position}nonfinal positions) are lower in \is{spoken language}speech. In \il{Irish English}IrE, the percentage is relatively stable across modes for clauses in \isi{nonfinal position}, while for clauses in \isi{final position} it is lower in \is{spoken language}speech. The \il{Singapore English}SingE pattern is somewhat puzzling since values for both clause positions are slightly higher in \is{spoken language}speech, which goes directly against the general trend and the expectation that was formulated~– compare \figref{fig:11.2}, in which \il{Singapore English}SingE was the only variety characterised by a higher percentage of \is{nonfinite clauses}nonfinite clauses in \is{spoken language}speech.

\begin{figure}
\includegraphics{figures/CCs.Fig.11.6.pdf}
\caption{\label{bkm:Ref75876892}\label{fig:11.6}Average percentages of \is{nonfinite clauses}nonfinite subordinate clauses by variety, mode and clause position; fn~= final, nf~= nonfinal}
\end{figure}

Apart from the level pattern (with more or less horizontal lines in the upper panels of \figref{fig:11.6}) described above, several varieties show a very clear interaction effect, whereby percentages of \is{nonfinite clauses}nonfinite realisations in combination with \is{nonfinal position}nonfinal clause positions are pulled towards zero in \is{spoken language}speech~– that is, the environment normally favouring \is{nonfinite clauses}nonfinite realisations (that is, \isi{nonfinal position}) does not do so in this mode of production, and the expected percentages become much more similar to those associated with clauses in \isi{final position}. In the plots of difference in the upper panels of \figref{fig:11.6}, the resultant pattern is one with the difference in \is{spoken language}speech close to zero and a relatively steep downward slope when moving to the right-hand part of the plot, representing writing. This can be seen in \il{Canadian English}CanE, \il{Jamaican English}JamE, \il{Nigerian English}NigE and \il{Indian English}IndE.

Finally, in \il{British English}BrE and \il{Hong Kong English}HKE there is a “crossed” pattern: In \is{spoken language}speech, it is predominantly clauses in \isi{final position} that take \is{nonfinite clauses}nonfinite complements, while in \is{written language}writing it is clauses in \is{nonfinal position}nonfinal positions that do. There is thus a positive difference (\%~\is{final position}final\,$-$\.\%~\is{nonfinal position}nonfinal) in \is{spoken language}speech and a negative one in \is{written language}writing. This finding does not conform to the formulated expectations. Given the rather low overall token numbers for \is{nonfinite clauses}nonfinite realisations (see Appendix~\ref{appendix:A.3}), we can only speculate as to the reasons, which probably lie in \is{sampling error}sampling errors or confounding factors to do with the specific \isi{information structure} of the few instances that are involved.

In sum, the analyses in this section suggest that a \is{nonfinite clauses}nonfinite clause realisation is normally more likely if the subordinate clause is in \isi{nonfinal position}. This is not unexpected, as it aligns well with \citegen{QuirkEtAl1985} principle of “\isi{communicative dynamism}”, whereby heavier and more informative syntactic elements tend to occur later in a sentence (cf. \sectref{sec:2.3.1}). Since \is{nonfinite clauses}nonfinite clauses not only lack a \is{finite clauses}finite verb but very often also a subject, they have less material substance and their early placement therefore comes as no surprise. The matrix clause, on the other hand, has both a \is{finite clauses}finite verb and a syntactic subject and on average tends to be placed after a \is{nonfinite clauses}nonfinite subordinate clause.

\subsection{\label{bkm:Ref75784179}Markers}\label{sec:11.2.4}

In this section, the focus is on the relationships between the three concessive conjunctions \textit{although}, \textit{though} and \textit{even though} and the realisation of a subordinate clause as \is{finite clauses}finite or \is{nonfinite clauses}nonfinite. Because in this case three conditions are compared, \figref{fig:11.7} differs in design from the respective first plots in \sectref{sec:11.2.1}–\ref{sec:11.2.3}, and it does not show estimates of differences.\footnote{Pairwise differences between markers could have been estimated, but their interpretation would have been challenging.}

\vfill
\begin{figure}[H]
\includegraphics{figures/CCs.Fig.11.7.pdf}
\caption{\label{bkm:Ref75877064}\label{fig:11.7}Average percentages of \is{nonfinite clauses}nonfinite subordinate clauses by variety and marker; A~= \textit{although}, T~= \textit{though}, E~= \textit{even though}}
\end{figure}
\vfill\pagebreak

Although the slopes of lines connecting the three values in the individual panels differ in steepness, the general similarity of patterns is very striking. In eight varieties, the expected percentages follow a uniform ranking, namely \SchuetzlerIndexExpression{though} > \SchuetzlerIndexExpression{although} > \SchuetzlerIndexExpression{even though}. Averaging across the median estimates of all nine varieties, we get mean values of 12.4\% for \SchuetzlerIndexExpression{though}, 4.1\% for \SchuetzlerIndexExpression{although} and 2.7\% for \SchuetzlerIndexExpression{even though}. \il{Indian English}IndE constitutes the single exception to this pattern, with percentages of 3.6 for \SchuetzlerIndexExpression{even though} and 3.2 for \SchuetzlerIndexExpression{although}. However, the difference between \SchuetzlerIndexExpression{although} and \SchuetzlerIndexExpression{even though} is generally not very large, and the inverted ranking in \il{Indian English}IndE is not very striking. The affinity between \SchuetzlerIndexExpression{though} and \is{nonfinite clauses}nonfinite clauses is in accordance with \citegen{Hilpert2013a} findings (cf. \sectref{sec:5.1.4}). The highest overall value is estimated for \SchuetzlerIndexExpression{though} in \il{Jamaican English}JamE (16.4\% [8.5; 26.8]); the lowest value is found for \SchuetzlerIndexExpression{even though} in \il{Hong Kong English}HKE (1.0\% [0.2; 3.7]). Looking back to the previous sections of this chapter, it appears that the rather low overall percentage of \is{nonfinite clauses}nonfinite subordinate clauses in the global perspective~– with a cross-varietal mean value smaller than 7\% (cf. \figref{fig:11.1})~– results from the fact that in those parts of the analysis the effects of the markers themselves were neutralised.

The inspection of average values according to the two broad groups of \is{varieties of English!L1}L1 and \is{varieties of English!L2}L2 varieties in \tabref{tab:11.4} does not reveal any substantial difference between them concerning the correlation of (non)finiteness and specific markers. The general ranking described above (\SchuetzlerIndexExpression{though} > \SchuetzlerIndexExpression{although} > \SchuetzlerIndexExpression{even though}) holds within both groups, and the absolute percentage-point difference between \is{varieties of English!L1}L1 and \is{varieties of English!L2}L2 varieties for each individual marker does not seem remarkable, either.

\begin{table}
\caption{\label{bkm:Ref75877076}\label{tab:11.4}Nonfinite realisations of subordinate clauses by marker and variety type (mean \%)}
\begin{tabular}{l *3{S[table-format=-1.1]}}
\lsptoprule
 & {\itshape \SchuetzlerIndexExpression{although}} & {\itshape \SchuetzlerIndexExpression{though}} & {\itshape \SchuetzlerIndexExpression{even though}}\\\midrule
\is{varieties of English!L2}L2 & 3.4 & 12.8 & 2.2\\
\is{varieties of English!L1}L1 & 4.9 & 12.0 & 3.3\\
\is{varieties of English!L2}L2\,$-$\,\is{varieties of English!L1}L1 & -1.5 & 0.8 & -1.1\\
\lspbottomrule
\end{tabular}
\end{table}

The analysis next turns to the interaction of mode of production and individual conjunctions in conditioning the selection of (non)\is{finite clauses}finite subordinate clause realisations. The focus in \figref{fig:11.8} is on the absolute percentage-point difference between \is{written language}writing and \is{spoken language}speech, as shown in the respective upper panels of the nine subplots. Values above the dashed reference line indicate that \is{nonfinite clauses}nonfinite subordinate clauses are more frequent in \is{written language}writing (which is the expected pattern), while values below the line signify that they are more common in \is{spoken language}speech.

\begin{figure}
\includegraphics{figures/CCs.Fig.11.8.pdf}
\caption{\label{bkm:Ref75877117}\label{fig:11.8}Average percentages of \is{nonfinite clauses}nonfinite subordinate clauses by variety, mode and marker; A~= \textit{although}, T~= \textit{though}, E~= \textit{even though}, W~= written, S~= spoken}
 \end{figure}

Once again, there are relatively clear tendencies, although in comparison to \figref{fig:11.7} the number of exceptions is somewhat larger. Most differences (\%~written\,$-$\,\%~spoken) are in a positive direction or close to zero, which confirms that \is{written language}written language is characterised by a higher share of \is{nonfinite clauses}nonfinite subordinate clauses in CCs. The only exceptions (i.e. tendencies in the opposite direction) sufficiently different from zero to deserve discussion are \SchuetzlerIndexExpression{even though} in \il{Irish English}IrE ($-$5.8 [$-$18.0; 3.1]), \SchuetzlerIndexExpression{although} in \il{Australian English}AusE ($-$6.0 [$-$17.4; $-$0.3]), and, with some reservations, \SchuetzlerIndexExpression{though} in \il{Singapore English}SingE ($-$5.9 [$-$24.4; 8.1]).   Typically, \SchuetzlerIndexExpression{though} is the marker that responds most strongly to the difference in \isi{mode of production}, as reflected in the positive wedge-shaped patterns in the upper panels of \figref{fig:11.8} for \il{British English}BrE, \il{Canadian English}CanE, \il{Australian English}AusE, \il{Jamaican English}JamE, \il{Indian English}IndE and \il{Hong Kong English}HKE. In this set, \il{Australian English}AusE and \il{Hong Kong English}HKE display the most and the least pronounced patterns of this kind, respectively. Concerning the remaining three varieties, there are conflicting (i.e. hard-to-interpret) patterns in \il{Irish English}IrE and \il{Singapore English}SingE, and a level pattern in \il{Nigerian English}NigE. The strong affinity between \SchuetzlerIndexExpression{though} and \is{nonfinite clauses}nonfinite subordinate clauses suggests that this particular combination of formal characteristics qualifies as a \is{constructions!subconstructions}subconstruction, and this view is further supported by its particular sensitivity to differences in \isi{mode of production}.

The ranking of conjunctions according to their co-occurrence with \is{nonfinite clauses}nonfinite subordinate clauses is also interesting at a more general level. The shortest conjunction (\SchuetzlerIndexExpression{though}) is most likely to introduce \is{nonfinite clauses}nonfinite subordinate clauses, which will on average also be relatively short, due to the absence of a \is{finite clauses}finite verb and a grammatical subject. Conversely, the longest conjunction (\SchuetzlerIndexExpression{even though}) is the one most likely to combine with \is{finite clauses}finite~– and therefore longer~– subordinate clauses, followed by the second longest marker, \SchuetzlerIndexExpression{although}. Thus, in terms of the weight of subordinate clauses, we effectively get a split into
(i)~longer constructions that combine complex/long markers with syntactically unreduced/\is{finite clauses}finite clauses and
(ii)~shorter ones that combine the marker \SchuetzlerIndexExpression{though} with reduced/\is{nonfinite clauses}nonfinite clauses. It could be argued that the special function of \SchuetzlerIndexExpression{though} aids \is{addressee/reader}AD/R in \isi{parsing} the sentence, since the occurrence of this particular marker signals an increased likelihood of a following \is{nonfinite clauses}nonfinite (and therefore cognitively somewhat more complex) clause. However, despite the tendencies shown in this section, it is of course still the case that in combination with \textit{any} of the three conjunctions \is{finite clauses}finite clauses remain in the majority. As will be shown in the next section, this is true even if all factors are set against nonfiniteness.

\subsection{\label{bkm:Ref75783807}Complete factor combinations}\label{sec:11.2.5}

This section provides the final, most detailed perspective on the estimated share of \is{nonfinite clauses}nonfinite subordinate clauses expected to occur under different circumstances. In contrast to the previous sections, no averaging across specific conditions is applied but all possible combinations of factor settings are shown. Their total number is \textit{n}~=~216 (9~varieties × 2~\is{mode of production}modes of production × 2~semantics × 2~clause positions × 3~markers). Due to the large number of conditions, ranked estimates are shown in three consecutive plots: \figref{fig:11.9} shows ranks 1–72, \figref{fig:11.10} shows ranks 73–144, and \figref{fig:11.11} shows ranks 145–216. The percentage scale once again has a horizontal orientation.{\interfootnotelinepenalty=10000\footnote{The numerous low percentage values are very difficult to discriminate in \figref{fig:11.10} and \figref{fig:11.11}. An alternative way of plotting percentages using \is{logits}logit scaling and thus increasing the resolution of low (and high) values is discussed in \citet{Schützler2023} but not applied here.}} To the right of each figure, the scheme of grey-scale symbols known from earlier chapters (cf. \sectref{sec:7.3}, \sectref{sec:9.2.2} and \sectref{sec:10.2.4}) is used to highlight structure in the data. In the first four columns, black squares represent \is{varieties of English!L1}L1 varieties, written language, \is{concessives (types of)!anticausal}anticausal semantics and subordinate clauses in \isi{final position}, respectively; conversely, white squares denote \is{varieties of English!L2}L2 varieties, spoken language, \is{concessives (types of)!dialogic}dialogic semantics and subordinate clauses in \isi{nonfinal position}. The fifth column differentiates between the three subordinators \SchuetzlerIndexExpression{although}, \SchuetzlerIndexExpression{though} and \SchuetzlerIndexExpression{even though}, using black, grey and white boxes, respectively. Additionally, the mean ranks of groups of conditions~– “black” vs “white” (vs “grey”)~– are indicated in each column by the triangular markers known from earlier plots. Those average ranks (like the ranks themselves) are established based on all three figures in combination. Since this way of plotting the data merely reveals the underlying specifics and does not add novel insights to the analysis, the discussion in this section will be kept relatively brief. Note that, in contrast to earlier plots, only 50\% uncertainty intervals are shown.

The first rank is occupied by \is{concessives (types of)!dialogic}dialogic CCs in written \il{British English}BrE whose subordinate clauses are in \isi{nonfinal position} and headed by \SchuetzlerIndexExpression{though}, with an expected percentage of \is{nonfinite clauses}nonfinite clauses of 43.6\%, directly followed by \is{concessives (types of)!anticausal}anticausal CCs in written \il{Jamaican English}JamE with subordinate clauses in nonfinal position introduced by \SchuetzlerIndexExpression{though} (43.4\%). The models predict a number of relatively high values (at the top of \figref{fig:11.9}), but only \textit{n}~=~36 of the estimated \textit{n}~=~216 specific median percentages are actually above 10\% (i.e. exactly one in six conditions). Ranks 165–216 (\textit{n}~=~52, which corresponds to 24\% of all cases) round to a whole-number value of zero on the percentage scale, as indicated by the dashed horizontal line in \figref{fig:11.11}. Turning to the mean ranks calculated for sets of conditions grouped according to basic predictor values, we necessarily obtain patterns that support the findings documented in \sectref{sec:11.2.1}–\ref{sec:11.2.4}, as discussed in the following paragraph. Most indicators of ranks for such groups (that is, the triangular markers added to the columns on the right of the three plots) are found in \figref{fig:11.10}, i.e. among the middle third of ranks; only the mean ranks for conditions involving \SchuetzlerIndexExpression{though} and \SchuetzlerIndexExpression{even though} are found in Figures \ref{fig:11.9} \& \ref{fig:11.11}, due to the strong association of these conjunctions with \is{finite clauses}finite and \is{nonfinite clauses}nonfinite clause realisations, respectively.

\begin{figure}
\includegraphics{figures/CCs.Fig.11.9.pdf}
\caption{\label{bkm:Ref75877136}\label{fig:11.9}Ranked percentages of \is{nonfinite clauses}nonfinite clauses by specific conditions, ranks 1–72; with 50\% uncertainty intervals; W~= written, S~= spoken, a~= anticausal, d~= dialogic, fn~= final, nf~= nonfinal, A~= \textit{although}, T~= \textit{though}, E~= \textit{even though}}
 \end{figure}

\begin{figure}
\includegraphics{figures/CCs.Fig.11.10.pdf}
\caption{\label{bkm:Ref75877141}\label{fig:11.10}Ranked percentages of \is{nonfinite clauses}nonfinite clauses by specific conditions, ranks 73–144; with 50\% uncertainty intervals; W~= written, S~= spoken, a~= anticausal, d~= dialogic, fn~= final, nf~= nonfinal, A~= \textit{although}, T~= \textit{though}, E~= \textit{even though}}
 \end{figure}

\begin{figure}
\includegraphics{figures/CCs.Fig.11.11.pdf}
\caption{\label{bkm:Ref60636703}\label{fig:11.11}Ranked percentages of \is{nonfinite clauses}nonfinite clauses by specific conditions, ranks 145–216; with 50\% uncertainty intervals; W~= written, S~= spoken, a~= anticausal, d~= dialogic, fn~= final, nf~= nonfinal, A~= \textit{although}, T~= \textit{though}, E~= \textit{even though}}
 \end{figure}

The very slight~– not to say, negligible~– general difference between the two broad variety types (\is{varieties of English!L1}L1 vs \is{varieties of English!L2}L2) that was discussed above (cf. \figref{fig:11.1}) corresponds to the only marginally higher mean rank of conditions involving \is{varieties of English!L1}L1 varieties (\textit{M}~=~105.7) compared to conditions involving \is{varieties of English!L2}L2 varieties (\textit{M}~=~110.7). The pattern of black and white squares in the first of the analytic columns in the three figures above does not reveal any obvious structure. Contrasting written and spoken varieties in the second column, there is a clearer pattern: Particularly if we compare \figref{fig:11.9} to \figref{fig:11.11}, we observe a greater density of black squares in the former (among  the top 72 ranks) and a greater density of white squares in the latter (among the bottom 72 ranks). The mean ranks are \textit{M}~= 96.3 for \is{written language}writing and \textit{M}~= 120.7 for \is{spoken language}speech, which agrees with the general patterns observed in \figref{fig:11.2} above. A considerably smaller difference is once again found for groups based on semantics, as shown in the third column in the three figures. The mean rank for \is{concessives (types of)!dialogic}dialogic CCs is 105.8, while for \is{concessives (types of)!anticausal}anticausal CCs it is 111.2~– the difference between the two semantic types was even difficult to perceive in \figref{fig:11.3} above (see also \tabref{tab:11.2}). On the other hand, a very substantial difference was found in the general comparison of subordinate clauses in \is{final position}final and \isi{nonfinal position} (see \figref{fig:11.5}), and the three figures in this section reflect this, with mean ranks of 131.1 for the former and 85.9 for the latter. Finally, the associations of the three concessive conjunctions with the (non)finiteness of subordinate clauses can be traced in the fifth column of Figures \ref{fig:11.9}–\ref{fig:11.11}. In \sectref{sec:11.2.4}, \SchuetzlerIndexExpression{though} emerged as the marker correlating most strongly with \is{nonfinite clauses}nonfinite subordinate clauses, while \SchuetzlerIndexExpression{although} and particularly \SchuetzlerIndexExpression{even though} are much less likely to introduce such clauses (see \figref{fig:11.7} above). Figures \ref{fig:11.9}–\ref{fig:11.11} throw these patterns into relief: The average rank of conditions involving \SchuetzlerIndexExpression{even though} is 146.0, for \SchuetzlerIndexExpression{although} it is 115.9, and for \SchuetzlerIndexExpression{though} it is 63.6, which is the only value to make it into the top 72 ranks.

The presentation of fully specified conditions in this section has naturally confirmed the more general scenarios discussed in the earlier parts of the chapter. However, the inspection of all $n=216$ possible factor combinations provides a more realistic impression of how those values were arrived at, namely by averaging across a large number of low percentages~– very often close to zero~– and a small number of higher values. Under most circumstances, the realisation of a subordinate clause as \is{nonfinite clauses}nonfinite remains the exception: As Figures \ref{fig:11.9}–\ref{fig:11.11} show, written discourse, \is{nonfinal position}nonfinal clause position and the conjunction \SchuetzlerIndexExpression{though} need to coincide to generate a more substantial share of this particular syntactic type.

\section{\label{bkm:Ref118502315}Summary and discussion}\label{sec:11.3}

The main factors that play a role in the selection of \is{nonfinite clauses}nonfinite and \is{finite clauses}finite subordinate clauses in CCs are
(i)~mode of production,
(ii)~clause position and
(iii)~the subordinating conjunction. There is neither a systematic difference between \is{varieties of English!L1}L1 and \is{varieties of English!L2}L2 varieties, nor do the intra-constructional semantics of a CC seem to have an impact on the internal structure of subclauses.

Concerning \isi{mode of production}, results support the hypothesis formulated in \sectref{sec:5.3}: Less \is{explicitness}explicit (elliptical) \is{nonfinite clauses}nonfinite subordinate clauses are more common in \is{written language}writing, arguably because the challenges they pose for the processor are alleviated in this mode. The association of \is{nonfinite clauses}nonfinite structures with written discourse is well-known from the literature. It is a typical feature of a more \is{compression}compressed style, and therefore requires no additional discussion here.

No hypotheses were formulated concerning the relationship between clause position and the (non)finiteness of subordinate clauses. I have argued that \is{nonfinite clauses}nonfinite subordinate clauses preceding the matrix clause should be more problematic from a \isi{processing} perspective: They not only lack a \is{finite clauses}finite verb but usually also a subject (cf. \sectref{sec:2.3.2}), so that their full interpretation must be suspended at least until the matrix clause subject is parsed. On the other hand, according to Quirk et al.’s (\citeyear[1036]{QuirkEtAl1985}) notion of \textit{resolution} (i.e. \isi{end-weight} applied at the sentence level), the heavier clause would be expected at the end of a sentence, and this will typically be the \is{finite clauses}finite matrix clause. There were thus diametrically opposed predictions, whose relative importance can only be established empirically. Results in this study suggest a strong alignment of the \is{nonfinal position}nonfinal placement of a subordinate clause and its realisation as \is{nonfinite clauses}nonfinite. It appears that the weight of component clauses plays a more important role than the challenge presented by a suspended subject.

\begin{sloppypar}
Finally, the finding that \is{nonfinite clauses}nonfinite subordinate clauses are more likely to be attached to the conjunction \SchuetzlerIndexExpression{though} is in agreement with expectations~– expectations, however, that are based exclusively on findings by \citet{Hilpert2013a}, not on theoretical considerations. Like the association of \SchuetzlerIndexExpression{although} with subordinate clauses in \isi{nonfinal position} (see \chapref{sec:9}), the association of \SchuetzlerIndexExpression{though} with \is{nonfinite clauses}nonfinite clauses makes the range of possible specific constructions somewhat tidier~– in concrete terms, it makes the combination of formal characteristics less arbitrary. This higher degree of orderliness in itself can motivate constructional patterns~– even without invoking additional semantic, formal or language-external factors~– and a more principled account of these thoughts will be provided in the final chapter of this volume. However, a few remarks on the combination of \SchuetzlerIndexExpression{though} with \is{nonfinite clauses}nonfinite subordinate clauses are nevertheless in place, if only to provide pointers for future research. It is curious, for instance, that the shortest marker (\SchuetzlerIndexExpression{though}) should be the one that most readily combines with \is{nonfinite clauses}nonfinite (and therefore shorter) clauses, and, conversely, that the longest marker (\SchuetzlerIndexExpression{even though}) should most readily combine with \is{finite clauses}finite (and therefore longer) clauses. Although no more than an informed speculation, there appears to be a tendency for the economy of clauses at the sentence level not to strive towards balanced constructions (short marker +~long clause; long/complex marker +~short clause) but to favour a somewhat more obvious differentiation into subordinate clauses that are either heavier or lighter on both counts. It would of course be interesting (and perhaps necessary) to see the emergence of such tendencies in \is{diachronic approaches}diachrony, and thus to shed light on a specific kind of \isi{constructional change}. It seems quite possible that language users actively exploit different degrees of clause weight to emphasise certain parts of sentences and thus to generate specific information structures. Secondly, \SchuetzlerIndexExpression{though} is historically primary, while \SchuetzlerIndexExpression{although} and \SchuetzlerIndexExpression{even though} are somewhat later additions to this set of conjunctions. Thus, there appears to be an attraction between the potentially most grammaticalised marker and types of subordinate clauses that are cognitively more complex since they contain less explicit information. Although we cannot truly derive such theories from the present research, the final chapter will point towards some possible avenues for future research that may incorporate assumptions of this kind.
\end{sloppypar}
