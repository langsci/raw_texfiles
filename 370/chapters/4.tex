\chapter{\label{bkm:Ref3457828}Dimensions and mechanisms of variation}\label{sec:4}

This chapter discusses three aspects crucial in the context of the present study. Firstly, \isi{Construction Grammar} is introduced as a theoretical framework used to account for the formal and functional variation of concessives (\sectref{sec:4.1}). A hierarchical \is{constructional choice model}choice model of \isi{constructional variation} is proposed, in which higher-order properties of a construction have an impact on lower-order properties. These relationships can be employed to predict formal characteristics of CCs. Along with these intra-constructional factors, two \isi{language-external factors} are introduced in \sectref{sec:4.2} and \sectref{sec:4.3}, respectively: \isi{mode of production} (speech vs writing) and different geographical or national \isi{varieties of English}. Analyses of genre that go beyond the general distinction of speech and writing will not be undertaken in the present study.

\section{\label{bkm:Ref488418230}\label{bkm:Ref465410054}\label{bkm:Ref465410417}Constructions and constructional variation}\label{sec:4.1}

This section starts with a definition of constructions and \isi{Construction Grammar} (\is{Construction Grammar}CxG) in \sectref{sec:4.1.1}, followed by a discussion of how the \is{Construction Grammar}CxG framework relates to the \isi{usage-based approach}, as advocated by \citet{Bybee2001, Bybee2006, Bybee2010}, in \sectref{sec:4.1.2}. These two sections inform the \is{constructional choice model}choice model proposed in \sectref{sec:4.1.3}, which makes special reference to CCs but can be adapted to other constructions as well. The model is cognitively motivated but has direct consequences for quantitative (statistical) models implemented on its basis.

  For a brief history of the emergence of \is{Construction Grammar}CxG, see \citet{ÖstmanFried2005}; for alternative views of \isi{Construction Grammar}(s) that partly diverge from the approach taken in this study, see \citet[165–289]{CroftCruse2004}; concerning \is{Construction Grammar}CxG and language acquisition, see \citet[222]{Goldberg2003}, \citet{Tomasello2005, Tomasello2006}, and the chapters in part IV of \citet{HoffmannTrousdale2013}, to name but a few. For a seminal early introduction to the rationale behind \is{Construction Grammar}CxG, see \citet{Fillmore1988}.

\subsection{\label{bkm:Ref485630178}Constructions and Construction Grammar}\label{sec:4.1.1}

\isi{Construction Grammar} (or \is{Construction Grammar}CxG) as defined by \citet[1]{BergsDiewald2008} aims to describe grammatical systems in terms of their inventories of constructions at all linguistic levels. Constructions are defined as form-meaning pairings, for example by Goldberg (\citeyear{Goldberg2003}: 219; cf. \citealt{Langacker1987,Croft2005}: 274, \citealt{Trousdale2012}: 168), and in a \is{Construction Grammar}CxG framework, descriptions of (or theories about) language always need to consider both formal and functional aspects. \citet[275]{Croft2005}, for instance, uses the term “\is{constructions!elements of}element” to refer to any identifiable formal aspect of a construction and the term “component” to refer to any identifiable meaning aspect of a construction. While the labels seem problematic (because easily interchangeable), these two concepts are very much applicable to the present study. A classic example of a construction given by \citet[220]{Goldberg2003} is the “covariational-conditional construction”, an instance of which is shown here:

\ea\label{ex:74}
The more you shout, the less they will listen.
\z

This construction has a characteristic form: Within each of the two elements separated by the comma, the determiner (\textit{the}) does not take a nominal complement but some kind of “comparative phrase” \citep[220]{Goldberg2003}, the two parts are most likely interpreted as clauses but are characterised by non-canonical word order, and they are simply juxtaposed, i.e. not overtly linked by a connective. On the function side, the covariational-conditional meaning of the construction is only accessible from the construction as a whole, i.e. it cannot be derived from the component parts~– formal and functional aspects interact and are stored (and used) as a single unit. In contrast to a \isi{Generative Grammar} approach, for instance, which would either classify the above example as marginal or try to derive it from some underlying main-clause-cum-conditional-clause structure via transformational rules, the constructionist view is “non-reductionist” in assuming that there is nothing beyond (or underlying) the observed form and the associated meaning (\citealt{Trousdale2012}: 170; cf. \citealt{Goldberg2006}: 222).

\begin{sloppypar}
Constructions like the covariational-conditional construction pose obvious problems for traditional syntactic analyses, and they are therefore strong pieces of evidence for a \is{Construction Grammar}CxG analysis: Non-canonical and unusual forms do not need to be explained as aberrations from a prototypical pattern but can be directly motivated from the specific functions they serve. However, constructions that \textit{do} conform to canonical patterns are also captured by the \is{Construction Grammar}CxG framework, although they are less conspicuous (see \sectref{sec:4.1.2}). For instance, a simple SVO clause structure clearly qualifies as a (very general) construction, as does a straightforward combination of matrix and subordinate clause. Thus, while first insights into \is{Construction Grammar}CxG are most easily generated by the inspection of syntactically striking examples, a general \is{Construction Grammar}CxG framework must necessarily capture \textit{all} linguistic expressions.
\end{sloppypar}

An important aspect of constructions highlighted by \citet[221]{Goldberg2003} is that “different surface forms are typically associated with slightly different semantic or discourse functions”~– that is, in a \is{Construction Grammar}CxG framework one would naturally hypothesise that a difference in form between two expressions is likely to correspond to some difference in function (or meaning). An example \citep[221]{Goldberg2003} is the difference between ditransitive constructions (S~V~O\textsubscript{i~}O\textsubscript{d} : \textit{I bought him} X.) and prepositional-object constructions (S~V~O\textsubscript{d} O\textsubscript{prep~}: \textit{I bought} X \textit{for him.})~– the formal difference between the two is argued to correspond to some difference in function or meaning. This function-form relationship is a crucial element in the analyses presented in this book.

  In theory, the term \textit{construction} always refers to a \isi{schema} (e.g. the “ditransitive construction”), while \is{constructs}\textit{constructs} or \is{constructions!allostructions}\textit{allostructions} are realisations of constructions, i.e. lexically filled expressions in use (\citealt{BergsDiewald2008}: 5, \citealt{Cappelle2006,Fried2008}: 52). There are of course fixed constructions with very few (or no) options as to how to fill individual slots in the \isi{schema}. If both syntactic frame and lexical content are relatively fixed, the construction is “\is{constructions!lexicalised}lexicalised” or “\is{constructions!idiomatic}idiomatic”; if the syntactic \isi{schema} can be relatively freely filled with lexical content, the construction is “abstract” or “productive” (\citealt{BergsDiewald2008}: 1–2; cf. \citealt{Goldberg2013}: 18)~– the latter type is often referred to simply as “\is{schematicity}schematic”. In the present study, the notion of the \textit{subconstruction} also plays a role. In my definition of this concept, and without tying it explicitly into existing \is{Construction Grammar}CxG frameworks, \is{constructions!subconstructions}subconstructions can be located at levels of \isi{schematicity} that are intermediate between highly general constructions and \isi{constructs} that are syntactically fully specific and lexically filled. For instance, if we treat anticausal CCs with subordinate clauses as our maximally \is{schematicity}schematic construction, then CCs with sentence-initial subordinates and CCs with sentence-final subordinates would be \is{constructions!subconstructions}subconstructions at a lower level. These are still lexically unfilled, but syntactically more specific than the general \isi{schema}. At the next level of specificity, we would then identify the conjunction that is used to connect matrix clause and subordinate clause. Finally, the grammatical status of the subordinate clause (finite vs nonfinite) can be included at an even finer level of granularity. The exact (hierarchical) arrangement of such layers will be partly open to debate, however~– for instance, one could disagree about whether it is the choice of a conjunction or the ordering of component clauses that ranks higher, or whether one should place the two on the same level.

  The \is{constructional choice model}choice model that informs the quantitative analyses in this study is an attempt to formalise a framework of \is{constructions!subconstructions}subconstructions for CCs. In this framework, information at more general (“higher”) functional and formal levels can be used to predict realisations at more specific (“lower”) levels. On the one hand, CCs as constructions do in principle allow for all combinations of functions and forms as defined in this study; on the other hand, there are probabilistic ties between the different functional and formal facets of CCs. This reasoning provides the main link between \is{Construction Grammar}CxG proper and the quantitative analyses presented in the later chapters of this book. However, both the exact sequence of ranked causes and effects in the proposed \is{constructional choice model}choice model as well as the idea of a hierarchy itself may be challenged. Ultimately, the question will be whether the approach contributes to a cognitively grounded explanatory model or simply establishes useful correlations between functional and formal facets of CCs. The latter case would be of value in itself but of course theoretically less satisfying.

\subsection{\label{bkm:Ref485723728}Constructions and the usage-based approach}\label{sec:4.1.2}

Combining \is{Construction Grammar}CxG with the \isi{usage-based approach} (\citealt{Langacker1987,Langacker1988,Bybee2001,Bybee2006,Bybee2010,Bybee2013,Phillips2006}) can generate theories concerning both the emergence and the cognitive representation of constructions, as well as the paths along which those representations change through language use. Bybee’s \is{usage-based approach}usage-based model~– particularly in its version that is geared more specifically towards \is{Construction Grammar}CxG (\citealt{Bybee2013,Bybee2001}: 171–177)~– is appealing in its capacity for taking into account the multi-faceted (or multidimensional) nature of constructions. \citet[51]{Bybee2013} argues that “[c]onstructions, with their direct pairing of form to meaning without intermediate structures, are particularly appropriate for \is{usage-based approach}usage-based models.” Combinations of linguistic structure and meaning become \is{entrenchment}entrenched as constructions~– i.e. they are turned into “processing units or \isi{chunks}”~– if they are frequently encountered in use. This, Bybee says, happens even if they lack the unpredictable (or idiosyncratic) formal or functional behaviour sometimes regarded as a defining characteristic of constructions (e.g. \citealt{Goldberg2003}). Thus, even fully predictable structures qualify as constructions if they occur frequently enough (\citealt{Bybee2001}: 173, \citealt{Goldberg2006}: 5, \citealt{Trousdale2012}: 170).

  According to \citet[53–54]{Bybee2013}, linguistic experience is stored in mental categories called \textit{exemplars}, which exist at all levels of language and also pertain to non-linguistic parameters. Each language event will therefore trigger and be connected to different \isi{exemplars}, e.g. one that best represents its phonetic properties, one that represents its concrete semantics, one that represents the context of production, and so forth. Exemplars are grouped in an \is{exemplar clouds}\textit{exemplar cloud} when they store information concerning the same parameter. For instance, different meanings will be stored in \isi{exemplars} that belong in a semantic \is{exemplar clouds}exemplar cloud, and the context in which each utterance is made is stored in the respective \is{exemplars}exemplar of a stylistic/contextual \is{exemplar clouds}exemplar cloud.\footnote{\citet{Bybee2013} calls these parameters “criteria”.} In other words: Each \is{exemplar clouds}exemplar cloud corresponds to one of the relevant characteristics (formal, functional and \is{language-external factors}language-external) needed for a full description of a particular construction; the \isi{exemplars} contained within each of these, on the other hand, correspond to a possible realisation of the respective characteristic. If a certain construction is encountered frequently, the relevant \isi{exemplars} (within their respective \is{exemplars}exemplar clouds) will be strengthened, as will the connections (or ties) between them. \figref{fig:4.1} provides a schematic illustration, which represents categories relevant to this study. Information about a CC is stored in four \is{exemplars}exemplar clouds: Cloud 1 contains \isi{exemplars} of the different semantic types; cloud 2 contains \isi{exemplars} of different clause positions; cloud 3 stores \isi{exemplars} of the different conjunctions; and cloud 4 contains \isi{exemplars} of different complement realisations. The grey lines in the figure suggest that there are connections between any \is{exemplars}exemplar in one particular cloud and all \isi{exemplars} in the other clouds. One such combination is highlighted. We can think about this model as a compartmentalised representation: In category E1 (semantics), all instances of concessives encountered by the language user are stored by sorting them into (in this case three) subcategories, or \isi{exemplars} (e.g. anticausal, epistemic and dialogic). The same happens within the formal categories E2–E4. Along with the \isi{exemplars} in each cloud, the language user stores degrees of interconnectedness between them, across \is{exemplars}exemplar clouds. That is, in processing a CC encountered in use, the links between the four involved \isi{exemplars} are triggered along with the \isi{exemplars} themselves. Frequent triggering of this kind leads to a general strengthening of particular combinations of functional and formal characteristics, which will then be easier to produce and process. In other words, these CCs become strengthened as subconstructions, as indicated by the black lines in \figref{fig:4.1}.

\begin{figure}
\includegraphics[width=\textwidth]{figures/CCs.Fig.4.1.pdf}
\caption{\label{bkm:Ref497025942}\label{fig:4.1}Exemplars and exemplar clouds applied to CCs}
\end{figure}

By measuring the strength of certain connections between \isi{exemplars} in the network based on frequency of use, typical constructional patterns can be identified. According to \citet[54]{Bybee2013}, the establishment of such links in the ex\-em\-plar-based model is one way of conceptualising the emergence of constructions as cognitive representations, and it is in such processes that \is{Construction Grammar}CxG and the \isi{usage-based approach} come together. Similarly, \citet[50]{Fried2008} in her constructionist approach views grammar as consisting of “networks of partially overlapping patterns organized around shared features”, which comes quite close to the marriage of the \isi{usage-based approach} and \is{Construction Grammar}CxG in \citet{Bybee2013}. Certain elements of the \isi{usage-based approach} also seem to be implied in publications by Goldberg, when, for example, she refers to a construction as being acquired “on the basis of positive input” (\citeyear{Goldberg2003}: 222) or as a “\textit{learned} pairing of form and function” (\citealt{Goldberg2013}: 15; my emphasis, OS), or when discussing the concept of “\isi{statistical preemption}” in the emergence of constructions (\citeyear{Goldberg2011}: 133). The latter appears to be a process rather similar to that involved in the strengthening and weakening of \isi{exemplars} in the sense of Bybee.

The advantages of this view of constructions as being defined through the strength of ties between \isi{exemplars} at different levels (semantic, syntactic, etc.) are twofold. For one, it is not merely tolerant of but in fact ideal for the charting of variation, since all \isi{exemplars} are part of the network, not only the strongest (or most strongly connected) ones. For another, it is quantifiable, since the strength of connections can be measured, either based on the relative frequencies of certain \is{exemplars}exemplar combinations, or as directional relationships in regression models. The latter approach is taken in this study and will be explained in more detail in the following section. Operationalising connections between \is{exemplars}exemplar clouds in this (directional, or sequential) way has the disadvantage that the strictly simultaneous view implicit in \figref{fig:4.1} is abandoned: Functional categories have an impact on formal categories, and higher-order formal categories have an impact on lower-order ones, while in a strictly \is{Construction Grammar}CxG approach, different components would be seen as being on a par with each other. I will argue, however, that the conceptualisation of constructions as tightly integrated sets of ordered choices is useful not only for practical reasons, but also for theoretical ones.

\subsection{\label{bkm:Ref35374759}A choice model of constructional variation}\label{sec:4.1.3}

In this section, the \is{usage-based approach}usage-based model introduced above will be modified by giving the ties between elements in different \isi{exemplar clouds} a particular direction. While this does not abandon the idea that function and form are inextricably linked (or fused) in constructions, it introduces a certain hierarchical thinking: Higher-order and lower-order characteristics of constructions are assumed to exist, and this ranking can be put to use in theoretical and empirical work. There will be a brief discussion of what I call a \is{constructional choice model}\textit{\is{constructional choice model}choice model of constructional variation} for English subordinating CCs, building directly upon definitions and descriptions found in \sectref{sec:4.1.2} and pointing ahead to the quantitative analyses and their interpretation in the later chapters.

  The functional (or meaning) side of a CC is defined by the four semantico-pragmatic types discussed in \sectref{sec:2.2}, namely
(i)~\is{concessives (types of)!anticausal}anticausal,
(ii)~\is{concessives (types of)!epistemic}epistemic,
(iii)~\is{concessives (types of)!dialogic}dialogic and
(iv)~\is{concessives (types of)!narrow-scope dialogic}narrow-scope dialogic. As has been explained in \chapref{sec:2}, only the two most frequent categories (anticausal vs dialogic) will be used in the statistical analysis, but this is irrelevant for the principles outlined here. In this study, then, \textit{function} denotes the relationship between propositions within the construction, or the function of intra-constructional propositions relative to each other. As an alternative to (or expansion of) this relatively local view, which I will call \textit{hermetic}, one could inspect a CC’s communicative or discourse function and the relations that hold between it and its wider context of use (see discussion in \chapref{sec:1}).

  The formal (grammatical) parameters relevant for CCs are threefold:
(i)~the position of the subordinate clause relative to the matrix clause (cf. \sectref{sec:2.3.1});
(ii)~the \is{connectives}connective that introduces the subordinate clause; and
(iii)~the internal syntactic structure of a subordinate clause (cf. \sectref{sec:2.3.2}). In the quantitative analysis, the three-way distinction between initial, medial and final position will be reduced to a binary scheme with the categories \is{nonfinal position}\textit{nonfinal} (including medial) and \is{final position}\textit{final}. Concerning the third aspect, two complement types are possible in combination with subordinating conjunctions: \is{finite clauses}finite clauses (including \is{subjunctive}subjunctives and \textit{though}-inverted clauses) and \is{nonfinite clauses}nonfinite (i.e. participial or \is{verbless clauses}verbless) clauses. This simplified inventory of distinct form-function combinations thus comprises $n=24$ categories: 2~semantic types\,×\,2~clause positions\,×\,3~conjunctions\,×\,2~complement types. Accordingly, for each of the two functional types that are included (\is{concessives (types of)!anticausal}anticausal and \is{concessives (types of)!dialogic}dialogic), the number of possible formal realisations is $n=12$.

The \is{constructional choice model}choice model that informs the quantitative analyses in this study is an attempt to formalise a framework of \isi{constructional variation} for CCs in English. In the model, information at more general (higher) functional and formal levels can be used to predict realisations at more specific (lower) levels. The following five assumptions are made:

\begin{enumerate}
\item Language users produce (and store) constructions.
\item Constructions are correlated functional and formal properties of language.
\item Functional and formal properties of constructions can be ranked:
  \begin{enumerate}
  \item Form follows function.
  \item Lower-order formal properties follow higher-order ones.
  \end{enumerate}
  
\item The ranking can be employed for
  \begin{enumerate}
  \item the identification of \is{constructions!subconstructions}subconstructions and
  \item the statistical modelling of \isi{constructional variation} and change.
  \end{enumerate}

\item The production of constructions (and \is{constructions!subconstructions}subconstructions) is not categorical but probabilistic.
\end{enumerate}

Assumptions 1 and 2 are in broad agreement with existing \is{Construction Grammar}CxG approaches, have been discussed in different terms in \sectref{sec:4.1.1} and will therefore be taken for granted here. The third assumption is based on the (\is{onomasiological approaches}onomasiological) view that the need to express semantic and/or pragmatic meaning is primary, and the linguistic choices that are made to express it are secondary. Further, it is assumed that broader (or more general) formal choices~– e.g. located in superordinate structures or heads, in a traditional sense~– take precedence over choices corresponding to traditionally lower-ranking structures~– e.g. located in subordinate structures or complements/postmodifications. In concrete terms, and with reference to CCs, selecting a general syntactic grid consisting of an arrangement of matrix and subordinate clause (\textsc{matrix}→\textsc{sub} or \textsc{sub}→\textsc{matrix}) is followed by the choice of the marker that introduces the subordinate clause, which in turn is followed by selecting a specific syntactic type of subordinate clause. \figref{fig:4.2} shows the \is{constructional choice model}choice model in a schematic form that contains only categories included in the quantitative analyses.

\begin{figure}
\includegraphics[width=10cm]{figures/CCs.Fig.4.2.pdf}
\caption{\label{bkm:Ref35424208}\label{fig:4.2}A choice model for CCs}
\end{figure}

Constructions are still regarded as unitary concepts, with functional and formal parameters inextricably linked. What the model is additionally meant to supply, however, is a framework for the identification of \is{constructions!subconstructions}subconstructions at different levels. Starting from a certain functional (or meaning) category, we proceed to different formal layers: There are two \is{constructions!subconstructions}subconstructions at the highest and most \is{schematicity}schematic level, namely CCs with subordinate clauses in final and nonfinal position, respectively. Each of these breaks down into three \is{constructions!subconstructions}subconstructions at a lower level, distinguished by means of the three conjunctions. At the lowest level, \is{constructions!subconstructions}subconstructions are additionally specified for the syntactic class of the subordinate clause.

\figref{fig:4.3} shows the consequences of the \is{constructional choice model}choice model for the notion of \is{usage-based approach}usage-based \is{Construction Grammar}CxG: Ties between members of different \is{exemplars}exemplar clouds are shown only for adjacent levels, once again with one particular combination highlighted in black. The three choices that are made are indicated using arrows, and indexed using the letters A, B and C. We still assume that for any \is{constructs}construct, all four parameters~– linked by a single path through the four \is{exemplars}exemplar clouds~– must be stored in combination and are not triggered independently. However, the hierarchy of parameters enables us to identify more or less \is{schematicity}schematic \is{constructions!subconstructions}subconstructions, and it will give our quantitative analysis a direction. If this was not the case, it would be hard to decide which parameters to use as predictors, and which as outcomes. These issues will become clearer in Chapters \ref{sec:9}–\ref{sec:11}.

\begin{figure}
\includegraphics[width=\textwidth]{figures/CCs.Fig.4.3.pdf}
\caption{\label{bkm:Ref35430401}\label{fig:4.3}Merging exemplar clouds and the choice model}
\end{figure}

The model is problematic on two counts:
(i)~It re-introduces traditional grammatical concepts to \is{Construction Grammar}CxG (e.g. hierarchies, headedness), at least notionally; and
(ii)~it can at present make no claims regarding cognitive validity. However, it is consonant with the idea that constructions may differ in their degree of \isi{schematicity}, and it can be used to postulate \is{constructions!subconstructions}subconstructions at different levels.

\pagebreak\section{\label{bkm:Ref89705747}\label{bkm:Ref89705748}\label{bkm:Ref89705749}\label{bkm:Ref485629936}Mode of production}\label{sec:4.2}

A fine-grained investigation of variation across \is{genre}genres of English is beyond the scope of the present study, which will limit itself to inspections of the two \is{mode of production}modes of production, speech and writing. While particularly corpora from the \is{International Corpus of English}ICE family are structured so as to enable the comparison of various written and spoken \is{genre}genres or \is{register}register (see \sectref{sec:6.1} and Appendix~\ref{appendix:A.1}), the connectives under investigation are not frequent enough to allow meaningful comparisons at finer levels of granularity. Larger corpora that contain spoken and written material (e.g. \is{corpora!COHA}COHA, \is{corpora!BNC}BNC) are of limited use in the \isi{World Englishes} paradigm since they are restricted to the two main reference dialects of the language, \il{American English}AmE and \il{British English}BrE.

According to \citet[42–45]{Chafe1994}, prototypical speech is evanescent, relatively quickly and spontaneously produced, and clearly situated concerning place, time and interlocutors; writing, on the other hand, is produced more slow\-ly than speech, takes a more permanent form and may be edited and revised (cf. \citealt{Linell2005}: 21). Furthermore, it is desituated, i.e. less clearly tied to a particular temporal, local or circumstantial context. Another basic difference between the two modes of language production is highlighted by \citet[59]{Fowler1991}, who associates printed language in particular with “\isi{formality} and authority” and speech with “\isi{informality} and solidarity”. Fowler also acknowledges that \isi{text types} written in one medium may assume certain characteristics of the other, so that prototypical writing and prototypical speech constitute the poles of a continuum, rather than discrete categories. \citet[45–46]{Biber1988} cites \isi{face-to-face conversation} as an example of typical speech and \is{academic language}academic expository prose as an example of typical writing. According to him, \is{academic language}academic lectures are an example of speech with characteristics of writing, while personal letters could be described as writing with characteristics of speech. The broad division of the \is{International Corpus of English}ICE corpus into spoken and written sections merges several such ambivalent \is{genre}genres. As higher-level categories, speech and writing in \is{International Corpus of English}ICE therefore lack in focus and specificity.\footnote{\citeauthor{KochOesterreicher1985} (e.g. \citeyear{KochOesterreicher1985}) describe many central differences between speech and writing, in fact anticipating much of what was (independently) formulated by \citeauthor{Biber1988} (e.g. \citeyear{Biber1988}) and others. As \citet[24, 36–37]{Biber1988} points out, there is no dimension of variation that simply corresponds to the dichotomy “spoken” vs “written”; from a more general perspective, i.e. ignoring finer textual distinctions within each category, there may be as much variation \textit{within} speech and writing, respectively, as there is \textit{between} them.}

  Concerning structural (i.e. linguistic) differences between speech and writing, two central dimensions of variation are proposed by \citeauthor{Chafe1982} (\citeyear{Chafe1982}: 38–49; cf. \citealt{Chafe1985,ChafeDanielewicz1987,Biber1988}: 21):
(i)~fragmentation vs integration and
(ii)~involvement vs detachment. One symptom of the fragmentation of speech is the succession of coordinated (shorter) clauses and the consequently relatively low number of subordinate clauses and the connectives that introduce them (cf. \citealt{Akinnaso1982}: 104)~– a finding that is relevant in the context of the present study. Writing, on the other hand, is more integrated. It contains nominalisations, participles, attributive adjectives, prepositional phrases, and dependent clausal structures (certainly including subordinate adverbial clauses, which are not explicitly mentioned by Chafe, however).

  Involvement in oral texts can manifest itself in higher text frequencies of first person pronouns, emphatic particles and hedges; detachment in written texts, by contrast, may result in higher frequencies of passives and nominalisations. It can also be hypothesised that involvement correlates with different pragmatic strategies relevant with regard to different semantico-pragmatic types of concessives (cf. \sectref{sec:2.2}). For instance, according to \citet[45–48]{Chafe1982}, involvement may be reflected in “[r]eferences to a speaker’s own mental processes” (e.g. thinking, remembering, reasoning, etc.). Such processes are arguably more transparent in \is{concessives (types of)!epistemic}epistemic and \is{concessives (types of)!dialogic}dialogic concessives, and less transparent in \is{concessives (types of)!anticausal}anticausal concessives. \citegen[47--49]{Biber1988} discussion of explicitness (in writing) and implicitness (in speech) points in a similar direction: Writing is explicit in that it overtly encodes assumptions and logical relations in a text; speech, on the other hand, is more implicit, constructing meaning between interlocutors who jointly contribute to the interpretation process~– according to \citet[18]{Linell2005}, \is{speaker/writer}SP/W and \is{addressee/reader}AD/R “co-construct inter\-pretations” in conversation. Thus, it could be hypothesised that the incidence partic\-u\-larly of \is{concessives (types of)!dialogic}dialogic concessives, with their pragmatically ambivalent character, will be higher in more involved types of text, and thus in spoken registers.

  Finally, there is also a crucial difference between speech and writing in \isi{language acquisition}, which can help to account for corresponding differences in the use of certain constructions in the two \is{mode of production}modes of production. As \citet[111]{Akinnaso1982} points out, speech is for the most part acquired “naturally”, not at school. The same point is made by \citet[23]{Linell2005}, who argues that more explicit instruction is involved in learning to write. Acquiring literacy involves what Linell calls “goal-directed study”, based on “explicit norms”. Such explicit norms are endorsed by language teachers and codified in grammars, usage guides and teaching materials. The presence (or absence) of certain norms concerning the use of concessives in such reference works may thus contribute to explanations of patterns found particularly in \is{varieties of English!L2}L2 varieties, in which English is acquired scholastically to a greater extent.

  It is possible to view speech and writing simply as very general high-level \is{genre}genres. Precisely this is done by Miller \& Weinert (\citeyear{MillerWeinert1998}: 17; cf. \citealt{Chafe1994}: 48). Within the broad \isi{genre} category of writing, they argue, one can differentiate between various “sub-genres”, e.g. literature, business correspondence, company reports and academic books, which may in turn break down into “sub-sub-genres” (e.g. subdivision of literature into novels, plays, poetry, autobiography and diary). This hierarchical view of \isi{genre} is also reflected in the \is{sample}sampling scheme adopted for the \textit{International Corpus of English}, for instance (cf. \sectref{sec:6.1} and Appendix~\ref{appendix:A.1}). At the analytic level, however, only the first-order difference between speech and writing plays a role in the present study; nevertheless, \isi{genre} differences will sometimes be referred to in a more general way. I use the term \is{genre}\textit{genre} in the same sense as \citegen{BiberConrad2009} “text variety”, i.e. a sort of text that is produced under certain communicative circumstances. In this terminological decision I follow Smitterberg \& Kytö (\citeyear{SmitterbergKytö2015}: 118; cf. \citealt{Meurman-Solin2001}: 243, \citealt{Moessner2001}: 134–135), who use \textit{genre} for “categories of texts that are defined on extralinguistic or text-external grounds”. By contrast,   \citet[118]{SmitterbergKytö2015} use the term “\is{text types}text type” to refer to categories of text that differ on linguistic grounds. Thus, according to them, “the linguistic make-up of the text itself […] does not determine what \isi{genre} it belongs to”. This is very much in line with the predominant approach in studies that make use of \is{International Corpus of English}ICE \is{International Corpus of English!components}components: Different kinds of text (\is{genre}genres) are sampled from different communicative situations to be then analysed in terms of their linguistic structure.

\section{\label{bkm:Ref35373991}\label{bkm:Ref35882627}Varieties of English}\label{sec:4.3}

Varieties of English are one of the dimensions across which constructions are assumed to vary in the present study. Section \ref{sec:4.3.1} summarises some general and conceptual issues involved in what has been called the “\isi{World Englishes} paradigm” \citep{Mesthrie2003}, i.e. the investigation of variation and change in the English language against the background of its spread and diversification across the globe. Furthermore, it intro\-duces the varieties that are studied. Relevant models that have been proposed to describe \isi{World Englishes} and processes involved in their emergence will be discussed in \sectref{sec:4.3.2}.

\subsection{\label{bkm:Ref497158198}General aspects}\label{sec:4.3.1}

English is a \is{pluricentricity}pluricentric language (cf. \citealt{Kachru1988}: 3, \citealt{Clyne1992,Leitner1992}) spoken in various locations throughout the world, all of which have the potential of developing their own linguistic norms and standards. \citet[vii]{Ferguson1982} considers the spread of English across the globe to be “one of the most significant linguistic phenomena of our time”, and for   \citet[12–17]{MesthrieBhatt2008} it is a defining characteristic of the Modern English period (cf. \citealt{McArthur1998}: 87). These views are also reflected in the amount of research on \isi{World Englishes} that has been and continues to be produced.

  Three models of English will be discussed in this section: Kachru’s (\citeyear{Kachru1985, Kachru1988}) \textit{Concentric Circles of English} model, \citegen{McArthur1987} \textit{Circle of World English}, and Schneider’s (\citeyear{Schneider2003, Schneider2007}) \textit{\isi{Dynamic Model} of the Evolution of Postcolonial Englishes}.\footnote{\citet[2–56]{Jenkins2015} provides detailed summaries of several other models of English.} Traditional terms that play a more or less central role in many discussions are \textit{English as a native language} (\is{varieties of English!L1}L1 / \is{varieties of English!ENL}ENL), \textit{English as a second language} (\is{varieties of English!L2}L2 / \is{varieties of English!ESL}ESL), and \textit{English as a foreign language} (\is{varieties of English!EFL}EFL).\footnote{The terms \textit{English as a lingua franca} (ELF) and \textit{English as an International Language} (EIL) and the~– sometimes overlapping~– concepts they stand for play no role in my study (cf. \citealt{Pennycook1994,Modiano1999,Jenkins2000,Jenkins2007,Seidlhofer2011}).} Although the analyses in Chapters \ref{sec:7}–\ref{sec:11} do inspect patterns in individual varieties, their main objective is to assess cross-varietal stability and variation, not to discuss socio-stylistic patterns and their implications for the status of individual varieties. Models like Schneider’s (\citeyear{Schneider2003,Schneider2007}; see below) therefore serve as a general background to this study but are not exploited to the full. Their discussion in this section is accordingly kept relatively short.

Data from $n=9$ \isi{varieties of English} are discussed: \ili{British English} (\il{British English}BrE), \ili{Irish English} (\il{Irish English}IrE), \ili{Canadian English} (\il{Canadian English}CanE), \ili{Australian English} (\il{Australian English}AusE), \ili{Jamaican English} (\il{Jamaican English}JamE), \ili{Nigerian English} (\il{Nigerian English}NigE), \ili{Indian English} (\il{Indian English}IndE), \ili{Singapore English} (\il{Singapore English}SingE) and \ili{Hong Kong English} (\il{Hong Kong English}HKE). \tabref{tab:4.1} lists the following parameters for each variety:
(i)~\is{varieties of English!L1}L1/\is{varieties of English!L2}L2 status,
(ii)~variety label,
(iii)~\is{world regions}world region, and
(iv)~the developmental phase according to \citegen{Schneider2003} \isi{Dynamic Model}. Information concerning the latter is taken from \citeauthor{Schneider2007} (\citeyear{Schneider2007}; also cf. \citealt{Schneider2011}). As the table shows, there are four \is{varieties of English!L1}L1 varieties and five \is{varieties of English!L2}L2 varieties, covering six of the eight Anglophone \isi{world regions} (cf. \citealt{KortmannSzmrecsanyi2011}: 275, \citealt{KortmannEtAl2020}): the \isi{British Isles} (\il{British English}BrE, \il{Irish English}IrE), \isi{America} (\il{Canadian English}CanE), the \isi{Caribbean} (\il{Jamaican English}JamE), \isi{Africa} (\il{Nigerian English}NigE), \is{South Asia}South and \isi{Southeast Asia} (\il{Indian English}IndE, \il{Singapore English}SingE, \il{Hong Kong English}HKE), and Australia (\il{Australian English}AusE). The \is{Pacific region}Pacific and the \is{South Atlantic region}South Atlantic are not represented in the study.

\begin{table}
\caption{\label{tab:4.1}Varieties of English in this study}
\begin{tabular}{llll}
\lsptoprule
\is{varieties of English!L1}L1/\is{varieties of English!L2}L2 & Variety & World region & Phase\\\midrule
\is{varieties of English!L1}L1 & \il{British English}BrE & British Isles & 5 / n.a.\\
& \il{Irish English}IrE & British Isles & 5\\
& \il{Canadian English}CanE & America & 5\\
& \il{Australian English}AusE & Australia & 5\\
\is{varieties of English!L2}L2 & \il{Jamaican English}JamE & Caribbean & 4\\
& \il{Nigerian English}NigE & Africa & 3\\
& \il{Indian English}IndE & S and SE Asia & 3–4\\
& \il{Singapore English}SingE & S and SE Asia & 4\\
& \il{Hong Kong English}HKE & S and SE Asia & 3\\
\lspbottomrule
\end{tabular}
\end{table}

At a higher level, the arrangement of varieties in the table and in the visualisations of results follows the division into \is{varieties of English!L1}L1 and \is{varieties of English!L2}L2; within each of these sets, geographical principles are applied, with \is{varieties of English!L1}L1 varieties ordered according to distance from Britain (\il{British English}BrE, \il{Irish English}IrE, \il{Canadian English}CanE, \il{Australian English}AusE) and \is{varieties of English!L2}L2 varieties arranged from West to East (\il{Jamaican English}JamE, \il{Nigerian English}NigE, \il{Indian English}IndE, \il{Singapore English}SingE and \il{Hong Kong English}HKE).

\subsection{\label{bkm:Ref497158185}Models of English}\label{sec:4.3.2}

The three influential models of English proposed by \citet{Kachru1985,Kachru1988,McArthur1987} and \citet{Schneider2003, Schneider2007} will be summarised and discussed below. \figref{fig:4.4} gives a first overview, which shows all three models in juxtaposition.

\begin{figure}
\includegraphics[width=\textwidth]{figures/CCs.Fig.4.4.pdf}
\caption{\label{bkm:Ref475024343}\label{bkm:Ref497947644}\label{fig:4.4}Models of English}
\end{figure}

Kachru’s is probably the most influential one among models based on circles (\citealt{Werner2014}: 34, \citealt{Jenkins2015}: 13). It is motivated by a critique of “a monolingual model for linguistic description and analysis” \citep[11]{Kachru1985}, which prevailed at the time and which is to some extent still reflected in other models (e.g. the model by McArthur discussed below; also cf. \citealt{Görlach1990}). Kachru’s \textit{Inner Circle} of Englishes contains varieties which Kachru calls “the traditional bases of English” where it is “the primary language”, or \is{varieties of English!L1}L1 \citep[12]{Kachru1985}. The \is{varieties of English!Outer Circle}\textit{Outer Circle} comprises \isi{varieties of English} that have emerged through colonisation~– these are what \citet[3–4]{PlattEtAl1984} call \textit{New Englishes}.\footnote{It is striking how rarely the alternative term “Extended Circle” features in later publications on the subject, although it rather elegantly reduces the terminological distance between the two innermost circles. Perhaps “extended” and “expanding” are too similar and thus too easily mixed up.} In outer-circle countries, English is a non-native second language (\is{varieties of English!L2}L2), which, however, is given some institutionalised role within the speech community (\citealt{Kachru1985}: 12–13). \citet[4]{Quirk1985} calls these functions of an \is{varieties of English!L2}L2 within the speech community “internal purposes”, as opposed to the “external purposes” of communicating with non-members of the speech community (cf. \citealt{Greenbaum1996}: 4). The official role of English as an \is{varieties of English!L2}L2 is often, but not necessarily, decreed by political agencies (\citealt{PlattWeberHo1984}: 198). Further, \is{varieties of English!L2}L2 English will very often not be the primary language of daily interaction in the home and will therefore first be transmitted through the school system (\citealt{PlattWeberHo1984}: 2; cf. \citealt{MesthrieBhatt2008}: 11). Finally, the \is{varieties of English!Expanding Circle}\textit{Expanding Circle} comprises those countries or territories which do not have an English colonial background. Here, English is a \is{varieties of English!EFL}foreign language not used for internal purposes among members of the speech community \citep[13]{Kachru1985}. The three circles also differ in the way they adhere to norms or standards (\citealt{Kachru1985}: 16–17). Inner-circle varieties are \is{varieties of English!norm-providing}\textit{norm-providing} (or \is{varieties of English!endonormative}\textit{endonormative}), because they are recognised as being used by the native speaker. However, there are considerable differences within this circle in this regard. For instance, \il{Australian English}Australian and \ili{New Zealand English} (and probably also \il{Canadian English}Canadian and \ili{Irish English}) are less widely recognised as norms than \il{British English}BrE and \il{American English}AmE. The \is{varieties of English!Outer Circle}Outer Circle is categorised as \is{varieties of English!norm-developing}\textit{norm-developing} by Kachru. Varieties in this circle are both \is{varieties of English!exonormative}\textit{exo}- and \is{varieties of English!endonormative}\textit{endonormative}, i.e. outward- and inward-looking for their norms. This implies that one cannot assume a single norm for all levels of the linguistic system, i.e. certain features (or groups of features) may follow an external norm, while others have truly nativised. Even if an outer-circle variety has developed into a norm provider in usage, the new norms will not necessarily be available as a model for language learners, due to a lack of codification (cf. \citealt{Kachru1985}: 17). In other words, the emerging norms are entirely sociolinguistic and implicit, not pedagogical.\footnote{This is also true with regard to several \is{varieties of English!L1}L1 varieties. Take, for instance, \ili{Scottish Standard English} (SSE; cf. \citealt{Schützler2015}), for which there are few attempts to promote salient and positively evaluated features (for example in pronunciation) and give them a place in education (but see \citealt{Grant1914,Abercrombie1991}: 53).} Finally, varieties in the \is{varieties of English!Expanding Circle}Expanding Circle are \textit{norm-dependent}, or \is{varieties of English!exonormative}\textit{exonormative}; as a general rule, they do not develop norms of their own.

Another circle-shaped model is the one by \citet[11]{McArthur1987}, which is shown in \figref{fig:4.4}b above. For the sake of simplicity, the figure does not display specific varieties at the periphery of the model (for full details, see \citealt{McArthur1987}: 11, \citealt{McArthur1998}:~97). The model is constructed in such a way as to reflect

\begin{quote}
the broad three-part spectrum that ranges from the ‘innumerable’ popular Englishes through the various national and regional standards to the remarkably homogeneous but negotiable ‘common core’ of \ili{World Standard English}.
\end{quote}

In McArthur’s model, regional and national standards and non-standardised varieties form continua in the respective local or national domain.\footnote{This idea~– as well as much of what is said about “\ili{World Standard English}” by \citet{McArthur1987}~– is anticipated in an earlier publication (\citealt{McArthur1979}: 54–57) much quoted in Scottish English studies, where the concept of a bipolar \ili{Scots}-English continuum is developed (see also \citealt{Schützler2015,SchützlerGutFuchs2017}).} \citet[11]{McArthur1987} is very much aware of some shortcomings of his model, e.g. concerning the relative status of British, Irish, Scottish and English (i.e. Southern British) English, as well as the perhaps overstated difference between \il{American English}American and \ili{Canadian English}~– indeed, despite low speaker numbers, \ili{Scottish Standard English} may well be a more independent standard than \ili{Canadian English}, for instance. However, it seems much more interesting to focus on the underlying principles, rather than the exact placement of varieties in the model. For further criticism of McArthur’s model, see \citet[27–28]{MesthrieBhatt2008}.

  McArthur’s model also implies that the continua between regional and national standards and their associated non-standard (or “popular”) varieties may be extended inwards, resulting in more global continua between the respective world-regional standards and \ili{World Standard English}. The latter is sometimes associated with certain types of (written) text; for example, \citet{McArthur1987} likens the present-day situation of English to that of classical \ili{Latin}~– whose stability lies in its written form~– and refers to “a text-linked World Standard” (\citealt{McArthur1987}: 10; also cf. \citealt{McArthur2003}: 56). That is, \ili{World Standard English} in the sense of McArthur is not codified as such, nor would speakers across the globe have strong intuitions or sentiments about it. Rather, it is “negotiated among a variety of more or less established national standards” \citep[10]{McArthur1987} whenever the contextual need arises.

  The final model to be discussed is Schneider’s (\citeyear{Schneider2003}, \citeyear[21–70]{Schneider2007}) \textit{\isi{Dynamic Model} of the Evolution of Postcolonial Englishes}, usually referred to more simply as the \textit{Dynamic Model}. Schneider’s model is dynamic since it assumes five developmental phases through which a postcolonial variety may pass in a certain order, as shown in \figref{fig:4.4}c. Varieties can then be characterised according to the stage they have reached. The model assigns a strong role to the speech community and the way its members construct their (postcolonial) identities relative to the (former) coloniser. The model predicts that different kinds of \is{social identity}identity construction will result in different kinds of linguistic accommodation (\citealt{Schneider2007}: 26–29). Like the models discussed above, Schneider’s contribution does not claim exclusive validity: The \isi{Dynamic Model} takes a particular (namely postcolonial) perspective on \isi{World Englishes} and focuses on aspects not rigorously addressed before. The five phases of the model are briefly summarised in the following paragraph.\footnote{For a detailed account, see \citet[33–55]{Schneider2007}. See also \citet[32–33]{MesthrieBhatt2008}, \citet[11–12]{Schneider2014} and \citet{Werner2014} for summaries, as well as contributions in \citet{BuschfeldEtAl2014}.}

In phase 1 (\is{foundation (Dynamic Model)}\textit{foundation}), English is brought to a new territory and comes to be used on a regular basis. Settlers and indigenous population have separate identities, the former emphasising the affiliation to the country of origin, the latter regarding themselves as the true and rightful inhabitants of the territory. Cross-cultural communication is relatively limited. There is incipient pidgini\-sation as well as limited lexical borrowing and a modest degree of bilingualism. In phase 2 (\is{exonormative stabilisation (Dynamic Model)}\textit{exonormative stabilisation}), the colonial setting becomes more stable, both politically and linguistically, and colonial administrative structures and the orientation towards the linguistic norms of the colonisers are strengthened. English is spoken more widely. Settlers still view themselves as such, but also perceive a difference between themselves and those at home who do not share the “colonial experience” \citep[37]{Schneider2007}. Many among the indigenous population are beginning to see the benefits of speaking English, and English-speaking indigenous elites emerge. English is beginning to show the first signs of developing into a local variety. In phase 3 (\is{nativisation (Dynamic Model)}\textit{nativisation}), there are movements towards political and linguistic independence. Cultural, ethnic, economic and linguistic differences between settlers and natives are reduced, and all inhabitants share a sense of belonging to the same territory, despite their different origins. Among the indigenous population, bilingualism is common, and local features begin to stabilise at all linguistic levels. The settler population, on the other hand, divides into two camps: those who readily adopt nativised local features and those who resist. Phase 4 (\is{endonormative stabilisation (Dynamic Model)}\textit{endonormative stabilisation}) typically follows the achievement of political independence. Crucially, local \is{social identity}\is{social identity}identity is now constructed so as to emphasise difference from the mother country. Local forms of English lose their stigma and are widely used to express the new (national) \is{social identity}identity; the variety is now \is{varieties of English!endonormative}endonormative. While local forms are also used in phase 3, in phase 4 such forms are available in more contexts, i.e. not only in the vernacular but also in formal and official contexts. The language variety is perceived as highly homogeneous, even if this need not be supported by the facts of actual usage. Finally, in phase 5 (\is{differentiation (Dynamic Model)}\textit{differentiation}), the new nation has become politically independent and self-reliant. There is no longer the need to demonstrate linguistic homogeneity, and, accordingly, internal patterns of variation in the speech community emerge more strongly, potentially leading to dialect birth. Identity construction is increasingly driven by social, rather than national factors.

\begin{sloppypar}
\citet[10, 16]{Schneider2014} is quite explicit about the fact that the \isi{Dynamic Model} was developed specifically for postcolonial Englishes, i.e. varieties found in \citegen{Kachru1988} \is{varieties of English!Inner Circle}Inner and \is{varieties of English!Outer Circle}Outer Circles. Testing his own model against English in East Asian expanding-circle contexts, he comes to the conclusion that the model is indeed of limited applicability there (\citealt{Schneider2014}: 27–28). Based on Schneider’s work, \citet{BuschfeldKautzsch2017} propose a more flexible and general model they call the model of \is{EIF model}\textit{Extra- and Intra-territorial Forces} (EIF), which is not discussed here.
\end{sloppypar}

\section{Summary}\label{sec:4.4}

This chapter has discussed three central topics that provide part of the background for the present study. The theoretical framework of Construction Grammar (CxG) was introduced, with a particular focus on its application to English concessives. The choice model proposed in this context will be instrumental in understanding the approach taken in the quantitative analyses, particularly in Chapters \ref{sec:9}–\ref{sec:11}. Further, two extralinguistic dimensions of variation were introduced:
(i)~the difference between speech and writing and
(ii)~the difference between geographical or national varieties of English. In sum, then, the present study treats English concessives as constructions whose different subconstructional levels are assumed to be interconnected and correlated, but which may also be subject to variation induced by different contexts of use. Showing those different aspects in combination will be the task of the analytic chapters.
