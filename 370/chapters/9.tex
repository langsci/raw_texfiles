\chapter{Clause position}\label{ch:9}\label{bkm:Ref3391695}\label{bkm:Ref4146297}\label{sec:9}

This chapter focuses on the relative frequencies of the two basic configurations of matrix and subordinate clause, “\is{final position}final” and “\is{nonfinal position}nonfinal”. The \is{nonfinal position}nonfinal category comprises sentences with subclauses in \is{medial position}medial and \isi{initial position} (cf. \sectref{sec:2.3.1}, \sectref{sec:5.1.3} and \sectref{sec:6.3.6}). Unlike the analyses in Chapters \ref{sec:7} \& \ref{sec:8}, the approach is not based on text frequency counts but on the inspection of \isi{variable contexts}, i.e. the choice that is made between the two positional variants each time a CC occurs. Results are presented as percentages.

The main expectations concerning the variation of clause positions are that \is{nonfinal position}nonfinal clause placement is somewhat more likely in \is{varieties of English!L1}L1 varieties, in written discourse and in connection with \is{concessives (types of)!anticausal}anticausal semantics. For the reasoning behind these assumptions, see \sectref{sec:5.3}, which also highlights the place of positional variation in the choice model of constructional variation for CCs (see \figref{fig:5.11}). Similar in general structure to the foregoing ones, this chapter will first present a minimal discussion of how the data were approached in the statistical model (\sectref{sec:9.1}), followed by the results in \sectref{sec:9.2}. Finally, \sectref{sec:9.3} focuses on a concluding summary and discussion of the quantitative findings.

\section{\label{bkm:Ref52571821}\label{bkm:Ref52609859}Statistical model}\label{sec:9.1}

Clause positions are classified as either \is{final position}final or \is{nonfinal position}nonfinal, and the outcome is thus a binary variable called \textsc{final}, with the reference level “nonfinal” comprising \is{initial position}initial and \is{medial position}medial positions (cf. \sectref{sec:2.3.1}; for a detailed definition of variables, see \tabref{tab:6.3} in \sectref{sec:6.3.6}).\footnote{Note that the (uncentred) outcome variable \textsc{final} is different from the (centred) predictor variable \textsc{final.ct} (cf. \tabref{tab:6.3}).} Accordingly, a binary \is{regression!logistic}logistic regression model was used to estimate the outcome. In line with the sequence of chapters, this is called “Model~C”; its structure is shown in (\ref{eq:9.1}). Again, one model was specified for each variety, resulting in a total of $n=9$ models. The interaction term for the predictors \textsc{spoken.ct} and \textsc{anti.ct} was included; further, slopes of \textsc{anti.ct} were specified as varying randomly across the two \isi{cluster variables} \textsc{genre} and \textsc{text}. Note that, as in all models in this study, there is no \is{cluster variables}cluster variable for variety, since the nine varieties were each assigned a separate model. The predictor \textsc{length.ct} is used as a control variable. It measures the logged \is{length (of clauses)}length (in words) of subordinate clauses, centred on the mean logged \is{length (of clauses)}length of all occurrences, across all varieties (see \sectref{sec:6.3.6}). This predictor will play no prominent role in the discussion of results.

\ea
\label{bkm:Ref41473527}\label{eq:9.1}Model C: Syntax
\begin{lstlisting}
final ~ spoken.ct * anti.ct + length.ct
        + (anti.ct | genre)
        + (anti.ct | text)
\end{lstlisting}
\z


More information can be found in Appendix~\ref{appendix:B.3}, e.g. concerning token numbers, the number of levels of both random factors (\textsc{genre} and \textsc{text}), the \isi{priors} (which were the same for all nine models) and the number of posterior samples. Data, scripts and detailed model summaries can be retrieved from the online repositories as outlined in \sectref{sec:1.4}.

\section{\label{bkm:Ref51055367}Results}\label{sec:9.2}

The three parts of this section present the results from two different perspectives, first averaging across varieties (\sectref{sec:9.2.1}), then inspecting the relationship between intra-constructional semantics on the position of the subordinate clause (\sectref{sec:9.2.2}). Both sections also take the spoken-written dimension into account.

Before embarking on the main analyses, a few words about the effect of clause \is{length (of clauses)}length (operationalised as the predictor \textsc{length.ct}) are in place. This predictor was used purely as a control variable: The effect of \is{length (of clauses)}length on the placement of syntactic elements (or on syntactic variation more generally) is well documented, and while the present study takes no theoretical interest in it, ignoring the effect would have been problematic. While the predictor \textsc{length.ct} is therefore included in all the models, it is held constant at its mean value. Values of this variable are positively correlated with the outcome \textsc{final}, as seen in \figref{fig:9.1}, which plots the coefficients (\isi{logits}) for all nine varieties, ordered by magnitude and including 50\% and 90\% \isi{uncertainty intervals}.\footnote{The coefficient signifies the change in the log odds of \isi{final position} corresponding to a unit change of \textsc{length.ct}, which in turn corresponds to a change in actual \is{length (of clauses)}length (measured in words) by factor $e \approx 2.72$.}  While the interpretability of these values is limited, they show that the effect is always positive, and sometimes substantially so.

\begin{figure}
\includegraphics{figures/CCs.Fig.9.1.pdf}
\caption{\label{bkm:Ref62285257}\label{fig:9.1}The coefficient \textsc{length.ct} in nine varieties}
\end{figure}

Compared to a preliminary model run without the predictor \textsc{length.ct}, the coefficients for the intercept, \textsc{spoken.ct} and \textsc{anti.ct} in the fixed part changed only very slightly, in the vast majority of cases by no more than ±0.04 on the log odds scale. Random coefficients tended to increase in the model that included \is{length (of clauses)}\textsc{length.ct} as a predictor. On closer inspection, it emerged that this predictor was neither correlated with any of the others, nor with the individual varieties. Using a reduced model without \textsc{length.ct} would therefore have yielded only marginally different results and would certainly not have affected the general conclusions.

\subsection{\label{bkm:Ref51143922}Average percentages}\label{sec:9.2.1}

The first approach to the positioning of subordinate concessive clauses relative to their associated matrix clauses was to establish average values for individual varieties. This perspective makes the perhaps questionable assumption of neutral values for \isi{mode of production} (intermediate between spoken and written) and semantics (intermediate between \is{concessives (types of)!anticausal}anticausal and \is{concessives (types of)!dialogic}dialogic), but it is nevertheless informative and provides an easy-to-grasp point of departure for subsequent analyses (see discussion in \sectref{sec:6.3.4}). \figref{fig:9.2} shows percentages of subordinate clauses in sentence-\isi{final position} for all nine varieties under investigation, arranged in ascending order. \is{varieties of English!L1}L1 varieties are shown in black, \is{varieties of English!L2}L2 varieties in grey; the central dashed line represents the 50\% mark, which can be used to assess whether or not the data encourage the conclusion that either \is{final position}final or \is{nonfinal position}nonfinal clause placement can actually be considered the majority variant.

\begin{figure}
\includegraphics{figures/CCs.Fig.9.2.pdf}
\caption{\label{bkm:Ref50750170}\label{fig:9.2}Average percentages of subordinate clauses in final position}
\end{figure}

Values range from 39.5\% of clauses in \isi{final position} in \il{Jamaican English}JamE to 55.9\% in \il{Canadian English}CanE. The three varieties on the left (\il{Jamaican English}JamE, \il{Indian English}IndE and \il{Hong Kong English}HKE) show a very clear preference of \is{nonfinal position}nonfinal placement; the other varieties are considerably closer to the 50\% value. Further, all \is{varieties of English!L2}L2 varieties prefer subordinate clauses in \isi{nonfinal position}, while \is{varieties of English!L1}L1 varieties prefer \is{final position}final placement. In \figref{fig:9.2}, this difference is emphasized not only by the left-to-right orientation of the two groups, but also by the indication of means for \is{varieties of English!L1}L1 and \is{varieties of English!L2}L2 varieties, based on the respective variety-specific averages: For \is{varieties of English!L2}L2 varieties, the average percentage of sentence-\is{final position}final subordinate clauses is 42.9\%, while for \is{varieties of English!L1}L1 varieties it is 54.6\%, a difference of 11.7 in absolute percentage points.\largerpage

As mentioned above, the effects of the \is{mode of production}mode of language production (spoken vs written) as well as the intra-constructional semantics of concessives are neutralised~– or controlled for~– in \figref{fig:9.2}. A leading question in the subsequent analyses will be whether or not the general difference between \is{varieties of English!L1}L1 and \is{varieties of English!L2}L2 varieties detected here unfolds into further differences concerning specific effects. In other words: Are the \is{varieties of English!L1}L1 and \is{varieties of English!L2}L2 varieties investigated here characterised by different general preferences concerning clause placement but otherwise affected similarly by differences in the \isi{mode of production} or the semantic type of a concessive, or do they also respond differently to those factors?

The first step towards a more nuanced assessment of differences in the preferred clause placement patterns is taken in \figref{fig:9.3}, which again orders varieties according to their average percentage of sentence-\is{final position}final subordinate clauses. However, each variety category in the lower panel is now subdivided into “spoken” and “written”, shown in grey and black, respectively. Once again, a line is drawn at the value of 50\% since values that depart more markedly from this reference point indicate that there is an actual preference in a given (spoken or written) variety. The upper panel shows estimates of absolute percentage-point differences between speech and writing in each variety.

\begin{figure}
\includegraphics{figures/CCs.Fig.9.3.pdf}
\caption{\label{bkm:Ref51140714}\label{fig:9.3}Sentence-final placement of subordinate clauses by mode of production}
\end{figure}

Percentages of subordinate clauses in \isi{final position} tend to be higher in spoken language in most varieties. The contrast~– written minus spoken~– takes a (rather small) positive value in only two varieties, \il{Singapore English}SingE and \il{British English}BrE. Differences between \is{mode of production}modes of production generally tend to be moderate and come with a high degree of \isi{uncertainty}~– observe, for instance, that the 90\% uncertainty intervals include the critical value of zero in all varieties except \il{Canadian English}CanE.\footnote{Readers are invited to inspect the regression tables that are published online (cf. \sectref{sec:1.4}), which show, for instance, that \il{Canadian English}CanE not only has a high \textsc{intercept} for final position~– reflected here in its position at the very right of the figure~– but also an exceptionally high coefficient for \textsc{spoken.ct}. Consulting the supplementary materials in this way is generally recommended to readers who do not wish to rely exclusively on the visualisations.} If we inspect the average differences between speech and writing across subsets of varieties, we see that \is{varieties of English!L1}L1 and \is{varieties of English!L2}L2 varieties behave similarly in this regard, even if their overall percentages differ substantially (as shown in \figref{fig:9.2}): Compared to writing, the mean share of subordinate clauses in \isi{final position} in spoken \is{varieties of English!L2}L2 varieties is on average higher by 5.0 percentage points; in \is{varieties of English!L1}L1 varieties, this average difference is 8.1 absolute percentage points, as summarised in \tabref{tab:9.1}.\footnote{Note that only the mean values themselves are directly derived from the model-based estimates. Differences in the table are then calculated on this basis to present a consistent picture: If, for the rightmost column, variety-specific mean differences between speech and writing were estimated and then averaged for each group, slight and uncritical (but potentially confusing) discrepancies might arise. This also applies to the corresponding tables in Chapters \ref{sec:10} \& \ref{sec:11}.\label{fn87}}\largerpage

\begin{table}
\caption{\label{bkm:Ref51144092}\label{tab:9.1}Sentence-final placement of subordinate clauses in speech and writing by variety type (mean \%)}
\begin{tabular}{l *3{S[table-format=-2.1]}}
\lsptoprule
& {written} & {spoken} & {written\,$-$\,spoken}\\\midrule
\is{varieties of English!L2}L2 & 40.4 & 45.4 & -5.0\\
\is{varieties of English!L1}L1 & 50.6 & 58.7 & -8.1\\
\is{varieties of English!L2}L2\,$-$\,\is{varieties of English!L1}L1 & -10.2 & -13.3 & \\
\lspbottomrule
\end{tabular}
\end{table}

It follows that \tabref{tab:9.1} also shows that the general difference between \is{varieties of English!L2}L2 and \is{varieties of English!L1}L1 varieties~– with the former characterised by fewer subordinate clauses in sentence-\isi{final position}~– persists in both \is{mode of production}modes of production. From the bird’s-eye perspective discussed above, this difference was 11.7 absolute percentage points. In speech and writing, it is relatively similar, at 13.3 and 10.2 percentage points, respectively.

\subsection{\label{bkm:Ref59129356}Semantics}\label{sec:9.2.2}

In an approach strictly analogous to the one taken in the previous section, the lower part of \figref{fig:9.4} isolates the effect of the intra-constructional semantics of a CC~– with \is{concessives (types of)!dialogic}dialogic and \is{concessives (types of)!anticausal}anticausal types shown in grey and black, respectively~– on the placement of subordinate clauses. In the upper panel, the differences between the two conditions are shown. The same ranking of varieties as in \sectref{sec:9.2.1} is applied. Once again, lines of reference are drawn at 50\% in the lower panel and at zero in the upper panel.

\begin{figure}
\includegraphics{figures/CCs.Fig.9.4.pdf}
\caption{\label{bkm:Ref59219422}\label{fig:9.4}Sentence-final placement of subordinate clauses by semantic type}
\end{figure}

The relationship between clause positions and semantic types does not appear to be systematic. In four varieties (\il{Jamaican English}JamE, \il{Hong Kong English}HKE, \il{Singapore English}SingE and \il{Canadian English}CanE), there is a tendency for \is{concessives (types of)!anticausal}anticausal semantics to be associated with higher percentages of concessive clauses in \isi{final position}, although this effect is rather small in \il{Jamaican English}JamE; in four varieties (\il{Indian English}IndE, \il{Irish English}IrE, \il{British English}BrE and \il{Australian English}AusE), the inverse pattern obtains, with relatively weak or uncertain patterns in \il{Indian English}IndE and \il{Irish English}IrE; and in \il{Nigerian English}NigE there is virtually no difference, the expected percentage of subclauses in \isi{final position} being lower by a mere 0.23 percentage points for \is{concessives (types of)!anticausal}anticausal CCs. Overall, there is no clear, general tendency with a few exceptions (as in the analysis of speech vs writing above) but a mix of indifferent or even conflicting patterns.

In the following paragraphs, the relationship between semantics and clause positions will be explored in more detail by including the spoken-written dimension as a superordinate level of variation~– in other words, the interaction of the predictors \textsc{spoken.ct} and \textsc{anti.ct} is taken into account. Results are shown in \figref{fig:9.5}, which requires a few words of introduction. For each of the nine varieties, there is one component plot with two panels (lower and upper). Varieties are no longer ordered by median percentages but according to the sequence introduced in \sectref{sec:4.3} and \sectref{sec:6.1} (cf. \tabref{tab:4.1} and \figref{fig:6.1}). The lower panel of each plot shows the estimated percentages of subordinate clauses in \isi{final position} in the four possible conditions (2 modes~× 2 semantics), while the upper panels show the difference between \is{concessives (types of)!anticausal}anticausal and \is{concessives (types of)!dialogic}dialogic semantics in speech and writing.

\begin{figure}
\includegraphics{figures/CCs.Fig.9.5.pdf}
\caption{\label{bkm:Ref51165021}\label{fig:9.5}Clause position: The interaction of mode and semantics; a~= anticausal, d~= dialogic}
\end{figure}

The very first component plot (representing \il{British English}BrE) shows the level of detail that may be revealed by including the interaction between \isi{mode of production} and semantics: In speech, \is{concessives (types of)!anticausal}anticausal concessives are considerably less likely to be constructed with a sentence-\is{final position}final subordinate clause ($-$21.5 absolute percentage points), while in writing there is a less pronounced tendency in the opposite direction (+7.4 percentage points). There are multiple strategies for reading \figref{fig:9.5}. Focusing on the plots of differences (i.e. the upper panels in each subplot), point estimates below the dashed line signal that, compared to \is{concessives (types of)!anticausal}anticausal CCs, \is{concessives (types of)!dialogic}dialogic CCs have a higher percentage of subordinate clauses in \isi{final position}. This is the expected outcome (cf. \sectref{sec:5.3}) on the assumption that clause arrangements should be \is{iconicity}iconic of the underlying \textsc{if→then} relation in \is{concessives (types of)!anticausal}anticausal CCs, but should follow patterns that are easier to parse in \is{concessives (types of)!dialogic}dialogic CCs. On the other hand, if the point estimate is above zero, a higher percentage of subordinate clauses in \isi{final position} is associated with \is{concessives (types of)!anticausal}anticausal CCs, which is contra expectations. Secondly, if the connecting line in a plot of difference (upper panels) has a relatively flat slope, or is even parallel to the $x$-axis, the effect of intra-constructional semantics on clause positions is similar in speech and writing. If there is a marked difference, i.e. if the connecting line slopes steeply, the semantic effect differs substantially between \is{mode of production}modes of production.

The plot confirms what the previous section has shown, albeit in more detail: The response of clause arrangement to the semantic predictor is relatively erratic and unsystematic. Additionally, there is often a difference in the semantic effect between the two \is{mode of production}modes of production that is equally surprising and difficult to explain. The unexpected result found in \il{British English}BrE was already discussed above, with the expected pattern in speech but no effect in writing; \il{Irish English}IrE has an unexpected effect in speech and virtually no effect in writing; \il{Canadian English}CanE is characterised by tendencies (in both modes of production) whose directions run counter to hypotheses; \il{Australian English}AusE seems to conform to the hypothesised patterns and is stable across both modes of production, even if the effects are not particularly large; in \il{Jamaican English}JamE, there is virtually no effect in speech but an unexpected tendency in writing; in \il{Nigerian English}NigE, there is a tendency in the expected direction in spoken language, but its complete reversal in writing; \il{Indian English}IndE shows virtually no semantic effect, irrespective of the \isi{mode of production}; both \il{Singapore English}SingE and \il{Hong Kong English}HKE have effects in the expected direction, but their strength varies between speech and writing in an inconsistent way. In short: There is no evidence to support the idea of an \is{iconicity}iconic arrangement of matrix and subordinate clause relative to each other, and the balance of evidence and counter-evidence discourages further interpretation. Clause positions thus cannot be explained using the model that was proposed for this chapter. I will return to this finding in the concluding section and offer a few suggestions for future approaches to the issue.

\figref{fig:9.6} is similar in design to \figref{fig:7.5} and provides a final visual summary for this chapter. In its main panel, it does not contain information that goes beyond what was shown in \figref{fig:9.5} above but arranges values in a different, cross-varietal manner (cf. Figures \ref{fig:7.5}–\ref{fig:7.7}). Ranked percentages of subordinate clauses in \isi{final position} are plotted for all $n=36$ possible conditions (9~varieties × 2~modes of production × 2~semantics). The black, white and grey boxes to the right of the figure highlight structure in the data. In the first three columns, black squares denote \is{varieties of English!L1}L1 varieties, written language and \is{concessives (types of)!anticausal}anticausal semantics, respectively; conversely, white squares denote \is{varieties of English!L2}L2 varieties, spoken language and \is{concessives (types of)!dialogic}dialogic semantics. In the fourth column, three categories are established, defined by the interaction of mode and semantics (cf. \figref{fig:9.5}). Additionally, the mean ranks of groups of conditions~– “black” vs “white” (vs “grey”)~– are indicated in each column by triangles that jut out to the left and right. Darker colours in the right-hand part of the figure correspond to conditions that should favour the \is{nonfinal position}nonfinal placement of subordinate clauses (\is{varieties of English!L1}L1 varieties, writing and \is{concessives (types of)!anticausal}anticausal semantics), according to the hypotheses. A concentration of darker shades nearer the bottom of each column (and corresponding mean ranks) would therefore signal agreement between results and expectations.

\begin{figure}
\includegraphics{figures/CCs.Fig.9.6.pdf}
\caption{\label{bkm:Ref51234537}\label{fig:9.6}Clause position: Ranking of specific conditions; W~= written, S~= spoken, a~= anticausal, d~= dialogic}
\end{figure}

The reversal of the expected varieties-based pattern is clearly visible in the clustering of black squares towards the top of the first column~in the right-hand part of \figref{fig:9.6} (mean rank: 11.6) and the higher concentration of \is{varieties of English!L2}L2 varieties towards the bottom (mean rank: 24.1). The general effect of \isi{mode of production} is as expected: White squares representing speech in column two are on average nearer the top (mean rank: 15.1) than black squares representing writing (mean rank: 21.9), but there is a high degree of overlap between the two sets of specific conditions. Concerning the difference between \is{concessives (types of)!anticausal}anticausal and \is{concessives (types of)!dialogic}dialogic CCs, \figref{fig:9.6}~– like \figref{fig:9.4} above~– shows that it has no systematic impact on the positioning of clauses, as the mean ranks for both types are very close to each other (17.4 and 19.6, respectively). Finally, the pattern seen in the column at the very right of \figref{fig:9.6} highlights once again that hypotheses concerning the sequencing of clauses are not, or only very partially, supported. While \is{concessives (types of)!dialogic}dialogic CCs in speech are indeed most likely to be associated with subordinate clauses in \isi{final position} (mean rank: 13.8), the intermediate combinations of factors (spoken \is{concessives (types of)!anticausal}anticausal and written \is{concessives (types of)!dialogic}dialogic) have a mean rank of 20.9, which is lower than for the hypothetically most strongly disfavouring combination, written \is{concessives (types of)!anticausal}anticausal (mean rank: 18.3).

There are thus many patterns that are unsystematic (or noisy) or even run counter to the hypotheses that were set up to account for alternating clause positions. Except for the difference between speech and writing, results match poorly with theory. At present, it can therefore only be concluded that
(i)~the theoretical assumptions for this part of the investigation may not be adequate, particularly concerning the \isi{iconicity} principle, or that
(ii)~there are other (and potentially stronger) factors at work that were not operationalised for this study. One such candidate factor will be discussed in the following final section of this chapter.

\section{\label{bkm:Ref51055521}Summary and discussion}\label{sec:9.3}

The investigation of factors that correlate with the \is{final position}final or \is{nonfinal position}nonfinal placement of subordinate clauses in CCs yielded mainly three results:
(i)~Sentence-\isi{final position} is more common in \is{varieties of English!L1}L1 varieties than in \is{varieties of English!L2}L2 varieties,
(ii)~spoken language correlates with subordinate clauses in \isi{final position}, and
(iii)~there is no systematic general link between the semantic relation that holds within a CC and the arrangement of its component clauses relative to each other. Only the second finding is in support of the corresponding hypothesis formulated in \sectref{sec:5.3}. In the following paragraphs, I will briefly discuss these rather ambivalent results and speculate as to the reasons for their lack of coherence. Ultimately, I will argue that clause position as an outcome variable is inherently problematic, at least in an analytic design that focuses on hermetic constructions and ignores the wider discourse context.

Concerning the link between types of varieties (\is{varieties of English!L1}L1 vs \is{varieties of English!L2}L2) and clause placement, the \is{initial position}initial hypothesis was that it would be \is{varieties of English!L2}L2 varieties that favour subclauses in \isi{final position}. As discussed in \sectref{sec:2.3.1}, theories connected to \isi{production} and \isi{parsing} suggest that \is{final position}final placement is cognitively the optimal configuration. In \is{varieties of English!L2}L2 varieties, the role of English will on average be somewhat less secure, compared to \is{varieties of English!L1}L1 varieties, and its share in everyday language use will be smaller. Under such conditions, it was argued, the selection of cognitively less complex (or more “natural”) patterns would be more likely. However, the opposite seems to be the case in the data at hand, as it is the \is{varieties of English!L1}L1 varieties that are characterised by more subordinate clauses in \isi{final position}. Cognitive mechanisms in language \isi{production} and \isi{processing} cannot provide an explanation. It is tempting to resort to a post-hoc inversion of the hypothesis. For instance, one could build a two-stage argument based on the assumption that concessive subordinate clauses in \isi{initial position} are more frequent in \is{varieties of English!L2}L2 varieties due to the way in which language is acquired:
(i)~Many standard grammars of the language tend to focus on what may be seen as prototypical CCs, i.e. \is{concessives (types of)!anticausal}anticausal semantics with a preposed subordinate clause, and
(ii)~the \is{language acquisition}acquisition of \is{varieties of English!L2}L2 Englishes may be viewed as more “scholastic”, i.e. happening  to a much greater extent in formal school settings, which depend on input from such grammar books and derived materials. \is{varieties of English!L1}L1 Englishes, on the other hand, could then be viewed as more emancipated from what is codified in grammars, which would allow it to follow the more “natural” tendencies predicted from a \isi{production} or \isi{parsing} perspective.\footnote{My observation concerning the predominance in grammars of \is{concessives (types of)!anticausal}anticausal CCs with preposed subordinate clauses is largely impressionistic and has not been tested systematically. This further undermines the alternative hypothesis, in addition to the fact that it is post hoc.} However, exploring this alternative set of hypotheses would require a new, independent research effort, probably incorporating theories of \isi{language acquisition}, an inspection of teaching materials and practices, and possibly \is{experimental approaches}experimental techniques.

The second result concerns the relationship between modes of \isi{production} and the placement of clauses. There is a fairly consistent tendency in the data for spoken language to favour subordinate clauses in \isi{final position}, even if the effect is not particularly strong (with the exception of \il{Canadian English}CanE). This finding is in line with the hypothesis outlined in \sectref{sec:5.3}: From a \isi{production}-and-\isi{processing} perspective, \is{final position}final placement was considered to be cognitively less demanding than \is{nonfinal position}nonfinal placement and was therefore expected to be favoured even more strongly when the linguistic signal is purely acoustic and thus transient. The finding that speech tends to favour \is{final position}final clause placement also agrees with \citegen{Altenberg1986} results.

Thirdly and finally, the association between semantics and clause position in this study does not follow a clear and interpretable pattern. There is no support for the hypothesis that the arrangement of clauses in \is{concessives (types of)!anticausal}anticausal CCs should be \is{iconicity}iconic of the intra-constructional semantic relation between propositions (again, see \sectref{sec:2.3.1} and \sectref{sec:5.3}). Individual patterns that confirm the hypothesis co-occur with patterns that run counter to it, so that the overall picture is very difficult to interpret.

In view of the results presented in this chapter, there remains a feeling of unease with the treatment of clause position as an outcome variable. One obvious general conclusion could be that the factors operationalised for the analysis are not the centrally important ones. In other words, they may at the very least be obfuscated, if not outright overridden, by other determinants not even considered here. One such factor with a potentially strong effect on the sequencing of matrix and subordinate clauses is information-structural in nature. This means that the particular arrangement of clauses in a CC depends on which proposition SP/W wishes to place in focus position in order to give the sentence as a whole a specific \isi{theme-rheme} (or \isi{topic-comment}) structure. This decision, it can be assumed, will be partly \is{subjectivity}subjective but probably to a larger part determined by the wider discourse context and the pragmatic function of the entire CC within it. In other words, the conditioning factor may in this case be external to the construction itself, at least as defined in the present study. Conceivably, such information-structural mechanisms may be stronger than (and, of course, independent of) factors related to \isi{production} and \isi{processing}, or the \is{iconicity}iconic relationship between syntactic and semantic structures. Thus, in order to understand better why a particular clause arrangement is selected, we may need to look beyond the CC and inspect its discourse function relative to what follows and goes before. If it is a rather taxing exercise to classify CCs as belonging to one of the categories established for this study, operationalising the wider discourse context would be even more involved. As discussed in \sectref{sec:1.1}, a discourse-analytic approach was explicitly not taken in this study, and the decision to conduct analyses entirely at the level of the CC itself was made to enable the quantitative approach.

Thus, as far as positional variation is concerned, the success of the analysis is limited. Disappointing though this may be, the findings from this chapter are in fact valuable pointers for future research on concessives and perhaps other types of \isi{adverbials}. Crucially, while the analyses in this chapter provide only limited insights into clause position as an outcome variable, this does not automatically disqualify it as a predictor variable for subsequent analyses. The \is{constructional choice model}choice model introduced in \sectref{sec:4.1.3} (see \figref{fig:4.2} there) is rather tolerant regarding explanatory gaps: We can accept that our understanding of why SP/W selects a certain clause arrangement remains limited, perhaps because we have given insufficient consideration to additional predictor variables, or because variation is to a large extent unsystematic. In spite of this, we can still use clause position (along with \isi{mode of production} and semantics) as a predictor in subsequent stages of the study.
