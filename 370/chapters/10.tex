\chapter{Choice of conjunction}\label{bkm:Ref34987799}\label{ch:10}\label{sec:10}

In this chapter, the focus lies on factors that affect the choice between the three conjunctions \SchuetzlerIndexExpression{although}, \SchuetzlerIndexExpression{though} and \SchuetzlerIndexExpression{even though}. This perspective could be argued to take centre stage in the study as a whole, since specific conjunctions constitute concrete morphological forms and are therefore perhaps more immediately noticeable (or salient) in a CC than, for example, semantic structures or clause positions~– they are, after all, the connecting devices upon which a CC hinges. Predictors used at this stage of the analysis follow from the \is{constructional choice model}choice model presented in \figref{fig:4.2} (see \sectref{sec:4.1.3}) and include the mode of production, the semantic type of a CC and the internal arrangement of clauses. Section \ref{sec:10.1} defines the statistical model used in this chapter, followed by a presentation of results in \sectref{sec:10.2}. The summary in \sectref{sec:10.3} reflects upon these results against the background of the expectations formulated in \sectref{sec:5.3}.

\section{\label{bkm:Ref59515395}Statistical model}\label{sec:10.1}

The outcome variable \textsc{marker} takes three values: \SchuetzlerIndexExpression{although} (the reference category), \textit{though} and \textit{even though}. Thus, “Model~D”~– shown in (\ref{eq:10.1})~– is a \is{regression!multinomial}multinomial \is{regression!mixed-effects}mixed-effects model (cf. \sectref{sec:6.3.3.3}). The two fixed-effects terms \textsc{anti.ct} and \textsc{final.ct} interact with \textsc{spoken.ct} but not with each other. That is, the effects of semantics and clause position on the selection of the marker may differ between speech and writing but are treated as independent of each other. Both \textsc{anti.ct} and \textsc{final.ct} vary randomly across the two grouping factors \textsc{genre} and \textsc{text}, which are the same as in Model C (see \sectref{sec:9.1} above). As in all other chapters, a separate model of the same syntax was fitted for each variety.

\ea\label{bkm:Ref41508745}\label{eq:10.1}Model D: Syntax
\begin{lstlisting}
marker ~ spoken.ct * (anti.ct + final.ct)
         + (anti.ct + final.ct | genre)
         + (anti.ct + final.ct | text)
\end{lstlisting}
\z

Appendix~\ref{appendix:B.4} contains information regarding token numbers, the number of levels of both random variables (\textsc{genre} and \textsc{text}), the \isi{priors} (constant across all nine models) as well as the number of posterior samples. For data, scripts and model summaries (i.e. tables with regression coefficients), see the online repositories (cf. \sectref{sec:1.4}).

\section{\label{bkm:Ref59515477}Results}\label{sec:10.2}

Results are presented in four sections. The first three of these (\sectref{sec:10.2.1}--\ref{sec:10.2.3}) take the following perspectives:
(i)~a global one in which the effects of both semantics and clause position are controlled for (cf. \sectref{sec:6.3.4}),
(ii)~one in which the focus lies on the effects of the two semantic types (controlling for positional effects), and
(iii)~one in which the focus lies on positional effects (controlling for semantic effects). In each case, a hypothetical, average scenario (poised between \is{spoken language}speech and \is{written language}writing) is given first, followed by one that takes the effects of \isi{mode of production} into account. Finally, \sectref{sec:10.2.4} documents the full, most detailed range of results by showing for each conjunction the \textit{n}~=~72 specific conditions that affect its probability of occurrence.

\subsection{\label{bkm:Ref59530971}\label{bkm:Ref59994510}\label{bkm:Ref60080151}\label{bkm:Ref60239585}\label{bkm:Ref60241218}\label{bkm:Ref60417331}Average percentages}\label{sec:10.2.1}

As in the foregoing chapters, the first perspective on the outcome~– in this case the estimated percentages of the three concessive conjunctions~– is based on global averages in the nine varieties under investigation. As pointed out earlier, the approach is hypothetical in suggesting that semantics can be indeterminate between \is{concessives (types of)!anticausal}anticausal and \is{concessives (types of)!dialogic}dialogic, and that clause positions can be indeterminate between \is{final position}final and \is{nonfinal position}nonfinal. However, showing all factors in combination and thus applying no generalisation and simplification would seriously hamper the understanding of individual effects.

\figref{fig:10.1} displays average percentages of the three conjunctions \SchuetzlerIndexExpression{although}, \SchuetzlerIndexExpression{though} and \SchuetzlerIndexExpression{even though} in the nine varieties from a global perspective. The arrangement of varieties follows the sequence in \figref{fig:6.1} (and \tabref{tab:4.1}); the horizontal arrangement of conjunctions in each panel is in accordance with the order in which the markers were introduced in the theoretical part. Connecting lines are added to facilitate the direct comparison of patterns.

\begin{figure}
\includegraphics{figures/CCs.Fig.10.1.pdf}
\caption{\label{bkm:Ref59787429}\label{fig:10.1}Average percentages of conjunctions; A~= \textit{although}, T~= \SchuetzlerIndexExpression{though}, E~= \SchuetzlerIndexExpression{even though}}
\end{figure}

Typically, varieties are characterised by a “hockey-stick” pattern: \SchuetzlerIndexExpression[although]{Although} is expected to be most commonly selected ($M_A=47.4\%$), while values for \SchuetzlerIndexExpression{though} and \SchuetzlerIndexExpression{even though} are much lower and roughly on the same level ($M_T=25.0\%$ and $M_E=27.5\%$, respectively). There is considerable variation between varieties, however: For \SchuetzlerIndexExpression{although}, extreme values are found at 58.6\% in \il{British English}BrE and 25.2\% in \il{Indian English}IndE; for \SchuetzlerIndexExpression{though}, the range of values is between 57.6\% in \il{Indian English}IndE and 13.5\% in \il{Canadian English}CanE; and for \SchuetzlerIndexExpression{even though}, extremes are at 36.3\% in \il{Nigerian English}NigE and 17.1\% in \il{Indian English}IndE. The most striking patterns are found in \il{Nigerian English}NigE, which does not seem to give precedence to any of the three markers, as well as in \il{Indian English}IndE with its remarkably high value for \SchuetzlerIndexExpression{though} (and, accordingly, a low value for \SchuetzlerIndexExpression{although}). Both patterns can also be seen in \figref{fig:7.3}, although this is based entirely on \isi{text frequency} and does not control for semantics and clause position. Among the \is{varieties of English!L1}L1 varieties, \il{Canadian English}CanE stands out slightly in using a relatively high percentage of \SchuetzlerIndexExpression{even though}. Compared to the purely count-based analysis in \chapref{sec:7}, the conjunction \SchuetzlerIndexExpression{even though} has considerably greater weight in \figref{fig:10.1}, appearing in second place (after \SchuetzlerIndexExpression{although}) in six out of the nine varieties and even being the preferred marker in \il{Nigerian English}NigE. As will be shown below, there are two main reasons for this:
(i)~There is a strong positive correlation between \is{concessives (types of)!anticausal}anticausal semantics and the use of \SchuetzlerIndexExpression{even though}, and
(ii)~\is{concessives (types of)!anticausal}anticausal semantics are on the whole considerably less common than \is{concessives (types of)!dialogic}dialogic semantics. Since the initial analysis in this chapter assumes a balance between the two semantic types, the estimated percentages will be positively biased for \SchuetzlerIndexExpression{even though} and negatively biased for \SchuetzlerIndexExpression{although}, compared to actual rates of occurrence. Unfolding this general picture into a perspective that does consider semantics as a factor is therefore all the more important, as will be shown in \sectref{sec:10.2.2}.\largerpage

Some evidence of a general difference between \is{varieties of English!L1}L1 and \is{varieties of English!L2}L2 varieties is produced by the inspection of the mean percentages of the three conjunctions for those two subgroups. On average, \SchuetzlerIndexExpression{although} occurs 54.2\% of the time in \is{varieties of English!L1}L1 varieties, but only 41.9\% of the time in \is{varieties of English!L2}L2 varieties. The respective values for \SchuetzlerIndexExpression{though} are 17.7\% (\is{varieties of English!L1}L1) and 30.9\% (\is{varieties of English!L2}L2), while for \SchuetzlerIndexExpression{even though}, mean percentages in both subgroups are very similar (\is{varieties of English!L1}L1: 28.0\%; \is{varieties of English!L2}L2: 27.1\%). It is tempting to make conjectures concerning possible explanations of this pattern (e.g. from \isi{grammaticalisation} theory), and some such notions will be touched upon in the concluding part of this chapter (\sectref{sec:10.3}), but we need to bear in mind that much of the difference between \SchuetzlerIndexExpression{although} and \SchuetzlerIndexExpression{though} is due to the rather idiosyncratic and extreme behaviour of a single variety, \il{Indian English}IndE. Thus, it seems risky to make even tentative generalisations.

The global perspective in \figref{fig:10.1} becomes more nuanced in \figref{fig:10.2}, which is arranged along the same general lines but compares separate percentages of markers for spoken and written \is{genre}genres. Accordingly, there are two sets of values in the lower panel of each variety-specific subplot, rendered in grey and black and internally connected with lines to facilitate the recognition of patterns. In the panels above the percentage plots, differences between \is{written language}writing and \is{spoken language}speech (in absolute percentage points) are plotted for each conjunction. The relevant reference value of zero (that is,~“no difference”) is highlighted by a dashed line. In the discussion of tendencies for the individual conjunctions, varieties are ordered according to \is{effect size}effect sizes, but note that large effects may also come with high degrees of \isi{uncertainty}, as indicated in the text and visible in \figref{fig:10.2}.

\begin{figure}
\includegraphics{figures/CCs.Fig.10.2.pdf}
\caption{\label{bkm:Ref59897259}\label{fig:10.2}Average percentages of conjunctions in \is{spoken language}speech and \is{written language}writing; A~= \SchuetzlerIndexExpression{although}, T~= \SchuetzlerIndexExpression{though}, E~= \SchuetzlerIndexExpression{even though}, W~= written, S~= spoken}
\end{figure}

The conjunction \SchuetzlerIndexExpression{although} tends to be selected more often in \is{written language}written language. There are patterns of this kind in \il{Canadian English}CanE (\textit{D}~=~16.8 [5.3; 29.1]), \il{Australian English}AusE (\textit{D}~=~12.3 [1.0; 24.1]), \il{Indian English}IndE (\textit{D}~=~9.1 [$-$0.4; 19.0]), \il{Nigerian English}NigE (\textit{D}~=~7.2 [$-$5.9; 19.6]), \il{Irish English}IrE (\textit{D}~=~6.2 [$-$6.2; 19.0]), \il{Jamaican English}JamE (\textit{D}~=~3.1 [$-$11.3; 17.9]), \il{Hong Kong English}HKE (\textit{D}~=~2.2 [$-$9.8; 13.7]) and \il{Singapore English}SingE (\textit{D}~=~1.1 [$-$14.7; 16.1]). Only in \il{British English}BrE is the tendency reversed, with an extremely small increase in the percentage of \SchuetzlerIndexExpression{although} in \is{spoken language}speech (\textit{D}~=~$-$0.8 [$-$12.8; 11.4]). That is, in eight out of the nine varieties under investigation there is a tendency for \SchuetzlerIndexExpression{although} to be more frequent in \is{written language}writing. However, based on the \isi{uncertainty intervals} shown in \figref{fig:10.2} we can speak of a more robust effect in only two of them, \il{Canadian English}CanE and \il{Australian English}AusE, perhaps with the addition of \il{Indian English}IndE.

The conjunction \SchuetzlerIndexExpression{though} is also generally more likely to be selected in \is{written language}writing, compared to \is{spoken language}speech, namely in \il{Hong Kong English}HKE (\textit{D}~=~11.3 [0.4; 21.7]), \il{Jamaican English}JamE (\textit{D}~=~9.4 [$-$0.3; 19.6]), \il{Canadian English}CanE (\textit{D}~=~7.6 [0; 15.4]), \il{Singapore English}SingE (\textit{D}~=~6.9 [$-$6.8; 20.0]), \il{Australian English}AusE (\textit{D}~=~6.1 [$-$3.4; 15.0]), \il{Irish English}IrE (\textit{D}~=~3.0 [$-$5.6; 11.1]) and \il{Nigerian English}NigE (\textit{D}~=~1.4 [$-$12.8; 15.4]). A slight reversal is once again found in \il{British English}BrE, i.e. in this variety \SchuetzlerIndexExpression{though} tends to be selected more often in \isi{spoken language} (\textit{D}~=~$-$1.4 [$-$12.1; 8.6]). A more substantial preference of this conjunction in \is{spoken language}speech is found in \il{Indian English}IndE (\textit{D}~=~$-$7.5 [$-$19.4; 4.3]). Not unlike \SchuetzlerIndexExpression{although}, percentages of the conjunction \SchuetzlerIndexExpression{though} are higher in \is{written language}writing in seven out of the nine varieties. However, the effect is substantially different from zero in only two of them, \il{Hong Kong English}HKE and \il{Canadian English}CanE.

In marked contrast to the other two conjunctions, \SchuetzlerIndexExpression{even though} is more common in \is{spoken language}speech. This is the case in \il{Canadian English}CanE (\textit{D}~=~$-$24.4 [$-$36.6; $-$12.9]), \il{Australian English}AusE ($D = -18.3$ $[-29.8;\allowbreak -7.1]$), \il{Jamaican English}JamE (\textit{D}~=~$-$12.6~[$-$26.5; 1.2]), \il{Hong Kong English}HKE (\textit{D}~=~$-$12.2 [$-$24.5; $-$2.3]), \il{Irish English}IrE (\textit{D}~= $-$9.1~[$-$21.5; 2.7]), \il{Nigerian English}NigE (\textit{D}~=~$-$8.3~[$-$22.8; 5.8]), \il{Singapore English}SingE (\textit{D}~=~$-$7.9~[$-$23.4; 7.8]) and \il{Indian English}IndE (\textit{D}~=~$-$1.5 [$-$10.3; 6.9]). It is only in \il{British English}BrE that we find an effect in the opposite direction (\textit{D~}= 2.4~[$-$7.8; 12.1]). Thus, from a general, cross-varietal perspective, eight varieties conform to the mainstream tendency for \SchuetzlerIndexExpression{even though} to be more frequent in \is{spoken language}speech relative to the other two conjunctions.\largerpage

Finally, the general differences between the three conjunctions are captured if we average the written-spoken differences across all nine varieties: For \SchuetzlerIndexExpression{although}, the mean difference in absolute percentage points between \is{written language}writing and \is{spoken language}speech is +6.4; for \SchuetzlerIndexExpression{though}, the mean difference is +4.1; and for \SchuetzlerIndexExpression{even though} it is $-$10.3. While exceptions do of course exist, we can tentatively conclude that \SchuetzlerIndexExpression{although} and \SchuetzlerIndexExpression{even though} are most sensitive to differences in \isi{mode of production}, and by extension perhaps also to \is{style}stylistic differences more generally (cf. \sectref{sec:4.2}).

\tabref{tab:10.1} summarises the global differences between \is{varieties of English!L1}L1 and \is{varieties of English!L2}L2 varieties concerning the effect of \isi{mode of production} on the selection of subordinators. It is organised so as to show, for each subset of varieties, the mean percentage of each conjunction in written and in spoken discourse, as well as the difference between those means (written minus spoken) in absolute percentage points (see comment in Footnote~\ref{fn87} on p. \pageref{fn87} regarding this as well as Tables \ref{tab:10.2}–\ref{tab:10.5} below).

\begin{table}
\caption{\label{bkm:Ref60066185}\label{tab:10.1}Mean percentages of conjunctions in \is{written language}written and \is{spoken language}spoken L1 and L2 varieties}
\begin{tabular}{ll *3{S[table-format=-2.1]}}
\lsptoprule
&  & {\itshape \SchuetzlerIndexExpression{although}} & {\itshape \SchuetzlerIndexExpression{though}} & {\itshape \SchuetzlerIndexExpression{even though}}\\\midrule
\is{varieties of English!L1}L1 & \is{written language}written & 58.5 & 19.5 & 21.7\\
& spoken & 49.8 & 15.6 & 34.2\\
& \is{written language}written\,$-$\,spoken & +8.7 & +3.9 & -12.5\\\midrule
\is{varieties of English!L2}L2 & \is{written language}written & 44.1 & 33.0 & 22.6\\
& spoken & 39.6 & 28.5 & 31.4\\
& \is{written language}written\,$-$\,spoken & +4.5 & +4.5 & -8.8\\
\lspbottomrule
\end{tabular}
\end{table}

With regard to \SchuetzlerIndexExpression{although} and \SchuetzlerIndexExpression{even though}, \is{varieties of English!L1}L1 varieties are on average characterised by a larger percentage-point difference between \is{written language}writing and \is{spoken language}speech, in a positive direction for \SchuetzlerIndexExpression{although} and in a negative direction for \SchuetzlerIndexExpression{even though}. For \SchuetzlerIndexExpression{though}, the difference between variety types is much smaller. Due to the small number of varieties included in this study, we cannot draw very strong conclusions based on this finding. It is, however, in agreement with the idea that, unlike \is{varieties of English!L2}L2 varieties, \is{varieties of English!L1}L1 varieties have progressed to the \is{differentiation (Dynamic Model)}differentiation stage in \citegen{Schneider2003} \isi{Dynamic Model} (see \sectref{sec:4.3.2}): At a very general level, the greater similarity of percentage patterns in \is{written language}written and \is{spoken language}spoken \is{varieties of English!L2}L2 varieties may suggest that these two (admittedly very broad) \is{style}stylistic categories are formally not differentiated to the same extent as in \is{varieties of English!L1}L1 varieties. We will return to this idea in the conclusion to this chapter.

\subsection{\label{bkm:Ref59897180}Semantics}\label{sec:10.2.2}

In an approach analogous to the one taken in \figref{fig:10.2} in the previous section, the lower panels of \figref{fig:10.3} isolate the correlation of the two semantic categories~– \is{concessives (types of)!dialogic}dialogic and \is{concessives (types of)!anticausal}anticausal (shown in grey and black, respectively)~– with the selection of conjunctions. In the upper panels, differences (in absolute percentage points) between the two conditions are shown, subtracting estimated percentages in \is{concessives (types of)!dialogic}dialogic CCs from estimated percentages in \is{concessives (types of)!anticausal}anticausal CCs. Once again, dashed lines of reference are drawn at the value of zero (denoting “no difference”) in the upper panels.

In \is{concessives (types of)!dialogic}dialogic CCs, the typical ranking of conjunctions is \SchuetzlerIndexExpression{although} > \SchuetzlerIndexExpression{though} > \SchuetzlerIndexExpression{even though}. This pattern is in line with the general frequency pattern described in \chapref{sec:7}, and it is explicable from the fact that \is{concessives (types of)!dialogic}dialogic CCs are the dominant type~– the pattern typical of \is{concessives (types of)!dialogic}dialogic semantics will thus have a disproportionately high influence on general \is{text frequency}text frequencies. Once again, however, \il{Indian English}IndE with its exceptionally high relative frequency of \SchuetzlerIndexExpression{though} is a striking exception. Further, \SchuetzlerIndexExpression{though} and \SchuetzlerIndexExpression{even though} are roughly on a par in \il{Canadian English}CanE (even with a slightly higher frequency of the latter), and in \il{Nigerian English}NigE the percentages of \SchuetzlerIndexExpression{although} and \SchuetzlerIndexExpression{though} are almost the same.

Within the category of \is{concessives (types of)!anticausal}anticausal CCs, estimated percentages of \SchuetzlerIndexExpression{even though} in \figref{fig:10.3} are astonishingly high when compared against the initial impressions gained from \chapref{sec:7}: In five varieties, this conjunction is the most frequent one of the three, namely in \il{Canadian English}CanE (54.5\% [43.9; 64.9]), \il{Nigerian English}NigE (47.0\% [32.9; 61.4]), \il{Australian English}AusE (42.1\% [30.5; 52.8]), \il{Hong Kong English}HKE (41.0\% [31.4; 50.5]) and \il{Singapore English}SingE (39.6\% [26.5; 52.7]); in another three varieties, it ranks second only to \SchuetzlerIndexExpression{although}, namely in \il{Irish English}IrE (45.0\% [33.2; 56.1]), \il{Jamaican English}JamE (37.2\% [25.1; 48.6]) and \il{British English}BrE (32.7\% [20.8; 43.3]).

\begin{figure}[p]
\includegraphics{figures/CCs.Fig.10.3.pdf}
\caption{\label{bkm:Ref59921288}\label{fig:10.3}Average percentages of conjunctions by semantic type; A~= \SchuetzlerIndexExpression{although}, T~= \SchuetzlerIndexExpression{though}, E~= \SchuetzlerIndexExpression{even though}, a~= anticausal, d~= dialogic}
\end{figure}

\begin{sloppypar}
There is evidently a fundamental difference between \is{concessives (types of)!anticausal}anticausal and \is{concessives (types of)!dialogic}dialogic CCs concerning the roles of the conjunctions \SchuetzlerIndexExpression{although} and \SchuetzlerIndexExpression{even though}. The bird’s-eye perspective fully confirms this: The mean percentage of \SchuetzlerIndexExpression{although} across all varieties in \is{concessives (types of)!dialogic}dialogic CCs is 56.4\%, while in \is{concessives (types of)!anticausal}anticausal CCs it is 38.3\%; conversely, the average percentage of \SchuetzlerIndexExpression{even though} in \is{concessives (types of)!dialogic}dialogic CCs is a mere 14.5\%, while in \is{concessives (types of)!anticausal}anticausal CCs it is 40.2\%. The conjunction \SchuetzlerIndexExpression{though} does not partake in this semantically conditioned variation to the same extent. As can be seen in \figref{fig:10.3}, this marker also tends to be less frequent in \is{concessives (types of)!anticausal}anticausal CCs (21.0\%) as compared to \is{concessives (types of)!dialogic}dialogic CCs (28.9\%), but the more modest difference between these numbers suggests that the main division of labour for the marking of specific semantic relations within a construction seems to be between \SchuetzlerIndexExpression{although} and \SchuetzlerIndexExpression{even though}. It is only in \il{Irish English}IrE and \il{Indian English}IndE that \SchuetzlerIndexExpression{though} is more strongly affected by semantics than \SchuetzlerIndexExpression{although}. Interestingly, this mirrors the results presented in \sectref{sec:10.2.1} above, where it was found that \SchuetzlerIndexExpression{although} and \SchuetzlerIndexExpression{even though} also respond more strongly to the difference between \is{spoken language}speech and \is{written language}writing. While \SchuetzlerIndexExpression{though} tends to be functionally more similar to \SchuetzlerIndexExpression{although} in both dimensions of variation (mode of production and semantics), it is apparently somewhat more versatile~– that is, its likelihood of occurrence does not differ as radically between conditions as is the case for the other two conjunctions.
\end{sloppypar}

Again, we will inspect the data for general differences between \is{varieties of English!L1}L1 and \is{varieties of English!L2}L2 varieties concerning the effect of semantic types on the selection of conjunctions. \tabref{tab:10.2} provides a summary, showing for each subgroup of varieties the mean percentage of each conjunction in connection with the two semantic types, as well as the difference between these conditions.

\vfill
\begin{table}[H]
\caption{\label{bkm:Ref60005374}\label{tab:10.2}Mean percentages of conjunctions in anticausal and dialogic CCs in L1 and L2 varieties}
\is{varieties of English!L1}\is{concessives (types of)!dialogic}\is{concessives (types of)!anticausal}\is{concessives (types of)!anticausal}\is{concessives (types of)!dialogic}\is{concessives (types of)!anticausal}\is{concessives (types of)!dialogic}\is{varieties of English!L2}\is{concessives (types of)!dialogic}\is{concessives (types of)!anticausal}
\begin{tabular}{ll *3{S[table-format=-2.1]}}
\lsptoprule
   &  & {\itshape although} & {\itshape though} & {\itshape even though}\\\midrule
L1 & anticausal & 43.0 & 13.0 & 43.5\\
   & dialogic & 65.4 & 22.3 & 12.2\\
   & anticausal\,$-$\,dialogic & -22.4 & -9.3 & +31.3\\\midrule
L2 & anticausal & 34.1 & 28.7 & 36.8\\
   & dialogic & 45.7 & 37.9 & 16.2\\
   & anticausal\,$-$\,dialogic & -11.6 & -9.2 & +20.6\\
\lspbottomrule
\end{tabular}
\end{table}
\vfill\pagebreak

Like the general effect of mode of production (cf. \tabref{tab:10.2}), the impact of the semantic structure of a CC on the selection of the conjunction tends to be smaller in \is{varieties of English!L2}L2 varieties than in \is{varieties of English!L1}L1 varieties. Further, and again similarly to what was shown in \sectref{sec:10.2.1}, this difference between the two subsets of varieties surfaces only with regard to \SchuetzlerIndexExpression{although} and \SchuetzlerIndexExpression{even though}, while there is no such difference in connection with \SchuetzlerIndexExpression{though}. If it was tentatively argued above that \is{varieties of English!L2}L2 varieties appear to be stylistically less differentiated, results in this section suggest that there is also less intra-linguistic differentiation. In other words: In \is{varieties of English!L1}L1 varieties, the semantic difference between \is{concessives (types of)!anticausal}anticausal and \is{concessives (types of)!dialogic}dialogic CCs corresponds to a more substantial formal difference (i.e. a different selection of conjunctions) than in \is{varieties of English!L2}L2 varieties.

\figref{fig:10.4} presents the same comparison between \is{concessives (types of)!anticausal}anticausal and \is{concessives (types of)!dialogic}dialogic CCs but additionally includes the \is{spoken language}spoken-\is{written language}written dimension. The percentage panels at the bottom of each of the nine subplots thus contain two sets of values of the kind presented in \figref{fig:10.3}. General effects of mode and semantics as discussed earlier in this chapter can partly be traced in this plot. For instance, in several varieties the highest percentage of \SchuetzlerIndexExpression{even though} is found in \is{spoken language}spoken \is{concessives (types of)!anticausal}anticausal CCs, followed by \is{written language}written \is{concessives (types of)!anticausal}anticausal, \is{spoken language}spoken \is{concessives (types of)!dialogic}dialogic and \is{written language}written \is{concessives (types of)!dialogic}dialogic CCs, as in \il{Irish English}IrE, \il{Canadian English}CanE, \il{Australian English}AusE, \il{Singapore English}SingE and \il{Hong Kong English}HKE. However, other varieties show that there is no perfect regularity in the ranking of constraints. This is even more clearly the case for the other two conjunctions, and we will therefore turn to a more general assessment of patterns, averaging across conditions and groups of varieties. The focus will be on the magnitude of differences in the two \is{mode of production}modes of production as displayed in the upper panels of \figref{fig:10.4}. The purely visual inspection suggests that, while in most varieties patterns in \is{spoken language}speech and \is{written language}writing are similar, they often appear to be more compact in \is{written language}writing, as in \il{Canadian English}CanE, \il{Australian English}AusE, \il{Jamaican English}JamE, \il{Indian English}IndE and \il{Singapore English}SingE, for instance. A prime example of this tendency is \il{Canadian English}CanE: In both \is{mode of production}modes of production, \SchuetzlerIndexExpression{although} and \SchuetzlerIndexExpression{though} associate with \is{concessives (types of)!dialogic}dialogic CCs and \SchuetzlerIndexExpression{even though} associates with \is{concessives (types of)!anticausal}anticausal CCs (with the respective negative and positive values in the upper panel); in \is{written language}writing, however, all values are closer to zero. That is, while the general pattern is preserved, it is less extreme in \is{written language}writing.

\begin{figure}
\includegraphics{figures/CCs.Fig.10.4.pdf}
\caption{\label{bkm:Ref59994927}\label{fig:10.4}Average percentages of conjunctions by semantic type in \is{spoken language}speech and \is{written language}writing; A~= \SchuetzlerIndexExpression{although}, T~= \SchuetzlerIndexExpression{though}, E~= \SchuetzlerIndexExpression{even though}, a~= anticausal, d~= dialogic}
\end{figure}

General patterns and the magnitudes of semantic effects are summarised in \tabref{tab:10.3}, which shows for both \is{mode of production}modes of production the cross-varietal average percentages of the three conjunctions, given one or the other semantic type, as well as the mean differences between them. Thus, the table effectively sums up the interaction of semantics and mode of production. It once more illustrates the general association between semantic types and specific conjunctions: In both \is{mode of production}modes of production, percentages of \SchuetzlerIndexExpression{although} and \SchuetzlerIndexExpression{though} are higher in connection with \is{concessives (types of)!dialogic}dialogic CCs, while percentages of \SchuetzlerIndexExpression{even though} are higher in connection with \is{concessives (types of)!anticausal}anticausal CCs. Due to the organisation of the table, another general tendency is more difficult to detect, namely the increased percentages of both \SchuetzlerIndexExpression{although} and \SchuetzlerIndexExpression{though} in \is{written language}written \is{genre}genres and the increased percentages of \SchuetzlerIndexExpression{even though} in \is{spoken language}speech, regardless of semantics.

\begin{table}
\caption{\label{bkm:Ref60003282}\label{tab:10.3}Mean percentages of conjunctions in anticausal and dialogic CCs in \is{written language}writing and \is{spoken language}speech}
\begin{tabular}{ll *3{S[table-format=-2.1]}}
\lsptoprule
 &  & {\itshape \SchuetzlerIndexExpression{although}} & {\itshape \SchuetzlerIndexExpression{though}} & {\itshape \SchuetzlerIndexExpression{even though}}\\\midrule
Written & \is{concessives (types of)!anticausal}anticausal & 42.1 & 23.8 & 33.3\\
& \is{concessives (types of)!dialogic}dialogic & 58.9 & 29.9 & 10.7\\
& \is{concessives (types of)!anticausal}anticausal\,$-$\,\is{concessives (types of)!dialogic}dialogic & -16.8 & -6.1 & +22.6\\\midrule
Spoken & \is{concessives (types of)!anticausal}anticausal & 34.4 & 17.4 & 46.9\\
& \is{concessives (types of)!dialogic}dialogic & 53.8 & 27.6 & 18.0\\
& \is{concessives (types of)!anticausal}anticausal\,$-$\,\is{concessives (types of)!dialogic}dialogic & -19.4 & -10.2 & +28.9\\
\lspbottomrule
\end{tabular}
\end{table}

A complex design that takes several intra- and extra-linguistic factors (including different global varieties) into account is bound to generate results that will not be homogeneous from all perspectives. For instance, individual varieties will diverge from general patterns, possibly due to the data quality in specific corpus components (in this case of ICE), or due to other factors unknown. Cases that do not conform to the majority pattern may provide points of departure for linguists with expert knowledge and a particular interest in the respective varieties, but they are not discussed any further in this study in order to avoid the risk of post-hoc, speculative argumentation.

\subsection{\label{bkm:Ref59530974}\label{bkm:Ref60239588}\label{bkm:Ref60241323}\label{bkm:Ref60417335}Clause position}\label{sec:10.2.3}

This section is organised in parallel to the two preceding ones. A first, general approach to the effects of clause position on the selection of conjunctions is presented in \figref{fig:10.5}, which consists of nine subplots corresponding to the varieties under investigation. The panels in the lower part of each subplot show percentages of \SchuetzlerIndexExpression{although}, \SchuetzlerIndexExpression{though} and \SchuetzlerIndexExpression{even though} in sentence-final subordinate clauses (in black) and clauses that are in \isi{nonfinal position} (grey). The upper panels show absolute percentage-point differences between the two conditions, i.e. the subtraction of percentages in \isi{nonfinal position} from percentages in \isi{final position}. A dashed reference line at the value of zero (“no difference”) is added to the upper panel of each subplot.

\begin{figure}
\includegraphics{figures/CCs.Fig.10.5.pdf}
\caption{\label{bkm:Ref60159375}\label{fig:10.5}Average percentages of conjunctions by clause position; A~= \SchuetzlerIndexExpression{although}, T~= \SchuetzlerIndexExpression{though}, E~= \SchuetzlerIndexExpression{even though}, fn~= final, nf~= nonfinal}
\end{figure}

The main difference between the two clause arrangements concerns \SchuetzlerIndexExpression{although} and \SchuetzlerIndexExpression{even though}: Across varieties, the percentages of these two conjunctions in sentence-final clauses are 40.4\% and 31.6\%, respectively; in other clauses, they are 54.2\% (up by 13.8 percentage points) and 23.3\% (down by 8.3 percentage points). Relative frequencies of the third conjunction (\SchuetzlerIndexExpression{though}) are affected less (27.7\% if sentence-\is{final position}final; otherwise 22.2\%). The most common pattern for a variety across both clause positions can be described as follows:
(i)~In \is{nonfinal position}nonfinal subordinate clauses, \SchuetzlerIndexExpression{although} is the most commonly selected conjunction, usually followed by \SchuetzlerIndexExpression{even though}, with \SchuetzlerIndexExpression{though} coming third~– albeit sometimes by a small margin;
(ii)~clauses in \isi{final position} preserve this general pattern, with a smaller percentage-point difference between \SchuetzlerIndexExpression{although} and \SchuetzlerIndexExpression{though}. \il{Irish English}IrE, \il{Canadian English}CanE, \il{Australian English}AusE, \il{Jamaican English}JamE, \il{Singapore English}SingE and \il{Hong Kong English}HKE conform to this pattern. When inspecting them in \figref{fig:10.5}, we can see that, in contrast to clauses in \isi{nonfinal position}, sentence-\is{final position}final clauses are characterised by a “flattened hockey-stick pattern”, or even a V-shaped pattern.

Varieties that do not conform to this pattern are \il{British English}BrE, \il{Nigerian English}NigE and \il{Indian English}IndE. However, in \il{British English}BrE and \il{Nigerian English}NigE, sentence-\is{final position}final clauses are still associated with lower percentages of \SchuetzlerIndexExpression{although} and higher percentages of \SchuetzlerIndexExpression{even though}. \il{British English}BrE differs in using \SchuetzlerIndexExpression{though} more frequently than \textit{even} \SchuetzlerIndexExpression{though} throughout, and in \il{Nigerian English}NigE the difference between clause positions effects a complete reversal of the frequency ranking of conjunctions. \il{Indian English}IndE simply stands out in having a unique and rather different pattern, with disproportionately high percentages of \SchuetzlerIndexExpression{though}. On the whole it is once again predominantly \SchuetzlerIndexExpression{although} and \SchuetzlerIndexExpression{even though} that correlate with a change in condition; \SchuetzlerIndexExpression{though} only shows a moderately higher percentage in sentence-\is{final position}final clauses.

Again, the data are inspected for general differences between \is{varieties of English!L1}L1 and \is{varieties of English!L2}L2 varieties, this time concerning the effect of clause position on the selection of conjunctions. \tabref{tab:10.4} summarises for each subgroup of varieties the mean percentage of each conjunction in association with subordinate clauses in final and \isi{nonfinal position}. Similarly to what was found in the inspection of mode and semantics in the previous sections, the effect tends to be smaller in \is{varieties of English!L2}L2 varieties. This is true for \SchuetzlerIndexExpression{although} and \SchuetzlerIndexExpression{though}, but not for \SchuetzlerIndexExpression{even though}. We can also see that the general percentage patterns in combination with each of the two clause arrangements is more level in \is{varieties of English!L2}L2 varieties~– i.e. values for the three connectives are closer to each other. This is particularly visible in connection with clauses in \isi{final position}, where a share of roughly one third is estimated for each of the three markers. Once again, we can carefully draw on the concept of \is{differentiation (Dynamic Model)}differentiation \citep{Schneider2003}, or a slight modification thereof: The conjunctions under investigation are possibly used less discriminately in \is{varieties of English!L2}L2 varieties, while in \is{varieties of English!L1}L1 varieties there appears to be a higher degree of specialisation, with a general preference of \SchuetzlerIndexExpression{although}, particularly in subordinate clauses that precede the matrix clause. This is an interesting finding because it persists in the different analyses conducted in this section. Whether or not these patterns are the result of a \is{diachronic approaches}diachronic process of \isi{grammaticalisation} and differentiation that has progressed further in \is{varieties of English!L1}L1 varieties is beyond what this study can investigate.

\begin{table}
\caption{\label{bkm:Ref75366070}\label{tab:10.4}Mean percentages of conjunctions by clause position in L1 and L2 varieties}
\begin{tabular}{ll *3{S[table-format=-2.1]}}
\lsptoprule
 &  & {\itshape \SchuetzlerIndexExpression{although}} & {\itshape \SchuetzlerIndexExpression{though}} & {\itshape \SchuetzlerIndexExpression{even though}}\\\midrule
\is{varieties of English!L1}L1 & \is{final position}final & 46.1 & 21.7 & 32.0\\
& \is{nonfinal position}nonfinal & 62.2 & 13.7 & 23.9\\
& \is{final position}final\,$-$\,\is{nonfinal position}nonfinal & -16.1 & +8.0 & +8.1\\\midrule
\is{varieties of English!L2}L2 & \is{final position}final & 35.9 & 32.6 & 31.2\\
& \is{nonfinal position}nonfinal & 47.8 & 29.1 & 22.9\\
& \is{final position}final\,$-$\,\is{nonfinal position}nonfinal & -11.9 & +3.5 & +8.3\\
\lspbottomrule
\end{tabular}
\end{table}

In \figref{fig:10.6}, the inspection of clause placement and its effects on the selection of conjunctions is unfolded into \is{spoken language}speech and \is{written language}writing. For each mode, there are again two sets of values in the percentage panels at the bottom of each of the nine subplots, rendered in grey (nonfinal) and black (final). Once again, the focus will lie on the direction and magnitude of differences in the two modes of production in the upper panels of the figure. Patterns are manifold and it is difficult to generalise across them. Three varieties have an indifferent or relatively flat pattern in \is{spoken language}speech that is augmented in a regular fashion in \is{written language}writing: \il{Jamaican English}JamE, \il{Nigerian English}NigE and \il{Hong Kong English}HKE. The remaining four varieties (\il{British English}BrE, \il{Irish English}IrE, \il{Canadian English}CanE and \il{Indian English}IndE), however, are not captured by this generalisation, since effects either do not differ much (or not systematically) between modes, or because the pattern is reversed, as in \il{Canadian English}CanE and \il{Indian English}IndE. It would appear, then, that \isi{mode of production} and clause position do not interact systematically in conditioning the selection of concessive conjunctions. The tendencies discussed above are also summarised in \tabref{tab:10.5}, which compares the effects of clause position on the selection of conjunctions for both modes of production, showing in each case mean percentages of clauses in \is{final position}final and \isi{nonfinal position} as well as the difference between conditions.

\begin{figure}
\includegraphics{figures/CCs.Fig.10.6.pdf}
\caption{\label{bkm:Ref60209688}\label{fig:10.6}Average proportions of conjunctions by clause position in \is{spoken language}speech and \is{written language}writing; A~= \SchuetzlerIndexExpression{although}, T~= \SchuetzlerIndexExpression{though}, E~= \SchuetzlerIndexExpression{even though}, fn~= final, nf~= nonfinal}
\end{figure}

\begin{table}
\caption{\label{bkm:Ref60005389}\label{tab:10.5}Mean percentages of conjunctions by clause position in \is{written language}writing and \is{spoken language}speech}
\begin{tabular}{ll *3{S[table-format=-2.1]}}
\lsptoprule
 &  & {\itshape \SchuetzlerIndexExpression{although}} & {\itshape \SchuetzlerIndexExpression{though}} & {\itshape \SchuetzlerIndexExpression{even though}}\\\midrule
Written & \is{final position}final & 40.4 & 29.5 & 29.7\\
& \is{nonfinal position}nonfinal & 60.6 & 24.4 & 14.4\\
& \is{final position}final\,$-$\,\is{nonfinal position}nonfinal & -20.2 & +5.1 & +15.3\\\midrule
Spoken & \is{final position}final & 40.3 & 25.6 & 33.3\\
& \is{nonfinal position}nonfinal & 47.8 & 19.6 & 31.8\\
& \is{final position}final\,$-$\,\is{nonfinal position}nonfinal & -7.5 & +6.0 & +1.5\\
\lspbottomrule
\end{tabular}
\end{table}

The table highlights numerically what was stated above, namely that the \is{written language}written mode tends to augment the effect of clause position on the choice of marker. This is in contrast to the finding in \sectref{sec:10.2.2} that semantic effects are reduced in \is{written language}writing. It appears that \isi{written language} does not generally minimise the constraints that operate on the realisation of CCs. However, clause positions could be argued to constitute a slightly different case: While semantic properties of CCs are broadly language-internal, describing the relationship between the two propositions that make up the construction, clause positions are a formal property of CCs, and we are thus looking at a correlation of one formal parameter (clause position) with another formal parameter (choice of \is{connectives}connective). It could be the case that particular surface forms have been codified as part of the \is{written language}written mode with some degree of independence from functional parameters, which would explain why
(i)~semantic effects are somewhat subdued in \is{written language}writing (see \sectref{sec:10.2.2}) and why
(ii)~more distinct formal realisations are brought out in \is{written language}writing. Of course, this interpretation is post hoc, not motivated from theory, and it therefore has to be treated with due caution.

\subsection{\label{bkm:Ref59531598}Complete factor combinations}\label{sec:10.2.4}

This section shows all individual conditions and their effects on the selection of markers. Sections \ref{sec:10.2.1}–\ref{sec:10.2.3} involved some degree of simplification (or abstraction), as the plots and discussions there were based on average values for certain conditions in which one or several of the factors were controlled for. In contrast, this section shows all the details and thus makes the underlying specific values transparent. It cannot, however, result in alternative interpretations.

The logic behind the three complex plots presented in this section is the same as in Figures \ref{fig:7.5}–\ref{fig:7.7}, as well as in \figref{fig:9.6}, but it will nevertheless be explained in brief. There is one plot for each of the three conjunctions (\SchuetzlerIndexExpression{although}, \SchuetzlerIndexExpression{though} and \SchuetzlerIndexExpression{even though}), each showing the expected percentages of the respective marker in all of the \textit{n}~=~72 conditions. This number of conditions results from the fact that specific estimates differ according to variety (\textit{n}~=~9), mode of production (×~2), semantics (×~2) and clause position (×~2). For each conjunction, these conditions are made explicit on the left-hand side of the plot, and they are arranged in descending order according to their median estimates. The right-hand part of the plot highlights groupings of conditions based on variety type (\is{varieties of English!L1}L1 vs \is{varieties of English!L2}L2), mode, semantics and clause position. Once again, by comparing higher and lower concentrations of white and black squares, the reader has quicker visual access to general tendencies in the data. Each column in this part of the plot additionally shows the mean rank for each group, using triangular markers corresponding in colour to the respective group. Let us first turn to \figref{fig:10.7}, which shows the complete set of individual estimates for \SchuetzlerIndexExpression{although}.

\begin{figure}
\includegraphics{figures/CCs.Fig.10.7.pdf}
\caption{\label{bkm:Ref60240952}\label{fig:10.7}Ranked percentages of \SchuetzlerIndexExpression{although} by specific conditions; W~= \is{written language}written, S~= \is{spoken language}spoken, a~= anticausal, d~= dialogic, fn~= final, nf~= nonfinal}
\end{figure}

The highest percentage (rank \#1) of \SchuetzlerIndexExpression{although} is estimated for \is{concessives (types of)!dialogic}dialogic CCs with subordinate clauses in \isi{nonfinal position} in written \il{Australian English}AusE, at 82.3\% [72.4; 90.9]; the lowest percentage (rank \#72) is estimated for \is{concessives (types of)!anticausal}anticausal CCs that occur in sentence-\is{final position}final subordinate clauses in written \il{Nigerian English}NigE, at a mere 6.6\% [1.4; 19.4]. This minimum value constitutes an outlier, but between rank \#1 and rank \#71 (\is{concessives (types of)!anticausal}anticausal CCs in nonfinal subclauses in spoken \il{Indian English}IndE) there is a fairly even distribution of median values. Turning to the right-hand part of \figref{fig:10.7}, we see that the distribution of black and white squares in the four columns reflects the results discussed earlier in this section. For instance, the mean rank of specific conditions from the \is{varieties of English!L1}L1 group of varieties is 28.7, while conditions in the \is{varieties of English!L2}L2 group rank considerably lower, at an average rank of 42.8; this is in line with the discussion in \sectref{sec:10.2.1} to the effect that percentages of \SchuetzlerIndexExpression{although} are, on average, higher in \is{varieties of English!L1}L1 varieties. Likewise, \is{written language}written varieties rank higher than \is{spoken language}spoken varieties ($M_{\text{~W}}$~=~32.9; $M_{\text{~S}}$~=~40.1)~– again, see \sectref{sec:10.2.1} for percentage-based results that correspond to this finding. Looking at the third column, the rank-based approach illustrates that semantics (i.e. the difference between \is{concessives (types of)!anticausal}anticausal and \is{concessives (types of)!dialogic}dialogic CCs) have a larger effect than variety status and \isi{mode of production}: Dialogic CCs strongly favour \SchuetzlerIndexExpression{although}, with a mean rank of 26.3, while the mean rank of \is{concessives (types of)!anticausal}anticausal conditions is considerably lower, at 46.7 (cf. \sectref{sec:10.2.2}). Finally, note the clear tendency for conditions to favour \SchuetzlerIndexExpression{although} when a subordinate clause is in \isi{nonfinal position} (mean rank: 28.2) as compared to cases with clauses in \isi{final position} (mean rank: 44.8)~– this finding points to the “\isi{grounding}” function of \SchuetzlerIndexExpression{although} (cf. \sectref{sec:10.2.3}). As stated above, the presentation of results in \figref{fig:10.7} does not add substantially to the earlier discussion. It does, however, make the individual patterns transparent: While the scenarios compared in the previous three sections involved some degree of simplification since they backgrounded one or several factors, all details are shown here. Further, it is demonstrated that mean ranks as indicated in \figref{fig:10.7} are quite reliable as a basic~– and relatively intuitive~– measure of \isi{effect size}:  The further apart the triangular indicators attached to the columns on the right, the more distinct the two basic groups that are being compared. The discussion of the conjunction \textit{though} in \figref{fig:10.8} happens along similar lines but will be kept somewhat shorter.

\begin{figure}
\includegraphics{figures/CCs.Fig.10.8.pdf}
\caption{\label{bkm:Ref60241527}\label{fig:10.8}Ranked percentages of \SchuetzlerIndexExpression{though} by specific conditions across varieties; W~= \is{written language}written, S~= \is{spoken language}spoken,  a~= anticausal, d~= dialogic, fn~= final, nf~= nonfinal}
\end{figure}

The highest-ranking estimate for \SchuetzlerIndexExpression{though} is for \is{concessives (types of)!dialogic}dialogic CCs with subordinate clauses in final position in spoken \il{Indian English}IndE, at a value of 74.3\% [61.3; 86.4]; the lowest-ranking percentage is estimated for \is{concessives (types of)!anticausal}anticausal CCs with subordinate clauses in nonfinal position in spoken \il{Canadian English}CanE, at 1.7\% [0.2; 10.4]. Ranks 66–72 as well as the top eight ranks appear to break away from the central part of the distribution in  \figref{fig:10.8}, which makes a somewhat more skewed impression compared to \figref{fig:10.7}~– in other words: The ordinary range of values is somewhat narrower if we disregard the more extreme ranks. Once again, the distribution of black and white squares in the four columns on the right of \figref{fig:10.8} is in accordance with earlier discussions in this section. \is{varieties of English!L2}L2 varieties (mean rank: 27.8) are considerably more likely than \is{varieties of English!L1}L1 varieties (mean rank: 47.4) to select \SchuetzlerIndexExpression{though}~– this is of course partly due to the exceptional position of \il{Indian English}IndE, which occupies the top eight ranks. \is{written language}Written varieties are generally more likely to select \SchuetzlerIndexExpression{though} than \is{spoken language}spoken varieties, with a mean rank of 31.1 (as compared to 41.9 in \is{spoken language}speech). In the third column, we see that semantics have a similar (if somewhat weaker) effect when compared to \SchuetzlerIndexExpression{although} in \figref{fig:10.7} above: The conjunction \SchuetzlerIndexExpression{though} is more likely in \is{concessives (types of)!dialogic}dialogic and less likely in \is{concessives (types of)!anticausal}anticausal CCs, with mean ranks of 29.7 and 43.3, respectively. The effect of clause position on the selection of \SchuetzlerIndexExpression{though} is the inverse of its effect on the selection of \SchuetzlerIndexExpression{although}, and it is also somewhat weaker: The average rank of conditions that involve subordinate clauses in \isi{nonfinal position} is 41.2, while for sentence-\is{final position}final clauses this value is 31.8.

\begin{figure}
\includegraphics{figures/CCs.Fig.10.9.pdf}
\caption{\label{bkm:Ref60241698}\label{fig:10.9}Ranked percentages of \SchuetzlerIndexExpression{even though} by specific conditions across varieties; W~= \is{written language}written, S~= \is{spoken language}spoken, a~= anticausal, d~= dialogic, fn~= final, nf~= nonfinal}
\end{figure}

Let us now turn to the discussion of specific estimates for the conjunction \SchuetzlerIndexExpression{even though} in \figref{fig:10.9}. This marker is most frequent in \is{concessives (types of)!anticausal}anticausal CCs with subclauses in \isi{nonfinal position} in spoken \il{Canadian English}CanE, at 76.2\% [55.8; 91.3], and its occurrence is least likely in \is{concessives (types of)!dialogic}dialogic CCs with subordinate clauses in \isi{nonfinal position} in written \il{British English}BrE, at 2.1\% [0.3; 6.9]. Apart from the top four ranks, the distribution of values is quite even. The mean ranks of specific conditions from the \is{varieties of English!L1}L1 and \is{varieties of English!L2}L2 groups of varieties in the right-hand part of the figure are virtually the same, at 36.7 and 36.4, respectively. Again, the perspective taken in this section does not generate new insights but merely shows results that were discussed earlier in a different light: Note, for instance, that the similarity of ranks for \is{varieties of English!L1}L1 and \is{varieties of English!L2}L2 varieties necessarily corresponds to the similarity of mean percentages discussed in \sectref{sec:10.2.1} (\is{varieties of English!L1}L1: 28.0\%; \is{varieties of English!L2}L2: 27.1\%). In contrast to the other two conjunctions, \SchuetzlerIndexExpression{even though} is more likely in \is{spoken language}speech (mean rank: 30.7) than in \is{written language}writing (mean rank: 42.3). The strong semantic effect highlighted in the third column is also the inverse of what was found for \SchuetzlerIndexExpression{although} and \SchuetzlerIndexExpression{though}: The mean rank for conditions that involve \is{concessives (types of)!anticausal}anticausal semantics is 21.1; for \is{concessives (types of)!dialogic}dialogic semantics, the mean rank is considerably lower, at 51.9. Finally, from the perspective of ranked specific conditions, the general relationship between clause position and the selection of \SchuetzlerIndexExpression{even though} seems to be very similar to what was found for \SchuetzlerIndexExpression{though}: When a subordinate clause in \isi{nonfinal position} is involved, the mean rank of conditions is 41.3 (\SchuetzlerIndexExpression{though}: 41.2); when there is a clause in \isi{final position}, the mean rank is 31.7 (\SchuetzlerIndexExpression{though}: 31.8). This is in marked contrast to the rankings found for \SchuetzlerIndexExpression{although} with regard to this parameter.

The rank-based assessments of the four basic contrasts~– according to variety type (\is{varieties of English!L1}L1 vs \is{varieties of English!L2}L2), mode of production, semantics and clause position~– show from a different perspective the same tendencies that were discussed in \sectref{sec:10.2.1}–\ref{sec:10.2.3}. At the same time, they make the estimates for specific conditions~– each defined by a unique combination of factors~– maximally transparent. The conjunction \SchuetzlerIndexExpression{although} remains the most frequent conjunction in most scenarios. However, for a sizeable number of factor combinations it is \SchuetzlerIndexExpression{even though} that is estimated to be the most likely choice. These tendencies as well as the precise numbers of particular rankings are shown in \tabref{tab:10.6}.

\begin{table}
\caption{\label{bkm:Ref60636585}\label{tab:10.6}Frequency rankings of conjunctions relative to each other, based on \textit{n~}=~72 conditions}
\begin{tabular}{llr}
\lsptoprule
 & Three-way ranking & $n$\\\midrule
1 & \SchuetzlerIndexExpression{although} > \SchuetzlerIndexExpression{though} > \SchuetzlerIndexExpression{even though} & 27\\
2 & \textit{even though} > \textit{although} > \textit{though} & 17\\
3 & \textit{although} > \textit{even though} > \textit{though} & 15\\
4 & \textit{though} > \textit{although} > \textit{even though} & 7\\
5 & \textit{though} > \textit{even though} > \textit{although} & 3\\
& \textit{even though} > \textit{though} > \textit{although} & 3\\\midrule
& Pairwise comparison & $n$\\\midrule
1 & \textit{although} > \textit{though} & 59\\
2 & \textit{although} > \textit{even though} & 49\\
3 & \textit{though} > \textit{even though} & 37\\
4 & \textit{even though} > \textit{though} & 35\\
5 & \textit{even though} > \textit{although} & 23\\
6 & \textit{though} > \textit{although} & 13\\\midrule
& Most frequent conjunction & $n$\\\midrule
1 & {\itshape although} & 42\\
2 & {\itshape even though} & 20\\
3 & {\itshape though} & 10\\
\lspbottomrule
\end{tabular}
\end{table}

As a matter of course, a condition that is likely to produce higher percentages of one of the three conjunctions must produce lower percentages of one or both of the others. Therefore, the \textit{n~}=~72 percentages discussed above will under normal circumstances be negatively correlated for any pair of conjunctions. In \figref{fig:10.10}, these relationships are explored in some more detail, taking \SchuetzlerIndexExpression{although}, \SchuetzlerIndexExpression{though} and \SchuetzlerIndexExpression{even though} as the respective points of reference on the $x$-axis in the three panels of the plot. Relationships are gauged more precisely by additionally showing the respective coefficient from a simple linear regression model, which indicates by how much the percentage on the $y$-axis changes as the percentage on the $x$-axis increases by one point.

\begin{figure}
\includegraphics{figures/CCs.Fig.10.10.pdf}
\caption{\label{bkm:Ref60442245}\label{fig:10.10}Relationship of median percentages of three conjunctions for all conditions}
\end{figure}


There are two basic, methodologically reassuring findings. Firstly, all correlations are negative. This means that any increase in the percentage of one of the conjunctions comes at the expense of both of the others~– any other pattern would have been surprising in view of the results that were discussed earlier. Secondly, the beta-coefficients in each of the three parts of the figure roughly add up to one. This must necessarily be the case: The sum of percentages across all three markers must remain at 100\%, so that an increase by one percentage point in one of them must be accompanied by a total decrease of one percentage point in the other two combined. The interesting detail on whose discussion we can conclude this section is that the strongest negative correlation exists between \SchuetzlerIndexExpression{although} and \SchuetzlerIndexExpression{even though}, indicated by the regression coefficients and the steepness of the regression lines in the second plot in \figref{fig:10.10}a and the first plot in \figref{fig:10.10}c. For instance, if moving from one condition to another increases the estimated percentage of \SchuetzlerIndexExpression{although} by ten points, the estimated percentage of \SchuetzlerIndexExpression{even though} will on average decrease by 6.8 points, while for \SchuetzlerIndexExpression{though} the decrease will only be 3.1 points. This dovetails with the earlier discussions, in which it was found that for all four basic factors~– variety type (\is{varieties of English!L1}L1/\is{varieties of English!L2}L2), mode, semantics and clause position~– the greatest average swing in percentages, as we move from one factor level to the other, tends to be between \SchuetzlerIndexExpression{although} and \textit{even though}. By contrast, \SchuetzlerIndexExpression{though} is also systematically affected but shows a more moderate response. To use once again an expression introduced in \sectref{sec:10.2.2}: The conjunction \SchuetzlerIndexExpression{though} is functionally more versatile, and the greatest functional contrast is between \SchuetzlerIndexExpression{although} and \SchuetzlerIndexExpression{even though}.

\section{\label{bkm:Ref59531919}\label{bkm:Ref59920988}Summary and discussion}\label{sec:10.3}\largerpage

This final section will first summarise the main conditions likely to result in the selection of each of the three conjunctions. After thus highlighting the functional differences between the three markers, the discussion will turn to the moderating effect that \isi{mode of production} has on the other main factors, as well as the general differences that were found in the comparison of \is{varieties of English!L1}L1 and \is{varieties of English!L2}L2 \isi{varieties of English} in this chapter.

Under most circumstances, the conjunction \SchuetzlerIndexExpression{although} is the most frequent one of the three markers (see \tabref{tab:10.6}). It is particularly common in \is{written language}writing, when \is{concessives (types of)!dialogic}dialogic meaning is expressed, and when the subordinate clause is in \isi{nonfinal position}. The tendency for \SchuetzlerIndexExpression{although} to be more frequently selected in \is{written language}writing is in broad agreement with the general notion that this conjunction is more \is{formality}formal than, for instance, \SchuetzlerIndexExpression{though} (\citealt{QuirkEtAl1985,BiberEtAl1999,HuddlestonPullum2002}; also \citealt{Aarts1988}). Further, the association of \SchuetzlerIndexExpression{although} with \is{concessives (types of)!dialogic}dialogic semantics agrees with patterns in \il{American English}AmE data in \citet{Schützler2018b} as well as \il{New Zealand English}NZE data in \citet{Schützler2017}, but it is contra the general tendencies described in \citet{Hilpert2013a}.{\interfootnotelinepenalty=10000\footnote{\citet{Schützler2017} includes data from the British, Canadian and New Zealand components of \is{International Corpus of English}ICE; the first two data sources can of course not be cited as additional (independent) evidence, since they also feature in the present study.}}  The clear difference between the present study and \citet{Hilpert2013a} is surprising only at first glance: Due to his particular research interest in concessive \isi{parentheticals}, Hilpert focuses on CCs with \is{co-referentiality}co-referential subjects in both clauses, which narrows the eligible constructions down to a grammatically more restricted and therefore smaller set. The strong association of \is{concessives (types of)!dialogic}dialogic CCs with \SchuetzlerIndexExpression{although} can also account for the high \isi{text frequency} of this marker (cf. \chapref{sec:7}): Since the \is{concessives (types of)!dialogic}dialogic type is the most common kind of CC (as shown in \chapref{sec:8}), and since \is{concessives (types of)!dialogic}dialogic semantics tend to be expressed with \SchuetzlerIndexExpression{although}, the high overall frequency of this conjunction naturally follows.

\begin{sloppypar}
  When we average across different conditions and thus gloss over the differences induced by the various internal and external factors, the conjunction \SchuetzlerIndexExpression{though} seems to be of roughly the same relative frequency as \SchuetzlerIndexExpression{even though}. However, like \SchuetzlerIndexExpression{although} (and in contrast to \SchuetzlerIndexExpression{even though}) it associates mostly with \is{concessives (types of)!dialogic}dialogic CCs, which strengthens it in terms of \isi{text frequency} (cf. \chapref{sec:7}). The probability of selecting \SchuetzlerIndexExpression{though} is higher in \is{written language}writing, but this effect is usually weaker than for \SchuetzlerIndexExpression{although}. This finding casts doubt on the assertion that \SchuetzlerIndexExpression{though} is a less \is{formality}formal variant, which is found in some of the literature (particularly in the major standard grammars; but see also \citealt{Aarts1988}). If we accept writing and speech as very basic \is{style}stylistic categories, we would expect less \is{formality}formal items to occur at higher frequencies in \is{spoken language}spoken discourse. This is not the case for \SchuetzlerIndexExpression{though}. All we can say is that this conjunction is affected less strongly by a difference in mode than \SchuetzlerIndexExpression{although}, but both effects are in the same direction. In contrast to \SchuetzlerIndexExpression{although}, however, the conjunction \SchuetzlerIndexExpression{though} tends to be used more in subordinate clauses in \isi{final position} and resembles \SchuetzlerIndexExpression{even though} in this respect.
\end{sloppypar}

  Lastly, \SchuetzlerIndexExpression{even though} is considerably more frequent in \is{spoken language}speech, in contrast to the other two conjunctions. The literature has very little to say about this marker’s \isi{formality} value but regularly stresses its \is{emphasis}emphatic character, presumably triggered by the adverb \textit{even}. It could be argued that \isi{emphasis} and immediacy are more characteristic of \is{spoken language}speech, in the sense that SP/W draws on material that is felt to be stronger or more emotive in order to persuade AD/R. This higher degree of \isi{emphasis} coincides with the fact that \SchuetzlerIndexExpression{even though} is morphologically the most complex of the three markers (cf. \sectref{sec:2.3}). Once again in contrast to the other two conjunctions, \SchuetzlerIndexExpression{even though} is strongly associated with \is{concessives (types of)!anticausal}anticausal CCs. This finding explains why this marker appears to be quite rare from a perspective purely based on \isi{text frequency}; in the variationist approach, i.e. when we consider \isi{variable contexts} and the factors that play a role in the selection of conjunctions, there are many scenarios (particularly in \is{concessives (types of)!anticausal}anticausal CCs) in which \SchuetzlerIndexExpression{even though} can be quite frequent, or even the most frequent variant. In marked contrast to \SchuetzlerIndexExpression{although} but to some extent similar to \SchuetzlerIndexExpression{though}, \SchuetzlerIndexExpression{even though} is more common in subordinate clauses that follow the matrix clause.

  In the more general comparison of the three markers, it is striking that \SchuetzlerIndexExpression{although} and \SchuetzlerIndexExpression{even though} seem to form the poles of a functionally motivated (probabilistic) continuum: \SchuetzlerIndexExpression{although} is associated with \is{written language}written discourse, \is{concessives (types of)!dialogic}dialogic semantics and subordinate clauses in \isi{nonfinal position}; \SchuetzlerIndexExpression{even though} is associated with \is{spoken language}spoken discourse, \is{concessives (types of)!anticausal}anticausal semantics and subordinate clauses in \isi{final position}. This goes hand in hand with the finding that these two markers respond more strongly to differences in \isi{mode of production} and semantics, compared to \SchuetzlerIndexExpression{though}.

As regards the differences between \is{varieties of English!L1}L1 and \is{varieties of English!L2}L2 \isi{varieties of English}, the sample in the present study is of course too small for sweeping generalisations ($n_{\text{~L1}} = 4$; $n_{\text{~L2}} = 5$). Thus, the tendencies that were detected need to be treated with caution and can be taken as no more than indicators with the potential of providing guidance for future research. In the \is{varieties of English!L1}L1 varieties, the average effect of \isi{mode of production} on the selection of subordinators is greater than in the \is{varieties of English!L2}L2 varieties. The same is true with regard to the effect of semantics. A tentative conclusion that agrees with a broad understanding of \citegen{Schneider2007} notion of \is{differentiation (Dynamic Model)}differentiation in Phase 5 of his \isi{Dynamic Model} is the following: Patterns of use in \is{varieties of English!L2}L2 varieties are somewhat more fixed, i.e. they respond less sensitively to conditioning factors, be they external (e.g. mode, or \isi{style} more generally) or internal (e.g. semantic or information-structural). In other words, varieties from this broad subset have undergone less formal differentiation along contextual and functional lines. The systematic variability of rules (to use a key concept from variationist linguistics) is equally visible and tends to be in the same direction in \is{varieties of English!L1}L1 and \is{varieties of English!L2}L2 varieties, but effects tend to be smaller in the latter.

For the final part of this summary, I will return to the differences and similarities between the three conjunctions. It was argued that the main division of labour is between \SchuetzlerIndexExpression{although} and \SchuetzlerIndexExpression{even though}, while \SchuetzlerIndexExpression{though} tends to be functionally more intermediate: Typically, \SchuetzlerIndexExpression{although} is used for \isi{grounding} purposes (putting the matrix clause in focus position at the sentence level), for \is{concessives (types of)!dialogic}dialogic CCs and for \is{written language}written discourse; \SchuetzlerIndexExpression{even though} associates with \is{concessives (types of)!anticausal}anticausal CCs in which the subordinate clause is in \is{final position}final (i.e. focus) position, and it is more common in \is{spoken language}speech; \SchuetzlerIndexExpression{though}, like \SchuetzlerIndexExpression{although}, is more typical of \is{written language}writing and \is{concessives (types of)!dialogic}dialogic CCs, but~– like \SchuetzlerIndexExpression{even though}~– it tends to be attached to subordinate clauses that follow the matrix clause. Seeing that there is a functional continuum with \SchuetzlerIndexExpression{though} at its centre, and that the three markers can (and regularly do) serve exactly the same purposes, it is unsurprising that the literature thus far has either treated them as functionally equivalent or has tried to capture differences exclusively in terms of categories like “\isi{emphasis}” or “\isi{formality}”. However, as this chapter has shown, we can profile the differences between the three conjunctions in a more nuanced way and demonstrate that they are not only measurably different but also form a system in which specific tasks are assigned to specific markers. Naturally, those tendencies are not categorical but probabilistic. At present, we can only speculate as to why it is \SchuetzlerIndexExpression{although} and \SchuetzlerIndexExpression{even though} that are (or have become) particularly specialised in several respects. Ultimately, the answer to this question needs to be sought in \is{diachronic approaches}diachronic studies on a similar scale as the present synchronic one, i.e. studies based on data sets large enough to include the same (or perhaps even more) factors. A new, \is{diachronic approaches}diachronic hypothesis generated from the present research would be that, over time, the morphological variants \SchuetzlerIndexExpression{although} and \SchuetzlerIndexExpression{even though} grammaticalised into functionally somewhat different items. The pattern I would expect to find in \is{diachronic approaches}diachronic data is therefore one of gradual functional divergence. On the one hand, we might see \SchuetzlerIndexExpression{although} and \SchuetzlerIndexExpression{even though} slowly breaking away in different directions, increasingly specialising on the marking of constructional variants diametrically opposed in terms of typical contexts of use (e.g. mode), semantics and general syntactic design (clause sequencing). On the other hand, we would expect that \SchuetzlerIndexExpression{though} does not undergo the same degree of specialisation but borrows characteristics from the other two connectives, because in PDE it is more likely in \isi{final position}, in \is{written language}writing and in combination with \is{concessives (types of)!dialogic}dialogic CCs. Filling in the \is{diachronic approaches}diachronic details of such processes, or investigating whether or not such processes can be shown to have taken place at all, goes far beyond what the present study can achieve. I will return to these thoughts and their implications for future research as part of the final discussion in \chapref{sec:12}.
