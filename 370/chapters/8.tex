\chapter{Frequencies of semantic types}\label{bkm:Ref487278308}\label{bkm:Ref497550521}\label{bkm:Ref1727470}\label{bkm:Ref34987676}\label{ch:8}\label{sec:8}

In this chapter, the focus lies on the \is{text frequency}text frequencies of the four types of intra-constructional semantics that characterise CCs in the present study. As in \chapref{sec:7}, the analysis is purely count-based. Whereas the previous chapter inspected forms only (i.e. the three conjunctions), this chapter examines functions only. It is somewhat shorter than the previous one since it contains no analogue of the detailed displays in Figures~\ref{fig:7.5}–\ref{fig:7.7}.

One important caveat concerns the limited perspective of the present study with its focus on only three subordinating conjunctions. This chapter therefore cannot truly show whether certain semantic types are generally more frequent in any of the varieties under investigation~– it only shows their frequencies in connection with \textit{although}, \textit{though} and \textit{even though}. We could say that these conjunctions constitute a sample of markers, whose representativeness would need to be discussed. Accordingly, it is quite possible that frequency differences as shown in this chapter do not hold true when a wider range~– or, ideally, the complete inventory~– of concessive markers is taken into account. Therefore, even more so than \chapref{sec:7} above, this chapter generates few insights that can truly stand alone, which is another reason for its relative shortness.

The expectations formulated in \sectref{sec:5.3} are that both \is{varieties of English!L2}L2 varieties and the written mode should lean towards higher rates of \is{concessives (types of)!anticausal}anticausal CCs (and, perhaps, the related \is{concessives (types of)!epistemic}epistemic CCs, too), while \is{concessives (types of)!dialogic}dialogic CCs will occur at higher rates in \is{varieties of English!L1}L1 varieties and in speech. \is{concessives (types of)!narrow-scope dialogic}Narrow-scope CCs, while they are semantically part of the \is{concessives (types of)!dialogic}dialogic class of CCs, cannot easily be captured by the same general hypothesis regarding the effect of mode. They are more integrated as they encode concessive relationships at the phrase level, and they should therefore probably be regarded as cognitively quite complex, making their appearance in writing more likely. The general frequencies of \is{concessives (types of)!epistemic}epistemic CCs are expected to fall between those of \is{concessives (types of)!anticausal}anticausal and \is{concessives (types of)!dialogic}dialogic CCs, due to the alleged intermediate historical status of this type~– that is, if we work on the assumption that in the history of English \is{concessives (types of)!anticausal}anticausal CCs are primary and develop towards \is{concessives (types of)!dialogic}dialogic CCs via \is{concessives (types of)!epistemic}epistemic CCs (see discussion in \citealt{Sweetser1990}). Within the chapter, \sectref{sec:8.1} introduces the statistical model for the main frequency analysis in \sectref{sec:8.2}, and there will be a concluding discussion in \sectref{sec:8.3}.

\section{\label{bkm:Ref52538317}Statistical model}\label{sec:8.1}

The estimation procedure for rates of occurrence of the four semantic types is similar to the approach in \chapref{sec:7}, i.e. a \is{Bayesian statistics}Bayesian \is{regression!negative binomial}negative binomial \is{regression!mixed-effects}mixed-effects regression model of identical form is fitted for each variety. Instead of \textsc{marker}, however, it is \textsc{type} that is included in the fixed part. This predictor has four levels in this chapter~– “anticausal”, “epistemic”, “dialogic” and “narrow-scope”~– while in later analyses (Chapters \ref{sec:9}–\ref{sec:11}) only the two main variants, “anticausal” and “dialogic”, are included. The model is called “Model~B”, and its syntax is shown in (\ref{eq:8.1});  for a detailed specification of variables, see \tabref{tab:6.3} in \sectref{sec:6.3.6}. In analogy to Model~A above, the two predictors in the fixed part of the model interact, and slopes for \textsc{type} vary randomly across \textsc{genre}. Again, the \is{cluster variables}cluster variable \textsc{text} is not included, because the individual text is the smallest unit of observation, and variety does not feature as a \is{cluster variables}cluster variable either, since an independent submodel is run for each variety. As in Model A, the variable \textsc{log\_words} stands for the logged number of words per text; again, note that this variable is not listed in \tabref{tab:6.3}.

\ea
\label{bkm:Ref41326003}\label{eq:8.1}Model B: Syntax
\begin{lstlisting}
count ~ spoken.ct * type
        + (type | genre)
        + offset(log_words)
\end{lstlisting}
\z

Appendix~\ref{appendix:B.2} provides more information concerning token numbers, the number of levels of the random factor \textsc{genre}, the \isi{priors} that were specified, and the number of posterior samples. Data, scripts and regression tables can be found in the online materials (cf. \sectref{sec:1.4}).

\section{\label{bkm:Ref52538360}Results}\label{sec:8.2}

In analogy to \chapref{sec:7}, frequencies of the four semantic types in all varieties will first be discussed at a global level before differences between speech and writing are shown. Summing up the text frequencies of all four types would make little sense, as the result would be exactly the same as the summed frequencies of conjunctions (cf. \figref{fig:7.2} in \sectref{sec:7.3} above).

\figref{fig:8.1} shows the frequencies of the four semantic types~– \is{concessives (types of)!dialogic}dialogic, \is{concessives (types of)!anticausal}anticausal, \is{concessives (types of)!epistemic}epistemic and \is{concessives (types of)!narrow-scope dialogic}narrow-scope~– in all nine varieties under investigation, based on the respective single-variety components of Model~B (see Appendix~\ref{appendix:B.2}). The effects of mode are once again controlled for by estimating average values across written and spoken conditions (cf. \sectref{sec:6.3.4} for a discussion). The horizontal arrangement of semantic types in the plots in this chapter only partly follows theoretical considerations as outlined in \chapref{sec:3}: Anticausal, epistemic and dialogic CCs are grouped together because they are characterised by the same syntactic flexibility and their subordinate clauses have \isi{scope} over the entire matrix clause; narrow-scope CCs are set apart, because their syntactic behaviour is more restricted and they have \isi{scope} over the matrix-clause VP only (cf. \sectref{sec:2.2.4}). However, within the group of the three central (wide-scope) types, the horizontal arrangement is not based on the putative sequence of their historical development (\is{concessives (types of)!anticausal}anticausal > \is{concessives (types of)!epistemic}epistemic > \is{concessives (types of)!dialogic}dialogic) but follows their typical frequency ranking ($f_d$ > $f_a$ > $f_e$) to facilitate the comparison of patterns in the plots.

\begin{figure}
\includegraphics{figures/CCs.Fig.8.1.pdf}
\caption{\label{bkm:Ref489544014}\label{fig:8.1}Average frequencies of semantic types; d~= \is{concessives (types of)!dialogic}dialogic, a~= \is{concessives (types of)!anticausal}anticausal, e~= \is{concessives (types of)!epistemic}epistemic, d*~= \is{concessives (types of)!narrow-scope dialogic}narrow-scope dialogic}
\end{figure}

The literature implicitly suggests that \is{concessives (types of)!anticausal}anticausal CCs are somehow primary, or prototypical, possibly because there are relatively straightforward connections between their semantics and those of related kinds of \isi{adverbials} (e.g. \is{adverbials!condition}conditional, \is{adverbials!reason}causal, or \is{adverbials!result}consecutive). As shown in \figref{fig:8.1}, however, \is{concessives (types of)!dialogic}dialogic CCs are the most frequent semantic type in all nine varieties under investigation, with a mean text frequency of 266 pmw (not shown), followed at a considerable distance by \is{concessives (types of)!anticausal}anticausal CCs (\textit{M}~=~72 pmw).\footnote{Again, the \isi{geometric mean} of variety-specific medians was used in this and the following calculations. See Footnote~\ref{fn81} on p. \pageref{fn81}.} For \is{concessives (types of)!epistemic}epistemic CCs, \textit{M}~=~11 pmw across all varieties, and for \is{concessives (types of)!narrow-scope dialogic}narrow-scope CCs \textit{M}~=~14 pmw. With regard to these latter two types, there is greater variability between varieties: \is{concessives (types of)!epistemic}Epistemic CCs are more frequent than \is{concessives (types of)!narrow-scope dialogic}narrow-scope CCs in \il{Irish English}IrE, \il{Jamaican English}JamE, \il{Singapore English}SingE and \il{Hong Kong English}HKE, while the opposite is the case in the remaining five varieties. Since they are semantically related, “regular” \is{concessives (types of)!dialogic}dialogic CCs (with \isi{scope} over the entire matrix clause) and \is{concessives (types of)!narrow-scope dialogic}narrow-scope dialogic CCs could alternatively be treated as a single category, with a summed frequency value. However, due to the logarithmic treatment of the frequency scale in \figref{fig:8.1}, this would hardly affect the position of dialogic CCs relative to the others.

In the next step, the estimated frequencies of CCs of the four semantic types are compared between spoken and written varieties. Thus, in analogy to \figref{fig:7.4} in the previous chapter, two estimates are shown for each type in each variety in the lower panels of \figref{fig:8.2}, complemented by the estimated ratio of the two in the upper panels. In all nine varieties, \is{concessives (types of)!dialogic}dialogic, \is{concessives (types of)!anticausal}anticausal and \is{concessives (types of)!narrow-scope dialogic}narrow-scope CCs are more frequent in writing than in speech, and the writing-to-speech ratio is relatively robust in most cases: All of the 50\% intervals and most of the 90\% intervals are above the value of $Ro=1$. Across varieties, the mean ratios ($f_W$/$f_S$) are $Ro_d=2.7$, $Ro_a=2.1$, $Ro_e=1.6$ and $Ro_d*=3.6$, respectively. Thus, particularly \is{concessives (types of)!narrow-scope dialogic}narrow-scope CCs are much more common in written language. Another way to look at this phenomenon is to compare the rankings of the two less frequent types: In the written mode, \is{concessives (types of)!narrow-scope dialogic}narrow-scope CCs are more frequent than \is{concessives (types of)!epistemic}epistemic CCs in seven out of nine varieties (exceptions being \il{Singapore English}SingE and \il{Hong Kong English}HKE); in the spoken mode, this is the case in only three varieties (\il{Canadian English}CanE, \il{Australian English}AusE and \il{Nigerian English}NigE). In contrast to \is{concessives (types of)!narrow-scope dialogic}narrow-scope CCs, the frequencies of \is{concessives (types of)!epistemic}epistemic CCs differ less markedly between modes of production: There are four varieties (\il{British English}BrE, \il{Jamaican English}JamE, \il{Nigerian English}NigE and \il{Indian English}IndE) in which there is virtually no difference.

\begin{figure}
\includegraphics{figures/CCs.Fig.8.2.pdf}
\caption{\label{bkm:Ref52893515}\label{fig:8.2}Frequencies of semantic types in speech and writing; d~= dialogic, a~= anticausal, e~= epistemic, d*~= narrow-scope dialogic, W~= written, S~= spoken}
\end{figure}

Although \is{concessives (types of)!dialogic}dialogic CCs could be argued to require less \isi{planning} than \is{concessives (types of)!anticausal}anticausal or \is{concessives (types of)!epistemic}epistemic CCs, as they are characterised by semantically less tightly connected (or integrated) propositions, there is no evidence that they are more frequent in spoken language. If this were the case, the written/spoken ratios of \is{concessives (types of)!dialogic}dialogic CCs should be lower (perhaps even below $Ro=1$) when compared to \is{concessives (types of)!anticausal}anticausal CCs, for instance. However, text frequencies of \is{concessives (types of)!dialogic}dialogic, \is{concessives (types of)!anticausal}anticausal and \is{concessives (types of)!narrow-scope dialogic}narrow-scope CCs all mirror the general pattern established in the inspection of the frequencies of individual markers.

\section{\label{bkm:Ref494882555}Summary and discussion}\label{sec:8.3}

In the analyses in this chapter, \is{concessives (types of)!dialogic}dialogic CCs emerged as by far the most frequent type in all varieties. Examples of CCs typically cited in the literature tend to belong to the \is{concessives (types of)!anticausal}anticausal type, which seems to be implicitly treated as the semantic prototype. However, \is{concessives (types of)!anticausal}anticausal CCs are only the second most frequent semantic type in the present study. At far lower text frequencies, we find \is{concessives (types of)!epistemic}epistemic and \is{concessives (types of)!narrow-scope dialogic}narrow-scope dialogic CCs. In some varieties, the former is more frequent than the latter; in others, the opposite pattern obtains. The comparison of spoken and written varieties confirms the frequency patterns discussed in \chapref{sec:7}, with generally higher frequencies in writing for all types, perhaps with the exception of \is{concessives (types of)!epistemic}epistemic CCs, which are often enough of similar frequencies in both modes. There is no evidence of the \is{concessives (types of)!dialogic}dialogic type being more frequent in speech, although it could be argued to be more coordinated in character and thus to require less advance \isi{planning}. The frequencies of \is{concessives (types of)!narrow-scope dialogic}narrow-scope CCs, on the other hand, differ more radically between speech and writing than those of the other types, with considerably higher frequencies in writing. This may be because in most CCs of this type the complement of the conjunction is \is{nonfinite clauses}nonfinite~– usually it is not even clausal or interpretable as a \is{verbless clauses}verbless clause (see examples in \sectref{sec:2.2.4}). Moreover, \is{concessives (types of)!narrow-scope dialogic}narrow-scope CCs are more locally embedded at the phrase level, and both this and their nonfiniteness make them cognitively complex and thus likely to correlate with written language.

There are no previous findings to which results presented in this chapter could be related, with the exception of \citet{Hilpert2013a} and some of my own earlier research. Although not including \textit{even though}, Hilpert’s study suggests that the \is{concessives (types of)!anticausal}anticausal type is much more frequent  than seems to be the case in the present study. As pointed out earlier, however, Hilpert focuses on a specific construction type (with \is{co-referentiality}co-referential subjects in both component clauses), whose semantic versatility is possibly restricted. Schützler’s (\citeyear{Schützler2017, Schützler2018b}) results anticipate the outcome of this chapter, even if, like Hilpert’s, these studies take a \is{semasiological approaches}semasiological approach to markers, investigating the percentages of semantic types found for each of them.

Finally, it has to be mentioned once again that the approach of counting semantic types the way it was done in this chapter has its limitations and is attached to certain caveats. Since the analysis is strictly limited to the three conjunctions \textit{although}, \textit{though} and \textit{even though}, it is not transparent how many of the different semantic types are encoded using other formal means, like prepositional constructions or coordination with \SchuetzlerIndexExpression{but}, for example. It may well be the case that the total number of CCs (at least of the \is{concessives (types of)!dialogic}dialogic, \is{concessives (types of)!anticausal}anticausal and \is{concessives (types of)!epistemic}epistemic types) does not differ substantially between speech and writing when inspected from a more holistic, global perspective that takes more concessive markers into account. In other words: Frequencies of semantic types shown in this chapter are to a considerable extent likely to be artefacts of the frequencies of subordinating conjunctions.

