\chapter{\label{bkm:Ref411411802}Introduction}\label{sec:1}

Studies of concessives have been undertaken from mainly three perspectives, as pointed out by \citet[382–383]{Couper-KuhlenThompson2000}. Some of the research focuses on concession as the establishment of a particular syntactic link between clauses (or clause-like structures), using concessive \isi{connectives}. Secondly, concession may be of interest as a semantic text relation, with a focus on the underlying assumptions and semantic mechanisms. A third approach looks at concession from the perspective of rhetoric, that is, it emphasises the role of concession in spoken interaction (cf. \citealt{Couper-KuhlenKortmann2000Introduction}: 2, \citealt{Barth2000,Barth-Weingarten2003}). Those three perspectives broadly correspond to the fields of syntax, \isi{text linguistics} and \isi{discourse analysis}; accordingly, certain phenomena and methodologies will take centre stage, depending on the focus that is selected. For example, the syntax-oriented approach will only consider concessives in which an overt \is{connectives}connective grammatically marks the relation between \isi{matrix clause} and dependent structure, while the \is{discourse analysis}discourse-analytical approach is much more interested in conceding moves between discourse participants, which may or may not be supported by typical grammatical \isi{markers}.

  The present study is informed by the first two perspectives above, i.e. it focuses on syntactic constructions and the semantic relations that they express (and thus also on local textual coherence). In this approach, concessives only qualify as objects of investigation if they are attached to certain \isi{markers}, which are in this case restricted to the three conjunctions \SchuetzlerIndexExpression{although}, \SchuetzlerIndexExpression{though} and \SchuetzlerIndexExpression{even though}. The analysis of semantics and syntax is conducted at the level of the construction, not at the discourse level, and a construction is comprised of
(i)~a semantico-pragmatic relation that holds between two propositions,
(ii)~two syntactic units (clausal or clause-like) that correspond to propositions, and
(iii)~a \is{connectives}connective. While corpus queries in this study are essentially form-driven, the analysis goes beyond formal aspects (e.g. the counting of markers) and gives semantic and pragmatic considerations a central place. In contrast to \is{discourse analysis}discourse-analytical approaches, however, the concrete objects of study are relatively local, in the sense that they do not extend beyond complete sentences and thus treat \is{addressee/reader}addressees or interlocutors and their contributions as no more than abstract givens operating in the background. This can of course be viewed as a shortcoming, but it was considered a necessary restriction; its implications will be discussed in the relevant contexts.

The analytic, quantitative parts of the study are found in Chapters~\ref{sec:7}–\ref{sec:11} of the volume. In Chapters \ref{sec:7} \& \ref{sec:8}, two surface characteristics of \isi{concessive constructions} are described in detail:
(i)~the \is{text frequency}text frequencies of conjunctions and
(ii)~the \is{text frequency}text frequencies of semantic types. These chapters do not establish any relationships between the different \is{function and form}functional and formal facets of \isi{concessive constructions}, and their main function is to prepare and support the more complex scenarios analysed in the later chapters. Particularly assessing the \is{text frequency}text frequencies of the three conjunctions offers a perspective known from traditional, first-generation corpus-linguistic research, highlighting rates of occurrence without taking recourse to the under\-lying factors that motivate them, and thus without describing in detail the partly predictable choices made by language users. What places the study as a whole in a \isi{Construction Grammar} (\is{Construction Grammar}CxG) context is the approach taken in Chapters \ref{sec:9}–\ref{sec:11}. Here, it is assumed that \isi{concessive constructions} are characterised by \is{function and form}functional and formal properties that matter in combination and therefore need to be explored together. It is of course a challenge for quantitative research to treat the construction as an indivisible whole, rather than focus on one of its characteristics (e.g. semantic type, clause position): Instead of a single variable (e.g. an alternation) I predict the behaviour of constructions comprised of several variable parameters. In this book, I propose a \is{constructional choice model}model of constructional choice that rests on five assumptions, as explained in more detail in \sectref{sec:4.1.3}; as a cognitive model, it will inform the quantitative analyses and the line of argumentation followed in presenting the results.

\section{\label{bkm:Ref51150486}Existing research on concessives in English}\label{sec:1.1}

Characteristics of concessive adverbials are discussed in several edited volumes, which usually treat a wider range of semantic relations, often including or even focusing on languages other than English (e.g. \citealt{Kortmann1996,Rudolph1996,vanderAuwera1998,Couper-KuhlenKortmann2000,Ferraresi2011}). There are also many individual articles and chapters, in those volumes and elsewhere, which discuss the semantics of concessives, their relatedness to and overlap with other types of \isi{adverbials}, as well as the origin and etymology of concessive \isi{connectives} (e.g. \citealt{König1985,König1988,Hermodsson1994,Azar1997,DiMeola1998,KönigSiemund2000}). Fewer publications take a quantitative, \isi{usage-based approach} to concessives; unsurprisingly, all of them approach the topic using corpus-linguistic methods (e.g. \citealt{Altenberg1986,Aarts1988,Rissanen2002,Hoffmann2005,Berlage2009,Hilpert2013a}). However, all of these contributions are based on present-day and historical \il{British English}British and \il{American English}American Standard English (hereafter: \il{British English}BrE and \il{American English}AmE), and they focus on selected aspects of variation and change (e.g. variable syntactic structures, semantics, frequencies, the choice of connectives), but not on their interaction (or association) in constructions. \citet{Hilpert2013a} is exceptional in going some way towards the inspection of multiple dimensions of variation (semantic types, syntactic realisations), although he, too, stays within the bounds of \il{American English}AmE. What is still lacking, therefore, is research on concessive adverbials that
(i)~goes beyond \il{British English}BrE and \il{American English}AmE,
(ii)~inspects several dimensions of variation based on the same data, and
(iii)~takes steps in the direction of a more holistic \is{Construction Grammar}CxG account that considers \is{function and form}functional and formal criteria in combination. All three points strongly inform the approach taken in this book, which is, moreover, firmly regression-based and transparent in the sense that \is{uncertainty}uncertainties are communicated along with \is{effect size}effect sizes. The result is a complex corpus-linguistic research design that generates important insights but also raises interesting methodological questions that can be of value for future research in a \is{Construction Grammar}CxG framework.

The approach taken in this book may inform research on other (i.e. non-con\-ces\-sive) types of \is{adverbials}adverbial linkage, although the semantico-pragmatic mechanisms at work in most of them would require considerable adjustments. However, taking the insights gained in this study as starting points for research on other \is{adverbials}adverbial constructions may be worthwhile since the quantitative and methodological gap outlined above can more or less be argued to hold for the entire domain of \is{adverbials}adverbial linkage. This becomes evident, for example, if one looks at the amount of research conducted on aspects of the English verb phrase, either based on the central standard dialects (e.g. contributions in \citealt{AartsWallis2013}) or actively engaging with \isi{World Englishes} (e.g. \citealt{HundtGut2012}). Similarly, the noun phrase in English has been investigated quantitatively with corpora (e.g. \citealt{Jucker1993,Pastor-Gómez2011,Berlage2014}). In contrast, explicitly quantitative studies of \is{adverbials}adverbial constructions and linkage are few and far between, although this aspect of grammar is certainly rather central. \citet[2]{Lenker2010} therefore still seems to have a point in saying that connectives (and the constructions they bind together) are an understudied area.

\section{\label{bkm:Ref411515113}\label{bkm:Ref411603806}Concessives as constructions}\label{sec:1.2}

Adverbial (or “circumstantial”, \citealt{KönigSiemund2000}: 341) relations expressed by a concessive construction are grouped among “adverbials of \is{adverbials!contingency}contingency” by \citet[479, 484]{QuirkEtAl1985}, together with adverbials of \is{adverbials!reason}reason, \is{adverbials!purpose}purpose, \is{adverbials!result}result and \is{adverbials!condition}condition. In adverbial constructions of this class, two propositions are shown in relation to each other, one of which depends (“is contingent”) upon the other (cf. \citealt{Burnham1911}: 1, \citealt{BiberEtAl1999}: 779). Sometimes it is suggested that, among adverbials, concessives are particularly complex (e.g. \citealt{Kortmann1996}: 167–175, \citealt{DiMeola1998}: 348, \citealt{Hoffmann2005}: 111, \citealt{König2006}: 821), which has implications for their historical development, their late emergence in language learning, and their cross-linguistic \isi{markedness}. The particular intra-constructional complexity of concessives is a consideration when formulating hypotheses and expectations prior to the analyses in this book.

Similarly to \citet[633]{König1991b}, the present study uses the term \textit{concessive} to refer to the entire bipartite \is{concessive constructions}concessive construction~– “bipartite” in the sense that it consists of two syntactic structures with propositional content (in this case: clauses or structures interpretable as reduced clauses) that are in some way connected, usually through overt concessive marking. The terms \is{connectives}\textit{connective} and \is{markers}\textit{marker} will be used interchangeably: It is assumed that marking a grammatical structure in order to encourage a concessive reading invariably involves connecting (or linking) two components. The concessive as a whole is characterised by at least four variable parameters (or facets), as proposed by \citet[176]{Hilpert2013a}:
(i)~the semantic relationship that holds between the two component parts;
(ii)~the syntactic arrangement of components (e.g. \is{initial position}initial, \is{final position}final or \is{medial position}medial placement of the dependent structure relative to the \isi{matrix clause});
(iii)~the selection of the concessive \is{markers}marker(s); and
(iv)~the internal syntactic form of the subordinate part (e.g. \isi{finite clauses} vs reduced or \isi{nonfinite clauses}).\footnote{Hilpert (\citeyear{Hilpert2013a}) focuses on concessive \isi{parentheticals}, but his four dimensions of variation are undoubtedly applicable to concessives in general.} Whenever reference is made to a \is{concessive constructions}concessive construction (or CC, for short) in this study, it is implied that this entity can be described in terms of these aspects. Crucially, it is assumed that the four dimensions are not independent but linked in a certain way; this view, detailed in \sectref{sec:4.1.3}, will strongly inform the quantitative analyses in Chapters \ref{sec:9}–\ref{sec:11} and their sequential arrangement. Some priority is given to functional aspects, which then have formal consequences: It is the primary need to express a certain semantic relation that triggers the selection of a basic syntactic frame (i.e. an arrangement of super- and subordinate), a certain concessive \is{markers}marker and its (\is{finite clauses}finite or \is{nonfinite clauses}nonfinite) clausal complement. I will thus argue that the emergence and \isi{entrenchment}~– and therefore the patterns of use~– of \isi{concessive constructions} can be conceptualised as following a cognitively motivated trajectory, which proceeds from the need to express semantico-pragmatic relations to the formal realisation of these relations, at increasingly fine levels of detail. While formal realisation is of course instantaneous, and therefore happens simultaneously at different levels (clause arrangement, \is{markers}marker selection, realisation of complement), I hope to demonstrate that a cognitively sequential model is helpful, both theoretically and methodologically. Alternative, truly holistic approaches to \isi{constructional variation}~– as foreshadowed in the final chapter~– will be explored in future research.

Against the background of this constructional, multifaceted view of CCs, previous research naturally provides an incomplete picture. Thus, \citet{Aarts1988} and \citet{Hoffmann2005} study the frequencies of different concessive \isi{markers} and analyse their \is{style}stylistic distribution, but are not concerned with semantic types. \citet{Aarts1988} does investigate the syntactic ordering of CCs marked by certain \isi{connectives}, but not its interaction with other factors. \citet{Berlage2009}, on the other hand, correlates the use of \SchuetzlerIndexExpression{notwithstanding} as a pre- or postposition with the complexity of the attached noun phrase (NP), but does not differentiate between semantic types either. \citet{Hilpert2013a} includes all four dimensions discussed above in his analysis of concessive \isi{parentheticals}, which is very much consistent with the \is{Construction Grammar}CxG framework and its assumptions of an indissoluble link between form and meaning in a construction (hence their definition as \textit{form-meaning pairings}; cf. \citealt{Goldberg2003}). Therefore, his study is an important point of reference for the present one, although it differs in methodology.

\section{\label{bkm:Ref495941783}Aims, scope and structure of the study}\label{sec:1.3}

\citet[111]{Hoffmann2005} is relatively pessimistic about the feasibility of full-scale studies of concession~– that is, \isi{onomasiological approaches} exploring all possible ways of expressing this relation:

\begin{quote}
Given the relatively large range of linguistic realizations, a comprehensive study of concessive relations is certainly not an easy undertaking. This is particularly true given the fact that some sentences may carry a concessive interpretation even though they do not contain an overt marker of concessiveness.
\end{quote}

Indeed, the number of possible \isi{markers} is large, and constructions associated with each of them are potentially characterised by formal and semantic variability across several dimensions (cf. \sectref{sec:1.2}). Hoffmann also rightly identifies the problem of formally \is{concessives (types of)!unmarked}unmarked CCs that are virtually impossible to retrieve automatically from a corpus.

In consequence, the present study does not aim to be comprehensive but, as mentioned above, focuses on the \is{subordinators}subordinating conjunctions \SchuetzlerIndexExpression{although}, \SchuetzlerIndexExpression{though} and \SchuetzlerIndexExpression{even though}.\footnote{In \sectref{sec:7.1}, more \isi{markers} will be cursorily inspected regarding their frequencies in speech and writing. Further, see \citet{Schützler2018a} for a \is{diachronic approaches}diachronic study of \textit{notwithstanding}; see also \citet{Schützler2018c}. A comprehensive treatment of concession would ideally rely on the study of a single, medium-sized corpus, using the automatic retrieval of overtly marked constructions in combination with the manual identification of cases that do not carry an overt \is{markers}marker. This would essentially require reading the corpus.} Concentrating on them as a pseudo-closed set was considered appropriate for the following reasons:

\begin{itemize}
\item Unlike \isi{conjuncts} (e.g. \SchuetzlerIndexExpression{however}, \SchuetzlerIndexExpression{nevertheless}), which are always attached to the consequent proposition and follow the antecedent proposition, the conjunctions under investigation partake in the full range of syntactic variation. That is, they (along with their complements) can be in \is{initial position}initial, \is{medial position}medial and \isi{final position} relative to the corresponding (in this case: superordinate) component of the CC.
\item\sloppy Unlike other conjunctions that may convey concessive meaning (e.g. \textit{while}, \textit{when}, \textit{if}), they are captured rather well by \citegen{Sweetser1990} semantic framework (cf. \sectref{sec:2.2}) and do not overlap with other, “primary” (e.g. \is{adverbials!condition}conditional or \is{adverbials!time}temporal) \is{adverbials}adverbial relations to the same extent (see \sectref{sec:2.1.1}).
\item Due to their etymological relatedness and morphological similarity, they are often treated as a set of \is{synonymy}quasi-synonymous items; this set, however, is internally underspecified, and its inspection can therefore close a gap in this niche of English grammar.
\end{itemize}

The general research questions that inform the analyses in this book were partly discussed in \sectref{sec:1.2} above. They will be given more substance by formulating hypotheses and expectations in \sectref{sec:5.3}, which in turn will be addressed empirically in Chapters~\ref{sec:9}–\ref{sec:11}, following the more descriptive approaches of Chapters \ref{sec:7} \& \ref{sec:8}. The underlying broad questions are the following:

\begin{enumerate}\sloppy
\item Is the variation between the \is{final position}final and \is{nonfinal position}nonfinal placement of subordinate clauses systematic, i.e. can we account for it in terms of semantico-prag\-mat\-ics, principles of \isi{production} and \isi{processing}, or variety status? [→~\chapref{sec:9}]
\item Is the choice between the three conjunctions systematically conditioned by the semantic relation that holds within the construction, or by the general syntactic frame, i.e. the arrangement of matrix and subordinate clause relative to each other? Do \isi{language-external factors} (\is{varieties of English}variety, \isi{mode of production}) affect the choice of conjunction? In short, can the three conjunctions be regarded as \is{synonymy}quasi-synonymous at all? [→~\chapref{sec:10}]
\item Likewise, is the alternation of \is{finite clauses}finite and \is{nonfinite clauses}nonfinite/reduced subordinate clauses systematically affected by the same semantic or contextual factors? In addition, are some conjunctions more likely to attract \is{nonfinite clauses}nonfinite clauses than others? [→ \chapref{sec:11}]
\end{enumerate}

These three broad questions will be linked to concrete expectations (or hypotheses) at the end of \chapref{sec:5}, immediately before embarking on the quantitative analyses. Concrete expectations are framed relatively late because they depend on the background provided in Chapters~\ref{sec:2}–\ref{sec:4}.

The book as a whole is structured as follows: \chapref{sec:2} discusses
(i)~the etymology and historical development of the markers under investigation,
(ii)~the different semantic types of concessives that are assumed to exist, and
(iii)~the forms of concessives both in terms of the position of dependent structures relative to matrix clauses and the internal syntactic structure of complements. \chapref{sec:3} provides corpus examples to illustrate the semantic (and syntactic) categories relevant in the present study and serves as a qualitative counterpoint to the otherwise strongly quantitative analyses. \chapref{sec:4} sketches the theoretical framework of \isi{Construction Grammar} (\is{Construction Grammar}CxG) as well as the two dimensions along which variation is mainly explored in this book, namely \isi{mode of production} and national \isi{varieties of English}. \chapref{sec:5} presents short summaries of existing research and formulates more concrete expectations and hypotheses on this basis. \chapref{sec:6} lays out the methodologies that were employed. It includes discussions of
(i)~the corpus material that was used,
(ii)~the steps that were followed in retrieving, processing and coding the data, and
(iii)~the applied methods of statistical analysis and visualisation. Chapters \ref{sec:7} \& \ref{sec:8} contain the descriptive analyses discussed above, which focus on the \is{text frequency}text frequencies of \isi{markers} and semantic types. Chapters \ref{sec:9}–\ref{sec:11} inspect factors that have an influence on
(i)~the placement of clauses,
(ii)~the selection of \isi{markers} and
(iii)~the clause-internal syntax of subordinates. These three chapters (and to some extent also Chapters \ref{sec:7} \& \ref{sec:8}) are essentially parallel in structure and thus provide accessible, in-depth and easy-to-compare treatments of individual aspects. At the same time, they follow the logic of the \is{constructional choice model}model of constructional choice presented in \sectref{sec:4.1.3}. Finally, \chapref{sec:12} contains a general summary of results, discusses their descriptive, theoretical and methodological implications, and points to avenues of future research.

\section{\label{bkm:Ref60730938}Open data}\label{sec:1.4}

The data used in this monograph are published as \citet{Schützler2021} at the \textit{Tromsø Repository of Language and Linguistics} (TROLLing) and can be retrieved via the identifier \url{https://doi.org/10.18710/1JMFVR}. Annotated R scripts used for the analyses and visualisations in this monograph can be retrieved from the \textit{Open Science Framework} at \url{https://osf.io/m4tfc/}. This repository will occasionally be referred to as “the online appendix”, and it also contains all graphics files from this volume. In combination with the original data published at TROLLing, these materials enable readers to rerun all analyses exactly as in this volume, revisualise the data or integrate them into their own research, adapt the models (e.g. by using different \isi{priors}, including more interactions, or specifying different random effects) or implement altogether different kinds of models (e.g. of a \is{Bayesian statistics}non-Bayesian type) or statistical tests. While individual data tables, scripts and figures at \url{https://osf.io/m4tfc} have their unique, direct links, I do not refer to these in the text for the sake of simplicity. However, the repository is structured so as to support easy navigation through the individual components, and there is a ReadMe file explaining how the different parts interrelate.
